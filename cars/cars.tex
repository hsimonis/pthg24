\PassOptionsToPackage{table}{xcolor}
\documentclass[a4paper]{article}
\usepackage[a4paper,margin=2cm,landscape]{geometry}
\usepackage{tabularx}
\usepackage{tikz}
\usepackage{graphicx}
\usepackage{rotating}
\usepackage{float}
\usepackage{calc}
\usepackage{pdflscape}
\usepackage{booktabs}
\usepackage{colortbl}
\usepackage{longtable}
\usepackage{stackengine}
\usepackage{multicol}
%\usepackage{showkeys}
\newcounter{rowcounter}
\newcommand{\rowlabel}[1]{\refstepcounter{rowcounter}\label{#1}}

\usepackage{url}
\usepackage{hyperref}

\newcommand{\su}[1]{\Shortunderstack[l]{#1}}

\title{CP Papers on Car Sequencing}
\author{Helmut Simonis and Cemalettin Öztürk}
\begin{document}
\rowcolors{2}{gray!20}{white}

\maketitle
\section{Introduction}

This document shows the result of a survey on "Constraint Programming and Scheduling", which tries to find and classify all publications on the combination of these two concepts. It is based on a manually collected bibfile containing reference to relevant papers and articles, and on an automatic and manual analysis of local copies of the cited papers. For copyright reasons, we are obviously not able to distribute the collected copies, but we provide links to the original sources of the files. 

We identify the papers by a key which is the last name of the first author, the first character of the last names of all other authors, and a two digit year code for the date of publication. If multiple works would define the same key, we differentiate by adding a suffix "a", "b", etc, to the second and subsequent works found.

Most of the content of this document is generated by a Java program that parses the bib files, adds any manually extracted information, and which then extracts concept occurrences from the local copies of the works. It then produces tables and other LaTeX  artifacts that are included in a manually defined top-level document.

To add new works, first add bibtex entries for each work in the main \texttt{overview/bib.bib} file, then add local copies of the pdf of the work to the \texttt{overview/works/} directory, using the key of the bibtex entry as the file name (plus extension .pdf), and then run the main Java program \texttt{org.insightcentre.pthg24.JfxApp} to consolidate the information and extract the relevant concepts. Finally, run \texttt{pdflatex} on the \texttt{overview/scheduling.tex} file to produce this pdf document. Manually extracted information for the files can be added in the \texttt{imports/manual.csv} file. New concepts can be added in the file \texttt{imports/concepts.json}, new concept types need to be directly defined in the Java code.

We start the document by providing a table of all defined keys in the bib file in alphabetical order. This table can be helpful to see if a candidate paper is already in the survey, it suffices to see if the key is already present, and matches the authors, title and origin of the candidate paper. In the table link given by the key points to the local copy of the file, while the citation number links to the bibliography entry. That entry typically also contains a link to the original source of the paper.

This document heavily depends on the use of hyper links in the document, it has been tested with Acrobat Reader, other pdf reader may not use links in the same way. 

\clearpage
\begin{longtable}{*{6}{l}}
\rowcolor{white}\caption{Key Overview (Total: 29)}\\ \toprule
\rowcolor{white}1 & 2 & 3 & 4 & 5 & 6\\ \midrule
\endhead
\bottomrule
\endfoot
\href{cars/works/ArtiguesHM0W14.pdf}{ArtiguesHM0W14}~\cite{ArtiguesHM0W14} & \href{}{BoysenFS09}~\cite{BoysenFS09} & \href{cars/works/ButaruH05.pdf}{ButaruH05}~\cite{ButaruH05} & \href{cars/works/DincbasSH88.pdf}{DincbasSH88}~\cite{DincbasSH88} & \href{}{Gent98}~\cite{Gent98} & \href{}{GolleRB14}~\cite{GolleRB14}\\ 
\href{cars/works/GottliebPS03.pdf}{GottliebPS03}~\cite{GottliebPS03} & \href{}{HindiP94}~\cite{HindiP94} & \href{cars/works/HoevePRS06.pdf}{HoevePRS06}~\cite{HoevePRS06} & \href{}{HoevePRS09}~\cite{HoevePRS09} & \href{}{Kis04}~\cite{Kis04} & \href{cars/works/Mayer-EichbergerW13.pdf}{Mayer-EichbergerW13}~\cite{Mayer-EichbergerW13}\\ 
\href{}{MazurN15}~\cite{MazurN15} & \href{}{MoyaCB19}~\cite{MoyaCB19} & \href{}{OzturkTHO13}~\cite{OzturkTHO13} & \href{cars/works/ParrelloK86.pdf}{ParrelloK86}~\cite{ParrelloK86} & \href{cars/works/PerronS04.pdf}{PerronS04}~\cite{PerronS04} & \href{cars/works/ReginP97.pdf}{ReginP97}~\cite{ReginP97}\\ 
\href{}{Schaus09}~\cite{Schaus09} & \href{cars/works/Siala15.pdf}{Siala15}~\cite{Siala15} & \href{cars/works/SialaHH14.pdf}{SialaHH14}~\cite{SialaHH14} & \href{}{SialaHH155}~\cite{SialaHH155} & \href{}{SolnonCNA08}~\cite{SolnonCNA08} & \href{}{ThiruvadyME11}~\cite{ThiruvadyME11}\\ 
\href{}{WarwickT95}~\cite{WarwickT95} & \href{}{WinterM21}~\cite{WinterM21} & \href{}{YavuzE18}~\cite{YavuzE18} & \href{}{YuLZCLW22}~\cite{YuLZCLW22} & \href{}{ZhangGWH17}~\cite{ZhangGWH17} & \end{longtable}


\section{Conference Paper List}

This section presents the information for all conference papers included in the survey. For space reasons, not all information about the papers can be presented in a single table, we therefore split the data into three parts. The first part contains the main bibliographical information for the paper. The paper are sorted by year of publication (newest first), and then alphabetically by key. 

The key contains a hyperlink to the original source URL of the paper. You may have to navigate manually to download the actual paper content, and you may be unable to access the paper completely if it is behind a paywall for which you (or your organization) do not have access.

We then list the authors of the paper, in the other given in the bibtex file, abbreviating first names for space where we can identify them. Note that names with non-latin characters are not handled by latex. We use the form that is given in the bibtex file, but have excluded entries that cause latex to fail.  

We then give the title of the publication, using the original capitalization of the title entry in the bibtex entry, which may differ from the format shown in the bibliography. We then (column LC) provide a link to a local copy, if it is present, and a link to the bibliography entry of the paper.  We also show the year of publication, and the conference where the paper was published, using a short form abbreviation of the conference. This relies on a matching routine in the Java code to find the short title, new conference series may require an additional entry in \texttt{ImportBibtex.java} to work properly. Finally we list the number of pages of the paper, this information is using the bibtex entry where possible, otherwise uses \texttt{pdfinfo} to extract the actual number of pages from the local copy. The final columns b and c provide links to the corresponding tables of extracted concepts and manual information. Note that the links to typically show the correct page, not do not necessarily scroll to the correct line in the table.

\clearpage
\subsection{Papers from bibtex}
{\scriptsize
\begin{longtable}{>{\raggedright\arraybackslash}p{3cm}>{\raggedright\arraybackslash}p{6cm}>{\raggedright\arraybackslash}p{6.5cm}rrrp{2.5cm}rrrrr}
\rowcolor{white}\caption{Works from bibtex (Total 9)}\\ \toprule
\rowcolor{white}Key & Authors & Title & LC & Cite & Year & \shortstack{Conference\\/Journal} & Pages & \shortstack{Nr\\Cites} & \shortstack{Nr\\Refs} & b & c \\ \midrule\endhead
\bottomrule
\endfoot
\rowlabel{a:ArtiguesHM0W14}ArtiguesHM0W14 \href{https://doi.org/10.1007/978-3-319-07046-9\_19}{ArtiguesHM0W14} & \hyperref[auth:a8]{C. Artigues}, \hyperref[auth:a12]{E. Hebrard}, \hyperref[auth:a35]{V. Mayer{-}Eichberger}, \hyperref[auth:a11]{M. Siala}, \hyperref[auth:a36]{T. Walsh} & {SAT} and Hybrid Models of the Car Sequencing Problem & \href{cars/works/ArtiguesHM0W14.pdf}{Yes} & \cite{ArtiguesHM0W14} & 2014 & CPAIOR 2014 & 16 & 2 & 16 & \ref{b:ArtiguesHM0W14} & \ref{c:ArtiguesHM0W14}\\
\rowlabel{a:Mayer-EichbergerW13}Mayer-EichbergerW13 \href{https://doi.org/10.29007/jrsp}{Mayer-EichbergerW13} & \hyperref[auth:a35]{V. Mayer{-}Eichberger}, \hyperref[auth:a36]{T. Walsh} & {SAT} Encodings for the Car Sequencing Problem & \href{cars/works/Mayer-EichbergerW13.pdf}{Yes} & \cite{Mayer-EichbergerW13} & 2013 & SAT 2013 & 13 & 0 & 0 & \ref{b:Mayer-EichbergerW13} & \ref{c:Mayer-EichbergerW13}\\
\rowlabel{a:ThiruvadyME11}ThiruvadyME11 \href{}{ThiruvadyME11} & \hyperref[auth:a26]{Dhananjay Raghavan Thiruvady}, \hyperref[auth:a27]{B. Meyer}, \hyperref[auth:a28]{A. Ernst} & Car sequencing with constraint-based ACO & No & \cite{ThiruvadyME11} & 2011 & Genetic and evolutionary computation 2011 & 8 & 0 & 0 & No & \ref{c:ThiruvadyME11}\\
\rowlabel{a:HoevePRS06}HoevePRS06 \href{https://doi.org/10.1007/11889205\_44}{HoevePRS06} & \hyperref[auth:a43]{Willem Jan van Hoeve}, \hyperref[auth:a40]{G. Pesant}, \hyperref[auth:a44]{L. Rousseau}, \hyperref[auth:a42]{A. Sabharwal} & Revisiting the Sequence Constraint & \href{cars/works/HoevePRS06.pdf}{Yes} & \cite{HoevePRS06} & 2006 & CP 2006 & 15 & 33 & 7 & \ref{b:HoevePRS06} & \ref{c:HoevePRS06}\\
\rowlabel{a:ButaruH05}ButaruH05 \href{}{ButaruH05} & \hyperref[auth:a29]{M. Butaru}, \hyperref[auth:a30]{Z. Habbas} & The car-sequencing problem as n-ary CSP--Sequential and parallel solving & \href{cars/works/ButaruH05.pdf}{Yes} & \cite{ButaruH05} & 2005 & Australian Joint Conference on Artificial Intelligence 2005 & 4 & 0 & 0 & \ref{b:ButaruH05} & \ref{c:ButaruH05}\\
\rowlabel{a:PerronS04}PerronS04 \href{https://doi.org/10.1007/978-3-540-24664-0_16}{PerronS04} & \hyperref[auth:a20]{L. Perron}, \hyperref[auth:a21]{P. Shaw} & Combining Forces to Solve the Car Sequencing Problem & \href{cars/works/PerronS04.pdf}{Yes} & \cite{PerronS04} & 2004 & CPAIOR 2004 & 15 & 17 & 9 & \ref{b:PerronS04} & \ref{c:PerronS04}\\
\rowlabel{a:GottliebPS03}GottliebPS03 \href{https://doi.org/10.1007/3-540-36605-9_23}{GottliebPS03} & \hyperref[auth:a9]{J. Gottlieb}, \hyperref[auth:a10]{M. Puchta}, \hyperref[auth:a5]{C. Solnon} & A Study of Greedy, Local Search, and Ant Colony Optimization Approaches for Car Sequencing Problems & \href{cars/works/GottliebPS03.pdf}{Yes} & \cite{GottliebPS03} & 2003 & EvoWorkshop 2003 & 12 & 46 & 5 & \ref{b:GottliebPS03} & \ref{c:GottliebPS03}\\
\rowlabel{a:ReginP97}ReginP97 \href{https://doi.org/10.1007/BFb0017428}{ReginP97} & \hyperref[auth:a33]{J. R{\'{e}}gin}, \hyperref[auth:a34]{J. Puget} & A Filtering Algorithm for Global Sequencing Constraints & \href{cars/works/ReginP97.pdf}{Yes} & \cite{ReginP97} & 1997 & CP 1997 & 15 & 53 & 3 & \ref{b:ReginP97} & \ref{c:ReginP97}\\
\rowlabel{a:DincbasSH88}DincbasSH88 \href{}{DincbasSH88} & \hyperref[auth:a2]{M. Dincbas}, \hyperref[auth:a3]{H. Simonis}, \hyperref[auth:a4]{Pascal Van Hentenryck} & Solving the Car-Sequencing Problem in Constraint Logic Programming & \href{cars/works/DincbasSH88.pdf}{Yes} & \cite{DincbasSH88} & 1988 & ECAI 1988 & 6 & 0 & 0 & \ref{b:DincbasSH88} & \ref{c:DincbasSH88}\\
\end{longtable}
}



\clearpage
\subsection{Extracted Concepts}
{\scriptsize
\begin{longtable}{>{\raggedright\arraybackslash}p{3cm}r>{\raggedright\arraybackslash}p{4cm}p{1.5cm}p{2cm}p{1.5cm}p{1.5cm}p{1.5cm}p{1.5cm}p{2cm}p{1.5cm}rr}
\rowcolor{white}\caption{Automatically Extracted PAPER Properties (Requires Local Copy)}\\ \toprule
\rowcolor{white}Work & Pages & Concepts & Classification & Constraints & \shortstack{Prog\\Languages} & \shortstack{CP\\Systems} & Areas & Industries & Benchmarks & Algorithm & a & c\\ \midrule\endhead
\bottomrule
\endfoot
\rowlabel{b:ArtiguesHM0W14}\href{../cars/works/ArtiguesHM0W14.pdf}{ArtiguesHM0W14}~\cite{ArtiguesHM0W14} & 16 & activity, lazy clause generation, order &  & Among constraint, Cardinality constraint, cumulative &  & Mistral &  &  & CSPlib, Roadef, benchmark, github &  & \ref{a:ArtiguesHM0W14} & \ref{c:ArtiguesHM0W14}\\
\rowlabel{b:ButaruH05}\href{../cars/works/ButaruH05.pdf}{ButaruH05}~\cite{ButaruH05} & 4 & job, job-shop, machine, order, task &  &  & C++ & Ilog Solver &  &  & CSPlib &  & \ref{a:ButaruH05} & \ref{c:ButaruH05}\\
\rowlabel{b:DincbasSH88}\href{../cars/works/DincbasSH88.pdf}{DincbasSH88}~\cite{DincbasSH88} & 6 & distributed, job, job-shop, machine, order, precedence, resource, scheduling, task &  & Disjunctive constraint, circuit, disjunctive & Prolog & CHIP, OPL &  &  & real-life &  & \ref{a:DincbasSH88} & \ref{c:DincbasSH88}\\
\rowlabel{b:GottliebPS03}\href{../cars/works/GottliebPS03.pdf}{GottliebPS03}~\cite{GottliebPS03} & 12 & machine, order, scheduling &  & cycle &  &  &  &  & CSPlib, benchmark, real-world &  & \ref{a:GottliebPS03} & \ref{c:GottliebPS03}\\
\rowlabel{b:HoevePRS06}\href{../cars/works/HoevePRS06.pdf}{HoevePRS06}~\cite{HoevePRS06} & 15 & order, transportation &  & Among constraint, Cardinality constraint, Regular constraint &  & CHIP, Ilog Solver & nurse &  & real-life &  & \ref{a:HoevePRS06} & \ref{c:HoevePRS06}\\
\rowlabel{b:Mayer-EichbergerW13}\href{../cars/works/Mayer-EichbergerW13.pdf}{Mayer-EichbergerW13}~\cite{Mayer-EichbergerW13} & 13 & scheduling, task &  & Atmost constraint, Cardinality constraint, cumulative &  &  &  &  & CSPlib, benchmark, github &  & \ref{a:Mayer-EichbergerW13} & \ref{c:Mayer-EichbergerW13}\\
\rowlabel{b:PerronS04}\href{../cars/works/PerronS04.pdf}{PerronS04}~\cite{PerronS04} & 15 & job, job-shop, order, resource, scheduling &  &  &  & Ilog Scheduler, OZ &  &  & generated instance &  & \ref{a:PerronS04} & \ref{c:PerronS04}\\
\rowlabel{b:ReginP97}\href{../cars/works/ReginP97.pdf}{ReginP97}~\cite{ReginP97} & 15 & machine, order, resource, scheduling &  & Cardinality constraint, cumulative &  & CHIP, Ilog Scheduler, Ilog Solver & automotive, crew-scheduling &  & benchmark, random instance, real-life & edge-finder, time-tabling & \ref{a:ReginP97} & \ref{c:ReginP97}\\
\end{longtable}
}



\clearpage
\subsection{Manually Defined Fields}
{\scriptsize
\begin{longtable}{>{\raggedright\arraybackslash}p{3cm}>{\raggedright\arraybackslash}p{6cm}lp{2cm}rrrrlp{2cm}p{2cm}rr}
\rowcolor{white}\caption{Manually Defined PAPER Properties}\\ \toprule
\rowcolor{white}Key & Title (Local Copy) & \shortstack{CP\\System} & Bench & Links & \shortstack{Data\\Avail} & \shortstack{Sol\\Avail} & \shortstack{Code\\Avail} & \shortstack{Related\\To} & Classification & Constraints & a & b\\ \midrule\endhead
\bottomrule
\endfoot
\rowlabel{c:ArtiguesHM0W14}ArtiguesHM0W14 \href{https://doi.org/10.1007/978-3-319-07046-9\_19}{ArtiguesHM0W14}~\cite{ArtiguesHM0W14} & \href{cars/works/ArtiguesHM0W14.pdf}{{SAT} and Hybrid Models of the Car Sequencing Problem} &  & github, CSPlib, Roadef, benchmark & 0 &  &  &  &  &  &  & \ref{a:ArtiguesHM0W14} & \ref{b:ArtiguesHM0W14}\\
\rowlabel{c:Mayer-EichbergerW13}Mayer-EichbergerW13 \href{https://doi.org/10.29007/jrsp}{Mayer-EichbergerW13}~\cite{Mayer-EichbergerW13} & \href{cars/works/Mayer-EichbergerW13.pdf}{{SAT} Encodings for the Car Sequencing Problem} &  & CSPlib, benchmark, github & 0 &  &  &  &  &  &  & \ref{a:Mayer-EichbergerW13} & \ref{b:Mayer-EichbergerW13}\\
\rowlabel{c:ThiruvadyME11}ThiruvadyME11 \href{}{ThiruvadyME11}~\cite{ThiruvadyME11} & \href{}{Car sequencing with constraint-based ACO} &  &  & 0 &  &  &  &  &  &  & \ref{a:ThiruvadyME11} & No\\
\rowlabel{c:HoevePRS06}HoevePRS06 \href{https://doi.org/10.1007/11889205\_44}{HoevePRS06}~\cite{HoevePRS06} & \href{cars/works/HoevePRS06.pdf}{Revisiting the Sequence Constraint} &  & real-life & 0 &  &  &  &  &  &  & \ref{a:HoevePRS06} & \ref{b:HoevePRS06}\\
\rowlabel{c:ButaruH05}ButaruH05 \href{}{ButaruH05}~\cite{ButaruH05} & \href{cars/works/ButaruH05.pdf}{The car-sequencing problem as n-ary CSP--Sequential and parallel solving} &  & CSPlib & 0 &  &  &  &  &  &  & \ref{a:ButaruH05} & \ref{b:ButaruH05}\\
\rowlabel{c:PerronS04}PerronS04 \href{https://doi.org/10.1007/978-3-540-24664-0_16}{PerronS04}~\cite{PerronS04} & \href{cars/works/PerronS04.pdf}{Combining Forces to Solve the Car Sequencing Problem} &  & generated instance & 0 &  &  &  &  &  &  & \ref{a:PerronS04} & \ref{b:PerronS04}\\
\rowlabel{c:GottliebPS03}GottliebPS03 \href{https://doi.org/10.1007/3-540-36605-9_23}{GottliebPS03}~\cite{GottliebPS03} & \href{cars/works/GottliebPS03.pdf}{A Study of Greedy, Local Search, and Ant Colony Optimization Approaches for Car Sequencing Problems} &  & real-world, benchmark, CSPlib & 0 &  &  &  &  &  &  & \ref{a:GottliebPS03} & \ref{b:GottliebPS03}\\
\rowlabel{c:ReginP97}ReginP97 \href{https://doi.org/10.1007/BFb0017428}{ReginP97}~\cite{ReginP97} & \href{cars/works/ReginP97.pdf}{A Filtering Algorithm for Global Sequencing Constraints} &  & random instance, benchmark, real-life & 0 &  &  &  &  &  &  & \ref{a:ReginP97} & \ref{b:ReginP97}\\
\rowlabel{c:DincbasSH88}DincbasSH88 \href{}{DincbasSH88}~\cite{DincbasSH88} & \href{cars/works/DincbasSH88.pdf}{Solving the Car-Sequencing Problem in Constraint Logic Programming} &  & real-life & 0 &  &  &  &  &  &  & \ref{a:DincbasSH88} & \ref{b:DincbasSH88}\\
\end{longtable}
}



\clearpage
\section{Journal Articles}

\clearpage
\subsection{Articles from bibtex}
{\scriptsize
\begin{longtable}{>{\raggedright\arraybackslash}p{3cm}>{\raggedright\arraybackslash}p{6cm}>{\raggedright\arraybackslash}p{6.5cm}rrrp{2.5cm}rrrrr}
\rowcolor{white}\caption{Works from bibtex (Total 19)}\\ \toprule
\rowcolor{white}\shortstack{Key\\Source} & Authors & Title & LC & Cite & Year & \shortstack{Conference\\/Journal\\/School} & Pages & \shortstack{Nr\\Cites} & \shortstack{Nr\\Refs} & b & c \\ \midrule\endhead
\bottomrule
\endfoot
\rowlabel{a:YuLZCLW22}YuLZCLW22 \href{http://dx.doi.org/10.1186/s13638-022-02113-7}{YuLZCLW22} & \hyperref[auth:a55]{Y. Yu}, \hyperref[auth:a56]{X. Lu}, \hyperref[auth:a57]{T. Zhao}, \hyperref[auth:a58]{M. Cheng}, \hyperref[auth:a59]{L. Liu}, \hyperref[auth:a60]{W. Wei} & Heuristic approaches for the car sequencing problems with block batches & No & \cite{YuLZCLW22} & 2022 & EURASIP Journal on Wireless Communications and Networking & null & 2 & 37 & No & \ref{c:YuLZCLW22}\\
\rowlabel{a:WinterM21}WinterM21 \href{}{WinterM21} & \hyperref[auth:a24]{F. Winter}, \hyperref[auth:a25]{N. Musliu} & Constraint-based Scheduling for Paint Shops in the Automotive Supply Industry & No & \cite{WinterM21} & 2021 & ACM Transactions on Intelligent Systems and Technology (TIST) & 25 & 0 & 0 & No & \ref{c:WinterM21}\\
\rowlabel{a:MoyaCB19}MoyaCB19 \href{http://dx.doi.org/10.1016/j.cie.2019.106048}{MoyaCB19} & \hyperref[auth:a63]{I. Moya}, \hyperref[auth:a64]{M. Chica}, \hyperref[auth:a65]{J. Bautista} & Constructive metaheuristics for solving the Car Sequencing Problem under uncertain partial demand & No & \cite{MoyaCB19} & 2019 & Computers \  Industrial Engineering & 1 & 8 & 44 & No & \ref{c:MoyaCB19}\\
\rowlabel{a:YavuzE18}YavuzE18 \href{}{YavuzE18} & \hyperref[auth:a22]{M. Yavuz}, \hyperref[auth:a23]{H. Ergin} & Advanced constraint propagation for the combined car sequencing and level scheduling problem & No & \cite{YavuzE18} & 2018 & Computers \  Operations Research & 12 & 0 & 0 & No & \ref{c:YavuzE18}\\
\rowlabel{a:ZhangGWH17}ZhangGWH17 \href{http://dx.doi.org/10.1007/s10033-017-0083-7}{ZhangGWH17} & \hyperref[auth:a51]{X. ZHANG}, \hyperref[auth:a52]{L. GAO}, \hyperref[auth:a53]{L. WEN}, \hyperref[auth:a54]{Z. HUANG} & Parallel Construction Heuristic Combined with Constraint Propagation for the Car Sequencing Problem & No & \cite{ZhangGWH17} & 2017 & Chinese Journal of Mechanical Engineering & null & 3 & 32 & No & \ref{c:ZhangGWH17}\\
\rowlabel{a:MazurN15}MazurN15 \href{}{MazurN15} & \hyperref[auth:a31]{M. Mazur}, \hyperref[auth:a32]{A. Niederli{\'n}ski} & A Two-stage approach for an optimum solution of the car assembly scheduling problem. Part 2. CLP solution and real-world example & No & \cite{MazurN15} & 2015 & Archives of Control Sciences & 9 & 0 & 0 & No & \ref{c:MazurN15}\\
\rowlabel{a:SialaHH155}SialaHH155 \href{https://doi.org/10.1016/j.engappai.2014.10.009}{SialaHH155} & \hyperref[auth:a11]{M. Siala}, \hyperref[auth:a12]{E. Hebrard}, \hyperref[auth:a13]{M. Huguet} & A study of constraint programming heuristics for the car-sequencing problem & No & \cite{SialaHH155} & 2015 & Eng. Appl. Artif. Intell. & 11 & 15 & 10 & No & \ref{c:SialaHH155}\\
\rowlabel{a:GolleRB14}GolleRB14 \href{http://dx.doi.org/10.1007/s10479-014-1733-0}{GolleRB14} & \hyperref[auth:a61]{U. Golle}, \hyperref[auth:a62]{F. Rothlauf}, \hyperref[auth:a48]{N. Boysen} & Iterative beam search for car sequencing & No & \cite{GolleRB14} & 2014 & Annals of Operations Research & null & 15 & 15 & No & \ref{c:GolleRB14}\\
\rowlabel{a:SialaHH14}SialaHH14 \href{https://doi.org/10.1007/s10601-013-9150-6}{SialaHH14} & \hyperref[auth:a11]{M. Siala}, \hyperref[auth:a12]{E. Hebrard}, \hyperref[auth:a13]{M. Huguet} & An optimal arc consistency algorithm for a particular case of sequence constraint & \href{../cars/works/SialaHH14.pdf}{Yes} & \cite{SialaHH14} & 2014 & Constraints An Int. J. & 27 & 3 & 14 & \ref{b:SialaHH14} & \ref{c:SialaHH14}\\
\rowlabel{a:OzturkTHO13}OzturkTHO13 \href{https://doi.org/10.1007/s10601-013-9142-6}{OzturkTHO13} & \hyperref[auth:a14]{C. {\"{O}}zt{\"{u}}rk}, \hyperref[auth:a15]{S. Tunali}, \hyperref[auth:a16]{B. Hnich}, \hyperref[auth:a17]{M. Arslan Ornek} & Balancing and scheduling of flexible mixed model assembly lines & No & \cite{OzturkTHO13} & 2013 & Constraints An Int. J. & 36 & 31 & 44 & No & \ref{c:OzturkTHO13}\\
\rowlabel{a:BoysenFS09}BoysenFS09 \href{http://dx.doi.org/10.1016/j.ejor.2007.09.013}{BoysenFS09} & \hyperref[auth:a48]{N. Boysen}, \hyperref[auth:a49]{M. Fliedner}, \hyperref[auth:a50]{A. Scholl} & Sequencing mixed-model assembly lines: Survey,  classification and model critique & No & \cite{BoysenFS09} & 2009 & European Journal of Operational Research & null & 308 & 167 & No & \ref{c:BoysenFS09}\\
\rowlabel{a:HoevePRS09}HoevePRS09 \href{http://dx.doi.org/10.1007/s10601-008-9067-7}{HoevePRS09} & \hyperref[auth:a39]{Willem-Jan van Hoeve}, \hyperref[auth:a40]{G. Pesant}, \hyperref[auth:a41]{L. Rousseau}, \hyperref[auth:a42]{A. Sabharwal} & New filtering algorithms for combinations of among constraints & No & \cite{HoevePRS09} & 2009 & Constraints An Int. J. & null & 13 & 8 & No & \ref{c:HoevePRS09}\\
\rowlabel{a:Schaus09}Schaus09 \href{}{Schaus09} & \hyperref[auth:a19]{P. Schaus} & Solving balancing and bin-packing problems with constraint programming & No & \cite{Schaus09} & 2009 & These de doctorat, Universit{\'e} catholique de Louvain & null & 0 & 0 & No & \ref{c:Schaus09}\\
\rowlabel{a:SolnonCNA08}SolnonCNA08 \href{https://doi.org/10.1016/j.ejor.2007.04.033}{SolnonCNA08} & \hyperref[auth:a5]{C. Solnon}, \hyperref[auth:a6]{V. Cung}, \hyperref[auth:a7]{A. Nguyen}, \hyperref[auth:a8]{C. Artigues} & The car sequencing problem: Overview of state-of-the-art methods and industrial case-study of the ROADEF'2005 challenge problem & No & \cite{SolnonCNA08} & 2008 & European Journal of Operational Research & 16 & 146 & 22 & No & \ref{c:SolnonCNA08}\\
\rowlabel{a:Kis04}Kis04 \href{http://dx.doi.org/10.1016/j.orl.2003.09.003}{Kis04} & \hyperref[auth:a47]{T. Kis} & On the complexity of the car sequencing problem & No & \cite{Kis04} & 2004 & Operations Research Letters & null & 69 & 3 & No & \ref{c:Kis04}\\
\rowlabel{a:Gent98}Gent98 \href{}{Gent98} & \hyperref[auth:a18]{Ian P Gent} & Two results on car-sequencing problems & No & \cite{Gent98} & 1998 & Report University of Strathclyde, APES-02-98 & null & 0 & 0 & No & \ref{c:Gent98}\\
\rowlabel{a:WarwickT95}WarwickT95 \href{http://dx.doi.org/10.1162/evco.1995.3.3.267}{WarwickT95} & \hyperref[auth:a45]{T. Warwick}, \hyperref[auth:a46]{Edward P. K. Tsang} & Tackling Car Sequencing Problems Using a Generic Genetic Algorithm & No & \cite{WarwickT95} & 1995 & Evolutionary Computation & null & 28 & 0 & No & \ref{c:WarwickT95}\\
\rowlabel{a:HindiP94}HindiP94 \href{http://dx.doi.org/10.1016/0360-8352(94)90038-8}{HindiP94} & \hyperref[auth:a37]{Khalil S. Hindi}, \hyperref[auth:a38]{G. Ploszajski} & Formulation and solution of a selection and sequencing problem in car manufacture & No & \cite{HindiP94} & 1994 & Computers \  Industrial Engineering & null & 24 & 4 & No & \ref{c:HindiP94}\\
\rowlabel{a:ParrelloK86}ParrelloK86 \href{https://doi.org/10.1007/BF00246021}{ParrelloK86} & \hyperref[auth:a0]{Bruce D. Parrello}, \hyperref[auth:a1]{Waldo C. Kabat} & Job-Shop Scheduling Using Automated Reasoning: {A} Case Study of the Car-Sequencing Problem & \href{../cars/works/ParrelloK86.pdf}{Yes} & \cite{ParrelloK86} & 1986 & J. Autom. Reason. & 42 & 74 & 0 & \ref{b:ParrelloK86} & \ref{c:ParrelloK86}\\
\end{longtable}
}




\clearpage
\subsection{Extracted Concepts}
{\scriptsize
\begin{longtable}{>{\raggedright\arraybackslash}p{3cm}r>{\raggedright\arraybackslash}p{4cm}p{1.5cm}p{2cm}p{1.5cm}p{1.5cm}p{1.5cm}p{1.5cm}p{2cm}p{1.5cm}rr}
\rowcolor{white}\caption{Automatically Extracted ARTICLE Properties (Requires Local Copy)}\\ \toprule
\rowcolor{white}Work & Pages & Concepts & Classification & Constraints & \shortstack{Prog\\Languages} & \shortstack{CP\\Systems} & Areas & Industries & Benchmarks & Algorithm & a & c\\ \midrule\endhead
\bottomrule
\endfoot
\rowlabel{b:BoysenFS09}\href{../cars/works/BoysenFS09.pdf}{BoysenFS09}~\cite{BoysenFS09} & 25 & distributed, due-date, inventory, job, job-shop, machine, multi-agent, order, precedence, preempt, resource, scheduling, setup-time, task, transportation &  & cumulative, cycle &  & OZ & automotive & automobile industry, automotive industry & Roadef, real-life, real-world &  & \ref{a:BoysenFS09} & \ref{c:BoysenFS09}\\
\rowlabel{b:GolleRB14}\href{../cars/works/GolleRB14.pdf}{GolleRB14}~\cite{GolleRB14} & 16 & job, job-shop, order, resource, scheduling &  & cycle & Java &  &  &  & CSPlib, Roadef, real-life, real-world &  & \ref{a:GolleRB14} & \ref{c:GolleRB14}\\
\rowlabel{b:HoevePRS09}\href{../cars/works/HoevePRS09.pdf}{HoevePRS09}~\cite{HoevePRS09} & 20 & machine, order, scheduling &  & Among constraint, Cardinality constraint, GCC constraint, Regular constraint, cumulative &  & CHIP, Ilog Solver & nurse &  & CSPlib, benchmark, real-life & time-tabling & \ref{a:HoevePRS09} & \ref{c:HoevePRS09}\\
\rowlabel{b:Kis04}\href{../cars/works/Kis04.pdf}{Kis04}~\cite{Kis04} & 5 & job, job-shop, order, scheduling &  &  &  &  &  &  & benchmark &  & \ref{a:Kis04} & \ref{c:Kis04}\\
\rowlabel{b:MoyaCB19}\href{../cars/works/MoyaCB19.pdf}{MoyaCB19}~\cite{MoyaCB19} & 13 & distributed, flow-shop, job, job-shop, make-span, order, scheduling, task &  & cycle & Java & OZ & crew-scheduling, railway, robot &  & CSPlib, Roadef, benchmark, bitbucket, generated instance, real-life & GRASP & \ref{a:MoyaCB19} & \ref{c:MoyaCB19}\\
\rowlabel{b:OzturkTHO13}\href{../cars/works/OzturkTHO13.pdf}{OzturkTHO13}~\cite{OzturkTHO13} & 36 & activity, cmax, completion-time, flow-shop, job, machine, make-span, order, precedence, preempt, resource, scheduling, setup-time, task & SBSFMMAL & Channeling constraint, Disjunctive constraint, cumulative, cycle, disjunctive &  & CHIP, Cplex, Ilog Solver, OPL, OZ &  &  & real-life, real-world & edge-finding & \ref{a:OzturkTHO13} & \ref{c:OzturkTHO13}\\
\rowlabel{b:ParrelloK86}\href{../cars/works/ParrelloK86.pdf}{ParrelloK86}~\cite{ParrelloK86} & 42 & job-shop, machine, scheduling, job, order &  &  & Prolog & OPL & nurse &  & real-life &  & \ref{a:ParrelloK86} & \ref{c:ParrelloK86}\\
\rowlabel{b:SialaHH14}\href{../cars/works/SialaHH14.pdf}{SialaHH14}~\cite{SialaHH14} & 27 & resource, scheduling, order &  & AtMostSeqCard, Atmost constraint, Cardinality constraint, AmongSeq constraint, CardPath, Regular constraint, MultiAtMostSeqCard, AtMostSeq, Among constraint &  & CHIP &  &  & Roadef, CSPlib, benchmark &  & \ref{a:SialaHH14} & \ref{c:SialaHH14}\\
\rowlabel{b:SialaHH155}\href{../cars/works/SialaHH155.pdf}{SialaHH155}~\cite{SialaHH155} & 11 & machine, order, resource, task &  & Among constraint, AtMostSeq, AtMostSeqCard &  & CHIP & automotive & automotive industry & CSPlib, Roadef, benchmark &  & \ref{a:SialaHH155} & \ref{c:SialaHH155}\\
\rowlabel{b:SolnonCNA08}\href{../cars/works/SolnonCNA08.pdf}{SolnonCNA08}~\cite{SolnonCNA08} & 16 & distributed, due-date, inventory, job, job-shop, order, scheduling, task &  &  & C++ & CHIP, Ilog Solver, OZ &  &  & CSPlib, Roadef, benchmark, generated instance, industrial instance, industrial partner, real-life &  & \ref{a:SolnonCNA08} & \ref{c:SolnonCNA08}\\
\rowlabel{b:YuLZCLW22}\href{../cars/works/YuLZCLW22.pdf}{YuLZCLW22}~\cite{YuLZCLW22} & 17 & BOM, bill of material, inventory, job, job-shop, machine, order, resource, scheduling, setup-time & parallel machine &  & C++, Java & Cplex, Gurobi & automotive, car manufacturing & automobile industry & CSPlib, Roadef, benchmark, real-life, real-world & GRASP & \ref{a:YuLZCLW22} & \ref{c:YuLZCLW22}\\
\rowlabel{b:ZhangGWH17}\href{../cars/works/ZhangGWH17.pdf}{ZhangGWH17}~\cite{ZhangGWH17} & 12 & job, job-shop, machine, order, scheduling, transportation &  &  &  & Cplex &  &  & CSPlib, Roadef, benchmark, real-life & GRASP & \ref{a:ZhangGWH17} & \ref{c:ZhangGWH17}\\
\end{longtable}
}



\clearpage
\subsection{Manually Defined Fields}
{\scriptsize
\begin{longtable}{>{\raggedright\arraybackslash}p{3cm}>{\raggedright\arraybackslash}p{6cm}lp{2cm}rrrrlp{2cm}p{2cm}rr}
\rowcolor{white}\caption{Manually Defined ARTICLE Properties}\\ \toprule
\rowcolor{white}Key & Title (Local Copy) & \shortstack{CP\\System} & Bench & Links & \shortstack{Data\\Avail} & \shortstack{Sol\\Avail} & \shortstack{Code\\Avail} & \shortstack{Related\\To} & Classification & Constraints & a & b\\ \midrule\endhead
\bottomrule
\endfoot
\rowlabel{c:YuLZCLW22}YuLZCLW22 \href{http://dx.doi.org/10.1186/s13638-022-02113-7}{YuLZCLW22}~\cite{YuLZCLW22} & \href{../cars/works/YuLZCLW22.pdf}{Heuristic approaches for the car sequencing problems with block batches} &  & CSPlib, Roadef, benchmark, real-life, real-world & 0 &  &  &  &  &  &  & \ref{a:YuLZCLW22} & \ref{b:YuLZCLW22}\\
\rowlabel{c:WinterM21}WinterM21 \href{}{WinterM21}~\cite{WinterM21} & \href{../}{Constraint-based Scheduling for Paint Shops in the Automotive Supply Industry} &  &  & 0 &  &  &  &  &  &  & \ref{a:WinterM21} & No\\
\rowlabel{c:MoyaCB19}MoyaCB19 \href{http://dx.doi.org/10.1016/j.cie.2019.106048}{MoyaCB19}~\cite{MoyaCB19} & \href{../cars/works/MoyaCB19.pdf}{Constructive metaheuristics for solving the Car Sequencing Problem under uncertain partial demand} &  & CSPlib, Roadef, benchmark, bitbucket, generated instance, real-life & 1 &  &  &  &  &  &  & \ref{a:MoyaCB19} & \ref{b:MoyaCB19}\\
\rowlabel{c:YavuzE18}YavuzE18 \href{}{YavuzE18}~\cite{YavuzE18} & \href{../}{Advanced constraint propagation for the combined car sequencing and level scheduling problem} &  &  & 0 &  &  &  &  &  &  & \ref{a:YavuzE18} & No\\
\rowlabel{c:ZhangGWH17}ZhangGWH17 \href{http://dx.doi.org/10.1007/s10033-017-0083-7}{ZhangGWH17}~\cite{ZhangGWH17} & \href{../cars/works/ZhangGWH17.pdf}{Parallel Construction Heuristic Combined with Constraint Propagation for the Car Sequencing Problem} &  & CSPlib, Roadef, benchmark, real-life & 1 &  &  &  &  &  &  & \ref{a:ZhangGWH17} & \ref{b:ZhangGWH17}\\
\rowlabel{c:MazurN15}MazurN15 \href{}{MazurN15}~\cite{MazurN15} & \href{../}{A Two-stage approach for an optimum solution of the car assembly scheduling problem. Part 2. CLP solution and real-world example} &  &  & 0 &  &  &  &  &  &  & \ref{a:MazurN15} & No\\
\rowlabel{c:SialaHH155}SialaHH155 \href{https://doi.org/10.1016/j.engappai.2014.10.009}{SialaHH155}~\cite{SialaHH155} & \href{../cars/works/SialaHH155.pdf}{A study of constraint programming heuristics for the car-sequencing problem} &  & CSPlib, Roadef, benchmark & 2 &  &  &  &  &  &  & \ref{a:SialaHH155} & \ref{b:SialaHH155}\\
\rowlabel{c:GolleRB14}GolleRB14 \href{http://dx.doi.org/10.1007/s10479-014-1733-0}{GolleRB14}~\cite{GolleRB14} & \href{../cars/works/GolleRB14.pdf}{Iterative beam search for car sequencing} &  & CSPlib, Roadef, real-life, real-world & 0 &  &  &  &  &  &  & \ref{a:GolleRB14} & \ref{b:GolleRB14}\\
\rowlabel{c:SialaHH14}SialaHH14 \href{https://doi.org/10.1007/s10601-013-9150-6}{SialaHH14}~\cite{SialaHH14} & \href{../cars/works/SialaHH14.pdf}{An optimal arc consistency algorithm for a particular case of sequence constraint} &  & Roadef, CSPlib, benchmark & 0 &  &  &  &  &  &  & \ref{a:SialaHH14} & \ref{b:SialaHH14}\\
\rowlabel{c:OzturkTHO13}OzturkTHO13 \href{https://doi.org/10.1007/s10601-013-9142-6}{OzturkTHO13}~\cite{OzturkTHO13} & \href{../cars/works/OzturkTHO13.pdf}{Balancing and scheduling of flexible mixed model assembly lines} &  & real-life, real-world & 2 &  &  &  &  &  &  & \ref{a:OzturkTHO13} & \ref{b:OzturkTHO13}\\
\rowlabel{c:BoysenFS09}BoysenFS09 \href{http://dx.doi.org/10.1016/j.ejor.2007.09.013}{BoysenFS09}~\cite{BoysenFS09} & \href{../cars/works/BoysenFS09.pdf}{Sequencing mixed-model assembly lines: Survey,  classification and model critique} &  & Roadef, real-life, real-world & 0 &  &  &  &  &  &  & \ref{a:BoysenFS09} & \ref{b:BoysenFS09}\\
\rowlabel{c:HoevePRS09}HoevePRS09 \href{http://dx.doi.org/10.1007/s10601-008-9067-7}{HoevePRS09}~\cite{HoevePRS09} & \href{../cars/works/HoevePRS09.pdf}{New filtering algorithms for combinations of among constraints} &  & CSPlib, benchmark, real-life & 1 &  &  &  &  &  &  & \ref{a:HoevePRS09} & \ref{b:HoevePRS09}\\
\rowlabel{c:Schaus09}Schaus09 \href{}{Schaus09}~\cite{Schaus09} & \href{../}{Solving balancing and bin-packing problems with constraint programming} &  &  & 0 &  &  &  &  &  &  & \ref{a:Schaus09} & No\\
\rowlabel{c:SolnonCNA08}SolnonCNA08 \href{https://doi.org/10.1016/j.ejor.2007.04.033}{SolnonCNA08}~\cite{SolnonCNA08} & \href{../cars/works/SolnonCNA08.pdf}{The car sequencing problem: Overview of state-of-the-art methods and industrial case-study of the ROADEF'2005 challenge problem} &  & CSPlib, Roadef, benchmark, generated instance, industrial instance, industrial partner, real-life & 4 &  &  &  &  &  &  & \ref{a:SolnonCNA08} & \ref{b:SolnonCNA08}\\
\rowlabel{c:Kis04}Kis04 \href{http://dx.doi.org/10.1016/j.orl.2003.09.003}{Kis04}~\cite{Kis04} & \href{../cars/works/Kis04.pdf}{On the complexity of the car sequencing problem} &  & benchmark & 0 &  &  &  &  &  &  & \ref{a:Kis04} & \ref{b:Kis04}\\
\rowlabel{c:Gent98}Gent98 \href{}{Gent98}~\cite{Gent98} & \href{../}{Two results on car-sequencing problems} &  &  & 0 &  &  &  &  &  &  & \ref{a:Gent98} & No\\
\rowlabel{c:WarwickT95}WarwickT95 \href{http://dx.doi.org/10.1162/evco.1995.3.3.267}{WarwickT95}~\cite{WarwickT95} & \href{../}{Tackling Car Sequencing Problems Using a Generic Genetic Algorithm} &  &  & 0 &  &  &  &  &  &  & \ref{a:WarwickT95} & No\\
\rowlabel{c:HindiP94}HindiP94 \href{http://dx.doi.org/10.1016/0360-8352(94)90038-8}{HindiP94}~\cite{HindiP94} & \href{../}{Formulation and solution of a selection and sequencing problem in car manufacture} &  &  & 0 &  &  &  &  &  &  & \ref{a:HindiP94} & No\\
\rowlabel{c:ParrelloK86}ParrelloK86 \href{https://doi.org/10.1007/BF00246021}{ParrelloK86}~\cite{ParrelloK86} & \href{../cars/works/ParrelloK86.pdf}{Job-Shop Scheduling Using Automated Reasoning: {A} Case Study of the Car-Sequencing Problem} &  & real-life & 0 &  &  &  &  &  &  & \ref{a:ParrelloK86} & \ref{b:ParrelloK86}\\
\end{longtable}
}



\clearpage
\section{Authors}

{\scriptsize
\begin{longtable}{p{4cm}rrp{18cm}}
\rowcolor{white}\caption{Co-Authors of Articles/Papers}\\ \toprule
\rowcolor{white}Author & \shortstack{Nr\\Works} & \shortstack{Nr\\Cites} & Entries \\ \midrule\endhead
\bottomrule
\endfoot
\rowlabel{auth:a11}Mohamed Siala & 4 &20 &\href{../cars/works/Siala15.pdf}{Siala15}~\cite{Siala15}, \href{../cars/works/SialaHH155.pdf}{SialaHH155}~\cite{SialaHH155}, \href{../cars/works/SialaHH14.pdf}{SialaHH14}~\cite{SialaHH14}, \href{../cars/works/ArtiguesHM0W14.pdf}{ArtiguesHM0W14}~\cite{ArtiguesHM0W14}\\
\rowlabel{auth:a12}Emmanuel Hebrard & 3 &20 &\href{../cars/works/SialaHH155.pdf}{SialaHH155}~\cite{SialaHH155}, \href{../cars/works/SialaHH14.pdf}{SialaHH14}~\cite{SialaHH14}, \href{../cars/works/ArtiguesHM0W14.pdf}{ArtiguesHM0W14}~\cite{ArtiguesHM0W14}\\
\rowlabel{auth:a8}Christian Artigues & 2 &148 &\href{../cars/works/ArtiguesHM0W14.pdf}{ArtiguesHM0W14}~\cite{ArtiguesHM0W14}, \href{../cars/works/SolnonCNA08.pdf}{SolnonCNA08}~\cite{SolnonCNA08}\\
\rowlabel{auth:a48}Nils Boysen & 2 &323 &\href{../cars/works/GolleRB14.pdf}{GolleRB14}~\cite{GolleRB14}, \href{../cars/works/BoysenFS09.pdf}{BoysenFS09}~\cite{BoysenFS09}\\
\rowlabel{auth:a13}Marie{-}Jos{\'{e}} Huguet & 2 &18 &\href{../cars/works/SialaHH155.pdf}{SialaHH155}~\cite{SialaHH155}, \href{../cars/works/SialaHH14.pdf}{SialaHH14}~\cite{SialaHH14}\\
\rowlabel{auth:a35}Valentin Mayer{-}Eichberger & 2 &2 &\href{../cars/works/ArtiguesHM0W14.pdf}{ArtiguesHM0W14}~\cite{ArtiguesHM0W14}, \href{../cars/works/Mayer-EichbergerW13.pdf}{Mayer-EichbergerW13}~\cite{Mayer-EichbergerW13}\\
\rowlabel{auth:a40}Gilles Pesant & 2 &46 &\href{../cars/works/HoevePRS09.pdf}{HoevePRS09}~\cite{HoevePRS09}, \href{../cars/works/HoevePRS06.pdf}{HoevePRS06}~\cite{HoevePRS06}\\
\rowlabel{auth:a42}Ashish Sabharwal & 2 &46 &\href{../cars/works/HoevePRS09.pdf}{HoevePRS09}~\cite{HoevePRS09}, \href{../cars/works/HoevePRS06.pdf}{HoevePRS06}~\cite{HoevePRS06}\\
\rowlabel{auth:a5}Christine Solnon & 2 &192 &\href{../cars/works/SolnonCNA08.pdf}{SolnonCNA08}~\cite{SolnonCNA08}, \href{../cars/works/GottliebPS03.pdf}{GottliebPS03}~\cite{GottliebPS03}\\
\rowlabel{auth:a36}Toby Walsh & 2 &2 &\href{../cars/works/ArtiguesHM0W14.pdf}{ArtiguesHM0W14}~\cite{ArtiguesHM0W14}, \href{../cars/works/Mayer-EichbergerW13.pdf}{Mayer-EichbergerW13}~\cite{Mayer-EichbergerW13}\\
\rowlabel{auth:a17}M. Arslan Ornek & 1 &31 &\href{../cars/works/OzturkTHO13.pdf}{OzturkTHO13}~\cite{OzturkTHO13}\\
\rowlabel{auth:a65}Joaquín Bautista & 1 &8 &\href{../cars/works/MoyaCB19.pdf}{MoyaCB19}~\cite{MoyaCB19}\\
\rowlabel{auth:a29}Mihaela Butaru & 1 &0 &\href{../cars/works/ButaruH05.pdf}{ButaruH05}~\cite{ButaruH05}\\
\rowlabel{auth:a1}Waldo C. Kabat & 1 &74 &\href{../cars/works/ParrelloK86.pdf}{ParrelloK86}~\cite{ParrelloK86}\\
\rowlabel{auth:a58}Minjiao Cheng & 1 &2 &\href{../cars/works/YuLZCLW22.pdf}{YuLZCLW22}~\cite{YuLZCLW22}\\
\rowlabel{auth:a64}Manuel Chica & 1 &8 &\href{../cars/works/MoyaCB19.pdf}{MoyaCB19}~\cite{MoyaCB19}\\
\rowlabel{auth:a6}Van{-}Dat Cung & 1 &146 &\href{../cars/works/SolnonCNA08.pdf}{SolnonCNA08}~\cite{SolnonCNA08}\\
\rowlabel{auth:a0}Bruce D. Parrello & 1 &74 &\href{../cars/works/ParrelloK86.pdf}{ParrelloK86}~\cite{ParrelloK86}\\
\rowlabel{auth:a2}Mehmet Dincbas & 1 &0 &\href{../cars/works/DincbasSH88.pdf}{DincbasSH88}~\cite{DincbasSH88}\\
\rowlabel{auth:a23}H{\"u}seyin Ergin & 1 &0 &\href{../}{YavuzE18}~\cite{YavuzE18}\\
\rowlabel{auth:a28}Andreas Ernst & 1 &0 &\href{../}{ThiruvadyME11}~\cite{ThiruvadyME11}\\
\rowlabel{auth:a49}Malte Fliedner & 1 &308 &\href{../cars/works/BoysenFS09.pdf}{BoysenFS09}~\cite{BoysenFS09}\\
\rowlabel{auth:a52}Liang GAO & 1 &3 &\href{../cars/works/ZhangGWH17.pdf}{ZhangGWH17}~\cite{ZhangGWH17}\\
\rowlabel{auth:a61}Uli Golle & 1 &15 &\href{../cars/works/GolleRB14.pdf}{GolleRB14}~\cite{GolleRB14}\\
\rowlabel{auth:a9}Jens Gottlieb & 1 &46 &\href{../cars/works/GottliebPS03.pdf}{GottliebPS03}~\cite{GottliebPS03}\\
\rowlabel{auth:a54}Zhaodong HUANG & 1 &3 &\href{../cars/works/ZhangGWH17.pdf}{ZhangGWH17}~\cite{ZhangGWH17}\\
\rowlabel{auth:a30}Zineb Habbas & 1 &0 &\href{../cars/works/ButaruH05.pdf}{ButaruH05}~\cite{ButaruH05}\\
\rowlabel{auth:a16}Brahim Hnich & 1 &31 &\href{../cars/works/OzturkTHO13.pdf}{OzturkTHO13}~\cite{OzturkTHO13}\\
\rowlabel{auth:a43}Willem Jan van Hoeve & 1 &33 &\href{../cars/works/HoevePRS06.pdf}{HoevePRS06}~\cite{HoevePRS06}\\
\rowlabel{auth:a47}Tamás Kis & 1 &69 &\href{../cars/works/Kis04.pdf}{Kis04}~\cite{Kis04}\\
\rowlabel{auth:a59}Lin Liu & 1 &2 &\href{../cars/works/YuLZCLW22.pdf}{YuLZCLW22}~\cite{YuLZCLW22}\\
\rowlabel{auth:a56}Xiaochun Lu & 1 &2 &\href{../cars/works/YuLZCLW22.pdf}{YuLZCLW22}~\cite{YuLZCLW22}\\
\rowlabel{auth:a31}Micha{\l} Mazur & 1 &0 &\href{../}{MazurN15}~\cite{MazurN15}\\
\rowlabel{auth:a27}Bernd Meyer & 1 &0 &\href{../}{ThiruvadyME11}~\cite{ThiruvadyME11}\\
\rowlabel{auth:a63}Ignacio Moya & 1 &8 &\href{../cars/works/MoyaCB19.pdf}{MoyaCB19}~\cite{MoyaCB19}\\
\rowlabel{auth:a25}Nysret Musliu & 1 &0 &\href{../}{WinterM21}~\cite{WinterM21}\\
\rowlabel{auth:a7}Alain Nguyen & 1 &146 &\href{../cars/works/SolnonCNA08.pdf}{SolnonCNA08}~\cite{SolnonCNA08}\\
\rowlabel{auth:a32}Antoni Niederli{\'n}ski & 1 &0 &\href{../}{MazurN15}~\cite{MazurN15}\\
\rowlabel{auth:a18}Ian P Gent & 1 &0 &\href{../}{Gent98}~\cite{Gent98}\\
\rowlabel{auth:a46}Edward P. K. Tsang & 1 &28 &\href{../}{WarwickT95}~\cite{WarwickT95}\\
\rowlabel{auth:a20}Laurent Perron & 1 &17 &\href{../cars/works/PerronS04.pdf}{PerronS04}~\cite{PerronS04}\\
\rowlabel{auth:a38}Grzegorz Ploszajski & 1 &24 &\href{../}{HindiP94}~\cite{HindiP94}\\
\rowlabel{auth:a10}Markus Puchta & 1 &46 &\href{../cars/works/GottliebPS03.pdf}{GottliebPS03}~\cite{GottliebPS03}\\
\rowlabel{auth:a34}Jean{-}Francois Puget & 1 &53 &\href{../cars/works/ReginP97.pdf}{ReginP97}~\cite{ReginP97}\\
\rowlabel{auth:a26}Dhananjay Raghavan Thiruvady & 1 &0 &\href{../}{ThiruvadyME11}~\cite{ThiruvadyME11}\\
\rowlabel{auth:a62}Franz Rothlauf & 1 &15 &\href{../cars/works/GolleRB14.pdf}{GolleRB14}~\cite{GolleRB14}\\
\rowlabel{auth:a41}Louis-Martin Rousseau & 1 &13 &\href{../cars/works/HoevePRS09.pdf}{HoevePRS09}~\cite{HoevePRS09}\\
\rowlabel{auth:a44}Louis{-}Martin Rousseau & 1 &33 &\href{../cars/works/HoevePRS06.pdf}{HoevePRS06}~\cite{HoevePRS06}\\
\rowlabel{auth:a33}Jean{-}Charles R{\'{e}}gin & 1 &53 &\href{../cars/works/ReginP97.pdf}{ReginP97}~\cite{ReginP97}\\
\rowlabel{auth:a37}Khalil S. Hindi & 1 &24 &\href{../}{HindiP94}~\cite{HindiP94}\\
\rowlabel{auth:a19}Pierre Schaus & 1 &0 &\href{../}{Schaus09}~\cite{Schaus09}\\
\rowlabel{auth:a50}Armin Scholl & 1 &308 &\href{../cars/works/BoysenFS09.pdf}{BoysenFS09}~\cite{BoysenFS09}\\
\rowlabel{auth:a21}Paul Shaw & 1 &17 &\href{../cars/works/PerronS04.pdf}{PerronS04}~\cite{PerronS04}\\
\rowlabel{auth:a3}Helmut Simonis & 1 &0 &\href{../cars/works/DincbasSH88.pdf}{DincbasSH88}~\cite{DincbasSH88}\\
\rowlabel{auth:a15}Semra Tunali & 1 &31 &\href{../cars/works/OzturkTHO13.pdf}{OzturkTHO13}~\cite{OzturkTHO13}\\
\rowlabel{auth:a4}Pascal Van Hentenryck & 1 &0 &\href{../cars/works/DincbasSH88.pdf}{DincbasSH88}~\cite{DincbasSH88}\\
\rowlabel{auth:a53}Long WEN & 1 &3 &\href{../cars/works/ZhangGWH17.pdf}{ZhangGWH17}~\cite{ZhangGWH17}\\
\rowlabel{auth:a45}Terry Warwick & 1 &28 &\href{../}{WarwickT95}~\cite{WarwickT95}\\
\rowlabel{auth:a60}Wenchao Wei & 1 &2 &\href{../cars/works/YuLZCLW22.pdf}{YuLZCLW22}~\cite{YuLZCLW22}\\
\rowlabel{auth:a24}Felix Winter & 1 &0 &\href{../}{WinterM21}~\cite{WinterM21}\\
\rowlabel{auth:a22}Mesut Yavuz & 1 &0 &\href{../}{YavuzE18}~\cite{YavuzE18}\\
\rowlabel{auth:a55}Yingjie Yu & 1 &2 &\href{../cars/works/YuLZCLW22.pdf}{YuLZCLW22}~\cite{YuLZCLW22}\\
\rowlabel{auth:a51}Xiangyang ZHANG & 1 &3 &\href{../cars/works/ZhangGWH17.pdf}{ZhangGWH17}~\cite{ZhangGWH17}\\
\rowlabel{auth:a57}Tao Zhao & 1 &2 &\href{../cars/works/YuLZCLW22.pdf}{YuLZCLW22}~\cite{YuLZCLW22}\\
\rowlabel{auth:a39}Willem-Jan van Hoeve & 1 &13 &\href{../cars/works/HoevePRS09.pdf}{HoevePRS09}~\cite{HoevePRS09}\\
\rowlabel{auth:a14}Cemalettin {\"{O}}zt{\"{u}}rk & 1 &31 &\href{../cars/works/OzturkTHO13.pdf}{OzturkTHO13}~\cite{OzturkTHO13}\\
\end{longtable}
}



\clearpage
\section{Most Cited Works}

{\scriptsize
\begin{longtable}{>{\raggedright\arraybackslash}p{3cm}>{\raggedright\arraybackslash}p{6cm}>{\raggedright\arraybackslash}p{6.5cm}rrrp{2.5cm}rrrrr}
\rowcolor{white}\caption{Works from bibtex (Total 29)}\\ \toprule
\rowcolor{white}Key & Authors & Title & LC & Cite & Year & \shortstack{Conference\\/Journal} & Pages & \shortstack{Nr\\Cites} & \shortstack{Nr\\Refs} & b & c \\ \midrule\endhead
\bottomrule
\endfoot
BoysenFS09 \href{http://dx.doi.org/10.1016/j.ejor.2007.09.013}{BoysenFS09} & \hyperref[auth:a48]{N. Boysen}, \hyperref[auth:a49]{M. Fliedner}, \hyperref[auth:a50]{A. Scholl} & Sequencing mixed-model assembly lines: Survey,  classification and model critique & No & \cite{BoysenFS09} & 2009 & European Journal of Operational Research & null & 308 & 167 & No & \ref{c:BoysenFS09}\\
SolnonCNA08 \href{https://doi.org/10.1016/j.ejor.2007.04.033}{SolnonCNA08} & \hyperref[auth:a5]{C. Solnon}, \hyperref[auth:a6]{V. Cung}, \hyperref[auth:a7]{A. Nguyen}, \hyperref[auth:a8]{C. Artigues} & The car sequencing problem: Overview of state-of-the-art methods and industrial case-study of the ROADEF'2005 challenge problem & No & \cite{SolnonCNA08} & 2008 & European Journal of Operational Research & 16 & 146 & 22 & No & \ref{c:SolnonCNA08}\\
ParrelloK86 \href{https://doi.org/10.1007/BF00246021}{ParrelloK86} & \hyperref[auth:a0]{Bruce D. Parrello}, \hyperref[auth:a1]{Waldo C. Kabat} & Job-Shop Scheduling Using Automated Reasoning: {A} Case Study of the Car-Sequencing Problem & \href{cars/works/ParrelloK86.pdf}{Yes} & \cite{ParrelloK86} & 1986 & J. Autom. Reason. & 42 & 74 & 0 & \ref{b:ParrelloK86} & \ref{c:ParrelloK86}\\
Kis04 \href{http://dx.doi.org/10.1016/j.orl.2003.09.003}{Kis04} & \hyperref[auth:a47]{T. Kis} & On the complexity of the car sequencing problem & No & \cite{Kis04} & 2004 & Operations Research Letters & null & 69 & 3 & No & \ref{c:Kis04}\\
ReginP97 \href{https://doi.org/10.1007/BFb0017428}{ReginP97} & \hyperref[auth:a33]{J. R{\'{e}}gin}, \hyperref[auth:a34]{J. Puget} & A Filtering Algorithm for Global Sequencing Constraints & \href{cars/works/ReginP97.pdf}{Yes} & \cite{ReginP97} & 1997 & CP 1997 & 15 & 53 & 3 & \ref{b:ReginP97} & \ref{c:ReginP97}\\
GottliebPS03 \href{https://doi.org/10.1007/3-540-36605-9_23}{GottliebPS03} & \hyperref[auth:a9]{J. Gottlieb}, \hyperref[auth:a10]{M. Puchta}, \hyperref[auth:a5]{C. Solnon} & A Study of Greedy, Local Search, and Ant Colony Optimization Approaches for Car Sequencing Problems & \href{cars/works/GottliebPS03.pdf}{Yes} & \cite{GottliebPS03} & 2003 & EvoWorkshop 2003 & 12 & 46 & 5 & \ref{b:GottliebPS03} & \ref{c:GottliebPS03}\\
HoevePRS06 \href{https://doi.org/10.1007/11889205\_44}{HoevePRS06} & \hyperref[auth:a43]{Willem Jan van Hoeve}, \hyperref[auth:a40]{G. Pesant}, \hyperref[auth:a44]{L. Rousseau}, \hyperref[auth:a42]{A. Sabharwal} & Revisiting the Sequence Constraint & \href{cars/works/HoevePRS06.pdf}{Yes} & \cite{HoevePRS06} & 2006 & CP 2006 & 15 & 33 & 7 & \ref{b:HoevePRS06} & \ref{c:HoevePRS06}\\
OzturkTHO13 \href{https://doi.org/10.1007/s10601-013-9142-6}{OzturkTHO13} & \hyperref[auth:a14]{C. {\"{O}}zt{\"{u}}rk}, \hyperref[auth:a15]{S. Tunali}, \hyperref[auth:a16]{B. Hnich}, \hyperref[auth:a17]{M. Arslan Ornek} & Balancing and scheduling of flexible mixed model assembly lines & No & \cite{OzturkTHO13} & 2013 & Constraints An Int. J. & 36 & 31 & 44 & No & \ref{c:OzturkTHO13}\\
WarwickT95 \href{http://dx.doi.org/10.1162/evco.1995.3.3.267}{WarwickT95} & \hyperref[auth:a45]{T. Warwick}, \hyperref[auth:a46]{Edward P. K. Tsang} & Tackling Car Sequencing Problems Using a Generic Genetic Algorithm & No & \cite{WarwickT95} & 1995 & Evolutionary Computation & null & 28 & 0 & No & \ref{c:WarwickT95}\\
HindiP94 \href{http://dx.doi.org/10.1016/0360-8352(94)90038-8}{HindiP94} & \hyperref[auth:a37]{Khalil S. Hindi}, \hyperref[auth:a38]{G. Ploszajski} & Formulation and solution of a selection and sequencing problem in car manufacture & No & \cite{HindiP94} & 1994 & Computers \  Industrial Engineering & null & 24 & 4 & No & \ref{c:HindiP94}\\
PerronS04 \href{https://doi.org/10.1007/978-3-540-24664-0_16}{PerronS04} & \hyperref[auth:a20]{L. Perron}, \hyperref[auth:a21]{P. Shaw} & Combining Forces to Solve the Car Sequencing Problem & \href{cars/works/PerronS04.pdf}{Yes} & \cite{PerronS04} & 2004 & CPAIOR 2004 & 15 & 17 & 9 & \ref{b:PerronS04} & \ref{c:PerronS04}\\
SialaHH155 \href{https://doi.org/10.1016/j.engappai.2014.10.009}{SialaHH155} & \hyperref[auth:a11]{M. Siala}, \hyperref[auth:a12]{E. Hebrard}, \hyperref[auth:a13]{M. Huguet} & A study of constraint programming heuristics for the car-sequencing problem & No & \cite{SialaHH155} & 2015 & Eng. Appl. Artif. Intell. & 11 & 15 & 10 & No & \ref{c:SialaHH155}\\
GolleRB14 \href{http://dx.doi.org/10.1007/s10479-014-1733-0}{GolleRB14} & \hyperref[auth:a61]{U. Golle}, \hyperref[auth:a62]{F. Rothlauf}, \hyperref[auth:a48]{N. Boysen} & Iterative beam search for car sequencing & No & \cite{GolleRB14} & 2014 & Annals of Operations Research & null & 15 & 15 & No & \ref{c:GolleRB14}\\
HoevePRS09 \href{http://dx.doi.org/10.1007/s10601-008-9067-7}{HoevePRS09} & \hyperref[auth:a39]{Willem-Jan van Hoeve}, \hyperref[auth:a40]{G. Pesant}, \hyperref[auth:a41]{L. Rousseau}, \hyperref[auth:a42]{A. Sabharwal} & New filtering algorithms for combinations of among constraints & No & \cite{HoevePRS09} & 2009 & Constraints An Int. J. & null & 13 & 8 & No & \ref{c:HoevePRS09}\\
MoyaCB19 \href{http://dx.doi.org/10.1016/j.cie.2019.106048}{MoyaCB19} & \hyperref[auth:a63]{I. Moya}, \hyperref[auth:a64]{M. Chica}, \hyperref[auth:a65]{J. Bautista} & Constructive metaheuristics for solving the Car Sequencing Problem under uncertain partial demand & No & \cite{MoyaCB19} & 2019 & Computers \  Industrial Engineering & 1 & 8 & 44 & No & \ref{c:MoyaCB19}\\
SialaHH14 \href{https://doi.org/10.1007/s10601-013-9150-6}{SialaHH14} & \hyperref[auth:a11]{M. Siala}, \hyperref[auth:a12]{E. Hebrard}, \hyperref[auth:a13]{M. Huguet} & An optimal arc consistency algorithm for a particular case of sequence constraint & \href{cars/works/SialaHH14.pdf}{Yes} & \cite{SialaHH14} & 2014 & Constraints An Int. J. & 27 & 3 & 14 & \ref{b:SialaHH14} & \ref{c:SialaHH14}\\
ZhangGWH17 \href{http://dx.doi.org/10.1007/s10033-017-0083-7}{ZhangGWH17} & \hyperref[auth:a51]{X. ZHANG}, \hyperref[auth:a52]{L. GAO}, \hyperref[auth:a53]{L. WEN}, \hyperref[auth:a54]{Z. HUANG} & Parallel Construction Heuristic Combined with Constraint Propagation for the Car Sequencing Problem & No & \cite{ZhangGWH17} & 2017 & Chinese Journal of Mechanical Engineering & null & 3 & 32 & No & \ref{c:ZhangGWH17}\\
ArtiguesHM0W14 \href{https://doi.org/10.1007/978-3-319-07046-9\_19}{ArtiguesHM0W14} & \hyperref[auth:a8]{C. Artigues}, \hyperref[auth:a12]{E. Hebrard}, \hyperref[auth:a35]{V. Mayer{-}Eichberger}, \hyperref[auth:a11]{M. Siala}, \hyperref[auth:a36]{T. Walsh} & {SAT} and Hybrid Models of the Car Sequencing Problem & \href{cars/works/ArtiguesHM0W14.pdf}{Yes} & \cite{ArtiguesHM0W14} & 2014 & CPAIOR 2014 & 16 & 2 & 16 & \ref{b:ArtiguesHM0W14} & \ref{c:ArtiguesHM0W14}\\
YuLZCLW22 \href{http://dx.doi.org/10.1186/s13638-022-02113-7}{YuLZCLW22} & \hyperref[auth:a55]{Y. Yu}, \hyperref[auth:a56]{X. Lu}, \hyperref[auth:a57]{T. Zhao}, \hyperref[auth:a58]{M. Cheng}, \hyperref[auth:a59]{L. Liu}, \hyperref[auth:a60]{W. Wei} & Heuristic approaches for the car sequencing problems with block batches & No & \cite{YuLZCLW22} & 2022 & EURASIP Journal on Wireless Communications and Networking & null & 2 & 37 & No & \ref{c:YuLZCLW22}\\
DincbasSH88 \href{}{DincbasSH88} & \hyperref[auth:a2]{M. Dincbas}, \hyperref[auth:a3]{H. Simonis}, \hyperref[auth:a4]{Pascal Van Hentenryck} & Solving the Car-Sequencing Problem in Constraint Logic Programming & \href{cars/works/DincbasSH88.pdf}{Yes} & \cite{DincbasSH88} & 1988 & ECAI 1988 & 6 & 0 & 0 & \ref{b:DincbasSH88} & \ref{c:DincbasSH88}\\
Siala15 \href{https://tel.archives-ouvertes.fr/tel-01164291}{Siala15} & \hyperref[auth:a11]{M. Siala} & Search, propagation, and learning in sequencing and scheduling problems. (Recherche, propagation et apprentissage dans les probl{\`{e}}mes de s{\'{e}}quencement et d'ordonnancement) & \href{cars/works/Siala15.pdf}{Yes} & \cite{Siala15} & 2015 & {INSA} Toulouse, France & 200 & 0 & 0 & \ref{b:Siala15} & n/a\\
Gent98 \href{}{Gent98} & \hyperref[auth:a18]{Ian P Gent} & Two results on car-sequencing problems & No & \cite{Gent98} & 1998 & Report University of Strathclyde, APES-02-98 & null & 0 & 0 & No & \ref{c:Gent98}\\
Schaus09 \href{}{Schaus09} & \hyperref[auth:a19]{P. Schaus} & Solving balancing and bin-packing problems with constraint programming & No & \cite{Schaus09} & 2009 & These de doctorat, Universit{\'e} catholique de Louvain & null & 0 & 0 & No & \ref{c:Schaus09}\\
YavuzE18 \href{}{YavuzE18} & \hyperref[auth:a22]{M. Yavuz}, \hyperref[auth:a23]{H. Ergin} & Advanced constraint propagation for the combined car sequencing and level scheduling problem & No & \cite{YavuzE18} & 2018 & Computers \  Operations Research & 12 & 0 & 0 & No & \ref{c:YavuzE18}\\
WinterM21 \href{}{WinterM21} & \hyperref[auth:a24]{F. Winter}, \hyperref[auth:a25]{N. Musliu} & Constraint-based Scheduling for Paint Shops in the Automotive Supply Industry & No & \cite{WinterM21} & 2021 & ACM Transactions on Intelligent Systems and Technology (TIST) & 25 & 0 & 0 & No & \ref{c:WinterM21}\\
ThiruvadyME11 \href{}{ThiruvadyME11} & \hyperref[auth:a26]{Dhananjay Raghavan Thiruvady}, \hyperref[auth:a27]{B. Meyer}, \hyperref[auth:a28]{A. Ernst} & Car sequencing with constraint-based ACO & No & \cite{ThiruvadyME11} & 2011 & Genetic and evolutionary computation 2011 & 8 & 0 & 0 & No & \ref{c:ThiruvadyME11}\\
ButaruH05 \href{}{ButaruH05} & \hyperref[auth:a29]{M. Butaru}, \hyperref[auth:a30]{Z. Habbas} & The car-sequencing problem as n-ary CSP--Sequential and parallel solving & \href{cars/works/ButaruH05.pdf}{Yes} & \cite{ButaruH05} & 2005 & Australian Joint Conference on Artificial Intelligence 2005 & 4 & 0 & 0 & \ref{b:ButaruH05} & \ref{c:ButaruH05}\\
MazurN15 \href{}{MazurN15} & \hyperref[auth:a31]{M. Mazur}, \hyperref[auth:a32]{A. Niederli{\'n}ski} & A Two-stage approach for an optimum solution of the car assembly scheduling problem. Part 2. CLP solution and real-world example & No & \cite{MazurN15} & 2015 & Archives of Control Sciences & 9 & 0 & 0 & No & \ref{c:MazurN15}\\
Mayer-EichbergerW13 \href{https://doi.org/10.29007/jrsp}{Mayer-EichbergerW13} & \hyperref[auth:a35]{V. Mayer{-}Eichberger}, \hyperref[auth:a36]{T. Walsh} & {SAT} Encodings for the Car Sequencing Problem & \href{cars/works/Mayer-EichbergerW13.pdf}{Yes} & \cite{Mayer-EichbergerW13} & 2013 & SAT 2013 & 13 & 0 & 0 & \ref{b:Mayer-EichbergerW13} & \ref{c:Mayer-EichbergerW13}\\
\end{longtable}
}




\clearpage
\section{Problem Classification}




\clearpage
\section{Concept Matching}

In order to automatically find out properties of the articles, we try to find certain concepts in the pdf versions of the articles. We manually defined an ontology of important concepts to look for, and defined regular expressions that would recognize these concepts in the text. We use the \emph{pdfgrep} command to search for the number of occurrences of certain regular expressions in the files. This often clearly identifies the constraints used in the model. We group the results by number of occurrences of the concept in the text of the work. Note that this is only approximate, as we do include the full pdf file in the search. A concept might only be mentioned in some of the title of citations used in the paper, we do count them in our results, as we were not able to remove the bibliography from the main body of the work.

Overall, if a work is not mentioned as using the concept, the the text does not contain a match to the corresponding regular expression. A fundamental limitation of this approach is that it only really works for text written in the language the regular expressions are designed for (in our case English), and not those written in another language. We could overcome this limitation by defining all concepts in other languages as well, and then using a language flag to identify the language the text is written in. 

Note that we only show the first 30 matching entries in each concept category, and list the total number of matches if there are more than 30 matches.


\clearpage
\subsection{Concept Type Concepts}
\label{sec:Concepts}
{\scriptsize
\begin{longtable}{lp{3cm}>{\raggedright\arraybackslash}p{6cm}>{\raggedright\arraybackslash}p{6cm}>{\raggedright\arraybackslash}p{8cm}}
\rowcolor{white}\caption{Works for Concepts of Type Concepts}\\ \toprule
\rowcolor{white}Type & Keyword & High & Medium & Low\\ \midrule\endhead
\bottomrule
\endfoot
Concepts & Allen's algebra &  &  & \\
Concepts & BOM &  &  & \\
Concepts & Benders Decomposition &  &  & \\
Concepts & Logic-Based Benders Decomposition &  &  & \\
Concepts & activity & \href{cars/works/Siala15.pdf}{Siala15}~\cite{Siala15} &  & \href{cars/works/ArtiguesHM0W14.pdf}{ArtiguesHM0W14}~\cite{ArtiguesHM0W14}\\
Concepts & batch process &  &  & \\
Concepts & bill of material &  &  & \\
Concepts & blocking constraint &  &  & \\
Concepts & buffer-capacity &  &  & \\
Concepts & cmax & \href{cars/works/Siala15.pdf}{Siala15}~\cite{Siala15} &  & \\
Concepts & completion-time &  &  & \\
Concepts & continuous-process &  &  & \\
Concepts & distributed &  &  & \href{cars/works/DincbasSH88.pdf}{DincbasSH88}~\cite{DincbasSH88}\\
Concepts & due-date &  &  & \href{cars/works/Siala15.pdf}{Siala15}~\cite{Siala15}\\
Concepts & earliness &  &  & \href{cars/works/Siala15.pdf}{Siala15}~\cite{Siala15}\\
Concepts & flow-shop &  &  & \\
Concepts & flow-time &  &  & \\
Concepts & inventory &  &  & \\
Concepts & job & \href{cars/works/Siala15.pdf}{Siala15}~\cite{Siala15}, \href{cars/works/ParrelloK86.pdf}{ParrelloK86}~\cite{ParrelloK86} & \href{cars/works/PerronS04.pdf}{PerronS04}~\cite{PerronS04} & \href{cars/works/ButaruH05.pdf}{ButaruH05}~\cite{ButaruH05}, \href{cars/works/DincbasSH88.pdf}{DincbasSH88}~\cite{DincbasSH88}\\
Concepts & job-shop & \href{cars/works/Siala15.pdf}{Siala15}~\cite{Siala15}, \href{cars/works/ParrelloK86.pdf}{ParrelloK86}~\cite{ParrelloK86} & \href{cars/works/PerronS04.pdf}{PerronS04}~\cite{PerronS04} & \href{cars/works/ButaruH05.pdf}{ButaruH05}~\cite{ButaruH05}, \href{cars/works/DincbasSH88.pdf}{DincbasSH88}~\cite{DincbasSH88}\\
Concepts & lateness &  &  & \\
Concepts & lazy clause generation & \href{cars/works/Siala15.pdf}{Siala15}~\cite{Siala15} &  & \href{cars/works/ArtiguesHM0W14.pdf}{ArtiguesHM0W14}~\cite{ArtiguesHM0W14}\\
Concepts & machine & \href{cars/works/Siala15.pdf}{Siala15}~\cite{Siala15} &  & \href{cars/works/ButaruH05.pdf}{ButaruH05}~\cite{ButaruH05}, \href{cars/works/GottliebPS03.pdf}{GottliebPS03}~\cite{GottliebPS03}, \href{cars/works/ReginP97.pdf}{ReginP97}~\cite{ReginP97}, \href{cars/works/DincbasSH88.pdf}{DincbasSH88}~\cite{DincbasSH88}, \href{cars/works/ParrelloK86.pdf}{ParrelloK86}~\cite{ParrelloK86}\\
Concepts & make to order &  &  & \\
Concepts & make to stock &  &  & \\
Concepts & make-span & \href{cars/works/Siala15.pdf}{Siala15}~\cite{Siala15} &  & \\
Concepts & manpower &  &  & \\
Concepts & multi-agent &  &  & \\
Concepts & no preempt &  &  & \\
Concepts & no-wait &  &  & \\
Concepts & open-shop & \href{cars/works/Siala15.pdf}{Siala15}~\cite{Siala15} &  & \\
Concepts & order & \href{cars/works/Siala15.pdf}{Siala15}~\cite{Siala15}, \href{cars/works/SialaHH14.pdf}{SialaHH14}~\cite{SialaHH14}, \href{cars/works/HoevePRS06.pdf}{HoevePRS06}~\cite{HoevePRS06}, \href{cars/works/ButaruH05.pdf}{ButaruH05}~\cite{ButaruH05}, \href{cars/works/PerronS04.pdf}{PerronS04}~\cite{PerronS04}, \href{cars/works/GottliebPS03.pdf}{GottliebPS03}~\cite{GottliebPS03}, \href{cars/works/ReginP97.pdf}{ReginP97}~\cite{ReginP97}, \href{cars/works/DincbasSH88.pdf}{DincbasSH88}~\cite{DincbasSH88}, \href{cars/works/ParrelloK86.pdf}{ParrelloK86}~\cite{ParrelloK86} & \href{cars/works/ArtiguesHM0W14.pdf}{ArtiguesHM0W14}~\cite{ArtiguesHM0W14} & \\
Concepts & precedence & \href{cars/works/Siala15.pdf}{Siala15}~\cite{Siala15}, \href{cars/works/DincbasSH88.pdf}{DincbasSH88}~\cite{DincbasSH88} &  & \\
Concepts & preempt &  &  & \\
Concepts & producer/consumer &  &  & \\
Concepts & re-scheduling &  &  & \\
Concepts & release-date &  &  & \\
Concepts & resource & \href{cars/works/Siala15.pdf}{Siala15}~\cite{Siala15} & \href{cars/works/DincbasSH88.pdf}{DincbasSH88}~\cite{DincbasSH88} & \href{cars/works/SialaHH14.pdf}{SialaHH14}~\cite{SialaHH14}, \href{cars/works/PerronS04.pdf}{PerronS04}~\cite{PerronS04}, \href{cars/works/ReginP97.pdf}{ReginP97}~\cite{ReginP97}\\
Concepts & scheduling & \href{cars/works/Siala15.pdf}{Siala15}~\cite{Siala15}, \href{cars/works/DincbasSH88.pdf}{DincbasSH88}~\cite{DincbasSH88}, \href{cars/works/ParrelloK86.pdf}{ParrelloK86}~\cite{ParrelloK86} & \href{cars/works/PerronS04.pdf}{PerronS04}~\cite{PerronS04}, \href{cars/works/ReginP97.pdf}{ReginP97}~\cite{ReginP97} & \href{cars/works/SialaHH14.pdf}{SialaHH14}~\cite{SialaHH14}, \href{cars/works/Mayer-EichbergerW13.pdf}{Mayer-EichbergerW13}~\cite{Mayer-EichbergerW13}, \href{cars/works/GottliebPS03.pdf}{GottliebPS03}~\cite{GottliebPS03}\\
Concepts & sequence dependent setup &  &  & \href{cars/works/Siala15.pdf}{Siala15}~\cite{Siala15}\\
Concepts & setup-time &  &  & \href{cars/works/Siala15.pdf}{Siala15}~\cite{Siala15}\\
Concepts & stock level &  &  & \\
Concepts & tardiness &  &  & \href{cars/works/Siala15.pdf}{Siala15}~\cite{Siala15}\\
Concepts & task & \href{cars/works/Siala15.pdf}{Siala15}~\cite{Siala15}, \href{cars/works/DincbasSH88.pdf}{DincbasSH88}~\cite{DincbasSH88} &  & \href{cars/works/Mayer-EichbergerW13.pdf}{Mayer-EichbergerW13}~\cite{Mayer-EichbergerW13}, \href{cars/works/ButaruH05.pdf}{ButaruH05}~\cite{ButaruH05}\\
Concepts & temporal constraint reasoning &  &  & \\
Concepts & transportation &  &  & \href{cars/works/HoevePRS06.pdf}{HoevePRS06}~\cite{HoevePRS06}\\
\end{longtable}
}


\clearpage
\subsection{Concept Type Classification}
\label{sec:Classification}
{\scriptsize
\begin{longtable}{lp{3cm}>{\raggedright\arraybackslash}p{6cm}>{\raggedright\arraybackslash}p{6cm}>{\raggedright\arraybackslash}p{8cm}}
\rowcolor{white}\caption{Works for Concepts of Type Classification}\\ \toprule
\rowcolor{white}Type & Keyword & High & Medium & Low\\ \midrule\endhead
\bottomrule
\endfoot
Classification & OSP & \href{../cars/works/Siala15.pdf}{Siala15}~\cite{Siala15} &  & \\
Classification & RCPSP &  &  & \href{../cars/works/Siala15.pdf}{Siala15}~\cite{Siala15}\\
Classification & SBSFMMAL & \href{../cars/works/OzturkTHO13.pdf}{OzturkTHO13}~\cite{OzturkTHO13} &  & \\
Classification & TMS &  &  & \href{../cars/works/Siala15.pdf}{Siala15}~\cite{Siala15}\\
Classification & parallel machine &  &  & \href{../cars/works/YuLZCLW22.pdf}{YuLZCLW22}~\cite{YuLZCLW22}\\
Classification & single machine &  &  & \href{../cars/works/Siala15.pdf}{Siala15}~\cite{Siala15}\\
\end{longtable}
}


\clearpage
\subsection{Concept Type Constraints}
\label{sec:Constraints}
{\scriptsize
\begin{longtable}{lp{3cm}>{\raggedright\arraybackslash}p{6cm}>{\raggedright\arraybackslash}p{6cm}>{\raggedright\arraybackslash}p{8cm}}
\rowcolor{white}\caption{Works for Concepts of Type Constraints}\\ \toprule
\rowcolor{white}Type & Keyword & High & Medium & Low\\ \midrule\endhead
\bottomrule
\endfoot
Constraints & AllDiff constraint &  &  & \\
Constraints & AllDiffPrec constraint &  &  & \\
Constraints & AlwaysConstant &  &  & \\
Constraints & Among constraint & \href{cars/works/Siala15.pdf}{Siala15}~\cite{Siala15}, \href{cars/works/SialaHH14.pdf}{SialaHH14}~\cite{SialaHH14}, \href{cars/works/HoevePRS06.pdf}{HoevePRS06}~\cite{HoevePRS06} &  & \href{cars/works/ArtiguesHM0W14.pdf}{ArtiguesHM0W14}~\cite{ArtiguesHM0W14}\\
Constraints & AmongSeq constraint &  & \href{cars/works/Siala15.pdf}{Siala15}~\cite{Siala15}, \href{cars/works/SialaHH14.pdf}{SialaHH14}~\cite{SialaHH14} & \\
Constraints & Arithmetic constraint &  &  & \\
Constraints & AtMostSeq & \href{cars/works/Siala15.pdf}{Siala15}~\cite{Siala15}, \href{cars/works/SialaHH14.pdf}{SialaHH14}~\cite{SialaHH14} &  & \\
Constraints & AtMostSeqCard & \href{cars/works/Siala15.pdf}{Siala15}~\cite{Siala15}, \href{cars/works/SialaHH14.pdf}{SialaHH14}~\cite{SialaHH14} &  & \\
Constraints & Atmost constraint & \href{cars/works/Siala15.pdf}{Siala15}~\cite{Siala15}, \href{cars/works/SialaHH14.pdf}{SialaHH14}~\cite{SialaHH14} &  & \href{cars/works/Mayer-EichbergerW13.pdf}{Mayer-EichbergerW13}~\cite{Mayer-EichbergerW13}\\
Constraints & Balance constraint &  &  & \href{cars/works/Siala15.pdf}{Siala15}~\cite{Siala15}\\
Constraints & BinPacking constraint &  &  & \\
Constraints & Blocking constraint &  &  & \\
Constraints & BufferedResource &  &  & \\
Constraints & Calendar constraint &  &  & \\
Constraints & CardPath &  &  & \href{cars/works/Siala15.pdf}{Siala15}~\cite{Siala15}, \href{cars/works/SialaHH14.pdf}{SialaHH14}~\cite{SialaHH14}\\
Constraints & Cardinality constraint & \href{cars/works/Siala15.pdf}{Siala15}~\cite{Siala15}, \href{cars/works/SialaHH14.pdf}{SialaHH14}~\cite{SialaHH14}, \href{cars/works/ArtiguesHM0W14.pdf}{ArtiguesHM0W14}~\cite{ArtiguesHM0W14}, \href{cars/works/Mayer-EichbergerW13.pdf}{Mayer-EichbergerW13}~\cite{Mayer-EichbergerW13}, \href{cars/works/ReginP97.pdf}{ReginP97}~\cite{ReginP97} & \href{cars/works/HoevePRS06.pdf}{HoevePRS06}~\cite{HoevePRS06} & \\
Constraints & Channeling constraint &  &  & \\
Constraints & CumulativeCost &  &  & \\
Constraints & Cumulatives constraint &  &  & \\
Constraints & Diff2 constraint &  &  & \\
Constraints & Disjunctive constraint & \href{cars/works/DincbasSH88.pdf}{DincbasSH88}~\cite{DincbasSH88} & \href{cars/works/Siala15.pdf}{Siala15}~\cite{Siala15} & \\
Constraints & Element constraint &  &  & \\
Constraints & GCC constraint &  &  & \href{cars/works/Siala15.pdf}{Siala15}~\cite{Siala15}\\
Constraints & GeneralizedAllDiffPrec &  &  & \\
Constraints & IloAlternative &  &  & \\
Constraints & IloAlwaysIn &  &  & \\
Constraints & IloForbidEnd &  &  & \\
Constraints & IloNoOverlap &  &  & \\
Constraints & IloPack &  &  & \\
Constraints & IloPulse &  &  & \\
Constraints & MinWeightAllDiff &  &  & \\
Constraints & MultiAtMostSeqCard & \href{cars/works/Siala15.pdf}{Siala15}~\cite{Siala15} & \href{cars/works/SialaHH14.pdf}{SialaHH14}~\cite{SialaHH14} & \\
Constraints & PreemptiveNoOverlap &  &  & \\
Constraints & Pulse constraint &  &  & \\
Constraints & Regular constraint & \href{cars/works/Siala15.pdf}{Siala15}~\cite{Siala15}, \href{cars/works/SialaHH14.pdf}{SialaHH14}~\cite{SialaHH14} & \href{cars/works/HoevePRS06.pdf}{HoevePRS06}~\cite{HoevePRS06} & \\
Constraints & Reified constraint &  &  & \href{cars/works/Siala15.pdf}{Siala15}~\cite{Siala15}\\
Constraints & RelSoftCumulative &  &  & \\
Constraints & RelSoftCumulativeSum &  &  & \\
Constraints & SoftCumulative &  &  & \\
Constraints & SoftCumulativeSum &  &  & \\
Constraints & TaskIntersection constraint &  &  & \\
Constraints & UTVPI constraint &  &  & \\
Constraints & WeightAllDiff &  &  & \\
Constraints & WeightedSum &  &  & \\
Constraints & WeightedTaskSum &  &  & \\
Constraints & alldifferent & \href{cars/works/Siala15.pdf}{Siala15}~\cite{Siala15} &  & \\
Constraints & alternative constraint &  &  & \\
Constraints & alwaysEqual constraint &  &  & \\
Constraints & alwaysIn &  &  & \\
Constraints & bin-packing &  &  & \\
Constraints & circuit &  & \href{cars/works/Siala15.pdf}{Siala15}~\cite{Siala15}, \href{cars/works/DincbasSH88.pdf}{DincbasSH88}~\cite{DincbasSH88} & \\
Constraints & cumulative &  & \href{cars/works/ArtiguesHM0W14.pdf}{ArtiguesHM0W14}~\cite{ArtiguesHM0W14}, \href{cars/works/Mayer-EichbergerW13.pdf}{Mayer-EichbergerW13}~\cite{Mayer-EichbergerW13} & \href{cars/works/Siala15.pdf}{Siala15}~\cite{Siala15}, \href{cars/works/ReginP97.pdf}{ReginP97}~\cite{ReginP97}\\
Constraints & cycle & \href{cars/works/GottliebPS03.pdf}{GottliebPS03}~\cite{GottliebPS03} & \href{cars/works/Siala15.pdf}{Siala15}~\cite{Siala15} & \\
Constraints & diffn &  &  & \\
Constraints & disjunctive & \href{cars/works/Siala15.pdf}{Siala15}~\cite{Siala15}, \href{cars/works/DincbasSH88.pdf}{DincbasSH88}~\cite{DincbasSH88} &  & \\
Constraints & endBeforeStart &  &  & \\
Constraints & geost &  &  & \\
Constraints & noOverlap &  &  & \\
Constraints & regular expression &  &  & \\
Constraints & span constraint &  &  & \\
Constraints & table constraint &  &  & \href{cars/works/Siala15.pdf}{Siala15}~\cite{Siala15}\\
\end{longtable}
}


\clearpage
\subsection{Concept Type ProgLanguages}
\label{sec:ProgLanguages}
{\scriptsize
\begin{longtable}{lp{3cm}>{\raggedright\arraybackslash}p{6cm}>{\raggedright\arraybackslash}p{6cm}>{\raggedright\arraybackslash}p{8cm}}
\rowcolor{white}\caption{Works for Concepts of Type ProgLanguages}\\ \toprule
\rowcolor{white}Type & Keyword & High & Medium & Low\\ \midrule\endhead
\bottomrule
\endfoot
ProgLanguages & C++ &  &  & \href{../cars/works/YuLZCLW22.pdf}{YuLZCLW22}~\cite{YuLZCLW22}, \href{../cars/works/SolnonCNA08.pdf}{SolnonCNA08}~\cite{SolnonCNA08}, \href{../cars/works/ButaruH05.pdf}{ButaruH05}~\cite{ButaruH05}\\
ProgLanguages & Java &  &  & \href{../cars/works/YuLZCLW22.pdf}{YuLZCLW22}~\cite{YuLZCLW22}, \href{../cars/works/MoyaCB19.pdf}{MoyaCB19}~\cite{MoyaCB19}, \href{../cars/works/GolleRB14.pdf}{GolleRB14}~\cite{GolleRB14}\\
ProgLanguages & Prolog & \href{../cars/works/DincbasSH88.pdf}{DincbasSH88}~\cite{DincbasSH88}, \href{../cars/works/ParrelloK86.pdf}{ParrelloK86}~\cite{ParrelloK86} &  & \\
\end{longtable}
}


\clearpage
\subsection{Concept Type CPSystems}
\label{sec:CPSystems}
{\scriptsize
\begin{longtable}{lp{3cm}>{\raggedright\arraybackslash}p{6cm}>{\raggedright\arraybackslash}p{6cm}>{\raggedright\arraybackslash}p{8cm}}
\rowcolor{white}\caption{Works for Concepts of Type CPSystems}\\ \toprule
\rowcolor{white}Type & Keyword & High & Medium & Low\\ \midrule\endhead
\bottomrule
\endfoot
CPSystems & CHIP & \href{../cars/works/DincbasSH88.pdf}{DincbasSH88}~\cite{DincbasSH88} &  & \href{../cars/works/Siala15.pdf}{Siala15}~\cite{Siala15}, \href{../cars/works/SialaHH155.pdf}{SialaHH155}~\cite{SialaHH155}, \href{../cars/works/SialaHH14.pdf}{SialaHH14}~\cite{SialaHH14}, \href{../cars/works/OzturkTHO13.pdf}{OzturkTHO13}~\cite{OzturkTHO13}, \href{../cars/works/HoevePRS09.pdf}{HoevePRS09}~\cite{HoevePRS09}, \href{../cars/works/SolnonCNA08.pdf}{SolnonCNA08}~\cite{SolnonCNA08}, \href{../cars/works/HoevePRS06.pdf}{HoevePRS06}~\cite{HoevePRS06}, \href{../cars/works/ReginP97.pdf}{ReginP97}~\cite{ReginP97}\\
CPSystems & Claire & \href{../cars/works/Siala15.pdf}{Siala15}~\cite{Siala15} &  & \\
CPSystems & Cplex & \href{../cars/works/YuLZCLW22.pdf}{YuLZCLW22}~\cite{YuLZCLW22} & \href{../cars/works/ZhangGWH17.pdf}{ZhangGWH17}~\cite{ZhangGWH17} & \href{../cars/works/OzturkTHO13.pdf}{OzturkTHO13}~\cite{OzturkTHO13}\\
CPSystems & Gurobi &  &  & \href{../cars/works/YuLZCLW22.pdf}{YuLZCLW22}~\cite{YuLZCLW22}\\
CPSystems & Ilog Scheduler &  &  & \href{../cars/works/PerronS04.pdf}{PerronS04}~\cite{PerronS04}, \href{../cars/works/ReginP97.pdf}{ReginP97}~\cite{ReginP97}\\
CPSystems & Ilog Solver &  & \href{../cars/works/HoevePRS09.pdf}{HoevePRS09}~\cite{HoevePRS09}, \href{../cars/works/ReginP97.pdf}{ReginP97}~\cite{ReginP97} & \href{../cars/works/Siala15.pdf}{Siala15}~\cite{Siala15}, \href{../cars/works/OzturkTHO13.pdf}{OzturkTHO13}~\cite{OzturkTHO13}, \href{../cars/works/SolnonCNA08.pdf}{SolnonCNA08}~\cite{SolnonCNA08}, \href{../cars/works/HoevePRS06.pdf}{HoevePRS06}~\cite{HoevePRS06}, \href{../cars/works/ButaruH05.pdf}{ButaruH05}~\cite{ButaruH05}\\
CPSystems & Mistral & \href{../cars/works/Siala15.pdf}{Siala15}~\cite{Siala15} & \href{../cars/works/ArtiguesHM0W14.pdf}{ArtiguesHM0W14}~\cite{ArtiguesHM0W14} & \\
CPSystems & OPL &  & \href{../cars/works/OzturkTHO13.pdf}{OzturkTHO13}~\cite{OzturkTHO13} & \href{../cars/works/Siala15.pdf}{Siala15}~\cite{Siala15}, \href{../cars/works/DincbasSH88.pdf}{DincbasSH88}~\cite{DincbasSH88}, \href{../cars/works/ParrelloK86.pdf}{ParrelloK86}~\cite{ParrelloK86}\\
CPSystems & OZ &  & \href{../cars/works/MoyaCB19.pdf}{MoyaCB19}~\cite{MoyaCB19}, \href{../cars/works/BoysenFS09.pdf}{BoysenFS09}~\cite{BoysenFS09} & \href{../cars/works/OzturkTHO13.pdf}{OzturkTHO13}~\cite{OzturkTHO13}, \href{../cars/works/SolnonCNA08.pdf}{SolnonCNA08}~\cite{SolnonCNA08}, \href{../cars/works/PerronS04.pdf}{PerronS04}~\cite{PerronS04}\\
\end{longtable}
}


\clearpage
\subsection{Concept Type ApplicationAreas}
\label{sec:ApplicationAreas}
{\scriptsize
\begin{longtable}{lp{3cm}>{\raggedright\arraybackslash}p{6cm}>{\raggedright\arraybackslash}p{6cm}>{\raggedright\arraybackslash}p{8cm}}
\rowcolor{white}\caption{Works for Concepts of Type ApplicationAreas}\\ \toprule
\rowcolor{white}Type & Keyword & High & Medium & Low\\ \midrule\endhead
\bottomrule
\endfoot
ApplicationAreas & COVID &  &  & \\
ApplicationAreas & HVAC &  &  & \\
ApplicationAreas & agriculture &  &  & \\
ApplicationAreas & aircraft &  &  & \\
ApplicationAreas & automotive &  &  & \href{cars/works/Siala15.pdf}{Siala15}~\cite{Siala15}, \href{cars/works/ReginP97.pdf}{ReginP97}~\cite{ReginP97}\\
ApplicationAreas & cable tree &  &  & \\
ApplicationAreas & car manufacturing &  &  & \\
ApplicationAreas & container terminal &  &  & \\
ApplicationAreas & crew-scheduling &  &  & \href{cars/works/ReginP97.pdf}{ReginP97}~\cite{ReginP97}\\
ApplicationAreas & dairies &  &  & \\
ApplicationAreas & dairy &  &  & \\
ApplicationAreas & datacenter &  &  & \\
ApplicationAreas & datacentre &  &  & \\
ApplicationAreas & day-ahead market &  &  & \\
ApplicationAreas & deep space &  &  & \\
ApplicationAreas & drone &  &  & \\
ApplicationAreas & earth observation &  &  & \\
ApplicationAreas & earth orbit &  &  & \\
ApplicationAreas & electroplating &  &  & \\
ApplicationAreas & emergency service &  &  & \\
ApplicationAreas & energy-price &  &  & \\
ApplicationAreas & farming &  &  & \\
ApplicationAreas & forestry &  &  & \\
ApplicationAreas & hoist &  &  & \\
ApplicationAreas & medical &  &  & \\
ApplicationAreas & nurse &  &  & \href{cars/works/HoevePRS06.pdf}{HoevePRS06}~\cite{HoevePRS06}, \href{cars/works/ParrelloK86.pdf}{ParrelloK86}~\cite{ParrelloK86}\\
ApplicationAreas & offshore &  &  & \\
ApplicationAreas & operating room &  &  & \\
ApplicationAreas & oven scheduling &  &  & \\
ApplicationAreas & patient &  &  & \\
ApplicationAreas & perfect-square &  &  & \\
ApplicationAreas & physician &  &  & \\
ApplicationAreas & pipeline &  &  & \\
ApplicationAreas & radiation therapy &  &  & \\
ApplicationAreas & railway &  &  & \\
ApplicationAreas & real-time pricing &  &  & \\
ApplicationAreas & rectangle-packing &  &  & \href{cars/works/Siala15.pdf}{Siala15}~\cite{Siala15}\\
ApplicationAreas & robot &  &  & \\
ApplicationAreas & satellite &  &  & \\
ApplicationAreas & semiconductor &  &  & \\
ApplicationAreas & ship building &  &  & \\
ApplicationAreas & shipping line &  &  & \\
ApplicationAreas & steel cable &  &  & \\
ApplicationAreas & steel mill &  &  & \\
ApplicationAreas & super-computer &  &  & \\
ApplicationAreas & surgery &  &  & \\
ApplicationAreas & torpedo &  &  & \\
ApplicationAreas & vaccine &  &  & \\
ApplicationAreas & yard crane &  &  & \\
\end{longtable}
}


\clearpage
\subsection{Concept Type Industries}
\label{sec:Industries}
{\scriptsize
\begin{longtable}{lp{3cm}>{\raggedright\arraybackslash}p{6cm}>{\raggedright\arraybackslash}p{6cm}>{\raggedright\arraybackslash}p{8cm}}
\rowcolor{white}\caption{Works for Concepts of Type Industries}\\ \toprule
\rowcolor{white}Type & Keyword & High & Medium & Low\\ \midrule\endhead
\bottomrule
\endfoot
Industries & automobile industry & \href{../cars/works/BoysenFS09.pdf}{BoysenFS09}~\cite{BoysenFS09} &  & \href{../cars/works/YuLZCLW22.pdf}{YuLZCLW22}~\cite{YuLZCLW22}\\
Industries & automotive industry &  & \href{../cars/works/BoysenFS09.pdf}{BoysenFS09}~\cite{BoysenFS09} & \href{../cars/works/SialaHH155.pdf}{SialaHH155}~\cite{SialaHH155}\\
\end{longtable}
}


\clearpage
\subsection{Concept Type Benchmarks}
\label{sec:Benchmarks}
{\scriptsize
\begin{longtable}{lp{3cm}>{\raggedright\arraybackslash}p{6cm}>{\raggedright\arraybackslash}p{6cm}>{\raggedright\arraybackslash}p{8cm}}
\rowcolor{white}\caption{Works for Concepts of Type Benchmarks}\\ \toprule
\rowcolor{white}Type & Keyword & High & Medium & Low\\ \midrule\endhead
\bottomrule
\endfoot
Benchmarks & CSPlib & \href{../cars/works/Siala15.pdf}{Siala15}~\cite{Siala15}, \href{../cars/works/ArtiguesHM0W14.pdf}{ArtiguesHM0W14}~\cite{ArtiguesHM0W14}, \href{../cars/works/Mayer-EichbergerW13.pdf}{Mayer-EichbergerW13}~\cite{Mayer-EichbergerW13} & \href{../cars/works/SialaHH14.pdf}{SialaHH14}~\cite{SialaHH14}, \href{../cars/works/GottliebPS03.pdf}{GottliebPS03}~\cite{GottliebPS03} & \href{../cars/works/ButaruH05.pdf}{ButaruH05}~\cite{ButaruH05}\\
Benchmarks & Roadef & \href{../cars/works/Siala15.pdf}{Siala15}~\cite{Siala15} &  & \href{../cars/works/ArtiguesHM0W14.pdf}{ArtiguesHM0W14}~\cite{ArtiguesHM0W14}, \href{../cars/works/SialaHH14.pdf}{SialaHH14}~\cite{SialaHH14}\\
Benchmarks & benchmark & \href{../cars/works/Siala15.pdf}{Siala15}~\cite{Siala15}, \href{../cars/works/SialaHH14.pdf}{SialaHH14}~\cite{SialaHH14}, \href{../cars/works/Mayer-EichbergerW13.pdf}{Mayer-EichbergerW13}~\cite{Mayer-EichbergerW13}, \href{../cars/works/GottliebPS03.pdf}{GottliebPS03}~\cite{GottliebPS03} &  & \href{../cars/works/ArtiguesHM0W14.pdf}{ArtiguesHM0W14}~\cite{ArtiguesHM0W14}, \href{../cars/works/ReginP97.pdf}{ReginP97}~\cite{ReginP97}\\
Benchmarks & generated instance &  &  & \href{../cars/works/PerronS04.pdf}{PerronS04}~\cite{PerronS04}\\
Benchmarks & github &  & \href{../cars/works/Siala15.pdf}{Siala15}~\cite{Siala15} & \href{../cars/works/ArtiguesHM0W14.pdf}{ArtiguesHM0W14}~\cite{ArtiguesHM0W14}, \href{../cars/works/Mayer-EichbergerW13.pdf}{Mayer-EichbergerW13}~\cite{Mayer-EichbergerW13}\\
Benchmarks & random instance &  &  & \href{../cars/works/Siala15.pdf}{Siala15}~\cite{Siala15}, \href{../cars/works/ReginP97.pdf}{ReginP97}~\cite{ReginP97}\\
Benchmarks & real-life &  & \href{../cars/works/DincbasSH88.pdf}{DincbasSH88}~\cite{DincbasSH88} & \href{../cars/works/HoevePRS06.pdf}{HoevePRS06}~\cite{HoevePRS06}, \href{../cars/works/ReginP97.pdf}{ReginP97}~\cite{ReginP97}, \href{../cars/works/ParrelloK86.pdf}{ParrelloK86}~\cite{ParrelloK86}\\
Benchmarks & real-world &  &  & \href{../cars/works/Siala15.pdf}{Siala15}~\cite{Siala15}, \href{../cars/works/GottliebPS03.pdf}{GottliebPS03}~\cite{GottliebPS03}\\
\end{longtable}
}


\clearpage
\subsection{Concept Type Algorithms}
\label{sec:Algorithms}
{\scriptsize
\begin{longtable}{lp{3cm}>{\raggedright\arraybackslash}p{6cm}>{\raggedright\arraybackslash}p{6cm}>{\raggedright\arraybackslash}p{8cm}}
\rowcolor{white}\caption{Works for Concepts of Type Algorithms}\\ \toprule
\rowcolor{white}Type & Keyword & High & Medium & Low\\ \midrule\endhead
\bottomrule
\endfoot
Algorithms & GRASP &  &  & \href{cars/works/Siala15.pdf}{Siala15}~\cite{Siala15}\\
Algorithms & IGT &  &  & \\
Algorithms & NEH &  &  & \\
Algorithms & bi-partite matching &  &  & \\
Algorithms & edge-finder &  &  & \href{cars/works/ReginP97.pdf}{ReginP97}~\cite{ReginP97}\\
Algorithms & edge-finding &  & \href{cars/works/Siala15.pdf}{Siala15}~\cite{Siala15} & \\
Algorithms & energetic reasoning &  &  & \\
Algorithms & max-flow &  &  & \\
Algorithms & not-first &  &  & \\
Algorithms & not-last &  &  & \\
Algorithms & sweep &  &  & \\
Algorithms & time-tabling &  & \href{cars/works/Siala15.pdf}{Siala15}~\cite{Siala15} & \href{cars/works/ReginP97.pdf}{ReginP97}~\cite{ReginP97}\\
\end{longtable}
}





\clearpage
\phantomsection
\addcontentsline{toc}{section}{Bibliography}
\bibliographystyle{plainurl}
\bibliography{imports/cars}



\appendix
\clearpage
\section{Papers and Articles Missing a Local Copy}

This section lists all papers and articles for which we were not able to locate an electronic copy that we could download to our system. This might be because the work is behind a paywall for which we do not have access, or since the paper only exists in hardcopy, for works from the start of the period covered. As in either case we are not able to extract useful information from the work, either automatically, or manually, without the actual text itself, these gaps should be closed where possible.

{\scriptsize
\begin{longtable}{p{2cm}p{2cm}p{5cm}p{10cm}rp{3cm}l}
\rowcolor{white}\caption{PAPER without Local Copy}\\ \toprule
\rowcolor{white}Key & URL & Authors & Title & Year & \shortstack{Conference\\/Journal} & Cite\\ \midrule
\endhead
\bottomrule
\endfoot
ThiruvadyME11 & \href{}{ThiruvadyME11} & \hyperref[auth:a26]{Dhananjay Raghavan Thiruvady}, \hyperref[auth:a27]{B. Meyer}, \hyperref[auth:a28]{A. Ernst} & Car sequencing with constraint-based ACO & 2011 & GECCO 2011 & \cite{ThiruvadyME11}\\\end{longtable}
}



{\scriptsize
\begin{longtable}{p{2cm}p{2cm}p{5cm}p{10cm}rp{3cm}l}
\rowcolor{white}\caption{ARTICLE without Local Copy}\\ \toprule
\rowcolor{white}Key & URL & Authors & Title & Year & \shortstack{Conference\\/Journal} & Cite\\ \midrule
\endhead
\bottomrule
\endfoot
YuLZCLW22 & \href{http://dx.doi.org/10.1186/s13638-022-02113-7}{YuLZCLW22} & \hyperref[auth:a55]{Y. Yu}, \hyperref[auth:a56]{X. Lu}, \hyperref[auth:a57]{T. Zhao}, \hyperref[auth:a58]{M. Cheng}, \hyperref[auth:a59]{L. Liu}, \hyperref[auth:a60]{W. Wei} & Heuristic approaches for the car sequencing problems with block batches & 2022 & EURASIP Journal on Wireless Communications and Networking & \cite{YuLZCLW22}\\WinterM21 & \href{}{WinterM21} & \hyperref[auth:a24]{F. Winter}, \hyperref[auth:a25]{N. Musliu} & Constraint-based Scheduling for Paint Shops in the Automotive Supply Industry & 2021 & ACM Transactions on Intelligent Systems and Technology (TIST) & \cite{WinterM21}\\MoyaCB19 & \href{http://dx.doi.org/10.1016/j.cie.2019.106048}{MoyaCB19} & \hyperref[auth:a63]{I. Moya}, \hyperref[auth:a64]{M. Chica}, \hyperref[auth:a65]{J. Bautista} & Constructive metaheuristics for solving the Car Sequencing Problem under uncertain partial demand & 2019 & Computers \  Industrial Engineering & \cite{MoyaCB19}\\YavuzE18 & \href{}{YavuzE18} & \hyperref[auth:a22]{M. Yavuz}, \hyperref[auth:a23]{H. Ergin} & Advanced constraint propagation for the combined car sequencing and level scheduling problem & 2018 & Computers \  Operations Research & \cite{YavuzE18}\\ZhangGWH17 & \href{http://dx.doi.org/10.1007/s10033-017-0083-7}{ZhangGWH17} & \hyperref[auth:a51]{X. ZHANG}, \hyperref[auth:a52]{L. GAO}, \hyperref[auth:a53]{L. WEN}, \hyperref[auth:a54]{Z. HUANG} & Parallel Construction Heuristic Combined with Constraint Propagation for the Car Sequencing Problem & 2017 & Chinese Journal of Mechanical Engineering & \cite{ZhangGWH17}\\MazurN15 & \href{}{MazurN15} & \hyperref[auth:a31]{M. Mazur}, \hyperref[auth:a32]{A. Niederli{\'n}ski} & A Two-stage approach for an optimum solution of the car assembly scheduling problem. Part 2. CLP solution and real-world example & 2015 & Archives of Control Sciences & \cite{MazurN15}\\SialaHH155 & \href{https://doi.org/10.1016/j.engappai.2014.10.009}{SialaHH155} & \hyperref[auth:a11]{M. Siala}, \hyperref[auth:a12]{E. Hebrard}, \hyperref[auth:a13]{M. Huguet} & A study of constraint programming heuristics for the car-sequencing problem & 2015 & Eng. Appl. Artif. Intell. & \cite{SialaHH155}\\GolleRB14 & \href{http://dx.doi.org/10.1007/s10479-014-1733-0}{GolleRB14} & \hyperref[auth:a61]{U. Golle}, \hyperref[auth:a62]{F. Rothlauf}, \hyperref[auth:a48]{N. Boysen} & Iterative beam search for car sequencing & 2014 & Annals of Operations Research & \cite{GolleRB14}\\OzturkTHO13 & \href{https://doi.org/10.1007/s10601-013-9142-6}{OzturkTHO13} & \hyperref[auth:a14]{C. {\"{O}}zt{\"{u}}rk}, \hyperref[auth:a15]{S. Tunali}, \hyperref[auth:a16]{B. Hnich}, \hyperref[auth:a17]{M. Arslan Ornek} & Balancing and scheduling of flexible mixed model assembly lines & 2013 & Constraints An Int. J. & \cite{OzturkTHO13}\\BoysenFS09 & \href{http://dx.doi.org/10.1016/j.ejor.2007.09.013}{BoysenFS09} & \hyperref[auth:a48]{N. Boysen}, \hyperref[auth:a49]{M. Fliedner}, \hyperref[auth:a50]{A. Scholl} & Sequencing mixed-model assembly lines: Survey,  classification and model critique & 2009 & European Journal of Operational Research & \cite{BoysenFS09}\\HoevePRS09 & \href{http://dx.doi.org/10.1007/s10601-008-9067-7}{HoevePRS09} & \hyperref[auth:a39]{Willem-Jan van Hoeve}, \hyperref[auth:a40]{G. Pesant}, \hyperref[auth:a41]{L. Rousseau}, \hyperref[auth:a42]{A. Sabharwal} & New filtering algorithms for combinations of among constraints & 2009 & Constraints An Int. J. & \cite{HoevePRS09}\\Schaus09 & \href{}{Schaus09} & \hyperref[auth:a19]{P. Schaus} & Solving balancing and bin-packing problems with constraint programming & 2009 & These de doctorat, Universit{\'e} catholique de Louvain & \cite{Schaus09}\\SolnonCNA08 & \href{https://doi.org/10.1016/j.ejor.2007.04.033}{SolnonCNA08} & \hyperref[auth:a5]{C. Solnon}, \hyperref[auth:a6]{V. Cung}, \hyperref[auth:a7]{A. Nguyen}, \hyperref[auth:a8]{C. Artigues} & The car sequencing problem: Overview of state-of-the-art methods and industrial case-study of the ROADEF'2005 challenge problem & 2008 & European Journal of Operational Research & \cite{SolnonCNA08}\\Kis04 & \href{http://dx.doi.org/10.1016/j.orl.2003.09.003}{Kis04} & \hyperref[auth:a47]{T. Kis} & On the complexity of the car sequencing problem & 2004 & Operations Research Letters & \cite{Kis04}\\Gent98 & \href{}{Gent98} & \hyperref[auth:a18]{Ian P Gent} & Two results on car-sequencing problems & 1998 & Report University of Strathclyde, APES-02-98 & \cite{Gent98}\\WarwickT95 & \href{http://dx.doi.org/10.1162/evco.1995.3.3.267}{WarwickT95} & \hyperref[auth:a45]{T. Warwick}, \hyperref[auth:a46]{Edward P. K. Tsang} & Tackling Car Sequencing Problems Using a Generic Genetic Algorithm & 1995 & Evolutionary Computation & \cite{WarwickT95}\\HindiP94 & \href{http://dx.doi.org/10.1016/0360-8352(94)90038-8}{HindiP94} & \hyperref[auth:a37]{Khalil S. Hindi}, \hyperref[auth:a38]{G. Ploszajski} & Formulation and solution of a selection and sequencing problem in car manufacture & 1994 & Computers \  Industrial Engineering & \cite{HindiP94}\\\end{longtable}
}



\clearpage
\section{Papers and Articles Without Recognized Concepts}

This section lists papers and articles for which we have a pdf local copy, but where we were not able to extract any of the defined concepts. This can basically have two reasons. We either have included a paper which is not at all related to scheduling, so that none of the defined concepts occur in the paper. A  more likely cause is that the pdf file is a scanned document for which optical character recognition was not run or not successful, so that the pdf consists of a series of bitmap images. In that case, pdfgrep is unable to find any text in the document, and no matches for concepts are found. It may be useful to check the pdf files to see if that is the case.

{\scriptsize
\begin{longtable}{llp{5cm}p{10cm}rp{3cm}l}
\caption{Paper without Concepts}\\ \toprule
Key & \shortstack{Local\\Copy} & Authors & Title & Year & \shortstack{Conference\\/Journal} & Cite\\ \midrule
\endhead
\bottomrule
\endfoot
BaptisteLV92 & \href{papers/BaptisteLV92.pdf}{Yes} & Pierre Baptiste and Bruno Legeard and Christophe Varnier & Hoist scheduling problem: an approach based on constraint logic programming & 1992 & ICRA 1992 & \cite{BaptisteLV92}\\\end{longtable}
}



{\scriptsize
\begin{longtable}{llp{5cm}p{10cm}rp{3cm}lr}
\rowcolor{white}\caption{ARTICLE without Concepts}\\ \toprule
\rowcolor{white}Key & \shortstack{Local\\Copy} & Authors & Title & Year & \shortstack{Conference\\/Journal} & Cite & Pages\\ \midrule
\endhead
\bottomrule
\endfoot
KorbaaYG00 & \href{works/KorbaaYG00.pdf}{Yes} & O. Korbaa, P. Yim, J. Gentina & Solving Transient Scheduling Problems with Constraint Programming & 2000 & Eur. J. Control & \cite{KorbaaYG00} & 10\\LopezAKYG00 & \href{works/LopezAKYG00.pdf}{Yes} & P. Lopez, H. Alla, O. Korbaa, P. Yim, J. Gentina & Discussion on: 'Solving Transient Scheduling Problems with Constraint Programming' by O. Korbaa, P. Yim, and {J.-C.} Gentina & 2000 & Eur. J. Control & \cite{LopezAKYG00} & 4\\\end{longtable}
}



\clearpage
\section{Unmatched Concepts}

This section lists those concepts for which no matches were found. The most likely cause is a mistake in the regular expression used to find the concept, but it is also possible that some concept simply is not mentioned in any of the documents. 

{\scriptsize
\begin{longtable}{lp{10cm}rr}
\rowcolor{white}\caption{Unmatched Concepts}\\ \toprule
\rowcolor{white}Type & Name & CaseSensitive & Revision\\ \midrule
\endhead
\bottomrule
\endfoot
Algorithms & IGT & Y & 0\\Algorithms & NEH & Y & 0\\Algorithms & bi-partite matching &  & 0\\Algorithms & energetic reasoning &  & 0\\Algorithms & max-flow &  & 0\\Algorithms & not-first &  & 0\\Algorithms & not-last &  & 0\\Algorithms & sweep &  & 0\\Benchmarks & gitlab &  & 0\\Benchmarks & industry partner &  & 0\\Benchmarks & instance generator &  & 0\\Benchmarks & supplementary material &  & 0\\Benchmarks & zenodo &  & 0\\CPSystems & CPO &  & 0\\CPSystems & Choco Solver &  & 0\\CPSystems & Chuffed &  & 0\\CPSystems & ECLiPSe &  & 0\\CPSystems & Gecode &  & 0\\CPSystems & MiniZinc &  & 0\\CPSystems & OR-Tools &  & 0\\CPSystems & SCIP & Y & 0\\CPSystems & SICStus &  & 0\\CPSystems & Z3 &  & 0\\ProgLanguages & C  &  & 0\\ProgLanguages & Julia &  & 0\\ProgLanguages & Lisp &  & 0\\ProgLanguages & Python &  & 0\\Industries & IT industry & Y & 0\\Industries & PCB industry &  & 0\\Industries & aerospace industry &  & 0\\Industries & agricultural industry &  & 0\\Industries & agrifood industry &  & 0\\Industries & airline industry &  & 0\\Industries & aviation industry &  & 0\\Industries & cable industry &  & 0\\Industries & carpet industry &  & 0\\Industries & chemical industry &  & 0\\Industries & chemical processing industry &  & 0\\Industries & chemistry industry &  & 0\\Industries & chips industry &  & 0\\Industries & circuit boards industry &  & 0\\Industries & control system industry &  & 0\\Industries & cutting industry &  & 0\\Industries & dairy industry &  & 0\\Industries & dismantling industry &  & 0\\Industries & drawing industry &  & 0\\Industries & electricity industry &  & 0\\Industries & electronics industry &  & 0\\Industries & electroplating industry &  & 0\\Industries & energy industry &  & 0\\Industries & fashion industry &  & 0\\Industries & food industry &  & 0\\Industries & food-processing industry &  & 0\\Industries & forest industry &  & 0\\Industries & forging industry &  & 0\\Industries & foundry industry &  & 0\\Industries & garment industry &  & 0\\Industries & gas industry &  & 0\\Industries & glass industry &  & 0\\Industries & heavy industry &  & 0\\Industries & insulation industry &  & 0\\Industries & leisure industry &  & 0\\Industries & lumber industry &  & 0\\Industries & manufacturing industry &  & 0\\Industries & maritime industry &  & 0\\Industries & metal industry &  & 0\\Industries & metalworking industry &  & 0\\Industries & mineral industry &  & 0\\Industries & mining industry &  & 0\\Industries & nuclear industry &  & 0\\Industries & oil industry &  & 0\\Industries & packaging industry &  & 0\\Industries & painting industry &  & 0\\Industries & paper industry &  & 0\\Industries & petro-chemical industry &  & 0\\Industries & pharmaceutical industry &  & 0\\Industries & potash industry &  & 0\\Industries & power industry &  & 0\\Industries & printing industry &  & 0\\Industries & process industry &  & 0\\Industries & processing industry &  & 0\\Industries & railway industry &  & 0\\Industries & repair industry &  & 0\\Industries & retail industry &  & 0\\Industries & semiconductor industry &  & 0\\Industries & semiprocess industry &  & 0\\Industries & service industry &  & 0\\Industries & ship repair industry &  & 0\\Industries & shipping industry &  & 0\\Industries & software industry &  & 0\\Industries & solar cell industry &  & 0\\Industries & steel industry &  & 0\\Industries & steel making industry &  & 0\\Industries & sugar industry &  & 0\\Industries & taxi industry &  & 0\\Industries & telecommunication industry &  & 0\\Industries & textile industry &  & 0\\Industries & tire industry &  & 0\\Industries & tourism industry &  & 0\\Industries & trade industry &  & 0\\Industries & transportation industry &  & 0\\Industries & wind industry &  & 0\\ApplicationAreas & COVID &  & 0\\ApplicationAreas & HVAC &  & 0\\ApplicationAreas & agriculture &  & 0\\ApplicationAreas & aircraft &  & 0\\ApplicationAreas & cable tree &  & 0\\ApplicationAreas & container terminal &  & 0\\ApplicationAreas & dairies &  & 0\\ApplicationAreas & dairy &  & 0\\ApplicationAreas & datacenter &  & 0\\ApplicationAreas & datacentre &  & 0\\ApplicationAreas & day-ahead market &  & 0\\ApplicationAreas & deep space &  & 0\\ApplicationAreas & drone &  & 0\\ApplicationAreas & earth observation &  & 0\\ApplicationAreas & earth orbit &  & 0\\ApplicationAreas & electroplating &  & 0\\ApplicationAreas & emergency service &  & 0\\ApplicationAreas & energy-price &  & 0\\ApplicationAreas & farming &  & 0\\ApplicationAreas & forestry &  & 0\\ApplicationAreas & hoist &  & 0\\ApplicationAreas & medical &  & 0\\ApplicationAreas & offshore &  & 0\\ApplicationAreas & operating room &  & 0\\ApplicationAreas & oven scheduling &  & 0\\ApplicationAreas & patient &  & 0\\ApplicationAreas & perfect-square &  & 0\\ApplicationAreas & physician &  & 0\\ApplicationAreas & pipeline &  & 0\\ApplicationAreas & radiation therapy &  & 0\\ApplicationAreas & real-time pricing &  & 0\\ApplicationAreas & satellite &  & 0\\ApplicationAreas & semiconductor &  & 0\\ApplicationAreas & ship building &  & 0\\ApplicationAreas & shipping line &  & 0\\ApplicationAreas & steel cable &  & 0\\ApplicationAreas & steel mill &  & 0\\ApplicationAreas & super-computer &  & 0\\ApplicationAreas & surgery &  & 0\\ApplicationAreas & torpedo &  & 0\\ApplicationAreas & vaccine &  & 0\\ApplicationAreas & yard crane &  & 0\\Constraints & AllDiff constraint &  & 0\\Constraints & AllDiffPrec constraint &  & 0\\Constraints & AlwaysConstant &  & 0\\Constraints & Arithmetic constraint &  & 0\\Constraints & BinPacking constraint &  & 0\\Constraints & Blocking constraint &  & 0\\Constraints & BufferedResource &  & 0\\Constraints & Calendar constraint &  & 0\\Constraints & CumulativeCost &  & 0\\Constraints & Cumulatives constraint &  & 0\\Constraints & Diff2 constraint &  & 0\\Constraints & Element constraint &  & 0\\Constraints & GeneralizedAllDiffPrec &  & 0\\Constraints & IloAlternative &  & 0\\Constraints & IloAlwaysIn &  & 0\\Constraints & IloForbidEnd &  & 0\\Constraints & IloNoOverlap &  & 0\\Constraints & IloPack &  & 0\\Constraints & IloPulse &  & 0\\Constraints & MinWeightAllDiff &  & 0\\Constraints & PreemptiveNoOverlap &  & 0\\Constraints & Pulse constraint &  & 0\\Constraints & RelSoftCumulative &  & 0\\Constraints & RelSoftCumulativeSum &  & 0\\Constraints & SoftCumulative &  & 0\\Constraints & SoftCumulativeSum &  & 0\\Constraints & TaskIntersection constraint &  & 0\\Constraints & UTVPI constraint &  & 0\\Constraints & WeightAllDiff &  & 0\\Constraints & WeightedSum &  & 0\\Constraints & WeightedTaskSum &  & 0\\Constraints & alternative constraint &  & 0\\Constraints & alwaysEqual constraint &  & 0\\Constraints & alwaysIn &  & 0\\Constraints & bin-packing &  & 0\\Constraints & diffn &  & 0\\Constraints & endBeforeStart &  & 0\\Constraints & geost &  & 0\\Constraints & noOverlap &  & 0\\Constraints & regular expression &  & 0\\Constraints & span constraint &  & 0\\Classification & 2BPHFSP & Y & 1\\Classification & BPCTOP & Y & 1\\Classification & Bulk Port Cargo Throughput Optimisation Problem &  & 0\\Classification & CECSP & Y & 1\\Classification & CHSP & Y & 1\\Classification & CTW & Y & 1\\Classification & CuSP &  & 0\\Classification & EOSP & Y & 1\\Classification & Earth Observation Scheduling Problem &  & 0\\Classification & FJS & Y & 1\\Classification & Fixed Job Scheduling &  & 0\\Classification & GCSP & Y & 1\\Classification & HFF & Y & 1\\Classification & HFFTT & Y & 1\\Classification & HFS & Y & 1\\Classification & JSPT & Y & 1\\Classification & JSSP & Y & 1\\Classification & KRFP & Y & 1\\Classification & LSFRP & Y & 1\\Classification & Liner Shipping Fleet Repositioning Problem &  & 0\\Classification & MGAP & Y & 1\\Classification & Modified Generalized Assignment Problem &  & 0\\Classification & OSSP & Y & 1\\Classification & Open Shop Scheduling Problem &  & 0\\Classification & PJSSP & Y & 1\\Classification & PMSP & Y & 1\\Classification & PP-MS-MMRCPSP & Y & 1\\Classification & PTC & Y & 1\\Classification & Pre-emptive Job-Shop scheduling Problem &  & 0\\Classification & RCMPSP & Y & 1\\Classification & RCPSPDC & Y & 1\\Classification & Resource-constrained Project Scheduling Problem with Discounted Cashflow &  & 0\\Classification & SCC & Y & 1\\Classification & SMSDP & Y & 1\\Classification & Steel-making and continuous casting &  & 0\\Classification & TCSP & Y & 1\\Classification & Temporal Constraint Satisfaction Problem &  & 0\\Classification & psplib &  & 0\\Concepts & Allen's algebra &  & 0\\Concepts & Benders Decomposition &  & 0\\Concepts & Logic-Based Benders Decomposition &  & 0\\Concepts & batch process &  & 0\\Concepts & blocking constraint &  & 0\\Concepts & buffer-capacity &  & 0\\Concepts & continuous-process &  & 0\\Concepts & flow-time &  & 0\\Concepts & lateness &  & 0\\Concepts & make to order &  & 0\\Concepts & make to stock &  & 1\\Concepts & manpower &  & 0\\Concepts & no preempt &  & 0\\Concepts & no-wait &  & 0\\Concepts & producer/consumer &  & 0\\Concepts & re-scheduling &  & 0\\Concepts & release-date &  & 0\\Concepts & stock level &  & 0\\Concepts & temporal constraint reasoning &  & 0\\\end{longtable}
}



\clearpage
\section{Works by Author}

\subsection{Works by J. Christopher Beck}
\label{sec:a89}
{\scriptsize
\begin{longtable}{>{\raggedright\arraybackslash}p{3cm}>{\raggedright\arraybackslash}p{6cm}>{\raggedright\arraybackslash}p{7cm}rrrp{3cm}rrr}
\rowcolor{white}\caption{Works from bibtex (Total 46)}\\ \toprule
\rowcolor{white}Key & Authors & Title & LC & Cite & Year & \shortstack{Conference\\/Journal} & Pages & b & c \\ \midrule\endhead
\bottomrule
\endfoot
LuoB22 \href{https://doi.org/10.1007/978-3-031-08011-1\_17}{LuoB22} & \hyperref[auth:a754]{Yiqing L. Luo}, \hyperref[auth:a89]{J. Christopher Beck} & Packing by Scheduling: Using Constraint Programming to Solve a Complex 2D Cutting Stock Problem & \href{works/LuoB22.pdf}{Yes} & \cite{LuoB22} & 2022 & CPAIOR 2022 & 17 & \ref{b:LuoB22} & \ref{c:LuoB22}\\
ZhangBB22 \href{https://ojs.aaai.org/index.php/ICAPS/article/view/19826}{ZhangBB22} & \hyperref[auth:a808]{J. Zhang}, \hyperref[auth:a809]{Giovanni Lo Bianco}, \hyperref[auth:a89]{J. Christopher Beck} & Solving Job-Shop Scheduling Problems with QUBO-Based Specialized Hardware & \href{works/ZhangBB22.pdf}{Yes} & \cite{ZhangBB22} & 2022 & ICAPS 2022 & 9 & \ref{b:ZhangBB22} & \ref{c:ZhangBB22}\\
TangB20 \href{https://doi.org/10.1007/978-3-030-58942-4\_28}{TangB20} & \hyperref[auth:a88]{Tanya Y. Tang}, \hyperref[auth:a89]{J. Christopher Beck} & {CP} and Hybrid Models for Two-Stage Batching and Scheduling & \href{works/TangB20.pdf}{Yes} & \cite{TangB20} & 2020 & CPAIOR 2020 & 16 & \ref{b:TangB20} & \ref{c:TangB20}\\
TranPZLDB18 \href{https://doi.org/10.1007/s10951-017-0537-x}{TranPZLDB18} & \hyperref[auth:a810]{Tony T. Tran}, \hyperref[auth:a811]{M. Padmanabhan}, \hyperref[auth:a812]{Peter Yun Zhang}, \hyperref[auth:a813]{H. Li}, \hyperref[auth:a814]{Douglas G. Down}, \hyperref[auth:a89]{J. Christopher Beck} & Multi-stage resource-aware scheduling for data centers with heterogeneous servers & \href{works/TranPZLDB18.pdf}{Yes} & \cite{TranPZLDB18} & 2018 & J. Sched. & 17 & \ref{b:TranPZLDB18} & \ref{c:TranPZLDB18}\\
CohenHB17 \href{https://doi.org/10.1007/978-3-319-66263-3\_10}{CohenHB17} & \hyperref[auth:a816]{E. Cohen}, \hyperref[auth:a817]{G. Huang}, \hyperref[auth:a89]{J. Christopher Beck} & {(I} Can Get) Satisfaction: Preference-Based Scheduling for Concert-Goers at Multi-venue Music Festivals & \href{works/CohenHB17.pdf}{Yes} & \cite{CohenHB17} & 2017 & SAT 2017 & 17 & \ref{b:CohenHB17} & \ref{c:CohenHB17}\\
TranVNB17 \href{https://doi.org/10.1613/jair.5306}{TranVNB17} & \hyperref[auth:a810]{Tony T. Tran}, \hyperref[auth:a815]{Tiago Stegun Vaquero}, \hyperref[auth:a209]{G. Nejat}, \hyperref[auth:a89]{J. Christopher Beck} & Robots in Retirement Homes: Applying Off-the-Shelf Planning and Scheduling to a Team of Assistive Robots & \href{works/TranVNB17.pdf}{Yes} & \cite{TranVNB17} & 2017 & J. Artif. Intell. Res. & 68 & \ref{b:TranVNB17} & \ref{c:TranVNB17}\\
TranVNB17a \href{https://doi.org/10.24963/ijcai.2017/726}{TranVNB17a} & \hyperref[auth:a810]{Tony T. Tran}, \hyperref[auth:a815]{Tiago Stegun Vaquero}, \hyperref[auth:a209]{G. Nejat}, \hyperref[auth:a89]{J. Christopher Beck} & Robots in Retirement Homes: Applying Off-the-Shelf Planning and Scheduling to a Team of Assistive Robots (Extended Abstract) & \href{works/TranVNB17a.pdf}{Yes} & \cite{TranVNB17a} & 2017 & IJCAI 2017 & 5 & \ref{b:TranVNB17a} & \ref{c:TranVNB17a}\\
BoothNB16 \href{https://doi.org/10.1007/978-3-319-44953-1\_34}{BoothNB16} & \hyperref[auth:a208]{Kyle E. C. Booth}, \hyperref[auth:a209]{G. Nejat}, \hyperref[auth:a89]{J. Christopher Beck} & A Constraint Programming Approach to Multi-Robot Task Allocation and Scheduling in Retirement Homes & \href{works/BoothNB16.pdf}{Yes} & \cite{BoothNB16} & 2016 & CP 2016 & 17 & \ref{b:BoothNB16} & \ref{c:BoothNB16}\\
KuB16 \href{https://doi.org/10.1016/j.cor.2016.04.006}{KuB16} & \hyperref[auth:a336]{W. Ku}, \hyperref[auth:a89]{J. Christopher Beck} & Mixed Integer Programming models for job shop scheduling: {A} computational analysis & No & \cite{KuB16} & 2016 & Comput. Oper. Res. & 9 & No & \ref{c:KuB16}\\
LuoVLBM16 \href{http://www.aaai.org/ocs/index.php/KR/KR16/paper/view/12909}{LuoVLBM16} & \hyperref[auth:a824]{R. Luo}, \hyperref[auth:a825]{Richard Anthony Valenzano}, \hyperref[auth:a826]{Y. Li}, \hyperref[auth:a89]{J. Christopher Beck}, \hyperref[auth:a827]{Sheila A. McIlraith} & Using Metric Temporal Logic to Specify Scheduling Problems & \href{works/LuoVLBM16.pdf}{Yes} & \cite{LuoVLBM16} & 2016 & KR 2016 & 4 & \ref{b:LuoVLBM16} & \ref{c:LuoVLBM16}\\
TranAB16 \href{https://doi.org/10.1287/ijoc.2015.0666}{TranAB16} & \hyperref[auth:a810]{Tony T. Tran}, \hyperref[auth:a818]{A. Araujo}, \hyperref[auth:a89]{J. Christopher Beck} & Decomposition Methods for the Parallel Machine Scheduling Problem with Setups & No & \cite{TranAB16} & 2016 & {INFORMS} J. Comput. & 13 & No & \ref{c:TranAB16}\\
TranDRFWOVB16 \href{https://doi.org/10.1609/socs.v7i1.18390}{TranDRFWOVB16} & \hyperref[auth:a810]{Tony T. Tran}, \hyperref[auth:a820]{M. Do}, \hyperref[auth:a821]{Eleanor Gilbert Rieffel}, \hyperref[auth:a383]{J. Frank}, \hyperref[auth:a819]{Z. Wang}, \hyperref[auth:a822]{B. O'Gorman}, \hyperref[auth:a823]{D. Venturelli}, \hyperref[auth:a89]{J. Christopher Beck} & A Hybrid Quantum-Classical Approach to Solving Scheduling Problems & \href{works/TranDRFWOVB16.pdf}{Yes} & \cite{TranDRFWOVB16} & 2016 & SOCS 2016 & 9 & \ref{b:TranDRFWOVB16} & \ref{c:TranDRFWOVB16}\\
TranWDRFOVB16 \href{http://www.aaai.org/ocs/index.php/WS/AAAIW16/paper/view/12664}{TranWDRFOVB16} & \hyperref[auth:a810]{Tony T. Tran}, \hyperref[auth:a819]{Z. Wang}, \hyperref[auth:a820]{M. Do}, \hyperref[auth:a821]{Eleanor Gilbert Rieffel}, \hyperref[auth:a383]{J. Frank}, \hyperref[auth:a822]{B. O'Gorman}, \hyperref[auth:a823]{D. Venturelli}, \hyperref[auth:a89]{J. Christopher Beck} & Explorations of Quantum-Classical Approaches to Scheduling a Mars Lander Activity Problem & \href{works/TranWDRFOVB16.pdf}{Yes} & \cite{TranWDRFOVB16} & 2016 & AAAI 2016 & 9 & \ref{b:TranWDRFOVB16} & \ref{c:TranWDRFOVB16}\\
BajestaniB15 \href{https://doi.org/10.1007/s10951-015-0416-2}{BajestaniB15} & \hyperref[auth:a828]{Maliheh Aramon Bajestani}, \hyperref[auth:a89]{J. Christopher Beck} & A two-stage coupled algorithm for an integrated maintenance planning and flowshop scheduling problem with deteriorating machines & \href{works/BajestaniB15.pdf}{Yes} & \cite{BajestaniB15} & 2015 & J. Sched. & 16 & \ref{b:BajestaniB15} & \ref{c:BajestaniB15}\\
KoschB14 \href{https://doi.org/10.1007/978-3-319-07046-9\_5}{KoschB14} & \hyperref[auth:a332]{S. Kosch}, \hyperref[auth:a89]{J. Christopher Beck} & A New {MIP} Model for Parallel-Batch Scheduling with Non-identical Job Sizes & \href{works/KoschB14.pdf}{Yes} & \cite{KoschB14} & 2014 & CPAIOR 2014 & 16 & \ref{b:KoschB14} & \ref{c:KoschB14}\\
LouieVNB14 \href{https://doi.org/10.1109/ICRA.2014.6907637}{LouieVNB14} & \hyperref[auth:a830]{Wing{-}Yue Geoffrey Louie}, \hyperref[auth:a815]{Tiago Stegun Vaquero}, \hyperref[auth:a209]{G. Nejat}, \hyperref[auth:a89]{J. Christopher Beck} & An autonomous assistive robot for planning, scheduling and facilitating multi-user activities & No & \cite{LouieVNB14} & 2014 & ICRA 2014 & 7 & No & \ref{c:LouieVNB14}\\
TerekhovTDB14 \href{https://doi.org/10.1613/jair.4278}{TerekhovTDB14} & \hyperref[auth:a829]{D. Terekhov}, \hyperref[auth:a810]{Tony T. Tran}, \hyperref[auth:a814]{Douglas G. Down}, \hyperref[auth:a89]{J. Christopher Beck} & Integrating Queueing Theory and Scheduling for Dynamic Scheduling Problems & \href{works/TerekhovTDB14.pdf}{Yes} & \cite{TerekhovTDB14} & 2014 & J. Artif. Intell. Res. & 38 & \ref{b:TerekhovTDB14} & \ref{c:TerekhovTDB14}\\
BajestaniB13 \href{https://doi.org/10.1613/jair.3902}{BajestaniB13} & \hyperref[auth:a828]{Maliheh Aramon Bajestani}, \hyperref[auth:a89]{J. Christopher Beck} & Scheduling a Dynamic Aircraft Repair Shop with Limited Repair Resources & \href{works/BajestaniB13.pdf}{Yes} & \cite{BajestaniB13} & 2013 & J. Artif. Intell. Res. & 36 & \ref{b:BajestaniB13} & \ref{c:BajestaniB13}\\
HeinzKB13 \href{https://doi.org/10.1007/978-3-642-38171-3\_2}{HeinzKB13} & \hyperref[auth:a133]{S. Heinz}, \hyperref[auth:a336]{W. Ku}, \hyperref[auth:a89]{J. Christopher Beck} & Recent Improvements Using Constraint Integer Programming for Resource Allocation and Scheduling & \href{works/HeinzKB13.pdf}{Yes} & \cite{HeinzKB13} & 2013 & CPAIOR 2013 & 16 & \ref{b:HeinzKB13} & \ref{c:HeinzKB13}\\
HeinzSB13 \href{https://doi.org/10.1007/s10601-012-9136-9}{HeinzSB13} & \hyperref[auth:a133]{S. Heinz}, \hyperref[auth:a134]{J. Schulz}, \hyperref[auth:a89]{J. Christopher Beck} & Using dual presolving reductions to reformulate cumulative constraints & \href{works/HeinzSB13.pdf}{Yes} & \cite{HeinzSB13} & 2013 & Constraints An Int. J. & 36 & \ref{b:HeinzSB13} & \ref{c:HeinzSB13}\\
TranTDB13 \href{http://www.aaai.org/ocs/index.php/ICAPS/ICAPS13/paper/view/6005}{TranTDB13} & \hyperref[auth:a810]{Tony T. Tran}, \hyperref[auth:a829]{D. Terekhov}, \hyperref[auth:a814]{Douglas G. Down}, \hyperref[auth:a89]{J. Christopher Beck} & Hybrid Queueing Theory and Scheduling Models for Dynamic Environments with Sequence-Dependent Setup Times & \href{works/TranTDB13.pdf}{Yes} & \cite{TranTDB13} & 2013 & ICAPS 2013 & 9 & \ref{b:TranTDB13} & \ref{c:TranTDB13}\\
HeinzB12 \href{https://doi.org/10.1007/978-3-642-29828-8\_14}{HeinzB12} & \hyperref[auth:a133]{S. Heinz}, \hyperref[auth:a89]{J. Christopher Beck} & Reconsidering Mixed Integer Programming and MIP-Based Hybrids for Scheduling & \href{works/HeinzB12.pdf}{Yes} & \cite{HeinzB12} & 2012 & CPAIOR 2012 & 17 & \ref{b:HeinzB12} & \ref{c:HeinzB12}\\
TerekhovDOB12 \href{https://doi.org/10.1016/j.cie.2012.02.006}{TerekhovDOB12} & \hyperref[auth:a829]{D. Terekhov}, \hyperref[auth:a831]{Mustafa K. Dogru}, \hyperref[auth:a832]{U. {\"{O}}zen}, \hyperref[auth:a89]{J. Christopher Beck} & Solving two-machine assembly scheduling problems with inventory constraints & No & \cite{TerekhovDOB12} & 2012 & Comput. Ind. Eng. & 15 & No & \ref{c:TerekhovDOB12}\\
TranB12 \href{https://doi.org/10.3233/978-1-61499-098-7-774}{TranB12} & \hyperref[auth:a810]{Tony T. Tran}, \hyperref[auth:a89]{J. Christopher Beck} & Logic-based Benders Decomposition for Alternative Resource Scheduling with Sequence Dependent Setups & \href{works/TranB12.pdf}{Yes} & \cite{TranB12} & 2012 & ECAI 2012 & 6 & \ref{b:TranB12} & \ref{c:TranB12}\\
BajestaniB11 \href{http://aaai.org/ocs/index.php/ICAPS/ICAPS11/paper/view/2680}{BajestaniB11} & \hyperref[auth:a828]{Maliheh Aramon Bajestani}, \hyperref[auth:a89]{J. Christopher Beck} & Scheduling an Aircraft Repair Shop & \href{works/BajestaniB11.pdf}{Yes} & \cite{BajestaniB11} & 2011 & ICAPS 2011 & 8 & \ref{b:BajestaniB11} & \ref{c:BajestaniB11}\\
BeckFW11 \href{https://doi.org/10.1287/ijoc.1100.0388}{BeckFW11} & \hyperref[auth:a89]{J. Christopher Beck}, \hyperref[auth:a833]{T. K. Feng}, \hyperref[auth:a364]{J. Watson} & Combining Constraint Programming and Local Search for Job-Shop Scheduling & \href{works/BeckFW11.pdf}{Yes} & \cite{BeckFW11} & 2011 & {INFORMS} J. Comput. & 14 & \ref{b:BeckFW11} & \ref{c:BeckFW11}\\
HeckmanB11 \href{https://doi.org/10.1007/s10951-009-0113-0}{HeckmanB11} & \hyperref[auth:a834]{I. Heckman}, \hyperref[auth:a89]{J. Christopher Beck} & Understanding the behavior of Solution-Guided Search for job-shop scheduling & \href{works/HeckmanB11.pdf}{Yes} & \cite{HeckmanB11} & 2011 & J. Sched. & 20 & \ref{b:HeckmanB11} & \ref{c:HeckmanB11}\\
KovacsB11 \href{https://doi.org/10.1007/s10601-009-9088-x}{KovacsB11} & \hyperref[auth:a146]{A. Kov{\'{a}}cs}, \hyperref[auth:a89]{J. Christopher Beck} & A global constraint for total weighted completion time for unary resources & \href{works/KovacsB11.pdf}{Yes} & \cite{KovacsB11} & 2011 & Constraints An Int. J. & 24 & \ref{b:KovacsB11} & \ref{c:KovacsB11}\\
BidotVLB09 \href{https://doi.org/10.1007/s10951-008-0080-x}{BidotVLB09} & \hyperref[auth:a835]{J. Bidot}, \hyperref[auth:a836]{T. Vidal}, \hyperref[auth:a118]{P. Laborie}, \hyperref[auth:a89]{J. Christopher Beck} & A theoretic and practical framework for scheduling in a stochastic environment & \href{works/BidotVLB09.pdf}{Yes} & \cite{BidotVLB09} & 2009 & J. Sched. & 30 & \ref{b:BidotVLB09} & \ref{c:BidotVLB09}\\
WuBB09 \href{https://doi.org/10.1016/j.cor.2008.08.008}{WuBB09} & \hyperref[auth:a276]{Christine Wei Wu}, \hyperref[auth:a222]{Kenneth N. Brown}, \hyperref[auth:a89]{J. Christopher Beck} & Scheduling with uncertain durations: Modeling beta-robust scheduling with constraints & No & \cite{WuBB09} & 2009 & Comput. Oper. Res. & 9 & No & \ref{c:WuBB09}\\
KovacsB08 \href{https://doi.org/10.1016/j.engappai.2008.03.004}{KovacsB08} & \hyperref[auth:a146]{A. Kov{\'{a}}cs}, \hyperref[auth:a89]{J. Christopher Beck} & A global constraint for total weighted completion time for cumulative resources & \href{works/KovacsB08.pdf}{Yes} & \cite{KovacsB08} & 2008 & Eng. Appl. Artif. Intell. & 7 & \ref{b:KovacsB08} & \ref{c:KovacsB08}\\
WatsonB08 \href{https://doi.org/10.1007/978-3-540-68155-7\_21}{WatsonB08} & \hyperref[auth:a364]{J. Watson}, \hyperref[auth:a89]{J. Christopher Beck} & A Hybrid Constraint Programming / Local Search Approach to the Job-Shop Scheduling Problem & \href{works/WatsonB08.pdf}{Yes} & \cite{WatsonB08} & 2008 & CPAIOR 2008 & 15 & \ref{b:WatsonB08} & \ref{c:WatsonB08}\\
Beck07 \href{https://doi.org/10.1613/jair.2169}{Beck07} & \hyperref[auth:a89]{J. Christopher Beck} & Solution-Guided Multi-Point Constructive Search for Job Shop Scheduling & \href{works/Beck07.pdf}{Yes} & \cite{Beck07} & 2007 & J. Artif. Intell. Res. & 29 & \ref{b:Beck07} & \ref{c:Beck07}\\
BeckW07 \href{https://doi.org/10.1613/jair.2080}{BeckW07} & \hyperref[auth:a89]{J. Christopher Beck}, \hyperref[auth:a837]{N. Wilson} & Proactive Algorithms for Job Shop Scheduling with Probabilistic Durations & \href{works/BeckW07.pdf}{Yes} & \cite{BeckW07} & 2007 & J. Artif. Intell. Res. & 50 & \ref{b:BeckW07} & \ref{c:BeckW07}\\
KovacsB07 \href{https://doi.org/10.1007/978-3-540-72397-4\_9}{KovacsB07} & \hyperref[auth:a146]{A. Kov{\'{a}}cs}, \hyperref[auth:a89]{J. Christopher Beck} & A Global Constraint for Total Weighted Completion Time & \href{works/KovacsB07.pdf}{Yes} & \cite{KovacsB07} & 2007 & CPAIOR 2007 & 15 & \ref{b:KovacsB07} & \ref{c:KovacsB07}\\
Beck06 \href{http://www.aaai.org/Library/ICAPS/2006/icaps06-028.php}{Beck06} & \hyperref[auth:a89]{J. Christopher Beck} & An Empirical Study of Multi-Point Constructive Search for Constraint-Based Scheduling & No & \cite{Beck06} & 2006 & ICAPS 2006 & 10 & No & \ref{c:Beck06}\\
BeckW05 \href{http://ijcai.org/Proceedings/05/Papers/0748.pdf}{BeckW05} & \hyperref[auth:a89]{J. Christopher Beck}, \hyperref[auth:a837]{N. Wilson} & Proactive Algorithms for Scheduling with Probabilistic Durations & \href{works/BeckW05.pdf}{Yes} & \cite{BeckW05} & 2005 & IJCAI 2005 & 6 & \ref{b:BeckW05} & \ref{c:BeckW05}\\
CarchraeBF05 \href{https://doi.org/10.1007/11564751\_80}{CarchraeBF05} & \hyperref[auth:a274]{T. Carchrae}, \hyperref[auth:a89]{J. Christopher Beck}, \hyperref[auth:a275]{Eugene C. Freuder} & Methods to Learn Abstract Scheduling Models & \href{works/CarchraeBF05.pdf}{Yes} & \cite{CarchraeBF05} & 2005 & CP 2005 & 1 & \ref{b:CarchraeBF05} & \ref{c:CarchraeBF05}\\
WuBB05 \href{https://doi.org/10.1007/11564751\_110}{WuBB05} & \hyperref[auth:a276]{Christine Wei Wu}, \hyperref[auth:a222]{Kenneth N. Brown}, \hyperref[auth:a89]{J. Christopher Beck} & Scheduling with Uncertain Start Dates & \href{works/WuBB05.pdf}{Yes} & \cite{WuBB05} & 2005 & CP 2005 & 1 & \ref{b:WuBB05} & \ref{c:WuBB05}\\
BeckW04 \href{}{BeckW04} & \hyperref[auth:a89]{J. Christopher Beck}, \hyperref[auth:a837]{N. Wilson} & Job Shop Scheduling with Probabilistic Durations & No & \cite{BeckW04} & 2004 & ECAI 2004 & 5 & No & \ref{c:BeckW04}\\
BeckPS03 \href{http://www.aaai.org/Library/ICAPS/2003/icaps03-027.php}{BeckPS03} & \hyperref[auth:a89]{J. Christopher Beck}, \hyperref[auth:a838]{P. Prosser}, \hyperref[auth:a839]{E. Selensky} & Vehicle Routing and Job Shop Scheduling: What's the Difference? & No & \cite{BeckPS03} & 2003 & ICAPS 2003 & 10 & No & \ref{c:BeckPS03}\\
BeckR03 \href{https://doi.org/10.1023/A:1021849405707}{BeckR03} & \hyperref[auth:a89]{J. Christopher Beck}, \hyperref[auth:a256]{P. Refalo} & A Hybrid Approach to Scheduling with Earliness and Tardiness Costs & \href{works/BeckR03.pdf}{Yes} & \cite{BeckR03} & 2003 & Ann. Oper. Res. & 23 & \ref{b:BeckR03} & \ref{c:BeckR03}\\
BeckF00 \href{https://doi.org/10.1016/S0004-3702(99)00099-5}{BeckF00} & \hyperref[auth:a89]{J. Christopher Beck}, \hyperref[auth:a304]{Mark S. Fox} & Dynamic problem structure analysis as a basis for constraint-directed scheduling heuristics & \href{works/BeckF00.pdf}{Yes} & \cite{BeckF00} & 2000 & Artif. Intell. & 51 & \ref{b:BeckF00} & \ref{c:BeckF00}\\
Beck99 \href{https://librarysearch.library.utoronto.ca/permalink/01UTORONTO\_INST/14bjeso/alma991106162342106196}{Beck99} & \hyperref[auth:a89]{J. Christopher Beck} & Texture measurements as a basis for heuristic commitment techniques in constraint-directed scheduling & No & \cite{Beck99} & 1999 & n/a & null & No & \ref{c:Beck99}\\
BeckF98 \href{https://doi.org/10.1609/aimag.v19i4.1426}{BeckF98} & \hyperref[auth:a89]{J. Christopher Beck}, \hyperref[auth:a304]{Mark S. Fox} & A Generic Framework for Constraint-Directed Search and Scheduling & \href{works/BeckF98.pdf}{Yes} & \cite{BeckF98} & 1998 & {AI} Mag. & 30 & \ref{b:BeckF98} & \ref{c:BeckF98}\\
BeckDF97 \href{https://doi.org/10.1007/BFb0017455}{BeckDF97} & \hyperref[auth:a89]{J. Christopher Beck}, \hyperref[auth:a250]{Andrew J. Davenport}, \hyperref[auth:a304]{Mark S. Fox} & Five Pitfalls of Empirical Scheduling Research & \href{works/BeckDF97.pdf}{Yes} & \cite{BeckDF97} & 1997 & CP 1997 & 15 & \ref{b:BeckDF97} & \ref{c:BeckDF97}\\
\end{longtable}
}

\subsection{Works by Michela Milano}
\label{sec:a143}
{\scriptsize
\begin{longtable}{>{\raggedright\arraybackslash}p{3cm}>{\raggedright\arraybackslash}p{6cm}>{\raggedright\arraybackslash}p{7cm}rrrp{3cm}rrr}
\rowcolor{white}\caption{Works from bibtex (Total 24)}\\ \toprule
\rowcolor{white}Key & Authors & Title & LC & Cite & Year & \shortstack{Conference\\/Journal} & Pages & b & c \\ \midrule\endhead
\bottomrule
\endfoot
BorghesiBLMB18 \href{https://doi.org/10.1016/j.suscom.2018.05.007}{BorghesiBLMB18} & \hyperref[auth:a231]{A. Borghesi}, \hyperref[auth:a230]{A. Bartolini}, \hyperref[auth:a142]{M. Lombardi}, \hyperref[auth:a143]{M. Milano}, \hyperref[auth:a247]{L. Benini} & Scheduling-based power capping in high performance computing systems & \href{works/BorghesiBLMB18.pdf}{Yes} & \cite{BorghesiBLMB18} & 2018 & Sustain. Comput. Informatics Syst. & 13 & \ref{b:BorghesiBLMB18} & \ref{c:BorghesiBLMB18}\\
BonfiettiZLM16 \href{https://doi.org/10.1007/978-3-319-44953-1\_8}{BonfiettiZLM16} & \hyperref[auth:a203]{A. Bonfietti}, \hyperref[auth:a204]{A. Zanarini}, \hyperref[auth:a142]{M. Lombardi}, \hyperref[auth:a143]{M. Milano} & The Multirate Resource Constraint & \href{works/BonfiettiZLM16.pdf}{Yes} & \cite{BonfiettiZLM16} & 2016 & CP 2016 & 17 & \ref{b:BonfiettiZLM16} & \ref{c:BonfiettiZLM16}\\
BridiBLMB16 \href{https://doi.org/10.1109/TPDS.2016.2516997}{BridiBLMB16} & \hyperref[auth:a232]{T. Bridi}, \hyperref[auth:a230]{A. Bartolini}, \hyperref[auth:a142]{M. Lombardi}, \hyperref[auth:a143]{M. Milano}, \hyperref[auth:a247]{L. Benini} & A Constraint Programming Scheduler for Heterogeneous High-Performance Computing Machines & \href{works/BridiBLMB16.pdf}{Yes} & \cite{BridiBLMB16} & 2016 & {IEEE} Trans. Parallel Distributed Syst. & 14 & \ref{b:BridiBLMB16} & \ref{c:BridiBLMB16}\\
BridiLBBM16 \href{https://doi.org/10.3233/978-1-61499-672-9-1598}{BridiLBBM16} & \hyperref[auth:a232]{T. Bridi}, \hyperref[auth:a142]{M. Lombardi}, \hyperref[auth:a230]{A. Bartolini}, \hyperref[auth:a247]{L. Benini}, \hyperref[auth:a143]{M. Milano} & {DARDIS:} Distributed And Randomized DIspatching and Scheduling & \href{works/BridiLBBM16.pdf}{Yes} & \cite{BridiLBBM16} & 2016 & ECAI 2016 & 2 & \ref{b:BridiLBBM16} & \ref{c:BridiLBBM16}\\
LombardiBM15 \href{https://doi.org/10.1007/978-3-319-23219-5\_20}{LombardiBM15} & \hyperref[auth:a142]{M. Lombardi}, \hyperref[auth:a203]{A. Bonfietti}, \hyperref[auth:a143]{M. Milano} & Deterministic Estimation of the Expected Makespan of a {POS} Under Duration Uncertainty & \href{works/LombardiBM15.pdf}{Yes} & \cite{LombardiBM15} & 2015 & CP 2015 & 16 & \ref{b:LombardiBM15} & \ref{c:LombardiBM15}\\
BartoliniBBLM14 \href{https://doi.org/10.1007/978-3-319-10428-7\_55}{BartoliniBBLM14} & \hyperref[auth:a230]{A. Bartolini}, \hyperref[auth:a231]{A. Borghesi}, \hyperref[auth:a232]{T. Bridi}, \hyperref[auth:a142]{M. Lombardi}, \hyperref[auth:a143]{M. Milano} & Proactive Workload Dispatching on the {EURORA} Supercomputer & \href{works/BartoliniBBLM14.pdf}{Yes} & \cite{BartoliniBBLM14} & 2014 & CP 2014 & 16 & \ref{b:BartoliniBBLM14} & \ref{c:BartoliniBBLM14}\\
BonfiettiLBM14 \href{https://doi.org/10.1016/j.artint.2013.09.006}{BonfiettiLBM14} & \hyperref[auth:a203]{A. Bonfietti}, \hyperref[auth:a142]{M. Lombardi}, \hyperref[auth:a247]{L. Benini}, \hyperref[auth:a143]{M. Milano} & {CROSS} cyclic resource-constrained scheduling solver & \href{works/BonfiettiLBM14.pdf}{Yes} & \cite{BonfiettiLBM14} & 2014 & Artif. Intell. & 28 & \ref{b:BonfiettiLBM14} & \ref{c:BonfiettiLBM14}\\
BonfiettiLM14 \href{https://doi.org/10.1007/978-3-319-07046-9\_15}{BonfiettiLM14} & \hyperref[auth:a203]{A. Bonfietti}, \hyperref[auth:a142]{M. Lombardi}, \hyperref[auth:a143]{M. Milano} & Disregarding Duration Uncertainty in Partial Order Schedules? Yes, We Can! & \href{works/BonfiettiLM14.pdf}{Yes} & \cite{BonfiettiLM14} & 2014 & CPAIOR 2014 & 16 & \ref{b:BonfiettiLM14} & \ref{c:BonfiettiLM14}\\
BonfiettiLM13 \href{http://www.aaai.org/ocs/index.php/ICAPS/ICAPS13/paper/view/6050}{BonfiettiLM13} & \hyperref[auth:a203]{A. Bonfietti}, \hyperref[auth:a142]{M. Lombardi}, \hyperref[auth:a143]{M. Milano} & De-Cycling Cyclic Scheduling Problems & \href{works/BonfiettiLM13.pdf}{Yes} & \cite{BonfiettiLM13} & 2013 & ICAPS 2013 & 5 & \ref{b:BonfiettiLM13} & \ref{c:BonfiettiLM13}\\
LombardiM13 \href{http://www.aaai.org/ocs/index.php/ICAPS/ICAPS13/paper/view/6052}{LombardiM13} & \hyperref[auth:a142]{M. Lombardi}, \hyperref[auth:a143]{M. Milano} & A Min-Flow Algorithm for Minimal Critical Set Detection in Resource Constrained Project Scheduling & \href{works/LombardiM13.pdf}{Yes} & \cite{LombardiM13} & 2013 & ICAPS 2013 & 2 & \ref{b:LombardiM13} & \ref{c:LombardiM13}\\
BonfiettiLBM12 \href{https://doi.org/10.1007/978-3-642-29828-8\_6}{BonfiettiLBM12} & \hyperref[auth:a203]{A. Bonfietti}, \hyperref[auth:a142]{M. Lombardi}, \hyperref[auth:a247]{L. Benini}, \hyperref[auth:a143]{M. Milano} & Global Cyclic Cumulative Constraint & \href{works/BonfiettiLBM12.pdf}{Yes} & \cite{BonfiettiLBM12} & 2012 & CPAIOR 2012 & 16 & \ref{b:BonfiettiLBM12} & \ref{c:BonfiettiLBM12}\\
BonfiettiM12 \href{https://ceur-ws.org/Vol-926/paper2.pdf}{BonfiettiM12} & \hyperref[auth:a203]{A. Bonfietti}, \hyperref[auth:a143]{M. Milano} & A Constraint-based Approach to Cyclic Resource-Constrained Scheduling Problem & \href{works/BonfiettiM12.pdf}{Yes} & \cite{BonfiettiM12} & 2012 & DC SIAAI 2012 & 3 & \ref{b:BonfiettiM12} & \ref{c:BonfiettiM12}\\
LombardiM12 \href{https://doi.org/10.1007/s10601-011-9115-6}{LombardiM12} & \hyperref[auth:a142]{M. Lombardi}, \hyperref[auth:a143]{M. Milano} & Optimal methods for resource allocation and scheduling: a cross-disciplinary survey & \href{works/LombardiM12.pdf}{Yes} & \cite{LombardiM12} & 2012 & Constraints An Int. J. & 35 & \ref{b:LombardiM12} & \ref{c:LombardiM12}\\
LombardiM12a \href{https://doi.org/10.1016/j.artint.2011.12.001}{LombardiM12a} & \hyperref[auth:a142]{M. Lombardi}, \hyperref[auth:a143]{M. Milano} & A min-flow algorithm for Minimal Critical Set detection in Resource Constrained Project Scheduling & \href{works/LombardiM12a.pdf}{Yes} & \cite{LombardiM12a} & 2012 & Artif. Intell. & 10 & \ref{b:LombardiM12a} & \ref{c:LombardiM12a}\\
BeniniLMR11 \href{https://doi.org/10.1007/s10479-010-0718-x}{BeniniLMR11} & \hyperref[auth:a247]{L. Benini}, \hyperref[auth:a142]{M. Lombardi}, \hyperref[auth:a143]{M. Milano}, \hyperref[auth:a727]{M. Ruggiero} & Optimal resource allocation and scheduling for the {CELL} {BE} platform & \href{works/BeniniLMR11.pdf}{Yes} & \cite{BeniniLMR11} & 2011 & Ann. Oper. Res. & 27 & \ref{b:BeniniLMR11} & \ref{c:BeniniLMR11}\\
BonfiettiLBM11 \href{https://doi.org/10.1007/978-3-642-23786-7\_12}{BonfiettiLBM11} & \hyperref[auth:a203]{A. Bonfietti}, \hyperref[auth:a142]{M. Lombardi}, \hyperref[auth:a247]{L. Benini}, \hyperref[auth:a143]{M. Milano} & A Constraint Based Approach to Cyclic {RCPSP} & \href{works/BonfiettiLBM11.pdf}{Yes} & \cite{BonfiettiLBM11} & 2011 & CP 2011 & 15 & \ref{b:BonfiettiLBM11} & \ref{c:BonfiettiLBM11}\\
LombardiBMB11 \href{https://doi.org/10.1007/978-3-642-21311-3\_14}{LombardiBMB11} & \hyperref[auth:a142]{M. Lombardi}, \hyperref[auth:a203]{A. Bonfietti}, \hyperref[auth:a143]{M. Milano}, \hyperref[auth:a247]{L. Benini} & Precedence Constraint Posting for Cyclic Scheduling Problems & \href{works/LombardiBMB11.pdf}{Yes} & \cite{LombardiBMB11} & 2011 & CPAIOR 2011 & 17 & \ref{b:LombardiBMB11} & \ref{c:LombardiBMB11}\\
LombardiM10 \href{https://doi.org/10.1007/978-3-642-15396-9\_32}{LombardiM10} & \hyperref[auth:a142]{M. Lombardi}, \hyperref[auth:a143]{M. Milano} & Constraint Based Scheduling to Deal with Uncertain Durations and Self-Timed Execution & \href{works/LombardiM10.pdf}{Yes} & \cite{LombardiM10} & 2010 & CP 2010 & 15 & \ref{b:LombardiM10} & \ref{c:LombardiM10}\\
LombardiM10a \href{https://doi.org/10.1016/j.artint.2010.02.004}{LombardiM10a} & \hyperref[auth:a142]{M. Lombardi}, \hyperref[auth:a143]{M. Milano} & Allocation and scheduling of Conditional Task Graphs & \href{works/LombardiM10a.pdf}{Yes} & \cite{LombardiM10a} & 2010 & Artif. Intell. & 30 & \ref{b:LombardiM10a} & \ref{c:LombardiM10a}\\
LombardiM09 \href{https://doi.org/10.1007/978-3-642-04244-7\_45}{LombardiM09} & \hyperref[auth:a142]{M. Lombardi}, \hyperref[auth:a143]{M. Milano} & A Precedence Constraint Posting Approach for the {RCPSP} with Time Lags and Variable Durations & \href{works/LombardiM09.pdf}{Yes} & \cite{LombardiM09} & 2009 & CP 2009 & 15 & \ref{b:LombardiM09} & \ref{c:LombardiM09}\\
RuggieroBBMA09 \href{https://doi.org/10.1109/TCAD.2009.2013536}{RuggieroBBMA09} & \hyperref[auth:a727]{M. Ruggiero}, \hyperref[auth:a379]{D. Bertozzi}, \hyperref[auth:a247]{L. Benini}, \hyperref[auth:a143]{M. Milano}, \hyperref[auth:a728]{A. Andrei} & Reducing the Abstraction and Optimality Gaps in the Allocation and Scheduling for Variable Voltage/Frequency MPSoC Platforms & \href{works/RuggieroBBMA09.pdf}{Yes} & \cite{RuggieroBBMA09} & 2009 & {IEEE} Trans. Comput. Aided Des. Integr. Circuits Syst. & 14 & \ref{b:RuggieroBBMA09} & \ref{c:RuggieroBBMA09}\\
BeniniBGM06 \href{https://doi.org/10.1007/11757375\_6}{BeniniBGM06} & \hyperref[auth:a247]{L. Benini}, \hyperref[auth:a379]{D. Bertozzi}, \hyperref[auth:a380]{A. Guerri}, \hyperref[auth:a143]{M. Milano} & Allocation, Scheduling and Voltage Scaling on Energy Aware MPSoCs & \href{works/BeniniBGM06.pdf}{Yes} & \cite{BeniniBGM06} & 2006 & CPAIOR 2006 & 15 & \ref{b:BeniniBGM06} & \ref{c:BeniniBGM06}\\
LammaMM97 \href{https://doi.org/10.1016/S0954-1810(96)00002-7}{LammaMM97} & \hyperref[auth:a729]{E. Lamma}, \hyperref[auth:a730]{P. Mello}, \hyperref[auth:a143]{M. Milano} & A distributed constraint-based scheduler & \href{works/LammaMM97.pdf}{Yes} & \cite{LammaMM97} & 1997 & Artif. Intell. Eng. & 15 & \ref{b:LammaMM97} & \ref{c:LammaMM97}\\
BrusoniCLMMT96 \href{https://doi.org/10.1007/3-540-61286-6\_157}{BrusoniCLMMT96} & \hyperref[auth:a731]{V. Brusoni}, \hyperref[auth:a732]{L. Console}, \hyperref[auth:a729]{E. Lamma}, \hyperref[auth:a730]{P. Mello}, \hyperref[auth:a143]{M. Milano}, \hyperref[auth:a733]{P. Terenziani} & Resource-Based vs. Task-Based Approaches for Scheduling Problems & \href{works/BrusoniCLMMT96.pdf}{Yes} & \cite{BrusoniCLMMT96} & 1996 & ISMIS 1996 & 10 & \ref{b:BrusoniCLMMT96} & \ref{c:BrusoniCLMMT96}\\
\end{longtable}
}

\subsection{Works by Andreas Schutt}
\label{sec:a124}
{\scriptsize
\begin{longtable}{>{\raggedright\arraybackslash}p{3cm}>{\raggedright\arraybackslash}p{6cm}>{\raggedright\arraybackslash}p{7cm}rrrp{3cm}rrr}
\rowcolor{white}\caption{Works from bibtex (Total 24)}\\ \toprule
\rowcolor{white}Key & Authors & Title & LC & Cite & Year & \shortstack{Conference\\/Journal} & Pages & b & c \\ \midrule\endhead
\bottomrule
\endfoot
YangSS19 \href{https://doi.org/10.1007/978-3-030-19212-9\_42}{YangSS19} & \hyperref[auth:a311]{M. Yang}, \hyperref[auth:a124]{A. Schutt}, \hyperref[auth:a125]{Peter J. Stuckey} & Time Table Edge Finding with Energy Variables & \href{works/YangSS19.pdf}{Yes} & \cite{YangSS19} & 2019 & CPAIOR 2019 & 10 & \ref{b:YangSS19} & \ref{c:YangSS19}\\
GoldwaserS18 \href{https://doi.org/10.1613/jair.1.11268}{GoldwaserS18} & \hyperref[auth:a194]{A. Goldwaser}, \hyperref[auth:a124]{A. Schutt} & Optimal Torpedo Scheduling & \href{works/GoldwaserS18.pdf}{Yes} & \cite{GoldwaserS18} & 2018 & J. Artif. Intell. Res. & 32 & \ref{b:GoldwaserS18} & \ref{c:GoldwaserS18}\\
KreterSSZ18 \href{https://doi.org/10.1016/j.ejor.2017.10.014}{KreterSSZ18} & \hyperref[auth:a123]{S. Kreter}, \hyperref[auth:a124]{A. Schutt}, \hyperref[auth:a125]{Peter J. Stuckey}, \hyperref[auth:a803]{J. Zimmermann} & Mixed-integer linear programming and constraint programming formulations for solving resource availability cost problems & No & \cite{KreterSSZ18} & 2018 & Eur. J. Oper. Res. & 15 & No & \ref{c:KreterSSZ18}\\
MusliuSS18 \href{https://doi.org/10.1007/978-3-319-93031-2\_31}{MusliuSS18} & \hyperref[auth:a45]{N. Musliu}, \hyperref[auth:a124]{A. Schutt}, \hyperref[auth:a125]{Peter J. Stuckey} & Solver Independent Rotating Workforce Scheduling & \href{works/MusliuSS18.pdf}{Yes} & \cite{MusliuSS18} & 2018 & CPAIOR 2018 & 17 & \ref{b:MusliuSS18} & \ref{c:MusliuSS18}\\
GoldwaserS17 \href{https://doi.org/10.1007/978-3-319-66158-2\_22}{GoldwaserS17} & \hyperref[auth:a194]{A. Goldwaser}, \hyperref[auth:a124]{A. Schutt} & Optimal Torpedo Scheduling & \href{works/GoldwaserS17.pdf}{Yes} & \cite{GoldwaserS17} & 2017 & CP 2017 & 16 & \ref{b:GoldwaserS17} & \ref{c:GoldwaserS17}\\
KreterSS17 \href{https://doi.org/10.1007/s10601-016-9266-6}{KreterSS17} & \hyperref[auth:a123]{S. Kreter}, \hyperref[auth:a124]{A. Schutt}, \hyperref[auth:a125]{Peter J. Stuckey} & Using constraint programming for solving RCPSP/max-cal & \href{works/KreterSS17.pdf}{Yes} & \cite{KreterSS17} & 2017 & Constraints An Int. J. & 31 & \ref{b:KreterSS17} & \ref{c:KreterSS17}\\
YoungFS17 \href{https://doi.org/10.1007/978-3-319-66158-2\_20}{YoungFS17} & \hyperref[auth:a193]{Kenneth D. Young}, \hyperref[auth:a154]{T. Feydy}, \hyperref[auth:a124]{A. Schutt} & Constraint Programming Applied to the Multi-Skill Project Scheduling Problem & \href{works/YoungFS17.pdf}{Yes} & \cite{YoungFS17} & 2017 & CP 2017 & 10 & \ref{b:YoungFS17} & \ref{c:YoungFS17}\\
SchuttS16 \href{https://doi.org/10.1007/978-3-319-44953-1\_28}{SchuttS16} & \hyperref[auth:a124]{A. Schutt}, \hyperref[auth:a125]{Peter J. Stuckey} & Explaining Producer/Consumer Constraints & \href{works/SchuttS16.pdf}{Yes} & \cite{SchuttS16} & 2016 & CP 2016 & 17 & \ref{b:SchuttS16} & \ref{c:SchuttS16}\\
SzerediS16 \href{https://doi.org/10.1007/978-3-319-44953-1\_31}{SzerediS16} & \hyperref[auth:a205]{R. Szeredi}, \hyperref[auth:a124]{A. Schutt} & Modelling and Solving Multi-mode Resource-Constrained Project Scheduling & \href{works/SzerediS16.pdf}{Yes} & \cite{SzerediS16} & 2016 & CP 2016 & 10 & \ref{b:SzerediS16} & \ref{c:SzerediS16}\\
EvenSH15 \href{https://doi.org/10.1007/978-3-319-23219-5\_40}{EvenSH15} & \hyperref[auth:a219]{C. Even}, \hyperref[auth:a124]{A. Schutt}, \hyperref[auth:a148]{Pascal Van Hentenryck} & A Constraint Programming Approach for Non-preemptive Evacuation Scheduling & \href{works/EvenSH15.pdf}{Yes} & \cite{EvenSH15} & 2015 & CP 2015 & 18 & \ref{b:EvenSH15} & \ref{c:EvenSH15}\\
EvenSH15a \href{http://arxiv.org/abs/1505.02487}{EvenSH15a} & \hyperref[auth:a219]{C. Even}, \hyperref[auth:a124]{A. Schutt}, \hyperref[auth:a148]{Pascal Van Hentenryck} & A Constraint Programming Approach for Non-Preemptive Evacuation Scheduling & \href{works/EvenSH15a.pdf}{Yes} & \cite{EvenSH15a} & 2015 & CoRR & 16 & \ref{b:EvenSH15a} & \ref{c:EvenSH15a}\\
KreterSS15 \href{https://doi.org/10.1007/978-3-319-23219-5\_19}{KreterSS15} & \hyperref[auth:a123]{S. Kreter}, \hyperref[auth:a124]{A. Schutt}, \hyperref[auth:a125]{Peter J. Stuckey} & Modeling and Solving Project Scheduling with Calendars & \href{works/KreterSS15.pdf}{Yes} & \cite{KreterSS15} & 2015 & CP 2015 & 17 & \ref{b:KreterSS15} & \ref{c:KreterSS15}\\
ThiruvadyWGS14 \href{https://doi.org/10.1007/s10732-014-9260-3}{ThiruvadyWGS14} & \hyperref[auth:a400]{Dhananjay R. Thiruvady}, \hyperref[auth:a117]{M. Wallace}, \hyperref[auth:a341]{H. Gu}, \hyperref[auth:a124]{A. Schutt} & A Lagrangian relaxation and {ACO} hybrid for resource constrained project scheduling with discounted cash flows & \href{works/ThiruvadyWGS14.pdf}{Yes} & \cite{ThiruvadyWGS14} & 2014 & J. Heuristics & 34 & \ref{b:ThiruvadyWGS14} & \ref{c:ThiruvadyWGS14}\\
ChuGNSW13 \href{http://www.aaai.org/ocs/index.php/IJCAI/IJCAI13/paper/view/6878}{ChuGNSW13} & \hyperref[auth:a348]{G. Chu}, \hyperref[auth:a804]{S. Gaspers}, \hyperref[auth:a805]{N. Narodytska}, \hyperref[auth:a124]{A. Schutt}, \hyperref[auth:a278]{T. Walsh} & On the Complexity of Global Scheduling Constraints under Structural Restrictions & \href{works/ChuGNSW13.pdf}{Yes} & \cite{ChuGNSW13} & 2013 & IJCAI 2013 & 7 & \ref{b:ChuGNSW13} & \ref{c:ChuGNSW13}\\
GuSS13 \href{https://doi.org/10.1007/978-3-642-38171-3\_24}{GuSS13} & \hyperref[auth:a341]{H. Gu}, \hyperref[auth:a124]{A. Schutt}, \hyperref[auth:a125]{Peter J. Stuckey} & A Lagrangian Relaxation Based Forward-Backward Improvement Heuristic for Maximising the Net Present Value of Resource-Constrained Projects & \href{works/GuSS13.pdf}{Yes} & \cite{GuSS13} & 2013 & CPAIOR 2013 & 7 & \ref{b:GuSS13} & \ref{c:GuSS13}\\
SchuttFS13 \href{https://doi.org/10.1007/978-3-642-40627-0\_47}{SchuttFS13} & \hyperref[auth:a124]{A. Schutt}, \hyperref[auth:a154]{T. Feydy}, \hyperref[auth:a125]{Peter J. Stuckey} & Scheduling Optional Tasks with Explanation & \href{works/SchuttFS13.pdf}{Yes} & \cite{SchuttFS13} & 2013 & CP 2013 & 17 & \ref{b:SchuttFS13} & \ref{c:SchuttFS13}\\
SchuttFS13a \href{https://doi.org/10.1007/978-3-642-38171-3\_16}{SchuttFS13a} & \hyperref[auth:a124]{A. Schutt}, \hyperref[auth:a154]{T. Feydy}, \hyperref[auth:a125]{Peter J. Stuckey} & Explaining Time-Table-Edge-Finding Propagation for the Cumulative Resource Constraint & \href{works/SchuttFS13a.pdf}{Yes} & \cite{SchuttFS13a} & 2013 & CPAIOR 2013 & 17 & \ref{b:SchuttFS13a} & \ref{c:SchuttFS13a}\\
SchuttFSW13 \href{https://doi.org/10.1007/s10951-012-0285-x}{SchuttFSW13} & \hyperref[auth:a124]{A. Schutt}, \hyperref[auth:a154]{T. Feydy}, \hyperref[auth:a125]{Peter J. Stuckey}, \hyperref[auth:a155]{Mark G. Wallace} & Solving RCPSP/max by lazy clause generation & \href{works/SchuttFSW13.pdf}{Yes} & \cite{SchuttFSW13} & 2013 & J. Sched. & 17 & \ref{b:SchuttFSW13} & \ref{c:SchuttFSW13}\\
SchuttCSW12 \href{https://doi.org/10.1007/978-3-642-29828-8\_24}{SchuttCSW12} & \hyperref[auth:a124]{A. Schutt}, \hyperref[auth:a348]{G. Chu}, \hyperref[auth:a125]{Peter J. Stuckey}, \hyperref[auth:a155]{Mark G. Wallace} & Maximising the Net Present Value for Resource-Constrained Project Scheduling & \href{works/SchuttCSW12.pdf}{Yes} & \cite{SchuttCSW12} & 2012 & CPAIOR 2012 & 17 & \ref{b:SchuttCSW12} & \ref{c:SchuttCSW12}\\
SchuttFSW11 \href{https://doi.org/10.1007/s10601-010-9103-2}{SchuttFSW11} & \hyperref[auth:a124]{A. Schutt}, \hyperref[auth:a154]{T. Feydy}, \hyperref[auth:a125]{Peter J. Stuckey}, \hyperref[auth:a155]{Mark G. Wallace} & Explaining the cumulative propagator & \href{works/SchuttFSW11.pdf}{Yes} & \cite{SchuttFSW11} & 2011 & Constraints An Int. J. & 33 & \ref{b:SchuttFSW11} & \ref{c:SchuttFSW11}\\
SchuttW10 \href{https://doi.org/10.1007/978-3-642-15396-9\_36}{SchuttW10} & \hyperref[auth:a124]{A. Schutt}, \hyperref[auth:a51]{A. Wolf} & A New \emph{O}(\emph{n}\({}^{\mbox{2}}\)log\emph{n}) Not-First/Not-Last Pruning Algorithm for Cumulative Resource Constraints & \href{works/SchuttW10.pdf}{Yes} & \cite{SchuttW10} & 2010 & CP 2010 & 15 & \ref{b:SchuttW10} & \ref{c:SchuttW10}\\
abs-1009-0347 \href{http://arxiv.org/abs/1009.0347}{abs-1009-0347} & \hyperref[auth:a124]{A. Schutt}, \hyperref[auth:a154]{T. Feydy}, \hyperref[auth:a125]{Peter J. Stuckey}, \hyperref[auth:a155]{Mark G. Wallace} & Solving the Resource Constrained Project Scheduling Problem with Generalized Precedences by Lazy Clause Generation & \href{works/abs-1009-0347.pdf}{Yes} & \cite{abs-1009-0347} & 2010 & CoRR & 37 & \ref{b:abs-1009-0347} & \ref{c:abs-1009-0347}\\
SchuttFSW09 \href{https://doi.org/10.1007/978-3-642-04244-7\_58}{SchuttFSW09} & \hyperref[auth:a124]{A. Schutt}, \hyperref[auth:a154]{T. Feydy}, \hyperref[auth:a125]{Peter J. Stuckey}, \hyperref[auth:a117]{M. Wallace} & Why Cumulative Decomposition Is Not as Bad as It Sounds & \href{works/SchuttFSW09.pdf}{Yes} & \cite{SchuttFSW09} & 2009 & CP 2009 & 16 & \ref{b:SchuttFSW09} & \ref{c:SchuttFSW09}\\
SchuttWS05 \href{https://doi.org/10.1007/11963578\_6}{SchuttWS05} & \hyperref[auth:a124]{A. Schutt}, \hyperref[auth:a51]{A. Wolf}, \hyperref[auth:a720]{G. Schrader} & Not-First and Not-Last Detection for Cumulative Scheduling in \emph{O}(\emph{n}\({}^{\mbox{3}}\)log\emph{n}) & \href{works/SchuttWS05.pdf}{Yes} & \cite{SchuttWS05} & 2005 & INAP 2005 & 15 & \ref{b:SchuttWS05} & \ref{c:SchuttWS05}\\
\end{longtable}
}

\subsection{Works by Peter J. Stuckey}
\label{sec:a125}
{\scriptsize
\begin{longtable}{>{\raggedright\arraybackslash}p{3cm}>{\raggedright\arraybackslash}p{6cm}>{\raggedright\arraybackslash}p{7cm}rrrp{3cm}rrr}
\rowcolor{white}\caption{Works from bibtex (Total 21)}\\ \toprule
\rowcolor{white}Key & Authors & Title & LC & Cite & Year & \shortstack{Conference\\/Journal} & Pages & b & c \\ \midrule\endhead
\bottomrule
\endfoot
YangSS19 \href{https://doi.org/10.1007/978-3-030-19212-9\_42}{YangSS19} & \hyperref[auth:a311]{M. Yang}, \hyperref[auth:a124]{A. Schutt}, \hyperref[auth:a125]{Peter J. Stuckey} & Time Table Edge Finding with Energy Variables & \href{works/YangSS19.pdf}{Yes} & \cite{YangSS19} & 2019 & CPAIOR 2019 & 10 & \ref{b:YangSS19} & \ref{c:YangSS19}\\
DemirovicS18 \href{https://doi.org/10.1007/978-3-319-93031-2\_10}{DemirovicS18} & \hyperref[auth:a314]{E. Demirovic}, \hyperref[auth:a125]{Peter J. Stuckey} & Constraint Programming for High School Timetabling: {A} Scheduling-Based Model with Hot Starts & \href{works/DemirovicS18.pdf}{Yes} & \cite{DemirovicS18} & 2018 & CPAIOR 2018 & 18 & \ref{b:DemirovicS18} & \ref{c:DemirovicS18}\\
KreterSSZ18 \href{https://doi.org/10.1016/j.ejor.2017.10.014}{KreterSSZ18} & \hyperref[auth:a123]{S. Kreter}, \hyperref[auth:a124]{A. Schutt}, \hyperref[auth:a125]{Peter J. Stuckey}, \hyperref[auth:a803]{J. Zimmermann} & Mixed-integer linear programming and constraint programming formulations for solving resource availability cost problems & No & \cite{KreterSSZ18} & 2018 & Eur. J. Oper. Res. & 15 & No & \ref{c:KreterSSZ18}\\
MusliuSS18 \href{https://doi.org/10.1007/978-3-319-93031-2\_31}{MusliuSS18} & \hyperref[auth:a45]{N. Musliu}, \hyperref[auth:a124]{A. Schutt}, \hyperref[auth:a125]{Peter J. Stuckey} & Solver Independent Rotating Workforce Scheduling & \href{works/MusliuSS18.pdf}{Yes} & \cite{MusliuSS18} & 2018 & CPAIOR 2018 & 17 & \ref{b:MusliuSS18} & \ref{c:MusliuSS18}\\
KreterSS17 \href{https://doi.org/10.1007/s10601-016-9266-6}{KreterSS17} & \hyperref[auth:a123]{S. Kreter}, \hyperref[auth:a124]{A. Schutt}, \hyperref[auth:a125]{Peter J. Stuckey} & Using constraint programming for solving RCPSP/max-cal & \href{works/KreterSS17.pdf}{Yes} & \cite{KreterSS17} & 2017 & Constraints An Int. J. & 31 & \ref{b:KreterSS17} & \ref{c:KreterSS17}\\
BlomPS16 \href{https://doi.org/10.1287/mnsc.2015.2284}{BlomPS16} & \hyperref[auth:a806]{Michelle L. Blom}, \hyperref[auth:a327]{Adrian R. Pearce}, \hyperref[auth:a125]{Peter J. Stuckey} & A Decomposition-Based Algorithm for the Scheduling of Open-Pit Networks Over Multiple Time Periods & No & \cite{BlomPS16} & 2016 & Manag. Sci. & 26 & No & \ref{c:BlomPS16}\\
SchuttS16 \href{https://doi.org/10.1007/978-3-319-44953-1\_28}{SchuttS16} & \hyperref[auth:a124]{A. Schutt}, \hyperref[auth:a125]{Peter J. Stuckey} & Explaining Producer/Consumer Constraints & \href{works/SchuttS16.pdf}{Yes} & \cite{SchuttS16} & 2016 & CP 2016 & 17 & \ref{b:SchuttS16} & \ref{c:SchuttS16}\\
BurtLPS15 \href{https://doi.org/10.1007/978-3-319-18008-3\_7}{BurtLPS15} & \hyperref[auth:a325]{Christina N. Burt}, \hyperref[auth:a326]{N. Lipovetzky}, \hyperref[auth:a327]{Adrian R. Pearce}, \hyperref[auth:a125]{Peter J. Stuckey} & Scheduling with Fixed Maintenance, Shared Resources and Nonlinear Feedrate Constraints: {A} Mine Planning Case Study & \href{works/BurtLPS15.pdf}{Yes} & \cite{BurtLPS15} & 2015 & CPAIOR 2015 & 17 & \ref{b:BurtLPS15} & \ref{c:BurtLPS15}\\
KreterSS15 \href{https://doi.org/10.1007/978-3-319-23219-5\_19}{KreterSS15} & \hyperref[auth:a123]{S. Kreter}, \hyperref[auth:a124]{A. Schutt}, \hyperref[auth:a125]{Peter J. Stuckey} & Modeling and Solving Project Scheduling with Calendars & \href{works/KreterSS15.pdf}{Yes} & \cite{KreterSS15} & 2015 & CP 2015 & 17 & \ref{b:KreterSS15} & \ref{c:KreterSS15}\\
BlomBPS14 \href{https://doi.org/10.1287/ijoc.2013.0590}{BlomBPS14} & \hyperref[auth:a806]{Michelle L. Blom}, \hyperref[auth:a325]{Christina N. Burt}, \hyperref[auth:a327]{Adrian R. Pearce}, \hyperref[auth:a125]{Peter J. Stuckey} & A Decomposition-Based Heuristic for Collaborative Scheduling in a Network of Open-Pit Mines & No & \cite{BlomBPS14} & 2014 & {INFORMS} J. Comput. & 19 & No & \ref{c:BlomBPS14}\\
LipovetzkyBPS14 \href{http://www.aaai.org/ocs/index.php/ICAPS/ICAPS14/paper/view/7942}{LipovetzkyBPS14} & \hyperref[auth:a326]{N. Lipovetzky}, \hyperref[auth:a325]{Christina N. Burt}, \hyperref[auth:a327]{Adrian R. Pearce}, \hyperref[auth:a125]{Peter J. Stuckey} & Planning for Mining Operations with Time and Resource Constraints & \href{works/LipovetzkyBPS14.pdf}{Yes} & \cite{LipovetzkyBPS14} & 2014 & ICAPS 2014 & 9 & \ref{b:LipovetzkyBPS14} & \ref{c:LipovetzkyBPS14}\\
GuSS13 \href{https://doi.org/10.1007/978-3-642-38171-3\_24}{GuSS13} & \hyperref[auth:a341]{H. Gu}, \hyperref[auth:a124]{A. Schutt}, \hyperref[auth:a125]{Peter J. Stuckey} & A Lagrangian Relaxation Based Forward-Backward Improvement Heuristic for Maximising the Net Present Value of Resource-Constrained Projects & \href{works/GuSS13.pdf}{Yes} & \cite{GuSS13} & 2013 & CPAIOR 2013 & 7 & \ref{b:GuSS13} & \ref{c:GuSS13}\\
SchuttFS13 \href{https://doi.org/10.1007/978-3-642-40627-0\_47}{SchuttFS13} & \hyperref[auth:a124]{A. Schutt}, \hyperref[auth:a154]{T. Feydy}, \hyperref[auth:a125]{Peter J. Stuckey} & Scheduling Optional Tasks with Explanation & \href{works/SchuttFS13.pdf}{Yes} & \cite{SchuttFS13} & 2013 & CP 2013 & 17 & \ref{b:SchuttFS13} & \ref{c:SchuttFS13}\\
SchuttFS13a \href{https://doi.org/10.1007/978-3-642-38171-3\_16}{SchuttFS13a} & \hyperref[auth:a124]{A. Schutt}, \hyperref[auth:a154]{T. Feydy}, \hyperref[auth:a125]{Peter J. Stuckey} & Explaining Time-Table-Edge-Finding Propagation for the Cumulative Resource Constraint & \href{works/SchuttFS13a.pdf}{Yes} & \cite{SchuttFS13a} & 2013 & CPAIOR 2013 & 17 & \ref{b:SchuttFS13a} & \ref{c:SchuttFS13a}\\
SchuttFSW13 \href{https://doi.org/10.1007/s10951-012-0285-x}{SchuttFSW13} & \hyperref[auth:a124]{A. Schutt}, \hyperref[auth:a154]{T. Feydy}, \hyperref[auth:a125]{Peter J. Stuckey}, \hyperref[auth:a155]{Mark G. Wallace} & Solving RCPSP/max by lazy clause generation & \href{works/SchuttFSW13.pdf}{Yes} & \cite{SchuttFSW13} & 2013 & J. Sched. & 17 & \ref{b:SchuttFSW13} & \ref{c:SchuttFSW13}\\
GuSW12 \href{https://doi.org/10.1007/978-3-642-33558-7\_55}{GuSW12} & \hyperref[auth:a341]{H. Gu}, \hyperref[auth:a125]{Peter J. Stuckey}, \hyperref[auth:a155]{Mark G. Wallace} & Maximising the Net Present Value of Large Resource-Constrained Projects & \href{works/GuSW12.pdf}{Yes} & \cite{GuSW12} & 2012 & CP 2012 & 15 & \ref{b:GuSW12} & \ref{c:GuSW12}\\
SchuttCSW12 \href{https://doi.org/10.1007/978-3-642-29828-8\_24}{SchuttCSW12} & \hyperref[auth:a124]{A. Schutt}, \hyperref[auth:a348]{G. Chu}, \hyperref[auth:a125]{Peter J. Stuckey}, \hyperref[auth:a155]{Mark G. Wallace} & Maximising the Net Present Value for Resource-Constrained Project Scheduling & \href{works/SchuttCSW12.pdf}{Yes} & \cite{SchuttCSW12} & 2012 & CPAIOR 2012 & 17 & \ref{b:SchuttCSW12} & \ref{c:SchuttCSW12}\\
BandaSC11 \href{https://doi.org/10.1287/ijoc.1090.0378}{BandaSC11} & \hyperref[auth:a807]{Maria Garcia de la Banda}, \hyperref[auth:a125]{Peter J. Stuckey}, \hyperref[auth:a348]{G. Chu} & Solving Talent Scheduling with Dynamic Programming & No & \cite{BandaSC11} & 2011 & {INFORMS} J. Comput. & 18 & No & \ref{c:BandaSC11}\\
SchuttFSW11 \href{https://doi.org/10.1007/s10601-010-9103-2}{SchuttFSW11} & \hyperref[auth:a124]{A. Schutt}, \hyperref[auth:a154]{T. Feydy}, \hyperref[auth:a125]{Peter J. Stuckey}, \hyperref[auth:a155]{Mark G. Wallace} & Explaining the cumulative propagator & \href{works/SchuttFSW11.pdf}{Yes} & \cite{SchuttFSW11} & 2011 & Constraints An Int. J. & 33 & \ref{b:SchuttFSW11} & \ref{c:SchuttFSW11}\\
abs-1009-0347 \href{http://arxiv.org/abs/1009.0347}{abs-1009-0347} & \hyperref[auth:a124]{A. Schutt}, \hyperref[auth:a154]{T. Feydy}, \hyperref[auth:a125]{Peter J. Stuckey}, \hyperref[auth:a155]{Mark G. Wallace} & Solving the Resource Constrained Project Scheduling Problem with Generalized Precedences by Lazy Clause Generation & \href{works/abs-1009-0347.pdf}{Yes} & \cite{abs-1009-0347} & 2010 & CoRR & 37 & \ref{b:abs-1009-0347} & \ref{c:abs-1009-0347}\\
SchuttFSW09 \href{https://doi.org/10.1007/978-3-642-04244-7\_58}{SchuttFSW09} & \hyperref[auth:a124]{A. Schutt}, \hyperref[auth:a154]{T. Feydy}, \hyperref[auth:a125]{Peter J. Stuckey}, \hyperref[auth:a117]{M. Wallace} & Why Cumulative Decomposition Is Not as Bad as It Sounds & \href{works/SchuttFSW09.pdf}{Yes} & \cite{SchuttFSW09} & 2009 & CP 2009 & 16 & \ref{b:SchuttFSW09} & \ref{c:SchuttFSW09}\\
\end{longtable}
}

\subsection{Works by Michele Lombardi}
\label{sec:a142}
{\scriptsize
\begin{longtable}{>{\raggedright\arraybackslash}p{3cm}>{\raggedright\arraybackslash}p{6cm}>{\raggedright\arraybackslash}p{7cm}rrrp{3cm}rrr}
\rowcolor{white}\caption{Works from bibtex (Total 20)}\\ \toprule
\rowcolor{white}Key & Authors & Title & LC & Cite & Year & \shortstack{Conference\\/Journal} & Pages & b & c \\ \midrule\endhead
\bottomrule
\endfoot
BorghesiBLMB18 \href{https://doi.org/10.1016/j.suscom.2018.05.007}{BorghesiBLMB18} & \hyperref[auth:a231]{A. Borghesi}, \hyperref[auth:a230]{A. Bartolini}, \hyperref[auth:a142]{M. Lombardi}, \hyperref[auth:a143]{M. Milano}, \hyperref[auth:a247]{L. Benini} & Scheduling-based power capping in high performance computing systems & \href{works/BorghesiBLMB18.pdf}{Yes} & \cite{BorghesiBLMB18} & 2018 & Sustain. Comput. Informatics Syst. & 13 & \ref{b:BorghesiBLMB18} & \ref{c:BorghesiBLMB18}\\
BonfiettiZLM16 \href{https://doi.org/10.1007/978-3-319-44953-1\_8}{BonfiettiZLM16} & \hyperref[auth:a203]{A. Bonfietti}, \hyperref[auth:a204]{A. Zanarini}, \hyperref[auth:a142]{M. Lombardi}, \hyperref[auth:a143]{M. Milano} & The Multirate Resource Constraint & \href{works/BonfiettiZLM16.pdf}{Yes} & \cite{BonfiettiZLM16} & 2016 & CP 2016 & 17 & \ref{b:BonfiettiZLM16} & \ref{c:BonfiettiZLM16}\\
BridiBLMB16 \href{https://doi.org/10.1109/TPDS.2016.2516997}{BridiBLMB16} & \hyperref[auth:a232]{T. Bridi}, \hyperref[auth:a230]{A. Bartolini}, \hyperref[auth:a142]{M. Lombardi}, \hyperref[auth:a143]{M. Milano}, \hyperref[auth:a247]{L. Benini} & A Constraint Programming Scheduler for Heterogeneous High-Performance Computing Machines & \href{works/BridiBLMB16.pdf}{Yes} & \cite{BridiBLMB16} & 2016 & {IEEE} Trans. Parallel Distributed Syst. & 14 & \ref{b:BridiBLMB16} & \ref{c:BridiBLMB16}\\
BridiLBBM16 \href{https://doi.org/10.3233/978-1-61499-672-9-1598}{BridiLBBM16} & \hyperref[auth:a232]{T. Bridi}, \hyperref[auth:a142]{M. Lombardi}, \hyperref[auth:a230]{A. Bartolini}, \hyperref[auth:a247]{L. Benini}, \hyperref[auth:a143]{M. Milano} & {DARDIS:} Distributed And Randomized DIspatching and Scheduling & \href{works/BridiLBBM16.pdf}{Yes} & \cite{BridiLBBM16} & 2016 & ECAI 2016 & 2 & \ref{b:BridiLBBM16} & \ref{c:BridiLBBM16}\\
LombardiBM15 \href{https://doi.org/10.1007/978-3-319-23219-5\_20}{LombardiBM15} & \hyperref[auth:a142]{M. Lombardi}, \hyperref[auth:a203]{A. Bonfietti}, \hyperref[auth:a143]{M. Milano} & Deterministic Estimation of the Expected Makespan of a {POS} Under Duration Uncertainty & \href{works/LombardiBM15.pdf}{Yes} & \cite{LombardiBM15} & 2015 & CP 2015 & 16 & \ref{b:LombardiBM15} & \ref{c:LombardiBM15}\\
BartoliniBBLM14 \href{https://doi.org/10.1007/978-3-319-10428-7\_55}{BartoliniBBLM14} & \hyperref[auth:a230]{A. Bartolini}, \hyperref[auth:a231]{A. Borghesi}, \hyperref[auth:a232]{T. Bridi}, \hyperref[auth:a142]{M. Lombardi}, \hyperref[auth:a143]{M. Milano} & Proactive Workload Dispatching on the {EURORA} Supercomputer & \href{works/BartoliniBBLM14.pdf}{Yes} & \cite{BartoliniBBLM14} & 2014 & CP 2014 & 16 & \ref{b:BartoliniBBLM14} & \ref{c:BartoliniBBLM14}\\
BonfiettiLBM14 \href{https://doi.org/10.1016/j.artint.2013.09.006}{BonfiettiLBM14} & \hyperref[auth:a203]{A. Bonfietti}, \hyperref[auth:a142]{M. Lombardi}, \hyperref[auth:a247]{L. Benini}, \hyperref[auth:a143]{M. Milano} & {CROSS} cyclic resource-constrained scheduling solver & \href{works/BonfiettiLBM14.pdf}{Yes} & \cite{BonfiettiLBM14} & 2014 & Artif. Intell. & 28 & \ref{b:BonfiettiLBM14} & \ref{c:BonfiettiLBM14}\\
BonfiettiLM14 \href{https://doi.org/10.1007/978-3-319-07046-9\_15}{BonfiettiLM14} & \hyperref[auth:a203]{A. Bonfietti}, \hyperref[auth:a142]{M. Lombardi}, \hyperref[auth:a143]{M. Milano} & Disregarding Duration Uncertainty in Partial Order Schedules? Yes, We Can! & \href{works/BonfiettiLM14.pdf}{Yes} & \cite{BonfiettiLM14} & 2014 & CPAIOR 2014 & 16 & \ref{b:BonfiettiLM14} & \ref{c:BonfiettiLM14}\\
BonfiettiLM13 \href{http://www.aaai.org/ocs/index.php/ICAPS/ICAPS13/paper/view/6050}{BonfiettiLM13} & \hyperref[auth:a203]{A. Bonfietti}, \hyperref[auth:a142]{M. Lombardi}, \hyperref[auth:a143]{M. Milano} & De-Cycling Cyclic Scheduling Problems & \href{works/BonfiettiLM13.pdf}{Yes} & \cite{BonfiettiLM13} & 2013 & ICAPS 2013 & 5 & \ref{b:BonfiettiLM13} & \ref{c:BonfiettiLM13}\\
LombardiM13 \href{http://www.aaai.org/ocs/index.php/ICAPS/ICAPS13/paper/view/6052}{LombardiM13} & \hyperref[auth:a142]{M. Lombardi}, \hyperref[auth:a143]{M. Milano} & A Min-Flow Algorithm for Minimal Critical Set Detection in Resource Constrained Project Scheduling & \href{works/LombardiM13.pdf}{Yes} & \cite{LombardiM13} & 2013 & ICAPS 2013 & 2 & \ref{b:LombardiM13} & \ref{c:LombardiM13}\\
BonfiettiLBM12 \href{https://doi.org/10.1007/978-3-642-29828-8\_6}{BonfiettiLBM12} & \hyperref[auth:a203]{A. Bonfietti}, \hyperref[auth:a142]{M. Lombardi}, \hyperref[auth:a247]{L. Benini}, \hyperref[auth:a143]{M. Milano} & Global Cyclic Cumulative Constraint & \href{works/BonfiettiLBM12.pdf}{Yes} & \cite{BonfiettiLBM12} & 2012 & CPAIOR 2012 & 16 & \ref{b:BonfiettiLBM12} & \ref{c:BonfiettiLBM12}\\
LombardiM12 \href{https://doi.org/10.1007/s10601-011-9115-6}{LombardiM12} & \hyperref[auth:a142]{M. Lombardi}, \hyperref[auth:a143]{M. Milano} & Optimal methods for resource allocation and scheduling: a cross-disciplinary survey & \href{works/LombardiM12.pdf}{Yes} & \cite{LombardiM12} & 2012 & Constraints An Int. J. & 35 & \ref{b:LombardiM12} & \ref{c:LombardiM12}\\
LombardiM12a \href{https://doi.org/10.1016/j.artint.2011.12.001}{LombardiM12a} & \hyperref[auth:a142]{M. Lombardi}, \hyperref[auth:a143]{M. Milano} & A min-flow algorithm for Minimal Critical Set detection in Resource Constrained Project Scheduling & \href{works/LombardiM12a.pdf}{Yes} & \cite{LombardiM12a} & 2012 & Artif. Intell. & 10 & \ref{b:LombardiM12a} & \ref{c:LombardiM12a}\\
BeniniLMR11 \href{https://doi.org/10.1007/s10479-010-0718-x}{BeniniLMR11} & \hyperref[auth:a247]{L. Benini}, \hyperref[auth:a142]{M. Lombardi}, \hyperref[auth:a143]{M. Milano}, \hyperref[auth:a727]{M. Ruggiero} & Optimal resource allocation and scheduling for the {CELL} {BE} platform & \href{works/BeniniLMR11.pdf}{Yes} & \cite{BeniniLMR11} & 2011 & Ann. Oper. Res. & 27 & \ref{b:BeniniLMR11} & \ref{c:BeniniLMR11}\\
BonfiettiLBM11 \href{https://doi.org/10.1007/978-3-642-23786-7\_12}{BonfiettiLBM11} & \hyperref[auth:a203]{A. Bonfietti}, \hyperref[auth:a142]{M. Lombardi}, \hyperref[auth:a247]{L. Benini}, \hyperref[auth:a143]{M. Milano} & A Constraint Based Approach to Cyclic {RCPSP} & \href{works/BonfiettiLBM11.pdf}{Yes} & \cite{BonfiettiLBM11} & 2011 & CP 2011 & 15 & \ref{b:BonfiettiLBM11} & \ref{c:BonfiettiLBM11}\\
LombardiBMB11 \href{https://doi.org/10.1007/978-3-642-21311-3\_14}{LombardiBMB11} & \hyperref[auth:a142]{M. Lombardi}, \hyperref[auth:a203]{A. Bonfietti}, \hyperref[auth:a143]{M. Milano}, \hyperref[auth:a247]{L. Benini} & Precedence Constraint Posting for Cyclic Scheduling Problems & \href{works/LombardiBMB11.pdf}{Yes} & \cite{LombardiBMB11} & 2011 & CPAIOR 2011 & 17 & \ref{b:LombardiBMB11} & \ref{c:LombardiBMB11}\\
LombardiM10 \href{https://doi.org/10.1007/978-3-642-15396-9\_32}{LombardiM10} & \hyperref[auth:a142]{M. Lombardi}, \hyperref[auth:a143]{M. Milano} & Constraint Based Scheduling to Deal with Uncertain Durations and Self-Timed Execution & \href{works/LombardiM10.pdf}{Yes} & \cite{LombardiM10} & 2010 & CP 2010 & 15 & \ref{b:LombardiM10} & \ref{c:LombardiM10}\\
LombardiM10a \href{https://doi.org/10.1016/j.artint.2010.02.004}{LombardiM10a} & \hyperref[auth:a142]{M. Lombardi}, \hyperref[auth:a143]{M. Milano} & Allocation and scheduling of Conditional Task Graphs & \href{works/LombardiM10a.pdf}{Yes} & \cite{LombardiM10a} & 2010 & Artif. Intell. & 30 & \ref{b:LombardiM10a} & \ref{c:LombardiM10a}\\
LombardiM09 \href{https://doi.org/10.1007/978-3-642-04244-7\_45}{LombardiM09} & \hyperref[auth:a142]{M. Lombardi}, \hyperref[auth:a143]{M. Milano} & A Precedence Constraint Posting Approach for the {RCPSP} with Time Lags and Variable Durations & \href{works/LombardiM09.pdf}{Yes} & \cite{LombardiM09} & 2009 & CP 2009 & 15 & \ref{b:LombardiM09} & \ref{c:LombardiM09}\\
HoeveGSL07 \href{http://www.aaai.org/Library/AAAI/2007/aaai07-291.php}{HoeveGSL07} & \hyperref[auth:a651]{Willem Jan van Hoeve}, \hyperref[auth:a652]{Carla P. Gomes}, \hyperref[auth:a653]{B. Selman}, \hyperref[auth:a142]{M. Lombardi} & Optimal Multi-Agent Scheduling with Constraint Programming & \href{works/HoeveGSL07.pdf}{Yes} & \cite{HoeveGSL07} & 2007 & AAAI 2007 & 6 & \ref{b:HoeveGSL07} & \ref{c:HoeveGSL07}\\
\end{longtable}
}

\subsection{Works by Emmanuel Hebrard}
\label{sec:a1}
{\scriptsize
\begin{longtable}{>{\raggedright\arraybackslash}p{3cm}>{\raggedright\arraybackslash}p{6cm}>{\raggedright\arraybackslash}p{7cm}rrrp{3cm}rrr}
\rowcolor{white}\caption{Works from bibtex (Total 17)}\\ \toprule
\rowcolor{white}Key & Authors & Title & LC & Cite & Year & \shortstack{Conference\\/Journal} & Pages & b & c \\ \midrule\endhead
\bottomrule
\endfoot
JuvinHHL23 \href{https://doi.org/10.4230/LIPIcs.CP.2023.19}{JuvinHHL23} & \hyperref[auth:a0]{C. Juvin}, \hyperref[auth:a1]{E. Hebrard}, \hyperref[auth:a2]{L. Houssin}, \hyperref[auth:a3]{P. Lopez} & An Efficient Constraint Programming Approach to Preemptive Job Shop Scheduling & \href{works/JuvinHHL23.pdf}{Yes} & \cite{JuvinHHL23} & 2023 & CP 2023 & 16 & \ref{b:JuvinHHL23} & \ref{c:JuvinHHL23}\\
HebrardALLCMR22 \href{https://doi.org/10.24963/ijcai.2022/643}{HebrardALLCMR22} & \hyperref[auth:a1]{E. Hebrard}, \hyperref[auth:a6]{C. Artigues}, \hyperref[auth:a3]{P. Lopez}, \hyperref[auth:a796]{A. Lusson}, \hyperref[auth:a797]{Steve A. Chien}, \hyperref[auth:a798]{A. Maillard}, \hyperref[auth:a799]{Gregg R. Rabideau} & An Efficient Approach to Data Transfer Scheduling for Long Range Space Exploration & \href{works/HebrardALLCMR22.pdf}{Yes} & \cite{HebrardALLCMR22} & 2022 & IJCAI 2022 & 7 & \ref{b:HebrardALLCMR22} & \ref{c:HebrardALLCMR22}\\
AntuoriHHEN21 \href{https://doi.org/10.4230/LIPIcs.CP.2021.14}{AntuoriHHEN21} & \hyperref[auth:a53]{V. Antuori}, \hyperref[auth:a1]{E. Hebrard}, \hyperref[auth:a54]{M. Huguet}, \hyperref[auth:a55]{S. Essodaigui}, \hyperref[auth:a56]{A. Nguyen} & Combining Monte Carlo Tree Search and Depth First Search Methods for a Car Manufacturing Workshop Scheduling Problem & \href{works/AntuoriHHEN21.pdf}{Yes} & \cite{AntuoriHHEN21} & 2021 & CP 2021 & 16 & \ref{b:AntuoriHHEN21} & \ref{c:AntuoriHHEN21}\\
ArtiguesHQT21 \href{https://doi.org/10.5220/0010190101290136}{ArtiguesHQT21} & \hyperref[auth:a6]{C. Artigues}, \hyperref[auth:a1]{E. Hebrard}, \hyperref[auth:a800]{A. Quilliot}, \hyperref[auth:a801]{H. Toussaint} & Multi-Mode {RCPSP} with Safety Margin Maximization: Models and Algorithms & No & \cite{ArtiguesHQT21} & 2021 & ICORES 2021 & 8 & No & \ref{c:ArtiguesHQT21}\\
AntuoriHHEN20 \href{https://doi.org/10.1007/978-3-030-58475-7\_38}{AntuoriHHEN20} & \hyperref[auth:a53]{V. Antuori}, \hyperref[auth:a1]{E. Hebrard}, \hyperref[auth:a54]{M. Huguet}, \hyperref[auth:a55]{S. Essodaigui}, \hyperref[auth:a56]{A. Nguyen} & Leveraging Reinforcement Learning, Constraint Programming and Local Search: {A} Case Study in Car Manufacturing & \href{works/AntuoriHHEN20.pdf}{Yes} & \cite{AntuoriHHEN20} & 2020 & CP 2020 & 16 & \ref{b:AntuoriHHEN20} & \ref{c:AntuoriHHEN20}\\
GodetLHS20 \href{https://doi.org/10.1609/aaai.v34i02.5510}{GodetLHS20} & \hyperref[auth:a476]{A. Godet}, \hyperref[auth:a246]{X. Lorca}, \hyperref[auth:a1]{E. Hebrard}, \hyperref[auth:a126]{G. Simonin} & Using Approximation within Constraint Programming to Solve the Parallel Machine Scheduling Problem with Additional Unit Resources & \href{works/GodetLHS20.pdf}{Yes} & \cite{GodetLHS20} & 2020 & AAAI 2020 & 8 & \ref{b:GodetLHS20} & \ref{c:GodetLHS20}\\
HebrardHJMPV16 \href{https://doi.org/10.1016/j.dam.2016.07.003}{HebrardHJMPV16} & \hyperref[auth:a1]{E. Hebrard}, \hyperref[auth:a54]{M. Huguet}, \hyperref[auth:a802]{N. Jozefowiez}, \hyperref[auth:a798]{A. Maillard}, \hyperref[auth:a21]{C. Pralet}, \hyperref[auth:a174]{G. Verfaillie} & Approximation of the parallel machine scheduling problem with additional unit resources & \href{works/HebrardHJMPV16.pdf}{Yes} & \cite{HebrardHJMPV16} & 2016 & Discret. Appl. Math. & 10 & \ref{b:HebrardHJMPV16} & \ref{c:HebrardHJMPV16}\\
GrimesH15 \href{https://doi.org/10.1287/ijoc.2014.0625}{GrimesH15} & \hyperref[auth:a182]{D. Grimes}, \hyperref[auth:a1]{E. Hebrard} & Solving Variants of the Job Shop Scheduling Problem Through Conflict-Directed Search & No & \cite{GrimesH15} & 2015 & {INFORMS} J. Comput. & 17 & No & \ref{c:GrimesH15}\\
SialaAH15 \href{https://doi.org/10.1007/978-3-319-23219-5\_28}{SialaAH15} & \hyperref[auth:a129]{M. Siala}, \hyperref[auth:a6]{C. Artigues}, \hyperref[auth:a1]{E. Hebrard} & Two Clause Learning Approaches for Disjunctive Scheduling & \href{works/SialaAH15.pdf}{Yes} & \cite{SialaAH15} & 2015 & CP 2015 & 10 & \ref{b:SialaAH15} & \ref{c:SialaAH15}\\
SimoninAHL15 \href{https://doi.org/10.1007/s10601-014-9169-3}{SimoninAHL15} & \hyperref[auth:a126]{G. Simonin}, \hyperref[auth:a6]{C. Artigues}, \hyperref[auth:a1]{E. Hebrard}, \hyperref[auth:a3]{P. Lopez} & Scheduling scientific experiments for comet exploration & \href{works/SimoninAHL15.pdf}{Yes} & \cite{SimoninAHL15} & 2015 & Constraints An Int. J. & 23 & \ref{b:SimoninAHL15} & \ref{c:SimoninAHL15}\\
BessiereHMQW14 \href{https://doi.org/10.1007/978-3-319-07046-9\_23}{BessiereHMQW14} & \hyperref[auth:a333]{C. Bessiere}, \hyperref[auth:a1]{E. Hebrard}, \hyperref[auth:a334]{M. M{\'{e}}nard}, \hyperref[auth:a37]{C. Quimper}, \hyperref[auth:a278]{T. Walsh} & Buffered Resource Constraint: Algorithms and Complexity & \href{works/BessiereHMQW14.pdf}{Yes} & \cite{BessiereHMQW14} & 2014 & CPAIOR 2014 & 16 & \ref{b:BessiereHMQW14} & \ref{c:BessiereHMQW14}\\
BillautHL12 \href{https://doi.org/10.1007/978-3-642-29828-8\_5}{BillautHL12} & \hyperref[auth:a342]{J. Billaut}, \hyperref[auth:a1]{E. Hebrard}, \hyperref[auth:a3]{P. Lopez} & Complete Characterization of Near-Optimal Sequences for the Two-Machine Flow Shop Scheduling Problem & \href{works/BillautHL12.pdf}{Yes} & \cite{BillautHL12} & 2012 & CPAIOR 2012 & 15 & \ref{b:BillautHL12} & \ref{c:BillautHL12}\\
SimoninAHL12 \href{https://doi.org/10.1007/978-3-642-33558-7\_5}{SimoninAHL12} & \hyperref[auth:a126]{G. Simonin}, \hyperref[auth:a6]{C. Artigues}, \hyperref[auth:a1]{E. Hebrard}, \hyperref[auth:a3]{P. Lopez} & Scheduling Scientific Experiments on the Rosetta/Philae Mission & \href{works/SimoninAHL12.pdf}{Yes} & \cite{SimoninAHL12} & 2012 & CP 2012 & 15 & \ref{b:SimoninAHL12} & \ref{c:SimoninAHL12}\\
GrimesH11 \href{https://doi.org/10.1007/978-3-642-23786-7\_28}{GrimesH11} & \hyperref[auth:a182]{D. Grimes}, \hyperref[auth:a1]{E. Hebrard} & Models and Strategies for Variants of the Job Shop Scheduling Problem & \href{works/GrimesH11.pdf}{Yes} & \cite{GrimesH11} & 2011 & CP 2011 & 17 & \ref{b:GrimesH11} & \ref{c:GrimesH11}\\
GrimesH10 \href{https://doi.org/10.1007/978-3-642-13520-0\_19}{GrimesH10} & \hyperref[auth:a182]{D. Grimes}, \hyperref[auth:a1]{E. Hebrard} & Job Shop Scheduling with Setup Times and Maximal Time-Lags: {A} Simple Constraint Programming Approach & \href{works/GrimesH10.pdf}{Yes} & \cite{GrimesH10} & 2010 & CPAIOR 2010 & 15 & \ref{b:GrimesH10} & \ref{c:GrimesH10}\\
GrimesHM09 \href{https://doi.org/10.1007/978-3-642-04244-7\_33}{GrimesHM09} & \hyperref[auth:a182]{D. Grimes}, \hyperref[auth:a1]{E. Hebrard}, \hyperref[auth:a82]{A. Malapert} & Closing the Open Shop: Contradicting Conventional Wisdom & \href{works/GrimesHM09.pdf}{Yes} & \cite{GrimesHM09} & 2009 & CP 2009 & 9 & \ref{b:GrimesHM09} & \ref{c:GrimesHM09}\\
HebrardTW05 \href{https://doi.org/10.1007/11564751\_117}{HebrardTW05} & \hyperref[auth:a1]{E. Hebrard}, \hyperref[auth:a277]{P. Tyler}, \hyperref[auth:a278]{T. Walsh} & Computing Super-Schedules & \href{works/HebrardTW05.pdf}{Yes} & \cite{HebrardTW05} & 2005 & CP 2005 & 1 & \ref{b:HebrardTW05} & \ref{c:HebrardTW05}\\
\end{longtable}
}

\subsection{Works by Nicolas Beldiceanu}
\label{sec:a128}
{\scriptsize
\begin{longtable}{>{\raggedright\arraybackslash}p{3cm}>{\raggedright\arraybackslash}p{6cm}>{\raggedright\arraybackslash}p{7cm}rrrp{3cm}rrr}
\rowcolor{white}\caption{Works from bibtex (Total 13)}\\ \toprule
\rowcolor{white}Key & Authors & Title & LC & Cite & Year & \shortstack{Conference\\/Journal} & Pages & b & c \\ \midrule\endhead
\bottomrule
\endfoot
Madi-WambaLOBM17 \href{https://doi.org/10.1109/ICPADS.2017.00089}{Madi-WambaLOBM17} & \hyperref[auth:a323]{G. Madi{-}Wamba}, \hyperref[auth:a723]{Y. Li}, \hyperref[auth:a724]{A. Orgerie}, \hyperref[auth:a128]{N. Beldiceanu}, \hyperref[auth:a725]{J. Menaud} & Green Energy Aware Scheduling Problem in Virtualized Datacenters & \href{works/Madi-WambaLOBM17.pdf}{Yes} & \cite{Madi-WambaLOBM17} & 2017 & ICPADS 2017 & 8 & \ref{b:Madi-WambaLOBM17} & \ref{c:Madi-WambaLOBM17}\\
Madi-WambaB16 \href{https://doi.org/10.1007/978-3-319-33954-2\_18}{Madi-WambaB16} & \hyperref[auth:a323]{G. Madi{-}Wamba}, \hyperref[auth:a128]{N. Beldiceanu} & The TaskIntersection Constraint & \href{works/Madi-WambaB16.pdf}{Yes} & \cite{Madi-WambaB16} & 2016 & CPAIOR 2016 & 16 & \ref{b:Madi-WambaB16} & \ref{c:Madi-WambaB16}\\
LetortCB15 \href{https://doi.org/10.1007/s10601-014-9172-8}{LetortCB15} & \hyperref[auth:a127]{A. Letort}, \hyperref[auth:a91]{M. Carlsson}, \hyperref[auth:a128]{N. Beldiceanu} & Synchronized sweep algorithms for scalable scheduling constraints & \href{works/LetortCB15.pdf}{Yes} & \cite{LetortCB15} & 2015 & Constraints An Int. J. & 52 & \ref{b:LetortCB15} & \ref{c:LetortCB15}\\
LetortCB13 \href{https://doi.org/10.1007/978-3-642-38171-3\_10}{LetortCB13} & \hyperref[auth:a127]{A. Letort}, \hyperref[auth:a91]{M. Carlsson}, \hyperref[auth:a128]{N. Beldiceanu} & A Synchronized Sweep Algorithm for the \emph{k-dimensional cumulative} Constraint & \href{works/LetortCB13.pdf}{Yes} & \cite{LetortCB13} & 2013 & CPAIOR 2013 & 16 & \ref{b:LetortCB13} & \ref{c:LetortCB13}\\
LetortBC12 \href{https://doi.org/10.1007/978-3-642-33558-7\_33}{LetortBC12} & \hyperref[auth:a127]{A. Letort}, \hyperref[auth:a128]{N. Beldiceanu}, \hyperref[auth:a91]{M. Carlsson} & A Scalable Sweep Algorithm for the cumulative Constraint & \href{works/LetortBC12.pdf}{Yes} & \cite{LetortBC12} & 2012 & CP 2012 & 16 & \ref{b:LetortBC12} & \ref{c:LetortBC12}\\
BeldiceanuCDP11 \href{https://doi.org/10.1007/s10479-010-0731-0}{BeldiceanuCDP11} & \hyperref[auth:a128]{N. Beldiceanu}, \hyperref[auth:a91]{M. Carlsson}, \hyperref[auth:a245]{S. Demassey}, \hyperref[auth:a362]{E. Poder} & New filtering for the \emph{cumulative} constraint in the context of non-overlapping rectangles & \href{works/BeldiceanuCDP11.pdf}{Yes} & \cite{BeldiceanuCDP11} & 2011 & Ann. Oper. Res. & 24 & \ref{b:BeldiceanuCDP11} & \ref{c:BeldiceanuCDP11}\\
ClercqPBJ11 \href{https://doi.org/10.1007/978-3-642-23786-7\_20}{ClercqPBJ11} & \hyperref[auth:a248]{Alexis De Clercq}, \hyperref[auth:a226]{T. Petit}, \hyperref[auth:a128]{N. Beldiceanu}, \hyperref[auth:a249]{N. Jussien} & Filtering Algorithms for Discrete Cumulative Problems with Overloads of Resource & \href{works/ClercqPBJ11.pdf}{Yes} & \cite{ClercqPBJ11} & 2011 & CP 2011 & 16 & \ref{b:ClercqPBJ11} & \ref{c:ClercqPBJ11}\\
BeldiceanuCP08 \href{https://doi.org/10.1007/978-3-540-68155-7\_5}{BeldiceanuCP08} & \hyperref[auth:a128]{N. Beldiceanu}, \hyperref[auth:a91]{M. Carlsson}, \hyperref[auth:a362]{E. Poder} & New Filtering for the cumulative Constraint in the Context of Non-Overlapping Rectangles & \href{works/BeldiceanuCP08.pdf}{Yes} & \cite{BeldiceanuCP08} & 2008 & CPAIOR 2008 & 15 & \ref{b:BeldiceanuCP08} & \ref{c:BeldiceanuCP08}\\
PoderB08 \href{http://www.aaai.org/Library/ICAPS/2008/icaps08-033.php}{PoderB08} & \hyperref[auth:a362]{E. Poder}, \hyperref[auth:a128]{N. Beldiceanu} & Filtering for a Continuous Multi-Resources cumulative Constraint with Resource Consumption and Production & \href{works/PoderB08.pdf}{Yes} & \cite{PoderB08} & 2008 & ICAPS 2008 & 8 & \ref{b:PoderB08} & \ref{c:PoderB08}\\
BeldiceanuP07 \href{https://doi.org/10.1007/978-3-540-72397-4\_16}{BeldiceanuP07} & \hyperref[auth:a128]{N. Beldiceanu}, \hyperref[auth:a362]{E. Poder} & A Continuous Multi-resources \emph{cumulative} Constraint with Positive-Negative Resource Consumption-Production & \href{works/BeldiceanuP07.pdf}{Yes} & \cite{BeldiceanuP07} & 2007 & CPAIOR 2007 & 15 & \ref{b:BeldiceanuP07} & \ref{c:BeldiceanuP07}\\
PoderBS04 \href{https://doi.org/10.1016/S0377-2217(02)00756-7}{PoderBS04} & \hyperref[auth:a362]{E. Poder}, \hyperref[auth:a128]{N. Beldiceanu}, \hyperref[auth:a722]{E. Sanlaville} & Computing a lower approximation of the compulsory part of a task with varying duration and varying resource consumption & \href{works/PoderBS04.pdf}{Yes} & \cite{PoderBS04} & 2004 & Eur. J. Oper. Res. & 16 & \ref{b:PoderBS04} & \ref{c:PoderBS04}\\
BeldiceanuC02 \href{https://doi.org/10.1007/3-540-46135-3\_5}{BeldiceanuC02} & \hyperref[auth:a128]{N. Beldiceanu}, \hyperref[auth:a91]{M. Carlsson} & A New Multi-resource cumulatives Constraint with Negative Heights & \href{works/BeldiceanuC02.pdf}{Yes} & \cite{BeldiceanuC02} & 2002 & CP 2002 & 17 & \ref{b:BeldiceanuC02} & \ref{c:BeldiceanuC02}\\
AggounB93 \href{https://www.sciencedirect.com/science/article/pii/089571779390068A}{AggounB93} & \hyperref[auth:a734]{A. Aggoun}, \hyperref[auth:a128]{N. Beldiceanu} & Extending {CHIP} in order to solve complex scheduling and placement problems & \href{works/AggounB93.pdf}{Yes} & \cite{AggounB93} & 1993 & Mathematical and Computer Modelling & 17 & \ref{b:AggounB93} & \ref{c:AggounB93}\\
\end{longtable}
}

\subsection{Works by Christian Artigues}
\label{sec:a6}
{\scriptsize
\begin{longtable}{>{\raggedright\arraybackslash}p{3cm}>{\raggedright\arraybackslash}p{6cm}>{\raggedright\arraybackslash}p{7cm}rrrp{3cm}rrr}
\rowcolor{white}\caption{Works from bibtex (Total 12)}\\ \toprule
\rowcolor{white}Key & Authors & Title & LC & Cite & Year & \shortstack{Conference\\/Journal} & Pages & b & c \\ \midrule\endhead
\bottomrule
\endfoot
PovedaAA23 \href{https://doi.org/10.4230/LIPIcs.CP.2023.31}{PovedaAA23} & \hyperref[auth:a4]{G. Pov{\'{e}}da}, \hyperref[auth:a5]{N. {\'{A}}lvarez}, \hyperref[auth:a6]{C. Artigues} & Partially Preemptive Multi Skill/Mode Resource-Constrained Project Scheduling with Generalized Precedence Relations and Calendars & \href{works/PovedaAA23.pdf}{Yes} & \cite{PovedaAA23} & 2023 & CP 2023 & 21 & \ref{b:PovedaAA23} & \ref{c:PovedaAA23}\\
HebrardALLCMR22 \href{https://doi.org/10.24963/ijcai.2022/643}{HebrardALLCMR22} & \hyperref[auth:a1]{E. Hebrard}, \hyperref[auth:a6]{C. Artigues}, \hyperref[auth:a3]{P. Lopez}, \hyperref[auth:a796]{A. Lusson}, \hyperref[auth:a797]{Steve A. Chien}, \hyperref[auth:a798]{A. Maillard}, \hyperref[auth:a799]{Gregg R. Rabideau} & An Efficient Approach to Data Transfer Scheduling for Long Range Space Exploration & \href{works/HebrardALLCMR22.pdf}{Yes} & \cite{HebrardALLCMR22} & 2022 & IJCAI 2022 & 7 & \ref{b:HebrardALLCMR22} & \ref{c:HebrardALLCMR22}\\
PohlAK22 \href{https://doi.org/10.1016/j.ejor.2021.08.028}{PohlAK22} & \hyperref[auth:a444]{M. Pohl}, \hyperref[auth:a6]{C. Artigues}, \hyperref[auth:a445]{R. Kolisch} & Solving the time-discrete winter runway scheduling problem: {A} column generation and constraint programming approach & \href{works/PohlAK22.pdf}{Yes} & \cite{PohlAK22} & 2022 & Eur. J. Oper. Res. & 16 & \ref{b:PohlAK22} & \ref{c:PohlAK22}\\
ArtiguesHQT21 \href{https://doi.org/10.5220/0010190101290136}{ArtiguesHQT21} & \hyperref[auth:a6]{C. Artigues}, \hyperref[auth:a1]{E. Hebrard}, \hyperref[auth:a800]{A. Quilliot}, \hyperref[auth:a801]{H. Toussaint} & Multi-Mode {RCPSP} with Safety Margin Maximization: Models and Algorithms & No & \cite{ArtiguesHQT21} & 2021 & ICORES 2021 & 8 & No & \ref{c:ArtiguesHQT21}\\
Polo-MejiaALB20 \href{https://doi.org/10.1080/00207543.2019.1693654}{Polo-MejiaALB20} & \hyperref[auth:a522]{O. Polo{-}Mej{\'{\i}}a}, \hyperref[auth:a6]{C. Artigues}, \hyperref[auth:a3]{P. Lopez}, \hyperref[auth:a523]{V. Basini} & Mixed-integer/linear and constraint programming approaches for activity scheduling in a nuclear research facility & \href{works/Polo-MejiaALB20.pdf}{Yes} & \cite{Polo-MejiaALB20} & 2020 & Int. J. Prod. Res. & 18 & \ref{b:Polo-MejiaALB20} & \ref{c:Polo-MejiaALB20}\\
NattafAL17 \href{https://doi.org/10.1007/s10601-017-9271-4}{NattafAL17} & \hyperref[auth:a81]{M. Nattaf}, \hyperref[auth:a6]{C. Artigues}, \hyperref[auth:a3]{P. Lopez} & Cumulative scheduling with variable task profiles and concave piecewise linear processing rate functions & \href{works/NattafAL17.pdf}{Yes} & \cite{NattafAL17} & 2017 & Constraints An Int. J. & 18 & \ref{b:NattafAL17} & \ref{c:NattafAL17}\\
NattafAL15 \href{https://doi.org/10.1007/s10601-015-9192-z}{NattafAL15} & \hyperref[auth:a81]{M. Nattaf}, \hyperref[auth:a6]{C. Artigues}, \hyperref[auth:a3]{P. Lopez} & A hybrid exact method for a scheduling problem with a continuous resource and energy constraints & \href{works/NattafAL15.pdf}{Yes} & \cite{NattafAL15} & 2015 & Constraints An Int. J. & 21 & \ref{b:NattafAL15} & \ref{c:NattafAL15}\\
SialaAH15 \href{https://doi.org/10.1007/978-3-319-23219-5\_28}{SialaAH15} & \hyperref[auth:a129]{M. Siala}, \hyperref[auth:a6]{C. Artigues}, \hyperref[auth:a1]{E. Hebrard} & Two Clause Learning Approaches for Disjunctive Scheduling & \href{works/SialaAH15.pdf}{Yes} & \cite{SialaAH15} & 2015 & CP 2015 & 10 & \ref{b:SialaAH15} & \ref{c:SialaAH15}\\
SimoninAHL15 \href{https://doi.org/10.1007/s10601-014-9169-3}{SimoninAHL15} & \hyperref[auth:a126]{G. Simonin}, \hyperref[auth:a6]{C. Artigues}, \hyperref[auth:a1]{E. Hebrard}, \hyperref[auth:a3]{P. Lopez} & Scheduling scientific experiments for comet exploration & \href{works/SimoninAHL15.pdf}{Yes} & \cite{SimoninAHL15} & 2015 & Constraints An Int. J. & 23 & \ref{b:SimoninAHL15} & \ref{c:SimoninAHL15}\\
SimoninAHL12 \href{https://doi.org/10.1007/978-3-642-33558-7\_5}{SimoninAHL12} & \hyperref[auth:a126]{G. Simonin}, \hyperref[auth:a6]{C. Artigues}, \hyperref[auth:a1]{E. Hebrard}, \hyperref[auth:a3]{P. Lopez} & Scheduling Scientific Experiments on the Rosetta/Philae Mission & \href{works/SimoninAHL12.pdf}{Yes} & \cite{SimoninAHL12} & 2012 & CP 2012 & 15 & \ref{b:SimoninAHL12} & \ref{c:SimoninAHL12}\\
ArtiguesBF04 \href{https://doi.org/10.1007/978-3-540-24664-0\_3}{ArtiguesBF04} & \hyperref[auth:a6]{C. Artigues}, \hyperref[auth:a387]{S. Belmokhtar}, \hyperref[auth:a360]{D. Feillet} & A New Exact Solution Algorithm for the Job Shop Problem with Sequence-Dependent Setup Times & \href{works/ArtiguesBF04.pdf}{Yes} & \cite{ArtiguesBF04} & 2004 & CPAIOR 2004 & 13 & \ref{b:ArtiguesBF04} & \ref{c:ArtiguesBF04}\\
ArtiguesR00 \href{https://doi.org/10.1016/S0377-2217(99)00496-8}{ArtiguesR00} & \hyperref[auth:a6]{C. Artigues}, \hyperref[auth:a721]{F. Roubellat} & A polynomial activity insertion algorithm in a multi-resource schedule with cumulative constraints and multiple modes & \href{works/ArtiguesR00.pdf}{Yes} & \cite{ArtiguesR00} & 2000 & Eur. J. Oper. Res. & 20 & \ref{b:ArtiguesR00} & \ref{c:ArtiguesR00}\\
\end{longtable}
}

\subsection{Works by Pierre Lopez}
\label{sec:a3}
{\scriptsize
\begin{longtable}{>{\raggedright\arraybackslash}p{3cm}>{\raggedright\arraybackslash}p{6cm}>{\raggedright\arraybackslash}p{7cm}rrrp{3cm}rrr}
\rowcolor{white}\caption{Works from bibtex (Total 12)}\\ \toprule
\rowcolor{white}Key & Authors & Title & LC & Cite & Year & \shortstack{Conference\\/Journal} & Pages & b & c \\ \midrule\endhead
\bottomrule
\endfoot
JuvinHHL23 \href{https://doi.org/10.4230/LIPIcs.CP.2023.19}{JuvinHHL23} & \hyperref[auth:a0]{C. Juvin}, \hyperref[auth:a1]{E. Hebrard}, \hyperref[auth:a2]{L. Houssin}, \hyperref[auth:a3]{P. Lopez} & An Efficient Constraint Programming Approach to Preemptive Job Shop Scheduling & \href{works/JuvinHHL23.pdf}{Yes} & \cite{JuvinHHL23} & 2023 & CP 2023 & 16 & \ref{b:JuvinHHL23} & \ref{c:JuvinHHL23}\\
JuvinHL23 \href{https://doi.org/10.1007/978-3-031-33271-5\_23}{JuvinHL23} & \hyperref[auth:a0]{C. Juvin}, \hyperref[auth:a2]{L. Houssin}, \hyperref[auth:a3]{P. Lopez} & Constraint Programming for the Robust Two-Machine Flow-Shop Scheduling Problem with Budgeted Uncertainty & \href{works/JuvinHL23.pdf}{Yes} & \cite{JuvinHL23} & 2023 & CPAIOR 2023 & 16 & \ref{b:JuvinHL23} & \ref{c:JuvinHL23}\\
HebrardALLCMR22 \href{https://doi.org/10.24963/ijcai.2022/643}{HebrardALLCMR22} & \hyperref[auth:a1]{E. Hebrard}, \hyperref[auth:a6]{C. Artigues}, \hyperref[auth:a3]{P. Lopez}, \hyperref[auth:a796]{A. Lusson}, \hyperref[auth:a797]{Steve A. Chien}, \hyperref[auth:a798]{A. Maillard}, \hyperref[auth:a799]{Gregg R. Rabideau} & An Efficient Approach to Data Transfer Scheduling for Long Range Space Exploration & \href{works/HebrardALLCMR22.pdf}{Yes} & \cite{HebrardALLCMR22} & 2022 & IJCAI 2022 & 7 & \ref{b:HebrardALLCMR22} & \ref{c:HebrardALLCMR22}\\
Polo-MejiaALB20 \href{https://doi.org/10.1080/00207543.2019.1693654}{Polo-MejiaALB20} & \hyperref[auth:a522]{O. Polo{-}Mej{\'{\i}}a}, \hyperref[auth:a6]{C. Artigues}, \hyperref[auth:a3]{P. Lopez}, \hyperref[auth:a523]{V. Basini} & Mixed-integer/linear and constraint programming approaches for activity scheduling in a nuclear research facility & \href{works/Polo-MejiaALB20.pdf}{Yes} & \cite{Polo-MejiaALB20} & 2020 & Int. J. Prod. Res. & 18 & \ref{b:Polo-MejiaALB20} & \ref{c:Polo-MejiaALB20}\\
NattafAL17 \href{https://doi.org/10.1007/s10601-017-9271-4}{NattafAL17} & \hyperref[auth:a81]{M. Nattaf}, \hyperref[auth:a6]{C. Artigues}, \hyperref[auth:a3]{P. Lopez} & Cumulative scheduling with variable task profiles and concave piecewise linear processing rate functions & \href{works/NattafAL17.pdf}{Yes} & \cite{NattafAL17} & 2017 & Constraints An Int. J. & 18 & \ref{b:NattafAL17} & \ref{c:NattafAL17}\\
NattafAL15 \href{https://doi.org/10.1007/s10601-015-9192-z}{NattafAL15} & \hyperref[auth:a81]{M. Nattaf}, \hyperref[auth:a6]{C. Artigues}, \hyperref[auth:a3]{P. Lopez} & A hybrid exact method for a scheduling problem with a continuous resource and energy constraints & \href{works/NattafAL15.pdf}{Yes} & \cite{NattafAL15} & 2015 & Constraints An Int. J. & 21 & \ref{b:NattafAL15} & \ref{c:NattafAL15}\\
SimoninAHL15 \href{https://doi.org/10.1007/s10601-014-9169-3}{SimoninAHL15} & \hyperref[auth:a126]{G. Simonin}, \hyperref[auth:a6]{C. Artigues}, \hyperref[auth:a1]{E. Hebrard}, \hyperref[auth:a3]{P. Lopez} & Scheduling scientific experiments for comet exploration & \href{works/SimoninAHL15.pdf}{Yes} & \cite{SimoninAHL15} & 2015 & Constraints An Int. J. & 23 & \ref{b:SimoninAHL15} & \ref{c:SimoninAHL15}\\
BillautHL12 \href{https://doi.org/10.1007/978-3-642-29828-8\_5}{BillautHL12} & \hyperref[auth:a342]{J. Billaut}, \hyperref[auth:a1]{E. Hebrard}, \hyperref[auth:a3]{P. Lopez} & Complete Characterization of Near-Optimal Sequences for the Two-Machine Flow Shop Scheduling Problem & \href{works/BillautHL12.pdf}{Yes} & \cite{BillautHL12} & 2012 & CPAIOR 2012 & 15 & \ref{b:BillautHL12} & \ref{c:BillautHL12}\\
SimoninAHL12 \href{https://doi.org/10.1007/978-3-642-33558-7\_5}{SimoninAHL12} & \hyperref[auth:a126]{G. Simonin}, \hyperref[auth:a6]{C. Artigues}, \hyperref[auth:a1]{E. Hebrard}, \hyperref[auth:a3]{P. Lopez} & Scheduling Scientific Experiments on the Rosetta/Philae Mission & \href{works/SimoninAHL12.pdf}{Yes} & \cite{SimoninAHL12} & 2012 & CP 2012 & 15 & \ref{b:SimoninAHL12} & \ref{c:SimoninAHL12}\\
LahimerLH11 \href{https://doi.org/10.1007/978-3-642-21311-3\_12}{LahimerLH11} & \hyperref[auth:a353]{A. Lahimer}, \hyperref[auth:a3]{P. Lopez}, \hyperref[auth:a354]{M. Haouari} & Climbing Depth-Bounded Adjacent Discrepancy Search for Solving Hybrid Flow Shop Scheduling Problems with Multiprocessor Tasks & \href{works/LahimerLH11.pdf}{Yes} & \cite{LahimerLH11} & 2011 & CPAIOR 2011 & 14 & \ref{b:LahimerLH11} & \ref{c:LahimerLH11}\\
TrojetHL11 \href{https://doi.org/10.1016/j.cie.2010.08.014}{TrojetHL11} & \hyperref[auth:a715]{M. Trojet}, \hyperref[auth:a716]{F. H'Mida}, \hyperref[auth:a3]{P. Lopez} & Project scheduling under resource constraints: Application of the cumulative global constraint in a decision support framework & \href{works/TrojetHL11.pdf}{Yes} & \cite{TrojetHL11} & 2011 & Comput. Ind. Eng. & 7 & \ref{b:TrojetHL11} & \ref{c:TrojetHL11}\\
LopezAKYG00 \href{https://doi.org/10.1016/S0947-3580(00)71114-9}{LopezAKYG00} & \hyperref[auth:a3]{P. Lopez}, \hyperref[auth:a693]{H. Alla}, \hyperref[auth:a690]{O. Korbaa}, \hyperref[auth:a691]{P. Yim}, \hyperref[auth:a692]{J. Gentina} & Discussion on: 'Solving Transient Scheduling Problems with Constraint Programming' by O. Korbaa, P. Yim, and {J.-C.} Gentina & \href{works/LopezAKYG00.pdf}{Yes} & \cite{LopezAKYG00} & 2000 & Eur. J. Control & 4 & \ref{b:LopezAKYG00} & \ref{c:LopezAKYG00}\\
\end{longtable}
}

\subsection{Works by Roman Bart{\'{a}}k}
\label{sec:a152}
{\scriptsize
\begin{longtable}{>{\raggedright\arraybackslash}p{3cm}>{\raggedright\arraybackslash}p{6cm}>{\raggedright\arraybackslash}p{7cm}rrrp{3cm}rrr}
\rowcolor{white}\caption{Works from bibtex (Total 11)}\\ \toprule
\rowcolor{white}Key & Authors & Title & LC & Cite & Year & \shortstack{Conference\\/Journal} & Pages & b & c \\ \midrule\endhead
\bottomrule
\endfoot
SvancaraB22 \href{https://doi.org/10.5220/0010869700003116}{SvancaraB22} & \hyperref[auth:a787]{J. Svancara}, \hyperref[auth:a152]{R. Bart{\'{a}}k} & Tackling Train Routing via Multi-agent Pathfinding and Constraint-based Scheduling & \href{works/SvancaraB22.pdf}{Yes} & \cite{SvancaraB22} & 2022 & ICAART 2022 & 8 & \ref{b:SvancaraB22} & \ref{c:SvancaraB22}\\
JelinekB16 \href{https://doi.org/10.1007/978-3-319-28228-2\_1}{JelinekB16} & \hyperref[auth:a788]{J. Jel{\'{\i}}nek}, \hyperref[auth:a152]{R. Bart{\'{a}}k} & Using Constraint Logic Programming to Schedule Solar Array Operations on the International Space Station & \href{works/JelinekB16.pdf}{Yes} & \cite{JelinekB16} & 2016 & PADL 2016 & 10 & \ref{b:JelinekB16} & \ref{c:JelinekB16}\\
BartakV15 \href{}{BartakV15} & \hyperref[auth:a152]{R. Bart{\'{a}}k}, \hyperref[auth:a313]{M. Vlk} & Reactive Recovery from Machine Breakdown in Production Scheduling with Temporal Distance and Resource Constraints & \href{works/BartakV15.pdf}{Yes} & \cite{BartakV15} & 2015 & ICAART 2015 & 12 & \ref{b:BartakV15} & \ref{c:BartakV15}\\
Bartak14 \href{}{Bartak14} & \hyperref[auth:a152]{R. Bart{\'{a}}k} & Planning and Scheduling & No & \cite{Bartak14} & 2014 & n/a & null & No & \ref{c:Bartak14}\\
BartakS11 \href{https://doi.org/10.1007/s10601-011-9109-4}{BartakS11} & \hyperref[auth:a152]{R. Bart{\'{a}}k}, \hyperref[auth:a153]{Miguel A. Salido} & Constraint satisfaction for planning and scheduling problems & \href{works/BartakS11.pdf}{Yes} & \cite{BartakS11} & 2011 & Constraints An Int. J. & 5 & \ref{b:BartakS11} & \ref{c:BartakS11}\\
BartakCS10 \href{https://doi.org/10.1007/s10479-008-0492-1}{BartakCS10} & \hyperref[auth:a152]{R. Bart{\'{a}}k}, \hyperref[auth:a162]{O. Cepek}, \hyperref[auth:a789]{P. Surynek} & Discovering implied constraints in precedence graphs with alternatives & \href{works/BartakCS10.pdf}{Yes} & \cite{BartakCS10} & 2010 & Ann. Oper. Res. & 31 & \ref{b:BartakCS10} & \ref{c:BartakCS10}\\
BartakSR10 \href{https://doi.org/10.1017/S0269888910000202}{BartakSR10} & \hyperref[auth:a152]{R. Bart{\'{a}}k}, \hyperref[auth:a153]{Miguel A. Salido}, \hyperref[auth:a318]{F. Rossi} & New trends in constraint satisfaction, planning, and scheduling: a survey & \href{works/BartakSR10.pdf}{Yes} & \cite{BartakSR10} & 2010 & Knowl. Eng. Rev. & 31 & \ref{b:BartakSR10} & \ref{c:BartakSR10}\\
VilimBC05 \href{https://doi.org/10.1007/s10601-005-2814-0}{VilimBC05} & \hyperref[auth:a121]{P. Vil{\'{\i}}m}, \hyperref[auth:a152]{R. Bart{\'{a}}k}, \hyperref[auth:a162]{O. Cepek} & Extension of \emph{O}(\emph{n} log \emph{n}) Filtering Algorithms for the Unary Resource Constraint to Optional Activities & \href{works/VilimBC05.pdf}{Yes} & \cite{VilimBC05} & 2005 & Constraints An Int. J. & 23 & \ref{b:VilimBC05} & \ref{c:VilimBC05}\\
VilimBC04 \href{https://doi.org/10.1007/978-3-540-30201-8\_8}{VilimBC04} & \hyperref[auth:a121]{P. Vil{\'{\i}}m}, \hyperref[auth:a152]{R. Bart{\'{a}}k}, \hyperref[auth:a162]{O. Cepek} & Unary Resource Constraint with Optional Activities & \href{works/VilimBC04.pdf}{Yes} & \cite{VilimBC04} & 2004 & CP 2004 & 15 & \ref{b:VilimBC04} & \ref{c:VilimBC04}\\
Bartak02 \href{https://doi.org/10.1007/3-540-46135-3\_39}{Bartak02} & \hyperref[auth:a152]{R. Bart{\'{a}}k} & Visopt ShopFloor: On the Edge of Planning and Scheduling & \href{works/Bartak02.pdf}{Yes} & \cite{Bartak02} & 2002 & CP 2002 & 16 & \ref{b:Bartak02} & \ref{c:Bartak02}\\
Bartak02a \href{https://doi.org/10.1007/3-540-36607-5\_14}{Bartak02a} & \hyperref[auth:a152]{R. Bart{\'{a}}k} & Visopt ShopFloor: Going Beyond Traditional Scheduling & \href{works/Bartak02a.pdf}{Yes} & \cite{Bartak02a} & 2002 & ERCIM/CologNet 2002 & 15 & \ref{b:Bartak02a} & \ref{c:Bartak02a}\\
\end{longtable}
}

\subsection{Works by Petr Vil{\'{\i}}m}
\label{sec:a121}
{\scriptsize
\begin{longtable}{>{\raggedright\arraybackslash}p{3cm}>{\raggedright\arraybackslash}p{6cm}>{\raggedright\arraybackslash}p{7cm}rrrp{3cm}rrr}
\rowcolor{white}\caption{Works from bibtex (Total 11)}\\ \toprule
\rowcolor{white}Key & Authors & Title & LC & Cite & Year & \shortstack{Conference\\/Journal} & Pages & b & c \\ \midrule\endhead
\bottomrule
\endfoot
LaborieRSV18 \href{https://doi.org/10.1007/s10601-018-9281-x}{LaborieRSV18} & \hyperref[auth:a118]{P. Laborie}, \hyperref[auth:a119]{J. Rogerie}, \hyperref[auth:a120]{P. Shaw}, \hyperref[auth:a121]{P. Vil{\'{\i}}m} & {IBM} {ILOG} {CP} optimizer for scheduling - 20+ years of scheduling with constraints at {IBM/ILOG} & \href{works/LaborieRSV18.pdf}{Yes} & \cite{LaborieRSV18} & 2018 & Constraints An Int. J. & 41 & \ref{b:LaborieRSV18} & \ref{c:LaborieRSV18}\\
VilimLS15 \href{https://doi.org/10.1007/978-3-319-18008-3\_30}{VilimLS15} & \hyperref[auth:a121]{P. Vil{\'{\i}}m}, \hyperref[auth:a118]{P. Laborie}, \hyperref[auth:a120]{P. Shaw} & Failure-Directed Search for Constraint-Based Scheduling & \href{works/VilimLS15.pdf}{Yes} & \cite{VilimLS15} & 2015 & CPAIOR 2015 & 17 & \ref{b:VilimLS15} & \ref{c:VilimLS15}\\
Vilim11 \href{https://doi.org/10.1007/978-3-642-21311-3\_22}{Vilim11} & \hyperref[auth:a121]{P. Vil{\'{\i}}m} & Timetable Edge Finding Filtering Algorithm for Discrete Cumulative Resources & \href{works/Vilim11.pdf}{Yes} & \cite{Vilim11} & 2011 & CPAIOR 2011 & 16 & \ref{b:Vilim11} & \ref{c:Vilim11}\\
Vilim09 \href{https://doi.org/10.1007/978-3-642-04244-7\_62}{Vilim09} & \hyperref[auth:a121]{P. Vil{\'{\i}}m} & Edge Finding Filtering Algorithm for Discrete Cumulative Resources in \emph{O}(\emph{kn} log \emph{n})\{{\textbackslash}mathcal O\}(kn \{{\textbackslash}rm log\} n) & \href{works/Vilim09.pdf}{Yes} & \cite{Vilim09} & 2009 & CP 2009 & 15 & \ref{b:Vilim09} & \ref{c:Vilim09}\\
Vilim09a \href{https://doi.org/10.1007/978-3-642-01929-6\_22}{Vilim09a} & \hyperref[auth:a121]{P. Vil{\'{\i}}m} & Max Energy Filtering Algorithm for Discrete Cumulative Resources & \href{works/Vilim09a.pdf}{Yes} & \cite{Vilim09a} & 2009 & CPAIOR 2009 & 15 & \ref{b:Vilim09a} & \ref{c:Vilim09a}\\
Vilim05 \href{https://doi.org/10.1007/11493853\_29}{Vilim05} & \hyperref[auth:a121]{P. Vil{\'{\i}}m} & Computing Explanations for the Unary Resource Constraint & \href{works/Vilim05.pdf}{Yes} & \cite{Vilim05} & 2005 & CPAIOR 2005 & 14 & \ref{b:Vilim05} & \ref{c:Vilim05}\\
VilimBC05 \href{https://doi.org/10.1007/s10601-005-2814-0}{VilimBC05} & \hyperref[auth:a121]{P. Vil{\'{\i}}m}, \hyperref[auth:a152]{R. Bart{\'{a}}k}, \hyperref[auth:a162]{O. Cepek} & Extension of \emph{O}(\emph{n} log \emph{n}) Filtering Algorithms for the Unary Resource Constraint to Optional Activities & \href{works/VilimBC05.pdf}{Yes} & \cite{VilimBC05} & 2005 & Constraints An Int. J. & 23 & \ref{b:VilimBC05} & \ref{c:VilimBC05}\\
Vilim04 \href{https://doi.org/10.1007/978-3-540-24664-0\_23}{Vilim04} & \hyperref[auth:a121]{P. Vil{\'{\i}}m} & O(n log n) Filtering Algorithms for Unary Resource Constraint & \href{works/Vilim04.pdf}{Yes} & \cite{Vilim04} & 2004 & CPAIOR 2004 & 13 & \ref{b:Vilim04} & \ref{c:Vilim04}\\
VilimBC04 \href{https://doi.org/10.1007/978-3-540-30201-8\_8}{VilimBC04} & \hyperref[auth:a121]{P. Vil{\'{\i}}m}, \hyperref[auth:a152]{R. Bart{\'{a}}k}, \hyperref[auth:a162]{O. Cepek} & Unary Resource Constraint with Optional Activities & \href{works/VilimBC04.pdf}{Yes} & \cite{VilimBC04} & 2004 & CP 2004 & 15 & \ref{b:VilimBC04} & \ref{c:VilimBC04}\\
Vilim03 \href{https://doi.org/10.1007/978-3-540-45193-8\_124}{Vilim03} & \hyperref[auth:a121]{P. Vil{\'{\i}}m} & Computing Explanations for Global Scheduling Constraints & \href{works/Vilim03.pdf}{Yes} & \cite{Vilim03} & 2003 & CP 2003 & 1 & \ref{b:Vilim03} & \ref{c:Vilim03}\\
Vilim02 \href{https://doi.org/10.1007/3-540-46135-3\_62}{Vilim02} & \hyperref[auth:a121]{P. Vil{\'{\i}}m} & Batch Processing with Sequence Dependent Setup Times & \href{works/Vilim02.pdf}{Yes} & \cite{Vilim02} & 2002 & CP 2002 & 1 & \ref{b:Vilim02} & \ref{c:Vilim02}\\
\end{longtable}
}

\subsection{Works by Luca Benini}
\label{sec:a247}
{\scriptsize
\begin{longtable}{>{\raggedright\arraybackslash}p{3cm}>{\raggedright\arraybackslash}p{6cm}>{\raggedright\arraybackslash}p{7cm}rrrp{3cm}rrr}
\rowcolor{white}\caption{Works from bibtex (Total 10)}\\ \toprule
\rowcolor{white}Key & Authors & Title & LC & Cite & Year & \shortstack{Conference\\/Journal} & Pages & b & c \\ \midrule\endhead
\bottomrule
\endfoot
BorghesiBLMB18 \href{https://doi.org/10.1016/j.suscom.2018.05.007}{BorghesiBLMB18} & \hyperref[auth:a231]{A. Borghesi}, \hyperref[auth:a230]{A. Bartolini}, \hyperref[auth:a142]{M. Lombardi}, \hyperref[auth:a143]{M. Milano}, \hyperref[auth:a247]{L. Benini} & Scheduling-based power capping in high performance computing systems & \href{works/BorghesiBLMB18.pdf}{Yes} & \cite{BorghesiBLMB18} & 2018 & Sustain. Comput. Informatics Syst. & 13 & \ref{b:BorghesiBLMB18} & \ref{c:BorghesiBLMB18}\\
BridiBLMB16 \href{https://doi.org/10.1109/TPDS.2016.2516997}{BridiBLMB16} & \hyperref[auth:a232]{T. Bridi}, \hyperref[auth:a230]{A. Bartolini}, \hyperref[auth:a142]{M. Lombardi}, \hyperref[auth:a143]{M. Milano}, \hyperref[auth:a247]{L. Benini} & A Constraint Programming Scheduler for Heterogeneous High-Performance Computing Machines & \href{works/BridiBLMB16.pdf}{Yes} & \cite{BridiBLMB16} & 2016 & {IEEE} Trans. Parallel Distributed Syst. & 14 & \ref{b:BridiBLMB16} & \ref{c:BridiBLMB16}\\
BridiLBBM16 \href{https://doi.org/10.3233/978-1-61499-672-9-1598}{BridiLBBM16} & \hyperref[auth:a232]{T. Bridi}, \hyperref[auth:a142]{M. Lombardi}, \hyperref[auth:a230]{A. Bartolini}, \hyperref[auth:a247]{L. Benini}, \hyperref[auth:a143]{M. Milano} & {DARDIS:} Distributed And Randomized DIspatching and Scheduling & \href{works/BridiLBBM16.pdf}{Yes} & \cite{BridiLBBM16} & 2016 & ECAI 2016 & 2 & \ref{b:BridiLBBM16} & \ref{c:BridiLBBM16}\\
BonfiettiLBM14 \href{https://doi.org/10.1016/j.artint.2013.09.006}{BonfiettiLBM14} & \hyperref[auth:a203]{A. Bonfietti}, \hyperref[auth:a142]{M. Lombardi}, \hyperref[auth:a247]{L. Benini}, \hyperref[auth:a143]{M. Milano} & {CROSS} cyclic resource-constrained scheduling solver & \href{works/BonfiettiLBM14.pdf}{Yes} & \cite{BonfiettiLBM14} & 2014 & Artif. Intell. & 28 & \ref{b:BonfiettiLBM14} & \ref{c:BonfiettiLBM14}\\
BonfiettiLBM12 \href{https://doi.org/10.1007/978-3-642-29828-8\_6}{BonfiettiLBM12} & \hyperref[auth:a203]{A. Bonfietti}, \hyperref[auth:a142]{M. Lombardi}, \hyperref[auth:a247]{L. Benini}, \hyperref[auth:a143]{M. Milano} & Global Cyclic Cumulative Constraint & \href{works/BonfiettiLBM12.pdf}{Yes} & \cite{BonfiettiLBM12} & 2012 & CPAIOR 2012 & 16 & \ref{b:BonfiettiLBM12} & \ref{c:BonfiettiLBM12}\\
BeniniLMR11 \href{https://doi.org/10.1007/s10479-010-0718-x}{BeniniLMR11} & \hyperref[auth:a247]{L. Benini}, \hyperref[auth:a142]{M. Lombardi}, \hyperref[auth:a143]{M. Milano}, \hyperref[auth:a727]{M. Ruggiero} & Optimal resource allocation and scheduling for the {CELL} {BE} platform & \href{works/BeniniLMR11.pdf}{Yes} & \cite{BeniniLMR11} & 2011 & Ann. Oper. Res. & 27 & \ref{b:BeniniLMR11} & \ref{c:BeniniLMR11}\\
BonfiettiLBM11 \href{https://doi.org/10.1007/978-3-642-23786-7\_12}{BonfiettiLBM11} & \hyperref[auth:a203]{A. Bonfietti}, \hyperref[auth:a142]{M. Lombardi}, \hyperref[auth:a247]{L. Benini}, \hyperref[auth:a143]{M. Milano} & A Constraint Based Approach to Cyclic {RCPSP} & \href{works/BonfiettiLBM11.pdf}{Yes} & \cite{BonfiettiLBM11} & 2011 & CP 2011 & 15 & \ref{b:BonfiettiLBM11} & \ref{c:BonfiettiLBM11}\\
LombardiBMB11 \href{https://doi.org/10.1007/978-3-642-21311-3\_14}{LombardiBMB11} & \hyperref[auth:a142]{M. Lombardi}, \hyperref[auth:a203]{A. Bonfietti}, \hyperref[auth:a143]{M. Milano}, \hyperref[auth:a247]{L. Benini} & Precedence Constraint Posting for Cyclic Scheduling Problems & \href{works/LombardiBMB11.pdf}{Yes} & \cite{LombardiBMB11} & 2011 & CPAIOR 2011 & 17 & \ref{b:LombardiBMB11} & \ref{c:LombardiBMB11}\\
RuggieroBBMA09 \href{https://doi.org/10.1109/TCAD.2009.2013536}{RuggieroBBMA09} & \hyperref[auth:a727]{M. Ruggiero}, \hyperref[auth:a379]{D. Bertozzi}, \hyperref[auth:a247]{L. Benini}, \hyperref[auth:a143]{M. Milano}, \hyperref[auth:a728]{A. Andrei} & Reducing the Abstraction and Optimality Gaps in the Allocation and Scheduling for Variable Voltage/Frequency MPSoC Platforms & \href{works/RuggieroBBMA09.pdf}{Yes} & \cite{RuggieroBBMA09} & 2009 & {IEEE} Trans. Comput. Aided Des. Integr. Circuits Syst. & 14 & \ref{b:RuggieroBBMA09} & \ref{c:RuggieroBBMA09}\\
BeniniBGM06 \href{https://doi.org/10.1007/11757375\_6}{BeniniBGM06} & \hyperref[auth:a247]{L. Benini}, \hyperref[auth:a379]{D. Bertozzi}, \hyperref[auth:a380]{A. Guerri}, \hyperref[auth:a143]{M. Milano} & Allocation, Scheduling and Voltage Scaling on Energy Aware MPSoCs & \href{works/BeniniBGM06.pdf}{Yes} & \cite{BeniniBGM06} & 2006 & CPAIOR 2006 & 15 & \ref{b:BeniniBGM06} & \ref{c:BeniniBGM06}\\
\end{longtable}
}

\subsection{Works by Alessio Bonfietti}
\label{sec:a203}
{\scriptsize
\begin{longtable}{>{\raggedright\arraybackslash}p{3cm}>{\raggedright\arraybackslash}p{6cm}>{\raggedright\arraybackslash}p{7cm}rrrp{3cm}rrr}
\rowcolor{white}\caption{Works from bibtex (Total 10)}\\ \toprule
\rowcolor{white}Key & Authors & Title & LC & Cite & Year & \shortstack{Conference\\/Journal} & Pages & b & c \\ \midrule\endhead
\bottomrule
\endfoot
Bonfietti16 \href{https://doi.org/10.3233/IA-160095}{Bonfietti16} & \hyperref[auth:a203]{A. Bonfietti} & A constraint programming scheduling solver for the MPOpt programming environment & \href{works/Bonfietti16.pdf}{Yes} & \cite{Bonfietti16} & 2016 & Intelligenza Artificiale & 13 & \ref{b:Bonfietti16} & \ref{c:Bonfietti16}\\
BonfiettiZLM16 \href{https://doi.org/10.1007/978-3-319-44953-1\_8}{BonfiettiZLM16} & \hyperref[auth:a203]{A. Bonfietti}, \hyperref[auth:a204]{A. Zanarini}, \hyperref[auth:a142]{M. Lombardi}, \hyperref[auth:a143]{M. Milano} & The Multirate Resource Constraint & \href{works/BonfiettiZLM16.pdf}{Yes} & \cite{BonfiettiZLM16} & 2016 & CP 2016 & 17 & \ref{b:BonfiettiZLM16} & \ref{c:BonfiettiZLM16}\\
LombardiBM15 \href{https://doi.org/10.1007/978-3-319-23219-5\_20}{LombardiBM15} & \hyperref[auth:a142]{M. Lombardi}, \hyperref[auth:a203]{A. Bonfietti}, \hyperref[auth:a143]{M. Milano} & Deterministic Estimation of the Expected Makespan of a {POS} Under Duration Uncertainty & \href{works/LombardiBM15.pdf}{Yes} & \cite{LombardiBM15} & 2015 & CP 2015 & 16 & \ref{b:LombardiBM15} & \ref{c:LombardiBM15}\\
BonfiettiLBM14 \href{https://doi.org/10.1016/j.artint.2013.09.006}{BonfiettiLBM14} & \hyperref[auth:a203]{A. Bonfietti}, \hyperref[auth:a142]{M. Lombardi}, \hyperref[auth:a247]{L. Benini}, \hyperref[auth:a143]{M. Milano} & {CROSS} cyclic resource-constrained scheduling solver & \href{works/BonfiettiLBM14.pdf}{Yes} & \cite{BonfiettiLBM14} & 2014 & Artif. Intell. & 28 & \ref{b:BonfiettiLBM14} & \ref{c:BonfiettiLBM14}\\
BonfiettiLM14 \href{https://doi.org/10.1007/978-3-319-07046-9\_15}{BonfiettiLM14} & \hyperref[auth:a203]{A. Bonfietti}, \hyperref[auth:a142]{M. Lombardi}, \hyperref[auth:a143]{M. Milano} & Disregarding Duration Uncertainty in Partial Order Schedules? Yes, We Can! & \href{works/BonfiettiLM14.pdf}{Yes} & \cite{BonfiettiLM14} & 2014 & CPAIOR 2014 & 16 & \ref{b:BonfiettiLM14} & \ref{c:BonfiettiLM14}\\
BonfiettiLM13 \href{http://www.aaai.org/ocs/index.php/ICAPS/ICAPS13/paper/view/6050}{BonfiettiLM13} & \hyperref[auth:a203]{A. Bonfietti}, \hyperref[auth:a142]{M. Lombardi}, \hyperref[auth:a143]{M. Milano} & De-Cycling Cyclic Scheduling Problems & \href{works/BonfiettiLM13.pdf}{Yes} & \cite{BonfiettiLM13} & 2013 & ICAPS 2013 & 5 & \ref{b:BonfiettiLM13} & \ref{c:BonfiettiLM13}\\
BonfiettiLBM12 \href{https://doi.org/10.1007/978-3-642-29828-8\_6}{BonfiettiLBM12} & \hyperref[auth:a203]{A. Bonfietti}, \hyperref[auth:a142]{M. Lombardi}, \hyperref[auth:a247]{L. Benini}, \hyperref[auth:a143]{M. Milano} & Global Cyclic Cumulative Constraint & \href{works/BonfiettiLBM12.pdf}{Yes} & \cite{BonfiettiLBM12} & 2012 & CPAIOR 2012 & 16 & \ref{b:BonfiettiLBM12} & \ref{c:BonfiettiLBM12}\\
BonfiettiM12 \href{https://ceur-ws.org/Vol-926/paper2.pdf}{BonfiettiM12} & \hyperref[auth:a203]{A. Bonfietti}, \hyperref[auth:a143]{M. Milano} & A Constraint-based Approach to Cyclic Resource-Constrained Scheduling Problem & \href{works/BonfiettiM12.pdf}{Yes} & \cite{BonfiettiM12} & 2012 & DC SIAAI 2012 & 3 & \ref{b:BonfiettiM12} & \ref{c:BonfiettiM12}\\
BonfiettiLBM11 \href{https://doi.org/10.1007/978-3-642-23786-7\_12}{BonfiettiLBM11} & \hyperref[auth:a203]{A. Bonfietti}, \hyperref[auth:a142]{M. Lombardi}, \hyperref[auth:a247]{L. Benini}, \hyperref[auth:a143]{M. Milano} & A Constraint Based Approach to Cyclic {RCPSP} & \href{works/BonfiettiLBM11.pdf}{Yes} & \cite{BonfiettiLBM11} & 2011 & CP 2011 & 15 & \ref{b:BonfiettiLBM11} & \ref{c:BonfiettiLBM11}\\
LombardiBMB11 \href{https://doi.org/10.1007/978-3-642-21311-3\_14}{LombardiBMB11} & \hyperref[auth:a142]{M. Lombardi}, \hyperref[auth:a203]{A. Bonfietti}, \hyperref[auth:a143]{M. Milano}, \hyperref[auth:a247]{L. Benini} & Precedence Constraint Posting for Cyclic Scheduling Problems & \href{works/LombardiBMB11.pdf}{Yes} & \cite{LombardiBMB11} & 2011 & CPAIOR 2011 & 17 & \ref{b:LombardiBMB11} & \ref{c:LombardiBMB11}\\
\end{longtable}
}

\subsection{Works by Philippe Laborie}
\label{sec:a118}
{\scriptsize
\begin{longtable}{>{\raggedright\arraybackslash}p{3cm}>{\raggedright\arraybackslash}p{6cm}>{\raggedright\arraybackslash}p{7cm}rrrp{3cm}rrr}
\rowcolor{white}\caption{Works from bibtex (Total 10)}\\ \toprule
\rowcolor{white}Key & Authors & Title & LC & Cite & Year & \shortstack{Conference\\/Journal} & Pages & b & c \\ \midrule\endhead
\bottomrule
\endfoot
LunardiBLRV20 \href{https://doi.org/10.1016/j.cor.2020.105020}{LunardiBLRV20} & \hyperref[auth:a510]{Willian T. Lunardi}, \hyperref[auth:a511]{Ernesto G. Birgin}, \hyperref[auth:a118]{P. Laborie}, \hyperref[auth:a512]{D{\'{e}}bora P. Ronconi}, \hyperref[auth:a513]{H. Voos} & Mixed Integer linear programming and constraint programming models for the online printing shop scheduling problem & \href{works/LunardiBLRV20.pdf}{Yes} & \cite{LunardiBLRV20} & 2020 & Comput. Oper. Res. & 20 & \ref{b:LunardiBLRV20} & \ref{c:LunardiBLRV20}\\
Laborie18a \href{https://doi.org/10.1007/978-3-319-93031-2\_29}{Laborie18a} & \hyperref[auth:a118]{P. Laborie} & An Update on the Comparison of MIP, {CP} and Hybrid Approaches for Mixed Resource Allocation and Scheduling & \href{works/Laborie18a.pdf}{Yes} & \cite{Laborie18a} & 2018 & CPAIOR 2018 & 9 & \ref{b:Laborie18a} & \ref{c:Laborie18a}\\
LaborieRSV18 \href{https://doi.org/10.1007/s10601-018-9281-x}{LaborieRSV18} & \hyperref[auth:a118]{P. Laborie}, \hyperref[auth:a119]{J. Rogerie}, \hyperref[auth:a120]{P. Shaw}, \hyperref[auth:a121]{P. Vil{\'{\i}}m} & {IBM} {ILOG} {CP} optimizer for scheduling - 20+ years of scheduling with constraints at {IBM/ILOG} & \href{works/LaborieRSV18.pdf}{Yes} & \cite{LaborieRSV18} & 2018 & Constraints An Int. J. & 41 & \ref{b:LaborieRSV18} & \ref{c:LaborieRSV18}\\
MelgarejoLS15 \href{https://doi.org/10.1007/978-3-319-18008-3\_1}{MelgarejoLS15} & \hyperref[auth:a324]{P. Aguiar{-}Melgarejo}, \hyperref[auth:a118]{P. Laborie}, \hyperref[auth:a85]{C. Solnon} & A Time-Dependent No-Overlap Constraint: Application to Urban Delivery Problems & \href{works/MelgarejoLS15.pdf}{Yes} & \cite{MelgarejoLS15} & 2015 & CPAIOR 2015 & 17 & \ref{b:MelgarejoLS15} & \ref{c:MelgarejoLS15}\\
VilimLS15 \href{https://doi.org/10.1007/978-3-319-18008-3\_30}{VilimLS15} & \hyperref[auth:a121]{P. Vil{\'{\i}}m}, \hyperref[auth:a118]{P. Laborie}, \hyperref[auth:a120]{P. Shaw} & Failure-Directed Search for Constraint-Based Scheduling & \href{works/VilimLS15.pdf}{Yes} & \cite{VilimLS15} & 2015 & CPAIOR 2015 & 17 & \ref{b:VilimLS15} & \ref{c:VilimLS15}\\
BidotVLB09 \href{https://doi.org/10.1007/s10951-008-0080-x}{BidotVLB09} & \hyperref[auth:a835]{J. Bidot}, \hyperref[auth:a836]{T. Vidal}, \hyperref[auth:a118]{P. Laborie}, \hyperref[auth:a89]{J. Christopher Beck} & A theoretic and practical framework for scheduling in a stochastic environment & \href{works/BidotVLB09.pdf}{Yes} & \cite{BidotVLB09} & 2009 & J. Sched. & 30 & \ref{b:BidotVLB09} & \ref{c:BidotVLB09}\\
Laborie09 \href{https://doi.org/10.1007/978-3-642-01929-6\_12}{Laborie09} & \hyperref[auth:a118]{P. Laborie} & {IBM} {ILOG} {CP} Optimizer for Detailed Scheduling Illustrated on Three Problems & \href{works/Laborie09.pdf}{Yes} & \cite{Laborie09} & 2009 & CPAIOR 2009 & 15 & \ref{b:Laborie09} & \ref{c:Laborie09}\\
BaptisteLPN06 \href{https://doi.org/10.1016/S1574-6526(06)80026-X}{BaptisteLPN06} & \hyperref[auth:a163]{P. Baptiste}, \hyperref[auth:a118]{P. Laborie}, \hyperref[auth:a164]{Claude Le Pape}, \hyperref[auth:a666]{W. Nuijten} & Constraint-Based Scheduling and Planning & No & \cite{BaptisteLPN06} & 2006 & n/a & 39 & No & \ref{c:BaptisteLPN06}\\
GodardLN05 \href{http://www.aaai.org/Library/ICAPS/2005/icaps05-009.php}{GodardLN05} & \hyperref[auth:a782]{D. Godard}, \hyperref[auth:a118]{P. Laborie}, \hyperref[auth:a666]{W. Nuijten} & Randomized Large Neighborhood Search for Cumulative Scheduling & \href{works/GodardLN05.pdf}{Yes} & \cite{GodardLN05} & 2005 & ICAPS 2005 & 9 & \ref{b:GodardLN05} & \ref{c:GodardLN05}\\
FocacciLN00 \href{http://www.aaai.org/Library/AIPS/2000/aips00-010.php}{FocacciLN00} & \hyperref[auth:a784]{F. Focacci}, \hyperref[auth:a118]{P. Laborie}, \hyperref[auth:a666]{W. Nuijten} & Solving Scheduling Problems with Setup Times and Alternative Resources & \href{works/FocacciLN00.pdf}{Yes} & \cite{FocacciLN00} & 2000 & AIPS 2000 & 10 & \ref{b:FocacciLN00} & \ref{c:FocacciLN00}\\
\end{longtable}
}

\subsection{Works by Nysret Musliu}
\label{sec:a45}
{\scriptsize
\begin{longtable}{>{\raggedright\arraybackslash}p{3cm}>{\raggedright\arraybackslash}p{6cm}>{\raggedright\arraybackslash}p{7cm}rrrp{3cm}rrr}
\rowcolor{white}\caption{Works from bibtex (Total 9)}\\ \toprule
\rowcolor{white}Key & Authors & Title & LC & Cite & Year & \shortstack{Conference\\/Journal} & Pages & b & c \\ \midrule\endhead
\bottomrule
\endfoot
LacknerMMWW23 \href{https://doi.org/10.1007/s10601-023-09347-2}{LacknerMMWW23} & \hyperref[auth:a62]{M. Lackner}, \hyperref[auth:a63]{C. Mrkvicka}, \hyperref[auth:a45]{N. Musliu}, \hyperref[auth:a46]{D. Walkiewicz}, \hyperref[auth:a43]{F. Winter} & Exact methods for the Oven Scheduling Problem & \href{works/LacknerMMWW23.pdf}{Yes} & \cite{LacknerMMWW23} & 2023 & Constraints An Int. J. & 42 & \ref{b:LacknerMMWW23} & \ref{c:LacknerMMWW23}\\
WinterMMW22 \href{https://doi.org/10.4230/LIPIcs.CP.2022.41}{WinterMMW22} & \hyperref[auth:a43]{F. Winter}, \hyperref[auth:a44]{S. Meiswinkel}, \hyperref[auth:a45]{N. Musliu}, \hyperref[auth:a46]{D. Walkiewicz} & Modeling and Solving Parallel Machine Scheduling with Contamination Constraints in the Agricultural Industry & \href{works/WinterMMW22.pdf}{Yes} & \cite{WinterMMW22} & 2022 & CP 2022 & 18 & \ref{b:WinterMMW22} & \ref{c:WinterMMW22}\\
GeibingerKKMMW21 \href{https://doi.org/10.1007/978-3-030-78230-6\_29}{GeibingerKKMMW21} & \hyperref[auth:a77]{T. Geibinger}, \hyperref[auth:a78]{L. Kletzander}, \hyperref[auth:a79]{M. Krainz}, \hyperref[auth:a80]{F. Mischek}, \hyperref[auth:a45]{N. Musliu}, \hyperref[auth:a43]{F. Winter} & Physician Scheduling During a Pandemic & \href{works/GeibingerKKMMW21.pdf}{Yes} & \cite{GeibingerKKMMW21} & 2021 & CPAIOR 2021 & 10 & \ref{b:GeibingerKKMMW21} & \ref{c:GeibingerKKMMW21}\\
GeibingerMM21 \href{https://doi.org/10.1609/aaai.v35i7.16789}{GeibingerMM21} & \hyperref[auth:a77]{T. Geibinger}, \hyperref[auth:a80]{F. Mischek}, \hyperref[auth:a45]{N. Musliu} & Constraint Logic Programming for Real-World Test Laboratory Scheduling & \href{works/GeibingerMM21.pdf}{Yes} & \cite{GeibingerMM21} & 2021 & AAAI 2021 & 9 & \ref{b:GeibingerMM21} & \ref{c:GeibingerMM21}\\
LacknerMMWW21 \href{https://doi.org/10.4230/LIPIcs.CP.2021.37}{LacknerMMWW21} & \hyperref[auth:a62]{M. Lackner}, \hyperref[auth:a63]{C. Mrkvicka}, \hyperref[auth:a45]{N. Musliu}, \hyperref[auth:a46]{D. Walkiewicz}, \hyperref[auth:a43]{F. Winter} & Minimizing Cumulative Batch Processing Time for an Industrial Oven Scheduling Problem & \href{works/LacknerMMWW21.pdf}{Yes} & \cite{LacknerMMWW21} & 2021 & CP 2021 & 18 & \ref{b:LacknerMMWW21} & \ref{c:LacknerMMWW21}\\
GeibingerMM19 \href{https://doi.org/10.1007/978-3-030-19212-9\_20}{GeibingerMM19} & \hyperref[auth:a77]{T. Geibinger}, \hyperref[auth:a80]{F. Mischek}, \hyperref[auth:a45]{N. Musliu} & Investigating Constraint Programming for Real World Industrial Test Laboratory Scheduling & \href{works/GeibingerMM19.pdf}{Yes} & \cite{GeibingerMM19} & 2019 & CPAIOR 2019 & 16 & \ref{b:GeibingerMM19} & \ref{c:GeibingerMM19}\\
abs-1911-04766 \href{http://arxiv.org/abs/1911.04766}{abs-1911-04766} & \hyperref[auth:a77]{T. Geibinger}, \hyperref[auth:a80]{F. Mischek}, \hyperref[auth:a45]{N. Musliu} & Investigating Constraint Programming and Hybrid Methods for Real World Industrial Test Laboratory Scheduling & \href{works/abs-1911-04766.pdf}{Yes} & \cite{abs-1911-04766} & 2019 & CoRR & 16 & \ref{b:abs-1911-04766} & \ref{c:abs-1911-04766}\\
MusliuSS18 \href{https://doi.org/10.1007/978-3-319-93031-2\_31}{MusliuSS18} & \hyperref[auth:a45]{N. Musliu}, \hyperref[auth:a124]{A. Schutt}, \hyperref[auth:a125]{Peter J. Stuckey} & Solver Independent Rotating Workforce Scheduling & \href{works/MusliuSS18.pdf}{Yes} & \cite{MusliuSS18} & 2018 & CPAIOR 2018 & 17 & \ref{b:MusliuSS18} & \ref{c:MusliuSS18}\\
KletzanderM17 \href{https://doi.org/10.1007/978-3-319-59776-8\_28}{KletzanderM17} & \hyperref[auth:a78]{L. Kletzander}, \hyperref[auth:a45]{N. Musliu} & A Multi-stage Simulated Annealing Algorithm for the Torpedo Scheduling Problem & \href{works/KletzanderM17.pdf}{Yes} & \cite{KletzanderM17} & 2017 & CPAIOR 2017 & 15 & \ref{b:KletzanderM17} & \ref{c:KletzanderM17}\\
\end{longtable}
}

\subsection{Works by John N. Hooker}
\label{sec:a161}
{\scriptsize
\begin{longtable}{>{\raggedright\arraybackslash}p{3cm}>{\raggedright\arraybackslash}p{6cm}>{\raggedright\arraybackslash}p{7cm}rrrp{3cm}rrr}
\rowcolor{white}\caption{Works from bibtex (Total 9)}\\ \toprule
\rowcolor{white}Key & Authors & Title & LC & Cite & Year & \shortstack{Conference\\/Journal} & Pages & b & c \\ \midrule\endhead
\bottomrule
\endfoot
Hooker17 \href{https://doi.org/10.1007/978-3-319-66158-2\_36}{Hooker17} & \hyperref[auth:a161]{John N. Hooker} & Job Sequencing Bounds from Decision Diagrams & \href{works/Hooker17.pdf}{Yes} & \cite{Hooker17} & 2017 & CP 2017 & 14 & \ref{b:Hooker17} & \ref{c:Hooker17}\\
HechingH16 \href{https://doi.org/10.1007/978-3-319-33954-2\_14}{HechingH16} & \hyperref[auth:a322]{Aliza R. Heching}, \hyperref[auth:a161]{John N. Hooker} & Scheduling Home Hospice Care with Logic-Based Benders Decomposition & \href{works/HechingH16.pdf}{Yes} & \cite{HechingH16} & 2016 & CPAIOR 2016 & 11 & \ref{b:HechingH16} & \ref{c:HechingH16}\\
CireCH13 \href{https://doi.org/10.1007/978-3-642-38171-3\_22}{CireCH13} & \hyperref[auth:a158]{Andr{\'{e}} A. Cir{\'{e}}}, \hyperref[auth:a340]{E. Coban}, \hyperref[auth:a161]{John N. Hooker} & Mixed Integer Programming vs. Logic-Based Benders Decomposition for Planning and Scheduling & \href{works/CireCH13.pdf}{Yes} & \cite{CireCH13} & 2013 & CPAIOR 2013 & 7 & \ref{b:CireCH13} & \ref{c:CireCH13}\\
CobanH10 \href{https://doi.org/10.1007/978-3-642-13520-0\_11}{CobanH10} & \hyperref[auth:a340]{E. Coban}, \hyperref[auth:a161]{John N. Hooker} & Single-Facility Scheduling over Long Time Horizons by Logic-Based Benders Decomposition & \href{works/CobanH10.pdf}{Yes} & \cite{CobanH10} & 2010 & CPAIOR 2010 & 5 & \ref{b:CobanH10} & \ref{c:CobanH10}\\
Hooker06 \href{https://doi.org/10.1007/s10601-006-8060-2}{Hooker06} & \hyperref[auth:a161]{John N. Hooker} & An Integrated Method for Planning and Scheduling to Minimize Tardiness & \href{works/Hooker06.pdf}{Yes} & \cite{Hooker06} & 2006 & Constraints An Int. J. & 19 & \ref{b:Hooker06} & \ref{c:Hooker06}\\
Hooker05 \href{https://doi.org/10.1007/s10601-005-2812-2}{Hooker05} & \hyperref[auth:a161]{John N. Hooker} & A Hybrid Method for the Planning and Scheduling & \href{works/Hooker05.pdf}{Yes} & \cite{Hooker05} & 2005 & Constraints An Int. J. & 17 & \ref{b:Hooker05} & \ref{c:Hooker05}\\
Hooker05a \href{https://doi.org/10.1007/11564751\_25}{Hooker05a} & \hyperref[auth:a161]{John N. Hooker} & Planning and Scheduling to Minimize Tardiness & \href{works/Hooker05a.pdf}{Yes} & \cite{Hooker05a} & 2005 & CP 2005 & 14 & \ref{b:Hooker05a} & \ref{c:Hooker05a}\\
Hooker04 \href{https://doi.org/10.1007/978-3-540-30201-8\_24}{Hooker04} & \hyperref[auth:a161]{John N. Hooker} & A Hybrid Method for Planning and Scheduling & \href{works/Hooker04.pdf}{Yes} & \cite{Hooker04} & 2004 & CP 2004 & 12 & \ref{b:Hooker04} & \ref{c:Hooker04}\\
HookerY02 \href{https://doi.org/10.1007/3-540-46135-3\_46}{HookerY02} & \hyperref[auth:a161]{John N. Hooker}, \hyperref[auth:a293]{H. Yan} & A Relaxation of the Cumulative Constraint & \href{works/HookerY02.pdf}{Yes} & \cite{HookerY02} & 2002 & CP 2002 & 5 & \ref{b:HookerY02} & \ref{c:HookerY02}\\
\end{longtable}
}

\subsection{Works by Claude{-}Guy Quimper}
\label{sec:a37}
{\scriptsize
\begin{longtable}{>{\raggedright\arraybackslash}p{3cm}>{\raggedright\arraybackslash}p{6cm}>{\raggedright\arraybackslash}p{7cm}rrrp{3cm}rrr}
\rowcolor{white}\caption{Works from bibtex (Total 9)}\\ \toprule
\rowcolor{white}Key & Authors & Title & LC & Cite & Year & \shortstack{Conference\\/Journal} & Pages & b & c \\ \midrule\endhead
\bottomrule
\endfoot
BoudreaultSLQ22 \href{https://doi.org/10.4230/LIPIcs.CP.2022.10}{BoudreaultSLQ22} & \hyperref[auth:a34]{R. Boudreault}, \hyperref[auth:a35]{V. Simard}, \hyperref[auth:a36]{D. Lafond}, \hyperref[auth:a37]{C. Quimper} & A Constraint Programming Approach to Ship Refit Project Scheduling & \href{works/BoudreaultSLQ22.pdf}{Yes} & \cite{BoudreaultSLQ22} & 2022 & CP 2022 & 16 & \ref{b:BoudreaultSLQ22} & \ref{c:BoudreaultSLQ22}\\
OuelletQ22 \href{https://doi.org/10.1007/978-3-031-08011-1\_21}{OuelletQ22} & \hyperref[auth:a52]{Y. Ouellet}, \hyperref[auth:a37]{C. Quimper} & A MinCumulative Resource Constraint & \href{works/OuelletQ22.pdf}{Yes} & \cite{OuelletQ22} & 2022 & CPAIOR 2022 & 17 & \ref{b:OuelletQ22} & \ref{c:OuelletQ22}\\
Mercier-AubinGQ20 \href{https://doi.org/10.1007/978-3-030-58942-4\_22}{Mercier-AubinGQ20} & \hyperref[auth:a86]{A. Mercier{-}Aubin}, \hyperref[auth:a87]{J. Gaudreault}, \hyperref[auth:a37]{C. Quimper} & Leveraging Constraint Scheduling: {A} Case Study to the Textile Industry & \href{works/Mercier-AubinGQ20.pdf}{Yes} & \cite{Mercier-AubinGQ20} & 2020 & CPAIOR 2020 & 13 & \ref{b:Mercier-AubinGQ20} & \ref{c:Mercier-AubinGQ20}\\
FahimiOQ18 \href{https://doi.org/10.1007/s10601-018-9282-9}{FahimiOQ18} & \hyperref[auth:a122]{H. Fahimi}, \hyperref[auth:a52]{Y. Ouellet}, \hyperref[auth:a37]{C. Quimper} & Linear-time filtering algorithms for the disjunctive constraint and a quadratic filtering algorithm for the cumulative not-first not-last & \href{works/FahimiOQ18.pdf}{Yes} & \cite{FahimiOQ18} & 2018 & Constraints An Int. J. & 22 & \ref{b:FahimiOQ18} & \ref{c:FahimiOQ18}\\
KameugneFGOQ18 \href{https://doi.org/10.1007/978-3-319-93031-2\_23}{KameugneFGOQ18} & \hyperref[auth:a10]{R. Kameugne}, \hyperref[auth:a11]{S{\'{e}}v{\'{e}}rine Betmbe Fetgo}, \hyperref[auth:a315]{V. Gingras}, \hyperref[auth:a52]{Y. Ouellet}, \hyperref[auth:a37]{C. Quimper} & Horizontally Elastic Not-First/Not-Last Filtering Algorithm for Cumulative Resource Constraint & \href{works/KameugneFGOQ18.pdf}{Yes} & \cite{KameugneFGOQ18} & 2018 & CPAIOR 2018 & 17 & \ref{b:KameugneFGOQ18} & \ref{c:KameugneFGOQ18}\\
OuelletQ18 \href{https://doi.org/10.1007/978-3-319-93031-2\_34}{OuelletQ18} & \hyperref[auth:a52]{Y. Ouellet}, \hyperref[auth:a37]{C. Quimper} & A O(n {\textbackslash}log {\^{}}2 n) Checker and O(n{\^{}}2 {\textbackslash}log n) Filtering Algorithm for the Energetic Reasoning & \href{works/OuelletQ18.pdf}{Yes} & \cite{OuelletQ18} & 2018 & CPAIOR 2018 & 18 & \ref{b:OuelletQ18} & \ref{c:OuelletQ18}\\
GingrasQ16 \href{http://www.ijcai.org/Abstract/16/440}{GingrasQ16} & \hyperref[auth:a315]{V. Gingras}, \hyperref[auth:a37]{C. Quimper} & Generalizing the Edge-Finder Rule for the Cumulative Constraint & \href{works/GingrasQ16.pdf}{Yes} & \cite{GingrasQ16} & 2016 & IJCAI 2016 & 7 & \ref{b:GingrasQ16} & \ref{c:GingrasQ16}\\
BessiereHMQW14 \href{https://doi.org/10.1007/978-3-319-07046-9\_23}{BessiereHMQW14} & \hyperref[auth:a333]{C. Bessiere}, \hyperref[auth:a1]{E. Hebrard}, \hyperref[auth:a334]{M. M{\'{e}}nard}, \hyperref[auth:a37]{C. Quimper}, \hyperref[auth:a278]{T. Walsh} & Buffered Resource Constraint: Algorithms and Complexity & \href{works/BessiereHMQW14.pdf}{Yes} & \cite{BessiereHMQW14} & 2014 & CPAIOR 2014 & 16 & \ref{b:BessiereHMQW14} & \ref{c:BessiereHMQW14}\\
OuelletQ13 \href{https://doi.org/10.1007/978-3-642-40627-0\_42}{OuelletQ13} & \hyperref[auth:a240]{P. Ouellet}, \hyperref[auth:a37]{C. Quimper} & Time-Table Extended-Edge-Finding for the Cumulative Constraint & \href{works/OuelletQ13.pdf}{Yes} & \cite{OuelletQ13} & 2013 & CP 2013 & 16 & \ref{b:OuelletQ13} & \ref{c:OuelletQ13}\\
\end{longtable}
}

\subsection{Works by Pierre Schaus}
\label{sec:a147}
{\scriptsize
\begin{longtable}{>{\raggedright\arraybackslash}p{3cm}>{\raggedright\arraybackslash}p{6cm}>{\raggedright\arraybackslash}p{7cm}rrrp{3cm}rrr}
\rowcolor{white}\caption{Works from bibtex (Total 9)}\\ \toprule
\rowcolor{white}Key & Authors & Title & LC & Cite & Year & \shortstack{Conference\\/Journal} & Pages & b & c \\ \midrule\endhead
\bottomrule
\endfoot
CappartS17 \href{https://doi.org/10.1007/978-3-319-59776-8\_26}{CappartS17} & \hyperref[auth:a42]{Q. Cappart}, \hyperref[auth:a147]{P. Schaus} & Rescheduling Railway Traffic on Real Time Situations Using Time-Interval Variables & \href{works/CappartS17.pdf}{Yes} & \cite{CappartS17} & 2017 & CPAIOR 2017 & 16 & \ref{b:CappartS17} & \ref{c:CappartS17}\\
CauwelaertDMS16 \href{https://doi.org/10.1007/978-3-319-44953-1\_33}{CauwelaertDMS16} & \hyperref[auth:a206]{Sascha Van Cauwelaert}, \hyperref[auth:a207]{C. Dejemeppe}, \hyperref[auth:a149]{J. Monette}, \hyperref[auth:a147]{P. Schaus} & Efficient Filtering for the Unary Resource with Family-Based Transition Times & \href{works/CauwelaertDMS16.pdf}{Yes} & \cite{CauwelaertDMS16} & 2016 & CP 2016 & 16 & \ref{b:CauwelaertDMS16} & \ref{c:CauwelaertDMS16}\\
DejemeppeCS15 \href{https://doi.org/10.1007/978-3-319-23219-5\_7}{DejemeppeCS15} & \hyperref[auth:a207]{C. Dejemeppe}, \hyperref[auth:a206]{Sascha Van Cauwelaert}, \hyperref[auth:a147]{P. Schaus} & The Unary Resource with Transition Times & \href{works/DejemeppeCS15.pdf}{Yes} & \cite{DejemeppeCS15} & 2015 & CP 2015 & 16 & \ref{b:DejemeppeCS15} & \ref{c:DejemeppeCS15}\\
GayHLS15 \href{https://doi.org/10.1007/978-3-319-23219-5\_10}{GayHLS15} & \hyperref[auth:a216]{S. Gay}, \hyperref[auth:a217]{R. Hartert}, \hyperref[auth:a218]{C. Lecoutre}, \hyperref[auth:a147]{P. Schaus} & Conflict Ordering Search for Scheduling Problems & \href{works/GayHLS15.pdf}{Yes} & \cite{GayHLS15} & 2015 & CP 2015 & 9 & \ref{b:GayHLS15} & \ref{c:GayHLS15}\\
GayHS15 \href{https://doi.org/10.1007/978-3-319-23219-5\_11}{GayHS15} & \hyperref[auth:a216]{S. Gay}, \hyperref[auth:a217]{R. Hartert}, \hyperref[auth:a147]{P. Schaus} & Simple and Scalable Time-Table Filtering for the Cumulative Constraint & \href{works/GayHS15.pdf}{Yes} & \cite{GayHS15} & 2015 & CP 2015 & 9 & \ref{b:GayHS15} & \ref{c:GayHS15}\\
GayHS15a \href{https://doi.org/10.1007/978-3-319-18008-3\_11}{GayHS15a} & \hyperref[auth:a216]{S. Gay}, \hyperref[auth:a217]{R. Hartert}, \hyperref[auth:a147]{P. Schaus} & Time-Table Disjunctive Reasoning for the Cumulative Constraint & \href{works/GayHS15a.pdf}{Yes} & \cite{GayHS15a} & 2015 & CPAIOR 2015 & 16 & \ref{b:GayHS15a} & \ref{c:GayHS15a}\\
GaySS14 \href{https://doi.org/10.1007/978-3-319-10428-7\_59}{GaySS14} & \hyperref[auth:a216]{S. Gay}, \hyperref[auth:a147]{P. Schaus}, \hyperref[auth:a239]{Vivian De Smedt} & Continuous Casting Scheduling with Constraint Programming & \href{works/GaySS14.pdf}{Yes} & \cite{GaySS14} & 2014 & CP 2014 & 15 & \ref{b:GaySS14} & \ref{c:GaySS14}\\
HoundjiSWD14 \href{https://doi.org/10.1007/978-3-319-10428-7\_29}{HoundjiSWD14} & \hyperref[auth:a228]{Vinas{\'{e}}tan Ratheil Houndji}, \hyperref[auth:a147]{P. Schaus}, \hyperref[auth:a229]{Laurence A. Wolsey}, \hyperref[auth:a151]{Y. Deville} & The StockingCost Constraint & \href{works/HoundjiSWD14.pdf}{Yes} & \cite{HoundjiSWD14} & 2014 & CP 2014 & 16 & \ref{b:HoundjiSWD14} & \ref{c:HoundjiSWD14}\\
SchausHMCMD11 \href{https://doi.org/10.1007/s10601-010-9100-5}{SchausHMCMD11} & \hyperref[auth:a147]{P. Schaus}, \hyperref[auth:a148]{Pascal Van Hentenryck}, \hyperref[auth:a149]{J. Monette}, \hyperref[auth:a150]{C. Coffrin}, \hyperref[auth:a32]{L. Michel}, \hyperref[auth:a151]{Y. Deville} & Solving Steel Mill Slab Problems with constraint-based techniques: CP, LNS, and {CBLS} & \href{works/SchausHMCMD11.pdf}{Yes} & \cite{SchausHMCMD11} & 2011 & Constraints An Int. J. & 23 & \ref{b:SchausHMCMD11} & \ref{c:SchausHMCMD11}\\
\end{longtable}
}

\subsection{Works by Tony T. Tran}
\label{sec:a810}
{\scriptsize
\begin{longtable}{>{\raggedright\arraybackslash}p{3cm}>{\raggedright\arraybackslash}p{6cm}>{\raggedright\arraybackslash}p{7cm}rrrp{3cm}rrr}
\rowcolor{white}\caption{Works from bibtex (Total 9)}\\ \toprule
\rowcolor{white}Key & Authors & Title & LC & Cite & Year & \shortstack{Conference\\/Journal} & Pages & b & c \\ \midrule\endhead
\bottomrule
\endfoot
TranPZLDB18 \href{https://doi.org/10.1007/s10951-017-0537-x}{TranPZLDB18} & \hyperref[auth:a810]{Tony T. Tran}, \hyperref[auth:a811]{M. Padmanabhan}, \hyperref[auth:a812]{Peter Yun Zhang}, \hyperref[auth:a813]{H. Li}, \hyperref[auth:a814]{Douglas G. Down}, \hyperref[auth:a89]{J. Christopher Beck} & Multi-stage resource-aware scheduling for data centers with heterogeneous servers & \href{works/TranPZLDB18.pdf}{Yes} & \cite{TranPZLDB18} & 2018 & J. Sched. & 17 & \ref{b:TranPZLDB18} & \ref{c:TranPZLDB18}\\
TranVNB17 \href{https://doi.org/10.1613/jair.5306}{TranVNB17} & \hyperref[auth:a810]{Tony T. Tran}, \hyperref[auth:a815]{Tiago Stegun Vaquero}, \hyperref[auth:a209]{G. Nejat}, \hyperref[auth:a89]{J. Christopher Beck} & Robots in Retirement Homes: Applying Off-the-Shelf Planning and Scheduling to a Team of Assistive Robots & \href{works/TranVNB17.pdf}{Yes} & \cite{TranVNB17} & 2017 & J. Artif. Intell. Res. & 68 & \ref{b:TranVNB17} & \ref{c:TranVNB17}\\
TranVNB17a \href{https://doi.org/10.24963/ijcai.2017/726}{TranVNB17a} & \hyperref[auth:a810]{Tony T. Tran}, \hyperref[auth:a815]{Tiago Stegun Vaquero}, \hyperref[auth:a209]{G. Nejat}, \hyperref[auth:a89]{J. Christopher Beck} & Robots in Retirement Homes: Applying Off-the-Shelf Planning and Scheduling to a Team of Assistive Robots (Extended Abstract) & \href{works/TranVNB17a.pdf}{Yes} & \cite{TranVNB17a} & 2017 & IJCAI 2017 & 5 & \ref{b:TranVNB17a} & \ref{c:TranVNB17a}\\
TranAB16 \href{https://doi.org/10.1287/ijoc.2015.0666}{TranAB16} & \hyperref[auth:a810]{Tony T. Tran}, \hyperref[auth:a818]{A. Araujo}, \hyperref[auth:a89]{J. Christopher Beck} & Decomposition Methods for the Parallel Machine Scheduling Problem with Setups & No & \cite{TranAB16} & 2016 & {INFORMS} J. Comput. & 13 & No & \ref{c:TranAB16}\\
TranDRFWOVB16 \href{https://doi.org/10.1609/socs.v7i1.18390}{TranDRFWOVB16} & \hyperref[auth:a810]{Tony T. Tran}, \hyperref[auth:a820]{M. Do}, \hyperref[auth:a821]{Eleanor Gilbert Rieffel}, \hyperref[auth:a383]{J. Frank}, \hyperref[auth:a819]{Z. Wang}, \hyperref[auth:a822]{B. O'Gorman}, \hyperref[auth:a823]{D. Venturelli}, \hyperref[auth:a89]{J. Christopher Beck} & A Hybrid Quantum-Classical Approach to Solving Scheduling Problems & \href{works/TranDRFWOVB16.pdf}{Yes} & \cite{TranDRFWOVB16} & 2016 & SOCS 2016 & 9 & \ref{b:TranDRFWOVB16} & \ref{c:TranDRFWOVB16}\\
TranWDRFOVB16 \href{http://www.aaai.org/ocs/index.php/WS/AAAIW16/paper/view/12664}{TranWDRFOVB16} & \hyperref[auth:a810]{Tony T. Tran}, \hyperref[auth:a819]{Z. Wang}, \hyperref[auth:a820]{M. Do}, \hyperref[auth:a821]{Eleanor Gilbert Rieffel}, \hyperref[auth:a383]{J. Frank}, \hyperref[auth:a822]{B. O'Gorman}, \hyperref[auth:a823]{D. Venturelli}, \hyperref[auth:a89]{J. Christopher Beck} & Explorations of Quantum-Classical Approaches to Scheduling a Mars Lander Activity Problem & \href{works/TranWDRFOVB16.pdf}{Yes} & \cite{TranWDRFOVB16} & 2016 & AAAI 2016 & 9 & \ref{b:TranWDRFOVB16} & \ref{c:TranWDRFOVB16}\\
TerekhovTDB14 \href{https://doi.org/10.1613/jair.4278}{TerekhovTDB14} & \hyperref[auth:a829]{D. Terekhov}, \hyperref[auth:a810]{Tony T. Tran}, \hyperref[auth:a814]{Douglas G. Down}, \hyperref[auth:a89]{J. Christopher Beck} & Integrating Queueing Theory and Scheduling for Dynamic Scheduling Problems & \href{works/TerekhovTDB14.pdf}{Yes} & \cite{TerekhovTDB14} & 2014 & J. Artif. Intell. Res. & 38 & \ref{b:TerekhovTDB14} & \ref{c:TerekhovTDB14}\\
TranTDB13 \href{http://www.aaai.org/ocs/index.php/ICAPS/ICAPS13/paper/view/6005}{TranTDB13} & \hyperref[auth:a810]{Tony T. Tran}, \hyperref[auth:a829]{D. Terekhov}, \hyperref[auth:a814]{Douglas G. Down}, \hyperref[auth:a89]{J. Christopher Beck} & Hybrid Queueing Theory and Scheduling Models for Dynamic Environments with Sequence-Dependent Setup Times & \href{works/TranTDB13.pdf}{Yes} & \cite{TranTDB13} & 2013 & ICAPS 2013 & 9 & \ref{b:TranTDB13} & \ref{c:TranTDB13}\\
TranB12 \href{https://doi.org/10.3233/978-1-61499-098-7-774}{TranB12} & \hyperref[auth:a810]{Tony T. Tran}, \hyperref[auth:a89]{J. Christopher Beck} & Logic-based Benders Decomposition for Alternative Resource Scheduling with Sequence Dependent Setups & \href{works/TranB12.pdf}{Yes} & \cite{TranB12} & 2012 & ECAI 2012 & 6 & \ref{b:TranB12} & \ref{c:TranB12}\\
\end{longtable}
}

\subsection{Works by Pascal Van Hentenryck}
\label{sec:a148}
{\scriptsize
\begin{longtable}{>{\raggedright\arraybackslash}p{3cm}>{\raggedright\arraybackslash}p{6cm}>{\raggedright\arraybackslash}p{7cm}rrrp{3cm}rrr}
\rowcolor{white}\caption{Works from bibtex (Total 9)}\\ \toprule
\rowcolor{white}Key & Authors & Title & LC & Cite & Year & \shortstack{Conference\\/Journal} & Pages & b & c \\ \midrule\endhead
\bottomrule
\endfoot
FontaineMH16 \href{https://doi.org/10.1007/978-3-319-33954-2\_12}{FontaineMH16} & \hyperref[auth:a320]{D. Fontaine}, \hyperref[auth:a321]{Laurent D. Michel}, \hyperref[auth:a148]{Pascal Van Hentenryck} & Parallel Composition of Scheduling Solvers & \href{works/FontaineMH16.pdf}{Yes} & \cite{FontaineMH16} & 2016 & CPAIOR 2016 & 11 & \ref{b:FontaineMH16} & \ref{c:FontaineMH16}\\
EvenSH15 \href{https://doi.org/10.1007/978-3-319-23219-5\_40}{EvenSH15} & \hyperref[auth:a219]{C. Even}, \hyperref[auth:a124]{A. Schutt}, \hyperref[auth:a148]{Pascal Van Hentenryck} & A Constraint Programming Approach for Non-preemptive Evacuation Scheduling & \href{works/EvenSH15.pdf}{Yes} & \cite{EvenSH15} & 2015 & CP 2015 & 18 & \ref{b:EvenSH15} & \ref{c:EvenSH15}\\
EvenSH15a \href{http://arxiv.org/abs/1505.02487}{EvenSH15a} & \hyperref[auth:a219]{C. Even}, \hyperref[auth:a124]{A. Schutt}, \hyperref[auth:a148]{Pascal Van Hentenryck} & A Constraint Programming Approach for Non-Preemptive Evacuation Scheduling & \href{works/EvenSH15a.pdf}{Yes} & \cite{EvenSH15a} & 2015 & CoRR & 16 & \ref{b:EvenSH15a} & \ref{c:EvenSH15a}\\
SchausHMCMD11 \href{https://doi.org/10.1007/s10601-010-9100-5}{SchausHMCMD11} & \hyperref[auth:a147]{P. Schaus}, \hyperref[auth:a148]{Pascal Van Hentenryck}, \hyperref[auth:a149]{J. Monette}, \hyperref[auth:a150]{C. Coffrin}, \hyperref[auth:a32]{L. Michel}, \hyperref[auth:a151]{Y. Deville} & Solving Steel Mill Slab Problems with constraint-based techniques: CP, LNS, and {CBLS} & \href{works/SchausHMCMD11.pdf}{Yes} & \cite{SchausHMCMD11} & 2011 & Constraints An Int. J. & 23 & \ref{b:SchausHMCMD11} & \ref{c:SchausHMCMD11}\\
MonetteDH09 \href{http://aaai.org/ocs/index.php/ICAPS/ICAPS09/paper/view/712}{MonetteDH09} & \hyperref[auth:a149]{J. Monette}, \hyperref[auth:a151]{Y. Deville}, \hyperref[auth:a148]{Pascal Van Hentenryck} & Just-In-Time Scheduling with Constraint Programming & \href{works/MonetteDH09.pdf}{Yes} & \cite{MonetteDH09} & 2009 & ICAPS 2009 & 8 & \ref{b:MonetteDH09} & \ref{c:MonetteDH09}\\
DoomsH08 \href{https://doi.org/10.1007/978-3-540-68155-7\_8}{DoomsH08} & \hyperref[auth:a363]{G. Dooms}, \hyperref[auth:a148]{Pascal Van Hentenryck} & Gap Reduction Techniques for Online Stochastic Project Scheduling & \href{works/DoomsH08.pdf}{Yes} & \cite{DoomsH08} & 2008 & CPAIOR 2008 & 16 & \ref{b:DoomsH08} & \ref{c:DoomsH08}\\
HentenryckM08 \href{https://doi.org/10.1007/978-3-540-68155-7\_41}{HentenryckM08} & \hyperref[auth:a148]{Pascal Van Hentenryck}, \hyperref[auth:a32]{L. Michel} & The Steel Mill Slab Design Problem Revisited & \href{works/HentenryckM08.pdf}{Yes} & \cite{HentenryckM08} & 2008 & CPAIOR 2008 & 5 & \ref{b:HentenryckM08} & \ref{c:HentenryckM08}\\
HentenryckM04 \href{https://doi.org/10.1007/978-3-540-24664-0\_22}{HentenryckM04} & \hyperref[auth:a148]{Pascal Van Hentenryck}, \hyperref[auth:a32]{L. Michel} & Scheduling Abstractions for Local Search & \href{works/HentenryckM04.pdf}{Yes} & \cite{HentenryckM04} & 2004 & CPAIOR 2004 & 16 & \ref{b:HentenryckM04} & \ref{c:HentenryckM04}\\
DincbasSH90 \href{https://doi.org/10.1016/0743-1066(90)90052-7}{DincbasSH90} & \hyperref[auth:a726]{M. Dincbas}, \hyperref[auth:a17]{H. Simonis}, \hyperref[auth:a148]{Pascal Van Hentenryck} & Solving Large Combinatorial Problems in Logic Programming & \href{works/DincbasSH90.pdf}{Yes} & \cite{DincbasSH90} & 1990 & J. Log. Program. & 19 & \ref{b:DincbasSH90} & \ref{c:DincbasSH90}\\
\end{longtable}
}

\subsection{Works by Philippe Baptiste}
\label{sec:a163}
{\scriptsize
\begin{longtable}{>{\raggedright\arraybackslash}p{3cm}>{\raggedright\arraybackslash}p{6cm}>{\raggedright\arraybackslash}p{7cm}rrrp{3cm}rrr}
\rowcolor{white}\caption{Works from bibtex (Total 8)}\\ \toprule
\rowcolor{white}Key & Authors & Title & LC & Cite & Year & \shortstack{Conference\\/Journal} & Pages & b & c \\ \midrule\endhead
\bottomrule
\endfoot
BaptisteB18 \href{https://doi.org/10.1016/j.dam.2017.05.001}{BaptisteB18} & \hyperref[auth:a163]{P. Baptiste}, \hyperref[auth:a714]{N. Bonifas} & Redundant cumulative constraints to compute preemptive bounds & \href{works/BaptisteB18.pdf}{Yes} & \cite{BaptisteB18} & 2018 & Discret. Appl. Math. & 10 & \ref{b:BaptisteB18} & \ref{c:BaptisteB18}\\
Baptiste09 \href{https://doi.org/10.1007/978-3-642-04244-7\_1}{Baptiste09} & \hyperref[auth:a163]{P. Baptiste} & Constraint-Based Schedulers, Do They Really Work? & \href{works/Baptiste09.pdf}{Yes} & \cite{Baptiste09} & 2009 & CP 2009 & 1 & \ref{b:Baptiste09} & \ref{c:Baptiste09}\\
BaptisteLPN06 \href{https://doi.org/10.1016/S1574-6526(06)80026-X}{BaptisteLPN06} & \hyperref[auth:a163]{P. Baptiste}, \hyperref[auth:a118]{P. Laborie}, \hyperref[auth:a164]{Claude Le Pape}, \hyperref[auth:a666]{W. Nuijten} & Constraint-Based Scheduling and Planning & No & \cite{BaptisteLPN06} & 2006 & n/a & 39 & No & \ref{c:BaptisteLPN06}\\
ArtiouchineB05 \href{https://doi.org/10.1007/11564751\_8}{ArtiouchineB05} & \hyperref[auth:a264]{K. Artiouchine}, \hyperref[auth:a163]{P. Baptiste} & Inter-distance Constraint: An Extension of the All-Different Constraint for Scheduling Equal Length Jobs & \href{works/ArtiouchineB05.pdf}{Yes} & \cite{ArtiouchineB05} & 2005 & CP 2005 & 15 & \ref{b:ArtiouchineB05} & \ref{c:ArtiouchineB05}\\
BaptisteP00 \href{https://doi.org/10.1023/A:1009822502231}{BaptisteP00} & \hyperref[auth:a163]{P. Baptiste}, \hyperref[auth:a164]{Claude Le Pape} & Constraint Propagation and Decomposition Techniques for Highly Disjunctive and Highly Cumulative Project Scheduling Problems & \href{works/BaptisteP00.pdf}{Yes} & \cite{BaptisteP00} & 2000 & Constraints An Int. J. & 21 & \ref{b:BaptisteP00} & \ref{c:BaptisteP00}\\
PapaB98 \href{https://doi.org/10.1023/A:1009723704757}{PapaB98} & \hyperref[auth:a164]{Claude Le Pape}, \hyperref[auth:a163]{P. Baptiste} & Resource Constraints for Preemptive Job-shop Scheduling & \href{works/PapaB98.pdf}{Yes} & \cite{PapaB98} & 1998 & Constraints An Int. J. & 25 & \ref{b:PapaB98} & \ref{c:PapaB98}\\
BaptisteP97 \href{https://doi.org/10.1007/BFb0017454}{BaptisteP97} & \hyperref[auth:a163]{P. Baptiste}, \hyperref[auth:a164]{Claude Le Pape} & Constraint Propagation and Decomposition Techniques for Highly Disjunctive and Highly Cumulative Project Scheduling Problems & \href{works/BaptisteP97.pdf}{Yes} & \cite{BaptisteP97} & 1997 & CP 1997 & 15 & \ref{b:BaptisteP97} & \ref{c:BaptisteP97}\\
PapeB97 \href{}{PapeB97} & \hyperref[auth:a164]{Claude Le Pape}, \hyperref[auth:a163]{P. Baptiste} & A Constraint Programming Library for Preemptive and Non-Preemptive Scheduling & No & \cite{PapeB97} & 1997 & PACT 1997 & 20 & No & \ref{c:PapeB97}\\
\end{longtable}
}

\subsection{Works by Mats Carlsson}
\label{sec:a91}
{\scriptsize
\begin{longtable}{>{\raggedright\arraybackslash}p{3cm}>{\raggedright\arraybackslash}p{6cm}>{\raggedright\arraybackslash}p{7cm}rrrp{3cm}rrr}
\rowcolor{white}\caption{Works from bibtex (Total 8)}\\ \toprule
\rowcolor{white}Key & Authors & Title & LC & Cite & Year & \shortstack{Conference\\/Journal} & Pages & b & c \\ \midrule\endhead
\bottomrule
\endfoot
WessenCS20 \href{https://doi.org/10.1007/978-3-030-58942-4\_33}{WessenCS20} & \hyperref[auth:a90]{J. Wess{\'{e}}n}, \hyperref[auth:a91]{M. Carlsson}, \hyperref[auth:a92]{C. Schulte} & Scheduling of Dual-Arm Multi-tool Assembly Robots and Workspace Layout Optimization & \href{works/WessenCS20.pdf}{Yes} & \cite{WessenCS20} & 2020 & CPAIOR 2020 & 10 & \ref{b:WessenCS20} & \ref{c:WessenCS20}\\
MossigeGSMC17 \href{https://doi.org/10.1007/978-3-319-66158-2\_25}{MossigeGSMC17} & \hyperref[auth:a199]{M. Mossige}, \hyperref[auth:a200]{A. Gotlieb}, \hyperref[auth:a201]{H. Spieker}, \hyperref[auth:a202]{H. Meling}, \hyperref[auth:a91]{M. Carlsson} & Time-Aware Test Case Execution Scheduling for Cyber-Physical Systems & \href{works/MossigeGSMC17.pdf}{Yes} & \cite{MossigeGSMC17} & 2017 & CP 2017 & 18 & \ref{b:MossigeGSMC17} & \ref{c:MossigeGSMC17}\\
LetortCB15 \href{https://doi.org/10.1007/s10601-014-9172-8}{LetortCB15} & \hyperref[auth:a127]{A. Letort}, \hyperref[auth:a91]{M. Carlsson}, \hyperref[auth:a128]{N. Beldiceanu} & Synchronized sweep algorithms for scalable scheduling constraints & \href{works/LetortCB15.pdf}{Yes} & \cite{LetortCB15} & 2015 & Constraints An Int. J. & 52 & \ref{b:LetortCB15} & \ref{c:LetortCB15}\\
LetortCB13 \href{https://doi.org/10.1007/978-3-642-38171-3\_10}{LetortCB13} & \hyperref[auth:a127]{A. Letort}, \hyperref[auth:a91]{M. Carlsson}, \hyperref[auth:a128]{N. Beldiceanu} & A Synchronized Sweep Algorithm for the \emph{k-dimensional cumulative} Constraint & \href{works/LetortCB13.pdf}{Yes} & \cite{LetortCB13} & 2013 & CPAIOR 2013 & 16 & \ref{b:LetortCB13} & \ref{c:LetortCB13}\\
LetortBC12 \href{https://doi.org/10.1007/978-3-642-33558-7\_33}{LetortBC12} & \hyperref[auth:a127]{A. Letort}, \hyperref[auth:a128]{N. Beldiceanu}, \hyperref[auth:a91]{M. Carlsson} & A Scalable Sweep Algorithm for the cumulative Constraint & \href{works/LetortBC12.pdf}{Yes} & \cite{LetortBC12} & 2012 & CP 2012 & 16 & \ref{b:LetortBC12} & \ref{c:LetortBC12}\\
BeldiceanuCDP11 \href{https://doi.org/10.1007/s10479-010-0731-0}{BeldiceanuCDP11} & \hyperref[auth:a128]{N. Beldiceanu}, \hyperref[auth:a91]{M. Carlsson}, \hyperref[auth:a245]{S. Demassey}, \hyperref[auth:a362]{E. Poder} & New filtering for the \emph{cumulative} constraint in the context of non-overlapping rectangles & \href{works/BeldiceanuCDP11.pdf}{Yes} & \cite{BeldiceanuCDP11} & 2011 & Ann. Oper. Res. & 24 & \ref{b:BeldiceanuCDP11} & \ref{c:BeldiceanuCDP11}\\
BeldiceanuCP08 \href{https://doi.org/10.1007/978-3-540-68155-7\_5}{BeldiceanuCP08} & \hyperref[auth:a128]{N. Beldiceanu}, \hyperref[auth:a91]{M. Carlsson}, \hyperref[auth:a362]{E. Poder} & New Filtering for the cumulative Constraint in the Context of Non-Overlapping Rectangles & \href{works/BeldiceanuCP08.pdf}{Yes} & \cite{BeldiceanuCP08} & 2008 & CPAIOR 2008 & 15 & \ref{b:BeldiceanuCP08} & \ref{c:BeldiceanuCP08}\\
BeldiceanuC02 \href{https://doi.org/10.1007/3-540-46135-3\_5}{BeldiceanuC02} & \hyperref[auth:a128]{N. Beldiceanu}, \hyperref[auth:a91]{M. Carlsson} & A New Multi-resource cumulatives Constraint with Negative Heights & \href{works/BeldiceanuC02.pdf}{Yes} & \cite{BeldiceanuC02} & 2002 & CP 2002 & 17 & \ref{b:BeldiceanuC02} & \ref{c:BeldiceanuC02}\\
\end{longtable}
}

\subsection{Works by Helmut Simonis}
\label{sec:a17}
{\scriptsize
\begin{longtable}{>{\raggedright\arraybackslash}p{3cm}>{\raggedright\arraybackslash}p{6cm}>{\raggedright\arraybackslash}p{7cm}rrrp{3cm}rrr}
\rowcolor{white}\caption{Works from bibtex (Total 8)}\\ \toprule
\rowcolor{white}Key & Authors & Title & LC & Cite & Year & \shortstack{Conference\\/Journal} & Pages & b & c \\ \midrule\endhead
\bottomrule
\endfoot
ArmstrongGOS22 \href{https://doi.org/10.1007/978-3-031-08011-1\_1}{ArmstrongGOS22} & \hyperref[auth:a14]{E. Armstrong}, \hyperref[auth:a15]{M. Garraffa}, \hyperref[auth:a16]{B. O'Sullivan}, \hyperref[auth:a17]{H. Simonis} & A Two-Phase Hybrid Approach for the Hybrid Flexible Flowshop with Transportation Times & \href{works/ArmstrongGOS22.pdf}{Yes} & \cite{ArmstrongGOS22} & 2022 & CPAIOR 2022 & 13 & \ref{b:ArmstrongGOS22} & \ref{c:ArmstrongGOS22}\\
ArmstrongGOS21 \href{https://doi.org/10.4230/LIPIcs.CP.2021.16}{ArmstrongGOS21} & \hyperref[auth:a14]{E. Armstrong}, \hyperref[auth:a15]{M. Garraffa}, \hyperref[auth:a16]{B. O'Sullivan}, \hyperref[auth:a17]{H. Simonis} & The Hybrid Flexible Flowshop with Transportation Times & \href{works/ArmstrongGOS21.pdf}{Yes} & \cite{ArmstrongGOS21} & 2021 & CP 2021 & 18 & \ref{b:ArmstrongGOS21} & \ref{c:ArmstrongGOS21}\\
GrimesIOS14 \href{https://doi.org/10.1016/j.suscom.2014.08.009}{GrimesIOS14} & \hyperref[auth:a182]{D. Grimes}, \hyperref[auth:a183]{G. Ifrim}, \hyperref[auth:a16]{B. O'Sullivan}, \hyperref[auth:a17]{H. Simonis} & Analyzing the impact of electricity price forecasting on energy cost-aware scheduling & \href{works/GrimesIOS14.pdf}{Yes} & \cite{GrimesIOS14} & 2014 & Sustain. Comput. Informatics Syst. & 16 & \ref{b:GrimesIOS14} & \ref{c:GrimesIOS14}\\
IfrimOS12 \href{https://doi.org/10.1007/978-3-642-33558-7\_68}{IfrimOS12} & \hyperref[auth:a183]{G. Ifrim}, \hyperref[auth:a16]{B. O'Sullivan}, \hyperref[auth:a17]{H. Simonis} & Properties of Energy-Price Forecasts for Scheduling & \href{works/IfrimOS12.pdf}{Yes} & \cite{IfrimOS12} & 2012 & CP 2012 & 16 & \ref{b:IfrimOS12} & \ref{c:IfrimOS12}\\
Simonis07 \href{https://doi.org/10.1007/s10601-006-9011-7}{Simonis07} & \hyperref[auth:a17]{H. Simonis} & Models for Global Constraint Applications & \href{works/Simonis07.pdf}{Yes} & \cite{Simonis07} & 2007 & Constraints An Int. J. & 30 & \ref{b:Simonis07} & \ref{c:Simonis07}\\
Simonis95 \href{https://doi.org/10.1007/3-540-60299-2\_42}{Simonis95} & \hyperref[auth:a17]{H. Simonis} & The {CHIP} System and Its Applications & \href{works/Simonis95.pdf}{Yes} & \cite{Simonis95} & 1995 & CP 1995 & 4 & \ref{b:Simonis95} & \ref{c:Simonis95}\\
SimonisC95 \href{https://doi.org/10.1007/3-540-60299-2\_27}{SimonisC95} & \hyperref[auth:a17]{H. Simonis}, \hyperref[auth:a305]{T. Cornelissens} & Modelling Producer/Consumer Constraints & \href{works/SimonisC95.pdf}{Yes} & \cite{SimonisC95} & 1995 & CP 1995 & 14 & \ref{b:SimonisC95} & \ref{c:SimonisC95}\\
DincbasSH90 \href{https://doi.org/10.1016/0743-1066(90)90052-7}{DincbasSH90} & \hyperref[auth:a726]{M. Dincbas}, \hyperref[auth:a17]{H. Simonis}, \hyperref[auth:a148]{Pascal Van Hentenryck} & Solving Large Combinatorial Problems in Logic Programming & \href{works/DincbasSH90.pdf}{Yes} & \cite{DincbasSH90} & 1990 & J. Log. Program. & 19 & \ref{b:DincbasSH90} & \ref{c:DincbasSH90}\\
\end{longtable}
}

\subsection{Works by Mark Wallace}
\label{sec:a117}
{\scriptsize
\begin{longtable}{>{\raggedright\arraybackslash}p{3cm}>{\raggedright\arraybackslash}p{6cm}>{\raggedright\arraybackslash}p{7cm}rrrp{3cm}rrr}
\rowcolor{white}\caption{Works from bibtex (Total 8)}\\ \toprule
\rowcolor{white}Key & Authors & Title & LC & Cite & Year & \shortstack{Conference\\/Journal} & Pages & b & c \\ \midrule\endhead
\bottomrule
\endfoot
WallaceY20 \href{https://doi.org/10.1007/s10601-020-09316-z}{WallaceY20} & \hyperref[auth:a117]{M. Wallace}, \hyperref[auth:a19]{N. Yorke{-}Smith} & A new constraint programming model and solving for the cyclic hoist scheduling problem & \href{works/WallaceY20.pdf}{Yes} & \cite{WallaceY20} & 2020 & Constraints An Int. J. & 19 & \ref{b:WallaceY20} & \ref{c:WallaceY20}\\
He0GLW18 \href{https://doi.org/10.1007/978-3-319-98334-9\_42}{He0GLW18} & \hyperref[auth:a185]{S. He}, \hyperref[auth:a117]{M. Wallace}, \hyperref[auth:a186]{G. Gange}, \hyperref[auth:a187]{A. Liebman}, \hyperref[auth:a188]{C. Wilson} & A Fast and Scalable Algorithm for Scheduling Large Numbers of Devices Under Real-Time Pricing & \href{works/He0GLW18.pdf}{Yes} & \cite{He0GLW18} & 2018 & CP 2018 & 18 & \ref{b:He0GLW18} & \ref{c:He0GLW18}\\
ThiruvadyWGS14 \href{https://doi.org/10.1007/s10732-014-9260-3}{ThiruvadyWGS14} & \hyperref[auth:a400]{Dhananjay R. Thiruvady}, \hyperref[auth:a117]{M. Wallace}, \hyperref[auth:a341]{H. Gu}, \hyperref[auth:a124]{A. Schutt} & A Lagrangian relaxation and {ACO} hybrid for resource constrained project scheduling with discounted cash flows & \href{works/ThiruvadyWGS14.pdf}{Yes} & \cite{ThiruvadyWGS14} & 2014 & J. Heuristics & 34 & \ref{b:ThiruvadyWGS14} & \ref{c:ThiruvadyWGS14}\\
SchuttFSW09 \href{https://doi.org/10.1007/978-3-642-04244-7\_58}{SchuttFSW09} & \hyperref[auth:a124]{A. Schutt}, \hyperref[auth:a154]{T. Feydy}, \hyperref[auth:a125]{Peter J. Stuckey}, \hyperref[auth:a117]{M. Wallace} & Why Cumulative Decomposition Is Not as Bad as It Sounds & \href{works/SchuttFSW09.pdf}{Yes} & \cite{SchuttFSW09} & 2009 & CP 2009 & 16 & \ref{b:SchuttFSW09} & \ref{c:SchuttFSW09}\\
SakkoutW00 \href{https://doi.org/10.1023/A:1009856210543}{SakkoutW00} & \hyperref[auth:a167]{Hani El Sakkout}, \hyperref[auth:a117]{M. Wallace} & Probe Backtrack Search for Minimal Perturbation in Dynamic Scheduling & \href{works/SakkoutW00.pdf}{Yes} & \cite{SakkoutW00} & 2000 & Constraints An Int. J. & 30 & \ref{b:SakkoutW00} & \ref{c:SakkoutW00}\\
RodosekW98 \href{https://doi.org/10.1007/3-540-49481-2\_28}{RodosekW98} & \hyperref[auth:a299]{R. Rodosek}, \hyperref[auth:a117]{M. Wallace} & A Generic Model and Hybrid Algorithm for Hoist Scheduling Problems & \href{works/RodosekW98.pdf}{Yes} & \cite{RodosekW98} & 1998 & CP 1998 & 15 & \ref{b:RodosekW98} & \ref{c:RodosekW98}\\
Wallace96 \href{https://doi.org/10.1007/BF00143881}{Wallace96} & \hyperref[auth:a117]{M. Wallace} & Practical Applications of Constraint Programming & \href{works/Wallace96.pdf}{Yes} & \cite{Wallace96} & 1996 & Constraints An Int. J. & 30 & \ref{b:Wallace96} & \ref{c:Wallace96}\\
Wallace94 \href{}{Wallace94} & \hyperref[auth:a117]{M. Wallace} & Applying Constraints for Scheduling & No & \cite{Wallace94} & 1994 & Constraint Programming 1994 & 19 & No & \ref{c:Wallace94}\\
\end{longtable}
}

\subsection{Works by Thibaut Feydy}
\label{sec:a154}
{\scriptsize
\begin{longtable}{>{\raggedright\arraybackslash}p{3cm}>{\raggedright\arraybackslash}p{6cm}>{\raggedright\arraybackslash}p{7cm}rrrp{3cm}rrr}
\rowcolor{white}\caption{Works from bibtex (Total 7)}\\ \toprule
\rowcolor{white}Key & Authors & Title & LC & Cite & Year & \shortstack{Conference\\/Journal} & Pages & b & c \\ \midrule\endhead
\bottomrule
\endfoot
YoungFS17 \href{https://doi.org/10.1007/978-3-319-66158-2\_20}{YoungFS17} & \hyperref[auth:a193]{Kenneth D. Young}, \hyperref[auth:a154]{T. Feydy}, \hyperref[auth:a124]{A. Schutt} & Constraint Programming Applied to the Multi-Skill Project Scheduling Problem & \href{works/YoungFS17.pdf}{Yes} & \cite{YoungFS17} & 2017 & CP 2017 & 10 & \ref{b:YoungFS17} & \ref{c:YoungFS17}\\
SchuttFS13 \href{https://doi.org/10.1007/978-3-642-40627-0\_47}{SchuttFS13} & \hyperref[auth:a124]{A. Schutt}, \hyperref[auth:a154]{T. Feydy}, \hyperref[auth:a125]{Peter J. Stuckey} & Scheduling Optional Tasks with Explanation & \href{works/SchuttFS13.pdf}{Yes} & \cite{SchuttFS13} & 2013 & CP 2013 & 17 & \ref{b:SchuttFS13} & \ref{c:SchuttFS13}\\
SchuttFS13a \href{https://doi.org/10.1007/978-3-642-38171-3\_16}{SchuttFS13a} & \hyperref[auth:a124]{A. Schutt}, \hyperref[auth:a154]{T. Feydy}, \hyperref[auth:a125]{Peter J. Stuckey} & Explaining Time-Table-Edge-Finding Propagation for the Cumulative Resource Constraint & \href{works/SchuttFS13a.pdf}{Yes} & \cite{SchuttFS13a} & 2013 & CPAIOR 2013 & 17 & \ref{b:SchuttFS13a} & \ref{c:SchuttFS13a}\\
SchuttFSW13 \href{https://doi.org/10.1007/s10951-012-0285-x}{SchuttFSW13} & \hyperref[auth:a124]{A. Schutt}, \hyperref[auth:a154]{T. Feydy}, \hyperref[auth:a125]{Peter J. Stuckey}, \hyperref[auth:a155]{Mark G. Wallace} & Solving RCPSP/max by lazy clause generation & \href{works/SchuttFSW13.pdf}{Yes} & \cite{SchuttFSW13} & 2013 & J. Sched. & 17 & \ref{b:SchuttFSW13} & \ref{c:SchuttFSW13}\\
SchuttFSW11 \href{https://doi.org/10.1007/s10601-010-9103-2}{SchuttFSW11} & \hyperref[auth:a124]{A. Schutt}, \hyperref[auth:a154]{T. Feydy}, \hyperref[auth:a125]{Peter J. Stuckey}, \hyperref[auth:a155]{Mark G. Wallace} & Explaining the cumulative propagator & \href{works/SchuttFSW11.pdf}{Yes} & \cite{SchuttFSW11} & 2011 & Constraints An Int. J. & 33 & \ref{b:SchuttFSW11} & \ref{c:SchuttFSW11}\\
abs-1009-0347 \href{http://arxiv.org/abs/1009.0347}{abs-1009-0347} & \hyperref[auth:a124]{A. Schutt}, \hyperref[auth:a154]{T. Feydy}, \hyperref[auth:a125]{Peter J. Stuckey}, \hyperref[auth:a155]{Mark G. Wallace} & Solving the Resource Constrained Project Scheduling Problem with Generalized Precedences by Lazy Clause Generation & \href{works/abs-1009-0347.pdf}{Yes} & \cite{abs-1009-0347} & 2010 & CoRR & 37 & \ref{b:abs-1009-0347} & \ref{c:abs-1009-0347}\\
SchuttFSW09 \href{https://doi.org/10.1007/978-3-642-04244-7\_58}{SchuttFSW09} & \hyperref[auth:a124]{A. Schutt}, \hyperref[auth:a154]{T. Feydy}, \hyperref[auth:a125]{Peter J. Stuckey}, \hyperref[auth:a117]{M. Wallace} & Why Cumulative Decomposition Is Not as Bad as It Sounds & \href{works/SchuttFSW09.pdf}{Yes} & \cite{SchuttFSW09} & 2009 & CP 2009 & 16 & \ref{b:SchuttFSW09} & \ref{c:SchuttFSW09}\\
\end{longtable}
}

\subsection{Works by Zdenek Hanz{\'{a}}lek}
\label{sec:a116}
{\scriptsize
\begin{longtable}{>{\raggedright\arraybackslash}p{3cm}>{\raggedright\arraybackslash}p{6cm}>{\raggedright\arraybackslash}p{7cm}rrrp{3cm}rrr}
\rowcolor{white}\caption{Works from bibtex (Total 7)}\\ \toprule
\rowcolor{white}Key & Authors & Title & LC & Cite & Year & \shortstack{Conference\\/Journal} & Pages & b & c \\ \midrule\endhead
\bottomrule
\endfoot
Mehdizadeh-Somarin23 \href{https://doi.org/10.1007/978-3-031-43670-3\_33}{Mehdizadeh-Somarin23} & \hyperref[auth:a433]{Z. Mehdizadeh{-}Somarin}, \hyperref[auth:a434]{R. Tavakkoli{-}Moghaddam}, \hyperref[auth:a435]{M. Rohaninejad}, \hyperref[auth:a116]{Z. Hanz{\'{a}}lek}, \hyperref[auth:a436]{Behdin Vahedi Nouri} & A Constraint Programming Model for a Reconfigurable Job Shop Scheduling Problem with Machine Availability & \href{works/Mehdizadeh-Somarin23.pdf}{Yes} & \cite{Mehdizadeh-Somarin23} & 2023 & APMS 2023 & 14 & \ref{b:Mehdizadeh-Somarin23} & \ref{c:Mehdizadeh-Somarin23}\\
abs-2305-19888 \href{https://doi.org/10.48550/arXiv.2305.19888}{abs-2305-19888} & \hyperref[auth:a437]{V. Heinz}, \hyperref[auth:a438]{A. Nov{\'{a}}k}, \hyperref[auth:a313]{M. Vlk}, \hyperref[auth:a116]{Z. Hanz{\'{a}}lek} & Constraint Programming and Constructive Heuristics for Parallel Machine Scheduling with Sequence-Dependent Setups and Common Servers & \href{works/abs-2305-19888.pdf}{Yes} & \cite{abs-2305-19888} & 2023 & CoRR & 42 & \ref{b:abs-2305-19888} & \ref{c:abs-2305-19888}\\
HeinzNVH22 \href{https://doi.org/10.1016/j.cie.2022.108586}{HeinzNVH22} & \hyperref[auth:a437]{V. Heinz}, \hyperref[auth:a438]{A. Nov{\'{a}}k}, \hyperref[auth:a313]{M. Vlk}, \hyperref[auth:a116]{Z. Hanz{\'{a}}lek} & Constraint Programming and constructive heuristics for parallel machine scheduling with sequence-dependent setups and common servers & \href{works/HeinzNVH22.pdf}{Yes} & \cite{HeinzNVH22} & 2022 & Comput. Ind. Eng. & 16 & \ref{b:HeinzNVH22} & \ref{c:HeinzNVH22}\\
VlkHT21 \href{https://doi.org/10.1016/j.cie.2021.107317}{VlkHT21} & \hyperref[auth:a313]{M. Vlk}, \hyperref[auth:a116]{Z. Hanz{\'{a}}lek}, \hyperref[auth:a480]{S. Tang} & Constraint programming approaches to joint routing and scheduling in time-sensitive networks & \href{works/VlkHT21.pdf}{Yes} & \cite{VlkHT21} & 2021 & Comput. Ind. Eng. & 14 & \ref{b:VlkHT21} & \ref{c:VlkHT21}\\
BenediktMH20 \href{https://doi.org/10.1007/s10601-020-09317-y}{BenediktMH20} & \hyperref[auth:a114]{O. Benedikt}, \hyperref[auth:a115]{I. M{\'{o}}dos}, \hyperref[auth:a116]{Z. Hanz{\'{a}}lek} & Power of pre-processing: production scheduling with variable energy pricing and power-saving states & \href{works/BenediktMH20.pdf}{Yes} & \cite{BenediktMH20} & 2020 & Constraints An Int. J. & 19 & \ref{b:BenediktMH20} & \ref{c:BenediktMH20}\\
BenediktSMVH18 \href{https://doi.org/10.1007/978-3-319-93031-2\_6}{BenediktSMVH18} & \hyperref[auth:a114]{O. Benedikt}, \hyperref[auth:a312]{P. Sucha}, \hyperref[auth:a115]{I. M{\'{o}}dos}, \hyperref[auth:a313]{M. Vlk}, \hyperref[auth:a116]{Z. Hanz{\'{a}}lek} & Energy-Aware Production Scheduling with Power-Saving Modes & \href{works/BenediktSMVH18.pdf}{Yes} & \cite{BenediktSMVH18} & 2018 & CPAIOR 2018 & 10 & \ref{b:BenediktSMVH18} & \ref{c:BenediktSMVH18}\\
KelbelH11 \href{https://doi.org/10.1007/s10845-009-0318-2}{KelbelH11} & \hyperref[auth:a627]{J. Kelbel}, \hyperref[auth:a116]{Z. Hanz{\'{a}}lek} & Solving production scheduling with earliness/tardiness penalties by constraint programming & \href{works/KelbelH11.pdf}{Yes} & \cite{KelbelH11} & 2011 & J. Intell. Manuf. & 10 & \ref{b:KelbelH11} & \ref{c:KelbelH11}\\
\end{longtable}
}

\subsection{Works by Andr{\'{a}}s Kov{\'{a}}cs}
\label{sec:a146}
{\scriptsize
\begin{longtable}{>{\raggedright\arraybackslash}p{3cm}>{\raggedright\arraybackslash}p{6cm}>{\raggedright\arraybackslash}p{7cm}rrrp{3cm}rrr}
\rowcolor{white}\caption{Works from bibtex (Total 7)}\\ \toprule
\rowcolor{white}Key & Authors & Title & LC & Cite & Year & \shortstack{Conference\\/Journal} & Pages & b & c \\ \midrule\endhead
\bottomrule
\endfoot
KovacsB11 \href{https://doi.org/10.1007/s10601-009-9088-x}{KovacsB11} & \hyperref[auth:a146]{A. Kov{\'{a}}cs}, \hyperref[auth:a89]{J. Christopher Beck} & A global constraint for total weighted completion time for unary resources & \href{works/KovacsB11.pdf}{Yes} & \cite{KovacsB11} & 2011 & Constraints An Int. J. & 24 & \ref{b:KovacsB11} & \ref{c:KovacsB11}\\
KovacsK11 \href{https://doi.org/10.1007/s10601-010-9102-3}{KovacsK11} & \hyperref[auth:a146]{A. Kov{\'{a}}cs}, \hyperref[auth:a156]{T. Kis} & Constraint programming approach to a bilevel scheduling problem & \href{works/KovacsK11.pdf}{Yes} & \cite{KovacsK11} & 2011 & Constraints An Int. J. & 24 & \ref{b:KovacsK11} & \ref{c:KovacsK11}\\
KovacsB08 \href{https://doi.org/10.1016/j.engappai.2008.03.004}{KovacsB08} & \hyperref[auth:a146]{A. Kov{\'{a}}cs}, \hyperref[auth:a89]{J. Christopher Beck} & A global constraint for total weighted completion time for cumulative resources & \href{works/KovacsB08.pdf}{Yes} & \cite{KovacsB08} & 2008 & Eng. Appl. Artif. Intell. & 7 & \ref{b:KovacsB08} & \ref{c:KovacsB08}\\
KovacsB07 \href{https://doi.org/10.1007/978-3-540-72397-4\_9}{KovacsB07} & \hyperref[auth:a146]{A. Kov{\'{a}}cs}, \hyperref[auth:a89]{J. Christopher Beck} & A Global Constraint for Total Weighted Completion Time & \href{works/KovacsB07.pdf}{Yes} & \cite{KovacsB07} & 2007 & CPAIOR 2007 & 15 & \ref{b:KovacsB07} & \ref{c:KovacsB07}\\
KovacsV06 \href{https://doi.org/10.1007/11757375\_13}{KovacsV06} & \hyperref[auth:a146]{A. Kov{\'{a}}cs}, \hyperref[auth:a280]{J. V{\'{a}}ncza} & Progressive Solutions: {A} Simple but Efficient Dominance Rule for Practical {RCPSP} & \href{works/KovacsV06.pdf}{Yes} & \cite{KovacsV06} & 2006 & CPAIOR 2006 & 13 & \ref{b:KovacsV06} & \ref{c:KovacsV06}\\
KovacsEKV05 \href{https://doi.org/10.1007/11564751\_118}{KovacsEKV05} & \hyperref[auth:a146]{A. Kov{\'{a}}cs}, \hyperref[auth:a279]{P. Egri}, \hyperref[auth:a156]{T. Kis}, \hyperref[auth:a280]{J. V{\'{a}}ncza} & Proterv-II: An Integrated Production Planning and Scheduling System & \href{works/KovacsEKV05.pdf}{Yes} & \cite{KovacsEKV05} & 2005 & CP 2005 & 1 & \ref{b:KovacsEKV05} & \ref{c:KovacsEKV05}\\
KovacsV04 \href{https://doi.org/10.1007/978-3-540-30201-8\_26}{KovacsV04} & \hyperref[auth:a146]{A. Kov{\'{a}}cs}, \hyperref[auth:a280]{J. V{\'{a}}ncza} & Completable Partial Solutions in Constraint Programming and Constraint-Based Scheduling & \href{works/KovacsV04.pdf}{Yes} & \cite{KovacsV04} & 2004 & CP 2004 & 15 & \ref{b:KovacsV04} & \ref{c:KovacsV04}\\
\end{longtable}
}

\subsection{Works by Gabriela P. Henning}
\label{sec:a596}
{\scriptsize
\begin{longtable}{>{\raggedright\arraybackslash}p{3cm}>{\raggedright\arraybackslash}p{6cm}>{\raggedright\arraybackslash}p{7cm}rrrp{3cm}rrr}
\rowcolor{white}\caption{Works from bibtex (Total 7)}\\ \toprule
\rowcolor{white}Key & Authors & Title & LC & Cite & Year & \shortstack{Conference\\/Journal} & Pages & b & c \\ \midrule\endhead
\bottomrule
\endfoot
NovaraNH16 \href{https://doi.org/10.1016/j.compchemeng.2016.04.030}{NovaraNH16} & \hyperref[auth:a595]{Franco M. Novara}, \hyperref[auth:a529]{Juan M. Novas}, \hyperref[auth:a596]{Gabriela P. Henning} & A novel constraint programming model for large-scale scheduling problems in multiproduct multistage batch plants: Limited resources and campaign-based operation & \href{works/NovaraNH16.pdf}{Yes} & \cite{NovaraNH16} & 2016 & Comput. Chem. Eng. & 17 & \ref{b:NovaraNH16} & \ref{c:NovaraNH16}\\
NovasH14 \href{https://doi.org/10.1016/j.eswa.2013.09.026}{NovasH14} & \hyperref[auth:a529]{Juan M. Novas}, \hyperref[auth:a596]{Gabriela P. Henning} & Integrated scheduling of resource-constrained flexible manufacturing systems using constraint programming & \href{works/NovasH14.pdf}{Yes} & \cite{NovasH14} & 2014 & Expert Syst. Appl. & 14 & \ref{b:NovasH14} & \ref{c:NovasH14}\\
NovasH12 \href{https://doi.org/10.1016/j.compchemeng.2012.01.005}{NovasH12} & \hyperref[auth:a529]{Juan M. Novas}, \hyperref[auth:a596]{Gabriela P. Henning} & A comprehensive constraint programming approach for the rolling horizon-based scheduling of automated wet-etch stations & \href{works/NovasH12.pdf}{Yes} & \cite{NovasH12} & 2012 & Comput. Chem. Eng. & 17 & \ref{b:NovasH12} & \ref{c:NovasH12}\\
NovasH10 \href{https://doi.org/10.1016/j.compchemeng.2010.07.011}{NovasH10} & \hyperref[auth:a529]{Juan M. Novas}, \hyperref[auth:a596]{Gabriela P. Henning} & Reactive scheduling framework based on domain knowledge and constraint programming & \href{works/NovasH10.pdf}{Yes} & \cite{NovasH10} & 2010 & Comput. Chem. Eng. & 20 & \ref{b:NovasH10} & \ref{c:NovasH10}\\
ZeballosQH10 \href{https://doi.org/10.1016/j.engappai.2009.07.002}{ZeballosQH10} & \hyperref[auth:a630]{L. Zeballos}, \hyperref[auth:a631]{O. Quiroga}, \hyperref[auth:a596]{Gabriela P. Henning} & A constraint programming model for the scheduling of flexible manufacturing systems with machine and tool limitations & \href{works/ZeballosQH10.pdf}{Yes} & \cite{ZeballosQH10} & 2010 & Eng. Appl. Artif. Intell. & 20 & \ref{b:ZeballosQH10} & \ref{c:ZeballosQH10}\\
QuirogaZH05 \href{https://doi.org/10.1109/ROBOT.2005.1570686}{QuirogaZH05} & \hyperref[auth:a631]{O. Quiroga}, \hyperref[auth:a630]{L. Zeballos}, \hyperref[auth:a596]{Gabriela P. Henning} & A Constraint Programming Approach to Tool Allocation and Resource Scheduling in {FMS} & \href{works/QuirogaZH05.pdf}{Yes} & \cite{QuirogaZH05} & 2005 & ICRA 2005 & 6 & \ref{b:QuirogaZH05} & \ref{c:QuirogaZH05}\\
ZeballosH05 \href{http://journal.iberamia.org/index.php/ia/article/view/452/article\%20\%281\%29.pdf}{ZeballosH05} & \hyperref[auth:a630]{L. Zeballos}, \hyperref[auth:a596]{Gabriela P. Henning} & A Constraint Programming Approach to {FMS} Scheduling. Consideration of Storage and Transportation Resources & \href{works/ZeballosH05.pdf}{Yes} & \cite{ZeballosH05} & 2005 & Inteligencia Artif. & 10 & \ref{b:ZeballosH05} & \ref{c:ZeballosH05}\\
\end{longtable}
}

\subsection{Works by Stefan Heinz}
\label{sec:a133}
{\scriptsize
\begin{longtable}{>{\raggedright\arraybackslash}p{3cm}>{\raggedright\arraybackslash}p{6cm}>{\raggedright\arraybackslash}p{7cm}rrrp{3cm}rrr}
\rowcolor{white}\caption{Works from bibtex (Total 6)}\\ \toprule
\rowcolor{white}Key & Authors & Title & LC & Cite & Year & \shortstack{Conference\\/Journal} & Pages & b & c \\ \midrule\endhead
\bottomrule
\endfoot
HeinzKB13 \href{https://doi.org/10.1007/978-3-642-38171-3\_2}{HeinzKB13} & \hyperref[auth:a133]{S. Heinz}, \hyperref[auth:a336]{W. Ku}, \hyperref[auth:a89]{J. Christopher Beck} & Recent Improvements Using Constraint Integer Programming for Resource Allocation and Scheduling & \href{works/HeinzKB13.pdf}{Yes} & \cite{HeinzKB13} & 2013 & CPAIOR 2013 & 16 & \ref{b:HeinzKB13} & \ref{c:HeinzKB13}\\
HeinzSB13 \href{https://doi.org/10.1007/s10601-012-9136-9}{HeinzSB13} & \hyperref[auth:a133]{S. Heinz}, \hyperref[auth:a134]{J. Schulz}, \hyperref[auth:a89]{J. Christopher Beck} & Using dual presolving reductions to reformulate cumulative constraints & \href{works/HeinzSB13.pdf}{Yes} & \cite{HeinzSB13} & 2013 & Constraints An Int. J. & 36 & \ref{b:HeinzSB13} & \ref{c:HeinzSB13}\\
HeinzB12 \href{https://doi.org/10.1007/978-3-642-29828-8\_14}{HeinzB12} & \hyperref[auth:a133]{S. Heinz}, \hyperref[auth:a89]{J. Christopher Beck} & Reconsidering Mixed Integer Programming and MIP-Based Hybrids for Scheduling & \href{works/HeinzB12.pdf}{Yes} & \cite{HeinzB12} & 2012 & CPAIOR 2012 & 17 & \ref{b:HeinzB12} & \ref{c:HeinzB12}\\
HeinzSSW12 \href{https://doi.org/10.1007/s10601-011-9113-8}{HeinzSSW12} & \hyperref[auth:a133]{S. Heinz}, \hyperref[auth:a139]{T. Schlechte}, \hyperref[auth:a140]{R. Stephan}, \hyperref[auth:a141]{M. Winkler} & Solving steel mill slab design problems & \href{works/HeinzSSW12.pdf}{Yes} & \cite{HeinzSSW12} & 2012 & Constraints An Int. J. & 12 & \ref{b:HeinzSSW12} & \ref{c:HeinzSSW12}\\
HeinzS11 \href{https://doi.org/10.1007/978-3-642-20662-7\_34}{HeinzS11} & \hyperref[auth:a133]{S. Heinz}, \hyperref[auth:a134]{J. Schulz} & Explanations for the Cumulative Constraint: An Experimental Study & \href{works/HeinzS11.pdf}{Yes} & \cite{HeinzS11} & 2011 & SEA 2011 & 10 & \ref{b:HeinzS11} & \ref{c:HeinzS11}\\
BertholdHLMS10 \href{https://doi.org/10.1007/978-3-642-13520-0\_34}{BertholdHLMS10} & \hyperref[auth:a355]{T. Berthold}, \hyperref[auth:a133]{S. Heinz}, \hyperref[auth:a356]{Marco E. L{\"{u}}bbecke}, \hyperref[auth:a357]{Rolf H. M{\"{o}}hring}, \hyperref[auth:a134]{J. Schulz} & A Constraint Integer Programming Approach for Resource-Constrained Project Scheduling & \href{works/BertholdHLMS10.pdf}{Yes} & \cite{BertholdHLMS10} & 2010 & CPAIOR 2010 & 5 & \ref{b:BertholdHLMS10} & \ref{c:BertholdHLMS10}\\
\end{longtable}
}

\subsection{Works by Claude Le Pape}
\label{sec:a164}
{\scriptsize
\begin{longtable}{>{\raggedright\arraybackslash}p{3cm}>{\raggedright\arraybackslash}p{6cm}>{\raggedright\arraybackslash}p{7cm}rrrp{3cm}rrr}
\rowcolor{white}\caption{Works from bibtex (Total 6)}\\ \toprule
\rowcolor{white}Key & Authors & Title & LC & Cite & Year & \shortstack{Conference\\/Journal} & Pages & b & c \\ \midrule\endhead
\bottomrule
\endfoot
BaptisteLPN06 \href{https://doi.org/10.1016/S1574-6526(06)80026-X}{BaptisteLPN06} & \hyperref[auth:a163]{P. Baptiste}, \hyperref[auth:a118]{P. Laborie}, \hyperref[auth:a164]{Claude Le Pape}, \hyperref[auth:a666]{W. Nuijten} & Constraint-Based Scheduling and Planning & No & \cite{BaptisteLPN06} & 2006 & n/a & 39 & No & \ref{c:BaptisteLPN06}\\
BaptisteP00 \href{https://doi.org/10.1023/A:1009822502231}{BaptisteP00} & \hyperref[auth:a163]{P. Baptiste}, \hyperref[auth:a164]{Claude Le Pape} & Constraint Propagation and Decomposition Techniques for Highly Disjunctive and Highly Cumulative Project Scheduling Problems & \href{works/BaptisteP00.pdf}{Yes} & \cite{BaptisteP00} & 2000 & Constraints An Int. J. & 21 & \ref{b:BaptisteP00} & \ref{c:BaptisteP00}\\
NuijtenP98 \href{https://doi.org/10.1023/A:1009687210594}{NuijtenP98} & \hyperref[auth:a666]{W. Nuijten}, \hyperref[auth:a164]{Claude Le Pape} & Constraint-Based Job Shop Scheduling with {\textbackslash}sc Ilog Scheduler & \href{works/NuijtenP98.pdf}{Yes} & \cite{NuijtenP98} & 1998 & J. Heuristics & 16 & \ref{b:NuijtenP98} & \ref{c:NuijtenP98}\\
PapaB98 \href{https://doi.org/10.1023/A:1009723704757}{PapaB98} & \hyperref[auth:a164]{Claude Le Pape}, \hyperref[auth:a163]{P. Baptiste} & Resource Constraints for Preemptive Job-shop Scheduling & \href{works/PapaB98.pdf}{Yes} & \cite{PapaB98} & 1998 & Constraints An Int. J. & 25 & \ref{b:PapaB98} & \ref{c:PapaB98}\\
BaptisteP97 \href{https://doi.org/10.1007/BFb0017454}{BaptisteP97} & \hyperref[auth:a163]{P. Baptiste}, \hyperref[auth:a164]{Claude Le Pape} & Constraint Propagation and Decomposition Techniques for Highly Disjunctive and Highly Cumulative Project Scheduling Problems & \href{works/BaptisteP97.pdf}{Yes} & \cite{BaptisteP97} & 1997 & CP 1997 & 15 & \ref{b:BaptisteP97} & \ref{c:BaptisteP97}\\
PapeB97 \href{}{PapeB97} & \hyperref[auth:a164]{Claude Le Pape}, \hyperref[auth:a163]{P. Baptiste} & A Constraint Programming Library for Preemptive and Non-Preemptive Scheduling & No & \cite{PapeB97} & 1997 & PACT 1997 & 20 & No & \ref{c:PapeB97}\\
\end{longtable}
}

\subsection{Works by Emmanuel Poder}
\label{sec:a362}
{\scriptsize
\begin{longtable}{>{\raggedright\arraybackslash}p{3cm}>{\raggedright\arraybackslash}p{6cm}>{\raggedright\arraybackslash}p{7cm}rrrp{3cm}rrr}
\rowcolor{white}\caption{Works from bibtex (Total 6)}\\ \toprule
\rowcolor{white}Key & Authors & Title & LC & Cite & Year & \shortstack{Conference\\/Journal} & Pages & b & c \\ \midrule\endhead
\bottomrule
\endfoot
BeldiceanuCDP11 \href{https://doi.org/10.1007/s10479-010-0731-0}{BeldiceanuCDP11} & \hyperref[auth:a128]{N. Beldiceanu}, \hyperref[auth:a91]{M. Carlsson}, \hyperref[auth:a245]{S. Demassey}, \hyperref[auth:a362]{E. Poder} & New filtering for the \emph{cumulative} constraint in the context of non-overlapping rectangles & \href{works/BeldiceanuCDP11.pdf}{Yes} & \cite{BeldiceanuCDP11} & 2011 & Ann. Oper. Res. & 24 & \ref{b:BeldiceanuCDP11} & \ref{c:BeldiceanuCDP11}\\
abs-0907-0939 \href{http://arxiv.org/abs/0907.0939}{abs-0907-0939} & \hyperref[auth:a226]{T. Petit}, \hyperref[auth:a362]{E. Poder} & The Soft Cumulative Constraint & \href{works/abs-0907-0939.pdf}{Yes} & \cite{abs-0907-0939} & 2009 & CoRR & 12 & \ref{b:abs-0907-0939} & \ref{c:abs-0907-0939}\\
BeldiceanuCP08 \href{https://doi.org/10.1007/978-3-540-68155-7\_5}{BeldiceanuCP08} & \hyperref[auth:a128]{N. Beldiceanu}, \hyperref[auth:a91]{M. Carlsson}, \hyperref[auth:a362]{E. Poder} & New Filtering for the cumulative Constraint in the Context of Non-Overlapping Rectangles & \href{works/BeldiceanuCP08.pdf}{Yes} & \cite{BeldiceanuCP08} & 2008 & CPAIOR 2008 & 15 & \ref{b:BeldiceanuCP08} & \ref{c:BeldiceanuCP08}\\
PoderB08 \href{http://www.aaai.org/Library/ICAPS/2008/icaps08-033.php}{PoderB08} & \hyperref[auth:a362]{E. Poder}, \hyperref[auth:a128]{N. Beldiceanu} & Filtering for a Continuous Multi-Resources cumulative Constraint with Resource Consumption and Production & \href{works/PoderB08.pdf}{Yes} & \cite{PoderB08} & 2008 & ICAPS 2008 & 8 & \ref{b:PoderB08} & \ref{c:PoderB08}\\
BeldiceanuP07 \href{https://doi.org/10.1007/978-3-540-72397-4\_16}{BeldiceanuP07} & \hyperref[auth:a128]{N. Beldiceanu}, \hyperref[auth:a362]{E. Poder} & A Continuous Multi-resources \emph{cumulative} Constraint with Positive-Negative Resource Consumption-Production & \href{works/BeldiceanuP07.pdf}{Yes} & \cite{BeldiceanuP07} & 2007 & CPAIOR 2007 & 15 & \ref{b:BeldiceanuP07} & \ref{c:BeldiceanuP07}\\
PoderBS04 \href{https://doi.org/10.1016/S0377-2217(02)00756-7}{PoderBS04} & \hyperref[auth:a362]{E. Poder}, \hyperref[auth:a128]{N. Beldiceanu}, \hyperref[auth:a722]{E. Sanlaville} & Computing a lower approximation of the compulsory part of a task with varying duration and varying resource consumption & \href{works/PoderBS04.pdf}{Yes} & \cite{PoderBS04} & 2004 & Eur. J. Oper. Res. & 16 & \ref{b:PoderBS04} & \ref{c:PoderBS04}\\
\end{longtable}
}

\subsection{Works by Yves Deville}
\label{sec:a151}
{\scriptsize
\begin{longtable}{>{\raggedright\arraybackslash}p{3cm}>{\raggedright\arraybackslash}p{6cm}>{\raggedright\arraybackslash}p{7cm}rrrp{3cm}rrr}
\rowcolor{white}\caption{Works from bibtex (Total 5)}\\ \toprule
\rowcolor{white}Key & Authors & Title & LC & Cite & Year & \shortstack{Conference\\/Journal} & Pages & b & c \\ \midrule\endhead
\bottomrule
\endfoot
DejemeppeD14 \href{https://doi.org/10.1007/978-3-319-07046-9\_20}{DejemeppeD14} & \hyperref[auth:a207]{C. Dejemeppe}, \hyperref[auth:a151]{Y. Deville} & Continuously Degrading Resource and Interval Dependent Activity Durations in Nuclear Medicine Patient Scheduling & \href{works/DejemeppeD14.pdf}{Yes} & \cite{DejemeppeD14} & 2014 & CPAIOR 2014 & 9 & \ref{b:DejemeppeD14} & \ref{c:DejemeppeD14}\\
HoundjiSWD14 \href{https://doi.org/10.1007/978-3-319-10428-7\_29}{HoundjiSWD14} & \hyperref[auth:a228]{Vinas{\'{e}}tan Ratheil Houndji}, \hyperref[auth:a147]{P. Schaus}, \hyperref[auth:a229]{Laurence A. Wolsey}, \hyperref[auth:a151]{Y. Deville} & The StockingCost Constraint & \href{works/HoundjiSWD14.pdf}{Yes} & \cite{HoundjiSWD14} & 2014 & CP 2014 & 16 & \ref{b:HoundjiSWD14} & \ref{c:HoundjiSWD14}\\
SchausHMCMD11 \href{https://doi.org/10.1007/s10601-010-9100-5}{SchausHMCMD11} & \hyperref[auth:a147]{P. Schaus}, \hyperref[auth:a148]{Pascal Van Hentenryck}, \hyperref[auth:a149]{J. Monette}, \hyperref[auth:a150]{C. Coffrin}, \hyperref[auth:a32]{L. Michel}, \hyperref[auth:a151]{Y. Deville} & Solving Steel Mill Slab Problems with constraint-based techniques: CP, LNS, and {CBLS} & \href{works/SchausHMCMD11.pdf}{Yes} & \cite{SchausHMCMD11} & 2011 & Constraints An Int. J. & 23 & \ref{b:SchausHMCMD11} & \ref{c:SchausHMCMD11}\\
MonetteDH09 \href{http://aaai.org/ocs/index.php/ICAPS/ICAPS09/paper/view/712}{MonetteDH09} & \hyperref[auth:a149]{J. Monette}, \hyperref[auth:a151]{Y. Deville}, \hyperref[auth:a148]{Pascal Van Hentenryck} & Just-In-Time Scheduling with Constraint Programming & \href{works/MonetteDH09.pdf}{Yes} & \cite{MonetteDH09} & 2009 & ICAPS 2009 & 8 & \ref{b:MonetteDH09} & \ref{c:MonetteDH09}\\
MonetteDD07 \href{https://doi.org/10.1007/978-3-540-72397-4\_14}{MonetteDD07} & \hyperref[auth:a149]{J. Monette}, \hyperref[auth:a151]{Y. Deville}, \hyperref[auth:a372]{P. Dupont} & A Position-Based Propagator for the Open-Shop Problem & \href{works/MonetteDD07.pdf}{Yes} & \cite{MonetteDD07} & 2007 & CPAIOR 2007 & 14 & \ref{b:MonetteDD07} & \ref{c:MonetteDD07}\\
\end{longtable}
}

\subsection{Works by Mark G. Wallace}
\label{sec:a155}
{\scriptsize
\begin{longtable}{>{\raggedright\arraybackslash}p{3cm}>{\raggedright\arraybackslash}p{6cm}>{\raggedright\arraybackslash}p{7cm}rrrp{3cm}rrr}
\rowcolor{white}\caption{Works from bibtex (Total 5)}\\ \toprule
\rowcolor{white}Key & Authors & Title & LC & Cite & Year & \shortstack{Conference\\/Journal} & Pages & b & c \\ \midrule\endhead
\bottomrule
\endfoot
SchuttFSW13 \href{https://doi.org/10.1007/s10951-012-0285-x}{SchuttFSW13} & \hyperref[auth:a124]{A. Schutt}, \hyperref[auth:a154]{T. Feydy}, \hyperref[auth:a125]{Peter J. Stuckey}, \hyperref[auth:a155]{Mark G. Wallace} & Solving RCPSP/max by lazy clause generation & \href{works/SchuttFSW13.pdf}{Yes} & \cite{SchuttFSW13} & 2013 & J. Sched. & 17 & \ref{b:SchuttFSW13} & \ref{c:SchuttFSW13}\\
GuSW12 \href{https://doi.org/10.1007/978-3-642-33558-7\_55}{GuSW12} & \hyperref[auth:a341]{H. Gu}, \hyperref[auth:a125]{Peter J. Stuckey}, \hyperref[auth:a155]{Mark G. Wallace} & Maximising the Net Present Value of Large Resource-Constrained Projects & \href{works/GuSW12.pdf}{Yes} & \cite{GuSW12} & 2012 & CP 2012 & 15 & \ref{b:GuSW12} & \ref{c:GuSW12}\\
SchuttCSW12 \href{https://doi.org/10.1007/978-3-642-29828-8\_24}{SchuttCSW12} & \hyperref[auth:a124]{A. Schutt}, \hyperref[auth:a348]{G. Chu}, \hyperref[auth:a125]{Peter J. Stuckey}, \hyperref[auth:a155]{Mark G. Wallace} & Maximising the Net Present Value for Resource-Constrained Project Scheduling & \href{works/SchuttCSW12.pdf}{Yes} & \cite{SchuttCSW12} & 2012 & CPAIOR 2012 & 17 & \ref{b:SchuttCSW12} & \ref{c:SchuttCSW12}\\
SchuttFSW11 \href{https://doi.org/10.1007/s10601-010-9103-2}{SchuttFSW11} & \hyperref[auth:a124]{A. Schutt}, \hyperref[auth:a154]{T. Feydy}, \hyperref[auth:a125]{Peter J. Stuckey}, \hyperref[auth:a155]{Mark G. Wallace} & Explaining the cumulative propagator & \href{works/SchuttFSW11.pdf}{Yes} & \cite{SchuttFSW11} & 2011 & Constraints An Int. J. & 33 & \ref{b:SchuttFSW11} & \ref{c:SchuttFSW11}\\
abs-1009-0347 \href{http://arxiv.org/abs/1009.0347}{abs-1009-0347} & \hyperref[auth:a124]{A. Schutt}, \hyperref[auth:a154]{T. Feydy}, \hyperref[auth:a125]{Peter J. Stuckey}, \hyperref[auth:a155]{Mark G. Wallace} & Solving the Resource Constrained Project Scheduling Problem with Generalized Precedences by Lazy Clause Generation & \href{works/abs-1009-0347.pdf}{Yes} & \cite{abs-1009-0347} & 2010 & CoRR & 37 & \ref{b:abs-1009-0347} & \ref{c:abs-1009-0347}\\
\end{longtable}
}

\subsection{Works by Diarmuid Grimes}
\label{sec:a182}
{\scriptsize
\begin{longtable}{>{\raggedright\arraybackslash}p{3cm}>{\raggedright\arraybackslash}p{6cm}>{\raggedright\arraybackslash}p{7cm}rrrp{3cm}rrr}
\rowcolor{white}\caption{Works from bibtex (Total 5)}\\ \toprule
\rowcolor{white}Key & Authors & Title & LC & Cite & Year & \shortstack{Conference\\/Journal} & Pages & b & c \\ \midrule\endhead
\bottomrule
\endfoot
GrimesH15 \href{https://doi.org/10.1287/ijoc.2014.0625}{GrimesH15} & \hyperref[auth:a182]{D. Grimes}, \hyperref[auth:a1]{E. Hebrard} & Solving Variants of the Job Shop Scheduling Problem Through Conflict-Directed Search & No & \cite{GrimesH15} & 2015 & {INFORMS} J. Comput. & 17 & No & \ref{c:GrimesH15}\\
GrimesIOS14 \href{https://doi.org/10.1016/j.suscom.2014.08.009}{GrimesIOS14} & \hyperref[auth:a182]{D. Grimes}, \hyperref[auth:a183]{G. Ifrim}, \hyperref[auth:a16]{B. O'Sullivan}, \hyperref[auth:a17]{H. Simonis} & Analyzing the impact of electricity price forecasting on energy cost-aware scheduling & \href{works/GrimesIOS14.pdf}{Yes} & \cite{GrimesIOS14} & 2014 & Sustain. Comput. Informatics Syst. & 16 & \ref{b:GrimesIOS14} & \ref{c:GrimesIOS14}\\
GrimesH11 \href{https://doi.org/10.1007/978-3-642-23786-7\_28}{GrimesH11} & \hyperref[auth:a182]{D. Grimes}, \hyperref[auth:a1]{E. Hebrard} & Models and Strategies for Variants of the Job Shop Scheduling Problem & \href{works/GrimesH11.pdf}{Yes} & \cite{GrimesH11} & 2011 & CP 2011 & 17 & \ref{b:GrimesH11} & \ref{c:GrimesH11}\\
GrimesH10 \href{https://doi.org/10.1007/978-3-642-13520-0\_19}{GrimesH10} & \hyperref[auth:a182]{D. Grimes}, \hyperref[auth:a1]{E. Hebrard} & Job Shop Scheduling with Setup Times and Maximal Time-Lags: {A} Simple Constraint Programming Approach & \href{works/GrimesH10.pdf}{Yes} & \cite{GrimesH10} & 2010 & CPAIOR 2010 & 15 & \ref{b:GrimesH10} & \ref{c:GrimesH10}\\
GrimesHM09 \href{https://doi.org/10.1007/978-3-642-04244-7\_33}{GrimesHM09} & \hyperref[auth:a182]{D. Grimes}, \hyperref[auth:a1]{E. Hebrard}, \hyperref[auth:a82]{A. Malapert} & Closing the Open Shop: Contradicting Conventional Wisdom & \href{works/GrimesHM09.pdf}{Yes} & \cite{GrimesHM09} & 2009 & CP 2009 & 9 & \ref{b:GrimesHM09} & \ref{c:GrimesHM09}\\
\end{longtable}
}

\subsection{Works by Roger Kameugne}
\label{sec:a10}
{\scriptsize
\begin{longtable}{>{\raggedright\arraybackslash}p{3cm}>{\raggedright\arraybackslash}p{6cm}>{\raggedright\arraybackslash}p{7cm}rrrp{3cm}rrr}
\rowcolor{white}\caption{Works from bibtex (Total 5)}\\ \toprule
\rowcolor{white}Key & Authors & Title & LC & Cite & Year & \shortstack{Conference\\/Journal} & Pages & b & c \\ \midrule\endhead
\bottomrule
\endfoot
KameugneFND23 \href{https://doi.org/10.4230/LIPIcs.CP.2023.20}{KameugneFND23} & \hyperref[auth:a10]{R. Kameugne}, \hyperref[auth:a11]{S{\'{e}}v{\'{e}}rine Betmbe Fetgo}, \hyperref[auth:a12]{T. Noulamo}, \hyperref[auth:a13]{Cl{\'{e}}mentin Tayou Djam{\'{e}}gni} & Horizontally Elastic Edge Finder Rule for Cumulative Constraint Based on Slack and Density & \href{works/KameugneFND23.pdf}{Yes} & \cite{KameugneFND23} & 2023 & CP 2023 & 17 & \ref{b:KameugneFND23} & \ref{c:KameugneFND23}\\
KameugneFGOQ18 \href{https://doi.org/10.1007/978-3-319-93031-2\_23}{KameugneFGOQ18} & \hyperref[auth:a10]{R. Kameugne}, \hyperref[auth:a11]{S{\'{e}}v{\'{e}}rine Betmbe Fetgo}, \hyperref[auth:a315]{V. Gingras}, \hyperref[auth:a52]{Y. Ouellet}, \hyperref[auth:a37]{C. Quimper} & Horizontally Elastic Not-First/Not-Last Filtering Algorithm for Cumulative Resource Constraint & \href{works/KameugneFGOQ18.pdf}{Yes} & \cite{KameugneFGOQ18} & 2018 & CPAIOR 2018 & 17 & \ref{b:KameugneFGOQ18} & \ref{c:KameugneFGOQ18}\\
Kameugne15 \href{https://doi.org/10.1007/s10601-015-9227-5}{Kameugne15} & \hyperref[auth:a10]{R. Kameugne} & Propagation techniques of resource constraint for cumulative scheduling & \href{works/Kameugne15.pdf}{Yes} & \cite{Kameugne15} & 2015 & Constraints An Int. J. & 2 & \ref{b:Kameugne15} & \ref{c:Kameugne15}\\
KameugneFSN14 \href{https://doi.org/10.1007/s10601-013-9157-z}{KameugneFSN14} & \hyperref[auth:a10]{R. Kameugne}, \hyperref[auth:a130]{Laure Pauline Fotso}, \hyperref[auth:a131]{Joseph D. Scott}, \hyperref[auth:a132]{Y. Ngo{-}Kateu} & A quadratic edge-finding filtering algorithm for cumulative resource constraints & \href{works/KameugneFSN14.pdf}{Yes} & \cite{KameugneFSN14} & 2014 & Constraints An Int. J. & 27 & \ref{b:KameugneFSN14} & \ref{c:KameugneFSN14}\\
KameugneFSN11 \href{https://doi.org/10.1007/978-3-642-23786-7\_37}{KameugneFSN11} & \hyperref[auth:a10]{R. Kameugne}, \hyperref[auth:a130]{Laure Pauline Fotso}, \hyperref[auth:a131]{Joseph D. Scott}, \hyperref[auth:a132]{Y. Ngo{-}Kateu} & A Quadratic Edge-Finding Filtering Algorithm for Cumulative Resource Constraints & \href{works/KameugneFSN11.pdf}{Yes} & \cite{KameugneFSN11} & 2011 & CP 2011 & 15 & \ref{b:KameugneFSN11} & \ref{c:KameugneFSN11}\\
\end{longtable}
}

\subsection{Works by Juan M. Novas}
\label{sec:a529}
{\scriptsize
\begin{longtable}{>{\raggedright\arraybackslash}p{3cm}>{\raggedright\arraybackslash}p{6cm}>{\raggedright\arraybackslash}p{7cm}rrrp{3cm}rrr}
\rowcolor{white}\caption{Works from bibtex (Total 5)}\\ \toprule
\rowcolor{white}Key & Authors & Title & LC & Cite & Year & \shortstack{Conference\\/Journal} & Pages & b & c \\ \midrule\endhead
\bottomrule
\endfoot
Novas19 \href{https://doi.org/10.1016/j.cie.2019.07.011}{Novas19} & \hyperref[auth:a529]{Juan M. Novas} & Production scheduling and lot streaming at flexible job-shops environments using constraint programming & \href{works/Novas19.pdf}{Yes} & \cite{Novas19} & 2019 & Comput. Ind. Eng. & 13 & \ref{b:Novas19} & \ref{c:Novas19}\\
NovaraNH16 \href{https://doi.org/10.1016/j.compchemeng.2016.04.030}{NovaraNH16} & \hyperref[auth:a595]{Franco M. Novara}, \hyperref[auth:a529]{Juan M. Novas}, \hyperref[auth:a596]{Gabriela P. Henning} & A novel constraint programming model for large-scale scheduling problems in multiproduct multistage batch plants: Limited resources and campaign-based operation & \href{works/NovaraNH16.pdf}{Yes} & \cite{NovaraNH16} & 2016 & Comput. Chem. Eng. & 17 & \ref{b:NovaraNH16} & \ref{c:NovaraNH16}\\
NovasH14 \href{https://doi.org/10.1016/j.eswa.2013.09.026}{NovasH14} & \hyperref[auth:a529]{Juan M. Novas}, \hyperref[auth:a596]{Gabriela P. Henning} & Integrated scheduling of resource-constrained flexible manufacturing systems using constraint programming & \href{works/NovasH14.pdf}{Yes} & \cite{NovasH14} & 2014 & Expert Syst. Appl. & 14 & \ref{b:NovasH14} & \ref{c:NovasH14}\\
NovasH12 \href{https://doi.org/10.1016/j.compchemeng.2012.01.005}{NovasH12} & \hyperref[auth:a529]{Juan M. Novas}, \hyperref[auth:a596]{Gabriela P. Henning} & A comprehensive constraint programming approach for the rolling horizon-based scheduling of automated wet-etch stations & \href{works/NovasH12.pdf}{Yes} & \cite{NovasH12} & 2012 & Comput. Chem. Eng. & 17 & \ref{b:NovasH12} & \ref{c:NovasH12}\\
NovasH10 \href{https://doi.org/10.1016/j.compchemeng.2010.07.011}{NovasH10} & \hyperref[auth:a529]{Juan M. Novas}, \hyperref[auth:a596]{Gabriela P. Henning} & Reactive scheduling framework based on domain knowledge and constraint programming & \href{works/NovasH10.pdf}{Yes} & \cite{NovasH10} & 2010 & Comput. Chem. Eng. & 20 & \ref{b:NovasH10} & \ref{c:NovasH10}\\
\end{longtable}
}

\subsection{Works by Wim Nuijten}
\label{sec:a666}
{\scriptsize
\begin{longtable}{>{\raggedright\arraybackslash}p{3cm}>{\raggedright\arraybackslash}p{6cm}>{\raggedright\arraybackslash}p{7cm}rrrp{3cm}rrr}
\rowcolor{white}\caption{Works from bibtex (Total 5)}\\ \toprule
\rowcolor{white}Key & Authors & Title & LC & Cite & Year & \shortstack{Conference\\/Journal} & Pages & b & c \\ \midrule\endhead
\bottomrule
\endfoot
BaptisteLPN06 \href{https://doi.org/10.1016/S1574-6526(06)80026-X}{BaptisteLPN06} & \hyperref[auth:a163]{P. Baptiste}, \hyperref[auth:a118]{P. Laborie}, \hyperref[auth:a164]{Claude Le Pape}, \hyperref[auth:a666]{W. Nuijten} & Constraint-Based Scheduling and Planning & No & \cite{BaptisteLPN06} & 2006 & n/a & 39 & No & \ref{c:BaptisteLPN06}\\
GodardLN05 \href{http://www.aaai.org/Library/ICAPS/2005/icaps05-009.php}{GodardLN05} & \hyperref[auth:a782]{D. Godard}, \hyperref[auth:a118]{P. Laborie}, \hyperref[auth:a666]{W. Nuijten} & Randomized Large Neighborhood Search for Cumulative Scheduling & \href{works/GodardLN05.pdf}{Yes} & \cite{GodardLN05} & 2005 & ICAPS 2005 & 9 & \ref{b:GodardLN05} & \ref{c:GodardLN05}\\
FocacciLN00 \href{http://www.aaai.org/Library/AIPS/2000/aips00-010.php}{FocacciLN00} & \hyperref[auth:a784]{F. Focacci}, \hyperref[auth:a118]{P. Laborie}, \hyperref[auth:a666]{W. Nuijten} & Solving Scheduling Problems with Setup Times and Alternative Resources & \href{works/FocacciLN00.pdf}{Yes} & \cite{FocacciLN00} & 2000 & AIPS 2000 & 10 & \ref{b:FocacciLN00} & \ref{c:FocacciLN00}\\
SourdN00 \href{https://doi.org/10.1287/ijoc.12.4.341.11881}{SourdN00} & \hyperref[auth:a783]{F. Sourd}, \hyperref[auth:a666]{W. Nuijten} & Multiple-Machine Lower Bounds for Shop-Scheduling Problems & \href{works/SourdN00.pdf}{Yes} & \cite{SourdN00} & 2000 & {INFORMS} J. Comput. & 12 & \ref{b:SourdN00} & \ref{c:SourdN00}\\
NuijtenP98 \href{https://doi.org/10.1023/A:1009687210594}{NuijtenP98} & \hyperref[auth:a666]{W. Nuijten}, \hyperref[auth:a164]{Claude Le Pape} & Constraint-Based Job Shop Scheduling with {\textbackslash}sc Ilog Scheduler & \href{works/NuijtenP98.pdf}{Yes} & \cite{NuijtenP98} & 1998 & J. Heuristics & 16 & \ref{b:NuijtenP98} & \ref{c:NuijtenP98}\\
\end{longtable}
}

\subsection{Works by Louis{-}Martin Rousseau}
\label{sec:a331}
{\scriptsize
\begin{longtable}{>{\raggedright\arraybackslash}p{3cm}>{\raggedright\arraybackslash}p{6cm}>{\raggedright\arraybackslash}p{7cm}rrrp{3cm}rrr}
\rowcolor{white}\caption{Works from bibtex (Total 5)}\\ \toprule
\rowcolor{white}Key & Authors & Title & LC & Cite & Year & \shortstack{Conference\\/Journal} & Pages & b & c \\ \midrule\endhead
\bottomrule
\endfoot
DoulabiRP16 \href{https://doi.org/10.1287/ijoc.2015.0686}{DoulabiRP16} & \hyperref[auth:a335]{Seyed Hossein Hashemi Doulabi}, \hyperref[auth:a331]{L. Rousseau}, \hyperref[auth:a8]{G. Pesant} & A Constraint-Programming-Based Branch-and-Price-and-Cut Approach for Operating Room Planning and Scheduling & \href{works/DoulabiRP16.pdf}{Yes} & \cite{DoulabiRP16} & 2016 & {INFORMS} J. Comput. & 17 & \ref{b:DoulabiRP16} & \ref{c:DoulabiRP16}\\
PesantRR15 \href{https://doi.org/10.1007/978-3-319-18008-3\_21}{PesantRR15} & \hyperref[auth:a8]{G. Pesant}, \hyperref[auth:a330]{G. Rix}, \hyperref[auth:a331]{L. Rousseau} & A Comparative Study of {MIP} and {CP} Formulations for the {B2B} Scheduling Optimization Problem & \href{works/PesantRR15.pdf}{Yes} & \cite{PesantRR15} & 2015 & CPAIOR 2015 & 16 & \ref{b:PesantRR15} & \ref{c:PesantRR15}\\
DoulabiRP14 \href{https://doi.org/10.1007/978-3-319-07046-9\_32}{DoulabiRP14} & \hyperref[auth:a335]{Seyed Hossein Hashemi Doulabi}, \hyperref[auth:a331]{L. Rousseau}, \hyperref[auth:a8]{G. Pesant} & A Constraint Programming-Based Column Generation Approach for Operating Room Planning and Scheduling & \href{works/DoulabiRP14.pdf}{Yes} & \cite{DoulabiRP14} & 2014 & CPAIOR 2014 & 9 & \ref{b:DoulabiRP14} & \ref{c:DoulabiRP14}\\
ChapadosJR11 \href{https://doi.org/10.1007/978-3-642-21311-3\_7}{ChapadosJR11} & \hyperref[auth:a349]{N. Chapados}, \hyperref[auth:a350]{M. Joliveau}, \hyperref[auth:a331]{L. Rousseau} & Retail Store Workforce Scheduling by Expected Operating Income Maximization & \href{works/ChapadosJR11.pdf}{Yes} & \cite{ChapadosJR11} & 2011 & CPAIOR 2011 & 6 & \ref{b:ChapadosJR11} & \ref{c:ChapadosJR11}\\
HachemiGR11 \href{https://doi.org/10.1007/s10479-010-0698-x}{HachemiGR11} & \hyperref[auth:a623]{Nizar El Hachemi}, \hyperref[auth:a624]{M. Gendreau}, \hyperref[auth:a331]{L. Rousseau} & A hybrid constraint programming approach to the log-truck scheduling problem & \href{works/HachemiGR11.pdf}{Yes} & \cite{HachemiGR11} & 2011 & Ann. Oper. Res. & 16 & \ref{b:HachemiGR11} & \ref{c:HachemiGR11}\\
\end{longtable}
}

\subsection{Works by Marek Vlk}
\label{sec:a313}
{\scriptsize
\begin{longtable}{>{\raggedright\arraybackslash}p{3cm}>{\raggedright\arraybackslash}p{6cm}>{\raggedright\arraybackslash}p{7cm}rrrp{3cm}rrr}
\rowcolor{white}\caption{Works from bibtex (Total 5)}\\ \toprule
\rowcolor{white}Key & Authors & Title & LC & Cite & Year & \shortstack{Conference\\/Journal} & Pages & b & c \\ \midrule\endhead
\bottomrule
\endfoot
abs-2305-19888 \href{https://doi.org/10.48550/arXiv.2305.19888}{abs-2305-19888} & \hyperref[auth:a437]{V. Heinz}, \hyperref[auth:a438]{A. Nov{\'{a}}k}, \hyperref[auth:a313]{M. Vlk}, \hyperref[auth:a116]{Z. Hanz{\'{a}}lek} & Constraint Programming and Constructive Heuristics for Parallel Machine Scheduling with Sequence-Dependent Setups and Common Servers & \href{works/abs-2305-19888.pdf}{Yes} & \cite{abs-2305-19888} & 2023 & CoRR & 42 & \ref{b:abs-2305-19888} & \ref{c:abs-2305-19888}\\
HeinzNVH22 \href{https://doi.org/10.1016/j.cie.2022.108586}{HeinzNVH22} & \hyperref[auth:a437]{V. Heinz}, \hyperref[auth:a438]{A. Nov{\'{a}}k}, \hyperref[auth:a313]{M. Vlk}, \hyperref[auth:a116]{Z. Hanz{\'{a}}lek} & Constraint Programming and constructive heuristics for parallel machine scheduling with sequence-dependent setups and common servers & \href{works/HeinzNVH22.pdf}{Yes} & \cite{HeinzNVH22} & 2022 & Comput. Ind. Eng. & 16 & \ref{b:HeinzNVH22} & \ref{c:HeinzNVH22}\\
VlkHT21 \href{https://doi.org/10.1016/j.cie.2021.107317}{VlkHT21} & \hyperref[auth:a313]{M. Vlk}, \hyperref[auth:a116]{Z. Hanz{\'{a}}lek}, \hyperref[auth:a480]{S. Tang} & Constraint programming approaches to joint routing and scheduling in time-sensitive networks & \href{works/VlkHT21.pdf}{Yes} & \cite{VlkHT21} & 2021 & Comput. Ind. Eng. & 14 & \ref{b:VlkHT21} & \ref{c:VlkHT21}\\
BenediktSMVH18 \href{https://doi.org/10.1007/978-3-319-93031-2\_6}{BenediktSMVH18} & \hyperref[auth:a114]{O. Benedikt}, \hyperref[auth:a312]{P. Sucha}, \hyperref[auth:a115]{I. M{\'{o}}dos}, \hyperref[auth:a313]{M. Vlk}, \hyperref[auth:a116]{Z. Hanz{\'{a}}lek} & Energy-Aware Production Scheduling with Power-Saving Modes & \href{works/BenediktSMVH18.pdf}{Yes} & \cite{BenediktSMVH18} & 2018 & CPAIOR 2018 & 10 & \ref{b:BenediktSMVH18} & \ref{c:BenediktSMVH18}\\
BartakV15 \href{}{BartakV15} & \hyperref[auth:a152]{R. Bart{\'{a}}k}, \hyperref[auth:a313]{M. Vlk} & Reactive Recovery from Machine Breakdown in Production Scheduling with Temporal Distance and Resource Constraints & \href{works/BartakV15.pdf}{Yes} & \cite{BartakV15} & 2015 & ICAART 2015 & 12 & \ref{b:BartakV15} & \ref{c:BartakV15}\\
\end{longtable}
}

\subsection{Works by Armin Wolf}
\label{sec:a51}
{\scriptsize
\begin{longtable}{>{\raggedright\arraybackslash}p{3cm}>{\raggedright\arraybackslash}p{6cm}>{\raggedright\arraybackslash}p{7cm}rrrp{3cm}rrr}
\rowcolor{white}\caption{Works from bibtex (Total 5)}\\ \toprule
\rowcolor{white}Key & Authors & Title & LC & Cite & Year & \shortstack{Conference\\/Journal} & Pages & b & c \\ \midrule\endhead
\bottomrule
\endfoot
GeitzGSSW22 \href{https://doi.org/10.1007/978-3-031-08011-1\_10}{GeitzGSSW22} & \hyperref[auth:a47]{M. Geitz}, \hyperref[auth:a48]{C. Grozea}, \hyperref[auth:a49]{W. Steigerwald}, \hyperref[auth:a50]{R. St{\"{o}}hr}, \hyperref[auth:a51]{A. Wolf} & Solving the Extended Job Shop Scheduling Problem with AGVs - Classical and Quantum Approaches & \href{works/GeitzGSSW22.pdf}{Yes} & \cite{GeitzGSSW22} & 2022 & CPAIOR 2022 & 18 & \ref{b:GeitzGSSW22} & \ref{c:GeitzGSSW22}\\
SchuttW10 \href{https://doi.org/10.1007/978-3-642-15396-9\_36}{SchuttW10} & \hyperref[auth:a124]{A. Schutt}, \hyperref[auth:a51]{A. Wolf} & A New \emph{O}(\emph{n}\({}^{\mbox{2}}\)log\emph{n}) Not-First/Not-Last Pruning Algorithm for Cumulative Resource Constraints & \href{works/SchuttW10.pdf}{Yes} & \cite{SchuttW10} & 2010 & CP 2010 & 15 & \ref{b:SchuttW10} & \ref{c:SchuttW10}\\
SchuttWS05 \href{https://doi.org/10.1007/11963578\_6}{SchuttWS05} & \hyperref[auth:a124]{A. Schutt}, \hyperref[auth:a51]{A. Wolf}, \hyperref[auth:a720]{G. Schrader} & Not-First and Not-Last Detection for Cumulative Scheduling in \emph{O}(\emph{n}\({}^{\mbox{3}}\)log\emph{n}) & \href{works/SchuttWS05.pdf}{Yes} & \cite{SchuttWS05} & 2005 & INAP 2005 & 15 & \ref{b:SchuttWS05} & \ref{c:SchuttWS05}\\
WolfS05 \href{https://doi.org/10.1007/11963578\_8}{WolfS05} & \hyperref[auth:a51]{A. Wolf}, \hyperref[auth:a720]{G. Schrader} & \emph{O}(\emph{n} log\emph{n}) Overload Checking for the Cumulative Constraint and Its Application & \href{works/WolfS05.pdf}{Yes} & \cite{WolfS05} & 2005 & INAP 2005 & 14 & \ref{b:WolfS05} & \ref{c:WolfS05}\\
Wolf03 \href{https://doi.org/10.1007/978-3-540-45193-8\_50}{Wolf03} & \hyperref[auth:a51]{A. Wolf} & Pruning while Sweeping over Task Intervals & \href{works/Wolf03.pdf}{Yes} & \cite{Wolf03} & 2003 & CP 2003 & 15 & \ref{b:Wolf03} & \ref{c:Wolf03}\\
\end{longtable}
}



\clearpage
\section{Other Works}

\clearpage
\subsection{Books from bibtex}
{\scriptsize
\begin{longtable}{>{\raggedright\arraybackslash}p{3cm}>{\raggedright\arraybackslash}p{6cm}>{\raggedright\arraybackslash}p{6.5cm}rrrp{2.5cm}rrrrr}
\rowcolor{white}\caption{Works from bibtex (Total 3)}\\ \toprule
\rowcolor{white}\shortstack{Key\\Source} & Authors & Title & LC & Cite & Year & \shortstack{Conference\\/Journal\\/School} & Pages & \shortstack{Nr\\Cites} & \shortstack{Nr\\Refs} & b & c \\ \midrule\endhead
\bottomrule
\endfoot
\rowlabel{a:ArtiguesDN08}ArtiguesDN08 \href{http://dx.doi.org/10.1002/9780470611227}{ArtiguesDN08} & \hyperref[auth:a941]{} & Resource Constrained Project Scheduling & No & \cite{ArtiguesDN08} & 2008 & Book & null & 63 & 0 & No & n/a\\
\rowlabel{a:BaptistePN01}BaptistePN01 \href{http://dx.doi.org/10.1007/978-1-4615-1479-4}{BaptistePN01} & \hyperref[auth:a163]{P. Baptiste}, \hyperref[auth:a164]{Claude Le Pape}, \hyperref[auth:a664]{W. Nuijten} & Constraint-Based Scheduling & No & \cite{BaptistePN01} & 2001 & Book & null & 296 & 0 & No & n/a\\
\rowlabel{a:Hooker00}Hooker00 \href{http://dx.doi.org/10.1002/9781118033036}{Hooker00} & \hyperref[auth:a161]{John N. Hooker} & Logic Based Methods for Optimization: Combining Optimization and Constraint Satisfaction & No & \cite{Hooker00} & 2000 & Book & null & 185 & 0 & No & n/a\\
\end{longtable}
}



\clearpage
\subsection{PhDThesis from bibtex}
{\scriptsize
\begin{longtable}{>{\raggedright\arraybackslash}p{3cm}>{\raggedright\arraybackslash}p{6cm}>{\raggedright\arraybackslash}p{6.5cm}rrrp{2.5cm}rrrrr}
\rowcolor{white}\caption{Works from bibtex (Total 1)}\\ \toprule
\rowcolor{white}Key & Authors & Title & LC & Cite & Year & \shortstack{Conference\\/Journal} & Pages & \shortstack{Nr\\Cites} & \shortstack{Nr\\Refs} & b & c \\ \midrule\endhead
\bottomrule
\endfoot
\rowlabel{a:Siala15}Siala15 \href{https://tel.archives-ouvertes.fr/tel-01164291}{Siala15} & \hyperref[auth:a11]{M. Siala} & Search, propagation, and learning in sequencing and scheduling problems. (Recherche, propagation et apprentissage dans les probl{\`{e}}mes de s{\'{e}}quencement et d'ordonnancement) & \href{cars/works/Siala15.pdf}{Yes} & \cite{Siala15} & 2015 & {INSA} Toulouse, France & 200 & 0 & 0 & \ref{b:Siala15} & n/a\\
\end{longtable}
}



\clearpage
{\scriptsize
\begin{longtable}{>{\raggedright\arraybackslash}p{3cm}r>{\raggedright\arraybackslash}p{4cm}p{1.5cm}p{2cm}p{1.5cm}p{1.5cm}p{1.5cm}p{1.5cm}p{2cm}p{1.5cm}rr}
\rowcolor{white}\caption{Automatically Extracted THESIS Properties (Requires Local Copy)}\\ \toprule
\rowcolor{white}Work & Pages & Concepts & Classification & Constraints & \shortstack{Prog\\Languages} & \shortstack{CP\\Systems} & Areas & Industries & Benchmarks & Algorithm & a & c\\ \midrule\endhead
\bottomrule
\endfoot
\rowlabel{b:Siala15}\href{cars/works/Siala15.pdf}{Siala15}~\cite{Siala15} & 200 & earliness, sequence dependent setup, setup-time, lazy clause generation, order, due-date, cmax, machine, job-shop, task, tardiness, resource, scheduling, make-span, activity, open-shop, job, precedence & single machine, TMS, RCPSP, OSP & disjunctive, alldifferent, AtMostSeq, table constraint, GCC constraint, Cardinality constraint, CardPath, circuit, Reified constraint, MultiAtMostSeqCard, AmongSeq constraint, Disjunctive constraint, Regular constraint, Atmost constraint, AtMostSeqCard, Balance constraint, Among constraint, cumulative, cycle &  & Claire, Ilog Solver, CHIP, OPL, Mistral & automotive, rectangle-packing &  & Roadef, real-world, random instance, github, CSPlib, benchmark & GRASP, time-tabling, edge-finding & \ref{a:Siala15} & n/a\\
\end{longtable}
}




\clearpage
\subsection{InBook from bibtex}
{\scriptsize
\begin{longtable}{>{\raggedright\arraybackslash}p{3cm}>{\raggedright\arraybackslash}p{6cm}>{\raggedright\arraybackslash}p{6.5cm}rrrp{2.5cm}rrrrr}
\rowcolor{white}\caption{Works from bibtex (Total 16)}\\ \toprule
\rowcolor{white}\shortstack{Key\\Source} & Authors & Title & LC & Cite & Year & \shortstack{Conference\\/Journal\\/School} & Pages & \shortstack{Nr\\Cites} & \shortstack{Nr\\Refs} & b & c \\ \midrule\endhead
\bottomrule
\endfoot
\rowlabel{a:SchuttFSW15}SchuttFSW15 \href{https://doi.org/10.1007/978-3-319-05443-8_7}{SchuttFSW15} & \hyperref[auth:a125]{A. Schutt}, \hyperref[auth:a155]{T. Feydy}, \hyperref[auth:a126]{Peter J. Stuckey}, \hyperref[auth:a117]{Mark G. Wallace} & A Satisfiability Solving Approach & No & \cite{SchuttFSW15} & 2015 & Handbook on Project Management and Scheduling Vol.1 & 26 & 3 & 28 & No & n/a\\
\rowlabel{a:CestaOPS14}CestaOPS14 \href{http://dx.doi.org/10.1007/978-3-319-05443-8_6}{CestaOPS14} & \hyperref[auth:a286]{A. Cesta}, \hyperref[auth:a284]{A. Oddi}, \hyperref[auth:a285]{N. Policella}, \hyperref[auth:a300]{Stephen F. Smith} & A Precedence Constraint Posting Approach & No & \cite{CestaOPS14} & 2014 & Handbook on Project Management and Scheduling Vol.1 & null & 2 & 17 & No & n/a\\
\rowlabel{a:GuSSWC14}GuSSWC14 \href{http://dx.doi.org/10.1007/978-3-319-05443-8_14}{GuSSWC14} & \hyperref[auth:a339]{H. Gu}, \hyperref[auth:a125]{A. Schutt}, \hyperref[auth:a126]{Peter J. Stuckey}, \hyperref[auth:a117]{Mark G. Wallace}, \hyperref[auth:a346]{G. Chu} & Exact and Heuristic Methods for the Resource-Constrained Net Present Value Problem & No & \cite{GuSSWC14} & 2014 & Handbook on Project Management and Scheduling Vol.1 & null & 5 & 35 & No & n/a\\
\rowlabel{a:Milano11}Milano11 \href{http://dx.doi.org/10.1002/9780470400531.eorms0473}{Milano11} & \hyperref[auth:a144]{M. Milano} & Constraint Programming Links with Math Programming & No & \cite{Milano11} & 2011 & Wiley Encyclopedia of Operations Research and Management Science & null & 0 & 28 & No & n/a\\
\rowlabel{a:CastroGR10}CastroGR10 \href{http://dx.doi.org/10.1007/978-1-4419-1644-0_4}{CastroGR10} & \hyperref[auth:a895]{Pedro M. Castro}, \hyperref[auth:a385]{Ignacio E. Grossmann}, \hyperref[auth:a329]{L. Rousseau} & Decomposition Techniques for Hybrid MILP/CP Models applied to Scheduling and Routing Problems & No & \cite{CastroGR10} & 2010 & Hybrid Optimization & null & 0 & 67 & No & n/a\\
\rowlabel{a:Hooker10}Hooker10 \href{http://dx.doi.org/10.1007/978-1-4419-1644-0_2}{Hooker10} & \hyperref[auth:a161]{John N. Hooker} & Hybrid Modeling & No & \cite{Hooker10} & 2010 & Hybrid Optimization & null & 9 & 39 & No & n/a\\
\rowlabel{a:GongLMW09}GongLMW09 \href{http://dx.doi.org/10.1007/978-0-387-88617-6_11}{GongLMW09} & \hyperref[auth:a1250]{J. Gong}, \hyperref[auth:a1251]{Earl E. Lee}, \hyperref[auth:a1252]{John E. Mitchell}, \hyperref[auth:a1253]{William A. Wallace} & Logic-based MultiObjective Optimization for Restoration Planning & No & \cite{GongLMW09} & 2009 & Optimization and Logistics Challenges in the Enterprise & null & 14 & 13 & No & n/a\\
\rowlabel{a:AggounMV08}AggounMV08 \href{http://dx.doi.org/10.1007/978-0-387-74759-0_396}{AggounMV08} & \hyperref[auth:a728]{A. Aggoun}, \hyperref[auth:a911]{C. Maravelias}, \hyperref[auth:a912]{A. Vazacopoulos} & Mixed Integer Programming/Constraint Programming Hybrid Methods & No & \cite{AggounMV08} & 2008 & Encyclopedia of Optimization & null & 0 & 34 & No & n/a\\
\rowlabel{a:Hooker06a}Hooker06a \href{http://dx.doi.org/10.1016/s1574-6526(06)80019-2}{Hooker06a} & \hyperref[auth:a161]{John N. Hooker} & Operations Research Methods in Constraint Programming & No & \cite{Hooker06a} & 2006 & Foundations of Artificial Intelligence & null & 11 & 44 & No & n/a\\
\rowlabel{a:NeronABCDD06}NeronABCDD06 \href{http://dx.doi.org/10.1007/978-0-387-33768-5_7}{NeronABCDD06} & \hyperref[auth:a903]{E. Néron}, \hyperref[auth:a6]{C. Artigues}, \hyperref[auth:a163]{P. Baptiste}, \hyperref[auth:a849]{J. Carlier}, \hyperref[auth:a904]{J. Damay}, \hyperref[auth:a245]{S. Demassey}, \hyperref[auth:a118]{P. Laborie} & Lower Bounds for Resource Constrained Project Scheduling Problem & No & \cite{NeronABCDD06} & 2006 & Perspectives in Modern Project Scheduling & null & 3 & 34 & No & n/a\\
\rowlabel{a:WolfS05a}WolfS05a \href{http://dx.doi.org/10.1007/11415763_12}{WolfS05a} & \hyperref[auth:a51]{A. Wolf}, \hyperref[auth:a714]{H. Schlenker} & Realising the Alternative Resources Constraint & \href{../works/WolfS05a.pdf}{Yes} & \cite{WolfS05a} & 2005 & Applications of Declarative Programming and Knowledge Management & 15 & 5 & 6 & \ref{b:WolfS05a} & n/a\\
\rowlabel{a:AggounV04}AggounV04 \href{http://dx.doi.org/10.1007/978-3-540-24734-0_15}{AggounV04} & \hyperref[auth:a728]{A. Aggoun}, \hyperref[auth:a912]{A. Vazacopoulos} & Solving Sports Scheduling and Timetabling Problems with Constraint Programming & No & \cite{AggounV04} & 2004 & Economics, Management and Optimization in Sports & null & 7 & 4 & No & n/a\\
\rowlabel{a:AjiliW04}AjiliW04 \href{http://dx.doi.org/10.1007/978-1-4419-8917-8_6}{AjiliW04} & \hyperref[auth:a957]{F. Ajili}, \hyperref[auth:a117]{Mark G. Wallace} & Hybrid Problem Solving in ECLiPSe & No & \cite{AjiliW04} & 2004 & Constraint and Integer Programming & null & 4 & 24 & No & n/a\\
\rowlabel{a:DannaP04}DannaP04 \href{http://dx.doi.org/10.1007/978-1-4419-8917-8_2}{DannaP04} & \hyperref[auth:a289]{E. Danna}, \hyperref[auth:a164]{Claude Le Pape} & Two Generic Schemes for Efficient and Robust Cooperative Algorithms & No & \cite{DannaP04} & 2004 & Constraints and Integer Programming & null & 2 & 34 & No & n/a\\
\rowlabel{a:DomdorfPH03}DomdorfPH03 \href{http://dx.doi.org/10.1007/978-3-642-18965-4_31}{DomdorfPH03} & \hyperref[auth:a967]{U. Domdorf}, \hyperref[auth:a441]{E. Pesch}, \hyperref[auth:a968]{To\"{a}n Phan Huy} & Machine Learning by Schedule Decomposition — Prospects for an Integration of AI and OR Techniques for Job Shop Scheduling & No & \cite{DomdorfPH03} & 2003 & Advances in Evolutionary Computing & null & 0 & 57 & No & n/a\\
\rowlabel{a:DorndorfHP99}DorndorfHP99 \href{http://dx.doi.org/10.1007/978-1-4615-5533-9_10}{DorndorfHP99} & \hyperref[auth:a908]{U. Dorndorf}, \hyperref[auth:a909]{Toàn Phan Huy}, \hyperref[auth:a441]{E. Pesch} & A Survey of Interval Capacity Consistency Tests for Time- and Resource-Constrained Scheduling & No & \cite{DorndorfHP99} & 1999 & Project Scheduling & null & 18 & 20 & No & n/a\\
\end{longtable}
}



\clearpage
\subsection{InCollection from bibtex}
{\scriptsize
\begin{longtable}{>{\raggedright\arraybackslash}p{3cm}>{\raggedright\arraybackslash}p{6cm}>{\raggedright\arraybackslash}p{6.5cm}rrrp{2.5cm}rrrrr}
\rowcolor{white}\caption{Works from bibtex (Total 6)}\\ \toprule
\rowcolor{white}Key & Authors & Title & LC & Cite & Year & \shortstack{Conference\\/Journal} & Pages & \shortstack{Nr\\Cites} & \shortstack{Nr\\Refs} & b & c \\ \midrule\endhead
\bottomrule
\endfoot
\rowlabel{a:BlazewiczEP19}BlazewiczEP19 \href{https://ideas.repec.org/h/spr/ihichp/978-3-319-99849-7_16.html}{BlazewiczEP19} & \hyperref[auth:a774]{J. Blazewicz}, \hyperref[auth:a775]{Klaus H. Ecker}, \hyperref[auth:a443]{E. Pesch}, \hyperref[auth:a776]{G. Schmidt}, \hyperref[auth:a777]{M. Sterna}, \hyperref[auth:a778]{J. Weglarz} & {Constraint Programming and Disjunctive Scheduling} & No & \cite{BlazewiczEP19} & 2019 & {Handbook on Scheduling} & 62 & 38 & 0 & No & \ref{c:BlazewiczEP19}\\
\rowlabel{a:Hooker19}Hooker19 \href{https://ideas.repec.org/h/spr/spochp/978-3-030-22788-3_1.html}{Hooker19} & \hyperref[auth:a161]{John N. Hooker} & {Logic-Based Benders Decomposition for Large-Scale Optimization} & No & \cite{Hooker19} & 2019 & {Large Scale Optimization in Supply Chains and Smart Manufacturing} & 26 & 8 & 0 & No & \ref{c:Hooker19}\\
\rowlabel{a:Bartak14}Bartak14 \href{}{Bartak14} & \hyperref[auth:a152]{R. Bart{\'{a}}k} & Planning and Scheduling & No & \cite{Bartak14} & 2014 & Computing Handbook, Third Edition: Computer Science and Software Engineering & null & 0 & 0 & No & \ref{c:Bartak14}\\
\rowlabel{a:BaptisteLPN06}BaptisteLPN06 \href{https://doi.org/10.1016/S1574-6526(06)80026-X}{BaptisteLPN06} & \hyperref[auth:a163]{P. Baptiste}, \hyperref[auth:a118]{P. Laborie}, \hyperref[auth:a164]{Claude Le Pape}, \hyperref[auth:a666]{W. Nuijten} & Constraint-Based Scheduling and Planning & No & \cite{BaptisteLPN06} & 2006 & Handbook of Constraint Programming & 39 & 30 & 25 & No & \ref{c:BaptisteLPN06}\\
\rowlabel{a:KanetAG04}KanetAG04 \href{http://www.crcnetbase.com/doi/abs/10.1201/9780203489802.ch47}{KanetAG04} & \hyperref[auth:a672]{John J. Kanet}, \hyperref[auth:a673]{S. Ahire}, \hyperref[auth:a674]{Michael F. Gorman} & Constraint Programming for Scheduling & No & \cite{KanetAG04} & 2004 & Handbook of Scheduling - Algorithms, Models, and Performance Analysis & null & 0 & 0 & No & \ref{c:KanetAG04}\\
\rowlabel{a:BreitingerL95}BreitingerL95 \href{}{BreitingerL95} & \hyperref[auth:a705]{S. Breitinger}, \hyperref[auth:a706]{Hendrik C. R. Lock} & Using Constraint Logic Programming for Industrial Scheduling Problems & No & \cite{BreitingerL95} & 1995 & Logic Programming: Formal Methods and Practical Applications, Studies in Computer Science and Artificial Intelligence & 27 & 0 & 0 & No & \ref{c:BreitingerL95}\\
\end{longtable}
}



\clearpage
{\scriptsize
\begin{longtable}{>{\raggedright\arraybackslash}p{3cm}r>{\raggedright\arraybackslash}p{4cm}p{1.5cm}p{2cm}p{1.5cm}p{1.5cm}p{1.5cm}p{1.5cm}p{2cm}p{1.5cm}rr}
\rowcolor{white}\caption{Automatically Extracted INCOLLECTION Properties (Requires Local Copy)}\\ \toprule
\rowcolor{white}Work & Pages & Concepts & Classification & Constraints & \shortstack{Prog\\Languages} & \shortstack{CP\\Systems} & Areas & Industries & Benchmarks & Algorithm & a & c\\ \midrule\endhead
\bottomrule
\endfoot
\end{longtable}
}



\clearpage
\section{Background Works}

{\scriptsize
\begin{longtable}{>{\raggedright\arraybackslash}p{3cm}>{\raggedright\arraybackslash}p{6cm}>{\raggedright\arraybackslash}p{6.5cm}rrrp{2.5cm}rrrrr}
\rowcolor{white}\caption{Works from bibtex (Total 0)}\\ \toprule
\rowcolor{white}\shortstack{Key\\Source} & Authors & Title & LC & Cite & Year & \shortstack{Conference\\/Journal\\/School} & Pages & \shortstack{Nr\\Cites} & \shortstack{Nr\\Refs} & b & c \\ \midrule\endhead
\bottomrule
\endfoot
\end{longtable}
}







\end{document}


& \href{papers/.pdf}{} & \cite{} & 2019 & CP & & \su{} & & & & & \su{} \\

& \href{articles/.pdf}{} & \cite{} & & & & & & & & & \\
& \href{articles/.pdf}{} & \cite{} & & & & & & & & & \\
& \href{articles/.pdf}{} & \cite{} & & & & & & & & & \\
& \href{articles/.pdf}{} & \cite{} & & & & & & & & & \\
& \href{articles/.pdf}{} & \cite{} & & & & & & & & & \\
& \href{articles/.pdf}{} & \cite{} & & & & & & & & & \\
& \href{articles/.pdf}{} & \cite{} & & & & & & & & & \\
& \href{articles/.pdf}{} & \cite{} & & & & & & & & & \\

& \href{papers/.pdf}{} & \cite{} & & CP & & & & & & & \\

& \href{papers/.pdf}{} & \cite{} &  & CPAIOR & & & & & & & \\
& \href{papers/.pdf}{} & \cite{} &  & CPAIOR & & & & & & & \\
& \href{papers/.pdf}{} & \cite{} &  & CPAIOR & & & & & & & \\
& \href{papers/.pdf}{} & \cite{} &  & CPAIOR & & & & & & & \\
& \href{papers/.pdf}{} & \cite{} &  & CPAIOR & & & & & & & \\
& \href{papers/.pdf}{} & \cite{} &  & CPAIOR & & & & & & & \\
& \href{papers/.pdf}{} & \cite{} &  & CPAIOR & & & & & & & \\
& \href{papers/.pdf}{} & \cite{} &  & CPAIOR & & & & & & & \\
& \href{papers/.pdf}{} & \cite{} &  & CPAIOR & & & & & & & \\
& \href{papers/.pdf}{} & \cite{} &  & CPAIOR & & & & & & & \\
& \href{papers/.pdf}{} & \cite{} &  & CPAIOR & & & & & & & \\
& \href{papers/.pdf}{} & \cite{} &  & CPAIOR & & & & & & & \\
& \href{papers/.pdf}{} & \cite{} &  & CPAIOR & & & & & & & \\
