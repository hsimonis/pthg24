\PassOptionsToPackage{table}{xcolor}
\documentclass[a4paper]{article}
\usepackage[a4paper,margin=2cm,landscape]{geometry}
\usepackage{tabularx}
\usepackage{tikz}
\usepackage{graphicx}
\usepackage{rotating}
\usepackage{float}
\usepackage{calc}
\usepackage{pdflscape}
\usepackage{booktabs}
\usepackage{colortbl}
\usepackage{longtable}
\usepackage{stackengine}
\usepackage{multicol}
%\usepackage{showkeys}
\newcounter{rowcounter}
\newcommand{\rowlabel}[1]{\refstepcounter{rowcounter}\label{#1}}
\usepackage{soul}
\definecolor{gold}{rgb}{1.0, 0.84, 0.0}
\definecolor{goldenrod}{rgb}{0.85, 0.65, 0.13}
\definecolor{darkspringgreen}{rgb}{0.09, 0.45, 0.27}
\usepackage{imakeidx}
\makeindex[columns=3, title=Alphabetical Index, intoc]

\usepackage{url}
\usepackage[pdfnewwindow]{hyperref}
\usepackage{pdfcomment}

\newcommand{\su}[1]{\Shortunderstack[l]{#1}}

\title{CP Papers on Scheduling}
\author{Helmut Simonis and Cemalettin Öztürk}
\begin{document}
\rowcolors{2}{gray!20}{white}

\maketitle
\section{Introduction}

This document shows the result of a survey on "Constraint Programming and Scheduling", which tries to find and classify all publications on the combination of these two concepts. It is based on a manually collected bibfile containing reference to relevant papers and articles, and on an automatic and manual analysis of local copies of the cited papers. For copyright reasons, we are obviously not able to distribute the collected copies, but we provide links to the original sources of the files. 

We identify the papers by a key which is the last name of the first author, the first character of the last names of all other authors, and a two digit year code for the date of publication. If multiple works would define the same key, we differentiate by adding a suffix "a", "b", etc, to the second and subsequent works found.

Most of the content of this document is generated by a Java program that parses the bib files, adds any manually extracted information, and which then extracts concept occurrences from the local copies of the works. It then produces tables and other LaTeX  artifacts that are included in a manually defined top-level document.

To add new works, first add bibtex entries for each work in the main \texttt{overview/bib.bib} file, then add local copies of the pdf of the work to the \texttt{works/} directory, using the key of the bibtex entry as the file name (plus extension .pdf), and then run the main Java program \texttt{org.insightcentre.pthg24.JfxApp} to consolidate the information and extract the relevant concepts. Finally, run \texttt{pdflatex} on the \texttt{overview/scheduling.tex} file to produce this pdf document. Manually extracted information for the files can be added in the \texttt{imports/manual.csv} file. New concepts can be added in the file \texttt{imports/concepts.json}, new concept types need to be directly defined in the Java code.

We start the document by providing a table of all defined keys in the bib file in alphabetical order. This table can be helpful to see if a candidate paper is already in the survey, it suffices to see if the key is already present, and matches the authors, title and origin of the candidate paper. In the table the hyper-link given by the key points to the local copy of the file, while the citation number links to the bibliography entry. That entry typically also contains a link to the original source of the paper.

This document heavily depends on the use of hyper links in the document, it has been tested with Acrobat Reader, other pdf reader may not use links in the same way. If you read the on-line version of the document, the links to the local files will not be active.

\clearpage
\begin{longtable}{*{6}{l}}
\rowcolor{white}\caption{Key Overview (Total: 1207)}\\ \toprule
\rowcolor{white}1 & 2 & 3 & 4 & 5 & 6\\ \midrule
\endhead
\bottomrule
\endfoot
\href{../}{2007}~\cite{2007} & \href{../works/AalianPG23.pdf}{AalianPG23}~\cite{AalianPG23} & \href{../works/AbdennadherS99.pdf}{AbdennadherS99}~\cite{AbdennadherS99} & \href{../works/Abdul-Niby2016.pdf}{Abdul-Niby2016}~\cite{Abdul-Niby2016} & \href{../works/AbidinK20.pdf}{AbidinK20}~\cite{AbidinK20} & \href{../works/AbohashimaEG21.pdf}{AbohashimaEG21}~\cite{AbohashimaEG21}\\ 
\href{../works/Abreu2023.pdf}{Abreu2023}~\cite{Abreu2023} & \href{../works/AbreuAPNM21.pdf}{AbreuAPNM21}~\cite{AbreuAPNM21} & \href{../works/AbreuN22.pdf}{AbreuN22}~\cite{AbreuN22} & \href{../works/AbreuNP23.pdf}{AbreuNP23}~\cite{AbreuNP23} & \href{../works/AbreuPNF23.pdf}{AbreuPNF23}~\cite{AbreuPNF23} & \href{../works/AbrilSB05.pdf}{AbrilSB05}~\cite{AbrilSB05}\\ 
\href{../}{Abuwarda2019}~\cite{Abuwarda2019} & \href{../works/AchterbergBKW08.pdf}{AchterbergBKW08}~\cite{AchterbergBKW08} & \href{../works/Acuna-Agost2011.pdf}{Acuna-Agost2011}~\cite{Acuna-Agost2011} & \href{../works/Acuna-AgostMFG09.pdf}{Acuna-AgostMFG09}~\cite{Acuna-AgostMFG09} & \href{../works/Adelgren2023.pdf}{Adelgren2023}~\cite{Adelgren2023} & \href{../works/AfsarVPG23.pdf}{AfsarVPG23}~\cite{AfsarVPG23}\\ 
\href{../works/AggounB93.pdf}{AggounB93}~\cite{AggounB93} & \href{../}{AggounMV08}~\cite{AggounMV08} & \href{../}{AggounV04}~\cite{AggounV04} & \href{../works/AgussurjaKL18.pdf}{AgussurjaKL18}~\cite{AgussurjaKL18} & \href{../}{Ahmadi-Javid2023}~\cite{Ahmadi-Javid2023} & \href{../works/Ahmed2006.pdf}{Ahmed2006}~\cite{Ahmed2006}\\ 
\href{../}{AjiliW04}~\cite{AjiliW04} & \href{../works/Akan2023.pdf}{Akan2023}~\cite{Akan2023} & \href{../works/AkkerDH07.pdf}{AkkerDH07}~\cite{AkkerDH07} & \href{../works/AkramNHRSA23.pdf}{AkramNHRSA23}~\cite{AkramNHRSA23} & \href{../works/Alaka21.pdf}{Alaka21}~\cite{Alaka21} & \href{../works/AlakaP23.pdf}{AlakaP23}~\cite{AlakaP23}\\ 
\href{../works/AlakaPY19.pdf}{AlakaPY19}~\cite{AlakaPY19} & \href{../works/Alesio2013.pdf}{Alesio2013}~\cite{Alesio2013} & \href{../works/AlesioBNG15.pdf}{AlesioBNG15}~\cite{AlesioBNG15} & \href{../works/AlesioNBG14.pdf}{AlesioNBG14}~\cite{AlesioNBG14} & \href{../works/AlfieriGPS23.pdf}{AlfieriGPS23}~\cite{AlfieriGPS23} & \href{../}{AlizdehS20}~\cite{AlizdehS20}\\ 
\href{../works/Amadini2014.pdf}{Amadini2014}~\cite{Amadini2014} & \href{../works/AmadiniGM16.pdf}{AmadiniGM16}~\cite{AmadiniGM16} & \href{../}{Ammar2013}~\cite{Ammar2013} & \href{../works/AngelsmarkJ00.pdf}{AngelsmarkJ00}~\cite{AngelsmarkJ00} & \href{../works/AntunesABD18.pdf}{AntunesABD18}~\cite{AntunesABD18} & \href{../works/AntunesABD20.pdf}{AntunesABD20}~\cite{AntunesABD20}\\ 
\href{../works/AntuoriHHEN20.pdf}{AntuoriHHEN20}~\cite{AntuoriHHEN20} & \href{../works/AntuoriHHEN21.pdf}{AntuoriHHEN21}~\cite{AntuoriHHEN21} & \href{../works/Apt2001.pdf}{Apt2001}~\cite{Apt2001} & \href{../works/ArbaouiY18.pdf}{ArbaouiY18}~\cite{ArbaouiY18} & \href{../}{Arkhipov19}~\cite{Arkhipov19} & \href{../works/ArkhipovBL19.pdf}{ArkhipovBL19}~\cite{ArkhipovBL19}\\ 
\href{../works/ArmstrongGOS21.pdf}{ArmstrongGOS21}~\cite{ArmstrongGOS21} & \href{../works/ArmstrongGOS22.pdf}{ArmstrongGOS22}~\cite{ArmstrongGOS22} & \href{../works/AronssonBK09.pdf}{AronssonBK09}~\cite{AronssonBK09} & \href{../works/Artigues2011.pdf}{Artigues2011}~\cite{Artigues2011} & \href{../works/ArtiguesBF04.pdf}{ArtiguesBF04}~\cite{ArtiguesBF04} & \href{../}{ArtiguesDN08}~\cite{ArtiguesDN08}\\ 
\href{../works/ArtiguesF07.pdf}{ArtiguesF07}~\cite{ArtiguesF07} & \href{../works/ArtiguesHQT21.pdf}{ArtiguesHQT21}~\cite{ArtiguesHQT21} & \href{../works/ArtiguesL14.pdf}{ArtiguesL14}~\cite{ArtiguesL14} & \href{../works/ArtiguesLH13.pdf}{ArtiguesLH13}~\cite{ArtiguesLH13} & \href{../works/ArtiguesR00.pdf}{ArtiguesR00}~\cite{ArtiguesR00} & \href{../works/ArtiouchineB05.pdf}{ArtiouchineB05}~\cite{ArtiouchineB05}\\ 
\href{../works/Astrand0F21.pdf}{Astrand0F21}~\cite{Astrand0F21} & \href{../works/Astrand2020.pdf}{Astrand2020}~\cite{Astrand2020} & \href{../works/Astrand21.pdf}{Astrand21}~\cite{Astrand21} & \href{../works/AstrandJZ18.pdf}{AstrandJZ18}~\cite{AstrandJZ18} & \href{../works/AstrandJZ20.pdf}{AstrandJZ20}~\cite{AstrandJZ20} & \href{../works/Austrin2013.pdf}{Austrin2013}~\cite{Austrin2013}\\ 
\href{../works/AwadMDMT22.pdf}{AwadMDMT22}~\cite{AwadMDMT22} & \href{../works/BadicaBI20.pdf}{BadicaBI20}~\cite{BadicaBI20} & \href{../works/BadicaBIL19.pdf}{BadicaBIL19}~\cite{BadicaBIL19} & \href{../works/BajestaniB11.pdf}{BajestaniB11}~\cite{BajestaniB11} & \href{../works/BajestaniB13.pdf}{BajestaniB13}~\cite{BajestaniB13} & \href{../works/BajestaniB15.pdf}{BajestaniB15}~\cite{BajestaniB15}\\ 
\href{../works/Balduccini11.pdf}{Balduccini11}~\cite{Balduccini11} & \href{../works/Balduccini2017.pdf}{Balduccini2017}~\cite{Balduccini2017} & \href{../}{BalochG20}~\cite{BalochG20} & \href{../works/Banaszak2008.pdf}{Banaszak2008}~\cite{Banaszak2008} & \href{../works/Banaszak2014.pdf}{Banaszak2014}~\cite{Banaszak2014} & \href{../works/BandaSC11.pdf}{BandaSC11}~\cite{BandaSC11}\\ 
\href{../works/Baptiste02.pdf}{Baptiste02}~\cite{Baptiste02} & \href{../works/Baptiste09.pdf}{Baptiste09}~\cite{Baptiste09} & \href{../works/Baptiste1998.pdf}{Baptiste1998}~\cite{Baptiste1998} & \href{../}{Baptiste2001}~\cite{Baptiste2001} & \href{../works/BaptisteB18.pdf}{BaptisteB18}~\cite{BaptisteB18} & \href{../}{BaptisteLPN06}~\cite{BaptisteLPN06}\\ 
\href{../works/BaptisteLV92.pdf}{BaptisteLV92}~\cite{BaptisteLV92} & \href{../works/BaptisteP00.pdf}{BaptisteP00}~\cite{BaptisteP00} & \href{../works/BaptisteP95.pdf}{BaptisteP95}~\cite{BaptisteP95} & \href{../works/BaptisteP97.pdf}{BaptisteP97}~\cite{BaptisteP97} & \href{../}{BaptistePN01}~\cite{BaptistePN01} & \href{../works/BaptistePN99.pdf}{BaptistePN99}~\cite{BaptistePN99}\\ 
\href{../works/Barber1993.pdf}{Barber1993}~\cite{Barber1993} & \href{../works/BarbulescuWH04.pdf}{BarbulescuWH04}~\cite{BarbulescuWH04} & \href{../works/BarlattCG08.pdf}{BarlattCG08}~\cite{BarlattCG08} & \href{../works/Bartak02.pdf}{Bartak02}~\cite{Bartak02} & \href{../works/Bartak02a.pdf}{Bartak02a}~\cite{Bartak02a} & \href{../}{Bartak14}~\cite{Bartak14}\\ 
\href{../}{Bartak2005}~\cite{Bartak2005} & \href{../works/BartakCS10.pdf}{BartakCS10}~\cite{BartakCS10} & \href{../works/BartakS11.pdf}{BartakS11}~\cite{BartakS11} & \href{../works/BartakSR08.pdf}{BartakSR08}~\cite{BartakSR08} & \href{../works/BartakSR10.pdf}{BartakSR10}~\cite{BartakSR10} & \href{../works/BartakV15.pdf}{BartakV15}~\cite{BartakV15}\\ 
\href{../works/Bartk2010.pdf}{Bartk2010}~\cite{Bartk2010} & \href{../works/BartoliniBBLM14.pdf}{BartoliniBBLM14}~\cite{BartoliniBBLM14} & \href{../works/BarzegaranZP20.pdf}{BarzegaranZP20}~\cite{BarzegaranZP20} & \href{../works/Baykan1997.pdf}{Baykan1997}~\cite{Baykan1997} & \href{../works/Beck06.pdf}{Beck06}~\cite{Beck06} & \href{../works/Beck07.pdf}{Beck07}~\cite{Beck07}\\ 
\href{../works/Beck10.pdf}{Beck10}~\cite{Beck10} & \href{../works/Beck99.pdf}{Beck99}~\cite{Beck99} & \href{../works/BeckDDF98.pdf}{BeckDDF98}~\cite{BeckDDF98} & \href{../works/BeckDF97.pdf}{BeckDF97}~\cite{BeckDF97} & \href{../works/BeckDSF97.pdf}{BeckDSF97}~\cite{BeckDSF97} & \href{../works/BeckDSF97a.pdf}{BeckDSF97a}~\cite{BeckDSF97a}\\ 
\href{../works/BeckF00.pdf}{BeckF00}~\cite{BeckF00} & \href{../works/BeckF00a.pdf}{BeckF00a}~\cite{BeckF00a} & \href{../works/BeckF98.pdf}{BeckF98}~\cite{BeckF98} & \href{../works/BeckF99.pdf}{BeckF99}~\cite{BeckF99} & \href{../works/BeckFW11.pdf}{BeckFW11}~\cite{BeckFW11} & \href{../works/BeckPS03.pdf}{BeckPS03}~\cite{BeckPS03}\\ 
\href{../works/BeckR03.pdf}{BeckR03}~\cite{BeckR03} & \href{../works/BeckW04.pdf}{BeckW04}~\cite{BeckW04} & \href{../works/BeckW05.pdf}{BeckW05}~\cite{BeckW05} & \href{../works/BeckW07.pdf}{BeckW07}~\cite{BeckW07} & \href{../works/Bedhief21.pdf}{Bedhief21}~\cite{Bedhief21} & \href{../works/BegB13.pdf}{BegB13}~\cite{BegB13}\\ 
\href{../works/BehrensLM19.pdf}{BehrensLM19}~\cite{BehrensLM19} & \href{../works/BeldiceanuC01.pdf}{BeldiceanuC01}~\cite{BeldiceanuC01} & \href{../works/BeldiceanuC02.pdf}{BeldiceanuC02}~\cite{BeldiceanuC02} & \href{../works/BeldiceanuC94.pdf}{BeldiceanuC94}~\cite{BeldiceanuC94} & \href{../works/BeldiceanuCDP11.pdf}{BeldiceanuCDP11}~\cite{BeldiceanuCDP11} & \href{../works/BeldiceanuCP08.pdf}{BeldiceanuCP08}~\cite{BeldiceanuCP08}\\ 
\href{../works/BeldiceanuP07.pdf}{BeldiceanuP07}~\cite{BeldiceanuP07} & \href{../works/BelhadjiI98.pdf}{BelhadjiI98}~\cite{BelhadjiI98} & \href{../works/Benda2019.pdf}{Benda2019}~\cite{Benda2019} & \href{../works/BenderWS21.pdf}{BenderWS21}~\cite{BenderWS21} & \href{../works/Benedetti2008.pdf}{Benedetti2008}~\cite{Benedetti2008} & \href{../works/BenediktMH20.pdf}{BenediktMH20}~\cite{BenediktMH20}\\ 
\href{../works/BenediktSMVH18.pdf}{BenediktSMVH18}~\cite{BenediktSMVH18} & \href{../works/BeniniBGM05.pdf}{BeniniBGM05}~\cite{BeniniBGM05} & \href{../works/BeniniBGM05a.pdf}{BeniniBGM05a}~\cite{BeniniBGM05a} & \href{../works/BeniniBGM06.pdf}{BeniniBGM06}~\cite{BeniniBGM06} & \href{../works/BeniniLMMR08.pdf}{BeniniLMMR08}~\cite{BeniniLMMR08} & \href{../works/BeniniLMR08.pdf}{BeniniLMR08}~\cite{BeniniLMR08}\\ 
\href{../works/BeniniLMR11.pdf}{BeniniLMR11}~\cite{BeniniLMR11} & \href{../works/BenoistGR02.pdf}{BenoistGR02}~\cite{BenoistGR02} & \href{../works/BensanaLV99.pdf}{BensanaLV99}~\cite{BensanaLV99} & \href{../}{Berbeglia2012}~\cite{Berbeglia2012} & \href{../works/Bergman2014.pdf}{Bergman2014}~\cite{Bergman2014} & \href{../works/BertholdHLMS10.pdf}{BertholdHLMS10}~\cite{BertholdHLMS10}\\ 
\href{../works/BessiereHMQW14.pdf}{BessiereHMQW14}~\cite{BessiereHMQW14} & \href{../}{Bgler2016}~\cite{Bgler2016} & \href{../}{Bgler2016a}~\cite{Bgler2016a} & \href{../works/BhatnagarKL19.pdf}{BhatnagarKL19}~\cite{BhatnagarKL19} & \href{../works/Bidot2006.pdf}{Bidot2006}~\cite{Bidot2006} & \href{../works/BidotVLB07.pdf}{BidotVLB07}~\cite{BidotVLB07}\\ 
\href{../works/BidotVLB09.pdf}{BidotVLB09}~\cite{BidotVLB09} & \href{../works/BillautHL12.pdf}{BillautHL12}~\cite{BillautHL12} & \href{../works/Biswas2010.pdf}{Biswas2010}~\cite{Biswas2010} & \href{../works/Bit-Monnot23.pdf}{Bit-Monnot23}~\cite{Bit-Monnot23} & \href{../works/Bittle2009.pdf}{Bittle2009}~\cite{Bittle2009} & \href{../works/Bixby2006.pdf}{Bixby2006}~\cite{Bixby2006}\\ 
\href{../works/BlazewiczDP96.pdf}{BlazewiczDP96}~\cite{BlazewiczDP96} & \href{../}{BlazewiczEP19}~\cite{BlazewiczEP19} & \href{../works/Bley2023.pdf}{Bley2023}~\cite{Bley2023} & \href{../works/BlomBPS14.pdf}{BlomBPS14}~\cite{BlomBPS14} & \href{../works/BlomPS16.pdf}{BlomPS16}~\cite{BlomPS16} & \href{../works/Bocewicz2009.pdf}{Bocewicz2009}~\cite{Bocewicz2009}\\ 
\href{../works/Bocewicz2013.pdf}{Bocewicz2013}~\cite{Bocewicz2013} & \href{../works/Bocewicz2021.pdf}{Bocewicz2021}~\cite{Bocewicz2021} & \href{../works/Bocewicz2023.pdf}{Bocewicz2023}~\cite{Bocewicz2023} & \href{../works/BocewiczBB09.pdf}{BocewiczBB09}~\cite{BocewiczBB09} & \href{../}{BockmayrK98}~\cite{BockmayrK98} & \href{../works/BockmayrP06.pdf}{BockmayrP06}~\cite{BockmayrP06}\\ 
\href{../works/Boek2016.pdf}{Boek2016}~\cite{Boek2016} & \href{../works/BofillCGGPSV23.pdf}{BofillCGGPSV23}~\cite{BofillCGGPSV23} & \href{../works/BofillCSV17.pdf}{BofillCSV17}~\cite{BofillCSV17} & \href{../works/BofillCSV17a.pdf}{BofillCSV17a}~\cite{BofillCSV17a} & \href{../works/BofillEGPSV14.pdf}{BofillEGPSV14}~\cite{BofillEGPSV14} & \href{../works/BofillGSV15.pdf}{BofillGSV15}~\cite{BofillGSV15}\\ 
\href{../works/BogaerdtW19.pdf}{BogaerdtW19}~\cite{BogaerdtW19} & \href{../works/Bonfietti16.pdf}{Bonfietti16}~\cite{Bonfietti16} & \href{../works/BonfiettiLBM11.pdf}{BonfiettiLBM11}~\cite{BonfiettiLBM11} & \href{../works/BonfiettiLBM12.pdf}{BonfiettiLBM12}~\cite{BonfiettiLBM12} & \href{../works/BonfiettiLBM14.pdf}{BonfiettiLBM14}~\cite{BonfiettiLBM14} & \href{../works/BonfiettiLM13.pdf}{BonfiettiLM13}~\cite{BonfiettiLM13}\\ 
\href{../works/BonfiettiLM14.pdf}{BonfiettiLM14}~\cite{BonfiettiLM14} & \href{../works/BonfiettiM12.pdf}{BonfiettiM12}~\cite{BonfiettiM12} & \href{../works/BonfiettiZLM16.pdf}{BonfiettiZLM16}~\cite{BonfiettiZLM16} & \href{../works/BonninMNE24.pdf}{BonninMNE24}~\cite{BonninMNE24} & \href{../works/BoothNB16.pdf}{BoothNB16}~\cite{BoothNB16} & \href{../works/BoothTNB16.pdf}{BoothTNB16}~\cite{BoothTNB16}\\ 
\href{../works/BorghesiBLMB18.pdf}{BorghesiBLMB18}~\cite{BorghesiBLMB18} & \href{../works/BosiM2001.pdf}{BosiM2001}~\cite{BosiM2001} & \href{../}{BoucherBVBL97}~\cite{BoucherBVBL97} & \href{../works/BoudreaultSLQ22.pdf}{BoudreaultSLQ22}~\cite{BoudreaultSLQ22} & \href{../works/BourdaisGP03.pdf}{BourdaisGP03}~\cite{BourdaisGP03} & \href{../}{Bourdeaudhuy2011}~\cite{Bourdeaudhuy2011}\\ 
\href{../works/BourreauGGLT22.pdf}{BourreauGGLT22}~\cite{BourreauGGLT22} & \href{../works/Braune2022.pdf}{Braune2022}~\cite{Braune2022} & \href{../}{Breitinger1994}~\cite{Breitinger1994} & \href{../}{BreitingerL95}~\cite{BreitingerL95} & \href{../}{BriandHHL08}~\cite{BriandHHL08} & \href{../works/BridiBLMB16.pdf}{BridiBLMB16}~\cite{BridiBLMB16}\\ 
\href{../works/BridiLBBM16.pdf}{BridiLBBM16}~\cite{BridiLBBM16} & \href{../works/Brucker2002.pdf}{Brucker2002}~\cite{Brucker2002} & \href{../works/BruckerK00.pdf}{BruckerK00}~\cite{BruckerK00} & \href{../works/BrusoniCLMMT96.pdf}{BrusoniCLMMT96}~\cite{BrusoniCLMMT96} & \href{../works/BukchinR18.pdf}{BukchinR18}~\cite{BukchinR18} & \href{../works/BulckG22.pdf}{BulckG22}~\cite{BulckG22}\\ 
\href{../works/BurtLPS15.pdf}{BurtLPS15}~\cite{BurtLPS15} & \href{../}{Bzdyra2015}~\cite{Bzdyra2015} & \href{../works/Caballero19.pdf}{Caballero19}~\cite{Caballero19} & \href{../works/Caballero23.pdf}{Caballero23}~\cite{Caballero23} & \href{../works/CambazardHDJT04.pdf}{CambazardHDJT04}~\cite{CambazardHDJT04} & \href{../works/CambazardJ05.pdf}{CambazardJ05}~\cite{CambazardJ05}\\ 
\href{../works/CampeauG22.pdf}{CampeauG22}~\cite{CampeauG22} & \href{../works/Capone2009.pdf}{Capone2009}~\cite{Capone2009} & \href{../works/CappartS17.pdf}{CappartS17}~\cite{CappartS17} & \href{../works/CappartTSR18.pdf}{CappartTSR18}~\cite{CappartTSR18} & \href{../works/CarchraeB09.pdf}{CarchraeB09}~\cite{CarchraeB09} & \href{../works/CarchraeBF05.pdf}{CarchraeBF05}~\cite{CarchraeBF05}\\ 
\href{../works/Caricato2020.pdf}{Caricato2020}~\cite{Caricato2020} & \href{../works/CarlierPSJ20.pdf}{CarlierPSJ20}~\cite{CarlierPSJ20} & \href{../}{CarlierSJP21}~\cite{CarlierSJP21} & \href{../works/CarlssonJL17.pdf}{CarlssonJL17}~\cite{CarlssonJL17} & \href{../works/CarlssonKA99.pdf}{CarlssonKA99}~\cite{CarlssonKA99} & \href{../}{Caseau1996}~\cite{Caseau1996}\\ 
\href{../works/Caseau2001.pdf}{Caseau2001}~\cite{Caseau2001} & \href{../works/Caseau97.pdf}{Caseau97}~\cite{Caseau97} & \href{../}{CastroGR10}~\cite{CastroGR10} & \href{../works/CatusseCBL16.pdf}{CatusseCBL16}~\cite{CatusseCBL16} & \href{../works/CauwelaertDMS16.pdf}{CauwelaertDMS16}~\cite{CauwelaertDMS16} & \href{../works/CauwelaertDS20.pdf}{CauwelaertDS20}~\cite{CauwelaertDS20}\\ 
\href{../works/CauwelaertLS15.pdf}{CauwelaertLS15}~\cite{CauwelaertLS15} & \href{../works/CauwelaertLS18.pdf}{CauwelaertLS18}~\cite{CauwelaertLS18} & \href{../works/CestaOF99.pdf}{CestaOF99}~\cite{CestaOF99} & \href{../}{CestaOPS14}~\cite{CestaOPS14} & \href{../works/CestaOS00.pdf}{CestaOS00}~\cite{CestaOS00} & \href{../works/CestaOS98.pdf}{CestaOS98}~\cite{CestaOS98}\\ 
\href{../works/Chakrabortty2019.pdf}{Chakrabortty2019}~\cite{Chakrabortty2019} & \href{../works/Chaleshtarti2014.pdf}{Chaleshtarti2014}~\cite{Chaleshtarti2014} & \href{../works/Chan2001.pdf}{Chan2001}~\cite{Chan2001} & \href{../}{Chan2002}~\cite{Chan2002} & \href{../works/ChapadosJR11.pdf}{ChapadosJR11}~\cite{ChapadosJR11} & \href{../works/Chen2021.pdf}{Chen2021}~\cite{Chen2021}\\ 
\href{../works/ChenGPSH10.pdf}{ChenGPSH10}~\cite{ChenGPSH10} & \href{../works/Choi2007.pdf}{Choi2007}~\cite{Choi2007} & \href{../works/Choosri2011.pdf}{Choosri2011}~\cite{Choosri2011} & \href{../works/ChuGNSW13.pdf}{ChuGNSW13}~\cite{ChuGNSW13} & \href{../works/ChuX05.pdf}{ChuX05}~\cite{ChuX05} & \href{../works/Chun2011.pdf}{Chun2011}~\cite{Chun2011}\\ 
\href{../works/ChunCTY99.pdf}{ChunCTY99}~\cite{ChunCTY99} & \href{../works/ChunS14.pdf}{ChunS14}~\cite{ChunS14} & \href{../works/CilKLO22.pdf}{CilKLO22}~\cite{CilKLO22} & \href{../works/CireCH13.pdf}{CireCH13}~\cite{CireCH13} & \href{../works/CireCH16.pdf}{CireCH16}~\cite{CireCH16} & \href{../}{Clautiaux2013}~\cite{Clautiaux2013}\\ 
\href{../works/ClautiauxJCM08.pdf}{ClautiauxJCM08}~\cite{ClautiauxJCM08} & \href{../}{Clearwater1991}~\cite{Clearwater1991} & \href{../works/Clercq12.pdf}{Clercq12}~\cite{Clercq12} & \href{../works/ClercqPBJ11.pdf}{ClercqPBJ11}~\cite{ClercqPBJ11} & \href{../works/CobanH10.pdf}{CobanH10}~\cite{CobanH10} & \href{../works/CobanH11.pdf}{CobanH11}~\cite{CobanH11}\\ 
\href{../works/Coelho2011.pdf}{Coelho2011}~\cite{Coelho2011} & \href{../works/CohenHB17.pdf}{CohenHB17}~\cite{CohenHB17} & \href{../works/ColT19.pdf}{ColT19}~\cite{ColT19} & \href{../works/ColT2019a.pdf}{ColT2019a}~\cite{ColT2019a} & \href{../works/ColT22.pdf}{ColT22}~\cite{ColT22} & \href{../works/Colombani96.pdf}{Colombani96}~\cite{Colombani96}\\ 
\href{../works/CorreaLR07.pdf}{CorreaLR07}~\cite{CorreaLR07} & \href{../works/Cox2019.pdf}{Cox2019}~\cite{Cox2019} & \href{../works/CrawfordB94.pdf}{CrawfordB94}~\cite{CrawfordB94} & \href{../works/Cucu-Grosjean2009.pdf}{Cucu-Grosjean2009}~\cite{Cucu-Grosjean2009} & \href{../works/CzerniachowskaWZ23.pdf}{CzerniachowskaWZ23}~\cite{CzerniachowskaWZ23} & \href{../works/Daneshamooz2021.pdf}{Daneshamooz2021}~\cite{Daneshamooz2021}\\ 
\href{../works/DannaP03.pdf}{DannaP03}~\cite{DannaP03} & \href{../}{DannaP04}~\cite{DannaP04} & \href{../works/Danzinger2020.pdf}{Danzinger2020}~\cite{Danzinger2020} & \href{../works/Danzinger2023.pdf}{Danzinger2023}~\cite{Danzinger2023} & \href{../works/Darby-DowmanLMZ97.pdf}{Darby-DowmanLMZ97}~\cite{Darby-DowmanLMZ97} & \href{../}{DarbyDowmanL98}~\cite{DarbyDowmanL98}\\ 
\href{../}{Dasygenis2018}~\cite{Dasygenis2018} & \href{../works/Davenport10.pdf}{Davenport10}~\cite{Davenport10} & \href{../works/DavenportKRSH07.pdf}{DavenportKRSH07}~\cite{DavenportKRSH07} & \href{../works/Davis87.pdf}{Davis87}~\cite{Davis87} & \href{../works/Deblaere2011.pdf}{Deblaere2011}~\cite{Deblaere2011} & \href{../works/Dejemeppe16.pdf}{Dejemeppe16}~\cite{Dejemeppe16}\\ 
\href{../works/DejemeppeCS15.pdf}{DejemeppeCS15}~\cite{DejemeppeCS15} & \href{../works/DejemeppeD14.pdf}{DejemeppeD14}~\cite{DejemeppeD14} & \href{../works/Demassey03.pdf}{Demassey03}~\cite{Demassey03} & \href{../works/DemasseyAM05.pdf}{DemasseyAM05}~\cite{DemasseyAM05} & \href{../}{Demeulemeester1992}~\cite{Demeulemeester1992} & \href{../}{Demeulemeester1997}~\cite{Demeulemeester1997}\\ 
\href{../works/DemirovicS18.pdf}{DemirovicS18}~\cite{DemirovicS18} & \href{../works/Derrien15.pdf}{Derrien15}~\cite{Derrien15} & \href{../works/DerrienP14.pdf}{DerrienP14}~\cite{DerrienP14} & \href{../works/DerrienPZ14.pdf}{DerrienPZ14}~\cite{DerrienPZ14} & \href{../works/DilkinaDH05.pdf}{DilkinaDH05}~\cite{DilkinaDH05} & \href{../works/DilkinaH04.pdf}{DilkinaH04}~\cite{DilkinaH04}\\ 
\href{../works/Dimny2023.pdf}{Dimny2023}~\cite{Dimny2023} & \href{../works/DincbasS91.pdf}{DincbasS91}~\cite{DincbasS91} & \href{../works/DincbasSH90.pdf}{DincbasSH90}~\cite{DincbasSH90} & \href{../works/DoRZ08.pdf}{DoRZ08}~\cite{DoRZ08} & \href{../works/Dolabi2014.pdf}{Dolabi2014}~\cite{Dolabi2014} & \href{../}{DomdorfPH03}~\cite{DomdorfPH03}\\ 
\href{../works/Dong2010.pdf}{Dong2010}~\cite{Dong2010} & \href{../works/Doolaard2022.pdf}{Doolaard2022}~\cite{Doolaard2022} & \href{../works/DoomsH08.pdf}{DoomsH08}~\cite{DoomsH08} & \href{../works/Dorndorf2000.pdf}{Dorndorf2000}~\cite{Dorndorf2000} & \href{../}{Dorndorf2000a}~\cite{Dorndorf2000a} & \href{../}{DorndorfHP99}~\cite{DorndorfHP99}\\ 
\href{../}{DorndorfPH99}~\cite{DorndorfPH99} & \href{../works/DoulabiRP14.pdf}{DoulabiRP14}~\cite{DoulabiRP14} & \href{../works/DoulabiRP16.pdf}{DoulabiRP16}~\cite{DoulabiRP16} & \href{../works/DraperJCJ99.pdf}{DraperJCJ99}~\cite{DraperJCJ99} & \href{../works/EastonNT02.pdf}{EastonNT02}~\cite{EastonNT02} & \href{../works/Edis21.pdf}{Edis21}~\cite{Edis21}\\ 
\href{../works/EdisO11.pdf}{EdisO11}~\cite{EdisO11} & \href{../}{EdisO11a}~\cite{EdisO11a} & \href{../}{EdwardsBSE19}~\cite{EdwardsBSE19} & \href{../works/EfthymiouY23.pdf}{EfthymiouY23}~\cite{EfthymiouY23} & \href{../}{Eirinakis2012}~\cite{Eirinakis2012} & \href{../works/Eiter2021.pdf}{Eiter2021}~\cite{Eiter2021}\\ 
\href{../works/Eiter2023.pdf}{Eiter2023}~\cite{Eiter2023} & \href{../works/El-Kholany2022.pdf}{El-Kholany2022}~\cite{El-Kholany2022} & \href{../works/ElciOH22.pdf}{ElciOH22}~\cite{ElciOH22} & \href{../works/ElfJR03.pdf}{ElfJR03}~\cite{ElfJR03} & \href{../works/ElhouraniDM07.pdf}{ElhouraniDM07}~\cite{ElhouraniDM07} & \href{../works/Elkhyari03.pdf}{Elkhyari03}~\cite{Elkhyari03}\\ 
\href{../works/Elkhyari2006.pdf}{Elkhyari2006}~\cite{Elkhyari2006} & \href{../works/Elkhyari2017.pdf}{Elkhyari2017}~\cite{Elkhyari2017} & \href{../works/ElkhyariGJ02.pdf}{ElkhyariGJ02}~\cite{ElkhyariGJ02} & \href{../works/ElkhyariGJ02a.pdf}{ElkhyariGJ02a}~\cite{ElkhyariGJ02a} & \href{../}{Elmaghraby1992}~\cite{Elmaghraby1992} & \href{../works/EmdeZD22.pdf}{EmdeZD22}~\cite{EmdeZD22}\\ 
\href{../works/Emeretlis2014.pdf}{Emeretlis2014}~\cite{Emeretlis2014} & \href{../works/EmeretlisTAV17.pdf}{EmeretlisTAV17}~\cite{EmeretlisTAV17} & \href{../works/EreminW01.pdf}{EreminW01}~\cite{EreminW01} & \href{../works/ErkingerM17.pdf}{ErkingerM17}~\cite{ErkingerM17} & \href{../works/ErtlK91.pdf}{ErtlK91}~\cite{ErtlK91} & \href{../works/EscobetPQPRA19.pdf}{EscobetPQPRA19}~\cite{EscobetPQPRA19}\\ 
\href{../works/EskeyZ90.pdf}{EskeyZ90}~\cite{EskeyZ90} & \href{../}{EsquirolLH2008}~\cite{EsquirolLH2008} & \href{../works/EtminaniesfahaniGNMS22.pdf}{EtminaniesfahaniGNMS22}~\cite{EtminaniesfahaniGNMS22} & \href{../works/EvenSH15.pdf}{EvenSH15}~\cite{EvenSH15} & \href{../works/EvenSH15a.pdf}{EvenSH15a}~\cite{EvenSH15a} & \href{../works/FachiniA20.pdf}{FachiniA20}~\cite{FachiniA20}\\ 
\href{../works/Fahimi16.pdf}{Fahimi16}~\cite{Fahimi16} & \href{../works/FahimiOQ18.pdf}{FahimiOQ18}~\cite{FahimiOQ18} & \href{../}{FahimiQ23}~\cite{FahimiQ23} & \href{../works/FalaschiGMP97.pdf}{FalaschiGMP97}~\cite{FalaschiGMP97} & \href{../works/FallahiAC20.pdf}{FallahiAC20}~\cite{FallahiAC20} & \href{../works/FalqueALM24.pdf}{FalqueALM24}~\cite{FalqueALM24}\\ 
\href{../works/FanXG21.pdf}{FanXG21}~\cite{FanXG21} & \href{../works/Farias2001.pdf}{Farias2001}~\cite{Farias2001} & \href{../works/FarsiTM22.pdf}{FarsiTM22}~\cite{FarsiTM22} & \href{../works/Fatemi-AnarakiTFV23.pdf}{Fatemi-AnarakiTFV23}~\cite{Fatemi-AnarakiTFV23} & \href{../works/FeldmanG89.pdf}{FeldmanG89}~\cite{FeldmanG89} & \href{../}{FelizariAL09}~\cite{FelizariAL09}\\ 
\href{../works/Feng2022.pdf}{Feng2022}~\cite{Feng2022} & \href{../works/FetgoD22.pdf}{FetgoD22}~\cite{FetgoD22} & \href{../works/Filho2012.pdf}{Filho2012}~\cite{Filho2012} & \href{../}{Fisher1985}~\cite{Fisher1985} & \href{../works/FocacciLN00.pdf}{FocacciLN00}~\cite{FocacciLN00} & \href{../works/FontaineMH16.pdf}{FontaineMH16}~\cite{FontaineMH16}\\ 
\href{../works/ForbesHJST24.pdf}{ForbesHJST24}~\cite{ForbesHJST24} & \href{../works/FortinZDF05.pdf}{FortinZDF05}~\cite{FortinZDF05} & \href{../works/FoxAS82.pdf}{FoxAS82}~\cite{FoxAS82} & \href{../works/FoxS90.pdf}{FoxS90}~\cite{FoxS90} & \href{../works/FrankDT16.pdf}{FrankDT16}~\cite{FrankDT16} & \href{../works/FrankK03.pdf}{FrankK03}~\cite{FrankK03}\\ 
\href{../works/FrankK05.pdf}{FrankK05}~\cite{FrankK05} & \href{../}{Freuder1994}~\cite{Freuder1994} & \href{../}{FriedrichFMRSST14}~\cite{FriedrichFMRSST14} & \href{../works/FrimodigECM23.pdf}{FrimodigECM23}~\cite{FrimodigECM23} & \href{../works/FrimodigS19.pdf}{FrimodigS19}~\cite{FrimodigS19} & \href{../works/Frisch2006.pdf}{Frisch2006}~\cite{Frisch2006}\\ 
\href{../works/Froger16.pdf}{Froger16}~\cite{Froger16} & \href{../works/FrohnerTR19.pdf}{FrohnerTR19}~\cite{FrohnerTR19} & \href{../works/FrostD98.pdf}{FrostD98}~\cite{FrostD98} & \href{../works/FukunagaHFAMN02.pdf}{FukunagaHFAMN02}~\cite{FukunagaHFAMN02} & \href{../works/Galipienso2001.pdf}{Galipienso2001}~\cite{Galipienso2001} & \href{../works/GalleguillosKSB19.pdf}{GalleguillosKSB19}~\cite{GalleguillosKSB19}\\ 
\href{../works/Gao2018.pdf}{Gao2018}~\cite{Gao2018} & \href{../works/Gao2022.pdf}{Gao2022}~\cite{Gao2022} & \href{../works/GarcaNieves2018.pdf}{GarcaNieves2018}~\cite{GarcaNieves2018} & \href{../works/GarganiR07.pdf}{GarganiR07}~\cite{GarganiR07} & \href{../works/GarridoAO09.pdf}{GarridoAO09}~\cite{GarridoAO09} & \href{../works/GarridoOS08.pdf}{GarridoOS08}~\cite{GarridoOS08}\\ 
\href{../works/Gaspero2014.pdf}{Gaspero2014}~\cite{Gaspero2014} & \href{../works/GayHLS15.pdf}{GayHLS15}~\cite{GayHLS15} & \href{../works/GayHS15.pdf}{GayHS15}~\cite{GayHS15} & \href{../works/GayHS15a.pdf}{GayHS15a}~\cite{GayHS15a} & \href{../works/GaySS14.pdf}{GaySS14}~\cite{GaySS14} & \href{../works/GedikKBR17.pdf}{GedikKBR17}~\cite{GedikKBR17}\\ 
\href{../works/GedikKEK18.pdf}{GedikKEK18}~\cite{GedikKEK18} & \href{../works/GeibingerKKMMW21.pdf}{GeibingerKKMMW21}~\cite{GeibingerKKMMW21} & \href{../works/GeibingerMM19.pdf}{GeibingerMM19}~\cite{GeibingerMM19} & \href{../works/GeibingerMM21.pdf}{GeibingerMM21}~\cite{GeibingerMM21} & \href{../works/Geiger2019.pdf}{Geiger2019}~\cite{Geiger2019} & \href{../works/GeitzGSSW22.pdf}{GeitzGSSW22}~\cite{GeitzGSSW22}\\ 
\href{../works/GelainPRVW17.pdf}{GelainPRVW17}~\cite{GelainPRVW17} & \href{../works/Gembarski2022.pdf}{Gembarski2022}~\cite{Gembarski2022} & \href{../works/Gent1996.pdf}{Gent1996}~\cite{Gent1996} & \href{../works/German18.pdf}{German18}~\cite{German18} & \href{../works/Geske05.pdf}{Geske05}~\cite{Geske05} & \href{../works/GetoorOFC97.pdf}{GetoorOFC97}~\cite{GetoorOFC97}\\ 
\href{../works/GhandehariK22.pdf}{GhandehariK22}~\cite{GhandehariK22} & \href{../}{GhasemiMH23}~\cite{GhasemiMH23} & \href{../works/GilesH16.pdf}{GilesH16}~\cite{GilesH16} & \href{../works/GingrasQ16.pdf}{GingrasQ16}~\cite{GingrasQ16} & \href{../works/GlobusCLP04.pdf}{GlobusCLP04}~\cite{GlobusCLP04} & \href{../works/GodardLN05.pdf}{GodardLN05}~\cite{GodardLN05}\\ 
\href{../works/Godet21a.pdf}{Godet21a}~\cite{Godet21a} & \href{../works/GodetLHS20.pdf}{GodetLHS20}~\cite{GodetLHS20} & \href{../works/GoelSHFS15.pdf}{GoelSHFS15}~\cite{GoelSHFS15} & \href{../works/GokGSTO20.pdf}{GokGSTO20}~\cite{GokGSTO20} & \href{../works/GokPTGO23.pdf}{GokPTGO23}~\cite{GokPTGO23} & \href{../works/Gokgur2022.pdf}{Gokgur2022}~\cite{Gokgur2022}\\ 
\href{../works/GokgurHO18.pdf}{GokgurHO18}~\cite{GokgurHO18} & \href{../works/GoldwaserS17.pdf}{GoldwaserS17}~\cite{GoldwaserS17} & \href{../works/GoldwaserS18.pdf}{GoldwaserS18}~\cite{GoldwaserS18} & \href{../works/Goltz95.pdf}{Goltz95}~\cite{Goltz95} & \href{../works/GombolayWS18.pdf}{GombolayWS18}~\cite{GombolayWS18} & \href{../works/GomesHS06.pdf}{GomesHS06}~\cite{GomesHS06}\\ 
\href{../works/GomesM17.pdf}{GomesM17}~\cite{GomesM17} & \href{../}{GongLMW09}~\cite{GongLMW09} & \href{../works/Gonzlez2017.pdf}{Gonzlez2017}~\cite{Gonzlez2017} & \href{../works/GrimesH10.pdf}{GrimesH10}~\cite{GrimesH10} & \href{../works/GrimesH11.pdf}{GrimesH11}~\cite{GrimesH11} & \href{../works/GrimesH15.pdf}{GrimesH15}~\cite{GrimesH15}\\ 
\href{../works/GrimesHM09.pdf}{GrimesHM09}~\cite{GrimesHM09} & \href{../works/GrimesIOS14.pdf}{GrimesIOS14}~\cite{GrimesIOS14} & \href{../works/Groleaz21.pdf}{Groleaz21}~\cite{Groleaz21} & \href{../works/GroleazNS20.pdf}{GroleazNS20}~\cite{GroleazNS20} & \href{../works/GroleazNS20a.pdf}{GroleazNS20a}~\cite{GroleazNS20a} & \href{../works/Gronkvist06.pdf}{Gronkvist06}~\cite{Gronkvist06}\\ 
\href{../works/GruianK98.pdf}{GruianK98}~\cite{GruianK98} & \href{../works/Grzegorz2021.pdf}{Grzegorz2021}~\cite{Grzegorz2021} & \href{../works/GuSS13.pdf}{GuSS13}~\cite{GuSS13} & \href{../}{GuSSWC14}~\cite{GuSSWC14} & \href{../works/GuSW12.pdf}{GuSW12}~\cite{GuSW12} & \href{../works/Guerinik1995.pdf}{Guerinik1995}~\cite{Guerinik1995}\\ 
\href{../}{Guimarans2013}~\cite{Guimarans2013} & \href{../}{GunerGSKD23}~\cite{GunerGSKD23} & \href{../}{GuoHLW20}~\cite{GuoHLW20} & \href{../works/GuoZ23.pdf}{GuoZ23}~\cite{GuoZ23} & \href{../works/GurEA19.pdf}{GurEA19}~\cite{GurEA19} & \href{../works/GurPAE23.pdf}{GurPAE23}~\cite{GurPAE23}\\ 
\href{../works/GuyonLPR12.pdf}{GuyonLPR12}~\cite{GuyonLPR12} & \href{../works/Hachemi2008.pdf}{Hachemi2008}~\cite{Hachemi2008} & \href{../works/Hachemi2009.pdf}{Hachemi2009}~\cite{Hachemi2009} & \href{../works/HachemiGR11.pdf}{HachemiGR11}~\cite{HachemiGR11} & \href{../works/Hajji2023.pdf}{Hajji2023}~\cite{Hajji2023} & \href{../works/Ham18.pdf}{Ham18}~\cite{Ham18}\\ 
\href{../works/Ham18a.pdf}{Ham18a}~\cite{Ham18a} & \href{../}{Ham20}~\cite{Ham20} & \href{../works/Ham20a.pdf}{Ham20a}~\cite{Ham20a} & \href{../works/HamC16.pdf}{HamC16}~\cite{HamC16} & \href{../works/HamFC17.pdf}{HamFC17}~\cite{HamFC17} & \href{../works/HamP21.pdf}{HamP21}~\cite{HamP21}\\ 
\href{../works/HamPK21.pdf}{HamPK21}~\cite{HamPK21} & \href{../works/HamdiL13.pdf}{HamdiL13}~\cite{HamdiL13} & \href{../works/Hamscher91.pdf}{Hamscher91}~\cite{Hamscher91} & \href{../works/Han2014.pdf}{Han2014}~\cite{Han2014} & \href{../works/HanenKP21.pdf}{HanenKP21}~\cite{HanenKP21} & \href{../works/Hannebauer2001.pdf}{Hannebauer2001}~\cite{Hannebauer2001}\\ 
\href{../}{Harjunkoski2001}~\cite{Harjunkoski2001} & \href{../works/HarjunkoskiG02.pdf}{HarjunkoskiG02}~\cite{HarjunkoskiG02} & \href{../works/HarjunkoskiJG00.pdf}{HarjunkoskiJG00}~\cite{HarjunkoskiJG00} & \href{../works/HarjunkoskiMBC14.pdf}{HarjunkoskiMBC14}~\cite{HarjunkoskiMBC14} & \href{../}{Hat2011}~\cite{Hat2011} & \href{../works/HauderBRPA20.pdf}{HauderBRPA20}~\cite{HauderBRPA20}\\ 
\href{../works/He0GLW18.pdf}{He0GLW18}~\cite{He0GLW18} & \href{../works/He2019.pdf}{He2019}~\cite{He2019} & \href{../works/HebrardALLCMR22.pdf}{HebrardALLCMR22}~\cite{HebrardALLCMR22} & \href{../works/HebrardHJMPV16.pdf}{HebrardHJMPV16}~\cite{HebrardHJMPV16} & \href{../works/HebrardTW05.pdf}{HebrardTW05}~\cite{HebrardTW05} & \href{../works/HechingH16.pdf}{HechingH16}~\cite{HechingH16}\\ 
\href{../}{HechingHK19}~\cite{HechingHK19} & \href{../works/HeckmanB11.pdf}{HeckmanB11}~\cite{HeckmanB11} & \href{../works/HeinzB12.pdf}{HeinzB12}~\cite{HeinzB12} & \href{../works/HeinzKB13.pdf}{HeinzKB13}~\cite{HeinzKB13} & \href{../works/HeinzNVH22.pdf}{HeinzNVH22}~\cite{HeinzNVH22} & \href{../works/HeinzS11.pdf}{HeinzS11}~\cite{HeinzS11}\\ 
\href{../works/HeinzSB13.pdf}{HeinzSB13}~\cite{HeinzSB13} & \href{../works/HeinzSSW12.pdf}{HeinzSSW12}~\cite{HeinzSSW12} & \href{../works/HeipckeCCS00.pdf}{HeipckeCCS00}~\cite{HeipckeCCS00} & \href{../works/Hentenryck2000.pdf}{Hentenryck2000}~\cite{Hentenryck2000} & \href{../works/HentenryckM04.pdf}{HentenryckM04}~\cite{HentenryckM04} & \href{../works/HentenryckM08.pdf}{HentenryckM08}~\cite{HentenryckM08}\\ 
\href{../}{Henz01}~\cite{Henz01} & \href{../works/HenzMT04.pdf}{HenzMT04}~\cite{HenzMT04} & \href{../works/HermenierDL11.pdf}{HermenierDL11}~\cite{HermenierDL11} & \href{../}{HillBCGN22}~\cite{HillBCGN22} & \href{../works/HillTV21.pdf}{HillTV21}~\cite{HillTV21} & \href{../works/Hindi2004.pdf}{Hindi2004}~\cite{Hindi2004}\\ 
\href{../works/HladikCDJ08.pdf}{HladikCDJ08}~\cite{HladikCDJ08} & \href{../works/HoYCLLCLC18.pdf}{HoYCLLCLC18}~\cite{HoYCLLCLC18} & \href{../works/Hoc2012.pdf}{Hoc2012}~\cite{Hoc2012} & \href{../works/HoeveGSL07.pdf}{HoeveGSL07}~\cite{HoeveGSL07} & \href{../}{Hofe2001}~\cite{Hofe2001} & \href{../}{Hooker00}~\cite{Hooker00}\\ 
\href{../}{Hooker02}~\cite{Hooker02} & \href{../works/Hooker04.pdf}{Hooker04}~\cite{Hooker04} & \href{../works/Hooker05.pdf}{Hooker05}~\cite{Hooker05} & \href{../works/Hooker05a.pdf}{Hooker05a}~\cite{Hooker05a} & \href{../works/Hooker05b.pdf}{Hooker05b}~\cite{Hooker05b} & \href{../works/Hooker06.pdf}{Hooker06}~\cite{Hooker06}\\ 
\href{../}{Hooker06a}~\cite{Hooker06a} & \href{../works/Hooker07.pdf}{Hooker07}~\cite{Hooker07} & \href{../}{Hooker10}~\cite{Hooker10} & \href{../works/Hooker17.pdf}{Hooker17}~\cite{Hooker17} & \href{../works/Hooker19.pdf}{Hooker19}~\cite{Hooker19} & \href{../works/HookerH17.pdf}{HookerH17}~\cite{HookerH17}\\ 
\href{../works/HookerO03.pdf}{HookerO03}~\cite{HookerO03} & \href{../works/HookerO99.pdf}{HookerO99}~\cite{HookerO99} & \href{../works/HookerOTK00.pdf}{HookerOTK00}~\cite{HookerOTK00} & \href{../works/HookerY02.pdf}{HookerY02}~\cite{HookerY02} & \href{../works/Hosseinian2019.pdf}{Hosseinian2019}~\cite{Hosseinian2019} & \href{../works/Hosseinian2021.pdf}{Hosseinian2021}~\cite{Hosseinian2021}\\ 
\href{../works/HoundjiSW19.pdf}{HoundjiSW19}~\cite{HoundjiSW19} & \href{../works/HoundjiSWD14.pdf}{HoundjiSWD14}~\cite{HoundjiSWD14} & \href{../works/Hu2009.pdf}{Hu2009}~\cite{Hu2009} & \href{../works/Huang2000.pdf}{Huang2000}~\cite{Huang2000} & \href{../works/HubnerGSV21.pdf}{HubnerGSV21}~\cite{HubnerGSV21} & \href{../works/Hunsberger08.pdf}{Hunsberger08}~\cite{Hunsberger08}\\ 
\href{../works/HurleyOS16.pdf}{HurleyOS16}~\cite{HurleyOS16} & \href{../works/Icmeli1993.pdf}{Icmeli1993}~\cite{Icmeli1993} & \href{../}{Icmeli1996}~\cite{Icmeli1996} & \href{../works/IfrimOS12.pdf}{IfrimOS12}~\cite{IfrimOS12} & \href{../works/IklassovMR023.pdf}{IklassovMR023}~\cite{IklassovMR023} & \href{../works/IsikYA23.pdf}{IsikYA23}~\cite{IsikYA23}\\ 
\href{../}{Jaffar1998}~\cite{Jaffar1998} & \href{../works/JainG01.pdf}{JainG01}~\cite{JainG01} & \href{../works/JainM99.pdf}{JainM99}~\cite{JainM99} & \href{../works/Janosikova2013.pdf}{Janosikova2013}~\cite{Janosikova2013} & \href{../works/Jans09.pdf}{Jans09}~\cite{Jans09} & \href{../works/JelinekB16.pdf}{JelinekB16}~\cite{JelinekB16}\\ 
\href{../works/JoLLH99.pdf}{JoLLH99}~\cite{JoLLH99} & \href{../works/Johnston05.pdf}{Johnston05}~\cite{Johnston05} & \href{../}{JourdanFRD94}~\cite{JourdanFRD94} & \href{../}{Juan2014}~\cite{Juan2014} & \href{../works/JungblutK22.pdf}{JungblutK22}~\cite{JungblutK22} & \href{../works/Junker00.pdf}{Junker00}~\cite{Junker00}\\ 
\href{../works/Junker2012.pdf}{Junker2012}~\cite{Junker2012} & \href{../works/JussienL02.pdf}{JussienL02}~\cite{JussienL02} & \href{../works/JuvinHHL23.pdf}{JuvinHHL23}~\cite{JuvinHHL23} & \href{../works/JuvinHL22.pdf}{JuvinHL22}~\cite{JuvinHL22} & \href{../works/JuvinHL23.pdf}{JuvinHL23}~\cite{JuvinHL23} & \href{../works/JuvinHL23a.pdf}{JuvinHL23a}~\cite{JuvinHL23a}\\ 
\href{../works/KamarainenS02.pdf}{KamarainenS02}~\cite{KamarainenS02} & \href{../works/Kambhampati2000.pdf}{Kambhampati2000}~\cite{Kambhampati2000} & \href{../works/Kameugne14.pdf}{Kameugne14}~\cite{Kameugne14} & \href{../works/Kameugne15.pdf}{Kameugne15}~\cite{Kameugne15} & \href{../works/KameugneF13.pdf}{KameugneF13}~\cite{KameugneF13} & \href{../works/KameugneFGOQ18.pdf}{KameugneFGOQ18}~\cite{KameugneFGOQ18}\\ 
\href{../works/KameugneFND23.pdf}{KameugneFND23}~\cite{KameugneFND23} & \href{../works/KameugneFSN11.pdf}{KameugneFSN11}~\cite{KameugneFSN11} & \href{../works/KameugneFSN14.pdf}{KameugneFSN14}~\cite{KameugneFSN14} & \href{../works/KanetAG04.pdf}{KanetAG04}~\cite{KanetAG04} & \href{../}{Kasapidis2021}~\cite{Kasapidis2021} & \href{../works/Kasapidis2023.pdf}{Kasapidis2023}~\cite{Kasapidis2023}\\ 
\href{../works/Kelareva2012.pdf}{Kelareva2012}~\cite{Kelareva2012} & \href{../}{Kelareva2014}~\cite{Kelareva2014} & \href{../works/KelarevaTK13.pdf}{KelarevaTK13}~\cite{KelarevaTK13} & \href{../works/KelbelH11.pdf}{KelbelH11}~\cite{KelbelH11} & \href{../works/KendallKRU10.pdf}{KendallKRU10}~\cite{KendallKRU10} & \href{../works/KengY89.pdf}{KengY89}~\cite{KengY89}\\ 
\href{../works/KeriK07.pdf}{KeriK07}~\cite{KeriK07} & \href{../works/KhayatLR06.pdf}{KhayatLR06}~\cite{KhayatLR06} & \href{../works/KhemmoudjPB06.pdf}{KhemmoudjPB06}~\cite{KhemmoudjPB06} & \href{../works/Kim2004.pdf}{Kim2004}~\cite{Kim2004} & \href{../works/KimCMLLP23.pdf}{KimCMLLP23}~\cite{KimCMLLP23} & \href{../works/KinsellaS0OS16.pdf}{KinsellaS0OS16}~\cite{KinsellaS0OS16}\\ 
\href{../works/Kizilay2019.pdf}{Kizilay2019}~\cite{Kizilay2019} & \href{../}{KizilayC20}~\cite{KizilayC20} & \href{../works/KlankeBYE21.pdf}{KlankeBYE21}~\cite{KlankeBYE21} & \href{../works/KletzanderM17.pdf}{KletzanderM17}~\cite{KletzanderM17} & \href{../works/KletzanderM20.pdf}{KletzanderM20}~\cite{KletzanderM20} & \href{../works/KletzanderMH21.pdf}{KletzanderMH21}~\cite{KletzanderMH21}\\ 
\href{../works/KoehlerBFFHPSSS21.pdf}{KoehlerBFFHPSSS21}~\cite{KoehlerBFFHPSSS21} & \href{../}{Kong2020}~\cite{Kong2020} & \href{../}{Kong2021}~\cite{Kong2021} & \href{../works/KonowalenkoMM19.pdf}{KonowalenkoMM19}~\cite{KonowalenkoMM19} & \href{../works/KorbaaYG00.pdf}{KorbaaYG00}~\cite{KorbaaYG00} & \href{../works/KorbaaYG99.pdf}{KorbaaYG99}~\cite{KorbaaYG99}\\ 
\href{../works/KoschB14.pdf}{KoschB14}~\cite{KoschB14} & \href{../works/KotaryFH22.pdf}{KotaryFH22}~\cite{KotaryFH22} & \href{../works/KovacsB07.pdf}{KovacsB07}~\cite{KovacsB07} & \href{../works/KovacsB08.pdf}{KovacsB08}~\cite{KovacsB08} & \href{../works/KovacsB11.pdf}{KovacsB11}~\cite{KovacsB11} & \href{../works/KovacsEKV05.pdf}{KovacsEKV05}~\cite{KovacsEKV05}\\ 
\href{../works/KovacsK11.pdf}{KovacsK11}~\cite{KovacsK11} & \href{../works/KovacsTKSG21.pdf}{KovacsTKSG21}~\cite{KovacsTKSG21} & \href{../works/KovacsV04.pdf}{KovacsV04}~\cite{KovacsV04} & \href{../works/KovacsV06.pdf}{KovacsV06}~\cite{KovacsV06} & \href{../works/Kovcs2003.pdf}{Kovcs2003}~\cite{Kovcs2003} & \href{../works/KreterSS15.pdf}{KreterSS15}~\cite{KreterSS15}\\ 
\href{../works/KreterSS17.pdf}{KreterSS17}~\cite{KreterSS17} & \href{../works/KreterSSZ18.pdf}{KreterSSZ18}~\cite{KreterSSZ18} & \href{../works/KrogtLPHJ07.pdf}{KrogtLPHJ07}~\cite{KrogtLPHJ07} & \href{../works/KuB16.pdf}{KuB16}~\cite{KuB16} & \href{../works/Kuchcinski03.pdf}{Kuchcinski03}~\cite{Kuchcinski03} & \href{../works/KuchcinskiW03.pdf}{KuchcinskiW03}~\cite{KuchcinskiW03}\\ 
\href{../works/KucukY19.pdf}{KucukY19}~\cite{KucukY19} & \href{../works/Kumar03.pdf}{Kumar03}~\cite{Kumar03} & \href{../works/Kuramata2022.pdf}{Kuramata2022}~\cite{Kuramata2022} & \href{../works/KusterJF07.pdf}{KusterJF07}~\cite{KusterJF07} & \href{../works/Laborie03.pdf}{Laborie03}~\cite{Laborie03} & \href{../works/Laborie05.pdf}{Laborie05}~\cite{Laborie05}\\ 
\href{../works/Laborie09.pdf}{Laborie09}~\cite{Laborie09} & \href{../works/Laborie18a.pdf}{Laborie18a}~\cite{Laborie18a} & \href{../}{Laborie2011}~\cite{Laborie2011} & \href{../works/Laborie2017.pdf}{Laborie2017}~\cite{Laborie2017} & \href{../works/LaborieR14.pdf}{LaborieR14}~\cite{LaborieR14} & \href{../works/LaborieRSV18.pdf}{LaborieRSV18}~\cite{LaborieRSV18}\\ 
\href{../works/LacknerMMWW21.pdf}{LacknerMMWW21}~\cite{LacknerMMWW21} & \href{../works/LacknerMMWW23.pdf}{LacknerMMWW23}~\cite{LacknerMMWW23} & \href{../works/Lacomme2011.pdf}{Lacomme2011}~\cite{Lacomme2011} & \href{../works/LahimerLH11.pdf}{LahimerLH11}~\cite{LahimerLH11} & \href{../}{Lallouet2007}~\cite{Lallouet2007} & \href{../}{Lambert2014}~\cite{Lambert2014}\\ 
\href{../works/LammaMM97.pdf}{LammaMM97}~\cite{LammaMM97} & \href{../}{Larrosa1998}~\cite{Larrosa1998} & \href{../works/Larrosa2002.pdf}{Larrosa2002}~\cite{Larrosa2002} & \href{../works/LarsonJC14.pdf}{LarsonJC14}~\cite{LarsonJC14} & \href{../works/LauLN08.pdf}{LauLN08}~\cite{LauLN08} & \href{../works/Layfield02.pdf}{Layfield02}~\cite{Layfield02}\\ 
\href{../works/LeeKLKKYHP97.pdf}{LeeKLKKYHP97}~\cite{LeeKLKKYHP97} & \href{../works/Lemos21.pdf}{Lemos21}~\cite{Lemos21} & \href{../works/Letort13.pdf}{Letort13}~\cite{Letort13} & \href{../works/LetortBC12.pdf}{LetortBC12}~\cite{LetortBC12} & \href{../works/LetortCB13.pdf}{LetortCB13}~\cite{LetortCB13} & \href{../works/LetortCB15.pdf}{LetortCB15}~\cite{LetortCB15}\\ 
\href{../works/Levine2014.pdf}{Levine2014}~\cite{Levine2014} & \href{../}{Li2014}~\cite{Li2014} & \href{../works/Li2014a.pdf}{Li2014a}~\cite{Li2014a} & \href{../works/Li2014b.pdf}{Li2014b}~\cite{Li2014b} & \href{../works/Li2015.pdf}{Li2015}~\cite{Li2015} & \href{../works/Li2016.pdf}{Li2016}~\cite{Li2016}\\ 
\href{../works/Li2018.pdf}{Li2018}~\cite{Li2018} & \href{../works/Li2020.pdf}{Li2020}~\cite{Li2020} & \href{../works/LiFJZLL22.pdf}{LiFJZLL22}~\cite{LiFJZLL22} & \href{../works/LiLZDZW24.pdf}{LiLZDZW24}~\cite{LiLZDZW24} & \href{../works/LiW08.pdf}{LiW08}~\cite{LiW08} & \href{../works/LiessM08.pdf}{LiessM08}~\cite{LiessM08}\\ 
\href{../works/Lim2004.pdf}{Lim2004}~\cite{Lim2004} & \href{../works/Lim2014.pdf}{Lim2014}~\cite{Lim2014} & \href{../works/Lim2015.pdf}{Lim2015}~\cite{Lim2015} & \href{../works/LimAHO02a.pdf}{LimAHO02a}~\cite{LimAHO02a} & \href{../works/LimBTBB15.pdf}{LimBTBB15}~\cite{LimBTBB15} & \href{../works/LimBTBB15a.pdf}{LimBTBB15a}~\cite{LimBTBB15a}\\ 
\href{../works/LimHTB16.pdf}{LimHTB16}~\cite{LimHTB16} & \href{../works/LimRX04.pdf}{LimRX04}~\cite{LimRX04} & \href{../works/Limtanyakul07.pdf}{Limtanyakul07}~\cite{Limtanyakul07} & \href{../works/LimtanyakulS12.pdf}{LimtanyakulS12}~\cite{LimtanyakulS12} & \href{../works/Lindauer2015.pdf}{Lindauer2015}~\cite{Lindauer2015} & \href{../works/LipovetzkyBPS14.pdf}{LipovetzkyBPS14}~\cite{LipovetzkyBPS14}\\ 
\href{../works/Liu2020.pdf}{Liu2020}~\cite{Liu2020} & \href{../works/Liu2021.pdf}{Liu2021}~\cite{Liu2021} & \href{../works/Liu2021a.pdf}{Liu2021a}~\cite{Liu2021a} & \href{../works/Liu2021b.pdf}{Liu2021b}~\cite{Liu2021b} & \href{../works/Liu2023.pdf}{Liu2023}~\cite{Liu2023} & \href{../works/LiuCGM17.pdf}{LiuCGM17}~\cite{LiuCGM17}\\ 
\href{../works/LiuGT10.pdf}{LiuGT10}~\cite{LiuGT10} & \href{../works/LiuJ06.pdf}{LiuJ06}~\cite{LiuJ06} & \href{../works/LiuLH18.pdf}{LiuLH18}~\cite{LiuLH18} & \href{../works/LiuLH19.pdf}{LiuLH19}~\cite{LiuLH19} & \href{../works/LiuLH19a.pdf}{LiuLH19a}~\cite{LiuLH19a} & \href{../works/LiuW11.pdf}{LiuW11}~\cite{LiuW11}\\ 
\href{../works/Lizarralde2011.pdf}{Lizarralde2011}~\cite{Lizarralde2011} & \href{../works/Lombardi10.pdf}{Lombardi10}~\cite{Lombardi10} & \href{../works/LombardiBM15.pdf}{LombardiBM15}~\cite{LombardiBM15} & \href{../works/LombardiBMB11.pdf}{LombardiBMB11}~\cite{LombardiBMB11} & \href{../works/LombardiM09.pdf}{LombardiM09}~\cite{LombardiM09} & \href{../works/LombardiM10.pdf}{LombardiM10}~\cite{LombardiM10}\\ 
\href{../works/LombardiM10a.pdf}{LombardiM10a}~\cite{LombardiM10a} & \href{../works/LombardiM12.pdf}{LombardiM12}~\cite{LombardiM12} & \href{../works/LombardiM12a.pdf}{LombardiM12a}~\cite{LombardiM12a} & \href{../works/LombardiM13.pdf}{LombardiM13}~\cite{LombardiM13} & \href{../works/LombardiMB13.pdf}{LombardiMB13}~\cite{LombardiMB13} & \href{../works/LombardiMRB10.pdf}{LombardiMRB10}~\cite{LombardiMRB10}\\ 
\href{../works/LopesCSM10.pdf}{LopesCSM10}~\cite{LopesCSM10} & \href{../works/LopezAKYG00.pdf}{LopezAKYG00}~\cite{LopezAKYG00} & \href{../works/Lorca2016.pdf}{Lorca2016}~\cite{Lorca2016} & \href{../works/LorigeonBB02.pdf}{LorigeonBB02}~\cite{LorigeonBB02} & \href{../}{Lorterapong2009}~\cite{Lorterapong2009} & \href{../}{Lorterapong2013}~\cite{Lorterapong2013}\\ 
\href{../works/Loth2013.pdf}{Loth2013}~\cite{Loth2013} & \href{../works/LouieVNB14.pdf}{LouieVNB14}~\cite{LouieVNB14} & \href{../works/Lozano2014.pdf}{Lozano2014}~\cite{Lozano2014} & \href{../works/Lozano2019.pdf}{Lozano2019}~\cite{Lozano2019} & \href{../works/Lozano2019a.pdf}{Lozano2019a}~\cite{Lozano2019a} & \href{../works/LozanoCDS12.pdf}{LozanoCDS12}~\cite{LozanoCDS12}\\ 
\href{../works/Lu2021.pdf}{Lu2021}~\cite{Lu2021} & \href{../works/LuZZYW24.pdf}{LuZZYW24}~\cite{LuZZYW24} & \href{../works/LudwigKRBMS14.pdf}{LudwigKRBMS14}~\cite{LudwigKRBMS14} & \href{../works/Lunardi20.pdf}{Lunardi20}~\cite{Lunardi20} & \href{../works/LunardiBLRV20.pdf}{LunardiBLRV20}~\cite{LunardiBLRV20} & \href{../works/LuoB22.pdf}{LuoB22}~\cite{LuoB22}\\ 
\href{../works/LuoVLBM16.pdf}{LuoVLBM16}~\cite{LuoVLBM16} & \href{../works/Lyons2023.pdf}{Lyons2023}~\cite{Lyons2023} & \href{../works/Madi-WambaB16.pdf}{Madi-WambaB16}~\cite{Madi-WambaB16} & \href{../works/Madi-WambaLOBM17.pdf}{Madi-WambaLOBM17}~\cite{Madi-WambaLOBM17} & \href{../}{MagataoAN05}~\cite{MagataoAN05} & \href{../works/Magato2008.pdf}{Magato2008}~\cite{Magato2008}\\ 
\href{../works/Magato2010.pdf}{Magato2010}~\cite{Magato2010} & \href{../works/Maillard15.pdf}{Maillard15}~\cite{Maillard15} & \href{../works/MakMS10.pdf}{MakMS10}~\cite{MakMS10} & \href{../works/Malapert11.pdf}{Malapert11}~\cite{Malapert11} & \href{../works/MalapertCGJLR12.pdf}{MalapertCGJLR12}~\cite{MalapertCGJLR12} & \href{../works/MalapertCGJLR13.pdf}{MalapertCGJLR13}~\cite{MalapertCGJLR13}\\ 
\href{../works/MalapertGR12.pdf}{MalapertGR12}~\cite{MalapertGR12} & \href{../works/MalapertN19.pdf}{MalapertN19}~\cite{MalapertN19} & \href{../works/Malik08.pdf}{Malik08}~\cite{Malik08} & \href{../works/Malik2008.pdf}{Malik2008}~\cite{Malik2008} & \href{../works/MalikMB08.pdf}{MalikMB08}~\cite{MalikMB08} & \href{../works/MaraveliasCG04.pdf}{MaraveliasCG04}~\cite{MaraveliasCG04}\\ 
\href{../works/MaraveliasG04.pdf}{MaraveliasG04}~\cite{MaraveliasG04} & \href{../}{Marcolini2022}~\cite{Marcolini2022} & \href{../works/MarliereSPR23.pdf}{MarliereSPR23}~\cite{MarliereSPR23} & \href{../works/Martin2012.pdf}{Martin2012}~\cite{Martin2012} & \href{../works/MartinPY01.pdf}{MartinPY01}~\cite{MartinPY01} & \href{../}{MartnezAJ22}~\cite{MartnezAJ22}\\ 
\href{../works/Mason01.pdf}{Mason01}~\cite{Mason01} & \href{../works/Mehdizadeh-Somarin23.pdf}{Mehdizadeh-Somarin23}~\cite{Mehdizadeh-Somarin23} & \href{../works/MejiaY20.pdf}{MejiaY20}~\cite{MejiaY20} & \href{../works/MelgarejoLS15.pdf}{MelgarejoLS15}~\cite{MelgarejoLS15} & \href{../works/Menana11.pdf}{Menana11}~\cite{Menana11} & \href{../works/MenciaSV12.pdf}{MenciaSV12}~\cite{MenciaSV12}\\ 
\href{../works/MenciaSV13.pdf}{MenciaSV13}~\cite{MenciaSV13} & \href{../works/MengGRZSC22.pdf}{MengGRZSC22}~\cite{MengGRZSC22} & \href{../works/MengLZB21.pdf}{MengLZB21}~\cite{MengLZB21} & \href{../works/MengZRZL20.pdf}{MengZRZL20}~\cite{MengZRZL20} & \href{../works/Menouer2016.pdf}{Menouer2016}~\cite{Menouer2016} & \href{../works/Mercier-AubinGQ20.pdf}{Mercier-AubinGQ20}~\cite{Mercier-AubinGQ20}\\ 
\href{../works/MercierH07.pdf}{MercierH07}~\cite{MercierH07} & \href{../works/MercierH08.pdf}{MercierH08}~\cite{MercierH08} & \href{../}{Mesghouni1997}~\cite{Mesghouni1997} & \href{../works/MeskensDHG11.pdf}{MeskensDHG11}~\cite{MeskensDHG11} & \href{../works/MeskensDL13.pdf}{MeskensDL13}~\cite{MeskensDL13} & \href{../works/MeyerE04.pdf}{MeyerE04}~\cite{MeyerE04}\\ 
\href{../works/Michel2004.pdf}{Michel2004}~\cite{Michel2004} & \href{../}{Michel2009}~\cite{Michel2009} & \href{../works/Michel2012.pdf}{Michel2012}~\cite{Michel2012} & \href{../works/Michels2022.pdf}{Michels2022}~\cite{Michels2022} & \href{../}{Milano11}~\cite{Milano11} & \href{../}{MilanoORT02}~\cite{MilanoORT02}\\ 
\href{../works/MilanoW06.pdf}{MilanoW06}~\cite{MilanoW06} & \href{../works/MilanoW09.pdf}{MilanoW09}~\cite{MilanoW09} & \href{../works/MintonJPL90.pdf}{MintonJPL90}~\cite{MintonJPL90} & \href{../works/MintonJPL92.pdf}{MintonJPL92}~\cite{MintonJPL92} & \href{../works/Mischek2021.pdf}{Mischek2021}~\cite{Mischek2021} & \href{../works/Mischek2021a.pdf}{Mischek2021a}~\cite{Mischek2021a}\\ 
\href{../works/Misra2022.pdf}{Misra2022}~\cite{Misra2022} & \href{../works/Mladenovic2007.pdf}{Mladenovic2007}~\cite{Mladenovic2007} & \href{../works/Mladenovic2015.pdf}{Mladenovic2015}~\cite{Mladenovic2015} & \href{../works/Mller2003.pdf}{Mller2003}~\cite{Mller2003} & \href{../}{Mnif2020}~\cite{Mnif2020} & \href{../works/Moccia2005.pdf}{Moccia2005}~\cite{Moccia2005}\\ 
\href{../works/MoffittPP05.pdf}{MoffittPP05}~\cite{MoffittPP05} & \href{../works/MokhtarzadehTNF20.pdf}{MokhtarzadehTNF20}~\cite{MokhtarzadehTNF20} & \href{../works/MonetteDD07.pdf}{MonetteDD07}~\cite{MonetteDD07} & \href{../works/MonetteDH09.pdf}{MonetteDH09}~\cite{MonetteDH09} & \href{../works/MontemanniD23.pdf}{MontemanniD23}~\cite{MontemanniD23} & \href{../works/MontemanniD23a.pdf}{MontemanniD23a}~\cite{MontemanniD23a}\\ 
\href{../works/Moreno-Scott2016.pdf}{Moreno-Scott2016}~\cite{Moreno-Scott2016} & \href{../works/MorgadoM97.pdf}{MorgadoM97}~\cite{MorgadoM97} & \href{../works/Morillo2017.pdf}{Morillo2017}~\cite{Morillo2017} & \href{../works/MossigeGSMC17.pdf}{MossigeGSMC17}~\cite{MossigeGSMC17} & \href{../}{Moukrim2014}~\cite{Moukrim2014} & \href{../works/MouraSCL08.pdf}{MouraSCL08}~\cite{MouraSCL08}\\ 
\href{../works/MouraSCL08a.pdf}{MouraSCL08a}~\cite{MouraSCL08a} & \href{../works/MullerMKP22.pdf}{MullerMKP22}~\cite{MullerMKP22} & \href{../works/MurinR19.pdf}{MurinR19}~\cite{MurinR19} & \href{../works/MurphyMB15.pdf}{MurphyMB15}~\cite{MurphyMB15} & \href{../works/MurphyRFSS97.pdf}{MurphyRFSS97}~\cite{MurphyRFSS97} & \href{../}{MurthyRAW97}~\cite{MurthyRAW97}\\ 
\href{../works/Muscettola02.pdf}{Muscettola02}~\cite{Muscettola02} & \href{../works/Muscettola94.pdf}{Muscettola94}~\cite{Muscettola94} & \href{../works/Musliu05.pdf}{Musliu05}~\cite{Musliu05} & \href{../works/MusliuSS18.pdf}{MusliuSS18}~\cite{MusliuSS18} & \href{../}{Mutha2017}~\cite{Mutha2017} & \href{../works/NaderiBZ22.pdf}{NaderiBZ22}~\cite{NaderiBZ22}\\ 
\href{../works/NaderiBZ22a.pdf}{NaderiBZ22a}~\cite{NaderiBZ22a} & \href{../works/NaderiBZ23.pdf}{NaderiBZ23}~\cite{NaderiBZ23} & \href{../works/NaderiBZR23.pdf}{NaderiBZR23}~\cite{NaderiBZR23} & \href{../}{NaderiR22}~\cite{NaderiR22} & \href{../}{NaderiRBAU21}~\cite{NaderiRBAU21} & \href{../works/NaderiRR23.pdf}{NaderiRR23}~\cite{NaderiRR23}\\ 
\href{../works/NaqviAIAAA22.pdf}{NaqviAIAAA22}~\cite{NaqviAIAAA22} & \href{../works/Nattaf16.pdf}{Nattaf16}~\cite{Nattaf16} & \href{../works/NattafAL15.pdf}{NattafAL15}~\cite{NattafAL15} & \href{../works/NattafAL17.pdf}{NattafAL17}~\cite{NattafAL17} & \href{../works/NattafALR16.pdf}{NattafALR16}~\cite{NattafALR16} & \href{../works/NattafDYW19.pdf}{NattafDYW19}~\cite{NattafDYW19}\\ 
\href{../works/NattafHKAL19.pdf}{NattafHKAL19}~\cite{NattafHKAL19} & \href{../works/NattafM20.pdf}{NattafM20}~\cite{NattafM20} & \href{../}{NeronABCDD06}~\cite{NeronABCDD06} & \href{../works/NishikawaSTT18.pdf}{NishikawaSTT18}~\cite{NishikawaSTT18} & \href{../works/NishikawaSTT18a.pdf}{NishikawaSTT18a}~\cite{NishikawaSTT18a} & \href{../works/NishikawaSTT19.pdf}{NishikawaSTT19}~\cite{NishikawaSTT19}\\ 
\href{../}{NouriMHD23}~\cite{NouriMHD23} & \href{../}{Novara2013}~\cite{Novara2013} & \href{../}{Novara2015}~\cite{Novara2015} & \href{../works/NovaraNH16.pdf}{NovaraNH16}~\cite{NovaraNH16} & \href{../works/Novas19.pdf}{Novas19}~\cite{Novas19} & \href{../works/NovasH10.pdf}{NovasH10}~\cite{NovasH10}\\ 
\href{../works/NovasH12.pdf}{NovasH12}~\cite{NovasH12} & \href{../works/NovasH14.pdf}{NovasH14}~\cite{NovasH14} & \href{../works/Nowatzki2013.pdf}{Nowatzki2013}~\cite{Nowatzki2013} & \href{../works/Nuijten94.pdf}{Nuijten94}~\cite{Nuijten94} & \href{../works/NuijtenA94.pdf}{NuijtenA94}~\cite{NuijtenA94} & \href{../}{NuijtenA94a}~\cite{NuijtenA94a}\\ 
\href{../works/NuijtenA96.pdf}{NuijtenA96}~\cite{NuijtenA96} & \href{../works/NuijtenP98.pdf}{NuijtenP98}~\cite{NuijtenP98} & \href{../works/OddiPCC03.pdf}{OddiPCC03}~\cite{OddiPCC03} & \href{../}{OddiPCC05}~\cite{OddiPCC05} & \href{../works/OddiRC10.pdf}{OddiRC10}~\cite{OddiRC10} & \href{../works/OddiRCS11.pdf}{OddiRCS11}~\cite{OddiRCS11}\\ 
\href{../works/OddiS97.pdf}{OddiS97}~\cite{OddiS97} & \href{../works/OhrimenkoSC09.pdf}{OhrimenkoSC09}~\cite{OhrimenkoSC09} & \href{../}{OkanoDTRYA04}~\cite{OkanoDTRYA04} & \href{../}{Oliveira2015}~\cite{Oliveira2015} & \href{../works/OrnekO16.pdf}{OrnekO16}~\cite{OrnekO16} & \href{../works/OrnekOS20.pdf}{OrnekOS20}~\cite{OrnekOS20}\\ 
\href{../works/Ortiz-Bayliss2013.pdf}{Ortiz-Bayliss2013}~\cite{Ortiz-Bayliss2013} & \href{../}{Ortiz-Bayliss2014}~\cite{Ortiz-Bayliss2014} & \href{../works/Ortiz-Bayliss2018.pdf}{Ortiz-Bayliss2018}~\cite{Ortiz-Bayliss2018} & \href{../works/Ortiz-Bayliss2021.pdf}{Ortiz-Bayliss2021}~\cite{Ortiz-Bayliss2021} & \href{../works/Ouaja2004.pdf}{Ouaja2004}~\cite{Ouaja2004} & \href{../works/Ouellet2022.pdf}{Ouellet2022}~\cite{Ouellet2022}\\ 
\href{../works/OuelletQ13.pdf}{OuelletQ13}~\cite{OuelletQ13} & \href{../works/OuelletQ18.pdf}{OuelletQ18}~\cite{OuelletQ18} & \href{../works/OuelletQ22.pdf}{OuelletQ22}~\cite{OuelletQ22} & \href{../works/Oujana2023.pdf}{Oujana2023}~\cite{Oujana2023} & \href{../works/OujanaAYB22.pdf}{OujanaAYB22}~\cite{OujanaAYB22} & \href{../works/Ozder2019.pdf}{Ozder2019}~\cite{Ozder2019}\\ 
\href{../works/OzturkTHO10.pdf}{OzturkTHO10}~\cite{OzturkTHO10} & \href{../works/OzturkTHO12.pdf}{OzturkTHO12}~\cite{OzturkTHO12} & \href{../works/OzturkTHO13.pdf}{OzturkTHO13}~\cite{OzturkTHO13} & \href{../works/OzturkTHO15.pdf}{OzturkTHO15}~\cite{OzturkTHO15} & \href{../works/PachecoPR19.pdf}{PachecoPR19}~\cite{PachecoPR19} & \href{../works/PacinoH11.pdf}{PacinoH11}~\cite{PacinoH11}\\ 
\href{../works/PandeyS21a.pdf}{PandeyS21a}~\cite{PandeyS21a} & \href{../works/PapaB98.pdf}{PapaB98}~\cite{PapaB98} & \href{../works/Pape94.pdf}{Pape94}~\cite{Pape94} & \href{../}{PapeB96}~\cite{PapeB96} & \href{../works/PapeB97.pdf}{PapeB97}~\cite{PapeB97} & \href{../}{Paredis1992}~\cite{Paredis1992}\\ 
\href{../}{Park2016}~\cite{Park2016} & \href{../works/ParkUJR19.pdf}{ParkUJR19}~\cite{ParkUJR19} & \href{../works/PembertonG98.pdf}{PembertonG98}~\cite{PembertonG98} & \href{../}{Peng2012}~\cite{Peng2012} & \href{../works/PengLC14.pdf}{PengLC14}~\cite{PengLC14} & \href{../works/PenzDN23.pdf}{PenzDN23}~\cite{PenzDN23}\\ 
\href{../works/PerezGSL23.pdf}{PerezGSL23}~\cite{PerezGSL23} & \href{../works/Perron05.pdf}{Perron05}~\cite{Perron05} & \href{../works/PerronSF04.pdf}{PerronSF04}~\cite{PerronSF04} & \href{../works/Pesant2012.pdf}{Pesant2012}~\cite{Pesant2012} & \href{../works/PesantGPR99.pdf}{PesantGPR99}~\cite{PesantGPR99} & \href{../works/PesantRR15.pdf}{PesantRR15}~\cite{PesantRR15}\\ 
\href{../}{PeschT96}~\cite{PeschT96} & \href{../works/Pessoa2013.pdf}{Pessoa2013}~\cite{Pessoa2013} & \href{../works/Petith2002.pdf}{Petith2002}~\cite{Petith2002} & \href{../}{Petrovic2008}~\cite{Petrovic2008} & \href{../}{Pinarbasi21}~\cite{Pinarbasi21} & \href{../}{PinarbasiA20}~\cite{PinarbasiA20}\\ 
\href{../works/PinarbasiAY19.pdf}{PinarbasiAY19}~\cite{PinarbasiAY19} & \href{../works/Pinto2012.pdf}{Pinto2012}~\cite{Pinto2012} & \href{../works/PintoG97.pdf}{PintoG97}~\cite{PintoG97} & \href{../works/PoderB08.pdf}{PoderB08}~\cite{PoderB08} & \href{../works/PoderBS04.pdf}{PoderBS04}~\cite{PoderBS04} & \href{../works/PohlAK22.pdf}{PohlAK22}~\cite{PohlAK22}\\ 
\href{../works/PolicellaWSO05.pdf}{PolicellaWSO05}~\cite{PolicellaWSO05} & \href{../works/Polo-MejiaALB20.pdf}{Polo-MejiaALB20}~\cite{Polo-MejiaALB20} & \href{../works/PopovicCGNC22.pdf}{PopovicCGNC22}~\cite{PopovicCGNC22} & \href{../works/PourDERB18.pdf}{PourDERB18}~\cite{PourDERB18} & \href{../works/PovedaAA23.pdf}{PovedaAA23}~\cite{PovedaAA23} & \href{../works/Pralet17.pdf}{Pralet17}~\cite{Pralet17}\\ 
\href{../works/PraletLJ15.pdf}{PraletLJ15}~\cite{PraletLJ15} & \href{../works/PrataAN23.pdf}{PrataAN23}~\cite{PrataAN23} & \href{../works/Priore2003.pdf}{Priore2003}~\cite{Priore2003} & \href{../works/Prosser89.pdf}{Prosser89}~\cite{Prosser89} & \href{../works/Psarras1997.pdf}{Psarras1997}~\cite{Psarras1997} & \href{../works/Puget95.pdf}{Puget95}~\cite{Puget95}\\ 
\href{../works/QinDCS20.pdf}{QinDCS20}~\cite{QinDCS20} & \href{../works/QinDS16.pdf}{QinDS16}~\cite{QinDS16} & \href{../works/QinWSLS21.pdf}{QinWSLS21}~\cite{QinWSLS21} & \href{../works/QuSN06.pdf}{QuSN06}~\cite{QuSN06} & \href{../works/QuirogaZH05.pdf}{QuirogaZH05}~\cite{QuirogaZH05} & \href{../}{RabbaniMM21}~\cite{RabbaniMM21}\\ 
\href{../works/Radzki2021.pdf}{Radzki2021}~\cite{Radzki2021} & \href{../}{Raffin2012}~\cite{Raffin2012} & \href{../works/Ramos2021.pdf}{Ramos2021}~\cite{Ramos2021} & \href{../works/Ramos2023.pdf}{Ramos2023}~\cite{Ramos2023} & \href{../works/RasmussenT06.pdf}{RasmussenT06}~\cite{RasmussenT06} & \href{../works/RasmussenT07.pdf}{RasmussenT07}~\cite{RasmussenT07}\\ 
\href{../works/RasmussenT09.pdf}{RasmussenT09}~\cite{RasmussenT09} & \href{../works/Reale2014.pdf}{Reale2014}~\cite{Reale2014} & \href{../works/ReddyFIBKAJ11.pdf}{ReddyFIBKAJ11}~\cite{ReddyFIBKAJ11} & \href{../works/Refalo00.pdf}{Refalo00}~\cite{Refalo00} & \href{../works/Refanidis2010.pdf}{Refanidis2010}~\cite{Refanidis2010} & \href{../works/Relich2020.pdf}{Relich2020}~\cite{Relich2020}\\ 
\href{../works/Relich2022.pdf}{Relich2022}~\cite{Relich2022} & \href{../works/Relich2023.pdf}{Relich2023}~\cite{Relich2023} & \href{../works/Ren2016.pdf}{Ren2016}~\cite{Ren2016} & \href{../works/RenT09.pdf}{RenT09}~\cite{RenT09} & \href{../works/RendlPHPR12.pdf}{RendlPHPR12}~\cite{RendlPHPR12} & \href{../}{Rgin2001}~\cite{Rgin2001}\\ 
\href{../works/RiahiNS018.pdf}{RiahiNS018}~\cite{RiahiNS018} & \href{../works/Ribeiro12.pdf}{Ribeiro12}~\cite{Ribeiro12} & \href{../works/Richard1998.pdf}{Richard1998}~\cite{Richard1998} & \href{../works/Richard2002.pdf}{Richard2002}~\cite{Richard2002} & \href{../works/Rieber2021.pdf}{Rieber2021}~\cite{Rieber2021} & \href{../works/RiiseML16.pdf}{RiiseML16}~\cite{RiiseML16}\\ 
\href{../works/Rit86.pdf}{Rit86}~\cite{Rit86} & \href{../}{Rodosek94}~\cite{Rodosek94} & \href{../works/RodosekW98.pdf}{RodosekW98}~\cite{RodosekW98} & \href{../works/RodosekWH99.pdf}{RodosekWH99}~\cite{RodosekWH99} & \href{../works/Rodriguez07.pdf}{Rodriguez07}~\cite{Rodriguez07} & \href{../works/Rodriguez07b.pdf}{Rodriguez07b}~\cite{Rodriguez07b}\\ 
\href{../works/RodriguezDG02.pdf}{RodriguezDG02}~\cite{RodriguezDG02} & \href{../works/RodriguezS09.pdf}{RodriguezS09}~\cite{RodriguezS09} & \href{../}{Roe2003}~\cite{Roe2003} & \href{../works/RoePS05.pdf}{RoePS05}~\cite{RoePS05} & \href{../works/RoshanaeiBAUB20.pdf}{RoshanaeiBAUB20}~\cite{RoshanaeiBAUB20} & \href{../works/RoshanaeiLAU17.pdf}{RoshanaeiLAU17}~\cite{RoshanaeiLAU17}\\ 
\href{../}{RoshanaeiLAU17a}~\cite{RoshanaeiLAU17a} & \href{../works/RoshanaeiN21.pdf}{RoshanaeiN21}~\cite{RoshanaeiN21} & \href{../works/RossiTHP07.pdf}{RossiTHP07}~\cite{RossiTHP07} & \href{../works/RoweJCA96.pdf}{RoweJCA96}~\cite{RoweJCA96} & \href{../works/RuggieroBBMA09.pdf}{RuggieroBBMA09}~\cite{RuggieroBBMA09} & \href{../works/Ruixin2018.pdf}{Ruixin2018}~\cite{Ruixin2018}\\ 
\href{../works/RussellU06.pdf}{RussellU06}~\cite{RussellU06} & \href{../works/SacramentoSP20.pdf}{SacramentoSP20}~\cite{SacramentoSP20} & \href{../works/Sadeh1995.pdf}{Sadeh1995}~\cite{Sadeh1995} & \href{../works/SadehF96.pdf}{SadehF96}~\cite{SadehF96} & \href{../works/Sadykov04.pdf}{Sadykov04}~\cite{Sadykov04} & \href{../works/Sadykov2003.pdf}{Sadykov2003}~\cite{Sadykov2003}\\ 
\href{../works/SadykovW06.pdf}{SadykovW06}~\cite{SadykovW06} & \href{../works/Sahli2021.pdf}{Sahli2021}~\cite{Sahli2021} & \href{../works/Sahraeian2015.pdf}{Sahraeian2015}~\cite{Sahraeian2015} & \href{../}{SakkoutRW98}~\cite{SakkoutRW98} & \href{../works/SakkoutW00.pdf}{SakkoutW00}~\cite{SakkoutW00} & \href{../works/Salido10.pdf}{Salido10}~\cite{Salido10}\\ 
\href{../works/Salido2008.pdf}{Salido2008}~\cite{Salido2008} & \href{../works/Salido2008a.pdf}{Salido2008a}~\cite{Salido2008a} & \href{../works/Salvagnin2012.pdf}{Salvagnin2012}~\cite{Salvagnin2012} & \href{../works/Satish2007.pdf}{Satish2007}~\cite{Satish2007} & \href{../}{Schaerf96}~\cite{Schaerf96} & \href{../works/Schaerf97.pdf}{Schaerf97}~\cite{Schaerf97}\\ 
\href{../works/SchausD08.pdf}{SchausD08}~\cite{SchausD08} & \href{../works/SchausHMCMD11.pdf}{SchausHMCMD11}~\cite{SchausHMCMD11} & \href{../}{Schiex1994}~\cite{Schiex1994} & \href{../works/SchildW00.pdf}{SchildW00}~\cite{SchildW00} & \href{../works/SchnellH15.pdf}{SchnellH15}~\cite{SchnellH15} & \href{../works/SchnellH17.pdf}{SchnellH17}~\cite{SchnellH17}\\ 
\href{../works/Schutt11.pdf}{Schutt11}~\cite{Schutt11} & \href{../works/SchuttCSW12.pdf}{SchuttCSW12}~\cite{SchuttCSW12} & \href{../works/SchuttFS13.pdf}{SchuttFS13}~\cite{SchuttFS13} & \href{../works/SchuttFS13a.pdf}{SchuttFS13a}~\cite{SchuttFS13a} & \href{../works/SchuttFSW09.pdf}{SchuttFSW09}~\cite{SchuttFSW09} & \href{../works/SchuttFSW11.pdf}{SchuttFSW11}~\cite{SchuttFSW11}\\ 
\href{../works/SchuttFSW13.pdf}{SchuttFSW13}~\cite{SchuttFSW13} & \href{../}{SchuttFSW15}~\cite{SchuttFSW15} & \href{../works/SchuttS16.pdf}{SchuttS16}~\cite{SchuttS16} & \href{../works/SchuttW10.pdf}{SchuttW10}~\cite{SchuttW10} & \href{../works/SchuttWS05.pdf}{SchuttWS05}~\cite{SchuttWS05} & \href{../works/Schwarz2019.pdf}{Schwarz2019}~\cite{Schwarz2019}\\ 
\href{../works/Schweitzer2023.pdf}{Schweitzer2023}~\cite{Schweitzer2023} & \href{../works/SenderovichBB19.pdf}{SenderovichBB19}~\cite{SenderovichBB19} & \href{../works/SerraNM12.pdf}{SerraNM12}~\cite{SerraNM12} & \href{../works/ShaikhK23.pdf}{ShaikhK23}~\cite{ShaikhK23} & \href{../}{ShangGuan2012}~\cite{ShangGuan2012} & \href{../}{ShiYXQ22}~\cite{ShiYXQ22}\\ 
\href{../works/ShinBBHO18.pdf}{ShinBBHO18}~\cite{ShinBBHO18} & \href{../works/Shobaki2013.pdf}{Shobaki2013}~\cite{Shobaki2013} & \href{../works/Siala15.pdf}{Siala15}~\cite{Siala15} & \href{../works/Siala15a.pdf}{Siala15a}~\cite{Siala15a} & \href{../works/SialaAH15.pdf}{SialaAH15}~\cite{SialaAH15} & \href{../works/Silva2014.pdf}{Silva2014}~\cite{Silva2014}\\ 
\href{../works/SimoninAHL12.pdf}{SimoninAHL12}~\cite{SimoninAHL12} & \href{../works/SimoninAHL15.pdf}{SimoninAHL15}~\cite{SimoninAHL15} & \href{../works/Simonis07.pdf}{Simonis07}~\cite{Simonis07} & \href{../works/Simonis95.pdf}{Simonis95}~\cite{Simonis95} & \href{../works/Simonis95a.pdf}{Simonis95a}~\cite{Simonis95a} & \href{../works/Simonis99.pdf}{Simonis99}~\cite{Simonis99}\\ 
\href{../works/SimonisC95.pdf}{SimonisC95}~\cite{SimonisC95} & \href{../works/SimonisCK00.pdf}{SimonisCK00}~\cite{SimonisCK00} & \href{../works/SimonisH11.pdf}{SimonisH11}~\cite{SimonisH11} & \href{../works/Sitek2016.pdf}{Sitek2016}~\cite{Sitek2016} & \href{../works/Sitek2017.pdf}{Sitek2017}~\cite{Sitek2017} & \href{../}{Sitek2017a}~\cite{Sitek2017a}\\ 
\href{../works/Sitek2018.pdf}{Sitek2018}~\cite{Sitek2018} & \href{../works/Smith-Miles2009.pdf}{Smith-Miles2009}~\cite{Smith-Miles2009} & \href{../works/SmithBHW96.pdf}{SmithBHW96}~\cite{SmithBHW96} & \href{../works/SmithC93.pdf}{SmithC93}~\cite{SmithC93} & \href{../works/Soh2015.pdf}{Soh2015}~\cite{Soh2015} & \href{../works/Song2022.pdf}{Song2022}~\cite{Song2022}\\ 
\href{../works/Soto2015.pdf}{Soto2015}~\cite{Soto2015} & \href{../works/SourdN00.pdf}{SourdN00}~\cite{SourdN00} & \href{../works/Spieker2021.pdf}{Spieker2021}~\cite{Spieker2021} & \href{../works/Squillaci2022.pdf}{Squillaci2022}~\cite{Squillaci2022} & \href{../works/SquillaciPR23.pdf}{SquillaciPR23}~\cite{SquillaciPR23} & \href{../}{Stebel2006}~\cite{Stebel2006}\\ 
\href{../}{StidsenKM96}~\cite{StidsenKM96} & \href{../works/Stobbe1999.pdf}{Stobbe1999}~\cite{Stobbe1999} & \href{../works/Strak2021.pdf}{Strak2021}~\cite{Strak2021} & \href{../works/SuCC13.pdf}{SuCC13}~\cite{SuCC13} & \href{../works/SubulanC22.pdf}{SubulanC22}~\cite{SubulanC22} & \href{../works/SultanikMR07.pdf}{SultanikMR07}~\cite{SultanikMR07}\\ 
\href{../works/Sun2006.pdf}{Sun2006}~\cite{Sun2006} & \href{../works/SunLYL10.pdf}{SunLYL10}~\cite{SunLYL10} & \href{../works/SunTB19.pdf}{SunTB19}~\cite{SunTB19} & \href{../works/SureshMOK06.pdf}{SureshMOK06}~\cite{SureshMOK06} & \href{../works/SvancaraB22.pdf}{SvancaraB22}~\cite{SvancaraB22} & \href{../works/SzerediS16.pdf}{SzerediS16}~\cite{SzerediS16}\\ 
\href{../works/Talbi2013.pdf}{Talbi2013}~\cite{Talbi2013} & \href{../}{Talbi2013a}~\cite{Talbi2013a} & \href{../works/Talbi2015.pdf}{Talbi2015}~\cite{Talbi2015} & \href{../}{Talbot1978}~\cite{Talbot1978} & \href{../works/TanSD10.pdf}{TanSD10}~\cite{TanSD10} & \href{../works/TanT18.pdf}{TanT18}~\cite{TanT18}\\ 
\href{../works/TanZWGQ19.pdf}{TanZWGQ19}~\cite{TanZWGQ19} & \href{../works/Tang2014.pdf}{Tang2014}~\cite{Tang2014} & \href{../works/Tang2018.pdf}{Tang2018}~\cite{Tang2018} & \href{../works/TangB20.pdf}{TangB20}~\cite{TangB20} & \href{../works/TangLWSK18.pdf}{TangLWSK18}~\cite{TangLWSK18} & \href{../}{Tapkan2022}~\cite{Tapkan2022}\\ 
\href{../works/TardivoDFMP23.pdf}{TardivoDFMP23}~\cite{TardivoDFMP23} & \href{../works/Tassel22.pdf}{Tassel22}~\cite{Tassel22} & \href{../works/TasselGS23.pdf}{TasselGS23}~\cite{TasselGS23} & \href{../}{Tay92}~\cite{Tay92} & \href{../works/Tayyab2023.pdf}{Tayyab2023}~\cite{Tayyab2023} & \href{../works/Teppan22.pdf}{Teppan22}~\cite{Teppan22}\\ 
\href{../works/Terashima-Marn2008.pdf}{Terashima-Marn2008}~\cite{Terashima-Marn2008} & \href{../works/Terashima-Marn2008a.pdf}{Terashima-Marn2008a}~\cite{Terashima-Marn2008a} & \href{../works/TerekhovDOB12.pdf}{TerekhovDOB12}~\cite{TerekhovDOB12} & \href{../works/TerekhovTDB14.pdf}{TerekhovTDB14}~\cite{TerekhovTDB14} & \href{../works/Tesch16.pdf}{Tesch16}~\cite{Tesch16} & \href{../works/Tesch18.pdf}{Tesch18}~\cite{Tesch18}\\ 
\href{../works/Tesch2020.pdf}{Tesch2020}~\cite{Tesch2020} & \href{../works/Teschemacher2016.pdf}{Teschemacher2016}~\cite{Teschemacher2016} & \href{../works/ThiruvadyBME09.pdf}{ThiruvadyBME09}~\cite{ThiruvadyBME09} & \href{../works/ThiruvadyWGS14.pdf}{ThiruvadyWGS14}~\cite{ThiruvadyWGS14} & \href{../works/ThomasKS20.pdf}{ThomasKS20}~\cite{ThomasKS20} & \href{../works/Thorsteinsson01.pdf}{Thorsteinsson01}~\cite{Thorsteinsson01}\\ 
\href{../works/Timpe02.pdf}{Timpe02}~\cite{Timpe02} & \href{../}{Timpe2003}~\cite{Timpe2003} & \href{../works/Tom19.pdf}{Tom19}~\cite{Tom19} & \href{../works/Tomczak2022.pdf}{Tomczak2022}~\cite{Tomczak2022} & \href{../works/TopalogluO11.pdf}{TopalogluO11}~\cite{TopalogluO11} & \href{../works/TopalogluSS12.pdf}{TopalogluSS12}~\cite{TopalogluSS12}\\ 
\href{../works/TorresL00.pdf}{TorresL00}~\cite{TorresL00} & \href{../works/TouatBT22.pdf}{TouatBT22}~\cite{TouatBT22} & \href{../works/Touraivane95.pdf}{Touraivane95}~\cite{Touraivane95} & \href{../works/TranAB16.pdf}{TranAB16}~\cite{TranAB16} & \href{../works/TranB12.pdf}{TranB12}~\cite{TranB12} & \href{../works/TranDRFWOVB16.pdf}{TranDRFWOVB16}~\cite{TranDRFWOVB16}\\ 
\href{../works/TranPZLDB18.pdf}{TranPZLDB18}~\cite{TranPZLDB18} & \href{../works/TranTDB13.pdf}{TranTDB13}~\cite{TranTDB13} & \href{../works/TranVNB17.pdf}{TranVNB17}~\cite{TranVNB17} & \href{../works/TranVNB17a.pdf}{TranVNB17a}~\cite{TranVNB17a} & \href{../works/TranWDRFOVB16.pdf}{TranWDRFOVB16}~\cite{TranWDRFOVB16} & \href{../works/TrentesauxPT01.pdf}{TrentesauxPT01}~\cite{TrentesauxPT01}\\ 
\href{../works/Trick03.pdf}{Trick03}~\cite{Trick03} & \href{../}{Trick11}~\cite{Trick11} & \href{../works/Trilling2006.pdf}{Trilling2006}~\cite{Trilling2006} & \href{../}{Triska2011}~\cite{Triska2011} & \href{../works/Trker2018.pdf}{Trker2018}~\cite{Trker2018} & \href{../works/TrojetHL11.pdf}{TrojetHL11}~\cite{TrojetHL11}\\ 
\href{../works/Tsang03.pdf}{Tsang03}~\cite{Tsang03} & \href{../works/Tseng2008.pdf}{Tseng2008}~\cite{Tseng2008} & \href{../}{TsurutaS00}~\cite{TsurutaS00} & \href{../works/UnsalO13.pdf}{UnsalO13}~\cite{UnsalO13} & \href{../works/UnsalO19.pdf}{UnsalO19}~\cite{UnsalO19} & \href{../works/Valdes87.pdf}{Valdes87}~\cite{Valdes87}\\ 
\href{../works/ValleMGT03.pdf}{ValleMGT03}~\cite{ValleMGT03} & \href{../works/Valouxis2022.pdf}{Valouxis2022}~\cite{Valouxis2022} & \href{../works/VanczaM01.pdf}{VanczaM01}~\cite{VanczaM01} & \href{../works/Varnier2002.pdf}{Varnier2002}~\cite{Varnier2002} & \href{../}{Vazacopoulos2005}~\cite{Vazacopoulos2005} & \href{../works/Velez2013.pdf}{Velez2013}~\cite{Velez2013}\\ 
\href{../works/Velez2014.pdf}{Velez2014}~\cite{Velez2014} & \href{../works/Verfaillie2010.pdf}{Verfaillie2010}~\cite{Verfaillie2010} & \href{../works/VerfaillieL01.pdf}{VerfaillieL01}~\cite{VerfaillieL01} & \href{../works/Vilim02.pdf}{Vilim02}~\cite{Vilim02} & \href{../works/Vilim03.pdf}{Vilim03}~\cite{Vilim03} & \href{../works/Vilim04.pdf}{Vilim04}~\cite{Vilim04}\\ 
\href{../works/Vilim05.pdf}{Vilim05}~\cite{Vilim05} & \href{../works/Vilim09.pdf}{Vilim09}~\cite{Vilim09} & \href{../works/Vilim09a.pdf}{Vilim09a}~\cite{Vilim09a} & \href{../works/Vilim11.pdf}{Vilim11}~\cite{Vilim11} & \href{../works/VilimBC04.pdf}{VilimBC04}~\cite{VilimBC04} & \href{../works/VilimBC05.pdf}{VilimBC05}~\cite{VilimBC05}\\ 
\href{../works/VilimLS15.pdf}{VilimLS15}~\cite{VilimLS15} & \href{../}{VillaverdeP04}~\cite{VillaverdeP04} & \href{../works/VlkHT21.pdf}{VlkHT21}~\cite{VlkHT21} & \href{../works/Wallace06.pdf}{Wallace06}~\cite{Wallace06} & \href{../}{Wallace2008}~\cite{Wallace2008} & \href{../}{Wallace94}~\cite{Wallace94}\\ 
\href{../works/Wallace96.pdf}{Wallace96}~\cite{Wallace96} & \href{../works/WallaceF00.pdf}{WallaceF00}~\cite{WallaceF00} & \href{../works/WallaceY20.pdf}{WallaceY20}~\cite{WallaceY20} & \href{../works/Wang2007.pdf}{Wang2007}~\cite{Wang2007} & \href{../}{Wang2013}~\cite{Wang2013} & \href{../works/Wang2014.pdf}{Wang2014}~\cite{Wang2014}\\ 
\href{../works/Wang2015.pdf}{Wang2015}~\cite{Wang2015} & \href{../works/Wang2021.pdf}{Wang2021}~\cite{Wang2021} & \href{../works/WangB20.pdf}{WangB20}~\cite{WangB20} & \href{../works/WangB23.pdf}{WangB23}~\cite{WangB23} & \href{../works/WangMD15.pdf}{WangMD15}~\cite{WangMD15} & \href{../}{WariZ19}~\cite{WariZ19}\\ 
\href{../works/Watermeyer2020.pdf}{Watermeyer2020}~\cite{Watermeyer2020} & \href{../works/WatsonB08.pdf}{WatsonB08}~\cite{WatsonB08} & \href{../works/WatsonBHW99.pdf}{WatsonBHW99}~\cite{WatsonBHW99} & \href{../works/Weil1992.pdf}{Weil1992}~\cite{Weil1992} & \href{../works/WeilHFP95.pdf}{WeilHFP95}~\cite{WeilHFP95} & \href{../works/WessenCS20.pdf}{WessenCS20}~\cite{WessenCS20}\\ 
\href{../works/WessenCSFPM23.pdf}{WessenCSFPM23}~\cite{WessenCSFPM23} & \href{../}{Wikarek2019}~\cite{Wikarek2019} & \href{../works/WikarekS19.pdf}{WikarekS19}~\cite{WikarekS19} & \href{../works/WinterMMW22.pdf}{WinterMMW22}~\cite{WinterMMW22} & \href{../works/Wolf03.pdf}{Wolf03}~\cite{Wolf03} & \href{../works/Wolf05.pdf}{Wolf05}~\cite{Wolf05}\\ 
\href{../works/Wolf09.pdf}{Wolf09}~\cite{Wolf09} & \href{../works/Wolf11.pdf}{Wolf11}~\cite{Wolf11} & \href{../works/WolfS05.pdf}{WolfS05}~\cite{WolfS05} & \href{../works/WolfS05a.pdf}{WolfS05a}~\cite{WolfS05a} & \href{../works/WolinskiKG04.pdf}{WolinskiKG04}~\cite{WolinskiKG04} & \href{../}{Wu2008}~\cite{Wu2008}\\ 
\href{../works/WuBB05.pdf}{WuBB05}~\cite{WuBB05} & \href{../works/WuBB09.pdf}{WuBB09}~\cite{WuBB09} & \href{../}{Xia2021}~\cite{Xia2021} & \href{../works/Xidias2019.pdf}{Xidias2019}~\cite{Xidias2019} & \href{../works/Xing2006.pdf}{Xing2006}~\cite{Xing2006} & \href{../works/Xu2023.pdf}{Xu2023}~\cite{Xu2023}\\ 
\href{../works/Xujun2009.pdf}{Xujun2009}~\cite{Xujun2009} & \href{../works/Yan2003.pdf}{Yan2003}~\cite{Yan2003} & \href{../works/Yang2000.pdf}{Yang2000}~\cite{Yang2000} & \href{../works/Yang2009.pdf}{Yang2009}~\cite{Yang2009} & \href{../works/YangSS19.pdf}{YangSS19}~\cite{YangSS19} & \href{../works/Yasmin2017.pdf}{Yasmin2017}~\cite{Yasmin2017}\\ 
\href{../}{YeGMH94}~\cite{YeGMH94} & \href{../works/Yin2007.pdf}{Yin2007}~\cite{Yin2007} & \href{../works/YoshikawaKNW94.pdf}{YoshikawaKNW94}~\cite{YoshikawaKNW94} & \href{../works/Younes2003.pdf}{Younes2003}~\cite{Younes2003} & \href{../works/YounespourAKE19.pdf}{YounespourAKE19}~\cite{YounespourAKE19} & \href{../works/YoungFS17.pdf}{YoungFS17}~\cite{YoungFS17}\\ 
\href{../works/YunG02.pdf}{YunG02}~\cite{YunG02} & \href{../}{Yunes2005}~\cite{Yunes2005} & \href{../works/YunusogluY22.pdf}{YunusogluY22}~\cite{YunusogluY22} & \href{../works/YuraszeckMC23.pdf}{YuraszeckMC23}~\cite{YuraszeckMC23} & \href{../works/YuraszeckMCCR23.pdf}{YuraszeckMCCR23}~\cite{YuraszeckMCCR23} & \href{../works/YuraszeckMPV22.pdf}{YuraszeckMPV22}~\cite{YuraszeckMPV22}\\ 
\href{../works/Yvars2018.pdf}{Yvars2018}~\cite{Yvars2018} & \href{../works/Zahout21.pdf}{Zahout21}~\cite{Zahout21} & \href{../works/ZampelliVSDR13.pdf}{ZampelliVSDR13}~\cite{ZampelliVSDR13} & \href{../works/ZarandiASC20.pdf}{ZarandiASC20}~\cite{ZarandiASC20} & \href{../}{ZarandiB12}~\cite{ZarandiB12} & \href{../works/ZarandiKS16.pdf}{ZarandiKS16}~\cite{ZarandiKS16}\\ 
\href{../works/Zeballos10.pdf}{Zeballos10}~\cite{Zeballos10} & \href{../}{Zeballos2006}~\cite{Zeballos2006} & \href{../works/ZeballosCM10.pdf}{ZeballosCM10}~\cite{ZeballosCM10} & \href{../works/ZeballosH05.pdf}{ZeballosH05}~\cite{ZeballosH05} & \href{../works/ZeballosM09.pdf}{ZeballosM09}~\cite{ZeballosM09} & \href{../works/ZeballosNH11.pdf}{ZeballosNH11}~\cite{ZeballosNH11}\\ 
\href{../works/ZeballosQH10.pdf}{ZeballosQH10}~\cite{ZeballosQH10} & \href{../works/ZengM12.pdf}{ZengM12}~\cite{ZengM12} & \href{../works/Zhang2005.pdf}{Zhang2005}~\cite{Zhang2005} & \href{../works/Zhang2013.pdf}{Zhang2013}~\cite{Zhang2013} & \href{../}{Zhang2019}~\cite{Zhang2019} & \href{../works/ZhangBB22.pdf}{ZhangBB22}~\cite{ZhangBB22}\\ 
\href{../works/ZhangJZL22.pdf}{ZhangJZL22}~\cite{ZhangJZL22} & \href{../works/ZhangLS12.pdf}{ZhangLS12}~\cite{ZhangLS12} & \href{../works/ZhangW18.pdf}{ZhangW18}~\cite{ZhangW18} & \href{../works/ZhangYW21.pdf}{ZhangYW21}~\cite{ZhangYW21} & \href{../works/ZhaoL14.pdf}{ZhaoL14}~\cite{ZhaoL14} & \href{../works/Zhou96.pdf}{Zhou96}~\cite{Zhou96}\\ 
\href{../works/Zhou97.pdf}{Zhou97}~\cite{Zhou97} & \href{../works/ZhouGL15.pdf}{ZhouGL15}~\cite{ZhouGL15} & \href{../}{Zhu2006}~\cite{Zhu2006} & \href{../works/ZhuS02.pdf}{ZhuS02}~\cite{ZhuS02} & \href{../works/ZhuSZW23.pdf}{ZhuSZW23}~\cite{ZhuSZW23} & \href{../works/ZibranR11.pdf}{ZibranR11}~\cite{ZibranR11}\\ 
\href{../works/ZibranR11a.pdf}{ZibranR11a}~\cite{ZibranR11a} & \href{../}{Zohali2022}~\cite{Zohali2022} & \href{../works/Zou2012.pdf}{Zou2012}~\cite{Zou2012} & \href{../works/Zou2021.pdf}{Zou2021}~\cite{Zou2021} & \href{../works/ZouZ20.pdf}{ZouZ20}~\cite{ZouZ20} & \href{../}{Zoulfaghari2013}~\cite{Zoulfaghari2013}\\ 
\href{../works/Zuenko2021.pdf}{Zuenko2021}~\cite{Zuenko2021} & \href{../works/abs-0907-0939.pdf}{abs-0907-0939}~\cite{abs-0907-0939} & \href{../works/abs-1009-0347.pdf}{abs-1009-0347}~\cite{abs-1009-0347} & \href{../works/abs-1901-07914.pdf}{abs-1901-07914}~\cite{abs-1901-07914} & \href{../works/abs-1902-01193.pdf}{abs-1902-01193}~\cite{abs-1902-01193} & \href{../works/abs-1902-09244.pdf}{abs-1902-09244}~\cite{abs-1902-09244}\\ 
\href{../works/abs-1911-04766.pdf}{abs-1911-04766}~\cite{abs-1911-04766} & \href{../works/abs-2102-08778.pdf}{abs-2102-08778}~\cite{abs-2102-08778} & \href{../works/abs-2211-14492.pdf}{abs-2211-14492}~\cite{abs-2211-14492} & \href{../works/abs-2305-19888.pdf}{abs-2305-19888}~\cite{abs-2305-19888} & \href{../works/abs-2306-05747.pdf}{abs-2306-05747}~\cite{abs-2306-05747} & \href{../works/abs-2312-13682.pdf}{abs-2312-13682}~\cite{abs-2312-13682}\\ 
\href{../works/abs-2402-00459.pdf}{abs-2402-00459}~\cite{abs-2402-00459} & \end{longtable}


\section{Conference Paper List}

This section presents the information for all conference papers included in the survey. For space reasons, not all information about the papers can be presented in a single table, we therefore split the data into three parts. The first part contains the main bibliographical information for the paper. The papers are sorted by year of publication (newest first), and then alphabetically by key. 

The key field of the table contains a hyperlink to the original source URL of the paper. You may have to navigate manually to see or download the actual paper content, and you may be unable to access the paper at all if it is behind a paywall for which you (or your organization) do not have access.

We then list the authors of the paper, in the order given in the bibtex file, abbreviating first names for space where we can identify them. Note that names with non-latin characters are not handled by latex. We use the form that is given in the bibtex file, but have excluded entries that cause latex to fail.  

We then give the title of the publication, using the original capitalization of the title entry in the bibtex entry, which may differ from the format shown in the bibliography. We then (column LC) provide a link to a local copy, if it is present, and a link to the bibliography entry of the paper.  We also show the year of publication, and the conference where the paper was published, using a short form abbreviation of the conference. This relies on a matching routine in the Java code to find the short title, new conference series may require an additional entry in \texttt{ImportBibtex.java} to work properly. Finally we list the number of pages of the paper, this information is using the bibtex entry where possible, otherwise uses \texttt{pdfinfo} to extract the actual number of pages from the local copy. The final columns b and c provide links to the corresponding tables of extracted concepts and manual information. Note that the links typically go to the correct page, but do not necessarily scroll to the correct line in the table.

\clearpage
\subsection{Papers from bibtex}
{\scriptsize
\begin{longtable}{>{\raggedright\arraybackslash}p{3cm}>{\raggedright\arraybackslash}p{6cm}>{\raggedright\arraybackslash}p{6.5cm}rrrp{2.5cm}rrrrr}
\rowcolor{white}\caption{Works from bibtex (Total 320)}\\ \toprule
\rowcolor{white}Key & Authors & Title & LC & Cite & Year & \shortstack{Conference\\/Journal} & Pages & \shortstack{Nr\\Cites} & \shortstack{Nr\\Refs} & b & c \\ \midrule\endhead
\bottomrule
\endfoot
\rowlabel{a:AalianPG23}AalianPG23 \href{https://doi.org/10.4230/LIPIcs.CP.2023.6}{AalianPG23} & \hyperref[auth:a7]{Y. Aalian}, \hyperref[auth:a8]{G. Pesant}, \hyperref[auth:a9]{M. Gamache} & Optimization of Short-Term Underground Mine Planning Using Constraint Programming & \href{works/AalianPG23.pdf}{Yes} & \cite{AalianPG23} & 2023 & CP 2023 & 16 & 0 & 0 & \ref{b:AalianPG23} & \ref{c:AalianPG23}\\
\rowlabel{a:Bit-Monnot23}Bit-Monnot23 \href{https://doi.org/10.3233/FAIA230278}{Bit-Monnot23} & \hyperref[auth:a398]{A. Bit{-}Monnot} & Enhancing Hybrid {CP-SAT} Search for Disjunctive Scheduling & \href{works/Bit-Monnot23.pdf}{Yes} & \cite{Bit-Monnot23} & 2023 & ECAI 2023 & 8 & 0 & 0 & \ref{b:Bit-Monnot23} & \ref{c:Bit-Monnot23}\\
\rowlabel{a:EfthymiouY23}EfthymiouY23 \href{https://doi.org/10.1007/978-3-031-33271-5\_16}{EfthymiouY23} & \hyperref[auth:a18]{N. Efthymiou}, \hyperref[auth:a19]{N. Yorke{-}Smith} & Predicting the Optimal Period for Cyclic Hoist Scheduling Problems & \href{works/EfthymiouY23.pdf}{Yes} & \cite{EfthymiouY23} & 2023 & CPAIOR 2023 & 16 & 0 & 23 & \ref{b:EfthymiouY23} & \ref{c:EfthymiouY23}\\
\rowlabel{a:JuvinHHL23}JuvinHHL23 \href{https://doi.org/10.4230/LIPIcs.CP.2023.19}{JuvinHHL23} & \hyperref[auth:a0]{C. Juvin}, \hyperref[auth:a1]{E. Hebrard}, \hyperref[auth:a2]{L. Houssin}, \hyperref[auth:a3]{P. Lopez} & An Efficient Constraint Programming Approach to Preemptive Job Shop Scheduling & \href{works/JuvinHHL23.pdf}{Yes} & \cite{JuvinHHL23} & 2023 & CP 2023 & 16 & 0 & 0 & \ref{b:JuvinHHL23} & \ref{c:JuvinHHL23}\\
\rowlabel{a:JuvinHL23}JuvinHL23 \href{https://doi.org/10.1007/978-3-031-33271-5\_23}{JuvinHL23} & \hyperref[auth:a0]{C. Juvin}, \hyperref[auth:a2]{L. Houssin}, \hyperref[auth:a3]{P. Lopez} & Constraint Programming for the Robust Two-Machine Flow-Shop Scheduling Problem with Budgeted Uncertainty & \href{works/JuvinHL23.pdf}{Yes} & \cite{JuvinHL23} & 2023 & CPAIOR 2023 & 16 & 0 & 11 & \ref{b:JuvinHL23} & \ref{c:JuvinHL23}\\
\rowlabel{a:KameugneFND23}KameugneFND23 \href{https://doi.org/10.4230/LIPIcs.CP.2023.20}{KameugneFND23} & \hyperref[auth:a10]{R. Kameugne}, \hyperref[auth:a11]{S{\'{e}}v{\'{e}}rine Betmbe Fetgo}, \hyperref[auth:a12]{T. Noulamo}, \hyperref[auth:a13]{Cl{\'{e}}mentin Tayou Djam{\'{e}}gni} & Horizontally Elastic Edge Finder Rule for Cumulative Constraint Based on Slack and Density & \href{works/KameugneFND23.pdf}{Yes} & \cite{KameugneFND23} & 2023 & CP 2023 & 17 & 0 & 0 & \ref{b:KameugneFND23} & \ref{c:KameugneFND23}\\
\rowlabel{a:KimCMLLP23}KimCMLLP23 \href{https://doi.org/10.1007/978-3-031-33271-5\_31}{KimCMLLP23} & \hyperref[auth:a23]{D. Kim}, \hyperref[auth:a24]{Y. Choi}, \hyperref[auth:a25]{K. Moon}, \hyperref[auth:a26]{M. Lee}, \hyperref[auth:a27]{K. Lee}, \hyperref[auth:a28]{Michael L. Pinedo} & Iterated Greedy Constraint Programming for Scheduling Steelmaking Continuous Casting & \href{works/KimCMLLP23.pdf}{Yes} & \cite{KimCMLLP23} & 2023 & CPAIOR 2023 & 16 & 0 & 13 & \ref{b:KimCMLLP23} & \ref{c:KimCMLLP23}\\
\rowlabel{a:Mehdizadeh-Somarin23}Mehdizadeh-Somarin23 \href{https://doi.org/10.1007/978-3-031-43670-3\_33}{Mehdizadeh-Somarin23} & \hyperref[auth:a435]{Z. Mehdizadeh{-}Somarin}, \hyperref[auth:a436]{R. Tavakkoli{-}Moghaddam}, \hyperref[auth:a437]{M. Rohaninejad}, \hyperref[auth:a116]{Z. Hanz{\'{a}}lek}, \hyperref[auth:a438]{Behdin Vahedi Nouri} & A Constraint Programming Model for a Reconfigurable Job Shop Scheduling Problem with Machine Availability & \href{works/Mehdizadeh-Somarin23.pdf}{Yes} & \cite{Mehdizadeh-Somarin23} & 2023 & APMS 2023 & 14 & 0 & 0 & \ref{b:Mehdizadeh-Somarin23} & \ref{c:Mehdizadeh-Somarin23}\\
\rowlabel{a:PerezGSL23}PerezGSL23 \href{https://doi.org/10.1109/ICTAI59109.2023.00108}{PerezGSL23} & \hyperref[auth:a431]{G. Perez}, \hyperref[auth:a432]{G. Glorian}, \hyperref[auth:a433]{W. Suijlen}, \hyperref[auth:a434]{A. Lallouet} & A Constraint Programming Model for Scheduling the Unloading of Trains in Ports & \href{works/PerezGSL23.pdf}{Yes} & \cite{PerezGSL23} & 2023 & ICTAI 2023 & 7 & 0 & 0 & \ref{b:PerezGSL23} & \ref{c:PerezGSL23}\\
\rowlabel{a:PovedaAA23}PovedaAA23 \href{https://doi.org/10.4230/LIPIcs.CP.2023.31}{PovedaAA23} & \hyperref[auth:a4]{G. Pov{\'{e}}da}, \hyperref[auth:a5]{N. {\'{A}}lvarez}, \hyperref[auth:a6]{C. Artigues} & Partially Preemptive Multi Skill/Mode Resource-Constrained Project Scheduling with Generalized Precedence Relations and Calendars & \href{works/PovedaAA23.pdf}{Yes} & \cite{PovedaAA23} & 2023 & CP 2023 & 21 & 0 & 0 & \ref{b:PovedaAA23} & \ref{c:PovedaAA23}\\
\rowlabel{a:SquillaciPR23}SquillaciPR23 \href{https://doi.org/10.1007/978-3-031-33271-5\_29}{SquillaciPR23} & \hyperref[auth:a20]{S. Squillaci}, \hyperref[auth:a21]{C. Pralet}, \hyperref[auth:a22]{S. Roussel} & Scheduling Complex Observation Requests for a Constellation of Satellites: Large Neighborhood Search Approaches & \href{works/SquillaciPR23.pdf}{Yes} & \cite{SquillaciPR23} & 2023 & CPAIOR 2023 & 17 & 0 & 19 & \ref{b:SquillaciPR23} & \ref{c:SquillaciPR23}\\
\rowlabel{a:TardivoDFMP23}TardivoDFMP23 \href{https://doi.org/10.1007/978-3-031-33271-5\_22}{TardivoDFMP23} & \hyperref[auth:a29]{F. Tardivo}, \hyperref[auth:a30]{A. Dovier}, \hyperref[auth:a31]{A. Formisano}, \hyperref[auth:a32]{L. Michel}, \hyperref[auth:a33]{E. Pontelli} & Constraint Propagation on {GPU:} {A} Case Study for the Cumulative Constraint & \href{works/TardivoDFMP23.pdf}{Yes} & \cite{TardivoDFMP23} & 2023 & CPAIOR 2023 & 18 & 0 & 30 & \ref{b:TardivoDFMP23} & \ref{c:TardivoDFMP23}\\
\rowlabel{a:TasselGS23}TasselGS23 \href{https://doi.org/10.1609/icaps.v33i1.27243}{TasselGS23} & \hyperref[auth:a58]{P. Tassel}, \hyperref[auth:a61]{M. Gebser}, \hyperref[auth:a429]{K. Schekotihin} & An End-to-End Reinforcement Learning Approach for Job-Shop Scheduling Problems Based on Constraint Programming & \href{works/TasselGS23.pdf}{Yes} & \cite{TasselGS23} & 2023 & ICAPS 2023 & 9 & 0 & 0 & \ref{b:TasselGS23} & \ref{c:TasselGS23}\\
\rowlabel{a:WangB23}WangB23 \href{https://doi.org/10.1109/ICTAI59109.2023.00062}{WangB23} & \hyperref[auth:a399]{R. Wang}, \hyperref[auth:a400]{N. Barnier} & Dynamic All-Different and Maximal Cliques Constraints for Fixed Job Scheduling & \href{works/WangB23.pdf}{Yes} & \cite{WangB23} & 2023 & ICTAI 2023 & 8 & 0 & 0 & \ref{b:WangB23} & \ref{c:WangB23}\\
\rowlabel{a:YuraszeckMC23}YuraszeckMC23 \href{https://doi.org/10.1016/j.procs.2023.03.130}{YuraszeckMC23} & \hyperref[auth:a411]{F. Yuraszeck}, \hyperref[auth:a430]{G. Mej{\'{\i}}a}, \hyperref[auth:a413]{D. Canut{-}de{-}Bon} & A competitive constraint programming approach for the group shop scheduling problem & \href{works/YuraszeckMC23.pdf}{Yes} & \cite{YuraszeckMC23} & 2023 & ANT 2023 & 6 & 1 & 15 & \ref{b:YuraszeckMC23} & \ref{c:YuraszeckMC23}\\
\rowlabel{a:ArmstrongGOS22}ArmstrongGOS22 \href{https://doi.org/10.1007/978-3-031-08011-1\_1}{ArmstrongGOS22} & \hyperref[auth:a14]{E. Armstrong}, \hyperref[auth:a15]{M. Garraffa}, \hyperref[auth:a16]{B. O'Sullivan}, \hyperref[auth:a17]{H. Simonis} & A Two-Phase Hybrid Approach for the Hybrid Flexible Flowshop with Transportation Times & \href{works/ArmstrongGOS22.pdf}{Yes} & \cite{ArmstrongGOS22} & 2022 & CPAIOR 2022 & 13 & 0 & 14 & \ref{b:ArmstrongGOS22} & \ref{c:ArmstrongGOS22}\\
\rowlabel{a:BoudreaultSLQ22}BoudreaultSLQ22 \href{https://doi.org/10.4230/LIPIcs.CP.2022.10}{BoudreaultSLQ22} & \hyperref[auth:a34]{R. Boudreault}, \hyperref[auth:a35]{V. Simard}, \hyperref[auth:a36]{D. Lafond}, \hyperref[auth:a37]{C. Quimper} & A Constraint Programming Approach to Ship Refit Project Scheduling & \href{works/BoudreaultSLQ22.pdf}{Yes} & \cite{BoudreaultSLQ22} & 2022 & CP 2022 & 16 & 0 & 0 & \ref{b:BoudreaultSLQ22} & \ref{c:BoudreaultSLQ22}\\
\rowlabel{a:GeitzGSSW22}GeitzGSSW22 \href{https://doi.org/10.1007/978-3-031-08011-1\_10}{GeitzGSSW22} & \hyperref[auth:a47]{M. Geitz}, \hyperref[auth:a48]{C. Grozea}, \hyperref[auth:a49]{W. Steigerwald}, \hyperref[auth:a50]{R. St{\"{o}}hr}, \hyperref[auth:a51]{A. Wolf} & Solving the Extended Job Shop Scheduling Problem with AGVs - Classical and Quantum Approaches & \href{works/GeitzGSSW22.pdf}{Yes} & \cite{GeitzGSSW22} & 2022 & CPAIOR 2022 & 18 & 0 & 24 & \ref{b:GeitzGSSW22} & \ref{c:GeitzGSSW22}\\
\rowlabel{a:HebrardALLCMR22}HebrardALLCMR22 \href{https://doi.org/10.24963/ijcai.2022/643}{HebrardALLCMR22} & \hyperref[auth:a1]{E. Hebrard}, \hyperref[auth:a6]{C. Artigues}, \hyperref[auth:a3]{P. Lopez}, \hyperref[auth:a797]{A. Lusson}, \hyperref[auth:a798]{Steve A. Chien}, \hyperref[auth:a799]{A. Maillard}, \hyperref[auth:a800]{Gregg R. Rabideau} & An Efficient Approach to Data Transfer Scheduling for Long Range Space Exploration & \href{works/HebrardALLCMR22.pdf}{Yes} & \cite{HebrardALLCMR22} & 2022 & IJCAI 2022 & 7 & 0 & 0 & \ref{b:HebrardALLCMR22} & \ref{c:HebrardALLCMR22}\\
\rowlabel{a:JungblutK22}JungblutK22 \href{https://doi.org/10.1109/IPDPSW55747.2022.00025}{JungblutK22} & \hyperref[auth:a750]{P. Jungblut}, \hyperref[auth:a751]{D. Kranzlm{\"{u}}ller} & Optimal Schedules for High-Level Programming Environments on FPGAs with Constraint Programming & \href{works/JungblutK22.pdf}{Yes} & \cite{JungblutK22} & 2022 & IPDPS 2022 & 4 & 0 & 0 & \ref{b:JungblutK22} & \ref{c:JungblutK22}\\
\rowlabel{a:LiFJZLL22}LiFJZLL22 \href{https://doi.org/10.1109/ICNSC55942.2022.10004158}{LiFJZLL22} & \hyperref[auth:a467]{X. Li}, \hyperref[auth:a468]{J. Fu}, \hyperref[auth:a469]{Z. Jia}, \hyperref[auth:a470]{Z. Zhao}, \hyperref[auth:a471]{S. Li}, \hyperref[auth:a472]{S. Liu} & Constraint Programming for a Novel Integrated Optimization of Blocking Job Shop Scheduling and Variable-Speed Transfer Robot Assignment & \href{works/LiFJZLL22.pdf}{Yes} & \cite{LiFJZLL22} & 2022 & ICNSC 2022 & 6 & 0 & 31 & \ref{b:LiFJZLL22} & \ref{c:LiFJZLL22}\\
\rowlabel{a:LuoB22}LuoB22 \href{https://doi.org/10.1007/978-3-031-08011-1\_17}{LuoB22} & \hyperref[auth:a755]{Yiqing L. Luo}, \hyperref[auth:a89]{J. Christopher Beck} & Packing by Scheduling: Using Constraint Programming to Solve a Complex 2D Cutting Stock Problem & \href{works/LuoB22.pdf}{Yes} & \cite{LuoB22} & 2022 & CPAIOR 2022 & 17 & 0 & 28 & \ref{b:LuoB22} & \ref{c:LuoB22}\\
\rowlabel{a:OuelletQ22}OuelletQ22 \href{https://doi.org/10.1007/978-3-031-08011-1\_21}{OuelletQ22} & \hyperref[auth:a52]{Y. Ouellet}, \hyperref[auth:a37]{C. Quimper} & A MinCumulative Resource Constraint & \href{works/OuelletQ22.pdf}{Yes} & \cite{OuelletQ22} & 2022 & CPAIOR 2022 & 17 & 1 & 22 & \ref{b:OuelletQ22} & \ref{c:OuelletQ22}\\
\rowlabel{a:OujanaAYB22}OujanaAYB22 \href{https://doi.org/10.1109/CoDIT55151.2022.9803972}{OujanaAYB22} & \hyperref[auth:a460]{S. Oujana}, \hyperref[auth:a461]{L. Amodeo}, \hyperref[auth:a462]{F. Yalaoui}, \hyperref[auth:a463]{D. Brodart} & Solving a realistic hybrid and flexible flow shop scheduling problem through constraint programming: industrial case in a packaging company & \href{works/OujanaAYB22.pdf}{Yes} & \cite{OujanaAYB22} & 2022 & CoDIT 2022 & 6 & 1 & 21 & \ref{b:OujanaAYB22} & \ref{c:OujanaAYB22}\\
\rowlabel{a:PopovicCGNC22}PopovicCGNC22 \href{https://doi.org/10.4230/LIPIcs.CP.2022.34}{PopovicCGNC22} & \hyperref[auth:a38]{L. Popovic}, \hyperref[auth:a39]{A. C{\^{o}}t{\'{e}}}, \hyperref[auth:a40]{M. Gaha}, \hyperref[auth:a41]{F. Nguewouo}, \hyperref[auth:a42]{Q. Cappart} & Scheduling the Equipment Maintenance of an Electric Power Transmission Network Using Constraint Programming & \href{works/PopovicCGNC22.pdf}{Yes} & \cite{PopovicCGNC22} & 2022 & CP 2022 & 15 & 0 & 0 & \ref{b:PopovicCGNC22} & \ref{c:PopovicCGNC22}\\
\rowlabel{a:SvancaraB22}SvancaraB22 \href{https://doi.org/10.5220/0010869700003116}{SvancaraB22} & \hyperref[auth:a788]{J. Svancara}, \hyperref[auth:a153]{R. Bart{\'{a}}k} & Tackling Train Routing via Multi-agent Pathfinding and Constraint-based Scheduling & \href{works/SvancaraB22.pdf}{Yes} & \cite{SvancaraB22} & 2022 & ICAART 2022 & 8 & 0 & 0 & \ref{b:SvancaraB22} & \ref{c:SvancaraB22}\\
\rowlabel{a:Teppan22}Teppan22 \href{https://doi.org/10.5220/0010849900003116}{Teppan22} & \hyperref[auth:a94]{Erich Christian Teppan} & Types of Flexible Job Shop Scheduling: {A} Constraint Programming Experiment & \href{works/Teppan22.pdf}{Yes} & \cite{Teppan22} & 2022 & ICAART 2022 & 8 & 0 & 0 & \ref{b:Teppan22} & \ref{c:Teppan22}\\
\rowlabel{a:TouatBT22}TouatBT22 \href{}{TouatBT22} & \hyperref[auth:a464]{M. Touat}, \hyperref[auth:a465]{B. Benhamou}, \hyperref[auth:a466]{Fatima Benbouzid{-}Si Tayeb} & A Constraint Programming Model for the Scheduling Problem with Flexible Maintenance under Human Resource Constraints & \href{works/TouatBT22.pdf}{Yes} & \cite{TouatBT22} & 2022 & ICAART 2022 & 8 & 0 & 0 & \ref{b:TouatBT22} & \ref{c:TouatBT22}\\
\rowlabel{a:WinterMMW22}WinterMMW22 \href{https://doi.org/10.4230/LIPIcs.CP.2022.41}{WinterMMW22} & \hyperref[auth:a43]{F. Winter}, \hyperref[auth:a44]{S. Meiswinkel}, \hyperref[auth:a45]{N. Musliu}, \hyperref[auth:a46]{D. Walkiewicz} & Modeling and Solving Parallel Machine Scheduling with Contamination Constraints in the Agricultural Industry & \href{works/WinterMMW22.pdf}{Yes} & \cite{WinterMMW22} & 2022 & CP 2022 & 18 & 0 & 0 & \ref{b:WinterMMW22} & \ref{c:WinterMMW22}\\
\rowlabel{a:ZhangBB22}ZhangBB22 \href{https://ojs.aaai.org/index.php/ICAPS/article/view/19826}{ZhangBB22} & \hyperref[auth:a809]{J. Zhang}, \hyperref[auth:a810]{Giovanni Lo Bianco}, \hyperref[auth:a89]{J. Christopher Beck} & Solving Job-Shop Scheduling Problems with QUBO-Based Specialized Hardware & \href{works/ZhangBB22.pdf}{Yes} & \cite{ZhangBB22} & 2022 & ICAPS 2022 & 9 & 0 & 0 & \ref{b:ZhangBB22} & \ref{c:ZhangBB22}\\
\rowlabel{a:ZhangJZL22}ZhangJZL22 \href{https://doi.org/10.1109/ICNSC55942.2022.10004154}{ZhangJZL22} & \hyperref[auth:a473]{H. Zhang}, \hyperref[auth:a474]{Y. Ji}, \hyperref[auth:a470]{Z. Zhao}, \hyperref[auth:a472]{S. Liu} & Constraint Programming for Modeling and Solving a Hybrid Flow Shop Scheduling Problem & \href{works/ZhangJZL22.pdf}{Yes} & \cite{ZhangJZL22} & 2022 & ICNSC 2022 & 6 & 0 & 21 & \ref{b:ZhangJZL22} & \ref{c:ZhangJZL22}\\
\rowlabel{a:AntuoriHHEN21}AntuoriHHEN21 \href{https://doi.org/10.4230/LIPIcs.CP.2021.14}{AntuoriHHEN21} & \hyperref[auth:a53]{V. Antuori}, \hyperref[auth:a1]{E. Hebrard}, \hyperref[auth:a54]{M. Huguet}, \hyperref[auth:a55]{S. Essodaigui}, \hyperref[auth:a56]{A. Nguyen} & Combining Monte Carlo Tree Search and Depth First Search Methods for a Car Manufacturing Workshop Scheduling Problem & \href{works/AntuoriHHEN21.pdf}{Yes} & \cite{AntuoriHHEN21} & 2021 & CP 2021 & 16 & 0 & 0 & \ref{b:AntuoriHHEN21} & \ref{c:AntuoriHHEN21}\\
\rowlabel{a:ArmstrongGOS21}ArmstrongGOS21 \href{https://doi.org/10.4230/LIPIcs.CP.2021.16}{ArmstrongGOS21} & \hyperref[auth:a14]{E. Armstrong}, \hyperref[auth:a15]{M. Garraffa}, \hyperref[auth:a16]{B. O'Sullivan}, \hyperref[auth:a17]{H. Simonis} & The Hybrid Flexible Flowshop with Transportation Times & \href{works/ArmstrongGOS21.pdf}{Yes} & \cite{ArmstrongGOS21} & 2021 & CP 2021 & 18 & 1 & 0 & \ref{b:ArmstrongGOS21} & \ref{c:ArmstrongGOS21}\\
\rowlabel{a:ArtiguesHQT21}ArtiguesHQT21 \href{https://doi.org/10.5220/0010190101290136}{ArtiguesHQT21} & \hyperref[auth:a6]{C. Artigues}, \hyperref[auth:a1]{E. Hebrard}, \hyperref[auth:a801]{A. Quilliot}, \hyperref[auth:a802]{H. Toussaint} & Multi-Mode {RCPSP} with Safety Margin Maximization: Models and Algorithms & No & \cite{ArtiguesHQT21} & 2021 & ICORES 2021 & 8 & 0 & 0 & No & \ref{c:ArtiguesHQT21}\\
\rowlabel{a:Astrand0F21}Astrand0F21 \href{https://doi.org/10.1007/978-3-030-78230-6\_23}{Astrand0F21} & \hyperref[auth:a74]{M. {\AA}strand}, \hyperref[auth:a75]{M. Johansson}, \hyperref[auth:a76]{Hamid Reza Feyzmahdavian} & Short-Term Scheduling of Production Fleets in Underground Mines Using CP-Based {LNS} & \href{works/Astrand0F21.pdf}{Yes} & \cite{Astrand0F21} & 2021 & CPAIOR 2021 & 18 & 2 & 25 & \ref{b:Astrand0F21} & \ref{c:Astrand0F21}\\
\rowlabel{a:BenderWS21}BenderWS21 \href{https://doi.org/10.1007/978-3-030-87672-2\_37}{BenderWS21} & \hyperref[auth:a500]{T. Bender}, \hyperref[auth:a501]{D. Wittwer}, \hyperref[auth:a502]{T. Schmidt} & Applying Constraint Programming to the Multi-mode Scheduling Problem in Harvest Logistics & \href{works/BenderWS21.pdf}{Yes} & \cite{BenderWS21} & 2021 & ICCL 2021 & 16 & 1 & 16 & \ref{b:BenderWS21} & \ref{c:BenderWS21}\\
\rowlabel{a:GeibingerKKMMW21}GeibingerKKMMW21 \href{https://doi.org/10.1007/978-3-030-78230-6\_29}{GeibingerKKMMW21} & \hyperref[auth:a77]{T. Geibinger}, \hyperref[auth:a78]{L. Kletzander}, \hyperref[auth:a79]{M. Krainz}, \hyperref[auth:a80]{F. Mischek}, \hyperref[auth:a45]{N. Musliu}, \hyperref[auth:a43]{F. Winter} & Physician Scheduling During a Pandemic & \href{works/GeibingerKKMMW21.pdf}{Yes} & \cite{GeibingerKKMMW21} & 2021 & CPAIOR 2021 & 10 & 0 & 6 & \ref{b:GeibingerKKMMW21} & \ref{c:GeibingerKKMMW21}\\
\rowlabel{a:GeibingerMM21}GeibingerMM21 \href{https://doi.org/10.1609/aaai.v35i7.16789}{GeibingerMM21} & \hyperref[auth:a77]{T. Geibinger}, \hyperref[auth:a80]{F. Mischek}, \hyperref[auth:a45]{N. Musliu} & Constraint Logic Programming for Real-World Test Laboratory Scheduling & \href{works/GeibingerMM21.pdf}{Yes} & \cite{GeibingerMM21} & 2021 & AAAI 2021 & 9 & 0 & 0 & \ref{b:GeibingerMM21} & \ref{c:GeibingerMM21}\\
\rowlabel{a:HanenKP21}HanenKP21 \href{https://doi.org/10.1007/978-3-030-78230-6\_14}{HanenKP21} & \hyperref[auth:a71]{C. Hanen}, \hyperref[auth:a72]{Alix Munier Kordon}, \hyperref[auth:a73]{T. Pedersen} & Two Deadline Reduction Algorithms for Scheduling Dependent Tasks on Parallel Processors & \href{works/HanenKP21.pdf}{Yes} & \cite{HanenKP21} & 2021 & CPAIOR 2021 & 17 & 1 & 24 & \ref{b:HanenKP21} & \ref{c:HanenKP21}\\
\rowlabel{a:HillTV21}HillTV21 \href{https://doi.org/10.1007/978-3-030-78230-6\_2}{HillTV21} & \hyperref[auth:a64]{A. Hill}, \hyperref[auth:a65]{J. Ticktin}, \hyperref[auth:a66]{Thomas W. M. Vossen} & A Computational Study of Constraint Programming Approaches for Resource-Constrained Project Scheduling with Autonomous Learning Effects & \href{works/HillTV21.pdf}{Yes} & \cite{HillTV21} & 2021 & CPAIOR 2021 & 19 & 0 & 38 & \ref{b:HillTV21} & \ref{c:HillTV21}\\
\rowlabel{a:KlankeBYE21}KlankeBYE21 \href{https://doi.org/10.1007/978-3-030-78230-6\_9}{KlankeBYE21} & \hyperref[auth:a67]{C. Klanke}, \hyperref[auth:a68]{Dominik R. Bleidorn}, \hyperref[auth:a69]{V. Yfantis}, \hyperref[auth:a70]{S. Engell} & Combining Constraint Programming and Temporal Decomposition Approaches - Scheduling of an Industrial Formulation Plant & \href{works/KlankeBYE21.pdf}{Yes} & \cite{KlankeBYE21} & 2021 & CPAIOR 2021 & 16 & 3 & 13 & \ref{b:KlankeBYE21} & \ref{c:KlankeBYE21}\\
\rowlabel{a:KovacsTKSG21}KovacsTKSG21 \href{https://doi.org/10.4230/LIPIcs.CP.2021.36}{KovacsTKSG21} & \hyperref[auth:a57]{B. Kov{\'{a}}cs}, \hyperref[auth:a58]{P. Tassel}, \hyperref[auth:a59]{W. Kohlenbrein}, \hyperref[auth:a60]{P. Schrott{-}Kostwein}, \hyperref[auth:a61]{M. Gebser} & Utilizing Constraint Optimization for Industrial Machine Workload Balancing & \href{works/KovacsTKSG21.pdf}{Yes} & \cite{KovacsTKSG21} & 2021 & CP 2021 & 17 & 0 & 0 & \ref{b:KovacsTKSG21} & \ref{c:KovacsTKSG21}\\
\rowlabel{a:LacknerMMWW21}LacknerMMWW21 \href{https://doi.org/10.4230/LIPIcs.CP.2021.37}{LacknerMMWW21} & \hyperref[auth:a62]{M. Lackner}, \hyperref[auth:a63]{C. Mrkvicka}, \hyperref[auth:a45]{N. Musliu}, \hyperref[auth:a46]{D. Walkiewicz}, \hyperref[auth:a43]{F. Winter} & Minimizing Cumulative Batch Processing Time for an Industrial Oven Scheduling Problem & \href{works/LacknerMMWW21.pdf}{Yes} & \cite{LacknerMMWW21} & 2021 & CP 2021 & 18 & 0 & 0 & \ref{b:LacknerMMWW21} & \ref{c:LacknerMMWW21}\\
\rowlabel{a:AntuoriHHEN20}AntuoriHHEN20 \href{https://doi.org/10.1007/978-3-030-58475-7\_38}{AntuoriHHEN20} & \hyperref[auth:a53]{V. Antuori}, \hyperref[auth:a1]{E. Hebrard}, \hyperref[auth:a54]{M. Huguet}, \hyperref[auth:a55]{S. Essodaigui}, \hyperref[auth:a56]{A. Nguyen} & Leveraging Reinforcement Learning, Constraint Programming and Local Search: {A} Case Study in Car Manufacturing & \href{works/AntuoriHHEN20.pdf}{Yes} & \cite{AntuoriHHEN20} & 2020 & CP 2020 & 16 & 3 & 8 & \ref{b:AntuoriHHEN20} & \ref{c:AntuoriHHEN20}\\
\rowlabel{a:BarzegaranZP20}BarzegaranZP20 \href{https://doi.org/10.4230/OASIcs.Fog-IoT.2020.3}{BarzegaranZP20} & \hyperref[auth:a528]{M. Barzegaran}, \hyperref[auth:a529]{B. Zarrin}, \hyperref[auth:a530]{P. Pop} & Quality-Of-Control-Aware Scheduling of Communication in TSN-Based Fog Computing Platforms Using Constraint Programming & \href{works/BarzegaranZP20.pdf}{Yes} & \cite{BarzegaranZP20} & 2020 & Fog-IoT 2020 & 9 & 0 & 0 & \ref{b:BarzegaranZP20} & \ref{c:BarzegaranZP20}\\
\rowlabel{a:GodetLHS20}GodetLHS20 \href{https://doi.org/10.1609/aaai.v34i02.5510}{GodetLHS20} & \hyperref[auth:a478]{A. Godet}, \hyperref[auth:a247]{X. Lorca}, \hyperref[auth:a1]{E. Hebrard}, \hyperref[auth:a127]{G. Simonin} & Using Approximation within Constraint Programming to Solve the Parallel Machine Scheduling Problem with Additional Unit Resources & \href{works/GodetLHS20.pdf}{Yes} & \cite{GodetLHS20} & 2020 & AAAI 2020 & 8 & 1 & 0 & \ref{b:GodetLHS20} & \ref{c:GodetLHS20}\\
\rowlabel{a:GroleazNS20}GroleazNS20 \href{https://doi.org/10.1007/978-3-030-58475-7\_36}{GroleazNS20} & \hyperref[auth:a83]{L. Groleaz}, \hyperref[auth:a84]{Samba Ndojh Ndiaye}, \hyperref[auth:a85]{C. Solnon} & Solving the Group Cumulative Scheduling Problem with {CPO} and {ACO} & \href{works/GroleazNS20.pdf}{Yes} & \cite{GroleazNS20} & 2020 & CP 2020 & 17 & 1 & 25 & \ref{b:GroleazNS20} & \ref{c:GroleazNS20}\\
\rowlabel{a:GroleazNS20a}GroleazNS20a \href{https://doi.org/10.1145/3377930.3389818}{GroleazNS20a} & \hyperref[auth:a83]{L. Groleaz}, \hyperref[auth:a84]{Samba Ndojh Ndiaye}, \hyperref[auth:a85]{C. Solnon} & {ACO} with automatic parameter selection for a scheduling problem with a group cumulative constraint & \href{works/GroleazNS20a.pdf}{Yes} & \cite{GroleazNS20a} & 2020 & GECCO 2020 & 9 & 3 & 28 & \ref{b:GroleazNS20a} & \ref{c:GroleazNS20a}\\
\rowlabel{a:Mercier-AubinGQ20}Mercier-AubinGQ20 \href{https://doi.org/10.1007/978-3-030-58942-4\_22}{Mercier-AubinGQ20} & \hyperref[auth:a86]{A. Mercier{-}Aubin}, \hyperref[auth:a87]{J. Gaudreault}, \hyperref[auth:a37]{C. Quimper} & Leveraging Constraint Scheduling: {A} Case Study to the Textile Industry & \href{works/Mercier-AubinGQ20.pdf}{Yes} & \cite{Mercier-AubinGQ20} & 2020 & CPAIOR 2020 & 13 & 2 & 13 & \ref{b:Mercier-AubinGQ20} & \ref{c:Mercier-AubinGQ20}\\
\rowlabel{a:NattafM20}NattafM20 \href{https://doi.org/10.1007/978-3-030-58475-7\_27}{NattafM20} & \hyperref[auth:a81]{M. Nattaf}, \hyperref[auth:a82]{A. Malapert} & Filtering Rules for Flow Time Minimization in a Parallel Machine Scheduling Problem & \href{works/NattafM20.pdf}{Yes} & \cite{NattafM20} & 2020 & CP 2020 & 16 & 0 & 6 & \ref{b:NattafM20} & \ref{c:NattafM20}\\
\rowlabel{a:TangB20}TangB20 \href{https://doi.org/10.1007/978-3-030-58942-4\_28}{TangB20} & \hyperref[auth:a88]{Tanya Y. Tang}, \hyperref[auth:a89]{J. Christopher Beck} & {CP} and Hybrid Models for Two-Stage Batching and Scheduling & \href{works/TangB20.pdf}{Yes} & \cite{TangB20} & 2020 & CPAIOR 2020 & 16 & 6 & 12 & \ref{b:TangB20} & \ref{c:TangB20}\\
\rowlabel{a:WangB20}WangB20 \href{https://doi.org/10.3233/FAIA200114}{WangB20} & \hyperref[auth:a399]{R. Wang}, \hyperref[auth:a400]{N. Barnier} & Global Propagation of Transition Cost for Fixed Job Scheduling & \href{works/WangB20.pdf}{Yes} & \cite{WangB20} & 2020 & ECAI 2020 & 8 & 0 & 0 & \ref{b:WangB20} & \ref{c:WangB20}\\
\rowlabel{a:WessenCS20}WessenCS20 \href{https://doi.org/10.1007/978-3-030-58942-4\_33}{WessenCS20} & \hyperref[auth:a90]{J. Wess{\'{e}}n}, \hyperref[auth:a91]{M. Carlsson}, \hyperref[auth:a92]{C. Schulte} & Scheduling of Dual-Arm Multi-tool Assembly Robots and Workspace Layout Optimization & \href{works/WessenCS20.pdf}{Yes} & \cite{WessenCS20} & 2020 & CPAIOR 2020 & 10 & 2 & 11 & \ref{b:WessenCS20} & \ref{c:WessenCS20}\\
\rowlabel{a:BadicaBIL19}BadicaBIL19 \href{https://doi.org/10.1007/978-3-030-32258-8\_17}{BadicaBIL19} & \hyperref[auth:a504]{A. Badica}, \hyperref[auth:a505]{C. Badica}, \hyperref[auth:a506]{M. Ivanovic}, \hyperref[auth:a550]{D. Logofatu} & Exploring the Space of Block Structured Scheduling Processes Using Constraint Logic Programming & \href{works/BadicaBIL19.pdf}{Yes} & \cite{BadicaBIL19} & 2019 & IDC 2019 & 11 & 2 & 6 & \ref{b:BadicaBIL19} & \ref{c:BadicaBIL19}\\
\rowlabel{a:BehrensLM19}BehrensLM19 \href{https://doi.org/10.1109/ICRA.2019.8794022}{BehrensLM19} & \hyperref[auth:a547]{Jan Kristof Behrens}, \hyperref[auth:a548]{R. Lange}, \hyperref[auth:a549]{M. Mansouri} & A Constraint Programming Approach to Simultaneous Task Allocation and Motion Scheduling for Industrial Dual-Arm Manipulation Tasks & \href{works/BehrensLM19.pdf}{Yes} & \cite{BehrensLM19} & 2019 & ICRA 2019 & 7 & 12 & 18 & \ref{b:BehrensLM19} & \ref{c:BehrensLM19}\\
\rowlabel{a:BogaerdtW19}BogaerdtW19 \href{https://doi.org/10.1007/978-3-030-19212-9\_38}{BogaerdtW19} & \hyperref[auth:a310]{Pim van den Bogaerdt}, \hyperref[auth:a311]{Mathijs de Weerdt} & Lower Bounds for Uniform Machine Scheduling Using Decision Diagrams & \href{works/BogaerdtW19.pdf}{Yes} & \cite{BogaerdtW19} & 2019 & CPAIOR 2019 & 16 & 1 & 16 & \ref{b:BogaerdtW19} & \ref{c:BogaerdtW19}\\
\rowlabel{a:ColT19}ColT19 \href{https://doi.org/10.1007/978-3-030-30048-7\_9}{ColT19} & \hyperref[auth:a93]{Giacomo Da Col}, \hyperref[auth:a94]{Erich Christian Teppan} & Industrial Size Job Shop Scheduling Tackled by Present Day {CP} Solvers & \href{works/ColT19.pdf}{Yes} & \cite{ColT19} & 2019 & CP 2019 & 17 & 11 & 12 & \ref{b:ColT19} & \ref{c:ColT19}\\
\rowlabel{a:FrimodigS19}FrimodigS19 \href{https://doi.org/10.1007/978-3-030-30048-7\_25}{FrimodigS19} & \hyperref[auth:a95]{S. Frimodig}, \hyperref[auth:a92]{C. Schulte} & Models for Radiation Therapy Patient Scheduling & \href{works/FrimodigS19.pdf}{Yes} & \cite{FrimodigS19} & 2019 & CP 2019 & 17 & 3 & 26 & \ref{b:FrimodigS19} & \ref{c:FrimodigS19}\\
\rowlabel{a:FrohnerTR19}FrohnerTR19 \href{https://doi.org/10.1007/978-3-030-45093-9\_34}{FrohnerTR19} & \hyperref[auth:a544]{N. Frohner}, \hyperref[auth:a545]{S. Teuschl}, \hyperref[auth:a348]{G{\"{u}}nther R. Raidl} & Casual Employee Scheduling with Constraint Programming and Metaheuristics & \href{works/FrohnerTR19.pdf}{Yes} & \cite{FrohnerTR19} & 2019 & EUROCAST 2019 & 9 & 0 & 6 & \ref{b:FrohnerTR19} & \ref{c:FrohnerTR19}\\
\rowlabel{a:GalleguillosKSB19}GalleguillosKSB19 \href{https://doi.org/10.1007/978-3-030-30048-7\_26}{GalleguillosKSB19} & \hyperref[auth:a96]{C. Galleguillos}, \hyperref[auth:a97]{Z. Kiziltan}, \hyperref[auth:a98]{A. S{\^{\i}}rbu}, \hyperref[auth:a99]{{\"{O}}zalp Babaoglu} & Constraint Programming-Based Job Dispatching for Modern {HPC} Applications & \href{works/GalleguillosKSB19.pdf}{Yes} & \cite{GalleguillosKSB19} & 2019 & CP 2019 & 18 & 1 & 27 & \ref{b:GalleguillosKSB19} & \ref{c:GalleguillosKSB19}\\
\rowlabel{a:GeibingerMM19}GeibingerMM19 \href{https://doi.org/10.1007/978-3-030-19212-9\_20}{GeibingerMM19} & \hyperref[auth:a77]{T. Geibinger}, \hyperref[auth:a80]{F. Mischek}, \hyperref[auth:a45]{N. Musliu} & Investigating Constraint Programming for Real World Industrial Test Laboratory Scheduling & \href{works/GeibingerMM19.pdf}{Yes} & \cite{GeibingerMM19} & 2019 & CPAIOR 2019 & 16 & 6 & 15 & \ref{b:GeibingerMM19} & \ref{c:GeibingerMM19}\\
\rowlabel{a:KucukY19}KucukY19 \href{https://api.semanticscholar.org/CorpusID:198146161}{KucukY19} & \hyperref[auth:a772]{M. K{\"u}ç{\"u}k}, \hyperref[auth:a427]{Seyda Topaloglu Yildiz} & A Constraint Programming Approach for Agile Earth Observation Satellite Scheduling Problem & \href{works/KucukY19.pdf}{Yes} & \cite{KucukY19} & 2019 & RAST 2019 & 5 & 0 & 0 & \ref{b:KucukY19} & \ref{c:KucukY19}\\
\rowlabel{a:LiuLH19}LiuLH19 \href{https://doi.org/10.1007/978-3-030-19823-7\_19}{LiuLH19} & \hyperref[auth:a551]{K. Liu}, \hyperref[auth:a552]{S. L{\"{o}}ffler}, \hyperref[auth:a553]{P. Hofstedt} & Solving the Talent Scheduling Problem by Parallel Constraint Programming & \href{works/LiuLH19.pdf}{Yes} & \cite{LiuLH19} & 2019 & AIAI 2019 & 9 & 1 & 5 & \ref{b:LiuLH19} & \ref{c:LiuLH19}\\
\rowlabel{a:MalapertN19}MalapertN19 \href{https://doi.org/10.1007/978-3-030-19212-9\_28}{MalapertN19} & \hyperref[auth:a82]{A. Malapert}, \hyperref[auth:a81]{M. Nattaf} & A New CP-Approach for a Parallel Machine Scheduling Problem with Time Constraints on Machine Qualifications & \href{works/MalapertN19.pdf}{Yes} & \cite{MalapertN19} & 2019 & CPAIOR 2019 & 17 & 1 & 7 & \ref{b:MalapertN19} & \ref{c:MalapertN19}\\
\rowlabel{a:MurinR19}MurinR19 \href{https://doi.org/10.1007/978-3-030-30048-7\_27}{MurinR19} & \hyperref[auth:a100]{S. Mur{\'{\i}}n}, \hyperref[auth:a101]{H. Rudov{\'{a}}} & Scheduling of Mobile Robots Using Constraint Programming & \href{works/MurinR19.pdf}{Yes} & \cite{MurinR19} & 2019 & CP 2019 & 16 & 2 & 22 & \ref{b:MurinR19} & \ref{c:MurinR19}\\
\rowlabel{a:ParkUJR19}ParkUJR19 \href{https://doi.org/10.1007/978-3-030-19648-6\_15}{ParkUJR19} & \hyperref[auth:a554]{H. Park}, \hyperref[auth:a555]{J. Um}, \hyperref[auth:a556]{J. Jung}, \hyperref[auth:a557]{M. Ruskowski} & Developing a Production Scheduling System for Modular Factory Using Constraint Programming & \href{works/ParkUJR19.pdf}{Yes} & \cite{ParkUJR19} & 2019 & RAAD 2019 & 8 & 1 & 3 & \ref{b:ParkUJR19} & \ref{c:ParkUJR19}\\
\rowlabel{a:Tom19}Tom19 \href{https://doi.org/10.1109/FUZZ-IEEE.2019.8859029}{Tom19} & \hyperref[auth:a546]{M. Tom} & Fuzzy Multi-Constraint Programming Model for Weekly Meals Scheduling & \href{works/Tom19.pdf}{Yes} & \cite{Tom19} & 2019 & FUZZ-IEEE 2019 & 6 & 0 & 21 & \ref{b:Tom19} & \ref{c:Tom19}\\
\rowlabel{a:YangSS19}YangSS19 \href{https://doi.org/10.1007/978-3-030-19212-9\_42}{YangSS19} & \hyperref[auth:a312]{M. Yang}, \hyperref[auth:a125]{A. Schutt}, \hyperref[auth:a126]{Peter J. Stuckey} & Time Table Edge Finding with Energy Variables & \href{works/YangSS19.pdf}{Yes} & \cite{YangSS19} & 2019 & CPAIOR 2019 & 10 & 1 & 14 & \ref{b:YangSS19} & \ref{c:YangSS19}\\
\rowlabel{a:AntunesABDEGGOL18}AntunesABDEGGOL18 \href{https://doi.org/10.1109/ICTAI.2018.00027}{AntunesABDEGGOL18} & \hyperref[auth:a891]{M. Antunes}, \hyperref[auth:a892]{V. Armant}, \hyperref[auth:a223]{Kenneth N. Brown}, \hyperref[auth:a893]{Daniel A. Desmond}, \hyperref[auth:a894]{G. Escamocher}, \hyperref[auth:a895]{A. George}, \hyperref[auth:a183]{D. Grimes}, \hyperref[auth:a896]{M. O'Keeffe}, \hyperref[auth:a897]{Y. Lin}, \hyperref[auth:a16]{B. O'Sullivan}, \hyperref[auth:a898]{C. Ozturk}, \hyperref[auth:a899]{L. Quesada}, \hyperref[auth:a130]{M. Siala}, \hyperref[auth:a17]{H. Simonis}, \hyperref[auth:a838]{N. Wilson} & Assigning and Scheduling Service Visits in a Mixed Urban/Rural Setting & \href{works/AntunesABDEGGOL18.pdf}{Yes} & \cite{AntunesABDEGGOL18} & 2018 & ICTAI 2018 & 8 & 1 & 24 & \ref{b:AntunesABDEGGOL18} & \ref{c:AntunesABDEGGOL18}\\
\rowlabel{a:ArbaouiY18}ArbaouiY18 \href{https://doi.org/10.1007/978-3-319-75420-8\_67}{ArbaouiY18} & \hyperref[auth:a588]{T. Arbaoui}, \hyperref[auth:a462]{F. Yalaoui} & Solving the Unrelated Parallel Machine Scheduling Problem with Additional Resources Using Constraint Programming & \href{works/ArbaouiY18.pdf}{Yes} & \cite{ArbaouiY18} & 2018 & ACIIDS 2018 & 10 & 2 & 14 & \ref{b:ArbaouiY18} & \ref{c:ArbaouiY18}\\
\rowlabel{a:AstrandJZ18}AstrandJZ18 \href{https://doi.org/10.1007/978-3-319-93031-2\_44}{AstrandJZ18} & \hyperref[auth:a74]{M. {\AA}strand}, \hyperref[auth:a75]{M. Johansson}, \hyperref[auth:a205]{A. Zanarini} & Fleet Scheduling in Underground Mines Using Constraint Programming & \href{works/AstrandJZ18.pdf}{Yes} & \cite{AstrandJZ18} & 2018 & CPAIOR 2018 & 9 & 9 & 10 & \ref{b:AstrandJZ18} & \ref{c:AstrandJZ18}\\
\rowlabel{a:BenediktSMVH18}BenediktSMVH18 \href{https://doi.org/10.1007/978-3-319-93031-2\_6}{BenediktSMVH18} & \hyperref[auth:a114]{O. Benedikt}, \hyperref[auth:a313]{P. Sucha}, \hyperref[auth:a115]{I. M{\'{o}}dos}, \hyperref[auth:a314]{M. Vlk}, \hyperref[auth:a116]{Z. Hanz{\'{a}}lek} & Energy-Aware Production Scheduling with Power-Saving Modes & \href{works/BenediktSMVH18.pdf}{Yes} & \cite{BenediktSMVH18} & 2018 & CPAIOR 2018 & 10 & 2 & 12 & \ref{b:BenediktSMVH18} & \ref{c:BenediktSMVH18}\\
\rowlabel{a:CappartTSR18}CappartTSR18 \href{https://doi.org/10.1007/978-3-319-98334-9\_32}{CappartTSR18} & \hyperref[auth:a42]{Q. Cappart}, \hyperref[auth:a847]{C. Thomas}, \hyperref[auth:a148]{P. Schaus}, \hyperref[auth:a332]{L. Rousseau} & A Constraint Programming Approach for Solving Patient Transportation Problems & \href{works/CappartTSR18.pdf}{Yes} & \cite{CappartTSR18} & 2018 & CP 2018 & 17 & 6 & 31 & \ref{b:CappartTSR18} & \ref{c:CappartTSR18}\\
\rowlabel{a:DemirovicS18}DemirovicS18 \href{https://doi.org/10.1007/978-3-319-93031-2\_10}{DemirovicS18} & \hyperref[auth:a315]{E. Demirovic}, \hyperref[auth:a126]{Peter J. Stuckey} & Constraint Programming for High School Timetabling: {A} Scheduling-Based Model with Hot Starts & \href{works/DemirovicS18.pdf}{Yes} & \cite{DemirovicS18} & 2018 & CPAIOR 2018 & 18 & 4 & 16 & \ref{b:DemirovicS18} & \ref{c:DemirovicS18}\\
\rowlabel{a:He0GLW18}He0GLW18 \href{https://doi.org/10.1007/978-3-319-98334-9\_42}{He0GLW18} & \hyperref[auth:a186]{S. He}, \hyperref[auth:a117]{M. Wallace}, \hyperref[auth:a187]{G. Gange}, \hyperref[auth:a188]{A. Liebman}, \hyperref[auth:a189]{C. Wilson} & A Fast and Scalable Algorithm for Scheduling Large Numbers of Devices Under Real-Time Pricing & \href{works/He0GLW18.pdf}{Yes} & \cite{He0GLW18} & 2018 & CP 2018 & 18 & 6 & 26 & \ref{b:He0GLW18} & \ref{c:He0GLW18}\\
\rowlabel{a:HoYCLLCLC18}HoYCLLCLC18 \href{https://doi.org/10.1145/3299819.3299825}{HoYCLLCLC18} & \hyperref[auth:a589]{T. Ho}, \hyperref[auth:a590]{J. Yao}, \hyperref[auth:a591]{Y. Chang}, \hyperref[auth:a592]{F. Lai}, \hyperref[auth:a593]{J. Lai}, \hyperref[auth:a594]{S. Chu}, \hyperref[auth:a595]{W. Liao}, \hyperref[auth:a596]{H. Chiu} & A Platform for Dynamic Optimal Nurse Scheduling Based on Integer Linear Programming along with Multiple Criteria Constraints & \href{works/HoYCLLCLC18.pdf}{Yes} & \cite{HoYCLLCLC18} & 2018 & AICCC 2018 & 6 & 2 & 14 & \ref{b:HoYCLLCLC18} & \ref{c:HoYCLLCLC18}\\
\rowlabel{a:KameugneFGOQ18}KameugneFGOQ18 \href{https://doi.org/10.1007/978-3-319-93031-2\_23}{KameugneFGOQ18} & \hyperref[auth:a10]{R. Kameugne}, \hyperref[auth:a11]{S{\'{e}}v{\'{e}}rine Betmbe Fetgo}, \hyperref[auth:a316]{V. Gingras}, \hyperref[auth:a52]{Y. Ouellet}, \hyperref[auth:a37]{C. Quimper} & Horizontally Elastic Not-First/Not-Last Filtering Algorithm for Cumulative Resource Constraint & \href{works/KameugneFGOQ18.pdf}{Yes} & \cite{KameugneFGOQ18} & 2018 & CPAIOR 2018 & 17 & 1 & 12 & \ref{b:KameugneFGOQ18} & \ref{c:KameugneFGOQ18}\\
\rowlabel{a:Laborie18a}Laborie18a \href{https://doi.org/10.1007/978-3-319-93031-2\_29}{Laborie18a} & \hyperref[auth:a118]{P. Laborie} & An Update on the Comparison of MIP, {CP} and Hybrid Approaches for Mixed Resource Allocation and Scheduling & \href{works/Laborie18a.pdf}{Yes} & \cite{Laborie18a} & 2018 & CPAIOR 2018 & 9 & 18 & 10 & \ref{b:Laborie18a} & \ref{c:Laborie18a}\\
\rowlabel{a:MusliuSS18}MusliuSS18 \href{https://doi.org/10.1007/978-3-319-93031-2\_31}{MusliuSS18} & \hyperref[auth:a45]{N. Musliu}, \hyperref[auth:a125]{A. Schutt}, \hyperref[auth:a126]{Peter J. Stuckey} & Solver Independent Rotating Workforce Scheduling & \href{works/MusliuSS18.pdf}{Yes} & \cite{MusliuSS18} & 2018 & CPAIOR 2018 & 17 & 7 & 23 & \ref{b:MusliuSS18} & \ref{c:MusliuSS18}\\
\rowlabel{a:NishikawaSTT18}NishikawaSTT18 \href{https://doi.org/10.1109/CANDAR.2018.00025}{NishikawaSTT18} & \hyperref[auth:a538]{H. Nishikawa}, \hyperref[auth:a539]{K. Shimada}, \hyperref[auth:a540]{I. Taniguchi}, \hyperref[auth:a541]{H. Tomiyama} & Scheduling of Malleable Fork-Join Tasks with Constraint Programming & \href{works/NishikawaSTT18.pdf}{Yes} & \cite{NishikawaSTT18} & 2018 & CANDAR 2018 & 6 & 2 & 14 & \ref{b:NishikawaSTT18} & \ref{c:NishikawaSTT18}\\
\rowlabel{a:NishikawaSTT18a}NishikawaSTT18a \href{https://doi.org/10.1109/TENCON.2018.8650168}{NishikawaSTT18a} & \hyperref[auth:a538]{H. Nishikawa}, \hyperref[auth:a539]{K. Shimada}, \hyperref[auth:a540]{I. Taniguchi}, \hyperref[auth:a541]{H. Tomiyama} & Scheduling of Malleable Tasks Based on Constraint Programming & \href{works/NishikawaSTT18a.pdf}{Yes} & \cite{NishikawaSTT18a} & 2018 & TENCON 2018 & 6 & 1 & 9 & \ref{b:NishikawaSTT18a} & \ref{c:NishikawaSTT18a}\\
\rowlabel{a:OuelletQ18}OuelletQ18 \href{https://doi.org/10.1007/978-3-319-93031-2\_34}{OuelletQ18} & \hyperref[auth:a52]{Y. Ouellet}, \hyperref[auth:a37]{C. Quimper} & A O(n {\textbackslash}log {\^{}}2 n) Checker and O(n{\^{}}2 {\textbackslash}log n) Filtering Algorithm for the Energetic Reasoning & \href{works/OuelletQ18.pdf}{Yes} & \cite{OuelletQ18} & 2018 & CPAIOR 2018 & 18 & 6 & 16 & \ref{b:OuelletQ18} & \ref{c:OuelletQ18}\\
\rowlabel{a:RiahiNS018}RiahiNS018 \href{https://aaai.org/ocs/index.php/ICAPS/ICAPS18/paper/view/17755}{RiahiNS018} & \hyperref[auth:a394]{V. Riahi}, \hyperref[auth:a395]{M. A. Hakim Newton}, \hyperref[auth:a396]{K. Su}, \hyperref[auth:a397]{A. Sattar} & Local Search for Flowshops with Setup Times and Blocking Constraints & \href{works/RiahiNS018.pdf}{Yes} & \cite{RiahiNS018} & 2018 & ICAPS 2018 & 9 & 0 & 0 & \ref{b:RiahiNS018} & \ref{c:RiahiNS018}\\
\rowlabel{a:TanT18}TanT18 \href{http://dx.doi.org/10.1007/978-3-319-89656-4_5}{TanT18} & \hyperref[auth:a928]{Y. Tan}, \hyperref[auth:a830]{D. Terekhov} & Logic-Based Benders Decomposition for Two-Stage Flexible Flow Shop Scheduling with Unrelated Parallel Machines & \href{works/TanT18.pdf}{Yes} & \cite{TanT18} & 2018 & Canadian AI 2018 & 12 & 1 & 23 & \ref{b:TanT18} & \ref{c:TanT18}\\
\rowlabel{a:Tesch18}Tesch18 \href{https://doi.org/10.1007/978-3-319-98334-9\_41}{Tesch18} & \hyperref[auth:a185]{A. Tesch} & Improving Energetic Propagations for Cumulative Scheduling & \href{works/Tesch18.pdf}{Yes} & \cite{Tesch18} & 2018 & CP 2018 & 17 & 5 & 21 & \ref{b:Tesch18} & \ref{c:Tesch18}\\
\rowlabel{a:BofillCSV17}BofillCSV17 \href{https://doi.org/10.1007/978-3-319-66158-2\_5}{BofillCSV17} & \hyperref[auth:a190]{M. Bofill}, \hyperref[auth:a191]{J. Coll}, \hyperref[auth:a192]{J. Suy}, \hyperref[auth:a193]{M. Villaret} & An Efficient {SMT} Approach to Solve MRCPSP/max Instances with Tight Constraints on Resources & \href{works/BofillCSV17.pdf}{Yes} & \cite{BofillCSV17} & 2017 & CP 2017 & 9 & 1 & 12 & \ref{b:BofillCSV17} & \ref{c:BofillCSV17}\\
\rowlabel{a:CappartS17}CappartS17 \href{https://doi.org/10.1007/978-3-319-59776-8\_26}{CappartS17} & \hyperref[auth:a42]{Q. Cappart}, \hyperref[auth:a148]{P. Schaus} & Rescheduling Railway Traffic on Real Time Situations Using Time-Interval Variables & \href{works/CappartS17.pdf}{Yes} & \cite{CappartS17} & 2017 & CPAIOR 2017 & 16 & 2 & 28 & \ref{b:CappartS17} & \ref{c:CappartS17}\\
\rowlabel{a:CohenHB17}CohenHB17 \href{https://doi.org/10.1007/978-3-319-66263-3\_10}{CohenHB17} & \hyperref[auth:a817]{E. Cohen}, \hyperref[auth:a818]{G. Huang}, \hyperref[auth:a89]{J. Christopher Beck} & {(I} Can Get) Satisfaction: Preference-Based Scheduling for Concert-Goers at Multi-venue Music Festivals & \href{works/CohenHB17.pdf}{Yes} & \cite{CohenHB17} & 2017 & SAT 2017 & 17 & 1 & 12 & \ref{b:CohenHB17} & \ref{c:CohenHB17}\\
\rowlabel{a:GelainPRVW17}GelainPRVW17 \href{https://doi.org/10.1007/978-3-319-59776-8\_32}{GelainPRVW17} & \hyperref[auth:a317]{M. Gelain}, \hyperref[auth:a318]{Maria Silvia Pini}, \hyperref[auth:a319]{F. Rossi}, \hyperref[auth:a320]{Kristen Brent Venable}, \hyperref[auth:a279]{T. Walsh} & A Local Search Approach for Incomplete Soft Constraint Problems: Experimental Results on Meeting Scheduling Problems & \href{works/GelainPRVW17.pdf}{Yes} & \cite{GelainPRVW17} & 2017 & CPAIOR 2017 & 16 & 1 & 5 & \ref{b:GelainPRVW17} & \ref{c:GelainPRVW17}\\
\rowlabel{a:GoldwaserS17}GoldwaserS17 \href{https://doi.org/10.1007/978-3-319-66158-2\_22}{GoldwaserS17} & \hyperref[auth:a195]{A. Goldwaser}, \hyperref[auth:a125]{A. Schutt} & Optimal Torpedo Scheduling & \href{works/GoldwaserS17.pdf}{Yes} & \cite{GoldwaserS17} & 2017 & CP 2017 & 16 & 0 & 10 & \ref{b:GoldwaserS17} & \ref{c:GoldwaserS17}\\
\rowlabel{a:Hooker17}Hooker17 \href{https://doi.org/10.1007/978-3-319-66158-2\_36}{Hooker17} & \hyperref[auth:a162]{John N. Hooker} & Job Sequencing Bounds from Decision Diagrams & \href{works/Hooker17.pdf}{Yes} & \cite{Hooker17} & 2017 & CP 2017 & 14 & 6 & 24 & \ref{b:Hooker17} & \ref{c:Hooker17}\\
\rowlabel{a:KletzanderM17}KletzanderM17 \href{https://doi.org/10.1007/978-3-319-59776-8\_28}{KletzanderM17} & \hyperref[auth:a78]{L. Kletzander}, \hyperref[auth:a45]{N. Musliu} & A Multi-stage Simulated Annealing Algorithm for the Torpedo Scheduling Problem & \href{works/KletzanderM17.pdf}{Yes} & \cite{KletzanderM17} & 2017 & CPAIOR 2017 & 15 & 1 & 9 & \ref{b:KletzanderM17} & \ref{c:KletzanderM17}\\
\rowlabel{a:LiuCGM17}LiuCGM17 \href{https://doi.org/10.1007/978-3-319-66158-2\_24}{LiuCGM17} & \hyperref[auth:a196]{T. Liu}, \hyperref[auth:a197]{Roberto Di Cosmo}, \hyperref[auth:a198]{M. Gabbrielli}, \hyperref[auth:a199]{J. Mauro} & NightSplitter: {A} Scheduling Tool to Optimize (Sub)group Activities & \href{works/LiuCGM17.pdf}{Yes} & \cite{LiuCGM17} & 2017 & CP 2017 & 17 & 0 & 15 & \ref{b:LiuCGM17} & \ref{c:LiuCGM17}\\
\rowlabel{a:Madi-WambaLOBM17}Madi-WambaLOBM17 \href{https://doi.org/10.1109/ICPADS.2017.00089}{Madi-WambaLOBM17} & \hyperref[auth:a324]{G. Madi{-}Wamba}, \hyperref[auth:a723]{Y. Li}, \hyperref[auth:a724]{A. Orgerie}, \hyperref[auth:a129]{N. Beldiceanu}, \hyperref[auth:a725]{J. Menaud} & Green Energy Aware Scheduling Problem in Virtualized Datacenters & \href{works/Madi-WambaLOBM17.pdf}{Yes} & \cite{Madi-WambaLOBM17} & 2017 & ICPADS 2017 & 8 & 1 & 8 & \ref{b:Madi-WambaLOBM17} & \ref{c:Madi-WambaLOBM17}\\
\rowlabel{a:MossigeGSMC17}MossigeGSMC17 \href{https://doi.org/10.1007/978-3-319-66158-2\_25}{MossigeGSMC17} & \hyperref[auth:a200]{M. Mossige}, \hyperref[auth:a201]{A. Gotlieb}, \hyperref[auth:a202]{H. Spieker}, \hyperref[auth:a203]{H. Meling}, \hyperref[auth:a91]{M. Carlsson} & Time-Aware Test Case Execution Scheduling for Cyber-Physical Systems & \href{works/MossigeGSMC17.pdf}{Yes} & \cite{MossigeGSMC17} & 2017 & CP 2017 & 18 & 6 & 33 & \ref{b:MossigeGSMC17} & \ref{c:MossigeGSMC17}\\
\rowlabel{a:Pralet17}Pralet17 \href{https://doi.org/10.1007/978-3-319-66158-2\_16}{Pralet17} & \hyperref[auth:a21]{C. Pralet} & An Incomplete Constraint-Based System for Scheduling with Renewable Resources & \href{works/Pralet17.pdf}{Yes} & \cite{Pralet17} & 2017 & CP 2017 & 19 & 1 & 30 & \ref{b:Pralet17} & \ref{c:Pralet17}\\
\rowlabel{a:TranVNB17a}TranVNB17a \href{https://doi.org/10.24963/ijcai.2017/726}{TranVNB17a} & \hyperref[auth:a811]{Tony T. Tran}, \hyperref[auth:a816]{Tiago Stegun Vaquero}, \hyperref[auth:a210]{G. Nejat}, \hyperref[auth:a89]{J. Christopher Beck} & Robots in Retirement Homes: Applying Off-the-Shelf Planning and Scheduling to a Team of Assistive Robots (Extended Abstract) & \href{works/TranVNB17a.pdf}{Yes} & \cite{TranVNB17a} & 2017 & IJCAI 2017 & 5 & 1 & 0 & \ref{b:TranVNB17a} & \ref{c:TranVNB17a}\\
\rowlabel{a:YoungFS17}YoungFS17 \href{https://doi.org/10.1007/978-3-319-66158-2\_20}{YoungFS17} & \hyperref[auth:a194]{Kenneth D. Young}, \hyperref[auth:a155]{T. Feydy}, \hyperref[auth:a125]{A. Schutt} & Constraint Programming Applied to the Multi-Skill Project Scheduling Problem & \href{works/YoungFS17.pdf}{Yes} & \cite{YoungFS17} & 2017 & CP 2017 & 10 & 6 & 21 & \ref{b:YoungFS17} & \ref{c:YoungFS17}\\
\rowlabel{a:AmadiniGM16}AmadiniGM16 \href{http://dx.doi.org/10.1007/978-3-319-50349-3_16}{AmadiniGM16} & \hyperref[auth:a929]{R. Amadini}, \hyperref[auth:a198]{M. Gabbrielli}, \hyperref[auth:a199]{J. Mauro} & Parallelizing Constraint Solvers for Hard RCPSP Instances & \href{works/AmadiniGM16.pdf}{Yes} & \cite{AmadiniGM16} & 2016 & LION 2016 & 7 & 2 & 16 & \ref{b:AmadiniGM16} & \ref{c:AmadiniGM16}\\
\rowlabel{a:BonfiettiZLM16}BonfiettiZLM16 \href{https://doi.org/10.1007/978-3-319-44953-1\_8}{BonfiettiZLM16} & \hyperref[auth:a204]{A. Bonfietti}, \hyperref[auth:a205]{A. Zanarini}, \hyperref[auth:a143]{M. Lombardi}, \hyperref[auth:a144]{M. Milano} & The Multirate Resource Constraint & \href{works/BonfiettiZLM16.pdf}{Yes} & \cite{BonfiettiZLM16} & 2016 & CP 2016 & 17 & 0 & 11 & \ref{b:BonfiettiZLM16} & \ref{c:BonfiettiZLM16}\\
\rowlabel{a:BoothNB16}BoothNB16 \href{https://doi.org/10.1007/978-3-319-44953-1\_34}{BoothNB16} & \hyperref[auth:a209]{Kyle E. C. Booth}, \hyperref[auth:a210]{G. Nejat}, \hyperref[auth:a89]{J. Christopher Beck} & A Constraint Programming Approach to Multi-Robot Task Allocation and Scheduling in Retirement Homes & \href{works/BoothNB16.pdf}{Yes} & \cite{BoothNB16} & 2016 & CP 2016 & 17 & 21 & 24 & \ref{b:BoothNB16} & \ref{c:BoothNB16}\\
\rowlabel{a:BridiLBBM16}BridiLBBM16 \href{https://doi.org/10.3233/978-1-61499-672-9-1598}{BridiLBBM16} & \hyperref[auth:a233]{T. Bridi}, \hyperref[auth:a143]{M. Lombardi}, \hyperref[auth:a231]{A. Bartolini}, \hyperref[auth:a248]{L. Benini}, \hyperref[auth:a144]{M. Milano} & {DARDIS:} Distributed And Randomized DIspatching and Scheduling & \href{works/BridiLBBM16.pdf}{Yes} & \cite{BridiLBBM16} & 2016 & ECAI 2016 & 2 & 0 & 0 & \ref{b:BridiLBBM16} & \ref{c:BridiLBBM16}\\
\rowlabel{a:CauwelaertDMS16}CauwelaertDMS16 \href{https://doi.org/10.1007/978-3-319-44953-1\_33}{CauwelaertDMS16} & \hyperref[auth:a207]{Sascha Van Cauwelaert}, \hyperref[auth:a208]{C. Dejemeppe}, \hyperref[auth:a150]{J. Monette}, \hyperref[auth:a148]{P. Schaus} & Efficient Filtering for the Unary Resource with Family-Based Transition Times & \href{works/CauwelaertDMS16.pdf}{Yes} & \cite{CauwelaertDMS16} & 2016 & CP 2016 & 16 & 1 & 12 & \ref{b:CauwelaertDMS16} & \ref{c:CauwelaertDMS16}\\
\rowlabel{a:FontaineMH16}FontaineMH16 \href{https://doi.org/10.1007/978-3-319-33954-2\_12}{FontaineMH16} & \hyperref[auth:a321]{D. Fontaine}, \hyperref[auth:a322]{Laurent D. Michel}, \hyperref[auth:a149]{Pascal Van Hentenryck} & Parallel Composition of Scheduling Solvers & \href{works/FontaineMH16.pdf}{Yes} & \cite{FontaineMH16} & 2016 & CPAIOR 2016 & 11 & 3 & 0 & \ref{b:FontaineMH16} & \ref{c:FontaineMH16}\\
\rowlabel{a:GilesH16}GilesH16 \href{https://doi.org/10.1007/978-3-319-44953-1\_38}{GilesH16} & \hyperref[auth:a211]{K. Giles}, \hyperref[auth:a212]{Willem{-}Jan van Hoeve} & Solving a Supply-Delivery Scheduling Problem with Constraint Programming & \href{works/GilesH16.pdf}{Yes} & \cite{GilesH16} & 2016 & CP 2016 & 16 & 2 & 6 & \ref{b:GilesH16} & \ref{c:GilesH16}\\
\rowlabel{a:GingrasQ16}GingrasQ16 \href{http://www.ijcai.org/Abstract/16/440}{GingrasQ16} & \hyperref[auth:a316]{V. Gingras}, \hyperref[auth:a37]{C. Quimper} & Generalizing the Edge-Finder Rule for the Cumulative Constraint & \href{works/GingrasQ16.pdf}{Yes} & \cite{GingrasQ16} & 2016 & IJCAI 2016 & 7 & 0 & 0 & \ref{b:GingrasQ16} & \ref{c:GingrasQ16}\\
\rowlabel{a:HechingH16}HechingH16 \href{https://doi.org/10.1007/978-3-319-33954-2\_14}{HechingH16} & \hyperref[auth:a323]{Aliza R. Heching}, \hyperref[auth:a162]{John N. Hooker} & Scheduling Home Hospice Care with Logic-Based Benders Decomposition & \href{works/HechingH16.pdf}{Yes} & \cite{HechingH16} & 2016 & CPAIOR 2016 & 11 & 10 & 0 & \ref{b:HechingH16} & \ref{c:HechingH16}\\
\rowlabel{a:JelinekB16}JelinekB16 \href{https://doi.org/10.1007/978-3-319-28228-2\_1}{JelinekB16} & \hyperref[auth:a789]{J. Jel{\'{\i}}nek}, \hyperref[auth:a153]{R. Bart{\'{a}}k} & Using Constraint Logic Programming to Schedule Solar Array Operations on the International Space Station & \href{works/JelinekB16.pdf}{Yes} & \cite{JelinekB16} & 2016 & PADL 2016 & 10 & 0 & 5 & \ref{b:JelinekB16} & \ref{c:JelinekB16}\\
\rowlabel{a:LimHTB16}LimHTB16 \href{https://doi.org/10.1007/978-3-319-44953-1\_43}{LimHTB16} & \hyperref[auth:a213]{B. Lim}, \hyperref[auth:a214]{Hassan L. Hijazi}, \hyperref[auth:a215]{S. Thi{\'{e}}baux}, \hyperref[auth:a216]{Menkes van den Briel} & Online HVAC-Aware Occupancy Scheduling with Adaptive Temperature Control & \href{works/LimHTB16.pdf}{Yes} & \cite{LimHTB16} & 2016 & CP 2016 & 18 & 2 & 23 & \ref{b:LimHTB16} & \ref{c:LimHTB16}\\
\rowlabel{a:LuoVLBM16}LuoVLBM16 \href{http://www.aaai.org/ocs/index.php/KR/KR16/paper/view/12909}{LuoVLBM16} & \hyperref[auth:a825]{R. Luo}, \hyperref[auth:a826]{Richard Anthony Valenzano}, \hyperref[auth:a827]{Y. Li}, \hyperref[auth:a89]{J. Christopher Beck}, \hyperref[auth:a828]{Sheila A. McIlraith} & Using Metric Temporal Logic to Specify Scheduling Problems & \href{works/LuoVLBM16.pdf}{Yes} & \cite{LuoVLBM16} & 2016 & KR 2016 & 4 & 0 & 0 & \ref{b:LuoVLBM16} & \ref{c:LuoVLBM16}\\
\rowlabel{a:Madi-WambaB16}Madi-WambaB16 \href{https://doi.org/10.1007/978-3-319-33954-2\_18}{Madi-WambaB16} & \hyperref[auth:a324]{G. Madi{-}Wamba}, \hyperref[auth:a129]{N. Beldiceanu} & The TaskIntersection Constraint & \href{works/Madi-WambaB16.pdf}{Yes} & \cite{Madi-WambaB16} & 2016 & CPAIOR 2016 & 16 & 0 & 0 & \ref{b:Madi-WambaB16} & \ref{c:Madi-WambaB16}\\
\rowlabel{a:SchuttS16}SchuttS16 \href{https://doi.org/10.1007/978-3-319-44953-1\_28}{SchuttS16} & \hyperref[auth:a125]{A. Schutt}, \hyperref[auth:a126]{Peter J. Stuckey} & Explaining Producer/Consumer Constraints & \href{works/SchuttS16.pdf}{Yes} & \cite{SchuttS16} & 2016 & CP 2016 & 17 & 3 & 23 & \ref{b:SchuttS16} & \ref{c:SchuttS16}\\
\rowlabel{a:SzerediS16}SzerediS16 \href{https://doi.org/10.1007/978-3-319-44953-1\_31}{SzerediS16} & \hyperref[auth:a206]{R. Szeredi}, \hyperref[auth:a125]{A. Schutt} & Modelling and Solving Multi-mode Resource-Constrained Project Scheduling & \href{works/SzerediS16.pdf}{Yes} & \cite{SzerediS16} & 2016 & CP 2016 & 10 & 9 & 14 & \ref{b:SzerediS16} & \ref{c:SzerediS16}\\
\rowlabel{a:Tesch16}Tesch16 \href{https://doi.org/10.1007/978-3-319-44953-1\_32}{Tesch16} & \hyperref[auth:a185]{A. Tesch} & A Nearly Exact Propagation Algorithm for Energetic Reasoning in {\textbackslash}mathcal O(n{\^{}}2 {\textbackslash}log n) & \href{works/Tesch16.pdf}{Yes} & \cite{Tesch16} & 2016 & CP 2016 & 27 & 4 & 14 & \ref{b:Tesch16} & \ref{c:Tesch16}\\
\rowlabel{a:TranDRFWOVB16}TranDRFWOVB16 \href{https://doi.org/10.1609/socs.v7i1.18390}{TranDRFWOVB16} & \hyperref[auth:a811]{Tony T. Tran}, \hyperref[auth:a821]{M. Do}, \hyperref[auth:a822]{Eleanor Gilbert Rieffel}, \hyperref[auth:a385]{J. Frank}, \hyperref[auth:a820]{Z. Wang}, \hyperref[auth:a823]{B. O'Gorman}, \hyperref[auth:a824]{D. Venturelli}, \hyperref[auth:a89]{J. Christopher Beck} & A Hybrid Quantum-Classical Approach to Solving Scheduling Problems & \href{works/TranDRFWOVB16.pdf}{Yes} & \cite{TranDRFWOVB16} & 2016 & SOCS 2016 & 9 & 3 & 0 & \ref{b:TranDRFWOVB16} & \ref{c:TranDRFWOVB16}\\
\rowlabel{a:TranWDRFOVB16}TranWDRFOVB16 \href{http://www.aaai.org/ocs/index.php/WS/AAAIW16/paper/view/12664}{TranWDRFOVB16} & \hyperref[auth:a811]{Tony T. Tran}, \hyperref[auth:a820]{Z. Wang}, \hyperref[auth:a821]{M. Do}, \hyperref[auth:a822]{Eleanor Gilbert Rieffel}, \hyperref[auth:a385]{J. Frank}, \hyperref[auth:a823]{B. O'Gorman}, \hyperref[auth:a824]{D. Venturelli}, \hyperref[auth:a89]{J. Christopher Beck} & Explorations of Quantum-Classical Approaches to Scheduling a Mars Lander Activity Problem & \href{works/TranWDRFOVB16.pdf}{Yes} & \cite{TranWDRFOVB16} & 2016 & AAAI 2016 & 9 & 0 & 0 & \ref{b:TranWDRFOVB16} & \ref{c:TranWDRFOVB16}\\
\rowlabel{a:BartakV15}BartakV15 \href{}{BartakV15} & \hyperref[auth:a153]{R. Bart{\'{a}}k}, \hyperref[auth:a314]{M. Vlk} & Reactive Recovery from Machine Breakdown in Production Scheduling with Temporal Distance and Resource Constraints & \href{works/BartakV15.pdf}{Yes} & \cite{BartakV15} & 2015 & ICAART 2015 & 12 & 0 & 0 & \ref{b:BartakV15} & \ref{c:BartakV15}\\
\rowlabel{a:BofillGSV15}BofillGSV15 \href{https://doi.org/10.1007/978-3-319-18008-3\_5}{BofillGSV15} & \hyperref[auth:a190]{M. Bofill}, \hyperref[auth:a235]{M. Garcia}, \hyperref[auth:a192]{J. Suy}, \hyperref[auth:a193]{M. Villaret} & MaxSAT-Based Scheduling of {B2B} Meetings & \href{works/BofillGSV15.pdf}{Yes} & \cite{BofillGSV15} & 2015 & CPAIOR 2015 & 9 & 7 & 8 & \ref{b:BofillGSV15} & \ref{c:BofillGSV15}\\
\rowlabel{a:BurtLPS15}BurtLPS15 \href{https://doi.org/10.1007/978-3-319-18008-3\_7}{BurtLPS15} & \hyperref[auth:a326]{Christina N. Burt}, \hyperref[auth:a327]{N. Lipovetzky}, \hyperref[auth:a328]{Adrian R. Pearce}, \hyperref[auth:a126]{Peter J. Stuckey} & Scheduling with Fixed Maintenance, Shared Resources and Nonlinear Feedrate Constraints: {A} Mine Planning Case Study & \href{works/BurtLPS15.pdf}{Yes} & \cite{BurtLPS15} & 2015 & CPAIOR 2015 & 17 & 0 & 8 & \ref{b:BurtLPS15} & \ref{c:BurtLPS15}\\
\rowlabel{a:DejemeppeCS15}DejemeppeCS15 \href{https://doi.org/10.1007/978-3-319-23219-5\_7}{DejemeppeCS15} & \hyperref[auth:a208]{C. Dejemeppe}, \hyperref[auth:a207]{Sascha Van Cauwelaert}, \hyperref[auth:a148]{P. Schaus} & The Unary Resource with Transition Times & \href{works/DejemeppeCS15.pdf}{Yes} & \cite{DejemeppeCS15} & 2015 & CP 2015 & 16 & 5 & 11 & \ref{b:DejemeppeCS15} & \ref{c:DejemeppeCS15}\\
\rowlabel{a:EvenSH15}EvenSH15 \href{https://doi.org/10.1007/978-3-319-23219-5\_40}{EvenSH15} & \hyperref[auth:a220]{C. Even}, \hyperref[auth:a125]{A. Schutt}, \hyperref[auth:a149]{Pascal Van Hentenryck} & A Constraint Programming Approach for Non-preemptive Evacuation Scheduling & \href{works/EvenSH15.pdf}{Yes} & \cite{EvenSH15} & 2015 & CP 2015 & 18 & 3 & 12 & \ref{b:EvenSH15} & \ref{c:EvenSH15}\\
\rowlabel{a:GayHLS15}GayHLS15 \href{https://doi.org/10.1007/978-3-319-23219-5\_10}{GayHLS15} & \hyperref[auth:a217]{S. Gay}, \hyperref[auth:a218]{R. Hartert}, \hyperref[auth:a219]{C. Lecoutre}, \hyperref[auth:a148]{P. Schaus} & Conflict Ordering Search for Scheduling Problems & \href{works/GayHLS15.pdf}{Yes} & \cite{GayHLS15} & 2015 & CP 2015 & 9 & 20 & 15 & \ref{b:GayHLS15} & \ref{c:GayHLS15}\\
\rowlabel{a:GayHS15}GayHS15 \href{https://doi.org/10.1007/978-3-319-23219-5\_11}{GayHS15} & \hyperref[auth:a217]{S. Gay}, \hyperref[auth:a218]{R. Hartert}, \hyperref[auth:a148]{P. Schaus} & Simple and Scalable Time-Table Filtering for the Cumulative Constraint & \href{works/GayHS15.pdf}{Yes} & \cite{GayHS15} & 2015 & CP 2015 & 9 & 10 & 9 & \ref{b:GayHS15} & \ref{c:GayHS15}\\
\rowlabel{a:GayHS15a}GayHS15a \href{https://doi.org/10.1007/978-3-319-18008-3\_11}{GayHS15a} & \hyperref[auth:a217]{S. Gay}, \hyperref[auth:a218]{R. Hartert}, \hyperref[auth:a148]{P. Schaus} & Time-Table Disjunctive Reasoning for the Cumulative Constraint & \href{works/GayHS15a.pdf}{Yes} & \cite{GayHS15a} & 2015 & CPAIOR 2015 & 16 & 5 & 12 & \ref{b:GayHS15a} & \ref{c:GayHS15a}\\
\rowlabel{a:KreterSS15}KreterSS15 \href{https://doi.org/10.1007/978-3-319-23219-5\_19}{KreterSS15} & \hyperref[auth:a124]{S. Kreter}, \hyperref[auth:a125]{A. Schutt}, \hyperref[auth:a126]{Peter J. Stuckey} & Modeling and Solving Project Scheduling with Calendars & \href{works/KreterSS15.pdf}{Yes} & \cite{KreterSS15} & 2015 & CP 2015 & 17 & 7 & 16 & \ref{b:KreterSS15} & \ref{c:KreterSS15}\\
\rowlabel{a:LimBTBB15}LimBTBB15 \href{https://doi.org/10.1007/978-3-319-18008-3\_17}{LimBTBB15} & \hyperref[auth:a213]{B. Lim}, \hyperref[auth:a216]{Menkes van den Briel}, \hyperref[auth:a215]{S. Thi{\'{e}}baux}, \hyperref[auth:a329]{R. Bent}, \hyperref[auth:a330]{S. Backhaus} & Large Neighborhood Search for Energy Aware Meeting Scheduling in Smart Buildings & \href{works/LimBTBB15.pdf}{Yes} & \cite{LimBTBB15} & 2015 & CPAIOR 2015 & 15 & 4 & 18 & \ref{b:LimBTBB15} & \ref{c:LimBTBB15}\\
\rowlabel{a:LombardiBM15}LombardiBM15 \href{https://doi.org/10.1007/978-3-319-23219-5\_20}{LombardiBM15} & \hyperref[auth:a143]{M. Lombardi}, \hyperref[auth:a204]{A. Bonfietti}, \hyperref[auth:a144]{M. Milano} & Deterministic Estimation of the Expected Makespan of a {POS} Under Duration Uncertainty & \href{works/LombardiBM15.pdf}{Yes} & \cite{LombardiBM15} & 2015 & CP 2015 & 16 & 0 & 8 & \ref{b:LombardiBM15} & \ref{c:LombardiBM15}\\
\rowlabel{a:MelgarejoLS15}MelgarejoLS15 \href{https://doi.org/10.1007/978-3-319-18008-3\_1}{MelgarejoLS15} & \hyperref[auth:a325]{P. Aguiar{-}Melgarejo}, \hyperref[auth:a118]{P. Laborie}, \hyperref[auth:a85]{C. Solnon} & A Time-Dependent No-Overlap Constraint: Application to Urban Delivery Problems & \href{works/MelgarejoLS15.pdf}{Yes} & \cite{MelgarejoLS15} & 2015 & CPAIOR 2015 & 17 & 14 & 17 & \ref{b:MelgarejoLS15} & \ref{c:MelgarejoLS15}\\
\rowlabel{a:MurphyMB15}MurphyMB15 \href{https://doi.org/10.1007/978-3-319-23219-5\_47}{MurphyMB15} & \hyperref[auth:a221]{Se{\'{a}}n {\'{O}}g Murphy}, \hyperref[auth:a222]{O. Manzano}, \hyperref[auth:a223]{Kenneth N. Brown} & Design and Evaluation of a Constraint-Based Energy Saving and Scheduling Recommender System & \href{works/MurphyMB15.pdf}{Yes} & \cite{MurphyMB15} & 2015 & CP 2015 & 17 & 1 & 20 & \ref{b:MurphyMB15} & \ref{c:MurphyMB15}\\
\rowlabel{a:PesantRR15}PesantRR15 \href{https://doi.org/10.1007/978-3-319-18008-3\_21}{PesantRR15} & \hyperref[auth:a8]{G. Pesant}, \hyperref[auth:a331]{G. Rix}, \hyperref[auth:a332]{L. Rousseau} & A Comparative Study of {MIP} and {CP} Formulations for the {B2B} Scheduling Optimization Problem & \href{works/PesantRR15.pdf}{Yes} & \cite{PesantRR15} & 2015 & CPAIOR 2015 & 16 & 1 & 7 & \ref{b:PesantRR15} & \ref{c:PesantRR15}\\
\rowlabel{a:PraletLJ15}PraletLJ15 \href{https://doi.org/10.1007/978-3-319-23219-5\_48}{PraletLJ15} & \hyperref[auth:a21]{C. Pralet}, \hyperref[auth:a224]{S. Lemai{-}Chenevier}, \hyperref[auth:a225]{J. Jaubert} & Scheduling Running Modes of Satellite Instruments Using Constraint-Based Local Search & \href{works/PraletLJ15.pdf}{Yes} & \cite{PraletLJ15} & 2015 & CP 2015 & 16 & 0 & 8 & \ref{b:PraletLJ15} & \ref{c:PraletLJ15}\\
\rowlabel{a:SialaAH15}SialaAH15 \href{https://doi.org/10.1007/978-3-319-23219-5\_28}{SialaAH15} & \hyperref[auth:a130]{M. Siala}, \hyperref[auth:a6]{C. Artigues}, \hyperref[auth:a1]{E. Hebrard} & Two Clause Learning Approaches for Disjunctive Scheduling & \href{works/SialaAH15.pdf}{Yes} & \cite{SialaAH15} & 2015 & CP 2015 & 10 & 4 & 17 & \ref{b:SialaAH15} & \ref{c:SialaAH15}\\
\rowlabel{a:VilimLS15}VilimLS15 \href{https://doi.org/10.1007/978-3-319-18008-3\_30}{VilimLS15} & \hyperref[auth:a121]{P. Vil{\'{\i}}m}, \hyperref[auth:a118]{P. Laborie}, \hyperref[auth:a120]{P. Shaw} & Failure-Directed Search for Constraint-Based Scheduling & \href{works/VilimLS15.pdf}{Yes} & \cite{VilimLS15} & 2015 & CPAIOR 2015 & 17 & 31 & 19 & \ref{b:VilimLS15} & \ref{c:VilimLS15}\\
\rowlabel{a:ZhouGL15}ZhouGL15 \href{https://doi.org/10.1109/FSKD.2015.7382064}{ZhouGL15} & \hyperref[auth:a609]{J. Zhou}, \hyperref[auth:a610]{Y. Guo}, \hyperref[auth:a611]{G. Li} & On complex hybrid flexible flowshop scheduling problems based on constraint programming & \href{works/ZhouGL15.pdf}{Yes} & \cite{ZhouGL15} & 2015 & FSKD 2015 & 5 & 0 & 16 & \ref{b:ZhouGL15} & \ref{c:ZhouGL15}\\
\rowlabel{a:AlesioNBG14}AlesioNBG14 \href{https://doi.org/10.1007/978-3-319-10428-7\_58}{AlesioNBG14} & \hyperref[auth:a237]{Stefano {Di Alesio}}, \hyperref[auth:a238]{S. Nejati}, \hyperref[auth:a239]{Lionel C. Briand}, \hyperref[auth:a201]{A. Gotlieb} & Worst-Case Scheduling of Software Tasks - {A} Constraint Optimization Model to Support Performance Testing & \href{works/AlesioNBG14.pdf}{Yes} & \cite{AlesioNBG14} & 2014 & CP 2014 & 18 & 3 & 19 & \ref{b:AlesioNBG14} & \ref{c:AlesioNBG14}\\
\rowlabel{a:BartoliniBBLM14}BartoliniBBLM14 \href{https://doi.org/10.1007/978-3-319-10428-7\_55}{BartoliniBBLM14} & \hyperref[auth:a231]{A. Bartolini}, \hyperref[auth:a232]{A. Borghesi}, \hyperref[auth:a233]{T. Bridi}, \hyperref[auth:a143]{M. Lombardi}, \hyperref[auth:a144]{M. Milano} & Proactive Workload Dispatching on the {EURORA} Supercomputer & \href{works/BartoliniBBLM14.pdf}{Yes} & \cite{BartoliniBBLM14} & 2014 & CP 2014 & 16 & 12 & 3 & \ref{b:BartoliniBBLM14} & \ref{c:BartoliniBBLM14}\\
\rowlabel{a:BessiereHMQW14}BessiereHMQW14 \href{https://doi.org/10.1007/978-3-319-07046-9\_23}{BessiereHMQW14} & \hyperref[auth:a334]{C. Bessiere}, \hyperref[auth:a1]{E. Hebrard}, \hyperref[auth:a335]{M. M{\'{e}}nard}, \hyperref[auth:a37]{C. Quimper}, \hyperref[auth:a279]{T. Walsh} & Buffered Resource Constraint: Algorithms and Complexity & \href{works/BessiereHMQW14.pdf}{Yes} & \cite{BessiereHMQW14} & 2014 & CPAIOR 2014 & 16 & 1 & 3 & \ref{b:BessiereHMQW14} & \ref{c:BessiereHMQW14}\\
\rowlabel{a:BofillEGPSV14}BofillEGPSV14 \href{https://doi.org/10.1007/978-3-319-10428-7\_56}{BofillEGPSV14} & \hyperref[auth:a190]{M. Bofill}, \hyperref[auth:a234]{J. Espasa}, \hyperref[auth:a235]{M. Garcia}, \hyperref[auth:a236]{M. Palah{\'{\i}}}, \hyperref[auth:a192]{J. Suy}, \hyperref[auth:a193]{M. Villaret} & Scheduling {B2B} Meetings & \href{works/BofillEGPSV14.pdf}{Yes} & \cite{BofillEGPSV14} & 2014 & CP 2014 & 16 & 3 & 10 & \ref{b:BofillEGPSV14} & \ref{c:BofillEGPSV14}\\
\rowlabel{a:BonfiettiLM14}BonfiettiLM14 \href{https://doi.org/10.1007/978-3-319-07046-9\_15}{BonfiettiLM14} & \hyperref[auth:a204]{A. Bonfietti}, \hyperref[auth:a143]{M. Lombardi}, \hyperref[auth:a144]{M. Milano} & Disregarding Duration Uncertainty in Partial Order Schedules? Yes, We Can! & \href{works/BonfiettiLM14.pdf}{Yes} & \cite{BonfiettiLM14} & 2014 & CPAIOR 2014 & 16 & 3 & 12 & \ref{b:BonfiettiLM14} & \ref{c:BonfiettiLM14}\\
\rowlabel{a:DejemeppeD14}DejemeppeD14 \href{https://doi.org/10.1007/978-3-319-07046-9\_20}{DejemeppeD14} & \hyperref[auth:a208]{C. Dejemeppe}, \hyperref[auth:a152]{Y. Deville} & Continuously Degrading Resource and Interval Dependent Activity Durations in Nuclear Medicine Patient Scheduling & \href{works/DejemeppeD14.pdf}{Yes} & \cite{DejemeppeD14} & 2014 & CPAIOR 2014 & 9 & 0 & 7 & \ref{b:DejemeppeD14} & \ref{c:DejemeppeD14}\\
\rowlabel{a:DerrienP14}DerrienP14 \href{https://doi.org/10.1007/978-3-319-10428-7\_22}{DerrienP14} & \hyperref[auth:a226]{A. Derrien}, \hyperref[auth:a227]{T. Petit} & A New Characterization of Relevant Intervals for Energetic Reasoning & \href{works/DerrienP14.pdf}{Yes} & \cite{DerrienP14} & 2014 & CP 2014 & 9 & 14 & 0 & \ref{b:DerrienP14} & \ref{c:DerrienP14}\\
\rowlabel{a:DerrienPZ14}DerrienPZ14 \href{https://doi.org/10.1007/978-3-319-10428-7\_23}{DerrienPZ14} & \hyperref[auth:a226]{A. Derrien}, \hyperref[auth:a227]{T. Petit}, \hyperref[auth:a228]{S. Zampelli} & A Declarative Paradigm for Robust Cumulative Scheduling & \href{works/DerrienPZ14.pdf}{Yes} & \cite{DerrienPZ14} & 2014 & CP 2014 & 9 & 3 & 10 & \ref{b:DerrienPZ14} & \ref{c:DerrienPZ14}\\
\rowlabel{a:DoulabiRP14}DoulabiRP14 \href{https://doi.org/10.1007/978-3-319-07046-9\_32}{DoulabiRP14} & \hyperref[auth:a336]{Seyed Hossein Hashemi Doulabi}, \hyperref[auth:a332]{L. Rousseau}, \hyperref[auth:a8]{G. Pesant} & A Constraint Programming-Based Column Generation Approach for Operating Room Planning and Scheduling & \href{works/DoulabiRP14.pdf}{Yes} & \cite{DoulabiRP14} & 2014 & CPAIOR 2014 & 9 & 3 & 10 & \ref{b:DoulabiRP14} & \ref{c:DoulabiRP14}\\
\rowlabel{a:FriedrichFMRSST14}FriedrichFMRSST14 \href{https://doi.org/10.1007/978-3-319-28697-6\_23}{FriedrichFMRSST14} & \hyperref[auth:a612]{G. Friedrich}, \hyperref[auth:a613]{M. Fr{\"{u}}hst{\"{u}}ck}, \hyperref[auth:a614]{V. Mersheeva}, \hyperref[auth:a615]{A. Ryabokon}, \hyperref[auth:a616]{M. Sander}, \hyperref[auth:a617]{A. Starzacher}, \hyperref[auth:a618]{E. Teppan} & Representing Production Scheduling with Constraint Answer Set Programming & No & \cite{FriedrichFMRSST14} & 2014 & GOR 2014 & 7 & 3 & 2 & No & \ref{c:FriedrichFMRSST14}\\
\rowlabel{a:GaySS14}GaySS14 \href{https://doi.org/10.1007/978-3-319-10428-7\_59}{GaySS14} & \hyperref[auth:a217]{S. Gay}, \hyperref[auth:a148]{P. Schaus}, \hyperref[auth:a240]{Vivian De Smedt} & Continuous Casting Scheduling with Constraint Programming & \href{works/GaySS14.pdf}{Yes} & \cite{GaySS14} & 2014 & CP 2014 & 15 & 7 & 11 & \ref{b:GaySS14} & \ref{c:GaySS14}\\
\rowlabel{a:HoundjiSWD14}HoundjiSWD14 \href{https://doi.org/10.1007/978-3-319-10428-7\_29}{HoundjiSWD14} & \hyperref[auth:a229]{Vinas{\'{e}}tan Ratheil Houndji}, \hyperref[auth:a148]{P. Schaus}, \hyperref[auth:a230]{Laurence A. Wolsey}, \hyperref[auth:a152]{Y. Deville} & The StockingCost Constraint & \href{works/HoundjiSWD14.pdf}{Yes} & \cite{HoundjiSWD14} & 2014 & CP 2014 & 16 & 5 & 7 & \ref{b:HoundjiSWD14} & \ref{c:HoundjiSWD14}\\
\rowlabel{a:KoschB14}KoschB14 \href{https://doi.org/10.1007/978-3-319-07046-9\_5}{KoschB14} & \hyperref[auth:a333]{S. Kosch}, \hyperref[auth:a89]{J. Christopher Beck} & A New {MIP} Model for Parallel-Batch Scheduling with Non-identical Job Sizes & \href{works/KoschB14.pdf}{Yes} & \cite{KoschB14} & 2014 & CPAIOR 2014 & 16 & 4 & 18 & \ref{b:KoschB14} & \ref{c:KoschB14}\\
\rowlabel{a:LipovetzkyBPS14}LipovetzkyBPS14 \href{http://www.aaai.org/ocs/index.php/ICAPS/ICAPS14/paper/view/7942}{LipovetzkyBPS14} & \hyperref[auth:a327]{N. Lipovetzky}, \hyperref[auth:a326]{Christina N. Burt}, \hyperref[auth:a328]{Adrian R. Pearce}, \hyperref[auth:a126]{Peter J. Stuckey} & Planning for Mining Operations with Time and Resource Constraints & \href{works/LipovetzkyBPS14.pdf}{Yes} & \cite{LipovetzkyBPS14} & 2014 & ICAPS 2014 & 9 & 0 & 0 & \ref{b:LipovetzkyBPS14} & \ref{c:LipovetzkyBPS14}\\
\rowlabel{a:LouieVNB14}LouieVNB14 \href{https://doi.org/10.1109/ICRA.2014.6907637}{LouieVNB14} & \hyperref[auth:a831]{Wing{-}Yue Geoffrey Louie}, \hyperref[auth:a816]{Tiago Stegun Vaquero}, \hyperref[auth:a210]{G. Nejat}, \hyperref[auth:a89]{J. Christopher Beck} & An autonomous assistive robot for planning, scheduling and facilitating multi-user activities & \href{works/LouieVNB14.pdf}{Yes} & \cite{LouieVNB14} & 2014 & ICRA 2014 & 7 & 16 & 9 & \ref{b:LouieVNB14} & \ref{c:LouieVNB14}\\
\rowlabel{a:BonfiettiLM13}BonfiettiLM13 \href{http://www.aaai.org/ocs/index.php/ICAPS/ICAPS13/paper/view/6050}{BonfiettiLM13} & \hyperref[auth:a204]{A. Bonfietti}, \hyperref[auth:a143]{M. Lombardi}, \hyperref[auth:a144]{M. Milano} & De-Cycling Cyclic Scheduling Problems & \href{works/BonfiettiLM13.pdf}{Yes} & \cite{BonfiettiLM13} & 2013 & ICAPS 2013 & 5 & 0 & 0 & \ref{b:BonfiettiLM13} & \ref{c:BonfiettiLM13}\\
\rowlabel{a:ChuGNSW13}ChuGNSW13 \href{http://www.aaai.org/ocs/index.php/IJCAI/IJCAI13/paper/view/6878}{ChuGNSW13} & \hyperref[auth:a349]{G. Chu}, \hyperref[auth:a805]{S. Gaspers}, \hyperref[auth:a806]{N. Narodytska}, \hyperref[auth:a125]{A. Schutt}, \hyperref[auth:a279]{T. Walsh} & On the Complexity of Global Scheduling Constraints under Structural Restrictions & \href{works/ChuGNSW13.pdf}{Yes} & \cite{ChuGNSW13} & 2013 & IJCAI 2013 & 7 & 0 & 0 & \ref{b:ChuGNSW13} & \ref{c:ChuGNSW13}\\
\rowlabel{a:CireCH13}CireCH13 \href{https://doi.org/10.1007/978-3-642-38171-3\_22}{CireCH13} & \hyperref[auth:a159]{Andr{\'{e}} A. Cir{\'{e}}}, \hyperref[auth:a341]{E. Coban}, \hyperref[auth:a162]{John N. Hooker} & Mixed Integer Programming vs. Logic-Based Benders Decomposition for Planning and Scheduling & \href{works/CireCH13.pdf}{Yes} & \cite{CireCH13} & 2013 & CPAIOR 2013 & 7 & 3 & 23 & \ref{b:CireCH13} & \ref{c:CireCH13}\\
\rowlabel{a:GuSS13}GuSS13 \href{https://doi.org/10.1007/978-3-642-38171-3\_24}{GuSS13} & \hyperref[auth:a342]{H. Gu}, \hyperref[auth:a125]{A. Schutt}, \hyperref[auth:a126]{Peter J. Stuckey} & A Lagrangian Relaxation Based Forward-Backward Improvement Heuristic for Maximising the Net Present Value of Resource-Constrained Projects & \href{works/GuSS13.pdf}{Yes} & \cite{GuSS13} & 2013 & CPAIOR 2013 & 7 & 10 & 24 & \ref{b:GuSS13} & \ref{c:GuSS13}\\
\rowlabel{a:HeinzKB13}HeinzKB13 \href{https://doi.org/10.1007/978-3-642-38171-3\_2}{HeinzKB13} & \hyperref[auth:a134]{S. Heinz}, \hyperref[auth:a337]{W. Ku}, \hyperref[auth:a89]{J. Christopher Beck} & Recent Improvements Using Constraint Integer Programming for Resource Allocation and Scheduling & \href{works/HeinzKB13.pdf}{Yes} & \cite{HeinzKB13} & 2013 & CPAIOR 2013 & 16 & 9 & 15 & \ref{b:HeinzKB13} & \ref{c:HeinzKB13}\\
\rowlabel{a:KelarevaTK13}KelarevaTK13 \href{https://doi.org/10.1007/978-3-642-38171-3\_8}{KelarevaTK13} & \hyperref[auth:a338]{E. Kelareva}, \hyperref[auth:a339]{K. Tierney}, \hyperref[auth:a340]{P. Kilby} & {CP} Methods for Scheduling and Routing with Time-Dependent Task Costs & \href{works/KelarevaTK13.pdf}{Yes} & \cite{KelarevaTK13} & 2013 & CPAIOR 2013 & 17 & 16 & 28 & \ref{b:KelarevaTK13} & \ref{c:KelarevaTK13}\\
\rowlabel{a:LetortCB13}LetortCB13 \href{https://doi.org/10.1007/978-3-642-38171-3\_10}{LetortCB13} & \hyperref[auth:a128]{A. Letort}, \hyperref[auth:a91]{M. Carlsson}, \hyperref[auth:a129]{N. Beldiceanu} & A Synchronized Sweep Algorithm for the \emph{k-dimensional cumulative} Constraint & \href{works/LetortCB13.pdf}{Yes} & \cite{LetortCB13} & 2013 & CPAIOR 2013 & 16 & 3 & 10 & \ref{b:LetortCB13} & \ref{c:LetortCB13}\\
\rowlabel{a:LombardiM13}LombardiM13 \href{http://www.aaai.org/ocs/index.php/ICAPS/ICAPS13/paper/view/6052}{LombardiM13} & \hyperref[auth:a143]{M. Lombardi}, \hyperref[auth:a144]{M. Milano} & A Min-Flow Algorithm for Minimal Critical Set Detection in Resource Constrained Project Scheduling & \href{works/LombardiM13.pdf}{Yes} & \cite{LombardiM13} & 2013 & ICAPS 2013 & 2 & 0 & 0 & \ref{b:LombardiM13} & \ref{c:LombardiM13}\\
\rowlabel{a:OuelletQ13}OuelletQ13 \href{https://doi.org/10.1007/978-3-642-40627-0\_42}{OuelletQ13} & \hyperref[auth:a241]{P. Ouellet}, \hyperref[auth:a37]{C. Quimper} & Time-Table Extended-Edge-Finding for the Cumulative Constraint & \href{works/OuelletQ13.pdf}{Yes} & \cite{OuelletQ13} & 2013 & CP 2013 & 16 & 12 & 14 & \ref{b:OuelletQ13} & \ref{c:OuelletQ13}\\
\rowlabel{a:SchuttFS13}SchuttFS13 \href{https://doi.org/10.1007/978-3-642-40627-0\_47}{SchuttFS13} & \hyperref[auth:a125]{A. Schutt}, \hyperref[auth:a155]{T. Feydy}, \hyperref[auth:a126]{Peter J. Stuckey} & Scheduling Optional Tasks with Explanation & \href{works/SchuttFS13.pdf}{Yes} & \cite{SchuttFS13} & 2013 & CP 2013 & 17 & 10 & 20 & \ref{b:SchuttFS13} & \ref{c:SchuttFS13}\\
\rowlabel{a:SchuttFS13a}SchuttFS13a \href{https://doi.org/10.1007/978-3-642-38171-3\_16}{SchuttFS13a} & \hyperref[auth:a125]{A. Schutt}, \hyperref[auth:a155]{T. Feydy}, \hyperref[auth:a126]{Peter J. Stuckey} & Explaining Time-Table-Edge-Finding Propagation for the Cumulative Resource Constraint & \href{works/SchuttFS13a.pdf}{Yes} & \cite{SchuttFS13a} & 2013 & CPAIOR 2013 & 17 & 20 & 27 & \ref{b:SchuttFS13a} & \ref{c:SchuttFS13a}\\
\rowlabel{a:TranTDB13}TranTDB13 \href{http://www.aaai.org/ocs/index.php/ICAPS/ICAPS13/paper/view/6005}{TranTDB13} & \hyperref[auth:a811]{Tony T. Tran}, \hyperref[auth:a830]{D. Terekhov}, \hyperref[auth:a815]{Douglas G. Down}, \hyperref[auth:a89]{J. Christopher Beck} & Hybrid Queueing Theory and Scheduling Models for Dynamic Environments with Sequence-Dependent Setup Times & \href{works/TranTDB13.pdf}{Yes} & \cite{TranTDB13} & 2013 & ICAPS 2013 & 9 & 0 & 0 & \ref{b:TranTDB13} & \ref{c:TranTDB13}\\
\rowlabel{a:BillautHL12}BillautHL12 \href{https://doi.org/10.1007/978-3-642-29828-8\_5}{BillautHL12} & \hyperref[auth:a343]{J. Billaut}, \hyperref[auth:a1]{E. Hebrard}, \hyperref[auth:a3]{P. Lopez} & Complete Characterization of Near-Optimal Sequences for the Two-Machine Flow Shop Scheduling Problem & \href{works/BillautHL12.pdf}{Yes} & \cite{BillautHL12} & 2012 & CPAIOR 2012 & 15 & 1 & 19 & \ref{b:BillautHL12} & \ref{c:BillautHL12}\\
\rowlabel{a:BonfiettiLBM12}BonfiettiLBM12 \href{https://doi.org/10.1007/978-3-642-29828-8\_6}{BonfiettiLBM12} & \hyperref[auth:a204]{A. Bonfietti}, \hyperref[auth:a143]{M. Lombardi}, \hyperref[auth:a248]{L. Benini}, \hyperref[auth:a144]{M. Milano} & Global Cyclic Cumulative Constraint & \href{works/BonfiettiLBM12.pdf}{Yes} & \cite{BonfiettiLBM12} & 2012 & CPAIOR 2012 & 16 & 2 & 11 & \ref{b:BonfiettiLBM12} & \ref{c:BonfiettiLBM12}\\
\rowlabel{a:BonfiettiM12}BonfiettiM12 \href{https://ceur-ws.org/Vol-926/paper2.pdf}{BonfiettiM12} & \hyperref[auth:a204]{A. Bonfietti}, \hyperref[auth:a144]{M. Milano} & A Constraint-based Approach to Cyclic Resource-Constrained Scheduling Problem & \href{works/BonfiettiM12.pdf}{Yes} & \cite{BonfiettiM12} & 2012 & DC SIAAI 2012 & 3 & 0 & 0 & \ref{b:BonfiettiM12} & \ref{c:BonfiettiM12}\\
\rowlabel{a:GuSW12}GuSW12 \href{https://doi.org/10.1007/978-3-642-33558-7\_55}{GuSW12} & \hyperref[auth:a342]{H. Gu}, \hyperref[auth:a126]{Peter J. Stuckey}, \hyperref[auth:a156]{Mark G. Wallace} & Maximising the Net Present Value of Large Resource-Constrained Projects & \href{works/GuSW12.pdf}{Yes} & \cite{GuSW12} & 2012 & CP 2012 & 15 & 5 & 20 & \ref{b:GuSW12} & \ref{c:GuSW12}\\
\rowlabel{a:HeinzB12}HeinzB12 \href{https://doi.org/10.1007/978-3-642-29828-8\_14}{HeinzB12} & \hyperref[auth:a134]{S. Heinz}, \hyperref[auth:a89]{J. Christopher Beck} & Reconsidering Mixed Integer Programming and MIP-Based Hybrids for Scheduling & \href{works/HeinzB12.pdf}{Yes} & \cite{HeinzB12} & 2012 & CPAIOR 2012 & 17 & 8 & 21 & \ref{b:HeinzB12} & \ref{c:HeinzB12}\\
\rowlabel{a:IfrimOS12}IfrimOS12 \href{https://doi.org/10.1007/978-3-642-33558-7\_68}{IfrimOS12} & \hyperref[auth:a184]{G. Ifrim}, \hyperref[auth:a16]{B. O'Sullivan}, \hyperref[auth:a17]{H. Simonis} & Properties of Energy-Price Forecasts for Scheduling & \href{works/IfrimOS12.pdf}{Yes} & \cite{IfrimOS12} & 2012 & CP 2012 & 16 & 6 & 20 & \ref{b:IfrimOS12} & \ref{c:IfrimOS12}\\
\rowlabel{a:LetortBC12}LetortBC12 \href{https://doi.org/10.1007/978-3-642-33558-7\_33}{LetortBC12} & \hyperref[auth:a128]{A. Letort}, \hyperref[auth:a129]{N. Beldiceanu}, \hyperref[auth:a91]{M. Carlsson} & A Scalable Sweep Algorithm for the cumulative Constraint & \href{works/LetortBC12.pdf}{Yes} & \cite{LetortBC12} & 2012 & CP 2012 & 16 & 18 & 12 & \ref{b:LetortBC12} & \ref{c:LetortBC12}\\
\rowlabel{a:RendlPHPR12}RendlPHPR12 \href{https://doi.org/10.1007/978-3-642-29828-8\_22}{RendlPHPR12} & \hyperref[auth:a344]{A. Rendl}, \hyperref[auth:a345]{M. Prandtstetter}, \hyperref[auth:a346]{G. Hiermann}, \hyperref[auth:a347]{J. Puchinger}, \hyperref[auth:a348]{G{\"{u}}nther R. Raidl} & Hybrid Heuristics for Multimodal Homecare Scheduling & \href{works/RendlPHPR12.pdf}{Yes} & \cite{RendlPHPR12} & 2012 & CPAIOR 2012 & 17 & 14 & 14 & \ref{b:RendlPHPR12} & \ref{c:RendlPHPR12}\\
\rowlabel{a:SchuttCSW12}SchuttCSW12 \href{https://doi.org/10.1007/978-3-642-29828-8\_24}{SchuttCSW12} & \hyperref[auth:a125]{A. Schutt}, \hyperref[auth:a349]{G. Chu}, \hyperref[auth:a126]{Peter J. Stuckey}, \hyperref[auth:a156]{Mark G. Wallace} & Maximising the Net Present Value for Resource-Constrained Project Scheduling & \href{works/SchuttCSW12.pdf}{Yes} & \cite{SchuttCSW12} & 2012 & CPAIOR 2012 & 17 & 18 & 21 & \ref{b:SchuttCSW12} & \ref{c:SchuttCSW12}\\
\rowlabel{a:SerraNM12}SerraNM12 \href{https://doi.org/10.1007/978-3-642-33558-7\_59}{SerraNM12} & \hyperref[auth:a242]{T. Serra}, \hyperref[auth:a243]{G. Nishioka}, \hyperref[auth:a244]{Fernando J. M. Marcellino} & The Offshore Resources Scheduling Problem: Detailing a Constraint Programming Approach & \href{works/SerraNM12.pdf}{Yes} & \cite{SerraNM12} & 2012 & CP 2012 & 17 & 0 & 8 & \ref{b:SerraNM12} & \ref{c:SerraNM12}\\
\rowlabel{a:SimoninAHL12}SimoninAHL12 \href{https://doi.org/10.1007/978-3-642-33558-7\_5}{SimoninAHL12} & \hyperref[auth:a127]{G. Simonin}, \hyperref[auth:a6]{C. Artigues}, \hyperref[auth:a1]{E. Hebrard}, \hyperref[auth:a3]{P. Lopez} & Scheduling Scientific Experiments on the Rosetta/Philae Mission & \href{works/SimoninAHL12.pdf}{Yes} & \cite{SimoninAHL12} & 2012 & CP 2012 & 15 & 3 & 8 & \ref{b:SimoninAHL12} & \ref{c:SimoninAHL12}\\
\rowlabel{a:TranB12}TranB12 \href{https://doi.org/10.3233/978-1-61499-098-7-774}{TranB12} & \hyperref[auth:a811]{Tony T. Tran}, \hyperref[auth:a89]{J. Christopher Beck} & Logic-based Benders Decomposition for Alternative Resource Scheduling with Sequence Dependent Setups & \href{works/TranB12.pdf}{Yes} & \cite{TranB12} & 2012 & ECAI 2012 & 6 & 0 & 0 & \ref{b:TranB12} & \ref{c:TranB12}\\
\rowlabel{a:ZhangLS12}ZhangLS12 \href{https://doi.org/10.1109/CIT.2012.96}{ZhangLS12} & \hyperref[auth:a621]{X. Zhang}, \hyperref[auth:a622]{Z. Lv}, \hyperref[auth:a623]{X. Song} & Model and Solution for Hot Strip Rolling Scheduling Problem Based on Constraint Programming Method & \href{works/ZhangLS12.pdf}{Yes} & \cite{ZhangLS12} & 2012 & CIT 2012 & 4 & 1 & 3 & \ref{b:ZhangLS12} & \ref{c:ZhangLS12}\\
\rowlabel{a:BajestaniB11}BajestaniB11 \href{http://aaai.org/ocs/index.php/ICAPS/ICAPS11/paper/view/2680}{BajestaniB11} & \hyperref[auth:a829]{Maliheh Aramon Bajestani}, \hyperref[auth:a89]{J. Christopher Beck} & Scheduling an Aircraft Repair Shop & \href{works/BajestaniB11.pdf}{Yes} & \cite{BajestaniB11} & 2011 & ICAPS 2011 & 8 & 0 & 0 & \ref{b:BajestaniB11} & \ref{c:BajestaniB11}\\
\rowlabel{a:BonfiettiLBM11}BonfiettiLBM11 \href{https://doi.org/10.1007/978-3-642-23786-7\_12}{BonfiettiLBM11} & \hyperref[auth:a204]{A. Bonfietti}, \hyperref[auth:a143]{M. Lombardi}, \hyperref[auth:a248]{L. Benini}, \hyperref[auth:a144]{M. Milano} & A Constraint Based Approach to Cyclic {RCPSP} & \href{works/BonfiettiLBM11.pdf}{Yes} & \cite{BonfiettiLBM11} & 2011 & CP 2011 & 15 & 3 & 14 & \ref{b:BonfiettiLBM11} & \ref{c:BonfiettiLBM11}\\
\rowlabel{a:ChapadosJR11}ChapadosJR11 \href{https://doi.org/10.1007/978-3-642-21311-3\_7}{ChapadosJR11} & \hyperref[auth:a350]{N. Chapados}, \hyperref[auth:a351]{M. Joliveau}, \hyperref[auth:a332]{L. Rousseau} & Retail Store Workforce Scheduling by Expected Operating Income Maximization & \href{works/ChapadosJR11.pdf}{Yes} & \cite{ChapadosJR11} & 2011 & CPAIOR 2011 & 6 & 5 & 12 & \ref{b:ChapadosJR11} & \ref{c:ChapadosJR11}\\
\rowlabel{a:ClercqPBJ11}ClercqPBJ11 \href{https://doi.org/10.1007/978-3-642-23786-7\_20}{ClercqPBJ11} & \hyperref[auth:a249]{Alexis De Clercq}, \hyperref[auth:a227]{T. Petit}, \hyperref[auth:a129]{N. Beldiceanu}, \hyperref[auth:a250]{N. Jussien} & Filtering Algorithms for Discrete Cumulative Problems with Overloads of Resource & \href{works/ClercqPBJ11.pdf}{Yes} & \cite{ClercqPBJ11} & 2011 & CP 2011 & 16 & 3 & 11 & \ref{b:ClercqPBJ11} & \ref{c:ClercqPBJ11}\\
\rowlabel{a:EdisO11}EdisO11 \href{https://doi.org/10.1007/978-3-642-21311-3\_10}{EdisO11} & \hyperref[auth:a352]{Emrah B. Edis}, \hyperref[auth:a353]{C. Oguz} & Parallel Machine Scheduling with Additional Resources: {A} Lagrangian-Based Constraint Programming Approach & \href{works/EdisO11.pdf}{Yes} & \cite{EdisO11} & 2011 & CPAIOR 2011 & 7 & 5 & 16 & \ref{b:EdisO11} & \ref{c:EdisO11}\\
\rowlabel{a:GrimesH11}GrimesH11 \href{https://doi.org/10.1007/978-3-642-23786-7\_28}{GrimesH11} & \hyperref[auth:a183]{D. Grimes}, \hyperref[auth:a1]{E. Hebrard} & Models and Strategies for Variants of the Job Shop Scheduling Problem & \href{works/GrimesH11.pdf}{Yes} & \cite{GrimesH11} & 2011 & CP 2011 & 17 & 5 & 18 & \ref{b:GrimesH11} & \ref{c:GrimesH11}\\
\rowlabel{a:HeinzS11}HeinzS11 \href{https://doi.org/10.1007/978-3-642-20662-7\_34}{HeinzS11} & \hyperref[auth:a134]{S. Heinz}, \hyperref[auth:a135]{J. Schulz} & Explanations for the Cumulative Constraint: An Experimental Study & \href{works/HeinzS11.pdf}{Yes} & \cite{HeinzS11} & 2011 & SEA 2011 & 10 & 5 & 12 & \ref{b:HeinzS11} & \ref{c:HeinzS11}\\
\rowlabel{a:HermenierDL11}HermenierDL11 \href{https://doi.org/10.1007/978-3-642-23786-7\_5}{HermenierDL11} & \hyperref[auth:a245]{F. Hermenier}, \hyperref[auth:a246]{S. Demassey}, \hyperref[auth:a247]{X. Lorca} & Bin Repacking Scheduling in Virtualized Datacenters & \href{works/HermenierDL11.pdf}{Yes} & \cite{HermenierDL11} & 2011 & CP 2011 & 15 & 28 & 5 & \ref{b:HermenierDL11} & \ref{c:HermenierDL11}\\
\rowlabel{a:KameugneFSN11}KameugneFSN11 \href{https://doi.org/10.1007/978-3-642-23786-7\_37}{KameugneFSN11} & \hyperref[auth:a10]{R. Kameugne}, \hyperref[auth:a131]{Laure Pauline Fotso}, \hyperref[auth:a132]{Joseph D. Scott}, \hyperref[auth:a133]{Y. Ngo{-}Kateu} & A Quadratic Edge-Finding Filtering Algorithm for Cumulative Resource Constraints & \href{works/KameugneFSN11.pdf}{Yes} & \cite{KameugneFSN11} & 2011 & CP 2011 & 15 & 7 & 9 & \ref{b:KameugneFSN11} & \ref{c:KameugneFSN11}\\
\rowlabel{a:LahimerLH11}LahimerLH11 \href{https://doi.org/10.1007/978-3-642-21311-3\_12}{LahimerLH11} & \hyperref[auth:a355]{A. Lahimer}, \hyperref[auth:a3]{P. Lopez}, \hyperref[auth:a356]{M. Haouari} & Climbing Depth-Bounded Adjacent Discrepancy Search for Solving Hybrid Flow Shop Scheduling Problems with Multiprocessor Tasks & \href{works/LahimerLH11.pdf}{Yes} & \cite{LahimerLH11} & 2011 & CPAIOR 2011 & 14 & 3 & 15 & \ref{b:LahimerLH11} & \ref{c:LahimerLH11}\\
\rowlabel{a:LombardiBMB11}LombardiBMB11 \href{https://doi.org/10.1007/978-3-642-21311-3\_14}{LombardiBMB11} & \hyperref[auth:a143]{M. Lombardi}, \hyperref[auth:a204]{A. Bonfietti}, \hyperref[auth:a144]{M. Milano}, \hyperref[auth:a248]{L. Benini} & Precedence Constraint Posting for Cyclic Scheduling Problems & \href{works/LombardiBMB11.pdf}{Yes} & \cite{LombardiBMB11} & 2011 & CPAIOR 2011 & 17 & 1 & 13 & \ref{b:LombardiBMB11} & \ref{c:LombardiBMB11}\\
\rowlabel{a:SimonisH11}SimonisH11 \href{http://dx.doi.org/10.1007/978-3-642-19486-3_5}{SimonisH11} & \hyperref[auth:a17]{H. Simonis}, \hyperref[auth:a924]{T. Hadzic} & A Resource Cost Aware Cumulative & \href{works/SimonisH11.pdf}{Yes} & \cite{SimonisH11} & 2011 & CSCLP 2011 & 14 & 3 & 9 & \ref{b:SimonisH11} & \ref{c:SimonisH11}\\
\rowlabel{a:Vilim11}Vilim11 \href{https://doi.org/10.1007/978-3-642-21311-3\_22}{Vilim11} & \hyperref[auth:a121]{P. Vil{\'{\i}}m} & Timetable Edge Finding Filtering Algorithm for Discrete Cumulative Resources & \href{works/Vilim11.pdf}{Yes} & \cite{Vilim11} & 2011 & CPAIOR 2011 & 16 & 28 & 6 & \ref{b:Vilim11} & \ref{c:Vilim11}\\
\rowlabel{a:Wolf11}Wolf11 \href{http://dx.doi.org/10.1007/978-3-642-19486-3_8}{Wolf11} & \hyperref[auth:a51]{A. Wolf} & Constraint-Based Modeling and Scheduling of Clinical Pathways & \href{works/Wolf11.pdf}{Yes} & \cite{Wolf11} & 2011 & CSCLP 2011 & 17 & 5 & 19 & \ref{b:Wolf11} & \ref{c:Wolf11}\\
\rowlabel{a:ZibranR11}ZibranR11 \href{https://doi.org/10.1109/ICPC.2011.45}{ZibranR11} & \hyperref[auth:a629]{Minhaz F. Zibran}, \hyperref[auth:a630]{Chanchal K. Roy} & Conflict-Aware Optimal Scheduling of Code Clone Refactoring: {A} Constraint Programming Approach & \href{works/ZibranR11.pdf}{Yes} & \cite{ZibranR11} & 2011 & ICPC 2011 & 4 & 17 & 18 & \ref{b:ZibranR11} & \ref{c:ZibranR11}\\
\rowlabel{a:ZibranR11a}ZibranR11a \href{https://doi.org/10.1109/SCAM.2011.21}{ZibranR11a} & \hyperref[auth:a629]{Minhaz F. Zibran}, \hyperref[auth:a630]{Chanchal K. Roy} & A Constraint Programming Approach to Conflict-Aware Optimal Scheduling of Prioritized Code Clone Refactoring & \href{works/ZibranR11a.pdf}{Yes} & \cite{ZibranR11a} & 2011 & SCAM 2011 & 10 & 26 & 27 & \ref{b:ZibranR11a} & \ref{c:ZibranR11a}\\
\rowlabel{a:BertholdHLMS10}BertholdHLMS10 \href{https://doi.org/10.1007/978-3-642-13520-0\_34}{BertholdHLMS10} & \hyperref[auth:a357]{T. Berthold}, \hyperref[auth:a134]{S. Heinz}, \hyperref[auth:a358]{Marco E. L{\"{u}}bbecke}, \hyperref[auth:a359]{Rolf H. M{\"{o}}hring}, \hyperref[auth:a135]{J. Schulz} & A Constraint Integer Programming Approach for Resource-Constrained Project Scheduling & \href{works/BertholdHLMS10.pdf}{Yes} & \cite{BertholdHLMS10} & 2010 & CPAIOR 2010 & 5 & 28 & 10 & \ref{b:BertholdHLMS10} & \ref{c:BertholdHLMS10}\\
\rowlabel{a:CobanH10}CobanH10 \href{https://doi.org/10.1007/978-3-642-13520-0\_11}{CobanH10} & \hyperref[auth:a341]{E. Coban}, \hyperref[auth:a162]{John N. Hooker} & Single-Facility Scheduling over Long Time Horizons by Logic-Based Benders Decomposition & \href{works/CobanH10.pdf}{Yes} & \cite{CobanH10} & 2010 & CPAIOR 2010 & 5 & 9 & 9 & \ref{b:CobanH10} & \ref{c:CobanH10}\\
\rowlabel{a:Davenport10}Davenport10 \href{https://doi.org/10.1007/978-3-642-13520-0\_12}{Davenport10} & \hyperref[auth:a251]{Andrew J. Davenport} & Integrated Maintenance Scheduling for Semiconductor Manufacturing & \href{works/Davenport10.pdf}{Yes} & \cite{Davenport10} & 2010 & CPAIOR 2010 & 5 & 9 & 2 & \ref{b:Davenport10} & \ref{c:Davenport10}\\
\rowlabel{a:GrimesH10}GrimesH10 \href{https://doi.org/10.1007/978-3-642-13520-0\_19}{GrimesH10} & \hyperref[auth:a183]{D. Grimes}, \hyperref[auth:a1]{E. Hebrard} & Job Shop Scheduling with Setup Times and Maximal Time-Lags: {A} Simple Constraint Programming Approach & \href{works/GrimesH10.pdf}{Yes} & \cite{GrimesH10} & 2010 & CPAIOR 2010 & 15 & 13 & 20 & \ref{b:GrimesH10} & \ref{c:GrimesH10}\\
\rowlabel{a:LombardiM10}LombardiM10 \href{https://doi.org/10.1007/978-3-642-15396-9\_32}{LombardiM10} & \hyperref[auth:a143]{M. Lombardi}, \hyperref[auth:a144]{M. Milano} & Constraint Based Scheduling to Deal with Uncertain Durations and Self-Timed Execution & \href{works/LombardiM10.pdf}{Yes} & \cite{LombardiM10} & 2010 & CP 2010 & 15 & 1 & 11 & \ref{b:LombardiM10} & \ref{c:LombardiM10}\\
\rowlabel{a:MakMS10}MakMS10 \href{https://doi.org/10.1109/ICNC.2010.5583494}{MakMS10} & \hyperref[auth:a637]{K. Mak}, \hyperref[auth:a638]{J. Ma}, \hyperref[auth:a639]{W. Su} & A constraint programming approach for production scheduling of multi-period virtual cellular manufacturing systems & \href{works/MakMS10.pdf}{Yes} & \cite{MakMS10} & 2010 & ICNC 2010 & 5 & 1 & 3 & \ref{b:MakMS10} & \ref{c:MakMS10}\\
\rowlabel{a:SchuttW10}SchuttW10 \href{https://doi.org/10.1007/978-3-642-15396-9\_36}{SchuttW10} & \hyperref[auth:a125]{A. Schutt}, \hyperref[auth:a51]{A. Wolf} & A New \emph{O}(\emph{n}\({}^{\mbox{2}}\)log\emph{n}) Not-First/Not-Last Pruning Algorithm for Cumulative Resource Constraints & \href{works/SchuttW10.pdf}{Yes} & \cite{SchuttW10} & 2010 & CP 2010 & 15 & 13 & 14 & \ref{b:SchuttW10} & \ref{c:SchuttW10}\\
\rowlabel{a:SunLYL10}SunLYL10 \href{https://doi.org/10.1109/GreenCom-CPSCom.2010.111}{SunLYL10} & \hyperref[auth:a633]{Z. Sun}, \hyperref[auth:a634]{H. Li}, \hyperref[auth:a635]{M. Yao}, \hyperref[auth:a636]{N. Li} & Scheduling Optimization Techniques for FlexRay Using Constraint-Programming & \href{works/SunLYL10.pdf}{Yes} & \cite{SunLYL10} & 2010 & GreenCom 2010 & 6 & 4 & 8 & \ref{b:SunLYL10} & \ref{c:SunLYL10}\\
\rowlabel{a:Acuna-AgostMFG09}Acuna-AgostMFG09 \href{https://doi.org/10.1007/978-3-642-01929-6\_24}{Acuna-AgostMFG09} & \hyperref[auth:a360]{R. Acuna{-}Agost}, \hyperref[auth:a361]{P. Michelon}, \hyperref[auth:a362]{D. Feillet}, \hyperref[auth:a363]{S. Gueye} & Constraint Programming and Mixed Integer Linear Programming for Rescheduling Trains under Disrupted Operations & \href{works/Acuna-AgostMFG09.pdf}{Yes} & \cite{Acuna-AgostMFG09} & 2009 & CPAIOR 2009 & 2 & 3 & 2 & \ref{b:Acuna-AgostMFG09} & \ref{c:Acuna-AgostMFG09}\\
\rowlabel{a:AronssonBK09}AronssonBK09 \href{http://drops.dagstuhl.de/opus/volltexte/2009/2141}{AronssonBK09} & \hyperref[auth:a717]{M. Aronsson}, \hyperref[auth:a718]{M. Bohlin}, \hyperref[auth:a719]{P. Kreuger} & {MILP} formulations of cumulative constraints for railway scheduling - {A} comparative study & \href{works/AronssonBK09.pdf}{Yes} & \cite{AronssonBK09} & 2009 & ATMOS 2009 & 13 & 0 & 0 & \ref{b:AronssonBK09} & \ref{c:AronssonBK09}\\
\rowlabel{a:Baptiste09}Baptiste09 \href{https://doi.org/10.1007/978-3-642-04244-7\_1}{Baptiste09} & \hyperref[auth:a164]{P. Baptiste} & Constraint-Based Schedulers, Do They Really Work? & \href{works/Baptiste09.pdf}{Yes} & \cite{Baptiste09} & 2009 & CP 2009 & 1 & 0 & 0 & \ref{b:Baptiste09} & \ref{c:Baptiste09}\\
\rowlabel{a:GrimesHM09}GrimesHM09 \href{https://doi.org/10.1007/978-3-642-04244-7\_33}{GrimesHM09} & \hyperref[auth:a183]{D. Grimes}, \hyperref[auth:a1]{E. Hebrard}, \hyperref[auth:a82]{A. Malapert} & Closing the Open Shop: Contradicting Conventional Wisdom & \href{works/GrimesHM09.pdf}{Yes} & \cite{GrimesHM09} & 2009 & CP 2009 & 9 & 15 & 12 & \ref{b:GrimesHM09} & \ref{c:GrimesHM09}\\
\rowlabel{a:Laborie09}Laborie09 \href{https://doi.org/10.1007/978-3-642-01929-6\_12}{Laborie09} & \hyperref[auth:a118]{P. Laborie} & {IBM} {ILOG} {CP} Optimizer for Detailed Scheduling Illustrated on Three Problems & \href{works/Laborie09.pdf}{Yes} & \cite{Laborie09} & 2009 & CPAIOR 2009 & 15 & 53 & 2 & \ref{b:Laborie09} & \ref{c:Laborie09}\\
\rowlabel{a:LombardiM09}LombardiM09 \href{https://doi.org/10.1007/978-3-642-04244-7\_45}{LombardiM09} & \hyperref[auth:a143]{M. Lombardi}, \hyperref[auth:a144]{M. Milano} & A Precedence Constraint Posting Approach for the {RCPSP} with Time Lags and Variable Durations & \href{works/LombardiM09.pdf}{Yes} & \cite{LombardiM09} & 2009 & CP 2009 & 15 & 7 & 12 & \ref{b:LombardiM09} & \ref{c:LombardiM09}\\
\rowlabel{a:MonetteDH09}MonetteDH09 \href{http://aaai.org/ocs/index.php/ICAPS/ICAPS09/paper/view/712}{MonetteDH09} & \hyperref[auth:a150]{J. Monette}, \hyperref[auth:a152]{Y. Deville}, \hyperref[auth:a149]{Pascal Van Hentenryck} & Just-In-Time Scheduling with Constraint Programming & \href{works/MonetteDH09.pdf}{Yes} & \cite{MonetteDH09} & 2009 & ICAPS 2009 & 8 & 0 & 0 & \ref{b:MonetteDH09} & \ref{c:MonetteDH09}\\
\rowlabel{a:SchuttFSW09}SchuttFSW09 \href{https://doi.org/10.1007/978-3-642-04244-7\_58}{SchuttFSW09} & \hyperref[auth:a125]{A. Schutt}, \hyperref[auth:a155]{T. Feydy}, \hyperref[auth:a126]{Peter J. Stuckey}, \hyperref[auth:a117]{M. Wallace} & Why Cumulative Decomposition Is Not as Bad as It Sounds & \href{works/SchuttFSW09.pdf}{Yes} & \cite{SchuttFSW09} & 2009 & CP 2009 & 16 & 34 & 11 & \ref{b:SchuttFSW09} & \ref{c:SchuttFSW09}\\
\rowlabel{a:ThiruvadyBME09}ThiruvadyBME09 \href{https://doi.org/10.1007/978-3-642-04918-7\_3}{ThiruvadyBME09} & \hyperref[auth:a402]{Dhananjay R. Thiruvady}, \hyperref[auth:a646]{C. Blum}, \hyperref[auth:a647]{B. Meyer}, \hyperref[auth:a476]{Andreas T. Ernst} & Hybridizing Beam-ACO with Constraint Programming for Single Machine Job Scheduling & \href{works/ThiruvadyBME09.pdf}{Yes} & \cite{ThiruvadyBME09} & 2009 & HM 2009 & 15 & 13 & 12 & \ref{b:ThiruvadyBME09} & \ref{c:ThiruvadyBME09}\\
\rowlabel{a:Vilim09}Vilim09 \href{https://doi.org/10.1007/978-3-642-04244-7\_62}{Vilim09} & \hyperref[auth:a121]{P. Vil{\'{\i}}m} & Edge Finding Filtering Algorithm for Discrete Cumulative Resources in \emph{O}(\emph{kn} log \emph{n})\{{\textbackslash}mathcal O\}(kn \{{\textbackslash}rm log\} n) & \href{works/Vilim09.pdf}{Yes} & \cite{Vilim09} & 2009 & CP 2009 & 15 & 25 & 4 & \ref{b:Vilim09} & \ref{c:Vilim09}\\
\rowlabel{a:Vilim09a}Vilim09a \href{https://doi.org/10.1007/978-3-642-01929-6\_22}{Vilim09a} & \hyperref[auth:a121]{P. Vil{\'{\i}}m} & Max Energy Filtering Algorithm for Discrete Cumulative Resources & \href{works/Vilim09a.pdf}{Yes} & \cite{Vilim09a} & 2009 & CPAIOR 2009 & 15 & 13 & 4 & \ref{b:Vilim09a} & \ref{c:Vilim09a}\\
\rowlabel{a:Wolf09}Wolf09 \href{http://dx.doi.org/10.1007/978-3-642-00675-3_2}{Wolf09} & \hyperref[auth:a51]{A. Wolf}, \hyperref[auth:a720]{G. Schrader} & Linear Weighted-Task-Sum – Scheduling Prioritized Tasks on a Single Resource & \href{works/Wolf09.pdf}{Yes} & \cite{Wolf09} & 2009 & INAP 2009 & 17 & 1 & 12 & \ref{b:Wolf09} & \ref{c:Wolf09}\\
\rowlabel{a:BarlattCG08}BarlattCG08 \href{https://doi.org/10.1007/978-3-540-68155-7\_24}{BarlattCG08} & \hyperref[auth:a367]{A. Barlatt}, \hyperref[auth:a368]{Amy Mainville Cohn}, \hyperref[auth:a369]{Oleg Yu. Gusikhin} & A Hybrid Approach for Solving Shift-Selection and Task-Sequencing Problems & \href{works/BarlattCG08.pdf}{Yes} & \cite{BarlattCG08} & 2008 & CPAIOR 2008 & 5 & 1 & 9 & \ref{b:BarlattCG08} & \ref{c:BarlattCG08}\\
\rowlabel{a:BeldiceanuCP08}BeldiceanuCP08 \href{https://doi.org/10.1007/978-3-540-68155-7\_5}{BeldiceanuCP08} & \hyperref[auth:a129]{N. Beldiceanu}, \hyperref[auth:a91]{M. Carlsson}, \hyperref[auth:a364]{E. Poder} & New Filtering for the cumulative Constraint in the Context of Non-Overlapping Rectangles & \href{works/BeldiceanuCP08.pdf}{Yes} & \cite{BeldiceanuCP08} & 2008 & CPAIOR 2008 & 15 & 8 & 9 & \ref{b:BeldiceanuCP08} & \ref{c:BeldiceanuCP08}\\
\rowlabel{a:BeniniLMR08}BeniniLMR08 \href{http://dx.doi.org/10.1007/978-3-540-85958-1_2}{BeniniLMR08} & \hyperref[auth:a248]{L. Benini}, \hyperref[auth:a143]{M. Lombardi}, \hyperref[auth:a144]{M. Milano}, \hyperref[auth:a727]{M. Ruggiero} & A Constraint Programming Approach for Allocation and Scheduling on the CELL Broadband Engine & \href{works/BeniniLMR08.pdf}{Yes} & \cite{BeniniLMR08} & 2008 & CP 2008 & 15 & 7 & 23 & \ref{b:BeniniLMR08} & \ref{c:BeniniLMR08}\\
\rowlabel{a:DoomsH08}DoomsH08 \href{https://doi.org/10.1007/978-3-540-68155-7\_8}{DoomsH08} & \hyperref[auth:a365]{G. Dooms}, \hyperref[auth:a149]{Pascal Van Hentenryck} & Gap Reduction Techniques for Online Stochastic Project Scheduling & \href{works/DoomsH08.pdf}{Yes} & \cite{DoomsH08} & 2008 & CPAIOR 2008 & 16 & 1 & 2 & \ref{b:DoomsH08} & \ref{c:DoomsH08}\\
\rowlabel{a:HentenryckM08}HentenryckM08 \href{https://doi.org/10.1007/978-3-540-68155-7\_41}{HentenryckM08} & \hyperref[auth:a149]{Pascal Van Hentenryck}, \hyperref[auth:a32]{L. Michel} & The Steel Mill Slab Design Problem Revisited & \href{works/HentenryckM08.pdf}{Yes} & \cite{HentenryckM08} & 2008 & CPAIOR 2008 & 5 & 13 & 3 & \ref{b:HentenryckM08} & \ref{c:HentenryckM08}\\
\rowlabel{a:LauLN08}LauLN08 \href{https://doi.org/10.1007/978-3-540-68155-7\_33}{LauLN08} & \hyperref[auth:a370]{Hoong Chuin Lau}, \hyperref[auth:a371]{Kong Wei Lye}, \hyperref[auth:a372]{Viet Bang Nguyen} & A Combinatorial Auction Framework for Solving Decentralized Scheduling Problems (Extended Abstract) & \href{works/LauLN08.pdf}{Yes} & \cite{LauLN08} & 2008 & CPAIOR 2008 & 5 & 0 & 4 & \ref{b:LauLN08} & \ref{c:LauLN08}\\
\rowlabel{a:MouraSCL08}MouraSCL08 \href{https://doi.org/10.1007/978-3-540-85958-1\_3}{MouraSCL08} & \hyperref[auth:a161]{Arnaldo Vieira Moura}, \hyperref[auth:a172]{Cid C. de Souza}, \hyperref[auth:a159]{Andr{\'{e}} A. Cir{\'{e}}}, \hyperref[auth:a158]{Tony Minoru Tamura Lopes} & Planning and Scheduling the Operation of a Very Large Oil Pipeline Network & \href{works/MouraSCL08.pdf}{Yes} & \cite{MouraSCL08} & 2008 & CP 2008 & 16 & 11 & 10 & \ref{b:MouraSCL08} & \ref{c:MouraSCL08}\\
\rowlabel{a:MouraSCL08a}MouraSCL08a \href{https://doi.org/10.1109/CSE.2008.24}{MouraSCL08a} & \hyperref[auth:a161]{Arnaldo Vieira Moura}, \hyperref[auth:a172]{Cid C. de Souza}, \hyperref[auth:a159]{Andr{\'{e}} A. Cir{\'{e}}}, \hyperref[auth:a158]{Tony Minoru Tamura Lopes} & Heuristics and Constraint Programming Hybridizations for a Real Pipeline Planning and Scheduling Problem & \href{works/MouraSCL08a.pdf}{Yes} & \cite{MouraSCL08a} & 2008 & CSE 2008 & 8 & 5 & 14 & \ref{b:MouraSCL08a} & \ref{c:MouraSCL08a}\\
\rowlabel{a:PoderB08}PoderB08 \href{http://www.aaai.org/Library/ICAPS/2008/icaps08-033.php}{PoderB08} & \hyperref[auth:a364]{E. Poder}, \hyperref[auth:a129]{N. Beldiceanu} & Filtering for a Continuous Multi-Resources cumulative Constraint with Resource Consumption and Production & \href{works/PoderB08.pdf}{Yes} & \cite{PoderB08} & 2008 & ICAPS 2008 & 8 & 0 & 0 & \ref{b:PoderB08} & \ref{c:PoderB08}\\
\rowlabel{a:WatsonB08}WatsonB08 \href{https://doi.org/10.1007/978-3-540-68155-7\_21}{WatsonB08} & \hyperref[auth:a366]{J. Watson}, \hyperref[auth:a89]{J. Christopher Beck} & A Hybrid Constraint Programming / Local Search Approach to the Job-Shop Scheduling Problem & \href{works/WatsonB08.pdf}{Yes} & \cite{WatsonB08} & 2008 & CPAIOR 2008 & 15 & 14 & 17 & \ref{b:WatsonB08} & \ref{c:WatsonB08}\\
\rowlabel{a:AkkerDH07}AkkerDH07 \href{https://doi.org/10.1007/978-3-540-72397-4\_27}{AkkerDH07} & \hyperref[auth:a378]{J. M. van den Akker}, \hyperref[auth:a379]{G. Diepen}, \hyperref[auth:a380]{J. A. Hoogeveen} & A Column Generation Based Destructive Lower Bound for Resource Constrained Project Scheduling Problems & \href{works/AkkerDH07.pdf}{Yes} & \cite{AkkerDH07} & 2007 & CPAIOR 2007 & 15 & 2 & 8 & \ref{b:AkkerDH07} & \ref{c:AkkerDH07}\\
\rowlabel{a:BeldiceanuP07}BeldiceanuP07 \href{https://doi.org/10.1007/978-3-540-72397-4\_16}{BeldiceanuP07} & \hyperref[auth:a129]{N. Beldiceanu}, \hyperref[auth:a364]{E. Poder} & A Continuous Multi-resources \emph{cumulative} Constraint with Positive-Negative Resource Consumption-Production & \href{works/BeldiceanuP07.pdf}{Yes} & \cite{BeldiceanuP07} & 2007 & CPAIOR 2007 & 15 & 4 & 7 & \ref{b:BeldiceanuP07} & \ref{c:BeldiceanuP07}\\
\rowlabel{a:DavenportKRSH07}DavenportKRSH07 \href{https://doi.org/10.1007/978-3-540-74970-7\_7}{DavenportKRSH07} & \hyperref[auth:a251]{Andrew J. Davenport}, \hyperref[auth:a252]{J. Kalagnanam}, \hyperref[auth:a253]{C. Reddy}, \hyperref[auth:a254]{S. Siegel}, \hyperref[auth:a255]{J. Hou} & An Application of Constraint Programming to Generating Detailed Operations Schedules for Steel Manufacturing & \href{works/DavenportKRSH07.pdf}{Yes} & \cite{DavenportKRSH07} & 2007 & CP 2007 & 13 & 1 & 2 & \ref{b:DavenportKRSH07} & \ref{c:DavenportKRSH07}\\
\rowlabel{a:GarganiR07}GarganiR07 \href{https://doi.org/10.1007/978-3-540-74970-7\_8}{GarganiR07} & \hyperref[auth:a256]{A. Gargani}, \hyperref[auth:a257]{P. Refalo} & An Efficient Model and Strategy for the Steel Mill Slab Design Problem & \href{works/GarganiR07.pdf}{Yes} & \cite{GarganiR07} & 2007 & CP 2007 & 13 & 17 & 5 & \ref{b:GarganiR07} & \ref{c:GarganiR07}\\
\rowlabel{a:HoeveGSL07}HoeveGSL07 \href{http://www.aaai.org/Library/AAAI/2007/aaai07-291.php}{HoeveGSL07} & \hyperref[auth:a212]{Willem{-}Jan van Hoeve}, \hyperref[auth:a652]{Carla P. Gomes}, \hyperref[auth:a653]{B. Selman}, \hyperref[auth:a143]{M. Lombardi} & Optimal Multi-Agent Scheduling with Constraint Programming & \href{works/HoeveGSL07.pdf}{Yes} & \cite{HoeveGSL07} & 2007 & AAAI 2007 & 6 & 0 & 0 & \ref{b:HoeveGSL07} & \ref{c:HoeveGSL07}\\
\rowlabel{a:KeriK07}KeriK07 \href{https://doi.org/10.1007/978-3-540-72397-4\_10}{KeriK07} & \hyperref[auth:a373]{A. K{\'{e}}ri}, \hyperref[auth:a157]{T. Kis} & Computing Tight Time Windows for {RCPSPWET} with the Primal-Dual Method & \href{works/KeriK07.pdf}{Yes} & \cite{KeriK07} & 2007 & CPAIOR 2007 & 14 & 1 & 13 & \ref{b:KeriK07} & \ref{c:KeriK07}\\
\rowlabel{a:KovacsB07}KovacsB07 \href{https://doi.org/10.1007/978-3-540-72397-4\_9}{KovacsB07} & \hyperref[auth:a147]{A. Kov{\'{a}}cs}, \hyperref[auth:a89]{J. Christopher Beck} & A Global Constraint for Total Weighted Completion Time & \href{works/KovacsB07.pdf}{Yes} & \cite{KovacsB07} & 2007 & CPAIOR 2007 & 15 & 2 & 12 & \ref{b:KovacsB07} & \ref{c:KovacsB07}\\
\rowlabel{a:KrogtLPHJ07}KrogtLPHJ07 \href{https://doi.org/10.1007/978-3-540-74970-7\_10}{KrogtLPHJ07} & \hyperref[auth:a258]{Roman van der Krogt}, \hyperref[auth:a180]{J. Little}, \hyperref[auth:a259]{K. Pulliam}, \hyperref[auth:a260]{S. Hanhilammi}, \hyperref[auth:a261]{Y. Jin} & Scheduling for Cellular Manufacturing & \href{works/KrogtLPHJ07.pdf}{Yes} & \cite{KrogtLPHJ07} & 2007 & CP 2007 & 13 & 2 & 3 & \ref{b:KrogtLPHJ07} & \ref{c:KrogtLPHJ07}\\
\rowlabel{a:Limtanyakul07}Limtanyakul07 \href{https://doi.org/10.1007/978-3-540-77903-2\_65}{Limtanyakul07} & \hyperref[auth:a145]{K. Limtanyakul} & Scheduling of Tests on Vehicle Prototypes Using Constraint and Integer Programming & \href{works/Limtanyakul07.pdf}{Yes} & \cite{Limtanyakul07} & 2007 & GOR 2007 & 6 & 2 & 3 & \ref{b:Limtanyakul07} & \ref{c:Limtanyakul07}\\
\rowlabel{a:MonetteDD07}MonetteDD07 \href{https://doi.org/10.1007/978-3-540-72397-4\_14}{MonetteDD07} & \hyperref[auth:a150]{J. Monette}, \hyperref[auth:a152]{Y. Deville}, \hyperref[auth:a374]{P. Dupont} & A Position-Based Propagator for the Open-Shop Problem & \href{works/MonetteDD07.pdf}{Yes} & \cite{MonetteDD07} & 2007 & CPAIOR 2007 & 14 & 0 & 12 & \ref{b:MonetteDD07} & \ref{c:MonetteDD07}\\
\rowlabel{a:RossiTHP07}RossiTHP07 \href{https://doi.org/10.1007/978-3-540-72397-4\_17}{RossiTHP07} & \hyperref[auth:a375]{R. Rossi}, \hyperref[auth:a376]{A. Tarim}, \hyperref[auth:a138]{B. Hnich}, \hyperref[auth:a377]{Steven D. Prestwich} & Replenishment Planning for Stochastic Inventory Systems with Shortage Cost & \href{works/RossiTHP07.pdf}{Yes} & \cite{RossiTHP07} & 2007 & CPAIOR 2007 & 15 & 6 & 10 & \ref{b:RossiTHP07} & \ref{c:RossiTHP07}\\
\rowlabel{a:Beck06}Beck06 \href{http://www.aaai.org/Library/ICAPS/2006/icaps06-028.php}{Beck06} & \hyperref[auth:a89]{J. Christopher Beck} & An Empirical Study of Multi-Point Constructive Search for Constraint-Based Scheduling & \href{works/Beck06.pdf}{Yes} & \cite{Beck06} & 2006 & ICAPS 2006 & 10 & 0 & 0 & \ref{b:Beck06} & \ref{c:Beck06}\\
\rowlabel{a:BeniniBGM06}BeniniBGM06 \href{https://doi.org/10.1007/11757375\_6}{BeniniBGM06} & \hyperref[auth:a248]{L. Benini}, \hyperref[auth:a381]{D. Bertozzi}, \hyperref[auth:a382]{A. Guerri}, \hyperref[auth:a144]{M. Milano} & Allocation, Scheduling and Voltage Scaling on Energy Aware MPSoCs & \href{works/BeniniBGM06.pdf}{Yes} & \cite{BeniniBGM06} & 2006 & CPAIOR 2006 & 15 & 18 & 10 & \ref{b:BeniniBGM06} & \ref{c:BeniniBGM06}\\
\rowlabel{a:GomesHS06}GomesHS06 \href{http://www.aaai.org/Library/Symposia/Spring/2006/ss06-04-024.php}{GomesHS06} & \hyperref[auth:a652]{Carla P. Gomes}, \hyperref[auth:a212]{Willem{-}Jan van Hoeve}, \hyperref[auth:a653]{B. Selman} & Constraint Programming for Distributed Planning and Scheduling & \href{works/GomesHS06.pdf}{Yes} & \cite{GomesHS06} & 2006 & AAAI 2006 & 2 & 0 & 0 & \ref{b:GomesHS06} & \ref{c:GomesHS06}\\
\rowlabel{a:KhemmoudjPB06}KhemmoudjPB06 \href{https://doi.org/10.1007/11889205\_21}{KhemmoudjPB06} & \hyperref[auth:a262]{Mohand Ou Idir Khemmoudj}, \hyperref[auth:a263]{M. Porcheron}, \hyperref[auth:a264]{H. Bennaceur} & When Constraint Programming and Local Search Solve the Scheduling Problem of Electricit{\'{e}} de France Nuclear Power Plant Outages & \href{works/KhemmoudjPB06.pdf}{Yes} & \cite{KhemmoudjPB06} & 2006 & CP 2006 & 13 & 8 & 8 & \ref{b:KhemmoudjPB06} & \ref{c:KhemmoudjPB06}\\
\rowlabel{a:KovacsV06}KovacsV06 \href{https://doi.org/10.1007/11757375\_13}{KovacsV06} & \hyperref[auth:a147]{A. Kov{\'{a}}cs}, \hyperref[auth:a281]{J. V{\'{a}}ncza} & Progressive Solutions: {A} Simple but Efficient Dominance Rule for Practical {RCPSP} & \href{works/KovacsV06.pdf}{Yes} & \cite{KovacsV06} & 2006 & CPAIOR 2006 & 13 & 2 & 7 & \ref{b:KovacsV06} & \ref{c:KovacsV06}\\
\rowlabel{a:LiuJ06}LiuJ06 \href{https://doi.org/10.1007/11801603\_92}{LiuJ06} & \hyperref[auth:a664]{Y. Liu}, \hyperref[auth:a665]{Y. Jiang} & {LP-TPOP:} Integrating Planning and Scheduling Through Constraint Programming & \href{works/LiuJ06.pdf}{Yes} & \cite{LiuJ06} & 2006 & PRICAI 2006 & 5 & 0 & 0 & \ref{b:LiuJ06} & \ref{c:LiuJ06}\\
\rowlabel{a:QuSN06}QuSN06 \href{https://doi.org/10.1109/ISSOC.2006.321973}{QuSN06} & \hyperref[auth:a661]{Y. Qu}, \hyperref[auth:a662]{J. Soininen}, \hyperref[auth:a663]{J. Nurmi} & Using Constraint Programming to Achieve Optimal Prefetch Scheduling for Dependent Tasks on Run-Time Reconfigurable Devices & \href{works/QuSN06.pdf}{Yes} & \cite{QuSN06} & 2006 & SoC 2006 & 4 & 2 & 5 & \ref{b:QuSN06} & \ref{c:QuSN06}\\
\rowlabel{a:Wallace06}Wallace06 \href{http://dx.doi.org/10.1007/978-3-540-73817-6_1}{Wallace06} & \hyperref[auth:a117]{M. Wallace} & Hybrid Algorithms in Constraint Programming & \href{works/Wallace06.pdf}{Yes} & \cite{Wallace06} & 2006 & CSCLP 2006 & 32 & 1 & 35 & \ref{b:Wallace06} & \ref{c:Wallace06}\\
\rowlabel{a:AbrilSB05}AbrilSB05 \href{https://doi.org/10.1007/11564751\_75}{AbrilSB05} & \hyperref[auth:a273]{M. Abril}, \hyperref[auth:a154]{Miguel A. Salido}, \hyperref[auth:a274]{F. Barber} & Distributed Constraints for Large-Scale Scheduling Problems & \href{works/AbrilSB05.pdf}{Yes} & \cite{AbrilSB05} & 2005 & CP 2005 & 1 & 0 & 0 & \ref{b:AbrilSB05} & \ref{c:AbrilSB05}\\
\rowlabel{a:ArtiouchineB05}ArtiouchineB05 \href{https://doi.org/10.1007/11564751\_8}{ArtiouchineB05} & \hyperref[auth:a265]{K. Artiouchine}, \hyperref[auth:a164]{P. Baptiste} & Inter-distance Constraint: An Extension of the All-Different Constraint for Scheduling Equal Length Jobs & \href{works/ArtiouchineB05.pdf}{Yes} & \cite{ArtiouchineB05} & 2005 & CP 2005 & 15 & 3 & 11 & \ref{b:ArtiouchineB05} & \ref{c:ArtiouchineB05}\\
\rowlabel{a:BeckW05}BeckW05 \href{http://ijcai.org/Proceedings/05/Papers/0748.pdf}{BeckW05} & \hyperref[auth:a89]{J. Christopher Beck}, \hyperref[auth:a838]{N. Wilson} & Proactive Algorithms for Scheduling with Probabilistic Durations & \href{works/BeckW05.pdf}{Yes} & \cite{BeckW05} & 2005 & IJCAI 2005 & 6 & 0 & 0 & \ref{b:BeckW05} & \ref{c:BeckW05}\\
\rowlabel{a:CarchraeBF05}CarchraeBF05 \href{https://doi.org/10.1007/11564751\_80}{CarchraeBF05} & \hyperref[auth:a275]{T. Carchrae}, \hyperref[auth:a89]{J. Christopher Beck}, \hyperref[auth:a276]{Eugene C. Freuder} & Methods to Learn Abstract Scheduling Models & \href{works/CarchraeBF05.pdf}{Yes} & \cite{CarchraeBF05} & 2005 & CP 2005 & 1 & 0 & 0 & \ref{b:CarchraeBF05} & \ref{c:CarchraeBF05}\\
\rowlabel{a:ChuX05}ChuX05 \href{https://doi.org/10.1007/11493853\_10}{ChuX05} & \hyperref[auth:a383]{Y. Chu}, \hyperref[auth:a384]{Q. Xia} & A Hybrid Algorithm for a Class of Resource Constrained Scheduling Problems & \href{works/ChuX05.pdf}{Yes} & \cite{ChuX05} & 2005 & CPAIOR 2005 & 15 & 13 & 13 & \ref{b:ChuX05} & \ref{c:ChuX05}\\
\rowlabel{a:DilkinaDH05}DilkinaDH05 \href{https://doi.org/10.1007/11564751\_60}{DilkinaDH05} & \hyperref[auth:a270]{B. Dilkina}, \hyperref[auth:a271]{L. Duan}, \hyperref[auth:a272]{William S. Havens} & Extending Systematic Local Search for Job Shop Scheduling Problems & \href{works/DilkinaDH05.pdf}{Yes} & \cite{DilkinaDH05} & 2005 & CP 2005 & 5 & 2 & 7 & \ref{b:DilkinaDH05} & \ref{c:DilkinaDH05}\\
\rowlabel{a:FortinZDF05}FortinZDF05 \href{https://doi.org/10.1007/11564751\_19}{FortinZDF05} & \hyperref[auth:a266]{J. Fortin}, \hyperref[auth:a267]{P. Zielinski}, \hyperref[auth:a268]{D. Dubois}, \hyperref[auth:a269]{H. Fargier} & Interval Analysis in Scheduling & \href{works/FortinZDF05.pdf}{Yes} & \cite{FortinZDF05} & 2005 & CP 2005 & 15 & 13 & 11 & \ref{b:FortinZDF05} & \ref{c:FortinZDF05}\\
\rowlabel{a:FrankK05}FrankK05 \href{https://doi.org/10.1007/11493853\_15}{FrankK05} & \hyperref[auth:a385]{J. Frank}, \hyperref[auth:a386]{E. K{\"{u}}rkl{\"{u}}} & Mixed Discrete and Continuous Algorithms for Scheduling Airborne Astronomy Observations & \href{works/FrankK05.pdf}{Yes} & \cite{FrankK05} & 2005 & CPAIOR 2005 & 18 & 4 & 4 & \ref{b:FrankK05} & \ref{c:FrankK05}\\
\rowlabel{a:Geske05}Geske05 \href{https://doi.org/10.1007/11963578\_10}{Geske05} & \hyperref[auth:a667]{U. Geske} & Railway Scheduling with Declarative Constraint Programming & \href{works/Geske05.pdf}{Yes} & \cite{Geske05} & 2005 & INAP 2005 & 18 & 2 & 3 & \ref{b:Geske05} & \ref{c:Geske05}\\
\rowlabel{a:GodardLN05}GodardLN05 \href{http://www.aaai.org/Library/ICAPS/2005/icaps05-009.php}{GodardLN05} & \hyperref[auth:a783]{D. Godard}, \hyperref[auth:a118]{P. Laborie}, \hyperref[auth:a666]{W. Nuijten} & Randomized Large Neighborhood Search for Cumulative Scheduling & \href{works/GodardLN05.pdf}{Yes} & \cite{GodardLN05} & 2005 & ICAPS 2005 & 9 & 0 & 0 & \ref{b:GodardLN05} & \ref{c:GodardLN05}\\
\rowlabel{a:HebrardTW05}HebrardTW05 \href{https://doi.org/10.1007/11564751\_117}{HebrardTW05} & \hyperref[auth:a1]{E. Hebrard}, \hyperref[auth:a278]{P. Tyler}, \hyperref[auth:a279]{T. Walsh} & Computing Super-Schedules & \href{works/HebrardTW05.pdf}{Yes} & \cite{HebrardTW05} & 2005 & CP 2005 & 1 & 0 & 3 & \ref{b:HebrardTW05} & \ref{c:HebrardTW05}\\
\rowlabel{a:Hooker05a}Hooker05a \href{https://doi.org/10.1007/11564751\_25}{Hooker05a} & \hyperref[auth:a162]{John N. Hooker} & Planning and Scheduling to Minimize Tardiness & \href{works/Hooker05a.pdf}{Yes} & \cite{Hooker05a} & 2005 & CP 2005 & 14 & 30 & 10 & \ref{b:Hooker05a} & \ref{c:Hooker05a}\\
\rowlabel{a:KovacsEKV05}KovacsEKV05 \href{https://doi.org/10.1007/11564751\_118}{KovacsEKV05} & \hyperref[auth:a147]{A. Kov{\'{a}}cs}, \hyperref[auth:a280]{P. Egri}, \hyperref[auth:a157]{T. Kis}, \hyperref[auth:a281]{J. V{\'{a}}ncza} & Proterv-II: An Integrated Production Planning and Scheduling System & \href{works/KovacsEKV05.pdf}{Yes} & \cite{KovacsEKV05} & 2005 & CP 2005 & 1 & 2 & 3 & \ref{b:KovacsEKV05} & \ref{c:KovacsEKV05}\\
\rowlabel{a:MoffittPP05}MoffittPP05 \href{http://www.aaai.org/Library/AAAI/2005/aaai05-188.php}{MoffittPP05} & \hyperref[auth:a780]{Michael D. Moffitt}, \hyperref[auth:a781]{B. Peintner}, \hyperref[auth:a782]{Martha E. Pollack} & Augmenting Disjunctive Temporal Problems with Finite-Domain Constraints & \href{works/MoffittPP05.pdf}{Yes} & \cite{MoffittPP05} & 2005 & AAAI 2005 & 6 & 0 & 0 & \ref{b:MoffittPP05} & \ref{c:MoffittPP05}\\
\rowlabel{a:QuirogaZH05}QuirogaZH05 \href{https://doi.org/10.1109/ROBOT.2005.1570686}{QuirogaZH05} & \hyperref[auth:a632]{O. Quiroga}, \hyperref[auth:a631]{L. Zeballos}, \hyperref[auth:a598]{Gabriela P. Henning} & A Constraint Programming Approach to Tool Allocation and Resource Scheduling in {FMS} & \href{works/QuirogaZH05.pdf}{Yes} & \cite{QuirogaZH05} & 2005 & ICRA 2005 & 6 & 2 & 7 & \ref{b:QuirogaZH05} & \ref{c:QuirogaZH05}\\
\rowlabel{a:SchuttWS05}SchuttWS05 \href{https://doi.org/10.1007/11963578\_6}{SchuttWS05} & \hyperref[auth:a125]{A. Schutt}, \hyperref[auth:a51]{A. Wolf}, \hyperref[auth:a720]{G. Schrader} & Not-First and Not-Last Detection for Cumulative Scheduling in \emph{O}(\emph{n}\({}^{\mbox{3}}\)log\emph{n}) & \href{works/SchuttWS05.pdf}{Yes} & \cite{SchuttWS05} & 2005 & INAP 2005 & 15 & 6 & 4 & \ref{b:SchuttWS05} & \ref{c:SchuttWS05}\\
\rowlabel{a:Vilim05}Vilim05 \href{https://doi.org/10.1007/11493853\_29}{Vilim05} & \hyperref[auth:a121]{P. Vil{\'{\i}}m} & Computing Explanations for the Unary Resource Constraint & \href{works/Vilim05.pdf}{Yes} & \cite{Vilim05} & 2005 & CPAIOR 2005 & 14 & 5 & 8 & \ref{b:Vilim05} & \ref{c:Vilim05}\\
\rowlabel{a:Wolf05}Wolf05 \href{http://dx.doi.org/10.1007/11402763_15}{Wolf05} & \hyperref[auth:a51]{A. Wolf} & Better Propagation for Non-preemptive Single-Resource Constraint Problems & \href{works/Wolf05.pdf}{Yes} & \cite{Wolf05} & 2005 & CSCLP 2005 & 15 & 4 & 8 & \ref{b:Wolf05} & \ref{c:Wolf05}\\
\rowlabel{a:WolfS05}WolfS05 \href{https://doi.org/10.1007/11963578\_8}{WolfS05} & \hyperref[auth:a51]{A. Wolf}, \hyperref[auth:a720]{G. Schrader} & \emph{O}(\emph{n} log\emph{n}) Overload Checking for the Cumulative Constraint and Its Application & \href{works/WolfS05.pdf}{Yes} & \cite{WolfS05} & 2005 & INAP 2005 & 14 & 6 & 6 & \ref{b:WolfS05} & \ref{c:WolfS05}\\
\rowlabel{a:WuBB05}WuBB05 \href{https://doi.org/10.1007/11564751\_110}{WuBB05} & \hyperref[auth:a277]{Christine Wei Wu}, \hyperref[auth:a223]{Kenneth N. Brown}, \hyperref[auth:a89]{J. Christopher Beck} & Scheduling with Uncertain Start Dates & \href{works/WuBB05.pdf}{Yes} & \cite{WuBB05} & 2005 & CP 2005 & 1 & 0 & 0 & \ref{b:WuBB05} & \ref{c:WuBB05}\\
\rowlabel{a:ArtiguesBF04}ArtiguesBF04 \href{https://doi.org/10.1007/978-3-540-24664-0\_3}{ArtiguesBF04} & \hyperref[auth:a6]{C. Artigues}, \hyperref[auth:a389]{S. Belmokhtar}, \hyperref[auth:a362]{D. Feillet} & A New Exact Solution Algorithm for the Job Shop Problem with Sequence-Dependent Setup Times & \href{works/ArtiguesBF04.pdf}{Yes} & \cite{ArtiguesBF04} & 2004 & CPAIOR 2004 & 13 & 16 & 9 & \ref{b:ArtiguesBF04} & \ref{c:ArtiguesBF04}\\
\rowlabel{a:BeckW04}BeckW04 \href{}{BeckW04} & \hyperref[auth:a89]{J. Christopher Beck}, \hyperref[auth:a838]{N. Wilson} & Job Shop Scheduling with Probabilistic Durations & \href{works/BeckW04.pdf}{Yes} & \cite{BeckW04} & 2004 & ECAI 2004 & 5 & 0 & 0 & \ref{b:BeckW04} & \ref{c:BeckW04}\\
\rowlabel{a:HentenryckM04}HentenryckM04 \href{https://doi.org/10.1007/978-3-540-24664-0\_22}{HentenryckM04} & \hyperref[auth:a149]{Pascal Van Hentenryck}, \hyperref[auth:a32]{L. Michel} & Scheduling Abstractions for Local Search & \href{works/HentenryckM04.pdf}{Yes} & \cite{HentenryckM04} & 2004 & CPAIOR 2004 & 16 & 12 & 14 & \ref{b:HentenryckM04} & \ref{c:HentenryckM04}\\
\rowlabel{a:Hooker04}Hooker04 \href{https://doi.org/10.1007/978-3-540-30201-8\_24}{Hooker04} & \hyperref[auth:a162]{John N. Hooker} & A Hybrid Method for Planning and Scheduling & \href{works/Hooker04.pdf}{Yes} & \cite{Hooker04} & 2004 & CP 2004 & 12 & 39 & 9 & \ref{b:Hooker04} & \ref{c:Hooker04}\\
\rowlabel{a:KovacsV04}KovacsV04 \href{https://doi.org/10.1007/978-3-540-30201-8\_26}{KovacsV04} & \hyperref[auth:a147]{A. Kov{\'{a}}cs}, \hyperref[auth:a281]{J. V{\'{a}}ncza} & Completable Partial Solutions in Constraint Programming and Constraint-Based Scheduling & \href{works/KovacsV04.pdf}{Yes} & \cite{KovacsV04} & 2004 & CP 2004 & 15 & 3 & 12 & \ref{b:KovacsV04} & \ref{c:KovacsV04}\\
\rowlabel{a:LimRX04}LimRX04 \href{https://doi.org/10.1007/978-3-540-30201-8\_59}{LimRX04} & \hyperref[auth:a282]{A. Lim}, \hyperref[auth:a283]{B. Rodrigues}, \hyperref[auth:a284]{Z. Xu} & Solving the Crane Scheduling Problem Using Intelligent Search Schemes & \href{works/LimRX04.pdf}{Yes} & \cite{LimRX04} & 2004 & CP 2004 & 5 & 5 & 6 & \ref{b:LimRX04} & \ref{c:LimRX04}\\
\rowlabel{a:MaraveliasG04}MaraveliasG04 \href{https://doi.org/10.1007/978-3-540-24664-0\_1}{MaraveliasG04} & \hyperref[auth:a387]{Christos T. Maravelias}, \hyperref[auth:a388]{Ignacio E. Grossmann} & Using {MILP} and {CP} for the Scheduling of Batch Chemical Processes & \href{works/MaraveliasG04.pdf}{Yes} & \cite{MaraveliasG04} & 2004 & CPAIOR 2004 & 20 & 15 & 15 & \ref{b:MaraveliasG04} & \ref{c:MaraveliasG04}\\
\rowlabel{a:Sadykov04}Sadykov04 \href{https://doi.org/10.1007/978-3-540-24664-0\_31}{Sadykov04} & \hyperref[auth:a390]{R. Sadykov} & A Hybrid Branch-And-Cut Algorithm for the One-Machine Scheduling Problem & \href{works/Sadykov04.pdf}{Yes} & \cite{Sadykov04} & 2004 & CPAIOR 2004 & 7 & 11 & 7 & \ref{b:Sadykov04} & \ref{c:Sadykov04}\\
\rowlabel{a:Vilim04}Vilim04 \href{https://doi.org/10.1007/978-3-540-24664-0\_23}{Vilim04} & \hyperref[auth:a121]{P. Vil{\'{\i}}m} & O(n log n) Filtering Algorithms for Unary Resource Constraint & \href{works/Vilim04.pdf}{Yes} & \cite{Vilim04} & 2004 & CPAIOR 2004 & 13 & 22 & 5 & \ref{b:Vilim04} & \ref{c:Vilim04}\\
\rowlabel{a:VilimBC04}VilimBC04 \href{https://doi.org/10.1007/978-3-540-30201-8\_8}{VilimBC04} & \hyperref[auth:a121]{P. Vil{\'{\i}}m}, \hyperref[auth:a153]{R. Bart{\'{a}}k}, \hyperref[auth:a163]{O. Cepek} & Unary Resource Constraint with Optional Activities & \href{works/VilimBC04.pdf}{Yes} & \cite{VilimBC04} & 2004 & CP 2004 & 15 & 13 & 4 & \ref{b:VilimBC04} & \ref{c:VilimBC04}\\
\rowlabel{a:VillaverdeP04}VillaverdeP04 \href{}{VillaverdeP04} & \hyperref[auth:a668]{K. Villaverde}, \hyperref[auth:a33]{E. Pontelli} & An Investigation of Scheduling in Distributed Constraint Logic Programming & No & \cite{VillaverdeP04} & 2004 & ISCA 2004 & 6 & 0 & 0 & No & \ref{c:VillaverdeP04}\\
\rowlabel{a:WolinskiKG04}WolinskiKG04 \href{https://doi.org/10.1109/DSD.2004.1333291}{WolinskiKG04} & \hyperref[auth:a669]{C. Wolinski}, \hyperref[auth:a670]{K. Kuchcinski}, \hyperref[auth:a671]{Maya B. Gokhale} & A Constraints Programming Approach to Communication Scheduling on SoPC Architectures & \href{works/WolinskiKG04.pdf}{Yes} & \cite{WolinskiKG04} & 2004 & DSD 2004 & 8 & 0 & 9 & \ref{b:WolinskiKG04} & \ref{c:WolinskiKG04}\\
\rowlabel{a:BeckPS03}BeckPS03 \href{http://www.aaai.org/Library/ICAPS/2003/icaps03-027.php}{BeckPS03} & \hyperref[auth:a89]{J. Christopher Beck}, \hyperref[auth:a839]{P. Prosser}, \hyperref[auth:a840]{E. Selensky} & Vehicle Routing and Job Shop Scheduling: What's the Difference? & \href{works/BeckPS03.pdf}{Yes} & \cite{BeckPS03} & 2003 & ICAPS 2003 & 10 & 0 & 0 & \ref{b:BeckPS03} & \ref{c:BeckPS03}\\
\rowlabel{a:DannaP03}DannaP03 \href{https://doi.org/10.1007/978-3-540-45193-8\_59}{DannaP03} & \hyperref[auth:a290]{E. Danna}, \hyperref[auth:a291]{L. Perron} & Structured vs. Unstructured Large Neighborhood Search: {A} Case Study on Job-Shop Scheduling Problems with Earliness and Tardiness Costs & \href{works/DannaP03.pdf}{Yes} & \cite{DannaP03} & 2003 & CP 2003 & 5 & 21 & 3 & \ref{b:DannaP03} & \ref{c:DannaP03}\\
\rowlabel{a:Kumar03}Kumar03 \href{https://doi.org/10.1007/978-3-540-45193-8\_45}{Kumar03} & \hyperref[auth:a289]{T. K. Satish Kumar} & Incremental Computation of Resource-Envelopes in Producer-Consumer Models & \href{works/Kumar03.pdf}{Yes} & \cite{Kumar03} & 2003 & CP 2003 & 15 & 4 & 2 & \ref{b:Kumar03} & \ref{c:Kumar03}\\
\rowlabel{a:OddiPCC03}OddiPCC03 \href{https://doi.org/10.1007/978-3-540-45193-8\_39}{OddiPCC03} & \hyperref[auth:a285]{A. Oddi}, \hyperref[auth:a286]{N. Policella}, \hyperref[auth:a287]{A. Cesta}, \hyperref[auth:a288]{G. Cortellessa} & Generating High Quality Schedules for a Spacecraft Memory Downlink Problem & \href{works/OddiPCC03.pdf}{Yes} & \cite{OddiPCC03} & 2003 & CP 2003 & 15 & 8 & 6 & \ref{b:OddiPCC03} & \ref{c:OddiPCC03}\\
\rowlabel{a:ValleMGT03}ValleMGT03 \href{https://doi.org/10.1007/978-3-540-45226-3\_180}{ValleMGT03} & \hyperref[auth:a676]{Carmelo Del Valle}, \hyperref[auth:a677]{Antonio A. M{\'{a}}rquez}, \hyperref[auth:a678]{Rafael M. Gasca}, \hyperref[auth:a679]{M. Toro} & On Selecting and Scheduling Assembly Plans Using Constraint Programming & \href{works/ValleMGT03.pdf}{Yes} & \cite{ValleMGT03} & 2003 & KES 2003 & 8 & 7 & 7 & \ref{b:ValleMGT03} & \ref{c:ValleMGT03}\\
\rowlabel{a:Vilim03}Vilim03 \href{https://doi.org/10.1007/978-3-540-45193-8\_124}{Vilim03} & \hyperref[auth:a121]{P. Vil{\'{\i}}m} & Computing Explanations for Global Scheduling Constraints & \href{works/Vilim03.pdf}{Yes} & \cite{Vilim03} & 2003 & CP 2003 & 1 & 1 & 1 & \ref{b:Vilim03} & \ref{c:Vilim03}\\
\rowlabel{a:Wolf03}Wolf03 \href{https://doi.org/10.1007/978-3-540-45193-8\_50}{Wolf03} & \hyperref[auth:a51]{A. Wolf} & Pruning while Sweeping over Task Intervals & \href{works/Wolf03.pdf}{Yes} & \cite{Wolf03} & 2003 & CP 2003 & 15 & 11 & 7 & \ref{b:Wolf03} & \ref{c:Wolf03}\\
\rowlabel{a:Bartak02}Bartak02 \href{https://doi.org/10.1007/3-540-46135-3\_39}{Bartak02} & \hyperref[auth:a153]{R. Bart{\'{a}}k} & Visopt ShopFloor: On the Edge of Planning and Scheduling & \href{works/Bartak02.pdf}{Yes} & \cite{Bartak02} & 2002 & CP 2002 & 16 & 6 & 4 & \ref{b:Bartak02} & \ref{c:Bartak02}\\
\rowlabel{a:Bartak02a}Bartak02a \href{https://doi.org/10.1007/3-540-36607-5\_14}{Bartak02a} & \hyperref[auth:a153]{R. Bart{\'{a}}k} & Visopt ShopFloor: Going Beyond Traditional Scheduling & \href{works/Bartak02a.pdf}{Yes} & \cite{Bartak02a} & 2002 & ERCIM/CologNet 2002 & 15 & 1 & 9 & \ref{b:Bartak02a} & \ref{c:Bartak02a}\\
\rowlabel{a:BeldiceanuC02}BeldiceanuC02 \href{https://doi.org/10.1007/3-540-46135-3\_5}{BeldiceanuC02} & \hyperref[auth:a129]{N. Beldiceanu}, \hyperref[auth:a91]{M. Carlsson} & A New Multi-resource cumulatives Constraint with Negative Heights & \href{works/BeldiceanuC02.pdf}{Yes} & \cite{BeldiceanuC02} & 2002 & CP 2002 & 17 & 33 & 9 & \ref{b:BeldiceanuC02} & \ref{c:BeldiceanuC02}\\
\rowlabel{a:ElkhyariGJ02}ElkhyariGJ02 \href{https://doi.org/10.1007/3-540-46135-3\_49}{ElkhyariGJ02} & \hyperref[auth:a295]{A. Elkhyari}, \hyperref[auth:a296]{C. Gu{\'{e}}ret}, \hyperref[auth:a250]{N. Jussien} & Conflict-Based Repair Techniques for Solving Dynamic Scheduling Problems & \href{works/ElkhyariGJ02.pdf}{Yes} & \cite{ElkhyariGJ02} & 2002 & CP 2002 & 6 & 1 & 6 & \ref{b:ElkhyariGJ02} & \ref{c:ElkhyariGJ02}\\
\rowlabel{a:ElkhyariGJ02a}ElkhyariGJ02a \href{https://doi.org/10.1007/978-3-540-45157-0\_3}{ElkhyariGJ02a} & \hyperref[auth:a295]{A. Elkhyari}, \hyperref[auth:a296]{C. Gu{\'{e}}ret}, \hyperref[auth:a250]{N. Jussien} & Solving Dynamic Resource Constraint Project Scheduling Problems Using New Constraint Programming Tools & \href{works/ElkhyariGJ02a.pdf}{Yes} & \cite{ElkhyariGJ02a} & 2002 & PATAT 2002 & 24 & 9 & 20 & \ref{b:ElkhyariGJ02a} & \ref{c:ElkhyariGJ02a}\\
\rowlabel{a:HookerY02}HookerY02 \href{https://doi.org/10.1007/3-540-46135-3\_46}{HookerY02} & \hyperref[auth:a162]{John N. Hooker}, \hyperref[auth:a294]{H. Yan} & A Relaxation of the Cumulative Constraint & \href{works/HookerY02.pdf}{Yes} & \cite{HookerY02} & 2002 & CP 2002 & 5 & 8 & 7 & \ref{b:HookerY02} & \ref{c:HookerY02}\\
\rowlabel{a:KamarainenS02}KamarainenS02 \href{https://doi.org/10.1007/3-540-46135-3\_11}{KamarainenS02} & \hyperref[auth:a293]{O. Kamarainen}, \hyperref[auth:a168]{Hani El Sakkout} & Local Probing Applied to Scheduling & \href{works/KamarainenS02.pdf}{Yes} & \cite{KamarainenS02} & 2002 & CP 2002 & 17 & 9 & 13 & \ref{b:KamarainenS02} & \ref{c:KamarainenS02}\\
\rowlabel{a:Muscettola02}Muscettola02 \href{https://doi.org/10.1007/3-540-46135-3\_10}{Muscettola02} & \hyperref[auth:a292]{N. Muscettola} & Computing the Envelope for Stepwise-Constant Resource Allocations & \href{works/Muscettola02.pdf}{Yes} & \cite{Muscettola02} & 2002 & CP 2002 & 16 & 14 & 4 & \ref{b:Muscettola02} & \ref{c:Muscettola02}\\
\rowlabel{a:Vilim02}Vilim02 \href{https://doi.org/10.1007/3-540-46135-3\_62}{Vilim02} & \hyperref[auth:a121]{P. Vil{\'{\i}}m} & Batch Processing with Sequence Dependent Setup Times & \href{works/Vilim02.pdf}{Yes} & \cite{Vilim02} & 2002 & CP 2002 & 1 & 6 & 1 & \ref{b:Vilim02} & \ref{c:Vilim02}\\
\rowlabel{a:ZhuS02}ZhuS02 \href{https://doi.org/10.1007/3-540-47961-9\_69}{ZhuS02} & \hyperref[auth:a684]{Kenny Qili Zhu}, \hyperref[auth:a685]{Andrew E. Santosa} & A Meeting Scheduling System Based on Open Constraint Programming & \href{works/ZhuS02.pdf}{Yes} & \cite{ZhuS02} & 2002 & CAiSE 2002 & 5 & 0 & 5 & \ref{b:ZhuS02} & \ref{c:ZhuS02}\\
\rowlabel{a:Thorsteinsson01}Thorsteinsson01 \href{https://doi.org/10.1007/3-540-45578-7\_2}{Thorsteinsson01} & \hyperref[auth:a887]{Erlendur S. Thorsteinsson} & Branch-and-Check: {A} Hybrid Framework Integrating Mixed Integer Programming and Constraint Logic Programming & \href{works/Thorsteinsson01.pdf}{Yes} & \cite{Thorsteinsson01} & 2001 & CP 2001 & 15 & 67 & 12 & \ref{b:Thorsteinsson01} & \ref{c:Thorsteinsson01}\\
\rowlabel{a:VanczaM01}VanczaM01 \href{https://doi.org/10.1007/3-540-45578-7\_60}{VanczaM01} & \hyperref[auth:a281]{J. V{\'{a}}ncza}, \hyperref[auth:a297]{A. M{\'{a}}rkus} & A Constraint Engine for Manufacturing Process Planning & \href{works/VanczaM01.pdf}{Yes} & \cite{VanczaM01} & 2001 & CP 2001 & 15 & 2 & 19 & \ref{b:VanczaM01} & \ref{c:VanczaM01}\\
\rowlabel{a:VerfaillieL01}VerfaillieL01 \href{https://doi.org/10.1007/3-540-45578-7\_55}{VerfaillieL01} & \hyperref[auth:a175]{G. Verfaillie}, \hyperref[auth:a174]{M. Lema{\^{\i}}tre} & Selecting and Scheduling Observations for Agile Satellites: Some Lessons from the Constraint Reasoning Community Point of View & \href{works/VerfaillieL01.pdf}{Yes} & \cite{VerfaillieL01} & 2001 & CP 2001 & 15 & 11 & 6 & \ref{b:VerfaillieL01} & \ref{c:VerfaillieL01}\\
\rowlabel{a:AngelsmarkJ00}AngelsmarkJ00 \href{https://doi.org/10.1007/3-540-45349-0\_35}{AngelsmarkJ00} & \hyperref[auth:a298]{O. Angelsmark}, \hyperref[auth:a299]{P. Jonsson} & Some Observations on Durations, Scheduling and Allen's Algebra & \href{works/AngelsmarkJ00.pdf}{Yes} & \cite{AngelsmarkJ00} & 2000 & CP 2000 & 5 & 1 & 9 & \ref{b:AngelsmarkJ00} & \ref{c:AngelsmarkJ00}\\
\rowlabel{a:FocacciLN00}FocacciLN00 \href{http://www.aaai.org/Library/AIPS/2000/aips00-010.php}{FocacciLN00} & \hyperref[auth:a785]{F. Focacci}, \hyperref[auth:a118]{P. Laborie}, \hyperref[auth:a666]{W. Nuijten} & Solving Scheduling Problems with Setup Times and Alternative Resources & \href{works/FocacciLN00.pdf}{Yes} & \cite{FocacciLN00} & 2000 & AIPS 2000 & 10 & 0 & 0 & \ref{b:FocacciLN00} & \ref{c:FocacciLN00}\\
\rowlabel{a:DorndorfPH99}DorndorfPH99 \href{http://dx.doi.org/10.1007/978-3-642-58409-1_35}{DorndorfPH99} & \hyperref[auth:a922]{U. Dorndorf}, \hyperref[auth:a445]{E. Pesch}, \hyperref[auth:a923]{Toàn Phan Huy} & Recent Developments in Scheduling & No & \cite{DorndorfPH99} & 1999 & Operations Research Proceedings 1999 & null & 0 & 34 & No & \ref{c:DorndorfPH99}\\
\rowlabel{a:KorbaaYG99}KorbaaYG99 \href{https://doi.org/10.23919/ECC.1999.7099947}{KorbaaYG99} & \hyperref[auth:a690]{O. Korbaa}, \hyperref[auth:a691]{P. Yim}, \hyperref[auth:a692]{J. Gentina} & Solving transient scheduling problem for cyclic production using timed Petri nets and constraint programming & \href{works/KorbaaYG99.pdf}{Yes} & \cite{KorbaaYG99} & 1999 & ECC 1999 & 8 & 1 & 0 & \ref{b:KorbaaYG99} & \ref{c:KorbaaYG99}\\
\rowlabel{a:Simonis99}Simonis99 \href{https://doi.org/10.1007/3-540-45406-3\_6}{Simonis99} & \hyperref[auth:a17]{H. Simonis} & Building Industrial Applications with Constraint Programming & \href{works/Simonis99.pdf}{Yes} & \cite{Simonis99} & 1999 & CCL'99 1999 & 39 & 5 & 18 & \ref{b:Simonis99} & \ref{c:Simonis99}\\
\rowlabel{a:CestaOS98}CestaOS98 \href{https://doi.org/10.1007/3-540-49481-2\_36}{CestaOS98} & \hyperref[auth:a287]{A. Cesta}, \hyperref[auth:a285]{A. Oddi}, \hyperref[auth:a301]{Stephen F. Smith} & Scheduling Multi-capacitated Resources Under Complex Temporal Constraints & \href{works/CestaOS98.pdf}{Yes} & \cite{CestaOS98} & 1998 & CP 1998 & 1 & 5 & 0 & \ref{b:CestaOS98} & \ref{c:CestaOS98}\\
\rowlabel{a:FrostD98}FrostD98 \href{https://doi.org/10.1007/3-540-49481-2\_40}{FrostD98} & \hyperref[auth:a302]{D. Frost}, \hyperref[auth:a303]{R. Dechter} & Optimizing with Constraints: {A} Case Study in Scheduling Maintenance of Electric Power Units & \href{works/FrostD98.pdf}{Yes} & \cite{FrostD98} & 1998 & CP 1998 & 1 & 10 & 2 & \ref{b:FrostD98} & \ref{c:FrostD98}\\
\rowlabel{a:GruianK98}GruianK98 \href{https://doi.org/10.1109/EURMIC.1998.711781}{GruianK98} & \hyperref[auth:a696]{F. Gruian}, \hyperref[auth:a670]{K. Kuchcinski} & Operation Binding and Scheduling for Low Power Using Constraint Logic Programming & \href{works/GruianK98.pdf}{Yes} & \cite{GruianK98} & 1998 & EUROMICRO 1998 & 8 & 5 & 10 & \ref{b:GruianK98} & \ref{c:GruianK98}\\
\rowlabel{a:PembertonG98}PembertonG98 \href{https://doi.org/10.1090/dimacs/057/06}{PembertonG98} & \hyperref[auth:a694]{Joseph C. Pemberton}, \hyperref[auth:a695]{Flavius Galiber III} & A constraint-based approach to satellite scheduling & \href{works/PembertonG98.pdf}{Yes} & \cite{PembertonG98} & 1998 & DIMACS 1998 & 14 & 26 & 0 & \ref{b:PembertonG98} & \ref{c:PembertonG98}\\
\rowlabel{a:RodosekW98}RodosekW98 \href{https://doi.org/10.1007/3-540-49481-2\_28}{RodosekW98} & \hyperref[auth:a300]{R. Rodosek}, \hyperref[auth:a117]{M. Wallace} & A Generic Model and Hybrid Algorithm for Hoist Scheduling Problems & \href{works/RodosekW98.pdf}{Yes} & \cite{RodosekW98} & 1998 & CP 1998 & 15 & 19 & 10 & \ref{b:RodosekW98} & \ref{c:RodosekW98}\\
\rowlabel{a:BaptisteP97}BaptisteP97 \href{https://doi.org/10.1007/BFb0017454}{BaptisteP97} & \hyperref[auth:a164]{P. Baptiste}, \hyperref[auth:a165]{Claude Le Pape} & Constraint Propagation and Decomposition Techniques for Highly Disjunctive and Highly Cumulative Project Scheduling Problems & \href{works/BaptisteP97.pdf}{Yes} & \cite{BaptisteP97} & 1997 & CP 1997 & 15 & 8 & 10 & \ref{b:BaptisteP97} & \ref{c:BaptisteP97}\\
\rowlabel{a:BeckDF97}BeckDF97 \href{https://doi.org/10.1007/BFb0017455}{BeckDF97} & \hyperref[auth:a89]{J. Christopher Beck}, \hyperref[auth:a251]{Andrew J. Davenport}, \hyperref[auth:a305]{Mark S. Fox} & Five Pitfalls of Empirical Scheduling Research & \href{works/BeckDF97.pdf}{Yes} & \cite{BeckDF97} & 1997 & CP 1997 & 15 & 3 & 12 & \ref{b:BeckDF97} & \ref{c:BeckDF97}\\
\rowlabel{a:BoucherBVBL97}BoucherBVBL97 \href{}{BoucherBVBL97} & \hyperref[auth:a700]{E. Boucher}, \hyperref[auth:a701]{A. Bachelu}, \hyperref[auth:a702]{C. Varnier}, \hyperref[auth:a703]{P. Baptiste}, \hyperref[auth:a704]{B. Legeard} & Multi-criteria Comparison Between Algorithmic, Constraint Logic and Specific Constraint Programming on a Real Schedulingt Problem & No & \cite{BoucherBVBL97} & 1997 & PACT 1997 & 18 & 0 & 0 & No & \ref{c:BoucherBVBL97}\\
\rowlabel{a:Caseau97}Caseau97 \href{https://doi.org/10.1007/BFb0017437}{Caseau97} & \hyperref[auth:a304]{Y. Caseau} & Using Constraint Propagation for Complex Scheduling Problems: Managing Size, Complex Resources and Travel & \href{works/Caseau97.pdf}{Yes} & \cite{Caseau97} & 1997 & CP 1997 & 4 & 0 & 0 & \ref{b:Caseau97} & \ref{c:Caseau97}\\
\rowlabel{a:PapeB97}PapeB97 \href{}{PapeB97} & \hyperref[auth:a165]{Claude Le Pape}, \hyperref[auth:a164]{P. Baptiste} & A Constraint Programming Library for Preemptive and Non-Preemptive Scheduling & No & \cite{PapeB97} & 1997 & PACT 1997 & 20 & 0 & 0 & No & \ref{c:PapeB97}\\
\rowlabel{a:BrusoniCLMMT96}BrusoniCLMMT96 \href{https://doi.org/10.1007/3-540-61286-6\_157}{BrusoniCLMMT96} & \hyperref[auth:a731]{V. Brusoni}, \hyperref[auth:a732]{L. Console}, \hyperref[auth:a729]{E. Lamma}, \hyperref[auth:a730]{P. Mello}, \hyperref[auth:a144]{M. Milano}, \hyperref[auth:a733]{P. Terenziani} & Resource-Based vs. Task-Based Approaches for Scheduling Problems & \href{works/BrusoniCLMMT96.pdf}{Yes} & \cite{BrusoniCLMMT96} & 1996 & ISMIS 1996 & 10 & 1 & 9 & \ref{b:BrusoniCLMMT96} & \ref{c:BrusoniCLMMT96}\\
\rowlabel{a:Colombani96}Colombani96 \href{https://doi.org/10.1007/3-540-61551-2\_72}{Colombani96} & \hyperref[auth:a170]{Y. Colombani} & Constraint Programming: an Efficient and Practical Approach to Solving the Job-Shop Problem & \href{works/Colombani96.pdf}{Yes} & \cite{Colombani96} & 1996 & CP 1996 & 15 & 4 & 5 & \ref{b:Colombani96} & \ref{c:Colombani96}\\
\rowlabel{a:Zhou96}Zhou96 \href{https://doi.org/10.1007/3-540-61551-2\_97}{Zhou96} & \hyperref[auth:a178]{J. Zhou} & A Constraint Program for Solving the Job-Shop Problem & \href{works/Zhou96.pdf}{Yes} & \cite{Zhou96} & 1996 & CP 1996 & 15 & 10 & 7 & \ref{b:Zhou96} & \ref{c:Zhou96}\\
\rowlabel{a:Goltz95}Goltz95 \href{https://doi.org/10.1007/3-540-60299-2\_33}{Goltz95} & \hyperref[auth:a307]{H. Goltz} & Reducing Domains for Search in {CLP(FD)} and Its Application to Job-Shop Scheduling & \href{works/Goltz95.pdf}{Yes} & \cite{Goltz95} & 1995 & CP 1995 & 14 & 7 & 7 & \ref{b:Goltz95} & \ref{c:Goltz95}\\
\rowlabel{a:Puget95}Puget95 \href{https://doi.org/10.1007/3-540-60299-2\_43}{Puget95} & \hyperref[auth:a308]{J. Puget} & Applications of Constraint Programming & \href{works/Puget95.pdf}{Yes} & \cite{Puget95} & 1995 & CP 1995 & 4 & 6 & 2 & \ref{b:Puget95} & \ref{c:Puget95}\\
\rowlabel{a:Simonis95}Simonis95 \href{https://doi.org/10.1007/3-540-60299-2\_42}{Simonis95} & \hyperref[auth:a17]{H. Simonis} & The {CHIP} System and Its Applications & \href{works/Simonis95.pdf}{Yes} & \cite{Simonis95} & 1995 & CP 1995 & 4 & 7 & 3 & \ref{b:Simonis95} & \ref{c:Simonis95}\\
\rowlabel{a:Simonis95a}Simonis95a \href{https://doi.org/10.1007/3-540-60794-3\_11}{Simonis95a} & \hyperref[auth:a17]{H. Simonis} & Application Development with the {CHIP} System & \href{works/Simonis95a.pdf}{Yes} & \cite{Simonis95a} & 1995 & CONTESSA 1995 & 21 & 1 & 12 & \ref{b:Simonis95a} & \ref{c:Simonis95a}\\
\rowlabel{a:SimonisC95}SimonisC95 \href{https://doi.org/10.1007/3-540-60299-2\_27}{SimonisC95} & \hyperref[auth:a17]{H. Simonis}, \hyperref[auth:a306]{T. Cornelissens} & Modelling Producer/Consumer Constraints & \href{works/SimonisC95.pdf}{Yes} & \cite{SimonisC95} & 1995 & CP 1995 & 14 & 17 & 8 & \ref{b:SimonisC95} & \ref{c:SimonisC95}\\
\rowlabel{a:Touraivane95}Touraivane95 \href{https://doi.org/10.1007/3-540-60299-2\_41}{Touraivane95} & \hyperref[auth:a309]{Toura{\"{\i}}vane} & Constraint Programming and Industrial Applications & \href{works/Touraivane95.pdf}{Yes} & \cite{Touraivane95} & 1995 & CP 1995 & 3 & 2 & 1 & \ref{b:Touraivane95} & \ref{c:Touraivane95}\\
\rowlabel{a:JourdanFRD94}JourdanFRD94 \href{}{JourdanFRD94} & \hyperref[auth:a707]{J. Jourdan}, \hyperref[auth:a708]{F. Fages}, \hyperref[auth:a709]{D. Rozzonelli}, \hyperref[auth:a710]{A. Demeure} & Data Alignment and Task Scheduling On Parallel Machines Using Concurrent Constraint Model-based Programming & No & \cite{JourdanFRD94} & 1994 & ILPS 1994 & 1 & 0 & 0 & No & \ref{c:JourdanFRD94}\\
\rowlabel{a:NuijtenA94}NuijtenA94 \href{}{NuijtenA94} & \hyperref[auth:a786]{W. P. M. Nuijten}, \hyperref[auth:a787]{Emile H. L. Aarts} & Constraint Satisfaction for Multiple Capacitated Job Shop Scheduling & \href{works/NuijtenA94.pdf}{Yes} & \cite{NuijtenA94} & 1994 & ECAI 1994 & 5 & 0 & 0 & \ref{b:NuijtenA94} & \ref{c:NuijtenA94}\\
\rowlabel{a:Wallace94}Wallace94 \href{}{Wallace94} & \hyperref[auth:a117]{M. Wallace} & Applying Constraints for Scheduling & No & \cite{Wallace94} & 1994 & Constraint Programming 1994 & 19 & 0 & 0 & No & \ref{c:Wallace94}\\
\rowlabel{a:BaptisteLV92}BaptisteLV92 \href{https://doi.org/10.1109/ROBOT.1992.220195}{BaptisteLV92} & \hyperref[auth:a703]{P. Baptiste}, \hyperref[auth:a704]{B. Legeard}, \hyperref[auth:a702]{C. Varnier} & Hoist scheduling problem: an approach based on constraint logic programming & \href{works/BaptisteLV92.pdf}{Yes} & \cite{BaptisteLV92} & 1992 & ICRA 1992 & 6 & 13 & 6 & \ref{b:BaptisteLV92} & \ref{c:BaptisteLV92}\\
\rowlabel{a:ErtlK91}ErtlK91 \href{https://doi.org/10.1007/3-540-54444-5\_89}{ErtlK91} & \hyperref[auth:a712]{M. Anton Ertl}, \hyperref[auth:a713]{A. Krall} & Optimal Instruction Scheduling using Constraint Logic Programming & \href{works/ErtlK91.pdf}{Yes} & \cite{ErtlK91} & 1991 & PLILP 1991 & 12 & 14 & 14 & \ref{b:ErtlK91} & \ref{c:ErtlK91}\\
\end{longtable}
}



\clearpage
\subsection{Extracted Concepts}
{\scriptsize
\begin{longtable}{>{\raggedright\arraybackslash}p{3cm}r>{\raggedright\arraybackslash}p{4cm}p{1.5cm}p{2cm}p{1.5cm}p{1.5cm}p{1.5cm}p{1.5cm}p{2cm}p{1.5cm}rr}
\rowcolor{white}\caption{Automatically Extracted PAPER Properties (Requires Local Copy)}\\ \toprule
\rowcolor{white}Work & Pages & Concepts & Classification & Constraints & \shortstack{Prog\\Languages} & \shortstack{CP\\Systems} & Areas & Industries & Benchmarks & Algorithm & a & c\\ \midrule\endhead
\bottomrule
\endfoot
\rowlabel{b:AalianPG23}\href{../works/AalianPG23.pdf}{AalianPG23}~\cite{AalianPG23} & 16 & scheduling, preempt, transportation, machine, make-span, activity, flow-shop, order, resource, preemptive &  & cycle, noOverlap, endBeforeStart, alwaysIn, cumulative &  & CPO, Cplex & steel cable & mining industry & real-world &  & \ref{a:AalianPG23} & \ref{c:AalianPG23}\\
\rowlabel{b:AbrilSB05}\href{../works/AbrilSB05.pdf}{AbrilSB05}~\cite{AbrilSB05} & 1 & distributed, multi-agent, scheduling, order &  &  &  &  & railway &  &  &  & \ref{a:AbrilSB05} & \ref{c:AbrilSB05}\\
\rowlabel{b:Acuna-AgostMFG09}\href{../works/Acuna-AgostMFG09.pdf}{Acuna-AgostMFG09}~\cite{Acuna-AgostMFG09} & 2 & re-scheduling, order, scheduling, transportation &  &  &  &  & railway &  & Roadef &  & \ref{a:Acuna-AgostMFG09} & \ref{c:Acuna-AgostMFG09}\\
\rowlabel{b:AkkerDH07}\href{../works/AkkerDH07.pdf}{AkkerDH07}~\cite{AkkerDH07} & 15 & due-date, cmax, machine, job, lateness, sequence dependent setup, preempt, resource, no-wait, scheduling, precedence, order, make-span, completion-time, release-date, preemptive & parallel machine, RCPSP, single machine & cumulative &  & Cplex &  &  &  &  & \ref{a:AkkerDH07} & \ref{c:AkkerDH07}\\
\rowlabel{b:AlesioNBG14}\href{../works/AlesioNBG14.pdf}{AlesioNBG14}~\cite{AlesioNBG14} & 18 & preempt, scheduling, completion-time, resource, task, job-shop, distributed, make-span, open-shop, order, job, activity, periodic, preemptive &  & alldifferent &  & OPL, Cplex & automotive &  & benchmark &  & \ref{a:AlesioNBG14} & \ref{c:AlesioNBG14}\\
\rowlabel{b:AmadiniGM16}\href{../works/AmadiniGM16.pdf}{AmadiniGM16}~\cite{AmadiniGM16} & 7 & make-span, lazy clause generation, scheduling, resource, task, distributed, precedence & RCPSP & cumulative &  & MiniZinc, Choco Solver, Gurobi, Gecode, OR-Tools &  &  & benchmark, real-life, github &  & \ref{a:AmadiniGM16} & \ref{c:AmadiniGM16}\\
\rowlabel{b:AngelsmarkJ00}\href{../works/AngelsmarkJ00.pdf}{AngelsmarkJ00}~\cite{AngelsmarkJ00} & 5 & resource, job, order, scheduling, task, job-shop &  &  &  &  &  &  &  &  & \ref{a:AngelsmarkJ00} & \ref{c:AngelsmarkJ00}\\
\rowlabel{b:AntunesABD18}\href{../works/AntunesABD18.pdf}{AntunesABD18}~\cite{AntunesABD18} & 8 & earliness, scheduling, machine, order, lateness, activity, due-date, re-scheduling, task, Benders Decomposition, Logic-Based Benders Decomposition, periodic, stochastic &  & bin-packing, BinPacking constraint &  & Cplex & workforce scheduling, maintenance scheduling & electricity industry & real-world, industry partner, industrial partner &  & \ref{a:AntunesABD18} & \ref{c:AntunesABD18}\\
\rowlabel{b:AntuoriHHEN20}\href{../works/AntuoriHHEN20.pdf}{AntuoriHHEN20}~\cite{AntuoriHHEN20} & 16 & due-date, task, job-shop, precedence, release-date, resource, job, order, completion-time, tardiness, scheduling, machine, periodic, stochastic &  & alldifferent, circuit, Element constraint, cycle, Channeling constraint &  & Choco Solver & torpedo &  & random instance, generated instance, gitlab, benchmark, industrial instance &  & \ref{a:AntuoriHHEN20} & \ref{c:AntuoriHHEN20}\\
\rowlabel{b:AntuoriHHEN21}\href{../works/AntuoriHHEN21.pdf}{AntuoriHHEN21}~\cite{AntuoriHHEN21} & 16 & transportation, due-date, task, job-shop, precedence, release-date, resource, job, order, tardiness, scheduling, machine, stochastic &  & cycle & C++, Java & Choco Solver, Gecode & automotive, car manufacturing, drone & automotive industry & gitlab, supplementary material & GRASP & \ref{a:AntuoriHHEN21} & \ref{c:AntuoriHHEN21}\\
\rowlabel{b:ArbaouiY18}\href{../works/ArbaouiY18.pdf}{ArbaouiY18}~\cite{ArbaouiY18} & 10 & order, sequence dependent setup, resource, job, scheduling, setup-time, machine, make-span, no-wait, completion-time, cmax & single machine, parallel machine & Pulse constraint, alternative constraint, noOverlap, cumulative & C++ & Cplex &  &  & benchmark &  & \ref{a:ArbaouiY18} & \ref{c:ArbaouiY18}\\
\rowlabel{b:ArmstrongGOS21}\href{../works/ArmstrongGOS21.pdf}{ArmstrongGOS21}~\cite{ArmstrongGOS21} & 18 & machine, flow-shop, job-shop, job, order, sequence dependent setup, cmax, transportation, scheduling, make-span, completion-time, preempt, resource, setup-time, precedence, task, preemptive & HFF, HFFTT, HFS & cycle, alternative constraint, table constraint, circuit, diffn, bin-packing, cumulative & Java, Prolog & Gecode, CHIP, MiniZinc, CPO, Chuffed, SICStus, Cplex & robot & packaging industry & instance generator, industry partner, zenodo, supplementary material, real-world, industrial partner, benchmark & energetic reasoning & \ref{a:ArmstrongGOS21} & \ref{c:ArmstrongGOS21}\\
\rowlabel{b:ArmstrongGOS22}\href{../works/ArmstrongGOS22.pdf}{ArmstrongGOS22}~\cite{ArmstrongGOS22} & 13 & machine, flow-shop, job, re-scheduling, order, cmax, no-wait, transportation, scheduling, make-span, completion-time, resource, task & HFF, parallel machine, HFFTT, HFS & noOverlap, cumulative & Prolog & OPL, SICStus &  &  & real-world, benchmark & IGT, GRASP, NEH & \ref{a:ArmstrongGOS22} & \ref{c:ArmstrongGOS22}\\
\rowlabel{b:AronssonBK09}\href{../works/AronssonBK09.pdf}{AronssonBK09}~\cite{AronssonBK09} & 13 & job-shop, transportation, order, job, task &  & cumulative & Prolog & CHIP, Cplex & railway &  & real-world, real-life & sweep & \ref{a:AronssonBK09} & \ref{c:AronssonBK09}\\
\rowlabel{b:ArtiguesBF04}\href{../works/ArtiguesBF04.pdf}{ArtiguesBF04}~\cite{ArtiguesBF04} & 13 & batch process, cmax, resource, completion-time, scheduling, machine, job, make-span, release-date, precedence, sequence dependent setup, job-shop, setup-time, preempt, order, one-machine scheduling, preemptive &  & Disjunctive constraint, disjunctive & C++ & Ilog Solver, Ilog Scheduler &  &  & benchmark & edge-finding & \ref{a:ArtiguesBF04} & \ref{c:ArtiguesBF04}\\
\rowlabel{b:ArtiguesHQT21}\href{../works/ArtiguesHQT21.pdf}{ArtiguesHQT21}~\cite{ArtiguesHQT21} & 8 & order, resource, preempt, scheduling, release-date, machine, job, preemptive & RCPSP & cumulative &  & Cplex &  &  &  &  & \ref{a:ArtiguesHQT21} & \ref{c:ArtiguesHQT21}\\
\rowlabel{b:ArtiouchineB05}\href{../works/ArtiouchineB05.pdf}{ArtiouchineB05}~\cite{ArtiouchineB05} & 15 & release-date, completion-time, job, resource, activity, open-shop, machine, job-shop, re-scheduling, scheduling, order, make-span, preempt, precedence, preemptive & parallel machine, single machine & Disjunctive constraint, cumulative, disjunctive &  & Ilog Scheduler & aircraft &  & generated instance, random instance & not-last, edge-finding, not-first & \ref{a:ArtiouchineB05} & \ref{c:ArtiouchineB05}\\
\rowlabel{b:Astrand0F21}\href{../works/Astrand0F21.pdf}{Astrand0F21}~\cite{Astrand0F21} & 18 & open-shop, task, precedence, make-span, order, job, activity, scheduling, resource, machine, job-shop &  & cycle, disjunctive, Disjunctive constraint &  & Gecode & farming, forestry, agriculture, drone, robot, satellite & potash industry, mining industry, mineral industry & benchmark, real-life, real-world, generated instance &  & \ref{a:Astrand0F21} & \ref{c:Astrand0F21}\\
\rowlabel{b:AstrandJZ18}\href{../works/AstrandJZ18.pdf}{AstrandJZ18}~\cite{AstrandJZ18} & 9 & task, make-span, order, activity, scheduling, resource, machine, periodic & single machine & disjunctive, cumulative, cycle &  & Gecode & hoist, robot & potash industry &  & time-tabling & \ref{a:AstrandJZ18} & \ref{c:AstrandJZ18}\\
\rowlabel{b:BadicaBIL19}\href{../works/BadicaBIL19.pdf}{BadicaBIL19}~\cite{BadicaBIL19} & 11 & completion-time, resource, distributed, order, activity, machine, multi-agent, make-span, scheduling &  & cycle, Arithmetic constraint &  & ECLiPSe, Gecode &  &  & github &  & \ref{a:BadicaBIL19} & \ref{c:BadicaBIL19}\\
\rowlabel{b:BajestaniB11}\href{../works/BajestaniB11.pdf}{BajestaniB11}~\cite{BajestaniB11} & 8 & re-scheduling, Benders Decomposition, scheduling, machine, transportation, order, tardiness, make-span, resource, inventory, due-date, job, Logic-Based Benders Decomposition, periodic, single-machine scheduling, stochastic & JSSP, single machine & cycle, Cardinality constraint, cumulative, circuit &  & Ilog Solver, Cplex & railway, maintenance scheduling, aircraft &  &  &  & \ref{a:BajestaniB11} & \ref{c:BajestaniB11}\\
\rowlabel{b:Baptiste09}\href{../works/Baptiste09.pdf}{Baptiste09}~\cite{Baptiste09} & 1 & scheduling &  &  &  &  &  &  &  &  & \ref{a:Baptiste09} & \ref{c:Baptiste09}\\
\rowlabel{b:BaptisteLV92}\href{../works/BaptisteLV92.pdf}{BaptisteLV92}~\cite{BaptisteLV92} & 6 &  &  &  &  &  &  &  &  &  & \ref{a:BaptisteLV92} & \ref{c:BaptisteLV92}\\
\rowlabel{b:BaptisteP97}\href{../works/BaptisteP97.pdf}{BaptisteP97}~\cite{BaptisteP97} & 15 & resource, preempt, job-shop, scheduling, re-scheduling, due-date, task, precedence, release-date, flow-shop, make-span, order, job, activity, preemptive & RCPSP & Disjunctive constraint, disjunctive, cumulative & C++ & Claire, CHIP &  &  & benchmark & edge-finding, edge-finder & \ref{a:BaptisteP97} & \ref{c:BaptisteP97}\\
\rowlabel{b:BarlattCG08}\href{../works/BarlattCG08.pdf}{BarlattCG08}~\cite{BarlattCG08} & 5 & scheduling, resource, setup-time, task, job-shop, transportation, job, machine, flow-shop &  &  &  &  & automotive, pipeline &  & real-world &  & \ref{a:BarlattCG08} & \ref{c:BarlattCG08}\\
\rowlabel{b:Bartak02}\href{../works/Bartak02.pdf}{Bartak02}~\cite{Bartak02} & 16 & make-span, machine, job, activity, resource, lateness, job-shop, precedence, earliness, scheduling, continuous-process, task, order &  & cumulative, disjunctive, Disjunctive constraint & Prolog & SICStus & dairies &  & real-life & edge-finding, time-tabling & \ref{a:Bartak02} & \ref{c:Bartak02}\\
\rowlabel{b:Bartak02a}\href{../works/Bartak02a.pdf}{Bartak02a}~\cite{Bartak02a} & 15 & activity, earliness, scheduling, make-span, task, machine, job, re-scheduling, job-shop, resource, precedence, order, tardiness &  & Disjunctive constraint, cumulative, disjunctive &  & Ilog Scheduler & dairies &  & benchmark, real-life & time-tabling, edge-finding & \ref{a:Bartak02a} & \ref{c:Bartak02a}\\
\rowlabel{b:BartakV15}\href{../works/BartakV15.pdf}{BartakV15}~\cite{BartakV15} & 12 & scheduling, make-span, machine, job, lateness, re-scheduling, job-shop, resource, precedence, order, activity, setup-time &  &  &  &  &  &  & real-world, real-life & sweep & \ref{a:BartakV15} & \ref{c:BartakV15}\\
\rowlabel{b:BartoliniBBLM14}\href{../works/BartoliniBBLM14.pdf}{BartoliniBBLM14}~\cite{BartoliniBBLM14} & 16 & tardiness, make-span, scheduling, resource, task, job, activity, machine &  & alternative constraint, cumulative &  &  & super-computer &  &  &  & \ref{a:BartoliniBBLM14} & \ref{c:BartoliniBBLM14}\\
\rowlabel{b:BarzegaranZP20}\href{../works/BarzegaranZP20.pdf}{BarzegaranZP20}~\cite{BarzegaranZP20} & 9 & resource, re-scheduling, distributed, machine, scheduling, order, task &  &  & Java & OR-Tools & automotive, robot &  &  &  & \ref{a:BarzegaranZP20} & \ref{c:BarzegaranZP20}\\
\rowlabel{b:Beck06}\href{../works/Beck06.pdf}{Beck06}~\cite{Beck06} & 10 & due-date, order, scheduling, machine, job-shop, tardiness, flow-shop, make-span, resource, job &  &  &  & Ilog Scheduler &  &  & benchmark &  & \ref{a:Beck06} & \ref{c:Beck06}\\
\rowlabel{b:BeckDF97}\href{../works/BeckDF97.pdf}{BeckDF97}~\cite{BeckDF97} & 15 & activity, release-date, make-span, resource, inventory, job-shop, precedence, due-date, re-scheduling, order, scheduling, machine, job, task & single machine & cycle, cumulative &  &  & robot &  & benchmark, real-world & edge-finding & \ref{a:BeckDF97} & \ref{c:BeckDF97}\\
\rowlabel{b:BeckPS03}\href{../works/BeckPS03.pdf}{BeckPS03}~\cite{BeckPS03} & 10 & job, task, activity, release-date, make-span, transportation, earliness, flow-time, resource, job-shop, precedence, due-date, re-scheduling, order, tardiness, scheduling, completion-time, machine, setup-time, stochastic & RCPSP &  &  & Ilog Scheduler & robot &  & benchmark, real-world &  & \ref{a:BeckPS03} & \ref{c:BeckPS03}\\
\rowlabel{b:BeckW04}\href{../works/BeckW04.pdf}{BeckW04}~\cite{BeckW04} & 5 & job-shop, machine, activity, distributed, flow-shop, resource, job, order, make-span, scheduling, one-machine scheduling, stochastic & single machine &  &  & Ilog Scheduler &  &  &  & edge-finding, time-tabling & \ref{a:BeckW04} & \ref{c:BeckW04}\\
\rowlabel{b:BeckW05}\href{../works/BeckW05.pdf}{BeckW05}~\cite{BeckW05} & 6 & job-shop, activity, flow-shop, resource, job, order, make-span, scheduling, stochastic &  & Balance constraint &  & Ilog Scheduler &  &  &  & edge-finder & \ref{a:BeckW05} & \ref{c:BeckW05}\\
\rowlabel{b:BehrensLM19}\href{../works/BehrensLM19.pdf}{BehrensLM19}~\cite{BehrensLM19} & 7 & order, resource, machine, scheduling, setup-time, task, distributed, multi-agent, make-span &  &  & Python & OR-Tools, MiniZinc & robot &  & github, real-world &  & \ref{a:BehrensLM19} & \ref{c:BehrensLM19}\\
\rowlabel{b:BeldiceanuC02}\href{../works/BeldiceanuC02.pdf}{BeldiceanuC02}~\cite{BeldiceanuC02} & 17 & task, resource, activity, order, producer/consumer, scheduling, machine & single machine & Cumulatives constraint, cumulative & Prolog & CHIP, SICStus & crew-scheduling &  & real-life, random instance, benchmark & sweep & \ref{a:BeldiceanuC02} & \ref{c:BeldiceanuC02}\\
\rowlabel{b:BeldiceanuCP08}\href{../works/BeldiceanuCP08.pdf}{BeldiceanuCP08}~\cite{BeldiceanuCP08} & 15 & scheduling, order, resource, task &  & disjunctive, geost, cumulative & Prolog & CHIP, SICStus, OPL & rectangle-packing, perfect-square &  & benchmark & edge-finding, sweep & \ref{a:BeldiceanuCP08} & \ref{c:BeldiceanuCP08}\\
\rowlabel{b:BeldiceanuP07}\href{../works/BeldiceanuP07.pdf}{BeldiceanuP07}~\cite{BeldiceanuP07} & 15 & preempt, task, resource, order, scheduling, release-date, due-date, preemptive &  & disjunctive, cumulative &  &  &  &  &  & sweep & \ref{a:BeldiceanuP07} & \ref{c:BeldiceanuP07}\\
\rowlabel{b:BenderWS21}\href{../works/BenderWS21.pdf}{BenderWS21}~\cite{BenderWS21} & 16 & activity, order, resource, scheduling, preempt, task, machine, make-span, job, distributed, setup-time, preemptive & RCPSP & noOverlap & Python &  & agriculture &  &  &  & \ref{a:BenderWS21} & \ref{c:BenderWS21}\\
\rowlabel{b:BenediktSMVH18}\href{../works/BenediktSMVH18.pdf}{BenediktSMVH18}~\cite{BenediktSMVH18} & 10 & job-shop, scheduling, order, preempt, resource, job, machine, single-machine scheduling & single machine, parallel machine & noOverlap &  & Gurobi & energy-price &  & github, random instance, generated instance &  & \ref{a:BenediktSMVH18} & \ref{c:BenediktSMVH18}\\
\rowlabel{b:BeniniBGM06}\href{../works/BeniniBGM06.pdf}{BeniniBGM06}~\cite{BeniniBGM06} & 15 & Benders Decomposition, task, distributed, precedence, make-span, order, activity, tardiness, scheduling, resource, setup-time, Logic-Based Benders Decomposition &  & cycle, cumulative &  & ECLiPSe, Cplex, Ilog Solver & automotive, pipeline &  & real-life &  & \ref{a:BeniniBGM06} & \ref{c:BeniniBGM06}\\
\rowlabel{b:BeniniLMR08}\href{../works/BeniniLMR08.pdf}{BeniniLMR08}~\cite{BeniniLMR08} & 15 & resource, Benders Decomposition, task, distributed, precedence, make-span, order, activity, machine, preempt, release-date, tardiness, scheduling, Logic-Based Benders Decomposition, periodic, preemptive & SCC & circuit &  & Ilog Scheduler, Cplex & medical, pipeline &  & benchmark &  & \ref{a:BeniniLMR08} & \ref{c:BeniniLMR08}\\
\rowlabel{b:BertholdHLMS10}\href{../works/BertholdHLMS10.pdf}{BertholdHLMS10}~\cite{BertholdHLMS10} & 5 & scheduling, order, preempt, precedence, completion-time, job, resource & psplib, RCPSP & disjunctive, cumulative &  & Cplex, SCIP, Z3 &  &  &  &  & \ref{a:BertholdHLMS10} & \ref{c:BertholdHLMS10}\\
\rowlabel{b:BessiereHMQW14}\href{../works/BessiereHMQW14.pdf}{BessiereHMQW14}~\cite{BessiereHMQW14} & 16 & scheduling, order, resource, setup-time, task, machine, job &  & BufferedResource, cycle, Cardinality constraint, alldifferent, Element constraint &  & Choco Solver & satellite & textile industry & benchmark, real-life &  & \ref{a:BessiereHMQW14} & \ref{c:BessiereHMQW14}\\
\rowlabel{b:BillautHL12}\href{../works/BillautHL12.pdf}{BillautHL12}~\cite{BillautHL12} & 15 & tardiness, job-shop, setup-time, due-date, open-shop, precedence, release-date, flow-shop, make-span, order, job, scheduling, completion-time, resource, machine, cmax, stochastic & single machine & cycle &  & Cplex, Mistral &  &  & random instance &  & \ref{a:BillautHL12} & \ref{c:BillautHL12}\\
\rowlabel{b:Bit-Monnot23}\href{../works/Bit-Monnot23.pdf}{Bit-Monnot23}~\cite{Bit-Monnot23} & 8 & distributed, job, open-shop, task, lazy clause generation, precedence, scheduling, machine, order, make-span, job-shop, resource, activity & OSP, Open Shop Scheduling Problem & Disjunctive constraint, cycle, cumulative, disjunctive &  & OR-Tools, CPO, MiniZinc, Mistral &  &  & benchmark, real-world, github &  & \ref{a:Bit-Monnot23} & \ref{c:Bit-Monnot23}\\
\rowlabel{b:BofillCSV17}\href{../works/BofillCSV17.pdf}{BofillCSV17}~\cite{BofillCSV17} & 9 & precedence, make-span, order, activity, machine, preempt, cmax, lazy clause generation, scheduling, resource, preemptive & RCPSP, psplib & cumulative &  & Z3, SCIP &  &  & benchmark & energetic reasoning & \ref{a:BofillCSV17} & \ref{c:BofillCSV17}\\
\rowlabel{b:BofillEGPSV14}\href{../works/BofillEGPSV14.pdf}{BofillEGPSV14}~\cite{BofillEGPSV14} & 16 & machine, order, scheduling, lazy clause generation, task &  & Channeling constraint &  & Cplex, Gecode, MiniZinc, SCIP &  &  & industrial instance & time-tabling & \ref{a:BofillEGPSV14} & \ref{c:BofillEGPSV14}\\
\rowlabel{b:BofillGSV15}\href{../works/BofillGSV15.pdf}{BofillGSV15}~\cite{BofillGSV15} & 9 & machine, scheduling, order &  & Channeling constraint, Cardinality constraint &  & Cplex &  &  & industrial instance & time-tabling & \ref{a:BofillGSV15} & \ref{c:BofillGSV15}\\
\rowlabel{b:BogaerdtW19}\href{../works/BogaerdtW19.pdf}{BogaerdtW19}~\cite{BogaerdtW19} & 16 & scheduling, completion-time, setup-time, job-shop, precedence, order, job, machine, tardiness, single-machine scheduling & single machine, parallel machine & noOverlap & C  & OPL, Cplex & railway &  & benchmark &  & \ref{a:BogaerdtW19} & \ref{c:BogaerdtW19}\\
\rowlabel{b:BonfiettiLBM11}\href{../works/BonfiettiLBM11.pdf}{BonfiettiLBM11}~\cite{BonfiettiLBM11} & 15 & scheduling, order, make-span, precedence, task, job, resource, activity, machine, job-shop, periodic & RCPSP & cumulative, cycle &  & Ilog Solver & hoist, robot &  & benchmark, generated instance, industrial instance &  & \ref{a:BonfiettiLBM11} & \ref{c:BonfiettiLBM11}\\
\rowlabel{b:BonfiettiLBM12}\href{../works/BonfiettiLBM12.pdf}{BonfiettiLBM12}~\cite{BonfiettiLBM12} & 16 & scheduling, order, make-span, precedence, job, resource, activity, distributed, machine, job-shop, periodic & RCPSP & cumulative, cycle &  & Ilog Solver & hoist, robot &  & benchmark & time-tabling & \ref{a:BonfiettiLBM12} & \ref{c:BonfiettiLBM12}\\
\rowlabel{b:BonfiettiLM13}\href{../works/BonfiettiLM13.pdf}{BonfiettiLM13}~\cite{BonfiettiLM13} & 5 & scheduling, make-span, job-shop, precedence, resource, activity, job, order, periodic & RCPSP & cycle, cumulative &  & Cplex &  &  &  &  & \ref{a:BonfiettiLM13} & \ref{c:BonfiettiLM13}\\
\rowlabel{b:BonfiettiLM14}\href{../works/BonfiettiLM14.pdf}{BonfiettiLM14}~\cite{BonfiettiLM14} & 16 & scheduling, machine, open-shop, distributed, make-span, task, job-shop, precedence, resource, activity, job, order, stochastic & RCPSP, psplib & cumulative &  &  &  &  & benchmark, real-world &  & \ref{a:BonfiettiLM14} & \ref{c:BonfiettiLM14}\\
\rowlabel{b:BonfiettiM12}\href{../works/BonfiettiM12.pdf}{BonfiettiM12}~\cite{BonfiettiM12} & 3 & job, task, scheduling, machine, precedence, job-shop, resource, activity, periodic & RCPSP & cumulative &  &  & hoist &  & industrial instance &  & \ref{a:BonfiettiM12} & \ref{c:BonfiettiM12}\\
\rowlabel{b:BonfiettiZLM16}\href{../works/BonfiettiZLM16.pdf}{BonfiettiZLM16}~\cite{BonfiettiZLM16} & 17 & resource, activity, scheduling, order, make-span, precedence, periodic & RCPSP & cumulative, cycle, disjunctive &  & OR-Tools & automotive & automotive industry, control system industry & generated instance, github, industrial instance, benchmark, real-world & sweep, edge-finder & \ref{a:BonfiettiZLM16} & \ref{c:BonfiettiZLM16}\\
\rowlabel{b:BonninMNE24}\href{../works/BonninMNE24.pdf}{BonninMNE24}~\cite{BonninMNE24} & 12 & open-shop, order, job, activity, flow-time, machine, preempt, precedence, release-date, flow-shop, make-span, scheduling, completion-time, resource, task, job-shop, preemptive, single-machine scheduling & single machine & noOverlap, Flowtime constraint, Completion constraint, disjunctive, cumulative, Disjunctive constraint & C++ & Cplex & patient, COVID, vaccine &  & benchmark, real-life & edge-finding, sweep, time-tabling & \ref{a:BonninMNE24} & \ref{c:BonninMNE24}\\
\rowlabel{b:BoothNB16}\href{../works/BoothNB16.pdf}{BoothNB16}~\cite{BoothNB16} & 17 & distributed, resource, machine, Benders Decomposition, precedence, order, activity, scheduling, task, re-scheduling, Logic-Based Benders Decomposition &  & cumulative, noOverlap, disjunctive & C++ & Cplex & robot, medical &  & real-world &  & \ref{a:BoothNB16} & \ref{c:BoothNB16}\\
\rowlabel{b:BoudreaultSLQ22}\href{../works/BoudreaultSLQ22.pdf}{BoudreaultSLQ22}~\cite{BoudreaultSLQ22} & 16 & activity, machine, transportation, distributed, lazy clause generation, order, make-span, scheduling, cmax, resource, preempt, precedence, task & RCPSP, psplib & disjunctive, Cumulatives constraint, Disjunctive constraint, cumulative &  & Chuffed, MiniZinc, OPL, OR-Tools & offshore & repair industry, ship repair industry & supplementary material, gitlab, benchmark, generated instance, real-life, industrial partner, github, real-world & edge-finding, not-first, not-last, energetic reasoning & \ref{a:BoudreaultSLQ22} & \ref{c:BoudreaultSLQ22}\\
\rowlabel{b:BridiLBBM16}\href{../works/BridiLBBM16.pdf}{BridiLBBM16}~\cite{BridiLBBM16} & 2 & task, distributed, make-span, order, job, activity, scheduling, resource, machine, periodic &  &  &  &  &  &  &  &  & \ref{a:BridiLBBM16} & \ref{c:BridiLBBM16}\\
\rowlabel{b:BrusoniCLMMT96}\href{../works/BrusoniCLMMT96.pdf}{BrusoniCLMMT96}~\cite{BrusoniCLMMT96} & 10 & no-wait, due-date, scheduling, order, resource, activity, precedence, task, distributed, job-shop, job &  & disjunctive, Disjunctive constraint & Prolog &  & railway, train schedule &  &  &  & \ref{a:BrusoniCLMMT96} & \ref{c:BrusoniCLMMT96}\\
\rowlabel{b:BurtLPS15}\href{../works/BurtLPS15.pdf}{BurtLPS15}~\cite{BurtLPS15} & 17 & task, job, job-shop, resource, machine, Benders Decomposition, precedence, order, tardiness, scheduling, make-span, completion-time, periodic, single-machine scheduling, stochastic & parallel machine, single machine & cumulative, cycle &  & Gurobi, Gecode, Cplex, MiniZinc &  &  & industry partner, real-world, benchmark &  & \ref{a:BurtLPS15} & \ref{c:BurtLPS15}\\
\rowlabel{b:CappartS17}\href{../works/CappartS17.pdf}{CappartS17}~\cite{CappartS17} & 16 & re-scheduling, resource, scheduling, task, machine, activity, job, precedence, job-shop, completion-time, order & TMS & cumulative, span constraint, noOverlap, alternative constraint &  & OPL & train schedule, railway &  & bitbucket, real-life, random instance &  & \ref{a:CappartS17} & \ref{c:CappartS17}\\
\rowlabel{b:CappartTSR18}\href{../works/CappartTSR18.pdf}{CappartTSR18}~\cite{CappartTSR18} & 17 & resource, setup-time, producer/consumer, activity, Benders Decomposition, scheduling, transportation, order, Logic-Based Benders Decomposition, periodic &  & cumulative, circuit, disjunctive, noOverlap &  & Cplex, CPO, MiniZinc, OPL & medical, patient &  & bitbucket, real-life, CSPlib &  & \ref{a:CappartTSR18} & \ref{c:CappartTSR18}\\
\rowlabel{b:CarchraeBF05}\href{../works/CarchraeBF05.pdf}{CarchraeBF05}~\cite{CarchraeBF05} & 1 & scheduling, task, make-span, order &  &  &  &  &  &  &  &  & \ref{a:CarchraeBF05} & \ref{c:CarchraeBF05}\\
\rowlabel{b:Caseau97}\href{../works/Caseau97.pdf}{Caseau97}~\cite{Caseau97} & 4 & preempt, order, scheduling, task, make-span, job, resource, job-shop, preemptive &  & cumulative &  &  & robot &  & benchmark & edge-finding & \ref{a:Caseau97} & \ref{c:Caseau97}\\
\rowlabel{b:CatusseCBL16}\href{../works/CatusseCBL16.pdf}{CatusseCBL16}~\cite{CatusseCBL16} & 7 & release-date, order, resource, due-date, scheduling, machine, job, task, stochastic & parallel machine, single machine & disjunctive & Julia & OPL &  &  &  &  & \ref{a:CatusseCBL16} & \ref{c:CatusseCBL16}\\
\rowlabel{b:CauwelaertDMS16}\href{../works/CauwelaertDMS16.pdf}{CauwelaertDMS16}~\cite{CauwelaertDMS16} & 16 & batch process, order, make-span, scheduling, completion-time, setup-time, resource, preempt, precedence, task, job, job-shop, activity, machine, sequence dependent setup, preemptive &  & cumulative, disjunctive & Java &  & container terminal &  & real-life, bitbucket, benchmark & not-last, edge-finding, not-first & \ref{a:CauwelaertDMS16} & \ref{c:CauwelaertDMS16}\\
\rowlabel{b:CestaOS98}\href{../works/CestaOS98.pdf}{CestaOS98}~\cite{CestaOS98} & 1 & job, resource, scheduling &  &  &  &  & robot &  &  &  & \ref{a:CestaOS98} & \ref{c:CestaOS98}\\
\rowlabel{b:ChapadosJR11}\href{../works/ChapadosJR11.pdf}{ChapadosJR11}~\cite{ChapadosJR11} & 6 & activity, task, scheduling, order &  & cycle, cumulative &  & OPL &  & retail industry &  & time-tabling & \ref{a:ChapadosJR11} & \ref{c:ChapadosJR11}\\
\rowlabel{b:ChuGNSW13}\href{../works/ChuGNSW13.pdf}{ChuGNSW13}~\cite{ChuGNSW13} & 7 & distributed, resource, machine, job, scheduling, precedence, order, task &  & cumulative, alldifferent, Cardinality constraint, disjunctive &  & CHIP &  &  &  & not-first, not-last, edge-finding & \ref{a:ChuGNSW13} & \ref{c:ChuGNSW13}\\
\rowlabel{b:ChuX05}\href{../works/ChuX05.pdf}{ChuX05}~\cite{ChuX05} & 15 & scheduling, machine, release-date, order, completion-time, resource, job, due-date, Benders Decomposition, Logic-Based Benders Decomposition, one-machine scheduling, single-machine scheduling & single machine & disjunctive, cumulative, Disjunctive constraint &  & ECLiPSe &  &  &  &  & \ref{a:ChuX05} & \ref{c:ChuX05}\\
\rowlabel{b:CireCH13}\href{../works/CireCH13.pdf}{CireCH13}~\cite{CireCH13} & 7 & tardiness, scheduling, Benders Decomposition, precedence, task, order, make-span, machine, job, resource, Logic-Based Benders Decomposition, stochastic &  & circuit, cumulative &  & SCIP, OPL, Cplex &  &  &  &  & \ref{a:CireCH13} & \ref{c:CireCH13}\\
\rowlabel{b:ClercqPBJ11}\href{../works/ClercqPBJ11.pdf}{ClercqPBJ11}~\cite{ClercqPBJ11} & 16 & order, activity, release-date, scheduling, completion-time, resource, due-date, distributed, precedence &  & cumulative, SoftCumulative, Cumulatives constraint, alldifferent, SoftCumulativeSum, Cardinality constraint & Java & Choco Solver, CHIP &  &  & benchmark & time-tabling, sweep, energetic reasoning, edge-finding & \ref{a:ClercqPBJ11} & \ref{c:ClercqPBJ11}\\
\rowlabel{b:CobanH10}\href{../works/CobanH10.pdf}{CobanH10}~\cite{CobanH10} & 5 & job, make-span, distributed, tardiness, Benders Decomposition, preempt, re-scheduling, order, scheduling, Logic-Based Benders Decomposition, preemptive &  & disjunctive, circuit &  & OPL, Cplex &  &  &  &  & \ref{a:CobanH10} & \ref{c:CobanH10}\\
\rowlabel{b:CohenHB17}\href{../works/CohenHB17.pdf}{CohenHB17}~\cite{CohenHB17} & 17 & machine, order, activity, scheduling, task &  & noOverlap, alternative constraint &  & Cplex, OPL &  &  &  & time-tabling & \ref{a:CohenHB17} & \ref{c:CohenHB17}\\
\rowlabel{b:ColT19}\href{../works/ColT19.pdf}{ColT19}~\cite{ColT19} & 17 & scheduling, machine, job-shop, earliness, order, precedence, make-span, resource, job & JSSP & noOverlap, disjunctive & Java & OR-Tools, MiniZinc, CPO &  &  & github, benchmark, real-world &  & \ref{a:ColT19} & \ref{c:ColT19}\\
\rowlabel{b:Colombani96}\href{../works/Colombani96.pdf}{Colombani96}~\cite{Colombani96} & 15 & job, scheduling, resource, preempt, due-date, job-shop, task, order, activity, machine, precedence, release-date, stochastic &  & disjunctive &  & CHIP &  &  &  &  & \ref{a:Colombani96} & \ref{c:Colombani96}\\
\rowlabel{b:DannaP03}\href{../works/DannaP03.pdf}{DannaP03}~\cite{DannaP03} & 5 & job-shop, order, tardiness, scheduling, machine, job, activity, earliness, resource &  & disjunctive &  & Cplex, Ilog Solver, Ilog Scheduler &  &  & benchmark &  & \ref{a:DannaP03} & \ref{c:DannaP03}\\
\rowlabel{b:Davenport10}\href{../works/Davenport10.pdf}{Davenport10}~\cite{Davenport10} & 5 & order, resource, release-date, tardiness, scheduling, completion-time, earliness, due-date, periodic &  &  &  & Cplex & semiconductor, maintenance scheduling &  &  &  & \ref{a:Davenport10} & \ref{c:Davenport10}\\
\rowlabel{b:DavenportKRSH07}\href{../works/DavenportKRSH07.pdf}{DavenportKRSH07}~\cite{DavenportKRSH07} & 13 & make to order, activity, machine, preempt, precedence, job-shop, sequence dependent setup, resource, inventory, order, scheduling, job, setup-time &  & disjunctive, bin-packing & C++ & Cplex, CHIP &  & steel industry &  &  & \ref{a:DavenportKRSH07} & \ref{c:DavenportKRSH07}\\
\rowlabel{b:DejemeppeCS15}\href{../works/DejemeppeCS15.pdf}{DejemeppeCS15}~\cite{DejemeppeCS15} & 16 & make-span, task, precedence, setup-time, resource, preempt, activity, completion-time, tardiness, job-shop, sequence dependent setup, scheduling, release-date, machine, job, order, preemptive & single machine & disjunctive, cumulative, cycle &  &  & container terminal &  & bitbucket, real-world, generated instance, benchmark & not-last, not-first, edge-finding & \ref{a:DejemeppeCS15} & \ref{c:DejemeppeCS15}\\
\rowlabel{b:DejemeppeD14}\href{../works/DejemeppeD14.pdf}{DejemeppeD14}~\cite{DejemeppeD14} & 9 & make-span, precedence, job-shop, resource, activity, setup-time, job, scheduling, order &  & cumulative &  &  & medical, patient &  & bitbucket &  & \ref{a:DejemeppeD14} & \ref{c:DejemeppeD14}\\
\rowlabel{b:DemirovicS18}\href{../works/DemirovicS18.pdf}{DemirovicS18}~\cite{DemirovicS18} & 18 & scheduling, task, precedence, order, resource, activity &  & Disjunctive constraint, cumulative, disjunctive &  & MiniZinc, Gurobi &  &  & benchmark, real-world & time-tabling & \ref{a:DemirovicS18} & \ref{c:DemirovicS18}\\
\rowlabel{b:DerrienP14}\href{../works/DerrienP14.pdf}{DerrienP14}~\cite{DerrienP14} & 9 & resource, scheduling, make-span, activity, order & psplib, CuSP & cumulative & Java & Choco Solver &  &  & random instance & sweep, edge-finding, energetic reasoning & \ref{a:DerrienP14} & \ref{c:DerrienP14}\\
\rowlabel{b:DerrienPZ14}\href{../works/DerrienPZ14.pdf}{DerrienPZ14}~\cite{DerrienPZ14} & 9 & re-scheduling, order, job, activity, machine, precedence, make-span, scheduling, resource & RCPSP, CuSP & cumulative, Balance constraint, Cumulatives constraint &  & Choco Solver, CHIP &  &  & real-world, benchmark, random instance & sweep & \ref{a:DerrienPZ14} & \ref{c:DerrienPZ14}\\
\rowlabel{b:DilkinaDH05}\href{../works/DilkinaDH05.pdf}{DilkinaDH05}~\cite{DilkinaDH05} & 5 & machine, precedence, make-span, job, scheduling, job-shop, order, stochastic &  &  &  & OPL &  &  &  &  & \ref{a:DilkinaDH05} & \ref{c:DilkinaDH05}\\
\rowlabel{b:DoomsH08}\href{../works/DoomsH08.pdf}{DoomsH08}~\cite{DoomsH08} & 16 & scheduling, completion-time, machine, job, activity, resource, job-shop, task, order, online scheduling, stochastic & RCPSP &  &  &  &  & service industry &  &  & \ref{a:DoomsH08} & \ref{c:DoomsH08}\\
\rowlabel{b:DoulabiRP14}\href{../works/DoulabiRP14.pdf}{DoulabiRP14}~\cite{DoulabiRP14} & 9 & due-date, task, order, activity, scheduling, resource &  & Cardinality constraint, bin-packing, Element constraint &  & Cplex & medical, patient, nurse, surgery, operating room &  &  &  & \ref{a:DoulabiRP14} & \ref{c:DoulabiRP14}\\
\rowlabel{b:EdisO11}\href{../works/EdisO11.pdf}{EdisO11}~\cite{EdisO11} & 7 & task, job, resource, make-span, scheduling, flow-time, tardiness, due-date, machine, completion-time, activity, lateness, earliness, Benders Decomposition, preempt, Logic-Based Benders Decomposition & parallel machine & bin-packing, noOverlap, cumulative &  & OPL, Cplex &  &  &  &  & \ref{a:EdisO11} & \ref{c:EdisO11}\\
\rowlabel{b:EfthymiouY23}\href{../works/EfthymiouY23.pdf}{EfthymiouY23}~\cite{EfthymiouY23} & 16 & setup-time, order, make-span, job-shop, job, re-scheduling, task, scheduling, machine & CHSP, JSSP & cumulative, disjunctive, cycle & Python & OPL, OR-Tools & pipeline, hoist, satellite, electroplating &  & generated instance, benchmark, random instance, real-life, industrial instance &  & \ref{a:EfthymiouY23} & \ref{c:EfthymiouY23}\\
\rowlabel{b:ElkhyariGJ02}\href{../works/ElkhyariGJ02.pdf}{ElkhyariGJ02}~\cite{ElkhyariGJ02} & 6 & precedence, scheduling, machine, preempt, make-span, resource, activity, due-date, re-scheduling, task & RCPSP & cumulative, disjunctive, table constraint &  &  &  &  &  &  & \ref{a:ElkhyariGJ02} & \ref{c:ElkhyariGJ02}\\
\rowlabel{b:ElkhyariGJ02a}\href{../works/ElkhyariGJ02a.pdf}{ElkhyariGJ02a}~\cite{ElkhyariGJ02a} & 24 & activity, re-scheduling, order, scheduling, open-shop, due-date, task, precedence, resource, online scheduling & RCPSP, psplib & cumulative, Disjunctive constraint, Arithmetic constraint, disjunctive &  & OPL &  &  & benchmark, real-life & time-tabling & \ref{a:ElkhyariGJ02a} & \ref{c:ElkhyariGJ02a}\\
\rowlabel{b:ErtlK91}\href{../works/ErtlK91.pdf}{ErtlK91}~\cite{ErtlK91} & 12 & setup-time, task, resource, scheduling, order, machine &  & cycle & Prolog &  & pipeline &  & real-world, benchmark &  & \ref{a:ErtlK91} & \ref{c:ErtlK91}\\
\rowlabel{b:EvenSH15}\href{../works/EvenSH15.pdf}{EvenSH15}~\cite{EvenSH15} & 18 & transportation, machine, distributed, resource, preempt, order, scheduling, Benders Decomposition, completion-time, task, preemptive &  & cumulative, disjunctive, Disjunctive constraint &  & OPL, Choco Solver & emergency service &  & real-life, real-world & sweep & \ref{a:EvenSH15} & \ref{c:EvenSH15}\\
\rowlabel{b:FocacciLN00}\href{../works/FocacciLN00.pdf}{FocacciLN00}~\cite{FocacciLN00} & 10 & machine, preempt, cmax, scheduling, resource, setup-time, due-date, task, job-shop, distributed, precedence, make-span, sequence dependent setup, open-shop, order, job, activity &  & Disjunctive constraint, disjunctive &  &  &  &  & real-world & edge-finding & \ref{a:FocacciLN00} & \ref{c:FocacciLN00}\\
\rowlabel{b:FontaineMH16}\href{../works/FontaineMH16.pdf}{FontaineMH16}~\cite{FontaineMH16} & 11 & order, job-shop, resource, scheduling, machine, job, task, completion-time, Benders Decomposition, make-span, precedence, Logic-Based Benders Decomposition & parallel machine & disjunctive &  & MiniZinc, Gurobi, CHIP &  &  & benchmark &  & \ref{a:FontaineMH16} & \ref{c:FontaineMH16}\\
\rowlabel{b:FortinZDF05}\href{../works/FortinZDF05.pdf}{FortinZDF05}~\cite{FortinZDF05} & 15 & resource, task, order, activity, precedence, temporal constraint reasoning, make-span, scheduling, stochastic & psplib &  &  &  &  &  &  &  & \ref{a:FortinZDF05} & \ref{c:FortinZDF05}\\
\rowlabel{b:FrankK05}\href{../works/FrankK05.pdf}{FrankK05}~\cite{FrankK05} & 18 & order, job, resource, precedence, scheduling, due-date, task, periodic, stochastic &  & cycle &  &  & satellite, aircraft &  & benchmark &  & \ref{a:FrankK05} & \ref{c:FrankK05}\\
\rowlabel{b:FrimodigS19}\href{../works/FrimodigS19.pdf}{FrimodigS19}~\cite{FrimodigS19} & 17 & order, machine, job, scheduling, resource, Benders Decomposition, task, job-shop, stochastic &  & cumulative, bin-packing, regular expression, Regular constraint & Python & Cplex, MiniZinc, Gecode & medical, patient, nurse, physician, radiation therapy, surgery &  & benchmark, real-world &  & \ref{a:FrimodigS19} & \ref{c:FrimodigS19}\\
\rowlabel{b:FrohnerTR19}\href{../works/FrohnerTR19.pdf}{FrohnerTR19}~\cite{FrohnerTR19} & 9 & order, scheduling, distributed &  &  & Java, Python & MiniZinc, Gecode, Gurobi & nurse &  & benchmark, real-world &  & \ref{a:FrohnerTR19} & \ref{c:FrohnerTR19}\\
\rowlabel{b:FrostD98}\href{../works/FrostD98.pdf}{FrostD98}~\cite{FrostD98} & 1 & scheduling, order &  &  &  &  & maintenance scheduling & power industry &  &  & \ref{a:FrostD98} & \ref{c:FrostD98}\\
\rowlabel{b:GalleguillosKSB19}\href{../works/GalleguillosKSB19.pdf}{GalleguillosKSB19}~\cite{GalleguillosKSB19} & 18 & resource, order, job, activity, make-span, re-scheduling, machine, distributed, scheduling, stochastic & JSSP & alternative constraint, cumulative & Python & OR-Tools & datacenter, super-computer &  &  &  & \ref{a:GalleguillosKSB19} & \ref{c:GalleguillosKSB19}\\
\rowlabel{b:GarganiR07}\href{../works/GarganiR07.pdf}{GarganiR07}~\cite{GarganiR07} & 13 & machine, inventory, order, resource &  & bin-packing, Channeling constraint, Element constraint & C++ & OPL & steel mill & steel industry & real-life, CSPlib &  & \ref{a:GarganiR07} & \ref{c:GarganiR07}\\
\rowlabel{b:GayHLS15}\href{../works/GayHLS15.pdf}{GayHLS15}~\cite{GayHLS15} & 9 & resource, scheduling, precedence, task, order, make-span, activity & RCPSP, OSP, psplib & cumulative, disjunctive &  &  &  &  & bitbucket, benchmark & time-tabling, edge-finding & \ref{a:GayHLS15} & \ref{c:GayHLS15}\\
\rowlabel{b:GayHS15}\href{../works/GayHS15.pdf}{GayHS15}~\cite{GayHS15} & 9 & resource, task, order, scheduling, precedence, preempt, preemptive &  & Cumulatives constraint, cumulative, table constraint, disjunctive &  & Choco Solver, OR-Tools, Gecode &  &  & bitbucket & time-tabling, sweep & \ref{a:GayHS15} & \ref{c:GayHS15}\\
\rowlabel{b:GayHS15a}\href{../works/GayHS15a.pdf}{GayHS15a}~\cite{GayHS15a} & 16 & task, order, machine, manpower, preempt, resource, scheduling, preemptive & psplib, RCPSP & Cumulatives constraint, cumulative, disjunctive & Java &  &  &  & benchmark, real-world, bitbucket & time-tabling, not-first, not-last, energetic reasoning, edge-finding, sweep & \ref{a:GayHS15a} & \ref{c:GayHS15a}\\
\rowlabel{b:GaySS14}\href{../works/GaySS14.pdf}{GaySS14}~\cite{GaySS14} & 15 & machine, completion-time, activity, setup-time, continuous-process, resource, job, order, make-span, scheduling, precedence, manpower, job-shop &  & cycle, cumulative, disjunctive &  &  & steel mill &  & real-life, CSPlib & sweep & \ref{a:GaySS14} & \ref{c:GaySS14}\\
\rowlabel{b:GeibingerKKMMW21}\href{../works/GeibingerKKMMW21.pdf}{GeibingerKKMMW21}~\cite{GeibingerKKMMW21} & 10 & scheduling, distributed &  & Cardinality constraint &  & MiniZinc, OR-Tools, Gurobi, Cplex, Gecode & nurse, physician, COVID, medical, patient & pharmaceutical industry & real-world &  & \ref{a:GeibingerKKMMW21} & \ref{c:GeibingerKKMMW21}\\
\rowlabel{b:GeibingerMM19}\href{../works/GeibingerMM19.pdf}{GeibingerMM19}~\cite{GeibingerMM19} & 16 & precedence, release-date, resource, activity, re-scheduling, job, order, completion-time, scheduling, due-date, make-span, task & RCPSP & alternative constraint, cumulative, endBeforeStart, Pulse constraint, noOverlap & Java & Cplex, Gecode, MiniZinc, CPO & automotive &  & real-world, benchmark, real-life, generated instance, industrial partner & time-tabling & \ref{a:GeibingerMM19} & \ref{c:GeibingerMM19}\\
\rowlabel{b:GeibingerMM21}\href{../works/GeibingerMM21.pdf}{GeibingerMM21}~\cite{GeibingerMM21} & 9 & precedence, release-date, resource, activity, job, order, completion-time, tardiness, scheduling, machine, lazy clause generation, due-date, task & RCPSP & disjunctive, cumulative &  & Chuffed, Cplex, CPO & nurse, train schedule, operating room &  & github, real-world, benchmark, real-life, generated instance & time-tabling & \ref{a:GeibingerMM21} & \ref{c:GeibingerMM21}\\
\rowlabel{b:GeitzGSSW22}\href{../works/GeitzGSSW22.pdf}{GeitzGSSW22}~\cite{GeitzGSSW22} & 18 & setup-time, sequence dependent setup, task, lateness, precedence, batch process, make-span, order, job, scheduling, completion-time, resource, machine, preempt, producer/consumer, lazy clause generation, job-shop, transportation, preemptive & single machine, RCPSP, JSSP & cumulative &  & OPL & robot &  & real-world, real-life, github & sweep, not-last & \ref{a:GeitzGSSW22} & \ref{c:GeitzGSSW22}\\
\rowlabel{b:GelainPRVW17}\href{../works/GelainPRVW17.pdf}{GelainPRVW17}~\cite{GelainPRVW17} & 16 & order, resource, scheduling &  &  &  &  &  &  & real-life, CSPlib, benchmark &  & \ref{a:GelainPRVW17} & \ref{c:GelainPRVW17}\\
\rowlabel{b:Geske05}\href{../works/Geske05.pdf}{Geske05}~\cite{Geske05} & 18 & machine, re-scheduling, activity, distributed, task, job, order, resource, scheduling, lateness, job-shop &  & cumulative & Prolog & SICStus, CHIP & train schedule, railway & railway industry & real-life &  & \ref{a:Geske05} & \ref{c:Geske05}\\
\rowlabel{b:GilesH16}\href{../works/GilesH16.pdf}{GilesH16}~\cite{GilesH16} & 16 & setup-time, activity, transportation, resource, inventory, task, order, scheduling &  & disjunctive, cumulative &  & Cplex & pipeline & chemical industry, processing industry, petro-chemical industry, chemical processing industry &  &  & \ref{a:GilesH16} & \ref{c:GilesH16}\\
\rowlabel{b:GingrasQ16}\href{../works/GingrasQ16.pdf}{GingrasQ16}~\cite{GingrasQ16} & 7 & resource, scheduling, task, make-span, completion-time, precedence, order & psplib, RCPSP, CuSP & disjunctive, cumulative &  & Choco Solver &  &  & benchmark & energetic reasoning, sweep, edge-finder, edge-finding & \ref{a:GingrasQ16} & \ref{c:GingrasQ16}\\
\rowlabel{b:GodardLN05}\href{../works/GodardLN05.pdf}{GodardLN05}~\cite{GodardLN05} & 9 & job-shop, activity, completion-time, order, earliness, tardiness, resource, scheduling, machine, make-span, job, precedence & JSSP & cumulative, disjunctive, table constraint &  & Ilog Solver, Ilog Scheduler &  &  & benchmark &  & \ref{a:GodardLN05} & \ref{c:GodardLN05}\\
\rowlabel{b:GodetLHS20}\href{../works/GodetLHS20.pdf}{GodetLHS20}~\cite{GodetLHS20} & 8 & lazy clause generation, release-date, scheduling, task, machine, make-span, completion-time, setup-time, order, cmax, resource, job & single machine, parallel machine, PMSP & alldifferent, bin-packing, Disjunctive constraint, cumulative, disjunctive &  & CHIP, Chuffed, Choco Solver & satellite &  & real-life, benchmark, generated instance, github & not-last, time-tabling & \ref{a:GodetLHS20} & \ref{c:GodetLHS20}\\
\rowlabel{b:GokGSTO20}\href{../works/GokGSTO20.pdf}{GokGSTO20}~\cite{GokGSTO20} & 17 & distributed, task, job-shop, resource, multi-agent, job, setup-time, scheduling, precedence, order, tardiness, activity, stochastic & RCPSP & cumulative, circuit, disjunctive & Python & Gecode, Z3, MiniZinc, Gurobi & aircraft &  & real-world, Roadef & GRASP & \ref{a:GokGSTO20} & \ref{c:GokGSTO20}\\
\rowlabel{b:GoldwaserS17}\href{../works/GoldwaserS17.pdf}{GoldwaserS17}~\cite{GoldwaserS17} & 16 & scheduling, machine, transportation, order, resource, due-date, lazy clause generation, Benders Decomposition, Logic-Based Benders Decomposition &  & cumulative, disjunctive & Python & Gurobi, Gecode & torpedo & steel industry & github, generated instance, instance generator &  & \ref{a:GoldwaserS17} & \ref{c:GoldwaserS17}\\
\rowlabel{b:Goltz95}\href{../works/Goltz95.pdf}{Goltz95}~\cite{Goltz95} & 14 & task, job, order, resource, scheduling, precedence, job-shop, due-date, machine, completion-time &  & cumulative, disjunctive & Prolog & CHIP &  &  & benchmark & edge-finding & \ref{a:Goltz95} & \ref{c:Goltz95}\\
\rowlabel{b:GomesHS06}\href{../works/GomesHS06.pdf}{GomesHS06}~\cite{GomesHS06} & 2 & order, scheduling, distributed, task, multi-agent &  &  &  & Ilog Solver &  &  & real-life &  & \ref{a:GomesHS06} & \ref{c:GomesHS06}\\
\rowlabel{b:GrimesH10}\href{../works/GrimesH10.pdf}{GrimesH10}~\cite{GrimesH10} & 15 & cmax, machine, job, job-shop, setup-time, flow-shop, no-wait, open-shop, scheduling, precedence, order, make-span, sequence dependent setup, task, batch process, resource & Open Shop Scheduling Problem & cycle, disjunctive, Disjunctive constraint, cumulative &  &  &  & steel industry & benchmark & time-tabling, edge-finding & \ref{a:GrimesH10} & \ref{c:GrimesH10}\\
\rowlabel{b:GrimesH11}\href{../works/GrimesH11.pdf}{GrimesH11}~\cite{GrimesH11} & 17 & cmax, machine, job, job-shop, flow-shop, no-wait, open-shop, scheduling, precedence, order, make-span, completion-time, tardiness, release-date, earliness, lazy clause generation, task, due-date, resource & RCPSP & disjunctive, Disjunctive constraint, cumulative &  & Cplex, Ilog Solver, OPL, Ilog Scheduler &  &  & benchmark & edge-finding & \ref{a:GrimesH11} & \ref{c:GrimesH11}\\
\rowlabel{b:GrimesHM09}\href{../works/GrimesHM09.pdf}{GrimesHM09}~\cite{GrimesHM09} & 9 & open-shop, order, make-span, resource, job, precedence, scheduling, task, job-shop, machine & OSP, Open Shop Scheduling Problem & Balance constraint, disjunctive, Disjunctive constraint & Java & Ilog Scheduler, Choco Solver, Mistral &  &  & benchmark & edge-finding, not-last & \ref{a:GrimesHM09} & \ref{c:GrimesHM09}\\
\rowlabel{b:GroleazNS20}\href{../works/GroleazNS20.pdf}{GroleazNS20}~\cite{GroleazNS20} & 17 & precedence, release-date, job, scheduling, resource, machine, preempt, due-date, tardiness, job-shop, setup-time, order, inventory, preemptive & GCSP & circuit, noOverlap, cycle, cumulative &  & OR-Tools, CPO &  & food industry & industrial instance, benchmark &  & \ref{a:GroleazNS20} & \ref{c:GroleazNS20}\\
\rowlabel{b:GroleazNS20a}\href{../works/GroleazNS20a.pdf}{GroleazNS20a}~\cite{GroleazNS20a} & 9 & scheduling, machine, transportation, order, tardiness, release-date, precedence, resource, setup-time, preempt, inventory, due-date, distributed, job, preemptive & parallel machine, RCPSP & noOverlap, cumulative, cycle &  & Cplex, CPO &  & food industry & industrial partner, benchmark & GRASP & \ref{a:GroleazNS20a} & \ref{c:GroleazNS20a}\\
\rowlabel{b:GruianK98}\href{../works/GruianK98.pdf}{GruianK98}~\cite{GruianK98} & 8 & task, resource, re-scheduling, scheduling, order, activity &  & cumulative, cycle, circuit, diffn &  & OPL, CHIP & pipeline, aircraft &  & benchmark &  & \ref{a:GruianK98} & \ref{c:GruianK98}\\
\rowlabel{b:GuSS13}\href{../works/GuSS13.pdf}{GuSS13}~\cite{GuSS13} & 7 & lazy clause generation, activity, order, precedence, make-span, resource, distributed, scheduling, machine, single-machine scheduling & single machine & cumulative &  &  &  &  & benchmark & edge-finding, edge-finder, time-tabling & \ref{a:GuSS13} & \ref{c:GuSS13}\\
\rowlabel{b:GuSW12}\href{../works/GuSW12.pdf}{GuSW12}~\cite{GuSW12} & 15 & lazy clause generation, activity, order, precedence, make-span, resource, job, preempt, scheduling, cmax, preemptive &  & cumulative & C++ &  &  &  & benchmark &  & \ref{a:GuSW12} & \ref{c:GuSW12}\\
\rowlabel{b:HanenKP21}\href{../works/HanenKP21.pdf}{HanenKP21}~\cite{HanenKP21} & 17 & job-shop, resource, machine, precedence, order, tardiness, preempt, release-date, scheduling, make-span, completion-time, task, cmax, job, lateness, due-date, preemptive & RCPSP, CuSP, parallel machine & cumulative & Python & Claire & pipeline &  & Roadef, generated instance, random instance & energetic reasoning & \ref{a:HanenKP21} & \ref{c:HanenKP21}\\
\rowlabel{b:He0GLW18}\href{../works/He0GLW18.pdf}{He0GLW18}~\cite{He0GLW18} & 18 & machine, transportation, multi-agent, distributed, precedence, re-scheduling, order, scheduling &  &  & Python & Gurobi & energy-price, real-time pricing &  & real-world, bitbucket &  & \ref{a:He0GLW18} & \ref{c:He0GLW18}\\
\rowlabel{b:HebrardALLCMR22}\href{../works/HebrardALLCMR22.pdf}{HebrardALLCMR22}~\cite{HebrardALLCMR22} & 7 & order, scheduling, activity &  & cumulative & Julia & Claire & deep space &  &  & sweep & \ref{a:HebrardALLCMR22} & \ref{c:HebrardALLCMR22}\\
\rowlabel{b:HebrardTW05}\href{../works/HebrardTW05.pdf}{HebrardTW05}~\cite{HebrardTW05} & 1 & job-shop, order, job, machine, scheduling &  &  &  &  &  &  &  &  & \ref{a:HebrardTW05} & \ref{c:HebrardTW05}\\
\rowlabel{b:HechingH16}\href{../works/HechingH16.pdf}{HechingH16}~\cite{HechingH16} & 11 & order, scheduling, manpower, re-scheduling, job, Benders Decomposition, task, Logic-Based Benders Decomposition, stochastic &  & circuit, noOverlap &  & OPL, Cplex & patient, medical &  & real-world &  & \ref{a:HechingH16} & \ref{c:HechingH16}\\
\rowlabel{b:HeinzB12}\href{../works/HeinzB12.pdf}{HeinzB12}~\cite{HeinzB12} & 17 & precedence, due-date, order, tardiness, scheduling, completion-time, machine, job, activity, release-date, earliness, resource, Benders Decomposition, Logic-Based Benders Decomposition, single-machine scheduling & single machine & cumulative, Channeling constraint, cycle, alternative constraint, IloAlternative &  & SCIP, Ilog Solver, OPL, Cplex, Ilog Scheduler &  &  &  & GRASP & \ref{a:HeinzB12} & \ref{c:HeinzB12}\\
\rowlabel{b:HeinzKB13}\href{../works/HeinzKB13.pdf}{HeinzKB13}~\cite{HeinzKB13} & 16 & release-date, job-shop, resource, machine, job, scheduling, Benders Decomposition, order, tardiness, Logic-Based Benders Decomposition & single machine & cumulative, Channeling constraint &  & SCIP, Cplex, OPL &  &  &  &  & \ref{a:HeinzKB13} & \ref{c:HeinzKB13}\\
\rowlabel{b:HeinzS11}\href{../works/HeinzS11.pdf}{HeinzS11}~\cite{HeinzS11} & 10 & preempt, order, scheduling, completion-time, machine, job, resource & psplib, RCPSP & disjunctive, cumulative &  & SCIP, Cplex &  &  & benchmark & time-tabling, energetic reasoning & \ref{a:HeinzS11} & \ref{c:HeinzS11}\\
\rowlabel{b:HentenryckM04}\href{../works/HentenryckM04.pdf}{HentenryckM04}~\cite{HentenryckM04} & 16 & resource, activity, job, completion-time, tardiness, scheduling, machine, open-shop, order, due-date, make-span, task, job-shop, precedence &  & disjunctive, cumulative, cycle &  &  &  &  & benchmark &  & \ref{a:HentenryckM04} & \ref{c:HentenryckM04}\\
\rowlabel{b:HentenryckM08}\href{../works/HentenryckM08.pdf}{HentenryckM08}~\cite{HentenryckM08} & 5 & order &  & bin-packing &  &  & steel mill &  & CSPlib &  & \ref{a:HentenryckM08} & \ref{c:HentenryckM08}\\
\rowlabel{b:HermenierDL11}\href{../works/HermenierDL11.pdf}{HermenierDL11}~\cite{HermenierDL11} & 15 & task, precedence, distributed, resource, completion-time, producer/consumer, machine, no-wait, order, scheduling, periodic &  & bin-packing, disjunctive, table constraint, alldifferent, cumulative, cycle &  & Choco Solver & datacenter &  &  &  & \ref{a:HermenierDL11} & \ref{c:HermenierDL11}\\
\rowlabel{b:HillTV21}\href{../works/HillTV21.pdf}{HillTV21}~\cite{HillTV21} & 19 & machine, job, activity, resource, release-date, precedence, preempt, lazy clause generation, scheduling, flow-shop, task, order, make-span, preemptive, single-machine scheduling & RCPSP, psplib, single machine & cycle, cumulative, alternative constraint &  &  &  &  & real-world &  & \ref{a:HillTV21} & \ref{c:HillTV21}\\
\rowlabel{b:HoYCLLCLC18}\href{../works/HoYCLLCLC18.pdf}{HoYCLLCLC18}~\cite{HoYCLLCLC18} & 6 & task, distributed, order, job, scheduling, resource, machine, re-scheduling, stochastic &  &  & C  &  & medical, patient, nurse &  & real-world &  & \ref{a:HoYCLLCLC18} & \ref{c:HoYCLLCLC18}\\
\rowlabel{b:HoeveGSL07}\href{../works/HoeveGSL07.pdf}{HoeveGSL07}~\cite{HoeveGSL07} & 6 & resource, multi-agent, scheduling, re-scheduling, job, precedence, distributed, task, job-shop, machine, order &  & disjunctive &  & Ilog Scheduler, Cplex &  &  & benchmark & edge-finding & \ref{a:HoeveGSL07} & \ref{c:HoeveGSL07}\\
\rowlabel{b:Hooker04}\href{../works/Hooker04.pdf}{Hooker04}~\cite{Hooker04} & 12 & machine, task, release-date, make-span, distributed, resource, precedence, order, tardiness, scheduling, Benders Decomposition, Logic-Based Benders Decomposition &  & disjunctive, cumulative, circuit &  & OPL, Ilog Scheduler, Cplex &  &  & random instance &  & \ref{a:Hooker04} & \ref{c:Hooker04}\\
\rowlabel{b:Hooker05a}\href{../works/Hooker05a.pdf}{Hooker05a}~\cite{Hooker05a} & 14 & release-date, scheduling, make-span, task, machine, job, due-date, resource, Benders Decomposition, precedence, order, tardiness, Logic-Based Benders Decomposition &  & circuit, cumulative, disjunctive &  & Ilog Scheduler, OPL, Cplex &  &  &  &  & \ref{a:Hooker05a} & \ref{c:Hooker05a}\\
\rowlabel{b:Hooker17}\href{../works/Hooker17.pdf}{Hooker17}~\cite{Hooker17} & 14 & job, resource, due-date, order, tardiness, scheduling &  & circuit &  &  &  &  & benchmark, random instance &  & \ref{a:Hooker17} & \ref{c:Hooker17}\\
\rowlabel{b:HookerY02}\href{../works/HookerY02.pdf}{HookerY02}~\cite{HookerY02} & 5 & scheduling, machine, job, resource, Benders Decomposition, order, Logic-Based Benders Decomposition & RCPSP & cumulative, disjunctive &  &  &  &  &  &  & \ref{a:HookerY02} & \ref{c:HookerY02}\\
\rowlabel{b:HoundjiSWD14}\href{../works/HoundjiSWD14.pdf}{HoundjiSWD14}~\cite{HoundjiSWD14} & 16 & scheduling, machine, transportation, order, precedence, resource, inventory, due-date & single machine & circuit, Cardinality constraint, Element constraint, GCC constraint &  &  &  &  & bitbucket, generated instance &  & \ref{a:HoundjiSWD14} & \ref{c:HoundjiSWD14}\\
\rowlabel{b:IfrimOS12}\href{../works/IfrimOS12.pdf}{IfrimOS12}~\cite{IfrimOS12} & 16 & order, scheduling, task, machine, job, re-scheduling, distributed, due-date, resource, periodic, stochastic &  & disjunctive &  &  & datacenter, energy-price &  & real-life &  & \ref{a:IfrimOS12} & \ref{c:IfrimOS12}\\
\rowlabel{b:JelinekB16}\href{../works/JelinekB16.pdf}{JelinekB16}~\cite{JelinekB16} & 10 & completion-time, order, scheduling, task &  & cumulative, table constraint & Prolog & SICStus, OPL &  &  & real-life &  & \ref{a:JelinekB16} & \ref{c:JelinekB16}\\
\rowlabel{b:JungblutK22}\href{../works/JungblutK22.pdf}{JungblutK22}~\cite{JungblutK22} & 4 & distributed, machine, make-span, scheduling, resource, preempt, task, order &  & circuit &  & MiniZinc &  &  & benchmark, github, real-world &  & \ref{a:JungblutK22} & \ref{c:JungblutK22}\\
\rowlabel{b:JuvinHHL23}\href{../works/JuvinHHL23.pdf}{JuvinHHL23}~\cite{JuvinHHL23} & 16 & resource, job, scheduling, task, job-shop, due-date, machine, make-span, flow-shop, completion-time, precedence, Benders Decomposition, cmax, setup-time, order, preempt, Logic-Based Benders Decomposition, preemptive & JSSP, parallel machine & disjunctive, Disjunctive constraint, PreemptiveNoOverlap, alldifferent, noOverlap, endBeforeStart, AllDiffPrec constraint, cumulative & C++ & CPO, Mistral &  &  & github, benchmark, supplementary material & not-last, edge-finding, not-first & \ref{a:JuvinHHL23} & \ref{c:JuvinHHL23}\\
\rowlabel{b:JuvinHL23}\href{../works/JuvinHL23.pdf}{JuvinHL23}~\cite{JuvinHL23} & 16 & precedence, order, tardiness, setup-time, scheduling, make-span, completion-time, task, cmax, machine, job, job-shop, flow-shop, stochastic &  & noOverlap, endBeforeStart &  & Cplex, CPO &  &  & real-world &  & \ref{a:JuvinHL23} & \ref{c:JuvinHL23}\\
\rowlabel{b:KamarainenS02}\href{../works/KamarainenS02.pdf}{KamarainenS02}~\cite{KamarainenS02} & 17 & job-shop, resource, earliness, activity, job, order, scheduling, machine, precedence, transportation, preempt, preemptive & KRFP &  &  & ECLiPSe &  &  & real-world, benchmark &  & \ref{a:KamarainenS02} & \ref{c:KamarainenS02}\\
\rowlabel{b:KameugneFGOQ18}\href{../works/KameugneFGOQ18.pdf}{KameugneFGOQ18}~\cite{KameugneFGOQ18} & 17 & cmax, precedence, make-span, completion-time, resource, task, scheduling, order & RCPSP, CuSP & Disjunctive constraint, cumulative, disjunctive & Java & CHIP, Choco Solver &  &  & real-world, benchmark & time-tabling, sweep, not-last, energetic reasoning, not-first & \ref{a:KameugneFGOQ18} & \ref{c:KameugneFGOQ18}\\
\rowlabel{b:KameugneFND23}\href{../works/KameugneFND23.pdf}{KameugneFND23}~\cite{KameugneFND23} & 17 & precedence, cmax, preempt, make-span, task, completion-time, machine, resource, order, scheduling, lazy clause generation & RCPSP, psplib, CuSP & Disjunctive constraint, disjunctive, Cumulatives constraint, cumulative & Java & Choco Solver, CHIP &  &  & benchmark & sweep, energetic reasoning, not-last, not-first, edge-finder, time-tabling, edge-finding & \ref{a:KameugneFND23} & \ref{c:KameugneFND23}\\
\rowlabel{b:KameugneFSN11}\href{../works/KameugneFSN11.pdf}{KameugneFSN11}~\cite{KameugneFSN11} & 15 & completion-time, job-shop, release-date, resource, job, order, scheduling, precedence, preempt, make-span, task, preemptive & RCPSP, psplib, CuSP & cumulative, disjunctive &  & Gecode &  &  & benchmark & edge-finding, not-last, not-first, time-tabling & \ref{a:KameugneFSN11} & \ref{c:KameugneFSN11}\\
\rowlabel{b:KelarevaTK13}\href{../works/KelarevaTK13.pdf}{KelarevaTK13}~\cite{KelarevaTK13} & 17 & re-scheduling, task, Benders Decomposition, precedence, scheduling, transportation, setup-time, order, tardiness, make-span, resource, activity, lazy clause generation, inventory & Liner Shipping Fleet Repositioning Problem, BPCTOP, LSFRP, Bulk Port Cargo Throughput Optimisation Problem & alldifferent &  & Cplex, SCIP, MiniZinc & earth observation, shipping line, satellite &  & real-world &  & \ref{a:KelarevaTK13} & \ref{c:KelarevaTK13}\\
\rowlabel{b:KeriK07}\href{../works/KeriK07.pdf}{KeriK07}~\cite{KeriK07} & 14 & due-date, activity, earliness, resource, tardiness, job, temporal constraint reasoning, order, make-span, scheduling, precedence, cmax, job-shop & RCPSP & cycle & C++ &  &  &  &  & edge-finding & \ref{a:KeriK07} & \ref{c:KeriK07}\\
\rowlabel{b:KhemmoudjPB06}\href{../works/KhemmoudjPB06.pdf}{KhemmoudjPB06}~\cite{KhemmoudjPB06} & 13 & distributed, resource, stock level, order, scheduling &  & cycle, cumulative & C++ & CHIP &  &  & real-world &  & \ref{a:KhemmoudjPB06} & \ref{c:KhemmoudjPB06}\\
\rowlabel{b:KimCMLLP23}\href{../works/KimCMLLP23.pdf}{KimCMLLP23}~\cite{KimCMLLP23} & 16 & open-shop, tardiness, earliness, scheduling, transportation, machine, make-span, job, precedence, distributed, setup-time, job-shop, due-date, order & parallel machine, SCC & noOverlap & Python & OR-Tools, Gurobi &  & steel industry & real-world, zenodo, benchmark &  & \ref{a:KimCMLLP23} & \ref{c:KimCMLLP23}\\
\rowlabel{b:KlankeBYE21}\href{../works/KlankeBYE21.pdf}{KlankeBYE21}~\cite{KlankeBYE21} & 16 & make-span, order, job, activity, scheduling, completion-time, resource, machine, producer/consumer, job-shop, re-scheduling, due-date, task, batch process &  & circuit, noOverlap, disjunctive, cumulative & Python & CHIP, OR-Tools, Gurobi, Cplex &  & processing industry, food-processing industry & random instance, benchmark, real-life &  & \ref{a:KlankeBYE21} & \ref{c:KlankeBYE21}\\
\rowlabel{b:KletzanderM17}\href{../works/KletzanderM17.pdf}{KletzanderM17}~\cite{KletzanderM17} & 15 & machine, resource, order, scheduling, transportation & parallel machine &  &  &  & torpedo & steel industry &  &  & \ref{a:KletzanderM17} & \ref{c:KletzanderM17}\\
\rowlabel{b:KorbaaYG99}\href{../works/KorbaaYG99.pdf}{KorbaaYG99}~\cite{KorbaaYG99} & 8 & resource, scheduling, transportation, make-span, job, task, job-shop, machine, flow-shop, order, periodic &  & circuit, cycle & Prolog & Ilog Solver, CHIP, OZ & robot, hoist &  &  &  & \ref{a:KorbaaYG99} & \ref{c:KorbaaYG99}\\
\rowlabel{b:KoschB14}\href{../works/KoschB14.pdf}{KoschB14}~\cite{KoschB14} & 16 & resource, lateness, job-shop, release-date, multi-agent, cmax, scheduling, Benders Decomposition, completion-time, batch process, due-date, order, make-span, machine, job, distributed, Logic-Based Benders Decomposition & RCPSP, single machine & cumulative, disjunctive, bin-packing & Java & Choco Solver, Cplex & semiconductor &  & benchmark &  & \ref{a:KoschB14} & \ref{c:KoschB14}\\
\rowlabel{b:KovacsB07}\href{../works/KovacsB07.pdf}{KovacsB07}~\cite{KovacsB07} & 15 & order, tardiness, activity, preempt, release-date, earliness, scheduling, make-span, completion-time, job, due-date, job-shop, flow-shop, resource, machine, preemptive, single-machine scheduling & parallel machine, single machine & Completion constraint, cumulative & C++ & Ilog Solver &  &  & benchmark &  & \ref{a:KovacsB07} & \ref{c:KovacsB07}\\
\rowlabel{b:KovacsEKV05}\href{../works/KovacsEKV05.pdf}{KovacsEKV05}~\cite{KovacsEKV05} & 1 & scheduling, resource, setup-time, job-shop, precedence, job &  &  &  &  &  &  & real-life &  & \ref{a:KovacsEKV05} & \ref{c:KovacsEKV05}\\
\rowlabel{b:KovacsTKSG21}\href{../works/KovacsTKSG21.pdf}{KovacsTKSG21}~\cite{KovacsTKSG21} & 17 & precedence, job-shop, preempt, order, tardiness, inventory, distributed, resource, due-date, scheduling, machine, flow-shop, job, re-scheduling, task, release-date, preemptive & RCPSP, single machine & cumulative &  & Gurobi, OR-Tools, Cplex &  &  & github, supplementary material, real-world, benchmark &  & \ref{a:KovacsTKSG21} & \ref{c:KovacsTKSG21}\\
\rowlabel{b:KovacsV04}\href{../works/KovacsV04.pdf}{KovacsV04}~\cite{KovacsV04} & 15 & scheduling, make-span, task, job, job-shop, resource, machine, precedence, order & single machine & disjunctive, cumulative &  & Ilog Scheduler &  &  & industrial partner, benchmark, real-life & edge-finding & \ref{a:KovacsV04} & \ref{c:KovacsV04}\\
\rowlabel{b:KovacsV06}\href{../works/KovacsV06.pdf}{KovacsV06}~\cite{KovacsV06} & 13 & tardiness, setup-time, earliness, scheduling, make-span, task, job, job-shop, resource, machine, precedence, order & single machine, RCPSP & cumulative &  & Ilog Scheduler & automotive & energy industry & industrial partner, benchmark, generated instance &  & \ref{a:KovacsV06} & \ref{c:KovacsV06}\\
\rowlabel{b:KreterSS15}\href{../works/KreterSS15.pdf}{KreterSS15}~\cite{KreterSS15} & 17 & order, preempt, resource, lazy clause generation, scheduling, task, machine, activity, make-span, completion-time, periodic, preemptive & RCPSP, parallel machine & cumulative, diffn, Element constraint, Calendar constraint &  & Cplex, MiniZinc, CHIP, Chuffed &  &  & benchmark &  & \ref{a:KreterSS15} & \ref{c:KreterSS15}\\
\rowlabel{b:KrogtLPHJ07}\href{../works/KrogtLPHJ07.pdf}{KrogtLPHJ07}~\cite{KrogtLPHJ07} & 13 & resource, due-date, job-shop, precedence, order, job, inventory, activity, machine, scheduling &  & circuit & Prolog & OPL & semiconductor, aircraft & semiconductor industry & real-world &  & \ref{a:KrogtLPHJ07} & \ref{c:KrogtLPHJ07}\\
\rowlabel{b:KucukY19}\href{../works/KucukY19.pdf}{KucukY19}~\cite{KucukY19} & 5 & distributed, resource, sequence dependent setup, task, order, scheduling, setup-time, stochastic &  & disjunctive, noOverlap, cycle &  & Cplex & earth observation, satellite &  & benchmark, generated instance & time-tabling & \ref{a:KucukY19} & \ref{c:KucukY19}\\
\rowlabel{b:Kumar03}\href{../works/Kumar03.pdf}{Kumar03}~\cite{Kumar03} & 15 & order, scheduling, producer/consumer, activity, resource &  & cycle &  &  &  &  &  & max-flow, bi-partite matching & \ref{a:Kumar03} & \ref{c:Kumar03}\\
\rowlabel{b:Laborie09}\href{../works/Laborie09.pdf}{Laborie09}~\cite{Laborie09} & 15 & task, machine, job, sequence dependent setup, inventory, due-date, job-shop, preempt, resource, precedence, order, tardiness, activity, setup-time, release-date, earliness, scheduling, preemptive &  & noOverlap, endBeforeStart, cumulative, disjunctive, alternative constraint & C  & CPO, OPL & satellite, aircraft &  & real-world, benchmark &  & \ref{a:Laborie09} & \ref{c:Laborie09}\\
\rowlabel{b:Laborie18a}\href{../works/Laborie18a.pdf}{Laborie18a}~\cite{Laborie18a} & 9 & resource, job, release-date, scheduling, task, due-date, machine, precedence, Benders Decomposition, Logic-Based Benders Decomposition &  & cumulative, alternative constraint &  & Ilog Scheduler, CPO, OPL &  &  & real-world, real-life, benchmark & energetic reasoning & \ref{a:Laborie18a} & \ref{c:Laborie18a}\\
\rowlabel{b:LacknerMMWW21}\href{../works/LacknerMMWW21.pdf}{LacknerMMWW21}~\cite{LacknerMMWW21} & 18 & release-date, flow-shop, job, order, tardiness, scheduling, machine, lateness, earliness, batch process, setup-time, due-date, make-span, task & OSP, single machine, parallel machine & cumulative, endBeforeStart, noOverlap, Element constraint &  & Chuffed, Cplex, OPL, CPO, MiniZinc, Gurobi, OR-Tools & semiconductor, oven scheduling & manufacturing industry, electronics industry, steel industry & benchmark, instance generator, real-life, random instance, industrial partner, supplementary material & GRASP & \ref{a:LacknerMMWW21} & \ref{c:LacknerMMWW21}\\
\rowlabel{b:LahimerLH11}\href{../works/LahimerLH11.pdf}{LahimerLH11}~\cite{LahimerLH11} & 14 & resource, machine, preempt, cmax, task, precedence, make-span, order, job, scheduling, completion-time, preemptive & parallel machine, RCPSP & Disjunctive constraint, disjunctive & C++ & Ilog Scheduler &  &  & benchmark & energetic reasoning & \ref{a:LahimerLH11} & \ref{c:LahimerLH11}\\
\rowlabel{b:LauLN08}\href{../works/LauLN08.pdf}{LauLN08}~\cite{LauLN08} & 5 & job, order, resource, scheduling, transportation, job-shop, machine, distributed, inventory, flow-shop &  &  &  &  &  &  & real-world, benchmark &  & \ref{a:LauLN08} & \ref{c:LauLN08}\\
\rowlabel{b:LetortBC12}\href{../works/LetortBC12.pdf}{LetortBC12}~\cite{LetortBC12} & 16 & task, machine, make-span, precedence, order, resource, scheduling & psplib & Cumulatives constraint, cumulative, geost, bin-packing & Java, Prolog & Choco Solver, CHIP, SICStus & datacenter &  & Roadef, benchmark, random instance & sweep, edge-finding & \ref{a:LetortBC12} & \ref{c:LetortBC12}\\
\rowlabel{b:LetortCB13}\href{../works/LetortCB13.pdf}{LetortCB13}~\cite{LetortCB13} & 16 & machine, make-span, precedence, order, resource, scheduling, task & psplib, RCPSP & Disjunctive constraint, cumulative, disjunctive, bin-packing & Java, Prolog & Choco Solver, SICStus &  &  & Roadef, benchmark, random instance & energetic reasoning, sweep, edge-finding & \ref{a:LetortCB13} & \ref{c:LetortCB13}\\
\rowlabel{b:LiFJZLL22}\href{../works/LiFJZLL22.pdf}{LiFJZLL22}~\cite{LiFJZLL22} & 6 & completion-time, task, tardiness, buffer-capacity, flow-time, blocking constraint, distributed, job-shop, batch process, flow-shop, transportation, machine, job, setup-time, no-wait, scheduling, order, make-span, stochastic & single machine & Blocking constraint &  & OPL & robot &  & benchmark &  & \ref{a:LiFJZLL22} & \ref{c:LiFJZLL22}\\
\rowlabel{b:LimBTBB15}\href{../works/LimBTBB15.pdf}{LimBTBB15}~\cite{LimBTBB15} & 15 & scheduling, order, tardiness, earliness, job-shop, multi-agent, machine, job, re-scheduling, stochastic &  &  &  & OPL & HVAC &  & benchmark & time-tabling & \ref{a:LimBTBB15} & \ref{c:LimBTBB15}\\
\rowlabel{b:LimHTB16}\href{../works/LimHTB16.pdf}{LimHTB16}~\cite{LimHTB16} & 18 & machine, activity, multi-agent, distributed, re-scheduling, order, scheduling, online scheduling, stochastic &  & cumulative &  & OPL & HVAC, energy-price, real-time pricing &  & real-world &  & \ref{a:LimHTB16} & \ref{c:LimHTB16}\\
\rowlabel{b:LimRX04}\href{../works/LimRX04.pdf}{LimRX04}~\cite{LimRX04} & 5 & scheduling, machine, preempt, completion-time, transportation, job, order, preemptive, stochastic &  &  &  &  & container terminal &  & generated instance &  & \ref{a:LimRX04} & \ref{c:LimRX04}\\
\rowlabel{b:Limtanyakul07}\href{../works/Limtanyakul07.pdf}{Limtanyakul07}~\cite{Limtanyakul07} & 6 & make-span, task, release-date, machine, resource, job, order, scheduling, due-date, precedence &  & cumulative &  & OPL & robot & automobile industry & real-life & energetic reasoning & \ref{a:Limtanyakul07} & \ref{c:Limtanyakul07}\\
\rowlabel{b:LipovetzkyBPS14}\href{../works/LipovetzkyBPS14.pdf}{LipovetzkyBPS14}~\cite{LipovetzkyBPS14} & 9 & make-span, scheduling, resource, precedence, Benders Decomposition, task, order, transportation &  & disjunctive &  & Cplex & crew-scheduling &  & real-life, real-world, industrial partner, industry partner, benchmark, generated instance &  & \ref{a:LipovetzkyBPS14} & \ref{c:LipovetzkyBPS14}\\
\rowlabel{b:LiuCGM17}\href{../works/LiuCGM17.pdf}{LiuCGM17}~\cite{LiuCGM17} & 17 & order, scheduling, machine, task, activity, transportation, cmax &  & Element constraint & Python & OR-Tools, OPL, MiniZinc &  & tourism industry & github &  & \ref{a:LiuCGM17} & \ref{c:LiuCGM17}\\
\rowlabel{b:LiuJ06}\href{../works/LiuJ06.pdf}{LiuJ06}~\cite{LiuJ06} & 5 & make-span, resource, task, order, scheduling &  & disjunctive, Disjunctive constraint, cycle &  &  &  &  &  &  & \ref{a:LiuJ06} & \ref{c:LiuJ06}\\
\rowlabel{b:LiuLH19}\href{../works/LiuLH19.pdf}{LiuLH19}~\cite{LiuLH19} & 9 & order, resource, scheduling &  & Channeling constraint &  & Choco Solver &  &  & benchmark, CSPlib & time-tabling & \ref{a:LiuLH19} & \ref{c:LiuLH19}\\
\rowlabel{b:LombardiBM15}\href{../works/LombardiBM15.pdf}{LombardiBM15}~\cite{LombardiBM15} & 16 & task, completion-time, precedence, scheduling, machine, order, make-span, job-shop, resource, activity, distributed, job, stochastic & JSSP, RCPSP, psplib &  &  &  &  &  & benchmark, real-world &  & \ref{a:LombardiBM15} & \ref{c:LombardiBM15}\\
\rowlabel{b:LombardiBMB11}\href{../works/LombardiBMB11.pdf}{LombardiBMB11}~\cite{LombardiBMB11} & 17 & order, make-span, task, precedence, resource, activity, completion-time, scheduling, machine, periodic, stochastic & RCPSP & cycle, cumulative & C++ &  & hoist &  & benchmark, industrial instance, real-life &  & \ref{a:LombardiBMB11} & \ref{c:LombardiBMB11}\\
\rowlabel{b:LombardiM09}\href{../works/LombardiM09.pdf}{LombardiM09}~\cite{LombardiM09} & 15 & precedence, make-span, order, activity, scheduling, resource, preempt, completion-time, task, preemptive, stochastic & RCPSP & Balance constraint &  & Ilog Solver &  &  & instance generator, real-world &  & \ref{a:LombardiM09} & \ref{c:LombardiM09}\\
\rowlabel{b:LombardiM10}\href{../works/LombardiM10.pdf}{LombardiM10}~\cite{LombardiM10} & 15 & precedence, make-span, order, activity, scheduling, resource, completion-time, task, stochastic & RCPSP & Disjunctive constraint, disjunctive, cumulative &  & Ilog Solver &  &  & real-world, benchmark &  & \ref{a:LombardiM10} & \ref{c:LombardiM10}\\
\rowlabel{b:LombardiM13}\href{../works/LombardiM13.pdf}{LombardiM13}~\cite{LombardiM13} & 2 & precedence, make-span, order, activity, scheduling, resource, task & RCPSP, psplib &  &  &  &  &  &  &  & \ref{a:LombardiM13} & \ref{c:LombardiM13}\\
\rowlabel{b:LouieVNB14}\href{../works/LouieVNB14.pdf}{LouieVNB14}~\cite{LouieVNB14} & 7 & order, resource, job, scheduling, task, machine, activity, periodic &  & cycle &  & OPL & patient, robot &  &  &  & \ref{a:LouieVNB14} & \ref{c:LouieVNB14}\\
\rowlabel{b:LuoB22}\href{../works/LuoB22.pdf}{LuoB22}~\cite{LuoB22} & 17 & order, scheduling, re-scheduling, job, Benders Decomposition, resource, machine, batch process, job-shop &  & AlwaysConstant, bin-packing, diffn, Element constraint, cumulative, alwaysIn & Python & CHIP, Cplex & super-computer, rectangle-packing, railway & metal industry, forging industry & real-life, industry partner, real-world, generated instance, github, industrial instance &  & \ref{a:LuoB22} & \ref{c:LuoB22}\\
\rowlabel{b:LuoVLBM16}\href{../works/LuoVLBM16.pdf}{LuoVLBM16}~\cite{LuoVLBM16} & 4 & task, job, job-shop, resource, machine, precedence, order, activity, scheduling &  &  &  &  & nurse &  &  & time-tabling & \ref{a:LuoVLBM16} & \ref{c:LuoVLBM16}\\
\rowlabel{b:Madi-WambaB16}\href{../works/Madi-WambaB16.pdf}{Madi-WambaB16}~\cite{Madi-WambaB16} & 16 & precedence, task, resource, job, order, scheduling &  & cumulative, TaskIntersection constraint & Java & Choco Solver, CHIP &  &  & real-world, benchmark, random instance, generated instance &  & \ref{a:Madi-WambaB16} & \ref{c:Madi-WambaB16}\\
\rowlabel{b:Madi-WambaLOBM17}\href{../works/Madi-WambaLOBM17.pdf}{Madi-WambaLOBM17}~\cite{Madi-WambaLOBM17} & 8 & job, distributed, scheduling, order, machine, task, re-scheduling, activity, precedence, resource &  & bin-packing, cumulative, Cumulatives constraint, Element constraint & Prolog & SICStus & datacenter &  & real-world & sweep & \ref{a:Madi-WambaLOBM17} & \ref{c:Madi-WambaLOBM17}\\
\rowlabel{b:MakMS10}\href{../works/MakMS10.pdf}{MakMS10}~\cite{MakMS10} & 5 & inventory, task, job, resource, scheduling, due-date, order, machine, activity, transportation, precedence &  & cycle &  &  &  &  &  &  & \ref{a:MakMS10} & \ref{c:MakMS10}\\
\rowlabel{b:MalapertCGJLR13}\href{../works/MalapertCGJLR13.pdf}{MalapertCGJLR13}~\cite{MalapertCGJLR13} & 2 & flow-shop, order, make-span, scheduling, cmax, open-shop, resource, preempt, precedence, task, job, job-shop, machine, preemptive & single machine, Open Shop Scheduling Problem & disjunctive, cumulative & Java & Choco Solver &  &  & benchmark, real-life &  & \ref{a:MalapertCGJLR13} & \ref{c:MalapertCGJLR13}\\
\rowlabel{b:MalapertN19}\href{../works/MalapertN19.pdf}{MalapertN19}~\cite{MalapertN19} & 17 & sequence dependent setup, order, job, flow-time, machine, cmax, make-span, scheduling, completion-time, resource, setup-time, task, single-machine scheduling & PMSP, PTC, parallel machine, single machine & noOverlap, cumulative, alternative constraint, alwaysIn &  & Cplex, CPO & semiconductor &  & benchmark, generated instance, industrial instance, Roadef &  & \ref{a:MalapertN19} & \ref{c:MalapertN19}\\
\rowlabel{b:MaraveliasG04}\href{../works/MaraveliasG04.pdf}{MaraveliasG04}~\cite{MaraveliasG04} & 20 &  &  &  &  & OZ &  &  &  &  & \ref{a:MaraveliasG04} & \ref{c:MaraveliasG04}\\
\rowlabel{b:Mehdizadeh-Somarin23}\href{../works/Mehdizadeh-Somarin23.pdf}{Mehdizadeh-Somarin23}~\cite{Mehdizadeh-Somarin23} & 14 & make-span, preempt, multi-agent, completion-time, tardiness, scheduling, cmax, job, setup-time, precedence, order, job-shop, re-scheduling, machine, flow-shop, task, online scheduling, periodic, preemptive, single-machine scheduling, stochastic & JSSP, parallel machine, single machine &  & Python & Cplex & COVID, robot &  & random instance &  & \ref{a:Mehdizadeh-Somarin23} & \ref{c:Mehdizadeh-Somarin23}\\
\rowlabel{b:MelgarejoLS15}\href{../works/MelgarejoLS15.pdf}{MelgarejoLS15}~\cite{MelgarejoLS15} & 17 & tardiness, scheduling, machine, order, task, precedence, transportation, setup-time, resource, job, one-machine scheduling & single machine & alldifferent, noOverlap, circuit, Disjunctive constraint, disjunctive, table constraint &  & Cplex &  &  & real-world, benchmark &  & \ref{a:MelgarejoLS15} & \ref{c:MelgarejoLS15}\\
\rowlabel{b:Mercier-AubinGQ20}\href{../works/Mercier-AubinGQ20.pdf}{Mercier-AubinGQ20}~\cite{Mercier-AubinGQ20} & 13 & order, Benders Decomposition, job, make-span, sequence dependent setup, tardiness, resource, precedence, completion-time, machine, activity, due-date, preempt, task, setup-time, earliness, lazy clause generation, job-shop, scheduling, Logic-Based Benders Decomposition, preemptive & RCPSP & circuit, cumulative, disjunctive, cycle & C++, Python & OPL, MiniZinc &  & textile industry, manufacturing industry & industrial instance, industrial partner &  & \ref{a:Mercier-AubinGQ20} & \ref{c:Mercier-AubinGQ20}\\
\rowlabel{b:MoffittPP05}\href{../works/MoffittPP05.pdf}{MoffittPP05}~\cite{MoffittPP05} & 6 & order, activity, machine, cmax, make-span, scheduling, resource & Temporal Constraint Satisfaction Problem & cycle, disjunctive &  &  &  &  &  &  & \ref{a:MoffittPP05} & \ref{c:MoffittPP05}\\
\rowlabel{b:MonetteDD07}\href{../works/MonetteDD07.pdf}{MonetteDD07}~\cite{MonetteDD07} & 14 & machine, precedence, make-span, job, scheduling, completion-time, resource, preempt, no preempt, task, job-shop, open-shop, order, preemptive & Open Shop Scheduling Problem, OSP & disjunctive &  & Gecode &  &  & benchmark & not-last, not-first, edge-finding & \ref{a:MonetteDD07} & \ref{c:MonetteDD07}\\
\rowlabel{b:MonetteDH09}\href{../works/MonetteDH09.pdf}{MonetteDH09}~\cite{MonetteDH09} & 8 & machine, precedence, release-date, tardiness, make-span, job, scheduling, completion-time, resource, preempt, earliness, due-date, task, job-shop, order, activity, distributed, preemptive &  & cycle, disjunctive, cumulative &  &  &  &  & benchmark & not-last & \ref{a:MonetteDH09} & \ref{c:MonetteDH09}\\
\rowlabel{b:MossigeGSMC17}\href{../works/MossigeGSMC17.pdf}{MossigeGSMC17}~\cite{MossigeGSMC17} & 18 & activity, job, order, completion-time, scheduling, machine, precedence, distributed, preempt, make-span, task, job-shop, resource, preemptive & single machine, FJS, RCPSP & Cumulatives constraint, cumulative, cycle, disjunctive & Prolog & CHIP, SICStus & robot, rectangle-packing &  & real-world, benchmark, random instance, CSPlib, generated instance, industrial partner &  & \ref{a:MossigeGSMC17} & \ref{c:MossigeGSMC17}\\
\rowlabel{b:MouraSCL08}\href{../works/MouraSCL08.pdf}{MouraSCL08}~\cite{MouraSCL08} & 16 & scheduling, preempt, transportation, precedence, distributed, activity, order, inventory, resource, preemptive &  & table constraint, Element constraint, Channeling constraint, cycle, disjunctive & C++ & Ilog Solver, Ilog Scheduler & pipeline &  &  & max-flow & \ref{a:MouraSCL08} & \ref{c:MouraSCL08}\\
\rowlabel{b:MouraSCL08a}\href{../works/MouraSCL08a.pdf}{MouraSCL08a}~\cite{MouraSCL08a} & 8 & order, scheduling, resource, transportation, re-scheduling, due-date, inventory, distributed &  & Channeling constraint, disjunctive, cumulative & C++ & Ilog Scheduler, Ilog Solver & pipeline &  & real-world, benchmark &  & \ref{a:MouraSCL08a} & \ref{c:MouraSCL08a}\\
\rowlabel{b:MurinR19}\href{../works/MurinR19.pdf}{MurinR19}~\cite{MurinR19} & 16 & job-shop, make-span, transportation, resource, scheduling, Benders Decomposition, completion-time, precedence, task, order, machine, setup-time, job, activity, Logic-Based Benders Decomposition & JSPT & alternative constraint, noOverlap, endBeforeStart &  & Cplex, OPL & robot, patient &  & github, benchmark, real-life &  & \ref{a:MurinR19} & \ref{c:MurinR19}\\
\rowlabel{b:MurphyMB15}\href{../works/MurphyMB15.pdf}{MurphyMB15}~\cite{MurphyMB15} & 17 & scheduling, task, machine, activity, order, re-scheduling, resource, periodic, stochastic &  & cycle, circuit, Disjunctive constraint, cumulative, disjunctive & Java & Choco Solver &  &  & real-world &  & \ref{a:MurphyMB15} & \ref{c:MurphyMB15}\\
\rowlabel{b:Muscettola02}\href{../works/Muscettola02.pdf}{Muscettola02}~\cite{Muscettola02} & 16 & job-shop, resource, activity, job, cmax, precedence, scheduling, order, stochastic &  & cycle, Balance constraint &  &  &  &  &  & edge-finding, max-flow & \ref{a:Muscettola02} & \ref{c:Muscettola02}\\
\rowlabel{b:MusliuSS18}\href{../works/MusliuSS18.pdf}{MusliuSS18}~\cite{MusliuSS18} & 17 & distributed, activity, order, scheduling, manpower, task, machine &  & Regular constraint, cycle, Cardinality constraint &  & Gecode, Gurobi, MiniZinc & workforce scheduling, operating room, nurse &  & generated instance, benchmark, real-life &  & \ref{a:MusliuSS18} & \ref{c:MusliuSS18}\\
\rowlabel{b:NattafM20}\href{../works/NattafM20.pdf}{NattafM20}~\cite{NattafM20} & 16 & setup-time, scheduling, order, make-span, completion-time, flow-time, resource, machine, job & single machine, PMSP, parallel machine, PTC & cumulative, noOverlap &  & CPO, Cplex & semiconductor &  & benchmark, industrial instance &  & \ref{a:NattafM20} & \ref{c:NattafM20}\\
\rowlabel{b:NishikawaSTT18}\href{../works/NishikawaSTT18.pdf}{NishikawaSTT18}~\cite{NishikawaSTT18} & 6 & order, precedence, scheduling, make-span, resource, activity, task, distributed &  & alternative constraint, endBeforeStart &  & Cplex & pipeline, robot &  & real-world, benchmark &  & \ref{a:NishikawaSTT18} & \ref{c:NishikawaSTT18}\\
\rowlabel{b:NishikawaSTT18a}\href{../works/NishikawaSTT18a.pdf}{NishikawaSTT18a}~\cite{NishikawaSTT18a} & 6 & order, make-span, scheduling, resource, precedence, task, activity, distributed, re-scheduling &  & endBeforeStart, alternative constraint &  & Cplex & nurse, pipeline, robot &  & benchmark, real-life, real-world &  & \ref{a:NishikawaSTT18a} & \ref{c:NishikawaSTT18a}\\
\rowlabel{b:NuijtenA94}\href{../works/NuijtenA94.pdf}{NuijtenA94}~\cite{NuijtenA94} & 5 & resource, scheduling, preempt, machine, make-span, job, precedence, job-shop, completion-time, order, preemptive & JSSP & disjunctive, Disjunctive constraint & C++ & Ilog Solver, CPO &  &  &  & time-tabling & \ref{a:NuijtenA94} & \ref{c:NuijtenA94}\\
\rowlabel{b:OddiPCC03}\href{../works/OddiPCC03.pdf}{OddiPCC03}~\cite{OddiPCC03} & 15 & distributed, resource, machine, preempt, scheduling, precedence, order, completion-time, task, activity, periodic, single-machine scheduling & single machine & cycle & Java &  & satellite, earth observation &  & benchmark &  & \ref{a:OddiPCC03} & \ref{c:OddiPCC03}\\
\rowlabel{b:OuelletQ13}\href{../works/OuelletQ13.pdf}{OuelletQ13}~\cite{OuelletQ13} & 16 & scheduling, task, make-span, completion-time, precedence, order, preempt, resource, preemptive & RCPSP, CuSP, psplib & Cumulatives constraint, cumulative, disjunctive &  & Choco Solver &  &  & benchmark & edge-finder, energetic reasoning, time-tabling, sweep, edge-finding, not-first, not-last & \ref{a:OuelletQ13} & \ref{c:OuelletQ13}\\
\rowlabel{b:OuelletQ18}\href{../works/OuelletQ18.pdf}{OuelletQ18}~\cite{OuelletQ18} & 18 & scheduling, task, make-span, completion-time, precedence, order, resource & RCPSP, psplib & Cumulatives constraint, cumulative, disjunctive & Java & Choco Solver &  &  & benchmark, Roadef & energetic reasoning, time-tabling, edge-finding, not-first, not-last & \ref{a:OuelletQ18} & \ref{c:OuelletQ18}\\
\rowlabel{b:OuelletQ22}\href{../works/OuelletQ22.pdf}{OuelletQ22}~\cite{OuelletQ22} & 17 & scheduling, task, activity, completion-time, order, preempt, resource, lazy clause generation &  & GCC constraint, Cumulatives constraint, cumulative, Cardinality constraint, disjunctive, SoftCumulative & Java & MiniZinc, Choco Solver & nurse &  & github, benchmark, random instance & energetic reasoning, time-tabling, sweep, edge-finding, not-first, not-last & \ref{a:OuelletQ22} & \ref{c:OuelletQ22}\\
\rowlabel{b:OujanaAYB22}\href{../works/OujanaAYB22.pdf}{OujanaAYB22}~\cite{OujanaAYB22} & 6 & due-date, tardiness, make to order, job-shop, buffer-capacity, setup-time, sequence dependent setup, open-shop, task, order, distributed, precedence, flow-shop, batch process, make-span, job, scheduling, completion-time, resource, machine, preempt & HFF, PMSP, parallel machine, FJS & span constraint, noOverlap, disjunctive &  & CPO, OPL & robot, COVID & steel industry, food industry & industrial instance, real-world, benchmark, real-life &  & \ref{a:OujanaAYB22} & \ref{c:OujanaAYB22}\\
\rowlabel{b:ParkUJR19}\href{../works/ParkUJR19.pdf}{ParkUJR19}~\cite{ParkUJR19} & 8 & machine, order, tardiness, preempt, scheduling, make-span, completion-time, task, flow-time, cmax, job, lateness, no preempt, distributed, due-date, job-shop, flow-shop, resource, open-shop, stochastic & parallel machine, single machine & endBeforeStart, cycle, noOverlap &  &  &  & trade industry & real-world &  & \ref{a:ParkUJR19} & \ref{c:ParkUJR19}\\
\rowlabel{b:PembertonG98}\href{../works/PembertonG98.pdf}{PembertonG98}~\cite{PembertonG98} & 14 & scheduling, machine, order, job-shop, resource, activity, preempt, job, task, periodic, preemptive, stochastic &  & geost, cycle &  & Ilog Solver, OPL & robot, satellite &  &  &  & \ref{a:PembertonG98} & \ref{c:PembertonG98}\\
\rowlabel{b:PerezGSL23}\href{../works/PerezGSL23.pdf}{PerezGSL23}~\cite{PerezGSL23} & 7 & inventory, order, transportation, re-scheduling, resource, scheduling, task, machine, activity, make-span, completion-time &  & table constraint, cumulative &  & OPL & container terminal, operating room, nurse, steel mill &  & real-world, generated instance &  & \ref{a:PerezGSL23} & \ref{c:PerezGSL23}\\
\rowlabel{b:PesantRR15}\href{../works/PesantRR15.pdf}{PesantRR15}~\cite{PesantRR15} & 16 & transportation, lazy clause generation, scheduling, activity, order &  & cumulative, Cardinality constraint, Regular constraint, table constraint &  & Ilog Solver, Gecode, Gurobi &  &  &  &  & \ref{a:PesantRR15} & \ref{c:PesantRR15}\\
\rowlabel{b:PoderB08}\href{../works/PoderB08.pdf}{PoderB08}~\cite{PoderB08} & 8 & resource, release-date, preempt, due-date, order, scheduling, producer/consumer, task, activity, preemptive &  & cumulative &  & CHIP &  &  &  & sweep & \ref{a:PoderB08} & \ref{c:PoderB08}\\
\rowlabel{b:PopovicCGNC22}\href{../works/PopovicCGNC22.pdf}{PopovicCGNC22}~\cite{PopovicCGNC22} & 15 & order, completion-time, scheduling, machine, transportation, make-span, task, resource, activity, periodic, stochastic & TMS & Balance constraint, cumulative, noOverlap, alwaysIn & C++, Prolog & SICStus, Cplex, CHIP & pipeline, maintenance scheduling & electricity industry &  &  & \ref{a:PopovicCGNC22} & \ref{c:PopovicCGNC22}\\
\rowlabel{b:PovedaAA23}\href{../works/PovedaAA23.pdf}{PovedaAA23}~\cite{PovedaAA23} & 21 & make-span, resource, job, precedence, Benders Decomposition, lazy clause generation, release-date, task, job-shop, activity, order, scheduling, preempt, preemptive & RCPSP & Calendar constraint, cumulative, disjunctive & Python & Cplex, MiniZinc, Chuffed, CPO & automotive, aircraft &  & github, benchmark, industrial instance, real-world, real-life & GRASP & \ref{a:PovedaAA23} & \ref{c:PovedaAA23}\\
\rowlabel{b:Pralet17}\href{../works/Pralet17.pdf}{Pralet17}~\cite{Pralet17} & 19 & setup-time, job, activity, job-shop, sequence dependent setup, resource, scheduling, precedence, due-date, order, make-span, machine & JSSP, RCPSP, psplib & cycle, cumulative, disjunctive &  & CPO, Cplex, CHIP & satellite &  & benchmark &  & \ref{a:Pralet17} & \ref{c:Pralet17}\\
\rowlabel{b:PraletLJ15}\href{../works/PraletLJ15.pdf}{PraletLJ15}~\cite{PraletLJ15} & 16 & task, job-shop, activity, make-span, precedence, due-date, tardiness, order, resource, job, scheduling & JSSP & alternative constraint, Regular constraint, noOverlap, cycle &  & CPO, Cplex & earth observation, satellite &  &  &  & \ref{a:PraletLJ15} & \ref{c:PraletLJ15}\\
\rowlabel{b:Puget95}\href{../works/Puget95.pdf}{Puget95}~\cite{Puget95} & 4 & resource, task, job, order, scheduling, transportation, manpower, job-shop, activity &  & disjunctive &  & OPL & maintenance scheduling &  & benchmark &  & \ref{a:Puget95} & \ref{c:Puget95}\\
\rowlabel{b:QuSN06}\href{../works/QuSN06.pdf}{QuSN06}~\cite{QuSN06} & 4 & task, scheduling, precedence, distributed, resource &  & circuit & Prolog & SICStus &  &  &  &  & \ref{a:QuSN06} & \ref{c:QuSN06}\\
\rowlabel{b:QuirogaZH05}\href{../works/QuirogaZH05.pdf}{QuirogaZH05}~\cite{QuirogaZH05} & 6 & machine, release-date, tardiness, scheduling, completion-time, resource, earliness, due-date, task, precedence, flow-shop, make-span, order, inventory, activity, flow-time &  &  &  & Ilog Solver, OPL, ECLiPSe, Ilog Scheduler & robot &  &  &  & \ref{a:QuirogaZH05} & \ref{c:QuirogaZH05}\\
\rowlabel{b:RendlPHPR12}\href{../works/RendlPHPR12.pdf}{RendlPHPR12}~\cite{RendlPHPR12} & 17 & job, scheduling, machine, transportation, re-scheduling, order, periodic &  &  & Java &  & medical, patient, nurse &  & real-world, CSPlib, benchmark &  & \ref{a:RendlPHPR12} & \ref{c:RendlPHPR12}\\
\rowlabel{b:RiahiNS018}\href{../works/RiahiNS018.pdf}{RiahiNS018}~\cite{RiahiNS018} & 9 & no-wait, flow-shop, completion-time, tardiness, order, buffer-capacity, sequence dependent setup, job, scheduling, blocking constraint, distributed, setup-time, machine, make-span &  & Blocking constraint &  &  &  & cutting industry, painting industry & real-world, real-life, benchmark & NEH, GRASP & \ref{a:RiahiNS018} & \ref{c:RiahiNS018}\\
\rowlabel{b:RodosekW98}\href{../works/RodosekW98.pdf}{RodosekW98}~\cite{RodosekW98} & 15 & order, resource, scheduling, task, transportation, machine, activity, make-span, job &  & disjunctive, cycle, circuit, Disjunctive constraint & Prolog & OPL, CHIP, ECLiPSe, Cplex & hoist, electroplating &  & benchmark &  & \ref{a:RodosekW98} & \ref{c:RodosekW98}\\
\rowlabel{b:Rodriguez07b}\href{../works/Rodriguez07b.pdf}{Rodriguez07b}~\cite{Rodriguez07b} & 14 & re-scheduling, task, blocking constraint, release-date, precedence, scheduling, transportation, order, no-wait, job-shop, resource, activity, job &  & Blocking constraint, Disjunctive constraint, circuit, disjunctive &  & Ilog Scheduler, Z3, Ilog Solver & railway, train schedule & railway industry &  & edge-finding & \ref{a:Rodriguez07b} & \ref{c:Rodriguez07b}\\
\rowlabel{b:RodriguezS09}\href{../works/RodriguezS09.pdf}{RodriguezS09}~\cite{RodriguezS09} & 14 & blocking constraint, completion-time, Benders Decomposition, precedence, scheduling, transportation, order, no-wait, job-shop, resource, activity, job, task, Logic-Based Benders Decomposition &  & Blocking constraint, Disjunctive constraint, circuit, disjunctive &  & Ilog Scheduler, Ilog Solver & railway, train schedule &  &  & edge-finding & \ref{a:RodriguezS09} & \ref{c:RodriguezS09}\\
\rowlabel{b:RossiTHP07}\href{../works/RossiTHP07.pdf}{RossiTHP07}~\cite{RossiTHP07} & 15 & inventory, order, resource, scheduling, distributed, stock level, periodic, stochastic &  & cumulative, cycle &  & OPL, Choco Solver &  &  &  &  & \ref{a:RossiTHP07} & \ref{c:RossiTHP07}\\
\rowlabel{b:Sadykov04}\href{../works/Sadykov04.pdf}{Sadykov04}~\cite{Sadykov04} & 7 & release-date, scheduling, completion-time, task, machine, job, lateness, due-date, preempt, precedence, one-machine scheduling & parallel machine, single machine & disjunctive &  &  &  &  &  & edge-finding & \ref{a:Sadykov04} & \ref{c:Sadykov04}\\
\rowlabel{b:SchausD08}\href{../works/SchausD08.pdf}{SchausD08}~\cite{SchausD08} & 6 & precedence, order, task, preempt, preemptive &  & IloPack, bin-packing, cycle, Reified constraint, Element constraint &  & Ilog Solver, OPL &  &  & real-life, benchmark &  & \ref{a:SchausD08} & \ref{c:SchausD08}\\
\rowlabel{b:SchuttCSW12}\href{../works/SchuttCSW12.pdf}{SchuttCSW12}~\cite{SchuttCSW12} & 17 & scheduling, resource, preempt, lazy clause generation, order, activity, precedence, make-span, preemptive &  & cumulative &  & CHIP &  &  & benchmark &  & \ref{a:SchuttCSW12} & \ref{c:SchuttCSW12}\\
\rowlabel{b:SchuttFS13}\href{../works/SchuttFS13.pdf}{SchuttFS13}~\cite{SchuttFS13} & 17 & resource, job, lazy clause generation, scheduling, task, job-shop, machine, activity, make-span, completion-time, precedence, order & RCPSP, FJS & disjunctive, Disjunctive constraint, span constraint, alternative constraint, cumulative &  & MiniZinc &  &  & benchmark & energetic reasoning, time-tabling & \ref{a:SchuttFS13} & \ref{c:SchuttFS13}\\
\rowlabel{b:SchuttFS13a}\href{../works/SchuttFS13a.pdf}{SchuttFS13a}~\cite{SchuttFS13a} & 17 & make-span, scheduling, completion-time, resource, machine, preempt, lazy clause generation, task, order, activity, precedence, preemptive & psplib, RCPSP & circuit, disjunctive, cumulative &  & SCIP, CHIP &  &  & benchmark & not-last, energetic reasoning, edge-finding & \ref{a:SchuttFS13a} & \ref{c:SchuttFS13a}\\
\rowlabel{b:SchuttFSW09}\href{../works/SchuttFSW09.pdf}{SchuttFSW09}~\cite{SchuttFSW09} & 16 & scheduling, resource, machine, preempt, lazy clause generation, open-shop, task, order, activity, precedence, make-span, job, periodic, preemptive & psplib & Disjunctive constraint, disjunctive, cumulative &  & ECLiPSe, CHIP, SICStus &  &  & real-world, benchmark & edge-finder & \ref{a:SchuttFSW09} & \ref{c:SchuttFSW09}\\
\rowlabel{b:SchuttS16}\href{../works/SchuttS16.pdf}{SchuttS16}~\cite{SchuttS16} & 17 & machine, precedence, order, inventory, activity, preempt, manpower, scheduling, make-span, producer/consumer, lazy clause generation, resource, preemptive & RCPSP & Balance constraint, Cumulatives constraint, cumulative &  & Chuffed, MiniZinc, OPL, Ilog Scheduler &  &  & benchmark &  & \ref{a:SchuttS16} & \ref{c:SchuttS16}\\
\rowlabel{b:SchuttW10}\href{../works/SchuttW10.pdf}{SchuttW10}~\cite{SchuttW10} & 15 & order, activity, preempt, release-date, scheduling, make-span, task, lazy clause generation, due-date, resource, preemptive & CuSP, psplib, RCPSP & disjunctive, Disjunctive constraint, cumulative & Java & CHIP & rectangle-packing &  & benchmark & not-last, edge-finding, not-first & \ref{a:SchuttW10} & \ref{c:SchuttW10}\\
\rowlabel{b:SchuttWS05}\href{../works/SchuttWS05.pdf}{SchuttWS05}~\cite{SchuttWS05} & 15 & task, due-date, machine, order, preempt, resource, release-date, scheduling, preemptive &  & cumulative, disjunctive &  & OPL, CHIP &  &  & benchmark & not-last & \ref{a:SchuttWS05} & \ref{c:SchuttWS05}\\
\rowlabel{b:SerraNM12}\href{../works/SerraNM12.pdf}{SerraNM12}~\cite{SerraNM12} & 17 & inventory, preempt, resource, precedence, order, activity, release-date, scheduling, machine, preemptive &  & cumulative, alwaysIn, cycle &  & OPL, Cplex &  &  & real-world, benchmark & GRASP & \ref{a:SerraNM12} & \ref{c:SerraNM12}\\
\rowlabel{b:SialaAH15}\href{../works/SialaAH15.pdf}{SialaAH15}~\cite{SialaAH15} & 10 & make-span, task, cmax, job, job-shop, resource, open-shop, machine, precedence, order, tardiness, setup-time, earliness, lazy clause generation, scheduling & RCPSP, JSSP & Disjunctive constraint, cumulative, disjunctive &  & Mistral &  &  & github, benchmark & edge-finding & \ref{a:SialaAH15} & \ref{c:SialaAH15}\\
\rowlabel{b:SimoninAHL12}\href{../works/SimoninAHL12.pdf}{SimoninAHL12}~\cite{SimoninAHL12} & 15 & resource, activity, scheduling, task, precedence, preempt, order, periodic, preemptive &  & disjunctive, span constraint, cycle, cumulative &  & CHIP & satellite &  &  & sweep & \ref{a:SimoninAHL12} & \ref{c:SimoninAHL12}\\
\rowlabel{b:Simonis95}\href{../works/Simonis95.pdf}{Simonis95}~\cite{Simonis95} & 4 & scheduling, task, producer/consumer, resource, transportation, machine, precedence, order &  & diffn, Among constraint, cumulative, cycle, circuit & Prolog & CHIP & aircraft & food industry &  &  & \ref{a:Simonis95} & \ref{c:Simonis95}\\
\rowlabel{b:Simonis95a}\href{../works/Simonis95a.pdf}{Simonis95a}~\cite{Simonis95a} & 21 & scheduling, manpower, task, machine, job, precedence, distributed, stock level, due-date, order, inventory, producer/consumer, resource &  & cycle, diffn, circuit, cumulative & Prolog, C++ & OPL, CHIP & aircraft, pipeline & chemical industry, drawing industry & real-life, benchmark &  & \ref{a:Simonis95a} & \ref{c:Simonis95a}\\
\rowlabel{b:Simonis99}\href{../works/Simonis99.pdf}{Simonis99}~\cite{Simonis99} & 39 & scheduling, task, producer/consumer, job, inventory, due-date, manpower, resource, transportation, stock level, machine, precedence, order, activity, stochastic &  & disjunctive, Disjunctive constraint, diffn, cumulative, alldifferent, cycle, circuit & C++, Prolog & OPL, CHIP, ECLiPSe, SICStus & aircraft, pipeline, maintenance scheduling, nurse & chemical industry, food industry, process industry & benchmark, real-world, real-life & bi-partite matching & \ref{a:Simonis99} & \ref{c:Simonis99}\\
\rowlabel{b:SimonisC95}\href{../works/SimonisC95.pdf}{SimonisC95}~\cite{SimonisC95} & 14 & scheduling, manpower, task, transportation, machine, job, stock level, continuous-process, job-shop, due-date, flow-shop, order, inventory, batch process, producer/consumer, resource &  & diffn, cumulative & Prolog & CHIP & aircraft, pipeline, maintenance scheduling & food industry & real-life &  & \ref{a:SimonisC95} & \ref{c:SimonisC95}\\
\rowlabel{b:SimonisH11}\href{../works/SimonisH11.pdf}{SimonisH11}~\cite{SimonisH11} & 14 & preempt, manpower, task, order, producer/consumer, resource, scheduling, preemptive &  & Element constraint, CumulativeCost, Cumulatives constraint, cumulative &  & Choco Solver, CHIP, Cplex &  &  & real-life, real-world & sweep, edge-finding & \ref{a:SimonisH11} & \ref{c:SimonisH11}\\
\rowlabel{b:SquillaciPR23}\href{../works/SquillaciPR23.pdf}{SquillaciPR23}~\cite{SquillaciPR23} & 17 & multi-agent, distributed, task, resource, activity, order, scheduling, periodic & EOSP, OSP, Earth Observation Scheduling Problem & noOverlap & Python & Cplex & earth orbit, earth observation, satellite &  & github, benchmark & GRASP & \ref{a:SquillaciPR23} & \ref{c:SquillaciPR23}\\
\rowlabel{b:SunLYL10}\href{../works/SunLYL10.pdf}{SunLYL10}~\cite{SunLYL10} & 6 & task, order, distributed, scheduling, periodic &  & cycle &  & OPL, Cplex & automotive &  &  &  & \ref{a:SunLYL10} & \ref{c:SunLYL10}\\
\rowlabel{b:SvancaraB22}\href{../works/SvancaraB22.pdf}{SvancaraB22}~\cite{SvancaraB22} & 8 & multi-agent, batch process, make-span, order, activity, scheduling, resource, task &  & alternative constraint, noOverlap &  &  & train schedule, railway &  & benchmark, real-world & time-tabling & \ref{a:SvancaraB22} & \ref{c:SvancaraB22}\\
\rowlabel{b:SzerediS16}\href{../works/SzerediS16.pdf}{SzerediS16}~\cite{SzerediS16} & 10 & task, machine, activity, order, preempt, make-span, resource, precedence, lazy clause generation, scheduling, preemptive & RCPSP, psplib & Element constraint, cumulative &  & Cplex, MiniZinc, SCIP, Chuffed, Gecode &  &  & benchmark &  & \ref{a:SzerediS16} & \ref{c:SzerediS16}\\
\rowlabel{b:TanT18}\href{../works/TanT18.pdf}{TanT18}~\cite{TanT18} & 12 & flow-shop, Benders Decomposition, machine, cmax, release-date, job-shop, task, scheduling, completion-time, precedence, make-span, re-scheduling, job, setup-time, Logic-Based Benders Decomposition, single-machine scheduling & single machine, parallel machine & Disjunctive constraint, disjunctive &  & Cplex & medical, operating room, patient, robot &  & benchmark &  & \ref{a:TanT18} & \ref{c:TanT18}\\
\rowlabel{b:TangB20}\href{../works/TangB20.pdf}{TangB20}~\cite{TangB20} & 16 & job, flow-shop, resource, make-span, scheduling, tardiness, due-date, order, batch process, machine, precedence, Benders Decomposition, Logic-Based Benders Decomposition, two-stage scheduling & HFS, 2BPHFSP, single machine & span constraint, bin-packing, alwaysIn, Cardinality constraint, Element constraint, cycle, endBeforeStart, GCC constraint & Java & CPO, Cplex & semiconductor & manufacturing industry & real-world &  & \ref{a:TangB20} & \ref{c:TangB20}\\
\rowlabel{b:TardivoDFMP23}\href{../works/TardivoDFMP23.pdf}{TardivoDFMP23}~\cite{TardivoDFMP23} & 18 & activity, order, scheduling, lazy clause generation, task, precedence, preempt, make-span, resource & RCPSP, psplib, CuSP & cumulative, disjunctive, Cumulatives constraint & C++ & CHIP, Gecode, MiniZinc &  &  & benchmark, bitbucket, github, real-world & sweep, energetic reasoning, not-last, not-first, edge-finding, time-tabling & \ref{a:TardivoDFMP23} & \ref{c:TardivoDFMP23}\\
\rowlabel{b:TasselGS23}\href{../works/TasselGS23.pdf}{TasselGS23}~\cite{TasselGS23} & 9 & flow-shop, completion-time, order, tardiness, resource, scheduling, preempt, flow-time, task, machine, re-scheduling, make-span, job, precedence, job-shop, periodic & JSSP & cumulative, disjunctive, noOverlap & Java & Choco Solver &  &  & industrial instance, real-world, supplementary material, github, benchmark &  & \ref{a:TasselGS23} & \ref{c:TasselGS23}\\
\rowlabel{b:Teppan22}\href{../works/Teppan22.pdf}{Teppan22}~\cite{Teppan22} & 8 & job-shop, make-span, cmax, preempt, distributed, resource, scheduling, flow-shop, task, order, completion-time, machine, setup-time, job & parallel machine, JSSP, PTC, FJS & noOverlap, endBeforeStart & Java & OR-Tools, OPL &  &  & benchmark, real-life &  & \ref{a:Teppan22} & \ref{c:Teppan22}\\
\rowlabel{b:Tesch16}\href{../works/Tesch16.pdf}{Tesch16}~\cite{Tesch16} & 27 & job, resource, make-span, scheduling, order, completion-time, precedence & CuSP, psplib, RCPSP & cumulative, disjunctive & C++ & OPL &  &  & Roadef & energetic reasoning, not-first, sweep, edge-finding, not-last, time-tabling & \ref{a:Tesch16} & \ref{c:Tesch16}\\
\rowlabel{b:Tesch18}\href{../works/Tesch18.pdf}{Tesch18}~\cite{Tesch18} & 17 & preempt, task, job, release-date, resource, make-span, scheduling, due-date, order, machine, completion-time, precedence, lateness, preemptive, single-machine scheduling & CuSP, psplib, RCPSP, single machine & cumulative &  &  &  &  & Roadef & energetic reasoning, sweep, edge-finding, not-last, time-tabling & \ref{a:Tesch18} & \ref{c:Tesch18}\\
\rowlabel{b:ThiruvadyBME09}\href{../works/ThiruvadyBME09.pdf}{ThiruvadyBME09}~\cite{ThiruvadyBME09} & 15 & due-date, make-span, resource, setup-time, tardiness, open-shop, machine, job, scheduling, order, single-machine scheduling, stochastic & single machine & cumulative & C++ & Gecode &  &  &  &  & \ref{a:ThiruvadyBME09} & \ref{c:ThiruvadyBME09}\\
\rowlabel{b:ThomasKS20}\href{../works/ThomasKS20.pdf}{ThomasKS20}~\cite{ThomasKS20} & 18 & order, transportation, resource, scheduling, activity &  & cumulative & C , Java & CPO, OR-Tools, OPL, Cplex & medical, patient &  & CSPlib, benchmark, generated instance, bitbucket &  & \ref{a:ThomasKS20} & \ref{c:ThomasKS20}\\
\rowlabel{b:Thorsteinsson01}\href{../works/Thorsteinsson01.pdf}{Thorsteinsson01}~\cite{Thorsteinsson01} & 15 & order, Benders Decomposition, scheduling, job, machine, precedence, task, due-date, Logic-Based Benders Decomposition & parallel machine & alldifferent, cumulative, circuit, Arithmetic constraint &  & OPL &  &  &  &  & \ref{a:Thorsteinsson01} & \ref{c:Thorsteinsson01}\\
\rowlabel{b:Tom19}\href{../works/Tom19.pdf}{Tom19}~\cite{Tom19} & 6 & task, tardiness, resource, job-shop, job, re-scheduling, activity, scheduling, make-span, machine, transportation, single-machine scheduling & single machine &  & Java & OPL &  &  & real-world &  & \ref{a:Tom19} & \ref{c:Tom19}\\
\rowlabel{b:TouatBT22}\href{../works/TouatBT22.pdf}{TouatBT22}~\cite{TouatBT22} & 8 & job, no preempt, distributed, due-date, job-shop, flow-shop, resource, machine, precedence, order, tardiness, activity, preempt, release-date, earliness, scheduling, make-span, completion-time, task, periodic, single-machine scheduling & RCPSP, single machine & noOverlap &  & Cplex, OPL & robot, satellite, container terminal &  & generated instance, benchmark & time-tabling & \ref{a:TouatBT22} & \ref{c:TouatBT22}\\
\rowlabel{b:Touraivane95}\href{../works/Touraivane95.pdf}{Touraivane95}~\cite{Touraivane95} & 3 & order, scheduling, task &  &  & Prolog &  & crew-scheduling &  & real-life &  & \ref{a:Touraivane95} & \ref{c:Touraivane95}\\
\rowlabel{b:TranB12}\href{../works/TranB12.pdf}{TranB12}~\cite{TranB12} & 6 & setup-time, due-date, Benders Decomposition, release-date, resource, make-span, scheduling, sequence dependent setup, tardiness, job, order, machine, completion-time, distributed, precedence, cmax, Logic-Based Benders Decomposition, single-machine scheduling & PMSP, single machine, parallel machine & cycle, circuit & C++ & Cplex &  &  & benchmark &  & \ref{a:TranB12} & \ref{c:TranB12}\\
\rowlabel{b:TranDRFWOVB16}\href{../works/TranDRFWOVB16.pdf}{TranDRFWOVB16}~\cite{TranDRFWOVB16} & 9 & resource, activity, re-scheduling, job, order, scheduling, machine, task, job-shop, precedence, stochastic &  & cycle & Python & OPL & aircraft &  &  &  & \ref{a:TranDRFWOVB16} & \ref{c:TranDRFWOVB16}\\
\rowlabel{b:TranTDB13}\href{../works/TranTDB13.pdf}{TranTDB13}~\cite{TranTDB13} & 9 & flow-shop, resource, cmax, machine, job, re-scheduling, setup-time, scheduling, order, make-span, task, flow-time, distributed, periodic, stochastic & parallel machine & cycle & C++ & Cplex &  &  & real-world &  & \ref{a:TranTDB13} & \ref{c:TranTDB13}\\
\rowlabel{b:TranVNB17a}\href{../works/TranVNB17a.pdf}{TranVNB17a}~\cite{TranVNB17a} & 5 & scheduling, task, transportation, machine, activity, setup-time, order, resource &  & alternative constraint, cumulative &  & Cplex & medical, robot &  & real-world &  & \ref{a:TranVNB17a} & \ref{c:TranVNB17a}\\
\rowlabel{b:TranWDRFOVB16}\href{../works/TranWDRFOVB16.pdf}{TranWDRFOVB16}~\cite{TranWDRFOVB16} & 9 & job, order, scheduling, task, precedence, activity, job-shop, machine, single-machine scheduling, stochastic & single machine & cumulative, cycle & Python & OPL, Ilog Scheduler & robot, satellite &  & benchmark &  & \ref{a:TranWDRFOVB16} & \ref{c:TranWDRFOVB16}\\
\rowlabel{b:ValleMGT03}\href{../works/ValleMGT03.pdf}{ValleMGT03}~\cite{ValleMGT03} & 8 & machine, order, scheduling, transportation, make-span, resource, job, precedence, task, job-shop &  &  &  & Ilog Solver & robot &  & real-life & edge-finder & \ref{a:ValleMGT03} & \ref{c:ValleMGT03}\\
\rowlabel{b:VanczaM01}\href{../works/VanczaM01.pdf}{VanczaM01}~\cite{VanczaM01} & 15 & resource, machine, order, scheduling, precedence, task &  & cycle, disjunctive, Disjunctive constraint &  &  & robot &  & real-world, real-life &  & \ref{a:VanczaM01} & \ref{c:VanczaM01}\\
\rowlabel{b:VerfaillieL01}\href{../works/VerfaillieL01.pdf}{VerfaillieL01}~\cite{VerfaillieL01} & 15 & task, job-shop, job, open-shop, order, scheduling, stochastic & Open Shop Scheduling Problem & cycle &  & Cplex, OPL & earth observation, satellite &  &  &  & \ref{a:VerfaillieL01} & \ref{c:VerfaillieL01}\\
\rowlabel{b:Vilim02}\href{../works/Vilim02.pdf}{Vilim02}~\cite{Vilim02} & 1 & resource, scheduling, precedence, sequence dependent setup, batch process, activity, setup-time &  & cumulative, disjunctive &  &  &  &  &  & edge-finding & \ref{a:Vilim02} & \ref{c:Vilim02}\\
\rowlabel{b:Vilim03}\href{../works/Vilim03.pdf}{Vilim03}~\cite{Vilim03} & 1 & job, open-shop, order, scheduling, job-shop &  & cumulative, disjunctive &  &  &  &  &  & edge-finding, not-last & \ref{a:Vilim03} & \ref{c:Vilim03}\\
\rowlabel{b:Vilim04}\href{../works/Vilim04.pdf}{Vilim04}~\cite{Vilim04} & 13 & task, job, order, resource, scheduling, precedence, sequence dependent setup, batch process, machine, completion-time, activity, setup-time, job-shop &  & cumulative, disjunctive &  &  &  &  & benchmark & edge-finding, sweep, not-last & \ref{a:Vilim04} & \ref{c:Vilim04}\\
\rowlabel{b:Vilim05}\href{../works/Vilim05.pdf}{Vilim05}~\cite{Vilim05} & 14 & preempt, task, job, open-shop, order, resource, make-span, scheduling, precedence, machine, completion-time, activity, job-shop, preemptive &  & cumulative, disjunctive & C++ &  &  &  & benchmark & not-last & \ref{a:Vilim05} & \ref{c:Vilim05}\\
\rowlabel{b:Vilim09}\href{../works/Vilim09.pdf}{Vilim09}~\cite{Vilim09} & 15 & preempt, job, order, resource, scheduling, precedence, completion-time, activity, job-shop, preemptive &  & cumulative, cycle &  & CPO &  &  &  & energetic reasoning, edge-finding, not-first, not-last & \ref{a:Vilim09} & \ref{c:Vilim09}\\
\rowlabel{b:Vilim09a}\href{../works/Vilim09a.pdf}{Vilim09a}~\cite{Vilim09a} & 15 & order, scheduling, completion-time, task, activity, resource, preempt, preemptive &  & cycle, cumulative &  & Ilog Scheduler &  &  &  & edge-finding, not-last, energetic reasoning & \ref{a:Vilim09a} & \ref{c:Vilim09a}\\
\rowlabel{b:Vilim11}\href{../works/Vilim11.pdf}{Vilim11}~\cite{Vilim11} & 16 & preempt, task, order, resource, scheduling, precedence, machine, completion-time, activity, manpower, preemptive & psplib, RCPSP & cumulative, disjunctive, cycle &  &  &  &  & benchmark & energetic reasoning, edge-finding, sweep, not-last, time-tabling & \ref{a:Vilim11} & \ref{c:Vilim11}\\
\rowlabel{b:VilimBC04}\href{../works/VilimBC04.pdf}{VilimBC04}~\cite{VilimBC04} & 15 & scheduling, make-span, completion-time, job, distributed, job-shop, resource, open-shop, machine, precedence, order, activity &  & disjunctive, cumulative &  &  &  &  & benchmark, real-life & edge-finding, not-first, not-last & \ref{a:VilimBC04} & \ref{c:VilimBC04}\\
\rowlabel{b:VilimLS15}\href{../works/VilimLS15.pdf}{VilimLS15}~\cite{VilimLS15} & 17 & machine, precedence, order, activity, earliness, scheduling, make-span, completion-time, task, cmax, job, job-shop, resource, periodic & psplib, RCPSP & disjunctive, noOverlap, cumulative &  & Cplex, CPO & rectangle-packing &  & benchmark & time-tabling & \ref{a:VilimLS15} & \ref{c:VilimLS15}\\
\rowlabel{b:Wallace06}\href{../works/Wallace06.pdf}{Wallace06}~\cite{Wallace06} & 32 & earliness, task, resource, machine, job, job-shop, transportation, scheduling, Benders Decomposition, order, tardiness, Logic-Based Benders Decomposition &  & cycle, Channeling constraint, circuit &  & Z3, CHIP, Cplex, ECLiPSe, OPL & workforce scheduling, hoist &  & benchmark, real-world, Roadef &  & \ref{a:Wallace06} & \ref{c:Wallace06}\\
\rowlabel{b:WangB20}\href{../works/WangB20.pdf}{WangB20}~\cite{WangB20} & 8 & task, resource, scheduling, job, order, machine, distributed & Fixed Job Scheduling, FJS & AllDiff constraint, alldifferent, MinWeightAllDiff, WeightAllDiff &  & Gurobi & aircraft &  & github &  & \ref{a:WangB20} & \ref{c:WangB20}\\
\rowlabel{b:WangB23}\href{../works/WangB23.pdf}{WangB23}~\cite{WangB23} & 8 & task, resource, scheduling, job, lazy clause generation, order, transportation & Fixed Job Scheduling, FJS & alldifferent, Channeling constraint, MinWeightAllDiff, WeightAllDiff &  & Gurobi & crew-scheduling, operating room, aircraft &  & random instance, real-world &  & \ref{a:WangB23} & \ref{c:WangB23}\\
\rowlabel{b:WatsonB08}\href{../works/WatsonB08.pdf}{WatsonB08}~\cite{WatsonB08} & 15 & job-shop, resource, machine, order, scheduling, make-span, completion-time, cmax, job, periodic &  & disjunctive & C++ & Ilog Scheduler &  &  & real-world, benchmark &  & \ref{a:WatsonB08} & \ref{c:WatsonB08}\\
\rowlabel{b:WessenCS20}\href{../works/WessenCS20.pdf}{WessenCS20}~\cite{WessenCS20} & 10 & make-span, completion-time, precedence, order, multi-agent, job, scheduling, task, job-shop &  & circuit &  & Gecode & robot &  & real-world &  & \ref{a:WessenCS20} & \ref{c:WessenCS20}\\
\rowlabel{b:WinterMMW22}\href{../works/WinterMMW22.pdf}{WinterMMW22}~\cite{WinterMMW22} & 18 & tardiness, setup-time, task, order, distributed, precedence, release-date, job, scheduling, completion-time, resource, machine, due-date & PMSP, parallel machine & noOverlap, alternative constraint &  & CPO, Gurobi, Cplex & farming & manufacturing industry, agricultural industry & supplementary material, zenodo, industrial partner, benchmark, real-life, industry partner &  & \ref{a:WinterMMW22} & \ref{c:WinterMMW22}\\
\rowlabel{b:Wolf03}\href{../works/Wolf03.pdf}{Wolf03}~\cite{Wolf03} & 15 & resource, job, machine, job-shop, task, order, preempt, scheduling, completion-time, make-span, activity, preemptive &  & cumulative, Disjunctive constraint, disjunctive & Java &  & pipeline &  & benchmark & not-last, edge-finding, not-first, sweep & \ref{a:Wolf03} & \ref{c:Wolf03}\\
\rowlabel{b:Wolf05}\href{../works/Wolf05.pdf}{Wolf05}~\cite{Wolf05} & 15 & resource, job, machine, job-shop, task, order, preempt, scheduling, completion-time, precedence, make-span, activity, preemptive &  & cumulative & Java & Ilog Scheduler &  &  & benchmark & not-last, edge-finding, not-first, sweep & \ref{a:Wolf05} & \ref{c:Wolf05}\\
\rowlabel{b:Wolf09}\href{../works/Wolf09.pdf}{Wolf09}~\cite{Wolf09} & 17 & resource, job, machine, job-shop, task, order, preempt, scheduling, preemptive &  & WeightedSum, WeightedTaskSum & Java & CHIP, SICStus, OPL & operating room, patient, surgery &  & real-life & not-last, edge-finding, not-first, sweep & \ref{a:Wolf09} & \ref{c:Wolf09}\\
\rowlabel{b:Wolf11}\href{../works/Wolf11.pdf}{Wolf11}~\cite{Wolf11} & 17 & distributed, resource, inventory, machine, producer/consumer, task, order, preempt, scheduling, sequence dependent setup, activity, transportation, setup-time, preemptive & single machine & cumulative, Element constraint, Cumulatives constraint, alternative constraint & Java & CHIP, OPL & medical, nurse, physician, operating room, patient, surgery &  &  &  & \ref{a:Wolf11} & \ref{c:Wolf11}\\
\rowlabel{b:WolfS05}\href{../works/WolfS05.pdf}{WolfS05}~\cite{WolfS05} & 14 & order, completion-time, scheduling, distributed, preempt, activity, task, resource, preemptive &  & cumulative &  & CHIP &  &  & real-world & energetic reasoning, sweep, not-last & \ref{a:WolfS05} & \ref{c:WolfS05}\\
\rowlabel{b:WolinskiKG04}\href{../works/WolinskiKG04.pdf}{WolinskiKG04}~\cite{WolinskiKG04} & 8 & resource, precedence, scheduling, machine, order, distributed & SCC & Diff2 constraint, cycle & Java &  & pipeline &  &  &  & \ref{a:WolinskiKG04} & \ref{c:WolinskiKG04}\\
\rowlabel{b:WuBB05}\href{../works/WuBB05.pdf}{WuBB05}~\cite{WuBB05} & 1 & resource, job, release-date, scheduling, make-span, stochastic &  &  &  & Ilog Scheduler &  &  & benchmark &  & \ref{a:WuBB05} & \ref{c:WuBB05}\\
\rowlabel{b:YangSS19}\href{../works/YangSS19.pdf}{YangSS19}~\cite{YangSS19} & 10 & resource, preempt, order, scheduling, completion-time, machine, task, activity, lazy clause generation, preemptive &  & cumulative, disjunctive & Prolog & Choco Solver, Gecode, CHIP, OR-Tools, SICStus, OPL & rectangle-packing &  & generated instance & energetic reasoning, edge-finding, not-last & \ref{a:YangSS19} & \ref{c:YangSS19}\\
\rowlabel{b:YoungFS17}\href{../works/YoungFS17.pdf}{YoungFS17}~\cite{YoungFS17} & 10 & lazy clause generation, scheduling, make-span, task, resource, machine, precedence, order, activity, preempt, preemptive & psplib, RCPSP & disjunctive, cumulative &  & Chuffed, MiniZinc &  &  & benchmark, github, instance generator & time-tabling & \ref{a:YoungFS17} & \ref{c:YoungFS17}\\
\rowlabel{b:YuraszeckMC23}\href{../works/YuraszeckMC23.pdf}{YuraszeckMC23}~\cite{YuraszeckMC23} & 6 & job, open-shop, order, scheduling, due-date, make-span, precedence, cmax, distributed, preempt, job-shop, flow-time, release-date, machine, preemptive, stochastic & OSSP, JSSP & noOverlap &  &  &  &  & benchmark, github &  & \ref{a:YuraszeckMC23} & \ref{c:YuraszeckMC23}\\
\rowlabel{b:ZhangBB22}\href{../works/ZhangBB22.pdf}{ZhangBB22}~\cite{ZhangBB22} & 9 & preempt, scheduling, precedence, order, make-span, completion-time, task, distributed, job-shop, resource, cmax, machine, job, lateness, one-machine scheduling & single machine & disjunctive, span constraint, Disjunctive constraint, cycle & Python & OPL, Gurobi, CPO &  &  & benchmark, generated instance &  & \ref{a:ZhangBB22} & \ref{c:ZhangBB22}\\
\rowlabel{b:ZhangJZL22}\href{../works/ZhangJZL22.pdf}{ZhangJZL22}~\cite{ZhangJZL22} & 6 & resource, scheduling, task, transportation, machine, make-span, job, precedence, setup-time, due-date, flow-shop, completion-time, order, tardiness, single-machine scheduling, stochastic & single machine, parallel machine, HFS & noOverlap, endBeforeStart, alternative constraint, cumulative &  &  & semiconductor &  & benchmark &  & \ref{a:ZhangJZL22} & \ref{c:ZhangJZL22}\\
\rowlabel{b:ZhangLS12}\href{../works/ZhangLS12.pdf}{ZhangLS12}~\cite{ZhangLS12} & 4 & scheduling, order, cmax &  &  &  &  &  &  &  & time-tabling & \ref{a:ZhangLS12} & \ref{c:ZhangLS12}\\
\rowlabel{b:Zhou96}\href{../works/Zhou96.pdf}{Zhou96}~\cite{Zhou96} & 15 & release-date, job-shop, due-date, task, order, scheduling, completion-time, precedence, job, machine &  & Disjunctive constraint, disjunctive & Prolog & Z3 &  &  &  & edge-finding & \ref{a:Zhou96} & \ref{c:Zhou96}\\
\rowlabel{b:ZhouGL15}\href{../works/ZhouGL15.pdf}{ZhouGL15}~\cite{ZhouGL15} & 5 & distributed, resource, tardiness, job-shop, flow-shop, re-scheduling, task, order, scheduling, completion-time, machine, setup-time, job, make-span, transportation, cmax, online scheduling, stochastic & HFF, FJS, HFS, parallel machine & cumulative &  & CHIP, Gecode, OR-Tools & railway &  & real-world & GRASP, NEH & \ref{a:ZhouGL15} & \ref{c:ZhouGL15}\\
\rowlabel{b:ZhuS02}\href{../works/ZhuS02.pdf}{ZhuS02}~\cite{ZhuS02} & 5 & activity, distributed, resource, scheduling &  &  &  &  &  &  &  &  & \ref{a:ZhuS02} & \ref{c:ZhuS02}\\
\rowlabel{b:ZibranR11}\href{../works/ZibranR11.pdf}{ZibranR11}~\cite{ZibranR11} & 4 & scheduling, order, activity &  &  & Java & Cplex, OPL &  &  &  &  & \ref{a:ZibranR11} & \ref{c:ZibranR11}\\
\rowlabel{b:ZibranR11a}\href{../works/ZibranR11a.pdf}{ZibranR11a}~\cite{ZibranR11a} & 10 & scheduling, distributed, activity, order, resource &  &  &  & Cplex, OPL &  &  &  & time-tabling & \ref{a:ZibranR11a} & \ref{c:ZibranR11a}\\
\end{longtable}
}



%\clearpage
%\subsection{Manually Defined Fields}
%{\scriptsize
\begin{longtable}{>{\raggedright\arraybackslash}p{3cm}>{\raggedright\arraybackslash}p{6cm}lp{2cm}rrrrlp{2cm}p{2cm}rr}
\rowcolor{white}\caption{Manually Defined PAPER Properties}\\ \toprule
\rowcolor{white}Key & Title (Local Copy) & \shortstack{CP\\System} & Bench & Links & \shortstack{Data\\Avail} & \shortstack{Sol\\Avail} & \shortstack{Code\\Avail} & \shortstack{Based\\On} & Classification & Constraints & a & b\\ \midrule\endhead
\bottomrule
\endfoot
\rowlabel{c:AalianPG23}AalianPG23 \href{https://doi.org/10.4230/LIPIcs.CP.2023.6}{AalianPG23}~\cite{AalianPG23} & \href{works/AalianPG23.pdf}{Optimization of Short-Term Underground Mine Planning Using Constraint Programming} & CP Opt & real-world & 1 & n &  & n &  &  & ? & \ref{a:AalianPG23} & \ref{b:AalianPG23}\\
\rowlabel{c:Bit-Monnot23}Bit-Monnot23 \href{https://doi.org/10.3233/FAIA230278}{Bit-Monnot23}~\cite{Bit-Monnot23} & \href{works/Bit-Monnot23.pdf}{Enhancing Hybrid {CP-SAT} Search for Disjunctive Scheduling} & \su{ARIES {CP Opt} OR-Tools Mistral} & real-world, github, benchmark & 1 & \href{https://github.com/plaans/aries}{y} &  & \href{https://github.com/plaans/aries}{y} & - & \su{JSSP OSSP} & - & \ref{a:Bit-Monnot23} & \ref{b:Bit-Monnot23}\\
\rowlabel{c:EfthymiouY23}EfthymiouY23 \href{https://doi.org/10.1007/978-3-031-33271-5\_16}{EfthymiouY23}~\cite{EfthymiouY23} & \href{works/EfthymiouY23.pdf}{Predicting the Optimal Period for Cyclic Hoist Scheduling Problems} & OR-Tools & benchmark, random instance, generated instance, real-life, industrial instance & 3 & n &  & n & - & CHSP & - & \ref{a:EfthymiouY23} & \ref{b:EfthymiouY23}\\
\rowlabel{c:JuvinHHL23}JuvinHHL23 \href{https://doi.org/10.4230/LIPIcs.CP.2023.19}{JuvinHHL23}~\cite{JuvinHHL23} & \href{works/JuvinHHL23.pdf}{An Efficient Constraint Programming Approach to Preemptive Job Shop Scheduling} & \su{{CP Opt} Mistral} & supplementary material, github, benchmark & 6 & ref &  & y &  & PJSSP & \su{endBeforeStart span noOverlap} & \ref{a:JuvinHHL23} & \ref{b:JuvinHHL23}\\
\rowlabel{c:JuvinHL23}JuvinHL23 \href{https://doi.org/10.1007/978-3-031-33271-5\_23}{JuvinHL23}~\cite{JuvinHL23} & \href{works/JuvinHL23.pdf}{Constraint Programming for the Robust Two-Machine Flow-Shop Scheduling Problem with Budgeted Uncertainty} & \su{{CP Opt} Cplex} & real-world & 0 & ref &  & n & - & Perm FSSP & \su{endBeforeStart noOverlap sameSequence} & \ref{a:JuvinHL23} & \ref{b:JuvinHL23}\\
\rowlabel{c:KameugneFND23}KameugneFND23 \href{https://doi.org/10.4230/LIPIcs.CP.2023.20}{KameugneFND23}~\cite{KameugneFND23} & \href{works/KameugneFND23.pdf}{Horizontally Elastic Edge Finder Rule for Cumulative Constraint Based on Slack and Density} & ? & benchmark & 5 & \su{BL PSPlib} &  & n & - & RCPSPs & cumulative & \ref{a:KameugneFND23} & \ref{b:KameugneFND23}\\
\rowlabel{c:KimCMLLP23}KimCMLLP23 \href{https://doi.org/10.1007/978-3-031-33271-5\_31}{KimCMLLP23}~\cite{KimCMLLP23} & \href{works/KimCMLLP23.pdf}{Iterated Greedy Constraint Programming for Scheduling Steelmaking Continuous Casting} & \su{Gurobi OR-Tools} & real-world, benchmark, zenodo & 0 & \href{https://zenodo.org/records/5126007}{y} &  & n & - & SCC & \su{alternative noOverlap} & \ref{a:KimCMLLP23} & \ref{b:KimCMLLP23}\\
\rowlabel{c:Mehdizadeh-Somarin23}Mehdizadeh-Somarin23 \href{https://doi.org/10.1007/978-3-031-43670-3\_33}{Mehdizadeh-Somarin23}~\cite{Mehdizadeh-Somarin23} & \href{works/Mehdizadeh-Somarin23.pdf}{A Constraint Programming Model for a Reconfigurable Job Shop Scheduling Problem with Machine Availability} & CP Opt & random instance & 0 & n &  & n & - & \su{JSSP RMS} & \su{alternative endBeforeStart noOverlap} & \ref{a:Mehdizadeh-Somarin23} & \ref{b:Mehdizadeh-Somarin23}\\
\rowlabel{c:PerezGSL23}PerezGSL23 \href{https://doi.org/10.1109/ICTAI59109.2023.00108}{PerezGSL23}~\cite{PerezGSL23} & \href{works/PerezGSL23.pdf}{A Constraint Programming Model for Scheduling the Unloading of Trains in Ports} & custom & real-world, generated instance & 0 & n &  & n & - & SUTP & \su{table disjunctive} & \ref{a:PerezGSL23} & \ref{b:PerezGSL23}\\
\rowlabel{c:PovedaAA23}PovedaAA23 \href{https://doi.org/10.4230/LIPIcs.CP.2023.31}{PovedaAA23}~\cite{PovedaAA23} & \href{works/PovedaAA23.pdf}{Partially Preemptive Multi Skill/Mode Resource-Constrained Project Scheduling with Generalized Precedence Relations and Calendars} & \su{{CP Opt} MiniZinc Chuffed} & real-world, github, benchmark, industrial instance, real-life & 4 & y &  & \href{https://github.com/youngkd/MSPSP-InstLib/blob/master/models/mspsp.mzn}{y} &  & PP-MS-MMRCPSP/max-cal &  & \ref{a:PovedaAA23} & \ref{b:PovedaAA23}\\
\rowlabel{c:SquillaciPR23}SquillaciPR23 \href{https://doi.org/10.1007/978-3-031-33271-5\_29}{SquillaciPR23}~\cite{SquillaciPR23} & \href{works/SquillaciPR23.pdf}{Scheduling Complex Observation Requests for a Constellation of Satellites: Large Neighborhood Search Approaches} & Cplex Studio & github, benchmark & 2 & \href{https://github.com/ssquilla/Earth_Observing_Satellites_benchmarks}{y} &  & n & - & EOSP & ? & \ref{a:SquillaciPR23} & \ref{b:SquillaciPR23}\\
\rowlabel{c:TardivoDFMP23}TardivoDFMP23 \href{https://doi.org/10.1007/978-3-031-33271-5\_22}{TardivoDFMP23}~\cite{TardivoDFMP23} & \href{works/TardivoDFMP23.pdf}{Constraint Propagation on {GPU:} {A} Case Study for the Cumulative Constraint} & \su{MiniCPP MiniZinc} & bitbucket, github, benchmark, real-world & 9 & \href{https://bitbucket.org/constraint-programming/minicpp-benchmarks/src/main/rcpsp/}{\su{PSPLib BL Pack}} &  & y & - & RCPSP & cumulative & \ref{a:TardivoDFMP23} & \ref{b:TardivoDFMP23}\\
\rowlabel{c:TasselGS23}TasselGS23 \href{https://doi.org/10.1609/icaps.v33i1.27243}{TasselGS23}~\cite{TasselGS23} & \href{works/TasselGS23.pdf}{An End-to-End Reinforcement Learning Approach for Job-Shop Scheduling Problems Based on Constraint Programming} & \su{custom Choco} & industrial instance, real-world, supplementary material, github, benchmark & 0 & ref &  & \href{https://github.com/ingambe/End2End-Job-Shop-Scheduling-CP}{y} & - & JSSP & noOverlap & \ref{a:TasselGS23} & \ref{b:TasselGS23}\\
\rowlabel{c:WangB23}WangB23 \href{https://doi.org/10.1109/ICTAI59109.2023.00062}{WangB23}~\cite{WangB23} & \href{works/WangB23.pdf}{Dynamic All-Different and Maximal Cliques Constraints for Fixed Job Scheduling} & FaCiLe & real-world, random instance & 0 & (y) &  & n & \cite{WangB20} & FJS & - & \ref{a:WangB23} & \ref{b:WangB23}\\
\rowlabel{c:YuraszeckMC23}YuraszeckMC23 \href{https://doi.org/10.1016/j.procs.2023.03.130}{YuraszeckMC23}~\cite{YuraszeckMC23} & \href{works/YuraszeckMC23.pdf}{A competitive constraint programming approach for the group shop scheduling problem} & CP Opt & github, benchmark & 0 & ref &  & n & - & GSSP & \su{noOverlap endBeforeStart} & \ref{a:YuraszeckMC23} & \ref{b:YuraszeckMC23}\\
\rowlabel{c:ArmstrongGOS22}ArmstrongGOS22 \href{https://doi.org/10.1007/978-3-031-08011-1\_1}{ArmstrongGOS22}~\cite{ArmstrongGOS22} & \href{works/ArmstrongGOS22.pdf}{A Two-Phase Hybrid Approach for the Hybrid Flexible Flowshop with Transportation Times} & CP Opt & real-world, benchmark & 0 & (y) &  & - & \cite{ArmstrongGOS21} & $HFFm|tt|C_{\max}$ & \su{endBeforeStart alternative cumulative noOverlap} & \ref{a:ArmstrongGOS22} & \ref{b:ArmstrongGOS22}\\
\rowlabel{c:BoudreaultSLQ22}BoudreaultSLQ22 \href{https://doi.org/10.4230/LIPIcs.CP.2022.10}{BoudreaultSLQ22}~\cite{BoudreaultSLQ22} & \href{works/BoudreaultSLQ22.pdf}{A Constraint Programming Approach to Ship Refit Project Scheduling} & \su{MiniZinc Chuffed} & benchmark, generated instance, supplementary material, gitlab, real-life, industrial partner, github, real-world & 9 &  &  & \href{https://github.com/raphaelboudreault/chuffed/releases/tag/SBPS}{y} & - & RCPSP & cumulative & \ref{a:BoudreaultSLQ22} & \ref{b:BoudreaultSLQ22}\\
\rowlabel{c:GeitzGSSW22}GeitzGSSW22 \href{https://doi.org/10.1007/978-3-031-08011-1\_10}{GeitzGSSW22}~\cite{GeitzGSSW22} & \href{works/GeitzGSSW22.pdf}{Solving the Extended Job Shop Scheduling Problem with AGVs - Classical and Quantum Approaches} & \su{firstCS QUBO} & real-life, github, real-world & 8 & \href{https://github.com/cgrozea/Data4ExtJSSAGV}{y} &  & n & - & JSSP &  & \ref{a:GeitzGSSW22} & \ref{b:GeitzGSSW22}\\
\rowlabel{c:HebrardALLCMR22}HebrardALLCMR22 \href{https://doi.org/10.24963/ijcai.2022/643}{HebrardALLCMR22}~\cite{HebrardALLCMR22} & \href{works/HebrardALLCMR22.pdf}{An Efficient Approach to Data Transfer Scheduling for Long Range Space Exploration} &  &  & 0 &  &  &  &  &  &  & \ref{a:HebrardALLCMR22} & \ref{b:HebrardALLCMR22}\\
\rowlabel{c:JungblutK22}JungblutK22 \href{https://doi.org/10.1109/IPDPSW55747.2022.00025}{JungblutK22}~\cite{JungblutK22} & \href{works/JungblutK22.pdf}{Optimal Schedules for High-Level Programming Environments on FPGAs with Constraint Programming} & MiniZinc & benchmark, github, real-world & 0 & \href{https://github.com/pascalj/reconf-scheduling}{y} &  & y & - &  &  & \ref{a:JungblutK22} & \ref{b:JungblutK22}\\
\rowlabel{c:LiFJZLL22}LiFJZLL22 \href{https://doi.org/10.1109/ICNSC55942.2022.10004158}{LiFJZLL22}~\cite{LiFJZLL22} & \href{works/LiFJZLL22.pdf}{Constraint Programming for a Novel Integrated Optimization of Blocking Job Shop Scheduling and Variable-Speed Transfer Robot Assignment} & \su{OPL {CP Opt}} & benchmark & 0 & ref &  & n & - & BJSSP & \su{endBEforeStart alternative noOverlap} & \ref{a:LiFJZLL22} & \ref{b:LiFJZLL22}\\
\rowlabel{c:LuoB22}LuoB22 \href{https://doi.org/10.1007/978-3-031-08011-1\_17}{LuoB22}~\cite{LuoB22} & \href{works/LuoB22.pdf}{Packing by Scheduling: Using Constraint Programming to Solve a Complex 2D Cutting Stock Problem} & CPO & generated instance, github, real-life, real-world, industry partner, industrial instance & 2 & n &  & n & - & 2SCSP-FF & \su{pulse alwaysIn forbidExtent stateFunction} & \ref{a:LuoB22} & \ref{b:LuoB22}\\
\rowlabel{c:OuelletQ22}OuelletQ22 \href{https://doi.org/10.1007/978-3-031-08011-1\_21}{OuelletQ22}~\cite{OuelletQ22} & \href{works/OuelletQ22.pdf}{A MinCumulative Resource Constraint} & Choco & github, benchmark, random instance & 1 & \href{https://github.com/yanickouellet/min-cumulative-paper-public}{y} &  & \href{https://github.com/yanickouellet/min-cumulative-paper-public}{y} & - &  & \su{cumulative minCumulative} & \ref{a:OuelletQ22} & \ref{b:OuelletQ22}\\
\rowlabel{c:OujanaAYB22}OujanaAYB22 \href{https://doi.org/10.1109/CoDIT55151.2022.9803972}{OujanaAYB22}~\cite{OujanaAYB22} & \href{works/OujanaAYB22.pdf}{Solving a realistic hybrid and flexible flow shop scheduling problem through constraint programming: industrial case in a packaging company} & CP Opt & benchmark, industrial instance, real-world, real-life & 0 & n &  & n & - & HFFS & \su{alternative span noOverlap endBeforeStart} & \ref{a:OujanaAYB22} & \ref{b:OujanaAYB22}\\
\rowlabel{c:PopovicCGNC22}PopovicCGNC22 \href{https://doi.org/10.4230/LIPIcs.CP.2022.34}{PopovicCGNC22}~\cite{PopovicCGNC22} & \href{works/PopovicCGNC22.pdf}{Scheduling the Equipment Maintenance of an Electric Power Transmission Network Using Constraint Programming} & CP Opt &  & 0 & n &  & n & - & TMS & \su{alwaysIn noOverlap} & \ref{a:PopovicCGNC22} & \ref{b:PopovicCGNC22}\\
\rowlabel{c:SvancaraB22}SvancaraB22 \href{https://doi.org/10.5220/0010869700003116}{SvancaraB22}~\cite{SvancaraB22} & \href{works/SvancaraB22.pdf}{Tackling Train Routing via Multi-agent Pathfinding and Constraint-based Scheduling} &  & benchmark, real-world & 0 &  &  &  &  &  &  & \ref{a:SvancaraB22} & \ref{b:SvancaraB22}\\
\rowlabel{c:Teppan22}Teppan22 \href{https://doi.org/10.5220/0010849900003116}{Teppan22}~\cite{Teppan22} & \href{works/Teppan22.pdf}{Types of Flexible Job Shop Scheduling: {A} Constraint Programming Experiment} & OPL & real-life, benchmark & 0 & ref &  & n & - & FJSSP & \su{noOverlap alternative endBeforeStart} & \ref{a:Teppan22} & \ref{b:Teppan22}\\
\rowlabel{c:TouatBT22}TouatBT22 \href{}{TouatBT22}~\cite{TouatBT22} & \href{works/TouatBT22.pdf}{A Constraint Programming Model for the Scheduling Problem with Flexible Maintenance under Human Resource Constraints} & OPL & benchmark, generated instance & 0 & n &  & n & - & Single Machine Scheduling & \su{alternative noOverlap forbidExtent} & \ref{a:TouatBT22} & \ref{b:TouatBT22}\\
\rowlabel{c:WinterMMW22}WinterMMW22 \href{https://doi.org/10.4230/LIPIcs.CP.2022.41}{WinterMMW22}~\cite{WinterMMW22} & \href{works/WinterMMW22.pdf}{Modeling and Solving Parallel Machine Scheduling with Contamination Constraints in the Agricultural Industry} & \su{Cplex Gurobi {CP Opt} {Sim Anneal}} & supplementary material, real-life, industry partner, zenodo, industrial partner, benchmark & 0 & \href{https://zenodo.org/records/6797397}{y} &  & \href{https://zenodo.org/records/6797397}{y} & - & PMSP & \su{alternative noOverlap} & \ref{a:WinterMMW22} & \ref{b:WinterMMW22}\\
\rowlabel{c:ZhangBB22}ZhangBB22 \href{https://ojs.aaai.org/index.php/ICAPS/article/view/19826}{ZhangBB22}~\cite{ZhangBB22} & \href{works/ZhangBB22.pdf}{Solving Job-Shop Scheduling Problems with QUBO-Based Specialized Hardware} &  & benchmark, generated instance & 0 &  &  &  &  &  &  & \ref{a:ZhangBB22} & \ref{b:ZhangBB22}\\
\rowlabel{c:ZhangJZL22}ZhangJZL22 \href{https://doi.org/10.1109/ICNSC55942.2022.10004154}{ZhangJZL22}~\cite{ZhangJZL22} & \href{works/ZhangJZL22.pdf}{Constraint Programming for Modeling and Solving a Hybrid Flow Shop Scheduling Problem} & OP Opt & benchmark & 0 & ref &  & n & - & HFSP & \su{alternative endBeforeStart noOverlap cumulative} & \ref{a:ZhangJZL22} & \ref{b:ZhangJZL22}\\
\rowlabel{c:AntuoriHHEN21}AntuoriHHEN21 \href{https://doi.org/10.4230/LIPIcs.CP.2021.14}{AntuoriHHEN21}~\cite{AntuoriHHEN21} & \href{works/AntuoriHHEN21.pdf}{Combining Monte Carlo Tree Search and Depth First Search Methods for a Car Manufacturing Workshop Scheduling Problem} & MCTS & gitlab, supplementary material & 1 & \href{https://gitlab.laas.fr/vantuori/mcts-cp}{y} &  & \href{https://gitlab.laas.fr/vantuori/mcts-cp}{y} &  &  &  & \ref{a:AntuoriHHEN21} & \ref{b:AntuoriHHEN21}\\
\rowlabel{c:ArmstrongGOS21}ArmstrongGOS21 \href{https://doi.org/10.4230/LIPIcs.CP.2021.16}{ArmstrongGOS21}~\cite{ArmstrongGOS21} & \href{works/ArmstrongGOS21.pdf}{The Hybrid Flexible Flowshop with Transportation Times} & \su{MiniZinc Chuffed {CP Opt} SICStus} & instance generator, industry partner, zenodo, supplementary material, real-world, industrial partner, benchmark & 1 & \href{https://zenodo.org/record/5168966}{y} &  & y & - & $HFFm|tt|C_{\max}$ & \su{cumulative diffn table} & \ref{a:ArmstrongGOS21} & \ref{b:ArmstrongGOS21}\\
\rowlabel{c:ArtiguesHQT21}ArtiguesHQT21 \href{https://doi.org/10.5220/0010190101290136}{ArtiguesHQT21}~\cite{ArtiguesHQT21} & \href{}{Multi-Mode {RCPSP} with Safety Margin Maximization: Models and Algorithms} &  &  & 0 &  &  &  &  &  &  & \ref{a:ArtiguesHQT21} & No\\
\rowlabel{c:Astrand0F21}Astrand0F21 \href{https://doi.org/10.1007/978-3-030-78230-6\_23}{Astrand0F21}~\cite{Astrand0F21} & \href{works/Astrand0F21.pdf}{Short-Term Scheduling of Production Fleets in Underground Mines Using CP-Based {LNS}} & Gecode & benchmark, real-world, real-life, generated instance & 0 & \su{ref generated} &  & n & - &  & - & \ref{a:Astrand0F21} & \ref{b:Astrand0F21}\\
\rowlabel{c:BenderWS21}BenderWS21 \href{https://doi.org/10.1007/978-3-030-87672-2\_37}{BenderWS21}~\cite{BenderWS21} & \href{works/BenderWS21.pdf}{Applying Constraint Programming to the Multi-mode Scheduling Problem in Harvest Logistics} & CP Opt &  & 9 & \href{https://tud.link/47mz}{y} &  & n & - & MRCPSP & \su{noOverlap alternative} & \ref{a:BenderWS21} & \ref{b:BenderWS21}\\
\rowlabel{c:GeibingerKKMMW21}GeibingerKKMMW21 \href{https://doi.org/10.1007/978-3-030-78230-6\_29}{GeibingerKKMMW21}~\cite{GeibingerKKMMW21} & \href{works/GeibingerKKMMW21.pdf}{Physician Scheduling During a Pandemic} & MiniZinc & real-world & 3 & \href{https://cdlab-artis.dbai.tuwien.ac.at/papers/pandemic-scheduling/}{y} &  & n & - &  & nvalue & \ref{a:GeibingerKKMMW21} & \ref{b:GeibingerKKMMW21}\\
\rowlabel{c:GeibingerMM21}GeibingerMM21 \href{https://doi.org/10.1609/aaai.v35i7.16789}{GeibingerMM21}~\cite{GeibingerMM21} & \href{works/GeibingerMM21.pdf}{Constraint Logic Programming for Real-World Test Laboratory Scheduling} & clingcon & real-life, github, generated instance, real-world, benchmark & 0 & \href{dbai.tuwien.ac.at/staff/fmischek/TLSP}{y} &  &  &  & \su{TLSP RCPSP} & disjunctive & \ref{a:GeibingerMM21} & \ref{b:GeibingerMM21}\\
\rowlabel{c:HanenKP21}HanenKP21 \href{https://doi.org/10.1007/978-3-030-78230-6\_14}{HanenKP21}~\cite{HanenKP21} & \href{works/HanenKP21.pdf}{Two Deadline Reduction Algorithms for Scheduling Dependent Tasks on Parallel Processors} & Python & Roadef, generated instance, random instance & 1 & ref &  & n & - & $P|prec, r_i, d_i|*$ & - & \ref{a:HanenKP21} & \ref{b:HanenKP21}\\
\rowlabel{c:HillTV21}HillTV21 \href{https://doi.org/10.1007/978-3-030-78230-6\_2}{HillTV21}~\cite{HillTV21} & \href{works/HillTV21.pdf}{A Computational Study of Constraint Programming Approaches for Resource-Constrained Project Scheduling with Autonomous Learning Effects} & CP Opt & real-world & 0 & PSPlib &  & n & - & RCPSP & \su{cumulative alternative endBeforeStart} & \ref{a:HillTV21} & \ref{b:HillTV21}\\
\rowlabel{c:KlankeBYE21}KlankeBYE21 \href{https://doi.org/10.1007/978-3-030-78230-6\_9}{KlankeBYE21}~\cite{KlankeBYE21} & \href{works/KlankeBYE21.pdf}{Combining Constraint Programming and Temporal Decomposition Approaches - Scheduling of an Industrial Formulation Plant} & OR-Tools & benchmark, random instance, real-life & 0 & n &  & n & - &  & \su{cumulative circuit noOverlap} & \ref{a:KlankeBYE21} & \ref{b:KlankeBYE21}\\
\rowlabel{c:KovacsTKSG21}KovacsTKSG21 \href{https://doi.org/10.4230/LIPIcs.CP.2021.36}{KovacsTKSG21}~\cite{KovacsTKSG21} & \href{works/KovacsTKSG21.pdf}{Utilizing Constraint Optimization for Industrial Machine Workload Balancing} & \su{Gurobi OR-Tools Cplex {CP Opt}} & github, supplementary material, real-world, benchmark & 2 & \href{https://github.com/prosysscience/CPWorkloadBalancing}{y} &  & \href{https://github.com/prosysscience/CPWorkloadBalancing}{y} & - & extended RCPSP & cumulative & \ref{a:KovacsTKSG21} & \ref{b:KovacsTKSG21}\\
\rowlabel{c:LacknerMMWW21}LacknerMMWW21 \href{https://doi.org/10.4230/LIPIcs.CP.2021.37}{LacknerMMWW21}~\cite{LacknerMMWW21} & \href{works/LacknerMMWW21.pdf}{Minimizing Cumulative Batch Processing Time for an Industrial Oven Scheduling Problem} & \su{{CP Opt} Chuffed OR-Tools Gurobi OPL} & random instance, industrial partner, benchmark, instance generator, real-life, supplementary material & 3 & \href{https://cdlab-artis.dbai.tuwien.ac.at/papers/ovenscheduling/}{y} &  & \href{https://cdlab-artis.dbai.tuwien.ac.at/papers/ovenscheduling/}{y} &  & OSP &  & \ref{a:LacknerMMWW21} & \ref{b:LacknerMMWW21}\\
\rowlabel{c:AntuoriHHEN20}AntuoriHHEN20 \href{https://doi.org/10.1007/978-3-030-58475-7\_38}{AntuoriHHEN20}~\cite{AntuoriHHEN20} & \href{works/AntuoriHHEN20.pdf}{Leveraging Reinforcement Learning, Constraint Programming and Local Search: {A} Case Study in Car Manufacturing} &  & random instance, generated instance, gitlab, benchmark, industrial instance & 4 &  &  &  &  &  &  & \ref{a:AntuoriHHEN20} & \ref{b:AntuoriHHEN20}\\
\rowlabel{c:BarzegaranZP20}BarzegaranZP20 \href{https://doi.org/10.4230/OASIcs.Fog-IoT.2020.3}{BarzegaranZP20}~\cite{BarzegaranZP20} & \href{works/BarzegaranZP20.pdf}{Quality-Of-Control-Aware Scheduling of Communication in TSN-Based Fog Computing Platforms Using Constraint Programming} & OR-Tools &  & 5 & n &  & n & - & FCP &  & \ref{a:BarzegaranZP20} & \ref{b:BarzegaranZP20}\\
\rowlabel{c:GodetLHS20}GodetLHS20 \href{https://doi.org/10.1609/aaai.v34i02.5510}{GodetLHS20}~\cite{GodetLHS20} & \href{works/GodetLHS20.pdf}{Using Approximation within Constraint Programming to Solve the Parallel Machine Scheduling Problem with Additional Unit Resources} & \su{MiniZinc Choco Chuffed} & github, real-life, benchmark, generated instance & 0 & \href{https://github.com/ArthurGodet/PMSPAUR-public}{JSON} &  & \href{https://github.com/ArthurGodet/PMSPAUR-public}{y} & - & PMSPAUR & \su{disjunctive cumulative alldifferent enqueueCstr approxCstr} & \ref{a:GodetLHS20} & \ref{b:GodetLHS20}\\
\rowlabel{c:GroleazNS20}GroleazNS20 \href{https://doi.org/10.1007/978-3-030-58475-7\_36}{GroleazNS20}~\cite{GroleazNS20} & \href{works/GroleazNS20.pdf}{Solving the Group Cumulative Scheduling Problem with {CPO} and {ACO}} & \su{{CP Opt} ACO} & benchmark, industrial instance & 0 & - &  & - & \cite{GroleazNS20} & GCSP & groupCumulative & \ref{a:GroleazNS20} & \ref{b:GroleazNS20}\\
\rowlabel{c:GroleazNS20a}GroleazNS20a \href{https://doi.org/10.1145/3377930.3389818}{GroleazNS20a}~\cite{GroleazNS20a} & \href{works/GroleazNS20a.pdf}{{ACO} with automatic parameter selection for a scheduling problem with a group cumulative constraint} & \su{CPO ACO} & industrial partner, benchmark & 0 & \href{https://perso.citi-lab.fr/csolnon/gc-sched.html}{y} &  & n & - & GCSP & \su{groupCumulative} & \ref{a:GroleazNS20a} & \ref{b:GroleazNS20a}\\
\rowlabel{c:Mercier-AubinGQ20}Mercier-AubinGQ20 \href{https://doi.org/10.1007/978-3-030-58942-4\_22}{Mercier-AubinGQ20}~\cite{Mercier-AubinGQ20} & \href{works/Mercier-AubinGQ20.pdf}{Leveraging Constraint Scheduling: {A} Case Study to the Textile Industry} & \su{MiniZinc Chuffed} & industrial instance, industrial partner & 1 & a &  & a & - &  & \su{circuit cumulative} & \ref{a:Mercier-AubinGQ20} & \ref{b:Mercier-AubinGQ20}\\
\rowlabel{c:NattafM20}NattafM20 \href{https://doi.org/10.1007/978-3-030-58475-7\_27}{NattafM20}~\cite{NattafM20} & \href{works/NattafM20.pdf}{Filtering Rules for Flow Time Minimization in a Parallel Machine Scheduling Problem} & \su{Cplex {CP Opt}} & benchmark, industrial instance & 7 & - &  & - & \cite{MalapertN19} & PTC & \su{alternative noOverlap} & \ref{a:NattafM20} & \ref{b:NattafM20}\\
\rowlabel{c:TangB20}TangB20 \href{https://doi.org/10.1007/978-3-030-58942-4\_28}{TangB20}~\cite{TangB20} & \href{works/TangB20.pdf}{{CP} and Hybrid Models for Two-Stage Batching and Scheduling} & \su{Cplex {CP Opt}} & real-world & 0 & n &  & n & - & 2BPHFSP & \su{span alwaysIn} & \ref{a:TangB20} & \ref{b:TangB20}\\
\rowlabel{c:WangB20}WangB20 \href{https://doi.org/10.3233/FAIA200114}{WangB20}~\cite{WangB20} & \href{works/WangB20.pdf}{Global Propagation of Transition Cost for Fixed Job Scheduling} & FaCiLe & github & 0 & \href{http://recherche.enac.fr/~wangrx/ecai_gap/}{y} &  & n & - & FJS & - & \ref{a:WangB20} & \ref{b:WangB20}\\
\rowlabel{c:WessenCS20}WessenCS20 \href{https://doi.org/10.1007/978-3-030-58942-4\_33}{WessenCS20}~\cite{WessenCS20} & \href{works/WessenCS20.pdf}{Scheduling of Dual-Arm Multi-tool Assembly Robots and Workspace Layout Optimization} & Gecode & real-world & 10 & n &  & n & - &  & \su{circuit alldifferent} & \ref{a:WessenCS20} & \ref{b:WessenCS20}\\
\rowlabel{c:BadicaBIL19}BadicaBIL19 \href{https://doi.org/10.1007/978-3-030-32258-8\_17}{BadicaBIL19}~\cite{BadicaBIL19} & \href{works/BadicaBIL19.pdf}{Exploring the Space of Block Structured Scheduling Processes Using Constraint Logic Programming} & ECLiPSe & github & 0 & dead &  & dead & - &  &  & \ref{a:BadicaBIL19} & \ref{b:BadicaBIL19}\\
\rowlabel{c:BehrensLM19}BehrensLM19 \href{https://doi.org/10.1109/ICRA.2019.8794022}{BehrensLM19}~\cite{BehrensLM19} & \href{works/BehrensLM19.pdf}{A Constraint Programming Approach to Simultaneous Task Allocation and Motion Scheduling for Industrial Dual-Arm Manipulation Tasks} & OR-Tools & real-world, github & 0 & \href{https://github.com/boschresearch/STAAMS-SOLVER}{y} &  & \href{https://github.com/boschresearch/STAAMS-SOLVER}{y} & - & STAAMS &  & \ref{a:BehrensLM19} & \ref{b:BehrensLM19}\\
\rowlabel{c:BogaerdtW19}BogaerdtW19 \href{https://doi.org/10.1007/978-3-030-19212-9\_38}{BogaerdtW19}~\cite{BogaerdtW19} & \href{works/BogaerdtW19.pdf}{Lower Bounds for Uniform Machine Scheduling Using Decision Diagrams} & \su{custom Cplex CPO} & benchmark & 4 & n &  & n & - & Multi Machine Scheduling & \su{noOverlap} & \ref{a:BogaerdtW19} & \ref{b:BogaerdtW19}\\
\rowlabel{c:ColT19}ColT19 \href{https://doi.org/10.1007/978-3-030-30048-7\_9}{ColT19}~\cite{ColT19} & \href{works/ColT19.pdf}{Industrial Size Job Shop Scheduling Tackled by Present Day {CP} Solvers} & \su{{CP Opt} OR-Tools} & github, benchmark, real-world & 2 & \href{https://drive.google.com/drive/folders/1QuKEABR9aiNKPIFe0VMFXP7BNor8KW9b}{y} &  & \href{https://drive.google.com/drive/folders/1QuKEABR9aiNKPIFe0VMFXP7BNor8KW9b}{y} & - & JSSP & \su{noOverlap} & \ref{a:ColT19} & \ref{b:ColT19}\\
\rowlabel{c:FrimodigS19}FrimodigS19 \href{https://doi.org/10.1007/978-3-030-30048-7\_25}{FrimodigS19}~\cite{FrimodigS19} & \href{works/FrimodigS19.pdf}{Models for Radiation Therapy Patient Scheduling} & \su{Mini-Zinc Gecode Cplex} & benchmark, real-world & 1 & n &  & n & - &  & \su{cumulative regular bin-packing} & \ref{a:FrimodigS19} & \ref{b:FrimodigS19}\\
\rowlabel{c:FrohnerTR19}FrohnerTR19 \href{https://doi.org/10.1007/978-3-030-45093-9\_34}{FrohnerTR19}~\cite{FrohnerTR19} & \href{works/FrohnerTR19.pdf}{Casual Employee Scheduling with Constraint Programming and Metaheuristics} &  & benchmark, real-world & 0 &  &  &  &  &  &  & \ref{a:FrohnerTR19} & \ref{b:FrohnerTR19}\\
\rowlabel{c:GalleguillosKSB19}GalleguillosKSB19 \href{https://doi.org/10.1007/978-3-030-30048-7\_26}{GalleguillosKSB19}~\cite{GalleguillosKSB19} & \href{works/GalleguillosKSB19.pdf}{Constraint Programming-Based Job Dispatching for Modern {HPC} Applications} & \su{OR-Tools} &  & 5 &  &  & \href{https://github.com/cgalleguillosm/cp_dispatchers}{y} &  & on-line dispatch &  & \ref{a:GalleguillosKSB19} & \ref{b:GalleguillosKSB19}\\
\rowlabel{c:GeibingerMM19}GeibingerMM19 \href{https://doi.org/10.1007/978-3-030-19212-9\_20}{GeibingerMM19}~\cite{GeibingerMM19} & \href{works/GeibingerMM19.pdf}{Investigating Constraint Programming for Real World Industrial Test Laboratory Scheduling} &  & real-life, generated instance, industrial partner, real-world, benchmark & 3 &  &  &  &  &  &  & \ref{a:GeibingerMM19} & \ref{b:GeibingerMM19}\\
\rowlabel{c:KucukY19}KucukY19 \href{https://api.semanticscholar.org/CorpusID:198146161}{KucukY19}~\cite{KucukY19} & \href{works/KucukY19.pdf}{A Constraint Programming Approach for Agile Earth Observation Satellite Scheduling Problem} &  & benchmark, generated instance & 0 &  &  &  &  &  &  & \ref{a:KucukY19} & \ref{b:KucukY19}\\
\rowlabel{c:LiuLH19}LiuLH19 \href{https://doi.org/10.1007/978-3-030-19823-7\_19}{LiuLH19}~\cite{LiuLH19} & \href{works/LiuLH19.pdf}{Solving the Talent Scheduling Problem by Parallel Constraint Programming} &  & CSPlib, benchmark & 0 &  &  &  &  &  &  & \ref{a:LiuLH19} & \ref{b:LiuLH19}\\
\rowlabel{c:MalapertN19}MalapertN19 \href{https://doi.org/10.1007/978-3-030-19212-9\_28}{MalapertN19}~\cite{MalapertN19} & \href{works/MalapertN19.pdf}{A New CP-Approach for a Parallel Machine Scheduling Problem with Time Constraints on Machine Qualifications} &  & generated instance, benchmark, industrial instance, Roadef & 3 &  &  &  &  &  &  & \ref{a:MalapertN19} & \ref{b:MalapertN19}\\
\rowlabel{c:MurinR19}MurinR19 \href{https://doi.org/10.1007/978-3-030-30048-7\_27}{MurinR19}~\cite{MurinR19} & \href{works/MurinR19.pdf}{Scheduling of Mobile Robots Using Constraint Programming} & \su{{CP Opt} Cplex OPL} & real-life, benchmark, github & 3 & \href{https://github.com/StanislavMurin/Scheduling-of-Mobile-Robots-using-Constraint-Programming}{y} &  & \href{https://github.com/StanislavMurin/Scheduling-of-Mobile-Robots-using-Constraint-Programming}{y} &  & JSPT & \su{endBeforeStart alternative noOverlap} & \ref{a:MurinR19} & \ref{b:MurinR19}\\
\rowlabel{c:ParkUJR19}ParkUJR19 \href{https://doi.org/10.1007/978-3-030-19648-6\_15}{ParkUJR19}~\cite{ParkUJR19} & \href{works/ParkUJR19.pdf}{Developing a Production Scheduling System for Modular Factory Using Constraint Programming} &  & real-world & 0 &  &  &  &  &  &  & \ref{a:ParkUJR19} & \ref{b:ParkUJR19}\\
\rowlabel{c:Tom19}Tom19 \href{https://doi.org/10.1109/FUZZ-IEEE.2019.8859029}{Tom19}~\cite{Tom19} & \href{works/Tom19.pdf}{Fuzzy Multi-Constraint Programming Model for Weekly Meals Scheduling} &  & real-world & 0 &  &  &  &  &  &  & \ref{a:Tom19} & \ref{b:Tom19}\\
\rowlabel{c:YangSS19}YangSS19 \href{https://doi.org/10.1007/978-3-030-19212-9\_42}{YangSS19}~\cite{YangSS19} & \href{works/YangSS19.pdf}{Time Table Edge Finding with Energy Variables} &  & generated instance & 1 &  &  &  &  &  &  & \ref{a:YangSS19} & \ref{b:YangSS19}\\
\rowlabel{c:ArbaouiY18}ArbaouiY18 \href{https://doi.org/10.1007/978-3-319-75420-8\_67}{ArbaouiY18}~\cite{ArbaouiY18} & \href{works/ArbaouiY18.pdf}{Solving the Unrelated Parallel Machine Scheduling Problem with Additional Resources Using Constraint Programming} &  & benchmark & 0 &  &  &  &  &  &  & \ref{a:ArbaouiY18} & \ref{b:ArbaouiY18}\\
\rowlabel{c:AstrandJZ18}AstrandJZ18 \href{https://doi.org/10.1007/978-3-319-93031-2\_44}{AstrandJZ18}~\cite{AstrandJZ18} & \href{works/AstrandJZ18.pdf}{Fleet Scheduling in Underground Mines Using Constraint Programming} &  &  & 0 &  &  &  &  &  &  & \ref{a:AstrandJZ18} & \ref{b:AstrandJZ18}\\
\rowlabel{c:BenediktSMVH18}BenediktSMVH18 \href{https://doi.org/10.1007/978-3-319-93031-2\_6}{BenediktSMVH18}~\cite{BenediktSMVH18} & \href{works/BenediktSMVH18.pdf}{Energy-Aware Production Scheduling with Power-Saving Modes} & \su{CPO Gurobi} & github, random instance, generated instance & 1 & \href{https://github.com/CTU-IIG/PSPSM}{y} &  & \href{https://github.com/CTU-IIG/PSPSM}{y} & - & Energy Aware Production Scheduling &  & \ref{a:BenediktSMVH18} & \ref{b:BenediktSMVH18}\\
\rowlabel{c:CappartTSR18}CappartTSR18 \href{https://doi.org/10.1007/978-3-319-98334-9\_32}{CappartTSR18}~\cite{CappartTSR18} & \href{works/CappartTSR18.pdf}{A Constraint Programming Approach for Solving Patient Transportation Problems} &  & bitbucket, CSPlib, real-life & 1 &  &  &  &  &  &  & \ref{a:CappartTSR18} & \ref{b:CappartTSR18}\\
\rowlabel{c:DemirovicS18}DemirovicS18 \href{https://doi.org/10.1007/978-3-319-93031-2\_10}{DemirovicS18}~\cite{DemirovicS18} & \href{works/DemirovicS18.pdf}{Constraint Programming for High School Timetabling: {A} Scheduling-Based Model with Hot Starts} &  & real-world, benchmark & 5 &  &  &  &  &  &  & \ref{a:DemirovicS18} & \ref{b:DemirovicS18}\\
\rowlabel{c:He0GLW18}He0GLW18 \href{https://doi.org/10.1007/978-3-319-98334-9\_42}{He0GLW18}~\cite{He0GLW18} & \href{works/He0GLW18.pdf}{A Fast and Scalable Algorithm for Scheduling Large Numbers of Devices Under Real-Time Pricing} & \su{Gurobi Python} & real-world, bitbucket & 8 & \href{https://bitbucket.org/monash-dr/deterministic-rtp-ad/src/master/}{y} &  & \href{https://bitbucket.org/monash-dr/deterministic-rtp-ad/src/master/}{y} & - & \su{FSDN-DS DSP-MH-RTP} &  & \ref{a:He0GLW18} & \ref{b:He0GLW18}\\
\rowlabel{c:HoYCLLCLC18}HoYCLLCLC18 \href{https://doi.org/10.1145/3299819.3299825}{HoYCLLCLC18}~\cite{HoYCLLCLC18} & \href{works/HoYCLLCLC18.pdf}{A Platform for Dynamic Optimal Nurse Scheduling Based on Integer Linear Programming along with Multiple Criteria Constraints} &  & real-world & 0 &  &  &  &  &  &  & \ref{a:HoYCLLCLC18} & \ref{b:HoYCLLCLC18}\\
\rowlabel{c:KameugneFGOQ18}KameugneFGOQ18 \href{https://doi.org/10.1007/978-3-319-93031-2\_23}{KameugneFGOQ18}~\cite{KameugneFGOQ18} & \href{works/KameugneFGOQ18.pdf}{Horizontally Elastic Not-First/Not-Last Filtering Algorithm for Cumulative Resource Constraint} &  & benchmark, real-world & 0 &  &  &  &  &  &  & \ref{a:KameugneFGOQ18} & \ref{b:KameugneFGOQ18}\\
\rowlabel{c:Laborie18a}Laborie18a \href{https://doi.org/10.1007/978-3-319-93031-2\_29}{Laborie18a}~\cite{Laborie18a} & \href{works/Laborie18a.pdf}{An Update on the Comparison of MIP, {CP} and Hybrid Approaches for Mixed Resource Allocation and Scheduling} &  & real-life, benchmark, real-world & 0 &  &  &  &  &  &  & \ref{a:Laborie18a} & \ref{b:Laborie18a}\\
\rowlabel{c:MusliuSS18}MusliuSS18 \href{https://doi.org/10.1007/978-3-319-93031-2\_31}{MusliuSS18}~\cite{MusliuSS18} & \href{works/MusliuSS18.pdf}{Solver Independent Rotating Workforce Scheduling} &  & generated instance, benchmark, real-life & 2 &  &  &  &  &  &  & \ref{a:MusliuSS18} & \ref{b:MusliuSS18}\\
\rowlabel{c:NishikawaSTT18}NishikawaSTT18 \href{https://doi.org/10.1109/CANDAR.2018.00025}{NishikawaSTT18}~\cite{NishikawaSTT18} & \href{works/NishikawaSTT18.pdf}{Scheduling of Malleable Fork-Join Tasks with Constraint Programming} &  & real-world, benchmark & 0 &  &  &  &  &  &  & \ref{a:NishikawaSTT18} & \ref{b:NishikawaSTT18}\\
\rowlabel{c:NishikawaSTT18a}NishikawaSTT18a \href{https://doi.org/10.1109/TENCON.2018.8650168}{NishikawaSTT18a}~\cite{NishikawaSTT18a} & \href{works/NishikawaSTT18a.pdf}{Scheduling of Malleable Tasks Based on Constraint Programming} &  & real-world, benchmark, real-life & 0 &  &  &  &  &  &  & \ref{a:NishikawaSTT18a} & \ref{b:NishikawaSTT18a}\\
\rowlabel{c:OuelletQ18}OuelletQ18 \href{https://doi.org/10.1007/978-3-319-93031-2\_34}{OuelletQ18}~\cite{OuelletQ18} & \href{works/OuelletQ18.pdf}{A O(n {\textbackslash}log {\^{}}2 n) Checker and O(n{\^{}}2 {\textbackslash}log n) Filtering Algorithm for the Energetic Reasoning} &  & benchmark, Roadef & 0 &  &  &  &  &  &  & \ref{a:OuelletQ18} & \ref{b:OuelletQ18}\\
\rowlabel{c:RiahiNS018}RiahiNS018 \href{https://aaai.org/ocs/index.php/ICAPS/ICAPS18/paper/view/17755}{RiahiNS018}~\cite{RiahiNS018} & \href{works/RiahiNS018.pdf}{Local Search for Flowshops with Setup Times and Blocking Constraints} &  & real-world, real-life, benchmark & 0 &  &  &  &  &  &  & \ref{a:RiahiNS018} & \ref{b:RiahiNS018}\\
\rowlabel{c:Tesch18}Tesch18 \href{https://doi.org/10.1007/978-3-319-98334-9\_41}{Tesch18}~\cite{Tesch18} & \href{works/Tesch18.pdf}{Improving Energetic Propagations for Cumulative Scheduling} &  & Roadef & 0 &  &  &  &  &  &  & \ref{a:Tesch18} & \ref{b:Tesch18}\\
\rowlabel{c:BofillCSV17}BofillCSV17 \href{https://doi.org/10.1007/978-3-319-66158-2\_5}{BofillCSV17}~\cite{BofillCSV17} & \href{works/BofillCSV17.pdf}{An Efficient {SMT} Approach to Solve MRCPSP/max Instances with Tight Constraints on Resources} &  & benchmark & 2 &  &  &  &  &  &  & \ref{a:BofillCSV17} & \ref{b:BofillCSV17}\\
\rowlabel{c:CappartS17}CappartS17 \href{https://doi.org/10.1007/978-3-319-59776-8\_26}{CappartS17}~\cite{CappartS17} & \href{works/CappartS17.pdf}{Rescheduling Railway Traffic on Real Time Situations Using Time-Interval Variables} & CPO & bitbucket, random instance, real-life & 1 & \href{https://bitbucket.org/qcappart/qcappart_opendata/src/master/}{y} &  & n & - & Rescheduling Railway Traffic &  & \ref{a:CappartS17} & \ref{b:CappartS17}\\
\rowlabel{c:CohenHB17}CohenHB17 \href{https://doi.org/10.1007/978-3-319-66263-3\_10}{CohenHB17}~\cite{CohenHB17} & \href{works/CohenHB17.pdf}{{(I} Can Get) Satisfaction: Preference-Based Scheduling for Concert-Goers at Multi-venue Music Festivals} &  &  & 12 &  &  &  &  &  &  & \ref{a:CohenHB17} & \ref{b:CohenHB17}\\
\rowlabel{c:GelainPRVW17}GelainPRVW17 \href{https://doi.org/10.1007/978-3-319-59776-8\_32}{GelainPRVW17}~\cite{GelainPRVW17} & \href{works/GelainPRVW17.pdf}{A Local Search Approach for Incomplete Soft Constraint Problems: Experimental Results on Meeting Scheduling Problems} &  & CSPlib, real-life, benchmark & 2 &  &  &  &  &  &  & \ref{a:GelainPRVW17} & \ref{b:GelainPRVW17}\\
\rowlabel{c:GoldwaserS17}GoldwaserS17 \href{https://doi.org/10.1007/978-3-319-66158-2\_22}{GoldwaserS17}~\cite{GoldwaserS17} & \href{works/GoldwaserS17.pdf}{Optimal Torpedo Scheduling} & \su{Chuffed Gurobi} & instance generator, github, generated instance & 4 & \href{https://github.com/AdGold/TorpedoSchedulingInstances}{y} &  & n & - & Torpedo Scheduling &  & \ref{a:GoldwaserS17} & \ref{b:GoldwaserS17}\\
\rowlabel{c:Hooker17}Hooker17 \href{https://doi.org/10.1007/978-3-319-66158-2\_36}{Hooker17}~\cite{Hooker17} & \href{works/Hooker17.pdf}{Job Sequencing Bounds from Decision Diagrams} &  & benchmark, random instance & 0 &  &  &  &  &  &  & \ref{a:Hooker17} & \ref{b:Hooker17}\\
\rowlabel{c:KletzanderM17}KletzanderM17 \href{https://doi.org/10.1007/978-3-319-59776-8\_28}{KletzanderM17}~\cite{KletzanderM17} & \href{works/KletzanderM17.pdf}{A Multi-stage Simulated Annealing Algorithm for the Torpedo Scheduling Problem} &  &  & 2 &  &  &  &  &  &  & \ref{a:KletzanderM17} & \ref{b:KletzanderM17}\\
\rowlabel{c:LiuCGM17}LiuCGM17 \href{https://doi.org/10.1007/978-3-319-66158-2\_24}{LiuCGM17}~\cite{LiuCGM17} & \href{works/LiuCGM17.pdf}{NightSplitter: {A} Scheduling Tool to Optimize (Sub)group Activities} & \su{Chuffed OR-Tools HCSP SA} & github & 11 & n &  & \href{https://cs.unibo.it/t.liu/nightsplitter/mzn.html} & - & NightSplit &  & \ref{a:LiuCGM17} & \ref{b:LiuCGM17}\\
\rowlabel{c:Madi-WambaLOBM17}Madi-WambaLOBM17 \href{https://doi.org/10.1109/ICPADS.2017.00089}{Madi-WambaLOBM17}~\cite{Madi-WambaLOBM17} & \href{works/Madi-WambaLOBM17.pdf}{Green Energy Aware Scheduling Problem in Virtualized Datacenters} &  & real-world & 0 &  &  &  &  &  &  & \ref{a:Madi-WambaLOBM17} & \ref{b:Madi-WambaLOBM17}\\
\rowlabel{c:MossigeGSMC17}MossigeGSMC17 \href{https://doi.org/10.1007/978-3-319-66158-2\_25}{MossigeGSMC17}~\cite{MossigeGSMC17} & \href{works/MossigeGSMC17.pdf}{Time-Aware Test Case Execution Scheduling for Cyber-Physical Systems} &  & industrial partner, real-world, benchmark, random instance, CSPlib, generated instance & 4 &  &  &  &  &  &  & \ref{a:MossigeGSMC17} & \ref{b:MossigeGSMC17}\\
\rowlabel{c:Pralet17}Pralet17 \href{https://doi.org/10.1007/978-3-319-66158-2\_16}{Pralet17}~\cite{Pralet17} & \href{works/Pralet17.pdf}{An Incomplete Constraint-Based System for Scheduling with Renewable Resources} &  & benchmark & 1 &  &  &  &  &  &  & \ref{a:Pralet17} & \ref{b:Pralet17}\\
\rowlabel{c:TranVNB17a}TranVNB17a \href{https://doi.org/10.24963/ijcai.2017/726}{TranVNB17a}~\cite{TranVNB17a} & \href{works/TranVNB17a.pdf}{Robots in Retirement Homes: Applying Off-the-Shelf Planning and Scheduling to a Team of Assistive Robots (Extended Abstract)} &  & real-world & 0 &  &  &  &  &  &  & \ref{a:TranVNB17a} & \ref{b:TranVNB17a}\\
\rowlabel{c:YoungFS17}YoungFS17 \href{https://doi.org/10.1007/978-3-319-66158-2\_20}{YoungFS17}~\cite{YoungFS17} & \href{works/YoungFS17.pdf}{Constraint Programming Applied to the Multi-Skill Project Scheduling Problem} &  & benchmark, github, instance generator & 6 &  &  &  &  &  &  & \ref{a:YoungFS17} & \ref{b:YoungFS17}\\
\rowlabel{c:BonfiettiZLM16}BonfiettiZLM16 \href{https://doi.org/10.1007/978-3-319-44953-1\_8}{BonfiettiZLM16}~\cite{BonfiettiZLM16} & \href{works/BonfiettiZLM16.pdf}{The Multirate Resource Constraint} &  & generated instance, github, industrial instance, benchmark, real-world & 1 &  &  &  &  &  &  & \ref{a:BonfiettiZLM16} & \ref{b:BonfiettiZLM16}\\
\rowlabel{c:BoothNB16}BoothNB16 \href{https://doi.org/10.1007/978-3-319-44953-1\_34}{BoothNB16}~\cite{BoothNB16} & \href{works/BoothNB16.pdf}{A Constraint Programming Approach to Multi-Robot Task Allocation and Scheduling in Retirement Homes} &  & real-world & 0 &  &  &  &  &  &  & \ref{a:BoothNB16} & \ref{b:BoothNB16}\\
\rowlabel{c:BridiLBBM16}BridiLBBM16 \href{https://doi.org/10.3233/978-1-61499-672-9-1598}{BridiLBBM16}~\cite{BridiLBBM16} & \href{works/BridiLBBM16.pdf}{{DARDIS:} Distributed And Randomized DIspatching and Scheduling} &  &  & 0 &  &  &  &  &  &  & \ref{a:BridiLBBM16} & \ref{b:BridiLBBM16}\\
\rowlabel{c:CauwelaertDMS16}CauwelaertDMS16 \href{https://doi.org/10.1007/978-3-319-44953-1\_33}{CauwelaertDMS16}~\cite{CauwelaertDMS16} & \href{works/CauwelaertDMS16.pdf}{Efficient Filtering for the Unary Resource with Family-Based Transition Times} &  & real-life, bitbucket, benchmark & 2 &  &  &  &  &  &  & \ref{a:CauwelaertDMS16} & \ref{b:CauwelaertDMS16}\\
\rowlabel{c:FontaineMH16}FontaineMH16 \href{https://doi.org/10.1007/978-3-319-33954-2\_12}{FontaineMH16}~\cite{FontaineMH16} & \href{works/FontaineMH16.pdf}{Parallel Composition of Scheduling Solvers} &  & benchmark & 2 &  &  &  &  &  &  & \ref{a:FontaineMH16} & \ref{b:FontaineMH16}\\
\rowlabel{c:GilesH16}GilesH16 \href{https://doi.org/10.1007/978-3-319-44953-1\_38}{GilesH16}~\cite{GilesH16} & \href{works/GilesH16.pdf}{Solving a Supply-Delivery Scheduling Problem with Constraint Programming} &  &  & 0 &  &  &  &  &  &  & \ref{a:GilesH16} & \ref{b:GilesH16}\\
\rowlabel{c:GingrasQ16}GingrasQ16 \href{http://www.ijcai.org/Abstract/16/440}{GingrasQ16}~\cite{GingrasQ16} & \href{works/GingrasQ16.pdf}{Generalizing the Edge-Finder Rule for the Cumulative Constraint} &  & benchmark & 0 &  &  &  &  &  &  & \ref{a:GingrasQ16} & \ref{b:GingrasQ16}\\
\rowlabel{c:HechingH16}HechingH16 \href{https://doi.org/10.1007/978-3-319-33954-2\_14}{HechingH16}~\cite{HechingH16} & \href{works/HechingH16.pdf}{Scheduling Home Hospice Care with Logic-Based Benders Decomposition} &  & real-world & 0 &  &  &  &  &  &  & \ref{a:HechingH16} & \ref{b:HechingH16}\\
\rowlabel{c:JelinekB16}JelinekB16 \href{https://doi.org/10.1007/978-3-319-28228-2\_1}{JelinekB16}~\cite{JelinekB16} & \href{works/JelinekB16.pdf}{Using Constraint Logic Programming to Schedule Solar Array Operations on the International Space Station} &  & real-life & 2 &  &  &  &  &  &  & \ref{a:JelinekB16} & \ref{b:JelinekB16}\\
\rowlabel{c:LimHTB16}LimHTB16 \href{https://doi.org/10.1007/978-3-319-44953-1\_43}{LimHTB16}~\cite{LimHTB16} & \href{works/LimHTB16.pdf}{Online HVAC-Aware Occupancy Scheduling with Adaptive Temperature Control} &  & real-world & 4 &  &  &  &  &  &  & \ref{a:LimHTB16} & \ref{b:LimHTB16}\\
\rowlabel{c:LuoVLBM16}LuoVLBM16 \href{http://www.aaai.org/ocs/index.php/KR/KR16/paper/view/12909}{LuoVLBM16}~\cite{LuoVLBM16} & \href{works/LuoVLBM16.pdf}{Using Metric Temporal Logic to Specify Scheduling Problems} &  &  & 0 &  &  &  &  &  &  & \ref{a:LuoVLBM16} & \ref{b:LuoVLBM16}\\
\rowlabel{c:Madi-WambaB16}Madi-WambaB16 \href{https://doi.org/10.1007/978-3-319-33954-2\_18}{Madi-WambaB16}~\cite{Madi-WambaB16} & \href{works/Madi-WambaB16.pdf}{The TaskIntersection Constraint} &  & real-world, benchmark, random instance, generated instance & 3 &  &  &  &  &  &  & \ref{a:Madi-WambaB16} & \ref{b:Madi-WambaB16}\\
\rowlabel{c:SchuttS16}SchuttS16 \href{https://doi.org/10.1007/978-3-319-44953-1\_28}{SchuttS16}~\cite{SchuttS16} & \href{works/SchuttS16.pdf}{Explaining Producer/Consumer Constraints} &  & benchmark & 1 &  &  &  &  &  &  & \ref{a:SchuttS16} & \ref{b:SchuttS16}\\
\rowlabel{c:SzerediS16}SzerediS16 \href{https://doi.org/10.1007/978-3-319-44953-1\_31}{SzerediS16}~\cite{SzerediS16} & \href{works/SzerediS16.pdf}{Modelling and Solving Multi-mode Resource-Constrained Project Scheduling} &  & benchmark & 2 &  &  &  &  &  &  & \ref{a:SzerediS16} & \ref{b:SzerediS16}\\
\rowlabel{c:Tesch16}Tesch16 \href{https://doi.org/10.1007/978-3-319-44953-1\_32}{Tesch16}~\cite{Tesch16} & \href{works/Tesch16.pdf}{A Nearly Exact Propagation Algorithm for Energetic Reasoning in {\textbackslash}mathcal O(n{\^{}}2 {\textbackslash}log n)} &  & Roadef & 1 &  &  &  &  &  &  & \ref{a:Tesch16} & \ref{b:Tesch16}\\
\rowlabel{c:TranDRFWOVB16}TranDRFWOVB16 \href{https://doi.org/10.1609/socs.v7i1.18390}{TranDRFWOVB16}~\cite{TranDRFWOVB16} & \href{works/TranDRFWOVB16.pdf}{A Hybrid Quantum-Classical Approach to Solving Scheduling Problems} &  &  & 0 &  &  &  &  &  &  & \ref{a:TranDRFWOVB16} & \ref{b:TranDRFWOVB16}\\
\rowlabel{c:TranWDRFOVB16}TranWDRFOVB16 \href{http://www.aaai.org/ocs/index.php/WS/AAAIW16/paper/view/12664}{TranWDRFOVB16}~\cite{TranWDRFOVB16} & \href{works/TranWDRFOVB16.pdf}{Explorations of Quantum-Classical Approaches to Scheduling a Mars Lander Activity Problem} &  & benchmark & 0 &  &  &  &  &  &  & \ref{a:TranWDRFOVB16} & \ref{b:TranWDRFOVB16}\\
\rowlabel{c:BartakV15}BartakV15 \href{}{BartakV15}~\cite{BartakV15} & \href{works/BartakV15.pdf}{Reactive Recovery from Machine Breakdown in Production Scheduling with Temporal Distance and Resource Constraints} &  & real-world, real-life & 0 &  &  &  &  &  &  & \ref{a:BartakV15} & \ref{b:BartakV15}\\
\rowlabel{c:BofillGSV15}BofillGSV15 \href{https://doi.org/10.1007/978-3-319-18008-3\_5}{BofillGSV15}~\cite{BofillGSV15} & \href{works/BofillGSV15.pdf}{MaxSAT-Based Scheduling of {B2B} Meetings} &  & industrial instance & 3 &  &  &  &  &  &  & \ref{a:BofillGSV15} & \ref{b:BofillGSV15}\\
\rowlabel{c:BurtLPS15}BurtLPS15 \href{https://doi.org/10.1007/978-3-319-18008-3\_7}{BurtLPS15}~\cite{BurtLPS15} & \href{works/BurtLPS15.pdf}{Scheduling with Fixed Maintenance, Shared Resources and Nonlinear Feedrate Constraints: {A} Mine Planning Case Study} &  & real-world, benchmark, industry partner & 5 &  &  &  &  &  &  & \ref{a:BurtLPS15} & \ref{b:BurtLPS15}\\
\rowlabel{c:DejemeppeCS15}DejemeppeCS15 \href{https://doi.org/10.1007/978-3-319-23219-5\_7}{DejemeppeCS15}~\cite{DejemeppeCS15} & \href{works/DejemeppeCS15.pdf}{The Unary Resource with Transition Times} &  & real-world, bitbucket, generated instance, benchmark & 4 &  &  &  &  &  &  & \ref{a:DejemeppeCS15} & \ref{b:DejemeppeCS15}\\
\rowlabel{c:EvenSH15}EvenSH15 \href{https://doi.org/10.1007/978-3-319-23219-5\_40}{EvenSH15}~\cite{EvenSH15} & \href{works/EvenSH15.pdf}{A Constraint Programming Approach for Non-preemptive Evacuation Scheduling} &  & real-life, real-world & 0 &  &  &  &  &  &  & \ref{a:EvenSH15} & \ref{b:EvenSH15}\\
\rowlabel{c:GayHLS15}GayHLS15 \href{https://doi.org/10.1007/978-3-319-23219-5\_10}{GayHLS15}~\cite{GayHLS15} & \href{works/GayHLS15.pdf}{Conflict Ordering Search for Scheduling Problems} &  & benchmark, bitbucket & 0 &  &  &  &  &  &  & \ref{a:GayHLS15} & \ref{b:GayHLS15}\\
\rowlabel{c:GayHS15}GayHS15 \href{https://doi.org/10.1007/978-3-319-23219-5\_11}{GayHS15}~\cite{GayHS15} & \href{works/GayHS15.pdf}{Simple and Scalable Time-Table Filtering for the Cumulative Constraint} &  & bitbucket & 2 &  &  &  &  &  &  & \ref{a:GayHS15} & \ref{b:GayHS15}\\
\rowlabel{c:GayHS15a}GayHS15a \href{https://doi.org/10.1007/978-3-319-18008-3\_11}{GayHS15a}~\cite{GayHS15a} & \href{works/GayHS15a.pdf}{Time-Table Disjunctive Reasoning for the Cumulative Constraint} &  & benchmark, bitbucket, real-world & 0 &  &  &  &  &  &  & \ref{a:GayHS15a} & \ref{b:GayHS15a}\\
\rowlabel{c:KreterSS15}KreterSS15 \href{https://doi.org/10.1007/978-3-319-23219-5\_19}{KreterSS15}~\cite{KreterSS15} & \href{works/KreterSS15.pdf}{Modeling and Solving Project Scheduling with Calendars} &  & benchmark & 3 &  &  &  &  &  &  & \ref{a:KreterSS15} & \ref{b:KreterSS15}\\
\rowlabel{c:LimBTBB15}LimBTBB15 \href{https://doi.org/10.1007/978-3-319-18008-3\_17}{LimBTBB15}~\cite{LimBTBB15} & \href{works/LimBTBB15.pdf}{Large Neighborhood Search for Energy Aware Meeting Scheduling in Smart Buildings} &  & benchmark & 3 &  &  &  &  &  &  & \ref{a:LimBTBB15} & \ref{b:LimBTBB15}\\
\rowlabel{c:LombardiBM15}LombardiBM15 \href{https://doi.org/10.1007/978-3-319-23219-5\_20}{LombardiBM15}~\cite{LombardiBM15} & \href{works/LombardiBM15.pdf}{Deterministic Estimation of the Expected Makespan of a {POS} Under Duration Uncertainty} &  & benchmark, real-world & 0 &  &  &  &  &  &  & \ref{a:LombardiBM15} & \ref{b:LombardiBM15}\\
\rowlabel{c:MelgarejoLS15}MelgarejoLS15 \href{https://doi.org/10.1007/978-3-319-18008-3\_1}{MelgarejoLS15}~\cite{MelgarejoLS15} & \href{works/MelgarejoLS15.pdf}{A Time-Dependent No-Overlap Constraint: Application to Urban Delivery Problems} &  & real-world, benchmark & 1 &  &  &  &  &  &  & \ref{a:MelgarejoLS15} & \ref{b:MelgarejoLS15}\\
\rowlabel{c:MurphyMB15}MurphyMB15 \href{https://doi.org/10.1007/978-3-319-23219-5\_47}{MurphyMB15}~\cite{MurphyMB15} & \href{works/MurphyMB15.pdf}{Design and Evaluation of a Constraint-Based Energy Saving and Scheduling Recommender System} &  & real-world & 3 &  &  &  &  &  &  & \ref{a:MurphyMB15} & \ref{b:MurphyMB15}\\
\rowlabel{c:PesantRR15}PesantRR15 \href{https://doi.org/10.1007/978-3-319-18008-3\_21}{PesantRR15}~\cite{PesantRR15} & \href{works/PesantRR15.pdf}{A Comparative Study of {MIP} and {CP} Formulations for the {B2B} Scheduling Optimization Problem} &  &  & 1 &  &  &  &  &  &  & \ref{a:PesantRR15} & \ref{b:PesantRR15}\\
\rowlabel{c:PraletLJ15}PraletLJ15 \href{https://doi.org/10.1007/978-3-319-23219-5\_48}{PraletLJ15}~\cite{PraletLJ15} & \href{works/PraletLJ15.pdf}{Scheduling Running Modes of Satellite Instruments Using Constraint-Based Local Search} &  &  & 0 &  &  &  &  &  &  & \ref{a:PraletLJ15} & \ref{b:PraletLJ15}\\
\rowlabel{c:SialaAH15}SialaAH15 \href{https://doi.org/10.1007/978-3-319-23219-5\_28}{SialaAH15}~\cite{SialaAH15} & \href{works/SialaAH15.pdf}{Two Clause Learning Approaches for Disjunctive Scheduling} &  & github, benchmark & 5 &  &  &  &  &  &  & \ref{a:SialaAH15} & \ref{b:SialaAH15}\\
\rowlabel{c:VilimLS15}VilimLS15 \href{https://doi.org/10.1007/978-3-319-18008-3\_30}{VilimLS15}~\cite{VilimLS15} & \href{works/VilimLS15.pdf}{Failure-Directed Search for Constraint-Based Scheduling} &  & benchmark & 8 &  &  &  &  &  &  & \ref{a:VilimLS15} & \ref{b:VilimLS15}\\
\rowlabel{c:ZhouGL15}ZhouGL15 \href{https://doi.org/10.1109/FSKD.2015.7382064}{ZhouGL15}~\cite{ZhouGL15} & \href{works/ZhouGL15.pdf}{On complex hybrid flexible flowshop scheduling problems based on constraint programming} &  & real-world & 0 &  &  &  &  &  &  & \ref{a:ZhouGL15} & \ref{b:ZhouGL15}\\
\rowlabel{c:AlesioNBG14}AlesioNBG14 \href{https://doi.org/10.1007/978-3-319-10428-7\_58}{AlesioNBG14}~\cite{AlesioNBG14} & \href{works/AlesioNBG14.pdf}{Worst-Case Scheduling of Software Tasks - {A} Constraint Optimization Model to Support Performance Testing} &  & benchmark & 2 &  &  &  &  &  &  & \ref{a:AlesioNBG14} & \ref{b:AlesioNBG14}\\
\rowlabel{c:BartoliniBBLM14}BartoliniBBLM14 \href{https://doi.org/10.1007/978-3-319-10428-7\_55}{BartoliniBBLM14}~\cite{BartoliniBBLM14} & \href{works/BartoliniBBLM14.pdf}{Proactive Workload Dispatching on the {EURORA} Supercomputer} &  &  & 4 &  &  &  &  &  &  & \ref{a:BartoliniBBLM14} & \ref{b:BartoliniBBLM14}\\
\rowlabel{c:BessiereHMQW14}BessiereHMQW14 \href{https://doi.org/10.1007/978-3-319-07046-9\_23}{BessiereHMQW14}~\cite{BessiereHMQW14} & \href{works/BessiereHMQW14.pdf}{Buffered Resource Constraint: Algorithms and Complexity} &  & benchmark, real-life & 0 &  &  &  &  &  &  & \ref{a:BessiereHMQW14} & \ref{b:BessiereHMQW14}\\
\rowlabel{c:BofillEGPSV14}BofillEGPSV14 \href{https://doi.org/10.1007/978-3-319-10428-7\_56}{BofillEGPSV14}~\cite{BofillEGPSV14} & \href{works/BofillEGPSV14.pdf}{Scheduling {B2B} Meetings} &  & industrial instance & 6 &  &  &  &  &  &  & \ref{a:BofillEGPSV14} & \ref{b:BofillEGPSV14}\\
\rowlabel{c:BonfiettiLM14}BonfiettiLM14 \href{https://doi.org/10.1007/978-3-319-07046-9\_15}{BonfiettiLM14}~\cite{BonfiettiLM14} & \href{works/BonfiettiLM14.pdf}{Disregarding Duration Uncertainty in Partial Order Schedules? Yes, We Can!} &  & real-world, benchmark & 2 &  &  &  &  &  &  & \ref{a:BonfiettiLM14} & \ref{b:BonfiettiLM14}\\
\rowlabel{c:DejemeppeD14}DejemeppeD14 \href{https://doi.org/10.1007/978-3-319-07046-9\_20}{DejemeppeD14}~\cite{DejemeppeD14} & \href{works/DejemeppeD14.pdf}{Continuously Degrading Resource and Interval Dependent Activity Durations in Nuclear Medicine Patient Scheduling} &  & bitbucket & 0 &  &  &  &  &  &  & \ref{a:DejemeppeD14} & \ref{b:DejemeppeD14}\\
\rowlabel{c:DerrienP14}DerrienP14 \href{https://doi.org/10.1007/978-3-319-10428-7\_22}{DerrienP14}~\cite{DerrienP14} & \href{works/DerrienP14.pdf}{A New Characterization of Relevant Intervals for Energetic Reasoning} &  & random instance & 0 &  &  &  &  &  &  & \ref{a:DerrienP14} & \ref{b:DerrienP14}\\
\rowlabel{c:DerrienPZ14}DerrienPZ14 \href{https://doi.org/10.1007/978-3-319-10428-7\_23}{DerrienPZ14}~\cite{DerrienPZ14} & \href{works/DerrienPZ14.pdf}{A Declarative Paradigm for Robust Cumulative Scheduling} &  & benchmark, random instance, real-world & 0 &  &  &  &  &  &  & \ref{a:DerrienPZ14} & \ref{b:DerrienPZ14}\\
\rowlabel{c:DoulabiRP14}DoulabiRP14 \href{https://doi.org/10.1007/978-3-319-07046-9\_32}{DoulabiRP14}~\cite{DoulabiRP14} & \href{works/DoulabiRP14.pdf}{A Constraint Programming-Based Column Generation Approach for Operating Room Planning and Scheduling} &  &  & 0 &  &  &  &  &  &  & \ref{a:DoulabiRP14} & \ref{b:DoulabiRP14}\\
\rowlabel{c:FriedrichFMRSST14}FriedrichFMRSST14 \href{https://doi.org/10.1007/978-3-319-28697-6\_23}{FriedrichFMRSST14}~\cite{FriedrichFMRSST14} & \href{}{Representing Production Scheduling with Constraint Answer Set Programming} &  &  & 0 &  &  &  &  &  &  & \ref{a:FriedrichFMRSST14} & No\\
\rowlabel{c:GaySS14}GaySS14 \href{https://doi.org/10.1007/978-3-319-10428-7\_59}{GaySS14}~\cite{GaySS14} & \href{works/GaySS14.pdf}{Continuous Casting Scheduling with Constraint Programming} &  & real-life, CSPlib & 0 &  &  &  &  &  &  & \ref{a:GaySS14} & \ref{b:GaySS14}\\
\rowlabel{c:HoundjiSWD14}HoundjiSWD14 \href{https://doi.org/10.1007/978-3-319-10428-7\_29}{HoundjiSWD14}~\cite{HoundjiSWD14} & \href{works/HoundjiSWD14.pdf}{The StockingCost Constraint} &  & bitbucket, generated instance & 0 &  &  &  &  &  &  & \ref{a:HoundjiSWD14} & \ref{b:HoundjiSWD14}\\
\rowlabel{c:KoschB14}KoschB14 \href{https://doi.org/10.1007/978-3-319-07046-9\_5}{KoschB14}~\cite{KoschB14} & \href{works/KoschB14.pdf}{A New {MIP} Model for Parallel-Batch Scheduling with Non-identical Job Sizes} &  & benchmark & 0 &  &  &  &  &  &  & \ref{a:KoschB14} & \ref{b:KoschB14}\\
\rowlabel{c:LipovetzkyBPS14}LipovetzkyBPS14 \href{http://www.aaai.org/ocs/index.php/ICAPS/ICAPS14/paper/view/7942}{LipovetzkyBPS14}~\cite{LipovetzkyBPS14} & \href{works/LipovetzkyBPS14.pdf}{Planning for Mining Operations with Time and Resource Constraints} &  & industrial partner, real-life, industry partner, real-world, benchmark, generated instance & 0 &  &  &  &  &  &  & \ref{a:LipovetzkyBPS14} & \ref{b:LipovetzkyBPS14}\\
\rowlabel{c:LouieVNB14}LouieVNB14 \href{https://doi.org/10.1109/ICRA.2014.6907637}{LouieVNB14}~\cite{LouieVNB14} & \href{}{An autonomous assistive robot for planning, scheduling and facilitating multi-user activities} &  &  & 0 &  &  &  &  &  &  & \ref{a:LouieVNB14} & No\\
\rowlabel{c:BonfiettiLM13}BonfiettiLM13 \href{http://www.aaai.org/ocs/index.php/ICAPS/ICAPS13/paper/view/6050}{BonfiettiLM13}~\cite{BonfiettiLM13} & \href{works/BonfiettiLM13.pdf}{De-Cycling Cyclic Scheduling Problems} &  &  & 0 &  &  &  &  &  &  & \ref{a:BonfiettiLM13} & \ref{b:BonfiettiLM13}\\
\rowlabel{c:ChuGNSW13}ChuGNSW13 \href{http://www.aaai.org/ocs/index.php/IJCAI/IJCAI13/paper/view/6878}{ChuGNSW13}~\cite{ChuGNSW13} & \href{works/ChuGNSW13.pdf}{On the Complexity of Global Scheduling Constraints under Structural Restrictions} &  &  & 0 &  &  &  &  &  &  & \ref{a:ChuGNSW13} & \ref{b:ChuGNSW13}\\
\rowlabel{c:CireCH13}CireCH13 \href{https://doi.org/10.1007/978-3-642-38171-3\_22}{CireCH13}~\cite{CireCH13} & \href{works/CireCH13.pdf}{Mixed Integer Programming vs. Logic-Based Benders Decomposition for Planning and Scheduling} & \su{{CP Opt} Cplex} &  & 1 & dead &  & n & - &  &  & \ref{a:CireCH13} & \ref{b:CireCH13}\\
\rowlabel{c:GuSS13}GuSS13 \href{https://doi.org/10.1007/978-3-642-38171-3\_24}{GuSS13}~\cite{GuSS13} & \href{works/GuSS13.pdf}{A Lagrangian Relaxation Based Forward-Backward Improvement Heuristic for Maximising the Net Present Value of Resource-Constrained Projects} & Chuffed & benchmark & 1 & dead &  &  & - & RCPSPDC & \su{cumulative maxNVPProp} & \ref{a:GuSS13} & \ref{b:GuSS13}\\
\rowlabel{c:HeinzKB13}HeinzKB13 \href{https://doi.org/10.1007/978-3-642-38171-3\_2}{HeinzKB13}~\cite{HeinzKB13} & \href{works/HeinzKB13.pdf}{Recent Improvements Using Constraint Integer Programming for Resource Allocation and Scheduling} &  &  & 0 &  &  &  &  &  &  & \ref{a:HeinzKB13} & \ref{b:HeinzKB13}\\
\rowlabel{c:KelarevaTK13}KelarevaTK13 \href{https://doi.org/10.1007/978-3-642-38171-3\_8}{KelarevaTK13}~\cite{KelarevaTK13} & \href{works/KelarevaTK13.pdf}{{CP} Methods for Scheduling and Routing with Time-Dependent Task Costs} & \su{MiniZinc CPX G12FD} & real-world & 5 & ref &  & - & - & \su{LSFRP BPCTOP} & \su{alldifferent alldifferentExcept0} & \ref{a:KelarevaTK13} & \ref{b:KelarevaTK13}\\
\rowlabel{c:LetortCB13}LetortCB13 \href{https://doi.org/10.1007/978-3-642-38171-3\_10}{LetortCB13}~\cite{LetortCB13} & \href{works/LetortCB13.pdf}{A Synchronized Sweep Algorithm for the \emph{k-dimensional cumulative} Constraint} & \su{SICStus Choco} & Roadef, benchmark, random instance & 2 & PSPlib &  & - & - & RCPSP & \su{cumulative kDimensionalCumulative} & \ref{a:LetortCB13} & \ref{b:LetortCB13}\\
\rowlabel{c:LombardiM13}LombardiM13 \href{http://www.aaai.org/ocs/index.php/ICAPS/ICAPS13/paper/view/6052}{LombardiM13}~\cite{LombardiM13} & \href{works/LombardiM13.pdf}{A Min-Flow Algorithm for Minimal Critical Set Detection in Resource Constrained Project Scheduling} &  &  & 0 &  &  &  &  &  &  & \ref{a:LombardiM13} & \ref{b:LombardiM13}\\
\rowlabel{c:OuelletQ13}OuelletQ13 \href{https://doi.org/10.1007/978-3-642-40627-0\_42}{OuelletQ13}~\cite{OuelletQ13} & \href{works/OuelletQ13.pdf}{Time-Table Extended-Edge-Finding for the Cumulative Constraint} &  & benchmark & 1 &  &  &  &  &  &  & \ref{a:OuelletQ13} & \ref{b:OuelletQ13}\\
\rowlabel{c:SchuttFS13}SchuttFS13 \href{https://doi.org/10.1007/978-3-642-40627-0\_47}{SchuttFS13}~\cite{SchuttFS13} & \href{works/SchuttFS13.pdf}{Scheduling Optional Tasks with Explanation} &  & benchmark & 1 &  &  &  &  &  &  & \ref{a:SchuttFS13} & \ref{b:SchuttFS13}\\
\rowlabel{c:SchuttFS13a}SchuttFS13a \href{https://doi.org/10.1007/978-3-642-38171-3\_16}{SchuttFS13a}~\cite{SchuttFS13a} & \href{works/SchuttFS13a.pdf}{Explaining Time-Table-Edge-Finding Propagation for the Cumulative Resource Constraint} & \su{Mercury G12} & benchmark & 5 & \su{PSPlib AT BL Pack KSD15D PackD} &  & - & - & RCPSP & cumulative & \ref{a:SchuttFS13a} & \ref{b:SchuttFS13a}\\
\rowlabel{c:TranTDB13}TranTDB13 \href{http://www.aaai.org/ocs/index.php/ICAPS/ICAPS13/paper/view/6005}{TranTDB13}~\cite{TranTDB13} & \href{works/TranTDB13.pdf}{Hybrid Queueing Theory and Scheduling Models for Dynamic Environments with Sequence-Dependent Setup Times} &  & real-world & 0 &  &  &  &  &  &  & \ref{a:TranTDB13} & \ref{b:TranTDB13}\\
\rowlabel{c:BillautHL12}BillautHL12 \href{https://doi.org/10.1007/978-3-642-29828-8\_5}{BillautHL12}~\cite{BillautHL12} & \href{works/BillautHL12.pdf}{Complete Characterization of Near-Optimal Sequences for the Two-Machine Flow Shop Scheduling Problem} &  & random instance & 0 &  &  &  &  &  &  & \ref{a:BillautHL12} & \ref{b:BillautHL12}\\
\rowlabel{c:BonfiettiLBM12}BonfiettiLBM12 \href{https://doi.org/10.1007/978-3-642-29828-8\_6}{BonfiettiLBM12}~\cite{BonfiettiLBM12} & \href{works/BonfiettiLBM12.pdf}{Global Cyclic Cumulative Constraint} &  & benchmark & 3 &  &  &  &  &  &  & \ref{a:BonfiettiLBM12} & \ref{b:BonfiettiLBM12}\\
\rowlabel{c:BonfiettiM12}BonfiettiM12 \href{https://ceur-ws.org/Vol-926/paper2.pdf}{BonfiettiM12}~\cite{BonfiettiM12} & \href{works/BonfiettiM12.pdf}{A Constraint-based Approach to Cyclic Resource-Constrained Scheduling Problem} &  & industrial instance & 0 &  &  &  &  &  &  & \ref{a:BonfiettiM12} & \ref{b:BonfiettiM12}\\
\rowlabel{c:GuSW12}GuSW12 \href{https://doi.org/10.1007/978-3-642-33558-7\_55}{GuSW12}~\cite{GuSW12} & \href{works/GuSW12.pdf}{Maximising the Net Present Value of Large Resource-Constrained Projects} &  & benchmark & 2 &  &  &  &  &  &  & \ref{a:GuSW12} & \ref{b:GuSW12}\\
\rowlabel{c:HeinzB12}HeinzB12 \href{https://doi.org/10.1007/978-3-642-29828-8\_14}{HeinzB12}~\cite{HeinzB12} & \href{works/HeinzB12.pdf}{Reconsidering Mixed Integer Programming and MIP-Based Hybrids for Scheduling} &  &  & 0 &  &  &  &  &  &  & \ref{a:HeinzB12} & \ref{b:HeinzB12}\\
\rowlabel{c:IfrimOS12}IfrimOS12 \href{https://doi.org/10.1007/978-3-642-33558-7\_68}{IfrimOS12}~\cite{IfrimOS12} & \href{works/IfrimOS12.pdf}{Properties of Energy-Price Forecasts for Scheduling} &  & real-life & 1 &  &  &  &  &  &  & \ref{a:IfrimOS12} & \ref{b:IfrimOS12}\\
\rowlabel{c:LetortBC12}LetortBC12 \href{https://doi.org/10.1007/978-3-642-33558-7\_33}{LetortBC12}~\cite{LetortBC12} & \href{works/LetortBC12.pdf}{A Scalable Sweep Algorithm for the cumulative Constraint} &  & Roadef, benchmark, random instance & 2 &  &  &  &  &  &  & \ref{a:LetortBC12} & \ref{b:LetortBC12}\\
\rowlabel{c:RendlPHPR12}RendlPHPR12 \href{https://doi.org/10.1007/978-3-642-29828-8\_22}{RendlPHPR12}~\cite{RendlPHPR12} & \href{works/RendlPHPR12.pdf}{Hybrid Heuristics for Multimodal Homecare Scheduling} &  & real-world, CSPlib, benchmark & 2 &  &  &  &  &  &  & \ref{a:RendlPHPR12} & \ref{b:RendlPHPR12}\\
\rowlabel{c:SchuttCSW12}SchuttCSW12 \href{https://doi.org/10.1007/978-3-642-29828-8\_24}{SchuttCSW12}~\cite{SchuttCSW12} & \href{works/SchuttCSW12.pdf}{Maximising the Net Present Value for Resource-Constrained Project Scheduling} &  & benchmark & 1 &  &  &  &  &  &  & \ref{a:SchuttCSW12} & \ref{b:SchuttCSW12}\\
\rowlabel{c:SerraNM12}SerraNM12 \href{https://doi.org/10.1007/978-3-642-33558-7\_59}{SerraNM12}~\cite{SerraNM12} & \href{works/SerraNM12.pdf}{The Offshore Resources Scheduling Problem: Detailing a Constraint Programming Approach} &  & benchmark, real-world & 4 &  &  &  &  &  &  & \ref{a:SerraNM12} & \ref{b:SerraNM12}\\
\rowlabel{c:SimoninAHL12}SimoninAHL12 \href{https://doi.org/10.1007/978-3-642-33558-7\_5}{SimoninAHL12}~\cite{SimoninAHL12} & \href{works/SimoninAHL12.pdf}{Scheduling Scientific Experiments on the Rosetta/Philae Mission} & \su{MOST {Ilog Scheduler}} &  & 0 & n &  & n & - &  & \su{cumulative dataTransfer} & \ref{a:SimoninAHL12} & \ref{b:SimoninAHL12}\\
\rowlabel{c:TranB12}TranB12 \href{https://doi.org/10.3233/978-1-61499-098-7-774}{TranB12}~\cite{TranB12} & \href{works/TranB12.pdf}{Logic-based Benders Decomposition for Alternative Resource Scheduling with Sequence Dependent Setups} &  & benchmark & 0 &  &  &  &  &  &  & \ref{a:TranB12} & \ref{b:TranB12}\\
\rowlabel{c:ZhangLS12}ZhangLS12 \href{https://doi.org/10.1109/CIT.2012.96}{ZhangLS12}~\cite{ZhangLS12} & \href{works/ZhangLS12.pdf}{Model and Solution for Hot Strip Rolling Scheduling Problem Based on Constraint Programming Method} &  &  & 0 &  &  &  &  &  &  & \ref{a:ZhangLS12} & \ref{b:ZhangLS12}\\
\rowlabel{c:BajestaniB11}BajestaniB11 \href{http://aaai.org/ocs/index.php/ICAPS/ICAPS11/paper/view/2680}{BajestaniB11}~\cite{BajestaniB11} & \href{works/BajestaniB11.pdf}{Scheduling an Aircraft Repair Shop} &  &  & 0 &  &  &  &  &  &  & \ref{a:BajestaniB11} & \ref{b:BajestaniB11}\\
\rowlabel{c:BonfiettiLBM11}BonfiettiLBM11 \href{https://doi.org/10.1007/978-3-642-23786-7\_12}{BonfiettiLBM11}~\cite{BonfiettiLBM11} & \href{works/BonfiettiLBM11.pdf}{A Constraint Based Approach to Cyclic {RCPSP}} &  & generated instance, industrial instance, benchmark & 3 &  &  &  &  &  &  & \ref{a:BonfiettiLBM11} & \ref{b:BonfiettiLBM11}\\
\rowlabel{c:ChapadosJR11}ChapadosJR11 \href{https://doi.org/10.1007/978-3-642-21311-3\_7}{ChapadosJR11}~\cite{ChapadosJR11} & \href{works/ChapadosJR11.pdf}{Retail Store Workforce Scheduling by Expected Operating Income Maximization} &  &  & 0 &  &  &  &  &  &  & \ref{a:ChapadosJR11} & \ref{b:ChapadosJR11}\\
\rowlabel{c:ClercqPBJ11}ClercqPBJ11 \href{https://doi.org/10.1007/978-3-642-23786-7\_20}{ClercqPBJ11}~\cite{ClercqPBJ11} & \href{works/ClercqPBJ11.pdf}{Filtering Algorithms for Discrete Cumulative Problems with Overloads of Resource} &  & benchmark & 1 &  &  &  &  &  &  & \ref{a:ClercqPBJ11} & \ref{b:ClercqPBJ11}\\
\rowlabel{c:EdisO11}EdisO11 \href{https://doi.org/10.1007/978-3-642-21311-3\_10}{EdisO11}~\cite{EdisO11} & \href{works/EdisO11.pdf}{Parallel Machine Scheduling with Additional Resources: {A} Lagrangian-Based Constraint Programming Approach} &  &  & 0 &  &  &  &  &  &  & \ref{a:EdisO11} & \ref{b:EdisO11}\\
\rowlabel{c:GrimesH11}GrimesH11 \href{https://doi.org/10.1007/978-3-642-23786-7\_28}{GrimesH11}~\cite{GrimesH11} & \href{works/GrimesH11.pdf}{Models and Strategies for Variants of the Job Shop Scheduling Problem} &  & benchmark & 1 &  &  &  &  &  &  & \ref{a:GrimesH11} & \ref{b:GrimesH11}\\
\rowlabel{c:HeinzS11}HeinzS11 \href{https://doi.org/10.1007/978-3-642-20662-7\_34}{HeinzS11}~\cite{HeinzS11} & \href{works/HeinzS11.pdf}{Explanations for the Cumulative Constraint: An Experimental Study} &  & benchmark & 1 &  &  &  &  &  &  & \ref{a:HeinzS11} & \ref{b:HeinzS11}\\
\rowlabel{c:HermenierDL11}HermenierDL11 \href{https://doi.org/10.1007/978-3-642-23786-7\_5}{HermenierDL11}~\cite{HermenierDL11} & \href{works/HermenierDL11.pdf}{Bin Repacking Scheduling in Virtualized Datacenters} &  &  & 1 &  &  &  &  &  &  & \ref{a:HermenierDL11} & \ref{b:HermenierDL11}\\
\rowlabel{c:KameugneFSN11}KameugneFSN11 \href{https://doi.org/10.1007/978-3-642-23786-7\_37}{KameugneFSN11}~\cite{KameugneFSN11} & \href{works/KameugneFSN11.pdf}{A Quadratic Edge-Finding Filtering Algorithm for Cumulative Resource Constraints} &  & benchmark & 1 &  &  &  &  &  &  & \ref{a:KameugneFSN11} & \ref{b:KameugneFSN11}\\
\rowlabel{c:LahimerLH11}LahimerLH11 \href{https://doi.org/10.1007/978-3-642-21311-3\_12}{LahimerLH11}~\cite{LahimerLH11} & \href{works/LahimerLH11.pdf}{Climbing Depth-Bounded Adjacent Discrepancy Search for Solving Hybrid Flow Shop Scheduling Problems with Multiprocessor Tasks} &  & benchmark & 2 &  &  &  &  &  &  & \ref{a:LahimerLH11} & \ref{b:LahimerLH11}\\
\rowlabel{c:LombardiBMB11}LombardiBMB11 \href{https://doi.org/10.1007/978-3-642-21311-3\_14}{LombardiBMB11}~\cite{LombardiBMB11} & \href{works/LombardiBMB11.pdf}{Precedence Constraint Posting for Cyclic Scheduling Problems} &  & benchmark, industrial instance, real-life & 0 &  &  &  &  &  &  & \ref{a:LombardiBMB11} & \ref{b:LombardiBMB11}\\
\rowlabel{c:Vilim11}Vilim11 \href{https://doi.org/10.1007/978-3-642-21311-3\_22}{Vilim11}~\cite{Vilim11} & \href{works/Vilim11.pdf}{Timetable Edge Finding Filtering Algorithm for Discrete Cumulative Resources} &  & benchmark & 1 &  &  &  &  &  &  & \ref{a:Vilim11} & \ref{b:Vilim11}\\
\rowlabel{c:ZibranR11}ZibranR11 \href{https://doi.org/10.1109/ICPC.2011.45}{ZibranR11}~\cite{ZibranR11} & \href{works/ZibranR11.pdf}{Conflict-Aware Optimal Scheduling of Code Clone Refactoring: {A} Constraint Programming Approach} &  &  & 0 &  &  &  &  &  &  & \ref{a:ZibranR11} & \ref{b:ZibranR11}\\
\rowlabel{c:ZibranR11a}ZibranR11a \href{https://doi.org/10.1109/SCAM.2011.21}{ZibranR11a}~\cite{ZibranR11a} & \href{works/ZibranR11a.pdf}{A Constraint Programming Approach to Conflict-Aware Optimal Scheduling of Prioritized Code Clone Refactoring} &  &  & 0 &  &  &  &  &  &  & \ref{a:ZibranR11a} & \ref{b:ZibranR11a}\\
\rowlabel{c:BertholdHLMS10}BertholdHLMS10 \href{https://doi.org/10.1007/978-3-642-13520-0\_34}{BertholdHLMS10}~\cite{BertholdHLMS10} & \href{works/BertholdHLMS10.pdf}{A Constraint Integer Programming Approach for Resource-Constrained Project Scheduling} &  &  & 1 &  &  &  &  &  &  & \ref{a:BertholdHLMS10} & \ref{b:BertholdHLMS10}\\
\rowlabel{c:CobanH10}CobanH10 \href{https://doi.org/10.1007/978-3-642-13520-0\_11}{CobanH10}~\cite{CobanH10} & \href{works/CobanH10.pdf}{Single-Facility Scheduling over Long Time Horizons by Logic-Based Benders Decomposition} &  &  & 0 &  &  &  &  &  &  & \ref{a:CobanH10} & \ref{b:CobanH10}\\
\rowlabel{c:Davenport10}Davenport10 \href{https://doi.org/10.1007/978-3-642-13520-0\_12}{Davenport10}~\cite{Davenport10} & \href{works/Davenport10.pdf}{Integrated Maintenance Scheduling for Semiconductor Manufacturing} &  &  & 0 &  &  &  &  &  &  & \ref{a:Davenport10} & \ref{b:Davenport10}\\
\rowlabel{c:GrimesH10}GrimesH10 \href{https://doi.org/10.1007/978-3-642-13520-0\_19}{GrimesH10}~\cite{GrimesH10} & \href{works/GrimesH10.pdf}{Job Shop Scheduling with Setup Times and Maximal Time-Lags: {A} Simple Constraint Programming Approach} &  & benchmark & 1 &  &  &  &  &  &  & \ref{a:GrimesH10} & \ref{b:GrimesH10}\\
\rowlabel{c:LombardiM10}LombardiM10 \href{https://doi.org/10.1007/978-3-642-15396-9\_32}{LombardiM10}~\cite{LombardiM10} & \href{works/LombardiM10.pdf}{Constraint Based Scheduling to Deal with Uncertain Durations and Self-Timed Execution} &  & real-world, benchmark & 1 &  &  &  &  &  &  & \ref{a:LombardiM10} & \ref{b:LombardiM10}\\
\rowlabel{c:MakMS10}MakMS10 \href{https://doi.org/10.1109/ICNC.2010.5583494}{MakMS10}~\cite{MakMS10} & \href{works/MakMS10.pdf}{A constraint programming approach for production scheduling of multi-period virtual cellular manufacturing systems} &  &  & 0 &  &  &  &  &  &  & \ref{a:MakMS10} & \ref{b:MakMS10}\\
\rowlabel{c:SchuttW10}SchuttW10 \href{https://doi.org/10.1007/978-3-642-15396-9\_36}{SchuttW10}~\cite{SchuttW10} & \href{works/SchuttW10.pdf}{A New \emph{O}(\emph{n}\({}^{\mbox{2}}\)log\emph{n}) Not-First/Not-Last Pruning Algorithm for Cumulative Resource Constraints} &  & benchmark & 1 &  &  &  &  &  &  & \ref{a:SchuttW10} & \ref{b:SchuttW10}\\
\rowlabel{c:SunLYL10}SunLYL10 \href{https://doi.org/10.1109/GreenCom-CPSCom.2010.111}{SunLYL10}~\cite{SunLYL10} & \href{works/SunLYL10.pdf}{Scheduling Optimization Techniques for FlexRay Using Constraint-Programming} &  &  & 0 &  &  &  &  &  &  & \ref{a:SunLYL10} & \ref{b:SunLYL10}\\
\rowlabel{c:Acuna-AgostMFG09}Acuna-AgostMFG09 \href{https://doi.org/10.1007/978-3-642-01929-6\_24}{Acuna-AgostMFG09}~\cite{Acuna-AgostMFG09} & \href{works/Acuna-AgostMFG09.pdf}{Constraint Programming and Mixed Integer Linear Programming for Rescheduling Trains under Disrupted Operations} &  & Roadef & 1 &  &  &  &  &  &  & \ref{a:Acuna-AgostMFG09} & \ref{b:Acuna-AgostMFG09}\\
\rowlabel{c:AronssonBK09}AronssonBK09 \href{http://drops.dagstuhl.de/opus/volltexte/2009/2141}{AronssonBK09}~\cite{AronssonBK09} & \href{works/AronssonBK09.pdf}{{MILP} formulations of cumulative constraints for railway scheduling - {A} comparative study} &  & real-world, real-life & 0 &  &  &  &  &  &  & \ref{a:AronssonBK09} & \ref{b:AronssonBK09}\\
\rowlabel{c:Baptiste09}Baptiste09 \href{https://doi.org/10.1007/978-3-642-04244-7\_1}{Baptiste09}~\cite{Baptiste09} & \href{works/Baptiste09.pdf}{Constraint-Based Schedulers, Do They Really Work?} &  &  & 0 &  &  &  &  &  &  & \ref{a:Baptiste09} & \ref{b:Baptiste09}\\
\rowlabel{c:GrimesHM09}GrimesHM09 \href{https://doi.org/10.1007/978-3-642-04244-7\_33}{GrimesHM09}~\cite{GrimesHM09} & \href{works/GrimesHM09.pdf}{Closing the Open Shop: Contradicting Conventional Wisdom} &  & benchmark & 0 &  &  &  &  &  &  & \ref{a:GrimesHM09} & \ref{b:GrimesHM09}\\
\rowlabel{c:Laborie09}Laborie09 \href{https://doi.org/10.1007/978-3-642-01929-6\_12}{Laborie09}~\cite{Laborie09} & \href{works/Laborie09.pdf}{{IBM} {ILOG} {CP} Optimizer for Detailed Scheduling Illustrated on Three Problems} &  & real-world, benchmark & 2 &  &  &  &  &  &  & \ref{a:Laborie09} & \ref{b:Laborie09}\\
\rowlabel{c:LombardiM09}LombardiM09 \href{https://doi.org/10.1007/978-3-642-04244-7\_45}{LombardiM09}~\cite{LombardiM09} & \href{works/LombardiM09.pdf}{A Precedence Constraint Posting Approach for the {RCPSP} with Time Lags and Variable Durations} &  & real-world, instance generator & 1 &  &  &  &  &  &  & \ref{a:LombardiM09} & \ref{b:LombardiM09}\\
\rowlabel{c:MonetteDH09}MonetteDH09 \href{http://aaai.org/ocs/index.php/ICAPS/ICAPS09/paper/view/712}{MonetteDH09}~\cite{MonetteDH09} & \href{works/MonetteDH09.pdf}{Just-In-Time Scheduling with Constraint Programming} &  & benchmark & 0 &  &  &  &  &  &  & \ref{a:MonetteDH09} & \ref{b:MonetteDH09}\\
\rowlabel{c:SchuttFSW09}SchuttFSW09 \href{https://doi.org/10.1007/978-3-642-04244-7\_58}{SchuttFSW09}~\cite{SchuttFSW09} & \href{works/SchuttFSW09.pdf}{Why Cumulative Decomposition Is Not as Bad as It Sounds} &  & benchmark, real-world & 1 &  &  &  &  &  &  & \ref{a:SchuttFSW09} & \ref{b:SchuttFSW09}\\
\rowlabel{c:ThiruvadyBME09}ThiruvadyBME09 \href{https://doi.org/10.1007/978-3-642-04918-7\_3}{ThiruvadyBME09}~\cite{ThiruvadyBME09} & \href{works/ThiruvadyBME09.pdf}{Hybridizing Beam-ACO with Constraint Programming for Single Machine Job Scheduling} &  &  & 0 &  &  &  &  &  &  & \ref{a:ThiruvadyBME09} & \ref{b:ThiruvadyBME09}\\
\rowlabel{c:Vilim09}Vilim09 \href{https://doi.org/10.1007/978-3-642-04244-7\_62}{Vilim09}~\cite{Vilim09} & \href{works/Vilim09.pdf}{Edge Finding Filtering Algorithm for Discrete Cumulative Resources in \emph{O}(\emph{kn} log \emph{n})\{{\textbackslash}mathcal O\}(kn \{{\textbackslash}rm log\} n)} &  &  & 0 &  &  &  &  &  &  & \ref{a:Vilim09} & \ref{b:Vilim09}\\
\rowlabel{c:Vilim09a}Vilim09a \href{https://doi.org/10.1007/978-3-642-01929-6\_22}{Vilim09a}~\cite{Vilim09a} & \href{works/Vilim09a.pdf}{Max Energy Filtering Algorithm for Discrete Cumulative Resources} &  &  & 1 &  &  &  &  &  &  & \ref{a:Vilim09a} & \ref{b:Vilim09a}\\
\rowlabel{c:BarlattCG08}BarlattCG08 \href{https://doi.org/10.1007/978-3-540-68155-7\_24}{BarlattCG08}~\cite{BarlattCG08} & \href{works/BarlattCG08.pdf}{A Hybrid Approach for Solving Shift-Selection and Task-Sequencing Problems} &  & real-world & 1 &  &  &  &  &  &  & \ref{a:BarlattCG08} & \ref{b:BarlattCG08}\\
\rowlabel{c:BeldiceanuCP08}BeldiceanuCP08 \href{https://doi.org/10.1007/978-3-540-68155-7\_5}{BeldiceanuCP08}~\cite{BeldiceanuCP08} & \href{works/BeldiceanuCP08.pdf}{New Filtering for the cumulative Constraint in the Context of Non-Overlapping Rectangles} &  & benchmark & 0 &  &  &  &  &  &  & \ref{a:BeldiceanuCP08} & \ref{b:BeldiceanuCP08}\\
\rowlabel{c:DoomsH08}DoomsH08 \href{https://doi.org/10.1007/978-3-540-68155-7\_8}{DoomsH08}~\cite{DoomsH08} & \href{works/DoomsH08.pdf}{Gap Reduction Techniques for Online Stochastic Project Scheduling} &  &  & 0 &  &  &  &  &  &  & \ref{a:DoomsH08} & \ref{b:DoomsH08}\\
\rowlabel{c:HentenryckM08}HentenryckM08 \href{https://doi.org/10.1007/978-3-540-68155-7\_41}{HentenryckM08}~\cite{HentenryckM08} & \href{works/HentenryckM08.pdf}{The Steel Mill Slab Design Problem Revisited} &  & CSPlib & 0 &  &  &  &  &  &  & \ref{a:HentenryckM08} & \ref{b:HentenryckM08}\\
\rowlabel{c:LauLN08}LauLN08 \href{https://doi.org/10.1007/978-3-540-68155-7\_33}{LauLN08}~\cite{LauLN08} & \href{works/LauLN08.pdf}{A Combinatorial Auction Framework for Solving Decentralized Scheduling Problems (Extended Abstract)} &  & benchmark, real-world & 0 &  &  &  &  &  &  & \ref{a:LauLN08} & \ref{b:LauLN08}\\
\rowlabel{c:MouraSCL08}MouraSCL08 \href{https://doi.org/10.1007/978-3-540-85958-1\_3}{MouraSCL08}~\cite{MouraSCL08} & \href{works/MouraSCL08.pdf}{Planning and Scheduling the Operation of a Very Large Oil Pipeline Network} &  &  & 0 &  &  &  &  &  &  & \ref{a:MouraSCL08} & \ref{b:MouraSCL08}\\
\rowlabel{c:MouraSCL08a}MouraSCL08a \href{https://doi.org/10.1109/CSE.2008.24}{MouraSCL08a}~\cite{MouraSCL08a} & \href{works/MouraSCL08a.pdf}{Heuristics and Constraint Programming Hybridizations for a Real Pipeline Planning and Scheduling Problem} &  & real-world, benchmark & 0 &  &  &  &  &  &  & \ref{a:MouraSCL08a} & \ref{b:MouraSCL08a}\\
\rowlabel{c:PoderB08}PoderB08 \href{http://www.aaai.org/Library/ICAPS/2008/icaps08-033.php}{PoderB08}~\cite{PoderB08} & \href{works/PoderB08.pdf}{Filtering for a Continuous Multi-Resources cumulative Constraint with Resource Consumption and Production} &  &  & 0 &  &  &  &  &  &  & \ref{a:PoderB08} & \ref{b:PoderB08}\\
\rowlabel{c:WatsonB08}WatsonB08 \href{https://doi.org/10.1007/978-3-540-68155-7\_21}{WatsonB08}~\cite{WatsonB08} & \href{works/WatsonB08.pdf}{A Hybrid Constraint Programming / Local Search Approach to the Job-Shop Scheduling Problem} &  & benchmark, real-world & 1 &  &  &  &  &  &  & \ref{a:WatsonB08} & \ref{b:WatsonB08}\\
\rowlabel{c:AkkerDH07}AkkerDH07 \href{https://doi.org/10.1007/978-3-540-72397-4\_27}{AkkerDH07}~\cite{AkkerDH07} & \href{works/AkkerDH07.pdf}{A Column Generation Based Destructive Lower Bound for Resource Constrained Project Scheduling Problems} &  &  & 0 &  &  &  &  &  &  & \ref{a:AkkerDH07} & \ref{b:AkkerDH07}\\
\rowlabel{c:BeldiceanuP07}BeldiceanuP07 \href{https://doi.org/10.1007/978-3-540-72397-4\_16}{BeldiceanuP07}~\cite{BeldiceanuP07} & \href{works/BeldiceanuP07.pdf}{A Continuous Multi-resources \emph{cumulative} Constraint with Positive-Negative Resource Consumption-Production} &  &  & 0 &  &  &  &  &  &  & \ref{a:BeldiceanuP07} & \ref{b:BeldiceanuP07}\\
\rowlabel{c:DavenportKRSH07}DavenportKRSH07 \href{https://doi.org/10.1007/978-3-540-74970-7\_7}{DavenportKRSH07}~\cite{DavenportKRSH07} & \href{works/DavenportKRSH07.pdf}{An Application of Constraint Programming to Generating Detailed Operations Schedules for Steel Manufacturing} &  &  & 0 &  &  &  &  &  &  & \ref{a:DavenportKRSH07} & \ref{b:DavenportKRSH07}\\
\rowlabel{c:GarganiR07}GarganiR07 \href{https://doi.org/10.1007/978-3-540-74970-7\_8}{GarganiR07}~\cite{GarganiR07} & \href{works/GarganiR07.pdf}{An Efficient Model and Strategy for the Steel Mill Slab Design Problem} &  & real-life, CSPlib & 0 &  &  &  &  &  &  & \ref{a:GarganiR07} & \ref{b:GarganiR07}\\
\rowlabel{c:HoeveGSL07}HoeveGSL07 \href{http://www.aaai.org/Library/AAAI/2007/aaai07-291.php}{HoeveGSL07}~\cite{HoeveGSL07} & \href{works/HoeveGSL07.pdf}{Optimal Multi-Agent Scheduling with Constraint Programming} &  & benchmark & 0 &  &  &  &  &  &  & \ref{a:HoeveGSL07} & \ref{b:HoeveGSL07}\\
\rowlabel{c:KeriK07}KeriK07 \href{https://doi.org/10.1007/978-3-540-72397-4\_10}{KeriK07}~\cite{KeriK07} & \href{works/KeriK07.pdf}{Computing Tight Time Windows for {RCPSPWET} with the Primal-Dual Method} &  &  & 2 &  &  &  &  &  &  & \ref{a:KeriK07} & \ref{b:KeriK07}\\
\rowlabel{c:KovacsB07}KovacsB07 \href{https://doi.org/10.1007/978-3-540-72397-4\_9}{KovacsB07}~\cite{KovacsB07} & \href{works/KovacsB07.pdf}{A Global Constraint for Total Weighted Completion Time} &  & benchmark & 0 &  &  &  &  &  &  & \ref{a:KovacsB07} & \ref{b:KovacsB07}\\
\rowlabel{c:KrogtLPHJ07}KrogtLPHJ07 \href{https://doi.org/10.1007/978-3-540-74970-7\_10}{KrogtLPHJ07}~\cite{KrogtLPHJ07} & \href{works/KrogtLPHJ07.pdf}{Scheduling for Cellular Manufacturing} &  & real-world & 0 &  &  &  &  &  &  & \ref{a:KrogtLPHJ07} & \ref{b:KrogtLPHJ07}\\
\rowlabel{c:Limtanyakul07}Limtanyakul07 \href{https://doi.org/10.1007/978-3-540-77903-2\_65}{Limtanyakul07}~\cite{Limtanyakul07} & \href{works/Limtanyakul07.pdf}{Scheduling of Tests on Vehicle Prototypes Using Constraint and Integer Programming} &  & real-life & 0 &  &  &  &  &  &  & \ref{a:Limtanyakul07} & \ref{b:Limtanyakul07}\\
\rowlabel{c:MonetteDD07}MonetteDD07 \href{https://doi.org/10.1007/978-3-540-72397-4\_14}{MonetteDD07}~\cite{MonetteDD07} & \href{works/MonetteDD07.pdf}{A Position-Based Propagator for the Open-Shop Problem} &  & benchmark & 0 &  &  &  &  &  &  & \ref{a:MonetteDD07} & \ref{b:MonetteDD07}\\
\rowlabel{c:NethercoteSBBDT07}NethercoteSBBDT07 \href{https://doi.org/10.1007/978-3-540-74970-7\_38}{NethercoteSBBDT07}~\cite{NethercoteSBBDT07} & \href{works/NethercoteSBBDT07.pdf}{MiniZinc: Towards a Standard {CP} Modelling Language} &  & CSPlib, benchmark & 1 &  &  &  &  &  &  & \ref{a:NethercoteSBBDT07} & \ref{b:NethercoteSBBDT07}\\
\rowlabel{c:RossiTHP07}RossiTHP07 \href{https://doi.org/10.1007/978-3-540-72397-4\_17}{RossiTHP07}~\cite{RossiTHP07} & \href{works/RossiTHP07.pdf}{Replenishment Planning for Stochastic Inventory Systems with Shortage Cost} &  &  & 0 &  &  &  &  &  &  & \ref{a:RossiTHP07} & \ref{b:RossiTHP07}\\
\rowlabel{c:Beck06}Beck06 \href{http://www.aaai.org/Library/ICAPS/2006/icaps06-028.php}{Beck06}~\cite{Beck06} & \href{works/Beck06.pdf}{An Empirical Study of Multi-Point Constructive Search for Constraint-Based Scheduling} &  & benchmark & 0 &  &  &  &  &  &  & \ref{a:Beck06} & \ref{b:Beck06}\\
\rowlabel{c:BeniniBGM06}BeniniBGM06 \href{https://doi.org/10.1007/11757375\_6}{BeniniBGM06}~\cite{BeniniBGM06} & \href{works/BeniniBGM06.pdf}{Allocation, Scheduling and Voltage Scaling on Energy Aware MPSoCs} &  & real-life & 0 &  &  &  &  &  &  & \ref{a:BeniniBGM06} & \ref{b:BeniniBGM06}\\
\rowlabel{c:GomesHS06}GomesHS06 \href{http://www.aaai.org/Library/Symposia/Spring/2006/ss06-04-024.php}{GomesHS06}~\cite{GomesHS06} & \href{works/GomesHS06.pdf}{Constraint Programming for Distributed Planning and Scheduling} &  & real-life & 0 &  &  &  &  &  &  & \ref{a:GomesHS06} & \ref{b:GomesHS06}\\
\rowlabel{c:KhemmoudjPB06}KhemmoudjPB06 \href{https://doi.org/10.1007/11889205\_21}{KhemmoudjPB06}~\cite{KhemmoudjPB06} & \href{works/KhemmoudjPB06.pdf}{When Constraint Programming and Local Search Solve the Scheduling Problem of Electricit{\'{e}} de France Nuclear Power Plant Outages} &  & real-world & 0 &  &  &  &  &  &  & \ref{a:KhemmoudjPB06} & \ref{b:KhemmoudjPB06}\\
\rowlabel{c:KovacsV06}KovacsV06 \href{https://doi.org/10.1007/11757375\_13}{KovacsV06}~\cite{KovacsV06} & \href{works/KovacsV06.pdf}{Progressive Solutions: {A} Simple but Efficient Dominance Rule for Practical {RCPSP}} &  & industrial partner, benchmark, generated instance & 0 &  &  &  &  &  &  & \ref{a:KovacsV06} & \ref{b:KovacsV06}\\
\rowlabel{c:LiuJ06}LiuJ06 \href{https://doi.org/10.1007/11801603\_92}{LiuJ06}~\cite{LiuJ06} & \href{works/LiuJ06.pdf}{{LP-TPOP:} Integrating Planning and Scheduling Through Constraint Programming} &  &  & 0 &  &  &  &  &  &  & \ref{a:LiuJ06} & \ref{b:LiuJ06}\\
\rowlabel{c:QuSN06}QuSN06 \href{https://doi.org/10.1109/ISSOC.2006.321973}{QuSN06}~\cite{QuSN06} & \href{works/QuSN06.pdf}{Using Constraint Programming to Achieve Optimal Prefetch Scheduling for Dependent Tasks on Run-Time Reconfigurable Devices} &  &  & 0 &  &  &  &  &  &  & \ref{a:QuSN06} & \ref{b:QuSN06}\\
\rowlabel{c:AbrilSB05}AbrilSB05 \href{https://doi.org/10.1007/11564751\_75}{AbrilSB05}~\cite{AbrilSB05} & \href{works/AbrilSB05.pdf}{Distributed Constraints for Large-Scale Scheduling Problems} &  &  & 0 &  &  &  &  &  &  & \ref{a:AbrilSB05} & \ref{b:AbrilSB05}\\
\rowlabel{c:ArtiouchineB05}ArtiouchineB05 \href{https://doi.org/10.1007/11564751\_8}{ArtiouchineB05}~\cite{ArtiouchineB05} & \href{works/ArtiouchineB05.pdf}{Inter-distance Constraint: An Extension of the All-Different Constraint for Scheduling Equal Length Jobs} &  & generated instance, random instance & 0 &  &  &  &  &  &  & \ref{a:ArtiouchineB05} & \ref{b:ArtiouchineB05}\\
\rowlabel{c:BeckW05}BeckW05 \href{http://ijcai.org/Proceedings/05/Papers/0748.pdf}{BeckW05}~\cite{BeckW05} & \href{works/BeckW05.pdf}{Proactive Algorithms for Scheduling with Probabilistic Durations} &  &  & 0 &  &  &  &  &  &  & \ref{a:BeckW05} & \ref{b:BeckW05}\\
\rowlabel{c:CarchraeBF05}CarchraeBF05 \href{https://doi.org/10.1007/11564751\_80}{CarchraeBF05}~\cite{CarchraeBF05} & \href{works/CarchraeBF05.pdf}{Methods to Learn Abstract Scheduling Models} &  &  & 0 &  &  &  &  &  &  & \ref{a:CarchraeBF05} & \ref{b:CarchraeBF05}\\
\rowlabel{c:ChuX05}ChuX05 \href{https://doi.org/10.1007/11493853\_10}{ChuX05}~\cite{ChuX05} & \href{works/ChuX05.pdf}{A Hybrid Algorithm for a Class of Resource Constrained Scheduling Problems} &  &  & 0 &  &  &  &  &  &  & \ref{a:ChuX05} & \ref{b:ChuX05}\\
\rowlabel{c:DilkinaDH05}DilkinaDH05 \href{https://doi.org/10.1007/11564751\_60}{DilkinaDH05}~\cite{DilkinaDH05} & \href{works/DilkinaDH05.pdf}{Extending Systematic Local Search for Job Shop Scheduling Problems} &  &  & 0 &  &  &  &  &  &  & \ref{a:DilkinaDH05} & \ref{b:DilkinaDH05}\\
\rowlabel{c:FortinZDF05}FortinZDF05 \href{https://doi.org/10.1007/11564751\_19}{FortinZDF05}~\cite{FortinZDF05} & \href{works/FortinZDF05.pdf}{Interval Analysis in Scheduling} &  &  & 0 &  &  &  &  &  &  & \ref{a:FortinZDF05} & \ref{b:FortinZDF05}\\
\rowlabel{c:FrankK05}FrankK05 \href{https://doi.org/10.1007/11493853\_15}{FrankK05}~\cite{FrankK05} & \href{works/FrankK05.pdf}{Mixed Discrete and Continuous Algorithms for Scheduling Airborne Astronomy Observations} &  & benchmark & 0 &  &  &  &  &  &  & \ref{a:FrankK05} & \ref{b:FrankK05}\\
\rowlabel{c:Geske05}Geske05 \href{https://doi.org/10.1007/11963578\_10}{Geske05}~\cite{Geske05} & \href{works/Geske05.pdf}{Railway Scheduling with Declarative Constraint Programming} &  & real-life & 0 &  &  &  &  &  &  & \ref{a:Geske05} & \ref{b:Geske05}\\
\rowlabel{c:GodardLN05}GodardLN05 \href{http://www.aaai.org/Library/ICAPS/2005/icaps05-009.php}{GodardLN05}~\cite{GodardLN05} & \href{works/GodardLN05.pdf}{Randomized Large Neighborhood Search for Cumulative Scheduling} &  & benchmark & 0 &  &  &  &  &  &  & \ref{a:GodardLN05} & \ref{b:GodardLN05}\\
\rowlabel{c:HebrardTW05}HebrardTW05 \href{https://doi.org/10.1007/11564751\_117}{HebrardTW05}~\cite{HebrardTW05} & \href{works/HebrardTW05.pdf}{Computing Super-Schedules} &  &  & 0 &  &  &  &  &  &  & \ref{a:HebrardTW05} & \ref{b:HebrardTW05}\\
\rowlabel{c:Hooker05a}Hooker05a \href{https://doi.org/10.1007/11564751\_25}{Hooker05a}~\cite{Hooker05a} & \href{works/Hooker05a.pdf}{Planning and Scheduling to Minimize Tardiness} &  &  & 0 &  &  &  &  &  &  & \ref{a:Hooker05a} & \ref{b:Hooker05a}\\
\rowlabel{c:KovacsEKV05}KovacsEKV05 \href{https://doi.org/10.1007/11564751\_118}{KovacsEKV05}~\cite{KovacsEKV05} & \href{works/KovacsEKV05.pdf}{Proterv-II: An Integrated Production Planning and Scheduling System} &  & real-life & 0 &  &  &  &  &  &  & \ref{a:KovacsEKV05} & \ref{b:KovacsEKV05}\\
\rowlabel{c:MoffittPP05}MoffittPP05 \href{http://www.aaai.org/Library/AAAI/2005/aaai05-188.php}{MoffittPP05}~\cite{MoffittPP05} & \href{works/MoffittPP05.pdf}{Augmenting Disjunctive Temporal Problems with Finite-Domain Constraints} &  &  & 0 &  &  &  &  &  &  & \ref{a:MoffittPP05} & \ref{b:MoffittPP05}\\
\rowlabel{c:QuirogaZH05}QuirogaZH05 \href{https://doi.org/10.1109/ROBOT.2005.1570686}{QuirogaZH05}~\cite{QuirogaZH05} & \href{works/QuirogaZH05.pdf}{A Constraint Programming Approach to Tool Allocation and Resource Scheduling in {FMS}} &  &  & 0 &  &  &  &  &  &  & \ref{a:QuirogaZH05} & \ref{b:QuirogaZH05}\\
\rowlabel{c:SchuttWS05}SchuttWS05 \href{https://doi.org/10.1007/11963578\_6}{SchuttWS05}~\cite{SchuttWS05} & \href{works/SchuttWS05.pdf}{Not-First and Not-Last Detection for Cumulative Scheduling in \emph{O}(\emph{n}\({}^{\mbox{3}}\)log\emph{n})} &  & benchmark & 0 &  &  &  &  &  &  & \ref{a:SchuttWS05} & \ref{b:SchuttWS05}\\
\rowlabel{c:Vilim05}Vilim05 \href{https://doi.org/10.1007/11493853\_29}{Vilim05}~\cite{Vilim05} & \href{works/Vilim05.pdf}{Computing Explanations for the Unary Resource Constraint} &  & benchmark & 4 &  &  &  &  &  &  & \ref{a:Vilim05} & \ref{b:Vilim05}\\
\rowlabel{c:WolfS05}WolfS05 \href{https://doi.org/10.1007/11963578\_8}{WolfS05}~\cite{WolfS05} & \href{works/WolfS05.pdf}{\emph{O}(\emph{n} log\emph{n}) Overload Checking for the Cumulative Constraint and Its Application} &  & real-world & 0 &  &  &  &  &  &  & \ref{a:WolfS05} & \ref{b:WolfS05}\\
\rowlabel{c:WuBB05}WuBB05 \href{https://doi.org/10.1007/11564751\_110}{WuBB05}~\cite{WuBB05} & \href{works/WuBB05.pdf}{Scheduling with Uncertain Start Dates} &  & benchmark & 0 &  &  &  &  &  &  & \ref{a:WuBB05} & \ref{b:WuBB05}\\
\rowlabel{c:ArtiguesBF04}ArtiguesBF04 \href{https://doi.org/10.1007/978-3-540-24664-0\_3}{ArtiguesBF04}~\cite{ArtiguesBF04} & \href{works/ArtiguesBF04.pdf}{A New Exact Solution Algorithm for the Job Shop Problem with Sequence-Dependent Setup Times} &  & benchmark & 0 &  &  &  &  &  &  & \ref{a:ArtiguesBF04} & \ref{b:ArtiguesBF04}\\
\rowlabel{c:BeckW04}BeckW04 \href{}{BeckW04}~\cite{BeckW04} & \href{works/BeckW04.pdf}{Job Shop Scheduling with Probabilistic Durations} &  &  & 0 &  &  &  &  &  &  & \ref{a:BeckW04} & \ref{b:BeckW04}\\
\rowlabel{c:HentenryckM04}HentenryckM04 \href{https://doi.org/10.1007/978-3-540-24664-0\_22}{HentenryckM04}~\cite{HentenryckM04} & \href{works/HentenryckM04.pdf}{Scheduling Abstractions for Local Search} &  & benchmark & 0 &  &  &  &  &  &  & \ref{a:HentenryckM04} & \ref{b:HentenryckM04}\\
\rowlabel{c:Hooker04}Hooker04 \href{https://doi.org/10.1007/978-3-540-30201-8\_24}{Hooker04}~\cite{Hooker04} & \href{works/Hooker04.pdf}{A Hybrid Method for Planning and Scheduling} &  & random instance & 0 &  &  &  &  &  &  & \ref{a:Hooker04} & \ref{b:Hooker04}\\
\rowlabel{c:KovacsV04}KovacsV04 \href{https://doi.org/10.1007/978-3-540-30201-8\_26}{KovacsV04}~\cite{KovacsV04} & \href{works/KovacsV04.pdf}{Completable Partial Solutions in Constraint Programming and Constraint-Based Scheduling} &  & industrial partner, benchmark, real-life & 0 &  &  &  &  &  &  & \ref{a:KovacsV04} & \ref{b:KovacsV04}\\
\rowlabel{c:LimRX04}LimRX04 \href{https://doi.org/10.1007/978-3-540-30201-8\_59}{LimRX04}~\cite{LimRX04} & \href{works/LimRX04.pdf}{Solving the Crane Scheduling Problem Using Intelligent Search Schemes} &  & generated instance & 0 &  &  &  &  &  &  & \ref{a:LimRX04} & \ref{b:LimRX04}\\
\rowlabel{c:MaraveliasG04}MaraveliasG04 \href{https://doi.org/10.1007/978-3-540-24664-0\_1}{MaraveliasG04}~\cite{MaraveliasG04} & \href{works/MaraveliasG04.pdf}{Using {MILP} and {CP} for the Scheduling of Batch Chemical Processes} &  &  & 0 &  &  &  &  &  &  & \ref{a:MaraveliasG04} & \ref{b:MaraveliasG04}\\
\rowlabel{c:Sadykov04}Sadykov04 \href{https://doi.org/10.1007/978-3-540-24664-0\_31}{Sadykov04}~\cite{Sadykov04} & \href{works/Sadykov04.pdf}{A Hybrid Branch-And-Cut Algorithm for the One-Machine Scheduling Problem} &  &  & 0 &  &  &  &  &  &  & \ref{a:Sadykov04} & \ref{b:Sadykov04}\\
\rowlabel{c:Vilim04}Vilim04 \href{https://doi.org/10.1007/978-3-540-24664-0\_23}{Vilim04}~\cite{Vilim04} & \href{works/Vilim04.pdf}{O(n log n) Filtering Algorithms for Unary Resource Constraint} &  & benchmark & 1 &  &  &  &  &  &  & \ref{a:Vilim04} & \ref{b:Vilim04}\\
\rowlabel{c:VilimBC04}VilimBC04 \href{https://doi.org/10.1007/978-3-540-30201-8\_8}{VilimBC04}~\cite{VilimBC04} & \href{works/VilimBC04.pdf}{Unary Resource Constraint with Optional Activities} &  & benchmark, real-life & 0 &  &  &  &  &  &  & \ref{a:VilimBC04} & \ref{b:VilimBC04}\\
\rowlabel{c:VillaverdeP04}VillaverdeP04 \href{}{VillaverdeP04}~\cite{VillaverdeP04} & \href{}{An Investigation of Scheduling in Distributed Constraint Logic Programming} &  &  & 0 &  &  &  &  &  &  & \ref{a:VillaverdeP04} & No\\
\rowlabel{c:WolinskiKG04}WolinskiKG04 \href{https://doi.org/10.1109/DSD.2004.1333291}{WolinskiKG04}~\cite{WolinskiKG04} & \href{works/WolinskiKG04.pdf}{A Constraints Programming Approach to Communication Scheduling on SoPC Architectures} &  &  & 0 &  &  &  &  &  &  & \ref{a:WolinskiKG04} & \ref{b:WolinskiKG04}\\
\rowlabel{c:BeckPS03}BeckPS03 \href{http://www.aaai.org/Library/ICAPS/2003/icaps03-027.php}{BeckPS03}~\cite{BeckPS03} & \href{works/BeckPS03.pdf}{Vehicle Routing and Job Shop Scheduling: What's the Difference?} &  & benchmark, real-world & 0 &  &  &  &  &  &  & \ref{a:BeckPS03} & \ref{b:BeckPS03}\\
\rowlabel{c:DannaP03}DannaP03 \href{https://doi.org/10.1007/978-3-540-45193-8\_59}{DannaP03}~\cite{DannaP03} & \href{works/DannaP03.pdf}{Structured vs. Unstructured Large Neighborhood Search: {A} Case Study on Job-Shop Scheduling Problems with Earliness and Tardiness Costs} &  & benchmark & 0 &  &  &  &  &  &  & \ref{a:DannaP03} & \ref{b:DannaP03}\\
\rowlabel{c:Kumar03}Kumar03 \href{https://doi.org/10.1007/978-3-540-45193-8\_45}{Kumar03}~\cite{Kumar03} & \href{works/Kumar03.pdf}{Incremental Computation of Resource-Envelopes in Producer-Consumer Models} &  &  & 0 &  &  &  &  &  &  & \ref{a:Kumar03} & \ref{b:Kumar03}\\
\rowlabel{c:OddiPCC03}OddiPCC03 \href{https://doi.org/10.1007/978-3-540-45193-8\_39}{OddiPCC03}~\cite{OddiPCC03} & \href{works/OddiPCC03.pdf}{Generating High Quality Schedules for a Spacecraft Memory Downlink Problem} &  & benchmark & 0 &  &  &  &  &  &  & \ref{a:OddiPCC03} & \ref{b:OddiPCC03}\\
\rowlabel{c:ValleMGT03}ValleMGT03 \href{https://doi.org/10.1007/978-3-540-45226-3\_180}{ValleMGT03}~\cite{ValleMGT03} & \href{works/ValleMGT03.pdf}{On Selecting and Scheduling Assembly Plans Using Constraint Programming} &  & real-life & 0 &  &  &  &  &  &  & \ref{a:ValleMGT03} & \ref{b:ValleMGT03}\\
\rowlabel{c:Vilim03}Vilim03 \href{https://doi.org/10.1007/978-3-540-45193-8\_124}{Vilim03}~\cite{Vilim03} & \href{works/Vilim03.pdf}{Computing Explanations for Global Scheduling Constraints} &  &  & 0 &  &  &  &  &  &  & \ref{a:Vilim03} & \ref{b:Vilim03}\\
\rowlabel{c:Wolf03}Wolf03 \href{https://doi.org/10.1007/978-3-540-45193-8\_50}{Wolf03}~\cite{Wolf03} & \href{works/Wolf03.pdf}{Pruning while Sweeping over Task Intervals} &  & benchmark & 0 &  &  &  &  &  &  & \ref{a:Wolf03} & \ref{b:Wolf03}\\
\rowlabel{c:Bartak02}Bartak02 \href{https://doi.org/10.1007/3-540-46135-3\_39}{Bartak02}~\cite{Bartak02} & \href{works/Bartak02.pdf}{Visopt ShopFloor: On the Edge of Planning and Scheduling} &  & real-life & 0 &  &  &  &  &  &  & \ref{a:Bartak02} & \ref{b:Bartak02}\\
\rowlabel{c:Bartak02a}Bartak02a \href{https://doi.org/10.1007/3-540-36607-5\_14}{Bartak02a}~\cite{Bartak02a} & \href{works/Bartak02a.pdf}{Visopt ShopFloor: Going Beyond Traditional Scheduling} &  & benchmark, real-life & 0 &  &  &  &  &  &  & \ref{a:Bartak02a} & \ref{b:Bartak02a}\\
\rowlabel{c:BeldiceanuC02}BeldiceanuC02 \href{https://doi.org/10.1007/3-540-46135-3\_5}{BeldiceanuC02}~\cite{BeldiceanuC02} & \href{works/BeldiceanuC02.pdf}{A New Multi-resource cumulatives Constraint with Negative Heights} &  & real-life, random instance, benchmark & 0 &  &  &  &  &  &  & \ref{a:BeldiceanuC02} & \ref{b:BeldiceanuC02}\\
\rowlabel{c:ElkhyariGJ02}ElkhyariGJ02 \href{https://doi.org/10.1007/3-540-46135-3\_49}{ElkhyariGJ02}~\cite{ElkhyariGJ02} & \href{works/ElkhyariGJ02.pdf}{Conflict-Based Repair Techniques for Solving Dynamic Scheduling Problems} &  &  & 0 &  &  &  &  &  &  & \ref{a:ElkhyariGJ02} & \ref{b:ElkhyariGJ02}\\
\rowlabel{c:ElkhyariGJ02a}ElkhyariGJ02a \href{https://doi.org/10.1007/978-3-540-45157-0\_3}{ElkhyariGJ02a}~\cite{ElkhyariGJ02a} & \href{works/ElkhyariGJ02a.pdf}{Solving Dynamic Resource Constraint Project Scheduling Problems Using New Constraint Programming Tools} &  & benchmark, real-life & 0 &  &  &  &  &  &  & \ref{a:ElkhyariGJ02a} & \ref{b:ElkhyariGJ02a}\\
\rowlabel{c:HookerY02}HookerY02 \href{https://doi.org/10.1007/3-540-46135-3\_46}{HookerY02}~\cite{HookerY02} & \href{works/HookerY02.pdf}{A Relaxation of the Cumulative Constraint} &  &  & 0 &  &  &  &  &  &  & \ref{a:HookerY02} & \ref{b:HookerY02}\\
\rowlabel{c:KamarainenS02}KamarainenS02 \href{https://doi.org/10.1007/3-540-46135-3\_11}{KamarainenS02}~\cite{KamarainenS02} & \href{works/KamarainenS02.pdf}{Local Probing Applied to Scheduling} &  & real-world, benchmark & 2 &  &  &  &  &  &  & \ref{a:KamarainenS02} & \ref{b:KamarainenS02}\\
\rowlabel{c:Muscettola02}Muscettola02 \href{https://doi.org/10.1007/3-540-46135-3\_10}{Muscettola02}~\cite{Muscettola02} & \href{works/Muscettola02.pdf}{Computing the Envelope for Stepwise-Constant Resource Allocations} &  &  & 0 &  &  &  &  &  &  & \ref{a:Muscettola02} & \ref{b:Muscettola02}\\
\rowlabel{c:Vilim02}Vilim02 \href{https://doi.org/10.1007/3-540-46135-3\_62}{Vilim02}~\cite{Vilim02} & \href{works/Vilim02.pdf}{Batch Processing with Sequence Dependent Setup Times} &  &  & 0 &  &  &  &  &  &  & \ref{a:Vilim02} & \ref{b:Vilim02}\\
\rowlabel{c:ZhuS02}ZhuS02 \href{https://doi.org/10.1007/3-540-47961-9\_69}{ZhuS02}~\cite{ZhuS02} & \href{works/ZhuS02.pdf}{A Meeting Scheduling System Based on Open Constraint Programming} &  &  & 0 &  &  &  &  &  &  & \ref{a:ZhuS02} & \ref{b:ZhuS02}\\
\rowlabel{c:Thorsteinsson01}Thorsteinsson01 \href{https://doi.org/10.1007/3-540-45578-7\_2}{Thorsteinsson01}~\cite{Thorsteinsson01} & \href{works/Thorsteinsson01.pdf}{Branch-and-Check: {A} Hybrid Framework Integrating Mixed Integer Programming and Constraint Logic Programming} &  &  & 0 &  &  &  &  &  &  & \ref{a:Thorsteinsson01} & \ref{b:Thorsteinsson01}\\
\rowlabel{c:VanczaM01}VanczaM01 \href{https://doi.org/10.1007/3-540-45578-7\_60}{VanczaM01}~\cite{VanczaM01} & \href{works/VanczaM01.pdf}{A Constraint Engine for Manufacturing Process Planning} &  & real-life, real-world & 0 &  &  &  &  &  &  & \ref{a:VanczaM01} & \ref{b:VanczaM01}\\
\rowlabel{c:VerfaillieL01}VerfaillieL01 \href{https://doi.org/10.1007/3-540-45578-7\_55}{VerfaillieL01}~\cite{VerfaillieL01} & \href{works/VerfaillieL01.pdf}{Selecting and Scheduling Observations for Agile Satellites: Some Lessons from the Constraint Reasoning Community Point of View} &  &  & 0 &  &  &  &  &  &  & \ref{a:VerfaillieL01} & \ref{b:VerfaillieL01}\\
\rowlabel{c:AngelsmarkJ00}AngelsmarkJ00 \href{https://doi.org/10.1007/3-540-45349-0\_35}{AngelsmarkJ00}~\cite{AngelsmarkJ00} & \href{works/AngelsmarkJ00.pdf}{Some Observations on Durations, Scheduling and Allen's Algebra} &  &  & 0 &  &  &  &  &  &  & \ref{a:AngelsmarkJ00} & \ref{b:AngelsmarkJ00}\\
\rowlabel{c:FocacciLN00}FocacciLN00 \href{http://www.aaai.org/Library/AIPS/2000/aips00-010.php}{FocacciLN00}~\cite{FocacciLN00} & \href{works/FocacciLN00.pdf}{Solving Scheduling Problems with Setup Times and Alternative Resources} &  & real-world & 0 &  &  &  &  &  &  & \ref{a:FocacciLN00} & \ref{b:FocacciLN00}\\
\rowlabel{c:KorbaaYG99}KorbaaYG99 \href{https://doi.org/10.23919/ECC.1999.7099947}{KorbaaYG99}~\cite{KorbaaYG99} & \href{works/KorbaaYG99.pdf}{Solving transient scheduling problem for cyclic production using timed Petri nets and constraint programming} &  &  & 0 &  &  &  &  &  &  & \ref{a:KorbaaYG99} & \ref{b:KorbaaYG99}\\
\rowlabel{c:CestaOS98}CestaOS98 \href{https://doi.org/10.1007/3-540-49481-2\_36}{CestaOS98}~\cite{CestaOS98} & \href{works/CestaOS98.pdf}{Scheduling Multi-capacitated Resources Under Complex Temporal Constraints} &  &  & 0 &  &  &  &  &  &  & \ref{a:CestaOS98} & \ref{b:CestaOS98}\\
\rowlabel{c:FrostD98}FrostD98 \href{https://doi.org/10.1007/3-540-49481-2\_40}{FrostD98}~\cite{FrostD98} & \href{works/FrostD98.pdf}{Optimizing with Constraints: {A} Case Study in Scheduling Maintenance of Electric Power Units} &  &  & 0 &  &  &  &  &  &  & \ref{a:FrostD98} & \ref{b:FrostD98}\\
\rowlabel{c:GruianK98}GruianK98 \href{https://doi.org/10.1109/EURMIC.1998.711781}{GruianK98}~\cite{GruianK98} & \href{works/GruianK98.pdf}{Operation Binding and Scheduling for Low Power Using Constraint Logic Programming} &  & benchmark & 0 &  &  &  &  &  &  & \ref{a:GruianK98} & \ref{b:GruianK98}\\
\rowlabel{c:PembertonG98}PembertonG98 \href{https://doi.org/10.1090/dimacs/057/06}{PembertonG98}~\cite{PembertonG98} & \href{works/PembertonG98.pdf}{A constraint-based approach to satellite scheduling} &  &  & 0 &  &  &  &  &  &  & \ref{a:PembertonG98} & \ref{b:PembertonG98}\\
\rowlabel{c:RodosekW98}RodosekW98 \href{https://doi.org/10.1007/3-540-49481-2\_28}{RodosekW98}~\cite{RodosekW98} & \href{works/RodosekW98.pdf}{A Generic Model and Hybrid Algorithm for Hoist Scheduling Problems} &  & benchmark & 0 &  &  &  &  &  &  & \ref{a:RodosekW98} & \ref{b:RodosekW98}\\
\rowlabel{c:Shaw98}Shaw98 \href{https://doi.org/10.1007/3-540-49481-2\_30}{Shaw98}~\cite{Shaw98} & \href{works/Shaw98.pdf}{Using Constraint Programming and Local Search Methods to Solve Vehicle Routing Problems} &  & benchmark & 0 &  &  &  &  &  &  & \ref{a:Shaw98} & \ref{b:Shaw98}\\
\rowlabel{c:BaptisteP97}BaptisteP97 \href{https://doi.org/10.1007/BFb0017454}{BaptisteP97}~\cite{BaptisteP97} & \href{works/BaptisteP97.pdf}{Constraint Propagation and Decomposition Techniques for Highly Disjunctive and Highly Cumulative Project Scheduling Problems} &  & benchmark & 0 &  &  &  &  &  &  & \ref{a:BaptisteP97} & \ref{b:BaptisteP97}\\
\rowlabel{c:BeckDF97}BeckDF97 \href{https://doi.org/10.1007/BFb0017455}{BeckDF97}~\cite{BeckDF97} & \href{works/BeckDF97.pdf}{Five Pitfalls of Empirical Scheduling Research} &  & benchmark, real-world & 0 &  &  &  &  &  &  & \ref{a:BeckDF97} & \ref{b:BeckDF97}\\
\rowlabel{c:BoucherBVBL97}BoucherBVBL97 \href{}{BoucherBVBL97}~\cite{BoucherBVBL97} & \href{}{Multi-criteria Comparison Between Algorithmic, Constraint Logic and Specific Constraint Programming on a Real Schedulingt Problem} &  &  & 0 &  &  &  &  &  &  & \ref{a:BoucherBVBL97} & No\\
\rowlabel{c:Caseau97}Caseau97 \href{https://doi.org/10.1007/BFb0017437}{Caseau97}~\cite{Caseau97} & \href{works/Caseau97.pdf}{Using Constraint Propagation for Complex Scheduling Problems: Managing Size, Complex Resources and Travel} &  & benchmark & 0 &  &  &  &  &  &  & \ref{a:Caseau97} & \ref{b:Caseau97}\\
\rowlabel{c:PapeB97}PapeB97 \href{}{PapeB97}~\cite{PapeB97} & \href{}{A Constraint Programming Library for Preemptive and Non-Preemptive Scheduling} &  &  & 0 &  &  &  &  &  &  & \ref{a:PapeB97} & No\\
\rowlabel{c:BrusoniCLMMT96}BrusoniCLMMT96 \href{https://doi.org/10.1007/3-540-61286-6\_157}{BrusoniCLMMT96}~\cite{BrusoniCLMMT96} & \href{works/BrusoniCLMMT96.pdf}{Resource-Based vs. Task-Based Approaches for Scheduling Problems} &  &  & 0 &  &  &  &  &  &  & \ref{a:BrusoniCLMMT96} & \ref{b:BrusoniCLMMT96}\\
\rowlabel{c:Colombani96}Colombani96 \href{https://doi.org/10.1007/3-540-61551-2\_72}{Colombani96}~\cite{Colombani96} & \href{works/Colombani96.pdf}{Constraint Programming: an Efficient and Practical Approach to Solving the Job-Shop Problem} &  &  & 0 &  &  &  &  &  &  & \ref{a:Colombani96} & \ref{b:Colombani96}\\
\rowlabel{c:Zhou96}Zhou96 \href{https://doi.org/10.1007/3-540-61551-2\_97}{Zhou96}~\cite{Zhou96} & \href{works/Zhou96.pdf}{A Constraint Program for Solving the Job-Shop Problem} &  &  & 0 &  &  &  &  &  &  & \ref{a:Zhou96} & \ref{b:Zhou96}\\
\rowlabel{c:Goltz95}Goltz95 \href{https://doi.org/10.1007/3-540-60299-2\_33}{Goltz95}~\cite{Goltz95} & \href{works/Goltz95.pdf}{Reducing Domains for Search in {CLP(FD)} and Its Application to Job-Shop Scheduling} &  & benchmark & 0 &  &  &  &  &  &  & \ref{a:Goltz95} & \ref{b:Goltz95}\\
\rowlabel{c:Puget95}Puget95 \href{https://doi.org/10.1007/3-540-60299-2\_43}{Puget95}~\cite{Puget95} & \href{works/Puget95.pdf}{Applications of Constraint Programming} &  & benchmark & 0 &  &  &  &  &  &  & \ref{a:Puget95} & \ref{b:Puget95}\\
\rowlabel{c:Simonis95}Simonis95 \href{https://doi.org/10.1007/3-540-60299-2\_42}{Simonis95}~\cite{Simonis95} & \href{works/Simonis95.pdf}{The {CHIP} System and Its Applications} &  &  & 0 &  &  &  &  &  &  & \ref{a:Simonis95} & \ref{b:Simonis95}\\
\rowlabel{c:SimonisC95}SimonisC95 \href{https://doi.org/10.1007/3-540-60299-2\_27}{SimonisC95}~\cite{SimonisC95} & \href{works/SimonisC95.pdf}{Modelling Producer/Consumer Constraints} &  & real-life & 0 &  &  &  &  &  &  & \ref{a:SimonisC95} & \ref{b:SimonisC95}\\
\rowlabel{c:Touraivane95}Touraivane95 \href{https://doi.org/10.1007/3-540-60299-2\_41}{Touraivane95}~\cite{Touraivane95} & \href{works/Touraivane95.pdf}{Constraint Programming and Industrial Applications} &  & real-life & 0 &  &  &  &  &  &  & \ref{a:Touraivane95} & \ref{b:Touraivane95}\\
\rowlabel{c:JourdanFRD94}JourdanFRD94 \href{}{JourdanFRD94}~\cite{JourdanFRD94} & \href{}{Data Alignment and Task Scheduling On Parallel Machines Using Concurrent Constraint Model-based Programming} &  &  & 0 &  &  &  &  &  &  & \ref{a:JourdanFRD94} & No\\
\rowlabel{c:NuijtenA94}NuijtenA94 \href{}{NuijtenA94}~\cite{NuijtenA94} & \href{works/NuijtenA94.pdf}{Constraint Satisfaction for Multiple Capacitated Job Shop Scheduling} &  &  & 0 &  &  &  &  &  &  & \ref{a:NuijtenA94} & \ref{b:NuijtenA94}\\
\rowlabel{c:Wallace94}Wallace94 \href{}{Wallace94}~\cite{Wallace94} & \href{}{Applying Constraints for Scheduling} &  &  & 0 &  &  &  &  &  &  & \ref{a:Wallace94} & No\\
\rowlabel{c:BaptisteLV92}BaptisteLV92 \href{https://doi.org/10.1109/ROBOT.1992.220195}{BaptisteLV92}~\cite{BaptisteLV92} & \href{works/BaptisteLV92.pdf}{Hoist scheduling problem: an approach based on constraint logic programming} &  &  & 0 &  &  &  &  &  &  & \ref{a:BaptisteLV92} & \ref{b:BaptisteLV92}\\
\rowlabel{c:ErtlK91}ErtlK91 \href{https://doi.org/10.1007/3-540-54444-5\_89}{ErtlK91}~\cite{ErtlK91} & \href{works/ErtlK91.pdf}{Optimal Instruction Scheduling using Constraint Logic Programming} &  & real-world, benchmark & 0 &  &  &  &  &  &  & \ref{a:ErtlK91} & \ref{b:ErtlK91}\\
\end{longtable}
}



\clearpage
\section{Journal Articles}

\clearpage
\subsection{Articles from bibtex}
{\scriptsize
\begin{longtable}{>{\raggedright\arraybackslash}p{3cm}>{\raggedright\arraybackslash}p{6cm}>{\raggedright\arraybackslash}p{6.5cm}rrrp{2.5cm}rrrrr}
\rowcolor{white}\caption{Works from bibtex (Total 352)}\\ \toprule
\rowcolor{white}\shortstack{Key\\Source} & Authors & Title & LC & Cite & Year & \shortstack{Conference\\/Journal\\/School} & Pages & \shortstack{Nr\\Cites} & \shortstack{Nr\\Refs} & b & c \\ \midrule\endhead
\bottomrule
\endfoot
\rowlabel{a:ForbesHJST24}ForbesHJST24 \href{http://dx.doi.org/10.1016/j.ejor.2023.07.032}{ForbesHJST24} & \hyperref[auth:a998]{M. Forbes}, \hyperref[auth:a999]{M. Harris}, \hyperref[auth:a1000]{H. Jansen}, \hyperref[auth:a1001]{F.A. van der Schoot}, \hyperref[auth:a1002]{T. Taimre} & Combining optimisation and simulation using logic-based Benders decomposition & \href{../works/ForbesHJST24.pdf}{Yes} & \cite{ForbesHJST24} & 2024 & European Journal of Operational Research & 15 & 0 & 26 & \ref{b:ForbesHJST24} & \ref{c:ForbesHJST24}\\
\rowlabel{a:LuZZYW24}LuZZYW24 \href{https://www.mdpi.com/2077-1312/12/1/124}{LuZZYW24} & \hyperref[auth:a1279]{X. Lu}, \hyperref[auth:a1280]{Y. Zhang}, \hyperref[auth:a1281]{L. Zheng}, \hyperref[auth:a1282]{C. Yang}, \hyperref[auth:a1283]{J. Wang} & Integrated Inbound and Outbound Scheduling for Coal Port: Constraint Programming and Adaptive Local Search & \href{../works/LuZZYW24.pdf}{Yes} & \cite{LuZZYW24} & 2024 & Journal of Marine Science and Engineering & 36 & 0 & 0 & \ref{b:LuZZYW24} & \ref{c:LuZZYW24}\\
\rowlabel{a:PrataAN23}PrataAN23 \href{https://www.sciencedirect.com/science/article/pii/S2666720723001522}{PrataAN23} & \hyperref[auth:a390]{Bruno A. Prata}, \hyperref[auth:a391]{Levi R. Abreu}, \hyperref[auth:a392]{Marcelo S. Nagano} & Applications of constraint programming in production scheduling problems: A descriptive bibliometric analysis & \href{../works/PrataAN23.pdf}{Yes} & \cite{PrataAN23} & 2024 & Results in Control and Optimization & 17 & 0 & 0 & \ref{b:PrataAN23} & \ref{c:PrataAN23}\\
\rowlabel{a:abs-2402-00459}abs-2402-00459 \href{https://doi.org/10.48550/arXiv.2402.00459}{abs-2402-00459} & \hyperref[auth:a400]{S. Nguyen}, \hyperref[auth:a401]{Dhananjay R. Thiruvady}, \hyperref[auth:a402]{Y. Sun}, \hyperref[auth:a403]{M. Zhang} & Genetic-based Constraint Programming for Resource Constrained Job Scheduling & \href{../works/abs-2402-00459.pdf}{Yes} & \cite{abs-2402-00459} & 2024 & CoRR & 21 & 0 & 0 & \ref{b:abs-2402-00459} & \ref{c:abs-2402-00459}\\
\rowlabel{a:AbreuNP23}AbreuNP23 \href{https://doi.org/10.1080/00207543.2022.2154404}{AbreuNP23} & \hyperref[auth:a423]{Levi Ribeiro de Abreu}, \hyperref[auth:a424]{Marcelo Seido Nagano}, \hyperref[auth:a390]{Bruno A. Prata} & A new two-stage constraint programming approach for open shop scheduling problem with machine blocking & \href{../works/AbreuNP23.pdf}{Yes} & \cite{AbreuNP23} & 2023 & International Journal of Production Research & 20 & 1 & 47 & \ref{b:AbreuNP23} & \ref{c:AbreuNP23}\\
\rowlabel{a:AbreuPNF23}AbreuPNF23 \href{https://www.sciencedirect.com/science/article/pii/S0305054823002502}{AbreuPNF23} & \hyperref[auth:a391]{Levi R. Abreu}, \hyperref[auth:a390]{Bruno A. Prata}, \hyperref[auth:a392]{Marcelo S. Nagano}, \hyperref[auth:a842]{Jose M. Framinan} & A constraint programming-based iterated greedy algorithm for the open shop with sequence-dependent processing times and makespan minimization & \href{../works/AbreuPNF23.pdf}{Yes} & \cite{AbreuPNF23} & 2023 & Computers \  Operations Research & 12 & 0 & 46 & \ref{b:AbreuPNF23} & \ref{c:AbreuPNF23}\\
\rowlabel{a:Adelgren2023}Adelgren2023 \href{http://dx.doi.org/10.1016/j.cie.2023.109330}{Adelgren2023} & \hyperref[auth:a980]{N. Adelgren}, \hyperref[auth:a386]{Christos T. Maravelias} & On the utility of production scheduling formulations including record keeping variables & \href{../works/Adelgren2023.pdf}{Yes} & \cite{Adelgren2023} & 2023 & Computers \  Industrial Engineering & 12 & 0 & 43 & \ref{b:Adelgren2023} & \ref{c:Adelgren2023}\\
\rowlabel{a:AfsarVPG23}AfsarVPG23 \href{http://dx.doi.org/10.1016/j.cie.2023.109454}{AfsarVPG23} & \hyperref[auth:a974]{S. Afsar}, \hyperref[auth:a975]{Camino R. Vela}, \hyperref[auth:a976]{Juan José Palacios}, \hyperref[auth:a977]{I. González-Rodríguez} & Mathematical models and benchmarking for the fuzzy job shop scheduling problem & \href{../works/AfsarVPG23.pdf}{Yes} & \cite{AfsarVPG23} & 2023 & Computers \  Industrial Engineering & 14 & 0 & 50 & \ref{b:AfsarVPG23} & \ref{c:AfsarVPG23}\\
\rowlabel{a:AkramNHRSA23}AkramNHRSA23 \href{https://doi.org/10.1109/ACCESS.2023.3343409}{AkramNHRSA23} & \hyperref[auth:a404]{Bilal Omar Akram}, \hyperref[auth:a405]{Nor Kamariah Noordin}, \hyperref[auth:a406]{F. Hashim}, \hyperref[auth:a407]{Mohd Fadlee A. Rasid}, \hyperref[auth:a408]{Mustafa Ismael Salman}, \hyperref[auth:a409]{Abdulrahman M. Abdulghani} & Joint Scheduling and Routing Optimization for Deterministic Hybrid Traffic in Time-Sensitive Networks Using Constraint Programming & \href{../works/AkramNHRSA23.pdf}{Yes} & \cite{AkramNHRSA23} & 2023 & {IEEE} Access & 16 & 0 & 0 & \ref{b:AkramNHRSA23} & \ref{c:AkramNHRSA23}\\
\rowlabel{a:AlfieriGPS23}AlfieriGPS23 \href{https://www.sciencedirect.com/science/article/pii/S0360835223000074}{AlfieriGPS23} & \hyperref[auth:a737]{A. Alfieri}, \hyperref[auth:a15]{M. Garraffa}, \hyperref[auth:a738]{E. Pastore}, \hyperref[auth:a739]{F. Salassa} & Permutation flowshop problems minimizing core waiting time and core idle time & \href{../works/AlfieriGPS23.pdf}{Yes} & \cite{AlfieriGPS23} & 2023 & Computers \  Industrial Engineering & 13 & 0 & 37 & \ref{b:AlfieriGPS23} & \ref{c:AlfieriGPS23}\\
\rowlabel{a:Caballero23}Caballero23 \href{https://doi.org/10.1007/s10601-023-09357-0}{Caballero23} & \hyperref[auth:a102]{Jordi Coll Caballero} & Scheduling through logic-based tools & \href{../works/Caballero23.pdf}{Yes} & \cite{Caballero23} & 2023 & Constraints An Int. J. & 1 & 0 & 0 & \ref{b:Caballero23} & \ref{c:Caballero23}\\
\rowlabel{a:CzerniachowskaWZ23}CzerniachowskaWZ23 \href{https://doi.org/10.12913/22998624/166588}{CzerniachowskaWZ23} & \hyperref[auth:a740]{K. Czerniachowska}, \hyperref[auth:a741]{R. Wichniarek}, \hyperref[auth:a742]{K. Żywicki} & Constraint Programming for Flexible Flow Shop Scheduling Problem with Repeated Jobs and Repeated Operations & \href{../works/CzerniachowskaWZ23.pdf}{Yes} & \cite{CzerniachowskaWZ23} & 2023 & Advances in Science and Technology Research Journal & 14 & 0 & 0 & \ref{b:CzerniachowskaWZ23} & \ref{c:CzerniachowskaWZ23}\\
\rowlabel{a:FahimiQ23}FahimiQ23 \href{http://dx.doi.org/10.1287/ijoc.2021.0138}{FahimiQ23} & \hyperref[auth:a122]{H. Fahimi}, \hyperref[auth:a123]{C. Quimper} & Overload-Checking and Edge-Finding for Robust Cumulative Scheduling & No & \cite{FahimiQ23} & 2023 & INFORMS Journal on Computing & null & 0 & 16 & No & \ref{c:FahimiQ23}\\
\rowlabel{a:Fatemi-AnarakiTFV23}Fatemi-AnarakiTFV23 \href{http://dx.doi.org/10.1016/j.omega.2022.102770}{Fatemi-AnarakiTFV23} & \hyperref[auth:a743]{S. Fatemi-Anaraki}, \hyperref[auth:a435]{R. Tavakkoli{-}Moghaddam}, \hyperref[auth:a744]{M. Foumani}, \hyperref[auth:a745]{B. Vahedi-Nouri} & Scheduling of Multi-Robot Job Shop Systems in Dynamic Environments: Mixed-Integer Linear Programming and Constraint Programming Approaches & \href{../works/Fatemi-AnarakiTFV23.pdf}{Yes} & \cite{Fatemi-AnarakiTFV23} & 2023 & Omega & 15 & 7 & 60 & \ref{b:Fatemi-AnarakiTFV23} & \ref{c:Fatemi-AnarakiTFV23}\\
\rowlabel{a:GhasemiMH23}GhasemiMH23 \href{http://dx.doi.org/10.1080/23302674.2023.2224509}{GhasemiMH23} & \hyperref[auth:a996]{S. Ghasemi}, \hyperref[auth:a435]{R. Tavakkoli{-}Moghaddam}, \hyperref[auth:a997]{M. Hamid} & Operating room scheduling by emphasising human factors and dynamic decision-making styles: a constraint programming method & No & \cite{GhasemiMH23} & 2023 & International Journal of Systems Science: Operations \  Logistics & null & 0 & 104 & No & \ref{c:GhasemiMH23}\\
\rowlabel{a:GokPTGO23}GokPTGO23 \href{https://ideas.repec.org/a/spr/annopr/v320y2023i2d10.1007_s10479-022-04547-0.html}{GokPTGO23} & \hyperref[auth:a1024]{Yagmur S. G{\"{o}}k}, \hyperref[auth:a1025]{S. Padr{\'{o}}n}, \hyperref[auth:a1026]{M. Tomasella}, \hyperref[auth:a1027]{D. Guimarans}, \hyperref[auth:a136]{C. {\"{O}}zt{\"{u}}rk} & {Constraint-based robust planning and scheduling of airport apron operations through simheuristics} & \href{../works/GokPTGO23.pdf}{Yes} & \cite{GokPTGO23} & 2023 & Annals of Operations Research & 36 & 0 & 0 & \ref{b:GokPTGO23} & \ref{c:GokPTGO23}\\
\rowlabel{a:GuoZ23}GuoZ23 \href{http://dx.doi.org/10.1016/j.ejor.2023.06.006}{GuoZ23} & \hyperref[auth:a955]{P. Guo}, \hyperref[auth:a956]{J. Zhu} & Capacity reservation for humanitarian relief: A logic-based Benders decomposition method with subgradient cut & \href{../works/GuoZ23.pdf}{Yes} & \cite{GuoZ23} & 2023 & European Journal of Operational Research & 29 & 0 & 112 & \ref{b:GuoZ23} & \ref{c:GuoZ23}\\
\rowlabel{a:GurPAE23}GurPAE23 \href{https://doi.org/10.1007/s10100-022-00835-z}{GurPAE23} & \hyperref[auth:a417]{S. G{\"{u}}r}, \hyperref[auth:a418]{M. Pinarbasi}, \hyperref[auth:a419]{Haci Mehmet Alakas}, \hyperref[auth:a420]{T. Eren} & Operating room scheduling with surgical team: a new approach with constraint programming and goal programming & \href{../works/GurPAE23.pdf}{Yes} & \cite{GurPAE23} & 2023 & Central Eur. J. Oper. Res. & 25 & 1 & 40 & \ref{b:GurPAE23} & \ref{c:GurPAE23}\\
\rowlabel{a:IsikYA23}IsikYA23 \href{https://doi.org/10.1007/s00500-023-09086-9}{IsikYA23} & \hyperref[auth:a425]{Ey{\"{u}}p Ensar Isik}, \hyperref[auth:a426]{Seyda Topaloglu Yildiz}, \hyperref[auth:a427]{{\"{O}}zge Satir Akpunar} & Constraint programming models for the hybrid flow shop scheduling problem and its extensions & \href{../works/IsikYA23.pdf}{Yes} & \cite{IsikYA23} & 2023 & Soft Comput. & 28 & 0 & 127 & \ref{b:IsikYA23} & \ref{c:IsikYA23}\\
\rowlabel{a:JuvinHL23a}JuvinHL23a \href{http://dx.doi.org/10.1016/j.cor.2023.106156}{JuvinHL23a} & \hyperref[auth:a0]{C. Juvin}, \hyperref[auth:a2]{L. Houssin}, \hyperref[auth:a3]{P. Lopez} & Logic-based Benders decomposition for the preemptive flexible job-shop scheduling problem & \href{../works/JuvinHL23a.pdf}{Yes} & \cite{JuvinHL23a} & 2023 & Computers \  Operations Research & 17 & 0 & 40 & \ref{b:JuvinHL23a} & \ref{c:JuvinHL23a}\\
\rowlabel{a:LacknerMMWW23}LacknerMMWW23 \href{https://doi.org/10.1007/s10601-023-09347-2}{LacknerMMWW23} & \hyperref[auth:a62]{M. Lackner}, \hyperref[auth:a63]{C. Mrkvicka}, \hyperref[auth:a45]{N. Musliu}, \hyperref[auth:a46]{D. Walkiewicz}, \hyperref[auth:a43]{F. Winter} & Exact methods for the Oven Scheduling Problem & \href{../works/LacknerMMWW23.pdf}{Yes} & \cite{LacknerMMWW23} & 2023 & Constraints An Int. J. & 42 & 0 & 32 & \ref{b:LacknerMMWW23} & \ref{c:LacknerMMWW23}\\
\rowlabel{a:MarliereSPR23}MarliereSPR23 \href{https://www.sciencedirect.com/science/article/pii/S0967066122002611}{MarliereSPR23} & \hyperref[auth:a1033]{G. Marlière}, \hyperref[auth:a1034]{Sonia {Sobieraj Richard}}, \hyperref[auth:a1035]{P. Pellegrini}, \hyperref[auth:a789]{J. Rodriguez} & A conditional time-intervals formulation of the real-time Railway Traffic Management Problem & \href{../works/MarliereSPR23.pdf}{Yes} & \cite{MarliereSPR23} & 2023 & Control Engineering Practice & 22 & 1 & 75 & \ref{b:MarliereSPR23} & \ref{c:MarliereSPR23}\\
\rowlabel{a:MontemanniD23}MontemanniD23 \href{https://doi.org/10.3390/a16010040}{MontemanniD23} & \hyperref[auth:a415]{R. Montemanni}, \hyperref[auth:a416]{M. Dell'Amico} & Solving the Parallel Drone Scheduling Traveling Salesman Problem via Constraint Programming & \href{../works/MontemanniD23.pdf}{Yes} & \cite{MontemanniD23} & 2023 & Algorithms & 13 & 2 & 18 & \ref{b:MontemanniD23} & \ref{c:MontemanniD23}\\
\rowlabel{a:MontemanniD23a}MontemanniD23a \href{https://doi.org/10.1016/j.ejco.2023.100078}{MontemanniD23a} & \hyperref[auth:a415]{R. Montemanni}, \hyperref[auth:a416]{M. Dell'Amico} & Constraint programming models for the parallel drone scheduling vehicle routing problem & \href{../works/MontemanniD23a.pdf}{Yes} & \cite{MontemanniD23a} & 2023 & {EURO} J. Comput. Optim. & 20 & 0 & 14 & \ref{b:MontemanniD23a} & \ref{c:MontemanniD23a}\\
\rowlabel{a:NaderiBZ23}NaderiBZ23 \href{http://dx.doi.org/10.2139/ssrn.4494381}{NaderiBZ23} & \hyperref[auth:a734]{B. Naderi}, \hyperref[auth:a845]{Mehmet A. Begen}, \hyperref[auth:a846]{G. Zhang} & Integrated Order Acceptance and Resource Decisions Under Uncertainty: Robust and Stochastic Approaches & \href{../works/NaderiBZ23.pdf}{Yes} & \cite{NaderiBZ23} & 2023 & SSRN & 32 & 0 & 46 & \ref{b:NaderiBZ23} & \ref{c:NaderiBZ23}\\
\rowlabel{a:NaderiBZR23}NaderiBZR23 \href{http://dx.doi.org/10.1016/j.omega.2022.102805}{NaderiBZR23} & \hyperref[auth:a734]{B. Naderi}, \hyperref[auth:a845]{Mehmet A. Begen}, \hyperref[auth:a847]{Gregory S. Zaric}, \hyperref[auth:a736]{V. Roshanaei} & A novel and efficient exact technique for integrated staffing,  assignment,  routing,  and scheduling of home care services under uncertainty & No & \cite{NaderiBZR23} & 2023 & Omega & 1 & 4 & 64 & No & \ref{c:NaderiBZR23}\\
\rowlabel{a:NaderiRR23}NaderiRR23 \href{https://doi.org/10.1287/ijoc.2023.1287}{NaderiRR23} & \hyperref[auth:a734]{B. Naderi}, \hyperref[auth:a735]{R. Ruiz}, \hyperref[auth:a736]{V. Roshanaei} & Mixed-Integer Programming vs. Constraint Programming for Shop Scheduling Problems: New Results and Outlook & \href{../works/NaderiRR23.pdf}{Yes} & \cite{NaderiRR23} & 2023 & INFORMS Journal on Computing & 27 & 2 & 50 & \ref{b:NaderiRR23} & \ref{c:NaderiRR23}\\
\rowlabel{a:NouriMHD23}NouriMHD23 \href{http://dx.doi.org/10.1080/00207543.2023.2173503}{NouriMHD23} & \hyperref[auth:a745]{B. Vahedi-Nouri}, \hyperref[auth:a958]{R. Tavakkoli-Moghaddam}, \hyperref[auth:a959]{Z. Hanzálek}, \hyperref[auth:a960]{A. Dolgui} & Production scheduling in a reconfigurable manufacturing system benefiting from human-robot collaboration & No & \cite{NouriMHD23} & 2023 & International Journal of Production Research & null & 2 & 44 & No & \ref{c:NouriMHD23}\\
\rowlabel{a:PenzDN23}PenzDN23 \href{https://doi.org/10.1016/j.cor.2022.106092}{PenzDN23} & \hyperref[auth:a1007]{L. Penz}, \hyperref[auth:a1008]{S. Dauz{\`{e}}re{-}P{\'{e}}r{\`{e}}s}, \hyperref[auth:a81]{M. Nattaf} & Minimizing the sum of completion times on a single machine with health index and flexible maintenance operations & \href{../works/PenzDN23.pdf}{Yes} & \cite{PenzDN23} & 2023 & Computers \  Operations Research & 13 & 0 & 34 & \ref{b:PenzDN23} & \ref{c:PenzDN23}\\
\rowlabel{a:ShaikhK23}ShaikhK23 \href{https://doi.org/10.1504/IJESDF.2023.10045616}{ShaikhK23} & \hyperref[auth:a421]{Aftab Ahmed Shaikh}, \hyperref[auth:a422]{Abdullah Ayub Khan} & Management of electronic ledger: a constraint programming approach for solving curricula scheduling problems & \href{../works/ShaikhK23.pdf}{Yes} & \cite{ShaikhK23} & 2023 & Int. J. Electron. Secur. Digit. Forensics & 12 & 0 & 0 & \ref{b:ShaikhK23} & \ref{c:ShaikhK23}\\
\rowlabel{a:YuraszeckMCCR23}YuraszeckMCCR23 \href{https://doi.org/10.1109/ACCESS.2023.3345793}{YuraszeckMCCR23} & \hyperref[auth:a410]{F. Yuraszeck}, \hyperref[auth:a411]{E. Montero}, \hyperref[auth:a412]{D. Canut{-}de{-}Bon}, \hyperref[auth:a413]{N. Cuneo}, \hyperref[auth:a414]{M. Rojel} & A Constraint Programming Formulation of the Multi-Mode Resource-Constrained Project Scheduling Problem for the Flexible Job Shop Scheduling Problem & \href{../works/YuraszeckMCCR23.pdf}{Yes} & \cite{YuraszeckMCCR23} & 2023 & {IEEE} Access & 11 & 0 & 0 & \ref{b:YuraszeckMCCR23} & \ref{c:YuraszeckMCCR23}\\
\rowlabel{a:ZhuSZW23}ZhuSZW23 \href{http://dx.doi.org/10.1016/j.omega.2022.102823}{ZhuSZW23} & \hyperref[auth:a1003]{X. Zhu}, \hyperref[auth:a1004]{J. Son}, \hyperref[auth:a1005]{X. Zhang}, \hyperref[auth:a1006]{J. Wu} & Constraint programming and logic-based Benders decomposition for the integrated process planning and scheduling problem & \href{../works/ZhuSZW23.pdf}{Yes} & \cite{ZhuSZW23} & 2023 & Omega & 22 & 1 & 36 & \ref{b:ZhuSZW23} & \ref{c:ZhuSZW23}\\
\rowlabel{a:abs-2305-19888}abs-2305-19888 \href{https://doi.org/10.48550/arXiv.2305.19888}{abs-2305-19888} & \hyperref[auth:a438]{V. Heinz}, \hyperref[auth:a439]{A. Nov{\'{a}}k}, \hyperref[auth:a313]{M. Vlk}, \hyperref[auth:a116]{Z. Hanz{\'{a}}lek} & Constraint Programming and Constructive Heuristics for Parallel Machine Scheduling with Sequence-Dependent Setups and Common Servers & \href{../works/abs-2305-19888.pdf}{Yes} & \cite{abs-2305-19888} & 2023 & CoRR & 42 & 0 & 0 & \ref{b:abs-2305-19888} & \ref{c:abs-2305-19888}\\
\rowlabel{a:abs-2306-05747}abs-2306-05747 \href{https://doi.org/10.48550/arXiv.2306.05747}{abs-2306-05747} & \hyperref[auth:a58]{P. Tassel}, \hyperref[auth:a61]{M. Gebser}, \hyperref[auth:a428]{K. Schekotihin} & An End-to-End Reinforcement Learning Approach for Job-Shop Scheduling Problems Based on Constraint Programming & \href{../works/abs-2306-05747.pdf}{Yes} & \cite{abs-2306-05747} & 2023 & CoRR & 9 & 0 & 0 & \ref{b:abs-2306-05747} & \ref{c:abs-2306-05747}\\
\rowlabel{a:abs-2312-13682}abs-2312-13682 \href{https://doi.org/10.48550/arXiv.2312.13682}{abs-2312-13682} & \hyperref[auth:a430]{G. Perez}, \hyperref[auth:a431]{G. Glorian}, \hyperref[auth:a432]{W. Suijlen}, \hyperref[auth:a433]{A. Lallouet} & A Constraint Programming Model for Scheduling the Unloading of Trains in Ports: Extended & \href{../works/abs-2312-13682.pdf}{Yes} & \cite{abs-2312-13682} & 2023 & CoRR & 20 & 0 & 0 & \ref{b:abs-2312-13682} & \ref{c:abs-2312-13682}\\
\rowlabel{a:AbreuN22}AbreuN22 \href{https://doi.org/10.1016/j.cie.2022.108128}{AbreuN22} & \hyperref[auth:a423]{Levi Ribeiro de Abreu}, \hyperref[auth:a424]{Marcelo Seido Nagano} & A new hybridization of adaptive large neighborhood search with constraint programming for open shop scheduling with sequence-dependent setup times & \href{../works/AbreuN22.pdf}{Yes} & \cite{AbreuN22} & 2022 & Computers \  Industrial Engineering & 20 & 10 & 56 & \ref{b:AbreuN22} & \ref{c:AbreuN22}\\
\rowlabel{a:AwadMDMT22}AwadMDMT22 \href{http://dx.doi.org/10.1016/j.compchemeng.2021.107565}{AwadMDMT22} & \hyperref[auth:a1195]{M. Awad}, \hyperref[auth:a1196]{K. Mulrennan}, \hyperref[auth:a1197]{J. Donovan}, \hyperref[auth:a1198]{R. Macpherson}, \hyperref[auth:a1199]{D. Tormey} & A constraint programming model for makespan minimisation in batch manufacturing pharmaceutical facilities & No & \cite{AwadMDMT22} & 2022 & Computers \  Chemical Engineering & 1 & 3 & 41 & No & \ref{c:AwadMDMT22}\\
\rowlabel{a:BourreauGGLT22}BourreauGGLT22 \href{https://doi.org/10.1080/00207543.2020.1856436}{BourreauGGLT22} & \hyperref[auth:a446]{E. Bourreau}, \hyperref[auth:a447]{T. Garaix}, \hyperref[auth:a448]{M. Gondran}, \hyperref[auth:a449]{P. Lacomme}, \hyperref[auth:a450]{N. Tchernev} & A constraint-programming based decomposition method for the Generalised Workforce Scheduling and Routing Problem {(GWSRP)} & \href{../works/BourreauGGLT22.pdf}{Yes} & \cite{BourreauGGLT22} & 2022 & International Journal of Production Research & 19 & 4 & 44 & \ref{b:BourreauGGLT22} & \ref{c:BourreauGGLT22}\\
\rowlabel{a:CampeauG22}CampeauG22 \href{https://doi.org/10.1007/s10601-022-09337-w}{CampeauG22} & \hyperref[auth:a103]{L. Campeau}, \hyperref[auth:a9]{M. Gamache} & Short- and medium-term optimization of underground mine planning using constraint programming & \href{../works/CampeauG22.pdf}{Yes} & \cite{CampeauG22} & 2022 & Constraints An Int. J. & 18 & 0 & 22 & \ref{b:CampeauG22} & \ref{c:CampeauG22}\\
\rowlabel{a:ColT22}ColT22 \href{http://dx.doi.org/10.1016/j.orp.2022.100249}{ColT22} & \hyperref[auth:a93]{Giacomo Da Col}, \hyperref[auth:a746]{Erich C. Teppan} & Industrial-size job shop scheduling with constraint programming & \href{../works/ColT22.pdf}{Yes} & \cite{ColT22} & 2022 & Operations Research Perspectives & 19 & 3 & 55 & \ref{b:ColT22} & \ref{c:ColT22}\\
\rowlabel{a:ElciOH22}ElciOH22 \href{http://dx.doi.org/10.1287/ijoc.2022.1184}{ElciOH22} & \hyperref[auth:a942]{\"{O}zg\"{u}n El\c{c}i}, \hyperref[auth:a161]{John N. Hooker} & Stochastic Planning and Scheduling with Logic-Based Benders Decomposition & \href{../works/ElciOH22.pdf}{Yes} & \cite{ElciOH22} & 2022 & INFORMS Journal on Computing & 21 & 2 & 34 & \ref{b:ElciOH22} & \ref{c:ElciOH22}\\
\rowlabel{a:EmdeZD22}EmdeZD22 \href{http://dx.doi.org/10.1007/s10479-022-04891-1}{EmdeZD22} & \hyperref[auth:a969]{S. Emde}, \hyperref[auth:a970]{S. Zehtabian}, \hyperref[auth:a971]{Y. Disser} & Point-to-point and milk run delivery scheduling: models,  complexity results,  and algorithms based on Benders decomposition & \href{../works/EmdeZD22.pdf}{Yes} & \cite{EmdeZD22} & 2022 & Annals of Operations Research & 30 & 0 & 52 & \ref{b:EmdeZD22} & \ref{c:EmdeZD22}\\
\rowlabel{a:EtminaniesfahaniGNMS22}EtminaniesfahaniGNMS22 \href{http://dx.doi.org/10.1007/s42979-022-01487-1}{EtminaniesfahaniGNMS22} & \hyperref[auth:a910]{A. Etminaniesfahani}, \hyperref[auth:a341]{H. Gu}, \hyperref[auth:a911]{Leila Moslemi Naeni}, \hyperref[auth:a912]{A. Salehipour} & A Forward–Backward Relax-and-Solve Algorithm for the Resource-Constrained Project Scheduling Problem & \href{../works/EtminaniesfahaniGNMS22.pdf}{Yes} & \cite{EtminaniesfahaniGNMS22} & 2022 & SN Computer Science & 10 & 0 & 57 & \ref{b:EtminaniesfahaniGNMS22} & \ref{c:EtminaniesfahaniGNMS22}\\
\rowlabel{a:FarsiTM22}FarsiTM22 \href{https://api.semanticscholar.org/CorpusID:250301745}{FarsiTM22} & \hyperref[auth:a521]{A. Farsi}, \hyperref[auth:a747]{S. Ali Torabi}, \hyperref[auth:a520]{M. Mokhtarzadeh} & Integrated surgery scheduling by constraint programming and meta-heuristics & \href{../works/FarsiTM22.pdf}{Yes} & \cite{FarsiTM22} & 2022 & International Journal of Management Science and Engineering Management & 14 & 5 & 47 & \ref{b:FarsiTM22} & \ref{c:FarsiTM22}\\
\rowlabel{a:FetgoD22}FetgoD22 \href{https://doi.org/10.1007/s43069-022-00172-6}{FetgoD22} & \hyperref[auth:a11]{S{\'{e}}v{\'{e}}rine Betmbe Fetgo}, \hyperref[auth:a13]{Cl{\'{e}}mentin Tayou Djam{\'{e}}gni} & Horizontally Elastic Edge-Finder Algorithm for Cumulative Resource Constraint Revisited & \href{../works/FetgoD22.pdf}{Yes} & \cite{FetgoD22} & 2022 & Oper. Res. Forum & 32 & 0 & 20 & \ref{b:FetgoD22} & \ref{c:FetgoD22}\\
\rowlabel{a:HeinzNVH22}HeinzNVH22 \href{https://doi.org/10.1016/j.cie.2022.108586}{HeinzNVH22} & \hyperref[auth:a438]{V. Heinz}, \hyperref[auth:a439]{A. Nov{\'{a}}k}, \hyperref[auth:a313]{M. Vlk}, \hyperref[auth:a116]{Z. Hanz{\'{a}}lek} & Constraint Programming and constructive heuristics for parallel machine scheduling with sequence-dependent setups and common servers & \href{../works/HeinzNVH22.pdf}{Yes} & \cite{HeinzNVH22} & 2022 & Computers \  Industrial Engineering & 16 & 5 & 25 & \ref{b:HeinzNVH22} & \ref{c:HeinzNVH22}\\
\rowlabel{a:HillBCGN22}HillBCGN22 \href{http://dx.doi.org/10.1287/ijoc.2022.1222}{HillBCGN22} & \hyperref[auth:a64]{A. Hill}, \hyperref[auth:a984]{Andrea J. Brickey}, \hyperref[auth:a985]{I. Cipriano}, \hyperref[auth:a986]{M. Goycoolea}, \hyperref[auth:a987]{A. Newman} & Optimization Strategies for Resource-Constrained Project Scheduling Problems in Underground Mining & No & \cite{HillBCGN22} & 2022 & INFORMS Journal on Computing & null & 0 & 53 & No & \ref{c:HillBCGN22}\\
\rowlabel{a:JuvinHL22}JuvinHL22 \href{http://dx.doi.org/10.2139/ssrn.4068164}{JuvinHL22} & \hyperref[auth:a0]{C. Juvin}, \hyperref[auth:a2]{L. Houssin}, \hyperref[auth:a3]{P. Lopez} & Logic-Based Benders Decomposition for the Preemptive Flexible Job-Shop Scheduling Problem & \href{../works/JuvinHL22.pdf}{Yes} & \cite{JuvinHL22} & 2022 & SSRN Electronic Journal & 32 & 0 & 29 & \ref{b:JuvinHL22} & \ref{c:JuvinHL22}\\
\rowlabel{a:MartnezAJ22}MartnezAJ22 \href{http://dx.doi.org/10.1287/ijoc.2021.1079}{MartnezAJ22} & \hyperref[auth:a947]{Karim Pérez Martínez}, \hyperref[auth:a948]{Y. Adulyasak}, \hyperref[auth:a850]{R. Jans} & Logic-Based Benders Decomposition for Integrated Process Configuration and Production Planning Problems & No & \cite{MartnezAJ22} & 2022 & INFORMS Journal on Computing & null & 1 & 29 & No & \ref{c:MartnezAJ22}\\
\rowlabel{a:MengGRZSC22}MengGRZSC22 \href{http://dx.doi.org/10.1016/j.swevo.2022.101058}{MengGRZSC22} & \hyperref[auth:a505]{L. Meng}, \hyperref[auth:a1200]{K. Gao}, \hyperref[auth:a507]{Y. Ren}, \hyperref[auth:a508]{B. Zhang}, \hyperref[auth:a1179]{H. Sang}, \hyperref[auth:a1201]{Z. Chaoyong} & Novel MILP and CP models for distributed hybrid flowshop scheduling problem with sequence-dependent setup times & No & \cite{MengGRZSC22} & 2022 & Swarm and Evolutionary Computation & 1 & 38 & 37 & No & \ref{c:MengGRZSC22}\\
\rowlabel{a:MullerMKP22}MullerMKP22 \href{https://doi.org/10.1016/j.ejor.2022.01.034}{MullerMKP22} & \hyperref[auth:a440]{D. M{\"{u}}ller}, \hyperref[auth:a441]{Marcus Gerhard M{\"{u}}ller}, \hyperref[auth:a442]{D. Kress}, \hyperref[auth:a443]{E. Pesch} & An algorithm selection approach for the flexible job shop scheduling problem: Choosing constraint programming solvers through machine learning & \href{../works/MullerMKP22.pdf}{Yes} & \cite{MullerMKP22} & 2022 & European Journal of Operational Research & 18 & 17 & 59 & \ref{b:MullerMKP22} & \ref{c:MullerMKP22}\\
\rowlabel{a:NaderiBZ22}NaderiBZ22 \href{http://dx.doi.org/10.2139/ssrn.4140716}{NaderiBZ22} & \hyperref[auth:a734]{B. Naderi}, \hyperref[auth:a845]{Mehmet A. Begen}, \hyperref[auth:a846]{G. Zhang} & Integrated Order Acceptance and Resource Decisions Under Uncertainty: Robust and Stochastic Approaches & \href{../works/NaderiBZ22.pdf}{Yes} & \cite{NaderiBZ22} & 2022 & SSRN Electronic Journal & 29 & 0 & 44 & \ref{b:NaderiBZ22} & \ref{c:NaderiBZ22}\\
\rowlabel{a:NaderiBZ22a}NaderiBZ22a \href{http://dx.doi.org/10.1016/j.cor.2022.105728}{NaderiBZ22a} & \hyperref[auth:a734]{B. Naderi}, \hyperref[auth:a845]{Mehmet A. Begen}, \hyperref[auth:a847]{Gregory S. Zaric} & Type-2 integrated process-planning and scheduling problem: Reformulation and solution algorithms & \href{../works/NaderiBZ22a.pdf}{Yes} & \cite{NaderiBZ22a} & 2022 & Computers \  Operations Research & 19 & 3 & 44 & \ref{b:NaderiBZ22a} & \ref{c:NaderiBZ22a}\\
\rowlabel{a:NaderiR22}NaderiR22 \href{http://dx.doi.org/10.1287/ijoo.2021.0056}{NaderiR22} & \hyperref[auth:a734]{B. Naderi}, \hyperref[auth:a736]{V. Roshanaei} & Critical-Path-Search Logic-Based Benders Decomposition Approaches for Flexible Job Shop Scheduling & No & \cite{NaderiR22} & 2022 & INFORMS Journal on Optimization & null & 5 & 49 & No & \ref{c:NaderiR22}\\
\rowlabel{a:OrnekOS20}OrnekOS20 \href{https://ideas.repec.org/a/spr/operea/v22y2022i1d10.1007_s12351-020-00563-9.html}{OrnekOS20} & \hyperref[auth:a139]{A. {\"{O}}rnek}, \hyperref[auth:a136]{C. {\"{O}}zt{\"{u}}rk}, \hyperref[auth:a1028]{I. Sugut} & {Integer and constraint programming model formulations for flight-gate assignment problem} & \href{../works/OrnekOS20.pdf}{Yes} & \cite{OrnekOS20} & 2022 & Operational Research & 29 & 0 & 0 & \ref{b:OrnekOS20} & \ref{c:OrnekOS20}\\
\rowlabel{a:PohlAK22}PohlAK22 \href{https://doi.org/10.1016/j.ejor.2021.08.028}{PohlAK22} & \hyperref[auth:a444]{M. Pohl}, \hyperref[auth:a6]{C. Artigues}, \hyperref[auth:a445]{R. Kolisch} & Solving the time-discrete winter runway scheduling problem: {A} column generation and constraint programming approach & \href{../works/PohlAK22.pdf}{Yes} & \cite{PohlAK22} & 2022 & European Journal of Operational Research & 16 & 4 & 31 & \ref{b:PohlAK22} & \ref{c:PohlAK22}\\
\rowlabel{a:ShiYXQ22}ShiYXQ22 \href{https://doi.org/10.1080/00207543.2021.1963496}{ShiYXQ22} & \hyperref[auth:a451]{G. Shi}, \hyperref[auth:a452]{Z. Yang}, \hyperref[auth:a453]{Y. Xu}, \hyperref[auth:a454]{Y. Quan} & Solving the integrated process planning and scheduling problem using an enhanced constraint programming-based approach & No & \cite{ShiYXQ22} & 2022 & International Journal of Production Research & 18 & 2 & 45 & No & \ref{c:ShiYXQ22}\\
\rowlabel{a:SubulanC22}SubulanC22 \href{https://doi.org/10.1007/s00500-021-06399-5}{SubulanC22} & \hyperref[auth:a456]{K. Subulan}, \hyperref[auth:a457]{G. {\c{C}}akir} & Constraint programming-based transformation approach for a mixed fuzzy-stochastic resource investment project scheduling problem & \href{../works/SubulanC22.pdf}{Yes} & \cite{SubulanC22} & 2022 & Soft Comput. & 38 & 5 & 86 & \ref{b:SubulanC22} & \ref{c:SubulanC22}\\
\rowlabel{a:YunusogluY22}YunusogluY22 \href{https://doi.org/10.1080/00207543.2021.1885068}{YunusogluY22} & \hyperref[auth:a455]{P. Yunusoglu}, \hyperref[auth:a426]{Seyda Topaloglu Yildiz} & Constraint programming approach for multi-resource-constrained unrelated parallel machine scheduling problem with sequence-dependent setup times & \href{../works/YunusogluY22.pdf}{Yes} & \cite{YunusogluY22} & 2022 & International Journal of Production Research & 18 & 20 & 58 & \ref{b:YunusogluY22} & \ref{c:YunusogluY22}\\
\rowlabel{a:YuraszeckMPV22}YuraszeckMPV22 \href{http://dx.doi.org/10.3390/math10030329}{YuraszeckMPV22} & \hyperref[auth:a410]{F. Yuraszeck}, \hyperref[auth:a750]{G. Mejía}, \hyperref[auth:a751]{J. Pereira}, \hyperref[auth:a752]{M. Vilà} & A Novel Constraint Programming Decomposition Approach for the Total Flow Time Fixed Group Shop Scheduling Problem & \href{../works/YuraszeckMPV22.pdf}{Yes} & \cite{YuraszeckMPV22} & 2022 & Mathematics & 26 & 6 & 29 & \ref{b:YuraszeckMPV22} & \ref{c:YuraszeckMPV22}\\
\rowlabel{a:abs-2211-14492}abs-2211-14492 \href{https://doi.org/10.48550/arXiv.2211.14492}{abs-2211-14492} & \hyperref[auth:a402]{Y. Sun}, \hyperref[auth:a400]{S. Nguyen}, \hyperref[auth:a401]{Dhananjay R. Thiruvady}, \hyperref[auth:a473]{X. Li}, \hyperref[auth:a474]{Andreas T. Ernst}, \hyperref[auth:a475]{U. Aickelin} & Enhancing Constraint Programming via Supervised Learning for Job Shop Scheduling & \href{../works/abs-2211-14492.pdf}{Yes} & \cite{abs-2211-14492} & 2022 & CoRR & 17 & 0 & 0 & \ref{b:abs-2211-14492} & \ref{c:abs-2211-14492}\\
\rowlabel{a:AbohashimaEG21}AbohashimaEG21 \href{https://doi.org/10.1109/ACCESS.2021.3112600}{AbohashimaEG21} & \hyperref[auth:a477]{H. Abohashima}, \hyperref[auth:a478]{Amr B. Eltawil}, \hyperref[auth:a479]{Mohamed S. Gheith} & A Mathematical Programming Model and a Firefly-Based Heuristic for Real-Time Traffic Signal Scheduling With Physical Constraints & \href{../works/AbohashimaEG21.pdf}{Yes} & \cite{AbohashimaEG21} & 2021 & {IEEE} Access & 14 & 1 & 25 & \ref{b:AbohashimaEG21} & \ref{c:AbohashimaEG21}\\
\rowlabel{a:AbreuAPNM21}AbreuAPNM21 \href{http://dx.doi.org/10.1080/0305215x.2021.1957101}{AbreuAPNM21} & \hyperref[auth:a423]{Levi Ribeiro de Abreu}, \hyperref[auth:a755]{Kennedy Anderson Guimarães Araújo}, \hyperref[auth:a756]{Bruno de Athayde Prata}, \hyperref[auth:a424]{Marcelo Seido Nagano}, \hyperref[auth:a757]{João Vitor Moccellin} & A new variable neighbourhood search with a constraint programming search strategy for the open shop scheduling problem with operation repetitions & \href{../works/AbreuAPNM21.pdf}{Yes} & \cite{AbreuAPNM21} & 2021 & Engineering Optimization & 21 & 5 & 50 & \ref{b:AbreuAPNM21} & \ref{c:AbreuAPNM21}\\
\rowlabel{a:Bedhief21}Bedhief21 \href{https://api.semanticscholar.org/CorpusID:240611192}{Bedhief21} & \hyperref[auth:a754]{Asma Ouled Bedhief} & Comparing Mixed-Integer Programming and Constraint Programming Models for the Hybrid Flow Shop Scheduling Problem with Dedicated Machines & \href{../works/Bedhief21.pdf}{Yes} & \cite{Bedhief21} & 2021 & Journal Europ{\'e}en des Syst{\`e}mes Automatis{\'e}s & 7 & 0 & 0 & \ref{b:Bedhief21} & \ref{c:Bedhief21}\\
\rowlabel{a:CarlierSJP21}CarlierSJP21 \href{http://dx.doi.org/10.1080/00207543.2021.1923853}{CarlierSJP21} & \hyperref[auth:a854]{J. Carlier}, \hyperref[auth:a939]{A. Sahli}, \hyperref[auth:a940]{A. Jouglet}, \hyperref[auth:a855]{E. Pinson} & A faster checker of the energetic reasoning for the cumulative scheduling problem & No & \cite{CarlierSJP21} & 2021 & International Journal of Production Research & null & 3 & 26 & No & \ref{c:CarlierSJP21}\\
\rowlabel{a:Edis21}Edis21 \href{http://dx.doi.org/10.1016/j.cor.2020.105111}{Edis21} & \hyperref[auth:a351]{Emrah B. Edis} & Constraint programming approaches to disassembly line balancing problem with sequencing decisions & No & \cite{Edis21} & 2021 & Computers \  Operations Research & 1 & 13 & 48 & No & \ref{c:Edis21}\\
\rowlabel{a:FanXG21}FanXG21 \href{https://doi.org/10.1016/j.cor.2021.105401}{FanXG21} & \hyperref[auth:a481]{H. Fan}, \hyperref[auth:a482]{H. Xiong}, \hyperref[auth:a483]{M. Goh} & Genetic programming-based hyper-heuristic approach for solving dynamic job shop scheduling problem with extended technical precedence constraints & \href{../works/FanXG21.pdf}{Yes} & \cite{FanXG21} & 2021 & Computers \  Operations Research & 15 & 18 & 57 & \ref{b:FanXG21} & \ref{c:FanXG21}\\
\rowlabel{a:HamP21}HamP21 \href{http://dx.doi.org/10.1109/lra.2021.3056069}{HamP21} & \hyperref[auth:a758]{A. Ham}, \hyperref[auth:a759]{M. Park} & Human–Robot Task Allocation and Scheduling: Boeing 777 Case Study & No & \cite{HamP21} & 2021 & IEEE Robotics and Automation Letters & null & 13 & 26 & No & \ref{c:HamP21}\\
\rowlabel{a:HamPK21}HamPK21 \href{https://api.semanticscholar.org/CorpusID:237898414}{HamPK21} & \hyperref[auth:a758]{A. Ham}, \hyperref[auth:a759]{M. Park}, \hyperref[auth:a760]{Kyung Min Kim} & Energy-Aware Flexible Job Shop Scheduling Using Mixed Integer Programming and Constraint Programming & \href{../works/HamPK21.pdf}{Yes} & \cite{HamPK21} & 2021 & Mathematical Problems in Engineering & 12 & 6 & 46 & \ref{b:HamPK21} & \ref{c:HamPK21}\\
\rowlabel{a:HubnerGSV21}HubnerGSV21 \href{https://doi.org/10.1007/s10951-021-00682-x}{HubnerGSV21} & \hyperref[auth:a487]{F. H{\"{u}}bner}, \hyperref[auth:a488]{P. Gerhards}, \hyperref[auth:a489]{C. St{\"{u}}rck}, \hyperref[auth:a490]{R. Volk} & Solving the nuclear dismantling project scheduling problem by combining mixed-integer and constraint programming techniques and metaheuristics & \href{../works/HubnerGSV21.pdf}{Yes} & \cite{HubnerGSV21} & 2021 & Journal of Scheduling & 22 & 0 & 37 & \ref{b:HubnerGSV21} & \ref{c:HubnerGSV21}\\
\rowlabel{a:KoehlerBFFHPSSS21}KoehlerBFFHPSSS21 \href{https://doi.org/10.1007/s10601-021-09321-w}{KoehlerBFFHPSSS21} & \hyperref[auth:a104]{J. Koehler}, \hyperref[auth:a105]{J. B{\"{u}}rgler}, \hyperref[auth:a106]{U. Fontana}, \hyperref[auth:a107]{E. Fux}, \hyperref[auth:a108]{Florian A. Herzog}, \hyperref[auth:a109]{M. Pouly}, \hyperref[auth:a110]{S. Saller}, \hyperref[auth:a111]{A. Salyaeva}, \hyperref[auth:a112]{P. Scheiblechner}, \hyperref[auth:a113]{K. Waelti} & Cable tree wiring - benchmarking solvers on a real-world scheduling problem with a variety of precedence constraints & \href{../works/KoehlerBFFHPSSS21.pdf}{Yes} & \cite{KoehlerBFFHPSSS21} & 2021 & Constraints An Int. J. & 51 & 2 & 52 & \ref{b:KoehlerBFFHPSSS21} & \ref{c:KoehlerBFFHPSSS21}\\
\rowlabel{a:MengLZB21}MengLZB21 \href{http://dx.doi.org/10.1049/cim2.12005}{MengLZB21} & \hyperref[auth:a505]{L. Meng}, \hyperref[auth:a1178]{C. Lu}, \hyperref[auth:a508]{B. Zhang}, \hyperref[auth:a507]{Y. Ren}, \hyperref[auth:a509]{C. Lv}, \hyperref[auth:a1179]{H. Sang}, \hyperref[auth:a1180]{J. Li}, \hyperref[auth:a506]{C. Zhang} & Constraint programing for solving four complex flexible shop scheduling problems & No & \cite{MengLZB21} & 2021 & IET Collaborative Intelligent Manufacturing & null & 5 & 39 & No & \ref{c:MengLZB21}\\
\rowlabel{a:NaderiRBAU21}NaderiRBAU21 \href{http://dx.doi.org/10.1111/poms.13397}{NaderiRBAU21} & \hyperref[auth:a734]{B. Naderi}, \hyperref[auth:a736]{V. Roshanaei}, \hyperref[auth:a845]{Mehmet A. Begen}, \hyperref[auth:a904]{Dionne M. Aleman}, \hyperref[auth:a905]{David R. Urbach} & Increased Surgical Capacity without Additional Resources: Generalized Operating Room Planning and Scheduling & No & \cite{NaderiRBAU21} & 2021 & Production and Operations Management & null & 22 & 61 & No & \ref{c:NaderiRBAU21}\\
\rowlabel{a:PandeyS21a}PandeyS21a \href{https://doi.org/10.1007/s11227-020-03516-3}{PandeyS21a} & \hyperref[auth:a496]{V. Pandey}, \hyperref[auth:a497]{P. Saini} & Constraint programming versus heuristic approach to MapReduce scheduling problem in Hadoop {YARN} for energy minimization & \href{../works/PandeyS21a.pdf}{Yes} & \cite{PandeyS21a} & 2021 & J. Supercomput. & 29 & 3 & 32 & \ref{b:PandeyS21a} & \ref{c:PandeyS21a}\\
\rowlabel{a:QinWSLS21}QinWSLS21 \href{https://doi.org/10.1109/TASE.2019.2947398}{QinWSLS21} & \hyperref[auth:a491]{M. Qin}, \hyperref[auth:a492]{R. Wang}, \hyperref[auth:a493]{Z. Shi}, \hyperref[auth:a494]{L. Liu}, \hyperref[auth:a495]{L. Shi} & A Genetic Programming-Based Scheduling Approach for Hybrid Flow Shop With a Batch Processor and Waiting Time Constraint & \href{../works/QinWSLS21.pdf}{Yes} & \cite{QinWSLS21} & 2021 & {IEEE} Trans Autom. Sci. Eng. & 12 & 12 & 30 & \ref{b:QinWSLS21} & \ref{c:QinWSLS21}\\
\rowlabel{a:RabbaniMM21}RabbaniMM21 \href{http://dx.doi.org/10.1080/17509653.2021.1905096}{RabbaniMM21} & \hyperref[auth:a1274]{M. Rabbani}, \hyperref[auth:a520]{M. Mokhtarzadeh}, \hyperref[auth:a1275]{N. Manavizadeh} & A constraint programming approach and a hybrid of genetic and K-means algorithms to solve the p-hub location-allocation problems & No & \cite{RabbaniMM21} & 2021 & International Journal of Management Science and Engineering Management & null & 4 & 44 & No & \ref{c:RabbaniMM21}\\
\rowlabel{a:RoshanaeiN21}RoshanaeiN21 \href{http://dx.doi.org/10.1016/j.ejor.2020.12.004}{RoshanaeiN21} & \hyperref[auth:a736]{V. Roshanaei}, \hyperref[auth:a734]{B. Naderi} & Solving integrated operating room planning and scheduling: Logic-based Benders decomposition versus Branch-Price-and-Cut & No & \cite{RoshanaeiN21} & 2021 & European Journal of Operational Research & null & 15 & 44 & No & \ref{c:RoshanaeiN21}\\
\rowlabel{a:VlkHT21}VlkHT21 \href{https://doi.org/10.1016/j.cie.2021.107317}{VlkHT21} & \hyperref[auth:a313]{M. Vlk}, \hyperref[auth:a116]{Z. Hanz{\'{a}}lek}, \hyperref[auth:a480]{S. Tang} & Constraint programming approaches to joint routing and scheduling in time-sensitive networks & \href{../works/VlkHT21.pdf}{Yes} & \cite{VlkHT21} & 2021 & Computers \  Industrial Engineering & 14 & 7 & 22 & \ref{b:VlkHT21} & \ref{c:VlkHT21}\\
\rowlabel{a:ZhangYW21}ZhangYW21 \href{https://doi.org/10.1016/j.cor.2021.105282}{ZhangYW21} & \hyperref[auth:a484]{L. Zhang}, \hyperref[auth:a485]{C. Yu}, \hyperref[auth:a486]{T. N. Wong} & A graph-based constraint programming approach for the integrated process planning and scheduling problem & \href{../works/ZhangYW21.pdf}{Yes} & \cite{ZhangYW21} & 2021 & Computers \  Operations Research & 10 & 6 & 35 & \ref{b:ZhangYW21} & \ref{c:ZhangYW21}\\
\rowlabel{a:abs-2102-08778}abs-2102-08778 \href{https://arxiv.org/abs/2102.08778}{abs-2102-08778} & \hyperref[auth:a93]{Giacomo Da Col}, \hyperref[auth:a616]{E. Teppan} & Large-Scale Benchmarks for the Job Shop Scheduling Problem & \href{../works/abs-2102-08778.pdf}{Yes} & \cite{abs-2102-08778} & 2021 & CoRR & 10 & 0 & 0 & \ref{b:abs-2102-08778} & \ref{c:abs-2102-08778}\\
\rowlabel{a:AlizdehS20}AlizdehS20 \href{https://doi.org/10.1504/IJAIP.2020.106687}{AlizdehS20} & \hyperref[auth:a518]{S. Alizdeh}, \hyperref[auth:a519]{S. Saeidi} & Fuzzy project scheduling with critical path including risk and resource constraints using linear programming & No & \cite{AlizdehS20} & 2020 & Int. J. Adv. Intell. Paradigms & 14 & 1 & 0 & No & \ref{c:AlizdehS20}\\
\rowlabel{a:AntunesABD20}AntunesABD20 \href{https://doi.org/10.1142/S0218213020600076}{AntunesABD20} & \hyperref[auth:a886]{M. Antunes}, \hyperref[auth:a887]{V. Armant}, \hyperref[auth:a222]{Kenneth N. Brown}, \hyperref[auth:a888]{Daniel A. Desmond}, \hyperref[auth:a889]{G. Escamocher}, \hyperref[auth:a890]{A. George}, \hyperref[auth:a182]{D. Grimes}, \hyperref[auth:a891]{M. O'Keeffe}, \hyperref[auth:a892]{Y. Lin}, \hyperref[auth:a16]{B. O'Sullivan}, \hyperref[auth:a136]{C. {\"{O}}zt{\"{u}}rk}, \hyperref[auth:a893]{L. Quesada}, \hyperref[auth:a130]{M. Siala}, \hyperref[auth:a17]{H. Simonis}, \hyperref[auth:a834]{N. Wilson} & Assigning and Scheduling Service Visits in a Mixed Urban/Rural Setting & \href{../works/AntunesABD20.pdf}{Yes} & \cite{AntunesABD20} & 2020 & Int. J. Artif. Intell. Tools & 31 & 0 & 16 & \ref{b:AntunesABD20} & \ref{c:AntunesABD20}\\
\rowlabel{a:AstrandJZ20}AstrandJZ20 \href{https://doi.org/10.1016/j.cor.2020.105036}{AstrandJZ20} & \hyperref[auth:a74]{M. {\AA}strand}, \hyperref[auth:a75]{M. Johansson}, \hyperref[auth:a204]{A. Zanarini} & Underground mine scheduling of mobile machines using Constraint Programming and Large Neighborhood Search & \href{../works/AstrandJZ20.pdf}{Yes} & \cite{AstrandJZ20} & 2020 & Computers \  Operations Research & 13 & 16 & 24 & \ref{b:AstrandJZ20} & \ref{c:AstrandJZ20}\\
\rowlabel{a:BadicaBI20}BadicaBI20 \href{https://doi.org/10.3233/AIC-200650}{BadicaBI20} & \hyperref[auth:a502]{A. Badica}, \hyperref[auth:a503]{C. Badica}, \hyperref[auth:a504]{M. Ivanovic} & Block structured scheduling using constraint logic programming & \href{../works/BadicaBI20.pdf}{Yes} & \cite{BadicaBI20} & 2020 & {AI} Commun. & 17 & 2 & 28 & \ref{b:BadicaBI20} & \ref{c:BadicaBI20}\\
\rowlabel{a:BalochG20}BalochG20 \href{http://dx.doi.org/10.1287/trsc.2019.0928}{BalochG20} & \hyperref[auth:a1263]{G. Baloch}, \hyperref[auth:a1264]{F. Gzara} & Strategic Network Design for Parcel Delivery with Drones Under Competition & No & \cite{BalochG20} & 2020 & Transportation Science & null & 25 & 46 & No & \ref{c:BalochG20}\\
\rowlabel{a:BenediktMH20}BenediktMH20 \href{https://doi.org/10.1007/s10601-020-09317-y}{BenediktMH20} & \hyperref[auth:a114]{O. Benedikt}, \hyperref[auth:a115]{I. M{\'{o}}dos}, \hyperref[auth:a116]{Z. Hanz{\'{a}}lek} & Power of pre-processing: production scheduling with variable energy pricing and power-saving states & \href{../works/BenediktMH20.pdf}{Yes} & \cite{BenediktMH20} & 2020 & Constraints An Int. J. & 19 & 1 & 18 & \ref{b:BenediktMH20} & \ref{c:BenediktMH20}\\
\rowlabel{a:CarlierPSJ20}CarlierPSJ20 \href{http://dx.doi.org/10.1016/j.ejor.2020.03.079}{CarlierPSJ20} & \hyperref[auth:a1265]{J. Carlier}, \hyperref[auth:a1266]{E. Pinson}, \hyperref[auth:a1267]{A. Sahli}, \hyperref[auth:a1268]{A. Jouglet} & An O(n2) algorithm for time-bound adjustments for the cumulative scheduling problem & No & \cite{CarlierPSJ20} & 2020 & European Journal of Operational Research & null & 6 & 10 & No & \ref{c:CarlierPSJ20}\\
\rowlabel{a:CauwelaertDS20}CauwelaertDS20 \href{http://dx.doi.org/10.1007/s10951-019-00632-8}{CauwelaertDS20} & \hyperref[auth:a844]{Sasha Van Cauwelaert}, \hyperref[auth:a207]{C. Dejemeppe}, \hyperref[auth:a148]{P. Schaus} & An Efficient Filtering Algorithm for the Unary Resource Constraint with Transition Times and Optional Activities & \href{../works/CauwelaertDS20.pdf}{Yes} & \cite{CauwelaertDS20} & 2020 & Journal of Scheduling & 19 & 2 & 21 & \ref{b:CauwelaertDS20} & \ref{c:CauwelaertDS20}\\
\rowlabel{a:FachiniA20}FachiniA20 \href{http://dx.doi.org/10.1016/j.cie.2020.106641}{FachiniA20} & \hyperref[auth:a1039]{Ramon Faganello Fachini}, \hyperref[auth:a1040]{Vinícius Amaral Armentano} & Logic-based Benders decomposition for the heterogeneous fixed fleet vehicle routing problem with time windows & No & \cite{FachiniA20} & 2020 & Computers \  Industrial Engineering & 1 & 25 & 55 & No & \ref{c:FachiniA20}\\
\rowlabel{a:FallahiAC20}FallahiAC20 \href{https://api.semanticscholar.org/CorpusID:213449737}{FallahiAC20} & \hyperref[auth:a761]{Abdellah El Fallahi}, \hyperref[auth:a762]{El Yaakoubi Anass}, \hyperref[auth:a763]{M. Cherkaoui} & Tabu search and constraint programming-based approach for a real scheduling and routing problem & \href{../works/FallahiAC20.pdf}{Yes} & \cite{FallahiAC20} & 2020 & International Journal of Applied Management Science & 18 & 0 & 0 & \ref{b:FallahiAC20} & \ref{c:FallahiAC20}\\
\rowlabel{a:GuoHLW20}GuoHLW20 \href{http://dx.doi.org/10.1080/0305215x.2019.1699919}{GuoHLW20} & \hyperref[auth:a943]{P. Guo}, \hyperref[auth:a944]{X. He}, \hyperref[auth:a945]{Y. Luan}, \hyperref[auth:a946]{Y. Wang} & Logic-based Benders decomposition for gantry crane scheduling with transferring position constraints in a rail–road container terminal & No & \cite{GuoHLW20} & 2020 & Engineering Optimization & null & 8 & 31 & No & \ref{c:GuoHLW20}\\
\rowlabel{a:Ham20}Ham20 \href{http://dx.doi.org/10.1080/00207543.2019.1709671}{Ham20} & \hyperref[auth:a758]{A. Ham} & Transfer-robot task scheduling in job shop & No & \cite{Ham20} & 2020 & International Journal of Production Research & null & 15 & 27 & No & \ref{c:Ham20}\\
\rowlabel{a:Ham20a}Ham20a \href{http://dx.doi.org/10.1109/tase.2019.2952523}{Ham20a} & \hyperref[auth:a758]{A. Ham} & Drone-Based Material Transfer System in a Robotic Mobile Fulfillment Center & No & \cite{Ham20a} & 2020 & IEEE Transactions on Automation Science and Engineering & null & 15 & 27 & No & \ref{c:Ham20a}\\
\rowlabel{a:HauderBRPA20}HauderBRPA20 \href{http://dx.doi.org/10.1016/j.cie.2020.106857}{HauderBRPA20} & \hyperref[auth:a558]{Viktoria A. Hauder}, \hyperref[auth:a559]{A. Beham}, \hyperref[auth:a560]{S. Raggl}, \hyperref[auth:a561]{Sophie N. Parragh}, \hyperref[auth:a562]{M. Affenzeller} & Resource-constrained multi-project scheduling with activity and time flexibility & \href{../works/HauderBRPA20.pdf}{Yes} & \cite{HauderBRPA20} & 2020 & Computers \  Industrial Engineering & 14 & 14 & 46 & \ref{b:HauderBRPA20} & \ref{c:HauderBRPA20}\\
\rowlabel{a:LunardiBLRV20}LunardiBLRV20 \href{https://doi.org/10.1016/j.cor.2020.105020}{LunardiBLRV20} & \hyperref[auth:a510]{Willian T. Lunardi}, \hyperref[auth:a511]{Ernesto G. Birgin}, \hyperref[auth:a118]{P. Laborie}, \hyperref[auth:a512]{D{\'{e}}bora P. Ronconi}, \hyperref[auth:a513]{H. Voos} & Mixed Integer linear programming and constraint programming models for the online printing shop scheduling problem & \href{../works/LunardiBLRV20.pdf}{Yes} & \cite{LunardiBLRV20} & 2020 & Computers \  Operations Research & 20 & 30 & 18 & \ref{b:LunardiBLRV20} & \ref{c:LunardiBLRV20}\\
\rowlabel{a:MejiaY20}MejiaY20 \href{https://doi.org/10.1016/j.ejor.2020.02.010}{MejiaY20} & \hyperref[auth:a429]{G. Mej{\'{\i}}a}, \hyperref[auth:a410]{F. Yuraszeck} & A self-tuning variable neighborhood search algorithm and an effective decoding scheme for open shop scheduling problems with travel/setup times & \href{../works/MejiaY20.pdf}{Yes} & \cite{MejiaY20} & 2020 & European Journal of Operational Research & 13 & 24 & 45 & \ref{b:MejiaY20} & \ref{c:MejiaY20}\\
\rowlabel{a:MengZRZL20}MengZRZL20 \href{https://doi.org/10.1016/j.cie.2020.106347}{MengZRZL20} & \hyperref[auth:a505]{L. Meng}, \hyperref[auth:a506]{C. Zhang}, \hyperref[auth:a507]{Y. Ren}, \hyperref[auth:a508]{B. Zhang}, \hyperref[auth:a509]{C. Lv} & Mixed-integer linear programming and constraint programming formulations for solving distributed flexible job shop scheduling problem & \href{../works/MengZRZL20.pdf}{Yes} & \cite{MengZRZL20} & 2020 & Computers \  Industrial Engineering & 13 & 100 & 62 & \ref{b:MengZRZL20} & \ref{c:MengZRZL20}\\
\rowlabel{a:MokhtarzadehTNF20}MokhtarzadehTNF20 \href{https://doi.org/10.1080/0951192X.2020.1736713}{MokhtarzadehTNF20} & \hyperref[auth:a520]{M. Mokhtarzadeh}, \hyperref[auth:a435]{R. Tavakkoli{-}Moghaddam}, \hyperref[auth:a437]{Behdin Vahedi Nouri}, \hyperref[auth:a521]{A. Farsi} & Scheduling of human-robot collaboration in assembly of printed circuit boards: a constraint programming approach & \href{../works/MokhtarzadehTNF20.pdf}{Yes} & \cite{MokhtarzadehTNF20} & 2020 & Int. J. Comput. Integr. Manuf. & 14 & 25 & 32 & \ref{b:MokhtarzadehTNF20} & \ref{c:MokhtarzadehTNF20}\\
\rowlabel{a:Polo-MejiaALB20}Polo-MejiaALB20 \href{https://doi.org/10.1080/00207543.2019.1693654}{Polo-MejiaALB20} & \hyperref[auth:a522]{O. Polo{-}Mej{\'{\i}}a}, \hyperref[auth:a6]{C. Artigues}, \hyperref[auth:a3]{P. Lopez}, \hyperref[auth:a523]{V. Basini} & Mixed-integer/linear and constraint programming approaches for activity scheduling in a nuclear research facility & \href{../works/Polo-MejiaALB20.pdf}{Yes} & \cite{Polo-MejiaALB20} & 2020 & International Journal of Production Research & 18 & 8 & 23 & \ref{b:Polo-MejiaALB20} & \ref{c:Polo-MejiaALB20}\\
\rowlabel{a:QinDCS20}QinDCS20 \href{https://doi.org/10.1016/j.ejor.2020.02.021}{QinDCS20} & \hyperref[auth:a514]{T. Qin}, \hyperref[auth:a515]{Y. Du}, \hyperref[auth:a516]{Jiang Hang Chen}, \hyperref[auth:a517]{M. Sha} & Combining mixed integer programming and constraint programming to solve the integrated scheduling problem of container handling operations of a single vessel & \href{../works/QinDCS20.pdf}{Yes} & \cite{QinDCS20} & 2020 & European Journal of Operational Research & 18 & 27 & 30 & \ref{b:QinDCS20} & \ref{c:QinDCS20}\\
\rowlabel{a:RoshanaeiBAUB20}RoshanaeiBAUB20 \href{http://dx.doi.org/10.1016/j.ijpe.2019.07.006}{RoshanaeiBAUB20} & \hyperref[auth:a736]{V. Roshanaei}, \hyperref[auth:a994]{Kyle E.C. Booth}, \hyperref[auth:a904]{Dionne M. Aleman}, \hyperref[auth:a905]{David R. Urbach}, \hyperref[auth:a89]{J. Christopher Beck} & Branch-and-check methods for multi-level operating room planning and scheduling & \href{../works/RoshanaeiBAUB20.pdf}{Yes} & \cite{RoshanaeiBAUB20} & 2020 & International Journal of Production Economics & 19 & 24 & 43 & \ref{b:RoshanaeiBAUB20} & \ref{c:RoshanaeiBAUB20}\\
\rowlabel{a:SacramentoSP20}SacramentoSP20 \href{https://doi.org/10.1007/s43069-020-00036-x}{SacramentoSP20} & \hyperref[auth:a524]{D. Sacramento}, \hyperref[auth:a85]{C. Solnon}, \hyperref[auth:a525]{D. Pisinger} & Constraint Programming and Local Search Heuristic: a Matheuristic Approach for Routing and Scheduling Feeder Vessels in Multi-terminal Ports & \href{../works/SacramentoSP20.pdf}{Yes} & \cite{SacramentoSP20} & 2020 & Oper. Res. Forum & 33 & 2 & 38 & \ref{b:SacramentoSP20} & \ref{c:SacramentoSP20}\\
\rowlabel{a:WallaceY20}WallaceY20 \href{https://doi.org/10.1007/s10601-020-09316-z}{WallaceY20} & \hyperref[auth:a117]{Mark G. Wallace}, \hyperref[auth:a19]{N. Yorke{-}Smith} & A new constraint programming model and solving for the cyclic hoist scheduling problem & \href{../works/WallaceY20.pdf}{Yes} & \cite{WallaceY20} & 2020 & Constraints An Int. J. & 19 & 5 & 18 & \ref{b:WallaceY20} & \ref{c:WallaceY20}\\
\rowlabel{a:ZarandiASC20}ZarandiASC20 \href{https://doi.org/10.1007/s10462-018-9667-6}{ZarandiASC20} & \hyperref[auth:a837]{Mohammad Hossein Fazel Zarandi}, \hyperref[auth:a838]{Ali Akbar Sadat Asl}, \hyperref[auth:a839]{S. Sotudian}, \hyperref[auth:a840]{O. Castillo} & A state of the art review of intelligent scheduling & \href{../works/ZarandiASC20.pdf}{Yes} & \cite{ZarandiASC20} & 2020 & Artif. Intell. Rev. & 93 & 55 & 445 & \ref{b:ZarandiASC20} & \ref{c:ZarandiASC20}\\
\rowlabel{a:ZouZ20}ZouZ20 \href{https://api.semanticscholar.org/CorpusID:208840808}{ZouZ20} & \hyperref[auth:a764]{X. Zou}, \hyperref[auth:a765]{L. Zhang} & A constraint programming approach for scheduling repetitive projects with atypical activities considering soft logic & \href{../works/ZouZ20.pdf}{Yes} & \cite{ZouZ20} & 2020 & Automation in Construction & 10 & 18 & 48 & \ref{b:ZouZ20} & \ref{c:ZouZ20}\\
\rowlabel{a:ArkhipovBL19}ArkhipovBL19 \href{http://dx.doi.org/10.1016/j.ejor.2018.11.005}{ArkhipovBL19} & \hyperref[auth:a934]{D. Arkhipov}, \hyperref[auth:a935]{O. Battaïa}, \hyperref[auth:a936]{A. Lazarev} & An efficient pseudo-polynomial algorithm for finding a lower bound on the makespan for the Resource Constrained Project Scheduling Problem & \href{../works/ArkhipovBL19.pdf}{Yes} & \cite{ArkhipovBL19} & 2019 & European Journal of Operational Research & 10 & 12 & 24 & \ref{b:ArkhipovBL19} & \ref{c:ArkhipovBL19}\\
\rowlabel{a:ColT2019a}ColT2019a \href{http://dx.doi.org/10.4204/eptcs.306.30}{ColT2019a} & \hyperref[auth:a93]{Giacomo Da Col}, \hyperref[auth:a616]{E. Teppan} & Google vs IBM: A Constraint Solving Challenge on the Job-Shop Scheduling Problem & No & \cite{ColT2019a} & 2019 & Electronic Proceedings in Theoretical Computer Science & null & 10 & 10 & No & \ref{c:ColT2019a}\\
\rowlabel{a:EdwardsBSE19}EdwardsBSE19 \href{http://dx.doi.org/10.1080/01605682.2019.1595192}{EdwardsBSE19} & \hyperref[auth:a901]{Steven J. Edwards}, \hyperref[auth:a902]{D. Baatar}, \hyperref[auth:a903]{K. Smith-Miles}, \hyperref[auth:a474]{Andreas T. Ernst} & Symmetry breaking of identical projects in the high-multiplicity RCPSP/max & No & \cite{EdwardsBSE19} & 2019 & Journal of the Operational Research Society & null & 3 & 40 & No & \ref{c:EdwardsBSE19}\\
\rowlabel{a:EscobetPQPRA19}EscobetPQPRA19 \href{https://doi.org/10.1016/j.compchemeng.2018.08.040}{EscobetPQPRA19} & \hyperref[auth:a530]{T. Escobet}, \hyperref[auth:a531]{V. Puig}, \hyperref[auth:a532]{J. Quevedo}, \hyperref[auth:a533]{P. Pal{\`{a}}{-}Sch{\"{o}}nw{\"{a}}lder}, \hyperref[auth:a534]{J. Romera}, \hyperref[auth:a535]{W. Adelman} & Optimal batch scheduling of a multiproduct dairy process using a combined optimization/constraint programming approach & \href{../works/EscobetPQPRA19.pdf}{Yes} & \cite{EscobetPQPRA19} & 2019 & Computers \  Chemical Engineering & 10 & 17 & 18 & \ref{b:EscobetPQPRA19} & \ref{c:EscobetPQPRA19}\\
\rowlabel{a:GurEA19}GurEA19 \href{https://api.semanticscholar.org/CorpusID:88492001}{GurEA19} & \hyperref[auth:a771]{Şeyda G{\"u}r}, \hyperref[auth:a420]{T. Eren}, \hyperref[auth:a772]{Hacı Mehmet Alakaş} & Surgical Operation Scheduling with Goal Programming and Constraint Programming: A Case Study & \href{../works/GurEA19.pdf}{Yes} & \cite{GurEA19} & 2019 & Mathematics & 24 & 19 & 30 & \ref{b:GurEA19} & \ref{c:GurEA19}\\
\rowlabel{a:HechingHK19}HechingHK19 \href{http://dx.doi.org/10.1287/trsc.2018.0830}{HechingHK19} & \hyperref[auth:a1036]{A. Heching}, \hyperref[auth:a1037]{J. N. Hooker}, \hyperref[auth:a1038]{R. Kimura} & A Logic-Based Benders Approach to Home Healthcare Delivery & No & \cite{HechingHK19} & 2019 & Transportation Science & null & 35 & 29 & No & \ref{c:HechingHK19}\\
\rowlabel{a:HoundjiSW19}HoundjiSW19 \href{https://doi.org/10.1007/s10601-018-9300-y}{HoundjiSW19} & \hyperref[auth:a228]{Vinas{\'{e}}tan Ratheil Houndji}, \hyperref[auth:a148]{P. Schaus}, \hyperref[auth:a229]{Laurence A. Wolsey} & The item dependent stockingcost constraint & \href{../works/HoundjiSW19.pdf}{Yes} & \cite{HoundjiSW19} & 2019 & Constraints An Int. J. & 27 & 0 & 17 & \ref{b:HoundjiSW19} & \ref{c:HoundjiSW19}\\
\rowlabel{a:NattafDYW19}NattafDYW19 \href{https://doi.org/10.1016/j.cor.2019.03.004}{NattafDYW19} & \hyperref[auth:a81]{M. Nattaf}, \hyperref[auth:a1008]{S. Dauz{\`{e}}re{-}P{\'{e}}r{\`{e}}s}, \hyperref[auth:a1009]{C. Yugma}, \hyperref[auth:a1010]{C. Wu} & Parallel machine scheduling with time constraints on machine qualifications & \href{../works/NattafDYW19.pdf}{Yes} & \cite{NattafDYW19} & 2019 & Computers \  Operations Research & 16 & 14 & 21 & \ref{b:NattafDYW19} & \ref{c:NattafDYW19}\\
\rowlabel{a:NattafHKAL19}NattafHKAL19 \href{https://doi.org/10.1016/j.dam.2018.11.008}{NattafHKAL19} & \hyperref[auth:a81]{M. Nattaf}, \hyperref[auth:a1011]{M. Horv{\'{a}}th}, \hyperref[auth:a156]{T. Kis}, \hyperref[auth:a6]{C. Artigues}, \hyperref[auth:a3]{P. Lopez} & Polyhedral results and valid inequalities for the continuous energy-constrained scheduling problem & \href{../works/NattafHKAL19.pdf}{Yes} & \cite{NattafHKAL19} & 2019 & Discret. Appl. Math. & 16 & 5 & 12 & \ref{b:NattafHKAL19} & \ref{c:NattafHKAL19}\\
\rowlabel{a:NishikawaSTT19}NishikawaSTT19 \href{http://www.ijnc.org/index.php/ijnc/article/view/201}{NishikawaSTT19} & \hyperref[auth:a536]{H. Nishikawa}, \hyperref[auth:a537]{K. Shimada}, \hyperref[auth:a538]{I. Taniguchi}, \hyperref[auth:a539]{H. Tomiyama} & A Constraint Programming Approach to Scheduling of Malleable Tasks & \href{../works/NishikawaSTT19.pdf}{Yes} & \cite{NishikawaSTT19} & 2019 & Int. J. Netw. Comput. & 16 & 3 & 20 & \ref{b:NishikawaSTT19} & \ref{c:NishikawaSTT19}\\
\rowlabel{a:Novas19}Novas19 \href{https://doi.org/10.1016/j.cie.2019.07.011}{Novas19} & \hyperref[auth:a529]{Juan M. Novas} & Production scheduling and lot streaming at flexible job-shops environments using constraint programming & \href{../works/Novas19.pdf}{Yes} & \cite{Novas19} & 2019 & Computers \  Industrial Engineering & 13 & 30 & 29 & \ref{b:Novas19} & \ref{c:Novas19}\\
\rowlabel{a:SunTB19}SunTB19 \href{http://dx.doi.org/10.1016/j.ejor.2018.08.009}{SunTB19} & \hyperref[auth:a1221]{D. Sun}, \hyperref[auth:a1222]{L. Tang}, \hyperref[auth:a1223]{R. Baldacci} & A Benders decomposition-based framework for solving quay crane scheduling problems & No & \cite{SunTB19} & 2019 & European Journal of Operational Research & null & 31 & 29 & No & \ref{c:SunTB19}\\
\rowlabel{a:TanZWGQ19}TanZWGQ19 \href{http://dx.doi.org/10.1109/tase.2019.2894093}{TanZWGQ19} & \hyperref[auth:a1207]{Y. Tan}, \hyperref[auth:a1208]{M. Zhou}, \hyperref[auth:a1209]{Y. Wang}, \hyperref[auth:a1210]{X. Guo}, \hyperref[auth:a1211]{L. Qi} & A Hybrid MIP–CP Approach to Multistage Scheduling Problem in Continuous Casting and Hot-Rolling Processes & No & \cite{TanZWGQ19} & 2019 & IEEE Transactions on Automation Science and Engineering & null & 40 & 40 & No & \ref{c:TanZWGQ19}\\
\rowlabel{a:UnsalO19}UnsalO19 \href{http://dx.doi.org/10.1016/j.tre.2019.03.018}{UnsalO19} & \hyperref[auth:a1243]{O. Unsal}, \hyperref[auth:a352]{C. Oguz} & An exact algorithm for integrated planning of operations in dry bulk terminals & No & \cite{UnsalO19} & 2019 & Transportation Research Part E: Logistics and Transportation Review & null & 44 & 27 & No & \ref{c:UnsalO19}\\
\rowlabel{a:WariZ19}WariZ19 \href{http://dx.doi.org/10.1080/00207543.2019.1571250}{WariZ19} & \hyperref[auth:a848]{E. Wari}, \hyperref[auth:a849]{W. Zhu} & A Constraint Programming model for food processing industry: a case for an ice cream processing facility & No & \cite{WariZ19} & 2019 & International Journal of Production Research & null & 11 & 42 & No & \ref{c:WariZ19}\\
\rowlabel{a:WikarekS19}WikarekS19 \href{https://doi.org/10.1142/S2196888819500027}{WikarekS19} & \hyperref[auth:a540]{J. Wikarek}, \hyperref[auth:a541]{P. Sitek} & A Constraint-Based Declarative Programming Framework for Scheduling and Resource Allocation Problems & \href{../works/WikarekS19.pdf}{Yes} & \cite{WikarekS19} & 2019 & Vietnam. J. Comput. Sci. & 22 & 0 & 11 & \ref{b:WikarekS19} & \ref{c:WikarekS19}\\
\rowlabel{a:YounespourAKE19}YounespourAKE19 \href{https://api.semanticscholar.org/CorpusID:208103305}{YounespourAKE19} & \hyperref[auth:a766]{M. Younespour}, \hyperref[auth:a767]{A. Atighehchian}, \hyperref[auth:a768]{K. Kianfar}, \hyperref[auth:a769]{Ehsan Tarkesh Esfahani} & Using mixed integer programming and constraint programming for operating rooms scheduling with modified block strategy & \href{../works/YounespourAKE19.pdf}{Yes} & \cite{YounespourAKE19} & 2019 & Operations research for health care & 11 & 7 & 15 & \ref{b:YounespourAKE19} & \ref{c:YounespourAKE19}\\
\rowlabel{a:abs-1901-07914}abs-1901-07914 \href{http://arxiv.org/abs/1901.07914}{abs-1901-07914} & \hyperref[auth:a545]{Jan Kristof Behrens}, \hyperref[auth:a546]{R. Lange}, \hyperref[auth:a547]{M. Mansouri} & A Constraint Programming Approach to Simultaneous Task Allocation and Motion Scheduling for Industrial Dual-Arm Manipulation Tasks & \href{../works/abs-1901-07914.pdf}{Yes} & \cite{abs-1901-07914} & 2019 & CoRR & 8 & 0 & 0 & \ref{b:abs-1901-07914} & \ref{c:abs-1901-07914}\\
\rowlabel{a:abs-1902-01193}abs-1902-01193 \href{http://arxiv.org/abs/1902.01193}{abs-1902-01193} & \hyperref[auth:a556]{O. M. Alade}, \hyperref[auth:a557]{A. O. Amusat} & Solving Nurse Scheduling Problem Using Constraint Programming Technique & \href{../works/abs-1902-01193.pdf}{Yes} & \cite{abs-1902-01193} & 2019 & CoRR & 9 & 0 & 0 & \ref{b:abs-1902-01193} & \ref{c:abs-1902-01193}\\
\rowlabel{a:abs-1902-09244}abs-1902-09244 \href{http://arxiv.org/abs/1902.09244}{abs-1902-09244} & \hyperref[auth:a558]{Viktoria A. Hauder}, \hyperref[auth:a559]{A. Beham}, \hyperref[auth:a560]{S. Raggl}, \hyperref[auth:a561]{Sophie N. Parragh}, \hyperref[auth:a562]{M. Affenzeller} & On constraint programming for a new flexible project scheduling problem with resource constraints & \href{../works/abs-1902-09244.pdf}{Yes} & \cite{abs-1902-09244} & 2019 & CoRR & 62 & 0 & 0 & \ref{b:abs-1902-09244} & \ref{c:abs-1902-09244}\\
\rowlabel{a:abs-1911-04766}abs-1911-04766 \href{http://arxiv.org/abs/1911.04766}{abs-1911-04766} & \hyperref[auth:a77]{T. Geibinger}, \hyperref[auth:a80]{F. Mischek}, \hyperref[auth:a45]{N. Musliu} & Investigating Constraint Programming and Hybrid Methods for Real World Industrial Test Laboratory Scheduling & \href{../works/abs-1911-04766.pdf}{Yes} & \cite{abs-1911-04766} & 2019 & CoRR & 16 & 0 & 0 & \ref{b:abs-1911-04766} & \ref{c:abs-1911-04766}\\
\rowlabel{a:BaptisteB18}BaptisteB18 \href{https://doi.org/10.1016/j.dam.2017.05.001}{BaptisteB18} & \hyperref[auth:a163]{P. Baptiste}, \hyperref[auth:a712]{N. Bonifas} & Redundant cumulative constraints to compute preemptive bounds & \href{../works/BaptisteB18.pdf}{Yes} & \cite{BaptisteB18} & 2018 & Discret. Appl. Math. & 10 & 3 & 13 & \ref{b:BaptisteB18} & \ref{c:BaptisteB18}\\
\rowlabel{a:BorghesiBLMB18}BorghesiBLMB18 \href{https://doi.org/10.1016/j.suscom.2018.05.007}{BorghesiBLMB18} & \hyperref[auth:a231]{A. Borghesi}, \hyperref[auth:a230]{A. Bartolini}, \hyperref[auth:a143]{M. Lombardi}, \hyperref[auth:a144]{M. Milano}, \hyperref[auth:a247]{L. Benini} & Scheduling-based power capping in high performance computing systems & \href{../works/BorghesiBLMB18.pdf}{Yes} & \cite{BorghesiBLMB18} & 2018 & Sustain. Comput. Informatics Syst. & 13 & 11 & 22 & \ref{b:BorghesiBLMB18} & \ref{c:BorghesiBLMB18}\\
\rowlabel{a:BukchinR18}BukchinR18 \href{http://dx.doi.org/10.1016/j.omega.2017.06.008}{BukchinR18} & \hyperref[auth:a1205]{Y. Bukchin}, \hyperref[auth:a1206]{T. Raviv} & Constraint programming for solving various assembly line balancing problems & No & \cite{BukchinR18} & 2018 & Omega & null & 66 & 29 & No & \ref{c:BukchinR18}\\
\rowlabel{a:CauwelaertLS18}CauwelaertLS18 \href{https://doi.org/10.1007/s10601-017-9277-y}{CauwelaertLS18} & \hyperref[auth:a206]{Sascha Van Cauwelaert}, \hyperref[auth:a143]{M. Lombardi}, \hyperref[auth:a148]{P. Schaus} & How efficient is a global constraint in practice? - {A} fair experimental framework & \href{../works/CauwelaertLS18.pdf}{Yes} & \cite{CauwelaertLS18} & 2018 & Constraints An Int. J. & 36 & 2 & 39 & \ref{b:CauwelaertLS18} & \ref{c:CauwelaertLS18}\\
\rowlabel{a:FahimiOQ18}FahimiOQ18 \href{https://doi.org/10.1007/s10601-018-9282-9}{FahimiOQ18} & \hyperref[auth:a122]{H. Fahimi}, \hyperref[auth:a52]{Y. Ouellet}, \hyperref[auth:a37]{C. Quimper} & Linear-time filtering algorithms for the disjunctive constraint and a quadratic filtering algorithm for the cumulative not-first not-last & \href{../works/FahimiOQ18.pdf}{Yes} & \cite{FahimiOQ18} & 2018 & Constraints An Int. J. & 22 & 2 & 20 & \ref{b:FahimiOQ18} & \ref{c:FahimiOQ18}\\
\rowlabel{a:GedikKEK18}GedikKEK18 \href{https://doi.org/10.1016/j.cie.2018.05.014}{GedikKEK18} & \hyperref[auth:a568]{R. Gedik}, \hyperref[auth:a569]{D. Kalathia}, \hyperref[auth:a570]{G. Egilmez}, \hyperref[auth:a571]{E. Kirac} & A constraint programming approach for solving unrelated parallel machine scheduling problem & \href{../works/GedikKEK18.pdf}{Yes} & \cite{GedikKEK18} & 2018 & Computers \  Industrial Engineering & 11 & 43 & 22 & \ref{b:GedikKEK18} & \ref{c:GedikKEK18}\\
\rowlabel{a:GokgurHO18}GokgurHO18 \href{https://doi.org/10.1080/00207543.2017.1421781}{GokgurHO18} & \hyperref[auth:a577]{B. G{\"{o}}kg{\"{u}}r}, \hyperref[auth:a138]{B. Hnich}, \hyperref[auth:a578]{S. {\"{O}}zpeynirci} & Parallel machine scheduling with tool loading: a constraint programming approach & \href{../works/GokgurHO18.pdf}{Yes} & \cite{GokgurHO18} & 2018 & International Journal of Production Research & 17 & 31 & 43 & \ref{b:GokgurHO18} & \ref{c:GokgurHO18}\\
\rowlabel{a:GoldwaserS18}GoldwaserS18 \href{https://doi.org/10.1613/jair.1.11268}{GoldwaserS18} & \hyperref[auth:a194]{A. Goldwaser}, \hyperref[auth:a125]{A. Schutt} & Optimal Torpedo Scheduling & \href{../works/GoldwaserS18.pdf}{Yes} & \cite{GoldwaserS18} & 2018 & J. Artif. Intell. Res. & 32 & 8 & 0 & \ref{b:GoldwaserS18} & \ref{c:GoldwaserS18}\\
\rowlabel{a:GombolayWS18}GombolayWS18 \href{http://dx.doi.org/10.1109/tro.2018.2795034}{GombolayWS18} & \hyperref[auth:a931]{Matthew C. Gombolay}, \hyperref[auth:a932]{Ronald J. Wilcox}, \hyperref[auth:a933]{Julie A. Shah} & Fast Scheduling of Robot Teams Performing Tasks With Temporospatial Constraints & \href{../works/GombolayWS18.pdf}{Yes} & \cite{GombolayWS18} & 2018 & IEEE Transactions on Robotics & 20 & 71 & 75 & \ref{b:GombolayWS18} & \ref{c:GombolayWS18}\\
\rowlabel{a:Ham18}Ham18 \href{http://dx.doi.org/10.1016/j.trc.2018.03.025}{Ham18} & \hyperref[auth:a778]{Andy M. Ham} & Integrated scheduling of m-truck,  m-drone,  and m-depot constrained by time-window,  drop-pickup,  and m-visit using constraint programming & \href{../works/Ham18.pdf}{Yes} & \cite{Ham18} & 2018 & Transportation Research Part C: Emerging Technologies & 14 & 164 & 14 & \ref{b:Ham18} & \ref{c:Ham18}\\
\rowlabel{a:Ham18a}Ham18a \href{http://dx.doi.org/10.1109/tsm.2017.2768899}{Ham18a} & \hyperref[auth:a758]{A. Ham} & Scheduling of Dual Resource Constrained Lithography Production: Using CP and MIP/CP & \href{../works/Ham18a.pdf}{Yes} & \cite{Ham18a} & 2018 & IEEE Transactions on Semiconductor Manufacturing & 10 & 20 & 21 & \ref{b:Ham18a} & \ref{c:Ham18a}\\
\rowlabel{a:KreterSSZ18}KreterSSZ18 \href{https://doi.org/10.1016/j.ejor.2017.10.014}{KreterSSZ18} & \hyperref[auth:a124]{S. Kreter}, \hyperref[auth:a125]{A. Schutt}, \hyperref[auth:a126]{Peter J. Stuckey}, \hyperref[auth:a800]{J. Zimmermann} & Mixed-integer linear programming and constraint programming formulations for solving resource availability cost problems & \href{../works/KreterSSZ18.pdf}{Yes} & \cite{KreterSSZ18} & 2018 & European Journal of Operational Research & 15 & 25 & 31 & \ref{b:KreterSSZ18} & \ref{c:KreterSSZ18}\\
\rowlabel{a:LaborieRSV18}LaborieRSV18 \href{https://doi.org/10.1007/s10601-018-9281-x}{LaborieRSV18} & \hyperref[auth:a118]{P. Laborie}, \hyperref[auth:a119]{J. Rogerie}, \hyperref[auth:a120]{P. Shaw}, \hyperref[auth:a121]{P. Vil{\'{\i}}m} & {IBM} {ILOG} {CP} optimizer for scheduling - 20+ years of scheduling with constraints at {IBM/ILOG} & \href{../works/LaborieRSV18.pdf}{Yes} & \cite{LaborieRSV18} & 2018 & Constraints An Int. J. & 41 & 148 & 35 & \ref{b:LaborieRSV18} & \ref{c:LaborieRSV18}\\
\rowlabel{a:PourDERB18}PourDERB18 \href{https://doi.org/10.1016/j.ejor.2017.08.033}{PourDERB18} & \hyperref[auth:a572]{Shahrzad M. Pour}, \hyperref[auth:a573]{John H. Drake}, \hyperref[auth:a574]{Lena Secher Ejlertsen}, \hyperref[auth:a575]{Kourosh Marjani Rasmussen}, \hyperref[auth:a576]{Edmund K. Burke} & A hybrid Constraint Programming/Mixed Integer Programming framework for the preventive signaling maintenance crew scheduling problem & \href{../works/PourDERB18.pdf}{Yes} & \cite{PourDERB18} & 2018 & European Journal of Operational Research & 12 & 41 & 13 & \ref{b:PourDERB18} & \ref{c:PourDERB18}\\
\rowlabel{a:ShinBBHO18}ShinBBHO18 \href{https://doi.org/10.1109/TSMC.2017.2681623}{ShinBBHO18} & \hyperref[auth:a581]{Seung Yeob Shin}, \hyperref[auth:a582]{Y. Brun}, \hyperref[auth:a583]{H. Balasubramanian}, \hyperref[auth:a584]{Philip L. Henneman}, \hyperref[auth:a585]{Leon J. Osterweil} & Discrete-Event Simulation and Integer Linear Programming for Constraint-Aware Resource Scheduling & \href{../works/ShinBBHO18.pdf}{Yes} & \cite{ShinBBHO18} & 2018 & {IEEE} Trans. Syst. Man Cybern. Syst. & 16 & 9 & 31 & \ref{b:ShinBBHO18} & \ref{c:ShinBBHO18}\\
\rowlabel{a:TangLWSK18}TangLWSK18 \href{https://doi.org/10.1111/mice.12277}{TangLWSK18} & \hyperref[auth:a563]{Y. Tang}, \hyperref[auth:a564]{R. Liu}, \hyperref[auth:a565]{F. Wang}, \hyperref[auth:a566]{Q. Sun}, \hyperref[auth:a567]{Amr A. Kandil} & Scheduling Optimization of Linear Schedule with Constraint Programming & \href{../works/TangLWSK18.pdf}{Yes} & \cite{TangLWSK18} & 2018 & Comput. Aided Civ. Infrastructure Eng. & 28 & 24 & 76 & \ref{b:TangLWSK18} & \ref{c:TangLWSK18}\\
\rowlabel{a:TranPZLDB18}TranPZLDB18 \href{https://doi.org/10.1007/s10951-017-0537-x}{TranPZLDB18} & \hyperref[auth:a807]{Tony T. Tran}, \hyperref[auth:a808]{M. Padmanabhan}, \hyperref[auth:a809]{Peter Yun Zhang}, \hyperref[auth:a810]{H. Li}, \hyperref[auth:a811]{Douglas G. Down}, \hyperref[auth:a89]{J. Christopher Beck} & Multi-stage resource-aware scheduling for data centers with heterogeneous servers & \href{../works/TranPZLDB18.pdf}{Yes} & \cite{TranPZLDB18} & 2018 & Journal of Scheduling & 17 & 8 & 26 & \ref{b:TranPZLDB18} & \ref{c:TranPZLDB18}\\
\rowlabel{a:ZhangW18}ZhangW18 \href{https://doi.org/10.1109/TEM.2017.2785774}{ZhangW18} & \hyperref[auth:a579]{S. Zhang}, \hyperref[auth:a580]{S. Wang} & Flexible Assembly Job-Shop Scheduling With Sequence-Dependent Setup Times and Part Sharing in a Dynamic Environment: Constraint Programming Model, Mixed-Integer Programming Model, and Dispatching Rules & \href{../works/ZhangW18.pdf}{Yes} & \cite{ZhangW18} & 2018 & {IEEE} Trans. Engineering Management & 18 & 49 & 28 & \ref{b:ZhangW18} & \ref{c:ZhangW18}\\
\rowlabel{a:EmeretlisTAV17}EmeretlisTAV17 \href{http://dx.doi.org/10.1145/3133219}{EmeretlisTAV17} & \hyperref[auth:a1253]{A. Emeretlis}, \hyperref[auth:a1254]{G. Theodoridis}, \hyperref[auth:a1255]{P. Alefragis}, \hyperref[auth:a1256]{N. Voros} & Static Mapping of Applications on Heterogeneous Multi-Core Platforms Combining Logic-Based Benders Decomposition with Integer Linear Programming & No & \cite{EmeretlisTAV17} & 2017 & ACM Transactions on Design Automation of Electronic Systems & null & 4 & 42 & No & \ref{c:EmeretlisTAV17}\\
\rowlabel{a:GedikKBR17}GedikKBR17 \href{http://dx.doi.org/10.1016/j.cie.2017.03.017}{GedikKBR17} & \hyperref[auth:a568]{R. Gedik}, \hyperref[auth:a571]{E. Kirac}, \hyperref[auth:a1175]{Ashlea Bennet Milburn}, \hyperref[auth:a1176]{C. Rainwater} & A constraint programming approach for the team orienteering problem with time windows & No & \cite{GedikKBR17} & 2017 & Computers \  Industrial Engineering & null & 20 & 32 & No & \ref{c:GedikKBR17}\\
\rowlabel{a:GomesM17}GomesM17 \href{http://dx.doi.org/10.1155/2017/9452762}{GomesM17} & \hyperref[auth:a978]{Francisco Regis Abreu Gomes}, \hyperref[auth:a979]{Geraldo Robson Mateus} & Improved Combinatorial Benders Decomposition for a Scheduling Problem with Unrelated Parallel Machines & \href{../works/GomesM17.pdf}{Yes} & \cite{GomesM17} & 2017 & Journal of Applied Mathematics & 11 & 1 & 43 & \ref{b:GomesM17} & \ref{c:GomesM17}\\
\rowlabel{a:HamFC17}HamFC17 \href{http://dx.doi.org/10.1109/tsm.2017.2740340}{HamFC17} & \hyperref[auth:a758]{A. Ham}, \hyperref[auth:a1227]{John W. Fowler}, \hyperref[auth:a884]{E. Cakici} & Constraint Programming Approach for Scheduling Jobs With Release Times,  Non-Identical Sizes,  and Incompatible Families on Parallel Batching Machines & No & \cite{HamFC17} & 2017 & IEEE Transactions on Semiconductor Manufacturing & null & 21 & 28 & No & \ref{c:HamFC17}\\
\rowlabel{a:HookerH17}HookerH17 \href{http://dx.doi.org/10.1007/s10601-017-9280-3}{HookerH17} & \hyperref[auth:a161]{John N. Hooker}, \hyperref[auth:a841]{Willem-Jan van Hoeve} & Constraint programming and operations research & \href{../works/HookerH17.pdf}{Yes} & \cite{HookerH17} & 2017 & Constraints An Int. J. & 24 & 12 & 189 & \ref{b:HookerH17} & \ref{c:HookerH17}\\
\rowlabel{a:KreterSS17}KreterSS17 \href{https://doi.org/10.1007/s10601-016-9266-6}{KreterSS17} & \hyperref[auth:a124]{S. Kreter}, \hyperref[auth:a125]{A. Schutt}, \hyperref[auth:a126]{Peter J. Stuckey} & Using constraint programming for solving RCPSP/max-cal & \href{../works/KreterSS17.pdf}{Yes} & \cite{KreterSS17} & 2017 & Constraints An Int. J. & 31 & 15 & 20 & \ref{b:KreterSS17} & \ref{c:KreterSS17}\\
\rowlabel{a:NattafAL17}NattafAL17 \href{https://doi.org/10.1007/s10601-017-9271-4}{NattafAL17} & \hyperref[auth:a81]{M. Nattaf}, \hyperref[auth:a6]{C. Artigues}, \hyperref[auth:a3]{P. Lopez} & Cumulative scheduling with variable task profiles and concave piecewise linear processing rate functions & \href{../works/NattafAL17.pdf}{Yes} & \cite{NattafAL17} & 2017 & Constraints An Int. J. & 18 & 5 & 10 & \ref{b:NattafAL17} & \ref{c:NattafAL17}\\
\rowlabel{a:RoshanaeiLAU17}RoshanaeiLAU17 \href{http://dx.doi.org/10.1016/j.ejor.2016.08.024}{RoshanaeiLAU17} & \hyperref[auth:a736]{V. Roshanaei}, \hyperref[auth:a937]{C. Luong}, \hyperref[auth:a904]{Dionne M. Aleman}, \hyperref[auth:a938]{D. Urbach} & Propagating logic-based Benders' decomposition approaches for distributed operating room scheduling & \href{../works/RoshanaeiLAU17.pdf}{Yes} & \cite{RoshanaeiLAU17} & 2017 & European Journal of Operational Research & 17 & 61 & 46 & \ref{b:RoshanaeiLAU17} & \ref{c:RoshanaeiLAU17}\\
\rowlabel{a:RoshanaeiLAU17a}RoshanaeiLAU17a \href{http://dx.doi.org/10.1287/ijoc.2017.0745}{RoshanaeiLAU17a} & \hyperref[auth:a736]{V. Roshanaei}, \hyperref[auth:a937]{C. Luong}, \hyperref[auth:a904]{Dionne M. Aleman}, \hyperref[auth:a905]{David R. Urbach} & Collaborative Operating Room Planning and Scheduling & No & \cite{RoshanaeiLAU17a} & 2017 & INFORMS Journal on Computing & null & 54 & 42 & No & \ref{c:RoshanaeiLAU17a}\\
\rowlabel{a:SchnellH17}SchnellH17 \href{http://dx.doi.org/10.1016/j.orp.2017.01.002}{SchnellH17} & \hyperref[auth:a963]{A. Schnell}, \hyperref[auth:a964]{Richard F. Hartl} & On the generalization of constraint programming and boolean satisfiability solving techniques to schedule a resource-constrained project consisting of multi-mode jobs & No & \cite{SchnellH17} & 2017 & Operations Research Perspectives & null & 12 & 20 & No & \ref{c:SchnellH17}\\
\rowlabel{a:TranVNB17}TranVNB17 \href{https://doi.org/10.1613/jair.5306}{TranVNB17} & \hyperref[auth:a807]{Tony T. Tran}, \hyperref[auth:a812]{Tiago Stegun Vaquero}, \hyperref[auth:a209]{G. Nejat}, \hyperref[auth:a89]{J. Christopher Beck} & Robots in Retirement Homes: Applying Off-the-Shelf Planning and Scheduling to a Team of Assistive Robots & \href{../works/TranVNB17.pdf}{Yes} & \cite{TranVNB17} & 2017 & J. Artif. Intell. Res. & 68 & 12 & 0 & \ref{b:TranVNB17} & \ref{c:TranVNB17}\\
\rowlabel{a:BlomPS16}BlomPS16 \href{https://doi.org/10.1287/mnsc.2015.2284}{BlomPS16} & \hyperref[auth:a803]{Michelle L. Blom}, \hyperref[auth:a327]{Adrian R. Pearce}, \hyperref[auth:a126]{Peter J. Stuckey} & A Decomposition-Based Algorithm for the Scheduling of Open-Pit Networks Over Multiple Time Periods & \href{../works/BlomPS16.pdf}{Yes} & \cite{BlomPS16} & 2016 & Manag. Sci. & 26 & 20 & 36 & \ref{b:BlomPS16} & \ref{c:BlomPS16}\\
\rowlabel{a:Bonfietti16}Bonfietti16 \href{https://doi.org/10.3233/IA-160095}{Bonfietti16} & \hyperref[auth:a203]{A. Bonfietti} & A constraint programming scheduling solver for the MPOpt programming environment & \href{../works/Bonfietti16.pdf}{Yes} & \cite{Bonfietti16} & 2016 & Intelligenza Artificiale & 13 & 0 & 19 & \ref{b:Bonfietti16} & \ref{c:Bonfietti16}\\
\rowlabel{a:BoothTNB16}BoothTNB16 \href{http://dx.doi.org/10.1109/lra.2016.2522096}{BoothTNB16} & \hyperref[auth:a208]{Kyle E. C. Booth}, \hyperref[auth:a807]{Tony T. Tran}, \hyperref[auth:a209]{G. Nejat}, \hyperref[auth:a89]{J. Christopher Beck} & Mixed-Integer and Constraint Programming Techniques for Mobile Robot Task Planning & No & \cite{BoothTNB16} & 2016 & IEEE Robotics and Automation Letters & null & 27 & 21 & No & \ref{c:BoothTNB16}\\
\rowlabel{a:BridiBLMB16}BridiBLMB16 \href{https://doi.org/10.1109/TPDS.2016.2516997}{BridiBLMB16} & \hyperref[auth:a232]{T. Bridi}, \hyperref[auth:a230]{A. Bartolini}, \hyperref[auth:a143]{M. Lombardi}, \hyperref[auth:a144]{M. Milano}, \hyperref[auth:a247]{L. Benini} & A Constraint Programming Scheduler for Heterogeneous High-Performance Computing Machines & \href{../works/BridiBLMB16.pdf}{Yes} & \cite{BridiBLMB16} & 2016 & {IEEE} Trans. Parallel Distributed Syst. & 14 & 17 & 22 & \ref{b:BridiBLMB16} & \ref{c:BridiBLMB16}\\
\rowlabel{a:CireCH16}CireCH16 \href{http://dx.doi.org/10.1017/s0269888916000254}{CireCH16} & \hyperref[auth:a158]{Andr{\'{e}} A. Cir{\'{e}}}, \hyperref[auth:a340]{E. Coban}, \hyperref[auth:a161]{John N. Hooker} & Logic-based Benders decomposition for planning and scheduling: a computational analysis & \href{../works/CireCH16.pdf}{Yes} & \cite{CireCH16} & 2016 & The Knowledge Engineering Review & 12 & 15 & 21 & \ref{b:CireCH16} & \ref{c:CireCH16}\\
\rowlabel{a:DoulabiRP16}DoulabiRP16 \href{https://doi.org/10.1287/ijoc.2015.0686}{DoulabiRP16} & \hyperref[auth:a335]{Seyed Hossein Hashemi Doulabi}, \hyperref[auth:a331]{L. Rousseau}, \hyperref[auth:a8]{G. Pesant} & A Constraint-Programming-Based Branch-and-Price-and-Cut Approach for Operating Room Planning and Scheduling & \href{../works/DoulabiRP16.pdf}{Yes} & \cite{DoulabiRP16} & 2016 & INFORMS Journal on Computing & 17 & 56 & 28 & \ref{b:DoulabiRP16} & \ref{c:DoulabiRP16}\\
\rowlabel{a:HamC16}HamC16 \href{http://dx.doi.org/10.1016/j.cie.2016.11.001}{HamC16} & \hyperref[auth:a778]{Andy M. Ham}, \hyperref[auth:a884]{E. Cakici} & Flexible job shop scheduling problem with parallel batch processing machines: MIP and CP approaches & \href{../works/HamC16.pdf}{Yes} & \cite{HamC16} & 2016 & Computers \  Industrial Engineering & 6 & 50 & 26 & \ref{b:HamC16} & \ref{c:HamC16}\\
\rowlabel{a:HebrardHJMPV16}HebrardHJMPV16 \href{https://doi.org/10.1016/j.dam.2016.07.003}{HebrardHJMPV16} & \hyperref[auth:a1]{E. Hebrard}, \hyperref[auth:a54]{M. Huguet}, \hyperref[auth:a799]{N. Jozefowiez}, \hyperref[auth:a795]{A. Maillard}, \hyperref[auth:a21]{C. Pralet}, \hyperref[auth:a174]{G. Verfaillie} & Approximation of the parallel machine scheduling problem with additional unit resources & \href{../works/HebrardHJMPV16.pdf}{Yes} & \cite{HebrardHJMPV16} & 2016 & Discret. Appl. Math. & 10 & 9 & 8 & \ref{b:HebrardHJMPV16} & \ref{c:HebrardHJMPV16}\\
\rowlabel{a:KuB16}KuB16 \href{https://doi.org/10.1016/j.cor.2016.04.006}{KuB16} & \hyperref[auth:a336]{W. Ku}, \hyperref[auth:a89]{J. Christopher Beck} & Mixed Integer Programming models for job shop scheduling: {A} computational analysis & \href{../works/KuB16.pdf}{Yes} & \cite{KuB16} & 2016 & Computers \  Operations Research & 9 & 119 & 17 & \ref{b:KuB16} & \ref{c:KuB16}\\
\rowlabel{a:NattafALR16}NattafALR16 \href{https://doi.org/10.1007/s00291-015-0423-x}{NattafALR16} & \hyperref[auth:a81]{M. Nattaf}, \hyperref[auth:a6]{C. Artigues}, \hyperref[auth:a3]{P. Lopez}, \hyperref[auth:a993]{D. Rivreau} & Energetic reasoning and mixed-integer linear programming for scheduling with a continuous resource and linear efficiency functions & \href{../works/NattafALR16.pdf}{Yes} & \cite{NattafALR16} & 2016 & {OR} Spectr. & 34 & 10 & 15 & \ref{b:NattafALR16} & \ref{c:NattafALR16}\\
\rowlabel{a:NovaraNH16}NovaraNH16 \href{https://doi.org/10.1016/j.compchemeng.2016.04.030}{NovaraNH16} & \hyperref[auth:a595]{Franco M. Novara}, \hyperref[auth:a529]{Juan M. Novas}, \hyperref[auth:a596]{Gabriela P. Henning} & A novel constraint programming model for large-scale scheduling problems in multiproduct multistage batch plants: Limited resources and campaign-based operation & \href{../works/NovaraNH16.pdf}{Yes} & \cite{NovaraNH16} & 2016 & Computers \  Chemical Engineering & 17 & 18 & 31 & \ref{b:NovaraNH16} & \ref{c:NovaraNH16}\\
\rowlabel{a:OrnekO16}OrnekO16 \href{https://journals.sfu.ca/ijietap/index.php/ijie/article/view/1930}{OrnekO16} & \hyperref[auth:a139]{A. {\"{O}}rnek}, \hyperref[auth:a136]{C. {\"{O}}zt{\"{u}}rk} & Optimisation and Constraint Based Heuristic Methods for Advanced Planning and Scheduling Systems & \href{../works/OrnekO16.pdf}{Yes} & \cite{OrnekO16} & 2016 & International Journal of Industrial Engineering: Theory, Applications and Practice & 25 & 0 & 0 & \ref{b:OrnekO16} & \ref{c:OrnekO16}\\
\rowlabel{a:QinDS16}QinDS16 \href{http://dx.doi.org/10.1016/j.tre.2016.01.007}{QinDS16} & \hyperref[auth:a514]{T. Qin}, \hyperref[auth:a515]{Y. Du}, \hyperref[auth:a517]{M. Sha} & Evaluating the solution performance of IP and CP for berth allocation with time-varying water depth & No & \cite{QinDS16} & 2016 & Transportation Research Part E: Logistics and Transportation Review & null & 17 & 40 & No & \ref{c:QinDS16}\\
\rowlabel{a:RiiseML16}RiiseML16 \href{http://dx.doi.org/10.1016/j.ejor.2016.06.015}{RiiseML16} & \hyperref[auth:a1082]{A. Riise}, \hyperref[auth:a1083]{C. Mannino}, \hyperref[auth:a1084]{L. Lamorgese} & Recursive logic-based Benders' decomposition for multi-mode outpatient scheduling & No & \cite{RiiseML16} & 2016 & European Journal of Operational Research & null & 27 & 29 & No & \ref{c:RiiseML16}\\
\rowlabel{a:TranAB16}TranAB16 \href{https://doi.org/10.1287/ijoc.2015.0666}{TranAB16} & \hyperref[auth:a807]{Tony T. Tran}, \hyperref[auth:a815]{A. Araujo}, \hyperref[auth:a89]{J. Christopher Beck} & Decomposition Methods for the Parallel Machine Scheduling Problem with Setups & \href{../works/TranAB16.pdf}{Yes} & \cite{TranAB16} & 2016 & INFORMS Journal on Computing & 13 & 72 & 28 & \ref{b:TranAB16} & \ref{c:TranAB16}\\
\rowlabel{a:ZarandiKS16}ZarandiKS16 \href{https://doi.org/10.1007/s10845-013-0860-9}{ZarandiKS16} & \hyperref[auth:a597]{M. H. Fazel Zarandi}, \hyperref[auth:a598]{H. Khorshidian}, \hyperref[auth:a599]{Mohsen Akbarpour Shirazi} & A constraint programming model for the scheduling of {JIT} cross-docking systems with preemption & \href{../works/ZarandiKS16.pdf}{Yes} & \cite{ZarandiKS16} & 2016 & Journal of Intelligent Manufacturing & 17 & 28 & 14 & \ref{b:ZarandiKS16} & \ref{c:ZarandiKS16}\\
\rowlabel{a:AlesioBNG15}AlesioBNG15 \href{http://dx.doi.org/10.1145/2818640}{AlesioBNG15} & \hyperref[auth:a1249]{Stefano Di Alesio}, \hyperref[auth:a238]{Lionel C. Briand}, \hyperref[auth:a237]{S. Nejati}, \hyperref[auth:a200]{A. Gotlieb} & Combining Genetic Algorithms and Constraint Programming to Support Stress Testing of Task Deadlines & No & \cite{AlesioBNG15} & 2015 & ACM Transactions on Software Engineering and Methodology & null & 13 & 51 & No & \ref{c:AlesioBNG15}\\
\rowlabel{a:BajestaniB15}BajestaniB15 \href{https://doi.org/10.1007/s10951-015-0416-2}{BajestaniB15} & \hyperref[auth:a825]{Maliheh Aramon Bajestani}, \hyperref[auth:a89]{J. Christopher Beck} & A two-stage coupled algorithm for an integrated maintenance planning and flowshop scheduling problem with deteriorating machines & \href{../works/BajestaniB15.pdf}{Yes} & \cite{BajestaniB15} & 2015 & Journal of Scheduling & 16 & 17 & 59 & \ref{b:BajestaniB15} & \ref{c:BajestaniB15}\\
\rowlabel{a:EvenSH15a}EvenSH15a \href{http://arxiv.org/abs/1505.02487}{EvenSH15a} & \hyperref[auth:a219]{C. Even}, \hyperref[auth:a125]{A. Schutt}, \hyperref[auth:a149]{Pascal Van Hentenryck} & A Constraint Programming Approach for Non-Preemptive Evacuation Scheduling & \href{../works/EvenSH15a.pdf}{Yes} & \cite{EvenSH15a} & 2015 & CoRR & 16 & 0 & 0 & \ref{b:EvenSH15a} & \ref{c:EvenSH15a}\\
\rowlabel{a:GoelSHFS15}GoelSHFS15 \href{https://doi.org/10.1016/j.ejor.2014.09.048}{GoelSHFS15} & \hyperref[auth:a600]{V. Goel}, \hyperref[auth:a601]{M. Slusky}, \hyperref[auth:a211]{Willem{-}Jan van Hoeve}, \hyperref[auth:a602]{Kevin C. Furman}, \hyperref[auth:a603]{Y. Shao} & Constraint programming for {LNG} ship scheduling and inventory management & \href{../works/GoelSHFS15.pdf}{Yes} & \cite{GoelSHFS15} & 2015 & European Journal of Operational Research & 12 & 48 & 4 & \ref{b:GoelSHFS15} & \ref{c:GoelSHFS15}\\
\rowlabel{a:GrimesH15}GrimesH15 \href{https://doi.org/10.1287/ijoc.2014.0625}{GrimesH15} & \hyperref[auth:a182]{D. Grimes}, \hyperref[auth:a1]{E. Hebrard} & Solving Variants of the Job Shop Scheduling Problem Through Conflict-Directed Search & \href{../works/GrimesH15.pdf}{Yes} & \cite{GrimesH15} & 2015 & INFORMS Journal on Computing & 17 & 12 & 41 & \ref{b:GrimesH15} & \ref{c:GrimesH15}\\
\rowlabel{a:Kameugne15}Kameugne15 \href{https://doi.org/10.1007/s10601-015-9227-5}{Kameugne15} & \hyperref[auth:a10]{R. Kameugne} & Propagation techniques of resource constraint for cumulative scheduling & \href{../works/Kameugne15.pdf}{Yes} & \cite{Kameugne15} & 2015 & Constraints An Int. J. & 2 & 0 & 0 & \ref{b:Kameugne15} & \ref{c:Kameugne15}\\
\rowlabel{a:LetortCB15}LetortCB15 \href{https://doi.org/10.1007/s10601-014-9172-8}{LetortCB15} & \hyperref[auth:a128]{A. Letort}, \hyperref[auth:a91]{M. Carlsson}, \hyperref[auth:a129]{N. Beldiceanu} & Synchronized sweep algorithms for scalable scheduling constraints & \href{../works/LetortCB15.pdf}{Yes} & \cite{LetortCB15} & 2015 & Constraints An Int. J. & 52 & 2 & 14 & \ref{b:LetortCB15} & \ref{c:LetortCB15}\\
\rowlabel{a:NattafAL15}NattafAL15 \href{https://doi.org/10.1007/s10601-015-9192-z}{NattafAL15} & \hyperref[auth:a81]{M. Nattaf}, \hyperref[auth:a6]{C. Artigues}, \hyperref[auth:a3]{P. Lopez} & A hybrid exact method for a scheduling problem with a continuous resource and energy constraints & \href{../works/NattafAL15.pdf}{Yes} & \cite{NattafAL15} & 2015 & Constraints An Int. J. & 21 & 14 & 13 & \ref{b:NattafAL15} & \ref{c:NattafAL15}\\
\rowlabel{a:OzturkTHO15}OzturkTHO15 \href{https://www.sciencedirect.com/science/article/pii/S0278612515000527}{OzturkTHO15} & \hyperref[auth:a136]{C. {\"{O}}zt{\"{u}}rk}, \hyperref[auth:a1031]{S. Tunalı}, \hyperref[auth:a138]{B. Hnich}, \hyperref[auth:a139]{A. {\"{O}}rnek} & Cyclic scheduling of flexible mixed model assembly lines with parallel stations & \href{../works/OzturkTHO15.pdf}{Yes} & \cite{OzturkTHO15} & 2015 & Journal of Manufacturing Systems & 12 & 27 & 17 & \ref{b:OzturkTHO15} & \ref{c:OzturkTHO15}\\
\rowlabel{a:SchnellH15}SchnellH15 \href{http://dx.doi.org/10.1007/s00291-015-0419-6}{SchnellH15} & \hyperref[auth:a963]{A. Schnell}, \hyperref[auth:a964]{Richard F. Hartl} & On the efficient modeling and solution of the multi-mode resource-constrained project scheduling problem with generalized precedence relations & \href{../works/SchnellH15.pdf}{Yes} & \cite{SchnellH15} & 2015 & OR Spectrum & 21 & 24 & 20 & \ref{b:SchnellH15} & \ref{c:SchnellH15}\\
\rowlabel{a:Siala15}Siala15 \href{https://doi.org/10.1007/s10601-015-9213-y}{Siala15} & \hyperref[auth:a130]{M. Siala} & Search, propagation, and learning in sequencing and scheduling problems & \href{../works/Siala15.pdf}{Yes} & \cite{Siala15} & 2015 & Constraints An Int. J. & 2 & 4 & 0 & \ref{b:Siala15} & \ref{c:Siala15}\\
\rowlabel{a:SimoninAHL15}SimoninAHL15 \href{https://doi.org/10.1007/s10601-014-9169-3}{SimoninAHL15} & \hyperref[auth:a127]{G. Simonin}, \hyperref[auth:a6]{C. Artigues}, \hyperref[auth:a1]{E. Hebrard}, \hyperref[auth:a3]{P. Lopez} & Scheduling scientific experiments for comet exploration & \href{../works/SimoninAHL15.pdf}{Yes} & \cite{SimoninAHL15} & 2015 & Constraints An Int. J. & 23 & 4 & 5 & \ref{b:SimoninAHL15} & \ref{c:SimoninAHL15}\\
\rowlabel{a:WangMD15}WangMD15 \href{https://doi.org/10.1016/j.ejor.2015.06.008}{WangMD15} & \hyperref[auth:a604]{T. Wang}, \hyperref[auth:a605]{N. Meskens}, \hyperref[auth:a606]{D. Duvivier} & Scheduling operating theatres: Mixed integer programming vs. constraint programming & \href{../works/WangMD15.pdf}{Yes} & \cite{WangMD15} & 2015 & European Journal of Operational Research & 13 & 36 & 33 & \ref{b:WangMD15} & \ref{c:WangMD15}\\
\rowlabel{a:ArtiguesL14}ArtiguesL14 \href{http://dx.doi.org/10.1007/s10951-014-0404-y}{ArtiguesL14} & \hyperref[auth:a6]{C. Artigues}, \hyperref[auth:a3]{P. Lopez} & Energetic reasoning for energy-constrained scheduling with a continuous resource & No & \cite{ArtiguesL14} & 2014 & Journal of Scheduling & null & 11 & 19 & No & \ref{c:ArtiguesL14}\\
\rowlabel{a:BlomBPS14}BlomBPS14 \href{https://doi.org/10.1287/ijoc.2013.0590}{BlomBPS14} & \hyperref[auth:a803]{Michelle L. Blom}, \hyperref[auth:a325]{Christina N. Burt}, \hyperref[auth:a327]{Adrian R. Pearce}, \hyperref[auth:a126]{Peter J. Stuckey} & A Decomposition-Based Heuristic for Collaborative Scheduling in a Network of Open-Pit Mines & \href{../works/BlomBPS14.pdf}{Yes} & \cite{BlomBPS14} & 2014 & INFORMS Journal on Computing & 19 & 15 & 47 & \ref{b:BlomBPS14} & \ref{c:BlomBPS14}\\
\rowlabel{a:BonfiettiLBM14}BonfiettiLBM14 \href{https://doi.org/10.1016/j.artint.2013.09.006}{BonfiettiLBM14} & \hyperref[auth:a203]{A. Bonfietti}, \hyperref[auth:a143]{M. Lombardi}, \hyperref[auth:a247]{L. Benini}, \hyperref[auth:a144]{M. Milano} & {CROSS} cyclic resource-constrained scheduling solver & \href{../works/BonfiettiLBM14.pdf}{Yes} & \cite{BonfiettiLBM14} & 2014 & Artificial Intelligence & 28 & 8 & 15 & \ref{b:BonfiettiLBM14} & \ref{c:BonfiettiLBM14}\\
\rowlabel{a:GrimesIOS14}GrimesIOS14 \href{https://doi.org/10.1016/j.suscom.2014.08.009}{GrimesIOS14} & \hyperref[auth:a182]{D. Grimes}, \hyperref[auth:a183]{G. Ifrim}, \hyperref[auth:a16]{B. O'Sullivan}, \hyperref[auth:a17]{H. Simonis} & Analyzing the impact of electricity price forecasting on energy cost-aware scheduling & \href{../works/GrimesIOS14.pdf}{Yes} & \cite{GrimesIOS14} & 2014 & Sustain. Comput. Informatics Syst. & 16 & 6 & 7 & \ref{b:GrimesIOS14} & \ref{c:GrimesIOS14}\\
\rowlabel{a:HarjunkoskiMBC14}HarjunkoskiMBC14 \href{http://dx.doi.org/10.1016/j.compchemeng.2013.12.001}{HarjunkoskiMBC14} & \hyperref[auth:a880]{I. Harjunkoski}, \hyperref[auth:a386]{Christos T. Maravelias}, \hyperref[auth:a949]{P. Bongers}, \hyperref[auth:a900]{Pedro M. Castro}, \hyperref[auth:a70]{S. Engell}, \hyperref[auth:a387]{Ignacio E. Grossmann}, \hyperref[auth:a161]{John N. Hooker}, \hyperref[auth:a950]{C. Méndez}, \hyperref[auth:a951]{G. Sand}, \hyperref[auth:a952]{J. Wassick} & Scope for industrial applications of production scheduling models and solution methods & \href{../works/HarjunkoskiMBC14.pdf}{Yes} & \cite{HarjunkoskiMBC14} & 2014 & Computers \  Chemical Engineering & 33 & 381 & 176 & \ref{b:HarjunkoskiMBC14} & \ref{c:HarjunkoskiMBC14}\\
\rowlabel{a:KameugneFSN14}KameugneFSN14 \href{https://doi.org/10.1007/s10601-013-9157-z}{KameugneFSN14} & \hyperref[auth:a10]{R. Kameugne}, \hyperref[auth:a131]{Laure Pauline Fotso}, \hyperref[auth:a132]{Joseph D. Scott}, \hyperref[auth:a133]{Y. Ngo{-}Kateu} & A quadratic edge-finding filtering algorithm for cumulative resource constraints & \href{../works/KameugneFSN14.pdf}{Yes} & \cite{KameugneFSN14} & 2014 & Constraints An Int. J. & 27 & 6 & 10 & \ref{b:KameugneFSN14} & \ref{c:KameugneFSN14}\\
\rowlabel{a:LaborieR14}LaborieR14 \href{http://dx.doi.org/10.1007/s10951-014-0408-7}{LaborieR14} & \hyperref[auth:a118]{P. Laborie}, \hyperref[auth:a1087]{J. Rogerie} & Temporal linear relaxation in IBM ILOG CP Optimizer & \href{../works/LaborieR14.pdf}{Yes} & \cite{LaborieR14} & 2014 & Journal of Scheduling & 10 & 17 & 13 & \ref{b:LaborieR14} & \ref{c:LaborieR14}\\
\rowlabel{a:NovasH14}NovasH14 \href{https://doi.org/10.1016/j.eswa.2013.09.026}{NovasH14} & \hyperref[auth:a529]{Juan M. Novas}, \hyperref[auth:a596]{Gabriela P. Henning} & Integrated scheduling of resource-constrained flexible manufacturing systems using constraint programming & \href{../works/NovasH14.pdf}{Yes} & \cite{NovasH14} & 2014 & Expert Syst. Appl. & 14 & 35 & 26 & \ref{b:NovasH14} & \ref{c:NovasH14}\\
\rowlabel{a:TerekhovTDB14}TerekhovTDB14 \href{https://doi.org/10.1613/jair.4278}{TerekhovTDB14} & \hyperref[auth:a826]{D. Terekhov}, \hyperref[auth:a807]{Tony T. Tran}, \hyperref[auth:a811]{Douglas G. Down}, \hyperref[auth:a89]{J. Christopher Beck} & Integrating Queueing Theory and Scheduling for Dynamic Scheduling Problems & \href{../works/TerekhovTDB14.pdf}{Yes} & \cite{TerekhovTDB14} & 2014 & J. Artif. Intell. Res. & 38 & 12 & 0 & \ref{b:TerekhovTDB14} & \ref{c:TerekhovTDB14}\\
\rowlabel{a:ThiruvadyWGS14}ThiruvadyWGS14 \href{https://doi.org/10.1007/s10732-014-9260-3}{ThiruvadyWGS14} & \hyperref[auth:a401]{Dhananjay R. Thiruvady}, \hyperref[auth:a117]{Mark G. Wallace}, \hyperref[auth:a341]{H. Gu}, \hyperref[auth:a125]{A. Schutt} & A Lagrangian relaxation and {ACO} hybrid for resource constrained project scheduling with discounted cash flows & \href{../works/ThiruvadyWGS14.pdf}{Yes} & \cite{ThiruvadyWGS14} & 2014 & J. Heuristics & 34 & 19 & 18 & \ref{b:ThiruvadyWGS14} & \ref{c:ThiruvadyWGS14}\\
\rowlabel{a:ArtiguesLH13}ArtiguesLH13 \href{http://dx.doi.org/10.1016/j.ijpe.2010.09.030}{ArtiguesLH13} & \hyperref[auth:a6]{C. Artigues}, \hyperref[auth:a3]{P. Lopez}, \hyperref[auth:a1184]{A. Haït} & The energy scheduling problem: Industrial case-study and constraint propagation techniques & No & \cite{ArtiguesLH13} & 2013 & International Journal of Production Economics & null & 76 & 16 & No & \ref{c:ArtiguesLH13}\\
\rowlabel{a:BajestaniB13}BajestaniB13 \href{https://doi.org/10.1613/jair.3902}{BajestaniB13} & \hyperref[auth:a825]{Maliheh Aramon Bajestani}, \hyperref[auth:a89]{J. Christopher Beck} & Scheduling a Dynamic Aircraft Repair Shop with Limited Repair Resources & \href{../works/BajestaniB13.pdf}{Yes} & \cite{BajestaniB13} & 2013 & J. Artif. Intell. Res. & 36 & 14 & 0 & \ref{b:BajestaniB13} & \ref{c:BajestaniB13}\\
\rowlabel{a:BegB13}BegB13 \href{http://doi.acm.org/10.1145/2512470}{BegB13} & \hyperref[auth:a617]{Mirza Omer Beg}, \hyperref[auth:a618]{Peter van Beek} & A constraint programming approach for integrated spatial and temporal scheduling for clustered architectures & \href{../works/BegB13.pdf}{Yes} & \cite{BegB13} & 2013 & {ACM} Trans. Embed. Comput. Syst. & 23 & 1 & 28 & \ref{b:BegB13} & \ref{c:BegB13}\\
\rowlabel{a:HeinzSB13}HeinzSB13 \href{https://doi.org/10.1007/s10601-012-9136-9}{HeinzSB13} & \hyperref[auth:a134]{S. Heinz}, \hyperref[auth:a135]{J. Schulz}, \hyperref[auth:a89]{J. Christopher Beck} & Using dual presolving reductions to reformulate cumulative constraints & \href{../works/HeinzSB13.pdf}{Yes} & \cite{HeinzSB13} & 2013 & Constraints An Int. J. & 36 & 7 & 31 & \ref{b:HeinzSB13} & \ref{c:HeinzSB13}\\
\rowlabel{a:KameugneF13}KameugneF13 \href{http://dx.doi.org/10.1007/s13226-013-0005-z}{KameugneF13} & \hyperref[auth:a10]{R. Kameugne}, \hyperref[auth:a131]{Laure Pauline Fotso} & A cumulative not-first/not-last filtering algorithm in O(n 2log(n)) & No & \cite{KameugneF13} & 2013 & Indian Journal of Pure and Applied Mathematics & null & 6 & 4 & No & \ref{c:KameugneF13}\\
\rowlabel{a:LombardiMB13}LombardiMB13 \href{http://dx.doi.org/10.1109/tc.2011.203}{LombardiMB13} & \hyperref[auth:a143]{M. Lombardi}, \hyperref[auth:a144]{M. Milano}, \hyperref[auth:a247]{L. Benini} & Robust Scheduling of Task Graphs under Execution Time Uncertainty & \href{../works/LombardiMB13.pdf}{Yes} & \cite{LombardiMB13} & 2013 & IEEE Transactions on Computers & 14 & 28 & 29 & \ref{b:LombardiMB13} & \ref{c:LombardiMB13}\\
\rowlabel{a:MenciaSV13}MenciaSV13 \href{http://dx.doi.org/10.1007/s10845-012-0726-6}{MenciaSV13} & \hyperref[auth:a928]{C. Mencía}, \hyperref[auth:a929]{María R. Sierra}, \hyperref[auth:a930]{R. Varela} & Intensified iterative deepening A* with application to job shop scheduling & \href{../works/MenciaSV13.pdf}{Yes} & \cite{MenciaSV13} & 2013 & Journal of Intelligent Manufacturing & 11 & 9 & 43 & \ref{b:MenciaSV13} & \ref{c:MenciaSV13}\\
\rowlabel{a:OzturkTHO13}OzturkTHO13 \href{https://doi.org/10.1007/s10601-013-9142-6}{OzturkTHO13} & \hyperref[auth:a136]{C. {\"{O}}zt{\"{u}}rk}, \hyperref[auth:a137]{S. Tunali}, \hyperref[auth:a138]{B. Hnich}, \hyperref[auth:a139]{A. {\"{O}}rnek} & Balancing and scheduling of flexible mixed model assembly lines & \href{../works/OzturkTHO13.pdf}{Yes} & \cite{OzturkTHO13} & 2013 & Constraints An Int. J. & 36 & 31 & 44 & \ref{b:OzturkTHO13} & \ref{c:OzturkTHO13}\\
\rowlabel{a:SchuttFSW13}SchuttFSW13 \href{https://doi.org/10.1007/s10951-012-0285-x}{SchuttFSW13} & \hyperref[auth:a125]{A. Schutt}, \hyperref[auth:a155]{T. Feydy}, \hyperref[auth:a126]{Peter J. Stuckey}, \hyperref[auth:a117]{Mark G. Wallace} & Solving RCPSP/max by lazy clause generation & \href{../works/SchuttFSW13.pdf}{Yes} & \cite{SchuttFSW13} & 2013 & Journal of Scheduling & 17 & 43 & 23 & \ref{b:SchuttFSW13} & \ref{c:SchuttFSW13}\\
\rowlabel{a:UnsalO13}UnsalO13 \href{http://dx.doi.org/10.1016/j.tre.2013.08.006}{UnsalO13} & \hyperref[auth:a1243]{O. Unsal}, \hyperref[auth:a352]{C. Oguz} & Constraint programming approach to quay crane scheduling problem & No & \cite{UnsalO13} & 2013 & Transportation Research Part E: Logistics and Transportation Review & null & 44 & 25 & No & \ref{c:UnsalO13}\\
\rowlabel{a:GuyonLPR12}GuyonLPR12 \href{http://dx.doi.org/10.1007/s10479-012-1132-3}{GuyonLPR12} & \hyperref[auth:a990]{O. Guyon}, \hyperref[auth:a991]{P. Lemaire}, \hyperref[auth:a992]{Éric Pinson}, \hyperref[auth:a993]{D. Rivreau} & Solving an integrated job-shop problem with human resource constraints & \href{../works/GuyonLPR12.pdf}{Yes} & \cite{GuyonLPR12} & 2012 & Annals of Operations Research & 25 & 32 & 25 & \ref{b:GuyonLPR12} & \ref{c:GuyonLPR12}\\
\rowlabel{a:HeinzSSW12}HeinzSSW12 \href{https://doi.org/10.1007/s10601-011-9113-8}{HeinzSSW12} & \hyperref[auth:a134]{S. Heinz}, \hyperref[auth:a140]{T. Schlechte}, \hyperref[auth:a141]{R. Stephan}, \hyperref[auth:a142]{M. Winkler} & Solving steel mill slab design problems & \href{../works/HeinzSSW12.pdf}{Yes} & \cite{HeinzSSW12} & 2012 & Constraints An Int. J. & 12 & 10 & 9 & \ref{b:HeinzSSW12} & \ref{c:HeinzSSW12}\\
\rowlabel{a:LimtanyakulS12}LimtanyakulS12 \href{https://doi.org/10.1007/s10601-012-9118-y}{LimtanyakulS12} & \hyperref[auth:a145]{K. Limtanyakul}, \hyperref[auth:a146]{U. Schwiegelshohn} & Improvements of constraint programming and hybrid methods for scheduling of tests on vehicle prototypes & \href{../works/LimtanyakulS12.pdf}{Yes} & \cite{LimtanyakulS12} & 2012 & Constraints An Int. J. & 32 & 4 & 16 & \ref{b:LimtanyakulS12} & \ref{c:LimtanyakulS12}\\
\rowlabel{a:LombardiM12}LombardiM12 \href{https://doi.org/10.1007/s10601-011-9115-6}{LombardiM12} & \hyperref[auth:a143]{M. Lombardi}, \hyperref[auth:a144]{M. Milano} & Optimal methods for resource allocation and scheduling: a cross-disciplinary survey & \href{../works/LombardiM12.pdf}{Yes} & \cite{LombardiM12} & 2012 & Constraints An Int. J. & 35 & 39 & 68 & \ref{b:LombardiM12} & \ref{c:LombardiM12}\\
\rowlabel{a:LombardiM12a}LombardiM12a \href{https://doi.org/10.1016/j.artint.2011.12.001}{LombardiM12a} & \hyperref[auth:a143]{M. Lombardi}, \hyperref[auth:a144]{M. Milano} & A min-flow algorithm for Minimal Critical Set detection in Resource Constrained Project Scheduling & \href{../works/LombardiM12a.pdf}{Yes} & \cite{LombardiM12a} & 2012 & Artificial Intelligence & 10 & 3 & 13 & \ref{b:LombardiM12a} & \ref{c:LombardiM12a}\\
\rowlabel{a:MalapertCGJLR12}MalapertCGJLR12 \href{https://doi.org/10.1287/ijoc.1100.0446}{MalapertCGJLR12} & \hyperref[auth:a82]{A. Malapert}, \hyperref[auth:a1013]{H. Cambazard}, \hyperref[auth:a295]{C. Gu{\'{e}}ret}, \hyperref[auth:a249]{N. Jussien}, \hyperref[auth:a653]{A. Langevin}, \hyperref[auth:a331]{L. Rousseau} & An Optimal Constraint Programming Approach to the Open-Shop Problem & \href{../works/MalapertCGJLR12.pdf}{Yes} & \cite{MalapertCGJLR12} & 2012 & INFORMS Journal on Computing & 17 & 23 & 21 & \ref{b:MalapertCGJLR12} & \ref{c:MalapertCGJLR12}\\
\rowlabel{a:MenciaSV12}MenciaSV12 \href{http://dx.doi.org/10.1007/s10479-012-1296-x}{MenciaSV12} & \hyperref[auth:a928]{C. Mencía}, \hyperref[auth:a929]{María R. Sierra}, \hyperref[auth:a930]{R. Varela} & Depth-first heuristic search for the job shop scheduling problem & \href{../works/MenciaSV12.pdf}{Yes} & \cite{MenciaSV12} & 2012 & Annals of Operations Research & 32 & 16 & 40 & \ref{b:MenciaSV12} & \ref{c:MenciaSV12}\\
\rowlabel{a:NovasH12}NovasH12 \href{https://doi.org/10.1016/j.compchemeng.2012.01.005}{NovasH12} & \hyperref[auth:a529]{Juan M. Novas}, \hyperref[auth:a596]{Gabriela P. Henning} & A comprehensive constraint programming approach for the rolling horizon-based scheduling of automated wet-etch stations & \href{../works/NovasH12.pdf}{Yes} & \cite{NovasH12} & 2012 & Computers \  Chemical Engineering & 17 & 17 & 15 & \ref{b:NovasH12} & \ref{c:NovasH12}\\
\rowlabel{a:OzturkTHO12}OzturkTHO12 \href{https://www.sciencedirect.com/science/article/pii/S1474667016331858}{OzturkTHO12} & \hyperref[auth:a1030]{C. {\"{O}}zt{\"{u}}rk}, \hyperref[auth:a1031]{S. Tunalı}, \hyperref[auth:a138]{B. Hnich}, \hyperref[auth:a139]{A. {\"{O}}rnek} & A Constraint Programming Model for Balancing and Scheduling of Flexible Mixed Model Assembly Lines with Parallel Stations & \href{../works/OzturkTHO12.pdf}{Yes} & \cite{OzturkTHO12} & 2012 & IFAC Proceedings Volumes & 6 & 5 & 5 & \ref{b:OzturkTHO12} & \ref{c:OzturkTHO12}\\
\rowlabel{a:TerekhovDOB12}TerekhovDOB12 \href{https://doi.org/10.1016/j.cie.2012.02.006}{TerekhovDOB12} & \hyperref[auth:a826]{D. Terekhov}, \hyperref[auth:a828]{Mustafa K. Dogru}, \hyperref[auth:a829]{U. {\"{O}}zen}, \hyperref[auth:a89]{J. Christopher Beck} & Solving two-machine assembly scheduling problems with inventory constraints & \href{../works/TerekhovDOB12.pdf}{Yes} & \cite{TerekhovDOB12} & 2012 & Computers \  Industrial Engineering & 15 & 8 & 48 & \ref{b:TerekhovDOB12} & \ref{c:TerekhovDOB12}\\
\rowlabel{a:ZarandiB12}ZarandiB12 \href{http://dx.doi.org/10.1287/ijoc.1110.0458}{ZarandiB12} & \hyperref[auth:a957]{Mohammad M. Fazel-Zarandi}, \hyperref[auth:a89]{J. Christopher Beck} & Using Logic-Based Benders Decomposition to Solve the Capacity- and Distance-Constrained Plant Location Problem & No & \cite{ZarandiB12} & 2012 & INFORMS Journal on Computing & null & 38 & 57 & No & \ref{c:ZarandiB12}\\
\rowlabel{a:BandaSC11}BandaSC11 \href{https://doi.org/10.1287/ijoc.1090.0378}{BandaSC11} & \hyperref[auth:a804]{Maria Garcia de la Banda}, \hyperref[auth:a126]{Peter J. Stuckey}, \hyperref[auth:a348]{G. Chu} & Solving Talent Scheduling with Dynamic Programming & \href{../works/BandaSC11.pdf}{Yes} & \cite{BandaSC11} & 2011 & INFORMS Journal on Computing & 18 & 24 & 17 & \ref{b:BandaSC11} & \ref{c:BandaSC11}\\
\rowlabel{a:BartakS11}BartakS11 \href{https://doi.org/10.1007/s10601-011-9109-4}{BartakS11} & \hyperref[auth:a153]{R. Bart{\'{a}}k}, \hyperref[auth:a154]{Miguel A. Salido} & Constraint satisfaction for planning and scheduling problems & \href{../works/BartakS11.pdf}{Yes} & \cite{BartakS11} & 2011 & Constraints An Int. J. & 5 & 17 & 3 & \ref{b:BartakS11} & \ref{c:BartakS11}\\
\rowlabel{a:BeckFW11}BeckFW11 \href{https://doi.org/10.1287/ijoc.1100.0388}{BeckFW11} & \hyperref[auth:a89]{J. Christopher Beck}, \hyperref[auth:a830]{T. K. Feng}, \hyperref[auth:a365]{J. Watson} & Combining Constraint Programming and Local Search for Job-Shop Scheduling & \href{../works/BeckFW11.pdf}{Yes} & \cite{BeckFW11} & 2011 & INFORMS Journal on Computing & 14 & 43 & 23 & \ref{b:BeckFW11} & \ref{c:BeckFW11}\\
\rowlabel{a:BeldiceanuCDP11}BeldiceanuCDP11 \href{https://doi.org/10.1007/s10479-010-0731-0}{BeldiceanuCDP11} & \hyperref[auth:a129]{N. Beldiceanu}, \hyperref[auth:a91]{M. Carlsson}, \hyperref[auth:a245]{S. Demassey}, \hyperref[auth:a363]{E. Poder} & New filtering for the \emph{cumulative} constraint in the context of non-overlapping rectangles & \href{../works/BeldiceanuCDP11.pdf}{Yes} & \cite{BeldiceanuCDP11} & 2011 & Annals of Operations Research & 24 & 8 & 8 & \ref{b:BeldiceanuCDP11} & \ref{c:BeldiceanuCDP11}\\
\rowlabel{a:BeniniLMR11}BeniniLMR11 \href{https://doi.org/10.1007/s10479-010-0718-x}{BeniniLMR11} & \hyperref[auth:a247]{L. Benini}, \hyperref[auth:a143]{M. Lombardi}, \hyperref[auth:a144]{M. Milano}, \hyperref[auth:a726]{M. Ruggiero} & Optimal resource allocation and scheduling for the {CELL} {BE} platform & \href{../works/BeniniLMR11.pdf}{Yes} & \cite{BeniniLMR11} & 2011 & Annals of Operations Research & 27 & 18 & 16 & \ref{b:BeniniLMR11} & \ref{c:BeniniLMR11}\\
\rowlabel{a:CobanH11}CobanH11 \href{http://dx.doi.org/10.1007/s10479-011-1031-z}{CobanH11} & \hyperref[auth:a340]{E. Coban}, \hyperref[auth:a161]{John N. Hooker} & Single-facility scheduling by logic-based Benders decomposition & \href{../works/CobanH11.pdf}{Yes} & \cite{CobanH11} & 2011 & Annals of Operations Research & 28 & 14 & 37 & \ref{b:CobanH11} & \ref{c:CobanH11}\\
\rowlabel{a:EdisO11a}EdisO11a \href{http://dx.doi.org/10.1080/03052151003759117}{EdisO11a} & \hyperref[auth:a351]{Emrah B. Edis}, \hyperref[auth:a353]{I. Ozkarahan} & A combined integer/constraint programming approach to a resource-constrained parallel machine scheduling problem with machine eligibility restrictions & No & \cite{EdisO11a} & 2011 & Engineering Optimization & null & 43 & 37 & No & \ref{c:EdisO11a}\\
\rowlabel{a:HachemiGR11}HachemiGR11 \href{https://doi.org/10.1007/s10479-010-0698-x}{HachemiGR11} & \hyperref[auth:a623]{Nizar El Hachemi}, \hyperref[auth:a624]{M. Gendreau}, \hyperref[auth:a331]{L. Rousseau} & A hybrid constraint programming approach to the log-truck scheduling problem & \href{../works/HachemiGR11.pdf}{Yes} & \cite{HachemiGR11} & 2011 & Annals of Operations Research & 16 & 32 & 19 & \ref{b:HachemiGR11} & \ref{c:HachemiGR11}\\
\rowlabel{a:HeckmanB11}HeckmanB11 \href{https://doi.org/10.1007/s10951-009-0113-0}{HeckmanB11} & \hyperref[auth:a831]{I. Heckman}, \hyperref[auth:a89]{J. Christopher Beck} & Understanding the behavior of Solution-Guided Search for job-shop scheduling & \href{../works/HeckmanB11.pdf}{Yes} & \cite{HeckmanB11} & 2011 & Journal of Scheduling & 20 & 0 & 22 & \ref{b:HeckmanB11} & \ref{c:HeckmanB11}\\
\rowlabel{a:KelbelH11}KelbelH11 \href{https://doi.org/10.1007/s10845-009-0318-2}{KelbelH11} & \hyperref[auth:a626]{J. Kelbel}, \hyperref[auth:a116]{Z. Hanz{\'{a}}lek} & Solving production scheduling with earliness/tardiness penalties by constraint programming & \href{../works/KelbelH11.pdf}{Yes} & \cite{KelbelH11} & 2011 & Journal of Intelligent Manufacturing & 10 & 12 & 14 & \ref{b:KelbelH11} & \ref{c:KelbelH11}\\
\rowlabel{a:KovacsB11}KovacsB11 \href{https://doi.org/10.1007/s10601-009-9088-x}{KovacsB11} & \hyperref[auth:a147]{A. Kov{\'{a}}cs}, \hyperref[auth:a89]{J. Christopher Beck} & A global constraint for total weighted completion time for unary resources & \href{../works/KovacsB11.pdf}{Yes} & \cite{KovacsB11} & 2011 & Constraints An Int. J. & 24 & 4 & 26 & \ref{b:KovacsB11} & \ref{c:KovacsB11}\\
\rowlabel{a:KovacsK11}KovacsK11 \href{https://doi.org/10.1007/s10601-010-9102-3}{KovacsK11} & \hyperref[auth:a147]{A. Kov{\'{a}}cs}, \hyperref[auth:a156]{T. Kis} & Constraint programming approach to a bilevel scheduling problem & \href{../works/KovacsK11.pdf}{Yes} & \cite{KovacsK11} & 2011 & Constraints An Int. J. & 24 & 3 & 24 & \ref{b:KovacsK11} & \ref{c:KovacsK11}\\
\rowlabel{a:LiuW11}LiuW11 \href{http://dx.doi.org/10.1016/j.autcon.2011.04.012}{LiuW11} & \hyperref[auth:a1272]{S. Liu}, \hyperref[auth:a1273]{C. Wang} & Optimizing project selection and scheduling problems with time-dependent resource constraints & No & \cite{LiuW11} & 2011 & Automation in Construction & null & 57 & 35 & No & \ref{c:LiuW11}\\
\rowlabel{a:ReddyFIBKAJ11}ReddyFIBKAJ11 \href{https://doi.org/10.1145/1989734.1989745}{ReddyFIBKAJ11} & \hyperref[auth:a1055]{Sudhakar Y. Reddy}, \hyperref[auth:a384]{J. Frank}, \hyperref[auth:a1056]{M. Iatauro}, \hyperref[auth:a1057]{Matthew E. Boyce}, \hyperref[auth:a385]{E. K{\"{u}}rkl{\"{u}}}, \hyperref[auth:a1058]{M. Ai{-}Chang}, \hyperref[auth:a1059]{Ari K. J{\'{o}}nsson} & Planning solar array operations on the international space station & No & \cite{ReddyFIBKAJ11} & 2011 & {ACM} Trans. Intell. Syst. Technol. & 24 & 3 & 8 & No & \ref{c:ReddyFIBKAJ11}\\
\rowlabel{a:SchausHMCMD11}SchausHMCMD11 \href{https://doi.org/10.1007/s10601-010-9100-5}{SchausHMCMD11} & \hyperref[auth:a148]{P. Schaus}, \hyperref[auth:a149]{Pascal Van Hentenryck}, \hyperref[auth:a150]{J. Monette}, \hyperref[auth:a151]{C. Coffrin}, \hyperref[auth:a32]{L. Michel}, \hyperref[auth:a152]{Y. Deville} & Solving Steel Mill Slab Problems with constraint-based techniques: CP, LNS, and {CBLS} & \href{../works/SchausHMCMD11.pdf}{Yes} & \cite{SchausHMCMD11} & 2011 & Constraints An Int. J. & 23 & 14 & 5 & \ref{b:SchausHMCMD11} & \ref{c:SchausHMCMD11}\\
\rowlabel{a:SchuttFSW11}SchuttFSW11 \href{https://doi.org/10.1007/s10601-010-9103-2}{SchuttFSW11} & \hyperref[auth:a125]{A. Schutt}, \hyperref[auth:a155]{T. Feydy}, \hyperref[auth:a126]{Peter J. Stuckey}, \hyperref[auth:a117]{Mark G. Wallace} & Explaining the cumulative propagator & \href{../works/SchuttFSW11.pdf}{Yes} & \cite{SchuttFSW11} & 2011 & Constraints An Int. J. & 33 & 57 & 23 & \ref{b:SchuttFSW11} & \ref{c:SchuttFSW11}\\
\rowlabel{a:TopalogluO11}TopalogluO11 \href{https://doi.org/10.1016/j.cor.2010.04.018}{TopalogluO11} & \hyperref[auth:a625]{S. Topaloglu}, \hyperref[auth:a353]{I. Ozkarahan} & A constraint programming-based solution approach for medical resident scheduling problems & \href{../works/TopalogluO11.pdf}{Yes} & \cite{TopalogluO11} & 2011 & Computers \  Operations Research & 10 & 46 & 24 & \ref{b:TopalogluO11} & \ref{c:TopalogluO11}\\
\rowlabel{a:TrojetHL11}TrojetHL11 \href{https://doi.org/10.1016/j.cie.2010.08.014}{TrojetHL11} & \hyperref[auth:a713]{M. Trojet}, \hyperref[auth:a714]{F. H'Mida}, \hyperref[auth:a3]{P. Lopez} & Project scheduling under resource constraints: Application of the cumulative global constraint in a decision support framework & \href{../works/TrojetHL11.pdf}{Yes} & \cite{TrojetHL11} & 2011 & Computers \  Industrial Engineering & 7 & 11 & 17 & \ref{b:TrojetHL11} & \ref{c:TrojetHL11}\\
\rowlabel{a:ZeballosNH11}ZeballosNH11 \href{http://dx.doi.org/10.1016/j.compchemeng.2011.01.043}{ZeballosNH11} & \hyperref[auth:a1177]{Luis J. Zeballos}, \hyperref[auth:a529]{Juan M. Novas}, \hyperref[auth:a596]{Gabriela P. Henning} & A CP formulation for scheduling multiproduct multistage batch plants & No & \cite{ZeballosNH11} & 2011 & Computers \  Chemical Engineering & null & 26 & 29 & No & \ref{c:ZeballosNH11}\\
\rowlabel{a:BartakCS10}BartakCS10 \href{https://doi.org/10.1007/s10479-008-0492-1}{BartakCS10} & \hyperref[auth:a153]{R. Bart{\'{a}}k}, \hyperref[auth:a162]{O. Cepek}, \hyperref[auth:a788]{P. Surynek} & Discovering implied constraints in precedence graphs with alternatives & \href{../works/BartakCS10.pdf}{Yes} & \cite{BartakCS10} & 2010 & Annals of Operations Research & 31 & 2 & 9 & \ref{b:BartakCS10} & \ref{c:BartakCS10}\\
\rowlabel{a:BartakSR10}BartakSR10 \href{https://doi.org/10.1017/S0269888910000202}{BartakSR10} & \hyperref[auth:a153]{R. Bart{\'{a}}k}, \hyperref[auth:a154]{Miguel A. Salido}, \hyperref[auth:a318]{F. Rossi} & New trends in constraint satisfaction, planning, and scheduling: a survey & \href{../works/BartakSR10.pdf}{Yes} & \cite{BartakSR10} & 2010 & Knowl. Eng. Rev. & 31 & 28 & 47 & \ref{b:BartakSR10} & \ref{c:BartakSR10}\\
\rowlabel{a:ChenGPSH10}ChenGPSH10 \href{http://dx.doi.org/10.1007/s11465-010-0106-x}{ChenGPSH10} & \hyperref[auth:a923]{Y. Chen}, \hyperref[auth:a924]{Z. Guan}, \hyperref[auth:a925]{Y. Peng}, \hyperref[auth:a926]{X. Shao}, \hyperref[auth:a927]{M. Hasseb} & Technology and system of constraint programming for industry production scheduling — Part I: A brief survey and potential directions & \href{../works/ChenGPSH10.pdf}{Yes} & \cite{ChenGPSH10} & 2010 & Frontiers of Mechanical Engineering in China & 10 & 2 & 32 & \ref{b:ChenGPSH10} & \ref{c:ChenGPSH10}\\
\rowlabel{a:LiuGT10}LiuGT10 \href{http://dx.doi.org/10.3724/sp.j.1004.2010.00603}{LiuGT10} & \hyperref[auth:a1246]{S. Liu}, \hyperref[auth:a1247]{Z. Guo}, \hyperref[auth:a1248]{J. Tang} & Constraint Propagation for Cumulative Scheduling Problems with Precedences: Constraint Propagation for Cumulative Scheduling Problems with Precedences & No & \cite{LiuGT10} & 2010 & Acta Automatica Sinica & null & 2 & 15 & No & \ref{c:LiuGT10}\\
\rowlabel{a:LombardiM10a}LombardiM10a \href{https://doi.org/10.1016/j.artint.2010.02.004}{LombardiM10a} & \hyperref[auth:a143]{M. Lombardi}, \hyperref[auth:a144]{M. Milano} & Allocation and scheduling of Conditional Task Graphs & \href{../works/LombardiM10a.pdf}{Yes} & \cite{LombardiM10a} & 2010 & Artificial Intelligence & 30 & 8 & 24 & \ref{b:LombardiM10a} & \ref{c:LombardiM10a}\\
\rowlabel{a:LombardiMRB10}LombardiMRB10 \href{http://dx.doi.org/10.1007/s10951-010-0184-y}{LombardiMRB10} & \hyperref[auth:a143]{M. Lombardi}, \hyperref[auth:a144]{M. Milano}, \hyperref[auth:a726]{M. Ruggiero}, \hyperref[auth:a247]{L. Benini} & Stochastic allocation and scheduling for conditional task graphs in multi-processor systems-on-chip & \href{../works/LombardiMRB10.pdf}{Yes} & \cite{LombardiMRB10} & 2010 & Journal of Scheduling & 31 & 24 & 41 & \ref{b:LombardiMRB10} & \ref{c:LombardiMRB10}\\
\rowlabel{a:LopesCSM10}LopesCSM10 \href{https://doi.org/10.1007/s10601-009-9086-z}{LopesCSM10} & \hyperref[auth:a157]{Tony Minoru Tamura Lopes}, \hyperref[auth:a158]{Andr{\'{e}} A. Cir{\'{e}}}, \hyperref[auth:a159]{Cid Carvalho de Souza}, \hyperref[auth:a160]{Arnaldo Vieira Moura} & A hybrid model for a multiproduct pipeline planning and scheduling problem & \href{../works/LopesCSM10.pdf}{Yes} & \cite{LopesCSM10} & 2010 & Constraints An Int. J. & 39 & 31 & 18 & \ref{b:LopesCSM10} & \ref{c:LopesCSM10}\\
\rowlabel{a:NovasH10}NovasH10 \href{https://doi.org/10.1016/j.compchemeng.2010.07.011}{NovasH10} & \hyperref[auth:a529]{Juan M. Novas}, \hyperref[auth:a596]{Gabriela P. Henning} & Reactive scheduling framework based on domain knowledge and constraint programming & \href{../works/NovasH10.pdf}{Yes} & \cite{NovasH10} & 2010 & Computers \  Chemical Engineering & 20 & 48 & 19 & \ref{b:NovasH10} & \ref{c:NovasH10}\\
\rowlabel{a:OzturkTHO10}OzturkTHO10 \href{https://www.sciencedirect.com/science/article/pii/S1571065310000107}{OzturkTHO10} & \hyperref[auth:a136]{C. {\"{O}}zt{\"{u}}rk}, \hyperref[auth:a137]{S. Tunali}, \hyperref[auth:a138]{B. Hnich}, \hyperref[auth:a139]{A. {\"{O}}rnek} & Simultaneous Balancing and Scheduling of Flexible Mixed Model Assembly Lines with Sequence-Dependent Setup Times & \href{../works/OzturkTHO10.pdf}{Yes} & \cite{OzturkTHO10} & 2010 & Electronic Notes in Discrete Mathematics & 8 & 15 & 1 & \ref{b:OzturkTHO10} & \ref{c:OzturkTHO10}\\
\rowlabel{a:Zeballos10}Zeballos10 \href{http://dx.doi.org/10.1016/j.rcim.2010.04.005}{Zeballos10} & \hyperref[auth:a1185]{L. Zeballos} & A constraint programming approach to tool allocation and production scheduling in flexible manufacturing systems & No & \cite{Zeballos10} & 2010 & Robotics and Computer-Integrated Manufacturing & null & 41 & 16 & No & \ref{c:Zeballos10}\\
\rowlabel{a:ZeballosCM10}ZeballosCM10 \href{http://dx.doi.org/10.1021/ie1016199}{ZeballosCM10} & \hyperref[auth:a1177]{Luis J. Zeballos}, \hyperref[auth:a900]{Pedro M. Castro}, \hyperref[auth:a1216]{Carlos A. Méndez} & Integrated Constraint Programming Scheduling Approach for Automated Wet-Etch Stations in Semiconductor Manufacturing & No & \cite{ZeballosCM10} & 2010 & Industrial \  Engineering Chemistry Research & null & 22 & 30 & No & \ref{c:ZeballosCM10}\\
\rowlabel{a:ZeballosQH10}ZeballosQH10 \href{https://doi.org/10.1016/j.engappai.2009.07.002}{ZeballosQH10} & \hyperref[auth:a629]{L. Zeballos}, \hyperref[auth:a630]{O. Quiroga}, \hyperref[auth:a596]{Gabriela P. Henning} & A constraint programming model for the scheduling of flexible manufacturing systems with machine and tool limitations & \href{../works/ZeballosQH10.pdf}{Yes} & \cite{ZeballosQH10} & 2010 & Eng. Appl. Artif. Intell. & 20 & 33 & 28 & \ref{b:ZeballosQH10} & \ref{c:ZeballosQH10}\\
\rowlabel{a:abs-1009-0347}abs-1009-0347 \href{http://arxiv.org/abs/1009.0347}{abs-1009-0347} & \hyperref[auth:a125]{A. Schutt}, \hyperref[auth:a155]{T. Feydy}, \hyperref[auth:a126]{Peter J. Stuckey}, \hyperref[auth:a117]{Mark G. Wallace} & Solving the Resource Constrained Project Scheduling Problem with Generalized Precedences by Lazy Clause Generation & \href{../works/abs-1009-0347.pdf}{Yes} & \cite{abs-1009-0347} & 2010 & CoRR & 37 & 0 & 0 & \ref{b:abs-1009-0347} & \ref{c:abs-1009-0347}\\
\rowlabel{a:BidotVLB09}BidotVLB09 \href{https://doi.org/10.1007/s10951-008-0080-x}{BidotVLB09} & \hyperref[auth:a832]{J. Bidot}, \hyperref[auth:a833]{T. Vidal}, \hyperref[auth:a118]{P. Laborie}, \hyperref[auth:a89]{J. Christopher Beck} & A theoretic and practical framework for scheduling in a stochastic environment & \href{../works/BidotVLB09.pdf}{Yes} & \cite{BidotVLB09} & 2009 & Journal of Scheduling & 30 & 58 & 20 & \ref{b:BidotVLB09} & \ref{c:BidotVLB09}\\
\rowlabel{a:BocewiczBB09}BocewiczBB09 \href{https://doi.org/10.1504/IJIIDS.2009.023038}{BocewiczBB09} & \hyperref[auth:a638]{G. Bocewicz}, \hyperref[auth:a639]{I. Bach}, \hyperref[auth:a640]{Zbigniew Antoni Banaszak} & Logic-algebraic method based and constraints programming driven approach to AGVs scheduling & \href{../works/BocewiczBB09.pdf}{Yes} & \cite{BocewiczBB09} & 2009 & Int. J. Intell. Inf. Database Syst. & 19 & 0 & 0 & \ref{b:BocewiczBB09} & \ref{c:BocewiczBB09}\\
\rowlabel{a:CarchraeB09}CarchraeB09 \href{http://dx.doi.org/10.1007/s10852-008-9100-2}{CarchraeB09} & \hyperref[auth:a274]{T. Carchrae}, \hyperref[auth:a89]{J. Christopher Beck} & Principles for the Design of Large Neighborhood Search & \href{../works/CarchraeB09.pdf}{Yes} & \cite{CarchraeB09} & 2009 & Journal of Mathematical Modelling and Algorithms & 26 & 16 & 19 & \ref{b:CarchraeB09} & \ref{c:CarchraeB09}\\
\rowlabel{a:GarridoAO09}GarridoAO09 \href{https://doi.org/10.1007/s10951-008-0083-7}{GarridoAO09} & \hyperref[auth:a641]{A. Garrido}, \hyperref[auth:a642]{M. Arang{\'{u}}}, \hyperref[auth:a643]{E. Onaindia} & A constraint programming formulation for planning: from plan scheduling to plan generation & \href{../works/GarridoAO09.pdf}{Yes} & \cite{GarridoAO09} & 2009 & Journal of Scheduling & 30 & 5 & 14 & \ref{b:GarridoAO09} & \ref{c:GarridoAO09}\\
\rowlabel{a:Jans09}Jans09 \href{http://dx.doi.org/10.1287/ijoc.1080.0283}{Jans09} & \hyperref[auth:a850]{R. Jans} & Solving Lot-Sizing Problems on Parallel Identical Machines Using Symmetry-Breaking Constraints & \href{../works/Jans09.pdf}{Yes} & \cite{Jans09} & 2009 & INFORMS Journal on Computing & 24 & 59 & 73 & \ref{b:Jans09} & \ref{c:Jans09}\\
\rowlabel{a:MilanoW09}MilanoW09 \href{http://dx.doi.org/10.1007/s10479-009-0654-9}{MilanoW09} & \hyperref[auth:a144]{M. Milano}, \hyperref[auth:a117]{Mark G. Wallace} & Integrating Operations Research in Constraint Programming & \href{../works/MilanoW09.pdf}{Yes} & \cite{MilanoW09} & 2009 & Annals of Operations Research & 40 & 34 & 46 & \ref{b:MilanoW09} & \ref{c:MilanoW09}\\
\rowlabel{a:OhrimenkoSC09}OhrimenkoSC09 \href{http://dx.doi.org/10.1007/s10601-008-9064-x}{OhrimenkoSC09} & \hyperref[auth:a870]{O. Ohrimenko}, \hyperref[auth:a126]{Peter J. Stuckey}, \hyperref[auth:a871]{M. Codish} & Propagation via lazy clause generation & \href{../works/OhrimenkoSC09.pdf}{Yes} & \cite{OhrimenkoSC09} & 2009 & Constraints An Int. J. & 35 & 127 & 15 & \ref{b:OhrimenkoSC09} & \ref{c:OhrimenkoSC09}\\
\rowlabel{a:RuggieroBBMA09}RuggieroBBMA09 \href{https://doi.org/10.1109/TCAD.2009.2013536}{RuggieroBBMA09} & \hyperref[auth:a726]{M. Ruggiero}, \hyperref[auth:a380]{D. Bertozzi}, \hyperref[auth:a247]{L. Benini}, \hyperref[auth:a144]{M. Milano}, \hyperref[auth:a727]{A. Andrei} & Reducing the Abstraction and Optimality Gaps in the Allocation and Scheduling for Variable Voltage/Frequency MPSoC Platforms & \href{../works/RuggieroBBMA09.pdf}{Yes} & \cite{RuggieroBBMA09} & 2009 & {IEEE} Trans. Comput. Aided Des. Integr. Circuits Syst. & 14 & 9 & 27 & \ref{b:RuggieroBBMA09} & \ref{c:RuggieroBBMA09}\\
\rowlabel{a:WuBB09}WuBB09 \href{https://doi.org/10.1016/j.cor.2008.08.008}{WuBB09} & \hyperref[auth:a276]{Christine Wei Wu}, \hyperref[auth:a222]{Kenneth N. Brown}, \hyperref[auth:a89]{J. Christopher Beck} & Scheduling with uncertain durations: Modeling beta-robust scheduling with constraints & \href{../works/WuBB09.pdf}{Yes} & \cite{WuBB09} & 2009 & Computers \  Operations Research & 9 & 42 & 5 & \ref{b:WuBB09} & \ref{c:WuBB09}\\
\rowlabel{a:abs-0907-0939}abs-0907-0939 \href{http://arxiv.org/abs/0907.0939}{abs-0907-0939} & \hyperref[auth:a226]{T. Petit}, \hyperref[auth:a363]{E. Poder} & The Soft Cumulative Constraint & \href{../works/abs-0907-0939.pdf}{Yes} & \cite{abs-0907-0939} & 2009 & CoRR & 12 & 0 & 0 & \ref{b:abs-0907-0939} & \ref{c:abs-0907-0939}\\
\rowlabel{a:BartakSR08}BartakSR08 \href{http://dx.doi.org/10.1007/s10845-008-0203-4}{BartakSR08} & \hyperref[auth:a1081]{R. Barták}, \hyperref[auth:a154]{Miguel A. Salido}, \hyperref[auth:a318]{F. Rossi} & Constraint satisfaction techniques in planning and scheduling & No & \cite{BartakSR08} & 2008 & Journal of Intelligent Manufacturing & null & 54 & 21 & No & \ref{c:BartakSR08}\\
\rowlabel{a:ClautiauxJCM08}ClautiauxJCM08 \href{http://dx.doi.org/10.1016/j.cor.2006.05.012}{ClautiauxJCM08} & \hyperref[auth:a1193]{F. Clautiaux}, \hyperref[auth:a940]{A. Jouglet}, \hyperref[auth:a854]{J. Carlier}, \hyperref[auth:a1194]{A. Moukrim} & A new constraint programming approach for the orthogonal packing problem & No & \cite{ClautiauxJCM08} & 2008 & Computers  \  Operations Research & null & 64 & 14 & No & \ref{c:ClautiauxJCM08}\\
\rowlabel{a:GarridoOS08}GarridoOS08 \href{https://doi.org/10.1016/j.engappai.2008.03.009}{GarridoOS08} & \hyperref[auth:a641]{A. Garrido}, \hyperref[auth:a643]{E. Onaindia}, \hyperref[auth:a648]{{\'{O}}scar Sapena} & Planning and scheduling in an e-learning environment. {A} constraint-programming-based approach & \href{../works/GarridoOS08.pdf}{Yes} & \cite{GarridoOS08} & 2008 & Eng. Appl. Artif. Intell. & 11 & 22 & 7 & \ref{b:GarridoOS08} & \ref{c:GarridoOS08}\\
\rowlabel{a:HladikCDJ08}HladikCDJ08 \href{http://dx.doi.org/10.1016/j.jss.2007.02.032}{HladikCDJ08} & \hyperref[auth:a1182]{P. Hladik}, \hyperref[auth:a1013]{H. Cambazard}, \hyperref[auth:a1183]{A. Déplanche}, \hyperref[auth:a249]{N. Jussien} & Solving a real-time allocation problem with constraint programming & No & \cite{HladikCDJ08} & 2008 & Journal of Systems and Software & null & 36 & 27 & No & \ref{c:HladikCDJ08}\\
\rowlabel{a:KovacsB08}KovacsB08 \href{https://doi.org/10.1016/j.engappai.2008.03.004}{KovacsB08} & \hyperref[auth:a147]{A. Kov{\'{a}}cs}, \hyperref[auth:a89]{J. Christopher Beck} & A global constraint for total weighted completion time for cumulative resources & \href{../works/KovacsB08.pdf}{Yes} & \cite{KovacsB08} & 2008 & Eng. Appl. Artif. Intell. & 7 & 5 & 14 & \ref{b:KovacsB08} & \ref{c:KovacsB08}\\
\rowlabel{a:LiW08}LiW08 \href{http://dx.doi.org/10.1007/s10951-008-0079-3}{LiW08} & \hyperref[auth:a965]{H. Li}, \hyperref[auth:a966]{K. Womer} & Scheduling projects with multi-skilled personnel by a hybrid MILP/CP benders decomposition algorithm & \href{../works/LiW08.pdf}{Yes} & \cite{LiW08} & 2008 & Journal of Scheduling & 18 & 113 & 31 & \ref{b:LiW08} & \ref{c:LiW08}\\
\rowlabel{a:LiessM08}LiessM08 \href{https://doi.org/10.1007/s10479-007-0188-y}{LiessM08} & \hyperref[auth:a647]{O. Liess}, \hyperref[auth:a360]{P. Michelon} & A constraint programming approach for the resource-constrained project scheduling problem & \href{../works/LiessM08.pdf}{Yes} & \cite{LiessM08} & 2008 & Annals of Operations Research & 12 & 22 & 14 & \ref{b:LiessM08} & \ref{c:LiessM08}\\
\rowlabel{a:MalikMB08}MalikMB08 \href{https://doi.org/10.1142/S0218213008003765}{MalikMB08} & \hyperref[auth:a646]{Abid M. Malik}, \hyperref[auth:a649]{J. McInnes}, \hyperref[auth:a618]{Peter van Beek} & Optimal Basic Block Instruction Scheduling for Multiple-Issue Processors Using Constraint Programming & \href{../works/MalikMB08.pdf}{Yes} & \cite{MalikMB08} & 2008 & Int. J. Artif. Intell. Tools & 18 & 15 & 8 & \ref{b:MalikMB08} & \ref{c:MalikMB08}\\
\rowlabel{a:MercierH08}MercierH08 \href{http://dx.doi.org/10.1287/ijoc.1070.0226}{MercierH08} & \hyperref[auth:a860]{L. Mercier}, \hyperref[auth:a149]{Pascal Van Hentenryck} & Edge Finding for Cumulative Scheduling & \href{../works/MercierH08.pdf}{Yes} & \cite{MercierH08} & 2008 & INFORMS Journal on Computing & 21 & 32 & 5 & \ref{b:MercierH08} & \ref{c:MercierH08}\\
\rowlabel{a:ArtiguesF07}ArtiguesF07 \href{http://dx.doi.org/10.1007/s10479-007-0283-0}{ArtiguesF07} & \hyperref[auth:a6]{C. Artigues}, \hyperref[auth:a361]{D. Feillet} & A branch and bound method for the job-shop problem with sequence-dependent setup times & \href{../works/ArtiguesF07.pdf}{Yes} & \cite{ArtiguesF07} & 2007 & Annals of Operations Research & 25 & 49 & 32 & \ref{b:ArtiguesF07} & \ref{c:ArtiguesF07}\\
\rowlabel{a:Beck07}Beck07 \href{https://doi.org/10.1613/jair.2169}{Beck07} & \hyperref[auth:a89]{J. Christopher Beck} & Solution-Guided Multi-Point Constructive Search for Job Shop Scheduling & \href{../works/Beck07.pdf}{Yes} & \cite{Beck07} & 2007 & J. Artif. Intell. Res. & 29 & 34 & 0 & \ref{b:Beck07} & \ref{c:Beck07}\\
\rowlabel{a:BeckW07}BeckW07 \href{https://doi.org/10.1613/jair.2080}{BeckW07} & \hyperref[auth:a89]{J. Christopher Beck}, \hyperref[auth:a834]{N. Wilson} & Proactive Algorithms for Job Shop Scheduling with Probabilistic Durations & \href{../works/BeckW07.pdf}{Yes} & \cite{BeckW07} & 2007 & J. Artif. Intell. Res. & 50 & 27 & 0 & \ref{b:BeckW07} & \ref{c:BeckW07}\\
\rowlabel{a:CorreaLR07}CorreaLR07 \href{http://dx.doi.org/10.1016/j.cor.2005.07.004}{CorreaLR07} & \hyperref[auth:a961]{Ayoub Insa Corr{\'{e}}a}, \hyperref[auth:a653]{A. Langevin}, \hyperref[auth:a331]{L. Rousseau} & Scheduling and routing of automated guided vehicles: A hybrid approach & \href{../works/CorreaLR07.pdf}{Yes} & \cite{CorreaLR07} & 2007 & Computers \  Operations Research & 20 & 106 & 20 & \ref{b:CorreaLR07} & \ref{c:CorreaLR07}\\
\rowlabel{a:Hooker07}Hooker07 \href{http://dx.doi.org/10.1287/opre.1060.0371}{Hooker07} & \hyperref[auth:a161]{John N. Hooker} & Planning and Scheduling by Logic-Based Benders Decomposition & \href{../works/Hooker07.pdf}{Yes} & \cite{Hooker07} & 2007 & Operations Research & 29 & 181 & 19 & \ref{b:Hooker07} & \ref{c:Hooker07}\\
\rowlabel{a:MercierH07}MercierH07 \href{http://dx.doi.org/10.1016/j.disopt.2007.01.001}{MercierH07} & \hyperref[auth:a860]{L. Mercier}, \hyperref[auth:a149]{Pascal Van Hentenryck} & Strong polynomiality of resource constraint propagation & No & \cite{MercierH07} & 2007 & Discrete Optimization & null & 5 & 8 & No & \ref{c:MercierH07}\\
\rowlabel{a:Rodriguez07}Rodriguez07 \href{https://www.sciencedirect.com/science/article/pii/S0191261506000233}{Rodriguez07} & \hyperref[auth:a789]{J. Rodriguez} & A constraint programming model for real-time train scheduling at junctions & \href{../works/Rodriguez07.pdf}{Yes} & \cite{Rodriguez07} & 2007 & Transportation Research Part B: Methodological & 15 & 117 & 6 & \ref{b:Rodriguez07} & \ref{c:Rodriguez07}\\
\rowlabel{a:Simonis07}Simonis07 \href{https://doi.org/10.1007/s10601-006-9011-7}{Simonis07} & \hyperref[auth:a17]{H. Simonis} & Models for Global Constraint Applications & \href{../works/Simonis07.pdf}{Yes} & \cite{Simonis07} & 2007 & Constraints An Int. J. & 30 & 10 & 17 & \ref{b:Simonis07} & \ref{c:Simonis07}\\
\rowlabel{a:BockmayrP06}BockmayrP06 \href{http://dx.doi.org/10.1016/j.cor.2005.01.010}{BockmayrP06} & \hyperref[auth:a918]{A. Bockmayr}, \hyperref[auth:a1202]{N. Pisaruk} & Detecting infeasibility and generating cuts for mixed integer programming using constraint programming & No & \cite{BockmayrP06} & 2006 & Computers \  Operations Research & null & 12 & 7 & No & \ref{c:BockmayrP06}\\
\rowlabel{a:Gronkvist06}Gronkvist06 \href{http://dx.doi.org/10.1016/j.cor.2005.01.017}{Gronkvist06} & \hyperref[auth:a1240]{M. Gr\"{o}nkvist} & Accelerating column generation for aircraft scheduling using constraint propagation & No & \cite{Gronkvist06} & 2006 & Computers \  Operations Research & null & 28 & 15 & No & \ref{c:Gronkvist06}\\
\rowlabel{a:Hooker06}Hooker06 \href{https://doi.org/10.1007/s10601-006-8060-2}{Hooker06} & \hyperref[auth:a161]{John N. Hooker} & An Integrated Method for Planning and Scheduling to Minimize Tardiness & \href{../works/Hooker06.pdf}{Yes} & \cite{Hooker06} & 2006 & Constraints An Int. J. & 19 & 19 & 13 & \ref{b:Hooker06} & \ref{c:Hooker06}\\
\rowlabel{a:KhayatLR06}KhayatLR06 \href{https://doi.org/10.1016/j.ejor.2005.02.077}{KhayatLR06} & \hyperref[auth:a652]{Ghada El Khayat}, \hyperref[auth:a653]{A. Langevin}, \hyperref[auth:a654]{D. Riopel} & Integrated production and material handling scheduling using mathematical programming and constraint programming & \href{../works/KhayatLR06.pdf}{Yes} & \cite{KhayatLR06} & 2006 & European Journal of Operational Research & 15 & 84 & 14 & \ref{b:KhayatLR06} & \ref{c:KhayatLR06}\\
\rowlabel{a:MilanoW06}MilanoW06 \href{http://dx.doi.org/10.1007/s10288-006-0019-z}{MilanoW06} & \hyperref[auth:a144]{M. Milano}, \hyperref[auth:a117]{Mark G. Wallace} & Integrating operations research in constraint programming & \href{../works/MilanoW06.pdf}{Yes} & \cite{MilanoW06} & 2006 & 4OR & 45 & 18 & 46 & \ref{b:MilanoW06} & \ref{c:MilanoW06}\\
\rowlabel{a:SadykovW06}SadykovW06 \href{https://doi.org/10.1287/ijoc.1040.0110}{SadykovW06} & \hyperref[auth:a389]{R. Sadykov}, \hyperref[auth:a229]{Laurence A. Wolsey} & Integer Programming and Constraint Programming in Solving a Multimachine Assignment Scheduling Problem with Deadlines and Release Dates & \href{../works/SadykovW06.pdf}{Yes} & \cite{SadykovW06} & 2006 & INFORMS Journal on Computing & 9 & 45 & 6 & \ref{b:SadykovW06} & \ref{c:SadykovW06}\\
\rowlabel{a:SureshMOK06}SureshMOK06 \href{https://doi.org/10.1080/17445760600567842}{SureshMOK06} & \hyperref[auth:a655]{S. Sundaram}, \hyperref[auth:a656]{V. Mani}, \hyperref[auth:a657]{S. N. Omkar}, \hyperref[auth:a658]{H. J. Kim} & Divisible load scheduling in distributed system with buffer constraints: genetic algorithm and linear programming approach & \href{../works/SureshMOK06.pdf}{Yes} & \cite{SureshMOK06} & 2006 & Int. J. Parallel Emergent Distributed Syst. & 19 & 12 & 23 & \ref{b:SureshMOK06} & \ref{c:SureshMOK06}\\
\rowlabel{a:DemasseyAM05}DemasseyAM05 \href{http://dx.doi.org/10.1287/ijoc.1030.0043}{DemasseyAM05} & \hyperref[auth:a245]{S. Demassey}, \hyperref[auth:a6]{C. Artigues}, \hyperref[auth:a360]{P. Michelon} & Constraint-Propagation-Based Cutting Planes: An Application to the Resource-Constrained Project Scheduling Problem & \href{../works/DemasseyAM05.pdf}{Yes} & \cite{DemasseyAM05} & 2005 & INFORMS Journal on Computing & 18 & 43 & 25 & \ref{b:DemasseyAM05} & \ref{c:DemasseyAM05}\\
\rowlabel{a:Hooker05}Hooker05 \href{https://doi.org/10.1007/s10601-005-2812-2}{Hooker05} & \hyperref[auth:a161]{John N. Hooker} & A Hybrid Method for the Planning and Scheduling & \href{../works/Hooker05.pdf}{Yes} & \cite{Hooker05} & 2005 & Constraints An Int. J. & 17 & 68 & 11 & \ref{b:Hooker05} & \ref{c:Hooker05}\\
\rowlabel{a:RoePS05}RoePS05 \href{http://dx.doi.org/10.1016/j.compchemeng.2005.02.024}{RoePS05} & \hyperref[auth:a1269]{B. Roe}, \hyperref[auth:a1270]{Lazaros G. Papageorgiou}, \hyperref[auth:a1271]{N. Shah} & A hybrid MILP/CLP algorithm for multipurpose batch process scheduling & No & \cite{RoePS05} & 2005 & Computers \  Chemical Engineering & null & 48 & 15 & No & \ref{c:RoePS05}\\
\rowlabel{a:VilimBC05}VilimBC05 \href{https://doi.org/10.1007/s10601-005-2814-0}{VilimBC05} & \hyperref[auth:a121]{P. Vil{\'{\i}}m}, \hyperref[auth:a153]{R. Bart{\'{a}}k}, \hyperref[auth:a162]{O. Cepek} & Extension of \emph{O}(\emph{n} log \emph{n}) Filtering Algorithms for the Unary Resource Constraint to Optional Activities & \href{../works/VilimBC05.pdf}{Yes} & \cite{VilimBC05} & 2005 & Constraints An Int. J. & 23 & 21 & 5 & \ref{b:VilimBC05} & \ref{c:VilimBC05}\\
\rowlabel{a:ZeballosH05}ZeballosH05 \href{http://journal.iberamia.org/index.php/ia/article/view/452/article\%20\%281\%29.pdf}{ZeballosH05} & \hyperref[auth:a629]{L. Zeballos}, \hyperref[auth:a596]{Gabriela P. Henning} & A Constraint Programming Approach to {FMS} Scheduling. Consideration of Storage and Transportation Resources & \href{../works/ZeballosH05.pdf}{Yes} & \cite{ZeballosH05} & 2005 & Inteligencia Artif. & 10 & 0 & 0 & \ref{b:ZeballosH05} & \ref{c:ZeballosH05}\\
\rowlabel{a:MaraveliasCG04}MaraveliasCG04 \href{http://dx.doi.org/10.1016/j.compchemeng.2004.03.016}{MaraveliasCG04} & \hyperref[auth:a386]{Christos T. Maravelias}, \hyperref[auth:a387]{Ignacio E. Grossmann} & A hybrid MILP/CP decomposition approach for the continuous time scheduling of multipurpose batch plants & No & \cite{MaraveliasCG04} & 2004 & Computers \  Chemical Engineering & null & 116 & 24 & No & \ref{c:MaraveliasCG04}\\
\rowlabel{a:PoderBS04}PoderBS04 \href{https://doi.org/10.1016/S0377-2217(02)00756-7}{PoderBS04} & \hyperref[auth:a363]{E. Poder}, \hyperref[auth:a129]{N. Beldiceanu}, \hyperref[auth:a721]{E. Sanlaville} & Computing a lower approximation of the compulsory part of a task with varying duration and varying resource consumption & \href{../works/PoderBS04.pdf}{Yes} & \cite{PoderBS04} & 2004 & European Journal of Operational Research & 16 & 7 & 8 & \ref{b:PoderBS04} & \ref{c:PoderBS04}\\
\rowlabel{a:BeckR03}BeckR03 \href{https://doi.org/10.1023/A:1021849405707}{BeckR03} & \hyperref[auth:a89]{J. Christopher Beck}, \hyperref[auth:a256]{P. Refalo} & A Hybrid Approach to Scheduling with Earliness and Tardiness Costs & \href{../works/BeckR03.pdf}{Yes} & \cite{BeckR03} & 2003 & Annals of Operations Research & 23 & 29 & 0 & \ref{b:BeckR03} & \ref{c:BeckR03}\\
\rowlabel{a:HookerO03}HookerO03 \href{http://dx.doi.org/10.1007/s10107-003-0375-9}{HookerO03} & \hyperref[auth:a161]{John N. Hooker}, \hyperref[auth:a861]{G. Ottosson} & Logic-based Benders decomposition & \href{../works/HookerO03.pdf}{Yes} & \cite{HookerO03} & 2003 & Mathematical Programming & 28 & 317 & 0 & \ref{b:HookerO03} & \ref{c:HookerO03}\\
\rowlabel{a:Kuchcinski03}Kuchcinski03 \href{http://dx.doi.org/10.1145/785411.785416}{Kuchcinski03} & \hyperref[auth:a668]{K. Kuchcinski} & Constraints-driven scheduling and resource assignment & No & \cite{Kuchcinski03} & 2003 & ACM Transactions on Design Automation of Electronic Systems & null & 105 & 15 & No & \ref{c:Kuchcinski03}\\
\rowlabel{a:KuchcinskiW03}KuchcinskiW03 \href{https://doi.org/10.1016/S1383-7621(03)00075-4}{KuchcinskiW03} & \hyperref[auth:a668]{K. Kuchcinski}, \hyperref[auth:a667]{C. Wolinski} & Global approach to assignment and scheduling of complex behaviors based on {HCDG} and constraint programming & \href{../works/KuchcinskiW03.pdf}{Yes} & \cite{KuchcinskiW03} & 2003 & J. Syst. Archit. & 15 & 19 & 18 & \ref{b:KuchcinskiW03} & \ref{c:KuchcinskiW03}\\
\rowlabel{a:Laborie03}Laborie03 \href{http://dx.doi.org/10.1016/s0004-3702(02)00362-4}{Laborie03} & \hyperref[auth:a118]{P. Laborie} & Algorithms for propagating resource constraints in AI planning and scheduling: Existing approaches and new results & \href{../works/Laborie03.pdf}{Yes} & \cite{Laborie03} & 2003 & Artificial Intelligence & 38 & 128 & 10 & \ref{b:Laborie03} & \ref{c:Laborie03}\\
\rowlabel{a:Tsang03}Tsang03 \href{https://doi.org/10.1023/A:1024016929283}{Tsang03} & \hyperref[auth:a673]{Edward P. K. Tsang} & Constraint Based Scheduling: Applying Constraint Programming to Scheduling Problems & \href{../works/Tsang03.pdf}{Yes} & \cite{Tsang03} & 2003 & Journal of Scheduling & 2 & 1 & 0 & \ref{b:Tsang03} & \ref{c:Tsang03}\\
\rowlabel{a:HarjunkoskiG02}HarjunkoskiG02 \href{http://dx.doi.org/10.1016/s0098-1354(02)00100-x}{HarjunkoskiG02} & \hyperref[auth:a880]{I. Harjunkoski}, \hyperref[auth:a387]{Ignacio E. Grossmann} & Decomposition techniques for multistage scheduling problems using mixed-integer and constraint programming methods & \href{../works/HarjunkoskiG02.pdf}{Yes} & \cite{HarjunkoskiG02} & 2002 & Computers \  Chemical Engineering & 20 & 169 & 11 & \ref{b:HarjunkoskiG02} & \ref{c:HarjunkoskiG02}\\
\rowlabel{a:Hooker02}Hooker02 \href{http://dx.doi.org/10.1287/ijoc.14.4.295.2828}{Hooker02} & \hyperref[auth:a161]{John N. Hooker} & Logic,  Optimization,  and Constraint Programming & No & \cite{Hooker02} & 2002 & INFORMS Journal on Computing & null & 94 & 84 & No & \ref{c:Hooker02}\\
\rowlabel{a:JussienL02}JussienL02 \href{http://dx.doi.org/10.1016/s0004-3702(02)00221-7}{JussienL02} & \hyperref[auth:a249]{N. Jussien}, \hyperref[auth:a1091]{O. Lhomme} & Local search with constraint propagation and conflict-based heuristics & \href{../works/JussienL02.pdf}{Yes} & \cite{JussienL02} & 2002 & Artificial Intelligence & 25 & 88 & 16 & \ref{b:JussienL02} & \ref{c:JussienL02}\\
\rowlabel{a:LorigeonBB02}LorigeonBB02 \href{https://doi.org/10.1057/palgrave.jors.2601421}{LorigeonBB02} & \hyperref[auth:a679]{T. Lorigeon}, \hyperref[auth:a342]{J. Billaut}, \hyperref[auth:a680]{J. Bouquard} & A dynamic programming algorithm for scheduling jobs in a two-machine open shop with an availability constraint & \href{../works/LorigeonBB02.pdf}{Yes} & \cite{LorigeonBB02} & 2002 & Journal of the Operational Research Society & 8 & 22 & 0 & \ref{b:LorigeonBB02} & \ref{c:LorigeonBB02}\\
\rowlabel{a:MilanoORT02}MilanoORT02 \href{http://dx.doi.org/10.1287/ijoc.14.4.387.2830}{MilanoORT02} & \hyperref[auth:a144]{M. Milano}, \hyperref[auth:a861]{G. Ottosson}, \hyperref[auth:a256]{P. Refalo}, \hyperref[auth:a883]{Erlendur S. Thorsteinsson} & The Role of Integer Programming Techniques in Constraint Programming's Global Constraints & No & \cite{MilanoORT02} & 2002 & INFORMS Journal on Computing & null & 14 & 31 & No & \ref{c:MilanoORT02}\\
\rowlabel{a:RodriguezDG02}RodriguezDG02 \href{}{RodriguezDG02} & \hyperref[auth:a789]{J. Rodriguez}, \hyperref[auth:a790]{X. Delorme}, \hyperref[auth:a791]{X. Gandibleux} & Railway infrastructure saturation using constraint programming approach & \href{../works/RodriguezDG02.pdf}{Yes} & \cite{RodriguezDG02} & 2002 & Computers in Railways VIII & 10 & 0 & 0 & \ref{b:RodriguezDG02} & \ref{c:RodriguezDG02}\\
\rowlabel{a:Timpe02}Timpe02 \href{https://doi.org/10.1007/s00291-002-0107-1}{Timpe02} & \hyperref[auth:a681]{C. Timpe} & Solving planning and scheduling problems with combined integer and constraint programming & \href{../works/Timpe02.pdf}{Yes} & \cite{Timpe02} & 2002 & {OR} Spectr. & 18 & 42 & 0 & \ref{b:Timpe02} & \ref{c:Timpe02}\\
\rowlabel{a:BosiM2001}BosiM2001 \href{http://dx.doi.org/10.1002/1097-024x(200101)31:1<17::aid-spe355>3.0.co;2-l}{BosiM2001} & \hyperref[auth:a1250]{F. Bosi}, \hyperref[auth:a144]{M. Milano} & Enhancing CLP branch and bound techniques for scheduling problems & No & \cite{BosiM2001} & 2001 & Software: Practice and Experience & null & 3 & 12 & No & \ref{c:BosiM2001}\\
\rowlabel{a:JainG01}JainG01 \href{http://dx.doi.org/10.1287/ijoc.13.4.258.9733}{JainG01} & \hyperref[auth:a853]{V. Jain}, \hyperref[auth:a387]{Ignacio E. Grossmann} & Algorithms for Hybrid MILP/CP Models for a Class of Optimization Problems & \href{../works/JainG01.pdf}{Yes} & \cite{JainG01} & 2001 & INFORMS Journal on Computing & 19 & 279 & 23 & \ref{b:JainG01} & \ref{c:JainG01}\\
\rowlabel{a:MartinPY01}MartinPY01 \href{https://doi.org/10.1023/A:1016067230126}{MartinPY01} & \hyperref[auth:a684]{F. Martin}, \hyperref[auth:a685]{A. Pinkney}, \hyperref[auth:a686]{X. Yu} & Cane Railway Scheduling via Constraint Logic Programming: Labelling Order and Constraints in a Real-Life Application & \href{../works/MartinPY01.pdf}{Yes} & \cite{MartinPY01} & 2001 & Annals of Operations Research & 17 & 11 & 0 & \ref{b:MartinPY01} & \ref{c:MartinPY01}\\
\rowlabel{a:Mason01}Mason01 \href{https://doi.org/10.1023/A:1016023415105}{Mason01} & \hyperref[auth:a687]{Andrew J. Mason} & Elastic Constraint Branching, the Wedelin/Carmen Lagrangian Heuristic and Integer Programming for Personnel Scheduling & \href{../works/Mason01.pdf}{Yes} & \cite{Mason01} & 2001 & Annals of Operations Research & 38 & 5 & 0 & \ref{b:Mason01} & \ref{c:Mason01}\\
\rowlabel{a:ArtiguesR00}ArtiguesR00 \href{https://doi.org/10.1016/S0377-2217(99)00496-8}{ArtiguesR00} & \hyperref[auth:a6]{C. Artigues}, \hyperref[auth:a720]{F. Roubellat} & A polynomial activity insertion algorithm in a multi-resource schedule with cumulative constraints and multiple modes & \href{../works/ArtiguesR00.pdf}{Yes} & \cite{ArtiguesR00} & 2000 & European Journal of Operational Research & 20 & 84 & 3 & \ref{b:ArtiguesR00} & \ref{c:ArtiguesR00}\\
\rowlabel{a:BaptisteP00}BaptisteP00 \href{https://doi.org/10.1023/A:1009822502231}{BaptisteP00} & \hyperref[auth:a163]{P. Baptiste}, \hyperref[auth:a164]{Claude Le Pape} & Constraint Propagation and Decomposition Techniques for Highly Disjunctive and Highly Cumulative Project Scheduling Problems & \href{../works/BaptisteP00.pdf}{Yes} & \cite{BaptisteP00} & 2000 & Constraints An Int. J. & 21 & 46 & 0 & \ref{b:BaptisteP00} & \ref{c:BaptisteP00}\\
\rowlabel{a:BeckF00}BeckF00 \href{https://doi.org/10.1016/S0004-3702(99)00099-5}{BeckF00} & \hyperref[auth:a89]{J. Christopher Beck}, \hyperref[auth:a304]{Mark S. Fox} & Dynamic problem structure analysis as a basis for constraint-directed scheduling heuristics & \href{../works/BeckF00.pdf}{Yes} & \cite{BeckF00} & 2000 & Artificial Intelligence & 51 & 24 & 19 & \ref{b:BeckF00} & \ref{c:BeckF00}\\
\rowlabel{a:BeckF00a}BeckF00a \href{http://dx.doi.org/10.1016/s0004-3702(00)00035-7}{BeckF00a} & \hyperref[auth:a89]{J. Christopher Beck}, \hyperref[auth:a304]{Mark S. Fox} & Constraint-directed techniques for scheduling alternative activities & No & \cite{BeckF00a} & 2000 & Artificial Intelligence & null & 48 & 10 & No & \ref{c:BeckF00a}\\
\rowlabel{a:BruckerK00}BruckerK00 \href{http://dx.doi.org/10.1016/s0377-2217(99)00489-0}{BruckerK00} & \hyperref[auth:a856]{P. Brucker}, \hyperref[auth:a1190]{S. Knust} & A linear programming and constraint propagation-based lower bound for the RCPSP & No & \cite{BruckerK00} & 2000 & European Journal of Operational Research & null & 66 & 8 & No & \ref{c:BruckerK00}\\
\rowlabel{a:Dorndorf2000}Dorndorf2000 \href{http://dx.doi.org/10.1016/s0004-3702(00)00040-0}{Dorndorf2000} & \hyperref[auth:a913]{U. Dorndorf}, \hyperref[auth:a443]{E. Pesch}, \hyperref[auth:a1064]{T. Phan-Huy} & Constraint propagation techniques for the disjunctive scheduling problem & No & \cite{Dorndorf2000} & 2000 & Artificial Intelligence & null & 47 & 33 & No & \ref{c:Dorndorf2000}\\
\rowlabel{a:HarjunkoskiJG00}HarjunkoskiJG00 \href{http://dx.doi.org/10.1016/s0098-1354(00)00470-1}{HarjunkoskiJG00} & \hyperref[auth:a880]{I. Harjunkoski}, \hyperref[auth:a853]{V. Jain}, \hyperref[auth:a1181]{Ignacio E. Grossman} & Hybrid mixed-integer/constraint logic programming strategies for solving scheduling and combinatorial optimization problems & No & \cite{HarjunkoskiJG00} & 2000 & Computers \  Chemical Engineering & null & 44 & 3 & No & \ref{c:HarjunkoskiJG00}\\
\rowlabel{a:HeipckeCCS00}HeipckeCCS00 \href{https://doi.org/10.1023/A:1009860311452}{HeipckeCCS00} & \hyperref[auth:a168]{S. Heipcke}, \hyperref[auth:a169]{Y. Colombani}, \hyperref[auth:a170]{Cristina C. B. Cavalcante}, \hyperref[auth:a171]{Cid C. de Souza} & Scheduling under Labour Resource Constraints & \href{../works/HeipckeCCS00.pdf}{Yes} & \cite{HeipckeCCS00} & 2000 & Constraints An Int. J. & 8 & 5 & 0 & \ref{b:HeipckeCCS00} & \ref{c:HeipckeCCS00}\\
\rowlabel{a:HookerOTK00}HookerOTK00 \href{http://dx.doi.org/10.1017/s0269888900001077}{HookerOTK00} & \hyperref[auth:a1212]{J. HOOKER}, \hyperref[auth:a1213]{G. OTTOSSON}, \hyperref[auth:a1214]{ERLENDER S. THORSTEINSSON}, \hyperref[auth:a1215]{H. KIM} & A scheme for unifying optimization and constraint satisfaction methods & No & \cite{HookerOTK00} & 2000 & The Knowledge Engineering Review & null & 30 & 0 & No & \ref{c:HookerOTK00}\\
\rowlabel{a:KorbaaYG00}KorbaaYG00 \href{https://doi.org/10.1016/S0947-3580(00)71113-7}{KorbaaYG00} & \hyperref[auth:a688]{O. Korbaa}, \hyperref[auth:a689]{P. Yim}, \hyperref[auth:a690]{J. Gentina} & Solving Transient Scheduling Problems with Constraint Programming & \href{../works/KorbaaYG00.pdf}{Yes} & \cite{KorbaaYG00} & 2000 & Eur. J. Control & 10 & 7 & 4 & \ref{b:KorbaaYG00} & \ref{c:KorbaaYG00}\\
\rowlabel{a:LopezAKYG00}LopezAKYG00 \href{https://doi.org/10.1016/S0947-3580(00)71114-9}{LopezAKYG00} & \hyperref[auth:a3]{P. Lopez}, \hyperref[auth:a691]{H. Alla}, \hyperref[auth:a688]{O. Korbaa}, \hyperref[auth:a689]{P. Yim}, \hyperref[auth:a690]{J. Gentina} & Discussion on: 'Solving Transient Scheduling Problems with Constraint Programming' by O. Korbaa, P. Yim, and {J.-C.} Gentina & \href{../works/LopezAKYG00.pdf}{Yes} & \cite{LopezAKYG00} & 2000 & Eur. J. Control & 4 & 0 & 0 & \ref{b:LopezAKYG00} & \ref{c:LopezAKYG00}\\
\rowlabel{a:SakkoutW00}SakkoutW00 \href{https://doi.org/10.1023/A:1009856210543}{SakkoutW00} & \hyperref[auth:a167]{Hani El Sakkout}, \hyperref[auth:a117]{Mark G. Wallace} & Probe Backtrack Search for Minimal Perturbation in Dynamic Scheduling & \href{../works/SakkoutW00.pdf}{Yes} & \cite{SakkoutW00} & 2000 & Constraints An Int. J. & 30 & 73 & 0 & \ref{b:SakkoutW00} & \ref{c:SakkoutW00}\\
\rowlabel{a:SchildW00}SchildW00 \href{https://doi.org/10.1023/A:1009804226473}{SchildW00} & \hyperref[auth:a165]{K. Schild}, \hyperref[auth:a166]{J. W{\"{u}}rtz} & Scheduling of Time-Triggered Real-Time Systems & \href{../works/SchildW00.pdf}{Yes} & \cite{SchildW00} & 2000 & Constraints An Int. J. & 23 & 23 & 0 & \ref{b:SchildW00} & \ref{c:SchildW00}\\
\rowlabel{a:SimonisCK00}SimonisCK00 \href{https://doi.org/10.1109/5254.820326}{SimonisCK00} & \hyperref[auth:a17]{H. Simonis}, \hyperref[auth:a895]{P. Charlier}, \hyperref[auth:a896]{P. Kay} & Constraint Handling in an Integrated Transportation Problem & \href{../works/SimonisCK00.pdf}{Yes} & \cite{SimonisCK00} & 2000 & {IEEE} Intell. Syst. & 7 & 11 & 5 & \ref{b:SimonisCK00} & \ref{c:SimonisCK00}\\
\rowlabel{a:SourdN00}SourdN00 \href{https://doi.org/10.1287/ijoc.12.4.341.11881}{SourdN00} & \hyperref[auth:a783]{F. Sourd}, \hyperref[auth:a664]{W. Nuijten} & Multiple-Machine Lower Bounds for Shop-Scheduling Problems & \href{../works/SourdN00.pdf}{Yes} & \cite{SourdN00} & 2000 & INFORMS Journal on Computing & 12 & 7 & 14 & \ref{b:SourdN00} & \ref{c:SourdN00}\\
\rowlabel{a:TorresL00}TorresL00 \href{http://dx.doi.org/10.1016/s0377-2217(99)00497-x}{TorresL00} & \hyperref[auth:a882]{P. Torres}, \hyperref[auth:a3]{P. Lopez} & On Not-First/Not-Last conditions in disjunctive scheduling & \href{../works/TorresL00.pdf}{Yes} & \cite{TorresL00} & 2000 & European Journal of Operational Research & 12 & 26 & 13 & \ref{b:TorresL00} & \ref{c:TorresL00}\\
\rowlabel{a:BaptistePN99}BaptistePN99 \href{http://dx.doi.org/10.1023/a:1018995000688}{BaptistePN99} & \hyperref[auth:a163]{P. Baptiste}, \hyperref[auth:a164]{Claude Le Pape}, \hyperref[auth:a664]{W. Nuijten} & Satisfiability tests and time-bound adjustments for cumulative scheduling problems & \href{../works/BaptistePN99.pdf}{Yes} & \cite{BaptistePN99} & 1999 & Annals of Operations Research & 29 & 72 & 0 & \ref{b:BaptistePN99} & \ref{c:BaptistePN99}\\
\rowlabel{a:BensanaLV99}BensanaLV99 \href{https://doi.org/10.1023/A:1026488509554}{BensanaLV99} & \hyperref[auth:a172]{E. Bensana}, \hyperref[auth:a173]{M. Lema{\^{\i}}tre}, \hyperref[auth:a174]{G. Verfaillie} & Earth Observation Satellite Management & \href{../works/BensanaLV99.pdf}{Yes} & \cite{BensanaLV99} & 1999 & Constraints An Int. J. & 7 & 99 & 0 & \ref{b:BensanaLV99} & \ref{c:BensanaLV99}\\
\rowlabel{a:HookerO99}HookerO99 \href{http://dx.doi.org/10.1016/s0166-218x(99)00100-6}{HookerO99} & \hyperref[auth:a1172]{J. Hooker}, \hyperref[auth:a1173]{M. Osorio} & Mixed logical-linear programming & No & \cite{HookerO99} & 1999 & Discrete Applied Mathematics & null & 92 & 48 & No & \ref{c:HookerO99}\\
\rowlabel{a:JainM99}JainM99 \href{http://dx.doi.org/10.1016/s0377-2217(98)00113-1}{JainM99} & \hyperref[auth:a967]{A. Jain}, \hyperref[auth:a968]{S. Meeran} & Deterministic job-shop scheduling: Past, present and future & \href{../works/JainM99.pdf}{Yes} & \cite{JainM99} & 1999 & European Journal of Operational Research & 45 & 490 & 150 & \ref{b:JainM99} & \ref{c:JainM99}\\
\rowlabel{a:PesantGPR99}PesantGPR99 \href{http://dx.doi.org/10.1016/s0377-2217(98)00248-3}{PesantGPR99} & \hyperref[auth:a8]{G. Pesant}, \hyperref[auth:a624]{M. Gendreau}, \hyperref[auth:a1228]{J. Potvin}, \hyperref[auth:a1229]{J. Rousseau} & On the flexibility of constraint programming models: From single to multiple time windows for the traveling salesman problem & No & \cite{PesantGPR99} & 1999 & European Journal of Operational Research & null & 26 & 18 & No & \ref{c:PesantGPR99}\\
\rowlabel{a:RodosekWH99}RodosekWH99 \href{http://dx.doi.org/10.1023/a:1018904229454}{RodosekWH99} & \hyperref[auth:a299]{R. Rodosek}, \hyperref[auth:a117]{Mark G. Wallace}, \hyperref[auth:a1048]{M. Hajian} & A new approach to integrating mixed integer programming and constraint logic programming & No & \cite{RodosekWH99} & 1999 & Annals of Operations Research & null & 53 & 0 & No & \ref{c:RodosekWH99}\\
\rowlabel{a:BeckDDF98}BeckDDF98 \href{http://dx.doi.org/10.1002/(sici)1099-1425(199808)1:2<89::aid-jos9>3.0.co;2-h}{BeckDDF98} & \hyperref[auth:a89]{J. Christopher Beck}, \hyperref[auth:a250]{Andrew J. Davenport}, \hyperref[auth:a1244]{Eugene D. Davis}, \hyperref[auth:a304]{Mark S. Fox} & The ODO project: toward a unified basis for constraint-directed scheduling & No & \cite{BeckDDF98} & 1998 & Journal of Scheduling & null & 9 & 0 & No & \ref{c:BeckDDF98}\\
\rowlabel{a:BeckF98}BeckF98 \href{https://doi.org/10.1609/aimag.v19i4.1426}{BeckF98} & \hyperref[auth:a89]{J. Christopher Beck}, \hyperref[auth:a304]{Mark S. Fox} & A Generic Framework for Constraint-Directed Search and Scheduling & \href{../works/BeckF98.pdf}{Yes} & \cite{BeckF98} & 1998 & {AI} Mag. & 30 & 0 & 0 & \ref{b:BeckF98} & \ref{c:BeckF98}\\
\rowlabel{a:BelhadjiI98}BelhadjiI98 \href{https://doi.org/10.1023/A:1009777711218}{BelhadjiI98} & \hyperref[auth:a175]{S. Belhadji}, \hyperref[auth:a176]{A. Isli} & Temporal Constraint Satisfaction Techniques in Job Shop Scheduling Problem Solving & \href{../works/BelhadjiI98.pdf}{Yes} & \cite{BelhadjiI98} & 1998 & Constraints An Int. J. & 9 & 3 & 0 & \ref{b:BelhadjiI98} & \ref{c:BelhadjiI98}\\
\rowlabel{a:BockmayrK98}BockmayrK98 \href{http://dx.doi.org/10.1287/ijoc.10.3.287}{BockmayrK98} & \hyperref[auth:a918]{A. Bockmayr}, \hyperref[auth:a1063]{T. Kasper} & Branch and Infer: A Unifying Framework for Integer and Finite Domain Constraint Programming & No & \cite{BockmayrK98} & 1998 & INFORMS Journal on Computing & null & 79 & 27 & No & \ref{c:BockmayrK98}\\
\rowlabel{a:DarbyDowmanL98}DarbyDowmanL98 \href{http://dx.doi.org/10.1287/ijoc.10.3.276}{DarbyDowmanL98} & \hyperref[auth:a1089]{K. Darby-Dowman}, \hyperref[auth:a179]{J. Little} & Properties of Some Combinatorial Optimization Problems and Their Effect on the Performance of Integer Programming and Constraint Logic Programming & No & \cite{DarbyDowmanL98} & 1998 & INFORMS Journal on Computing & null & 28 & 6 & No & \ref{c:DarbyDowmanL98}\\
\rowlabel{a:NuijtenP98}NuijtenP98 \href{https://doi.org/10.1023/A:1009687210594}{NuijtenP98} & \hyperref[auth:a664]{W. Nuijten}, \hyperref[auth:a164]{Claude Le Pape} & Constraint-Based Job Shop Scheduling with {\textbackslash}sc Ilog Scheduler & \href{../works/NuijtenP98.pdf}{Yes} & \cite{NuijtenP98} & 1998 & J. Heuristics & 16 & 42 & 0 & \ref{b:NuijtenP98} & \ref{c:NuijtenP98}\\
\rowlabel{a:PapaB98}PapaB98 \href{https://doi.org/10.1023/A:1009723704757}{PapaB98} & \hyperref[auth:a164]{Claude Le Pape}, \hyperref[auth:a163]{P. Baptiste} & Resource Constraints for Preemptive Job-shop Scheduling & \href{../works/PapaB98.pdf}{Yes} & \cite{PapaB98} & 1998 & Constraints An Int. J. & 25 & 14 & 0 & \ref{b:PapaB98} & \ref{c:PapaB98}\\
\rowlabel{a:Darby-DowmanLMZ97}Darby-DowmanLMZ97 \href{https://doi.org/10.1007/BF00137871}{Darby-DowmanLMZ97} & \hyperref[auth:a178]{K. Darby{-}Dowman}, \hyperref[auth:a179]{J. Little}, \hyperref[auth:a180]{G. Mitra}, \hyperref[auth:a181]{M. Zaffalon} & Constraint Logic Programming and Integer Programming Approaches and Their Collaboration in Solving an Assignment Scheduling Problem & \href{../works/Darby-DowmanLMZ97.pdf}{Yes} & \cite{Darby-DowmanLMZ97} & 1997 & Constraints An Int. J. & 20 & 28 & 5 & \ref{b:Darby-DowmanLMZ97} & \ref{c:Darby-DowmanLMZ97}\\
\rowlabel{a:FalaschiGMP97}FalaschiGMP97 \href{https://doi.org/10.1006/inco.1997.2638}{FalaschiGMP97} & \hyperref[auth:a695]{M. Falaschi}, \hyperref[auth:a197]{M. Gabbrielli}, \hyperref[auth:a696]{K. Marriott}, \hyperref[auth:a697]{C. Palamidessi} & Constraint Logic Programming with Dynamic Scheduling: {A} Semantics Based on Closure Operators & \href{../works/FalaschiGMP97.pdf}{Yes} & \cite{FalaschiGMP97} & 1997 & Inf. Comput. & 27 & 10 & 9 & \ref{b:FalaschiGMP97} & \ref{c:FalaschiGMP97}\\
\rowlabel{a:LammaMM97}LammaMM97 \href{https://doi.org/10.1016/S0954-1810(96)00002-7}{LammaMM97} & \hyperref[auth:a728]{E. Lamma}, \hyperref[auth:a729]{P. Mello}, \hyperref[auth:a144]{M. Milano} & A distributed constraint-based scheduler & \href{../works/LammaMM97.pdf}{Yes} & \cite{LammaMM97} & 1997 & Artif. Intell. Eng. & 15 & 11 & 7 & \ref{b:LammaMM97} & \ref{c:LammaMM97}\\
\rowlabel{a:Zhou97}Zhou97 \href{https://doi.org/10.1023/A:1009757726572}{Zhou97} & \hyperref[auth:a177]{J. Zhou} & A Permutation-Based Approach for Solving the Job-Shop Problem & \href{../works/Zhou97.pdf}{Yes} & \cite{Zhou97} & 1997 & Constraints An Int. J. & 29 & 14 & 0 & \ref{b:Zhou97} & \ref{c:Zhou97}\\
\rowlabel{a:BlazewiczDP96}BlazewiczDP96 \href{http://dx.doi.org/10.1016/0377-2217(95)00362-2}{BlazewiczDP96} & \hyperref[auth:a988]{J. Błażewicz}, \hyperref[auth:a989]{W. Domschke}, \hyperref[auth:a443]{E. Pesch} & The job shop scheduling problem: Conventional and new solution techniques & \href{../works/BlazewiczDP96.pdf}{Yes} & \cite{BlazewiczDP96} & 1996 & European Journal of Operational Research & 33 & 344 & 127 & \ref{b:BlazewiczDP96} & \ref{c:BlazewiczDP96}\\
\rowlabel{a:NuijtenA96}NuijtenA96 \href{http://dx.doi.org/10.1016/0377-2217(95)00354-1}{NuijtenA96} & \hyperref[auth:a664]{W. Nuijten}, \hyperref[auth:a785]{E. Aarts} & A computational study of constraint satisfaction for multiple capacitated job shop scheduling & \href{../works/NuijtenA96.pdf}{Yes} & \cite{NuijtenA96} & 1996 & European Journal of Operational Research & 16 & 65 & 6 & \ref{b:NuijtenA96} & \ref{c:NuijtenA96}\\
\rowlabel{a:PeschT96}PeschT96 \href{http://dx.doi.org/10.1287/ijoc.8.2.144}{PeschT96} & \hyperref[auth:a443]{E. Pesch}, \hyperref[auth:a1242]{Ulrich A. W. Tetzlaff} & Constraint Propagation Based Scheduling of Job Shops & No & \cite{PeschT96} & 1996 & INFORMS Journal on Computing & null & 22 & 0 & No & \ref{c:PeschT96}\\
\rowlabel{a:SadehF96}SadehF96 \href{http://dx.doi.org/10.1016/0004-3702(95)00098-4}{SadehF96} & \hyperref[auth:a1189]{N. Sadeh}, \hyperref[auth:a304]{Mark S. Fox} & Variable and value ordering heuristics for the job shop scheduling constraint satisfaction problem & No & \cite{SadehF96} & 1996 & Artificial Intelligence & null & 95 & 17 & No & \ref{c:SadehF96}\\
\rowlabel{a:SmithBHW96}SmithBHW96 \href{http://dx.doi.org/10.1007/bf00143880}{SmithBHW96} & \hyperref[auth:a1071]{Barbara M. Smith}, \hyperref[auth:a1069]{Sally C. Brailsford}, \hyperref[auth:a1203]{Peter M. Hubbard}, \hyperref[auth:a1204]{H. Paul Williams} & The progressive party problem: Integer linear programming and constraint programming compared & No & \cite{SmithBHW96} & 1996 & Constraints An Int. J. & null & 56 & 4 & No & \ref{c:SmithBHW96}\\
\rowlabel{a:Wallace96}Wallace96 \href{https://doi.org/10.1007/BF00143881}{Wallace96} & \hyperref[auth:a117]{Mark G. Wallace} & Practical Applications of Constraint Programming & \href{../works/Wallace96.pdf}{Yes} & \cite{Wallace96} & 1996 & Constraints An Int. J. & 30 & 87 & 55 & \ref{b:Wallace96} & \ref{c:Wallace96}\\
\rowlabel{a:WeilHFP95}WeilHFP95 \href{http://dx.doi.org/10.1109/51.395324}{WeilHFP95} & \hyperref[auth:a1217]{G. Weil}, \hyperref[auth:a1218]{K. Heus}, \hyperref[auth:a1219]{P. Francois}, \hyperref[auth:a1220]{M. Poujade} & Constraint programming for nurse scheduling & No & \cite{WeilHFP95} & 1995 & IEEE Engineering in Medicine and Biology Magazine & null & 56 & 9 & No & \ref{c:WeilHFP95}\\
\rowlabel{a:BeldiceanuC94}BeldiceanuC94 \href{https://www.sciencedirect.com/science/article/pii/0895717794901279}{BeldiceanuC94} & \hyperref[auth:a129]{N. Beldiceanu}, \hyperref[auth:a792]{E. Contejean} & Introducing Global Constraints in {CHIP} & \href{../works/BeldiceanuC94.pdf}{Yes} & \cite{BeldiceanuC94} & 1994 & Mathematical and Computer Modelling & 27 & 167 & 8 & \ref{b:BeldiceanuC94} & \ref{c:BeldiceanuC94}\\
\rowlabel{a:Pape94}Pape94 \href{http://dx.doi.org/10.1049/ise.1994.0009}{Pape94} & \hyperref[auth:a164]{Claude Le Pape} & Implementation of resource constraints in ILOG SCHEDULE: a library for the development of constraint-based scheduling systems & \href{../works/Pape94.pdf}{Yes} & \cite{Pape94} & 1994 & Intelligent Systems Engineering & 34 & 98 & 0 & \ref{b:Pape94} & \ref{c:Pape94}\\
\rowlabel{a:AggounB93}AggounB93 \href{https://www.sciencedirect.com/science/article/pii/089571779390068A}{AggounB93} & \hyperref[auth:a733]{A. Aggoun}, \hyperref[auth:a129]{N. Beldiceanu} & Extending {CHIP} in order to solve complex scheduling and placement problems & \href{../works/AggounB93.pdf}{Yes} & \cite{AggounB93} & 1993 & Mathematical and Computer Modelling & 17 & 187 & 11 & \ref{b:AggounB93} & \ref{c:AggounB93}\\
\rowlabel{a:MintonJPL92}MintonJPL92 \href{http://dx.doi.org/10.1016/0004-3702(92)90007-k}{MintonJPL92} & \hyperref[auth:a1236]{S. Minton}, \hyperref[auth:a1237]{Mark D. Johnston}, \hyperref[auth:a1238]{Andrew B. Philips}, \hyperref[auth:a1239]{P. Laird} & Minimizing conflicts: a heuristic repair method for constraint satisfaction and scheduling problems & No & \cite{MintonJPL92} & 1992 & Artificial Intelligence & null & 437 & 13 & No & \ref{c:MintonJPL92}\\
\rowlabel{a:Tay92}Tay92 \href{}{Tay92} & \hyperref[auth:a709]{David B. H. Tay} & {COPS:} {A} Constraint Programming Approach to Resource-Limited Project Scheduling & No & \cite{Tay92} & 1992 & Comput. J. & null & 0 & 0 & No & \ref{c:Tay92}\\
\rowlabel{a:DincbasSH90}DincbasSH90 \href{https://doi.org/10.1016/0743-1066(90)90052-7}{DincbasSH90} & \hyperref[auth:a725]{M. Dincbas}, \hyperref[auth:a17]{H. Simonis}, \hyperref[auth:a149]{Pascal Van Hentenryck} & Solving Large Combinatorial Problems in Logic Programming & \href{../works/DincbasSH90.pdf}{Yes} & \cite{DincbasSH90} & 1990 & The Journal of Logic Programming & 19 & 86 & 9 & \ref{b:DincbasSH90} & \ref{c:DincbasSH90}\\
\rowlabel{a:Davis87}Davis87 \href{http://dx.doi.org/10.1016/0004-3702(87)90091-9}{Davis87} & \hyperref[auth:a1241]{E. Davis} & Constraint propagation with interval labels & No & \cite{Davis87} & 1987 & Artificial Intelligence & null & 308 & 21 & No & \ref{c:Davis87}\\
\end{longtable}
}




\clearpage
\subsection{Extracted Concepts}
{\scriptsize
\begin{longtable}{>{\raggedright\arraybackslash}p{3cm}r>{\raggedright\arraybackslash}p{4cm}p{1.5cm}p{2cm}p{1.5cm}p{1.5cm}p{1.5cm}p{1.5cm}p{2cm}p{1.5cm}rr}
\rowcolor{white}\caption{Automatically Extracted ARTICLE Features (Requires Local Copy)}\\ \toprule
\rowcolor{white}Work & Pages & Concepts & Classification & Constraints & \shortstack{Prog\\Languages} & \shortstack{CP\\Systems} & Areas & Industries & Benchmarks & Algorithm & a & c\\ \midrule\endhead
\bottomrule
\endfoot
\index{AbohashimaEG21}\rowlabel{b:AbohashimaEG21}\href{../works/AbohashimaEG21.pdf}{AbohashimaEG21}~\cite{AbohashimaEG21} & 14 & setup-time, machine, multi-objective, scheduling, order, stochastic, CP, resource, explanation, cmax, transportation & parallel machine & cycle & Python & Gurobi &  &  & real-world, generated instance, github & genetic algorithm, ant colony, memetic algorithm, machine learning, meta heuristic & \ref{a:AbohashimaEG21} & \ref{c:AbohashimaEG21}\\
\index{AbreuAPNM21}\rowlabel{b:AbreuAPNM21}\href{../works/AbreuAPNM21.pdf}{AbreuAPNM21}~\cite{AbreuAPNM21} & 20 & multi-objective, make-span, open-shop, order, machine, preempt, multi-agent, breakdown, cmax, tardiness, periodic, scheduling, no-wait, job-shop, distributed, job, resource, order scheduling, release-date, preemptive, completion-time, setup-time, CP, task, stochastic, constraint programming, precedence, flow-shop & parallel machine, OSSP, single machine, Open Shop Scheduling Problem & cycle, noOverlap & Python, C++ & Cplex & medical, patient, automotive & oil industry & generated instance, benchmark, real-world & mat heuristic, meta heuristic, genetic algorithm, simulated annealing, particle swarm, large neighborhood search & \ref{a:AbreuAPNM21} & n/a\\
\index{AbreuN22}\rowlabel{b:AbreuN22}\href{../works/AbreuN22.pdf}{AbreuN22}~\cite{AbreuN22} & 20 & CP, make-span, flow-time, distributed, resource, batch process, cmax, preemptive, order, tardiness, scheduling, multi-objective, completion-time, machine, job, task, no-wait, open-shop, transportation, stochastic, constraint programming, job-shop, flow-shop, bi-objective, preempt, inventory, setup-time & OSSP, single machine, Open Shop Scheduling Problem & cumulative, noOverlap, cycle & Python & Cplex & medical & chips industry & real-world, benchmark & meta heuristic, simulated annealing, mat heuristic, large neighborhood search, Lagrangian relaxation, ant colony, particle swarm, genetic algorithm & \ref{a:AbreuN22} & \ref{c:AbreuN22}\\
\index{AbreuNP23}\rowlabel{b:AbreuNP23}\href{../works/AbreuNP23.pdf}{AbreuNP23}~\cite{AbreuNP23} & 20 & order, completion-time, distributed, job-shop, resource, job, preempt, setup-time, no-wait, scheduling, make-span, tardiness, earliness, two-machine scheduling, energy efficiency, flow-shop, cmax, CP, machine, blocking constraint, stochastic, constraint programming, transportation, open-shop & OSSP, parallel machine, Open Shop Scheduling Problem & noOverlap, Blocking constraint & Python & Cplex, OPL & medical & oil industry & real-world, benchmark & genetic algorithm, mat heuristic, meta heuristic, simulated annealing, time-tabling, large neighborhood search & \ref{a:AbreuNP23} & \ref{c:AbreuNP23}\\
\index{AbreuPNF23}\rowlabel{b:AbreuPNF23}\href{../works/AbreuPNF23.pdf}{AbreuPNF23}~\cite{AbreuPNF23} & 12 & distributed, preemptive, job-shop, due-date, no-wait, flow-shop, constraint programming, completion-time, CP, stochastic, sustainability, setup-time, open-shop, order, order scheduling, multi-objective, transportation, bi-objective, job, scheduling, machine, make-span, periodic, tardiness, earliness, preempt, resource & RCPSP, parallel machine, OSSP, Open Shop Scheduling Problem & cumulative, disjunctive, noOverlap & Python & Cplex, OPL & medical, robot &  & real-life, benchmark, real-world & lazy clause generation, meta heuristic, ant colony, NEH, large neighborhood search, mat heuristic, simulated annealing, genetic algorithm & \ref{a:AbreuPNF23} & n/a\\
\index{Adelgren2023}\rowlabel{b:Adelgren2023}\href{../works/Adelgren2023.pdf}{Adelgren2023}~\cite{Adelgren2023} & 12 & setup-time, preempt, order, inventory, distributed, CP, resource, task, constraint programming, preemptive, release-date, sequence dependent setup, job-shop, transportation, periodic, batch process, completion-time, scheduling, machine, job, re-scheduling, make-span & parallel machine & disjunctive &  & Gurobi, Cplex & drone, crew-scheduling, aircraft, operating room, pipeline &  & benchmark, real-life, github, generated instance, supplementary material & MINLP, column generation & \ref{a:Adelgren2023} & \ref{c:Adelgren2023}\\
\index{AfsarVPG23}\rowlabel{b:AfsarVPG23}\href{../works/AfsarVPG23.pdf}{AfsarVPG23}~\cite{AfsarVPG23} & 14 & transportation, bi-objective, job, constraint programming, precedence, stochastic, task, setup-time, machine, multi-objective, scheduling, preempt, make-span, resource, job-shop, due-date, activity, flow-shop, completion-time, CP, open-shop, order &  & disjunctive &  & Cplex &  &  & benchmark, real-life, supplementary material, real-world & reinforcement learning, genetic algorithm, meta heuristic, memetic algorithm, neural network, particle swarm & \ref{a:AfsarVPG23} & \ref{c:AfsarVPG23}\\
\index{AggounB93}\rowlabel{b:AggounB93}\href{../works/AggounB93.pdf}{AggounB93}~\cite{AggounB93} & 17 & job-shop, resource, order, activity, constraint satisfaction, scheduling, task, constraint logic programming, job, due-date, flow-shop, machine, precedence, CLP &  & Cardinality constraint, circuit, Arithmetic constraint, disjunctive, Disjunctive constraint, bin-packing, Among constraint, cumulative & Prolog & CHIP, OPL & perfect-square, rectangle-packing &  & real-world & simulated annealing & \ref{a:AggounB93} & n/a\\
\index{AkramNHRSA23}\rowlabel{b:AkramNHRSA23}\href{../works/AkramNHRSA23.pdf}{AkramNHRSA23}~\cite{AkramNHRSA23} & 16 & CP, resource, task, constraint programming, periodic, completion-time, scheduling, machine, preempt, order, distributed &  & cycle, bin-packing & Python & OR-Tools & agriculture, medical &  & benchmark & genetic algorithm, reinforcement learning, machine learning, deep learning, GRASP, ant colony, simulated annealing & \ref{a:AkramNHRSA23} & \ref{c:AkramNHRSA23}\\
\index{Alaka21}\rowlabel{b:Alaka21}\href{../works/Alaka21.pdf}{Alaka21}~\cite{Alaka21} & 11 & CP, precedence, constraint programming, multi-objective, stochastic, completion-time, task, resource, cyclic scheduling, order, machine, scheduling &  & cycle &  &  &  &  & real-life & meta heuristic, genetic algorithm & \ref{a:Alaka21} & n/a\\
\index{AlakaP23}\rowlabel{b:AlakaP23}\href{../works/AlakaP23.pdf}{AlakaP23}~\cite{AlakaP23} & 14 & CP, make-span, resource, cyclic scheduling, precedence, order, scheduling, machine, task, stochastic, multi-objective, constraint programming &  & cycle &  & Cplex & robot & garment industry & real-life & meta heuristic, ant colony, time-tabling, genetic algorithm & \ref{a:AlakaP23} & n/a\\
\index{AlakaPY19}\rowlabel{b:AlakaPY19}\href{../works/AlakaPY19.pdf}{AlakaPY19}~\cite{AlakaPY19} & 9 & scheduling, precedence, CP, machine, stochastic, constraint programming, order, completion-time, cyclic scheduling, multi-objective, task, resource &  & bin-packing, cycle &  & Cplex &  &  & real-life & genetic algorithm & \ref{a:AlakaPY19} & n/a\\
\index{AlesioBNG15}\rowlabel{b:AlesioBNG15}\href{../works/AlesioBNG15.pdf}{AlesioBNG15}~\cite{AlesioBNG15} & 37 & multi-objective, open-shop, order, machine, preempt, periodic, scheduling, resource, job-shop, distributed, constraint optimization, job, activity, constraint programming, preemptive, completion-time, CP, task &  &  &  & Cplex, Ilog Solver, OPL & aircraft, satellite, telescope, automotive &  & benchmark, industrial partner & meta heuristic, genetic algorithm, Lagrangian relaxation, memetic algorithm & \ref{a:AlesioBNG15} & n/a\\
\index{AlfieriGPS23}\rowlabel{b:AlfieriGPS23}\href{../works/AlfieriGPS23.pdf}{AlfieriGPS23}~\cite{AlfieriGPS23} & 13 & stochastic, constraint programming, flow-time, completion-time, precedence, earliness, scheduling, machine, transportation, tardiness, make-span, inventory, flow-shop, job, CP, Benders Decomposition, multi-objective, setup-time, single-machine scheduling, order, distributed, no-wait, job-shop, resource & single machine, parallel machine &  & Java & Cplex & surgery, patient &  & benchmark & NEH, memetic algorithm, ant colony, meta heuristic, mat heuristic, particle swarm & \ref{a:AlfieriGPS23} & n/a\\
\index{AntunesABD20}\rowlabel{b:AntunesABD20}\href{../works/AntunesABD20.pdf}{AntunesABD20}~\cite{AntunesABD20} & 31 & precedence, earliness, scheduling, transportation, periodic, planned maintenance, activity, due-date, re-scheduling, CP, Benders Decomposition, COP, order, distributed, lateness, stochastic, task, constraint programming &  & bin-packing &  & Cplex & workforce scheduling, maintenance scheduling & electricity industry & real-world, industrial partner & column generation, genetic algorithm, meta heuristic & \ref{a:AntunesABD20} & n/a\\
\index{ArkhipovBL19}\rowlabel{b:ArkhipovBL19}\href{../works/ArkhipovBL19.pdf}{ArkhipovBL19}~\cite{ArkhipovBL19} & 10 & cmax, task, constraint programming, preemptive, completion-time, release-date, precedence, job-shop, constraint satisfaction, scheduling, machine, job, CP, make-span, preempt, order, lateness, resource & parallel machine, Resource-constrained Project Scheduling Problem, psplib, RCPSP & cycle, Cumulatives constraint, cumulative, disjunctive &  & Z3 &  &  & benchmark & sweep, time-tabling & \ref{a:ArkhipovBL19} & n/a\\
\index{ArtiguesF07}\rowlabel{b:ArtiguesF07}\href{../works/ArtiguesF07.pdf}{ArtiguesF07}~\cite{ArtiguesF07} & 25 & constraint satisfaction, constraint programming, precedence, batch process, sequence dependent setup, make-span, order, machine, preempt, cmax, tardiness, scheduling, resource, job-shop, job, preemptive, completion-time, setup-time, one-machine scheduling, CP & single machine, Resource-constrained Project Scheduling Problem & cycle, disjunctive, Disjunctive constraint & C++ & Ilog Scheduler & semiconductor &  & benchmark & large neighborhood search, meta heuristic, edge-finding, genetic algorithm & \ref{a:ArtiguesF07} & n/a\\
\index{ArtiguesL14}\rowlabel{b:ArtiguesL14}\href{../works/ArtiguesL14.pdf}{ArtiguesL14}~\cite{ArtiguesL14} & 17 & task, constraint programming, precedence, make-span, order, preempt, scheduling, re-scheduling, resource, due-date, CSP, transportation, activity, release-date, preemptive, CP & CECSP, CuSP, RCPSP & cumulative &  &  &  &  &  & energetic reasoning, lazy clause generation, column generation & \ref{a:ArtiguesL14} & n/a\\
\index{ArtiguesLH13}\rowlabel{b:ArtiguesLH13}\href{../works/ArtiguesLH13.pdf}{ArtiguesLH13}~\cite{ArtiguesLH13} & 11 & setup-time, preempt, order, CP, resource, task, constraint programming, preemptive, release-date, precedence, periodic, CSP, tardiness, activity, completion-time, constraint satisfaction, due-date, scheduling, machine, job, make-span, unavailability & CuSP, PMSP, parallel machine, single machine & circuit, cumulative & C++ & Cplex, OPL & electroplating, hoist &  & real-world & energetic reasoning, column generation & \ref{a:ArtiguesLH13} & n/a\\
\index{ArtiguesR00}\rowlabel{b:ArtiguesR00}\href{../works/ArtiguesR00.pdf}{ArtiguesR00}~\cite{ArtiguesR00} & 20 & no preempt, lateness, precedence, make-span, order, machine, preempt, cmax, scheduling, re-scheduling, resource, due-date, job-shop, transportation, job, activity, release-date, completion-time, setup-time, earliness, CP & RCMPSP, RCPSP & cumulative, cycle, disjunctive &  &  &  &  &  & simulated annealing & \ref{a:ArtiguesR00} & n/a\\
\index{AstrandJZ20}\rowlabel{b:AstrandJZ20}\href{../works/AstrandJZ20.pdf}{AstrandJZ20}~\cite{AstrandJZ20} & 13 & task, stochastic, constraint satisfaction, constraint programming, precedence, flow-shop, make-span, order, completion-time, machine, CP, breakdown, periodic, re-scheduling, unavailability, due-date, open-shop, net present value, job, activity, scheduling, resource, CSP, job-shop, setup-time & parallel machine, Resource-constrained Project Scheduling Problem & disjunctive, cycle, alldifferent, Disjunctive constraint & C++ & Gecode & robot & potash industry, mineral industry, mining industry & benchmark, real-world, real-life & large neighborhood search, meta heuristic, genetic algorithm & \ref{a:AstrandJZ20} & n/a\\
\index{AwadMDMT22}\rowlabel{b:AwadMDMT22}\href{../works/AwadMDMT22.pdf}{AwadMDMT22}~\cite{AwadMDMT22} & 22 & order, cmax, tardiness, earliness, scheduling, preempt, task, machine, make-span, preemptive, job-shop, due-date, activity, no-wait, flow-shop, completion-time, lateness, CP, inventory, batch process, resource, breakdown, job, constraint programming, precedence, release-date, stochastic, setup-time & parallel machine & cumulative, disjunctive, span constraint, noOverlap, endBeforeStart, cycle, alternative constraint, alwaysIn &  & OPL, Cplex & crew-scheduling & pharmaceutical industry & real-life, benchmark & simulated annealing, meta heuristic, genetic algorithm & \ref{a:AwadMDMT22} & n/a\\
\index{BadicaBI20}\rowlabel{b:BadicaBI20}\href{../works/BadicaBI20.pdf}{BadicaBI20}~\cite{BadicaBI20} & 17 & constraint programming, precedence, scheduling, task, CLP, machine, make-span, constraint logic programming, manpower, constraint satisfaction, resource, distributed, activity, completion-time, CP, stochastic, order & psplib, Resource-constrained Project Scheduling Problem & bin-packing, Reified constraint, Arithmetic constraint, cycle & Prolog & Gecode, ECLiPSe & business process &  & benchmark, real-world & meta heuristic & \ref{a:BadicaBI20} & n/a\\
\index{BajestaniB13}\rowlabel{b:BajestaniB13}\href{../works/BajestaniB13.pdf}{BajestaniB13}~\cite{BajestaniB13} & 36 & scheduling, machine, transportation, tardiness, make-span, setup-time, preempt, single-machine scheduling, inventory, due-date, job, re-scheduling, CP, Benders Decomposition, stochastic, breakdown, constraint programming, order, preemptive, precedence, earliness, Logic-Based Benders Decomposition, job-shop, resource, periodic, CSP, reactive scheduling & single machine, parallel machine & GCC constraint, alwaysIn, circuit, IloPulse, Cardinality constraint, cumulative, IloAlwaysIn &  & Cplex & railway, maintenance scheduling, aircraft &  &  & meta heuristic, reinforcement learning, machine learning & \ref{a:BajestaniB13} & n/a\\
\index{BajestaniB15}\rowlabel{b:BajestaniB15}\href{../works/BajestaniB15.pdf}{BajestaniB15}~\cite{BajestaniB15} & 16 & completion-time, scheduling, machine, tardiness, make-span, unavailability, planned maintenance, activity, setup-time, preempt, single-machine scheduling, due-date, distributed, flow-shop, job, CP, Benders Decomposition, stochastic, breakdown, constraint programming, flow-time, order, preemptive, precedence, sequence dependent setup, Logic-Based Benders Decomposition, job-shop, resource, periodic & single machine & circuit, disjunctive, cumulative, Disjunctive constraint &  & Cplex & railway, robot, semiconductor, maintenance scheduling & semiconductor industry & real-world & genetic algorithm, meta heuristic & \ref{a:BajestaniB15} & n/a\\
\index{BandaSC11}\rowlabel{b:BandaSC11}\href{../works/BandaSC11.pdf}{BandaSC11}~\cite{BandaSC11} & 18 & precedence, constraint optimization, CSP, CP, constraint programming, order, scheduling, task &  &  &  & Ilog Solver &  &  & benchmark, CSPlib, random instance &  & \ref{a:BandaSC11} & n/a\\
\index{BaptisteB18}\rowlabel{b:BaptisteB18}\href{../works/BaptisteB18.pdf}{BaptisteB18}~\cite{BaptisteB18} & 10 & machine, preempt, CP, scheduling, manpower, job, resource, preemptive, task, constraint satisfaction, constraint programming, precedence, make-span, order & psplib, Resource-constrained Project Scheduling Problem, RCPSP, parallel machine & cumulative, bin-packing &  & CHIP &  &  &  & lazy clause generation, time-tabling, edge-finding, edge-finder, Lagrangian relaxation & \ref{a:BaptisteB18} & n/a\\
\index{BaptisteP00}\rowlabel{b:BaptisteP00}\href{../works/BaptisteP00.pdf}{BaptisteP00}~\cite{BaptisteP00} & 21 & preempt, CP, cmax, scheduling, re-scheduling, due-date, CSP, job, activity, resource, preemptive, job-shop, task, constraint satisfaction, constraint programming, precedence, release-date, flow-shop, make-span, order & RCPSP & cumulative, Disjunctive constraint, disjunctive & C++ & Claire, CHIP, Ilog Scheduler &  &  & benchmark & energetic reasoning, edge-finding, edge-finder & \ref{a:BaptisteP00} & \ref{c:BaptisteP00}\\
\index{BaptistePN99}\rowlabel{b:BaptistePN99}\href{../works/BaptistePN99.pdf}{BaptistePN99}~\cite{BaptistePN99} & 29 & cmax, task, constraint programming, preemptive, release-date, precedence, CLP, activity, constraint satisfaction, due-date, scheduling, flow-shop, machine, job, re-scheduling, CP, make-span, setup-time, preempt, order, job-shop, resource & Resource-constrained Project Scheduling Problem, CuSP, RCPSP & cumulative, disjunctive & C++ & Claire &  &  & benchmark, real-life & energetic reasoning, edge-finding & \ref{a:BaptistePN99} & n/a\\
\index{BartakCS10}\rowlabel{b:BartakCS10}\href{../works/BartakCS10.pdf}{BartakCS10}~\cite{BartakCS10} & 31 & scheduling, constraint satisfaction, job-shop, activity, flow-shop, CP, resource, job, CSP, precedence, task, setup-time, machine, order & RCPSP & disjunctive & Prolog & SICStus &  &  & real-life, real-world, benchmark &  & \ref{a:BartakCS10} & n/a\\
\index{BartakS11}\rowlabel{b:BartakS11}\href{../works/BartakS11.pdf}{BartakS11}~\cite{BartakS11} & 5 & CSP, multi-agent, CP, distributed, constraint programming, resource, constraint optimization, order, constraint satisfaction, scheduling, task, COP, explanation & Resource-constrained Project Scheduling Problem & cumulative &  & OPL &  & software industry & random instance, real-world, real-life &  & \ref{a:BartakS11} & \ref{c:BartakS11}\\
\index{BartakSR08}\rowlabel{b:BartakSR08}\href{../works/BartakSR08.pdf}{BartakSR08}~\cite{BartakSR08} & 11 & job, constraint programming, CSP, precedence, release-date, stochastic, task, machine, order, cmax, tardiness, lateness, scheduling, preempt, make-span, constraint satisfaction, preemptive, job-shop, activity, flow-shop, completion-time, CP, open-shop, resource & single machine & Disjunctive constraint, cumulative, disjunctive &  &  & robot &  & real-life, real-world & not-last, meta heuristic, edge-finding, not-first & \ref{a:BartakSR08} & n/a\\
\index{BartakSR10}\rowlabel{b:BartakSR10}\href{../works/BartakSR10.pdf}{BartakSR10}~\cite{BartakSR10} & 31 & job, constraint programming, CSP, precedence, release-date, stochastic, task, CLP, machine, temporal constraint reasoning, order, cmax, tardiness, lateness, multi-agent, scheduling, preempt, make-span, constraint satisfaction, distributed, preemptive, job-shop, due-date, activity, flow-shop, explanation, completion-time, CP, open-shop, resource & single machine, TCSP, Temporal Constraint Satisfaction Problem & Disjunctive constraint, cumulative, disjunctive &  & CPO, Choco Solver, OPL & meeting scheduling, robot &  & real-life, real-world & sweep, machine learning, not-last, meta heuristic, edge-finding, not-first & \ref{a:BartakSR10} & n/a\\
\index{Beck07}\rowlabel{b:Beck07}\href{../works/Beck07.pdf}{Beck07}~\cite{Beck07} & 29 & order, explanation, CP, tardiness, activity, flow-shop, precedence, resource, job, stochastic, scheduling, machine, job-shop, constraint programming, periodic, constraint satisfaction, make-span &  & disjunctive, Disjunctive constraint &  & Ilog Scheduler &  &  & benchmark & systematic local search, machine learning, meta heuristic, genetic algorithm & \ref{a:Beck07} & n/a\\
\index{BeckDDF98}\rowlabel{b:BeckDDF98}\href{../works/BeckDDF98.pdf}{BeckDDF98}~\cite{BeckDDF98} & 37 & activity, constraint satisfaction, breakdown, release-date, scheduling, make-span, distributed, task, COP, job, due-date, preempt, precedence, tardiness, flow-time, reactive scheduling, earliness, preemptive, CSP, multi-agent, CP, machine, lateness, re-scheduling, stochastic, inventory, constraint programming, job-shop, resource, transportation, constraint optimization, CLP, order & single machine & cumulative, disjunctive & Prolog, C  & OPL & robot, telescope &  & real-world & edge-finding, column generation, simulated annealing, genetic algorithm & \ref{a:BeckDDF98} & n/a\\
\index{BeckF00}\rowlabel{b:BeckF00}\href{../works/BeckF00.pdf}{BeckF00}~\cite{BeckF00} & 51 & precedence, CLP, job-shop, constraint programming, explanation, due-date, preempt, release-date, task, stochastic, make-span, transportation, CSP, preemptive, reactive scheduling, machine, activity, inventory, resource, constraint satisfaction, job, order, scheduling, CP & single machine & disjunctive, cumulative, Disjunctive constraint, Cardinality constraint &  &  & robot &  & real-world, benchmark & not-last, not-first, edge-finding & \ref{a:BeckF00} & n/a\\
\index{BeckF00a}\rowlabel{b:BeckF00a}\href{../works/BeckF00a.pdf}{BeckF00a}~\cite{BeckF00a} & 40 & constraint programming, job-shop, constraint satisfaction, stochastic, CSP, activity, CP, make-span, explanation, distributed, resource, precedence, re-scheduling, order, CLP, scheduling, machine, job, task &  & disjunctive, Disjunctive constraint, cumulative &  & Ilog Solver, Ilog Scheduler & robot &  & real-world & not-last, not-first, edge-finding & \ref{a:BeckF00a} & n/a\\
\index{BeckF98}\rowlabel{b:BeckF98}\href{../works/BeckF98.pdf}{BeckF98}~\cite{BeckF98} & 30 & precedence, CLP, job-shop, explanation, due-date, preempt, re-scheduling, release-date, task, tardiness, make-span, CSP, preemptive, machine, multi-agent, activity, COP, distributed, inventory, resource, constraint satisfaction, job, order, scheduling, CP & single machine & disjunctive, circuit, cumulative & Prolog &  & robot, business process &  & real-world, benchmark & deep learning, machine learning, simulated annealing, genetic algorithm, column generation, edge-finding & \ref{a:BeckF98} & n/a\\
\index{BeckFW11}\rowlabel{b:BeckFW11}\href{../works/BeckFW11.pdf}{BeckFW11}~\cite{BeckFW11} & 14 & cmax, breakdown, constraint programming, job-shop, constraint satisfaction, preempt, periodic, CP, make-span, explanation, resource, precedence, order, scheduling, completion-time, machine, job &  & disjunctive, table constraint, cumulative & C++ & Ilog Scheduler &  &  & benchmark, real-world & machine learning, meta heuristic, simulated annealing, reinforcement learning & \ref{a:BeckFW11} & n/a\\
\index{BeckR03}\rowlabel{b:BeckR03}\href{../works/BeckR03.pdf}{BeckR03}~\cite{BeckR03} & 23 & job-shop, constraint programming, explanation, due-date, re-scheduling, completion-time, earliness, release-date, constraint logic programming, tardiness, make-span, flow-time, machine, activity, breakdown, inventory, flow-shop, resource, constraint satisfaction, job, order, scheduling, CP, precedence &  & disjunctive &  & Ilog Scheduler, Ilog Solver, Cplex & hoist &  & benchmark & genetic algorithm, column generation, edge-finder & \ref{a:BeckR03} & n/a\\
\index{BeckW07}\rowlabel{b:BeckW07}\href{../works/BeckW07.pdf}{BeckW07}~\cite{BeckW07} & 50 & job-shop, constraint programming, explanation, multi-objective, preempt, re-scheduling, no preempt, task, tardiness, stochastic, make-span, flow-time, reactive scheduling, machine, activity, distributed, flow-shop, resource, job, order, constraint optimization, scheduling, CP, precedence & RCPSP, single machine, Resource-constrained Project Scheduling Problem & Balance constraint &  & Ilog Scheduler & robot, telescope &  & benchmark & column generation, edge-finder, edge-finding & \ref{a:BeckW07} & n/a\\
\index{Bedhief21}\rowlabel{b:Bedhief21}\href{../works/Bedhief21.pdf}{Bedhief21}~\cite{Bedhief21} & 7 & preempt, setup-time, no-wait, scheduling, make-span, sequence dependent setup, due-date, flow-shop, machine, tardiness, constraint programming, completion-time, release-date, CP, no preempt, transportation, job, order & single machine, parallel machine, HFS & noOverlap &  & Cplex, OPL & robot, medical &  & real-life & meta heuristic, genetic algorithm & \ref{a:Bedhief21} & n/a\\
\index{BegB13}\rowlabel{b:BegB13}\href{../works/BegB13.pdf}{BegB13}~\cite{BegB13} & 23 & CP, re-scheduling, resource, scheduling, machine, breakdown, constraint programming, task, completion-time, order, distributed & TMS & cycle &  &  & pipeline &  & benchmark &  & \ref{a:BegB13} & n/a\\
\index{BeldiceanuC94}\rowlabel{b:BeldiceanuC94}\href{../works/BeldiceanuC94.pdf}{BeldiceanuC94}~\cite{BeldiceanuC94} & 27 & task, precedence, CP, scheduling, machine, resource, constraint logic programming, order, completion-time &  & circuit, Element constraint, Among constraint, Atmost constraint, diffn, Arithmetic constraint, cycle, bin-packing, cumulative, alldifferent & Prolog & CHIP, CPO, OZ, OPL & car manufacturing, pipeline &  & real-world, benchmark, real-life &  & \ref{a:BeldiceanuC94} & n/a\\
\index{BeldiceanuCDP11}\rowlabel{b:BeldiceanuCDP11}\href{../works/BeldiceanuCDP11.pdf}{BeldiceanuCDP11}~\cite{BeldiceanuCDP11} & 24 & cmax, constraint programming, preemptive, scheduling, preempt, task, CP, resource, order &  & disjunctive, diffn, bin-packing, geost, cumulative & Prolog & SICStus, CHIP & rectangle-packing, perfect-square &  & benchmark & sweep, energetic reasoning, edge-finding & \ref{a:BeldiceanuCDP11} & n/a\\
\index{BelhadjiI98}\rowlabel{b:BelhadjiI98}\href{../works/BelhadjiI98.pdf}{BelhadjiI98}~\cite{BelhadjiI98} & 9 & job, scheduling, explanation, resource, due-date, CSP, job-shop, task, precedence, release-date, preemptive, order, machine, preempt, constraint satisfaction & JSSP, Temporal Constraint Satisfaction Problem, TCSP & Disjunctive constraint, disjunctive &  &  &  &  & real-life &  & \ref{a:BelhadjiI98} & \ref{c:BelhadjiI98}\\
\index{BenediktMH20}\rowlabel{b:BenediktMH20}\href{../works/BenediktMH20.pdf}{BenediktMH20}~\cite{BenediktMH20} & 19 & task, preemptive, scheduling, machine, sustainability, energy efficiency, job, re-scheduling, CP, preempt, constraint programming, single-machine scheduling, order, job-shop & single machine & noOverlap, endBeforeStart &  & Gurobi & robot &  & benchmark, generated instance, random instance, github &  & \ref{a:BenediktMH20} & \ref{c:BenediktMH20}\\
\index{BeniniLMR11}\rowlabel{b:BeniniLMR11}\href{../works/BeniniLMR11.pdf}{BeniniLMR11}~\cite{BeniniLMR11} & 27 & one-machine scheduling, CP, Benders Decomposition, Logic-Based Benders Decomposition, task, precedence, make-span, order, machine, preempt, tardiness, periodic, scheduling, re-scheduling, resource, CSP, activity, explanation, constraint programming, release-date, energy efficiency, preemptive & SCC, single machine & table constraint, circuit, cumulative &  & Cplex, Ilog Scheduler & pipeline &  & real-world, benchmark, instance generator & column generation, machine learning & \ref{a:BeniniLMR11} & n/a\\
\index{BensanaLV99}\rowlabel{b:BensanaLV99}\href{../works/BensanaLV99.pdf}{BensanaLV99}~\cite{BensanaLV99} & 7 & constraint satisfaction, constraint programming, order, CP, CSP, constraint optimization, explanation &  & cycle &  & Ilog Solver, Cplex & satellite, earth observation &  & benchmark &  & \ref{a:BensanaLV99} & \ref{c:BensanaLV99}\\
\index{BidotVLB09}\rowlabel{b:BidotVLB09}\href{../works/BidotVLB09.pdf}{BidotVLB09}~\cite{BidotVLB09} & 30 & job-shop, due-date, activity, CP, inventory, order, breakdown, re-scheduling, reactive scheduling, job, constraint programming, precedence, release-date, stochastic, scheduling, task, machine, tardiness, make-span, resource, periodic, distributed & Resource-constrained Project Scheduling Problem, JSSP & cumulative, disjunctive & C++ & Ilog Scheduler, OPL & telescope, robot &  & real-world, real-life & edge-finder, edge-finding & \ref{a:BidotVLB09} & n/a\\
\index{BlazewiczDP96}\rowlabel{b:BlazewiczDP96}\href{../works/BlazewiczDP96.pdf}{BlazewiczDP96}~\cite{BlazewiczDP96} & 33 & CSP, stochastic, due-date, distributed, preempt, make-span, task, precedence, preemptive, flow-shop, CP, no-wait, activity, scheduling, machine, lateness, constraint satisfaction, inventory, job-shop, setup-time, constraint programming, release-date, resource, one-machine scheduling, job, order, completion-time, single-machine scheduling & single machine, parallel machine & disjunctive, cumulative, cycle, Disjunctive constraint &  & CHIP, OPL & robot &  & benchmark & neural network, edge-finding, simulated annealing, machine learning, energetic reasoning, Lagrangian relaxation, genetic algorithm & \ref{a:BlazewiczDP96} & n/a\\
\index{BlomBPS14}\rowlabel{b:BlomBPS14}\href{../works/BlomBPS14.pdf}{BlomBPS14}~\cite{BlomBPS14} & 19 & task, net present value, CP, stochastic, transportation, Benders Decomposition, order, distributed, resource, scheduling, precedence &  & disjunctive &  & Cplex & offshore & mineral industry & benchmark, industry partner & MINLP & \ref{a:BlomBPS14} & n/a\\
\index{BlomPS16}\rowlabel{b:BlomPS16}\href{../works/BlomPS16.pdf}{BlomPS16}~\cite{BlomPS16} & 26 & activity, CP, distributed, resource, precedence, producer/consumer, batch process, re-scheduling, order, scheduling, machine, task, transportation &  & disjunctive &  & Cplex & pipeline, offshore & process industry & industry partner, benchmark & genetic algorithm, Lagrangian relaxation, MINLP & \ref{a:BlomPS16} & n/a\\
\index{BocewiczBB09}\rowlabel{b:BocewiczBB09}\href{../works/BocewiczBB09.pdf}{BocewiczBB09}~\cite{BocewiczBB09} & 19 & precedence, scheduling, machine, transportation, periodic, CLP, tardiness, multi-agent, constraint satisfaction, job, CP, order, distributed, job-shop, resource, CSP, constraint logic programming, task, constraint programming, completion-time &  & cycle &  &  & robot &  &  & not-last & \ref{a:BocewiczBB09} & n/a\\
\index{BockmayrP06}\rowlabel{b:BockmayrP06}\href{../works/BockmayrP06.pdf}{BockmayrP06}~\cite{BockmayrP06} & 10 & scheduling, resource, due-date, task, constraint programming, release-date, order, CLP, completion-time, machine, CP &  & cycle, cumulative &  & CHIP &  &  & real-world &  & \ref{a:BockmayrP06} & n/a\\
\index{Bonfietti16}\rowlabel{b:Bonfietti16}\href{../works/Bonfietti16.pdf}{Bonfietti16}~\cite{Bonfietti16} & 13 & distributed, activity, scheduling, resource, task, constraint programming, precedence, order, CP, periodic &  & cumulative, disjunctive, circuit & C++ &  & pipeline &  & benchmark &  & \ref{a:Bonfietti16} & n/a\\
\index{BonfiettiLBM14}\rowlabel{b:BonfiettiLBM14}\href{../works/BonfiettiLBM14.pdf}{BonfiettiLBM14}~\cite{BonfiettiLBM14} & 28 & CSP, constraint satisfaction, precedence, task, buffer-capacity, activity, distributed, machine, scheduling, order, make-span, cyclic scheduling, constraint programming, CP, resource, job, periodic, job-shop & Partial Order Schedule, RCPSP & cycle, circuit, cumulative &  & Ilog Solver & hoist, medical, pipeline, robot &  & benchmark, real-world, generated instance, industrial instance & time-tabling, sweep & \ref{a:BonfiettiLBM14} & n/a\\
\index{BoothTNB16}\rowlabel{b:BoothTNB16}\href{../works/BoothTNB16.pdf}{BoothTNB16}~\cite{BoothTNB16} & 8 & constraint logic programming, scheduling, task, machine, job-shop, activity, completion-time, CP, resource, job, constraint programming, setup-time, order & parallel machine & noOverlap, Disjunctive constraint, cumulative, disjunctive & C++ & OPL, Cplex & robot &  & real-world, random instance &  & \ref{a:BoothTNB16} & n/a\\
\index{BorghesiBLMB18}\rowlabel{b:BorghesiBLMB18}\href{../works/BorghesiBLMB18.pdf}{BorghesiBLMB18}~\cite{BorghesiBLMB18} & 13 & distributed, scheduling, order, make-span, activity, constraint programming, machine, CP, job, re-scheduling, constraint satisfaction, resource, task &  & cycle, cumulative &  &  & high performance computing, super-computer &  & real-life, benchmark & machine learning & \ref{a:BorghesiBLMB18} & n/a\\
\index{BosiM2001}\rowlabel{b:BosiM2001}\href{../works/BosiM2001.pdf}{BosiM2001}~\cite{BosiM2001} & 26 & constraint satisfaction, explanation, single-machine scheduling, scheduling, precedence, make-span, due-date, cmax, CP, machine, constraint programming, CLP, order, release-date, one-machine scheduling, task, job-shop, resource, constraint logic programming, job & single machine & cumulative, Among constraint & Prolog & Ilog Scheduler, CHIP & crew-scheduling &  & real-life, real-world & meta heuristic, edge-finding & \ref{a:BosiM2001} & n/a\\
\index{BourreauGGLT22}\rowlabel{b:BourreauGGLT22}\href{../works/BourreauGGLT22.pdf}{BourreauGGLT22}~\cite{BourreauGGLT22} & 19 & re-scheduling, scheduling, order, CP, resource, job, manpower, no-wait, precedence, transportation, constraint programming &  & disjunctive, alldifferent, diffn, Disjunctive constraint, cycle & C++ & Choco Solver, CHIP, Cplex & workforce scheduling, crew-scheduling, maintenance scheduling, airport, nurse & printing industry & real-world, benchmark & large neighborhood search, meta heuristic, column generation, genetic algorithm & \ref{a:BourreauGGLT22} & n/a\\
\index{BridiBLMB16}\rowlabel{b:BridiBLMB16}\href{../works/BridiBLMB16.pdf}{BridiBLMB16}~\cite{BridiBLMB16} & 14 & energy efficiency, make-span, machine, CP, tardiness, re-scheduling, sustainability, order, distributed, job, scheduling, resource, activity, stochastic, Pareto, constraint programming &  & circuit, cycle, cumulative &  &  & medical, super-computer &  & real-world, real-life & large neighborhood search, meta heuristic, genetic algorithm, machine learning & \ref{a:BridiBLMB16} & n/a\\
\index{BruckerK00}\rowlabel{b:BruckerK00}\href{../works/BruckerK00.pdf}{BruckerK00}~\cite{BruckerK00} & 8 & make-span, cmax, preempt, resource, no preempt, preemptive, activity, completion-time, precedence, CP, order, release-date, scheduling & RCPSP, Resource-constrained Project Scheduling Problem, psplib &  &  & Cplex &  &  & benchmark & meta heuristic, column generation & \ref{a:BruckerK00} & n/a\\
\index{BukchinR18}\rowlabel{b:BukchinR18}\href{../works/BukchinR18.pdf}{BukchinR18}~\cite{BukchinR18} & 12 & scheduling, task, machine, job, constraint programming, precedence, CP, order &  & cycle, bin-packing &  & Cplex & robot &  & benchmark & machine learning, time-tabling & \ref{a:BukchinR18} & n/a\\
\index{BulckG22}\rowlabel{b:BulckG22}\href{../works/BulckG22.pdf}{BulckG22}~\cite{BulckG22} & 11 & CP, Logic-Based Benders Decomposition, distributed, order, unavailability, scheduling, constraint programming, Benders Decomposition &  &  &  & Gurobi, Cplex & round-robin, travelling tournament problem, tournament, sports scheduling &  & real-life, benchmark & memetic algorithm, meta heuristic, time-tabling, large neighborhood search & \ref{a:BulckG22} & n/a\\
\index{Caballero23}\rowlabel{b:Caballero23}\href{../works/Caballero23.pdf}{Caballero23}~\cite{Caballero23} & 1 & CP, scheduling, resource & Resource-constrained Project Scheduling Problem, RCPSP &  &  &  &  &  &  &  & \ref{a:Caballero23} & \ref{c:Caballero23}\\
\index{CampeauG22}\rowlabel{b:CampeauG22}\href{../works/CampeauG22.pdf}{CampeauG22}~\cite{CampeauG22} & 18 & activity, explanation, completion-time, precedence, CP, stochastic, order, job, constraint programming, scheduling, task, make-span, net present value, resource & RCPSP, Resource-constrained Project Scheduling Problem, RCPSPDC & noOverlap, endBeforeStart, cumulative, alwaysIn, cycle & Python & Cplex &  & mining industry & real-life, real-world & column generation, edge-finding & \ref{a:CampeauG22} & \ref{c:CampeauG22}\\
\index{CarchraeB09}\rowlabel{b:CarchraeB09}\href{../works/CarchraeB09.pdf}{CarchraeB09}~\cite{CarchraeB09} & 26 & make-span, order, machine, tardiness, scheduling, resource, job-shop, job, explanation, constraint programming, earliness, CP, task, constraint satisfaction, precedence &  & cumulative & C++ & Ilog Scheduler, OPL &  &  & benchmark, real-world & reinforcement learning, meta heuristic, machine learning, sweep, large neighborhood search & \ref{a:CarchraeB09} & n/a\\
\index{CarlierPSJ20}\rowlabel{b:CarlierPSJ20}\href{../works/CarlierPSJ20.pdf}{CarlierPSJ20}~\cite{CarlierPSJ20} & 9 & scheduling, machine, job, make-span, preempt, order, CP, resource, task, constraint programming, preemptive, release-date, job-shop, cmax & CuSP, RCPSP & cumulative &  &  &  &  & Roadef & energetic reasoning & \ref{a:CarlierPSJ20} & n/a\\
\index{CauwelaertDS20}\rowlabel{b:CauwelaertDS20}\href{../works/CauwelaertDS20.pdf}{CauwelaertDS20}~\cite{CauwelaertDS20} & 19 & constraint programming, CP, completion-time, resource, job, preemptive, constraint logic programming, job-shop, batch process, sequence dependent setup, precedence, transportation, task, CLP, activity, machine, scheduling, order, make-span, preempt, setup-time &  & cycle, Cardinality constraint, cumulative, disjunctive & Java &  & patient, container terminal &  & real-life, generated instance, benchmark, bitbucket & edge-finding, Lagrangian relaxation, not-last, not-first & \ref{a:CauwelaertDS20} & \ref{c:CauwelaertDS20}\\
\index{CauwelaertLS18}\rowlabel{b:CauwelaertLS18}\href{../works/CauwelaertLS18.pdf}{CauwelaertLS18}~\cite{CauwelaertLS18} & 36 & CSP, task, activity, machine, scheduling, order, constraint programming, CP, resource, job, constraint logic programming, job-shop & RCPSP, Resource-constrained Project Scheduling Problem, psplib & bin-packing, table constraint, circuit, alldifferent, cumulative, disjunctive, Reified constraint, GCC constraint & Java, Prolog & OPL, Gecode, CHIP &  &  & benchmark, bitbucket & meta heuristic, time-tabling, sweep, large neighborhood search, not-last, not-first, energetic reasoning, edge-finding & \ref{a:CauwelaertLS18} & \ref{c:CauwelaertLS18}\\
\index{ChenGPSH10}\rowlabel{b:ChenGPSH10}\href{../works/ChenGPSH10.pdf}{ChenGPSH10}~\cite{ChenGPSH10} & 10 & activity, job, constraint programming, CSP, precedence, Benders Decomposition, stochastic, open-shop, order, re-scheduling, constraint optimization, scheduling, preempt, manpower, task, machine, make-span, constraint satisfaction, job-shop, due-date, completion-time, lateness, COP, CP, producer/consumer, resource, preemptive, transportation & JSSP & Disjunctive constraint, cumulative, disjunctive, diffn, cycle & C++ & Ilog Scheduler, Ilog Solver &  & semiprocess industry, chemistry industry, process industry, chemical industry & real-life & particle swarm, not-last, energetic reasoning, genetic algorithm, time-tabling, neural network & \ref{a:ChenGPSH10} & n/a\\
\index{CireCH16}\rowlabel{b:CireCH16}\href{../works/CireCH16.pdf}{CireCH16}~\cite{CireCH16} & 12 & tardiness, scheduling, constraint programming, Benders Decomposition, constraint satisfaction, task, stochastic, CP, Logic-Based Benders Decomposition, order, make-span, resource, breakdown &  & cumulative &  & Cplex &  &  &  & mat heuristic & \ref{a:CireCH16} & n/a\\
\index{ClautiauxJCM08}\rowlabel{b:ClautiauxJCM08}\href{../works/ClautiauxJCM08.pdf}{ClautiauxJCM08}~\cite{ClautiauxJCM08} & 16 & scheduling, order, preempt, activity, constraint programming, CP, job, preemptive, resource, task &  & bin-packing, disjunctive, cumulative, Disjunctive constraint & Prolog & Z3, SICStus &  &  & benchmark & sweep, energetic reasoning & \ref{a:ClautiauxJCM08} & n/a\\
\index{CobanH11}\rowlabel{b:CobanH11}\href{../works/CobanH11.pdf}{CobanH11}~\cite{CobanH11} & 28 & stochastic, CP, make-span, constraint logic programming, distributed, resource, due-date, re-scheduling, preemptive, order, CLP, scheduling, completion-time, machine, job, task, release-date, tardiness, constraint programming, Benders Decomposition, Logic-Based Benders Decomposition, preempt & single machine & cumulative, circuit, noOverlap &  & OPL, Cplex & round-robin, tournament, sports scheduling &  & random instance & time-tabling & \ref{a:CobanH11} & n/a\\
\index{ColT2019a}\rowlabel{b:ColT2019a}\href{../works/ColT2019a.pdf}{ColT2019a}~\cite{ColT2019a} & 7 & constraint programming, open-shop, CLP, order, job-shop, resource, job, constraint satisfaction, scheduling, make-span, earliness, CP, machine & PTC & noOverlap, cumulative & Java & OPL, CPO, MiniZinc, OR-Tools &  &  & benchmark, github, real-world & machine learning, genetic algorithm, large neighborhood search & \ref{a:ColT2019a} & \ref{c:ColT2019a}\\
\index{ColT22}\rowlabel{b:ColT22}\href{../works/ColT22.pdf}{ColT22}~\cite{ColT22} & 19 & setup-time, scheduling, machine, batch process, breakdown, job-shop, constraint programming, task, order, one-machine scheduling, constraint satisfaction, completion-time, constraint logic programming, make-span, no preempt, due-date, COP, distributed, preempt, explanation, preemptive, CLP, multi-objective, open-shop, CP, lateness, CSP, tardiness, transportation, flow-shop, precedence, resource, job & Open Shop Scheduling Problem, FJS, single machine, PMSP, JSSP, OSSP, parallel machine & alldifferent, circuit, cumulative, noOverlap, Arithmetic constraint, disjunctive & C++, Java & OR-Tools, CPO, OPL, MiniZinc, Cplex & robot, semiconductor, oven scheduling &  & supplementary material, benchmark, generated instance, github, real-life, real-world & machine learning, genetic algorithm, particle swarm, memetic algorithm, simulated annealing, large neighborhood search & \ref{a:ColT22} & \ref{c:ColT22}\\
\index{CorreaLR07}\rowlabel{b:CorreaLR07}\href{../works/CorreaLR07.pdf}{CorreaLR07}~\cite{CorreaLR07} & 20 & task, machine, make-span, constraint logic programming, constraint satisfaction, Logic-Based Benders Decomposition, constraint programming, precedence, CP, Benders Decomposition, order, transportation, CSP, release-date, scheduling & parallel machine & disjunctive &  & OPL, Cplex, Choco Solver, Ilog Solver & workforce scheduling, container terminal & heavy industry & real-world & column generation & \ref{a:CorreaLR07} & n/a\\
\index{CzerniachowskaWZ23}\rowlabel{b:CzerniachowskaWZ23}\href{../works/CzerniachowskaWZ23.pdf}{CzerniachowskaWZ23}~\cite{CzerniachowskaWZ23} & 14 & periodic, make-span, constraint programming, transportation, flow-shop, activity, task, constraint satisfaction, resource, job-shop, sustainability, multi-objective, scheduling, Pareto, CP, setup-time, machine, order, completion-time, job & JSSP, PTC, parallel machine & endBeforeStart, noOverlap &  & CPO, OPL, Cplex & robot, automotive & pharmaceutical industry, manufacturing industry, automotive industry & real-world, benchmark, Roadef & meta heuristic, particle swarm & \ref{a:CzerniachowskaWZ23} & n/a\\
\index{Darby-DowmanLMZ97}\rowlabel{b:Darby-DowmanLMZ97}\href{../works/Darby-DowmanLMZ97.pdf}{Darby-DowmanLMZ97}~\cite{Darby-DowmanLMZ97} & 20 & constraint logic programming, scheduling, make-span, resource, CLP, CP, order, CSP, constraint programming, machine, task & single machine, MGAP & disjunctive, Element constraint, span constraint, Disjunctive constraint & Prolog & ECLiPSe, Cplex & pipeline, workforce scheduling, aircraft &  & benchmark, real-life, real-world &  & \ref{a:Darby-DowmanLMZ97} & \ref{c:Darby-DowmanLMZ97}\\
\index{Davis87}\rowlabel{b:Davis87}\href{../works/Davis87.pdf}{Davis87}~\cite{Davis87} & 51 & machine, task, constraint satisfaction, job, order, scheduling, CP &  & disjunctive, circuit, cycle &  &  & robot &  &  &  & \ref{a:Davis87} & n/a\\
\index{DemasseyAM05}\rowlabel{b:DemasseyAM05}\href{../works/DemasseyAM05.pdf}{DemasseyAM05}~\cite{DemasseyAM05} & 14 & preempt, order, CP, resource, CSP, constraint logic programming, task, constraint programming, preemptive, completion-time, release-date, precedence, job-shop, activity, constraint satisfaction, scheduling, machine, job, make-span & Resource-constrained Project Scheduling Problem, psplib, RCPSP, single machine & cumulative, disjunctive, cycle &  & Cplex &  &  & benchmark & edge-finding, energetic reasoning, Lagrangian relaxation, column generation & \ref{a:DemasseyAM05} & n/a\\
\index{DincbasSH90}\rowlabel{b:DincbasSH90}\href{../works/DincbasSH90.pdf}{DincbasSH90}~\cite{DincbasSH90} & 19 & job-shop, distributed, job, constraint programming, CLP, constraint logic programming, task, precedence, order, machine, scheduling, resource &  & Disjunctive constraint, disjunctive, circuit & Prolog & CHIP, OPL &  &  & real-life &  & \ref{a:DincbasSH90} & n/a\\
\index{Dorndorf2000}\rowlabel{b:Dorndorf2000}\href{../works/Dorndorf2000.pdf}{Dorndorf2000}~\cite{Dorndorf2000} & 52 & explanation, setup-time, preempt, single-machine scheduling, due-date, open-shop, constraint optimization, constraint logic programming, stochastic, breakdown, constraint programming, order, preemptive, precedence, sequence dependent setup, job-shop, resource, CSP, cmax, task, completion-time, constraint satisfaction, scheduling, machine, job, make-span, COP & parallel machine, OSP, Open Shop Scheduling Problem, single machine & disjunctive, cumulative, Disjunctive constraint &  & Ilog Scheduler &  &  &  & edge-finding, energetic reasoning & \ref{a:Dorndorf2000} & n/a\\
\index{DoulabiRP16}\rowlabel{b:DoulabiRP16}\href{../works/DoulabiRP16.pdf}{DoulabiRP16}~\cite{DoulabiRP16} & 17 & distributed, scheduling, resource, transportation, stochastic, constraint programming, order, machine, CP, single-machine scheduling & single machine & cycle, bin-packing, Element constraint &  & Cplex, OPL & medical, patient, nurse, operating room, rectangle-packing, surgery, steel mill, crew-scheduling, robot &  & real-world, generated instance & genetic algorithm, column generation & \ref{a:DoulabiRP16} & n/a\\
\index{Edis21}\rowlabel{b:Edis21}\href{../works/Edis21.pdf}{Edis21}~\cite{Edis21} & 20 & precedence, resource, job, constraint optimization, stochastic, scheduling, constraint programming, task, order, Pareto, COP, distributed, preempt, preemptive, multi-objective, CP, CSP &  & disjunctive, cycle &  & Cplex, Z3, OPL & crew-scheduling, medical, robot &  & benchmark & MINLP, simulated annealing, meta heuristic, genetic algorithm, particle swarm, ant colony, reinforcement learning & \ref{a:Edis21} & n/a\\
\index{ElciOH22}\rowlabel{b:ElciOH22}\href{../works/ElciOH22.pdf}{ElciOH22}~\cite{ElciOH22} & 15 & resource, due-date, order, tardiness, scheduling, Benders Decomposition, job, task, transportation, stochastic, constraint programming, CP, Logic-Based Benders Decomposition, constraint logic programming, make-span, single-machine scheduling, machine, distributed & single machine & cumulative, disjunctive & Julia & Cplex & patient, crew-scheduling, aircraft, operating room, surgery &  & benchmark, random instance, real-life &  & \ref{a:ElciOH22} & n/a\\
\index{EmdeZD22}\rowlabel{b:EmdeZD22}\href{../works/EmdeZD22.pdf}{EmdeZD22}~\cite{EmdeZD22} & 30 & constraint logic programming, flow-time, distributed, resource, inventory, precedence, batch process, order, machine, job, no-wait, release-date, transportation, tardiness, scheduling, constraint programming, Benders Decomposition, completion-time, constraint satisfaction, task, open-shop, stochastic, job-shop, CP, Logic-Based Benders Decomposition, make-span, single-machine scheduling, bi-objective & single machine, parallel machine & noOverlap, bin-packing & C  & Cplex & pipeline, drone, semiconductor, yard crane, automotive & automotive industry & github, random instance &  & \ref{a:EmdeZD22} & \ref{c:EmdeZD22}\\
\index{EmeretlisTAV17}\rowlabel{b:EmeretlisTAV17}\href{../works/EmeretlisTAV17.pdf}{EmeretlisTAV17}~\cite{EmeretlisTAV17} & 24 & stochastic, constraint programming, Logic-Based Benders Decomposition, CP, completion-time, resource, job, re-scheduling, periodic, explanation, sequence dependent setup, precedence, constraint optimization, task, release-date, tardiness, distributed, machine, multi-objective, scheduling, order, make-span, Benders Decomposition, preempt, setup-time &  & cycle, circuit, disjunctive &  &  & pipeline &  & real-life, generated instance & edge-finding, machine learning, meta heuristic, genetic algorithm & \ref{a:EmeretlisTAV17} & n/a\\
\index{EscobetPQPRA19}\rowlabel{b:EscobetPQPRA19}\href{../works/EscobetPQPRA19.pdf}{EscobetPQPRA19}~\cite{EscobetPQPRA19} & 10 & task, release-date, Pareto, activity, distributed, machine, multi-objective, scheduling, order, due-date, reactive scheduling, constraint programming, CP, resource, job, periodic, job-shop, batch process, constraint satisfaction &  & cycle, alternative constraint, noOverlap, circuit &  & OPL, Cplex & energy-price, dairy & food industry, manufacturing industry, dairy industry &  & MINLP, meta heuristic & \ref{a:EscobetPQPRA19} & n/a\\
\index{EtminaniesfahaniGNMS22}\rowlabel{b:EtminaniesfahaniGNMS22}\href{../works/EtminaniesfahaniGNMS22.pdf}{EtminaniesfahaniGNMS22}~\cite{EtminaniesfahaniGNMS22} & 10 & order, earliness, cmax, open-shop, resource, CP, preemptive, constraint programming, job, job-shop, preempt, machine, precedence, tardiness, net present value, stochastic, activity, make-span, task, scheduling & RCPSP, Resource-constrained Project Scheduling Problem, psplib, parallel machine &  & Python & Cplex, OR-Tools & crew-scheduling, aircraft &  & real-world & lazy clause generation, memetic algorithm, mat heuristic, ant colony, Lagrangian relaxation, meta heuristic, large neighborhood search, particle swarm, genetic algorithm & \ref{a:EtminaniesfahaniGNMS22} & n/a\\
\index{EvenSH15a}\rowlabel{b:EvenSH15a}\href{../works/EvenSH15a.pdf}{EvenSH15a}~\cite{EvenSH15a} & 16 & distributed, constraint programming, resource, transportation, Benders Decomposition, order, preempt, scheduling, task, machine, preemptive, completion-time, CP &  & disjunctive, Disjunctive constraint, cumulative & Java & Choco Solver, OPL & emergency service, evacuation &  & real-world, real-life & ant colony, mat heuristic, meta heuristic, column generation, sweep & \ref{a:EvenSH15a} & n/a\\
\index{FachiniA20}\rowlabel{b:FachiniA20}\href{../works/FachiniA20.pdf}{FachiniA20}~\cite{FachiniA20} & 18 & distributed, Logic-Based Benders Decomposition, CP, resource, transportation, constraint programming, Benders Decomposition, inventory, order, constraint logic programming, scheduling, task, machine & HFF & cycle, cumulative & C  & Gurobi, OR-Tools & robot, operating room, satellite &  & supplementary material, real-world, github, benchmark, real-life & meta heuristic, sweep, large neighborhood search, particle swarm, column generation, mat heuristic, simulated annealing & \ref{a:FachiniA20} & \ref{c:FachiniA20}\\
\index{FahimiOQ18}\rowlabel{b:FahimiOQ18}\href{../works/FahimiOQ18.pdf}{FahimiOQ18}~\cite{FahimiOQ18} & 22 & explanation, completion-time, CP, batch process, open-shop, order, job, constraint programming, precedence, scheduling, task, setup-time, machine, make-span, lateness, preempt, sequence dependent setup, constraint satisfaction, resource, distributed, preemptive, job-shop, due-date & Resource-constrained Project Scheduling Problem, psplib, RCPSP & Disjunctive constraint, AllDiff constraint, cumulative, disjunctive, alldifferent, Cumulatives constraint &  & Choco Solver &  &  & benchmark, random instance & time-tabling, not-first, sweep, edge-finding, not-last, lazy clause generation & \ref{a:FahimiOQ18} & \ref{c:FahimiOQ18}\\
\index{FalaschiGMP97}\rowlabel{b:FalaschiGMP97}\href{../works/FalaschiGMP97.pdf}{FalaschiGMP97}~\cite{FalaschiGMP97} & 27 & constraint programming, constraint logic programming, scheduling, CLP, order &  & Arithmetic constraint & Prolog &  &  &  &  &  & \ref{a:FalaschiGMP97} & n/a\\
\index{FallahiAC20}\rowlabel{b:FallahiAC20}\href{../works/FallahiAC20.pdf}{FallahiAC20}~\cite{FallahiAC20} & 18 & resource, COP, transportation, constraint logic programming, task, constraint satisfaction, constraint programming, order, CP, scheduling, CSP, constraint optimization &  & cycle &  & OR-Tools & nurse, container terminal, workforce scheduling, robot, medical &  & real-life, github & Lagrangian relaxation, sweep, simulated annealing, memetic algorithm, large neighborhood search, column generation, neural network, meta heuristic & \ref{a:FallahiAC20} & \ref{c:FallahiAC20}\\
\index{FanXG21}\rowlabel{b:FanXG21}\href{../works/FanXG21.pdf}{FanXG21}~\cite{FanXG21} & 15 & tardiness, multi-objective, stochastic, completion-time, no preempt, breakdown, task, flow-shop, resource, make-span, flow-time, job, order, reactive scheduling, batch process, machine, distributed, precedence, setup-time, job-shop, unavailability, due-date, preempt, earliness, one-machine scheduling, scheduling, CP & single machine, parallel machine & cycle & Python, Java & Cplex, ECLiPSe, Gurobi & semiconductor, tournament & manufacturing industry & benchmark & simulated annealing, neural network, ant colony, max-flow, machine learning, meta heuristic & \ref{a:FanXG21} & n/a\\
\index{FarsiTM22}\rowlabel{b:FarsiTM22}\href{../works/FarsiTM22.pdf}{FarsiTM22}~\cite{FarsiTM22} & 14 & tardiness, earliness, CP, re-scheduling, stochastic, constraint programming, Benders Decomposition, completion-time, multi-objective, periodic, Logic-Based Benders Decomposition, distributed, task, resource, bi-objective, continuous-process, Pareto, no-wait, scheduling, make-span &  & alldifferent, circuit &  & Cplex & physician, nurse, patient, operating room, surgery, robot, medical &  & supplementary material & time-tabling, ant colony, genetic algorithm, meta heuristic & \ref{a:FarsiTM22} & \ref{c:FarsiTM22}\\
\index{Fatemi-AnarakiTFV23}\rowlabel{b:Fatemi-AnarakiTFV23}\href{../works/Fatemi-AnarakiTFV23.pdf}{Fatemi-AnarakiTFV23}~\cite{Fatemi-AnarakiTFV23} & 15 & multi-agent, machine, cmax, resource, no-wait, order, completion-time, scheduling, CP, re-scheduling, distributed, constraint programming, job-shop, single-machine scheduling, breakdown, job, transportation, setup-time, task, cyclic scheduling, make-span & parallel machine, single machine & bin-packing, circuit, cycle, disjunctive & Python & Cplex, OPL & electroplating, COVID, robot, hoist, semiconductor & food industry & random instance, github, real-world & ant colony, mat heuristic, meta heuristic, time-tabling & \ref{a:Fatemi-AnarakiTFV23} & \ref{c:Fatemi-AnarakiTFV23}\\
\index{FetgoD22}\rowlabel{b:FetgoD22}\href{../works/FetgoD22.pdf}{FetgoD22}~\cite{FetgoD22} & 32 & CP, explanation, constraint optimization, preempt, COP, CSP, make-span, resource, precedence, cmax, order, scheduling, constraint programming, completion-time, constraint satisfaction, task & Resource-constrained Project Scheduling Problem, RCPSP, CuSP & cumulative & Java, Python & CHIP, Choco Solver &  &  & real-world, benchmark & not-first, not-last, edge-finding, lazy clause generation, edge-finder, time-tabling, energetic reasoning, sweep & \ref{a:FetgoD22} & n/a\\
\index{ForbesHJST24}\rowlabel{b:ForbesHJST24}\href{../works/ForbesHJST24.pdf}{ForbesHJST24}~\cite{ForbesHJST24} & 15 & Logic-Based Benders Decomposition, job-shop, constraint satisfaction, scheduling, machine, job, re-scheduling, make-span, order, distributed, CP, resource, Benders Decomposition, constraint logic programming, stochastic, task, constraint programming, release-date &  & cumulative & Python & Gurobi, OPL & emergency service, surgery, airport, patient, operating room &  & benchmark, real-life, github & genetic algorithm & \ref{a:ForbesHJST24} & \ref{c:ForbesHJST24}\\
\index{FrimodigECM23}\rowlabel{b:FrimodigECM23}\href{../works/FrimodigECM23.pdf}{FrimodigECM23}~\cite{FrimodigECM23} & 38 & CP, scheduling, unavailability, task, activity, machine, resource, order, constraint satisfaction, setup-time, constraint programming, Pareto, stochastic, distributed, multi-objective &  & bin-packing, cumulative & Python & Chuffed, Cplex, Gecode, MiniZinc & patient, nurse, surgery, radiation therapy, medical, operating room, physician &  & real-world, benchmark, instance generator & large neighborhood search, lazy clause generation, column generation & \ref{a:FrimodigECM23} & n/a\\
\index{GarridoAO09}\rowlabel{b:GarridoAO09}\href{../works/GarridoAO09.pdf}{GarridoAO09}~\cite{GarridoAO09} & 30 & multi-objective, CP, task, constraint satisfaction, re-scheduling, precedence, make-span, order, scheduling, resource, CSP, explanation, constraint programming &  & disjunctive & Java & OPL, Choco Solver, CPO & airport &  & benchmark &  & \ref{a:GarridoAO09} & n/a\\
\index{GarridoOS08}\rowlabel{b:GarridoOS08}\href{../works/GarridoOS08.pdf}{GarridoOS08}~\cite{GarridoOS08} & 11 & scheduling, resource, CSP, activity, explanation, constraint programming, CP, task, constraint satisfaction, make-span, order, machine &  &  & Java, C  & CPO, Choco Solver &  &  & real-world &  & \ref{a:GarridoOS08} & n/a\\
\index{GedikKBR17}\rowlabel{b:GedikKBR17}\href{../works/GedikKBR17.pdf}{GedikKBR17}~\cite{GedikKBR17} & 18 & sequence dependent setup, resource, explanation, constraint programming, CP, setup-time, order, transportation, job, scheduling, task, machine & parallel machine & alternative constraint, cumulative, noOverlap &  & OR-Tools, Cplex, Gecode & medical, tournament, nurse &  & real-world, benchmark & meta heuristic, ant colony, large neighborhood search, column generation, simulated annealing & \ref{a:GedikKBR17} & n/a\\
\index{GedikKEK18}\rowlabel{b:GedikKEK18}\href{../works/GedikKEK18.pdf}{GedikKEK18}~\cite{GedikKEK18} & 11 & resource, preemptive, constraint programming, completion-time, CP, stochastic, setup-time, order, breakdown, multi-objective, transportation, job, scheduling, task, machine, make-span, cmax, due-date, tardiness, preempt, sequence dependent setup & parallel machine, single machine, PMSP & cumulative, noOverlap &  & Cplex & medical, nurse, sports scheduling & manufacturing industry & benchmark & meta heuristic, ant colony, large neighborhood search, column generation, simulated annealing, genetic algorithm & \ref{a:GedikKEK18} & n/a\\
\index{GoelSHFS15}\rowlabel{b:GoelSHFS15}\href{../works/GoelSHFS15.pdf}{GoelSHFS15}~\cite{GoelSHFS15} & 12 & constraint programming, precedence, inventory, setup-time, order, scheduling, task, unavailability, machine, activity, CP, resource, transportation &  & cumulative, disjunctive, noOverlap, alwaysEqual constraint, alwaysIn &  & OPL, Cplex, CPO & pipeline & gas industry, transportation industry &  & large neighborhood search & \ref{a:GoelSHFS15} & n/a\\
\index{GokPTGO23}\rowlabel{b:GokPTGO23}\href{../works/GokPTGO23.pdf}{GokPTGO23}~\cite{GokPTGO23} & 36 & order, completion-time, multi-objective, activity, constraint satisfaction, distributed, task, resource, bi-objective, job, setup-time, Pareto, scheduling, precedence, make-span, tardiness, multi-agent, CP, machine, re-scheduling, stochastic, inventory, constraint programming, job-shop, transportation & RCPSP, Resource-constrained Project Scheduling Problem & circuit, alldifferent, disjunctive, cumulative, cycle &  & OPL & offshore, workforce scheduling, airport, aircraft & airline industry & real-world, github & machine learning, genetic algorithm, meta heuristic, reinforcement learning, large neighborhood search & \ref{a:GokPTGO23} & \ref{c:GokPTGO23}\\
\index{GokgurHO18}\rowlabel{b:GokgurHO18}\href{../works/GokgurHO18.pdf}{GokgurHO18}~\cite{GokgurHO18} & 17 & task, setup-time, CLP, machine, order, cmax, tardiness, earliness, scheduling, preempt, make-span, constraint satisfaction, preemptive, job-shop, due-date, activity, flow-shop, explanation, completion-time, CP, resource, transportation, job, constraint programming, CSP, precedence, release-date & parallel machine, single machine & cumulative, disjunctive, alternative constraint, Channeling constraint, Disjunctive constraint &  & OPL, CHIP & robot, semiconductor &  & real-life, real-world & mat heuristic, not-last, meta heuristic, edge-finding, energetic reasoning, genetic algorithm, not-first & \ref{a:GokgurHO18} & n/a\\
\index{GoldwaserS18}\rowlabel{b:GoldwaserS18}\href{../works/GoldwaserS18.pdf}{GoldwaserS18}~\cite{GoldwaserS18} & 32 & constraint programming, order, Logic-Based Benders Decomposition, resource, task, scheduling, machine, transportation, explanation, due-date, flow-shop, CP, Benders Decomposition, constraint optimization &  & cumulative & Python & Gurobi, CHIP, Gecode, Chuffed & torpedo & steel industry & github, instance generator, benchmark, generated instance & sweep, column generation, lazy clause generation, simulated annealing, time-tabling & \ref{a:GoldwaserS18} & \ref{c:GoldwaserS18}\\
\index{GombolayWS18}\rowlabel{b:GombolayWS18}\href{../works/GombolayWS18.pdf}{GombolayWS18}~\cite{GombolayWS18} & 20 & scheduling, machine, job, re-scheduling, open-shop, make-span, breakdown, setup-time, multi-agent, preempt, order, distributed, flow-shop, CP, resource, Benders Decomposition, task, constraint programming, preemptive, precedence, Logic-Based Benders Decomposition, job-shop, periodic, completion-time, constraint satisfaction & Resource-constrained Project Scheduling Problem, OSP & cumulative, disjunctive & Java & OPL, Gurobi & robot, patient, aircraft, crew-scheduling &  & real-world, instance generator, benchmark & simulated annealing, genetic algorithm, edge-finding, meta heuristic & \ref{a:GombolayWS18} & n/a\\
\index{GomesM17}\rowlabel{b:GomesM17}\href{../works/GomesM17.pdf}{GomesM17}~\cite{GomesM17} & 10 & Pareto, CP, Logic-Based Benders Decomposition, inventory, setup-time, make-span, single-machine scheduling, machine, distributed, resource, release-date, due-date, order, tardiness, scheduling, Benders Decomposition, completion-time, job, transportation, stochastic & parallel machine, PMSP, single machine & cycle & C++ & Cplex &  &  &  & genetic algorithm, meta heuristic, simulated annealing, Lagrangian relaxation, ant colony & \ref{a:GomesM17} & n/a\\
\index{GrimesH15}\rowlabel{b:GrimesH15}\href{../works/GrimesH15.pdf}{GrimesH15}~\cite{GrimesH15} & 17 & job, preempt, flow-shop, setup-time, no-wait, scheduling, precedence, make-span, tardiness, earliness, preemptive, sequence dependent setup, CSP, due-date, batch process, cmax, CP, machine, lateness, constraint programming, job-shop, constraint optimization, open-shop, order, completion-time, constraint satisfaction, release-date, distributed, task, COP, resource & OSP, Open Shop Scheduling Problem, JSSP & noOverlap, disjunctive, IloNoOverlap, cumulative, Balance constraint, endBeforeStart, Disjunctive constraint &  & Choco Solver, Mistral, Ilog Scheduler, CPO & semiconductor & semiconductor industry & real-world, benchmark & genetic algorithm, not-first, meta heuristic, not-last, simulated annealing, time-tabling, large neighborhood search, particle swarm, memetic algorithm, edge-finding & \ref{a:GrimesH15} & n/a\\
\index{GrimesIOS14}\rowlabel{b:GrimesIOS14}\href{../works/GrimesIOS14.pdf}{GrimesIOS14}~\cite{GrimesIOS14} & 16 & completion-time, machine, preempt, CP, periodic, re-scheduling, sustainability, due-date, distributed, activity, scheduling, resource, task, stochastic, constraint programming, preemptive, order &  & disjunctive &  & Cplex, CHIP & energy-price, real-time pricing, HVAC &  & real-world, real-life & machine learning & \ref{a:GrimesIOS14} & n/a\\
\index{Gronkvist06}\rowlabel{b:Gronkvist06}\href{../works/Gronkvist06.pdf}{Gronkvist06}~\cite{Gronkvist06} & 17 & CP, scheduling, CSP, activity, resource, transportation, constraint logic programming, constraint satisfaction, constraint programming, order &  & cycle, GCC constraint, Cardinality constraint &  & Ilog Solver, Cplex & aircraft, crew-scheduling, railway, airport &  & real-world & column generation, Lagrangian relaxation & \ref{a:Gronkvist06} & n/a\\
\index{GuoZ23}\rowlabel{b:GuoZ23}\href{../works/GuoZ23.pdf}{GuoZ23}~\cite{GuoZ23} & 29 & activity, sequence dependent setup, resource, job, setup-time, stochastic, Benders Decomposition, scheduling, inventory, machine, job-shop, constraint programming, task, order, unavailability, make-span, Logic-Based Benders Decomposition, transportation, distributed, constraint optimization, multi-objective, CP & parallel machine & bin-packing, cycle, Balance constraint & Python & Cplex, Gurobi, SCIP, OPL & railway, drone, medical, physician, operating room, patient, vaccine, COVID, automotive & automotive industry, garment industry & real-world, supplementary material, benchmark, github & machine learning, column generation, ant colony & \ref{a:GuoZ23} & \ref{c:GuoZ23}\\
\index{GurEA19}\rowlabel{b:GurEA19}\href{../works/GurEA19.pdf}{GurEA19}~\cite{GurEA19} & 24 & order, scheduling, CP, constraint programming, multi-objective, stochastic, re-scheduling, completion-time, Pareto, resource, distributed, job-shop, job &  &  &  & Cplex & medical, surgery, patient, operating room & service industry & real-life & meta heuristic, ant colony, Lagrangian relaxation & \ref{a:GurEA19} & n/a\\
\index{GurPAE23}\rowlabel{b:GurPAE23}\href{../works/GurPAE23.pdf}{GurPAE23}~\cite{GurPAE23} & 25 & re-scheduling, order, scheduling, machine, multi-objective, constraint programming, bi-objective, stochastic, CP, distributed, resource, inventory &  & cumulative &  & Cplex, OPL & patient, operating room, physician, surgery, nurse, COVID &  & real-life & neural network, meta heuristic, machine learning & \ref{a:GurPAE23} & \ref{c:GurPAE23}\\
\index{GuyonLPR12}\rowlabel{b:GuyonLPR12}\href{../works/GuyonLPR12.pdf}{GuyonLPR12}~\cite{GuyonLPR12} & 25 & precedence, CP, Benders Decomposition, order, cmax, release-date, scheduling, preempt, manpower, task, unavailability, machine, make-span, Logic-Based Benders Decomposition, COP, resource, preemptive, job-shop, activity, flow-shop, job, constraint programming, completion-time & parallel machine, single machine & disjunctive, cycle &  & Cplex & satellite &  & generated instance, benchmark, instance generator & time-tabling, column generation, Lagrangian relaxation, energetic reasoning & \ref{a:GuyonLPR12} & n/a\\
\index{HachemiGR11}\rowlabel{b:HachemiGR11}\href{../works/HachemiGR11.pdf}{HachemiGR11}~\cite{HachemiGR11} & 16 & constraint programming, precedence, make-span, CP, Benders Decomposition, Logic-Based Benders Decomposition, task, constraint satisfaction, order, scheduling, explanation, resource, job-shop, transportation, job, activity &  & GCC constraint, Cardinality constraint, alldifferent, cycle &  & OPL, Cplex, Ilog Scheduler & forestry, crew-scheduling & food industry, airline industry, forest industry &  & column generation, meta heuristic & \ref{a:HachemiGR11} & n/a\\
\index{Ham18}\rowlabel{b:Ham18}\href{../works/Ham18.pdf}{Ham18}~\cite{Ham18} & 14 & cmax, scheduling, inventory, constraint programming, distributed, constraint satisfaction, CP, CSP, batch process, resource, job, due-date, order, precedence, make-span, machine, transportation, task, completion-time, job-shop, sequence dependent setup & parallel machine & endBeforeStart, cumulative, disjunctive, cycle, noOverlap &  & Cplex, OPL & semiconductor, robot, drone, aircraft & taxi industry &  & genetic algorithm, meta heuristic & \ref{a:Ham18} & n/a\\
\index{Ham18a}\rowlabel{b:Ham18a}\href{../works/Ham18a.pdf}{Ham18a}~\cite{Ham18a} & 10 & CP, CSP, resource, job, setup-time, scheduling, inventory, machine, batch process, cmax, job-shop, constraint programming, task, order, constraint satisfaction, completion-time, make-span, tardiness, due-date & parallel machine & cycle, noOverlap, alternative constraint, disjunctive, circuit &  & Cplex, CPO, OPL & semiconductor, drone, robot &  & real-world & meta heuristic & \ref{a:Ham18a} & n/a\\
\index{Ham20a}\rowlabel{b:Ham20a}\href{../works/Ham20a.pdf}{Ham20a}~\cite{Ham20a} & 9 & CP, precedence, resource, job, multi-agent, scheduling, inventory, machine, batch process, job-shop, constraint programming, task, order, completion-time, make-span, transportation & parallel machine & cycle, noOverlap, endBeforeStart, cumulative &  & Cplex, Choco Solver, OPL & semiconductor, drone, robot, deep space &  & benchmark, generated instance & large neighborhood search, machine learning, meta heuristic & \ref{a:Ham20a} & n/a\\
\index{HamC16}\rowlabel{b:HamC16}\href{../works/HamC16.pdf}{HamC16}~\cite{HamC16} & 6 & multi-objective, CP, sequence dependent setup, precedence, resource, job, setup-time, scheduling, machine, batch process, cmax, bi-objective, job-shop, constraint programming, task, order, constraint satisfaction, completion-time, make-span, transportation & FJS & alwaysEqual constraint, cycle, endBeforeStart &  & Cplex, OPL & semiconductor & pharmaceutical industry & benchmark & particle swarm, meta heuristic, genetic algorithm & \ref{a:HamC16} & n/a\\
\index{HamFC17}\rowlabel{b:HamFC17}\href{../works/HamFC17.pdf}{HamFC17}~\cite{HamFC17} & 8 & order, batch process, machine, job-shop, preempt, job, scheduling, CP, constraint programming, tardiness, due-date, completion-time, preemptive, inventory, resource, make-span, single-machine scheduling & parallel machine, single machine & cycle, alwaysEqual constraint, bin-packing &  & Cplex, OPL & semiconductor, robot, drone & semiconductor industry & benchmark & meta heuristic, genetic algorithm & \ref{a:HamFC17} & n/a\\
\index{HamP21}\rowlabel{b:HamP21}\href{../works/HamP21.pdf}{HamP21}~\cite{HamP21} & 8 & distributed, CP, precedence, resource, job, scheduling, machine, cmax, job-shop, constraint programming, task, order, completion-time, make-span & FJS & noOverlap, endBeforeStart &  & Cplex, CPO, OPL & semiconductor, drone, robot, aircraft &  & real-world, benchmark, github & large neighborhood search, machine learning & \ref{a:HamP21} & \ref{c:HamP21}\\
\index{HamPK21}\rowlabel{b:HamPK21}\href{../works/HamPK21.pdf}{HamPK21}~\cite{HamPK21} & 12 & scheduling, CP, constraint programming, tardiness, multi-objective, completion-time, task, flow-shop, resource, make-span, single-machine scheduling, sequence dependent setup, order, machine, distributed, precedence, cmax, setup-time, job-shop, re-scheduling, bi-objective, job & single machine, FJS, parallel machine & noOverlap, cycle, endBeforeStart &  & OPL, Cplex & semiconductor, agriculture, robot &  & benchmark, github & genetic algorithm, simulated annealing, swarm intelligence, particle swarm, ant colony, Lagrangian relaxation, meta heuristic & \ref{a:HamPK21} & \ref{c:HamPK21}\\
\index{HarjunkoskiG02}\rowlabel{b:HarjunkoskiG02}\href{../works/HarjunkoskiG02.pdf}{HarjunkoskiG02}~\cite{HarjunkoskiG02} & 20 & CLP, scheduling, order, setup-time, activity, single-stage scheduling, constraint programming, machine, flow-shop, CP, job, constraint logic programming, due-date, constraint satisfaction, resource, task, release-date, job-shop &  & cumulative &  & ECLiPSe, Ilog Scheduler, CHIP, Ilog Solver, Cplex, OPL &  &  &  & simulated annealing, genetic algorithm & \ref{a:HarjunkoskiG02} & n/a\\
\index{HarjunkoskiJG00}\rowlabel{b:HarjunkoskiJG00}\href{../works/HarjunkoskiJG00.pdf}{HarjunkoskiJG00}~\cite{HarjunkoskiJG00} & 7 & due-date, constraint satisfaction, scheduling, machine, job-shop, CP, order, constraint logic programming, job, CLP & parallel machine & disjunctive &  & OPL, ECLiPSe, Ilog Solver, Cplex, CHIP &  &  &  & MINLP & \ref{a:HarjunkoskiJG00} & n/a\\
\index{HarjunkoskiMBC14}\rowlabel{b:HarjunkoskiMBC14}\href{../works/HarjunkoskiMBC14.pdf}{HarjunkoskiMBC14}~\cite{HarjunkoskiMBC14} & 33 & distributed, CLP, Benders Decomposition, precedence, lateness, unavailability, task, release-date, activity, setup-time, due-date, breakdown, Pareto, job, continuous-process, batch process, constraint programming, reactive scheduling, make-span, manpower, stochastic, constraint logic programming, make to stock, machine, re-scheduling, multi-objective, earliness, order, job-shop, Logic-Based Benders Decomposition, resource, inventory, CP, COP, periodic, scheduling, transportation, cyclic scheduling, tardiness & single machine & circuit, cycle, disjunctive &  & ECLiPSe, CHIP, Gurobi, Cplex, Gecode, SCIP, OPL & dairy, airport, semiconductor, automotive, pipeline & petro-chemical industry, oil industry, chemical industry, paper industry, pharmaceutical industry, dairy industry, process industry & benchmark, real-world, real-life & simulated annealing, particle swarm, column generation, MINLP, large neighborhood search, meta heuristic & \ref{a:HarjunkoskiMBC14} & n/a\\
\index{HauderBRPA20}\rowlabel{b:HauderBRPA20}\href{../works/HauderBRPA20.pdf}{HauderBRPA20}~\cite{HauderBRPA20} & 14 & setup-time, order, bi-objective, no-wait, job-shop, resource, stochastic, task, constraint programming, completion-time, precedence, earliness, machine, transportation, tardiness, make-span, activity, explanation, inventory, due-date, scheduling, flow-shop, job, CP, multi-objective, breakdown, manpower & RCPSP, RCMPSP, FJS, Resource-constrained Project Scheduling Problem & cumulative, cycle &  & OPL, Cplex & aircraft & automobile industry, food-processing industry, steel industry, processing industry & industry partner, benchmark, real-world, supplementary material & particle swarm, genetic algorithm, meta heuristic & \ref{a:HauderBRPA20} & \ref{c:HauderBRPA20}\\
\index{HebrardHJMPV16}\rowlabel{b:HebrardHJMPV16}\href{../works/HebrardHJMPV16.pdf}{HebrardHJMPV16}~\cite{HebrardHJMPV16} & 10 & online scheduling, cmax, scheduling, order, make-span, distributed, machine, job, completion-time, resource, task & parallel machine & cumulative &  &  & satellite, earth observation &  & industrial partner &  & \ref{a:HebrardHJMPV16} & n/a\\
\index{HeckmanB11}\rowlabel{b:HeckmanB11}\href{../works/HeckmanB11.pdf}{HeckmanB11}~\cite{HeckmanB11} & 20 & order, job, CSP, scheduling, Pareto, machine, make-span, tardiness, constraint satisfaction, resource, job-shop, activity, flow-shop, explanation, constraint programming, precedence, CP &  & disjunctive, Completion constraint &  & Ilog Scheduler &  &  & benchmark, real-world & simulated annealing, edge-finding, genetic algorithm, meta heuristic, edge-finder & \ref{a:HeckmanB11} & n/a\\
\index{HeinzNVH22}\rowlabel{b:HeinzNVH22}\href{../works/HeinzNVH22.pdf}{HeinzNVH22}~\cite{HeinzNVH22} & 16 & re-scheduling, bi-objective, scheduling, preempt, sequence dependent setup, task, unavailability, machine, make-span, distributed, flow-shop, completion-time, CP, resource, preemptive, activity, explanation, job, constraint programming, precedence, setup-time, order & parallel machine & cumulative, noOverlap, alternative constraint &  & Gurobi & high performance computing, robot, crew-scheduling &  & real-world, generated instance, benchmark, gitlab & meta heuristic, genetic algorithm, Lagrangian relaxation & \ref{a:HeinzNVH22} & \ref{c:HeinzNVH22}\\
\index{HeinzSB13}\rowlabel{b:HeinzSB13}\href{../works/HeinzSB13.pdf}{HeinzSB13}~\cite{HeinzSB13} & 36 & constraint programming, constraint optimization, order, CSP, completion-time, release-date, resource, job, constraint satisfaction, preempt, scheduling, precedence, COP, due-date, CP, machine, explanation & psplib, single machine, RCPSP, Resource-constrained Project Scheduling Problem & disjunctive, cumulative &  & MiniZinc, SCIP, Cplex & satellite &  & benchmark & edge-finding, time-tabling & \ref{a:HeinzSB13} & \ref{c:HeinzSB13}\\
\index{HeinzSSW12}\rowlabel{b:HeinzSSW12}\href{../works/HeinzSSW12.pdf}{HeinzSSW12}~\cite{HeinzSSW12} & 12 & explanation, constraint programming, inventory, order, task, constraint satisfaction, CP &  & bin-packing &  & Cplex & steel mill & steel industry, process industry & CSPlib, real-world & large neighborhood search, column generation & \ref{a:HeinzSSW12} & \ref{c:HeinzSSW12}\\
\index{HeipckeCCS00}\rowlabel{b:HeipckeCCS00}\href{../works/HeipckeCCS00.pdf}{HeipckeCCS00}~\cite{HeipckeCCS00} & 8 & CP, resource, task, constraint programming, preemptive, release-date, precedence, job-shop, activity, completion-time, due-date, scheduling, machine, job, make-span, preempt, order & RCPSP, single machine, Resource-constrained Project Scheduling Problem & disjunctive, cumulative, Disjunctive constraint &  &  &  &  & instance generator, benchmark &  & \ref{a:HeipckeCCS00} & \ref{c:HeipckeCCS00}\\
\index{HladikCDJ08}\rowlabel{b:HladikCDJ08}\href{../works/HladikCDJ08.pdf}{HladikCDJ08}~\cite{HladikCDJ08} & 18 & distributed, activity, Pareto, constraint programming, cyclic scheduling, precedence, multi-objective, preemptive, CLP, completion-time, preempt, CP, Benders Decomposition, constraint logic programming, Logic-Based Benders Decomposition, task, constraint satisfaction, order, machine, periodic, scheduling, explanation, resource &  & cycle &  & Choco Solver & automotive, robot, aircraft &  & benchmark & genetic algorithm, simulated annealing, neural network & \ref{a:HladikCDJ08} & n/a\\
\index{Hooker05}\rowlabel{b:Hooker05}\href{../works/Hooker05.pdf}{Hooker05}~\cite{Hooker05} & 17 & constraint satisfaction, machine, job, task, release-date, constraint programming, Logic-Based Benders Decomposition, CP, make-span, explanation, constraint logic programming, distributed, resource, precedence, due-date, order, tardiness, CLP, scheduling, Benders Decomposition &  & disjunctive, cumulative, circuit &  & OPL, Ilog Scheduler, Cplex &  &  & random instance & MINLP, edge-finding & \ref{a:Hooker05} & \ref{c:Hooker05}\\
\index{Hooker06}\rowlabel{b:Hooker06}\href{../works/Hooker06.pdf}{Hooker06}~\cite{Hooker06} & 19 & constraint satisfaction, machine, job, task, release-date, constraint programming, Logic-Based Benders Decomposition, CP, make-span, constraint logic programming, resource, precedence, due-date, order, tardiness, scheduling, Benders Decomposition &  & disjunctive, cumulative, circuit &  & OPL, Ilog Scheduler, Cplex &  &  & random instance & MINLP & \ref{a:Hooker06} & \ref{c:Hooker06}\\
\index{Hooker07}\rowlabel{b:Hooker07}\href{../works/Hooker07.pdf}{Hooker07}~\cite{Hooker07} & 15 & constraint satisfaction, machine, job, task, release-date, constraint programming, Logic-Based Benders Decomposition, inventory, activity, CP, make-span, constraint logic programming, distributed, resource, precedence, due-date, order, tardiness, scheduling, Benders Decomposition &  & disjunctive, cumulative, circuit &  & OPL, Ilog Scheduler, Cplex &  &  & random instance, generated instance & MINLP, edge-finding & \ref{a:Hooker07} & n/a\\
\index{HookerH17}\rowlabel{b:HookerH17}\href{../works/HookerH17.pdf}{HookerH17}~\cite{HookerH17} & 24 & preemptive, CLP, CSP, multi-agent, CP, machine, stochastic, constraint programming, one-machine scheduling, job-shop, resource, transportation, constraint optimization, open-shop, Benders Decomposition, order, multi-objective, activity, constraint satisfaction, setup-time, release-date, scheduling, Logic-Based Benders Decomposition, task, constraint logic programming, job, sequence dependent setup, preempt, flow-shop, net present value, explanation, tardiness & Open Shop Scheduling Problem, parallel machine, RCPSP & bin-packing, regular expression, Regular constraint, Cardinality constraint, Among constraint, circuit, cumulative, alldifferent, disjunctive &  & OPL, CHIP, SCIP, ECLiPSe, MiniZinc, Ilog Solver & aircraft, crew-scheduling, travelling tournament problem, sports scheduling, operating room, radiation therapy, tournament, nurse, physician &  & real-world, real-life & time-tabling, MINLP, bi-partite matching, energetic reasoning, edge-finding, column generation, not-first, not-last, neural network, Lagrangian relaxation & \ref{a:HookerH17} & n/a\\
\index{HookerO03}\rowlabel{b:HookerO03}\href{../works/HookerO03.pdf}{HookerO03}~\cite{HookerO03} & 28 & CLP, CP, machine, constraint programming, one-machine scheduling, resource, Benders Decomposition, order, constraint satisfaction, release-date, scheduling, Logic-Based Benders Decomposition, task, constraint logic programming, job, due-date &  & circuit, cumulative, disjunctive &  & Ilog Scheduler, OPL, Cplex &  &  & generated instance &  & \ref{a:HookerO03} & n/a\\
\index{HookerO99}\rowlabel{b:HookerO99}\href{../works/HookerO99.pdf}{HookerO99}~\cite{HookerO99} & 48 & CLP, CP, machine, stochastic, constraint programming, job-shop, resource, Benders Decomposition, order, make to order, constraint satisfaction, scheduling, Logic-Based Benders Decomposition, make-span, task, constraint logic programming, job, explanation &  & Disjunctive constraint, disjunctive & C++, Prolog & CHIP, Cplex, Ilog Solver &  & chemical industry &  & MINLP & \ref{a:HookerO99} & n/a\\
\index{HookerOTK00}\rowlabel{b:HookerOTK00}\href{../works/HookerOTK00.pdf}{HookerOTK00}~\cite{HookerOTK00} & 20 & order, machine, job, scheduling, explanation, resource, CSP, job-shop, transportation, constraint logic programming, stochastic, constraint programming, CLP, CP, Benders Decomposition, Logic-Based Benders Decomposition, task, constraint satisfaction &  & cumulative, Element constraint, disjunctive, Cardinality constraint &  & Cplex, Ilog Solver, CHIP, OPL & hoist &  &  & MINLP & \ref{a:HookerOTK00} & n/a\\
\index{HoundjiSW19}\rowlabel{b:HoundjiSW19}\href{../works/HoundjiSW19.pdf}{HoundjiSW19}~\cite{HoundjiSW19} & 27 & scheduling, resource, BOM, due-date, transportation, inventory, constraint programming, CP, constraint logic programming, task, order, machine & single machine & alldifferent, GCC constraint, circuit, Cardinality constraint, cumulative &  &  &  &  & random instance, benchmark, bitbucket & sweep, column generation, max-flow & \ref{a:HoundjiSW19} & \ref{c:HoundjiSW19}\\
\index{HubnerGSV21}\rowlabel{b:HubnerGSV21}\href{../works/HubnerGSV21.pdf}{HubnerGSV21}~\cite{HubnerGSV21} & 22 & completion-time, CP, task, stochastic, precedence, reactive scheduling, order, machine, preempt, cmax, tardiness, scheduling, resource, due-date, no-wait, transportation, inventory, job, activity, constraint programming, make-span, preemptive & RCPSPDC, Resource-constrained Project Scheduling Problem, RCPSP & cumulative, cycle, alternative constraint, endBeforeStart & C  & Cplex, Gurobi, OPL & automotive, tournament & dismantling industry & real-life, benchmark & large neighborhood search, meta heuristic, mat heuristic, genetic algorithm, simulated annealing & \ref{a:HubnerGSV21} & n/a\\
\index{IsikYA23}\rowlabel{b:IsikYA23}\href{../works/IsikYA23.pdf}{IsikYA23}~\cite{IsikYA23} & 28 & tardiness, scheduling, energy efficiency, multi-objective, constraint programming, completion-time, flow-shop, constraint satisfaction, task, no-wait, job-shop, blocking constraint, CP, explanation, earliness, bi-objective, constraint logic programming, preempt, batch process, setup-time, due-date, order, make-span, machine, job, distributed, resource, release-date, transportation, precedence, cmax, sequence dependent setup, breakdown & single machine, parallel machine, HFS & noOverlap, endBeforeStart, Calendar constraint, circuit, Blocking constraint, cumulative &  & OPL, Cplex & robot, medical & steel industry & benchmark, generated instance, real-life, real-world & Lagrangian relaxation, energetic reasoning, NEH, GRASP, genetic algorithm, memetic algorithm, machine learning, mat heuristic, meta heuristic, reinforcement learning, neural network, ant colony, particle swarm, simulated annealing & \ref{a:IsikYA23} & \ref{c:IsikYA23}\\
\index{JainG01}\rowlabel{b:JainG01}\href{../works/JainG01.pdf}{JainG01}~\cite{JainG01} & 19 & job-shop, constraint satisfaction, job, order, release-date, scheduling, constraint logic programming, CP, CLP, constraint programming, due-date, Benders Decomposition, task, resource, machine, activity & single machine, parallel machine & disjunctive, cumulative & Prolog & Ilog Solver, ECLiPSe, CHIP, OPL, Ilog Scheduler, Cplex & crew-scheduling &  &  & column generation, MINLP & \ref{a:JainG01} & n/a\\
\index{JainM99}\rowlabel{b:JainM99}\href{../works/JainM99.pdf}{JainM99}~\cite{JainM99} & 45 & flow-shop, preempt, constraint satisfaction, one-machine scheduling, job, open-shop, order, release-date, constraint optimization, scheduling, CP, precedence, CLP, cmax, tardiness, stochastic, due-date, re-scheduling, completion-time, CSP, preemptive, lateness, task, resource, make-span, single-machine scheduling, machine, distributed, inventory, job-shop & single machine & disjunctive, cycle &  & OPL & semiconductor, robot &  & real-world, benchmark, real-life & Lagrangian relaxation, edge-finder, memetic algorithm, simulated annealing, meta heuristic, GRASP, neural network, genetic algorithm, machine learning & \ref{a:JainM99} & n/a\\
\index{Jans09}\rowlabel{b:Jans09}\href{../works/Jans09.pdf}{Jans09}~\cite{Jans09} & 14 & distributed, CP, scheduling, sequence dependent setup, resource, job, setup-time, multi-agent, inventory, machine, order & single machine, parallel machine &  &  & Cplex & offshore, business process & foundry industry, tire industry, fashion industry, process industry & benchmark & column generation, meta heuristic & \ref{a:Jans09} & n/a\\
\index{JussienL02}\rowlabel{b:JussienL02}\href{../works/JussienL02.pdf}{JussienL02}~\cite{JussienL02} & 25 & job, constraint programming, CSP, precedence, stochastic, task, CLP, machine, order, constraint logic programming, scheduling, preempt, make-span, constraint satisfaction, preemptive, job-shop, explanation, CP, open-shop, resource & Open Shop Scheduling Problem, TMS &  &  &  & satellite &  & benchmark, real-life & genetic algorithm, time-tabling, neural network & \ref{a:JussienL02} & n/a\\
\index{JuvinHL22}\rowlabel{b:JuvinHL22}\href{../works/JuvinHL22.pdf}{JuvinHL22}~\cite{JuvinHL22} & 32 & precedence, preemptive, completion-time, cmax, CP, machine, re-scheduling, distributed, constraint programming, job-shop, resource, Benders Decomposition, order, activity, constraint satisfaction, setup-time, Pareto, release-date, scheduling, Logic-Based Benders Decomposition, make-span, task, job, preempt, flow-shop & parallel machine, single machine, JSSP, FJS & endBeforeStart, disjunctive, Disjunctive constraint, noOverlap, circuit, cumulative &  & Cplex, CPO &  &  & benchmark & meta heuristic, simulated annealing, genetic algorithm & \ref{a:JuvinHL22} & n/a\\
\index{JuvinHL23a}\rowlabel{b:JuvinHL23a}\href{../works/JuvinHL23a.pdf}{JuvinHL23a}~\cite{JuvinHL23a} & 17 & task, machine, make-span, preempt, constraint satisfaction, resource, distributed, Logic-Based Benders Decomposition, preemptive, job-shop, activity, flow-shop, constraint programming, precedence, CP, Benders Decomposition, stochastic, setup-time, order, re-scheduling, job, release-date, scheduling, Pareto & FJS, JSSP, parallel machine, single machine & noOverlap, endBeforeStart, bin-packing, cumulative, circuit, disjunctive, Disjunctive constraint &  & CPO, Cplex & vaccine, drone, operating room, COVID &  & benchmark & genetic algorithm, machine learning, simulated annealing, meta heuristic & \ref{a:JuvinHL23a} & n/a\\
\index{Kameugne15}\rowlabel{b:Kameugne15}\href{../works/Kameugne15.pdf}{Kameugne15}~\cite{Kameugne15} & 2 & resource, preemptive, completion-time, constraint programming, scheduling, task, preempt &  & cumulative &  &  &  &  &  & not-last, not-first, edge-finding & \ref{a:Kameugne15} & \ref{c:Kameugne15}\\
\index{KameugneF13}\rowlabel{b:KameugneF13}\href{../works/KameugneF13.pdf}{KameugneF13}~\cite{KameugneF13} & 21 & order, task, release-date &  & cumulative &  &  &  &  &  & not-first & \ref{a:KameugneF13} & n/a\\
\index{KameugneFSN14}\rowlabel{b:KameugneFSN14}\href{../works/KameugneFSN14.pdf}{KameugneFSN14}~\cite{KameugneFSN14} & 27 & completion-time, preemptive, CP, scheduling, CLP, precedence, job-shop, release-date, resource, job, order, constraint programming, preempt, make-span, task & RCPSP, psplib, CuSP, Resource-constrained Project Scheduling Problem & disjunctive, cumulative &  & CHIP, Gecode &  &  & random instance, benchmark & edge-finding, not-first, edge-finder, time-tabling, energetic reasoning, not-last & \ref{a:KameugneFSN14} & \ref{c:KameugneFSN14}\\
\index{KelbelH11}\rowlabel{b:KelbelH11}\href{../works/KelbelH11.pdf}{KelbelH11}~\cite{KelbelH11} & 10 & due-date, preempt, precedence, tardiness, earliness, CSP, CP, machine, inventory, constraint programming, job-shop, resource, order, completion-time, constraint satisfaction, release-date, scheduling, make-span, distributed, task, job & JSSP & cumulative, disjunctive &  & Cplex, Ilog Solver, OPL &  &  & generated instance, benchmark, random instance & edge-finder, large neighborhood search, edge-finding & \ref{a:KelbelH11} & n/a\\
\index{KhayatLR06}\rowlabel{b:KhayatLR06}\href{../works/KhayatLR06.pdf}{KhayatLR06}~\cite{KhayatLR06} & 15 & order, cmax, scheduling, preempt, task, machine, make-span, constraint satisfaction, job-shop, due-date, constraint logic programming, CP, resource, preemptive, activity, job, constraint programming, precedence, setup-time &  &  &  & OPL, Cplex &  &  & real-life, benchmark & genetic algorithm & \ref{a:KhayatLR06} & n/a\\
\index{KoehlerBFFHPSSS21}\rowlabel{b:KoehlerBFFHPSSS21}\href{../works/KoehlerBFFHPSSS21.pdf}{KoehlerBFFHPSSS21}~\cite{KoehlerBFFHPSSS21} & 51 & constraint satisfaction, flow-shop, CSP, job, constraint programming, tardiness, task, multi-objective, scheduling, single-machine scheduling, make-span, resource, precedence, job-shop, order, lateness, CP, machine, one-machine scheduling, flow-time, constraint optimization & CTW, single machine & Channeling constraint, alldifferent, Disjunctive constraint, circuit, cumulative, cycle, disjunctive & C , Python & MiniZinc, OPL, Cplex, Gurobi, OR-Tools, Chuffed, Z3 & cable tree, automotive, robot &  & real-world, benchmark, github & ant colony, genetic algorithm, particle swarm, simulated annealing & \ref{a:KoehlerBFFHPSSS21} & \ref{c:KoehlerBFFHPSSS21}\\
\index{KorbaaYG00}\rowlabel{b:KorbaaYG00}\href{../works/KorbaaYG00.pdf}{KorbaaYG00}~\cite{KorbaaYG00} & 10 &  &  &  &  &  &  &  &  &  & \ref{a:KorbaaYG00} & n/a\\
\index{KovacsB08}\rowlabel{b:KovacsB08}\href{../works/KovacsB08.pdf}{KovacsB08}~\cite{KovacsB08} & 7 & order, activity, preempt, release-date, single-machine scheduling, scheduling, job, preemptive, machine, tardiness, constraint programming, CSP, completion-time, CP, resource & single machine & disjunctive, Disjunctive constraint, bin-packing, cumulative, cycle, Regular constraint, Completion constraint, Cardinality constraint &  & Ilog Solver, Ilog Scheduler & aircraft &  & benchmark & genetic algorithm, sweep & \ref{a:KovacsB08} & n/a\\
\index{KovacsB11}\rowlabel{b:KovacsB11}\href{../works/KovacsB11.pdf}{KovacsB11}~\cite{KovacsB11} & 24 & order, activity, preempt, release-date, single-machine scheduling, scheduling, make-span, flow-time, job, preemptive, due-date, flow-shop, machine, precedence, tardiness, constraint programming, earliness, completion-time, CP, distributed, job-shop, resource & single machine, parallel machine, Resource-constrained Project Scheduling Problem & disjunctive, Disjunctive constraint, cumulative, cycle, Regular constraint, Completion constraint, Cardinality constraint, Channeling constraint & C++ & Ilog Solver, Ilog Scheduler &  &  & benchmark & edge-finding, column generation & \ref{a:KovacsB11} & \ref{c:KovacsB11}\\
\index{KovacsK11}\rowlabel{b:KovacsK11}\href{../works/KovacsK11.pdf}{KovacsK11}~\cite{KovacsK11} & 24 & Benders Decomposition, order, constraint satisfaction, breakdown, Pareto, release-date, scheduling, task, COP, job, sequence dependent setup, due-date, flow-shop, machine, stochastic, tardiness, constraint programming, constraint optimization, earliness, CSP, completion-time, CP, Logic-Based Benders Decomposition, job-shop, resource, transportation & single machine & cycle, Reified constraint & C++ & Cplex, Ilog Solver, Gecode &  &  &  &  & \ref{a:KovacsK11} & \ref{c:KovacsK11}\\
\index{KreterSS17}\rowlabel{b:KreterSS17}\href{../works/KreterSS17.pdf}{KreterSS17}~\cite{KreterSS17} & 31 & order, scheduling, task, unavailability, machine, make-span, periodic, preempt, resource, preemptive, activity, explanation, constraint programming, completion-time, precedence, CP & Resource-constrained Project Scheduling Problem, RCPSP, parallel machine & IloPulse, cumulative, IloForbidEnd, Pulse constraint, Reified constraint, Calendar constraint, alwaysIn, diffn, cycle, IloAlwaysIn, Element constraint &  & MiniZinc, Chuffed, CPO, Cplex, CHIP &  &  & benchmark & edge-finding, lazy clause generation & \ref{a:KreterSS17} & \ref{c:KreterSS17}\\
\index{KreterSSZ18}\rowlabel{b:KreterSSZ18}\href{../works/KreterSSZ18.pdf}{KreterSSZ18}~\cite{KreterSSZ18} & 15 & activity, constraint programming, precedence, release-date, preemptive, completion-time, preempt, CP, task, unavailability, order, machine, tardiness, periodic, scheduling, explanation, resource & Resource-constrained Project Scheduling Problem, RCPSP, psplib & cumulative, Element constraint, Calendar constraint &  & Chuffed, Cplex, MiniZinc &  &  & benchmark & genetic algorithm, particle swarm, GRASP, lazy clause generation, Lagrangian relaxation & \ref{a:KreterSSZ18} & n/a\\
\index{KuB16}\rowlabel{b:KuB16}\href{../works/KuB16.pdf}{KuB16}~\cite{KuB16} & 9 & earliness, job, order, precedence, make-span, machine, tardiness, completion-time, job-shop, scheduling, constraint programming, constraint satisfaction, CP &  & Disjunctive constraint, disjunctive &  & Ilog Scheduler, Cplex, SCIP, Gurobi &  &  & benchmark & meta heuristic, genetic algorithm & \ref{a:KuB16} & n/a\\
\index{Kuchcinski03}\rowlabel{b:Kuchcinski03}\href{../works/Kuchcinski03.pdf}{Kuchcinski03}~\cite{Kuchcinski03} & 29 & constraint logic programming, task, constraint programming, completion-time, precedence, periodic, CLP, scheduling, job, re-scheduling, CP, multi-objective, order, distributed, job-shop, resource, CSP, constraint optimization &  & Reified constraint, Disjunctive constraint, Arithmetic constraint, diffn, cumulative, circuit, Element constraint, disjunctive, Diff2 constraint, cycle & Prolog, Java & SICStus, CHIP & pipeline &  & benchmark & genetic algorithm, meta heuristic, edge-finding & \ref{a:Kuchcinski03} & n/a\\
\index{KuchcinskiW03}\rowlabel{b:KuchcinskiW03}\href{../works/KuchcinskiW03.pdf}{KuchcinskiW03}~\cite{KuchcinskiW03} & 15 & distributed, precedence, CP, CSP, scheduling, constraint programming, resource, constraint logic programming, order &  & cycle, circuit, Diff2 constraint & Java &  & pipeline &  & benchmark &  & \ref{a:KuchcinskiW03} & n/a\\
\index{Laborie03}\rowlabel{b:Laborie03}\href{../works/Laborie03.pdf}{Laborie03}~\cite{Laborie03} & 38 & task, job, preempt, precedence, preemptive, cmax, CP, machine, re-scheduling, inventory, constraint programming, job-shop, resource, order, activity, constraint satisfaction, setup-time, release-date, scheduling, make-span &  & cumulative, disjunctive, table constraint, cycle, Balance constraint, Disjunctive constraint & C++ & Ilog Scheduler &  &  & benchmark & edge-finding, not-first, not-last, time-tabling, energetic reasoning & \ref{a:Laborie03} & n/a\\
\index{LaborieR14}\rowlabel{b:LaborieR14}\href{../works/LaborieR14.pdf}{LaborieR14}~\cite{LaborieR14} & 10 & single-machine scheduling, order, breakdown, transportation, job, constraint programming, scheduling, task, machine, net present value, tardiness, earliness, preempt, resource, Logic-Based Benders Decomposition, job-shop, due-date, activity, flow-shop, precedence, CP, Benders Decomposition & single machine, Partial Order Schedule, RCPSP & disjunctive, span constraint, noOverlap, endBeforeStart, cumulative, alternative constraint &  & Cplex & satellite, aircraft &  & benchmark, real-world & column generation, large neighborhood search, machine learning & \ref{a:LaborieR14} & n/a\\
\index{LaborieRSV18}\rowlabel{b:LaborieRSV18}\href{../works/LaborieRSV18.pdf}{LaborieRSV18}~\cite{LaborieRSV18} & 41 & Benders Decomposition, multi-objective, breakdown, manpower, setup-time, order, distributed, Logic-Based Benders Decomposition, job-shop, resource, CSP, net present value, batch process, stochastic, task, constraint programming, release-date, precedence, earliness, sequence dependent setup, scheduling, machine, transportation, periodic, tardiness, make-span, activity, explanation, inventory, due-date, flow-shop, job, re-scheduling, CP & Resource-constrained Project Scheduling Problem, psplib, RCPSP, parallel machine & endBeforeStart, noOverlap, alternative constraint, cumulative, disjunctive, span constraint, cycle, Reified constraint, AlwaysConstant, Disjunctive constraint, alwaysEqual constraint, Arithmetic constraint, Calendar constraint, alwaysIn & Python, C++, C , Java & Ilog Scheduler, OPL, CHIP, Choco Solver, Gecode, Ilog Solver, Cplex, CPO & semiconductor, satellite, aircraft, robot, pipeline, shipping line, railway, container terminal & petro-chemical industry, chemical industry & CSPlib, benchmark, real-world & edge-finding, large neighborhood search & \ref{a:LaborieRSV18} & \ref{c:LaborieRSV18}\\
\index{LacknerMMWW23}\rowlabel{b:LacknerMMWW23}\href{../works/LacknerMMWW23.pdf}{LacknerMMWW23}~\cite{LacknerMMWW23} & 42 & explanation, CP, scheduling, machine, lateness, constraint optimization, batch process, job-shop, constraint programming, release-date, job, order, tardiness, earliness, setup-time, Pareto, due-date, multi-objective, make-span, task & single machine, parallel machine, OSP & disjunctive, bin-packing, alternative constraint, cumulative, endBeforeStart, noOverlap, Element constraint &  & Chuffed, Cplex, Gurobi, OPL, CPO, MiniZinc, OR-Tools & semiconductor, oven scheduling & manufacturing industry, electronics industry, steel industry & zenodo, random instance, benchmark, instance generator, real-life, industrial partner & ant colony, GRASP, simulated annealing, large neighborhood search, time-tabling, particle swarm, meta heuristic, genetic algorithm & \ref{a:LacknerMMWW23} & \ref{c:LacknerMMWW23}\\
\index{LammaMM97}\rowlabel{b:LammaMM97}\href{../works/LammaMM97.pdf}{LammaMM97}~\cite{LammaMM97} & 15 & CP, CLP, order, CSP, distributed, job-shop, resource, constraint logic programming, job, constraint satisfaction, no-wait, scheduling, precedence, task &  & Disjunctive constraint, circuit, disjunctive & C++, Prolog & ECLiPSe, OPL, CHIP & railway, train schedule &  & real-life &  & \ref{a:LammaMM97} & n/a\\
\index{LetortCB15}\rowlabel{b:LetortCB15}\href{../works/LetortCB15.pdf}{LetortCB15}~\cite{LetortCB15} & 52 & job, constraint programming, precedence, CP, order, scheduling, task, CLP, machine, make-span, resource & Resource-constrained Project Scheduling Problem, psplib & cumulative, Cumulatives constraint, cycle, bin-packing & Java, Prolog & Choco Solver, CHIP, SICStus &  &  & generated instance, benchmark, random instance, Roadef & energetic reasoning, large neighborhood search, meta heuristic, sweep, edge-finding & \ref{a:LetortCB15} & \ref{c:LetortCB15}\\
\index{LiW08}\rowlabel{b:LiW08}\href{../works/LiW08.pdf}{LiW08}~\cite{LiW08} & 18 & activity, setup-time, scheduling, constraint programming, no preempt, constraint satisfaction, CP, CSP, resource, explanation, job, due-date, order, precedence, make-span, machine, preempt, task, completion-time, job-shop, re-scheduling, open-shop, Benders Decomposition, constraint logic programming & RCPSP & disjunctive, bin-packing, cycle &  & Ilog Solver, Cplex, ECLiPSe, CHIP, OPL & tournament, travelling tournament problem, astronomy &  & real-world & Lagrangian relaxation & \ref{a:LiW08} & n/a\\
\index{LiessM08}\rowlabel{b:LiessM08}\href{../works/LiessM08.pdf}{LiessM08}~\cite{LiessM08} & 12 & activity, job-shop, CP, make-span, preempt, resource, precedence, preemptive, order, machine, job, cmax, scheduling, constraint programming, constraint satisfaction, task & psplib, Resource-constrained Project Scheduling Problem, RCPSP & cumulative, disjunctive & C++ &  &  &  & benchmark & edge-finding, meta heuristic, large neighborhood search, column generation & \ref{a:LiessM08} & n/a\\
\index{LimtanyakulS12}\rowlabel{b:LimtanyakulS12}\href{../works/LimtanyakulS12.pdf}{LimtanyakulS12}~\cite{LimtanyakulS12} & 32 & precedence, constraint optimization, release-date, activity, tardiness, constraint programming, machine, scheduling, order, Benders Decomposition, due-date, stochastic, CP, completion-time, resource, job, COP, CSP, constraint satisfaction & Resource-constrained Project Scheduling Problem & bin-packing, disjunctive, table constraint, Cardinality constraint, cumulative &  & Ilog Scheduler, Cplex & robot, automotive & automotive industry & real-life, generated instance, industrial partner, benchmark, random instance & not-last, not-first, genetic algorithm, energetic reasoning, edge-finding & \ref{a:LimtanyakulS12} & \ref{c:LimtanyakulS12}\\
\index{LiuW11}\rowlabel{b:LiuW11}\href{../works/LiuW11.pdf}{LiuW11}~\cite{LiuW11} & 10 & periodic, stochastic, inventory, machine, constraint programming, task, order, constraint satisfaction, completion-time, CP, transportation, due-date, distributed, resource, preempt, preemptive, multi-objective, CSP, activity, scheduling, precedence & Resource-constrained Project Scheduling Problem, RCPSP & cumulative &  &  & robot, train schedule &  &  & simulated annealing, time-tabling, genetic algorithm, neural network, ant colony & \ref{a:LiuW11} & n/a\\
\index{LombardiM10a}\rowlabel{b:LombardiM10a}\href{../works/LombardiM10a.pdf}{LombardiM10a}~\cite{LombardiM10a} & 30 & task, preemptive, completion-time, precedence, scheduling, machine, make-span, activity, preempt, constraint satisfaction, due-date, distributed, job, re-scheduling, CP, Benders Decomposition, stochastic, constraint programming, order, release-date, Logic-Based Benders Decomposition, resource, CSP & TCSP, Resource-constrained Project Scheduling Problem & Disjunctive constraint, cycle, table constraint, span constraint, cumulative, disjunctive & C  & Cplex & business process &  & benchmark, real-life, real-world & genetic algorithm, sweep & \ref{a:LombardiM10a} & n/a\\
\index{LombardiM12}\rowlabel{b:LombardiM12}\href{../works/LombardiM12.pdf}{LombardiM12}~\cite{LombardiM12} & 35 & precedence, flow-shop, reactive scheduling, sequence dependent setup, make-span, order, machine, preempt, CP, tardiness, re-scheduling, due-date, CSP, distributed, manpower, job, activity, scheduling, explanation, resource, energy efficiency, preemptive, job-shop, transportation, completion-time, setup-time, earliness, Benders Decomposition, Logic-Based Benders Decomposition, task, inventory, stochastic, constraint satisfaction, constraint programming & psplib, Partial Order Schedule, parallel machine, Resource-constrained Project Scheduling Problem, RCPSP & circuit, cycle, cumulative, Disjunctive constraint, disjunctive &  & OR-Tools & aircraft & chemical industry & real-world, benchmark & large neighborhood search, lazy clause generation, energetic reasoning, meta heuristic, edge-finding, genetic algorithm & \ref{a:LombardiM12} & \ref{c:LombardiM12}\\
\index{LombardiM12a}\rowlabel{b:LombardiM12a}\href{../works/LombardiM12a.pdf}{LombardiM12a}~\cite{LombardiM12a} & 10 & completion-time, precedence, scheduling, make-span, activity, producer/consumer, CP, stochastic, constraint programming, order, resource, CSP & psplib, Partial Order Schedule, Resource-constrained Project Scheduling Problem, RCPSP & disjunctive &  & Ilog Solver &  &  & benchmark &  & \ref{a:LombardiM12a} & n/a\\
\index{LombardiMB13}\rowlabel{b:LombardiMB13}\href{../works/LombardiMB13.pdf}{LombardiMB13}~\cite{LombardiMB13} & 14 & cmax, task, preemptive, completion-time, precedence, scheduling, periodic, make-span, energy efficiency, activity, explanation, preempt, constraint satisfaction, distributed, re-scheduling, CP, multi-objective, stochastic, constraint programming, order, resource, CSP & SCC, RCPSP & cycle, cumulative, circuit &  & Gecode, Ilog Solver, OR-Tools & pipeline, medical &  & benchmark, real-world &  & \ref{a:LombardiMB13} & n/a\\
\index{LombardiMRB10}\rowlabel{b:LombardiMRB10}\href{../works/LombardiMRB10.pdf}{LombardiMRB10}~\cite{LombardiMRB10} & 31 & constraint programming, stochastic, resource, constraint logic programming, periodic, Benders Decomposition, completion-time, tardiness, Logic-Based Benders Decomposition, distributed, no preempt, preempt, make-span, task, precedence, preemptive, CP, activity, re-scheduling, producer/consumer, scheduling, release-date, order, constraint satisfaction & SCC & table constraint, cumulative, Disjunctive constraint, circuit, disjunctive, cycle, bin-packing & C  & ECLiPSe, Cplex & pipeline, semiconductor & semiconductor industry & real-life, real-world, benchmark & genetic algorithm, simulated annealing & \ref{a:LombardiMRB10} & n/a\\
\index{LopesCSM10}\rowlabel{b:LopesCSM10}\href{../works/LopesCSM10.pdf}{LopesCSM10}~\cite{LopesCSM10} & 39 & distributed, stock level, job-shop, due-date, activity, explanation, CP, resource, transportation, reactive scheduling, job, constraint programming, precedence, inventory, CLP, order, multi-objective, re-scheduling, scheduling, task, make-span &  & table constraint, cycle, alldifferent, disjunctive & C++ & Ilog Scheduler, Ilog Solver, OPL & pipeline & oil industry & real-world, benchmark & meta heuristic, MINLP, genetic algorithm, max-flow & \ref{a:LopesCSM10} & \ref{c:LopesCSM10}\\
\index{LopezAKYG00}\rowlabel{b:LopezAKYG00}\href{../works/LopezAKYG00.pdf}{LopezAKYG00}~\cite{LopezAKYG00} & 4 &  &  &  &  &  &  &  &  &  & \ref{a:LopezAKYG00} & n/a\\
\index{LorigeonBB02}\rowlabel{b:LorigeonBB02}\href{../works/LorigeonBB02.pdf}{LorigeonBB02}~\cite{LorigeonBB02} & 8 & resource, cmax, completion-time, scheduling, machine, make-span, unavailability, activity, setup-time, preempt, flow-shop, job, CP, open-shop, order & parallel machine, Open Shop Scheduling Problem &  &  & Cplex, OPL &  &  &  &  & \ref{a:LorigeonBB02} & n/a\\
\index{LuZZYW24}\rowlabel{b:LuZZYW24}\href{../works/LuZZYW24.pdf}{LuZZYW24}~\cite{LuZZYW24} & 36 & job-shop, CP, explanation, single-machine scheduling, bi-objective, inventory, order, sustainability, machine, job, distributed, resource, transportation, precedence, scheduling, energy efficiency, multi-objective, constraint programming, Benders Decomposition, completion-time, flow-shop, constraint satisfaction, task, stochastic & Resource-constrained Project Scheduling Problem, single machine & cumulative, alwaysIn, disjunctive, noOverlap & Java & Cplex, OPL & energy-price, maintenance scheduling, train schedule, automotive, container terminal, railway & shipping industry & real-life, real-world & evolutionary computing, genetic algorithm, memetic algorithm, meta heuristic, ant colony, particle swarm, simulated annealing, large neighborhood search, column generation & \ref{a:LuZZYW24} & n/a\\
\index{LunardiBLRV20}\rowlabel{b:LunardiBLRV20}\href{../works/LunardiBLRV20.pdf}{LunardiBLRV20}~\cite{LunardiBLRV20} & 20 & job-shop, resource, job, order, tardiness, setup-time, constraint programming, preempt, make-span, unavailability, completion-time, flow-shop, CP, activity, re-scheduling, bi-objective, scheduling, due-date, machine, precedence & FJS & endBeforeStart, noOverlap & Python & Cplex & high performance computing & printing industry, glass industry & benchmark, github, random instance, generated instance & meta heuristic, genetic algorithm, large neighborhood search & \ref{a:LunardiBLRV20} & \ref{c:LunardiBLRV20}\\
\index{MalapertCGJLR12}\rowlabel{b:MalapertCGJLR12}\href{../works/MalapertCGJLR12.pdf}{MalapertCGJLR12}~\cite{MalapertCGJLR12} & 17 & constraint programming, transportation, flow-shop, CSP, order, cmax, open-shop, precedence, constraint satisfaction, completion-time, activity, machine, CP, preemptive, make-span, scheduling, resource, preempt, task, job, job-shop & OSP, Open Shop Scheduling Problem & disjunctive, cycle, Disjunctive constraint, cumulative & Java & Choco Solver &  &  & benchmark & meta heuristic, genetic algorithm, not-first, not-last, ant colony, edge-finding, particle swarm & \ref{a:MalapertCGJLR12} & n/a\\
\index{MalikMB08}\rowlabel{b:MalikMB08}\href{../works/MalikMB08.pdf}{MalikMB08}~\cite{MalikMB08} & 18 & machine, precedence, distributed, constraint programming, resource, constraint logic programming, order, scheduling &  & cycle, Cardinality constraint &  &  & pipeline &  & benchmark & edge-finding & \ref{a:MalikMB08} & n/a\\
\index{MaraveliasCG04}\rowlabel{b:MaraveliasCG04}\href{../works/MaraveliasCG04.pdf}{MaraveliasCG04}~\cite{MaraveliasCG04} & 29 & manpower, resource, task, inventory, constraint satisfaction, scheduling, order, make-span, activity, continuous-process, constraint programming, flow-shop, CP, job, re-scheduling, due-date &  & Balance constraint, cycle, disjunctive &  & Cplex, OPL &  & chemical industry &  & MINLP, edge-finding & \ref{a:MaraveliasCG04} & n/a\\
\index{MarliereSPR23}\rowlabel{b:MarliereSPR23}\href{../works/MarliereSPR23.pdf}{MarliereSPR23}~\cite{MarliereSPR23} & 22 & machine, precedence, transportation, job-shop, resource, job, order, constraint satisfaction, energy efficiency, constraint programming, distributed, multi-objective, task, explanation, CP, no-wait, activity, re-scheduling, scheduling & rtRTMP, RTMP & disjunctive, alternative constraint, cumulative, cycle, Disjunctive constraint, circuit, table constraint, noOverlap &  & Cplex & railway, robot, train schedule &  & benchmark, real-world & quadratic programming, time-tabling, Lagrangian relaxation, machine learning, meta heuristic & \ref{a:MarliereSPR23} & n/a\\
\index{MartinPY01}\rowlabel{b:MartinPY01}\href{../works/MartinPY01.pdf}{MartinPY01}~\cite{MartinPY01} & 17 & CP, order, breakdown, transportation, re-scheduling, constraint programming, scheduling, task, CLP, machine, constraint logic programming, constraint satisfaction, resource &  & circuit & Prolog & ECLiPSe, Ilog Solver & train schedule, railway, aircraft & sugar industry & real-life &  & \ref{a:MartinPY01} & n/a\\
\index{Mason01}\rowlabel{b:Mason01}\href{../works/Mason01.pdf}{Mason01}~\cite{Mason01} & 38 & cyclic scheduling, order, activity, scheduling, CP, transportation, task &  &  &  & Cplex, OPL & railway, airport, crew-scheduling, workforce scheduling, nurse & airline industry &  & Lagrangian relaxation, column generation & \ref{a:Mason01} & n/a\\
\index{MejiaY20}\rowlabel{b:MejiaY20}\href{../works/MejiaY20.pdf}{MejiaY20}~\cite{MejiaY20} & 13 & resource, cmax, sequence dependent setup, due-date, re-scheduling, preemptive, order, tardiness, scheduling, completion-time, machine, job, no-wait, open-shop, release-date, transportation, multi-agent, multi-objective, constraint programming, job-shop, bi-objective, preempt, setup-time, CP, make-span, distributed & OSSP, Open Shop Scheduling Problem, parallel machine & disjunctive, Disjunctive constraint & Java & Cplex, ECLiPSe & agriculture, robot &  & supplementary material, benchmark, generated instance & simulated annealing, ant colony, GRASP, particle swarm, genetic algorithm, meta heuristic & \ref{a:MejiaY20} & \ref{c:MejiaY20}\\
\index{MenciaSV12}\rowlabel{b:MenciaSV12}\href{../works/MenciaSV12.pdf}{MenciaSV12}~\cite{MenciaSV12} & 32 & lateness, preempt, sequence dependent setup, constraint satisfaction, resource, flow-time, preemptive, job-shop, job, constraint programming, completion-time, precedence, CP, setup-time, order, cmax, multi-objective, constraint optimization, scheduling, task, machine, make-span, distributed & JSSP, single machine & cycle, Disjunctive constraint, disjunctive &  &  & steel mill &  & benchmark, real-life & edge-finding, genetic algorithm, energetic reasoning, memetic algorithm, time-tabling, simulated annealing & \ref{a:MenciaSV12} & n/a\\
\index{MenciaSV13}\rowlabel{b:MenciaSV13}\href{../works/MenciaSV13.pdf}{MenciaSV13}~\cite{MenciaSV13} & 11 & lateness, preempt, sequence dependent setup, constraint satisfaction, resource, flow-time, preemptive, job-shop, flow-shop, job, constraint programming, completion-time, precedence, CP, setup-time, order, cmax, multi-objective, constraint optimization, scheduling, task, machine, make-span & JSSP, single machine & cycle, Disjunctive constraint, disjunctive &  &  & steel mill &  & benchmark, real-life, supplementary material & edge-finding, genetic algorithm, energetic reasoning, time-tabling, simulated annealing, meta heuristic & \ref{a:MenciaSV13} & \ref{c:MenciaSV13}\\
\index{MengGRZSC22}\rowlabel{b:MengGRZSC22}\href{../works/MengGRZSC22.pdf}{MengGRZSC22}~\cite{MengGRZSC22} & 13 & no-wait, distributed, job, job-shop, transportation, setup-time, Benders Decomposition, task, Pareto, constraint programming, bi-objective, precedence, flow-shop, multi-objective, make-span, order, machine, CP, tardiness, scheduling & parallel machine, FJS, HFS, PMSP &  &  & Cplex, Gurobi, OPL & semiconductor &  & benchmark & meta heuristic, genetic algorithm, memetic algorithm & \ref{a:MengGRZSC22} & n/a\\
\index{MengLZB21}\rowlabel{b:MengLZB21}\href{../works/MengLZB21.pdf}{MengLZB21}~\cite{MengLZB21} & 14 & flow-shop, machine, tardiness, earliness, completion-time, multi-agent, cmax, CP, distributed, job-shop, resource, transportation, order, multi-objective, setup-time, scheduling, make-span, task, job, sequence dependent setup, due-date & parallel machine, FJS, HFS & endBeforeStart, noOverlap, circuit, alternative constraint, cumulative &  & Cplex, OPL, Gurobi & semiconductor &  & benchmark & NEH, particle swarm, memetic algorithm, meta heuristic, simulated annealing, genetic algorithm & \ref{a:MengLZB21} & n/a\\
\index{MengZRZL20}\rowlabel{b:MengZRZL20}\href{../works/MengZRZL20.pdf}{MengZRZL20}~\cite{MengZRZL20} & 13 & job-shop, no-wait, flow-shop, completion-time, CP, batch process, open-shop, resource, energy efficiency, flow-time, transportation, job, constraint programming, precedence, Benders Decomposition, cyclic scheduling, task, setup-time, machine, order, cmax, multi-objective, tardiness, earliness, scheduling, preempt, sequence dependent setup, make-span, blocking constraint, distributed, no preempt & parallel machine, FJS, OSP, Open Shop Scheduling Problem, HFS & alternative constraint, Blocking constraint, noOverlap, endBeforeStart &  & OR-Tools, Gecode, OPL, Gurobi, Cplex & robot, semiconductor &  & benchmark, supplementary material & genetic algorithm, ant colony, particle swarm, simulated annealing, meta heuristic & \ref{a:MengZRZL20} & \ref{c:MengZRZL20}\\
\index{MercierH07}\rowlabel{b:MercierH07}\href{../works/MercierH07.pdf}{MercierH07}~\cite{MercierH07} & 27 & transportation, job, constraint programming, precedence, release-date, CLP, order, scheduling, task, machine, constraint satisfaction, job-shop, due-date, activity, explanation, completion-time, CP, resource & JSSP & cumulative, disjunctive &  & CHIP &  & steel industry & benchmark & not-last, edge-finder, edge-finding, energetic reasoning, not-first & \ref{a:MercierH07} & n/a\\
\index{MercierH08}\rowlabel{b:MercierH08}\href{../works/MercierH08.pdf}{MercierH08}~\cite{MercierH08} & 11 & preemptive, job, constraint programming, release-date, CLP, order, scheduling, preempt, task, job-shop, due-date, CP, resource &  & cumulative, disjunctive &  &  &  &  &  & edge-finder, edge-finding & \ref{a:MercierH08} & n/a\\
\index{MilanoW06}\rowlabel{b:MilanoW06}\href{../works/MilanoW06.pdf}{MilanoW06}~\cite{MilanoW06} & 45 & release-date, Logic-Based Benders Decomposition, distributed, one-machine scheduling, job-shop, resource, constraint logic programming, job, constraint satisfaction, preempt, setup-time, explanation, single-machine scheduling, scheduling, tardiness, task, preemptive, due-date, CP, machine, lateness, stochastic, constraint programming, transportation, Benders Decomposition, order, CSP, completion-time, activity & single machine, parallel machine & Cumulatives constraint, Reified constraint, Cardinality constraint, Channeling constraint, circuit, cumulative, alldifferent, GCC constraint &  & CHIP, ECLiPSe, Cplex, OPL & crew-scheduling &  & benchmark & column generation, edge-finder, meta heuristic, time-tabling, large neighborhood search, Lagrangian relaxation & \ref{a:MilanoW06} & n/a\\
\index{MilanoW09}\rowlabel{b:MilanoW09}\href{../works/MilanoW09.pdf}{MilanoW09}~\cite{MilanoW09} & 40 & release-date, Logic-Based Benders Decomposition, distributed, one-machine scheduling, job-shop, resource, constraint logic programming, job, constraint satisfaction, preempt, setup-time, explanation, single-machine scheduling, scheduling, tardiness, task, preemptive, due-date, CP, machine, lateness, stochastic, constraint programming, transportation, Benders Decomposition, order, CSP, completion-time, activity & single machine & Cumulatives constraint, Reified constraint, Cardinality constraint, Channeling constraint, circuit, cumulative, alldifferent, GCC constraint &  & CHIP, SCIP, ECLiPSe, Cplex, OPL & crew-scheduling &  & benchmark & column generation, edge-finder, meta heuristic, lazy clause generation, time-tabling, large neighborhood search, Lagrangian relaxation & \ref{a:MilanoW09} & n/a\\
\index{MintonJPL92}\rowlabel{b:MintonJPL92}\href{../works/MintonJPL92.pdf}{MintonJPL92}~\cite{MintonJPL92} & 45 & CSP, constraint satisfaction, distributed, explanation, job, job-shop, machine, order, periodic, re-scheduling, resource, scheduling, stochastic, task &  & cumulative, cycle & Lisp & OPL & astronomy, robot, telescope &  & benchmark, real-world & neural network & \ref{a:MintonJPL92} & n/a\\
\index{MokhtarzadehTNF20}\rowlabel{b:MokhtarzadehTNF20}\href{../works/MokhtarzadehTNF20.pdf}{MokhtarzadehTNF20}~\cite{MokhtarzadehTNF20} & 14 & CP, distributed, order, job, constraint programming, machine, task, multi-agent, setup-time, manpower, no-wait, scheduling, make-span, resource, precedence, completion-time & parallel machine & alldifferent, circuit, cycle &  & Cplex & robot, crew-scheduling & circuit boards industry & real-world, generated instance & simulated annealing, time-tabling, meta heuristic, particle swarm & \ref{a:MokhtarzadehTNF20} & n/a\\
\index{MontemanniD23}\rowlabel{b:MontemanniD23}\href{../works/MontemanniD23.pdf}{MontemanniD23}~\cite{MontemanniD23} & 13 & constraint programming, resource, order, distributed, sustainability, task, CP, scheduling, machine &  & circuit & Python & OPL, OR-Tools, Gurobi & robot, drone &  & benchmark, supplementary material & machine learning, swarm intelligence, meta heuristic, ant colony, mat heuristic & \ref{a:MontemanniD23} & \ref{c:MontemanniD23}\\
\index{MontemanniD23a}\rowlabel{b:MontemanniD23a}\href{../works/MontemanniD23a.pdf}{MontemanniD23a}~\cite{MontemanniD23a} & 20 & explanation, completion-time, task, sustainability, transportation, scheduling, order, constraint programming, CP &  & circuit & Python & OR-Tools & drone &  & benchmark & meta heuristic, mat heuristic, ant colony & \ref{a:MontemanniD23a} & \ref{c:MontemanniD23a}\\
\index{MullerMKP22}\rowlabel{b:MullerMKP22}\href{../works/MullerMKP22.pdf}{MullerMKP22}~\cite{MullerMKP22} & 18 & bi-objective, precedence, batch process, multi-objective, make-span, order, machine, preempt, breakdown, cmax, scheduling, sustainability, due-date, no-wait, job, activity, resource, preemptive, job-shop, completion-time, setup-time, CP, online scheduling, task, stochastic, constraint programming & Resource-constrained Project Scheduling Problem, FJS & circuit, disjunctive & Python, Java & Chuffed, Choco Solver, OPL, OR-Tools, Cplex, MiniZinc, Gecode & robot, semiconductor &  & random instance, benchmark, github, real-world & reinforcement learning, neural network, meta heuristic, genetic algorithm, deep learning, machine learning & \ref{a:MullerMKP22} & \ref{c:MullerMKP22}\\
\index{NaderiBZ22}\rowlabel{b:NaderiBZ22}\href{../works/NaderiBZ22.pdf}{NaderiBZ22}~\cite{NaderiBZ22} & 29 & stochastic, setup-time, open-shop, order, scheduling, machine, make-span, distributed, Logic-Based Benders Decomposition, job-shop, due-date, tardiness, flow-shop, lateness, CP, resource, transportation, no-wait, job, constraint programming, completion-time, Benders Decomposition & parallel machine, single machine & disjunctive, noOverlap, Disjunctive constraint &  & Cplex, CPO & crew-scheduling, nurse, surgery, patient, operating room, automotive &  & benchmark, real-life & meta heuristic, memetic algorithm & \ref{a:NaderiBZ22} & n/a\\
\index{NaderiBZ22a}\rowlabel{b:NaderiBZ22a}\href{../works/NaderiBZ22a.pdf}{NaderiBZ22a}~\cite{NaderiBZ22a} & 19 & job-shop, distributed, transportation, job, constraint programming, precedence, flow-shop, multi-objective, make-span, preemptive, completion-time, setup-time, CP, Benders Decomposition, Logic-Based Benders Decomposition, task, stochastic, re-scheduling, sequence dependent setup, order, machine, preempt, tardiness, scheduling, resource & parallel machine & Disjunctive constraint, noOverlap, disjunctive, endBeforeStart & C++ & CPO, Cplex & nurse, automotive, crew-scheduling, robot, operating room &  & benchmark & genetic algorithm, simulated annealing, ant colony, meta heuristic & \ref{a:NaderiBZ22a} & n/a\\
\index{NaderiBZ23}\rowlabel{b:NaderiBZ23}\href{../works/NaderiBZ23.pdf}{NaderiBZ23}~\cite{NaderiBZ23} & 32 & stochastic, setup-time, open-shop, order, scheduling, machine, make-span, distributed, Logic-Based Benders Decomposition, job-shop, due-date, tardiness, flow-shop, lateness, CP, resource, transportation, no-wait, job, constraint programming, completion-time, Benders Decomposition & parallel machine, single machine & disjunctive, noOverlap, Disjunctive constraint & Python & Cplex, CPO & crew-scheduling, nurse, surgery, patient, operating room, automotive &  & benchmark, real-world & meta heuristic, memetic algorithm & \ref{a:NaderiBZ23} & n/a\\
\index{NaderiBZR23}\rowlabel{b:NaderiBZR23}\href{../works/NaderiBZR23.pdf}{NaderiBZR23}~\cite{NaderiBZR23} & 15 & distributed, transportation, job, activity, constraint programming, precedence, completion-time, setup-time, CP, Benders Decomposition, constraint logic programming, Logic-Based Benders Decomposition, task, stochastic, re-scheduling, bi-objective, order, machine, breakdown, periodic, scheduling, resource & parallel machine & bin-packing & Python & Cplex & nurse, surgery, automotive, crew-scheduling, medical, patient, operating room &  & supplementary material, benchmark, real-world, real-life, generated instance & large neighborhood search, meta heuristic, mat heuristic & \ref{a:NaderiBZR23} & \ref{c:NaderiBZR23}\\
\index{NaderiRR23}\rowlabel{b:NaderiRR23}\href{../works/NaderiRR23.pdf}{NaderiRR23}~\cite{NaderiRR23} & 27 & tardiness, flow-shop, earliness, CP, resource, preemptive, transportation, no-wait, job, constraint programming, completion-time, precedence, Benders Decomposition, setup-time, open-shop, order, cmax, re-scheduling, bi-objective, scheduling, preempt, sequence dependent setup, task, machine, make-span, constraint satisfaction, distributed, Logic-Based Benders Decomposition, job-shop, due-date & Open Shop Scheduling Problem, PMSP, RCPSP, parallel machine, Resource-constrained Project Scheduling Problem, OSP, PTC, single machine, FJS & Disjunctive constraint, cumulative, disjunctive, noOverlap, endBeforeStart, alternative constraint & Python & CPO, Z3, Gurobi, SCIP, Cplex & operating room, automotive, crew-scheduling, airport &  & github, benchmark & genetic algorithm, large neighborhood search, meta heuristic & \ref{a:NaderiRR23} & \ref{c:NaderiRR23}\\
\index{NaqviAIAAA22}\rowlabel{b:NaqviAIAAA22}\href{../works/NaqviAIAAA22.pdf}{NaqviAIAAA22}~\cite{NaqviAIAAA22} & 18 & order, distributed, CP, multi-objective, task, constraint programming, scheduling, machine, job, unavailability & TMS & cumulative &  &  & tournament, round-robin, pipeline, sports scheduling, travelling tournament problem &  & real-life, real-world & simulated annealing, time-tabling & \ref{a:NaqviAIAAA22} & n/a\\
\index{NattafAL15}\rowlabel{b:NattafAL15}\href{../works/NattafAL15.pdf}{NattafAL15}~\cite{NattafAL15} & 21 & release-date, scheduling, preempt, task, make-span, due-date, resource, preemptive, activity, constraint programming, CSP, CP, order & CECSP, RCPSP, Resource-constrained Project Scheduling Problem, CuSP & cumulative & C++ & Cplex &  &  & generated instance & sweep, energetic reasoning & \ref{a:NattafAL15} & \ref{c:NattafAL15}\\
\index{NattafAL17}\rowlabel{b:NattafAL17}\href{../works/NattafAL17.pdf}{NattafAL17}~\cite{NattafAL17} & 18 & release-date, scheduling, task, make-span, resource, energy efficiency, activity, job, constraint programming, CSP, CP, order & CECSP & disjunctive, cumulative & C++ & Cplex &  &  & real-world & edge-finding, energetic reasoning & \ref{a:NattafAL17} & \ref{c:NattafAL17}\\
\index{NattafALR16}\rowlabel{b:NattafALR16}\href{../works/NattafALR16.pdf}{NattafALR16}~\cite{NattafALR16} & 34 & preemptive, no preempt, task, constraint programming, precedence, make-span, order, preempt, CP, scheduling, due-date, CSP, activity, explanation, resource, release-date & CECSP, CuSP, Resource-constrained Project Scheduling Problem, RCPSP & cumulative & C++ & Cplex &  &  & generated instance & sweep, energetic reasoning & \ref{a:NattafALR16} & n/a\\
\index{NattafDYW19}\rowlabel{b:NattafDYW19}\href{../works/NattafDYW19.pdf}{NattafDYW19}~\cite{NattafDYW19} & 16 & job-shop, completion-time, setup-time, stochastic, constraint satisfaction, constraint programming, make-span, order, machine, CP, cmax, periodic, single-machine scheduling, scheduling, job, resource & parallel machine, single machine, PTC & noOverlap, alternative constraint &  & OPL, Cplex & semiconductor & lumber industry, semiconductor industry & benchmark & simulated annealing, memetic algorithm, meta heuristic, genetic algorithm & \ref{a:NattafDYW19} & n/a\\
\index{NattafHKAL19}\rowlabel{b:NattafHKAL19}\href{../works/NattafHKAL19.pdf}{NattafHKAL19}~\cite{NattafHKAL19} & 16 & preempt, single-machine scheduling, order, CP, resource, CSP, task, preemptive, release-date, activity, scheduling, machine, make-span & RCPSP, single machine, CECSP, Resource-constrained Project Scheduling Problem & cumulative &  & Cplex &  &  & real-life, benchmark & energetic reasoning & \ref{a:NattafHKAL19} & n/a\\
\index{NishikawaSTT19}\rowlabel{b:NishikawaSTT19}\href{../works/NishikawaSTT19.pdf}{NishikawaSTT19}~\cite{NishikawaSTT19} & 16 & online scheduling, precedence, scheduling, make-span, preempt, activity, distributed, constraint programming, machine, re-scheduling, order, CP, resource, task, preemptive & parallel machine & alternative constraint, cumulative &  & Cplex & robot, pipeline &  & real-world, benchmark & genetic algorithm, large neighborhood search & \ref{a:NishikawaSTT19} & n/a\\
\index{NovaraNH16}\rowlabel{b:NovaraNH16}\href{../works/NovaraNH16.pdf}{NovaraNH16}~\cite{NovaraNH16} & 17 & job, constraint programming, CSP, precedence, setup-time, order, re-scheduling, tardiness, constraint logic programming, scheduling, sequence dependent setup, manpower, task, machine, make-span, constraint satisfaction, due-date, activity, completion-time, earliness, CP, batch process, resource &  & cumulative, disjunctive, noOverlap, endBeforeStart, alternative constraint &  & OPL, Cplex &  & pharmaceutical industry & benchmark, CSPlib &  & \ref{a:NovaraNH16} & n/a\\
\index{Novas19}\rowlabel{b:Novas19}\href{../works/Novas19.pdf}{Novas19}~\cite{Novas19} & 13 & scheduling, CP, precedence, cmax, job-shop, constraint programming, multi-objective, due-date, completion-time, lateness, release-date, task, tardiness, resource, make-span, flow-time, transportation, sequence dependent setup, machine, no-wait, activity, distributed, inventory, setup-time, flow-shop, constraint satisfaction, job, order & parallel machine, HFS, FJS & cycle, noOverlap, cumulative, endBeforeStart &  & OPL, Cplex & medical, semiconductor, robot, train schedule & solar cell industry & benchmark & meta heuristic, swarm intelligence, genetic algorithm, particle swarm & \ref{a:Novas19} & n/a\\
\index{NovasH10}\rowlabel{b:NovasH10}\href{../works/NovasH10.pdf}{NovasH10}~\cite{NovasH10} & 20 & unavailability, precedence, batch process, due-date, re-scheduling, order, tardiness, scheduling, completion-time, machine, job, task, no-wait, breakdown, periodic, multi-objective, constraint programming, reactive scheduling, CSP, setup-time, manpower, activity, CP, make-span, earliness, resource, lateness &  &  &  & OPL, Ilog Scheduler & pipeline &  &  & meta heuristic & \ref{a:NovasH10} & n/a\\
\index{NovasH12}\rowlabel{b:NovasH12}\href{../works/NovasH12.pdf}{NovasH12}~\cite{NovasH12} & 17 & precedence, order, scheduling, completion-time, machine, job, task, no-wait, transportation, breakdown, constraint programming, reactive scheduling, activity, CP, make-span, resource &  & cycle &  & OPL, Ilog Solver, Ilog Scheduler & hoist, electroplating, container terminal, semiconductor, robot & semiconductor industry, electroplating industry &  &  & \ref{a:NovasH12} & n/a\\
\index{NovasH14}\rowlabel{b:NovasH14}\href{../works/NovasH14.pdf}{NovasH14}~\cite{NovasH14} & 14 & unavailability, precedence, order, scheduling, completion-time, machine, job, task, transportation, multi-objective, constraint programming, job-shop, reactive scheduling, constraint satisfaction, activity, CP, make-span, buffer-capacity, resource & single machine, parallel machine &  &  & OPL, Ilog Solver, Ilog Scheduler & robot &  & benchmark & ant colony, genetic algorithm & \ref{a:NovasH14} & n/a\\
\index{NuijtenA96}\rowlabel{b:NuijtenA96}\href{../works/NuijtenA96.pdf}{NuijtenA96}~\cite{NuijtenA96} & 16 & scheduling, preempt, machine, make-span, constraint satisfaction, preemptive, job-shop, flow-shop, completion-time, CP, resource, job, constraint programming, CSP, precedence, CLP, order & JSSP & disjunctive, Disjunctive constraint &  & CPO &  &  &  & time-tabling & \ref{a:NuijtenA96} & n/a\\
\index{NuijtenP98}\rowlabel{b:NuijtenP98}\href{../works/NuijtenP98.pdf}{NuijtenP98}~\cite{NuijtenP98} & 16 & scheduling, preempt, manpower, task, machine, make-span, constraint satisfaction, preemptive, job-shop, flow-shop, completion-time, CP, resource, transportation, reactive scheduling, job, constraint programming, CSP, precedence, setup-time, CLP, single-machine scheduling, order & single machine, JSSP & disjunctive, Disjunctive constraint & C++ & Ilog Solver, OPL, Ilog Scheduler & satellite &  & real-life & simulated annealing, edge-finding & \ref{a:NuijtenP98} & n/a\\
\index{OhrimenkoSC09}\rowlabel{b:OhrimenkoSC09}\href{../works/OhrimenkoSC09.pdf}{OhrimenkoSC09}~\cite{OhrimenkoSC09} & 35 & scheduling, machine, order, constraint satisfaction, constraint programming, resource, job, completion-time, explanation, open-shop, CSP, make-span, CP & Open Shop Scheduling Problem & Reified constraint, Arithmetic constraint, alldifferent, Cardinality constraint, disjunctive &  & Gecode &  &  & benchmark & lazy clause generation & \ref{a:OhrimenkoSC09} & n/a\\
\index{OrnekO16}\rowlabel{b:OrnekO16}\href{../works/OrnekO16.pdf}{OrnekO16}~\cite{OrnekO16} & 25 & constraint satisfaction, preempt, inventory, setup-time, activity, CP, make-span, earliness, distributed, resource, precedence, cmax, due-date, preemptive, order, tardiness, scheduling, completion-time, machine, job, bill of material, release-date, BOM, multi-objective, constraint programming, job-shop & parallel machine & Disjunctive constraint, cumulative, Element constraint, disjunctive &  & Cplex, OPL &  &  & real-life, real-world & genetic algorithm, meta heuristic, neural network, edge-finding & \ref{a:OrnekO16} & n/a\\
\index{OrnekOS20}\rowlabel{b:OrnekOS20}\href{../works/OrnekOS20.pdf}{OrnekOS20}~\cite{OrnekOS20} & 29 & explanation, CP, machine, stochastic, distributed, constraint programming, resource, order, multi-objective, Pareto, periodic, scheduling & parallel machine & disjunctive, noOverlap &  & Cplex & aircraft, airport &  & generated instance, real-world & genetic algorithm, time-tabling, large neighborhood search, particle swarm, meta heuristic, simulated annealing & \ref{a:OrnekOS20} & n/a\\
\index{OzturkTHO10}\rowlabel{b:OzturkTHO10}\href{../works/OzturkTHO10.pdf}{OzturkTHO10}~\cite{OzturkTHO10} & 8 & job, activity, scheduling, resource, setup-time, task, constraint programming, precedence, make-span, order, completion-time, machine, CP, cmax & SBSFMMAL & disjunctive &  & Ilog Scheduler, OPL, Ilog Solver, Cplex & robot &  &  &  & \ref{a:OzturkTHO10} & n/a\\
\index{OzturkTHO12}\rowlabel{b:OzturkTHO12}\href{../works/OzturkTHO12.pdf}{OzturkTHO12}~\cite{OzturkTHO12} & 6 & job, activity, scheduling, resource, cyclic scheduling, job-shop, setup-time, task, constraint programming, precedence, make-span, preemptive, order, completion-time, machine, preempt, CP, distributed &  & disjunctive, Element constraint, cycle, cumulative &  & OPL, Cplex &  &  &  & edge-finding & \ref{a:OzturkTHO12} & n/a\\
\index{OzturkTHO13}\rowlabel{b:OzturkTHO13}\href{../works/OzturkTHO13.pdf}{OzturkTHO13}~\cite{OzturkTHO13} & 36 & job, activity, scheduling, resource, cyclic scheduling, setup-time, constraint logic programming, task, constraint satisfaction, constraint programming, precedence, flow-shop, make-span, preemptive, order, CLP, completion-time, machine, preempt, CP, breakdown, cmax, CSP & SBSFMMAL & Disjunctive constraint, disjunctive, Channeling constraint, cycle, cumulative &  & CHIP, OPL, Ilog Solver, Cplex &  &  & real-world, real-life & genetic algorithm, large neighborhood search, column generation, edge-finding & \ref{a:OzturkTHO13} & \ref{c:OzturkTHO13}\\
\index{OzturkTHO15}\rowlabel{b:OzturkTHO15}\href{../works/OzturkTHO15.pdf}{OzturkTHO15}~\cite{OzturkTHO15} & 12 & job, activity, scheduling, resource, cyclic scheduling, setup-time, task, inventory, constraint satisfaction, constraint programming, precedence, make-span, preemptive, order, completion-time, machine, preempt, CP, breakdown, distributed & SBSFMMAL & disjunctive, circuit, cycle, cumulative &  & OPL, Cplex &  &  & real-life & large neighborhood search & \ref{a:OzturkTHO15} & n/a\\
\index{PandeyS21a}\rowlabel{b:PandeyS21a}\href{../works/PandeyS21a.pdf}{PandeyS21a}~\cite{PandeyS21a} & 29 & constraint logic programming, scheduling, unavailability, make-span, constraint satisfaction, distributed, activity, flow-shop, completion-time, CP, resource, energy efficiency, re-scheduling, job, constraint programming, CSP, precedence, task, single-machine scheduling, machine, order & PMSP, parallel machine, single machine & cumulative, Pulse constraint, endBeforeStart, alternative constraint &  & OPL, Cplex & semiconductor &  & benchmark & quadratic programming, column generation, mat heuristic & \ref{a:PandeyS21a} & n/a\\
\index{PapaB98}\rowlabel{b:PapaB98}\href{../works/PapaB98.pdf}{PapaB98}~\cite{PapaB98} & 25 & reactive scheduling, machine, activity, task, flow-shop, resource, constraint satisfaction, job, order, scheduling, distributed, CP, CLP, cmax, setup-time, job-shop, constraint programming, due-date, preempt, re-scheduling, make-span, completion-time, CSP, preemptive & PJSSP, Resource-constrained Project Scheduling Problem, JSSP & cumulative, table constraint, Disjunctive constraint, Cardinality constraint, disjunctive & C++ & Ilog Solver, CHIP, Claire & hoist &  & benchmark & edge-finder, energetic reasoning, edge-finding & \ref{a:PapaB98} & \ref{c:PapaB98}\\
\index{Pape94}\rowlabel{b:Pape94}\href{../works/Pape94.pdf}{Pape94}~\cite{Pape94} & 34 & stochastic, multi-agent, inventory, machine, job-shop, constraint programming, task, order, constraint satisfaction, constraint logic programming, CP, activity, due-date, distributed, resource, CLP, release-date, scheduling, precedence, re-scheduling, job &  & cumulative, disjunctive & C++, Prolog, Lisp &  &  &  &  &  & \ref{a:Pape94} & n/a\\
\index{PengLC14}\rowlabel{b:PengLC14}\href{../works/PengLC14.pdf}{PengLC14}~\cite{PengLC14} & 7 & stochastic, job-shop, CP, earliness, constraint optimization, preempt, COP, CSP, setup-time, make-span, machine, job, distributed, resource, precedence, sequence dependent setup, due-date, preemptive, order, tardiness, scheduling, constraint programming, completion-time, constraint satisfaction, task & single machine &  &  &  &  &  & real-life, benchmark & genetic algorithm & \ref{a:PengLC14} & n/a\\
\index{PenzDN23}\rowlabel{b:PenzDN23}\href{../works/PenzDN23.pdf}{PenzDN23}~\cite{PenzDN23} & 13 & periodic, job-shop, CP, single-machine scheduling, earliness, preempt, COP, setup-time, activity, sustainability, make-span, machine, flow-time, job, resource, one-machine scheduling, release-date, unavailability, breakdown, preemptive, order, tardiness, scheduling, completion-time, stochastic & parallel machine, single machine &  &  & Cplex & maintenance scheduling, semiconductor & semiconductor industry &  & memetic algorithm, meta heuristic, ant colony, simulated annealing & \ref{a:PenzDN23} & n/a\\
\index{PesantGPR99}\rowlabel{b:PesantGPR99}\href{../works/PesantGPR99.pdf}{PesantGPR99}~\cite{PesantGPR99} & 11 & job-shop, transportation, constraint logic programming, task, constraint programming, make-span, order, CLP, CP, scheduling, re-scheduling, distributed, inventory, job, resource &  &  & Prolog, C++ & Ilog Solver, ECLiPSe &  &  & real-life, benchmark & time-tabling & \ref{a:PesantGPR99} & n/a\\
\index{PoderBS04}\rowlabel{b:PoderBS04}\href{../works/PoderBS04.pdf}{PoderBS04}~\cite{PoderBS04} & 16 & order, activity, release-date, resource, constraint satisfaction, preempt, scheduling, precedence, task, producer/consumer, preemptive, due-date, CP, machine, constraint programming & RCPSP & cumulative & Prolog & CHIP &  & chemical industry &  &  & \ref{a:PoderBS04} & n/a\\
\index{PohlAK22}\rowlabel{b:PohlAK22}\href{../works/PohlAK22.pdf}{PohlAK22}~\cite{PohlAK22} & 16 & job, resource, lateness, release-date, transportation, precedence, sequence dependent setup, re-scheduling, tardiness, scheduling, constraint programming, completion-time, stochastic, activity, CP, single-machine scheduling, earliness, inventory, setup-time, order, machine & single machine, SCC & cumulative, noOverlap & Python & Cplex, Gurobi & airport, aircraft &  & benchmark, real-world & simulated annealing, large neighborhood search, column generation & \ref{a:PohlAK22} & n/a\\
\index{Polo-MejiaALB20}\rowlabel{b:Polo-MejiaALB20}\href{../works/Polo-MejiaALB20.pdf}{Polo-MejiaALB20}~\cite{Polo-MejiaALB20} & 18 & setup-time, cmax, precedence, due-date, activity, machine, tardiness, CP, preemptive, make-span, scheduling, completion-time, periodic, multi-objective, resource, preempt, earliness, Benders Decomposition, task, job, constraint programming, order, release-date & RCPSP, Resource-constrained Project Scheduling Problem & alternative constraint, Calendar constraint, endBeforeStart, alwaysIn, Disjunctive constraint, cumulative, noOverlap, disjunctive & C++ & Cplex, CPO &  &  & github, Roadef & mat heuristic, meta heuristic, particle swarm & \ref{a:Polo-MejiaALB20} & \ref{c:Polo-MejiaALB20}\\
\index{PourDERB18}\rowlabel{b:PourDERB18}\href{../works/PourDERB18.pdf}{PourDERB18}~\cite{PourDERB18} & 12 & multi-objective, COP, constraint satisfaction, CP, stochastic, order, transportation, constraint optimization, job, constraint programming, CSP, scheduling, task, machine &  &  &  & OR-Tools, Cplex & maintenance scheduling, railway, crew-scheduling &  & real-world, generated instance, real-life, benchmark & genetic algorithm, meta heuristic, ant colony & \ref{a:PourDERB18} & n/a\\
\index{PrataAN23}\rowlabel{b:PrataAN23}\href{../works/PrataAN23.pdf}{PrataAN23}~\cite{PrataAN23} & 17 & order, multi-objective, activity, setup-time, release-date, no-wait, single-machine scheduling, scheduling, Logic-Based Benders Decomposition, make-span, task, bi-objective, order scheduling, job, sequence dependent setup, due-date, batch process, preempt, flow-shop, precedence, tardiness, flow-time, earliness, preemptive, completion-time, energy efficiency, online scheduling, CP, machine, lateness, re-scheduling, stochastic, inventory, distributed, constraint programming, job-shop, resource, open-shop, Benders Decomposition & single machine, Open Shop Scheduling Problem, parallel machine & circuit, cumulative &  & CHIP & dairy, robot, energy-price, aircraft & manufacturing industry & real-life, benchmark, real-world & mat heuristic, memetic algorithm, meta heuristic, machine learning, genetic algorithm, reinforcement learning, time-tabling, large neighborhood search, particle swarm & \ref{a:PrataAN23} & \ref{c:PrataAN23}\\
\index{QinDCS20}\rowlabel{b:QinDCS20}\href{../works/QinDCS20.pdf}{QinDCS20}~\cite{QinDCS20} & 18 & order, tardiness, scheduling, completion-time, machine, job, task, transportation, cmax, constraint programming, Benders Decomposition, Logic-Based Benders Decomposition, CSP, setup-time, activity, CP, make-span, explanation, resource, precedence & parallel machine & noOverlap, endBeforeStart, cycle &  & Cplex, OPL & shipping line, container terminal, yard crane & maritime industry, shipping industry & real-life, benchmark & meta heuristic, GRASP, particle swarm & \ref{a:QinDCS20} & n/a\\
\index{QinDS16}\rowlabel{b:QinDS16}\href{../works/QinDS16.pdf}{QinDS16}~\cite{QinDS16} & 19 & job, order, scheduling, CP, Logic-Based Benders Decomposition, constraint programming, explanation, stochastic, completion-time, transportation, Benders Decomposition, task, periodic, Pareto, resource, make-span, machine, activity, COP, stock level, constraint satisfaction, bi-objective & parallel machine & alwaysIn, noOverlap &  & OPL, SCIP, Cplex & sports scheduling, container terminal, crew-scheduling &  &  & simulated annealing, meta heuristic, column generation & \ref{a:QinDS16} & n/a\\
\index{QinWSLS21}\rowlabel{b:QinWSLS21}\href{../works/QinWSLS21.pdf}{QinWSLS21}~\cite{QinWSLS21} & 12 & job-shop, order, completion-time, multi-objective, single-machine scheduling, order scheduling, job, preempt, flow-shop, scheduling, make-span, two-stage scheduling, tardiness, preemptive, batch process, cmax, CP, machine, lateness & single machine &  & C++ & OPL, Cplex & tournament, agriculture, semiconductor & semiconductor industry &  & particle swarm, ant colony, memetic algorithm, meta heuristic, machine learning, genetic algorithm & \ref{a:QinWSLS21} & n/a\\
\index{RasmussenT09}\rowlabel{b:RasmussenT09}\href{../works/RasmussenT09.pdf}{RasmussenT09}~\cite{RasmussenT09} & 15 & scheduling, distributed, CP, Benders Decomposition, constraint logic programming, constraint programming, order &  &  &  & Ilog Solver, Cplex, OPL & sports scheduling, break minimization problem, travelling tournament problem, tournament, round-robin &  & benchmark & column generation, simulated annealing, time-tabling, meta heuristic & \ref{a:RasmussenT09} & n/a\\
\index{ReddyFIBKAJ11}\rowlabel{b:ReddyFIBKAJ11}\href{../works/ReddyFIBKAJ11.pdf}{ReddyFIBKAJ11}~\cite{ReddyFIBKAJ11} & 24 & distributed, task, CP, activity, CSP, scheduling, constraint optimization, precedence, machine, resource, order, tardiness, constraint satisfaction &  & table constraint, bin-packing, cycle &  &  & satellite, deep space, robot, telescope &  &  &  & \ref{a:ReddyFIBKAJ11} & n/a\\
\index{RiiseML16}\rowlabel{b:RiiseML16}\href{../works/RiiseML16.pdf}{RiiseML16}~\cite{RiiseML16} & 10 & stochastic, constraint programming, Benders Decomposition, CP, Logic-Based Benders Decomposition, job-shop, resource, job, order, activity, preempt, setup-time, scheduling, make-span, task, due-date & Resource-constrained Project Scheduling Problem, RCPSP & bin-packing &  & Cplex & nurse, sports scheduling, operating room, surgery, patient, medical &  & real-life, real-world & meta heuristic, column generation, genetic algorithm & \ref{a:RiiseML16} & n/a\\
\index{RodosekWH99}\rowlabel{b:RodosekWH99}\href{../works/RodosekWH99.pdf}{RodosekWH99}~\cite{RodosekWH99} & 25 & constraint programming, CLP, CP, constraint logic programming, task, constraint satisfaction, order, machine, scheduling, resource &  & disjunctive, cycle, Disjunctive constraint & Prolog & Cplex, ECLiPSe & hoist, pipeline, crew-scheduling &  & benchmark &  & \ref{a:RodosekWH99} & n/a\\
\index{Rodriguez07}\rowlabel{b:Rodriguez07}\href{../works/Rodriguez07.pdf}{Rodriguez07}~\cite{Rodriguez07} & 15 & blocking constraint, job, scheduling, explanation, resource, due-date, job-shop, transportation, task, activity, constraint programming, precedence, preemptive, preempt, CP, order &  & Disjunctive constraint, disjunctive, circuit, Blocking constraint &  & Ilog Solver, Z3, Ilog Scheduler, Cplex & railway, satellite, train schedule &  & real-life & GRASP, meta heuristic & \ref{a:Rodriguez07} & n/a\\
\index{RodriguezDG02}\rowlabel{b:RodriguezDG02}\href{../works/RodriguezDG02.pdf}{RodriguezDG02}~\cite{RodriguezDG02} & 10 & constraint programming, resource, order, completion-time, CP, activity, scheduling, transportation &  & circuit, disjunctive &  &  & railway, train schedule &  &  & edge-finding & \ref{a:RodriguezDG02} & n/a\\
\index{RoePS05}\rowlabel{b:RoePS05}\href{../works/RoePS05.pdf}{RoePS05}~\cite{RoePS05} & 15 & setup-time, job-shop, constraint programming, due-date, continuous-process, re-scheduling, constraint logic programming, tardiness, make-span, lateness, machine, inventory, task, flow-shop, periodic, resource, constraint satisfaction, job, order, batch process, scheduling, distributed, precedence, CLP &  & disjunctive, Balance constraint, cumulative, Disjunctive constraint & Prolog & ECLiPSe, CHIP & maintenance scheduling &  &  & time-tabling, MINLP & \ref{a:RoePS05} & n/a\\
\index{RoshanaeiBAUB20}\rowlabel{b:RoshanaeiBAUB20}\href{../works/RoshanaeiBAUB20.pdf}{RoshanaeiBAUB20}~\cite{RoshanaeiBAUB20} & 19 & activity, machine, stochastic, sequence dependent setup, CP, scheduling, Logic-Based Benders Decomposition, resource, order, Benders Decomposition, constraint logic programming, job, job-shop, setup-time, bi-objective, constraint programming, distributed, re-scheduling & parallel machine & bin-packing, noOverlap, disjunctive & C++ & Cplex & operating room, nurse, patient, surgery &  & real-world, benchmark, generated instance & genetic algorithm, column generation, meta heuristic & \ref{a:RoshanaeiBAUB20} & n/a\\
\index{RoshanaeiLAU17}\rowlabel{b:RoshanaeiLAU17}\href{../works/RoshanaeiLAU17.pdf}{RoshanaeiLAU17}~\cite{RoshanaeiLAU17} & 17 & Benders Decomposition, constraint logic programming, stochastic, Logic-Based Benders Decomposition, breakdown, resource, task, job-shop, tardiness, sequence dependent setup, transportation, scheduling, order, make-span, release-date, setup-time, distributed, constraint programming, machine, CP, job, re-scheduling & single machine, parallel machine & bin-packing &  & Cplex, Gurobi & operating room, patient, medical, surgery, nurse &  & real-world & meta heuristic, column generation & \ref{a:RoshanaeiLAU17} & n/a\\
\index{RoshanaeiN21}\rowlabel{b:RoshanaeiN21}\href{../works/RoshanaeiN21.pdf}{RoshanaeiN21}~\cite{RoshanaeiN21} & 14 & due-date, distributed, job, online scheduling, re-scheduling, CP, Benders Decomposition, constraint logic programming, stochastic, setup-time, constraint programming, order, Logic-Based Benders Decomposition, job-shop, resource, completion-time, scheduling, machine, explanation, inventory & parallel machine, OSSP & cumulative, noOverlap &  & Cplex, CPO & automotive, operating room, patient, surgery &  & benchmark & genetic algorithm, meta heuristic, column generation & \ref{a:RoshanaeiN21} & n/a\\
\index{RuggieroBBMA09}\rowlabel{b:RuggieroBBMA09}\href{../works/RuggieroBBMA09.pdf}{RuggieroBBMA09}~\cite{RuggieroBBMA09} & 14 & Logic-Based Benders Decomposition, CP, resource, energy efficiency, CSP, constraint satisfaction, precedence, task, Pareto, activity, distributed, machine, scheduling, order, Benders Decomposition, preempt, setup-time, constraint programming &  & circuit, cumulative, cycle &  & Ilog Solver, Cplex, Ilog Scheduler & pipeline, satellite &  & instance generator, real-life & genetic algorithm & \ref{a:RuggieroBBMA09} & n/a\\
\index{SacramentoSP20}\rowlabel{b:SacramentoSP20}\href{../works/SacramentoSP20.pdf}{SacramentoSP20}~\cite{SacramentoSP20} & 33 & CSP, precedence, task, open-shop, activity, distributed, machine, flow-shop, multi-objective, transportation, scheduling, order, make-span, preempt, stochastic, constraint programming, CP, completion-time, resource, job, preemptive, job-shop & parallel machine, Resource-constrained Project Scheduling Problem, Open Shop Scheduling Problem & alternative constraint, endBeforeStart, noOverlap, cumulative, disjunctive & Java & Cplex, CPO & container terminal & shipping industry, maritime industry & benchmark, real-life, zenodo, real-world & particle swarm, simulated annealing, meta heuristic, large neighborhood search, mat heuristic, genetic algorithm, machine learning & \ref{a:SacramentoSP20} & \ref{c:SacramentoSP20}\\
\index{SadehF96}\rowlabel{b:SadehF96}\href{../works/SadehF96.pdf}{SadehF96}~\cite{SadehF96} & 41 & inventory, CSP, activity, CP, make-span, resource, precedence, due-date, re-scheduling, order, tardiness, scheduling, constraint satisfaction, machine, job, task, release-date, multi-agent, stochastic, job-shop &  & Disjunctive constraint, cycle, circuit, disjunctive & Lisp, C++ &  & robot, aircraft &  & benchmark &  & \ref{a:SadehF96} & n/a\\
\index{SadykovW06}\rowlabel{b:SadykovW06}\href{../works/SadykovW06.pdf}{SadykovW06}~\cite{SadykovW06} & 9 & due-date, constraint programming, completion-time, CP, job, release-date, scheduling, CLP, machine, one-machine scheduling, lateness & parallel machine, single machine & disjunctive, Disjunctive constraint &  & CHIP & robot &  & generated instance & Lagrangian relaxation, column generation & \ref{a:SadykovW06} & n/a\\
\index{SakkoutW00}\rowlabel{b:SakkoutW00}\href{../works/SakkoutW00.pdf}{SakkoutW00}~\cite{SakkoutW00} & 30 & distributed, preemptive, job-shop, activity, precedence, CP, single-machine scheduling, order, transportation, re-scheduling, reactive scheduling, job, constraint programming, CSP, scheduling, task, machine, preempt, constraint satisfaction, resource & single machine, KRFP & bin-packing, disjunctive, Disjunctive constraint, Arithmetic constraint &  & Cplex, CHIP & emergency service, aircraft &  & benchmark, real-world & edge-finder, edge-finding, genetic algorithm, simulated annealing & \ref{a:SakkoutW00} & \ref{c:SakkoutW00}\\
\index{Salido10}\rowlabel{b:Salido10}\href{../works/Salido10.pdf}{Salido10}~\cite{Salido10} & 4 & COP, CP, CSP, activity, constraint programming, constraint satisfaction, distributed, flow-time, job, job-shop, machine, multi-agent, multi-objective, order, preempt, preemptive, resource, scheduling, tardiness, task, transportation &  & cycle &  & OPL & medical, patient, robot & software industry & benchmark, real-life, real-world & genetic algorithm, time-tabling & \ref{a:Salido10} & n/a\\
\index{SchausHMCMD11}\rowlabel{b:SchausHMCMD11}\href{../works/SchausHMCMD11.pdf}{SchausHMCMD11}~\cite{SchausHMCMD11} & 23 & stochastic, periodic, task, CP, CSP, constraint optimization, constraint programming, constraint logic programming, order & SCC & Element constraint, Cardinality constraint, bin-packing, GCC constraint &  &  & steel mill & steel industry & CSPlib, generated instance, benchmark & meta heuristic, large neighborhood search & \ref{a:SchausHMCMD11} & \ref{c:SchausHMCMD11}\\
\index{SchildW00}\rowlabel{b:SchildW00}\href{../works/SchildW00.pdf}{SchildW00}~\cite{SchildW00} & 23 & periodic, scheduling, task, constraint logic programming, job, flow-shop, machine, explanation, precedence, CLP, completion-time, distributed, constraint programming, job-shop, resource, order, constraint satisfaction & single machine & disjunctive, Disjunctive constraint, bin-packing, cycle, Reified constraint &  & Ilog Solver & automotive & automotive industry, aerospace industry &  & edge-finding, time-tabling & \ref{a:SchildW00} & \ref{c:SchildW00}\\
\index{SchnellH15}\rowlabel{b:SchnellH15}\href{../works/SchnellH15.pdf}{SchnellH15}~\cite{SchnellH15} & 21 & preempt, resource, preemptive, activity, explanation, precedence, CP, job, constraint programming, scheduling, machine, make-span, net present value, cmax & psplib, RCPSP, Resource-constrained Project Scheduling Problem & cycle, cumulative &  & SCIP & automotive & IT industry & real-life, benchmark, supplementary material & simulated annealing, lazy clause generation, meta heuristic, GRASP & \ref{a:SchnellH15} & \ref{c:SchnellH15}\\
\index{SchnellH17}\rowlabel{b:SchnellH17}\href{../works/SchnellH17.pdf}{SchnellH17}~\cite{SchnellH17} & 11 & preempt, resource, preemptive, explanation, precedence, CP, order, job, constraint programming, scheduling, machine, make-span, net present value & psplib, RCPSP, Resource-constrained Project Scheduling Problem & cumulative & Java & SCIP &  &  & benchmark, supplementary material & genetic algorithm, lazy clause generation, meta heuristic, GRASP & \ref{a:SchnellH17} & \ref{c:SchnellH17}\\
\index{SchuttFSW11}\rowlabel{b:SchuttFSW11}\href{../works/SchuttFSW11.pdf}{SchuttFSW11}~\cite{SchuttFSW11} & 33 & scheduling, explanation, resource, CSP, task, activity, constraint programming, precedence, make-span, preemptive, completion-time, machine, preempt, CP, periodic, constraint satisfaction, open-shop, order & psplib, Resource-constrained Project Scheduling Problem, RCPSP & Disjunctive constraint, span constraint, disjunctive, circuit, cumulative &  & ECLiPSe, CHIP, Ilog Scheduler, SICStus &  &  & benchmark, real-world & not-last, lazy clause generation, not-first, edge-finding, edge-finder & \ref{a:SchuttFSW11} & \ref{c:SchuttFSW11}\\
\index{SchuttFSW13}\rowlabel{b:SchuttFSW13}\href{../works/SchuttFSW13.pdf}{SchuttFSW13}~\cite{SchuttFSW13} & 17 & scheduling, explanation, resource, task, activity, constraint programming, precedence, release-date, preemptive, CLP, machine, setup-time, preempt, CP, cmax, constraint satisfaction, order & psplib, Resource-constrained Project Scheduling Problem, RCPSP, SCC & disjunctive, cycle, cumulative, Reified constraint & C++ & CHIP &  &  & supplementary material, benchmark & genetic algorithm, lazy clause generation, meta heuristic & \ref{a:SchuttFSW13} & \ref{c:SchuttFSW13}\\
\index{ShaikhK23}\rowlabel{b:ShaikhK23}\href{../works/ShaikhK23.pdf}{ShaikhK23}~\cite{ShaikhK23} & 12 & job, unavailability, machine, constraint programming, CLP, CP, re-scheduling, distributed, job-shop, resource, open-shop, order, activity, constraint satisfaction, scheduling, task &  &  &  &  & medical, drone &  & benchmark, real-world & genetic algorithm, time-tabling, meta heuristic, machine learning & \ref{a:ShaikhK23} & \ref{c:ShaikhK23}\\
\index{ShinBBHO18}\rowlabel{b:ShinBBHO18}\href{../works/ShinBBHO18.pdf}{ShinBBHO18}~\cite{ShinBBHO18} & 16 & order, transportation, job, scheduling, task, machine, preempt, resource, activity, stochastic, inventory &  &  &  &  & physician, nurse, patient, medical &  & github, real-world &  & \ref{a:ShinBBHO18} & \ref{c:ShinBBHO18}\\
\index{Siala15}\rowlabel{b:Siala15}\href{../works/Siala15.pdf}{Siala15}~\cite{Siala15} & 2 & sequence dependent setup, machine, activity, setup-time, job, open-shop, order, scheduling, CP, precedence, cmax, job-shop, constraint programming, explanation, due-date, earliness, task, tardiness, resource, make-span & single machine, OSP, RCPSP, TMS & AmongSeq constraint, circuit, alldifferent, Balance constraint, cumulative, table constraint, GCC constraint, AtMostSeqCard, Reified constraint, Regular constraint, Among constraint, Atmost constraint, Disjunctive constraint, Cardinality constraint, cycle, MultiAtMostSeqCard, disjunctive, CardPath, AtMostSeq &  & Ilog Solver, CHIP, Claire, OPL, Mistral & rectangle-packing, automotive &  & github, Roadef, CSPlib, real-world, benchmark, random instance & edge-finding, GRASP, time-tabling & \ref{a:Siala15} & \ref{c:Siala15}\\
\index{SimoninAHL15}\rowlabel{b:SimoninAHL15}\href{../works/SimoninAHL15.pdf}{SimoninAHL15}~\cite{SimoninAHL15} & 23 & CP, resource, task, constraint programming, precedence, periodic, activity, scheduling, transportation, make-span, preempt, order, inventory &  & disjunctive, span constraint, cycle, cumulative &  & CHIP & satellite, pipeline, earth observation, robot &  &  & sweep & \ref{a:SimoninAHL15} & \ref{c:SimoninAHL15}\\
\index{Simonis07}\rowlabel{b:Simonis07}\href{../works/Simonis07.pdf}{Simonis07}~\cite{Simonis07} & 30 & CLP, CSP, make to order, CP, re-scheduling, constraint programming, job-shop, resource, transportation, order, activity, constraint satisfaction, setup-time, release-date, periodic, scheduling, task, producer/consumer, bill of material, constraint logic programming, job, sequence dependent setup, due-date, batch process, machine &  & GCC constraint, Atmost constraint, diffn, Cardinality constraint, Cumulatives constraint, disjunctive, bin-packing, Among constraint, cumulative, alldifferent, cycle & Prolog & OPL, Ilog Scheduler, CHIP & aircraft, airport, patient, medical, round-robin, business process, nurse &  &  & sweep, bi-partite matching, meta heuristic, time-tabling & \ref{a:Simonis07} & \ref{c:Simonis07}\\
\index{SimonisCK00}\rowlabel{b:SimonisCK00}\href{../works/SimonisCK00.pdf}{SimonisCK00}~\cite{SimonisCK00} & 7 & activity, constraint programming, CP, task, stock level, order, machine, producer/consumer, scheduling, resource, transportation &  & disjunctive, bin-packing, circuit, cumulative, diffn, cycle & C++, Prolog & CHIP & business process, crew-scheduling, aircraft & food industry &  &  & \ref{a:SimonisCK00} & n/a\\
\index{SmithBHW96}\rowlabel{b:SmithBHW96}\href{../works/SmithBHW96.pdf}{SmithBHW96}~\cite{SmithBHW96} & 20 & constraint satisfaction, resource, order, constraint programming, CSP, task, CLP, constraint logic programming &  &  & C++ & OPL, Ilog Solver &  &  & real-life &  & \ref{a:SmithBHW96} & n/a\\
\index{SourdN00}\rowlabel{b:SourdN00}\href{../works/SourdN00.pdf}{SourdN00}~\cite{SourdN00} & 12 & CP, make-span, resource, precedence, cmax, preemptive, order, scheduling, completion-time, constraint satisfaction, machine, job, open-shop, release-date, job-shop, flow-shop, preempt, setup-time & single machine, JSSP & cumulative, disjunctive, Disjunctive constraint &  & Ilog Scheduler & robot &  & real-life, benchmark & not-first, edge-finding, genetic algorithm & \ref{a:SourdN00} & n/a\\
\index{SubulanC22}\rowlabel{b:SubulanC22}\href{../works/SubulanC22.pdf}{SubulanC22}~\cite{SubulanC22} & 38 & tardiness, preempt, resource, preemptive, due-date, activity, completion-time, precedence, CP, stochastic, inventory, order, BOM, breakdown, transportation, constraint programming, scheduling, task, machine, make-span, multi-objective & RCPSP, Resource-constrained Project Scheduling Problem & endBeforeStart, cumulative &  & Cplex, OPL & business process, offshore &  & real-world, real-life, benchmark & mat heuristic, genetic algorithm, meta heuristic, ant colony, particle swarm & \ref{a:SubulanC22} & n/a\\
\index{SunTB19}\rowlabel{b:SunTB19}\href{../works/SunTB19.pdf}{SunTB19}~\cite{SunTB19} & 12 & preempt, constraint satisfaction, job, order, scheduling, CP, precedence, Logic-Based Benders Decomposition, explanation, tardiness, completion-time, transportation, CSP, Benders Decomposition, task, resource, make-span &  &  &  & Cplex & container terminal, yard crane & maritime industry & generated instance, instance generator, benchmark, github, real-life & simulated annealing, meta heuristic, genetic algorithm & \ref{a:SunTB19} & \ref{c:SunTB19}\\
\index{SureshMOK06}\rowlabel{b:SureshMOK06}\href{../works/SureshMOK06.pdf}{SureshMOK06}~\cite{SureshMOK06} & 19 & task, stochastic, order, machine, CP, scheduling, buffer-capacity, distributed, job &  & cycle, cumulative &  & Z3 & tournament &  &  & genetic algorithm, machine learning & \ref{a:SureshMOK06} & n/a\\
\index{TanZWGQ19}\rowlabel{b:TanZWGQ19}\href{../works/TanZWGQ19.pdf}{TanZWGQ19}~\cite{TanZWGQ19} & 10 & resource, transportation, Benders Decomposition, order, completion-time, multi-objective, setup-time, periodic, task, job, flow-shop, Pareto, scheduling, precedence, make-span, two-stage scheduling, due-date, batch process, multi-agent, CP, machine, constraint programming & SCC, parallel machine & noOverlap & C++ & Cplex & robot, automotive &  & real-world, generated instance, supplementary material & meta heuristic, particle swarm & \ref{a:TanZWGQ19} & \ref{c:TanZWGQ19}\\
\index{TangLWSK18}\rowlabel{b:TangLWSK18}\href{../works/TangLWSK18.pdf}{TangLWSK18}~\cite{TangLWSK18} & 28 & preempt, constraint satisfaction, resource, preemptive, activity, job, constraint programming, CP, stochastic, order, multi-objective, transportation, re-scheduling, CSP, scheduling, task & RCPSP & cycle, circuit & C  & Cplex, OPL & pipeline, crew-scheduling, railway &  &  & meta heuristic, genetic algorithm, neural network, particle swarm & \ref{a:TangLWSK18} & n/a\\
\index{TerekhovDOB12}\rowlabel{b:TerekhovDOB12}\href{../works/TerekhovDOB12.pdf}{TerekhovDOB12}~\cite{TerekhovDOB12} & 15 & order scheduling, constraint programming, cmax, resource, periodic, job, Benders Decomposition, completion-time, tardiness, Logic-Based Benders Decomposition, flow-shop, earliness, open-shop, due-date, distributed, preempt, make-span, precedence, single-stage scheduling, inventory, CP, activity, job-shop, scheduling, release-date, machine, lateness, order, constraint satisfaction, single-machine scheduling & parallel machine, RCPSP, Resource-constrained Project Scheduling Problem, single machine & cumulative, Balance constraint, alldifferent, disjunctive & C++ & Ilog Scheduler, Ilog Solver, Cplex & robot &  & real-life & meta heuristic, genetic algorithm & \ref{a:TerekhovDOB12} & n/a\\
\index{TerekhovTDB14}\rowlabel{b:TerekhovTDB14}\href{../works/TerekhovTDB14.pdf}{TerekhovTDB14}~\cite{TerekhovTDB14} & 38 & flow-shop, distributed, no preempt, preempt, make-span, task, preemptive, inventory, CP, activity, re-scheduling, job-shop, scheduling, flow-time, release-date, machine, order, constraint programming, stochastic, cmax, resource, periodic, job, completion-time, tardiness, buffer-capacity, online scheduling & single machine, parallel machine &  &  & Cplex, Ilog Scheduler & round-robin, semiconductor, robot &  & real-world & meta heuristic, genetic algorithm & \ref{a:TerekhovTDB14} & n/a\\
\index{ThiruvadyWGS14}\rowlabel{b:ThiruvadyWGS14}\href{../works/ThiruvadyWGS14.pdf}{ThiruvadyWGS14}~\cite{ThiruvadyWGS14} & 34 & explanation, breakdown, precedence, task, make-span, activity, tardiness, distributed, constraint programming, machine, job, scheduling, order, net present value, stochastic, CP, completion-time, resource & single machine, Resource-constrained Project Scheduling Problem, psplib & cumulative &  &  &  & mining industry & benchmark & meta heuristic, Lagrangian relaxation, genetic algorithm, ant colony, simulated annealing, machine learning & \ref{a:ThiruvadyWGS14} & n/a\\
\index{Timpe02}\rowlabel{b:Timpe02}\href{../works/Timpe02.pdf}{Timpe02}~\cite{Timpe02} & 18 & breakdown, inventory, task, resource, make-span, order, machine, activity, stock level, setup-time, job, scheduling, constraint logic programming, CP, producer/consumer, constraint programming, due-date &  & Balance constraint, cumulative, cycle, diffn, disjunctive & C++ & CHIP, Cplex &  & chemical industry, process industry &  &  & \ref{a:Timpe02} & n/a\\
\index{TopalogluO11}\rowlabel{b:TopalogluO11}\href{../works/TopalogluO11.pdf}{TopalogluO11}~\cite{TopalogluO11} & 10 & scheduling, re-scheduling, CP, multi-objective, preempt, order, distributed, constraint logic programming, task, constraint programming, preemptive, transportation, constraint satisfaction &  &  &  & Cplex, OPL, Ilog Solver & nurse, physician, emergency service, patient, surgery, medical &  & real-life & column generation, time-tabling & \ref{a:TopalogluO11} & n/a\\
\index{TorresL00}\rowlabel{b:TorresL00}\href{../works/TorresL00.pdf}{TorresL00}~\cite{TorresL00} & 12 & precedence, constraint programming, CSP, CP, job-shop, resource, order, constraint satisfaction, preempt, release-date, scheduling, make-span, task, job, preemptive, machine & single machine, JSSP & disjunctive, cumulative, cycle & C++ &  & robot &  & benchmark & not-last, energetic reasoning, not-first & \ref{a:TorresL00} & n/a\\
\index{TranAB16}\rowlabel{b:TranAB16}\href{../works/TranAB16.pdf}{TranAB16}~\cite{TranAB16} & 13 & sequence dependent setup, due-date, order, tardiness, scheduling, completion-time, machine, job, release-date, cmax, constraint programming, Benders Decomposition, Logic-Based Benders Decomposition, stochastic, setup-time, CP, make-span, explanation, single-machine scheduling, constraint logic programming, resource, precedence & PMSP, single machine, parallel machine & cycle, circuit &  & SCIP, Gurobi, Cplex & aircraft &  & benchmark & simulated annealing, meta heuristic, ant colony, genetic algorithm, column generation & \ref{a:TranAB16} & n/a\\
\index{TranPZLDB18}\rowlabel{b:TranPZLDB18}\href{../works/TranPZLDB18.pdf}{TranPZLDB18}~\cite{TranPZLDB18} & 17 & machine, preempt, periodic, scheduling, resource, distributed, online scheduling, job, energy efficiency, make-span, preemptive, completion-time, CP, task, stochastic, re-scheduling, order & single machine & bin-packing & C++ & Cplex & high performance computing &  & generated instance, benchmark & machine learning & \ref{a:TranPZLDB18} & n/a\\
\index{TranVNB17}\rowlabel{b:TranVNB17}\href{../works/TranVNB17.pdf}{TranVNB17}~\cite{TranVNB17} & 68 & Benders Decomposition, order, multi-objective, activity, Logic-Based Benders Decomposition, resource, constraint logic programming, job, scheduling, precedence, task, unavailability, multi-agent, CP, machine, re-scheduling, constraint programming, transportation &  & alternative constraint, Cardinality constraint, noOverlap, cumulative &  & OPL, MiniZinc, Cplex & robot, medical, satellite &  & real-world &  & \ref{a:TranVNB17} & n/a\\
\index{TrojetHL11}\rowlabel{b:TrojetHL11}\href{../works/TrojetHL11.pdf}{TrojetHL11}~\cite{TrojetHL11} & 7 & job-shop, activity, job, constraint programming, completion-time, CSP, precedence, CP, order, scheduling, task, machine, make-span, constraint satisfaction, distributed, due-date, constraint logic programming, resource & RCPSP & cumulative, disjunctive, diffn, cycle, alldifferent & Prolog & CHIP, SICStus & robot &  & real-world &  & \ref{a:TrojetHL11} & n/a\\
\index{Tsang03}\rowlabel{b:Tsang03}\href{../works/Tsang03.pdf}{Tsang03}~\cite{Tsang03} & 2 & resource, constraint programming, explanation, constraint satisfaction, scheduling &  &  &  &  &  &  & real-life & time-tabling & \ref{a:Tsang03} & n/a\\
\index{UnsalO13}\rowlabel{b:UnsalO13}\href{../works/UnsalO13.pdf}{UnsalO13}~\cite{UnsalO13} & 15 & constraint programming, CP, preempt, CSP, activity, make-span, machine, resource, unavailability, precedence, due-date, preemptive, order, scheduling, completion-time, job, task, transportation & parallel machine & alldifferent, cumulative, disjunctive &  & Ilog Scheduler, Cplex & yard crane, hoist, container terminal &  & real-life, instance generator, real-world, benchmark & GRASP, genetic algorithm, meta heuristic, large neighborhood search & \ref{a:UnsalO13} & n/a\\
\index{UnsalO19}\rowlabel{b:UnsalO19}\href{../works/UnsalO19.pdf}{UnsalO19}~\cite{UnsalO19} & 19 & constraint programming, CP, Logic-Based Benders Decomposition, preempt, COP, activity, explanation, machine, resource, release-date, re-scheduling, preemptive, order, BOM, scheduling, Benders Decomposition, completion-time, stock level, task, transportation & parallel machine & endBeforeStart, disjunctive &  & OPL, Cplex & container terminal &  & real-world, random instance & genetic algorithm, column generation & \ref{a:UnsalO19} & n/a\\
\index{VilimBC05}\rowlabel{b:VilimBC05}\href{../works/VilimBC05.pdf}{VilimBC05}~\cite{VilimBC05} & 23 & setup-time, scheduling, make-span, task, job, sequence dependent setup, batch process, machine, precedence, CLP, completion-time, CP, distributed, constraint programming, job-shop, resource, open-shop, order, activity &  & disjunctive, cumulative, cycle &  &  &  &  & benchmark, real-life & edge-finding, not-first, not-last, sweep & \ref{a:VilimBC05} & \ref{c:VilimBC05}\\
\index{VlkHT21}\rowlabel{b:VlkHT21}\href{../works/VlkHT21.pdf}{VlkHT21}~\cite{VlkHT21} & 14 & scheduling, CP, Logic-Based Benders Decomposition, constraint programming, explanation, tardiness, stochastic, due-date, completion-time, Benders Decomposition, periodic, online scheduling, resource, no-wait, distributed, precedence, bi-objective, order & PMSP & noOverlap, alternative constraint &  & OPL, Cplex, Gurobi, Z3 & automotive, robot &  & benchmark, industrial partner, random instance, github & GRASP & \ref{a:VlkHT21} & \ref{c:VlkHT21}\\
\index{Wallace96}\rowlabel{b:Wallace96}\href{../works/Wallace96.pdf}{Wallace96}~\cite{Wallace96} & 30 & activity, constraint satisfaction, distributed, task, resource, constraint logic programming, job, explanation, scheduling, reactive scheduling, CSP, multi-agent, CP, machine, stochastic, constraint programming, job-shop, transportation, CLP, Benders Decomposition, order &  & circuit, disjunctive, cycle & Lisp, Prolog & CHIP, ECLiPSe, Ilog Solver, OPL & robot, train schedule, airport, railway, telescope, automotive, aircraft & automotive industry, process industry &  & column generation, genetic algorithm, neural network, Lagrangian relaxation, simulated annealing, time-tabling & \ref{a:Wallace96} & \ref{c:Wallace96}\\
\index{WallaceY20}\rowlabel{b:WallaceY20}\href{../works/WallaceY20.pdf}{WallaceY20}~\cite{WallaceY20} & 19 & job-shop, flow-shop, explanation, CP, resource, transportation, bi-objective, job, constraint programming, Benders Decomposition, cyclic scheduling, task, machine, order, scheduling, Logic-Based Benders Decomposition & CHSP & circuit, Disjunctive constraint, cumulative, disjunctive, cycle &  & Gecode, MiniZinc, Chuffed, OPL, Gurobi, Cplex & hoist, electroplating, container terminal, robot, yard crane &  & real-world, benchmark, random instance, real-life & edge-finding, genetic algorithm, time-tabling, lazy clause generation, meta heuristic & \ref{a:WallaceY20} & \ref{c:WallaceY20}\\
\index{WangMD15}\rowlabel{b:WangMD15}\href{../works/WangMD15.pdf}{WangMD15}~\cite{WangMD15} & 13 & stochastic, activity, job-shop, CP, CSP, order, make-span, job, resource, precedence, cmax, re-scheduling, scheduling, multi-objective, constraint programming, completion-time, constraint satisfaction, task, no-wait &  & cumulative, noOverlap &  & OPL, Cplex & nurse, operating room, physician, medical, patient, surgery &  & real-life, real-world & mat heuristic, particle swarm, time-tabling, column generation & \ref{a:WangMD15} & n/a\\
\index{WeilHFP95}\rowlabel{b:WeilHFP95}\href{../works/WeilHFP95.pdf}{WeilHFP95}~\cite{WeilHFP95} & 6 & task, resource, job, scheduling, CSP, constraint programming, job-shop, order, constraint satisfaction &  & cycle, Cardinality constraint & Lisp, Prolog, C++ & CHIP, OPL & patient, medical, nurse &  &  &  & \ref{a:WeilHFP95} & n/a\\
\index{WessenCSFPM23}\rowlabel{b:WessenCSFPM23}\href{../works/WessenCSFPM23.pdf}{WessenCSFPM23}~\cite{WessenCSFPM23} & 34 & precedence, blocking constraint, CP, no-wait, activity, scheduling, multi-agent, job-shop, constraint programming, resource, periodic, job, order, completion-time, flow-shop, distributed, cyclic scheduling, make-span, task &  & Blocking constraint, diffn, regular expression, alternative constraint, Regular constraint, cumulative, cycle, circuit &  & Gecode, MiniZinc & hoist, satellite, robot &  & real-world, benchmark, github &  & \ref{a:WessenCSFPM23} & \ref{c:WessenCSFPM23}\\
\index{WikarekS19}\rowlabel{b:WikarekS19}\href{../works/WikarekS19.pdf}{WikarekS19}~\cite{WikarekS19} & 22 & job, constraint programming, CSP, precedence, task, setup-time, CLP, machine, order, cmax, constraint logic programming, multi-agent, scheduling, preempt, manpower, make-span, constraint satisfaction, resource, distributed, preemptive, job-shop, flow-shop, CP, inventory & JSSP, RCPSP & cumulative, disjunctive &  & Z3, SCIP, ECLiPSe & robot &  &  & meta heuristic & \ref{a:WikarekS19} & n/a\\
\index{WuBB09}\rowlabel{b:WuBB09}\href{../works/WuBB09.pdf}{WuBB09}~\cite{WuBB09} & 9 & distributed, resource, job, constraint optimization, single-machine scheduling, CSP, scheduling, precedence, constraint satisfaction, stochastic, machine, job-shop, constraint programming, task, order, completion-time, CP, lateness, activity, flow-time, transportation & single machine & Channeling constraint, cumulative &  & Ilog Solver & railway, crew-scheduling &  & real-world &  & \ref{a:WuBB09} & n/a\\
\index{YounespourAKE19}\rowlabel{b:YounespourAKE19}\href{../works/YounespourAKE19.pdf}{YounespourAKE19}~\cite{YounespourAKE19} & 11 & re-scheduling, resource, inventory, order, cmax, precedence, constraint programming, Pareto, scheduling, completion-time, multi-objective, activity, machine, stochastic, CP, distributed, make-span &  & cumulative, noOverlap, alternative constraint, span constraint &  & OPL, Z3 & nurse, surgery, medical, operating room, patient &  & real-life, real-world & MINLP, ant colony & \ref{a:YounespourAKE19} & n/a\\
\index{YunusogluY22}\rowlabel{b:YunusogluY22}\href{../works/YunusogluY22.pdf}{YunusogluY22}~\cite{YunusogluY22} & 18 & constraint programming, order, release-date, bi-objective, lateness, precedence, sequence dependent setup, job-shop, resource, batch process, cmax, flow-time, completion-time, earliness, scheduling, machine, transportation, tardiness, make-span, unavailability, activity, setup-time, preempt, inventory, due-date, job, re-scheduling, CP, multi-objective, breakdown & PMSP, parallel machine & bin-packing, cumulative, noOverlap, endBeforeStart &  & OPL, Cplex & robot, medical & insulation industry & benchmark, real-life, real-world, generated instance, supplementary material & Lagrangian relaxation, mat heuristic, particle swarm, ant colony, simulated annealing, genetic algorithm, meta heuristic, GRASP & \ref{a:YunusogluY22} & \ref{c:YunusogluY22}\\
\index{YuraszeckMCCR23}\rowlabel{b:YuraszeckMCCR23}\href{../works/YuraszeckMCCR23.pdf}{YuraszeckMCCR23}~\cite{YuraszeckMCCR23} & 11 & flow-time, activity, machine, task, CP, make-span, multi-objective, resource, preempt, batch process, order, job, job-shop, setup-time, cmax, open-shop, precedence, constraint programming, flow-shop, scheduling & RCPSP, Open Shop Scheduling Problem, FJS, OSSP, Resource-constrained Project Scheduling Problem, JSSP & endBeforeStart, cumulative &  & OPL, Cplex &  & pharmaceutical industry & benchmark, github, real-world & GRASP, mat heuristic, meta heuristic & \ref{a:YuraszeckMCCR23} & \ref{c:YuraszeckMCCR23}\\
\index{YuraszeckMPV22}\rowlabel{b:YuraszeckMPV22}\href{../works/YuraszeckMPV22.pdf}{YuraszeckMPV22}~\cite{YuraszeckMPV22} & 26 & sequence dependent setup, no-wait, transportation, scheduling, order, make-span, release-date, setup-time, distributed, constraint programming, machine, flow-shop, CP, flow-time, job, re-scheduling, due-date, cyclic scheduling, stochastic, completion-time, resource, task, open-shop, job-shop & Open Shop Scheduling Problem, OSSP, single machine, JSSP & noOverlap, disjunctive, Disjunctive constraint & Java & Cplex & robot, semiconductor, automotive & manufacturing industry & real-life, generated instance, benchmark, github & meta heuristic, mat heuristic, genetic algorithm, ant colony, simulated annealing & \ref{a:YuraszeckMPV22} & \ref{c:YuraszeckMPV22}\\
\index{ZarandiASC20}\rowlabel{b:ZarandiASC20}\href{../works/ZarandiASC20.pdf}{ZarandiASC20}~\cite{ZarandiASC20} & 93 & tardiness, batch process, activity, multi-agent, completion-time, constraint satisfaction, due-date, scheduling, flow-shop, machine, job, re-scheduling, open-shop, make-span, energy efficiency, multi-objective, breakdown, explanation, setup-time, preempt, single-machine scheduling, order, inventory, bi-objective, distributed, lateness, no-wait, CP, resource, CSP, two-stage scheduling, net present value, one-machine scheduling, constraint logic programming, cmax, stochastic, reactive scheduling, task, constraint programming, flow-time, preemptive, Pareto, release-date, precedence, earliness, sequence dependent setup, job-shop, transportation, periodic, CLP & HFS, parallel machine, OSSP, JSSP, Resource-constrained Project Scheduling Problem, Open Shop Scheduling Problem, PMSP, RCPSP, single machine, FJS, Resource-constrained Project Scheduling Problem with Discounted Cashflow & disjunctive, cycle & Prolog & OPL & satellite, robot, sports scheduling, surgery, medical, round-robin, railway, business process, container terminal, nurse, semiconductor, tournament, evacuation, drone, crew-scheduling, train schedule, maintenance scheduling, aircraft, operating room, airport & textile industry, gas industry & real-world, benchmark, real-life & memetic algorithm, column generation, max-flow, time-tabling, neural network, meta heuristic, ant colony, simulated annealing, genetic algorithm, reinforcement learning, particle swarm, machine learning, Lagrangian relaxation, swarm intelligence & \ref{a:ZarandiASC20} & n/a\\
\index{ZarandiKS16}\rowlabel{b:ZarandiKS16}\href{../works/ZarandiKS16.pdf}{ZarandiKS16}~\cite{ZarandiKS16} & 17 & make-span, preemptive, completion-time, machine, preempt, earliness, CP, tardiness, single-machine scheduling, constraint satisfaction, order, distributed, breakdown, job, scheduling, resource, due-date, CSP, job-shop, transportation, task, constraint programming, flow-shop, multi-objective & single machine &  &  & Ilog Solver & robot &  & real-world & time-tabling, meta heuristic, genetic algorithm, machine learning, simulated annealing & \ref{a:ZarandiKS16} & n/a\\
\index{Zeballos10}\rowlabel{b:Zeballos10}\href{../works/Zeballos10.pdf}{Zeballos10}~\cite{Zeballos10} & 19 & order, multi-objective, tardiness, scheduling, task, machine, make-span, constraint satisfaction, due-date, activity, completion-time, CP, batch process, resource, breakdown, transportation, constraint programming, CSP, precedence &  &  &  & Ilog Solver, OPL, ECLiPSe, Ilog Scheduler & robot &  &  & ant colony & \ref{a:Zeballos10} & n/a\\
\index{ZeballosCM10}\rowlabel{b:ZeballosCM10}\href{../works/ZeballosCM10.pdf}{ZeballosCM10}~\cite{ZeballosCM10} & 11 & task, constraint programming, earliness, transportation, CLP, batch process, activity, scheduling, machine, job, re-scheduling, make-span, breakdown, order, inventory, flow-shop, CP, resource, reactive scheduling & single machine & circuit &  & Ilog Solver, Cplex, Z3, OPL & robot, semiconductor & process industry, semiconductor industry & real-life & simulated annealing, NEH & \ref{a:ZeballosCM10} & n/a\\
\index{ZeballosH05}\rowlabel{b:ZeballosH05}\href{../works/ZeballosH05.pdf}{ZeballosH05}~\cite{ZeballosH05} & 10 & make-span, order, machine, CP, tardiness, scheduling, buffer-capacity, due-date, CSP, job, activity, explanation, resource, transportation, completion-time, task, constraint satisfaction, constraint programming, precedence &  &  &  & OPL, Ilog Solver, Ilog Scheduler & robot &  &  & genetic algorithm & \ref{a:ZeballosH05} & n/a\\
\index{ZeballosNH11}\rowlabel{b:ZeballosNH11}\href{../works/ZeballosNH11.pdf}{ZeballosNH11}~\cite{ZeballosNH11} & 17 & due-date, scheduling, job, Benders Decomposition, make-span, manpower, setup-time, preempt, order, inventory, lateness, no-wait, CP, resource, CSP, task, constraint programming, preemptive, precedence, earliness, job-shop, CLP, tardiness, batch process, activity, completion-time, constraint satisfaction &  &  &  & Cplex, ECLiPSe, Ilog Scheduler, OPL, OZ, Ilog Solver &  & chemical industry & real-world & meta heuristic & \ref{a:ZeballosNH11} & n/a\\
\index{ZeballosQH10}\rowlabel{b:ZeballosQH10}\href{../works/ZeballosQH10.pdf}{ZeballosQH10}~\cite{ZeballosQH10} & 20 & make-span, breakdown, preempt, order, CP, resource, multi-objective, task, constraint programming, preemptive, precedence, earliness, job-shop, transportation, tardiness, cmax, activity, completion-time, constraint satisfaction, due-date, scheduling, machine, job &  &  &  & ECLiPSe, Ilog Scheduler, OPL, Ilog Solver, Cplex & robot &  & real-world, benchmark & ant colony, genetic algorithm & \ref{a:ZeballosQH10} & n/a\\
\index{ZhangW18}\rowlabel{b:ZhangW18}\href{../works/ZhangW18.pdf}{ZhangW18}~\cite{ZhangW18} & 18 & job, no-wait, lateness, transportation, unavailability, multi-agent, breakdown, tardiness, scheduling, multi-objective, constraint programming, completion-time, flow-shop, stochastic, setup-time, job-shop, CP, earliness, preempt, flow-time, distributed, resource, precedence, re-scheduling, order, make-span, machine & FJS & cumulative, noOverlap &  & Cplex, Z3, OPL & robot &  & benchmark & meta heuristic, ant colony, particle swarm, simulated annealing, genetic algorithm, memetic algorithm & \ref{a:ZhangW18} & n/a\\
\index{ZhangYW21}\rowlabel{b:ZhangYW21}\href{../works/ZhangYW21.pdf}{ZhangYW21}~\cite{ZhangYW21} & 10 & job, constraint satisfaction, preempt, setup-time, scheduling, precedence, make-span, task, preemptive, batch process, multi-agent, cmax, CP, machine, re-scheduling, constraint programming, order, multi-objective, activity, release-date, distributed, job-shop, resource & RCPSP, Resource-constrained Project Scheduling Problem & disjunctive, endBeforeStart &  & Cplex & robot &  & benchmark & memetic algorithm, meta heuristic, simulated annealing, genetic algorithm, particle swarm, ant colony & \ref{a:ZhangYW21} & n/a\\
\index{Zhou97}\rowlabel{b:Zhou97}\href{../works/Zhou97.pdf}{Zhou97}~\cite{Zhou97} & 29 & job-shop, due-date, constraint programming, task, order, preempt, preemptive, completion-time, constraint logic programming, CP, precedence, job, explanation, CLP, release-date, CSP, scheduling, constraint satisfaction, machine &  & Disjunctive constraint, disjunctive, cumulative & Prolog & CHIP, Z3, Ilog Scheduler &  &  & benchmark & edge-finder, edge-finding & \ref{a:Zhou97} & \ref{c:Zhou97}\\
\index{ZhuSZW23}\rowlabel{b:ZhuSZW23}\href{../works/ZhuSZW23.pdf}{ZhuSZW23}~\cite{ZhuSZW23} & 22 & order, scheduling, completion-time, machine, job, task, open-shop, transportation, multi-agent, cmax, constraint programming, job-shop, Benders Decomposition, Logic-Based Benders Decomposition, constraint satisfaction, preempt, CSP, setup-time, CP, make-span, distributed, resource, inventory, precedence, re-scheduling &  & alternative constraint, disjunctive, noOverlap, endBeforeStart &  & Cplex & robot & cable industry & real-world, benchmark & ant colony, particle swarm, genetic algorithm, column generation & \ref{a:ZhuSZW23} & n/a\\
\index{ZouZ20}\rowlabel{b:ZouZ20}\href{../works/ZouZ20.pdf}{ZouZ20}~\cite{ZouZ20} & 10 & resource, explanation, multi-objective, CSP, scheduling, constraint satisfaction, stochastic, constraint programming, task, order, completion-time, CP, activity, two-stage scheduling, precedence, distributed &  & noOverlap, span constraint, endBeforeStart, cumulative &  & Cplex, OPL & pipeline &  & benchmark & meta heuristic, genetic algorithm & \ref{a:ZouZ20} & n/a\\
\index{abs-0907-0939}\rowlabel{b:abs-0907-0939}\href{../works/abs-0907-0939.pdf}{abs-0907-0939}~\cite{abs-0907-0939} & 12 & constraint programming, resource, explanation, CSP, due-date, preempt, make-span, task, preemptive, CP, activity, scheduling, release-date, order &  & Cumulatives constraint, RelSoftCumulativeSum, cumulative, SoftCumulative, SoftCumulativeSum, Cardinality constraint, RelSoftCumulative & Java & Choco Solver, CHIP &  &  & real-world & sweep, energetic reasoning, edge-finding & \ref{a:abs-0907-0939} & n/a\\
\index{abs-1009-0347}\rowlabel{b:abs-1009-0347}\href{../works/abs-1009-0347.pdf}{abs-1009-0347}~\cite{abs-1009-0347} & 37 & constraint programming, cmax, resource, explanation, preempt, make-span, task, precedence, preemptive, CP, activity, scheduling, machine, order & RCPSP, SCC, Resource-constrained Project Scheduling Problem, psplib & cumulative, disjunctive, cycle & C++ & Ilog Scheduler, Ilog Solver, CHIP &  &  & benchmark, instance generator & genetic algorithm, lazy clause generation & \ref{a:abs-1009-0347} & n/a\\
\index{abs-1901-07914}\rowlabel{b:abs-1901-07914}\href{../works/abs-1901-07914.pdf}{abs-1901-07914}~\cite{abs-1901-07914} & 8 & constraint programming, CP, resource, CSP, constraint satisfaction, constraint optimization, task, distributed, machine, multi-agent, scheduling, order, make-span &  &  & Python & OR-Tools, MiniZinc & robot &  & real-world, github, benchmark &  & \ref{a:abs-1901-07914} & \ref{c:abs-1901-07914}\\
\index{abs-1902-01193}\rowlabel{b:abs-1902-01193}\href{../works/abs-1902-01193.pdf}{abs-1902-01193}~\cite{abs-1902-01193} & 9 & CSP, constraint satisfaction, constraint optimization, CLP, scheduling, activity, BOM, constraint programming, CP, order, constraint logic programming, stochastic, resource, task &  &  & Python, C++, Prolog & CHIP, Ilog Solver, OPL & medical, nurse &  &  & simulated annealing, meta heuristic, time-tabling, genetic algorithm, particle swarm & \ref{a:abs-1902-01193} & n/a\\
\index{abs-1902-09244}\rowlabel{b:abs-1902-09244}\href{../works/abs-1902-09244.pdf}{abs-1902-09244}~\cite{abs-1902-09244} & 62 & setup-time, activity, constraint programming, machine, flow-shop, CP, job, order, due-date, earliness, bi-objective, stochastic, explanation, completion-time, breakdown, resource, task, job-shop, tardiness, inventory, multi-objective, no-wait, precedence, transportation, scheduling, make-span, release-date & Resource-constrained Project Scheduling Problem, FJS, RCMPSP, RCPSP & cycle, cumulative, endBeforeStart &  & OPL, Cplex & aircraft & automobile industry, steel industry, food-processing industry, glass industry, processing industry & real-world, benchmark, industry partner & genetic algorithm, particle swarm, simulated annealing, meta heuristic & \ref{a:abs-1902-09244} & n/a\\
\index{abs-1911-04766}\rowlabel{b:abs-1911-04766}\href{../works/abs-1911-04766.pdf}{abs-1911-04766}~\cite{abs-1911-04766} & 16 & explanation, precedence, task, release-date, activity, multi-objective, scheduling, order, make-span, due-date, constraint programming, CP, completion-time, resource, job, re-scheduling & RCPSP, Resource-constrained Project Scheduling Problem & alternative constraint, endBeforeStart, noOverlap, Cardinality constraint, cumulative, disjunctive & Java & CPO, Cplex, MiniZinc, Chuffed, Gecode & automotive &  & real-world, benchmark, github, real-life, instance generator, generated instance, industrial partner & simulated annealing, meta heuristic, time-tabling, large neighborhood search & \ref{a:abs-1911-04766} & \ref{c:abs-1911-04766}\\
\index{abs-2102-08778}\rowlabel{b:abs-2102-08778}\href{../works/abs-2102-08778.pdf}{abs-2102-08778}~\cite{abs-2102-08778} & 10 & task, machine, flow-shop, scheduling, order, make-span, constraint programming, CP, resource, job, open-shop, job-shop, explanation & JSSP &  & Java & Cplex, OR-Tools, MiniZinc, OPL, CPO &  &  & benchmark, real-life, real-world, generated instance & genetic algorithm & \ref{a:abs-2102-08778} & n/a\\
\index{abs-2211-14492}\rowlabel{b:abs-2211-14492}\href{../works/abs-2211-14492.pdf}{abs-2211-14492}~\cite{abs-2211-14492} & 17 & distributed, flow-shop, multi-objective, transportation, scheduling, order, make-span, setup-time, activity, due-date, constraint programming, machine, CP, job, energy efficiency, cmax, completion-time, constraint satisfaction, resource, precedence, constraint optimization, task, job-shop, tardiness & single machine & bin-packing, disjunctive, cumulative, Disjunctive constraint & Python & OR-Tools, Cplex & semiconductor &  & generated instance, benchmark, random instance & quadratic programming, neural network, reinforcement learning, column generation, genetic algorithm, deep learning, ant colony, machine learning, meta heuristic & \ref{a:abs-2211-14492} & n/a\\
\index{abs-2305-19888}\rowlabel{b:abs-2305-19888}\href{../works/abs-2305-19888.pdf}{abs-2305-19888}~\cite{abs-2305-19888} & 42 & sequence dependent setup, distributed, flow-shop, scheduling, order, make-span, preempt, setup-time, activity, constraint programming, machine, CP, job, re-scheduling, unavailability, preemptive, bi-objective, explanation, cmax, completion-time, resource, precedence, task & parallel machine & alternative constraint, noOverlap, cumulative &  & Gurobi & robot, high performance computing &  & gitlab, generated instance, real-world, benchmark & meta heuristic, Lagrangian relaxation, genetic algorithm & \ref{a:abs-2305-19888} & \ref{c:abs-2305-19888}\\
\index{abs-2306-05747}\rowlabel{b:abs-2306-05747}\href{../works/abs-2306-05747.pdf}{abs-2306-05747}~\cite{abs-2306-05747} & 9 & re-scheduling, scheduling, order, make-span, preempt, constraint programming, CP, flow-time, completion-time, resource, job, periodic, job-shop, precedence, constraint optimization, task, tardiness, machine, flow-shop & JSSP & noOverlap, disjunctive, cumulative & Java & Choco Solver &  &  & real-world, github, industrial instance, supplementary material, benchmark & neural network, large neighborhood search, reinforcement learning, genetic algorithm, machine learning, meta heuristic, simulated annealing & \ref{a:abs-2306-05747} & \ref{c:abs-2306-05747}\\
\index{abs-2312-13682}\rowlabel{b:abs-2312-13682}\href{../works/abs-2312-13682.pdf}{abs-2312-13682}~\cite{abs-2312-13682} & 20 & activity, constraint programming, machine, inventory, re-scheduling, scheduling, order, make-span, CP, resource, transportation, task &  & table constraint, cumulative &  & OPL & container terminal, train schedule, nurse, steel mill, operating room &  & real-world, generated instance & large neighborhood search, mat heuristic, meta heuristic & \ref{a:abs-2312-13682} & \ref{c:abs-2312-13682}\\
\index{abs-2402-00459}\rowlabel{b:abs-2402-00459}\href{../works/abs-2402-00459.pdf}{abs-2402-00459}~\cite{abs-2402-00459} & 21 & job-shop, tardiness, multi-objective, scheduling, order, net present value, constraint programming, machine, CP, job, multi-agent, due-date, earliness, completion-time, resource, precedence, task & single machine, Resource-constrained Project Scheduling Problem & Disjunctive constraint, bin-packing, disjunctive, cumulative &  & OPL, OR-Tools & tournament & mining industry & instance generator, real-world, generated instance, benchmark, github & particle swarm, simulated annealing, meta heuristic, quadratic programming, Lagrangian relaxation, neural network, reinforcement learning, column generation, mat heuristic, genetic algorithm, ant colony, machine learning & \ref{a:abs-2402-00459} & \ref{c:abs-2402-00459}\\
\end{longtable}
}



%\clearpage
%\subsection{Manually Defined Fields}
%{\scriptsize
\begin{longtable}{>{\raggedright\arraybackslash}p{3cm}>{\raggedright\arraybackslash}p{6cm}lp{2cm}rrrrlp{2cm}p{2cm}rr}
\rowcolor{white}\caption{Manually Defined ARTICLE Properties}\\ \toprule
\rowcolor{white}Key & Title (Local Copy) & \shortstack{CP\\System} & Bench & Links & \shortstack{Data\\Avail} & \shortstack{Sol\\Avail} & \shortstack{Code\\Avail} & \shortstack{Based\\On} & Classification & Constraints & a & b\\ \midrule\endhead
\bottomrule
\endfoot
\rowlabel{c:PrataAN23}PrataAN23 \href{https://www.sciencedirect.com/science/article/pii/S2666720723001522}{PrataAN23}~\cite{PrataAN23} & \href{works/PrataAN23.pdf}{Applications of constraint programming in production scheduling problems: A descriptive bibliometric analysis} & - & benchmark, real-world, real-life & 1 & - &  & - & - & survey & - & \ref{a:PrataAN23} & \ref{b:PrataAN23}\\
\rowlabel{c:abs-2402-00459}abs-2402-00459 \href{https://doi.org/10.48550/arXiv.2402.00459}{abs-2402-00459}~\cite{abs-2402-00459} & \href{works/abs-2402-00459.pdf}{Genetic-based Constraint Programming for Resource Constrained Job Scheduling} & OR-Tools & instance generator, real-world, generated instance, github, benchmark & 2 & \href{https://github.com/andreas-ernst/Mathprog-ORlib/blob/master/data/RCJS_new_instances.zip}{y} &  & n & - & RCJS & cumulatives & \ref{a:abs-2402-00459} & \ref{b:abs-2402-00459}\\
\rowlabel{c:AbreuNP23}AbreuNP23 \href{https://doi.org/10.1080/00207543.2022.2154404}{AbreuNP23}~\cite{AbreuNP23} & \href{works/AbreuNP23.pdf}{A new two-stage constraint programming approach for open shop scheduling problem with machine blocking} & ? & real-world, benchmark & 10 & ? &  & ? & ? & ? & ? & \ref{a:AbreuNP23} & \ref{b:AbreuNP23}\\
\rowlabel{c:AkramNHRSA23}AkramNHRSA23 \href{https://doi.org/10.1109/ACCESS.2023.3343409}{AkramNHRSA23}~\cite{AkramNHRSA23} & \href{works/AkramNHRSA23.pdf}{Joint Scheduling and Routing Optimization for Deterministic Hybrid Traffic in Time-Sensitive Networks Using Constraint Programming} & OR-Tools & benchmark & 0 & n &  & n & - & TSN & - & \ref{a:AkramNHRSA23} & \ref{b:AkramNHRSA23}\\
\rowlabel{c:AlfieriGPS23}AlfieriGPS23 \href{https://www.sciencedirect.com/science/article/pii/S0360835223000074}{AlfieriGPS23}~\cite{AlfieriGPS23} & \href{works/AlfieriGPS23.pdf}{Permutation flowshop problems minimizing core waiting time and core idle time} &  & benchmark & 0 &  &  &  &  &  &  & \ref{a:AlfieriGPS23} & \ref{b:AlfieriGPS23}\\
\rowlabel{c:Caballero23}Caballero23 \href{https://doi.org/10.1007/s10601-023-09357-0}{Caballero23}~\cite{Caballero23} & \href{works/Caballero23.pdf}{Scheduling through logic-based tools} & SAT &  & 1 & - &  & - & \href{http://hdl.handle.net/10803/667963}{PhD Thesis} & RCPSP & - & \ref{a:Caballero23} & \ref{b:Caballero23}\\
\rowlabel{c:CzerniachowskaWZ23}CzerniachowskaWZ23 \href{https://doi.org/10.12913/22998624/166588}{CzerniachowskaWZ23}~\cite{CzerniachowskaWZ23} & \href{works/CzerniachowskaWZ23.pdf}{Constraint Programming for Flexible Flow Shop Scheduling Problem with Repeated Jobs and Repeated Operations} &  & benchmark, Roadef, real-world & 0 &  &  &  &  &  &  & \ref{a:CzerniachowskaWZ23} & \ref{b:CzerniachowskaWZ23}\\
\rowlabel{c:GurPAE23}GurPAE23 \href{https://doi.org/10.1007/s10100-022-00835-z}{GurPAE23}~\cite{GurPAE23} & \href{works/GurPAE23.pdf}{Operating room scheduling with surgical team: a new approach with constraint programming and goal programming} & Cplex & real-life & 0 & n &  & n & - & - & - & \ref{a:GurPAE23} & \ref{b:GurPAE23}\\
\rowlabel{c:IsikYA23}IsikYA23 \href{https://doi.org/10.1007/s00500-023-09086-9}{IsikYA23}~\cite{IsikYA23} & \href{works/IsikYA23.pdf}{Constraint programming models for the hybrid flow shop scheduling problem and its extensions} & \su{OPL {CP Opt}} & real-world, benchmark, generated instance, real-life & 4 & \href{https://data.mendeley.com/datasets/n4g8cfjg87/1}{y} &  & \href{https://data.mendeley.com/datasets/n4g8cfjg87/1}{y} & - & HFSP & \su{alternative endBeforeStart noOverlap cumulative} & \ref{a:IsikYA23} & \ref{b:IsikYA23}\\
\rowlabel{c:LacknerMMWW23}LacknerMMWW23 \href{https://doi.org/10.1007/s10601-023-09347-2}{LacknerMMWW23}~\cite{LacknerMMWW23} & \href{works/LacknerMMWW23.pdf}{Exact methods for the Oven Scheduling Problem} & \su{MiniZinc OPL} & random instance, industrial partner, benchmark, instance generator, zenodo, real-life & 0 & \href{https://zenodo.org/records/7456938}{\su{DZN JSON}} &  & \href{https://zenodo.org/records/7456938}{y} & \cite{LacknerMMWW21} & OSP & \su{alternative noOverlap forbidExtent} & \ref{a:LacknerMMWW23} & \ref{b:LacknerMMWW23}\\
\rowlabel{c:MontemanniD23}MontemanniD23 \href{https://doi.org/10.3390/a16010040}{MontemanniD23}~\cite{MontemanniD23} & \href{works/MontemanniD23.pdf}{Solving the Parallel Drone Scheduling Traveling Salesman Problem via Constraint Programming} & OR-Tools & benchmark, supplementary material & 6 & ref & \href{https://www.mdpi.com/article/10.3390/a16010040/s1}{y} & n & - & PDSTSP & \su{circuit} & \ref{a:MontemanniD23} & \ref{b:MontemanniD23}\\
\rowlabel{c:MontemanniD23a}MontemanniD23a \href{https://doi.org/10.1016/j.ejco.2023.100078}{MontemanniD23a}~\cite{MontemanniD23a} & \href{works/MontemanniD23a.pdf}{Constraint programming models for the parallel drone scheduling vehicle routing problem} & OR-Tools & benchmark & 0 & ref &  & n & - & PDSTSP & \su{circuit multipleCircuit} & \ref{a:MontemanniD23a} & \ref{b:MontemanniD23a}\\
\rowlabel{c:NaderiRR23}NaderiRR23 \href{https://doi.org/10.1287/ijoc.2023.1287}{NaderiRR23}~\cite{NaderiRR23} & \href{works/NaderiRR23.pdf}{Mixed-Integer Programming vs. Constraint Programming for Shop Scheduling Problems: New Results and Outlook} &  & github, benchmark & 8 &  &  &  &  &  &  & \ref{a:NaderiRR23} & \ref{b:NaderiRR23}\\
\rowlabel{c:ShaikhK23}ShaikhK23 \href{https://doi.org/10.1504/IJESDF.2023.10045616}{ShaikhK23}~\cite{ShaikhK23} & \href{works/ShaikhK23.pdf}{Management of electronic ledger: a constraint programming approach for solving curricula scheduling problems} & ? & benchmark, real-world & 2 & ? &  & ? & ? & ? & ? & \ref{a:ShaikhK23} & \ref{b:ShaikhK23}\\
\rowlabel{c:YuraszeckMCCR23}YuraszeckMCCR23 \href{https://doi.org/10.1109/ACCESS.2023.3345793}{YuraszeckMCCR23}~\cite{YuraszeckMCCR23} & \href{works/YuraszeckMCCR23.pdf}{A Constraint Programming Formulation of the Multi-Mode Resource-Constrained Project Scheduling Problem for the Flexible Job Shop Scheduling Problem} & CP Opt & github, real-world, benchmark & 0 & ref &  & n & - & FJSSP & \su{alternative endBeforeStart cumulative} & \ref{a:YuraszeckMCCR23} & \ref{b:YuraszeckMCCR23}\\
\rowlabel{c:abs-2305-19888}abs-2305-19888 \href{https://doi.org/10.48550/arXiv.2305.19888}{abs-2305-19888}~\cite{abs-2305-19888} & \href{works/abs-2305-19888.pdf}{Constraint Programming and Constructive Heuristics for Parallel Machine Scheduling with Sequence-Dependent Setups and Common Servers} & \su{{CP Opt} Gurobi} & real-world, generated instance, gitlab, benchmark & 1 & \href{https://gitlab.com/vilem_heinz/cp_heur_paper_evalutation}{y} & \href{https://gitlab.com/vilem_heinz/cp_heur_paper_evalutation}{y} & n & - & $P|seq, ser|C_{max}$ & \su{alternative noOverlap cumulative} & \ref{a:abs-2305-19888} & \ref{b:abs-2305-19888}\\
\rowlabel{c:abs-2306-05747}abs-2306-05747 \href{https://doi.org/10.48550/arXiv.2306.05747}{abs-2306-05747}~\cite{abs-2306-05747} & \href{works/abs-2306-05747.pdf}{An End-to-End Reinforcement Learning Approach for Job-Shop Scheduling Problems Based on Constraint Programming} & \su{custom Choco} & real-world, supplementary material, github, industrial instance, benchmark & 0 & ref &  & n & - & JSSP & \su{noOverlap} & \ref{a:abs-2306-05747} & \ref{b:abs-2306-05747}\\
\rowlabel{c:abs-2312-13682}abs-2312-13682 \href{https://doi.org/10.48550/arXiv.2312.13682}{abs-2312-13682}~\cite{abs-2312-13682} & \href{works/abs-2312-13682.pdf}{A Constraint Programming Model for Scheduling the Unloading of Trains in Ports: Extended} & custom & real-world, generated instance & 0 & n &  & n & - & SUTP & \su{table disjunctive} & \ref{a:abs-2312-13682} & \ref{b:abs-2312-13682}\\
\rowlabel{c:AbreuN22}AbreuN22 \href{https://doi.org/10.1016/j.cie.2022.108128}{AbreuN22}~\cite{AbreuN22} & \href{works/AbreuN22.pdf}{A new hybridization of adaptive large neighborhood search with constraint programming for open shop scheduling with sequence-dependent setup times} & \su{Cplex {CP Opt}} & real-world, benchmark & 0 & \href{https://bit.ly/392wfZa}{y} &  & n & - & OSSPST & \su{noOverlap} & \ref{a:AbreuN22} & \ref{b:AbreuN22}\\
\rowlabel{c:BourreauGGLT22}BourreauGGLT22 \href{https://doi.org/10.1080/00207543.2020.1856436}{BourreauGGLT22}~\cite{BourreauGGLT22} & \href{works/BourreauGGLT22.pdf}{A constraint-programming based decomposition method for the Generalised Workforce Scheduling and Routing Problem {(GWSRP)}} &  & real-world, benchmark & 2 &  &  &  &  &  &  & \ref{a:BourreauGGLT22} & \ref{b:BourreauGGLT22}\\
\rowlabel{c:CampeauG22}CampeauG22 \href{https://doi.org/10.1007/s10601-022-09337-w}{CampeauG22}~\cite{CampeauG22} & \href{works/CampeauG22.pdf}{Short- and medium-term optimization of underground mine planning using constraint programming} & CP Opt & real-life, real-world & 0 & ref &  & n &  &  & \su{pulse alwaysIn endBeforeStart noOverlap} & \ref{a:CampeauG22} & \ref{b:CampeauG22}\\
\rowlabel{c:ColT22}ColT22 \href{https://api.semanticscholar.org/CorpusID:251551160}{ColT22}~\cite{ColT22} & \href{works/ColT22.pdf}{Industrial-size job shop scheduling with constraint programming} &  & generated instance, supplementary material, github, real-life, benchmark, real-world & 4 &  &  &  &  &  &  & \ref{a:ColT22} & \ref{b:ColT22}\\
\rowlabel{c:FarsiTM22}FarsiTM22 \href{https://api.semanticscholar.org/CorpusID:250301745}{FarsiTM22}~\cite{FarsiTM22} & \href{works/FarsiTM22.pdf}{Integrated surgery scheduling by constraint programming and meta-heuristics} &  & supplementary material & 10 &  &  &  &  &  &  & \ref{a:FarsiTM22} & \ref{b:FarsiTM22}\\
\rowlabel{c:Fatemi-AnarakiMFN22}Fatemi-AnarakiMFN22 \href{https://api.semanticscholar.org/CorpusID:252524295}{Fatemi-AnarakiMFN22}~\cite{Fatemi-AnarakiMFN22} & \href{}{Scheduling of Multi-Robot Job Shop Systems in Dynamic Environments: Mixed-Integer Linear Programming and Constraint Programming Approaches} &  &  & 0 &  &  &  &  &  &  & \ref{a:Fatemi-AnarakiMFN22} & No\\
\rowlabel{c:FetgoD22}FetgoD22 \href{https://doi.org/10.1007/s43069-022-00172-6}{FetgoD22}~\cite{FetgoD22} & \href{works/FetgoD22.pdf}{Horizontally Elastic Edge-Finder Algorithm for Cumulative Resource Constraint Revisited} &  & benchmark, real-world & 7 &  &  &  &  &  &  & \ref{a:FetgoD22} & \ref{b:FetgoD22}\\
\rowlabel{c:HeinzNVH22}HeinzNVH22 \href{https://doi.org/10.1016/j.cie.2022.108586}{HeinzNVH22}~\cite{HeinzNVH22} & \href{works/HeinzNVH22.pdf}{Constraint Programming and constructive heuristics for parallel machine scheduling with sequence-dependent setups and common servers} &  & real-world, generated instance, benchmark, gitlab & 3 &  &  &  &  &  &  & \ref{a:HeinzNVH22} & \ref{b:HeinzNVH22}\\
\rowlabel{c:MullerMKP22}MullerMKP22 \href{https://doi.org/10.1016/j.ejor.2022.01.034}{MullerMKP22}~\cite{MullerMKP22} & \href{works/MullerMKP22.pdf}{An algorithm selection approach for the flexible job shop scheduling problem: Choosing constraint programming solvers through machine learning} &  & benchmark, random instance, real-world, github & 3 &  &  &  &  &  &  & \ref{a:MullerMKP22} & \ref{b:MullerMKP22}\\
\rowlabel{c:PohlAK22}PohlAK22 \href{https://doi.org/10.1016/j.ejor.2021.08.028}{PohlAK22}~\cite{PohlAK22} & \href{works/PohlAK22.pdf}{Solving the time-discrete winter runway scheduling problem: {A} column generation and constraint programming approach} &  & benchmark, real-world & 2 &  &  &  &  &  &  & \ref{a:PohlAK22} & \ref{b:PohlAK22}\\
\rowlabel{c:ShiYXQ22}ShiYXQ22 \href{https://doi.org/10.1080/00207543.2021.1963496}{ShiYXQ22}~\cite{ShiYXQ22} & \href{}{Solving the integrated process planning and scheduling problem using an enhanced constraint programming-based approach} &  &  & 0 &  &  &  &  &  &  & \ref{a:ShiYXQ22} & No\\
\rowlabel{c:SubulanC22}SubulanC22 \href{https://doi.org/10.1007/s00500-021-06399-5}{SubulanC22}~\cite{SubulanC22} & \href{works/SubulanC22.pdf}{Constraint programming-based transformation approach for a mixed fuzzy-stochastic resource investment project scheduling problem} &  & real-life, benchmark, real-world & 2 &  &  &  &  &  &  & \ref{a:SubulanC22} & \ref{b:SubulanC22}\\
\rowlabel{c:YunusogluY22}YunusogluY22 \href{https://doi.org/10.1080/00207543.2021.1885068}{YunusogluY22}~\cite{YunusogluY22} & \href{works/YunusogluY22.pdf}{Constraint programming approach for multi-resource-constrained unrelated parallel machine scheduling problem with sequence-dependent setup times} &  & real-world, benchmark, generated instance, real-life, supplementary material & 10 &  &  &  &  &  &  & \ref{a:YunusogluY22} & \ref{b:YunusogluY22}\\
\rowlabel{c:YuraszeckMPV22}YuraszeckMPV22 \href{https://api.semanticscholar.org/CorpusID:246320449}{YuraszeckMPV22}~\cite{YuraszeckMPV22} & \href{works/YuraszeckMPV22.pdf}{A Novel Constraint Programming Decomposition Approach for the Total Flow Time Fixed Group Shop Scheduling Problem} &  & generated instance, github, benchmark, real-life & 5 &  &  &  &  &  &  & \ref{a:YuraszeckMPV22} & \ref{b:YuraszeckMPV22}\\
\rowlabel{c:abs-2211-14492}abs-2211-14492 \href{https://doi.org/10.48550/arXiv.2211.14492}{abs-2211-14492}~\cite{abs-2211-14492} & \href{works/abs-2211-14492.pdf}{Enhancing Constraint Programming via Supervised Learning for Job Shop Scheduling} &  & benchmark, random instance, generated instance & 1 &  &  &  &  &  &  & \ref{a:abs-2211-14492} & \ref{b:abs-2211-14492}\\
\rowlabel{c:AbohashimaEG21}AbohashimaEG21 \href{https://doi.org/10.1109/ACCESS.2021.3112600}{AbohashimaEG21}~\cite{AbohashimaEG21} & \href{works/AbohashimaEG21.pdf}{A Mathematical Programming Model and a Firefly-Based Heuristic for Real-Time Traffic Signal Scheduling With Physical Constraints} &  & real-world, generated instance, github & 0 &  &  &  &  &  &  & \ref{a:AbohashimaEG21} & \ref{b:AbohashimaEG21}\\
\rowlabel{c:AbreuAPNM21}AbreuAPNM21 \href{https://api.semanticscholar.org/CorpusID:238794651}{AbreuAPNM21}~\cite{AbreuAPNM21} & \href{works/AbreuAPNM21.pdf}{A new variable neighbourhood search with a constraint programming search strategy for the open shop scheduling problem with operation repetitions} &  & generated instance, benchmark, real-world & 8 &  &  &  &  &  &  & \ref{a:AbreuAPNM21} & \ref{b:AbreuAPNM21}\\
\rowlabel{c:Bedhief21}Bedhief21 \href{https://api.semanticscholar.org/CorpusID:240611192}{Bedhief21}~\cite{Bedhief21} & \href{works/Bedhief21.pdf}{Comparing Mixed-Integer Programming and Constraint Programming Models for the Hybrid Flow Shop Scheduling Problem with Dedicated Machines} &  & real-life & 0 &  &  &  &  &  &  & \ref{a:Bedhief21} & \ref{b:Bedhief21}\\
\rowlabel{c:FanXG21}FanXG21 \href{https://doi.org/10.1016/j.cor.2021.105401}{FanXG21}~\cite{FanXG21} & \href{works/FanXG21.pdf}{Genetic programming-based hyper-heuristic approach for solving dynamic job shop scheduling problem with extended technical precedence constraints} &  & benchmark & 0 &  &  &  &  &  &  & \ref{a:FanXG21} & \ref{b:FanXG21}\\
\rowlabel{c:HamPK21}HamPK21 \href{https://api.semanticscholar.org/CorpusID:237898414}{HamPK21}~\cite{HamPK21} & \href{works/HamPK21.pdf}{Energy-Aware Flexible Job Shop Scheduling Using Mixed Integer Programming and Constraint Programming} &  & benchmark, github & 4 &  &  &  &  &  &  & \ref{a:HamPK21} & \ref{b:HamPK21}\\
\rowlabel{c:HubnerGSV21}HubnerGSV21 \href{https://doi.org/10.1007/s10951-021-00682-x}{HubnerGSV21}~\cite{HubnerGSV21} & \href{works/HubnerGSV21.pdf}{Solving the nuclear dismantling project scheduling problem by combining mixed-integer and constraint programming techniques and metaheuristics} &  & benchmark, real-life & 4 &  &  &  &  &  &  & \ref{a:HubnerGSV21} & \ref{b:HubnerGSV21}\\
\rowlabel{c:KoehlerBFFHPSSS21}KoehlerBFFHPSSS21 \href{https://doi.org/10.1007/s10601-021-09321-w}{KoehlerBFFHPSSS21}~\cite{KoehlerBFFHPSSS21} & \href{works/KoehlerBFFHPSSS21.pdf}{Cable tree wiring - benchmarking solvers on a real-world scheduling problem with a variety of precedence constraints} & \su{{CP Opt} OR-Tools Chuffed Cplex Gurobi Z3 OptiMathSat} & real-world, benchmark, github & 9 & \href{https://github.com/kw90/ctw_toolchain}{DZN} &  & y & - & CTW & \su{alldifferent inverse} & \ref{a:KoehlerBFFHPSSS21} & \ref{b:KoehlerBFFHPSSS21}\\
\rowlabel{c:PandeyS21a}PandeyS21a \href{https://doi.org/10.1007/s11227-020-03516-3}{PandeyS21a}~\cite{PandeyS21a} & \href{works/PandeyS21a.pdf}{Constraint programming versus heuristic approach to MapReduce scheduling problem in Hadoop {YARN} for energy minimization} &  & benchmark & 1 &  &  &  &  &  &  & \ref{a:PandeyS21a} & \ref{b:PandeyS21a}\\
\rowlabel{c:QinWSLS21}QinWSLS21 \href{https://doi.org/10.1109/TASE.2019.2947398}{QinWSLS21}~\cite{QinWSLS21} & \href{works/QinWSLS21.pdf}{A Genetic Programming-Based Scheduling Approach for Hybrid Flow Shop With a Batch Processor and Waiting Time Constraint} &  &  & 0 &  &  &  &  &  &  & \ref{a:QinWSLS21} & \ref{b:QinWSLS21}\\
\rowlabel{c:VlkHT21}VlkHT21 \href{https://doi.org/10.1016/j.cie.2021.107317}{VlkHT21}~\cite{VlkHT21} & \href{works/VlkHT21.pdf}{Constraint programming approaches to joint routing and scheduling in time-sensitive networks} &  & industrial partner, random instance, github, benchmark & 0 &  &  &  &  &  &  & \ref{a:VlkHT21} & \ref{b:VlkHT21}\\
\rowlabel{c:ZhangYW21}ZhangYW21 \href{https://doi.org/10.1016/j.cor.2021.105282}{ZhangYW21}~\cite{ZhangYW21} & \href{works/ZhangYW21.pdf}{A graph-based constraint programming approach for the integrated process planning and scheduling problem} &  & benchmark & 0 &  &  &  &  &  &  & \ref{a:ZhangYW21} & \ref{b:ZhangYW21}\\
\rowlabel{c:abs-2102-08778}abs-2102-08778 \href{https://arxiv.org/abs/2102.08778}{abs-2102-08778}~\cite{abs-2102-08778} & \href{works/abs-2102-08778.pdf}{Large-Scale Benchmarks for the Job Shop Scheduling Problem} &  & generated instance, benchmark, real-life, real-world & 0 &  &  &  &  &  &  & \ref{a:abs-2102-08778} & \ref{b:abs-2102-08778}\\
\rowlabel{c:AlizdehS20}AlizdehS20 \href{https://doi.org/10.1504/IJAIP.2020.106687}{AlizdehS20}~\cite{AlizdehS20} & \href{}{Fuzzy project scheduling with critical path including risk and resource constraints using linear programming} &  &  & 0 &  &  &  &  &  &  & \ref{a:AlizdehS20} & No\\
\rowlabel{c:AstrandJZ20}AstrandJZ20 \href{https://doi.org/10.1016/j.cor.2020.105036}{AstrandJZ20}~\cite{AstrandJZ20} & \href{works/AstrandJZ20.pdf}{Underground mine scheduling of mobile machines using Constraint Programming and Large Neighborhood Search} &  & benchmark, real-world, real-life & 0 &  &  &  &  &  &  & \ref{a:AstrandJZ20} & \ref{b:AstrandJZ20}\\
\rowlabel{c:BadicaBI20}BadicaBI20 \href{https://doi.org/10.3233/AIC-200650}{BadicaBI20}~\cite{BadicaBI20} & \href{works/BadicaBI20.pdf}{Block structured scheduling using constraint logic programming} &  & real-world, benchmark & 5 &  &  &  &  &  &  & \ref{a:BadicaBI20} & \ref{b:BadicaBI20}\\
\rowlabel{c:BenediktMH20}BenediktMH20 \href{https://doi.org/10.1007/s10601-020-09317-y}{BenediktMH20}~\cite{BenediktMH20} & \href{works/BenediktMH20.pdf}{Power of pre-processing: production scheduling with variable energy pricing and power-saving states} & \su{{CP Opt} Gurobi} & github, benchmark, random instance, generated instance & 4 & \href{https://github.com/CTU-IIG/EnergyStatesAndCostsSchedulingData}{JSON} &  & \href{https://github.com/CTU-IIG/EnergyStatesAndCostsScheduling}{y} &  &  &  & \ref{a:BenediktMH20} & \ref{b:BenediktMH20}\\
\rowlabel{c:FallahiAC20}FallahiAC20 \href{https://api.semanticscholar.org/CorpusID:213449737}{FallahiAC20}~\cite{FallahiAC20} & \href{works/FallahiAC20.pdf}{Tabu search and constraint programming-based approach for a real scheduling and routing problem} &  & github, real-life & 0 &  &  &  &  &  &  & \ref{a:FallahiAC20} & \ref{b:FallahiAC20}\\
\rowlabel{c:LunardiBLRV20}LunardiBLRV20 \href{https://doi.org/10.1016/j.cor.2020.105020}{LunardiBLRV20}~\cite{LunardiBLRV20} & \href{works/LunardiBLRV20.pdf}{Mixed Integer linear programming and constraint programming models for the online printing shop scheduling problem} &  & benchmark, random instance, generated instance, github & 1 &  &  &  &  &  &  & \ref{a:LunardiBLRV20} & \ref{b:LunardiBLRV20}\\
\rowlabel{c:MejiaY20}MejiaY20 \href{https://doi.org/10.1016/j.ejor.2020.02.010}{MejiaY20}~\cite{MejiaY20} & \href{works/MejiaY20.pdf}{A self-tuning variable neighborhood search algorithm and an effective decoding scheme for open shop scheduling problems with travel/setup times} &  & supplementary material, benchmark, generated instance & 2 &  &  &  &  &  &  & \ref{a:MejiaY20} & \ref{b:MejiaY20}\\
\rowlabel{c:MengZRZL20}MengZRZL20 \href{https://doi.org/10.1016/j.cie.2020.106347}{MengZRZL20}~\cite{MengZRZL20} & \href{works/MengZRZL20.pdf}{Mixed-integer linear programming and constraint programming formulations for solving distributed flexible job shop scheduling problem} &  & supplementary material, benchmark & 0 &  &  &  &  &  &  & \ref{a:MengZRZL20} & \ref{b:MengZRZL20}\\
\rowlabel{c:MokhtarzadehTNF20}MokhtarzadehTNF20 \href{https://doi.org/10.1080/0951192X.2020.1736713}{MokhtarzadehTNF20}~\cite{MokhtarzadehTNF20} & \href{works/MokhtarzadehTNF20.pdf}{Scheduling of human-robot collaboration in assembly of printed circuit boards: a constraint programming approach} &  & generated instance, real-world & 12 &  &  &  &  &  &  & \ref{a:MokhtarzadehTNF20} & \ref{b:MokhtarzadehTNF20}\\
\rowlabel{c:Polo-MejiaALB20}Polo-MejiaALB20 \href{https://doi.org/10.1080/00207543.2019.1693654}{Polo-MejiaALB20}~\cite{Polo-MejiaALB20} & \href{works/Polo-MejiaALB20.pdf}{Mixed-integer/linear and constraint programming approaches for activity scheduling in a nuclear research facility} &  & Roadef, github & 2 &  &  &  &  &  &  & \ref{a:Polo-MejiaALB20} & \ref{b:Polo-MejiaALB20}\\
\rowlabel{c:QinDCS20}QinDCS20 \href{https://doi.org/10.1016/j.ejor.2020.02.021}{QinDCS20}~\cite{QinDCS20} & \href{works/QinDCS20.pdf}{Combining mixed integer programming and constraint programming to solve the integrated scheduling problem of container handling operations of a single vessel} &  & real-life, benchmark & 0 &  &  &  &  &  &  & \ref{a:QinDCS20} & \ref{b:QinDCS20}\\
\rowlabel{c:SacramentoSP20}SacramentoSP20 \href{https://doi.org/10.1007/s43069-020-00036-x}{SacramentoSP20}~\cite{SacramentoSP20} & \href{works/SacramentoSP20.pdf}{Constraint Programming and Local Search Heuristic: a Matheuristic Approach for Routing and Scheduling Feeder Vessels in Multi-terminal Ports} &  & benchmark, real-life, zenodo, real-world & 4 &  &  &  &  &  &  & \ref{a:SacramentoSP20} & \ref{b:SacramentoSP20}\\
\rowlabel{c:WallaceY20}WallaceY20 \href{https://doi.org/10.1007/s10601-020-09316-z}{WallaceY20}~\cite{WallaceY20} & \href{works/WallaceY20.pdf}{A new constraint programming model and solving for the cyclic hoist scheduling problem} & MiniZinc & random instance, real-life, real-world, benchmark & 2 & \href{https://data.4tu.nl/articles/_/12912413}{DZN} &  & \href{https://data.4tu.nl/articles/_/12912413}{y} &  & CHSP &  & \ref{a:WallaceY20} & \ref{b:WallaceY20}\\
\rowlabel{c:ZouZ20}ZouZ20 \href{https://api.semanticscholar.org/CorpusID:208840808}{ZouZ20}~\cite{ZouZ20} & \href{works/ZouZ20.pdf}{A constraint programming approach for scheduling repetitive projects with atypical activities considering soft logic} &  & benchmark & 3 &  &  &  &  &  &  & \ref{a:ZouZ20} & \ref{b:ZouZ20}\\
\rowlabel{c:EscobetPQPRA19}EscobetPQPRA19 \href{https://doi.org/10.1016/j.compchemeng.2018.08.040}{EscobetPQPRA19}~\cite{EscobetPQPRA19} & \href{works/EscobetPQPRA19.pdf}{Optimal batch scheduling of a multiproduct dairy process using a combined optimization/constraint programming approach} &  &  & 1 &  &  &  &  &  &  & \ref{a:EscobetPQPRA19} & \ref{b:EscobetPQPRA19}\\
\rowlabel{c:GurEA19}GurEA19 \href{https://api.semanticscholar.org/CorpusID:88492001}{GurEA19}~\cite{GurEA19} & \href{works/GurEA19.pdf}{Surgical Operation Scheduling with Goal Programming and Constraint Programming: A Case Study} &  & real-life & 11 &  &  &  &  &  &  & \ref{a:GurEA19} & \ref{b:GurEA19}\\
\rowlabel{c:NishikawaSTT19}NishikawaSTT19 \href{http://www.ijnc.org/index.php/ijnc/article/view/201}{NishikawaSTT19}~\cite{NishikawaSTT19} & \href{works/NishikawaSTT19.pdf}{A Constraint Programming Approach to Scheduling of Malleable Tasks} &  & real-world, benchmark & 0 &  &  &  &  &  &  & \ref{a:NishikawaSTT19} & \ref{b:NishikawaSTT19}\\
\rowlabel{c:Novas19}Novas19 \href{https://doi.org/10.1016/j.cie.2019.07.011}{Novas19}~\cite{Novas19} & \href{works/Novas19.pdf}{Production scheduling and lot streaming at flexible job-shops environments using constraint programming} &  & benchmark & 0 &  &  &  &  &  &  & \ref{a:Novas19} & \ref{b:Novas19}\\
\rowlabel{c:WikarekS19}WikarekS19 \href{https://doi.org/10.1142/S2196888819500027}{WikarekS19}~\cite{WikarekS19} & \href{works/WikarekS19.pdf}{A Constraint-Based Declarative Programming Framework for Scheduling and Resource Allocation Problems} &  &  & 0 &  &  &  &  &  &  & \ref{a:WikarekS19} & \ref{b:WikarekS19}\\
\rowlabel{c:YounespourAKE19}YounespourAKE19 \href{https://api.semanticscholar.org/CorpusID:208103305}{YounespourAKE19}~\cite{YounespourAKE19} & \href{works/YounespourAKE19.pdf}{Using mixed integer programming and constraint programming for operating rooms scheduling with modified block strategy} &  & real-life, real-world & 6 &  &  &  &  &  &  & \ref{a:YounespourAKE19} & \ref{b:YounespourAKE19}\\
\rowlabel{c:abs-1901-07914}abs-1901-07914 \href{http://arxiv.org/abs/1901.07914}{abs-1901-07914}~\cite{abs-1901-07914} & \href{works/abs-1901-07914.pdf}{A Constraint Programming Approach to Simultaneous Task Allocation and Motion Scheduling for Industrial Dual-Arm Manipulation Tasks} &  & benchmark, real-world, github & 0 &  &  &  &  &  &  & \ref{a:abs-1901-07914} & \ref{b:abs-1901-07914}\\
\rowlabel{c:abs-1902-01193}abs-1902-01193 \href{http://arxiv.org/abs/1902.01193}{abs-1902-01193}~\cite{abs-1902-01193} & \href{works/abs-1902-01193.pdf}{Solving Nurse Scheduling Problem Using Constraint Programming Technique} &  &  & 0 &  &  &  &  &  &  & \ref{a:abs-1902-01193} & \ref{b:abs-1902-01193}\\
\rowlabel{c:abs-1902-09244}abs-1902-09244 \href{http://arxiv.org/abs/1902.09244}{abs-1902-09244}~\cite{abs-1902-09244} & \href{works/abs-1902-09244.pdf}{On constraint programming for a new flexible project scheduling problem with resource constraints} &  & benchmark, industry partner, real-world & 0 &  &  &  &  &  &  & \ref{a:abs-1902-09244} & \ref{b:abs-1902-09244}\\
\rowlabel{c:abs-1911-04766}abs-1911-04766 \href{http://arxiv.org/abs/1911.04766}{abs-1911-04766}~\cite{abs-1911-04766} & \href{works/abs-1911-04766.pdf}{Investigating Constraint Programming and Hybrid Methods for Real World Industrial Test Laboratory Scheduling} &  & real-world, generated instance, industrial partner, github, benchmark, instance generator, real-life & 10 &  &  &  &  &  &  & \ref{a:abs-1911-04766} & \ref{b:abs-1911-04766}\\
\rowlabel{c:BaptisteB18}BaptisteB18 \href{https://doi.org/10.1016/j.dam.2017.05.001}{BaptisteB18}~\cite{BaptisteB18} & \href{works/BaptisteB18.pdf}{Redundant cumulative constraints to compute preemptive bounds} &  &  & 1 &  &  &  &  &  &  & \ref{a:BaptisteB18} & \ref{b:BaptisteB18}\\
\rowlabel{c:BorghesiBLMB18}BorghesiBLMB18 \href{https://doi.org/10.1016/j.suscom.2018.05.007}{BorghesiBLMB18}~\cite{BorghesiBLMB18} & \href{works/BorghesiBLMB18.pdf}{Scheduling-based power capping in high performance computing systems} &  & benchmark, real-life & 3 &  &  &  &  &  &  & \ref{a:BorghesiBLMB18} & \ref{b:BorghesiBLMB18}\\
\rowlabel{c:FahimiOQ18}FahimiOQ18 \href{https://doi.org/10.1007/s10601-018-9282-9}{FahimiOQ18}~\cite{FahimiOQ18} & \href{works/FahimiOQ18.pdf}{Linear-time filtering algorithms for the disjunctive constraint and a quadratic filtering algorithm for the cumulative not-first not-last} & Choco & benchmark, random instance & 0 & (y) &  & n &  & RCPSP & \su{disjunctive cumulative} & \ref{a:FahimiOQ18} & \ref{b:FahimiOQ18}\\
\rowlabel{c:GedikKEK18}GedikKEK18 \href{https://doi.org/10.1016/j.cie.2018.05.014}{GedikKEK18}~\cite{GedikKEK18} & \href{works/GedikKEK18.pdf}{A constraint programming approach for solving unrelated parallel machine scheduling problem} &  & benchmark & 9 &  &  &  &  &  &  & \ref{a:GedikKEK18} & \ref{b:GedikKEK18}\\
\rowlabel{c:GokgurHO18}GokgurHO18 \href{https://doi.org/10.1080/00207543.2017.1421781}{GokgurHO18}~\cite{GokgurHO18} & \href{works/GokgurHO18.pdf}{Parallel machine scheduling with tool loading: a constraint programming approach} &  & real-life, real-world & 9 &  &  &  &  &  &  & \ref{a:GokgurHO18} & \ref{b:GokgurHO18}\\
\rowlabel{c:Ham18}Ham18 \href{https://api.semanticscholar.org/CorpusID:116853255}{Ham18}~\cite{Ham18} & \href{works/Ham18.pdf}{Integrated scheduling of m-truck, m-drone, and m-depot constrained by time-window, drop-pickup, and m-visit using constraint programming} &  &  & 7 &  &  &  &  &  &  & \ref{a:Ham18} & \ref{b:Ham18}\\
\rowlabel{c:LaborieRSV18}LaborieRSV18 \href{https://doi.org/10.1007/s10601-018-9281-x}{LaborieRSV18}~\cite{LaborieRSV18} & \href{works/LaborieRSV18.pdf}{{IBM} {ILOG} {CP} optimizer for scheduling - 20+ years of scheduling with constraints at {IBM/ILOG}} & OP Opt & real-world, CSPlib, benchmark & 3 & - &  & - & - & - & - & \ref{a:LaborieRSV18} & \ref{b:LaborieRSV18}\\
\rowlabel{c:PourDERB18}PourDERB18 \href{https://doi.org/10.1016/j.ejor.2017.08.033}{PourDERB18}~\cite{PourDERB18} & \href{works/PourDERB18.pdf}{A hybrid Constraint Programming/Mixed Integer Programming framework for the preventive signaling maintenance crew scheduling problem} &  & real-life, benchmark, real-world, generated instance & 1 &  &  &  &  &  &  & \ref{a:PourDERB18} & \ref{b:PourDERB18}\\
\rowlabel{c:ShinBBHO18}ShinBBHO18 \href{https://doi.org/10.1109/TSMC.2017.2681623}{ShinBBHO18}~\cite{ShinBBHO18} & \href{works/ShinBBHO18.pdf}{Discrete-Event Simulation and Integer Linear Programming for Constraint-Aware Resource Scheduling} &  & github, real-world & 4 &  &  &  &  &  &  & \ref{a:ShinBBHO18} & \ref{b:ShinBBHO18}\\
\rowlabel{c:TangLWSK18}TangLWSK18 \href{https://doi.org/10.1111/mice.12277}{TangLWSK18}~\cite{TangLWSK18} & \href{works/TangLWSK18.pdf}{Scheduling Optimization of Linear Schedule with Constraint Programming} &  &  & 0 &  &  &  &  &  &  & \ref{a:TangLWSK18} & \ref{b:TangLWSK18}\\
\rowlabel{c:ZhangW18}ZhangW18 \href{https://doi.org/10.1109/TEM.2017.2785774}{ZhangW18}~\cite{ZhangW18} & \href{works/ZhangW18.pdf}{Flexible Assembly Job-Shop Scheduling With Sequence-Dependent Setup Times and Part Sharing in a Dynamic Environment: Constraint Programming Model, Mixed-Integer Programming Model, and Dispatching Rules} &  & benchmark & 0 &  &  &  &  &  &  & \ref{a:ZhangW18} & \ref{b:ZhangW18}\\
\rowlabel{c:KreterSS17}KreterSS17 \href{https://doi.org/10.1007/s10601-016-9266-6}{KreterSS17}~\cite{KreterSS17} & \href{works/KreterSS17.pdf}{Using constraint programming for solving RCPSP/max-cal} & \su{MiniZinc Chuffed Cplex} & benchmark & 5 & dead &  &  & \cite{KreterSS15} & RCPSP & \su{cumulative cumulativeCalendar} & \ref{a:KreterSS17} & \ref{b:KreterSS17}\\
\rowlabel{c:NattafAL17}NattafAL17 \href{https://doi.org/10.1007/s10601-017-9271-4}{NattafAL17}~\cite{NattafAL17} & \href{works/NattafAL17.pdf}{Cumulative scheduling with variable task profiles and concave piecewise linear processing rate functions} & Cplex & real-world & 2 & n &  & n & - & CECSP & - & \ref{a:NattafAL17} & \ref{b:NattafAL17}\\
\rowlabel{c:Bonfietti16}Bonfietti16 \href{https://doi.org/10.3233/IA-160095}{Bonfietti16}~\cite{Bonfietti16} & \href{works/Bonfietti16.pdf}{A constraint programming scheduling solver for the MPOpt programming environment} &  & benchmark & 10 &  &  &  &  &  &  & \ref{a:Bonfietti16} & \ref{b:Bonfietti16}\\
\rowlabel{c:BridiBLMB16}BridiBLMB16 \href{https://doi.org/10.1109/TPDS.2016.2516997}{BridiBLMB16}~\cite{BridiBLMB16} & \href{works/BridiBLMB16.pdf}{A Constraint Programming Scheduler for Heterogeneous High-Performance Computing Machines} &  & real-world, real-life & 0 &  &  &  &  &  &  & \ref{a:BridiBLMB16} & \ref{b:BridiBLMB16}\\
\rowlabel{c:DoulabiRP16}DoulabiRP16 \href{https://doi.org/10.1287/ijoc.2015.0686}{DoulabiRP16}~\cite{DoulabiRP16} & \href{works/DoulabiRP16.pdf}{A Constraint-Programming-Based Branch-and-Price-and-Cut Approach for Operating Room Planning and Scheduling} &  & real-world, generated instance & 3 &  &  &  &  &  &  & \ref{a:DoulabiRP16} & \ref{b:DoulabiRP16}\\
\rowlabel{c:NovaraNH16}NovaraNH16 \href{https://doi.org/10.1016/j.compchemeng.2016.04.030}{NovaraNH16}~\cite{NovaraNH16} & \href{works/NovaraNH16.pdf}{A novel constraint programming model for large-scale scheduling problems in multiproduct multistage batch plants: Limited resources and campaign-based operation} &  & CSPlib, benchmark & 5 &  &  &  &  &  &  & \ref{a:NovaraNH16} & \ref{b:NovaraNH16}\\
\rowlabel{c:ZarandiKS16}ZarandiKS16 \href{https://doi.org/10.1007/s10845-013-0860-9}{ZarandiKS16}~\cite{ZarandiKS16} & \href{works/ZarandiKS16.pdf}{A constraint programming model for the scheduling of {JIT} cross-docking systems with preemption} &  & real-world & 0 &  &  &  &  &  &  & \ref{a:ZarandiKS16} & \ref{b:ZarandiKS16}\\
\rowlabel{c:EvenSH15a}EvenSH15a \href{http://arxiv.org/abs/1505.02487}{EvenSH15a}~\cite{EvenSH15a} & \href{works/EvenSH15a.pdf}{A Constraint Programming Approach for Non-Preemptive Evacuation Scheduling} &  & real-world, real-life & 2 &  &  &  &  &  &  & \ref{a:EvenSH15a} & \ref{b:EvenSH15a}\\
\rowlabel{c:GoelSHFS15}GoelSHFS15 \href{https://doi.org/10.1016/j.ejor.2014.09.048}{GoelSHFS15}~\cite{GoelSHFS15} & \href{works/GoelSHFS15.pdf}{Constraint programming for {LNG} ship scheduling and inventory management} &  &  & 0 &  &  &  &  &  &  & \ref{a:GoelSHFS15} & \ref{b:GoelSHFS15}\\
\rowlabel{c:Kameugne15}Kameugne15 \href{https://doi.org/10.1007/s10601-015-9227-5}{Kameugne15}~\cite{Kameugne15} & \href{works/Kameugne15.pdf}{Propagation techniques of resource constraint for cumulative scheduling} & - &  & 2 & - &  & - & \href{https://www.a4cp.org/sites/default/files/roger_kameugne_-_propagation_techniques_of_resource_constraint_for_cumulative_scheduling.pdf}{PhDThesis} & RCPSP &  & \ref{a:Kameugne15} & \ref{b:Kameugne15}\\
\rowlabel{c:LetortCB15}LetortCB15 \href{https://doi.org/10.1007/s10601-014-9172-8}{LetortCB15}~\cite{LetortCB15} & \href{works/LetortCB15.pdf}{Synchronized sweep algorithms for scalable scheduling constraints} & \su{Choco SICStus} & generated instance, Roadef, benchmark, random instance & 4 & dead &  & - & \cite{LetortCB13} & - & \su{cumulative dimCumulative dimCumulativePrecedences} & \ref{a:LetortCB15} & \ref{b:LetortCB15}\\
\rowlabel{c:NattafAL15}NattafAL15 \href{https://doi.org/10.1007/s10601-015-9192-z}{NattafAL15}~\cite{NattafAL15} & \href{works/NattafAL15.pdf}{A hybrid exact method for a scheduling problem with a continuous resource and energy constraints} & Cplex & generated instance & 1 & n &  & n &  & CSCSP &  & \ref{a:NattafAL15} & \ref{b:NattafAL15}\\
\rowlabel{c:Siala15}Siala15 \href{https://doi.org/10.1007/s10601-015-9213-y}{Siala15}~\cite{Siala15} & \href{works/Siala15.pdf}{Search, propagation, and learning in sequencing and scheduling problems} & - & benchmark & 2 & - &  & - & \href{https://www.a4cp.org/sites/default/files/mohamed_siala_-_search_propagation_and_learning_in_sequencing_and_scheduling_problems.pdf}{PhD Thesis} &  &  & \ref{a:Siala15} & \ref{b:Siala15}\\
\rowlabel{c:SimoninAHL15}SimoninAHL15 \href{https://doi.org/10.1007/s10601-014-9169-3}{SimoninAHL15}~\cite{SimoninAHL15} & \href{works/SimoninAHL15.pdf}{Scheduling scientific experiments for comet exploration} & \su{MOST Ilog Scheduler} &  & 0 & n &  & n & \cite{SimoninAHL12} &  & \su{cumulative dataTransfer} & \ref{a:SimoninAHL15} & \ref{b:SimoninAHL15}\\
\rowlabel{c:WangMD15}WangMD15 \href{https://doi.org/10.1016/j.ejor.2015.06.008}{WangMD15}~\cite{WangMD15} & \href{works/WangMD15.pdf}{Scheduling operating theatres: Mixed integer programming vs. constraint programming} &  & real-life, real-world & 2 &  &  &  &  &  &  & \ref{a:WangMD15} & \ref{b:WangMD15}\\
\rowlabel{c:BonfiettiLBM14}BonfiettiLBM14 \href{https://doi.org/10.1016/j.artint.2013.09.006}{BonfiettiLBM14}~\cite{BonfiettiLBM14} & \href{works/BonfiettiLBM14.pdf}{{CROSS} cyclic resource-constrained scheduling solver} &  & real-world, generated instance, industrial instance, benchmark & 0 &  &  &  &  &  &  & \ref{a:BonfiettiLBM14} & \ref{b:BonfiettiLBM14}\\
\rowlabel{c:GrimesIOS14}GrimesIOS14 \href{https://doi.org/10.1016/j.suscom.2014.08.009}{GrimesIOS14}~\cite{GrimesIOS14} & \href{works/GrimesIOS14.pdf}{Analyzing the impact of electricity price forecasting on energy cost-aware scheduling} &  & real-world, real-life & 9 &  &  &  &  &  &  & \ref{a:GrimesIOS14} & \ref{b:GrimesIOS14}\\
\rowlabel{c:KameugneFSN14}KameugneFSN14 \href{https://doi.org/10.1007/s10601-013-9157-z}{KameugneFSN14}~\cite{KameugneFSN14} & \href{works/KameugneFSN14.pdf}{A quadratic edge-finding filtering algorithm for cumulative resource constraints} & Gecode & random instance, benchmark & 2 & \href{https://figshare.com/articles/dataset/Comparison_of_edge_finding_and_extended_edge_finding_filtering_algorithms/736454}{y} &  &  & \cite{KameugneFSN11} & CuSP & cumulative & \ref{a:KameugneFSN14} & \ref{b:KameugneFSN14}\\
\rowlabel{c:NovasH14}NovasH14 \href{https://doi.org/10.1016/j.eswa.2013.09.026}{NovasH14}~\cite{NovasH14} & \href{works/NovasH14.pdf}{Integrated scheduling of resource-constrained flexible manufacturing systems using constraint programming} &  & benchmark & 0 &  &  &  &  &  &  & \ref{a:NovasH14} & \ref{b:NovasH14}\\
\rowlabel{c:BegB13}BegB13 \href{http://doi.acm.org/10.1145/2512470}{BegB13}~\cite{BegB13} & \href{works/BegB13.pdf}{A constraint programming approach for integrated spatial and temporal scheduling for clustered architectures} &  & benchmark & 0 &  &  &  &  &  &  & \ref{a:BegB13} & \ref{b:BegB13}\\
\rowlabel{c:HeinzSB13}HeinzSB13 \href{https://doi.org/10.1007/s10601-012-9136-9}{HeinzSB13}~\cite{HeinzSB13} & \href{works/HeinzSB13.pdf}{Using dual presolving reductions to reformulate cumulative constraints} & \su{Cplex SCIP} & benchmark & 1 & ref &  & - & - & \su{RCPSP RCPSP/max} & cumulative & \ref{a:HeinzSB13} & \ref{b:HeinzSB13}\\
\rowlabel{c:OzturkTHO13}OzturkTHO13 \href{https://doi.org/10.1007/s10601-013-9142-6}{OzturkTHO13}~\cite{OzturkTHO13} & \href{works/OzturkTHO13.pdf}{Balancing and scheduling of flexible mixed model assembly lines} & \su{{Ilog Solver} {Ilog Scheduler} Cplex} & real-world, real-life & 2 & \href{https://github.com/ozturkcemal/SBSFMMAL}{y} &  & - & - & SBSFMMAL & \su{alddifferent disjunctive} & \ref{a:OzturkTHO13} & \ref{b:OzturkTHO13}\\
\rowlabel{c:HeinzSSW12}HeinzSSW12 \href{https://doi.org/10.1007/s10601-011-9113-8}{HeinzSSW12}~\cite{HeinzSSW12} & \href{works/HeinzSSW12.pdf}{Solving steel mill slab design problems} &  & real-world, CSPlib & 2 & Cplex &  & dead & - & SMSDP & - & \ref{a:HeinzSSW12} & \ref{b:HeinzSSW12}\\
\rowlabel{c:LimtanyakulS12}LimtanyakulS12 \href{https://doi.org/10.1007/s10601-012-9118-y}{LimtanyakulS12}~\cite{LimtanyakulS12} & \href{works/LimtanyakulS12.pdf}{Improvements of constraint programming and hybrid methods for scheduling of tests on vehicle prototypes} & \su{Cplex {Ilog Scheduler}} & random instance, real-life, generated instance, industrial partner, benchmark & 1 & dead &  & - & - &  &  & \ref{a:LimtanyakulS12} & \ref{b:LimtanyakulS12}\\
\rowlabel{c:LombardiM12}LombardiM12 \href{https://doi.org/10.1007/s10601-011-9115-6}{LombardiM12}~\cite{LombardiM12} & \href{works/LombardiM12.pdf}{Optimal methods for resource allocation and scheduling: a cross-disciplinary survey} & - & real-world, benchmark & 0 & - &  & - & - & survey & - & \ref{a:LombardiM12} & \ref{b:LombardiM12}\\
\rowlabel{c:LombardiM12a}LombardiM12a \href{https://doi.org/10.1016/j.artint.2011.12.001}{LombardiM12a}~\cite{LombardiM12a} & \href{works/LombardiM12a.pdf}{A min-flow algorithm for Minimal Critical Set detection in Resource Constrained Project Scheduling} &  & benchmark & 1 &  &  &  &  &  &  & \ref{a:LombardiM12a} & \ref{b:LombardiM12a}\\
\rowlabel{c:NovasH12}NovasH12 \href{https://doi.org/10.1016/j.compchemeng.2012.01.005}{NovasH12}~\cite{NovasH12} & \href{works/NovasH12.pdf}{A comprehensive constraint programming approach for the rolling horizon-based scheduling of automated wet-etch stations} &  &  & 0 &  &  &  &  &  &  & \ref{a:NovasH12} & \ref{b:NovasH12}\\
\rowlabel{c:BartakS11}BartakS11 \href{https://doi.org/10.1007/s10601-011-9109-4}{BartakS11}~\cite{BartakS11} & \href{works/BartakS11.pdf}{Constraint satisfaction for planning and scheduling problems} & - & random instance, real-world, real-life & 2 & - &  & - &  & survey &  & \ref{a:BartakS11} & \ref{b:BartakS11}\\
\rowlabel{c:BeckFW11}BeckFW11 \href{https://doi.org/10.1287/ijoc.1100.0388}{BeckFW11}~\cite{BeckFW11} & \href{works/BeckFW11.pdf}{Combining Constraint Programming and Local Search for Job-Shop Scheduling} &  & real-world, benchmark & 0 &  &  &  &  &  &  & \ref{a:BeckFW11} & \ref{b:BeckFW11}\\
\rowlabel{c:BeldiceanuCDP11}BeldiceanuCDP11 \href{https://doi.org/10.1007/s10479-010-0731-0}{BeldiceanuCDP11}~\cite{BeldiceanuCDP11} & \href{works/BeldiceanuCDP11.pdf}{New filtering for the \emph{cumulative} constraint in the context of non-overlapping rectangles} &  & benchmark & 1 &  &  &  &  &  &  & \ref{a:BeldiceanuCDP11} & \ref{b:BeldiceanuCDP11}\\
\rowlabel{c:BeniniLMR11}BeniniLMR11 \href{https://doi.org/10.1007/s10479-010-0718-x}{BeniniLMR11}~\cite{BeniniLMR11} & \href{works/BeniniLMR11.pdf}{Optimal resource allocation and scheduling for the {CELL} {BE} platform} &  & benchmark, real-world, instance generator & 0 &  &  &  &  &  &  & \ref{a:BeniniLMR11} & \ref{b:BeniniLMR11}\\
\rowlabel{c:HachemiGR11}HachemiGR11 \href{https://doi.org/10.1007/s10479-010-0698-x}{HachemiGR11}~\cite{HachemiGR11} & \href{works/HachemiGR11.pdf}{A hybrid constraint programming approach to the log-truck scheduling problem} &  &  & 1 &  &  &  &  &  &  & \ref{a:HachemiGR11} & \ref{b:HachemiGR11}\\
\rowlabel{c:KelbelH11}KelbelH11 \href{https://doi.org/10.1007/s10845-009-0318-2}{KelbelH11}~\cite{KelbelH11} & \href{works/KelbelH11.pdf}{Solving production scheduling with earliness/tardiness penalties by constraint programming} &  & benchmark, random instance, generated instance & 3 &  &  &  &  &  &  & \ref{a:KelbelH11} & \ref{b:KelbelH11}\\
\rowlabel{c:KovacsB11}KovacsB11 \href{https://doi.org/10.1007/s10601-009-9088-x}{KovacsB11}~\cite{KovacsB11} & \href{works/KovacsB11.pdf}{A global constraint for total weighted completion time for unary resources} & Ilog Scheduler & benchmark & 2 & n &  & n & - &  & Completion & \ref{a:KovacsB11} & \ref{b:KovacsB11}\\
\rowlabel{c:KovacsK11}KovacsK11 \href{https://doi.org/10.1007/s10601-010-9102-3}{KovacsK11}~\cite{KovacsK11} & \href{works/KovacsK11.pdf}{Constraint programming approach to a bilevel scheduling problem} & Ilog Solver &  & 2 & n &  & n & - & Bilevel Opt &  & \ref{a:KovacsK11} & \ref{b:KovacsK11}\\
\rowlabel{c:SchausHMCMD11}SchausHMCMD11 \href{https://doi.org/10.1007/s10601-010-9100-5}{SchausHMCMD11}~\cite{SchausHMCMD11} & \href{works/SchausHMCMD11.pdf}{Solving Steel Mill Slab Problems with constraint-based techniques: CP, LNS, and {CBLS}} & Comet & benchmark, CSPlib, generated instance & 3 & dead &  &  &  & SMSDP &  & \ref{a:SchausHMCMD11} & \ref{b:SchausHMCMD11}\\
\rowlabel{c:SchuttFSW11}SchuttFSW11 \href{https://doi.org/10.1007/s10601-010-9103-2}{SchuttFSW11}~\cite{SchuttFSW11} & \href{works/SchuttFSW11.pdf}{Explaining the cumulative propagator} & MiniZinc & benchmark, real-world & 7 & PSPLib &  & - & - & RCPSP & cumulative & \ref{a:SchuttFSW11} & \ref{b:SchuttFSW11}\\
\rowlabel{c:TopalogluO11}TopalogluO11 \href{https://doi.org/10.1016/j.cor.2010.04.018}{TopalogluO11}~\cite{TopalogluO11} & \href{works/TopalogluO11.pdf}{A constraint programming-based solution approach for medical resident scheduling problems} &  & real-life & 2 &  &  &  &  &  &  & \ref{a:TopalogluO11} & \ref{b:TopalogluO11}\\
\rowlabel{c:TrojetHL11}TrojetHL11 \href{https://doi.org/10.1016/j.cie.2010.08.014}{TrojetHL11}~\cite{TrojetHL11} & \href{works/TrojetHL11.pdf}{Project scheduling under resource constraints: Application of the cumulative global constraint in a decision support framework} &  & real-world & 2 &  &  &  &  &  &  & \ref{a:TrojetHL11} & \ref{b:TrojetHL11}\\
\rowlabel{c:BartakCS10}BartakCS10 \href{https://doi.org/10.1007/s10479-008-0492-1}{BartakCS10}~\cite{BartakCS10} & \href{works/BartakCS10.pdf}{Discovering implied constraints in precedence graphs with alternatives} &  & benchmark, real-life, real-world & 3 &  &  &  &  &  &  & \ref{a:BartakCS10} & \ref{b:BartakCS10}\\
\rowlabel{c:BartakSR10}BartakSR10 \href{https://doi.org/10.1017/S0269888910000202}{BartakSR10}~\cite{BartakSR10} & \href{works/BartakSR10.pdf}{New trends in constraint satisfaction, planning, and scheduling: a survey} &  & real-life, real-world & 0 &  &  &  &  &  &  & \ref{a:BartakSR10} & \ref{b:BartakSR10}\\
\rowlabel{c:LombardiM10a}LombardiM10a \href{https://doi.org/10.1016/j.artint.2010.02.004}{LombardiM10a}~\cite{LombardiM10a} & \href{works/LombardiM10a.pdf}{Allocation and scheduling of Conditional Task Graphs} &  & real-world, benchmark, real-life & 3 &  &  &  &  &  &  & \ref{a:LombardiM10a} & \ref{b:LombardiM10a}\\
\rowlabel{c:LopesCSM10}LopesCSM10 \href{https://doi.org/10.1007/s10601-009-9086-z}{LopesCSM10}~\cite{LopesCSM10} & \href{works/LopesCSM10.pdf}{A hybrid model for a multiproduct pipeline planning and scheduling problem} & Ilog Solver & benchmark, real-world & 2 & - &  & - & \cite{MouraSCL08,MouraSCL08a} &  &  & \ref{a:LopesCSM10} & \ref{b:LopesCSM10}\\
\rowlabel{c:NovasH10}NovasH10 \href{https://doi.org/10.1016/j.compchemeng.2010.07.011}{NovasH10}~\cite{NovasH10} & \href{works/NovasH10.pdf}{Reactive scheduling framework based on domain knowledge and constraint programming} &  &  & 0 &  &  &  &  &  &  & \ref{a:NovasH10} & \ref{b:NovasH10}\\
\rowlabel{c:ZeballosQH10}ZeballosQH10 \href{https://doi.org/10.1016/j.engappai.2009.07.002}{ZeballosQH10}~\cite{ZeballosQH10} & \href{works/ZeballosQH10.pdf}{A constraint programming model for the scheduling of flexible manufacturing systems with machine and tool limitations} &  & benchmark, real-world & 4 &  &  &  &  &  &  & \ref{a:ZeballosQH10} & \ref{b:ZeballosQH10}\\
\rowlabel{c:BocewiczBB09}BocewiczBB09 \href{https://doi.org/10.1504/IJIIDS.2009.023038}{BocewiczBB09}~\cite{BocewiczBB09} & \href{works/BocewiczBB09.pdf}{Logic-algebraic method based and constraints programming driven approach to AGVs scheduling} &  &  & 0 &  &  &  &  &  &  & \ref{a:BocewiczBB09} & \ref{b:BocewiczBB09}\\
\rowlabel{c:GarridoAO09}GarridoAO09 \href{https://doi.org/10.1007/s10951-008-0083-7}{GarridoAO09}~\cite{GarridoAO09} & \href{works/GarridoAO09.pdf}{A constraint programming formulation for planning: from plan scheduling to plan generation} &  & benchmark & 8 &  &  &  &  &  &  & \ref{a:GarridoAO09} & \ref{b:GarridoAO09}\\
\rowlabel{c:RuggieroBBMA09}RuggieroBBMA09 \href{https://doi.org/10.1109/TCAD.2009.2013536}{RuggieroBBMA09}~\cite{RuggieroBBMA09} & \href{works/RuggieroBBMA09.pdf}{Reducing the Abstraction and Optimality Gaps in the Allocation and Scheduling for Variable Voltage/Frequency MPSoC Platforms} &  & instance generator, real-life & 0 &  &  &  &  &  &  & \ref{a:RuggieroBBMA09} & \ref{b:RuggieroBBMA09}\\
\rowlabel{c:abs-0907-0939}abs-0907-0939 \href{http://arxiv.org/abs/0907.0939}{abs-0907-0939}~\cite{abs-0907-0939} & \href{works/abs-0907-0939.pdf}{The Soft Cumulative Constraint} &  & real-world & 0 &  &  &  &  &  &  & \ref{a:abs-0907-0939} & \ref{b:abs-0907-0939}\\
\rowlabel{c:GarridoOS08}GarridoOS08 \href{https://doi.org/10.1016/j.engappai.2008.03.009}{GarridoOS08}~\cite{GarridoOS08} & \href{works/GarridoOS08.pdf}{Planning and scheduling in an e-learning environment. {A} constraint-programming-based approach} &  & real-world & 0 &  &  &  &  &  &  & \ref{a:GarridoOS08} & \ref{b:GarridoOS08}\\
\rowlabel{c:KovacsB08}KovacsB08 \href{https://doi.org/10.1016/j.engappai.2008.03.004}{KovacsB08}~\cite{KovacsB08} & \href{works/KovacsB08.pdf}{A global constraint for total weighted completion time for cumulative resources} &  & benchmark & 0 &  &  &  &  &  &  & \ref{a:KovacsB08} & \ref{b:KovacsB08}\\
\rowlabel{c:LiessM08}LiessM08 \href{https://doi.org/10.1007/s10479-007-0188-y}{LiessM08}~\cite{LiessM08} & \href{works/LiessM08.pdf}{A constraint programming approach for the resource-constrained project scheduling problem} &  & benchmark & 0 &  &  &  &  &  &  & \ref{a:LiessM08} & \ref{b:LiessM08}\\
\rowlabel{c:MalikMB08}MalikMB08 \href{https://doi.org/10.1142/S0218213008003765}{MalikMB08}~\cite{MalikMB08} & \href{works/MalikMB08.pdf}{Optimal Basic Block Instruction Scheduling for Multiple-Issue Processors Using Constraint Programming} &  & benchmark & 0 &  &  &  &  &  &  & \ref{a:MalikMB08} & \ref{b:MalikMB08}\\
\rowlabel{c:Rodriguez07}Rodriguez07 \href{https://www.sciencedirect.com/science/article/pii/S0191261506000233}{Rodriguez07}~\cite{Rodriguez07} & \href{works/Rodriguez07.pdf}{A constraint programming model for real-time train scheduling at junctions} &  & real-life & 2 &  &  &  &  &  &  & \ref{a:Rodriguez07} & \ref{b:Rodriguez07}\\
\rowlabel{c:Simonis07}Simonis07 \href{https://doi.org/10.1007/s10601-006-9011-7}{Simonis07}~\cite{Simonis07} & \href{works/Simonis07.pdf}{Models for Global Constraint Applications} & CHIP &  & 0 & n &  & n &  &  & \su{cumulative diffn cycle inverse} & \ref{a:Simonis07} & \ref{b:Simonis07}\\
\rowlabel{c:Hooker06}Hooker06 \href{https://doi.org/10.1007/s10601-006-8060-2}{Hooker06}~\cite{Hooker06} & \href{works/Hooker06.pdf}{An Integrated Method for Planning and Scheduling to Minimize Tardiness} & \su{OPL Cplex {Ilog Scheduler}} & random instance & 2 & n &  & n & \cite{Hooker05a} & CuSP & \su{cumulative} & \ref{a:Hooker06} & \ref{b:Hooker06}\\
\rowlabel{c:KhayatLR06}KhayatLR06 \href{https://doi.org/10.1016/j.ejor.2005.02.077}{KhayatLR06}~\cite{KhayatLR06} & \href{works/KhayatLR06.pdf}{Integrated production and material handling scheduling using mathematical programming and constraint programming} &  & real-life, benchmark & 1 &  &  &  &  &  &  & \ref{a:KhayatLR06} & \ref{b:KhayatLR06}\\
\rowlabel{c:SadykovW06}SadykovW06 \href{https://doi.org/10.1287/ijoc.1040.0110}{SadykovW06}~\cite{SadykovW06} & \href{works/SadykovW06.pdf}{Integer Programming and Constraint Programming in Solving a Multimachine Assignment Scheduling Problem with Deadlines and Release Dates} &  & generated instance & 1 &  &  &  &  &  &  & \ref{a:SadykovW06} & \ref{b:SadykovW06}\\
\rowlabel{c:SureshMOK06}SureshMOK06 \href{https://doi.org/10.1080/17445760600567842}{SureshMOK06}~\cite{SureshMOK06} & \href{works/SureshMOK06.pdf}{Divisible load scheduling in distributed system with buffer constraints: genetic algorithm and linear programming approach} &  &  & 0 &  &  &  &  &  &  & \ref{a:SureshMOK06} & \ref{b:SureshMOK06}\\
\rowlabel{c:Hooker05}Hooker05 \href{https://doi.org/10.1007/s10601-005-2812-2}{Hooker05}~\cite{Hooker05} & \href{works/Hooker05.pdf}{A Hybrid Method for the Planning and Scheduling} & \su{OPL Cplex {Ilog Scheduler}} & random instance & 0 & n &  & n & \cite{Hooker04} & CuSP & \su{cumulative} & \ref{a:Hooker05} & \ref{b:Hooker05}\\
\rowlabel{c:VilimBC05}VilimBC05 \href{https://doi.org/10.1007/s10601-005-2814-0}{VilimBC05}~\cite{VilimBC05} & \href{works/VilimBC05.pdf}{Extension of \emph{O}(\emph{n} log \emph{n}) Filtering Algorithms for the Unary Resource Constraint to Optional Activities} &  & benchmark, real-life & 0 & n &  & n & \cite{VilimBC04} & JSSP & disjunctive & \ref{a:VilimBC05} & \ref{b:VilimBC05}\\
\rowlabel{c:ZeballosH05}ZeballosH05 \href{http://journal.iberamia.org/index.php/ia/article/view/452/article\%20\%281\%29.pdf}{ZeballosH05}~\cite{ZeballosH05} & \href{works/ZeballosH05.pdf}{A Constraint Programming Approach to {FMS} Scheduling. Consideration of Storage and Transportation Resources} &  &  & 0 &  &  &  &  &  &  & \ref{a:ZeballosH05} & \ref{b:ZeballosH05}\\
\rowlabel{c:PoderBS04}PoderBS04 \href{https://doi.org/10.1016/S0377-2217(02)00756-7}{PoderBS04}~\cite{PoderBS04} & \href{works/PoderBS04.pdf}{Computing a lower approximation of the compulsory part of a task with varying duration and varying resource consumption} &  &  & 0 &  &  &  &  &  &  & \ref{a:PoderBS04} & \ref{b:PoderBS04}\\
\rowlabel{c:KuchcinskiW03}KuchcinskiW03 \href{https://doi.org/10.1016/S1383-7621(03)00075-4}{KuchcinskiW03}~\cite{KuchcinskiW03} & \href{works/KuchcinskiW03.pdf}{Global approach to assignment and scheduling of complex behaviors based on {HCDG} and constraint programming} &  & benchmark & 0 &  &  &  &  &  &  & \ref{a:KuchcinskiW03} & \ref{b:KuchcinskiW03}\\
\rowlabel{c:Tsang03}Tsang03 \href{https://doi.org/10.1023/A:1024016929283}{Tsang03}~\cite{Tsang03} & \href{works/Tsang03.pdf}{Constraint Based Scheduling: Applying Constraint Programming to Scheduling Problems} &  & real-life & 0 &  &  &  &  &  &  & \ref{a:Tsang03} & \ref{b:Tsang03}\\
\rowlabel{c:LorigeonBB02}LorigeonBB02 \href{https://doi.org/10.1057/palgrave.jors.2601421}{LorigeonBB02}~\cite{LorigeonBB02} & \href{works/LorigeonBB02.pdf}{A dynamic programming algorithm for scheduling jobs in a two-machine open shop with an availability constraint} &  &  & 0 &  &  &  &  &  &  & \ref{a:LorigeonBB02} & \ref{b:LorigeonBB02}\\
\rowlabel{c:RodriguezDG02}RodriguezDG02 \href{}{RodriguezDG02}~\cite{RodriguezDG02} & \href{works/RodriguezDG02.pdf}{Railway infrastructure saturation using constraint programming approach} &  &  & 0 &  &  &  &  &  &  & \ref{a:RodriguezDG02} & \ref{b:RodriguezDG02}\\
\rowlabel{c:Timpe02}Timpe02 \href{https://doi.org/10.1007/s00291-002-0107-1}{Timpe02}~\cite{Timpe02} & \href{works/Timpe02.pdf}{Solving planning and scheduling problems with combined integer and constraint programming} &  &  & 0 &  &  &  &  &  &  & \ref{a:Timpe02} & \ref{b:Timpe02}\\
\rowlabel{c:MartinPY01}MartinPY01 \href{https://doi.org/10.1023/A:1016067230126}{MartinPY01}~\cite{MartinPY01} & \href{works/MartinPY01.pdf}{Cane Railway Scheduling via Constraint Logic Programming: Labelling Order and Constraints in a Real-Life Application} &  & real-life & 0 &  &  &  &  &  &  & \ref{a:MartinPY01} & \ref{b:MartinPY01}\\
\rowlabel{c:Mason01}Mason01 \href{https://doi.org/10.1023/A:1016023415105}{Mason01}~\cite{Mason01} & \href{works/Mason01.pdf}{Elastic Constraint Branching, the Wedelin/Carmen Lagrangian Heuristic and Integer Programming for Personnel Scheduling} &  &  & 0 &  &  &  &  &  &  & \ref{a:Mason01} & \ref{b:Mason01}\\
\rowlabel{c:ArtiguesR00}ArtiguesR00 \href{https://doi.org/10.1016/S0377-2217(99)00496-8}{ArtiguesR00}~\cite{ArtiguesR00} & \href{works/ArtiguesR00.pdf}{A polynomial activity insertion algorithm in a multi-resource schedule with cumulative constraints and multiple modes} &  &  & 0 &  &  &  &  &  &  & \ref{a:ArtiguesR00} & \ref{b:ArtiguesR00}\\
\rowlabel{c:BaptisteP00}BaptisteP00 \href{https://doi.org/10.1023/A:1009822502231}{BaptisteP00}~\cite{BaptisteP00} & \href{works/BaptisteP00.pdf}{Constraint Propagation and Decomposition Techniques for Highly Disjunctive and Highly Cumulative Project Scheduling Problems} & CLAIRE & benchmark & 0 & n &  & n &  & RCCSP & cumulative & \ref{a:BaptisteP00} & \ref{b:BaptisteP00}\\
\rowlabel{c:HeipckeCCS00}HeipckeCCS00 \href{https://doi.org/10.1023/A:1009860311452}{HeipckeCCS00}~\cite{HeipckeCCS00} & \href{works/HeipckeCCS00.pdf}{Scheduling under Labour Resource Constraints} & \su{COME SchedEns} & benchmark, instance generator & 0 & dead &  & n & - &  &  & \ref{a:HeipckeCCS00} & \ref{b:HeipckeCCS00}\\
\rowlabel{c:KorbaaYG00}KorbaaYG00 \href{https://doi.org/10.1016/S0947-3580(00)71113-7}{KorbaaYG00}~\cite{KorbaaYG00} & \href{works/KorbaaYG00.pdf}{Solving Transient Scheduling Problems with Constraint Programming} &  &  & 0 &  &  &  &  &  &  & \ref{a:KorbaaYG00} & \ref{b:KorbaaYG00}\\
\rowlabel{c:LopezAKYG00}LopezAKYG00 \href{https://doi.org/10.1016/S0947-3580(00)71114-9}{LopezAKYG00}~\cite{LopezAKYG00} & \href{works/LopezAKYG00.pdf}{Discussion on: 'Solving Transient Scheduling Problems with Constraint Programming' by O. Korbaa, P. Yim, and {J.-C.} Gentina} &  &  & 0 &  &  &  &  &  &  & \ref{a:LopezAKYG00} & \ref{b:LopezAKYG00}\\
\rowlabel{c:SakkoutW00}SakkoutW00 \href{https://doi.org/10.1023/A:1009856210543}{SakkoutW00}~\cite{SakkoutW00} & \href{works/SakkoutW00.pdf}{Probe Backtrack Search for Minimal Perturbation in Dynamic Scheduling} & \su{Cplex ECLiPSe} & benchmark, real-world & 0 & n &  & n & - & KRFP &  & \ref{a:SakkoutW00} & \ref{b:SakkoutW00}\\
\rowlabel{c:SchildW00}SchildW00 \href{https://doi.org/10.1023/A:1009804226473}{SchildW00}~\cite{SchildW00} & \href{works/SchildW00.pdf}{Scheduling of Time-Triggered Real-Time Systems} & OZ &  & 0 & n &  & n & - &  & disjunctive & \ref{a:SchildW00} & \ref{b:SchildW00}\\
\rowlabel{c:SourdN00}SourdN00 \href{https://doi.org/10.1287/ijoc.12.4.341.11881}{SourdN00}~\cite{SourdN00} & \href{works/SourdN00.pdf}{Multiple-Machine Lower Bounds for Shop-Scheduling Problems} &  & real-life, benchmark & 1 &  &  &  &  &  &  & \ref{a:SourdN00} & \ref{b:SourdN00}\\
\rowlabel{c:BensanaLV99}BensanaLV99 \href{https://doi.org/10.1023/A:1026488509554}{BensanaLV99}~\cite{BensanaLV99} & \href{works/BensanaLV99.pdf}{Earth Observation Satellite Management} & Ilog Solver & benchmark & 0 & ? &  & - & - &  &  & \ref{a:BensanaLV99} & \ref{b:BensanaLV99}\\
\rowlabel{c:BelhadjiI98}BelhadjiI98 \href{https://doi.org/10.1023/A:1009777711218}{BelhadjiI98}~\cite{BelhadjiI98} & \href{works/BelhadjiI98.pdf}{Temporal Constraint Satisfaction Techniques in Job Shop Scheduling Problem Solving} & - & real-life & 0 & n &  & n & - & \su{TCSP JSSP} &  & \ref{a:BelhadjiI98} & \ref{b:BelhadjiI98}\\
\rowlabel{c:NuijtenP98}NuijtenP98 \href{https://doi.org/10.1023/A:1009687210594}{NuijtenP98}~\cite{NuijtenP98} & \href{works/NuijtenP98.pdf}{Constraint-Based Job Shop Scheduling with {\textbackslash}sc Ilog Scheduler} &  & real-life & 0 &  &  &  &  &  &  & \ref{a:NuijtenP98} & \ref{b:NuijtenP98}\\
\rowlabel{c:PapaB98}PapaB98 \href{https://doi.org/10.1023/A:1009723704757}{PapaB98}~\cite{PapaB98} & \href{works/PapaB98.pdf}{Resource Constraints for Preemptive Job-shop Scheduling} & \su{{Ilog Solver} Claire} & benchmark & 0 & dead &  & - & - & PJSSP & \su{disjunctive flow} & \ref{a:PapaB98} & \ref{b:PapaB98}\\
\rowlabel{c:Darby-DowmanLMZ97}Darby-DowmanLMZ97 \href{https://doi.org/10.1007/BF00137871}{Darby-DowmanLMZ97}~\cite{Darby-DowmanLMZ97} & \href{works/Darby-DowmanLMZ97.pdf}{Constraint Logic Programming and Integer Programming Approaches and Their Collaboration in Solving an Assignment Scheduling Problem} & \su{Cplex ECLiPSe} & real-life, real-world, benchmark & 0 & n &  & n & - & MGAP &  & \ref{a:Darby-DowmanLMZ97} & \ref{b:Darby-DowmanLMZ97}\\
\rowlabel{c:FalaschiGMP97}FalaschiGMP97 \href{https://doi.org/10.1006/inco.1997.2638}{FalaschiGMP97}~\cite{FalaschiGMP97} & \href{works/FalaschiGMP97.pdf}{Constraint Logic Programming with Dynamic Scheduling: {A} Semantics Based on Closure Operators} &  &  & 0 &  &  &  &  &  &  & \ref{a:FalaschiGMP97} & \ref{b:FalaschiGMP97}\\
\rowlabel{c:LammaMM97}LammaMM97 \href{https://doi.org/10.1016/S0954-1810(96)00002-7}{LammaMM97}~\cite{LammaMM97} & \href{works/LammaMM97.pdf}{A distributed constraint-based scheduler} &  & real-life & 0 &  &  &  &  &  &  & \ref{a:LammaMM97} & \ref{b:LammaMM97}\\
\rowlabel{c:Zhou97}Zhou97 \href{https://doi.org/10.1023/A:1009757726572}{Zhou97}~\cite{Zhou97} & \href{works/Zhou97.pdf}{A Permutation-Based Approach for Solving the Job-Shop Problem} & - & benchmark & 0 & n &  & n & \cite{Zhou96} & JSSP & \su{sort alldifferent permutation} & \ref{a:Zhou97} & \ref{b:Zhou97}\\
\rowlabel{c:Wallace96}Wallace96 \href{https://doi.org/10.1007/BF00143881}{Wallace96}~\cite{Wallace96} & \href{works/Wallace96.pdf}{Practical Applications of Constraint Programming} & - &  & 0 & - &  & - & - & Survey & - & \ref{a:Wallace96} & \ref{b:Wallace96}\\
\rowlabel{c:BeldiceanuC94}BeldiceanuC94 \href{https://www.sciencedirect.com/science/article/pii/0895717794901279}{BeldiceanuC94}~\cite{BeldiceanuC94} & \href{works/BeldiceanuC94.pdf}{Introducing Global Constraints in {CHIP}} &  & real-world, real-life, benchmark & 0 &  &  &  &  &  &  & \ref{a:BeldiceanuC94} & \ref{b:BeldiceanuC94}\\
\rowlabel{c:AggounB93}AggounB93 \href{https://www.sciencedirect.com/science/article/pii/089571779390068A}{AggounB93}~\cite{AggounB93} & \href{works/AggounB93.pdf}{Extending {CHIP} in order to solve complex scheduling and placement problems} &  & real-world & 0 &  &  &  &  &  &  & \ref{a:AggounB93} & \ref{b:AggounB93}\\
\rowlabel{c:Tay92}Tay92 \href{}{Tay92}~\cite{Tay92} & \href{}{{COPS:} {A} Constraint Programming Approach to Resource-Limited Project Scheduling} &  &  & 0 &  &  &  &  &  &  & \ref{a:Tay92} & No\\
\rowlabel{c:DincbasSH90}DincbasSH90 \href{https://doi.org/10.1016/0743-1066(90)90052-7}{DincbasSH90}~\cite{DincbasSH90} & \href{works/DincbasSH90.pdf}{Solving Large Combinatorial Problems in Logic Programming} &  & real-life & 0 &  &  &  &  &  &  & \ref{a:DincbasSH90} & \ref{b:DincbasSH90}\\
\end{longtable}
}



\clearpage
\section{Authors}

{\scriptsize
\begin{longtable}{p{4cm}rrp{18cm}}
\rowcolor{white}\caption{Co-Authors of Articles/Papers}\\ \toprule
\rowcolor{white}Author & \shortstack{Nr\\Works} & \shortstack{Nr\\Cites} & Entries \\ \midrule\endhead
\bottomrule
\endfoot
\rowlabel{auth:a89}J. Christopher Beck & 49 &701 &\href{works/LuoB22.pdf}{LuoB22}~\cite{LuoB22}, \href{works/ZhangBB22.pdf}{ZhangBB22}~\cite{ZhangBB22}, \href{works/TangB20.pdf}{TangB20}~\cite{TangB20}, \href{}{RoshanaeiBAUB20}~\cite{RoshanaeiBAUB20}, \href{works/TranPZLDB18.pdf}{TranPZLDB18}~\cite{TranPZLDB18}, \href{works/TranVNB17.pdf}{TranVNB17}~\cite{TranVNB17}, \href{works/TranVNB17a.pdf}{TranVNB17a}~\cite{TranVNB17a}, \href{works/CohenHB17.pdf}{CohenHB17}~\cite{CohenHB17}, \href{works/BoothNB16.pdf}{BoothNB16}~\cite{BoothNB16}, \href{works/KuB16.pdf}{KuB16}~\cite{KuB16}, \href{works/TranAB16.pdf}{TranAB16}~\cite{TranAB16}, \href{works/TranWDRFOVB16.pdf}{TranWDRFOVB16}~\cite{TranWDRFOVB16}, \href{works/LuoVLBM16.pdf}{LuoVLBM16}~\cite{LuoVLBM16}, \href{works/TranDRFWOVB16.pdf}{TranDRFWOVB16}~\cite{TranDRFWOVB16}, \href{works/BajestaniB15.pdf}{BajestaniB15}~\cite{BajestaniB15}, \href{works/KoschB14.pdf}{KoschB14}~\cite{KoschB14}, \href{works/TerekhovTDB14.pdf}{TerekhovTDB14}~\cite{TerekhovTDB14}, \href{works/LouieVNB14.pdf}{LouieVNB14}~\cite{LouieVNB14}, \href{works/HeinzSB13.pdf}{HeinzSB13}~\cite{HeinzSB13}, \href{works/HeinzKB13.pdf}{HeinzKB13}~\cite{HeinzKB13}, \href{works/BajestaniB13.pdf}{BajestaniB13}~\cite{BajestaniB13}, \href{works/TranTDB13.pdf}{TranTDB13}~\cite{TranTDB13}, \href{works/HeinzB12.pdf}{HeinzB12}~\cite{HeinzB12}, \href{works/TerekhovDOB12.pdf}{TerekhovDOB12}~\cite{TerekhovDOB12}, \href{works/TranB12.pdf}{TranB12}~\cite{TranB12}, \href{}{ZarandiB12}~\cite{ZarandiB12}, \href{works/KovacsB11.pdf}{KovacsB11}~\cite{KovacsB11}, \href{works/BeckFW11.pdf}{BeckFW11}~\cite{BeckFW11}, \href{works/HeckmanB11.pdf}{HeckmanB11}~\cite{HeckmanB11}, \href{works/BajestaniB11.pdf}{BajestaniB11}~\cite{BajestaniB11}, \href{works/WuBB09.pdf}{WuBB09}~\cite{WuBB09}, \href{works/BidotVLB09.pdf}{BidotVLB09}~\cite{BidotVLB09}, \href{works/CarchraeB09.pdf}{CarchraeB09}~\cite{CarchraeB09}, \href{works/WatsonB08.pdf}{WatsonB08}~\cite{WatsonB08}, \href{works/KovacsB08.pdf}{KovacsB08}~\cite{KovacsB08}, \href{works/BeckW07.pdf}{BeckW07}~\cite{BeckW07}, \href{works/Beck07.pdf}{Beck07}~\cite{Beck07}, \href{works/KovacsB07.pdf}{KovacsB07}~\cite{KovacsB07}, \href{works/Beck06.pdf}{Beck06}~\cite{Beck06}, \href{works/CarchraeBF05.pdf}{CarchraeBF05}~\cite{CarchraeBF05}, \href{works/WuBB05.pdf}{WuBB05}~\cite{WuBB05}, \href{works/BeckW05.pdf}{BeckW05}~\cite{BeckW05}, \href{works/BeckW04.pdf}{BeckW04}~\cite{BeckW04}, \href{works/BeckR03.pdf}{BeckR03}~\cite{BeckR03}, \href{works/BeckPS03.pdf}{BeckPS03}~\cite{BeckPS03}, \href{works/BeckF00.pdf}{BeckF00}~\cite{BeckF00}, \href{works/Beck99.pdf}{Beck99}~\cite{Beck99}, \href{works/BeckF98.pdf}{BeckF98}~\cite{BeckF98}, \href{works/BeckDF97.pdf}{BeckDF97}~\cite{BeckDF97}\\
\rowlabel{auth:a144}Michela Milano & 31 &297 &\href{works/BorghesiBLMB18.pdf}{BorghesiBLMB18}~\cite{BorghesiBLMB18}, \href{works/BonfiettiZLM16.pdf}{BonfiettiZLM16}~\cite{BonfiettiZLM16}, \href{works/BridiBLMB16.pdf}{BridiBLMB16}~\cite{BridiBLMB16}, \href{works/BridiLBBM16.pdf}{BridiLBBM16}~\cite{BridiLBBM16}, \href{works/LombardiBM15.pdf}{LombardiBM15}~\cite{LombardiBM15}, \href{works/BartoliniBBLM14.pdf}{BartoliniBBLM14}~\cite{BartoliniBBLM14}, \href{works/BonfiettiLM14.pdf}{BonfiettiLM14}~\cite{BonfiettiLM14}, \href{works/BonfiettiLBM14.pdf}{BonfiettiLBM14}~\cite{BonfiettiLBM14}, \href{works/BonfiettiLM13.pdf}{BonfiettiLM13}~\cite{BonfiettiLM13}, \href{works/LombardiM13.pdf}{LombardiM13}~\cite{LombardiM13}, \href{}{LombardiMB13}~\cite{LombardiMB13}, \href{works/LombardiM12.pdf}{LombardiM12}~\cite{LombardiM12}, \href{works/BonfiettiLBM12.pdf}{BonfiettiLBM12}~\cite{BonfiettiLBM12}, \href{works/LombardiM12a.pdf}{LombardiM12a}~\cite{LombardiM12a}, \href{works/BonfiettiM12.pdf}{BonfiettiM12}~\cite{BonfiettiM12}, \href{works/BonfiettiLBM11.pdf}{BonfiettiLBM11}~\cite{BonfiettiLBM11}, \href{works/LombardiBMB11.pdf}{LombardiBMB11}~\cite{LombardiBMB11}, \href{works/BeniniLMR11.pdf}{BeniniLMR11}~\cite{BeniniLMR11}, \href{}{Milano11}~\cite{Milano11}, \href{works/LombardiM10.pdf}{LombardiM10}~\cite{LombardiM10}, \href{works/LombardiM10a.pdf}{LombardiM10a}~\cite{LombardiM10a}, \href{works/LombardiMRB10.pdf}{LombardiMRB10}~\cite{LombardiMRB10}, \href{works/LombardiM09.pdf}{LombardiM09}~\cite{LombardiM09}, \href{works/RuggieroBBMA09.pdf}{RuggieroBBMA09}~\cite{RuggieroBBMA09}, \href{works/MilanoW09.pdf}{MilanoW09}~\cite{MilanoW09}, \href{works/BeniniLMR08.pdf}{BeniniLMR08}~\cite{BeniniLMR08}, \href{works/BeniniBGM06.pdf}{BeniniBGM06}~\cite{BeniniBGM06}, \href{works/MilanoW06.pdf}{MilanoW06}~\cite{MilanoW06}, \href{}{MilanoORT02}~\cite{MilanoORT02}, \href{works/LammaMM97.pdf}{LammaMM97}~\cite{LammaMM97}, \href{works/BrusoniCLMMT96.pdf}{BrusoniCLMMT96}~\cite{BrusoniCLMMT96}\\
\rowlabel{auth:a125}Andreas Schutt & 27 &322 &\href{works/YangSS19.pdf}{YangSS19}~\cite{YangSS19}, \href{works/KreterSSZ18.pdf}{KreterSSZ18}~\cite{KreterSSZ18}, \href{works/GoldwaserS18.pdf}{GoldwaserS18}~\cite{GoldwaserS18}, \href{works/MusliuSS18.pdf}{MusliuSS18}~\cite{MusliuSS18}, \href{works/KreterSS17.pdf}{KreterSS17}~\cite{KreterSS17}, \href{works/YoungFS17.pdf}{YoungFS17}~\cite{YoungFS17}, \href{works/GoldwaserS17.pdf}{GoldwaserS17}~\cite{GoldwaserS17}, \href{works/SchuttS16.pdf}{SchuttS16}~\cite{SchuttS16}, \href{works/SzerediS16.pdf}{SzerediS16}~\cite{SzerediS16}, \href{works/KreterSS15.pdf}{KreterSS15}~\cite{KreterSS15}, \href{works/EvenSH15.pdf}{EvenSH15}~\cite{EvenSH15}, \href{works/EvenSH15a.pdf}{EvenSH15a}~\cite{EvenSH15a}, \href{}{SchuttFSW15}~\cite{SchuttFSW15}, \href{works/ThiruvadyWGS14.pdf}{ThiruvadyWGS14}~\cite{ThiruvadyWGS14}, \href{}{GuSSWC14}~\cite{GuSSWC14}, \href{works/SchuttFS13.pdf}{SchuttFS13}~\cite{SchuttFS13}, \href{works/SchuttFS13a.pdf}{SchuttFS13a}~\cite{SchuttFS13a}, \href{works/GuSS13.pdf}{GuSS13}~\cite{GuSS13}, \href{works/SchuttFSW13.pdf}{SchuttFSW13}~\cite{SchuttFSW13}, \href{works/ChuGNSW13.pdf}{ChuGNSW13}~\cite{ChuGNSW13}, \href{works/SchuttCSW12.pdf}{SchuttCSW12}~\cite{SchuttCSW12}, \href{works/SchuttFSW11.pdf}{SchuttFSW11}~\cite{SchuttFSW11}, \href{works/Schutt11.pdf}{Schutt11}~\cite{Schutt11}, \href{works/SchuttW10.pdf}{SchuttW10}~\cite{SchuttW10}, \href{works/abs-1009-0347.pdf}{abs-1009-0347}~\cite{abs-1009-0347}, \href{works/SchuttFSW09.pdf}{SchuttFSW09}~\cite{SchuttFSW09}, \href{works/SchuttWS05.pdf}{SchuttWS05}~\cite{SchuttWS05}\\
\rowlabel{auth:a143}Michele Lombardi & 25 &194 &\href{works/BorghesiBLMB18.pdf}{BorghesiBLMB18}~\cite{BorghesiBLMB18}, \href{works/CauwelaertLS18.pdf}{CauwelaertLS18}~\cite{CauwelaertLS18}, \href{works/BonfiettiZLM16.pdf}{BonfiettiZLM16}~\cite{BonfiettiZLM16}, \href{works/BridiBLMB16.pdf}{BridiBLMB16}~\cite{BridiBLMB16}, \href{works/BridiLBBM16.pdf}{BridiLBBM16}~\cite{BridiLBBM16}, \href{works/LombardiBM15.pdf}{LombardiBM15}~\cite{LombardiBM15}, \href{works/BartoliniBBLM14.pdf}{BartoliniBBLM14}~\cite{BartoliniBBLM14}, \href{works/BonfiettiLM14.pdf}{BonfiettiLM14}~\cite{BonfiettiLM14}, \href{works/BonfiettiLBM14.pdf}{BonfiettiLBM14}~\cite{BonfiettiLBM14}, \href{works/BonfiettiLM13.pdf}{BonfiettiLM13}~\cite{BonfiettiLM13}, \href{works/LombardiM13.pdf}{LombardiM13}~\cite{LombardiM13}, \href{}{LombardiMB13}~\cite{LombardiMB13}, \href{works/LombardiM12.pdf}{LombardiM12}~\cite{LombardiM12}, \href{works/BonfiettiLBM12.pdf}{BonfiettiLBM12}~\cite{BonfiettiLBM12}, \href{works/LombardiM12a.pdf}{LombardiM12a}~\cite{LombardiM12a}, \href{works/BonfiettiLBM11.pdf}{BonfiettiLBM11}~\cite{BonfiettiLBM11}, \href{works/LombardiBMB11.pdf}{LombardiBMB11}~\cite{LombardiBMB11}, \href{works/BeniniLMR11.pdf}{BeniniLMR11}~\cite{BeniniLMR11}, \href{works/LombardiM10.pdf}{LombardiM10}~\cite{LombardiM10}, \href{works/LombardiM10a.pdf}{LombardiM10a}~\cite{LombardiM10a}, \href{works/Lombardi10.pdf}{Lombardi10}~\cite{Lombardi10}, \href{works/LombardiMRB10.pdf}{LombardiMRB10}~\cite{LombardiMRB10}, \href{works/LombardiM09.pdf}{LombardiM09}~\cite{LombardiM09}, \href{works/BeniniLMR08.pdf}{BeniniLMR08}~\cite{BeniniLMR08}, \href{works/HoeveGSL07.pdf}{HoeveGSL07}~\cite{HoeveGSL07}\\
\rowlabel{auth:a126}Peter J. Stuckey & 24 &453 &\href{works/YangSS19.pdf}{YangSS19}~\cite{YangSS19}, \href{works/DemirovicS18.pdf}{DemirovicS18}~\cite{DemirovicS18}, \href{works/KreterSSZ18.pdf}{KreterSSZ18}~\cite{KreterSSZ18}, \href{works/MusliuSS18.pdf}{MusliuSS18}~\cite{MusliuSS18}, \href{works/KreterSS17.pdf}{KreterSS17}~\cite{KreterSS17}, \href{works/SchuttS16.pdf}{SchuttS16}~\cite{SchuttS16}, \href{works/BlomPS16.pdf}{BlomPS16}~\cite{BlomPS16}, \href{works/KreterSS15.pdf}{KreterSS15}~\cite{KreterSS15}, \href{works/BurtLPS15.pdf}{BurtLPS15}~\cite{BurtLPS15}, \href{}{SchuttFSW15}~\cite{SchuttFSW15}, \href{works/BlomBPS14.pdf}{BlomBPS14}~\cite{BlomBPS14}, \href{works/LipovetzkyBPS14.pdf}{LipovetzkyBPS14}~\cite{LipovetzkyBPS14}, \href{}{GuSSWC14}~\cite{GuSSWC14}, \href{works/SchuttFS13.pdf}{SchuttFS13}~\cite{SchuttFS13}, \href{works/SchuttFS13a.pdf}{SchuttFS13a}~\cite{SchuttFS13a}, \href{works/GuSS13.pdf}{GuSS13}~\cite{GuSS13}, \href{works/SchuttFSW13.pdf}{SchuttFSW13}~\cite{SchuttFSW13}, \href{works/SchuttCSW12.pdf}{SchuttCSW12}~\cite{SchuttCSW12}, \href{works/GuSW12.pdf}{GuSW12}~\cite{GuSW12}, \href{works/SchuttFSW11.pdf}{SchuttFSW11}~\cite{SchuttFSW11}, \href{works/BandaSC11.pdf}{BandaSC11}~\cite{BandaSC11}, \href{works/abs-1009-0347.pdf}{abs-1009-0347}~\cite{abs-1009-0347}, \href{works/SchuttFSW09.pdf}{SchuttFSW09}~\cite{SchuttFSW09}, \href{works/OhrimenkoSC09.pdf}{OhrimenkoSC09}~\cite{OhrimenkoSC09}\\
\rowlabel{auth:a162}John N. Hooker & 19 &1316 &\href{}{ElciOH22}~\cite{ElciOH22}, \href{works/Hooker19.pdf}{Hooker19}~\cite{Hooker19}, \href{works/Hooker17.pdf}{Hooker17}~\cite{Hooker17}, \href{works/HookerH17.pdf}{HookerH17}~\cite{HookerH17}, \href{works/HechingH16.pdf}{HechingH16}~\cite{HechingH16}, \href{}{CireCH16}~\cite{CireCH16}, \href{}{HarjunkoskiMBC14}~\cite{HarjunkoskiMBC14}, \href{works/CireCH13.pdf}{CireCH13}~\cite{CireCH13}, \href{works/CobanH11.pdf}{CobanH11}~\cite{CobanH11}, \href{works/CobanH10.pdf}{CobanH10}~\cite{CobanH10}, \href{}{Hooker10}~\cite{Hooker10}, \href{works/Hooker07.pdf}{Hooker07}~\cite{Hooker07}, \href{works/Hooker06.pdf}{Hooker06}~\cite{Hooker06}, \href{works/Hooker05.pdf}{Hooker05}~\cite{Hooker05}, \href{works/Hooker05a.pdf}{Hooker05a}~\cite{Hooker05a}, \href{works/Hooker04.pdf}{Hooker04}~\cite{Hooker04}, \href{works/HookerO03.pdf}{HookerO03}~\cite{HookerO03}, \href{works/HookerY02.pdf}{HookerY02}~\cite{HookerY02}, \href{}{Hooker00}~\cite{Hooker00}\\
\rowlabel{auth:a1}Emmanuel Hebrard & 17 &71 &\href{works/JuvinHHL23.pdf}{JuvinHHL23}~\cite{JuvinHHL23}, \href{works/HebrardALLCMR22.pdf}{HebrardALLCMR22}~\cite{HebrardALLCMR22}, \href{works/AntuoriHHEN21.pdf}{AntuoriHHEN21}~\cite{AntuoriHHEN21}, \href{}{ArtiguesHQT21}~\cite{ArtiguesHQT21}, \href{works/GodetLHS20.pdf}{GodetLHS20}~\cite{GodetLHS20}, \href{works/AntuoriHHEN20.pdf}{AntuoriHHEN20}~\cite{AntuoriHHEN20}, \href{works/HebrardHJMPV16.pdf}{HebrardHJMPV16}~\cite{HebrardHJMPV16}, \href{works/SimoninAHL15.pdf}{SimoninAHL15}~\cite{SimoninAHL15}, \href{works/SialaAH15.pdf}{SialaAH15}~\cite{SialaAH15}, \href{works/GrimesH15.pdf}{GrimesH15}~\cite{GrimesH15}, \href{works/BessiereHMQW14.pdf}{BessiereHMQW14}~\cite{BessiereHMQW14}, \href{works/SimoninAHL12.pdf}{SimoninAHL12}~\cite{SimoninAHL12}, \href{works/BillautHL12.pdf}{BillautHL12}~\cite{BillautHL12}, \href{works/GrimesH11.pdf}{GrimesH11}~\cite{GrimesH11}, \href{works/GrimesH10.pdf}{GrimesH10}~\cite{GrimesH10}, \href{works/GrimesHM09.pdf}{GrimesHM09}~\cite{GrimesHM09}, \href{works/HebrardTW05.pdf}{HebrardTW05}~\cite{HebrardTW05}\\
\rowlabel{auth:a3}Pierre Lopez & 17 &90 &\href{works/JuvinHHL23.pdf}{JuvinHHL23}~\cite{JuvinHHL23}, \href{}{JuvinHL23a}~\cite{JuvinHL23a}, \href{works/JuvinHL23.pdf}{JuvinHL23}~\cite{JuvinHL23}, \href{works/HebrardALLCMR22.pdf}{HebrardALLCMR22}~\cite{HebrardALLCMR22}, \href{works/JuvinHL22.pdf}{JuvinHL22}~\cite{JuvinHL22}, \href{works/Polo-MejiaALB20.pdf}{Polo-MejiaALB20}~\cite{Polo-MejiaALB20}, \href{works/NattafHKAL19.pdf}{NattafHKAL19}~\cite{NattafHKAL19}, \href{works/NattafAL17.pdf}{NattafAL17}~\cite{NattafAL17}, \href{works/NattafALR16.pdf}{NattafALR16}~\cite{NattafALR16}, \href{works/SimoninAHL15.pdf}{SimoninAHL15}~\cite{SimoninAHL15}, \href{works/NattafAL15.pdf}{NattafAL15}~\cite{NattafAL15}, \href{works/SimoninAHL12.pdf}{SimoninAHL12}~\cite{SimoninAHL12}, \href{works/BillautHL12.pdf}{BillautHL12}~\cite{BillautHL12}, \href{works/LahimerLH11.pdf}{LahimerLH11}~\cite{LahimerLH11}, \href{works/TrojetHL11.pdf}{TrojetHL11}~\cite{TrojetHL11}, \href{works/LopezAKYG00.pdf}{LopezAKYG00}~\cite{LopezAKYG00}, \href{works/TorresL00.pdf}{TorresL00}~\cite{TorresL00}\\
\rowlabel{auth:a6}Christian Artigues & 16 &203 &\href{works/PovedaAA23.pdf}{PovedaAA23}~\cite{PovedaAA23}, \href{works/PohlAK22.pdf}{PohlAK22}~\cite{PohlAK22}, \href{works/HebrardALLCMR22.pdf}{HebrardALLCMR22}~\cite{HebrardALLCMR22}, \href{}{ArtiguesHQT21}~\cite{ArtiguesHQT21}, \href{works/Polo-MejiaALB20.pdf}{Polo-MejiaALB20}~\cite{Polo-MejiaALB20}, \href{works/NattafHKAL19.pdf}{NattafHKAL19}~\cite{NattafHKAL19}, \href{works/NattafAL17.pdf}{NattafAL17}~\cite{NattafAL17}, \href{works/NattafALR16.pdf}{NattafALR16}~\cite{NattafALR16}, \href{works/SimoninAHL15.pdf}{SimoninAHL15}~\cite{SimoninAHL15}, \href{works/NattafAL15.pdf}{NattafAL15}~\cite{NattafAL15}, \href{works/SialaAH15.pdf}{SialaAH15}~\cite{SialaAH15}, \href{works/SimoninAHL12.pdf}{SimoninAHL12}~\cite{SimoninAHL12}, \href{}{NeronABCDD06}~\cite{NeronABCDD06}, \href{}{DemasseyAM05}~\cite{DemasseyAM05}, \href{works/ArtiguesBF04.pdf}{ArtiguesBF04}~\cite{ArtiguesBF04}, \href{works/ArtiguesR00.pdf}{ArtiguesR00}~\cite{ArtiguesR00}\\
\rowlabel{auth:a148}Pierre Schaus & 15 &79 &\href{works/CauwelaertDS20.pdf}{CauwelaertDS20}~\cite{CauwelaertDS20}, \href{works/ThomasKS20.pdf}{ThomasKS20}~\cite{ThomasKS20}, \href{works/HoundjiSW19.pdf}{HoundjiSW19}~\cite{HoundjiSW19}, \href{works/CappartTSR18.pdf}{CappartTSR18}~\cite{CappartTSR18}, \href{works/CauwelaertLS18.pdf}{CauwelaertLS18}~\cite{CauwelaertLS18}, \href{works/CappartS17.pdf}{CappartS17}~\cite{CappartS17}, \href{works/CauwelaertDMS16.pdf}{CauwelaertDMS16}~\cite{CauwelaertDMS16}, \href{works/DejemeppeCS15.pdf}{DejemeppeCS15}~\cite{DejemeppeCS15}, \href{works/GayHLS15.pdf}{GayHLS15}~\cite{GayHLS15}, \href{works/GayHS15.pdf}{GayHS15}~\cite{GayHS15}, \href{works/GayHS15a.pdf}{GayHS15a}~\cite{GayHS15a}, \href{works/HoundjiSWD14.pdf}{HoundjiSWD14}~\cite{HoundjiSWD14}, \href{works/GaySS14.pdf}{GaySS14}~\cite{GaySS14}, \href{works/SchausHMCMD11.pdf}{SchausHMCMD11}~\cite{SchausHMCMD11}, \href{works/SchausD08.pdf}{SchausD08}~\cite{SchausD08}\\
\rowlabel{auth:a17}Helmut Simonis & 15 &154 &\href{works/ArmstrongGOS22.pdf}{ArmstrongGOS22}~\cite{ArmstrongGOS22}, \href{works/ArmstrongGOS21.pdf}{ArmstrongGOS21}~\cite{ArmstrongGOS21}, \href{works/AntunesABD20.pdf}{AntunesABD20}~\cite{AntunesABD20}, \href{works/AntunesABD18.pdf}{AntunesABD18}~\cite{AntunesABD18}, \href{works/HurleyOS16.pdf}{HurleyOS16}~\cite{HurleyOS16}, \href{works/GrimesIOS14.pdf}{GrimesIOS14}~\cite{GrimesIOS14}, \href{works/IfrimOS12.pdf}{IfrimOS12}~\cite{IfrimOS12}, \href{works/SimonisH11.pdf}{SimonisH11}~\cite{SimonisH11}, \href{works/Simonis07.pdf}{Simonis07}~\cite{Simonis07}, \href{works/SimonisCK00.pdf}{SimonisCK00}~\cite{SimonisCK00}, \href{works/Simonis99.pdf}{Simonis99}~\cite{Simonis99}, \href{works/SimonisC95.pdf}{SimonisC95}~\cite{SimonisC95}, \href{works/Simonis95.pdf}{Simonis95}~\cite{Simonis95}, \href{works/Simonis95a.pdf}{Simonis95a}~\cite{Simonis95a}, \href{works/DincbasSH90.pdf}{DincbasSH90}~\cite{DincbasSH90}\\
\rowlabel{auth:a129}Nicolas Beldiceanu & 13 &274 &\href{works/Madi-WambaLOBM17.pdf}{Madi-WambaLOBM17}~\cite{Madi-WambaLOBM17}, \href{works/Madi-WambaB16.pdf}{Madi-WambaB16}~\cite{Madi-WambaB16}, \href{works/LetortCB15.pdf}{LetortCB15}~\cite{LetortCB15}, \href{works/LetortCB13.pdf}{LetortCB13}~\cite{LetortCB13}, \href{works/LetortBC12.pdf}{LetortBC12}~\cite{LetortBC12}, \href{works/ClercqPBJ11.pdf}{ClercqPBJ11}~\cite{ClercqPBJ11}, \href{works/BeldiceanuCDP11.pdf}{BeldiceanuCDP11}~\cite{BeldiceanuCDP11}, \href{works/BeldiceanuCP08.pdf}{BeldiceanuCP08}~\cite{BeldiceanuCP08}, \href{works/PoderB08.pdf}{PoderB08}~\cite{PoderB08}, \href{works/BeldiceanuP07.pdf}{BeldiceanuP07}~\cite{BeldiceanuP07}, \href{works/PoderBS04.pdf}{PoderBS04}~\cite{PoderBS04}, \href{works/BeldiceanuC02.pdf}{BeldiceanuC02}~\cite{BeldiceanuC02}, \href{works/AggounB93.pdf}{AggounB93}~\cite{AggounB93}\\
\rowlabel{auth:a248}Luca Benini & 13 &146 &\href{works/BorghesiBLMB18.pdf}{BorghesiBLMB18}~\cite{BorghesiBLMB18}, \href{works/BridiBLMB16.pdf}{BridiBLMB16}~\cite{BridiBLMB16}, \href{works/BridiLBBM16.pdf}{BridiLBBM16}~\cite{BridiLBBM16}, \href{works/BonfiettiLBM14.pdf}{BonfiettiLBM14}~\cite{BonfiettiLBM14}, \href{}{LombardiMB13}~\cite{LombardiMB13}, \href{works/BonfiettiLBM12.pdf}{BonfiettiLBM12}~\cite{BonfiettiLBM12}, \href{works/BonfiettiLBM11.pdf}{BonfiettiLBM11}~\cite{BonfiettiLBM11}, \href{works/LombardiBMB11.pdf}{LombardiBMB11}~\cite{LombardiBMB11}, \href{works/BeniniLMR11.pdf}{BeniniLMR11}~\cite{BeniniLMR11}, \href{works/LombardiMRB10.pdf}{LombardiMRB10}~\cite{LombardiMRB10}, \href{works/RuggieroBBMA09.pdf}{RuggieroBBMA09}~\cite{RuggieroBBMA09}, \href{works/BeniniLMR08.pdf}{BeniniLMR08}~\cite{BeniniLMR08}, \href{works/BeniniBGM06.pdf}{BeniniBGM06}~\cite{BeniniBGM06}\\
\rowlabel{auth:a118}Philippe Laborie & 12 &513 &\href{works/LunardiBLRV20.pdf}{LunardiBLRV20}~\cite{LunardiBLRV20}, \href{works/LaborieRSV18.pdf}{LaborieRSV18}~\cite{LaborieRSV18}, \href{works/Laborie18a.pdf}{Laborie18a}~\cite{Laborie18a}, \href{works/MelgarejoLS15.pdf}{MelgarejoLS15}~\cite{MelgarejoLS15}, \href{works/VilimLS15.pdf}{VilimLS15}~\cite{VilimLS15}, \href{works/Laborie09.pdf}{Laborie09}~\cite{Laborie09}, \href{works/BidotVLB09.pdf}{BidotVLB09}~\cite{BidotVLB09}, \href{}{BaptisteLPN06}~\cite{BaptisteLPN06}, \href{}{NeronABCDD06}~\cite{NeronABCDD06}, \href{works/GodardLN05.pdf}{GodardLN05}~\cite{GodardLN05}, \href{works/Laborie03.pdf}{Laborie03}~\cite{Laborie03}, \href{works/FocacciLN00.pdf}{FocacciLN00}~\cite{FocacciLN00}\\
\rowlabel{auth:a164}Philippe Baptiste & 11 &403 &\href{works/BaptisteB18.pdf}{BaptisteB18}~\cite{BaptisteB18}, \href{works/Baptiste09.pdf}{Baptiste09}~\cite{Baptiste09}, \href{}{BaptisteLPN06}~\cite{BaptisteLPN06}, \href{}{NeronABCDD06}~\cite{NeronABCDD06}, \href{works/ArtiouchineB05.pdf}{ArtiouchineB05}~\cite{ArtiouchineB05}, \href{works/Baptiste02.pdf}{Baptiste02}~\cite{Baptiste02}, \href{}{BaptistePN01}~\cite{BaptistePN01}, \href{works/BaptisteP00.pdf}{BaptisteP00}~\cite{BaptisteP00}, \href{works/PapaB98.pdf}{PapaB98}~\cite{PapaB98}, \href{works/BaptisteP97.pdf}{BaptisteP97}~\cite{BaptisteP97}, \href{}{PapeB97}~\cite{PapeB97}\\
\rowlabel{auth:a153}Roman Bart{\'{a}}k & 11 &88 &\href{works/SvancaraB22.pdf}{SvancaraB22}~\cite{SvancaraB22}, \href{works/JelinekB16.pdf}{JelinekB16}~\cite{JelinekB16}, \href{works/BartakV15.pdf}{BartakV15}~\cite{BartakV15}, \href{}{Bartak14}~\cite{Bartak14}, \href{works/BartakS11.pdf}{BartakS11}~\cite{BartakS11}, \href{works/BartakCS10.pdf}{BartakCS10}~\cite{BartakCS10}, \href{works/BartakSR10.pdf}{BartakSR10}~\cite{BartakSR10}, \href{works/VilimBC05.pdf}{VilimBC05}~\cite{VilimBC05}, \href{works/VilimBC04.pdf}{VilimBC04}~\cite{VilimBC04}, \href{works/Bartak02.pdf}{Bartak02}~\cite{Bartak02}, \href{works/Bartak02a.pdf}{Bartak02a}~\cite{Bartak02a}\\
\rowlabel{auth:a121}Petr Vil{\'{\i}}m & 11 &313 &\href{works/LaborieRSV18.pdf}{LaborieRSV18}~\cite{LaborieRSV18}, \href{works/VilimLS15.pdf}{VilimLS15}~\cite{VilimLS15}, \href{works/Vilim11.pdf}{Vilim11}~\cite{Vilim11}, \href{works/Vilim09.pdf}{Vilim09}~\cite{Vilim09}, \href{works/Vilim09a.pdf}{Vilim09a}~\cite{Vilim09a}, \href{works/VilimBC05.pdf}{VilimBC05}~\cite{VilimBC05}, \href{works/Vilim05.pdf}{Vilim05}~\cite{Vilim05}, \href{works/VilimBC04.pdf}{VilimBC04}~\cite{VilimBC04}, \href{works/Vilim04.pdf}{Vilim04}~\cite{Vilim04}, \href{works/Vilim03.pdf}{Vilim03}~\cite{Vilim03}, \href{works/Vilim02.pdf}{Vilim02}~\cite{Vilim02}\\
\rowlabel{auth:a117}Mark Wallace & 11 &296 &\href{works/WallaceY20.pdf}{WallaceY20}~\cite{WallaceY20}, \href{works/He0GLW18.pdf}{He0GLW18}~\cite{He0GLW18}, \href{works/ThiruvadyWGS14.pdf}{ThiruvadyWGS14}~\cite{ThiruvadyWGS14}, \href{works/SchuttFSW09.pdf}{SchuttFSW09}~\cite{SchuttFSW09}, \href{works/MilanoW09.pdf}{MilanoW09}~\cite{MilanoW09}, \href{works/MilanoW06.pdf}{MilanoW06}~\cite{MilanoW06}, \href{works/Wallace06.pdf}{Wallace06}~\cite{Wallace06}, \href{works/SakkoutW00.pdf}{SakkoutW00}~\cite{SakkoutW00}, \href{works/RodosekW98.pdf}{RodosekW98}~\cite{RodosekW98}, \href{works/Wallace96.pdf}{Wallace96}~\cite{Wallace96}, \href{}{Wallace94}~\cite{Wallace94}\\
\rowlabel{auth:a204}Alessio Bonfietti & 10 &17 &\href{works/BonfiettiZLM16.pdf}{BonfiettiZLM16}~\cite{BonfiettiZLM16}, \href{works/Bonfietti16.pdf}{Bonfietti16}~\cite{Bonfietti16}, \href{works/LombardiBM15.pdf}{LombardiBM15}~\cite{LombardiBM15}, \href{works/BonfiettiLM14.pdf}{BonfiettiLM14}~\cite{BonfiettiLM14}, \href{works/BonfiettiLBM14.pdf}{BonfiettiLBM14}~\cite{BonfiettiLBM14}, \href{works/BonfiettiLM13.pdf}{BonfiettiLM13}~\cite{BonfiettiLM13}, \href{works/BonfiettiLBM12.pdf}{BonfiettiLBM12}~\cite{BonfiettiLBM12}, \href{works/BonfiettiM12.pdf}{BonfiettiM12}~\cite{BonfiettiM12}, \href{works/BonfiettiLBM11.pdf}{BonfiettiLBM11}~\cite{BonfiettiLBM11}, \href{works/LombardiBMB11.pdf}{LombardiBMB11}~\cite{LombardiBMB11}\\
\rowlabel{auth:a149}Pascal Van Hentenryck & 10 &164 &\href{works/FontaineMH16.pdf}{FontaineMH16}~\cite{FontaineMH16}, \href{works/EvenSH15.pdf}{EvenSH15}~\cite{EvenSH15}, \href{works/EvenSH15a.pdf}{EvenSH15a}~\cite{EvenSH15a}, \href{works/SchausHMCMD11.pdf}{SchausHMCMD11}~\cite{SchausHMCMD11}, \href{works/MonetteDH09.pdf}{MonetteDH09}~\cite{MonetteDH09}, \href{works/DoomsH08.pdf}{DoomsH08}~\cite{DoomsH08}, \href{works/HentenryckM08.pdf}{HentenryckM08}~\cite{HentenryckM08}, \href{works/MercierH08.pdf}{MercierH08}~\cite{MercierH08}, \href{works/HentenryckM04.pdf}{HentenryckM04}~\cite{HentenryckM04}, \href{works/DincbasSH90.pdf}{DincbasSH90}~\cite{DincbasSH90}\\
\rowlabel{auth:a165}Claude Le Pape & 9 &536 &\href{}{BaptisteLPN06}~\cite{BaptisteLPN06}, \href{}{DannaP04}~\cite{DannaP04}, \href{}{BaptistePN01}~\cite{BaptistePN01}, \href{works/BaptisteP00.pdf}{BaptisteP00}~\cite{BaptisteP00}, \href{works/PapaB98.pdf}{PapaB98}~\cite{PapaB98}, \href{works/NuijtenP98.pdf}{NuijtenP98}~\cite{NuijtenP98}, \href{works/BaptisteP97.pdf}{BaptisteP97}~\cite{BaptisteP97}, \href{}{PapeB97}~\cite{PapeB97}, \href{}{Pape94}~\cite{Pape94}\\
\rowlabel{auth:a45}Nysret Musliu & 9 &14 &\href{works/LacknerMMWW23.pdf}{LacknerMMWW23}~\cite{LacknerMMWW23}, \href{works/WinterMMW22.pdf}{WinterMMW22}~\cite{WinterMMW22}, \href{works/LacknerMMWW21.pdf}{LacknerMMWW21}~\cite{LacknerMMWW21}, \href{works/GeibingerKKMMW21.pdf}{GeibingerKKMMW21}~\cite{GeibingerKKMMW21}, \href{works/GeibingerMM21.pdf}{GeibingerMM21}~\cite{GeibingerMM21}, \href{works/GeibingerMM19.pdf}{GeibingerMM19}~\cite{GeibingerMM19}, \href{works/abs-1911-04766.pdf}{abs-1911-04766}~\cite{abs-1911-04766}, \href{works/MusliuSS18.pdf}{MusliuSS18}~\cite{MusliuSS18}, \href{works/KletzanderM17.pdf}{KletzanderM17}~\cite{KletzanderM17}\\
\rowlabel{auth:a81}Margaux Nattaf & 9 &49 &\href{}{PenzDN23}~\cite{PenzDN23}, \href{works/NattafM20.pdf}{NattafM20}~\cite{NattafM20}, \href{works/MalapertN19.pdf}{MalapertN19}~\cite{MalapertN19}, \href{}{NattafDYW19}~\cite{NattafDYW19}, \href{works/NattafHKAL19.pdf}{NattafHKAL19}~\cite{NattafHKAL19}, \href{works/NattafAL17.pdf}{NattafAL17}~\cite{NattafAL17}, \href{works/Nattaf16.pdf}{Nattaf16}~\cite{Nattaf16}, \href{works/NattafALR16.pdf}{NattafALR16}~\cite{NattafALR16}, \href{works/NattafAL15.pdf}{NattafAL15}~\cite{NattafAL15}\\
\rowlabel{auth:a37}Claude{-}Guy Quimper & 9 &25 &\href{works/BoudreaultSLQ22.pdf}{BoudreaultSLQ22}~\cite{BoudreaultSLQ22}, \href{works/OuelletQ22.pdf}{OuelletQ22}~\cite{OuelletQ22}, \href{works/Mercier-AubinGQ20.pdf}{Mercier-AubinGQ20}~\cite{Mercier-AubinGQ20}, \href{works/FahimiOQ18.pdf}{FahimiOQ18}~\cite{FahimiOQ18}, \href{works/KameugneFGOQ18.pdf}{KameugneFGOQ18}~\cite{KameugneFGOQ18}, \href{works/OuelletQ18.pdf}{OuelletQ18}~\cite{OuelletQ18}, \href{works/GingrasQ16.pdf}{GingrasQ16}~\cite{GingrasQ16}, \href{works/BessiereHMQW14.pdf}{BessiereHMQW14}~\cite{BessiereHMQW14}, \href{works/OuelletQ13.pdf}{OuelletQ13}~\cite{OuelletQ13}\\
\rowlabel{auth:a811}Tony T. Tran & 9 &108 &\href{works/TranPZLDB18.pdf}{TranPZLDB18}~\cite{TranPZLDB18}, \href{works/TranVNB17.pdf}{TranVNB17}~\cite{TranVNB17}, \href{works/TranVNB17a.pdf}{TranVNB17a}~\cite{TranVNB17a}, \href{works/TranAB16.pdf}{TranAB16}~\cite{TranAB16}, \href{works/TranWDRFOVB16.pdf}{TranWDRFOVB16}~\cite{TranWDRFOVB16}, \href{works/TranDRFWOVB16.pdf}{TranDRFWOVB16}~\cite{TranDRFWOVB16}, \href{works/TerekhovTDB14.pdf}{TerekhovTDB14}~\cite{TerekhovTDB14}, \href{works/TranTDB13.pdf}{TranTDB13}~\cite{TranTDB13}, \href{works/TranB12.pdf}{TranB12}~\cite{TranB12}\\
\rowlabel{auth:a91}Mats Carlsson & 8 &80 &\href{works/WessenCS20.pdf}{WessenCS20}~\cite{WessenCS20}, \href{works/MossigeGSMC17.pdf}{MossigeGSMC17}~\cite{MossigeGSMC17}, \href{works/LetortCB15.pdf}{LetortCB15}~\cite{LetortCB15}, \href{works/LetortCB13.pdf}{LetortCB13}~\cite{LetortCB13}, \href{works/LetortBC12.pdf}{LetortBC12}~\cite{LetortBC12}, \href{works/BeldiceanuCDP11.pdf}{BeldiceanuCDP11}~\cite{BeldiceanuCDP11}, \href{works/BeldiceanuCP08.pdf}{BeldiceanuCP08}~\cite{BeldiceanuCP08}, \href{works/BeldiceanuC02.pdf}{BeldiceanuC02}~\cite{BeldiceanuC02}\\
\rowlabel{auth:a155}Thibaut Feydy & 8 &173 &\href{works/YoungFS17.pdf}{YoungFS17}~\cite{YoungFS17}, \href{}{SchuttFSW15}~\cite{SchuttFSW15}, \href{works/SchuttFS13.pdf}{SchuttFS13}~\cite{SchuttFS13}, \href{works/SchuttFS13a.pdf}{SchuttFS13a}~\cite{SchuttFS13a}, \href{works/SchuttFSW13.pdf}{SchuttFSW13}~\cite{SchuttFSW13}, \href{works/SchuttFSW11.pdf}{SchuttFSW11}~\cite{SchuttFSW11}, \href{works/abs-1009-0347.pdf}{abs-1009-0347}~\cite{abs-1009-0347}, \href{works/SchuttFSW09.pdf}{SchuttFSW09}~\cite{SchuttFSW09}\\
\rowlabel{auth:a156}Mark G. Wallace & 8 &135 &\href{}{SchuttFSW15}~\cite{SchuttFSW15}, \href{}{GuSSWC14}~\cite{GuSSWC14}, \href{works/SchuttFSW13.pdf}{SchuttFSW13}~\cite{SchuttFSW13}, \href{works/SchuttCSW12.pdf}{SchuttCSW12}~\cite{SchuttCSW12}, \href{works/GuSW12.pdf}{GuSW12}~\cite{GuSW12}, \href{works/SchuttFSW11.pdf}{SchuttFSW11}~\cite{SchuttFSW11}, \href{works/abs-1009-0347.pdf}{abs-1009-0347}~\cite{abs-1009-0347}, \href{}{AjiliW04}~\cite{AjiliW04}\\
\rowlabel{auth:a332}Louis{-}Martin Rousseau & 8 &126 &\href{works/CappartTSR18.pdf}{CappartTSR18}~\cite{CappartTSR18}, \href{works/DoulabiRP16.pdf}{DoulabiRP16}~\cite{DoulabiRP16}, \href{works/PesantRR15.pdf}{PesantRR15}~\cite{PesantRR15}, \href{works/DoulabiRP14.pdf}{DoulabiRP14}~\cite{DoulabiRP14}, \href{works/MalapertCGJLR13.pdf}{MalapertCGJLR13}~\cite{MalapertCGJLR13}, \href{}{MalapertCGJLR12}~\cite{MalapertCGJLR12}, \href{works/ChapadosJR11.pdf}{ChapadosJR11}~\cite{ChapadosJR11}, \href{works/HachemiGR11.pdf}{HachemiGR11}~\cite{HachemiGR11}\\
\rowlabel{auth:a51}Armin Wolf & 8 &46 &\href{works/GeitzGSSW22.pdf}{GeitzGSSW22}~\cite{GeitzGSSW22}, \href{works/Wolf11.pdf}{Wolf11}~\cite{Wolf11}, \href{works/SchuttW10.pdf}{SchuttW10}~\cite{SchuttW10}, \href{works/Wolf09.pdf}{Wolf09}~\cite{Wolf09}, \href{works/WolfS05.pdf}{WolfS05}~\cite{WolfS05}, \href{works/SchuttWS05.pdf}{SchuttWS05}~\cite{SchuttWS05}, \href{works/Wolf05.pdf}{Wolf05}~\cite{Wolf05}, \href{works/Wolf03.pdf}{Wolf03}~\cite{Wolf03}\\
\rowlabel{auth:a183}Diarmuid Grimes & 7 &52 &\href{works/AntunesABD20.pdf}{AntunesABD20}~\cite{AntunesABD20}, \href{works/AntunesABD18.pdf}{AntunesABD18}~\cite{AntunesABD18}, \href{works/GrimesH15.pdf}{GrimesH15}~\cite{GrimesH15}, \href{works/GrimesIOS14.pdf}{GrimesIOS14}~\cite{GrimesIOS14}, \href{works/GrimesH11.pdf}{GrimesH11}~\cite{GrimesH11}, \href{works/GrimesH10.pdf}{GrimesH10}~\cite{GrimesH10}, \href{works/GrimesHM09.pdf}{GrimesHM09}~\cite{GrimesHM09}\\
\rowlabel{auth:a116}Zdenek Hanz{\'{a}}lek & 7 &27 &\href{works/Mehdizadeh-Somarin23.pdf}{Mehdizadeh-Somarin23}~\cite{Mehdizadeh-Somarin23}, \href{works/abs-2305-19888.pdf}{abs-2305-19888}~\cite{abs-2305-19888}, \href{works/HeinzNVH22.pdf}{HeinzNVH22}~\cite{HeinzNVH22}, \href{works/VlkHT21.pdf}{VlkHT21}~\cite{VlkHT21}, \href{works/BenediktMH20.pdf}{BenediktMH20}~\cite{BenediktMH20}, \href{works/BenediktSMVH18.pdf}{BenediktSMVH18}~\cite{BenediktSMVH18}, \href{works/KelbelH11.pdf}{KelbelH11}~\cite{KelbelH11}\\
\rowlabel{auth:a10}Roger Kameugne & 7 &14 &\href{works/KameugneFND23.pdf}{KameugneFND23}~\cite{KameugneFND23}, \href{works/ThomasKS20.pdf}{ThomasKS20}~\cite{ThomasKS20}, \href{works/KameugneFGOQ18.pdf}{KameugneFGOQ18}~\cite{KameugneFGOQ18}, \href{works/Kameugne15.pdf}{Kameugne15}~\cite{Kameugne15}, \href{works/KameugneFSN14.pdf}{KameugneFSN14}~\cite{KameugneFSN14}, \href{works/Kameugne14.pdf}{Kameugne14}~\cite{Kameugne14}, \href{works/KameugneFSN11.pdf}{KameugneFSN11}~\cite{KameugneFSN11}\\
\rowlabel{auth:a147}Andr{\'{a}}s Kov{\'{a}}cs & 7 &21 &\href{works/KovacsB11.pdf}{KovacsB11}~\cite{KovacsB11}, \href{works/KovacsK11.pdf}{KovacsK11}~\cite{KovacsK11}, \href{works/KovacsB08.pdf}{KovacsB08}~\cite{KovacsB08}, \href{works/KovacsB07.pdf}{KovacsB07}~\cite{KovacsB07}, \href{works/KovacsV06.pdf}{KovacsV06}~\cite{KovacsV06}, \href{works/KovacsEKV05.pdf}{KovacsEKV05}~\cite{KovacsEKV05}, \href{works/KovacsV04.pdf}{KovacsV04}~\cite{KovacsV04}\\
\rowlabel{auth:a16}Barry O'Sullivan & 7 &14 &\href{works/ArmstrongGOS22.pdf}{ArmstrongGOS22}~\cite{ArmstrongGOS22}, \href{works/ArmstrongGOS21.pdf}{ArmstrongGOS21}~\cite{ArmstrongGOS21}, \href{works/AntunesABD20.pdf}{AntunesABD20}~\cite{AntunesABD20}, \href{works/AntunesABD18.pdf}{AntunesABD18}~\cite{AntunesABD18}, \href{works/HurleyOS16.pdf}{HurleyOS16}~\cite{HurleyOS16}, \href{works/GrimesIOS14.pdf}{GrimesIOS14}~\cite{GrimesIOS14}, \href{works/IfrimOS12.pdf}{IfrimOS12}~\cite{IfrimOS12}\\
\rowlabel{auth:a598}Gabriela P. Henning & 7 &153 &\href{works/NovaraNH16.pdf}{NovaraNH16}~\cite{NovaraNH16}, \href{works/NovasH14.pdf}{NovasH14}~\cite{NovasH14}, \href{works/NovasH12.pdf}{NovasH12}~\cite{NovasH12}, \href{works/NovasH10.pdf}{NovasH10}~\cite{NovasH10}, \href{works/ZeballosQH10.pdf}{ZeballosQH10}~\cite{ZeballosQH10}, \href{works/ZeballosH05.pdf}{ZeballosH05}~\cite{ZeballosH05}, \href{works/QuirogaZH05.pdf}{QuirogaZH05}~\cite{QuirogaZH05}\\
\rowlabel{auth:a152}Yves Deville & 6 &19 &\href{works/HoundjiSWD14.pdf}{HoundjiSWD14}~\cite{HoundjiSWD14}, \href{works/DejemeppeD14.pdf}{DejemeppeD14}~\cite{DejemeppeD14}, \href{works/SchausHMCMD11.pdf}{SchausHMCMD11}~\cite{SchausHMCMD11}, \href{works/MonetteDH09.pdf}{MonetteDH09}~\cite{MonetteDH09}, \href{works/SchausD08.pdf}{SchausD08}~\cite{SchausD08}, \href{works/MonetteDD07.pdf}{MonetteDD07}~\cite{MonetteDD07}\\
\rowlabel{auth:a134}Stefan Heinz & 6 &67 &\href{works/HeinzSB13.pdf}{HeinzSB13}~\cite{HeinzSB13}, \href{works/HeinzKB13.pdf}{HeinzKB13}~\cite{HeinzKB13}, \href{works/HeinzSSW12.pdf}{HeinzSSW12}~\cite{HeinzSSW12}, \href{works/HeinzB12.pdf}{HeinzB12}~\cite{HeinzB12}, \href{works/HeinzS11.pdf}{HeinzS11}~\cite{HeinzS11}, \href{works/BertholdHLMS10.pdf}{BertholdHLMS10}~\cite{BertholdHLMS10}\\
\rowlabel{auth:a82}Arnaud Malapert & 6 &39 &\href{works/NattafM20.pdf}{NattafM20}~\cite{NattafM20}, \href{works/MalapertN19.pdf}{MalapertN19}~\cite{MalapertN19}, \href{works/MalapertCGJLR13.pdf}{MalapertCGJLR13}~\cite{MalapertCGJLR13}, \href{}{MalapertCGJLR12}~\cite{MalapertCGJLR12}, \href{works/Malapert11.pdf}{Malapert11}~\cite{Malapert11}, \href{works/GrimesHM09.pdf}{GrimesHM09}~\cite{GrimesHM09}\\
\rowlabel{auth:a666}Wim Nuijten & 6 &375 &\href{}{BaptisteLPN06}~\cite{BaptisteLPN06}, \href{works/GodardLN05.pdf}{GodardLN05}~\cite{GodardLN05}, \href{}{BaptistePN01}~\cite{BaptistePN01}, \href{works/SourdN00.pdf}{SourdN00}~\cite{SourdN00}, \href{works/FocacciLN00.pdf}{FocacciLN00}~\cite{FocacciLN00}, \href{works/NuijtenP98.pdf}{NuijtenP98}~\cite{NuijtenP98}\\
\rowlabel{auth:a445}Erwin Pesch & 6 &417 &\href{works/MullerMKP22.pdf}{MullerMKP22}~\cite{MullerMKP22}, \href{}{BlazewiczEP19}~\cite{BlazewiczEP19}, \href{}{DomdorfPH03}~\cite{DomdorfPH03}, \href{}{DorndorfPH99}~\cite{DorndorfPH99}, \href{}{DorndorfHP99}~\cite{DorndorfHP99}, \href{}{BlazewiczDP96}~\cite{BlazewiczDP96}\\
\rowlabel{auth:a364}Emmanuel Poder & 6 &27 &\href{works/BeldiceanuCDP11.pdf}{BeldiceanuCDP11}~\cite{BeldiceanuCDP11}, \href{works/abs-0907-0939.pdf}{abs-0907-0939}~\cite{abs-0907-0939}, \href{works/BeldiceanuCP08.pdf}{BeldiceanuCP08}~\cite{BeldiceanuCP08}, \href{works/PoderB08.pdf}{PoderB08}~\cite{PoderB08}, \href{works/BeldiceanuP07.pdf}{BeldiceanuP07}~\cite{BeldiceanuP07}, \href{works/PoderBS04.pdf}{PoderBS04}~\cite{PoderBS04}\\
\rowlabel{auth:a737}Vahid Roshanaei & 6 &168 &\href{works/NaderiRR23.pdf}{NaderiRR23}~\cite{NaderiRR23}, \href{}{NaderiR22}~\cite{NaderiR22}, \href{}{NaderiRBAU21}~\cite{NaderiRBAU21}, \href{}{RoshanaeiBAUB20}~\cite{RoshanaeiBAUB20}, \href{}{RoshanaeiLAU17}~\cite{RoshanaeiLAU17}, \href{}{RoshanaeiLAU17a}~\cite{RoshanaeiLAU17a}\\
\rowlabel{auth:a208}Cyrille Dejemeppe & 5 &8 &\href{works/CauwelaertDS20.pdf}{CauwelaertDS20}~\cite{CauwelaertDS20}, \href{works/CauwelaertDMS16.pdf}{CauwelaertDMS16}~\cite{CauwelaertDMS16}, \href{works/Dejemeppe16.pdf}{Dejemeppe16}~\cite{Dejemeppe16}, \href{works/DejemeppeCS15.pdf}{DejemeppeCS15}~\cite{DejemeppeCS15}, \href{works/DejemeppeD14.pdf}{DejemeppeD14}~\cite{DejemeppeD14}\\
\rowlabel{auth:a246}Sophie Demassey & 5 &82 &\href{works/HermenierDL11.pdf}{HermenierDL11}~\cite{HermenierDL11}, \href{works/BeldiceanuCDP11.pdf}{BeldiceanuCDP11}~\cite{BeldiceanuCDP11}, \href{}{NeronABCDD06}~\cite{NeronABCDD06}, \href{}{DemasseyAM05}~\cite{DemasseyAM05}, \href{works/Demassey03.pdf}{Demassey03}~\cite{Demassey03}\\
\rowlabel{auth:a388}Ignacio E. Grossmann & 5 &844 &\href{}{HarjunkoskiMBC14}~\cite{HarjunkoskiMBC14}, \href{}{CastroGR10}~\cite{CastroGR10}, \href{works/MaraveliasG04.pdf}{MaraveliasG04}~\cite{MaraveliasG04}, \href{works/HarjunkoskiG02.pdf}{HarjunkoskiG02}~\cite{HarjunkoskiG02}, \href{works/JainG01.pdf}{JainG01}~\cite{JainG01}\\
\rowlabel{auth:a342}Hanyu Gu & 5 &39 &\href{works/EtminaniesfahaniGNMS22.pdf}{EtminaniesfahaniGNMS22}~\cite{EtminaniesfahaniGNMS22}, \href{works/ThiruvadyWGS14.pdf}{ThiruvadyWGS14}~\cite{ThiruvadyWGS14}, \href{}{GuSSWC14}~\cite{GuSSWC14}, \href{works/GuSS13.pdf}{GuSS13}~\cite{GuSS13}, \href{works/GuSW12.pdf}{GuSW12}~\cite{GuSW12}\\
\rowlabel{auth:a250}Narendra Jussien & 5 &36 &\href{works/MalapertCGJLR13.pdf}{MalapertCGJLR13}~\cite{MalapertCGJLR13}, \href{}{MalapertCGJLR12}~\cite{MalapertCGJLR12}, \href{works/ClercqPBJ11.pdf}{ClercqPBJ11}~\cite{ClercqPBJ11}, \href{works/ElkhyariGJ02.pdf}{ElkhyariGJ02}~\cite{ElkhyariGJ02}, \href{works/ElkhyariGJ02a.pdf}{ElkhyariGJ02a}~\cite{ElkhyariGJ02a}\\
\rowlabel{auth:a531}Juan M. Novas & 5 &148 &\href{works/Novas19.pdf}{Novas19}~\cite{Novas19}, \href{works/NovaraNH16.pdf}{NovaraNH16}~\cite{NovaraNH16}, \href{works/NovasH14.pdf}{NovasH14}~\cite{NovasH14}, \href{works/NovasH12.pdf}{NovasH12}~\cite{NovasH12}, \href{works/NovasH10.pdf}{NovasH10}~\cite{NovasH10}\\
\rowlabel{auth:a223}Kenneth N. Brown & 5 &44 &\href{works/AntunesABD20.pdf}{AntunesABD20}~\cite{AntunesABD20}, \href{works/AntunesABD18.pdf}{AntunesABD18}~\cite{AntunesABD18}, \href{works/MurphyMB15.pdf}{MurphyMB15}~\cite{MurphyMB15}, \href{works/WuBB09.pdf}{WuBB09}~\cite{WuBB09}, \href{works/WuBB05.pdf}{WuBB05}~\cite{WuBB05}\\
\rowlabel{auth:a735}Bahman Naderi & 5 &32 &\href{works/NaderiRR23.pdf}{NaderiRR23}~\cite{NaderiRR23}, \href{works/NaderiBZ22.pdf}{NaderiBZ22}~\cite{NaderiBZ22}, \href{}{NaderiBZ22a}~\cite{NaderiBZ22a}, \href{}{NaderiR22}~\cite{NaderiR22}, \href{}{NaderiRBAU21}~\cite{NaderiRBAU21}\\
\rowlabel{auth:a130}Mohamed Siala & 5 &9 &\href{works/AntunesABD20.pdf}{AntunesABD20}~\cite{AntunesABD20}, \href{works/AntunesABD18.pdf}{AntunesABD18}~\cite{AntunesABD18}, \href{works/Siala15.pdf}{Siala15}~\cite{Siala15}, \href{works/SialaAH15.pdf}{SialaAH15}~\cite{SialaAH15}, \href{works/Siala15a.pdf}{Siala15a}~\cite{Siala15a}\\
\rowlabel{auth:a314}Marek Vlk & 5 &14 &\href{works/abs-2305-19888.pdf}{abs-2305-19888}~\cite{abs-2305-19888}, \href{works/HeinzNVH22.pdf}{HeinzNVH22}~\cite{HeinzNVH22}, \href{works/VlkHT21.pdf}{VlkHT21}~\cite{VlkHT21}, \href{works/BenediktSMVH18.pdf}{BenediktSMVH18}~\cite{BenediktSMVH18}, \href{works/BartakV15.pdf}{BartakV15}~\cite{BartakV15}\\
\rowlabel{auth:a838}Nic Wilson & 5 &28 &\href{works/AntunesABD20.pdf}{AntunesABD20}~\cite{AntunesABD20}, \href{works/AntunesABD18.pdf}{AntunesABD18}~\cite{AntunesABD18}, \href{works/BeckW07.pdf}{BeckW07}~\cite{BeckW07}, \href{works/BeckW05.pdf}{BeckW05}~\cite{BeckW05}, \href{works/BeckW04.pdf}{BeckW04}~\cite{BeckW04}\\
\rowlabel{auth:a159}Andr{\'{e}} A. Cir{\'{e}} & 4 &50 &\href{works/CireCH13.pdf}{CireCH13}~\cite{CireCH13}, \href{works/LopesCSM10.pdf}{LopesCSM10}~\cite{LopesCSM10}, \href{works/MouraSCL08.pdf}{MouraSCL08}~\cite{MouraSCL08}, \href{works/MouraSCL08a.pdf}{MouraSCL08a}~\cite{MouraSCL08a}\\
\rowlabel{auth:a231}Andrea Bartolini & 4 &40 &\href{works/BorghesiBLMB18.pdf}{BorghesiBLMB18}~\cite{BorghesiBLMB18}, \href{works/BridiBLMB16.pdf}{BridiBLMB16}~\cite{BridiBLMB16}, \href{works/BridiLBBM16.pdf}{BridiLBBM16}~\cite{BridiLBBM16}, \href{works/BartoliniBBLM14.pdf}{BartoliniBBLM14}~\cite{BartoliniBBLM14}\\
\rowlabel{auth:a349}Geoffrey Chu & 4 &47 &\href{}{GuSSWC14}~\cite{GuSSWC14}, \href{works/ChuGNSW13.pdf}{ChuGNSW13}~\cite{ChuGNSW13}, \href{works/SchuttCSW12.pdf}{SchuttCSW12}~\cite{SchuttCSW12}, \href{works/BandaSC11.pdf}{BandaSC11}~\cite{BandaSC11}\\
\rowlabel{auth:a341}Elvin Coban & 4 &41 &\href{}{CireCH16}~\cite{CireCH16}, \href{works/CireCH13.pdf}{CireCH13}~\cite{CireCH13}, \href{works/CobanH11.pdf}{CobanH11}~\cite{CobanH11}, \href{works/CobanH10.pdf}{CobanH10}~\cite{CobanH10}\\
\rowlabel{auth:a217}Steven Gay & 4 &42 &\href{works/GayHLS15.pdf}{GayHLS15}~\cite{GayHLS15}, \href{works/GayHS15.pdf}{GayHS15}~\cite{GayHS15}, \href{works/GayHS15a.pdf}{GayHS15a}~\cite{GayHS15a}, \href{works/GaySS14.pdf}{GaySS14}~\cite{GaySS14}\\
\rowlabel{auth:a77}Tobias Geibinger & 4 &6 &\href{works/GeibingerKKMMW21.pdf}{GeibingerKKMMW21}~\cite{GeibingerKKMMW21}, \href{works/GeibingerMM21.pdf}{GeibingerMM21}~\cite{GeibingerMM21}, \href{works/GeibingerMM19.pdf}{GeibingerMM19}~\cite{GeibingerMM19}, \href{works/abs-1911-04766.pdf}{abs-1911-04766}~\cite{abs-1911-04766}\\
\rowlabel{auth:a296}Christelle Gu{\'{e}}ret & 4 &33 &\href{works/MalapertCGJLR13.pdf}{MalapertCGJLR13}~\cite{MalapertCGJLR13}, \href{}{MalapertCGJLR12}~\cite{MalapertCGJLR12}, \href{works/ElkhyariGJ02.pdf}{ElkhyariGJ02}~\cite{ElkhyariGJ02}, \href{works/ElkhyariGJ02a.pdf}{ElkhyariGJ02a}~\cite{ElkhyariGJ02a}\\
\rowlabel{auth:a2}Laurent Houssin & 4 &0 &\href{works/JuvinHHL23.pdf}{JuvinHHL23}~\cite{JuvinHHL23}, \href{}{JuvinHL23a}~\cite{JuvinHL23a}, \href{works/JuvinHL23.pdf}{JuvinHL23}~\cite{JuvinHL23}, \href{works/JuvinHL22.pdf}{JuvinHL22}~\cite{JuvinHL22}\\
\rowlabel{auth:a0}Carla Juvin & 4 &0 &\href{works/JuvinHHL23.pdf}{JuvinHHL23}~\cite{JuvinHHL23}, \href{}{JuvinHL23a}~\cite{JuvinHL23a}, \href{works/JuvinHL23.pdf}{JuvinHL23}~\cite{JuvinHL23}, \href{works/JuvinHL22.pdf}{JuvinHL22}~\cite{JuvinHL22}\\
\rowlabel{auth:a157}Tam{\'{a}}s Kis & 4 &11 &\href{works/NattafHKAL19.pdf}{NattafHKAL19}~\cite{NattafHKAL19}, \href{works/KovacsK11.pdf}{KovacsK11}~\cite{KovacsK11}, \href{works/KeriK07.pdf}{KeriK07}~\cite{KeriK07}, \href{works/KovacsEKV05.pdf}{KovacsEKV05}~\cite{KovacsEKV05}\\
\rowlabel{auth:a128}Arnaud Letort & 4 &23 &\href{works/LetortCB15.pdf}{LetortCB15}~\cite{LetortCB15}, \href{works/LetortCB13.pdf}{LetortCB13}~\cite{LetortCB13}, \href{works/Letort13.pdf}{Letort13}~\cite{Letort13}, \href{works/LetortBC12.pdf}{LetortBC12}~\cite{LetortBC12}\\
\rowlabel{auth:a913}Dionne M. Aleman & 4 &161 &\href{}{NaderiRBAU21}~\cite{NaderiRBAU21}, \href{}{RoshanaeiBAUB20}~\cite{RoshanaeiBAUB20}, \href{}{RoshanaeiLAU17}~\cite{RoshanaeiLAU17}, \href{}{RoshanaeiLAU17a}~\cite{RoshanaeiLAU17a}\\
\rowlabel{auth:a32}Laurent Michel & 4 &39 &\href{works/TardivoDFMP23.pdf}{TardivoDFMP23}~\cite{TardivoDFMP23}, \href{works/SchausHMCMD11.pdf}{SchausHMCMD11}~\cite{SchausHMCMD11}, \href{works/HentenryckM08.pdf}{HentenryckM08}~\cite{HentenryckM08}, \href{works/HentenryckM04.pdf}{HentenryckM04}~\cite{HentenryckM04}\\
\rowlabel{auth:a80}Florian Mischek & 4 &6 &\href{works/GeibingerKKMMW21.pdf}{GeibingerKKMMW21}~\cite{GeibingerKKMMW21}, \href{works/GeibingerMM21.pdf}{GeibingerMM21}~\cite{GeibingerMM21}, \href{works/GeibingerMM19.pdf}{GeibingerMM19}~\cite{GeibingerMM19}, \href{works/abs-1911-04766.pdf}{abs-1911-04766}~\cite{abs-1911-04766}\\
\rowlabel{auth:a150}Jean{-}No{\"{e}}l Monette & 4 &15 &\href{works/CauwelaertDMS16.pdf}{CauwelaertDMS16}~\cite{CauwelaertDMS16}, \href{works/SchausHMCMD11.pdf}{SchausHMCMD11}~\cite{SchausHMCMD11}, \href{works/MonetteDH09.pdf}{MonetteDH09}~\cite{MonetteDH09}, \href{works/MonetteDD07.pdf}{MonetteDD07}~\cite{MonetteDD07}\\
\rowlabel{auth:a210}Goldie Nejat & 4 &50 &\href{works/TranVNB17.pdf}{TranVNB17}~\cite{TranVNB17}, \href{works/TranVNB17a.pdf}{TranVNB17a}~\cite{TranVNB17a}, \href{works/BoothNB16.pdf}{BoothNB16}~\cite{BoothNB16}, \href{works/LouieVNB14.pdf}{LouieVNB14}~\cite{LouieVNB14}\\
\rowlabel{auth:a52}Yanick Ouellet & 4 &10 &\href{works/OuelletQ22.pdf}{OuelletQ22}~\cite{OuelletQ22}, \href{works/FahimiOQ18.pdf}{FahimiOQ18}~\cite{FahimiOQ18}, \href{works/KameugneFGOQ18.pdf}{KameugneFGOQ18}~\cite{KameugneFGOQ18}, \href{works/OuelletQ18.pdf}{OuelletQ18}~\cite{OuelletQ18}\\
\rowlabel{auth:a8}Gilles Pesant & 4 &60 &\href{works/AalianPG23.pdf}{AalianPG23}~\cite{AalianPG23}, \href{works/DoulabiRP16.pdf}{DoulabiRP16}~\cite{DoulabiRP16}, \href{works/PesantRR15.pdf}{PesantRR15}~\cite{PesantRR15}, \href{works/DoulabiRP14.pdf}{DoulabiRP14}~\cite{DoulabiRP14}\\
\rowlabel{auth:a227}Thierry Petit & 4 &20 &\href{works/DerrienP14.pdf}{DerrienP14}~\cite{DerrienP14}, \href{works/DerrienPZ14.pdf}{DerrienPZ14}~\cite{DerrienPZ14}, \href{works/ClercqPBJ11.pdf}{ClercqPBJ11}~\cite{ClercqPBJ11}, \href{works/abs-0907-0939.pdf}{abs-0907-0939}~\cite{abs-0907-0939}\\
\rowlabel{auth:a21}C{\'{e}}dric Pralet & 4 &10 &\href{works/SquillaciPR23.pdf}{SquillaciPR23}~\cite{SquillaciPR23}, \href{works/Pralet17.pdf}{Pralet17}~\cite{Pralet17}, \href{works/HebrardHJMPV16.pdf}{HebrardHJMPV16}~\cite{HebrardHJMPV16}, \href{works/PraletLJ15.pdf}{PraletLJ15}~\cite{PraletLJ15}\\
\rowlabel{auth:a328}Adrian R. Pearce & 4 &35 &\href{works/BlomPS16.pdf}{BlomPS16}~\cite{BlomPS16}, \href{works/BurtLPS15.pdf}{BurtLPS15}~\cite{BurtLPS15}, \href{works/BlomBPS14.pdf}{BlomBPS14}~\cite{BlomBPS14}, \href{works/LipovetzkyBPS14.pdf}{LipovetzkyBPS14}~\cite{LipovetzkyBPS14}\\
\rowlabel{auth:a402}Dhananjay R. Thiruvady & 4 &32 &\href{works/abs-2402-00459.pdf}{abs-2402-00459}~\cite{abs-2402-00459}, \href{works/abs-2211-14492.pdf}{abs-2211-14492}~\cite{abs-2211-14492}, \href{works/ThiruvadyWGS14.pdf}{ThiruvadyWGS14}~\cite{ThiruvadyWGS14}, \href{works/ThiruvadyBME09.pdf}{ThiruvadyBME09}~\cite{ThiruvadyBME09}\\
\rowlabel{auth:a727}Martino Ruggiero & 4 &58 &\href{works/BeniniLMR11.pdf}{BeniniLMR11}~\cite{BeniniLMR11}, \href{works/LombardiMRB10.pdf}{LombardiMRB10}~\cite{LombardiMRB10}, \href{works/RuggieroBBMA09.pdf}{RuggieroBBMA09}~\cite{RuggieroBBMA09}, \href{works/BeniniLMR08.pdf}{BeniniLMR08}~\cite{BeniniLMR08}\\
\rowlabel{auth:a85}Christine Solnon & 4 &20 &\href{works/GroleazNS20.pdf}{GroleazNS20}~\cite{GroleazNS20}, \href{works/GroleazNS20a.pdf}{GroleazNS20a}~\cite{GroleazNS20a}, \href{works/SacramentoSP20.pdf}{SacramentoSP20}~\cite{SacramentoSP20}, \href{works/MelgarejoLS15.pdf}{MelgarejoLS15}~\cite{MelgarejoLS15}\\
\rowlabel{auth:a830}Daria Terekhov & 4 &21 &\href{works/TanT18.pdf}{TanT18}~\cite{TanT18}, \href{works/TerekhovTDB14.pdf}{TerekhovTDB14}~\cite{TerekhovTDB14}, \href{works/TranTDB13.pdf}{TranTDB13}~\cite{TranTDB13}, \href{works/TerekhovDOB12.pdf}{TerekhovDOB12}~\cite{TerekhovDOB12}\\
\rowlabel{auth:a281}J{\'{o}}zsef V{\'{a}}ncza & 4 &9 &\href{works/KovacsV06.pdf}{KovacsV06}~\cite{KovacsV06}, \href{works/KovacsEKV05.pdf}{KovacsEKV05}~\cite{KovacsEKV05}, \href{works/KovacsV04.pdf}{KovacsV04}~\cite{KovacsV04}, \href{works/VanczaM01.pdf}{VanczaM01}~\cite{VanczaM01}\\
\rowlabel{auth:a279}Toby Walsh & 4 &2 &\href{works/GelainPRVW17.pdf}{GelainPRVW17}~\cite{GelainPRVW17}, \href{works/BessiereHMQW14.pdf}{BessiereHMQW14}~\cite{BessiereHMQW14}, \href{works/ChuGNSW13.pdf}{ChuGNSW13}~\cite{ChuGNSW13}, \href{works/HebrardTW05.pdf}{HebrardTW05}~\cite{HebrardTW05}\\
\rowlabel{auth:a43}Felix Winter & 4 &0 &\href{works/LacknerMMWW23.pdf}{LacknerMMWW23}~\cite{LacknerMMWW23}, \href{works/WinterMMW22.pdf}{WinterMMW22}~\cite{WinterMMW22}, \href{works/LacknerMMWW21.pdf}{LacknerMMWW21}~\cite{LacknerMMWW21}, \href{works/GeibingerKKMMW21.pdf}{GeibingerKKMMW21}~\cite{GeibingerKKMMW21}\\
\rowlabel{auth:a411}Francisco Yuraszeck & 4 &31 &\href{works/YuraszeckMCCR23.pdf}{YuraszeckMCCR23}~\cite{YuraszeckMCCR23}, \href{works/YuraszeckMC23.pdf}{YuraszeckMC23}~\cite{YuraszeckMC23}, \href{works/YuraszeckMPV22.pdf}{YuraszeckMPV22}~\cite{YuraszeckMPV22}, \href{works/MejiaY20.pdf}{MejiaY20}~\cite{MejiaY20}\\
\rowlabel{auth:a212}Willem{-}Jan van Hoeve & 4 &50 &\href{works/GilesH16.pdf}{GilesH16}~\cite{GilesH16}, \href{works/GoelSHFS15.pdf}{GoelSHFS15}~\cite{GoelSHFS15}, \href{works/HoeveGSL07.pdf}{HoeveGSL07}~\cite{HoeveGSL07}, \href{works/GomesHS06.pdf}{GomesHS06}~\cite{GomesHS06}\\
\rowlabel{auth:a74}Max {\AA}strand & 4 &27 &\href{works/Astrand0F21.pdf}{Astrand0F21}~\cite{Astrand0F21}, \href{works/Astrand21.pdf}{Astrand21}~\cite{Astrand21}, \href{works/AstrandJZ20.pdf}{AstrandJZ20}~\cite{AstrandJZ20}, \href{works/AstrandJZ18.pdf}{AstrandJZ18}~\cite{AstrandJZ18}\\
\rowlabel{auth:a154}Miguel A. Salido & 3 &45 &\href{works/BartakS11.pdf}{BartakS11}~\cite{BartakS11}, \href{works/BartakSR10.pdf}{BartakSR10}~\cite{BartakSR10}, \href{works/AbrilSB05.pdf}{AbrilSB05}~\cite{AbrilSB05}\\
\rowlabel{auth:a230}Laurence A. Wolsey & 3 &50 &\href{works/HoundjiSW19.pdf}{HoundjiSW19}~\cite{HoundjiSW19}, \href{works/HoundjiSWD14.pdf}{HoundjiSWD14}~\cite{HoundjiSWD14}, \href{works/SadykovW06.pdf}{SadykovW06}~\cite{SadykovW06}\\
\rowlabel{auth:a391}Bruno A. Prata & 3 &1 &\href{works/PrataAN23.pdf}{PrataAN23}~\cite{PrataAN23}, \href{works/AbreuNP23.pdf}{AbreuNP23}~\cite{AbreuNP23}, \href{}{AbreuPNF23}~\cite{AbreuPNF23}\\
\rowlabel{auth:a849}Mehmet A. Begen & 3 &25 &\href{works/NaderiBZ22.pdf}{NaderiBZ22}~\cite{NaderiBZ22}, \href{}{NaderiBZ22a}~\cite{NaderiBZ22a}, \href{}{NaderiRBAU21}~\cite{NaderiRBAU21}\\
\rowlabel{auth:a829}Maliheh Aramon Bajestani & 3 &31 &\href{works/BajestaniB15.pdf}{BajestaniB15}~\cite{BajestaniB15}, \href{works/BajestaniB13.pdf}{BajestaniB13}~\cite{BajestaniB13}, \href{works/BajestaniB11.pdf}{BajestaniB11}~\cite{BajestaniB11}\\
\rowlabel{auth:a11}S{\'{e}}v{\'{e}}rine Betmbe Fetgo & 3 &1 &\href{works/KameugneFND23.pdf}{KameugneFND23}~\cite{KameugneFND23}, \href{works/FetgoD22.pdf}{FetgoD22}~\cite{FetgoD22}, \href{works/KameugneFGOQ18.pdf}{KameugneFGOQ18}~\cite{KameugneFGOQ18}\\
\rowlabel{auth:a190}Miquel Bofill & 3 &11 &\href{works/BofillCSV17.pdf}{BofillCSV17}~\cite{BofillCSV17}, \href{works/BofillGSV15.pdf}{BofillGSV15}~\cite{BofillGSV15}, \href{works/BofillEGPSV14.pdf}{BofillEGPSV14}~\cite{BofillEGPSV14}\\
\rowlabel{auth:a233}Thomas Bridi & 3 &29 &\href{works/BridiBLMB16.pdf}{BridiBLMB16}~\cite{BridiBLMB16}, \href{works/BridiLBBM16.pdf}{BridiLBBM16}~\cite{BridiLBBM16}, \href{works/BartoliniBBLM14.pdf}{BartoliniBBLM14}~\cite{BartoliniBBLM14}\\
\rowlabel{auth:a172}Cid C. de Souza & 3 &21 &\href{works/MouraSCL08.pdf}{MouraSCL08}~\cite{MouraSCL08}, \href{works/MouraSCL08a.pdf}{MouraSCL08a}~\cite{MouraSCL08a}, \href{works/HeipckeCCS00.pdf}{HeipckeCCS00}~\cite{HeipckeCCS00}\\
\rowlabel{auth:a1025}Hadrien Cambazard & 3 &23 &\href{works/CatusseCBL16.pdf}{CatusseCBL16}~\cite{CatusseCBL16}, \href{works/MalapertCGJLR13.pdf}{MalapertCGJLR13}~\cite{MalapertCGJLR13}, \href{}{MalapertCGJLR12}~\cite{MalapertCGJLR12}\\
\rowlabel{auth:a42}Quentin Cappart & 3 &8 &\href{works/PopovicCGNC22.pdf}{PopovicCGNC22}~\cite{PopovicCGNC22}, \href{works/CappartTSR18.pdf}{CappartTSR18}~\cite{CappartTSR18}, \href{works/CappartS17.pdf}{CappartS17}~\cite{CappartS17}\\
\rowlabel{auth:a163}Ondrej Cepek & 3 &36 &\href{works/BartakCS10.pdf}{BartakCS10}~\cite{BartakCS10}, \href{works/VilimBC05.pdf}{VilimBC05}~\cite{VilimBC05}, \href{works/VilimBC04.pdf}{VilimBC04}~\cite{VilimBC04}\\
\rowlabel{auth:a287}Amedeo Cesta & 3 &15 &\href{}{CestaOPS14}~\cite{CestaOPS14}, \href{works/OddiPCC03.pdf}{OddiPCC03}~\cite{OddiPCC03}, \href{works/CestaOS98.pdf}{CestaOS98}~\cite{CestaOS98}\\
\rowlabel{auth:a93}Giacomo Da Col & 3 &14 &\href{works/ColT22.pdf}{ColT22}~\cite{ColT22}, \href{works/abs-2102-08778.pdf}{abs-2102-08778}~\cite{abs-2102-08778}, \href{works/ColT19.pdf}{ColT19}~\cite{ColT19}\\
\rowlabel{auth:a226}Alban Derrien & 3 &17 &\href{works/Derrien15.pdf}{Derrien15}~\cite{Derrien15}, \href{works/DerrienP14.pdf}{DerrienP14}~\cite{DerrienP14}, \href{works/DerrienPZ14.pdf}{DerrienPZ14}~\cite{DerrienPZ14}\\
\rowlabel{auth:a295}Abdallah Elkhyari & 3 &10 &\href{works/Elkhyari03.pdf}{Elkhyari03}~\cite{Elkhyari03}, \href{works/ElkhyariGJ02.pdf}{ElkhyariGJ02}~\cite{ElkhyariGJ02}, \href{works/ElkhyariGJ02a.pdf}{ElkhyariGJ02a}~\cite{ElkhyariGJ02a}\\
\rowlabel{auth:a122}Hamed Fahimi & 3 &2 &\href{}{FahimiQ23}~\cite{FahimiQ23}, \href{works/FahimiOQ18.pdf}{FahimiOQ18}~\cite{FahimiOQ18}, \href{works/Fahimi16.pdf}{Fahimi16}~\cite{Fahimi16}\\
\rowlabel{auth:a385}Jeremy Frank & 3 &7 &\href{works/TranWDRFOVB16.pdf}{TranWDRFOVB16}~\cite{TranWDRFOVB16}, \href{works/TranDRFWOVB16.pdf}{TranDRFWOVB16}~\cite{TranDRFWOVB16}, \href{works/FrankK05.pdf}{FrankK05}~\cite{FrankK05}\\
\rowlabel{auth:a815}Douglas G. Down & 3 &20 &\href{works/TranPZLDB18.pdf}{TranPZLDB18}~\cite{TranPZLDB18}, \href{works/TerekhovTDB14.pdf}{TerekhovTDB14}~\cite{TerekhovTDB14}, \href{works/TranTDB13.pdf}{TranTDB13}~\cite{TranTDB13}\\
\rowlabel{auth:a198}Maurizio Gabbrielli & 3 &12 &\href{works/LiuCGM17.pdf}{LiuCGM17}~\cite{LiuCGM17}, \href{works/AmadiniGM16.pdf}{AmadiniGM16}~\cite{AmadiniGM16}, \href{works/FalaschiGMP97.pdf}{FalaschiGMP97}~\cite{FalaschiGMP97}\\
\rowlabel{auth:a15}Michele Garraffa & 3 &1 &\href{works/AlfieriGPS23.pdf}{AlfieriGPS23}~\cite{AlfieriGPS23}, \href{works/ArmstrongGOS22.pdf}{ArmstrongGOS22}~\cite{ArmstrongGOS22}, \href{works/ArmstrongGOS21.pdf}{ArmstrongGOS21}~\cite{ArmstrongGOS21}\\
\rowlabel{auth:a61}Martin Gebser & 3 &0 &\href{works/TasselGS23.pdf}{TasselGS23}~\cite{TasselGS23}, \href{works/abs-2306-05747.pdf}{abs-2306-05747}~\cite{abs-2306-05747}, \href{works/KovacsTKSG21.pdf}{KovacsTKSG21}~\cite{KovacsTKSG21}\\
\rowlabel{auth:a692}Jean{-}Claude Gentina & 3 &8 &\href{works/KorbaaYG00.pdf}{KorbaaYG00}~\cite{KorbaaYG00}, \href{works/LopezAKYG00.pdf}{LopezAKYG00}~\cite{LopezAKYG00}, \href{works/KorbaaYG99.pdf}{KorbaaYG99}~\cite{KorbaaYG99}\\
\rowlabel{auth:a83}Lucas Groleaz & 3 &4 &\href{works/Groleaz21.pdf}{Groleaz21}~\cite{Groleaz21}, \href{works/GroleazNS20.pdf}{GroleazNS20}~\cite{GroleazNS20}, \href{works/GroleazNS20a.pdf}{GroleazNS20a}~\cite{GroleazNS20a}\\
\rowlabel{auth:a760}Andy Ham & 3 &20 &\href{works/HamPK21.pdf}{HamPK21}~\cite{HamPK21}, \href{works/Ham18.pdf}{Ham18}~\cite{Ham18}, \href{}{Ham18a}~\cite{Ham18a}\\
\rowlabel{auth:a218}Renaud Hartert & 3 &35 &\href{works/GayHLS15.pdf}{GayHLS15}~\cite{GayHLS15}, \href{works/GayHS15.pdf}{GayHS15}~\cite{GayHS15}, \href{works/GayHS15a.pdf}{GayHS15a}~\cite{GayHS15a}\\
\rowlabel{auth:a138}Brahim Hnich & 3 &68 &\href{works/GokgurHO18.pdf}{GokgurHO18}~\cite{GokgurHO18}, \href{works/OzturkTHO13.pdf}{OzturkTHO13}~\cite{OzturkTHO13}, \href{works/RossiTHP07.pdf}{RossiTHP07}~\cite{RossiTHP07}\\
\rowlabel{auth:a54}Marie{-}Jos{\'{e}} Huguet & 3 &12 &\href{works/AntuoriHHEN21.pdf}{AntuoriHHEN21}~\cite{AntuoriHHEN21}, \href{works/AntuoriHHEN20.pdf}{AntuoriHHEN20}~\cite{AntuoriHHEN20}, \href{works/HebrardHJMPV16.pdf}{HebrardHJMPV16}~\cite{HebrardHJMPV16}\\
\rowlabel{auth:a251}Andrew J. Davenport & 3 &13 &\href{works/Davenport10.pdf}{Davenport10}~\cite{Davenport10}, \href{works/DavenportKRSH07.pdf}{DavenportKRSH07}~\cite{DavenportKRSH07}, \href{works/BeckDF97.pdf}{BeckDF97}~\cite{BeckDF97}\\
\rowlabel{auth:a75}Mikael Johansson & 3 &27 &\href{works/Astrand0F21.pdf}{Astrand0F21}~\cite{Astrand0F21}, \href{works/AstrandJZ20.pdf}{AstrandJZ20}~\cite{AstrandJZ20}, \href{works/AstrandJZ18.pdf}{AstrandJZ18}~\cite{AstrandJZ18}\\
\rowlabel{auth:a690}Ouajdi Korbaa & 3 &8 &\href{works/KorbaaYG00.pdf}{KorbaaYG00}~\cite{KorbaaYG00}, \href{works/LopezAKYG00.pdf}{LopezAKYG00}~\cite{LopezAKYG00}, \href{works/KorbaaYG99.pdf}{KorbaaYG99}~\cite{KorbaaYG99}\\
\rowlabel{auth:a124}Stefan Kreter & 3 &47 &\href{works/KreterSSZ18.pdf}{KreterSSZ18}~\cite{KreterSSZ18}, \href{works/KreterSS17.pdf}{KreterSS17}~\cite{KreterSS17}, \href{works/KreterSS15.pdf}{KreterSS15}~\cite{KreterSS15}\\
\rowlabel{auth:a670}Krzysztof Kuchcinski & 3 &24 &\href{works/WolinskiKG04.pdf}{WolinskiKG04}~\cite{WolinskiKG04}, \href{works/KuchcinskiW03.pdf}{KuchcinskiW03}~\cite{KuchcinskiW03}, \href{works/GruianK98.pdf}{GruianK98}~\cite{GruianK98}\\
\rowlabel{auth:a655}Andr{\'{e}} Langevin & 3 &107 &\href{works/MalapertCGJLR13.pdf}{MalapertCGJLR13}~\cite{MalapertCGJLR13}, \href{}{MalapertCGJLR12}~\cite{MalapertCGJLR12}, \href{works/KhayatLR06.pdf}{KhayatLR06}~\cite{KhayatLR06}\\
\rowlabel{auth:a361}Philippe Michelon & 3 &68 &\href{works/Acuna-AgostMFG09.pdf}{Acuna-AgostMFG09}~\cite{Acuna-AgostMFG09}, \href{works/LiessM08.pdf}{LiessM08}~\cite{LiessM08}, \href{}{DemasseyAM05}~\cite{DemasseyAM05}\\
\rowlabel{auth:a158}Tony Minoru Tamura Lopes & 3 &47 &\href{works/LopesCSM10.pdf}{LopesCSM10}~\cite{LopesCSM10}, \href{works/MouraSCL08.pdf}{MouraSCL08}~\cite{MouraSCL08}, \href{works/MouraSCL08a.pdf}{MouraSCL08a}~\cite{MouraSCL08a}\\
\rowlabel{auth:a326}Christina N. Burt & 3 &15 &\href{works/BurtLPS15.pdf}{BurtLPS15}~\cite{BurtLPS15}, \href{works/BlomBPS14.pdf}{BlomBPS14}~\cite{BlomBPS14}, \href{works/LipovetzkyBPS14.pdf}{LipovetzkyBPS14}~\cite{LipovetzkyBPS14}\\
\rowlabel{auth:a538}Hiroki Nishikawa & 3 &3 &\href{works/NishikawaSTT19.pdf}{NishikawaSTT19}~\cite{NishikawaSTT19}, \href{works/NishikawaSTT18.pdf}{NishikawaSTT18}~\cite{NishikawaSTT18}, \href{works/NishikawaSTT18a.pdf}{NishikawaSTT18a}~\cite{NishikawaSTT18a}\\
\rowlabel{auth:a285}Angelo Oddi & 3 &15 &\href{}{CestaOPS14}~\cite{CestaOPS14}, \href{works/OddiPCC03.pdf}{OddiPCC03}~\cite{OddiPCC03}, \href{works/CestaOS98.pdf}{CestaOS98}~\cite{CestaOS98}\\
\rowlabel{auth:a914}David R. Urbach & 3 &100 &\href{}{NaderiRBAU21}~\cite{NaderiRBAU21}, \href{}{RoshanaeiBAUB20}~\cite{RoshanaeiBAUB20}, \href{}{RoshanaeiLAU17a}~\cite{RoshanaeiLAU17a}\\
\rowlabel{auth:a257}Philippe Refalo & 3 &60 &\href{works/GarganiR07.pdf}{GarganiR07}~\cite{GarganiR07}, \href{works/BeckR03.pdf}{BeckR03}~\cite{BeckR03}, \href{}{MilanoORT02}~\cite{MilanoORT02}\\
\rowlabel{auth:a424}Levi Ribeiro de Abreu & 3 &11 &\href{works/AbreuNP23.pdf}{AbreuNP23}~\cite{AbreuNP23}, \href{works/AbreuN22.pdf}{AbreuN22}~\cite{AbreuN22}, \href{works/AbreuAPNM21.pdf}{AbreuAPNM21}~\cite{AbreuAPNM21}\\
\rowlabel{auth:a305}Mark S. Fox & 3 &27 &\href{works/BeckF00.pdf}{BeckF00}~\cite{BeckF00}, \href{works/BeckF98.pdf}{BeckF98}~\cite{BeckF98}, \href{works/BeckDF97.pdf}{BeckDF97}~\cite{BeckDF97}\\
\rowlabel{auth:a720}Gunnar Schrader & 3 &13 &\href{works/Wolf09.pdf}{Wolf09}~\cite{Wolf09}, \href{works/WolfS05.pdf}{WolfS05}~\cite{WolfS05}, \href{works/SchuttWS05.pdf}{SchuttWS05}~\cite{SchuttWS05}\\
\rowlabel{auth:a135}Jens Schulz & 3 &40 &\href{works/HeinzSB13.pdf}{HeinzSB13}~\cite{HeinzSB13}, \href{works/HeinzS11.pdf}{HeinzS11}~\cite{HeinzS11}, \href{works/BertholdHLMS10.pdf}{BertholdHLMS10}~\cite{BertholdHLMS10}\\
\rowlabel{auth:a425}Marcelo Seido Nagano & 3 &11 &\href{works/AbreuNP23.pdf}{AbreuNP23}~\cite{AbreuNP23}, \href{works/AbreuN22.pdf}{AbreuN22}~\cite{AbreuN22}, \href{works/AbreuAPNM21.pdf}{AbreuAPNM21}~\cite{AbreuAPNM21}\\
\rowlabel{auth:a539}Kana Shimada & 3 &3 &\href{works/NishikawaSTT19.pdf}{NishikawaSTT19}~\cite{NishikawaSTT19}, \href{works/NishikawaSTT18.pdf}{NishikawaSTT18}~\cite{NishikawaSTT18}, \href{works/NishikawaSTT18a.pdf}{NishikawaSTT18a}~\cite{NishikawaSTT18a}\\
\rowlabel{auth:a127}Gilles Simonin & 3 &8 &\href{works/GodetLHS20.pdf}{GodetLHS20}~\cite{GodetLHS20}, \href{works/SimoninAHL15.pdf}{SimoninAHL15}~\cite{SimoninAHL15}, \href{works/SimoninAHL12.pdf}{SimoninAHL12}~\cite{SimoninAHL12}\\
\rowlabel{auth:a816}Tiago Stegun Vaquero & 3 &29 &\href{works/TranVNB17.pdf}{TranVNB17}~\cite{TranVNB17}, \href{works/TranVNB17a.pdf}{TranVNB17a}~\cite{TranVNB17a}, \href{works/LouieVNB14.pdf}{LouieVNB14}~\cite{LouieVNB14}\\
\rowlabel{auth:a192}Josep Suy & 3 &11 &\href{works/BofillCSV17.pdf}{BofillCSV17}~\cite{BofillCSV17}, \href{works/BofillGSV15.pdf}{BofillGSV15}~\cite{BofillGSV15}, \href{works/BofillEGPSV14.pdf}{BofillEGPSV14}~\cite{BofillEGPSV14}\\
\rowlabel{auth:a387}Christos T. Maravelias & 3 &396 &\href{}{Adelgren2023}~\cite{Adelgren2023}, \href{}{HarjunkoskiMBC14}~\cite{HarjunkoskiMBC14}, \href{works/MaraveliasG04.pdf}{MaraveliasG04}~\cite{MaraveliasG04}\\
\rowlabel{auth:a476}Andreas T. Ernst & 3 &16 &\href{works/abs-2211-14492.pdf}{abs-2211-14492}~\cite{abs-2211-14492}, \href{}{EdwardsBSE19}~\cite{EdwardsBSE19}, \href{works/ThiruvadyBME09.pdf}{ThiruvadyBME09}~\cite{ThiruvadyBME09}\\
\rowlabel{auth:a540}Ittetsu Taniguchi & 3 &3 &\href{works/NishikawaSTT19.pdf}{NishikawaSTT19}~\cite{NishikawaSTT19}, \href{works/NishikawaSTT18.pdf}{NishikawaSTT18}~\cite{NishikawaSTT18}, \href{works/NishikawaSTT18a.pdf}{NishikawaSTT18a}~\cite{NishikawaSTT18a}\\
\rowlabel{auth:a58}Pierre Tassel & 3 &0 &\href{works/TasselGS23.pdf}{TasselGS23}~\cite{TasselGS23}, \href{works/abs-2306-05747.pdf}{abs-2306-05747}~\cite{abs-2306-05747}, \href{works/KovacsTKSG21.pdf}{KovacsTKSG21}~\cite{KovacsTKSG21}\\
\rowlabel{auth:a745}Reza Tavakkoli-Moghaddam & 3 &9 &\href{}{Fatemi-AnarakiTFV23}~\cite{Fatemi-AnarakiTFV23}, \href{}{NouriMHD23}~\cite{NouriMHD23}, \href{}{GhasemiMH23}~\cite{GhasemiMH23}\\
\rowlabel{auth:a541}Hiroyuki Tomiyama & 3 &3 &\href{works/NishikawaSTT19.pdf}{NishikawaSTT19}~\cite{NishikawaSTT19}, \href{works/NishikawaSTT18.pdf}{NishikawaSTT18}~\cite{NishikawaSTT18}, \href{works/NishikawaSTT18a.pdf}{NishikawaSTT18a}~\cite{NishikawaSTT18a}\\
\rowlabel{auth:a427}Seyda Topaloglu Yildiz & 3 &20 &\href{works/IsikYA23.pdf}{IsikYA23}~\cite{IsikYA23}, \href{works/YunusogluY22.pdf}{YunusogluY22}~\cite{YunusogluY22}, \href{works/KucukY19.pdf}{KucukY19}~\cite{KucukY19}\\
\rowlabel{auth:a207}Sascha Van Cauwelaert & 3 &8 &\href{works/CauwelaertLS18.pdf}{CauwelaertLS18}~\cite{CauwelaertLS18}, \href{works/CauwelaertDMS16.pdf}{CauwelaertDMS16}~\cite{CauwelaertDMS16}, \href{works/DejemeppeCS15.pdf}{DejemeppeCS15}~\cite{DejemeppeCS15}\\
\rowlabel{auth:a175}G{\'{e}}rard Verfaillie & 3 &119 &\href{works/HebrardHJMPV16.pdf}{HebrardHJMPV16}~\cite{HebrardHJMPV16}, \href{works/VerfaillieL01.pdf}{VerfaillieL01}~\cite{VerfaillieL01}, \href{works/BensanaLV99.pdf}{BensanaLV99}~\cite{BensanaLV99}\\
\rowlabel{auth:a161}Arnaldo Vieira Moura & 3 &47 &\href{works/LopesCSM10.pdf}{LopesCSM10}~\cite{LopesCSM10}, \href{works/MouraSCL08.pdf}{MouraSCL08}~\cite{MouraSCL08}, \href{works/MouraSCL08a.pdf}{MouraSCL08a}~\cite{MouraSCL08a}\\
\rowlabel{auth:a193}Mateu Villaret & 3 &11 &\href{works/BofillCSV17.pdf}{BofillCSV17}~\cite{BofillCSV17}, \href{works/BofillGSV15.pdf}{BofillGSV15}~\cite{BofillGSV15}, \href{works/BofillEGPSV14.pdf}{BofillEGPSV14}~\cite{BofillEGPSV14}\\
\rowlabel{auth:a46}Daniel Walkiewicz & 3 &0 &\href{works/LacknerMMWW23.pdf}{LacknerMMWW23}~\cite{LacknerMMWW23}, \href{works/WinterMMW22.pdf}{WinterMMW22}~\cite{WinterMMW22}, \href{works/LacknerMMWW21.pdf}{LacknerMMWW21}~\cite{LacknerMMWW21}\\
\rowlabel{auth:a691}Pascal Yim & 3 &8 &\href{works/KorbaaYG00.pdf}{KorbaaYG00}~\cite{KorbaaYG00}, \href{works/LopezAKYG00.pdf}{LopezAKYG00}~\cite{LopezAKYG00}, \href{works/KorbaaYG99.pdf}{KorbaaYG99}~\cite{KorbaaYG99}\\
\rowlabel{auth:a205}Alessandro Zanarini & 3 &25 &\href{works/AstrandJZ20.pdf}{AstrandJZ20}~\cite{AstrandJZ20}, \href{works/AstrandJZ18.pdf}{AstrandJZ18}~\cite{AstrandJZ18}, \href{works/BonfiettiZLM16.pdf}{BonfiettiZLM16}~\cite{BonfiettiZLM16}\\
\rowlabel{auth:a631}Luis Zeballos & 3 &35 &\href{works/ZeballosQH10.pdf}{ZeballosQH10}~\cite{ZeballosQH10}, \href{works/ZeballosH05.pdf}{ZeballosH05}~\cite{ZeballosH05}, \href{works/QuirogaZH05.pdf}{QuirogaZH05}~\cite{QuirogaZH05}\\
\rowlabel{auth:a560}Viktoria A. Hauder & 2 &14 &\href{}{HauderBRPA20}~\cite{HauderBRPA20}, \href{works/abs-1902-09244.pdf}{abs-1902-09244}~\cite{abs-1902-09244}\\
\rowlabel{auth:a893}Daniel A. Desmond & 2 &1 &\href{works/AntunesABD20.pdf}{AntunesABD20}~\cite{AntunesABD20}, \href{works/AntunesABD18.pdf}{AntunesABD18}~\cite{AntunesABD18}\\
\rowlabel{auth:a564}Michael Affenzeller & 2 &14 &\href{}{HauderBRPA20}~\cite{HauderBRPA20}, \href{works/abs-1902-09244.pdf}{abs-1902-09244}~\cite{abs-1902-09244}\\
\rowlabel{auth:a734}Abderrahmane Aggoun & 2 &187 &\href{}{AggounMV08}~\cite{AggounMV08}, \href{works/AggounB93.pdf}{AggounB93}~\cite{AggounB93}\\
\rowlabel{auth:a891}Mark Antunes & 2 &1 &\href{works/AntunesABD20.pdf}{AntunesABD20}~\cite{AntunesABD20}, \href{works/AntunesABD18.pdf}{AntunesABD18}~\cite{AntunesABD18}\\
\rowlabel{auth:a53}Valentin Antuori & 2 &3 &\href{works/AntuoriHHEN21.pdf}{AntuoriHHEN21}~\cite{AntuoriHHEN21}, \href{works/AntuoriHHEN20.pdf}{AntuoriHHEN20}~\cite{AntuoriHHEN20}\\
\rowlabel{auth:a892}Vincent Armant & 2 &1 &\href{works/AntunesABD20.pdf}{AntunesABD20}~\cite{AntunesABD20}, \href{works/AntunesABD18.pdf}{AntunesABD18}~\cite{AntunesABD18}\\
\rowlabel{auth:a14}Eddie Armstrong & 2 &1 &\href{works/ArmstrongGOS22.pdf}{ArmstrongGOS22}~\cite{ArmstrongGOS22}, \href{works/ArmstrongGOS21.pdf}{ArmstrongGOS21}~\cite{ArmstrongGOS21}\\
\rowlabel{auth:a352}Emrah B. Edis & 2 &48 &\href{works/EdisO11.pdf}{EdisO11}~\cite{EdisO11}, \href{}{EdisO11a}~\cite{EdisO11a}\\
\rowlabel{auth:a504}Amelia Badica & 2 &4 &\href{works/BadicaBI20.pdf}{BadicaBI20}~\cite{BadicaBI20}, \href{works/BadicaBIL19.pdf}{BadicaBIL19}~\cite{BadicaBIL19}\\
\rowlabel{auth:a505}Costin Badica & 2 &4 &\href{works/BadicaBI20.pdf}{BadicaBI20}~\cite{BadicaBI20}, \href{works/BadicaBIL19.pdf}{BadicaBIL19}~\cite{BadicaBIL19}\\
\rowlabel{auth:a703}Pierre Baptiste & 2 &13 &\href{}{BoucherBVBL97}~\cite{BoucherBVBL97}, \href{works/BaptisteLV92.pdf}{BaptisteLV92}~\cite{BaptisteLV92}\\
\rowlabel{auth:a400}Nicolas Barnier & 2 &0 &\href{works/WangB23.pdf}{WangB23}~\cite{WangB23}, \href{works/WangB20.pdf}{WangB20}~\cite{WangB20}\\
\rowlabel{auth:a561}Andreas Beham & 2 &14 &\href{}{HauderBRPA20}~\cite{HauderBRPA20}, \href{works/abs-1902-09244.pdf}{abs-1902-09244}~\cite{abs-1902-09244}\\
\rowlabel{auth:a114}Ondrej Benedikt & 2 &3 &\href{works/BenediktMH20.pdf}{BenediktMH20}~\cite{BenediktMH20}, \href{works/BenediktSMVH18.pdf}{BenediktSMVH18}~\cite{BenediktSMVH18}\\
\rowlabel{auth:a381}Davide Bertozzi & 2 &27 &\href{works/RuggieroBBMA09.pdf}{RuggieroBBMA09}~\cite{RuggieroBBMA09}, \href{works/BeniniBGM06.pdf}{BeniniBGM06}~\cite{BeniniBGM06}\\
\rowlabel{auth:a343}Jean{-}Charles Billaut & 2 &23 &\href{works/BillautHL12.pdf}{BillautHL12}~\cite{BillautHL12}, \href{works/LorigeonBB02.pdf}{LorigeonBB02}~\cite{LorigeonBB02}\\
\rowlabel{auth:a232}Andrea Borghesi & 2 &23 &\href{works/BorghesiBLMB18.pdf}{BorghesiBLMB18}~\cite{BorghesiBLMB18}, \href{works/BartoliniBBLM14.pdf}{BartoliniBBLM14}~\cite{BartoliniBBLM14}\\
\rowlabel{auth:a413}Dario Canut{-}de{-}Bon & 2 &1 &\href{works/YuraszeckMCCR23.pdf}{YuraszeckMCCR23}~\cite{YuraszeckMCCR23}, \href{works/YuraszeckMC23.pdf}{YuraszeckMC23}~\cite{YuraszeckMC23}\\
\rowlabel{auth:a275}Tom Carchrae & 2 &16 &\href{works/CarchraeB09.pdf}{CarchraeB09}~\cite{CarchraeB09}, \href{works/CarchraeBF05.pdf}{CarchraeBF05}~\cite{CarchraeBF05}\\
\rowlabel{auth:a858}Jacques Carlier & 2 &6 &\href{}{CarlierSJP21}~\cite{CarlierSJP21}, \href{}{NeronABCDD06}~\cite{NeronABCDD06}\\
\rowlabel{auth:a94}Erich Christian Teppan & 2 &11 &\href{works/Teppan22.pdf}{Teppan22}~\cite{Teppan22}, \href{works/ColT19.pdf}{ColT19}~\cite{ColT19}\\
\rowlabel{auth:a102}Jordi Coll Caballero & 2 &0 &\href{works/Caballero23.pdf}{Caballero23}~\cite{Caballero23}, \href{works/Caballero19.pdf}{Caballero19}~\cite{Caballero19}\\
\rowlabel{auth:a170}Yves Colombani & 2 &9 &\href{works/HeipckeCCS00.pdf}{HeipckeCCS00}~\cite{HeipckeCCS00}, \href{works/Colombani96.pdf}{Colombani96}~\cite{Colombani96}\\
\rowlabel{auth:a132}Joseph D. Scott & 2 &13 &\href{works/KameugneFSN14.pdf}{KameugneFSN14}~\cite{KameugneFSN14}, \href{works/KameugneFSN11.pdf}{KameugneFSN11}~\cite{KameugneFSN11}\\
\rowlabel{auth:a290}Emilie Danna & 2 &23 &\href{}{DannaP04}~\cite{DannaP04}, \href{works/DannaP03.pdf}{DannaP03}~\cite{DannaP03}\\
\rowlabel{auth:a1020}St{\'{e}}phane Dauz{\`{e}}re{-}P{\'{e}}r{\`{e}}s & 2 &14 &\href{}{PenzDN23}~\cite{PenzDN23}, \href{}{NattafDYW19}~\cite{NattafDYW19}\\
\rowlabel{auth:a417}Mauro Dell'Amico & 2 &2 &\href{works/MontemanniD23.pdf}{MontemanniD23}~\cite{MontemanniD23}, \href{works/MontemanniD23a.pdf}{MontemanniD23a}~\cite{MontemanniD23a}\\
\rowlabel{auth:a821}Minh Do & 2 &3 &\href{works/TranWDRFOVB16.pdf}{TranWDRFOVB16}~\cite{TranWDRFOVB16}, \href{works/TranDRFWOVB16.pdf}{TranDRFWOVB16}~\cite{TranDRFWOVB16}\\
\rowlabel{auth:a922}Ulrich Dorndorf & 2 &18 &\href{}{DorndorfPH99}~\cite{DorndorfPH99}, \href{}{DorndorfHP99}~\cite{DorndorfHP99}\\
\rowlabel{auth:a168}Hani El Sakkout & 2 &82 &\href{works/KamarainenS02.pdf}{KamarainenS02}~\cite{KamarainenS02}, \href{works/SakkoutW00.pdf}{SakkoutW00}~\cite{SakkoutW00}\\
\rowlabel{auth:a70}Sebastian Engell & 2 &384 &\href{works/KlankeBYE21.pdf}{KlankeBYE21}~\cite{KlankeBYE21}, \href{}{HarjunkoskiMBC14}~\cite{HarjunkoskiMBC14}\\
\rowlabel{auth:a421}Tamer Eren & 2 &1 &\href{works/GurPAE23.pdf}{GurPAE23}~\cite{GurPAE23}, \href{works/GurEA19.pdf}{GurEA19}~\cite{GurEA19}\\
\rowlabel{auth:a894}Guillaume Escamocher & 2 &1 &\href{works/AntunesABD20.pdf}{AntunesABD20}~\cite{AntunesABD20}, \href{works/AntunesABD18.pdf}{AntunesABD18}~\cite{AntunesABD18}\\
\rowlabel{auth:a55}Siham Essodaigui & 2 &3 &\href{works/AntuoriHHEN21.pdf}{AntuoriHHEN21}~\cite{AntuoriHHEN21}, \href{works/AntuoriHHEN20.pdf}{AntuoriHHEN20}~\cite{AntuoriHHEN20}\\
\rowlabel{auth:a220}Caroline Even & 2 &3 &\href{works/EvenSH15.pdf}{EvenSH15}~\cite{EvenSH15}, \href{works/EvenSH15a.pdf}{EvenSH15a}~\cite{EvenSH15a}\\
\rowlabel{auth:a301}Stephen F. Smith & 2 &7 &\href{}{CestaOPS14}~\cite{CestaOPS14}, \href{works/CestaOS98.pdf}{CestaOS98}~\cite{CestaOS98}\\
\rowlabel{auth:a629}Minhaz F. Zibran & 2 &43 &\href{works/ZibranR11.pdf}{ZibranR11}~\cite{ZibranR11}, \href{works/ZibranR11a.pdf}{ZibranR11a}~\cite{ZibranR11a}\\
\rowlabel{auth:a523}Azadeh Farsi & 2 &25 &\href{works/FarsiTM22.pdf}{FarsiTM22}~\cite{FarsiTM22}, \href{works/MokhtarzadehTNF20.pdf}{MokhtarzadehTNF20}~\cite{MokhtarzadehTNF20}\\
\rowlabel{auth:a362}Dominique Feillet & 2 &19 &\href{works/Acuna-AgostMFG09.pdf}{Acuna-AgostMFG09}~\cite{Acuna-AgostMFG09}, \href{works/ArtiguesBF04.pdf}{ArtiguesBF04}~\cite{ArtiguesBF04}\\
\rowlabel{auth:a9}Michel Gamache & 2 &0 &\href{works/AalianPG23.pdf}{AalianPG23}~\cite{AalianPG23}, \href{works/CampeauG22.pdf}{CampeauG22}~\cite{CampeauG22}\\
\rowlabel{auth:a235}Marc Garcia & 2 &10 &\href{works/BofillGSV15.pdf}{BofillGSV15}~\cite{BofillGSV15}, \href{works/BofillEGPSV14.pdf}{BofillEGPSV14}~\cite{BofillEGPSV14}\\
\rowlabel{auth:a643}Antonio Garrido & 2 &27 &\href{works/GarridoAO09.pdf}{GarridoAO09}~\cite{GarridoAO09}, \href{works/GarridoOS08.pdf}{GarridoOS08}~\cite{GarridoOS08}\\
\rowlabel{auth:a895}Anne{-}Marie George & 2 &1 &\href{works/AntunesABD20.pdf}{AntunesABD20}~\cite{AntunesABD20}, \href{works/AntunesABD18.pdf}{AntunesABD18}~\cite{AntunesABD18}\\
\rowlabel{auth:a822}Eleanor Gilbert Rieffel & 2 &3 &\href{works/TranWDRFOVB16.pdf}{TranWDRFOVB16}~\cite{TranWDRFOVB16}, \href{works/TranDRFWOVB16.pdf}{TranDRFWOVB16}~\cite{TranDRFWOVB16}\\
\rowlabel{auth:a316}Vincent Gingras & 2 &1 &\href{works/KameugneFGOQ18.pdf}{KameugneFGOQ18}~\cite{KameugneFGOQ18}, \href{works/GingrasQ16.pdf}{GingrasQ16}~\cite{GingrasQ16}\\
\rowlabel{auth:a478}Arthur Godet & 2 &1 &\href{works/Godet21a.pdf}{Godet21a}~\cite{Godet21a}, \href{works/GodetLHS20.pdf}{GodetLHS20}~\cite{GodetLHS20}\\
\rowlabel{auth:a195}Adrian Goldwaser & 2 &8 &\href{works/GoldwaserS18.pdf}{GoldwaserS18}~\cite{GoldwaserS18}, \href{works/GoldwaserS17.pdf}{GoldwaserS17}~\cite{GoldwaserS17}\\
\rowlabel{auth:a201}Arnaud Gotlieb & 2 &9 &\href{works/MossigeGSMC17.pdf}{MossigeGSMC17}~\cite{MossigeGSMC17}, \href{works/AlesioNBG14.pdf}{AlesioNBG14}~\cite{AlesioNBG14}\\
\rowlabel{auth:a884}Iiro Harjunkoski & 2 &550 &\href{}{HarjunkoskiMBC14}~\cite{HarjunkoskiMBC14}, \href{works/HarjunkoskiG02.pdf}{HarjunkoskiG02}~\cite{HarjunkoskiG02}\\
\rowlabel{auth:a439}Vil{\'{e}}m Heinz & 2 &5 &\href{works/abs-2305-19888.pdf}{abs-2305-19888}~\cite{abs-2305-19888}, \href{works/HeinzNVH22.pdf}{HeinzNVH22}~\cite{HeinzNVH22}\\
\rowlabel{auth:a64}Alessandro Hill & 2 &0 &\href{}{HillBCGN22}~\cite{HillBCGN22}, \href{works/HillTV21.pdf}{HillTV21}~\cite{HillTV21}\\
\rowlabel{auth:a336}Seyed Hossein Hashemi Doulabi & 2 &59 &\href{works/DoulabiRP16.pdf}{DoulabiRP16}~\cite{DoulabiRP16}, \href{works/DoulabiRP14.pdf}{DoulabiRP14}~\cite{DoulabiRP14}\\
\rowlabel{auth:a184}Georgiana Ifrim & 2 &12 &\href{works/GrimesIOS14.pdf}{GrimesIOS14}~\cite{GrimesIOS14}, \href{works/IfrimOS12.pdf}{IfrimOS12}~\cite{IfrimOS12}\\
\rowlabel{auth:a506}Mirjana Ivanovic & 2 &4 &\href{works/BadicaBI20.pdf}{BadicaBI20}~\cite{BadicaBI20}, \href{works/BadicaBIL19.pdf}{BadicaBIL19}~\cite{BadicaBIL19}\\
\rowlabel{auth:a854}Raf Jans & 2 &60 &\href{}{MartnezAJ22}~\cite{MartnezAJ22}, \href{works/Jans09.pdf}{Jans09}~\cite{Jans09}\\
\rowlabel{auth:a630}Chanchal K. Roy & 2 &43 &\href{works/ZibranR11.pdf}{ZibranR11}~\cite{ZibranR11}, \href{works/ZibranR11a.pdf}{ZibranR11a}~\cite{ZibranR11a}\\
\rowlabel{auth:a78}Lucas Kletzander & 2 &1 &\href{works/GeibingerKKMMW21.pdf}{GeibingerKKMMW21}~\cite{GeibingerKKMMW21}, \href{works/KletzanderM17.pdf}{KletzanderM17}~\cite{KletzanderM17}\\
\rowlabel{auth:a547}Jan Kristof Behrens & 2 &12 &\href{works/BehrensLM19.pdf}{BehrensLM19}~\cite{BehrensLM19}, \href{works/abs-1901-07914.pdf}{abs-1901-07914}~\cite{abs-1901-07914}\\
\rowlabel{auth:a337}Wen{-}Yang Ku & 2 &128 &\href{works/KuB16.pdf}{KuB16}~\cite{KuB16}, \href{works/HeinzKB13.pdf}{HeinzKB13}~\cite{HeinzKB13}\\
\rowlabel{auth:a807}Michelle L. Blom & 2 &35 &\href{works/BlomPS16.pdf}{BlomPS16}~\cite{BlomPS16}, \href{works/BlomBPS14.pdf}{BlomBPS14}~\cite{BlomBPS14}\\
\rowlabel{auth:a62}Marie{-}Louise Lackner & 2 &0 &\href{works/LacknerMMWW23.pdf}{LacknerMMWW23}~\cite{LacknerMMWW23}, \href{works/LacknerMMWW21.pdf}{LacknerMMWW21}~\cite{LacknerMMWW21}\\
\rowlabel{auth:a434}Arnaud Lallouet & 2 &0 &\href{works/PerezGSL23.pdf}{PerezGSL23}~\cite{PerezGSL23}, \href{works/abs-2312-13682.pdf}{abs-2312-13682}~\cite{abs-2312-13682}\\
\rowlabel{auth:a729}Evelina Lamma & 2 &12 &\href{works/LammaMM97.pdf}{LammaMM97}~\cite{LammaMM97}, \href{works/BrusoniCLMMT96.pdf}{BrusoniCLMMT96}~\cite{BrusoniCLMMT96}\\
\rowlabel{auth:a548}Ralph Lange & 2 &12 &\href{works/BehrensLM19.pdf}{BehrensLM19}~\cite{BehrensLM19}, \href{works/abs-1901-07914.pdf}{abs-1901-07914}~\cite{abs-1901-07914}\\
\rowlabel{auth:a704}Bruno Legeard & 2 &13 &\href{}{BoucherBVBL97}~\cite{BoucherBVBL97}, \href{works/BaptisteLV92.pdf}{BaptisteLV92}~\cite{BaptisteLV92}\\
\rowlabel{auth:a1001}Pierre Lemaire & 2 &32 &\href{works/CatusseCBL16.pdf}{CatusseCBL16}~\cite{CatusseCBL16}, \href{works/GuyonLPR12.pdf}{GuyonLPR12}~\cite{GuyonLPR12}\\
\rowlabel{auth:a174}Michel Lema{\^{\i}}tre & 2 &110 &\href{works/VerfaillieL01.pdf}{VerfaillieL01}~\cite{VerfaillieL01}, \href{works/BensanaLV99.pdf}{BensanaLV99}~\cite{BensanaLV99}\\
\rowlabel{auth:a213}BoonPing Lim & 2 &6 &\href{works/LimHTB16.pdf}{LimHTB16}~\cite{LimHTB16}, \href{works/LimBTBB15.pdf}{LimBTBB15}~\cite{LimBTBB15}\\
\rowlabel{auth:a145}Kamol Limtanyakul & 2 &6 &\href{works/LimtanyakulS12.pdf}{LimtanyakulS12}~\cite{LimtanyakulS12}, \href{works/Limtanyakul07.pdf}{Limtanyakul07}~\cite{Limtanyakul07}\\
\rowlabel{auth:a897}Yiqing Lin & 2 &1 &\href{works/AntunesABD20.pdf}{AntunesABD20}~\cite{AntunesABD20}, \href{works/AntunesABD18.pdf}{AntunesABD18}~\cite{AntunesABD18}\\
\rowlabel{auth:a327}Nir Lipovetzky & 2 &0 &\href{works/BurtLPS15.pdf}{BurtLPS15}~\cite{BurtLPS15}, \href{works/LipovetzkyBPS14.pdf}{LipovetzkyBPS14}~\cite{LipovetzkyBPS14}\\
\rowlabel{auth:a180}James Little & 2 &30 &\href{works/KrogtLPHJ07.pdf}{KrogtLPHJ07}~\cite{KrogtLPHJ07}, \href{works/Darby-DowmanLMZ97.pdf}{Darby-DowmanLMZ97}~\cite{Darby-DowmanLMZ97}\\
\rowlabel{auth:a472}Shixin Liu & 2 &0 &\href{works/LiFJZLL22.pdf}{LiFJZLL22}~\cite{LiFJZLL22}, \href{works/ZhangJZL22.pdf}{ZhangJZL22}~\cite{ZhangJZL22}\\
\rowlabel{auth:a247}Xavier Lorca & 2 &29 &\href{works/GodetLHS20.pdf}{GodetLHS20}~\cite{GodetLHS20}, \href{works/HermenierDL11.pdf}{HermenierDL11}~\cite{HermenierDL11}\\
\rowlabel{auth:a946}Curtiss Luong & 2 &115 &\href{}{RoshanaeiLAU17}~\cite{RoshanaeiLAU17}, \href{}{RoshanaeiLAU17a}~\cite{RoshanaeiLAU17a}\\
\rowlabel{auth:a648}Abid M. Malik & 2 &15 &\href{works/Malik08.pdf}{Malik08}~\cite{Malik08}, \href{works/MalikMB08.pdf}{MalikMB08}~\cite{MalikMB08}\\
\rowlabel{auth:a907}Pedro M. Castro & 2 &381 &\href{}{HarjunkoskiMBC14}~\cite{HarjunkoskiMBC14}, \href{}{CastroGR10}~\cite{CastroGR10}\\
\rowlabel{auth:a324}Gilles Madi{-}Wamba & 2 &1 &\href{works/Madi-WambaLOBM17.pdf}{Madi-WambaLOBM17}~\cite{Madi-WambaLOBM17}, \href{works/Madi-WambaB16.pdf}{Madi-WambaB16}~\cite{Madi-WambaB16}\\
\rowlabel{auth:a799}Adrien Maillard & 2 &9 &\href{works/HebrardALLCMR22.pdf}{HebrardALLCMR22}~\cite{HebrardALLCMR22}, \href{works/HebrardHJMPV16.pdf}{HebrardHJMPV16}~\cite{HebrardHJMPV16}\\
\rowlabel{auth:a549}Masoumeh Mansouri & 2 &12 &\href{works/BehrensLM19.pdf}{BehrensLM19}~\cite{BehrensLM19}, \href{works/abs-1901-07914.pdf}{abs-1901-07914}~\cite{abs-1901-07914}\\
\rowlabel{auth:a199}Jacopo Mauro & 2 &2 &\href{works/LiuCGM17.pdf}{LiuCGM17}~\cite{LiuCGM17}, \href{works/AmadiniGM16.pdf}{AmadiniGM16}~\cite{AmadiniGM16}\\
\rowlabel{auth:a430}Gonzalo Mej{\'{\i}}a & 2 &25 &\href{works/YuraszeckMC23.pdf}{YuraszeckMC23}~\cite{YuraszeckMC23}, \href{works/MejiaY20.pdf}{MejiaY20}~\cite{MejiaY20}\\
\rowlabel{auth:a730}Paola Mello & 2 &12 &\href{works/LammaMM97.pdf}{LammaMM97}~\cite{LammaMM97}, \href{works/BrusoniCLMMT96.pdf}{BrusoniCLMMT96}~\cite{BrusoniCLMMT96}\\
\rowlabel{auth:a937}Carlos Mencía & 2 &25 &\href{works/MenciaSV13.pdf}{MenciaSV13}~\cite{MenciaSV13}, \href{works/MenciaSV12.pdf}{MenciaSV12}~\cite{MenciaSV12}\\
\rowlabel{auth:a522}Mahdi Mokhtarzadeh & 2 &25 &\href{works/FarsiTM22.pdf}{FarsiTM22}~\cite{FarsiTM22}, \href{works/MokhtarzadehTNF20.pdf}{MokhtarzadehTNF20}~\cite{MokhtarzadehTNF20}\\
\rowlabel{auth:a416}Roberto Montemanni & 2 &2 &\href{works/MontemanniD23.pdf}{MontemanniD23}~\cite{MontemanniD23}, \href{works/MontemanniD23a.pdf}{MontemanniD23a}~\cite{MontemanniD23a}\\
\rowlabel{auth:a63}Christoph Mrkvicka & 2 &0 &\href{works/LacknerMMWW23.pdf}{LacknerMMWW23}~\cite{LacknerMMWW23}, \href{works/LacknerMMWW21.pdf}{LacknerMMWW21}~\cite{LacknerMMWW21}\\
\rowlabel{auth:a115}Istv{\'{a}}n M{\'{o}}dos & 2 &3 &\href{works/BenediktMH20.pdf}{BenediktMH20}~\cite{BenediktMH20}, \href{works/BenediktSMVH18.pdf}{BenediktSMVH18}~\cite{BenediktSMVH18}\\
\rowlabel{auth:a563}Sophie N. Parragh & 2 &14 &\href{}{HauderBRPA20}~\cite{HauderBRPA20}, \href{works/abs-1902-09244.pdf}{abs-1902-09244}~\cite{abs-1902-09244}\\
\rowlabel{auth:a84}Samba Ndojh Ndiaye & 2 &4 &\href{works/GroleazNS20.pdf}{GroleazNS20}~\cite{GroleazNS20}, \href{works/GroleazNS20a.pdf}{GroleazNS20a}~\cite{GroleazNS20a}\\
\rowlabel{auth:a133}Youcheu Ngo{-}Kateu & 2 &13 &\href{works/KameugneFSN14.pdf}{KameugneFSN14}~\cite{KameugneFSN14}, \href{works/KameugneFSN11.pdf}{KameugneFSN11}~\cite{KameugneFSN11}\\
\rowlabel{auth:a56}Alain Nguyen & 2 &3 &\href{works/AntuoriHHEN21.pdf}{AntuoriHHEN21}~\cite{AntuoriHHEN21}, \href{works/AntuoriHHEN20.pdf}{AntuoriHHEN20}~\cite{AntuoriHHEN20}\\
\rowlabel{auth:a401}Su Nguyen & 2 &0 &\href{works/abs-2402-00459.pdf}{abs-2402-00459}~\cite{abs-2402-00459}, \href{works/abs-2211-14492.pdf}{abs-2211-14492}~\cite{abs-2211-14492}\\
\rowlabel{auth:a440}Anton{\'{\i}}n Nov{\'{a}}k & 2 &5 &\href{works/abs-2305-19888.pdf}{abs-2305-19888}~\cite{abs-2305-19888}, \href{works/HeinzNVH22.pdf}{HeinzNVH22}~\cite{HeinzNVH22}\\
\rowlabel{auth:a823}Bryan O'Gorman & 2 &3 &\href{works/TranWDRFOVB16.pdf}{TranWDRFOVB16}~\cite{TranWDRFOVB16}, \href{works/TranDRFWOVB16.pdf}{TranDRFWOVB16}~\cite{TranDRFWOVB16}\\
\rowlabel{auth:a896}Mike O'Keeffe & 2 &1 &\href{works/AntunesABD20.pdf}{AntunesABD20}~\cite{AntunesABD20}, \href{works/AntunesABD18.pdf}{AntunesABD18}~\cite{AntunesABD18}\\
\rowlabel{auth:a645}Eva Onaindia & 2 &27 &\href{works/GarridoAO09.pdf}{GarridoAO09}~\cite{GarridoAO09}, \href{works/GarridoOS08.pdf}{GarridoOS08}~\cite{GarridoOS08}\\
\rowlabel{auth:a354}Irem Ozkarahan & 2 &89 &\href{}{EdisO11a}~\cite{EdisO11a}, \href{works/TopalogluO11.pdf}{TopalogluO11}~\cite{TopalogluO11}\\
\rowlabel{auth:a898}Cemalettin Ozturk & 2 &1 &\href{works/AntunesABD20.pdf}{AntunesABD20}~\cite{AntunesABD20}, \href{works/AntunesABD18.pdf}{AntunesABD18}~\cite{AntunesABD18}\\
\rowlabel{auth:a652}Carla P. Gomes & 2 &0 &\href{works/HoeveGSL07.pdf}{HoeveGSL07}~\cite{HoeveGSL07}, \href{works/GomesHS06.pdf}{GomesHS06}~\cite{GomesHS06}\\
\rowlabel{auth:a131}Laure Pauline Fotso & 2 &13 &\href{works/KameugneFSN14.pdf}{KameugneFSN14}~\cite{KameugneFSN14}, \href{works/KameugneFSN11.pdf}{KameugneFSN11}~\cite{KameugneFSN11}\\
\rowlabel{auth:a431}Guillaume Perez & 2 &0 &\href{works/PerezGSL23.pdf}{PerezGSL23}~\cite{PerezGSL23}, \href{works/abs-2312-13682.pdf}{abs-2312-13682}~\cite{abs-2312-13682}\\
\rowlabel{auth:a923}Toàn Phan Huy & 2 &18 &\href{}{DorndorfPH99}~\cite{DorndorfPH99}, \href{}{DorndorfHP99}~\cite{DorndorfHP99}\\
\rowlabel{auth:a286}Nicola Policella & 2 &10 &\href{}{CestaOPS14}~\cite{CestaOPS14}, \href{works/OddiPCC03.pdf}{OddiPCC03}~\cite{OddiPCC03}\\
\rowlabel{auth:a33}Enrico Pontelli & 2 &0 &\href{works/TardivoDFMP23.pdf}{TardivoDFMP23}~\cite{TardivoDFMP23}, \href{}{VillaverdeP04}~\cite{VillaverdeP04}\\
\rowlabel{auth:a899}Luis Quesada & 2 &1 &\href{works/AntunesABD20.pdf}{AntunesABD20}~\cite{AntunesABD20}, \href{works/AntunesABD18.pdf}{AntunesABD18}~\cite{AntunesABD18}\\
\rowlabel{auth:a632}Oscar Quiroga & 2 &35 &\href{works/ZeballosQH10.pdf}{ZeballosQH10}~\cite{ZeballosQH10}, \href{works/QuirogaZH05.pdf}{QuirogaZH05}~\cite{QuirogaZH05}\\
\rowlabel{auth:a348}G{\"{u}}nther R. Raidl & 2 &14 &\href{works/FrohnerTR19.pdf}{FrohnerTR19}~\cite{FrohnerTR19}, \href{works/RendlPHPR12.pdf}{RendlPHPR12}~\cite{RendlPHPR12}\\
\rowlabel{auth:a392}Levi R. Abreu & 2 &0 &\href{works/PrataAN23.pdf}{PrataAN23}~\cite{PrataAN23}, \href{}{AbreuPNF23}~\cite{AbreuPNF23}\\
\rowlabel{auth:a938}María R. Sierra & 2 &25 &\href{works/MenciaSV13.pdf}{MenciaSV13}~\cite{MenciaSV13}, \href{works/MenciaSV12.pdf}{MenciaSV12}~\cite{MenciaSV12}\\
\rowlabel{auth:a562}Sebastian Raggl & 2 &14 &\href{}{HauderBRPA20}~\cite{HauderBRPA20}, \href{works/abs-1902-09244.pdf}{abs-1902-09244}~\cite{abs-1902-09244}\\
\rowlabel{auth:a229}Vinas{\'{e}}tan Ratheil Houndji & 2 &5 &\href{works/HoundjiSW19.pdf}{HoundjiSW19}~\cite{HoundjiSW19}, \href{works/HoundjiSWD14.pdf}{HoundjiSWD14}~\cite{HoundjiSWD14}\\
\rowlabel{auth:a1003}David Rivreau & 2 &42 &\href{works/NattafALR16.pdf}{NattafALR16}~\cite{NattafALR16}, \href{works/GuyonLPR12.pdf}{GuyonLPR12}~\cite{GuyonLPR12}\\
\rowlabel{auth:a319}Francesca Rossi & 2 &29 &\href{works/GelainPRVW17.pdf}{GelainPRVW17}~\cite{GelainPRVW17}, \href{works/BartakSR10.pdf}{BartakSR10}~\cite{BartakSR10}\\
\rowlabel{auth:a908}Louis-Martin Rousseau & 2 &106 &\href{}{CastroGR10}~\cite{CastroGR10}, \href{}{CorreaLR07}~\cite{CorreaLR07}\\
\rowlabel{auth:a393}Marcelo S. Nagano & 2 &0 &\href{works/PrataAN23.pdf}{PrataAN23}~\cite{PrataAN23}, \href{}{AbreuPNF23}~\cite{AbreuPNF23}\\
\rowlabel{auth:a887}Erlendur S. Thorsteinsson & 2 &81 &\href{}{MilanoORT02}~\cite{MilanoORT02}, \href{works/Thorsteinsson01.pdf}{Thorsteinsson01}~\cite{Thorsteinsson01}\\
\rowlabel{auth:a390}Ruslan Sadykov & 2 &56 &\href{works/SadykovW06.pdf}{SadykovW06}~\cite{SadykovW06}, \href{works/Sadykov04.pdf}{Sadykov04}~\cite{Sadykov04}\\
\rowlabel{auth:a429}Konstantin Schekotihin & 2 &0 &\href{works/TasselGS23.pdf}{TasselGS23}~\cite{TasselGS23}, \href{works/abs-2306-05747.pdf}{abs-2306-05747}~\cite{abs-2306-05747}\\
\rowlabel{auth:a92}Christian Schulte & 2 &5 &\href{works/WessenCS20.pdf}{WessenCS20}~\cite{WessenCS20}, \href{works/FrimodigS19.pdf}{FrimodigS19}~\cite{FrimodigS19}\\
\rowlabel{auth:a653}Bart Selman & 2 &0 &\href{works/HoeveGSL07.pdf}{HoeveGSL07}~\cite{HoeveGSL07}, \href{works/GomesHS06.pdf}{GomesHS06}~\cite{GomesHS06}\\
\rowlabel{auth:a120}Paul Shaw & 2 &179 &\href{works/LaborieRSV18.pdf}{LaborieRSV18}~\cite{LaborieRSV18}, \href{works/VilimLS15.pdf}{VilimLS15}~\cite{VilimLS15}\\
\rowlabel{auth:a433}Wijnand Suijlen & 2 &0 &\href{works/PerezGSL23.pdf}{PerezGSL23}~\cite{PerezGSL23}, \href{works/abs-2312-13682.pdf}{abs-2312-13682}~\cite{abs-2312-13682}\\
\rowlabel{auth:a403}Yuan Sun & 2 &0 &\href{works/abs-2402-00459.pdf}{abs-2402-00459}~\cite{abs-2402-00459}, \href{works/abs-2211-14492.pdf}{abs-2211-14492}~\cite{abs-2211-14492}\\
\rowlabel{auth:a436}Reza Tavakkoli{-}Moghaddam & 2 &25 &\href{works/Mehdizadeh-Somarin23.pdf}{Mehdizadeh-Somarin23}~\cite{Mehdizadeh-Somarin23}, \href{works/MokhtarzadehTNF20.pdf}{MokhtarzadehTNF20}~\cite{MokhtarzadehTNF20}\\
\rowlabel{auth:a13}Cl{\'{e}}mentin Tayou Djam{\'{e}}gni & 2 &0 &\href{works/KameugneFND23.pdf}{KameugneFND23}~\cite{KameugneFND23}, \href{works/FetgoD22.pdf}{FetgoD22}~\cite{FetgoD22}\\
\rowlabel{auth:a618}Erich Teppan & 2 &3 &\href{works/abs-2102-08778.pdf}{abs-2102-08778}~\cite{abs-2102-08778}, \href{}{FriedrichFMRSST14}~\cite{FriedrichFMRSST14}\\
\rowlabel{auth:a185}Alexander Tesch & 2 &9 &\href{works/Tesch18.pdf}{Tesch18}~\cite{Tesch18}, \href{works/Tesch16.pdf}{Tesch16}~\cite{Tesch16}\\
\rowlabel{auth:a215}Sylvie Thi{\'{e}}baux & 2 &6 &\href{works/LimHTB16.pdf}{LimHTB16}~\cite{LimHTB16}, \href{works/LimBTBB15.pdf}{LimBTBB15}~\cite{LimBTBB15}\\
\rowlabel{auth:a847}Charles Thomas & 2 &6 &\href{works/ThomasKS20.pdf}{ThomasKS20}~\cite{ThomasKS20}, \href{works/CappartTSR18.pdf}{CappartTSR18}~\cite{CappartTSR18}\\
\rowlabel{auth:a438}Behdin Vahedi Nouri & 2 &25 &\href{works/Mehdizadeh-Somarin23.pdf}{Mehdizadeh-Somarin23}~\cite{Mehdizadeh-Somarin23}, \href{works/MokhtarzadehTNF20.pdf}{MokhtarzadehTNF20}~\cite{MokhtarzadehTNF20}\\
\rowlabel{auth:a747}Behdin Vahedi-Nouri & 2 &9 &\href{}{Fatemi-AnarakiTFV23}~\cite{Fatemi-AnarakiTFV23}, \href{}{NouriMHD23}~\cite{NouriMHD23}\\
\rowlabel{auth:a939}Ramiro Varela & 2 &25 &\href{works/MenciaSV13.pdf}{MenciaSV13}~\cite{MenciaSV13}, \href{works/MenciaSV12.pdf}{MenciaSV12}~\cite{MenciaSV12}\\
\rowlabel{auth:a702}Christophe Varnier & 2 &13 &\href{}{BoucherBVBL97}~\cite{BoucherBVBL97}, \href{works/BaptisteLV92.pdf}{BaptisteLV92}~\cite{BaptisteLV92}\\
\rowlabel{auth:a824}Davide Venturelli & 2 &3 &\href{works/TranWDRFOVB16.pdf}{TranWDRFOVB16}~\cite{TranWDRFOVB16}, \href{works/TranDRFWOVB16.pdf}{TranDRFWOVB16}~\cite{TranDRFWOVB16}\\
\rowlabel{auth:a399}Ruixin Wang & 2 &0 &\href{works/WangB23.pdf}{WangB23}~\cite{WangB23}, \href{works/WangB20.pdf}{WangB20}~\cite{WangB20}\\
\rowlabel{auth:a820}Zhihui Wang & 2 &3 &\href{works/TranWDRFOVB16.pdf}{TranWDRFOVB16}~\cite{TranWDRFOVB16}, \href{works/TranDRFWOVB16.pdf}{TranDRFWOVB16}~\cite{TranDRFWOVB16}\\
\rowlabel{auth:a366}Jean{-}Paul Watson & 2 &57 &\href{works/BeckFW11.pdf}{BeckFW11}~\cite{BeckFW11}, \href{works/WatsonB08.pdf}{WatsonB08}~\cite{WatsonB08}\\
\rowlabel{auth:a277}Christine Wei Wu & 2 &42 &\href{works/WuBB09.pdf}{WuBB09}~\cite{WuBB09}, \href{works/WuBB05.pdf}{WuBB05}~\cite{WuBB05}\\
\rowlabel{auth:a669}Christophe Wolinski & 2 &19 &\href{works/WolinskiKG04.pdf}{WolinskiKG04}~\cite{WolinskiKG04}, \href{works/KuchcinskiW03.pdf}{KuchcinskiW03}~\cite{KuchcinskiW03}\\
\rowlabel{auth:a462}Farouk Yalaoui & 2 &3 &\href{works/OujanaAYB22.pdf}{OujanaAYB22}~\cite{OujanaAYB22}, \href{works/ArbaouiY18.pdf}{ArbaouiY18}~\cite{ArbaouiY18}\\
\rowlabel{auth:a19}Neil Yorke{-}Smith & 2 &5 &\href{works/EfthymiouY23.pdf}{EfthymiouY23}~\cite{EfthymiouY23}, \href{works/WallaceY20.pdf}{WallaceY20}~\cite{WallaceY20}\\
\rowlabel{auth:a470}Ziyan Zhao & 2 &0 &\href{works/LiFJZLL22.pdf}{LiFJZLL22}~\cite{LiFJZLL22}, \href{works/ZhangJZL22.pdf}{ZhangJZL22}~\cite{ZhangJZL22}\\
\rowlabel{auth:a178}Jianyang Zhou & 2 &24 &\href{works/Zhou97.pdf}{Zhou97}~\cite{Zhou97}, \href{works/Zhou96.pdf}{Zhou96}~\cite{Zhou96}\\
\rowlabel{auth:a216}Menkes van den Briel & 2 &6 &\href{works/LimHTB16.pdf}{LimHTB16}~\cite{LimHTB16}, \href{works/LimBTBB15.pdf}{LimBTBB15}~\cite{LimBTBB15}\\
\rowlabel{auth:a620}Peter van Beek & 2 &16 &\href{works/BegB13.pdf}{BegB13}~\cite{BegB13}, \href{works/MalikMB08.pdf}{MalikMB08}~\cite{MalikMB08}\\
\rowlabel{auth:a951} & 1 &63 &\href{}{ArtiguesDN08}~\cite{ArtiguesDN08}\\
\rowlabel{auth:a108}Florian A. Herzog & 1 &2 &\href{works/KoehlerBFFHPSSS21.pdf}{KoehlerBFFHPSSS21}~\cite{KoehlerBFFHPSSS21}\\
\rowlabel{auth:a380}J. A. Hoogeveen & 1 &2 &\href{works/AkkerDH07.pdf}{AkkerDH07}~\cite{AkkerDH07}\\
\rowlabel{auth:a395}M. A. Hakim Newton & 1 &0 &\href{works/RiahiNS018.pdf}{RiahiNS018}~\cite{RiahiNS018}\\
\rowlabel{auth:a569}Amr A. Kandil & 1 &24 &\href{works/TangLWSK18.pdf}{TangLWSK18}~\cite{TangLWSK18}\\
\rowlabel{auth:a677}Antonio A. M{\'{a}}rquez & 1 &7 &\href{works/ValleMGT03.pdf}{ValleMGT03}~\cite{ValleMGT03}\\
\rowlabel{auth:a757}Kennedy A. G. Ara{\'u}jo & 1 &0 &\href{works/AbreuAPNM21.pdf}{AbreuAPNM21}~\cite{AbreuAPNM21}\\
\rowlabel{auth:a798}Steve A. Chien & 1 &0 &\href{works/HebrardALLCMR22.pdf}{HebrardALLCMR22}~\cite{HebrardALLCMR22}\\
\rowlabel{auth:a828}Sheila A. McIlraith & 1 &0 &\href{works/LuoVLBM16.pdf}{LuoVLBM16}~\cite{LuoVLBM16}\\
\rowlabel{auth:a912}Andre A. Ciré & 1 &15 &\href{}{CireCH16}~\cite{CireCH16}\\
\rowlabel{auth:a942}Julie A. Shah & 1 &71 &\href{}{GombolayWS18}~\cite{GombolayWS18}\\
\rowlabel{auth:a7}Younes Aalian & 1 &0 &\href{works/AalianPG23.pdf}{AalianPG23}~\cite{AalianPG23}\\
\rowlabel{auth:a1006}E.H.L. Aarts & 1 &65 &\href{}{NuijtenA96}~\cite{NuijtenA96}\\
\rowlabel{auth:a479}Hanaa Abohashima & 1 &1 &\href{works/AbohashimaEG21.pdf}{AbohashimaEG21}~\cite{AbohashimaEG21}\\
\rowlabel{auth:a273}Montserrat Abril & 1 &0 &\href{works/AbrilSB05.pdf}{AbrilSB05}~\cite{AbrilSB05}\\
\rowlabel{auth:a360}Rodrigo Acuna{-}Agost & 1 &3 &\href{works/Acuna-AgostMFG09.pdf}{Acuna-AgostMFG09}~\cite{Acuna-AgostMFG09}\\
\rowlabel{auth:a990}Nathan Adelgren & 1 &0 &\href{}{Adelgren2023}~\cite{Adelgren2023}\\
\rowlabel{auth:a537}W. Adelman & 1 &17 &\href{works/EscobetPQPRA19.pdf}{EscobetPQPRA19}~\cite{EscobetPQPRA19}\\
\rowlabel{auth:a958}Yossiri Adulyasak & 1 &1 &\href{}{MartnezAJ22}~\cite{MartnezAJ22}\\
\rowlabel{auth:a984}Sezin Afsar & 1 &0 &\href{}{AfsarVPG23}~\cite{AfsarVPG23}\\
\rowlabel{auth:a325}Pen{\'{e}}lope Aguiar{-}Melgarejo & 1 &14 &\href{works/MelgarejoLS15.pdf}{MelgarejoLS15}~\cite{MelgarejoLS15}\\
\rowlabel{auth:a673}Sanjay Ahire & 1 &0 &\href{works/KanetAG04.pdf}{KanetAG04}~\cite{KanetAG04}\\
\rowlabel{auth:a422}Aftab Ahmed Shaikh & 1 &0 &\href{works/ShaikhK23.pdf}{ShaikhK23}~\cite{ShaikhK23}\\
\rowlabel{auth:a477}Uwe Aickelin & 1 &0 &\href{works/abs-2211-14492.pdf}{abs-2211-14492}~\cite{abs-2211-14492}\\
\rowlabel{auth:a972}Farid Ajili & 1 &4 &\href{}{AjiliW04}~\cite{AjiliW04}\\
\rowlabel{auth:a842}Ali Akbar Sadat Asl & 1 &55 &\href{works/ZarandiASC20.pdf}{ZarandiASC20}~\cite{ZarandiASC20}\\
\rowlabel{auth:a601}Mohsen Akbarpour Shirazi & 1 &28 &\href{works/ZarandiKS16.pdf}{ZarandiKS16}~\cite{ZarandiKS16}\\
\rowlabel{auth:a738}Arianna Alfieri & 1 &0 &\href{works/AlfieriGPS23.pdf}{AlfieriGPS23}~\cite{AlfieriGPS23}\\
\rowlabel{auth:a749}S. Ali Torabi & 1 &0 &\href{works/FarsiTM22.pdf}{FarsiTM22}~\cite{FarsiTM22}\\
\rowlabel{auth:a520}Samira Alizdeh & 1 &1 &\href{}{AlizdehS20}~\cite{AlizdehS20}\\
\rowlabel{auth:a693}Hassane Alla & 1 &0 &\href{works/LopezAKYG00.pdf}{LopezAKYG00}~\cite{LopezAKYG00}\\
\rowlabel{auth:a929}Roberto Amadini & 1 &2 &\href{works/AmadiniGM16.pdf}{AmadiniGM16}~\cite{AmadiniGM16}\\
\rowlabel{auth:a461}Lionel Amodeo & 1 &1 &\href{works/OujanaAYB22.pdf}{OujanaAYB22}~\cite{OujanaAYB22}\\
\rowlabel{auth:a728}Alexandru Andrei & 1 &9 &\href{works/RuggieroBBMA09.pdf}{RuggieroBBMA09}~\cite{RuggieroBBMA09}\\
\rowlabel{auth:a298}Ola Angelsmark & 1 &1 &\href{works/AngelsmarkJ00.pdf}{AngelsmarkJ00}~\cite{AngelsmarkJ00}\\
\rowlabel{auth:a826}Richard Anthony Valenzano & 1 &0 &\href{works/LuoVLBM16.pdf}{LuoVLBM16}~\cite{LuoVLBM16}\\
\rowlabel{auth:a712}M. Anton Ertl & 1 &14 &\href{works/ErtlK91.pdf}{ErtlK91}~\cite{ErtlK91}\\
\rowlabel{auth:a642}Zbigniew Antoni Banaszak & 1 &0 &\href{works/BocewiczBB09.pdf}{BocewiczBB09}~\cite{BocewiczBB09}\\
\rowlabel{auth:a644}Marlene Arang{\'{u}} & 1 &5 &\href{works/GarridoAO09.pdf}{GarridoAO09}~\cite{GarridoAO09}\\
\rowlabel{auth:a819}Arthur Araujo & 1 &72 &\href{works/TranAB16.pdf}{TranAB16}~\cite{TranAB16}\\
\rowlabel{auth:a588}Taha Arbaoui & 1 &2 &\href{works/ArbaouiY18.pdf}{ArbaouiY18}~\cite{ArbaouiY18}\\
\rowlabel{auth:a943}Dmitry Arkhipov & 1 &12 &\href{}{ArkhipovBL19}~\cite{ArkhipovBL19}\\
\rowlabel{auth:a717}Martin Aronsson & 1 &0 &\href{works/AronssonBK09.pdf}{AronssonBK09}~\cite{AronssonBK09}\\
\rowlabel{auth:a139}M. Arslan Ornek & 1 &31 &\href{works/OzturkTHO13.pdf}{OzturkTHO13}~\cite{OzturkTHO13}\\
\rowlabel{auth:a265}Konstantin Artiouchine & 1 &3 &\href{works/ArtiouchineB05.pdf}{ArtiouchineB05}~\cite{ArtiouchineB05}\\
\rowlabel{auth:a769}Arezoo Atighehchian & 1 &0 &\href{works/YounespourAKE19.pdf}{YounespourAKE19}~\cite{YounespourAKE19}\\
\rowlabel{auth:a423}Abdullah Ayub Khan & 1 &0 &\href{works/ShaikhK23.pdf}{ShaikhK23}~\cite{ShaikhK23}\\
\rowlabel{auth:a480}Amr B. Eltawil & 1 &1 &\href{works/AbohashimaEG21.pdf}{AbohashimaEG21}~\cite{AbohashimaEG21}\\
\rowlabel{auth:a671}Maya B. Gokhale & 1 &0 &\href{works/WolinskiKG04.pdf}{WolinskiKG04}~\cite{WolinskiKG04}\\
\rowlabel{auth:a711}David B. H. Tay & 1 &0 &\href{}{Tay92}~\cite{Tay92}\\
\rowlabel{auth:a910}Davaatseren Baatar & 1 &3 &\href{}{EdwardsBSE19}~\cite{EdwardsBSE19}\\
\rowlabel{auth:a99}{\"{O}}zalp Babaoglu & 1 &1 &\href{works/GalleguillosKSB19.pdf}{GalleguillosKSB19}~\cite{GalleguillosKSB19}\\
\rowlabel{auth:a641}Irena Bach & 1 &0 &\href{works/BocewiczBB09.pdf}{BocewiczBB09}~\cite{BocewiczBB09}\\
\rowlabel{auth:a701}Astrid Bachelu & 1 &0 &\href{}{BoucherBVBL97}~\cite{BoucherBVBL97}\\
\rowlabel{auth:a330}Scott Backhaus & 1 &4 &\href{works/LimBTBB15.pdf}{LimBTBB15}~\cite{LimBTBB15}\\
\rowlabel{auth:a585}Hari Balasubramanian & 1 &9 &\href{works/ShinBBHO18.pdf}{ShinBBHO18}~\cite{ShinBBHO18}\\
\rowlabel{auth:a372}Viet Bang Nguyen & 1 &0 &\href{works/LauLN08.pdf}{LauLN08}~\cite{LauLN08}\\
\rowlabel{auth:a274}Federico Barber & 1 &0 &\href{works/AbrilSB05.pdf}{AbrilSB05}~\cite{AbrilSB05}\\
\rowlabel{auth:a367}Ada Barlatt & 1 &1 &\href{works/BarlattCG08.pdf}{BarlattCG08}~\cite{BarlattCG08}\\
\rowlabel{auth:a528}Mohammadreza Barzegaran & 1 &0 &\href{works/BarzegaranZP20.pdf}{BarzegaranZP20}~\cite{BarzegaranZP20}\\
\rowlabel{auth:a525}Virginie Basini & 1 &8 &\href{works/Polo-MejiaALB20.pdf}{Polo-MejiaALB20}~\cite{Polo-MejiaALB20}\\
\rowlabel{auth:a944}Olga Battaïa & 1 &12 &\href{}{ArkhipovBL19}~\cite{ArkhipovBL19}\\
\rowlabel{auth:a795}N Beldiceanu & 1 &167 &\href{works/BeldiceanuC94.pdf}{BeldiceanuC94}~\cite{BeldiceanuC94}\\
\rowlabel{auth:a176}Said Belhadji & 1 &3 &\href{works/BelhadjiI98.pdf}{BelhadjiI98}~\cite{BelhadjiI98}\\
\rowlabel{auth:a389}Sana Belmokhtar & 1 &16 &\href{works/ArtiguesBF04.pdf}{ArtiguesBF04}~\cite{ArtiguesBF04}\\
\rowlabel{auth:a466}Fatima Benbouzid{-}Si Tayeb & 1 &0 &\href{works/TouatBT22.pdf}{TouatBT22}~\cite{TouatBT22}\\
\rowlabel{auth:a500}Till Bender & 1 &1 &\href{works/BenderWS21.pdf}{BenderWS21}~\cite{BenderWS21}\\
\rowlabel{auth:a465}Belaid Benhamou & 1 &0 &\href{works/TouatBT22.pdf}{TouatBT22}~\cite{TouatBT22}\\
\rowlabel{auth:a264}Hachemi Bennaceur & 1 &8 &\href{works/KhemmoudjPB06.pdf}{KhemmoudjPB06}~\cite{KhemmoudjPB06}\\
\rowlabel{auth:a173}E. Bensana & 1 &99 &\href{works/BensanaLV99.pdf}{BensanaLV99}~\cite{BensanaLV99}\\
\rowlabel{auth:a329}Russell Bent & 1 &4 &\href{works/LimBTBB15.pdf}{LimBTBB15}~\cite{LimBTBB15}\\
\rowlabel{auth:a357}Timo Berthold & 1 &28 &\href{works/BertholdHLMS10.pdf}{BertholdHLMS10}~\cite{BertholdHLMS10}\\
\rowlabel{auth:a334}Christian Bessiere & 1 &1 &\href{works/BessiereHMQW14.pdf}{BessiereHMQW14}~\cite{BessiereHMQW14}\\
\rowlabel{auth:a836}Julien Bidot & 1 &58 &\href{works/BidotVLB09.pdf}{BidotVLB09}~\cite{BidotVLB09}\\
\rowlabel{auth:a398}Arthur Bit{-}Monnot & 1 &0 &\href{works/Bit-Monnot23.pdf}{Bit-Monnot23}~\cite{Bit-Monnot23}\\
\rowlabel{auth:a775}Jacek Blazewicz & 1 &38 &\href{}{BlazewiczEP19}~\cite{BlazewiczEP19}\\
\rowlabel{auth:a646}Christian Blum & 1 &13 &\href{works/ThiruvadyBME09.pdf}{ThiruvadyBME09}~\cite{ThiruvadyBME09}\\
\rowlabel{auth:a640}Grzegorz Bocewicz & 1 &0 &\href{works/BocewiczBB09.pdf}{BocewiczBB09}~\cite{BocewiczBB09}\\
\rowlabel{auth:a718}Markus Bohlin & 1 &0 &\href{works/AronssonBK09.pdf}{AronssonBK09}~\cite{AronssonBK09}\\
\rowlabel{auth:a959}Peter Bongers & 1 &381 &\href{}{HarjunkoskiMBC14}~\cite{HarjunkoskiMBC14}\\
\rowlabel{auth:a714}Nicolas Bonifas & 1 &3 &\href{works/BaptisteB18.pdf}{BaptisteB18}~\cite{BaptisteB18}\\
\rowlabel{auth:a700}Eric Boucher & 1 &0 &\href{}{BoucherBVBL97}~\cite{BoucherBVBL97}\\
\rowlabel{auth:a34}Rapha{\"{e}}l Boudreault & 1 &0 &\href{works/BoudreaultSLQ22.pdf}{BoudreaultSLQ22}~\cite{BoudreaultSLQ22}\\
\rowlabel{auth:a682}Jean{-}Louis Bouquard & 1 &22 &\href{works/LorigeonBB02.pdf}{LorigeonBB02}~\cite{LorigeonBB02}\\
\rowlabel{auth:a448}Eric Bourreau & 1 &4 &\href{works/BourreauGGLT22.pdf}{BourreauGGLT22}~\cite{BourreauGGLT22}\\
\rowlabel{auth:a1026}Nadia Brauner & 1 &0 &\href{works/CatusseCBL16.pdf}{CatusseCBL16}~\cite{CatusseCBL16}\\
\rowlabel{auth:a705}Silvia Breitinger & 1 &0 &\href{}{BreitingerL95}~\cite{BreitingerL95}\\
\rowlabel{auth:a320}Kristen Brent Venable & 1 &1 &\href{works/GelainPRVW17.pdf}{GelainPRVW17}~\cite{GelainPRVW17}\\
\rowlabel{auth:a463}D. Brodart & 1 &1 &\href{works/OujanaAYB22.pdf}{OujanaAYB22}~\cite{OujanaAYB22}\\
\rowlabel{auth:a584}Yuriy Brun & 1 &9 &\href{works/ShinBBHO18.pdf}{ShinBBHO18}~\cite{ShinBBHO18}\\
\rowlabel{auth:a731}Vittorio Brusoni & 1 &1 &\href{works/BrusoniCLMMT96.pdf}{BrusoniCLMMT96}~\cite{BrusoniCLMMT96}\\
\rowlabel{auth:a105}Josef B{\"{u}}rgler & 1 &2 &\href{works/KoehlerBFFHPSSS21.pdf}{KoehlerBFFHPSSS21}~\cite{KoehlerBFFHPSSS21}\\
\rowlabel{auth:a998}Jacek Błażewicz & 1 &344 &\href{}{BlazewiczDP96}~\cite{BlazewiczDP96}\\
\rowlabel{auth:a171}Cristina C. B. Cavalcante & 1 &5 &\href{works/HeipckeCCS00.pdf}{HeipckeCCS00}~\cite{HeipckeCCS00}\\
\rowlabel{auth:a239}Lionel C. Briand & 1 &3 &\href{works/AlesioNBG14.pdf}{AlesioNBG14}~\cite{AlesioNBG14}\\
\rowlabel{auth:a276}Eugene C. Freuder & 1 &0 &\href{works/CarchraeBF05.pdf}{CarchraeBF05}~\cite{CarchraeBF05}\\
\rowlabel{auth:a604}Kevin C. Furman & 1 &48 &\href{works/GoelSHFS15.pdf}{GoelSHFS15}~\cite{GoelSHFS15}\\
\rowlabel{auth:a694}Joseph C. Pemberton & 1 &26 &\href{works/PembertonG98.pdf}{PembertonG98}~\cite{PembertonG98}\\
\rowlabel{auth:a706}Hendrik C. R. Lock & 1 &0 &\href{}{BreitingerL95}~\cite{BreitingerL95}\\
\rowlabel{auth:a748}Erich C. Teppan & 1 &3 &\href{works/ColT22.pdf}{ColT22}~\cite{ColT22}\\
\rowlabel{auth:a940}Matthew C. Gombolay & 1 &71 &\href{}{GombolayWS18}~\cite{GombolayWS18}\\
\rowlabel{auth:a889}Eray Cakici & 1 &50 &\href{works/HamC16.pdf}{HamC16}~\cite{HamC16}\\
\rowlabel{auth:a103}Louis{-}Pierre Campeau & 1 &0 &\href{works/CampeauG22.pdf}{CampeauG22}~\cite{CampeauG22}\\
\rowlabel{auth:a160}Cid Carvalho de Souza & 1 &31 &\href{works/LopesCSM10.pdf}{LopesCSM10}~\cite{LopesCSM10}\\
\rowlabel{auth:a304}Yves Caseau & 1 &0 &\href{works/Caseau97.pdf}{Caseau97}~\cite{Caseau97}\\
\rowlabel{auth:a844}Oscar Castillo & 1 &55 &\href{works/ZarandiASC20.pdf}{ZarandiASC20}~\cite{ZarandiASC20}\\
\rowlabel{auth:a1024}Nicolas Catusse & 1 &0 &\href{works/CatusseCBL16.pdf}{CatusseCBL16}~\cite{CatusseCBL16}\\
\rowlabel{auth:a591}Yao{-}Ting Chang & 1 &2 &\href{works/HoYCLLCLC18.pdf}{HoYCLLCLC18}~\cite{HoYCLLCLC18}\\
\rowlabel{auth:a350}Nicolas Chapados & 1 &5 &\href{works/ChapadosJR11.pdf}{ChapadosJR11}~\cite{ChapadosJR11}\\
\rowlabel{auth:a901}Philippe Charlier & 1 &11 &\href{works/SimonisCK00.pdf}{SimonisCK00}~\cite{SimonisCK00}\\
\rowlabel{auth:a932}Yarong Chen & 1 &2 &\href{works/ChenGPSH10.pdf}{ChenGPSH10}~\cite{ChenGPSH10}\\
\rowlabel{auth:a765}Mohammad Cherkaoui & 1 &0 &\href{works/FallahiAC20.pdf}{FallahiAC20}~\cite{FallahiAC20}\\
\rowlabel{auth:a596}Han{-}Mo Chiu & 1 &2 &\href{works/HoYCLLCLC18.pdf}{HoYCLLCLC18}~\cite{HoYCLLCLC18}\\
\rowlabel{auth:a24}Yeonjun Choi & 1 &0 &\href{works/KimCMLLP23.pdf}{KimCMLLP23}~\cite{KimCMLLP23}\\
\rowlabel{auth:a383}Yingyi Chu & 1 &13 &\href{works/ChuX05.pdf}{ChuX05}~\cite{ChuX05}\\
\rowlabel{auth:a594}Sue{-}Min Chu & 1 &2 &\href{works/HoYCLLCLC18.pdf}{HoYCLLCLC18}~\cite{HoYCLLCLC18}\\
\rowlabel{auth:a370}Hoong Chuin Lau & 1 &0 &\href{works/LauLN08.pdf}{LauLN08}~\cite{LauLN08}\\
\rowlabel{auth:a995}Italo Cipriano & 1 &0 &\href{}{HillBCGN22}~\cite{HillBCGN22}\\
\rowlabel{auth:a875}Michael Codish & 1 &127 &\href{works/OhrimenkoSC09.pdf}{OhrimenkoSC09}~\cite{OhrimenkoSC09}\\
\rowlabel{auth:a151}Carleton Coffrin & 1 &14 &\href{works/SchausHMCMD11.pdf}{SchausHMCMD11}~\cite{SchausHMCMD11}\\
\rowlabel{auth:a817}Eldan Cohen & 1 &1 &\href{works/CohenHB17.pdf}{CohenHB17}~\cite{CohenHB17}\\
\rowlabel{auth:a191}Jordi Coll & 1 &1 &\href{works/BofillCSV17.pdf}{BofillCSV17}~\cite{BofillCSV17}\\
\rowlabel{auth:a732}Luca Console & 1 &1 &\href{works/BrusoniCLMMT96.pdf}{BrusoniCLMMT96}~\cite{BrusoniCLMMT96}\\
\rowlabel{auth:a796}E Contejean & 1 &167 &\href{works/BeldiceanuC94.pdf}{BeldiceanuC94}~\cite{BeldiceanuC94}\\
\rowlabel{auth:a306}Trijntje Cornelissens & 1 &17 &\href{works/SimonisC95.pdf}{SimonisC95}~\cite{SimonisC95}\\
\rowlabel{auth:a288}Gabriella Cortellessa & 1 &8 &\href{works/OddiPCC03.pdf}{OddiPCC03}~\cite{OddiPCC03}\\
\rowlabel{auth:a414}Nicol{\'{a}}s Cuneo & 1 &0 &\href{works/YuraszeckMCCR23.pdf}{YuraszeckMCCR23}~\cite{YuraszeckMCCR23}\\
\rowlabel{auth:a741}Kateryna Czerniachowska & 1 &0 &\href{works/CzerniachowskaWZ23.pdf}{CzerniachowskaWZ23}~\cite{CzerniachowskaWZ23}\\
\rowlabel{auth:a39}Alain C{\^{o}}t{\'{e}} & 1 &0 &\href{works/PopovicCGNC22.pdf}{PopovicCGNC22}~\cite{PopovicCGNC22}\\
\rowlabel{auth:a194}Kenneth D. Young & 1 &6 &\href{works/YoungFS17.pdf}{YoungFS17}~\cite{YoungFS17}\\
\rowlabel{auth:a322}Laurent D. Michel & 1 &3 &\href{works/FontaineMH16.pdf}{FontaineMH16}~\cite{FontaineMH16}\\
\rowlabel{auth:a377}Steven D. Prestwich & 1 &6 &\href{works/RossiTHP07.pdf}{RossiTHP07}~\cite{RossiTHP07}\\
\rowlabel{auth:a780}Michael D. Moffitt & 1 &0 &\href{works/MoffittPP05.pdf}{MoffittPP05}~\cite{MoffittPP05}\\
\rowlabel{auth:a918}Jean Damay & 1 &3 &\href{}{NeronABCDD06}~\cite{NeronABCDD06}\\
\rowlabel{auth:a179}Ken Darby{-}Dowman & 1 &28 &\href{works/Darby-DowmanLMZ97.pdf}{Darby-DowmanLMZ97}~\cite{Darby-DowmanLMZ97}\\
\rowlabel{auth:a240}Vivian De Smedt & 1 &7 &\href{works/GaySS14.pdf}{GaySS14}~\cite{GaySS14}\\
\rowlabel{auth:a249}Alexis De Clercq & 1 &3 &\href{works/ClercqPBJ11.pdf}{ClercqPBJ11}~\cite{ClercqPBJ11}\\
\rowlabel{auth:a303}Rina Dechter & 1 &10 &\href{works/FrostD98.pdf}{FrostD98}~\cite{FrostD98}\\
\rowlabel{auth:a676}Carmelo Del Valle & 1 &7 &\href{works/ValleMGT03.pdf}{ValleMGT03}~\cite{ValleMGT03}\\
\rowlabel{auth:a793}Xavier Delorme & 1 &0 &\href{works/RodriguezDG02.pdf}{RodriguezDG02}~\cite{RodriguezDG02}\\
\rowlabel{auth:a710}Alain Demeure & 1 &0 &\href{}{JourdanFRD94}~\cite{JourdanFRD94}\\
\rowlabel{auth:a315}Emir Demirovic & 1 &4 &\href{works/DemirovicS18.pdf}{DemirovicS18}~\cite{DemirovicS18}\\
\rowlabel{auth:a197}Roberto Di Cosmo & 1 &0 &\href{works/LiuCGM17.pdf}{LiuCGM17}~\cite{LiuCGM17}\\
\rowlabel{auth:a379}Guido Diepen & 1 &2 &\href{works/AkkerDH07.pdf}{AkkerDH07}~\cite{AkkerDH07}\\
\rowlabel{auth:a270}Bistra Dilkina & 1 &2 &\href{works/DilkinaDH05.pdf}{DilkinaDH05}~\cite{DilkinaDH05}\\
\rowlabel{auth:a726}Mehmet Dincbas & 1 &86 &\href{works/DincbasSH90.pdf}{DincbasSH90}~\cite{DincbasSH90}\\
\rowlabel{auth:a981}Yann Disser & 1 &0 &\href{works/EmdeZD22.pdf}{EmdeZD22}~\cite{EmdeZD22}\\
\rowlabel{auth:a969}Alexandre Dolgui & 1 &2 &\href{}{NouriMHD23}~\cite{NouriMHD23}\\
\rowlabel{auth:a982}Ulrich Domdorf & 1 &0 &\href{}{DomdorfPH03}~\cite{DomdorfPH03}\\
\rowlabel{auth:a999}Wolfgang Domschke & 1 &344 &\href{}{BlazewiczDP96}~\cite{BlazewiczDP96}\\
\rowlabel{auth:a365}Gr{\'{e}}goire Dooms & 1 &1 &\href{works/DoomsH08.pdf}{DoomsH08}~\cite{DoomsH08}\\
\rowlabel{auth:a30}Agostino Dovier & 1 &0 &\href{works/TardivoDFMP23.pdf}{TardivoDFMP23}~\cite{TardivoDFMP23}\\
\rowlabel{auth:a517}Yuquan Du & 1 &27 &\href{works/QinDCS20.pdf}{QinDCS20}~\cite{QinDCS20}\\
\rowlabel{auth:a271}Lei Duan & 1 &2 &\href{works/DilkinaDH05.pdf}{DilkinaDH05}~\cite{DilkinaDH05}\\
\rowlabel{auth:a890}Alexandre Duarte {de Almeida} Lemos & 1 &0 &\href{works/Lemos21.pdf}{Lemos21}~\cite{Lemos21}\\
\rowlabel{auth:a268}Didier Dubois & 1 &13 &\href{works/FortinZDF05.pdf}{FortinZDF05}~\cite{FortinZDF05}\\
\rowlabel{auth:a374}Pierre Dupont & 1 &0 &\href{works/MonetteDD07.pdf}{MonetteDD07}~\cite{MonetteDD07}\\
\rowlabel{auth:a608}David Duvivier & 1 &36 &\href{works/WangMD15.pdf}{WangMD15}~\cite{WangMD15}\\
\rowlabel{auth:a209}Kyle E. C. Booth & 1 &21 &\href{works/BoothNB16.pdf}{BoothNB16}~\cite{BoothNB16}\\
\rowlabel{auth:a358}Marco E. L{\"{u}}bbecke & 1 &28 &\href{works/BertholdHLMS10.pdf}{BertholdHLMS10}~\cite{BertholdHLMS10}\\
\rowlabel{auth:a685}Andrew E. Santosa & 1 &0 &\href{works/ZhuS02.pdf}{ZhuS02}~\cite{ZhuS02}\\
\rowlabel{auth:a782}Martha E. Pollack & 1 &0 &\href{works/MoffittPP05.pdf}{MoffittPP05}~\cite{MoffittPP05}\\
\rowlabel{auth:a1004}Kyle E.C. Booth & 1 &24 &\href{}{RoshanaeiBAUB20}~\cite{RoshanaeiBAUB20}\\
\rowlabel{auth:a18}Nikolaos Efthymiou & 1 &0 &\href{works/EfthymiouY23.pdf}{EfthymiouY23}~\cite{EfthymiouY23}\\
\rowlabel{auth:a572}Gokhan Egilmez & 1 &43 &\href{works/GedikKEK18.pdf}{GedikKEK18}~\cite{GedikKEK18}\\
\rowlabel{auth:a280}P{\'{e}}ter Egri & 1 &2 &\href{works/KovacsEKV05.pdf}{KovacsEKV05}~\cite{KovacsEKV05}\\
\rowlabel{auth:a625}Nizar El Hachemi & 1 &32 &\href{works/HachemiGR11.pdf}{HachemiGR11}~\cite{HachemiGR11}\\
\rowlabel{auth:a654}Ghada El Khayat & 1 &84 &\href{works/KhayatLR06.pdf}{KhayatLR06}~\cite{KhayatLR06}\\
\rowlabel{auth:a763}Abdellah El Fallahi & 1 &0 &\href{works/FallahiAC20.pdf}{FallahiAC20}~\cite{FallahiAC20}\\
\rowlabel{auth:a952}\"{O}zg\"{u}n El\c{c}i & 1 &2 &\href{}{ElciOH22}~\cite{ElciOH22}\\
\rowlabel{auth:a979}Simon Emde & 1 &0 &\href{works/EmdeZD22.pdf}{EmdeZD22}~\cite{EmdeZD22}\\
\rowlabel{auth:a426}Ey{\"{u}}p Ensar Isik & 1 &0 &\href{works/IsikYA23.pdf}{IsikYA23}~\cite{IsikYA23}\\
\rowlabel{auth:a532}Teresa Escobet & 1 &17 &\href{works/EscobetPQPRA19.pdf}{EscobetPQPRA19}~\cite{EscobetPQPRA19}\\
\rowlabel{auth:a234}Joan Espasa & 1 &3 &\href{works/BofillEGPSV14.pdf}{BofillEGPSV14}~\cite{BofillEGPSV14}\\
\rowlabel{auth:a919}Alireza Etminaniesfahani & 1 &0 &\href{works/EtminaniesfahaniGNMS22.pdf}{EtminaniesfahaniGNMS22}~\cite{EtminaniesfahaniGNMS22}\\
\rowlabel{auth:a674}Michael F. Gorman & 1 &0 &\href{works/KanetAG04.pdf}{KanetAG04}~\cite{KanetAG04}\\
\rowlabel{auth:a974}Richard F. Hartl & 1 &24 &\href{works/SchnellH15.pdf}{SchnellH15}~\cite{SchnellH15}\\
\rowlabel{auth:a408}Mohd Fadlee A. Rasid & 1 &0 &\href{works/AkramNHRSA23.pdf}{AkramNHRSA23}~\cite{AkramNHRSA23}\\
\rowlabel{auth:a708}Fran{\c{c}}ois Fages & 1 &0 &\href{}{JourdanFRD94}~\cite{JourdanFRD94}\\
\rowlabel{auth:a697}Moreno Falaschi & 1 &10 &\href{works/FalaschiGMP97.pdf}{FalaschiGMP97}~\cite{FalaschiGMP97}\\
\rowlabel{auth:a483}Huali Fan & 1 &18 &\href{works/FanXG21.pdf}{FanXG21}~\cite{FanXG21}\\
\rowlabel{auth:a269}H{\'{e}}l{\`{e}}ne Fargier & 1 &13 &\href{works/FortinZDF05.pdf}{FortinZDF05}~\cite{FortinZDF05}\\
\rowlabel{auth:a744}Soroush Fatemi-Anaraki & 1 &7 &\href{}{Fatemi-AnarakiTFV23}~\cite{Fatemi-AnarakiTFV23}\\
\rowlabel{auth:a785}Filippo Focacci & 1 &0 &\href{works/FocacciLN00.pdf}{FocacciLN00}~\cite{FocacciLN00}\\
\rowlabel{auth:a321}Daniel Fontaine & 1 &3 &\href{works/FontaineMH16.pdf}{FontaineMH16}~\cite{FontaineMH16}\\
\rowlabel{auth:a106}Urs Fontana & 1 &2 &\href{works/KoehlerBFFHPSSS21.pdf}{KoehlerBFFHPSSS21}~\cite{KoehlerBFFHPSSS21}\\
\rowlabel{auth:a1010}M.A. Forbes & 1 &0 &\href{works/ForbesHJST24.pdf}{ForbesHJST24}~\cite{ForbesHJST24}\\
\rowlabel{auth:a31}Andrea Formisano & 1 &0 &\href{works/TardivoDFMP23.pdf}{TardivoDFMP23}~\cite{TardivoDFMP23}\\
\rowlabel{auth:a266}J{\'{e}}r{\^{o}}me Fortin & 1 &13 &\href{works/FortinZDF05.pdf}{FortinZDF05}~\cite{FortinZDF05}\\
\rowlabel{auth:a746}Mehdi Foumani & 1 &7 &\href{}{Fatemi-AnarakiTFV23}~\cite{Fatemi-AnarakiTFV23}\\
\rowlabel{auth:a612}Gerhard Friedrich & 1 &3 &\href{}{FriedrichFMRSST14}~\cite{FriedrichFMRSST14}\\
\rowlabel{auth:a95}Sara Frimodig & 1 &3 &\href{works/FrimodigS19.pdf}{FrimodigS19}~\cite{FrimodigS19}\\
\rowlabel{auth:a904}Aur{\'e}lien Froger & 1 &0 &\href{works/Froger16.pdf}{Froger16}~\cite{Froger16}\\
\rowlabel{auth:a544}Nikolaus Frohner & 1 &0 &\href{works/FrohnerTR19.pdf}{FrohnerTR19}~\cite{FrohnerTR19}\\
\rowlabel{auth:a302}Daniel Frost & 1 &10 &\href{works/FrostD98.pdf}{FrostD98}~\cite{FrostD98}\\
\rowlabel{auth:a613}Melanie Fr{\"{u}}hst{\"{u}}ck & 1 &3 &\href{}{FriedrichFMRSST14}~\cite{FriedrichFMRSST14}\\
\rowlabel{auth:a468}Jun Fu & 1 &0 &\href{works/LiFJZLL22.pdf}{LiFJZLL22}~\cite{LiFJZLL22}\\
\rowlabel{auth:a107}Etienne Fux & 1 &2 &\href{works/KoehlerBFFHPSSS21.pdf}{KoehlerBFFHPSSS21}~\cite{KoehlerBFFHPSSS21}\\
\rowlabel{auth:a513}Ernesto G. Birgin & 1 &30 &\href{works/LunardiBLRV20.pdf}{LunardiBLRV20}~\cite{LunardiBLRV20}\\
\rowlabel{auth:a40}Mohamed Gaha & 1 &0 &\href{works/PopovicCGNC22.pdf}{PopovicCGNC22}~\cite{PopovicCGNC22}\\
\rowlabel{auth:a695}Flavius Galiber III & 1 &26 &\href{works/PembertonG98.pdf}{PembertonG98}~\cite{PembertonG98}\\
\rowlabel{auth:a96}Cristian Galleguillos & 1 &1 &\href{works/GalleguillosKSB19.pdf}{GalleguillosKSB19}~\cite{GalleguillosKSB19}\\
\rowlabel{auth:a794}Xavier Gandibleux & 1 &0 &\href{works/RodriguezDG02.pdf}{RodriguezDG02}~\cite{RodriguezDG02}\\
\rowlabel{auth:a187}Graeme Gange & 1 &6 &\href{works/He0GLW18.pdf}{He0GLW18}~\cite{He0GLW18}\\
\rowlabel{auth:a449}Thierry Garaix & 1 &4 &\href{works/BourreauGGLT22.pdf}{BourreauGGLT22}~\cite{BourreauGGLT22}\\
\rowlabel{auth:a808}Maria Garcia de la Banda & 1 &24 &\href{works/BandaSC11.pdf}{BandaSC11}~\cite{BandaSC11}\\
\rowlabel{auth:a256}Antoine Gargani & 1 &17 &\href{works/GarganiR07.pdf}{GarganiR07}~\cite{GarganiR07}\\
\rowlabel{auth:a805}Serge Gaspers & 1 &0 &\href{works/ChuGNSW13.pdf}{ChuGNSW13}~\cite{ChuGNSW13}\\
\rowlabel{auth:a87}Jonathan Gaudreault & 1 &2 &\href{works/Mercier-AubinGQ20.pdf}{Mercier-AubinGQ20}~\cite{Mercier-AubinGQ20}\\
\rowlabel{auth:a570}Ridvan Gedik & 1 &43 &\href{works/GedikKEK18.pdf}{GedikKEK18}~\cite{GedikKEK18}\\
\rowlabel{auth:a47}Marc Geitz & 1 &0 &\href{works/GeitzGSSW22.pdf}{GeitzGSSW22}~\cite{GeitzGSSW22}\\
\rowlabel{auth:a317}Mirco Gelain & 1 &1 &\href{works/GelainPRVW17.pdf}{GelainPRVW17}~\cite{GelainPRVW17}\\
\rowlabel{auth:a626}Michel Gendreau & 1 &32 &\href{works/HachemiGR11.pdf}{HachemiGR11}~\cite{HachemiGR11}\\
\rowlabel{auth:a831}Wing{-}Yue Geoffrey Louie & 1 &16 &\href{works/LouieVNB14.pdf}{LouieVNB14}~\cite{LouieVNB14}\\
\rowlabel{auth:a443}Marcus Gerhard M{\"{u}}ller & 1 &17 &\href{works/MullerMKP22.pdf}{MullerMKP22}~\cite{MullerMKP22}\\
\rowlabel{auth:a490}Patrick Gerhards & 1 &0 &\href{works/HubnerGSV21.pdf}{HubnerGSV21}~\cite{HubnerGSV21}\\
\rowlabel{auth:a906}Grigori German & 1 &0 &\href{works/German18.pdf}{German18}~\cite{German18}\\
\rowlabel{auth:a667}Ulrich Geske & 1 &2 &\href{works/Geske05.pdf}{Geske05}~\cite{Geske05}\\
\rowlabel{auth:a1008}Shirin Ghasemi & 1 &0 &\href{}{GhasemiMH23}~\cite{GhasemiMH23}\\
\rowlabel{auth:a211}Katherine Giles & 1 &2 &\href{works/GilesH16.pdf}{GilesH16}~\cite{GilesH16}\\
\rowlabel{auth:a432}Ga{\"{e}}l Glorian & 1 &0 &\href{works/PerezGSL23.pdf}{PerezGSL23}~\cite{PerezGSL23}\\
\rowlabel{auth:a441}Gael Glorian & 1 &0 &\href{works/abs-2312-13682.pdf}{abs-2312-13682}~\cite{abs-2312-13682}\\
\rowlabel{auth:a783}Daniel Godard & 1 &0 &\href{works/GodardLN05.pdf}{GodardLN05}~\cite{GodardLN05}\\
\rowlabel{auth:a602}Vikas Goel & 1 &48 &\href{works/GoelSHFS15.pdf}{GoelSHFS15}~\cite{GoelSHFS15}\\
\rowlabel{auth:a485}Mark Goh & 1 &18 &\href{works/FanXG21.pdf}{FanXG21}~\cite{FanXG21}\\
\rowlabel{auth:a307}Hans{-}Joachim Goltz & 1 &7 &\href{works/Goltz95.pdf}{Goltz95}~\cite{Goltz95}\\
\rowlabel{auth:a450}Matthieu Gondran & 1 &4 &\href{works/BourreauGGLT22.pdf}{BourreauGGLT22}~\cite{BourreauGGLT22}\\
\rowlabel{auth:a987}Inés González-Rodríguez & 1 &0 &\href{}{AfsarVPG23}~\cite{AfsarVPG23}\\
\rowlabel{auth:a996}Marcos Goycoolea & 1 &0 &\href{}{HillBCGN22}~\cite{HillBCGN22}\\
\rowlabel{auth:a48}Cristian Grozea & 1 &0 &\href{works/GeitzGSSW22.pdf}{GeitzGSSW22}~\cite{GeitzGSSW22}\\
\rowlabel{auth:a696}Flavius Gruian & 1 &5 &\href{works/GruianK98.pdf}{GruianK98}~\cite{GruianK98}\\
\rowlabel{auth:a933}Zailin Guan & 1 &2 &\href{works/ChenGPSH10.pdf}{ChenGPSH10}~\cite{ChenGPSH10}\\
\rowlabel{auth:a382}Alessio Guerri & 1 &18 &\href{works/BeniniBGM06.pdf}{BeniniBGM06}~\cite{BeniniBGM06}\\
\rowlabel{auth:a363}Serigne Gueye & 1 &3 &\href{works/Acuna-AgostMFG09.pdf}{Acuna-AgostMFG09}~\cite{Acuna-AgostMFG09}\\
\rowlabel{auth:a610}Ying Guo & 1 &0 &\href{works/ZhouGL15.pdf}{ZhouGL15}~\cite{ZhouGL15}\\
\rowlabel{auth:a953}Peng Guo & 1 &8 &\href{}{GuoHLW20}~\cite{GuoHLW20}\\
\rowlabel{auth:a965}Penghui Guo & 1 &0 &\href{}{GuoZ23}~\cite{GuoZ23}\\
\rowlabel{auth:a1000}Olivier Guyon & 1 &32 &\href{works/GuyonLPR12.pdf}{GuyonLPR12}~\cite{GuyonLPR12}\\
\rowlabel{auth:a773}Şeyda G{\"u}r & 1 &0 &\href{works/GurEA19.pdf}{GurEA19}~\cite{GurEA19}\\
\rowlabel{auth:a579}Burak G{\"{o}}kg{\"{u}}r & 1 &31 &\href{works/GokgurHO18.pdf}{GokgurHO18}~\cite{GokgurHO18}\\
\rowlabel{auth:a418}Seyda G{\"{u}}r & 1 &1 &\href{works/GurPAE23.pdf}{GurPAE23}~\cite{GurPAE23}\\
\rowlabel{auth:a716}Fehmi H'Mida & 1 &11 &\href{works/TrojetHL11.pdf}{TrojetHL11}~\cite{TrojetHL11}\\
\rowlabel{auth:a359}Rolf H. M{\"{o}}hring & 1 &28 &\href{works/BertholdHLMS10.pdf}{BertholdHLMS10}~\cite{BertholdHLMS10}\\
\rowlabel{auth:a575}John H. Drake & 1 &41 &\href{works/PourDERB18.pdf}{PourDERB18}~\cite{PourDERB18}\\
\rowlabel{auth:a599}M. H. Fazel Zarandi & 1 &28 &\href{works/ZarandiKS16.pdf}{ZarandiKS16}~\cite{ZarandiKS16}\\
\rowlabel{auth:a776}Klaus H. Ecker & 1 &38 &\href{}{BlazewiczEP19}~\cite{BlazewiczEP19}\\
\rowlabel{auth:a787}Emile H. L. Aarts & 1 &0 &\href{works/NuijtenA94.pdf}{NuijtenA94}~\cite{NuijtenA94}\\
\rowlabel{auth:a924}Tarik Hadzic & 1 &3 &\href{works/SimonisH11.pdf}{SimonisH11}~\cite{SimonisH11}\\
\rowlabel{auth:a1009}Mahdi Hamid & 1 &0 &\href{}{GhasemiMH23}~\cite{GhasemiMH23}\\
\rowlabel{auth:a71}Claire Hanen & 1 &1 &\href{works/HanenKP21.pdf}{HanenKP21}~\cite{HanenKP21}\\
\rowlabel{auth:a518}Jiang Hang Chen & 1 &27 &\href{works/QinDCS20.pdf}{QinDCS20}~\cite{QinDCS20}\\
\rowlabel{auth:a260}Sue Hanhilammi & 1 &2 &\href{works/KrogtLPHJ07.pdf}{KrogtLPHJ07}~\cite{KrogtLPHJ07}\\
\rowlabel{auth:a968}Zdeněk Hanzálek & 1 &2 &\href{}{NouriMHD23}~\cite{NouriMHD23}\\
\rowlabel{auth:a356}Mohamed Haouari & 1 &3 &\href{works/LahimerLH11.pdf}{LahimerLH11}~\cite{LahimerLH11}\\
\rowlabel{auth:a1011}M.G. Harris & 1 &0 &\href{works/ForbesHJST24.pdf}{ForbesHJST24}~\cite{ForbesHJST24}\\
\rowlabel{auth:a407}Fazirulhisyam Hashim & 1 &0 &\href{works/AkramNHRSA23.pdf}{AkramNHRSA23}~\cite{AkramNHRSA23}\\
\rowlabel{auth:a936}Muhammad Hasseb & 1 &2 &\href{works/ChenGPSH10.pdf}{ChenGPSH10}~\cite{ChenGPSH10}\\
\rowlabel{auth:a186}Shan He & 1 &6 &\href{works/He0GLW18.pdf}{He0GLW18}~\cite{He0GLW18}\\
\rowlabel{auth:a954}Xun He & 1 &8 &\href{}{GuoHLW20}~\cite{GuoHLW20}\\
\rowlabel{auth:a835}Ivan Heckman & 1 &0 &\href{works/HeckmanB11.pdf}{HeckmanB11}~\cite{HeckmanB11}\\
\rowlabel{auth:a169}Susanne Heipcke & 1 &5 &\href{works/HeipckeCCS00.pdf}{HeipckeCCS00}~\cite{HeipckeCCS00}\\
\rowlabel{auth:a245}Fabien Hermenier & 1 &28 &\href{works/HermenierDL11.pdf}{HermenierDL11}~\cite{HermenierDL11}\\
\rowlabel{auth:a346}Gerhard Hiermann & 1 &14 &\href{works/RendlPHPR12.pdf}{RendlPHPR12}~\cite{RendlPHPR12}\\
\rowlabel{auth:a589}Te{-}Wei Ho & 1 &2 &\href{works/HoYCLLCLC18.pdf}{HoYCLLCLC18}~\cite{HoYCLLCLC18}\\
\rowlabel{auth:a553}Petra Hofstedt & 1 &1 &\href{works/LiuLH19.pdf}{LiuLH19}~\cite{LiuLH19}\\
\rowlabel{auth:a1023}Mark{\'{o}} Horv{\'{a}}th & 1 &5 &\href{works/NattafHKAL19.pdf}{NattafHKAL19}~\cite{NattafHKAL19}\\
\rowlabel{auth:a841}Mohammad Hossein Fazel Zarandi & 1 &55 &\href{works/ZarandiASC20.pdf}{ZarandiASC20}~\cite{ZarandiASC20}\\
\rowlabel{auth:a255}John Hou & 1 &1 &\href{works/DavenportKRSH07.pdf}{DavenportKRSH07}~\cite{DavenportKRSH07}\\
\rowlabel{auth:a818}Guoyu Huang & 1 &1 &\href{works/CohenHB17.pdf}{CohenHB17}~\cite{CohenHB17}\\
\rowlabel{auth:a900}Barry Hurley & 1 &0 &\href{works/HurleyOS16.pdf}{HurleyOS16}~\cite{HurleyOS16}\\
\rowlabel{auth:a489}Felix H{\"{u}}bner & 1 &0 &\href{works/HubnerGSV21.pdf}{HubnerGSV21}~\cite{HubnerGSV21}\\
\rowlabel{auth:a970}Ayoub Insa Corréa & 1 &106 &\href{}{CorreaLR07}~\cite{CorreaLR07}\\
\rowlabel{auth:a177}Amar Isli & 1 &3 &\href{works/BelhadjiI98.pdf}{BelhadjiI98}~\cite{BelhadjiI98}\\
\rowlabel{auth:a409}Mustafa Ismael Salman & 1 &0 &\href{works/AkramNHRSA23.pdf}{AkramNHRSA23}~\cite{AkramNHRSA23}\\
\rowlabel{auth:a244}Fernando J. M. Marcellino & 1 &0 &\href{works/SerraNM12.pdf}{SerraNM12}~\cite{SerraNM12}\\
\rowlabel{auth:a587}Leon J. Osterweil & 1 &9 &\href{works/ShinBBHO18.pdf}{ShinBBHO18}~\cite{ShinBBHO18}\\
\rowlabel{auth:a660}H. J. Kim & 1 &12 &\href{works/SureshMOK06.pdf}{SureshMOK06}~\cite{SureshMOK06}\\
\rowlabel{auth:a672}John J. Kanet & 1 &0 &\href{works/KanetAG04.pdf}{KanetAG04}~\cite{KanetAG04}\\
\rowlabel{auth:a680}Colin J. Layfield & 1 &0 &\href{works/Layfield02.pdf}{Layfield02}~\cite{Layfield02}\\
\rowlabel{auth:a689}Andrew J. Mason & 1 &5 &\href{works/Mason01.pdf}{Mason01}~\cite{Mason01}\\
\rowlabel{auth:a909}Steven J. Edwards & 1 &3 &\href{}{EdwardsBSE19}~\cite{EdwardsBSE19}\\
\rowlabel{auth:a941}Ronald J. Wilcox & 1 &71 &\href{}{GombolayWS18}~\cite{GombolayWS18}\\
\rowlabel{auth:a994}Andrea J. Brickey & 1 &0 &\href{}{HillBCGN22}~\cite{HillBCGN22}\\
\rowlabel{auth:a857}Vipul Jain & 1 &279 &\href{works/JainG01.pdf}{JainG01}~\cite{JainG01}\\
\rowlabel{auth:a977}A.S. Jain & 1 &490 &\href{}{JainM99}~\cite{JainM99}\\
\rowlabel{auth:a1012}H.M. Jansen & 1 &0 &\href{works/ForbesHJST24.pdf}{ForbesHJST24}~\cite{ForbesHJST24}\\
\rowlabel{auth:a225}Jean Jaubert & 1 &0 &\href{works/PraletLJ15.pdf}{PraletLJ15}~\cite{PraletLJ15}\\
\rowlabel{auth:a789}Jan Jel{\'{\i}}nek & 1 &0 &\href{works/JelinekB16.pdf}{JelinekB16}~\cite{JelinekB16}\\
\rowlabel{auth:a474}Yingjun Ji & 1 &0 &\href{works/ZhangJZL22.pdf}{ZhangJZL22}~\cite{ZhangJZL22}\\
\rowlabel{auth:a469}Zixi Jia & 1 &0 &\href{works/LiFJZLL22.pdf}{LiFJZLL22}~\cite{LiFJZLL22}\\
\rowlabel{auth:a665}Yunfei Jiang & 1 &0 &\href{works/LiuJ06.pdf}{LiuJ06}~\cite{LiuJ06}\\
\rowlabel{auth:a261}Yue Jin & 1 &2 &\href{works/KrogtLPHJ07.pdf}{KrogtLPHJ07}~\cite{KrogtLPHJ07}\\
\rowlabel{auth:a351}Marc Joliveau & 1 &5 &\href{works/ChapadosJR11.pdf}{ChapadosJR11}~\cite{ChapadosJR11}\\
\rowlabel{auth:a299}Peter Jonsson & 1 &1 &\href{works/AngelsmarkJ00.pdf}{AngelsmarkJ00}~\cite{AngelsmarkJ00}\\
\rowlabel{auth:a986}Juan José Palacios & 1 &0 &\href{}{AfsarVPG23}~\cite{AfsarVPG23}\\
\rowlabel{auth:a949}Antoine Jouglet & 1 &3 &\href{}{CarlierSJP21}~\cite{CarlierSJP21}\\
\rowlabel{auth:a707}Jean Jourdan & 1 &0 &\href{}{JourdanFRD94}~\cite{JourdanFRD94}\\
\rowlabel{auth:a803}Nicolas Jozefowiez & 1 &9 &\href{works/HebrardHJMPV16.pdf}{HebrardHJMPV16}~\cite{HebrardHJMPV16}\\
\rowlabel{auth:a556}Jae{-}Yoon Jung & 1 &1 &\href{works/ParkUJR19.pdf}{ParkUJR19}~\cite{ParkUJR19}\\
\rowlabel{auth:a750}Pascal Jungblut & 1 &0 &\href{works/JungblutK22.pdf}{JungblutK22}~\cite{JungblutK22}\\
\rowlabel{auth:a289}T. K. Satish Kumar & 1 &4 &\href{works/Kumar03.pdf}{Kumar03}~\cite{Kumar03}\\
\rowlabel{auth:a578}Edmund K. Burke & 1 &41 &\href{works/PourDERB18.pdf}{PourDERB18}~\cite{PourDERB18}\\
\rowlabel{auth:a832}Mustafa K. Dogru & 1 &8 &\href{works/TerekhovDOB12.pdf}{TerekhovDOB12}~\cite{TerekhovDOB12}\\
\rowlabel{auth:a834}T. K. Feng & 1 &43 &\href{works/BeckFW11.pdf}{BeckFW11}~\cite{BeckFW11}\\
\rowlabel{auth:a252}Jayant Kalagnanam & 1 &1 &\href{works/DavenportKRSH07.pdf}{DavenportKRSH07}~\cite{DavenportKRSH07}\\
\rowlabel{auth:a571}Darshan Kalathia & 1 &43 &\href{works/GedikKEK18.pdf}{GedikKEK18}~\cite{GedikKEK18}\\
\rowlabel{auth:a293}Olli Kamarainen & 1 &9 &\href{works/KamarainenS02.pdf}{KamarainenS02}~\cite{KamarainenS02}\\
\rowlabel{auth:a406}Nor Kamariah Noordin & 1 &0 &\href{works/AkramNHRSA23.pdf}{AkramNHRSA23}~\cite{AkramNHRSA23}\\
\rowlabel{auth:a902}Philip Kay & 1 &11 &\href{works/SimonisCK00.pdf}{SimonisCK00}~\cite{SimonisCK00}\\
\rowlabel{auth:a338}Elena Kelareva & 1 &16 &\href{works/KelarevaTK13.pdf}{KelarevaTK13}~\cite{KelarevaTK13}\\
\rowlabel{auth:a628}Jan Kelbel & 1 &12 &\href{works/KelbelH11.pdf}{KelbelH11}~\cite{KelbelH11}\\
\rowlabel{auth:a600}H. Khorshidian & 1 &28 &\href{works/ZarandiKS16.pdf}{ZarandiKS16}~\cite{ZarandiKS16}\\
\rowlabel{auth:a770}Kamran Kianfar & 1 &0 &\href{works/YounespourAKE19.pdf}{YounespourAKE19}~\cite{YounespourAKE19}\\
\rowlabel{auth:a340}Philip Kilby & 1 &16 &\href{works/KelarevaTK13.pdf}{KelarevaTK13}~\cite{KelarevaTK13}\\
\rowlabel{auth:a23}Dongyun Kim & 1 &0 &\href{works/KimCMLLP23.pdf}{KimCMLLP23}~\cite{KimCMLLP23}\\
\rowlabel{auth:a573}Emre Kirac & 1 &43 &\href{works/GedikKEK18.pdf}{GedikKEK18}~\cite{GedikKEK18}\\
\rowlabel{auth:a97}Zeynep Kiziltan & 1 &1 &\href{works/GalleguillosKSB19.pdf}{GalleguillosKSB19}~\cite{GalleguillosKSB19}\\
\rowlabel{auth:a67}Christian Klanke & 1 &3 &\href{works/KlankeBYE21.pdf}{KlankeBYE21}~\cite{KlankeBYE21}\\
\rowlabel{auth:a104}Jana Koehler & 1 &2 &\href{works/KoehlerBFFHPSSS21.pdf}{KoehlerBFFHPSSS21}~\cite{KoehlerBFFHPSSS21}\\
\rowlabel{auth:a59}Wolfgang Kohlenbrein & 1 &0 &\href{works/KovacsTKSG21.pdf}{KovacsTKSG21}~\cite{KovacsTKSG21}\\
\rowlabel{auth:a447}Rainer Kolisch & 1 &4 &\href{works/PohlAK22.pdf}{PohlAK22}~\cite{PohlAK22}\\
\rowlabel{auth:a333}Sebastian Kosch & 1 &4 &\href{works/KoschB14.pdf}{KoschB14}~\cite{KoschB14}\\
\rowlabel{auth:a57}Benjamin Kov{\'{a}}cs & 1 &0 &\href{works/KovacsTKSG21.pdf}{KovacsTKSG21}~\cite{KovacsTKSG21}\\
\rowlabel{auth:a79}Matthias Krainz & 1 &0 &\href{works/GeibingerKKMMW21.pdf}{GeibingerKKMMW21}~\cite{GeibingerKKMMW21}\\
\rowlabel{auth:a713}Andreas Krall & 1 &14 &\href{works/ErtlK91.pdf}{ErtlK91}~\cite{ErtlK91}\\
\rowlabel{auth:a751}Dieter Kranzlm{\"{u}}ller & 1 &0 &\href{works/JungblutK22.pdf}{JungblutK22}~\cite{JungblutK22}\\
\rowlabel{auth:a444}Dominik Kress & 1 &17 &\href{works/MullerMKP22.pdf}{MullerMKP22}~\cite{MullerMKP22}\\
\rowlabel{auth:a719}Per Kreuger & 1 &0 &\href{works/AronssonBK09.pdf}{AronssonBK09}~\cite{AronssonBK09}\\
\rowlabel{auth:a772}Mustafa K{\"u}ç{\"u}k & 1 &0 &\href{works/KucukY19.pdf}{KucukY19}~\cite{KucukY19}\\
\rowlabel{auth:a386}Elif K{\"{u}}rkl{\"{u}} & 1 &4 &\href{works/FrankK05.pdf}{FrankK05}~\cite{FrankK05}\\
\rowlabel{auth:a373}Andr{\'{a}}s K{\'{e}}ri & 1 &1 &\href{works/KeriK07.pdf}{KeriK07}~\cite{KeriK07}\\
\rowlabel{auth:a28}Michael L. Pinedo & 1 &0 &\href{works/KimCMLLP23.pdf}{KimCMLLP23}~\cite{KimCMLLP23}\\
\rowlabel{auth:a214}Hassan L. Hijazi & 1 &2 &\href{works/LimHTB16.pdf}{LimHTB16}~\cite{LimHTB16}\\
\rowlabel{auth:a586}Philip L. Henneman & 1 &9 &\href{works/ShinBBHO18.pdf}{ShinBBHO18}~\cite{ShinBBHO18}\\
\rowlabel{auth:a755}Yiqing L. Luo & 1 &0 &\href{works/LuoB22.pdf}{LuoB22}~\cite{LuoB22}\\
\rowlabel{auth:a451}Philippe Lacomme & 1 &4 &\href{works/BourreauGGLT22.pdf}{BourreauGGLT22}~\cite{BourreauGGLT22}\\
\rowlabel{auth:a36}Daniel Lafond & 1 &0 &\href{works/BoudreaultSLQ22.pdf}{BoudreaultSLQ22}~\cite{BoudreaultSLQ22}\\
\rowlabel{auth:a1028}Anne{-}Marie Lagrange & 1 &0 &\href{works/CatusseCBL16.pdf}{CatusseCBL16}~\cite{CatusseCBL16}\\
\rowlabel{auth:a355}Asma Lahimer & 1 &3 &\href{works/LahimerLH11.pdf}{LahimerLH11}~\cite{LahimerLH11}\\
\rowlabel{auth:a592}Feipei Lai & 1 &2 &\href{works/HoYCLLCLC18.pdf}{HoYCLLCLC18}~\cite{HoYCLLCLC18}\\
\rowlabel{auth:a593}Jui{-}Fen Lai & 1 &2 &\href{works/HoYCLLCLC18.pdf}{HoYCLLCLC18}~\cite{HoYCLLCLC18}\\
\rowlabel{auth:a971}André Langevin & 1 &106 &\href{}{CorreaLR07}~\cite{CorreaLR07}\\
\rowlabel{auth:a945}Alexander Lazarev & 1 &12 &\href{}{ArkhipovBL19}~\cite{ArkhipovBL19}\\
\rowlabel{auth:a219}Christophe Lecoutre & 1 &20 &\href{works/GayHLS15.pdf}{GayHLS15}~\cite{GayHLS15}\\
\rowlabel{auth:a26}Myungho Lee & 1 &0 &\href{works/KimCMLLP23.pdf}{KimCMLLP23}~\cite{KimCMLLP23}\\
\rowlabel{auth:a27}Kangbok Lee & 1 &0 &\href{works/KimCMLLP23.pdf}{KimCMLLP23}~\cite{KimCMLLP23}\\
\rowlabel{auth:a224}Solange Lemai{-}Chenevier & 1 &0 &\href{works/PraletLJ15.pdf}{PraletLJ15}~\cite{PraletLJ15}\\
\rowlabel{auth:a467}Xingyang Li & 1 &0 &\href{works/LiFJZLL22.pdf}{LiFJZLL22}~\cite{LiFJZLL22}\\
\rowlabel{auth:a471}Siyi Li & 1 &0 &\href{works/LiFJZLL22.pdf}{LiFJZLL22}~\cite{LiFJZLL22}\\
\rowlabel{auth:a475}Xiaodong Li & 1 &0 &\href{works/abs-2211-14492.pdf}{abs-2211-14492}~\cite{abs-2211-14492}\\
\rowlabel{auth:a611}Guipeng Li & 1 &0 &\href{works/ZhouGL15.pdf}{ZhouGL15}~\cite{ZhouGL15}\\
\rowlabel{auth:a634}Hong Li & 1 &4 &\href{works/SunLYL10.pdf}{SunLYL10}~\cite{SunLYL10}\\
\rowlabel{auth:a636}Nan Li & 1 &4 &\href{works/SunLYL10.pdf}{SunLYL10}~\cite{SunLYL10}\\
\rowlabel{auth:a723}Yunbo Li & 1 &1 &\href{works/Madi-WambaLOBM17.pdf}{Madi-WambaLOBM17}~\cite{Madi-WambaLOBM17}\\
\rowlabel{auth:a814}Heyse Li & 1 &8 &\href{works/TranPZLDB18.pdf}{TranPZLDB18}~\cite{TranPZLDB18}\\
\rowlabel{auth:a827}Yi Li & 1 &0 &\href{works/LuoVLBM16.pdf}{LuoVLBM16}~\cite{LuoVLBM16}\\
\rowlabel{auth:a975}Haitao Li & 1 &113 &\href{works/LiW08.pdf}{LiW08}~\cite{LiW08}\\
\rowlabel{auth:a595}Wan{-}Chung Liao & 1 &2 &\href{works/HoYCLLCLC18.pdf}{HoYCLLCLC18}~\cite{HoYCLLCLC18}\\
\rowlabel{auth:a188}Ariel Liebman & 1 &6 &\href{works/He0GLW18.pdf}{He0GLW18}~\cite{He0GLW18}\\
\rowlabel{auth:a649}Olivier Liess & 1 &22 &\href{works/LiessM08.pdf}{LiessM08}~\cite{LiessM08}\\
\rowlabel{auth:a282}Andrew Lim & 1 &5 &\href{works/LimRX04.pdf}{LimRX04}~\cite{LimRX04}\\
\rowlabel{auth:a196}Tong Liu & 1 &0 &\href{works/LiuCGM17.pdf}{LiuCGM17}~\cite{LiuCGM17}\\
\rowlabel{auth:a496}Lingxuan Liu & 1 &12 &\href{works/QinWSLS21.pdf}{QinWSLS21}~\cite{QinWSLS21}\\
\rowlabel{auth:a551}Ke Liu & 1 &1 &\href{works/LiuLH19.pdf}{LiuLH19}~\cite{LiuLH19}\\
\rowlabel{auth:a566}Rengkui Liu & 1 &24 &\href{works/TangLWSK18.pdf}{TangLWSK18}~\cite{TangLWSK18}\\
\rowlabel{auth:a664}Yuechang Liu & 1 &0 &\href{works/LiuJ06.pdf}{LiuJ06}~\cite{LiuJ06}\\
\rowlabel{auth:a810}Giovanni Lo Bianco & 1 &0 &\href{works/ZhangBB22.pdf}{ZhangBB22}~\cite{ZhangBB22}\\
\rowlabel{auth:a550}Doina Logofatu & 1 &2 &\href{works/BadicaBIL19.pdf}{BadicaBIL19}~\cite{BadicaBIL19}\\
\rowlabel{auth:a681}Thomas Lorigeon & 1 &22 &\href{works/LorigeonBB02.pdf}{LorigeonBB02}~\cite{LorigeonBB02}\\
\rowlabel{auth:a955}Yulin Luan & 1 &8 &\href{}{GuoHLW20}~\cite{GuoHLW20}\\
\rowlabel{auth:a825}Roy Luo & 1 &0 &\href{works/LuoVLBM16.pdf}{LuoVLBM16}~\cite{LuoVLBM16}\\
\rowlabel{auth:a797}Arnaud Lusson & 1 &0 &\href{works/HebrardALLCMR22.pdf}{HebrardALLCMR22}~\cite{HebrardALLCMR22}\\
\rowlabel{auth:a511}Chang Lv & 1 &100 &\href{works/MengZRZL20.pdf}{MengZRZL20}~\cite{MengZRZL20}\\
\rowlabel{auth:a622}Zhimin Lv & 1 &1 &\href{works/ZhangLS12.pdf}{ZhangLS12}~\cite{ZhangLS12}\\
\rowlabel{auth:a552}Sven L{\"{o}}ffler & 1 &1 &\href{works/LiuLH19.pdf}{LiuLH19}~\cite{LiuLH19}\\
\rowlabel{auth:a378}J. M. van den Akker & 1 &2 &\href{works/AkkerDH07.pdf}{AkkerDH07}~\cite{AkkerDH07}\\
\rowlabel{auth:a410}Abdulrahman M. Abdulghani & 1 &0 &\href{works/AkramNHRSA23.pdf}{AkramNHRSA23}~\cite{AkramNHRSA23}\\
\rowlabel{auth:a558}O. M. Alade & 1 &0 &\href{works/abs-1902-01193.pdf}{abs-1902-01193}~\cite{abs-1902-01193}\\
\rowlabel{auth:a574}Shahrzad M. Pour & 1 &41 &\href{works/PourDERB18.pdf}{PourDERB18}~\cite{PourDERB18}\\
\rowlabel{auth:a597}Franco M. Novara & 1 &18 &\href{works/NovaraNH16.pdf}{NovaraNH16}~\cite{NovaraNH16}\\
\rowlabel{auth:a678}Rafael M. Gasca & 1 &7 &\href{works/ValleMGT03.pdf}{ValleMGT03}~\cite{ValleMGT03}\\
\rowlabel{auth:a846}Jose M. Framinan & 1 &0 &\href{}{AbreuPNF23}~\cite{AbreuPNF23}\\
\rowlabel{auth:a888}Andy M. Ham & 1 &50 &\href{works/HamC16.pdf}{HamC16}~\cite{HamC16}\\
\rowlabel{auth:a967}Mohammad M. Fazel-Zarandi & 1 &38 &\href{}{ZarandiB12}~\cite{ZarandiB12}\\
\rowlabel{auth:a638}Jun Ma & 1 &1 &\href{works/MakMS10.pdf}{MakMS10}~\cite{MakMS10}\\
\rowlabel{auth:a368}Amy Mainville Cohn & 1 &1 &\href{works/BarlattCG08.pdf}{BarlattCG08}~\cite{BarlattCG08}\\
\rowlabel{auth:a637}Kai{-}Ling Mak & 1 &1 &\href{works/MakMS10.pdf}{MakMS10}~\cite{MakMS10}\\
\rowlabel{auth:a658}V. Mani & 1 &12 &\href{works/SureshMOK06.pdf}{SureshMOK06}~\cite{SureshMOK06}\\
\rowlabel{auth:a222}Oscar Manzano & 1 &1 &\href{works/MurphyMB15.pdf}{MurphyMB15}~\cite{MurphyMB15}\\
\rowlabel{auth:a925}Christos Maravelias & 1 &0 &\href{}{AggounMV08}~\cite{AggounMV08}\\
\rowlabel{auth:a577}Kourosh Marjani Rasmussen & 1 &41 &\href{works/PourDERB18.pdf}{PourDERB18}~\cite{PourDERB18}\\
\rowlabel{auth:a698}Kim Marriott & 1 &10 &\href{works/FalaschiGMP97.pdf}{FalaschiGMP97}~\cite{FalaschiGMP97}\\
\rowlabel{auth:a686}Fae Martin & 1 &11 &\href{works/MartinPY01.pdf}{MartinPY01}~\cite{MartinPY01}\\
\rowlabel{auth:a651}Jim McInnes & 1 &15 &\href{works/MalikMB08.pdf}{MalikMB08}~\cite{MalikMB08}\\
\rowlabel{auth:a978}S. Meeran & 1 &490 &\href{}{JainM99}~\cite{JainM99}\\
\rowlabel{auth:a435}Zahra Mehdizadeh{-}Somarin & 1 &0 &\href{works/Mehdizadeh-Somarin23.pdf}{Mehdizadeh-Somarin23}~\cite{Mehdizadeh-Somarin23}\\
\rowlabel{auth:a420}Haci Mehmet Alakas & 1 &1 &\href{works/GurPAE23.pdf}{GurPAE23}~\cite{GurPAE23}\\
\rowlabel{auth:a774}Hacı Mehmet Alakaş & 1 &0 &\href{works/GurEA19.pdf}{GurEA19}~\cite{GurEA19}\\
\rowlabel{auth:a44}Sebastian Meiswinkel & 1 &0 &\href{works/WinterMMW22.pdf}{WinterMMW22}~\cite{WinterMMW22}\\
\rowlabel{auth:a752}Gonzalo Mejía & 1 &6 &\href{works/YuraszeckMPV22.pdf}{YuraszeckMPV22}~\cite{YuraszeckMPV22}\\
\rowlabel{auth:a203}Hein Meling & 1 &6 &\href{works/MossigeGSMC17.pdf}{MossigeGSMC17}~\cite{MossigeGSMC17}\\
\rowlabel{auth:a624}Julien Menana & 1 &0 &\href{works/Menana11.pdf}{Menana11}~\cite{Menana11}\\
\rowlabel{auth:a725}Jean{-}Marc Menaud & 1 &1 &\href{works/Madi-WambaLOBM17.pdf}{Madi-WambaLOBM17}~\cite{Madi-WambaLOBM17}\\
\rowlabel{auth:a507}Leilei Meng & 1 &100 &\href{works/MengZRZL20.pdf}{MengZRZL20}~\cite{MengZRZL20}\\
\rowlabel{auth:a864}Luc Mercier & 1 &32 &\href{works/MercierH08.pdf}{MercierH08}~\cite{MercierH08}\\
\rowlabel{auth:a86}Alexandre Mercier{-}Aubin & 1 &2 &\href{works/Mercier-AubinGQ20.pdf}{Mercier-AubinGQ20}~\cite{Mercier-AubinGQ20}\\
\rowlabel{auth:a614}Vera Mersheeva & 1 &3 &\href{}{FriedrichFMRSST14}~\cite{FriedrichFMRSST14}\\
\rowlabel{auth:a607}Nadine Meskens & 1 &36 &\href{works/WangMD15.pdf}{WangMD15}~\cite{WangMD15}\\
\rowlabel{auth:a647}Bernd Meyer & 1 &13 &\href{works/ThiruvadyBME09.pdf}{ThiruvadyBME09}~\cite{ThiruvadyBME09}\\
\rowlabel{auth:a762}Kyung Min Kim & 1 &0 &\href{works/HamPK21.pdf}{HamPK21}~\cite{HamPK21}\\
\rowlabel{auth:a181}Gautam Mitra & 1 &28 &\href{works/Darby-DowmanLMZ97.pdf}{Darby-DowmanLMZ97}~\cite{Darby-DowmanLMZ97}\\
\rowlabel{auth:a412}Elizabeth Montero & 1 &0 &\href{works/YuraszeckMCCR23.pdf}{YuraszeckMCCR23}~\cite{YuraszeckMCCR23}\\
\rowlabel{auth:a25}Kyungduk Moon & 1 &0 &\href{works/KimCMLLP23.pdf}{KimCMLLP23}~\cite{KimCMLLP23}\\
\rowlabel{auth:a920}Leila Moslemi Naeni & 1 &0 &\href{works/EtminaniesfahaniGNMS22.pdf}{EtminaniesfahaniGNMS22}~\cite{EtminaniesfahaniGNMS22}\\
\rowlabel{auth:a200}Morten Mossige & 1 &6 &\href{works/MossigeGSMC17.pdf}{MossigeGSMC17}~\cite{MossigeGSMC17}\\
\rowlabel{auth:a72}Alix Munier Kordon & 1 &1 &\href{works/HanenKP21.pdf}{HanenKP21}~\cite{HanenKP21}\\
\rowlabel{auth:a100}Stanislav Mur{\'{\i}}n & 1 &2 &\href{works/MurinR19.pdf}{MurinR19}~\cite{MurinR19}\\
\rowlabel{auth:a292}Nicola Muscettola & 1 &14 &\href{works/Muscettola02.pdf}{Muscettola02}~\cite{Muscettola02}\\
\rowlabel{auth:a442}David M{\"{u}}ller & 1 &17 &\href{works/MullerMKP22.pdf}{MullerMKP22}~\cite{MullerMKP22}\\
\rowlabel{auth:a297}Andr{\'{a}}s M{\'{a}}rkus & 1 &2 &\href{works/VanczaM01.pdf}{VanczaM01}~\cite{VanczaM01}\\
\rowlabel{auth:a335}Marc{-}Andr{\'{e}} M{\'{e}}nard & 1 &1 &\href{works/BessiereHMQW14.pdf}{BessiereHMQW14}~\cite{BessiereHMQW14}\\
\rowlabel{auth:a960}Carlos Méndez & 1 &381 &\href{}{HarjunkoskiMBC14}~\cite{HarjunkoskiMBC14}\\
\rowlabel{auth:a488}T. N. Wong & 1 &6 &\href{works/ZhangYW21.pdf}{ZhangYW21}~\cite{ZhangYW21}\\
\rowlabel{auth:a659}S. N. Omkar & 1 &12 &\href{works/SureshMOK06.pdf}{SureshMOK06}~\cite{SureshMOK06}\\
\rowlabel{auth:a806}Nina Narodytska & 1 &0 &\href{works/ChuGNSW13.pdf}{ChuGNSW13}~\cite{ChuGNSW13}\\
\rowlabel{auth:a238}Shiva Nejati & 1 &3 &\href{works/AlesioNBG14.pdf}{AlesioNBG14}~\cite{AlesioNBG14}\\
\rowlabel{auth:a997}Alexandra Newman & 1 &0 &\href{}{HillBCGN22}~\cite{HillBCGN22}\\
\rowlabel{auth:a41}Franklin Nguewouo & 1 &0 &\href{works/PopovicCGNC22.pdf}{PopovicCGNC22}~\cite{PopovicCGNC22}\\
\rowlabel{auth:a243}Gilberto Nishioka & 1 &0 &\href{works/SerraNM12.pdf}{SerraNM12}~\cite{SerraNM12}\\
\rowlabel{auth:a12}Thierry Noulamo & 1 &0 &\href{works/KameugneFND23.pdf}{KameugneFND23}~\cite{KameugneFND23}\\
\rowlabel{auth:a1005}W.P.M. Nuijten & 1 &65 &\href{}{NuijtenA96}~\cite{NuijtenA96}\\
\rowlabel{auth:a663}Jari Nurmi & 1 &2 &\href{works/QuSN06.pdf}{QuSN06}~\cite{QuSN06}\\
\rowlabel{auth:a917}Emmanuel Néron & 1 &3 &\href{}{NeronABCDD06}~\cite{NeronABCDD06}\\
\rowlabel{auth:a559}A. O. Amusat & 1 &0 &\href{works/abs-1902-01193.pdf}{abs-1902-01193}~\cite{abs-1902-01193}\\
\rowlabel{auth:a353}Ceyda Oguz & 1 &5 &\href{works/EdisO11.pdf}{EdisO11}~\cite{EdisO11}\\
\rowlabel{auth:a874}Olga Ohrimenko & 1 &127 &\href{works/OhrimenkoSC09.pdf}{OhrimenkoSC09}~\cite{OhrimenkoSC09}\\
\rowlabel{auth:a405}Bilal Omar Akram & 1 &0 &\href{works/AkramNHRSA23.pdf}{AkramNHRSA23}~\cite{AkramNHRSA23}\\
\rowlabel{auth:a619}Mirza Omer Beg & 1 &1 &\href{works/BegB13.pdf}{BegB13}~\cite{BegB13}\\
\rowlabel{auth:a724}Anne{-}C{\'{e}}cile Orgerie & 1 &1 &\href{works/Madi-WambaLOBM17.pdf}{Madi-WambaLOBM17}~\cite{Madi-WambaLOBM17}\\
\rowlabel{auth:a865}Gregor Ottosson & 1 &317 &\href{works/HookerO03.pdf}{HookerO03}~\cite{HookerO03}\\
\rowlabel{auth:a950}Greger Ottosson & 1 &14 &\href{}{MilanoORT02}~\cite{MilanoORT02}\\
\rowlabel{auth:a262}Mohand Ou Idir Khemmoudj & 1 &8 &\href{works/KhemmoudjPB06.pdf}{KhemmoudjPB06}~\cite{KhemmoudjPB06}\\
\rowlabel{auth:a241}Pierre Ouellet & 1 &12 &\href{works/OuelletQ13.pdf}{OuelletQ13}~\cite{OuelletQ13}\\
\rowlabel{auth:a460}Soukaina Oujana & 1 &1 &\href{works/OujanaAYB22.pdf}{OujanaAYB22}~\cite{OujanaAYB22}\\
\rowlabel{auth:a756}Asma Ouled Bedhief & 1 &0 &\href{works/Bedhief21.pdf}{Bedhief21}~\cite{Bedhief21}\\
\rowlabel{auth:a514}D{\'{e}}bora P. Ronconi & 1 &30 &\href{works/LunardiBLRV20.pdf}{LunardiBLRV20}~\cite{LunardiBLRV20}\\
\rowlabel{auth:a675}Edward P. K. Tsang & 1 &1 &\href{works/Tsang03.pdf}{Tsang03}~\cite{Tsang03}\\
\rowlabel{auth:a786}W. P. M. Nuijten & 1 &0 &\href{works/NuijtenA94.pdf}{NuijtenA94}~\cite{NuijtenA94}\\
\rowlabel{auth:a812}Meghana Padmanabhan & 1 &8 &\href{works/TranPZLDB18.pdf}{TranPZLDB18}~\cite{TranPZLDB18}\\
\rowlabel{auth:a236}Miquel Palah{\'{\i}} & 1 &3 &\href{works/BofillEGPSV14.pdf}{BofillEGPSV14}~\cite{BofillEGPSV14}\\
\rowlabel{auth:a699}Catuscia Palamidessi & 1 &10 &\href{works/FalaschiGMP97.pdf}{FalaschiGMP97}~\cite{FalaschiGMP97}\\
\rowlabel{auth:a535}Pere Pal{\`{a}}{-}Sch{\"{o}}nw{\"{a}}lder & 1 &17 &\href{works/EscobetPQPRA19.pdf}{EscobetPQPRA19}~\cite{EscobetPQPRA19}\\
\rowlabel{auth:a498}Vaibhav Pandey & 1 &3 &\href{works/PandeyS21a.pdf}{PandeyS21a}~\cite{PandeyS21a}\\
\rowlabel{auth:a554}Hoonseok Park & 1 &1 &\href{works/ParkUJR19.pdf}{ParkUJR19}~\cite{ParkUJR19}\\
\rowlabel{auth:a761}Myoung-Ju Park & 1 &0 &\href{works/HamPK21.pdf}{HamPK21}~\cite{HamPK21}\\
\rowlabel{auth:a739}Erica Pastore & 1 &0 &\href{works/AlfieriGPS23.pdf}{AlfieriGPS23}~\cite{AlfieriGPS23}\\
\rowlabel{auth:a73}Theo Pedersen & 1 &1 &\href{works/HanenKP21.pdf}{HanenKP21}~\cite{HanenKP21}\\
\rowlabel{auth:a781}Bart Peintner & 1 &0 &\href{works/MoffittPP05.pdf}{MoffittPP05}~\cite{MoffittPP05}\\
\rowlabel{auth:a934}Yunfang Peng & 1 &2 &\href{works/ChenGPSH10.pdf}{ChenGPSH10}~\cite{ChenGPSH10}\\
\rowlabel{auth:a1019}Louise Penz & 1 &0 &\href{}{PenzDN23}~\cite{PenzDN23}\\
\rowlabel{auth:a1027}Bernard Penz & 1 &0 &\href{works/CatusseCBL16.pdf}{CatusseCBL16}~\cite{CatusseCBL16}\\
\rowlabel{auth:a753}Jordi Pereira & 1 &6 &\href{works/YuraszeckMPV22.pdf}{YuraszeckMPV22}~\cite{YuraszeckMPV22}\\
\rowlabel{auth:a291}Laurent Perron & 1 &21 &\href{works/DannaP03.pdf}{DannaP03}~\cite{DannaP03}\\
\rowlabel{auth:a983}To\"{a}n Phan Huy & 1 &0 &\href{}{DomdorfPH03}~\cite{DomdorfPH03}\\
\rowlabel{auth:a419}Mehmet Pinarbasi & 1 &1 &\href{works/GurPAE23.pdf}{GurPAE23}~\cite{GurPAE23}\\
\rowlabel{auth:a687}Arthur Pinkney & 1 &11 &\href{works/MartinPY01.pdf}{MartinPY01}~\cite{MartinPY01}\\
\rowlabel{auth:a859}Eric Pinson & 1 &3 &\href{}{CarlierSJP21}~\cite{CarlierSJP21}\\
\rowlabel{auth:a1002}Éric Pinson & 1 &32 &\href{works/GuyonLPR12.pdf}{GuyonLPR12}~\cite{GuyonLPR12}\\
\rowlabel{auth:a527}David Pisinger & 1 &2 &\href{works/SacramentoSP20.pdf}{SacramentoSP20}~\cite{SacramentoSP20}\\
\rowlabel{auth:a446}Maximilian Pohl & 1 &4 &\href{works/PohlAK22.pdf}{PohlAK22}~\cite{PohlAK22}\\
\rowlabel{auth:a524}Oliver Polo{-}Mej{\'{\i}}a & 1 &8 &\href{works/Polo-MejiaALB20.pdf}{Polo-MejiaALB20}~\cite{Polo-MejiaALB20}\\
\rowlabel{auth:a530}Paul Pop & 1 &0 &\href{works/BarzegaranZP20.pdf}{BarzegaranZP20}~\cite{BarzegaranZP20}\\
\rowlabel{auth:a38}Louis Popovic & 1 &0 &\href{works/PopovicCGNC22.pdf}{PopovicCGNC22}~\cite{PopovicCGNC22}\\
\rowlabel{auth:a263}Marc Porcheron & 1 &8 &\href{works/KhemmoudjPB06.pdf}{KhemmoudjPB06}~\cite{KhemmoudjPB06}\\
\rowlabel{auth:a109}Marc Pouly & 1 &2 &\href{works/KoehlerBFFHPSSS21.pdf}{KoehlerBFFHPSSS21}~\cite{KoehlerBFFHPSSS21}\\
\rowlabel{auth:a4}Guillaume Pov{\'{e}}da & 1 &0 &\href{works/PovedaAA23.pdf}{PovedaAA23}~\cite{PovedaAA23}\\
\rowlabel{auth:a345}Matthias Prandtstetter & 1 &14 &\href{works/RendlPHPR12.pdf}{RendlPHPR12}~\cite{RendlPHPR12}\\
\rowlabel{auth:a839}Patrick Prosser & 1 &0 &\href{works/BeckPS03.pdf}{BeckPS03}~\cite{BeckPS03}\\
\rowlabel{auth:a347}Jakob Puchinger & 1 &14 &\href{works/RendlPHPR12.pdf}{RendlPHPR12}~\cite{RendlPHPR12}\\
\rowlabel{auth:a308}Jean{-}Francois Puget & 1 &6 &\href{works/Puget95.pdf}{Puget95}~\cite{Puget95}\\
\rowlabel{auth:a533}Vicen{\c{c}} Puig & 1 &17 &\href{works/EscobetPQPRA19.pdf}{EscobetPQPRA19}~\cite{EscobetPQPRA19}\\
\rowlabel{auth:a259}Kenneth Pulliam & 1 &2 &\href{works/KrogtLPHJ07.pdf}{KrogtLPHJ07}~\cite{KrogtLPHJ07}\\
\rowlabel{auth:a957}Karim Pérez Martínez & 1 &1 &\href{}{MartnezAJ22}~\cite{MartnezAJ22}\\
\rowlabel{auth:a684}Kenny Qili Zhu & 1 &0 &\href{works/ZhuS02.pdf}{ZhuS02}~\cite{ZhuS02}\\
\rowlabel{auth:a493}Ming Qin & 1 &12 &\href{works/QinWSLS21.pdf}{QinWSLS21}~\cite{QinWSLS21}\\
\rowlabel{auth:a516}Tianbao Qin & 1 &27 &\href{works/QinDCS20.pdf}{QinDCS20}~\cite{QinDCS20}\\
\rowlabel{auth:a661}Yang Qu & 1 &2 &\href{works/QuSN06.pdf}{QuSN06}~\cite{QuSN06}\\
\rowlabel{auth:a456}Yuchen Quan & 1 &2 &\href{}{ShiYXQ22}~\cite{ShiYXQ22}\\
\rowlabel{auth:a534}Joseba Quevedo & 1 &17 &\href{works/EscobetPQPRA19.pdf}{EscobetPQPRA19}~\cite{EscobetPQPRA19}\\
\rowlabel{auth:a801}Alain Quilliot & 1 &0 &\href{}{ArtiguesHQT21}~\cite{ArtiguesHQT21}\\
\rowlabel{auth:a123}Claude-Guy Quimper & 1 &0 &\href{}{FahimiQ23}~\cite{FahimiQ23}\\
\rowlabel{auth:a68}Dominik R. Bleidorn & 1 &3 &\href{works/KlankeBYE21.pdf}{KlankeBYE21}~\cite{KlankeBYE21}\\
\rowlabel{auth:a323}Aliza R. Heching & 1 &10 &\href{works/HechingH16.pdf}{HechingH16}~\cite{HechingH16}\\
\rowlabel{auth:a800}Gregg R. Rabideau & 1 &0 &\href{works/HebrardALLCMR22.pdf}{HebrardALLCMR22}~\cite{HebrardALLCMR22}\\
\rowlabel{auth:a985}Camino R. Vela & 1 &0 &\href{}{AfsarVPG23}~\cite{AfsarVPG23}\\
\rowlabel{auth:a253}Chandra Reddy & 1 &1 &\href{works/DavenportKRSH07.pdf}{DavenportKRSH07}~\cite{DavenportKRSH07}\\
\rowlabel{auth:a988}Francisco Regis Abreu Gomes & 1 &1 &\href{works/GomesM17.pdf}{GomesM17}~\cite{GomesM17}\\
\rowlabel{auth:a509}Yaping Ren & 1 &100 &\href{works/MengZRZL20.pdf}{MengZRZL20}~\cite{MengZRZL20}\\
\rowlabel{auth:a344}Andrea Rendl & 1 &14 &\href{works/RendlPHPR12.pdf}{RendlPHPR12}~\cite{RendlPHPR12}\\
\rowlabel{auth:a76}Hamid Reza Feyzmahdavian & 1 &2 &\href{works/Astrand0F21.pdf}{Astrand0F21}~\cite{Astrand0F21}\\
\rowlabel{auth:a394}Vahid Riahi & 1 &0 &\href{works/RiahiNS018.pdf}{RiahiNS018}~\cite{RiahiNS018}\\
\rowlabel{auth:a656}Diane Riopel & 1 &84 &\href{works/KhayatLR06.pdf}{KhayatLR06}~\cite{KhayatLR06}\\
\rowlabel{auth:a331}Gregory Rix & 1 &1 &\href{works/PesantRR15.pdf}{PesantRR15}~\cite{PesantRR15}\\
\rowlabel{auth:a989}Geraldo Robson Mateus & 1 &1 &\href{works/GomesM17.pdf}{GomesM17}~\cite{GomesM17}\\
\rowlabel{auth:a300}Robert Rodosek & 1 &19 &\href{works/RodosekW98.pdf}{RodosekW98}~\cite{RodosekW98}\\
\rowlabel{auth:a283}Brian Rodrigues & 1 &5 &\href{works/LimRX04.pdf}{LimRX04}~\cite{LimRX04}\\
\rowlabel{auth:a791}Joaquín Rodriguez & 1 &117 &\href{works/Rodriguez07.pdf}{Rodriguez07}~\cite{Rodriguez07}\\
\rowlabel{auth:a792}Joaquin Rodriguez & 1 &0 &\href{works/RodriguezDG02.pdf}{RodriguezDG02}~\cite{RodriguezDG02}\\
\rowlabel{auth:a119}Jerome Rogerie & 1 &148 &\href{works/LaborieRSV18.pdf}{LaborieRSV18}~\cite{LaborieRSV18}\\
\rowlabel{auth:a437}Mohammad Rohaninejad & 1 &0 &\href{works/Mehdizadeh-Somarin23.pdf}{Mehdizadeh-Somarin23}~\cite{Mehdizadeh-Somarin23}\\
\rowlabel{auth:a415}Maximiliano Rojel & 1 &0 &\href{works/YuraszeckMCCR23.pdf}{YuraszeckMCCR23}~\cite{YuraszeckMCCR23}\\
\rowlabel{auth:a536}Juli Romera & 1 &17 &\href{works/EscobetPQPRA19.pdf}{EscobetPQPRA19}~\cite{EscobetPQPRA19}\\
\rowlabel{auth:a375}Roberto Rossi & 1 &6 &\href{works/RossiTHP07.pdf}{RossiTHP07}~\cite{RossiTHP07}\\
\rowlabel{auth:a721}Fran{\c{c}}ois Roubellat & 1 &84 &\href{works/ArtiguesR00.pdf}{ArtiguesR00}~\cite{ArtiguesR00}\\
\rowlabel{auth:a22}St{\'{e}}phanie Roussel & 1 &0 &\href{works/SquillaciPR23.pdf}{SquillaciPR23}~\cite{SquillaciPR23}\\
\rowlabel{auth:a709}Didier Rozzonelli & 1 &0 &\href{}{JourdanFRD94}~\cite{JourdanFRD94}\\
\rowlabel{auth:a1029}Pascal Rubini & 1 &0 &\href{works/CatusseCBL16.pdf}{CatusseCBL16}~\cite{CatusseCBL16}\\
\rowlabel{auth:a101}Hana Rudov{\'{a}} & 1 &2 &\href{works/MurinR19.pdf}{MurinR19}~\cite{MurinR19}\\
\rowlabel{auth:a736}Rub\'{e}n Ruiz & 1 &2 &\href{works/NaderiRR23.pdf}{NaderiRR23}~\cite{NaderiRR23}\\
\rowlabel{auth:a557}Martin Ruskowski & 1 &1 &\href{works/ParkUJR19.pdf}{ParkUJR19}~\cite{ParkUJR19}\\
\rowlabel{auth:a615}Anna Ryabokon & 1 &3 &\href{}{FriedrichFMRSST14}~\cite{FriedrichFMRSST14}\\
\rowlabel{auth:a272}William S. Havens & 1 &2 &\href{works/DilkinaDH05.pdf}{DilkinaDH05}~\cite{DilkinaDH05}\\
\rowlabel{auth:a481}Mohamed S. Gheith & 1 &1 &\href{works/AbohashimaEG21.pdf}{AbohashimaEG21}~\cite{AbohashimaEG21}\\
\rowlabel{auth:a851}Gregory S. Zaric & 1 &3 &\href{}{NaderiBZ22a}~\cite{NaderiBZ22a}\\
\rowlabel{auth:a526}David Sacramento & 1 &2 &\href{works/SacramentoSP20.pdf}{SacramentoSP20}~\cite{SacramentoSP20}\\
\rowlabel{auth:a521}Shahram Saeidi & 1 &1 &\href{}{AlizdehS20}~\cite{AlizdehS20}\\
\rowlabel{auth:a948}Abderrahim Sahli & 1 &3 &\href{}{CarlierSJP21}~\cite{CarlierSJP21}\\
\rowlabel{auth:a499}Poonam Saini & 1 &3 &\href{works/PandeyS21a.pdf}{PandeyS21a}~\cite{PandeyS21a}\\
\rowlabel{auth:a740}Fabio Salassa & 1 &0 &\href{works/AlfieriGPS23.pdf}{AlfieriGPS23}~\cite{AlfieriGPS23}\\
\rowlabel{auth:a921}Amir Salehipour & 1 &0 &\href{works/EtminaniesfahaniGNMS22.pdf}{EtminaniesfahaniGNMS22}~\cite{EtminaniesfahaniGNMS22}\\
\rowlabel{auth:a110}Sophia Saller & 1 &2 &\href{works/KoehlerBFFHPSSS21.pdf}{KoehlerBFFHPSSS21}~\cite{KoehlerBFFHPSSS21}\\
\rowlabel{auth:a111}Anastasia Salyaeva & 1 &2 &\href{works/KoehlerBFFHPSSS21.pdf}{KoehlerBFFHPSSS21}~\cite{KoehlerBFFHPSSS21}\\
\rowlabel{auth:a961}Guido Sand & 1 &381 &\href{}{HarjunkoskiMBC14}~\cite{HarjunkoskiMBC14}\\
\rowlabel{auth:a616}Maria Sander & 1 &3 &\href{}{FriedrichFMRSST14}~\cite{FriedrichFMRSST14}\\
\rowlabel{auth:a722}Eric Sanlaville & 1 &7 &\href{works/PoderBS04.pdf}{PoderBS04}~\cite{PoderBS04}\\
\rowlabel{auth:a650}{\'{O}}scar Sapena & 1 &22 &\href{works/GarridoOS08.pdf}{GarridoOS08}~\cite{GarridoOS08}\\
\rowlabel{auth:a428}{\"{O}}zge Satir Akpunar & 1 &0 &\href{works/IsikYA23.pdf}{IsikYA23}~\cite{IsikYA23}\\
\rowlabel{auth:a397}Abdul Sattar & 1 &0 &\href{works/RiahiNS018.pdf}{RiahiNS018}~\cite{RiahiNS018}\\
\rowlabel{auth:a112}Peter Scheiblechner & 1 &2 &\href{works/KoehlerBFFHPSSS21.pdf}{KoehlerBFFHPSSS21}~\cite{KoehlerBFFHPSSS21}\\
\rowlabel{auth:a166}Klaus Schild & 1 &23 &\href{works/SchildW00.pdf}{SchildW00}~\cite{SchildW00}\\
\rowlabel{auth:a140}Thomas Schlechte & 1 &10 &\href{works/HeinzSSW12.pdf}{HeinzSSW12}~\cite{HeinzSSW12}\\
\rowlabel{auth:a502}Thorsten Schmidt & 1 &1 &\href{works/BenderWS21.pdf}{BenderWS21}~\cite{BenderWS21}\\
\rowlabel{auth:a777}Günter Schmidt & 1 &38 &\href{}{BlazewiczEP19}~\cite{BlazewiczEP19}\\
\rowlabel{auth:a973}Alexander Schnell & 1 &24 &\href{works/SchnellH15.pdf}{SchnellH15}~\cite{SchnellH15}\\
\rowlabel{auth:a60}Philipp Schrott{-}Kostwein & 1 &0 &\href{works/KovacsTKSG21.pdf}{KovacsTKSG21}~\cite{KovacsTKSG21}\\
\rowlabel{auth:a146}Uwe Schwiegelshohn & 1 &4 &\href{works/LimtanyakulS12.pdf}{LimtanyakulS12}~\cite{LimtanyakulS12}\\
\rowlabel{auth:a576}Lena Secher Ejlertsen & 1 &41 &\href{works/PourDERB18.pdf}{PourDERB18}~\cite{PourDERB18}\\
\rowlabel{auth:a840}Evgeny Selensky & 1 &0 &\href{works/BeckPS03.pdf}{BeckPS03}~\cite{BeckPS03}\\
\rowlabel{auth:a242}Thiago Serra & 1 &0 &\href{works/SerraNM12.pdf}{SerraNM12}~\cite{SerraNM12}\\
\rowlabel{auth:a519}Mei Sha & 1 &27 &\href{works/QinDCS20.pdf}{QinDCS20}~\cite{QinDCS20}\\
\rowlabel{auth:a605}Yufen Shao & 1 &48 &\href{works/GoelSHFS15.pdf}{GoelSHFS15}~\cite{GoelSHFS15}\\
\rowlabel{auth:a935}Xinyu Shao & 1 &2 &\href{works/ChenGPSH10.pdf}{ChenGPSH10}~\cite{ChenGPSH10}\\
\rowlabel{auth:a453}Ganquan Shi & 1 &2 &\href{}{ShiYXQ22}~\cite{ShiYXQ22}\\
\rowlabel{auth:a495}Zhongshun Shi & 1 &12 &\href{works/QinWSLS21.pdf}{QinWSLS21}~\cite{QinWSLS21}\\
\rowlabel{auth:a497}Leyuan Shi & 1 &12 &\href{works/QinWSLS21.pdf}{QinWSLS21}~\cite{QinWSLS21}\\
\rowlabel{auth:a254}Stuart Siegel & 1 &1 &\href{works/DavenportKRSH07.pdf}{DavenportKRSH07}~\cite{DavenportKRSH07}\\
\rowlabel{auth:a318}Maria Silvia Pini & 1 &1 &\href{works/GelainPRVW17.pdf}{GelainPRVW17}~\cite{GelainPRVW17}\\
\rowlabel{auth:a35}Vanessa Simard & 1 &0 &\href{works/BoudreaultSLQ22.pdf}{BoudreaultSLQ22}~\cite{BoudreaultSLQ22}\\
\rowlabel{auth:a543}Pawel Sitek & 1 &0 &\href{works/WikarekS19.pdf}{WikarekS19}~\cite{WikarekS19}\\
\rowlabel{auth:a603}M. Slusky & 1 &48 &\href{works/GoelSHFS15.pdf}{GoelSHFS15}~\cite{GoelSHFS15}\\
\rowlabel{auth:a911}Kate Smith-Miles & 1 &3 &\href{}{EdwardsBSE19}~\cite{EdwardsBSE19}\\
\rowlabel{auth:a662}Juha{-}Pekka Soininen & 1 &2 &\href{works/QuSN06.pdf}{QuSN06}~\cite{QuSN06}\\
\rowlabel{auth:a1016}Junbo Son & 1 &1 &\href{}{ZhuSZW23}~\cite{ZhuSZW23}\\
\rowlabel{auth:a623}Xiaoqing Song & 1 &1 &\href{works/ZhangLS12.pdf}{ZhangLS12}~\cite{ZhangLS12}\\
\rowlabel{auth:a843}Shahabeddin Sotudian & 1 &55 &\href{works/ZarandiASC20.pdf}{ZarandiASC20}~\cite{ZarandiASC20}\\
\rowlabel{auth:a784}Francis Sourd & 1 &7 &\href{works/SourdN00.pdf}{SourdN00}~\cite{SourdN00}\\
\rowlabel{auth:a202}Helge Spieker & 1 &6 &\href{works/MossigeGSMC17.pdf}{MossigeGSMC17}~\cite{MossigeGSMC17}\\
\rowlabel{auth:a20}Samuel Squillaci & 1 &0 &\href{works/SquillaciPR23.pdf}{SquillaciPR23}~\cite{SquillaciPR23}\\
\rowlabel{auth:a617}Andreas Starzacher & 1 &3 &\href{}{FriedrichFMRSST14}~\cite{FriedrichFMRSST14}\\
\rowlabel{auth:a49}Wolfgang Steigerwald & 1 &0 &\href{works/GeitzGSSW22.pdf}{GeitzGSSW22}~\cite{GeitzGSSW22}\\
\rowlabel{auth:a141}R{\"{u}}diger Stephan & 1 &10 &\href{works/HeinzSSW12.pdf}{HeinzSSW12}~\cite{HeinzSSW12}\\
\rowlabel{auth:a778}Malgorzata Sterna & 1 &38 &\href{}{BlazewiczEP19}~\cite{BlazewiczEP19}\\
\rowlabel{auth:a50}Robin St{\"{o}}hr & 1 &0 &\href{works/GeitzGSSW22.pdf}{GeitzGSSW22}~\cite{GeitzGSSW22}\\
\rowlabel{auth:a491}Christian St{\"{u}}rck & 1 &0 &\href{works/HubnerGSV21.pdf}{HubnerGSV21}~\cite{HubnerGSV21}\\
\rowlabel{auth:a396}Kaile Su & 1 &0 &\href{works/RiahiNS018.pdf}{RiahiNS018}~\cite{RiahiNS018}\\
\rowlabel{auth:a639}Wei Su & 1 &1 &\href{works/MakMS10.pdf}{MakMS10}~\cite{MakMS10}\\
\rowlabel{auth:a458}Kemal Subulan & 1 &5 &\href{works/SubulanC22.pdf}{SubulanC22}~\cite{SubulanC22}\\
\rowlabel{auth:a313}Premysl Sucha & 1 &2 &\href{works/BenediktSMVH18.pdf}{BenediktSMVH18}~\cite{BenediktSMVH18}\\
\rowlabel{auth:a568}Quanxin Sun & 1 &24 &\href{works/TangLWSK18.pdf}{TangLWSK18}~\cite{TangLWSK18}\\
\rowlabel{auth:a633}Zheng Sun & 1 &4 &\href{works/SunLYL10.pdf}{SunLYL10}~\cite{SunLYL10}\\
\rowlabel{auth:a657}Suresh Sundaram & 1 &12 &\href{works/SureshMOK06.pdf}{SureshMOK06}~\cite{SureshMOK06}\\
\rowlabel{auth:a790}Pavel Surynek & 1 &2 &\href{works/BartakCS10.pdf}{BartakCS10}~\cite{BartakCS10}\\
\rowlabel{auth:a788}Jir{\'{\i}} Svancara & 1 &0 &\href{works/SvancaraB22.pdf}{SvancaraB22}~\cite{SvancaraB22}\\
\rowlabel{auth:a206}Ria Szeredi & 1 &9 &\href{works/SzerediS16.pdf}{SzerediS16}~\cite{SzerediS16}\\
\rowlabel{auth:a98}Alina S{\^{\i}}rbu & 1 &1 &\href{works/GalleguillosKSB19.pdf}{GalleguillosKSB19}~\cite{GalleguillosKSB19}\\
\rowlabel{auth:a512}Willian T. Lunardi & 1 &30 &\href{works/LunardiBLRV20.pdf}{LunardiBLRV20}~\cite{LunardiBLRV20}\\
\rowlabel{auth:a1014}T. Taimre & 1 &0 &\href{works/ForbesHJST24.pdf}{ForbesHJST24}~\cite{ForbesHJST24}\\
\rowlabel{auth:a928}Yingcong Tan & 1 &1 &\href{works/TanT18.pdf}{TanT18}~\cite{TanT18}\\
\rowlabel{auth:a482}Siyu Tang & 1 &7 &\href{works/VlkHT21.pdf}{VlkHT21}~\cite{VlkHT21}\\
\rowlabel{auth:a565}Yuanjie Tang & 1 &24 &\href{works/TangLWSK18.pdf}{TangLWSK18}~\cite{TangLWSK18}\\
\rowlabel{auth:a29}Fabio Tardivo & 1 &0 &\href{works/TardivoDFMP23.pdf}{TardivoDFMP23}~\cite{TardivoDFMP23}\\
\rowlabel{auth:a376}Armagan Tarim & 1 &6 &\href{works/RossiTHP07.pdf}{RossiTHP07}~\cite{RossiTHP07}\\
\rowlabel{auth:a771}Ehsan Tarkesh Esfahani & 1 &0 &\href{works/YounespourAKE19.pdf}{YounespourAKE19}~\cite{YounespourAKE19}\\
\rowlabel{auth:a452}Nikolay Tchernev & 1 &4 &\href{works/BourreauGGLT22.pdf}{BourreauGGLT22}~\cite{BourreauGGLT22}\\
\rowlabel{auth:a733}Paolo Terenziani & 1 &1 &\href{works/BrusoniCLMMT96.pdf}{BrusoniCLMMT96}~\cite{BrusoniCLMMT96}\\
\rowlabel{auth:a503}Willian Tessaro Lunardi & 1 &0 &\href{works/Lunardi20.pdf}{Lunardi20}~\cite{Lunardi20}\\
\rowlabel{auth:a545}Stephan Teuschl & 1 &0 &\href{works/FrohnerTR19.pdf}{FrohnerTR19}~\cite{FrohnerTR19}\\
\rowlabel{auth:a65}Jordan Ticktin & 1 &0 &\href{works/HillTV21.pdf}{HillTV21}~\cite{HillTV21}\\
\rowlabel{auth:a339}Kevin Tierney & 1 &16 &\href{works/KelarevaTK13.pdf}{KelarevaTK13}~\cite{KelarevaTK13}\\
\rowlabel{auth:a683}Christian Timpe & 1 &42 &\href{works/Timpe02.pdf}{Timpe02}~\cite{Timpe02}\\
\rowlabel{auth:a546}Mary Tom & 1 &0 &\href{works/Tom19.pdf}{Tom19}~\cite{Tom19}\\
\rowlabel{auth:a627}Seyda Topaloglu & 1 &46 &\href{works/TopalogluO11.pdf}{TopalogluO11}~\cite{TopalogluO11}\\
\rowlabel{auth:a679}Miguel Toro & 1 &7 &\href{works/ValleMGT03.pdf}{ValleMGT03}~\cite{ValleMGT03}\\
\rowlabel{auth:a886}Philippe Torres & 1 &26 &\href{works/TorresL00.pdf}{TorresL00}~\cite{TorresL00}\\
\rowlabel{auth:a464}Meriem Touat & 1 &0 &\href{works/TouatBT22.pdf}{TouatBT22}~\cite{TouatBT22}\\
\rowlabel{auth:a309}Toura{\"{\i}}vane & 1 &2 &\href{works/Touraivane95.pdf}{Touraivane95}~\cite{Touraivane95}\\
\rowlabel{auth:a802}H{\'{e}}l{\`{e}}ne Toussaint & 1 &0 &\href{}{ArtiguesHQT21}~\cite{ArtiguesHQT21}\\
\rowlabel{auth:a715}Mariem Trojet & 1 &11 &\href{works/TrojetHL11.pdf}{TrojetHL11}~\cite{TrojetHL11}\\
\rowlabel{auth:a137}Semra Tunali & 1 &31 &\href{works/OzturkTHO13.pdf}{OzturkTHO13}~\cite{OzturkTHO13}\\
\rowlabel{auth:a278}Paul Tyler & 1 &0 &\href{works/HebrardTW05.pdf}{HebrardTW05}~\cite{HebrardTW05}\\
\rowlabel{auth:a555}Jumyung Um & 1 &1 &\href{works/ParkUJR19.pdf}{ParkUJR19}~\cite{ParkUJR19}\\
\rowlabel{auth:a947}David Urbach & 1 &61 &\href{}{RoshanaeiLAU17}~\cite{RoshanaeiLAU17}\\
\rowlabel{auth:a759}J. V. Moccellin & 1 &0 &\href{works/AbreuAPNM21.pdf}{AbreuAPNM21}~\cite{AbreuAPNM21}\\
\rowlabel{auth:a848}Sasha Van Cauwelaert & 1 &2 &\href{works/CauwelaertDS20.pdf}{CauwelaertDS20}~\cite{CauwelaertDS20}\\
\rowlabel{auth:a926}Alkis Vazacopoulos & 1 &0 &\href{}{AggounMV08}~\cite{AggounMV08}\\
\rowlabel{auth:a837}Thierry Vidal & 1 &58 &\href{works/BidotVLB09.pdf}{BidotVLB09}~\cite{BidotVLB09}\\
\rowlabel{auth:a668}Karen Villaverde & 1 &0 &\href{}{VillaverdeP04}~\cite{VillaverdeP04}\\
\rowlabel{auth:a754}Mariona Vilà & 1 &6 &\href{works/YuraszeckMPV22.pdf}{YuraszeckMPV22}~\cite{YuraszeckMPV22}\\
\rowlabel{auth:a492}Rebekka Volk & 1 &0 &\href{works/HubnerGSV21.pdf}{HubnerGSV21}~\cite{HubnerGSV21}\\
\rowlabel{auth:a515}Holger Voos & 1 &30 &\href{works/LunardiBLRV20.pdf}{LunardiBLRV20}~\cite{LunardiBLRV20}\\
\rowlabel{auth:a66}Thomas W. M. Vossen & 1 &0 &\href{works/HillTV21.pdf}{HillTV21}~\cite{HillTV21}\\
\rowlabel{auth:a113}Kai Waelti & 1 &2 &\href{works/KoehlerBFFHPSSS21.pdf}{KoehlerBFFHPSSS21}~\cite{KoehlerBFFHPSSS21}\\
\rowlabel{auth:a494}Runsen Wang & 1 &12 &\href{works/QinWSLS21.pdf}{QinWSLS21}~\cite{QinWSLS21}\\
\rowlabel{auth:a567}Futian Wang & 1 &24 &\href{works/TangLWSK18.pdf}{TangLWSK18}~\cite{TangLWSK18}\\
\rowlabel{auth:a582}Shouyang Wang & 1 &49 &\href{works/ZhangW18.pdf}{ZhangW18}~\cite{ZhangW18}\\
\rowlabel{auth:a606}Tao Wang & 1 &36 &\href{works/WangMD15.pdf}{WangMD15}~\cite{WangMD15}\\
\rowlabel{auth:a956}Yi Wang & 1 &8 &\href{}{GuoHLW20}~\cite{GuoHLW20}\\
\rowlabel{auth:a852}Ezra Wari & 1 &11 &\href{}{WariZ19}~\cite{WariZ19}\\
\rowlabel{auth:a962}John Wassick & 1 &381 &\href{}{HarjunkoskiMBC14}~\cite{HarjunkoskiMBC14}\\
\rowlabel{auth:a779}Jan Weglarz & 1 &38 &\href{}{BlazewiczEP19}~\cite{BlazewiczEP19}\\
\rowlabel{auth:a371}Kong Wei Lye & 1 &0 &\href{works/LauLN08.pdf}{LauLN08}~\cite{LauLN08}\\
\rowlabel{auth:a90}Johan Wess{\'{e}}n & 1 &2 &\href{works/WessenCS20.pdf}{WessenCS20}~\cite{WessenCS20}\\
\rowlabel{auth:a742}Radosław Wichniarek & 1 &0 &\href{works/CzerniachowskaWZ23.pdf}{CzerniachowskaWZ23}~\cite{CzerniachowskaWZ23}\\
\rowlabel{auth:a542}Jaroslaw Wikarek & 1 &0 &\href{works/WikarekS19.pdf}{WikarekS19}~\cite{WikarekS19}\\
\rowlabel{auth:a189}Campbell Wilson & 1 &6 &\href{works/He0GLW18.pdf}{He0GLW18}~\cite{He0GLW18}\\
\rowlabel{auth:a142}Michael Winkler & 1 &10 &\href{works/HeinzSSW12.pdf}{HeinzSSW12}~\cite{HeinzSSW12}\\
\rowlabel{auth:a501}David Wittwer & 1 &1 &\href{works/BenderWS21.pdf}{BenderWS21}~\cite{BenderWS21}\\
\rowlabel{auth:a976}Keith Womer & 1 &113 &\href{works/LiW08.pdf}{LiW08}~\cite{LiW08}\\
\rowlabel{auth:a1018}Jianguo Wu & 1 &1 &\href{}{ZhuSZW23}~\cite{ZhuSZW23}\\
\rowlabel{auth:a1022}Cheng{-}Hung Wu & 1 &14 &\href{}{NattafDYW19}~\cite{NattafDYW19}\\
\rowlabel{auth:a167}J{\"{o}}rg W{\"{u}}rtz & 1 &23 &\href{works/SchildW00.pdf}{SchildW00}~\cite{SchildW00}\\
\rowlabel{auth:a384}Quanshi Xia & 1 &13 &\href{works/ChuX05.pdf}{ChuX05}~\cite{ChuX05}\\
\rowlabel{auth:a484}Hegen Xiong & 1 &18 &\href{works/FanXG21.pdf}{FanXG21}~\cite{FanXG21}\\
\rowlabel{auth:a284}Zhou Xu & 1 &5 &\href{works/LimRX04.pdf}{LimRX04}~\cite{LimRX04}\\
\rowlabel{auth:a455}Yang Xu & 1 &2 &\href{}{ShiYXQ22}~\cite{ShiYXQ22}\\
\rowlabel{auth:a88}Tanya Y. Tang & 1 &6 &\href{works/TangB20.pdf}{TangB20}~\cite{TangB20}\\
\rowlabel{auth:a764}El Yaakoubi Anass & 1 &0 &\href{works/FallahiAC20.pdf}{FallahiAC20}~\cite{FallahiAC20}\\
\rowlabel{auth:a294}Hong Yan & 1 &8 &\href{works/HookerY02.pdf}{HookerY02}~\cite{HookerY02}\\
\rowlabel{auth:a312}Moli Yang & 1 &1 &\href{works/YangSS19.pdf}{YangSS19}~\cite{YangSS19}\\
\rowlabel{auth:a454}Zhouwang Yang & 1 &2 &\href{}{ShiYXQ22}~\cite{ShiYXQ22}\\
\rowlabel{auth:a590}Jia{-}Sheng Yao & 1 &2 &\href{works/HoYCLLCLC18.pdf}{HoYCLLCLC18}~\cite{HoYCLLCLC18}\\
\rowlabel{auth:a635}Min Yao & 1 &4 &\href{works/SunLYL10.pdf}{SunLYL10}~\cite{SunLYL10}\\
\rowlabel{auth:a583}Seung Yeob Shin & 1 &9 &\href{works/ShinBBHO18.pdf}{ShinBBHO18}~\cite{ShinBBHO18}\\
\rowlabel{auth:a69}Vassilios Yfantis & 1 &3 &\href{works/KlankeBYE21.pdf}{KlankeBYE21}~\cite{KlankeBYE21}\\
\rowlabel{auth:a768}Maryam Younespour & 1 &0 &\href{works/YounespourAKE19.pdf}{YounespourAKE19}~\cite{YounespourAKE19}\\
\rowlabel{auth:a487}Chunxia Yu & 1 &6 &\href{works/ZhangYW21.pdf}{ZhangYW21}~\cite{ZhangYW21}\\
\rowlabel{auth:a688}Xinghuo Yu & 1 &11 &\href{works/MartinPY01.pdf}{MartinPY01}~\cite{MartinPY01}\\
\rowlabel{auth:a369}Oleg Yu. Gusikhin & 1 &1 &\href{works/BarlattCG08.pdf}{BarlattCG08}~\cite{BarlattCG08}\\
\rowlabel{auth:a1021}Claude Yugma & 1 &14 &\href{}{NattafDYW19}~\cite{NattafDYW19}\\
\rowlabel{auth:a813}Peter Yun Zhang & 1 &8 &\href{works/TranPZLDB18.pdf}{TranPZLDB18}~\cite{TranPZLDB18}\\
\rowlabel{auth:a457}Pinar Yunusoglu & 1 &20 &\href{works/YunusogluY22.pdf}{YunusogluY22}~\cite{YunusogluY22}\\
\rowlabel{auth:a182}Marco Zaffalon & 1 &28 &\href{works/Darby-DowmanLMZ97.pdf}{Darby-DowmanLMZ97}~\cite{Darby-DowmanLMZ97}\\
\rowlabel{auth:a905}Boukhalfa Zahout & 1 &0 &\href{works/Zahout21.pdf}{Zahout21}~\cite{Zahout21}\\
\rowlabel{auth:a228}St{\'{e}}phane Zampelli & 1 &3 &\href{works/DerrienPZ14.pdf}{DerrienPZ14}~\cite{DerrienPZ14}\\
\rowlabel{auth:a529}Bahram Zarrin & 1 &0 &\href{works/BarzegaranZP20.pdf}{BarzegaranZP20}~\cite{BarzegaranZP20}\\
\rowlabel{auth:a980}Shohre Zehtabian & 1 &0 &\href{works/EmdeZD22.pdf}{EmdeZD22}~\cite{EmdeZD22}\\
\rowlabel{auth:a404}Mengjie Zhang & 1 &0 &\href{works/abs-2402-00459.pdf}{abs-2402-00459}~\cite{abs-2402-00459}\\
\rowlabel{auth:a473}Haotian Zhang & 1 &0 &\href{works/ZhangJZL22.pdf}{ZhangJZL22}~\cite{ZhangJZL22}\\
\rowlabel{auth:a486}Luping Zhang & 1 &6 &\href{works/ZhangYW21.pdf}{ZhangYW21}~\cite{ZhangYW21}\\
\rowlabel{auth:a508}Chaoyong Zhang & 1 &100 &\href{works/MengZRZL20.pdf}{MengZRZL20}~\cite{MengZRZL20}\\
\rowlabel{auth:a510}Biao Zhang & 1 &100 &\href{works/MengZRZL20.pdf}{MengZRZL20}~\cite{MengZRZL20}\\
\rowlabel{auth:a581}Sicheng Zhang & 1 &49 &\href{works/ZhangW18.pdf}{ZhangW18}~\cite{ZhangW18}\\
\rowlabel{auth:a621}Xujun Zhang & 1 &1 &\href{works/ZhangLS12.pdf}{ZhangLS12}~\cite{ZhangLS12}\\
\rowlabel{auth:a767}Lihui Zhang & 1 &0 &\href{works/ZouZ20.pdf}{ZouZ20}~\cite{ZouZ20}\\
\rowlabel{auth:a809}Jiachen Zhang & 1 &0 &\href{works/ZhangBB22.pdf}{ZhangBB22}~\cite{ZhangBB22}\\
\rowlabel{auth:a850}Guoqing Zhang & 1 &0 &\href{works/NaderiBZ22.pdf}{NaderiBZ22}~\cite{NaderiBZ22}\\
\rowlabel{auth:a1017}Xi Zhang & 1 &1 &\href{}{ZhuSZW23}~\cite{ZhuSZW23}\\
\rowlabel{auth:a609}Jinlian Zhou & 1 &0 &\href{works/ZhouGL15.pdf}{ZhouGL15}~\cite{ZhouGL15}\\
\rowlabel{auth:a853}Weihang Zhu & 1 &11 &\href{}{WariZ19}~\cite{WariZ19}\\
\rowlabel{auth:a966}Jianjun Zhu & 1 &0 &\href{}{GuoZ23}~\cite{GuoZ23}\\
\rowlabel{auth:a1015}Xuedong Zhu & 1 &1 &\href{}{ZhuSZW23}~\cite{ZhuSZW23}\\
\rowlabel{auth:a267}Pawel Zielinski & 1 &13 &\href{works/FortinZDF05.pdf}{FortinZDF05}~\cite{FortinZDF05}\\
\rowlabel{auth:a804}J{\"{u}}rgen Zimmermann & 1 &25 &\href{works/KreterSSZ18.pdf}{KreterSSZ18}~\cite{KreterSSZ18}\\
\rowlabel{auth:a766}Xin Zou & 1 &0 &\href{works/ZouZ20.pdf}{ZouZ20}~\cite{ZouZ20}\\
\rowlabel{auth:a311}Mathijs de Weerdt & 1 &1 &\href{works/BogaerdtW19.pdf}{BogaerdtW19}~\cite{BogaerdtW19}\\
\rowlabel{auth:a758}Bruno de Athayde Prata & 1 &0 &\href{works/AbreuAPNM21.pdf}{AbreuAPNM21}~\cite{AbreuAPNM21}\\
\rowlabel{auth:a903}Alexis de Clercq & 1 &0 &\href{works/Clercq12.pdf}{Clercq12}~\cite{Clercq12}\\
\rowlabel{auth:a258}Roman van der Krogt & 1 &2 &\href{works/KrogtLPHJ07.pdf}{KrogtLPHJ07}~\cite{KrogtLPHJ07}\\
\rowlabel{auth:a310}Pim van den Bogaerdt & 1 &1 &\href{works/BogaerdtW19.pdf}{BogaerdtW19}~\cite{BogaerdtW19}\\
\rowlabel{auth:a845}Willem-Jan van Hoeve & 1 &12 &\href{works/HookerH17.pdf}{HookerH17}~\cite{HookerH17}\\
\rowlabel{auth:a1013}F.A. van der Schoot & 1 &0 &\href{works/ForbesHJST24.pdf}{ForbesHJST24}~\cite{ForbesHJST24}\\
\rowlabel{auth:a237}Stefano {Di Alesio} & 1 &3 &\href{works/AlesioNBG14.pdf}{AlesioNBG14}~\cite{AlesioNBG14}\\
\rowlabel{auth:a833}Ulas {\"{O}}zen & 1 &8 &\href{works/TerekhovDOB12.pdf}{TerekhovDOB12}~\cite{TerekhovDOB12}\\
\rowlabel{auth:a580}Selin {\"{O}}zpeynirci & 1 &31 &\href{works/GokgurHO18.pdf}{GokgurHO18}~\cite{GokgurHO18}\\
\rowlabel{auth:a136}Cemalettin {\"{O}}zt{\"{u}}rk & 1 &31 &\href{works/OzturkTHO13.pdf}{OzturkTHO13}~\cite{OzturkTHO13}\\
\rowlabel{auth:a5}Nahum {\'{A}}lvarez & 1 &0 &\href{works/PovedaAA23.pdf}{PovedaAA23}~\cite{PovedaAA23}\\
\rowlabel{auth:a221}Se{\'{a}}n {\'{O}}g Murphy & 1 &1 &\href{works/MurphyMB15.pdf}{MurphyMB15}~\cite{MurphyMB15}\\
\rowlabel{auth:a459}Gizem {\c{C}}akir & 1 &5 &\href{works/SubulanC22.pdf}{SubulanC22}~\cite{SubulanC22}\\
\rowlabel{auth:a743}Krzysztof Żywicki & 1 &0 &\href{works/CzerniachowskaWZ23.pdf}{CzerniachowskaWZ23}~\cite{CzerniachowskaWZ23}\\
\end{longtable}
}



\clearpage
\section{Most Cited Works}

{\scriptsize
\begin{longtable}{>{\raggedright\arraybackslash}p{3cm}>{\raggedright\arraybackslash}p{6cm}>{\raggedright\arraybackslash}p{6.5cm}rrrp{2.5cm}rrrrr}
\rowcolor{white}\caption{Works from bibtex (Total 30)}\\ \toprule
\rowcolor{white}\shortstack{Key\\Source} & Authors & Title & LC & Cite & Year & \shortstack{Conference\\/Journal\\/School} & Pages & \shortstack{Nr\\Cites} & \shortstack{Nr\\Refs} & b & c \\ \midrule\endhead
\bottomrule
\endfoot
JainM99 \href{http://dx.doi.org/10.1016/s0377-2217(98)00113-1}{JainM99} & \hyperref[auth:a962]{A. Jain}, \hyperref[auth:a963]{S. Meeran} & Deterministic job-shop scheduling: Past, present and future & \href{../works/JainM99.pdf}{Yes} & \cite{JainM99} & 1999 & European Journal of Operational Research & 45 & 490 & 150 & \ref{b:JainM99} & n/a\\
MintonJPL92 \href{http://dx.doi.org/10.1016/0004-3702(92)90007-k}{MintonJPL92} & \hyperref[auth:a1227]{S. Minton}, \hyperref[auth:a1228]{Mark D. Johnston}, \hyperref[auth:a1229]{Andrew B. Philips}, \hyperref[auth:a1230]{P. Laird} & Minimizing conflicts: a heuristic repair method for constraint satisfaction and scheduling problems & No & \cite{MintonJPL92} & 1992 & Artificial Intelligence & null & 437 & 13 & No & n/a\\
HarjunkoskiMBC14 \href{http://dx.doi.org/10.1016/j.compchemeng.2013.12.001}{HarjunkoskiMBC14} & \hyperref[auth:a875]{I. Harjunkoski}, \hyperref[auth:a384]{Christos T. Maravelias}, \hyperref[auth:a944]{P. Bongers}, \hyperref[auth:a895]{Pedro M. Castro}, \hyperref[auth:a70]{S. Engell}, \hyperref[auth:a385]{Ignacio E. Grossmann}, \hyperref[auth:a161]{John N. Hooker}, \hyperref[auth:a945]{C. Méndez}, \hyperref[auth:a946]{G. Sand}, \hyperref[auth:a947]{J. Wassick} & Scope for industrial applications of production scheduling models and solution methods & \href{../works/HarjunkoskiMBC14.pdf}{Yes} & \cite{HarjunkoskiMBC14} & 2014 & Computers \  Chemical Engineering & 33 & 381 & 176 & \ref{b:HarjunkoskiMBC14} & n/a\\
BlazewiczDP96 \href{http://dx.doi.org/10.1016/0377-2217(95)00362-2}{BlazewiczDP96} & \hyperref[auth:a983]{J. Błażewicz}, \hyperref[auth:a984]{W. Domschke}, \hyperref[auth:a441]{E. Pesch} & The job shop scheduling problem: Conventional and new solution techniques & \href{../works/BlazewiczDP96.pdf}{Yes} & \cite{BlazewiczDP96} & 1996 & European Journal of Operational Research & 33 & 344 & 127 & \ref{b:BlazewiczDP96} & n/a\\
HookerO03 \href{http://dx.doi.org/10.1007/s10107-003-0375-9}{HookerO03} & \hyperref[auth:a161]{John N. Hooker}, \hyperref[auth:a856]{G. Ottosson} & Logic-based Benders decomposition & \href{../works/HookerO03.pdf}{Yes} & \cite{HookerO03} & 2003 & Mathematical Programming & 28 & 317 & 0 & \ref{b:HookerO03} & n/a\\
Davis87 \href{http://dx.doi.org/10.1016/0004-3702(87)90091-9}{Davis87} & \hyperref[auth:a1232]{E. Davis} & Constraint propagation with interval labels & \href{../works/Davis87.pdf}{Yes} & \cite{Davis87} & 1987 & Artificial Intelligence & 51 & 308 & 21 & \ref{b:Davis87} & n/a\\
BaptistePN01 \href{http://dx.doi.org/10.1007/978-1-4615-1479-4}{BaptistePN01} & \hyperref[auth:a163]{P. Baptiste}, \hyperref[auth:a164]{Claude Le Pape}, \hyperref[auth:a659]{W. Nuijten} & Constraint-Based Scheduling & No & \cite{BaptistePN01} & 2001 & Book & null & 296 & 0 & No & n/a\\
JainG01 \href{http://dx.doi.org/10.1287/ijoc.13.4.258.9733}{JainG01} & \hyperref[auth:a848]{V. Jain}, \hyperref[auth:a385]{Ignacio E. Grossmann} & Algorithms for Hybrid MILP/CP Models for a Class of Optimization Problems & \href{../works/JainG01.pdf}{Yes} & \cite{JainG01} & 2001 & INFORMS Journal on Computing & 19 & 279 & 23 & \ref{b:JainG01} & n/a\\
AggounB93 \href{https://www.sciencedirect.com/science/article/pii/089571779390068A}{AggounB93} & \hyperref[auth:a728]{A. Aggoun}, \hyperref[auth:a129]{N. Beldiceanu} & Extending {CHIP} in order to solve complex scheduling and placement problems & \href{../works/AggounB93.pdf}{Yes} & \cite{AggounB93} & 1993 & Mathematical and Computer Modelling & 17 & 187 & 11 & \ref{b:AggounB93} & n/a\\
Hooker00 \href{http://dx.doi.org/10.1002/9781118033036}{Hooker00} & \hyperref[auth:a161]{John N. Hooker} & Logic Based Methods for Optimization: Combining Optimization and Constraint Satisfaction & No & \cite{Hooker00} & 2000 & Book & null & 185 & 0 & No & n/a\\
Hooker07 \href{http://dx.doi.org/10.1287/opre.1060.0371}{Hooker07} & \hyperref[auth:a161]{John N. Hooker} & Planning and Scheduling by Logic-Based Benders Decomposition & \href{../works/Hooker07.pdf}{Yes} & \cite{Hooker07} & 2007 & Operations Research & 29 & 181 & 19 & \ref{b:Hooker07} & n/a\\
KendallKRU10 \href{}{KendallKRU10} & \hyperref[auth:a1410]{G. Kendall}, \hyperref[auth:a1183]{S. Knust}, \hyperref[auth:a1409]{Celso C. Ribeiro}, \hyperref[auth:a1411]{S. Urrutia} & Scheduling in sports: An annotated bibliography & No & \cite{KendallKRU10} & 2010 & COMPUTERS \  OPERATIONS RESEARCH & 19 & 181 & 0 & No & n/a\\
HarjunkoskiG02 \href{http://dx.doi.org/10.1016/s0098-1354(02)00100-x}{HarjunkoskiG02} & \hyperref[auth:a875]{I. Harjunkoski}, \hyperref[auth:a385]{Ignacio E. Grossmann} & Decomposition techniques for multistage scheduling problems using mixed-integer and constraint programming methods & \href{../works/HarjunkoskiG02.pdf}{Yes} & \cite{HarjunkoskiG02} & 2002 & Computers \  Chemical Engineering & 20 & 169 & 11 & \ref{b:HarjunkoskiG02} & n/a\\
BeldiceanuC94 \href{https://www.sciencedirect.com/science/article/pii/0895717794901279}{BeldiceanuC94} & \hyperref[auth:a129]{N. Beldiceanu}, \hyperref[auth:a787]{E. Contejean} & Introducing Global Constraints in {CHIP} & \href{../works/BeldiceanuC94.pdf}{Yes} & \cite{BeldiceanuC94} & 1994 & Mathematical and Computer Modelling & 27 & 167 & 8 & \ref{b:BeldiceanuC94} & n/a\\
Ham18 \href{http://dx.doi.org/10.1016/j.trc.2018.03.025}{Ham18} & \hyperref[auth:a773]{Andy M. Ham} & Integrated scheduling of m-truck, m-drone, and m-depot constrained by time-window, drop-pickup, and m-visit using constraint programming & \href{../works/Ham18.pdf}{Yes} & \cite{Ham18} & 2018 & Transportation Research Part C: Emerging Technologies & 14 & 164 & 14 & \ref{b:Ham18} & n/a\\
LaborieRSV18 \href{https://doi.org/10.1007/s10601-018-9281-x}{LaborieRSV18} & \hyperref[auth:a118]{P. Laborie}, \hyperref[auth:a119]{J. Rogerie}, \hyperref[auth:a120]{P. Shaw}, \hyperref[auth:a121]{P. Vil{\'{\i}}m} & {IBM} {ILOG} {CP} optimizer for scheduling - 20+ years of scheduling with constraints at {IBM/ILOG} & \href{../works/LaborieRSV18.pdf}{Yes} & \cite{LaborieRSV18} & 2018 & Constraints An Int. J. & 41 & 148 & 35 & \ref{b:LaborieRSV18} & \ref{c:LaborieRSV18}\\
Laborie03 \href{http://dx.doi.org/10.1016/s0004-3702(02)00362-4}{Laborie03} & \hyperref[auth:a118]{P. Laborie} & Algorithms for propagating resource constraints in AI planning and scheduling: Existing approaches and new results & \href{../works/Laborie03.pdf}{Yes} & \cite{Laborie03} & 2003 & Artificial Intelligence & 38 & 128 & 10 & \ref{b:Laborie03} & n/a\\
OhrimenkoSC09 \href{http://dx.doi.org/10.1007/s10601-008-9064-x}{OhrimenkoSC09} & \hyperref[auth:a865]{O. Ohrimenko}, \hyperref[auth:a126]{Peter J. Stuckey}, \hyperref[auth:a866]{M. Codish} & Propagation via lazy clause generation & \href{../works/OhrimenkoSC09.pdf}{Yes} & \cite{OhrimenkoSC09} & 2009 & Constraints An Int. J. & 35 & 127 & 15 & \ref{b:OhrimenkoSC09} & n/a\\
KuB16 \href{https://doi.org/10.1016/j.cor.2016.04.006}{KuB16} & \hyperref[auth:a334]{W. Ku}, \hyperref[auth:a89]{J. Christopher Beck} & Mixed Integer Programming models for job shop scheduling: {A} computational analysis & \href{../works/KuB16.pdf}{Yes} & \cite{KuB16} & 2016 & Computers \  Operations Research & 9 & 119 & 17 & \ref{b:KuB16} & n/a\\
Rodriguez07 \href{https://www.sciencedirect.com/science/article/pii/S0191261506000233}{Rodriguez07} & \hyperref[auth:a784]{J. Rodriguez} & A constraint programming model for real-time train scheduling at junctions & \href{../works/Rodriguez07.pdf}{Yes} & \cite{Rodriguez07} & 2007 & Transportation Research Part B: Methodological & 15 & 117 & 6 & \ref{b:Rodriguez07} & n/a\\
MaraveliasCG04 \href{http://dx.doi.org/10.1016/j.compchemeng.2004.03.016}{MaraveliasCG04} & \hyperref[auth:a384]{Christos T. Maravelias}, \hyperref[auth:a385]{Ignacio E. Grossmann} & A hybrid MILP/CP decomposition approach for the continuous time scheduling of multipurpose batch plants & \href{../works/MaraveliasCG04.pdf}{Yes} & \cite{MaraveliasCG04} & 2004 & Computers \  Chemical Engineering & 29 & 116 & 24 & \ref{b:MaraveliasCG04} & n/a\\
LiW08 \href{http://dx.doi.org/10.1007/s10951-008-0079-3}{LiW08} & \hyperref[auth:a960]{H. Li}, \hyperref[auth:a961]{K. Womer} & Scheduling projects with multi-skilled personnel by a hybrid MILP/CP benders decomposition algorithm & \href{../works/LiW08.pdf}{Yes} & \cite{LiW08} & 2008 & Journal of Scheduling & 18 & 113 & 31 & \ref{b:LiW08} & n/a\\
CorreaLR07 \href{http://dx.doi.org/10.1016/j.cor.2005.07.004}{CorreaLR07} & \hyperref[auth:a956]{Ayoub Insa Corr{\'{e}}a}, \hyperref[auth:a648]{A. Langevin}, \hyperref[auth:a329]{L. Rousseau} & Scheduling and routing of automated guided vehicles: A hybrid approach & \href{../works/CorreaLR07.pdf}{Yes} & \cite{CorreaLR07} & 2007 & Computers \  Operations Research & 20 & 106 & 20 & \ref{b:CorreaLR07} & n/a\\
Kuchcinski03 \href{http://dx.doi.org/10.1145/785411.785416}{Kuchcinski03} & \hyperref[auth:a663]{K. Kuchcinski} & Constraints-driven scheduling and resource assignment & \href{../works/Kuchcinski03.pdf}{Yes} & \cite{Kuchcinski03} & 2003 & ACM Transactions on Design Automation of Electronic Systems & 29 & 105 & 15 & \ref{b:Kuchcinski03} & n/a\\
MengZRZL20 \href{https://doi.org/10.1016/j.cie.2020.106347}{MengZRZL20} & \hyperref[auth:a503]{L. Meng}, \hyperref[auth:a504]{C. Zhang}, \hyperref[auth:a505]{Y. Ren}, \hyperref[auth:a506]{B. Zhang}, \hyperref[auth:a507]{C. Lv} & Mixed-integer linear programming and constraint programming formulations for solving distributed flexible job shop scheduling problem & \href{../works/MengZRZL20.pdf}{Yes} & \cite{MengZRZL20} & 2020 & Computers \  Industrial Engineering & 13 & 100 & 62 & \ref{b:MengZRZL20} & \ref{c:MengZRZL20}\\
BensanaLV99 \href{https://doi.org/10.1023/A:1026488509554}{BensanaLV99} & \hyperref[auth:a172]{E. Bensana}, \hyperref[auth:a173]{M. Lema{\^{\i}}tre}, \hyperref[auth:a174]{G. Verfaillie} & Earth Observation Satellite Management & \href{../works/BensanaLV99.pdf}{Yes} & \cite{BensanaLV99} & 1999 & Constraints An Int. J. & 7 & 99 & 0 & \ref{b:BensanaLV99} & \ref{c:BensanaLV99}\\
Pape94 \href{http://dx.doi.org/10.1049/ise.1994.0009}{Pape94} & \hyperref[auth:a164]{Claude Le Pape} & Implementation of resource constraints in ILOG SCHEDULE: a library for the development of constraint-based scheduling systems & \href{../works/Pape94.pdf}{Yes} & \cite{Pape94} & 1994 & Intelligent Systems Engineering & 34 & 98 & 0 & \ref{b:Pape94} & n/a\\
SadehF96 \href{http://dx.doi.org/10.1016/0004-3702(95)00098-4}{SadehF96} & \hyperref[auth:a1182]{N. Sadeh}, \hyperref[auth:a304]{Mark S. Fox} & Variable and value ordering heuristics for the job shop scheduling constraint satisfaction problem & \href{../works/SadehF96.pdf}{Yes} & \cite{SadehF96} & 1996 & Artificial Intelligence & 41 & 95 & 17 & \ref{b:SadehF96} & n/a\\
Hooker02 \href{http://dx.doi.org/10.1287/ijoc.14.4.295.2828}{Hooker02} & \hyperref[auth:a161]{John N. Hooker} & Logic, Optimization, and Constraint Programming & No & \cite{Hooker02} & 2002 & INFORMS Journal on Computing & null & 94 & 84 & No & n/a\\
HookerO99 \href{http://dx.doi.org/10.1016/s0166-218x(99)00100-6}{HookerO99} & \hyperref[auth:a161]{John N. Hooker}, \hyperref[auth:a1166]{M. Osorio} & Mixed logical-linear programming & \href{../works/HookerO99.pdf}{Yes} & \cite{HookerO99} & 1999 & Discrete Applied Mathematics & 48 & 92 & 48 & \ref{b:HookerO99} & n/a\\
\end{longtable}
}



\clearpage
\section{Most Cited Relevant Works}

{\scriptsize
\begin{longtable}{>{\raggedright\arraybackslash}p{2.5cm}>{\raggedright\arraybackslash}p{4.5cm}>{\raggedright\arraybackslash}p{6.0cm}p{1.0cm}rr>{\raggedright\arraybackslash}p{2.0cm}r>{\raggedright\arraybackslash}p{1cm}p{1cm}p{1cm}p{1cm}}
\rowcolor{white}\caption{Most Cited Relevant Works (Total 30)}\\ \toprule
\rowcolor{white}\shortstack{Key\\Source} & Authors & Title (Colored by Open Access)& \shortstack{Details\\LC} & Cite & Year & \shortstack{Conference\\/Journal\\/School} & Pages & Relevance &\shortstack{Cites\\OC XR\\SC} & \shortstack{Refs\\OC\\XR} & \shortstack{Links\\Cites\\Refs}\\ \midrule\endhead
\bottomrule
\endfoot
JainM99 \href{http://dx.doi.org/10.1016/s0377-2217(98)00113-1}{JainM99} & \hyperref[auth:a953]{A. Jain}, \hyperref[auth:a954]{S. Meeran} & Deterministic job-shop scheduling: Past, present and future & \hyperref[detail:JainM99]{Details} \href{../works/JainM99.pdf}{Yes} & \cite{JainM99} & 1999 & European Journal of Operational Research & 45 & \noindent{}\textcolor{black!50}{0.00} \textcolor{black!50}{0.00} \textbf{7.96} & 490 503 630 & 150 262 & 26 10 16\\
MintonJPL92 \href{http://dx.doi.org/10.1016/0004-3702(92)90007-k}{MintonJPL92} & \hyperref[auth:a1209]{S. Minton}, \hyperref[auth:a1210]{M. D. Johnston}, \hyperref[auth:a1211]{A. B. Philips}, \hyperref[auth:a1212]{P. Laird} & \cellcolor{green!10}Minimizing conflicts: a heuristic repair method for constraint satisfaction and scheduling problems & \hyperref[detail:MintonJPL92]{Details} \href{../works/MintonJPL92.pdf}{Yes} & \cite{MintonJPL92} & 1992 & Artificial Intelligence & 45 & \noindent{}\textbf{1.00} \textbf{1.00} \textbf{1.68} & 437 440 525 & 13 46 & 18 18 0\\
HarjunkoskiMBC14 \href{http://dx.doi.org/10.1016/j.compchemeng.2013.12.001}{HarjunkoskiMBC14} & \hyperref[auth:a870]{I. Harjunkoski}, \hyperref[auth:a381]{C. T. Maravelias}, \hyperref[auth:a936]{P. Bongers}, \hyperref[auth:a890]{P. M. Castro}, \hyperref[auth:a70]{S. Engell}, \hyperref[auth:a382]{I. E. Grossmann}, \hyperref[auth:a160]{J. N. Hooker}, \hyperref[auth:a937]{C. Méndez}, \hyperref[auth:a938]{G. Sand}, \hyperref[auth:a939]{J. Wassick} & \cellcolor{green!10}Scope for industrial applications of production scheduling models and solution methods & \hyperref[detail:HarjunkoskiMBC14]{Details} \href{../works/HarjunkoskiMBC14.pdf}{Yes} & \cite{HarjunkoskiMBC14} & 2014 & Computers \  Chemical Engineering & 33 & \noindent{}\textcolor{black!50}{0.00} \textcolor{black!50}{0.00} \textbf{41.04} & 381 393 418 & 176 229 & 28 10 18\\
BlazewiczDP96 \href{http://dx.doi.org/10.1016/0377-2217(95)00362-2}{BlazewiczDP96} & \hyperref[auth:a974]{J. Błażewicz}, \hyperref[auth:a975]{W. Domschke}, \hyperref[auth:a437]{E. Pesch} & The job shop scheduling problem: Conventional and new solution techniques & \hyperref[detail:BlazewiczDP96]{Details} \href{../works/BlazewiczDP96.pdf}{Yes} & \cite{BlazewiczDP96} & 1996 & European Journal of Operational Research & 33 & \noindent{}\textcolor{black!50}{0.00} \textcolor{black!50}{0.00} \textbf{14.88} & 344 357 412 & 127 224 & 33 20 13\\
Davis87 \href{http://dx.doi.org/10.1016/0004-3702(87)90091-9}{Davis87} & \hyperref[auth:a1214]{E. Davis} & \cellcolor{gold!20}Constraint propagation with interval labels & \hyperref[detail:Davis87]{Details} \href{../works/Davis87.pdf}{Yes} & \cite{Davis87} & 1987 & Artificial Intelligence & 51 & \noindent{}\textcolor{black!50}{0.00} \textcolor{black!50}{0.00} \textbf{2.34} & 308 312 332 & 21 51 & 12 11 1\\
JainG01 \href{http://dx.doi.org/10.1287/ijoc.13.4.258.9733}{JainG01} & \hyperref[auth:a843]{V. Jain}, \hyperref[auth:a382]{I. E. Grossmann} & Algorithms for Hybrid MILP/CP Models for a Class of Optimization Problems & \hyperref[detail:JainG01]{Details} \href{../works/JainG01.pdf}{Yes} & \cite{JainG01} & 2001 & \cellcolor{red!20}INFORMS Journal on Computing & 19 & \noindent{}\textcolor{black!50}{0.00} \textcolor{black!50}{0.00} \textbf{29.84} & 279 284 321 & 23 38 & 101 89 12\\
BeldiceanuC94 \href{https://www.sciencedirect.com/science/article/pii/0895717794901279}{BeldiceanuC94} & \hyperref[auth:a128]{N. Beldiceanu}, \hyperref[auth:a783]{E. Contejean} & \cellcolor{gold!20}Introducing Global Constraints in {CHIP} \hyperref[abs:BeldiceanuC94]{Abstract} & \hyperref[detail:BeldiceanuC94]{Details} \href{../works/BeldiceanuC94.pdf}{Yes} & \cite{BeldiceanuC94} & 1994 & Mathematical and Computer Modelling & 27 & \noindent{}\textcolor{black!50}{0.00} \textbf{1.00} \textbf{1.72} & 167 169 223 & 8 21 & 37 34 3\\
KendallKRU10 \href{http://dx.doi.org/10.1016/j.cor.2009.05.013}{KendallKRU10} & \hyperref[auth:a1386]{G. Kendall}, \hyperref[auth:a1165]{S. Knust}, \hyperref[auth:a1385]{C. C. Ribeiro}, \hyperref[auth:a1387]{S. Urrutia} & Scheduling in sports: An annotated bibliography \hyperref[abs:KendallKRU10]{Abstract} & \hyperref[detail:KendallKRU10]{Details} \href{../works/KendallKRU10.pdf}{Yes} & \cite{KendallKRU10} & 2010 & Computers \  Operations Research & 19 & \noindent{}\textcolor{black!50}{0.00} \textcolor{black!50}{0.00} \textbf{9.51} & 181 186 220 & 0 0 & 6 6 0\\
AggounB93 \href{https://www.sciencedirect.com/science/article/pii/089571779390068A}{AggounB93} & \hyperref[auth:a724]{A. Aggoun}, \hyperref[auth:a128]{N. Beldiceanu} & \cellcolor{gold!20}Extending {CHIP} in order to solve complex scheduling and placement problems \hyperref[abs:AggounB93]{Abstract} & \hyperref[detail:AggounB93]{Details} \href{../works/AggounB93.pdf}{Yes} & \cite{AggounB93} & 1993 & Mathematical and Computer Modelling & 17 & \noindent{}\textcolor{black!50}{0.00} \textbf{3.00} \textbf{3.18} & 187 191 214 & 11 36 & 91 89 2\\
Hooker07 \href{http://dx.doi.org/10.1287/opre.1060.0371}{Hooker07} & \hyperref[auth:a160]{J. N. Hooker} & Planning and Scheduling by Logic-Based Benders Decomposition & \hyperref[detail:Hooker07]{Details} \href{../works/Hooker07.pdf}{Yes} & \cite{Hooker07} & 2007 & \cellcolor{red!20}Operations Research & 15 & \noindent{}\textcolor{black!50}{0.00} \textcolor{black!50}{0.00} \textbf{14.07} & 181 197 205 & 19 20 & 66 52 14\\
LaborieRSV18 \href{https://doi.org/10.1007/s10601-018-9281-x}{LaborieRSV18} & \hyperref[auth:a118]{P. Laborie}, \hyperref[auth:a119]{J. Rogerie}, \hyperref[auth:a120]{P. Shaw}, \hyperref[auth:a121]{P. Vil{\'{\i}}m} & {IBM} {ILOG} {CP} optimizer for scheduling - 20+ years of scheduling with constraints at {IBM/ILOG} & \hyperref[detail:LaborieRSV18]{Details} \href{../works/LaborieRSV18.pdf}{Yes} & \cite{LaborieRSV18} & 2018 & Constraints An Int. J. & 41 & \noindent{}\textbf{1.00} \textbf{1.00} \textbf{54.37} & 148 178 203 & 35 54 & 92 69 23\\
OhrimenkoSC09 \href{http://dx.doi.org/10.1007/s10601-008-9064-x}{OhrimenkoSC09} & \hyperref[auth:a860]{O. Ohrimenko}, \hyperref[auth:a125]{P. J. Stuckey}, \hyperref[auth:a861]{M. Codish} & Propagation via lazy clause generation & \hyperref[detail:OhrimenkoSC09]{Details} \href{../works/OhrimenkoSC09.pdf}{Yes} & \cite{OhrimenkoSC09} & 2009 & Constraints An Int. J. & 35 & \noindent{}\textcolor{black!50}{0.00} \textcolor{black!50}{0.00} \textbf{3.19} & 127 128 198 & 15 35 & 33 31 2\\
Ham18 \href{http://dx.doi.org/10.1016/j.trc.2018.03.025}{Ham18} & \hyperref[auth:a769]{A. M. Ham} & Integrated scheduling of m-truck, m-drone, and m-depot constrained by time-window, drop-pickup, and m-visit using constraint programming & \hyperref[detail:Ham18]{Details} \href{../works/Ham18.pdf}{Yes} & \cite{Ham18} & 2018 & Transportation Research Part C: Emerging Technologies & 14 & \noindent{}\textbf{1.00} \textbf{1.00} \textbf{12.98} & 164 192 197 & 14 30 & 11 7 4\\
HarjunkoskiG02 \href{http://dx.doi.org/10.1016/s0098-1354(02)00100-x}{HarjunkoskiG02} & \hyperref[auth:a870]{I. Harjunkoski}, \hyperref[auth:a382]{I. E. Grossmann} & Decomposition techniques for multistage scheduling problems using mixed-integer and constraint programming methods & \hyperref[detail:HarjunkoskiG02]{Details} \href{../works/HarjunkoskiG02.pdf}{Yes} & \cite{HarjunkoskiG02} & 2002 & Computers \  Chemical Engineering & 20 & \noindent{}\textbf{1.00} \textbf{1.00} \textbf{20.38} & 169 173 192 & 11 25 & 42 39 3\\
Laborie03 \href{http://dx.doi.org/10.1016/s0004-3702(02)00362-4}{Laborie03} & \hyperref[auth:a118]{P. Laborie} & \cellcolor{gold!20}Algorithms for propagating resource constraints in AI planning and scheduling: Existing approaches and new results & \hyperref[detail:Laborie03]{Details} \href{../works/Laborie03.pdf}{Yes} & \cite{Laborie03} & 2003 & Artificial Intelligence & 38 & \noindent{}\textcolor{black!50}{0.00} \textcolor{black!50}{0.00} \textbf{8.42} & 128 129 175 & 10 31 & 48 43 5\\
MengZRZL20 \href{https://doi.org/10.1016/j.cie.2020.106347}{MengZRZL20} & \hyperref[auth:a499]{L. Meng}, \hyperref[auth:a500]{C. Zhang}, \hyperref[auth:a501]{Y. Ren}, \hyperref[auth:a502]{B. Zhang}, \hyperref[auth:a503]{C. Lv} & Mixed-integer linear programming and constraint programming formulations for solving distributed flexible job shop scheduling problem & \hyperref[detail:MengZRZL20]{Details} \href{../works/MengZRZL20.pdf}{Yes} & \cite{MengZRZL20} & 2020 & Computers \  Industrial Engineering & 13 & \noindent{}\textbf{2.00} \textbf{2.00} \textbf{36.82} & 100 133 152 & 62 69 & 26 16 10\\
LiW08 \href{http://dx.doi.org/10.1007/s10951-008-0079-3}{LiW08} & \hyperref[auth:a951]{H. Li}, \hyperref[auth:a952]{K. Womer} & Scheduling projects with multi-skilled personnel by a hybrid MILP/CP benders decomposition algorithm & \hyperref[detail:LiW08]{Details} \href{../works/LiW08.pdf}{Yes} & \cite{LiW08} & 2008 & Journal of Scheduling & 18 & \noindent{}\textbf{1.00} \textbf{1.00} \textbf{23.20} & 113 123 144 & 31 52 & 25 10 15\\
Rodriguez07 \href{https://www.sciencedirect.com/science/article/pii/S0191261506000233}{Rodriguez07} & \hyperref[auth:a780]{J. Rodriguez} & A constraint programming model for real-time train scheduling at junctions \hyperref[abs:Rodriguez07]{Abstract} & \hyperref[detail:Rodriguez07]{Details} \href{../works/Rodriguez07.pdf}{Yes} & \cite{Rodriguez07} & 2007 & Transportation Research Part B: Methodological & 15 & \noindent{}\textbf{1.00} \textbf{1.50} \textbf{6.88} & 117 121 141 & 6 14 & 11 9 2\\
KuB16 \href{https://doi.org/10.1016/j.cor.2016.04.006}{KuB16} & \hyperref[auth:a331]{W.-Y. Ku}, \hyperref[auth:a89]{J. C. Beck} & \cellcolor{green!10}Mixed Integer Programming models for job shop scheduling: {A} computational analysis & \hyperref[detail:KuB16]{Details} \href{../works/KuB16.pdf}{Yes} & \cite{KuB16} & 2016 & Computers \  Operations Research & 9 & \noindent{}\textcolor{black!50}{0.00} \textcolor{black!50}{0.00} \textbf{5.53} & 119 132 141 & 17 25 & 25 19 6\\
Wallace96 \href{https://doi.org/10.1007/BF00143881}{Wallace96} & \hyperref[auth:a117]{M. G. Wallace} & Practical Applications of Constraint Programming & \hyperref[detail:Wallace96]{Details} \href{../works/Wallace96.pdf}{Yes} & \cite{Wallace96} & 1996 & Constraints An Int. J. & 30 & \noindent{}\textcolor{black!50}{0.00} \textcolor{black!50}{0.00} \textbf{16.50} & 87 89 138 & 55 143 & 20 12 8\\
CorreaLR07 \href{http://dx.doi.org/10.1016/j.cor.2005.07.004}{CorreaLR07} & \hyperref[auth:a947]{A. I. Corr{\'{e}}a}, \hyperref[auth:a644]{A. Langevin}, \hyperref[auth:a326]{L.-M. Rousseau} & Scheduling and routing of automated guided vehicles: A hybrid approach & \hyperref[detail:CorreaLR07]{Details} \href{../works/CorreaLR07.pdf}{Yes} & \cite{CorreaLR07} & 2007 & Computers \  Operations Research & 20 & \noindent{}\textcolor{black!50}{0.00} \textcolor{black!50}{0.00} \textbf{9.26} & 106 114 137 & 20 28 & 13 4 9\\
SadehF96 \href{http://dx.doi.org/10.1016/0004-3702(95)00098-4}{SadehF96} & \hyperref[auth:a1042]{N. M. Sadeh}, \hyperref[auth:a302]{M. S. Fox} & \cellcolor{gold!20}Variable and value ordering heuristics for the job shop scheduling constraint satisfaction problem & \hyperref[detail:SadehF96]{Details} \href{../works/SadehF96.pdf}{Yes} & \cite{SadehF96} & 1996 & Artificial Intelligence & 41 & \noindent{}\textbf{2.50} \textbf{2.50} \textbf{22.27} & 95 97 131 & 17 56 & 19 16 3\\
MaraveliasCG04 \href{http://dx.doi.org/10.1016/j.compchemeng.2004.03.016}{MaraveliasCG04} & \hyperref[auth:a381]{C. T. Maravelias}, \hyperref[auth:a382]{I. E. Grossmann} & A hybrid MILP/CP decomposition approach for the continuous time scheduling of multipurpose batch plants & \hyperref[detail:MaraveliasCG04]{Details} \href{../works/MaraveliasCG04.pdf}{Yes} & \cite{MaraveliasCG04} & 2004 & Computers \  Chemical Engineering & 29 & \noindent{}\textbf{1.00} \textbf{1.00} \textbf{49.17} & 116 119 130 & 24 29 & 29 23 6\\
Kuchcinski03 \href{http://dx.doi.org/10.1145/785411.785416}{Kuchcinski03} & \hyperref[auth:a659]{K. Kuchcinski} & Constraints-driven scheduling and resource assignment & \hyperref[detail:Kuchcinski03]{Details} \href{../works/Kuchcinski03.pdf}{Yes} & \cite{Kuchcinski03} & 2003 & ACM Transactions on Design Automation of Electronic Systems & 29 & \noindent{}\textcolor{black!50}{0.00} \textcolor{black!50}{0.00} \textbf{11.59} & 105 105 116 & 15 42 & 13 11 2\\
MeskensDL13 \href{http://dx.doi.org/10.1016/j.dss.2012.10.019}{MeskensDL13} & \hyperref[auth:a596]{N. Meskens}, \hyperref[auth:a597]{D. Duvivier}, \hyperref[auth:a1459]{A. Lianset} & Multi-objective operating room scheduling considering desiderata of the surgical team \hyperref[abs:MeskensDL13]{Abstract} & \hyperref[detail:MeskensDL13]{Details} \href{../works/MeskensDL13.pdf}{Yes} & \cite{MeskensDL13} & 2013 & DECISION SUPPORT SYSTEMS & 10 & \noindent{}\textcolor{black!50}{0.00} \textbf{1.00} \textbf{1.50} & 102 102 116 & 31 39 & 5 5 0\\
HookerO99 \href{http://dx.doi.org/10.1016/s0166-218x(99)00100-6}{HookerO99} & \hyperref[auth:a160]{J. N. Hooker}, \hyperref[auth:a1152]{M. Osorio} & \cellcolor{gold!20}Mixed logical-linear programming & \hyperref[detail:HookerO99]{Details} \href{../works/HookerO99.pdf}{Yes} & \cite{HookerO99} & 1999 & Discrete Applied Mathematics & 48 & \noindent{}\textcolor{black!50}{0.00} \textcolor{black!50}{0.00} \textbf{1.61} & 92 95 111 & 48 75 & 19 19 0\\
JussienL02 \href{http://dx.doi.org/10.1016/s0004-3702(02)00221-7}{JussienL02} & \hyperref[auth:a247]{N. Jussien}, \hyperref[auth:a1071]{O. Lhomme} & \cellcolor{gold!20}Local search with constraint propagation and conflict-based heuristics & \hyperref[detail:JussienL02]{Details} \href{../works/JussienL02.pdf}{Yes} & \cite{JussienL02} & 2002 & Artificial Intelligence & 25 & \noindent{}\textcolor{black!50}{0.00} \textcolor{black!50}{0.00} \textbf{4.25} & 88 88 108 & 16 54 & 15 8 7\\
SakkoutW00 \href{https://doi.org/10.1023/A:1009856210543}{SakkoutW00} & \hyperref[auth:a166]{H. E. Sakkout}, \hyperref[auth:a117]{M. G. Wallace} & Probe Backtrack Search for Minimal Perturbation in Dynamic Scheduling & \hyperref[detail:SakkoutW00]{Details} \href{../works/SakkoutW00.pdf}{Yes} & \cite{SakkoutW00} & 2000 & Constraints An Int. J. & 30 & \noindent{}\textcolor{black!50}{0.00} \textcolor{black!50}{0.00} \textbf{7.62} & 73 0 105 & 0 0 & 18 18 0\\
Pape94 \href{http://dx.doi.org/10.1049/ise.1994.0009}{Pape94} & \hyperref[auth:a163]{C. L. Pape} & Implementation of resource constraints in ILOG SCHEDULE: a library for the development of constraint-based scheduling systems & \hyperref[detail:Pape94]{Details} \href{../works/Pape94.pdf}{Yes} & \cite{Pape94} & 1994 & Intelligent Systems Engineering & 34 & \noindent{}\textcolor{black!50}{0.00} \textcolor{black!50}{0.00} \textbf{12.56} & 98 98 103 & 0 53 & 38 38 0\\
DincbasSH90 \href{https://doi.org/10.1016/0743-1066(90)90052-7}{DincbasSH90} & \hyperref[auth:a716]{M. Dincbas}, \hyperref[auth:a17]{H. Simonis}, \hyperref[auth:a148]{P. V. Hentenryck} & \cellcolor{gold!20}Solving Large Combinatorial Problems in Logic Programming & \hyperref[detail:DincbasSH90]{Details} \href{../works/DincbasSH90.pdf}{Yes} & \cite{DincbasSH90} & 1990 & The Journal of Logic Programming & 19 & \noindent{}\textcolor{black!50}{0.00} \textcolor{black!50}{0.00} \textbf{1.34} & 86 85 99 & 9 28 & 17 15 2\\
\end{longtable}
}



\clearpage
\section{Most Relevant Works}

{\scriptsize
\begin{longtable}{>{\raggedright\arraybackslash}p{2.5cm}>{\raggedright\arraybackslash}p{4.5cm}>{\raggedright\arraybackslash}p{6.0cm}p{1.0cm}rr>{\raggedright\arraybackslash}p{2.0cm}r>{\raggedright\arraybackslash}p{1cm}p{1cm}p{1cm}p{1cm}}
\rowcolor{white}\caption{Most Relevant Works (Total 30)}\\ \toprule
\rowcolor{white}\shortstack{Key\\Source} & Authors & Title (Colored by Open Access)& \shortstack{Details\\LC} & Cite & Year & \shortstack{Conference\\/Journal\\/School} & Pages & Relevance &\shortstack{Cites\\OC XR\\SC} & \shortstack{Refs\\OC\\XR} & \shortstack{Links\\Cites\\Refs}\\ \midrule\endhead
\bottomrule
\endfoot
Baptiste02 \href{https://theses.hal.science/tel-00124998}{Baptiste02} & \hyperref[auth:a162]{P. Baptiste} & {R{\'e}sultats de complexit{\'e} et programmation par contraintes pour l'ordonnancement} & \hyperref[detail:Baptiste02]{Details} \href{../works/Baptiste02.pdf}{Yes} & \cite{Baptiste02} & 2002 & {Universit{\'e} de Technologie de Compi{\`e}gne} & 237 & \noindent{}\textcolor{black!50}{0.00} \textcolor{black!50}{0.00} \textbf{1096.83} & 0 0 0 & 0 0 & 0 0 0\\
ZarandiASC20 \href{https://doi.org/10.1007/s10462-018-9667-6}{ZarandiASC20} & \hyperref[auth:a828]{M. H. F. Zarandi}, \hyperref[auth:a829]{A. A. S. Asl}, \hyperref[auth:a830]{S. Sotudian}, \hyperref[auth:a831]{O. Castillo} & A state of the art review of intelligent scheduling & \hyperref[detail:ZarandiASC20]{Details} \href{../works/ZarandiASC20.pdf}{Yes} & \cite{ZarandiASC20} & 2020 & Artif. Intell. Rev. & 93 & \noindent{}\textcolor{black!50}{0.00} \textcolor{black!50}{0.00} \textbf{440.67} & 55 64 66 & 445 538 & 66 3 63\\
abs-1902-09244 \href{http://arxiv.org/abs/1902.09244}{abs-1902-09244} & \hyperref[auth:a549]{V. A. Hauder}, \hyperref[auth:a550]{A. Beham}, \hyperref[auth:a551]{S. Raggl}, \hyperref[auth:a552]{S. N. Parragh}, \hyperref[auth:a553]{M. Affenzeller} & On constraint programming for a new flexible project scheduling problem with resource constraints & \hyperref[detail:abs-1902-09244]{Details} \href{../works/abs-1902-09244.pdf}{Yes} & \cite{abs-1902-09244} & 2019 & CoRR & 62 & \noindent{}\textbf{1.50} \textbf{1.50} \textbf{350.76} & 0 0 0 & 0 0 & 0 0 0\\
Groleaz21 \href{https://hal.science/tel-03266690}{Groleaz21} & \hyperref[auth:a83]{L. Groleaz} & {The Group Cumulative Scheduling Problem} & \hyperref[detail:Groleaz21]{Details} \href{../works/Groleaz21.pdf}{Yes} & \cite{Groleaz21} & 2021 & {Universit{\'e} de Lyon} & 153 & \noindent{}\textcolor{black!50}{0.00} \textcolor{black!50}{0.00} \textbf{331.76} & 0 0 0 & 0 0 & 0 0 0\\
Astrand21 \href{https://nbn-resolving.org/urn:nbn:se:kth:diva-294959}{Astrand21} & \hyperref[auth:a74]{M. {\AA}strand} & Short-term Underground Mine Scheduling: An Industrial Application of Constraint Programming & \hyperref[detail:Astrand21]{Details} \href{../works/Astrand21.pdf}{Yes} & \cite{Astrand21} & 2021 & Royal Institute of Technology, Stockholm, Sweden & 142 & \noindent{}\textbf{1.00} \textbf{1.00} \textbf{310.47} & 0 0 0 & 0 0 & 0 0 0\\
Beck99 \href{https://librarysearch.library.utoronto.ca/permalink/01UTORONTO_INST/14bjeso/alma991106162342106196}{Beck99} & \hyperref[auth:a89]{J. C. Beck} & Texture measurements as a basis for heuristic commitment techniques in constraint-directed scheduling & \hyperref[detail:Beck99]{Details} \href{../works/Beck99.pdf}{Yes} & \cite{Beck99} & 1999 & University of Toronto, Canada & 418 & \noindent{}\textcolor{black!50}{0.00} \textcolor{black!50}{0.00} \textbf{270.43} & 0 0 0 & 0 0 & 0 0 0\\
Dejemeppe16 \href{https://hdl.handle.net/2078.1/178078}{Dejemeppe16} & \hyperref[auth:a202]{C. Dejemeppe} & Constraint programming algorithms and models for scheduling applications & \hyperref[detail:Dejemeppe16]{Details} \href{../works/Dejemeppe16.pdf}{Yes} & \cite{Dejemeppe16} & 2016 & Catholic University of Louvain, Louvain-la-Neuve, Belgium & 274 & \noindent{}\textbf{1.00} \textbf{1.00} \textbf{262.14} & 0 0 0 & 0 0 & 0 0 0\\
Lombardi10 \href{http://amsdottorato.unibo.it/2961/}{Lombardi10} & \hyperref[auth:a142]{M. Lombardi} & Hybrid Methods for Resource Allocation and Scheduling Problems in Deterministic and Stochastic Environments & \hyperref[detail:Lombardi10]{Details} \href{../works/Lombardi10.pdf}{Yes} & \cite{Lombardi10} & 2010 & University of Bologna, Italy & 175 & \noindent{}\textcolor{black!50}{0.00} \textcolor{black!50}{0.00} \textbf{251.65} & 0 0 0 & 0 0 & 0 0 0\\
Lunardi20 \href{http://orbilu.uni.lu/handle/10993/43893}{Lunardi20} & \hyperref[auth:a495]{W. T. Lunardi} & A Real-World Flexible Job Shop Scheduling Problem With Sequencing Flexibility: Mathematical Programming, Constraint Programming, and Metaheuristics & \hyperref[detail:Lunardi20]{Details} \href{../works/Lunardi20.pdf}{Yes} & \cite{Lunardi20} & 2020 & University of Luxembourg, Luxembourg City, Luxembourg & 181 & \noindent{}\textbf{2.00} \textbf{2.00} \textbf{239.22} & 0 0 0 & 0 0 & 0 0 0\\
NaderiRR23 \href{https://doi.org/10.1287/ijoc.2023.1287}{NaderiRR23} & \hyperref[auth:a725]{B. Naderi}, \hyperref[auth:a726]{R. Ruiz}, \hyperref[auth:a727]{V. Roshanaei} & Mixed-Integer Programming vs. Constraint Programming for Shop Scheduling Problems: New Results and Outlook & \hyperref[detail:NaderiRR23]{Details} \href{../works/NaderiRR23.pdf}{Yes} & \cite{NaderiRR23} & 2023 & \cellcolor{red!20}INFORMS Journal on Computing & 27 & \noindent{}\textbf{1.00} \textbf{1.00} \textbf{184.97} & 2 7 7 & 50 55 & 22 1 21\\
Godet21a \href{https://tel.archives-ouvertes.fr/tel-03681868}{Godet21a} & \hyperref[auth:a470]{A. Godet} & Sur le tri de t{\^{a}}ches pour r{\'{e}}soudre des probl{\`{e}}mes d'ordonnancement avec la programmation par contraintes. (On the use of tasks ordering to solve scheduling problems with constraint programming) & \hyperref[detail:Godet21a]{Details} \href{../works/Godet21a.pdf}{Yes} & \cite{Godet21a} & 2021 & {IMT} Atlantique Bretagne Pays de la Loire, Brest, France & 168 & \noindent{}\textbf{2.50} \textbf{2.50} \textbf{172.67} & 0 0 0 & 0 0 & 0 0 0\\
Fahimi16 \href{http://cp2014.a4cp.org/sites/default/files/hamed_fahimi_-_efficient_algorithms_to_solve_scheduling_problems_with_a_variety_of_optimization_criteria.pdf}{Fahimi16} & \hyperref[auth:a122]{H. Fahimi} & Efficient algorithms to solve scheduling problems with a variety of optimization criteria & \hyperref[detail:Fahimi16]{Details} \href{../works/Fahimi16.pdf}{Yes} & \cite{Fahimi16} & 2016 & Universit{\'{e}} Laval, Quebec, Canada & 120 & \noindent{}\textcolor{black!50}{0.00} \textcolor{black!50}{0.00} \textbf{142.81} & 0 0 0 & 0 0 & 0 0 0\\
Malapert11 \href{https://tel.archives-ouvertes.fr/tel-00630122}{Malapert11} & \hyperref[auth:a82]{A. Malapert} & Techniques d'ordonnancement d'atelier et de fourn{\'{e}}es bas{\'{e}}es sur la programmation par contraintes. (Shop and batch scheduling with constraints) & \hyperref[detail:Malapert11]{Details} \href{../works/Malapert11.pdf}{Yes} & \cite{Malapert11} & 2011 & {\'{E}}cole des mines de Nantes, France & 194 & \noindent{}\textcolor{black!50}{0.00} \textcolor{black!50}{0.00} \textbf{142.49} & 0 0 0 & 0 0 & 0 0 0\\
Lim2015 \href{http://dx.doi.org/10.1145/2668930.2688058}{Lim2015} & \hyperref[auth:a2001]{N. Lim}, \hyperref[auth:a2002]{S. Majumdar}, \hyperref[auth:a2003]{P. Ashwood-Smith} & A Constraint Programming Based Hadoop Scheduler for Handling MapReduce Jobs with Deadlines on Clouds & \hyperref[detail:Lim2015]{Details} \href{../works/Lim2015.pdf}{Yes} & \cite{Lim2015} & 2015 & ICPE 2015 & 12 & \noindent{}\textbf{1.00} \textbf{1.00} \textbf{136.40} & 8 8 11 & 12 19 & 1 0 1\\
Froger16 \href{https://theses.hal.science/tel-01440836}{Froger16} & \hyperref[auth:a887]{A. Froger} & {Maintenance scheduling in the electricity industry : a particular focus on a problem rising in the onshore wind industry} & \hyperref[detail:Froger16]{Details} \href{../works/Froger16.pdf}{Yes} & \cite{Froger16} & 2016 & {Universit{\'e} d'Angers} & 181 & \noindent{}\textcolor{black!50}{0.00} \textcolor{black!50}{0.00} \textbf{127.37} & 0 0 0 & 0 0 & 0 0 0\\
AlesioBNG15 \href{http://dx.doi.org/10.1145/2818640}{AlesioBNG15} & \hyperref[auth:a1222]{S. D. Alesio}, \hyperref[auth:a236]{L. C. Briand}, \hyperref[auth:a235]{S. Nejati}, \hyperref[auth:a195]{A. Gotlieb} & \cellcolor{green!10}Combining Genetic Algorithms and Constraint Programming to Support Stress Testing of Task Deadlines & \hyperref[detail:AlesioBNG15]{Details} \href{../works/AlesioBNG15.pdf}{Yes} & \cite{AlesioBNG15} & 2015 & ACM Transactions on Software Engineering and Methodology & 37 & \noindent{}\textbf{1.00} \textbf{1.00} \textbf{119.50} & 13 14 17 & 51 59 & 10 0 10\\
Schutt11 \href{https://www.a4cp.org/sites/default/files/andreas_schutt_-_improving_scheduling_by_learning.pdf}{Schutt11} & \hyperref[auth:a124]{A. Schutt} & Improving Scheduling by Learning & \hyperref[detail:Schutt11]{Details} \href{../works/Schutt11.pdf}{Yes} & \cite{Schutt11} & 2011 & University of Melbourne, Australia & 209 & \noindent{}\textcolor{black!50}{0.00} \textcolor{black!50}{0.00} \textbf{102.66} & 0 0 0 & 0 0 & 0 0 0\\
Siala15a \href{https://tel.archives-ouvertes.fr/tel-01164291}{Siala15a} & \hyperref[auth:a129]{M. Siala} & Search, propagation, and learning in sequencing and scheduling problems. (Recherche, propagation et apprentissage dans les probl{\`{e}}mes de s{\'{e}}quencement et d'ordonnancement) & \hyperref[detail:Siala15a]{Details} \href{../works/Siala15a.pdf}{Yes} & \cite{Siala15a} & 2015 & {INSA} Toulouse, France & 199 & \noindent{}0.50 0.50 \textbf{98.99} & 0 0 0 & 0 0 & 0 0 0\\
IsikYA23 \href{https://doi.org/10.1007/s00500-023-09086-9}{IsikYA23} & \hyperref[auth:a419]{E. E. Isik}, \hyperref[auth:a420]{S. T. Yildiz}, \hyperref[auth:a421]{{\"{O}}zge S. Akpunar} & Constraint programming models for the hybrid flow shop scheduling problem and its extensions & \hyperref[detail:IsikYA23]{Details} \href{../works/IsikYA23.pdf}{Yes} & \cite{IsikYA23} & 2023 & Soft Computing & 28 & \noindent{}\textbf{1.00} \textbf{1.00} \textbf{80.43} & 0 2 2 & 127 141 & 12 0 12\\
KanetAG04 \href{http://www.crcnetbase.com/doi/abs/10.1201/9780203489802.ch47}{KanetAG04} & \hyperref[auth:a661]{J. J. Kanet}, \hyperref[auth:a662]{S. Ahire}, \hyperref[auth:a663]{M. F. Gorman} & Constraint Programming for Scheduling & \hyperref[detail:KanetAG04]{Details} \href{../works/KanetAG04.pdf}{Yes} & \cite{KanetAG04} & 2004 & Handbook of Scheduling - Algorithms, Models, and Performance Analysis & 22 & \noindent{}\textbf{1.00} \textbf{1.00} \textbf{73.85} & 0 0 0 & 0 0 & 0 0 0\\
GokgurHO18 \href{https://doi.org/10.1080/00207543.2017.1421781}{GokgurHO18} & \hyperref[auth:a568]{B. G{\"{o}}kg{\"{u}}r}, \hyperref[auth:a137]{B. Hnich}, \hyperref[auth:a569]{S. {\"{O}}zpeynirci} & Parallel machine scheduling with tool loading: a constraint programming approach & \hyperref[detail:GokgurHO18]{Details} \href{../works/GokgurHO18.pdf}{Yes} & \cite{GokgurHO18} & 2018 & \cellcolor{red!20}International Journal of Production Research & 17 & \noindent{}\textbf{1.50} \textbf{1.50} \textbf{73.63} & 31 40 51 & 43 62 & 23 8 15\\
LuZZYW24 \href{https://www.mdpi.com/2077-1312/12/1/124}{LuZZYW24} & \hyperref[auth:a1249]{X. Lu}, \hyperref[auth:a1250]{Y. Zhang}, \hyperref[auth:a1251]{L. Zheng}, \hyperref[auth:a1252]{C. Yang}, \hyperref[auth:a1253]{J. Wang} & \cellcolor{gold!20}Integrated Inbound and Outbound Scheduling for Coal Port: Constraint Programming and Adaptive Local Search & \hyperref[detail:LuZZYW24]{Details} \href{../works/LuZZYW24.pdf}{Yes} & \cite{LuZZYW24} & 2024 & Journal of Marine Science and Engineering & 36 & \noindent{}\textbf{1.00} \textbf{1.00} \textbf{71.08} & 0 0 0 & 0 57 & 0 0 0\\
YunusogluY22 \href{https://doi.org/10.1080/00207543.2021.1885068}{YunusogluY22} & \hyperref[auth:a449]{P. Yunusoglu}, \hyperref[auth:a420]{S. T. Yildiz} & Constraint programming approach for multi-resource-constrained unrelated parallel machine scheduling problem with sequence-dependent setup times & \hyperref[detail:YunusogluY22]{Details} \href{../works/YunusogluY22.pdf}{Yes} & \cite{YunusogluY22} & 2022 & \cellcolor{red!20}International Journal of Production Research & 18 & \noindent{}\textbf{2.00} \textbf{2.00} \textbf{69.33} & 20 36 40 & 58 64 & 16 6 10\\
Nuijten94 \href{https://pure.tue.nl/ws/portalfiles/portal/2374269/431902.pdf}{Nuijten94} & \hyperref[auth:a655]{W. Nuijten} & Time and Resource Constrained Scheduling: a Constraint Satisfaction Approach & \hyperref[detail:Nuijten94]{Details} \href{../works/Nuijten94.pdf}{Yes} & \cite{Nuijten94} & 1994 & Eindhoven University of Technology & 172 & \noindent{}\textbf{1.50} \textbf{1.50} \textbf{68.38} & 0 0 0 & 0 0 & 0 0 0\\
BosiM2001 \href{http://dx.doi.org/10.1002/1097-024x(200101)31:1<17::aid-spe355>3.0.co;2-l}{BosiM2001} & \hyperref[auth:a1223]{F. Bosi}, \hyperref[auth:a143]{M. Milano} & Enhancing CLP branch and bound techniques for scheduling problems & \hyperref[detail:BosiM2001]{Details} \href{../works/BosiM2001.pdf}{Yes} & \cite{BosiM2001} & 2001 & Software: Practice and Experience & 26 & \noindent{}\textbf{1.00} \textbf{1.00} \textbf{65.58} & 3 3 0 & 12 41 & 9 0 9\\
TranVNB17 \href{https://doi.org/10.1613/jair.5306}{TranVNB17} & \hyperref[auth:a798]{T. T. Tran}, \hyperref[auth:a803]{T. S. Vaquero}, \hyperref[auth:a204]{G. Nejat}, \hyperref[auth:a89]{J. C. Beck} & \cellcolor{gold!20}Robots in Retirement Homes: Applying Off-the-Shelf Planning and Scheduling to a Team of Assistive Robots & \hyperref[detail:TranVNB17]{Details} \href{../works/TranVNB17.pdf}{Yes} & \cite{TranVNB17} & 2017 & Journal of Artificial Intelligence Research & 68 & \noindent{}\textcolor{black!50}{0.00} \textcolor{black!50}{0.00} \textbf{61.11} & 12 12 21 & 0 0 & 2 2 0\\
Edis21 \href{http://dx.doi.org/10.1016/j.cor.2020.105111}{Edis21} & \hyperref[auth:a346]{E. B. Edis} & Constraint programming approaches to disassembly line balancing problem with sequencing decisions & \hyperref[detail:Edis21]{Details} \href{../works/Edis21.pdf}{Yes} & \cite{Edis21} & 2021 & Computers \  Operations Research & 20 & \noindent{}\textcolor{black!50}{0.00} \textcolor{black!50}{0.00} \textbf{60.40} & 13 19 20 & 48 53 & 10 2 8\\
BartakSR10 \href{https://doi.org/10.1017/S0269888910000202}{BartakSR10} & \hyperref[auth:a152]{R. Bart{\'{a}}k}, \hyperref[auth:a153]{M. A. Salido}, \hyperref[auth:a316]{F. Rossi} & \cellcolor{green!10}New trends in constraint satisfaction, planning, and scheduling: a survey & \hyperref[detail:BartakSR10]{Details} \href{../works/BartakSR10.pdf}{Yes} & \cite{BartakSR10} & 2010 & The Knowledge Engineering Review & 31 & \noindent{}\textbf{1.00} \textbf{1.00} \textbf{59.34} & 28 29 31 & 47 88 & 18 4 14\\
AbreuN22 \href{https://doi.org/10.1016/j.cie.2022.108128}{AbreuN22} & \hyperref[auth:a418]{L. R. de Abreu}, \hyperref[auth:a387]{M. S. Nagano} & A new hybridization of adaptive large neighborhood search with constraint programming for open shop scheduling with sequence-dependent setup times & \hyperref[detail:AbreuN22]{Details} \href{../works/AbreuN22.pdf}{Yes} & \cite{AbreuN22} & 2022 & Computers \  Industrial Engineering & 20 & \noindent{}\textbf{1.00} \textbf{1.00} \textbf{57.71} & 10 14 13 & 56 74 & 7 2 5\\
LaborieRSV18 \href{https://doi.org/10.1007/s10601-018-9281-x}{LaborieRSV18} & \hyperref[auth:a118]{P. Laborie}, \hyperref[auth:a119]{J. Rogerie}, \hyperref[auth:a120]{P. Shaw}, \hyperref[auth:a121]{P. Vil{\'{\i}}m} & {IBM} {ILOG} {CP} optimizer for scheduling - 20+ years of scheduling with constraints at {IBM/ILOG} & \hyperref[detail:LaborieRSV18]{Details} \href{../works/LaborieRSV18.pdf}{Yes} & \cite{LaborieRSV18} & 2018 & Constraints An Int. J. & 41 & \noindent{}\textbf{1.00} \textbf{1.00} \textbf{54.37} & 148 178 203 & 35 54 & 92 69 23\\
\end{longtable}
}



\clearpage
\section{Most Connected Works}

{\scriptsize
\begin{longtable}{>{\raggedright\arraybackslash}p{2.5cm}>{\raggedright\arraybackslash}p{4.5cm}>{\raggedright\arraybackslash}p{6.0cm}p{1.0cm}rr>{\raggedright\arraybackslash}p{2.0cm}r>{\raggedright\arraybackslash}p{1cm}p{1cm}p{1cm}p{1cm}}
\rowcolor{white}\caption{Most Connected Works (Total 30)}\\ \toprule
\rowcolor{white}\shortstack{Key\\Source} & Authors & Title (Colored by Open Access)& \shortstack{Details\\LC} & Cite & Year & \shortstack{Conference\\/Journal\\/School} & Pages & Relevance &\shortstack{Cites\\OC XR\\SC} & \shortstack{Refs\\OC\\XR} & \shortstack{Links\\Cites\\Refs}\\ \midrule\endhead
\bottomrule
\endfoot
BaptistePN01 \href{http://dx.doi.org/10.1007/978-1-4615-1479-4}{BaptistePN01} & \hyperref[auth:a162]{P. Baptiste}, \hyperref[auth:a163]{C. L. Pape}, \hyperref[auth:a656]{W. Nuijten} & Constraint-Based Scheduling & \hyperref[detail:BaptistePN01]{Details} No & \cite{BaptistePN01} & 2001 & Book & null & \noindent{}\textcolor{black!50}{0.00} \textcolor{black!50}{0.00} n/a & 296 302 0 & 0 0 & 118 118 0\\
AggounB93 \href{https://www.sciencedirect.com/science/article/pii/089571779390068A}{AggounB93} & \hyperref[auth:a725]{A. Aggoun}, \hyperref[auth:a128]{N. Beldiceanu} & \cellcolor{gold!20}Extending {CHIP} in order to solve complex scheduling and placement problems \hyperref[abs:AggounB93]{Abstract} & \hyperref[detail:AggounB93]{Details} \href{../works/AggounB93.pdf}{Yes} & \cite{AggounB93} & 1993 & Mathematical and Computer Modelling & 17 & \noindent{}\textcolor{black!50}{0.00} \textbf{3.00} \textbf{3.18} & 187 191 214 & 11 36 & 91 89 2\\
JainG01 \href{http://dx.doi.org/10.1287/ijoc.13.4.258.9733}{JainG01} & \hyperref[auth:a844]{V. Jain}, \hyperref[auth:a382]{I. E. Grossmann} & Algorithms for Hybrid MILP/CP Models for a Class of Optimization Problems & \hyperref[detail:JainG01]{Details} \href{../works/JainG01.pdf}{Yes} & \cite{JainG01} & 2001 & \cellcolor{red!20}INFORMS Journal on Computing & 19 & \noindent{}\textcolor{black!50}{0.00} \textcolor{black!50}{0.00} \textbf{29.84} & 279 284 321 & 23 38 & 101 89 12\\
HookerO03 \href{http://dx.doi.org/10.1007/s10107-003-0375-9}{HookerO03} & \hyperref[auth:a160]{J. N. Hooker}, \hyperref[auth:a852]{G. Ottosson} & \cellcolor{green!10}Logic-based Benders decomposition & \hyperref[detail:HookerO03]{Details} \href{../works/HookerO03.pdf}{Yes} & \cite{HookerO03} & 2003 & Mathematical Programming & 28 & \noindent{}\textcolor{black!50}{0.00} \textcolor{black!50}{0.00} 0.61 & 317 333 371 & 0 0 & 78 78 0\\
LaborieRSV18 \href{https://doi.org/10.1007/s10601-018-9281-x}{LaborieRSV18} & \hyperref[auth:a118]{P. Laborie}, \hyperref[auth:a119]{J. Rogerie}, \hyperref[auth:a120]{P. Shaw}, \hyperref[auth:a121]{P. Vil{\'{\i}}m} & {IBM} {ILOG} {CP} optimizer for scheduling - 20+ years of scheduling with constraints at {IBM/ILOG} & \hyperref[detail:LaborieRSV18]{Details} \href{../works/LaborieRSV18.pdf}{Yes} & \cite{LaborieRSV18} & 2018 & Constraints An Int. J. & 41 & \noindent{}\textbf{1.00} \textbf{1.00} \textbf{54.37} & 148 178 203 & 35 54 & 92 69 23\\
Hooker07 \href{http://dx.doi.org/10.1287/opre.1060.0371}{Hooker07} & \hyperref[auth:a160]{J. N. Hooker} & Planning and Scheduling by Logic-Based Benders Decomposition & \hyperref[detail:Hooker07]{Details} \href{../works/Hooker07.pdf}{Yes} & \cite{Hooker07} & 2007 & \cellcolor{red!20}Operations Research & 15 & \noindent{}\textcolor{black!50}{0.00} \textcolor{black!50}{0.00} \textbf{14.07} & 181 197 205 & 19 20 & 66 52 14\\
Hooker00 \href{http://dx.doi.org/10.1002/9781118033036}{Hooker00} & \hyperref[auth:a160]{J. N. Hooker} & Logic Based Methods for Optimization: Combining Optimization and Constraint Satisfaction & \hyperref[detail:Hooker00]{Details} No & \cite{Hooker00} & 2000 & Book & null & \noindent{}\textcolor{black!50}{0.00} \textcolor{black!50}{0.00} n/a & 185 186 0 & 0 0 & 44 44 0\\
Laborie03 \href{http://dx.doi.org/10.1016/s0004-3702(02)00362-4}{Laborie03} & \hyperref[auth:a118]{P. Laborie} & \cellcolor{gold!20}Algorithms for propagating resource constraints in AI planning and scheduling: Existing approaches and new results & \hyperref[detail:Laborie03]{Details} \href{../works/Laborie03.pdf}{Yes} & \cite{Laborie03} & 2003 & Artificial Intelligence & 38 & \noindent{}\textcolor{black!50}{0.00} \textcolor{black!50}{0.00} \textbf{8.42} & 128 129 175 & 10 31 & 48 43 5\\
HarjunkoskiG02 \href{http://dx.doi.org/10.1016/s0098-1354(02)00100-x}{HarjunkoskiG02} & \hyperref[auth:a871]{I. Harjunkoski}, \hyperref[auth:a382]{I. E. Grossmann} & Decomposition techniques for multistage scheduling problems using mixed-integer and constraint programming methods & \hyperref[detail:HarjunkoskiG02]{Details} \href{../works/HarjunkoskiG02.pdf}{Yes} & \cite{HarjunkoskiG02} & 2002 & Computers \  Chemical Engineering & 20 & \noindent{}\textbf{1.00} \textbf{1.00} \textbf{20.38} & 169 173 192 & 11 25 & 42 39 3\\
Pape94 \href{http://dx.doi.org/10.1049/ise.1994.0009}{Pape94} & \hyperref[auth:a163]{C. L. Pape} & Implementation of resource constraints in ILOG SCHEDULE: a library for the development of constraint-based scheduling systems & \hyperref[detail:Pape94]{Details} \href{../works/Pape94.pdf}{Yes} & \cite{Pape94} & 1994 & Intelligent Systems Engineering & 34 & \noindent{}\textcolor{black!50}{0.00} \textcolor{black!50}{0.00} \textbf{12.56} & 98 98 103 & 0 53 & 38 38 0\\
Thorsteinsson01 \href{https://doi.org/10.1007/3-540-45578-7_2}{Thorsteinsson01} & \hyperref[auth:a874]{E. S. Thorsteinsson} & Branch-and-Check: {A} Hybrid Framework Integrating Mixed Integer Programming and Constraint Logic Programming & \hyperref[detail:Thorsteinsson01]{Details} \href{../works/Thorsteinsson01.pdf}{Yes} & \cite{Thorsteinsson01} & 2001 & CP 2001 & 15 & \noindent{}\textcolor{black!50}{0.00} \textcolor{black!50}{0.00} \textbf{2.42} & 67 68 97 & 12 25 & 46 38 8\\
SchuttFSW11 \href{https://doi.org/10.1007/s10601-010-9103-2}{SchuttFSW11} & \hyperref[auth:a124]{A. Schutt}, \hyperref[auth:a154]{T. Feydy}, \hyperref[auth:a125]{P. J. Stuckey}, \hyperref[auth:a117]{M. G. Wallace} & Explaining the cumulative propagator & \hyperref[detail:SchuttFSW11]{Details} \href{../works/SchuttFSW11.pdf}{Yes} & \cite{SchuttFSW11} & 2011 & Constraints An Int. J. & 33 & \noindent{}\textcolor{black!50}{0.00} \textcolor{black!50}{0.00} \textbf{14.77} & 57 61 65 & 23 39 & 48 34 14\\
BeldiceanuC94 \href{https://www.sciencedirect.com/science/article/pii/0895717794901279}{BeldiceanuC94} & \hyperref[auth:a128]{N. Beldiceanu}, \hyperref[auth:a784]{E. Contejean} & \cellcolor{gold!20}Introducing Global Constraints in {CHIP} \hyperref[abs:BeldiceanuC94]{Abstract} & \hyperref[detail:BeldiceanuC94]{Details} \href{../works/BeldiceanuC94.pdf}{Yes} & \cite{BeldiceanuC94} & 1994 & Mathematical and Computer Modelling & 27 & \noindent{}\textcolor{black!50}{0.00} \textbf{1.00} \textbf{1.72} & 167 169 223 & 8 21 & 37 34 3\\
OhrimenkoSC09 \href{http://dx.doi.org/10.1007/s10601-008-9064-x}{OhrimenkoSC09} & \hyperref[auth:a861]{O. Ohrimenko}, \hyperref[auth:a125]{P. J. Stuckey}, \hyperref[auth:a862]{M. Codish} & Propagation via lazy clause generation & \hyperref[detail:OhrimenkoSC09]{Details} \href{../works/OhrimenkoSC09.pdf}{Yes} & \cite{OhrimenkoSC09} & 2009 & Constraints An Int. J. & 35 & \noindent{}\textcolor{black!50}{0.00} \textcolor{black!50}{0.00} \textbf{3.19} & 127 128 198 & 15 35 & 33 31 2\\
Hooker05 \href{https://doi.org/10.1007/s10601-005-2812-2}{Hooker05} & \hyperref[auth:a160]{J. N. Hooker} & \cellcolor{green!10}A Hybrid Method for the Planning and Scheduling & \hyperref[detail:Hooker05]{Details} \href{../works/Hooker05.pdf}{Yes} & \cite{Hooker05} & 2005 & Constraints An Int. J. & 17 & \noindent{}\textcolor{black!50}{0.00} \textcolor{black!50}{0.00} \textbf{7.71} & 68 69 87 & 11 18 & 40 30 10\\
BaptisteP00 \href{https://doi.org/10.1023/A:1009822502231}{BaptisteP00} & \hyperref[auth:a162]{P. Baptiste}, \hyperref[auth:a163]{C. L. Pape} & Constraint Propagation and Decomposition Techniques for Highly Disjunctive and Highly Cumulative Project Scheduling Problems & \hyperref[detail:BaptisteP00]{Details} \href{../works/BaptisteP00.pdf}{Yes} & \cite{BaptisteP00} & 2000 & Constraints An Int. J. & 21 & \noindent{}\textbf{1.50} \textbf{1.50} \textbf{14.93} & 46 0 62 & 0 0 & 29 29 0\\
BockmayrK98 \href{http://dx.doi.org/10.1287/ijoc.10.3.287}{BockmayrK98} & \hyperref[auth:a908]{A. Bockmayr}, \hyperref[auth:a1045]{T. Kasper} & Branch and Infer: A Unifying Framework for Integer and Finite Domain Constraint Programming & \hyperref[detail:BockmayrK98]{Details} No & \cite{BockmayrK98} & 1998 & \cellcolor{red!20}INFORMS Journal on Computing & 14 & \noindent{}\textcolor{black!50}{0.00} \textcolor{black!50}{0.00} n/a & 79 79 92 & 27 42 & 32 28 4\\
Vilim11 \href{https://doi.org/10.1007/978-3-642-21311-3_22}{Vilim11} & \hyperref[auth:a121]{P. Vil{\'{\i}}m} & Timetable Edge Finding Filtering Algorithm for Discrete Cumulative Resources & \hyperref[detail:Vilim11]{Details} \href{../works/Vilim11.pdf}{Yes} & \cite{Vilim11} & 2011 & CPAIOR 2011 & 16 & \noindent{}\textcolor{black!50}{0.00} \textcolor{black!50}{0.00} 0.64 & 28 29 46 & 6 11 & 31 26 5\\
MercierH08 \href{http://dx.doi.org/10.1287/ijoc.1070.0226}{MercierH08} & \hyperref[auth:a851]{L. Mercier}, \hyperref[auth:a148]{P. V. Hentenryck} & Edge Finding for Cumulative Scheduling & \hyperref[detail:MercierH08]{Details} \href{../works/MercierH08.pdf}{Yes} & \cite{MercierH08} & 2008 & \cellcolor{red!20}INFORMS Journal on Computing & 11 & \noindent{}\textcolor{black!50}{0.00} \textcolor{black!50}{0.00} 0.71 & 32 33 0 & 5 8 & 31 26 5\\
Hooker04 \href{https://doi.org/10.1007/978-3-540-30201-8_24}{Hooker04} & \hyperref[auth:a160]{J. N. Hooker} & \cellcolor{green!10}A Hybrid Method for Planning and Scheduling & \hyperref[detail:Hooker04]{Details} \href{../works/Hooker04.pdf}{Yes} & \cite{Hooker04} & 2004 & CP 2004 & 12 & \noindent{}\textcolor{black!50}{0.00} \textcolor{black!50}{0.00} \textbf{2.51} & 39 40 46 & 9 10 & 32 25 7\\
BeldiceanuC02 \href{https://doi.org/10.1007/3-540-46135-3_5}{BeldiceanuC02} & \hyperref[auth:a128]{N. Beldiceanu}, \hyperref[auth:a91]{M. Carlsson} & \cellcolor{green!10}A New Multi-resource cumulatives Constraint with Negative Heights & \hyperref[detail:BeldiceanuC02]{Details} \href{../works/BeldiceanuC02.pdf}{Yes} & \cite{BeldiceanuC02} & 2002 & CP 2002 & 17 & \noindent{}\textcolor{black!50}{0.00} \textcolor{black!50}{0.00} \textbf{1.31} & 33 33 48 & 9 20 & 30 25 5\\
SchuttFSW13 \href{https://doi.org/10.1007/s10951-012-0285-x}{SchuttFSW13} & \hyperref[auth:a124]{A. Schutt}, \hyperref[auth:a154]{T. Feydy}, \hyperref[auth:a125]{P. J. Stuckey}, \hyperref[auth:a117]{M. G. Wallace} & \cellcolor{green!10}Solving RCPSP/max by lazy clause generation & \hyperref[detail:SchuttFSW13]{Details} \href{../works/SchuttFSW13.pdf}{Yes} & \cite{SchuttFSW13} & 2013 & Journal of Scheduling & 17 & \noindent{}\textcolor{black!50}{0.00} \textcolor{black!50}{0.00} \textbf{5.21} & 43 45 57 & 23 38 & 36 25 11\\
Laborie09 \href{https://doi.org/10.1007/978-3-642-01929-6_12}{Laborie09} & \hyperref[auth:a118]{P. Laborie} & {IBM} {ILOG} {CP} Optimizer for Detailed Scheduling Illustrated on Three Problems & \hyperref[detail:Laborie09]{Details} \href{../works/Laborie09.pdf}{Yes} & \cite{Laborie09} & 2009 & CPAIOR 2009 & 15 & \noindent{}\textbf{1.00} \textbf{1.00} \textbf{8.01} & 53 52 91 & 2 9 & 25 24 1\\
NuijtenP98 \href{https://doi.org/10.1023/A:1009687210594}{NuijtenP98} & \hyperref[auth:a656]{W. Nuijten}, \hyperref[auth:a163]{C. L. Pape} & Constraint-Based Job Shop Scheduling with {\textbackslash}sc Ilog Scheduler & \hyperref[detail:NuijtenP98]{Details} \href{../works/NuijtenP98.pdf}{Yes} & \cite{NuijtenP98} & 1998 & J. Heuristics & 16 & \noindent{}\textcolor{black!50}{0.00} \textcolor{black!50}{0.00} \textbf{10.57} & 42 0 50 & 0 0 & 24 24 0\\
Vilim09 \href{https://doi.org/10.1007/978-3-642-04244-7_62}{Vilim09} & \hyperref[auth:a121]{P. Vil{\'{\i}}m} & Edge Finding Filtering Algorithm for Discrete Cumulative Resources in \emph{O}(\emph{kn} log \emph{n})\{{\textbackslash}mathcal O\}(kn \{{\textbackslash}rm log\} n) & \hyperref[detail:Vilim09]{Details} \href{../works/Vilim09.pdf}{Yes} & \cite{Vilim09} & 2009 & CP 2009 & 15 & \noindent{}\textcolor{black!50}{0.00} \textcolor{black!50}{0.00} 0.33 & 25 26 34 & 4 7 & 27 23 4\\
NuijtenA96 \href{http://dx.doi.org/10.1016/0377-2217(95)00354-1}{NuijtenA96} & \hyperref[auth:a656]{W. Nuijten}, \hyperref[auth:a777]{E. H. L. Aarts} & A computational study of constraint satisfaction for multiple capacitated job shop scheduling & \hyperref[detail:NuijtenA96]{Details} \href{../works/NuijtenA96.pdf}{Yes} & \cite{NuijtenA96} & 1996 & European Journal of Operational Research & 16 & \noindent{}\textbf{2.00} \textbf{2.00} \textbf{2.88} & 65 65 90 & 6 21 & 27 23 4\\
RodosekWH99 \href{http://dx.doi.org/10.1023/a:1018904229454}{RodosekWH99} & \hyperref[auth:a297]{R. Rodosek}, \hyperref[auth:a117]{M. G. Wallace}, \hyperref[auth:a1030]{M. Hajian} & A new approach to integrating mixed integer programming and constraint logic programming & \hyperref[detail:RodosekWH99]{Details} \href{../works/RodosekWH99.pdf}{Yes} & \cite{RodosekWH99} & 1999 & Annals of Operations Research & 25 & \noindent{}\textcolor{black!50}{0.00} \textcolor{black!50}{0.00} \textbf{8.11} & 53 0 67 & 0 0 & 23 23 0\\
MaraveliasCG04 \href{http://dx.doi.org/10.1016/j.compchemeng.2004.03.016}{MaraveliasCG04} & \hyperref[auth:a381]{C. T. Maravelias}, \hyperref[auth:a382]{I. E. Grossmann} & A hybrid MILP/CP decomposition approach for the continuous time scheduling of multipurpose batch plants & \hyperref[detail:MaraveliasCG04]{Details} \href{../works/MaraveliasCG04.pdf}{Yes} & \cite{MaraveliasCG04} & 2004 & Computers \  Chemical Engineering & 29 & \noindent{}\textbf{1.00} \textbf{1.00} \textbf{49.17} & 116 119 130 & 24 29 & 29 23 6\\
Hooker02 \href{http://dx.doi.org/10.1287/ijoc.14.4.295.2828}{Hooker02} & \hyperref[auth:a160]{J. N. Hooker} & Logic, Optimization, and Constraint Programming & \hyperref[detail:Hooker02]{Details} No & \cite{Hooker02} & 2002 & \cellcolor{red!20}INFORMS Journal on Computing & 27 & \noindent{}\textcolor{black!50}{0.00} \textcolor{black!50}{0.00} n/a & 94 93 0 & 84 149 & 40 22 18\\
BukchinR18 \href{http://dx.doi.org/10.1016/j.omega.2017.06.008}{BukchinR18} & \hyperref[auth:a1181]{Y. Bukchin}, \hyperref[auth:a1182]{T. Raviv} & Constraint programming for solving various assembly line balancing problems & \hyperref[detail:BukchinR18]{Details} \href{../works/BukchinR18.pdf}{Yes} & \cite{BukchinR18} & 2018 & Omega & 12 & \noindent{}\textcolor{black!50}{0.00} \textcolor{black!50}{0.00} \textbf{21.89} & 66 68 81 & 29 43 & 23 22 1\\
\end{longtable}
}



\clearpage
\section{Longest Works}

{\scriptsize
\begin{longtable}{>{\raggedright\arraybackslash}p{2.5cm}>{\raggedright\arraybackslash}p{4.5cm}>{\raggedright\arraybackslash}p{6.0cm}p{1.0cm}rr>{\raggedright\arraybackslash}p{2.0cm}r>{\raggedright\arraybackslash}p{1cm}p{1cm}p{1cm}p{1cm}}
\rowcolor{white}\caption{Longest Works (Total 30)}\\ \toprule
\rowcolor{white}\shortstack{Key\\Source} & Authors & Title (Colored by Open Access)& \shortstack{Details\\LC} & Cite & Year & \shortstack{Conference\\/Journal\\/School} & Pages & Relevance &\shortstack{Cites\\OC XR\\SC} & \shortstack{Refs\\OC\\XR} & \shortstack{Links\\Cites\\Refs}\\ \midrule\endhead
\bottomrule
\endfoot
Beck99 \href{https://librarysearch.library.utoronto.ca/permalink/01UTORONTO_INST/14bjeso/alma991106162342106196}{Beck99} & \hyperref[auth:a89]{J. C. Beck} & Texture measurements as a basis for heuristic commitment techniques in constraint-directed scheduling & \hyperref[detail:Beck99]{Details} \href{../works/Beck99.pdf}{Yes} & \cite{Beck99} & 1999 & University of Toronto, Canada & 418 & \noindent{}\textcolor{black!50}{0.00} \textcolor{black!50}{0.00} \textbf{270.43} & 0 0 0 & 0 0 & 0 0 0\\
Elkhyari03 \href{https://theses.hal.science/tel-00008377}{Elkhyari03} & \hyperref[auth:a292]{A. Elkhyari} & {Outils d'aide {\`a} la d{\'e}cision pour des probl{\`e}mes d'ordonnancement dynamiques} & \hyperref[detail:Elkhyari03]{Details} \href{../works/Elkhyari03.pdf}{Yes} & \cite{Elkhyari03} & 2003 & {Universit{\'e} de Nantes} & 333 & \noindent{}\textcolor{black!50}{0.00} \textcolor{black!50}{0.00} \textbf{24.65} & 0 0 0 & 0 0 & 0 0 0\\
Dejemeppe16 \href{https://hdl.handle.net/2078.1/178078}{Dejemeppe16} & \hyperref[auth:a202]{C. Dejemeppe} & Constraint programming algorithms and models for scheduling applications & \hyperref[detail:Dejemeppe16]{Details} \href{../works/Dejemeppe16.pdf}{Yes} & \cite{Dejemeppe16} & 2016 & Catholic University of Louvain, Louvain-la-Neuve, Belgium & 274 & \noindent{}\textbf{1.00} \textbf{1.00} \textbf{262.14} & 0 0 0 & 0 0 & 0 0 0\\
Baptiste02 \href{https://theses.hal.science/tel-00124998}{Baptiste02} & \hyperref[auth:a162]{P. Baptiste} & {R{\'e}sultats de complexit{\'e} et programmation par contraintes pour l'ordonnancement} & \hyperref[detail:Baptiste02]{Details} \href{../works/Baptiste02.pdf}{Yes} & \cite{Baptiste02} & 2002 & {Universit{\'e} de Technologie de Compi{\`e}gne} & 237 & \noindent{}\textcolor{black!50}{0.00} \textcolor{black!50}{0.00} \textbf{1096.83} & 0 0 0 & 0 0 & 0 0 0\\
Layfield02 \href{http://etheses.whiterose.ac.uk/1301/}{Layfield02} & \hyperref[auth:a669]{C. J. Layfield} & A constraint programming pre-processor for duty scheduling & \hyperref[detail:Layfield02]{Details} \href{../works/Layfield02.pdf}{Yes} & \cite{Layfield02} & 2002 & University of Leeds, {UK} & 230 & \noindent{}\textbf{1.00} \textbf{1.00} \textcolor{black!50}{0.00} & 0 0 0 & 0 0 & 0 0 0\\
Schutt11 \href{https://www.a4cp.org/sites/default/files/andreas_schutt_-_improving_scheduling_by_learning.pdf}{Schutt11} & \hyperref[auth:a124]{A. Schutt} & Improving Scheduling by Learning & \hyperref[detail:Schutt11]{Details} \href{../works/Schutt11.pdf}{Yes} & \cite{Schutt11} & 2011 & University of Melbourne, Australia & 209 & \noindent{}\textcolor{black!50}{0.00} \textcolor{black!50}{0.00} \textbf{102.66} & 0 0 0 & 0 0 & 0 0 0\\
Siala15a \href{https://tel.archives-ouvertes.fr/tel-01164291}{Siala15a} & \hyperref[auth:a129]{M. Siala} & Search, propagation, and learning in sequencing and scheduling problems. (Recherche, propagation et apprentissage dans les probl{\`{e}}mes de s{\'{e}}quencement et d'ordonnancement) & \hyperref[detail:Siala15a]{Details} \href{../works/Siala15a.pdf}{Yes} & \cite{Siala15a} & 2015 & {INSA} Toulouse, France & 199 & \noindent{}0.50 0.50 \textbf{98.99} & 0 0 0 & 0 0 & 0 0 0\\
Nattaf16 \href{https://laas.hal.science/tel-01417288}{Nattaf16} & \hyperref[auth:a81]{M. Nattaf} & {Ordonnancement sous contraintes d'{\'e}nergie} & \hyperref[detail:Nattaf16]{Details} \href{../works/Nattaf16.pdf}{Yes} & \cite{Nattaf16} & 2016 & {UPS Toulouse - Universit{\'e} Toulouse 3 Paul Sabatier} & 199 & \noindent{}\textcolor{black!50}{0.00} \textcolor{black!50}{0.00} \textbf{11.76} & 0 0 0 & 0 0 & 0 0 0\\
Clercq12 \href{https://theses.hal.science/tel-00794323}{Clercq12} & \hyperref[auth:a246]{A. D. Clercq} & {Ordonnancement cumulatif avec d{\'e}passements de capacit{\'e} : Contrainte globale et d{\'e}compositions} & \hyperref[detail:Clercq12]{Details} \href{../works/Clercq12.pdf}{Yes} & \cite{Clercq12} & 2012 & {Ecole des Mines de Nantes} & 196 & \noindent{}\textcolor{black!50}{0.00} \textcolor{black!50}{0.00} \textbf{6.18} & 0 0 0 & 0 0 & 0 0 0\\
Malapert11 \href{https://tel.archives-ouvertes.fr/tel-00630122}{Malapert11} & \hyperref[auth:a82]{A. Malapert} & Techniques d'ordonnancement d'atelier et de fourn{\'{e}}es bas{\'{e}}es sur la programmation par contraintes. (Shop and batch scheduling with constraints) & \hyperref[detail:Malapert11]{Details} \href{../works/Malapert11.pdf}{Yes} & \cite{Malapert11} & 2011 & {\'{E}}cole des mines de Nantes, France & 194 & \noindent{}\textcolor{black!50}{0.00} \textcolor{black!50}{0.00} \textbf{142.49} & 0 0 0 & 0 0 & 0 0 0\\
Caballero19 \href{https://www.tesisenred.net/handle/10803/667963#page=1}{Caballero19} & \hyperref[auth:a102]{J. C. Caballero} & Scheduling Through Logic-Based Tools & \hyperref[detail:Caballero19]{Details} \href{../works/Caballero19.pdf}{Yes} & \cite{Caballero19} & 2019 & Universitat de Girona, Spain & 194 & \noindent{}\textcolor{black!50}{0.00} \textcolor{black!50}{0.00} \textbf{43.36} & 0 0 0 & 0 0 & 0 0 0\\
Lemos21 \href{https://scholar.tecnico.ulisboa.pt/records/u5RPHM-pu_yoOLXJF7BHrgJx47D827b0xHb3}{Lemos21} & \hyperref[auth:a875]{Alexandre Duarte {de Almeida} Lemos} & Solving scheduling problems under disruptions & \hyperref[detail:Lemos21]{Details} \href{../works/Lemos21.pdf}{Yes} & \cite{Lemos21} & 2021 & UNIVERSIDADE DE LISBOA INSTITUTO SUPERIOR TÉCNICO & 188 & \noindent{}\textcolor{black!50}{0.00} \textcolor{black!50}{0.00} \textbf{17.57} & 0 0 0 & 0 0 & 0 0 0\\
Zahout21 \href{https://hal.science/tel-03606639}{Zahout21} & \hyperref[auth:a888]{B. Zahout} & {Algorithmes exacts et approch{\'e}s pour l'ordonnancement des travaux multiressources {\`a} intervalles fixes dans des syst{\`e}mes distribu{\'e}s : approche monocrit{\`e}re et multiagent} & \hyperref[detail:Zahout21]{Details} \href{../works/Zahout21.pdf}{Yes} & \cite{Zahout21} & 2021 & {Universit{\'e} de Tours - LIFAT} & 185 & \noindent{}\textcolor{black!50}{0.00} \textcolor{black!50}{0.00} \textbf{17.91} & 0 0 0 & 0 0 & 0 0 0\\
Lunardi20 \href{http://orbilu.uni.lu/handle/10993/43893}{Lunardi20} & \hyperref[auth:a495]{W. T. Lunardi} & A Real-World Flexible Job Shop Scheduling Problem With Sequencing Flexibility: Mathematical Programming, Constraint Programming, and Metaheuristics & \hyperref[detail:Lunardi20]{Details} \href{../works/Lunardi20.pdf}{Yes} & \cite{Lunardi20} & 2020 & University of Luxembourg, Luxembourg City, Luxembourg & 181 & \noindent{}\textbf{2.00} \textbf{2.00} \textbf{239.22} & 0 0 0 & 0 0 & 0 0 0\\
Froger16 \href{https://theses.hal.science/tel-01440836}{Froger16} & \hyperref[auth:a887]{A. Froger} & {Maintenance scheduling in the electricity industry : a particular focus on a problem rising in the onshore wind industry} & \hyperref[detail:Froger16]{Details} \href{../works/Froger16.pdf}{Yes} & \cite{Froger16} & 2016 & {Universit{\'e} d'Angers} & 181 & \noindent{}\textcolor{black!50}{0.00} \textcolor{black!50}{0.00} \textbf{127.37} & 0 0 0 & 0 0 & 0 0 0\\
Lombardi10 \href{http://amsdottorato.unibo.it/2961/}{Lombardi10} & \hyperref[auth:a142]{M. Lombardi} & Hybrid Methods for Resource Allocation and Scheduling Problems in Deterministic and Stochastic Environments & \hyperref[detail:Lombardi10]{Details} \href{../works/Lombardi10.pdf}{Yes} & \cite{Lombardi10} & 2010 & University of Bologna, Italy & 175 & \noindent{}\textcolor{black!50}{0.00} \textcolor{black!50}{0.00} \textbf{251.65} & 0 0 0 & 0 0 & 0 0 0\\
Nuijten94 \href{https://pure.tue.nl/ws/portalfiles/portal/2374269/431902.pdf}{Nuijten94} & \hyperref[auth:a655]{W. Nuijten} & Time and Resource Constrained Scheduling: a Constraint Satisfaction Approach & \hyperref[detail:Nuijten94]{Details} \href{../works/Nuijten94.pdf}{Yes} & \cite{Nuijten94} & 1994 & Eindhoven University of Technology & 172 & \noindent{}\textbf{1.50} \textbf{1.50} \textbf{68.38} & 0 0 0 & 0 0 & 0 0 0\\
Godet21a \href{https://tel.archives-ouvertes.fr/tel-03681868}{Godet21a} & \hyperref[auth:a470]{A. Godet} & Sur le tri de t{\^{a}}ches pour r{\'{e}}soudre des probl{\`{e}}mes d'ordonnancement avec la programmation par contraintes. (On the use of tasks ordering to solve scheduling problems with constraint programming) & \hyperref[detail:Godet21a]{Details} \href{../works/Godet21a.pdf}{Yes} & \cite{Godet21a} & 2021 & {IMT} Atlantique Bretagne Pays de la Loire, Brest, France & 168 & \noindent{}\textbf{2.50} \textbf{2.50} \textbf{172.67} & 0 0 0 & 0 0 & 0 0 0\\
Groleaz21 \href{https://hal.science/tel-03266690}{Groleaz21} & \hyperref[auth:a83]{L. Groleaz} & {The Group Cumulative Scheduling Problem} & \hyperref[detail:Groleaz21]{Details} \href{../works/Groleaz21.pdf}{Yes} & \cite{Groleaz21} & 2021 & {Universit{\'e} de Lyon} & 153 & \noindent{}\textcolor{black!50}{0.00} \textcolor{black!50}{0.00} \textbf{331.76} & 0 0 0 & 0 0 & 0 0 0\\
Malik08 \href{https://hdl.handle.net/10012/3612}{Malik08} & \hyperref[auth:a637]{A. M. Malik} & Constraint Programming Techniques for Optimal Instruction Scheduling & \hyperref[detail:Malik08]{Details} \href{../works/Malik08.pdf}{Yes} & \cite{Malik08} & 2008 & University of Waterloo, Ontario, Canada & 151 & \noindent{}\textbf{1.00} \textbf{1.00} \textbf{44.76} & 0 0 0 & 0 0 & 0 0 0\\
Menana11 \href{https://tel.archives-ouvertes.fr/tel-00785838}{Menana11} & \hyperref[auth:a613]{J. Menana} & Automates et programmation par contraintes pour la planification de personnel. (Automata and Constraint Programming for Personnel Scheduling Problems) & \hyperref[detail:Menana11]{Details} \href{../works/Menana11.pdf}{Yes} & \cite{Menana11} & 2011 & University of Nantes, France & 148 & \noindent{}\textbf{1.00} \textbf{1.00} \textbf{2.73} & 0 0 0 & 0 0 & 0 0 0\\
Demassey03 \href{https://tel.archives-ouvertes.fr/tel-00293564}{Demassey03} & \hyperref[auth:a243]{S. Demassey} & M{\'{e}}thodes hybrides de programmation par contraintes et programmation lin{\'{e}}aire pour le probl{\`{e}}me d'ordonnancement de projet {\`{a}} contraintes de ressources. (Hybrid Constraint Programming-Integer Linear Programming approaches for the Resource-Constrained Project Scheduling Problem) & \hyperref[detail:Demassey03]{Details} \href{../works/Demassey03.pdf}{Yes} & \cite{Demassey03} & 2003 & University of Avignon, France & 148 & \noindent{}\textbf{1.50} \textbf{1.50} \textbf{15.90} & 0 0 0 & 0 0 & 0 0 0\\
Astrand21 \href{https://nbn-resolving.org/urn:nbn:se:kth:diva-294959}{Astrand21} & \hyperref[auth:a74]{M. {\AA}strand} & Short-term Underground Mine Scheduling: An Industrial Application of Constraint Programming & \hyperref[detail:Astrand21]{Details} \href{../works/Astrand21.pdf}{Yes} & \cite{Astrand21} & 2021 & Royal Institute of Technology, Stockholm, Sweden & 142 & \noindent{}\textbf{1.00} \textbf{1.00} \textbf{310.47} & 0 0 0 & 0 0 & 0 0 0\\
Kameugne14 \href{http://cp2013.a4cp.org/sites/default/files/roger_kameugne_-_propagation_techniques_of_resource_constraint_for_cumulative_scheduling.pdf}{Kameugne14} & \hyperref[auth:a10]{R. Kameugne} & Techniques de Propagation de la Contrainte de Ressource en Ordonnancement Cumulatif & \hyperref[detail:Kameugne14]{Details} \href{../works/Kameugne14.pdf}{Yes} & \cite{Kameugne14} & 2014 & University of Yaounde I, Cameroon & 139 & \noindent{}\textcolor{black!50}{0.00} \textcolor{black!50}{0.00} \textbf{8.15} & 0 0 0 & 0 0 & 0 0 0\\
Letort13 \href{https://theses.hal.science/tel-00932215}{Letort13} & \hyperref[auth:a127]{A. Letort} & {Passage {\`a} l'{\'e}chelle pour les contraintes d'ordonnancement multi-ressources} & \hyperref[detail:Letort13]{Details} \href{../works/Letort13.pdf}{Yes} & \cite{Letort13} & 2013 & {Ecole des Mines de Nantes} & 132 & \noindent{}\textcolor{black!50}{0.00} \textcolor{black!50}{0.00} \textbf{8.73} & 0 0 0 & 0 0 & 0 0 0\\
Fahimi16 \href{http://cp2014.a4cp.org/sites/default/files/hamed_fahimi_-_efficient_algorithms_to_solve_scheduling_problems_with_a_variety_of_optimization_criteria.pdf}{Fahimi16} & \hyperref[auth:a122]{H. Fahimi} & Efficient algorithms to solve scheduling problems with a variety of optimization criteria & \hyperref[detail:Fahimi16]{Details} \href{../works/Fahimi16.pdf}{Yes} & \cite{Fahimi16} & 2016 & Universit{\'{e}} Laval, Quebec, Canada & 120 & \noindent{}\textcolor{black!50}{0.00} \textcolor{black!50}{0.00} \textbf{142.81} & 0 0 0 & 0 0 & 0 0 0\\
Derrien15 \href{https://tel.archives-ouvertes.fr/tel-01242789}{Derrien15} & \hyperref[auth:a220]{A. Derrien} & Ordonnancement cumulatif en programmation par contraintes : caract{\'{e}}risation {\'{e}}nerg{\'{e}}tique des raisonnements et solutions robustes. (Cumulative scheduling in constraint programming : energetic characterization of reasoning and robust solutions) & \hyperref[detail:Derrien15]{Details} \href{../works/Derrien15.pdf}{Yes} & \cite{Derrien15} & 2015 & {\'{E}}cole des mines de Nantes, France & 113 & \noindent{}\textbf{1.00} \textbf{1.00} \textbf{3.83} & 0 0 0 & 0 0 & 0 0 0\\
German18 \href{https://theses.hal.science/tel-01896325}{German18} & \hyperref[auth:a889]{G. German} & {Constraint programming for lot-sizing problems} & \hyperref[detail:German18]{Details} \href{../works/German18.pdf}{Yes} & \cite{German18} & 2018 & {Universit{\'e} Grenoble Alpes} & 112 & \noindent{}\textcolor{black!50}{0.00} \textcolor{black!50}{0.00} \textbf{10.90} & 0 0 0 & 0 0 & 0 0 0\\
ZarandiASC20 \href{https://doi.org/10.1007/s10462-018-9667-6}{ZarandiASC20} & \hyperref[auth:a828]{M. H. F. Zarandi}, \hyperref[auth:a829]{A. A. S. Asl}, \hyperref[auth:a830]{S. Sotudian}, \hyperref[auth:a831]{O. Castillo} & A state of the art review of intelligent scheduling & \hyperref[detail:ZarandiASC20]{Details} \href{../works/ZarandiASC20.pdf}{Yes} & \cite{ZarandiASC20} & 2020 & Artif. Intell. Rev. & 93 & \noindent{}\textcolor{black!50}{0.00} \textcolor{black!50}{0.00} \textbf{440.67} & 55 64 66 & 445 538 & 66 3 63\\
TranVNB17 \href{https://doi.org/10.1613/jair.5306}{TranVNB17} & \hyperref[auth:a798]{T. T. Tran}, \hyperref[auth:a803]{T. S. Vaquero}, \hyperref[auth:a204]{G. Nejat}, \hyperref[auth:a89]{J. C. Beck} & \cellcolor{gold!20}Robots in Retirement Homes: Applying Off-the-Shelf Planning and Scheduling to a Team of Assistive Robots & \hyperref[detail:TranVNB17]{Details} \href{../works/TranVNB17.pdf}{Yes} & \cite{TranVNB17} & 2017 & Journal of Artificial Intelligence Research & 68 & \noindent{}\textcolor{black!50}{0.00} \textcolor{black!50}{0.00} \textbf{61.11} & 12 12 21 & 0 0 & 2 2 0\\
\end{longtable}
}



\clearpage
\section{Acronyms}

{\scriptsize
\begin{longtable}{llp{12cm}}
\caption{Acronym Concepts}\\ \toprule
Acronym & Type & Description\\ \midrule\endhead
\bottomrule
\endfoot
\index{2BPHFSP (Two-Stage Bin Packing and Hybrid Flow Shop Scheduling Problem)}\index{Two-Stage Bin Packing and Hybrid Flow Shop Scheduling Problem (2BPHFSP)}2BPHFSP & Classification & Two-Stage Bin Packing and Hybrid Flow Shop Scheduling Problem\\
\index{BOM (Bill of Material)}\index{Bill of Material (BOM)}BOM & Concepts & Bill of Material\\
\index{BPCTOP (Bulk Port Cargo Throughput Optimisation Problem)}\index{Bulk Port Cargo Throughput Optimisation Problem (BPCTOP)}BPCTOP & Classification & Bulk Port Cargo Throughput Optimisation Problem\\
\index{CECSP (Continuous Energy-Constrained Scheduling Problem)}\index{Continuous Energy-Constrained Scheduling Problem (CECSP)}CECSP & Classification & Continuous Energy-Constrained Scheduling Problem\\
\index{CHSP (Cyclic Hoist Scheduling Problem)}\index{Cyclic Hoist Scheduling Problem (CHSP)}CHSP & Classification & Cyclic Hoist Scheduling Problem\\
\index{CLP (Constraint Logic Programming)}\index{Constraint Logic Programming (CLP)}CLP & CP & Constraint Logic Programming\\
\index{COP (Constraint Optimization Problem)}\index{Constraint Optimization Problem (COP)}COP & CP & Constraint Optimization Problem\\
\index{COVID (Coronavirus disease)}\index{Coronavirus disease (COVID)}COVID & ApplicationAreas & Coronavirus disease\\
\index{CP (Constraint Programming)}\index{Constraint Programming (CP)}CP & CP & Constraint Programming\\
\index{CSP (Constraint Satisfaction Problem)}\index{Constraint Satisfaction Problem (CSP)}CSP & CP & Constraint Satisfaction Problem\\
\index{CTW (Cable Tree Wiring Problem)}\index{Cable Tree Wiring Problem (CTW)}CTW & Classification & Cable Tree Wiring Problem\\
\index{CuSP (Cumulative Scheduling Problem)}\index{Cumulative Scheduling Problem (CuSP)}CuSP & Classification & Cumulative Scheduling Problem\\
\index{EOSP (Earth Observation Scheduling Problem)}\index{Earth Observation Scheduling Problem (EOSP)}EOSP & Classification & Earth Observation Scheduling Problem\\
\index{FJS (Fixed Job Scheduling)}\index{Fixed Job Scheduling (FJS)}FJS & Classification & Fixed Job Scheduling\\
\index{GCC constraint (global cardinality constraint)}\index{global cardinality constraint (GCC constraint)}GCC constraint & Constraints & global cardinality constraint\\
\index{GCSP (Group Cumulative Scheduling Problem)}\index{Group Cumulative Scheduling Problem (GCSP)}GCSP & Classification & Group Cumulative Scheduling Problem\\
\index{GRASP (Greedy Randomized Adaptive Search Procedure)}\index{Greedy Randomized Adaptive Search Procedure (GRASP)}GRASP & Algorithms & Greedy Randomized Adaptive Search Procedure\\
\index{HFF (Hybrid Flexible Flow-shop)}\index{Hybrid Flexible Flow-shop (HFF)}HFF & Classification & Hybrid Flexible Flow-shop\\
\index{HFFTT (Hybrid Flexible Flowshop with Transportation Times)}\index{Hybrid Flexible Flowshop with Transportation Times (HFFTT)}HFFTT & Classification & Hybrid Flexible Flowshop with Transportation Times\\
\index{HFS (Hybrid Flow-shop Problem)}\index{Hybrid Flow-shop Problem (HFS)}HFS & Classification & Hybrid Flow-shop Problem\\
\index{HVAC (Heating Ventilation Air-Conditioning)}\index{Heating Ventilation Air-Conditioning (HVAC)}HVAC & ApplicationAreas & Heating Ventilation Air-Conditioning\\
\index{IGT (Improved Guo Tao algorithm)}\index{Improved Guo Tao algorithm (IGT)}IGT & Algorithms & Improved Guo Tao algorithm\\
\index{JSPT (Job-Shop Scheduling Problem with Transportation)}\index{Job-Shop Scheduling Problem with Transportation (JSPT)}JSPT & Classification & Job-Shop Scheduling Problem with Transportation\\
\index{JSSP (Job-Shop Scheduling Problem)}\index{Job-Shop Scheduling Problem (JSSP)}JSSP & Classification & Job-Shop Scheduling Problem\\
\index{KRFP (kernel resource feasibility problem)}\index{kernel resource feasibility problem (KRFP)}KRFP & Classification & kernel resource feasibility problem\\
\index{LSFRP (Liner Shipping Fleet Repositioning Problem)}\index{Liner Shipping Fleet Repositioning Problem (LSFRP)}LSFRP & Classification & Liner Shipping Fleet Repositioning Problem\\
\index{MGAP (Modified Generalized Assignment Problem)}\index{Modified Generalized Assignment Problem (MGAP)}MGAP & Classification & Modified Generalized Assignment Problem\\
\index{MINLP (Mixed-Integer Non-linear Programming)}\index{Mixed-Integer Non-linear Programming (MINLP)}MINLP & Algorithms & Mixed-Integer Non-linear Programming\\
\index{MIQP (Mixed-Integer Quadratic Programming)}\index{Mixed-Integer Quadratic Programming (MIQP)}MIQP & Algorithms & Mixed-Integer Quadratic Programming\\
\index{NEH (Scheduling Algorithm developed by Nawaz, Enscore, Ham)}\index{Scheduling Algorithm developed by Nawaz, Enscore, Ham (NEH)}NEH & Algorithms & Scheduling Algorithm developed by Nawaz, Enscore, Ham\\
\index{OSP (Oven Scheduling Problem)}\index{Oven Scheduling Problem (OSP)}OSP & Classification & Oven Scheduling Problem\\
\index{OSSP (Open Shop Scheduling Problem)}\index{Open Shop Scheduling Problem (OSSP)}OSSP & Classification & Open Shop Scheduling Problem\\
\index{PCB industry (printed circuit board industry)}\index{printed circuit board industry (PCB industry)}PCB industry & Industries & printed circuit board industry\\
\index{PJSSP (Pre-emptive Job-Shop scheduling Problem)}\index{Pre-emptive Job-Shop scheduling Problem (PJSSP)}PJSSP & Classification & Pre-emptive Job-Shop scheduling Problem\\
\index{PMSP (Parallel Machine Scheduling Problem)}\index{Parallel Machine Scheduling Problem (PMSP)}PMSP & Classification & Parallel Machine Scheduling Problem\\
\index{PP-MS-MMRCPSP (partially preemptive- multi-skill/mode resource-constrained project scheduling problem with generalized precedence relations and resource calendars)}\index{partially preemptive- multi-skill/mode resource-constrained project scheduling problem with generalized precedence relations and resource calendars (PP-MS-MMRCPSP)}PP-MS-MMRCPSP & Classification & partially preemptive- multi-skill/mode resource-constrained project scheduling problem with generalized precedence relations and resource calendars\\
\index{PTC (Scheduling Problem with Time Constraints)}\index{Scheduling Problem with Time Constraints (PTC)}PTC & Classification & Scheduling Problem with Time Constraints\\
\index{RCMPSP (Resource-Constrained Multi-Project Scheduling Problem)}\index{Resource-Constrained Multi-Project Scheduling Problem (RCMPSP)}RCMPSP & Classification & Resource-Constrained Multi-Project Scheduling Problem\\
\index{RCPSP (Resource-constrained Project Scheduling Problem)}\index{Resource-constrained Project Scheduling Problem (RCPSP)}RCPSP & Classification & Resource-constrained Project Scheduling Problem\\
\index{RCPSPDC (Resource-constrained Project Scheduling Problem with Discounted Cashflow)}\index{Resource-constrained Project Scheduling Problem with Discounted Cashflow (RCPSPDC)}RCPSPDC & Classification & Resource-constrained Project Scheduling Problem with Discounted Cashflow\\
\index{RTMP (Railway Traffic Management Problem)}\index{Railway Traffic Management Problem (RTMP)}RTMP & Classification & Railway Traffic Management Problem\\
\index{SBSFMMAL (Simultaneous Balancing and Scheduling of Flexible Mixed Model Assembly Lines)}\index{Simultaneous Balancing and Scheduling of Flexible Mixed Model Assembly Lines (SBSFMMAL)}SBSFMMAL & Classification & Simultaneous Balancing and Scheduling of Flexible Mixed Model Assembly Lines\\
\index{SCC (Steel-making and continuous casting)}\index{Steel-making and continuous casting (SCC)}SCC & Classification & Steel-making and continuous casting\\
\index{SMSDP (steel mill slab design problem)}\index{steel mill slab design problem (SMSDP)}SMSDP & Classification & steel mill slab design problem\\
\index{TCSP (Temporal Constraint Satisfaction Problem)}\index{Temporal Constraint Satisfaction Problem (TCSP)}TCSP & Classification & Temporal Constraint Satisfaction Problem\\
\index{TMS (Transmission Network Maintenance Planning)}\index{Transmission Network Maintenance Planning (TMS)}TMS & Classification & Transmission Network Maintenance Planning\\
\index{UTVPI constraint (unit two variable per inequality constraint)}\index{unit two variable per inequality constraint (UTVPI constraint)}UTVPI constraint & Constraints & unit two variable per inequality constraint\\
\index{psplib (Project Scheduling Problem Library)}\index{Project Scheduling Problem Library (psplib)}psplib & Classification & Project Scheduling Problem Library\\
\index{rtRTMP (real-time Railway Traffic Management Problem)}\index{real-time Railway Traffic Management Problem (rtRTMP)}rtRTMP & Classification & real-time Railway Traffic Management Problem\\
\end{longtable}
}



\clearpage
\section{Concept Matching}

In order to automatically find out properties of the articles, we try to find certain concepts in the pdf versions of the articles. We manually defined an ontology of important concepts to look for, and defined regular expressions that would recognize these concepts in the text. We use the \emph{pdfgrep} command to search for the number of occurrences of certain regular expressions in the files. This often clearly identifies the constraints used in the model. We group the results by number of occurrences of the concept in the text of the work. Note that this is only approximate, as we do include the full pdf file in the search. A concept might only be mentioned in some of the title of citations used in the paper, we do count them in our results, as we were not able to remove the bibliography from the main body of the work.

Overall, if a work is not mentioned as using the concept, the the text does not contain a match to the corresponding regular expression. A fundamental limitation of this approach is that it only really works for text written in the language the regular expressions are designed for (in our case English), and not those written in another language. We could overcome this limitation by defining all concepts in other languages as well, and then using a language flag to identify the language the text is written in. 

Note that we only show the first 30 matching entries in each concept category, and list the total number of matches if there are more than 30 matches.


\clearpage
\subsection{Concept Type Scheduling}
\label{sec:Scheduling}
\label{Scheduling}
{\scriptsize
\begin{longtable}{p{3cm}r>{\raggedright\arraybackslash}p{6cm}>{\raggedright\arraybackslash}p{6cm}>{\raggedright\arraybackslash}p{8cm}}
\rowcolor{white}\caption{Works for Concepts of Type Scheduling (Total 7 Concepts, 7 Used)}\\ \toprule
\rowcolor{white}Keyword & Weight & High & Medium & Low\\ \midrule\endhead
\bottomrule
\endfoot
\index{activity}\index{Scheduling!activity}activity &  0.50 & \href{../works/TardivoDFMP23.pdf}{TardivoDFMP23}~\cite{TardivoDFMP23}, \href{../works/GokPTGO23.pdf}{GokPTGO23}~\cite{GokPTGO23}, \href{../works/PovedaAA23.pdf}{PovedaAA23}~\cite{PovedaAA23}, \href{../works/PenzDN23.pdf}{PenzDN23}~\cite{PenzDN23}, \href{../works/AalianPG23.pdf}{AalianPG23}~\cite{AalianPG23}, \href{../works/MarliereSPR23.pdf}{MarliereSPR23}~\cite{MarliereSPR23}, \href{../works/CampeauG22.pdf}{CampeauG22}~\cite{CampeauG22}, \href{../works/SubulanC22.pdf}{SubulanC22}~\cite{SubulanC22}, \href{../works/AwadMDMT22.pdf}{AwadMDMT22}~\cite{AwadMDMT22}, \href{../works/SvancaraB22.pdf}{SvancaraB22}~\cite{SvancaraB22}, \href{../works/TouatBT22.pdf}{TouatBT22}~\cite{TouatBT22}, \href{../works/Godet21a.pdf}{Godet21a}~\cite{Godet21a}, \href{../works/BenderWS21.pdf}{BenderWS21}~\cite{BenderWS21}, \href{../works/KlankeBYE21.pdf}{KlankeBYE21}~\cite{KlankeBYE21}, \href{../works/Astrand21.pdf}{Astrand21}~\cite{Astrand21}, \href{../works/HubnerGSV21.pdf}{HubnerGSV21}~\cite{HubnerGSV21}, \href{../works/ZarandiASC20.pdf}{ZarandiASC20}~\cite{ZarandiASC20}, \href{../works/Polo-MejiaALB20.pdf}{Polo-MejiaALB20}~\cite{Polo-MejiaALB20}, \href{../works/CauwelaertDS20.pdf}{CauwelaertDS20}~\cite{CauwelaertDS20}...\href{../works/BeckF98.pdf}{BeckF98}~\cite{BeckF98}, \href{../works/LeeKLKKYHP97.pdf}{LeeKLKKYHP97}~\cite{LeeKLKKYHP97}, \href{../works/BeckDSF97a.pdf}{BeckDSF97a}~\cite{BeckDSF97a}, \href{../works/BaptisteP97.pdf}{BaptisteP97}~\cite{BaptisteP97}, \href{../works/BeckDSF97.pdf}{BeckDSF97}~\cite{BeckDSF97}, \href{../works/MorgadoM97.pdf}{MorgadoM97}~\cite{MorgadoM97}, \href{../works/PapeB97.pdf}{PapeB97}~\cite{PapeB97}, \href{../works/BaptisteP95.pdf}{BaptisteP95}~\cite{BaptisteP95}, \href{../works/Muscettola94.pdf}{Muscettola94}~\cite{Muscettola94}, \href{../works/Pape94.pdf}{Pape94}~\cite{Pape94} (Total: 219) & \href{../works/BonninMNE24.pdf}{BonninMNE24}~\cite{BonninMNE24}, \href{../works/YuraszeckMCCR23.pdf}{YuraszeckMCCR23}~\cite{YuraszeckMCCR23}, \href{../works/Bit-Monnot23.pdf}{Bit-Monnot23}~\cite{Bit-Monnot23}, \href{../works/AfsarVPG23.pdf}{AfsarVPG23}~\cite{AfsarVPG23}, \href{../works/GhandehariK22.pdf}{GhandehariK22}~\cite{GhandehariK22}, \href{../works/PopovicCGNC22.pdf}{PopovicCGNC22}~\cite{PopovicCGNC22}, \href{../works/BoudreaultSLQ22.pdf}{BoudreaultSLQ22}~\cite{BoudreaultSLQ22}, \href{../works/AntunesABD20.pdf}{AntunesABD20}~\cite{AntunesABD20}, \href{../works/GokGSTO20.pdf}{GokGSTO20}~\cite{GokGSTO20}, \href{../works/Lunardi20.pdf}{Lunardi20}~\cite{Lunardi20}, \href{../works/LunardiBLRV20.pdf}{LunardiBLRV20}~\cite{LunardiBLRV20}, \href{../works/EscobetPQPRA19.pdf}{EscobetPQPRA19}~\cite{EscobetPQPRA19}, \href{../works/YangSS19.pdf}{YangSS19}~\cite{YangSS19}, \href{../works/Hooker19.pdf}{Hooker19}~\cite{Hooker19}, \href{../works/Novas19.pdf}{Novas19}~\cite{Novas19}, \href{../works/ShinBBHO18.pdf}{ShinBBHO18}~\cite{ShinBBHO18}, \href{../works/SchuttS16.pdf}{SchuttS16}~\cite{SchuttS16}, \href{../works/BoothNB16.pdf}{BoothNB16}~\cite{BoothNB16}, \href{../works/OrnekO16.pdf}{OrnekO16}~\cite{OrnekO16}...\href{../works/Mason01.pdf}{Mason01}~\cite{Mason01}, \href{../works/JainG01.pdf}{JainG01}~\cite{JainG01}, \href{../works/SimonisCK00.pdf}{SimonisCK00}~\cite{SimonisCK00}, \href{../works/CestaOS00.pdf}{CestaOS00}~\cite{CestaOS00}, \href{../works/PembertonG98.pdf}{PembertonG98}~\cite{PembertonG98}, \href{../works/RodosekW98.pdf}{RodosekW98}~\cite{RodosekW98}, \href{../works/Schaerf97.pdf}{Schaerf97}~\cite{Schaerf97}, \href{../works/SadehF96.pdf}{SadehF96}~\cite{SadehF96}, \href{../works/BrusoniCLMMT96.pdf}{BrusoniCLMMT96}~\cite{BrusoniCLMMT96}, \href{../works/Nuijten94.pdf}{Nuijten94}~\cite{Nuijten94} (Total: 64) & \href{../works/PrataAN23.pdf}{PrataAN23}~\cite{PrataAN23}, \href{../works/GuoZ23.pdf}{GuoZ23}~\cite{GuoZ23}, \href{../works/abs-2312-13682.pdf}{abs-2312-13682}~\cite{abs-2312-13682}, \href{../works/WessenCSFPM23.pdf}{WessenCSFPM23}~\cite{WessenCSFPM23}, \href{../works/NaderiBZR23.pdf}{NaderiBZR23}~\cite{NaderiBZR23}, \href{../works/CzerniachowskaWZ23.pdf}{CzerniachowskaWZ23}~\cite{CzerniachowskaWZ23}, \href{../works/SquillaciPR23.pdf}{SquillaciPR23}~\cite{SquillaciPR23}, \href{../works/abs-2305-19888.pdf}{abs-2305-19888}~\cite{abs-2305-19888}, \href{../works/FrimodigECM23.pdf}{FrimodigECM23}~\cite{FrimodigECM23}, \href{../works/JuvinHL23a.pdf}{JuvinHL23a}~\cite{JuvinHL23a}, \href{../works/ShaikhK23.pdf}{ShaikhK23}~\cite{ShaikhK23}, \href{../works/PerezGSL23.pdf}{PerezGSL23}~\cite{PerezGSL23}, \href{../works/abs-2211-14492.pdf}{abs-2211-14492}~\cite{abs-2211-14492}, \href{../works/PohlAK22.pdf}{PohlAK22}~\cite{PohlAK22}, \href{../works/OuelletQ22.pdf}{OuelletQ22}~\cite{OuelletQ22}, \href{../works/MullerMKP22.pdf}{MullerMKP22}~\cite{MullerMKP22}, \href{../works/JuvinHL22.pdf}{JuvinHL22}~\cite{JuvinHL22}, \href{../works/YunusogluY22.pdf}{YunusogluY22}~\cite{YunusogluY22}, \href{../works/HeinzNVH22.pdf}{HeinzNVH22}~\cite{HeinzNVH22}...\href{../works/MurphyRFSS97.pdf}{MurphyRFSS97}~\cite{MurphyRFSS97}, \href{../works/BeckDF97.pdf}{BeckDF97}~\cite{BeckDF97}, \href{../works/OddiS97.pdf}{OddiS97}~\cite{OddiS97}, \href{../works/GetoorOFC97.pdf}{GetoorOFC97}~\cite{GetoorOFC97}, \href{../works/Wallace96.pdf}{Wallace96}~\cite{Wallace96}, \href{../works/BlazewiczDP96.pdf}{BlazewiczDP96}~\cite{BlazewiczDP96}, \href{../works/Colombani96.pdf}{Colombani96}~\cite{Colombani96}, \href{../works/RoweJCA96.pdf}{RoweJCA96}~\cite{RoweJCA96}, \href{../works/Puget95.pdf}{Puget95}~\cite{Puget95}, \href{../works/AggounB93.pdf}{AggounB93}~\cite{AggounB93} (Total: 116)\\
\index{job}\index{Scheduling!job}job &  1.00 & \href{../works/PrataAN23.pdf}{PrataAN23}~\cite{PrataAN23}, \href{../works/ForbesHJST24.pdf}{ForbesHJST24}~\cite{ForbesHJST24}, \href{../works/abs-2402-00459.pdf}{abs-2402-00459}~\cite{abs-2402-00459}, \href{../works/LiLZDZW24.pdf}{LiLZDZW24}~\cite{LiLZDZW24}, \href{../works/JuvinHHL23.pdf}{JuvinHHL23}~\cite{JuvinHHL23}, \href{../works/YuraszeckMC23.pdf}{YuraszeckMC23}~\cite{YuraszeckMC23}, \href{../works/AfsarVPG23.pdf}{AfsarVPG23}~\cite{AfsarVPG23}, \href{../works/ZhuSZW23.pdf}{ZhuSZW23}~\cite{ZhuSZW23}, \href{../works/Mehdizadeh-Somarin23.pdf}{Mehdizadeh-Somarin23}~\cite{Mehdizadeh-Somarin23}, \href{../works/AbreuNP23.pdf}{AbreuNP23}~\cite{AbreuNP23}, \href{../works/abs-2306-05747.pdf}{abs-2306-05747}~\cite{abs-2306-05747}, \href{../works/NaderiRR23.pdf}{NaderiRR23}~\cite{NaderiRR23}, \href{../works/TasselGS23.pdf}{TasselGS23}~\cite{TasselGS23}, \href{../works/YuraszeckMCCR23.pdf}{YuraszeckMCCR23}~\cite{YuraszeckMCCR23}, \href{../works/AbreuPNF23.pdf}{AbreuPNF23}~\cite{AbreuPNF23}, \href{../works/PenzDN23.pdf}{PenzDN23}~\cite{PenzDN23}, \href{../works/AlfieriGPS23.pdf}{AlfieriGPS23}~\cite{AlfieriGPS23}, \href{../works/LacknerMMWW23.pdf}{LacknerMMWW23}~\cite{LacknerMMWW23}, \href{../works/Bit-Monnot23.pdf}{Bit-Monnot23}~\cite{Bit-Monnot23}...\href{../works/SimonisC95.pdf}{SimonisC95}~\cite{SimonisC95}, \href{../works/BaptisteP95.pdf}{BaptisteP95}~\cite{BaptisteP95}, \href{../works/Goltz95.pdf}{Goltz95}~\cite{Goltz95}, \href{../works/CrawfordB94.pdf}{CrawfordB94}~\cite{CrawfordB94}, \href{../works/NuijtenA94.pdf}{NuijtenA94}~\cite{NuijtenA94}, \href{../works/Nuijten94.pdf}{Nuijten94}~\cite{Nuijten94}, \href{../works/SmithC93.pdf}{SmithC93}~\cite{SmithC93}, \href{../works/AggounB93.pdf}{AggounB93}~\cite{AggounB93}, \href{../works/FoxS90.pdf}{FoxS90}~\cite{FoxS90}, \href{../works/KengY89.pdf}{KengY89}~\cite{KengY89} (Total: 332) & \href{../works/BonninMNE24.pdf}{BonninMNE24}~\cite{BonninMNE24}, \href{../works/LuZZYW24.pdf}{LuZZYW24}~\cite{LuZZYW24}, \href{../works/ShaikhK23.pdf}{ShaikhK23}~\cite{ShaikhK23}, \href{../works/EfthymiouY23.pdf}{EfthymiouY23}~\cite{EfthymiouY23}, \href{../works/MarliereSPR23.pdf}{MarliereSPR23}~\cite{MarliereSPR23}, \href{../works/abs-2305-19888.pdf}{abs-2305-19888}~\cite{abs-2305-19888}, \href{../works/Adelgren2023.pdf}{Adelgren2023}~\cite{Adelgren2023}, \href{../works/LuoB22.pdf}{LuoB22}~\cite{LuoB22}, \href{../works/BourreauGGLT22.pdf}{BourreauGGLT22}~\cite{BourreauGGLT22}, \href{../works/HeinzNVH22.pdf}{HeinzNVH22}~\cite{HeinzNVH22}, \href{../works/RoshanaeiN21.pdf}{RoshanaeiN21}~\cite{RoshanaeiN21}, \href{../works/Lemos21.pdf}{Lemos21}~\cite{Lemos21}, \href{../works/HanenKP21.pdf}{HanenKP21}~\cite{HanenKP21}, \href{../works/Mercier-AubinGQ20.pdf}{Mercier-AubinGQ20}~\cite{Mercier-AubinGQ20}, \href{../works/CarlierPSJ20.pdf}{CarlierPSJ20}~\cite{CarlierPSJ20}, \href{../works/MokhtarzadehTNF20.pdf}{MokhtarzadehTNF20}~\cite{MokhtarzadehTNF20}, \href{../works/GokGSTO20.pdf}{GokGSTO20}~\cite{GokGSTO20}, \href{../works/RoshanaeiBAUB20.pdf}{RoshanaeiBAUB20}~\cite{RoshanaeiBAUB20}, \href{../works/PinarbasiAY19.pdf}{PinarbasiAY19}~\cite{PinarbasiAY19}...\href{../works/CestaOF99.pdf}{CestaOF99}~\cite{CestaOF99}, \href{../works/CarlssonKA99.pdf}{CarlssonKA99}~\cite{CarlssonKA99}, \href{../works/JoLLH99.pdf}{JoLLH99}~\cite{JoLLH99}, \href{../works/Caseau97.pdf}{Caseau97}~\cite{Caseau97}, \href{../works/BaptisteP97.pdf}{BaptisteP97}~\cite{BaptisteP97}, \href{../works/LammaMM97.pdf}{LammaMM97}~\cite{LammaMM97}, \href{../works/MorgadoM97.pdf}{MorgadoM97}~\cite{MorgadoM97}, \href{../works/Puget95.pdf}{Puget95}~\cite{Puget95}, \href{../works/Muscettola94.pdf}{Muscettola94}~\cite{Muscettola94}, \href{../works/Pape94.pdf}{Pape94}~\cite{Pape94} (Total: 83) & \href{../works/FalqueALM24.pdf}{FalqueALM24}~\cite{FalqueALM24}, \href{../works/NaderiBZR23.pdf}{NaderiBZR23}~\cite{NaderiBZR23}, \href{../works/GuoZ23.pdf}{GuoZ23}~\cite{GuoZ23}, \href{../works/GokPTGO23.pdf}{GokPTGO23}~\cite{GokPTGO23}, \href{../works/PovedaAA23.pdf}{PovedaAA23}~\cite{PovedaAA23}, \href{../works/WessenCSFPM23.pdf}{WessenCSFPM23}~\cite{WessenCSFPM23}, \href{../works/PohlAK22.pdf}{PohlAK22}~\cite{PohlAK22}, \href{../works/CampeauG22.pdf}{CampeauG22}~\cite{CampeauG22}, \href{../works/GhandehariK22.pdf}{GhandehariK22}~\cite{GhandehariK22}, \href{../works/NaqviAIAAA22.pdf}{NaqviAIAAA22}~\cite{NaqviAIAAA22}, \href{../works/Edis21.pdf}{Edis21}~\cite{Edis21}, \href{../works/AntuoriHHEN21.pdf}{AntuoriHHEN21}~\cite{AntuoriHHEN21}, \href{../works/BenderWS21.pdf}{BenderWS21}~\cite{BenderWS21}, \href{../works/KlankeBYE21.pdf}{KlankeBYE21}~\cite{KlankeBYE21}, \href{../works/HubnerGSV21.pdf}{HubnerGSV21}~\cite{HubnerGSV21}, \href{../works/QinDCS20.pdf}{QinDCS20}~\cite{QinDCS20}, \href{../works/AntuoriHHEN20.pdf}{AntuoriHHEN20}~\cite{AntuoriHHEN20}, \href{../works/WessenCS20.pdf}{WessenCS20}~\cite{WessenCS20}, \href{../works/AbidinK20.pdf}{AbidinK20}~\cite{AbidinK20}...\href{../works/RoweJCA96.pdf}{RoweJCA96}~\cite{RoweJCA96}, \href{../works/Wallace96.pdf}{Wallace96}~\cite{Wallace96}, \href{../works/BrusoniCLMMT96.pdf}{BrusoniCLMMT96}~\cite{BrusoniCLMMT96}, \href{../works/Simonis95a.pdf}{Simonis95a}~\cite{Simonis95a}, \href{../works/WeilHFP95.pdf}{WeilHFP95}~\cite{WeilHFP95}, \href{../works/YoshikawaKNW94.pdf}{YoshikawaKNW94}~\cite{YoshikawaKNW94}, \href{../works/MintonJPL92.pdf}{MintonJPL92}~\cite{MintonJPL92}, \href{../works/DincbasSH90.pdf}{DincbasSH90}~\cite{DincbasSH90}, \href{../works/EskeyZ90.pdf}{EskeyZ90}~\cite{EskeyZ90}, \href{../works/Davis87.pdf}{Davis87}~\cite{Davis87} (Total: 117)\\
\index{machine}\index{Scheduling!machine}machine &  0.50 & \href{../works/LiLZDZW24.pdf}{LiLZDZW24}~\cite{LiLZDZW24}, \href{../works/abs-2402-00459.pdf}{abs-2402-00459}~\cite{abs-2402-00459}, \href{../works/BonninMNE24.pdf}{BonninMNE24}~\cite{BonninMNE24}, \href{../works/PrataAN23.pdf}{PrataAN23}~\cite{PrataAN23}, \href{../works/Fatemi-AnarakiTFV23.pdf}{Fatemi-AnarakiTFV23}~\cite{Fatemi-AnarakiTFV23}, \href{../works/YuraszeckMCCR23.pdf}{YuraszeckMCCR23}~\cite{YuraszeckMCCR23}, \href{../works/JuvinHL23a.pdf}{JuvinHL23a}~\cite{JuvinHL23a}, \href{../works/ZhuSZW23.pdf}{ZhuSZW23}~\cite{ZhuSZW23}, \href{../works/JuvinHHL23.pdf}{JuvinHHL23}~\cite{JuvinHHL23}, \href{../works/abs-2312-13682.pdf}{abs-2312-13682}~\cite{abs-2312-13682}, \href{../works/LacknerMMWW23.pdf}{LacknerMMWW23}~\cite{LacknerMMWW23}, \href{../works/AlfieriGPS23.pdf}{AlfieriGPS23}~\cite{AlfieriGPS23}, \href{../works/AfsarVPG23.pdf}{AfsarVPG23}~\cite{AfsarVPG23}, \href{../works/KimCMLLP23.pdf}{KimCMLLP23}~\cite{KimCMLLP23}, \href{../works/IklassovMR023.pdf}{IklassovMR023}~\cite{IklassovMR023}, \href{../works/JuvinHL23.pdf}{JuvinHL23}~\cite{JuvinHL23}, \href{../works/GuoZ23.pdf}{GuoZ23}~\cite{GuoZ23}, \href{../works/PerezGSL23.pdf}{PerezGSL23}~\cite{PerezGSL23}, \href{../works/NaderiBZ23.pdf}{NaderiBZ23}~\cite{NaderiBZ23}...\href{../works/Zhou96.pdf}{Zhou96}~\cite{Zhou96}, \href{../works/Colombani96.pdf}{Colombani96}~\cite{Colombani96}, \href{../works/Goltz95.pdf}{Goltz95}~\cite{Goltz95}, \href{../works/Pape94.pdf}{Pape94}~\cite{Pape94}, \href{../works/NuijtenA94.pdf}{NuijtenA94}~\cite{NuijtenA94}, \href{../works/BeldiceanuC94.pdf}{BeldiceanuC94}~\cite{BeldiceanuC94}, \href{../works/Nuijten94.pdf}{Nuijten94}~\cite{Nuijten94}, \href{../works/AggounB93.pdf}{AggounB93}~\cite{AggounB93}, \href{../works/ErtlK91.pdf}{ErtlK91}~\cite{ErtlK91}, \href{../works/EskeyZ90.pdf}{EskeyZ90}~\cite{EskeyZ90} (Total: 308) & \href{../works/ForbesHJST24.pdf}{ForbesHJST24}~\cite{ForbesHJST24}, \href{../works/GurPAE23.pdf}{GurPAE23}~\cite{GurPAE23}, \href{../works/Bit-Monnot23.pdf}{Bit-Monnot23}~\cite{Bit-Monnot23}, \href{../works/AkramNHRSA23.pdf}{AkramNHRSA23}~\cite{AkramNHRSA23}, \href{../works/GokPTGO23.pdf}{GokPTGO23}~\cite{GokPTGO23}, \href{../works/LuoB22.pdf}{LuoB22}~\cite{LuoB22}, \href{../works/OrnekOS20.pdf}{OrnekOS20}~\cite{OrnekOS20}, \href{../works/EtminaniesfahaniGNMS22.pdf}{EtminaniesfahaniGNMS22}~\cite{EtminaniesfahaniGNMS22}, \href{../works/ElciOH22.pdf}{ElciOH22}~\cite{ElciOH22}, \href{../works/KlankeBYE21.pdf}{KlankeBYE21}~\cite{KlankeBYE21}, \href{../works/Lemos21.pdf}{Lemos21}~\cite{Lemos21}, \href{../works/AbohashimaEG21.pdf}{AbohashimaEG21}~\cite{AbohashimaEG21}, \href{../works/HillTV21.pdf}{HillTV21}~\cite{HillTV21}, \href{../works/HamP21.pdf}{HamP21}~\cite{HamP21}, \href{../works/RoshanaeiN21.pdf}{RoshanaeiN21}~\cite{RoshanaeiN21}, \href{../works/Alaka21.pdf}{Alaka21}~\cite{Alaka21}, \href{../works/CarlierPSJ20.pdf}{CarlierPSJ20}~\cite{CarlierPSJ20}, \href{../works/Polo-MejiaALB20.pdf}{Polo-MejiaALB20}~\cite{Polo-MejiaALB20}, \href{../works/RoshanaeiBAUB20.pdf}{RoshanaeiBAUB20}~\cite{RoshanaeiBAUB20}...\href{../works/NuijtenP98.pdf}{NuijtenP98}~\cite{NuijtenP98}, \href{../works/MorgadoM97.pdf}{MorgadoM97}~\cite{MorgadoM97}, \href{../works/RoweJCA96.pdf}{RoweJCA96}~\cite{RoweJCA96}, \href{../works/Wallace96.pdf}{Wallace96}~\cite{Wallace96}, \href{../works/Simonis95a.pdf}{Simonis95a}~\cite{Simonis95a}, \href{../works/CrawfordB94.pdf}{CrawfordB94}~\cite{CrawfordB94}, \href{../works/SmithC93.pdf}{SmithC93}~\cite{SmithC93}, \href{../works/MintonJPL92.pdf}{MintonJPL92}~\cite{MintonJPL92}, \href{../works/FoxS90.pdf}{FoxS90}~\cite{FoxS90}, \href{../works/Prosser89.pdf}{Prosser89}~\cite{Prosser89} (Total: 95) & \href{../works/LuZZYW24.pdf}{LuZZYW24}~\cite{LuZZYW24}, \href{../works/MarliereSPR23.pdf}{MarliereSPR23}~\cite{MarliereSPR23}, \href{../works/ShaikhK23.pdf}{ShaikhK23}~\cite{ShaikhK23}, \href{../works/AlakaP23.pdf}{AlakaP23}~\cite{AlakaP23}, \href{../works/NaderiBZR23.pdf}{NaderiBZR23}~\cite{NaderiBZR23}, \href{../works/KameugneFND23.pdf}{KameugneFND23}~\cite{KameugneFND23}, \href{../works/MontemanniD23.pdf}{MontemanniD23}~\cite{MontemanniD23}, \href{../works/GhandehariK22.pdf}{GhandehariK22}~\cite{GhandehariK22}, \href{../works/CilKLO22.pdf}{CilKLO22}~\cite{CilKLO22}, \href{../works/BoudreaultSLQ22.pdf}{BoudreaultSLQ22}~\cite{BoudreaultSLQ22}, \href{../works/PopovicCGNC22.pdf}{PopovicCGNC22}~\cite{PopovicCGNC22}, \href{../works/SubulanC22.pdf}{SubulanC22}~\cite{SubulanC22}, \href{../works/NaqviAIAAA22.pdf}{NaqviAIAAA22}~\cite{NaqviAIAAA22}, \href{../works/PohlAK22.pdf}{PohlAK22}~\cite{PohlAK22}, \href{../works/GeibingerMM21.pdf}{GeibingerMM21}~\cite{GeibingerMM21}, \href{../works/ArtiguesHQT21.pdf}{ArtiguesHQT21}~\cite{ArtiguesHQT21}, \href{../works/Mercier-AubinGQ20.pdf}{Mercier-AubinGQ20}~\cite{Mercier-AubinGQ20}, \href{../works/WallaceY20.pdf}{WallaceY20}~\cite{WallaceY20}, \href{../works/BarzegaranZP20.pdf}{BarzegaranZP20}~\cite{BarzegaranZP20}...\href{../works/PintoG97.pdf}{PintoG97}~\cite{PintoG97}, \href{../works/BeckDSF97a.pdf}{BeckDSF97a}~\cite{BeckDSF97a}, \href{../works/BeckDF97.pdf}{BeckDF97}~\cite{BeckDF97}, \href{../works/SadehF96.pdf}{SadehF96}~\cite{SadehF96}, \href{../works/BaptisteP95.pdf}{BaptisteP95}~\cite{BaptisteP95}, \href{../works/SimonisC95.pdf}{SimonisC95}~\cite{SimonisC95}, \href{../works/Simonis95.pdf}{Simonis95}~\cite{Simonis95}, \href{../works/Hamscher91.pdf}{Hamscher91}~\cite{Hamscher91}, \href{../works/DincbasSH90.pdf}{DincbasSH90}~\cite{DincbasSH90}, \href{../works/MintonJPL90.pdf}{MintonJPL90}~\cite{MintonJPL90} (Total: 173)\\
\index{order}\index{Scheduling!order}order &  0.50 & \href{../works/PrataAN23.pdf}{PrataAN23}~\cite{PrataAN23}, \href{../works/abs-2402-00459.pdf}{abs-2402-00459}~\cite{abs-2402-00459}, \href{../works/LuZZYW24.pdf}{LuZZYW24}~\cite{LuZZYW24}, \href{../works/LiLZDZW24.pdf}{LiLZDZW24}~\cite{LiLZDZW24}, \href{../works/BonninMNE24.pdf}{BonninMNE24}~\cite{BonninMNE24}, \href{../works/GokPTGO23.pdf}{GokPTGO23}~\cite{GokPTGO23}, \href{../works/ZhuSZW23.pdf}{ZhuSZW23}~\cite{ZhuSZW23}, \href{../works/AbreuNP23.pdf}{AbreuNP23}~\cite{AbreuNP23}, \href{../works/Fatemi-AnarakiTFV23.pdf}{Fatemi-AnarakiTFV23}~\cite{Fatemi-AnarakiTFV23}, \href{../works/Adelgren2023.pdf}{Adelgren2023}~\cite{Adelgren2023}, \href{../works/abs-2306-05747.pdf}{abs-2306-05747}~\cite{abs-2306-05747}, \href{../works/PerezGSL23.pdf}{PerezGSL23}~\cite{PerezGSL23}, \href{../works/NaderiBZ23.pdf}{NaderiBZ23}~\cite{NaderiBZ23}, \href{../works/abs-2312-13682.pdf}{abs-2312-13682}~\cite{abs-2312-13682}, \href{../works/AalianPG23.pdf}{AalianPG23}~\cite{AalianPG23}, \href{../works/AbreuPNF23.pdf}{AbreuPNF23}~\cite{AbreuPNF23}, \href{../works/MarliereSPR23.pdf}{MarliereSPR23}~\cite{MarliereSPR23}, \href{../works/YuraszeckMCCR23.pdf}{YuraszeckMCCR23}~\cite{YuraszeckMCCR23}, \href{../works/GuoZ23.pdf}{GuoZ23}~\cite{GuoZ23}...\href{../works/MintonJPL92.pdf}{MintonJPL92}~\cite{MintonJPL92}, \href{../works/ErtlK91.pdf}{ErtlK91}~\cite{ErtlK91}, \href{../works/FoxS90.pdf}{FoxS90}~\cite{FoxS90}, \href{../works/DincbasSH90.pdf}{DincbasSH90}~\cite{DincbasSH90}, \href{../works/MintonJPL90.pdf}{MintonJPL90}~\cite{MintonJPL90}, \href{../works/KengY89.pdf}{KengY89}~\cite{KengY89}, \href{../works/Prosser89.pdf}{Prosser89}~\cite{Prosser89}, \href{../works/FeldmanG89.pdf}{FeldmanG89}~\cite{FeldmanG89}, \href{../works/Valdes87.pdf}{Valdes87}~\cite{Valdes87}, \href{../works/Davis87.pdf}{Davis87}~\cite{Davis87} (Total: 536) & \href{../works/FalqueALM24.pdf}{FalqueALM24}~\cite{FalqueALM24}, \href{../works/ForbesHJST24.pdf}{ForbesHJST24}~\cite{ForbesHJST24}, \href{../works/YuraszeckMC23.pdf}{YuraszeckMC23}~\cite{YuraszeckMC23}, \href{../works/GurPAE23.pdf}{GurPAE23}~\cite{GurPAE23}, \href{../works/abs-2305-19888.pdf}{abs-2305-19888}~\cite{abs-2305-19888}, \href{../works/MontemanniD23a.pdf}{MontemanniD23a}~\cite{MontemanniD23a}, \href{../works/NaderiRR23.pdf}{NaderiRR23}~\cite{NaderiRR23}, \href{../works/TardivoDFMP23.pdf}{TardivoDFMP23}~\cite{TardivoDFMP23}, \href{../works/NaderiBZR23.pdf}{NaderiBZR23}~\cite{NaderiBZR23}, \href{../works/ShaikhK23.pdf}{ShaikhK23}~\cite{ShaikhK23}, \href{../works/NaqviAIAAA22.pdf}{NaqviAIAAA22}~\cite{NaqviAIAAA22}, \href{../works/SvancaraB22.pdf}{SvancaraB22}~\cite{SvancaraB22}, \href{../works/ElciOH22.pdf}{ElciOH22}~\cite{ElciOH22}, \href{../works/CilKLO22.pdf}{CilKLO22}~\cite{CilKLO22}, \href{../works/JungblutK22.pdf}{JungblutK22}~\cite{JungblutK22}, \href{../works/ZhangBB22.pdf}{ZhangBB22}~\cite{ZhangBB22}, \href{../works/ArmstrongGOS22.pdf}{ArmstrongGOS22}~\cite{ArmstrongGOS22}, \href{../works/WinterMMW22.pdf}{WinterMMW22}~\cite{WinterMMW22}, \href{../works/GhandehariK22.pdf}{GhandehariK22}~\cite{GhandehariK22}...\href{../works/AbdennadherS99.pdf}{AbdennadherS99}~\cite{AbdennadherS99}, \href{../works/BeckF99.pdf}{BeckF99}~\cite{BeckF99}, \href{../works/BelhadjiI98.pdf}{BelhadjiI98}~\cite{BelhadjiI98}, \href{../works/FrostD98.pdf}{FrostD98}~\cite{FrostD98}, \href{../works/Caseau97.pdf}{Caseau97}~\cite{Caseau97}, \href{../works/NuijtenA96.pdf}{NuijtenA96}~\cite{NuijtenA96}, \href{../works/Puget95.pdf}{Puget95}~\cite{Puget95}, \href{../works/Touraivane95.pdf}{Touraivane95}~\cite{Touraivane95}, \href{../works/NuijtenA94.pdf}{NuijtenA94}~\cite{NuijtenA94}, \href{../works/EskeyZ90.pdf}{EskeyZ90}~\cite{EskeyZ90} (Total: 145) & \href{../works/Mehdizadeh-Somarin23.pdf}{Mehdizadeh-Somarin23}~\cite{Mehdizadeh-Somarin23}, \href{../works/AlakaP23.pdf}{AlakaP23}~\cite{AlakaP23}, \href{../works/MontemanniD23.pdf}{MontemanniD23}~\cite{MontemanniD23}, \href{../works/BofillCGGPSV23.pdf}{BofillCGGPSV23}~\cite{BofillCGGPSV23}, \href{../works/AkramNHRSA23.pdf}{AkramNHRSA23}~\cite{AkramNHRSA23}, \href{../works/ZhangJZL22.pdf}{ZhangJZL22}~\cite{ZhangJZL22}, \href{../works/BulckG22.pdf}{BulckG22}~\cite{BulckG22}, \href{../works/JuvinHL22.pdf}{JuvinHL22}~\cite{JuvinHL22}, \href{../works/Tassel22.pdf}{Tassel22}~\cite{Tassel22}, \href{../works/NaderiBZ22a.pdf}{NaderiBZ22a}~\cite{NaderiBZ22a}, \href{../works/AbohashimaEG21.pdf}{AbohashimaEG21}~\cite{AbohashimaEG21}, \href{../works/MengLZB21.pdf}{MengLZB21}~\cite{MengLZB21}, \href{../works/ZhangYW21.pdf}{ZhangYW21}~\cite{ZhangYW21}, \href{../works/MokhtarzadehTNF20.pdf}{MokhtarzadehTNF20}~\cite{MokhtarzadehTNF20}, \href{../works/RoshanaeiBAUB20.pdf}{RoshanaeiBAUB20}~\cite{RoshanaeiBAUB20}, \href{../works/GalleguillosKSB19.pdf}{GalleguillosKSB19}~\cite{GalleguillosKSB19}, \href{../works/AlakaPY19.pdf}{AlakaPY19}~\cite{AlakaPY19}, \href{../works/abs-1902-01193.pdf}{abs-1902-01193}~\cite{abs-1902-01193}, \href{../works/BhatnagarKL19.pdf}{BhatnagarKL19}~\cite{BhatnagarKL19}...\href{../works/HookerY02.pdf}{HookerY02}~\cite{HookerY02}, \href{../works/LimAHO02a.pdf}{LimAHO02a}~\cite{LimAHO02a}, \href{../works/AngelsmarkJ00.pdf}{AngelsmarkJ00}~\cite{AngelsmarkJ00}, \href{../works/RodosekWH99.pdf}{RodosekWH99}~\cite{RodosekWH99}, \href{../works/ChunCTY99.pdf}{ChunCTY99}~\cite{ChunCTY99}, \href{../works/RodosekW98.pdf}{RodosekW98}~\cite{RodosekW98}, \href{../works/LeeKLKKYHP97.pdf}{LeeKLKKYHP97}~\cite{LeeKLKKYHP97}, \href{../works/BeckDF97.pdf}{BeckDF97}~\cite{BeckDF97}, \href{../works/Simonis95.pdf}{Simonis95}~\cite{Simonis95}, \href{../works/WeilHFP95.pdf}{WeilHFP95}~\cite{WeilHFP95} (Total: 96)\\
\index{resource}\index{Scheduling!resource}resource &  0.50 & \href{../works/LuZZYW24.pdf}{LuZZYW24}~\cite{LuZZYW24}, \href{../works/ForbesHJST24.pdf}{ForbesHJST24}~\cite{ForbesHJST24}, \href{../works/BonninMNE24.pdf}{BonninMNE24}~\cite{BonninMNE24}, \href{../works/PrataAN23.pdf}{PrataAN23}~\cite{PrataAN23}, \href{../works/abs-2402-00459.pdf}{abs-2402-00459}~\cite{abs-2402-00459}, \href{../works/Fatemi-AnarakiTFV23.pdf}{Fatemi-AnarakiTFV23}~\cite{Fatemi-AnarakiTFV23}, \href{../works/PovedaAA23.pdf}{PovedaAA23}~\cite{PovedaAA23}, \href{../works/AlakaP23.pdf}{AlakaP23}~\cite{AlakaP23}, \href{../works/GuoZ23.pdf}{GuoZ23}~\cite{GuoZ23}, \href{../works/NaderiRR23.pdf}{NaderiRR23}~\cite{NaderiRR23}, \href{../works/GokPTGO23.pdf}{GokPTGO23}~\cite{GokPTGO23}, \href{../works/MarliereSPR23.pdf}{MarliereSPR23}~\cite{MarliereSPR23}, \href{../works/YuraszeckMCCR23.pdf}{YuraszeckMCCR23}~\cite{YuraszeckMCCR23}, \href{../works/CzerniachowskaWZ23.pdf}{CzerniachowskaWZ23}~\cite{CzerniachowskaWZ23}, \href{../works/JuvinHL23a.pdf}{JuvinHL23a}~\cite{JuvinHL23a}, \href{../works/TardivoDFMP23.pdf}{TardivoDFMP23}~\cite{TardivoDFMP23}, \href{../works/JuvinHHL23.pdf}{JuvinHHL23}~\cite{JuvinHHL23}, \href{../works/WessenCSFPM23.pdf}{WessenCSFPM23}~\cite{WessenCSFPM23}, \href{../works/ShaikhK23.pdf}{ShaikhK23}~\cite{ShaikhK23}...\href{../works/Pape94.pdf}{Pape94}~\cite{Pape94}, \href{../works/BeldiceanuC94.pdf}{BeldiceanuC94}~\cite{BeldiceanuC94}, \href{../works/Nuijten94.pdf}{Nuijten94}~\cite{Nuijten94}, \href{../works/AggounB93.pdf}{AggounB93}~\cite{AggounB93}, \href{../works/SmithC93.pdf}{SmithC93}~\cite{SmithC93}, \href{../works/MintonJPL92.pdf}{MintonJPL92}~\cite{MintonJPL92}, \href{../works/DincbasS91.pdf}{DincbasS91}~\cite{DincbasS91}, \href{../works/EskeyZ90.pdf}{EskeyZ90}~\cite{EskeyZ90}, \href{../works/FoxS90.pdf}{FoxS90}~\cite{FoxS90}, \href{../works/Prosser89.pdf}{Prosser89}~\cite{Prosser89} (Total: 511) & \href{../works/FalqueALM24.pdf}{FalqueALM24}~\cite{FalqueALM24}, \href{../works/Adelgren2023.pdf}{Adelgren2023}~\cite{Adelgren2023}, \href{../works/TasselGS23.pdf}{TasselGS23}~\cite{TasselGS23}, \href{../works/AbreuNP23.pdf}{AbreuNP23}~\cite{AbreuNP23}, \href{../works/abs-2306-05747.pdf}{abs-2306-05747}~\cite{abs-2306-05747}, \href{../works/Caballero23.pdf}{Caballero23}~\cite{Caballero23}, \href{../works/FrimodigECM23.pdf}{FrimodigECM23}~\cite{FrimodigECM23}, \href{../works/AfsarVPG23.pdf}{AfsarVPG23}~\cite{AfsarVPG23}, \href{../works/abs-2312-13682.pdf}{abs-2312-13682}~\cite{abs-2312-13682}, \href{../works/PerezGSL23.pdf}{PerezGSL23}~\cite{PerezGSL23}, \href{../works/IsikYA23.pdf}{IsikYA23}~\cite{IsikYA23}, \href{../works/NaderiBZR23.pdf}{NaderiBZR23}~\cite{NaderiBZR23}, \href{../works/Bit-Monnot23.pdf}{Bit-Monnot23}~\cite{Bit-Monnot23}, \href{../works/ElciOH22.pdf}{ElciOH22}~\cite{ElciOH22}, \href{../works/PohlAK22.pdf}{PohlAK22}~\cite{PohlAK22}, \href{../works/MullerMKP22.pdf}{MullerMKP22}~\cite{MullerMKP22}, \href{../works/SvancaraB22.pdf}{SvancaraB22}~\cite{SvancaraB22}, \href{../works/abs-2211-14492.pdf}{abs-2211-14492}~\cite{abs-2211-14492}, \href{../works/YuraszeckMPV22.pdf}{YuraszeckMPV22}~\cite{YuraszeckMPV22}...\href{../works/CarlssonKA99.pdf}{CarlssonKA99}~\cite{CarlssonKA99}, \href{../works/Darby-DowmanLMZ97.pdf}{Darby-DowmanLMZ97}~\cite{Darby-DowmanLMZ97}, \href{../works/LeeKLKKYHP97.pdf}{LeeKLKKYHP97}~\cite{LeeKLKKYHP97}, \href{../works/NuijtenA96.pdf}{NuijtenA96}~\cite{NuijtenA96}, \href{../works/Goltz95.pdf}{Goltz95}~\cite{Goltz95}, \href{../works/NuijtenA94.pdf}{NuijtenA94}~\cite{NuijtenA94}, \href{../works/ErtlK91.pdf}{ErtlK91}~\cite{ErtlK91}, \href{../works/MintonJPL90.pdf}{MintonJPL90}~\cite{MintonJPL90}, \href{../works/DincbasSH90.pdf}{DincbasSH90}~\cite{DincbasSH90}, \href{../works/Rit86.pdf}{Rit86}~\cite{Rit86} (Total: 91) & \href{../works/AkramNHRSA23.pdf}{AkramNHRSA23}~\cite{AkramNHRSA23}, \href{../works/MontemanniD23.pdf}{MontemanniD23}~\cite{MontemanniD23}, \href{../works/PenzDN23.pdf}{PenzDN23}~\cite{PenzDN23}, \href{../works/IklassovMR023.pdf}{IklassovMR023}~\cite{IklassovMR023}, \href{../works/SquillaciPR23.pdf}{SquillaciPR23}~\cite{SquillaciPR23}, \href{../works/ZhuSZW23.pdf}{ZhuSZW23}~\cite{ZhuSZW23}, \href{../works/ZhangJZL22.pdf}{ZhangJZL22}~\cite{ZhangJZL22}, \href{../works/EmdeZD22.pdf}{EmdeZD22}~\cite{EmdeZD22}, \href{../works/Tassel22.pdf}{Tassel22}~\cite{Tassel22}, \href{../works/Teppan22.pdf}{Teppan22}~\cite{Teppan22}, \href{../works/JungblutK22.pdf}{JungblutK22}~\cite{JungblutK22}, \href{../works/PopovicCGNC22.pdf}{PopovicCGNC22}~\cite{PopovicCGNC22}, \href{../works/ArmstrongGOS22.pdf}{ArmstrongGOS22}~\cite{ArmstrongGOS22}, \href{../works/AntuoriHHEN21.pdf}{AntuoriHHEN21}~\cite{AntuoriHHEN21}, \href{../works/FanXG21.pdf}{FanXG21}~\cite{FanXG21}, \href{../works/HamPK21.pdf}{HamPK21}~\cite{HamPK21}, \href{../works/AbreuAPNM21.pdf}{AbreuAPNM21}~\cite{AbreuAPNM21}, \href{../works/AbohashimaEG21.pdf}{AbohashimaEG21}~\cite{AbohashimaEG21}, \href{../works/KoehlerBFFHPSSS21.pdf}{KoehlerBFFHPSSS21}~\cite{KoehlerBFFHPSSS21}...\href{../works/BeldiceanuC01.pdf}{BeldiceanuC01}~\cite{BeldiceanuC01}, \href{../works/MartinPY01.pdf}{MartinPY01}~\cite{MartinPY01}, \href{../works/AngelsmarkJ00.pdf}{AngelsmarkJ00}~\cite{AngelsmarkJ00}, \href{../works/HookerO99.pdf}{HookerO99}~\cite{HookerO99}, \href{../works/JoLLH99.pdf}{JoLLH99}~\cite{JoLLH99}, \href{../works/RodosekWH99.pdf}{RodosekWH99}~\cite{RodosekWH99}, \href{../works/PesantGPR99.pdf}{PesantGPR99}~\cite{PesantGPR99}, \href{../works/RodosekW98.pdf}{RodosekW98}~\cite{RodosekW98}, \href{../works/RoweJCA96.pdf}{RoweJCA96}~\cite{RoweJCA96}, \href{../works/SmithBHW96.pdf}{SmithBHW96}~\cite{SmithBHW96} (Total: 94)\\
\index{scheduling}\index{Scheduling!scheduling}scheduling &  1.00 & \href{../works/PrataAN23.pdf}{PrataAN23}~\cite{PrataAN23}, \href{../works/ForbesHJST24.pdf}{ForbesHJST24}~\cite{ForbesHJST24}, \href{../works/abs-2402-00459.pdf}{abs-2402-00459}~\cite{abs-2402-00459}, \href{../works/LiLZDZW24.pdf}{LiLZDZW24}~\cite{LiLZDZW24}, \href{../works/LuZZYW24.pdf}{LuZZYW24}~\cite{LuZZYW24}, \href{../works/BonninMNE24.pdf}{BonninMNE24}~\cite{BonninMNE24}, \href{../works/ZhuSZW23.pdf}{ZhuSZW23}~\cite{ZhuSZW23}, \href{../works/IsikYA23.pdf}{IsikYA23}~\cite{IsikYA23}, \href{../works/abs-2306-05747.pdf}{abs-2306-05747}~\cite{abs-2306-05747}, \href{../works/JuvinHHL23.pdf}{JuvinHHL23}~\cite{JuvinHHL23}, \href{../works/TardivoDFMP23.pdf}{TardivoDFMP23}~\cite{TardivoDFMP23}, \href{../works/Fatemi-AnarakiTFV23.pdf}{Fatemi-AnarakiTFV23}~\cite{Fatemi-AnarakiTFV23}, \href{../works/Mehdizadeh-Somarin23.pdf}{Mehdizadeh-Somarin23}~\cite{Mehdizadeh-Somarin23}, \href{../works/KimCMLLP23.pdf}{KimCMLLP23}~\cite{KimCMLLP23}, \href{../works/FrimodigECM23.pdf}{FrimodigECM23}~\cite{FrimodigECM23}, \href{../works/LacknerMMWW23.pdf}{LacknerMMWW23}~\cite{LacknerMMWW23}, \href{../works/GurPAE23.pdf}{GurPAE23}~\cite{GurPAE23}, \href{../works/AlfieriGPS23.pdf}{AlfieriGPS23}~\cite{AlfieriGPS23}, \href{../works/WessenCSFPM23.pdf}{WessenCSFPM23}~\cite{WessenCSFPM23}...\href{../works/DincbasS91.pdf}{DincbasS91}~\cite{DincbasS91}, \href{../works/ErtlK91.pdf}{ErtlK91}~\cite{ErtlK91}, \href{../works/EskeyZ90.pdf}{EskeyZ90}~\cite{EskeyZ90}, \href{../works/FoxS90.pdf}{FoxS90}~\cite{FoxS90}, \href{../works/DincbasSH90.pdf}{DincbasSH90}~\cite{DincbasSH90}, \href{../works/MintonJPL90.pdf}{MintonJPL90}~\cite{MintonJPL90}, \href{../works/FeldmanG89.pdf}{FeldmanG89}~\cite{FeldmanG89}, \href{../works/Prosser89.pdf}{Prosser89}~\cite{Prosser89}, \href{../works/KengY89.pdf}{KengY89}~\cite{KengY89}, \href{../works/Rit86.pdf}{Rit86}~\cite{Rit86} (Total: 753) & \href{../works/FalqueALM24.pdf}{FalqueALM24}~\cite{FalqueALM24}, \href{../works/HebrardALLCMR22.pdf}{HebrardALLCMR22}~\cite{HebrardALLCMR22}, \href{../works/Alaka21.pdf}{Alaka21}~\cite{Alaka21}, \href{../works/AlakaPY19.pdf}{AlakaPY19}~\cite{AlakaPY19}, \href{../works/LiuLH19a.pdf}{LiuLH19a}~\cite{LiuLH19a}, \href{../works/BukchinR18.pdf}{BukchinR18}~\cite{BukchinR18}, \href{../works/Kameugne15.pdf}{Kameugne15}~\cite{Kameugne15}, \href{../works/GayHS15.pdf}{GayHS15}~\cite{GayHS15}, \href{../works/BessiereHMQW14.pdf}{BessiereHMQW14}~\cite{BessiereHMQW14}, \href{../works/HoundjiSWD14.pdf}{HoundjiSWD14}~\cite{HoundjiSWD14}, \href{../works/LetortCB13.pdf}{LetortCB13}~\cite{LetortCB13}, \href{../works/LetortBC12.pdf}{LetortBC12}~\cite{LetortBC12}, \href{../works/ClercqPBJ11.pdf}{ClercqPBJ11}~\cite{ClercqPBJ11}, \href{../works/ChapadosJR11.pdf}{ChapadosJR11}~\cite{ChapadosJR11}, \href{../works/Baptiste09.pdf}{Baptiste09}~\cite{Baptiste09}, \href{../works/Acuna-AgostMFG09.pdf}{Acuna-AgostMFG09}~\cite{Acuna-AgostMFG09}, \href{../works/abs-0907-0939.pdf}{abs-0907-0939}~\cite{abs-0907-0939}, \href{../works/GomesHS06.pdf}{GomesHS06}~\cite{GomesHS06}, \href{../works/HebrardTW05.pdf}{HebrardTW05}~\cite{HebrardTW05}...\href{../works/ValleMGT03.pdf}{ValleMGT03}~\cite{ValleMGT03}, \href{../works/Vilim03.pdf}{Vilim03}~\cite{Vilim03}, \href{../works/ElfJR03.pdf}{ElfJR03}~\cite{ElfJR03}, \href{../works/HookerY02.pdf}{HookerY02}~\cite{HookerY02}, \href{../works/BenoistGR02.pdf}{BenoistGR02}~\cite{BenoistGR02}, \href{../works/Vilim02.pdf}{Vilim02}~\cite{Vilim02}, \href{../works/RodriguezDG02.pdf}{RodriguezDG02}~\cite{RodriguezDG02}, \href{../works/BeldiceanuC01.pdf}{BeldiceanuC01}~\cite{BeldiceanuC01}, \href{../works/FrostD98.pdf}{FrostD98}~\cite{FrostD98}, \href{../works/CestaOS98.pdf}{CestaOS98}~\cite{CestaOS98} (Total: 33) & \href{../works/Hooker17.pdf}{Hooker17}~\cite{Hooker17}, \href{../works/TopalogluSS12.pdf}{TopalogluSS12}~\cite{TopalogluSS12}, \href{../works/AchterbergBKW08.pdf}{AchterbergBKW08}~\cite{AchterbergBKW08}, \href{../works/RossiTHP07.pdf}{RossiTHP07}~\cite{RossiTHP07}, \href{../works/CambazardJ05.pdf}{CambazardJ05}~\cite{CambazardJ05}, \href{../works/AbrilSB05.pdf}{AbrilSB05}~\cite{AbrilSB05}, \href{../works/WolfS05a.pdf}{WolfS05a}~\cite{WolfS05a}, \href{../works/Hooker05b.pdf}{Hooker05b}~\cite{Hooker05b}, \href{../works/VanczaM01.pdf}{VanczaM01}~\cite{VanczaM01}, \href{../works/LeeKLKKYHP97.pdf}{LeeKLKKYHP97}~\cite{LeeKLKKYHP97}, \href{../works/Davis87.pdf}{Davis87}~\cite{Davis87}\\
\index{task}\index{Scheduling!task}task &  1.00 & \href{../works/PrataAN23.pdf}{PrataAN23}~\cite{PrataAN23}, \href{../works/FalqueALM24.pdf}{FalqueALM24}~\cite{FalqueALM24}, \href{../works/BonninMNE24.pdf}{BonninMNE24}~\cite{BonninMNE24}, \href{../works/LuZZYW24.pdf}{LuZZYW24}~\cite{LuZZYW24}, \href{../works/ForbesHJST24.pdf}{ForbesHJST24}~\cite{ForbesHJST24}, \href{../works/abs-2402-00459.pdf}{abs-2402-00459}~\cite{abs-2402-00459}, \href{../works/JuvinHHL23.pdf}{JuvinHHL23}~\cite{JuvinHHL23}, \href{../works/YuraszeckMCCR23.pdf}{YuraszeckMCCR23}~\cite{YuraszeckMCCR23}, \href{../works/AfsarVPG23.pdf}{AfsarVPG23}~\cite{AfsarVPG23}, \href{../works/KameugneFND23.pdf}{KameugneFND23}~\cite{KameugneFND23}, \href{../works/GokPTGO23.pdf}{GokPTGO23}~\cite{GokPTGO23}, \href{../works/AkramNHRSA23.pdf}{AkramNHRSA23}~\cite{AkramNHRSA23}, \href{../works/CzerniachowskaWZ23.pdf}{CzerniachowskaWZ23}~\cite{CzerniachowskaWZ23}, \href{../works/Adelgren2023.pdf}{Adelgren2023}~\cite{Adelgren2023}, \href{../works/AlakaP23.pdf}{AlakaP23}~\cite{AlakaP23}, \href{../works/WangB23.pdf}{WangB23}~\cite{WangB23}, \href{../works/PovedaAA23.pdf}{PovedaAA23}~\cite{PovedaAA23}, \href{../works/JuvinHL23.pdf}{JuvinHL23}~\cite{JuvinHL23}, \href{../works/Fatemi-AnarakiTFV23.pdf}{Fatemi-AnarakiTFV23}~\cite{Fatemi-AnarakiTFV23}...\href{../works/BaptisteP95.pdf}{BaptisteP95}~\cite{BaptisteP95}, \href{../works/BeldiceanuC94.pdf}{BeldiceanuC94}~\cite{BeldiceanuC94}, \href{../works/Nuijten94.pdf}{Nuijten94}~\cite{Nuijten94}, \href{../works/Pape94.pdf}{Pape94}~\cite{Pape94}, \href{../works/AggounB93.pdf}{AggounB93}~\cite{AggounB93}, \href{../works/MintonJPL92.pdf}{MintonJPL92}~\cite{MintonJPL92}, \href{../works/DincbasSH90.pdf}{DincbasSH90}~\cite{DincbasSH90}, \href{../works/EskeyZ90.pdf}{EskeyZ90}~\cite{EskeyZ90}, \href{../works/MintonJPL90.pdf}{MintonJPL90}~\cite{MintonJPL90}, \href{../works/KengY89.pdf}{KengY89}~\cite{KengY89} (Total: 340) & \href{../works/JuvinHL23a.pdf}{JuvinHL23a}~\cite{JuvinHL23a}, \href{../works/IklassovMR023.pdf}{IklassovMR023}~\cite{IklassovMR023}, \href{../works/MontemanniD23a.pdf}{MontemanniD23a}~\cite{MontemanniD23a}, \href{../works/Bit-Monnot23.pdf}{Bit-Monnot23}~\cite{Bit-Monnot23}, \href{../works/IsikYA23.pdf}{IsikYA23}~\cite{IsikYA23}, \href{../works/SquillaciPR23.pdf}{SquillaciPR23}~\cite{SquillaciPR23}, \href{../works/FrimodigECM23.pdf}{FrimodigECM23}~\cite{FrimodigECM23}, \href{../works/NaderiBZR23.pdf}{NaderiBZR23}~\cite{NaderiBZR23}, \href{../works/MontemanniD23.pdf}{MontemanniD23}~\cite{MontemanniD23}, \href{../works/LacknerMMWW23.pdf}{LacknerMMWW23}~\cite{LacknerMMWW23}, \href{../works/ShaikhK23.pdf}{ShaikhK23}~\cite{ShaikhK23}, \href{../works/WinterMMW22.pdf}{WinterMMW22}~\cite{WinterMMW22}, \href{../works/PopovicCGNC22.pdf}{PopovicCGNC22}~\cite{PopovicCGNC22}, \href{../works/AbreuN22.pdf}{AbreuN22}~\cite{AbreuN22}, \href{../works/MengGRZSC22.pdf}{MengGRZSC22}~\cite{MengGRZSC22}, \href{../works/FarsiTM22.pdf}{FarsiTM22}~\cite{FarsiTM22}, \href{../works/OujanaAYB22.pdf}{OujanaAYB22}~\cite{OujanaAYB22}, \href{../works/Tassel22.pdf}{Tassel22}~\cite{Tassel22}, \href{../works/YuraszeckMPV22.pdf}{YuraszeckMPV22}~\cite{YuraszeckMPV22}...\href{../works/RodosekWH99.pdf}{RodosekWH99}~\cite{RodosekWH99}, \href{../works/HookerO99.pdf}{HookerO99}~\cite{HookerO99}, \href{../works/PapaB98.pdf}{PapaB98}~\cite{PapaB98}, \href{../works/BeckDDF98.pdf}{BeckDDF98}~\cite{BeckDDF98}, \href{../works/PapeB97.pdf}{PapeB97}~\cite{PapeB97}, \href{../works/LeeKLKKYHP97.pdf}{LeeKLKKYHP97}~\cite{LeeKLKKYHP97}, \href{../works/MurphyRFSS97.pdf}{MurphyRFSS97}~\cite{MurphyRFSS97}, \href{../works/RoweJCA96.pdf}{RoweJCA96}~\cite{RoweJCA96}, \href{../works/WeilHFP95.pdf}{WeilHFP95}~\cite{WeilHFP95}, \href{../works/FoxS90.pdf}{FoxS90}~\cite{FoxS90} (Total: 99) & \href{../works/LiLZDZW24.pdf}{LiLZDZW24}~\cite{LiLZDZW24}, \href{../works/BofillCGGPSV23.pdf}{BofillCGGPSV23}~\cite{BofillCGGPSV23}, \href{../works/ZhuSZW23.pdf}{ZhuSZW23}~\cite{ZhuSZW23}, \href{../works/TardivoDFMP23.pdf}{TardivoDFMP23}~\cite{TardivoDFMP23}, \href{../works/PerezGSL23.pdf}{PerezGSL23}~\cite{PerezGSL23}, \href{../works/GuoZ23.pdf}{GuoZ23}~\cite{GuoZ23}, \href{../works/abs-2306-05747.pdf}{abs-2306-05747}~\cite{abs-2306-05747}, \href{../works/MarliereSPR23.pdf}{MarliereSPR23}~\cite{MarliereSPR23}, \href{../works/NaderiRR23.pdf}{NaderiRR23}~\cite{NaderiRR23}, \href{../works/TasselGS23.pdf}{TasselGS23}~\cite{TasselGS23}, \href{../works/EfthymiouY23.pdf}{EfthymiouY23}~\cite{EfthymiouY23}, \href{../works/abs-2312-13682.pdf}{abs-2312-13682}~\cite{abs-2312-13682}, \href{../works/Mehdizadeh-Somarin23.pdf}{Mehdizadeh-Somarin23}~\cite{Mehdizadeh-Somarin23}, \href{../works/GhandehariK22.pdf}{GhandehariK22}~\cite{GhandehariK22}, \href{../works/ZhangBB22.pdf}{ZhangBB22}~\cite{ZhangBB22}, \href{../works/Teppan22.pdf}{Teppan22}~\cite{Teppan22}, \href{../works/NaqviAIAAA22.pdf}{NaqviAIAAA22}~\cite{NaqviAIAAA22}, \href{../works/ZhangJZL22.pdf}{ZhangJZL22}~\cite{ZhangJZL22}, \href{../works/EmdeZD22.pdf}{EmdeZD22}~\cite{EmdeZD22}...\href{../works/SmithBHW96.pdf}{SmithBHW96}~\cite{SmithBHW96}, \href{../works/Puget95.pdf}{Puget95}~\cite{Puget95}, \href{../works/Simonis95.pdf}{Simonis95}~\cite{Simonis95}, \href{../works/Touraivane95.pdf}{Touraivane95}~\cite{Touraivane95}, \href{../works/YoshikawaKNW94.pdf}{YoshikawaKNW94}~\cite{YoshikawaKNW94}, \href{../works/CrawfordB94.pdf}{CrawfordB94}~\cite{CrawfordB94}, \href{../works/ErtlK91.pdf}{ErtlK91}~\cite{ErtlK91}, \href{../works/DincbasS91.pdf}{DincbasS91}~\cite{DincbasS91}, \href{../works/Davis87.pdf}{Davis87}~\cite{Davis87}, \href{../works/Valdes87.pdf}{Valdes87}~\cite{Valdes87} (Total: 155)\\
\end{longtable}
}

\clearpage
\subsection{Concept Type CP}
\label{sec:CP}
\label{CP}
{\scriptsize
\begin{longtable}{p{3cm}r>{\raggedright\arraybackslash}p{6cm}>{\raggedright\arraybackslash}p{6cm}>{\raggedright\arraybackslash}p{8cm}}
\rowcolor{white}\caption{Works for Concepts of Type CP (Total 10 Concepts, 10 Used)}\\ \toprule
\rowcolor{white}Keyword & Weight & High & Medium & Low\\ \midrule\endhead
\bottomrule
\endfoot
\index{CLP}\index{CP!CLP}CLP &  1.00 & \href{../works/BadicaBI20.pdf}{BadicaBI20}~\cite{BadicaBI20}, \href{../works/WikarekS19.pdf}{WikarekS19}~\cite{WikarekS19}, \href{../works/BadicaBIL19.pdf}{BadicaBIL19}~\cite{BadicaBIL19}, \href{../works/RenT09.pdf}{RenT09}~\cite{RenT09}, \href{../works/RoePS05.pdf}{RoePS05}~\cite{RoePS05}, \href{../works/Kuchcinski03.pdf}{Kuchcinski03}~\cite{Kuchcinski03}, \href{../works/ZhuS02.pdf}{ZhuS02}~\cite{ZhuS02}, \href{../works/Bartak02.pdf}{Bartak02}~\cite{Bartak02}, \href{../works/Thorsteinsson01.pdf}{Thorsteinsson01}~\cite{Thorsteinsson01}, \href{../works/TrentesauxPT01.pdf}{TrentesauxPT01}~\cite{TrentesauxPT01}, \href{../works/MartinPY01.pdf}{MartinPY01}~\cite{MartinPY01}, \href{../works/BosiM2001.pdf}{BosiM2001}~\cite{BosiM2001}, \href{../works/HarjunkoskiJG00.pdf}{HarjunkoskiJG00}~\cite{HarjunkoskiJG00}, \href{../works/Simonis99.pdf}{Simonis99}~\cite{Simonis99}, \href{../works/RodosekWH99.pdf}{RodosekWH99}~\cite{RodosekWH99}, \href{../works/AbdennadherS99.pdf}{AbdennadherS99}~\cite{AbdennadherS99}, \href{../works/GruianK98.pdf}{GruianK98}~\cite{GruianK98}, \href{../works/RodosekW98.pdf}{RodosekW98}~\cite{RodosekW98}, \href{../works/LammaMM97.pdf}{LammaMM97}~\cite{LammaMM97}, \href{../works/Darby-DowmanLMZ97.pdf}{Darby-DowmanLMZ97}~\cite{Darby-DowmanLMZ97}, \href{../works/FalaschiGMP97.pdf}{FalaschiGMP97}~\cite{FalaschiGMP97}, \href{../works/BrusoniCLMMT96.pdf}{BrusoniCLMMT96}~\cite{BrusoniCLMMT96}, \href{../works/Wallace96.pdf}{Wallace96}~\cite{Wallace96}, \href{../works/Touraivane95.pdf}{Touraivane95}~\cite{Touraivane95}, \href{../works/Simonis95a.pdf}{Simonis95a}~\cite{Simonis95a}, \href{../works/Goltz95.pdf}{Goltz95}~\cite{Goltz95} & \href{../works/ShaikhK23.pdf}{ShaikhK23}~\cite{ShaikhK23}, \href{../works/KonowalenkoMM19.pdf}{KonowalenkoMM19}~\cite{KonowalenkoMM19}, \href{../works/HookerH17.pdf}{HookerH17}~\cite{HookerH17}, \href{../works/Malapert11.pdf}{Malapert11}~\cite{Malapert11}, \href{../works/AchterbergBKW08.pdf}{AchterbergBKW08}~\cite{AchterbergBKW08}, \href{../works/Baptiste02.pdf}{Baptiste02}~\cite{Baptiste02}, \href{../works/HookerOTK00.pdf}{HookerOTK00}~\cite{HookerOTK00}, \href{../works/HookerO99.pdf}{HookerO99}~\cite{HookerO99}, \href{../works/Simonis95.pdf}{Simonis95}~\cite{Simonis95}, \href{../works/FoxS90.pdf}{FoxS90}~\cite{FoxS90}, \href{../works/DincbasSH90.pdf}{DincbasSH90}~\cite{DincbasSH90} & \href{../works/KameugneFND23.pdf}{KameugneFND23}~\cite{KameugneFND23}, \href{../works/CilKLO22.pdf}{CilKLO22}~\cite{CilKLO22}, \href{../works/ColT22.pdf}{ColT22}~\cite{ColT22}, \href{../works/GeitzGSSW22.pdf}{GeitzGSSW22}~\cite{GeitzGSSW22}, \href{../works/ZarandiASC20.pdf}{ZarandiASC20}~\cite{ZarandiASC20}, \href{../works/abs-1902-01193.pdf}{abs-1902-01193}~\cite{abs-1902-01193}, \href{../works/GokgurHO18.pdf}{GokgurHO18}~\cite{GokgurHO18}, \href{../works/HarjunkoskiMBC14.pdf}{HarjunkoskiMBC14}~\cite{HarjunkoskiMBC14}, \href{../works/KameugneFSN14.pdf}{KameugneFSN14}~\cite{KameugneFSN14}, \href{../works/Kameugne14.pdf}{Kameugne14}~\cite{Kameugne14}, \href{../works/OzturkTHO13.pdf}{OzturkTHO13}~\cite{OzturkTHO13}, \href{../works/KameugneFSN11.pdf}{KameugneFSN11}~\cite{KameugneFSN11}, \href{../works/Balduccini11.pdf}{Balduccini11}~\cite{Balduccini11}, \href{../works/ZeballosNH11.pdf}{ZeballosNH11}~\cite{ZeballosNH11}, \href{../works/ZeballosCM10.pdf}{ZeballosCM10}~\cite{ZeballosCM10}, \href{../works/LopesCSM10.pdf}{LopesCSM10}~\cite{LopesCSM10}, \href{../works/AronssonBK09.pdf}{AronssonBK09}~\cite{AronssonBK09}, \href{../works/ZeballosM09.pdf}{ZeballosM09}~\cite{ZeballosM09}, \href{../works/BocewiczBB09.pdf}{BocewiczBB09}~\cite{BocewiczBB09}...\href{../works/Zhou97.pdf}{Zhou97}~\cite{Zhou97}, \href{../works/Zhou96.pdf}{Zhou96}~\cite{Zhou96}, \href{../works/Colombani96.pdf}{Colombani96}~\cite{Colombani96}, \href{../works/SmithBHW96.pdf}{SmithBHW96}~\cite{SmithBHW96}, \href{../works/NuijtenA96.pdf}{NuijtenA96}~\cite{NuijtenA96}, \href{../works/SimonisC95.pdf}{SimonisC95}~\cite{SimonisC95}, \href{../works/Puget95.pdf}{Puget95}~\cite{Puget95}, \href{../works/BaptisteP95.pdf}{BaptisteP95}~\cite{BaptisteP95}, \href{../works/Pape94.pdf}{Pape94}~\cite{Pape94}, \href{../works/NuijtenA94.pdf}{NuijtenA94}~\cite{NuijtenA94} (Total: 71)\\
\index{COP}\index{CP!COP}COP &  1.00 & \href{../works/Lemos21.pdf}{Lemos21}~\cite{Lemos21}, \href{../works/Godet21a.pdf}{Godet21a}~\cite{Godet21a}, \href{../works/Groleaz21.pdf}{Groleaz21}~\cite{Groleaz21}, \href{../works/PourDERB18.pdf}{PourDERB18}~\cite{PourDERB18}, \href{../works/Dejemeppe16.pdf}{Dejemeppe16}~\cite{Dejemeppe16}, \href{../works/AlesioNBG14.pdf}{AlesioNBG14}~\cite{AlesioNBG14}, \href{../works/PengLC14.pdf}{PengLC14}~\cite{PengLC14}, \href{../works/HeinzSB13.pdf}{HeinzSB13}~\cite{HeinzSB13}, \href{../works/Malapert11.pdf}{Malapert11}~\cite{Malapert11}, \href{../works/Dorndorf2000.pdf}{Dorndorf2000}~\cite{Dorndorf2000} & \href{../works/FalqueALM24.pdf}{FalqueALM24}~\cite{FalqueALM24}, \href{../works/FetgoD22.pdf}{FetgoD22}~\cite{FetgoD22}, \href{../works/CarlssonJL17.pdf}{CarlssonJL17}~\cite{CarlssonJL17}, \href{../works/Beck99.pdf}{Beck99}~\cite{Beck99}, \href{../works/BeckDDF98.pdf}{BeckDDF98}~\cite{BeckDDF98} & \href{../works/BoudreaultSLQ22.pdf}{BoudreaultSLQ22}~\cite{BoudreaultSLQ22}, \href{../works/ColT22.pdf}{ColT22}~\cite{ColT22}, \href{../works/Edis21.pdf}{Edis21}~\cite{Edis21}, \href{../works/FallahiAC20.pdf}{FallahiAC20}~\cite{FallahiAC20}, \href{../works/PinarbasiAY19.pdf}{PinarbasiAY19}~\cite{PinarbasiAY19}, \href{../works/LiuCGM17.pdf}{LiuCGM17}~\cite{LiuCGM17}, \href{../works/Froger16.pdf}{Froger16}~\cite{Froger16}, \href{../works/AmadiniGM16.pdf}{AmadiniGM16}~\cite{AmadiniGM16}, \href{../works/Derrien15.pdf}{Derrien15}~\cite{Derrien15}, \href{../works/GrimesH15.pdf}{GrimesH15}~\cite{GrimesH15}, \href{../works/DejemeppeD14.pdf}{DejemeppeD14}~\cite{DejemeppeD14}, \href{../works/Clercq12.pdf}{Clercq12}~\cite{Clercq12}, \href{../works/LimtanyakulS12.pdf}{LimtanyakulS12}~\cite{LimtanyakulS12}, \href{../works/ChenGPSH10.pdf}{ChenGPSH10}~\cite{ChenGPSH10}, \href{../works/Wolf09.pdf}{Wolf09}~\cite{Wolf09}, \href{../works/ElhouraniDM07.pdf}{ElhouraniDM07}~\cite{ElhouraniDM07}, \href{../works/DilkinaDH05.pdf}{DilkinaDH05}~\cite{DilkinaDH05}, \href{../works/KanetAG04.pdf}{KanetAG04}~\cite{KanetAG04}, \href{../works/Elkhyari03.pdf}{Elkhyari03}~\cite{Elkhyari03}, \href{../works/BeckF98.pdf}{BeckF98}~\cite{BeckF98}, \href{../works/FoxS90.pdf}{FoxS90}~\cite{FoxS90}\\
\index{CP}\index{CP!CP}CP &  1.00 & \href{../works/LuZZYW24.pdf}{LuZZYW24}~\cite{LuZZYW24}, \href{../works/FalqueALM24.pdf}{FalqueALM24}~\cite{FalqueALM24}, \href{../works/BonninMNE24.pdf}{BonninMNE24}~\cite{BonninMNE24}, \href{../works/abs-2402-00459.pdf}{abs-2402-00459}~\cite{abs-2402-00459}, \href{../works/ForbesHJST24.pdf}{ForbesHJST24}~\cite{ForbesHJST24}, \href{../works/PrataAN23.pdf}{PrataAN23}~\cite{PrataAN23}, \href{../works/TardivoDFMP23.pdf}{TardivoDFMP23}~\cite{TardivoDFMP23}, \href{../works/AkramNHRSA23.pdf}{AkramNHRSA23}~\cite{AkramNHRSA23}, \href{../works/FrimodigECM23.pdf}{FrimodigECM23}~\cite{FrimodigECM23}, \href{../works/NaderiRR23.pdf}{NaderiRR23}~\cite{NaderiRR23}, \href{../works/LacknerMMWW23.pdf}{LacknerMMWW23}~\cite{LacknerMMWW23}, \href{../works/AlakaP23.pdf}{AlakaP23}~\cite{AlakaP23}, \href{../works/Mehdizadeh-Somarin23.pdf}{Mehdizadeh-Somarin23}~\cite{Mehdizadeh-Somarin23}, \href{../works/TasselGS23.pdf}{TasselGS23}~\cite{TasselGS23}, \href{../works/WessenCSFPM23.pdf}{WessenCSFPM23}~\cite{WessenCSFPM23}, \href{../works/AbreuPNF23.pdf}{AbreuPNF23}~\cite{AbreuPNF23}, \href{../works/Fatemi-AnarakiTFV23.pdf}{Fatemi-AnarakiTFV23}~\cite{Fatemi-AnarakiTFV23}, \href{../works/YuraszeckMCCR23.pdf}{YuraszeckMCCR23}~\cite{YuraszeckMCCR23}, \href{../works/PovedaAA23.pdf}{PovedaAA23}~\cite{PovedaAA23}...\href{../works/RodriguezDG02.pdf}{RodriguezDG02}~\cite{RodriguezDG02}, \href{../works/Timpe02.pdf}{Timpe02}~\cite{Timpe02}, \href{../works/Baptiste02.pdf}{Baptiste02}~\cite{Baptiste02}, \href{../works/JainG01.pdf}{JainG01}~\cite{JainG01}, \href{../works/VanczaM01.pdf}{VanczaM01}~\cite{VanczaM01}, \href{../works/FocacciLN00.pdf}{FocacciLN00}~\cite{FocacciLN00}, \href{../works/Refalo00.pdf}{Refalo00}~\cite{Refalo00}, \href{../works/Beck99.pdf}{Beck99}~\cite{Beck99}, \href{../works/PesantGPR99.pdf}{PesantGPR99}~\cite{PesantGPR99}, \href{../works/BeckDDF98.pdf}{BeckDDF98}~\cite{BeckDDF98} (Total: 448) & \href{../works/NaderiBZR23.pdf}{NaderiBZR23}~\cite{NaderiBZR23}, \href{../works/NaqviAIAAA22.pdf}{NaqviAIAAA22}~\cite{NaqviAIAAA22}, \href{../works/Tassel22.pdf}{Tassel22}~\cite{Tassel22}, \href{../works/OuelletQ22.pdf}{OuelletQ22}~\cite{OuelletQ22}, \href{../works/Lemos21.pdf}{Lemos21}~\cite{Lemos21}, \href{../works/WessenCS20.pdf}{WessenCS20}~\cite{WessenCS20}, \href{../works/AntunesABD20.pdf}{AntunesABD20}~\cite{AntunesABD20}, \href{../works/BadicaBI20.pdf}{BadicaBI20}~\cite{BadicaBI20}, \href{../works/GroleazNS20a.pdf}{GroleazNS20a}~\cite{GroleazNS20a}, \href{../works/GurEA19.pdf}{GurEA19}~\cite{GurEA19}, \href{../works/LiuLH19a.pdf}{LiuLH19a}~\cite{LiuLH19a}, \href{../works/LiuLH19.pdf}{LiuLH19}~\cite{LiuLH19}, \href{../works/FahimiOQ18.pdf}{FahimiOQ18}~\cite{FahimiOQ18}, \href{../works/AstrandJZ18.pdf}{AstrandJZ18}~\cite{AstrandJZ18}, \href{../works/AntunesABD18.pdf}{AntunesABD18}~\cite{AntunesABD18}, \href{../works/OuelletQ18.pdf}{OuelletQ18}~\cite{OuelletQ18}, \href{../works/LiuLH18.pdf}{LiuLH18}~\cite{LiuLH18}, \href{../works/Pralet17.pdf}{Pralet17}~\cite{Pralet17}, \href{../works/GelainPRVW17.pdf}{GelainPRVW17}~\cite{GelainPRVW17}...\href{../works/VerfaillieL01.pdf}{VerfaillieL01}~\cite{VerfaillieL01}, \href{../works/BosiM2001.pdf}{BosiM2001}~\cite{BosiM2001}, \href{../works/EreminW01.pdf}{EreminW01}~\cite{EreminW01}, \href{../works/Thorsteinsson01.pdf}{Thorsteinsson01}~\cite{Thorsteinsson01}, \href{../works/HeipckeCCS00.pdf}{HeipckeCCS00}~\cite{HeipckeCCS00}, \href{../works/SakkoutW00.pdf}{SakkoutW00}~\cite{SakkoutW00}, \href{../works/BeckF00.pdf}{BeckF00}~\cite{BeckF00}, \href{../works/Simonis99.pdf}{Simonis99}~\cite{Simonis99}, \href{../works/Zhou97.pdf}{Zhou97}~\cite{Zhou97}, \href{../works/MorgadoM97.pdf}{MorgadoM97}~\cite{MorgadoM97} (Total: 88) & \href{../works/LiLZDZW24.pdf}{LiLZDZW24}~\cite{LiLZDZW24}, \href{../works/IklassovMR023.pdf}{IklassovMR023}~\cite{IklassovMR023}, \href{../works/Adelgren2023.pdf}{Adelgren2023}~\cite{Adelgren2023}, \href{../works/SvancaraB22.pdf}{SvancaraB22}~\cite{SvancaraB22}, \href{../works/JungblutK22.pdf}{JungblutK22}~\cite{JungblutK22}, \href{../works/HanenKP21.pdf}{HanenKP21}~\cite{HanenKP21}, \href{../works/FachiniA20.pdf}{FachiniA20}~\cite{FachiniA20}, \href{../works/KletzanderM20.pdf}{KletzanderM20}~\cite{KletzanderM20}, \href{../works/BadicaBIL19.pdf}{BadicaBIL19}~\cite{BadicaBIL19}, \href{../works/ArkhipovBL19.pdf}{ArkhipovBL19}~\cite{ArkhipovBL19}, \href{../works/KonowalenkoMM19.pdf}{KonowalenkoMM19}~\cite{KonowalenkoMM19}, \href{../works/WikarekS19.pdf}{WikarekS19}~\cite{WikarekS19}, \href{../works/BaptisteB18.pdf}{BaptisteB18}~\cite{BaptisteB18}, \href{../works/He0GLW18.pdf}{He0GLW18}~\cite{He0GLW18}, \href{../works/GomesM17.pdf}{GomesM17}~\cite{GomesM17}, \href{../works/NattafAL17.pdf}{NattafAL17}~\cite{NattafAL17}, \href{../works/BofillCSV17.pdf}{BofillCSV17}~\cite{BofillCSV17}, \href{../works/TranDRFWOVB16.pdf}{TranDRFWOVB16}~\cite{TranDRFWOVB16}, \href{../works/Madi-WambaB16.pdf}{Madi-WambaB16}~\cite{Madi-WambaB16}...\href{../works/HarjunkoskiJG00.pdf}{HarjunkoskiJG00}~\cite{HarjunkoskiJG00}, \href{../works/HookerOTK00.pdf}{HookerOTK00}~\cite{HookerOTK00}, \href{../works/WatsonBHW99.pdf}{WatsonBHW99}~\cite{WatsonBHW99}, \href{../works/ChunCTY99.pdf}{ChunCTY99}~\cite{ChunCTY99}, \href{../works/FrostD98.pdf}{FrostD98}~\cite{FrostD98}, \href{../works/RodosekW98.pdf}{RodosekW98}~\cite{RodosekW98}, \href{../works/CestaOS98.pdf}{CestaOS98}~\cite{CestaOS98}, \href{../works/BeckF98.pdf}{BeckF98}~\cite{BeckF98}, \href{../works/Darby-DowmanLMZ97.pdf}{Darby-DowmanLMZ97}~\cite{Darby-DowmanLMZ97}, \href{../works/Caseau97.pdf}{Caseau97}~\cite{Caseau97} (Total: 107)\\
\index{CSP}\index{CP!CSP}CSP &  1.00 & \href{../works/ZhuSZW23.pdf}{ZhuSZW23}~\cite{ZhuSZW23}, \href{../works/FetgoD22.pdf}{FetgoD22}~\cite{FetgoD22}, \href{../works/ColT22.pdf}{ColT22}~\cite{ColT22}, \href{../works/Lemos21.pdf}{Lemos21}~\cite{Lemos21}, \href{../works/Groleaz21.pdf}{Groleaz21}~\cite{Groleaz21}, \href{../works/Godet21a.pdf}{Godet21a}~\cite{Godet21a}, \href{../works/PinarbasiAY19.pdf}{PinarbasiAY19}~\cite{PinarbasiAY19}, \href{../works/PourDERB18.pdf}{PourDERB18}~\cite{PourDERB18}, \href{../works/TangLWSK18.pdf}{TangLWSK18}~\cite{TangLWSK18}, \href{../works/GokgurHO18.pdf}{GokgurHO18}~\cite{GokgurHO18}, \href{../works/Nattaf16.pdf}{Nattaf16}~\cite{Nattaf16}, \href{../works/Fahimi16.pdf}{Fahimi16}~\cite{Fahimi16}, \href{../works/ZarandiKS16.pdf}{ZarandiKS16}~\cite{ZarandiKS16}, \href{../works/Dejemeppe16.pdf}{Dejemeppe16}~\cite{Dejemeppe16}, \href{../works/Siala15a.pdf}{Siala15a}~\cite{Siala15a}, \href{../works/GayHLS15.pdf}{GayHLS15}~\cite{GayHLS15}, \href{../works/ChunS14.pdf}{ChunS14}~\cite{ChunS14}, \href{../works/PengLC14.pdf}{PengLC14}~\cite{PengLC14}, \href{../works/Kameugne14.pdf}{Kameugne14}~\cite{Kameugne14}...\href{../works/ChunCTY99.pdf}{ChunCTY99}~\cite{ChunCTY99}, \href{../works/BeckF98.pdf}{BeckF98}~\cite{BeckF98}, \href{../works/BeckDDF98.pdf}{BeckDDF98}~\cite{BeckDDF98}, \href{../works/SadehF96.pdf}{SadehF96}~\cite{SadehF96}, \href{../works/SmithBHW96.pdf}{SmithBHW96}~\cite{SmithBHW96}, \href{../works/Nuijten94.pdf}{Nuijten94}~\cite{Nuijten94}, \href{../works/MintonJPL92.pdf}{MintonJPL92}~\cite{MintonJPL92}, \href{../works/MintonJPL90.pdf}{MintonJPL90}~\cite{MintonJPL90}, \href{../works/FoxS90.pdf}{FoxS90}~\cite{FoxS90}, \href{../works/Prosser89.pdf}{Prosser89}~\cite{Prosser89} (Total: 63) & \href{../works/JuvinHHL23.pdf}{JuvinHHL23}~\cite{JuvinHHL23}, \href{../works/TardivoDFMP23.pdf}{TardivoDFMP23}~\cite{TardivoDFMP23}, \href{../works/Edis21.pdf}{Edis21}~\cite{Edis21}, \href{../works/ZarandiASC20.pdf}{ZarandiASC20}~\cite{ZarandiASC20}, \href{../works/FallahiAC20.pdf}{FallahiAC20}~\cite{FallahiAC20}, \href{../works/LiuLH19.pdf}{LiuLH19}~\cite{LiuLH19}, \href{../works/Caballero19.pdf}{Caballero19}~\cite{Caballero19}, \href{../works/German18.pdf}{German18}~\cite{German18}, \href{../works/CarlssonJL17.pdf}{CarlssonJL17}~\cite{CarlssonJL17}, \href{../works/Madi-WambaLOBM17.pdf}{Madi-WambaLOBM17}~\cite{Madi-WambaLOBM17}, \href{../works/GelainPRVW17.pdf}{GelainPRVW17}~\cite{GelainPRVW17}, \href{../works/LiuCGM17.pdf}{LiuCGM17}~\cite{LiuCGM17}, \href{../works/Derrien15.pdf}{Derrien15}~\cite{Derrien15}, \href{../works/GrimesH15.pdf}{GrimesH15}~\cite{GrimesH15}, \href{../works/BofillEGPSV14.pdf}{BofillEGPSV14}~\cite{BofillEGPSV14}, \href{../works/ChuGNSW13.pdf}{ChuGNSW13}~\cite{ChuGNSW13}, \href{../works/ZhangLS12.pdf}{ZhangLS12}~\cite{ZhangLS12}, \href{../works/MalapertGR12.pdf}{MalapertGR12}~\cite{MalapertGR12}, \href{../works/BartakS11.pdf}{BartakS11}~\cite{BartakS11}...\href{../works/KorbaaYG99.pdf}{KorbaaYG99}~\cite{KorbaaYG99}, \href{../works/NuijtenP98.pdf}{NuijtenP98}~\cite{NuijtenP98}, \href{../works/PembertonG98.pdf}{PembertonG98}~\cite{PembertonG98}, \href{../works/OddiS97.pdf}{OddiS97}~\cite{OddiS97}, \href{../works/LammaMM97.pdf}{LammaMM97}~\cite{LammaMM97}, \href{../works/Wallace96.pdf}{Wallace96}~\cite{Wallace96}, \href{../works/NuijtenA96.pdf}{NuijtenA96}~\cite{NuijtenA96}, \href{../works/WeilHFP95.pdf}{WeilHFP95}~\cite{WeilHFP95}, \href{../works/NuijtenA94.pdf}{NuijtenA94}~\cite{NuijtenA94}, \href{../works/SmithC93.pdf}{SmithC93}~\cite{SmithC93} (Total: 48) & \href{../works/EfthymiouY23.pdf}{EfthymiouY23}~\cite{EfthymiouY23}, \href{../works/Bit-Monnot23.pdf}{Bit-Monnot23}~\cite{Bit-Monnot23}, \href{../works/LuoB22.pdf}{LuoB22}~\cite{LuoB22}, \href{../works/KoehlerBFFHPSSS21.pdf}{KoehlerBFFHPSSS21}~\cite{KoehlerBFFHPSSS21}, \href{../works/Astrand21.pdf}{Astrand21}~\cite{Astrand21}, \href{../works/Zahout21.pdf}{Zahout21}~\cite{Zahout21}, \href{../works/AntuoriHHEN21.pdf}{AntuoriHHEN21}~\cite{AntuoriHHEN21}, \href{../works/PandeyS21a.pdf}{PandeyS21a}~\cite{PandeyS21a}, \href{../works/Mercier-AubinGQ20.pdf}{Mercier-AubinGQ20}~\cite{Mercier-AubinGQ20}, \href{../works/ZouZ20.pdf}{ZouZ20}~\cite{ZouZ20}, \href{../works/AstrandJZ20.pdf}{AstrandJZ20}~\cite{AstrandJZ20}, \href{../works/LiuLH19a.pdf}{LiuLH19a}~\cite{LiuLH19a}, \href{../works/abs-1902-01193.pdf}{abs-1902-01193}~\cite{abs-1902-01193}, \href{../works/PachecoPR19.pdf}{PachecoPR19}~\cite{PachecoPR19}, \href{../works/abs-1901-07914.pdf}{abs-1901-07914}~\cite{abs-1901-07914}, \href{../works/BehrensLM19.pdf}{BehrensLM19}~\cite{BehrensLM19}, \href{../works/FrohnerTR19.pdf}{FrohnerTR19}~\cite{FrohnerTR19}, \href{../works/LiuLH18.pdf}{LiuLH18}~\cite{LiuLH18}, \href{../works/CauwelaertLS18.pdf}{CauwelaertLS18}~\cite{CauwelaertLS18}...\href{../works/BeckF00.pdf}{BeckF00}~\cite{BeckF00}, \href{../works/JoLLH99.pdf}{JoLLH99}~\cite{JoLLH99}, \href{../works/BeckF99.pdf}{BeckF99}~\cite{BeckF99}, \href{../works/GruianK98.pdf}{GruianK98}~\cite{GruianK98}, \href{../works/BelhadjiI98.pdf}{BelhadjiI98}~\cite{BelhadjiI98}, \href{../works/CestaOS98.pdf}{CestaOS98}~\cite{CestaOS98}, \href{../works/BlazewiczDP96.pdf}{BlazewiczDP96}~\cite{BlazewiczDP96}, \href{../works/BrusoniCLMMT96.pdf}{BrusoniCLMMT96}~\cite{BrusoniCLMMT96}, \href{../works/Puget95.pdf}{Puget95}~\cite{Puget95}, \href{../works/YoshikawaKNW94.pdf}{YoshikawaKNW94}~\cite{YoshikawaKNW94} (Total: 90)\\
\index{constraint logic programming}\index{CP!constraint logic programming}constraint logic programming &  1.00 & \href{../works/BadicaBI20.pdf}{BadicaBI20}~\cite{BadicaBI20}, \href{../works/KonowalenkoMM19.pdf}{KonowalenkoMM19}~\cite{KonowalenkoMM19}, \href{../works/Simonis07.pdf}{Simonis07}~\cite{Simonis07}, \href{../works/RoePS05.pdf}{RoePS05}~\cite{RoePS05}, \href{../works/MartinPY01.pdf}{MartinPY01}~\cite{MartinPY01}, \href{../works/TrentesauxPT01.pdf}{TrentesauxPT01}~\cite{TrentesauxPT01}, \href{../works/BosiM2001.pdf}{BosiM2001}~\cite{BosiM2001}, \href{../works/RodosekWH99.pdf}{RodosekWH99}~\cite{RodosekWH99}, \href{../works/Simonis99.pdf}{Simonis99}~\cite{Simonis99}, \href{../works/GruianK98.pdf}{GruianK98}~\cite{GruianK98}, \href{../works/Darby-DowmanLMZ97.pdf}{Darby-DowmanLMZ97}~\cite{Darby-DowmanLMZ97}, \href{../works/LammaMM97.pdf}{LammaMM97}~\cite{LammaMM97}, \href{../works/BrusoniCLMMT96.pdf}{BrusoniCLMMT96}~\cite{BrusoniCLMMT96}, \href{../works/Wallace96.pdf}{Wallace96}~\cite{Wallace96}, \href{../works/SimonisC95.pdf}{SimonisC95}~\cite{SimonisC95}, \href{../works/Simonis95a.pdf}{Simonis95a}~\cite{Simonis95a}, \href{../works/Goltz95.pdf}{Goltz95}~\cite{Goltz95}, \href{../works/BeldiceanuC94.pdf}{BeldiceanuC94}~\cite{BeldiceanuC94}, \href{../works/AggounB93.pdf}{AggounB93}~\cite{AggounB93} & \href{../works/ZarandiASC20.pdf}{ZarandiASC20}~\cite{ZarandiASC20}, \href{../works/WikarekS19.pdf}{WikarekS19}~\cite{WikarekS19}, \href{../works/HookerH17.pdf}{HookerH17}~\cite{HookerH17}, \href{../works/JelinekB16.pdf}{JelinekB16}~\cite{JelinekB16}, \href{../works/MilanoW09.pdf}{MilanoW09}~\cite{MilanoW09}, \href{../works/LiW08.pdf}{LiW08}~\cite{LiW08}, \href{../works/MilanoW06.pdf}{MilanoW06}~\cite{MilanoW06}, \href{../works/Wallace06.pdf}{Wallace06}~\cite{Wallace06}, \href{../works/Kuchcinski03.pdf}{Kuchcinski03}~\cite{Kuchcinski03}, \href{../works/Demassey03.pdf}{Demassey03}~\cite{Demassey03}, \href{../works/Elkhyari03.pdf}{Elkhyari03}~\cite{Elkhyari03}, \href{../works/Bartak02.pdf}{Bartak02}~\cite{Bartak02}, \href{../works/Timpe02.pdf}{Timpe02}~\cite{Timpe02}, \href{../works/Thorsteinsson01.pdf}{Thorsteinsson01}~\cite{Thorsteinsson01}, \href{../works/JainG01.pdf}{JainG01}~\cite{JainG01}, \href{../works/HarjunkoskiJG00.pdf}{HarjunkoskiJG00}~\cite{HarjunkoskiJG00}, \href{../works/HookerO99.pdf}{HookerO99}~\cite{HookerO99}, \href{../works/RodosekW98.pdf}{RodosekW98}~\cite{RodosekW98}, \href{../works/FalaschiGMP97.pdf}{FalaschiGMP97}~\cite{FalaschiGMP97}, \href{../works/Simonis95.pdf}{Simonis95}~\cite{Simonis95}, \href{../works/DincbasS91.pdf}{DincbasS91}~\cite{DincbasS91}, \href{../works/ErtlK91.pdf}{ErtlK91}~\cite{ErtlK91} & \href{../works/ForbesHJST24.pdf}{ForbesHJST24}~\cite{ForbesHJST24}, \href{../works/NaderiBZR23.pdf}{NaderiBZR23}~\cite{NaderiBZR23}, \href{../works/IsikYA23.pdf}{IsikYA23}~\cite{IsikYA23}, \href{../works/EmdeZD22.pdf}{EmdeZD22}~\cite{EmdeZD22}, \href{../works/ColT22.pdf}{ColT22}~\cite{ColT22}, \href{../works/LuoB22.pdf}{LuoB22}~\cite{LuoB22}, \href{../works/ElciOH22.pdf}{ElciOH22}~\cite{ElciOH22}, \href{../works/PandeyS21a.pdf}{PandeyS21a}~\cite{PandeyS21a}, \href{../works/RoshanaeiN21.pdf}{RoshanaeiN21}~\cite{RoshanaeiN21}, \href{../works/GeibingerMM21.pdf}{GeibingerMM21}~\cite{GeibingerMM21}, \href{../works/ArmstrongGOS21.pdf}{ArmstrongGOS21}~\cite{ArmstrongGOS21}, \href{../works/KlankeBYE21.pdf}{KlankeBYE21}~\cite{KlankeBYE21}, \href{../works/Lemos21.pdf}{Lemos21}~\cite{Lemos21}, \href{../works/CauwelaertDS20.pdf}{CauwelaertDS20}~\cite{CauwelaertDS20}, \href{../works/FallahiAC20.pdf}{FallahiAC20}~\cite{FallahiAC20}, \href{../works/RoshanaeiBAUB20.pdf}{RoshanaeiBAUB20}~\cite{RoshanaeiBAUB20}, \href{../works/FachiniA20.pdf}{FachiniA20}~\cite{FachiniA20}, \href{../works/BadicaBIL19.pdf}{BadicaBIL19}~\cite{BadicaBIL19}, \href{../works/HoundjiSW19.pdf}{HoundjiSW19}~\cite{HoundjiSW19}...\href{../works/PesantGPR99.pdf}{PesantGPR99}~\cite{PesantGPR99}, \href{../works/AbdennadherS99.pdf}{AbdennadherS99}~\cite{AbdennadherS99}, \href{../works/JoLLH99.pdf}{JoLLH99}~\cite{JoLLH99}, \href{../works/Schaerf97.pdf}{Schaerf97}~\cite{Schaerf97}, \href{../works/Zhou97.pdf}{Zhou97}~\cite{Zhou97}, \href{../works/GetoorOFC97.pdf}{GetoorOFC97}~\cite{GetoorOFC97}, \href{../works/Zhou96.pdf}{Zhou96}~\cite{Zhou96}, \href{../works/SmithBHW96.pdf}{SmithBHW96}~\cite{SmithBHW96}, \href{../works/Touraivane95.pdf}{Touraivane95}~\cite{Touraivane95}, \href{../works/Pape94.pdf}{Pape94}~\cite{Pape94} (Total: 102)\\
\index{constraint optimization}\index{CP!constraint optimization}constraint optimization &  1.00 & \href{../works/KovacsTKSG21.pdf}{KovacsTKSG21}~\cite{KovacsTKSG21}, \href{../works/Dejemeppe16.pdf}{Dejemeppe16}~\cite{Dejemeppe16}, \href{../works/SultanikMR07.pdf}{SultanikMR07}~\cite{SultanikMR07} & \href{../works/abs-2211-14492.pdf}{abs-2211-14492}~\cite{abs-2211-14492}, \href{../works/Lemos21.pdf}{Lemos21}~\cite{Lemos21}, \href{../works/Godet21a.pdf}{Godet21a}~\cite{Godet21a}, \href{../works/abs-1901-07914.pdf}{abs-1901-07914}~\cite{abs-1901-07914}, \href{../works/LiuLH19.pdf}{LiuLH19}~\cite{LiuLH19}, \href{../works/PourDERB18.pdf}{PourDERB18}~\cite{PourDERB18}, \href{../works/MossigeGSMC17.pdf}{MossigeGSMC17}~\cite{MossigeGSMC17}, \href{../works/AlesioBNG15.pdf}{AlesioBNG15}~\cite{AlesioBNG15}, \href{../works/AlesioNBG14.pdf}{AlesioNBG14}~\cite{AlesioNBG14}, \href{../works/HeinzSB13.pdf}{HeinzSB13}~\cite{HeinzSB13}, \href{../works/MenciaSV13.pdf}{MenciaSV13}~\cite{MenciaSV13}, \href{../works/ChenGPSH10.pdf}{ChenGPSH10}~\cite{ChenGPSH10}, \href{../works/ElhouraniDM07.pdf}{ElhouraniDM07}~\cite{ElhouraniDM07}, \href{../works/Dorndorf2000.pdf}{Dorndorf2000}~\cite{Dorndorf2000} & \href{../works/FalqueALM24.pdf}{FalqueALM24}~\cite{FalqueALM24}, \href{../works/TardivoDFMP23.pdf}{TardivoDFMP23}~\cite{TardivoDFMP23}, \href{../works/LacknerMMWW23.pdf}{LacknerMMWW23}~\cite{LacknerMMWW23}, \href{../works/TasselGS23.pdf}{TasselGS23}~\cite{TasselGS23}, \href{../works/abs-2306-05747.pdf}{abs-2306-05747}~\cite{abs-2306-05747}, \href{../works/GuoZ23.pdf}{GuoZ23}~\cite{GuoZ23}, \href{../works/FetgoD22.pdf}{FetgoD22}~\cite{FetgoD22}, \href{../works/Tassel22.pdf}{Tassel22}~\cite{Tassel22}, \href{../works/Edis21.pdf}{Edis21}~\cite{Edis21}, \href{../works/KoehlerBFFHPSSS21.pdf}{KoehlerBFFHPSSS21}~\cite{KoehlerBFFHPSSS21}, \href{../works/FallahiAC20.pdf}{FallahiAC20}~\cite{FallahiAC20}, \href{../works/FrohnerTR19.pdf}{FrohnerTR19}~\cite{FrohnerTR19}, \href{../works/abs-1902-01193.pdf}{abs-1902-01193}~\cite{abs-1902-01193}, \href{../works/Hooker19.pdf}{Hooker19}~\cite{Hooker19}, \href{../works/PinarbasiAY19.pdf}{PinarbasiAY19}~\cite{PinarbasiAY19}, \href{../works/BehrensLM19.pdf}{BehrensLM19}~\cite{BehrensLM19}, \href{../works/LiuLH18.pdf}{LiuLH18}~\cite{LiuLH18}, \href{../works/GoldwaserS18.pdf}{GoldwaserS18}~\cite{GoldwaserS18}, \href{../works/HookerH17.pdf}{HookerH17}~\cite{HookerH17}...\href{../works/DilkinaDH05.pdf}{DilkinaDH05}~\cite{DilkinaDH05}, \href{../works/DilkinaH04.pdf}{DilkinaH04}~\cite{DilkinaH04}, \href{../works/Kuchcinski03.pdf}{Kuchcinski03}~\cite{Kuchcinski03}, \href{../works/JainM99.pdf}{JainM99}~\cite{JainM99}, \href{../works/Beck99.pdf}{Beck99}~\cite{Beck99}, \href{../works/BensanaLV99.pdf}{BensanaLV99}~\cite{BensanaLV99}, \href{../works/BeckDDF98.pdf}{BeckDDF98}~\cite{BeckDDF98}, \href{../works/BeckDSF97.pdf}{BeckDSF97}~\cite{BeckDSF97}, \href{../works/BeckDSF97a.pdf}{BeckDSF97a}~\cite{BeckDSF97a}, \href{../works/GetoorOFC97.pdf}{GetoorOFC97}~\cite{GetoorOFC97} (Total: 55)\\
\index{constraint programming}\index{CP!constraint programming}constraint programming &  1.00 & \href{../works/BonninMNE24.pdf}{BonninMNE24}~\cite{BonninMNE24}, \href{../works/abs-2402-00459.pdf}{abs-2402-00459}~\cite{abs-2402-00459}, \href{../works/LuZZYW24.pdf}{LuZZYW24}~\cite{LuZZYW24}, \href{../works/PrataAN23.pdf}{PrataAN23}~\cite{PrataAN23}, \href{../works/ForbesHJST24.pdf}{ForbesHJST24}~\cite{ForbesHJST24}, \href{../works/IsikYA23.pdf}{IsikYA23}~\cite{IsikYA23}, \href{../works/AfsarVPG23.pdf}{AfsarVPG23}~\cite{AfsarVPG23}, \href{../works/AalianPG23.pdf}{AalianPG23}~\cite{AalianPG23}, \href{../works/abs-2312-13682.pdf}{abs-2312-13682}~\cite{abs-2312-13682}, \href{../works/AbreuPNF23.pdf}{AbreuPNF23}~\cite{AbreuPNF23}, \href{../works/CzerniachowskaWZ23.pdf}{CzerniachowskaWZ23}~\cite{CzerniachowskaWZ23}, \href{../works/MontemanniD23.pdf}{MontemanniD23}~\cite{MontemanniD23}, \href{../works/AkramNHRSA23.pdf}{AkramNHRSA23}~\cite{AkramNHRSA23}, \href{../works/abs-2306-05747.pdf}{abs-2306-05747}~\cite{abs-2306-05747}, \href{../works/Fatemi-AnarakiTFV23.pdf}{Fatemi-AnarakiTFV23}~\cite{Fatemi-AnarakiTFV23}, \href{../works/NaderiRR23.pdf}{NaderiRR23}~\cite{NaderiRR23}, \href{../works/ZhuSZW23.pdf}{ZhuSZW23}~\cite{ZhuSZW23}, \href{../works/Adelgren2023.pdf}{Adelgren2023}~\cite{Adelgren2023}, \href{../works/TasselGS23.pdf}{TasselGS23}~\cite{TasselGS23}...\href{../works/NuijtenP98.pdf}{NuijtenP98}~\cite{NuijtenP98}, \href{../works/BeckDDF98.pdf}{BeckDDF98}~\cite{BeckDDF98}, \href{../works/Zhou97.pdf}{Zhou97}~\cite{Zhou97}, \href{../works/PapeB97.pdf}{PapeB97}~\cite{PapeB97}, \href{../works/BaptisteP97.pdf}{BaptisteP97}~\cite{BaptisteP97}, \href{../works/Colombani96.pdf}{Colombani96}~\cite{Colombani96}, \href{../works/SmithBHW96.pdf}{SmithBHW96}~\cite{SmithBHW96}, \href{../works/Wallace96.pdf}{Wallace96}~\cite{Wallace96}, \href{../works/Puget95.pdf}{Puget95}~\cite{Puget95}, \href{../works/WeilHFP95.pdf}{WeilHFP95}~\cite{WeilHFP95} (Total: 357) & \href{../works/FalqueALM24.pdf}{FalqueALM24}~\cite{FalqueALM24}, \href{../works/KameugneFND23.pdf}{KameugneFND23}~\cite{KameugneFND23}, \href{../works/AlfieriGPS23.pdf}{AlfieriGPS23}~\cite{AlfieriGPS23}, \href{../works/SquillaciPR23.pdf}{SquillaciPR23}~\cite{SquillaciPR23}, \href{../works/NaderiBZR23.pdf}{NaderiBZR23}~\cite{NaderiBZR23}, \href{../works/YuraszeckMC23.pdf}{YuraszeckMC23}~\cite{YuraszeckMC23}, \href{../works/ShaikhK23.pdf}{ShaikhK23}~\cite{ShaikhK23}, \href{../works/EfthymiouY23.pdf}{EfthymiouY23}~\cite{EfthymiouY23}, \href{../works/GokPTGO23.pdf}{GokPTGO23}~\cite{GokPTGO23}, \href{../works/WangB23.pdf}{WangB23}~\cite{WangB23}, \href{../works/JungblutK22.pdf}{JungblutK22}~\cite{JungblutK22}, \href{../works/NaqviAIAAA22.pdf}{NaqviAIAAA22}~\cite{NaqviAIAAA22}, \href{../works/LuoB22.pdf}{LuoB22}~\cite{LuoB22}, \href{../works/OuelletQ22.pdf}{OuelletQ22}~\cite{OuelletQ22}, \href{../works/VlkHT21.pdf}{VlkHT21}~\cite{VlkHT21}, \href{../works/KoehlerBFFHPSSS21.pdf}{KoehlerBFFHPSSS21}~\cite{KoehlerBFFHPSSS21}, \href{../works/Zahout21.pdf}{Zahout21}~\cite{Zahout21}, \href{../works/HanenKP21.pdf}{HanenKP21}~\cite{HanenKP21}, \href{../works/LacknerMMWW21.pdf}{LacknerMMWW21}~\cite{LacknerMMWW21}...\href{../works/MartinPY01.pdf}{MartinPY01}~\cite{MartinPY01}, \href{../works/TrentesauxPT01.pdf}{TrentesauxPT01}~\cite{TrentesauxPT01}, \href{../works/SimonisCK00.pdf}{SimonisCK00}~\cite{SimonisCK00}, \href{../works/Dorndorf2000.pdf}{Dorndorf2000}~\cite{Dorndorf2000}, \href{../works/FocacciLN00.pdf}{FocacciLN00}~\cite{FocacciLN00}, \href{../works/ChunCTY99.pdf}{ChunCTY99}~\cite{ChunCTY99}, \href{../works/PapaB98.pdf}{PapaB98}~\cite{PapaB98}, \href{../works/FalaschiGMP97.pdf}{FalaschiGMP97}~\cite{FalaschiGMP97}, \href{../works/Caseau97.pdf}{Caseau97}~\cite{Caseau97}, \href{../works/Zhou96.pdf}{Zhou96}~\cite{Zhou96} (Total: 185) & \href{../works/LiLZDZW24.pdf}{LiLZDZW24}~\cite{LiLZDZW24}, \href{../works/BofillCGGPSV23.pdf}{BofillCGGPSV23}~\cite{BofillCGGPSV23}, \href{../works/GuoZ23.pdf}{GuoZ23}~\cite{GuoZ23}, \href{../works/TardivoDFMP23.pdf}{TardivoDFMP23}~\cite{TardivoDFMP23}, \href{../works/IklassovMR023.pdf}{IklassovMR023}~\cite{IklassovMR023}, \href{../works/SvancaraB22.pdf}{SvancaraB22}~\cite{SvancaraB22}, \href{../works/KotaryFH22.pdf}{KotaryFH22}~\cite{KotaryFH22}, \href{../works/ElciOH22.pdf}{ElciOH22}~\cite{ElciOH22}, \href{../works/abs-2102-08778.pdf}{abs-2102-08778}~\cite{abs-2102-08778}, \href{../works/BadicaBI20.pdf}{BadicaBI20}~\cite{BadicaBI20}, \href{../works/GokGSTO20.pdf}{GokGSTO20}~\cite{GokGSTO20}, \href{../works/CarlierPSJ20.pdf}{CarlierPSJ20}~\cite{CarlierPSJ20}, \href{../works/ColT2019a.pdf}{ColT2019a}~\cite{ColT2019a}, \href{../works/KonowalenkoMM19.pdf}{KonowalenkoMM19}~\cite{KonowalenkoMM19}, \href{../works/PachecoPR19.pdf}{PachecoPR19}~\cite{PachecoPR19}, \href{../works/FrimodigS19.pdf}{FrimodigS19}~\cite{FrimodigS19}, \href{../works/BadicaBIL19.pdf}{BadicaBIL19}~\cite{BadicaBIL19}, \href{../works/HoYCLLCLC18.pdf}{HoYCLLCLC18}~\cite{HoYCLLCLC18}, \href{../works/GombolayWS18.pdf}{GombolayWS18}~\cite{GombolayWS18}...\href{../works/AbdennadherS99.pdf}{AbdennadherS99}~\cite{AbdennadherS99}, \href{../works/RodosekW98.pdf}{RodosekW98}~\cite{RodosekW98}, \href{../works/Darby-DowmanLMZ97.pdf}{Darby-DowmanLMZ97}~\cite{Darby-DowmanLMZ97}, \href{../works/GetoorOFC97.pdf}{GetoorOFC97}~\cite{GetoorOFC97}, \href{../works/BlazewiczDP96.pdf}{BlazewiczDP96}~\cite{BlazewiczDP96}, \href{../works/NuijtenA96.pdf}{NuijtenA96}~\cite{NuijtenA96}, \href{../works/SimonisC95.pdf}{SimonisC95}~\cite{SimonisC95}, \href{../works/Simonis95a.pdf}{Simonis95a}~\cite{Simonis95a}, \href{../works/BaptisteP95.pdf}{BaptisteP95}~\cite{BaptisteP95}, \href{../works/DincbasS91.pdf}{DincbasS91}~\cite{DincbasS91} (Total: 122)\\
\index{constraint propagation}\index{CP!constraint propagation}constraint propagation &  1.00 & \href{../works/LuZZYW24.pdf}{LuZZYW24}~\cite{LuZZYW24}, \href{../works/MarliereSPR23.pdf}{MarliereSPR23}~\cite{MarliereSPR23}, \href{../works/TardivoDFMP23.pdf}{TardivoDFMP23}~\cite{TardivoDFMP23}, \href{../works/PohlAK22.pdf}{PohlAK22}~\cite{PohlAK22}, \href{../works/Godet21a.pdf}{Godet21a}~\cite{Godet21a}, \href{../works/Groleaz21.pdf}{Groleaz21}~\cite{Groleaz21}, \href{../works/SacramentoSP20.pdf}{SacramentoSP20}~\cite{SacramentoSP20}, \href{../works/GokgurHO18.pdf}{GokgurHO18}~\cite{GokgurHO18}, \href{../works/HookerH17.pdf}{HookerH17}~\cite{HookerH17}, \href{../works/Fahimi16.pdf}{Fahimi16}~\cite{Fahimi16}, \href{../works/OrnekO16.pdf}{OrnekO16}~\cite{OrnekO16}, \href{../works/Dejemeppe16.pdf}{Dejemeppe16}~\cite{Dejemeppe16}, \href{../works/Siala15a.pdf}{Siala15a}~\cite{Siala15a}, \href{../works/VilimLS15.pdf}{VilimLS15}~\cite{VilimLS15}, \href{../works/GrimesH15.pdf}{GrimesH15}~\cite{GrimesH15}, \href{../works/PengLC14.pdf}{PengLC14}~\cite{PengLC14}, \href{../works/LombardiMB13.pdf}{LombardiMB13}~\cite{LombardiMB13}, \href{../works/MenciaSV13.pdf}{MenciaSV13}~\cite{MenciaSV13}, \href{../works/MenciaSV12.pdf}{MenciaSV12}~\cite{MenciaSV12}...\href{../works/NuijtenP98.pdf}{NuijtenP98}~\cite{NuijtenP98}, \href{../works/RodosekW98.pdf}{RodosekW98}~\cite{RodosekW98}, \href{../works/BaptisteP97.pdf}{BaptisteP97}~\cite{BaptisteP97}, \href{../works/Caseau97.pdf}{Caseau97}~\cite{Caseau97}, \href{../works/Zhou97.pdf}{Zhou97}~\cite{Zhou97}, \href{../works/PapeB97.pdf}{PapeB97}~\cite{PapeB97}, \href{../works/BlazewiczDP96.pdf}{BlazewiczDP96}~\cite{BlazewiczDP96}, \href{../works/Wallace96.pdf}{Wallace96}~\cite{Wallace96}, \href{../works/BaptisteP95.pdf}{BaptisteP95}~\cite{BaptisteP95}, \href{../works/Pape94.pdf}{Pape94}~\cite{Pape94} (Total: 75) & \href{../works/ZhuSZW23.pdf}{ZhuSZW23}~\cite{ZhuSZW23}, \href{../works/Fatemi-AnarakiTFV23.pdf}{Fatemi-AnarakiTFV23}~\cite{Fatemi-AnarakiTFV23}, \href{../works/WangB23.pdf}{WangB23}~\cite{WangB23}, \href{../works/JuvinHHL23.pdf}{JuvinHHL23}~\cite{JuvinHHL23}, \href{../works/IsikYA23.pdf}{IsikYA23}~\cite{IsikYA23}, \href{../works/MullerMKP22.pdf}{MullerMKP22}~\cite{MullerMKP22}, \href{../works/SubulanC22.pdf}{SubulanC22}~\cite{SubulanC22}, \href{../works/ZhangBB22.pdf}{ZhangBB22}~\cite{ZhangBB22}, \href{../works/Edis21.pdf}{Edis21}~\cite{Edis21}, \href{../works/KoehlerBFFHPSSS21.pdf}{KoehlerBFFHPSSS21}~\cite{KoehlerBFFHPSSS21}, \href{../works/NishikawaSTT19.pdf}{NishikawaSTT19}~\cite{NishikawaSTT19}, \href{../works/WikarekS19.pdf}{WikarekS19}~\cite{WikarekS19}, \href{../works/TangLWSK18.pdf}{TangLWSK18}~\cite{TangLWSK18}, \href{../works/LaborieRSV18.pdf}{LaborieRSV18}~\cite{LaborieRSV18}, \href{../works/SchnellH17.pdf}{SchnellH17}~\cite{SchnellH17}, \href{../works/SchuttS16.pdf}{SchuttS16}~\cite{SchuttS16}, \href{../works/Froger16.pdf}{Froger16}~\cite{Froger16}, \href{../works/ZhouGL15.pdf}{ZhouGL15}~\cite{ZhouGL15}, \href{../works/MelgarejoLS15.pdf}{MelgarejoLS15}~\cite{MelgarejoLS15}...\href{../works/HookerO99.pdf}{HookerO99}~\cite{HookerO99}, \href{../works/Simonis99.pdf}{Simonis99}~\cite{Simonis99}, \href{../works/PesantGPR99.pdf}{PesantGPR99}~\cite{PesantGPR99}, \href{../works/KorbaaYG99.pdf}{KorbaaYG99}~\cite{KorbaaYG99}, \href{../works/PembertonG98.pdf}{PembertonG98}~\cite{PembertonG98}, \href{../works/BeckDDF98.pdf}{BeckDDF98}~\cite{BeckDDF98}, \href{../works/Darby-DowmanLMZ97.pdf}{Darby-DowmanLMZ97}~\cite{Darby-DowmanLMZ97}, \href{../works/SmithBHW96.pdf}{SmithBHW96}~\cite{SmithBHW96}, \href{../works/Zhou96.pdf}{Zhou96}~\cite{Zhou96}, \href{../works/DincbasSH90.pdf}{DincbasSH90}~\cite{DincbasSH90} (Total: 70) & \href{../works/AbreuPNF23.pdf}{AbreuPNF23}~\cite{AbreuPNF23}, \href{../works/NaderiRR23.pdf}{NaderiRR23}~\cite{NaderiRR23}, \href{../works/Bit-Monnot23.pdf}{Bit-Monnot23}~\cite{Bit-Monnot23}, \href{../works/AlakaP23.pdf}{AlakaP23}~\cite{AlakaP23}, \href{../works/WessenCSFPM23.pdf}{WessenCSFPM23}~\cite{WessenCSFPM23}, \href{../works/YuraszeckMPV22.pdf}{YuraszeckMPV22}~\cite{YuraszeckMPV22}, \href{../works/OujanaAYB22.pdf}{OujanaAYB22}~\cite{OujanaAYB22}, \href{../works/CilKLO22.pdf}{CilKLO22}~\cite{CilKLO22}, \href{../works/YunusogluY22.pdf}{YunusogluY22}~\cite{YunusogluY22}, \href{../works/BoudreaultSLQ22.pdf}{BoudreaultSLQ22}~\cite{BoudreaultSLQ22}, \href{../works/Teppan22.pdf}{Teppan22}~\cite{Teppan22}, \href{../works/MengGRZSC22.pdf}{MengGRZSC22}~\cite{MengGRZSC22}, \href{../works/ElciOH22.pdf}{ElciOH22}~\cite{ElciOH22}, \href{../works/ColT22.pdf}{ColT22}~\cite{ColT22}, \href{../works/GeitzGSSW22.pdf}{GeitzGSSW22}~\cite{GeitzGSSW22}, \href{../works/ArmstrongGOS21.pdf}{ArmstrongGOS21}~\cite{ArmstrongGOS21}, \href{../works/KlankeBYE21.pdf}{KlankeBYE21}~\cite{KlankeBYE21}, \href{../works/HamP21.pdf}{HamP21}~\cite{HamP21}, \href{../works/HamPK21.pdf}{HamPK21}~\cite{HamPK21}...\href{../works/BeldiceanuC94.pdf}{BeldiceanuC94}~\cite{BeldiceanuC94}, \href{../works/NuijtenA94.pdf}{NuijtenA94}~\cite{NuijtenA94}, \href{../works/Muscettola94.pdf}{Muscettola94}~\cite{Muscettola94}, \href{../works/AggounB93.pdf}{AggounB93}~\cite{AggounB93}, \href{../works/SmithC93.pdf}{SmithC93}~\cite{SmithC93}, \href{../works/Hamscher91.pdf}{Hamscher91}~\cite{Hamscher91}, \href{../works/ErtlK91.pdf}{ErtlK91}~\cite{ErtlK91}, \href{../works/FoxS90.pdf}{FoxS90}~\cite{FoxS90}, \href{../works/FeldmanG89.pdf}{FeldmanG89}~\cite{FeldmanG89}, \href{../works/KengY89.pdf}{KengY89}~\cite{KengY89} (Total: 217)\\
\index{constraint satisfaction}\index{CP!constraint satisfaction}constraint satisfaction &  1.00 & \href{../works/abs-2211-14492.pdf}{abs-2211-14492}~\cite{abs-2211-14492}, \href{../works/Lemos21.pdf}{Lemos21}~\cite{Lemos21}, \href{../works/Godet21a.pdf}{Godet21a}~\cite{Godet21a}, \href{../works/ZarandiASC20.pdf}{ZarandiASC20}~\cite{ZarandiASC20}, \href{../works/LiuLH19a.pdf}{LiuLH19a}~\cite{LiuLH19a}, \href{../works/Madi-WambaLOBM17.pdf}{Madi-WambaLOBM17}~\cite{Madi-WambaLOBM17}, \href{../works/HookerH17.pdf}{HookerH17}~\cite{HookerH17}, \href{../works/Fahimi16.pdf}{Fahimi16}~\cite{Fahimi16}, \href{../works/ZarandiKS16.pdf}{ZarandiKS16}~\cite{ZarandiKS16}, \href{../works/Siala15a.pdf}{Siala15a}~\cite{Siala15a}, \href{../works/Derrien15.pdf}{Derrien15}~\cite{Derrien15}, \href{../works/Kameugne14.pdf}{Kameugne14}~\cite{Kameugne14}, \href{../works/ZhangLS12.pdf}{ZhangLS12}~\cite{ZhangLS12}, \href{../works/Clercq12.pdf}{Clercq12}~\cite{Clercq12}, \href{../works/BartakS11.pdf}{BartakS11}~\cite{BartakS11}, \href{../works/TrojetHL11.pdf}{TrojetHL11}~\cite{TrojetHL11}, \href{../works/Schutt11.pdf}{Schutt11}~\cite{Schutt11}, \href{../works/HeckmanB11.pdf}{HeckmanB11}~\cite{HeckmanB11}, \href{../works/BartakCS10.pdf}{BartakCS10}~\cite{BartakCS10}...\href{../works/BeckDSF97a.pdf}{BeckDSF97a}~\cite{BeckDSF97a}, \href{../works/GetoorOFC97.pdf}{GetoorOFC97}~\cite{GetoorOFC97}, \href{../works/Wallace96.pdf}{Wallace96}~\cite{Wallace96}, \href{../works/NuijtenA96.pdf}{NuijtenA96}~\cite{NuijtenA96}, \href{../works/BlazewiczDP96.pdf}{BlazewiczDP96}~\cite{BlazewiczDP96}, \href{../works/SadehF96.pdf}{SadehF96}~\cite{SadehF96}, \href{../works/BaptisteP95.pdf}{BaptisteP95}~\cite{BaptisteP95}, \href{../works/Nuijten94.pdf}{Nuijten94}~\cite{Nuijten94}, \href{../works/NuijtenA94.pdf}{NuijtenA94}~\cite{NuijtenA94}, \href{../works/MintonJPL90.pdf}{MintonJPL90}~\cite{MintonJPL90} (Total: 59) & \href{../works/ZhuSZW23.pdf}{ZhuSZW23}~\cite{ZhuSZW23}, \href{../works/JuvinHHL23.pdf}{JuvinHHL23}~\cite{JuvinHHL23}, \href{../works/TardivoDFMP23.pdf}{TardivoDFMP23}~\cite{TardivoDFMP23}, \href{../works/Astrand21.pdf}{Astrand21}~\cite{Astrand21}, \href{../works/FallahiAC20.pdf}{FallahiAC20}~\cite{FallahiAC20}, \href{../works/Caballero19.pdf}{Caballero19}~\cite{Caballero19}, \href{../works/abs-1902-01193.pdf}{abs-1902-01193}~\cite{abs-1902-01193}, \href{../works/LiuLH19.pdf}{LiuLH19}~\cite{LiuLH19}, \href{../works/TangLWSK18.pdf}{TangLWSK18}~\cite{TangLWSK18}, \href{../works/GokgurHO18.pdf}{GokgurHO18}~\cite{GokgurHO18}, \href{../works/German18.pdf}{German18}~\cite{German18}, \href{../works/GelainPRVW17.pdf}{GelainPRVW17}~\cite{GelainPRVW17}, \href{../works/SchuttS16.pdf}{SchuttS16}~\cite{SchuttS16}, \href{../works/JelinekB16.pdf}{JelinekB16}~\cite{JelinekB16}, \href{../works/Dejemeppe16.pdf}{Dejemeppe16}~\cite{Dejemeppe16}, \href{../works/Froger16.pdf}{Froger16}~\cite{Froger16}, \href{../works/MurphyMB15.pdf}{MurphyMB15}~\cite{MurphyMB15}, \href{../works/GayHLS15.pdf}{GayHLS15}~\cite{GayHLS15}, \href{../works/GrimesH15.pdf}{GrimesH15}~\cite{GrimesH15}...\href{../works/BeckDF97.pdf}{BeckDF97}~\cite{BeckDF97}, \href{../works/MurphyRFSS97.pdf}{MurphyRFSS97}~\cite{MurphyRFSS97}, \href{../works/BeckDSF97.pdf}{BeckDSF97}~\cite{BeckDSF97}, \href{../works/Zhou97.pdf}{Zhou97}~\cite{Zhou97}, \href{../works/Schaerf97.pdf}{Schaerf97}~\cite{Schaerf97}, \href{../works/SmithBHW96.pdf}{SmithBHW96}~\cite{SmithBHW96}, \href{../works/BrusoniCLMMT96.pdf}{BrusoniCLMMT96}~\cite{BrusoniCLMMT96}, \href{../works/Simonis95a.pdf}{Simonis95a}~\cite{Simonis95a}, \href{../works/Muscettola94.pdf}{Muscettola94}~\cite{Muscettola94}, \href{../works/Pape94.pdf}{Pape94}~\cite{Pape94} (Total: 77) & \href{../works/ForbesHJST24.pdf}{ForbesHJST24}~\cite{ForbesHJST24}, \href{../works/FalqueALM24.pdf}{FalqueALM24}~\cite{FalqueALM24}, \href{../works/LuZZYW24.pdf}{LuZZYW24}~\cite{LuZZYW24}, \href{../works/GokPTGO23.pdf}{GokPTGO23}~\cite{GokPTGO23}, \href{../works/IsikYA23.pdf}{IsikYA23}~\cite{IsikYA23}, \href{../works/CzerniachowskaWZ23.pdf}{CzerniachowskaWZ23}~\cite{CzerniachowskaWZ23}, \href{../works/JuvinHL23a.pdf}{JuvinHL23a}~\cite{JuvinHL23a}, \href{../works/MarliereSPR23.pdf}{MarliereSPR23}~\cite{MarliereSPR23}, \href{../works/Bit-Monnot23.pdf}{Bit-Monnot23}~\cite{Bit-Monnot23}, \href{../works/FrimodigECM23.pdf}{FrimodigECM23}~\cite{FrimodigECM23}, \href{../works/ShaikhK23.pdf}{ShaikhK23}~\cite{ShaikhK23}, \href{../works/NaderiRR23.pdf}{NaderiRR23}~\cite{NaderiRR23}, \href{../works/GeitzGSSW22.pdf}{GeitzGSSW22}~\cite{GeitzGSSW22}, \href{../works/SvancaraB22.pdf}{SvancaraB22}~\cite{SvancaraB22}, \href{../works/ZhangBB22.pdf}{ZhangBB22}~\cite{ZhangBB22}, \href{../works/ColT22.pdf}{ColT22}~\cite{ColT22}, \href{../works/GhandehariK22.pdf}{GhandehariK22}~\cite{GhandehariK22}, \href{../works/LuoB22.pdf}{LuoB22}~\cite{LuoB22}, \href{../works/EmdeZD22.pdf}{EmdeZD22}~\cite{EmdeZD22}...\href{../works/Colombani96.pdf}{Colombani96}~\cite{Colombani96}, \href{../works/Zhou96.pdf}{Zhou96}~\cite{Zhou96}, \href{../works/SimonisC95.pdf}{SimonisC95}~\cite{SimonisC95}, \href{../works/WeilHFP95.pdf}{WeilHFP95}~\cite{WeilHFP95}, \href{../works/AggounB93.pdf}{AggounB93}~\cite{AggounB93}, \href{../works/MintonJPL92.pdf}{MintonJPL92}~\cite{MintonJPL92}, \href{../works/ErtlK91.pdf}{ErtlK91}~\cite{ErtlK91}, \href{../works/EskeyZ90.pdf}{EskeyZ90}~\cite{EskeyZ90}, \href{../works/FeldmanG89.pdf}{FeldmanG89}~\cite{FeldmanG89}, \href{../works/Valdes87.pdf}{Valdes87}~\cite{Valdes87} (Total: 215)\\
\index{propagation}\index{CP!propagation}propagation &  0.50 & \href{../works/LuZZYW24.pdf}{LuZZYW24}~\cite{LuZZYW24}, \href{../works/MarliereSPR23.pdf}{MarliereSPR23}~\cite{MarliereSPR23}, \href{../works/Bit-Monnot23.pdf}{Bit-Monnot23}~\cite{Bit-Monnot23}, \href{../works/TardivoDFMP23.pdf}{TardivoDFMP23}~\cite{TardivoDFMP23}, \href{../works/ZhuSZW23.pdf}{ZhuSZW23}~\cite{ZhuSZW23}, \href{../works/JuvinHHL23.pdf}{JuvinHHL23}~\cite{JuvinHHL23}, \href{../works/WangB23.pdf}{WangB23}~\cite{WangB23}, \href{../works/PohlAK22.pdf}{PohlAK22}~\cite{PohlAK22}, \href{../works/ZhangBB22.pdf}{ZhangBB22}~\cite{ZhangBB22}, \href{../works/CilKLO22.pdf}{CilKLO22}~\cite{CilKLO22}, \href{../works/FetgoD22.pdf}{FetgoD22}~\cite{FetgoD22}, \href{../works/HanenKP21.pdf}{HanenKP21}~\cite{HanenKP21}, \href{../works/Astrand21.pdf}{Astrand21}~\cite{Astrand21}, \href{../works/Lemos21.pdf}{Lemos21}~\cite{Lemos21}, \href{../works/Godet21a.pdf}{Godet21a}~\cite{Godet21a}, \href{../works/Groleaz21.pdf}{Groleaz21}~\cite{Groleaz21}, \href{../works/AntuoriHHEN20.pdf}{AntuoriHHEN20}~\cite{AntuoriHHEN20}, \href{../works/SacramentoSP20.pdf}{SacramentoSP20}~\cite{SacramentoSP20}, \href{../works/AstrandJZ20.pdf}{AstrandJZ20}~\cite{AstrandJZ20}...\href{../works/Zhou96.pdf}{Zhou96}~\cite{Zhou96}, \href{../works/Wallace96.pdf}{Wallace96}~\cite{Wallace96}, \href{../works/WeilHFP95.pdf}{WeilHFP95}~\cite{WeilHFP95}, \href{../works/BaptisteP95.pdf}{BaptisteP95}~\cite{BaptisteP95}, \href{../works/Pape94.pdf}{Pape94}~\cite{Pape94}, \href{../works/BeldiceanuC94.pdf}{BeldiceanuC94}~\cite{BeldiceanuC94}, \href{../works/CrawfordB94.pdf}{CrawfordB94}~\cite{CrawfordB94}, \href{../works/Hamscher91.pdf}{Hamscher91}~\cite{Hamscher91}, \href{../works/DincbasSH90.pdf}{DincbasSH90}~\cite{DincbasSH90}, \href{../works/Davis87.pdf}{Davis87}~\cite{Davis87} (Total: 217) & \href{../works/BonninMNE24.pdf}{BonninMNE24}~\cite{BonninMNE24}, \href{../works/NaderiRR23.pdf}{NaderiRR23}~\cite{NaderiRR23}, \href{../works/IsikYA23.pdf}{IsikYA23}~\cite{IsikYA23}, \href{../works/Fatemi-AnarakiTFV23.pdf}{Fatemi-AnarakiTFV23}~\cite{Fatemi-AnarakiTFV23}, \href{../works/NaderiBZR23.pdf}{NaderiBZR23}~\cite{NaderiBZR23}, \href{../works/KameugneFND23.pdf}{KameugneFND23}~\cite{KameugneFND23}, \href{../works/GokPTGO23.pdf}{GokPTGO23}~\cite{GokPTGO23}, \href{../works/BoudreaultSLQ22.pdf}{BoudreaultSLQ22}~\cite{BoudreaultSLQ22}, \href{../works/SubulanC22.pdf}{SubulanC22}~\cite{SubulanC22}, \href{../works/MengGRZSC22.pdf}{MengGRZSC22}~\cite{MengGRZSC22}, \href{../works/YunusogluY22.pdf}{YunusogluY22}~\cite{YunusogluY22}, \href{../works/OuelletQ22.pdf}{OuelletQ22}~\cite{OuelletQ22}, \href{../works/MullerMKP22.pdf}{MullerMKP22}~\cite{MullerMKP22}, \href{../works/GeitzGSSW22.pdf}{GeitzGSSW22}~\cite{GeitzGSSW22}, \href{../works/ColT22.pdf}{ColT22}~\cite{ColT22}, \href{../works/Edis21.pdf}{Edis21}~\cite{Edis21}, \href{../works/AntuoriHHEN21.pdf}{AntuoriHHEN21}~\cite{AntuoriHHEN21}, \href{../works/Bedhief21.pdf}{Bedhief21}~\cite{Bedhief21}, \href{../works/KoehlerBFFHPSSS21.pdf}{KoehlerBFFHPSSS21}~\cite{KoehlerBFFHPSSS21}...\href{../works/HookerO99.pdf}{HookerO99}~\cite{HookerO99}, \href{../works/BaptistePN99.pdf}{BaptistePN99}~\cite{BaptistePN99}, \href{../works/KorbaaYG99.pdf}{KorbaaYG99}~\cite{KorbaaYG99}, \href{../works/PembertonG98.pdf}{PembertonG98}~\cite{PembertonG98}, \href{../works/BeckDSF97.pdf}{BeckDSF97}~\cite{BeckDSF97}, \href{../works/GetoorOFC97.pdf}{GetoorOFC97}~\cite{GetoorOFC97}, \href{../works/SmithBHW96.pdf}{SmithBHW96}~\cite{SmithBHW96}, \href{../works/RoweJCA96.pdf}{RoweJCA96}~\cite{RoweJCA96}, \href{../works/BrusoniCLMMT96.pdf}{BrusoniCLMMT96}~\cite{BrusoniCLMMT96}, \href{../works/Simonis95.pdf}{Simonis95}~\cite{Simonis95} (Total: 121) & \href{../works/ForbesHJST24.pdf}{ForbesHJST24}~\cite{ForbesHJST24}, \href{../works/abs-2306-05747.pdf}{abs-2306-05747}~\cite{abs-2306-05747}, \href{../works/TasselGS23.pdf}{TasselGS23}~\cite{TasselGS23}, \href{../works/AbreuPNF23.pdf}{AbreuPNF23}~\cite{AbreuPNF23}, \href{../works/abs-2312-13682.pdf}{abs-2312-13682}~\cite{abs-2312-13682}, \href{../works/PovedaAA23.pdf}{PovedaAA23}~\cite{PovedaAA23}, \href{../works/AlfieriGPS23.pdf}{AlfieriGPS23}~\cite{AlfieriGPS23}, \href{../works/KimCMLLP23.pdf}{KimCMLLP23}~\cite{KimCMLLP23}, \href{../works/FrimodigECM23.pdf}{FrimodigECM23}~\cite{FrimodigECM23}, \href{../works/WessenCSFPM23.pdf}{WessenCSFPM23}~\cite{WessenCSFPM23}, \href{../works/GurPAE23.pdf}{GurPAE23}~\cite{GurPAE23}, \href{../works/Adelgren2023.pdf}{Adelgren2023}~\cite{Adelgren2023}, \href{../works/AfsarVPG23.pdf}{AfsarVPG23}~\cite{AfsarVPG23}, \href{../works/AlakaP23.pdf}{AlakaP23}~\cite{AlakaP23}, \href{../works/AkramNHRSA23.pdf}{AkramNHRSA23}~\cite{AkramNHRSA23}, \href{../works/PerezGSL23.pdf}{PerezGSL23}~\cite{PerezGSL23}, \href{../works/OujanaAYB22.pdf}{OujanaAYB22}~\cite{OujanaAYB22}, \href{../works/Teppan22.pdf}{Teppan22}~\cite{Teppan22}, \href{../works/ElciOH22.pdf}{ElciOH22}~\cite{ElciOH22}...\href{../works/Puget95.pdf}{Puget95}~\cite{Puget95}, \href{../works/SimonisC95.pdf}{SimonisC95}~\cite{SimonisC95}, \href{../works/Simonis95a.pdf}{Simonis95a}~\cite{Simonis95a}, \href{../works/NuijtenA94.pdf}{NuijtenA94}~\cite{NuijtenA94}, \href{../works/Muscettola94.pdf}{Muscettola94}~\cite{Muscettola94}, \href{../works/SmithC93.pdf}{SmithC93}~\cite{SmithC93}, \href{../works/AggounB93.pdf}{AggounB93}~\cite{AggounB93}, \href{../works/ErtlK91.pdf}{ErtlK91}~\cite{ErtlK91}, \href{../works/KengY89.pdf}{KengY89}~\cite{KengY89}, \href{../works/FeldmanG89.pdf}{FeldmanG89}~\cite{FeldmanG89} (Total: 197)\\
\end{longtable}
}

\clearpage
\subsection{Concept Type Concepts}
\label{sec:Concepts}
\label{Concepts}
{\scriptsize
\begin{longtable}{p{3cm}r>{\raggedright\arraybackslash}p{6cm}>{\raggedright\arraybackslash}p{6cm}>{\raggedright\arraybackslash}p{8cm}}
\rowcolor{white}\caption{Works for Concepts of Type Concepts (Total 64 Concepts, 64 Used)}\\ \toprule
\rowcolor{white}Keyword & Weight & High & Medium & Low\\ \midrule\endhead
\bottomrule
\endfoot
\index{Allen's algebra}\index{Concepts!Allen's algebra}Allen's algebra &  1.00 &  &  & \\
\index{BOM}\index{Concepts!BOM}BOM &  1.00 & \href{../works/SubulanC22.pdf}{SubulanC22}~\cite{SubulanC22}, \href{../works/OrnekO16.pdf}{OrnekO16}~\cite{OrnekO16} &  & \href{../works/UnsalO19.pdf}{UnsalO19}~\cite{UnsalO19}\\
\index{Benders Decomposition}\index{Concepts!Benders Decomposition}Benders Decomposition &  1.00 & \href{../works/ForbesHJST24.pdf}{ForbesHJST24}~\cite{ForbesHJST24}, \href{../works/GuoZ23.pdf}{GuoZ23}~\cite{GuoZ23}, \href{../works/ZhuSZW23.pdf}{ZhuSZW23}~\cite{ZhuSZW23}, \href{../works/NaderiBZR23.pdf}{NaderiBZR23}~\cite{NaderiBZR23}, \href{../works/JuvinHL23a.pdf}{JuvinHL23a}~\cite{JuvinHL23a}, \href{../works/NaderiBZ23.pdf}{NaderiBZ23}~\cite{NaderiBZ23}, \href{../works/ElciOH22.pdf}{ElciOH22}~\cite{ElciOH22}, \href{../works/NaderiBZ22a.pdf}{NaderiBZ22a}~\cite{NaderiBZ22a}, \href{../works/JuvinHL22.pdf}{JuvinHL22}~\cite{JuvinHL22}, \href{../works/EmdeZD22.pdf}{EmdeZD22}~\cite{EmdeZD22}, \href{../works/NaderiBZ22.pdf}{NaderiBZ22}~\cite{NaderiBZ22}, \href{../works/RoshanaeiN21.pdf}{RoshanaeiN21}~\cite{RoshanaeiN21}, \href{../works/VlkHT21.pdf}{VlkHT21}~\cite{VlkHT21}, \href{../works/FachiniA20.pdf}{FachiniA20}~\cite{FachiniA20}, \href{../works/RoshanaeiBAUB20.pdf}{RoshanaeiBAUB20}~\cite{RoshanaeiBAUB20}, \href{../works/SunTB19.pdf}{SunTB19}~\cite{SunTB19}, \href{../works/UnsalO19.pdf}{UnsalO19}~\cite{UnsalO19}, \href{../works/Hooker19.pdf}{Hooker19}~\cite{Hooker19}, \href{../works/GombolayWS18.pdf}{GombolayWS18}~\cite{GombolayWS18}...\href{../works/Hooker05b.pdf}{Hooker05b}~\cite{Hooker05b}, \href{../works/CambazardJ05.pdf}{CambazardJ05}~\cite{CambazardJ05}, \href{../works/Hooker05.pdf}{Hooker05}~\cite{Hooker05}, \href{../works/ChuX05.pdf}{ChuX05}~\cite{ChuX05}, \href{../works/CambazardHDJT04.pdf}{CambazardHDJT04}~\cite{CambazardHDJT04}, \href{../works/Hooker04.pdf}{Hooker04}~\cite{Hooker04}, \href{../works/HookerO03.pdf}{HookerO03}~\cite{HookerO03}, \href{../works/BenoistGR02.pdf}{BenoistGR02}~\cite{BenoistGR02}, \href{../works/Thorsteinsson01.pdf}{Thorsteinsson01}~\cite{Thorsteinsson01}, \href{../works/EreminW01.pdf}{EreminW01}~\cite{EreminW01} (Total: 67) & \href{../works/LuZZYW24.pdf}{LuZZYW24}~\cite{LuZZYW24}, \href{../works/NaderiRR23.pdf}{NaderiRR23}~\cite{NaderiRR23}, \href{../works/MengGRZSC22.pdf}{MengGRZSC22}~\cite{MengGRZSC22}, \href{../works/BulckG22.pdf}{BulckG22}~\cite{BulckG22}, \href{../works/TangB20.pdf}{TangB20}~\cite{TangB20}, \href{../works/TanZWGQ19.pdf}{TanZWGQ19}~\cite{TanZWGQ19}, \href{../works/Laborie18a.pdf}{Laborie18a}~\cite{Laborie18a}, \href{../works/TranVNB17.pdf}{TranVNB17}~\cite{TranVNB17}, \href{../works/RoshanaeiLAU17.pdf}{RoshanaeiLAU17}~\cite{RoshanaeiLAU17}, \href{../works/GoldwaserS17.pdf}{GoldwaserS17}~\cite{GoldwaserS17}, \href{../works/RiiseML16.pdf}{RiiseML16}~\cite{RiiseML16}, \href{../works/HarjunkoskiMBC14.pdf}{HarjunkoskiMBC14}~\cite{HarjunkoskiMBC14}, \href{../works/GuyonLPR12.pdf}{GuyonLPR12}~\cite{GuyonLPR12}, \href{../works/LombardiMRB10.pdf}{LombardiMRB10}~\cite{LombardiMRB10}, \href{../works/BeniniLMR08.pdf}{BeniniLMR08}~\cite{BeniniLMR08}, \href{../works/RasmussenT07.pdf}{RasmussenT07}~\cite{RasmussenT07}, \href{../works/Hooker05a.pdf}{Hooker05a}~\cite{Hooker05a}, \href{../works/HookerY02.pdf}{HookerY02}~\cite{HookerY02} & \href{../works/PrataAN23.pdf}{PrataAN23}~\cite{PrataAN23}, \href{../works/PovedaAA23.pdf}{PovedaAA23}~\cite{PovedaAA23}, \href{../works/AlfieriGPS23.pdf}{AlfieriGPS23}~\cite{AlfieriGPS23}, \href{../works/JuvinHHL23.pdf}{JuvinHHL23}~\cite{JuvinHHL23}, \href{../works/LuoB22.pdf}{LuoB22}~\cite{LuoB22}, \href{../works/FarsiTM22.pdf}{FarsiTM22}~\cite{FarsiTM22}, \href{../works/Godet21a.pdf}{Godet21a}~\cite{Godet21a}, \href{../works/QinDCS20.pdf}{QinDCS20}~\cite{QinDCS20}, \href{../works/WallaceY20.pdf}{WallaceY20}~\cite{WallaceY20}, \href{../works/MengZRZL20.pdf}{MengZRZL20}~\cite{MengZRZL20}, \href{../works/AntunesABD20.pdf}{AntunesABD20}~\cite{AntunesABD20}, \href{../works/Mercier-AubinGQ20.pdf}{Mercier-AubinGQ20}~\cite{Mercier-AubinGQ20}, \href{../works/Polo-MejiaALB20.pdf}{Polo-MejiaALB20}~\cite{Polo-MejiaALB20}, \href{../works/MurinR19.pdf}{MurinR19}~\cite{MurinR19}, \href{../works/PachecoPR19.pdf}{PachecoPR19}~\cite{PachecoPR19}, \href{../works/FrimodigS19.pdf}{FrimodigS19}~\cite{FrimodigS19}, \href{../works/LaborieRSV18.pdf}{LaborieRSV18}~\cite{LaborieRSV18}, \href{../works/AntunesABD18.pdf}{AntunesABD18}~\cite{AntunesABD18}, \href{../works/CappartTSR18.pdf}{CappartTSR18}~\cite{CappartTSR18}...\href{../works/EdisO11.pdf}{EdisO11}~\cite{EdisO11}, \href{../works/ChenGPSH10.pdf}{ChenGPSH10}~\cite{ChenGPSH10}, \href{../works/KendallKRU10.pdf}{KendallKRU10}~\cite{KendallKRU10}, \href{../works/LombardiM10a.pdf}{LombardiM10a}~\cite{LombardiM10a}, \href{../works/RodriguezS09.pdf}{RodriguezS09}~\cite{RodriguezS09}, \href{../works/RasmussenT09.pdf}{RasmussenT09}~\cite{RasmussenT09}, \href{../works/Wallace06.pdf}{Wallace06}~\cite{Wallace06}, \href{../works/Demassey03.pdf}{Demassey03}~\cite{Demassey03}, \href{../works/JainG01.pdf}{JainG01}~\cite{JainG01}, \href{../works/HookerO99.pdf}{HookerO99}~\cite{HookerO99} (Total: 48)\\
\index{Infeasible}\index{Concepts!Infeasible}Infeasible &  1.00 & \href{../works/ForbesHJST24.pdf}{ForbesHJST24}~\cite{ForbesHJST24}, \href{../works/EmdeZD22.pdf}{EmdeZD22}~\cite{EmdeZD22}, \href{../works/BulckG22.pdf}{BulckG22}~\cite{BulckG22}, \href{../works/KoehlerBFFHPSSS21.pdf}{KoehlerBFFHPSSS21}~\cite{KoehlerBFFHPSSS21}, \href{../works/RoshanaeiN21.pdf}{RoshanaeiN21}~\cite{RoshanaeiN21}, \href{../works/Lemos21.pdf}{Lemos21}~\cite{Lemos21}, \href{../works/VlkHT21.pdf}{VlkHT21}~\cite{VlkHT21}, \href{../works/TangB20.pdf}{TangB20}~\cite{TangB20}, \href{../works/Lunardi20.pdf}{Lunardi20}~\cite{Lunardi20}, \href{../works/AntunesABD20.pdf}{AntunesABD20}~\cite{AntunesABD20}, \href{../works/RoshanaeiBAUB20.pdf}{RoshanaeiBAUB20}~\cite{RoshanaeiBAUB20}, \href{../works/Hooker19.pdf}{Hooker19}~\cite{Hooker19}, \href{../works/Caballero19.pdf}{Caballero19}~\cite{Caballero19}, \href{../works/UnsalO19.pdf}{UnsalO19}~\cite{UnsalO19}, \href{../works/GoldwaserS18.pdf}{GoldwaserS18}~\cite{GoldwaserS18}, \href{../works/AgussurjaKL18.pdf}{AgussurjaKL18}~\cite{AgussurjaKL18}, \href{../works/MusliuSS18.pdf}{MusliuSS18}~\cite{MusliuSS18}, \href{../works/LaborieRSV18.pdf}{LaborieRSV18}~\cite{LaborieRSV18}, \href{../works/AntunesABD18.pdf}{AntunesABD18}~\cite{AntunesABD18}...\href{../works/HookerO03.pdf}{HookerO03}~\cite{HookerO03}, \href{../works/KamarainenS02.pdf}{KamarainenS02}~\cite{KamarainenS02}, \href{../works/HarjunkoskiG02.pdf}{HarjunkoskiG02}~\cite{HarjunkoskiG02}, \href{../works/Timpe02.pdf}{Timpe02}~\cite{Timpe02}, \href{../works/JainG01.pdf}{JainG01}~\cite{JainG01}, \href{../works/EreminW01.pdf}{EreminW01}~\cite{EreminW01}, \href{../works/HookerOTK00.pdf}{HookerOTK00}~\cite{HookerOTK00}, \href{../works/HookerO99.pdf}{HookerO99}~\cite{HookerO99}, \href{../works/SmithBHW96.pdf}{SmithBHW96}~\cite{SmithBHW96}, \href{../works/SadehF96.pdf}{SadehF96}~\cite{SadehF96} (Total: 73) & \href{../works/GokPTGO23.pdf}{GokPTGO23}~\cite{GokPTGO23}, \href{../works/NaderiBZR23.pdf}{NaderiBZR23}~\cite{NaderiBZR23}, \href{../works/ZhangBB22.pdf}{ZhangBB22}~\cite{ZhangBB22}, \href{../works/Astrand21.pdf}{Astrand21}~\cite{Astrand21}, \href{../works/FachiniA20.pdf}{FachiniA20}~\cite{FachiniA20}, \href{../works/SacramentoSP20.pdf}{SacramentoSP20}~\cite{SacramentoSP20}, \href{../works/LunardiBLRV20.pdf}{LunardiBLRV20}~\cite{LunardiBLRV20}, \href{../works/BhatnagarKL19.pdf}{BhatnagarKL19}~\cite{BhatnagarKL19}, \href{../works/TanZWGQ19.pdf}{TanZWGQ19}~\cite{TanZWGQ19}, \href{../works/NattafAL17.pdf}{NattafAL17}~\cite{NattafAL17}, \href{../works/BridiBLMB16.pdf}{BridiBLMB16}~\cite{BridiBLMB16}, \href{../works/CireCH16.pdf}{CireCH16}~\cite{CireCH16}, \href{../works/GoelSHFS15.pdf}{GoelSHFS15}~\cite{GoelSHFS15}, \href{../works/ZhaoL14.pdf}{ZhaoL14}~\cite{ZhaoL14}, \href{../works/ArtiguesLH13.pdf}{ArtiguesLH13}~\cite{ArtiguesLH13}, \href{../works/HeinzSB13.pdf}{HeinzSB13}~\cite{HeinzSB13}, \href{../works/TranB12.pdf}{TranB12}~\cite{TranB12}, \href{../works/HeinzB12.pdf}{HeinzB12}~\cite{HeinzB12}, \href{../works/MalapertGR12.pdf}{MalapertGR12}~\cite{MalapertGR12}...\href{../works/SchuttWS05.pdf}{SchuttWS05}~\cite{SchuttWS05}, \href{../works/GlobusCLP04.pdf}{GlobusCLP04}~\cite{GlobusCLP04}, \href{../works/Trick03.pdf}{Trick03}~\cite{Trick03}, \href{../works/BosiM2001.pdf}{BosiM2001}~\cite{BosiM2001}, \href{../works/Thorsteinsson01.pdf}{Thorsteinsson01}~\cite{Thorsteinsson01}, \href{../works/SakkoutW00.pdf}{SakkoutW00}~\cite{SakkoutW00}, \href{../works/HarjunkoskiJG00.pdf}{HarjunkoskiJG00}~\cite{HarjunkoskiJG00}, \href{../works/PesantGPR99.pdf}{PesantGPR99}~\cite{PesantGPR99}, \href{../works/NuijtenA96.pdf}{NuijtenA96}~\cite{NuijtenA96}, \href{../works/NuijtenA94.pdf}{NuijtenA94}~\cite{NuijtenA94} (Total: 48) & \href{../works/abs-2402-00459.pdf}{abs-2402-00459}~\cite{abs-2402-00459}, \href{../works/LuZZYW24.pdf}{LuZZYW24}~\cite{LuZZYW24}, \href{../works/AalianPG23.pdf}{AalianPG23}~\cite{AalianPG23}, \href{../works/LacknerMMWW23.pdf}{LacknerMMWW23}~\cite{LacknerMMWW23}, \href{../works/GuoZ23.pdf}{GuoZ23}~\cite{GuoZ23}, \href{../works/WessenCSFPM23.pdf}{WessenCSFPM23}~\cite{WessenCSFPM23}, \href{../works/Fatemi-AnarakiTFV23.pdf}{Fatemi-AnarakiTFV23}~\cite{Fatemi-AnarakiTFV23}, \href{../works/SquillaciPR23.pdf}{SquillaciPR23}~\cite{SquillaciPR23}, \href{../works/abs-2305-19888.pdf}{abs-2305-19888}~\cite{abs-2305-19888}, \href{../works/AbreuNP23.pdf}{AbreuNP23}~\cite{AbreuNP23}, \href{../works/PovedaAA23.pdf}{PovedaAA23}~\cite{PovedaAA23}, \href{../works/JuvinHL23.pdf}{JuvinHL23}~\cite{JuvinHL23}, \href{../works/NaderiBZ23.pdf}{NaderiBZ23}~\cite{NaderiBZ23}, \href{../works/MarliereSPR23.pdf}{MarliereSPR23}~\cite{MarliereSPR23}, \href{../works/ZhangJZL22.pdf}{ZhangJZL22}~\cite{ZhangJZL22}, \href{../works/LuoB22.pdf}{LuoB22}~\cite{LuoB22}, \href{../works/ElciOH22.pdf}{ElciOH22}~\cite{ElciOH22}, \href{../works/CilKLO22.pdf}{CilKLO22}~\cite{CilKLO22}, \href{../works/HeinzNVH22.pdf}{HeinzNVH22}~\cite{HeinzNVH22}...\href{../works/CarlssonKA99.pdf}{CarlssonKA99}~\cite{CarlssonKA99}, \href{../works/NuijtenP98.pdf}{NuijtenP98}~\cite{NuijtenP98}, \href{../works/PintoG97.pdf}{PintoG97}~\cite{PintoG97}, \href{../works/Darby-DowmanLMZ97.pdf}{Darby-DowmanLMZ97}~\cite{Darby-DowmanLMZ97}, \href{../works/Zhou97.pdf}{Zhou97}~\cite{Zhou97}, \href{../works/Schaerf97.pdf}{Schaerf97}~\cite{Schaerf97}, \href{../works/Wallace96.pdf}{Wallace96}~\cite{Wallace96}, \href{../works/FoxS90.pdf}{FoxS90}~\cite{FoxS90}, \href{../works/DincbasSH90.pdf}{DincbasSH90}~\cite{DincbasSH90}, \href{../works/FeldmanG89.pdf}{FeldmanG89}~\cite{FeldmanG89} (Total: 179)\\
\index{Logic-Based Benders Decomposition}\index{Concepts!Logic-Based Benders Decomposition}Logic-Based Benders Decomposition &  1.00 & \href{../works/ForbesHJST24.pdf}{ForbesHJST24}~\cite{ForbesHJST24}, \href{../works/ZhuSZW23.pdf}{ZhuSZW23}~\cite{ZhuSZW23}, \href{../works/NaderiBZR23.pdf}{NaderiBZR23}~\cite{NaderiBZR23}, \href{../works/JuvinHL23a.pdf}{JuvinHL23a}~\cite{JuvinHL23a}, \href{../works/GuoZ23.pdf}{GuoZ23}~\cite{GuoZ23}, \href{../works/NaderiBZ22a.pdf}{NaderiBZ22a}~\cite{NaderiBZ22a}, \href{../works/ElciOH22.pdf}{ElciOH22}~\cite{ElciOH22}, \href{../works/JuvinHL22.pdf}{JuvinHL22}~\cite{JuvinHL22}, \href{../works/EmdeZD22.pdf}{EmdeZD22}~\cite{EmdeZD22}, \href{../works/VlkHT21.pdf}{VlkHT21}~\cite{VlkHT21}, \href{../works/RoshanaeiN21.pdf}{RoshanaeiN21}~\cite{RoshanaeiN21}, \href{../works/FachiniA20.pdf}{FachiniA20}~\cite{FachiniA20}, \href{../works/UnsalO19.pdf}{UnsalO19}~\cite{UnsalO19}, \href{../works/Hooker19.pdf}{Hooker19}~\cite{Hooker19}, \href{../works/SunTB19.pdf}{SunTB19}~\cite{SunTB19}, \href{../works/GoldwaserS18.pdf}{GoldwaserS18}~\cite{GoldwaserS18}, \href{../works/TanT18.pdf}{TanT18}~\cite{TanT18}, \href{../works/EmeretlisTAV17.pdf}{EmeretlisTAV17}~\cite{EmeretlisTAV17}, \href{../works/HookerH17.pdf}{HookerH17}~\cite{HookerH17}...\href{../works/BajestaniB13.pdf}{BajestaniB13}~\cite{BajestaniB13}, \href{../works/TranB12.pdf}{TranB12}~\cite{TranB12}, \href{../works/LombardiM12.pdf}{LombardiM12}~\cite{LombardiM12}, \href{../works/BeniniLMR11.pdf}{BeniniLMR11}~\cite{BeniniLMR11}, \href{../works/BajestaniB11.pdf}{BajestaniB11}~\cite{BajestaniB11}, \href{../works/CobanH11.pdf}{CobanH11}~\cite{CobanH11}, \href{../works/Beck10.pdf}{Beck10}~\cite{Beck10}, \href{../works/Hooker07.pdf}{Hooker07}~\cite{Hooker07}, \href{../works/Hooker05.pdf}{Hooker05}~\cite{Hooker05}, \href{../works/Hooker04.pdf}{Hooker04}~\cite{Hooker04} (Total: 37) & \href{../works/NaderiBZ23.pdf}{NaderiBZ23}~\cite{NaderiBZ23}, \href{../works/NaderiRR23.pdf}{NaderiRR23}~\cite{NaderiRR23}, \href{../works/NaderiBZ22.pdf}{NaderiBZ22}~\cite{NaderiBZ22}, \href{../works/TangB20.pdf}{TangB20}~\cite{TangB20}, \href{../works/RoshanaeiBAUB20.pdf}{RoshanaeiBAUB20}~\cite{RoshanaeiBAUB20}, \href{../works/Laborie18a.pdf}{Laborie18a}~\cite{Laborie18a}, \href{../works/GoldwaserS17.pdf}{GoldwaserS17}~\cite{GoldwaserS17}, \href{../works/Froger16.pdf}{Froger16}~\cite{Froger16}, \href{../works/HeinzB12.pdf}{HeinzB12}~\cite{HeinzB12}, \href{../works/GuyonLPR12.pdf}{GuyonLPR12}~\cite{GuyonLPR12}, \href{../works/Lombardi10.pdf}{Lombardi10}~\cite{Lombardi10}, \href{../works/CobanH10.pdf}{CobanH10}~\cite{CobanH10}, \href{../works/MilanoW09.pdf}{MilanoW09}~\cite{MilanoW09}, \href{../works/BeniniLMR08.pdf}{BeniniLMR08}~\cite{BeniniLMR08}, \href{../works/BeniniLMMR08.pdf}{BeniniLMMR08}~\cite{BeniniLMMR08}, \href{../works/HladikCDJ08.pdf}{HladikCDJ08}~\cite{HladikCDJ08}, \href{../works/CorreaLR07.pdf}{CorreaLR07}~\cite{CorreaLR07}, \href{../works/RasmussenT07.pdf}{RasmussenT07}~\cite{RasmussenT07}, \href{../works/Hooker06.pdf}{Hooker06}~\cite{Hooker06}, \href{../works/HookerY02.pdf}{HookerY02}~\cite{HookerY02} & \href{../works/PrataAN23.pdf}{PrataAN23}~\cite{PrataAN23}, \href{../works/JuvinHHL23.pdf}{JuvinHHL23}~\cite{JuvinHHL23}, \href{../works/BulckG22.pdf}{BulckG22}~\cite{BulckG22}, \href{../works/FarsiTM22.pdf}{FarsiTM22}~\cite{FarsiTM22}, \href{../works/Mercier-AubinGQ20.pdf}{Mercier-AubinGQ20}~\cite{Mercier-AubinGQ20}, \href{../works/QinDCS20.pdf}{QinDCS20}~\cite{QinDCS20}, \href{../works/WallaceY20.pdf}{WallaceY20}~\cite{WallaceY20}, \href{../works/MurinR19.pdf}{MurinR19}~\cite{MurinR19}, \href{../works/PachecoPR19.pdf}{PachecoPR19}~\cite{PachecoPR19}, \href{../works/CappartTSR18.pdf}{CappartTSR18}~\cite{CappartTSR18}, \href{../works/LaborieRSV18.pdf}{LaborieRSV18}~\cite{LaborieRSV18}, \href{../works/GombolayWS18.pdf}{GombolayWS18}~\cite{GombolayWS18}, \href{../works/AntunesABD18.pdf}{AntunesABD18}~\cite{AntunesABD18}, \href{../works/AgussurjaKL18.pdf}{AgussurjaKL18}~\cite{AgussurjaKL18}, \href{../works/GomesM17.pdf}{GomesM17}~\cite{GomesM17}, \href{../works/TranVNB17.pdf}{TranVNB17}~\cite{TranVNB17}, \href{../works/RoshanaeiLAU17.pdf}{RoshanaeiLAU17}~\cite{RoshanaeiLAU17}, \href{../works/QinDS16.pdf}{QinDS16}~\cite{QinDS16}, \href{../works/BoothNB16.pdf}{BoothNB16}~\cite{BoothNB16}...\href{../works/ChuX05.pdf}{ChuX05}~\cite{ChuX05}, \href{../works/Hooker05b.pdf}{Hooker05b}~\cite{Hooker05b}, \href{../works/Hooker05a.pdf}{Hooker05a}~\cite{Hooker05a}, \href{../works/CambazardJ05.pdf}{CambazardJ05}~\cite{CambazardJ05}, \href{../works/CambazardHDJT04.pdf}{CambazardHDJT04}~\cite{CambazardHDJT04}, \href{../works/Demassey03.pdf}{Demassey03}~\cite{Demassey03}, \href{../works/BenoistGR02.pdf}{BenoistGR02}~\cite{BenoistGR02}, \href{../works/EreminW01.pdf}{EreminW01}~\cite{EreminW01}, \href{../works/Thorsteinsson01.pdf}{Thorsteinsson01}~\cite{Thorsteinsson01}, \href{../works/HookerOTK00.pdf}{HookerOTK00}~\cite{HookerOTK00} (Total: 52)\\
\index{Over-constrained}\index{Concepts!Over-constrained}Over-constrained &  1.00 & \href{../works/BeckF00.pdf}{BeckF00}~\cite{BeckF00}, \href{../works/Beck99.pdf}{Beck99}~\cite{Beck99} & \href{../works/OrnekOS20.pdf}{OrnekOS20}~\cite{OrnekOS20}, \href{../works/ClercqPBJ11.pdf}{ClercqPBJ11}~\cite{ClercqPBJ11}, \href{../works/abs-0907-0939.pdf}{abs-0907-0939}~\cite{abs-0907-0939}, \href{../works/BeckF00a.pdf}{BeckF00a}~\cite{BeckF00a}, \href{../works/SimonisCK00.pdf}{SimonisCK00}~\cite{SimonisCK00}, \href{../works/Simonis99.pdf}{Simonis99}~\cite{Simonis99}, \href{../works/Wallace96.pdf}{Wallace96}~\cite{Wallace96}, \href{../works/MintonJPL92.pdf}{MintonJPL92}~\cite{MintonJPL92} & \href{../works/SquillaciPR23.pdf}{SquillaciPR23}~\cite{SquillaciPR23}, \href{../works/Teppan22.pdf}{Teppan22}~\cite{Teppan22}, \href{../works/BoudreaultSLQ22.pdf}{BoudreaultSLQ22}~\cite{BoudreaultSLQ22}, \href{../works/Lemos21.pdf}{Lemos21}~\cite{Lemos21}, \href{../works/Caballero19.pdf}{Caballero19}~\cite{Caballero19}, \href{../works/LaborieRSV18.pdf}{LaborieRSV18}~\cite{LaborieRSV18}, \href{../works/EmeretlisTAV17.pdf}{EmeretlisTAV17}~\cite{EmeretlisTAV17}, \href{../works/HookerH17.pdf}{HookerH17}~\cite{HookerH17}, \href{../works/FrankDT16.pdf}{FrankDT16}~\cite{FrankDT16}, \href{../works/Dejemeppe16.pdf}{Dejemeppe16}~\cite{Dejemeppe16}, \href{../works/LimBTBB15a.pdf}{LimBTBB15a}~\cite{LimBTBB15a}, \href{../works/MurphyMB15.pdf}{MurphyMB15}~\cite{MurphyMB15}, \href{../works/GaySS14.pdf}{GaySS14}~\cite{GaySS14}, \href{../works/HoundjiSWD14.pdf}{HoundjiSWD14}~\cite{HoundjiSWD14}, \href{../works/LombardiMB13.pdf}{LombardiMB13}~\cite{LombardiMB13}, \href{../works/SerraNM12.pdf}{SerraNM12}~\cite{SerraNM12}, \href{../works/Clercq12.pdf}{Clercq12}~\cite{Clercq12}, \href{../works/BeniniLMR11.pdf}{BeniniLMR11}~\cite{BeniniLMR11}, \href{../works/LopesCSM10.pdf}{LopesCSM10}~\cite{LopesCSM10}...\href{../works/BourdaisGP03.pdf}{BourdaisGP03}~\cite{BourdaisGP03}, \href{../works/ElkhyariGJ02a.pdf}{ElkhyariGJ02a}~\cite{ElkhyariGJ02a}, \href{../works/VanczaM01.pdf}{VanczaM01}~\cite{VanczaM01}, \href{../works/AbdennadherS99.pdf}{AbdennadherS99}~\cite{AbdennadherS99}, \href{../works/PembertonG98.pdf}{PembertonG98}~\cite{PembertonG98}, \href{../works/BeckDF97.pdf}{BeckDF97}~\cite{BeckDF97}, \href{../works/SadehF96.pdf}{SadehF96}~\cite{SadehF96}, \href{../works/Simonis95a.pdf}{Simonis95a}~\cite{Simonis95a}, \href{../works/CrawfordB94.pdf}{CrawfordB94}~\cite{CrawfordB94}, \href{../works/KengY89.pdf}{KengY89}~\cite{KengY89} (Total: 41)\\
\index{Pareto}\index{Concepts!Pareto}Pareto &  1.00 & \href{../works/FarsiTM22.pdf}{FarsiTM22}~\cite{FarsiTM22}, \href{../works/Zahout21.pdf}{Zahout21}~\cite{Zahout21}, \href{../works/Lemos21.pdf}{Lemos21}~\cite{Lemos21}, \href{../works/ZarandiASC20.pdf}{ZarandiASC20}~\cite{ZarandiASC20}, \href{../works/Dejemeppe16.pdf}{Dejemeppe16}~\cite{Dejemeppe16}, \href{../works/KovacsK11.pdf}{KovacsK11}~\cite{KovacsK11} & \href{../works/Edis21.pdf}{Edis21}~\cite{Edis21}, \href{../works/YounespourAKE19.pdf}{YounespourAKE19}~\cite{YounespourAKE19}, \href{../works/DejemeppeD14.pdf}{DejemeppeD14}~\cite{DejemeppeD14}, \href{../works/HeckmanB11.pdf}{HeckmanB11}~\cite{HeckmanB11} & \href{../works/GokPTGO23.pdf}{GokPTGO23}~\cite{GokPTGO23}, \href{../works/CzerniachowskaWZ23.pdf}{CzerniachowskaWZ23}~\cite{CzerniachowskaWZ23}, \href{../works/FrimodigECM23.pdf}{FrimodigECM23}~\cite{FrimodigECM23}, \href{../works/LacknerMMWW23.pdf}{LacknerMMWW23}~\cite{LacknerMMWW23}, \href{../works/JuvinHL23a.pdf}{JuvinHL23a}~\cite{JuvinHL23a}, \href{../works/WinterMMW22.pdf}{WinterMMW22}~\cite{WinterMMW22}, \href{../works/MengGRZSC22.pdf}{MengGRZSC22}~\cite{MengGRZSC22}, \href{../works/JuvinHL22.pdf}{JuvinHL22}~\cite{JuvinHL22}, \href{../works/OrnekOS20.pdf}{OrnekOS20}~\cite{OrnekOS20}, \href{../works/KletzanderMH21.pdf}{KletzanderMH21}~\cite{KletzanderMH21}, \href{../works/AntuoriHHEN20.pdf}{AntuoriHHEN20}~\cite{AntuoriHHEN20}, \href{../works/Lunardi20.pdf}{Lunardi20}~\cite{Lunardi20}, \href{../works/EscobetPQPRA19.pdf}{EscobetPQPRA19}~\cite{EscobetPQPRA19}, \href{../works/TanZWGQ19.pdf}{TanZWGQ19}~\cite{TanZWGQ19}, \href{../works/GurEA19.pdf}{GurEA19}~\cite{GurEA19}, \href{../works/CappartTSR18.pdf}{CappartTSR18}~\cite{CappartTSR18}, \href{../works/GomesM17.pdf}{GomesM17}~\cite{GomesM17}, \href{../works/QinDS16.pdf}{QinDS16}~\cite{QinDS16}, \href{../works/Froger16.pdf}{Froger16}~\cite{Froger16}, \href{../works/BridiBLMB16.pdf}{BridiBLMB16}~\cite{BridiBLMB16}, \href{../works/HarjunkoskiMBC14.pdf}{HarjunkoskiMBC14}~\cite{HarjunkoskiMBC14}, \href{../works/KendallKRU10.pdf}{KendallKRU10}~\cite{KendallKRU10}, \href{../works/RuggieroBBMA09.pdf}{RuggieroBBMA09}~\cite{RuggieroBBMA09}, \href{../works/HladikCDJ08.pdf}{HladikCDJ08}~\cite{HladikCDJ08}, \href{../works/Johnston05.pdf}{Johnston05}~\cite{Johnston05}, \href{../works/Baptiste02.pdf}{Baptiste02}~\cite{Baptiste02}, \href{../works/VanczaM01.pdf}{VanczaM01}~\cite{VanczaM01}, \href{../works/FocacciLN00.pdf}{FocacciLN00}~\cite{FocacciLN00}\\
\index{Unsatisfiable}\index{Concepts!Unsatisfiable}Unsatisfiable &  1.00 & \href{../works/WessenCSFPM23.pdf}{WessenCSFPM23}~\cite{WessenCSFPM23}, \href{../works/Godet21a.pdf}{Godet21a}~\cite{Godet21a}, \href{../works/KoehlerBFFHPSSS21.pdf}{KoehlerBFFHPSSS21}~\cite{KoehlerBFFHPSSS21}, \href{../works/Caballero19.pdf}{Caballero19}~\cite{Caballero19}, \href{../works/Siala15a.pdf}{Siala15a}~\cite{Siala15a}, \href{../works/Schutt11.pdf}{Schutt11}~\cite{Schutt11}, \href{../works/HookerO03.pdf}{HookerO03}~\cite{HookerO03} & \href{../works/OuelletQ22.pdf}{OuelletQ22}~\cite{OuelletQ22}, \href{../works/Lemos21.pdf}{Lemos21}~\cite{Lemos21}, \href{../works/Valdes87.pdf}{Valdes87}~\cite{Valdes87} & \href{../works/BoudreaultSLQ22.pdf}{BoudreaultSLQ22}~\cite{BoudreaultSLQ22}, \href{../works/HebrardALLCMR22.pdf}{HebrardALLCMR22}~\cite{HebrardALLCMR22}, \href{../works/GeibingerMM21.pdf}{GeibingerMM21}~\cite{GeibingerMM21}, \href{../works/VlkHT21.pdf}{VlkHT21}~\cite{VlkHT21}, \href{../works/CauwelaertDS20.pdf}{CauwelaertDS20}~\cite{CauwelaertDS20}, \href{../works/abs-1901-07914.pdf}{abs-1901-07914}~\cite{abs-1901-07914}, \href{../works/GoldwaserS18.pdf}{GoldwaserS18}~\cite{GoldwaserS18}, \href{../works/FahimiOQ18.pdf}{FahimiOQ18}~\cite{FahimiOQ18}, \href{../works/Dejemeppe16.pdf}{Dejemeppe16}~\cite{Dejemeppe16}, \href{../works/Fahimi16.pdf}{Fahimi16}~\cite{Fahimi16}, \href{../works/GrimesH15.pdf}{GrimesH15}~\cite{GrimesH15}, \href{../works/BertholdHLMS10.pdf}{BertholdHLMS10}~\cite{BertholdHLMS10}, \href{../works/GrimesH10.pdf}{GrimesH10}~\cite{GrimesH10}, \href{../works/OhrimenkoSC09.pdf}{OhrimenkoSC09}~\cite{OhrimenkoSC09}, \href{../works/FalaschiGMP97.pdf}{FalaschiGMP97}~\cite{FalaschiGMP97}, \href{../works/SimonisC95.pdf}{SimonisC95}~\cite{SimonisC95}, \href{../works/Nuijten94.pdf}{Nuijten94}~\cite{Nuijten94}\\
\index{batch process}\index{Concepts!batch process}batch process &  1.00 & \href{../works/LacknerMMWW23.pdf}{LacknerMMWW23}~\cite{LacknerMMWW23}, \href{../works/AwadMDMT22.pdf}{AwadMDMT22}~\cite{AwadMDMT22}, \href{../works/LacknerMMWW21.pdf}{LacknerMMWW21}~\cite{LacknerMMWW21}, \href{../works/QinWSLS21.pdf}{QinWSLS21}~\cite{QinWSLS21}, \href{../works/ZarandiASC20.pdf}{ZarandiASC20}~\cite{ZarandiASC20}, \href{../works/HamC16.pdf}{HamC16}~\cite{HamC16}, \href{../works/NovaraNH16.pdf}{NovaraNH16}~\cite{NovaraNH16}, \href{../works/KoschB14.pdf}{KoschB14}~\cite{KoschB14}, \href{../works/HarjunkoskiMBC14.pdf}{HarjunkoskiMBC14}~\cite{HarjunkoskiMBC14}, \href{../works/MalapertGR12.pdf}{MalapertGR12}~\cite{MalapertGR12}, \href{../works/Malapert11.pdf}{Malapert11}~\cite{Malapert11} & \href{../works/TangB20.pdf}{TangB20}~\cite{TangB20}, \href{../works/HamFC17.pdf}{HamFC17}~\cite{HamFC17}, \href{../works/NovasH10.pdf}{NovasH10}~\cite{NovasH10}, \href{../works/ZeballosCM10.pdf}{ZeballosCM10}~\cite{ZeballosCM10}, \href{../works/RenT09.pdf}{RenT09}~\cite{RenT09}, \href{../works/RoePS05.pdf}{RoePS05}~\cite{RoePS05}, \href{../works/Vilim02.pdf}{Vilim02}~\cite{Vilim02}, \href{../works/PintoG97.pdf}{PintoG97}~\cite{PintoG97}, \href{../works/SimonisC95.pdf}{SimonisC95}~\cite{SimonisC95} & \href{../works/PrataAN23.pdf}{PrataAN23}~\cite{PrataAN23}, \href{../works/YuraszeckMCCR23.pdf}{YuraszeckMCCR23}~\cite{YuraszeckMCCR23}, \href{../works/IsikYA23.pdf}{IsikYA23}~\cite{IsikYA23}, \href{../works/Adelgren2023.pdf}{Adelgren2023}~\cite{Adelgren2023}, \href{../works/MullerMKP22.pdf}{MullerMKP22}~\cite{MullerMKP22}, \href{../works/SvancaraB22.pdf}{SvancaraB22}~\cite{SvancaraB22}, \href{../works/EmdeZD22.pdf}{EmdeZD22}~\cite{EmdeZD22}, \href{../works/LiFJZLL22.pdf}{LiFJZLL22}~\cite{LiFJZLL22}, \href{../works/ColT22.pdf}{ColT22}~\cite{ColT22}, \href{../works/AbreuN22.pdf}{AbreuN22}~\cite{AbreuN22}, \href{../works/YunusogluY22.pdf}{YunusogluY22}~\cite{YunusogluY22}, \href{../works/LuoB22.pdf}{LuoB22}~\cite{LuoB22}, \href{../works/GeitzGSSW22.pdf}{GeitzGSSW22}~\cite{GeitzGSSW22}, \href{../works/OujanaAYB22.pdf}{OujanaAYB22}~\cite{OujanaAYB22}, \href{../works/ZhangYW21.pdf}{ZhangYW21}~\cite{ZhangYW21}, \href{../works/FanXG21.pdf}{FanXG21}~\cite{FanXG21}, \href{../works/KlankeBYE21.pdf}{KlankeBYE21}~\cite{KlankeBYE21}, \href{../works/MengZRZL20.pdf}{MengZRZL20}~\cite{MengZRZL20}, \href{../works/CauwelaertDS20.pdf}{CauwelaertDS20}~\cite{CauwelaertDS20}...\href{../works/Fahimi16.pdf}{Fahimi16}~\cite{Fahimi16}, \href{../works/GrimesH15.pdf}{GrimesH15}~\cite{GrimesH15}, \href{../works/ZeballosNH11.pdf}{ZeballosNH11}~\cite{ZeballosNH11}, \href{../works/GrimesH10.pdf}{GrimesH10}~\cite{GrimesH10}, \href{../works/Zeballos10.pdf}{Zeballos10}~\cite{Zeballos10}, \href{../works/ZeballosM09.pdf}{ZeballosM09}~\cite{ZeballosM09}, \href{../works/ArtiguesF07.pdf}{ArtiguesF07}~\cite{ArtiguesF07}, \href{../works/Simonis07.pdf}{Simonis07}~\cite{Simonis07}, \href{../works/VilimBC05.pdf}{VilimBC05}~\cite{VilimBC05}, \href{../works/ArtiguesBF04.pdf}{ArtiguesBF04}~\cite{ArtiguesBF04} (Total: 42)\\
\index{bi-objective}\index{Concepts!bi-objective}bi-objective &  1.00 & \href{../works/ZarandiASC20.pdf}{ZarandiASC20}~\cite{ZarandiASC20} & \href{../works/IsikYA23.pdf}{IsikYA23}~\cite{IsikYA23}, \href{../works/AbreuPNF23.pdf}{AbreuPNF23}~\cite{AbreuPNF23}, \href{../works/YunusogluY22.pdf}{YunusogluY22}~\cite{YunusogluY22}, \href{../works/HillTV21.pdf}{HillTV21}~\cite{HillTV21}, \href{../works/Lemos21.pdf}{Lemos21}~\cite{Lemos21}, \href{../works/NattafM20.pdf}{NattafM20}~\cite{NattafM20}, \href{../works/Dejemeppe16.pdf}{Dejemeppe16}~\cite{Dejemeppe16}, \href{../works/DejemeppeD14.pdf}{DejemeppeD14}~\cite{DejemeppeD14} & \href{../works/LuZZYW24.pdf}{LuZZYW24}~\cite{LuZZYW24}, \href{../works/PrataAN23.pdf}{PrataAN23}~\cite{PrataAN23}, \href{../works/AfsarVPG23.pdf}{AfsarVPG23}~\cite{AfsarVPG23}, \href{../works/Mehdizadeh-Somarin23.pdf}{Mehdizadeh-Somarin23}~\cite{Mehdizadeh-Somarin23}, \href{../works/GokPTGO23.pdf}{GokPTGO23}~\cite{GokPTGO23}, \href{../works/NaderiBZR23.pdf}{NaderiBZR23}~\cite{NaderiBZR23}, \href{../works/GurPAE23.pdf}{GurPAE23}~\cite{GurPAE23}, \href{../works/NaderiRR23.pdf}{NaderiRR23}~\cite{NaderiRR23}, \href{../works/abs-2305-19888.pdf}{abs-2305-19888}~\cite{abs-2305-19888}, \href{../works/MullerMKP22.pdf}{MullerMKP22}~\cite{MullerMKP22}, \href{../works/HeinzNVH22.pdf}{HeinzNVH22}~\cite{HeinzNVH22}, \href{../works/WinterMMW22.pdf}{WinterMMW22}~\cite{WinterMMW22}, \href{../works/PopovicCGNC22.pdf}{PopovicCGNC22}~\cite{PopovicCGNC22}, \href{../works/MengGRZSC22.pdf}{MengGRZSC22}~\cite{MengGRZSC22}, \href{../works/AbreuN22.pdf}{AbreuN22}~\cite{AbreuN22}, \href{../works/FarsiTM22.pdf}{FarsiTM22}~\cite{FarsiTM22}, \href{../works/EmdeZD22.pdf}{EmdeZD22}~\cite{EmdeZD22}, \href{../works/Groleaz21.pdf}{Groleaz21}~\cite{Groleaz21}, \href{../works/Zahout21.pdf}{Zahout21}~\cite{Zahout21}...\href{../works/LunardiBLRV20.pdf}{LunardiBLRV20}~\cite{LunardiBLRV20}, \href{../works/RoshanaeiBAUB20.pdf}{RoshanaeiBAUB20}~\cite{RoshanaeiBAUB20}, \href{../works/MalapertN19.pdf}{MalapertN19}~\cite{MalapertN19}, \href{../works/BhatnagarKL19.pdf}{BhatnagarKL19}~\cite{BhatnagarKL19}, \href{../works/abs-1902-09244.pdf}{abs-1902-09244}~\cite{abs-1902-09244}, \href{../works/AgussurjaKL18.pdf}{AgussurjaKL18}~\cite{AgussurjaKL18}, \href{../works/HamC16.pdf}{HamC16}~\cite{HamC16}, \href{../works/Nattaf16.pdf}{Nattaf16}~\cite{Nattaf16}, \href{../works/QinDS16.pdf}{QinDS16}~\cite{QinDS16}, \href{../works/BurtLPS15.pdf}{BurtLPS15}~\cite{BurtLPS15} (Total: 39)\\
\index{bill of material}\index{Concepts!bill of material}bill of material &  1.00 &  & \href{../works/OrnekO16.pdf}{OrnekO16}~\cite{OrnekO16} & \href{../works/Simonis07.pdf}{Simonis07}~\cite{Simonis07}\\
\index{blocking constraint}\index{Concepts!blocking constraint}blocking constraint &  1.00 & \href{../works/AbreuNP23.pdf}{AbreuNP23}~\cite{AbreuNP23}, \href{../works/RiahiNS018.pdf}{RiahiNS018}~\cite{RiahiNS018} &  & \href{../works/WessenCSFPM23.pdf}{WessenCSFPM23}~\cite{WessenCSFPM23}, \href{../works/IsikYA23.pdf}{IsikYA23}~\cite{IsikYA23}, \href{../works/LiFJZLL22.pdf}{LiFJZLL22}~\cite{LiFJZLL22}, \href{../works/MengZRZL20.pdf}{MengZRZL20}~\cite{MengZRZL20}, \href{../works/RodriguezS09.pdf}{RodriguezS09}~\cite{RodriguezS09}, \href{../works/Rodriguez07b.pdf}{Rodriguez07b}~\cite{Rodriguez07b}, \href{../works/Rodriguez07.pdf}{Rodriguez07}~\cite{Rodriguez07}\\
\index{breakdown}\index{Concepts!breakdown}breakdown &  1.00 & \href{../works/Groleaz21.pdf}{Groleaz21}~\cite{Groleaz21}, \href{../works/FanXG21.pdf}{FanXG21}~\cite{FanXG21}, \href{../works/ZarandiASC20.pdf}{ZarandiASC20}~\cite{ZarandiASC20}, \href{../works/LaborieRSV18.pdf}{LaborieRSV18}~\cite{LaborieRSV18}, \href{../works/ZhangW18.pdf}{ZhangW18}~\cite{ZhangW18}, \href{../works/Froger16.pdf}{Froger16}~\cite{Froger16}, \href{../works/BartakV15.pdf}{BartakV15}~\cite{BartakV15}, \href{../works/NovasH10.pdf}{NovasH10}~\cite{NovasH10}, \href{../works/BidotVLB09.pdf}{BidotVLB09}~\cite{BidotVLB09} & \href{../works/NaderiBZR23.pdf}{NaderiBZR23}~\cite{NaderiBZR23}, \href{../works/Lunardi20.pdf}{Lunardi20}~\cite{Lunardi20}, \href{../works/GombolayWS18.pdf}{GombolayWS18}~\cite{GombolayWS18}, \href{../works/RoshanaeiLAU17.pdf}{RoshanaeiLAU17}~\cite{RoshanaeiLAU17}, \href{../works/BajestaniB15.pdf}{BajestaniB15}~\cite{BajestaniB15}, \href{../works/ThiruvadyWGS14.pdf}{ThiruvadyWGS14}~\cite{ThiruvadyWGS14}, \href{../works/HarjunkoskiMBC14.pdf}{HarjunkoskiMBC14}~\cite{HarjunkoskiMBC14}, \href{../works/BajestaniB13.pdf}{BajestaniB13}~\cite{BajestaniB13}, \href{../works/BajestaniB11.pdf}{BajestaniB11}~\cite{BajestaniB11}, \href{../works/Elkhyari03.pdf}{Elkhyari03}~\cite{Elkhyari03}, \href{../works/MartinPY01.pdf}{MartinPY01}~\cite{MartinPY01}, \href{../works/BeckDDF98.pdf}{BeckDDF98}~\cite{BeckDDF98} & \href{../works/Fatemi-AnarakiTFV23.pdf}{Fatemi-AnarakiTFV23}~\cite{Fatemi-AnarakiTFV23}, \href{../works/JuvinHL23.pdf}{JuvinHL23}~\cite{JuvinHL23}, \href{../works/PenzDN23.pdf}{PenzDN23}~\cite{PenzDN23}, \href{../works/IsikYA23.pdf}{IsikYA23}~\cite{IsikYA23}, \href{../works/MullerMKP22.pdf}{MullerMKP22}~\cite{MullerMKP22}, \href{../works/ColT22.pdf}{ColT22}~\cite{ColT22}, \href{../works/SubulanC22.pdf}{SubulanC22}~\cite{SubulanC22}, \href{../works/AwadMDMT22.pdf}{AwadMDMT22}~\cite{AwadMDMT22}, \href{../works/Tassel22.pdf}{Tassel22}~\cite{Tassel22}, \href{../works/YunusogluY22.pdf}{YunusogluY22}~\cite{YunusogluY22}, \href{../works/AbreuAPNM21.pdf}{AbreuAPNM21}~\cite{AbreuAPNM21}, \href{../works/KovacsTKSG21.pdf}{KovacsTKSG21}~\cite{KovacsTKSG21}, \href{../works/Astrand21.pdf}{Astrand21}~\cite{Astrand21}, \href{../works/AstrandJZ20.pdf}{AstrandJZ20}~\cite{AstrandJZ20}, \href{../works/HauderBRPA20.pdf}{HauderBRPA20}~\cite{HauderBRPA20}, \href{../works/PachecoPR19.pdf}{PachecoPR19}~\cite{PachecoPR19}, \href{../works/MalapertN19.pdf}{MalapertN19}~\cite{MalapertN19}, \href{../works/abs-1902-09244.pdf}{abs-1902-09244}~\cite{abs-1902-09244}, \href{../works/GedikKEK18.pdf}{GedikKEK18}~\cite{GedikKEK18}...\href{../works/BeniniLMMR08.pdf}{BeniniLMMR08}~\cite{BeniniLMMR08}, \href{../works/QuirogaZH05.pdf}{QuirogaZH05}~\cite{QuirogaZH05}, \href{../works/BeckR03.pdf}{BeckR03}~\cite{BeckR03}, \href{../works/Timpe02.pdf}{Timpe02}~\cite{Timpe02}, \href{../works/ElkhyariGJ02a.pdf}{ElkhyariGJ02a}~\cite{ElkhyariGJ02a}, \href{../works/ElkhyariGJ02.pdf}{ElkhyariGJ02}~\cite{ElkhyariGJ02}, \href{../works/Dorndorf2000.pdf}{Dorndorf2000}~\cite{Dorndorf2000}, \href{../works/JoLLH99.pdf}{JoLLH99}~\cite{JoLLH99}, \href{../works/CarlssonKA99.pdf}{CarlssonKA99}~\cite{CarlssonKA99}, \href{../works/LeeKLKKYHP97.pdf}{LeeKLKKYHP97}~\cite{LeeKLKKYHP97} (Total: 47)\\
\index{buffer-capacity}\index{Concepts!buffer-capacity}buffer-capacity &  1.00 &  & \href{../works/SureshMOK06.pdf}{SureshMOK06}~\cite{SureshMOK06} & \href{../works/OujanaAYB22.pdf}{OujanaAYB22}~\cite{OujanaAYB22}, \href{../works/LiFJZLL22.pdf}{LiFJZLL22}~\cite{LiFJZLL22}, \href{../works/RiahiNS018.pdf}{RiahiNS018}~\cite{RiahiNS018}, \href{../works/BonfiettiLBM14.pdf}{BonfiettiLBM14}~\cite{BonfiettiLBM14}, \href{../works/NovasH14.pdf}{NovasH14}~\cite{NovasH14}, \href{../works/TerekhovTDB14.pdf}{TerekhovTDB14}~\cite{TerekhovTDB14}, \href{../works/ElhouraniDM07.pdf}{ElhouraniDM07}~\cite{ElhouraniDM07}, \href{../works/ZeballosH05.pdf}{ZeballosH05}~\cite{ZeballosH05}\\
\index{cmax}\index{Concepts!cmax}cmax &  1.00 & \href{../works/Fatemi-AnarakiTFV23.pdf}{Fatemi-AnarakiTFV23}~\cite{Fatemi-AnarakiTFV23}, \href{../works/KameugneFND23.pdf}{KameugneFND23}~\cite{KameugneFND23}, \href{../works/ZhuSZW23.pdf}{ZhuSZW23}~\cite{ZhuSZW23}, \href{../works/JuvinHHL23.pdf}{JuvinHHL23}~\cite{JuvinHHL23}, \href{../works/YuraszeckMC23.pdf}{YuraszeckMC23}~\cite{YuraszeckMC23}, \href{../works/YuraszeckMCCR23.pdf}{YuraszeckMCCR23}~\cite{YuraszeckMCCR23}, \href{../works/NaderiRR23.pdf}{NaderiRR23}~\cite{NaderiRR23}, \href{../works/AbreuNP23.pdf}{AbreuNP23}~\cite{AbreuNP23}, \href{../works/abs-2305-19888.pdf}{abs-2305-19888}~\cite{abs-2305-19888}, \href{../works/IsikYA23.pdf}{IsikYA23}~\cite{IsikYA23}, \href{../works/EtminaniesfahaniGNMS22.pdf}{EtminaniesfahaniGNMS22}~\cite{EtminaniesfahaniGNMS22}, \href{../works/AbreuN22.pdf}{AbreuN22}~\cite{AbreuN22}, \href{../works/YunusogluY22.pdf}{YunusogluY22}~\cite{YunusogluY22}, \href{../works/JuvinHL22.pdf}{JuvinHL22}~\cite{JuvinHL22}, \href{../works/ZhangBB22.pdf}{ZhangBB22}~\cite{ZhangBB22}, \href{../works/FetgoD22.pdf}{FetgoD22}~\cite{FetgoD22}, \href{../works/abs-2211-14492.pdf}{abs-2211-14492}~\cite{abs-2211-14492}, \href{../works/ArmstrongGOS21.pdf}{ArmstrongGOS21}~\cite{ArmstrongGOS21}, \href{../works/Godet21a.pdf}{Godet21a}~\cite{Godet21a}...\href{../works/ArtiguesF07.pdf}{ArtiguesF07}~\cite{ArtiguesF07}, \href{../works/ArtiguesBF04.pdf}{ArtiguesBF04}~\cite{ArtiguesBF04}, \href{../works/Elkhyari03.pdf}{Elkhyari03}~\cite{Elkhyari03}, \href{../works/LorigeonBB02.pdf}{LorigeonBB02}~\cite{LorigeonBB02}, \href{../works/Baptiste02.pdf}{Baptiste02}~\cite{Baptiste02}, \href{../works/BosiM2001.pdf}{BosiM2001}~\cite{BosiM2001}, \href{../works/BruckerK00.pdf}{BruckerK00}~\cite{BruckerK00}, \href{../works/Dorndorf2000.pdf}{Dorndorf2000}~\cite{Dorndorf2000}, \href{../works/JainM99.pdf}{JainM99}~\cite{JainM99}, \href{../works/PapaB98.pdf}{PapaB98}~\cite{PapaB98} (Total: 71) & \href{../works/Mehdizadeh-Somarin23.pdf}{Mehdizadeh-Somarin23}~\cite{Mehdizadeh-Somarin23}, \href{../works/MullerMKP22.pdf}{MullerMKP22}~\cite{MullerMKP22}, \href{../works/ArmstrongGOS22.pdf}{ArmstrongGOS22}~\cite{ArmstrongGOS22}, \href{../works/BoudreaultSLQ22.pdf}{BoudreaultSLQ22}~\cite{BoudreaultSLQ22}, \href{../works/AbreuAPNM21.pdf}{AbreuAPNM21}~\cite{AbreuAPNM21}, \href{../works/HamPK21.pdf}{HamPK21}~\cite{HamPK21}, \href{../works/ArkhipovBL19.pdf}{ArkhipovBL19}~\cite{ArkhipovBL19}, \href{../works/Novas19.pdf}{Novas19}~\cite{Novas19}, \href{../works/ParkUJR19.pdf}{ParkUJR19}~\cite{ParkUJR19}, \href{../works/ArbaouiY18.pdf}{ArbaouiY18}~\cite{ArbaouiY18}, \href{../works/ZhouGL15.pdf}{ZhouGL15}~\cite{ZhouGL15}, \href{../works/GrimesH15.pdf}{GrimesH15}~\cite{GrimesH15}, \href{../works/WangMD15.pdf}{WangMD15}~\cite{WangMD15}, \href{../works/MeskensDL13.pdf}{MeskensDL13}~\cite{MeskensDL13}, \href{../works/MenciaSV13.pdf}{MenciaSV13}~\cite{MenciaSV13}, \href{../works/ZhangLS12.pdf}{ZhangLS12}~\cite{ZhangLS12}, \href{../works/TopalogluSS12.pdf}{TopalogluSS12}~\cite{TopalogluSS12}, \href{../works/MenciaSV12.pdf}{MenciaSV12}~\cite{MenciaSV12}, \href{../works/BeckFW11.pdf}{BeckFW11}~\cite{BeckFW11}, \href{../works/MeskensDHG11.pdf}{MeskensDHG11}~\cite{MeskensDHG11}, \href{../works/BartakSR10.pdf}{BartakSR10}~\cite{BartakSR10}, \href{../works/TanSD10.pdf}{TanSD10}~\cite{TanSD10}, \href{../works/OzturkTHO10.pdf}{OzturkTHO10}~\cite{OzturkTHO10}, \href{../works/MoffittPP05.pdf}{MoffittPP05}~\cite{MoffittPP05}, \href{../works/Laborie05.pdf}{Laborie05}~\cite{Laborie05}, \href{../works/Muscettola02.pdf}{Muscettola02}~\cite{Muscettola02}, \href{../works/SourdN00.pdf}{SourdN00}~\cite{SourdN00}, \href{../works/ArtiguesR00.pdf}{ArtiguesR00}~\cite{ArtiguesR00} & \href{../works/JuvinHL23.pdf}{JuvinHL23}~\cite{JuvinHL23}, \href{../works/Teppan22.pdf}{Teppan22}~\cite{Teppan22}, \href{../works/AwadMDMT22.pdf}{AwadMDMT22}~\cite{AwadMDMT22}, \href{../works/MengLZB21.pdf}{MengLZB21}~\cite{MengLZB21}, \href{../works/HanenKP21.pdf}{HanenKP21}~\cite{HanenKP21}, \href{../works/HubnerGSV21.pdf}{HubnerGSV21}~\cite{HubnerGSV21}, \href{../works/ZhangYW21.pdf}{ZhangYW21}~\cite{ZhangYW21}, \href{../works/HamP21.pdf}{HamP21}~\cite{HamP21}, \href{../works/ZarandiASC20.pdf}{ZarandiASC20}~\cite{ZarandiASC20}, \href{../works/CarlierPSJ20.pdf}{CarlierPSJ20}~\cite{CarlierPSJ20}, \href{../works/SenderovichBB19.pdf}{SenderovichBB19}~\cite{SenderovichBB19}, \href{../works/GokgurHO18.pdf}{GokgurHO18}~\cite{GokgurHO18}, \href{../works/LiuCGM17.pdf}{LiuCGM17}~\cite{LiuCGM17}, \href{../works/BofillCSV17.pdf}{BofillCSV17}~\cite{BofillCSV17}, \href{../works/OrnekO16.pdf}{OrnekO16}~\cite{OrnekO16}, \href{../works/SialaAH15.pdf}{SialaAH15}~\cite{SialaAH15}, \href{../works/SchnellH15.pdf}{SchnellH15}~\cite{SchnellH15}, \href{../works/KoschB14.pdf}{KoschB14}~\cite{KoschB14}, \href{../works/LombardiMB13.pdf}{LombardiMB13}~\cite{LombardiMB13}...\href{../works/WatsonB08.pdf}{WatsonB08}~\cite{WatsonB08}, \href{../works/AkkerDH07.pdf}{AkkerDH07}~\cite{AkkerDH07}, \href{../works/KeriK07.pdf}{KeriK07}~\cite{KeriK07}, \href{../works/KhayatLR06.pdf}{KhayatLR06}~\cite{KhayatLR06}, \href{../works/MeyerE04.pdf}{MeyerE04}~\cite{MeyerE04}, \href{../works/Laborie03.pdf}{Laborie03}~\cite{Laborie03}, \href{../works/TrentesauxPT01.pdf}{TrentesauxPT01}~\cite{TrentesauxPT01}, \href{../works/BaptisteP00.pdf}{BaptisteP00}~\cite{BaptisteP00}, \href{../works/FocacciLN00.pdf}{FocacciLN00}~\cite{FocacciLN00}, \href{../works/BaptistePN99.pdf}{BaptistePN99}~\cite{BaptistePN99} (Total: 39)\\
\index{completion-time}\index{Concepts!completion-time}completion-time &  1.00 & \href{../works/BonninMNE24.pdf}{BonninMNE24}~\cite{BonninMNE24}, \href{../works/PrataAN23.pdf}{PrataAN23}~\cite{PrataAN23}, \href{../works/LuZZYW24.pdf}{LuZZYW24}~\cite{LuZZYW24}, \href{../works/AbreuNP23.pdf}{AbreuNP23}~\cite{AbreuNP23}, \href{../works/Mehdizadeh-Somarin23.pdf}{Mehdizadeh-Somarin23}~\cite{Mehdizadeh-Somarin23}, \href{../works/ZhuSZW23.pdf}{ZhuSZW23}~\cite{ZhuSZW23}, \href{../works/Fatemi-AnarakiTFV23.pdf}{Fatemi-AnarakiTFV23}~\cite{Fatemi-AnarakiTFV23}, \href{../works/AlfieriGPS23.pdf}{AlfieriGPS23}~\cite{AlfieriGPS23}, \href{../works/AbreuPNF23.pdf}{AbreuPNF23}~\cite{AbreuPNF23}, \href{../works/KameugneFND23.pdf}{KameugneFND23}~\cite{KameugneFND23}, \href{../works/JuvinHL23.pdf}{JuvinHL23}~\cite{JuvinHL23}, \href{../works/NaderiRR23.pdf}{NaderiRR23}~\cite{NaderiRR23}, \href{../works/PenzDN23.pdf}{PenzDN23}~\cite{PenzDN23}, \href{../works/NaderiBZ23.pdf}{NaderiBZ23}~\cite{NaderiBZ23}, \href{../works/JuvinHL22.pdf}{JuvinHL22}~\cite{JuvinHL22}, \href{../works/AbreuN22.pdf}{AbreuN22}~\cite{AbreuN22}, \href{../works/YunusogluY22.pdf}{YunusogluY22}~\cite{YunusogluY22}, \href{../works/SubulanC22.pdf}{SubulanC22}~\cite{SubulanC22}, \href{../works/AwadMDMT22.pdf}{AwadMDMT22}~\cite{AwadMDMT22}...\href{../works/KanetAG04.pdf}{KanetAG04}~\cite{KanetAG04}, \href{../works/Wolf03.pdf}{Wolf03}~\cite{Wolf03}, \href{../works/Baptiste02.pdf}{Baptiste02}~\cite{Baptiste02}, \href{../works/SourdN00.pdf}{SourdN00}~\cite{SourdN00}, \href{../works/SchildW00.pdf}{SchildW00}~\cite{SchildW00}, \href{../works/Dorndorf2000.pdf}{Dorndorf2000}~\cite{Dorndorf2000}, \href{../works/ArtiguesR00.pdf}{ArtiguesR00}~\cite{ArtiguesR00}, \href{../works/NuijtenP98.pdf}{NuijtenP98}~\cite{NuijtenP98}, \href{../works/NuijtenA96.pdf}{NuijtenA96}~\cite{NuijtenA96}, \href{../works/NuijtenA94.pdf}{NuijtenA94}~\cite{NuijtenA94} (Total: 105) & \href{../works/GokPTGO23.pdf}{GokPTGO23}~\cite{GokPTGO23}, \href{../works/IklassovMR023.pdf}{IklassovMR023}~\cite{IklassovMR023}, \href{../works/NaderiBZR23.pdf}{NaderiBZR23}~\cite{NaderiBZR23}, \href{../works/AfsarVPG23.pdf}{AfsarVPG23}~\cite{AfsarVPG23}, \href{../works/CzerniachowskaWZ23.pdf}{CzerniachowskaWZ23}~\cite{CzerniachowskaWZ23}, \href{../works/abs-2305-19888.pdf}{abs-2305-19888}~\cite{abs-2305-19888}, \href{../works/ColT22.pdf}{ColT22}~\cite{ColT22}, \href{../works/NaderiBZ22a.pdf}{NaderiBZ22a}~\cite{NaderiBZ22a}, \href{../works/TouatBT22.pdf}{TouatBT22}~\cite{TouatBT22}, \href{../works/HeinzNVH22.pdf}{HeinzNVH22}~\cite{HeinzNVH22}, \href{../works/LiFJZLL22.pdf}{LiFJZLL22}~\cite{LiFJZLL22}, \href{../works/ZhangBB22.pdf}{ZhangBB22}~\cite{ZhangBB22}, \href{../works/abs-2211-14492.pdf}{abs-2211-14492}~\cite{abs-2211-14492}, \href{../works/MullerMKP22.pdf}{MullerMKP22}~\cite{MullerMKP22}, \href{../works/Teppan22.pdf}{Teppan22}~\cite{Teppan22}, \href{../works/OujanaAYB22.pdf}{OujanaAYB22}~\cite{OujanaAYB22}, \href{../works/FanXG21.pdf}{FanXG21}~\cite{FanXG21}, \href{../works/QinWSLS21.pdf}{QinWSLS21}~\cite{QinWSLS21}, \href{../works/HanenKP21.pdf}{HanenKP21}~\cite{HanenKP21}...\href{../works/BartakSR08.pdf}{BartakSR08}~\cite{BartakSR08}, \href{../works/MonetteDD07.pdf}{MonetteDD07}~\cite{MonetteDD07}, \href{../works/VilimBC05.pdf}{VilimBC05}~\cite{VilimBC05}, \href{../works/ArtiguesBF04.pdf}{ArtiguesBF04}~\cite{ArtiguesBF04}, \href{../works/LimRX04.pdf}{LimRX04}~\cite{LimRX04}, \href{../works/VilimBC04.pdf}{VilimBC04}~\cite{VilimBC04}, \href{../works/Vilim04.pdf}{Vilim04}~\cite{Vilim04}, \href{../works/LorigeonBB02.pdf}{LorigeonBB02}~\cite{LorigeonBB02}, \href{../works/Zhou97.pdf}{Zhou97}~\cite{Zhou97}, \href{../works/BlazewiczDP96.pdf}{BlazewiczDP96}~\cite{BlazewiczDP96} (Total: 75) & \href{../works/abs-2402-00459.pdf}{abs-2402-00459}~\cite{abs-2402-00459}, \href{../works/TasselGS23.pdf}{TasselGS23}~\cite{TasselGS23}, \href{../works/MontemanniD23a.pdf}{MontemanniD23a}~\cite{MontemanniD23a}, \href{../works/IsikYA23.pdf}{IsikYA23}~\cite{IsikYA23}, \href{../works/abs-2306-05747.pdf}{abs-2306-05747}~\cite{abs-2306-05747}, \href{../works/AkramNHRSA23.pdf}{AkramNHRSA23}~\cite{AkramNHRSA23}, \href{../works/WessenCSFPM23.pdf}{WessenCSFPM23}~\cite{WessenCSFPM23}, \href{../works/JuvinHHL23.pdf}{JuvinHHL23}~\cite{JuvinHHL23}, \href{../works/Adelgren2023.pdf}{Adelgren2023}~\cite{Adelgren2023}, \href{../works/PerezGSL23.pdf}{PerezGSL23}~\cite{PerezGSL23}, \href{../works/PopovicCGNC22.pdf}{PopovicCGNC22}~\cite{PopovicCGNC22}, \href{../works/CampeauG22.pdf}{CampeauG22}~\cite{CampeauG22}, \href{../works/PohlAK22.pdf}{PohlAK22}~\cite{PohlAK22}, \href{../works/WinterMMW22.pdf}{WinterMMW22}~\cite{WinterMMW22}, \href{../works/FarsiTM22.pdf}{FarsiTM22}~\cite{FarsiTM22}, \href{../works/GhandehariK22.pdf}{GhandehariK22}~\cite{GhandehariK22}, \href{../works/CilKLO22.pdf}{CilKLO22}~\cite{CilKLO22}, \href{../works/GeitzGSSW22.pdf}{GeitzGSSW22}~\cite{GeitzGSSW22}, \href{../works/ZhangJZL22.pdf}{ZhangJZL22}~\cite{ZhangJZL22}...\href{../works/HeipckeCCS00.pdf}{HeipckeCCS00}~\cite{HeipckeCCS00}, \href{../works/DraperJCJ99.pdf}{DraperJCJ99}~\cite{DraperJCJ99}, \href{../works/CarlssonKA99.pdf}{CarlssonKA99}~\cite{CarlssonKA99}, \href{../works/JainM99.pdf}{JainM99}~\cite{JainM99}, \href{../works/CestaOF99.pdf}{CestaOF99}~\cite{CestaOF99}, \href{../works/PapaB98.pdf}{PapaB98}~\cite{PapaB98}, \href{../works/PintoG97.pdf}{PintoG97}~\cite{PintoG97}, \href{../works/Zhou96.pdf}{Zhou96}~\cite{Zhou96}, \href{../works/BaptisteP95.pdf}{BaptisteP95}~\cite{BaptisteP95}, \href{../works/BeldiceanuC94.pdf}{BeldiceanuC94}~\cite{BeldiceanuC94} (Total: 144)\\
\index{continuous-process}\index{Concepts!continuous-process}continuous-process &  1.00 & \href{../works/HarjunkoskiMBC14.pdf}{HarjunkoskiMBC14}~\cite{HarjunkoskiMBC14} &  & \href{../works/FarsiTM22.pdf}{FarsiTM22}~\cite{FarsiTM22}, \href{../works/Dejemeppe16.pdf}{Dejemeppe16}~\cite{Dejemeppe16}, \href{../works/GaySS14.pdf}{GaySS14}~\cite{GaySS14}, \href{../works/RoePS05.pdf}{RoePS05}~\cite{RoePS05}, \href{../works/MaraveliasCG04.pdf}{MaraveliasCG04}~\cite{MaraveliasCG04}, \href{../works/Bartak02.pdf}{Bartak02}~\cite{Bartak02}, \href{../works/TrentesauxPT01.pdf}{TrentesauxPT01}~\cite{TrentesauxPT01}, \href{../works/SimonisC95.pdf}{SimonisC95}~\cite{SimonisC95}\\
\index{cyclic scheduling}\index{Concepts!cyclic scheduling}cyclic scheduling &  1.00 & \href{../works/WessenCSFPM23.pdf}{WessenCSFPM23}~\cite{WessenCSFPM23}, \href{../works/OzturkTHO15.pdf}{OzturkTHO15}~\cite{OzturkTHO15}, \href{../works/BonfiettiLBM14.pdf}{BonfiettiLBM14}~\cite{BonfiettiLBM14}, \href{../works/HarjunkoskiMBC14.pdf}{HarjunkoskiMBC14}~\cite{HarjunkoskiMBC14}, \href{../works/BonfiettiLM13.pdf}{BonfiettiLM13}~\cite{BonfiettiLM13}, \href{../works/BonfiettiLBM12.pdf}{BonfiettiLBM12}~\cite{BonfiettiLBM12}, \href{../works/LombardiBMB11.pdf}{LombardiBMB11}~\cite{LombardiBMB11}, \href{../works/BonfiettiLBM11.pdf}{BonfiettiLBM11}~\cite{BonfiettiLBM11}, \href{../works/DraperJCJ99.pdf}{DraperJCJ99}~\cite{DraperJCJ99} & \href{../works/Fatemi-AnarakiTFV23.pdf}{Fatemi-AnarakiTFV23}~\cite{Fatemi-AnarakiTFV23}, \href{../works/BonfiettiZLM16.pdf}{BonfiettiZLM16}~\cite{BonfiettiZLM16}, \href{../works/BonfiettiM12.pdf}{BonfiettiM12}~\cite{BonfiettiM12}, \href{../works/HladikCDJ08.pdf}{HladikCDJ08}~\cite{HladikCDJ08}, \href{../works/KorbaaYG99.pdf}{KorbaaYG99}~\cite{KorbaaYG99}, \href{../works/RodosekW98.pdf}{RodosekW98}~\cite{RodosekW98} & \href{../works/AlakaP23.pdf}{AlakaP23}~\cite{AlakaP23}, \href{../works/GhandehariK22.pdf}{GhandehariK22}~\cite{GhandehariK22}, \href{../works/YuraszeckMPV22.pdf}{YuraszeckMPV22}~\cite{YuraszeckMPV22}, \href{../works/Alaka21.pdf}{Alaka21}~\cite{Alaka21}, \href{../works/WallaceY20.pdf}{WallaceY20}~\cite{WallaceY20}, \href{../works/MengZRZL20.pdf}{MengZRZL20}~\cite{MengZRZL20}, \href{../works/AlakaPY19.pdf}{AlakaPY19}~\cite{AlakaPY19}, \href{../works/MusliuSS18.pdf}{MusliuSS18}~\cite{MusliuSS18}, \href{../works/FrankDT16.pdf}{FrankDT16}~\cite{FrankDT16}, \href{../works/OzturkTHO13.pdf}{OzturkTHO13}~\cite{OzturkTHO13}, \href{../works/MeskensDL13.pdf}{MeskensDL13}~\cite{MeskensDL13}, \href{../works/OzturkTHO12.pdf}{OzturkTHO12}~\cite{OzturkTHO12}, \href{../works/Menana11.pdf}{Menana11}~\cite{Menana11}, \href{../works/Malik08.pdf}{Malik08}~\cite{Malik08}, \href{../works/Wallace06.pdf}{Wallace06}~\cite{Wallace06}, \href{../works/Mason01.pdf}{Mason01}~\cite{Mason01}\\
\index{distributed}\index{Concepts!distributed}distributed &  1.00 & \href{../works/PrataAN23.pdf}{PrataAN23}~\cite{PrataAN23}, \href{../works/GuoZ23.pdf}{GuoZ23}~\cite{GuoZ23}, \href{../works/NaderiRR23.pdf}{NaderiRR23}~\cite{NaderiRR23}, \href{../works/MengGRZSC22.pdf}{MengGRZSC22}~\cite{MengGRZSC22}, \href{../works/Zahout21.pdf}{Zahout21}~\cite{Zahout21}, \href{../works/ZarandiASC20.pdf}{ZarandiASC20}~\cite{ZarandiASC20}, \href{../works/MengZRZL20.pdf}{MengZRZL20}~\cite{MengZRZL20}, \href{../works/He0GLW18.pdf}{He0GLW18}~\cite{He0GLW18}, \href{../works/GombolayWS18.pdf}{GombolayWS18}~\cite{GombolayWS18}, \href{../works/TranPZLDB18.pdf}{TranPZLDB18}~\cite{TranPZLDB18}, \href{../works/RoshanaeiLAU17.pdf}{RoshanaeiLAU17}~\cite{RoshanaeiLAU17}, \href{../works/BridiBLMB16.pdf}{BridiBLMB16}~\cite{BridiBLMB16}, \href{../works/BridiLBBM16.pdf}{BridiLBBM16}~\cite{BridiLBBM16}, \href{../works/ZhouGL15.pdf}{ZhouGL15}~\cite{ZhouGL15}, \href{../works/AlesioBNG15.pdf}{AlesioBNG15}~\cite{AlesioBNG15}, \href{../works/TerekhovTDB14.pdf}{TerekhovTDB14}~\cite{TerekhovTDB14}, \href{../works/BonfiettiLM14.pdf}{BonfiettiLM14}~\cite{BonfiettiLM14}, \href{../works/BartakS11.pdf}{BartakS11}~\cite{BartakS11}, \href{../works/LombardiMRB10.pdf}{LombardiMRB10}~\cite{LombardiMRB10}...\href{../works/RossiTHP07.pdf}{RossiTHP07}~\cite{RossiTHP07}, \href{../works/SureshMOK06.pdf}{SureshMOK06}~\cite{SureshMOK06}, \href{../works/GomesHS06.pdf}{GomesHS06}~\cite{GomesHS06}, \href{../works/Geske05.pdf}{Geske05}~\cite{Geske05}, \href{../works/BeniniBGM05.pdf}{BeniniBGM05}~\cite{BeniniBGM05}, \href{../works/CambazardHDJT04.pdf}{CambazardHDJT04}~\cite{CambazardHDJT04}, \href{../works/BeckW04.pdf}{BeckW04}~\cite{BeckW04}, \href{../works/TrentesauxPT01.pdf}{TrentesauxPT01}~\cite{TrentesauxPT01}, \href{../works/Beck99.pdf}{Beck99}~\cite{Beck99}, \href{../works/LammaMM97.pdf}{LammaMM97}~\cite{LammaMM97} (Total: 38) & \href{../works/AbreuPNF23.pdf}{AbreuPNF23}~\cite{AbreuPNF23}, \href{../works/GokPTGO23.pdf}{GokPTGO23}~\cite{GokPTGO23}, \href{../works/AbreuNP23.pdf}{AbreuNP23}~\cite{AbreuNP23}, \href{../works/ShaikhK23.pdf}{ShaikhK23}~\cite{ShaikhK23}, \href{../works/MarliereSPR23.pdf}{MarliereSPR23}~\cite{MarliereSPR23}, \href{../works/IsikYA23.pdf}{IsikYA23}~\cite{IsikYA23}, \href{../works/NaderiBZ22a.pdf}{NaderiBZ22a}~\cite{NaderiBZ22a}, \href{../works/AbreuN22.pdf}{AbreuN22}~\cite{AbreuN22}, \href{../works/OujanaAYB22.pdf}{OujanaAYB22}~\cite{OujanaAYB22}, \href{../works/YuraszeckMPV22.pdf}{YuraszeckMPV22}~\cite{YuraszeckMPV22}, \href{../works/JungblutK22.pdf}{JungblutK22}~\cite{JungblutK22}, \href{../works/OrnekOS20.pdf}{OrnekOS20}~\cite{OrnekOS20}, \href{../works/ElciOH22.pdf}{ElciOH22}~\cite{ElciOH22}, \href{../works/RoshanaeiN21.pdf}{RoshanaeiN21}~\cite{RoshanaeiN21}, \href{../works/HamP21.pdf}{HamP21}~\cite{HamP21}, \href{../works/Godet21a.pdf}{Godet21a}~\cite{Godet21a}, \href{../works/AbreuAPNM21.pdf}{AbreuAPNM21}~\cite{AbreuAPNM21}, \href{../works/MokhtarzadehTNF20.pdf}{MokhtarzadehTNF20}~\cite{MokhtarzadehTNF20}, \href{../works/GokGSTO20.pdf}{GokGSTO20}~\cite{GokGSTO20}...\href{../works/PolicellaWSO05.pdf}{PolicellaWSO05}~\cite{PolicellaWSO05}, \href{../works/Kuchcinski03.pdf}{Kuchcinski03}~\cite{Kuchcinski03}, \href{../works/ZhuS02.pdf}{ZhuS02}~\cite{ZhuS02}, \href{../works/FukunagaHFAMN02.pdf}{FukunagaHFAMN02}~\cite{FukunagaHFAMN02}, \href{../works/SchildW00.pdf}{SchildW00}~\cite{SchildW00}, \href{../works/ChunCTY99.pdf}{ChunCTY99}~\cite{ChunCTY99}, \href{../works/JainM99.pdf}{JainM99}~\cite{JainM99}, \href{../works/BeckDDF98.pdf}{BeckDDF98}~\cite{BeckDDF98}, \href{../works/Wallace96.pdf}{Wallace96}~\cite{Wallace96}, \href{../works/Pape94.pdf}{Pape94}~\cite{Pape94} (Total: 55) & \href{../works/LiLZDZW24.pdf}{LiLZDZW24}~\cite{LiLZDZW24}, \href{../works/LuZZYW24.pdf}{LuZZYW24}~\cite{LuZZYW24}, \href{../works/ForbesHJST24.pdf}{ForbesHJST24}~\cite{ForbesHJST24}, \href{../works/NaderiBZR23.pdf}{NaderiBZR23}~\cite{NaderiBZR23}, \href{../works/Adelgren2023.pdf}{Adelgren2023}~\cite{Adelgren2023}, \href{../works/abs-2305-19888.pdf}{abs-2305-19888}~\cite{abs-2305-19888}, \href{../works/SquillaciPR23.pdf}{SquillaciPR23}~\cite{SquillaciPR23}, \href{../works/Fatemi-AnarakiTFV23.pdf}{Fatemi-AnarakiTFV23}~\cite{Fatemi-AnarakiTFV23}, \href{../works/YuraszeckMC23.pdf}{YuraszeckMC23}~\cite{YuraszeckMC23}, \href{../works/KimCMLLP23.pdf}{KimCMLLP23}~\cite{KimCMLLP23}, \href{../works/JuvinHL23a.pdf}{JuvinHL23a}~\cite{JuvinHL23a}, \href{../works/NaderiBZ23.pdf}{NaderiBZ23}~\cite{NaderiBZ23}, \href{../works/Bit-Monnot23.pdf}{Bit-Monnot23}~\cite{Bit-Monnot23}, \href{../works/MontemanniD23.pdf}{MontemanniD23}~\cite{MontemanniD23}, \href{../works/ZhuSZW23.pdf}{ZhuSZW23}~\cite{ZhuSZW23}, \href{../works/AlfieriGPS23.pdf}{AlfieriGPS23}~\cite{AlfieriGPS23}, \href{../works/FrimodigECM23.pdf}{FrimodigECM23}~\cite{FrimodigECM23}, \href{../works/WessenCSFPM23.pdf}{WessenCSFPM23}~\cite{WessenCSFPM23}, \href{../works/IklassovMR023.pdf}{IklassovMR023}~\cite{IklassovMR023}...\href{../works/BeckF98.pdf}{BeckF98}~\cite{BeckF98}, \href{../works/MorgadoM97.pdf}{MorgadoM97}~\cite{MorgadoM97}, \href{../works/BeckDSF97.pdf}{BeckDSF97}~\cite{BeckDSF97}, \href{../works/RoweJCA96.pdf}{RoweJCA96}~\cite{RoweJCA96}, \href{../works/BlazewiczDP96.pdf}{BlazewiczDP96}~\cite{BlazewiczDP96}, \href{../works/BrusoniCLMMT96.pdf}{BrusoniCLMMT96}~\cite{BrusoniCLMMT96}, \href{../works/Simonis95a.pdf}{Simonis95a}~\cite{Simonis95a}, \href{../works/CrawfordB94.pdf}{CrawfordB94}~\cite{CrawfordB94}, \href{../works/Nuijten94.pdf}{Nuijten94}~\cite{Nuijten94}, \href{../works/DincbasSH90.pdf}{DincbasSH90}~\cite{DincbasSH90} (Total: 174)\\
\index{due-date}\index{Concepts!due-date}due-date &  1.00 & \href{../works/NaderiBZ23.pdf}{NaderiBZ23}~\cite{NaderiBZ23}, \href{../works/AfsarVPG23.pdf}{AfsarVPG23}~\cite{AfsarVPG23}, \href{../works/AwadMDMT22.pdf}{AwadMDMT22}~\cite{AwadMDMT22}, \href{../works/OujanaAYB22.pdf}{OujanaAYB22}~\cite{OujanaAYB22}, \href{../works/ColT22.pdf}{ColT22}~\cite{ColT22}, \href{../works/NaderiBZ22.pdf}{NaderiBZ22}~\cite{NaderiBZ22}, \href{../works/RoshanaeiN21.pdf}{RoshanaeiN21}~\cite{RoshanaeiN21}, \href{../works/Groleaz21.pdf}{Groleaz21}~\cite{Groleaz21}, \href{../works/AntuoriHHEN21.pdf}{AntuoriHHEN21}~\cite{AntuoriHHEN21}, \href{../works/FanXG21.pdf}{FanXG21}~\cite{FanXG21}, \href{../works/ZarandiASC20.pdf}{ZarandiASC20}~\cite{ZarandiASC20}, \href{../works/TangB20.pdf}{TangB20}~\cite{TangB20}, \href{../works/Mercier-AubinGQ20.pdf}{Mercier-AubinGQ20}~\cite{Mercier-AubinGQ20}, \href{../works/AntunesABD20.pdf}{AntunesABD20}~\cite{AntunesABD20}, \href{../works/AntuoriHHEN20.pdf}{AntuoriHHEN20}~\cite{AntuoriHHEN20}, \href{../works/HauderBRPA20.pdf}{HauderBRPA20}~\cite{HauderBRPA20}, \href{../works/Lunardi20.pdf}{Lunardi20}~\cite{Lunardi20}, \href{../works/HoundjiSW19.pdf}{HoundjiSW19}~\cite{HoundjiSW19}, \href{../works/Novas19.pdf}{Novas19}~\cite{Novas19}...\href{../works/Beck99.pdf}{Beck99}~\cite{Beck99}, \href{../works/BeckDDF98.pdf}{BeckDDF98}~\cite{BeckDDF98}, \href{../works/PapaB98.pdf}{PapaB98}~\cite{PapaB98}, \href{../works/Zhou97.pdf}{Zhou97}~\cite{Zhou97}, \href{../works/PintoG97.pdf}{PintoG97}~\cite{PintoG97}, \href{../works/Zhou96.pdf}{Zhou96}~\cite{Zhou96}, \href{../works/BlazewiczDP96.pdf}{BlazewiczDP96}~\cite{BlazewiczDP96}, \href{../works/Colombani96.pdf}{Colombani96}~\cite{Colombani96}, \href{../works/SadehF96.pdf}{SadehF96}~\cite{SadehF96}, \href{../works/FoxS90.pdf}{FoxS90}~\cite{FoxS90} (Total: 75) & \href{../works/PrataAN23.pdf}{PrataAN23}~\cite{PrataAN23}, \href{../works/IsikYA23.pdf}{IsikYA23}~\cite{IsikYA23}, \href{../works/LacknerMMWW23.pdf}{LacknerMMWW23}~\cite{LacknerMMWW23}, \href{../works/NaderiRR23.pdf}{NaderiRR23}~\cite{NaderiRR23}, \href{../works/abs-2211-14492.pdf}{abs-2211-14492}~\cite{abs-2211-14492}, \href{../works/YunusogluY22.pdf}{YunusogluY22}~\cite{YunusogluY22}, \href{../works/WinterMMW22.pdf}{WinterMMW22}~\cite{WinterMMW22}, \href{../works/Godet21a.pdf}{Godet21a}~\cite{Godet21a}, \href{../works/LacknerMMWW21.pdf}{LacknerMMWW21}~\cite{LacknerMMWW21}, \href{../works/GeibingerMM21.pdf}{GeibingerMM21}~\cite{GeibingerMM21}, \href{../works/GroleazNS20a.pdf}{GroleazNS20a}~\cite{GroleazNS20a}, \href{../works/GeibingerMM19.pdf}{GeibingerMM19}~\cite{GeibingerMM19}, \href{../works/AntunesABD18.pdf}{AntunesABD18}~\cite{AntunesABD18}, \href{../works/FahimiOQ18.pdf}{FahimiOQ18}~\cite{FahimiOQ18}, \href{../works/CatusseCBL16.pdf}{CatusseCBL16}~\cite{CatusseCBL16}, \href{../works/ZarandiKS16.pdf}{ZarandiKS16}~\cite{ZarandiKS16}, \href{../works/RiiseML16.pdf}{RiiseML16}~\cite{RiiseML16}, \href{../works/GrimesH15.pdf}{GrimesH15}~\cite{GrimesH15}, \href{../works/GrimesIOS14.pdf}{GrimesIOS14}~\cite{GrimesIOS14}...\href{../works/BeckF00.pdf}{BeckF00}~\cite{BeckF00}, \href{../works/ArtiguesR00.pdf}{ArtiguesR00}~\cite{ArtiguesR00}, \href{../works/Junker00.pdf}{Junker00}~\cite{Junker00}, \href{../works/JainM99.pdf}{JainM99}~\cite{JainM99}, \href{../works/CarlssonKA99.pdf}{CarlssonKA99}~\cite{CarlssonKA99}, \href{../works/BeckF98.pdf}{BeckF98}~\cite{BeckF98}, \href{../works/BelhadjiI98.pdf}{BelhadjiI98}~\cite{BelhadjiI98}, \href{../works/BeckDF97.pdf}{BeckDF97}~\cite{BeckDF97}, \href{../works/OddiS97.pdf}{OddiS97}~\cite{OddiS97}, \href{../works/BrusoniCLMMT96.pdf}{BrusoniCLMMT96}~\cite{BrusoniCLMMT96} (Total: 55) & \href{../works/abs-2402-00459.pdf}{abs-2402-00459}~\cite{abs-2402-00459}, \href{../works/AbreuPNF23.pdf}{AbreuPNF23}~\cite{AbreuPNF23}, \href{../works/KimCMLLP23.pdf}{KimCMLLP23}~\cite{KimCMLLP23}, \href{../works/YuraszeckMC23.pdf}{YuraszeckMC23}~\cite{YuraszeckMC23}, \href{../works/JuvinHHL23.pdf}{JuvinHHL23}~\cite{JuvinHHL23}, \href{../works/TouatBT22.pdf}{TouatBT22}~\cite{TouatBT22}, \href{../works/ElciOH22.pdf}{ElciOH22}~\cite{ElciOH22}, \href{../works/SubulanC22.pdf}{SubulanC22}~\cite{SubulanC22}, \href{../works/MullerMKP22.pdf}{MullerMKP22}~\cite{MullerMKP22}, \href{../works/YuraszeckMPV22.pdf}{YuraszeckMPV22}~\cite{YuraszeckMPV22}, \href{../works/ZhangJZL22.pdf}{ZhangJZL22}~\cite{ZhangJZL22}, \href{../works/HanenKP21.pdf}{HanenKP21}~\cite{HanenKP21}, \href{../works/Astrand21.pdf}{Astrand21}~\cite{Astrand21}, \href{../works/VlkHT21.pdf}{VlkHT21}~\cite{VlkHT21}, \href{../works/KlankeBYE21.pdf}{KlankeBYE21}~\cite{KlankeBYE21}, \href{../works/Bedhief21.pdf}{Bedhief21}~\cite{Bedhief21}, \href{../works/KovacsTKSG21.pdf}{KovacsTKSG21}~\cite{KovacsTKSG21}, \href{../works/HubnerGSV21.pdf}{HubnerGSV21}~\cite{HubnerGSV21}, \href{../works/Zahout21.pdf}{Zahout21}~\cite{Zahout21}...\href{../works/BeckDSF97.pdf}{BeckDSF97}~\cite{BeckDSF97}, \href{../works/BeckDSF97a.pdf}{BeckDSF97a}~\cite{BeckDSF97a}, \href{../works/BaptisteP95.pdf}{BaptisteP95}~\cite{BaptisteP95}, \href{../works/SimonisC95.pdf}{SimonisC95}~\cite{SimonisC95}, \href{../works/Simonis95a.pdf}{Simonis95a}~\cite{Simonis95a}, \href{../works/Goltz95.pdf}{Goltz95}~\cite{Goltz95}, \href{../works/Muscettola94.pdf}{Muscettola94}~\cite{Muscettola94}, \href{../works/Pape94.pdf}{Pape94}~\cite{Pape94}, \href{../works/AggounB93.pdf}{AggounB93}~\cite{AggounB93}, \href{../works/SmithC93.pdf}{SmithC93}~\cite{SmithC93} (Total: 110)\\
\index{earliness}\index{Concepts!earliness}earliness &  1.00 & \href{../works/PrataAN23.pdf}{PrataAN23}~\cite{PrataAN23}, \href{../works/KimCMLLP23.pdf}{KimCMLLP23}~\cite{KimCMLLP23}, \href{../works/TouatBT22.pdf}{TouatBT22}~\cite{TouatBT22}, \href{../works/PohlAK22.pdf}{PohlAK22}~\cite{PohlAK22}, \href{../works/Groleaz21.pdf}{Groleaz21}~\cite{Groleaz21}, \href{../works/HauderBRPA20.pdf}{HauderBRPA20}~\cite{HauderBRPA20}, \href{../works/ZarandiASC20.pdf}{ZarandiASC20}~\cite{ZarandiASC20}, \href{../works/abs-1902-09244.pdf}{abs-1902-09244}~\cite{abs-1902-09244}, \href{../works/LaborieRSV18.pdf}{LaborieRSV18}~\cite{LaborieRSV18}, \href{../works/ZarandiKS16.pdf}{ZarandiKS16}~\cite{ZarandiKS16}, \href{../works/Dejemeppe16.pdf}{Dejemeppe16}~\cite{Dejemeppe16}, \href{../works/GrimesH15.pdf}{GrimesH15}~\cite{GrimesH15}, \href{../works/LaborieR14.pdf}{LaborieR14}~\cite{LaborieR14}, \href{../works/LombardiM12.pdf}{LombardiM12}~\cite{LombardiM12}, \href{../works/KelbelH11.pdf}{KelbelH11}~\cite{KelbelH11}, \href{../works/GrimesH11.pdf}{GrimesH11}~\cite{GrimesH11}, \href{../works/ZeballosNH11.pdf}{ZeballosNH11}~\cite{ZeballosNH11}, \href{../works/Laborie09.pdf}{Laborie09}~\cite{Laborie09}, \href{../works/MonetteDH09.pdf}{MonetteDH09}~\cite{MonetteDH09}, \href{../works/KeriK07.pdf}{KeriK07}~\cite{KeriK07}, \href{../works/BeckR03.pdf}{BeckR03}~\cite{BeckR03}, \href{../works/DannaP03.pdf}{DannaP03}~\cite{DannaP03}, \href{../works/PintoG97.pdf}{PintoG97}~\cite{PintoG97} & \href{../works/FarsiTM22.pdf}{FarsiTM22}~\cite{FarsiTM22}, \href{../works/AntunesABD20.pdf}{AntunesABD20}~\cite{AntunesABD20}, \href{../works/MengZRZL20.pdf}{MengZRZL20}~\cite{MengZRZL20}, \href{../works/TerekhovDOB12.pdf}{TerekhovDOB12}~\cite{TerekhovDOB12}, \href{../works/KovacsB11.pdf}{KovacsB11}~\cite{KovacsB11}, \href{../works/Davenport10.pdf}{Davenport10}~\cite{Davenport10}, \href{../works/PerronSF04.pdf}{PerronSF04}~\cite{PerronSF04}, \href{../works/Baptiste02.pdf}{Baptiste02}~\cite{Baptiste02} & \href{../works/abs-2402-00459.pdf}{abs-2402-00459}~\cite{abs-2402-00459}, \href{../works/NaderiRR23.pdf}{NaderiRR23}~\cite{NaderiRR23}, \href{../works/PenzDN23.pdf}{PenzDN23}~\cite{PenzDN23}, \href{../works/AlfieriGPS23.pdf}{AlfieriGPS23}~\cite{AlfieriGPS23}, \href{../works/IsikYA23.pdf}{IsikYA23}~\cite{IsikYA23}, \href{../works/AbreuNP23.pdf}{AbreuNP23}~\cite{AbreuNP23}, \href{../works/LacknerMMWW23.pdf}{LacknerMMWW23}~\cite{LacknerMMWW23}, \href{../works/AbreuPNF23.pdf}{AbreuPNF23}~\cite{AbreuPNF23}, \href{../works/EtminaniesfahaniGNMS22.pdf}{EtminaniesfahaniGNMS22}~\cite{EtminaniesfahaniGNMS22}, \href{../works/AwadMDMT22.pdf}{AwadMDMT22}~\cite{AwadMDMT22}, \href{../works/YunusogluY22.pdf}{YunusogluY22}~\cite{YunusogluY22}, \href{../works/GhandehariK22.pdf}{GhandehariK22}~\cite{GhandehariK22}, \href{../works/MengLZB21.pdf}{MengLZB21}~\cite{MengLZB21}, \href{../works/LacknerMMWW21.pdf}{LacknerMMWW21}~\cite{LacknerMMWW21}, \href{../works/FanXG21.pdf}{FanXG21}~\cite{FanXG21}, \href{../works/Mercier-AubinGQ20.pdf}{Mercier-AubinGQ20}~\cite{Mercier-AubinGQ20}, \href{../works/Polo-MejiaALB20.pdf}{Polo-MejiaALB20}~\cite{Polo-MejiaALB20}, \href{../works/ColT19.pdf}{ColT19}~\cite{ColT19}, \href{../works/ColT2019a.pdf}{ColT2019a}~\cite{ColT2019a}...\href{../works/KovacsV06.pdf}{KovacsV06}~\cite{KovacsV06}, \href{../works/GodardLN05.pdf}{GodardLN05}~\cite{GodardLN05}, \href{../works/QuirogaZH05.pdf}{QuirogaZH05}~\cite{QuirogaZH05}, \href{../works/KanetAG04.pdf}{KanetAG04}~\cite{KanetAG04}, \href{../works/BeckPS03.pdf}{BeckPS03}~\cite{BeckPS03}, \href{../works/Bartak02a.pdf}{Bartak02a}~\cite{Bartak02a}, \href{../works/KamarainenS02.pdf}{KamarainenS02}~\cite{KamarainenS02}, \href{../works/Bartak02.pdf}{Bartak02}~\cite{Bartak02}, \href{../works/ArtiguesR00.pdf}{ArtiguesR00}~\cite{ArtiguesR00}, \href{../works/BeckDDF98.pdf}{BeckDDF98}~\cite{BeckDDF98} (Total: 55)\\
\index{energy efficiency}\index{Concepts!energy efficiency}energy efficiency &  1.00 & \href{../works/PrataAN23.pdf}{PrataAN23}~\cite{PrataAN23}, \href{../works/PandeyS21a.pdf}{PandeyS21a}~\cite{PandeyS21a}, \href{../works/LimBTBB15a.pdf}{LimBTBB15a}~\cite{LimBTBB15a}, \href{../works/RuggieroBBMA09.pdf}{RuggieroBBMA09}~\cite{RuggieroBBMA09} & \href{../works/MarliereSPR23.pdf}{MarliereSPR23}~\cite{MarliereSPR23}, \href{../works/Zahout21.pdf}{Zahout21}~\cite{Zahout21}, \href{../works/BenediktMH20.pdf}{BenediktMH20}~\cite{BenediktMH20}, \href{../works/BridiBLMB16.pdf}{BridiBLMB16}~\cite{BridiBLMB16}, \href{../works/Lombardi10.pdf}{Lombardi10}~\cite{Lombardi10} & \href{../works/LuZZYW24.pdf}{LuZZYW24}~\cite{LuZZYW24}, \href{../works/IsikYA23.pdf}{IsikYA23}~\cite{IsikYA23}, \href{../works/AbreuNP23.pdf}{AbreuNP23}~\cite{AbreuNP23}, \href{../works/abs-2211-14492.pdf}{abs-2211-14492}~\cite{abs-2211-14492}, \href{../works/Lemos21.pdf}{Lemos21}~\cite{Lemos21}, \href{../works/MengZRZL20.pdf}{MengZRZL20}~\cite{MengZRZL20}, \href{../works/ZarandiASC20.pdf}{ZarandiASC20}~\cite{ZarandiASC20}, \href{../works/TranPZLDB18.pdf}{TranPZLDB18}~\cite{TranPZLDB18}, \href{../works/NattafAL17.pdf}{NattafAL17}~\cite{NattafAL17}, \href{../works/Dejemeppe16.pdf}{Dejemeppe16}~\cite{Dejemeppe16}, \href{../works/LombardiMB13.pdf}{LombardiMB13}~\cite{LombardiMB13}, \href{../works/LombardiM12.pdf}{LombardiM12}~\cite{LombardiM12}, \href{../works/BeniniLMR11.pdf}{BeniniLMR11}~\cite{BeniniLMR11}\\
\index{explainable}\index{Concepts!explainable}explainable &  1.00 &  &  & \href{../works/LaborieRSV18.pdf}{LaborieRSV18}~\cite{LaborieRSV18}, \href{../works/Schutt11.pdf}{Schutt11}~\cite{Schutt11}, \href{../works/abs-1009-0347.pdf}{abs-1009-0347}~\cite{abs-1009-0347}, \href{../works/EskeyZ90.pdf}{EskeyZ90}~\cite{EskeyZ90}\\
\index{explanation}\index{Concepts!explanation}explanation &  1.00 & \href{../works/Bit-Monnot23.pdf}{Bit-Monnot23}~\cite{Bit-Monnot23}, \href{../works/Godet21a.pdf}{Godet21a}~\cite{Godet21a}, \href{../works/YangSS19.pdf}{YangSS19}~\cite{YangSS19}, \href{../works/KreterSSZ18.pdf}{KreterSSZ18}~\cite{KreterSSZ18}, \href{../works/SchnellH17.pdf}{SchnellH17}~\cite{SchnellH17}, \href{../works/HookerH17.pdf}{HookerH17}~\cite{HookerH17}, \href{../works/KreterSS17.pdf}{KreterSS17}~\cite{KreterSS17}, \href{../works/SchuttS16.pdf}{SchuttS16}~\cite{SchuttS16}, \href{../works/Siala15a.pdf}{Siala15a}~\cite{Siala15a}, \href{../works/SchnellH15.pdf}{SchnellH15}~\cite{SchnellH15}, \href{../works/SialaAH15.pdf}{SialaAH15}~\cite{SialaAH15}, \href{../works/KreterSS15.pdf}{KreterSS15}~\cite{KreterSS15}, \href{../works/SchuttFSW13.pdf}{SchuttFSW13}~\cite{SchuttFSW13}, \href{../works/SchuttFS13a.pdf}{SchuttFS13a}~\cite{SchuttFS13a}, \href{../works/SchuttFS13.pdf}{SchuttFS13}~\cite{SchuttFS13}, \href{../works/HeinzB12.pdf}{HeinzB12}~\cite{HeinzB12}, \href{../works/SchuttCSW12.pdf}{SchuttCSW12}~\cite{SchuttCSW12}, \href{../works/SchuttFSW11.pdf}{SchuttFSW11}~\cite{SchuttFSW11}, \href{../works/HeinzS11.pdf}{HeinzS11}~\cite{HeinzS11}...\href{../works/Hooker05.pdf}{Hooker05}~\cite{Hooker05}, \href{../works/Vilim05.pdf}{Vilim05}~\cite{Vilim05}, \href{../works/CambazardHDJT04.pdf}{CambazardHDJT04}~\cite{CambazardHDJT04}, \href{../works/Elkhyari03.pdf}{Elkhyari03}~\cite{Elkhyari03}, \href{../works/Vilim03.pdf}{Vilim03}~\cite{Vilim03}, \href{../works/ElkhyariGJ02.pdf}{ElkhyariGJ02}~\cite{ElkhyariGJ02}, \href{../works/ElkhyariGJ02a.pdf}{ElkhyariGJ02a}~\cite{ElkhyariGJ02a}, \href{../works/JussienL02.pdf}{JussienL02}~\cite{JussienL02}, \href{../works/BeckF00.pdf}{BeckF00}~\cite{BeckF00}, \href{../works/Beck99.pdf}{Beck99}~\cite{Beck99} (Total: 36) & \href{../works/LuZZYW24.pdf}{LuZZYW24}~\cite{LuZZYW24}, \href{../works/LacknerMMWW23.pdf}{LacknerMMWW23}~\cite{LacknerMMWW23}, \href{../works/Caballero19.pdf}{Caballero19}~\cite{Caballero19}, \href{../works/EmeretlisTAV17.pdf}{EmeretlisTAV17}~\cite{EmeretlisTAV17}, \href{../works/CappartS17.pdf}{CappartS17}~\cite{CappartS17}, \href{../works/Froger16.pdf}{Froger16}~\cite{Froger16}, \href{../works/HeinzKB13.pdf}{HeinzKB13}~\cite{HeinzKB13}, \href{../works/BeniniLMR11.pdf}{BeniniLMR11}~\cite{BeniniLMR11}, \href{../works/BeniniLMMR08.pdf}{BeniniLMMR08}~\cite{BeniniLMMR08}, \href{../works/BeckW07.pdf}{BeckW07}~\cite{BeckW07}, \href{../works/Rodriguez07.pdf}{Rodriguez07}~\cite{Rodriguez07}, \href{../works/BeckW05.pdf}{BeckW05}~\cite{BeckW05}, \href{../works/BosiM2001.pdf}{BosiM2001}~\cite{BosiM2001}, \href{../works/BeckF00a.pdf}{BeckF00a}~\cite{BeckF00a}, \href{../works/BeckF99.pdf}{BeckF99}~\cite{BeckF99}, \href{../works/ChunCTY99.pdf}{ChunCTY99}~\cite{ChunCTY99}, \href{../works/Simonis99.pdf}{Simonis99}~\cite{Simonis99}, \href{../works/MorgadoM97.pdf}{MorgadoM97}~\cite{MorgadoM97}, \href{../works/BeckDF97.pdf}{BeckDF97}~\cite{BeckDF97}, \href{../works/MintonJPL92.pdf}{MintonJPL92}~\cite{MintonJPL92}, \href{../works/MintonJPL90.pdf}{MintonJPL90}~\cite{MintonJPL90} & \href{../works/BonninMNE24.pdf}{BonninMNE24}~\cite{BonninMNE24}, \href{../works/MontemanniD23a.pdf}{MontemanniD23a}~\cite{MontemanniD23a}, \href{../works/EfthymiouY23.pdf}{EfthymiouY23}~\cite{EfthymiouY23}, \href{../works/KameugneFND23.pdf}{KameugneFND23}~\cite{KameugneFND23}, \href{../works/IsikYA23.pdf}{IsikYA23}~\cite{IsikYA23}, \href{../works/abs-2305-19888.pdf}{abs-2305-19888}~\cite{abs-2305-19888}, \href{../works/MarliereSPR23.pdf}{MarliereSPR23}~\cite{MarliereSPR23}, \href{../works/CilKLO22.pdf}{CilKLO22}~\cite{CilKLO22}, \href{../works/FetgoD22.pdf}{FetgoD22}~\cite{FetgoD22}, \href{../works/CampeauG22.pdf}{CampeauG22}~\cite{CampeauG22}, \href{../works/OrnekOS20.pdf}{OrnekOS20}~\cite{OrnekOS20}, \href{../works/BoudreaultSLQ22.pdf}{BoudreaultSLQ22}~\cite{BoudreaultSLQ22}, \href{../works/ColT22.pdf}{ColT22}~\cite{ColT22}, \href{../works/HeinzNVH22.pdf}{HeinzNVH22}~\cite{HeinzNVH22}, \href{../works/ArmstrongGOS22.pdf}{ArmstrongGOS22}~\cite{ArmstrongGOS22}, \href{../works/BenderWS21.pdf}{BenderWS21}~\cite{BenderWS21}, \href{../works/VlkHT21.pdf}{VlkHT21}~\cite{VlkHT21}, \href{../works/Groleaz21.pdf}{Groleaz21}~\cite{Groleaz21}, \href{../works/Lemos21.pdf}{Lemos21}~\cite{Lemos21}...\href{../works/SchildW00.pdf}{SchildW00}~\cite{SchildW00}, \href{../works/JoLLH99.pdf}{JoLLH99}~\cite{JoLLH99}, \href{../works/HookerO99.pdf}{HookerO99}~\cite{HookerO99}, \href{../works/BensanaLV99.pdf}{BensanaLV99}~\cite{BensanaLV99}, \href{../works/BeckF98.pdf}{BeckF98}~\cite{BeckF98}, \href{../works/BelhadjiI98.pdf}{BelhadjiI98}~\cite{BelhadjiI98}, \href{../works/BeckDSF97a.pdf}{BeckDSF97a}~\cite{BeckDSF97a}, \href{../works/Zhou97.pdf}{Zhou97}~\cite{Zhou97}, \href{../works/Wallace96.pdf}{Wallace96}~\cite{Wallace96}, \href{../works/Puget95.pdf}{Puget95}~\cite{Puget95} (Total: 127)\\
\index{flow-shop}\index{Concepts!flow-shop}flow-shop &  1.00 & \href{../works/BonninMNE24.pdf}{BonninMNE24}~\cite{BonninMNE24}, \href{../works/PrataAN23.pdf}{PrataAN23}~\cite{PrataAN23}, \href{../works/LiLZDZW24.pdf}{LiLZDZW24}~\cite{LiLZDZW24}, \href{../works/NaderiRR23.pdf}{NaderiRR23}~\cite{NaderiRR23}, \href{../works/IsikYA23.pdf}{IsikYA23}~\cite{IsikYA23}, \href{../works/AbreuPNF23.pdf}{AbreuPNF23}~\cite{AbreuPNF23}, \href{../works/CzerniachowskaWZ23.pdf}{CzerniachowskaWZ23}~\cite{CzerniachowskaWZ23}, \href{../works/AlfieriGPS23.pdf}{AlfieriGPS23}~\cite{AlfieriGPS23}, \href{../works/AbreuNP23.pdf}{AbreuNP23}~\cite{AbreuNP23}, \href{../works/JuvinHL23.pdf}{JuvinHL23}~\cite{JuvinHL23}, \href{../works/LiFJZLL22.pdf}{LiFJZLL22}~\cite{LiFJZLL22}, \href{../works/ArmstrongGOS22.pdf}{ArmstrongGOS22}~\cite{ArmstrongGOS22}, \href{../works/AbreuN22.pdf}{AbreuN22}~\cite{AbreuN22}, \href{../works/OujanaAYB22.pdf}{OujanaAYB22}~\cite{OujanaAYB22}, \href{../works/ColT22.pdf}{ColT22}~\cite{ColT22}, \href{../works/ZhangJZL22.pdf}{ZhangJZL22}~\cite{ZhangJZL22}, \href{../works/MengLZB21.pdf}{MengLZB21}~\cite{MengLZB21}, \href{../works/Astrand21.pdf}{Astrand21}~\cite{Astrand21}, \href{../works/Bedhief21.pdf}{Bedhief21}~\cite{Bedhief21}...\href{../works/BajestaniB15.pdf}{BajestaniB15}~\cite{BajestaniB15}, \href{../works/ZhouGL15.pdf}{ZhouGL15}~\cite{ZhouGL15}, \href{../works/TerekhovTDB14.pdf}{TerekhovTDB14}~\cite{TerekhovTDB14}, \href{../works/HamdiL13.pdf}{HamdiL13}~\cite{HamdiL13}, \href{../works/Malapert11.pdf}{Malapert11}~\cite{Malapert11}, \href{../works/Demassey03.pdf}{Demassey03}~\cite{Demassey03}, \href{../works/LorigeonBB02.pdf}{LorigeonBB02}~\cite{LorigeonBB02}, \href{../works/Baptiste02.pdf}{Baptiste02}~\cite{Baptiste02}, \href{../works/SourdN00.pdf}{SourdN00}~\cite{SourdN00}, \href{../works/WatsonBHW99.pdf}{WatsonBHW99}~\cite{WatsonBHW99} (Total: 42) & \href{../works/Mehdizadeh-Somarin23.pdf}{Mehdizadeh-Somarin23}~\cite{Mehdizadeh-Somarin23}, \href{../works/JuvinHL23a.pdf}{JuvinHL23a}~\cite{JuvinHL23a}, \href{../works/NaderiBZ23.pdf}{NaderiBZ23}~\cite{NaderiBZ23}, \href{../works/YuraszeckMPV22.pdf}{YuraszeckMPV22}~\cite{YuraszeckMPV22}, \href{../works/NaderiBZ22.pdf}{NaderiBZ22}~\cite{NaderiBZ22}, \href{../works/MengGRZSC22.pdf}{MengGRZSC22}~\cite{MengGRZSC22}, \href{../works/JuvinHL22.pdf}{JuvinHL22}~\cite{JuvinHL22}, \href{../works/KoehlerBFFHPSSS21.pdf}{KoehlerBFFHPSSS21}~\cite{KoehlerBFFHPSSS21}, \href{../works/FanXG21.pdf}{FanXG21}~\cite{FanXG21}, \href{../works/Godet21a.pdf}{Godet21a}~\cite{Godet21a}, \href{../works/TangB20.pdf}{TangB20}~\cite{TangB20}, \href{../works/HauderBRPA20.pdf}{HauderBRPA20}~\cite{HauderBRPA20}, \href{../works/abs-1902-09244.pdf}{abs-1902-09244}~\cite{abs-1902-09244}, \href{../works/TanZWGQ19.pdf}{TanZWGQ19}~\cite{TanZWGQ19}, \href{../works/GombolayWS18.pdf}{GombolayWS18}~\cite{GombolayWS18}, \href{../works/LaborieRSV18.pdf}{LaborieRSV18}~\cite{LaborieRSV18}, \href{../works/Fahimi16.pdf}{Fahimi16}~\cite{Fahimi16}, \href{../works/Dejemeppe16.pdf}{Dejemeppe16}~\cite{Dejemeppe16}, \href{../works/GuyonLPR12.pdf}{GuyonLPR12}~\cite{GuyonLPR12}, \href{../works/GrimesH11.pdf}{GrimesH11}~\cite{GrimesH11}, \href{../works/KovacsB11.pdf}{KovacsB11}~\cite{KovacsB11}, \href{../works/ZeballosCM10.pdf}{ZeballosCM10}~\cite{ZeballosCM10}, \href{../works/BartakSR10.pdf}{BartakSR10}~\cite{BartakSR10}, \href{../works/BartakSR08.pdf}{BartakSR08}~\cite{BartakSR08}, \href{../works/RoePS05.pdf}{RoePS05}~\cite{RoePS05}, \href{../works/JainM99.pdf}{JainM99}~\cite{JainM99}, \href{../works/BaptistePN99.pdf}{BaptistePN99}~\cite{BaptistePN99}, \href{../works/AggounB93.pdf}{AggounB93}~\cite{AggounB93} & \href{../works/LuZZYW24.pdf}{LuZZYW24}~\cite{LuZZYW24}, \href{../works/TasselGS23.pdf}{TasselGS23}~\cite{TasselGS23}, \href{../works/abs-2305-19888.pdf}{abs-2305-19888}~\cite{abs-2305-19888}, \href{../works/YuraszeckMCCR23.pdf}{YuraszeckMCCR23}~\cite{YuraszeckMCCR23}, \href{../works/JuvinHHL23.pdf}{JuvinHHL23}~\cite{JuvinHHL23}, \href{../works/AfsarVPG23.pdf}{AfsarVPG23}~\cite{AfsarVPG23}, \href{../works/WessenCSFPM23.pdf}{WessenCSFPM23}~\cite{WessenCSFPM23}, \href{../works/AalianPG23.pdf}{AalianPG23}~\cite{AalianPG23}, \href{../works/abs-2306-05747.pdf}{abs-2306-05747}~\cite{abs-2306-05747}, \href{../works/abs-2211-14492.pdf}{abs-2211-14492}~\cite{abs-2211-14492}, \href{../works/TouatBT22.pdf}{TouatBT22}~\cite{TouatBT22}, \href{../works/Teppan22.pdf}{Teppan22}~\cite{Teppan22}, \href{../works/NaderiBZ22a.pdf}{NaderiBZ22a}~\cite{NaderiBZ22a}, \href{../works/HeinzNVH22.pdf}{HeinzNVH22}~\cite{HeinzNVH22}, \href{../works/AwadMDMT22.pdf}{AwadMDMT22}~\cite{AwadMDMT22}, \href{../works/LacknerMMWW21.pdf}{LacknerMMWW21}~\cite{LacknerMMWW21}, \href{../works/abs-2102-08778.pdf}{abs-2102-08778}~\cite{abs-2102-08778}, \href{../works/Zahout21.pdf}{Zahout21}~\cite{Zahout21}, \href{../works/KovacsTKSG21.pdf}{KovacsTKSG21}~\cite{KovacsTKSG21}...\href{../works/SchildW00.pdf}{SchildW00}~\cite{SchildW00}, \href{../works/BaptisteP00.pdf}{BaptisteP00}~\cite{BaptisteP00}, \href{../works/KorbaaYG99.pdf}{KorbaaYG99}~\cite{KorbaaYG99}, \href{../works/PapaB98.pdf}{PapaB98}~\cite{PapaB98}, \href{../works/NuijtenP98.pdf}{NuijtenP98}~\cite{NuijtenP98}, \href{../works/GetoorOFC97.pdf}{GetoorOFC97}~\cite{GetoorOFC97}, \href{../works/PintoG97.pdf}{PintoG97}~\cite{PintoG97}, \href{../works/BaptisteP97.pdf}{BaptisteP97}~\cite{BaptisteP97}, \href{../works/NuijtenA96.pdf}{NuijtenA96}~\cite{NuijtenA96}, \href{../works/SimonisC95.pdf}{SimonisC95}~\cite{SimonisC95} (Total: 73)\\
\index{flow-time}\index{Concepts!flow-time}flow-time &  1.00 & \href{../works/BonninMNE24.pdf}{BonninMNE24}~\cite{BonninMNE24}, \href{../works/PenzDN23.pdf}{PenzDN23}~\cite{PenzDN23}, \href{../works/EmdeZD22.pdf}{EmdeZD22}~\cite{EmdeZD22}, \href{../works/YuraszeckMPV22.pdf}{YuraszeckMPV22}~\cite{YuraszeckMPV22}, \href{../works/FanXG21.pdf}{FanXG21}~\cite{FanXG21}, \href{../works/NattafM20.pdf}{NattafM20}~\cite{NattafM20}, \href{../works/ZarandiASC20.pdf}{ZarandiASC20}~\cite{ZarandiASC20}, \href{../works/MalapertN19.pdf}{MalapertN19}~\cite{MalapertN19}, \href{../works/ZhangW18.pdf}{ZhangW18}~\cite{ZhangW18}, \href{../works/TerekhovTDB14.pdf}{TerekhovTDB14}~\cite{TerekhovTDB14}, \href{../works/TranTDB13.pdf}{TranTDB13}~\cite{TranTDB13}, \href{../works/WuBB09.pdf}{WuBB09}~\cite{WuBB09}, \href{../works/Baptiste02.pdf}{Baptiste02}~\cite{Baptiste02} & \href{../works/PrataAN23.pdf}{PrataAN23}~\cite{PrataAN23}, \href{../works/AlfieriGPS23.pdf}{AlfieriGPS23}~\cite{AlfieriGPS23}, \href{../works/YunusogluY22.pdf}{YunusogluY22}~\cite{YunusogluY22}, \href{../works/Malapert11.pdf}{Malapert11}~\cite{Malapert11}, \href{../works/BeckW07.pdf}{BeckW07}~\cite{BeckW07} & \href{../works/YuraszeckMCCR23.pdf}{YuraszeckMCCR23}~\cite{YuraszeckMCCR23}, \href{../works/TasselGS23.pdf}{TasselGS23}~\cite{TasselGS23}, \href{../works/abs-2306-05747.pdf}{abs-2306-05747}~\cite{abs-2306-05747}, \href{../works/IklassovMR023.pdf}{IklassovMR023}~\cite{IklassovMR023}, \href{../works/YuraszeckMC23.pdf}{YuraszeckMC23}~\cite{YuraszeckMC23}, \href{../works/LiFJZLL22.pdf}{LiFJZLL22}~\cite{LiFJZLL22}, \href{../works/AbreuN22.pdf}{AbreuN22}~\cite{AbreuN22}, \href{../works/KoehlerBFFHPSSS21.pdf}{KoehlerBFFHPSSS21}~\cite{KoehlerBFFHPSSS21}, \href{../works/MengZRZL20.pdf}{MengZRZL20}~\cite{MengZRZL20}, \href{../works/Novas19.pdf}{Novas19}~\cite{Novas19}, \href{../works/ParkUJR19.pdf}{ParkUJR19}~\cite{ParkUJR19}, \href{../works/BajestaniB15.pdf}{BajestaniB15}~\cite{BajestaniB15}, \href{../works/MenciaSV13.pdf}{MenciaSV13}~\cite{MenciaSV13}, \href{../works/MenciaSV12.pdf}{MenciaSV12}~\cite{MenciaSV12}, \href{../works/KovacsB11.pdf}{KovacsB11}~\cite{KovacsB11}, \href{../works/EdisO11.pdf}{EdisO11}~\cite{EdisO11}, \href{../works/Salido10.pdf}{Salido10}~\cite{Salido10}, \href{../works/QuirogaZH05.pdf}{QuirogaZH05}~\cite{QuirogaZH05}, \href{../works/BeckPS03.pdf}{BeckPS03}~\cite{BeckPS03}, \href{../works/BeckR03.pdf}{BeckR03}~\cite{BeckR03}, \href{../works/BeckDDF98.pdf}{BeckDDF98}~\cite{BeckDDF98}, \href{../works/FoxS90.pdf}{FoxS90}~\cite{FoxS90}\\
\index{inventory}\index{Concepts!inventory}inventory &  1.00 & \href{../works/LuZZYW24.pdf}{LuZZYW24}~\cite{LuZZYW24}, \href{../works/GuoZ23.pdf}{GuoZ23}~\cite{GuoZ23}, \href{../works/SubulanC22.pdf}{SubulanC22}~\cite{SubulanC22}, \href{../works/Astrand21.pdf}{Astrand21}~\cite{Astrand21}, \href{../works/German18.pdf}{German18}~\cite{German18}, \href{../works/GilesH16.pdf}{GilesH16}~\cite{GilesH16}, \href{../works/GoelSHFS15.pdf}{GoelSHFS15}~\cite{GoelSHFS15}, \href{../works/HarjunkoskiMBC14.pdf}{HarjunkoskiMBC14}~\cite{HarjunkoskiMBC14}, \href{../works/SerraNM12.pdf}{SerraNM12}~\cite{SerraNM12}, \href{../works/TerekhovDOB12.pdf}{TerekhovDOB12}~\cite{TerekhovDOB12}, \href{../works/LopesCSM10.pdf}{LopesCSM10}~\cite{LopesCSM10}, \href{../works/Jans09.pdf}{Jans09}~\cite{Jans09}, \href{../works/RossiTHP07.pdf}{RossiTHP07}~\cite{RossiTHP07}, \href{../works/RoePS05.pdf}{RoePS05}~\cite{RoePS05}, \href{../works/Timpe02.pdf}{Timpe02}~\cite{Timpe02}, \href{../works/Beck99.pdf}{Beck99}~\cite{Beck99}, \href{../works/BeckDDF98.pdf}{BeckDDF98}~\cite{BeckDDF98}, \href{../works/BeckDSF97.pdf}{BeckDSF97}~\cite{BeckDSF97}, \href{../works/MurphyRFSS97.pdf}{MurphyRFSS97}~\cite{MurphyRFSS97}, \href{../works/BeckDF97.pdf}{BeckDF97}~\cite{BeckDF97} & \href{../works/Adelgren2023.pdf}{Adelgren2023}~\cite{Adelgren2023}, \href{../works/EmdeZD22.pdf}{EmdeZD22}~\cite{EmdeZD22}, \href{../works/ZarandiASC20.pdf}{ZarandiASC20}~\cite{ZarandiASC20}, \href{../works/Ham20a.pdf}{Ham20a}~\cite{Ham20a}, \href{../works/Hooker19.pdf}{Hooker19}~\cite{Hooker19}, \href{../works/Novas19.pdf}{Novas19}~\cite{Novas19}, \href{../works/Ham18a.pdf}{Ham18a}~\cite{Ham18a}, \href{../works/HamFC17.pdf}{HamFC17}~\cite{HamFC17}, \href{../works/BajestaniB13.pdf}{BajestaniB13}~\cite{BajestaniB13}, \href{../works/MakMS10.pdf}{MakMS10}~\cite{MakMS10}, \href{../works/MouraSCL08a.pdf}{MouraSCL08a}~\cite{MouraSCL08a}, \href{../works/LauLN08.pdf}{LauLN08}~\cite{LauLN08}, \href{../works/DavenportKRSH07.pdf}{DavenportKRSH07}~\cite{DavenportKRSH07}, \href{../works/GarganiR07.pdf}{GarganiR07}~\cite{GarganiR07}, \href{../works/MaraveliasCG04.pdf}{MaraveliasCG04}~\cite{MaraveliasCG04}, \href{../works/BeckF00.pdf}{BeckF00}~\cite{BeckF00}, \href{../works/Simonis99.pdf}{Simonis99}~\cite{Simonis99}, \href{../works/BlazewiczDP96.pdf}{BlazewiczDP96}~\cite{BlazewiczDP96}, \href{../works/Simonis95a.pdf}{Simonis95a}~\cite{Simonis95a}, \href{../works/Hamscher91.pdf}{Hamscher91}~\cite{Hamscher91} & \href{../works/PrataAN23.pdf}{PrataAN23}~\cite{PrataAN23}, \href{../works/PerezGSL23.pdf}{PerezGSL23}~\cite{PerezGSL23}, \href{../works/abs-2312-13682.pdf}{abs-2312-13682}~\cite{abs-2312-13682}, \href{../works/AlfieriGPS23.pdf}{AlfieriGPS23}~\cite{AlfieriGPS23}, \href{../works/ZhuSZW23.pdf}{ZhuSZW23}~\cite{ZhuSZW23}, \href{../works/GokPTGO23.pdf}{GokPTGO23}~\cite{GokPTGO23}, \href{../works/GurPAE23.pdf}{GurPAE23}~\cite{GurPAE23}, \href{../works/AwadMDMT22.pdf}{AwadMDMT22}~\cite{AwadMDMT22}, \href{../works/PohlAK22.pdf}{PohlAK22}~\cite{PohlAK22}, \href{../works/YunusogluY22.pdf}{YunusogluY22}~\cite{YunusogluY22}, \href{../works/AbreuN22.pdf}{AbreuN22}~\cite{AbreuN22}, \href{../works/Groleaz21.pdf}{Groleaz21}~\cite{Groleaz21}, \href{../works/KovacsTKSG21.pdf}{KovacsTKSG21}~\cite{KovacsTKSG21}, \href{../works/HubnerGSV21.pdf}{HubnerGSV21}~\cite{HubnerGSV21}, \href{../works/RoshanaeiN21.pdf}{RoshanaeiN21}~\cite{RoshanaeiN21}, \href{../works/GroleazNS20.pdf}{GroleazNS20}~\cite{GroleazNS20}, \href{../works/FachiniA20.pdf}{FachiniA20}~\cite{FachiniA20}, \href{../works/HauderBRPA20.pdf}{HauderBRPA20}~\cite{HauderBRPA20}, \href{../works/GroleazNS20a.pdf}{GroleazNS20a}~\cite{GroleazNS20a}...\href{../works/Laborie03.pdf}{Laborie03}~\cite{Laborie03}, \href{../works/BeckR03.pdf}{BeckR03}~\cite{BeckR03}, \href{../works/Baptiste02.pdf}{Baptiste02}~\cite{Baptiste02}, \href{../works/TrentesauxPT01.pdf}{TrentesauxPT01}~\cite{TrentesauxPT01}, \href{../works/PesantGPR99.pdf}{PesantGPR99}~\cite{PesantGPR99}, \href{../works/JainM99.pdf}{JainM99}~\cite{JainM99}, \href{../works/BeckF98.pdf}{BeckF98}~\cite{BeckF98}, \href{../works/SadehF96.pdf}{SadehF96}~\cite{SadehF96}, \href{../works/SimonisC95.pdf}{SimonisC95}~\cite{SimonisC95}, \href{../works/Pape94.pdf}{Pape94}~\cite{Pape94} (Total: 68)\\
\index{job-shop}\index{Concepts!job-shop}job-shop &  1.00 & \href{../works/abs-2402-00459.pdf}{abs-2402-00459}~\cite{abs-2402-00459}, \href{../works/PrataAN23.pdf}{PrataAN23}~\cite{PrataAN23}, \href{../works/TasselGS23.pdf}{TasselGS23}~\cite{TasselGS23}, \href{../works/AbreuNP23.pdf}{AbreuNP23}~\cite{AbreuNP23}, \href{../works/Mehdizadeh-Somarin23.pdf}{Mehdizadeh-Somarin23}~\cite{Mehdizadeh-Somarin23}, \href{../works/Fatemi-AnarakiTFV23.pdf}{Fatemi-AnarakiTFV23}~\cite{Fatemi-AnarakiTFV23}, \href{../works/ZhuSZW23.pdf}{ZhuSZW23}~\cite{ZhuSZW23}, \href{../works/KimCMLLP23.pdf}{KimCMLLP23}~\cite{KimCMLLP23}, \href{../works/CzerniachowskaWZ23.pdf}{CzerniachowskaWZ23}~\cite{CzerniachowskaWZ23}, \href{../works/YuraszeckMCCR23.pdf}{YuraszeckMCCR23}~\cite{YuraszeckMCCR23}, \href{../works/abs-2306-05747.pdf}{abs-2306-05747}~\cite{abs-2306-05747}, \href{../works/JuvinHL23a.pdf}{JuvinHL23a}~\cite{JuvinHL23a}, \href{../works/IklassovMR023.pdf}{IklassovMR023}~\cite{IklassovMR023}, \href{../works/JuvinHHL23.pdf}{JuvinHHL23}~\cite{JuvinHHL23}, \href{../works/AfsarVPG23.pdf}{AfsarVPG23}~\cite{AfsarVPG23}, \href{../works/Bit-Monnot23.pdf}{Bit-Monnot23}~\cite{Bit-Monnot23}, \href{../works/NaderiRR23.pdf}{NaderiRR23}~\cite{NaderiRR23}, \href{../works/KotaryFH22.pdf}{KotaryFH22}~\cite{KotaryFH22}, \href{../works/NaderiBZ22a.pdf}{NaderiBZ22a}~\cite{NaderiBZ22a}...\href{../works/NuijtenA96.pdf}{NuijtenA96}~\cite{NuijtenA96}, \href{../works/BlazewiczDP96.pdf}{BlazewiczDP96}~\cite{BlazewiczDP96}, \href{../works/SadehF96.pdf}{SadehF96}~\cite{SadehF96}, \href{../works/BaptisteP95.pdf}{BaptisteP95}~\cite{BaptisteP95}, \href{../works/Goltz95.pdf}{Goltz95}~\cite{Goltz95}, \href{../works/NuijtenA94.pdf}{NuijtenA94}~\cite{NuijtenA94}, \href{../works/Nuijten94.pdf}{Nuijten94}~\cite{Nuijten94}, \href{../works/AggounB93.pdf}{AggounB93}~\cite{AggounB93}, \href{../works/SmithC93.pdf}{SmithC93}~\cite{SmithC93}, \href{../works/FoxS90.pdf}{FoxS90}~\cite{FoxS90} (Total: 167) & \href{../works/AbreuPNF23.pdf}{AbreuPNF23}~\cite{AbreuPNF23}, \href{../works/PenzDN23.pdf}{PenzDN23}~\cite{PenzDN23}, \href{../works/IsikYA23.pdf}{IsikYA23}~\cite{IsikYA23}, \href{../works/NaderiBZ23.pdf}{NaderiBZ23}~\cite{NaderiBZ23}, \href{../works/EfthymiouY23.pdf}{EfthymiouY23}~\cite{EfthymiouY23}, \href{../works/AlfieriGPS23.pdf}{AlfieriGPS23}~\cite{AlfieriGPS23}, \href{../works/YunusogluY22.pdf}{YunusogluY22}~\cite{YunusogluY22}, \href{../works/LuoB22.pdf}{LuoB22}~\cite{LuoB22}, \href{../works/NaderiBZ22.pdf}{NaderiBZ22}~\cite{NaderiBZ22}, \href{../works/EtminaniesfahaniGNMS22.pdf}{EtminaniesfahaniGNMS22}~\cite{EtminaniesfahaniGNMS22}, \href{../works/TouatBT22.pdf}{TouatBT22}~\cite{TouatBT22}, \href{../works/AbreuN22.pdf}{AbreuN22}~\cite{AbreuN22}, \href{../works/QinWSLS21.pdf}{QinWSLS21}~\cite{QinWSLS21}, \href{../works/ArmstrongGOS21.pdf}{ArmstrongGOS21}~\cite{ArmstrongGOS21}, \href{../works/RoshanaeiN21.pdf}{RoshanaeiN21}~\cite{RoshanaeiN21}, \href{../works/KoehlerBFFHPSSS21.pdf}{KoehlerBFFHPSSS21}~\cite{KoehlerBFFHPSSS21}, \href{../works/Godet21a.pdf}{Godet21a}~\cite{Godet21a}, \href{../works/Astrand0F21.pdf}{Astrand0F21}~\cite{Astrand0F21}, \href{../works/GroleazNS20.pdf}{GroleazNS20}~\cite{GroleazNS20}...\href{../works/CestaOF99.pdf}{CestaOF99}~\cite{CestaOF99}, \href{../works/PembertonG98.pdf}{PembertonG98}~\cite{PembertonG98}, \href{../works/MorgadoM97.pdf}{MorgadoM97}~\cite{MorgadoM97}, \href{../works/BaptisteP97.pdf}{BaptisteP97}~\cite{BaptisteP97}, \href{../works/Caseau97.pdf}{Caseau97}~\cite{Caseau97}, \href{../works/SimonisC95.pdf}{SimonisC95}~\cite{SimonisC95}, \href{../works/Puget95.pdf}{Puget95}~\cite{Puget95}, \href{../works/Pape94.pdf}{Pape94}~\cite{Pape94}, \href{../works/CrawfordB94.pdf}{CrawfordB94}~\cite{CrawfordB94}, \href{../works/Muscettola94.pdf}{Muscettola94}~\cite{Muscettola94} (Total: 72) & \href{../works/ForbesHJST24.pdf}{ForbesHJST24}~\cite{ForbesHJST24}, \href{../works/LuZZYW24.pdf}{LuZZYW24}~\cite{LuZZYW24}, \href{../works/BonninMNE24.pdf}{BonninMNE24}~\cite{BonninMNE24}, \href{../works/PovedaAA23.pdf}{PovedaAA23}~\cite{PovedaAA23}, \href{../works/MarliereSPR23.pdf}{MarliereSPR23}~\cite{MarliereSPR23}, \href{../works/GuoZ23.pdf}{GuoZ23}~\cite{GuoZ23}, \href{../works/LacknerMMWW23.pdf}{LacknerMMWW23}~\cite{LacknerMMWW23}, \href{../works/JuvinHL23.pdf}{JuvinHL23}~\cite{JuvinHL23}, \href{../works/Adelgren2023.pdf}{Adelgren2023}~\cite{Adelgren2023}, \href{../works/WessenCSFPM23.pdf}{WessenCSFPM23}~\cite{WessenCSFPM23}, \href{../works/ShaikhK23.pdf}{ShaikhK23}~\cite{ShaikhK23}, \href{../works/GokPTGO23.pdf}{GokPTGO23}~\cite{GokPTGO23}, \href{../works/YuraszeckMC23.pdf}{YuraszeckMC23}~\cite{YuraszeckMC23}, \href{../works/Tassel22.pdf}{Tassel22}~\cite{Tassel22}, \href{../works/GhandehariK22.pdf}{GhandehariK22}~\cite{GhandehariK22}, \href{../works/EmdeZD22.pdf}{EmdeZD22}~\cite{EmdeZD22}, \href{../works/AntuoriHHEN21.pdf}{AntuoriHHEN21}~\cite{AntuoriHHEN21}, \href{../works/Zahout21.pdf}{Zahout21}~\cite{Zahout21}, \href{../works/HanenKP21.pdf}{HanenKP21}~\cite{HanenKP21}...\href{../works/JoLLH99.pdf}{JoLLH99}~\cite{JoLLH99}, \href{../works/WatsonBHW99.pdf}{WatsonBHW99}~\cite{WatsonBHW99}, \href{../works/GetoorOFC97.pdf}{GetoorOFC97}~\cite{GetoorOFC97}, \href{../works/LammaMM97.pdf}{LammaMM97}~\cite{LammaMM97}, \href{../works/BrusoniCLMMT96.pdf}{BrusoniCLMMT96}~\cite{BrusoniCLMMT96}, \href{../works/Wallace96.pdf}{Wallace96}~\cite{Wallace96}, \href{../works/WeilHFP95.pdf}{WeilHFP95}~\cite{WeilHFP95}, \href{../works/MintonJPL92.pdf}{MintonJPL92}~\cite{MintonJPL92}, \href{../works/DincbasSH90.pdf}{DincbasSH90}~\cite{DincbasSH90}, \href{../works/EskeyZ90.pdf}{EskeyZ90}~\cite{EskeyZ90} (Total: 139)\\
\index{lateness}\index{Concepts!lateness}lateness &  1.00 & \href{../works/AwadMDMT22.pdf}{AwadMDMT22}~\cite{AwadMDMT22}, \href{../works/Groleaz21.pdf}{Groleaz21}~\cite{Groleaz21}, \href{../works/FahimiOQ18.pdf}{FahimiOQ18}~\cite{FahimiOQ18}, \href{../works/Dejemeppe16.pdf}{Dejemeppe16}~\cite{Dejemeppe16}, \href{../works/Fahimi16.pdf}{Fahimi16}~\cite{Fahimi16}, \href{../works/KoschB14.pdf}{KoschB14}~\cite{KoschB14}, \href{../works/ZampelliVSDR13.pdf}{ZampelliVSDR13}~\cite{ZampelliVSDR13}, \href{../works/MalapertGR12.pdf}{MalapertGR12}~\cite{MalapertGR12}, \href{../works/Malapert11.pdf}{Malapert11}~\cite{Malapert11}, \href{../works/BartakSR10.pdf}{BartakSR10}~\cite{BartakSR10}, \href{../works/Geske05.pdf}{Geske05}~\cite{Geske05}, \href{../works/Baptiste02.pdf}{Baptiste02}~\cite{Baptiste02}, \href{../works/ArtiguesR00.pdf}{ArtiguesR00}~\cite{ArtiguesR00}, \href{../works/BlazewiczDP96.pdf}{BlazewiczDP96}~\cite{BlazewiczDP96} & \href{../works/PrataAN23.pdf}{PrataAN23}~\cite{PrataAN23}, \href{../works/PohlAK22.pdf}{PohlAK22}~\cite{PohlAK22}, \href{../works/ZarandiASC20.pdf}{ZarandiASC20}~\cite{ZarandiASC20}, \href{../works/AntunesABD20.pdf}{AntunesABD20}~\cite{AntunesABD20}, \href{../works/ZhangW18.pdf}{ZhangW18}~\cite{ZhangW18}, \href{../works/HarjunkoskiMBC14.pdf}{HarjunkoskiMBC14}~\cite{HarjunkoskiMBC14}, \href{../works/MilanoW09.pdf}{MilanoW09}~\cite{MilanoW09}, \href{../works/AkkerDH07.pdf}{AkkerDH07}~\cite{AkkerDH07}, \href{../works/MilanoW06.pdf}{MilanoW06}~\cite{MilanoW06}, \href{../works/RoePS05.pdf}{RoePS05}~\cite{RoePS05}, \href{../works/Sadykov04.pdf}{Sadykov04}~\cite{Sadykov04}, \href{../works/FoxS90.pdf}{FoxS90}~\cite{FoxS90} & \href{../works/LacknerMMWW23.pdf}{LacknerMMWW23}~\cite{LacknerMMWW23}, \href{../works/NaderiBZ23.pdf}{NaderiBZ23}~\cite{NaderiBZ23}, \href{../works/YunusogluY22.pdf}{YunusogluY22}~\cite{YunusogluY22}, \href{../works/GhandehariK22.pdf}{GhandehariK22}~\cite{GhandehariK22}, \href{../works/ZhangBB22.pdf}{ZhangBB22}~\cite{ZhangBB22}, \href{../works/NaderiBZ22.pdf}{NaderiBZ22}~\cite{NaderiBZ22}, \href{../works/GeitzGSSW22.pdf}{GeitzGSSW22}~\cite{GeitzGSSW22}, \href{../works/ColT22.pdf}{ColT22}~\cite{ColT22}, \href{../works/LacknerMMWW21.pdf}{LacknerMMWW21}~\cite{LacknerMMWW21}, \href{../works/Godet21a.pdf}{Godet21a}~\cite{Godet21a}, \href{../works/HanenKP21.pdf}{HanenKP21}~\cite{HanenKP21}, \href{../works/KoehlerBFFHPSSS21.pdf}{KoehlerBFFHPSSS21}~\cite{KoehlerBFFHPSSS21}, \href{../works/QinWSLS21.pdf}{QinWSLS21}~\cite{QinWSLS21}, \href{../works/Lunardi20.pdf}{Lunardi20}~\cite{Lunardi20}, \href{../works/Novas19.pdf}{Novas19}~\cite{Novas19}, \href{../works/ParkUJR19.pdf}{ParkUJR19}~\cite{ParkUJR19}, \href{../works/ArkhipovBL19.pdf}{ArkhipovBL19}~\cite{ArkhipovBL19}, \href{../works/AntunesABD18.pdf}{AntunesABD18}~\cite{AntunesABD18}, \href{../works/Tesch18.pdf}{Tesch18}~\cite{Tesch18}...\href{../works/TerekhovDOB12.pdf}{TerekhovDOB12}~\cite{TerekhovDOB12}, \href{../works/EdisO11.pdf}{EdisO11}~\cite{EdisO11}, \href{../works/ZeballosNH11.pdf}{ZeballosNH11}~\cite{ZeballosNH11}, \href{../works/ChenGPSH10.pdf}{ChenGPSH10}~\cite{ChenGPSH10}, \href{../works/NovasH10.pdf}{NovasH10}~\cite{NovasH10}, \href{../works/WuBB09.pdf}{WuBB09}~\cite{WuBB09}, \href{../works/BartakSR08.pdf}{BartakSR08}~\cite{BartakSR08}, \href{../works/SadykovW06.pdf}{SadykovW06}~\cite{SadykovW06}, \href{../works/Bartak02.pdf}{Bartak02}~\cite{Bartak02}, \href{../works/JainM99.pdf}{JainM99}~\cite{JainM99} (Total: 34)\\
\index{make to order}\index{Concepts!make to order}make to order &  1.00 &  &  & \href{../works/OujanaAYB22.pdf}{OujanaAYB22}~\cite{OujanaAYB22}, \href{../works/Simonis07.pdf}{Simonis07}~\cite{Simonis07}, \href{../works/DavenportKRSH07.pdf}{DavenportKRSH07}~\cite{DavenportKRSH07}, \href{../works/HookerO99.pdf}{HookerO99}~\cite{HookerO99}\\
\index{make to stock}\index{Concepts!make to stock}make to stock &  1.00 &  &  & \href{../works/HarjunkoskiMBC14.pdf}{HarjunkoskiMBC14}~\cite{HarjunkoskiMBC14}\\
\index{make-span}\index{Concepts!make-span}make-span &  1.00 & \href{../works/PrataAN23.pdf}{PrataAN23}~\cite{PrataAN23}, \href{../works/LiLZDZW24.pdf}{LiLZDZW24}~\cite{LiLZDZW24}, \href{../works/Mehdizadeh-Somarin23.pdf}{Mehdizadeh-Somarin23}~\cite{Mehdizadeh-Somarin23}, \href{../works/IklassovMR023.pdf}{IklassovMR023}~\cite{IklassovMR023}, \href{../works/EfthymiouY23.pdf}{EfthymiouY23}~\cite{EfthymiouY23}, \href{../works/PovedaAA23.pdf}{PovedaAA23}~\cite{PovedaAA23}, \href{../works/JuvinHL23a.pdf}{JuvinHL23a}~\cite{JuvinHL23a}, \href{../works/abs-2306-05747.pdf}{abs-2306-05747}~\cite{abs-2306-05747}, \href{../works/CzerniachowskaWZ23.pdf}{CzerniachowskaWZ23}~\cite{CzerniachowskaWZ23}, \href{../works/JuvinHHL23.pdf}{JuvinHHL23}~\cite{JuvinHHL23}, \href{../works/AlfieriGPS23.pdf}{AlfieriGPS23}~\cite{AlfieriGPS23}, \href{../works/abs-2305-19888.pdf}{abs-2305-19888}~\cite{abs-2305-19888}, \href{../works/Bit-Monnot23.pdf}{Bit-Monnot23}~\cite{Bit-Monnot23}, \href{../works/AbreuNP23.pdf}{AbreuNP23}~\cite{AbreuNP23}, \href{../works/AfsarVPG23.pdf}{AfsarVPG23}~\cite{AfsarVPG23}, \href{../works/AalianPG23.pdf}{AalianPG23}~\cite{AalianPG23}, \href{../works/AbreuPNF23.pdf}{AbreuPNF23}~\cite{AbreuPNF23}, \href{../works/YuraszeckMC23.pdf}{YuraszeckMC23}~\cite{YuraszeckMC23}, \href{../works/ZhuSZW23.pdf}{ZhuSZW23}~\cite{ZhuSZW23}...\href{../works/PapaB98.pdf}{PapaB98}~\cite{PapaB98}, \href{../works/Darby-DowmanLMZ97.pdf}{Darby-DowmanLMZ97}~\cite{Darby-DowmanLMZ97}, \href{../works/BeckDF97.pdf}{BeckDF97}~\cite{BeckDF97}, \href{../works/BeckDSF97a.pdf}{BeckDSF97a}~\cite{BeckDSF97a}, \href{../works/PapeB97.pdf}{PapeB97}~\cite{PapeB97}, \href{../works/GetoorOFC97.pdf}{GetoorOFC97}~\cite{GetoorOFC97}, \href{../works/BaptisteP97.pdf}{BaptisteP97}~\cite{BaptisteP97}, \href{../works/BlazewiczDP96.pdf}{BlazewiczDP96}~\cite{BlazewiczDP96}, \href{../works/NuijtenA96.pdf}{NuijtenA96}~\cite{NuijtenA96}, \href{../works/Nuijten94.pdf}{Nuijten94}~\cite{Nuijten94} (Total: 241) & \href{../works/BonninMNE24.pdf}{BonninMNE24}~\cite{BonninMNE24}, \href{../works/KameugneFND23.pdf}{KameugneFND23}~\cite{KameugneFND23}, \href{../works/YuraszeckMCCR23.pdf}{YuraszeckMCCR23}~\cite{YuraszeckMCCR23}, \href{../works/abs-2312-13682.pdf}{abs-2312-13682}~\cite{abs-2312-13682}, \href{../works/PerezGSL23.pdf}{PerezGSL23}~\cite{PerezGSL23}, \href{../works/PenzDN23.pdf}{PenzDN23}~\cite{PenzDN23}, \href{../works/Adelgren2023.pdf}{Adelgren2023}~\cite{Adelgren2023}, \href{../works/MullerMKP22.pdf}{MullerMKP22}~\cite{MullerMKP22}, \href{../works/SvancaraB22.pdf}{SvancaraB22}~\cite{SvancaraB22}, \href{../works/abs-2211-14492.pdf}{abs-2211-14492}~\cite{abs-2211-14492}, \href{../works/YuraszeckMPV22.pdf}{YuraszeckMPV22}~\cite{YuraszeckMPV22}, \href{../works/ZhangJZL22.pdf}{ZhangJZL22}~\cite{ZhangJZL22}, \href{../works/OujanaAYB22.pdf}{OujanaAYB22}~\cite{OujanaAYB22}, \href{../works/LiFJZLL22.pdf}{LiFJZLL22}~\cite{LiFJZLL22}, \href{../works/PandeyS21a.pdf}{PandeyS21a}~\cite{PandeyS21a}, \href{../works/FanXG21.pdf}{FanXG21}~\cite{FanXG21}, \href{../works/QinDCS20.pdf}{QinDCS20}~\cite{QinDCS20}, \href{../works/NattafDYW19.pdf}{NattafDYW19}~\cite{NattafDYW19}, \href{../works/AstrandJZ18.pdf}{AstrandJZ18}~\cite{AstrandJZ18}...\href{../works/VilimBC04.pdf}{VilimBC04}~\cite{VilimBC04}, \href{../works/KovacsV04.pdf}{KovacsV04}~\cite{KovacsV04}, \href{../works/Wolf03.pdf}{Wolf03}~\cite{Wolf03}, \href{../works/Timpe02.pdf}{Timpe02}~\cite{Timpe02}, \href{../works/BruckerK00.pdf}{BruckerK00}~\cite{BruckerK00}, \href{../works/BeckF00a.pdf}{BeckF00a}~\cite{BeckF00a}, \href{../works/CarlssonKA99.pdf}{CarlssonKA99}~\cite{CarlssonKA99}, \href{../works/BeckF98.pdf}{BeckF98}~\cite{BeckF98}, \href{../works/LeeKLKKYHP97.pdf}{LeeKLKKYHP97}~\cite{LeeKLKKYHP97}, \href{../works/BeckDSF97.pdf}{BeckDSF97}~\cite{BeckDSF97} (Total: 71) & \href{../works/ForbesHJST24.pdf}{ForbesHJST24}~\cite{ForbesHJST24}, \href{../works/AlakaP23.pdf}{AlakaP23}~\cite{AlakaP23}, \href{../works/KimCMLLP23.pdf}{KimCMLLP23}~\cite{KimCMLLP23}, \href{../works/NaderiBZ23.pdf}{NaderiBZ23}~\cite{NaderiBZ23}, \href{../works/GokPTGO23.pdf}{GokPTGO23}~\cite{GokPTGO23}, \href{../works/GuoZ23.pdf}{GuoZ23}~\cite{GuoZ23}, \href{../works/TardivoDFMP23.pdf}{TardivoDFMP23}~\cite{TardivoDFMP23}, \href{../works/Fatemi-AnarakiTFV23.pdf}{Fatemi-AnarakiTFV23}~\cite{Fatemi-AnarakiTFV23}, \href{../works/JungblutK22.pdf}{JungblutK22}~\cite{JungblutK22}, \href{../works/PopovicCGNC22.pdf}{PopovicCGNC22}~\cite{PopovicCGNC22}, \href{../works/FetgoD22.pdf}{FetgoD22}~\cite{FetgoD22}, \href{../works/NaderiBZ22.pdf}{NaderiBZ22}~\cite{NaderiBZ22}, \href{../works/Teppan22.pdf}{Teppan22}~\cite{Teppan22}, \href{../works/CampeauG22.pdf}{CampeauG22}~\cite{CampeauG22}, \href{../works/EmdeZD22.pdf}{EmdeZD22}~\cite{EmdeZD22}, \href{../works/KoehlerBFFHPSSS21.pdf}{KoehlerBFFHPSSS21}~\cite{KoehlerBFFHPSSS21}, \href{../works/HanenKP21.pdf}{HanenKP21}~\cite{HanenKP21}, \href{../works/HamP21.pdf}{HamP21}~\cite{HamP21}, \href{../works/HubnerGSV21.pdf}{HubnerGSV21}~\cite{HubnerGSV21}...\href{../works/Bartak02.pdf}{Bartak02}~\cite{Bartak02}, \href{../works/Bartak02a.pdf}{Bartak02a}~\cite{Bartak02a}, \href{../works/Junker00.pdf}{Junker00}~\cite{Junker00}, \href{../works/HeipckeCCS00.pdf}{HeipckeCCS00}~\cite{HeipckeCCS00}, \href{../works/PesantGPR99.pdf}{PesantGPR99}~\cite{PesantGPR99}, \href{../works/HookerO99.pdf}{HookerO99}~\cite{HookerO99}, \href{../works/BaptistePN99.pdf}{BaptistePN99}~\cite{BaptistePN99}, \href{../works/RodosekW98.pdf}{RodosekW98}~\cite{RodosekW98}, \href{../works/Caseau97.pdf}{Caseau97}~\cite{Caseau97}, \href{../works/OddiS97.pdf}{OddiS97}~\cite{OddiS97} (Total: 126)\\
\index{manpower}\index{Concepts!manpower}manpower &  1.00 & \href{../works/NovaraNH16.pdf}{NovaraNH16}~\cite{NovaraNH16}, \href{../works/PintoG97.pdf}{PintoG97}~\cite{PintoG97} & \href{../works/LaborieRSV18.pdf}{LaborieRSV18}~\cite{LaborieRSV18}, \href{../works/Froger16.pdf}{Froger16}~\cite{Froger16}, \href{../works/ZeballosNH11.pdf}{ZeballosNH11}~\cite{ZeballosNH11} & \href{../works/BourreauGGLT22.pdf}{BourreauGGLT22}~\cite{BourreauGGLT22}, \href{../works/BadicaBI20.pdf}{BadicaBI20}~\cite{BadicaBI20}, \href{../works/MokhtarzadehTNF20.pdf}{MokhtarzadehTNF20}~\cite{MokhtarzadehTNF20}, \href{../works/HauderBRPA20.pdf}{HauderBRPA20}~\cite{HauderBRPA20}, \href{../works/WikarekS19.pdf}{WikarekS19}~\cite{WikarekS19}, \href{../works/BaptisteB18.pdf}{BaptisteB18}~\cite{BaptisteB18}, \href{../works/MusliuSS18.pdf}{MusliuSS18}~\cite{MusliuSS18}, \href{../works/ErkingerM17.pdf}{ErkingerM17}~\cite{ErkingerM17}, \href{../works/SchuttS16.pdf}{SchuttS16}~\cite{SchuttS16}, \href{../works/HechingH16.pdf}{HechingH16}~\cite{HechingH16}, \href{../works/GayHS15a.pdf}{GayHS15a}~\cite{GayHS15a}, \href{../works/GaySS14.pdf}{GaySS14}~\cite{GaySS14}, \href{../works/HarjunkoskiMBC14.pdf}{HarjunkoskiMBC14}~\cite{HarjunkoskiMBC14}, \href{../works/Clercq12.pdf}{Clercq12}~\cite{Clercq12}, \href{../works/GuyonLPR12.pdf}{GuyonLPR12}~\cite{GuyonLPR12}, \href{../works/LombardiM12.pdf}{LombardiM12}~\cite{LombardiM12}, \href{../works/Menana11.pdf}{Menana11}~\cite{Menana11}, \href{../works/Vilim11.pdf}{Vilim11}~\cite{Vilim11}, \href{../works/SimonisH11.pdf}{SimonisH11}~\cite{SimonisH11}, \href{../works/MeskensDHG11.pdf}{MeskensDHG11}~\cite{MeskensDHG11}, \href{../works/ChenGPSH10.pdf}{ChenGPSH10}~\cite{ChenGPSH10}, \href{../works/NovasH10.pdf}{NovasH10}~\cite{NovasH10}, \href{../works/MaraveliasCG04.pdf}{MaraveliasCG04}~\cite{MaraveliasCG04}, \href{../works/Simonis99.pdf}{Simonis99}~\cite{Simonis99}, \href{../works/NuijtenP98.pdf}{NuijtenP98}~\cite{NuijtenP98}, \href{../works/MorgadoM97.pdf}{MorgadoM97}~\cite{MorgadoM97}, \href{../works/LeeKLKKYHP97.pdf}{LeeKLKKYHP97}~\cite{LeeKLKKYHP97}, \href{../works/SimonisC95.pdf}{SimonisC95}~\cite{SimonisC95}, \href{../works/Simonis95a.pdf}{Simonis95a}~\cite{Simonis95a}, \href{../works/Puget95.pdf}{Puget95}~\cite{Puget95}\\
\index{multi-agent}\index{Concepts!multi-agent}multi-agent &  1.00 & \href{../works/SvancaraB22.pdf}{SvancaraB22}~\cite{SvancaraB22}, \href{../works/Zahout21.pdf}{Zahout21}~\cite{Zahout21}, \href{../works/ZarandiASC20.pdf}{ZarandiASC20}~\cite{ZarandiASC20}, \href{../works/BehrensLM19.pdf}{BehrensLM19}~\cite{BehrensLM19}, \href{../works/GombolayWS18.pdf}{GombolayWS18}~\cite{GombolayWS18}, \href{../works/He0GLW18.pdf}{He0GLW18}~\cite{He0GLW18}, \href{../works/HoeveGSL07.pdf}{HoeveGSL07}~\cite{HoeveGSL07}, \href{../works/SultanikMR07.pdf}{SultanikMR07}~\cite{SultanikMR07}, \href{../works/TrentesauxPT01.pdf}{TrentesauxPT01}~\cite{TrentesauxPT01} & \href{../works/Lemos21.pdf}{Lemos21}~\cite{Lemos21}, \href{../works/MokhtarzadehTNF20.pdf}{MokhtarzadehTNF20}~\cite{MokhtarzadehTNF20}, \href{../works/abs-1901-07914.pdf}{abs-1901-07914}~\cite{abs-1901-07914}, \href{../works/TranVNB17.pdf}{TranVNB17}~\cite{TranVNB17}, \href{../works/LimHTB16.pdf}{LimHTB16}~\cite{LimHTB16}, \href{../works/BartakSR10.pdf}{BartakSR10}~\cite{BartakSR10}, \href{../works/BocewiczBB09.pdf}{BocewiczBB09}~\cite{BocewiczBB09}, \href{../works/Hunsberger08.pdf}{Hunsberger08}~\cite{Hunsberger08} & \href{../works/abs-2402-00459.pdf}{abs-2402-00459}~\cite{abs-2402-00459}, \href{../works/Fatemi-AnarakiTFV23.pdf}{Fatemi-AnarakiTFV23}~\cite{Fatemi-AnarakiTFV23}, \href{../works/ZhuSZW23.pdf}{ZhuSZW23}~\cite{ZhuSZW23}, \href{../works/WessenCSFPM23.pdf}{WessenCSFPM23}~\cite{WessenCSFPM23}, \href{../works/IklassovMR023.pdf}{IklassovMR023}~\cite{IklassovMR023}, \href{../works/Mehdizadeh-Somarin23.pdf}{Mehdizadeh-Somarin23}~\cite{Mehdizadeh-Somarin23}, \href{../works/SquillaciPR23.pdf}{SquillaciPR23}~\cite{SquillaciPR23}, \href{../works/GokPTGO23.pdf}{GokPTGO23}~\cite{GokPTGO23}, \href{../works/AbreuAPNM21.pdf}{AbreuAPNM21}~\cite{AbreuAPNM21}, \href{../works/MengLZB21.pdf}{MengLZB21}~\cite{MengLZB21}, \href{../works/ZhangYW21.pdf}{ZhangYW21}~\cite{ZhangYW21}, \href{../works/GokGSTO20.pdf}{GokGSTO20}~\cite{GokGSTO20}, \href{../works/MejiaY20.pdf}{MejiaY20}~\cite{MejiaY20}, \href{../works/Ham20a.pdf}{Ham20a}~\cite{Ham20a}, \href{../works/WessenCS20.pdf}{WessenCS20}~\cite{WessenCS20}, \href{../works/BadicaBIL19.pdf}{BadicaBIL19}~\cite{BadicaBIL19}, \href{../works/WikarekS19.pdf}{WikarekS19}~\cite{WikarekS19}, \href{../works/TanZWGQ19.pdf}{TanZWGQ19}~\cite{TanZWGQ19}, \href{../works/ZhangW18.pdf}{ZhangW18}~\cite{ZhangW18}...\href{../works/ElhouraniDM07.pdf}{ElhouraniDM07}~\cite{ElhouraniDM07}, \href{../works/GomesHS06.pdf}{GomesHS06}~\cite{GomesHS06}, \href{../works/AbrilSB05.pdf}{AbrilSB05}~\cite{AbrilSB05}, \href{../works/DilkinaH04.pdf}{DilkinaH04}~\cite{DilkinaH04}, \href{../works/Beck99.pdf}{Beck99}~\cite{Beck99}, \href{../works/BeckF98.pdf}{BeckF98}~\cite{BeckF98}, \href{../works/BeckDDF98.pdf}{BeckDDF98}~\cite{BeckDDF98}, \href{../works/Wallace96.pdf}{Wallace96}~\cite{Wallace96}, \href{../works/SadehF96.pdf}{SadehF96}~\cite{SadehF96}, \href{../works/Pape94.pdf}{Pape94}~\cite{Pape94} (Total: 38)\\
\index{multi-objective}\index{Concepts!multi-objective}multi-objective &  1.00 & \href{../works/LuZZYW24.pdf}{LuZZYW24}~\cite{LuZZYW24}, \href{../works/IsikYA23.pdf}{IsikYA23}~\cite{IsikYA23}, \href{../works/AfsarVPG23.pdf}{AfsarVPG23}~\cite{AfsarVPG23}, \href{../works/FarsiTM22.pdf}{FarsiTM22}~\cite{FarsiTM22}, \href{../works/MengGRZSC22.pdf}{MengGRZSC22}~\cite{MengGRZSC22}, \href{../works/CilKLO22.pdf}{CilKLO22}~\cite{CilKLO22}, \href{../works/SubulanC22.pdf}{SubulanC22}~\cite{SubulanC22}, \href{../works/YunusogluY22.pdf}{YunusogluY22}~\cite{YunusogluY22}, \href{../works/HamPK21.pdf}{HamPK21}~\cite{HamPK21}, \href{../works/Lemos21.pdf}{Lemos21}~\cite{Lemos21}, \href{../works/Edis21.pdf}{Edis21}~\cite{Edis21}, \href{../works/ZarandiASC20.pdf}{ZarandiASC20}~\cite{ZarandiASC20}, \href{../works/Tom19.pdf}{Tom19}~\cite{Tom19}, \href{../works/TangLWSK18.pdf}{TangLWSK18}~\cite{TangLWSK18}, \href{../works/Dejemeppe16.pdf}{Dejemeppe16}~\cite{Dejemeppe16}, \href{../works/Froger16.pdf}{Froger16}~\cite{Froger16}, \href{../works/MeskensDL13.pdf}{MeskensDL13}~\cite{MeskensDL13}, \href{../works/TopalogluO11.pdf}{TopalogluO11}~\cite{TopalogluO11}, \href{../works/ZeballosQH10.pdf}{ZeballosQH10}~\cite{ZeballosQH10} & \href{../works/PrataAN23.pdf}{PrataAN23}~\cite{PrataAN23}, \href{../works/AbreuPNF23.pdf}{AbreuPNF23}~\cite{AbreuPNF23}, \href{../works/GurPAE23.pdf}{GurPAE23}~\cite{GurPAE23}, \href{../works/CzerniachowskaWZ23.pdf}{CzerniachowskaWZ23}~\cite{CzerniachowskaWZ23}, \href{../works/AlakaP23.pdf}{AlakaP23}~\cite{AlakaP23}, \href{../works/FrimodigECM23.pdf}{FrimodigECM23}~\cite{FrimodigECM23}, \href{../works/LacknerMMWW23.pdf}{LacknerMMWW23}~\cite{LacknerMMWW23}, \href{../works/NaderiBZ22a.pdf}{NaderiBZ22a}~\cite{NaderiBZ22a}, \href{../works/AbreuN22.pdf}{AbreuN22}~\cite{AbreuN22}, \href{../works/ZhangJZL22.pdf}{ZhangJZL22}~\cite{ZhangJZL22}, \href{../works/LiFJZLL22.pdf}{LiFJZLL22}~\cite{LiFJZLL22}, \href{../works/OrnekOS20.pdf}{OrnekOS20}~\cite{OrnekOS20}, \href{../works/AbreuAPNM21.pdf}{AbreuAPNM21}~\cite{AbreuAPNM21}, \href{../works/FanXG21.pdf}{FanXG21}~\cite{FanXG21}, \href{../works/Alaka21.pdf}{Alaka21}~\cite{Alaka21}, \href{../works/QinWSLS21.pdf}{QinWSLS21}~\cite{QinWSLS21}, \href{../works/AbohashimaEG21.pdf}{AbohashimaEG21}~\cite{AbohashimaEG21}, \href{../works/Zahout21.pdf}{Zahout21}~\cite{Zahout21}, \href{../works/MengLZB21.pdf}{MengLZB21}~\cite{MengLZB21}...\href{../works/MengZRZL20.pdf}{MengZRZL20}~\cite{MengZRZL20}, \href{../works/KletzanderM20.pdf}{KletzanderM20}~\cite{KletzanderM20}, \href{../works/EscobetPQPRA19.pdf}{EscobetPQPRA19}~\cite{EscobetPQPRA19}, \href{../works/YounespourAKE19.pdf}{YounespourAKE19}~\cite{YounespourAKE19}, \href{../works/PourDERB18.pdf}{PourDERB18}~\cite{PourDERB18}, \href{../works/CappartTSR18.pdf}{CappartTSR18}~\cite{CappartTSR18}, \href{../works/MenciaSV13.pdf}{MenciaSV13}~\cite{MenciaSV13}, \href{../works/MeskensDHG11.pdf}{MeskensDHG11}~\cite{MeskensDHG11}, \href{../works/Zeballos10.pdf}{Zeballos10}~\cite{Zeballos10}, \href{../works/Salido10.pdf}{Salido10}~\cite{Salido10} (Total: 35) & \href{../works/abs-2402-00459.pdf}{abs-2402-00459}~\cite{abs-2402-00459}, \href{../works/GokPTGO23.pdf}{GokPTGO23}~\cite{GokPTGO23}, \href{../works/SquillaciPR23.pdf}{SquillaciPR23}~\cite{SquillaciPR23}, \href{../works/YuraszeckMCCR23.pdf}{YuraszeckMCCR23}~\cite{YuraszeckMCCR23}, \href{../works/MarliereSPR23.pdf}{MarliereSPR23}~\cite{MarliereSPR23}, \href{../works/AlfieriGPS23.pdf}{AlfieriGPS23}~\cite{AlfieriGPS23}, \href{../works/GuoZ23.pdf}{GuoZ23}~\cite{GuoZ23}, \href{../works/MullerMKP22.pdf}{MullerMKP22}~\cite{MullerMKP22}, \href{../works/abs-2211-14492.pdf}{abs-2211-14492}~\cite{abs-2211-14492}, \href{../works/NaqviAIAAA22.pdf}{NaqviAIAAA22}~\cite{NaqviAIAAA22}, \href{../works/TouatBT22.pdf}{TouatBT22}~\cite{TouatBT22}, \href{../works/ColT22.pdf}{ColT22}~\cite{ColT22}, \href{../works/OujanaAYB22.pdf}{OujanaAYB22}~\cite{OujanaAYB22}, \href{../works/BoudreaultSLQ22.pdf}{BoudreaultSLQ22}~\cite{BoudreaultSLQ22}, \href{../works/GhandehariK22.pdf}{GhandehariK22}~\cite{GhandehariK22}, \href{../works/KletzanderMH21.pdf}{KletzanderMH21}~\cite{KletzanderMH21}, \href{../works/ArmstrongGOS21.pdf}{ArmstrongGOS21}~\cite{ArmstrongGOS21}, \href{../works/Astrand21.pdf}{Astrand21}~\cite{Astrand21}, \href{../works/KoehlerBFFHPSSS21.pdf}{KoehlerBFFHPSSS21}~\cite{KoehlerBFFHPSSS21}...\href{../works/Lombardi10.pdf}{Lombardi10}~\cite{Lombardi10}, \href{../works/NovasH10.pdf}{NovasH10}~\cite{NovasH10}, \href{../works/KendallKRU10.pdf}{KendallKRU10}~\cite{KendallKRU10}, \href{../works/LopesCSM10.pdf}{LopesCSM10}~\cite{LopesCSM10}, \href{../works/GarridoAO09.pdf}{GarridoAO09}~\cite{GarridoAO09}, \href{../works/HladikCDJ08.pdf}{HladikCDJ08}~\cite{HladikCDJ08}, \href{../works/BeckW07.pdf}{BeckW07}~\cite{BeckW07}, \href{../works/LiuJ06.pdf}{LiuJ06}~\cite{LiuJ06}, \href{../works/Kuchcinski03.pdf}{Kuchcinski03}~\cite{Kuchcinski03}, \href{../works/BenoistGR02.pdf}{BenoistGR02}~\cite{BenoistGR02} (Total: 75)\\
\index{net present value}\index{Concepts!net present value}net present value &  1.00 & \href{../works/ThiruvadyWGS14.pdf}{ThiruvadyWGS14}~\cite{ThiruvadyWGS14}, \href{../works/GuSS13.pdf}{GuSS13}~\cite{GuSS13}, \href{../works/GuSW12.pdf}{GuSW12}~\cite{GuSW12}, \href{../works/SchuttCSW12.pdf}{SchuttCSW12}~\cite{SchuttCSW12} & \href{../works/CampeauG22.pdf}{CampeauG22}~\cite{CampeauG22}, \href{../works/HillTV21.pdf}{HillTV21}~\cite{HillTV21}, \href{../works/KelarevaTK13.pdf}{KelarevaTK13}~\cite{KelarevaTK13} & \href{../works/abs-2402-00459.pdf}{abs-2402-00459}~\cite{abs-2402-00459}, \href{../works/EtminaniesfahaniGNMS22.pdf}{EtminaniesfahaniGNMS22}~\cite{EtminaniesfahaniGNMS22}, \href{../works/Astrand21.pdf}{Astrand21}~\cite{Astrand21}, \href{../works/AstrandJZ20.pdf}{AstrandJZ20}~\cite{AstrandJZ20}, \href{../works/ZarandiASC20.pdf}{ZarandiASC20}~\cite{ZarandiASC20}, \href{../works/LaborieRSV18.pdf}{LaborieRSV18}~\cite{LaborieRSV18}, \href{../works/MossigeGSMC17.pdf}{MossigeGSMC17}~\cite{MossigeGSMC17}, \href{../works/HookerH17.pdf}{HookerH17}~\cite{HookerH17}, \href{../works/SchnellH17.pdf}{SchnellH17}~\cite{SchnellH17}, \href{../works/SzerediS16.pdf}{SzerediS16}~\cite{SzerediS16}, \href{../works/SchuttS16.pdf}{SchuttS16}~\cite{SchuttS16}, \href{../works/SchnellH15.pdf}{SchnellH15}~\cite{SchnellH15}, \href{../works/BlomBPS14.pdf}{BlomBPS14}~\cite{BlomBPS14}, \href{../works/LaborieR14.pdf}{LaborieR14}~\cite{LaborieR14}, \href{../works/SchuttFS13.pdf}{SchuttFS13}~\cite{SchuttFS13}, \href{../works/Lombardi10.pdf}{Lombardi10}~\cite{Lombardi10}\\
\index{no preempt}\index{Concepts!no preempt}no preempt &  1.00 &  &  & \href{../works/ColT22.pdf}{ColT22}~\cite{ColT22}, \href{../works/TouatBT22.pdf}{TouatBT22}~\cite{TouatBT22}, \href{../works/FanXG21.pdf}{FanXG21}~\cite{FanXG21}, \href{../works/Bedhief21.pdf}{Bedhief21}~\cite{Bedhief21}, \href{../works/Lunardi20.pdf}{Lunardi20}~\cite{Lunardi20}, \href{../works/MengZRZL20.pdf}{MengZRZL20}~\cite{MengZRZL20}, \href{../works/ParkUJR19.pdf}{ParkUJR19}~\cite{ParkUJR19}, \href{../works/NattafALR16.pdf}{NattafALR16}~\cite{NattafALR16}, \href{../works/TerekhovTDB14.pdf}{TerekhovTDB14}~\cite{TerekhovTDB14}, \href{../works/OddiRCS11.pdf}{OddiRCS11}~\cite{OddiRCS11}, \href{../works/LombardiMRB10.pdf}{LombardiMRB10}~\cite{LombardiMRB10}, \href{../works/LiW08.pdf}{LiW08}~\cite{LiW08}, \href{../works/BeckW07.pdf}{BeckW07}~\cite{BeckW07}, \href{../works/MonetteDD07.pdf}{MonetteDD07}~\cite{MonetteDD07}, \href{../works/Baptiste02.pdf}{Baptiste02}~\cite{Baptiste02}, \href{../works/ArtiguesR00.pdf}{ArtiguesR00}~\cite{ArtiguesR00}, \href{../works/BruckerK00.pdf}{BruckerK00}~\cite{BruckerK00}\\
\index{no-wait}\index{Concepts!no-wait}no-wait &  1.00 & \href{../works/PrataAN23.pdf}{PrataAN23}~\cite{PrataAN23}, \href{../works/Fatemi-AnarakiTFV23.pdf}{Fatemi-AnarakiTFV23}~\cite{Fatemi-AnarakiTFV23}, \href{../works/IsikYA23.pdf}{IsikYA23}~\cite{IsikYA23}, \href{../works/NaderiRR23.pdf}{NaderiRR23}~\cite{NaderiRR23}, \href{../works/AbreuNP23.pdf}{AbreuNP23}~\cite{AbreuNP23}, \href{../works/AlfieriGPS23.pdf}{AlfieriGPS23}~\cite{AlfieriGPS23}, \href{../works/HubnerGSV21.pdf}{HubnerGSV21}~\cite{HubnerGSV21}, \href{../works/VlkHT21.pdf}{VlkHT21}~\cite{VlkHT21}, \href{../works/ZarandiASC20.pdf}{ZarandiASC20}~\cite{ZarandiASC20}, \href{../works/Novas19.pdf}{Novas19}~\cite{Novas19}, \href{../works/GrimesH15.pdf}{GrimesH15}~\cite{GrimesH15}, \href{../works/GrimesH11.pdf}{GrimesH11}~\cite{GrimesH11}, \href{../works/GrimesH10.pdf}{GrimesH10}~\cite{GrimesH10}, \href{../works/AkkerDH07.pdf}{AkkerDH07}~\cite{AkkerDH07} & \href{../works/MengGRZSC22.pdf}{MengGRZSC22}~\cite{MengGRZSC22}, \href{../works/AbreuN22.pdf}{AbreuN22}~\cite{AbreuN22}, \href{../works/AbreuAPNM21.pdf}{AbreuAPNM21}~\cite{AbreuAPNM21}, \href{../works/MengZRZL20.pdf}{MengZRZL20}~\cite{MengZRZL20}, \href{../works/MejiaY20.pdf}{MejiaY20}~\cite{MejiaY20}, \href{../works/MokhtarzadehTNF20.pdf}{MokhtarzadehTNF20}~\cite{MokhtarzadehTNF20}, \href{../works/Dejemeppe16.pdf}{Dejemeppe16}~\cite{Dejemeppe16}, \href{../works/Malapert11.pdf}{Malapert11}~\cite{Malapert11} & \href{../works/WessenCSFPM23.pdf}{WessenCSFPM23}~\cite{WessenCSFPM23}, \href{../works/AbreuPNF23.pdf}{AbreuPNF23}~\cite{AbreuPNF23}, \href{../works/MarliereSPR23.pdf}{MarliereSPR23}~\cite{MarliereSPR23}, \href{../works/NaderiBZ23.pdf}{NaderiBZ23}~\cite{NaderiBZ23}, \href{../works/YuraszeckMPV22.pdf}{YuraszeckMPV22}~\cite{YuraszeckMPV22}, \href{../works/ArmstrongGOS22.pdf}{ArmstrongGOS22}~\cite{ArmstrongGOS22}, \href{../works/EmdeZD22.pdf}{EmdeZD22}~\cite{EmdeZD22}, \href{../works/MullerMKP22.pdf}{MullerMKP22}~\cite{MullerMKP22}, \href{../works/AwadMDMT22.pdf}{AwadMDMT22}~\cite{AwadMDMT22}, \href{../works/BourreauGGLT22.pdf}{BourreauGGLT22}~\cite{BourreauGGLT22}, \href{../works/LiFJZLL22.pdf}{LiFJZLL22}~\cite{LiFJZLL22}, \href{../works/FarsiTM22.pdf}{FarsiTM22}~\cite{FarsiTM22}, \href{../works/NaderiBZ22.pdf}{NaderiBZ22}~\cite{NaderiBZ22}, \href{../works/Bedhief21.pdf}{Bedhief21}~\cite{Bedhief21}, \href{../works/HauderBRPA20.pdf}{HauderBRPA20}~\cite{HauderBRPA20}, \href{../works/abs-1902-09244.pdf}{abs-1902-09244}~\cite{abs-1902-09244}, \href{../works/RiahiNS018.pdf}{RiahiNS018}~\cite{RiahiNS018}, \href{../works/ZhangW18.pdf}{ZhangW18}~\cite{ZhangW18}, \href{../works/ArbaouiY18.pdf}{ArbaouiY18}~\cite{ArbaouiY18}, \href{../works/WangMD15.pdf}{WangMD15}~\cite{WangMD15}, \href{../works/NovasH12.pdf}{NovasH12}~\cite{NovasH12}, \href{../works/ZeballosNH11.pdf}{ZeballosNH11}~\cite{ZeballosNH11}, \href{../works/HermenierDL11.pdf}{HermenierDL11}~\cite{HermenierDL11}, \href{../works/NovasH10.pdf}{NovasH10}~\cite{NovasH10}, \href{../works/RodriguezS09.pdf}{RodriguezS09}~\cite{RodriguezS09}, \href{../works/Rodriguez07b.pdf}{Rodriguez07b}~\cite{Rodriguez07b}, \href{../works/LammaMM97.pdf}{LammaMM97}~\cite{LammaMM97}, \href{../works/BlazewiczDP96.pdf}{BlazewiczDP96}~\cite{BlazewiczDP96}, \href{../works/BrusoniCLMMT96.pdf}{BrusoniCLMMT96}~\cite{BrusoniCLMMT96}\\
\index{one-machine scheduling}\index{Concepts!one-machine scheduling}one-machine scheduling &  1.00 & \href{../works/MilanoW09.pdf}{MilanoW09}~\cite{MilanoW09}, \href{../works/MilanoW06.pdf}{MilanoW06}~\cite{MilanoW06}, \href{../works/BlazewiczDP96.pdf}{BlazewiczDP96}~\cite{BlazewiczDP96} & \href{../works/ZhangBB22.pdf}{ZhangBB22}~\cite{ZhangBB22}, \href{../works/Schutt11.pdf}{Schutt11}~\cite{Schutt11}, \href{../works/Baptiste02.pdf}{Baptiste02}~\cite{Baptiste02} & \href{../works/PenzDN23.pdf}{PenzDN23}~\cite{PenzDN23}, \href{../works/ColT22.pdf}{ColT22}~\cite{ColT22}, \href{../works/Astrand21.pdf}{Astrand21}~\cite{Astrand21}, \href{../works/FanXG21.pdf}{FanXG21}~\cite{FanXG21}, \href{../works/KoehlerBFFHPSSS21.pdf}{KoehlerBFFHPSSS21}~\cite{KoehlerBFFHPSSS21}, \href{../works/ZarandiASC20.pdf}{ZarandiASC20}~\cite{ZarandiASC20}, \href{../works/Hooker19.pdf}{Hooker19}~\cite{Hooker19}, \href{../works/HookerH17.pdf}{HookerH17}~\cite{HookerH17}, \href{../works/MelgarejoLS15.pdf}{MelgarejoLS15}~\cite{MelgarejoLS15}, \href{../works/BeniniLMR11.pdf}{BeniniLMR11}~\cite{BeniniLMR11}, \href{../works/ArtiguesF07.pdf}{ArtiguesF07}~\cite{ArtiguesF07}, \href{../works/SadykovW06.pdf}{SadykovW06}~\cite{SadykovW06}, \href{../works/ChuX05.pdf}{ChuX05}~\cite{ChuX05}, \href{../works/BeckW04.pdf}{BeckW04}~\cite{BeckW04}, \href{../works/Sadykov04.pdf}{Sadykov04}~\cite{Sadykov04}, \href{../works/ArtiguesBF04.pdf}{ArtiguesBF04}~\cite{ArtiguesBF04}, \href{../works/HookerO03.pdf}{HookerO03}~\cite{HookerO03}, \href{../works/BosiM2001.pdf}{BosiM2001}~\cite{BosiM2001}, \href{../works/JainM99.pdf}{JainM99}~\cite{JainM99}\\
\index{online scheduling}\index{Concepts!online scheduling}online scheduling &  1.00 & \href{../works/TerekhovTDB14.pdf}{TerekhovTDB14}~\cite{TerekhovTDB14}, \href{../works/GetoorOFC97.pdf}{GetoorOFC97}~\cite{GetoorOFC97} & \href{../works/Mehdizadeh-Somarin23.pdf}{Mehdizadeh-Somarin23}~\cite{Mehdizadeh-Somarin23}, \href{../works/Zahout21.pdf}{Zahout21}~\cite{Zahout21}, \href{../works/Groleaz21.pdf}{Groleaz21}~\cite{Groleaz21} & \href{../works/PrataAN23.pdf}{PrataAN23}~\cite{PrataAN23}, \href{../works/MullerMKP22.pdf}{MullerMKP22}~\cite{MullerMKP22}, \href{../works/RoshanaeiN21.pdf}{RoshanaeiN21}~\cite{RoshanaeiN21}, \href{../works/VlkHT21.pdf}{VlkHT21}~\cite{VlkHT21}, \href{../works/NishikawaSTT19.pdf}{NishikawaSTT19}~\cite{NishikawaSTT19}, \href{../works/TranPZLDB18.pdf}{TranPZLDB18}~\cite{TranPZLDB18}, \href{../works/HebrardHJMPV16.pdf}{HebrardHJMPV16}~\cite{HebrardHJMPV16}, \href{../works/LimHTB16.pdf}{LimHTB16}~\cite{LimHTB16}, \href{../works/LimBTBB15a.pdf}{LimBTBB15a}~\cite{LimBTBB15a}, \href{../works/ZhouGL15.pdf}{ZhouGL15}~\cite{ZhouGL15}, \href{../works/DoomsH08.pdf}{DoomsH08}~\cite{DoomsH08}, \href{../works/ElkhyariGJ02a.pdf}{ElkhyariGJ02a}~\cite{ElkhyariGJ02a}, \href{../works/DincbasS91.pdf}{DincbasS91}~\cite{DincbasS91}\\
\index{open-shop}\index{Concepts!open-shop}open-shop &  1.00 & \href{../works/PrataAN23.pdf}{PrataAN23}~\cite{PrataAN23}, \href{../works/AbreuPNF23.pdf}{AbreuPNF23}~\cite{AbreuPNF23}, \href{../works/NaderiRR23.pdf}{NaderiRR23}~\cite{NaderiRR23}, \href{../works/Bit-Monnot23.pdf}{Bit-Monnot23}~\cite{Bit-Monnot23}, \href{../works/AbreuNP23.pdf}{AbreuNP23}~\cite{AbreuNP23}, \href{../works/AbreuN22.pdf}{AbreuN22}~\cite{AbreuN22}, \href{../works/YuraszeckMPV22.pdf}{YuraszeckMPV22}~\cite{YuraszeckMPV22}, \href{../works/AbreuAPNM21.pdf}{AbreuAPNM21}~\cite{AbreuAPNM21}, \href{../works/Groleaz21.pdf}{Groleaz21}~\cite{Groleaz21}, \href{../works/ZarandiASC20.pdf}{ZarandiASC20}~\cite{ZarandiASC20}, \href{../works/MejiaY20.pdf}{MejiaY20}~\cite{MejiaY20}, \href{../works/Lunardi20.pdf}{Lunardi20}~\cite{Lunardi20}, \href{../works/FahimiOQ18.pdf}{FahimiOQ18}~\cite{FahimiOQ18}, \href{../works/Fahimi16.pdf}{Fahimi16}~\cite{Fahimi16}, \href{../works/Siala15.pdf}{Siala15}~\cite{Siala15}, \href{../works/GrimesH15.pdf}{GrimesH15}~\cite{GrimesH15}, \href{../works/Siala15a.pdf}{Siala15a}~\cite{Siala15a}, \href{../works/MalapertCGJLR13.pdf}{MalapertCGJLR13}~\cite{MalapertCGJLR13}, \href{../works/MalapertCGJLR12.pdf}{MalapertCGJLR12}~\cite{MalapertCGJLR12}, \href{../works/Malapert11.pdf}{Malapert11}~\cite{Malapert11}, \href{../works/GrimesHM09.pdf}{GrimesHM09}~\cite{GrimesHM09}, \href{../works/OhrimenkoSC09.pdf}{OhrimenkoSC09}~\cite{OhrimenkoSC09}, \href{../works/MonetteDD07.pdf}{MonetteDD07}~\cite{MonetteDD07}, \href{../works/Laborie05.pdf}{Laborie05}~\cite{Laborie05}, \href{../works/Elkhyari03.pdf}{Elkhyari03}~\cite{Elkhyari03}, \href{../works/LorigeonBB02.pdf}{LorigeonBB02}~\cite{LorigeonBB02}, \href{../works/JussienL02.pdf}{JussienL02}~\cite{JussienL02}, \href{../works/Baptiste02.pdf}{Baptiste02}~\cite{Baptiste02}, \href{../works/FocacciLN00.pdf}{FocacciLN00}~\cite{FocacciLN00} & \href{../works/ZhuSZW23.pdf}{ZhuSZW23}~\cite{ZhuSZW23}, \href{../works/Godet21a.pdf}{Godet21a}~\cite{Godet21a}, \href{../works/Astrand21.pdf}{Astrand21}~\cite{Astrand21}, \href{../works/SacramentoSP20.pdf}{SacramentoSP20}~\cite{SacramentoSP20}, \href{../works/MengZRZL20.pdf}{MengZRZL20}~\cite{MengZRZL20}, \href{../works/Dejemeppe16.pdf}{Dejemeppe16}~\cite{Dejemeppe16}, \href{../works/TerekhovDOB12.pdf}{TerekhovDOB12}~\cite{TerekhovDOB12}, \href{../works/Schutt11.pdf}{Schutt11}~\cite{Schutt11}, \href{../works/GrimesH10.pdf}{GrimesH10}~\cite{GrimesH10}, \href{../works/Vilim05.pdf}{Vilim05}~\cite{Vilim05}, \href{../works/Demassey03.pdf}{Demassey03}~\cite{Demassey03}, \href{../works/Dorndorf2000.pdf}{Dorndorf2000}~\cite{Dorndorf2000}, \href{../works/JainM99.pdf}{JainM99}~\cite{JainM99} & \href{../works/BonninMNE24.pdf}{BonninMNE24}~\cite{BonninMNE24}, \href{../works/NaderiBZ23.pdf}{NaderiBZ23}~\cite{NaderiBZ23}, \href{../works/YuraszeckMCCR23.pdf}{YuraszeckMCCR23}~\cite{YuraszeckMCCR23}, \href{../works/YuraszeckMC23.pdf}{YuraszeckMC23}~\cite{YuraszeckMC23}, \href{../works/KimCMLLP23.pdf}{KimCMLLP23}~\cite{KimCMLLP23}, \href{../works/ShaikhK23.pdf}{ShaikhK23}~\cite{ShaikhK23}, \href{../works/AfsarVPG23.pdf}{AfsarVPG23}~\cite{AfsarVPG23}, \href{../works/NaderiBZ22.pdf}{NaderiBZ22}~\cite{NaderiBZ22}, \href{../works/OujanaAYB22.pdf}{OujanaAYB22}~\cite{OujanaAYB22}, \href{../works/EtminaniesfahaniGNMS22.pdf}{EtminaniesfahaniGNMS22}~\cite{EtminaniesfahaniGNMS22}, \href{../works/EmdeZD22.pdf}{EmdeZD22}~\cite{EmdeZD22}, \href{../works/ColT22.pdf}{ColT22}~\cite{ColT22}, \href{../works/Astrand0F21.pdf}{Astrand0F21}~\cite{Astrand0F21}, \href{../works/abs-2102-08778.pdf}{abs-2102-08778}~\cite{abs-2102-08778}, \href{../works/AstrandJZ20.pdf}{AstrandJZ20}~\cite{AstrandJZ20}, \href{../works/ColT2019a.pdf}{ColT2019a}~\cite{ColT2019a}, \href{../works/ParkUJR19.pdf}{ParkUJR19}~\cite{ParkUJR19}, \href{../works/GombolayWS18.pdf}{GombolayWS18}~\cite{GombolayWS18}, \href{../works/HookerH17.pdf}{HookerH17}~\cite{HookerH17}...\href{../works/BartakSR08.pdf}{BartakSR08}~\cite{BartakSR08}, \href{../works/LiW08.pdf}{LiW08}~\cite{LiW08}, \href{../works/KusterJF07.pdf}{KusterJF07}~\cite{KusterJF07}, \href{../works/ArtiouchineB05.pdf}{ArtiouchineB05}~\cite{ArtiouchineB05}, \href{../works/VilimBC05.pdf}{VilimBC05}~\cite{VilimBC05}, \href{../works/HentenryckM04.pdf}{HentenryckM04}~\cite{HentenryckM04}, \href{../works/VilimBC04.pdf}{VilimBC04}~\cite{VilimBC04}, \href{../works/Vilim03.pdf}{Vilim03}~\cite{Vilim03}, \href{../works/ElkhyariGJ02a.pdf}{ElkhyariGJ02a}~\cite{ElkhyariGJ02a}, \href{../works/VerfaillieL01.pdf}{VerfaillieL01}~\cite{VerfaillieL01} (Total: 43)\\
\index{order scheduling}\index{Concepts!order scheduling}order scheduling &  1.00 & \href{../works/TerekhovDOB12.pdf}{TerekhovDOB12}~\cite{TerekhovDOB12} & \href{../works/PrataAN23.pdf}{PrataAN23}~\cite{PrataAN23}, \href{../works/AbreuPNF23.pdf}{AbreuPNF23}~\cite{AbreuPNF23} & \href{../works/QinWSLS21.pdf}{QinWSLS21}~\cite{QinWSLS21}, \href{../works/AbreuAPNM21.pdf}{AbreuAPNM21}~\cite{AbreuAPNM21}, \href{../works/DoRZ08.pdf}{DoRZ08}~\cite{DoRZ08}\\
\index{periodic}\index{Concepts!periodic}periodic &  1.00 & \href{../works/SquillaciPR23.pdf}{SquillaciPR23}~\cite{SquillaciPR23}, \href{../works/Groleaz21.pdf}{Groleaz21}~\cite{Groleaz21}, \href{../works/Lemos21.pdf}{Lemos21}~\cite{Lemos21}, \href{../works/BonfiettiZLM16.pdf}{BonfiettiZLM16}~\cite{BonfiettiZLM16}, \href{../works/Fahimi16.pdf}{Fahimi16}~\cite{Fahimi16}, \href{../works/AlesioBNG15.pdf}{AlesioBNG15}~\cite{AlesioBNG15}, \href{../works/AlesioNBG14.pdf}{AlesioNBG14}~\cite{AlesioNBG14}, \href{../works/BonfiettiLBM14.pdf}{BonfiettiLBM14}~\cite{BonfiettiLBM14}, \href{../works/TerekhovTDB14.pdf}{TerekhovTDB14}~\cite{TerekhovTDB14}, \href{../works/BonfiettiLM13.pdf}{BonfiettiLM13}~\cite{BonfiettiLM13}, \href{../works/TranTDB13.pdf}{TranTDB13}~\cite{TranTDB13}, \href{../works/SimoninAHL12.pdf}{SimoninAHL12}~\cite{SimoninAHL12}, \href{../works/BonfiettiLBM12.pdf}{BonfiettiLBM12}~\cite{BonfiettiLBM12}, \href{../works/LiuW11.pdf}{LiuW11}~\cite{LiuW11}, \href{../works/LombardiBMB11.pdf}{LombardiBMB11}~\cite{LombardiBMB11}, \href{../works/Lombardi10.pdf}{Lombardi10}~\cite{Lombardi10}, \href{../works/HladikCDJ08.pdf}{HladikCDJ08}~\cite{HladikCDJ08}, \href{../works/Johnston05.pdf}{Johnston05}~\cite{Johnston05}, \href{../works/CambazardHDJT04.pdf}{CambazardHDJT04}~\cite{CambazardHDJT04}, \href{../works/SchildW00.pdf}{SchildW00}~\cite{SchildW00}, \href{../works/KorbaaYG99.pdf}{KorbaaYG99}~\cite{KorbaaYG99}, \href{../works/PembertonG98.pdf}{PembertonG98}~\cite{PembertonG98} & \href{../works/Mehdizadeh-Somarin23.pdf}{Mehdizadeh-Somarin23}~\cite{Mehdizadeh-Somarin23}, \href{../works/TouatBT22.pdf}{TouatBT22}~\cite{TouatBT22}, \href{../works/VlkHT21.pdf}{VlkHT21}~\cite{VlkHT21}, \href{../works/Astrand21.pdf}{Astrand21}~\cite{Astrand21}, \href{../works/Bonfietti16.pdf}{Bonfietti16}~\cite{Bonfietti16}, \href{../works/BajestaniB15.pdf}{BajestaniB15}~\cite{BajestaniB15}, \href{../works/HarjunkoskiMBC14.pdf}{HarjunkoskiMBC14}~\cite{HarjunkoskiMBC14}, \href{../works/BonfiettiLBM11.pdf}{BonfiettiLBM11}~\cite{BonfiettiLBM11}, \href{../works/NovasH10.pdf}{NovasH10}~\cite{NovasH10}, \href{../works/Davenport10.pdf}{Davenport10}~\cite{Davenport10}, \href{../works/BocewiczBB09.pdf}{BocewiczBB09}~\cite{BocewiczBB09}, \href{../works/BeniniLMR08.pdf}{BeniniLMR08}~\cite{BeniniLMR08}, \href{../works/BeniniBGM05.pdf}{BeniniBGM05}~\cite{BeniniBGM05} & \href{../works/FalqueALM24.pdf}{FalqueALM24}~\cite{FalqueALM24}, \href{../works/CzerniachowskaWZ23.pdf}{CzerniachowskaWZ23}~\cite{CzerniachowskaWZ23}, \href{../works/PenzDN23.pdf}{PenzDN23}~\cite{PenzDN23}, \href{../works/AkramNHRSA23.pdf}{AkramNHRSA23}~\cite{AkramNHRSA23}, \href{../works/abs-2306-05747.pdf}{abs-2306-05747}~\cite{abs-2306-05747}, \href{../works/Adelgren2023.pdf}{Adelgren2023}~\cite{Adelgren2023}, \href{../works/WessenCSFPM23.pdf}{WessenCSFPM23}~\cite{WessenCSFPM23}, \href{../works/AbreuPNF23.pdf}{AbreuPNF23}~\cite{AbreuPNF23}, \href{../works/NaderiBZR23.pdf}{NaderiBZR23}~\cite{NaderiBZR23}, \href{../works/TasselGS23.pdf}{TasselGS23}~\cite{TasselGS23}, \href{../works/FarsiTM22.pdf}{FarsiTM22}~\cite{FarsiTM22}, \href{../works/OrnekOS20.pdf}{OrnekOS20}~\cite{OrnekOS20}, \href{../works/PopovicCGNC22.pdf}{PopovicCGNC22}~\cite{PopovicCGNC22}, \href{../works/AbreuAPNM21.pdf}{AbreuAPNM21}~\cite{AbreuAPNM21}, \href{../works/Godet21a.pdf}{Godet21a}~\cite{Godet21a}, \href{../works/AntunesABD20.pdf}{AntunesABD20}~\cite{AntunesABD20}, \href{../works/AntuoriHHEN20.pdf}{AntuoriHHEN20}~\cite{AntuoriHHEN20}, \href{../works/AstrandJZ20.pdf}{AstrandJZ20}~\cite{AstrandJZ20}, \href{../works/Polo-MejiaALB20.pdf}{Polo-MejiaALB20}~\cite{Polo-MejiaALB20}...\href{../works/Beck07.pdf}{Beck07}~\cite{Beck07}, \href{../works/RossiTHP07.pdf}{RossiTHP07}~\cite{RossiTHP07}, \href{../works/BidotVLB07.pdf}{BidotVLB07}~\cite{BidotVLB07}, \href{../works/FrankK05.pdf}{FrankK05}~\cite{FrankK05}, \href{../works/RoePS05.pdf}{RoePS05}~\cite{RoePS05}, \href{../works/Elkhyari03.pdf}{Elkhyari03}~\cite{Elkhyari03}, \href{../works/Kuchcinski03.pdf}{Kuchcinski03}~\cite{Kuchcinski03}, \href{../works/OddiPCC03.pdf}{OddiPCC03}~\cite{OddiPCC03}, \href{../works/FukunagaHFAMN02.pdf}{FukunagaHFAMN02}~\cite{FukunagaHFAMN02}, \href{../works/RoweJCA96.pdf}{RoweJCA96}~\cite{RoweJCA96} (Total: 76)\\
\index{planned maintenance}\index{Concepts!planned maintenance}planned maintenance &  1.00 &  & \href{../works/Malapert11.pdf}{Malapert11}~\cite{Malapert11}, \href{../works/Davenport10.pdf}{Davenport10}~\cite{Davenport10} & \href{../works/TouatBT22.pdf}{TouatBT22}~\cite{TouatBT22}, \href{../works/KovacsTKSG21.pdf}{KovacsTKSG21}~\cite{KovacsTKSG21}, \href{../works/Astrand21.pdf}{Astrand21}~\cite{Astrand21}, \href{../works/AntunesABD20.pdf}{AntunesABD20}~\cite{AntunesABD20}, \href{../works/BajestaniB15.pdf}{BajestaniB15}~\cite{BajestaniB15}, \href{../works/AkkerDH07.pdf}{AkkerDH07}~\cite{AkkerDH07}\\
\index{precedence}\index{Concepts!precedence}precedence &  1.00 & \href{../works/BonninMNE24.pdf}{BonninMNE24}~\cite{BonninMNE24}, \href{../works/LuZZYW24.pdf}{LuZZYW24}~\cite{LuZZYW24}, \href{../works/abs-2402-00459.pdf}{abs-2402-00459}~\cite{abs-2402-00459}, \href{../works/WessenCSFPM23.pdf}{WessenCSFPM23}~\cite{WessenCSFPM23}, \href{../works/MarliereSPR23.pdf}{MarliereSPR23}~\cite{MarliereSPR23}, \href{../works/AlfieriGPS23.pdf}{AlfieriGPS23}~\cite{AlfieriGPS23}, \href{../works/AlakaP23.pdf}{AlakaP23}~\cite{AlakaP23}, \href{../works/NaderiRR23.pdf}{NaderiRR23}~\cite{NaderiRR23}, \href{../works/YuraszeckMCCR23.pdf}{YuraszeckMCCR23}~\cite{YuraszeckMCCR23}, \href{../works/PovedaAA23.pdf}{PovedaAA23}~\cite{PovedaAA23}, \href{../works/JuvinHHL23.pdf}{JuvinHHL23}~\cite{JuvinHHL23}, \href{../works/ZhuSZW23.pdf}{ZhuSZW23}~\cite{ZhuSZW23}, \href{../works/IsikYA23.pdf}{IsikYA23}~\cite{IsikYA23}, \href{../works/KotaryFH22.pdf}{KotaryFH22}~\cite{KotaryFH22}, \href{../works/PohlAK22.pdf}{PohlAK22}~\cite{PohlAK22}, \href{../works/CampeauG22.pdf}{CampeauG22}~\cite{CampeauG22}, \href{../works/YunusogluY22.pdf}{YunusogluY22}~\cite{YunusogluY22}, \href{../works/ZhangBB22.pdf}{ZhangBB22}~\cite{ZhangBB22}, \href{../works/NaderiBZ22a.pdf}{NaderiBZ22a}~\cite{NaderiBZ22a}...\href{../works/BrusoniCLMMT96.pdf}{BrusoniCLMMT96}~\cite{BrusoniCLMMT96}, \href{../works/SadehF96.pdf}{SadehF96}~\cite{SadehF96}, \href{../works/NuijtenA96.pdf}{NuijtenA96}~\cite{NuijtenA96}, \href{../works/Nuijten94.pdf}{Nuijten94}~\cite{Nuijten94}, \href{../works/NuijtenA94.pdf}{NuijtenA94}~\cite{NuijtenA94}, \href{../works/Muscettola94.pdf}{Muscettola94}~\cite{Muscettola94}, \href{../works/Pape94.pdf}{Pape94}~\cite{Pape94}, \href{../works/AggounB93.pdf}{AggounB93}~\cite{AggounB93}, \href{../works/SmithC93.pdf}{SmithC93}~\cite{SmithC93}, \href{../works/FoxS90.pdf}{FoxS90}~\cite{FoxS90} (Total: 223) & \href{../works/TardivoDFMP23.pdf}{TardivoDFMP23}~\cite{TardivoDFMP23}, \href{../works/Bit-Monnot23.pdf}{Bit-Monnot23}~\cite{Bit-Monnot23}, \href{../works/GokPTGO23.pdf}{GokPTGO23}~\cite{GokPTGO23}, \href{../works/KameugneFND23.pdf}{KameugneFND23}~\cite{KameugneFND23}, \href{../works/JuvinHL23a.pdf}{JuvinHL23a}~\cite{JuvinHL23a}, \href{../works/SubulanC22.pdf}{SubulanC22}~\cite{SubulanC22}, \href{../works/OujanaAYB22.pdf}{OujanaAYB22}~\cite{OujanaAYB22}, \href{../works/ColT22.pdf}{ColT22}~\cite{ColT22}, \href{../works/HamP21.pdf}{HamP21}~\cite{HamP21}, \href{../works/VlkHT21.pdf}{VlkHT21}~\cite{VlkHT21}, \href{../works/AntuoriHHEN21.pdf}{AntuoriHHEN21}~\cite{AntuoriHHEN21}, \href{../works/Zahout21.pdf}{Zahout21}~\cite{Zahout21}, \href{../works/WessenCS20.pdf}{WessenCS20}~\cite{WessenCS20}, \href{../works/Ham20a.pdf}{Ham20a}~\cite{Ham20a}, \href{../works/MokhtarzadehTNF20.pdf}{MokhtarzadehTNF20}~\cite{MokhtarzadehTNF20}, \href{../works/GokGSTO20.pdf}{GokGSTO20}~\cite{GokGSTO20}, \href{../works/QinDCS20.pdf}{QinDCS20}~\cite{QinDCS20}, \href{../works/GeibingerMM19.pdf}{GeibingerMM19}~\cite{GeibingerMM19}, \href{../works/Novas19.pdf}{Novas19}~\cite{Novas19}...\href{../works/TorresL00.pdf}{TorresL00}~\cite{TorresL00}, \href{../works/BaptistePN99.pdf}{BaptistePN99}~\cite{BaptistePN99}, \href{../works/Simonis99.pdf}{Simonis99}~\cite{Simonis99}, \href{../works/BelhadjiI98.pdf}{BelhadjiI98}~\cite{BelhadjiI98}, \href{../works/PapeB97.pdf}{PapeB97}~\cite{PapeB97}, \href{../works/Zhou97.pdf}{Zhou97}~\cite{Zhou97}, \href{../works/OddiS97.pdf}{OddiS97}~\cite{OddiS97}, \href{../works/BeckDF97.pdf}{BeckDF97}~\cite{BeckDF97}, \href{../works/BeckDSF97.pdf}{BeckDSF97}~\cite{BeckDSF97}, \href{../works/Zhou96.pdf}{Zhou96}~\cite{Zhou96} (Total: 99) & \href{../works/PrataAN23.pdf}{PrataAN23}~\cite{PrataAN23}, \href{../works/BofillCGGPSV23.pdf}{BofillCGGPSV23}~\cite{BofillCGGPSV23}, \href{../works/JuvinHL23.pdf}{JuvinHL23}~\cite{JuvinHL23}, \href{../works/AfsarVPG23.pdf}{AfsarVPG23}~\cite{AfsarVPG23}, \href{../works/NaderiBZR23.pdf}{NaderiBZR23}~\cite{NaderiBZR23}, \href{../works/IklassovMR023.pdf}{IklassovMR023}~\cite{IklassovMR023}, \href{../works/TasselGS23.pdf}{TasselGS23}~\cite{TasselGS23}, \href{../works/Mehdizadeh-Somarin23.pdf}{Mehdizadeh-Somarin23}~\cite{Mehdizadeh-Somarin23}, \href{../works/abs-2306-05747.pdf}{abs-2306-05747}~\cite{abs-2306-05747}, \href{../works/YuraszeckMC23.pdf}{YuraszeckMC23}~\cite{YuraszeckMC23}, \href{../works/KimCMLLP23.pdf}{KimCMLLP23}~\cite{KimCMLLP23}, \href{../works/abs-2305-19888.pdf}{abs-2305-19888}~\cite{abs-2305-19888}, \href{../works/MullerMKP22.pdf}{MullerMKP22}~\cite{MullerMKP22}, \href{../works/GhandehariK22.pdf}{GhandehariK22}~\cite{GhandehariK22}, \href{../works/JuvinHL22.pdf}{JuvinHL22}~\cite{JuvinHL22}, \href{../works/EmdeZD22.pdf}{EmdeZD22}~\cite{EmdeZD22}, \href{../works/ZhangJZL22.pdf}{ZhangJZL22}~\cite{ZhangJZL22}, \href{../works/TouatBT22.pdf}{TouatBT22}~\cite{TouatBT22}, \href{../works/WinterMMW22.pdf}{WinterMMW22}~\cite{WinterMMW22}...\href{../works/BeckDSF97a.pdf}{BeckDSF97a}~\cite{BeckDSF97a}, \href{../works/GetoorOFC97.pdf}{GetoorOFC97}~\cite{GetoorOFC97}, \href{../works/Colombani96.pdf}{Colombani96}~\cite{Colombani96}, \href{../works/Simonis95a.pdf}{Simonis95a}~\cite{Simonis95a}, \href{../works/Goltz95.pdf}{Goltz95}~\cite{Goltz95}, \href{../works/BaptisteP95.pdf}{BaptisteP95}~\cite{BaptisteP95}, \href{../works/Simonis95.pdf}{Simonis95}~\cite{Simonis95}, \href{../works/Prosser89.pdf}{Prosser89}~\cite{Prosser89}, \href{../works/Valdes87.pdf}{Valdes87}~\cite{Valdes87}, \href{../works/Rit86.pdf}{Rit86}~\cite{Rit86} (Total: 133)\\
\index{preempt}\index{Concepts!preempt}preempt &  1.00 & \href{../works/BonninMNE24.pdf}{BonninMNE24}~\cite{BonninMNE24}, \href{../works/JuvinHL23a.pdf}{JuvinHL23a}~\cite{JuvinHL23a}, \href{../works/JuvinHHL23.pdf}{JuvinHHL23}~\cite{JuvinHHL23}, \href{../works/PovedaAA23.pdf}{PovedaAA23}~\cite{PovedaAA23}, \href{../works/SubulanC22.pdf}{SubulanC22}~\cite{SubulanC22}, \href{../works/AwadMDMT22.pdf}{AwadMDMT22}~\cite{AwadMDMT22}, \href{../works/JuvinHL22.pdf}{JuvinHL22}~\cite{JuvinHL22}, \href{../works/Groleaz21.pdf}{Groleaz21}~\cite{Groleaz21}, \href{../works/HanenKP21.pdf}{HanenKP21}~\cite{HanenKP21}, \href{../works/ArtiguesHQT21.pdf}{ArtiguesHQT21}~\cite{ArtiguesHQT21}, \href{../works/Godet21a.pdf}{Godet21a}~\cite{Godet21a}, \href{../works/ZarandiASC20.pdf}{ZarandiASC20}~\cite{ZarandiASC20}, \href{../works/Polo-MejiaALB20.pdf}{Polo-MejiaALB20}~\cite{Polo-MejiaALB20}, \href{../works/NattafHKAL19.pdf}{NattafHKAL19}~\cite{NattafHKAL19}, \href{../works/BaptisteB18.pdf}{BaptisteB18}~\cite{BaptisteB18}, \href{../works/GokgurHO18.pdf}{GokgurHO18}~\cite{GokgurHO18}, \href{../works/FahimiOQ18.pdf}{FahimiOQ18}~\cite{FahimiOQ18}, \href{../works/ZarandiKS16.pdf}{ZarandiKS16}~\cite{ZarandiKS16}, \href{../works/Fahimi16.pdf}{Fahimi16}~\cite{Fahimi16}...\href{../works/Baptiste02.pdf}{Baptiste02}~\cite{Baptiste02}, \href{../works/Dorndorf2000.pdf}{Dorndorf2000}~\cite{Dorndorf2000}, \href{../works/BaptisteP00.pdf}{BaptisteP00}~\cite{BaptisteP00}, \href{../works/BruckerK00.pdf}{BruckerK00}~\cite{BruckerK00}, \href{../works/BaptistePN99.pdf}{BaptistePN99}~\cite{BaptistePN99}, \href{../works/PembertonG98.pdf}{PembertonG98}~\cite{PembertonG98}, \href{../works/PapaB98.pdf}{PapaB98}~\cite{PapaB98}, \href{../works/BaptisteP97.pdf}{BaptisteP97}~\cite{BaptisteP97}, \href{../works/PapeB97.pdf}{PapeB97}~\cite{PapeB97}, \href{../works/BlazewiczDP96.pdf}{BlazewiczDP96}~\cite{BlazewiczDP96} (Total: 50) & \href{../works/PrataAN23.pdf}{PrataAN23}~\cite{PrataAN23}, \href{../works/Adelgren2023.pdf}{Adelgren2023}~\cite{Adelgren2023}, \href{../works/abs-2305-19888.pdf}{abs-2305-19888}~\cite{abs-2305-19888}, \href{../works/AbreuPNF23.pdf}{AbreuPNF23}~\cite{AbreuPNF23}, \href{../works/FetgoD22.pdf}{FetgoD22}~\cite{FetgoD22}, \href{../works/HeinzNVH22.pdf}{HeinzNVH22}~\cite{HeinzNVH22}, \href{../works/OuelletQ22.pdf}{OuelletQ22}~\cite{OuelletQ22}, \href{../works/Zahout21.pdf}{Zahout21}~\cite{Zahout21}, \href{../works/Astrand21.pdf}{Astrand21}~\cite{Astrand21}, \href{../works/Edis21.pdf}{Edis21}~\cite{Edis21}, \href{../works/CarlierPSJ20.pdf}{CarlierPSJ20}~\cite{CarlierPSJ20}, \href{../works/LunardiBLRV20.pdf}{LunardiBLRV20}~\cite{LunardiBLRV20}, \href{../works/SacramentoSP20.pdf}{SacramentoSP20}~\cite{SacramentoSP20}, \href{../works/Mercier-AubinGQ20.pdf}{Mercier-AubinGQ20}~\cite{Mercier-AubinGQ20}, \href{../works/Lunardi20.pdf}{Lunardi20}~\cite{Lunardi20}, \href{../works/Caballero19.pdf}{Caballero19}~\cite{Caballero19}, \href{../works/ArkhipovBL19.pdf}{ArkhipovBL19}~\cite{ArkhipovBL19}, \href{../works/GombolayWS18.pdf}{GombolayWS18}~\cite{GombolayWS18}, \href{../works/HamFC17.pdf}{HamFC17}~\cite{HamFC17}...\href{../works/Laborie09.pdf}{Laborie09}~\cite{Laborie09}, \href{../works/Wolf09.pdf}{Wolf09}~\cite{Wolf09}, \href{../works/SchuttFSW09.pdf}{SchuttFSW09}~\cite{SchuttFSW09}, \href{../works/KovacsB08.pdf}{KovacsB08}~\cite{KovacsB08}, \href{../works/SchausD08.pdf}{SchausD08}~\cite{SchausD08}, \href{../works/ArtiouchineB05.pdf}{ArtiouchineB05}~\cite{ArtiouchineB05}, \href{../works/CambazardHDJT04.pdf}{CambazardHDJT04}~\cite{CambazardHDJT04}, \href{../works/SourdN00.pdf}{SourdN00}~\cite{SourdN00}, \href{../works/Beck99.pdf}{Beck99}~\cite{Beck99}, \href{../works/NuijtenP98.pdf}{NuijtenP98}~\cite{NuijtenP98} (Total: 50) & \href{../works/Mehdizadeh-Somarin23.pdf}{Mehdizadeh-Somarin23}~\cite{Mehdizadeh-Somarin23}, \href{../works/KameugneFND23.pdf}{KameugneFND23}~\cite{KameugneFND23}, \href{../works/abs-2306-05747.pdf}{abs-2306-05747}~\cite{abs-2306-05747}, \href{../works/PenzDN23.pdf}{PenzDN23}~\cite{PenzDN23}, \href{../works/YuraszeckMC23.pdf}{YuraszeckMC23}~\cite{YuraszeckMC23}, \href{../works/YuraszeckMCCR23.pdf}{YuraszeckMCCR23}~\cite{YuraszeckMCCR23}, \href{../works/AbreuNP23.pdf}{AbreuNP23}~\cite{AbreuNP23}, \href{../works/ZhuSZW23.pdf}{ZhuSZW23}~\cite{ZhuSZW23}, \href{../works/IsikYA23.pdf}{IsikYA23}~\cite{IsikYA23}, \href{../works/AfsarVPG23.pdf}{AfsarVPG23}~\cite{AfsarVPG23}, \href{../works/AalianPG23.pdf}{AalianPG23}~\cite{AalianPG23}, \href{../works/NaderiRR23.pdf}{NaderiRR23}~\cite{NaderiRR23}, \href{../works/TasselGS23.pdf}{TasselGS23}~\cite{TasselGS23}, \href{../works/TardivoDFMP23.pdf}{TardivoDFMP23}~\cite{TardivoDFMP23}, \href{../works/AkramNHRSA23.pdf}{AkramNHRSA23}~\cite{AkramNHRSA23}, \href{../works/IklassovMR023.pdf}{IklassovMR023}~\cite{IklassovMR023}, \href{../works/ZhangBB22.pdf}{ZhangBB22}~\cite{ZhangBB22}, \href{../works/MullerMKP22.pdf}{MullerMKP22}~\cite{MullerMKP22}, \href{../works/JungblutK22.pdf}{JungblutK22}~\cite{JungblutK22}...\href{../works/HeipckeCCS00.pdf}{HeipckeCCS00}~\cite{HeipckeCCS00}, \href{../works/JainM99.pdf}{JainM99}~\cite{JainM99}, \href{../works/WatsonBHW99.pdf}{WatsonBHW99}~\cite{WatsonBHW99}, \href{../works/BeckF98.pdf}{BeckF98}~\cite{BeckF98}, \href{../works/BeckDDF98.pdf}{BeckDDF98}~\cite{BeckDDF98}, \href{../works/BelhadjiI98.pdf}{BelhadjiI98}~\cite{BelhadjiI98}, \href{../works/Caseau97.pdf}{Caseau97}~\cite{Caseau97}, \href{../works/Zhou97.pdf}{Zhou97}~\cite{Zhou97}, \href{../works/NuijtenA96.pdf}{NuijtenA96}~\cite{NuijtenA96}, \href{../works/Colombani96.pdf}{Colombani96}~\cite{Colombani96} (Total: 178)\\
\index{preemptive}\index{Concepts!preemptive}preemptive &  1.00 & \href{../works/BonninMNE24.pdf}{BonninMNE24}~\cite{BonninMNE24}, \href{../works/PovedaAA23.pdf}{PovedaAA23}~\cite{PovedaAA23}, \href{../works/JuvinHL23a.pdf}{JuvinHL23a}~\cite{JuvinHL23a}, \href{../works/JuvinHHL23.pdf}{JuvinHHL23}~\cite{JuvinHHL23}, \href{../works/JuvinHL22.pdf}{JuvinHL22}~\cite{JuvinHL22}, \href{../works/AwadMDMT22.pdf}{AwadMDMT22}~\cite{AwadMDMT22}, \href{../works/ArtiguesHQT21.pdf}{ArtiguesHQT21}~\cite{ArtiguesHQT21}, \href{../works/HanenKP21.pdf}{HanenKP21}~\cite{HanenKP21}, \href{../works/Godet21a.pdf}{Godet21a}~\cite{Godet21a}, \href{../works/Polo-MejiaALB20.pdf}{Polo-MejiaALB20}~\cite{Polo-MejiaALB20}, \href{../works/ZarandiASC20.pdf}{ZarandiASC20}~\cite{ZarandiASC20}, \href{../works/NattafHKAL19.pdf}{NattafHKAL19}~\cite{NattafHKAL19}, \href{../works/GokgurHO18.pdf}{GokgurHO18}~\cite{GokgurHO18}, \href{../works/BaptisteB18.pdf}{BaptisteB18}~\cite{BaptisteB18}, \href{../works/Dejemeppe16.pdf}{Dejemeppe16}~\cite{Dejemeppe16}, \href{../works/Fahimi16.pdf}{Fahimi16}~\cite{Fahimi16}, \href{../works/EvenSH15.pdf}{EvenSH15}~\cite{EvenSH15}, \href{../works/EvenSH15a.pdf}{EvenSH15a}~\cite{EvenSH15a}, \href{../works/AlesioNBG14.pdf}{AlesioNBG14}~\cite{AlesioNBG14}...\href{../works/Baptiste02.pdf}{Baptiste02}~\cite{Baptiste02}, \href{../works/Dorndorf2000.pdf}{Dorndorf2000}~\cite{Dorndorf2000}, \href{../works/BruckerK00.pdf}{BruckerK00}~\cite{BruckerK00}, \href{../works/BaptisteP00.pdf}{BaptisteP00}~\cite{BaptisteP00}, \href{../works/BaptistePN99.pdf}{BaptistePN99}~\cite{BaptistePN99}, \href{../works/PembertonG98.pdf}{PembertonG98}~\cite{PembertonG98}, \href{../works/PapaB98.pdf}{PapaB98}~\cite{PapaB98}, \href{../works/BaptisteP97.pdf}{BaptisteP97}~\cite{BaptisteP97}, \href{../works/PapeB97.pdf}{PapeB97}~\cite{PapeB97}, \href{../works/BlazewiczDP96.pdf}{BlazewiczDP96}~\cite{BlazewiczDP96} (Total: 44) & \href{../works/PrataAN23.pdf}{PrataAN23}~\cite{PrataAN23}, \href{../works/AbreuPNF23.pdf}{AbreuPNF23}~\cite{AbreuPNF23}, \href{../works/Adelgren2023.pdf}{Adelgren2023}~\cite{Adelgren2023}, \href{../works/Groleaz21.pdf}{Groleaz21}~\cite{Groleaz21}, \href{../works/Edis21.pdf}{Edis21}~\cite{Edis21}, \href{../works/Mercier-AubinGQ20.pdf}{Mercier-AubinGQ20}~\cite{Mercier-AubinGQ20}, \href{../works/CarlierPSJ20.pdf}{CarlierPSJ20}~\cite{CarlierPSJ20}, \href{../works/SacramentoSP20.pdf}{SacramentoSP20}~\cite{SacramentoSP20}, \href{../works/ArkhipovBL19.pdf}{ArkhipovBL19}~\cite{ArkhipovBL19}, \href{../works/Caballero19.pdf}{Caballero19}~\cite{Caballero19}, \href{../works/FahimiOQ18.pdf}{FahimiOQ18}~\cite{FahimiOQ18}, \href{../works/YoungFS17.pdf}{YoungFS17}~\cite{YoungFS17}, \href{../works/HamFC17.pdf}{HamFC17}~\cite{HamFC17}, \href{../works/NattafALR16.pdf}{NattafALR16}~\cite{NattafALR16}, \href{../works/ZarandiKS16.pdf}{ZarandiKS16}~\cite{ZarandiKS16}, \href{../works/OrnekO16.pdf}{OrnekO16}~\cite{OrnekO16}, \href{../works/OzturkTHO15.pdf}{OzturkTHO15}~\cite{OzturkTHO15}, \href{../works/NattafAL15.pdf}{NattafAL15}~\cite{NattafAL15}, \href{../works/MenciaSV13.pdf}{MenciaSV13}~\cite{MenciaSV13}...\href{../works/ChenGPSH10.pdf}{ChenGPSH10}~\cite{ChenGPSH10}, \href{../works/Wolf09.pdf}{Wolf09}~\cite{Wolf09}, \href{../works/Laborie09.pdf}{Laborie09}~\cite{Laborie09}, \href{../works/SchuttFSW09.pdf}{SchuttFSW09}~\cite{SchuttFSW09}, \href{../works/KovacsB08.pdf}{KovacsB08}~\cite{KovacsB08}, \href{../works/ArtiouchineB05.pdf}{ArtiouchineB05}~\cite{ArtiouchineB05}, \href{../works/CambazardHDJT04.pdf}{CambazardHDJT04}~\cite{CambazardHDJT04}, \href{../works/SourdN00.pdf}{SourdN00}~\cite{SourdN00}, \href{../works/Beck99.pdf}{Beck99}~\cite{Beck99}, \href{../works/NuijtenP98.pdf}{NuijtenP98}~\cite{NuijtenP98} (Total: 37) & \href{../works/IklassovMR023.pdf}{IklassovMR023}~\cite{IklassovMR023}, \href{../works/AalianPG23.pdf}{AalianPG23}~\cite{AalianPG23}, \href{../works/NaderiRR23.pdf}{NaderiRR23}~\cite{NaderiRR23}, \href{../works/Mehdizadeh-Somarin23.pdf}{Mehdizadeh-Somarin23}~\cite{Mehdizadeh-Somarin23}, \href{../works/abs-2305-19888.pdf}{abs-2305-19888}~\cite{abs-2305-19888}, \href{../works/PenzDN23.pdf}{PenzDN23}~\cite{PenzDN23}, \href{../works/YuraszeckMC23.pdf}{YuraszeckMC23}~\cite{YuraszeckMC23}, \href{../works/AbreuN22.pdf}{AbreuN22}~\cite{AbreuN22}, \href{../works/SubulanC22.pdf}{SubulanC22}~\cite{SubulanC22}, \href{../works/EtminaniesfahaniGNMS22.pdf}{EtminaniesfahaniGNMS22}~\cite{EtminaniesfahaniGNMS22}, \href{../works/NaderiBZ22a.pdf}{NaderiBZ22a}~\cite{NaderiBZ22a}, \href{../works/ColT22.pdf}{ColT22}~\cite{ColT22}, \href{../works/HeinzNVH22.pdf}{HeinzNVH22}~\cite{HeinzNVH22}, \href{../works/MullerMKP22.pdf}{MullerMKP22}~\cite{MullerMKP22}, \href{../works/GeitzGSSW22.pdf}{GeitzGSSW22}~\cite{GeitzGSSW22}, \href{../works/ZhangYW21.pdf}{ZhangYW21}~\cite{ZhangYW21}, \href{../works/HillTV21.pdf}{HillTV21}~\cite{HillTV21}, \href{../works/AbreuAPNM21.pdf}{AbreuAPNM21}~\cite{AbreuAPNM21}, \href{../works/ArmstrongGOS21.pdf}{ArmstrongGOS21}~\cite{ArmstrongGOS21}...\href{../works/BeckF00.pdf}{BeckF00}~\cite{BeckF00}, \href{../works/TorresL00.pdf}{TorresL00}~\cite{TorresL00}, \href{../works/WatsonBHW99.pdf}{WatsonBHW99}~\cite{WatsonBHW99}, \href{../works/JainM99.pdf}{JainM99}~\cite{JainM99}, \href{../works/BeckF98.pdf}{BeckF98}~\cite{BeckF98}, \href{../works/BelhadjiI98.pdf}{BelhadjiI98}~\cite{BelhadjiI98}, \href{../works/BeckDDF98.pdf}{BeckDDF98}~\cite{BeckDDF98}, \href{../works/Zhou97.pdf}{Zhou97}~\cite{Zhou97}, \href{../works/Caseau97.pdf}{Caseau97}~\cite{Caseau97}, \href{../works/NuijtenA96.pdf}{NuijtenA96}~\cite{NuijtenA96} (Total: 142)\\
\index{producer/consumer}\index{Concepts!producer/consumer}producer/consumer &  1.00 & \href{../works/SchuttS16.pdf}{SchuttS16}~\cite{SchuttS16}, \href{../works/PoderBS04.pdf}{PoderBS04}~\cite{PoderBS04}, \href{../works/Kumar03.pdf}{Kumar03}~\cite{Kumar03}, \href{../works/Beck99.pdf}{Beck99}~\cite{Beck99}, \href{../works/SimonisC95.pdf}{SimonisC95}~\cite{SimonisC95} & \href{../works/HermenierDL11.pdf}{HermenierDL11}~\cite{HermenierDL11}, \href{../works/BeldiceanuC02.pdf}{BeldiceanuC02}~\cite{BeldiceanuC02}, \href{../works/Simonis99.pdf}{Simonis99}~\cite{Simonis99}, \href{../works/Simonis95a.pdf}{Simonis95a}~\cite{Simonis95a} & \href{../works/GeitzGSSW22.pdf}{GeitzGSSW22}~\cite{GeitzGSSW22}, \href{../works/KlankeBYE21.pdf}{KlankeBYE21}~\cite{KlankeBYE21}, \href{../works/CappartTSR18.pdf}{CappartTSR18}~\cite{CappartTSR18}, \href{../works/BlomPS16.pdf}{BlomPS16}~\cite{BlomPS16}, \href{../works/LombardiM12a.pdf}{LombardiM12a}~\cite{LombardiM12a}, \href{../works/Wolf11.pdf}{Wolf11}~\cite{Wolf11}, \href{../works/SimonisH11.pdf}{SimonisH11}~\cite{SimonisH11}, \href{../works/LombardiMRB10.pdf}{LombardiMRB10}~\cite{LombardiMRB10}, \href{../works/ChenGPSH10.pdf}{ChenGPSH10}~\cite{ChenGPSH10}, \href{../works/PoderB08.pdf}{PoderB08}~\cite{PoderB08}, \href{../works/Simonis07.pdf}{Simonis07}~\cite{Simonis07}, \href{../works/PolicellaWSO05.pdf}{PolicellaWSO05}~\cite{PolicellaWSO05}, \href{../works/Timpe02.pdf}{Timpe02}~\cite{Timpe02}, \href{../works/SimonisCK00.pdf}{SimonisCK00}~\cite{SimonisCK00}, \href{../works/Simonis95.pdf}{Simonis95}~\cite{Simonis95}\\
\index{re-scheduling}\index{Concepts!re-scheduling}re-scheduling &  1.00 & \href{../works/Fatemi-AnarakiTFV23.pdf}{Fatemi-AnarakiTFV23}~\cite{Fatemi-AnarakiTFV23}, \href{../works/MarliereSPR23.pdf}{MarliereSPR23}~\cite{MarliereSPR23}, \href{../works/Astrand21.pdf}{Astrand21}~\cite{Astrand21}, \href{../works/Groleaz21.pdf}{Groleaz21}~\cite{Groleaz21}, \href{../works/Lemos21.pdf}{Lemos21}~\cite{Lemos21}, \href{../works/HamPK21.pdf}{HamPK21}~\cite{HamPK21}, \href{../works/BarzegaranZP20.pdf}{BarzegaranZP20}~\cite{BarzegaranZP20}, \href{../works/ZarandiASC20.pdf}{ZarandiASC20}~\cite{ZarandiASC20}, \href{../works/ZhangW18.pdf}{ZhangW18}~\cite{ZhangW18}, \href{../works/CappartS17.pdf}{CappartS17}~\cite{CappartS17}, \href{../works/Madi-WambaLOBM17.pdf}{Madi-WambaLOBM17}~\cite{Madi-WambaLOBM17}, \href{../works/Froger16.pdf}{Froger16}~\cite{Froger16}, \href{../works/BartakV15.pdf}{BartakV15}~\cite{BartakV15}, \href{../works/GrimesIOS14.pdf}{GrimesIOS14}~\cite{GrimesIOS14}, \href{../works/HarjunkoskiMBC14.pdf}{HarjunkoskiMBC14}~\cite{HarjunkoskiMBC14}, \href{../works/ChunS14.pdf}{ChunS14}~\cite{ChunS14}, \href{../works/BajestaniB13.pdf}{BajestaniB13}~\cite{BajestaniB13}, \href{../works/TranTDB13.pdf}{TranTDB13}~\cite{TranTDB13}, \href{../works/LombardiM12.pdf}{LombardiM12}~\cite{LombardiM12}, \href{../works/RendlPHPR12.pdf}{RendlPHPR12}~\cite{RendlPHPR12}, \href{../works/IfrimOS12.pdf}{IfrimOS12}~\cite{IfrimOS12}, \href{../works/NovasH10.pdf}{NovasH10}~\cite{NovasH10}, \href{../works/BidotVLB09.pdf}{BidotVLB09}~\cite{BidotVLB09}, \href{../works/KusterJF07.pdf}{KusterJF07}~\cite{KusterJF07}, \href{../works/BidotVLB07.pdf}{BidotVLB07}~\cite{BidotVLB07}, \href{../works/Laborie03.pdf}{Laborie03}~\cite{Laborie03}, \href{../works/Baptiste02.pdf}{Baptiste02}~\cite{Baptiste02}, \href{../works/MartinPY01.pdf}{MartinPY01}~\cite{MartinPY01}, \href{../works/ArtiguesR00.pdf}{ArtiguesR00}~\cite{ArtiguesR00} & \href{../works/Mehdizadeh-Somarin23.pdf}{Mehdizadeh-Somarin23}~\cite{Mehdizadeh-Somarin23}, \href{../works/NaderiBZR23.pdf}{NaderiBZR23}~\cite{NaderiBZR23}, \href{../works/NaderiBZ22a.pdf}{NaderiBZ22a}~\cite{NaderiBZ22a}, \href{../works/KovacsTKSG21.pdf}{KovacsTKSG21}~\cite{KovacsTKSG21}, \href{../works/Zahout21.pdf}{Zahout21}~\cite{Zahout21}, \href{../works/AstrandJZ20.pdf}{AstrandJZ20}~\cite{AstrandJZ20}, \href{../works/AntunesABD20.pdf}{AntunesABD20}~\cite{AntunesABD20}, \href{../works/RoshanaeiBAUB20.pdf}{RoshanaeiBAUB20}~\cite{RoshanaeiBAUB20}, \href{../works/BhatnagarKL19.pdf}{BhatnagarKL19}~\cite{BhatnagarKL19}, \href{../works/GombolayWS18.pdf}{GombolayWS18}~\cite{GombolayWS18}, \href{../works/AgussurjaKL18.pdf}{AgussurjaKL18}~\cite{AgussurjaKL18}, \href{../works/HoYCLLCLC18.pdf}{HoYCLLCLC18}~\cite{HoYCLLCLC18}, \href{../works/AntunesABD18.pdf}{AntunesABD18}~\cite{AntunesABD18}, \href{../works/TranPZLDB18.pdf}{TranPZLDB18}~\cite{TranPZLDB18}, \href{../works/FrankDT16.pdf}{FrankDT16}~\cite{FrankDT16}, \href{../works/KinsellaS0OS16.pdf}{KinsellaS0OS16}~\cite{KinsellaS0OS16}, \href{../works/HurleyOS16.pdf}{HurleyOS16}~\cite{HurleyOS16}, \href{../works/LimHTB16.pdf}{LimHTB16}~\cite{LimHTB16}, \href{../works/LimBTBB15.pdf}{LimBTBB15}~\cite{LimBTBB15}...\href{../works/CobanH10.pdf}{CobanH10}~\cite{CobanH10}, \href{../works/Acuna-AgostMFG09.pdf}{Acuna-AgostMFG09}~\cite{Acuna-AgostMFG09}, \href{../works/Johnston05.pdf}{Johnston05}~\cite{Johnston05}, \href{../works/Elkhyari03.pdf}{Elkhyari03}~\cite{Elkhyari03}, \href{../works/DraperJCJ99.pdf}{DraperJCJ99}~\cite{DraperJCJ99}, \href{../works/Beck99.pdf}{Beck99}~\cite{Beck99}, \href{../works/JoLLH99.pdf}{JoLLH99}~\cite{JoLLH99}, \href{../works/MorgadoM97.pdf}{MorgadoM97}~\cite{MorgadoM97}, \href{../works/SadehF96.pdf}{SadehF96}~\cite{SadehF96}, \href{../works/MintonJPL92.pdf}{MintonJPL92}~\cite{MintonJPL92} (Total: 32) & \href{../works/ForbesHJST24.pdf}{ForbesHJST24}~\cite{ForbesHJST24}, \href{../works/PrataAN23.pdf}{PrataAN23}~\cite{PrataAN23}, \href{../works/abs-2306-05747.pdf}{abs-2306-05747}~\cite{abs-2306-05747}, \href{../works/GurPAE23.pdf}{GurPAE23}~\cite{GurPAE23}, \href{../works/PerezGSL23.pdf}{PerezGSL23}~\cite{PerezGSL23}, \href{../works/abs-2312-13682.pdf}{abs-2312-13682}~\cite{abs-2312-13682}, \href{../works/EfthymiouY23.pdf}{EfthymiouY23}~\cite{EfthymiouY23}, \href{../works/TasselGS23.pdf}{TasselGS23}~\cite{TasselGS23}, \href{../works/abs-2305-19888.pdf}{abs-2305-19888}~\cite{abs-2305-19888}, \href{../works/ShaikhK23.pdf}{ShaikhK23}~\cite{ShaikhK23}, \href{../works/NaderiRR23.pdf}{NaderiRR23}~\cite{NaderiRR23}, \href{../works/GokPTGO23.pdf}{GokPTGO23}~\cite{GokPTGO23}, \href{../works/Adelgren2023.pdf}{Adelgren2023}~\cite{Adelgren2023}, \href{../works/JuvinHL23a.pdf}{JuvinHL23a}~\cite{JuvinHL23a}, \href{../works/ZhuSZW23.pdf}{ZhuSZW23}~\cite{ZhuSZW23}, \href{../works/BourreauGGLT22.pdf}{BourreauGGLT22}~\cite{BourreauGGLT22}, \href{../works/HeinzNVH22.pdf}{HeinzNVH22}~\cite{HeinzNVH22}, \href{../works/LuoB22.pdf}{LuoB22}~\cite{LuoB22}, \href{../works/PohlAK22.pdf}{PohlAK22}~\cite{PohlAK22}...\href{../works/BaptisteP97.pdf}{BaptisteP97}~\cite{BaptisteP97}, \href{../works/BeckDSF97.pdf}{BeckDSF97}~\cite{BeckDSF97}, \href{../works/Schaerf97.pdf}{Schaerf97}~\cite{Schaerf97}, \href{../works/BeckDF97.pdf}{BeckDF97}~\cite{BeckDF97}, \href{../works/Pape94.pdf}{Pape94}~\cite{Pape94}, \href{../works/SmithC93.pdf}{SmithC93}~\cite{SmithC93}, \href{../works/MintonJPL90.pdf}{MintonJPL90}~\cite{MintonJPL90}, \href{../works/FoxS90.pdf}{FoxS90}~\cite{FoxS90}, \href{../works/EskeyZ90.pdf}{EskeyZ90}~\cite{EskeyZ90}, \href{../works/Prosser89.pdf}{Prosser89}~\cite{Prosser89} (Total: 123)\\
\index{reactive scheduling}\index{Concepts!reactive scheduling}reactive scheduling &  1.00 & \href{../works/NovasH10.pdf}{NovasH10}~\cite{NovasH10}, \href{../works/JoLLH99.pdf}{JoLLH99}~\cite{JoLLH99} & \href{../works/Groleaz21.pdf}{Groleaz21}~\cite{Groleaz21}, \href{../works/ZarandiASC20.pdf}{ZarandiASC20}~\cite{ZarandiASC20}, \href{../works/BartakV15.pdf}{BartakV15}~\cite{BartakV15}, \href{../works/HarjunkoskiMBC14.pdf}{HarjunkoskiMBC14}~\cite{HarjunkoskiMBC14}, \href{../works/ChunCTY99.pdf}{ChunCTY99}~\cite{ChunCTY99}, \href{../works/Prosser89.pdf}{Prosser89}~\cite{Prosser89} & \href{../works/Mehdizadeh-Somarin23.pdf}{Mehdizadeh-Somarin23}~\cite{Mehdizadeh-Somarin23}, \href{../works/HubnerGSV21.pdf}{HubnerGSV21}~\cite{HubnerGSV21}, \href{../works/FanXG21.pdf}{FanXG21}~\cite{FanXG21}, \href{../works/Lunardi20.pdf}{Lunardi20}~\cite{Lunardi20}, \href{../works/EscobetPQPRA19.pdf}{EscobetPQPRA19}~\cite{EscobetPQPRA19}, \href{../works/Fahimi16.pdf}{Fahimi16}~\cite{Fahimi16}, \href{../works/Froger16.pdf}{Froger16}~\cite{Froger16}, \href{../works/NovasH14.pdf}{NovasH14}~\cite{NovasH14}, \href{../works/BonfiettiLM14.pdf}{BonfiettiLM14}~\cite{BonfiettiLM14}, \href{../works/BajestaniB13.pdf}{BajestaniB13}~\cite{BajestaniB13}, \href{../works/LombardiM12.pdf}{LombardiM12}~\cite{LombardiM12}, \href{../works/BillautHL12.pdf}{BillautHL12}~\cite{BillautHL12}, \href{../works/NovasH12.pdf}{NovasH12}~\cite{NovasH12}, \href{../works/LopesCSM10.pdf}{LopesCSM10}~\cite{LopesCSM10}, \href{../works/ZeballosCM10.pdf}{ZeballosCM10}~\cite{ZeballosCM10}, \href{../works/BidotVLB09.pdf}{BidotVLB09}~\cite{BidotVLB09}, \href{../works/MouraSCL08a.pdf}{MouraSCL08a}~\cite{MouraSCL08a}, \href{../works/KusterJF07.pdf}{KusterJF07}~\cite{KusterJF07}, \href{../works/BidotVLB07.pdf}{BidotVLB07}~\cite{BidotVLB07}, \href{../works/BeckW07.pdf}{BeckW07}~\cite{BeckW07}, \href{../works/Elkhyari03.pdf}{Elkhyari03}~\cite{Elkhyari03}, \href{../works/Baptiste02.pdf}{Baptiste02}~\cite{Baptiste02}, \href{../works/SakkoutW00.pdf}{SakkoutW00}~\cite{SakkoutW00}, \href{../works/BeckF00.pdf}{BeckF00}~\cite{BeckF00}, \href{../works/PapaB98.pdf}{PapaB98}~\cite{PapaB98}, \href{../works/BeckDDF98.pdf}{BeckDDF98}~\cite{BeckDDF98}, \href{../works/NuijtenP98.pdf}{NuijtenP98}~\cite{NuijtenP98}, \href{../works/Wallace96.pdf}{Wallace96}~\cite{Wallace96}, \href{../works/FoxS90.pdf}{FoxS90}~\cite{FoxS90}\\
\index{release-date}\index{Concepts!release-date}release-date &  1.00 & \href{../works/BonninMNE24.pdf}{BonninMNE24}~\cite{BonninMNE24}, \href{../works/YunusogluY22.pdf}{YunusogluY22}~\cite{YunusogluY22}, \href{../works/YuraszeckMPV22.pdf}{YuraszeckMPV22}~\cite{YuraszeckMPV22}, \href{../works/WinterMMW22.pdf}{WinterMMW22}~\cite{WinterMMW22}, \href{../works/EmdeZD22.pdf}{EmdeZD22}~\cite{EmdeZD22}, \href{../works/JuvinHL22.pdf}{JuvinHL22}~\cite{JuvinHL22}, \href{../works/Groleaz21.pdf}{Groleaz21}~\cite{Groleaz21}, \href{../works/HanenKP21.pdf}{HanenKP21}~\cite{HanenKP21}, \href{../works/Bedhief21.pdf}{Bedhief21}~\cite{Bedhief21}, \href{../works/CarlierPSJ20.pdf}{CarlierPSJ20}~\cite{CarlierPSJ20}, \href{../works/Polo-MejiaALB20.pdf}{Polo-MejiaALB20}~\cite{Polo-MejiaALB20}, \href{../works/EscobetPQPRA19.pdf}{EscobetPQPRA19}~\cite{EscobetPQPRA19}, \href{../works/Tesch18.pdf}{Tesch18}~\cite{Tesch18}, \href{../works/KameugneFSN14.pdf}{KameugneFSN14}~\cite{KameugneFSN14}, \href{../works/ArtiguesLH13.pdf}{ArtiguesLH13}~\cite{ArtiguesLH13}, \href{../works/LimtanyakulS12.pdf}{LimtanyakulS12}~\cite{LimtanyakulS12}, \href{../works/SerraNM12.pdf}{SerraNM12}~\cite{SerraNM12}, \href{../works/TerekhovDOB12.pdf}{TerekhovDOB12}~\cite{TerekhovDOB12}, \href{../works/KovacsB11.pdf}{KovacsB11}~\cite{KovacsB11}...\href{../works/ArtiouchineB05.pdf}{ArtiouchineB05}~\cite{ArtiouchineB05}, \href{../works/Hooker05.pdf}{Hooker05}~\cite{Hooker05}, \href{../works/SchuttWS05.pdf}{SchuttWS05}~\cite{SchuttWS05}, \href{../works/Hooker04.pdf}{Hooker04}~\cite{Hooker04}, \href{../works/Baptiste02.pdf}{Baptiste02}~\cite{Baptiste02}, \href{../works/BaptistePN99.pdf}{BaptistePN99}~\cite{BaptistePN99}, \href{../works/Zhou97.pdf}{Zhou97}~\cite{Zhou97}, \href{../works/Colombani96.pdf}{Colombani96}~\cite{Colombani96}, \href{../works/Zhou96.pdf}{Zhou96}~\cite{Zhou96}, \href{../works/BlazewiczDP96.pdf}{BlazewiczDP96}~\cite{BlazewiczDP96} (Total: 41) & \href{../works/PrataAN23.pdf}{PrataAN23}~\cite{PrataAN23}, \href{../works/LacknerMMWW23.pdf}{LacknerMMWW23}~\cite{LacknerMMWW23}, \href{../works/JuvinHL23a.pdf}{JuvinHL23a}~\cite{JuvinHL23a}, \href{../works/ArtiguesHQT21.pdf}{ArtiguesHQT21}~\cite{ArtiguesHQT21}, \href{../works/LacknerMMWW21.pdf}{LacknerMMWW21}~\cite{LacknerMMWW21}, \href{../works/Godet21a.pdf}{Godet21a}~\cite{Godet21a}, \href{../works/GroleazNS20a.pdf}{GroleazNS20a}~\cite{GroleazNS20a}, \href{../works/AntuoriHHEN20.pdf}{AntuoriHHEN20}~\cite{AntuoriHHEN20}, \href{../works/GroleazNS20.pdf}{GroleazNS20}~\cite{GroleazNS20}, \href{../works/ZarandiASC20.pdf}{ZarandiASC20}~\cite{ZarandiASC20}, \href{../works/ArkhipovBL19.pdf}{ArkhipovBL19}~\cite{ArkhipovBL19}, \href{../works/abs-1911-04766.pdf}{abs-1911-04766}~\cite{abs-1911-04766}, \href{../works/GeibingerMM19.pdf}{GeibingerMM19}~\cite{GeibingerMM19}, \href{../works/Dejemeppe16.pdf}{Dejemeppe16}~\cite{Dejemeppe16}, \href{../works/HeinzSB13.pdf}{HeinzSB13}~\cite{HeinzSB13}, \href{../works/KelbelH11.pdf}{KelbelH11}~\cite{KelbelH11}, \href{../works/Beck10.pdf}{Beck10}~\cite{Beck10}, \href{../works/MilanoW09.pdf}{MilanoW09}~\cite{MilanoW09}, \href{../works/Laborie09.pdf}{Laborie09}~\cite{Laborie09}...\href{../works/Hooker05a.pdf}{Hooker05a}~\cite{Hooker05a}, \href{../works/PolicellaWSO05.pdf}{PolicellaWSO05}~\cite{PolicellaWSO05}, \href{../works/Sadykov04.pdf}{Sadykov04}~\cite{Sadykov04}, \href{../works/HarjunkoskiG02.pdf}{HarjunkoskiG02}~\cite{HarjunkoskiG02}, \href{../works/JainG01.pdf}{JainG01}~\cite{JainG01}, \href{../works/BosiM2001.pdf}{BosiM2001}~\cite{BosiM2001}, \href{../works/SourdN00.pdf}{SourdN00}~\cite{SourdN00}, \href{../works/TorresL00.pdf}{TorresL00}~\cite{TorresL00}, \href{../works/Beck99.pdf}{Beck99}~\cite{Beck99}, \href{../works/BeckDDF98.pdf}{BeckDDF98}~\cite{BeckDDF98} (Total: 36) & \href{../works/ForbesHJST24.pdf}{ForbesHJST24}~\cite{ForbesHJST24}, \href{../works/Adelgren2023.pdf}{Adelgren2023}~\cite{Adelgren2023}, \href{../works/PovedaAA23.pdf}{PovedaAA23}~\cite{PovedaAA23}, \href{../works/PenzDN23.pdf}{PenzDN23}~\cite{PenzDN23}, \href{../works/IsikYA23.pdf}{IsikYA23}~\cite{IsikYA23}, \href{../works/YuraszeckMC23.pdf}{YuraszeckMC23}~\cite{YuraszeckMC23}, \href{../works/PohlAK22.pdf}{PohlAK22}~\cite{PohlAK22}, \href{../works/TouatBT22.pdf}{TouatBT22}~\cite{TouatBT22}, \href{../works/AwadMDMT22.pdf}{AwadMDMT22}~\cite{AwadMDMT22}, \href{../works/Zahout21.pdf}{Zahout21}~\cite{Zahout21}, \href{../works/AntuoriHHEN21.pdf}{AntuoriHHEN21}~\cite{AntuoriHHEN21}, \href{../works/GeibingerMM21.pdf}{GeibingerMM21}~\cite{GeibingerMM21}, \href{../works/HillTV21.pdf}{HillTV21}~\cite{HillTV21}, \href{../works/AbreuAPNM21.pdf}{AbreuAPNM21}~\cite{AbreuAPNM21}, \href{../works/Astrand21.pdf}{Astrand21}~\cite{Astrand21}, \href{../works/ZhangYW21.pdf}{ZhangYW21}~\cite{ZhangYW21}, \href{../works/KovacsTKSG21.pdf}{KovacsTKSG21}~\cite{KovacsTKSG21}, \href{../works/GodetLHS20.pdf}{GodetLHS20}~\cite{GodetLHS20}, \href{../works/MejiaY20.pdf}{MejiaY20}~\cite{MejiaY20}...\href{../works/Junker00.pdf}{Junker00}~\cite{Junker00}, \href{../works/BruckerK00.pdf}{BruckerK00}~\cite{BruckerK00}, \href{../works/JainM99.pdf}{JainM99}~\cite{JainM99}, \href{../works/CestaOF99.pdf}{CestaOF99}~\cite{CestaOF99}, \href{../works/BelhadjiI98.pdf}{BelhadjiI98}~\cite{BelhadjiI98}, \href{../works/BeckDF97.pdf}{BeckDF97}~\cite{BeckDF97}, \href{../works/BeckDSF97.pdf}{BeckDSF97}~\cite{BeckDSF97}, \href{../works/BeckDSF97a.pdf}{BeckDSF97a}~\cite{BeckDSF97a}, \href{../works/OddiS97.pdf}{OddiS97}~\cite{OddiS97}, \href{../works/BaptisteP97.pdf}{BaptisteP97}~\cite{BaptisteP97} (Total: 104)\\
\index{sequence dependent setup}\index{Concepts!sequence dependent setup}sequence dependent setup &  1.00 & \href{../works/Groleaz21.pdf}{Groleaz21}~\cite{Groleaz21}, \href{../works/GedikKEK18.pdf}{GedikKEK18}~\cite{GedikKEK18}, \href{../works/TranAB16.pdf}{TranAB16}~\cite{TranAB16}, \href{../works/HamC16.pdf}{HamC16}~\cite{HamC16}, \href{../works/TranB12.pdf}{TranB12}~\cite{TranB12}, \href{../works/Wolf11.pdf}{Wolf11}~\cite{Wolf11}, \href{../works/ArtiguesF07.pdf}{ArtiguesF07}~\cite{ArtiguesF07}, \href{../works/FocacciLN00.pdf}{FocacciLN00}~\cite{FocacciLN00} & \href{../works/IsikYA23.pdf}{IsikYA23}~\cite{IsikYA23}, \href{../works/YuraszeckMPV22.pdf}{YuraszeckMPV22}~\cite{YuraszeckMPV22}, \href{../works/GeitzGSSW22.pdf}{GeitzGSSW22}~\cite{GeitzGSSW22}, \href{../works/MengLZB21.pdf}{MengLZB21}~\cite{MengLZB21}, \href{../works/CauwelaertDS20.pdf}{CauwelaertDS20}~\cite{CauwelaertDS20}, \href{../works/MengZRZL20.pdf}{MengZRZL20}~\cite{MengZRZL20}, \href{../works/ZarandiASC20.pdf}{ZarandiASC20}~\cite{ZarandiASC20}, \href{../works/RiahiNS018.pdf}{RiahiNS018}~\cite{RiahiNS018}, \href{../works/Dejemeppe16.pdf}{Dejemeppe16}~\cite{Dejemeppe16}, \href{../works/GrimesH15.pdf}{GrimesH15}~\cite{GrimesH15}, \href{../works/LombardiM12.pdf}{LombardiM12}~\cite{LombardiM12}, \href{../works/Simonis07.pdf}{Simonis07}~\cite{Simonis07}, \href{../works/ArtiguesBF04.pdf}{ArtiguesBF04}~\cite{ArtiguesBF04} & \href{../works/PrataAN23.pdf}{PrataAN23}~\cite{PrataAN23}, \href{../works/GuoZ23.pdf}{GuoZ23}~\cite{GuoZ23}, \href{../works/abs-2305-19888.pdf}{abs-2305-19888}~\cite{abs-2305-19888}, \href{../works/Adelgren2023.pdf}{Adelgren2023}~\cite{Adelgren2023}, \href{../works/NaderiRR23.pdf}{NaderiRR23}~\cite{NaderiRR23}, \href{../works/YunusogluY22.pdf}{YunusogluY22}~\cite{YunusogluY22}, \href{../works/PohlAK22.pdf}{PohlAK22}~\cite{PohlAK22}, \href{../works/HeinzNVH22.pdf}{HeinzNVH22}~\cite{HeinzNVH22}, \href{../works/OujanaAYB22.pdf}{OujanaAYB22}~\cite{OujanaAYB22}, \href{../works/NaderiBZ22a.pdf}{NaderiBZ22a}~\cite{NaderiBZ22a}, \href{../works/Astrand21.pdf}{Astrand21}~\cite{Astrand21}, \href{../works/ArmstrongGOS21.pdf}{ArmstrongGOS21}~\cite{ArmstrongGOS21}, \href{../works/Bedhief21.pdf}{Bedhief21}~\cite{Bedhief21}, \href{../works/HamPK21.pdf}{HamPK21}~\cite{HamPK21}, \href{../works/Mercier-AubinGQ20.pdf}{Mercier-AubinGQ20}~\cite{Mercier-AubinGQ20}, \href{../works/MejiaY20.pdf}{MejiaY20}~\cite{MejiaY20}, \href{../works/RoshanaeiBAUB20.pdf}{RoshanaeiBAUB20}~\cite{RoshanaeiBAUB20}, \href{../works/MalapertN19.pdf}{MalapertN19}~\cite{MalapertN19}, \href{../works/KonowalenkoMM19.pdf}{KonowalenkoMM19}~\cite{KonowalenkoMM19}...\href{../works/KovacsK11.pdf}{KovacsK11}~\cite{KovacsK11}, \href{../works/GrimesH10.pdf}{GrimesH10}~\cite{GrimesH10}, \href{../works/Laborie09.pdf}{Laborie09}~\cite{Laborie09}, \href{../works/Jans09.pdf}{Jans09}~\cite{Jans09}, \href{../works/AkkerDH07.pdf}{AkkerDH07}~\cite{AkkerDH07}, \href{../works/DavenportKRSH07.pdf}{DavenportKRSH07}~\cite{DavenportKRSH07}, \href{../works/VilimBC05.pdf}{VilimBC05}~\cite{VilimBC05}, \href{../works/Vilim04.pdf}{Vilim04}~\cite{Vilim04}, \href{../works/Vilim02.pdf}{Vilim02}~\cite{Vilim02}, \href{../works/Baptiste02.pdf}{Baptiste02}~\cite{Baptiste02} (Total: 54)\\
\index{setup-time}\index{Concepts!setup-time}setup-time &  1.00 & \href{../works/PrataAN23.pdf}{PrataAN23}~\cite{PrataAN23}, \href{../works/NaderiBZ23.pdf}{NaderiBZ23}~\cite{NaderiBZ23}, \href{../works/AbreuPNF23.pdf}{AbreuPNF23}~\cite{AbreuPNF23}, \href{../works/abs-2305-19888.pdf}{abs-2305-19888}~\cite{abs-2305-19888}, \href{../works/AbreuNP23.pdf}{AbreuNP23}~\cite{AbreuNP23}, \href{../works/NaderiRR23.pdf}{NaderiRR23}~\cite{NaderiRR23}, \href{../works/IsikYA23.pdf}{IsikYA23}~\cite{IsikYA23}, \href{../works/LacknerMMWW23.pdf}{LacknerMMWW23}~\cite{LacknerMMWW23}, \href{../works/ColT22.pdf}{ColT22}~\cite{ColT22}, \href{../works/NaderiBZ22.pdf}{NaderiBZ22}~\cite{NaderiBZ22}, \href{../works/WinterMMW22.pdf}{WinterMMW22}~\cite{WinterMMW22}, \href{../works/CilKLO22.pdf}{CilKLO22}~\cite{CilKLO22}, \href{../works/YuraszeckMPV22.pdf}{YuraszeckMPV22}~\cite{YuraszeckMPV22}, \href{../works/MengGRZSC22.pdf}{MengGRZSC22}~\cite{MengGRZSC22}, \href{../works/GeitzGSSW22.pdf}{GeitzGSSW22}~\cite{GeitzGSSW22}, \href{../works/OujanaAYB22.pdf}{OujanaAYB22}~\cite{OujanaAYB22}, \href{../works/YunusogluY22.pdf}{YunusogluY22}~\cite{YunusogluY22}, \href{../works/PohlAK22.pdf}{PohlAK22}~\cite{PohlAK22}, \href{../works/HeinzNVH22.pdf}{HeinzNVH22}~\cite{HeinzNVH22}...\href{../works/ZeballosNH11.pdf}{ZeballosNH11}~\cite{ZeballosNH11}, \href{../works/GrimesH10.pdf}{GrimesH10}~\cite{GrimesH10}, \href{../works/OzturkTHO10.pdf}{OzturkTHO10}~\cite{OzturkTHO10}, \href{../works/Jans09.pdf}{Jans09}~\cite{Jans09}, \href{../works/Simonis07.pdf}{Simonis07}~\cite{Simonis07}, \href{../works/DavenportKRSH07.pdf}{DavenportKRSH07}~\cite{DavenportKRSH07}, \href{../works/ArtiguesF07.pdf}{ArtiguesF07}~\cite{ArtiguesF07}, \href{../works/ArtiguesBF04.pdf}{ArtiguesBF04}~\cite{ArtiguesBF04}, \href{../works/MeyerE04.pdf}{MeyerE04}~\cite{MeyerE04}, \href{../works/Baptiste02.pdf}{Baptiste02}~\cite{Baptiste02} (Total: 74) & \href{../works/Adelgren2023.pdf}{Adelgren2023}~\cite{Adelgren2023}, \href{../works/PenzDN23.pdf}{PenzDN23}~\cite{PenzDN23}, \href{../works/GokPTGO23.pdf}{GokPTGO23}~\cite{GokPTGO23}, \href{../works/ZhuSZW23.pdf}{ZhuSZW23}~\cite{ZhuSZW23}, \href{../works/AlfieriGPS23.pdf}{AlfieriGPS23}~\cite{AlfieriGPS23}, \href{../works/CzerniachowskaWZ23.pdf}{CzerniachowskaWZ23}~\cite{CzerniachowskaWZ23}, \href{../works/KimCMLLP23.pdf}{KimCMLLP23}~\cite{KimCMLLP23}, \href{../works/LiFJZLL22.pdf}{LiFJZLL22}~\cite{LiFJZLL22}, \href{../works/Bedhief21.pdf}{Bedhief21}~\cite{Bedhief21}, \href{../works/ArmstrongGOS21.pdf}{ArmstrongGOS21}~\cite{ArmstrongGOS21}, \href{../works/FanXG21.pdf}{FanXG21}~\cite{FanXG21}, \href{../works/AbreuAPNM21.pdf}{AbreuAPNM21}~\cite{AbreuAPNM21}, \href{../works/AstrandJZ20.pdf}{AstrandJZ20}~\cite{AstrandJZ20}, \href{../works/LaborieRSV18.pdf}{LaborieRSV18}~\cite{LaborieRSV18}, \href{../works/HookerH17.pdf}{HookerH17}~\cite{HookerH17}, \href{../works/NovaraNH16.pdf}{NovaraNH16}~\cite{NovaraNH16}, \href{../works/OrnekO16.pdf}{OrnekO16}~\cite{OrnekO16}, \href{../works/HamC16.pdf}{HamC16}~\cite{HamC16}, \href{../works/GaySS14.pdf}{GaySS14}~\cite{GaySS14}...\href{../works/KelarevaTK13.pdf}{KelarevaTK13}~\cite{KelarevaTK13}, \href{../works/OzturkTHO13.pdf}{OzturkTHO13}~\cite{OzturkTHO13}, \href{../works/ZampelliVSDR13.pdf}{ZampelliVSDR13}~\cite{ZampelliVSDR13}, \href{../works/Malapert11.pdf}{Malapert11}~\cite{Malapert11}, \href{../works/Wolf11.pdf}{Wolf11}~\cite{Wolf11}, \href{../works/ThiruvadyBME09.pdf}{ThiruvadyBME09}~\cite{ThiruvadyBME09}, \href{../works/BeniniBGM06.pdf}{BeniniBGM06}~\cite{BeniniBGM06}, \href{../works/HarjunkoskiG02.pdf}{HarjunkoskiG02}~\cite{HarjunkoskiG02}, \href{../works/Timpe02.pdf}{Timpe02}~\cite{Timpe02}, \href{../works/Vilim02.pdf}{Vilim02}~\cite{Vilim02} (Total: 31) & \href{../works/AfsarVPG23.pdf}{AfsarVPG23}~\cite{AfsarVPG23}, \href{../works/Mehdizadeh-Somarin23.pdf}{Mehdizadeh-Somarin23}~\cite{Mehdizadeh-Somarin23}, \href{../works/GuoZ23.pdf}{GuoZ23}~\cite{GuoZ23}, \href{../works/NaderiBZR23.pdf}{NaderiBZR23}~\cite{NaderiBZR23}, \href{../works/EfthymiouY23.pdf}{EfthymiouY23}~\cite{EfthymiouY23}, \href{../works/YuraszeckMCCR23.pdf}{YuraszeckMCCR23}~\cite{YuraszeckMCCR23}, \href{../works/JuvinHL23.pdf}{JuvinHL23}~\cite{JuvinHL23}, \href{../works/JuvinHL23a.pdf}{JuvinHL23a}~\cite{JuvinHL23a}, \href{../works/Fatemi-AnarakiTFV23.pdf}{Fatemi-AnarakiTFV23}~\cite{Fatemi-AnarakiTFV23}, \href{../works/FrimodigECM23.pdf}{FrimodigECM23}~\cite{FrimodigECM23}, \href{../works/JuvinHHL23.pdf}{JuvinHHL23}~\cite{JuvinHHL23}, \href{../works/abs-2211-14492.pdf}{abs-2211-14492}~\cite{abs-2211-14492}, \href{../works/ZhangJZL22.pdf}{ZhangJZL22}~\cite{ZhangJZL22}, \href{../works/Teppan22.pdf}{Teppan22}~\cite{Teppan22}, \href{../works/NaderiBZ22a.pdf}{NaderiBZ22a}~\cite{NaderiBZ22a}, \href{../works/GhandehariK22.pdf}{GhandehariK22}~\cite{GhandehariK22}, \href{../works/JuvinHL22.pdf}{JuvinHL22}~\cite{JuvinHL22}, \href{../works/MullerMKP22.pdf}{MullerMKP22}~\cite{MullerMKP22}, \href{../works/ZhangYW21.pdf}{ZhangYW21}~\cite{ZhangYW21}...\href{../works/LorigeonBB02.pdf}{LorigeonBB02}~\cite{LorigeonBB02}, \href{../works/Dorndorf2000.pdf}{Dorndorf2000}~\cite{Dorndorf2000}, \href{../works/SourdN00.pdf}{SourdN00}~\cite{SourdN00}, \href{../works/ArtiguesR00.pdf}{ArtiguesR00}~\cite{ArtiguesR00}, \href{../works/BaptistePN99.pdf}{BaptistePN99}~\cite{BaptistePN99}, \href{../works/PapaB98.pdf}{PapaB98}~\cite{PapaB98}, \href{../works/NuijtenP98.pdf}{NuijtenP98}~\cite{NuijtenP98}, \href{../works/PintoG97.pdf}{PintoG97}~\cite{PintoG97}, \href{../works/BlazewiczDP96.pdf}{BlazewiczDP96}~\cite{BlazewiczDP96}, \href{../works/ErtlK91.pdf}{ErtlK91}~\cite{ErtlK91} (Total: 101)\\
\index{single-machine scheduling}\index{Concepts!single-machine scheduling}single-machine scheduling &  1.00 & \href{../works/PenzDN23.pdf}{PenzDN23}~\cite{PenzDN23}, \href{../works/TouatBT22.pdf}{TouatBT22}~\cite{TouatBT22}, \href{../works/ZarandiASC20.pdf}{ZarandiASC20}~\cite{ZarandiASC20}, \href{../works/BajestaniB15.pdf}{BajestaniB15}~\cite{BajestaniB15} & \href{../works/PrataAN23.pdf}{PrataAN23}~\cite{PrataAN23}, \href{../works/AlfieriGPS23.pdf}{AlfieriGPS23}~\cite{AlfieriGPS23}, \href{../works/Groleaz21.pdf}{Groleaz21}~\cite{Groleaz21}, \href{../works/BenediktMH20.pdf}{BenediktMH20}~\cite{BenediktMH20}, \href{../works/BogaerdtW19.pdf}{BogaerdtW19}~\cite{BogaerdtW19}, \href{../works/TerekhovDOB12.pdf}{TerekhovDOB12}~\cite{TerekhovDOB12}, \href{../works/KovacsB11.pdf}{KovacsB11}~\cite{KovacsB11}, \href{../works/WuBB09.pdf}{WuBB09}~\cite{WuBB09}, \href{../works/JainM99.pdf}{JainM99}~\cite{JainM99} & \href{../works/LuZZYW24.pdf}{LuZZYW24}~\cite{LuZZYW24}, \href{../works/BonninMNE24.pdf}{BonninMNE24}~\cite{BonninMNE24}, \href{../works/Fatemi-AnarakiTFV23.pdf}{Fatemi-AnarakiTFV23}~\cite{Fatemi-AnarakiTFV23}, \href{../works/Mehdizadeh-Somarin23.pdf}{Mehdizadeh-Somarin23}~\cite{Mehdizadeh-Somarin23}, \href{../works/ZhangJZL22.pdf}{ZhangJZL22}~\cite{ZhangJZL22}, \href{../works/PohlAK22.pdf}{PohlAK22}~\cite{PohlAK22}, \href{../works/ElciOH22.pdf}{ElciOH22}~\cite{ElciOH22}, \href{../works/EmdeZD22.pdf}{EmdeZD22}~\cite{EmdeZD22}, \href{../works/HillTV21.pdf}{HillTV21}~\cite{HillTV21}, \href{../works/QinWSLS21.pdf}{QinWSLS21}~\cite{QinWSLS21}, \href{../works/KoehlerBFFHPSSS21.pdf}{KoehlerBFFHPSSS21}~\cite{KoehlerBFFHPSSS21}, \href{../works/HamPK21.pdf}{HamPK21}~\cite{HamPK21}, \href{../works/PandeyS21a.pdf}{PandeyS21a}~\cite{PandeyS21a}, \href{../works/NattafHKAL19.pdf}{NattafHKAL19}~\cite{NattafHKAL19}, \href{../works/NattafDYW19.pdf}{NattafDYW19}~\cite{NattafDYW19}, \href{../works/Tom19.pdf}{Tom19}~\cite{Tom19}, \href{../works/Hooker19.pdf}{Hooker19}~\cite{Hooker19}, \href{../works/MalapertN19.pdf}{MalapertN19}~\cite{MalapertN19}, \href{../works/BenediktSMVH18.pdf}{BenediktSMVH18}~\cite{BenediktSMVH18}...\href{../works/ChuX05.pdf}{ChuX05}~\cite{ChuX05}, \href{../works/KanetAG04.pdf}{KanetAG04}~\cite{KanetAG04}, \href{../works/Demassey03.pdf}{Demassey03}~\cite{Demassey03}, \href{../works/OddiPCC03.pdf}{OddiPCC03}~\cite{OddiPCC03}, \href{../works/Baptiste02.pdf}{Baptiste02}~\cite{Baptiste02}, \href{../works/BosiM2001.pdf}{BosiM2001}~\cite{BosiM2001}, \href{../works/Dorndorf2000.pdf}{Dorndorf2000}~\cite{Dorndorf2000}, \href{../works/SakkoutW00.pdf}{SakkoutW00}~\cite{SakkoutW00}, \href{../works/NuijtenP98.pdf}{NuijtenP98}~\cite{NuijtenP98}, \href{../works/BlazewiczDP96.pdf}{BlazewiczDP96}~\cite{BlazewiczDP96} (Total: 52)\\
\index{single-stage scheduling}\index{Concepts!single-stage scheduling}single-stage scheduling &  1.00 &  & \href{../works/HarjunkoskiG02.pdf}{HarjunkoskiG02}~\cite{HarjunkoskiG02} & \href{../works/TerekhovDOB12.pdf}{TerekhovDOB12}~\cite{TerekhovDOB12}\\
\index{stochastic}\index{Concepts!stochastic}stochastic &  1.00 & \href{../works/ForbesHJST24.pdf}{ForbesHJST24}~\cite{ForbesHJST24}, \href{../works/NaderiBZ23.pdf}{NaderiBZ23}~\cite{NaderiBZ23}, \href{../works/AfsarVPG23.pdf}{AfsarVPG23}~\cite{AfsarVPG23}, \href{../works/GuoZ23.pdf}{GuoZ23}~\cite{GuoZ23}, \href{../works/NaderiBZR23.pdf}{NaderiBZR23}~\cite{NaderiBZR23}, \href{../works/AlakaP23.pdf}{AlakaP23}~\cite{AlakaP23}, \href{../works/GurPAE23.pdf}{GurPAE23}~\cite{GurPAE23}, \href{../works/GokPTGO23.pdf}{GokPTGO23}~\cite{GokPTGO23}, \href{../works/PenzDN23.pdf}{PenzDN23}~\cite{PenzDN23}, \href{../works/NaderiBZ22.pdf}{NaderiBZ22}~\cite{NaderiBZ22}, \href{../works/SubulanC22.pdf}{SubulanC22}~\cite{SubulanC22}, \href{../works/ElciOH22.pdf}{ElciOH22}~\cite{ElciOH22}, \href{../works/GhandehariK22.pdf}{GhandehariK22}~\cite{GhandehariK22}, \href{../works/Astrand21.pdf}{Astrand21}~\cite{Astrand21}, \href{../works/Edis21.pdf}{Edis21}~\cite{Edis21}, \href{../works/Groleaz21.pdf}{Groleaz21}~\cite{Groleaz21}, \href{../works/RoshanaeiBAUB20.pdf}{RoshanaeiBAUB20}~\cite{RoshanaeiBAUB20}, \href{../works/AntuoriHHEN20.pdf}{AntuoriHHEN20}~\cite{AntuoriHHEN20}, \href{../works/ZarandiASC20.pdf}{ZarandiASC20}~\cite{ZarandiASC20}...\href{../works/MilanoW09.pdf}{MilanoW09}~\cite{MilanoW09}, \href{../works/LombardiM09.pdf}{LombardiM09}~\cite{LombardiM09}, \href{../works/DoomsH08.pdf}{DoomsH08}~\cite{DoomsH08}, \href{../works/RossiTHP07.pdf}{RossiTHP07}~\cite{RossiTHP07}, \href{../works/BeckW07.pdf}{BeckW07}~\cite{BeckW07}, \href{../works/BidotVLB07.pdf}{BidotVLB07}~\cite{BidotVLB07}, \href{../works/MilanoW06.pdf}{MilanoW06}~\cite{MilanoW06}, \href{../works/FrankK05.pdf}{FrankK05}~\cite{FrankK05}, \href{../works/WatsonBHW99.pdf}{WatsonBHW99}~\cite{WatsonBHW99}, \href{../works/OddiS97.pdf}{OddiS97}~\cite{OddiS97} (Total: 48) & \href{../works/YuraszeckMC23.pdf}{YuraszeckMC23}~\cite{YuraszeckMC23}, \href{../works/FrimodigECM23.pdf}{FrimodigECM23}~\cite{FrimodigECM23}, \href{../works/FarsiTM22.pdf}{FarsiTM22}~\cite{FarsiTM22}, \href{../works/OrnekOS20.pdf}{OrnekOS20}~\cite{OrnekOS20}, \href{../works/YuraszeckMPV22.pdf}{YuraszeckMPV22}~\cite{YuraszeckMPV22}, \href{../works/CilKLO22.pdf}{CilKLO22}~\cite{CilKLO22}, \href{../works/HubnerGSV21.pdf}{HubnerGSV21}~\cite{HubnerGSV21}, \href{../works/Alaka21.pdf}{Alaka21}~\cite{Alaka21}, \href{../works/AntuoriHHEN21.pdf}{AntuoriHHEN21}~\cite{AntuoriHHEN21}, \href{../works/AstrandJZ20.pdf}{AstrandJZ20}~\cite{AstrandJZ20}, \href{../works/Lunardi20.pdf}{Lunardi20}~\cite{Lunardi20}, \href{../works/SacramentoSP20.pdf}{SacramentoSP20}~\cite{SacramentoSP20}, \href{../works/GurEA19.pdf}{GurEA19}~\cite{GurEA19}, \href{../works/AlakaPY19.pdf}{AlakaPY19}~\cite{AlakaPY19}, \href{../works/ParkUJR19.pdf}{ParkUJR19}~\cite{ParkUJR19}, \href{../works/FrimodigS19.pdf}{FrimodigS19}~\cite{FrimodigS19}, \href{../works/SenderovichBB19.pdf}{SenderovichBB19}~\cite{SenderovichBB19}, \href{../works/LaborieRSV18.pdf}{LaborieRSV18}~\cite{LaborieRSV18}, \href{../works/ZhangW18.pdf}{ZhangW18}~\cite{ZhangW18}...\href{../works/LombardiM10.pdf}{LombardiM10}~\cite{LombardiM10}, \href{../works/WuBB09.pdf}{WuBB09}~\cite{WuBB09}, \href{../works/DilkinaDH05.pdf}{DilkinaDH05}~\cite{DilkinaDH05}, \href{../works/BeckW04.pdf}{BeckW04}~\cite{BeckW04}, \href{../works/GlobusCLP04.pdf}{GlobusCLP04}~\cite{GlobusCLP04}, \href{../works/FrankK03.pdf}{FrankK03}~\cite{FrankK03}, \href{../works/JainM99.pdf}{JainM99}~\cite{JainM99}, \href{../works/SadehF96.pdf}{SadehF96}~\cite{SadehF96}, \href{../works/Nuijten94.pdf}{Nuijten94}~\cite{Nuijten94}, \href{../works/MintonJPL92.pdf}{MintonJPL92}~\cite{MintonJPL92} (Total: 40) & \href{../works/PrataAN23.pdf}{PrataAN23}~\cite{PrataAN23}, \href{../works/LuZZYW24.pdf}{LuZZYW24}~\cite{LuZZYW24}, \href{../works/AlfieriGPS23.pdf}{AlfieriGPS23}~\cite{AlfieriGPS23}, \href{../works/AbreuPNF23.pdf}{AbreuPNF23}~\cite{AbreuPNF23}, \href{../works/Mehdizadeh-Somarin23.pdf}{Mehdizadeh-Somarin23}~\cite{Mehdizadeh-Somarin23}, \href{../works/JuvinHL23.pdf}{JuvinHL23}~\cite{JuvinHL23}, \href{../works/JuvinHL23a.pdf}{JuvinHL23a}~\cite{JuvinHL23a}, \href{../works/AbreuNP23.pdf}{AbreuNP23}~\cite{AbreuNP23}, \href{../works/IklassovMR023.pdf}{IklassovMR023}~\cite{IklassovMR023}, \href{../works/KotaryFH22.pdf}{KotaryFH22}~\cite{KotaryFH22}, \href{../works/CampeauG22.pdf}{CampeauG22}~\cite{CampeauG22}, \href{../works/ZhangJZL22.pdf}{ZhangJZL22}~\cite{ZhangJZL22}, \href{../works/PopovicCGNC22.pdf}{PopovicCGNC22}~\cite{PopovicCGNC22}, \href{../works/LiFJZLL22.pdf}{LiFJZLL22}~\cite{LiFJZLL22}, \href{../works/PohlAK22.pdf}{PohlAK22}~\cite{PohlAK22}, \href{../works/NaderiBZ22a.pdf}{NaderiBZ22a}~\cite{NaderiBZ22a}, \href{../works/AbreuN22.pdf}{AbreuN22}~\cite{AbreuN22}, \href{../works/EmdeZD22.pdf}{EmdeZD22}~\cite{EmdeZD22}, \href{../works/EtminaniesfahaniGNMS22.pdf}{EtminaniesfahaniGNMS22}~\cite{EtminaniesfahaniGNMS22}...\href{../works/Beck99.pdf}{Beck99}~\cite{Beck99}, \href{../works/CestaOF99.pdf}{CestaOF99}~\cite{CestaOF99}, \href{../works/PembertonG98.pdf}{PembertonG98}~\cite{PembertonG98}, \href{../works/BeckDDF98.pdf}{BeckDDF98}~\cite{BeckDDF98}, \href{../works/BlazewiczDP96.pdf}{BlazewiczDP96}~\cite{BlazewiczDP96}, \href{../works/Colombani96.pdf}{Colombani96}~\cite{Colombani96}, \href{../works/Wallace96.pdf}{Wallace96}~\cite{Wallace96}, \href{../works/Pape94.pdf}{Pape94}~\cite{Pape94}, \href{../works/SmithC93.pdf}{SmithC93}~\cite{SmithC93}, \href{../works/FoxS90.pdf}{FoxS90}~\cite{FoxS90} (Total: 121)\\
\index{stock level}\index{Concepts!stock level}stock level &  1.00 & \href{../works/LopesCSM10.pdf}{LopesCSM10}~\cite{LopesCSM10}, \href{../works/SimonisC95.pdf}{SimonisC95}~\cite{SimonisC95} & \href{../works/German18.pdf}{German18}~\cite{German18}, \href{../works/RossiTHP07.pdf}{RossiTHP07}~\cite{RossiTHP07}, \href{../works/Timpe02.pdf}{Timpe02}~\cite{Timpe02}, \href{../works/Simonis99.pdf}{Simonis99}~\cite{Simonis99} & \href{../works/UnsalO19.pdf}{UnsalO19}~\cite{UnsalO19}, \href{../works/QinDS16.pdf}{QinDS16}~\cite{QinDS16}, \href{../works/KhemmoudjPB06.pdf}{KhemmoudjPB06}~\cite{KhemmoudjPB06}, \href{../works/SimonisCK00.pdf}{SimonisCK00}~\cite{SimonisCK00}, \href{../works/Beck99.pdf}{Beck99}~\cite{Beck99}, \href{../works/RoweJCA96.pdf}{RoweJCA96}~\cite{RoweJCA96}, \href{../works/Simonis95a.pdf}{Simonis95a}~\cite{Simonis95a}\\
\index{sustainability}\index{Concepts!sustainability}sustainability &  1.00 &  &  & \href{../works/LuZZYW24.pdf}{LuZZYW24}~\cite{LuZZYW24}, \href{../works/MontemanniD23a.pdf}{MontemanniD23a}~\cite{MontemanniD23a}, \href{../works/AbreuPNF23.pdf}{AbreuPNF23}~\cite{AbreuPNF23}, \href{../works/PenzDN23.pdf}{PenzDN23}~\cite{PenzDN23}, \href{../works/CzerniachowskaWZ23.pdf}{CzerniachowskaWZ23}~\cite{CzerniachowskaWZ23}, \href{../works/MontemanniD23.pdf}{MontemanniD23}~\cite{MontemanniD23}, \href{../works/Mehdizadeh-Somarin23.pdf}{Mehdizadeh-Somarin23}~\cite{Mehdizadeh-Somarin23}, \href{../works/PopovicCGNC22.pdf}{PopovicCGNC22}~\cite{PopovicCGNC22}, \href{../works/MullerMKP22.pdf}{MullerMKP22}~\cite{MullerMKP22}, \href{../works/CilKLO22.pdf}{CilKLO22}~\cite{CilKLO22}, \href{../works/BenediktMH20.pdf}{BenediktMH20}~\cite{BenediktMH20}, \href{../works/HoYCLLCLC18.pdf}{HoYCLLCLC18}~\cite{HoYCLLCLC18}, \href{../works/Froger16.pdf}{Froger16}~\cite{Froger16}, \href{../works/BridiBLMB16.pdf}{BridiBLMB16}~\cite{BridiBLMB16}, \href{../works/Madi-WambaB16.pdf}{Madi-WambaB16}~\cite{Madi-WambaB16}, \href{../works/GrimesIOS14.pdf}{GrimesIOS14}~\cite{GrimesIOS14}, \href{../works/IfrimOS12.pdf}{IfrimOS12}~\cite{IfrimOS12}\\
\index{tardiness}\index{Concepts!tardiness}tardiness &  1.00 & \href{../works/PrataAN23.pdf}{PrataAN23}~\cite{PrataAN23}, \href{../works/NaderiRR23.pdf}{NaderiRR23}~\cite{NaderiRR23}, \href{../works/IsikYA23.pdf}{IsikYA23}~\cite{IsikYA23}, \href{../works/KimCMLLP23.pdf}{KimCMLLP23}~\cite{KimCMLLP23}, \href{../works/AlfieriGPS23.pdf}{AlfieriGPS23}~\cite{AlfieriGPS23}, \href{../works/GokPTGO23.pdf}{GokPTGO23}~\cite{GokPTGO23}, \href{../works/NaderiBZ23.pdf}{NaderiBZ23}~\cite{NaderiBZ23}, \href{../works/LacknerMMWW23.pdf}{LacknerMMWW23}~\cite{LacknerMMWW23}, \href{../works/AbreuPNF23.pdf}{AbreuPNF23}~\cite{AbreuPNF23}, \href{../works/OujanaAYB22.pdf}{OujanaAYB22}~\cite{OujanaAYB22}, \href{../works/AwadMDMT22.pdf}{AwadMDMT22}~\cite{AwadMDMT22}, \href{../works/PohlAK22.pdf}{PohlAK22}~\cite{PohlAK22}, \href{../works/TouatBT22.pdf}{TouatBT22}~\cite{TouatBT22}, \href{../works/AbreuN22.pdf}{AbreuN22}~\cite{AbreuN22}, \href{../works/WinterMMW22.pdf}{WinterMMW22}~\cite{WinterMMW22}, \href{../works/YunusogluY22.pdf}{YunusogluY22}~\cite{YunusogluY22}, \href{../works/NaderiBZ22.pdf}{NaderiBZ22}~\cite{NaderiBZ22}, \href{../works/abs-2211-14492.pdf}{abs-2211-14492}~\cite{abs-2211-14492}, \href{../works/FanXG21.pdf}{FanXG21}~\cite{FanXG21}...\href{../works/KeriK07.pdf}{KeriK07}~\cite{KeriK07}, \href{../works/Hooker07.pdf}{Hooker07}~\cite{Hooker07}, \href{../works/Hooker06.pdf}{Hooker06}~\cite{Hooker06}, \href{../works/ZeballosH05.pdf}{ZeballosH05}~\cite{ZeballosH05}, \href{../works/Hooker05a.pdf}{Hooker05a}~\cite{Hooker05a}, \href{../works/KanetAG04.pdf}{KanetAG04}~\cite{KanetAG04}, \href{../works/HentenryckM04.pdf}{HentenryckM04}~\cite{HentenryckM04}, \href{../works/BeckR03.pdf}{BeckR03}~\cite{BeckR03}, \href{../works/DannaP03.pdf}{DannaP03}~\cite{DannaP03}, \href{../works/Baptiste02.pdf}{Baptiste02}~\cite{Baptiste02} (Total: 72) & \href{../works/abs-2402-00459.pdf}{abs-2402-00459}~\cite{abs-2402-00459}, \href{../works/AbreuNP23.pdf}{AbreuNP23}~\cite{AbreuNP23}, \href{../works/PenzDN23.pdf}{PenzDN23}~\cite{PenzDN23}, \href{../works/SubulanC22.pdf}{SubulanC22}~\cite{SubulanC22}, \href{../works/FarsiTM22.pdf}{FarsiTM22}~\cite{FarsiTM22}, \href{../works/ElciOH22.pdf}{ElciOH22}~\cite{ElciOH22}, \href{../works/EmdeZD22.pdf}{EmdeZD22}~\cite{EmdeZD22}, \href{../works/ColT22.pdf}{ColT22}~\cite{ColT22}, \href{../works/KovacsTKSG21.pdf}{KovacsTKSG21}~\cite{KovacsTKSG21}, \href{../works/AbreuAPNM21.pdf}{AbreuAPNM21}~\cite{AbreuAPNM21}, \href{../works/GroleazNS20.pdf}{GroleazNS20}~\cite{GroleazNS20}, \href{../works/GokGSTO20.pdf}{GokGSTO20}~\cite{GokGSTO20}, \href{../works/Lunardi20.pdf}{Lunardi20}~\cite{Lunardi20}, \href{../works/GokgurHO18.pdf}{GokgurHO18}~\cite{GokgurHO18}, \href{../works/GedikKEK18.pdf}{GedikKEK18}~\cite{GedikKEK18}, \href{../works/Hooker17.pdf}{Hooker17}~\cite{Hooker17}, \href{../works/CireCH16.pdf}{CireCH16}~\cite{CireCH16}, \href{../works/TranAB16.pdf}{TranAB16}~\cite{TranAB16}, \href{../works/ThiruvadyWGS14.pdf}{ThiruvadyWGS14}~\cite{ThiruvadyWGS14}...\href{../works/NovasH10.pdf}{NovasH10}~\cite{NovasH10}, \href{../works/BartakSR10.pdf}{BartakSR10}~\cite{BartakSR10}, \href{../works/Beck06.pdf}{Beck06}~\cite{Beck06}, \href{../works/QuirogaZH05.pdf}{QuirogaZH05}~\cite{QuirogaZH05}, \href{../works/GodardLN05.pdf}{GodardLN05}~\cite{GodardLN05}, \href{../works/Hooker05.pdf}{Hooker05}~\cite{Hooker05}, \href{../works/MeyerE04.pdf}{MeyerE04}~\cite{MeyerE04}, \href{../works/PerronSF04.pdf}{PerronSF04}~\cite{PerronSF04}, \href{../works/BeckPS03.pdf}{BeckPS03}~\cite{BeckPS03}, \href{../works/SmithC93.pdf}{SmithC93}~\cite{SmithC93} (Total: 35) & \href{../works/JuvinHL23.pdf}{JuvinHL23}~\cite{JuvinHL23}, \href{../works/Mehdizadeh-Somarin23.pdf}{Mehdizadeh-Somarin23}~\cite{Mehdizadeh-Somarin23}, \href{../works/TasselGS23.pdf}{TasselGS23}~\cite{TasselGS23}, \href{../works/IklassovMR023.pdf}{IklassovMR023}~\cite{IklassovMR023}, \href{../works/abs-2306-05747.pdf}{abs-2306-05747}~\cite{abs-2306-05747}, \href{../works/LiFJZLL22.pdf}{LiFJZLL22}~\cite{LiFJZLL22}, \href{../works/ZhangJZL22.pdf}{ZhangJZL22}~\cite{ZhangJZL22}, \href{../works/GhandehariK22.pdf}{GhandehariK22}~\cite{GhandehariK22}, \href{../works/EtminaniesfahaniGNMS22.pdf}{EtminaniesfahaniGNMS22}~\cite{EtminaniesfahaniGNMS22}, \href{../works/NaderiBZ22a.pdf}{NaderiBZ22a}~\cite{NaderiBZ22a}, \href{../works/MengGRZSC22.pdf}{MengGRZSC22}~\cite{MengGRZSC22}, \href{../works/VlkHT21.pdf}{VlkHT21}~\cite{VlkHT21}, \href{../works/KoehlerBFFHPSSS21.pdf}{KoehlerBFFHPSSS21}~\cite{KoehlerBFFHPSSS21}, \href{../works/HanenKP21.pdf}{HanenKP21}~\cite{HanenKP21}, \href{../works/HamPK21.pdf}{HamPK21}~\cite{HamPK21}, \href{../works/Astrand21.pdf}{Astrand21}~\cite{Astrand21}, \href{../works/HubnerGSV21.pdf}{HubnerGSV21}~\cite{HubnerGSV21}, \href{../works/Bedhief21.pdf}{Bedhief21}~\cite{Bedhief21}, \href{../works/GeibingerMM21.pdf}{GeibingerMM21}~\cite{GeibingerMM21}...\href{../works/RoePS05.pdf}{RoePS05}~\cite{RoePS05}, \href{../works/BeniniBGM05.pdf}{BeniniBGM05}~\cite{BeniniBGM05}, \href{../works/Hooker04.pdf}{Hooker04}~\cite{Hooker04}, \href{../works/Elkhyari03.pdf}{Elkhyari03}~\cite{Elkhyari03}, \href{../works/Bartak02a.pdf}{Bartak02a}~\cite{Bartak02a}, \href{../works/Beck99.pdf}{Beck99}~\cite{Beck99}, \href{../works/JainM99.pdf}{JainM99}~\cite{JainM99}, \href{../works/BeckF98.pdf}{BeckF98}~\cite{BeckF98}, \href{../works/SadehF96.pdf}{SadehF96}~\cite{SadehF96}, \href{../works/Nuijten94.pdf}{Nuijten94}~\cite{Nuijten94} (Total: 96)\\
\index{temporal constraint reasoning}\index{Concepts!temporal constraint reasoning}temporal constraint reasoning &  1.00 &  &  & \href{../works/BartakSR10.pdf}{BartakSR10}~\cite{BartakSR10}, \href{../works/KeriK07.pdf}{KeriK07}~\cite{KeriK07}, \href{../works/FortinZDF05.pdf}{FortinZDF05}~\cite{FortinZDF05}\\
\index{transportation}\index{Concepts!transportation}transportation &  1.00 & \href{../works/LuZZYW24.pdf}{LuZZYW24}~\cite{LuZZYW24}, \href{../works/MarliereSPR23.pdf}{MarliereSPR23}~\cite{MarliereSPR23}, \href{../works/CzerniachowskaWZ23.pdf}{CzerniachowskaWZ23}~\cite{CzerniachowskaWZ23}, \href{../works/GuoZ23.pdf}{GuoZ23}~\cite{GuoZ23}, \href{../works/PohlAK22.pdf}{PohlAK22}~\cite{PohlAK22}, \href{../works/ArmstrongGOS22.pdf}{ArmstrongGOS22}~\cite{ArmstrongGOS22}, \href{../works/EmdeZD22.pdf}{EmdeZD22}~\cite{EmdeZD22}, \href{../works/GeitzGSSW22.pdf}{GeitzGSSW22}~\cite{GeitzGSSW22}, \href{../works/BourreauGGLT22.pdf}{BourreauGGLT22}~\cite{BourreauGGLT22}, \href{../works/Lemos21.pdf}{Lemos21}~\cite{Lemos21}, \href{../works/KletzanderMH21.pdf}{KletzanderMH21}~\cite{KletzanderMH21}, \href{../works/ArmstrongGOS21.pdf}{ArmstrongGOS21}~\cite{ArmstrongGOS21}, \href{../works/ThomasKS20.pdf}{ThomasKS20}~\cite{ThomasKS20}, \href{../works/QinDCS20.pdf}{QinDCS20}~\cite{QinDCS20}, \href{../works/Lunardi20.pdf}{Lunardi20}~\cite{Lunardi20}, \href{../works/FachiniA20.pdf}{FachiniA20}~\cite{FachiniA20}, \href{../works/SacramentoSP20.pdf}{SacramentoSP20}~\cite{SacramentoSP20}, \href{../works/KletzanderM20.pdf}{KletzanderM20}~\cite{KletzanderM20}, \href{../works/TanZWGQ19.pdf}{TanZWGQ19}~\cite{TanZWGQ19}...\href{../works/HachemiGR11.pdf}{HachemiGR11}~\cite{HachemiGR11}, \href{../works/ZeballosCM10.pdf}{ZeballosCM10}~\cite{ZeballosCM10}, \href{../works/LopesCSM10.pdf}{LopesCSM10}~\cite{LopesCSM10}, \href{../works/BocewiczBB09.pdf}{BocewiczBB09}~\cite{BocewiczBB09}, \href{../works/ZeballosM09.pdf}{ZeballosM09}~\cite{ZeballosM09}, \href{../works/MilanoW09.pdf}{MilanoW09}~\cite{MilanoW09}, \href{../works/Rodriguez07.pdf}{Rodriguez07}~\cite{Rodriguez07}, \href{../works/CorreaLR07.pdf}{CorreaLR07}~\cite{CorreaLR07}, \href{../works/MilanoW06.pdf}{MilanoW06}~\cite{MilanoW06}, \href{../works/ZeballosH05.pdf}{ZeballosH05}~\cite{ZeballosH05} (Total: 47) & \href{../works/FalqueALM24.pdf}{FalqueALM24}~\cite{FalqueALM24}, \href{../works/AfsarVPG23.pdf}{AfsarVPG23}~\cite{AfsarVPG23}, \href{../works/NaderiRR23.pdf}{NaderiRR23}~\cite{NaderiRR23}, \href{../works/AbreuPNF23.pdf}{AbreuPNF23}~\cite{AbreuPNF23}, \href{../works/KimCMLLP23.pdf}{KimCMLLP23}~\cite{KimCMLLP23}, \href{../works/Fatemi-AnarakiTFV23.pdf}{Fatemi-AnarakiTFV23}~\cite{Fatemi-AnarakiTFV23}, \href{../works/GokPTGO23.pdf}{GokPTGO23}~\cite{GokPTGO23}, \href{../works/NaderiBZ23.pdf}{NaderiBZ23}~\cite{NaderiBZ23}, \href{../works/MengGRZSC22.pdf}{MengGRZSC22}~\cite{MengGRZSC22}, \href{../works/ElciOH22.pdf}{ElciOH22}~\cite{ElciOH22}, \href{../works/AbreuN22.pdf}{AbreuN22}~\cite{AbreuN22}, \href{../works/SubulanC22.pdf}{SubulanC22}~\cite{SubulanC22}, \href{../works/PopovicCGNC22.pdf}{PopovicCGNC22}~\cite{PopovicCGNC22}, \href{../works/NaderiBZ22.pdf}{NaderiBZ22}~\cite{NaderiBZ22}, \href{../works/Astrand21.pdf}{Astrand21}~\cite{Astrand21}, \href{../works/Godet21a.pdf}{Godet21a}~\cite{Godet21a}, \href{../works/AbohashimaEG21.pdf}{AbohashimaEG21}~\cite{AbohashimaEG21}, \href{../works/FallahiAC20.pdf}{FallahiAC20}~\cite{FallahiAC20}, \href{../works/MengZRZL20.pdf}{MengZRZL20}~\cite{MengZRZL20}...\href{../works/RendlPHPR12.pdf}{RendlPHPR12}~\cite{RendlPHPR12}, \href{../works/Malapert11.pdf}{Malapert11}~\cite{Malapert11}, \href{../works/MakMS10.pdf}{MakMS10}~\cite{MakMS10}, \href{../works/MouraSCL08.pdf}{MouraSCL08}~\cite{MouraSCL08}, \href{../works/MouraSCL08a.pdf}{MouraSCL08a}~\cite{MouraSCL08a}, \href{../works/LimRX04.pdf}{LimRX04}~\cite{LimRX04}, \href{../works/Mason01.pdf}{Mason01}~\cite{Mason01}, \href{../works/ArtiguesR00.pdf}{ArtiguesR00}~\cite{ArtiguesR00}, \href{../works/Simonis99.pdf}{Simonis99}~\cite{Simonis99}, \href{../works/BeckDDF98.pdf}{BeckDDF98}~\cite{BeckDDF98} (Total: 37) & \href{../works/LiLZDZW24.pdf}{LiLZDZW24}~\cite{LiLZDZW24}, \href{../works/NaderiBZR23.pdf}{NaderiBZR23}~\cite{NaderiBZR23}, \href{../works/AalianPG23.pdf}{AalianPG23}~\cite{AalianPG23}, \href{../works/PerezGSL23.pdf}{PerezGSL23}~\cite{PerezGSL23}, \href{../works/AlfieriGPS23.pdf}{AlfieriGPS23}~\cite{AlfieriGPS23}, \href{../works/ZhuSZW23.pdf}{ZhuSZW23}~\cite{ZhuSZW23}, \href{../works/IklassovMR023.pdf}{IklassovMR023}~\cite{IklassovMR023}, \href{../works/WangB23.pdf}{WangB23}~\cite{WangB23}, \href{../works/MontemanniD23a.pdf}{MontemanniD23a}~\cite{MontemanniD23a}, \href{../works/Adelgren2023.pdf}{Adelgren2023}~\cite{Adelgren2023}, \href{../works/IsikYA23.pdf}{IsikYA23}~\cite{IsikYA23}, \href{../works/AbreuNP23.pdf}{AbreuNP23}~\cite{AbreuNP23}, \href{../works/abs-2312-13682.pdf}{abs-2312-13682}~\cite{abs-2312-13682}, \href{../works/NaderiBZ22a.pdf}{NaderiBZ22a}~\cite{NaderiBZ22a}, \href{../works/abs-2211-14492.pdf}{abs-2211-14492}~\cite{abs-2211-14492}, \href{../works/BoudreaultSLQ22.pdf}{BoudreaultSLQ22}~\cite{BoudreaultSLQ22}, \href{../works/ZhangJZL22.pdf}{ZhangJZL22}~\cite{ZhangJZL22}, \href{../works/YuraszeckMPV22.pdf}{YuraszeckMPV22}~\cite{YuraszeckMPV22}, \href{../works/YunusogluY22.pdf}{YunusogluY22}~\cite{YunusogluY22}...\href{../works/BeckF00.pdf}{BeckF00}~\cite{BeckF00}, \href{../works/PesantGPR99.pdf}{PesantGPR99}~\cite{PesantGPR99}, \href{../works/Beck99.pdf}{Beck99}~\cite{Beck99}, \href{../works/KorbaaYG99.pdf}{KorbaaYG99}~\cite{KorbaaYG99}, \href{../works/ChunCTY99.pdf}{ChunCTY99}~\cite{ChunCTY99}, \href{../works/NuijtenP98.pdf}{NuijtenP98}~\cite{NuijtenP98}, \href{../works/RodosekW98.pdf}{RodosekW98}~\cite{RodosekW98}, \href{../works/MorgadoM97.pdf}{MorgadoM97}~\cite{MorgadoM97}, \href{../works/SimonisC95.pdf}{SimonisC95}~\cite{SimonisC95}, \href{../works/Puget95.pdf}{Puget95}~\cite{Puget95} (Total: 114)\\
\index{two-machine scheduling}\index{Concepts!two-machine scheduling}two-machine scheduling &  1.00 &  &  & \href{../works/AbreuNP23.pdf}{AbreuNP23}~\cite{AbreuNP23}\\
\index{two-stage scheduling}\index{Concepts!two-stage scheduling}two-stage scheduling &  1.00 &  &  & \href{../works/Astrand21.pdf}{Astrand21}~\cite{Astrand21}, \href{../works/QinWSLS21.pdf}{QinWSLS21}~\cite{QinWSLS21}, \href{../works/ZarandiASC20.pdf}{ZarandiASC20}~\cite{ZarandiASC20}, \href{../works/ZouZ20.pdf}{ZouZ20}~\cite{ZouZ20}, \href{../works/TangB20.pdf}{TangB20}~\cite{TangB20}, \href{../works/TanZWGQ19.pdf}{TanZWGQ19}~\cite{TanZWGQ19}\\
\index{unavailability}\index{Concepts!unavailability}unavailability &  1.00 & \href{../works/Lemos21.pdf}{Lemos21}~\cite{Lemos21}, \href{../works/Astrand21.pdf}{Astrand21}~\cite{Astrand21}, \href{../works/Lunardi20.pdf}{Lunardi20}~\cite{Lunardi20}, \href{../works/LunardiBLRV20.pdf}{LunardiBLRV20}~\cite{LunardiBLRV20}, \href{../works/ZhangW18.pdf}{ZhangW18}~\cite{ZhangW18}, \href{../works/Froger16.pdf}{Froger16}~\cite{Froger16}, \href{../works/BajestaniB15.pdf}{BajestaniB15}~\cite{BajestaniB15}, \href{../works/UnsalO13.pdf}{UnsalO13}~\cite{UnsalO13}, \href{../works/AkkerDH07.pdf}{AkkerDH07}~\cite{AkkerDH07}, \href{../works/KhemmoudjPB06.pdf}{KhemmoudjPB06}~\cite{KhemmoudjPB06} & \href{../works/Mehdizadeh-Somarin23.pdf}{Mehdizadeh-Somarin23}~\cite{Mehdizadeh-Somarin23}, \href{../works/PenzDN23.pdf}{PenzDN23}~\cite{PenzDN23}, \href{../works/TouatBT22.pdf}{TouatBT22}~\cite{TouatBT22}, \href{../works/KovacsTKSG21.pdf}{KovacsTKSG21}~\cite{KovacsTKSG21}, \href{../works/SerraNM12.pdf}{SerraNM12}~\cite{SerraNM12}, \href{../works/LorigeonBB02.pdf}{LorigeonBB02}~\cite{LorigeonBB02} & \href{../works/WangB23.pdf}{WangB23}~\cite{WangB23}, \href{../works/PovedaAA23.pdf}{PovedaAA23}~\cite{PovedaAA23}, \href{../works/ShaikhK23.pdf}{ShaikhK23}~\cite{ShaikhK23}, \href{../works/FrimodigECM23.pdf}{FrimodigECM23}~\cite{FrimodigECM23}, \href{../works/abs-2305-19888.pdf}{abs-2305-19888}~\cite{abs-2305-19888}, \href{../works/GuoZ23.pdf}{GuoZ23}~\cite{GuoZ23}, \href{../works/HeinzNVH22.pdf}{HeinzNVH22}~\cite{HeinzNVH22}, \href{../works/YunusogluY22.pdf}{YunusogluY22}~\cite{YunusogluY22}, \href{../works/BulckG22.pdf}{BulckG22}~\cite{BulckG22}, \href{../works/NaqviAIAAA22.pdf}{NaqviAIAAA22}~\cite{NaqviAIAAA22}, \href{../works/PandeyS21a.pdf}{PandeyS21a}~\cite{PandeyS21a}, \href{../works/FanXG21.pdf}{FanXG21}~\cite{FanXG21}, \href{../works/WangB20.pdf}{WangB20}~\cite{WangB20}, \href{../works/AstrandJZ20.pdf}{AstrandJZ20}~\cite{AstrandJZ20}, \href{../works/KreterSSZ18.pdf}{KreterSSZ18}~\cite{KreterSSZ18}, \href{../works/ArbaouiY18.pdf}{ArbaouiY18}~\cite{ArbaouiY18}, \href{../works/KreterSS17.pdf}{KreterSS17}~\cite{KreterSS17}, \href{../works/TranVNB17.pdf}{TranVNB17}~\cite{TranVNB17}, \href{../works/BurtLPS15.pdf}{BurtLPS15}~\cite{BurtLPS15}, \href{../works/KreterSS15.pdf}{KreterSS15}~\cite{KreterSS15}, \href{../works/GoelSHFS15.pdf}{GoelSHFS15}~\cite{GoelSHFS15}, \href{../works/NovasH14.pdf}{NovasH14}~\cite{NovasH14}, \href{../works/HarjunkoskiMBC14.pdf}{HarjunkoskiMBC14}~\cite{HarjunkoskiMBC14}, \href{../works/ArtiguesLH13.pdf}{ArtiguesLH13}~\cite{ArtiguesLH13}, \href{../works/GuyonLPR12.pdf}{GuyonLPR12}~\cite{GuyonLPR12}, \href{../works/NovasH10.pdf}{NovasH10}~\cite{NovasH10}, \href{../works/FoxS90.pdf}{FoxS90}~\cite{FoxS90}\\
\end{longtable}
}

\clearpage
\subsection{Concept Type Classification}
\label{sec:Classification}
\label{Classification}
{\scriptsize
\begin{longtable}{p{3cm}r>{\raggedright\arraybackslash}p{6cm}>{\raggedright\arraybackslash}p{6cm}>{\raggedright\arraybackslash}p{8cm}}
\rowcolor{white}\caption{Works for Concepts of Type Classification (Total 48 Concepts, 43 Used)}\\ \toprule
\rowcolor{white}Keyword & Weight & High & Medium & Low\\ \midrule\endhead
\bottomrule
\endfoot
\index{2BPHFSP}\index{Classification!2BPHFSP}2BPHFSP &  1.00 & \href{../works/TangB20.pdf}{TangB20}~\cite{TangB20} &  & \\
\index{BPCTOP}\index{Classification!BPCTOP}BPCTOP &  1.00 & \href{../works/KelarevaTK13.pdf}{KelarevaTK13}~\cite{KelarevaTK13} &  & \\
\index{Bulk Port Cargo Throughput Optimisation Problem}\index{Classification!Bulk Port Cargo Throughput Optimisation Problem}Bulk Port Cargo Throughput Optimisation Problem &  1.00 &  &  & \href{../works/KelarevaTK13.pdf}{KelarevaTK13}~\cite{KelarevaTK13}\\
\index{CECSP}\index{Classification!CECSP}CECSP &  1.00 & \href{../works/NattafHKAL19.pdf}{NattafHKAL19}~\cite{NattafHKAL19}, \href{../works/NattafAL17.pdf}{NattafAL17}~\cite{NattafAL17}, \href{../works/Nattaf16.pdf}{Nattaf16}~\cite{Nattaf16}, \href{../works/NattafALR16.pdf}{NattafALR16}~\cite{NattafALR16}, \href{../works/NattafAL15.pdf}{NattafAL15}~\cite{NattafAL15}, \href{../works/ArtiguesL14.pdf}{ArtiguesL14}~\cite{ArtiguesL14} &  & \\
\index{CHSP}\index{Classification!CHSP}CHSP &  1.00 & \href{../works/EfthymiouY23.pdf}{EfthymiouY23}~\cite{EfthymiouY23}, \href{../works/WallaceY20.pdf}{WallaceY20}~\cite{WallaceY20} &  & \\
\index{CTW}\index{Classification!CTW}CTW &  1.00 & \href{../works/KoehlerBFFHPSSS21.pdf}{KoehlerBFFHPSSS21}~\cite{KoehlerBFFHPSSS21} & \href{../works/Lombardi10.pdf}{Lombardi10}~\cite{Lombardi10} & \\
\index{CuSP}\index{Classification!CuSP}CuSP &  1.00 & \href{../works/KameugneFND23.pdf}{KameugneFND23}~\cite{KameugneFND23}, \href{../works/FetgoD22.pdf}{FetgoD22}~\cite{FetgoD22}, \href{../works/CarlierPSJ20.pdf}{CarlierPSJ20}~\cite{CarlierPSJ20}, \href{../works/Tesch18.pdf}{Tesch18}~\cite{Tesch18}, \href{../works/KameugneFGOQ18.pdf}{KameugneFGOQ18}~\cite{KameugneFGOQ18}, \href{../works/Tesch16.pdf}{Tesch16}~\cite{Tesch16}, \href{../works/NattafALR16.pdf}{NattafALR16}~\cite{NattafALR16}, \href{../works/Nattaf16.pdf}{Nattaf16}~\cite{Nattaf16}, \href{../works/Froger16.pdf}{Froger16}~\cite{Froger16}, \href{../works/NattafAL15.pdf}{NattafAL15}~\cite{NattafAL15}, \href{../works/Derrien15.pdf}{Derrien15}~\cite{Derrien15}, \href{../works/KameugneFSN14.pdf}{KameugneFSN14}~\cite{KameugneFSN14}, \href{../works/ArtiguesL14.pdf}{ArtiguesL14}~\cite{ArtiguesL14}, \href{../works/Kameugne14.pdf}{Kameugne14}~\cite{Kameugne14}, \href{../works/DerrienPZ14.pdf}{DerrienPZ14}~\cite{DerrienPZ14}, \href{../works/ArtiguesLH13.pdf}{ArtiguesLH13}~\cite{ArtiguesLH13}, \href{../works/KameugneFSN11.pdf}{KameugneFSN11}~\cite{KameugneFSN11}, \href{../works/SchuttW10.pdf}{SchuttW10}~\cite{SchuttW10}, \href{../works/Demassey03.pdf}{Demassey03}~\cite{Demassey03}, \href{../works/BaptistePN99.pdf}{BaptistePN99}~\cite{BaptistePN99} & \href{../works/Fahimi16.pdf}{Fahimi16}~\cite{Fahimi16}, \href{../works/GingrasQ16.pdf}{GingrasQ16}~\cite{GingrasQ16}, \href{../works/OuelletQ13.pdf}{OuelletQ13}~\cite{OuelletQ13}, \href{../works/Elkhyari03.pdf}{Elkhyari03}~\cite{Elkhyari03} & \href{../works/TardivoDFMP23.pdf}{TardivoDFMP23}~\cite{TardivoDFMP23}, \href{../works/HanenKP21.pdf}{HanenKP21}~\cite{HanenKP21}, \href{../works/Zahout21.pdf}{Zahout21}~\cite{Zahout21}, \href{../works/DerrienP14.pdf}{DerrienP14}~\cite{DerrienP14}\\
\index{EOSP}\index{Classification!EOSP}EOSP &  1.00 &  & \href{../works/SquillaciPR23.pdf}{SquillaciPR23}~\cite{SquillaciPR23} & \\
\index{Earth Observation Scheduling Problem}\index{Classification!Earth Observation Scheduling Problem}Earth Observation Scheduling Problem &  1.00 &  & \href{../works/SquillaciPR23.pdf}{SquillaciPR23}~\cite{SquillaciPR23} & \\
\index{FJS}\index{Classification!FJS}FJS &  1.00 & \href{../works/JuvinHL23a.pdf}{JuvinHL23a}~\cite{JuvinHL23a}, \href{../works/YuraszeckMCCR23.pdf}{YuraszeckMCCR23}~\cite{YuraszeckMCCR23}, \href{../works/WangB23.pdf}{WangB23}~\cite{WangB23}, \href{../works/MengGRZSC22.pdf}{MengGRZSC22}~\cite{MengGRZSC22}, \href{../works/JuvinHL22.pdf}{JuvinHL22}~\cite{JuvinHL22}, \href{../works/MullerMKP22.pdf}{MullerMKP22}~\cite{MullerMKP22}, \href{../works/Teppan22.pdf}{Teppan22}~\cite{Teppan22}, \href{../works/HamPK21.pdf}{HamPK21}~\cite{HamPK21}, \href{../works/HamP21.pdf}{HamP21}~\cite{HamP21}, \href{../works/MengLZB21.pdf}{MengLZB21}~\cite{MengLZB21}, \href{../works/MengZRZL20.pdf}{MengZRZL20}~\cite{MengZRZL20}, \href{../works/WangB20.pdf}{WangB20}~\cite{WangB20}, \href{../works/Lunardi20.pdf}{Lunardi20}~\cite{Lunardi20}, \href{../works/LunardiBLRV20.pdf}{LunardiBLRV20}~\cite{LunardiBLRV20}, \href{../works/ZarandiASC20.pdf}{ZarandiASC20}~\cite{ZarandiASC20}, \href{../works/Novas19.pdf}{Novas19}~\cite{Novas19}, \href{../works/MossigeGSMC17.pdf}{MossigeGSMC17}~\cite{MossigeGSMC17}, \href{../works/HamC16.pdf}{HamC16}~\cite{HamC16}, \href{../works/OddiRCS11.pdf}{OddiRCS11}~\cite{OddiRCS11} & \href{../works/OujanaAYB22.pdf}{OujanaAYB22}~\cite{OujanaAYB22}, \href{../works/HauderBRPA20.pdf}{HauderBRPA20}~\cite{HauderBRPA20}, \href{../works/abs-1902-09244.pdf}{abs-1902-09244}~\cite{abs-1902-09244}, \href{../works/ZhangW18.pdf}{ZhangW18}~\cite{ZhangW18}, \href{../works/SchuttFS13.pdf}{SchuttFS13}~\cite{SchuttFS13} & \href{../works/NaderiRR23.pdf}{NaderiRR23}~\cite{NaderiRR23}, \href{../works/ColT22.pdf}{ColT22}~\cite{ColT22}, \href{../works/ZhouGL15.pdf}{ZhouGL15}~\cite{ZhouGL15}\\
\index{Fixed Job Scheduling}\index{Classification!Fixed Job Scheduling}Fixed Job Scheduling &  1.00 & \href{../works/WangB20.pdf}{WangB20}~\cite{WangB20} & \href{../works/WangB23.pdf}{WangB23}~\cite{WangB23} & \\
\index{GCSP}\index{Classification!GCSP}GCSP &  1.00 & \href{../works/Groleaz21.pdf}{Groleaz21}~\cite{Groleaz21}, \href{../works/GroleazNS20.pdf}{GroleazNS20}~\cite{GroleazNS20} &  & \\
\index{HFF}\index{Classification!HFF}HFF &  1.00 & \href{../works/ArmstrongGOS22.pdf}{ArmstrongGOS22}~\cite{ArmstrongGOS22}, \href{../works/OujanaAYB22.pdf}{OujanaAYB22}~\cite{OujanaAYB22}, \href{../works/ArmstrongGOS21.pdf}{ArmstrongGOS21}~\cite{ArmstrongGOS21}, \href{../works/FachiniA20.pdf}{FachiniA20}~\cite{FachiniA20}, \href{../works/ZhouGL15.pdf}{ZhouGL15}~\cite{ZhouGL15} &  & \\
\index{HFFTT}\index{Classification!HFFTT}HFFTT &  1.00 & \href{../works/ArmstrongGOS22.pdf}{ArmstrongGOS22}~\cite{ArmstrongGOS22}, \href{../works/ArmstrongGOS21.pdf}{ArmstrongGOS21}~\cite{ArmstrongGOS21} &  & \\
\index{HFS}\index{Classification!HFS}HFS &  1.00 & \href{../works/IsikYA23.pdf}{IsikYA23}~\cite{IsikYA23}, \href{../works/MengGRZSC22.pdf}{MengGRZSC22}~\cite{MengGRZSC22}, \href{../works/ZhangJZL22.pdf}{ZhangJZL22}~\cite{ZhangJZL22}, \href{../works/ArmstrongGOS21.pdf}{ArmstrongGOS21}~\cite{ArmstrongGOS21}, \href{../works/MengLZB21.pdf}{MengLZB21}~\cite{MengLZB21}, \href{../works/Astrand21.pdf}{Astrand21}~\cite{Astrand21}, \href{../works/Bedhief21.pdf}{Bedhief21}~\cite{Bedhief21}, \href{../works/TangB20.pdf}{TangB20}~\cite{TangB20}, \href{../works/MengZRZL20.pdf}{MengZRZL20}~\cite{MengZRZL20}, \href{../works/Baptiste02.pdf}{Baptiste02}~\cite{Baptiste02} &  & \href{../works/ArmstrongGOS22.pdf}{ArmstrongGOS22}~\cite{ArmstrongGOS22}, \href{../works/ZarandiASC20.pdf}{ZarandiASC20}~\cite{ZarandiASC20}, \href{../works/Novas19.pdf}{Novas19}~\cite{Novas19}, \href{../works/ZhouGL15.pdf}{ZhouGL15}~\cite{ZhouGL15}\\
\index{JSPT}\index{Classification!JSPT}JSPT &  1.00 &  & \href{../works/MurinR19.pdf}{MurinR19}~\cite{MurinR19} & \\
\index{JSSP}\index{Classification!JSSP}JSSP &  1.00 & \href{../works/JuvinHL23a.pdf}{JuvinHL23a}~\cite{JuvinHL23a}, \href{../works/JuvinHHL23.pdf}{JuvinHHL23}~\cite{JuvinHHL23}, \href{../works/abs-2306-05747.pdf}{abs-2306-05747}~\cite{abs-2306-05747}, \href{../works/TasselGS23.pdf}{TasselGS23}~\cite{TasselGS23}, \href{../works/YuraszeckMC23.pdf}{YuraszeckMC23}~\cite{YuraszeckMC23}, \href{../works/YuraszeckMCCR23.pdf}{YuraszeckMCCR23}~\cite{YuraszeckMCCR23}, \href{../works/JuvinHL22.pdf}{JuvinHL22}~\cite{JuvinHL22}, \href{../works/Teppan22.pdf}{Teppan22}~\cite{Teppan22}, \href{../works/GeitzGSSW22.pdf}{GeitzGSSW22}~\cite{GeitzGSSW22}, \href{../works/ColT22.pdf}{ColT22}~\cite{ColT22}, \href{../works/YuraszeckMPV22.pdf}{YuraszeckMPV22}~\cite{YuraszeckMPV22}, \href{../works/Godet21a.pdf}{Godet21a}~\cite{Godet21a}, \href{../works/abs-2102-08778.pdf}{abs-2102-08778}~\cite{abs-2102-08778}, \href{../works/ZarandiASC20.pdf}{ZarandiASC20}~\cite{ZarandiASC20}, \href{../works/ColT19.pdf}{ColT19}~\cite{ColT19}, \href{../works/Pralet17.pdf}{Pralet17}~\cite{Pralet17}, \href{../works/MenciaSV13.pdf}{MenciaSV13}~\cite{MenciaSV13}, \href{../works/MenciaSV12.pdf}{MenciaSV12}~\cite{MenciaSV12}, \href{../works/OddiRCS11.pdf}{OddiRCS11}~\cite{OddiRCS11}...\href{../works/BidotVLB09.pdf}{BidotVLB09}~\cite{BidotVLB09}, \href{../works/GodardLN05.pdf}{GodardLN05}~\cite{GodardLN05}, \href{../works/Baptiste02.pdf}{Baptiste02}~\cite{Baptiste02}, \href{../works/CestaOS00.pdf}{CestaOS00}~\cite{CestaOS00}, \href{../works/SourdN00.pdf}{SourdN00}~\cite{SourdN00}, \href{../works/TorresL00.pdf}{TorresL00}~\cite{TorresL00}, \href{../works/PapaB98.pdf}{PapaB98}~\cite{PapaB98}, \href{../works/NuijtenP98.pdf}{NuijtenP98}~\cite{NuijtenP98}, \href{../works/NuijtenA96.pdf}{NuijtenA96}~\cite{NuijtenA96}, \href{../works/Nuijten94.pdf}{Nuijten94}~\cite{Nuijten94} (Total: 31) & \href{../works/GalleguillosKSB19.pdf}{GalleguillosKSB19}~\cite{GalleguillosKSB19}, \href{../works/LombardiBM15.pdf}{LombardiBM15}~\cite{LombardiBM15}, \href{../works/SialaAH15.pdf}{SialaAH15}~\cite{SialaAH15}, \href{../works/BelhadjiI98.pdf}{BelhadjiI98}~\cite{BelhadjiI98} & \href{../works/EfthymiouY23.pdf}{EfthymiouY23}~\cite{EfthymiouY23}, \href{../works/Mehdizadeh-Somarin23.pdf}{Mehdizadeh-Somarin23}~\cite{Mehdizadeh-Somarin23}, \href{../works/CzerniachowskaWZ23.pdf}{CzerniachowskaWZ23}~\cite{CzerniachowskaWZ23}, \href{../works/WikarekS19.pdf}{WikarekS19}~\cite{WikarekS19}, \href{../works/PraletLJ15.pdf}{PraletLJ15}~\cite{PraletLJ15}, \href{../works/GrimesH15.pdf}{GrimesH15}~\cite{GrimesH15}, \href{../works/BajestaniB11.pdf}{BajestaniB11}~\cite{BajestaniB11}, \href{../works/ChenGPSH10.pdf}{ChenGPSH10}~\cite{ChenGPSH10}, \href{../works/MercierH07.pdf}{MercierH07}~\cite{MercierH07}\\
\index{KRFP}\index{Classification!KRFP}KRFP &  1.00 & \href{../works/KamarainenS02.pdf}{KamarainenS02}~\cite{KamarainenS02}, \href{../works/SakkoutW00.pdf}{SakkoutW00}~\cite{SakkoutW00} &  & \\
\index{LSFRP}\index{Classification!LSFRP}LSFRP &  1.00 & \href{../works/KelarevaTK13.pdf}{KelarevaTK13}~\cite{KelarevaTK13} &  & \\
\index{Liner Shipping Fleet Repositioning Problem}\index{Classification!Liner Shipping Fleet Repositioning Problem}Liner Shipping Fleet Repositioning Problem &  1.00 &  & \href{../works/KelarevaTK13.pdf}{KelarevaTK13}~\cite{KelarevaTK13} & \\
\index{MGAP}\index{Classification!MGAP}MGAP &  1.00 & \href{../works/Darby-DowmanLMZ97.pdf}{Darby-DowmanLMZ97}~\cite{Darby-DowmanLMZ97} &  & \\
\index{OSP}\index{Classification!OSP}OSP &  1.00 & \href{../works/Bit-Monnot23.pdf}{Bit-Monnot23}~\cite{Bit-Monnot23}, \href{../works/NaderiRR23.pdf}{NaderiRR23}~\cite{NaderiRR23}, \href{../works/LacknerMMWW23.pdf}{LacknerMMWW23}~\cite{LacknerMMWW23}, \href{../works/LacknerMMWW21.pdf}{LacknerMMWW21}~\cite{LacknerMMWW21}, \href{../works/Groleaz21.pdf}{Groleaz21}~\cite{Groleaz21}, \href{../works/GombolayWS18.pdf}{GombolayWS18}~\cite{GombolayWS18}, \href{../works/GrimesH15.pdf}{GrimesH15}~\cite{GrimesH15}, \href{../works/GayHLS15.pdf}{GayHLS15}~\cite{GayHLS15}, \href{../works/Siala15a.pdf}{Siala15a}~\cite{Siala15a}, \href{../works/Siala15.pdf}{Siala15}~\cite{Siala15}, \href{../works/MalapertCGJLR12.pdf}{MalapertCGJLR12}~\cite{MalapertCGJLR12} & \href{../works/SquillaciPR23.pdf}{SquillaciPR23}~\cite{SquillaciPR23}, \href{../works/GrimesHM09.pdf}{GrimesHM09}~\cite{GrimesHM09}, \href{../works/MonetteDD07.pdf}{MonetteDD07}~\cite{MonetteDD07} & \href{../works/MengZRZL20.pdf}{MengZRZL20}~\cite{MengZRZL20}, \href{../works/LiuLH18.pdf}{LiuLH18}~\cite{LiuLH18}, \href{../works/Dorndorf2000.pdf}{Dorndorf2000}~\cite{Dorndorf2000}\\
\index{OSSP}\index{Classification!OSSP}OSSP &  1.00 & \href{../works/YuraszeckMC23.pdf}{YuraszeckMC23}~\cite{YuraszeckMC23}, \href{../works/AbreuNP23.pdf}{AbreuNP23}~\cite{AbreuNP23}, \href{../works/AbreuPNF23.pdf}{AbreuPNF23}~\cite{AbreuPNF23}, \href{../works/AbreuN22.pdf}{AbreuN22}~\cite{AbreuN22}, \href{../works/YuraszeckMPV22.pdf}{YuraszeckMPV22}~\cite{YuraszeckMPV22}, \href{../works/ColT22.pdf}{ColT22}~\cite{ColT22}, \href{../works/AbreuAPNM21.pdf}{AbreuAPNM21}~\cite{AbreuAPNM21}, \href{../works/RoshanaeiN21.pdf}{RoshanaeiN21}~\cite{RoshanaeiN21}, \href{../works/MejiaY20.pdf}{MejiaY20}~\cite{MejiaY20}, \href{../works/Baptiste02.pdf}{Baptiste02}~\cite{Baptiste02} &  & \href{../works/YuraszeckMCCR23.pdf}{YuraszeckMCCR23}~\cite{YuraszeckMCCR23}, \href{../works/ZarandiASC20.pdf}{ZarandiASC20}~\cite{ZarandiASC20}\\
\index{Open Shop Scheduling Problem}\index{Classification!Open Shop Scheduling Problem}Open Shop Scheduling Problem &  1.00 & \href{../works/AbreuPNF23.pdf}{AbreuPNF23}~\cite{AbreuPNF23}, \href{../works/AbreuNP23.pdf}{AbreuNP23}~\cite{AbreuNP23}, \href{../works/AbreuN22.pdf}{AbreuN22}~\cite{AbreuN22}, \href{../works/AbreuAPNM21.pdf}{AbreuAPNM21}~\cite{AbreuAPNM21}, \href{../works/ZarandiASC20.pdf}{ZarandiASC20}~\cite{ZarandiASC20}, \href{../works/MejiaY20.pdf}{MejiaY20}~\cite{MejiaY20} & \href{../works/Malapert11.pdf}{Malapert11}~\cite{Malapert11}, \href{../works/LorigeonBB02.pdf}{LorigeonBB02}~\cite{LorigeonBB02}, \href{../works/Dorndorf2000.pdf}{Dorndorf2000}~\cite{Dorndorf2000} & \href{../works/PrataAN23.pdf}{PrataAN23}~\cite{PrataAN23}, \href{../works/NaderiRR23.pdf}{NaderiRR23}~\cite{NaderiRR23}, \href{../works/YuraszeckMCCR23.pdf}{YuraszeckMCCR23}~\cite{YuraszeckMCCR23}, \href{../works/Bit-Monnot23.pdf}{Bit-Monnot23}~\cite{Bit-Monnot23}, \href{../works/YuraszeckMPV22.pdf}{YuraszeckMPV22}~\cite{YuraszeckMPV22}, \href{../works/ColT22.pdf}{ColT22}~\cite{ColT22}, \href{../works/Groleaz21.pdf}{Groleaz21}~\cite{Groleaz21}, \href{../works/MengZRZL20.pdf}{MengZRZL20}~\cite{MengZRZL20}, \href{../works/SacramentoSP20.pdf}{SacramentoSP20}~\cite{SacramentoSP20}, \href{../works/HookerH17.pdf}{HookerH17}~\cite{HookerH17}, \href{../works/GrimesH15.pdf}{GrimesH15}~\cite{GrimesH15}, \href{../works/MalapertCGJLR13.pdf}{MalapertCGJLR13}~\cite{MalapertCGJLR13}, \href{../works/MalapertCGJLR12.pdf}{MalapertCGJLR12}~\cite{MalapertCGJLR12}, \href{../works/Schutt11.pdf}{Schutt11}~\cite{Schutt11}, \href{../works/GrimesH10.pdf}{GrimesH10}~\cite{GrimesH10}, \href{../works/OhrimenkoSC09.pdf}{OhrimenkoSC09}~\cite{OhrimenkoSC09}, \href{../works/GrimesHM09.pdf}{GrimesHM09}~\cite{GrimesHM09}, \href{../works/MonetteDD07.pdf}{MonetteDD07}~\cite{MonetteDD07}, \href{../works/JussienL02.pdf}{JussienL02}~\cite{JussienL02}, \href{../works/Baptiste02.pdf}{Baptiste02}~\cite{Baptiste02}, \href{../works/VerfaillieL01.pdf}{VerfaillieL01}~\cite{VerfaillieL01}\\
\index{PJSSP}\index{Classification!PJSSP}PJSSP &  1.00 & \href{../works/Baptiste02.pdf}{Baptiste02}~\cite{Baptiste02} & \href{../works/PapaB98.pdf}{PapaB98}~\cite{PapaB98} & \\
\index{PMSP}\index{Classification!PMSP}PMSP &  1.00 & \href{../works/NaderiRR23.pdf}{NaderiRR23}~\cite{NaderiRR23}, \href{../works/YunusogluY22.pdf}{YunusogluY22}~\cite{YunusogluY22}, \href{../works/WinterMMW22.pdf}{WinterMMW22}~\cite{WinterMMW22}, \href{../works/PandeyS21a.pdf}{PandeyS21a}~\cite{PandeyS21a}, \href{../works/Godet21a.pdf}{Godet21a}~\cite{Godet21a}, \href{../works/GodetLHS20.pdf}{GodetLHS20}~\cite{GodetLHS20}, \href{../works/MalapertN19.pdf}{MalapertN19}~\cite{MalapertN19}, \href{../works/GedikKEK18.pdf}{GedikKEK18}~\cite{GedikKEK18}, \href{../works/GomesM17.pdf}{GomesM17}~\cite{GomesM17}, \href{../works/TranAB16.pdf}{TranAB16}~\cite{TranAB16}, \href{../works/TranB12.pdf}{TranB12}~\cite{TranB12} & \href{../works/VlkHT21.pdf}{VlkHT21}~\cite{VlkHT21}, \href{../works/NattafM20.pdf}{NattafM20}~\cite{NattafM20}, \href{../works/ArtiguesLH13.pdf}{ArtiguesLH13}~\cite{ArtiguesLH13} & \href{../works/OujanaAYB22.pdf}{OujanaAYB22}~\cite{OujanaAYB22}, \href{../works/ColT22.pdf}{ColT22}~\cite{ColT22}, \href{../works/MengGRZSC22.pdf}{MengGRZSC22}~\cite{MengGRZSC22}, \href{../works/ZarandiASC20.pdf}{ZarandiASC20}~\cite{ZarandiASC20}\\
\index{PTC}\index{Classification!PTC}PTC &  1.00 & \href{../works/NattafM20.pdf}{NattafM20}~\cite{NattafM20}, \href{../works/MalapertN19.pdf}{MalapertN19}~\cite{MalapertN19}, \href{../works/NattafDYW19.pdf}{NattafDYW19}~\cite{NattafDYW19} & \href{../works/NaderiRR23.pdf}{NaderiRR23}~\cite{NaderiRR23} & \href{../works/CzerniachowskaWZ23.pdf}{CzerniachowskaWZ23}~\cite{CzerniachowskaWZ23}, \href{../works/Teppan22.pdf}{Teppan22}~\cite{Teppan22}, \href{../works/ColT2019a.pdf}{ColT2019a}~\cite{ColT2019a}, \href{../works/Dejemeppe16.pdf}{Dejemeppe16}~\cite{Dejemeppe16}\\
\index{Partial Order Schedule}\index{Classification!Partial Order Schedule}Partial Order Schedule &  1.00 & \href{../works/PolicellaWSO05.pdf}{PolicellaWSO05}~\cite{PolicellaWSO05} & \href{../works/LombardiBM15.pdf}{LombardiBM15}~\cite{LombardiBM15}, \href{../works/BonfiettiLM14.pdf}{BonfiettiLM14}~\cite{BonfiettiLM14} & \href{../works/Bit-Monnot23.pdf}{Bit-Monnot23}~\cite{Bit-Monnot23}, \href{../works/Astrand21.pdf}{Astrand21}~\cite{Astrand21}, \href{../works/Astrand0F21.pdf}{Astrand0F21}~\cite{Astrand0F21}, \href{../works/CappartTSR18.pdf}{CappartTSR18}~\cite{CappartTSR18}, \href{../works/BonfiettiLBM14.pdf}{BonfiettiLBM14}~\cite{BonfiettiLBM14}, \href{../works/LaborieR14.pdf}{LaborieR14}~\cite{LaborieR14}, \href{../works/GaySS14.pdf}{GaySS14}~\cite{GaySS14}, \href{../works/LombardiM12.pdf}{LombardiM12}~\cite{LombardiM12}, \href{../works/LombardiM12a.pdf}{LombardiM12a}~\cite{LombardiM12a}, \href{../works/LombardiM10.pdf}{LombardiM10}~\cite{LombardiM10}, \href{../works/CarchraeBF05.pdf}{CarchraeBF05}~\cite{CarchraeBF05}\\
\index{RCMPSP}\index{Classification!RCMPSP}RCMPSP &  1.00 & \href{../works/HauderBRPA20.pdf}{HauderBRPA20}~\cite{HauderBRPA20}, \href{../works/abs-1902-09244.pdf}{abs-1902-09244}~\cite{abs-1902-09244} &  & \href{../works/ArtiguesR00.pdf}{ArtiguesR00}~\cite{ArtiguesR00}\\
\index{RCPSP}\index{Classification!RCPSP}RCPSP &  1.00 & \href{../works/YuraszeckMCCR23.pdf}{YuraszeckMCCR23}~\cite{YuraszeckMCCR23}, \href{../works/GokPTGO23.pdf}{GokPTGO23}~\cite{GokPTGO23}, \href{../works/PovedaAA23.pdf}{PovedaAA23}~\cite{PovedaAA23}, \href{../works/CampeauG22.pdf}{CampeauG22}~\cite{CampeauG22}, \href{../works/EtminaniesfahaniGNMS22.pdf}{EtminaniesfahaniGNMS22}~\cite{EtminaniesfahaniGNMS22}, \href{../works/SubulanC22.pdf}{SubulanC22}~\cite{SubulanC22}, \href{../works/BoudreaultSLQ22.pdf}{BoudreaultSLQ22}~\cite{BoudreaultSLQ22}, \href{../works/FetgoD22.pdf}{FetgoD22}~\cite{FetgoD22}, \href{../works/GeibingerMM21.pdf}{GeibingerMM21}~\cite{GeibingerMM21}, \href{../works/BenderWS21.pdf}{BenderWS21}~\cite{BenderWS21}, \href{../works/Zahout21.pdf}{Zahout21}~\cite{Zahout21}, \href{../works/Groleaz21.pdf}{Groleaz21}~\cite{Groleaz21}, \href{../works/HubnerGSV21.pdf}{HubnerGSV21}~\cite{HubnerGSV21}, \href{../works/Godet21a.pdf}{Godet21a}~\cite{Godet21a}, \href{../works/HillTV21.pdf}{HillTV21}~\cite{HillTV21}, \href{../works/ArtiguesHQT21.pdf}{ArtiguesHQT21}~\cite{ArtiguesHQT21}, \href{../works/HauderBRPA20.pdf}{HauderBRPA20}~\cite{HauderBRPA20}, \href{../works/Polo-MejiaALB20.pdf}{Polo-MejiaALB20}~\cite{Polo-MejiaALB20}, \href{../works/ZarandiASC20.pdf}{ZarandiASC20}~\cite{ZarandiASC20}...\href{../works/Laborie05.pdf}{Laborie05}~\cite{Laborie05}, \href{../works/DemasseyAM05.pdf}{DemasseyAM05}~\cite{DemasseyAM05}, \href{../works/Elkhyari03.pdf}{Elkhyari03}~\cite{Elkhyari03}, \href{../works/Demassey03.pdf}{Demassey03}~\cite{Demassey03}, \href{../works/ElkhyariGJ02a.pdf}{ElkhyariGJ02a}~\cite{ElkhyariGJ02a}, \href{../works/Baptiste02.pdf}{Baptiste02}~\cite{Baptiste02}, \href{../works/BaptisteP00.pdf}{BaptisteP00}~\cite{BaptisteP00}, \href{../works/BruckerK00.pdf}{BruckerK00}~\cite{BruckerK00}, \href{../works/CestaOF99.pdf}{CestaOF99}~\cite{CestaOF99}, \href{../works/BaptistePN99.pdf}{BaptistePN99}~\cite{BaptistePN99} (Total: 77) & \href{../works/KameugneFND23.pdf}{KameugneFND23}~\cite{KameugneFND23}, \href{../works/Caballero23.pdf}{Caballero23}~\cite{Caballero23}, \href{../works/TardivoDFMP23.pdf}{TardivoDFMP23}~\cite{TardivoDFMP23}, \href{../works/KovacsTKSG21.pdf}{KovacsTKSG21}~\cite{KovacsTKSG21}, \href{../works/GroleazNS20a.pdf}{GroleazNS20a}~\cite{GroleazNS20a}, \href{../works/CarlierPSJ20.pdf}{CarlierPSJ20}~\cite{CarlierPSJ20}, \href{../works/CauwelaertLS18.pdf}{CauwelaertLS18}~\cite{CauwelaertLS18}, \href{../works/BaptisteB18.pdf}{BaptisteB18}~\cite{BaptisteB18}, \href{../works/Tesch18.pdf}{Tesch18}~\cite{Tesch18}, \href{../works/Dejemeppe16.pdf}{Dejemeppe16}~\cite{Dejemeppe16}, \href{../works/LombardiBM15.pdf}{LombardiBM15}~\cite{LombardiBM15}, \href{../works/NattafAL15.pdf}{NattafAL15}~\cite{NattafAL15}, \href{../works/GayHLS15.pdf}{GayHLS15}~\cite{GayHLS15}, \href{../works/KameugneFSN14.pdf}{KameugneFSN14}~\cite{KameugneFSN14}, \href{../works/ArtiguesL14.pdf}{ArtiguesL14}~\cite{ArtiguesL14}, \href{../works/LaborieR14.pdf}{LaborieR14}~\cite{LaborieR14}, \href{../works/LombardiM13.pdf}{LombardiM13}~\cite{LombardiM13}, \href{../works/LombardiMB13.pdf}{LombardiMB13}~\cite{LombardiMB13}, \href{../works/KameugneFSN11.pdf}{KameugneFSN11}~\cite{KameugneFSN11}, \href{../works/HeinzS11.pdf}{HeinzS11}~\cite{HeinzS11}, \href{../works/abs-1009-0347.pdf}{abs-1009-0347}~\cite{abs-1009-0347}, \href{../works/KeriK07.pdf}{KeriK07}~\cite{KeriK07}, \href{../works/KovacsV06.pdf}{KovacsV06}~\cite{KovacsV06}, \href{../works/HeipckeCCS00.pdf}{HeipckeCCS00}~\cite{HeipckeCCS00}, \href{../works/ArtiguesR00.pdf}{ArtiguesR00}~\cite{ArtiguesR00} & \href{../works/AbreuPNF23.pdf}{AbreuPNF23}~\cite{AbreuPNF23}, \href{../works/NaderiRR23.pdf}{NaderiRR23}~\cite{NaderiRR23}, \href{../works/TouatBT22.pdf}{TouatBT22}~\cite{TouatBT22}, \href{../works/GeitzGSSW22.pdf}{GeitzGSSW22}~\cite{GeitzGSSW22}, \href{../works/HanenKP21.pdf}{HanenKP21}~\cite{HanenKP21}, \href{../works/Astrand21.pdf}{Astrand21}~\cite{Astrand21}, \href{../works/ZhangYW21.pdf}{ZhangYW21}~\cite{ZhangYW21}, \href{../works/Lemos21.pdf}{Lemos21}~\cite{Lemos21}, \href{../works/Mercier-AubinGQ20.pdf}{Mercier-AubinGQ20}~\cite{Mercier-AubinGQ20}, \href{../works/NattafHKAL19.pdf}{NattafHKAL19}~\cite{NattafHKAL19}, \href{../works/WikarekS19.pdf}{WikarekS19}~\cite{WikarekS19}, \href{../works/OuelletQ18.pdf}{OuelletQ18}~\cite{OuelletQ18}, \href{../works/FahimiOQ18.pdf}{FahimiOQ18}~\cite{FahimiOQ18}, \href{../works/HookerH17.pdf}{HookerH17}~\cite{HookerH17}, \href{../works/BonfiettiZLM16.pdf}{BonfiettiZLM16}~\cite{BonfiettiZLM16}, \href{../works/GingrasQ16.pdf}{GingrasQ16}~\cite{GingrasQ16}, \href{../works/Tesch16.pdf}{Tesch16}~\cite{Tesch16}, \href{../works/NattafALR16.pdf}{NattafALR16}~\cite{NattafALR16}, \href{../works/Fahimi16.pdf}{Fahimi16}~\cite{Fahimi16}...\href{../works/SchuttFSW11.pdf}{SchuttFSW11}~\cite{SchuttFSW11}, \href{../works/LombardiBMB11.pdf}{LombardiBMB11}~\cite{LombardiBMB11}, \href{../works/Vilim11.pdf}{Vilim11}~\cite{Vilim11}, \href{../works/LahimerLH11.pdf}{LahimerLH11}~\cite{LahimerLH11}, \href{../works/BartakCS10.pdf}{BartakCS10}~\cite{BartakCS10}, \href{../works/AkkerDH07.pdf}{AkkerDH07}~\cite{AkkerDH07}, \href{../works/PerronSF04.pdf}{PerronSF04}~\cite{PerronSF04}, \href{../works/PoderBS04.pdf}{PoderBS04}~\cite{PoderBS04}, \href{../works/BeckPS03.pdf}{BeckPS03}~\cite{BeckPS03}, \href{../works/HookerY02.pdf}{HookerY02}~\cite{HookerY02} (Total: 49)\\
\index{RCPSPDC}\index{Classification!RCPSPDC}RCPSPDC &  1.00 &  &  & \href{../works/CampeauG22.pdf}{CampeauG22}~\cite{CampeauG22}, \href{../works/HubnerGSV21.pdf}{HubnerGSV21}~\cite{HubnerGSV21}\\
\index{RTMP}\index{Classification!RTMP}RTMP &  1.00 & \href{../works/MarliereSPR23.pdf}{MarliereSPR23}~\cite{MarliereSPR23} &  & \\
\index{Resource-constrained Project Scheduling Problem}\index{Classification!Resource-constrained Project Scheduling Problem}Resource-constrained Project Scheduling Problem &  1.00 & \href{../works/PovedaAA23.pdf}{PovedaAA23}~\cite{PovedaAA23}, \href{../works/EtminaniesfahaniGNMS22.pdf}{EtminaniesfahaniGNMS22}~\cite{EtminaniesfahaniGNMS22}, \href{../works/SubulanC22.pdf}{SubulanC22}~\cite{SubulanC22}, \href{../works/BoudreaultSLQ22.pdf}{BoudreaultSLQ22}~\cite{BoudreaultSLQ22}, \href{../works/Godet21a.pdf}{Godet21a}~\cite{Godet21a}, \href{../works/HillTV21.pdf}{HillTV21}~\cite{HillTV21}, \href{../works/ZarandiASC20.pdf}{ZarandiASC20}~\cite{ZarandiASC20}, \href{../works/abs-1902-09244.pdf}{abs-1902-09244}~\cite{abs-1902-09244}, \href{../works/Caballero19.pdf}{Caballero19}~\cite{Caballero19}, \href{../works/SchnellH15.pdf}{SchnellH15}~\cite{SchnellH15}, \href{../works/HeinzSB13.pdf}{HeinzSB13}~\cite{HeinzSB13}, \href{../works/LombardiM12.pdf}{LombardiM12}~\cite{LombardiM12}, \href{../works/SchuttFSW11.pdf}{SchuttFSW11}~\cite{SchuttFSW11}, \href{../works/Schutt11.pdf}{Schutt11}~\cite{Schutt11}, \href{../works/Lombardi10.pdf}{Lombardi10}~\cite{Lombardi10}, \href{../works/DemasseyAM05.pdf}{DemasseyAM05}~\cite{DemasseyAM05}, \href{../works/Demassey03.pdf}{Demassey03}~\cite{Demassey03} & \href{../works/KameugneFND23.pdf}{KameugneFND23}~\cite{KameugneFND23}, \href{../works/YuraszeckMCCR23.pdf}{YuraszeckMCCR23}~\cite{YuraszeckMCCR23}, \href{../works/Astrand21.pdf}{Astrand21}~\cite{Astrand21}, \href{../works/Groleaz21.pdf}{Groleaz21}~\cite{Groleaz21}, \href{../works/HubnerGSV21.pdf}{HubnerGSV21}~\cite{HubnerGSV21}, \href{../works/GokGSTO20.pdf}{GokGSTO20}~\cite{GokGSTO20}, \href{../works/Polo-MejiaALB20.pdf}{Polo-MejiaALB20}~\cite{Polo-MejiaALB20}, \href{../works/HauderBRPA20.pdf}{HauderBRPA20}~\cite{HauderBRPA20}, \href{../works/ArkhipovBL19.pdf}{ArkhipovBL19}~\cite{ArkhipovBL19}, \href{../works/NattafHKAL19.pdf}{NattafHKAL19}~\cite{NattafHKAL19}, \href{../works/KameugneFGOQ18.pdf}{KameugneFGOQ18}~\cite{KameugneFGOQ18}, \href{../works/SchnellH17.pdf}{SchnellH17}~\cite{SchnellH17}, \href{../works/BofillCSV17a.pdf}{BofillCSV17a}~\cite{BofillCSV17a}, \href{../works/BofillCSV17.pdf}{BofillCSV17}~\cite{BofillCSV17}, \href{../works/YoungFS17.pdf}{YoungFS17}~\cite{YoungFS17}, \href{../works/AmadiniGM16.pdf}{AmadiniGM16}~\cite{AmadiniGM16}, \href{../works/SchuttS16.pdf}{SchuttS16}~\cite{SchuttS16}, \href{../works/Nattaf16.pdf}{Nattaf16}~\cite{Nattaf16}, \href{../works/SzerediS16.pdf}{SzerediS16}~\cite{SzerediS16}...\href{../works/abs-1009-0347.pdf}{abs-1009-0347}~\cite{abs-1009-0347}, \href{../works/LiessM08.pdf}{LiessM08}~\cite{LiessM08}, \href{../works/BeckW07.pdf}{BeckW07}~\cite{BeckW07}, \href{../works/KusterJF07.pdf}{KusterJF07}~\cite{KusterJF07}, \href{../works/Laborie05.pdf}{Laborie05}~\cite{Laborie05}, \href{../works/KovacsV04.pdf}{KovacsV04}~\cite{KovacsV04}, \href{../works/Elkhyari03.pdf}{Elkhyari03}~\cite{Elkhyari03}, \href{../works/ElkhyariGJ02.pdf}{ElkhyariGJ02}~\cite{ElkhyariGJ02}, \href{../works/Baptiste02.pdf}{Baptiste02}~\cite{Baptiste02}, \href{../works/BaptistePN99.pdf}{BaptistePN99}~\cite{BaptistePN99} (Total: 38) & \href{../works/abs-2402-00459.pdf}{abs-2402-00459}~\cite{abs-2402-00459}, \href{../works/LuZZYW24.pdf}{LuZZYW24}~\cite{LuZZYW24}, \href{../works/NaderiRR23.pdf}{NaderiRR23}~\cite{NaderiRR23}, \href{../works/Caballero23.pdf}{Caballero23}~\cite{Caballero23}, \href{../works/GokPTGO23.pdf}{GokPTGO23}~\cite{GokPTGO23}, \href{../works/CampeauG22.pdf}{CampeauG22}~\cite{CampeauG22}, \href{../works/FetgoD22.pdf}{FetgoD22}~\cite{FetgoD22}, \href{../works/MullerMKP22.pdf}{MullerMKP22}~\cite{MullerMKP22}, \href{../works/HanenKP21.pdf}{HanenKP21}~\cite{HanenKP21}, \href{../works/ArtiguesHQT21.pdf}{ArtiguesHQT21}~\cite{ArtiguesHQT21}, \href{../works/ZhangYW21.pdf}{ZhangYW21}~\cite{ZhangYW21}, \href{../works/GeibingerMM21.pdf}{GeibingerMM21}~\cite{GeibingerMM21}, \href{../works/GroleazNS20a.pdf}{GroleazNS20a}~\cite{GroleazNS20a}, \href{../works/GroleazNS20.pdf}{GroleazNS20}~\cite{GroleazNS20}, \href{../works/SacramentoSP20.pdf}{SacramentoSP20}~\cite{SacramentoSP20}, \href{../works/AstrandJZ20.pdf}{AstrandJZ20}~\cite{AstrandJZ20}, \href{../works/BadicaBI20.pdf}{BadicaBI20}~\cite{BadicaBI20}, \href{../works/GalleguillosKSB19.pdf}{GalleguillosKSB19}~\cite{GalleguillosKSB19}, \href{../works/abs-1911-04766.pdf}{abs-1911-04766}~\cite{abs-1911-04766}...\href{../works/LombardiM09.pdf}{LombardiM09}~\cite{LombardiM09}, \href{../works/KeriK07.pdf}{KeriK07}~\cite{KeriK07}, \href{../works/AkkerDH07.pdf}{AkkerDH07}~\cite{AkkerDH07}, \href{../works/ArtiguesF07.pdf}{ArtiguesF07}~\cite{ArtiguesF07}, \href{../works/GodardLN05.pdf}{GodardLN05}~\cite{GodardLN05}, \href{../works/KovacsEKV05.pdf}{KovacsEKV05}~\cite{KovacsEKV05}, \href{../works/ElkhyariGJ02a.pdf}{ElkhyariGJ02a}~\cite{ElkhyariGJ02a}, \href{../works/BruckerK00.pdf}{BruckerK00}~\cite{BruckerK00}, \href{../works/HeipckeCCS00.pdf}{HeipckeCCS00}~\cite{HeipckeCCS00}, \href{../works/PapaB98.pdf}{PapaB98}~\cite{PapaB98} (Total: 72)\\
\index{Resource-constrained Project Scheduling Problem with Discounted Cashflow}\index{Classification!Resource-constrained Project Scheduling Problem with Discounted Cashflow}Resource-constrained Project Scheduling Problem with Discounted Cashflow &  1.00 &  &  & \href{../works/ZarandiASC20.pdf}{ZarandiASC20}~\cite{ZarandiASC20}\\
\index{SBSFMMAL}\index{Classification!SBSFMMAL}SBSFMMAL &  1.00 & \href{../works/OzturkTHO13.pdf}{OzturkTHO13}~\cite{OzturkTHO13}, \href{../works/OzturkTHO10.pdf}{OzturkTHO10}~\cite{OzturkTHO10} & \href{../works/OzturkTHO15.pdf}{OzturkTHO15}~\cite{OzturkTHO15} & \\
\index{SCC}\index{Classification!SCC}SCC &  1.00 & \href{../works/KimCMLLP23.pdf}{KimCMLLP23}~\cite{KimCMLLP23}, \href{../works/WolinskiKG04.pdf}{WolinskiKG04}~\cite{WolinskiKG04} & \href{../works/SchuttFSW13.pdf}{SchuttFSW13}~\cite{SchuttFSW13}, \href{../works/Lombardi10.pdf}{Lombardi10}~\cite{Lombardi10}, \href{../works/abs-1009-0347.pdf}{abs-1009-0347}~\cite{abs-1009-0347} & \href{../works/PohlAK22.pdf}{PohlAK22}~\cite{PohlAK22}, \href{../works/Zahout21.pdf}{Zahout21}~\cite{Zahout21}, \href{../works/TanZWGQ19.pdf}{TanZWGQ19}~\cite{TanZWGQ19}, \href{../works/PachecoPR19.pdf}{PachecoPR19}~\cite{PachecoPR19}, \href{../works/LombardiMB13.pdf}{LombardiMB13}~\cite{LombardiMB13}, \href{../works/BeniniLMR11.pdf}{BeniniLMR11}~\cite{BeniniLMR11}, \href{../works/SchausHMCMD11.pdf}{SchausHMCMD11}~\cite{SchausHMCMD11}, \href{../works/LombardiMRB10.pdf}{LombardiMRB10}~\cite{LombardiMRB10}, \href{../works/BeniniLMR08.pdf}{BeniniLMR08}~\cite{BeniniLMR08}, \href{../works/BeniniLMMR08.pdf}{BeniniLMMR08}~\cite{BeniniLMMR08}\\
\index{TCSP}\index{Classification!TCSP}TCSP &  1.00 & \href{../works/BelhadjiI98.pdf}{BelhadjiI98}~\cite{BelhadjiI98} & \href{../works/CestaOS00.pdf}{CestaOS00}~\cite{CestaOS00} & \href{../works/Zahout21.pdf}{Zahout21}~\cite{Zahout21}, \href{../works/LombardiM10a.pdf}{LombardiM10a}~\cite{LombardiM10a}, \href{../works/Lombardi10.pdf}{Lombardi10}~\cite{Lombardi10}, \href{../works/BartakSR10.pdf}{BartakSR10}~\cite{BartakSR10}, \href{../works/Demassey03.pdf}{Demassey03}~\cite{Demassey03}\\
\index{TMS}\index{Classification!TMS}TMS &  1.00 & \href{../works/PopovicCGNC22.pdf}{PopovicCGNC22}~\cite{PopovicCGNC22}, \href{../works/Froger16.pdf}{Froger16}~\cite{Froger16}, \href{../works/ChunS14.pdf}{ChunS14}~\cite{ChunS14}, \href{../works/Junker00.pdf}{Junker00}~\cite{Junker00}, \href{../works/Hamscher91.pdf}{Hamscher91}~\cite{Hamscher91} & \href{../works/BegB13.pdf}{BegB13}~\cite{BegB13} & \href{../works/NaqviAIAAA22.pdf}{NaqviAIAAA22}~\cite{NaqviAIAAA22}, \href{../works/CappartS17.pdf}{CappartS17}~\cite{CappartS17}, \href{../works/Siala15a.pdf}{Siala15a}~\cite{Siala15a}, \href{../works/Siala15.pdf}{Siala15}~\cite{Siala15}, \href{../works/JussienL02.pdf}{JussienL02}~\cite{JussienL02}, \href{../works/ChunCTY99.pdf}{ChunCTY99}~\cite{ChunCTY99}, \href{../works/MorgadoM97.pdf}{MorgadoM97}~\cite{MorgadoM97}, \href{../works/Prosser89.pdf}{Prosser89}~\cite{Prosser89}\\
\index{Temporal Constraint Satisfaction Problem}\index{Classification!Temporal Constraint Satisfaction Problem}Temporal Constraint Satisfaction Problem &  1.00 &  & \href{../works/BelhadjiI98.pdf}{BelhadjiI98}~\cite{BelhadjiI98} & \href{../works/BartakSR10.pdf}{BartakSR10}~\cite{BartakSR10}, \href{../works/MoffittPP05.pdf}{MoffittPP05}~\cite{MoffittPP05}, \href{../works/Elkhyari03.pdf}{Elkhyari03}~\cite{Elkhyari03}\\
\index{parallel machine}\index{Classification!parallel machine}parallel machine &  1.00 & \href{../works/PrataAN23.pdf}{PrataAN23}~\cite{PrataAN23}, \href{../works/abs-2305-19888.pdf}{abs-2305-19888}~\cite{abs-2305-19888}, \href{../works/IsikYA23.pdf}{IsikYA23}~\cite{IsikYA23}, \href{../works/NaderiRR23.pdf}{NaderiRR23}~\cite{NaderiRR23}, \href{../works/Adelgren2023.pdf}{Adelgren2023}~\cite{Adelgren2023}, \href{../works/CzerniachowskaWZ23.pdf}{CzerniachowskaWZ23}~\cite{CzerniachowskaWZ23}, \href{../works/ZhangJZL22.pdf}{ZhangJZL22}~\cite{ZhangJZL22}, \href{../works/WinterMMW22.pdf}{WinterMMW22}~\cite{WinterMMW22}, \href{../works/MengGRZSC22.pdf}{MengGRZSC22}~\cite{MengGRZSC22}, \href{../works/YunusogluY22.pdf}{YunusogluY22}~\cite{YunusogluY22}, \href{../works/HeinzNVH22.pdf}{HeinzNVH22}~\cite{HeinzNVH22}, \href{../works/OujanaAYB22.pdf}{OujanaAYB22}~\cite{OujanaAYB22}, \href{../works/MengLZB21.pdf}{MengLZB21}~\cite{MengLZB21}, \href{../works/Astrand21.pdf}{Astrand21}~\cite{Astrand21}, \href{../works/Groleaz21.pdf}{Groleaz21}~\cite{Groleaz21}, \href{../works/PandeyS21a.pdf}{PandeyS21a}~\cite{PandeyS21a}, \href{../works/Godet21a.pdf}{Godet21a}~\cite{Godet21a}, \href{../works/ZarandiASC20.pdf}{ZarandiASC20}~\cite{ZarandiASC20}, \href{../works/MengZRZL20.pdf}{MengZRZL20}~\cite{MengZRZL20}...\href{../works/ArbaouiY18.pdf}{ArbaouiY18}~\cite{ArbaouiY18}, \href{../works/GomesM17.pdf}{GomesM17}~\cite{GomesM17}, \href{../works/HebrardHJMPV16.pdf}{HebrardHJMPV16}~\cite{HebrardHJMPV16}, \href{../works/TranAB16.pdf}{TranAB16}~\cite{TranAB16}, \href{../works/Nattaf16.pdf}{Nattaf16}~\cite{Nattaf16}, \href{../works/ArtiguesLH13.pdf}{ArtiguesLH13}~\cite{ArtiguesLH13}, \href{../works/TranB12.pdf}{TranB12}~\cite{TranB12}, \href{../works/EdisO11.pdf}{EdisO11}~\cite{EdisO11}, \href{../works/Jans09.pdf}{Jans09}~\cite{Jans09}, \href{../works/Baptiste02.pdf}{Baptiste02}~\cite{Baptiste02} (Total: 38) & \href{../works/NaderiBZ23.pdf}{NaderiBZ23}~\cite{NaderiBZ23}, \href{../works/PenzDN23.pdf}{PenzDN23}~\cite{PenzDN23}, \href{../works/Fatemi-AnarakiTFV23.pdf}{Fatemi-AnarakiTFV23}~\cite{Fatemi-AnarakiTFV23}, \href{../works/AbreuPNF23.pdf}{AbreuPNF23}~\cite{AbreuPNF23}, \href{../works/JuvinHL23a.pdf}{JuvinHL23a}~\cite{JuvinHL23a}, \href{../works/AbreuNP23.pdf}{AbreuNP23}~\cite{AbreuNP23}, \href{../works/NaderiBZ22.pdf}{NaderiBZ22}~\cite{NaderiBZ22}, \href{../works/Teppan22.pdf}{Teppan22}~\cite{Teppan22}, \href{../works/EmdeZD22.pdf}{EmdeZD22}~\cite{EmdeZD22}, \href{../works/ColT22.pdf}{ColT22}~\cite{ColT22}, \href{../works/Bedhief21.pdf}{Bedhief21}~\cite{Bedhief21}, \href{../works/Zahout21.pdf}{Zahout21}~\cite{Zahout21}, \href{../works/MokhtarzadehTNF20.pdf}{MokhtarzadehTNF20}~\cite{MokhtarzadehTNF20}, \href{../works/SacramentoSP20.pdf}{SacramentoSP20}~\cite{SacramentoSP20}, \href{../works/MejiaY20.pdf}{MejiaY20}~\cite{MejiaY20}, \href{../works/ParkUJR19.pdf}{ParkUJR19}~\cite{ParkUJR19}, \href{../works/Novas19.pdf}{Novas19}~\cite{Novas19}, \href{../works/BogaerdtW19.pdf}{BogaerdtW19}~\cite{BogaerdtW19}, \href{../works/Ham18a.pdf}{Ham18a}~\cite{Ham18a}...\href{../works/CatusseCBL16.pdf}{CatusseCBL16}~\cite{CatusseCBL16}, \href{../works/QinDS16.pdf}{QinDS16}~\cite{QinDS16}, \href{../works/ZhouGL15.pdf}{ZhouGL15}~\cite{ZhouGL15}, \href{../works/TerekhovTDB14.pdf}{TerekhovTDB14}~\cite{TerekhovTDB14}, \href{../works/BajestaniB13.pdf}{BajestaniB13}~\cite{BajestaniB13}, \href{../works/TranTDB13.pdf}{TranTDB13}~\cite{TranTDB13}, \href{../works/GuyonLPR12.pdf}{GuyonLPR12}~\cite{GuyonLPR12}, \href{../works/KovacsB11.pdf}{KovacsB11}~\cite{KovacsB11}, \href{../works/AkkerDH07.pdf}{AkkerDH07}~\cite{AkkerDH07}, \href{../works/SadykovW06.pdf}{SadykovW06}~\cite{SadykovW06} (Total: 32) & \href{../works/GuoZ23.pdf}{GuoZ23}~\cite{GuoZ23}, \href{../works/NaderiBZR23.pdf}{NaderiBZR23}~\cite{NaderiBZR23}, \href{../works/LacknerMMWW23.pdf}{LacknerMMWW23}~\cite{LacknerMMWW23}, \href{../works/Mehdizadeh-Somarin23.pdf}{Mehdizadeh-Somarin23}~\cite{Mehdizadeh-Somarin23}, \href{../works/AlfieriGPS23.pdf}{AlfieriGPS23}~\cite{AlfieriGPS23}, \href{../works/KimCMLLP23.pdf}{KimCMLLP23}~\cite{KimCMLLP23}, \href{../works/JuvinHHL23.pdf}{JuvinHHL23}~\cite{JuvinHHL23}, \href{../works/JuvinHL22.pdf}{JuvinHL22}~\cite{JuvinHL22}, \href{../works/ArmstrongGOS22.pdf}{ArmstrongGOS22}~\cite{ArmstrongGOS22}, \href{../works/OrnekOS20.pdf}{OrnekOS20}~\cite{OrnekOS20}, \href{../works/NaderiBZ22a.pdf}{NaderiBZ22a}~\cite{NaderiBZ22a}, \href{../works/AwadMDMT22.pdf}{AwadMDMT22}~\cite{AwadMDMT22}, \href{../works/EtminaniesfahaniGNMS22.pdf}{EtminaniesfahaniGNMS22}~\cite{EtminaniesfahaniGNMS22}, \href{../works/HanenKP21.pdf}{HanenKP21}~\cite{HanenKP21}, \href{../works/AbohashimaEG21.pdf}{AbohashimaEG21}~\cite{AbohashimaEG21}, \href{../works/LacknerMMWW21.pdf}{LacknerMMWW21}~\cite{LacknerMMWW21}, \href{../works/FanXG21.pdf}{FanXG21}~\cite{FanXG21}, \href{../works/AbreuAPNM21.pdf}{AbreuAPNM21}~\cite{AbreuAPNM21}, \href{../works/HamPK21.pdf}{HamPK21}~\cite{HamPK21}...\href{../works/MilanoW06.pdf}{MilanoW06}~\cite{MilanoW06}, \href{../works/ArtiouchineB05.pdf}{ArtiouchineB05}~\cite{ArtiouchineB05}, \href{../works/BeniniBGM05.pdf}{BeniniBGM05}~\cite{BeniniBGM05}, \href{../works/Sadykov04.pdf}{Sadykov04}~\cite{Sadykov04}, \href{../works/KanetAG04.pdf}{KanetAG04}~\cite{KanetAG04}, \href{../works/CambazardHDJT04.pdf}{CambazardHDJT04}~\cite{CambazardHDJT04}, \href{../works/Elkhyari03.pdf}{Elkhyari03}~\cite{Elkhyari03}, \href{../works/LorigeonBB02.pdf}{LorigeonBB02}~\cite{LorigeonBB02}, \href{../works/Dorndorf2000.pdf}{Dorndorf2000}~\cite{Dorndorf2000}, \href{../works/HarjunkoskiJG00.pdf}{HarjunkoskiJG00}~\cite{HarjunkoskiJG00} (Total: 66)\\
\index{psplib}\index{Classification!psplib}psplib &  1.00 & \href{../works/TardivoDFMP23.pdf}{TardivoDFMP23}~\cite{TardivoDFMP23}, \href{../works/Caballero19.pdf}{Caballero19}~\cite{Caballero19}, \href{../works/ArkhipovBL19.pdf}{ArkhipovBL19}~\cite{ArkhipovBL19}, \href{../works/KreterSSZ18.pdf}{KreterSSZ18}~\cite{KreterSSZ18}, \href{../works/OuelletQ18.pdf}{OuelletQ18}~\cite{OuelletQ18}, \href{../works/GayHS15a.pdf}{GayHS15a}~\cite{GayHS15a}, \href{../works/LetortCB15.pdf}{LetortCB15}~\cite{LetortCB15}, \href{../works/Derrien15.pdf}{Derrien15}~\cite{Derrien15}, \href{../works/KameugneFSN14.pdf}{KameugneFSN14}~\cite{KameugneFSN14}, \href{../works/DerrienP14.pdf}{DerrienP14}~\cite{DerrienP14}, \href{../works/Kameugne14.pdf}{Kameugne14}~\cite{Kameugne14}, \href{../works/SchuttFSW13.pdf}{SchuttFSW13}~\cite{SchuttFSW13}, \href{../works/SchuttFS13a.pdf}{SchuttFS13a}~\cite{SchuttFS13a}, \href{../works/HeinzSB13.pdf}{HeinzSB13}~\cite{HeinzSB13}, \href{../works/Letort13.pdf}{Letort13}~\cite{Letort13}, \href{../works/Clercq12.pdf}{Clercq12}~\cite{Clercq12}, \href{../works/SchuttFSW11.pdf}{SchuttFSW11}~\cite{SchuttFSW11}, \href{../works/Schutt11.pdf}{Schutt11}~\cite{Schutt11}, \href{../works/BertholdHLMS10.pdf}{BertholdHLMS10}~\cite{BertholdHLMS10}, \href{../works/SchuttFSW09.pdf}{SchuttFSW09}~\cite{SchuttFSW09}, \href{../works/Laborie05.pdf}{Laborie05}~\cite{Laborie05}, \href{../works/Demassey03.pdf}{Demassey03}~\cite{Demassey03} & \href{../works/KameugneFND23.pdf}{KameugneFND23}~\cite{KameugneFND23}, \href{../works/BoudreaultSLQ22.pdf}{BoudreaultSLQ22}~\cite{BoudreaultSLQ22}, \href{../works/EtminaniesfahaniGNMS22.pdf}{EtminaniesfahaniGNMS22}~\cite{EtminaniesfahaniGNMS22}, \href{../works/HillTV21.pdf}{HillTV21}~\cite{HillTV21}, \href{../works/BadicaBI20.pdf}{BadicaBI20}~\cite{BadicaBI20}, \href{../works/BaptisteB18.pdf}{BaptisteB18}~\cite{BaptisteB18}, \href{../works/Tesch18.pdf}{Tesch18}~\cite{Tesch18}, \href{../works/FahimiOQ18.pdf}{FahimiOQ18}~\cite{FahimiOQ18}, \href{../works/SchnellH17.pdf}{SchnellH17}~\cite{SchnellH17}, \href{../works/BofillCSV17a.pdf}{BofillCSV17a}~\cite{BofillCSV17a}, \href{../works/GingrasQ16.pdf}{GingrasQ16}~\cite{GingrasQ16}, \href{../works/Nattaf16.pdf}{Nattaf16}~\cite{Nattaf16}, \href{../works/SzerediS16.pdf}{SzerediS16}~\cite{SzerediS16}, \href{../works/Tesch16.pdf}{Tesch16}~\cite{Tesch16}, \href{../works/GayHLS15.pdf}{GayHLS15}~\cite{GayHLS15}, \href{../works/VilimLS15.pdf}{VilimLS15}~\cite{VilimLS15}, \href{../works/LombardiBM15.pdf}{LombardiBM15}~\cite{LombardiBM15}, \href{../works/BonfiettiLM14.pdf}{BonfiettiLM14}~\cite{BonfiettiLM14}, \href{../works/LetortCB13.pdf}{LetortCB13}~\cite{LetortCB13}, \href{../works/LetortBC12.pdf}{LetortBC12}~\cite{LetortBC12}, \href{../works/LombardiM12a.pdf}{LombardiM12a}~\cite{LombardiM12a}, \href{../works/HeinzS11.pdf}{HeinzS11}~\cite{HeinzS11}, \href{../works/Vilim11.pdf}{Vilim11}~\cite{Vilim11}, \href{../works/abs-1009-0347.pdf}{abs-1009-0347}~\cite{abs-1009-0347}, \href{../works/SchuttW10.pdf}{SchuttW10}~\cite{SchuttW10}, \href{../works/BruckerK00.pdf}{BruckerK00}~\cite{BruckerK00} & \href{../works/Godet21a.pdf}{Godet21a}~\cite{Godet21a}, \href{../works/LaborieRSV18.pdf}{LaborieRSV18}~\cite{LaborieRSV18}, \href{../works/CauwelaertLS18.pdf}{CauwelaertLS18}~\cite{CauwelaertLS18}, \href{../works/YoungFS17.pdf}{YoungFS17}~\cite{YoungFS17}, \href{../works/Pralet17.pdf}{Pralet17}~\cite{Pralet17}, \href{../works/BofillCSV17.pdf}{BofillCSV17}~\cite{BofillCSV17}, \href{../works/Dejemeppe16.pdf}{Dejemeppe16}~\cite{Dejemeppe16}, \href{../works/SchnellH15.pdf}{SchnellH15}~\cite{SchnellH15}, \href{../works/CauwelaertLS15.pdf}{CauwelaertLS15}~\cite{CauwelaertLS15}, \href{../works/ThiruvadyWGS14.pdf}{ThiruvadyWGS14}~\cite{ThiruvadyWGS14}, \href{../works/LombardiM13.pdf}{LombardiM13}~\cite{LombardiM13}, \href{../works/OuelletQ13.pdf}{OuelletQ13}~\cite{OuelletQ13}, \href{../works/LombardiM12.pdf}{LombardiM12}~\cite{LombardiM12}, \href{../works/KameugneFSN11.pdf}{KameugneFSN11}~\cite{KameugneFSN11}, \href{../works/LiessM08.pdf}{LiessM08}~\cite{LiessM08}, \href{../works/FortinZDF05.pdf}{FortinZDF05}~\cite{FortinZDF05}, \href{../works/DemasseyAM05.pdf}{DemasseyAM05}~\cite{DemasseyAM05}, \href{../works/PolicellaWSO05.pdf}{PolicellaWSO05}~\cite{PolicellaWSO05}, \href{../works/ElkhyariGJ02a.pdf}{ElkhyariGJ02a}~\cite{ElkhyariGJ02a}\\
\index{rtRTMP}\index{Classification!rtRTMP}rtRTMP &  1.00 & \href{../works/MarliereSPR23.pdf}{MarliereSPR23}~\cite{MarliereSPR23} &  & \\
\index{single machine}\index{Classification!single machine}single machine &  1.00 & \href{../works/BonninMNE24.pdf}{BonninMNE24}~\cite{BonninMNE24}, \href{../works/PrataAN23.pdf}{PrataAN23}~\cite{PrataAN23}, \href{../works/AlfieriGPS23.pdf}{AlfieriGPS23}~\cite{AlfieriGPS23}, \href{../works/LacknerMMWW23.pdf}{LacknerMMWW23}~\cite{LacknerMMWW23}, \href{../works/PenzDN23.pdf}{PenzDN23}~\cite{PenzDN23}, \href{../works/TouatBT22.pdf}{TouatBT22}~\cite{TouatBT22}, \href{../works/HamPK21.pdf}{HamPK21}~\cite{HamPK21}, \href{../works/Groleaz21.pdf}{Groleaz21}~\cite{Groleaz21}, \href{../works/BenediktMH20.pdf}{BenediktMH20}~\cite{BenediktMH20}, \href{../works/ZarandiASC20.pdf}{ZarandiASC20}~\cite{ZarandiASC20}, \href{../works/BogaerdtW19.pdf}{BogaerdtW19}~\cite{BogaerdtW19}, \href{../works/BajestaniB15.pdf}{BajestaniB15}~\cite{BajestaniB15}, \href{../works/BajestaniB13.pdf}{BajestaniB13}~\cite{BajestaniB13}, \href{../works/TerekhovDOB12.pdf}{TerekhovDOB12}~\cite{TerekhovDOB12}, \href{../works/MalapertGR12.pdf}{MalapertGR12}~\cite{MalapertGR12}, \href{../works/KovacsB11.pdf}{KovacsB11}~\cite{KovacsB11}, \href{../works/WuBB09.pdf}{WuBB09}~\cite{WuBB09}, \href{../works/ThiruvadyBME09.pdf}{ThiruvadyBME09}~\cite{ThiruvadyBME09}, \href{../works/KovacsB07.pdf}{KovacsB07}~\cite{KovacsB07}, \href{../works/SadykovW06.pdf}{SadykovW06}~\cite{SadykovW06}, \href{../works/KanetAG04.pdf}{KanetAG04}~\cite{KanetAG04}, \href{../works/Elkhyari03.pdf}{Elkhyari03}~\cite{Elkhyari03}, \href{../works/Baptiste02.pdf}{Baptiste02}~\cite{Baptiste02}, \href{../works/BosiM2001.pdf}{BosiM2001}~\cite{BosiM2001}, \href{../works/SourdN00.pdf}{SourdN00}~\cite{SourdN00}, \href{../works/BlazewiczDP96.pdf}{BlazewiczDP96}~\cite{BlazewiczDP96} & \href{../works/NaderiBZ23.pdf}{NaderiBZ23}~\cite{NaderiBZ23}, \href{../works/ZhangBB22.pdf}{ZhangBB22}~\cite{ZhangBB22}, \href{../works/NaderiBZ22.pdf}{NaderiBZ22}~\cite{NaderiBZ22}, \href{../works/ElciOH22.pdf}{ElciOH22}~\cite{ElciOH22}, \href{../works/EmdeZD22.pdf}{EmdeZD22}~\cite{EmdeZD22}, \href{../works/YuraszeckMPV22.pdf}{YuraszeckMPV22}~\cite{YuraszeckMPV22}, \href{../works/Bedhief21.pdf}{Bedhief21}~\cite{Bedhief21}, \href{../works/LacknerMMWW21.pdf}{LacknerMMWW21}~\cite{LacknerMMWW21}, \href{../works/Zahout21.pdf}{Zahout21}~\cite{Zahout21}, \href{../works/KoehlerBFFHPSSS21.pdf}{KoehlerBFFHPSSS21}~\cite{KoehlerBFFHPSSS21}, \href{../works/PandeyS21a.pdf}{PandeyS21a}~\cite{PandeyS21a}, \href{../works/Astrand21.pdf}{Astrand21}~\cite{Astrand21}, \href{../works/HillTV21.pdf}{HillTV21}~\cite{HillTV21}, \href{../works/AbreuAPNM21.pdf}{AbreuAPNM21}~\cite{AbreuAPNM21}, \href{../works/NattafM20.pdf}{NattafM20}~\cite{NattafM20}, \href{../works/Lunardi20.pdf}{Lunardi20}~\cite{Lunardi20}, \href{../works/TanT18.pdf}{TanT18}~\cite{TanT18}, \href{../works/BenediktSMVH18.pdf}{BenediktSMVH18}~\cite{BenediktSMVH18}, \href{../works/Tesch18.pdf}{Tesch18}~\cite{Tesch18}...\href{../works/MilanoW09.pdf}{MilanoW09}~\cite{MilanoW09}, \href{../works/Jans09.pdf}{Jans09}~\cite{Jans09}, \href{../works/AkkerDH07.pdf}{AkkerDH07}~\cite{AkkerDH07}, \href{../works/MilanoW06.pdf}{MilanoW06}~\cite{MilanoW06}, \href{../works/MeyerE04.pdf}{MeyerE04}~\cite{MeyerE04}, \href{../works/Sadykov04.pdf}{Sadykov04}~\cite{Sadykov04}, \href{../works/OddiPCC03.pdf}{OddiPCC03}~\cite{OddiPCC03}, \href{../works/SchildW00.pdf}{SchildW00}~\cite{SchildW00}, \href{../works/JainM99.pdf}{JainM99}~\cite{JainM99}, \href{../works/BeckF98.pdf}{BeckF98}~\cite{BeckF98} (Total: 40) & \href{../works/abs-2402-00459.pdf}{abs-2402-00459}~\cite{abs-2402-00459}, \href{../works/LuZZYW24.pdf}{LuZZYW24}~\cite{LuZZYW24}, \href{../works/IsikYA23.pdf}{IsikYA23}~\cite{IsikYA23}, \href{../works/Fatemi-AnarakiTFV23.pdf}{Fatemi-AnarakiTFV23}~\cite{Fatemi-AnarakiTFV23}, \href{../works/NaderiRR23.pdf}{NaderiRR23}~\cite{NaderiRR23}, \href{../works/JuvinHL23a.pdf}{JuvinHL23a}~\cite{JuvinHL23a}, \href{../works/Mehdizadeh-Somarin23.pdf}{Mehdizadeh-Somarin23}~\cite{Mehdizadeh-Somarin23}, \href{../works/GeitzGSSW22.pdf}{GeitzGSSW22}~\cite{GeitzGSSW22}, \href{../works/JuvinHL22.pdf}{JuvinHL22}~\cite{JuvinHL22}, \href{../works/ZhangJZL22.pdf}{ZhangJZL22}~\cite{ZhangJZL22}, \href{../works/ColT22.pdf}{ColT22}~\cite{ColT22}, \href{../works/PohlAK22.pdf}{PohlAK22}~\cite{PohlAK22}, \href{../works/AbreuN22.pdf}{AbreuN22}~\cite{AbreuN22}, \href{../works/abs-2211-14492.pdf}{abs-2211-14492}~\cite{abs-2211-14492}, \href{../works/LiFJZLL22.pdf}{LiFJZLL22}~\cite{LiFJZLL22}, \href{../works/FanXG21.pdf}{FanXG21}~\cite{FanXG21}, \href{../works/Godet21a.pdf}{Godet21a}~\cite{Godet21a}, \href{../works/QinWSLS21.pdf}{QinWSLS21}~\cite{QinWSLS21}, \href{../works/KovacsTKSG21.pdf}{KovacsTKSG21}~\cite{KovacsTKSG21}...\href{../works/HeipckeCCS00.pdf}{HeipckeCCS00}~\cite{HeipckeCCS00}, \href{../works/Dorndorf2000.pdf}{Dorndorf2000}~\cite{Dorndorf2000}, \href{../works/TorresL00.pdf}{TorresL00}~\cite{TorresL00}, \href{../works/Beck99.pdf}{Beck99}~\cite{Beck99}, \href{../works/BeckDDF98.pdf}{BeckDDF98}~\cite{BeckDDF98}, \href{../works/NuijtenP98.pdf}{NuijtenP98}~\cite{NuijtenP98}, \href{../works/Darby-DowmanLMZ97.pdf}{Darby-DowmanLMZ97}~\cite{Darby-DowmanLMZ97}, \href{../works/BeckDF97.pdf}{BeckDF97}~\cite{BeckDF97}, \href{../works/BeckDSF97a.pdf}{BeckDSF97a}~\cite{BeckDSF97a}, \href{../works/SmithC93.pdf}{SmithC93}~\cite{SmithC93} (Total: 98)\\
\end{longtable}
}

\clearpage
\subsection{Concept Type Constraints}
\label{sec:Constraints}
\label{Constraints}
{\scriptsize
\begin{longtable}{p{3cm}r>{\raggedright\arraybackslash}p{6cm}>{\raggedright\arraybackslash}p{6cm}>{\raggedright\arraybackslash}p{8cm}}
\rowcolor{white}\caption{Works for Concepts of Type Constraints (Total 63 Concepts, 63 Used)}\\ \toprule
\rowcolor{white}Keyword & Weight & High & Medium & Low\\ \midrule\endhead
\bottomrule
\endfoot
\index{AllDiff constraint}\index{Constraints!AllDiff constraint}AllDiff constraint &  1.00 & \href{../works/WangB20.pdf}{WangB20}~\cite{WangB20} &  & \href{../works/Godet21a.pdf}{Godet21a}~\cite{Godet21a}, \href{../works/FahimiOQ18.pdf}{FahimiOQ18}~\cite{FahimiOQ18}, \href{../works/CarlssonJL17.pdf}{CarlssonJL17}~\cite{CarlssonJL17}, \href{../works/Fahimi16.pdf}{Fahimi16}~\cite{Fahimi16}, \href{../works/LarsonJC14.pdf}{LarsonJC14}~\cite{LarsonJC14}, \href{../works/Lombardi10.pdf}{Lombardi10}~\cite{Lombardi10}\\
\index{AllDiffPrec constraint}\index{Constraints!AllDiffPrec constraint}AllDiffPrec constraint &  1.00 & \href{../works/Godet21a.pdf}{Godet21a}~\cite{Godet21a} &  & \href{../works/JuvinHHL23.pdf}{JuvinHHL23}~\cite{JuvinHHL23}\\
\index{AlwaysConstant}\index{Constraints!AlwaysConstant}AlwaysConstant &  1.00 &  & \href{../works/LuoB22.pdf}{LuoB22}~\cite{LuoB22}, \href{../works/LaborieRSV18.pdf}{LaborieRSV18}~\cite{LaborieRSV18} & \\
\index{Among constraint}\index{Constraints!Among constraint}Among constraint &  1.00 & \href{../works/Siala15a.pdf}{Siala15a}~\cite{Siala15a}, \href{../works/Siala15.pdf}{Siala15}~\cite{Siala15}, \href{../works/BeldiceanuC94.pdf}{BeldiceanuC94}~\cite{BeldiceanuC94} & \href{../works/Simonis07.pdf}{Simonis07}~\cite{Simonis07}, \href{../works/BosiM2001.pdf}{BosiM2001}~\cite{BosiM2001} & \href{../works/German18.pdf}{German18}~\cite{German18}, \href{../works/HookerH17.pdf}{HookerH17}~\cite{HookerH17}, \href{../works/Refalo00.pdf}{Refalo00}~\cite{Refalo00}, \href{../works/Simonis95.pdf}{Simonis95}~\cite{Simonis95}, \href{../works/AggounB93.pdf}{AggounB93}~\cite{AggounB93}\\
\index{AmongSeq constraint}\index{Constraints!AmongSeq constraint}AmongSeq constraint &  1.00 &  & \href{../works/Siala15.pdf}{Siala15}~\cite{Siala15}, \href{../works/Siala15a.pdf}{Siala15a}~\cite{Siala15a} & \\
\index{Arithmetic constraint}\index{Constraints!Arithmetic constraint}Arithmetic constraint &  1.00 &  & \href{../works/ColT22.pdf}{ColT22}~\cite{ColT22} & \href{../works/BadicaBI20.pdf}{BadicaBI20}~\cite{BadicaBI20}, \href{../works/Caballero19.pdf}{Caballero19}~\cite{Caballero19}, \href{../works/BadicaBIL19.pdf}{BadicaBIL19}~\cite{BadicaBIL19}, \href{../works/LiuLH19a.pdf}{LiuLH19a}~\cite{LiuLH19a}, \href{../works/LaborieRSV18.pdf}{LaborieRSV18}~\cite{LaborieRSV18}, \href{../works/Schutt11.pdf}{Schutt11}~\cite{Schutt11}, \href{../works/OhrimenkoSC09.pdf}{OhrimenkoSC09}~\cite{OhrimenkoSC09}, \href{../works/Kuchcinski03.pdf}{Kuchcinski03}~\cite{Kuchcinski03}, \href{../works/Baptiste02.pdf}{Baptiste02}~\cite{Baptiste02}, \href{../works/ElkhyariGJ02a.pdf}{ElkhyariGJ02a}~\cite{ElkhyariGJ02a}, \href{../works/Thorsteinsson01.pdf}{Thorsteinsson01}~\cite{Thorsteinsson01}, \href{../works/Refalo00.pdf}{Refalo00}~\cite{Refalo00}, \href{../works/SakkoutW00.pdf}{SakkoutW00}~\cite{SakkoutW00}, \href{../works/AbdennadherS99.pdf}{AbdennadherS99}~\cite{AbdennadherS99}, \href{../works/FalaschiGMP97.pdf}{FalaschiGMP97}~\cite{FalaschiGMP97}, \href{../works/BeldiceanuC94.pdf}{BeldiceanuC94}~\cite{BeldiceanuC94}, \href{../works/AggounB93.pdf}{AggounB93}~\cite{AggounB93}\\
\index{AtMostSeq}\index{Constraints!AtMostSeq}AtMostSeq &  1.00 & \href{../works/Siala15a.pdf}{Siala15a}~\cite{Siala15a}, \href{../works/Siala15.pdf}{Siala15}~\cite{Siala15} &  & \\
\index{AtMostSeqCard}\index{Constraints!AtMostSeqCard}AtMostSeqCard &  1.00 & \href{../works/Siala15.pdf}{Siala15}~\cite{Siala15}, \href{../works/Siala15a.pdf}{Siala15a}~\cite{Siala15a} &  & \\
\index{Atmost constraint}\index{Constraints!Atmost constraint}Atmost constraint &  1.00 & \href{../works/Siala15a.pdf}{Siala15a}~\cite{Siala15a}, \href{../works/Siala15.pdf}{Siala15}~\cite{Siala15} &  & \href{../works/Simonis07.pdf}{Simonis07}~\cite{Simonis07}, \href{../works/BeldiceanuC94.pdf}{BeldiceanuC94}~\cite{BeldiceanuC94}\\
\index{Balance constraint}\index{Constraints!Balance constraint}Balance constraint &  1.00 & \href{../works/RoePS05.pdf}{RoePS05}~\cite{RoePS05}, \href{../works/Laborie03.pdf}{Laborie03}~\cite{Laborie03} & \href{../works/Timpe02.pdf}{Timpe02}~\cite{Timpe02}, \href{../works/Muscettola02.pdf}{Muscettola02}~\cite{Muscettola02} & \href{../works/GuoZ23.pdf}{GuoZ23}~\cite{GuoZ23}, \href{../works/PopovicCGNC22.pdf}{PopovicCGNC22}~\cite{PopovicCGNC22}, \href{../works/German18.pdf}{German18}~\cite{German18}, \href{../works/SchuttS16.pdf}{SchuttS16}~\cite{SchuttS16}, \href{../works/Siala15.pdf}{Siala15}~\cite{Siala15}, \href{../works/Siala15a.pdf}{Siala15a}~\cite{Siala15a}, \href{../works/GrimesH15.pdf}{GrimesH15}~\cite{GrimesH15}, \href{../works/Kameugne14.pdf}{Kameugne14}~\cite{Kameugne14}, \href{../works/DerrienPZ14.pdf}{DerrienPZ14}~\cite{DerrienPZ14}, \href{../works/TerekhovDOB12.pdf}{TerekhovDOB12}~\cite{TerekhovDOB12}, \href{../works/Lombardi10.pdf}{Lombardi10}~\cite{Lombardi10}, \href{../works/GrimesHM09.pdf}{GrimesHM09}~\cite{GrimesHM09}, \href{../works/LombardiM09.pdf}{LombardiM09}~\cite{LombardiM09}, \href{../works/BeckW07.pdf}{BeckW07}~\cite{BeckW07}, \href{../works/BeckW05.pdf}{BeckW05}~\cite{BeckW05}, \href{../works/Laborie05.pdf}{Laborie05}~\cite{Laborie05}, \href{../works/MaraveliasCG04.pdf}{MaraveliasCG04}~\cite{MaraveliasCG04}\\
\index{BinPacking constraint}\index{Constraints!BinPacking constraint}BinPacking constraint &  1.00 &  &  & \href{../works/Godet21a.pdf}{Godet21a}~\cite{Godet21a}, \href{../works/AntunesABD18.pdf}{AntunesABD18}~\cite{AntunesABD18}\\
\index{Blocking constraint}\index{Constraints!Blocking constraint}Blocking constraint &  1.00 & \href{../works/AbreuNP23.pdf}{AbreuNP23}~\cite{AbreuNP23}, \href{../works/RiahiNS018.pdf}{RiahiNS018}~\cite{RiahiNS018} &  & \href{../works/WessenCSFPM23.pdf}{WessenCSFPM23}~\cite{WessenCSFPM23}, \href{../works/IsikYA23.pdf}{IsikYA23}~\cite{IsikYA23}, \href{../works/LiFJZLL22.pdf}{LiFJZLL22}~\cite{LiFJZLL22}, \href{../works/MengZRZL20.pdf}{MengZRZL20}~\cite{MengZRZL20}, \href{../works/RodriguezS09.pdf}{RodriguezS09}~\cite{RodriguezS09}, \href{../works/Rodriguez07b.pdf}{Rodriguez07b}~\cite{Rodriguez07b}, \href{../works/Rodriguez07.pdf}{Rodriguez07}~\cite{Rodriguez07}\\
\index{BufferedResource}\index{Constraints!BufferedResource}BufferedResource &  1.00 & \href{../works/BessiereHMQW14.pdf}{BessiereHMQW14}~\cite{BessiereHMQW14} &  & \\
\index{Calendar constraint}\index{Constraints!Calendar constraint}Calendar constraint &  1.00 & \href{../works/KreterSSZ18.pdf}{KreterSSZ18}~\cite{KreterSSZ18}, \href{../works/KreterSS17.pdf}{KreterSS17}~\cite{KreterSS17} & \href{../works/KreterSS15.pdf}{KreterSS15}~\cite{KreterSS15} & \href{../works/PovedaAA23.pdf}{PovedaAA23}~\cite{PovedaAA23}, \href{../works/IsikYA23.pdf}{IsikYA23}~\cite{IsikYA23}, \href{../works/Polo-MejiaALB20.pdf}{Polo-MejiaALB20}~\cite{Polo-MejiaALB20}, \href{../works/LaborieRSV18.pdf}{LaborieRSV18}~\cite{LaborieRSV18}\\
\index{CardPath}\index{Constraints!CardPath}CardPath &  1.00 &  &  & \href{../works/Siala15a.pdf}{Siala15a}~\cite{Siala15a}, \href{../works/Siala15.pdf}{Siala15}~\cite{Siala15}\\
\index{Cardinality constraint}\index{Constraints!Cardinality constraint}Cardinality constraint &  1.00 & \href{../works/Caballero19.pdf}{Caballero19}~\cite{Caballero19}, \href{../works/Dejemeppe16.pdf}{Dejemeppe16}~\cite{Dejemeppe16}, \href{../works/Siala15.pdf}{Siala15}~\cite{Siala15}, \href{../works/Siala15a.pdf}{Siala15a}~\cite{Siala15a}, \href{../works/SchausHMCMD11.pdf}{SchausHMCMD11}~\cite{SchausHMCMD11}, \href{../works/Malik08.pdf}{Malik08}~\cite{Malik08} & \href{../works/OuelletQ22.pdf}{OuelletQ22}~\cite{OuelletQ22}, \href{../works/HoundjiSW19.pdf}{HoundjiSW19}~\cite{HoundjiSW19}, \href{../works/German18.pdf}{German18}~\cite{German18}, \href{../works/MusliuSS18.pdf}{MusliuSS18}~\cite{MusliuSS18}, \href{../works/HookerH17.pdf}{HookerH17}~\cite{HookerH17}, \href{../works/Fahimi16.pdf}{Fahimi16}~\cite{Fahimi16}, \href{../works/BofillGSV15.pdf}{BofillGSV15}~\cite{BofillGSV15}, \href{../works/HoundjiSWD14.pdf}{HoundjiSWD14}~\cite{HoundjiSWD14}, \href{../works/ChuGNSW13.pdf}{ChuGNSW13}~\cite{ChuGNSW13}, \href{../works/HachemiGR11.pdf}{HachemiGR11}~\cite{HachemiGR11}, \href{../works/MilanoW09.pdf}{MilanoW09}~\cite{MilanoW09}, \href{../works/MalikMB08.pdf}{MalikMB08}~\cite{MalikMB08}, \href{../works/Simonis07.pdf}{Simonis07}~\cite{Simonis07}, \href{../works/MilanoW06.pdf}{MilanoW06}~\cite{MilanoW06}, \href{../works/RussellU06.pdf}{RussellU06}~\cite{RussellU06}, \href{../works/Gronkvist06.pdf}{Gronkvist06}~\cite{Gronkvist06}, \href{../works/Perron05.pdf}{Perron05}~\cite{Perron05} & \href{../works/Godet21a.pdf}{Godet21a}~\cite{Godet21a}, \href{../works/Lemos21.pdf}{Lemos21}~\cite{Lemos21}, \href{../works/GeibingerKKMMW21.pdf}{GeibingerKKMMW21}~\cite{GeibingerKKMMW21}, \href{../works/CauwelaertDS20.pdf}{CauwelaertDS20}~\cite{CauwelaertDS20}, \href{../works/TangB20.pdf}{TangB20}~\cite{TangB20}, \href{../works/abs-1911-04766.pdf}{abs-1911-04766}~\cite{abs-1911-04766}, \href{../works/CarlssonJL17.pdf}{CarlssonJL17}~\cite{CarlssonJL17}, \href{../works/TranVNB17.pdf}{TranVNB17}~\cite{TranVNB17}, \href{../works/PesantRR15.pdf}{PesantRR15}~\cite{PesantRR15}, \href{../works/LarsonJC14.pdf}{LarsonJC14}~\cite{LarsonJC14}, \href{../works/DoulabiRP14.pdf}{DoulabiRP14}~\cite{DoulabiRP14}, \href{../works/BessiereHMQW14.pdf}{BessiereHMQW14}~\cite{BessiereHMQW14}, \href{../works/BajestaniB13.pdf}{BajestaniB13}~\cite{BajestaniB13}, \href{../works/LimtanyakulS12.pdf}{LimtanyakulS12}~\cite{LimtanyakulS12}, \href{../works/Menana11.pdf}{Menana11}~\cite{Menana11}, \href{../works/BajestaniB11.pdf}{BajestaniB11}~\cite{BajestaniB11}, \href{../works/ClercqPBJ11.pdf}{ClercqPBJ11}~\cite{ClercqPBJ11}, \href{../works/KovacsB11.pdf}{KovacsB11}~\cite{KovacsB11}, \href{../works/abs-0907-0939.pdf}{abs-0907-0939}~\cite{abs-0907-0939}...\href{../works/CambazardHDJT04.pdf}{CambazardHDJT04}~\cite{CambazardHDJT04}, \href{../works/BourdaisGP03.pdf}{BourdaisGP03}~\cite{BourdaisGP03}, \href{../works/Baptiste02.pdf}{Baptiste02}~\cite{Baptiste02}, \href{../works/Refalo00.pdf}{Refalo00}~\cite{Refalo00}, \href{../works/HookerOTK00.pdf}{HookerOTK00}~\cite{HookerOTK00}, \href{../works/BeckF00.pdf}{BeckF00}~\cite{BeckF00}, \href{../works/AbdennadherS99.pdf}{AbdennadherS99}~\cite{AbdennadherS99}, \href{../works/PapaB98.pdf}{PapaB98}~\cite{PapaB98}, \href{../works/PapeB97.pdf}{PapeB97}~\cite{PapeB97}, \href{../works/WeilHFP95.pdf}{WeilHFP95}~\cite{WeilHFP95} (Total: 32)\\
\index{Channeling constraint}\index{Constraints!Channeling constraint}Channeling constraint &  1.00 & \href{../works/OzturkTHO13.pdf}{OzturkTHO13}~\cite{OzturkTHO13}, \href{../works/Wallace06.pdf}{Wallace06}~\cite{Wallace06} & \href{../works/KoehlerBFFHPSSS21.pdf}{KoehlerBFFHPSSS21}~\cite{KoehlerBFFHPSSS21}, \href{../works/BofillEGPSV14.pdf}{BofillEGPSV14}~\cite{BofillEGPSV14}, \href{../works/HeinzB12.pdf}{HeinzB12}~\cite{HeinzB12} & \href{../works/WangB23.pdf}{WangB23}~\cite{WangB23}, \href{../works/AntuoriHHEN20.pdf}{AntuoriHHEN20}~\cite{AntuoriHHEN20}, \href{../works/LiuLH19.pdf}{LiuLH19}~\cite{LiuLH19}, \href{../works/GokgurHO18.pdf}{GokgurHO18}~\cite{GokgurHO18}, \href{../works/CarlssonJL17.pdf}{CarlssonJL17}~\cite{CarlssonJL17}, \href{../works/BofillGSV15.pdf}{BofillGSV15}~\cite{BofillGSV15}, \href{../works/HeinzKB13.pdf}{HeinzKB13}~\cite{HeinzKB13}, \href{../works/MalapertGR12.pdf}{MalapertGR12}~\cite{MalapertGR12}, \href{../works/KovacsB11.pdf}{KovacsB11}~\cite{KovacsB11}, \href{../works/WuBB09.pdf}{WuBB09}~\cite{WuBB09}, \href{../works/MilanoW09.pdf}{MilanoW09}~\cite{MilanoW09}, \href{../works/MouraSCL08.pdf}{MouraSCL08}~\cite{MouraSCL08}, \href{../works/MouraSCL08a.pdf}{MouraSCL08a}~\cite{MouraSCL08a}, \href{../works/GarganiR07.pdf}{GarganiR07}~\cite{GarganiR07}, \href{../works/MilanoW06.pdf}{MilanoW06}~\cite{MilanoW06}, \href{../works/Perron05.pdf}{Perron05}~\cite{Perron05}, \href{../works/CambazardHDJT04.pdf}{CambazardHDJT04}~\cite{CambazardHDJT04}\\
\index{Completion constraint}\index{Constraints!Completion constraint}Completion constraint &  1.00 & \href{../works/KovacsB11.pdf}{KovacsB11}~\cite{KovacsB11}, \href{../works/KovacsB08.pdf}{KovacsB08}~\cite{KovacsB08}, \href{../works/KovacsB07.pdf}{KovacsB07}~\cite{KovacsB07} & \href{../works/BonninMNE24.pdf}{BonninMNE24}~\cite{BonninMNE24} & \href{../works/HeckmanB11.pdf}{HeckmanB11}~\cite{HeckmanB11}\\
\index{CumulativeCost}\index{Constraints!CumulativeCost}CumulativeCost &  1.00 & \href{../works/SimonisH11.pdf}{SimonisH11}~\cite{SimonisH11} &  & \\
\index{Cumulatives constraint}\index{Constraints!Cumulatives constraint}Cumulatives constraint &  1.00 & \href{../works/BeldiceanuC02.pdf}{BeldiceanuC02}~\cite{BeldiceanuC02} & \href{../works/MossigeGSMC17.pdf}{MossigeGSMC17}~\cite{MossigeGSMC17}, \href{../works/Madi-WambaLOBM17.pdf}{Madi-WambaLOBM17}~\cite{Madi-WambaLOBM17} & \href{../works/KameugneFND23.pdf}{KameugneFND23}~\cite{KameugneFND23}, \href{../works/TardivoDFMP23.pdf}{TardivoDFMP23}~\cite{TardivoDFMP23}, \href{../works/OuelletQ22.pdf}{OuelletQ22}~\cite{OuelletQ22}, \href{../works/BoudreaultSLQ22.pdf}{BoudreaultSLQ22}~\cite{BoudreaultSLQ22}, \href{../works/ArkhipovBL19.pdf}{ArkhipovBL19}~\cite{ArkhipovBL19}, \href{../works/OuelletQ18.pdf}{OuelletQ18}~\cite{OuelletQ18}, \href{../works/FahimiOQ18.pdf}{FahimiOQ18}~\cite{FahimiOQ18}, \href{../works/Fahimi16.pdf}{Fahimi16}~\cite{Fahimi16}, \href{../works/SchuttS16.pdf}{SchuttS16}~\cite{SchuttS16}, \href{../works/Dejemeppe16.pdf}{Dejemeppe16}~\cite{Dejemeppe16}, \href{../works/GayHS15a.pdf}{GayHS15a}~\cite{GayHS15a}, \href{../works/LetortCB15.pdf}{LetortCB15}~\cite{LetortCB15}, \href{../works/GayHS15.pdf}{GayHS15}~\cite{GayHS15}, \href{../works/CauwelaertLS15.pdf}{CauwelaertLS15}~\cite{CauwelaertLS15}, \href{../works/Kameugne14.pdf}{Kameugne14}~\cite{Kameugne14}, \href{../works/DerrienPZ14.pdf}{DerrienPZ14}~\cite{DerrienPZ14}, \href{../works/Letort13.pdf}{Letort13}~\cite{Letort13}, \href{../works/OuelletQ13.pdf}{OuelletQ13}~\cite{OuelletQ13}, \href{../works/Clercq12.pdf}{Clercq12}~\cite{Clercq12}, \href{../works/LetortBC12.pdf}{LetortBC12}~\cite{LetortBC12}, \href{../works/SimonisH11.pdf}{SimonisH11}~\cite{SimonisH11}, \href{../works/ClercqPBJ11.pdf}{ClercqPBJ11}~\cite{ClercqPBJ11}, \href{../works/Malapert11.pdf}{Malapert11}~\cite{Malapert11}, \href{../works/Wolf11.pdf}{Wolf11}~\cite{Wolf11}, \href{../works/MilanoW09.pdf}{MilanoW09}~\cite{MilanoW09}, \href{../works/abs-0907-0939.pdf}{abs-0907-0939}~\cite{abs-0907-0939}, \href{../works/Simonis07.pdf}{Simonis07}~\cite{Simonis07}, \href{../works/MilanoW06.pdf}{MilanoW06}~\cite{MilanoW06}\\
\index{Diff2 constraint}\index{Constraints!Diff2 constraint}Diff2 constraint &  1.00 & \href{../works/Kuchcinski03.pdf}{Kuchcinski03}~\cite{Kuchcinski03} &  & \href{../works/WolinskiKG04.pdf}{WolinskiKG04}~\cite{WolinskiKG04}, \href{../works/KuchcinskiW03.pdf}{KuchcinskiW03}~\cite{KuchcinskiW03}\\
\index{Disjunctive constraint}\index{Constraints!Disjunctive constraint}Disjunctive constraint &  1.00 & \href{../works/KoehlerBFFHPSSS21.pdf}{KoehlerBFFHPSSS21}~\cite{KoehlerBFFHPSSS21}, \href{../works/Godet21a.pdf}{Godet21a}~\cite{Godet21a}, \href{../works/GrimesH15.pdf}{GrimesH15}~\cite{GrimesH15}, \href{../works/Malapert11.pdf}{Malapert11}~\cite{Malapert11}, \href{../works/RoePS05.pdf}{RoePS05}~\cite{RoePS05}, \href{../works/Baptiste02.pdf}{Baptiste02}~\cite{Baptiste02}, \href{../works/Dorndorf2000.pdf}{Dorndorf2000}~\cite{Dorndorf2000}, \href{../works/SourdN00.pdf}{SourdN00}~\cite{SourdN00}, \href{../works/HookerO99.pdf}{HookerO99}~\cite{HookerO99}, \href{../works/RodosekWH99.pdf}{RodosekWH99}~\cite{RodosekWH99}, \href{../works/PapaB98.pdf}{PapaB98}~\cite{PapaB98}, \href{../works/RodosekW98.pdf}{RodosekW98}~\cite{RodosekW98}, \href{../works/Zhou97.pdf}{Zhou97}~\cite{Zhou97}, \href{../works/BaptisteP95.pdf}{BaptisteP95}~\cite{BaptisteP95}, \href{../works/DincbasSH90.pdf}{DincbasSH90}~\cite{DincbasSH90} & \href{../works/BonninMNE24.pdf}{BonninMNE24}~\cite{BonninMNE24}, \href{../works/JuvinHHL23.pdf}{JuvinHHL23}~\cite{JuvinHHL23}, \href{../works/NaderiRR23.pdf}{NaderiRR23}~\cite{NaderiRR23}, \href{../works/BourreauGGLT22.pdf}{BourreauGGLT22}~\cite{BourreauGGLT22}, \href{../works/GodetLHS20.pdf}{GodetLHS20}~\cite{GodetLHS20}, \href{../works/GokgurHO18.pdf}{GokgurHO18}~\cite{GokgurHO18}, \href{../works/KuB16.pdf}{KuB16}~\cite{KuB16}, \href{../works/Fahimi16.pdf}{Fahimi16}~\cite{Fahimi16}, \href{../works/Siala15a.pdf}{Siala15a}~\cite{Siala15a}, \href{../works/MelgarejoLS15.pdf}{MelgarejoLS15}~\cite{MelgarejoLS15}, \href{../works/Siala15.pdf}{Siala15}~\cite{Siala15}, \href{../works/SialaAH15.pdf}{SialaAH15}~\cite{SialaAH15}, \href{../works/SchuttFS13.pdf}{SchuttFS13}~\cite{SchuttFS13}, \href{../works/OzturkTHO13.pdf}{OzturkTHO13}~\cite{OzturkTHO13}, \href{../works/GrimesH11.pdf}{GrimesH11}~\cite{GrimesH11}, \href{../works/LombardiM10a.pdf}{LombardiM10a}~\cite{LombardiM10a}, \href{../works/Lombardi10.pdf}{Lombardi10}~\cite{Lombardi10}, \href{../works/BartakSR10.pdf}{BartakSR10}~\cite{BartakSR10}, \href{../works/GrimesH10.pdf}{GrimesH10}~\cite{GrimesH10}...\href{../works/Kuchcinski03.pdf}{Kuchcinski03}~\cite{Kuchcinski03}, \href{../works/Laborie03.pdf}{Laborie03}~\cite{Laborie03}, \href{../works/ElkhyariGJ02a.pdf}{ElkhyariGJ02a}~\cite{ElkhyariGJ02a}, \href{../works/SchildW00.pdf}{SchildW00}~\cite{SchildW00}, \href{../works/FocacciLN00.pdf}{FocacciLN00}~\cite{FocacciLN00}, \href{../works/SakkoutW00.pdf}{SakkoutW00}~\cite{SakkoutW00}, \href{../works/BeckF00.pdf}{BeckF00}~\cite{BeckF00}, \href{../works/BelhadjiI98.pdf}{BelhadjiI98}~\cite{BelhadjiI98}, \href{../works/Darby-DowmanLMZ97.pdf}{Darby-DowmanLMZ97}~\cite{Darby-DowmanLMZ97}, \href{../works/Zhou96.pdf}{Zhou96}~\cite{Zhou96} (Total: 34) & \href{../works/abs-2402-00459.pdf}{abs-2402-00459}~\cite{abs-2402-00459}, \href{../works/KameugneFND23.pdf}{KameugneFND23}~\cite{KameugneFND23}, \href{../works/Bit-Monnot23.pdf}{Bit-Monnot23}~\cite{Bit-Monnot23}, \href{../works/MarliereSPR23.pdf}{MarliereSPR23}~\cite{MarliereSPR23}, \href{../works/JuvinHL23a.pdf}{JuvinHL23a}~\cite{JuvinHL23a}, \href{../works/NaderiBZ23.pdf}{NaderiBZ23}~\cite{NaderiBZ23}, \href{../works/NaderiBZ22a.pdf}{NaderiBZ22a}~\cite{NaderiBZ22a}, \href{../works/KotaryFH22.pdf}{KotaryFH22}~\cite{KotaryFH22}, \href{../works/JuvinHL22.pdf}{JuvinHL22}~\cite{JuvinHL22}, \href{../works/ZhangBB22.pdf}{ZhangBB22}~\cite{ZhangBB22}, \href{../works/abs-2211-14492.pdf}{abs-2211-14492}~\cite{abs-2211-14492}, \href{../works/BoudreaultSLQ22.pdf}{BoudreaultSLQ22}~\cite{BoudreaultSLQ22}, \href{../works/YuraszeckMPV22.pdf}{YuraszeckMPV22}~\cite{YuraszeckMPV22}, \href{../works/NaderiBZ22.pdf}{NaderiBZ22}~\cite{NaderiBZ22}, \href{../works/Groleaz21.pdf}{Groleaz21}~\cite{Groleaz21}, \href{../works/Astrand21.pdf}{Astrand21}~\cite{Astrand21}, \href{../works/Astrand0F21.pdf}{Astrand0F21}~\cite{Astrand0F21}, \href{../works/WallaceY20.pdf}{WallaceY20}~\cite{WallaceY20}, \href{../works/Polo-MejiaALB20.pdf}{Polo-MejiaALB20}~\cite{Polo-MejiaALB20}...\href{../works/NuijtenP98.pdf}{NuijtenP98}~\cite{NuijtenP98}, \href{../works/PintoG97.pdf}{PintoG97}~\cite{PintoG97}, \href{../works/LammaMM97.pdf}{LammaMM97}~\cite{LammaMM97}, \href{../works/BaptisteP97.pdf}{BaptisteP97}~\cite{BaptisteP97}, \href{../works/SadehF96.pdf}{SadehF96}~\cite{SadehF96}, \href{../works/BrusoniCLMMT96.pdf}{BrusoniCLMMT96}~\cite{BrusoniCLMMT96}, \href{../works/NuijtenA96.pdf}{NuijtenA96}~\cite{NuijtenA96}, \href{../works/BlazewiczDP96.pdf}{BlazewiczDP96}~\cite{BlazewiczDP96}, \href{../works/NuijtenA94.pdf}{NuijtenA94}~\cite{NuijtenA94}, \href{../works/FoxS90.pdf}{FoxS90}~\cite{FoxS90} (Total: 83)\\
\index{Element constraint}\index{Constraints!Element constraint}Element constraint &  1.00 & \href{../works/Dejemeppe16.pdf}{Dejemeppe16}~\cite{Dejemeppe16}, \href{../works/HookerOTK00.pdf}{HookerOTK00}~\cite{HookerOTK00} & \href{../works/LiuLH18.pdf}{LiuLH18}~\cite{LiuLH18}, \href{../works/KreterSS17.pdf}{KreterSS17}~\cite{KreterSS17}, \href{../works/Wolf11.pdf}{Wolf11}~\cite{Wolf11}, \href{../works/Kuchcinski03.pdf}{Kuchcinski03}~\cite{Kuchcinski03}, \href{../works/Darby-DowmanLMZ97.pdf}{Darby-DowmanLMZ97}~\cite{Darby-DowmanLMZ97} & \href{../works/LacknerMMWW23.pdf}{LacknerMMWW23}~\cite{LacknerMMWW23}, \href{../works/LuoB22.pdf}{LuoB22}~\cite{LuoB22}, \href{../works/Godet21a.pdf}{Godet21a}~\cite{Godet21a}, \href{../works/LacknerMMWW21.pdf}{LacknerMMWW21}~\cite{LacknerMMWW21}, \href{../works/TangB20.pdf}{TangB20}~\cite{TangB20}, \href{../works/AntuoriHHEN20.pdf}{AntuoriHHEN20}~\cite{AntuoriHHEN20}, \href{../works/KreterSSZ18.pdf}{KreterSSZ18}~\cite{KreterSSZ18}, \href{../works/LiuCGM17.pdf}{LiuCGM17}~\cite{LiuCGM17}, \href{../works/Madi-WambaLOBM17.pdf}{Madi-WambaLOBM17}~\cite{Madi-WambaLOBM17}, \href{../works/SzerediS16.pdf}{SzerediS16}~\cite{SzerediS16}, \href{../works/OrnekO16.pdf}{OrnekO16}~\cite{OrnekO16}, \href{../works/DoulabiRP16.pdf}{DoulabiRP16}~\cite{DoulabiRP16}, \href{../works/KreterSS15.pdf}{KreterSS15}~\cite{KreterSS15}, \href{../works/HoundjiSWD14.pdf}{HoundjiSWD14}~\cite{HoundjiSWD14}, \href{../works/BessiereHMQW14.pdf}{BessiereHMQW14}~\cite{BessiereHMQW14}, \href{../works/DoulabiRP14.pdf}{DoulabiRP14}~\cite{DoulabiRP14}, \href{../works/OzturkTHO12.pdf}{OzturkTHO12}~\cite{OzturkTHO12}, \href{../works/ZengM12.pdf}{ZengM12}~\cite{ZengM12}, \href{../works/SimonisH11.pdf}{SimonisH11}~\cite{SimonisH11}, \href{../works/SchausHMCMD11.pdf}{SchausHMCMD11}~\cite{SchausHMCMD11}, \href{../works/Malapert11.pdf}{Malapert11}~\cite{Malapert11}, \href{../works/Schutt11.pdf}{Schutt11}~\cite{Schutt11}, \href{../works/MouraSCL08.pdf}{MouraSCL08}~\cite{MouraSCL08}, \href{../works/SchausD08.pdf}{SchausD08}~\cite{SchausD08}, \href{../works/GarganiR07.pdf}{GarganiR07}~\cite{GarganiR07}, \href{../works/CambazardHDJT04.pdf}{CambazardHDJT04}~\cite{CambazardHDJT04}, \href{../works/Refalo00.pdf}{Refalo00}~\cite{Refalo00}, \href{../works/BeldiceanuC94.pdf}{BeldiceanuC94}~\cite{BeldiceanuC94}\\
\index{Flowtime constraint}\index{Constraints!Flowtime constraint}Flowtime constraint &  1.00 & \href{../works/BonninMNE24.pdf}{BonninMNE24}~\cite{BonninMNE24} &  & \\
\index{GCC constraint}\index{Constraints!GCC constraint}GCC constraint &  1.00 & \href{../works/HoundjiSW19.pdf}{HoundjiSW19}~\cite{HoundjiSW19}, \href{../works/Dejemeppe16.pdf}{Dejemeppe16}~\cite{Dejemeppe16}, \href{../works/HoundjiSWD14.pdf}{HoundjiSWD14}~\cite{HoundjiSWD14} & \href{../works/SchausHMCMD11.pdf}{SchausHMCMD11}~\cite{SchausHMCMD11} & \href{../works/OuelletQ22.pdf}{OuelletQ22}~\cite{OuelletQ22}, \href{../works/TangB20.pdf}{TangB20}~\cite{TangB20}, \href{../works/CauwelaertLS18.pdf}{CauwelaertLS18}~\cite{CauwelaertLS18}, \href{../works/Siala15.pdf}{Siala15}~\cite{Siala15}, \href{../works/Siala15a.pdf}{Siala15a}~\cite{Siala15a}, \href{../works/CauwelaertLS15.pdf}{CauwelaertLS15}~\cite{CauwelaertLS15}, \href{../works/BajestaniB13.pdf}{BajestaniB13}~\cite{BajestaniB13}, \href{../works/HachemiGR11.pdf}{HachemiGR11}~\cite{HachemiGR11}, \href{../works/MilanoW09.pdf}{MilanoW09}~\cite{MilanoW09}, \href{../works/Simonis07.pdf}{Simonis07}~\cite{Simonis07}, \href{../works/Gronkvist06.pdf}{Gronkvist06}~\cite{Gronkvist06}, \href{../works/MilanoW06.pdf}{MilanoW06}~\cite{MilanoW06}\\
\index{GeneralizedAllDiffPrec}\index{Constraints!GeneralizedAllDiffPrec}GeneralizedAllDiffPrec &  1.00 & \href{../works/Godet21a.pdf}{Godet21a}~\cite{Godet21a} &  & \\
\index{IloAlternative}\index{Constraints!IloAlternative}IloAlternative &  1.00 &  &  & \href{../works/HeinzB12.pdf}{HeinzB12}~\cite{HeinzB12}\\
\index{IloAlwaysIn}\index{Constraints!IloAlwaysIn}IloAlwaysIn &  1.00 &  &  & \href{../works/KreterSS17.pdf}{KreterSS17}~\cite{KreterSS17}, \href{../works/BajestaniB13.pdf}{BajestaniB13}~\cite{BajestaniB13}\\
\index{IloForbidEnd}\index{Constraints!IloForbidEnd}IloForbidEnd &  1.00 &  &  & \href{../works/KreterSS17.pdf}{KreterSS17}~\cite{KreterSS17}\\
\index{IloNoOverlap}\index{Constraints!IloNoOverlap}IloNoOverlap &  1.00 &  &  & \href{../works/GrimesH15.pdf}{GrimesH15}~\cite{GrimesH15}\\
\index{IloPack}\index{Constraints!IloPack}IloPack &  1.00 &  & \href{../works/SchausD08.pdf}{SchausD08}~\cite{SchausD08} & \\
\index{IloPulse}\index{Constraints!IloPulse}IloPulse &  1.00 &  &  & \href{../works/KreterSS17.pdf}{KreterSS17}~\cite{KreterSS17}, \href{../works/BajestaniB13.pdf}{BajestaniB13}~\cite{BajestaniB13}\\
\index{MinWeightAllDiff}\index{Constraints!MinWeightAllDiff}MinWeightAllDiff &  1.00 & \href{../works/WangB20.pdf}{WangB20}~\cite{WangB20} &  & \href{../works/WangB23.pdf}{WangB23}~\cite{WangB23}\\
\index{MultiAtMostSeqCard}\index{Constraints!MultiAtMostSeqCard}MultiAtMostSeqCard &  1.00 & \href{../works/Siala15.pdf}{Siala15}~\cite{Siala15}, \href{../works/Siala15a.pdf}{Siala15a}~\cite{Siala15a} &  & \\
\index{PreemptiveNoOverlap}\index{Constraints!PreemptiveNoOverlap}PreemptiveNoOverlap &  1.00 & \href{../works/JuvinHHL23.pdf}{JuvinHHL23}~\cite{JuvinHHL23} &  & \\
\index{Pulse constraint}\index{Constraints!Pulse constraint}Pulse constraint &  1.00 &  &  & \href{../works/PandeyS21a.pdf}{PandeyS21a}~\cite{PandeyS21a}, \href{../works/GeibingerMM19.pdf}{GeibingerMM19}~\cite{GeibingerMM19}, \href{../works/ArbaouiY18.pdf}{ArbaouiY18}~\cite{ArbaouiY18}, \href{../works/KreterSS17.pdf}{KreterSS17}~\cite{KreterSS17}\\
\index{Regular constraint}\index{Constraints!Regular constraint}Regular constraint &  1.00 & \href{../works/MusliuSS18.pdf}{MusliuSS18}~\cite{MusliuSS18}, \href{../works/Siala15.pdf}{Siala15}~\cite{Siala15}, \href{../works/Siala15a.pdf}{Siala15a}~\cite{Siala15a}, \href{../works/PesantRR15.pdf}{PesantRR15}~\cite{PesantRR15} & \href{../works/HookerH17.pdf}{HookerH17}~\cite{HookerH17}, \href{../works/Dejemeppe16.pdf}{Dejemeppe16}~\cite{Dejemeppe16} & \href{../works/WessenCSFPM23.pdf}{WessenCSFPM23}~\cite{WessenCSFPM23}, \href{../works/FrimodigS19.pdf}{FrimodigS19}~\cite{FrimodigS19}, \href{../works/PraletLJ15.pdf}{PraletLJ15}~\cite{PraletLJ15}, \href{../works/LarsonJC14.pdf}{LarsonJC14}~\cite{LarsonJC14}, \href{../works/KovacsB11.pdf}{KovacsB11}~\cite{KovacsB11}, \href{../works/Menana11.pdf}{Menana11}~\cite{Menana11}, \href{../works/KovacsB08.pdf}{KovacsB08}~\cite{KovacsB08}, \href{../works/HenzMT04.pdf}{HenzMT04}~\cite{HenzMT04}\\
\index{Reified constraint}\index{Constraints!Reified constraint}Reified constraint &  1.00 & \href{../works/Schutt11.pdf}{Schutt11}~\cite{Schutt11}, \href{../works/MilanoW09.pdf}{MilanoW09}~\cite{MilanoW09}, \href{../works/Kuchcinski03.pdf}{Kuchcinski03}~\cite{Kuchcinski03} & \href{../works/LiuLH19a.pdf}{LiuLH19a}~\cite{LiuLH19a}, \href{../works/KovacsK11.pdf}{KovacsK11}~\cite{KovacsK11}, \href{../works/MilanoW06.pdf}{MilanoW06}~\cite{MilanoW06} & \href{../works/Astrand21.pdf}{Astrand21}~\cite{Astrand21}, \href{../works/BadicaBI20.pdf}{BadicaBI20}~\cite{BadicaBI20}, \href{../works/LaborieRSV18.pdf}{LaborieRSV18}~\cite{LaborieRSV18}, \href{../works/CauwelaertLS18.pdf}{CauwelaertLS18}~\cite{CauwelaertLS18}, \href{../works/KreterSS17.pdf}{KreterSS17}~\cite{KreterSS17}, \href{../works/Dejemeppe16.pdf}{Dejemeppe16}~\cite{Dejemeppe16}, \href{../works/Siala15.pdf}{Siala15}~\cite{Siala15}, \href{../works/Siala15a.pdf}{Siala15a}~\cite{Siala15a}, \href{../works/SchuttFSW13.pdf}{SchuttFSW13}~\cite{SchuttFSW13}, \href{../works/OhrimenkoSC09.pdf}{OhrimenkoSC09}~\cite{OhrimenkoSC09}, \href{../works/SchausD08.pdf}{SchausD08}~\cite{SchausD08}, \href{../works/SchildW00.pdf}{SchildW00}~\cite{SchildW00}\\
\index{RelSoftCumulative}\index{Constraints!RelSoftCumulative}RelSoftCumulative &  1.00 & \href{../works/abs-0907-0939.pdf}{abs-0907-0939}~\cite{abs-0907-0939} &  & \\
\index{RelSoftCumulativeSum}\index{Constraints!RelSoftCumulativeSum}RelSoftCumulativeSum &  1.00 &  &  & \href{../works/abs-0907-0939.pdf}{abs-0907-0939}~\cite{abs-0907-0939}\\
\index{SoftCumulative}\index{Constraints!SoftCumulative}SoftCumulative &  1.00 & \href{../works/Clercq12.pdf}{Clercq12}~\cite{Clercq12}, \href{../works/ClercqPBJ11.pdf}{ClercqPBJ11}~\cite{ClercqPBJ11}, \href{../works/abs-0907-0939.pdf}{abs-0907-0939}~\cite{abs-0907-0939} & \href{../works/OuelletQ22.pdf}{OuelletQ22}~\cite{OuelletQ22} & \\
\index{SoftCumulativeSum}\index{Constraints!SoftCumulativeSum}SoftCumulativeSum &  1.00 & \href{../works/Clercq12.pdf}{Clercq12}~\cite{Clercq12}, \href{../works/abs-0907-0939.pdf}{abs-0907-0939}~\cite{abs-0907-0939} &  & \href{../works/ClercqPBJ11.pdf}{ClercqPBJ11}~\cite{ClercqPBJ11}\\
\index{TaskIntersection constraint}\index{Constraints!TaskIntersection constraint}TaskIntersection constraint &  1.00 & \href{../works/Madi-WambaB16.pdf}{Madi-WambaB16}~\cite{Madi-WambaB16} &  & \\
\index{UTVPI constraint}\index{Constraints!UTVPI constraint}UTVPI constraint &  1.00 & \href{../works/Schutt11.pdf}{Schutt11}~\cite{Schutt11} &  & \\
\index{WeightAllDiff}\index{Constraints!WeightAllDiff}WeightAllDiff &  1.00 & \href{../works/WangB20.pdf}{WangB20}~\cite{WangB20} &  & \href{../works/WangB23.pdf}{WangB23}~\cite{WangB23}\\
\index{WeightedSum}\index{Constraints!WeightedSum}WeightedSum &  1.00 & \href{../works/Wolf09.pdf}{Wolf09}~\cite{Wolf09} &  & \\
\index{WeightedTaskSum}\index{Constraints!WeightedTaskSum}WeightedTaskSum &  1.00 & \href{../works/Wolf09.pdf}{Wolf09}~\cite{Wolf09} &  & \\
\index{alldifferent}\index{Constraints!alldifferent}alldifferent &  1.00 & \href{../works/FalqueALM24.pdf}{FalqueALM24}~\cite{FalqueALM24}, \href{../works/JuvinHHL23.pdf}{JuvinHHL23}~\cite{JuvinHHL23}, \href{../works/Lemos21.pdf}{Lemos21}~\cite{Lemos21}, \href{../works/KoehlerBFFHPSSS21.pdf}{KoehlerBFFHPSSS21}~\cite{KoehlerBFFHPSSS21}, \href{../works/Godet21a.pdf}{Godet21a}~\cite{Godet21a}, \href{../works/HoundjiSW19.pdf}{HoundjiSW19}~\cite{HoundjiSW19}, \href{../works/LiuLH19a.pdf}{LiuLH19a}~\cite{LiuLH19a}, \href{../works/CauwelaertLS18.pdf}{CauwelaertLS18}~\cite{CauwelaertLS18}, \href{../works/CarlssonJL17.pdf}{CarlssonJL17}~\cite{CarlssonJL17}, \href{../works/Dejemeppe16.pdf}{Dejemeppe16}~\cite{Dejemeppe16}, \href{../works/Siala15.pdf}{Siala15}~\cite{Siala15}, \href{../works/Derrien15.pdf}{Derrien15}~\cite{Derrien15}, \href{../works/Siala15a.pdf}{Siala15a}~\cite{Siala15a}, \href{../works/LarsonJC14.pdf}{LarsonJC14}~\cite{LarsonJC14}, \href{../works/Clercq12.pdf}{Clercq12}~\cite{Clercq12}, \href{../works/Menana11.pdf}{Menana11}~\cite{Menana11}, \href{../works/Malapert11.pdf}{Malapert11}~\cite{Malapert11}, \href{../works/MilanoW09.pdf}{MilanoW09}~\cite{MilanoW09}, \href{../works/OhrimenkoSC09.pdf}{OhrimenkoSC09}~\cite{OhrimenkoSC09}, \href{../works/Simonis07.pdf}{Simonis07}~\cite{Simonis07}, \href{../works/RussellU06.pdf}{RussellU06}~\cite{RussellU06}, \href{../works/MilanoW06.pdf}{MilanoW06}~\cite{MilanoW06}, \href{../works/KanetAG04.pdf}{KanetAG04}~\cite{KanetAG04} & \href{../works/GodetLHS20.pdf}{GodetLHS20}~\cite{GodetLHS20}, \href{../works/HookerH17.pdf}{HookerH17}~\cite{HookerH17}, \href{../works/Fahimi16.pdf}{Fahimi16}~\cite{Fahimi16}, \href{../works/BessiereHMQW14.pdf}{BessiereHMQW14}~\cite{BessiereHMQW14}, \href{../works/UnsalO13.pdf}{UnsalO13}~\cite{UnsalO13}, \href{../works/KelarevaTK13.pdf}{KelarevaTK13}~\cite{KelarevaTK13}, \href{../works/TerekhovDOB12.pdf}{TerekhovDOB12}~\cite{TerekhovDOB12}, \href{../works/ZengM12.pdf}{ZengM12}~\cite{ZengM12}, \href{../works/Schutt11.pdf}{Schutt11}~\cite{Schutt11} & \href{../works/GokPTGO23.pdf}{GokPTGO23}~\cite{GokPTGO23}, \href{../works/WangB23.pdf}{WangB23}~\cite{WangB23}, \href{../works/ColT22.pdf}{ColT22}~\cite{ColT22}, \href{../works/FarsiTM22.pdf}{FarsiTM22}~\cite{FarsiTM22}, \href{../works/BourreauGGLT22.pdf}{BourreauGGLT22}~\cite{BourreauGGLT22}, \href{../works/Astrand21.pdf}{Astrand21}~\cite{Astrand21}, \href{../works/MokhtarzadehTNF20.pdf}{MokhtarzadehTNF20}~\cite{MokhtarzadehTNF20}, \href{../works/AntuoriHHEN20.pdf}{AntuoriHHEN20}~\cite{AntuoriHHEN20}, \href{../works/AstrandJZ20.pdf}{AstrandJZ20}~\cite{AstrandJZ20}, \href{../works/WangB20.pdf}{WangB20}~\cite{WangB20}, \href{../works/Lunardi20.pdf}{Lunardi20}~\cite{Lunardi20}, \href{../works/Caballero19.pdf}{Caballero19}~\cite{Caballero19}, \href{../works/FahimiOQ18.pdf}{FahimiOQ18}~\cite{FahimiOQ18}, \href{../works/Nattaf16.pdf}{Nattaf16}~\cite{Nattaf16}, \href{../works/MelgarejoLS15.pdf}{MelgarejoLS15}~\cite{MelgarejoLS15}, \href{../works/AlesioNBG14.pdf}{AlesioNBG14}~\cite{AlesioNBG14}, \href{../works/Letort13.pdf}{Letort13}~\cite{Letort13}, \href{../works/ChuGNSW13.pdf}{ChuGNSW13}~\cite{ChuGNSW13}, \href{../works/ClercqPBJ11.pdf}{ClercqPBJ11}~\cite{ClercqPBJ11}, \href{../works/HachemiGR11.pdf}{HachemiGR11}~\cite{HachemiGR11}, \href{../works/HermenierDL11.pdf}{HermenierDL11}~\cite{HermenierDL11}, \href{../works/TrojetHL11.pdf}{TrojetHL11}~\cite{TrojetHL11}, \href{../works/LopesCSM10.pdf}{LopesCSM10}~\cite{LopesCSM10}, \href{../works/Malik08.pdf}{Malik08}~\cite{Malik08}, \href{../works/RasmussenT07.pdf}{RasmussenT07}~\cite{RasmussenT07}, \href{../works/Thorsteinsson01.pdf}{Thorsteinsson01}~\cite{Thorsteinsson01}, \href{../works/BeldiceanuC01.pdf}{BeldiceanuC01}~\cite{BeldiceanuC01}, \href{../works/Simonis99.pdf}{Simonis99}~\cite{Simonis99}, \href{../works/BeldiceanuC94.pdf}{BeldiceanuC94}~\cite{BeldiceanuC94}\\
\index{alternative constraint}\index{Constraints!alternative constraint}alternative constraint &  1.00 & \href{../works/LaborieRSV18.pdf}{LaborieRSV18}~\cite{LaborieRSV18} & \href{../works/abs-2305-19888.pdf}{abs-2305-19888}~\cite{abs-2305-19888}, \href{../works/MurinR19.pdf}{MurinR19}~\cite{MurinR19}, \href{../works/GokgurHO18.pdf}{GokgurHO18}~\cite{GokgurHO18}, \href{../works/GedikKBR17.pdf}{GedikKBR17}~\cite{GedikKBR17}, \href{../works/ZhaoL14.pdf}{ZhaoL14}~\cite{ZhaoL14}, \href{../works/LaborieR14.pdf}{LaborieR14}~\cite{LaborieR14} & \href{../works/ZhuSZW23.pdf}{ZhuSZW23}~\cite{ZhuSZW23}, \href{../works/MarliereSPR23.pdf}{MarliereSPR23}~\cite{MarliereSPR23}, \href{../works/LacknerMMWW23.pdf}{LacknerMMWW23}~\cite{LacknerMMWW23}, \href{../works/WessenCSFPM23.pdf}{WessenCSFPM23}~\cite{WessenCSFPM23}, \href{../works/NaderiRR23.pdf}{NaderiRR23}~\cite{NaderiRR23}, \href{../works/SvancaraB22.pdf}{SvancaraB22}~\cite{SvancaraB22}, \href{../works/WinterMMW22.pdf}{WinterMMW22}~\cite{WinterMMW22}, \href{../works/HeinzNVH22.pdf}{HeinzNVH22}~\cite{HeinzNVH22}, \href{../works/AwadMDMT22.pdf}{AwadMDMT22}~\cite{AwadMDMT22}, \href{../works/ZhangJZL22.pdf}{ZhangJZL22}~\cite{ZhangJZL22}, \href{../works/ArmstrongGOS21.pdf}{ArmstrongGOS21}~\cite{ArmstrongGOS21}, \href{../works/PandeyS21a.pdf}{PandeyS21a}~\cite{PandeyS21a}, \href{../works/VlkHT21.pdf}{VlkHT21}~\cite{VlkHT21}, \href{../works/HillTV21.pdf}{HillTV21}~\cite{HillTV21}, \href{../works/MengLZB21.pdf}{MengLZB21}~\cite{MengLZB21}, \href{../works/HubnerGSV21.pdf}{HubnerGSV21}~\cite{HubnerGSV21}, \href{../works/MengZRZL20.pdf}{MengZRZL20}~\cite{MengZRZL20}, \href{../works/Polo-MejiaALB20.pdf}{Polo-MejiaALB20}~\cite{Polo-MejiaALB20}, \href{../works/SacramentoSP20.pdf}{SacramentoSP20}~\cite{SacramentoSP20}...\href{../works/NovaraNH16.pdf}{NovaraNH16}~\cite{NovaraNH16}, \href{../works/Fahimi16.pdf}{Fahimi16}~\cite{Fahimi16}, \href{../works/PraletLJ15.pdf}{PraletLJ15}~\cite{PraletLJ15}, \href{../works/BartoliniBBLM14.pdf}{BartoliniBBLM14}~\cite{BartoliniBBLM14}, \href{../works/SchuttFS13.pdf}{SchuttFS13}~\cite{SchuttFS13}, \href{../works/HeinzB12.pdf}{HeinzB12}~\cite{HeinzB12}, \href{../works/Wolf11.pdf}{Wolf11}~\cite{Wolf11}, \href{../works/Laborie09.pdf}{Laborie09}~\cite{Laborie09}, \href{../works/WolfS05a.pdf}{WolfS05a}~\cite{WolfS05a}, \href{../works/Baptiste02.pdf}{Baptiste02}~\cite{Baptiste02} (Total: 48)\\
\index{alwaysEqual constraint}\index{Constraints!alwaysEqual constraint}alwaysEqual constraint &  1.00 &  & \href{../works/LaborieRSV18.pdf}{LaborieRSV18}~\cite{LaborieRSV18}, \href{../works/GoelSHFS15.pdf}{GoelSHFS15}~\cite{GoelSHFS15} & \href{../works/HamFC17.pdf}{HamFC17}~\cite{HamFC17}, \href{../works/HamC16.pdf}{HamC16}~\cite{HamC16}\\
\index{alwaysIn}\index{Constraints!alwaysIn}alwaysIn &  1.00 & \href{../works/PopovicCGNC22.pdf}{PopovicCGNC22}~\cite{PopovicCGNC22}, \href{../works/SerraNM12.pdf}{SerraNM12}~\cite{SerraNM12} & \href{../works/LuZZYW24.pdf}{LuZZYW24}~\cite{LuZZYW24}, \href{../works/AalianPG23.pdf}{AalianPG23}~\cite{AalianPG23}, \href{../works/LuoB22.pdf}{LuoB22}~\cite{LuoB22}, \href{../works/TangB20.pdf}{TangB20}~\cite{TangB20}, \href{../works/Polo-MejiaALB20.pdf}{Polo-MejiaALB20}~\cite{Polo-MejiaALB20}, \href{../works/MalapertN19.pdf}{MalapertN19}~\cite{MalapertN19}, \href{../works/LaborieRSV18.pdf}{LaborieRSV18}~\cite{LaborieRSV18}, \href{../works/GoelSHFS15.pdf}{GoelSHFS15}~\cite{GoelSHFS15} & \href{../works/AwadMDMT22.pdf}{AwadMDMT22}~\cite{AwadMDMT22}, \href{../works/CampeauG22.pdf}{CampeauG22}~\cite{CampeauG22}, \href{../works/KreterSS17.pdf}{KreterSS17}~\cite{KreterSS17}, \href{../works/QinDS16.pdf}{QinDS16}~\cite{QinDS16}, \href{../works/BajestaniB13.pdf}{BajestaniB13}~\cite{BajestaniB13}\\
\index{bin-packing}\index{Constraints!bin-packing}bin-packing &  1.00 & \href{../works/NaderiBZR23.pdf}{NaderiBZR23}~\cite{NaderiBZR23}, \href{../works/Zahout21.pdf}{Zahout21}~\cite{Zahout21}, \href{../works/Godet21a.pdf}{Godet21a}~\cite{Godet21a}, \href{../works/TangB20.pdf}{TangB20}~\cite{TangB20}, \href{../works/CauwelaertLS18.pdf}{CauwelaertLS18}~\cite{CauwelaertLS18}, \href{../works/RoshanaeiLAU17.pdf}{RoshanaeiLAU17}~\cite{RoshanaeiLAU17}, \href{../works/CauwelaertLS15.pdf}{CauwelaertLS15}~\cite{CauwelaertLS15}, \href{../works/LetortCB15.pdf}{LetortCB15}~\cite{LetortCB15}, \href{../works/Letort13.pdf}{Letort13}~\cite{Letort13}, \href{../works/LetortCB13.pdf}{LetortCB13}~\cite{LetortCB13}, \href{../works/MalapertGR12.pdf}{MalapertGR12}~\cite{MalapertGR12}, \href{../works/HeinzSSW12.pdf}{HeinzSSW12}~\cite{HeinzSSW12}, \href{../works/LetortBC12.pdf}{LetortBC12}~\cite{LetortBC12}, \href{../works/Malapert11.pdf}{Malapert11}~\cite{Malapert11}, \href{../works/SchausHMCMD11.pdf}{SchausHMCMD11}~\cite{SchausHMCMD11}, \href{../works/ClautiauxJCM08.pdf}{ClautiauxJCM08}~\cite{ClautiauxJCM08}, \href{../works/SchausD08.pdf}{SchausD08}~\cite{SchausD08} & \href{../works/FrimodigECM23.pdf}{FrimodigECM23}~\cite{FrimodigECM23}, \href{../works/JuvinHL23a.pdf}{JuvinHL23a}~\cite{JuvinHL23a}, \href{../works/LuoB22.pdf}{LuoB22}~\cite{LuoB22}, \href{../works/EmdeZD22.pdf}{EmdeZD22}~\cite{EmdeZD22}, \href{../works/BadicaBI20.pdf}{BadicaBI20}~\cite{BadicaBI20}, \href{../works/AntunesABD20.pdf}{AntunesABD20}~\cite{AntunesABD20}, \href{../works/FrimodigS19.pdf}{FrimodigS19}~\cite{FrimodigS19}, \href{../works/AntunesABD18.pdf}{AntunesABD18}~\cite{AntunesABD18}, \href{../works/BaptisteB18.pdf}{BaptisteB18}~\cite{BaptisteB18}, \href{../works/Beck10.pdf}{Beck10}~\cite{Beck10}, \href{../works/LiW08.pdf}{LiW08}~\cite{LiW08}, \href{../works/GarganiR07.pdf}{GarganiR07}~\cite{GarganiR07}, \href{../works/SchildW00.pdf}{SchildW00}~\cite{SchildW00}, \href{../works/SakkoutW00.pdf}{SakkoutW00}~\cite{SakkoutW00} & \href{../works/abs-2402-00459.pdf}{abs-2402-00459}~\cite{abs-2402-00459}, \href{../works/Fatemi-AnarakiTFV23.pdf}{Fatemi-AnarakiTFV23}~\cite{Fatemi-AnarakiTFV23}, \href{../works/LacknerMMWW23.pdf}{LacknerMMWW23}~\cite{LacknerMMWW23}, \href{../works/GuoZ23.pdf}{GuoZ23}~\cite{GuoZ23}, \href{../works/AkramNHRSA23.pdf}{AkramNHRSA23}~\cite{AkramNHRSA23}, \href{../works/YunusogluY22.pdf}{YunusogluY22}~\cite{YunusogluY22}, \href{../works/abs-2211-14492.pdf}{abs-2211-14492}~\cite{abs-2211-14492}, \href{../works/GhandehariK22.pdf}{GhandehariK22}~\cite{GhandehariK22}, \href{../works/ArmstrongGOS21.pdf}{ArmstrongGOS21}~\cite{ArmstrongGOS21}, \href{../works/RoshanaeiBAUB20.pdf}{RoshanaeiBAUB20}~\cite{RoshanaeiBAUB20}, \href{../works/GodetLHS20.pdf}{GodetLHS20}~\cite{GodetLHS20}, \href{../works/PinarbasiAY19.pdf}{PinarbasiAY19}~\cite{PinarbasiAY19}, \href{../works/AlakaPY19.pdf}{AlakaPY19}~\cite{AlakaPY19}, \href{../works/TranPZLDB18.pdf}{TranPZLDB18}~\cite{TranPZLDB18}, \href{../works/BukchinR18.pdf}{BukchinR18}~\cite{BukchinR18}, \href{../works/German18.pdf}{German18}~\cite{German18}, \href{../works/HookerH17.pdf}{HookerH17}~\cite{HookerH17}, \href{../works/HamFC17.pdf}{HamFC17}~\cite{HamFC17}, \href{../works/Madi-WambaLOBM17.pdf}{Madi-WambaLOBM17}~\cite{Madi-WambaLOBM17}...\href{../works/Lombardi10.pdf}{Lombardi10}~\cite{Lombardi10}, \href{../works/LombardiMRB10.pdf}{LombardiMRB10}~\cite{LombardiMRB10}, \href{../works/KovacsB08.pdf}{KovacsB08}~\cite{KovacsB08}, \href{../works/HentenryckM08.pdf}{HentenryckM08}~\cite{HentenryckM08}, \href{../works/Simonis07.pdf}{Simonis07}~\cite{Simonis07}, \href{../works/DavenportKRSH07.pdf}{DavenportKRSH07}~\cite{DavenportKRSH07}, \href{../works/SimonisCK00.pdf}{SimonisCK00}~\cite{SimonisCK00}, \href{../works/MurphyRFSS97.pdf}{MurphyRFSS97}~\cite{MurphyRFSS97}, \href{../works/BeldiceanuC94.pdf}{BeldiceanuC94}~\cite{BeldiceanuC94}, \href{../works/AggounB93.pdf}{AggounB93}~\cite{AggounB93} (Total: 40)\\
\index{circuit}\index{Constraints!circuit}circuit &  1.00 & \href{../works/MontemanniD23a.pdf}{MontemanniD23a}~\cite{MontemanniD23a}, \href{../works/KlankeBYE21.pdf}{KlankeBYE21}~\cite{KlankeBYE21}, \href{../works/MokhtarzadehTNF20.pdf}{MokhtarzadehTNF20}~\cite{MokhtarzadehTNF20}, \href{../works/Mercier-AubinGQ20.pdf}{Mercier-AubinGQ20}~\cite{Mercier-AubinGQ20}, \href{../works/Caballero19.pdf}{Caballero19}~\cite{Caballero19}, \href{../works/HookerH17.pdf}{HookerH17}~\cite{HookerH17}, \href{../works/Lombardi10.pdf}{Lombardi10}~\cite{Lombardi10}, \href{../works/RuggieroBBMA09.pdf}{RuggieroBBMA09}~\cite{RuggieroBBMA09}, \href{../works/RodriguezS09.pdf}{RodriguezS09}~\cite{RodriguezS09}, \href{../works/AchterbergBKW08.pdf}{AchterbergBKW08}~\cite{AchterbergBKW08}, \href{../works/Rodriguez07b.pdf}{Rodriguez07b}~\cite{Rodriguez07b}, \href{../works/Rodriguez07.pdf}{Rodriguez07}~\cite{Rodriguez07}, \href{../works/BeniniBGM05.pdf}{BeniniBGM05}~\cite{BeniniBGM05}, \href{../works/RodriguezDG02.pdf}{RodriguezDG02}~\cite{RodriguezDG02}, \href{../works/GruianK98.pdf}{GruianK98}~\cite{GruianK98}, \href{../works/Wallace96.pdf}{Wallace96}~\cite{Wallace96}, \href{../works/BeldiceanuC94.pdf}{BeldiceanuC94}~\cite{BeldiceanuC94} & \href{../works/WessenCSFPM23.pdf}{WessenCSFPM23}~\cite{WessenCSFPM23}, \href{../works/Groleaz21.pdf}{Groleaz21}~\cite{Groleaz21}, \href{../works/WessenCS20.pdf}{WessenCS20}~\cite{WessenCS20}, \href{../works/AntuoriHHEN20.pdf}{AntuoriHHEN20}~\cite{AntuoriHHEN20}, \href{../works/Siala15.pdf}{Siala15}~\cite{Siala15}, \href{../works/Siala15a.pdf}{Siala15a}~\cite{Siala15a}, \href{../works/LombardiMB13.pdf}{LombardiMB13}~\cite{LombardiMB13}, \href{../works/TranB12.pdf}{TranB12}~\cite{TranB12}, \href{../works/Malapert11.pdf}{Malapert11}~\cite{Malapert11}, \href{../works/ZeballosCM10.pdf}{ZeballosCM10}~\cite{ZeballosCM10}, \href{../works/KrogtLPHJ07.pdf}{KrogtLPHJ07}~\cite{KrogtLPHJ07}, \href{../works/KuchcinskiW03.pdf}{KuchcinskiW03}~\cite{KuchcinskiW03}, \href{../works/HookerO03.pdf}{HookerO03}~\cite{HookerO03}, \href{../works/Thorsteinsson01.pdf}{Thorsteinsson01}~\cite{Thorsteinsson01}, \href{../works/WatsonBHW99.pdf}{WatsonBHW99}~\cite{WatsonBHW99}, \href{../works/Simonis99.pdf}{Simonis99}~\cite{Simonis99}, \href{../works/Simonis95a.pdf}{Simonis95a}~\cite{Simonis95a}, \href{../works/DincbasSH90.pdf}{DincbasSH90}~\cite{DincbasSH90} & \href{../works/PrataAN23.pdf}{PrataAN23}~\cite{PrataAN23}, \href{../works/Fatemi-AnarakiTFV23.pdf}{Fatemi-AnarakiTFV23}~\cite{Fatemi-AnarakiTFV23}, \href{../works/GokPTGO23.pdf}{GokPTGO23}~\cite{GokPTGO23}, \href{../works/IsikYA23.pdf}{IsikYA23}~\cite{IsikYA23}, \href{../works/MontemanniD23.pdf}{MontemanniD23}~\cite{MontemanniD23}, \href{../works/MarliereSPR23.pdf}{MarliereSPR23}~\cite{MarliereSPR23}, \href{../works/JuvinHL23a.pdf}{JuvinHL23a}~\cite{JuvinHL23a}, \href{../works/ColT22.pdf}{ColT22}~\cite{ColT22}, \href{../works/MullerMKP22.pdf}{MullerMKP22}~\cite{MullerMKP22}, \href{../works/JungblutK22.pdf}{JungblutK22}~\cite{JungblutK22}, \href{../works/FarsiTM22.pdf}{FarsiTM22}~\cite{FarsiTM22}, \href{../works/JuvinHL22.pdf}{JuvinHL22}~\cite{JuvinHL22}, \href{../works/KoehlerBFFHPSSS21.pdf}{KoehlerBFFHPSSS21}~\cite{KoehlerBFFHPSSS21}, \href{../works/MengLZB21.pdf}{MengLZB21}~\cite{MengLZB21}, \href{../works/Astrand21.pdf}{Astrand21}~\cite{Astrand21}, \href{../works/Zahout21.pdf}{Zahout21}~\cite{Zahout21}, \href{../works/ArmstrongGOS21.pdf}{ArmstrongGOS21}~\cite{ArmstrongGOS21}, \href{../works/WallaceY20.pdf}{WallaceY20}~\cite{WallaceY20}, \href{../works/GokGSTO20.pdf}{GokGSTO20}~\cite{GokGSTO20}...\href{../works/Beck99.pdf}{Beck99}~\cite{Beck99}, \href{../works/KorbaaYG99.pdf}{KorbaaYG99}~\cite{KorbaaYG99}, \href{../works/BeckF98.pdf}{BeckF98}~\cite{BeckF98}, \href{../works/RodosekW98.pdf}{RodosekW98}~\cite{RodosekW98}, \href{../works/LammaMM97.pdf}{LammaMM97}~\cite{LammaMM97}, \href{../works/SadehF96.pdf}{SadehF96}~\cite{SadehF96}, \href{../works/Simonis95.pdf}{Simonis95}~\cite{Simonis95}, \href{../works/Nuijten94.pdf}{Nuijten94}~\cite{Nuijten94}, \href{../works/AggounB93.pdf}{AggounB93}~\cite{AggounB93}, \href{../works/Valdes87.pdf}{Valdes87}~\cite{Valdes87} (Total: 90)\\
\index{cumulative}\index{Constraints!cumulative}cumulative &  1.00 & \href{../works/AalianPG23.pdf}{AalianPG23}~\cite{AalianPG23}, \href{../works/TardivoDFMP23.pdf}{TardivoDFMP23}~\cite{TardivoDFMP23}, \href{../works/NaderiRR23.pdf}{NaderiRR23}~\cite{NaderiRR23}, \href{../works/LacknerMMWW23.pdf}{LacknerMMWW23}~\cite{LacknerMMWW23}, \href{../works/WessenCSFPM23.pdf}{WessenCSFPM23}~\cite{WessenCSFPM23}, \href{../works/PovedaAA23.pdf}{PovedaAA23}~\cite{PovedaAA23}, \href{../works/KameugneFND23.pdf}{KameugneFND23}~\cite{KameugneFND23}, \href{../works/IsikYA23.pdf}{IsikYA23}~\cite{IsikYA23}, \href{../works/PohlAK22.pdf}{PohlAK22}~\cite{PohlAK22}, \href{../works/AwadMDMT22.pdf}{AwadMDMT22}~\cite{AwadMDMT22}, \href{../works/ZhangJZL22.pdf}{ZhangJZL22}~\cite{ZhangJZL22}, \href{../works/LuoB22.pdf}{LuoB22}~\cite{LuoB22}, \href{../works/FetgoD22.pdf}{FetgoD22}~\cite{FetgoD22}, \href{../works/OuelletQ22.pdf}{OuelletQ22}~\cite{OuelletQ22}, \href{../works/BoudreaultSLQ22.pdf}{BoudreaultSLQ22}~\cite{BoudreaultSLQ22}, \href{../works/Lemos21.pdf}{Lemos21}~\cite{Lemos21}, \href{../works/Godet21a.pdf}{Godet21a}~\cite{Godet21a}, \href{../works/Groleaz21.pdf}{Groleaz21}~\cite{Groleaz21}, \href{../works/LacknerMMWW21.pdf}{LacknerMMWW21}~\cite{LacknerMMWW21}...\href{../works/SimonisCK00.pdf}{SimonisCK00}~\cite{SimonisCK00}, \href{../works/Beck99.pdf}{Beck99}~\cite{Beck99}, \href{../works/Simonis99.pdf}{Simonis99}~\cite{Simonis99}, \href{../works/BaptistePN99.pdf}{BaptistePN99}~\cite{BaptistePN99}, \href{../works/PapaB98.pdf}{PapaB98}~\cite{PapaB98}, \href{../works/BaptisteP97.pdf}{BaptisteP97}~\cite{BaptisteP97}, \href{../works/PapeB97.pdf}{PapeB97}~\cite{PapeB97}, \href{../works/Goltz95.pdf}{Goltz95}~\cite{Goltz95}, \href{../works/SimonisC95.pdf}{SimonisC95}~\cite{SimonisC95}, \href{../works/BeldiceanuC94.pdf}{BeldiceanuC94}~\cite{BeldiceanuC94} (Total: 181) & \href{../works/ForbesHJST24.pdf}{ForbesHJST24}~\cite{ForbesHJST24}, \href{../works/PrataAN23.pdf}{PrataAN23}~\cite{PrataAN23}, \href{../works/LuZZYW24.pdf}{LuZZYW24}~\cite{LuZZYW24}, \href{../works/BonninMNE24.pdf}{BonninMNE24}~\cite{BonninMNE24}, \href{../works/abs-2402-00459.pdf}{abs-2402-00459}~\cite{abs-2402-00459}, \href{../works/PerezGSL23.pdf}{PerezGSL23}~\cite{PerezGSL23}, \href{../works/EfthymiouY23.pdf}{EfthymiouY23}~\cite{EfthymiouY23}, \href{../works/abs-2312-13682.pdf}{abs-2312-13682}~\cite{abs-2312-13682}, \href{../works/GokPTGO23.pdf}{GokPTGO23}~\cite{GokPTGO23}, \href{../works/ElciOH22.pdf}{ElciOH22}~\cite{ElciOH22}, \href{../works/YunusogluY22.pdf}{YunusogluY22}~\cite{YunusogluY22}, \href{../works/GeitzGSSW22.pdf}{GeitzGSSW22}~\cite{GeitzGSSW22}, \href{../works/AbreuN22.pdf}{AbreuN22}~\cite{AbreuN22}, \href{../works/NaqviAIAAA22.pdf}{NaqviAIAAA22}~\cite{NaqviAIAAA22}, \href{../works/ColT22.pdf}{ColT22}~\cite{ColT22}, \href{../works/CampeauG22.pdf}{CampeauG22}~\cite{CampeauG22}, \href{../works/HubnerGSV21.pdf}{HubnerGSV21}~\cite{HubnerGSV21}, \href{../works/KlankeBYE21.pdf}{KlankeBYE21}~\cite{KlankeBYE21}, \href{../works/HillTV21.pdf}{HillTV21}~\cite{HillTV21}...\href{../works/ChuX05.pdf}{ChuX05}~\cite{ChuX05}, \href{../works/VilimBC04.pdf}{VilimBC04}~\cite{VilimBC04}, \href{../works/KovacsV04.pdf}{KovacsV04}~\cite{KovacsV04}, \href{../works/Laborie03.pdf}{Laborie03}~\cite{Laborie03}, \href{../works/Elkhyari03.pdf}{Elkhyari03}~\cite{Elkhyari03}, \href{../works/Bartak02a.pdf}{Bartak02a}~\cite{Bartak02a}, \href{../works/Thorsteinsson01.pdf}{Thorsteinsson01}~\cite{Thorsteinsson01}, \href{../works/TorresL00.pdf}{TorresL00}~\cite{TorresL00}, \href{../works/CestaOF99.pdf}{CestaOF99}~\cite{CestaOF99}, \href{../works/Caseau97.pdf}{Caseau97}~\cite{Caseau97} (Total: 70) & \href{../works/GurPAE23.pdf}{GurPAE23}~\cite{GurPAE23}, \href{../works/abs-2306-05747.pdf}{abs-2306-05747}~\cite{abs-2306-05747}, \href{../works/AbreuPNF23.pdf}{AbreuPNF23}~\cite{AbreuPNF23}, \href{../works/YuraszeckMCCR23.pdf}{YuraszeckMCCR23}~\cite{YuraszeckMCCR23}, \href{../works/MarliereSPR23.pdf}{MarliereSPR23}~\cite{MarliereSPR23}, \href{../works/TasselGS23.pdf}{TasselGS23}~\cite{TasselGS23}, \href{../works/FrimodigECM23.pdf}{FrimodigECM23}~\cite{FrimodigECM23}, \href{../works/IklassovMR023.pdf}{IklassovMR023}~\cite{IklassovMR023}, \href{../works/JuvinHL23a.pdf}{JuvinHL23a}~\cite{JuvinHL23a}, \href{../works/abs-2305-19888.pdf}{abs-2305-19888}~\cite{abs-2305-19888}, \href{../works/Bit-Monnot23.pdf}{Bit-Monnot23}~\cite{Bit-Monnot23}, \href{../works/JuvinHHL23.pdf}{JuvinHHL23}~\cite{JuvinHHL23}, \href{../works/HeinzNVH22.pdf}{HeinzNVH22}~\cite{HeinzNVH22}, \href{../works/PopovicCGNC22.pdf}{PopovicCGNC22}~\cite{PopovicCGNC22}, \href{../works/HebrardALLCMR22.pdf}{HebrardALLCMR22}~\cite{HebrardALLCMR22}, \href{../works/Tassel22.pdf}{Tassel22}~\cite{Tassel22}, \href{../works/abs-2211-14492.pdf}{abs-2211-14492}~\cite{abs-2211-14492}, \href{../works/SubulanC22.pdf}{SubulanC22}~\cite{SubulanC22}, \href{../works/JuvinHL22.pdf}{JuvinHL22}~\cite{JuvinHL22}...\href{../works/BeckDDF98.pdf}{BeckDDF98}~\cite{BeckDDF98}, \href{../works/BeckF98.pdf}{BeckF98}~\cite{BeckF98}, \href{../works/BeckDF97.pdf}{BeckDF97}~\cite{BeckDF97}, \href{../works/Zhou97.pdf}{Zhou97}~\cite{Zhou97}, \href{../works/BlazewiczDP96.pdf}{BlazewiczDP96}~\cite{BlazewiczDP96}, \href{../works/Simonis95a.pdf}{Simonis95a}~\cite{Simonis95a}, \href{../works/Simonis95.pdf}{Simonis95}~\cite{Simonis95}, \href{../works/Pape94.pdf}{Pape94}~\cite{Pape94}, \href{../works/Nuijten94.pdf}{Nuijten94}~\cite{Nuijten94}, \href{../works/MintonJPL92.pdf}{MintonJPL92}~\cite{MintonJPL92} (Total: 154)\\
\index{cycle}\index{Constraints!cycle}cycle &  1.00 & \href{../works/AalianPG23.pdf}{AalianPG23}~\cite{AalianPG23}, \href{../works/WessenCSFPM23.pdf}{WessenCSFPM23}~\cite{WessenCSFPM23}, \href{../works/AlakaP23.pdf}{AlakaP23}~\cite{AlakaP23}, \href{../works/AwadMDMT22.pdf}{AwadMDMT22}~\cite{AwadMDMT22}, \href{../works/CilKLO22.pdf}{CilKLO22}~\cite{CilKLO22}, \href{../works/Astrand21.pdf}{Astrand21}~\cite{Astrand21}, \href{../works/AbohashimaEG21.pdf}{AbohashimaEG21}~\cite{AbohashimaEG21}, \href{../works/AntuoriHHEN21.pdf}{AntuoriHHEN21}~\cite{AntuoriHHEN21}, \href{../works/Groleaz21.pdf}{Groleaz21}~\cite{Groleaz21}, \href{../works/Astrand0F21.pdf}{Astrand0F21}~\cite{Astrand0F21}, \href{../works/Edis21.pdf}{Edis21}~\cite{Edis21}, \href{../works/Alaka21.pdf}{Alaka21}~\cite{Alaka21}, \href{../works/AbidinK20.pdf}{AbidinK20}~\cite{AbidinK20}, \href{../works/GroleazNS20a.pdf}{GroleazNS20a}~\cite{GroleazNS20a}, \href{../works/AntuoriHHEN20.pdf}{AntuoriHHEN20}~\cite{AntuoriHHEN20}, \href{../works/AstrandJZ20.pdf}{AstrandJZ20}~\cite{AstrandJZ20}, \href{../works/WallaceY20.pdf}{WallaceY20}~\cite{WallaceY20}, \href{../works/PinarbasiAY19.pdf}{PinarbasiAY19}~\cite{PinarbasiAY19}, \href{../works/Caballero19.pdf}{Caballero19}~\cite{Caballero19}...\href{../works/Refalo00.pdf}{Refalo00}~\cite{Refalo00}, \href{../works/KorbaaYG99.pdf}{KorbaaYG99}~\cite{KorbaaYG99}, \href{../works/DraperJCJ99.pdf}{DraperJCJ99}~\cite{DraperJCJ99}, \href{../works/CarlssonKA99.pdf}{CarlssonKA99}~\cite{CarlssonKA99}, \href{../works/GruianK98.pdf}{GruianK98}~\cite{GruianK98}, \href{../works/RodosekW98.pdf}{RodosekW98}~\cite{RodosekW98}, \href{../works/RoweJCA96.pdf}{RoweJCA96}~\cite{RoweJCA96}, \href{../works/BlazewiczDP96.pdf}{BlazewiczDP96}~\cite{BlazewiczDP96}, \href{../works/BeldiceanuC94.pdf}{BeldiceanuC94}~\cite{BeldiceanuC94}, \href{../works/Muscettola94.pdf}{Muscettola94}~\cite{Muscettola94} (Total: 65) & \href{../works/EfthymiouY23.pdf}{EfthymiouY23}~\cite{EfthymiouY23}, \href{../works/CampeauG22.pdf}{CampeauG22}~\cite{CampeauG22}, \href{../works/Godet21a.pdf}{Godet21a}~\cite{Godet21a}, \href{../works/HillTV21.pdf}{HillTV21}~\cite{HillTV21}, \href{../works/Lemos21.pdf}{Lemos21}~\cite{Lemos21}, \href{../works/KoehlerBFFHPSSS21.pdf}{KoehlerBFFHPSSS21}~\cite{KoehlerBFFHPSSS21}, \href{../works/HubnerGSV21.pdf}{HubnerGSV21}~\cite{HubnerGSV21}, \href{../works/CauwelaertDS20.pdf}{CauwelaertDS20}~\cite{CauwelaertDS20}, \href{../works/Lunardi20.pdf}{Lunardi20}~\cite{Lunardi20}, \href{../works/ZarandiASC20.pdf}{ZarandiASC20}~\cite{ZarandiASC20}, \href{../works/GroleazNS20.pdf}{GroleazNS20}~\cite{GroleazNS20}, \href{../works/ArkhipovBL19.pdf}{ArkhipovBL19}~\cite{ArkhipovBL19}, \href{../works/MossigeGSMC17.pdf}{MossigeGSMC17}~\cite{MossigeGSMC17}, \href{../works/HamFC17.pdf}{HamFC17}~\cite{HamFC17}, \href{../works/TranAB16.pdf}{TranAB16}~\cite{TranAB16}, \href{../works/Froger16.pdf}{Froger16}~\cite{Froger16}, \href{../works/SimoninAHL15.pdf}{SimoninAHL15}~\cite{SimoninAHL15}, \href{../works/BurtLPS15.pdf}{BurtLPS15}~\cite{BurtLPS15}, \href{../works/Siala15.pdf}{Siala15}~\cite{Siala15}...\href{../works/Laborie03.pdf}{Laborie03}~\cite{Laborie03}, \href{../works/OddiPCC03.pdf}{OddiPCC03}~\cite{OddiPCC03}, \href{../works/Demassey03.pdf}{Demassey03}~\cite{Demassey03}, \href{../works/SimonisCK00.pdf}{SimonisCK00}~\cite{SimonisCK00}, \href{../works/ChunCTY99.pdf}{ChunCTY99}~\cite{ChunCTY99}, \href{../works/Simonis99.pdf}{Simonis99}~\cite{Simonis99}, \href{../works/Wallace96.pdf}{Wallace96}~\cite{Wallace96}, \href{../works/Simonis95a.pdf}{Simonis95a}~\cite{Simonis95a}, \href{../works/MintonJPL92.pdf}{MintonJPL92}~\cite{MintonJPL92}, \href{../works/MintonJPL90.pdf}{MintonJPL90}~\cite{MintonJPL90} (Total: 56) & \href{../works/Bit-Monnot23.pdf}{Bit-Monnot23}~\cite{Bit-Monnot23}, \href{../works/AkramNHRSA23.pdf}{AkramNHRSA23}~\cite{AkramNHRSA23}, \href{../works/Fatemi-AnarakiTFV23.pdf}{Fatemi-AnarakiTFV23}~\cite{Fatemi-AnarakiTFV23}, \href{../works/MarliereSPR23.pdf}{MarliereSPR23}~\cite{MarliereSPR23}, \href{../works/IklassovMR023.pdf}{IklassovMR023}~\cite{IklassovMR023}, \href{../works/GokPTGO23.pdf}{GokPTGO23}~\cite{GokPTGO23}, \href{../works/GuoZ23.pdf}{GuoZ23}~\cite{GuoZ23}, \href{../works/ZhangBB22.pdf}{ZhangBB22}~\cite{ZhangBB22}, \href{../works/BourreauGGLT22.pdf}{BourreauGGLT22}~\cite{BourreauGGLT22}, \href{../works/AbreuN22.pdf}{AbreuN22}~\cite{AbreuN22}, \href{../works/KotaryFH22.pdf}{KotaryFH22}~\cite{KotaryFH22}, \href{../works/AbreuAPNM21.pdf}{AbreuAPNM21}~\cite{AbreuAPNM21}, \href{../works/ArmstrongGOS21.pdf}{ArmstrongGOS21}~\cite{ArmstrongGOS21}, \href{../works/Zahout21.pdf}{Zahout21}~\cite{Zahout21}, \href{../works/FanXG21.pdf}{FanXG21}~\cite{FanXG21}, \href{../works/HamPK21.pdf}{HamPK21}~\cite{HamPK21}, \href{../works/Ham20a.pdf}{Ham20a}~\cite{Ham20a}, \href{../works/HauderBRPA20.pdf}{HauderBRPA20}~\cite{HauderBRPA20}, \href{../works/FachiniA20.pdf}{FachiniA20}~\cite{FachiniA20}...\href{../works/JainM99.pdf}{JainM99}~\cite{JainM99}, \href{../works/PembertonG98.pdf}{PembertonG98}~\cite{PembertonG98}, \href{../works/BeckDF97.pdf}{BeckDF97}~\cite{BeckDF97}, \href{../works/OddiS97.pdf}{OddiS97}~\cite{OddiS97}, \href{../works/LeeKLKKYHP97.pdf}{LeeKLKKYHP97}~\cite{LeeKLKKYHP97}, \href{../works/MurphyRFSS97.pdf}{MurphyRFSS97}~\cite{MurphyRFSS97}, \href{../works/SadehF96.pdf}{SadehF96}~\cite{SadehF96}, \href{../works/WeilHFP95.pdf}{WeilHFP95}~\cite{WeilHFP95}, \href{../works/Simonis95.pdf}{Simonis95}~\cite{Simonis95}, \href{../works/Valdes87.pdf}{Valdes87}~\cite{Valdes87} (Total: 119)\\
\index{diffn}\index{Constraints!diffn}diffn &  1.00 & \href{../works/ArmstrongGOS21.pdf}{ArmstrongGOS21}~\cite{ArmstrongGOS21}, \href{../works/Simonis07.pdf}{Simonis07}~\cite{Simonis07}, \href{../works/SimonisCK00.pdf}{SimonisCK00}~\cite{SimonisCK00}, \href{../works/BeldiceanuC94.pdf}{BeldiceanuC94}~\cite{BeldiceanuC94} & \href{../works/WessenCSFPM23.pdf}{WessenCSFPM23}~\cite{WessenCSFPM23}, \href{../works/BeldiceanuCDP11.pdf}{BeldiceanuCDP11}~\cite{BeldiceanuCDP11} & \href{../works/LuoB22.pdf}{LuoB22}~\cite{LuoB22}, \href{../works/BourreauGGLT22.pdf}{BourreauGGLT22}~\cite{BourreauGGLT22}, \href{../works/KreterSS17.pdf}{KreterSS17}~\cite{KreterSS17}, \href{../works/KreterSS15.pdf}{KreterSS15}~\cite{KreterSS15}, \href{../works/Malapert11.pdf}{Malapert11}~\cite{Malapert11}, \href{../works/TrojetHL11.pdf}{TrojetHL11}~\cite{TrojetHL11}, \href{../works/ChenGPSH10.pdf}{ChenGPSH10}~\cite{ChenGPSH10}, \href{../works/Kuchcinski03.pdf}{Kuchcinski03}~\cite{Kuchcinski03}, \href{../works/Timpe02.pdf}{Timpe02}~\cite{Timpe02}, \href{../works/Simonis99.pdf}{Simonis99}~\cite{Simonis99}, \href{../works/GruianK98.pdf}{GruianK98}~\cite{GruianK98}, \href{../works/Simonis95.pdf}{Simonis95}~\cite{Simonis95}, \href{../works/SimonisC95.pdf}{SimonisC95}~\cite{SimonisC95}, \href{../works/Simonis95a.pdf}{Simonis95a}~\cite{Simonis95a}\\
\index{disjunctive}\index{Constraints!disjunctive}disjunctive &  1.00 & \href{../works/BonninMNE24.pdf}{BonninMNE24}~\cite{BonninMNE24}, \href{../works/JuvinHHL23.pdf}{JuvinHHL23}~\cite{JuvinHHL23}, \href{../works/AfsarVPG23.pdf}{AfsarVPG23}~\cite{AfsarVPG23}, \href{../works/NaderiRR23.pdf}{NaderiRR23}~\cite{NaderiRR23}, \href{../works/Bit-Monnot23.pdf}{Bit-Monnot23}~\cite{Bit-Monnot23}, \href{../works/BourreauGGLT22.pdf}{BourreauGGLT22}~\cite{BourreauGGLT22}, \href{../works/ZhangBB22.pdf}{ZhangBB22}~\cite{ZhangBB22}, \href{../works/YuraszeckMPV22.pdf}{YuraszeckMPV22}~\cite{YuraszeckMPV22}, \href{../works/JuvinHL22.pdf}{JuvinHL22}~\cite{JuvinHL22}, \href{../works/Groleaz21.pdf}{Groleaz21}~\cite{Groleaz21}, \href{../works/Godet21a.pdf}{Godet21a}~\cite{Godet21a}, \href{../works/KoehlerBFFHPSSS21.pdf}{KoehlerBFFHPSSS21}~\cite{KoehlerBFFHPSSS21}, \href{../works/Astrand21.pdf}{Astrand21}~\cite{Astrand21}, \href{../works/GodetLHS20.pdf}{GodetLHS20}~\cite{GodetLHS20}, \href{../works/GokgurHO18.pdf}{GokgurHO18}~\cite{GokgurHO18}, \href{../works/LaborieRSV18.pdf}{LaborieRSV18}~\cite{LaborieRSV18}, \href{../works/German18.pdf}{German18}~\cite{German18}, \href{../works/FahimiOQ18.pdf}{FahimiOQ18}~\cite{FahimiOQ18}, \href{../works/NattafAL17.pdf}{NattafAL17}~\cite{NattafAL17}...\href{../works/Zhou97.pdf}{Zhou97}~\cite{Zhou97}, \href{../works/BaptisteP97.pdf}{BaptisteP97}~\cite{BaptisteP97}, \href{../works/BlazewiczDP96.pdf}{BlazewiczDP96}~\cite{BlazewiczDP96}, \href{../works/Zhou96.pdf}{Zhou96}~\cite{Zhou96}, \href{../works/BaptisteP95.pdf}{BaptisteP95}~\cite{BaptisteP95}, \href{../works/Nuijten94.pdf}{Nuijten94}~\cite{Nuijten94}, \href{../works/Pape94.pdf}{Pape94}~\cite{Pape94}, \href{../works/AggounB93.pdf}{AggounB93}~\cite{AggounB93}, \href{../works/DincbasSH90.pdf}{DincbasSH90}~\cite{DincbasSH90}, \href{../works/Valdes87.pdf}{Valdes87}~\cite{Valdes87} (Total: 101) & \href{../works/MarliereSPR23.pdf}{MarliereSPR23}~\cite{MarliereSPR23}, \href{../works/Adelgren2023.pdf}{Adelgren2023}~\cite{Adelgren2023}, \href{../works/JuvinHL23a.pdf}{JuvinHL23a}~\cite{JuvinHL23a}, \href{../works/BoudreaultSLQ22.pdf}{BoudreaultSLQ22}~\cite{BoudreaultSLQ22}, \href{../works/OrnekOS20.pdf}{OrnekOS20}~\cite{OrnekOS20}, \href{../works/KotaryFH22.pdf}{KotaryFH22}~\cite{KotaryFH22}, \href{../works/Astrand0F21.pdf}{Astrand0F21}~\cite{Astrand0F21}, \href{../works/GeibingerMM21.pdf}{GeibingerMM21}~\cite{GeibingerMM21}, \href{../works/AstrandJZ20.pdf}{AstrandJZ20}~\cite{AstrandJZ20}, \href{../works/MejiaY20.pdf}{MejiaY20}~\cite{MejiaY20}, \href{../works/Polo-MejiaALB20.pdf}{Polo-MejiaALB20}~\cite{Polo-MejiaALB20}, \href{../works/SacramentoSP20.pdf}{SacramentoSP20}~\cite{SacramentoSP20}, \href{../works/RoshanaeiBAUB20.pdf}{RoshanaeiBAUB20}~\cite{RoshanaeiBAUB20}, \href{../works/PachecoPR19.pdf}{PachecoPR19}~\cite{PachecoPR19}, \href{../works/UnsalO19.pdf}{UnsalO19}~\cite{UnsalO19}, \href{../works/YangSS19.pdf}{YangSS19}~\cite{YangSS19}, \href{../works/DemirovicS18.pdf}{DemirovicS18}~\cite{DemirovicS18}, \href{../works/TanT18.pdf}{TanT18}~\cite{TanT18}, \href{../works/CauwelaertLS18.pdf}{CauwelaertLS18}~\cite{CauwelaertLS18}...\href{../works/Demassey03.pdf}{Demassey03}~\cite{Demassey03}, \href{../works/Elkhyari03.pdf}{Elkhyari03}~\cite{Elkhyari03}, \href{../works/Kuchcinski03.pdf}{Kuchcinski03}~\cite{Kuchcinski03}, \href{../works/SchildW00.pdf}{SchildW00}~\cite{SchildW00}, \href{../works/Refalo00.pdf}{Refalo00}~\cite{Refalo00}, \href{../works/HookerOTK00.pdf}{HookerOTK00}~\cite{HookerOTK00}, \href{../works/ArtiguesR00.pdf}{ArtiguesR00}~\cite{ArtiguesR00}, \href{../works/Beck99.pdf}{Beck99}~\cite{Beck99}, \href{../works/NuijtenP98.pdf}{NuijtenP98}~\cite{NuijtenP98}, \href{../works/Darby-DowmanLMZ97.pdf}{Darby-DowmanLMZ97}~\cite{Darby-DowmanLMZ97} (Total: 57) & \href{../works/LuZZYW24.pdf}{LuZZYW24}~\cite{LuZZYW24}, \href{../works/abs-2402-00459.pdf}{abs-2402-00459}~\cite{abs-2402-00459}, \href{../works/LacknerMMWW23.pdf}{LacknerMMWW23}~\cite{LacknerMMWW23}, \href{../works/abs-2306-05747.pdf}{abs-2306-05747}~\cite{abs-2306-05747}, \href{../works/IklassovMR023.pdf}{IklassovMR023}~\cite{IklassovMR023}, \href{../works/NaderiBZ23.pdf}{NaderiBZ23}~\cite{NaderiBZ23}, \href{../works/TardivoDFMP23.pdf}{TardivoDFMP23}~\cite{TardivoDFMP23}, \href{../works/ZhuSZW23.pdf}{ZhuSZW23}~\cite{ZhuSZW23}, \href{../works/GokPTGO23.pdf}{GokPTGO23}~\cite{GokPTGO23}, \href{../works/AbreuPNF23.pdf}{AbreuPNF23}~\cite{AbreuPNF23}, \href{../works/KameugneFND23.pdf}{KameugneFND23}~\cite{KameugneFND23}, \href{../works/EfthymiouY23.pdf}{EfthymiouY23}~\cite{EfthymiouY23}, \href{../works/TasselGS23.pdf}{TasselGS23}~\cite{TasselGS23}, \href{../works/Fatemi-AnarakiTFV23.pdf}{Fatemi-AnarakiTFV23}~\cite{Fatemi-AnarakiTFV23}, \href{../works/PovedaAA23.pdf}{PovedaAA23}~\cite{PovedaAA23}, \href{../works/GhandehariK22.pdf}{GhandehariK22}~\cite{GhandehariK22}, \href{../works/ElciOH22.pdf}{ElciOH22}~\cite{ElciOH22}, \href{../works/AwadMDMT22.pdf}{AwadMDMT22}~\cite{AwadMDMT22}, \href{../works/NaderiBZ22a.pdf}{NaderiBZ22a}~\cite{NaderiBZ22a}...\href{../works/LammaMM97.pdf}{LammaMM97}~\cite{LammaMM97}, \href{../works/BeckDSF97.pdf}{BeckDSF97}~\cite{BeckDSF97}, \href{../works/BrusoniCLMMT96.pdf}{BrusoniCLMMT96}~\cite{BrusoniCLMMT96}, \href{../works/NuijtenA96.pdf}{NuijtenA96}~\cite{NuijtenA96}, \href{../works/Wallace96.pdf}{Wallace96}~\cite{Wallace96}, \href{../works/Colombani96.pdf}{Colombani96}~\cite{Colombani96}, \href{../works/SadehF96.pdf}{SadehF96}~\cite{SadehF96}, \href{../works/Goltz95.pdf}{Goltz95}~\cite{Goltz95}, \href{../works/Puget95.pdf}{Puget95}~\cite{Puget95}, \href{../works/NuijtenA94.pdf}{NuijtenA94}~\cite{NuijtenA94} (Total: 170)\\
\index{endBeforeStart}\index{Constraints!endBeforeStart}endBeforeStart &  1.00 & \href{../works/SubulanC22.pdf}{SubulanC22}~\cite{SubulanC22}, \href{../works/QinDCS20.pdf}{QinDCS20}~\cite{QinDCS20} & \href{../works/IsikYA23.pdf}{IsikYA23}~\cite{IsikYA23}, \href{../works/ZhuSZW23.pdf}{ZhuSZW23}~\cite{ZhuSZW23}, \href{../works/NaderiRR23.pdf}{NaderiRR23}~\cite{NaderiRR23}, \href{../works/NaderiBZ22a.pdf}{NaderiBZ22a}~\cite{NaderiBZ22a}, \href{../works/PandeyS21a.pdf}{PandeyS21a}~\cite{PandeyS21a}, \href{../works/MengLZB21.pdf}{MengLZB21}~\cite{MengLZB21}, \href{../works/LunardiBLRV20.pdf}{LunardiBLRV20}~\cite{LunardiBLRV20}, \href{../works/Lunardi20.pdf}{Lunardi20}~\cite{Lunardi20}, \href{../works/MengZRZL20.pdf}{MengZRZL20}~\cite{MengZRZL20}, \href{../works/LaborieRSV18.pdf}{LaborieRSV18}~\cite{LaborieRSV18}, \href{../works/NovaraNH16.pdf}{NovaraNH16}~\cite{NovaraNH16}, \href{../works/Laborie09.pdf}{Laborie09}~\cite{Laborie09} & \href{../works/JuvinHL23a.pdf}{JuvinHL23a}~\cite{JuvinHL23a}, \href{../works/JuvinHHL23.pdf}{JuvinHHL23}~\cite{JuvinHHL23}, \href{../works/YuraszeckMCCR23.pdf}{YuraszeckMCCR23}~\cite{YuraszeckMCCR23}, \href{../works/JuvinHL23.pdf}{JuvinHL23}~\cite{JuvinHL23}, \href{../works/LacknerMMWW23.pdf}{LacknerMMWW23}~\cite{LacknerMMWW23}, \href{../works/AalianPG23.pdf}{AalianPG23}~\cite{AalianPG23}, \href{../works/CzerniachowskaWZ23.pdf}{CzerniachowskaWZ23}~\cite{CzerniachowskaWZ23}, \href{../works/Teppan22.pdf}{Teppan22}~\cite{Teppan22}, \href{../works/AwadMDMT22.pdf}{AwadMDMT22}~\cite{AwadMDMT22}, \href{../works/JuvinHL22.pdf}{JuvinHL22}~\cite{JuvinHL22}, \href{../works/CampeauG22.pdf}{CampeauG22}~\cite{CampeauG22}, \href{../works/ZhangJZL22.pdf}{ZhangJZL22}~\cite{ZhangJZL22}, \href{../works/CilKLO22.pdf}{CilKLO22}~\cite{CilKLO22}, \href{../works/YunusogluY22.pdf}{YunusogluY22}~\cite{YunusogluY22}, \href{../works/HamP21.pdf}{HamP21}~\cite{HamP21}, \href{../works/LacknerMMWW21.pdf}{LacknerMMWW21}~\cite{LacknerMMWW21}, \href{../works/HamPK21.pdf}{HamPK21}~\cite{HamPK21}, \href{../works/HubnerGSV21.pdf}{HubnerGSV21}~\cite{HubnerGSV21}, \href{../works/ZhangYW21.pdf}{ZhangYW21}~\cite{ZhangYW21}...\href{../works/ParkUJR19.pdf}{ParkUJR19}~\cite{ParkUJR19}, \href{../works/abs-1911-04766.pdf}{abs-1911-04766}~\cite{abs-1911-04766}, \href{../works/GeibingerMM19.pdf}{GeibingerMM19}~\cite{GeibingerMM19}, \href{../works/Novas19.pdf}{Novas19}~\cite{Novas19}, \href{../works/abs-1902-09244.pdf}{abs-1902-09244}~\cite{abs-1902-09244}, \href{../works/NishikawaSTT18a.pdf}{NishikawaSTT18a}~\cite{NishikawaSTT18a}, \href{../works/NishikawaSTT18.pdf}{NishikawaSTT18}~\cite{NishikawaSTT18}, \href{../works/Ham18.pdf}{Ham18}~\cite{Ham18}, \href{../works/HamC16.pdf}{HamC16}~\cite{HamC16}, \href{../works/GrimesH15.pdf}{GrimesH15}~\cite{GrimesH15} (Total: 40)\\
\index{geost}\index{Constraints!geost}geost &  1.00 & \href{../works/BeldiceanuCDP11.pdf}{BeldiceanuCDP11}~\cite{BeldiceanuCDP11} & \href{../works/LetortBC12.pdf}{LetortBC12}~\cite{LetortBC12}, \href{../works/PembertonG98.pdf}{PembertonG98}~\cite{PembertonG98} & \href{../works/FrankDT16.pdf}{FrankDT16}~\cite{FrankDT16}, \href{../works/Letort13.pdf}{Letort13}~\cite{Letort13}, \href{../works/Schutt11.pdf}{Schutt11}~\cite{Schutt11}, \href{../works/Malapert11.pdf}{Malapert11}~\cite{Malapert11}, \href{../works/BeldiceanuCP08.pdf}{BeldiceanuCP08}~\cite{BeldiceanuCP08}\\
\index{noOverlap}\index{Constraints!noOverlap}noOverlap &  1.00 & \href{../works/IsikYA23.pdf}{IsikYA23}~\cite{IsikYA23}, \href{../works/JuvinHHL23.pdf}{JuvinHHL23}~\cite{JuvinHHL23}, \href{../works/ZhuSZW23.pdf}{ZhuSZW23}~\cite{ZhuSZW23}, \href{../works/abs-2305-19888.pdf}{abs-2305-19888}~\cite{abs-2305-19888}, \href{../works/NaderiRR23.pdf}{NaderiRR23}~\cite{NaderiRR23}, \href{../works/PopovicCGNC22.pdf}{PopovicCGNC22}~\cite{PopovicCGNC22}, \href{../works/HeinzNVH22.pdf}{HeinzNVH22}~\cite{HeinzNVH22}, \href{../works/ColT22.pdf}{ColT22}~\cite{ColT22}, \href{../works/VlkHT21.pdf}{VlkHT21}~\cite{VlkHT21}, \href{../works/MengLZB21.pdf}{MengLZB21}~\cite{MengLZB21}, \href{../works/RoshanaeiN21.pdf}{RoshanaeiN21}~\cite{RoshanaeiN21}, \href{../works/Groleaz21.pdf}{Groleaz21}~\cite{Groleaz21}, \href{../works/QinDCS20.pdf}{QinDCS20}~\cite{QinDCS20}, \href{../works/Lunardi20.pdf}{Lunardi20}~\cite{Lunardi20}, \href{../works/LunardiBLRV20.pdf}{LunardiBLRV20}~\cite{LunardiBLRV20}, \href{../works/GedikKEK18.pdf}{GedikKEK18}~\cite{GedikKEK18}, \href{../works/MelgarejoLS15.pdf}{MelgarejoLS15}~\cite{MelgarejoLS15} & \href{../works/abs-2306-05747.pdf}{abs-2306-05747}~\cite{abs-2306-05747}, \href{../works/KimCMLLP23.pdf}{KimCMLLP23}~\cite{KimCMLLP23}, \href{../works/LacknerMMWW23.pdf}{LacknerMMWW23}~\cite{LacknerMMWW23}, \href{../works/TasselGS23.pdf}{TasselGS23}~\cite{TasselGS23}, \href{../works/NaderiBZ22a.pdf}{NaderiBZ22a}~\cite{NaderiBZ22a}, \href{../works/PohlAK22.pdf}{PohlAK22}~\cite{PohlAK22}, \href{../works/AwadMDMT22.pdf}{AwadMDMT22}~\cite{AwadMDMT22}, \href{../works/YuraszeckMPV22.pdf}{YuraszeckMPV22}~\cite{YuraszeckMPV22}, \href{../works/AbreuN22.pdf}{AbreuN22}~\cite{AbreuN22}, \href{../works/SvancaraB22.pdf}{SvancaraB22}~\cite{SvancaraB22}, \href{../works/GhandehariK22.pdf}{GhandehariK22}~\cite{GhandehariK22}, \href{../works/KlankeBYE21.pdf}{KlankeBYE21}~\cite{KlankeBYE21}, \href{../works/Bedhief21.pdf}{Bedhief21}~\cite{Bedhief21}, \href{../works/BenderWS21.pdf}{BenderWS21}~\cite{BenderWS21}, \href{../works/ZouZ20.pdf}{ZouZ20}~\cite{ZouZ20}, \href{../works/Ham20a.pdf}{Ham20a}~\cite{Ham20a}, \href{../works/BenediktMH20.pdf}{BenediktMH20}~\cite{BenediktMH20}, \href{../works/SacramentoSP20.pdf}{SacramentoSP20}~\cite{SacramentoSP20}, \href{../works/RoshanaeiBAUB20.pdf}{RoshanaeiBAUB20}~\cite{RoshanaeiBAUB20}...\href{../works/TranVNB17.pdf}{TranVNB17}~\cite{TranVNB17}, \href{../works/CohenHB17.pdf}{CohenHB17}~\cite{CohenHB17}, \href{../works/GedikKBR17.pdf}{GedikKBR17}~\cite{GedikKBR17}, \href{../works/BoothNB16.pdf}{BoothNB16}~\cite{BoothNB16}, \href{../works/QinDS16.pdf}{QinDS16}~\cite{QinDS16}, \href{../works/NovaraNH16.pdf}{NovaraNH16}~\cite{NovaraNH16}, \href{../works/GoelSHFS15.pdf}{GoelSHFS15}~\cite{GoelSHFS15}, \href{../works/PraletLJ15.pdf}{PraletLJ15}~\cite{PraletLJ15}, \href{../works/ZhaoL14.pdf}{ZhaoL14}~\cite{ZhaoL14}, \href{../works/EdisO11.pdf}{EdisO11}~\cite{EdisO11} (Total: 42) & \href{../works/BonninMNE24.pdf}{BonninMNE24}~\cite{BonninMNE24}, \href{../works/LuZZYW24.pdf}{LuZZYW24}~\cite{LuZZYW24}, \href{../works/JuvinHL23a.pdf}{JuvinHL23a}~\cite{JuvinHL23a}, \href{../works/AbreuNP23.pdf}{AbreuNP23}~\cite{AbreuNP23}, \href{../works/SquillaciPR23.pdf}{SquillaciPR23}~\cite{SquillaciPR23}, \href{../works/NaderiBZ23.pdf}{NaderiBZ23}~\cite{NaderiBZ23}, \href{../works/YuraszeckMC23.pdf}{YuraszeckMC23}~\cite{YuraszeckMC23}, \href{../works/AalianPG23.pdf}{AalianPG23}~\cite{AalianPG23}, \href{../works/AbreuPNF23.pdf}{AbreuPNF23}~\cite{AbreuPNF23}, \href{../works/JuvinHL23.pdf}{JuvinHL23}~\cite{JuvinHL23}, \href{../works/CzerniachowskaWZ23.pdf}{CzerniachowskaWZ23}~\cite{CzerniachowskaWZ23}, \href{../works/MarliereSPR23.pdf}{MarliereSPR23}~\cite{MarliereSPR23}, \href{../works/WinterMMW22.pdf}{WinterMMW22}~\cite{WinterMMW22}, \href{../works/Teppan22.pdf}{Teppan22}~\cite{Teppan22}, \href{../works/NaderiBZ22.pdf}{NaderiBZ22}~\cite{NaderiBZ22}, \href{../works/YunusogluY22.pdf}{YunusogluY22}~\cite{YunusogluY22}, \href{../works/CampeauG22.pdf}{CampeauG22}~\cite{CampeauG22}, \href{../works/OujanaAYB22.pdf}{OujanaAYB22}~\cite{OujanaAYB22}, \href{../works/ArmstrongGOS22.pdf}{ArmstrongGOS22}~\cite{ArmstrongGOS22}...\href{../works/CappartTSR18.pdf}{CappartTSR18}~\cite{CappartTSR18}, \href{../works/CappartS17.pdf}{CappartS17}~\cite{CappartS17}, \href{../works/BoothTNB16.pdf}{BoothTNB16}~\cite{BoothTNB16}, \href{../works/FrankDT16.pdf}{FrankDT16}~\cite{FrankDT16}, \href{../works/HechingH16.pdf}{HechingH16}~\cite{HechingH16}, \href{../works/GrimesH15.pdf}{GrimesH15}~\cite{GrimesH15}, \href{../works/VilimLS15.pdf}{VilimLS15}~\cite{VilimLS15}, \href{../works/WangMD15.pdf}{WangMD15}~\cite{WangMD15}, \href{../works/LaborieR14.pdf}{LaborieR14}~\cite{LaborieR14}, \href{../works/HamdiL13.pdf}{HamdiL13}~\cite{HamdiL13} (Total: 54)\\
\index{regular expression}\index{Constraints!regular expression}regular expression &  1.00 &  & \href{../works/WessenCSFPM23.pdf}{WessenCSFPM23}~\cite{WessenCSFPM23}, \href{../works/FrimodigS19.pdf}{FrimodigS19}~\cite{FrimodigS19} & \href{../works/HookerH17.pdf}{HookerH17}~\cite{HookerH17}, \href{../works/CarlssonJL17.pdf}{CarlssonJL17}~\cite{CarlssonJL17}, \href{../works/LarsonJC14.pdf}{LarsonJC14}~\cite{LarsonJC14}\\
\index{span constraint}\index{Constraints!span constraint}span constraint &  1.00 &  & \href{../works/Groleaz21.pdf}{Groleaz21}~\cite{Groleaz21}, \href{../works/CappartS17.pdf}{CappartS17}~\cite{CappartS17}, \href{../works/LaborieR14.pdf}{LaborieR14}~\cite{LaborieR14}, \href{../works/SchuttFS13.pdf}{SchuttFS13}~\cite{SchuttFS13}, \href{../works/Lombardi10.pdf}{Lombardi10}~\cite{Lombardi10}, \href{../works/LombardiM10a.pdf}{LombardiM10a}~\cite{LombardiM10a}, \href{../works/Darby-DowmanLMZ97.pdf}{Darby-DowmanLMZ97}~\cite{Darby-DowmanLMZ97} & \href{../works/ZhangBB22.pdf}{ZhangBB22}~\cite{ZhangBB22}, \href{../works/AwadMDMT22.pdf}{AwadMDMT22}~\cite{AwadMDMT22}, \href{../works/OujanaAYB22.pdf}{OujanaAYB22}~\cite{OujanaAYB22}, \href{../works/ZouZ20.pdf}{ZouZ20}~\cite{ZouZ20}, \href{../works/TangB20.pdf}{TangB20}~\cite{TangB20}, \href{../works/YounespourAKE19.pdf}{YounespourAKE19}~\cite{YounespourAKE19}, \href{../works/LaborieRSV18.pdf}{LaborieRSV18}~\cite{LaborieRSV18}, \href{../works/SimoninAHL15.pdf}{SimoninAHL15}~\cite{SimoninAHL15}, \href{../works/ZhaoL14.pdf}{ZhaoL14}~\cite{ZhaoL14}, \href{../works/SimoninAHL12.pdf}{SimoninAHL12}~\cite{SimoninAHL12}, \href{../works/SchuttFSW11.pdf}{SchuttFSW11}~\cite{SchuttFSW11}, \href{../works/GetoorOFC97.pdf}{GetoorOFC97}~\cite{GetoorOFC97}\\
\index{table constraint}\index{Constraints!table constraint}table constraint &  1.00 & \href{../works/FalqueALM24.pdf}{FalqueALM24}~\cite{FalqueALM24}, \href{../works/LiuLH18.pdf}{LiuLH18}~\cite{LiuLH18}, \href{../works/CarlssonJL17.pdf}{CarlssonJL17}~\cite{CarlssonJL17}, \href{../works/ReddyFIBKAJ11.pdf}{ReddyFIBKAJ11}~\cite{ReddyFIBKAJ11}, \href{../works/LombardiM10a.pdf}{LombardiM10a}~\cite{LombardiM10a}, \href{../works/Lombardi10.pdf}{Lombardi10}~\cite{Lombardi10}, \href{../works/Baptiste02.pdf}{Baptiste02}~\cite{Baptiste02}, \href{../works/PapaB98.pdf}{PapaB98}~\cite{PapaB98} & \href{../works/MarliereSPR23.pdf}{MarliereSPR23}~\cite{MarliereSPR23}, \href{../works/JelinekB16.pdf}{JelinekB16}~\cite{JelinekB16}, \href{../works/LombardiMRB10.pdf}{LombardiMRB10}~\cite{LombardiMRB10} & \href{../works/PerezGSL23.pdf}{PerezGSL23}~\cite{PerezGSL23}, \href{../works/abs-2312-13682.pdf}{abs-2312-13682}~\cite{abs-2312-13682}, \href{../works/CilKLO22.pdf}{CilKLO22}~\cite{CilKLO22}, \href{../works/ArmstrongGOS21.pdf}{ArmstrongGOS21}~\cite{ArmstrongGOS21}, \href{../works/LiuLH19a.pdf}{LiuLH19a}~\cite{LiuLH19a}, \href{../works/CauwelaertLS18.pdf}{CauwelaertLS18}~\cite{CauwelaertLS18}, \href{../works/Siala15.pdf}{Siala15}~\cite{Siala15}, \href{../works/GayHS15.pdf}{GayHS15}~\cite{GayHS15}, \href{../works/PesantRR15.pdf}{PesantRR15}~\cite{PesantRR15}, \href{../works/MelgarejoLS15.pdf}{MelgarejoLS15}~\cite{MelgarejoLS15}, \href{../works/Siala15a.pdf}{Siala15a}~\cite{Siala15a}, \href{../works/CauwelaertLS15.pdf}{CauwelaertLS15}~\cite{CauwelaertLS15}, \href{../works/LimtanyakulS12.pdf}{LimtanyakulS12}~\cite{LimtanyakulS12}, \href{../works/BeniniLMR11.pdf}{BeniniLMR11}~\cite{BeniniLMR11}, \href{../works/BeckFW11.pdf}{BeckFW11}~\cite{BeckFW11}, \href{../works/HermenierDL11.pdf}{HermenierDL11}~\cite{HermenierDL11}, \href{../works/LopesCSM10.pdf}{LopesCSM10}~\cite{LopesCSM10}, \href{../works/MouraSCL08.pdf}{MouraSCL08}~\cite{MouraSCL08}, \href{../works/Perron05.pdf}{Perron05}~\cite{Perron05}, \href{../works/GodardLN05.pdf}{GodardLN05}~\cite{GodardLN05}, \href{../works/Laborie03.pdf}{Laborie03}~\cite{Laborie03}, \href{../works/ElkhyariGJ02.pdf}{ElkhyariGJ02}~\cite{ElkhyariGJ02}, \href{../works/WallaceF00.pdf}{WallaceF00}~\cite{WallaceF00}, \href{../works/Nuijten94.pdf}{Nuijten94}~\cite{Nuijten94}\\
\end{longtable}
}

\clearpage
\subsection{Concept Type ApplicationAreas}
\label{sec:ApplicationAreas}
\label{ApplicationAreas}
{\scriptsize
\begin{longtable}{p{3cm}r>{\raggedright\arraybackslash}p{6cm}>{\raggedright\arraybackslash}p{6cm}>{\raggedright\arraybackslash}p{8cm}}
\rowcolor{white}\caption{Works for Concepts of Type ApplicationAreas (Total 67 Concepts, 66 Used)}\\ \toprule
\rowcolor{white}Keyword & Weight & High & Medium & Low\\ \midrule\endhead
\bottomrule
\endfoot
\index{COVID}\index{ApplicationAreas!COVID}COVID &  1.00 & \href{../works/GuoZ23.pdf}{GuoZ23}~\cite{GuoZ23} & \href{../works/GeibingerKKMMW21.pdf}{GeibingerKKMMW21}~\cite{GeibingerKKMMW21} & \href{../works/FalqueALM24.pdf}{FalqueALM24}~\cite{FalqueALM24}, \href{../works/BonninMNE24.pdf}{BonninMNE24}~\cite{BonninMNE24}, \href{../works/Mehdizadeh-Somarin23.pdf}{Mehdizadeh-Somarin23}~\cite{Mehdizadeh-Somarin23}, \href{../works/Fatemi-AnarakiTFV23.pdf}{Fatemi-AnarakiTFV23}~\cite{Fatemi-AnarakiTFV23}, \href{../works/JuvinHL23a.pdf}{JuvinHL23a}~\cite{JuvinHL23a}, \href{../works/GurPAE23.pdf}{GurPAE23}~\cite{GurPAE23}, \href{../works/OujanaAYB22.pdf}{OujanaAYB22}~\cite{OujanaAYB22}, \href{../works/GhandehariK22.pdf}{GhandehariK22}~\cite{GhandehariK22}, \href{../works/Lemos21.pdf}{Lemos21}~\cite{Lemos21}\\
\index{HVAC}\index{ApplicationAreas!HVAC}HVAC &  1.00 & \href{../works/LimHTB16.pdf}{LimHTB16}~\cite{LimHTB16}, \href{../works/LimBTBB15.pdf}{LimBTBB15}~\cite{LimBTBB15}, \href{../works/LimBTBB15a.pdf}{LimBTBB15a}~\cite{LimBTBB15a} & \href{../works/GrimesIOS14.pdf}{GrimesIOS14}~\cite{GrimesIOS14} & \\
\index{agriculture}\index{ApplicationAreas!agriculture}agriculture &  1.00 &  &  & \href{../works/AkramNHRSA23.pdf}{AkramNHRSA23}~\cite{AkramNHRSA23}, \href{../works/Astrand0F21.pdf}{Astrand0F21}~\cite{Astrand0F21}, \href{../works/HamPK21.pdf}{HamPK21}~\cite{HamPK21}, \href{../works/Astrand21.pdf}{Astrand21}~\cite{Astrand21}, \href{../works/BenderWS21.pdf}{BenderWS21}~\cite{BenderWS21}, \href{../works/QinWSLS21.pdf}{QinWSLS21}~\cite{QinWSLS21}, \href{../works/MejiaY20.pdf}{MejiaY20}~\cite{MejiaY20}\\
\index{aircraft}\index{ApplicationAreas!aircraft}aircraft &  1.00 & \href{../works/GokPTGO23.pdf}{GokPTGO23}~\cite{GokPTGO23}, \href{../works/PohlAK22.pdf}{PohlAK22}~\cite{PohlAK22}, \href{../works/OrnekOS20.pdf}{OrnekOS20}~\cite{OrnekOS20}, \href{../works/WangB20.pdf}{WangB20}~\cite{WangB20}, \href{../works/GokGSTO20.pdf}{GokGSTO20}~\cite{GokGSTO20}, \href{../works/TranDRFWOVB16.pdf}{TranDRFWOVB16}~\cite{TranDRFWOVB16}, \href{../works/Fahimi16.pdf}{Fahimi16}~\cite{Fahimi16}, \href{../works/BajestaniB13.pdf}{BajestaniB13}~\cite{BajestaniB13}, \href{../works/LombardiM12.pdf}{LombardiM12}~\cite{LombardiM12}, \href{../works/BajestaniB11.pdf}{BajestaniB11}~\cite{BajestaniB11}, \href{../works/Gronkvist06.pdf}{Gronkvist06}~\cite{Gronkvist06}, \href{../works/ArtiouchineB05.pdf}{ArtiouchineB05}~\cite{ArtiouchineB05}, \href{../works/FrankK05.pdf}{FrankK05}~\cite{FrankK05}, \href{../works/FrankK03.pdf}{FrankK03}~\cite{FrankK03}, \href{../works/Simonis99.pdf}{Simonis99}~\cite{Simonis99}, \href{../works/ChunCTY99.pdf}{ChunCTY99}~\cite{ChunCTY99}, \href{../works/JoLLH99.pdf}{JoLLH99}~\cite{JoLLH99} & \href{../works/WangB23.pdf}{WangB23}~\cite{WangB23}, \href{../works/GombolayWS18.pdf}{GombolayWS18}~\cite{GombolayWS18}, \href{../works/Ham18.pdf}{Ham18}~\cite{Ham18}, \href{../works/Simonis07.pdf}{Simonis07}~\cite{Simonis07}, \href{../works/KusterJF07.pdf}{KusterJF07}~\cite{KusterJF07}, \href{../works/SakkoutW00.pdf}{SakkoutW00}~\cite{SakkoutW00}, \href{../works/Simonis95a.pdf}{Simonis95a}~\cite{Simonis95a} & \href{../works/FalqueALM24.pdf}{FalqueALM24}~\cite{FalqueALM24}, \href{../works/PrataAN23.pdf}{PrataAN23}~\cite{PrataAN23}, \href{../works/PovedaAA23.pdf}{PovedaAA23}~\cite{PovedaAA23}, \href{../works/Adelgren2023.pdf}{Adelgren2023}~\cite{Adelgren2023}, \href{../works/ElciOH22.pdf}{ElciOH22}~\cite{ElciOH22}, \href{../works/Tassel22.pdf}{Tassel22}~\cite{Tassel22}, \href{../works/EtminaniesfahaniGNMS22.pdf}{EtminaniesfahaniGNMS22}~\cite{EtminaniesfahaniGNMS22}, \href{../works/HamP21.pdf}{HamP21}~\cite{HamP21}, \href{../works/HauderBRPA20.pdf}{HauderBRPA20}~\cite{HauderBRPA20}, \href{../works/ZarandiASC20.pdf}{ZarandiASC20}~\cite{ZarandiASC20}, \href{../works/abs-1902-09244.pdf}{abs-1902-09244}~\cite{abs-1902-09244}, \href{../works/Hooker19.pdf}{Hooker19}~\cite{Hooker19}, \href{../works/LaborieRSV18.pdf}{LaborieRSV18}~\cite{LaborieRSV18}, \href{../works/AgussurjaKL18.pdf}{AgussurjaKL18}~\cite{AgussurjaKL18}, \href{../works/HookerH17.pdf}{HookerH17}~\cite{HookerH17}, \href{../works/TranAB16.pdf}{TranAB16}~\cite{TranAB16}, \href{../works/AlesioBNG15.pdf}{AlesioBNG15}~\cite{AlesioBNG15}, \href{../works/LaborieR14.pdf}{LaborieR14}~\cite{LaborieR14}, \href{../works/Lombardi10.pdf}{Lombardi10}~\cite{Lombardi10}...\href{../works/KrogtLPHJ07.pdf}{KrogtLPHJ07}~\cite{KrogtLPHJ07}, \href{../works/CambazardHDJT04.pdf}{CambazardHDJT04}~\cite{CambazardHDJT04}, \href{../works/MartinPY01.pdf}{MartinPY01}~\cite{MartinPY01}, \href{../works/SimonisCK00.pdf}{SimonisCK00}~\cite{SimonisCK00}, \href{../works/GruianK98.pdf}{GruianK98}~\cite{GruianK98}, \href{../works/Darby-DowmanLMZ97.pdf}{Darby-DowmanLMZ97}~\cite{Darby-DowmanLMZ97}, \href{../works/SadehF96.pdf}{SadehF96}~\cite{SadehF96}, \href{../works/Wallace96.pdf}{Wallace96}~\cite{Wallace96}, \href{../works/Simonis95.pdf}{Simonis95}~\cite{Simonis95}, \href{../works/SimonisC95.pdf}{SimonisC95}~\cite{SimonisC95} (Total: 33)\\
\index{airport}\index{ApplicationAreas!airport}airport &  1.00 & \href{../works/FalqueALM24.pdf}{FalqueALM24}~\cite{FalqueALM24}, \href{../works/ForbesHJST24.pdf}{ForbesHJST24}~\cite{ForbesHJST24}, \href{../works/GokPTGO23.pdf}{GokPTGO23}~\cite{GokPTGO23}, \href{../works/PohlAK22.pdf}{PohlAK22}~\cite{PohlAK22}, \href{../works/OrnekOS20.pdf}{OrnekOS20}~\cite{OrnekOS20}, \href{../works/WangB20.pdf}{WangB20}~\cite{WangB20}, \href{../works/GokGSTO20.pdf}{GokGSTO20}~\cite{GokGSTO20}, \href{../works/TranDRFWOVB16.pdf}{TranDRFWOVB16}~\cite{TranDRFWOVB16}, \href{../works/Simonis07.pdf}{Simonis07}~\cite{Simonis07}, \href{../works/Gronkvist06.pdf}{Gronkvist06}~\cite{Gronkvist06}, \href{../works/FrankK05.pdf}{FrankK05}~\cite{FrankK05}, \href{../works/FrankK03.pdf}{FrankK03}~\cite{FrankK03}, \href{../works/ChunCTY99.pdf}{ChunCTY99}~\cite{ChunCTY99}, \href{../works/JoLLH99.pdf}{JoLLH99}~\cite{JoLLH99}, \href{../works/DincbasS91.pdf}{DincbasS91}~\cite{DincbasS91} & \href{../works/WangB23.pdf}{WangB23}~\cite{WangB23}, \href{../works/ZarandiASC20.pdf}{ZarandiASC20}~\cite{ZarandiASC20}, \href{../works/ChunS14.pdf}{ChunS14}~\cite{ChunS14}, \href{../works/GarridoAO09.pdf}{GarridoAO09}~\cite{GarridoAO09}, \href{../works/KusterJF07.pdf}{KusterJF07}~\cite{KusterJF07}, \href{../works/Mason01.pdf}{Mason01}~\cite{Mason01}, \href{../works/Simonis99.pdf}{Simonis99}~\cite{Simonis99}, \href{../works/Puget95.pdf}{Puget95}~\cite{Puget95} & \href{../works/SquillaciPR23.pdf}{SquillaciPR23}~\cite{SquillaciPR23}, \href{../works/NaderiRR23.pdf}{NaderiRR23}~\cite{NaderiRR23}, \href{../works/BourreauGGLT22.pdf}{BourreauGGLT22}~\cite{BourreauGGLT22}, \href{../works/Lunardi20.pdf}{Lunardi20}~\cite{Lunardi20}, \href{../works/DemirovicS18.pdf}{DemirovicS18}~\cite{DemirovicS18}, \href{../works/HurleyOS16.pdf}{HurleyOS16}~\cite{HurleyOS16}, \href{../works/Fahimi16.pdf}{Fahimi16}~\cite{Fahimi16}, \href{../works/HarjunkoskiMBC14.pdf}{HarjunkoskiMBC14}~\cite{HarjunkoskiMBC14}, \href{../works/ArtiouchineB05.pdf}{ArtiouchineB05}~\cite{ArtiouchineB05}, \href{../works/Elkhyari03.pdf}{Elkhyari03}~\cite{Elkhyari03}, \href{../works/Wallace96.pdf}{Wallace96}~\cite{Wallace96}\\
\index{astronomy}\index{ApplicationAreas!astronomy}astronomy &  1.00 &  & \href{../works/FrankK05.pdf}{FrankK05}~\cite{FrankK05} & \href{../works/CatusseCBL16.pdf}{CatusseCBL16}~\cite{CatusseCBL16}, \href{../works/LiW08.pdf}{LiW08}~\cite{LiW08}, \href{../works/Johnston05.pdf}{Johnston05}~\cite{Johnston05}, \href{../works/FrankK03.pdf}{FrankK03}~\cite{FrankK03}, \href{../works/MintonJPL92.pdf}{MintonJPL92}~\cite{MintonJPL92}, \href{../works/MintonJPL90.pdf}{MintonJPL90}~\cite{MintonJPL90}\\
\index{automotive}\index{ApplicationAreas!automotive}automotive &  1.00 &  & \href{../works/GuoZ23.pdf}{GuoZ23}~\cite{GuoZ23}, \href{../works/NaderiBZ23.pdf}{NaderiBZ23}~\cite{NaderiBZ23}, \href{../works/YuraszeckMPV22.pdf}{YuraszeckMPV22}~\cite{YuraszeckMPV22}, \href{../works/EmdeZD22.pdf}{EmdeZD22}~\cite{EmdeZD22}, \href{../works/Groleaz21.pdf}{Groleaz21}~\cite{Groleaz21}, \href{../works/AlesioBNG15.pdf}{AlesioBNG15}~\cite{AlesioBNG15}, \href{../works/LimtanyakulS12.pdf}{LimtanyakulS12}~\cite{LimtanyakulS12}, \href{../works/SunLYL10.pdf}{SunLYL10}~\cite{SunLYL10}, \href{../works/Lombardi10.pdf}{Lombardi10}~\cite{Lombardi10}, \href{../works/BarlattCG08.pdf}{BarlattCG08}~\cite{BarlattCG08}, \href{../works/SchildW00.pdf}{SchildW00}~\cite{SchildW00} & \href{../works/LuZZYW24.pdf}{LuZZYW24}~\cite{LuZZYW24}, \href{../works/NaderiRR23.pdf}{NaderiRR23}~\cite{NaderiRR23}, \href{../works/NaderiBZR23.pdf}{NaderiBZR23}~\cite{NaderiBZR23}, \href{../works/PovedaAA23.pdf}{PovedaAA23}~\cite{PovedaAA23}, \href{../works/CzerniachowskaWZ23.pdf}{CzerniachowskaWZ23}~\cite{CzerniachowskaWZ23}, \href{../works/NaderiBZ22a.pdf}{NaderiBZ22a}~\cite{NaderiBZ22a}, \href{../works/NaderiBZ22.pdf}{NaderiBZ22}~\cite{NaderiBZ22}, \href{../works/HubnerGSV21.pdf}{HubnerGSV21}~\cite{HubnerGSV21}, \href{../works/VlkHT21.pdf}{VlkHT21}~\cite{VlkHT21}, \href{../works/RoshanaeiN21.pdf}{RoshanaeiN21}~\cite{RoshanaeiN21}, \href{../works/AntuoriHHEN21.pdf}{AntuoriHHEN21}~\cite{AntuoriHHEN21}, \href{../works/AbreuAPNM21.pdf}{AbreuAPNM21}~\cite{AbreuAPNM21}, \href{../works/KoehlerBFFHPSSS21.pdf}{KoehlerBFFHPSSS21}~\cite{KoehlerBFFHPSSS21}, \href{../works/AbidinK20.pdf}{AbidinK20}~\cite{AbidinK20}, \href{../works/BarzegaranZP20.pdf}{BarzegaranZP20}~\cite{BarzegaranZP20}, \href{../works/KonowalenkoMM19.pdf}{KonowalenkoMM19}~\cite{KonowalenkoMM19}, \href{../works/abs-1911-04766.pdf}{abs-1911-04766}~\cite{abs-1911-04766}, \href{../works/TanZWGQ19.pdf}{TanZWGQ19}~\cite{TanZWGQ19}, \href{../works/GeibingerMM19.pdf}{GeibingerMM19}~\cite{GeibingerMM19}...\href{../works/Siala15.pdf}{Siala15}~\cite{Siala15}, \href{../works/Siala15a.pdf}{Siala15a}~\cite{Siala15a}, \href{../works/LudwigKRBMS14.pdf}{LudwigKRBMS14}~\cite{LudwigKRBMS14}, \href{../works/AlesioNBG14.pdf}{AlesioNBG14}~\cite{AlesioNBG14}, \href{../works/HarjunkoskiMBC14.pdf}{HarjunkoskiMBC14}~\cite{HarjunkoskiMBC14}, \href{../works/HladikCDJ08.pdf}{HladikCDJ08}~\cite{HladikCDJ08}, \href{../works/BeniniBGM06.pdf}{BeniniBGM06}~\cite{BeniniBGM06}, \href{../works/KovacsV06.pdf}{KovacsV06}~\cite{KovacsV06}, \href{../works/BeniniBGM05.pdf}{BeniniBGM05}~\cite{BeniniBGM05}, \href{../works/CambazardHDJT04.pdf}{CambazardHDJT04}~\cite{CambazardHDJT04} (Total: 32)\\
\index{break minimization problem}\index{ApplicationAreas!break minimization problem}break minimization problem &  1.00 & \href{../works/KendallKRU10.pdf}{KendallKRU10}~\cite{KendallKRU10} & \href{../works/ElfJR03.pdf}{ElfJR03}~\cite{ElfJR03} & \href{../works/ZengM12.pdf}{ZengM12}~\cite{ZengM12}, \href{../works/Ribeiro12.pdf}{Ribeiro12}~\cite{Ribeiro12}, \href{../works/RasmussenT09.pdf}{RasmussenT09}~\cite{RasmussenT09}, \href{../works/RasmussenT07.pdf}{RasmussenT07}~\cite{RasmussenT07}\\
\index{business process}\index{ApplicationAreas!business process}business process &  1.00 & \href{../works/BadicaBI20.pdf}{BadicaBI20}~\cite{BadicaBI20}, \href{../works/LombardiM10a.pdf}{LombardiM10a}~\cite{LombardiM10a}, \href{../works/Lombardi10.pdf}{Lombardi10}~\cite{Lombardi10}, \href{../works/RoweJCA96.pdf}{RoweJCA96}~\cite{RoweJCA96} & \href{../works/SenderovichBB19.pdf}{SenderovichBB19}~\cite{SenderovichBB19} & \href{../works/SubulanC22.pdf}{SubulanC22}~\cite{SubulanC22}, \href{../works/Groleaz21.pdf}{Groleaz21}~\cite{Groleaz21}, \href{../works/Zahout21.pdf}{Zahout21}~\cite{Zahout21}, \href{../works/ZarandiASC20.pdf}{ZarandiASC20}~\cite{ZarandiASC20}, \href{../works/BadicaBIL19.pdf}{BadicaBIL19}~\cite{BadicaBIL19}, \href{../works/Jans09.pdf}{Jans09}~\cite{Jans09}, \href{../works/Simonis07.pdf}{Simonis07}~\cite{Simonis07}, \href{../works/SimonisCK00.pdf}{SimonisCK00}~\cite{SimonisCK00}, \href{../works/Simonis99.pdf}{Simonis99}~\cite{Simonis99}, \href{../works/BeckF98.pdf}{BeckF98}~\cite{BeckF98}, \href{../works/Simonis95a.pdf}{Simonis95a}~\cite{Simonis95a}\\
\index{cable tree}\index{ApplicationAreas!cable tree}cable tree &  1.00 & \href{../works/KoehlerBFFHPSSS21.pdf}{KoehlerBFFHPSSS21}~\cite{KoehlerBFFHPSSS21} &  & \\
\index{car manufacturing}\index{ApplicationAreas!car manufacturing}car manufacturing &  1.00 &  & \href{../works/AntuoriHHEN21.pdf}{AntuoriHHEN21}~\cite{AntuoriHHEN21} & \href{../works/BeldiceanuC94.pdf}{BeldiceanuC94}~\cite{BeldiceanuC94}\\
\index{container terminal}\index{ApplicationAreas!container terminal}container terminal &  1.00 & \href{../works/QinDCS20.pdf}{QinDCS20}~\cite{QinDCS20}, \href{../works/SacramentoSP20.pdf}{SacramentoSP20}~\cite{SacramentoSP20}, \href{../works/SunTB19.pdf}{SunTB19}~\cite{SunTB19}, \href{../works/UnsalO19.pdf}{UnsalO19}~\cite{UnsalO19}, \href{../works/ZampelliVSDR13.pdf}{ZampelliVSDR13}~\cite{ZampelliVSDR13}, \href{../works/UnsalO13.pdf}{UnsalO13}~\cite{UnsalO13} & \href{../works/LaborieRSV18.pdf}{LaborieRSV18}~\cite{LaborieRSV18}, \href{../works/QinDS16.pdf}{QinDS16}~\cite{QinDS16} & \href{../works/LuZZYW24.pdf}{LuZZYW24}~\cite{LuZZYW24}, \href{../works/abs-2312-13682.pdf}{abs-2312-13682}~\cite{abs-2312-13682}, \href{../works/PerezGSL23.pdf}{PerezGSL23}~\cite{PerezGSL23}, \href{../works/TouatBT22.pdf}{TouatBT22}~\cite{TouatBT22}, \href{../works/ZarandiASC20.pdf}{ZarandiASC20}~\cite{ZarandiASC20}, \href{../works/AbidinK20.pdf}{AbidinK20}~\cite{AbidinK20}, \href{../works/FallahiAC20.pdf}{FallahiAC20}~\cite{FallahiAC20}, \href{../works/CauwelaertDS20.pdf}{CauwelaertDS20}~\cite{CauwelaertDS20}, \href{../works/WallaceY20.pdf}{WallaceY20}~\cite{WallaceY20}, \href{../works/Hooker19.pdf}{Hooker19}~\cite{Hooker19}, \href{../works/Dejemeppe16.pdf}{Dejemeppe16}~\cite{Dejemeppe16}, \href{../works/CauwelaertDMS16.pdf}{CauwelaertDMS16}~\cite{CauwelaertDMS16}, \href{../works/DejemeppeCS15.pdf}{DejemeppeCS15}~\cite{DejemeppeCS15}, \href{../works/NovasH12.pdf}{NovasH12}~\cite{NovasH12}, \href{../works/CorreaLR07.pdf}{CorreaLR07}~\cite{CorreaLR07}, \href{../works/LimRX04.pdf}{LimRX04}~\cite{LimRX04}\\
\index{crew-scheduling}\index{ApplicationAreas!crew-scheduling}crew-scheduling &  1.00 & \href{../works/ZarandiASC20.pdf}{ZarandiASC20}~\cite{ZarandiASC20}, \href{../works/PourDERB18.pdf}{PourDERB18}~\cite{PourDERB18}, \href{../works/MorgadoM97.pdf}{MorgadoM97}~\cite{MorgadoM97} & \href{../works/BourreauGGLT22.pdf}{BourreauGGLT22}~\cite{BourreauGGLT22}, \href{../works/KletzanderMH21.pdf}{KletzanderMH21}~\cite{KletzanderMH21}, \href{../works/Zahout21.pdf}{Zahout21}~\cite{Zahout21}, \href{../works/KletzanderM20.pdf}{KletzanderM20}~\cite{KletzanderM20}, \href{../works/GombolayWS18.pdf}{GombolayWS18}~\cite{GombolayWS18}, \href{../works/Mason01.pdf}{Mason01}~\cite{Mason01}, \href{../works/Touraivane95.pdf}{Touraivane95}~\cite{Touraivane95} & \href{../works/WangB23.pdf}{WangB23}~\cite{WangB23}, \href{../works/NaderiBZ23.pdf}{NaderiBZ23}~\cite{NaderiBZ23}, \href{../works/Adelgren2023.pdf}{Adelgren2023}~\cite{Adelgren2023}, \href{../works/NaderiBZR23.pdf}{NaderiBZR23}~\cite{NaderiBZR23}, \href{../works/NaderiRR23.pdf}{NaderiRR23}~\cite{NaderiRR23}, \href{../works/NaderiBZ22.pdf}{NaderiBZ22}~\cite{NaderiBZ22}, \href{../works/ElciOH22.pdf}{ElciOH22}~\cite{ElciOH22}, \href{../works/NaderiBZ22a.pdf}{NaderiBZ22a}~\cite{NaderiBZ22a}, \href{../works/AwadMDMT22.pdf}{AwadMDMT22}~\cite{AwadMDMT22}, \href{../works/EtminaniesfahaniGNMS22.pdf}{EtminaniesfahaniGNMS22}~\cite{EtminaniesfahaniGNMS22}, \href{../works/GhandehariK22.pdf}{GhandehariK22}~\cite{GhandehariK22}, \href{../works/HeinzNVH22.pdf}{HeinzNVH22}~\cite{HeinzNVH22}, \href{../works/Edis21.pdf}{Edis21}~\cite{Edis21}, \href{../works/Lemos21.pdf}{Lemos21}~\cite{Lemos21}, \href{../works/MokhtarzadehTNF20.pdf}{MokhtarzadehTNF20}~\cite{MokhtarzadehTNF20}, \href{../works/TangLWSK18.pdf}{TangLWSK18}~\cite{TangLWSK18}, \href{../works/AgussurjaKL18.pdf}{AgussurjaKL18}~\cite{AgussurjaKL18}, \href{../works/HookerH17.pdf}{HookerH17}~\cite{HookerH17}, \href{../works/DoulabiRP16.pdf}{DoulabiRP16}~\cite{DoulabiRP16}...\href{../works/KendallKRU10.pdf}{KendallKRU10}~\cite{KendallKRU10}, \href{../works/WuBB09.pdf}{WuBB09}~\cite{WuBB09}, \href{../works/MilanoW09.pdf}{MilanoW09}~\cite{MilanoW09}, \href{../works/Gronkvist06.pdf}{Gronkvist06}~\cite{Gronkvist06}, \href{../works/MilanoW06.pdf}{MilanoW06}~\cite{MilanoW06}, \href{../works/BeldiceanuC02.pdf}{BeldiceanuC02}~\cite{BeldiceanuC02}, \href{../works/JainG01.pdf}{JainG01}~\cite{JainG01}, \href{../works/BosiM2001.pdf}{BosiM2001}~\cite{BosiM2001}, \href{../works/EreminW01.pdf}{EreminW01}~\cite{EreminW01}, \href{../works/SimonisCK00.pdf}{SimonisCK00}~\cite{SimonisCK00} (Total: 35)\\
\index{dairies}\index{ApplicationAreas!dairies}dairies &  1.00 &  &  & \href{../works/Bartak02.pdf}{Bartak02}~\cite{Bartak02}, \href{../works/Bartak02a.pdf}{Bartak02a}~\cite{Bartak02a}\\
\index{dairy}\index{ApplicationAreas!dairy}dairy &  1.00 & \href{../works/EscobetPQPRA19.pdf}{EscobetPQPRA19}~\cite{EscobetPQPRA19} & \href{../works/PrataAN23.pdf}{PrataAN23}~\cite{PrataAN23}, \href{../works/HarjunkoskiMBC14.pdf}{HarjunkoskiMBC14}~\cite{HarjunkoskiMBC14} & \href{../works/Groleaz21.pdf}{Groleaz21}~\cite{Groleaz21}\\
\index{datacenter}\index{ApplicationAreas!datacenter}datacenter &  1.00 & \href{../works/HermenierDL11.pdf}{HermenierDL11}~\cite{HermenierDL11} &  & \href{../works/Zahout21.pdf}{Zahout21}~\cite{Zahout21}, \href{../works/GalleguillosKSB19.pdf}{GalleguillosKSB19}~\cite{GalleguillosKSB19}, \href{../works/Madi-WambaLOBM17.pdf}{Madi-WambaLOBM17}~\cite{Madi-WambaLOBM17}, \href{../works/Letort13.pdf}{Letort13}~\cite{Letort13}, \href{../works/LetortBC12.pdf}{LetortBC12}~\cite{LetortBC12}, \href{../works/IfrimOS12.pdf}{IfrimOS12}~\cite{IfrimOS12}\\
\index{datacentre}\index{ApplicationAreas!datacentre}datacentre &  1.00 &  & \href{../works/HurleyOS16.pdf}{HurleyOS16}~\cite{HurleyOS16} & \\
\index{day-ahead market}\index{ApplicationAreas!day-ahead market}day-ahead market &  1.00 &  &  & \\
\index{deep space}\index{ApplicationAreas!deep space}deep space &  1.00 & \href{../works/Johnston05.pdf}{Johnston05}~\cite{Johnston05} &  & \href{../works/HebrardALLCMR22.pdf}{HebrardALLCMR22}~\cite{HebrardALLCMR22}, \href{../works/Ham20a.pdf}{Ham20a}~\cite{Ham20a}, \href{../works/ReddyFIBKAJ11.pdf}{ReddyFIBKAJ11}~\cite{ReddyFIBKAJ11}, \href{../works/GlobusCLP04.pdf}{GlobusCLP04}~\cite{GlobusCLP04}, \href{../works/WallaceF00.pdf}{WallaceF00}~\cite{WallaceF00}\\
\index{drone}\index{ApplicationAreas!drone}drone &  1.00 & \href{../works/MontemanniD23a.pdf}{MontemanniD23a}~\cite{MontemanniD23a}, \href{../works/MontemanniD23.pdf}{MontemanniD23}~\cite{MontemanniD23}, \href{../works/Ham20a.pdf}{Ham20a}~\cite{Ham20a}, \href{../works/Ham18.pdf}{Ham18}~\cite{Ham18} &  & \href{../works/Adelgren2023.pdf}{Adelgren2023}~\cite{Adelgren2023}, \href{../works/GuoZ23.pdf}{GuoZ23}~\cite{GuoZ23}, \href{../works/JuvinHL23a.pdf}{JuvinHL23a}~\cite{JuvinHL23a}, \href{../works/ShaikhK23.pdf}{ShaikhK23}~\cite{ShaikhK23}, \href{../works/EmdeZD22.pdf}{EmdeZD22}~\cite{EmdeZD22}, \href{../works/Astrand0F21.pdf}{Astrand0F21}~\cite{Astrand0F21}, \href{../works/AntuoriHHEN21.pdf}{AntuoriHHEN21}~\cite{AntuoriHHEN21}, \href{../works/HamP21.pdf}{HamP21}~\cite{HamP21}, \href{../works/Astrand21.pdf}{Astrand21}~\cite{Astrand21}, \href{../works/ZarandiASC20.pdf}{ZarandiASC20}~\cite{ZarandiASC20}, \href{../works/Ham18a.pdf}{Ham18a}~\cite{Ham18a}, \href{../works/HamFC17.pdf}{HamFC17}~\cite{HamFC17}\\
\index{earth observation}\index{ApplicationAreas!earth observation}earth observation &  1.00 & \href{../works/SquillaciPR23.pdf}{SquillaciPR23}~\cite{SquillaciPR23}, \href{../works/KucukY19.pdf}{KucukY19}~\cite{KucukY19}, \href{../works/VerfaillieL01.pdf}{VerfaillieL01}~\cite{VerfaillieL01} & \href{../works/GlobusCLP04.pdf}{GlobusCLP04}~\cite{GlobusCLP04}, \href{../works/BensanaLV99.pdf}{BensanaLV99}~\cite{BensanaLV99} & \href{../works/HebrardHJMPV16.pdf}{HebrardHJMPV16}~\cite{HebrardHJMPV16}, \href{../works/Maillard15.pdf}{Maillard15}~\cite{Maillard15}, \href{../works/PraletLJ15.pdf}{PraletLJ15}~\cite{PraletLJ15}, \href{../works/SimoninAHL15.pdf}{SimoninAHL15}~\cite{SimoninAHL15}, \href{../works/KelarevaTK13.pdf}{KelarevaTK13}~\cite{KelarevaTK13}, \href{../works/OddiPCC03.pdf}{OddiPCC03}~\cite{OddiPCC03}\\
\index{earth orbit}\index{ApplicationAreas!earth orbit}earth orbit &  1.00 &  &  & \href{../works/SquillaciPR23.pdf}{SquillaciPR23}~\cite{SquillaciPR23}\\
\index{electroplating}\index{ApplicationAreas!electroplating}electroplating &  1.00 &  & \href{../works/RodosekW98.pdf}{RodosekW98}~\cite{RodosekW98} & \href{../works/Fatemi-AnarakiTFV23.pdf}{Fatemi-AnarakiTFV23}~\cite{Fatemi-AnarakiTFV23}, \href{../works/EfthymiouY23.pdf}{EfthymiouY23}~\cite{EfthymiouY23}, \href{../works/WallaceY20.pdf}{WallaceY20}~\cite{WallaceY20}, \href{../works/ArtiguesLH13.pdf}{ArtiguesLH13}~\cite{ArtiguesLH13}, \href{../works/NovasH12.pdf}{NovasH12}~\cite{NovasH12}\\
\index{emergency service}\index{ApplicationAreas!emergency service}emergency service &  1.00 &  & \href{../works/EvenSH15a.pdf}{EvenSH15a}~\cite{EvenSH15a}, \href{../works/TopalogluO11.pdf}{TopalogluO11}~\cite{TopalogluO11} & \href{../works/ForbesHJST24.pdf}{ForbesHJST24}~\cite{ForbesHJST24}, \href{../works/EvenSH15.pdf}{EvenSH15}~\cite{EvenSH15}, \href{../works/SakkoutW00.pdf}{SakkoutW00}~\cite{SakkoutW00}\\
\index{energy-price}\index{ApplicationAreas!energy-price}energy-price &  1.00 & \href{../works/KinsellaS0OS16.pdf}{KinsellaS0OS16}~\cite{KinsellaS0OS16}, \href{../works/GrimesIOS14.pdf}{GrimesIOS14}~\cite{GrimesIOS14}, \href{../works/IfrimOS12.pdf}{IfrimOS12}~\cite{IfrimOS12} & \href{../works/HurleyOS16.pdf}{HurleyOS16}~\cite{HurleyOS16}, \href{../works/Froger16.pdf}{Froger16}~\cite{Froger16} & \href{../works/LuZZYW24.pdf}{LuZZYW24}~\cite{LuZZYW24}, \href{../works/PrataAN23.pdf}{PrataAN23}~\cite{PrataAN23}, \href{../works/EscobetPQPRA19.pdf}{EscobetPQPRA19}~\cite{EscobetPQPRA19}, \href{../works/He0GLW18.pdf}{He0GLW18}~\cite{He0GLW18}, \href{../works/BenediktSMVH18.pdf}{BenediktSMVH18}~\cite{BenediktSMVH18}, \href{../works/LimHTB16.pdf}{LimHTB16}~\cite{LimHTB16}\\
\index{evacuation}\index{ApplicationAreas!evacuation}evacuation &  1.00 & \href{../works/ArtiguesHQT21.pdf}{ArtiguesHQT21}~\cite{ArtiguesHQT21}, \href{../works/ZarandiASC20.pdf}{ZarandiASC20}~\cite{ZarandiASC20}, \href{../works/YangSS19.pdf}{YangSS19}~\cite{YangSS19}, \href{../works/EvenSH15.pdf}{EvenSH15}~\cite{EvenSH15}, \href{../works/EvenSH15a.pdf}{EvenSH15a}~\cite{EvenSH15a} &  & \href{../works/AgussurjaKL18.pdf}{AgussurjaKL18}~\cite{AgussurjaKL18}, \href{../works/Beck99.pdf}{Beck99}~\cite{Beck99}\\
\index{farming}\index{ApplicationAreas!farming}farming &  1.00 &  &  & \href{../works/WinterMMW22.pdf}{WinterMMW22}~\cite{WinterMMW22}, \href{../works/Astrand0F21.pdf}{Astrand0F21}~\cite{Astrand0F21}\\
\index{forestry}\index{ApplicationAreas!forestry}forestry &  1.00 & \href{../works/HachemiGR11.pdf}{HachemiGR11}~\cite{HachemiGR11} &  & \href{../works/Astrand0F21.pdf}{Astrand0F21}~\cite{Astrand0F21}\\
\index{high performance computing}\index{ApplicationAreas!high performance computing}high performance computing &  1.00 & \href{../works/BorghesiBLMB18.pdf}{BorghesiBLMB18}~\cite{BorghesiBLMB18} & \href{../works/GalleguillosKSB19.pdf}{GalleguillosKSB19}~\cite{GalleguillosKSB19} & \href{../works/abs-2305-19888.pdf}{abs-2305-19888}~\cite{abs-2305-19888}, \href{../works/HeinzNVH22.pdf}{HeinzNVH22}~\cite{HeinzNVH22}, \href{../works/Zahout21.pdf}{Zahout21}~\cite{Zahout21}, \href{../works/Lunardi20.pdf}{Lunardi20}~\cite{Lunardi20}, \href{../works/LunardiBLRV20.pdf}{LunardiBLRV20}~\cite{LunardiBLRV20}, \href{../works/TranPZLDB18.pdf}{TranPZLDB18}~\cite{TranPZLDB18}, \href{../works/RiahiNS018.pdf}{RiahiNS018}~\cite{RiahiNS018}, \href{../works/HurleyOS16.pdf}{HurleyOS16}~\cite{HurleyOS16}, \href{../works/BartoliniBBLM14.pdf}{BartoliniBBLM14}~\cite{BartoliniBBLM14}\\
\index{high school timetabling}\index{ApplicationAreas!high school timetabling}high school timetabling &  1.00 & \href{../works/DemirovicS18.pdf}{DemirovicS18}~\cite{DemirovicS18}, \href{../works/YoshikawaKNW94.pdf}{YoshikawaKNW94}~\cite{YoshikawaKNW94} &  & \href{../works/Lemos21.pdf}{Lemos21}~\cite{Lemos21}, \href{../works/BofillGSV15.pdf}{BofillGSV15}~\cite{BofillGSV15}, \href{../works/KanetAG04.pdf}{KanetAG04}~\cite{KanetAG04}, \href{../works/ElkhyariGJ02a.pdf}{ElkhyariGJ02a}~\cite{ElkhyariGJ02a}, \href{../works/Schaerf97.pdf}{Schaerf97}~\cite{Schaerf97}\\
\index{hoist}\index{ApplicationAreas!hoist}hoist &  1.00 & \href{../works/EfthymiouY23.pdf}{EfthymiouY23}~\cite{EfthymiouY23}, \href{../works/WallaceY20.pdf}{WallaceY20}~\cite{WallaceY20}, \href{../works/RodosekWH99.pdf}{RodosekWH99}~\cite{RodosekWH99}, \href{../works/RodosekW98.pdf}{RodosekW98}~\cite{RodosekW98} & \href{../works/Fatemi-AnarakiTFV23.pdf}{Fatemi-AnarakiTFV23}~\cite{Fatemi-AnarakiTFV23}, \href{../works/NovasH12.pdf}{NovasH12}~\cite{NovasH12}, \href{../works/BonfiettiLBM11.pdf}{BonfiettiLBM11}~\cite{BonfiettiLBM11} & \href{../works/WessenCSFPM23.pdf}{WessenCSFPM23}~\cite{WessenCSFPM23}, \href{../works/AstrandJZ18.pdf}{AstrandJZ18}~\cite{AstrandJZ18}, \href{../works/BonfiettiLBM14.pdf}{BonfiettiLBM14}~\cite{BonfiettiLBM14}, \href{../works/UnsalO13.pdf}{UnsalO13}~\cite{UnsalO13}, \href{../works/ArtiguesLH13.pdf}{ArtiguesLH13}~\cite{ArtiguesLH13}, \href{../works/BonfiettiM12.pdf}{BonfiettiM12}~\cite{BonfiettiM12}, \href{../works/BonfiettiLBM12.pdf}{BonfiettiLBM12}~\cite{BonfiettiLBM12}, \href{../works/LombardiBMB11.pdf}{LombardiBMB11}~\cite{LombardiBMB11}, \href{../works/Wallace06.pdf}{Wallace06}~\cite{Wallace06}, \href{../works/BeckR03.pdf}{BeckR03}~\cite{BeckR03}, \href{../works/Baptiste02.pdf}{Baptiste02}~\cite{Baptiste02}, \href{../works/Refalo00.pdf}{Refalo00}~\cite{Refalo00}, \href{../works/HookerOTK00.pdf}{HookerOTK00}~\cite{HookerOTK00}, \href{../works/DraperJCJ99.pdf}{DraperJCJ99}~\cite{DraperJCJ99}, \href{../works/KorbaaYG99.pdf}{KorbaaYG99}~\cite{KorbaaYG99}, \href{../works/PapaB98.pdf}{PapaB98}~\cite{PapaB98}\\
\index{maintenance scheduling}\index{ApplicationAreas!maintenance scheduling}maintenance scheduling &  1.00 & \href{../works/PopovicCGNC22.pdf}{PopovicCGNC22}~\cite{PopovicCGNC22}, \href{../works/Froger16.pdf}{Froger16}~\cite{Froger16}, \href{../works/BajestaniB13.pdf}{BajestaniB13}~\cite{BajestaniB13}, \href{../works/Malapert11.pdf}{Malapert11}~\cite{Malapert11} & \href{../works/LuZZYW24.pdf}{LuZZYW24}~\cite{LuZZYW24}, \href{../works/PenzDN23.pdf}{PenzDN23}~\cite{PenzDN23}, \href{../works/AntunesABD20.pdf}{AntunesABD20}~\cite{AntunesABD20}, \href{../works/BajestaniB11.pdf}{BajestaniB11}~\cite{BajestaniB11}, \href{../works/Davenport10.pdf}{Davenport10}~\cite{Davenport10}, \href{../works/FrostD98.pdf}{FrostD98}~\cite{FrostD98} & \href{../works/BourreauGGLT22.pdf}{BourreauGGLT22}~\cite{BourreauGGLT22}, \href{../works/Godet21a.pdf}{Godet21a}~\cite{Godet21a}, \href{../works/ZarandiASC20.pdf}{ZarandiASC20}~\cite{ZarandiASC20}, \href{../works/Hooker19.pdf}{Hooker19}~\cite{Hooker19}, \href{../works/KonowalenkoMM19.pdf}{KonowalenkoMM19}~\cite{KonowalenkoMM19}, \href{../works/PourDERB18.pdf}{PourDERB18}~\cite{PourDERB18}, \href{../works/AntunesABD18.pdf}{AntunesABD18}~\cite{AntunesABD18}, \href{../works/Nattaf16.pdf}{Nattaf16}~\cite{Nattaf16}, \href{../works/BajestaniB15.pdf}{BajestaniB15}~\cite{BajestaniB15}, \href{../works/RoePS05.pdf}{RoePS05}~\cite{RoePS05}, \href{../works/Simonis99.pdf}{Simonis99}~\cite{Simonis99}, \href{../works/SimonisC95.pdf}{SimonisC95}~\cite{SimonisC95}, \href{../works/Puget95.pdf}{Puget95}~\cite{Puget95}\\
\index{medical}\index{ApplicationAreas!medical}medical &  1.00 & \href{../works/FrimodigECM23.pdf}{FrimodigECM23}~\cite{FrimodigECM23}, \href{../works/ShinBBHO18.pdf}{ShinBBHO18}~\cite{ShinBBHO18}, \href{../works/Dejemeppe16.pdf}{Dejemeppe16}~\cite{Dejemeppe16}, \href{../works/WangMD15.pdf}{WangMD15}~\cite{WangMD15}, \href{../works/ZhaoL14.pdf}{ZhaoL14}~\cite{ZhaoL14}, \href{../works/Wolf11.pdf}{Wolf11}~\cite{Wolf11}, \href{../works/TopalogluO11.pdf}{TopalogluO11}~\cite{TopalogluO11}, \href{../works/WeilHFP95.pdf}{WeilHFP95}~\cite{WeilHFP95} & \href{../works/GuoZ23.pdf}{GuoZ23}~\cite{GuoZ23}, \href{../works/ZarandiASC20.pdf}{ZarandiASC20}~\cite{ZarandiASC20}, \href{../works/HechingH16.pdf}{HechingH16}~\cite{HechingH16}, \href{../works/RiiseML16.pdf}{RiiseML16}~\cite{RiiseML16}, \href{../works/DejemeppeD14.pdf}{DejemeppeD14}~\cite{DejemeppeD14}, \href{../works/MeskensDL13.pdf}{MeskensDL13}~\cite{MeskensDL13}, \href{../works/RendlPHPR12.pdf}{RendlPHPR12}~\cite{RendlPHPR12} & \href{../works/ShaikhK23.pdf}{ShaikhK23}~\cite{ShaikhK23}, \href{../works/AbreuPNF23.pdf}{AbreuPNF23}~\cite{AbreuPNF23}, \href{../works/AbreuNP23.pdf}{AbreuNP23}~\cite{AbreuNP23}, \href{../works/IsikYA23.pdf}{IsikYA23}~\cite{IsikYA23}, \href{../works/NaderiBZR23.pdf}{NaderiBZR23}~\cite{NaderiBZR23}, \href{../works/AkramNHRSA23.pdf}{AkramNHRSA23}~\cite{AkramNHRSA23}, \href{../works/GhandehariK22.pdf}{GhandehariK22}~\cite{GhandehariK22}, \href{../works/YunusogluY22.pdf}{YunusogluY22}~\cite{YunusogluY22}, \href{../works/FarsiTM22.pdf}{FarsiTM22}~\cite{FarsiTM22}, \href{../works/AbreuN22.pdf}{AbreuN22}~\cite{AbreuN22}, \href{../works/GeibingerKKMMW21.pdf}{GeibingerKKMMW21}~\cite{GeibingerKKMMW21}, \href{../works/Bedhief21.pdf}{Bedhief21}~\cite{Bedhief21}, \href{../works/Edis21.pdf}{Edis21}~\cite{Edis21}, \href{../works/AbreuAPNM21.pdf}{AbreuAPNM21}~\cite{AbreuAPNM21}, \href{../works/Lemos21.pdf}{Lemos21}~\cite{Lemos21}, \href{../works/ThomasKS20.pdf}{ThomasKS20}~\cite{ThomasKS20}, \href{../works/FallahiAC20.pdf}{FallahiAC20}~\cite{FallahiAC20}, \href{../works/FrimodigS19.pdf}{FrimodigS19}~\cite{FrimodigS19}, \href{../works/abs-1902-01193.pdf}{abs-1902-01193}~\cite{abs-1902-01193}...\href{../works/BoothNB16.pdf}{BoothNB16}~\cite{BoothNB16}, \href{../works/DoulabiRP14.pdf}{DoulabiRP14}~\cite{DoulabiRP14}, \href{../works/BonfiettiLBM14.pdf}{BonfiettiLBM14}~\cite{BonfiettiLBM14}, \href{../works/LombardiMB13.pdf}{LombardiMB13}~\cite{LombardiMB13}, \href{../works/MeskensDHG11.pdf}{MeskensDHG11}~\cite{MeskensDHG11}, \href{../works/Salido10.pdf}{Salido10}~\cite{Salido10}, \href{../works/Lombardi10.pdf}{Lombardi10}~\cite{Lombardi10}, \href{../works/BeniniLMR08.pdf}{BeniniLMR08}~\cite{BeniniLMR08}, \href{../works/Simonis07.pdf}{Simonis07}~\cite{Simonis07}, \href{../works/WolfS05a.pdf}{WolfS05a}~\cite{WolfS05a} (Total: 44)\\
\index{meeting scheduling}\index{ApplicationAreas!meeting scheduling}meeting scheduling &  1.00 & \href{../works/BofillCGGPSV23.pdf}{BofillCGGPSV23}~\cite{BofillCGGPSV23}, \href{../works/GelainPRVW17.pdf}{GelainPRVW17}~\cite{GelainPRVW17}, \href{../works/LimHTB16.pdf}{LimHTB16}~\cite{LimHTB16}, \href{../works/LimBTBB15a.pdf}{LimBTBB15a}~\cite{LimBTBB15a}, \href{../works/LimBTBB15.pdf}{LimBTBB15}~\cite{LimBTBB15}, \href{../works/PesantRR15.pdf}{PesantRR15}~\cite{PesantRR15}, \href{../works/ZhuS02.pdf}{ZhuS02}~\cite{ZhuS02} & \href{../works/BofillEGPSV14.pdf}{BofillEGPSV14}~\cite{BofillEGPSV14} & \href{../works/Lemos21.pdf}{Lemos21}~\cite{Lemos21}, \href{../works/BofillGSV15.pdf}{BofillGSV15}~\cite{BofillGSV15}, \href{../works/MurphyMB15.pdf}{MurphyMB15}~\cite{MurphyMB15}, \href{../works/BartakSR10.pdf}{BartakSR10}~\cite{BartakSR10}, \href{../works/MoffittPP05.pdf}{MoffittPP05}~\cite{MoffittPP05}, \href{../works/FukunagaHFAMN02.pdf}{FukunagaHFAMN02}~\cite{FukunagaHFAMN02}\\
\index{music festival}\index{ApplicationAreas!music festival}music festival &  1.00 & \href{../works/CohenHB17.pdf}{CohenHB17}~\cite{CohenHB17} &  & \\
\index{nurse}\index{ApplicationAreas!nurse}nurse &  1.00 & \href{../works/NaderiBZR23.pdf}{NaderiBZR23}~\cite{NaderiBZR23}, \href{../works/GurPAE23.pdf}{GurPAE23}~\cite{GurPAE23}, \href{../works/FarsiTM22.pdf}{FarsiTM22}~\cite{FarsiTM22}, \href{../works/ZarandiASC20.pdf}{ZarandiASC20}~\cite{ZarandiASC20}, \href{../works/SenderovichBB19.pdf}{SenderovichBB19}~\cite{SenderovichBB19}, \href{../works/abs-1902-01193.pdf}{abs-1902-01193}~\cite{abs-1902-01193}, \href{../works/ShinBBHO18.pdf}{ShinBBHO18}~\cite{ShinBBHO18}, \href{../works/HoYCLLCLC18.pdf}{HoYCLLCLC18}~\cite{HoYCLLCLC18}, \href{../works/LuoVLBM16.pdf}{LuoVLBM16}~\cite{LuoVLBM16}, \href{../works/WangMD15.pdf}{WangMD15}~\cite{WangMD15}, \href{../works/MeskensDL13.pdf}{MeskensDL13}~\cite{MeskensDL13}, \href{../works/RendlPHPR12.pdf}{RendlPHPR12}~\cite{RendlPHPR12}, \href{../works/Menana11.pdf}{Menana11}~\cite{Menana11}, \href{../works/Wolf11.pdf}{Wolf11}~\cite{Wolf11}, \href{../works/MeskensDHG11.pdf}{MeskensDHG11}~\cite{MeskensDHG11}, \href{../works/Simonis07.pdf}{Simonis07}~\cite{Simonis07}, \href{../works/BourdaisGP03.pdf}{BourdaisGP03}~\cite{BourdaisGP03}, \href{../works/Mason01.pdf}{Mason01}~\cite{Mason01}, \href{../works/AbdennadherS99.pdf}{AbdennadherS99}~\cite{AbdennadherS99}, \href{../works/WeilHFP95.pdf}{WeilHFP95}~\cite{WeilHFP95} & \href{../works/OuelletQ22.pdf}{OuelletQ22}~\cite{OuelletQ22}, \href{../works/GeibingerMM21.pdf}{GeibingerMM21}~\cite{GeibingerMM21}, \href{../works/GeibingerKKMMW21.pdf}{GeibingerKKMMW21}~\cite{GeibingerKKMMW21}, \href{../works/YounespourAKE19.pdf}{YounespourAKE19}~\cite{YounespourAKE19}, \href{../works/FrohnerTR19.pdf}{FrohnerTR19}~\cite{FrohnerTR19}, \href{../works/RoshanaeiLAU17.pdf}{RoshanaeiLAU17}~\cite{RoshanaeiLAU17}, \href{../works/ZhaoL14.pdf}{ZhaoL14}~\cite{ZhaoL14} & \href{../works/abs-2312-13682.pdf}{abs-2312-13682}~\cite{abs-2312-13682}, \href{../works/PerezGSL23.pdf}{PerezGSL23}~\cite{PerezGSL23}, \href{../works/NaderiBZ23.pdf}{NaderiBZ23}~\cite{NaderiBZ23}, \href{../works/FrimodigECM23.pdf}{FrimodigECM23}~\cite{FrimodigECM23}, \href{../works/NaderiBZ22a.pdf}{NaderiBZ22a}~\cite{NaderiBZ22a}, \href{../works/NaderiBZ22.pdf}{NaderiBZ22}~\cite{NaderiBZ22}, \href{../works/BourreauGGLT22.pdf}{BourreauGGLT22}~\cite{BourreauGGLT22}, \href{../works/FallahiAC20.pdf}{FallahiAC20}~\cite{FallahiAC20}, \href{../works/RoshanaeiBAUB20.pdf}{RoshanaeiBAUB20}~\cite{RoshanaeiBAUB20}, \href{../works/FrimodigS19.pdf}{FrimodigS19}~\cite{FrimodigS19}, \href{../works/German18.pdf}{German18}~\cite{German18}, \href{../works/MusliuSS18.pdf}{MusliuSS18}~\cite{MusliuSS18}, \href{../works/GedikKEK18.pdf}{GedikKEK18}~\cite{GedikKEK18}, \href{../works/NishikawaSTT18a.pdf}{NishikawaSTT18a}~\cite{NishikawaSTT18a}, \href{../works/ErkingerM17.pdf}{ErkingerM17}~\cite{ErkingerM17}, \href{../works/GedikKBR17.pdf}{GedikKBR17}~\cite{GedikKBR17}, \href{../works/HookerH17.pdf}{HookerH17}~\cite{HookerH17}, \href{../works/RiiseML16.pdf}{RiiseML16}~\cite{RiiseML16}, \href{../works/Dejemeppe16.pdf}{Dejemeppe16}~\cite{Dejemeppe16}, \href{../works/DoulabiRP16.pdf}{DoulabiRP16}~\cite{DoulabiRP16}, \href{../works/DoulabiRP14.pdf}{DoulabiRP14}~\cite{DoulabiRP14}, \href{../works/TopalogluO11.pdf}{TopalogluO11}~\cite{TopalogluO11}, \href{../works/Simonis99.pdf}{Simonis99}~\cite{Simonis99}\\
\index{offshore}\index{ApplicationAreas!offshore}offshore &  1.00 &  & \href{../works/SubulanC22.pdf}{SubulanC22}~\cite{SubulanC22}, \href{../works/Froger16.pdf}{Froger16}~\cite{Froger16} & \href{../works/GokPTGO23.pdf}{GokPTGO23}~\cite{GokPTGO23}, \href{../works/BoudreaultSLQ22.pdf}{BoudreaultSLQ22}~\cite{BoudreaultSLQ22}, \href{../works/BlomPS16.pdf}{BlomPS16}~\cite{BlomPS16}, \href{../works/FrankDT16.pdf}{FrankDT16}~\cite{FrankDT16}, \href{../works/BlomBPS14.pdf}{BlomBPS14}~\cite{BlomBPS14}, \href{../works/Jans09.pdf}{Jans09}~\cite{Jans09}\\
\index{operating room}\index{ApplicationAreas!operating room}operating room &  1.00 & \href{../works/NaderiRR23.pdf}{NaderiRR23}~\cite{NaderiRR23}, \href{../works/GurPAE23.pdf}{GurPAE23}~\cite{GurPAE23}, \href{../works/NaderiBZ23.pdf}{NaderiBZ23}~\cite{NaderiBZ23}, \href{../works/NaderiBZR23.pdf}{NaderiBZR23}~\cite{NaderiBZR23}, \href{../works/FarsiTM22.pdf}{FarsiTM22}~\cite{FarsiTM22}, \href{../works/NaderiBZ22.pdf}{NaderiBZ22}~\cite{NaderiBZ22}, \href{../works/GhandehariK22.pdf}{GhandehariK22}~\cite{GhandehariK22}, \href{../works/RoshanaeiN21.pdf}{RoshanaeiN21}~\cite{RoshanaeiN21}, \href{../works/RoshanaeiBAUB20.pdf}{RoshanaeiBAUB20}~\cite{RoshanaeiBAUB20}, \href{../works/YounespourAKE19.pdf}{YounespourAKE19}~\cite{YounespourAKE19}, \href{../works/GurEA19.pdf}{GurEA19}~\cite{GurEA19}, \href{../works/RoshanaeiLAU17.pdf}{RoshanaeiLAU17}~\cite{RoshanaeiLAU17}, \href{../works/DoulabiRP16.pdf}{DoulabiRP16}~\cite{DoulabiRP16}, \href{../works/RiiseML16.pdf}{RiiseML16}~\cite{RiiseML16}, \href{../works/WangMD15.pdf}{WangMD15}~\cite{WangMD15}, \href{../works/ZhaoL14.pdf}{ZhaoL14}~\cite{ZhaoL14}, \href{../works/DoulabiRP14.pdf}{DoulabiRP14}~\cite{DoulabiRP14}, \href{../works/MeskensDL13.pdf}{MeskensDL13}~\cite{MeskensDL13}, \href{../works/Wolf11.pdf}{Wolf11}~\cite{Wolf11}, \href{../works/MeskensDHG11.pdf}{MeskensDHG11}~\cite{MeskensDHG11} & \href{../works/GuoZ23.pdf}{GuoZ23}~\cite{GuoZ23}, \href{../works/ElciOH22.pdf}{ElciOH22}~\cite{ElciOH22}, \href{../works/NaderiBZ22a.pdf}{NaderiBZ22a}~\cite{NaderiBZ22a}, \href{../works/ZarandiASC20.pdf}{ZarandiASC20}~\cite{ZarandiASC20}, \href{../works/Hooker19.pdf}{Hooker19}~\cite{Hooker19}, \href{../works/HookerH17.pdf}{HookerH17}~\cite{HookerH17} & \href{../works/ForbesHJST24.pdf}{ForbesHJST24}~\cite{ForbesHJST24}, \href{../works/WangB23.pdf}{WangB23}~\cite{WangB23}, \href{../works/JuvinHL23a.pdf}{JuvinHL23a}~\cite{JuvinHL23a}, \href{../works/Adelgren2023.pdf}{Adelgren2023}~\cite{Adelgren2023}, \href{../works/PerezGSL23.pdf}{PerezGSL23}~\cite{PerezGSL23}, \href{../works/abs-2312-13682.pdf}{abs-2312-13682}~\cite{abs-2312-13682}, \href{../works/FrimodigECM23.pdf}{FrimodigECM23}~\cite{FrimodigECM23}, \href{../works/GeibingerMM21.pdf}{GeibingerMM21}~\cite{GeibingerMM21}, \href{../works/FachiniA20.pdf}{FachiniA20}~\cite{FachiniA20}, \href{../works/MusliuSS18.pdf}{MusliuSS18}~\cite{MusliuSS18}, \href{../works/TanT18.pdf}{TanT18}~\cite{TanT18}, \href{../works/Wolf09.pdf}{Wolf09}~\cite{Wolf09}, \href{../works/CrawfordB94.pdf}{CrawfordB94}~\cite{CrawfordB94}\\
\index{oven scheduling}\index{ApplicationAreas!oven scheduling}oven scheduling &  1.00 & \href{../works/LacknerMMWW23.pdf}{LacknerMMWW23}~\cite{LacknerMMWW23}, \href{../works/LacknerMMWW21.pdf}{LacknerMMWW21}~\cite{LacknerMMWW21} &  & \href{../works/ColT22.pdf}{ColT22}~\cite{ColT22}\\
\index{patient}\index{ApplicationAreas!patient}patient &  1.00 & \href{../works/GurPAE23.pdf}{GurPAE23}~\cite{GurPAE23}, \href{../works/FrimodigECM23.pdf}{FrimodigECM23}~\cite{FrimodigECM23}, \href{../works/NaderiBZR23.pdf}{NaderiBZR23}~\cite{NaderiBZR23}, \href{../works/GhandehariK22.pdf}{GhandehariK22}~\cite{GhandehariK22}, \href{../works/FarsiTM22.pdf}{FarsiTM22}~\cite{FarsiTM22}, \href{../works/RoshanaeiN21.pdf}{RoshanaeiN21}~\cite{RoshanaeiN21}, \href{../works/ThomasKS20.pdf}{ThomasKS20}~\cite{ThomasKS20}, \href{../works/RoshanaeiBAUB20.pdf}{RoshanaeiBAUB20}~\cite{RoshanaeiBAUB20}, \href{../works/FrimodigS19.pdf}{FrimodigS19}~\cite{FrimodigS19}, \href{../works/GurEA19.pdf}{GurEA19}~\cite{GurEA19}, \href{../works/SenderovichBB19.pdf}{SenderovichBB19}~\cite{SenderovichBB19}, \href{../works/YounespourAKE19.pdf}{YounespourAKE19}~\cite{YounespourAKE19}, \href{../works/ShinBBHO18.pdf}{ShinBBHO18}~\cite{ShinBBHO18}, \href{../works/CappartTSR18.pdf}{CappartTSR18}~\cite{CappartTSR18}, \href{../works/RoshanaeiLAU17.pdf}{RoshanaeiLAU17}~\cite{RoshanaeiLAU17}, \href{../works/HechingH16.pdf}{HechingH16}~\cite{HechingH16}, \href{../works/DoulabiRP16.pdf}{DoulabiRP16}~\cite{DoulabiRP16}, \href{../works/Dejemeppe16.pdf}{Dejemeppe16}~\cite{Dejemeppe16}, \href{../works/RiiseML16.pdf}{RiiseML16}~\cite{RiiseML16}, \href{../works/WangMD15.pdf}{WangMD15}~\cite{WangMD15}, \href{../works/ZhaoL14.pdf}{ZhaoL14}~\cite{ZhaoL14}, \href{../works/DejemeppeD14.pdf}{DejemeppeD14}~\cite{DejemeppeD14}, \href{../works/MeskensDL13.pdf}{MeskensDL13}~\cite{MeskensDL13}, \href{../works/RendlPHPR12.pdf}{RendlPHPR12}~\cite{RendlPHPR12}, \href{../works/Wolf11.pdf}{Wolf11}~\cite{Wolf11}, \href{../works/TopalogluO11.pdf}{TopalogluO11}~\cite{TopalogluO11}, \href{../works/MeskensDHG11.pdf}{MeskensDHG11}~\cite{MeskensDHG11}, \href{../works/WeilHFP95.pdf}{WeilHFP95}~\cite{WeilHFP95} & \href{../works/NaderiBZ23.pdf}{NaderiBZ23}~\cite{NaderiBZ23}, \href{../works/GeibingerKKMMW21.pdf}{GeibingerKKMMW21}~\cite{GeibingerKKMMW21} & \href{../works/BonninMNE24.pdf}{BonninMNE24}~\cite{BonninMNE24}, \href{../works/ForbesHJST24.pdf}{ForbesHJST24}~\cite{ForbesHJST24}, \href{../works/GuoZ23.pdf}{GuoZ23}~\cite{GuoZ23}, \href{../works/AlfieriGPS23.pdf}{AlfieriGPS23}~\cite{AlfieriGPS23}, \href{../works/ElciOH22.pdf}{ElciOH22}~\cite{ElciOH22}, \href{../works/NaderiBZ22.pdf}{NaderiBZ22}~\cite{NaderiBZ22}, \href{../works/AbreuAPNM21.pdf}{AbreuAPNM21}~\cite{AbreuAPNM21}, \href{../works/CauwelaertDS20.pdf}{CauwelaertDS20}~\cite{CauwelaertDS20}, \href{../works/MurinR19.pdf}{MurinR19}~\cite{MurinR19}, \href{../works/Hooker19.pdf}{Hooker19}~\cite{Hooker19}, \href{../works/GombolayWS18.pdf}{GombolayWS18}~\cite{GombolayWS18}, \href{../works/HoYCLLCLC18.pdf}{HoYCLLCLC18}~\cite{HoYCLLCLC18}, \href{../works/TanT18.pdf}{TanT18}~\cite{TanT18}, \href{../works/LouieVNB14.pdf}{LouieVNB14}~\cite{LouieVNB14}, \href{../works/DoulabiRP14.pdf}{DoulabiRP14}~\cite{DoulabiRP14}, \href{../works/Clercq12.pdf}{Clercq12}~\cite{Clercq12}, \href{../works/Malapert11.pdf}{Malapert11}~\cite{Malapert11}, \href{../works/Salido10.pdf}{Salido10}~\cite{Salido10}, \href{../works/Wolf09.pdf}{Wolf09}~\cite{Wolf09}, \href{../works/Simonis07.pdf}{Simonis07}~\cite{Simonis07}, \href{../works/KanetAG04.pdf}{KanetAG04}~\cite{KanetAG04}, \href{../works/BourdaisGP03.pdf}{BourdaisGP03}~\cite{BourdaisGP03}\\
\index{perfect-square}\index{ApplicationAreas!perfect-square}perfect-square &  1.00 & \href{../works/BeldiceanuCDP11.pdf}{BeldiceanuCDP11}~\cite{BeldiceanuCDP11}, \href{../works/BeldiceanuCP08.pdf}{BeldiceanuCP08}~\cite{BeldiceanuCP08}, \href{../works/AggounB93.pdf}{AggounB93}~\cite{AggounB93} &  & \\
\index{physician}\index{ApplicationAreas!physician}physician &  1.00 & \href{../works/GeibingerKKMMW21.pdf}{GeibingerKKMMW21}~\cite{GeibingerKKMMW21}, \href{../works/ShinBBHO18.pdf}{ShinBBHO18}~\cite{ShinBBHO18} & \href{../works/Dejemeppe16.pdf}{Dejemeppe16}~\cite{Dejemeppe16}, \href{../works/BourdaisGP03.pdf}{BourdaisGP03}~\cite{BourdaisGP03} & \href{../works/GuoZ23.pdf}{GuoZ23}~\cite{GuoZ23}, \href{../works/GurPAE23.pdf}{GurPAE23}~\cite{GurPAE23}, \href{../works/FrimodigECM23.pdf}{FrimodigECM23}~\cite{FrimodigECM23}, \href{../works/FarsiTM22.pdf}{FarsiTM22}~\cite{FarsiTM22}, \href{../works/FrimodigS19.pdf}{FrimodigS19}~\cite{FrimodigS19}, \href{../works/HookerH17.pdf}{HookerH17}~\cite{HookerH17}, \href{../works/WangMD15.pdf}{WangMD15}~\cite{WangMD15}, \href{../works/Wolf11.pdf}{Wolf11}~\cite{Wolf11}, \href{../works/TopalogluO11.pdf}{TopalogluO11}~\cite{TopalogluO11}\\
\index{pipeline}\index{ApplicationAreas!pipeline}pipeline &  1.00 & \href{../works/KonowalenkoMM19.pdf}{KonowalenkoMM19}~\cite{KonowalenkoMM19}, \href{../works/HarjunkoskiMBC14.pdf}{HarjunkoskiMBC14}~\cite{HarjunkoskiMBC14}, \href{../works/BegB13.pdf}{BegB13}~\cite{BegB13}, \href{../works/Lombardi10.pdf}{Lombardi10}~\cite{Lombardi10}, \href{../works/LopesCSM10.pdf}{LopesCSM10}~\cite{LopesCSM10}, \href{../works/RuggieroBBMA09.pdf}{RuggieroBBMA09}~\cite{RuggieroBBMA09}, \href{../works/MouraSCL08a.pdf}{MouraSCL08a}~\cite{MouraSCL08a}, \href{../works/MouraSCL08.pdf}{MouraSCL08}~\cite{MouraSCL08}, \href{../works/BeniniLMR08.pdf}{BeniniLMR08}~\cite{BeniniLMR08}, \href{../works/Malik08.pdf}{Malik08}~\cite{Malik08}, \href{../works/BeniniBGM05.pdf}{BeniniBGM05}~\cite{BeniniBGM05}, \href{../works/Kuchcinski03.pdf}{Kuchcinski03}~\cite{Kuchcinski03}, \href{../works/ErtlK91.pdf}{ErtlK91}~\cite{ErtlK91} & \href{../works/ZouZ20.pdf}{ZouZ20}~\cite{ZouZ20}, \href{../works/TangLWSK18.pdf}{TangLWSK18}~\cite{TangLWSK18}, \href{../works/LombardiMRB10.pdf}{LombardiMRB10}~\cite{LombardiMRB10}, \href{../works/MalikMB08.pdf}{MalikMB08}~\cite{MalikMB08}, \href{../works/BeniniBGM06.pdf}{BeniniBGM06}~\cite{BeniniBGM06}, \href{../works/BeniniBGM05a.pdf}{BeniniBGM05a}~\cite{BeniniBGM05a}, \href{../works/WolinskiKG04.pdf}{WolinskiKG04}~\cite{WolinskiKG04}, \href{../works/BeldiceanuC94.pdf}{BeldiceanuC94}~\cite{BeldiceanuC94} & \href{../works/EfthymiouY23.pdf}{EfthymiouY23}~\cite{EfthymiouY23}, \href{../works/Adelgren2023.pdf}{Adelgren2023}~\cite{Adelgren2023}, \href{../works/PopovicCGNC22.pdf}{PopovicCGNC22}~\cite{PopovicCGNC22}, \href{../works/KotaryFH22.pdf}{KotaryFH22}~\cite{KotaryFH22}, \href{../works/EmdeZD22.pdf}{EmdeZD22}~\cite{EmdeZD22}, \href{../works/NaqviAIAAA22.pdf}{NaqviAIAAA22}~\cite{NaqviAIAAA22}, \href{../works/HanenKP21.pdf}{HanenKP21}~\cite{HanenKP21}, \href{../works/NishikawaSTT19.pdf}{NishikawaSTT19}~\cite{NishikawaSTT19}, \href{../works/NishikawaSTT18a.pdf}{NishikawaSTT18a}~\cite{NishikawaSTT18a}, \href{../works/LaborieRSV18.pdf}{LaborieRSV18}~\cite{LaborieRSV18}, \href{../works/NishikawaSTT18.pdf}{NishikawaSTT18}~\cite{NishikawaSTT18}, \href{../works/EmeretlisTAV17.pdf}{EmeretlisTAV17}~\cite{EmeretlisTAV17}, \href{../works/Bonfietti16.pdf}{Bonfietti16}~\cite{Bonfietti16}, \href{../works/KinsellaS0OS16.pdf}{KinsellaS0OS16}~\cite{KinsellaS0OS16}, \href{../works/BlomPS16.pdf}{BlomPS16}~\cite{BlomPS16}, \href{../works/GilesH16.pdf}{GilesH16}~\cite{GilesH16}, \href{../works/GoelSHFS15.pdf}{GoelSHFS15}~\cite{GoelSHFS15}, \href{../works/SimoninAHL15.pdf}{SimoninAHL15}~\cite{SimoninAHL15}, \href{../works/BonfiettiLBM14.pdf}{BonfiettiLBM14}~\cite{BonfiettiLBM14}...\href{../works/RenT09.pdf}{RenT09}~\cite{RenT09}, \href{../works/BarlattCG08.pdf}{BarlattCG08}~\cite{BarlattCG08}, \href{../works/KuchcinskiW03.pdf}{KuchcinskiW03}~\cite{KuchcinskiW03}, \href{../works/Wolf03.pdf}{Wolf03}~\cite{Wolf03}, \href{../works/Simonis99.pdf}{Simonis99}~\cite{Simonis99}, \href{../works/DraperJCJ99.pdf}{DraperJCJ99}~\cite{DraperJCJ99}, \href{../works/RodosekWH99.pdf}{RodosekWH99}~\cite{RodosekWH99}, \href{../works/GruianK98.pdf}{GruianK98}~\cite{GruianK98}, \href{../works/Darby-DowmanLMZ97.pdf}{Darby-DowmanLMZ97}~\cite{Darby-DowmanLMZ97}, \href{../works/SimonisC95.pdf}{SimonisC95}~\cite{SimonisC95} (Total: 35)\\
\index{radiation therapy}\index{ApplicationAreas!radiation therapy}radiation therapy &  1.00 & \href{../works/FrimodigECM23.pdf}{FrimodigECM23}~\cite{FrimodigECM23}, \href{../works/FrimodigS19.pdf}{FrimodigS19}~\cite{FrimodigS19} &  & \href{../works/HookerH17.pdf}{HookerH17}~\cite{HookerH17}\\
\index{railway}\index{ApplicationAreas!railway}railway &  1.00 & \href{../works/MarliereSPR23.pdf}{MarliereSPR23}~\cite{MarliereSPR23}, \href{../works/SvancaraB22.pdf}{SvancaraB22}~\cite{SvancaraB22}, \href{../works/Lemos21.pdf}{Lemos21}~\cite{Lemos21}, \href{../works/PourDERB18.pdf}{PourDERB18}~\cite{PourDERB18}, \href{../works/CappartS17.pdf}{CappartS17}~\cite{CappartS17}, \href{../works/ChunS14.pdf}{ChunS14}~\cite{ChunS14}, \href{../works/AronssonBK09.pdf}{AronssonBK09}~\cite{AronssonBK09}, \href{../works/Acuna-AgostMFG09.pdf}{Acuna-AgostMFG09}~\cite{Acuna-AgostMFG09}, \href{../works/RodriguezS09.pdf}{RodriguezS09}~\cite{RodriguezS09}, \href{../works/Rodriguez07.pdf}{Rodriguez07}~\cite{Rodriguez07}, \href{../works/Rodriguez07b.pdf}{Rodriguez07b}~\cite{Rodriguez07b}, \href{../works/Geske05.pdf}{Geske05}~\cite{Geske05}, \href{../works/RodriguezDG02.pdf}{RodriguezDG02}~\cite{RodriguezDG02}, \href{../works/MartinPY01.pdf}{MartinPY01}~\cite{MartinPY01}, \href{../works/MorgadoM97.pdf}{MorgadoM97}~\cite{MorgadoM97}, \href{../works/LammaMM97.pdf}{LammaMM97}~\cite{LammaMM97} & \href{../works/ZarandiASC20.pdf}{ZarandiASC20}~\cite{ZarandiASC20}, \href{../works/LaborieRSV18.pdf}{LaborieRSV18}~\cite{LaborieRSV18}, \href{../works/TangLWSK18.pdf}{TangLWSK18}~\cite{TangLWSK18}, \href{../works/Mason01.pdf}{Mason01}~\cite{Mason01}, \href{../works/BrusoniCLMMT96.pdf}{BrusoniCLMMT96}~\cite{BrusoniCLMMT96} & \href{../works/LuZZYW24.pdf}{LuZZYW24}~\cite{LuZZYW24}, \href{../works/GuoZ23.pdf}{GuoZ23}~\cite{GuoZ23}, \href{../works/LuoB22.pdf}{LuoB22}~\cite{LuoB22}, \href{../works/Godet21a.pdf}{Godet21a}~\cite{Godet21a}, \href{../works/BogaerdtW19.pdf}{BogaerdtW19}~\cite{BogaerdtW19}, \href{../works/Hooker19.pdf}{Hooker19}~\cite{Hooker19}, \href{../works/BajestaniB15.pdf}{BajestaniB15}~\cite{BajestaniB15}, \href{../works/ZhouGL15.pdf}{ZhouGL15}~\cite{ZhouGL15}, \href{../works/ZhaoL14.pdf}{ZhaoL14}~\cite{ZhaoL14}, \href{../works/BajestaniB13.pdf}{BajestaniB13}~\cite{BajestaniB13}, \href{../works/BajestaniB11.pdf}{BajestaniB11}~\cite{BajestaniB11}, \href{../works/WuBB09.pdf}{WuBB09}~\cite{WuBB09}, \href{../works/Gronkvist06.pdf}{Gronkvist06}~\cite{Gronkvist06}, \href{../works/AbrilSB05.pdf}{AbrilSB05}~\cite{AbrilSB05}, \href{../works/WolfS05a.pdf}{WolfS05a}~\cite{WolfS05a}, \href{../works/Wallace96.pdf}{Wallace96}~\cite{Wallace96}\\
\index{real-time pricing}\index{ApplicationAreas!real-time pricing}real-time pricing &  1.00 &  & \href{../works/He0GLW18.pdf}{He0GLW18}~\cite{He0GLW18}, \href{../works/GrimesIOS14.pdf}{GrimesIOS14}~\cite{GrimesIOS14} & \href{../works/LimHTB16.pdf}{LimHTB16}~\cite{LimHTB16}\\
\index{rectangle-packing}\index{ApplicationAreas!rectangle-packing}rectangle-packing &  1.00 & \href{../works/YangSS19.pdf}{YangSS19}~\cite{YangSS19}, \href{../works/AggounB93.pdf}{AggounB93}~\cite{AggounB93} & \href{../works/LuoB22.pdf}{LuoB22}~\cite{LuoB22}, \href{../works/Malapert11.pdf}{Malapert11}~\cite{Malapert11} & \href{../works/MossigeGSMC17.pdf}{MossigeGSMC17}~\cite{MossigeGSMC17}, \href{../works/DoulabiRP16.pdf}{DoulabiRP16}~\cite{DoulabiRP16}, \href{../works/VilimLS15.pdf}{VilimLS15}~\cite{VilimLS15}, \href{../works/Siala15a.pdf}{Siala15a}~\cite{Siala15a}, \href{../works/Siala15.pdf}{Siala15}~\cite{Siala15}, \href{../works/LozanoCDS12.pdf}{LozanoCDS12}~\cite{LozanoCDS12}, \href{../works/BeldiceanuCDP11.pdf}{BeldiceanuCDP11}~\cite{BeldiceanuCDP11}, \href{../works/Schutt11.pdf}{Schutt11}~\cite{Schutt11}, \href{../works/SchuttW10.pdf}{SchuttW10}~\cite{SchuttW10}, \href{../works/BeldiceanuCP08.pdf}{BeldiceanuCP08}~\cite{BeldiceanuCP08}\\
\index{robot}\index{ApplicationAreas!robot}robot &  1.00 & \href{../works/Fatemi-AnarakiTFV23.pdf}{Fatemi-AnarakiTFV23}~\cite{Fatemi-AnarakiTFV23}, \href{../works/IsikYA23.pdf}{IsikYA23}~\cite{IsikYA23}, \href{../works/WessenCSFPM23.pdf}{WessenCSFPM23}~\cite{WessenCSFPM23}, \href{../works/LiFJZLL22.pdf}{LiFJZLL22}~\cite{LiFJZLL22}, \href{../works/ArmstrongGOS21.pdf}{ArmstrongGOS21}~\cite{ArmstrongGOS21}, \href{../works/Edis21.pdf}{Edis21}~\cite{Edis21}, \href{../works/HamP21.pdf}{HamP21}~\cite{HamP21}, \href{../works/Astrand21.pdf}{Astrand21}~\cite{Astrand21}, \href{../works/KoehlerBFFHPSSS21.pdf}{KoehlerBFFHPSSS21}~\cite{KoehlerBFFHPSSS21}, \href{../works/ZarandiASC20.pdf}{ZarandiASC20}~\cite{ZarandiASC20}, \href{../works/MokhtarzadehTNF20.pdf}{MokhtarzadehTNF20}~\cite{MokhtarzadehTNF20}, \href{../works/Lunardi20.pdf}{Lunardi20}~\cite{Lunardi20}, \href{../works/WessenCS20.pdf}{WessenCS20}~\cite{WessenCS20}, \href{../works/Ham20a.pdf}{Ham20a}~\cite{Ham20a}, \href{../works/abs-1901-07914.pdf}{abs-1901-07914}~\cite{abs-1901-07914}, \href{../works/MurinR19.pdf}{MurinR19}~\cite{MurinR19}, \href{../works/BehrensLM19.pdf}{BehrensLM19}~\cite{BehrensLM19}, \href{../works/PachecoPR19.pdf}{PachecoPR19}~\cite{PachecoPR19}, \href{../works/GombolayWS18.pdf}{GombolayWS18}~\cite{GombolayWS18}...\href{../works/BoothNB16.pdf}{BoothNB16}~\cite{BoothNB16}, \href{../works/LouieVNB14.pdf}{LouieVNB14}~\cite{LouieVNB14}, \href{../works/NovasH14.pdf}{NovasH14}~\cite{NovasH14}, \href{../works/NovasH12.pdf}{NovasH12}~\cite{NovasH12}, \href{../works/ZeballosCM10.pdf}{ZeballosCM10}~\cite{ZeballosCM10}, \href{../works/Zeballos10.pdf}{Zeballos10}~\cite{Zeballos10}, \href{../works/BartakSR10.pdf}{BartakSR10}~\cite{BartakSR10}, \href{../works/BidotVLB09.pdf}{BidotVLB09}~\cite{BidotVLB09}, \href{../works/ValleMGT03.pdf}{ValleMGT03}~\cite{ValleMGT03}, \href{../works/BeckF98.pdf}{BeckF98}~\cite{BeckF98} (Total: 35) & \href{../works/PrataAN23.pdf}{PrataAN23}~\cite{PrataAN23}, \href{../works/AlakaP23.pdf}{AlakaP23}~\cite{AlakaP23}, \href{../works/Mehdizadeh-Somarin23.pdf}{Mehdizadeh-Somarin23}~\cite{Mehdizadeh-Somarin23}, \href{../works/CzerniachowskaWZ23.pdf}{CzerniachowskaWZ23}~\cite{CzerniachowskaWZ23}, \href{../works/ZhuSZW23.pdf}{ZhuSZW23}~\cite{ZhuSZW23}, \href{../works/TouatBT22.pdf}{TouatBT22}~\cite{TouatBT22}, \href{../works/OujanaAYB22.pdf}{OujanaAYB22}~\cite{OujanaAYB22}, \href{../works/YunusogluY22.pdf}{YunusogluY22}~\cite{YunusogluY22}, \href{../works/NaderiBZ22a.pdf}{NaderiBZ22a}~\cite{NaderiBZ22a}, \href{../works/Astrand0F21.pdf}{Astrand0F21}~\cite{Astrand0F21}, \href{../works/FachiniA20.pdf}{FachiniA20}~\cite{FachiniA20}, \href{../works/WallaceY20.pdf}{WallaceY20}~\cite{WallaceY20}, \href{../works/NishikawaSTT19.pdf}{NishikawaSTT19}~\cite{NishikawaSTT19}, \href{../works/WikarekS19.pdf}{WikarekS19}~\cite{WikarekS19}, \href{../works/NishikawaSTT18a.pdf}{NishikawaSTT18a}~\cite{NishikawaSTT18a}, \href{../works/NishikawaSTT18.pdf}{NishikawaSTT18}~\cite{NishikawaSTT18}, \href{../works/Dejemeppe16.pdf}{Dejemeppe16}~\cite{Dejemeppe16}, \href{../works/TopalogluSS12.pdf}{TopalogluSS12}~\cite{TopalogluSS12}, \href{../works/DoRZ08.pdf}{DoRZ08}~\cite{DoRZ08}, \href{../works/BidotVLB07.pdf}{BidotVLB07}~\cite{BidotVLB07}, \href{../works/PolicellaWSO05.pdf}{PolicellaWSO05}~\cite{PolicellaWSO05}, \href{../works/VanczaM01.pdf}{VanczaM01}~\cite{VanczaM01}, \href{../works/BeckF00.pdf}{BeckF00}~\cite{BeckF00}, \href{../works/CestaOF99.pdf}{CestaOF99}~\cite{CestaOF99}, \href{../works/Beck99.pdf}{Beck99}~\cite{Beck99}, \href{../works/BeckDDF98.pdf}{BeckDDF98}~\cite{BeckDDF98}, \href{../works/SmithC93.pdf}{SmithC93}~\cite{SmithC93}, \href{../works/MintonJPL92.pdf}{MintonJPL92}~\cite{MintonJPL92}, \href{../works/FoxS90.pdf}{FoxS90}~\cite{FoxS90} & \href{../works/MontemanniD23.pdf}{MontemanniD23}~\cite{MontemanniD23}, \href{../works/MarliereSPR23.pdf}{MarliereSPR23}~\cite{MarliereSPR23}, \href{../works/abs-2305-19888.pdf}{abs-2305-19888}~\cite{abs-2305-19888}, \href{../works/IklassovMR023.pdf}{IklassovMR023}~\cite{IklassovMR023}, \href{../works/AbreuPNF23.pdf}{AbreuPNF23}~\cite{AbreuPNF23}, \href{../works/GeitzGSSW22.pdf}{GeitzGSSW22}~\cite{GeitzGSSW22}, \href{../works/CilKLO22.pdf}{CilKLO22}~\cite{CilKLO22}, \href{../works/MullerMKP22.pdf}{MullerMKP22}~\cite{MullerMKP22}, \href{../works/ColT22.pdf}{ColT22}~\cite{ColT22}, \href{../works/YuraszeckMPV22.pdf}{YuraszeckMPV22}~\cite{YuraszeckMPV22}, \href{../works/HeinzNVH22.pdf}{HeinzNVH22}~\cite{HeinzNVH22}, \href{../works/FarsiTM22.pdf}{FarsiTM22}~\cite{FarsiTM22}, \href{../works/ZhangYW21.pdf}{ZhangYW21}~\cite{ZhangYW21}, \href{../works/Godet21a.pdf}{Godet21a}~\cite{Godet21a}, \href{../works/Bedhief21.pdf}{Bedhief21}~\cite{Bedhief21}, \href{../works/Groleaz21.pdf}{Groleaz21}~\cite{Groleaz21}, \href{../works/VlkHT21.pdf}{VlkHT21}~\cite{VlkHT21}, \href{../works/HamPK21.pdf}{HamPK21}~\cite{HamPK21}, \href{../works/BarzegaranZP20.pdf}{BarzegaranZP20}~\cite{BarzegaranZP20}...\href{../works/BeckDF97.pdf}{BeckDF97}~\cite{BeckDF97}, \href{../works/BeckDSF97.pdf}{BeckDSF97}~\cite{BeckDSF97}, \href{../works/BeckDSF97a.pdf}{BeckDSF97a}~\cite{BeckDSF97a}, \href{../works/Wallace96.pdf}{Wallace96}~\cite{Wallace96}, \href{../works/BlazewiczDP96.pdf}{BlazewiczDP96}~\cite{BlazewiczDP96}, \href{../works/Nuijten94.pdf}{Nuijten94}~\cite{Nuijten94}, \href{../works/Muscettola94.pdf}{Muscettola94}~\cite{Muscettola94}, \href{../works/MintonJPL90.pdf}{MintonJPL90}~\cite{MintonJPL90}, \href{../works/Prosser89.pdf}{Prosser89}~\cite{Prosser89}, \href{../works/Davis87.pdf}{Davis87}~\cite{Davis87} (Total: 104)\\
\index{round-robin}\index{ApplicationAreas!round-robin}round-robin &  1.00 & \href{../works/NaqviAIAAA22.pdf}{NaqviAIAAA22}~\cite{NaqviAIAAA22}, \href{../works/BulckG22.pdf}{BulckG22}~\cite{BulckG22}, \href{../works/CarlssonJL17.pdf}{CarlssonJL17}~\cite{CarlssonJL17}, \href{../works/TerekhovTDB14.pdf}{TerekhovTDB14}~\cite{TerekhovTDB14}, \href{../works/LarsonJC14.pdf}{LarsonJC14}~\cite{LarsonJC14}, \href{../works/TranTDB13.pdf}{TranTDB13}~\cite{TranTDB13}, \href{../works/SuCC13.pdf}{SuCC13}~\cite{SuCC13}, \href{../works/Ribeiro12.pdf}{Ribeiro12}~\cite{Ribeiro12}, \href{../works/ZengM12.pdf}{ZengM12}~\cite{ZengM12}, \href{../works/KendallKRU10.pdf}{KendallKRU10}~\cite{KendallKRU10}, \href{../works/RasmussenT09.pdf}{RasmussenT09}~\cite{RasmussenT09}, \href{../works/RasmussenT07.pdf}{RasmussenT07}~\cite{RasmussenT07}, \href{../works/RasmussenT06.pdf}{RasmussenT06}~\cite{RasmussenT06}, \href{../works/RussellU06.pdf}{RussellU06}~\cite{RussellU06}, \href{../works/HenzMT04.pdf}{HenzMT04}~\cite{HenzMT04}, \href{../works/Trick03.pdf}{Trick03}~\cite{Trick03} & \href{../works/HebrardALLCMR22.pdf}{HebrardALLCMR22}~\cite{HebrardALLCMR22}, \href{../works/ZarandiASC20.pdf}{ZarandiASC20}~\cite{ZarandiASC20}, \href{../works/Perron05.pdf}{Perron05}~\cite{Perron05}, \href{../works/EastonNT02.pdf}{EastonNT02}~\cite{EastonNT02} & \href{../works/Hooker19.pdf}{Hooker19}~\cite{Hooker19}, \href{../works/LiuLH18.pdf}{LiuLH18}~\cite{LiuLH18}, \href{../works/MossigeGSMC17.pdf}{MossigeGSMC17}~\cite{MossigeGSMC17}, \href{../works/CobanH11.pdf}{CobanH11}~\cite{CobanH11}, \href{../works/Simonis07.pdf}{Simonis07}~\cite{Simonis07}, \href{../works/BeniniBGM05.pdf}{BeniniBGM05}~\cite{BeniniBGM05}, \href{../works/DilkinaH04.pdf}{DilkinaH04}~\cite{DilkinaH04}, \href{../works/KanetAG04.pdf}{KanetAG04}~\cite{KanetAG04}, \href{../works/ElfJR03.pdf}{ElfJR03}~\cite{ElfJR03}, \href{../works/Schaerf97.pdf}{Schaerf97}~\cite{Schaerf97}\\
\index{satellite}\index{ApplicationAreas!satellite}satellite &  1.00 & \href{../works/SquillaciPR23.pdf}{SquillaciPR23}~\cite{SquillaciPR23}, \href{../works/Godet21a.pdf}{Godet21a}~\cite{Godet21a}, \href{../works/GodetLHS20.pdf}{GodetLHS20}~\cite{GodetLHS20}, \href{../works/KucukY19.pdf}{KucukY19}~\cite{KucukY19}, \href{../works/LaborieRSV18.pdf}{LaborieRSV18}~\cite{LaborieRSV18}, \href{../works/HebrardHJMPV16.pdf}{HebrardHJMPV16}~\cite{HebrardHJMPV16}, \href{../works/FrankDT16.pdf}{FrankDT16}~\cite{FrankDT16}, \href{../works/Maillard15.pdf}{Maillard15}~\cite{Maillard15}, \href{../works/PraletLJ15.pdf}{PraletLJ15}~\cite{PraletLJ15}, \href{../works/KelarevaTK13.pdf}{KelarevaTK13}~\cite{KelarevaTK13}, \href{../works/ReddyFIBKAJ11.pdf}{ReddyFIBKAJ11}~\cite{ReddyFIBKAJ11}, \href{../works/Johnston05.pdf}{Johnston05}~\cite{Johnston05}, \href{../works/BarbulescuWH04.pdf}{BarbulescuWH04}~\cite{BarbulescuWH04}, \href{../works/GlobusCLP04.pdf}{GlobusCLP04}~\cite{GlobusCLP04}, \href{../works/VerfaillieL01.pdf}{VerfaillieL01}~\cite{VerfaillieL01}, \href{../works/BensanaLV99.pdf}{BensanaLV99}~\cite{BensanaLV99}, \href{../works/PembertonG98.pdf}{PembertonG98}~\cite{PembertonG98} & \href{../works/AlesioBNG15.pdf}{AlesioBNG15}~\cite{AlesioBNG15}, \href{../works/Laborie09.pdf}{Laborie09}~\cite{Laborie09}, \href{../works/FrankK05.pdf}{FrankK05}~\cite{FrankK05}, \href{../works/JussienL02.pdf}{JussienL02}~\cite{JussienL02} & \href{../works/WessenCSFPM23.pdf}{WessenCSFPM23}~\cite{WessenCSFPM23}, \href{../works/EfthymiouY23.pdf}{EfthymiouY23}~\cite{EfthymiouY23}, \href{../works/TouatBT22.pdf}{TouatBT22}~\cite{TouatBT22}, \href{../works/Astrand21.pdf}{Astrand21}~\cite{Astrand21}, \href{../works/Astrand0F21.pdf}{Astrand0F21}~\cite{Astrand0F21}, \href{../works/Zahout21.pdf}{Zahout21}~\cite{Zahout21}, \href{../works/ZarandiASC20.pdf}{ZarandiASC20}~\cite{ZarandiASC20}, \href{../works/FachiniA20.pdf}{FachiniA20}~\cite{FachiniA20}, \href{../works/Hooker19.pdf}{Hooker19}~\cite{Hooker19}, \href{../works/Pralet17.pdf}{Pralet17}~\cite{Pralet17}, \href{../works/TranVNB17.pdf}{TranVNB17}~\cite{TranVNB17}, \href{../works/Froger16.pdf}{Froger16}~\cite{Froger16}, \href{../works/TranWDRFOVB16.pdf}{TranWDRFOVB16}~\cite{TranWDRFOVB16}, \href{../works/SimoninAHL15.pdf}{SimoninAHL15}~\cite{SimoninAHL15}, \href{../works/BessiereHMQW14.pdf}{BessiereHMQW14}~\cite{BessiereHMQW14}, \href{../works/LaborieR14.pdf}{LaborieR14}~\cite{LaborieR14}, \href{../works/HeinzSB13.pdf}{HeinzSB13}~\cite{HeinzSB13}, \href{../works/SimoninAHL12.pdf}{SimoninAHL12}~\cite{SimoninAHL12}, \href{../works/GuyonLPR12.pdf}{GuyonLPR12}~\cite{GuyonLPR12}, \href{../works/RuggieroBBMA09.pdf}{RuggieroBBMA09}~\cite{RuggieroBBMA09}, \href{../works/Rodriguez07.pdf}{Rodriguez07}~\cite{Rodriguez07}, \href{../works/OddiPCC03.pdf}{OddiPCC03}~\cite{OddiPCC03}, \href{../works/NuijtenP98.pdf}{NuijtenP98}~\cite{NuijtenP98}\\
\index{semiconductor}\index{ApplicationAreas!semiconductor}semiconductor &  1.00 & \href{../works/MengLZB21.pdf}{MengLZB21}~\cite{MengLZB21}, \href{../works/ZarandiASC20.pdf}{ZarandiASC20}~\cite{ZarandiASC20}, \href{../works/MalapertN19.pdf}{MalapertN19}~\cite{MalapertN19}, \href{../works/NattafDYW19.pdf}{NattafDYW19}~\cite{NattafDYW19}, \href{../works/Ham18a.pdf}{Ham18a}~\cite{Ham18a}, \href{../works/HamFC17.pdf}{HamFC17}~\cite{HamFC17}, \href{../works/BajestaniB15.pdf}{BajestaniB15}~\cite{BajestaniB15}, \href{../works/NovasH12.pdf}{NovasH12}~\cite{NovasH12}, \href{../works/ZeballosCM10.pdf}{ZeballosCM10}~\cite{ZeballosCM10} & \href{../works/PenzDN23.pdf}{PenzDN23}~\cite{PenzDN23}, \href{../works/Tassel22.pdf}{Tassel22}~\cite{Tassel22}, \href{../works/QinWSLS21.pdf}{QinWSLS21}~\cite{QinWSLS21}, \href{../works/GokgurHO18.pdf}{GokgurHO18}~\cite{GokgurHO18}, \href{../works/HamC16.pdf}{HamC16}~\cite{HamC16}, \href{../works/Davenport10.pdf}{Davenport10}~\cite{Davenport10}, \href{../works/LombardiMRB10.pdf}{LombardiMRB10}~\cite{LombardiMRB10}, \href{../works/KrogtLPHJ07.pdf}{KrogtLPHJ07}~\cite{KrogtLPHJ07}, \href{../works/JainM99.pdf}{JainM99}~\cite{JainM99} & \href{../works/LacknerMMWW23.pdf}{LacknerMMWW23}~\cite{LacknerMMWW23}, \href{../works/Fatemi-AnarakiTFV23.pdf}{Fatemi-AnarakiTFV23}~\cite{Fatemi-AnarakiTFV23}, \href{../works/abs-2211-14492.pdf}{abs-2211-14492}~\cite{abs-2211-14492}, \href{../works/MengGRZSC22.pdf}{MengGRZSC22}~\cite{MengGRZSC22}, \href{../works/EmdeZD22.pdf}{EmdeZD22}~\cite{EmdeZD22}, \href{../works/YuraszeckMPV22.pdf}{YuraszeckMPV22}~\cite{YuraszeckMPV22}, \href{../works/MullerMKP22.pdf}{MullerMKP22}~\cite{MullerMKP22}, \href{../works/ColT22.pdf}{ColT22}~\cite{ColT22}, \href{../works/ZhangJZL22.pdf}{ZhangJZL22}~\cite{ZhangJZL22}, \href{../works/FanXG21.pdf}{FanXG21}~\cite{FanXG21}, \href{../works/LacknerMMWW21.pdf}{LacknerMMWW21}~\cite{LacknerMMWW21}, \href{../works/HamP21.pdf}{HamP21}~\cite{HamP21}, \href{../works/HamPK21.pdf}{HamPK21}~\cite{HamPK21}, \href{../works/PandeyS21a.pdf}{PandeyS21a}~\cite{PandeyS21a}, \href{../works/Astrand21.pdf}{Astrand21}~\cite{Astrand21}, \href{../works/Ham20a.pdf}{Ham20a}~\cite{Ham20a}, \href{../works/TangB20.pdf}{TangB20}~\cite{TangB20}, \href{../works/MengZRZL20.pdf}{MengZRZL20}~\cite{MengZRZL20}, \href{../works/NattafM20.pdf}{NattafM20}~\cite{NattafM20}...\href{../works/LaborieRSV18.pdf}{LaborieRSV18}~\cite{LaborieRSV18}, \href{../works/Ham18.pdf}{Ham18}~\cite{Ham18}, \href{../works/GrimesH15.pdf}{GrimesH15}~\cite{GrimesH15}, \href{../works/TerekhovTDB14.pdf}{TerekhovTDB14}~\cite{TerekhovTDB14}, \href{../works/KoschB14.pdf}{KoschB14}~\cite{KoschB14}, \href{../works/HarjunkoskiMBC14.pdf}{HarjunkoskiMBC14}~\cite{HarjunkoskiMBC14}, \href{../works/MalapertGR12.pdf}{MalapertGR12}~\cite{MalapertGR12}, \href{../works/Malapert11.pdf}{Malapert11}~\cite{Malapert11}, \href{../works/Lombardi10.pdf}{Lombardi10}~\cite{Lombardi10}, \href{../works/AchterbergBKW08.pdf}{AchterbergBKW08}~\cite{AchterbergBKW08} (Total: 31)\\
\index{shipping line}\index{ApplicationAreas!shipping line}shipping line &  1.00 &  &  & \href{../works/QinDCS20.pdf}{QinDCS20}~\cite{QinDCS20}, \href{../works/LaborieRSV18.pdf}{LaborieRSV18}~\cite{LaborieRSV18}, \href{../works/KelarevaTK13.pdf}{KelarevaTK13}~\cite{KelarevaTK13}\\
\index{sports scheduling}\index{ApplicationAreas!sports scheduling}sports scheduling &  1.00 & \href{../works/NaqviAIAAA22.pdf}{NaqviAIAAA22}~\cite{NaqviAIAAA22}, \href{../works/LarsonJC14.pdf}{LarsonJC14}~\cite{LarsonJC14}, \href{../works/SuCC13.pdf}{SuCC13}~\cite{SuCC13}, \href{../works/Ribeiro12.pdf}{Ribeiro12}~\cite{Ribeiro12}, \href{../works/KendallKRU10.pdf}{KendallKRU10}~\cite{KendallKRU10}, \href{../works/RasmussenT09.pdf}{RasmussenT09}~\cite{RasmussenT09}, \href{../works/RasmussenT07.pdf}{RasmussenT07}~\cite{RasmussenT07}, \href{../works/RasmussenT06.pdf}{RasmussenT06}~\cite{RasmussenT06} & \href{../works/LiuLH18.pdf}{LiuLH18}~\cite{LiuLH18}, \href{../works/CarlssonJL17.pdf}{CarlssonJL17}~\cite{CarlssonJL17}, \href{../works/ZengM12.pdf}{ZengM12}~\cite{ZengM12}, \href{../works/Perron05.pdf}{Perron05}~\cite{Perron05}, \href{../works/Trick03.pdf}{Trick03}~\cite{Trick03}, \href{../works/ElfJR03.pdf}{ElfJR03}~\cite{ElfJR03} & \href{../works/BulckG22.pdf}{BulckG22}~\cite{BulckG22}, \href{../works/ZarandiASC20.pdf}{ZarandiASC20}~\cite{ZarandiASC20}, \href{../works/Hooker19.pdf}{Hooker19}~\cite{Hooker19}, \href{../works/LiuLH19a.pdf}{LiuLH19a}~\cite{LiuLH19a}, \href{../works/GedikKEK18.pdf}{GedikKEK18}~\cite{GedikKEK18}, \href{../works/HookerH17.pdf}{HookerH17}~\cite{HookerH17}, \href{../works/QinDS16.pdf}{QinDS16}~\cite{QinDS16}, \href{../works/RiiseML16.pdf}{RiiseML16}~\cite{RiiseML16}, \href{../works/CobanH11.pdf}{CobanH11}~\cite{CobanH11}, \href{../works/RussellU06.pdf}{RussellU06}~\cite{RussellU06}, \href{../works/DilkinaH04.pdf}{DilkinaH04}~\cite{DilkinaH04}, \href{../works/EastonNT02.pdf}{EastonNT02}~\cite{EastonNT02}\\
\index{steel cable}\index{ApplicationAreas!steel cable}steel cable &  1.00 &  &  & \href{../works/AalianPG23.pdf}{AalianPG23}~\cite{AalianPG23}\\
\index{steel mill}\index{ApplicationAreas!steel mill}steel mill &  1.00 & \href{../works/GaySS14.pdf}{GaySS14}~\cite{GaySS14}, \href{../works/Letort13.pdf}{Letort13}~\cite{Letort13}, \href{../works/HeinzSSW12.pdf}{HeinzSSW12}~\cite{HeinzSSW12}, \href{../works/SchausHMCMD11.pdf}{SchausHMCMD11}~\cite{SchausHMCMD11}, \href{../works/HentenryckM08.pdf}{HentenryckM08}~\cite{HentenryckM08}, \href{../works/GarganiR07.pdf}{GarganiR07}~\cite{GarganiR07} &  & \href{../works/abs-2312-13682.pdf}{abs-2312-13682}~\cite{abs-2312-13682}, \href{../works/PerezGSL23.pdf}{PerezGSL23}~\cite{PerezGSL23}, \href{../works/LiuLH19a.pdf}{LiuLH19a}~\cite{LiuLH19a}, \href{../works/DoulabiRP16.pdf}{DoulabiRP16}~\cite{DoulabiRP16}, \href{../works/MenciaSV13.pdf}{MenciaSV13}~\cite{MenciaSV13}, \href{../works/MenciaSV12.pdf}{MenciaSV12}~\cite{MenciaSV12}\\
\index{super-computer}\index{ApplicationAreas!super-computer}super-computer &  1.00 & \href{../works/BorghesiBLMB18.pdf}{BorghesiBLMB18}~\cite{BorghesiBLMB18}, \href{../works/BridiBLMB16.pdf}{BridiBLMB16}~\cite{BridiBLMB16}, \href{../works/BartoliniBBLM14.pdf}{BartoliniBBLM14}~\cite{BartoliniBBLM14} &  & \href{../works/LuoB22.pdf}{LuoB22}~\cite{LuoB22}, \href{../works/GalleguillosKSB19.pdf}{GalleguillosKSB19}~\cite{GalleguillosKSB19}, \href{../works/Dejemeppe16.pdf}{Dejemeppe16}~\cite{Dejemeppe16}, \href{../works/HurleyOS16.pdf}{HurleyOS16}~\cite{HurleyOS16}\\
\index{surgery}\index{ApplicationAreas!surgery}surgery &  1.00 & \href{../works/GurPAE23.pdf}{GurPAE23}~\cite{GurPAE23}, \href{../works/GhandehariK22.pdf}{GhandehariK22}~\cite{GhandehariK22}, \href{../works/FarsiTM22.pdf}{FarsiTM22}~\cite{FarsiTM22}, \href{../works/RoshanaeiN21.pdf}{RoshanaeiN21}~\cite{RoshanaeiN21}, \href{../works/RoshanaeiBAUB20.pdf}{RoshanaeiBAUB20}~\cite{RoshanaeiBAUB20}, \href{../works/GurEA19.pdf}{GurEA19}~\cite{GurEA19}, \href{../works/YounespourAKE19.pdf}{YounespourAKE19}~\cite{YounespourAKE19}, \href{../works/RoshanaeiLAU17.pdf}{RoshanaeiLAU17}~\cite{RoshanaeiLAU17}, \href{../works/RiiseML16.pdf}{RiiseML16}~\cite{RiiseML16}, \href{../works/DoulabiRP16.pdf}{DoulabiRP16}~\cite{DoulabiRP16}, \href{../works/WangMD15.pdf}{WangMD15}~\cite{WangMD15}, \href{../works/ZhaoL14.pdf}{ZhaoL14}~\cite{ZhaoL14}, \href{../works/DoulabiRP14.pdf}{DoulabiRP14}~\cite{DoulabiRP14}, \href{../works/MeskensDL13.pdf}{MeskensDL13}~\cite{MeskensDL13}, \href{../works/Wolf11.pdf}{Wolf11}~\cite{Wolf11}, \href{../works/Wolf09.pdf}{Wolf09}~\cite{Wolf09} & \href{../works/FrimodigECM23.pdf}{FrimodigECM23}~\cite{FrimodigECM23}, \href{../works/ZarandiASC20.pdf}{ZarandiASC20}~\cite{ZarandiASC20}, \href{../works/TopalogluO11.pdf}{TopalogluO11}~\cite{TopalogluO11} & \href{../works/ForbesHJST24.pdf}{ForbesHJST24}~\cite{ForbesHJST24}, \href{../works/AlfieriGPS23.pdf}{AlfieriGPS23}~\cite{AlfieriGPS23}, \href{../works/NaderiBZR23.pdf}{NaderiBZR23}~\cite{NaderiBZR23}, \href{../works/NaderiBZ23.pdf}{NaderiBZ23}~\cite{NaderiBZ23}, \href{../works/NaderiBZ22.pdf}{NaderiBZ22}~\cite{NaderiBZ22}, \href{../works/ElciOH22.pdf}{ElciOH22}~\cite{ElciOH22}, \href{../works/Lemos21.pdf}{Lemos21}~\cite{Lemos21}, \href{../works/FrimodigS19.pdf}{FrimodigS19}~\cite{FrimodigS19}, \href{../works/MeskensDHG11.pdf}{MeskensDHG11}~\cite{MeskensDHG11}\\
\index{telescope}\index{ApplicationAreas!telescope}telescope &  1.00 & \href{../works/FrankK05.pdf}{FrankK05}~\cite{FrankK05}, \href{../works/FrankK03.pdf}{FrankK03}~\cite{FrankK03}, \href{../works/MintonJPL92.pdf}{MintonJPL92}~\cite{MintonJPL92}, \href{../works/MintonJPL90.pdf}{MintonJPL90}~\cite{MintonJPL90} & \href{../works/CatusseCBL16.pdf}{CatusseCBL16}~\cite{CatusseCBL16}, \href{../works/Johnston05.pdf}{Johnston05}~\cite{Johnston05}, \href{../works/FoxS90.pdf}{FoxS90}~\cite{FoxS90} & \href{../works/AlesioBNG15.pdf}{AlesioBNG15}~\cite{AlesioBNG15}, \href{../works/Maillard15.pdf}{Maillard15}~\cite{Maillard15}, \href{../works/ReddyFIBKAJ11.pdf}{ReddyFIBKAJ11}~\cite{ReddyFIBKAJ11}, \href{../works/BidotVLB09.pdf}{BidotVLB09}~\cite{BidotVLB09}, \href{../works/BeckW07.pdf}{BeckW07}~\cite{BeckW07}, \href{../works/GlobusCLP04.pdf}{GlobusCLP04}~\cite{GlobusCLP04}, \href{../works/Beck99.pdf}{Beck99}~\cite{Beck99}, \href{../works/BeckDDF98.pdf}{BeckDDF98}~\cite{BeckDDF98}, \href{../works/PembertonG98.pdf}{PembertonG98}~\cite{PembertonG98}, \href{../works/Wallace96.pdf}{Wallace96}~\cite{Wallace96}, \href{../works/SmithC93.pdf}{SmithC93}~\cite{SmithC93}\\
\index{torpedo}\index{ApplicationAreas!torpedo}torpedo &  1.00 & \href{../works/GoldwaserS18.pdf}{GoldwaserS18}~\cite{GoldwaserS18}, \href{../works/KletzanderM17.pdf}{KletzanderM17}~\cite{KletzanderM17}, \href{../works/GoldwaserS17.pdf}{GoldwaserS17}~\cite{GoldwaserS17} & \href{../works/AntuoriHHEN20.pdf}{AntuoriHHEN20}~\cite{AntuoriHHEN20} & \href{../works/Hooker19.pdf}{Hooker19}~\cite{Hooker19}\\
\index{tournament}\index{ApplicationAreas!tournament}tournament &  1.00 & \href{../works/NaqviAIAAA22.pdf}{NaqviAIAAA22}~\cite{NaqviAIAAA22}, \href{../works/BulckG22.pdf}{BulckG22}~\cite{BulckG22}, \href{../works/FanXG21.pdf}{FanXG21}~\cite{FanXG21}, \href{../works/ZarandiASC20.pdf}{ZarandiASC20}~\cite{ZarandiASC20}, \href{../works/Lunardi20.pdf}{Lunardi20}~\cite{Lunardi20}, \href{../works/LiuLH18.pdf}{LiuLH18}~\cite{LiuLH18}, \href{../works/CarlssonJL17.pdf}{CarlssonJL17}~\cite{CarlssonJL17}, \href{../works/LarsonJC14.pdf}{LarsonJC14}~\cite{LarsonJC14}, \href{../works/SuCC13.pdf}{SuCC13}~\cite{SuCC13}, \href{../works/ZengM12.pdf}{ZengM12}~\cite{ZengM12}, \href{../works/Ribeiro12.pdf}{Ribeiro12}~\cite{Ribeiro12}, \href{../works/KendallKRU10.pdf}{KendallKRU10}~\cite{KendallKRU10}, \href{../works/RasmussenT09.pdf}{RasmussenT09}~\cite{RasmussenT09}, \href{../works/RasmussenT07.pdf}{RasmussenT07}~\cite{RasmussenT07}, \href{../works/RasmussenT06.pdf}{RasmussenT06}~\cite{RasmussenT06}, \href{../works/RussellU06.pdf}{RussellU06}~\cite{RussellU06}, \href{../works/HenzMT04.pdf}{HenzMT04}~\cite{HenzMT04}, \href{../works/GlobusCLP04.pdf}{GlobusCLP04}~\cite{GlobusCLP04}, \href{../works/Trick03.pdf}{Trick03}~\cite{Trick03}, \href{../works/ElfJR03.pdf}{ElfJR03}~\cite{ElfJR03}, \href{../works/EastonNT02.pdf}{EastonNT02}~\cite{EastonNT02}, \href{../works/Schaerf97.pdf}{Schaerf97}~\cite{Schaerf97} & \href{../works/CilKLO22.pdf}{CilKLO22}~\cite{CilKLO22}, \href{../works/QinWSLS21.pdf}{QinWSLS21}~\cite{QinWSLS21}, \href{../works/HookerH17.pdf}{HookerH17}~\cite{HookerH17} & \href{../works/abs-2402-00459.pdf}{abs-2402-00459}~\cite{abs-2402-00459}, \href{../works/Lemos21.pdf}{Lemos21}~\cite{Lemos21}, \href{../works/HubnerGSV21.pdf}{HubnerGSV21}~\cite{HubnerGSV21}, \href{../works/Hooker19.pdf}{Hooker19}~\cite{Hooker19}, \href{../works/LiuLH19a.pdf}{LiuLH19a}~\cite{LiuLH19a}, \href{../works/LiuLH19.pdf}{LiuLH19}~\cite{LiuLH19}, \href{../works/GedikKBR17.pdf}{GedikKBR17}~\cite{GedikKBR17}, \href{../works/Froger16.pdf}{Froger16}~\cite{Froger16}, \href{../works/Dejemeppe16.pdf}{Dejemeppe16}~\cite{Dejemeppe16}, \href{../works/RendlPHPR12.pdf}{RendlPHPR12}~\cite{RendlPHPR12}, \href{../works/CobanH11.pdf}{CobanH11}~\cite{CobanH11}, \href{../works/LiW08.pdf}{LiW08}~\cite{LiW08}, \href{../works/SureshMOK06.pdf}{SureshMOK06}~\cite{SureshMOK06}, \href{../works/Perron05.pdf}{Perron05}~\cite{Perron05}, \href{../works/KanetAG04.pdf}{KanetAG04}~\cite{KanetAG04}, \href{../works/DilkinaH04.pdf}{DilkinaH04}~\cite{DilkinaH04}, \href{../works/Demassey03.pdf}{Demassey03}~\cite{Demassey03}, \href{../works/EreminW01.pdf}{EreminW01}~\cite{EreminW01}\\
\index{train schedule}\index{ApplicationAreas!train schedule}train schedule &  1.00 & \href{../works/LuZZYW24.pdf}{LuZZYW24}~\cite{LuZZYW24}, \href{../works/MarliereSPR23.pdf}{MarliereSPR23}~\cite{MarliereSPR23}, \href{../works/Lemos21.pdf}{Lemos21}~\cite{Lemos21}, \href{../works/CappartS17.pdf}{CappartS17}~\cite{CappartS17}, \href{../works/RodriguezS09.pdf}{RodriguezS09}~\cite{RodriguezS09}, \href{../works/Rodriguez07b.pdf}{Rodriguez07b}~\cite{Rodriguez07b}, \href{../works/Geske05.pdf}{Geske05}~\cite{Geske05}, \href{../works/MurphyRFSS97.pdf}{MurphyRFSS97}~\cite{MurphyRFSS97} & \href{../works/ZarandiASC20.pdf}{ZarandiASC20}~\cite{ZarandiASC20}, \href{../works/LammaMM97.pdf}{LammaMM97}~\cite{LammaMM97}, \href{../works/BrusoniCLMMT96.pdf}{BrusoniCLMMT96}~\cite{BrusoniCLMMT96} & \href{../works/abs-2312-13682.pdf}{abs-2312-13682}~\cite{abs-2312-13682}, \href{../works/SvancaraB22.pdf}{SvancaraB22}~\cite{SvancaraB22}, \href{../works/GeibingerMM21.pdf}{GeibingerMM21}~\cite{GeibingerMM21}, \href{../works/Novas19.pdf}{Novas19}~\cite{Novas19}, \href{../works/AgussurjaKL18.pdf}{AgussurjaKL18}~\cite{AgussurjaKL18}, \href{../works/Froger16.pdf}{Froger16}~\cite{Froger16}, \href{../works/LiuW11.pdf}{LiuW11}~\cite{LiuW11}, \href{../works/Rodriguez07.pdf}{Rodriguez07}~\cite{Rodriguez07}, \href{../works/RodriguezDG02.pdf}{RodriguezDG02}~\cite{RodriguezDG02}, \href{../works/MartinPY01.pdf}{MartinPY01}~\cite{MartinPY01}, \href{../works/Wallace96.pdf}{Wallace96}~\cite{Wallace96}\\
\index{travelling tournament problem}\index{ApplicationAreas!travelling tournament problem}travelling tournament problem &  1.00 & \href{../works/LiuLH18.pdf}{LiuLH18}~\cite{LiuLH18}, \href{../works/Ribeiro12.pdf}{Ribeiro12}~\cite{Ribeiro12}, \href{../works/ZengM12.pdf}{ZengM12}~\cite{ZengM12}, \href{../works/KendallKRU10.pdf}{KendallKRU10}~\cite{KendallKRU10}, \href{../works/RasmussenT09.pdf}{RasmussenT09}~\cite{RasmussenT09}, \href{../works/RasmussenT07.pdf}{RasmussenT07}~\cite{RasmussenT07} & \href{../works/NaqviAIAAA22.pdf}{NaqviAIAAA22}~\cite{NaqviAIAAA22}, \href{../works/HookerH17.pdf}{HookerH17}~\cite{HookerH17}, \href{../works/EastonNT02.pdf}{EastonNT02}~\cite{EastonNT02} & \href{../works/BulckG22.pdf}{BulckG22}~\cite{BulckG22}, \href{../works/CarlssonJL17.pdf}{CarlssonJL17}~\cite{CarlssonJL17}, \href{../works/LarsonJC14.pdf}{LarsonJC14}~\cite{LarsonJC14}, \href{../works/LiW08.pdf}{LiW08}~\cite{LiW08}, \href{../works/RasmussenT06.pdf}{RasmussenT06}~\cite{RasmussenT06}, \href{../works/Demassey03.pdf}{Demassey03}~\cite{Demassey03}, \href{../works/EreminW01.pdf}{EreminW01}~\cite{EreminW01}\\
\index{vaccine}\index{ApplicationAreas!vaccine}vaccine &  1.00 &  & \href{../works/GuoZ23.pdf}{GuoZ23}~\cite{GuoZ23} & \href{../works/BonninMNE24.pdf}{BonninMNE24}~\cite{BonninMNE24}, \href{../works/JuvinHL23a.pdf}{JuvinHL23a}~\cite{JuvinHL23a}\\
\index{wildfire}\index{ApplicationAreas!wildfire}wildfire &  1.00 &  & \href{../works/ArtiguesHQT21.pdf}{ArtiguesHQT21}~\cite{ArtiguesHQT21} & \\
\index{workforce scheduling}\index{ApplicationAreas!workforce scheduling}workforce scheduling &  1.00 & \href{../works/BourreauGGLT22.pdf}{BourreauGGLT22}~\cite{BourreauGGLT22}, \href{../works/MusliuSS18.pdf}{MusliuSS18}~\cite{MusliuSS18}, \href{../works/ErkingerM17.pdf}{ErkingerM17}~\cite{ErkingerM17}, \href{../works/Wallace06.pdf}{Wallace06}~\cite{Wallace06}, \href{../works/Musliu05.pdf}{Musliu05}~\cite{Musliu05} & \href{../works/AntunesABD20.pdf}{AntunesABD20}~\cite{AntunesABD20}, \href{../works/AntunesABD18.pdf}{AntunesABD18}~\cite{AntunesABD18} & \href{../works/GokPTGO23.pdf}{GokPTGO23}~\cite{GokPTGO23}, \href{../works/FallahiAC20.pdf}{FallahiAC20}~\cite{FallahiAC20}, \href{../works/CorreaLR07.pdf}{CorreaLR07}~\cite{CorreaLR07}, \href{../works/BenoistGR02.pdf}{BenoistGR02}~\cite{BenoistGR02}, \href{../works/Mason01.pdf}{Mason01}~\cite{Mason01}, \href{../works/Darby-DowmanLMZ97.pdf}{Darby-DowmanLMZ97}~\cite{Darby-DowmanLMZ97}\\
\index{yard crane}\index{ApplicationAreas!yard crane}yard crane &  1.00 &  & \href{../works/QinDCS20.pdf}{QinDCS20}~\cite{QinDCS20}, \href{../works/Hooker19.pdf}{Hooker19}~\cite{Hooker19} & \href{../works/EmdeZD22.pdf}{EmdeZD22}~\cite{EmdeZD22}, \href{../works/WallaceY20.pdf}{WallaceY20}~\cite{WallaceY20}, \href{../works/SunTB19.pdf}{SunTB19}~\cite{SunTB19}, \href{../works/UnsalO13.pdf}{UnsalO13}~\cite{UnsalO13}\\
\end{longtable}
}

\clearpage
\subsection{Concept Type Industries}
\label{sec:Industries}
\label{Industries}
{\scriptsize
\begin{longtable}{p{3cm}r>{\raggedright\arraybackslash}p{6cm}>{\raggedright\arraybackslash}p{6cm}>{\raggedright\arraybackslash}p{8cm}}
\rowcolor{white}\caption{Works for Concepts of Type Industries (Total 77 Concepts, 72 Used)}\\ \toprule
\rowcolor{white}Keyword & Weight & High & Medium & Low\\ \midrule\endhead
\bottomrule
\endfoot
\index{IT industry}\index{Industries!IT industry}IT industry &  1.00 &  &  & \href{../works/SchnellH15.pdf}{SchnellH15}~\cite{SchnellH15}\\
\index{aerospace industry}\index{Industries!aerospace industry}aerospace industry &  1.00 &  &  & \href{../works/SchildW00.pdf}{SchildW00}~\cite{SchildW00}\\
\index{agricultural industry}\index{Industries!agricultural industry}agricultural industry &  1.00 & \href{../works/WinterMMW22.pdf}{WinterMMW22}~\cite{WinterMMW22} &  & \\
\index{agrifood industry}\index{Industries!agrifood industry}agrifood industry &  1.00 &  &  & \href{../works/Groleaz21.pdf}{Groleaz21}~\cite{Groleaz21}\\
\index{airline industry}\index{Industries!airline industry}airline industry &  1.00 &  &  & \href{../works/GokPTGO23.pdf}{GokPTGO23}~\cite{GokPTGO23}, \href{../works/HachemiGR11.pdf}{HachemiGR11}~\cite{HachemiGR11}, \href{../works/Mason01.pdf}{Mason01}~\cite{Mason01}\\
\index{automobile industry}\index{Industries!automobile industry}automobile industry &  1.00 &  &  & \href{../works/HauderBRPA20.pdf}{HauderBRPA20}~\cite{HauderBRPA20}, \href{../works/abs-1902-09244.pdf}{abs-1902-09244}~\cite{abs-1902-09244}, \href{../works/Limtanyakul07.pdf}{Limtanyakul07}~\cite{Limtanyakul07}\\
\index{automotive industry}\index{Industries!automotive industry}automotive industry &  1.00 &  & \href{../works/GuoZ23.pdf}{GuoZ23}~\cite{GuoZ23}, \href{../works/LimtanyakulS12.pdf}{LimtanyakulS12}~\cite{LimtanyakulS12} & \href{../works/CzerniachowskaWZ23.pdf}{CzerniachowskaWZ23}~\cite{CzerniachowskaWZ23}, \href{../works/EmdeZD22.pdf}{EmdeZD22}~\cite{EmdeZD22}, \href{../works/AntuoriHHEN21.pdf}{AntuoriHHEN21}~\cite{AntuoriHHEN21}, \href{../works/AbidinK20.pdf}{AbidinK20}~\cite{AbidinK20}, \href{../works/KonowalenkoMM19.pdf}{KonowalenkoMM19}~\cite{KonowalenkoMM19}, \href{../works/BonfiettiZLM16.pdf}{BonfiettiZLM16}~\cite{BonfiettiZLM16}, \href{../works/SchildW00.pdf}{SchildW00}~\cite{SchildW00}, \href{../works/Wallace96.pdf}{Wallace96}~\cite{Wallace96}\\
\index{aviation industry}\index{Industries!aviation industry}aviation industry &  1.00 &  &  & \\
\index{cable industry}\index{Industries!cable industry}cable industry &  1.00 &  &  & \href{../works/ZhuSZW23.pdf}{ZhuSZW23}~\cite{ZhuSZW23}\\
\index{carpet industry}\index{Industries!carpet industry}carpet industry &  1.00 &  &  & \href{../works/Schutt11.pdf}{Schutt11}~\cite{Schutt11}\\
\index{chemical industry}\index{Industries!chemical industry}chemical industry &  1.00 &  & \href{../works/Timpe02.pdf}{Timpe02}~\cite{Timpe02} & \href{../works/LaborieRSV18.pdf}{LaborieRSV18}~\cite{LaborieRSV18}, \href{../works/GilesH16.pdf}{GilesH16}~\cite{GilesH16}, \href{../works/HarjunkoskiMBC14.pdf}{HarjunkoskiMBC14}~\cite{HarjunkoskiMBC14}, \href{../works/LombardiM12.pdf}{LombardiM12}~\cite{LombardiM12}, \href{../works/ZeballosNH11.pdf}{ZeballosNH11}~\cite{ZeballosNH11}, \href{../works/ChenGPSH10.pdf}{ChenGPSH10}~\cite{ChenGPSH10}, \href{../works/MaraveliasCG04.pdf}{MaraveliasCG04}~\cite{MaraveliasCG04}, \href{../works/PoderBS04.pdf}{PoderBS04}~\cite{PoderBS04}, \href{../works/HookerO99.pdf}{HookerO99}~\cite{HookerO99}, \href{../works/Simonis99.pdf}{Simonis99}~\cite{Simonis99}, \href{../works/Simonis95a.pdf}{Simonis95a}~\cite{Simonis95a}\\
\index{chemical processing industry}\index{Industries!chemical processing industry}chemical processing industry &  1.00 &  &  & \href{../works/GilesH16.pdf}{GilesH16}~\cite{GilesH16}\\
\index{chemistry industry}\index{Industries!chemistry industry}chemistry industry &  1.00 &  &  & \href{../works/ChenGPSH10.pdf}{ChenGPSH10}~\cite{ChenGPSH10}\\
\index{chips industry}\index{Industries!chips industry}chips industry &  1.00 &  &  & \href{../works/AbreuN22.pdf}{AbreuN22}~\cite{AbreuN22}\\
\index{circuit boards industry}\index{Industries!circuit boards industry}circuit boards industry &  1.00 &  &  & \href{../works/MokhtarzadehTNF20.pdf}{MokhtarzadehTNF20}~\cite{MokhtarzadehTNF20}\\
\index{control system industry}\index{Industries!control system industry}control system industry &  1.00 &  &  & \href{../works/BonfiettiZLM16.pdf}{BonfiettiZLM16}~\cite{BonfiettiZLM16}\\
\index{cutting industry}\index{Industries!cutting industry}cutting industry &  1.00 &  &  & \href{../works/RiahiNS018.pdf}{RiahiNS018}~\cite{RiahiNS018}\\
\index{dairy industry}\index{Industries!dairy industry}dairy industry &  1.00 &  & \href{../works/EscobetPQPRA19.pdf}{EscobetPQPRA19}~\cite{EscobetPQPRA19}, \href{../works/HarjunkoskiMBC14.pdf}{HarjunkoskiMBC14}~\cite{HarjunkoskiMBC14} & \href{../works/Groleaz21.pdf}{Groleaz21}~\cite{Groleaz21}\\
\index{dismantling industry}\index{Industries!dismantling industry}dismantling industry &  1.00 &  &  & \href{../works/HubnerGSV21.pdf}{HubnerGSV21}~\cite{HubnerGSV21}\\
\index{drawing industry}\index{Industries!drawing industry}drawing industry &  1.00 &  &  & \href{../works/Simonis95a.pdf}{Simonis95a}~\cite{Simonis95a}\\
\index{electricity industry}\index{Industries!electricity industry}electricity industry &  1.00 & \href{../works/Froger16.pdf}{Froger16}~\cite{Froger16} &  & \href{../works/PopovicCGNC22.pdf}{PopovicCGNC22}~\cite{PopovicCGNC22}, \href{../works/Godet21a.pdf}{Godet21a}~\cite{Godet21a}, \href{../works/AntunesABD20.pdf}{AntunesABD20}~\cite{AntunesABD20}, \href{../works/AntunesABD18.pdf}{AntunesABD18}~\cite{AntunesABD18}\\
\index{electronics industry}\index{Industries!electronics industry}electronics industry &  1.00 &  &  & \href{../works/LacknerMMWW23.pdf}{LacknerMMWW23}~\cite{LacknerMMWW23}, \href{../works/LacknerMMWW21.pdf}{LacknerMMWW21}~\cite{LacknerMMWW21}\\
\index{electroplating industry}\index{Industries!electroplating industry}electroplating industry &  1.00 &  &  & \href{../works/NovasH12.pdf}{NovasH12}~\cite{NovasH12}\\
\index{energy industry}\index{Industries!energy industry}energy industry &  1.00 &  & \href{../works/Froger16.pdf}{Froger16}~\cite{Froger16} & \href{../works/KovacsV06.pdf}{KovacsV06}~\cite{KovacsV06}\\
\index{fashion industry}\index{Industries!fashion industry}fashion industry &  1.00 &  &  & \href{../works/Jans09.pdf}{Jans09}~\cite{Jans09}\\
\index{food industry}\index{Industries!food industry}food industry &  1.00 &  & \href{../works/Groleaz21.pdf}{Groleaz21}~\cite{Groleaz21} & \href{../works/Fatemi-AnarakiTFV23.pdf}{Fatemi-AnarakiTFV23}~\cite{Fatemi-AnarakiTFV23}, \href{../works/OujanaAYB22.pdf}{OujanaAYB22}~\cite{OujanaAYB22}, \href{../works/GroleazNS20a.pdf}{GroleazNS20a}~\cite{GroleazNS20a}, \href{../works/GroleazNS20.pdf}{GroleazNS20}~\cite{GroleazNS20}, \href{../works/EscobetPQPRA19.pdf}{EscobetPQPRA19}~\cite{EscobetPQPRA19}, \href{../works/HachemiGR11.pdf}{HachemiGR11}~\cite{HachemiGR11}, \href{../works/SimonisCK00.pdf}{SimonisCK00}~\cite{SimonisCK00}, \href{../works/Simonis99.pdf}{Simonis99}~\cite{Simonis99}, \href{../works/SimonisC95.pdf}{SimonisC95}~\cite{SimonisC95}, \href{../works/Simonis95.pdf}{Simonis95}~\cite{Simonis95}\\
\index{food-processing industry}\index{Industries!food-processing industry}food-processing industry &  1.00 &  &  & \href{../works/KlankeBYE21.pdf}{KlankeBYE21}~\cite{KlankeBYE21}, \href{../works/HauderBRPA20.pdf}{HauderBRPA20}~\cite{HauderBRPA20}, \href{../works/abs-1902-09244.pdf}{abs-1902-09244}~\cite{abs-1902-09244}\\
\index{forest industry}\index{Industries!forest industry}forest industry &  1.00 &  &  & \href{../works/ZhaoL14.pdf}{ZhaoL14}~\cite{ZhaoL14}, \href{../works/HachemiGR11.pdf}{HachemiGR11}~\cite{HachemiGR11}\\
\index{forging industry}\index{Industries!forging industry}forging industry &  1.00 &  &  & \href{../works/LuoB22.pdf}{LuoB22}~\cite{LuoB22}\\
\index{foundry industry}\index{Industries!foundry industry}foundry industry &  1.00 &  &  & \href{../works/Jans09.pdf}{Jans09}~\cite{Jans09}\\
\index{garment industry}\index{Industries!garment industry}garment industry &  1.00 &  &  & \href{../works/AlakaP23.pdf}{AlakaP23}~\cite{AlakaP23}, \href{../works/GuoZ23.pdf}{GuoZ23}~\cite{GuoZ23}\\
\index{gas industry}\index{Industries!gas industry}gas industry &  1.00 &  &  & \href{../works/ZarandiASC20.pdf}{ZarandiASC20}~\cite{ZarandiASC20}, \href{../works/GoelSHFS15.pdf}{GoelSHFS15}~\cite{GoelSHFS15}\\
\index{glass industry}\index{Industries!glass industry}glass industry &  1.00 &  &  & \href{../works/Lunardi20.pdf}{Lunardi20}~\cite{Lunardi20}, \href{../works/LunardiBLRV20.pdf}{LunardiBLRV20}~\cite{LunardiBLRV20}, \href{../works/abs-1902-09244.pdf}{abs-1902-09244}~\cite{abs-1902-09244}\\
\index{heavy industry}\index{Industries!heavy industry}heavy industry &  1.00 &  &  & \href{../works/CorreaLR07.pdf}{CorreaLR07}~\cite{CorreaLR07}\\
\index{insulation industry}\index{Industries!insulation industry}insulation industry &  1.00 &  &  & \href{../works/YunusogluY22.pdf}{YunusogluY22}~\cite{YunusogluY22}\\
\index{lumber industry}\index{Industries!lumber industry}lumber industry &  1.00 &  &  & \href{../works/NattafDYW19.pdf}{NattafDYW19}~\cite{NattafDYW19}\\
\index{manufacturing industry}\index{Industries!manufacturing industry}manufacturing industry &  1.00 &  &  & \href{../works/PrataAN23.pdf}{PrataAN23}~\cite{PrataAN23}, \href{../works/LacknerMMWW23.pdf}{LacknerMMWW23}~\cite{LacknerMMWW23}, \href{../works/CzerniachowskaWZ23.pdf}{CzerniachowskaWZ23}~\cite{CzerniachowskaWZ23}, \href{../works/WinterMMW22.pdf}{WinterMMW22}~\cite{WinterMMW22}, \href{../works/YuraszeckMPV22.pdf}{YuraszeckMPV22}~\cite{YuraszeckMPV22}, \href{../works/LacknerMMWW21.pdf}{LacknerMMWW21}~\cite{LacknerMMWW21}, \href{../works/FanXG21.pdf}{FanXG21}~\cite{FanXG21}, \href{../works/Mercier-AubinGQ20.pdf}{Mercier-AubinGQ20}~\cite{Mercier-AubinGQ20}, \href{../works/TangB20.pdf}{TangB20}~\cite{TangB20}, \href{../works/EscobetPQPRA19.pdf}{EscobetPQPRA19}~\cite{EscobetPQPRA19}, \href{../works/GedikKEK18.pdf}{GedikKEK18}~\cite{GedikKEK18}, \href{../works/RoweJCA96.pdf}{RoweJCA96}~\cite{RoweJCA96}\\
\index{maritime industry}\index{Industries!maritime industry}maritime industry &  1.00 &  &  & \href{../works/Astrand21.pdf}{Astrand21}~\cite{Astrand21}, \href{../works/QinDCS20.pdf}{QinDCS20}~\cite{QinDCS20}, \href{../works/SacramentoSP20.pdf}{SacramentoSP20}~\cite{SacramentoSP20}, \href{../works/SunTB19.pdf}{SunTB19}~\cite{SunTB19}, \href{../works/ZampelliVSDR13.pdf}{ZampelliVSDR13}~\cite{ZampelliVSDR13}\\
\index{metal industry}\index{Industries!metal industry}metal industry &  1.00 &  &  & \href{../works/LuoB22.pdf}{LuoB22}~\cite{LuoB22}\\
\index{mineral industry}\index{Industries!mineral industry}mineral industry &  1.00 &  &  & \href{../works/Astrand0F21.pdf}{Astrand0F21}~\cite{Astrand0F21}, \href{../works/Astrand21.pdf}{Astrand21}~\cite{Astrand21}, \href{../works/AstrandJZ20.pdf}{AstrandJZ20}~\cite{AstrandJZ20}, \href{../works/BlomBPS14.pdf}{BlomBPS14}~\cite{BlomBPS14}\\
\index{mining industry}\index{Industries!mining industry}mining industry &  1.00 &  & \href{../works/AalianPG23.pdf}{AalianPG23}~\cite{AalianPG23} & \href{../works/abs-2402-00459.pdf}{abs-2402-00459}~\cite{abs-2402-00459}, \href{../works/CampeauG22.pdf}{CampeauG22}~\cite{CampeauG22}, \href{../works/Astrand21.pdf}{Astrand21}~\cite{Astrand21}, \href{../works/Astrand0F21.pdf}{Astrand0F21}~\cite{Astrand0F21}, \href{../works/AstrandJZ20.pdf}{AstrandJZ20}~\cite{AstrandJZ20}, \href{../works/ThiruvadyWGS14.pdf}{ThiruvadyWGS14}~\cite{ThiruvadyWGS14}\\
\index{oil industry}\index{Industries!oil industry}oil industry &  1.00 &  &  & \href{../works/AbreuNP23.pdf}{AbreuNP23}~\cite{AbreuNP23}, \href{../works/AbreuAPNM21.pdf}{AbreuAPNM21}~\cite{AbreuAPNM21}, \href{../works/HarjunkoskiMBC14.pdf}{HarjunkoskiMBC14}~\cite{HarjunkoskiMBC14}, \href{../works/LopesCSM10.pdf}{LopesCSM10}~\cite{LopesCSM10}\\
\index{packaging industry}\index{Industries!packaging industry}packaging industry &  1.00 &  &  & \href{../works/ArmstrongGOS21.pdf}{ArmstrongGOS21}~\cite{ArmstrongGOS21}\\
\index{painting industry}\index{Industries!painting industry}painting industry &  1.00 &  &  & \href{../works/RiahiNS018.pdf}{RiahiNS018}~\cite{RiahiNS018}\\
\index{paper industry}\index{Industries!paper industry}paper industry &  1.00 &  &  & \href{../works/Dejemeppe16.pdf}{Dejemeppe16}~\cite{Dejemeppe16}, \href{../works/HarjunkoskiMBC14.pdf}{HarjunkoskiMBC14}~\cite{HarjunkoskiMBC14}\\
\index{petro-chemical industry}\index{Industries!petro-chemical industry}petro-chemical industry &  1.00 &  &  & \href{../works/LaborieRSV18.pdf}{LaborieRSV18}~\cite{LaborieRSV18}, \href{../works/GilesH16.pdf}{GilesH16}~\cite{GilesH16}, \href{../works/HarjunkoskiMBC14.pdf}{HarjunkoskiMBC14}~\cite{HarjunkoskiMBC14}\\
\index{pharmaceutical industry}\index{Industries!pharmaceutical industry}pharmaceutical industry &  1.00 & \href{../works/AwadMDMT22.pdf}{AwadMDMT22}~\cite{AwadMDMT22} &  & \href{../works/YuraszeckMCCR23.pdf}{YuraszeckMCCR23}~\cite{YuraszeckMCCR23}, \href{../works/CzerniachowskaWZ23.pdf}{CzerniachowskaWZ23}~\cite{CzerniachowskaWZ23}, \href{../works/GeibingerKKMMW21.pdf}{GeibingerKKMMW21}~\cite{GeibingerKKMMW21}, \href{../works/HamC16.pdf}{HamC16}~\cite{HamC16}, \href{../works/NovaraNH16.pdf}{NovaraNH16}~\cite{NovaraNH16}, \href{../works/HarjunkoskiMBC14.pdf}{HarjunkoskiMBC14}~\cite{HarjunkoskiMBC14}\\
\index{potash industry}\index{Industries!potash industry}potash industry &  1.00 &  &  & \href{../works/Astrand21.pdf}{Astrand21}~\cite{Astrand21}, \href{../works/Astrand0F21.pdf}{Astrand0F21}~\cite{Astrand0F21}, \href{../works/AstrandJZ20.pdf}{AstrandJZ20}~\cite{AstrandJZ20}, \href{../works/AstrandJZ18.pdf}{AstrandJZ18}~\cite{AstrandJZ18}\\
\index{power industry}\index{Industries!power industry}power industry &  1.00 & \href{../works/Froger16.pdf}{Froger16}~\cite{Froger16} &  & \href{../works/FrostD98.pdf}{FrostD98}~\cite{FrostD98}\\
\index{printing industry}\index{Industries!printing industry}printing industry &  1.00 & \href{../works/Lunardi20.pdf}{Lunardi20}~\cite{Lunardi20} & \href{../works/LunardiBLRV20.pdf}{LunardiBLRV20}~\cite{LunardiBLRV20} & \href{../works/BourreauGGLT22.pdf}{BourreauGGLT22}~\cite{BourreauGGLT22}\\
\index{process industry}\index{Industries!process industry}process industry &  1.00 &  & \href{../works/Timpe02.pdf}{Timpe02}~\cite{Timpe02} & \href{../works/Nattaf16.pdf}{Nattaf16}~\cite{Nattaf16}, \href{../works/BlomPS16.pdf}{BlomPS16}~\cite{BlomPS16}, \href{../works/HarjunkoskiMBC14.pdf}{HarjunkoskiMBC14}~\cite{HarjunkoskiMBC14}, \href{../works/HeinzSSW12.pdf}{HeinzSSW12}~\cite{HeinzSSW12}, \href{../works/ZeballosCM10.pdf}{ZeballosCM10}~\cite{ZeballosCM10}, \href{../works/ChenGPSH10.pdf}{ChenGPSH10}~\cite{ChenGPSH10}, \href{../works/TanSD10.pdf}{TanSD10}~\cite{TanSD10}, \href{../works/Jans09.pdf}{Jans09}~\cite{Jans09}, \href{../works/Simonis99.pdf}{Simonis99}~\cite{Simonis99}, \href{../works/Wallace96.pdf}{Wallace96}~\cite{Wallace96}\\
\index{processing industry}\index{Industries!processing industry}processing industry &  1.00 &  & \href{../works/HauderBRPA20.pdf}{HauderBRPA20}~\cite{HauderBRPA20} & \href{../works/KlankeBYE21.pdf}{KlankeBYE21}~\cite{KlankeBYE21}, \href{../works/abs-1902-09244.pdf}{abs-1902-09244}~\cite{abs-1902-09244}, \href{../works/GilesH16.pdf}{GilesH16}~\cite{GilesH16}\\
\index{railway industry}\index{Industries!railway industry}railway industry &  1.00 &  &  & \href{../works/Lemos21.pdf}{Lemos21}~\cite{Lemos21}, \href{../works/Rodriguez07b.pdf}{Rodriguez07b}~\cite{Rodriguez07b}, \href{../works/Geske05.pdf}{Geske05}~\cite{Geske05}\\
\index{repair industry}\index{Industries!repair industry}repair industry &  1.00 &  &  & \href{../works/BoudreaultSLQ22.pdf}{BoudreaultSLQ22}~\cite{BoudreaultSLQ22}\\
\index{retail industry}\index{Industries!retail industry}retail industry &  1.00 &  &  & \href{../works/ChapadosJR11.pdf}{ChapadosJR11}~\cite{ChapadosJR11}\\
\index{semiconductor industry}\index{Industries!semiconductor industry}semiconductor industry &  1.00 &  & \href{../works/HamFC17.pdf}{HamFC17}~\cite{HamFC17}, \href{../works/ZeballosCM10.pdf}{ZeballosCM10}~\cite{ZeballosCM10} & \href{../works/PenzDN23.pdf}{PenzDN23}~\cite{PenzDN23}, \href{../works/QinWSLS21.pdf}{QinWSLS21}~\cite{QinWSLS21}, \href{../works/NattafDYW19.pdf}{NattafDYW19}~\cite{NattafDYW19}, \href{../works/GrimesH15.pdf}{GrimesH15}~\cite{GrimesH15}, \href{../works/BajestaniB15.pdf}{BajestaniB15}~\cite{BajestaniB15}, \href{../works/NovasH12.pdf}{NovasH12}~\cite{NovasH12}, \href{../works/LombardiMRB10.pdf}{LombardiMRB10}~\cite{LombardiMRB10}, \href{../works/Lombardi10.pdf}{Lombardi10}~\cite{Lombardi10}, \href{../works/KrogtLPHJ07.pdf}{KrogtLPHJ07}~\cite{KrogtLPHJ07}\\
\index{semiprocess industry}\index{Industries!semiprocess industry}semiprocess industry &  1.00 &  &  & \href{../works/ChenGPSH10.pdf}{ChenGPSH10}~\cite{ChenGPSH10}\\
\index{service industry}\index{Industries!service industry}service industry &  1.00 &  &  & \href{../works/GurEA19.pdf}{GurEA19}~\cite{GurEA19}, \href{../works/DoomsH08.pdf}{DoomsH08}~\cite{DoomsH08}\\
\index{ship repair industry}\index{Industries!ship repair industry}ship repair industry &  1.00 &  &  & \href{../works/BoudreaultSLQ22.pdf}{BoudreaultSLQ22}~\cite{BoudreaultSLQ22}\\
\index{shipping industry}\index{Industries!shipping industry}shipping industry &  1.00 &  &  & \href{../works/LuZZYW24.pdf}{LuZZYW24}~\cite{LuZZYW24}, \href{../works/Astrand21.pdf}{Astrand21}~\cite{Astrand21}, \href{../works/SacramentoSP20.pdf}{SacramentoSP20}~\cite{SacramentoSP20}, \href{../works/QinDCS20.pdf}{QinDCS20}~\cite{QinDCS20}\\
\index{software industry}\index{Industries!software industry}software industry &  1.00 &  &  & \href{../works/BartakS11.pdf}{BartakS11}~\cite{BartakS11}, \href{../works/Salido10.pdf}{Salido10}~\cite{Salido10}\\
\index{solar cell industry}\index{Industries!solar cell industry}solar cell industry &  1.00 &  &  & \href{../works/Novas19.pdf}{Novas19}~\cite{Novas19}\\
\index{steel industry}\index{Industries!steel industry}steel industry &  1.00 &  & \href{../works/DavenportKRSH07.pdf}{DavenportKRSH07}~\cite{DavenportKRSH07} & \href{../works/LacknerMMWW23.pdf}{LacknerMMWW23}~\cite{LacknerMMWW23}, \href{../works/IsikYA23.pdf}{IsikYA23}~\cite{IsikYA23}, \href{../works/KimCMLLP23.pdf}{KimCMLLP23}~\cite{KimCMLLP23}, \href{../works/OujanaAYB22.pdf}{OujanaAYB22}~\cite{OujanaAYB22}, \href{../works/LacknerMMWW21.pdf}{LacknerMMWW21}~\cite{LacknerMMWW21}, \href{../works/HauderBRPA20.pdf}{HauderBRPA20}~\cite{HauderBRPA20}, \href{../works/abs-1902-09244.pdf}{abs-1902-09244}~\cite{abs-1902-09244}, \href{../works/GoldwaserS18.pdf}{GoldwaserS18}~\cite{GoldwaserS18}, \href{../works/GoldwaserS17.pdf}{GoldwaserS17}~\cite{GoldwaserS17}, \href{../works/KletzanderM17.pdf}{KletzanderM17}~\cite{KletzanderM17}, \href{../works/HeinzSSW12.pdf}{HeinzSSW12}~\cite{HeinzSSW12}, \href{../works/SchausHMCMD11.pdf}{SchausHMCMD11}~\cite{SchausHMCMD11}, \href{../works/GrimesH10.pdf}{GrimesH10}~\cite{GrimesH10}, \href{../works/MercierH07.pdf}{MercierH07}~\cite{MercierH07}, \href{../works/GarganiR07.pdf}{GarganiR07}~\cite{GarganiR07}\\
\index{sugar industry}\index{Industries!sugar industry}sugar industry &  1.00 &  &  & \href{../works/MartinPY01.pdf}{MartinPY01}~\cite{MartinPY01}\\
\index{taxi industry}\index{Industries!taxi industry}taxi industry &  1.00 &  &  & \href{../works/Ham18.pdf}{Ham18}~\cite{Ham18}\\
\index{telecommunication industry}\index{Industries!telecommunication industry}telecommunication industry &  1.00 &  &  & \\
\index{textile industry}\index{Industries!textile industry}textile industry &  1.00 & \href{../works/Mercier-AubinGQ20.pdf}{Mercier-AubinGQ20}~\cite{Mercier-AubinGQ20} &  & \href{../works/ZarandiASC20.pdf}{ZarandiASC20}~\cite{ZarandiASC20}, \href{../works/BessiereHMQW14.pdf}{BessiereHMQW14}~\cite{BessiereHMQW14}\\
\index{tire industry}\index{Industries!tire industry}tire industry &  1.00 &  &  & \href{../works/Jans09.pdf}{Jans09}~\cite{Jans09}\\
\index{tourism industry}\index{Industries!tourism industry}tourism industry &  1.00 &  &  & \href{../works/LiuCGM17.pdf}{LiuCGM17}~\cite{LiuCGM17}\\
\index{trade industry}\index{Industries!trade industry}trade industry &  1.00 &  &  & \href{../works/ParkUJR19.pdf}{ParkUJR19}~\cite{ParkUJR19}\\
\index{transportation industry}\index{Industries!transportation industry}transportation industry &  1.00 &  &  & \href{../works/GoelSHFS15.pdf}{GoelSHFS15}~\cite{GoelSHFS15}\\
\index{wind industry}\index{Industries!wind industry}wind industry &  1.00 & \href{../works/Froger16.pdf}{Froger16}~\cite{Froger16} &  & \\
\end{longtable}
}

\clearpage
\subsection{Concept Type CPSystems}
\label{sec:CPSystems}
\label{CPSystems}
{\scriptsize
\begin{longtable}{p{3cm}r>{\raggedright\arraybackslash}p{6cm}>{\raggedright\arraybackslash}p{6cm}>{\raggedright\arraybackslash}p{8cm}}
\rowcolor{white}\caption{Works for Concepts of Type CPSystems (Total 19 Concepts, 19 Used)}\\ \toprule
\rowcolor{white}Keyword & Weight & High & Medium & Low\\ \midrule\endhead
\bottomrule
\endfoot
\index{CHIP}\index{CPSystems!CHIP}CHIP &  1.00 & \href{../works/TrojetHL11.pdf}{TrojetHL11}~\cite{TrojetHL11}, \href{../works/Simonis07.pdf}{Simonis07}~\cite{Simonis07}, \href{../works/Kuchcinski03.pdf}{Kuchcinski03}~\cite{Kuchcinski03}, \href{../works/BosiM2001.pdf}{BosiM2001}~\cite{BosiM2001}, \href{../works/TrentesauxPT01.pdf}{TrentesauxPT01}~\cite{TrentesauxPT01}, \href{../works/SimonisCK00.pdf}{SimonisCK00}~\cite{SimonisCK00}, \href{../works/Simonis99.pdf}{Simonis99}~\cite{Simonis99}, \href{../works/GruianK98.pdf}{GruianK98}~\cite{GruianK98}, \href{../works/PapeB97.pdf}{PapeB97}~\cite{PapeB97}, \href{../works/Wallace96.pdf}{Wallace96}~\cite{Wallace96}, \href{../works/SimonisC95.pdf}{SimonisC95}~\cite{SimonisC95}, \href{../works/Simonis95a.pdf}{Simonis95a}~\cite{Simonis95a}, \href{../works/Simonis95.pdf}{Simonis95}~\cite{Simonis95}, \href{../works/Goltz95.pdf}{Goltz95}~\cite{Goltz95}, \href{../works/BeldiceanuC94.pdf}{BeldiceanuC94}~\cite{BeldiceanuC94}, \href{../works/AggounB93.pdf}{AggounB93}~\cite{AggounB93}, \href{../works/DincbasS91.pdf}{DincbasS91}~\cite{DincbasS91}, \href{../works/DincbasSH90.pdf}{DincbasSH90}~\cite{DincbasSH90} & \href{../works/ArmstrongGOS21.pdf}{ArmstrongGOS21}~\cite{ArmstrongGOS21}, \href{../works/YangSS19.pdf}{YangSS19}~\cite{YangSS19}, \href{../works/LaborieRSV18.pdf}{LaborieRSV18}~\cite{LaborieRSV18}, \href{../works/HookerH17.pdf}{HookerH17}~\cite{HookerH17}, \href{../works/Geske05.pdf}{Geske05}~\cite{Geske05}, \href{../works/PoderBS04.pdf}{PoderBS04}~\cite{PoderBS04}, \href{../works/Timpe02.pdf}{Timpe02}~\cite{Timpe02}, \href{../works/BeldiceanuC01.pdf}{BeldiceanuC01}~\cite{BeldiceanuC01}, \href{../works/HookerO99.pdf}{HookerO99}~\cite{HookerO99}, \href{../works/JoLLH99.pdf}{JoLLH99}~\cite{JoLLH99}, \href{../works/Beck99.pdf}{Beck99}~\cite{Beck99}, \href{../works/RodosekW98.pdf}{RodosekW98}~\cite{RodosekW98}, \href{../works/Zhou97.pdf}{Zhou97}~\cite{Zhou97}, \href{../works/LammaMM97.pdf}{LammaMM97}~\cite{LammaMM97} & \href{../works/PrataAN23.pdf}{PrataAN23}~\cite{PrataAN23}, \href{../works/TardivoDFMP23.pdf}{TardivoDFMP23}~\cite{TardivoDFMP23}, \href{../works/KameugneFND23.pdf}{KameugneFND23}~\cite{KameugneFND23}, \href{../works/BourreauGGLT22.pdf}{BourreauGGLT22}~\cite{BourreauGGLT22}, \href{../works/PopovicCGNC22.pdf}{PopovicCGNC22}~\cite{PopovicCGNC22}, \href{../works/LuoB22.pdf}{LuoB22}~\cite{LuoB22}, \href{../works/FetgoD22.pdf}{FetgoD22}~\cite{FetgoD22}, \href{../works/Godet21a.pdf}{Godet21a}~\cite{Godet21a}, \href{../works/KlankeBYE21.pdf}{KlankeBYE21}~\cite{KlankeBYE21}, \href{../works/AbidinK20.pdf}{AbidinK20}~\cite{AbidinK20}, \href{../works/GodetLHS20.pdf}{GodetLHS20}~\cite{GodetLHS20}, \href{../works/Caballero19.pdf}{Caballero19}~\cite{Caballero19}, \href{../works/abs-1902-01193.pdf}{abs-1902-01193}~\cite{abs-1902-01193}, \href{../works/GoldwaserS18.pdf}{GoldwaserS18}~\cite{GoldwaserS18}, \href{../works/BaptisteB18.pdf}{BaptisteB18}~\cite{BaptisteB18}, \href{../works/KameugneFGOQ18.pdf}{KameugneFGOQ18}~\cite{KameugneFGOQ18}, \href{../works/CauwelaertLS18.pdf}{CauwelaertLS18}~\cite{CauwelaertLS18}, \href{../works/GokgurHO18.pdf}{GokgurHO18}~\cite{GokgurHO18}, \href{../works/MossigeGSMC17.pdf}{MossigeGSMC17}~\cite{MossigeGSMC17}...\href{../works/HookerOTK00.pdf}{HookerOTK00}~\cite{HookerOTK00}, \href{../works/SakkoutW00.pdf}{SakkoutW00}~\cite{SakkoutW00}, \href{../works/HarjunkoskiJG00.pdf}{HarjunkoskiJG00}~\cite{HarjunkoskiJG00}, \href{../works/AbdennadherS99.pdf}{AbdennadherS99}~\cite{AbdennadherS99}, \href{../works/KorbaaYG99.pdf}{KorbaaYG99}~\cite{KorbaaYG99}, \href{../works/PapaB98.pdf}{PapaB98}~\cite{PapaB98}, \href{../works/BaptisteP97.pdf}{BaptisteP97}~\cite{BaptisteP97}, \href{../works/Colombani96.pdf}{Colombani96}~\cite{Colombani96}, \href{../works/BlazewiczDP96.pdf}{BlazewiczDP96}~\cite{BlazewiczDP96}, \href{../works/WeilHFP95.pdf}{WeilHFP95}~\cite{WeilHFP95} (Total: 96)\\
\index{CPO}\index{CPSystems!CPO}CPO &  1.00 & \href{../works/CzerniachowskaWZ23.pdf}{CzerniachowskaWZ23}~\cite{CzerniachowskaWZ23}, \href{../works/NaderiRR23.pdf}{NaderiRR23}~\cite{NaderiRR23}, \href{../works/JuvinHL23a.pdf}{JuvinHL23a}~\cite{JuvinHL23a}, \href{../works/LacknerMMWW23.pdf}{LacknerMMWW23}~\cite{LacknerMMWW23}, \href{../works/JuvinHHL23.pdf}{JuvinHHL23}~\cite{JuvinHHL23}, \href{../works/Bit-Monnot23.pdf}{Bit-Monnot23}~\cite{Bit-Monnot23}, \href{../works/NaderiBZ23.pdf}{NaderiBZ23}~\cite{NaderiBZ23}, \href{../works/WinterMMW22.pdf}{WinterMMW22}~\cite{WinterMMW22}, \href{../works/ColT22.pdf}{ColT22}~\cite{ColT22}, \href{../works/ZhangBB22.pdf}{ZhangBB22}~\cite{ZhangBB22}, \href{../works/NaderiBZ22.pdf}{NaderiBZ22}~\cite{NaderiBZ22}, \href{../works/RoshanaeiN21.pdf}{RoshanaeiN21}~\cite{RoshanaeiN21}, \href{../works/Zahout21.pdf}{Zahout21}~\cite{Zahout21}, \href{../works/LacknerMMWW21.pdf}{LacknerMMWW21}~\cite{LacknerMMWW21}, \href{../works/Groleaz21.pdf}{Groleaz21}~\cite{Groleaz21}, \href{../works/ArmstrongGOS21.pdf}{ArmstrongGOS21}~\cite{ArmstrongGOS21}, \href{../works/ThomasKS20.pdf}{ThomasKS20}~\cite{ThomasKS20}, \href{../works/Lunardi20.pdf}{Lunardi20}~\cite{Lunardi20}, \href{../works/NattafM20.pdf}{NattafM20}~\cite{NattafM20}...\href{../works/GeibingerMM19.pdf}{GeibingerMM19}~\cite{GeibingerMM19}, \href{../works/ColT19.pdf}{ColT19}~\cite{ColT19}, \href{../works/ColT2019a.pdf}{ColT2019a}~\cite{ColT2019a}, \href{../works/MalapertN19.pdf}{MalapertN19}~\cite{MalapertN19}, \href{../works/CappartTSR18.pdf}{CappartTSR18}~\cite{CappartTSR18}, \href{../works/LaborieRSV18.pdf}{LaborieRSV18}~\cite{LaborieRSV18}, \href{../works/KreterSS17.pdf}{KreterSS17}~\cite{KreterSS17}, \href{../works/GoelSHFS15.pdf}{GoelSHFS15}~\cite{GoelSHFS15}, \href{../works/PraletLJ15.pdf}{PraletLJ15}~\cite{PraletLJ15}, \href{../works/Laborie09.pdf}{Laborie09}~\cite{Laborie09} (Total: 34) & \href{../works/AalianPG23.pdf}{AalianPG23}~\cite{AalianPG23}, \href{../works/JuvinHL22.pdf}{JuvinHL22}~\cite{JuvinHL22}, \href{../works/HamP21.pdf}{HamP21}~\cite{HamP21}, \href{../works/abs-1911-04766.pdf}{abs-1911-04766}~\cite{abs-1911-04766}, \href{../works/Dejemeppe16.pdf}{Dejemeppe16}~\cite{Dejemeppe16}, \href{../works/GrimesH15.pdf}{GrimesH15}~\cite{GrimesH15}, \href{../works/NuijtenA96.pdf}{NuijtenA96}~\cite{NuijtenA96}, \href{../works/NuijtenA94.pdf}{NuijtenA94}~\cite{NuijtenA94} & \href{../works/JuvinHL23.pdf}{JuvinHL23}~\cite{JuvinHL23}, \href{../works/PovedaAA23.pdf}{PovedaAA23}~\cite{PovedaAA23}, \href{../works/NaderiBZ22a.pdf}{NaderiBZ22a}~\cite{NaderiBZ22a}, \href{../works/OujanaAYB22.pdf}{OujanaAYB22}~\cite{OujanaAYB22}, \href{../works/GeibingerMM21.pdf}{GeibingerMM21}~\cite{GeibingerMM21}, \href{../works/abs-2102-08778.pdf}{abs-2102-08778}~\cite{abs-2102-08778}, \href{../works/TangB20.pdf}{TangB20}~\cite{TangB20}, \href{../works/Caballero19.pdf}{Caballero19}~\cite{Caballero19}, \href{../works/Ham18a.pdf}{Ham18a}~\cite{Ham18a}, \href{../works/Laborie18a.pdf}{Laborie18a}~\cite{Laborie18a}, \href{../works/Pralet17.pdf}{Pralet17}~\cite{Pralet17}, \href{../works/VilimLS15.pdf}{VilimLS15}~\cite{VilimLS15}, \href{../works/BartakSR10.pdf}{BartakSR10}~\cite{BartakSR10}, \href{../works/Vilim09.pdf}{Vilim09}~\cite{Vilim09}, \href{../works/GarridoAO09.pdf}{GarridoAO09}~\cite{GarridoAO09}, \href{../works/GarridoOS08.pdf}{GarridoOS08}~\cite{GarridoOS08}, \href{../works/BeldiceanuC94.pdf}{BeldiceanuC94}~\cite{BeldiceanuC94}\\
\index{Choco Solver}\index{CPSystems!Choco Solver}Choco Solver &  1.00 & \href{../works/TasselGS23.pdf}{TasselGS23}~\cite{TasselGS23}, \href{../works/abs-2306-05747.pdf}{abs-2306-05747}~\cite{abs-2306-05747}, \href{../works/Godet21a.pdf}{Godet21a}~\cite{Godet21a}, \href{../works/LiuLH18.pdf}{LiuLH18}~\cite{LiuLH18}, \href{../works/German18.pdf}{German18}~\cite{German18}, \href{../works/Fahimi16.pdf}{Fahimi16}~\cite{Fahimi16}, \href{../works/LetortCB15.pdf}{LetortCB15}~\cite{LetortCB15}, \href{../works/Derrien15.pdf}{Derrien15}~\cite{Derrien15}, \href{../works/LetortCB13.pdf}{LetortCB13}~\cite{LetortCB13}, \href{../works/Letort13.pdf}{Letort13}~\cite{Letort13}, \href{../works/OuelletQ13.pdf}{OuelletQ13}~\cite{OuelletQ13}, \href{../works/LetortBC12.pdf}{LetortBC12}~\cite{LetortBC12}, \href{../works/Malapert11.pdf}{Malapert11}~\cite{Malapert11}, \href{../works/Menana11.pdf}{Menana11}~\cite{Menana11}, \href{../works/abs-0907-0939.pdf}{abs-0907-0939}~\cite{abs-0907-0939}, \href{../works/GarridoAO09.pdf}{GarridoAO09}~\cite{GarridoAO09}, \href{../works/GrimesHM09.pdf}{GrimesHM09}~\cite{GrimesHM09}, \href{../works/GarridoOS08.pdf}{GarridoOS08}~\cite{GarridoOS08}, \href{../works/Elkhyari03.pdf}{Elkhyari03}~\cite{Elkhyari03}, \href{../works/BenoistGR02.pdf}{BenoistGR02}~\cite{BenoistGR02} & \href{../works/KameugneFND23.pdf}{KameugneFND23}~\cite{KameugneFND23}, \href{../works/MullerMKP22.pdf}{MullerMKP22}~\cite{MullerMKP22}, \href{../works/FetgoD22.pdf}{FetgoD22}~\cite{FetgoD22}, \href{../works/AntuoriHHEN21.pdf}{AntuoriHHEN21}~\cite{AntuoriHHEN21}, \href{../works/AntuoriHHEN20.pdf}{AntuoriHHEN20}~\cite{AntuoriHHEN20}, \href{../works/LiuLH19.pdf}{LiuLH19}~\cite{LiuLH19}, \href{../works/LiuLH19a.pdf}{LiuLH19a}~\cite{LiuLH19a}, \href{../works/FahimiOQ18.pdf}{FahimiOQ18}~\cite{FahimiOQ18}, \href{../works/LaborieRSV18.pdf}{LaborieRSV18}~\cite{LaborieRSV18}, \href{../works/KameugneFGOQ18.pdf}{KameugneFGOQ18}~\cite{KameugneFGOQ18}, \href{../works/Froger16.pdf}{Froger16}~\cite{Froger16}, \href{../works/GayHS15.pdf}{GayHS15}~\cite{GayHS15}, \href{../works/KoschB14.pdf}{KoschB14}~\cite{KoschB14}, \href{../works/DerrienP14.pdf}{DerrienP14}~\cite{DerrienP14}, \href{../works/DerrienPZ14.pdf}{DerrienPZ14}~\cite{DerrienPZ14}, \href{../works/Kameugne14.pdf}{Kameugne14}~\cite{Kameugne14}, \href{../works/MeskensDL13.pdf}{MeskensDL13}~\cite{MeskensDL13}, \href{../works/Clercq12.pdf}{Clercq12}~\cite{Clercq12}, \href{../works/MalapertCGJLR12.pdf}{MalapertCGJLR12}~\cite{MalapertCGJLR12}, \href{../works/MalapertGR12.pdf}{MalapertGR12}~\cite{MalapertGR12}, \href{../works/ClercqPBJ11.pdf}{ClercqPBJ11}~\cite{ClercqPBJ11}, \href{../works/HermenierDL11.pdf}{HermenierDL11}~\cite{HermenierDL11}, \href{../works/MeskensDHG11.pdf}{MeskensDHG11}~\cite{MeskensDHG11}, \href{../works/HladikCDJ08.pdf}{HladikCDJ08}~\cite{HladikCDJ08} & \href{../works/FalqueALM24.pdf}{FalqueALM24}~\cite{FalqueALM24}, \href{../works/BourreauGGLT22.pdf}{BourreauGGLT22}~\cite{BourreauGGLT22}, \href{../works/OuelletQ22.pdf}{OuelletQ22}~\cite{OuelletQ22}, \href{../works/Groleaz21.pdf}{Groleaz21}~\cite{Groleaz21}, \href{../works/Ham20a.pdf}{Ham20a}~\cite{Ham20a}, \href{../works/GodetLHS20.pdf}{GodetLHS20}~\cite{GodetLHS20}, \href{../works/YangSS19.pdf}{YangSS19}~\cite{YangSS19}, \href{../works/OuelletQ18.pdf}{OuelletQ18}~\cite{OuelletQ18}, \href{../works/GingrasQ16.pdf}{GingrasQ16}~\cite{GingrasQ16}, \href{../works/AmadiniGM16.pdf}{AmadiniGM16}~\cite{AmadiniGM16}, \href{../works/Madi-WambaB16.pdf}{Madi-WambaB16}~\cite{Madi-WambaB16}, \href{../works/MurphyMB15.pdf}{MurphyMB15}~\cite{MurphyMB15}, \href{../works/EvenSH15.pdf}{EvenSH15}~\cite{EvenSH15}, \href{../works/GrimesH15.pdf}{GrimesH15}~\cite{GrimesH15}, \href{../works/EvenSH15a.pdf}{EvenSH15a}~\cite{EvenSH15a}, \href{../works/BessiereHMQW14.pdf}{BessiereHMQW14}~\cite{BessiereHMQW14}, \href{../works/MalapertCGJLR13.pdf}{MalapertCGJLR13}~\cite{MalapertCGJLR13}, \href{../works/SimonisH11.pdf}{SimonisH11}~\cite{SimonisH11}, \href{../works/BartakSR10.pdf}{BartakSR10}~\cite{BartakSR10}, \href{../works/RossiTHP07.pdf}{RossiTHP07}~\cite{RossiTHP07}, \href{../works/CorreaLR07.pdf}{CorreaLR07}~\cite{CorreaLR07}, \href{../works/CambazardJ05.pdf}{CambazardJ05}~\cite{CambazardJ05}, \href{../works/Baptiste02.pdf}{Baptiste02}~\cite{Baptiste02}\\
\index{Chuffed}\index{CPSystems!Chuffed}Chuffed &  1.00 & \href{../works/LacknerMMWW23.pdf}{LacknerMMWW23}~\cite{LacknerMMWW23}, \href{../works/PovedaAA23.pdf}{PovedaAA23}~\cite{PovedaAA23}, \href{../works/MullerMKP22.pdf}{MullerMKP22}~\cite{MullerMKP22}, \href{../works/BoudreaultSLQ22.pdf}{BoudreaultSLQ22}~\cite{BoudreaultSLQ22}, \href{../works/LacknerMMWW21.pdf}{LacknerMMWW21}~\cite{LacknerMMWW21}, \href{../works/GeibingerMM21.pdf}{GeibingerMM21}~\cite{GeibingerMM21}, \href{../works/Godet21a.pdf}{Godet21a}~\cite{Godet21a}, \href{../works/ArmstrongGOS21.pdf}{ArmstrongGOS21}~\cite{ArmstrongGOS21}, \href{../works/KoehlerBFFHPSSS21.pdf}{KoehlerBFFHPSSS21}~\cite{KoehlerBFFHPSSS21}, \href{../works/GodetLHS20.pdf}{GodetLHS20}~\cite{GodetLHS20}, \href{../works/WallaceY20.pdf}{WallaceY20}~\cite{WallaceY20}, \href{../works/abs-1911-04766.pdf}{abs-1911-04766}~\cite{abs-1911-04766}, \href{../works/KreterSSZ18.pdf}{KreterSSZ18}~\cite{KreterSSZ18}, \href{../works/KreterSS17.pdf}{KreterSS17}~\cite{KreterSS17}, \href{../works/CarlssonJL17.pdf}{CarlssonJL17}~\cite{CarlssonJL17}, \href{../works/YoungFS17.pdf}{YoungFS17}~\cite{YoungFS17}, \href{../works/SzerediS16.pdf}{SzerediS16}~\cite{SzerediS16}, \href{../works/KreterSS15.pdf}{KreterSS15}~\cite{KreterSS15} & \href{../works/GoldwaserS18.pdf}{GoldwaserS18}~\cite{GoldwaserS18} & \href{../works/FrimodigECM23.pdf}{FrimodigECM23}~\cite{FrimodigECM23}, \href{../works/Caballero19.pdf}{Caballero19}~\cite{Caballero19}, \href{../works/SchuttS16.pdf}{SchuttS16}~\cite{SchuttS16}\\
\index{Claire}\index{CPSystems!Claire}Claire &  1.00 & \href{../works/Nattaf16.pdf}{Nattaf16}~\cite{Nattaf16}, \href{../works/Siala15a.pdf}{Siala15a}~\cite{Siala15a}, \href{../works/Siala15.pdf}{Siala15}~\cite{Siala15}, \href{../works/Malapert11.pdf}{Malapert11}~\cite{Malapert11}, \href{../works/Demassey03.pdf}{Demassey03}~\cite{Demassey03}, \href{../works/Elkhyari03.pdf}{Elkhyari03}~\cite{Elkhyari03}, \href{../works/BaptisteP00.pdf}{BaptisteP00}~\cite{BaptisteP00}, \href{../works/PapeB97.pdf}{PapeB97}~\cite{PapeB97} & \href{../works/Zahout21.pdf}{Zahout21}~\cite{Zahout21}, \href{../works/Menana11.pdf}{Menana11}~\cite{Menana11}, \href{../works/BaptisteP97.pdf}{BaptisteP97}~\cite{BaptisteP97} & \href{../works/HebrardALLCMR22.pdf}{HebrardALLCMR22}~\cite{HebrardALLCMR22}, \href{../works/Godet21a.pdf}{Godet21a}~\cite{Godet21a}, \href{../works/HanenKP21.pdf}{HanenKP21}~\cite{HanenKP21}, \href{../works/PachecoPR19.pdf}{PachecoPR19}~\cite{PachecoPR19}, \href{../works/Derrien15.pdf}{Derrien15}~\cite{Derrien15}, \href{../works/Kameugne14.pdf}{Kameugne14}~\cite{Kameugne14}, \href{../works/Letort13.pdf}{Letort13}~\cite{Letort13}, \href{../works/Baptiste02.pdf}{Baptiste02}~\cite{Baptiste02}, \href{../works/BenoistGR02.pdf}{BenoistGR02}~\cite{BenoistGR02}, \href{../works/DraperJCJ99.pdf}{DraperJCJ99}~\cite{DraperJCJ99}, \href{../works/BaptistePN99.pdf}{BaptistePN99}~\cite{BaptistePN99}, \href{../works/PapaB98.pdf}{PapaB98}~\cite{PapaB98}\\
\index{Cplex}\index{CPSystems!Cplex}Cplex &  1.00 & \href{../works/GuoZ23.pdf}{GuoZ23}~\cite{GuoZ23}, \href{../works/ZhuSZW23.pdf}{ZhuSZW23}~\cite{ZhuSZW23}, \href{../works/CzerniachowskaWZ23.pdf}{CzerniachowskaWZ23}~\cite{CzerniachowskaWZ23}, \href{../works/NaderiBZ23.pdf}{NaderiBZ23}~\cite{NaderiBZ23}, \href{../works/AfsarVPG23.pdf}{AfsarVPG23}~\cite{AfsarVPG23}, \href{../works/NaderiBZR23.pdf}{NaderiBZR23}~\cite{NaderiBZR23}, \href{../works/Adelgren2023.pdf}{Adelgren2023}~\cite{Adelgren2023}, \href{../works/NaderiRR23.pdf}{NaderiRR23}~\cite{NaderiRR23}, \href{../works/MengGRZSC22.pdf}{MengGRZSC22}~\cite{MengGRZSC22}, \href{../works/ElciOH22.pdf}{ElciOH22}~\cite{ElciOH22}, \href{../works/OrnekOS20.pdf}{OrnekOS20}~\cite{OrnekOS20}, \href{../works/SubulanC22.pdf}{SubulanC22}~\cite{SubulanC22}, \href{../works/EtminaniesfahaniGNMS22.pdf}{EtminaniesfahaniGNMS22}~\cite{EtminaniesfahaniGNMS22}, \href{../works/AwadMDMT22.pdf}{AwadMDMT22}~\cite{AwadMDMT22}, \href{../works/EmdeZD22.pdf}{EmdeZD22}~\cite{EmdeZD22}, \href{../works/MullerMKP22.pdf}{MullerMKP22}~\cite{MullerMKP22}, \href{../works/NaderiBZ22.pdf}{NaderiBZ22}~\cite{NaderiBZ22}, \href{../works/BourreauGGLT22.pdf}{BourreauGGLT22}~\cite{BourreauGGLT22}, \href{../works/WinterMMW22.pdf}{WinterMMW22}~\cite{WinterMMW22}...\href{../works/HeinzSB13.pdf}{HeinzSB13}~\cite{HeinzSB13}, \href{../works/CireCH13.pdf}{CireCH13}~\cite{CireCH13}, \href{../works/Malapert11.pdf}{Malapert11}~\cite{Malapert11}, \href{../works/AchterbergBKW08.pdf}{AchterbergBKW08}~\cite{AchterbergBKW08}, \href{../works/Hooker07.pdf}{Hooker07}~\cite{Hooker07}, \href{../works/LorigeonBB02.pdf}{LorigeonBB02}~\cite{LorigeonBB02}, \href{../works/Mason01.pdf}{Mason01}~\cite{Mason01}, \href{../works/HarjunkoskiJG00.pdf}{HarjunkoskiJG00}~\cite{HarjunkoskiJG00}, \href{../works/RodosekWH99.pdf}{RodosekWH99}~\cite{RodosekWH99}, \href{../works/HookerO99.pdf}{HookerO99}~\cite{HookerO99} (Total: 69) & \href{../works/BonninMNE24.pdf}{BonninMNE24}~\cite{BonninMNE24}, \href{../works/LacknerMMWW23.pdf}{LacknerMMWW23}~\cite{LacknerMMWW23}, \href{../works/Fatemi-AnarakiTFV23.pdf}{Fatemi-AnarakiTFV23}~\cite{Fatemi-AnarakiTFV23}, \href{../works/Mehdizadeh-Somarin23.pdf}{Mehdizadeh-Somarin23}~\cite{Mehdizadeh-Somarin23}, \href{../works/AbreuNP23.pdf}{AbreuNP23}~\cite{AbreuNP23}, \href{../works/MarliereSPR23.pdf}{MarliereSPR23}~\cite{MarliereSPR23}, \href{../works/IsikYA23.pdf}{IsikYA23}~\cite{IsikYA23}, \href{../works/CampeauG22.pdf}{CampeauG22}~\cite{CampeauG22}, \href{../works/LuoB22.pdf}{LuoB22}~\cite{LuoB22}, \href{../works/TouatBT22.pdf}{TouatBT22}~\cite{TouatBT22}, \href{../works/CilKLO22.pdf}{CilKLO22}~\cite{CilKLO22}, \href{../works/NaderiBZ22a.pdf}{NaderiBZ22a}~\cite{NaderiBZ22a}, \href{../works/YunusogluY22.pdf}{YunusogluY22}~\cite{YunusogluY22}, \href{../works/ColT22.pdf}{ColT22}~\cite{ColT22}, \href{../works/LacknerMMWW21.pdf}{LacknerMMWW21}~\cite{LacknerMMWW21}, \href{../works/MengLZB21.pdf}{MengLZB21}~\cite{MengLZB21}, \href{../works/Zahout21.pdf}{Zahout21}~\cite{Zahout21}, \href{../works/KovacsTKSG21.pdf}{KovacsTKSG21}~\cite{KovacsTKSG21}, \href{../works/QinWSLS21.pdf}{QinWSLS21}~\cite{QinWSLS21}...\href{../works/KanetAG04.pdf}{KanetAG04}~\cite{KanetAG04}, \href{../works/PerronSF04.pdf}{PerronSF04}~\cite{PerronSF04}, \href{../works/Hooker04.pdf}{Hooker04}~\cite{Hooker04}, \href{../works/DannaP03.pdf}{DannaP03}~\cite{DannaP03}, \href{../works/Demassey03.pdf}{Demassey03}~\cite{Demassey03}, \href{../works/BeckR03.pdf}{BeckR03}~\cite{BeckR03}, \href{../works/HarjunkoskiG02.pdf}{HarjunkoskiG02}~\cite{HarjunkoskiG02}, \href{../works/EreminW01.pdf}{EreminW01}~\cite{EreminW01}, \href{../works/JainG01.pdf}{JainG01}~\cite{JainG01}, \href{../works/SakkoutW00.pdf}{SakkoutW00}~\cite{SakkoutW00} (Total: 81) & \href{../works/LuZZYW24.pdf}{LuZZYW24}~\cite{LuZZYW24}, \href{../works/BofillCGGPSV23.pdf}{BofillCGGPSV23}~\cite{BofillCGGPSV23}, \href{../works/JuvinHL23.pdf}{JuvinHL23}~\cite{JuvinHL23}, \href{../works/AbreuPNF23.pdf}{AbreuPNF23}~\cite{AbreuPNF23}, \href{../works/PovedaAA23.pdf}{PovedaAA23}~\cite{PovedaAA23}, \href{../works/AlakaP23.pdf}{AlakaP23}~\cite{AlakaP23}, \href{../works/SquillaciPR23.pdf}{SquillaciPR23}~\cite{SquillaciPR23}, \href{../works/FrimodigECM23.pdf}{FrimodigECM23}~\cite{FrimodigECM23}, \href{../works/GurPAE23.pdf}{GurPAE23}~\cite{GurPAE23}, \href{../works/YuraszeckMCCR23.pdf}{YuraszeckMCCR23}~\cite{YuraszeckMCCR23}, \href{../works/JuvinHL23a.pdf}{JuvinHL23a}~\cite{JuvinHL23a}, \href{../works/AlfieriGPS23.pdf}{AlfieriGPS23}~\cite{AlfieriGPS23}, \href{../works/PenzDN23.pdf}{PenzDN23}~\cite{PenzDN23}, \href{../works/AalianPG23.pdf}{AalianPG23}~\cite{AalianPG23}, \href{../works/JuvinHL22.pdf}{JuvinHL22}~\cite{JuvinHL22}, \href{../works/PohlAK22.pdf}{PohlAK22}~\cite{PohlAK22}, \href{../works/GhandehariK22.pdf}{GhandehariK22}~\cite{GhandehariK22}, \href{../works/AbreuN22.pdf}{AbreuN22}~\cite{AbreuN22}, \href{../works/FarsiTM22.pdf}{FarsiTM22}~\cite{FarsiTM22}...\href{../works/BeniniBGM05a.pdf}{BeniniBGM05a}~\cite{BeniniBGM05a}, \href{../works/Hooker05a.pdf}{Hooker05a}~\cite{Hooker05a}, \href{../works/DemasseyAM05.pdf}{DemasseyAM05}~\cite{DemasseyAM05}, \href{../works/ElfJR03.pdf}{ElfJR03}~\cite{ElfJR03}, \href{../works/HookerO03.pdf}{HookerO03}~\cite{HookerO03}, \href{../works/Timpe02.pdf}{Timpe02}~\cite{Timpe02}, \href{../works/VerfaillieL01.pdf}{VerfaillieL01}~\cite{VerfaillieL01}, \href{../works/HookerOTK00.pdf}{HookerOTK00}~\cite{HookerOTK00}, \href{../works/BruckerK00.pdf}{BruckerK00}~\cite{BruckerK00}, \href{../works/Refalo00.pdf}{Refalo00}~\cite{Refalo00} (Total: 151)\\
\index{ECLiPSe}\index{CPSystems!ECLiPSe}ECLiPSe &  1.00 & \href{../works/BadicaBI20.pdf}{BadicaBI20}~\cite{BadicaBI20}, \href{../works/BadicaBIL19.pdf}{BadicaBIL19}~\cite{BadicaBIL19}, \href{../works/RoePS05.pdf}{RoePS05}~\cite{RoePS05}, \href{../works/EreminW01.pdf}{EreminW01}~\cite{EreminW01}, \href{../works/RodosekWH99.pdf}{RodosekWH99}~\cite{RodosekWH99}, \href{../works/RodosekW98.pdf}{RodosekW98}~\cite{RodosekW98} & \href{../works/Kameugne14.pdf}{Kameugne14}~\cite{Kameugne14}, \href{../works/SchuttFSW11.pdf}{SchuttFSW11}~\cite{SchuttFSW11}, \href{../works/Malapert11.pdf}{Malapert11}~\cite{Malapert11}, \href{../works/Schutt11.pdf}{Schutt11}~\cite{Schutt11}, \href{../works/MilanoW09.pdf}{MilanoW09}~\cite{MilanoW09}, \href{../works/LiW08.pdf}{LiW08}~\cite{LiW08}, \href{../works/Wallace06.pdf}{Wallace06}~\cite{Wallace06}, \href{../works/MilanoW06.pdf}{MilanoW06}~\cite{MilanoW06}, \href{../works/KanetAG04.pdf}{KanetAG04}~\cite{KanetAG04}, \href{../works/KamarainenS02.pdf}{KamarainenS02}~\cite{KamarainenS02}, \href{../works/Simonis99.pdf}{Simonis99}~\cite{Simonis99}, \href{../works/Darby-DowmanLMZ97.pdf}{Darby-DowmanLMZ97}~\cite{Darby-DowmanLMZ97}, \href{../works/Wallace96.pdf}{Wallace96}~\cite{Wallace96} & \href{../works/FanXG21.pdf}{FanXG21}~\cite{FanXG21}, \href{../works/MejiaY20.pdf}{MejiaY20}~\cite{MejiaY20}, \href{../works/WikarekS19.pdf}{WikarekS19}~\cite{WikarekS19}, \href{../works/HookerH17.pdf}{HookerH17}~\cite{HookerH17}, \href{../works/HarjunkoskiMBC14.pdf}{HarjunkoskiMBC14}~\cite{HarjunkoskiMBC14}, \href{../works/Clercq12.pdf}{Clercq12}~\cite{Clercq12}, \href{../works/ZeballosNH11.pdf}{ZeballosNH11}~\cite{ZeballosNH11}, \href{../works/ZeballosQH10.pdf}{ZeballosQH10}~\cite{ZeballosQH10}, \href{../works/LombardiMRB10.pdf}{LombardiMRB10}~\cite{LombardiMRB10}, \href{../works/Zeballos10.pdf}{Zeballos10}~\cite{Zeballos10}, \href{../works/SchuttFSW09.pdf}{SchuttFSW09}~\cite{SchuttFSW09}, \href{../works/BeniniBGM06.pdf}{BeniniBGM06}~\cite{BeniniBGM06}, \href{../works/QuirogaZH05.pdf}{QuirogaZH05}~\cite{QuirogaZH05}, \href{../works/BeniniBGM05.pdf}{BeniniBGM05}~\cite{BeniniBGM05}, \href{../works/ChuX05.pdf}{ChuX05}~\cite{ChuX05}, \href{../works/HarjunkoskiG02.pdf}{HarjunkoskiG02}~\cite{HarjunkoskiG02}, \href{../works/Baptiste02.pdf}{Baptiste02}~\cite{Baptiste02}, \href{../works/MartinPY01.pdf}{MartinPY01}~\cite{MartinPY01}, \href{../works/JainG01.pdf}{JainG01}~\cite{JainG01}, \href{../works/HarjunkoskiJG00.pdf}{HarjunkoskiJG00}~\cite{HarjunkoskiJG00}, \href{../works/PesantGPR99.pdf}{PesantGPR99}~\cite{PesantGPR99}, \href{../works/LammaMM97.pdf}{LammaMM97}~\cite{LammaMM97}\\
\index{Gecode}\index{CPSystems!Gecode}Gecode &  1.00 & \href{../works/TardivoDFMP23.pdf}{TardivoDFMP23}~\cite{TardivoDFMP23}, \href{../works/Astrand21.pdf}{Astrand21}~\cite{Astrand21}, \href{../works/GokGSTO20.pdf}{GokGSTO20}~\cite{GokGSTO20}, \href{../works/BadicaBI20.pdf}{BadicaBI20}~\cite{BadicaBI20}, \href{../works/AstrandJZ20.pdf}{AstrandJZ20}~\cite{AstrandJZ20}, \href{../works/BadicaBIL19.pdf}{BadicaBIL19}~\cite{BadicaBIL19}, \href{../works/Fahimi16.pdf}{Fahimi16}~\cite{Fahimi16}, \href{../works/SzerediS16.pdf}{SzerediS16}~\cite{SzerediS16}, \href{../works/ZhouGL15.pdf}{ZhouGL15}~\cite{ZhouGL15}, \href{../works/GayHS15.pdf}{GayHS15}~\cite{GayHS15}, \href{../works/Kameugne14.pdf}{Kameugne14}~\cite{Kameugne14}, \href{../works/KameugneFSN14.pdf}{KameugneFSN14}~\cite{KameugneFSN14}, \href{../works/LarsonJC14.pdf}{LarsonJC14}~\cite{LarsonJC14}, \href{../works/OhrimenkoSC09.pdf}{OhrimenkoSC09}~\cite{OhrimenkoSC09} & \href{../works/FrimodigECM23.pdf}{FrimodigECM23}~\cite{FrimodigECM23}, \href{../works/MullerMKP22.pdf}{MullerMKP22}~\cite{MullerMKP22}, \href{../works/AntuoriHHEN21.pdf}{AntuoriHHEN21}~\cite{AntuoriHHEN21}, \href{../works/Groleaz21.pdf}{Groleaz21}~\cite{Groleaz21}, \href{../works/GeibingerKKMMW21.pdf}{GeibingerKKMMW21}~\cite{GeibingerKKMMW21}, \href{../works/Astrand0F21.pdf}{Astrand0F21}~\cite{Astrand0F21}, \href{../works/FrohnerTR19.pdf}{FrohnerTR19}~\cite{FrohnerTR19}, \href{../works/GeibingerMM19.pdf}{GeibingerMM19}~\cite{GeibingerMM19}, \href{../works/abs-1911-04766.pdf}{abs-1911-04766}~\cite{abs-1911-04766}, \href{../works/LaborieRSV18.pdf}{LaborieRSV18}~\cite{LaborieRSV18}, \href{../works/CarlssonJL17.pdf}{CarlssonJL17}~\cite{CarlssonJL17}, \href{../works/BurtLPS15.pdf}{BurtLPS15}~\cite{BurtLPS15}, \href{../works/BofillEGPSV14.pdf}{BofillEGPSV14}~\cite{BofillEGPSV14}, \href{../works/LozanoCDS12.pdf}{LozanoCDS12}~\cite{LozanoCDS12}, \href{../works/Malapert11.pdf}{Malapert11}~\cite{Malapert11}, \href{../works/KovacsK11.pdf}{KovacsK11}~\cite{KovacsK11}, \href{../works/KameugneFSN11.pdf}{KameugneFSN11}~\cite{KameugneFSN11}, \href{../works/ThiruvadyBME09.pdf}{ThiruvadyBME09}~\cite{ThiruvadyBME09} & \href{../works/WessenCSFPM23.pdf}{WessenCSFPM23}~\cite{WessenCSFPM23}, \href{../works/BofillCGGPSV23.pdf}{BofillCGGPSV23}~\cite{BofillCGGPSV23}, \href{../works/ArmstrongGOS21.pdf}{ArmstrongGOS21}~\cite{ArmstrongGOS21}, \href{../works/WessenCS20.pdf}{WessenCS20}~\cite{WessenCS20}, \href{../works/WallaceY20.pdf}{WallaceY20}~\cite{WallaceY20}, \href{../works/MengZRZL20.pdf}{MengZRZL20}~\cite{MengZRZL20}, \href{../works/YangSS19.pdf}{YangSS19}~\cite{YangSS19}, \href{../works/FrimodigS19.pdf}{FrimodigS19}~\cite{FrimodigS19}, \href{../works/MusliuSS18.pdf}{MusliuSS18}~\cite{MusliuSS18}, \href{../works/GoldwaserS18.pdf}{GoldwaserS18}~\cite{GoldwaserS18}, \href{../works/CauwelaertLS18.pdf}{CauwelaertLS18}~\cite{CauwelaertLS18}, \href{../works/AstrandJZ18.pdf}{AstrandJZ18}~\cite{AstrandJZ18}, \href{../works/GedikKBR17.pdf}{GedikKBR17}~\cite{GedikKBR17}, \href{../works/GoldwaserS17.pdf}{GoldwaserS17}~\cite{GoldwaserS17}, \href{../works/Dejemeppe16.pdf}{Dejemeppe16}~\cite{Dejemeppe16}, \href{../works/AmadiniGM16.pdf}{AmadiniGM16}~\cite{AmadiniGM16}, \href{../works/PesantRR15.pdf}{PesantRR15}~\cite{PesantRR15}, \href{../works/HarjunkoskiMBC14.pdf}{HarjunkoskiMBC14}~\cite{HarjunkoskiMBC14}, \href{../works/LombardiMB13.pdf}{LombardiMB13}~\cite{LombardiMB13}, \href{../works/Clercq12.pdf}{Clercq12}~\cite{Clercq12}, \href{../works/MonetteDD07.pdf}{MonetteDD07}~\cite{MonetteDD07}\\
\index{Gurobi}\index{CPSystems!Gurobi}Gurobi &  1.00 & \href{../works/WangB23.pdf}{WangB23}~\cite{WangB23}, \href{../works/Adelgren2023.pdf}{Adelgren2023}~\cite{Adelgren2023}, \href{../works/LacknerMMWW23.pdf}{LacknerMMWW23}~\cite{LacknerMMWW23}, \href{../works/NaderiRR23.pdf}{NaderiRR23}~\cite{NaderiRR23}, \href{../works/WinterMMW22.pdf}{WinterMMW22}~\cite{WinterMMW22}, \href{../works/ZhangBB22.pdf}{ZhangBB22}~\cite{ZhangBB22}, \href{../works/LacknerMMWW21.pdf}{LacknerMMWW21}~\cite{LacknerMMWW21}, \href{../works/KoehlerBFFHPSSS21.pdf}{KoehlerBFFHPSSS21}~\cite{KoehlerBFFHPSSS21}, \href{../works/Lemos21.pdf}{Lemos21}~\cite{Lemos21}, \href{../works/KovacsTKSG21.pdf}{KovacsTKSG21}~\cite{KovacsTKSG21}, \href{../works/GeibingerKKMMW21.pdf}{GeibingerKKMMW21}~\cite{GeibingerKKMMW21}, \href{../works/FachiniA20.pdf}{FachiniA20}~\cite{FachiniA20}, \href{../works/WangB20.pdf}{WangB20}~\cite{WangB20}, \href{../works/GokGSTO20.pdf}{GokGSTO20}~\cite{GokGSTO20}, \href{../works/WallaceY20.pdf}{WallaceY20}~\cite{WallaceY20}, \href{../works/FrohnerTR19.pdf}{FrohnerTR19}~\cite{FrohnerTR19}, \href{../works/MusliuSS18.pdf}{MusliuSS18}~\cite{MusliuSS18}, \href{../works/GombolayWS18.pdf}{GombolayWS18}~\cite{GombolayWS18}, \href{../works/RoshanaeiLAU17.pdf}{RoshanaeiLAU17}~\cite{RoshanaeiLAU17}, \href{../works/KuB16.pdf}{KuB16}~\cite{KuB16} & \href{../works/ForbesHJST24.pdf}{ForbesHJST24}~\cite{ForbesHJST24}, \href{../works/GuoZ23.pdf}{GuoZ23}~\cite{GuoZ23}, \href{../works/Groleaz21.pdf}{Groleaz21}~\cite{Groleaz21}, \href{../works/VlkHT21.pdf}{VlkHT21}~\cite{VlkHT21}, \href{../works/GoldwaserS18.pdf}{GoldwaserS18}~\cite{GoldwaserS18}, \href{../works/GoldwaserS17.pdf}{GoldwaserS17}~\cite{GoldwaserS17}, \href{../works/Froger16.pdf}{Froger16}~\cite{Froger16}, \href{../works/FontaineMH16.pdf}{FontaineMH16}~\cite{FontaineMH16}, \href{../works/LimBTBB15a.pdf}{LimBTBB15a}~\cite{LimBTBB15a} & \href{../works/abs-2305-19888.pdf}{abs-2305-19888}~\cite{abs-2305-19888}, \href{../works/KimCMLLP23.pdf}{KimCMLLP23}~\cite{KimCMLLP23}, \href{../works/MontemanniD23.pdf}{MontemanniD23}~\cite{MontemanniD23}, \href{../works/Tassel22.pdf}{Tassel22}~\cite{Tassel22}, \href{../works/HeinzNVH22.pdf}{HeinzNVH22}~\cite{HeinzNVH22}, \href{../works/BulckG22.pdf}{BulckG22}~\cite{BulckG22}, \href{../works/MengGRZSC22.pdf}{MengGRZSC22}~\cite{MengGRZSC22}, \href{../works/PohlAK22.pdf}{PohlAK22}~\cite{PohlAK22}, \href{../works/AbohashimaEG21.pdf}{AbohashimaEG21}~\cite{AbohashimaEG21}, \href{../works/MengLZB21.pdf}{MengLZB21}~\cite{MengLZB21}, \href{../works/HubnerGSV21.pdf}{HubnerGSV21}~\cite{HubnerGSV21}, \href{../works/FanXG21.pdf}{FanXG21}~\cite{FanXG21}, \href{../works/KlankeBYE21.pdf}{KlankeBYE21}~\cite{KlankeBYE21}, \href{../works/BenediktMH20.pdf}{BenediktMH20}~\cite{BenediktMH20}, \href{../works/MengZRZL20.pdf}{MengZRZL20}~\cite{MengZRZL20}, \href{../works/He0GLW18.pdf}{He0GLW18}~\cite{He0GLW18}, \href{../works/DemirovicS18.pdf}{DemirovicS18}~\cite{DemirovicS18}, \href{../works/BenediktSMVH18.pdf}{BenediktSMVH18}~\cite{BenediktSMVH18}, \href{../works/TranAB16.pdf}{TranAB16}~\cite{TranAB16}, \href{../works/AmadiniGM16.pdf}{AmadiniGM16}~\cite{AmadiniGM16}, \href{../works/BurtLPS15.pdf}{BurtLPS15}~\cite{BurtLPS15}, \href{../works/PesantRR15.pdf}{PesantRR15}~\cite{PesantRR15}, \href{../works/HarjunkoskiMBC14.pdf}{HarjunkoskiMBC14}~\cite{HarjunkoskiMBC14}\\
\index{Ilog Scheduler}\index{CPSystems!Ilog Scheduler}Ilog Scheduler &  1.00 & \href{../works/Malapert11.pdf}{Malapert11}~\cite{Malapert11}, \href{../works/GrimesH11.pdf}{GrimesH11}~\cite{GrimesH11}, \href{../works/ZeballosQH10.pdf}{ZeballosQH10}~\cite{ZeballosQH10}, \href{../works/Laborie03.pdf}{Laborie03}~\cite{Laborie03} & \href{../works/LaborieRSV18.pdf}{LaborieRSV18}~\cite{LaborieRSV18}, \href{../works/LimtanyakulS12.pdf}{LimtanyakulS12}~\cite{LimtanyakulS12}, \href{../works/HeinzB12.pdf}{HeinzB12}~\cite{HeinzB12}, \href{../works/NovasH12.pdf}{NovasH12}~\cite{NovasH12}, \href{../works/HeckmanB11.pdf}{HeckmanB11}~\cite{HeckmanB11}, \href{../works/BeckFW11.pdf}{BeckFW11}~\cite{BeckFW11}, \href{../works/Zeballos10.pdf}{Zeballos10}~\cite{Zeballos10}, \href{../works/RodriguezS09.pdf}{RodriguezS09}~\cite{RodriguezS09}, \href{../works/GrimesHM09.pdf}{GrimesHM09}~\cite{GrimesHM09}, \href{../works/RenT09.pdf}{RenT09}~\cite{RenT09}, \href{../works/WatsonB08.pdf}{WatsonB08}~\cite{WatsonB08}, \href{../works/Rodriguez07b.pdf}{Rodriguez07b}~\cite{Rodriguez07b}, \href{../works/ZeballosH05.pdf}{ZeballosH05}~\cite{ZeballosH05}, \href{../works/BeckR03.pdf}{BeckR03}~\cite{BeckR03}, \href{../works/JainG01.pdf}{JainG01}~\cite{JainG01}, \href{../works/Beck99.pdf}{Beck99}~\cite{Beck99}, \href{../works/NuijtenP98.pdf}{NuijtenP98}~\cite{NuijtenP98} & \href{../works/Laborie18a.pdf}{Laborie18a}~\cite{Laborie18a}, \href{../works/KuB16.pdf}{KuB16}~\cite{KuB16}, \href{../works/Fahimi16.pdf}{Fahimi16}~\cite{Fahimi16}, \href{../works/TranWDRFOVB16.pdf}{TranWDRFOVB16}~\cite{TranWDRFOVB16}, \href{../works/SchuttS16.pdf}{SchuttS16}~\cite{SchuttS16}, \href{../works/GrimesH15.pdf}{GrimesH15}~\cite{GrimesH15}, \href{../works/TerekhovTDB14.pdf}{TerekhovTDB14}~\cite{TerekhovTDB14}, \href{../works/NovasH14.pdf}{NovasH14}~\cite{NovasH14}, \href{../works/UnsalO13.pdf}{UnsalO13}~\cite{UnsalO13}, \href{../works/TerekhovDOB12.pdf}{TerekhovDOB12}~\cite{TerekhovDOB12}, \href{../works/Schutt11.pdf}{Schutt11}~\cite{Schutt11}, \href{../works/ZeballosNH11.pdf}{ZeballosNH11}~\cite{ZeballosNH11}, \href{../works/SchuttFSW11.pdf}{SchuttFSW11}~\cite{SchuttFSW11}, \href{../works/LahimerLH11.pdf}{LahimerLH11}~\cite{LahimerLH11}, \href{../works/BeniniLMR11.pdf}{BeniniLMR11}~\cite{BeniniLMR11}, \href{../works/KovacsB11.pdf}{KovacsB11}~\cite{KovacsB11}, \href{../works/HachemiGR11.pdf}{HachemiGR11}~\cite{HachemiGR11}, \href{../works/OzturkTHO10.pdf}{OzturkTHO10}~\cite{OzturkTHO10}, \href{../works/abs-1009-0347.pdf}{abs-1009-0347}~\cite{abs-1009-0347}...\href{../works/DannaP03.pdf}{DannaP03}~\cite{DannaP03}, \href{../works/HarjunkoskiG02.pdf}{HarjunkoskiG02}~\cite{HarjunkoskiG02}, \href{../works/Bartak02a.pdf}{Bartak02a}~\cite{Bartak02a}, \href{../works/Baptiste02.pdf}{Baptiste02}~\cite{Baptiste02}, \href{../works/BosiM2001.pdf}{BosiM2001}~\cite{BosiM2001}, \href{../works/SourdN00.pdf}{SourdN00}~\cite{SourdN00}, \href{../works/Dorndorf2000.pdf}{Dorndorf2000}~\cite{Dorndorf2000}, \href{../works/BaptisteP00.pdf}{BaptisteP00}~\cite{BaptisteP00}, \href{../works/Junker00.pdf}{Junker00}~\cite{Junker00}, \href{../works/BeckF00a.pdf}{BeckF00a}~\cite{BeckF00a} (Total: 68)\\
\index{Ilog Solver}\index{CPSystems!Ilog Solver}Ilog Solver &  1.00 &  & \href{../works/GrimesH11.pdf}{GrimesH11}~\cite{GrimesH11}, \href{../works/ZeballosNH11.pdf}{ZeballosNH11}~\cite{ZeballosNH11}, \href{../works/Zeballos10.pdf}{Zeballos10}~\cite{Zeballos10}, \href{../works/ZeballosQH10.pdf}{ZeballosQH10}~\cite{ZeballosQH10}, \href{../works/LiW08.pdf}{LiW08}~\cite{LiW08}, \href{../works/SchausD08.pdf}{SchausD08}~\cite{SchausD08}, \href{../works/HarjunkoskiG02.pdf}{HarjunkoskiG02}~\cite{HarjunkoskiG02}, \href{../works/JainG01.pdf}{JainG01}~\cite{JainG01}, \href{../works/Refalo00.pdf}{Refalo00}~\cite{Refalo00}, \href{../works/HarjunkoskiJG00.pdf}{HarjunkoskiJG00}~\cite{HarjunkoskiJG00}, \href{../works/PesantGPR99.pdf}{PesantGPR99}~\cite{PesantGPR99}, \href{../works/PapeB97.pdf}{PapeB97}~\cite{PapeB97} & \href{../works/CilKLO22.pdf}{CilKLO22}~\cite{CilKLO22}, \href{../works/PinarbasiAY19.pdf}{PinarbasiAY19}~\cite{PinarbasiAY19}, \href{../works/abs-1902-01193.pdf}{abs-1902-01193}~\cite{abs-1902-01193}, \href{../works/LaborieRSV18.pdf}{LaborieRSV18}~\cite{LaborieRSV18}, \href{../works/HookerH17.pdf}{HookerH17}~\cite{HookerH17}, \href{../works/ZarandiKS16.pdf}{ZarandiKS16}~\cite{ZarandiKS16}, \href{../works/Dejemeppe16.pdf}{Dejemeppe16}~\cite{Dejemeppe16}, \href{../works/AlesioBNG15.pdf}{AlesioBNG15}~\cite{AlesioBNG15}, \href{../works/Siala15.pdf}{Siala15}~\cite{Siala15}, \href{../works/Siala15a.pdf}{Siala15a}~\cite{Siala15a}, \href{../works/PesantRR15.pdf}{PesantRR15}~\cite{PesantRR15}, \href{../works/BonfiettiLBM14.pdf}{BonfiettiLBM14}~\cite{BonfiettiLBM14}, \href{../works/NovasH14.pdf}{NovasH14}~\cite{NovasH14}, \href{../works/LombardiMB13.pdf}{LombardiMB13}~\cite{LombardiMB13}, \href{../works/OzturkTHO13.pdf}{OzturkTHO13}~\cite{OzturkTHO13}, \href{../works/TopalogluSS12.pdf}{TopalogluSS12}~\cite{TopalogluSS12}, \href{../works/TerekhovDOB12.pdf}{TerekhovDOB12}~\cite{TerekhovDOB12}, \href{../works/LombardiM12a.pdf}{LombardiM12a}~\cite{LombardiM12a}, \href{../works/HeinzB12.pdf}{HeinzB12}~\cite{HeinzB12}...\href{../works/JoLLH99.pdf}{JoLLH99}~\cite{JoLLH99}, \href{../works/ChunCTY99.pdf}{ChunCTY99}~\cite{ChunCTY99}, \href{../works/BensanaLV99.pdf}{BensanaLV99}~\cite{BensanaLV99}, \href{../works/HookerO99.pdf}{HookerO99}~\cite{HookerO99}, \href{../works/PembertonG98.pdf}{PembertonG98}~\cite{PembertonG98}, \href{../works/NuijtenP98.pdf}{NuijtenP98}~\cite{NuijtenP98}, \href{../works/PapaB98.pdf}{PapaB98}~\cite{PapaB98}, \href{../works/Wallace96.pdf}{Wallace96}~\cite{Wallace96}, \href{../works/SmithBHW96.pdf}{SmithBHW96}~\cite{SmithBHW96}, \href{../works/BaptisteP95.pdf}{BaptisteP95}~\cite{BaptisteP95} (Total: 83)\\
\index{MiniZinc}\index{CPSystems!MiniZinc}MiniZinc &  1.00 & \href{../works/LacknerMMWW23.pdf}{LacknerMMWW23}~\cite{LacknerMMWW23}, \href{../works/WessenCSFPM23.pdf}{WessenCSFPM23}~\cite{WessenCSFPM23}, \href{../works/TardivoDFMP23.pdf}{TardivoDFMP23}~\cite{TardivoDFMP23}, \href{../works/BoudreaultSLQ22.pdf}{BoudreaultSLQ22}~\cite{BoudreaultSLQ22}, \href{../works/MullerMKP22.pdf}{MullerMKP22}~\cite{MullerMKP22}, \href{../works/JungblutK22.pdf}{JungblutK22}~\cite{JungblutK22}, \href{../works/ColT22.pdf}{ColT22}~\cite{ColT22}, \href{../works/KoehlerBFFHPSSS21.pdf}{KoehlerBFFHPSSS21}~\cite{KoehlerBFFHPSSS21}, \href{../works/ArmstrongGOS21.pdf}{ArmstrongGOS21}~\cite{ArmstrongGOS21}, \href{../works/LacknerMMWW21.pdf}{LacknerMMWW21}~\cite{LacknerMMWW21}, \href{../works/Mercier-AubinGQ20.pdf}{Mercier-AubinGQ20}~\cite{Mercier-AubinGQ20}, \href{../works/WallaceY20.pdf}{WallaceY20}~\cite{WallaceY20}, \href{../works/FrohnerTR19.pdf}{FrohnerTR19}~\cite{FrohnerTR19}, \href{../works/abs-1911-04766.pdf}{abs-1911-04766}~\cite{abs-1911-04766}, \href{../works/ColT2019a.pdf}{ColT2019a}~\cite{ColT2019a}, \href{../works/GeibingerMM19.pdf}{GeibingerMM19}~\cite{GeibingerMM19}, \href{../works/ColT19.pdf}{ColT19}~\cite{ColT19}, \href{../works/CarlssonJL17.pdf}{CarlssonJL17}~\cite{CarlssonJL17}, \href{../works/HookerH17.pdf}{HookerH17}~\cite{HookerH17}, \href{../works/YoungFS17.pdf}{YoungFS17}~\cite{YoungFS17}, \href{../works/LiuCGM17.pdf}{LiuCGM17}~\cite{LiuCGM17}, \href{../works/SzerediS16.pdf}{SzerediS16}~\cite{SzerediS16}, \href{../works/AmadiniGM16.pdf}{AmadiniGM16}~\cite{AmadiniGM16}, \href{../works/BofillEGPSV14.pdf}{BofillEGPSV14}~\cite{BofillEGPSV14}, \href{../works/KelarevaTK13.pdf}{KelarevaTK13}~\cite{KelarevaTK13} & \href{../works/FalqueALM24.pdf}{FalqueALM24}~\cite{FalqueALM24}, \href{../works/PovedaAA23.pdf}{PovedaAA23}~\cite{PovedaAA23}, \href{../works/Godet21a.pdf}{Godet21a}~\cite{Godet21a}, \href{../works/GokGSTO20.pdf}{GokGSTO20}~\cite{GokGSTO20}, \href{../works/MusliuSS18.pdf}{MusliuSS18}~\cite{MusliuSS18}, \href{../works/KreterSS17.pdf}{KreterSS17}~\cite{KreterSS17}, \href{../works/KreterSS15.pdf}{KreterSS15}~\cite{KreterSS15}, \href{../works/LarsonJC14.pdf}{LarsonJC14}~\cite{LarsonJC14} & \href{../works/Bit-Monnot23.pdf}{Bit-Monnot23}~\cite{Bit-Monnot23}, \href{../works/BofillCGGPSV23.pdf}{BofillCGGPSV23}~\cite{BofillCGGPSV23}, \href{../works/FrimodigECM23.pdf}{FrimodigECM23}~\cite{FrimodigECM23}, \href{../works/OuelletQ22.pdf}{OuelletQ22}~\cite{OuelletQ22}, \href{../works/abs-2102-08778.pdf}{abs-2102-08778}~\cite{abs-2102-08778}, \href{../works/GeibingerKKMMW21.pdf}{GeibingerKKMMW21}~\cite{GeibingerKKMMW21}, \href{../works/FrimodigS19.pdf}{FrimodigS19}~\cite{FrimodigS19}, \href{../works/Caballero19.pdf}{Caballero19}~\cite{Caballero19}, \href{../works/abs-1901-07914.pdf}{abs-1901-07914}~\cite{abs-1901-07914}, \href{../works/Hooker19.pdf}{Hooker19}~\cite{Hooker19}, \href{../works/BehrensLM19.pdf}{BehrensLM19}~\cite{BehrensLM19}, \href{../works/CappartTSR18.pdf}{CappartTSR18}~\cite{CappartTSR18}, \href{../works/KreterSSZ18.pdf}{KreterSSZ18}~\cite{KreterSSZ18}, \href{../works/DemirovicS18.pdf}{DemirovicS18}~\cite{DemirovicS18}, \href{../works/ErkingerM17.pdf}{ErkingerM17}~\cite{ErkingerM17}, \href{../works/TranVNB17.pdf}{TranVNB17}~\cite{TranVNB17}, \href{../works/FontaineMH16.pdf}{FontaineMH16}~\cite{FontaineMH16}, \href{../works/SchuttS16.pdf}{SchuttS16}~\cite{SchuttS16}, \href{../works/BurtLPS15.pdf}{BurtLPS15}~\cite{BurtLPS15}, \href{../works/SchuttFS13.pdf}{SchuttFS13}~\cite{SchuttFS13}, \href{../works/HeinzSB13.pdf}{HeinzSB13}~\cite{HeinzSB13}\\
\index{Mistral}\index{CPSystems!Mistral}Mistral &  1.00 & \href{../works/JuvinHHL23.pdf}{JuvinHHL23}~\cite{JuvinHHL23}, \href{../works/Siala15a.pdf}{Siala15a}~\cite{Siala15a}, \href{../works/Siala15.pdf}{Siala15}~\cite{Siala15}, \href{../works/Malapert11.pdf}{Malapert11}~\cite{Malapert11}, \href{../works/GrimesHM09.pdf}{GrimesHM09}~\cite{GrimesHM09} & \href{../works/Bit-Monnot23.pdf}{Bit-Monnot23}~\cite{Bit-Monnot23}, \href{../works/Kameugne14.pdf}{Kameugne14}~\cite{Kameugne14}, \href{../works/BillautHL12.pdf}{BillautHL12}~\cite{BillautHL12} & \href{../works/GrimesH15.pdf}{GrimesH15}~\cite{GrimesH15}, \href{../works/SialaAH15.pdf}{SialaAH15}~\cite{SialaAH15}\\
\index{OPL}\index{CPSystems!OPL}OPL &  1.00 & \href{../works/LacknerMMWW23.pdf}{LacknerMMWW23}~\cite{LacknerMMWW23}, \href{../works/GuoZ23.pdf}{GuoZ23}~\cite{GuoZ23}, \href{../works/YunusogluY22.pdf}{YunusogluY22}~\cite{YunusogluY22}, \href{../works/MullerMKP22.pdf}{MullerMKP22}~\cite{MullerMKP22}, \href{../works/TouatBT22.pdf}{TouatBT22}~\cite{TouatBT22}, \href{../works/ColT22.pdf}{ColT22}~\cite{ColT22}, \href{../works/PandeyS21a.pdf}{PandeyS21a}~\cite{PandeyS21a}, \href{../works/KoehlerBFFHPSSS21.pdf}{KoehlerBFFHPSSS21}~\cite{KoehlerBFFHPSSS21}, \href{../works/LacknerMMWW21.pdf}{LacknerMMWW21}~\cite{LacknerMMWW21}, \href{../works/AbidinK20.pdf}{AbidinK20}~\cite{AbidinK20}, \href{../works/QinDCS20.pdf}{QinDCS20}~\cite{QinDCS20}, \href{../works/Novas19.pdf}{Novas19}~\cite{Novas19}, \href{../works/EscobetPQPRA19.pdf}{EscobetPQPRA19}~\cite{EscobetPQPRA19}, \href{../works/LaborieRSV18.pdf}{LaborieRSV18}~\cite{LaborieRSV18}, \href{../works/TangLWSK18.pdf}{TangLWSK18}~\cite{TangLWSK18}, \href{../works/QinDS16.pdf}{QinDS16}~\cite{QinDS16}, \href{../works/Dejemeppe16.pdf}{Dejemeppe16}~\cite{Dejemeppe16}, \href{../works/NovaraNH16.pdf}{NovaraNH16}~\cite{NovaraNH16}, \href{../works/AlesioNBG14.pdf}{AlesioNBG14}~\cite{AlesioNBG14}...\href{../works/ZeballosM09.pdf}{ZeballosM09}~\cite{ZeballosM09}, \href{../works/Laborie09.pdf}{Laborie09}~\cite{Laborie09}, \href{../works/LiW08.pdf}{LiW08}~\cite{LiW08}, \href{../works/RussellU06.pdf}{RussellU06}~\cite{RussellU06}, \href{../works/KhayatLR06.pdf}{KhayatLR06}~\cite{KhayatLR06}, \href{../works/MaraveliasCG04.pdf}{MaraveliasCG04}~\cite{MaraveliasCG04}, \href{../works/KanetAG04.pdf}{KanetAG04}~\cite{KanetAG04}, \href{../works/Trick03.pdf}{Trick03}~\cite{Trick03}, \href{../works/JainG01.pdf}{JainG01}~\cite{JainG01}, \href{../works/Nuijten94.pdf}{Nuijten94}~\cite{Nuijten94} (Total: 39) & \href{../works/LuZZYW24.pdf}{LuZZYW24}~\cite{LuZZYW24}, \href{../works/AwadMDMT22.pdf}{AwadMDMT22}~\cite{AwadMDMT22}, \href{../works/Teppan22.pdf}{Teppan22}~\cite{Teppan22}, \href{../works/SubulanC22.pdf}{SubulanC22}~\cite{SubulanC22}, \href{../works/MengLZB21.pdf}{MengLZB21}~\cite{MengLZB21}, \href{../works/Mercier-AubinGQ20.pdf}{Mercier-AubinGQ20}~\cite{Mercier-AubinGQ20}, \href{../works/ZarandiASC20.pdf}{ZarandiASC20}~\cite{ZarandiASC20}, \href{../works/ZouZ20.pdf}{ZouZ20}~\cite{ZouZ20}, \href{../works/MurinR19.pdf}{MurinR19}~\cite{MurinR19}, \href{../works/Laborie18a.pdf}{Laborie18a}~\cite{Laborie18a}, \href{../works/CappartTSR18.pdf}{CappartTSR18}~\cite{CappartTSR18}, \href{../works/HookerH17.pdf}{HookerH17}~\cite{HookerH17}, \href{../works/OrnekO16.pdf}{OrnekO16}~\cite{OrnekO16}, \href{../works/WangMD15.pdf}{WangMD15}~\cite{WangMD15}, \href{../works/LimBTBB15.pdf}{LimBTBB15}~\cite{LimBTBB15}, \href{../works/EvenSH15a.pdf}{EvenSH15a}~\cite{EvenSH15a}, \href{../works/NovasH14.pdf}{NovasH14}~\cite{NovasH14}, \href{../works/HarjunkoskiMBC14.pdf}{HarjunkoskiMBC14}~\cite{HarjunkoskiMBC14}, \href{../works/ChunS14.pdf}{ChunS14}~\cite{ChunS14}...\href{../works/QuirogaZH05.pdf}{QuirogaZH05}~\cite{QuirogaZH05}, \href{../works/Baptiste02.pdf}{Baptiste02}~\cite{Baptiste02}, \href{../works/LorigeonBB02.pdf}{LorigeonBB02}~\cite{LorigeonBB02}, \href{../works/HarjunkoskiG02.pdf}{HarjunkoskiG02}~\cite{HarjunkoskiG02}, \href{../works/Thorsteinsson01.pdf}{Thorsteinsson01}~\cite{Thorsteinsson01}, \href{../works/VerfaillieL01.pdf}{VerfaillieL01}~\cite{VerfaillieL01}, \href{../works/HookerOTK00.pdf}{HookerOTK00}~\cite{HookerOTK00}, \href{../works/Simonis99.pdf}{Simonis99}~\cite{Simonis99}, \href{../works/RodosekW98.pdf}{RodosekW98}~\cite{RodosekW98}, \href{../works/BeckDDF98.pdf}{BeckDDF98}~\cite{BeckDDF98} (Total: 56) & \href{../works/abs-2402-00459.pdf}{abs-2402-00459}~\cite{abs-2402-00459}, \href{../works/FalqueALM24.pdf}{FalqueALM24}~\cite{FalqueALM24}, \href{../works/ForbesHJST24.pdf}{ForbesHJST24}~\cite{ForbesHJST24}, \href{../works/YuraszeckMCCR23.pdf}{YuraszeckMCCR23}~\cite{YuraszeckMCCR23}, \href{../works/CzerniachowskaWZ23.pdf}{CzerniachowskaWZ23}~\cite{CzerniachowskaWZ23}, \href{../works/MontemanniD23.pdf}{MontemanniD23}~\cite{MontemanniD23}, \href{../works/IsikYA23.pdf}{IsikYA23}~\cite{IsikYA23}, \href{../works/PerezGSL23.pdf}{PerezGSL23}~\cite{PerezGSL23}, \href{../works/EfthymiouY23.pdf}{EfthymiouY23}~\cite{EfthymiouY23}, \href{../works/AbreuPNF23.pdf}{AbreuPNF23}~\cite{AbreuPNF23}, \href{../works/abs-2312-13682.pdf}{abs-2312-13682}~\cite{abs-2312-13682}, \href{../works/GurPAE23.pdf}{GurPAE23}~\cite{GurPAE23}, \href{../works/Fatemi-AnarakiTFV23.pdf}{Fatemi-AnarakiTFV23}~\cite{Fatemi-AnarakiTFV23}, \href{../works/GokPTGO23.pdf}{GokPTGO23}~\cite{GokPTGO23}, \href{../works/AbreuNP23.pdf}{AbreuNP23}~\cite{AbreuNP23}, \href{../works/ArmstrongGOS22.pdf}{ArmstrongGOS22}~\cite{ArmstrongGOS22}, \href{../works/LiFJZLL22.pdf}{LiFJZLL22}~\cite{LiFJZLL22}, \href{../works/ZhangBB22.pdf}{ZhangBB22}~\cite{ZhangBB22}, \href{../works/GhandehariK22.pdf}{GhandehariK22}~\cite{GhandehariK22}...\href{../works/PintoG97.pdf}{PintoG97}~\cite{PintoG97}, \href{../works/SmithBHW96.pdf}{SmithBHW96}~\cite{SmithBHW96}, \href{../works/BlazewiczDP96.pdf}{BlazewiczDP96}~\cite{BlazewiczDP96}, \href{../works/Wallace96.pdf}{Wallace96}~\cite{Wallace96}, \href{../works/Simonis95a.pdf}{Simonis95a}~\cite{Simonis95a}, \href{../works/Puget95.pdf}{Puget95}~\cite{Puget95}, \href{../works/WeilHFP95.pdf}{WeilHFP95}~\cite{WeilHFP95}, \href{../works/Muscettola94.pdf}{Muscettola94}~\cite{Muscettola94}, \href{../works/BeldiceanuC94.pdf}{BeldiceanuC94}~\cite{BeldiceanuC94}, \href{../works/MintonJPL92.pdf}{MintonJPL92}~\cite{MintonJPL92} (Total: 148)\\
\index{OR-Tools}\index{CPSystems!OR-Tools}OR-Tools &  1.00 & \href{../works/abs-2402-00459.pdf}{abs-2402-00459}~\cite{abs-2402-00459}, \href{../works/LacknerMMWW23.pdf}{LacknerMMWW23}~\cite{LacknerMMWW23}, \href{../works/ColT22.pdf}{ColT22}~\cite{ColT22}, \href{../works/MullerMKP22.pdf}{MullerMKP22}~\cite{MullerMKP22}, \href{../works/abs-2211-14492.pdf}{abs-2211-14492}~\cite{abs-2211-14492}, \href{../works/Groleaz21.pdf}{Groleaz21}~\cite{Groleaz21}, \href{../works/abs-2102-08778.pdf}{abs-2102-08778}~\cite{abs-2102-08778}, \href{../works/KovacsTKSG21.pdf}{KovacsTKSG21}~\cite{KovacsTKSG21}, \href{../works/KoehlerBFFHPSSS21.pdf}{KoehlerBFFHPSSS21}~\cite{KoehlerBFFHPSSS21}, \href{../works/LacknerMMWW21.pdf}{LacknerMMWW21}~\cite{LacknerMMWW21}, \href{../works/FachiniA20.pdf}{FachiniA20}~\cite{FachiniA20}, \href{../works/FallahiAC20.pdf}{FallahiAC20}~\cite{FallahiAC20}, \href{../works/ColT19.pdf}{ColT19}~\cite{ColT19}, \href{../works/ColT2019a.pdf}{ColT2019a}~\cite{ColT2019a}, \href{../works/GayHS15.pdf}{GayHS15}~\cite{GayHS15} & \href{../works/EfthymiouY23.pdf}{EfthymiouY23}~\cite{EfthymiouY23}, \href{../works/BoudreaultSLQ22.pdf}{BoudreaultSLQ22}~\cite{BoudreaultSLQ22}, \href{../works/Godet21a.pdf}{Godet21a}~\cite{Godet21a}, \href{../works/GeibingerKKMMW21.pdf}{GeibingerKKMMW21}~\cite{GeibingerKKMMW21}, \href{../works/BarzegaranZP20.pdf}{BarzegaranZP20}~\cite{BarzegaranZP20}, \href{../works/ThomasKS20.pdf}{ThomasKS20}~\cite{ThomasKS20}, \href{../works/LiuCGM17.pdf}{LiuCGM17}~\cite{LiuCGM17}, \href{../works/Dejemeppe16.pdf}{Dejemeppe16}~\cite{Dejemeppe16} & \href{../works/Bit-Monnot23.pdf}{Bit-Monnot23}~\cite{Bit-Monnot23}, \href{../works/KimCMLLP23.pdf}{KimCMLLP23}~\cite{KimCMLLP23}, \href{../works/MontemanniD23.pdf}{MontemanniD23}~\cite{MontemanniD23}, \href{../works/AkramNHRSA23.pdf}{AkramNHRSA23}~\cite{AkramNHRSA23}, \href{../works/MontemanniD23a.pdf}{MontemanniD23a}~\cite{MontemanniD23a}, \href{../works/IklassovMR023.pdf}{IklassovMR023}~\cite{IklassovMR023}, \href{../works/Teppan22.pdf}{Teppan22}~\cite{Teppan22}, \href{../works/EtminaniesfahaniGNMS22.pdf}{EtminaniesfahaniGNMS22}~\cite{EtminaniesfahaniGNMS22}, \href{../works/Tassel22.pdf}{Tassel22}~\cite{Tassel22}, \href{../works/KlankeBYE21.pdf}{KlankeBYE21}~\cite{KlankeBYE21}, \href{../works/MengZRZL20.pdf}{MengZRZL20}~\cite{MengZRZL20}, \href{../works/GroleazNS20.pdf}{GroleazNS20}~\cite{GroleazNS20}, \href{../works/GalleguillosKSB19.pdf}{GalleguillosKSB19}~\cite{GalleguillosKSB19}, \href{../works/YangSS19.pdf}{YangSS19}~\cite{YangSS19}, \href{../works/BehrensLM19.pdf}{BehrensLM19}~\cite{BehrensLM19}, \href{../works/abs-1901-07914.pdf}{abs-1901-07914}~\cite{abs-1901-07914}, \href{../works/PourDERB18.pdf}{PourDERB18}~\cite{PourDERB18}, \href{../works/GedikKBR17.pdf}{GedikKBR17}~\cite{GedikKBR17}, \href{../works/BonfiettiZLM16.pdf}{BonfiettiZLM16}~\cite{BonfiettiZLM16}, \href{../works/AmadiniGM16.pdf}{AmadiniGM16}~\cite{AmadiniGM16}, \href{../works/ZhouGL15.pdf}{ZhouGL15}~\cite{ZhouGL15}, \href{../works/LombardiMB13.pdf}{LombardiMB13}~\cite{LombardiMB13}, \href{../works/LombardiM12.pdf}{LombardiM12}~\cite{LombardiM12}\\
\index{OZ}\index{CPSystems!OZ}OZ &  1.00 & \href{../works/Layfield02.pdf}{Layfield02}~\cite{Layfield02} & \href{../works/ZeballosNH11.pdf}{ZeballosNH11}~\cite{ZeballosNH11}, \href{../works/MaraveliasG04.pdf}{MaraveliasG04}~\cite{MaraveliasG04}, \href{../works/BeldiceanuC94.pdf}{BeldiceanuC94}~\cite{BeldiceanuC94} & \href{../works/Froger16.pdf}{Froger16}~\cite{Froger16}, \href{../works/LudwigKRBMS14.pdf}{LudwigKRBMS14}~\cite{LudwigKRBMS14}, \href{../works/KorbaaYG99.pdf}{KorbaaYG99}~\cite{KorbaaYG99}\\
\index{SCIP}\index{CPSystems!SCIP}SCIP &  1.00 & \href{../works/Caballero19.pdf}{Caballero19}~\cite{Caballero19}, \href{../works/SchnellH17.pdf}{SchnellH17}~\cite{SchnellH17}, \href{../works/KuB16.pdf}{KuB16}~\cite{KuB16}, \href{../works/SchnellH15.pdf}{SchnellH15}~\cite{SchnellH15}, \href{../works/HeinzSB13.pdf}{HeinzSB13}~\cite{HeinzSB13}, \href{../works/HeinzB12.pdf}{HeinzB12}~\cite{HeinzB12}, \href{../works/MilanoW09.pdf}{MilanoW09}~\cite{MilanoW09}, \href{../works/AchterbergBKW08.pdf}{AchterbergBKW08}~\cite{AchterbergBKW08} & \href{../works/BofillCSV17.pdf}{BofillCSV17}~\cite{BofillCSV17}, \href{../works/HookerH17.pdf}{HookerH17}~\cite{HookerH17}, \href{../works/TranAB16.pdf}{TranAB16}~\cite{TranAB16}, \href{../works/BofillEGPSV14.pdf}{BofillEGPSV14}~\cite{BofillEGPSV14}, \href{../works/SchuttFS13a.pdf}{SchuttFS13a}~\cite{SchuttFS13a}, \href{../works/HeinzKB13.pdf}{HeinzKB13}~\cite{HeinzKB13}, \href{../works/CireCH13.pdf}{CireCH13}~\cite{CireCH13} & \href{../works/GuoZ23.pdf}{GuoZ23}~\cite{GuoZ23}, \href{../works/NaderiRR23.pdf}{NaderiRR23}~\cite{NaderiRR23}, \href{../works/Groleaz21.pdf}{Groleaz21}~\cite{Groleaz21}, \href{../works/WikarekS19.pdf}{WikarekS19}~\cite{WikarekS19}, \href{../works/SzerediS16.pdf}{SzerediS16}~\cite{SzerediS16}, \href{../works/KinsellaS0OS16.pdf}{KinsellaS0OS16}~\cite{KinsellaS0OS16}, \href{../works/QinDS16.pdf}{QinDS16}~\cite{QinDS16}, \href{../works/HarjunkoskiMBC14.pdf}{HarjunkoskiMBC14}~\cite{HarjunkoskiMBC14}, \href{../works/KelarevaTK13.pdf}{KelarevaTK13}~\cite{KelarevaTK13}, \href{../works/ZampelliVSDR13.pdf}{ZampelliVSDR13}~\cite{ZampelliVSDR13}, \href{../works/Schutt11.pdf}{Schutt11}~\cite{Schutt11}, \href{../works/HeinzS11.pdf}{HeinzS11}~\cite{HeinzS11}, \href{../works/BertholdHLMS10.pdf}{BertholdHLMS10}~\cite{BertholdHLMS10}\\
\index{SICStus}\index{CPSystems!SICStus}SICStus &  1.00 & \href{../works/ArmstrongGOS21.pdf}{ArmstrongGOS21}~\cite{ArmstrongGOS21}, \href{../works/LetortCB15.pdf}{LetortCB15}~\cite{LetortCB15}, \href{../works/Letort13.pdf}{Letort13}~\cite{Letort13}, \href{../works/LetortCB13.pdf}{LetortCB13}~\cite{LetortCB13}, \href{../works/LetortBC12.pdf}{LetortBC12}~\cite{LetortBC12}, \href{../works/WolfS05a.pdf}{WolfS05a}~\cite{WolfS05a} & \href{../works/MossigeGSMC17.pdf}{MossigeGSMC17}~\cite{MossigeGSMC17}, \href{../works/Kameugne14.pdf}{Kameugne14}~\cite{Kameugne14}, \href{../works/Schutt11.pdf}{Schutt11}~\cite{Schutt11}, \href{../works/Malapert11.pdf}{Malapert11}~\cite{Malapert11}, \href{../works/SchuttFSW11.pdf}{SchuttFSW11}~\cite{SchuttFSW11}, \href{../works/QuSN06.pdf}{QuSN06}~\cite{QuSN06} & \href{../works/PopovicCGNC22.pdf}{PopovicCGNC22}~\cite{PopovicCGNC22}, \href{../works/ArmstrongGOS22.pdf}{ArmstrongGOS22}~\cite{ArmstrongGOS22}, \href{../works/YangSS19.pdf}{YangSS19}~\cite{YangSS19}, \href{../works/German18.pdf}{German18}~\cite{German18}, \href{../works/Madi-WambaLOBM17.pdf}{Madi-WambaLOBM17}~\cite{Madi-WambaLOBM17}, \href{../works/JelinekB16.pdf}{JelinekB16}~\cite{JelinekB16}, \href{../works/Clercq12.pdf}{Clercq12}~\cite{Clercq12}, \href{../works/BeldiceanuCDP11.pdf}{BeldiceanuCDP11}~\cite{BeldiceanuCDP11}, \href{../works/TrojetHL11.pdf}{TrojetHL11}~\cite{TrojetHL11}, \href{../works/BartakCS10.pdf}{BartakCS10}~\cite{BartakCS10}, \href{../works/Wolf09.pdf}{Wolf09}~\cite{Wolf09}, \href{../works/SchuttFSW09.pdf}{SchuttFSW09}~\cite{SchuttFSW09}, \href{../works/BeldiceanuCP08.pdf}{BeldiceanuCP08}~\cite{BeldiceanuCP08}, \href{../works/ClautiauxJCM08.pdf}{ClautiauxJCM08}~\cite{ClautiauxJCM08}, \href{../works/Geske05.pdf}{Geske05}~\cite{Geske05}, \href{../works/MeyerE04.pdf}{MeyerE04}~\cite{MeyerE04}, \href{../works/Kuchcinski03.pdf}{Kuchcinski03}~\cite{Kuchcinski03}, \href{../works/Bartak02.pdf}{Bartak02}~\cite{Bartak02}, \href{../works/BeldiceanuC02.pdf}{BeldiceanuC02}~\cite{BeldiceanuC02}, \href{../works/BeldiceanuC01.pdf}{BeldiceanuC01}~\cite{BeldiceanuC01}, \href{../works/Simonis99.pdf}{Simonis99}~\cite{Simonis99}, \href{../works/CarlssonKA99.pdf}{CarlssonKA99}~\cite{CarlssonKA99}, \href{../works/GetoorOFC97.pdf}{GetoorOFC97}~\cite{GetoorOFC97}\\
\index{Z3}\index{CPSystems!Z3}Z3 &  1.00 & \href{../works/Edis21.pdf}{Edis21}~\cite{Edis21}, \href{../works/KoehlerBFFHPSSS21.pdf}{KoehlerBFFHPSSS21}~\cite{KoehlerBFFHPSSS21}, \href{../works/GokGSTO20.pdf}{GokGSTO20}~\cite{GokGSTO20}, \href{../works/YounespourAKE19.pdf}{YounespourAKE19}~\cite{YounespourAKE19}, \href{../works/ErkingerM17.pdf}{ErkingerM17}~\cite{ErkingerM17}, \href{../works/Menana11.pdf}{Menana11}~\cite{Menana11}, \href{../works/SureshMOK06.pdf}{SureshMOK06}~\cite{SureshMOK06} & \href{../works/NaderiRR23.pdf}{NaderiRR23}~\cite{NaderiRR23}, \href{../works/VlkHT21.pdf}{VlkHT21}~\cite{VlkHT21}, \href{../works/WikarekS19.pdf}{WikarekS19}~\cite{WikarekS19}, \href{../works/ArkhipovBL19.pdf}{ArkhipovBL19}~\cite{ArkhipovBL19}, \href{../works/German18.pdf}{German18}~\cite{German18}, \href{../works/LimBTBB15a.pdf}{LimBTBB15a}~\cite{LimBTBB15a}, \href{../works/ZeballosCM10.pdf}{ZeballosCM10}~\cite{ZeballosCM10}, \href{../works/ClautiauxJCM08.pdf}{ClautiauxJCM08}~\cite{ClautiauxJCM08}, \href{../works/Baptiste02.pdf}{Baptiste02}~\cite{Baptiste02}, \href{../works/Zhou97.pdf}{Zhou97}~\cite{Zhou97} & \href{../works/KotaryFH22.pdf}{KotaryFH22}~\cite{KotaryFH22}, \href{../works/CilKLO22.pdf}{CilKLO22}~\cite{CilKLO22}, \href{../works/Groleaz21.pdf}{Groleaz21}~\cite{Groleaz21}, \href{../works/Caballero19.pdf}{Caballero19}~\cite{Caballero19}, \href{../works/ZhangW18.pdf}{ZhangW18}~\cite{ZhangW18}, \href{../works/BofillCSV17.pdf}{BofillCSV17}~\cite{BofillCSV17}, \href{../works/BertholdHLMS10.pdf}{BertholdHLMS10}~\cite{BertholdHLMS10}, \href{../works/Rodriguez07.pdf}{Rodriguez07}~\cite{Rodriguez07}, \href{../works/Rodriguez07b.pdf}{Rodriguez07b}~\cite{Rodriguez07b}, \href{../works/Wallace06.pdf}{Wallace06}~\cite{Wallace06}, \href{../works/Layfield02.pdf}{Layfield02}~\cite{Layfield02}, \href{../works/Zhou96.pdf}{Zhou96}~\cite{Zhou96}\\
\end{longtable}
}

\clearpage
\subsection{Concept Type Benchmarks}
\label{sec:Benchmarks}
\label{Benchmarks}
{\scriptsize
\begin{longtable}{p{3cm}r>{\raggedright\arraybackslash}p{6cm}>{\raggedright\arraybackslash}p{6cm}>{\raggedright\arraybackslash}p{8cm}}
\rowcolor{white}\caption{Works for Concepts of Type Benchmarks (Total 16 Concepts, 16 Used)}\\ \toprule
\rowcolor{white}Keyword & Weight & High & Medium & Low\\ \midrule\endhead
\bottomrule
\endfoot
\index{CSPlib}\index{Benchmarks!CSPlib}CSPlib &  1.00 & \href{../works/LiuLH19a.pdf}{LiuLH19a}~\cite{LiuLH19a}, \href{../works/Siala15.pdf}{Siala15}~\cite{Siala15}, \href{../works/Siala15a.pdf}{Siala15a}~\cite{Siala15a}, \href{../works/SchausHMCMD11.pdf}{SchausHMCMD11}~\cite{SchausHMCMD11}, \href{../works/GarganiR07.pdf}{GarganiR07}~\cite{GarganiR07} & \href{../works/LaborieRSV18.pdf}{LaborieRSV18}~\cite{LaborieRSV18}, \href{../works/German18.pdf}{German18}~\cite{German18}, \href{../works/CappartTSR18.pdf}{CappartTSR18}~\cite{CappartTSR18}, \href{../works/MossigeGSMC17.pdf}{MossigeGSMC17}~\cite{MossigeGSMC17}, \href{../works/NovaraNH16.pdf}{NovaraNH16}~\cite{NovaraNH16}, \href{../works/Letort13.pdf}{Letort13}~\cite{Letort13}, \href{../works/HeinzSSW12.pdf}{HeinzSSW12}~\cite{HeinzSSW12}, \href{../works/BandaSC11.pdf}{BandaSC11}~\cite{BandaSC11}, \href{../works/KendallKRU10.pdf}{KendallKRU10}~\cite{KendallKRU10} & \href{../works/ThomasKS20.pdf}{ThomasKS20}~\cite{ThomasKS20}, \href{../works/LiuLH19.pdf}{LiuLH19}~\cite{LiuLH19}, \href{../works/LiuLH18.pdf}{LiuLH18}~\cite{LiuLH18}, \href{../works/GelainPRVW17.pdf}{GelainPRVW17}~\cite{GelainPRVW17}, \href{../works/GaySS14.pdf}{GaySS14}~\cite{GaySS14}, \href{../works/RendlPHPR12.pdf}{RendlPHPR12}~\cite{RendlPHPR12}, \href{../works/HentenryckM08.pdf}{HentenryckM08}~\cite{HentenryckM08}\\
\index{Roadef}\index{Benchmarks!Roadef}Roadef &  1.00 & \href{../works/Froger16.pdf}{Froger16}~\cite{Froger16}, \href{../works/Siala15.pdf}{Siala15}~\cite{Siala15}, \href{../works/Siala15a.pdf}{Siala15a}~\cite{Siala15a} & \href{../works/Nattaf16.pdf}{Nattaf16}~\cite{Nattaf16}, \href{../works/LetortCB15.pdf}{LetortCB15}~\cite{LetortCB15}, \href{../works/Kameugne14.pdf}{Kameugne14}~\cite{Kameugne14}, \href{../works/Letort13.pdf}{Letort13}~\cite{Letort13}, \href{../works/LetortCB13.pdf}{LetortCB13}~\cite{LetortCB13}, \href{../works/LetortBC12.pdf}{LetortBC12}~\cite{LetortBC12} & \href{../works/CzerniachowskaWZ23.pdf}{CzerniachowskaWZ23}~\cite{CzerniachowskaWZ23}, \href{../works/HanenKP21.pdf}{HanenKP21}~\cite{HanenKP21}, \href{../works/Lemos21.pdf}{Lemos21}~\cite{Lemos21}, \href{../works/GokGSTO20.pdf}{GokGSTO20}~\cite{GokGSTO20}, \href{../works/CarlierPSJ20.pdf}{CarlierPSJ20}~\cite{CarlierPSJ20}, \href{../works/Polo-MejiaALB20.pdf}{Polo-MejiaALB20}~\cite{Polo-MejiaALB20}, \href{../works/MalapertN19.pdf}{MalapertN19}~\cite{MalapertN19}, \href{../works/OuelletQ18.pdf}{OuelletQ18}~\cite{OuelletQ18}, \href{../works/Tesch18.pdf}{Tesch18}~\cite{Tesch18}, \href{../works/Fahimi16.pdf}{Fahimi16}~\cite{Fahimi16}, \href{../works/Tesch16.pdf}{Tesch16}~\cite{Tesch16}, \href{../works/Menana11.pdf}{Menana11}~\cite{Menana11}, \href{../works/Acuna-AgostMFG09.pdf}{Acuna-AgostMFG09}~\cite{Acuna-AgostMFG09}, \href{../works/Wallace06.pdf}{Wallace06}~\cite{Wallace06}, \href{../works/Laborie05.pdf}{Laborie05}~\cite{Laborie05}, \href{../works/Elkhyari03.pdf}{Elkhyari03}~\cite{Elkhyari03}\\
\index{benchmark}\index{Benchmarks!benchmark}benchmark &  1.00 & \href{../works/LiLZDZW24.pdf}{LiLZDZW24}~\cite{LiLZDZW24}, \href{../works/JuvinHL23a.pdf}{JuvinHL23a}~\cite{JuvinHL23a}, \href{../works/IsikYA23.pdf}{IsikYA23}~\cite{IsikYA23}, \href{../works/AlfieriGPS23.pdf}{AlfieriGPS23}~\cite{AlfieriGPS23}, \href{../works/JuvinHHL23.pdf}{JuvinHHL23}~\cite{JuvinHHL23}, \href{../works/Bit-Monnot23.pdf}{Bit-Monnot23}~\cite{Bit-Monnot23}, \href{../works/NaderiBZR23.pdf}{NaderiBZR23}~\cite{NaderiBZR23}, \href{../works/AfsarVPG23.pdf}{AfsarVPG23}~\cite{AfsarVPG23}, \href{../works/YuraszeckMCCR23.pdf}{YuraszeckMCCR23}~\cite{YuraszeckMCCR23}, \href{../works/ShaikhK23.pdf}{ShaikhK23}~\cite{ShaikhK23}, \href{../works/ZhuSZW23.pdf}{ZhuSZW23}~\cite{ZhuSZW23}, \href{../works/NaderiRR23.pdf}{NaderiRR23}~\cite{NaderiRR23}, \href{../works/TasselGS23.pdf}{TasselGS23}~\cite{TasselGS23}, \href{../works/AbreuPNF23.pdf}{AbreuPNF23}~\cite{AbreuPNF23}, \href{../works/TardivoDFMP23.pdf}{TardivoDFMP23}~\cite{TardivoDFMP23}, \href{../works/LacknerMMWW23.pdf}{LacknerMMWW23}~\cite{LacknerMMWW23}, \href{../works/PovedaAA23.pdf}{PovedaAA23}~\cite{PovedaAA23}, \href{../works/abs-2306-05747.pdf}{abs-2306-05747}~\cite{abs-2306-05747}, \href{../works/AbreuNP23.pdf}{AbreuNP23}~\cite{AbreuNP23}...\href{../works/SakkoutW00.pdf}{SakkoutW00}~\cite{SakkoutW00}, \href{../works/HeipckeCCS00.pdf}{HeipckeCCS00}~\cite{HeipckeCCS00}, \href{../works/JainM99.pdf}{JainM99}~\cite{JainM99}, \href{../works/CestaOF99.pdf}{CestaOF99}~\cite{CestaOF99}, \href{../works/Beck99.pdf}{Beck99}~\cite{Beck99}, \href{../works/WatsonBHW99.pdf}{WatsonBHW99}~\cite{WatsonBHW99}, \href{../works/BensanaLV99.pdf}{BensanaLV99}~\cite{BensanaLV99}, \href{../works/BeckF98.pdf}{BeckF98}~\cite{BeckF98}, \href{../works/BeckDF97.pdf}{BeckDF97}~\cite{BeckDF97}, \href{../works/SadehF96.pdf}{SadehF96}~\cite{SadehF96} (Total: 133) & \href{../works/ForbesHJST24.pdf}{ForbesHJST24}~\cite{ForbesHJST24}, \href{../works/abs-2402-00459.pdf}{abs-2402-00459}~\cite{abs-2402-00459}, \href{../works/NaderiBZ23.pdf}{NaderiBZ23}~\cite{NaderiBZ23}, \href{../works/YuraszeckMC23.pdf}{YuraszeckMC23}~\cite{YuraszeckMC23}, \href{../works/MontemanniD23a.pdf}{MontemanniD23a}~\cite{MontemanniD23a}, \href{../works/MarliereSPR23.pdf}{MarliereSPR23}~\cite{MarliereSPR23}, \href{../works/AkramNHRSA23.pdf}{AkramNHRSA23}~\cite{AkramNHRSA23}, \href{../works/FrimodigECM23.pdf}{FrimodigECM23}~\cite{FrimodigECM23}, \href{../works/IklassovMR023.pdf}{IklassovMR023}~\cite{IklassovMR023}, \href{../works/KameugneFND23.pdf}{KameugneFND23}~\cite{KameugneFND23}, \href{../works/abs-2305-19888.pdf}{abs-2305-19888}~\cite{abs-2305-19888}, \href{../works/NaderiBZ22.pdf}{NaderiBZ22}~\cite{NaderiBZ22}, \href{../works/BourreauGGLT22.pdf}{BourreauGGLT22}~\cite{BourreauGGLT22}, \href{../works/KotaryFH22.pdf}{KotaryFH22}~\cite{KotaryFH22}, \href{../works/ZhangBB22.pdf}{ZhangBB22}~\cite{ZhangBB22}, \href{../works/FetgoD22.pdf}{FetgoD22}~\cite{FetgoD22}, \href{../works/Tassel22.pdf}{Tassel22}~\cite{Tassel22}, \href{../works/OujanaAYB22.pdf}{OujanaAYB22}~\cite{OujanaAYB22}, \href{../works/HeinzNVH22.pdf}{HeinzNVH22}~\cite{HeinzNVH22}...\href{../works/BourdaisGP03.pdf}{BourdaisGP03}~\cite{BourdaisGP03}, \href{../works/Bartak02a.pdf}{Bartak02a}~\cite{Bartak02a}, \href{../works/Baptiste02.pdf}{Baptiste02}~\cite{Baptiste02}, \href{../works/BeldiceanuC01.pdf}{BeldiceanuC01}~\cite{BeldiceanuC01}, \href{../works/BeckF00.pdf}{BeckF00}~\cite{BeckF00}, \href{../works/Simonis99.pdf}{Simonis99}~\cite{Simonis99}, \href{../works/GruianK98.pdf}{GruianK98}~\cite{GruianK98}, \href{../works/BeckDSF97a.pdf}{BeckDSF97a}~\cite{BeckDSF97a}, \href{../works/PapeB97.pdf}{PapeB97}~\cite{PapeB97}, \href{../works/Zhou97.pdf}{Zhou97}~\cite{Zhou97} (Total: 114) & \href{../works/BonninMNE24.pdf}{BonninMNE24}~\cite{BonninMNE24}, \href{../works/PrataAN23.pdf}{PrataAN23}~\cite{PrataAN23}, \href{../works/MontemanniD23.pdf}{MontemanniD23}~\cite{MontemanniD23}, \href{../works/GuoZ23.pdf}{GuoZ23}~\cite{GuoZ23}, \href{../works/WessenCSFPM23.pdf}{WessenCSFPM23}~\cite{WessenCSFPM23}, \href{../works/Adelgren2023.pdf}{Adelgren2023}~\cite{Adelgren2023}, \href{../works/CzerniachowskaWZ23.pdf}{CzerniachowskaWZ23}~\cite{CzerniachowskaWZ23}, \href{../works/EfthymiouY23.pdf}{EfthymiouY23}~\cite{EfthymiouY23}, \href{../works/KimCMLLP23.pdf}{KimCMLLP23}~\cite{KimCMLLP23}, \href{../works/SquillaciPR23.pdf}{SquillaciPR23}~\cite{SquillaciPR23}, \href{../works/SvancaraB22.pdf}{SvancaraB22}~\cite{SvancaraB22}, \href{../works/JungblutK22.pdf}{JungblutK22}~\cite{JungblutK22}, \href{../works/ElciOH22.pdf}{ElciOH22}~\cite{ElciOH22}, \href{../works/PohlAK22.pdf}{PohlAK22}~\cite{PohlAK22}, \href{../works/YunusogluY22.pdf}{YunusogluY22}~\cite{YunusogluY22}, \href{../works/SubulanC22.pdf}{SubulanC22}~\cite{SubulanC22}, \href{../works/YuraszeckMPV22.pdf}{YuraszeckMPV22}~\cite{YuraszeckMPV22}, \href{../works/AwadMDMT22.pdf}{AwadMDMT22}~\cite{AwadMDMT22}, \href{../works/ArmstrongGOS22.pdf}{ArmstrongGOS22}~\cite{ArmstrongGOS22}...\href{../works/Caseau97.pdf}{Caseau97}~\cite{Caseau97}, \href{../works/RoweJCA96.pdf}{RoweJCA96}~\cite{RoweJCA96}, \href{../works/Simonis95a.pdf}{Simonis95a}~\cite{Simonis95a}, \href{../works/Goltz95.pdf}{Goltz95}~\cite{Goltz95}, \href{../works/Puget95.pdf}{Puget95}~\cite{Puget95}, \href{../works/BeldiceanuC94.pdf}{BeldiceanuC94}~\cite{BeldiceanuC94}, \href{../works/Nuijten94.pdf}{Nuijten94}~\cite{Nuijten94}, \href{../works/SmithC93.pdf}{SmithC93}~\cite{SmithC93}, \href{../works/MintonJPL92.pdf}{MintonJPL92}~\cite{MintonJPL92}, \href{../works/ErtlK91.pdf}{ErtlK91}~\cite{ErtlK91} (Total: 183)\\
\index{bitbucket}\index{Benchmarks!bitbucket}bitbucket &  1.00 &  & \href{../works/TardivoDFMP23.pdf}{TardivoDFMP23}~\cite{TardivoDFMP23}, \href{../works/Dejemeppe16.pdf}{Dejemeppe16}~\cite{Dejemeppe16} & \href{../works/ThomasKS20.pdf}{ThomasKS20}~\cite{ThomasKS20}, \href{../works/CauwelaertDS20.pdf}{CauwelaertDS20}~\cite{CauwelaertDS20}, \href{../works/HoundjiSW19.pdf}{HoundjiSW19}~\cite{HoundjiSW19}, \href{../works/CappartTSR18.pdf}{CappartTSR18}~\cite{CappartTSR18}, \href{../works/CauwelaertLS18.pdf}{CauwelaertLS18}~\cite{CauwelaertLS18}, \href{../works/He0GLW18.pdf}{He0GLW18}~\cite{He0GLW18}, \href{../works/CappartS17.pdf}{CappartS17}~\cite{CappartS17}, \href{../works/CauwelaertDMS16.pdf}{CauwelaertDMS16}~\cite{CauwelaertDMS16}, \href{../works/GayHLS15.pdf}{GayHLS15}~\cite{GayHLS15}, \href{../works/CauwelaertLS15.pdf}{CauwelaertLS15}~\cite{CauwelaertLS15}, \href{../works/DejemeppeCS15.pdf}{DejemeppeCS15}~\cite{DejemeppeCS15}, \href{../works/GayHS15a.pdf}{GayHS15a}~\cite{GayHS15a}, \href{../works/GayHS15.pdf}{GayHS15}~\cite{GayHS15}, \href{../works/DejemeppeD14.pdf}{DejemeppeD14}~\cite{DejemeppeD14}, \href{../works/HoundjiSWD14.pdf}{HoundjiSWD14}~\cite{HoundjiSWD14}\\
\index{generated instance}\index{Benchmarks!generated instance}generated instance &  1.00 & \href{../works/IsikYA23.pdf}{IsikYA23}~\cite{IsikYA23}, \href{../works/LuoB22.pdf}{LuoB22}~\cite{LuoB22}, \href{../works/abs-1911-04766.pdf}{abs-1911-04766}~\cite{abs-1911-04766} & \href{../works/abs-2312-13682.pdf}{abs-2312-13682}~\cite{abs-2312-13682}, \href{../works/PerezGSL23.pdf}{PerezGSL23}~\cite{PerezGSL23}, \href{../works/OrnekOS20.pdf}{OrnekOS20}~\cite{OrnekOS20}, \href{../works/Godet21a.pdf}{Godet21a}~\cite{Godet21a}, \href{../works/GodetLHS20.pdf}{GodetLHS20}~\cite{GodetLHS20}, \href{../works/KletzanderM20.pdf}{KletzanderM20}~\cite{KletzanderM20}, \href{../works/MejiaY20.pdf}{MejiaY20}~\cite{MejiaY20}, \href{../works/SunTB19.pdf}{SunTB19}~\cite{SunTB19}, \href{../works/Madi-WambaB16.pdf}{Madi-WambaB16}~\cite{Madi-WambaB16}, \href{../works/NattafALR16.pdf}{NattafALR16}~\cite{NattafALR16}, \href{../works/Dejemeppe16.pdf}{Dejemeppe16}~\cite{Dejemeppe16}, \href{../works/KelbelH11.pdf}{KelbelH11}~\cite{KelbelH11}, \href{../works/SchausHMCMD11.pdf}{SchausHMCMD11}~\cite{SchausHMCMD11}, \href{../works/Nuijten94.pdf}{Nuijten94}~\cite{Nuijten94} & \href{../works/abs-2402-00459.pdf}{abs-2402-00459}~\cite{abs-2402-00459}, \href{../works/EfthymiouY23.pdf}{EfthymiouY23}~\cite{EfthymiouY23}, \href{../works/abs-2305-19888.pdf}{abs-2305-19888}~\cite{abs-2305-19888}, \href{../works/Adelgren2023.pdf}{Adelgren2023}~\cite{Adelgren2023}, \href{../works/NaderiBZR23.pdf}{NaderiBZR23}~\cite{NaderiBZR23}, \href{../works/CilKLO22.pdf}{CilKLO22}~\cite{CilKLO22}, \href{../works/TouatBT22.pdf}{TouatBT22}~\cite{TouatBT22}, \href{../works/GhandehariK22.pdf}{GhandehariK22}~\cite{GhandehariK22}, \href{../works/ZhangBB22.pdf}{ZhangBB22}~\cite{ZhangBB22}, \href{../works/abs-2211-14492.pdf}{abs-2211-14492}~\cite{abs-2211-14492}, \href{../works/ColT22.pdf}{ColT22}~\cite{ColT22}, \href{../works/YunusogluY22.pdf}{YunusogluY22}~\cite{YunusogluY22}, \href{../works/BoudreaultSLQ22.pdf}{BoudreaultSLQ22}~\cite{BoudreaultSLQ22}, \href{../works/YuraszeckMPV22.pdf}{YuraszeckMPV22}~\cite{YuraszeckMPV22}, \href{../works/HeinzNVH22.pdf}{HeinzNVH22}~\cite{HeinzNVH22}, \href{../works/HanenKP21.pdf}{HanenKP21}~\cite{HanenKP21}, \href{../works/Astrand21.pdf}{Astrand21}~\cite{Astrand21}, \href{../works/AbreuAPNM21.pdf}{AbreuAPNM21}~\cite{AbreuAPNM21}, \href{../works/GeibingerMM21.pdf}{GeibingerMM21}~\cite{GeibingerMM21}...\href{../works/KendallKRU10.pdf}{KendallKRU10}~\cite{KendallKRU10}, \href{../works/TanSD10.pdf}{TanSD10}~\cite{TanSD10}, \href{../works/Hooker07.pdf}{Hooker07}~\cite{Hooker07}, \href{../works/KusterJF07.pdf}{KusterJF07}~\cite{KusterJF07}, \href{../works/SadykovW06.pdf}{SadykovW06}~\cite{SadykovW06}, \href{../works/KovacsV06.pdf}{KovacsV06}~\cite{KovacsV06}, \href{../works/RasmussenT06.pdf}{RasmussenT06}~\cite{RasmussenT06}, \href{../works/ArtiouchineB05.pdf}{ArtiouchineB05}~\cite{ArtiouchineB05}, \href{../works/LimRX04.pdf}{LimRX04}~\cite{LimRX04}, \href{../works/HookerO03.pdf}{HookerO03}~\cite{HookerO03} (Total: 74)\\
\index{github}\index{Benchmarks!github}github &  1.00 & \href{../works/Lemos21.pdf}{Lemos21}~\cite{Lemos21}, \href{../works/Godet21a.pdf}{Godet21a}~\cite{Godet21a}, \href{../works/KoehlerBFFHPSSS21.pdf}{KoehlerBFFHPSSS21}~\cite{KoehlerBFFHPSSS21} & \href{../works/FalqueALM24.pdf}{FalqueALM24}~\cite{FalqueALM24}, \href{../works/PovedaAA23.pdf}{PovedaAA23}~\cite{PovedaAA23}, \href{../works/TardivoDFMP23.pdf}{TardivoDFMP23}~\cite{TardivoDFMP23}, \href{../works/BoudreaultSLQ22.pdf}{BoudreaultSLQ22}~\cite{BoudreaultSLQ22}, \href{../works/JungblutK22.pdf}{JungblutK22}~\cite{JungblutK22}, \href{../works/HamPK21.pdf}{HamPK21}~\cite{HamPK21}, \href{../works/LunardiBLRV20.pdf}{LunardiBLRV20}~\cite{LunardiBLRV20}, \href{../works/GodetLHS20.pdf}{GodetLHS20}~\cite{GodetLHS20}, \href{../works/BenediktMH20.pdf}{BenediktMH20}~\cite{BenediktMH20}, \href{../works/Siala15.pdf}{Siala15}~\cite{Siala15}, \href{../works/Siala15a.pdf}{Siala15a}~\cite{Siala15a} & \href{../works/LiLZDZW24.pdf}{LiLZDZW24}~\cite{LiLZDZW24}, \href{../works/ForbesHJST24.pdf}{ForbesHJST24}~\cite{ForbesHJST24}, \href{../works/abs-2402-00459.pdf}{abs-2402-00459}~\cite{abs-2402-00459}, \href{../works/SquillaciPR23.pdf}{SquillaciPR23}~\cite{SquillaciPR23}, \href{../works/JuvinHHL23.pdf}{JuvinHHL23}~\cite{JuvinHHL23}, \href{../works/YuraszeckMC23.pdf}{YuraszeckMC23}~\cite{YuraszeckMC23}, \href{../works/abs-2306-05747.pdf}{abs-2306-05747}~\cite{abs-2306-05747}, \href{../works/NaderiRR23.pdf}{NaderiRR23}~\cite{NaderiRR23}, \href{../works/Adelgren2023.pdf}{Adelgren2023}~\cite{Adelgren2023}, \href{../works/TasselGS23.pdf}{TasselGS23}~\cite{TasselGS23}, \href{../works/WessenCSFPM23.pdf}{WessenCSFPM23}~\cite{WessenCSFPM23}, \href{../works/YuraszeckMCCR23.pdf}{YuraszeckMCCR23}~\cite{YuraszeckMCCR23}, \href{../works/Fatemi-AnarakiTFV23.pdf}{Fatemi-AnarakiTFV23}~\cite{Fatemi-AnarakiTFV23}, \href{../works/GuoZ23.pdf}{GuoZ23}~\cite{GuoZ23}, \href{../works/GokPTGO23.pdf}{GokPTGO23}~\cite{GokPTGO23}, \href{../works/Bit-Monnot23.pdf}{Bit-Monnot23}~\cite{Bit-Monnot23}, \href{../works/IklassovMR023.pdf}{IklassovMR023}~\cite{IklassovMR023}, \href{../works/OuelletQ22.pdf}{OuelletQ22}~\cite{OuelletQ22}, \href{../works/EmdeZD22.pdf}{EmdeZD22}~\cite{EmdeZD22}...\href{../works/GoldwaserS18.pdf}{GoldwaserS18}~\cite{GoldwaserS18}, \href{../works/ShinBBHO18.pdf}{ShinBBHO18}~\cite{ShinBBHO18}, \href{../works/BenediktSMVH18.pdf}{BenediktSMVH18}~\cite{BenediktSMVH18}, \href{../works/CarlssonJL17.pdf}{CarlssonJL17}~\cite{CarlssonJL17}, \href{../works/GoldwaserS17.pdf}{GoldwaserS17}~\cite{GoldwaserS17}, \href{../works/YoungFS17.pdf}{YoungFS17}~\cite{YoungFS17}, \href{../works/LiuCGM17.pdf}{LiuCGM17}~\cite{LiuCGM17}, \href{../works/BonfiettiZLM16.pdf}{BonfiettiZLM16}~\cite{BonfiettiZLM16}, \href{../works/AmadiniGM16.pdf}{AmadiniGM16}~\cite{AmadiniGM16}, \href{../works/SialaAH15.pdf}{SialaAH15}~\cite{SialaAH15} (Total: 56)\\
\index{gitlab}\index{Benchmarks!gitlab}gitlab &  1.00 &  & \href{../works/HeinzNVH22.pdf}{HeinzNVH22}~\cite{HeinzNVH22} & \href{../works/FalqueALM24.pdf}{FalqueALM24}~\cite{FalqueALM24}, \href{../works/abs-2305-19888.pdf}{abs-2305-19888}~\cite{abs-2305-19888}, \href{../works/BoudreaultSLQ22.pdf}{BoudreaultSLQ22}~\cite{BoudreaultSLQ22}, \href{../works/AntuoriHHEN21.pdf}{AntuoriHHEN21}~\cite{AntuoriHHEN21}, \href{../works/AntuoriHHEN20.pdf}{AntuoriHHEN20}~\cite{AntuoriHHEN20}\\
\index{industrial instance}\index{Benchmarks!industrial instance}industrial instance &  1.00 & \href{../works/LuoB22.pdf}{LuoB22}~\cite{LuoB22}, \href{../works/AntuoriHHEN20.pdf}{AntuoriHHEN20}~\cite{AntuoriHHEN20} & \href{../works/BonfiettiZLM16.pdf}{BonfiettiZLM16}~\cite{BonfiettiZLM16}, \href{../works/BonfiettiLBM14.pdf}{BonfiettiLBM14}~\cite{BonfiettiLBM14}, \href{../works/Schutt11.pdf}{Schutt11}~\cite{Schutt11} & \href{../works/TasselGS23.pdf}{TasselGS23}~\cite{TasselGS23}, \href{../works/PovedaAA23.pdf}{PovedaAA23}~\cite{PovedaAA23}, \href{../works/EfthymiouY23.pdf}{EfthymiouY23}~\cite{EfthymiouY23}, \href{../works/abs-2306-05747.pdf}{abs-2306-05747}~\cite{abs-2306-05747}, \href{../works/Tassel22.pdf}{Tassel22}~\cite{Tassel22}, \href{../works/OujanaAYB22.pdf}{OujanaAYB22}~\cite{OujanaAYB22}, \href{../works/GroleazNS20.pdf}{GroleazNS20}~\cite{GroleazNS20}, \href{../works/NattafM20.pdf}{NattafM20}~\cite{NattafM20}, \href{../works/Mercier-AubinGQ20.pdf}{Mercier-AubinGQ20}~\cite{Mercier-AubinGQ20}, \href{../works/MalapertN19.pdf}{MalapertN19}~\cite{MalapertN19}, \href{../works/Hooker19.pdf}{Hooker19}~\cite{Hooker19}, \href{../works/BofillGSV15.pdf}{BofillGSV15}~\cite{BofillGSV15}, \href{../works/BofillEGPSV14.pdf}{BofillEGPSV14}~\cite{BofillEGPSV14}, \href{../works/ZampelliVSDR13.pdf}{ZampelliVSDR13}~\cite{ZampelliVSDR13}, \href{../works/BonfiettiM12.pdf}{BonfiettiM12}~\cite{BonfiettiM12}, \href{../works/LombardiBMB11.pdf}{LombardiBMB11}~\cite{LombardiBMB11}, \href{../works/BonfiettiLBM11.pdf}{BonfiettiLBM11}~\cite{BonfiettiLBM11}\\
\index{industrial partner}\index{Benchmarks!industrial partner}industrial partner &  1.00 & \href{../works/BoudreaultSLQ22.pdf}{BoudreaultSLQ22}~\cite{BoudreaultSLQ22}, \href{../works/Lunardi20.pdf}{Lunardi20}~\cite{Lunardi20}, \href{../works/Dejemeppe16.pdf}{Dejemeppe16}~\cite{Dejemeppe16} & \href{../works/LacknerMMWW23.pdf}{LacknerMMWW23}~\cite{LacknerMMWW23}, \href{../works/ArmstrongGOS21.pdf}{ArmstrongGOS21}~\cite{ArmstrongGOS21}, \href{../works/ZampelliVSDR13.pdf}{ZampelliVSDR13}~\cite{ZampelliVSDR13} & \href{../works/WinterMMW22.pdf}{WinterMMW22}~\cite{WinterMMW22}, \href{../works/Tassel22.pdf}{Tassel22}~\cite{Tassel22}, \href{../works/VlkHT21.pdf}{VlkHT21}~\cite{VlkHT21}, \href{../works/LacknerMMWW21.pdf}{LacknerMMWW21}~\cite{LacknerMMWW21}, \href{../works/GroleazNS20a.pdf}{GroleazNS20a}~\cite{GroleazNS20a}, \href{../works/Mercier-AubinGQ20.pdf}{Mercier-AubinGQ20}~\cite{Mercier-AubinGQ20}, \href{../works/AntunesABD20.pdf}{AntunesABD20}~\cite{AntunesABD20}, \href{../works/abs-1911-04766.pdf}{abs-1911-04766}~\cite{abs-1911-04766}, \href{../works/GeibingerMM19.pdf}{GeibingerMM19}~\cite{GeibingerMM19}, \href{../works/AntunesABD18.pdf}{AntunesABD18}~\cite{AntunesABD18}, \href{../works/MossigeGSMC17.pdf}{MossigeGSMC17}~\cite{MossigeGSMC17}, \href{../works/Froger16.pdf}{Froger16}~\cite{Froger16}, \href{../works/HebrardHJMPV16.pdf}{HebrardHJMPV16}~\cite{HebrardHJMPV16}, \href{../works/AlesioBNG15.pdf}{AlesioBNG15}~\cite{AlesioBNG15}, \href{../works/LipovetzkyBPS14.pdf}{LipovetzkyBPS14}~\cite{LipovetzkyBPS14}, \href{../works/LimtanyakulS12.pdf}{LimtanyakulS12}~\cite{LimtanyakulS12}, \href{../works/Malapert11.pdf}{Malapert11}~\cite{Malapert11}, \href{../works/DoRZ08.pdf}{DoRZ08}~\cite{DoRZ08}, \href{../works/KovacsV06.pdf}{KovacsV06}~\cite{KovacsV06}, \href{../works/KovacsV04.pdf}{KovacsV04}~\cite{KovacsV04}, \href{../works/EreminW01.pdf}{EreminW01}~\cite{EreminW01}\\
\index{industry partner}\index{Benchmarks!industry partner}industry partner &  1.00 & \href{../works/BurtLPS15.pdf}{BurtLPS15}~\cite{BurtLPS15}, \href{../works/LipovetzkyBPS14.pdf}{LipovetzkyBPS14}~\cite{LipovetzkyBPS14} & \href{../works/BlomBPS14.pdf}{BlomBPS14}~\cite{BlomBPS14} & \href{../works/LuoB22.pdf}{LuoB22}~\cite{LuoB22}, \href{../works/WinterMMW22.pdf}{WinterMMW22}~\cite{WinterMMW22}, \href{../works/ArmstrongGOS21.pdf}{ArmstrongGOS21}~\cite{ArmstrongGOS21}, \href{../works/HauderBRPA20.pdf}{HauderBRPA20}~\cite{HauderBRPA20}, \href{../works/abs-1902-09244.pdf}{abs-1902-09244}~\cite{abs-1902-09244}, \href{../works/AntunesABD18.pdf}{AntunesABD18}~\cite{AntunesABD18}, \href{../works/BlomPS16.pdf}{BlomPS16}~\cite{BlomPS16}\\
\index{instance generator}\index{Benchmarks!instance generator}instance generator &  1.00 & \href{../works/LacknerMMWW23.pdf}{LacknerMMWW23}~\cite{LacknerMMWW23}, \href{../works/LacknerMMWW21.pdf}{LacknerMMWW21}~\cite{LacknerMMWW21} & \href{../works/GoldwaserS18.pdf}{GoldwaserS18}~\cite{GoldwaserS18}, \href{../works/Froger16.pdf}{Froger16}~\cite{Froger16} & \href{../works/abs-2402-00459.pdf}{abs-2402-00459}~\cite{abs-2402-00459}, \href{../works/FrimodigECM23.pdf}{FrimodigECM23}~\cite{FrimodigECM23}, \href{../works/ArmstrongGOS21.pdf}{ArmstrongGOS21}~\cite{ArmstrongGOS21}, \href{../works/Lunardi20.pdf}{Lunardi20}~\cite{Lunardi20}, \href{../works/KletzanderM20.pdf}{KletzanderM20}~\cite{KletzanderM20}, \href{../works/SunTB19.pdf}{SunTB19}~\cite{SunTB19}, \href{../works/abs-1911-04766.pdf}{abs-1911-04766}~\cite{abs-1911-04766}, \href{../works/Caballero19.pdf}{Caballero19}~\cite{Caballero19}, \href{../works/GombolayWS18.pdf}{GombolayWS18}~\cite{GombolayWS18}, \href{../works/GoldwaserS17.pdf}{GoldwaserS17}~\cite{GoldwaserS17}, \href{../works/YoungFS17.pdf}{YoungFS17}~\cite{YoungFS17}, \href{../works/Dejemeppe16.pdf}{Dejemeppe16}~\cite{Dejemeppe16}, \href{../works/UnsalO13.pdf}{UnsalO13}~\cite{UnsalO13}, \href{../works/GuyonLPR12.pdf}{GuyonLPR12}~\cite{GuyonLPR12}, \href{../works/Schutt11.pdf}{Schutt11}~\cite{Schutt11}, \href{../works/BeniniLMR11.pdf}{BeniniLMR11}~\cite{BeniniLMR11}, \href{../works/Lombardi10.pdf}{Lombardi10}~\cite{Lombardi10}, \href{../works/abs-1009-0347.pdf}{abs-1009-0347}~\cite{abs-1009-0347}, \href{../works/RuggieroBBMA09.pdf}{RuggieroBBMA09}~\cite{RuggieroBBMA09}, \href{../works/LombardiM09.pdf}{LombardiM09}~\cite{LombardiM09}, \href{../works/HeipckeCCS00.pdf}{HeipckeCCS00}~\cite{HeipckeCCS00}\\
\index{random instance}\index{Benchmarks!random instance}random instance &  1.00 & \href{../works/LacknerMMWW21.pdf}{LacknerMMWW21}~\cite{LacknerMMWW21}, \href{../works/WallaceY20.pdf}{WallaceY20}~\cite{WallaceY20}, \href{../works/Dejemeppe16.pdf}{Dejemeppe16}~\cite{Dejemeppe16} & \href{../works/WangB23.pdf}{WangB23}~\cite{WangB23}, \href{../works/LacknerMMWW23.pdf}{LacknerMMWW23}~\cite{LacknerMMWW23}, \href{../works/EfthymiouY23.pdf}{EfthymiouY23}~\cite{EfthymiouY23}, \href{../works/CarlssonJL17.pdf}{CarlssonJL17}~\cite{CarlssonJL17}, \href{../works/LetortCB15.pdf}{LetortCB15}~\cite{LetortCB15}, \href{../works/ZhaoL14.pdf}{ZhaoL14}~\cite{ZhaoL14}, \href{../works/KelbelH11.pdf}{KelbelH11}~\cite{KelbelH11} & \href{../works/Mehdizadeh-Somarin23.pdf}{Mehdizadeh-Somarin23}~\cite{Mehdizadeh-Somarin23}, \href{../works/Fatemi-AnarakiTFV23.pdf}{Fatemi-AnarakiTFV23}~\cite{Fatemi-AnarakiTFV23}, \href{../works/OuelletQ22.pdf}{OuelletQ22}~\cite{OuelletQ22}, \href{../works/ElciOH22.pdf}{ElciOH22}~\cite{ElciOH22}, \href{../works/MullerMKP22.pdf}{MullerMKP22}~\cite{MullerMKP22}, \href{../works/EmdeZD22.pdf}{EmdeZD22}~\cite{EmdeZD22}, \href{../works/abs-2211-14492.pdf}{abs-2211-14492}~\cite{abs-2211-14492}, \href{../works/VlkHT21.pdf}{VlkHT21}~\cite{VlkHT21}, \href{../works/Godet21a.pdf}{Godet21a}~\cite{Godet21a}, \href{../works/KlankeBYE21.pdf}{KlankeBYE21}~\cite{KlankeBYE21}, \href{../works/HanenKP21.pdf}{HanenKP21}~\cite{HanenKP21}, \href{../works/AntuoriHHEN20.pdf}{AntuoriHHEN20}~\cite{AntuoriHHEN20}, \href{../works/Lunardi20.pdf}{Lunardi20}~\cite{Lunardi20}, \href{../works/BenediktMH20.pdf}{BenediktMH20}~\cite{BenediktMH20}, \href{../works/LunardiBLRV20.pdf}{LunardiBLRV20}~\cite{LunardiBLRV20}, \href{../works/HoundjiSW19.pdf}{HoundjiSW19}~\cite{HoundjiSW19}, \href{../works/UnsalO19.pdf}{UnsalO19}~\cite{UnsalO19}, \href{../works/FahimiOQ18.pdf}{FahimiOQ18}~\cite{FahimiOQ18}, \href{../works/BenediktSMVH18.pdf}{BenediktSMVH18}~\cite{BenediktSMVH18}...\href{../works/LimtanyakulS12.pdf}{LimtanyakulS12}~\cite{LimtanyakulS12}, \href{../works/BartakS11.pdf}{BartakS11}~\cite{BartakS11}, \href{../works/CobanH11.pdf}{CobanH11}~\cite{CobanH11}, \href{../works/BandaSC11.pdf}{BandaSC11}~\cite{BandaSC11}, \href{../works/Hooker07.pdf}{Hooker07}~\cite{Hooker07}, \href{../works/Hooker06.pdf}{Hooker06}~\cite{Hooker06}, \href{../works/Hooker05.pdf}{Hooker05}~\cite{Hooker05}, \href{../works/ArtiouchineB05.pdf}{ArtiouchineB05}~\cite{ArtiouchineB05}, \href{../works/Hooker04.pdf}{Hooker04}~\cite{Hooker04}, \href{../works/ElfJR03.pdf}{ElfJR03}~\cite{ElfJR03} (Total: 47)\\
\index{real-life}\index{Benchmarks!real-life}real-life &  1.00 & \href{../works/GurPAE23.pdf}{GurPAE23}~\cite{GurPAE23}, \href{../works/SubulanC22.pdf}{SubulanC22}~\cite{SubulanC22}, \href{../works/WinterMMW22.pdf}{WinterMMW22}~\cite{WinterMMW22}, \href{../works/HubnerGSV21.pdf}{HubnerGSV21}~\cite{HubnerGSV21}, \href{../works/Astrand21.pdf}{Astrand21}~\cite{Astrand21}, \href{../works/QinDCS20.pdf}{QinDCS20}~\cite{QinDCS20}, \href{../works/KletzanderM20.pdf}{KletzanderM20}~\cite{KletzanderM20}, \href{../works/GurEA19.pdf}{GurEA19}~\cite{GurEA19}, \href{../works/RiiseML16.pdf}{RiiseML16}~\cite{RiiseML16}, \href{../works/WangMD15.pdf}{WangMD15}~\cite{WangMD15}, \href{../works/MeskensDL13.pdf}{MeskensDL13}~\cite{MeskensDL13}, \href{../works/Ribeiro12.pdf}{Ribeiro12}~\cite{Ribeiro12}, \href{../works/BartakCS10.pdf}{BartakCS10}~\cite{BartakCS10}, \href{../works/ChenGPSH10.pdf}{ChenGPSH10}~\cite{ChenGPSH10}, \href{../works/BartakSR10.pdf}{BartakSR10}~\cite{BartakSR10}, \href{../works/KendallKRU10.pdf}{KendallKRU10}~\cite{KendallKRU10}, \href{../works/Baptiste02.pdf}{Baptiste02}~\cite{Baptiste02}, \href{../works/Bartak02a.pdf}{Bartak02a}~\cite{Bartak02a}, \href{../works/MartinPY01.pdf}{MartinPY01}~\cite{MartinPY01}, \href{../works/Nuijten94.pdf}{Nuijten94}~\cite{Nuijten94} & \href{../works/LuZZYW24.pdf}{LuZZYW24}~\cite{LuZZYW24}, \href{../works/AlakaP23.pdf}{AlakaP23}~\cite{AlakaP23}, \href{../works/AfsarVPG23.pdf}{AfsarVPG23}~\cite{AfsarVPG23}, \href{../works/LacknerMMWW23.pdf}{LacknerMMWW23}~\cite{LacknerMMWW23}, \href{../works/OujanaAYB22.pdf}{OujanaAYB22}~\cite{OujanaAYB22}, \href{../works/CilKLO22.pdf}{CilKLO22}~\cite{CilKLO22}, \href{../works/BulckG22.pdf}{BulckG22}~\cite{BulckG22}, \href{../works/Lemos21.pdf}{Lemos21}~\cite{Lemos21}, \href{../works/KlankeBYE21.pdf}{KlankeBYE21}~\cite{KlankeBYE21}, \href{../works/Astrand0F21.pdf}{Astrand0F21}~\cite{Astrand0F21}, \href{../works/LacknerMMWW21.pdf}{LacknerMMWW21}~\cite{LacknerMMWW21}, \href{../works/KletzanderMH21.pdf}{KletzanderMH21}~\cite{KletzanderMH21}, \href{../works/FallahiAC20.pdf}{FallahiAC20}~\cite{FallahiAC20}, \href{../works/Lunardi20.pdf}{Lunardi20}~\cite{Lunardi20}, \href{../works/PinarbasiAY19.pdf}{PinarbasiAY19}~\cite{PinarbasiAY19}, \href{../works/abs-1911-04766.pdf}{abs-1911-04766}~\cite{abs-1911-04766}, \href{../works/PourDERB18.pdf}{PourDERB18}~\cite{PourDERB18}, \href{../works/MusliuSS18.pdf}{MusliuSS18}~\cite{MusliuSS18}, \href{../works/AmadiniGM16.pdf}{AmadiniGM16}~\cite{AmadiniGM16}...\href{../works/MeskensDHG11.pdf}{MeskensDHG11}~\cite{MeskensDHG11}, \href{../works/LombardiMRB10.pdf}{LombardiMRB10}~\cite{LombardiMRB10}, \href{../works/RuggieroBBMA09.pdf}{RuggieroBBMA09}~\cite{RuggieroBBMA09}, \href{../works/BartakSR08.pdf}{BartakSR08}~\cite{BartakSR08}, \href{../works/Musliu05.pdf}{Musliu05}~\cite{Musliu05}, \href{../works/Tsang03.pdf}{Tsang03}~\cite{Tsang03}, \href{../works/BosiM2001.pdf}{BosiM2001}~\cite{BosiM2001}, \href{../works/JainM99.pdf}{JainM99}~\cite{JainM99}, \href{../works/NuijtenP98.pdf}{NuijtenP98}~\cite{NuijtenP98}, \href{../works/SimonisC95.pdf}{SimonisC95}~\cite{SimonisC95} (Total: 35) & \href{../works/PrataAN23.pdf}{PrataAN23}~\cite{PrataAN23}, \href{../works/LiLZDZW24.pdf}{LiLZDZW24}~\cite{LiLZDZW24}, \href{../works/BonninMNE24.pdf}{BonninMNE24}~\cite{BonninMNE24}, \href{../works/ForbesHJST24.pdf}{ForbesHJST24}~\cite{ForbesHJST24}, \href{../works/Adelgren2023.pdf}{Adelgren2023}~\cite{Adelgren2023}, \href{../works/AbreuPNF23.pdf}{AbreuPNF23}~\cite{AbreuPNF23}, \href{../works/IsikYA23.pdf}{IsikYA23}~\cite{IsikYA23}, \href{../works/NaderiBZR23.pdf}{NaderiBZR23}~\cite{NaderiBZR23}, \href{../works/EfthymiouY23.pdf}{EfthymiouY23}~\cite{EfthymiouY23}, \href{../works/PovedaAA23.pdf}{PovedaAA23}~\cite{PovedaAA23}, \href{../works/LuoB22.pdf}{LuoB22}~\cite{LuoB22}, \href{../works/GeitzGSSW22.pdf}{GeitzGSSW22}~\cite{GeitzGSSW22}, \href{../works/AwadMDMT22.pdf}{AwadMDMT22}~\cite{AwadMDMT22}, \href{../works/NaderiBZ22.pdf}{NaderiBZ22}~\cite{NaderiBZ22}, \href{../works/YunusogluY22.pdf}{YunusogluY22}~\cite{YunusogluY22}, \href{../works/CampeauG22.pdf}{CampeauG22}~\cite{CampeauG22}, \href{../works/NaqviAIAAA22.pdf}{NaqviAIAAA22}~\cite{NaqviAIAAA22}, \href{../works/YuraszeckMPV22.pdf}{YuraszeckMPV22}~\cite{YuraszeckMPV22}, \href{../works/ColT22.pdf}{ColT22}~\cite{ColT22}...\href{../works/Simonis99.pdf}{Simonis99}~\cite{Simonis99}, \href{../works/PesantGPR99.pdf}{PesantGPR99}~\cite{PesantGPR99}, \href{../works/BelhadjiI98.pdf}{BelhadjiI98}~\cite{BelhadjiI98}, \href{../works/LammaMM97.pdf}{LammaMM97}~\cite{LammaMM97}, \href{../works/Darby-DowmanLMZ97.pdf}{Darby-DowmanLMZ97}~\cite{Darby-DowmanLMZ97}, \href{../works/Schaerf97.pdf}{Schaerf97}~\cite{Schaerf97}, \href{../works/PapeB97.pdf}{PapeB97}~\cite{PapeB97}, \href{../works/SmithBHW96.pdf}{SmithBHW96}~\cite{SmithBHW96}, \href{../works/Touraivane95.pdf}{Touraivane95}~\cite{Touraivane95}, \href{../works/Simonis95a.pdf}{Simonis95a}~\cite{Simonis95a} (Total: 126)\\
\index{real-world}\index{Benchmarks!real-world}real-world &  1.00 & \href{../works/LuZZYW24.pdf}{LuZZYW24}~\cite{LuZZYW24}, \href{../works/GokPTGO23.pdf}{GokPTGO23}~\cite{GokPTGO23}, \href{../works/abs-2305-19888.pdf}{abs-2305-19888}~\cite{abs-2305-19888}, \href{../works/HeinzNVH22.pdf}{HeinzNVH22}~\cite{HeinzNVH22}, \href{../works/YunusogluY22.pdf}{YunusogluY22}~\cite{YunusogluY22}, \href{../works/ColT22.pdf}{ColT22}~\cite{ColT22}, \href{../works/GeibingerMM21.pdf}{GeibingerMM21}~\cite{GeibingerMM21}, \href{../works/KoehlerBFFHPSSS21.pdf}{KoehlerBFFHPSSS21}~\cite{KoehlerBFFHPSSS21}, \href{../works/Lemos21.pdf}{Lemos21}~\cite{Lemos21}, \href{../works/Astrand21.pdf}{Astrand21}~\cite{Astrand21}, \href{../works/Lunardi20.pdf}{Lunardi20}~\cite{Lunardi20}, \href{../works/MokhtarzadehTNF20.pdf}{MokhtarzadehTNF20}~\cite{MokhtarzadehTNF20}, \href{../works/HauderBRPA20.pdf}{HauderBRPA20}~\cite{HauderBRPA20}, \href{../works/abs-1911-04766.pdf}{abs-1911-04766}~\cite{abs-1911-04766}, \href{../works/GeibingerMM19.pdf}{GeibingerMM19}~\cite{GeibingerMM19}, \href{../works/abs-1902-09244.pdf}{abs-1902-09244}~\cite{abs-1902-09244}, \href{../works/FrohnerTR19.pdf}{FrohnerTR19}~\cite{FrohnerTR19}, \href{../works/SenderovichBB19.pdf}{SenderovichBB19}~\cite{SenderovichBB19}, \href{../works/GombolayWS18.pdf}{GombolayWS18}~\cite{GombolayWS18}...\href{../works/EvenSH15a.pdf}{EvenSH15a}~\cite{EvenSH15a}, \href{../works/EvenSH15.pdf}{EvenSH15}~\cite{EvenSH15}, \href{../works/UnsalO13.pdf}{UnsalO13}~\cite{UnsalO13}, \href{../works/RendlPHPR12.pdf}{RendlPHPR12}~\cite{RendlPHPR12}, \href{../works/KendallKRU10.pdf}{KendallKRU10}~\cite{KendallKRU10}, \href{../works/Lombardi10.pdf}{Lombardi10}~\cite{Lombardi10}, \href{../works/MouraSCL08a.pdf}{MouraSCL08a}~\cite{MouraSCL08a}, \href{../works/WatsonBHW99.pdf}{WatsonBHW99}~\cite{WatsonBHW99}, \href{../works/Beck99.pdf}{Beck99}~\cite{Beck99}, \href{../works/BeckDDF98.pdf}{BeckDDF98}~\cite{BeckDDF98} (Total: 33) & \href{../works/PrataAN23.pdf}{PrataAN23}~\cite{PrataAN23}, \href{../works/TasselGS23.pdf}{TasselGS23}~\cite{TasselGS23}, \href{../works/WessenCSFPM23.pdf}{WessenCSFPM23}~\cite{WessenCSFPM23}, \href{../works/abs-2306-05747.pdf}{abs-2306-05747}~\cite{abs-2306-05747}, \href{../works/AbreuNP23.pdf}{AbreuNP23}~\cite{AbreuNP23}, \href{../works/BofillCGGPSV23.pdf}{BofillCGGPSV23}~\cite{BofillCGGPSV23}, \href{../works/IsikYA23.pdf}{IsikYA23}~\cite{IsikYA23}, \href{../works/Fatemi-AnarakiTFV23.pdf}{Fatemi-AnarakiTFV23}~\cite{Fatemi-AnarakiTFV23}, \href{../works/AalianPG23.pdf}{AalianPG23}~\cite{AalianPG23}, \href{../works/AbreuPNF23.pdf}{AbreuPNF23}~\cite{AbreuPNF23}, \href{../works/WangB23.pdf}{WangB23}~\cite{WangB23}, \href{../works/YuraszeckMCCR23.pdf}{YuraszeckMCCR23}~\cite{YuraszeckMCCR23}, \href{../works/OujanaAYB22.pdf}{OujanaAYB22}~\cite{OujanaAYB22}, \href{../works/LuoB22.pdf}{LuoB22}~\cite{LuoB22}, \href{../works/SvancaraB22.pdf}{SvancaraB22}~\cite{SvancaraB22}, \href{../works/MullerMKP22.pdf}{MullerMKP22}~\cite{MullerMKP22}, \href{../works/ArmstrongGOS21.pdf}{ArmstrongGOS21}~\cite{ArmstrongGOS21}, \href{../works/ZarandiASC20.pdf}{ZarandiASC20}~\cite{ZarandiASC20}, \href{../works/WallaceY20.pdf}{WallaceY20}~\cite{WallaceY20}...\href{../works/Schutt11.pdf}{Schutt11}~\cite{Schutt11}, \href{../works/BartakS11.pdf}{BartakS11}~\cite{BartakS11}, \href{../works/LopesCSM10.pdf}{LopesCSM10}~\cite{LopesCSM10}, \href{../works/LombardiM10a.pdf}{LombardiM10a}~\cite{LombardiM10a}, \href{../works/BidotVLB09.pdf}{BidotVLB09}~\cite{BidotVLB09}, \href{../works/DoRZ08.pdf}{DoRZ08}~\cite{DoRZ08}, \href{../works/LiW08.pdf}{LiW08}~\cite{LiW08}, \href{../works/BeckPS03.pdf}{BeckPS03}~\cite{BeckPS03}, \href{../works/BosiM2001.pdf}{BosiM2001}~\cite{BosiM2001}, \href{../works/BeckF98.pdf}{BeckF98}~\cite{BeckF98} (Total: 56) & \href{../works/FalqueALM24.pdf}{FalqueALM24}~\cite{FalqueALM24}, \href{../works/abs-2402-00459.pdf}{abs-2402-00459}~\cite{abs-2402-00459}, \href{../works/ZhuSZW23.pdf}{ZhuSZW23}~\cite{ZhuSZW23}, \href{../works/GuoZ23.pdf}{GuoZ23}~\cite{GuoZ23}, \href{../works/IklassovMR023.pdf}{IklassovMR023}~\cite{IklassovMR023}, \href{../works/PovedaAA23.pdf}{PovedaAA23}~\cite{PovedaAA23}, \href{../works/Bit-Monnot23.pdf}{Bit-Monnot23}~\cite{Bit-Monnot23}, \href{../works/TardivoDFMP23.pdf}{TardivoDFMP23}~\cite{TardivoDFMP23}, \href{../works/CzerniachowskaWZ23.pdf}{CzerniachowskaWZ23}~\cite{CzerniachowskaWZ23}, \href{../works/FrimodigECM23.pdf}{FrimodigECM23}~\cite{FrimodigECM23}, \href{../works/abs-2312-13682.pdf}{abs-2312-13682}~\cite{abs-2312-13682}, \href{../works/KimCMLLP23.pdf}{KimCMLLP23}~\cite{KimCMLLP23}, \href{../works/NaderiBZR23.pdf}{NaderiBZR23}~\cite{NaderiBZR23}, \href{../works/JuvinHL23.pdf}{JuvinHL23}~\cite{JuvinHL23}, \href{../works/PerezGSL23.pdf}{PerezGSL23}~\cite{PerezGSL23}, \href{../works/NaderiBZ23.pdf}{NaderiBZ23}~\cite{NaderiBZ23}, \href{../works/ShaikhK23.pdf}{ShaikhK23}~\cite{ShaikhK23}, \href{../works/AfsarVPG23.pdf}{AfsarVPG23}~\cite{AfsarVPG23}, \href{../works/MarliereSPR23.pdf}{MarliereSPR23}~\cite{MarliereSPR23}...\href{../works/Darby-DowmanLMZ97.pdf}{Darby-DowmanLMZ97}~\cite{Darby-DowmanLMZ97}, \href{../works/BeckDF97.pdf}{BeckDF97}~\cite{BeckDF97}, \href{../works/BeckDSF97a.pdf}{BeckDSF97a}~\cite{BeckDSF97a}, \href{../works/RoweJCA96.pdf}{RoweJCA96}~\cite{RoweJCA96}, \href{../works/BeldiceanuC94.pdf}{BeldiceanuC94}~\cite{BeldiceanuC94}, \href{../works/AggounB93.pdf}{AggounB93}~\cite{AggounB93}, \href{../works/MintonJPL92.pdf}{MintonJPL92}~\cite{MintonJPL92}, \href{../works/ErtlK91.pdf}{ErtlK91}~\cite{ErtlK91}, \href{../works/FoxS90.pdf}{FoxS90}~\cite{FoxS90}, \href{../works/MintonJPL90.pdf}{MintonJPL90}~\cite{MintonJPL90} (Total: 186)\\
\index{supplementary material}\index{Benchmarks!supplementary material}supplementary material &  1.00 & \href{../works/GuoZ23.pdf}{GuoZ23}~\cite{GuoZ23}, \href{../works/FarsiTM22.pdf}{FarsiTM22}~\cite{FarsiTM22}, \href{../works/MejiaY20.pdf}{MejiaY20}~\cite{MejiaY20}, \href{../works/Lunardi20.pdf}{Lunardi20}~\cite{Lunardi20}, \href{../works/TanZWGQ19.pdf}{TanZWGQ19}~\cite{TanZWGQ19} & \href{../works/NaderiBZR23.pdf}{NaderiBZR23}~\cite{NaderiBZR23}, \href{../works/MontemanniD23.pdf}{MontemanniD23}~\cite{MontemanniD23}, \href{../works/AfsarVPG23.pdf}{AfsarVPG23}~\cite{AfsarVPG23}, \href{../works/FachiniA20.pdf}{FachiniA20}~\cite{FachiniA20}, \href{../works/SchuttFSW13.pdf}{SchuttFSW13}~\cite{SchuttFSW13} & \href{../works/FalqueALM24.pdf}{FalqueALM24}~\cite{FalqueALM24}, \href{../works/JuvinHHL23.pdf}{JuvinHHL23}~\cite{JuvinHHL23}, \href{../works/abs-2306-05747.pdf}{abs-2306-05747}~\cite{abs-2306-05747}, \href{../works/TasselGS23.pdf}{TasselGS23}~\cite{TasselGS23}, \href{../works/Adelgren2023.pdf}{Adelgren2023}~\cite{Adelgren2023}, \href{../works/WinterMMW22.pdf}{WinterMMW22}~\cite{WinterMMW22}, \href{../works/ColT22.pdf}{ColT22}~\cite{ColT22}, \href{../works/BoudreaultSLQ22.pdf}{BoudreaultSLQ22}~\cite{BoudreaultSLQ22}, \href{../works/YunusogluY22.pdf}{YunusogluY22}~\cite{YunusogluY22}, \href{../works/AntuoriHHEN21.pdf}{AntuoriHHEN21}~\cite{AntuoriHHEN21}, \href{../works/LacknerMMWW21.pdf}{LacknerMMWW21}~\cite{LacknerMMWW21}, \href{../works/KovacsTKSG21.pdf}{KovacsTKSG21}~\cite{KovacsTKSG21}, \href{../works/ArmstrongGOS21.pdf}{ArmstrongGOS21}~\cite{ArmstrongGOS21}, \href{../works/MengZRZL20.pdf}{MengZRZL20}~\cite{MengZRZL20}, \href{../works/HauderBRPA20.pdf}{HauderBRPA20}~\cite{HauderBRPA20}, \href{../works/SchnellH17.pdf}{SchnellH17}~\cite{SchnellH17}, \href{../works/SchnellH15.pdf}{SchnellH15}~\cite{SchnellH15}, \href{../works/MenciaSV13.pdf}{MenciaSV13}~\cite{MenciaSV13}\\
\index{zenodo}\index{Benchmarks!zenodo}zenodo &  1.00 & \href{../works/LacknerMMWW23.pdf}{LacknerMMWW23}~\cite{LacknerMMWW23}, \href{../works/SacramentoSP20.pdf}{SacramentoSP20}~\cite{SacramentoSP20} &  & \href{../works/KimCMLLP23.pdf}{KimCMLLP23}~\cite{KimCMLLP23}, \href{../works/WinterMMW22.pdf}{WinterMMW22}~\cite{WinterMMW22}, \href{../works/ArmstrongGOS21.pdf}{ArmstrongGOS21}~\cite{ArmstrongGOS21}\\
\end{longtable}
}

\clearpage
\subsection{Concept Type Algorithms}
\label{sec:Algorithms}
\label{Algorithms}
{\scriptsize
\begin{longtable}{p{3cm}r>{\raggedright\arraybackslash}p{6cm}>{\raggedright\arraybackslash}p{6cm}>{\raggedright\arraybackslash}p{8cm}}
\rowcolor{white}\caption{Works for Concepts of Type Algorithms (Total 36 Concepts, 35 Used)}\\ \toprule
\rowcolor{white}Keyword & Weight & High & Medium & Low\\ \midrule\endhead
\bottomrule
\endfoot
\index{GRASP}\index{Algorithms!GRASP}GRASP &  1.00 & \href{../works/Lemos21.pdf}{Lemos21}~\cite{Lemos21} & \href{../works/YuraszeckMCCR23.pdf}{YuraszeckMCCR23}~\cite{YuraszeckMCCR23}, \href{../works/PovedaAA23.pdf}{PovedaAA23}~\cite{PovedaAA23}, \href{../works/YunusogluY22.pdf}{YunusogluY22}~\cite{YunusogluY22}, \href{../works/RiahiNS018.pdf}{RiahiNS018}~\cite{RiahiNS018}, \href{../works/KendallKRU10.pdf}{KendallKRU10}~\cite{KendallKRU10} & \href{../works/LacknerMMWW23.pdf}{LacknerMMWW23}~\cite{LacknerMMWW23}, \href{../works/IsikYA23.pdf}{IsikYA23}~\cite{IsikYA23}, \href{../works/SquillaciPR23.pdf}{SquillaciPR23}~\cite{SquillaciPR23}, \href{../works/AkramNHRSA23.pdf}{AkramNHRSA23}~\cite{AkramNHRSA23}, \href{../works/ArmstrongGOS22.pdf}{ArmstrongGOS22}~\cite{ArmstrongGOS22}, \href{../works/KletzanderMH21.pdf}{KletzanderMH21}~\cite{KletzanderMH21}, \href{../works/LacknerMMWW21.pdf}{LacknerMMWW21}~\cite{LacknerMMWW21}, \href{../works/Zahout21.pdf}{Zahout21}~\cite{Zahout21}, \href{../works/VlkHT21.pdf}{VlkHT21}~\cite{VlkHT21}, \href{../works/AntuoriHHEN21.pdf}{AntuoriHHEN21}~\cite{AntuoriHHEN21}, \href{../works/GokGSTO20.pdf}{GokGSTO20}~\cite{GokGSTO20}, \href{../works/KletzanderM20.pdf}{KletzanderM20}~\cite{KletzanderM20}, \href{../works/QinDCS20.pdf}{QinDCS20}~\cite{QinDCS20}, \href{../works/MejiaY20.pdf}{MejiaY20}~\cite{MejiaY20}, \href{../works/GroleazNS20a.pdf}{GroleazNS20a}~\cite{GroleazNS20a}, \href{../works/Caballero19.pdf}{Caballero19}~\cite{Caballero19}, \href{../works/KreterSSZ18.pdf}{KreterSSZ18}~\cite{KreterSSZ18}, \href{../works/SchnellH17.pdf}{SchnellH17}~\cite{SchnellH17}, \href{../works/ZhouGL15.pdf}{ZhouGL15}~\cite{ZhouGL15}, \href{../works/Siala15a.pdf}{Siala15a}~\cite{Siala15a}, \href{../works/SchnellH15.pdf}{SchnellH15}~\cite{SchnellH15}, \href{../works/UnsalO13.pdf}{UnsalO13}~\cite{UnsalO13}, \href{../works/SerraNM12.pdf}{SerraNM12}~\cite{SerraNM12}, \href{../works/Ribeiro12.pdf}{Ribeiro12}~\cite{Ribeiro12}, \href{../works/HeinzB12.pdf}{HeinzB12}~\cite{HeinzB12}, \href{../works/AchterbergBKW08.pdf}{AchterbergBKW08}~\cite{AchterbergBKW08}, \href{../works/Rodriguez07.pdf}{Rodriguez07}~\cite{Rodriguez07}, \href{../works/Hooker05b.pdf}{Hooker05b}~\cite{Hooker05b}, \href{../works/JainM99.pdf}{JainM99}~\cite{JainM99}\\
\index{IGT}\index{Algorithms!IGT}IGT &  1.00 & \href{../works/ArmstrongGOS22.pdf}{ArmstrongGOS22}~\cite{ArmstrongGOS22} &  & \\
\index{Lagrangian relaxation}\index{Algorithms!Lagrangian relaxation}Lagrangian relaxation &  1.00 & \href{../works/HookerH17.pdf}{HookerH17}~\cite{HookerH17}, \href{../works/ThiruvadyWGS14.pdf}{ThiruvadyWGS14}~\cite{ThiruvadyWGS14}, \href{../works/GuSS13.pdf}{GuSS13}~\cite{GuSS13}, \href{../works/GuSW12.pdf}{GuSW12}~\cite{GuSW12}, \href{../works/MilanoW09.pdf}{MilanoW09}~\cite{MilanoW09}, \href{../works/MilanoW06.pdf}{MilanoW06}~\cite{MilanoW06}, \href{../works/Mason01.pdf}{Mason01}~\cite{Mason01} & \href{../works/IsikYA23.pdf}{IsikYA23}~\cite{IsikYA23}, \href{../works/AbreuN22.pdf}{AbreuN22}~\cite{AbreuN22}, \href{../works/KotaryFH22.pdf}{KotaryFH22}~\cite{KotaryFH22}, \href{../works/ZarandiASC20.pdf}{ZarandiASC20}~\cite{ZarandiASC20}, \href{../works/Dejemeppe16.pdf}{Dejemeppe16}~\cite{Dejemeppe16}, \href{../works/BlomPS16.pdf}{BlomPS16}~\cite{BlomPS16}, \href{../works/Froger16.pdf}{Froger16}~\cite{Froger16}, \href{../works/EdisO11.pdf}{EdisO11}~\cite{EdisO11}, \href{../works/Wallace06.pdf}{Wallace06}~\cite{Wallace06}, \href{../works/BlazewiczDP96.pdf}{BlazewiczDP96}~\cite{BlazewiczDP96} & \href{../works/abs-2402-00459.pdf}{abs-2402-00459}~\cite{abs-2402-00459}, \href{../works/abs-2305-19888.pdf}{abs-2305-19888}~\cite{abs-2305-19888}, \href{../works/MarliereSPR23.pdf}{MarliereSPR23}~\cite{MarliereSPR23}, \href{../works/KimCMLLP23.pdf}{KimCMLLP23}~\cite{KimCMLLP23}, \href{../works/YunusogluY22.pdf}{YunusogluY22}~\cite{YunusogluY22}, \href{../works/EtminaniesfahaniGNMS22.pdf}{EtminaniesfahaniGNMS22}~\cite{EtminaniesfahaniGNMS22}, \href{../works/HeinzNVH22.pdf}{HeinzNVH22}~\cite{HeinzNVH22}, \href{../works/HamPK21.pdf}{HamPK21}~\cite{HamPK21}, \href{../works/FallahiAC20.pdf}{FallahiAC20}~\cite{FallahiAC20}, \href{../works/CauwelaertDS20.pdf}{CauwelaertDS20}~\cite{CauwelaertDS20}, \href{../works/GurEA19.pdf}{GurEA19}~\cite{GurEA19}, \href{../works/BaptisteB18.pdf}{BaptisteB18}~\cite{BaptisteB18}, \href{../works/KreterSSZ18.pdf}{KreterSSZ18}~\cite{KreterSSZ18}, \href{../works/GomesM17.pdf}{GomesM17}~\cite{GomesM17}, \href{../works/YoungFS17.pdf}{YoungFS17}~\cite{YoungFS17}, \href{../works/AlesioBNG15.pdf}{AlesioBNG15}~\cite{AlesioBNG15}, \href{../works/DejemeppeCS15.pdf}{DejemeppeCS15}~\cite{DejemeppeCS15}, \href{../works/GaySS14.pdf}{GaySS14}~\cite{GaySS14}, \href{../works/ZampelliVSDR13.pdf}{ZampelliVSDR13}~\cite{ZampelliVSDR13}...\href{../works/LauLN08.pdf}{LauLN08}~\cite{LauLN08}, \href{../works/SadykovW06.pdf}{SadykovW06}~\cite{SadykovW06}, \href{../works/Gronkvist06.pdf}{Gronkvist06}~\cite{Gronkvist06}, \href{../works/DemasseyAM05.pdf}{DemasseyAM05}~\cite{DemasseyAM05}, \href{../works/Johnston05.pdf}{Johnston05}~\cite{Johnston05}, \href{../works/ArtiguesBF04.pdf}{ArtiguesBF04}~\cite{ArtiguesBF04}, \href{../works/Demassey03.pdf}{Demassey03}~\cite{Demassey03}, \href{../works/Baptiste02.pdf}{Baptiste02}~\cite{Baptiste02}, \href{../works/EreminW01.pdf}{EreminW01}~\cite{EreminW01}, \href{../works/JainM99.pdf}{JainM99}~\cite{JainM99} (Total: 34)\\
\index{MINLP}\index{Algorithms!MINLP}MINLP &  1.00 & \href{../works/BlomPS16.pdf}{BlomPS16}~\cite{BlomPS16}, \href{../works/BlomBPS14.pdf}{BlomBPS14}~\cite{BlomBPS14}, \href{../works/HarjunkoskiMBC14.pdf}{HarjunkoskiMBC14}~\cite{HarjunkoskiMBC14}, \href{../works/ZhaoL14.pdf}{ZhaoL14}~\cite{ZhaoL14}, \href{../works/HarjunkoskiJG00.pdf}{HarjunkoskiJG00}~\cite{HarjunkoskiJG00} & \href{../works/Adelgren2023.pdf}{Adelgren2023}~\cite{Adelgren2023}, \href{../works/BurtLPS15.pdf}{BurtLPS15}~\cite{BurtLPS15}, \href{../works/PintoG97.pdf}{PintoG97}~\cite{PintoG97} & \href{../works/Edis21.pdf}{Edis21}~\cite{Edis21}, \href{../works/YounespourAKE19.pdf}{YounespourAKE19}~\cite{YounespourAKE19}, \href{../works/EscobetPQPRA19.pdf}{EscobetPQPRA19}~\cite{EscobetPQPRA19}, \href{../works/HookerH17.pdf}{HookerH17}~\cite{HookerH17}, \href{../works/LimBTBB15a.pdf}{LimBTBB15a}~\cite{LimBTBB15a}, \href{../works/LimBTBB15.pdf}{LimBTBB15}~\cite{LimBTBB15}, \href{../works/LopesCSM10.pdf}{LopesCSM10}~\cite{LopesCSM10}, \href{../works/MouraSCL08a.pdf}{MouraSCL08a}~\cite{MouraSCL08a}, \href{../works/Hooker07.pdf}{Hooker07}~\cite{Hooker07}, \href{../works/Hooker06.pdf}{Hooker06}~\cite{Hooker06}, \href{../works/Hooker05a.pdf}{Hooker05a}~\cite{Hooker05a}, \href{../works/Hooker05.pdf}{Hooker05}~\cite{Hooker05}, \href{../works/ChuX05.pdf}{ChuX05}~\cite{ChuX05}, \href{../works/RoePS05.pdf}{RoePS05}~\cite{RoePS05}, \href{../works/Hooker04.pdf}{Hooker04}~\cite{Hooker04}, \href{../works/MaraveliasCG04.pdf}{MaraveliasCG04}~\cite{MaraveliasCG04}, \href{../works/JainG01.pdf}{JainG01}~\cite{JainG01}, \href{../works/HookerOTK00.pdf}{HookerOTK00}~\cite{HookerOTK00}, \href{../works/HookerO99.pdf}{HookerO99}~\cite{HookerO99}\\
\index{MIQP}\index{Algorithms!MIQP}MIQP &  1.00 & \href{../works/WinterMMW22.pdf}{WinterMMW22}~\cite{WinterMMW22} &  & \href{../works/BurtLPS15.pdf}{BurtLPS15}~\cite{BurtLPS15}\\
\index{NEH}\index{Algorithms!NEH}NEH &  1.00 & \href{../works/LiLZDZW24.pdf}{LiLZDZW24}~\cite{LiLZDZW24}, \href{../works/AlfieriGPS23.pdf}{AlfieriGPS23}~\cite{AlfieriGPS23}, \href{../works/ArmstrongGOS22.pdf}{ArmstrongGOS22}~\cite{ArmstrongGOS22}, \href{../works/Astrand21.pdf}{Astrand21}~\cite{Astrand21}, \href{../works/RiahiNS018.pdf}{RiahiNS018}~\cite{RiahiNS018}, \href{../works/WatsonBHW99.pdf}{WatsonBHW99}~\cite{WatsonBHW99} & \href{../works/MengLZB21.pdf}{MengLZB21}~\cite{MengLZB21} & \href{../works/AbreuPNF23.pdf}{AbreuPNF23}~\cite{AbreuPNF23}, \href{../works/IsikYA23.pdf}{IsikYA23}~\cite{IsikYA23}, \href{../works/ZhouGL15.pdf}{ZhouGL15}~\cite{ZhouGL15}\\
\index{ant colony}\index{Algorithms!ant colony}ant colony &  1.00 & \href{../works/abs-2402-00459.pdf}{abs-2402-00459}~\cite{abs-2402-00459}, \href{../works/Groleaz21.pdf}{Groleaz21}~\cite{Groleaz21}, \href{../works/ZarandiASC20.pdf}{ZarandiASC20}~\cite{ZarandiASC20}, \href{../works/GroleazNS20a.pdf}{GroleazNS20a}~\cite{GroleazNS20a}, \href{../works/ThiruvadyWGS14.pdf}{ThiruvadyWGS14}~\cite{ThiruvadyWGS14}, \href{../works/ThiruvadyBME09.pdf}{ThiruvadyBME09}~\cite{ThiruvadyBME09}, \href{../works/MeyerE04.pdf}{MeyerE04}~\cite{MeyerE04} & \href{../works/ZhuSZW23.pdf}{ZhuSZW23}~\cite{ZhuSZW23}, \href{../works/IsikYA23.pdf}{IsikYA23}~\cite{IsikYA23}, \href{../works/NaderiBZ22a.pdf}{NaderiBZ22a}~\cite{NaderiBZ22a}, \href{../works/YuraszeckMPV22.pdf}{YuraszeckMPV22}~\cite{YuraszeckMPV22}, \href{../works/CilKLO22.pdf}{CilKLO22}~\cite{CilKLO22}, \href{../works/FarsiTM22.pdf}{FarsiTM22}~\cite{FarsiTM22}, \href{../works/KoehlerBFFHPSSS21.pdf}{KoehlerBFFHPSSS21}~\cite{KoehlerBFFHPSSS21}, \href{../works/Edis21.pdf}{Edis21}~\cite{Edis21}, \href{../works/ZhangYW21.pdf}{ZhangYW21}~\cite{ZhangYW21}, \href{../works/GroleazNS20.pdf}{GroleazNS20}~\cite{GroleazNS20}, \href{../works/Lunardi20.pdf}{Lunardi20}~\cite{Lunardi20}, \href{../works/MejiaY20.pdf}{MejiaY20}~\cite{MejiaY20}, \href{../works/FrohnerTR19.pdf}{FrohnerTR19}~\cite{FrohnerTR19}, \href{../works/GedikKEK18.pdf}{GedikKEK18}~\cite{GedikKEK18}, \href{../works/GedikKBR17.pdf}{GedikKBR17}~\cite{GedikKBR17}, \href{../works/Dejemeppe16.pdf}{Dejemeppe16}~\cite{Dejemeppe16}, \href{../works/TranAB16.pdf}{TranAB16}~\cite{TranAB16}, \href{../works/Froger16.pdf}{Froger16}~\cite{Froger16}, \href{../works/DejemeppeCS15.pdf}{DejemeppeCS15}~\cite{DejemeppeCS15}, \href{../works/Siala15a.pdf}{Siala15a}~\cite{Siala15a}, \href{../works/MalapertCGJLR12.pdf}{MalapertCGJLR12}~\cite{MalapertCGJLR12}, \href{../works/TopalogluSS12.pdf}{TopalogluSS12}~\cite{TopalogluSS12}, \href{../works/Malapert11.pdf}{Malapert11}~\cite{Malapert11}, \href{../works/LiuW11.pdf}{LiuW11}~\cite{LiuW11} & \href{../works/LuZZYW24.pdf}{LuZZYW24}~\cite{LuZZYW24}, \href{../works/Fatemi-AnarakiTFV23.pdf}{Fatemi-AnarakiTFV23}~\cite{Fatemi-AnarakiTFV23}, \href{../works/LacknerMMWW23.pdf}{LacknerMMWW23}~\cite{LacknerMMWW23}, \href{../works/AbreuPNF23.pdf}{AbreuPNF23}~\cite{AbreuPNF23}, \href{../works/AlfieriGPS23.pdf}{AlfieriGPS23}~\cite{AlfieriGPS23}, \href{../works/AlakaP23.pdf}{AlakaP23}~\cite{AlakaP23}, \href{../works/PenzDN23.pdf}{PenzDN23}~\cite{PenzDN23}, \href{../works/MontemanniD23a.pdf}{MontemanniD23a}~\cite{MontemanniD23a}, \href{../works/MontemanniD23.pdf}{MontemanniD23}~\cite{MontemanniD23}, \href{../works/YuraszeckMC23.pdf}{YuraszeckMC23}~\cite{YuraszeckMC23}, \href{../works/GuoZ23.pdf}{GuoZ23}~\cite{GuoZ23}, \href{../works/AkramNHRSA23.pdf}{AkramNHRSA23}~\cite{AkramNHRSA23}, \href{../works/EtminaniesfahaniGNMS22.pdf}{EtminaniesfahaniGNMS22}~\cite{EtminaniesfahaniGNMS22}, \href{../works/SubulanC22.pdf}{SubulanC22}~\cite{SubulanC22}, \href{../works/abs-2211-14492.pdf}{abs-2211-14492}~\cite{abs-2211-14492}, \href{../works/AbreuN22.pdf}{AbreuN22}~\cite{AbreuN22}, \href{../works/YunusogluY22.pdf}{YunusogluY22}~\cite{YunusogluY22}, \href{../works/QinWSLS21.pdf}{QinWSLS21}~\cite{QinWSLS21}, \href{../works/LacknerMMWW21.pdf}{LacknerMMWW21}~\cite{LacknerMMWW21}...\href{../works/TranB12.pdf}{TranB12}~\cite{TranB12}, \href{../works/GuSW12.pdf}{GuSW12}~\cite{GuSW12}, \href{../works/Schutt11.pdf}{Schutt11}~\cite{Schutt11}, \href{../works/LahimerLH11.pdf}{LahimerLH11}~\cite{LahimerLH11}, \href{../works/Wolf11.pdf}{Wolf11}~\cite{Wolf11}, \href{../works/OddiRCS11.pdf}{OddiRCS11}~\cite{OddiRCS11}, \href{../works/ZeballosQH10.pdf}{ZeballosQH10}~\cite{ZeballosQH10}, \href{../works/KendallKRU10.pdf}{KendallKRU10}~\cite{KendallKRU10}, \href{../works/Zeballos10.pdf}{Zeballos10}~\cite{Zeballos10}, \href{../works/Wallace06.pdf}{Wallace06}~\cite{Wallace06} (Total: 52)\\
\index{bi-partite matching}\index{Algorithms!bi-partite matching}bi-partite matching &  1.00 &  &  & \href{../works/Caballero19.pdf}{Caballero19}~\cite{Caballero19}, \href{../works/HookerH17.pdf}{HookerH17}~\cite{HookerH17}, \href{../works/Simonis07.pdf}{Simonis07}~\cite{Simonis07}, \href{../works/Kumar03.pdf}{Kumar03}~\cite{Kumar03}, \href{../works/Trick03.pdf}{Trick03}~\cite{Trick03}, \href{../works/Simonis99.pdf}{Simonis99}~\cite{Simonis99}\\
\index{column generation}\index{Algorithms!column generation}column generation &  1.00 & \href{../works/FrimodigECM23.pdf}{FrimodigECM23}~\cite{FrimodigECM23}, \href{../works/PohlAK22.pdf}{PohlAK22}~\cite{PohlAK22}, \href{../works/GhandehariK22.pdf}{GhandehariK22}~\cite{GhandehariK22}, \href{../works/BourreauGGLT22.pdf}{BourreauGGLT22}~\cite{BourreauGGLT22}, \href{../works/KletzanderMH21.pdf}{KletzanderMH21}~\cite{KletzanderMH21}, \href{../works/HookerH17.pdf}{HookerH17}~\cite{HookerH17}, \href{../works/CatusseCBL16.pdf}{CatusseCBL16}~\cite{CatusseCBL16}, \href{../works/Froger16.pdf}{Froger16}~\cite{Froger16}, \href{../works/QinDS16.pdf}{QinDS16}~\cite{QinDS16}, \href{../works/DoulabiRP14.pdf}{DoulabiRP14}~\cite{DoulabiRP14}, \href{../works/TopalogluO11.pdf}{TopalogluO11}~\cite{TopalogluO11}, \href{../works/MilanoW09.pdf}{MilanoW09}~\cite{MilanoW09}, \href{../works/AkkerDH07.pdf}{AkkerDH07}~\cite{AkkerDH07}, \href{../works/MilanoW06.pdf}{MilanoW06}~\cite{MilanoW06}, \href{../works/Wallace06.pdf}{Wallace06}~\cite{Wallace06}, \href{../works/Gronkvist06.pdf}{Gronkvist06}~\cite{Gronkvist06}, \href{../works/SadykovW06.pdf}{SadykovW06}~\cite{SadykovW06}, \href{../works/Mason01.pdf}{Mason01}~\cite{Mason01}, \href{../works/EreminW01.pdf}{EreminW01}~\cite{EreminW01}, \href{../works/BruckerK00.pdf}{BruckerK00}~\cite{BruckerK00} & \href{../works/abs-2402-00459.pdf}{abs-2402-00459}~\cite{abs-2402-00459}, \href{../works/Adelgren2023.pdf}{Adelgren2023}~\cite{Adelgren2023}, \href{../works/abs-2211-14492.pdf}{abs-2211-14492}~\cite{abs-2211-14492}, \href{../works/Groleaz21.pdf}{Groleaz21}~\cite{Groleaz21}, \href{../works/FallahiAC20.pdf}{FallahiAC20}~\cite{FallahiAC20}, \href{../works/KletzanderM20.pdf}{KletzanderM20}~\cite{KletzanderM20}, \href{../works/BenediktSMVH18.pdf}{BenediktSMVH18}~\cite{BenediktSMVH18}, \href{../works/RoshanaeiLAU17.pdf}{RoshanaeiLAU17}~\cite{RoshanaeiLAU17}, \href{../works/DoulabiRP16.pdf}{DoulabiRP16}~\cite{DoulabiRP16}, \href{../works/EvenSH15.pdf}{EvenSH15}~\cite{EvenSH15}, \href{../works/EvenSH15a.pdf}{EvenSH15a}~\cite{EvenSH15a}, \href{../works/HeinzSSW12.pdf}{HeinzSSW12}~\cite{HeinzSSW12}, \href{../works/HachemiGR11.pdf}{HachemiGR11}~\cite{HachemiGR11}, \href{../works/KendallKRU10.pdf}{KendallKRU10}~\cite{KendallKRU10}, \href{../works/CorreaLR07.pdf}{CorreaLR07}~\cite{CorreaLR07}, \href{../works/BeckR03.pdf}{BeckR03}~\cite{BeckR03} & \href{../works/FalqueALM24.pdf}{FalqueALM24}~\cite{FalqueALM24}, \href{../works/LuZZYW24.pdf}{LuZZYW24}~\cite{LuZZYW24}, \href{../works/ZhuSZW23.pdf}{ZhuSZW23}~\cite{ZhuSZW23}, \href{../works/GuoZ23.pdf}{GuoZ23}~\cite{GuoZ23}, \href{../works/SquillaciPR23.pdf}{SquillaciPR23}~\cite{SquillaciPR23}, \href{../works/CampeauG22.pdf}{CampeauG22}~\cite{CampeauG22}, \href{../works/PandeyS21a.pdf}{PandeyS21a}~\cite{PandeyS21a}, \href{../works/Zahout21.pdf}{Zahout21}~\cite{Zahout21}, \href{../works/RoshanaeiN21.pdf}{RoshanaeiN21}~\cite{RoshanaeiN21}, \href{../works/ZarandiASC20.pdf}{ZarandiASC20}~\cite{ZarandiASC20}, \href{../works/AntunesABD20.pdf}{AntunesABD20}~\cite{AntunesABD20}, \href{../works/FachiniA20.pdf}{FachiniA20}~\cite{FachiniA20}, \href{../works/RoshanaeiBAUB20.pdf}{RoshanaeiBAUB20}~\cite{RoshanaeiBAUB20}, \href{../works/UnsalO19.pdf}{UnsalO19}~\cite{UnsalO19}, \href{../works/KucukY19.pdf}{KucukY19}~\cite{KucukY19}, \href{../works/Hooker19.pdf}{Hooker19}~\cite{Hooker19}, \href{../works/HoundjiSW19.pdf}{HoundjiSW19}~\cite{HoundjiSW19}, \href{../works/CappartTSR18.pdf}{CappartTSR18}~\cite{CappartTSR18}, \href{../works/AntunesABD18.pdf}{AntunesABD18}~\cite{AntunesABD18}...\href{../works/DemasseyAM05.pdf}{DemasseyAM05}~\cite{DemasseyAM05}, \href{../works/Sadykov04.pdf}{Sadykov04}~\cite{Sadykov04}, \href{../works/Demassey03.pdf}{Demassey03}~\cite{Demassey03}, \href{../works/BourdaisGP03.pdf}{BourdaisGP03}~\cite{BourdaisGP03}, \href{../works/Baptiste02.pdf}{Baptiste02}~\cite{Baptiste02}, \href{../works/EastonNT02.pdf}{EastonNT02}~\cite{EastonNT02}, \href{../works/JainG01.pdf}{JainG01}~\cite{JainG01}, \href{../works/Beck99.pdf}{Beck99}~\cite{Beck99}, \href{../works/BeckDDF98.pdf}{BeckDDF98}~\cite{BeckDDF98}, \href{../works/BeckF98.pdf}{BeckF98}~\cite{BeckF98} (Total: 71)\\
\index{conflict-driven clause learning}\index{Algorithms!conflict-driven clause learning}conflict-driven clause learning &  1.00 & \href{../works/Siala15a.pdf}{Siala15a}~\cite{Siala15a} &  & \href{../works/Lemos21.pdf}{Lemos21}~\cite{Lemos21}, \href{../works/Caballero19.pdf}{Caballero19}~\cite{Caballero19}, \href{../works/SialaAH15.pdf}{SialaAH15}~\cite{SialaAH15}\\
\index{deep learning}\index{Algorithms!deep learning}deep learning &  1.00 & \href{../works/KotaryFH22.pdf}{KotaryFH22}~\cite{KotaryFH22}, \href{../works/MullerMKP22.pdf}{MullerMKP22}~\cite{MullerMKP22} & \href{../works/IklassovMR023.pdf}{IklassovMR023}~\cite{IklassovMR023} & \href{../works/LiLZDZW24.pdf}{LiLZDZW24}~\cite{LiLZDZW24}, \href{../works/EfthymiouY23.pdf}{EfthymiouY23}~\cite{EfthymiouY23}, \href{../works/AkramNHRSA23.pdf}{AkramNHRSA23}~\cite{AkramNHRSA23}, \href{../works/Tassel22.pdf}{Tassel22}~\cite{Tassel22}, \href{../works/abs-2211-14492.pdf}{abs-2211-14492}~\cite{abs-2211-14492}, \href{../works/AntuoriHHEN21.pdf}{AntuoriHHEN21}~\cite{AntuoriHHEN21}, \href{../works/TranWDRFOVB16.pdf}{TranWDRFOVB16}~\cite{TranWDRFOVB16}, \href{../works/TranDRFWOVB16.pdf}{TranDRFWOVB16}~\cite{TranDRFWOVB16}, \href{../works/BeckF98.pdf}{BeckF98}~\cite{BeckF98}\\
\index{edge-finder}\index{Algorithms!edge-finder}edge-finder &  1.00 & \href{../works/KameugneFND23.pdf}{KameugneFND23}~\cite{KameugneFND23}, \href{../works/FetgoD22.pdf}{FetgoD22}~\cite{FetgoD22}, \href{../works/GingrasQ16.pdf}{GingrasQ16}~\cite{GingrasQ16}, \href{../works/KameugneFSN14.pdf}{KameugneFSN14}~\cite{KameugneFSN14}, \href{../works/Lombardi10.pdf}{Lombardi10}~\cite{Lombardi10}, \href{../works/MercierH08.pdf}{MercierH08}~\cite{MercierH08}, \href{../works/MercierH07.pdf}{MercierH07}~\cite{MercierH07}, \href{../works/BaptisteP00.pdf}{BaptisteP00}~\cite{BaptisteP00}, \href{../works/PapeB97.pdf}{PapeB97}~\cite{PapeB97} & \href{../works/OuelletQ13.pdf}{OuelletQ13}~\cite{OuelletQ13}, \href{../works/KelbelH11.pdf}{KelbelH11}~\cite{KelbelH11}, \href{../works/PapaB98.pdf}{PapaB98}~\cite{PapaB98} & \href{../works/BaptisteB18.pdf}{BaptisteB18}~\cite{BaptisteB18}, \href{../works/BonfiettiZLM16.pdf}{BonfiettiZLM16}~\cite{BonfiettiZLM16}, \href{../works/Kameugne14.pdf}{Kameugne14}~\cite{Kameugne14}, \href{../works/GuSS13.pdf}{GuSS13}~\cite{GuSS13}, \href{../works/Schutt11.pdf}{Schutt11}~\cite{Schutt11}, \href{../works/SchuttFSW11.pdf}{SchuttFSW11}~\cite{SchuttFSW11}, \href{../works/HeckmanB11.pdf}{HeckmanB11}~\cite{HeckmanB11}, \href{../works/BidotVLB09.pdf}{BidotVLB09}~\cite{BidotVLB09}, \href{../works/MilanoW09.pdf}{MilanoW09}~\cite{MilanoW09}, \href{../works/SchuttFSW09.pdf}{SchuttFSW09}~\cite{SchuttFSW09}, \href{../works/BeckW07.pdf}{BeckW07}~\cite{BeckW07}, \href{../works/MilanoW06.pdf}{MilanoW06}~\cite{MilanoW06}, \href{../works/BeckW05.pdf}{BeckW05}~\cite{BeckW05}, \href{../works/ValleMGT03.pdf}{ValleMGT03}~\cite{ValleMGT03}, \href{../works/BeckR03.pdf}{BeckR03}~\cite{BeckR03}, \href{../works/Junker00.pdf}{Junker00}~\cite{Junker00}, \href{../works/SakkoutW00.pdf}{SakkoutW00}~\cite{SakkoutW00}, \href{../works/JainM99.pdf}{JainM99}~\cite{JainM99}, \href{../works/Zhou97.pdf}{Zhou97}~\cite{Zhou97}, \href{../works/BaptisteP97.pdf}{BaptisteP97}~\cite{BaptisteP97}\\
\index{edge-finding}\index{Algorithms!edge-finding}edge-finding &  1.00 & \href{../works/KameugneFND23.pdf}{KameugneFND23}~\cite{KameugneFND23}, \href{../works/JuvinHHL23.pdf}{JuvinHHL23}~\cite{JuvinHHL23}, \href{../works/TardivoDFMP23.pdf}{TardivoDFMP23}~\cite{TardivoDFMP23}, \href{../works/OuelletQ22.pdf}{OuelletQ22}~\cite{OuelletQ22}, \href{../works/FetgoD22.pdf}{FetgoD22}~\cite{FetgoD22}, \href{../works/CauwelaertDS20.pdf}{CauwelaertDS20}~\cite{CauwelaertDS20}, \href{../works/Caballero19.pdf}{Caballero19}~\cite{Caballero19}, \href{../works/YangSS19.pdf}{YangSS19}~\cite{YangSS19}, \href{../works/BaptisteB18.pdf}{BaptisteB18}~\cite{BaptisteB18}, \href{../works/GokgurHO18.pdf}{GokgurHO18}~\cite{GokgurHO18}, \href{../works/FahimiOQ18.pdf}{FahimiOQ18}~\cite{FahimiOQ18}, \href{../works/KreterSS17.pdf}{KreterSS17}~\cite{KreterSS17}, \href{../works/HookerH17.pdf}{HookerH17}~\cite{HookerH17}, \href{../works/Nattaf16.pdf}{Nattaf16}~\cite{Nattaf16}, \href{../works/Fahimi16.pdf}{Fahimi16}~\cite{Fahimi16}, \href{../works/Dejemeppe16.pdf}{Dejemeppe16}~\cite{Dejemeppe16}, \href{../works/Derrien15.pdf}{Derrien15}~\cite{Derrien15}, \href{../works/GayHS15a.pdf}{GayHS15a}~\cite{GayHS15a}, \href{../works/Kameugne15.pdf}{Kameugne15}~\cite{Kameugne15}...\href{../works/Beck99.pdf}{Beck99}~\cite{Beck99}, \href{../works/BeckF99.pdf}{BeckF99}~\cite{BeckF99}, \href{../works/BeckDDF98.pdf}{BeckDDF98}~\cite{BeckDDF98}, \href{../works/PapaB98.pdf}{PapaB98}~\cite{PapaB98}, \href{../works/BeckF98.pdf}{BeckF98}~\cite{BeckF98}, \href{../works/PapeB97.pdf}{PapeB97}~\cite{PapeB97}, \href{../works/BaptisteP97.pdf}{BaptisteP97}~\cite{BaptisteP97}, \href{../works/BeckDSF97.pdf}{BeckDSF97}~\cite{BeckDSF97}, \href{../works/BeckDSF97a.pdf}{BeckDSF97a}~\cite{BeckDSF97a}, \href{../works/BeckDF97.pdf}{BeckDF97}~\cite{BeckDF97} (Total: 58) & \href{../works/BoudreaultSLQ22.pdf}{BoudreaultSLQ22}~\cite{BoudreaultSLQ22}, \href{../works/LaborieRSV18.pdf}{LaborieRSV18}~\cite{LaborieRSV18}, \href{../works/Tesch18.pdf}{Tesch18}~\cite{Tesch18}, \href{../works/GingrasQ16.pdf}{GingrasQ16}~\cite{GingrasQ16}, \href{../works/CauwelaertDMS16.pdf}{CauwelaertDMS16}~\cite{CauwelaertDMS16}, \href{../works/Siala15a.pdf}{Siala15a}~\cite{Siala15a}, \href{../works/Siala15.pdf}{Siala15}~\cite{Siala15}, \href{../works/LetortCB15.pdf}{LetortCB15}~\cite{LetortCB15}, \href{../works/DejemeppeCS15.pdf}{DejemeppeCS15}~\cite{DejemeppeCS15}, \href{../works/MenciaSV13.pdf}{MenciaSV13}~\cite{MenciaSV13}, \href{../works/LetortCB13.pdf}{LetortCB13}~\cite{LetortCB13}, \href{../works/LombardiM12.pdf}{LombardiM12}~\cite{LombardiM12}, \href{../works/LetortBC12.pdf}{LetortBC12}~\cite{LetortBC12}, \href{../works/Lombardi10.pdf}{Lombardi10}~\cite{Lombardi10}, \href{../works/BartakSR10.pdf}{BartakSR10}~\cite{BartakSR10}, \href{../works/LiessM08.pdf}{LiessM08}~\cite{LiessM08}, \href{../works/BartakSR08.pdf}{BartakSR08}~\cite{BartakSR08}, \href{../works/HoeveGSL07.pdf}{HoeveGSL07}~\cite{HoeveGSL07}, \href{../works/ArtiguesF07.pdf}{ArtiguesF07}~\cite{ArtiguesF07}, \href{../works/MonetteDD07.pdf}{MonetteDD07}~\cite{MonetteDD07}, \href{../works/Vilim04.pdf}{Vilim04}~\cite{Vilim04}, \href{../works/Kuchcinski03.pdf}{Kuchcinski03}~\cite{Kuchcinski03}, \href{../works/Bartak02.pdf}{Bartak02}~\cite{Bartak02}, \href{../works/SchildW00.pdf}{SchildW00}~\cite{SchildW00}, \href{../works/BaptistePN99.pdf}{BaptistePN99}~\cite{BaptistePN99}, \href{../works/Zhou97.pdf}{Zhou97}~\cite{Zhou97} & \href{../works/BonninMNE24.pdf}{BonninMNE24}~\cite{BonninMNE24}, \href{../works/CilKLO22.pdf}{CilKLO22}~\cite{CilKLO22}, \href{../works/CampeauG22.pdf}{CampeauG22}~\cite{CampeauG22}, \href{../works/Groleaz21.pdf}{Groleaz21}~\cite{Groleaz21}, \href{../works/Godet21a.pdf}{Godet21a}~\cite{Godet21a}, \href{../works/Astrand21.pdf}{Astrand21}~\cite{Astrand21}, \href{../works/WallaceY20.pdf}{WallaceY20}~\cite{WallaceY20}, \href{../works/OuelletQ18.pdf}{OuelletQ18}~\cite{OuelletQ18}, \href{../works/GombolayWS18.pdf}{GombolayWS18}~\cite{GombolayWS18}, \href{../works/CauwelaertLS18.pdf}{CauwelaertLS18}~\cite{CauwelaertLS18}, \href{../works/BofillCSV17a.pdf}{BofillCSV17a}~\cite{BofillCSV17a}, \href{../works/EmeretlisTAV17.pdf}{EmeretlisTAV17}~\cite{EmeretlisTAV17}, \href{../works/NattafAL17.pdf}{NattafAL17}~\cite{NattafAL17}, \href{../works/Tesch16.pdf}{Tesch16}~\cite{Tesch16}, \href{../works/OrnekO16.pdf}{OrnekO16}~\cite{OrnekO16}, \href{../works/GayHLS15.pdf}{GayHLS15}~\cite{GayHLS15}, \href{../works/SialaAH15.pdf}{SialaAH15}~\cite{SialaAH15}, \href{../works/DerrienP14.pdf}{DerrienP14}~\cite{DerrienP14}, \href{../works/GuSS13.pdf}{GuSS13}~\cite{GuSS13}...\href{../works/RodriguezDG02.pdf}{RodriguezDG02}~\cite{RodriguezDG02}, \href{../works/BosiM2001.pdf}{BosiM2001}~\cite{BosiM2001}, \href{../works/CestaOS00.pdf}{CestaOS00}~\cite{CestaOS00}, \href{../works/Dorndorf2000.pdf}{Dorndorf2000}~\cite{Dorndorf2000}, \href{../works/SourdN00.pdf}{SourdN00}~\cite{SourdN00}, \href{../works/SakkoutW00.pdf}{SakkoutW00}~\cite{SakkoutW00}, \href{../works/NuijtenP98.pdf}{NuijtenP98}~\cite{NuijtenP98}, \href{../works/Caseau97.pdf}{Caseau97}~\cite{Caseau97}, \href{../works/BlazewiczDP96.pdf}{BlazewiczDP96}~\cite{BlazewiczDP96}, \href{../works/Zhou96.pdf}{Zhou96}~\cite{Zhou96} (Total: 71)\\
\index{energetic reasoning}\index{Algorithms!energetic reasoning}energetic reasoning &  1.00 & \href{../works/TardivoDFMP23.pdf}{TardivoDFMP23}~\cite{TardivoDFMP23}, \href{../works/OuelletQ22.pdf}{OuelletQ22}~\cite{OuelletQ22}, \href{../works/FetgoD22.pdf}{FetgoD22}~\cite{FetgoD22}, \href{../works/HanenKP21.pdf}{HanenKP21}~\cite{HanenKP21}, \href{../works/CarlierPSJ20.pdf}{CarlierPSJ20}~\cite{CarlierPSJ20}, \href{../works/OuelletQ18.pdf}{OuelletQ18}~\cite{OuelletQ18}, \href{../works/Tesch18.pdf}{Tesch18}~\cite{Tesch18}, \href{../works/CauwelaertLS18.pdf}{CauwelaertLS18}~\cite{CauwelaertLS18}, \href{../works/NattafAL17.pdf}{NattafAL17}~\cite{NattafAL17}, \href{../works/Fahimi16.pdf}{Fahimi16}~\cite{Fahimi16}, \href{../works/NattafALR16.pdf}{NattafALR16}~\cite{NattafALR16}, \href{../works/Tesch16.pdf}{Tesch16}~\cite{Tesch16}, \href{../works/GayHS15a.pdf}{GayHS15a}~\cite{GayHS15a}, \href{../works/NattafAL15.pdf}{NattafAL15}~\cite{NattafAL15}, \href{../works/CauwelaertLS15.pdf}{CauwelaertLS15}~\cite{CauwelaertLS15}, \href{../works/ArtiguesL14.pdf}{ArtiguesL14}~\cite{ArtiguesL14}, \href{../works/DerrienP14.pdf}{DerrienP14}~\cite{DerrienP14}, \href{../works/ArtiguesLH13.pdf}{ArtiguesLH13}~\cite{ArtiguesLH13}, \href{../works/SchuttFS13a.pdf}{SchuttFS13a}~\cite{SchuttFS13a}, \href{../works/LimtanyakulS12.pdf}{LimtanyakulS12}~\cite{LimtanyakulS12}, \href{../works/HeinzS11.pdf}{HeinzS11}~\cite{HeinzS11}, \href{../works/Vilim11.pdf}{Vilim11}~\cite{Vilim11}, \href{../works/Lombardi10.pdf}{Lombardi10}~\cite{Lombardi10}, \href{../works/ClautiauxJCM08.pdf}{ClautiauxJCM08}~\cite{ClautiauxJCM08}, \href{../works/MercierH07.pdf}{MercierH07}~\cite{MercierH07}, \href{../works/Laborie03.pdf}{Laborie03}~\cite{Laborie03}, \href{../works/Baptiste02.pdf}{Baptiste02}~\cite{Baptiste02}, \href{../works/Dorndorf2000.pdf}{Dorndorf2000}~\cite{Dorndorf2000}, \href{../works/BaptisteP95.pdf}{BaptisteP95}~\cite{BaptisteP95} & \href{../works/KameugneFND23.pdf}{KameugneFND23}~\cite{KameugneFND23}, \href{../works/NattafHKAL19.pdf}{NattafHKAL19}~\cite{NattafHKAL19}, \href{../works/KameugneFGOQ18.pdf}{KameugneFGOQ18}~\cite{KameugneFGOQ18}, \href{../works/Nattaf16.pdf}{Nattaf16}~\cite{Nattaf16}, \href{../works/Kameugne14.pdf}{Kameugne14}~\cite{Kameugne14}, \href{../works/Letort13.pdf}{Letort13}~\cite{Letort13}, \href{../works/SchuttFS13.pdf}{SchuttFS13}~\cite{SchuttFS13}, \href{../works/Schutt11.pdf}{Schutt11}~\cite{Schutt11} & \href{../works/IsikYA23.pdf}{IsikYA23}~\cite{IsikYA23}, \href{../works/BoudreaultSLQ22.pdf}{BoudreaultSLQ22}~\cite{BoudreaultSLQ22}, \href{../works/ArmstrongGOS21.pdf}{ArmstrongGOS21}~\cite{ArmstrongGOS21}, \href{../works/Caballero19.pdf}{Caballero19}~\cite{Caballero19}, \href{../works/YangSS19.pdf}{YangSS19}~\cite{YangSS19}, \href{../works/GokgurHO18.pdf}{GokgurHO18}~\cite{GokgurHO18}, \href{../works/Laborie18a.pdf}{Laborie18a}~\cite{Laborie18a}, \href{../works/BofillCSV17.pdf}{BofillCSV17}~\cite{BofillCSV17}, \href{../works/HookerH17.pdf}{HookerH17}~\cite{HookerH17}, \href{../works/GingrasQ16.pdf}{GingrasQ16}~\cite{GingrasQ16}, \href{../works/LetortCB15.pdf}{LetortCB15}~\cite{LetortCB15}, \href{../works/Derrien15.pdf}{Derrien15}~\cite{Derrien15}, \href{../works/KameugneFSN14.pdf}{KameugneFSN14}~\cite{KameugneFSN14}, \href{../works/LetortCB13.pdf}{LetortCB13}~\cite{LetortCB13}, \href{../works/MenciaSV13.pdf}{MenciaSV13}~\cite{MenciaSV13}, \href{../works/OuelletQ13.pdf}{OuelletQ13}~\cite{OuelletQ13}, \href{../works/Clercq12.pdf}{Clercq12}~\cite{Clercq12}, \href{../works/LombardiM12.pdf}{LombardiM12}~\cite{LombardiM12}, \href{../works/MenciaSV12.pdf}{MenciaSV12}~\cite{MenciaSV12}...\href{../works/Vilim09a.pdf}{Vilim09a}~\cite{Vilim09a}, \href{../works/Vilim09.pdf}{Vilim09}~\cite{Vilim09}, \href{../works/Limtanyakul07.pdf}{Limtanyakul07}~\cite{Limtanyakul07}, \href{../works/DemasseyAM05.pdf}{DemasseyAM05}~\cite{DemasseyAM05}, \href{../works/WolfS05.pdf}{WolfS05}~\cite{WolfS05}, \href{../works/BaptisteP00.pdf}{BaptisteP00}~\cite{BaptisteP00}, \href{../works/TorresL00.pdf}{TorresL00}~\cite{TorresL00}, \href{../works/BaptistePN99.pdf}{BaptistePN99}~\cite{BaptistePN99}, \href{../works/PapaB98.pdf}{PapaB98}~\cite{PapaB98}, \href{../works/PapeB97.pdf}{PapeB97}~\cite{PapeB97} (Total: 37)\\
\index{evolutionary computing}\index{Algorithms!evolutionary computing}evolutionary computing &  1.00 &  &  & \href{../works/LuZZYW24.pdf}{LuZZYW24}~\cite{LuZZYW24}, \href{../works/Groleaz21.pdf}{Groleaz21}~\cite{Groleaz21}, \href{../works/Lemos21.pdf}{Lemos21}~\cite{Lemos21}, \href{../works/Siala15a.pdf}{Siala15a}~\cite{Siala15a}, \href{../works/PerronSF04.pdf}{PerronSF04}~\cite{PerronSF04}\\
\index{genetic algorithm}\index{Algorithms!genetic algorithm}genetic algorithm &  1.00 & \href{../works/LuZZYW24.pdf}{LuZZYW24}~\cite{LuZZYW24}, \href{../works/AbreuNP23.pdf}{AbreuNP23}~\cite{AbreuNP23}, \href{../works/ZhuSZW23.pdf}{ZhuSZW23}~\cite{ZhuSZW23}, \href{../works/IsikYA23.pdf}{IsikYA23}~\cite{IsikYA23}, \href{../works/AbreuPNF23.pdf}{AbreuPNF23}~\cite{AbreuPNF23}, \href{../works/NaderiBZ22a.pdf}{NaderiBZ22a}~\cite{NaderiBZ22a}, \href{../works/CilKLO22.pdf}{CilKLO22}~\cite{CilKLO22}, \href{../works/AbreuN22.pdf}{AbreuN22}~\cite{AbreuN22}, \href{../works/YunusogluY22.pdf}{YunusogluY22}~\cite{YunusogluY22}, \href{../works/BourreauGGLT22.pdf}{BourreauGGLT22}~\cite{BourreauGGLT22}, \href{../works/EtminaniesfahaniGNMS22.pdf}{EtminaniesfahaniGNMS22}~\cite{EtminaniesfahaniGNMS22}, \href{../works/HamPK21.pdf}{HamPK21}~\cite{HamPK21}, \href{../works/Groleaz21.pdf}{Groleaz21}~\cite{Groleaz21}, \href{../works/Alaka21.pdf}{Alaka21}~\cite{Alaka21}, \href{../works/ZhangYW21.pdf}{ZhangYW21}~\cite{ZhangYW21}, \href{../works/AbreuAPNM21.pdf}{AbreuAPNM21}~\cite{AbreuAPNM21}, \href{../works/Edis21.pdf}{Edis21}~\cite{Edis21}, \href{../works/Zahout21.pdf}{Zahout21}~\cite{Zahout21}, \href{../works/Astrand21.pdf}{Astrand21}~\cite{Astrand21}...\href{../works/MakMS10.pdf}{MakMS10}~\cite{MakMS10}, \href{../works/KusterJF07.pdf}{KusterJF07}~\cite{KusterJF07}, \href{../works/SureshMOK06.pdf}{SureshMOK06}~\cite{SureshMOK06}, \href{../works/BarbulescuWH04.pdf}{BarbulescuWH04}~\cite{BarbulescuWH04}, \href{../works/GlobusCLP04.pdf}{GlobusCLP04}~\cite{GlobusCLP04}, \href{../works/KamarainenS02.pdf}{KamarainenS02}~\cite{KamarainenS02}, \href{../works/JussienL02.pdf}{JussienL02}~\cite{JussienL02}, \href{../works/JainM99.pdf}{JainM99}~\cite{JainM99}, \href{../works/Beck99.pdf}{Beck99}~\cite{Beck99}, \href{../works/BeckF98.pdf}{BeckF98}~\cite{BeckF98} (Total: 42) & \href{../works/PrataAN23.pdf}{PrataAN23}~\cite{PrataAN23}, \href{../works/abs-2402-00459.pdf}{abs-2402-00459}~\cite{abs-2402-00459}, \href{../works/AkramNHRSA23.pdf}{AkramNHRSA23}~\cite{AkramNHRSA23}, \href{../works/ShaikhK23.pdf}{ShaikhK23}~\cite{ShaikhK23}, \href{../works/GokPTGO23.pdf}{GokPTGO23}~\cite{GokPTGO23}, \href{../works/JuvinHL23a.pdf}{JuvinHL23a}~\cite{JuvinHL23a}, \href{../works/abs-2305-19888.pdf}{abs-2305-19888}~\cite{abs-2305-19888}, \href{../works/AfsarVPG23.pdf}{AfsarVPG23}~\cite{AfsarVPG23}, \href{../works/KimCMLLP23.pdf}{KimCMLLP23}~\cite{KimCMLLP23}, \href{../works/AlakaP23.pdf}{AlakaP23}~\cite{AlakaP23}, \href{../works/LacknerMMWW23.pdf}{LacknerMMWW23}~\cite{LacknerMMWW23}, \href{../works/SubulanC22.pdf}{SubulanC22}~\cite{SubulanC22}, \href{../works/ColT22.pdf}{ColT22}~\cite{ColT22}, \href{../works/HeinzNVH22.pdf}{HeinzNVH22}~\cite{HeinzNVH22}, \href{../works/AwadMDMT22.pdf}{AwadMDMT22}~\cite{AwadMDMT22}, \href{../works/FarsiTM22.pdf}{FarsiTM22}~\cite{FarsiTM22}, \href{../works/JuvinHL22.pdf}{JuvinHL22}~\cite{JuvinHL22}, \href{../works/Bedhief21.pdf}{Bedhief21}~\cite{Bedhief21}, \href{../works/Lemos21.pdf}{Lemos21}~\cite{Lemos21}...\href{../works/HladikCDJ08.pdf}{HladikCDJ08}~\cite{HladikCDJ08}, \href{../works/MouraSCL08a.pdf}{MouraSCL08a}~\cite{MouraSCL08a}, \href{../works/Wallace06.pdf}{Wallace06}~\cite{Wallace06}, \href{../works/KhayatLR06.pdf}{KhayatLR06}~\cite{KhayatLR06}, \href{../works/CambazardHDJT04.pdf}{CambazardHDJT04}~\cite{CambazardHDJT04}, \href{../works/DannaP03.pdf}{DannaP03}~\cite{DannaP03}, \href{../works/BeckR03.pdf}{BeckR03}~\cite{BeckR03}, \href{../works/Kuchcinski03.pdf}{Kuchcinski03}~\cite{Kuchcinski03}, \href{../works/TrentesauxPT01.pdf}{TrentesauxPT01}~\cite{TrentesauxPT01}, \href{../works/VanczaM01.pdf}{VanczaM01}~\cite{VanczaM01} (Total: 61) & \href{../works/ForbesHJST24.pdf}{ForbesHJST24}~\cite{ForbesHJST24}, \href{../works/NaderiRR23.pdf}{NaderiRR23}~\cite{NaderiRR23}, \href{../works/TasselGS23.pdf}{TasselGS23}~\cite{TasselGS23}, \href{../works/Mehdizadeh-Somarin23.pdf}{Mehdizadeh-Somarin23}~\cite{Mehdizadeh-Somarin23}, \href{../works/WangB23.pdf}{WangB23}~\cite{WangB23}, \href{../works/abs-2306-05747.pdf}{abs-2306-05747}~\cite{abs-2306-05747}, \href{../works/PovedaAA23.pdf}{PovedaAA23}~\cite{PovedaAA23}, \href{../works/JuvinHHL23.pdf}{JuvinHHL23}~\cite{JuvinHHL23}, \href{../works/Bit-Monnot23.pdf}{Bit-Monnot23}~\cite{Bit-Monnot23}, \href{../works/AalianPG23.pdf}{AalianPG23}~\cite{AalianPG23}, \href{../works/OrnekOS20.pdf}{OrnekOS20}~\cite{OrnekOS20}, \href{../works/WinterMMW22.pdf}{WinterMMW22}~\cite{WinterMMW22}, \href{../works/LiFJZLL22.pdf}{LiFJZLL22}~\cite{LiFJZLL22}, \href{../works/MengGRZSC22.pdf}{MengGRZSC22}~\cite{MengGRZSC22}, \href{../works/abs-2211-14492.pdf}{abs-2211-14492}~\cite{abs-2211-14492}, \href{../works/YuraszeckMPV22.pdf}{YuraszeckMPV22}~\cite{YuraszeckMPV22}, \href{../works/MullerMKP22.pdf}{MullerMKP22}~\cite{MullerMKP22}, \href{../works/OujanaAYB22.pdf}{OujanaAYB22}~\cite{OujanaAYB22}, \href{../works/Teppan22.pdf}{Teppan22}~\cite{Teppan22}...\href{../works/HarjunkoskiG02.pdf}{HarjunkoskiG02}~\cite{HarjunkoskiG02}, \href{../works/Baptiste02.pdf}{Baptiste02}~\cite{Baptiste02}, \href{../works/VerfaillieL01.pdf}{VerfaillieL01}~\cite{VerfaillieL01}, \href{../works/SourdN00.pdf}{SourdN00}~\cite{SourdN00}, \href{../works/SakkoutW00.pdf}{SakkoutW00}~\cite{SakkoutW00}, \href{../works/WatsonBHW99.pdf}{WatsonBHW99}~\cite{WatsonBHW99}, \href{../works/BeckDDF98.pdf}{BeckDDF98}~\cite{BeckDDF98}, \href{../works/GruianK98.pdf}{GruianK98}~\cite{GruianK98}, \href{../works/Wallace96.pdf}{Wallace96}~\cite{Wallace96}, \href{../works/YoshikawaKNW94.pdf}{YoshikawaKNW94}~\cite{YoshikawaKNW94} (Total: 130)\\
\index{large neighborhood search}\index{Algorithms!large neighborhood search}large neighborhood search &  1.00 & \href{../works/SquillaciPR23.pdf}{SquillaciPR23}~\cite{SquillaciPR23}, \href{../works/PovedaAA23.pdf}{PovedaAA23}~\cite{PovedaAA23}, \href{../works/AbreuN22.pdf}{AbreuN22}~\cite{AbreuN22}, \href{../works/Astrand21.pdf}{Astrand21}~\cite{Astrand21}, \href{../works/Astrand0F21.pdf}{Astrand0F21}~\cite{Astrand0F21}, \href{../works/GeibingerMM21.pdf}{GeibingerMM21}~\cite{GeibingerMM21}, \href{../works/AstrandJZ20.pdf}{AstrandJZ20}~\cite{AstrandJZ20}, \href{../works/Mercier-AubinGQ20.pdf}{Mercier-AubinGQ20}~\cite{Mercier-AubinGQ20}, \href{../works/LaborieRSV18.pdf}{LaborieRSV18}~\cite{LaborieRSV18}, \href{../works/Dejemeppe16.pdf}{Dejemeppe16}~\cite{Dejemeppe16}, \href{../works/Froger16.pdf}{Froger16}~\cite{Froger16}, \href{../works/LimBTBB15.pdf}{LimBTBB15}~\cite{LimBTBB15}, \href{../works/GaySS14.pdf}{GaySS14}~\cite{GaySS14}, \href{../works/PacinoH11.pdf}{PacinoH11}~\cite{PacinoH11}, \href{../works/MonetteDH09.pdf}{MonetteDH09}~\cite{MonetteDH09}, \href{../works/CarchraeB09.pdf}{CarchraeB09}~\cite{CarchraeB09}, \href{../works/HentenryckM08.pdf}{HentenryckM08}~\cite{HentenryckM08}, \href{../works/PerronSF04.pdf}{PerronSF04}~\cite{PerronSF04}, \href{../works/DannaP03.pdf}{DannaP03}~\cite{DannaP03} & \href{../works/LuZZYW24.pdf}{LuZZYW24}~\cite{LuZZYW24}, \href{../works/PerezGSL23.pdf}{PerezGSL23}~\cite{PerezGSL23}, \href{../works/abs-2312-13682.pdf}{abs-2312-13682}~\cite{abs-2312-13682}, \href{../works/AbreuNP23.pdf}{AbreuNP23}~\cite{AbreuNP23}, \href{../works/KimCMLLP23.pdf}{KimCMLLP23}~\cite{KimCMLLP23}, \href{../works/ZhangBB22.pdf}{ZhangBB22}~\cite{ZhangBB22}, \href{../works/ColT22.pdf}{ColT22}~\cite{ColT22}, \href{../works/Lemos21.pdf}{Lemos21}~\cite{Lemos21}, \href{../works/Groleaz21.pdf}{Groleaz21}~\cite{Groleaz21}, \href{../works/GokGSTO20.pdf}{GokGSTO20}~\cite{GokGSTO20}, \href{../works/ThomasKS20.pdf}{ThomasKS20}~\cite{ThomasKS20}, \href{../works/SacramentoSP20.pdf}{SacramentoSP20}~\cite{SacramentoSP20}, \href{../works/FachiniA20.pdf}{FachiniA20}~\cite{FachiniA20}, \href{../works/abs-1911-04766.pdf}{abs-1911-04766}~\cite{abs-1911-04766}, \href{../works/DemirovicS18.pdf}{DemirovicS18}~\cite{DemirovicS18}, \href{../works/CappartTSR18.pdf}{CappartTSR18}~\cite{CappartTSR18}, \href{../works/FontaineMH16.pdf}{FontaineMH16}~\cite{FontaineMH16}, \href{../works/VilimLS15.pdf}{VilimLS15}~\cite{VilimLS15}, \href{../works/GrimesH15.pdf}{GrimesH15}~\cite{GrimesH15}, \href{../works/HarjunkoskiMBC14.pdf}{HarjunkoskiMBC14}~\cite{HarjunkoskiMBC14}, \href{../works/UnsalO13.pdf}{UnsalO13}~\cite{UnsalO13}, \href{../works/LombardiM12.pdf}{LombardiM12}~\cite{LombardiM12}, \href{../works/KelbelH11.pdf}{KelbelH11}~\cite{KelbelH11}, \href{../works/SchausHMCMD11.pdf}{SchausHMCMD11}~\cite{SchausHMCMD11}, \href{../works/GrimesH11.pdf}{GrimesH11}~\cite{GrimesH11}, \href{../works/Menana11.pdf}{Menana11}~\cite{Menana11}, \href{../works/Lombardi10.pdf}{Lombardi10}~\cite{Lombardi10}, \href{../works/GodardLN05.pdf}{GodardLN05}~\cite{GodardLN05} & \href{../works/FalqueALM24.pdf}{FalqueALM24}~\cite{FalqueALM24}, \href{../works/PrataAN23.pdf}{PrataAN23}~\cite{PrataAN23}, \href{../works/abs-2306-05747.pdf}{abs-2306-05747}~\cite{abs-2306-05747}, \href{../works/FrimodigECM23.pdf}{FrimodigECM23}~\cite{FrimodigECM23}, \href{../works/LacknerMMWW23.pdf}{LacknerMMWW23}~\cite{LacknerMMWW23}, \href{../works/NaderiBZR23.pdf}{NaderiBZR23}~\cite{NaderiBZR23}, \href{../works/AalianPG23.pdf}{AalianPG23}~\cite{AalianPG23}, \href{../works/AbreuPNF23.pdf}{AbreuPNF23}~\cite{AbreuPNF23}, \href{../works/NaderiRR23.pdf}{NaderiRR23}~\cite{NaderiRR23}, \href{../works/TasselGS23.pdf}{TasselGS23}~\cite{TasselGS23}, \href{../works/Bit-Monnot23.pdf}{Bit-Monnot23}~\cite{Bit-Monnot23}, \href{../works/GokPTGO23.pdf}{GokPTGO23}~\cite{GokPTGO23}, \href{../works/BoudreaultSLQ22.pdf}{BoudreaultSLQ22}~\cite{BoudreaultSLQ22}, \href{../works/BourreauGGLT22.pdf}{BourreauGGLT22}~\cite{BourreauGGLT22}, \href{../works/WinterMMW22.pdf}{WinterMMW22}~\cite{WinterMMW22}, \href{../works/PohlAK22.pdf}{PohlAK22}~\cite{PohlAK22}, \href{../works/OrnekOS20.pdf}{OrnekOS20}~\cite{OrnekOS20}, \href{../works/EtminaniesfahaniGNMS22.pdf}{EtminaniesfahaniGNMS22}~\cite{EtminaniesfahaniGNMS22}, \href{../works/BulckG22.pdf}{BulckG22}~\cite{BulckG22}...\href{../works/Vilim11.pdf}{Vilim11}~\cite{Vilim11}, \href{../works/OddiRCS11.pdf}{OddiRCS11}~\cite{OddiRCS11}, \href{../works/Laborie09.pdf}{Laborie09}~\cite{Laborie09}, \href{../works/MilanoW09.pdf}{MilanoW09}~\cite{MilanoW09}, \href{../works/SchausD08.pdf}{SchausD08}~\cite{SchausD08}, \href{../works/LiessM08.pdf}{LiessM08}~\cite{LiessM08}, \href{../works/GarganiR07.pdf}{GarganiR07}~\cite{GarganiR07}, \href{../works/ArtiguesF07.pdf}{ArtiguesF07}~\cite{ArtiguesF07}, \href{../works/DavenportKRSH07.pdf}{DavenportKRSH07}~\cite{DavenportKRSH07}, \href{../works/MilanoW06.pdf}{MilanoW06}~\cite{MilanoW06} (Total: 81)\\
\index{lazy clause generation}\index{Algorithms!lazy clause generation}lazy clause generation &  1.00 & \href{../works/Caballero19.pdf}{Caballero19}~\cite{Caballero19}, \href{../works/KreterSSZ18.pdf}{KreterSSZ18}~\cite{KreterSSZ18}, \href{../works/KreterSS17.pdf}{KreterSS17}~\cite{KreterSS17}, \href{../works/Siala15a.pdf}{Siala15a}~\cite{Siala15a}, \href{../works/KreterSS15.pdf}{KreterSS15}~\cite{KreterSS15}, \href{../works/SchuttFS13.pdf}{SchuttFS13}~\cite{SchuttFS13}, \href{../works/SchuttFSW13.pdf}{SchuttFSW13}~\cite{SchuttFSW13}, \href{../works/SchuttFS13a.pdf}{SchuttFS13a}~\cite{SchuttFS13a}, \href{../works/KelarevaTK13.pdf}{KelarevaTK13}~\cite{KelarevaTK13}, \href{../works/Schutt11.pdf}{Schutt11}~\cite{Schutt11}, \href{../works/SchuttFSW11.pdf}{SchuttFSW11}~\cite{SchuttFSW11}, \href{../works/abs-1009-0347.pdf}{abs-1009-0347}~\cite{abs-1009-0347}, \href{../works/SchuttFSW09.pdf}{SchuttFSW09}~\cite{SchuttFSW09}, \href{../works/OhrimenkoSC09.pdf}{OhrimenkoSC09}~\cite{OhrimenkoSC09} & \href{../works/PovedaAA23.pdf}{PovedaAA23}~\cite{PovedaAA23}, \href{../works/Bit-Monnot23.pdf}{Bit-Monnot23}~\cite{Bit-Monnot23}, \href{../works/BoudreaultSLQ22.pdf}{BoudreaultSLQ22}~\cite{BoudreaultSLQ22}, \href{../works/OuelletQ22.pdf}{OuelletQ22}~\cite{OuelletQ22}, \href{../works/GeitzGSSW22.pdf}{GeitzGSSW22}~\cite{GeitzGSSW22}, \href{../works/Godet21a.pdf}{Godet21a}~\cite{Godet21a}, \href{../works/WallaceY20.pdf}{WallaceY20}~\cite{WallaceY20}, \href{../works/FahimiOQ18.pdf}{FahimiOQ18}~\cite{FahimiOQ18}, \href{../works/SchnellH17.pdf}{SchnellH17}~\cite{SchnellH17}, \href{../works/SchuttS16.pdf}{SchuttS16}~\cite{SchuttS16}, \href{../works/SzerediS16.pdf}{SzerediS16}~\cite{SzerediS16}, \href{../works/SchnellH15.pdf}{SchnellH15}~\cite{SchnellH15}, \href{../works/SialaAH15.pdf}{SialaAH15}~\cite{SialaAH15}, \href{../works/BofillEGPSV14.pdf}{BofillEGPSV14}~\cite{BofillEGPSV14}, \href{../works/GuSS13.pdf}{GuSS13}~\cite{GuSS13}, \href{../works/SchuttCSW12.pdf}{SchuttCSW12}~\cite{SchuttCSW12} & \href{../works/FalqueALM24.pdf}{FalqueALM24}~\cite{FalqueALM24}, \href{../works/AbreuPNF23.pdf}{AbreuPNF23}~\cite{AbreuPNF23}, \href{../works/TardivoDFMP23.pdf}{TardivoDFMP23}~\cite{TardivoDFMP23}, \href{../works/FrimodigECM23.pdf}{FrimodigECM23}~\cite{FrimodigECM23}, \href{../works/BofillCGGPSV23.pdf}{BofillCGGPSV23}~\cite{BofillCGGPSV23}, \href{../works/KameugneFND23.pdf}{KameugneFND23}~\cite{KameugneFND23}, \href{../works/WangB23.pdf}{WangB23}~\cite{WangB23}, \href{../works/EtminaniesfahaniGNMS22.pdf}{EtminaniesfahaniGNMS22}~\cite{EtminaniesfahaniGNMS22}, \href{../works/FetgoD22.pdf}{FetgoD22}~\cite{FetgoD22}, \href{../works/GeibingerMM21.pdf}{GeibingerMM21}~\cite{GeibingerMM21}, \href{../works/HillTV21.pdf}{HillTV21}~\cite{HillTV21}, \href{../works/GodetLHS20.pdf}{GodetLHS20}~\cite{GodetLHS20}, \href{../works/Mercier-AubinGQ20.pdf}{Mercier-AubinGQ20}~\cite{Mercier-AubinGQ20}, \href{../works/YangSS19.pdf}{YangSS19}~\cite{YangSS19}, \href{../works/AgussurjaKL18.pdf}{AgussurjaKL18}~\cite{AgussurjaKL18}, \href{../works/BaptisteB18.pdf}{BaptisteB18}~\cite{BaptisteB18}, \href{../works/GoldwaserS18.pdf}{GoldwaserS18}~\cite{GoldwaserS18}, \href{../works/YoungFS17.pdf}{YoungFS17}~\cite{YoungFS17}, \href{../works/BofillCSV17.pdf}{BofillCSV17}~\cite{BofillCSV17}...\href{../works/GoldwaserS17.pdf}{GoldwaserS17}~\cite{GoldwaserS17}, \href{../works/MossigeGSMC17.pdf}{MossigeGSMC17}~\cite{MossigeGSMC17}, \href{../works/AmadiniGM16.pdf}{AmadiniGM16}~\cite{AmadiniGM16}, \href{../works/PesantRR15.pdf}{PesantRR15}~\cite{PesantRR15}, \href{../works/ArtiguesL14.pdf}{ArtiguesL14}~\cite{ArtiguesL14}, \href{../works/GuSW12.pdf}{GuSW12}~\cite{GuSW12}, \href{../works/LombardiM12.pdf}{LombardiM12}~\cite{LombardiM12}, \href{../works/GrimesH11.pdf}{GrimesH11}~\cite{GrimesH11}, \href{../works/SchuttW10.pdf}{SchuttW10}~\cite{SchuttW10}, \href{../works/Lombardi10.pdf}{Lombardi10}~\cite{Lombardi10} (Total: 31)\\
\index{machine learning}\index{Algorithms!machine learning}machine learning &  1.00 & \href{../works/abs-2402-00459.pdf}{abs-2402-00459}~\cite{abs-2402-00459}, \href{../works/EfthymiouY23.pdf}{EfthymiouY23}~\cite{EfthymiouY23}, \href{../works/abs-2211-14492.pdf}{abs-2211-14492}~\cite{abs-2211-14492}, \href{../works/MullerMKP22.pdf}{MullerMKP22}~\cite{MullerMKP22}, \href{../works/Groleaz21.pdf}{Groleaz21}~\cite{Groleaz21}, \href{../works/ZarandiASC20.pdf}{ZarandiASC20}~\cite{ZarandiASC20}, \href{../works/HurleyOS16.pdf}{HurleyOS16}~\cite{HurleyOS16}, \href{../works/EskeyZ90.pdf}{EskeyZ90}~\cite{EskeyZ90} & \href{../works/LiLZDZW24.pdf}{LiLZDZW24}~\cite{LiLZDZW24}, \href{../works/GokPTGO23.pdf}{GokPTGO23}~\cite{GokPTGO23}, \href{../works/IklassovMR023.pdf}{IklassovMR023}~\cite{IklassovMR023}, \href{../works/Tassel22.pdf}{Tassel22}~\cite{Tassel22}, \href{../works/KotaryFH22.pdf}{KotaryFH22}~\cite{KotaryFH22}, \href{../works/Lemos21.pdf}{Lemos21}~\cite{Lemos21}, \href{../works/AntuoriHHEN21.pdf}{AntuoriHHEN21}~\cite{AntuoriHHEN21}, \href{../works/KovacsTKSG21.pdf}{KovacsTKSG21}~\cite{KovacsTKSG21}, \href{../works/GalleguillosKSB19.pdf}{GalleguillosKSB19}~\cite{GalleguillosKSB19}, \href{../works/BukchinR18.pdf}{BukchinR18}~\cite{BukchinR18}, \href{../works/BorghesiBLMB18.pdf}{BorghesiBLMB18}~\cite{BorghesiBLMB18}, \href{../works/CohenHB17.pdf}{CohenHB17}~\cite{CohenHB17}, \href{../works/GrimesIOS14.pdf}{GrimesIOS14}~\cite{GrimesIOS14}, \href{../works/IfrimOS12.pdf}{IfrimOS12}~\cite{IfrimOS12}, \href{../works/CarchraeB09.pdf}{CarchraeB09}~\cite{CarchraeB09}, \href{../works/BlazewiczDP96.pdf}{BlazewiczDP96}~\cite{BlazewiczDP96} & \href{../works/PrataAN23.pdf}{PrataAN23}~\cite{PrataAN23}, \href{../works/Mehdizadeh-Somarin23.pdf}{Mehdizadeh-Somarin23}~\cite{Mehdizadeh-Somarin23}, \href{../works/MontemanniD23.pdf}{MontemanniD23}~\cite{MontemanniD23}, \href{../works/JuvinHL23a.pdf}{JuvinHL23a}~\cite{JuvinHL23a}, \href{../works/AkramNHRSA23.pdf}{AkramNHRSA23}~\cite{AkramNHRSA23}, \href{../works/GuoZ23.pdf}{GuoZ23}~\cite{GuoZ23}, \href{../works/abs-2306-05747.pdf}{abs-2306-05747}~\cite{abs-2306-05747}, \href{../works/MarliereSPR23.pdf}{MarliereSPR23}~\cite{MarliereSPR23}, \href{../works/IsikYA23.pdf}{IsikYA23}~\cite{IsikYA23}, \href{../works/TasselGS23.pdf}{TasselGS23}~\cite{TasselGS23}, \href{../works/GurPAE23.pdf}{GurPAE23}~\cite{GurPAE23}, \href{../works/ShaikhK23.pdf}{ShaikhK23}~\cite{ShaikhK23}, \href{../works/LiFJZLL22.pdf}{LiFJZLL22}~\cite{LiFJZLL22}, \href{../works/ColT22.pdf}{ColT22}~\cite{ColT22}, \href{../works/GeitzGSSW22.pdf}{GeitzGSSW22}~\cite{GeitzGSSW22}, \href{../works/ZhangJZL22.pdf}{ZhangJZL22}~\cite{ZhangJZL22}, \href{../works/PopovicCGNC22.pdf}{PopovicCGNC22}~\cite{PopovicCGNC22}, \href{../works/QinWSLS21.pdf}{QinWSLS21}~\cite{QinWSLS21}, \href{../works/HillTV21.pdf}{HillTV21}~\cite{HillTV21}...\href{../works/Lombardi10.pdf}{Lombardi10}~\cite{Lombardi10}, \href{../works/BartakSR10.pdf}{BartakSR10}~\cite{BartakSR10}, \href{../works/KendallKRU10.pdf}{KendallKRU10}~\cite{KendallKRU10}, \href{../works/Malik08.pdf}{Malik08}~\cite{Malik08}, \href{../works/Beck07.pdf}{Beck07}~\cite{Beck07}, \href{../works/Beck06.pdf}{Beck06}~\cite{Beck06}, \href{../works/SureshMOK06.pdf}{SureshMOK06}~\cite{SureshMOK06}, \href{../works/Beck99.pdf}{Beck99}~\cite{Beck99}, \href{../works/JainM99.pdf}{JainM99}~\cite{JainM99}, \href{../works/BeckF98.pdf}{BeckF98}~\cite{BeckF98} (Total: 53)\\
\index{mat heuristic}\index{Algorithms!mat heuristic}mat heuristic &  1.00 & \href{../works/abs-2402-00459.pdf}{abs-2402-00459}~\cite{abs-2402-00459}, \href{../works/AbreuPNF23.pdf}{AbreuPNF23}~\cite{AbreuPNF23}, \href{../works/MontemanniD23.pdf}{MontemanniD23}~\cite{MontemanniD23}, \href{../works/EtminaniesfahaniGNMS22.pdf}{EtminaniesfahaniGNMS22}~\cite{EtminaniesfahaniGNMS22}, \href{../works/SacramentoSP20.pdf}{SacramentoSP20}~\cite{SacramentoSP20}, \href{../works/FachiniA20.pdf}{FachiniA20}~\cite{FachiniA20}, \href{../works/ArbaouiY18.pdf}{ArbaouiY18}~\cite{ArbaouiY18}, \href{../works/Nattaf16.pdf}{Nattaf16}~\cite{Nattaf16} & \href{../works/KimCMLLP23.pdf}{KimCMLLP23}~\cite{KimCMLLP23}, \href{../works/NaderiBZR23.pdf}{NaderiBZR23}~\cite{NaderiBZR23}, \href{../works/AlfieriGPS23.pdf}{AlfieriGPS23}~\cite{AlfieriGPS23}, \href{../works/YunusogluY22.pdf}{YunusogluY22}~\cite{YunusogluY22}, \href{../works/YuraszeckMPV22.pdf}{YuraszeckMPV22}~\cite{YuraszeckMPV22}, \href{../works/ArmstrongGOS22.pdf}{ArmstrongGOS22}~\cite{ArmstrongGOS22}, \href{../works/AbreuAPNM21.pdf}{AbreuAPNM21}~\cite{AbreuAPNM21}, \href{../works/DemirovicS18.pdf}{DemirovicS18}~\cite{DemirovicS18}, \href{../works/Froger16.pdf}{Froger16}~\cite{Froger16} & \href{../works/PrataAN23.pdf}{PrataAN23}~\cite{PrataAN23}, \href{../works/Fatemi-AnarakiTFV23.pdf}{Fatemi-AnarakiTFV23}~\cite{Fatemi-AnarakiTFV23}, \href{../works/PerezGSL23.pdf}{PerezGSL23}~\cite{PerezGSL23}, \href{../works/YuraszeckMCCR23.pdf}{YuraszeckMCCR23}~\cite{YuraszeckMCCR23}, \href{../works/abs-2312-13682.pdf}{abs-2312-13682}~\cite{abs-2312-13682}, \href{../works/AbreuNP23.pdf}{AbreuNP23}~\cite{AbreuNP23}, \href{../works/MontemanniD23a.pdf}{MontemanniD23a}~\cite{MontemanniD23a}, \href{../works/IsikYA23.pdf}{IsikYA23}~\cite{IsikYA23}, \href{../works/SubulanC22.pdf}{SubulanC22}~\cite{SubulanC22}, \href{../works/AbreuN22.pdf}{AbreuN22}~\cite{AbreuN22}, \href{../works/WinterMMW22.pdf}{WinterMMW22}~\cite{WinterMMW22}, \href{../works/Groleaz21.pdf}{Groleaz21}~\cite{Groleaz21}, \href{../works/PandeyS21a.pdf}{PandeyS21a}~\cite{PandeyS21a}, \href{../works/HubnerGSV21.pdf}{HubnerGSV21}~\cite{HubnerGSV21}, \href{../works/GroleazNS20.pdf}{GroleazNS20}~\cite{GroleazNS20}, \href{../works/Lunardi20.pdf}{Lunardi20}~\cite{Lunardi20}, \href{../works/Polo-MejiaALB20.pdf}{Polo-MejiaALB20}~\cite{Polo-MejiaALB20}, \href{../works/GokGSTO20.pdf}{GokGSTO20}~\cite{GokGSTO20}, \href{../works/Hooker19.pdf}{Hooker19}~\cite{Hooker19}, \href{../works/GokgurHO18.pdf}{GokgurHO18}~\cite{GokgurHO18}, \href{../works/HechingH16.pdf}{HechingH16}~\cite{HechingH16}, \href{../works/CireCH16.pdf}{CireCH16}~\cite{CireCH16}, \href{../works/EvenSH15a.pdf}{EvenSH15a}~\cite{EvenSH15a}, \href{../works/WangMD15.pdf}{WangMD15}~\cite{WangMD15}, \href{../works/EvenSH15.pdf}{EvenSH15}~\cite{EvenSH15}, \href{../works/Elkhyari03.pdf}{Elkhyari03}~\cite{Elkhyari03}\\
\index{max-flow}\index{Algorithms!max-flow}max-flow &  1.00 &  & \href{../works/LopesCSM10.pdf}{LopesCSM10}~\cite{LopesCSM10}, \href{../works/MouraSCL08.pdf}{MouraSCL08}~\cite{MouraSCL08}, \href{../works/Muscettola02.pdf}{Muscettola02}~\cite{Muscettola02} & \href{../works/FanXG21.pdf}{FanXG21}~\cite{FanXG21}, \href{../works/ZarandiASC20.pdf}{ZarandiASC20}~\cite{ZarandiASC20}, \href{../works/HoundjiSW19.pdf}{HoundjiSW19}~\cite{HoundjiSW19}, \href{../works/Froger16.pdf}{Froger16}~\cite{Froger16}, \href{../works/Fahimi16.pdf}{Fahimi16}~\cite{Fahimi16}, \href{../works/Kumar03.pdf}{Kumar03}~\cite{Kumar03}\\
\index{memetic algorithm}\index{Algorithms!memetic algorithm}memetic algorithm &  1.00 & \href{../works/ZarandiASC20.pdf}{ZarandiASC20}~\cite{ZarandiASC20} & \href{../works/AfsarVPG23.pdf}{AfsarVPG23}~\cite{AfsarVPG23}, \href{../works/MengLZB21.pdf}{MengLZB21}~\cite{MengLZB21}, \href{../works/ArmstrongGOS21.pdf}{ArmstrongGOS21}~\cite{ArmstrongGOS21}, \href{../works/AlesioBNG15.pdf}{AlesioBNG15}~\cite{AlesioBNG15}, \href{../works/LahimerLH11.pdf}{LahimerLH11}~\cite{LahimerLH11} & \href{../works/PrataAN23.pdf}{PrataAN23}~\cite{PrataAN23}, \href{../works/LuZZYW24.pdf}{LuZZYW24}~\cite{LuZZYW24}, \href{../works/AlfieriGPS23.pdf}{AlfieriGPS23}~\cite{AlfieriGPS23}, \href{../works/PenzDN23.pdf}{PenzDN23}~\cite{PenzDN23}, \href{../works/IsikYA23.pdf}{IsikYA23}~\cite{IsikYA23}, \href{../works/NaderiBZ23.pdf}{NaderiBZ23}~\cite{NaderiBZ23}, \href{../works/EtminaniesfahaniGNMS22.pdf}{EtminaniesfahaniGNMS22}~\cite{EtminaniesfahaniGNMS22}, \href{../works/ZhangJZL22.pdf}{ZhangJZL22}~\cite{ZhangJZL22}, \href{../works/CilKLO22.pdf}{CilKLO22}~\cite{CilKLO22}, \href{../works/BulckG22.pdf}{BulckG22}~\cite{BulckG22}, \href{../works/MengGRZSC22.pdf}{MengGRZSC22}~\cite{MengGRZSC22}, \href{../works/ColT22.pdf}{ColT22}~\cite{ColT22}, \href{../works/LiFJZLL22.pdf}{LiFJZLL22}~\cite{LiFJZLL22}, \href{../works/NaderiBZ22.pdf}{NaderiBZ22}~\cite{NaderiBZ22}, \href{../works/Groleaz21.pdf}{Groleaz21}~\cite{Groleaz21}, \href{../works/ZhangYW21.pdf}{ZhangYW21}~\cite{ZhangYW21}, \href{../works/QinWSLS21.pdf}{QinWSLS21}~\cite{QinWSLS21}, \href{../works/AbohashimaEG21.pdf}{AbohashimaEG21}~\cite{AbohashimaEG21}, \href{../works/FallahiAC20.pdf}{FallahiAC20}~\cite{FallahiAC20}, \href{../works/Lunardi20.pdf}{Lunardi20}~\cite{Lunardi20}, \href{../works/NattafDYW19.pdf}{NattafDYW19}~\cite{NattafDYW19}, \href{../works/RiahiNS018.pdf}{RiahiNS018}~\cite{RiahiNS018}, \href{../works/ZhangW18.pdf}{ZhangW18}~\cite{ZhangW18}, \href{../works/GrimesH15.pdf}{GrimesH15}~\cite{GrimesH15}, \href{../works/RendlPHPR12.pdf}{RendlPHPR12}~\cite{RendlPHPR12}, \href{../works/MenciaSV12.pdf}{MenciaSV12}~\cite{MenciaSV12}, \href{../works/Ribeiro12.pdf}{Ribeiro12}~\cite{Ribeiro12}, \href{../works/GrimesH11.pdf}{GrimesH11}~\cite{GrimesH11}, \href{../works/KendallKRU10.pdf}{KendallKRU10}~\cite{KendallKRU10}, \href{../works/JainM99.pdf}{JainM99}~\cite{JainM99}\\
\index{meta heuristic}\index{Algorithms!meta heuristic}meta heuristic &  1.00 & \href{../works/PrataAN23.pdf}{PrataAN23}~\cite{PrataAN23}, \href{../works/abs-2402-00459.pdf}{abs-2402-00459}~\cite{abs-2402-00459}, \href{../works/LuZZYW24.pdf}{LuZZYW24}~\cite{LuZZYW24}, \href{../works/AbreuPNF23.pdf}{AbreuPNF23}~\cite{AbreuPNF23}, \href{../works/MontemanniD23a.pdf}{MontemanniD23a}~\cite{MontemanniD23a}, \href{../works/YuraszeckMC23.pdf}{YuraszeckMC23}~\cite{YuraszeckMC23}, \href{../works/GokPTGO23.pdf}{GokPTGO23}~\cite{GokPTGO23}, \href{../works/AbreuNP23.pdf}{AbreuNP23}~\cite{AbreuNP23}, \href{../works/YuraszeckMCCR23.pdf}{YuraszeckMCCR23}~\cite{YuraszeckMCCR23}, \href{../works/AlfieriGPS23.pdf}{AlfieriGPS23}~\cite{AlfieriGPS23}, \href{../works/AfsarVPG23.pdf}{AfsarVPG23}~\cite{AfsarVPG23}, \href{../works/IsikYA23.pdf}{IsikYA23}~\cite{IsikYA23}, \href{../works/NaderiRR23.pdf}{NaderiRR23}~\cite{NaderiRR23}, \href{../works/YuraszeckMPV22.pdf}{YuraszeckMPV22}~\cite{YuraszeckMPV22}, \href{../works/MengGRZSC22.pdf}{MengGRZSC22}~\cite{MengGRZSC22}, \href{../works/AbreuN22.pdf}{AbreuN22}~\cite{AbreuN22}, \href{../works/ArmstrongGOS22.pdf}{ArmstrongGOS22}~\cite{ArmstrongGOS22}, \href{../works/CilKLO22.pdf}{CilKLO22}~\cite{CilKLO22}, \href{../works/EtminaniesfahaniGNMS22.pdf}{EtminaniesfahaniGNMS22}~\cite{EtminaniesfahaniGNMS22}...\href{../works/MalapertCGJLR12.pdf}{MalapertCGJLR12}~\cite{MalapertCGJLR12}, \href{../works/TranB12.pdf}{TranB12}~\cite{TranB12}, \href{../works/Ribeiro12.pdf}{Ribeiro12}~\cite{Ribeiro12}, \href{../works/RendlPHPR12.pdf}{RendlPHPR12}~\cite{RendlPHPR12}, \href{../works/Schutt11.pdf}{Schutt11}~\cite{Schutt11}, \href{../works/KendallKRU10.pdf}{KendallKRU10}~\cite{KendallKRU10}, \href{../works/Beck07.pdf}{Beck07}~\cite{Beck07}, \href{../works/Beck06.pdf}{Beck06}~\cite{Beck06}, \href{../works/MeyerE04.pdf}{MeyerE04}~\cite{MeyerE04}, \href{../works/OddiPCC03.pdf}{OddiPCC03}~\cite{OddiPCC03} (Total: 58) & \href{../works/CzerniachowskaWZ23.pdf}{CzerniachowskaWZ23}~\cite{CzerniachowskaWZ23}, \href{../works/Fatemi-AnarakiTFV23.pdf}{Fatemi-AnarakiTFV23}~\cite{Fatemi-AnarakiTFV23}, \href{../works/MontemanniD23.pdf}{MontemanniD23}~\cite{MontemanniD23}, \href{../works/NaderiBZR23.pdf}{NaderiBZR23}~\cite{NaderiBZR23}, \href{../works/ZhangBB22.pdf}{ZhangBB22}~\cite{ZhangBB22}, \href{../works/BoudreaultSLQ22.pdf}{BoudreaultSLQ22}~\cite{BoudreaultSLQ22}, \href{../works/OujanaAYB22.pdf}{OujanaAYB22}~\cite{OujanaAYB22}, \href{../works/SubulanC22.pdf}{SubulanC22}~\cite{SubulanC22}, \href{../works/AwadMDMT22.pdf}{AwadMDMT22}~\cite{AwadMDMT22}, \href{../works/MullerMKP22.pdf}{MullerMKP22}~\cite{MullerMKP22}, \href{../works/NaderiBZ22a.pdf}{NaderiBZ22a}~\cite{NaderiBZ22a}, \href{../works/OrnekOS20.pdf}{OrnekOS20}~\cite{OrnekOS20}, \href{../works/TouatBT22.pdf}{TouatBT22}~\cite{TouatBT22}, \href{../works/GhandehariK22.pdf}{GhandehariK22}~\cite{GhandehariK22}, \href{../works/abs-2211-14492.pdf}{abs-2211-14492}~\cite{abs-2211-14492}, \href{../works/Bedhief21.pdf}{Bedhief21}~\cite{Bedhief21}, \href{../works/Zahout21.pdf}{Zahout21}~\cite{Zahout21}, \href{../works/KletzanderMH21.pdf}{KletzanderMH21}~\cite{KletzanderMH21}, \href{../works/ArmstrongGOS21.pdf}{ArmstrongGOS21}~\cite{ArmstrongGOS21}...\href{../works/BeckFW11.pdf}{BeckFW11}~\cite{BeckFW11}, \href{../works/OddiRCS11.pdf}{OddiRCS11}~\cite{OddiRCS11}, \href{../works/Malapert11.pdf}{Malapert11}~\cite{Malapert11}, \href{../works/MilanoW09.pdf}{MilanoW09}~\cite{MilanoW09}, \href{../works/ThiruvadyBME09.pdf}{ThiruvadyBME09}~\cite{ThiruvadyBME09}, \href{../works/RasmussenT09.pdf}{RasmussenT09}~\cite{RasmussenT09}, \href{../works/WatsonB08.pdf}{WatsonB08}~\cite{WatsonB08}, \href{../works/MilanoW06.pdf}{MilanoW06}~\cite{MilanoW06}, \href{../works/Wallace06.pdf}{Wallace06}~\cite{Wallace06}, \href{../works/BosiM2001.pdf}{BosiM2001}~\cite{BosiM2001} (Total: 54) & \href{../works/LiLZDZW24.pdf}{LiLZDZW24}~\cite{LiLZDZW24}, \href{../works/abs-2305-19888.pdf}{abs-2305-19888}~\cite{abs-2305-19888}, \href{../works/AlakaP23.pdf}{AlakaP23}~\cite{AlakaP23}, \href{../works/PovedaAA23.pdf}{PovedaAA23}~\cite{PovedaAA23}, \href{../works/SquillaciPR23.pdf}{SquillaciPR23}~\cite{SquillaciPR23}, \href{../works/GurPAE23.pdf}{GurPAE23}~\cite{GurPAE23}, \href{../works/Mehdizadeh-Somarin23.pdf}{Mehdizadeh-Somarin23}~\cite{Mehdizadeh-Somarin23}, \href{../works/KimCMLLP23.pdf}{KimCMLLP23}~\cite{KimCMLLP23}, \href{../works/EfthymiouY23.pdf}{EfthymiouY23}~\cite{EfthymiouY23}, \href{../works/JuvinHL23a.pdf}{JuvinHL23a}~\cite{JuvinHL23a}, \href{../works/PerezGSL23.pdf}{PerezGSL23}~\cite{PerezGSL23}, \href{../works/PenzDN23.pdf}{PenzDN23}~\cite{PenzDN23}, \href{../works/NaderiBZ23.pdf}{NaderiBZ23}~\cite{NaderiBZ23}, \href{../works/abs-2306-05747.pdf}{abs-2306-05747}~\cite{abs-2306-05747}, \href{../works/MarliereSPR23.pdf}{MarliereSPR23}~\cite{MarliereSPR23}, \href{../works/abs-2312-13682.pdf}{abs-2312-13682}~\cite{abs-2312-13682}, \href{../works/ShaikhK23.pdf}{ShaikhK23}~\cite{ShaikhK23}, \href{../works/LacknerMMWW23.pdf}{LacknerMMWW23}~\cite{LacknerMMWW23}, \href{../works/TasselGS23.pdf}{TasselGS23}~\cite{TasselGS23}...\href{../works/HentenryckM04.pdf}{HentenryckM04}~\cite{HentenryckM04}, \href{../works/ArtiguesBF04.pdf}{ArtiguesBF04}~\cite{ArtiguesBF04}, \href{../works/KanetAG04.pdf}{KanetAG04}~\cite{KanetAG04}, \href{../works/Elkhyari03.pdf}{Elkhyari03}~\cite{Elkhyari03}, \href{../works/BeckPS03.pdf}{BeckPS03}~\cite{BeckPS03}, \href{../works/Kuchcinski03.pdf}{Kuchcinski03}~\cite{Kuchcinski03}, \href{../works/Demassey03.pdf}{Demassey03}~\cite{Demassey03}, \href{../works/KamarainenS02.pdf}{KamarainenS02}~\cite{KamarainenS02}, \href{../works/BruckerK00.pdf}{BruckerK00}~\cite{BruckerK00}, \href{../works/Beck99.pdf}{Beck99}~\cite{Beck99} (Total: 153)\\
\index{neural network}\index{Algorithms!neural network}neural network &  1.00 & \href{../works/abs-2306-05747.pdf}{abs-2306-05747}~\cite{abs-2306-05747}, \href{../works/TasselGS23.pdf}{TasselGS23}~\cite{TasselGS23}, \href{../works/abs-2211-14492.pdf}{abs-2211-14492}~\cite{abs-2211-14492}, \href{../works/KotaryFH22.pdf}{KotaryFH22}~\cite{KotaryFH22}, \href{../works/MullerMKP22.pdf}{MullerMKP22}~\cite{MullerMKP22}, \href{../works/ZarandiASC20.pdf}{ZarandiASC20}~\cite{ZarandiASC20}, \href{../works/JainM99.pdf}{JainM99}~\cite{JainM99}, \href{../works/MintonJPL90.pdf}{MintonJPL90}~\cite{MintonJPL90} & \href{../works/LiLZDZW24.pdf}{LiLZDZW24}~\cite{LiLZDZW24}, \href{../works/IklassovMR023.pdf}{IklassovMR023}~\cite{IklassovMR023}, \href{../works/EfthymiouY23.pdf}{EfthymiouY23}~\cite{EfthymiouY23}, \href{../works/AntuoriHHEN20.pdf}{AntuoriHHEN20}~\cite{AntuoriHHEN20}, \href{../works/HookerH17.pdf}{HookerH17}~\cite{HookerH17}, \href{../works/MintonJPL92.pdf}{MintonJPL92}~\cite{MintonJPL92} & \href{../works/abs-2402-00459.pdf}{abs-2402-00459}~\cite{abs-2402-00459}, \href{../works/GurPAE23.pdf}{GurPAE23}~\cite{GurPAE23}, \href{../works/SquillaciPR23.pdf}{SquillaciPR23}~\cite{SquillaciPR23}, \href{../works/IsikYA23.pdf}{IsikYA23}~\cite{IsikYA23}, \href{../works/AfsarVPG23.pdf}{AfsarVPG23}~\cite{AfsarVPG23}, \href{../works/Tassel22.pdf}{Tassel22}~\cite{Tassel22}, \href{../works/Groleaz21.pdf}{Groleaz21}~\cite{Groleaz21}, \href{../works/FanXG21.pdf}{FanXG21}~\cite{FanXG21}, \href{../works/KovacsTKSG21.pdf}{KovacsTKSG21}~\cite{KovacsTKSG21}, \href{../works/AntuoriHHEN21.pdf}{AntuoriHHEN21}~\cite{AntuoriHHEN21}, \href{../works/Astrand21.pdf}{Astrand21}~\cite{Astrand21}, \href{../works/FallahiAC20.pdf}{FallahiAC20}~\cite{FallahiAC20}, \href{../works/Lunardi20.pdf}{Lunardi20}~\cite{Lunardi20}, \href{../works/GalleguillosKSB19.pdf}{GalleguillosKSB19}~\cite{GalleguillosKSB19}, \href{../works/TangLWSK18.pdf}{TangLWSK18}~\cite{TangLWSK18}, \href{../works/KletzanderM17.pdf}{KletzanderM17}~\cite{KletzanderM17}, \href{../works/TranWDRFOVB16.pdf}{TranWDRFOVB16}~\cite{TranWDRFOVB16}, \href{../works/Froger16.pdf}{Froger16}~\cite{Froger16}, \href{../works/OrnekO16.pdf}{OrnekO16}~\cite{OrnekO16}, \href{../works/TranDRFWOVB16.pdf}{TranDRFWOVB16}~\cite{TranDRFWOVB16}, \href{../works/IfrimOS12.pdf}{IfrimOS12}~\cite{IfrimOS12}, \href{../works/LiuW11.pdf}{LiuW11}~\cite{LiuW11}, \href{../works/ChenGPSH10.pdf}{ChenGPSH10}~\cite{ChenGPSH10}, \href{../works/HladikCDJ08.pdf}{HladikCDJ08}~\cite{HladikCDJ08}, \href{../works/JussienL02.pdf}{JussienL02}~\cite{JussienL02}, \href{../works/TrentesauxPT01.pdf}{TrentesauxPT01}~\cite{TrentesauxPT01}, \href{../works/BlazewiczDP96.pdf}{BlazewiczDP96}~\cite{BlazewiczDP96}, \href{../works/Wallace96.pdf}{Wallace96}~\cite{Wallace96}, \href{../works/Nuijten94.pdf}{Nuijten94}~\cite{Nuijten94}\\
\index{not-first}\index{Algorithms!not-first}not-first &  1.00 & \href{../works/KameugneFND23.pdf}{KameugneFND23}~\cite{KameugneFND23}, \href{../works/FahimiOQ18.pdf}{FahimiOQ18}~\cite{FahimiOQ18}, \href{../works/KameugneFGOQ18.pdf}{KameugneFGOQ18}~\cite{KameugneFGOQ18}, \href{../works/Fahimi16.pdf}{Fahimi16}~\cite{Fahimi16}, \href{../works/Dejemeppe16.pdf}{Dejemeppe16}~\cite{Dejemeppe16}, \href{../works/GayHS15a.pdf}{GayHS15a}~\cite{GayHS15a}, \href{../works/Kameugne14.pdf}{Kameugne14}~\cite{Kameugne14}, \href{../works/Clercq12.pdf}{Clercq12}~\cite{Clercq12}, \href{../works/Schutt11.pdf}{Schutt11}~\cite{Schutt11}, \href{../works/Malapert11.pdf}{Malapert11}~\cite{Malapert11}, \href{../works/SchuttFSW11.pdf}{SchuttFSW11}~\cite{SchuttFSW11}, \href{../works/MercierH07.pdf}{MercierH07}~\cite{MercierH07}, \href{../works/VilimBC05.pdf}{VilimBC05}~\cite{VilimBC05}, \href{../works/ArtiouchineB05.pdf}{ArtiouchineB05}~\cite{ArtiouchineB05}, \href{../works/Demassey03.pdf}{Demassey03}~\cite{Demassey03}, \href{../works/Baptiste02.pdf}{Baptiste02}~\cite{Baptiste02}, \href{../works/Beck99.pdf}{Beck99}~\cite{Beck99} & \href{../works/TardivoDFMP23.pdf}{TardivoDFMP23}~\cite{TardivoDFMP23}, \href{../works/FetgoD22.pdf}{FetgoD22}~\cite{FetgoD22}, \href{../works/GokgurHO18.pdf}{GokgurHO18}~\cite{GokgurHO18}, \href{../works/OuelletQ18.pdf}{OuelletQ18}~\cite{OuelletQ18}, \href{../works/HookerH17.pdf}{HookerH17}~\cite{HookerH17}, \href{../works/Kameugne15.pdf}{Kameugne15}~\cite{Kameugne15}, \href{../works/DejemeppeCS15.pdf}{DejemeppeCS15}~\cite{DejemeppeCS15}, \href{../works/KameugneFSN14.pdf}{KameugneFSN14}~\cite{KameugneFSN14}, \href{../works/Letort13.pdf}{Letort13}~\cite{Letort13}, \href{../works/OuelletQ13.pdf}{OuelletQ13}~\cite{OuelletQ13}, \href{../works/SchuttW10.pdf}{SchuttW10}~\cite{SchuttW10}, \href{../works/Lombardi10.pdf}{Lombardi10}~\cite{Lombardi10}, \href{../works/BartakSR10.pdf}{BartakSR10}~\cite{BartakSR10}, \href{../works/MonetteDD07.pdf}{MonetteDD07}~\cite{MonetteDD07}, \href{../works/VilimBC04.pdf}{VilimBC04}~\cite{VilimBC04}, \href{../works/Wolf03.pdf}{Wolf03}~\cite{Wolf03}, \href{../works/BeckF00.pdf}{BeckF00}~\cite{BeckF00}, \href{../works/BeckF00a.pdf}{BeckF00a}~\cite{BeckF00a}, \href{../works/TorresL00.pdf}{TorresL00}~\cite{TorresL00}, \href{../works/BeckF99.pdf}{BeckF99}~\cite{BeckF99} & \href{../works/JuvinHHL23.pdf}{JuvinHHL23}~\cite{JuvinHHL23}, \href{../works/BoudreaultSLQ22.pdf}{BoudreaultSLQ22}~\cite{BoudreaultSLQ22}, \href{../works/OuelletQ22.pdf}{OuelletQ22}~\cite{OuelletQ22}, \href{../works/Astrand21.pdf}{Astrand21}~\cite{Astrand21}, \href{../works/Groleaz21.pdf}{Groleaz21}~\cite{Groleaz21}, \href{../works/CauwelaertDS20.pdf}{CauwelaertDS20}~\cite{CauwelaertDS20}, \href{../works/CauwelaertLS18.pdf}{CauwelaertLS18}~\cite{CauwelaertLS18}, \href{../works/Tesch16.pdf}{Tesch16}~\cite{Tesch16}, \href{../works/CauwelaertDMS16.pdf}{CauwelaertDMS16}~\cite{CauwelaertDMS16}, \href{../works/GrimesH15.pdf}{GrimesH15}~\cite{GrimesH15}, \href{../works/KameugneF13.pdf}{KameugneF13}~\cite{KameugneF13}, \href{../works/ChuGNSW13.pdf}{ChuGNSW13}~\cite{ChuGNSW13}, \href{../works/LimtanyakulS12.pdf}{LimtanyakulS12}~\cite{LimtanyakulS12}, \href{../works/MalapertCGJLR12.pdf}{MalapertCGJLR12}~\cite{MalapertCGJLR12}, \href{../works/KameugneFSN11.pdf}{KameugneFSN11}~\cite{KameugneFSN11}, \href{../works/PacinoH11.pdf}{PacinoH11}~\cite{PacinoH11}, \href{../works/Vilim09.pdf}{Vilim09}~\cite{Vilim09}, \href{../works/Wolf09.pdf}{Wolf09}~\cite{Wolf09}, \href{../works/BartakSR08.pdf}{BartakSR08}~\cite{BartakSR08}, \href{../works/Wolf05.pdf}{Wolf05}~\cite{Wolf05}, \href{../works/Laborie03.pdf}{Laborie03}~\cite{Laborie03}, \href{../works/SourdN00.pdf}{SourdN00}~\cite{SourdN00}\\
\index{not-last}\index{Algorithms!not-last}not-last &  1.00 & \href{../works/KameugneFND23.pdf}{KameugneFND23}~\cite{KameugneFND23}, \href{../works/TardivoDFMP23.pdf}{TardivoDFMP23}~\cite{TardivoDFMP23}, \href{../works/OuelletQ18.pdf}{OuelletQ18}~\cite{OuelletQ18}, \href{../works/KameugneFGOQ18.pdf}{KameugneFGOQ18}~\cite{KameugneFGOQ18}, \href{../works/FahimiOQ18.pdf}{FahimiOQ18}~\cite{FahimiOQ18}, \href{../works/Dejemeppe16.pdf}{Dejemeppe16}~\cite{Dejemeppe16}, \href{../works/Fahimi16.pdf}{Fahimi16}~\cite{Fahimi16}, \href{../works/GayHS15a.pdf}{GayHS15a}~\cite{GayHS15a}, \href{../works/Kameugne14.pdf}{Kameugne14}~\cite{Kameugne14}, \href{../works/Clercq12.pdf}{Clercq12}~\cite{Clercq12}, \href{../works/Malapert11.pdf}{Malapert11}~\cite{Malapert11}, \href{../works/Schutt11.pdf}{Schutt11}~\cite{Schutt11}, \href{../works/SchuttW10.pdf}{SchuttW10}~\cite{SchuttW10}, \href{../works/ArtiouchineB05.pdf}{ArtiouchineB05}~\cite{ArtiouchineB05}, \href{../works/VilimBC05.pdf}{VilimBC05}~\cite{VilimBC05}, \href{../works/SchuttWS05.pdf}{SchuttWS05}~\cite{SchuttWS05}, \href{../works/Vilim05.pdf}{Vilim05}~\cite{Vilim05}, \href{../works/Vilim04.pdf}{Vilim04}~\cite{Vilim04}, \href{../works/Wolf03.pdf}{Wolf03}~\cite{Wolf03}, \href{../works/Demassey03.pdf}{Demassey03}~\cite{Demassey03}, \href{../works/Baptiste02.pdf}{Baptiste02}~\cite{Baptiste02}, \href{../works/Beck99.pdf}{Beck99}~\cite{Beck99} & \href{../works/FetgoD22.pdf}{FetgoD22}~\cite{FetgoD22}, \href{../works/CauwelaertDS20.pdf}{CauwelaertDS20}~\cite{CauwelaertDS20}, \href{../works/GokgurHO18.pdf}{GokgurHO18}~\cite{GokgurHO18}, \href{../works/Tesch18.pdf}{Tesch18}~\cite{Tesch18}, \href{../works/Kameugne15.pdf}{Kameugne15}~\cite{Kameugne15}, \href{../works/DejemeppeCS15.pdf}{DejemeppeCS15}~\cite{DejemeppeCS15}, \href{../works/KameugneFSN14.pdf}{KameugneFSN14}~\cite{KameugneFSN14}, \href{../works/OuelletQ13.pdf}{OuelletQ13}~\cite{OuelletQ13}, \href{../works/Letort13.pdf}{Letort13}~\cite{Letort13}, \href{../works/SchuttFS13a.pdf}{SchuttFS13a}~\cite{SchuttFS13a}, \href{../works/SchuttFSW11.pdf}{SchuttFSW11}~\cite{SchuttFSW11}, \href{../works/Vilim11.pdf}{Vilim11}~\cite{Vilim11}, \href{../works/KameugneFSN11.pdf}{KameugneFSN11}~\cite{KameugneFSN11}, \href{../works/Lombardi10.pdf}{Lombardi10}~\cite{Lombardi10}, \href{../works/BartakSR10.pdf}{BartakSR10}~\cite{BartakSR10}, \href{../works/MercierH07.pdf}{MercierH07}~\cite{MercierH07}, \href{../works/MonetteDD07.pdf}{MonetteDD07}~\cite{MonetteDD07}, \href{../works/Wolf05.pdf}{Wolf05}~\cite{Wolf05}, \href{../works/VilimBC04.pdf}{VilimBC04}~\cite{VilimBC04}, \href{../works/TorresL00.pdf}{TorresL00}~\cite{TorresL00}, \href{../works/BeckF00.pdf}{BeckF00}~\cite{BeckF00}, \href{../works/BeckF00a.pdf}{BeckF00a}~\cite{BeckF00a}, \href{../works/BeckF99.pdf}{BeckF99}~\cite{BeckF99} & \href{../works/JuvinHHL23.pdf}{JuvinHHL23}~\cite{JuvinHHL23}, \href{../works/OuelletQ22.pdf}{OuelletQ22}~\cite{OuelletQ22}, \href{../works/BoudreaultSLQ22.pdf}{BoudreaultSLQ22}~\cite{BoudreaultSLQ22}, \href{../works/GeitzGSSW22.pdf}{GeitzGSSW22}~\cite{GeitzGSSW22}, \href{../works/Astrand21.pdf}{Astrand21}~\cite{Astrand21}, \href{../works/Groleaz21.pdf}{Groleaz21}~\cite{Groleaz21}, \href{../works/GodetLHS20.pdf}{GodetLHS20}~\cite{GodetLHS20}, \href{../works/YangSS19.pdf}{YangSS19}~\cite{YangSS19}, \href{../works/CauwelaertLS18.pdf}{CauwelaertLS18}~\cite{CauwelaertLS18}, \href{../works/HookerH17.pdf}{HookerH17}~\cite{HookerH17}, \href{../works/CauwelaertDMS16.pdf}{CauwelaertDMS16}~\cite{CauwelaertDMS16}, \href{../works/Tesch16.pdf}{Tesch16}~\cite{Tesch16}, \href{../works/GrimesH15.pdf}{GrimesH15}~\cite{GrimesH15}, \href{../works/ChuGNSW13.pdf}{ChuGNSW13}~\cite{ChuGNSW13}, \href{../works/LimtanyakulS12.pdf}{LimtanyakulS12}~\cite{LimtanyakulS12}, \href{../works/MalapertCGJLR12.pdf}{MalapertCGJLR12}~\cite{MalapertCGJLR12}, \href{../works/PacinoH11.pdf}{PacinoH11}~\cite{PacinoH11}, \href{../works/ChenGPSH10.pdf}{ChenGPSH10}~\cite{ChenGPSH10}, \href{../works/GrimesHM09.pdf}{GrimesHM09}~\cite{GrimesHM09}, \href{../works/Wolf09.pdf}{Wolf09}~\cite{Wolf09}, \href{../works/MonetteDH09.pdf}{MonetteDH09}~\cite{MonetteDH09}, \href{../works/Vilim09a.pdf}{Vilim09a}~\cite{Vilim09a}, \href{../works/Vilim09.pdf}{Vilim09}~\cite{Vilim09}, \href{../works/BocewiczBB09.pdf}{BocewiczBB09}~\cite{BocewiczBB09}, \href{../works/BartakSR08.pdf}{BartakSR08}~\cite{BartakSR08}, \href{../works/WolfS05.pdf}{WolfS05}~\cite{WolfS05}, \href{../works/WolfS05a.pdf}{WolfS05a}~\cite{WolfS05a}, \href{../works/Laborie03.pdf}{Laborie03}~\cite{Laborie03}, \href{../works/Vilim03.pdf}{Vilim03}~\cite{Vilim03}\\
\index{particle swarm}\index{Algorithms!particle swarm}particle swarm &  1.00 & \href{../works/MengLZB21.pdf}{MengLZB21}~\cite{MengLZB21}, \href{../works/SacramentoSP20.pdf}{SacramentoSP20}~\cite{SacramentoSP20}, \href{../works/ZarandiASC20.pdf}{ZarandiASC20}~\cite{ZarandiASC20} & \href{../works/LuZZYW24.pdf}{LuZZYW24}~\cite{LuZZYW24}, \href{../works/ZhuSZW23.pdf}{ZhuSZW23}~\cite{ZhuSZW23}, \href{../works/AfsarVPG23.pdf}{AfsarVPG23}~\cite{AfsarVPG23}, \href{../works/IsikYA23.pdf}{IsikYA23}~\cite{IsikYA23}, \href{../works/EtminaniesfahaniGNMS22.pdf}{EtminaniesfahaniGNMS22}~\cite{EtminaniesfahaniGNMS22}, \href{../works/HamPK21.pdf}{HamPK21}~\cite{HamPK21}, \href{../works/Edis21.pdf}{Edis21}~\cite{Edis21}, \href{../works/Lunardi20.pdf}{Lunardi20}~\cite{Lunardi20}, \href{../works/MejiaY20.pdf}{MejiaY20}~\cite{MejiaY20}, \href{../works/MengZRZL20.pdf}{MengZRZL20}~\cite{MengZRZL20}, \href{../works/TanZWGQ19.pdf}{TanZWGQ19}~\cite{TanZWGQ19}, \href{../works/ZhangW18.pdf}{ZhangW18}~\cite{ZhangW18}, \href{../works/Froger16.pdf}{Froger16}~\cite{Froger16}, \href{../works/GrimesH15.pdf}{GrimesH15}~\cite{GrimesH15}, \href{../works/MalapertCGJLR12.pdf}{MalapertCGJLR12}~\cite{MalapertCGJLR12} & \href{../works/BonninMNE24.pdf}{BonninMNE24}~\cite{BonninMNE24}, \href{../works/abs-2402-00459.pdf}{abs-2402-00459}~\cite{abs-2402-00459}, \href{../works/PrataAN23.pdf}{PrataAN23}~\cite{PrataAN23}, \href{../works/Bit-Monnot23.pdf}{Bit-Monnot23}~\cite{Bit-Monnot23}, \href{../works/LacknerMMWW23.pdf}{LacknerMMWW23}~\cite{LacknerMMWW23}, \href{../works/CzerniachowskaWZ23.pdf}{CzerniachowskaWZ23}~\cite{CzerniachowskaWZ23}, \href{../works/AlfieriGPS23.pdf}{AlfieriGPS23}~\cite{AlfieriGPS23}, \href{../works/YunusogluY22.pdf}{YunusogluY22}~\cite{YunusogluY22}, \href{../works/SubulanC22.pdf}{SubulanC22}~\cite{SubulanC22}, \href{../works/OrnekOS20.pdf}{OrnekOS20}~\cite{OrnekOS20}, \href{../works/AbreuN22.pdf}{AbreuN22}~\cite{AbreuN22}, \href{../works/CilKLO22.pdf}{CilKLO22}~\cite{CilKLO22}, \href{../works/ColT22.pdf}{ColT22}~\cite{ColT22}, \href{../works/OujanaAYB22.pdf}{OujanaAYB22}~\cite{OujanaAYB22}, \href{../works/QinWSLS21.pdf}{QinWSLS21}~\cite{QinWSLS21}, \href{../works/KoehlerBFFHPSSS21.pdf}{KoehlerBFFHPSSS21}~\cite{KoehlerBFFHPSSS21}, \href{../works/AbreuAPNM21.pdf}{AbreuAPNM21}~\cite{AbreuAPNM21}, \href{../works/LacknerMMWW21.pdf}{LacknerMMWW21}~\cite{LacknerMMWW21}, \href{../works/ZhangYW21.pdf}{ZhangYW21}~\cite{ZhangYW21}...\href{../works/KreterSSZ18.pdf}{KreterSSZ18}~\cite{KreterSSZ18}, \href{../works/TangLWSK18.pdf}{TangLWSK18}~\cite{TangLWSK18}, \href{../works/HamC16.pdf}{HamC16}~\cite{HamC16}, \href{../works/Dejemeppe16.pdf}{Dejemeppe16}~\cite{Dejemeppe16}, \href{../works/ZhouGL15.pdf}{ZhouGL15}~\cite{ZhouGL15}, \href{../works/WangMD15.pdf}{WangMD15}~\cite{WangMD15}, \href{../works/HarjunkoskiMBC14.pdf}{HarjunkoskiMBC14}~\cite{HarjunkoskiMBC14}, \href{../works/MalapertCGJLR13.pdf}{MalapertCGJLR13}~\cite{MalapertCGJLR13}, \href{../works/Malapert11.pdf}{Malapert11}~\cite{Malapert11}, \href{../works/ChenGPSH10.pdf}{ChenGPSH10}~\cite{ChenGPSH10} (Total: 40)\\
\index{quadratic programming}\index{Algorithms!quadratic programming}quadratic programming &  1.00 &  & \href{../works/WinterMMW22.pdf}{WinterMMW22}~\cite{WinterMMW22}, \href{../works/BurtLPS15.pdf}{BurtLPS15}~\cite{BurtLPS15} & \href{../works/abs-2402-00459.pdf}{abs-2402-00459}~\cite{abs-2402-00459}, \href{../works/MarliereSPR23.pdf}{MarliereSPR23}~\cite{MarliereSPR23}, \href{../works/abs-2211-14492.pdf}{abs-2211-14492}~\cite{abs-2211-14492}, \href{../works/ZhangBB22.pdf}{ZhangBB22}~\cite{ZhangBB22}, \href{../works/PandeyS21a.pdf}{PandeyS21a}~\cite{PandeyS21a}, \href{../works/Hooker19.pdf}{Hooker19}~\cite{Hooker19}, \href{../works/He0GLW18.pdf}{He0GLW18}~\cite{He0GLW18}, \href{../works/Refalo00.pdf}{Refalo00}~\cite{Refalo00}\\
\index{reinforcement learning}\index{Algorithms!reinforcement learning}reinforcement learning &  1.00 & \href{../works/IklassovMR023.pdf}{IklassovMR023}~\cite{IklassovMR023}, \href{../works/abs-2211-14492.pdf}{abs-2211-14492}~\cite{abs-2211-14492}, \href{../works/Tassel22.pdf}{Tassel22}~\cite{Tassel22}, \href{../works/AntuoriHHEN20.pdf}{AntuoriHHEN20}~\cite{AntuoriHHEN20}, \href{../works/BeckFW11.pdf}{BeckFW11}~\cite{BeckFW11} & \href{../works/LiLZDZW24.pdf}{LiLZDZW24}~\cite{LiLZDZW24}, \href{../works/abs-2402-00459.pdf}{abs-2402-00459}~\cite{abs-2402-00459}, \href{../works/abs-2306-05747.pdf}{abs-2306-05747}~\cite{abs-2306-05747}, \href{../works/IsikYA23.pdf}{IsikYA23}~\cite{IsikYA23}, \href{../works/TasselGS23.pdf}{TasselGS23}~\cite{TasselGS23}, \href{../works/AntuoriHHEN21.pdf}{AntuoriHHEN21}~\cite{AntuoriHHEN21} & \href{../works/PrataAN23.pdf}{PrataAN23}~\cite{PrataAN23}, \href{../works/AkramNHRSA23.pdf}{AkramNHRSA23}~\cite{AkramNHRSA23}, \href{../works/Mehdizadeh-Somarin23.pdf}{Mehdizadeh-Somarin23}~\cite{Mehdizadeh-Somarin23}, \href{../works/AfsarVPG23.pdf}{AfsarVPG23}~\cite{AfsarVPG23}, \href{../works/GokPTGO23.pdf}{GokPTGO23}~\cite{GokPTGO23}, \href{../works/EfthymiouY23.pdf}{EfthymiouY23}~\cite{EfthymiouY23}, \href{../works/MullerMKP22.pdf}{MullerMKP22}~\cite{MullerMKP22}, \href{../works/SvancaraB22.pdf}{SvancaraB22}~\cite{SvancaraB22}, \href{../works/Zahout21.pdf}{Zahout21}~\cite{Zahout21}, \href{../works/Lemos21.pdf}{Lemos21}~\cite{Lemos21}, \href{../works/KovacsTKSG21.pdf}{KovacsTKSG21}~\cite{KovacsTKSG21}, \href{../works/Astrand21.pdf}{Astrand21}~\cite{Astrand21}, \href{../works/Edis21.pdf}{Edis21}~\cite{Edis21}, \href{../works/ZarandiASC20.pdf}{ZarandiASC20}~\cite{ZarandiASC20}, \href{../works/Lunardi20.pdf}{Lunardi20}~\cite{Lunardi20}, \href{../works/PachecoPR19.pdf}{PachecoPR19}~\cite{PachecoPR19}, \href{../works/BajestaniB13.pdf}{BajestaniB13}~\cite{BajestaniB13}, \href{../works/CarchraeB09.pdf}{CarchraeB09}~\cite{CarchraeB09}, \href{../works/MeyerE04.pdf}{MeyerE04}~\cite{MeyerE04}, \href{../works/PerronSF04.pdf}{PerronSF04}~\cite{PerronSF04}\\
\index{simulated annealing}\index{Algorithms!simulated annealing}simulated annealing &  1.00 & \href{../works/LuZZYW24.pdf}{LuZZYW24}~\cite{LuZZYW24}, \href{../works/PovedaAA23.pdf}{PovedaAA23}~\cite{PovedaAA23}, \href{../works/IsikYA23.pdf}{IsikYA23}~\cite{IsikYA23}, \href{../works/WinterMMW22.pdf}{WinterMMW22}~\cite{WinterMMW22}, \href{../works/Lemos21.pdf}{Lemos21}~\cite{Lemos21}, \href{../works/SacramentoSP20.pdf}{SacramentoSP20}~\cite{SacramentoSP20}, \href{../works/ZarandiASC20.pdf}{ZarandiASC20}~\cite{ZarandiASC20}, \href{../works/KletzanderM20.pdf}{KletzanderM20}~\cite{KletzanderM20}, \href{../works/NattafDYW19.pdf}{NattafDYW19}~\cite{NattafDYW19}, \href{../works/abs-1911-04766.pdf}{abs-1911-04766}~\cite{abs-1911-04766}, \href{../works/LiuCGM17.pdf}{LiuCGM17}~\cite{LiuCGM17}, \href{../works/KletzanderM17.pdf}{KletzanderM17}~\cite{KletzanderM17}, \href{../works/Froger16.pdf}{Froger16}~\cite{Froger16}, \href{../works/RendlPHPR12.pdf}{RendlPHPR12}~\cite{RendlPHPR12}, \href{../works/Ribeiro12.pdf}{Ribeiro12}~\cite{Ribeiro12}, \href{../works/KendallKRU10.pdf}{KendallKRU10}~\cite{KendallKRU10}, \href{../works/LimRX04.pdf}{LimRX04}~\cite{LimRX04}, \href{../works/GlobusCLP04.pdf}{GlobusCLP04}~\cite{GlobusCLP04}, \href{../works/JainM99.pdf}{JainM99}~\cite{JainM99}, \href{../works/Beck99.pdf}{Beck99}~\cite{Beck99}, \href{../works/BlazewiczDP96.pdf}{BlazewiczDP96}~\cite{BlazewiczDP96}, \href{../works/Nuijten94.pdf}{Nuijten94}~\cite{Nuijten94} & \href{../works/abs-2402-00459.pdf}{abs-2402-00459}~\cite{abs-2402-00459}, \href{../works/Mehdizadeh-Somarin23.pdf}{Mehdizadeh-Somarin23}~\cite{Mehdizadeh-Somarin23}, \href{../works/LacknerMMWW23.pdf}{LacknerMMWW23}~\cite{LacknerMMWW23}, \href{../works/NaqviAIAAA22.pdf}{NaqviAIAAA22}~\cite{NaqviAIAAA22}, \href{../works/CilKLO22.pdf}{CilKLO22}~\cite{CilKLO22}, \href{../works/GeitzGSSW22.pdf}{GeitzGSSW22}~\cite{GeitzGSSW22}, \href{../works/ColT22.pdf}{ColT22}~\cite{ColT22}, \href{../works/Edis21.pdf}{Edis21}~\cite{Edis21}, \href{../works/HubnerGSV21.pdf}{HubnerGSV21}~\cite{HubnerGSV21}, \href{../works/Astrand21.pdf}{Astrand21}~\cite{Astrand21}, \href{../works/MejiaY20.pdf}{MejiaY20}~\cite{MejiaY20}, \href{../works/Lunardi20.pdf}{Lunardi20}~\cite{Lunardi20}, \href{../works/AbidinK20.pdf}{AbidinK20}~\cite{AbidinK20}, \href{../works/GombolayWS18.pdf}{GombolayWS18}~\cite{GombolayWS18}, \href{../works/GedikKEK18.pdf}{GedikKEK18}~\cite{GedikKEK18}, \href{../works/GedikKBR17.pdf}{GedikKBR17}~\cite{GedikKBR17}, \href{../works/BeckFW11.pdf}{BeckFW11}~\cite{BeckFW11}, \href{../works/RasmussenT09.pdf}{RasmussenT09}~\cite{RasmussenT09}, \href{../works/HladikCDJ08.pdf}{HladikCDJ08}~\cite{HladikCDJ08}, \href{../works/WatsonB08.pdf}{WatsonB08}~\cite{WatsonB08}, \href{../works/BeniniLMR08.pdf}{BeniniLMR08}~\cite{BeniniLMR08}, \href{../works/BeniniBGM05.pdf}{BeniniBGM05}~\cite{BeniniBGM05}, \href{../works/BeckF98.pdf}{BeckF98}~\cite{BeckF98}, \href{../works/NuijtenP98.pdf}{NuijtenP98}~\cite{NuijtenP98}, \href{../works/BeckDDF98.pdf}{BeckDDF98}~\cite{BeckDDF98}, \href{../works/Wallace96.pdf}{Wallace96}~\cite{Wallace96}, \href{../works/AggounB93.pdf}{AggounB93}~\cite{AggounB93}, \href{../works/MintonJPL90.pdf}{MintonJPL90}~\cite{MintonJPL90} & \href{../works/JuvinHL23a.pdf}{JuvinHL23a}~\cite{JuvinHL23a}, \href{../works/AbreuNP23.pdf}{AbreuNP23}~\cite{AbreuNP23}, \href{../works/AbreuPNF23.pdf}{AbreuPNF23}~\cite{AbreuPNF23}, \href{../works/PenzDN23.pdf}{PenzDN23}~\cite{PenzDN23}, \href{../works/SquillaciPR23.pdf}{SquillaciPR23}~\cite{SquillaciPR23}, \href{../works/abs-2306-05747.pdf}{abs-2306-05747}~\cite{abs-2306-05747}, \href{../works/TasselGS23.pdf}{TasselGS23}~\cite{TasselGS23}, \href{../works/AkramNHRSA23.pdf}{AkramNHRSA23}~\cite{AkramNHRSA23}, \href{../works/PohlAK22.pdf}{PohlAK22}~\cite{PohlAK22}, \href{../works/NaderiBZ22a.pdf}{NaderiBZ22a}~\cite{NaderiBZ22a}, \href{../works/AbreuN22.pdf}{AbreuN22}~\cite{AbreuN22}, \href{../works/AwadMDMT22.pdf}{AwadMDMT22}~\cite{AwadMDMT22}, \href{../works/YunusogluY22.pdf}{YunusogluY22}~\cite{YunusogluY22}, \href{../works/JuvinHL22.pdf}{JuvinHL22}~\cite{JuvinHL22}, \href{../works/YuraszeckMPV22.pdf}{YuraszeckMPV22}~\cite{YuraszeckMPV22}, \href{../works/OrnekOS20.pdf}{OrnekOS20}~\cite{OrnekOS20}, \href{../works/FanXG21.pdf}{FanXG21}~\cite{FanXG21}, \href{../works/KletzanderMH21.pdf}{KletzanderMH21}~\cite{KletzanderMH21}, \href{../works/HamPK21.pdf}{HamPK21}~\cite{HamPK21}...\href{../works/HarjunkoskiG02.pdf}{HarjunkoskiG02}~\cite{HarjunkoskiG02}, \href{../works/Baptiste02.pdf}{Baptiste02}~\cite{Baptiste02}, \href{../works/KamarainenS02.pdf}{KamarainenS02}~\cite{KamarainenS02}, \href{../works/FukunagaHFAMN02.pdf}{FukunagaHFAMN02}~\cite{FukunagaHFAMN02}, \href{../works/TrentesauxPT01.pdf}{TrentesauxPT01}~\cite{TrentesauxPT01}, \href{../works/VerfaillieL01.pdf}{VerfaillieL01}~\cite{VerfaillieL01}, \href{../works/ArtiguesR00.pdf}{ArtiguesR00}~\cite{ArtiguesR00}, \href{../works/SakkoutW00.pdf}{SakkoutW00}~\cite{SakkoutW00}, \href{../works/Colombani96.pdf}{Colombani96}~\cite{Colombani96}, \href{../works/YoshikawaKNW94.pdf}{YoshikawaKNW94}~\cite{YoshikawaKNW94} (Total: 89)\\
\index{support vector regression}\index{Algorithms!support vector regression}support vector regression &  1.00 &  &  & \href{../works/CohenHB17.pdf}{CohenHB17}~\cite{CohenHB17}\\
\index{swarm intelligence}\index{Algorithms!swarm intelligence}swarm intelligence &  1.00 &  & \href{../works/ZarandiASC20.pdf}{ZarandiASC20}~\cite{ZarandiASC20}, \href{../works/Lunardi20.pdf}{Lunardi20}~\cite{Lunardi20} & \href{../works/MontemanniD23.pdf}{MontemanniD23}~\cite{MontemanniD23}, \href{../works/Groleaz21.pdf}{Groleaz21}~\cite{Groleaz21}, \href{../works/HamPK21.pdf}{HamPK21}~\cite{HamPK21}, \href{../works/GroleazNS20a.pdf}{GroleazNS20a}~\cite{GroleazNS20a}, \href{../works/Novas19.pdf}{Novas19}~\cite{Novas19}, \href{../works/Siala15a.pdf}{Siala15a}~\cite{Siala15a}\\
\index{sweep}\index{Algorithms!sweep}sweep &  1.00 & \href{../works/Tesch18.pdf}{Tesch18}~\cite{Tesch18}, \href{../works/NattafALR16.pdf}{NattafALR16}~\cite{NattafALR16}, \href{../works/Tesch16.pdf}{Tesch16}~\cite{Tesch16}, \href{../works/BonfiettiZLM16.pdf}{BonfiettiZLM16}~\cite{BonfiettiZLM16}, \href{../works/Derrien15.pdf}{Derrien15}~\cite{Derrien15}, \href{../works/SimoninAHL15.pdf}{SimoninAHL15}~\cite{SimoninAHL15}, \href{../works/NattafAL15.pdf}{NattafAL15}~\cite{NattafAL15}, \href{../works/GayHS15.pdf}{GayHS15}~\cite{GayHS15}, \href{../works/LetortCB15.pdf}{LetortCB15}~\cite{LetortCB15}, \href{../works/DerrienPZ14.pdf}{DerrienPZ14}~\cite{DerrienPZ14}, \href{../works/Letort13.pdf}{Letort13}~\cite{Letort13}, \href{../works/LetortCB13.pdf}{LetortCB13}~\cite{LetortCB13}, \href{../works/Clercq12.pdf}{Clercq12}~\cite{Clercq12}, \href{../works/SimoninAHL12.pdf}{SimoninAHL12}~\cite{SimoninAHL12}, \href{../works/LetortBC12.pdf}{LetortBC12}~\cite{LetortBC12}, \href{../works/ClercqPBJ11.pdf}{ClercqPBJ11}~\cite{ClercqPBJ11}, \href{../works/Malapert11.pdf}{Malapert11}~\cite{Malapert11}, \href{../works/abs-0907-0939.pdf}{abs-0907-0939}~\cite{abs-0907-0939}, \href{../works/ClautiauxJCM08.pdf}{ClautiauxJCM08}~\cite{ClautiauxJCM08}, \href{../works/BeldiceanuP07.pdf}{BeldiceanuP07}~\cite{BeldiceanuP07}, \href{../works/Wolf05.pdf}{Wolf05}~\cite{Wolf05}, \href{../works/WolfS05a.pdf}{WolfS05a}~\cite{WolfS05a}, \href{../works/Wolf03.pdf}{Wolf03}~\cite{Wolf03}, \href{../works/BeldiceanuC02.pdf}{BeldiceanuC02}~\cite{BeldiceanuC02}, \href{../works/BeldiceanuC01.pdf}{BeldiceanuC01}~\cite{BeldiceanuC01} & \href{../works/ArkhipovBL19.pdf}{ArkhipovBL19}~\cite{ArkhipovBL19}, \href{../works/GoldwaserS18.pdf}{GoldwaserS18}~\cite{GoldwaserS18}, \href{../works/FahimiOQ18.pdf}{FahimiOQ18}~\cite{FahimiOQ18}, \href{../works/GayHS15a.pdf}{GayHS15a}~\cite{GayHS15a}, \href{../works/Schutt11.pdf}{Schutt11}~\cite{Schutt11}, \href{../works/AronssonBK09.pdf}{AronssonBK09}~\cite{AronssonBK09}, \href{../works/PoderB08.pdf}{PoderB08}~\cite{PoderB08}, \href{../works/WolfS05.pdf}{WolfS05}~\cite{WolfS05} & \href{../works/BonninMNE24.pdf}{BonninMNE24}~\cite{BonninMNE24}, \href{../works/KameugneFND23.pdf}{KameugneFND23}~\cite{KameugneFND23}, \href{../works/TardivoDFMP23.pdf}{TardivoDFMP23}~\cite{TardivoDFMP23}, \href{../works/OuelletQ22.pdf}{OuelletQ22}~\cite{OuelletQ22}, \href{../works/HebrardALLCMR22.pdf}{HebrardALLCMR22}~\cite{HebrardALLCMR22}, \href{../works/GeitzGSSW22.pdf}{GeitzGSSW22}~\cite{GeitzGSSW22}, \href{../works/FetgoD22.pdf}{FetgoD22}~\cite{FetgoD22}, \href{../works/Godet21a.pdf}{Godet21a}~\cite{Godet21a}, \href{../works/FachiniA20.pdf}{FachiniA20}~\cite{FachiniA20}, \href{../works/FallahiAC20.pdf}{FallahiAC20}~\cite{FallahiAC20}, \href{../works/HoundjiSW19.pdf}{HoundjiSW19}~\cite{HoundjiSW19}, \href{../works/CauwelaertLS18.pdf}{CauwelaertLS18}~\cite{CauwelaertLS18}, \href{../works/KameugneFGOQ18.pdf}{KameugneFGOQ18}~\cite{KameugneFGOQ18}, \href{../works/Madi-WambaLOBM17.pdf}{Madi-WambaLOBM17}~\cite{Madi-WambaLOBM17}, \href{../works/Fahimi16.pdf}{Fahimi16}~\cite{Fahimi16}, \href{../works/GingrasQ16.pdf}{GingrasQ16}~\cite{GingrasQ16}, \href{../works/Dejemeppe16.pdf}{Dejemeppe16}~\cite{Dejemeppe16}, \href{../works/Nattaf16.pdf}{Nattaf16}~\cite{Nattaf16}, \href{../works/BartakV15.pdf}{BartakV15}~\cite{BartakV15}...\href{../works/BartakSR10.pdf}{BartakSR10}~\cite{BartakSR10}, \href{../works/Lombardi10.pdf}{Lombardi10}~\cite{Lombardi10}, \href{../works/LombardiM10a.pdf}{LombardiM10a}~\cite{LombardiM10a}, \href{../works/Wolf09.pdf}{Wolf09}~\cite{Wolf09}, \href{../works/CarchraeB09.pdf}{CarchraeB09}~\cite{CarchraeB09}, \href{../works/KovacsB08.pdf}{KovacsB08}~\cite{KovacsB08}, \href{../works/BeldiceanuCP08.pdf}{BeldiceanuCP08}~\cite{BeldiceanuCP08}, \href{../works/Simonis07.pdf}{Simonis07}~\cite{Simonis07}, \href{../works/VilimBC05.pdf}{VilimBC05}~\cite{VilimBC05}, \href{../works/Vilim04.pdf}{Vilim04}~\cite{Vilim04} (Total: 42)\\
\index{systematic local search}\index{Algorithms!systematic local search}systematic local search &  1.00 &  &  & \href{../works/Beck07.pdf}{Beck07}~\cite{Beck07}, \href{../works/DilkinaDH05.pdf}{DilkinaDH05}~\cite{DilkinaDH05}, \href{../works/DilkinaH04.pdf}{DilkinaH04}~\cite{DilkinaH04}\\
\index{time-tabling}\index{Algorithms!time-tabling}time-tabling &  1.00 & \href{../works/ShaikhK23.pdf}{ShaikhK23}~\cite{ShaikhK23}, \href{../works/TardivoDFMP23.pdf}{TardivoDFMP23}~\cite{TardivoDFMP23}, \href{../works/OrnekOS20.pdf}{OrnekOS20}~\cite{OrnekOS20}, \href{../works/OuelletQ22.pdf}{OuelletQ22}~\cite{OuelletQ22}, \href{../works/BulckG22.pdf}{BulckG22}~\cite{BulckG22}, \href{../works/Lemos21.pdf}{Lemos21}~\cite{Lemos21}, \href{../works/DemirovicS18.pdf}{DemirovicS18}~\cite{DemirovicS18}, \href{../works/FahimiOQ18.pdf}{FahimiOQ18}~\cite{FahimiOQ18}, \href{../works/Fahimi16.pdf}{Fahimi16}~\cite{Fahimi16}, \href{../works/GayHS15a.pdf}{GayHS15a}~\cite{GayHS15a}, \href{../works/Kameugne14.pdf}{Kameugne14}~\cite{Kameugne14}, \href{../works/OuelletQ13.pdf}{OuelletQ13}~\cite{OuelletQ13}, \href{../works/Letort13.pdf}{Letort13}~\cite{Letort13}, \href{../works/GuyonLPR12.pdf}{GuyonLPR12}~\cite{GuyonLPR12}, \href{../works/Ribeiro12.pdf}{Ribeiro12}~\cite{Ribeiro12}, \href{../works/HeinzS11.pdf}{HeinzS11}~\cite{HeinzS11}, \href{../works/Menana11.pdf}{Menana11}~\cite{Menana11}, \href{../works/KendallKRU10.pdf}{KendallKRU10}~\cite{KendallKRU10}, \href{../works/RussellU06.pdf}{RussellU06}~\cite{RussellU06}, \href{../works/KanetAG04.pdf}{KanetAG04}~\cite{KanetAG04}, \href{../works/Laborie03.pdf}{Laborie03}~\cite{Laborie03}, \href{../works/LimAHO02a.pdf}{LimAHO02a}~\cite{LimAHO02a}, \href{../works/ElkhyariGJ02a.pdf}{ElkhyariGJ02a}~\cite{ElkhyariGJ02a}, \href{../works/Schaerf97.pdf}{Schaerf97}~\cite{Schaerf97}, \href{../works/Wallace96.pdf}{Wallace96}~\cite{Wallace96}, \href{../works/Nuijten94.pdf}{Nuijten94}~\cite{Nuijten94}, \href{../works/YoshikawaKNW94.pdf}{YoshikawaKNW94}~\cite{YoshikawaKNW94} & \href{../works/Astrand21.pdf}{Astrand21}~\cite{Astrand21}, \href{../works/Godet21a.pdf}{Godet21a}~\cite{Godet21a}, \href{../works/WallaceY20.pdf}{WallaceY20}~\cite{WallaceY20}, \href{../works/ZarandiASC20.pdf}{ZarandiASC20}~\cite{ZarandiASC20}, \href{../works/abs-1902-01193.pdf}{abs-1902-01193}~\cite{abs-1902-01193}, \href{../works/CauwelaertLS18.pdf}{CauwelaertLS18}~\cite{CauwelaertLS18}, \href{../works/OuelletQ18.pdf}{OuelletQ18}~\cite{OuelletQ18}, \href{../works/Tesch18.pdf}{Tesch18}~\cite{Tesch18}, \href{../works/HookerH17.pdf}{HookerH17}~\cite{HookerH17}, \href{../works/Derrien15.pdf}{Derrien15}~\cite{Derrien15}, \href{../works/GayHS15.pdf}{GayHS15}~\cite{GayHS15}, \href{../works/BofillGSV15.pdf}{BofillGSV15}~\cite{BofillGSV15}, \href{../works/LimBTBB15a.pdf}{LimBTBB15a}~\cite{LimBTBB15a}, \href{../works/Siala15a.pdf}{Siala15a}~\cite{Siala15a}, \href{../works/Siala15.pdf}{Siala15}~\cite{Siala15}, \href{../works/Vilim11.pdf}{Vilim11}~\cite{Vilim11}, \href{../works/RasmussenT07.pdf}{RasmussenT07}~\cite{RasmussenT07}, \href{../works/RasmussenT06.pdf}{RasmussenT06}~\cite{RasmussenT06}, \href{../works/RoePS05.pdf}{RoePS05}~\cite{RoePS05}, \href{../works/HenzMT04.pdf}{HenzMT04}~\cite{HenzMT04}, \href{../works/MeyerE04.pdf}{MeyerE04}~\cite{MeyerE04}, \href{../works/Elkhyari03.pdf}{Elkhyari03}~\cite{Elkhyari03}, \href{../works/Demassey03.pdf}{Demassey03}~\cite{Demassey03}, \href{../works/EastonNT02.pdf}{EastonNT02}~\cite{EastonNT02}, \href{../works/Bartak02.pdf}{Bartak02}~\cite{Bartak02}, \href{../works/PesantGPR99.pdf}{PesantGPR99}~\cite{PesantGPR99} & \href{../works/BonninMNE24.pdf}{BonninMNE24}~\cite{BonninMNE24}, \href{../works/PrataAN23.pdf}{PrataAN23}~\cite{PrataAN23}, \href{../works/MarliereSPR23.pdf}{MarliereSPR23}~\cite{MarliereSPR23}, \href{../works/Fatemi-AnarakiTFV23.pdf}{Fatemi-AnarakiTFV23}~\cite{Fatemi-AnarakiTFV23}, \href{../works/LacknerMMWW23.pdf}{LacknerMMWW23}~\cite{LacknerMMWW23}, \href{../works/KameugneFND23.pdf}{KameugneFND23}~\cite{KameugneFND23}, \href{../works/AbreuNP23.pdf}{AbreuNP23}~\cite{AbreuNP23}, \href{../works/AlakaP23.pdf}{AlakaP23}~\cite{AlakaP23}, \href{../works/TouatBT22.pdf}{TouatBT22}~\cite{TouatBT22}, \href{../works/FarsiTM22.pdf}{FarsiTM22}~\cite{FarsiTM22}, \href{../works/SvancaraB22.pdf}{SvancaraB22}~\cite{SvancaraB22}, \href{../works/FetgoD22.pdf}{FetgoD22}~\cite{FetgoD22}, \href{../works/NaqviAIAAA22.pdf}{NaqviAIAAA22}~\cite{NaqviAIAAA22}, \href{../works/KletzanderMH21.pdf}{KletzanderMH21}~\cite{KletzanderMH21}, \href{../works/GeibingerMM21.pdf}{GeibingerMM21}~\cite{GeibingerMM21}, \href{../works/AbidinK20.pdf}{AbidinK20}~\cite{AbidinK20}, \href{../works/MokhtarzadehTNF20.pdf}{MokhtarzadehTNF20}~\cite{MokhtarzadehTNF20}, \href{../works/KletzanderM20.pdf}{KletzanderM20}~\cite{KletzanderM20}, \href{../works/GodetLHS20.pdf}{GodetLHS20}~\cite{GodetLHS20}...\href{../works/WolfS05a.pdf}{WolfS05a}~\cite{WolfS05a}, \href{../works/BeckW04.pdf}{BeckW04}~\cite{BeckW04}, \href{../works/Tsang03.pdf}{Tsang03}~\cite{Tsang03}, \href{../works/ElfJR03.pdf}{ElfJR03}~\cite{ElfJR03}, \href{../works/Trick03.pdf}{Trick03}~\cite{Trick03}, \href{../works/Bartak02a.pdf}{Bartak02a}~\cite{Bartak02a}, \href{../works/BenoistGR02.pdf}{BenoistGR02}~\cite{BenoistGR02}, \href{../works/JussienL02.pdf}{JussienL02}~\cite{JussienL02}, \href{../works/SchildW00.pdf}{SchildW00}~\cite{SchildW00}, \href{../works/NuijtenA96.pdf}{NuijtenA96}~\cite{NuijtenA96} (Total: 88)\\
\end{longtable}
}




\clearpage
\phantomsection
\addcontentsline{toc}{section}{Bibliography}
\bibliographystyle{plainurl}
\bibliography{imports/scheduling}



\appendix
\clearpage
\section{Papers and Articles Missing a Local Copy}

This section lists all papers and articles for which we were not able to locate an electronic copy that we could download to our system. This might be because the work is behind a paywall for which we do not have access, or since the paper only exists in hardcopy, for works from the start of the period covered. As in either case we are not able to extract useful information from the work, either automatically, or manually, without the actual text itself, these gaps should be closed where possible.

Some journals are highlighted in red, indicating that I cannot read these through our library. This helps to avoid repeatedly checking whether I can access the papers via the publication URL. The highlighting is controlled by the blocked.json file in the imports/ directory.

{\scriptsize
\begin{longtable}{llp{5cm}p{10cm}rp{3cm}l}
\caption{Paper without Local Copy}\\ \toprule
Key & URL & Authors & Title & Year & \shortstack{Conference\\/Journal} & Cite\\ \midrule
\endhead
\bottomrule
\endfoot
FriedrichFMRSST14 & \href{https://doi.org/10.1007/978-3-319-28697-6\_23}{FriedrichFMRSST14} & Gerhard Friedrich and Melanie Fr{\"{u}}hst{\"{u}}ck and Vera Mersheeva and Anna Ryabokon and Maria Sander and Andreas Starzacher and Erich Teppan & Representing Production Scheduling with Constraint Answer Set Programming & 2014 & GOR 2014 & \cite{FriedrichFMRSST14}\\VillaverdeP04 & \href{}{VillaverdeP04} & Karen Villaverde and Enrico Pontelli & An Investigation of Scheduling in Distributed Constraint Logic Programming & 2004 & ISCA 2004 & \cite{VillaverdeP04}\\WolinskiKG04a & \href{https://doi.org/10.1145/968280.968336}{WolinskiKG04a} & Christophe Wolinski and Krzysztof Kuchcinski and Maya B. Gokhale & A constraints programming approach to communication scheduling on SoPC architectures & 2004 & FPGA 2004 & \cite{WolinskiKG04a}\\BoucherBVBL97 & \href{}{BoucherBVBL97} & Eric Boucher and Astrid Bachelu and Christophe Varnier and Pierre Baptiste and Bruno Legeard & Multi-criteria Comparison Between Algorithmic, Constraint Logic and Specific Constraint Programming on a Real Schedulingt Problem & 1997 & PACT 1997 & \cite{BoucherBVBL97}\\PapeB97 & \href{}{PapeB97} & Claude Le Pape and Philippe Baptiste & A Constraint Programming Library for Preemptive and Non-Preemptive Scheduling & 1997 & PACT 1997 & \cite{PapeB97}\\JourdanFRD94 & \href{}{JourdanFRD94} & Jean Jourdan and Fran{\c{c}}ois Fages and Didier Rozzonelli and Alain Demeure & Data Alignment and Task Scheduling On Parallel Machines Using Concurrent Constraint Model-based Programming & 1994 & ILPS 1994 & \cite{JourdanFRD94}\\AggounB92 & \href{}{AggounB92} & Abderrahmane Aggoun and Nicolas Beldiceanu & Extending {CHIP} in order to solve complex scheduling and placement problems & 1992 & JFPL 1992 & \cite{AggounB92}\\\end{longtable}
}



{\scriptsize
\begin{longtable}{p{2cm}p{2cm}p{5cm}p{10cm}rp{3cm}l}
\rowcolor{white}\caption{ARTICLE without Local Copy}\\ \toprule
\rowcolor{white}Key & URL & Authors & Title & Year & \shortstack{Conference\\/Journal} & Cite\\ \midrule
\endhead
\bottomrule
\endfoot
AlakaP23 & \href{http://dx.doi.org/10.1007/s00500-023-09105-9}{AlakaP23} & \hyperref[auth:a770]{Hacı Mehmet Alakaş}, \hyperref[auth:a1410]{M. Pınarbaşı} & Balancing of cost-oriented U-type general resource-constrained assembly line: new constraint programming models & 2023 & Soft Computing & \cite{AlakaP23}\\FahimiQ23 & \href{http://dx.doi.org/10.1287/ijoc.2021.0138}{FahimiQ23} & \hyperref[auth:a122]{H. Fahimi}, \hyperref[auth:a123]{C. Quimper} & Overload-Checking and Edge-Finding for Robust Cumulative Scheduling & 2023 & INFORMS Journal on Computing & \cite{FahimiQ23}\\GhasemiMH23 & \href{http://dx.doi.org/10.1080/23302674.2023.2224509}{GhasemiMH23} & \hyperref[auth:a994]{S. Ghasemi}, \hyperref[auth:a433]{R. Tavakkoli{-}Moghaddam}, \hyperref[auth:a995]{M. Hamid} & Operating room scheduling by emphasising human factors and dynamic decision-making styles: a constraint programming method & 2023 & International Journal of Systems Science: Operations \  Logistics & \cite{GhasemiMH23}\\NouriMHD23 & \href{http://dx.doi.org/10.1080/00207543.2023.2173503}{NouriMHD23} & \hyperref[auth:a743]{B. Vahedi-Nouri}, \hyperref[auth:a956]{R. Tavakkoli-Moghaddam}, \hyperref[auth:a957]{Z. Hanzálek}, \hyperref[auth:a958]{A. Dolgui} & Production scheduling in a reconfigurable manufacturing system benefiting from human-robot collaboration & 2023 & International Journal of Production Research & \cite{NouriMHD23}\\CilKLO22 & \href{http://dx.doi.org/10.1016/j.eswa.2022.117529}{CilKLO22} & \hyperref[auth:a1407]{Zeynel Abidin Cil}, \hyperref[auth:a1406]{D. Kizilay}, \hyperref[auth:a1408]{Z. Li}, \hyperref[auth:a1409]{H. \"{O}ztop} & Two-sided disassembly line balancing problem with sequence-dependent setup time: A constraint programming model and artificial bee colony algorithm & 2022 & Expert Systems with Applications & \cite{CilKLO22}\\HillBCGN22 & \href{http://dx.doi.org/10.1287/ijoc.2022.1222}{HillBCGN22} & \hyperref[auth:a64]{A. Hill}, \hyperref[auth:a982]{Andrea J. Brickey}, \hyperref[auth:a983]{I. Cipriano}, \hyperref[auth:a984]{M. Goycoolea}, \hyperref[auth:a985]{A. Newman} & Optimization Strategies for Resource-Constrained Project Scheduling Problems in Underground Mining & 2022 & INFORMS Journal on Computing & \cite{HillBCGN22}\\MartnezAJ22 & \href{http://dx.doi.org/10.1287/ijoc.2021.1079}{MartnezAJ22} & \hyperref[auth:a945]{Karim Pérez Martínez}, \hyperref[auth:a946]{Y. Adulyasak}, \hyperref[auth:a848]{R. Jans} & Logic-Based Benders Decomposition for Integrated Process Configuration and Production Planning Problems & 2022 & INFORMS Journal on Computing & \cite{MartnezAJ22}\\NaderiR22 & \href{http://dx.doi.org/10.1287/ijoo.2021.0056}{NaderiR22} & \hyperref[auth:a732]{B. Naderi}, \hyperref[auth:a734]{V. Roshanaei} & Critical-Path-Search Logic-Based Benders Decomposition Approaches for Flexible Job Shop Scheduling & 2022 & INFORMS Journal on Optimization & \cite{NaderiR22}\\ShiYXQ22 & \href{https://doi.org/10.1080/00207543.2021.1963496}{ShiYXQ22} & \hyperref[auth:a449]{G. Shi}, \hyperref[auth:a450]{Z. Yang}, \hyperref[auth:a451]{Y. Xu}, \hyperref[auth:a452]{Y. Quan} & Solving the integrated process planning and scheduling problem using an enhanced constraint programming-based approach & 2022 & International Journal of Production Research & \cite{ShiYXQ22}\\Alaka21 & \href{http://dx.doi.org/10.1007/s00500-021-05602-x}{Alaka21} & \hyperref[auth:a770]{Hacı Mehmet Alakaş} & General resource-constrained assembly line balancing problem: conjunction normal form based constraint programming models & 2021 & Soft Computing & \cite{Alaka21}\\CarlierSJP21 & \href{http://dx.doi.org/10.1080/00207543.2021.1923853}{CarlierSJP21} & \hyperref[auth:a852]{J. Carlier}, \hyperref[auth:a937]{A. Sahli}, \hyperref[auth:a938]{A. Jouglet}, \hyperref[auth:a853]{E. Pinson} & A faster checker of the energetic reasoning for the cumulative scheduling problem & 2021 & International Journal of Production Research & \cite{CarlierSJP21}\\NaderiRBAU21 & \href{http://dx.doi.org/10.1111/poms.13397}{NaderiRBAU21} & \hyperref[auth:a732]{B. Naderi}, \hyperref[auth:a734]{V. Roshanaei}, \hyperref[auth:a843]{Mehmet A. Begen}, \hyperref[auth:a902]{Dionne M. Aleman}, \hyperref[auth:a903]{David R. Urbach} & Increased Surgical Capacity without Additional Resources: Generalized Operating Room Planning and Scheduling & 2021 & Production and Operations Management & \cite{NaderiRBAU21}\\RabbaniMM21 & \href{http://dx.doi.org/10.1080/17509653.2021.1905096}{RabbaniMM21} & \hyperref[auth:a1268]{M. Rabbani}, \hyperref[auth:a518]{M. Mokhtarzadeh}, \hyperref[auth:a1269]{N. Manavizadeh} & A constraint programming approach and a hybrid of genetic and K-means algorithms to solve the p-hub location-allocation problems & 2021 & International Journal of Management Science and Engineering Management & \cite{RabbaniMM21}\\AlizdehS20 & \href{https://doi.org/10.1504/IJAIP.2020.106687}{AlizdehS20} & \hyperref[auth:a516]{S. Alizdeh}, \hyperref[auth:a517]{S. Saeidi} & Fuzzy project scheduling with critical path including risk and resource constraints using linear programming & 2020 & Int. J. Adv. Intell. Paradigms & \cite{AlizdehS20}\\BalochG20 & \href{http://dx.doi.org/10.1287/trsc.2019.0928}{BalochG20} & \hyperref[auth:a1257]{G. Baloch}, \hyperref[auth:a1258]{F. Gzara} & Strategic Network Design for Parcel Delivery with Drones Under Competition & 2020 & Transportation Science & \cite{BalochG20}\\GuoHLW20 & \href{http://dx.doi.org/10.1080/0305215x.2019.1699919}{GuoHLW20} & \hyperref[auth:a941]{P. Guo}, \hyperref[auth:a942]{X. He}, \hyperref[auth:a943]{Y. Luan}, \hyperref[auth:a944]{Y. Wang} & Logic-based Benders decomposition for gantry crane scheduling with transferring position constraints in a rail–road container terminal & 2020 & Engineering Optimization & \cite{GuoHLW20}\\Ham20 & \href{http://dx.doi.org/10.1080/00207543.2019.1709671}{Ham20} & \hyperref[auth:a756]{A. Ham} & Transfer-robot task scheduling in job shop & 2020 & International Journal of Production Research & \cite{Ham20}\\KizilayC20 & \href{http://dx.doi.org/10.1080/0305215x.2020.1786081}{KizilayC20} & \hyperref[auth:a1406]{D. Kizilay}, \hyperref[auth:a1407]{Zeynel Abidin Cil} & Constraint programming approach for multi-objective two-sided assembly line balancing problem with multi-operator stations & 2020 & Engineering Optimization & \cite{KizilayC20}\\PinarbasiA20 & \href{http://dx.doi.org/10.1080/0305215x.2020.1716746}{PinarbasiA20} & \hyperref[auth:a1410]{M. Pınarbaşı}, \hyperref[auth:a770]{Hacı Mehmet Alakaş} & Balancing stochastic type-II assembly lines: chance-constrained mixed integer and constraint programming models & 2020 & Engineering Optimization & \cite{PinarbasiA20}\\EdwardsBSE19 & \href{http://dx.doi.org/10.1080/01605682.2019.1595192}{EdwardsBSE19} & \hyperref[auth:a899]{Steven J. Edwards}, \hyperref[auth:a900]{D. Baatar}, \hyperref[auth:a901]{K. Smith-Miles}, \hyperref[auth:a472]{Andreas T. Ernst} & Symmetry breaking of identical projects in the high-multiplicity RCPSP/max & 2019 & Journal of the Operational Research Society & \cite{EdwardsBSE19}\\HechingHK19 & \href{http://dx.doi.org/10.1287/trsc.2018.0830}{HechingHK19} & \hyperref[auth:a1034]{A. Heching}, \hyperref[auth:a161]{John N. Hooker}, \hyperref[auth:a1035]{R. Kimura} & A Logic-Based Benders Approach to Home Healthcare Delivery & 2019 & Transportation Science & \cite{HechingHK19}\\WariZ19 & \href{http://dx.doi.org/10.1080/00207543.2019.1571250}{WariZ19} & \hyperref[auth:a846]{E. Wari}, \hyperref[auth:a847]{W. Zhu} & A Constraint Programming model for food processing industry: a case for an ice cream processing facility & 2019 & International Journal of Production Research & \cite{WariZ19}\\RoshanaeiLAU17a & \href{http://dx.doi.org/10.1287/ijoc.2017.0745}{RoshanaeiLAU17a} & \hyperref[auth:a734]{V. Roshanaei}, \hyperref[auth:a935]{C. Luong}, \hyperref[auth:a902]{Dionne M. Aleman}, \hyperref[auth:a903]{David R. Urbach} & Collaborative Operating Room Planning and Scheduling & 2017 & INFORMS Journal on Computing & \cite{RoshanaeiLAU17a}\\PengLC14 & \href{http://dx.doi.org/10.1155/2014/917685}{PengLC14} & \hyperref[auth:a923]{Y. Peng}, \hyperref[auth:a1411]{D. Lu}, \hyperref[auth:a921]{Y. Chen} & A Constraint Programming Method for Advanced Planning and Scheduling System with Multilevel Structured Products & 2014 & Discrete Dynamics in Nature and Society & \cite{PengLC14}\\ZhaoL14 & \href{http://dx.doi.org/10.1016/j.orhc.2014.05.003}{ZhaoL14} & \hyperref[auth:a1402]{Z. Zhao}, \hyperref[auth:a1403]{X. Li} & Scheduling elective surgeries with sequence-dependent setup times to multiple operating rooms using constraint programming & 2014 & Operations Research for Health Care & \cite{ZhaoL14}\\MalapertGR12 & \href{http://dx.doi.org/10.1016/j.ejor.2012.04.008}{MalapertGR12} & \hyperref[auth:a82]{A. Malapert}, \hyperref[auth:a1400]{C. Guéret}, \hyperref[auth:a1401]{L. Rousseau} & A constraint programming approach for a batch processing problem with non-identical job sizes & 2012 & European Journal of Operational Research & \cite{MalapertGR12}\\TopalogluSS12 & \href{http://dx.doi.org/10.1016/j.eswa.2011.09.038}{TopalogluSS12} & \hyperref[auth:a623]{S. Topaloglu}, \hyperref[auth:a1404]{L. Salum}, \hyperref[auth:a1405]{Aliye Ayca Supciller} & Rule-based modeling and constraint programming based solution of the assembly line balancing problem & 2012 & Expert Systems with Applications & \cite{TopalogluSS12}\\ZarandiB12 & \href{http://dx.doi.org/10.1287/ijoc.1110.0458}{ZarandiB12} & \hyperref[auth:a955]{Mohammad M. Fazel-Zarandi}, \hyperref[auth:a89]{J. Christopher Beck} & Using Logic-Based Benders Decomposition to Solve the Capacity- and Distance-Constrained Plant Location Problem & 2012 & INFORMS Journal on Computing & \cite{ZarandiB12}\\EdisO11a & \href{http://dx.doi.org/10.1080/03052151003759117}{EdisO11a} & \hyperref[auth:a349]{Emrah B. Edis}, \hyperref[auth:a351]{I. Ozkarahan} & A combined integer/constraint programming approach to a resource-constrained parallel machine scheduling problem with machine eligibility restrictions & 2011 & Engineering Optimization & \cite{EdisO11a}\\LiuGT10 & \href{http://dx.doi.org/10.3724/sp.j.1004.2010.00603}{LiuGT10} & \hyperref[auth:a1240]{S. Liu}, \hyperref[auth:a1241]{Z. Guo}, \hyperref[auth:a1242]{J. Tang} & Constraint Propagation for Cumulative Scheduling Problems with Precedences: Constraint Propagation for Cumulative Scheduling Problems with Precedences & 2010 & Acta Automatica Sinica & \cite{LiuGT10}\\ZeballosM09 & \href{http://dx.doi.org/10.1021/ie901176n}{ZeballosM09} & \hyperref[auth:a1173]{Luis J. Zeballos}, \hyperref[auth:a1210]{Carlos A. Méndez} & An Integrated CP-Based Approach for Scheduling of Processing and Transport Units in Pipeless Plants & 2009 & Industrial \  Engineering Chemistry Research & \cite{ZeballosM09}\\OkanoDTRYA04 & \href{https://doi.org/10.1147/rd.485.0811}{OkanoDTRYA04} & \hyperref[auth:a1312]{H. Okano}, \hyperref[auth:a250]{Andrew J. Davenport}, \hyperref[auth:a1313]{M. Trumbo}, \hyperref[auth:a252]{C. Reddy}, \hyperref[auth:a1314]{K. Yoda}, \hyperref[auth:a1315]{M. Amano} & Finishing Line Scheduling in the steel industry & 2004 & {IBM} J. Res. Dev. & \cite{OkanoDTRYA04}\\Hentenryck02 & \href{http://dx.doi.org/10.1287/ijoc.14.4.345.2826}{Hentenryck02} & \hyperref[auth:a149]{Pascal Van Hentenryck} & Constraint and Integer Programming in OPL & 2002 & INFORMS Journal on Computing & \cite{Hentenryck02}\\Hooker02 & \href{http://dx.doi.org/10.1287/ijoc.14.4.295.2828}{Hooker02} & \hyperref[auth:a161]{John N. Hooker} & Logic, Optimization, and Constraint Programming & 2002 & INFORMS Journal on Computing & \cite{Hooker02}\\MilanoORT02 & \href{http://dx.doi.org/10.1287/ijoc.14.4.387.2830}{MilanoORT02} & \hyperref[auth:a144]{M. Milano}, \hyperref[auth:a859]{G. Ottosson}, \hyperref[auth:a256]{P. Refalo}, \hyperref[auth:a881]{Erlendur S. Thorsteinsson} & The Role of Integer Programming Techniques in Constraint Programming's Global Constraints & 2002 & INFORMS Journal on Computing & \cite{MilanoORT02}\\LustigP01 & \href{http://dx.doi.org/10.1287/inte.31.6.29.9647}{LustigP01} & Irvin J. Lustig, J. Puget & Program Does Not Equal Program: Constraint Programming and Its Relationship to Mathematical Programming & 2001 & Interfaces & \cite{LustigP01}\\BockmayrK98 & \href{http://dx.doi.org/10.1287/ijoc.10.3.287}{BockmayrK98} & \hyperref[auth:a916]{A. Bockmayr}, \hyperref[auth:a1060]{T. Kasper} & Branch and Infer: A Unifying Framework for Integer and Finite Domain Constraint Programming & 1998 & INFORMS Journal on Computing & \cite{BockmayrK98}\\DarbyDowmanL98 & \href{http://dx.doi.org/10.1287/ijoc.10.3.276}{DarbyDowmanL98} & \hyperref[auth:a1086]{K. Darby-Dowman}, \hyperref[auth:a179]{J. Little} & Properties of Some Combinatorial Optimization Problems and Their Effect on the Performance of Integer Programming and Constraint Logic Programming & 1998 & INFORMS Journal on Computing & \cite{DarbyDowmanL98}\\PintoG97 & \href{https://www.sciencedirect.com/science/article/pii/S0098135496003183}{PintoG97} & \hyperref[auth:a1277]{Jose M. Pinto}, \hyperref[auth:a385]{Ignacio E. Grossmann} & A logic-based approach to scheduling problems with resource constraints & 1997 & Computers \  Chemical Engineering & \cite{PintoG97}\\PeschT96 & \href{http://dx.doi.org/10.1287/ijoc.8.2.144}{PeschT96} & \hyperref[auth:a441]{E. Pesch}, \hyperref[auth:a1236]{Ulrich A. W. Tetzlaff} & Constraint Propagation Based Scheduling of Job Shops & 1996 & INFORMS Journal on Computing & \cite{PeschT96}\\LubySZ93 & \href{http://dx.doi.org/10.1016/0020-0190(93)90029-9}{LubySZ93} & M. Luby, A. Sinclair, D. Zuckerman & Optimal speedup of Las Vegas algorithms & 1993 & Information Processing Letters & \cite{LubySZ93}\\MintonJPL92 & \href{http://dx.doi.org/10.1016/0004-3702(92)90007-k}{MintonJPL92} & \hyperref[auth:a1230]{S. Minton}, \hyperref[auth:a1231]{Mark D. Johnston}, \hyperref[auth:a1232]{Andrew B. Philips}, \hyperref[auth:a1233]{P. Laird} & Minimizing conflicts: a heuristic repair method for constraint satisfaction and scheduling problems & 1992 & Artificial Intelligence & \cite{MintonJPL92}\\Tay92 & \href{}{Tay92} & \hyperref[auth:a707]{David B. H. Tay} & {COPS:} {A} Constraint Programming Approach to Resource-Limited Project Scheduling & 1992 & Comput. J. & \cite{Tay92}\\Carlier82 & \href{http://dx.doi.org/10.1016/s0377-2217(82)80007-6}{Carlier82} & \hyperref[auth:a852]{J. Carlier} & The one-machine sequencing problem & 1982 & European Journal of Operational Research & \cite{Carlier82}\\Lauriere78 & \href{http://dx.doi.org/10.1016/0004-3702(78)90029-2}{Lauriere78} & J. Lauriere & A language and a program for stating and solving combinatorial problems & 1978 & Artificial Intelligence & \cite{Lauriere78}\\Mackworth77 & \href{http://dx.doi.org/10.1016/0004-3702(77)90007-8}{Mackworth77} & Alan K. Mackworth & Consistency in networks of relations & 1977 & Artificial Intelligence & \cite{Mackworth77}\\GareyJS76 & \href{http://dx.doi.org/10.1287/moor.1.2.117}{GareyJS76} & M. R. Garey, D. S. Johnson, R. Sethi & The Complexity of Flowshop and Jobshop Scheduling & 1976 & Mathematics of Operations Research & \cite{GareyJS76}\\PritskerWW69 & \href{http://dx.doi.org/10.1287/mnsc.16.1.93}{PritskerWW69} & A. Alan B. Pritsker, Lawrence J. Waiters, Philip M. Wolfe & Multiproject Scheduling with Limited Resources: A Zero-One Programming Approach & 1969 & Management Science & \cite{PritskerWW69}\\\end{longtable}
}



{\scriptsize
\begin{longtable}{p{2cm}p{2cm}p{5cm}p{10cm}rp{3cm}l}
\rowcolor{white}\caption{INBOOK without Local Copy}\\ \toprule
\rowcolor{white}Key & URL & Authors & Title & Year & \shortstack{Conference\\/Journal} & Cite\\ \midrule
\endhead
\bottomrule
\endfoot
SchuttFSW15 & \href{https://doi.org/10.1007/978-3-319-05443-8_7}{SchuttFSW15} & \hyperref[auth:a124]{A. Schutt}, \hyperref[auth:a154]{T. Feydy}, \hyperref[auth:a125]{P. J. Stuckey}, \hyperref[auth:a117]{M. G. Wallace} & A Satisfiability Solving Approach & 2015 & Handbook on Project Management and Scheduling Vol.1 & \cite{SchuttFSW15}\\
CestaOPS14 & \href{http://dx.doi.org/10.1007/978-3-319-05443-8_6}{CestaOPS14} & \hyperref[auth:a284]{A. Cesta}, \hyperref[auth:a282]{A. Oddi}, \hyperref[auth:a283]{N. Policella}, \hyperref[auth:a298]{S. F. Smith} & A Precedence Constraint Posting Approach & 2014 & Handbook on Project Management and Scheduling Vol.1 & \cite{CestaOPS14}\\
GuSSWC14 & \href{http://dx.doi.org/10.1007/978-3-319-05443-8_14}{GuSSWC14} & \hyperref[auth:a336]{H. Gu}, \hyperref[auth:a124]{A. Schutt}, \hyperref[auth:a125]{P. J. Stuckey}, \hyperref[auth:a117]{M. G. Wallace}, \hyperref[auth:a343]{G. Chu} & Exact and Heuristic Methods for the Resource-Constrained Net Present Value Problem & 2014 & Handbook on Project Management and Scheduling Vol.1 & \cite{GuSSWC14}\\
Milano11 & \href{http://dx.doi.org/10.1002/9780470400531.eorms0473}{Milano11} & \hyperref[auth:a143]{M. Milano} & Constraint Programming Links with Math Programming & 2011 & Wiley Encyclopedia of Operations Research and Management Science & \cite{Milano11}\\
CastroGR10 & \href{http://dx.doi.org/10.1007/978-1-4419-1644-0_4}{CastroGR10} & \hyperref[auth:a891]{P. M. Castro}, \hyperref[auth:a382]{I. E. Grossmann}, \hyperref[auth:a326]{L.-M. Rousseau} & Decomposition Techniques for Hybrid MILP/CP Models applied to Scheduling and Routing Problems & 2010 & Hybrid Optimization & \cite{CastroGR10}\\
Hooker10 & \href{http://dx.doi.org/10.1007/978-1-4419-1644-0_2}{Hooker10} & \hyperref[auth:a160]{J. N. Hooker} & Hybrid Modeling & 2010 & Hybrid Optimization & \cite{Hooker10}\\
GongLMW09 & \href{http://dx.doi.org/10.1007/978-0-387-88617-6_11}{GongLMW09} & \hyperref[auth:a1234]{J. Gong}, \hyperref[auth:a1235]{E. E. Lee}, \hyperref[auth:a1236]{J. E. Mitchell}, \hyperref[auth:a1237]{W. A. Wallace} & Logic-based MultiObjective Optimization for Restoration Planning & 2009 & Optimization and Logistics Challenges in the Enterprise & \cite{GongLMW09}\\
AggounMV08 & \href{http://dx.doi.org/10.1007/978-0-387-74759-0_396}{AggounMV08} & \hyperref[auth:a725]{A. Aggoun}, \hyperref[auth:a381]{C. T. Maravelias}, \hyperref[auth:a907]{A. Vazacopoulos} & Mixed Integer Programming/Constraint Programming Hybrid Methods & 2008 & Encyclopedia of Optimization & \cite{AggounMV08}\\
Hooker06a & \href{http://dx.doi.org/10.1016/s1574-6526(06)80019-2}{Hooker06a} & \hyperref[auth:a160]{J. N. Hooker} & Operations Research Methods in Constraint Programming & 2006 & Foundations of Artificial Intelligence & \cite{Hooker06a}\\
NeronABCDD06 & \href{http://dx.doi.org/10.1007/978-0-387-33768-5_7}{NeronABCDD06} & \hyperref[auth:a899]{E. Néron}, \hyperref[auth:a6]{C. Artigues}, \hyperref[auth:a162]{P. Baptiste}, \hyperref[auth:a845]{J. Carlier}, \hyperref[auth:a900]{J. Damay}, \hyperref[auth:a243]{S. Demassey}, \hyperref[auth:a118]{P. Laborie} & Lower Bounds for Resource Constrained Project Scheduling Problem & 2006 & Perspectives in Modern Project Scheduling & \cite{NeronABCDD06}\\
AggounV04 & \href{http://dx.doi.org/10.1007/978-3-540-24734-0_15}{AggounV04} & \hyperref[auth:a725]{A. Aggoun}, \hyperref[auth:a907]{A. Vazacopoulos} & Solving Sports Scheduling and Timetabling Problems with Constraint Programming & 2004 & Economics, Management and Optimization in Sports & \cite{AggounV04}\\
AjiliW04 & \href{http://dx.doi.org/10.1007/978-1-4419-8917-8_6}{AjiliW04} & \hyperref[auth:a950]{F. Ajili}, \hyperref[auth:a117]{M. G. Wallace} & Hybrid Problem Solving in ECLiPSe & 2004 & Constraint and Integer Programming & \cite{AjiliW04}\\
DannaP04 & \href{http://dx.doi.org/10.1007/978-1-4419-8917-8_2}{DannaP04} & \hyperref[auth:a287]{E. Danna}, \hyperref[auth:a163]{C. L. Pape} & Two Generic Schemes for Efficient and Robust Cooperative Algorithms & 2004 & Constraints and Integer Programming & \cite{DannaP04}\\
DomdorfPH03 & \href{http://dx.doi.org/10.1007/978-3-642-18965-4_31}{DomdorfPH03} & \hyperref[auth:a960]{U. Domdorf}, \hyperref[auth:a438]{E. Pesch}, \hyperref[auth:a961]{T. P. Huy} & Machine Learning by Schedule Decomposition — Prospects for an Integration of AI and OR Techniques for Job Shop Scheduling & 2003 & Advances in Evolutionary Computing & \cite{DomdorfPH03}\\
Rgin2001 & \href{http://dx.doi.org/10.1090/dimacs/057/07}{Rgin2001} & \hyperref[auth:a1421]{J.-C. Régin} & Minimization of the number of breaks in sports scheduling problems using constraint programming & 2001 & DIMACS Series in Discrete Mathematics and Theoretical Computer Science & \cite{Rgin2001}\\
DorndorfHP99 & \href{http://dx.doi.org/10.1007/978-1-4615-5533-9_10}{DorndorfHP99} & \hyperref[auth:a904]{U. Dorndorf}, \hyperref[auth:a905]{T. P. Huy}, \hyperref[auth:a438]{E. Pesch} & A Survey of Interval Capacity Consistency Tests for Time- and Resource-Constrained Scheduling & 1999 & Project Scheduling & \cite{DorndorfHP99}\\
GrahamLLK79 & \href{http://dx.doi.org/10.1016/s0167-5060(08)70356-x}{GrahamLLK79} & R. L. Graham, E. L. Lawler, J. K. Lenstra, A. H. G. Rinnooy Kan & \cellcolor{green!10}Optimization and Approximation in Deterministic Sequencing and Scheduling: a Survey & 1979 & Annals of Discrete Mathematics & \cite{GrahamLLK79}\\
\end{longtable}
}



{\scriptsize
\begin{longtable}{p{3cm}p{5cm}p{10cm}p{1cm}rp{2.5cm}l}
\rowcolor{white}\caption{INCOLLECTION without Local Copy (Total 7)}\\ \toprule
\rowcolor{white}Key/URL & Authors & Title & Relevance &Year & \shortstack{Conference\\/Journal} & Cite\\ \midrule
\endhead
\bottomrule
\endfoot
BlazewiczEP19 \href{https://ideas.repec.org/h/spr/ihichp/978-3-319-99849-7_16.html}{BlazewiczEP19} & \hyperref[auth:a764]{J. Blazewicz}, \hyperref[auth:a765]{K. H. Ecker}, \hyperref[auth:a437]{E. Pesch}, \hyperref[auth:a766]{G. Schmidt}, \hyperref[auth:a767]{M. Sterna}, \hyperref[auth:a768]{J. Weglarz} & {Constraint Programming and Disjunctive Scheduling} & \noindent{}\textbf{1.00} \textbf{2.50} n/a & 2019 & {Handbook on Scheduling} & \cite{BlazewiczEP19}\\
Bartak14 \href{}{Bartak14} & \hyperref[auth:a152]{R. Bart{\'{a}}k} & Planning and Scheduling & \noindent{}\textcolor{black!50}{0.00} \textcolor{black!50}{0.00} n/a & 2014 & Computing Handbook, Third Edition: Computer Science and Software Engineering & \cite{Bartak14}\\
Trick11 \href{http://dx.doi.org/10.1007/978-1-4419-1644-0_15}{Trick11} & \hyperref[auth:a1388]{M. A. Trick} & Sports Scheduling & \noindent{}\textcolor{black!50}{0.00} \textbf{1.00} n/a & 2011 & HYBRID OPTIMIZATION: THE TEN YEARS OF CPAIOR & \cite{Trick11}\\
BriandHHL08 \href{http://dx.doi.org/10.1002/9780470611432.ch9}{BriandHHL08} & \hyperref[auth:a1197]{C. Briand}, \hyperref[auth:a1198]{M.-J. Huguet}, \hyperref[auth:a1199]{H. T. La}, \hyperref[auth:a3]{P. Lopez} & Constraint-based Approaches for Robust Scheduling & \noindent{}\textcolor{black!50}{0.00} \textcolor{black!50}{0.00} n/a & 2008 & Flexibility and Robustness in Scheduling & \cite{BriandHHL08}\\
EsquirolLH2008 \href{http://dx.doi.org/10.1002/9780470611050.ch5}{EsquirolLH2008} & \hyperref[auth:a1247]{P. Esquirol}, \hyperref[auth:a3]{P. Lopez}, \hyperref[auth:a1198]{M.-J. Huguet} & Constraint Propagation and Scheduling & \noindent{}\textbf{1.50} \textbf{1.50} n/a & 2008 & Production Scheduling & \cite{EsquirolLH2008}\\
BaptisteLPN06 \href{https://doi.org/10.1016/S1574-6526(06)80026-X}{BaptisteLPN06} & \hyperref[auth:a162]{P. Baptiste}, \hyperref[auth:a118]{P. Laborie}, \hyperref[auth:a163]{C. L. Pape}, \hyperref[auth:a655]{W. Nuijten} & Constraint-Based Scheduling and Planning & \noindent{}\textcolor{black!50}{0.00} \textcolor{black!50}{0.00} n/a & 2006 & Handbook of Constraint Programming & \cite{BaptisteLPN06}\\
BreitingerL95 \href{}{BreitingerL95} & \hyperref[auth:a694]{S. Breitinger}, \hyperref[auth:a695]{H. C. R. Lock} & Using Constraint Logic Programming for Industrial Scheduling Problems & \noindent{}\textbf{1.00} \textbf{1.00} n/a & 1995 & Logic Programming: Formal Methods and Practical Applications, Studies in Computer Science and Artificial Intelligence & \cite{BreitingerL95}\\
\end{longtable}
}



\clearpage
\section{Papers and Articles Without Recognized Concepts}

This section lists papers and articles for which we have a pdf local copy, but where we were not able to extract any of the defined concepts. This can basically have two reasons. We either have included a paper which is not at all related to scheduling, so that none of the defined concepts occur in the paper. A  more likely cause is that the pdf file is a scanned document for which optical character recognition was not run or not successful, so that the pdf consists of a series of bitmap images. In that case, pdfgrep is unable to find any text in the document, and no matches for concepts are found. It may be useful to check the pdf files to see if that is the case.

{\scriptsize
\begin{longtable}{llp{5cm}p{10cm}rp{3cm}lr}
\rowcolor{white}\caption{PAPER without Concepts}\\ \toprule
\rowcolor{white}Key & \shortstack{Local\\Copy} & Authors & Title & Year & \shortstack{Conference\\/Journal} & Cite & Pages\\ \midrule
\endhead
\bottomrule
\endfoot
BaptisteLV92 & \href{../works/BaptisteLV92.pdf}{Yes} & \hyperref[auth:a703]{P. Baptiste}, \hyperref[auth:a704]{B. Legeard}, \hyperref[auth:a702]{C. Varnier} & Hoist scheduling problem: an approach based on constraint logic programming & 1992 & ICRA 1992 & \cite{BaptisteLV92} & 6\\DincbasHSAGB88 & \href{../works/DincbasHSAGB88.pdf}{Yes} & \hyperref[auth:a726]{M. Dincbas}, \hyperref[auth:a149]{Pascal Van Hentenryck}, \hyperref[auth:a17]{H. Simonis}, \hyperref[auth:a734]{A. Aggoun}, T. Graf, F. Berthier & The Constraint Logic Programming Language {CHIP} & 1988 & FGCS 1988 & \cite{DincbasHSAGB88} & 10\\\end{longtable}
}



{\scriptsize
\begin{longtable}{llp{5cm}p{10cm}rp{3cm}lr}
\rowcolor{white}\caption{ARTICLE without Concepts}\\ \toprule
\rowcolor{white}Key & \shortstack{Local\\Copy} & Authors & Title & Year & \shortstack{Conference\\/Journal} & Cite & Pages\\ \midrule
\endhead
\bottomrule
\endfoot
KorbaaYG00 & \href{../works/KorbaaYG00.pdf}{Yes} & \hyperref[auth:a690]{O. Korbaa}, \hyperref[auth:a691]{P. Yim}, \hyperref[auth:a692]{J. Gentina} & Solving Transient Scheduling Problems with Constraint Programming & 2000 & Eur. J. Control & \cite{KorbaaYG00} & 10\\LopezAKYG00 & \href{../works/LopezAKYG00.pdf}{Yes} & \hyperref[auth:a3]{P. Lopez}, \hyperref[auth:a693]{H. Alla}, \hyperref[auth:a690]{O. Korbaa}, \hyperref[auth:a691]{P. Yim}, \hyperref[auth:a692]{J. Gentina} & Discussion on: 'Solving Transient Scheduling Problems with Constraint Programming' by O. Korbaa, P. Yim, and {J.-C.} Gentina & 2000 & Eur. J. Control & \cite{LopezAKYG00} & 4\\CarlierP94 & \href{../works/CarlierP94.pdf}{Yes} & \hyperref[auth:a857]{J. Carlier}, \hyperref[auth:a858]{E. Pinson} & Adjustment of heads and tails for the job-shop problem & 1994 & European Journal of Operational Research & \cite{CarlierP94} & 16\\ApplegateC91 & \href{../works/ApplegateC91.pdf}{Yes} & D. Applegate, W. Cook & A Computational Study of the Job-Shop Scheduling Problem & 1991 & ORSA Journal on Computing & \cite{ApplegateC91} & 8\\\end{longtable}
}



\subsection{Irrelevant Work Suspects}
The works in the following table might be irrelevant, as they either do not contain the concept "scheduling", or none of the "constraint" related concepts.

{\scriptsize
\begin{longtable}{>{\raggedright\arraybackslash}p{3cm}>{\raggedright\arraybackslash}p{4.5cm}>{\raggedright\arraybackslash}p{6.0cm}rrrp{2.5cm}rp{1cm}p{1cm}rr}
\rowcolor{white}\caption{Works that might be Irrelevant (Total 26)}\\ \toprule
\rowcolor{white}\shortstack{Key\\Source} & Authors & Title (Colored by Open Access)& LC & Cite & Year & \shortstack{Conference\\/Journal\\/School} & Pages & \shortstack{Cites\\OC XR\\SC} & \shortstack{Refs\\OC\\XR} & b & c \\ \midrule\endhead
\bottomrule
\endfoot
ShinBBHO18 \href{https://doi.org/10.1109/TSMC.2017.2681623}{ShinBBHO18} & \hyperref[auth:a573]{S. Y. Shin}, \hyperref[auth:a574]{Y. Brun}, \hyperref[auth:a575]{H. Balasubramanian}, \hyperref[auth:a576]{P. L. Henneman}, \hyperref[auth:a577]{L. J. Osterweil} & \cellcolor{gold!20}Discrete-Event Simulation and Integer Linear Programming for Constraint-Aware Resource Scheduling & \href{../works/ShinBBHO18.pdf}{Yes} & \cite{ShinBBHO18} & 2018 & {IEEE} Trans. Syst. Man Cybern. Syst. & 16 & 9 9 12 & 31 39 & \ref{b:ShinBBHO18} & \ref{c:ShinBBHO18}\\
HebrardHJMPV16 \href{https://doi.org/10.1016/j.dam.2016.07.003}{HebrardHJMPV16} & \hyperref[auth:a1]{E. Hebrard}, \hyperref[auth:a54]{M.-J. Huguet}, \hyperref[auth:a791]{N. Jozefowiez}, \hyperref[auth:a787]{A. Maillard}, \hyperref[auth:a21]{C. Pralet}, \hyperref[auth:a173]{G. Verfaillie} & \cellcolor{gold!20}Approximation of the parallel machine scheduling problem with additional unit resources & \href{../works/HebrardHJMPV16.pdf}{Yes} & \cite{HebrardHJMPV16} & 2016 & Discrete Applied Mathematics & 10 & 9 10 12 & 8 8 & \ref{b:HebrardHJMPV16} & n/a\\
LuoVLBM16 \href{http://www.aaai.org/ocs/index.php/KR/KR16/paper/view/12909}{LuoVLBM16} & \hyperref[auth:a813]{R. Luo}, \hyperref[auth:a814]{R. A. Valenzano}, \hyperref[auth:a815]{Y. Li}, \hyperref[auth:a89]{J. C. Beck}, \hyperref[auth:a816]{S. A. McIlraith} & Using Metric Temporal Logic to Specify Scheduling Problems & \href{../works/LuoVLBM16.pdf}{Yes} & \cite{LuoVLBM16} & 2016 & KR 2016 & 4 & 0 0 0 & 0 0 & \ref{b:LuoVLBM16} & n/a\\
LouieVNB14 \href{https://doi.org/10.1109/ICRA.2014.6907637}{LouieVNB14} & \hyperref[auth:a819]{W.-Y. G. Louie}, \hyperref[auth:a804]{T. S. Vaquero}, \hyperref[auth:a204]{G. Nejat}, \hyperref[auth:a89]{J. C. Beck} & An autonomous assistive robot for planning, scheduling and facilitating multi-user activities & \href{../works/LouieVNB14.pdf}{Yes} & \cite{LouieVNB14} & 2014 & ICRA 2014 & 7 & 16 16 28 & 9 21 & \ref{b:LouieVNB14} & n/a\\
LudwigKRBMS14 \href{https://doi.org/10.1609/aaai.v28i2.19030}{LudwigKRBMS14} & \hyperref[auth:a1349]{J. Ludwig}, \hyperref[auth:a1350]{A. Kalton}, \hyperref[auth:a1351]{R. Richards}, \hyperref[auth:a1352]{B. Bautsch}, \hyperref[auth:a1353]{C. Markusic}, \hyperref[auth:a1354]{J. Schumacher} & A Schedule Optimization Tool for Destructive and Non-Destructive Vehicle Tests & \href{../works/LudwigKRBMS14.pdf}{Yes} & \cite{LudwigKRBMS14} & 2014 & AAAI 2014 & 6 & 1 1 0 & 0 0 & \ref{b:LudwigKRBMS14} & n/a\\
KameugneF13 \href{http://dx.doi.org/10.1007/s13226-013-0005-z}{KameugneF13} & \hyperref[auth:a10]{R. Kameugne}, \hyperref[auth:a130]{L. P. Fotso} & A cumulative not-first/not-last filtering algorithm in O(n 2log(n)) & \href{../works/KameugneF13.pdf}{Yes} & \cite{KameugneF13} & 2013 & Indian Journal of Pure and Applied Mathematics & 21 & 6 8 8 & 4 19 & \ref{b:KameugneF13} & n/a\\
HeinzSSW12 \href{https://doi.org/10.1007/s10601-011-9113-8}{HeinzSSW12} & \hyperref[auth:a133]{S. Heinz}, \hyperref[auth:a139]{T. Schlechte}, \hyperref[auth:a140]{R. Stephan}, \hyperref[auth:a141]{M. Winkler} & Solving steel mill slab design problems & \href{../works/HeinzSSW12.pdf}{Yes} & \cite{HeinzSSW12} & 2012 & Constraints An Int. J. & 12 & 10 11 12 & 9 16 & \ref{b:HeinzSSW12} & \ref{c:HeinzSSW12}\\
SchausHMCMD11 \href{https://doi.org/10.1007/s10601-010-9100-5}{SchausHMCMD11} & \hyperref[auth:a147]{P. Schaus}, \hyperref[auth:a148]{P. V. Hentenryck}, \hyperref[auth:a149]{J.-N. Monette}, \hyperref[auth:a150]{C. Coffrin}, \hyperref[auth:a32]{L. Michel}, \hyperref[auth:a151]{Y. Deville} & \cellcolor{green!10}Solving Steel Mill Slab Problems with constraint-based techniques: CP, LNS, and {CBLS} & \href{../works/SchausHMCMD11.pdf}{Yes} & \cite{SchausHMCMD11} & 2011 & Constraints An Int. J. & 23 & 14 16 19 & 5 12 & \ref{b:SchausHMCMD11} & \ref{c:SchausHMCMD11}\\
AronssonBK09 \href{http://drops.dagstuhl.de/opus/volltexte/2009/2141}{AronssonBK09} & \hyperref[auth:a707]{M. Aronsson}, \hyperref[auth:a708]{M. Bohlin}, \hyperref[auth:a709]{P. Kreuger} & {MILP} formulations of cumulative constraints for railway scheduling - {A} comparative study & \href{../works/AronssonBK09.pdf}{Yes} & \cite{AronssonBK09} & 2009 & ATMOS 2009 & 13 & 0 0 0 & 0 0 & \ref{b:AronssonBK09} & n/a\\
HentenryckM08 \href{https://doi.org/10.1007/978-3-540-68155-7_41}{HentenryckM08} & \hyperref[auth:a148]{P. V. Hentenryck}, \hyperref[auth:a32]{L. Michel} & The Steel Mill Slab Design Problem Revisited & \href{../works/HentenryckM08.pdf}{Yes} & \cite{HentenryckM08} & 2008 & CPAIOR 2008 & 5 & 13 14 23 & 3 7 & \ref{b:HentenryckM08} & n/a\\
SchausD08 \href{http://www.aaai.org/Library/AAAI/2008/aaai08-058.php}{SchausD08} & \hyperref[auth:a147]{P. Schaus}, \hyperref[auth:a151]{Y. Deville} & A Global Constraint for Bin-Packing with Precedences: Application to the Assembly Line Balancing Problem & \href{../works/SchausD08.pdf}{Yes} & \cite{SchausD08} & 2008 & AAAI 2008 & 6 & 0 0 0 & 0 0 & \ref{b:SchausD08} & n/a\\
DoRZ08 \href{http://www.aaai.org/Library/AAAI/2008/aaai08-253.php}{DoRZ08} & \hyperref[auth:a1346]{M. B. Do}, \hyperref[auth:a1347]{W. Ruml}, \hyperref[auth:a1348]{R. Zhou} & On-line Planning and Scheduling: An Application to Controlling Modular Printers & \href{../works/DoRZ08.pdf}{Yes} & \cite{DoRZ08} & 2008 & AAAI 2008 & 5 & 0 0 0 & 0 0 & \ref{b:DoRZ08} & n/a\\
GarganiR07 \href{https://doi.org/10.1007/978-3-540-74970-7_8}{GarganiR07} & \hyperref[auth:a253]{A. Gargani}, \hyperref[auth:a254]{P. Refalo} & An Efficient Model and Strategy for the Steel Mill Slab Design Problem & \href{../works/GarganiR07.pdf}{Yes} & \cite{GarganiR07} & 2007 & CP 2007 & 13 & 17 18 28 & 5 12 & \ref{b:GarganiR07} & n/a\\
ElhouraniDM07 \href{http://www.aaai.org/Library/AAAI/2007/aaai07-213.php}{ElhouraniDM07} & \hyperref[auth:a1343]{T. Elhourani}, \hyperref[auth:a1344]{N. Denny}, \hyperref[auth:a1345]{M. M. Marefat} & A Distributed Constraint Optimization Solution to the {P2P} Video Streaming Problem & \href{../works/ElhouraniDM07.pdf}{Yes} & \cite{ElhouraniDM07} & 2007 & AAAI 2007 & 6 & 0 0 0 & 0 0 & \ref{b:ElhouraniDM07} & n/a\\
LiuJ06 \href{https://doi.org/10.1007/11801603_92}{LiuJ06} & \hyperref[auth:a654]{Y. Liu}, \hyperref[auth:a655]{Y. Jiang} & {LP-TPOP:} Integrating Planning and Scheduling Through Constraint Programming & \href{../works/LiuJ06.pdf}{Yes} & \cite{LiuJ06} & 2006 & PRICAI 2006 & 5 & 0 0 1 & 0 0 & \ref{b:LiuJ06} & n/a\\
MaraveliasG04 \href{https://doi.org/10.1007/978-3-540-24664-0_1}{MaraveliasG04} & \hyperref[auth:a381]{C. T. Maravelias}, \hyperref[auth:a382]{I. E. Grossmann} & Using {MILP} and {CP} for the Scheduling of Batch Chemical Processes & \href{../works/MaraveliasG04.pdf}{Yes} & \cite{MaraveliasG04} & 2004 & CPAIOR 2004 & 20 & 15 14 23 & 15 23 & \ref{b:MaraveliasG04} & n/a\\
FrankK03 \href{http://www.aaai.org/Library/ICAPS/2003/icaps03-023.php}{FrankK03} & \hyperref[auth:a379]{J. Frank}, \hyperref[auth:a380]{E. K{\"{u}}rkl{\"{u}}} & SOFIA's Choice: Scheduling Observations for an Airborne Observatory & \href{../works/FrankK03.pdf}{Yes} & \cite{FrankK03} & 2003 & ICAPS 2003 & 10 & 0 0 0 & 0 0 & \ref{b:FrankK03} & n/a\\
Layfield02 \href{http://etheses.whiterose.ac.uk/1301/}{Layfield02} & \hyperref[auth:a670]{C. J. Layfield} & A constraint programming pre-processor for duty scheduling & \href{../works/Layfield02.pdf}{Yes} & \cite{Layfield02} & 2002 & University of Leeds, {UK} & 230 & 0 0 0 & 0 0 & \ref{b:Layfield02} & n/a\\
Refalo00 \href{https://doi.org/10.1007/3-540-45349-0_27}{Refalo00} & \hyperref[auth:a254]{P. Refalo} & Linear Formulation of Constraint Programming Models and Hybrid Solvers & \href{../works/Refalo00.pdf}{Yes} & \cite{Refalo00} & 2000 & CP 2000 & 15 & 35 37 49 & 11 22 & \ref{b:Refalo00} & n/a\\
BensanaLV99 \href{https://doi.org/10.1023/A:1026488509554}{BensanaLV99} & \hyperref[auth:a171]{E. Bensana}, \hyperref[auth:a172]{M. Lema{\^{\i}}tre}, \hyperref[auth:a173]{G. Verfaillie} & Earth Observation Satellite Management & \href{../works/BensanaLV99.pdf}{Yes} & \cite{BensanaLV99} & 1999 & Constraints An Int. J. & 7 & 99 0 150 & 0 0 & \ref{b:BensanaLV99} & \ref{c:BensanaLV99}\\
SmithBHW96 \href{http://dx.doi.org/10.1007/bf00143880}{SmithBHW96} & \hyperref[auth:a1054]{B. M. Smith}, \hyperref[auth:a1052]{S. C. Brailsford}, \hyperref[auth:a1180]{P. M. Hubbard}, \hyperref[auth:a1181]{H. P. Williams} & The progressive party problem: Integer linear programming and constraint programming compared & \href{../works/SmithBHW96.pdf}{Yes} & \cite{SmithBHW96} & 1996 & Constraints An Int. J. & 20 & 56 57 61 & 4 9 & \ref{b:SmithBHW96} & n/a\\
CrawfordB94 \href{http://www.aaai.org/Library/AAAI/1994/aaai94-168.php}{CrawfordB94} & \hyperref[auth:a1278]{J. M. Crawford}, \hyperref[auth:a1279]{A. B. Baker} & Experimental Results on the Application of Satisfiability Algorithms to Scheduling Problems & \href{../works/CrawfordB94.pdf}{Yes} & \cite{CrawfordB94} & 1994 & AAAI 1994 & 6 & 0 0 0 & 0 0 & \ref{b:CrawfordB94} & n/a\\
Hamscher91 \href{http://www.aaai.org/Library/AAAI/1991/aaai91-079.php}{Hamscher91} & \hyperref[auth:a1276]{W. Hamscher} & {ACP:} Reason Maintenance and Inference Control for Constraint Propagation Over Intervals & \href{../works/Hamscher91.pdf}{Yes} & \cite{Hamscher91} & 1991 & AAAI 1991 & 6 & 0 0 0 & 0 0 & \ref{b:Hamscher91} & n/a\\
Valdes87 \href{http://www.aaai.org/Library/AAAI/1987/aaai87-046.php}{Valdes87} & \hyperref[auth:a1273]{R. E. Vald{\'{e}}s-P{\'{e}}rez} & The Satisfiability of Temporal Constraint Networks & \href{../works/Valdes87.pdf}{Yes} & \cite{Valdes87} & 1987 & AAAI 1987 & 5 & 0 0 0 & 0 0 & \ref{b:Valdes87} & n/a\\
Rit86 \href{http://www.aaai.org/Library/AAAI/1986/aaai86-064.php}{Rit86} & \hyperref[auth:a1272]{J.-F. Rit} & Propagating Temporal Constraints for Scheduling & \href{../works/Rit86.pdf}{Yes} & \cite{Rit86} & 1986 & AAAI 1986 & 6 & 0 0 0 & 0 0 & \ref{b:Rit86} & n/a\\
FoxAS82 \href{http://www.aaai.org/Library/AAAI/1982/aaai82-037.php}{FoxAS82} & \hyperref[auth:a302]{M. S. Fox}, \hyperref[auth:a1006]{B. P. Allen}, \hyperref[auth:a1007]{G. Strohm} & Job-Shop Scheduling: An Investigation in Constraint-Directed Reasoning & \href{../works/FoxAS82.pdf}{Yes} & \cite{FoxAS82} & 1982 & AAAI 1982 & 4 & 0 0 0 & 0 0 & \ref{b:FoxAS82} & n/a\\
\end{longtable}
}


{\scriptsize
\begin{longtable}{>{\raggedright\arraybackslash}p{3cm}r>{\raggedright\arraybackslash}p{4cm}p{1.5cm}p{2cm}p{1.5cm}p{1.5cm}p{1.5cm}p{1.5cm}p{2cm}p{1.5cm}rr}
\rowcolor{white}\caption{Features of Works that might be Irrelevant}\\ \toprule
\rowcolor{white}Work & Pages & Concepts & Classification & Constraints & \shortstack{Prog\\Languages} & \shortstack{CP\\Systems} & Areas & Industries & Benchmarks & Algorithm & a & c\\ \midrule\endhead
\bottomrule
\endfoot
\href{../works/ShinBBHO18.pdf}{ShinBBHO18}~\cite{ShinBBHO18} & 16 & order, transportation, job, scheduling, task, machine, preempt, resource, activity, stochastic, inventory &  &  &  &  & physician, nurse, patient, medical &  & github, real-world &  & \ref{a:ShinBBHO18} & \ref{c:ShinBBHO18}\\
\href{../works/HebrardHJMPV16.pdf}{HebrardHJMPV16}~\cite{HebrardHJMPV16} & 10 & online scheduling, cmax, scheduling, order, make-span, distributed, machine, job, completion-time, resource, task & parallel machine & cumulative &  &  & satellite, earth observation &  & industrial partner &  & \ref{a:HebrardHJMPV16} & n/a\\
\href{../works/LuoVLBM16.pdf}{LuoVLBM16}~\cite{LuoVLBM16} & 4 & job-shop, resource, order, activity, scheduling, task, job, machine, explanation, precedence &  &  &  &  & nurse &  &  & time-tabling & \ref{a:LuoVLBM16} & n/a\\
\href{../works/LouieVNB14.pdf}{LouieVNB14}~\cite{LouieVNB14} & 7 & resource, periodic, activity, order, job, scheduling, task, machine &  & cycle &  & OPL & patient, robot &  &  &  & \ref{a:LouieVNB14} & n/a\\
\href{../works/LudwigKRBMS14.pdf}{LudwigKRBMS14}~\cite{LudwigKRBMS14} & 6 & task, precedence, scheduling, transportation, resource, periodic, order &  & cycle &  & OZ & automotive, robot &  & real-world & sweep & \ref{a:LudwigKRBMS14} & n/a\\
\href{../works/KameugneF13.pdf}{KameugneF13}~\cite{KameugneF13} & 21 & order, task, release-date &  & cumulative &  &  &  &  &  & not-first & \ref{a:KameugneF13} & n/a\\
\href{../works/HeinzSSW12.pdf}{HeinzSSW12}~\cite{HeinzSSW12} & 12 & explanation, constraint programming, inventory, order, task, constraint satisfaction, CP &  & bin-packing &  & Cplex & steel mill & steel industry, process industry & CSPlib, real-world & large neighborhood search, column generation & \ref{a:HeinzSSW12} & \ref{c:HeinzSSW12}\\
\href{../works/SchausHMCMD11.pdf}{SchausHMCMD11}~\cite{SchausHMCMD11} & 23 & stochastic, periodic, task, CP, CSP, constraint optimization, constraint programming, constraint logic programming, order & SCC & Element constraint, Cardinality constraint, bin-packing, GCC constraint &  &  & steel mill & steel industry & CSPlib, generated instance, benchmark & meta heuristic, large neighborhood search & \ref{a:SchausHMCMD11} & \ref{c:SchausHMCMD11}\\
\href{../works/AronssonBK09.pdf}{AronssonBK09}~\cite{AronssonBK09} & 13 & job-shop, transportation, CLP, job, order, CP, task, constraint programming &  & cumulative & Prolog & Cplex, CHIP & railway &  & real-world, real-life & sweep & \ref{a:AronssonBK09} & n/a\\
\href{../works/HentenryckM08.pdf}{HentenryckM08}~\cite{HentenryckM08} & 5 & CSP, CP, order, constraint programming &  & bin-packing &  &  & steel mill &  & CSPlib & large neighborhood search & \ref{a:HentenryckM08} & n/a\\
\href{../works/SchausD08.pdf}{SchausD08}~\cite{SchausD08} & 6 & order, task, preempt, precedence, preemptive, CSP, CP, constraint programming &  & IloPack, bin-packing, cycle, Reified constraint, Element constraint &  & Ilog Solver, OPL &  &  & real-life, benchmark & large neighborhood search & \ref{a:SchausD08} & n/a\\
\href{../works/DoRZ08.pdf}{DoRZ08}~\cite{DoRZ08} & 5 & order scheduling, distributed, resource, job, stochastic, scheduling, machine, order, make-span &  &  &  &  & robot &  & real-world, industrial partner &  & \ref{a:DoRZ08} & n/a\\
\href{../works/GarganiR07.pdf}{GarganiR07}~\cite{GarganiR07} & 13 & machine, resource, CP, inventory, order, constraint programming, CSP &  & Element constraint, bin-packing, Channeling constraint & C++ & OPL & steel mill & steel industry & real-life, CSPlib & large neighborhood search, column generation & \ref{a:GarganiR07} & n/a\\
\href{../works/ElhouraniDM07.pdf}{ElhouraniDM07}~\cite{ElhouraniDM07} & 6 & constraint optimization, constraint satisfaction, multi-agent, COP, distributed, task, buffer-capacity &  & cycle &  & OPL &  &  &  &  & \ref{a:ElhouraniDM07} & n/a\\
\href{../works/LiuJ06.pdf}{LiuJ06}~\cite{LiuJ06} & 5 & task, order, make-span, resource, multi-objective, scheduling &  & disjunctive, Disjunctive constraint, cycle &  &  &  &  &  &  & \ref{a:LiuJ06} & n/a\\
\href{../works/MaraveliasG04.pdf}{MaraveliasG04}~\cite{MaraveliasG04} & 20 &  &  &  &  & OZ &  &  &  &  & \ref{a:MaraveliasG04} & n/a\\
\href{../works/FrankK03.pdf}{FrankK03}~\cite{FrankK03} & 10 & stochastic, order, scheduling &  &  &  &  & astronomy, airport, aircraft, telescope &  & benchmark &  & \ref{a:FrankK03} & n/a\\
\href{../works/Layfield02.pdf}{Layfield02}~\cite{Layfield02} & 230 & CP &  &  & C  & OPL, OZ, Z3 &  &  &  &  & \ref{a:Layfield02} & n/a\\
\href{../works/Refalo00.pdf}{Refalo00}~\cite{Refalo00} & 15 & transportation, constraint programming, constraint satisfaction, CP, order, CLP &  & Element constraint, Cardinality constraint, disjunctive, Arithmetic constraint, Among constraint, cycle, circuit &  & CHIP, Ilog Solver, Cplex & hoist &  &  & quadratic programming & \ref{a:Refalo00} & n/a\\
\href{../works/BensanaLV99.pdf}{BensanaLV99}~\cite{BensanaLV99} & 7 & constraint satisfaction, constraint programming, order, CP, CSP, constraint optimization, explanation &  & cycle &  & Ilog Solver, Cplex & satellite, earth observation &  & benchmark &  & \ref{a:BensanaLV99} & \ref{c:BensanaLV99}\\
\href{../works/SmithBHW96.pdf}{SmithBHW96}~\cite{SmithBHW96} & 20 & constraint satisfaction, resource, order, constraint programming, CSP, task, CLP, constraint logic programming &  &  & C++ & OPL, Ilog Solver &  &  & real-life &  & \ref{a:SmithBHW96} & n/a\\
\href{../works/CrawfordB94.pdf}{CrawfordB94}~\cite{CrawfordB94} & 6 & order, machine, scheduling, resource, job-shop, distributed, job, task &  &  &  &  & operating room &  &  &  & \ref{a:CrawfordB94} & n/a\\
\href{../works/Hamscher91.pdf}{Hamscher91}~\cite{Hamscher91} & 6 & machine, order, CP, inventory & TMS &  & Lisp &  &  &  &  &  & \ref{a:Hamscher91} & n/a\\
\href{../works/Valdes87.pdf}{Valdes87}~\cite{Valdes87} & 5 & precedence, order, constraint satisfaction, task &  & circuit, disjunctive, cycle &  &  &  &  &  &  & \ref{a:Valdes87} & n/a\\
\href{../works/Rit86.pdf}{Rit86}~\cite{Rit86} & 6 & activity, scheduling, resource, job, order, precedence, task, job-shop &  & Disjunctive constraint, disjunctive &  &  &  &  &  &  & \ref{a:Rit86} & n/a\\
\href{../works/FoxAS82.pdf}{FoxAS82}~\cite{FoxAS82} & 4 & job, order, scheduling, distributed, precedence, job-shop, due-date, re-scheduling, tardiness, machine, task, resource &  &  &  &  & robot &  &  &  & \ref{a:FoxAS82} & n/a\\
\end{longtable}
}



\subsection{Works with Few Extracted Concepts}

These works do match only few concepts, they might be discussing other topics that should be included in the list of concepts, or they might be irrelevant, and included in the survey by mistake.

{\scriptsize
\begin{longtable}{>{\raggedright\arraybackslash}p{3cm}>{\raggedright\arraybackslash}p{4.5cm}>{\raggedright\arraybackslash}p{6.0cm}rrrp{2.5cm}rp{1cm}p{1cm}rr}
\rowcolor{white}\caption{Works with Low Feature Count (Total 20)}\\ \toprule
\rowcolor{white}\shortstack{Key\\Source} & Authors & Title (Colored by Open Access)& LC & Cite & Year & \shortstack{Conference\\/Journal\\/School} & Pages & \shortstack{Cites\\OC XR\\SC} & \shortstack{Refs\\OC\\XR} & b & c \\ \midrule\endhead
\bottomrule
\endfoot
MaraveliasG04 \href{https://doi.org/10.1007/978-3-540-24664-0_1}{MaraveliasG04} & \hyperref[auth:a381]{C. T. Maravelias}, \hyperref[auth:a382]{I. E. Grossmann} & Using {MILP} and {CP} for the Scheduling of Batch Chemical Processes & \href{../works/MaraveliasG04.pdf}{Yes} & \cite{MaraveliasG04} & 2004 & CPAIOR 2004 & 20 & 15 14 23 & 15 23 & \ref{b:MaraveliasG04} & n/a\\
Layfield02 \href{http://etheses.whiterose.ac.uk/1301/}{Layfield02} & \hyperref[auth:a670]{C. J. Layfield} & A constraint programming pre-processor for duty scheduling & \href{../works/Layfield02.pdf}{Yes} & \cite{Layfield02} & 2002 & University of Leeds, {UK} & 230 & 0 0 0 & 0 0 & \ref{b:Layfield02} & n/a\\
KameugneF13 \href{http://dx.doi.org/10.1007/s13226-013-0005-z}{KameugneF13} & \hyperref[auth:a10]{R. Kameugne}, \hyperref[auth:a130]{L. P. Fotso} & A cumulative not-first/not-last filtering algorithm in O(n 2log(n)) & \href{../works/KameugneF13.pdf}{Yes} & \cite{KameugneF13} & 2013 & Indian Journal of Pure and Applied Mathematics & 21 & 6 8 8 & 4 19 & \ref{b:KameugneF13} & n/a\\
Hamscher91 \href{http://www.aaai.org/Library/AAAI/1991/aaai91-079.php}{Hamscher91} & \hyperref[auth:a1276]{W. Hamscher} & {ACP:} Reason Maintenance and Inference Control for Constraint Propagation Over Intervals & \href{../works/Hamscher91.pdf}{Yes} & \cite{Hamscher91} & 1991 & AAAI 1991 & 6 & 0 0 0 & 0 0 & \ref{b:Hamscher91} & n/a\\
FeldmanG89 \href{http://ijcai.org/Proceedings/89-2/Papers/026.pdf}{FeldmanG89} & \hyperref[auth:a1436]{R. Feldman}, \hyperref[auth:a1437]{M. C. Golumbic} & Constraint Satisfiability Algorithms for Interactive Student Scheduling & \href{../works/FeldmanG89.pdf}{Yes} & \cite{FeldmanG89} & 1989 & IJCAI 1989 & 7 & 0 0 0 & 0 0 & \ref{b:FeldmanG89} & n/a\\
FalaschiGMP97 \href{https://doi.org/10.1006/inco.1997.2638}{FalaschiGMP97} & \hyperref[auth:a687]{M. Falaschi}, \hyperref[auth:a192]{M. Gabbrielli}, \hyperref[auth:a688]{K. Marriott}, \hyperref[auth:a689]{C. Palamidessi} & \cellcolor{gold!20}Constraint Logic Programming with Dynamic Scheduling: {A} Semantics Based on Closure Operators & \href{../works/FalaschiGMP97.pdf}{Yes} & \cite{FalaschiGMP97} & 1997 & Inf. Comput. & 27 & 10 10 12 & 9 15 & \ref{b:FalaschiGMP97} & n/a\\
HentenryckM08 \href{https://doi.org/10.1007/978-3-540-68155-7_41}{HentenryckM08} & \hyperref[auth:a148]{P. V. Hentenryck}, \hyperref[auth:a32]{L. Michel} & The Steel Mill Slab Design Problem Revisited & \href{../works/HentenryckM08.pdf}{Yes} & \cite{HentenryckM08} & 2008 & CPAIOR 2008 & 5 & 13 14 23 & 3 7 & \ref{b:HentenryckM08} & n/a\\
FrankK03 \href{http://www.aaai.org/Library/ICAPS/2003/icaps03-023.php}{FrankK03} & \hyperref[auth:a379]{J. Frank}, \hyperref[auth:a380]{E. K{\"{u}}rkl{\"{u}}} & SOFIA's Choice: Scheduling Observations for an Airborne Observatory & \href{../works/FrankK03.pdf}{Yes} & \cite{FrankK03} & 2003 & ICAPS 2003 & 10 & 0 0 0 & 0 0 & \ref{b:FrankK03} & n/a\\
Kumar03 \href{https://doi.org/10.1007/978-3-540-45193-8_45}{Kumar03} & \hyperref[auth:a286]{T. K. S. Kumar} & Incremental Computation of Resource-Envelopes in Producer-Consumer Models & \href{../works/Kumar03.pdf}{Yes} & \cite{Kumar03} & 2003 & CP 2003 & 15 & 4 4 7 & 2 6 & \ref{b:Kumar03} & n/a\\
AngelsmarkJ00 \href{https://doi.org/10.1007/3-540-45349-0_35}{AngelsmarkJ00} & \hyperref[auth:a295]{O. Angelsmark}, \hyperref[auth:a296]{P. Jonsson} & Some Observations on Durations, Scheduling and Allen's Algebra & \href{../works/AngelsmarkJ00.pdf}{Yes} & \cite{AngelsmarkJ00} & 2000 & CP 2000 & 5 & 1 1 6 & 9 17 & \ref{b:AngelsmarkJ00} & n/a\\
Touraivane95 \href{https://doi.org/10.1007/3-540-60299-2_41}{Touraivane95} & \hyperref[auth:a306]{Toura{\"{\i}}vane} & Constraint Programming and Industrial Applications & \href{../works/Touraivane95.pdf}{Yes} & \cite{Touraivane95} & 1995 & CP 1995 & 3 & 2 0 1 & 1 0 & \ref{b:Touraivane95} & n/a\\
ZhuS02 \href{https://doi.org/10.1007/3-540-47961-9_69}{ZhuS02} & \hyperref[auth:a674]{K. Q. Zhu}, \hyperref[auth:a675]{A. E. Santosa} & A Meeting Scheduling System Based on Open Constraint Programming & \href{../works/ZhuS02.pdf}{Yes} & \cite{ZhuS02} & 2002 & CAiSE 2002 & 5 & 0 0 0 & 5 7 & \ref{b:ZhuS02} & n/a\\
Hunsberger08 \href{https://doi.org/10.3233/978-1-58603-891-5-553}{Hunsberger08} & \hyperref[auth:a1270]{L. Hunsberger} & A Practical Temporal Constraint Management System for Real-Time Applications & \href{../works/Hunsberger08.pdf}{Yes} & \cite{Hunsberger08} & 2008 & ECAI 2008 & 5 & 0 0 1 & 0 0 & \ref{b:Hunsberger08} & n/a\\
Valdes87 \href{http://www.aaai.org/Library/AAAI/1987/aaai87-046.php}{Valdes87} & \hyperref[auth:a1273]{R. E. Vald{\'{e}}s-P{\'{e}}rez} & The Satisfiability of Temporal Constraint Networks & \href{../works/Valdes87.pdf}{Yes} & \cite{Valdes87} & 1987 & AAAI 1987 & 5 & 0 0 0 & 0 0 & \ref{b:Valdes87} & n/a\\
ElhouraniDM07 \href{http://www.aaai.org/Library/AAAI/2007/aaai07-213.php}{ElhouraniDM07} & \hyperref[auth:a1343]{T. Elhourani}, \hyperref[auth:a1344]{N. Denny}, \hyperref[auth:a1345]{M. M. Marefat} & A Distributed Constraint Optimization Solution to the {P2P} Video Streaming Problem & \href{../works/ElhouraniDM07.pdf}{Yes} & \cite{ElhouraniDM07} & 2007 & AAAI 2007 & 6 & 0 0 0 & 0 0 & \ref{b:ElhouraniDM07} & n/a\\
ZhangLS12 \href{https://doi.org/10.1109/CIT.2012.96}{ZhangLS12} & \hyperref[auth:a611]{X. Zhang}, \hyperref[auth:a612]{Z. Lv}, \hyperref[auth:a613]{X. Song} & Model and Solution for Hot Strip Rolling Scheduling Problem Based on Constraint Programming Method & \href{../works/ZhangLS12.pdf}{Yes} & \cite{ZhangLS12} & 2012 & CIT 2012 & 4 & 1 1 1 & 3 9 & \ref{b:ZhangLS12} & n/a\\
QuSN06 \href{https://doi.org/10.1109/ISSOC.2006.321973}{QuSN06} & \hyperref[auth:a651]{Y. Qu}, \hyperref[auth:a652]{J.-P. Soininen}, \hyperref[auth:a653]{J. Nurmi} & Using Constraint Programming to Achieve Optimal Prefetch Scheduling for Dependent Tasks on Run-Time Reconfigurable Devices & \href{../works/QuSN06.pdf}{Yes} & \cite{QuSN06} & 2006 & SoC 2006 & 4 & 2 2 5 & 5 14 & \ref{b:QuSN06} & n/a\\
LiuJ06 \href{https://doi.org/10.1007/11801603_92}{LiuJ06} & \hyperref[auth:a654]{Y. Liu}, \hyperref[auth:a655]{Y. Jiang} & {LP-TPOP:} Integrating Planning and Scheduling Through Constraint Programming & \href{../works/LiuJ06.pdf}{Yes} & \cite{LiuJ06} & 2006 & PRICAI 2006 & 5 & 0 0 1 & 0 0 & \ref{b:LiuJ06} & n/a\\
HebrardALLCMR22 \href{https://doi.org/10.24963/ijcai.2022/643}{HebrardALLCMR22} & \hyperref[auth:a1]{E. Hebrard}, \hyperref[auth:a6]{C. Artigues}, \hyperref[auth:a3]{P. Lopez}, \hyperref[auth:a785]{A. Lusson}, \hyperref[auth:a786]{S. A. Chien}, \hyperref[auth:a787]{A. Maillard}, \hyperref[auth:a788]{G. R. Rabideau} & An Efficient Approach to Data Transfer Scheduling for Long Range Space Exploration & \href{../works/HebrardALLCMR22.pdf}{Yes} & \cite{HebrardALLCMR22} & 2022 & IJCAI 2022 & 7 & 0 0 0 & 0 0 & \ref{b:HebrardALLCMR22} & n/a\\
DincbasS91 \href{}{DincbasS91} & \hyperref[auth:a717]{M. Dincbas}, \hyperref[auth:a17]{H. Simonis} & Apache-a constraint based, automated stand allocation system & \href{../works/DincbasS91.pdf}{Yes} & \cite{DincbasS91} & 1991 & ASTAIR 1991 & 13 & 0 0 0 & 0 0 & \ref{b:DincbasS91} & n/a\\
\end{longtable}
}


{\scriptsize
\begin{longtable}{>{\raggedright\arraybackslash}p{3cm}r>{\raggedright\arraybackslash}p{1.0cm}>{\raggedright\arraybackslash}p{ 1.50cm}>{\raggedright\arraybackslash}p{ 1.50cm}>{\raggedright\arraybackslash}p{ 1.50cm}>{\raggedright\arraybackslash}p{ 1.50cm}>{\raggedright\arraybackslash}p{ 1.50cm}>{\raggedright\arraybackslash}p{ 1.50cm}>{\raggedright\arraybackslash}p{ 1.50cm}>{\raggedright\arraybackslash}p{ 1.50cm}>{\raggedright\arraybackslash}p{ 1.50cm}>{\raggedright\arraybackslash}p{ 1.50cm}}
\rowcolor{white}\caption{Features of Works with Low Feature Count}\\ \toprule
\rowcolor{white}Work/Title & Pages & Relevance & Scheduling& CP& Concepts& Classification& Constraints& ApplicationAreas& Industries& CPSystems& Benchmarks& Algorithms\\ \midrule\endhead
\bottomrule
\endfoot
\href{../works/MaraveliasG04.pdf}{MaraveliasG04}~\cite{MaraveliasG04} \hyperref[detail:MaraveliasG04]{Details} Using {MILP} and {CP} for the Scheduling of Batch Chemical Processes & 20 & \noindent{}\textbf{1.00} \textbf{1.00} \textcolor{black!50}{0.00} &  &  &  &  &  &  &  & \textcolor{black}{OZ} &  & \\
\href{../works/Layfield02.pdf}{Layfield02}~\cite{Layfield02} \hyperref[detail:Layfield02]{Details} A constraint programming pre-processor for duty scheduling & 230 & \noindent{}\textbf{1.00} \textbf{1.00} \textcolor{black!50}{0.00} &  & \textcolor{blue}{CP} &  &  &  &  &  & \textcolor{blue}{OZ}, \textcolor{black!40}{OPL}, \textcolor{black!40}{Z3} &  & \\
\href{../works/KameugneF13.pdf}{KameugneF13}~\cite{KameugneF13} \hyperref[detail:KameugneF13]{Details} A cumulative not-first/not-last filtering algorithm in O(n 2log(n)) & 21 & \noindent{}\textcolor{black!50}{0.00} \textcolor{black!50}{0.00} \textcolor{black!50}{0.00} & \textcolor{black}{task}, \textcolor{black!40}{order} &  & \textcolor{black!40}{release-date} &  & \textcolor{black!40}{cumulative} &  &  &  &  & \textcolor{black!40}{not-first}\\
\href{../works/Hamscher91.pdf}{Hamscher91}~\cite{Hamscher91} \hyperref[detail:Hamscher91]{Details} {ACP:} Reason Maintenance and Inference Control for Constraint Propagation Over Intervals & 6 & \noindent{}\textcolor{black!50}{0.00} \textcolor{black!50}{0.00} \textcolor{black!50}{0.02} & \textcolor{black!40}{order}, \textcolor{black!40}{machine} & \textcolor{blue}{propagation}, \textcolor{black!40}{constraint propagation} & \textcolor{black}{inventory} & \textcolor{blue}{TMS} &  &  &  &  &  & \\
\href{../works/FalaschiGMP97.pdf}{FalaschiGMP97}~\cite{FalaschiGMP97} \hyperref[detail:FalaschiGMP97]{Details} Constraint Logic Programming with Dynamic Scheduling: {A} Semantics Based on Closure Operators & 27 & \noindent{}\textbf{1.00} \textbf{1.00} \textbf{1.33} & \textcolor{blue}{scheduling}, \textcolor{blue}{order} & \textcolor{blue}{CLP}, \textcolor{black}{constraint logic programming}, \textcolor{black}{constraint programming} & \textcolor{black!40}{Unsatisfiable} &  & \textcolor{black!40}{Arithmetic constraint} &  &  &  &  & \\
\href{../works/FeldmanG89.pdf}{FeldmanG89}~\cite{FeldmanG89} \hyperref[detail:FeldmanG89]{Details} Constraint Satisfiability Algorithms for Interactive Student Scheduling & 7 & \noindent{}\textcolor{black!50}{0.00} \textcolor{black!50}{0.00} 0.75 & \textcolor{blue}{order}, \textcolor{blue}{scheduling} & \textcolor{blue}{CSP}, \textcolor{black!40}{constraint satisfaction}, \textcolor{black!40}{propagation}, \textcolor{black!40}{constraint propagation} & \textcolor{black!40}{Infeasible} &  &  &  &  &  &  & \\
\href{../works/HentenryckM08.pdf}{HentenryckM08}~\cite{HentenryckM08} \hyperref[detail:HentenryckM08]{Details} The Steel Mill Slab Design Problem Revisited & 5 & \noindent{}\textcolor{black!50}{0.00} \textcolor{black!50}{0.00} \textcolor{black!50}{0.15} & \textcolor{blue}{order} & \textcolor{black}{CSP}, \textcolor{black}{CP}, \textcolor{black}{constraint programming} &  &  & \textcolor{black!40}{bin-packing} & \textcolor{blue}{steel mill} &  &  & \textcolor{black!40}{CSPlib} & \textcolor{blue}{large neighborhood search}\\
\href{../works/FrankK03.pdf}{FrankK03}~\cite{FrankK03} \hyperref[detail:FrankK03]{Details} SOFIA's Choice: Scheduling Observations for an Airborne Observatory & 10 & \noindent{}\textcolor{black!50}{0.00} \textcolor{black!50}{0.00} \textcolor{black!50}{0.00} & \textcolor{blue}{scheduling}, \textcolor{blue}{order} &  & \textcolor{black}{stochastic} &  &  & \textcolor{blue}{aircraft}, \textcolor{blue}{airport}, \textcolor{blue}{telescope}, \textcolor{black!40}{astronomy} &  &  & \textcolor{black!40}{benchmark} & \\
\href{../works/Kumar03.pdf}{Kumar03}~\cite{Kumar03} \hyperref[detail:Kumar03]{Details} Incremental Computation of Resource-Envelopes in Producer-Consumer Models & 15 & \noindent{}\textcolor{black!50}{0.00} \textcolor{black!50}{0.00} \textcolor{black!50}{0.08} & \textcolor{blue}{resource}, \textcolor{blue}{scheduling}, \textcolor{black!40}{order}, \textcolor{black!40}{activity} & \textcolor{black!40}{CP} & \textcolor{blue}{producer/consumer} &  & \textcolor{blue}{cycle} &  &  &  &  & \textcolor{black!40}{bi-partite matching}, \textcolor{black!40}{max-flow}\\
\href{../works/Touraivane95.pdf}{Touraivane95}~\cite{Touraivane95} \hyperref[detail:Touraivane95]{Details} Constraint Programming and Industrial Applications & 3 & \noindent{}\textcolor{black!50}{0.00} \textcolor{black!50}{0.00} \textcolor{black!50}{0.11} & \textcolor{black}{scheduling}, \textcolor{black}{order}, \textcolor{black!40}{task} & \textcolor{blue}{CLP}, \textcolor{black}{constraint programming}, \textcolor{black!40}{CP}, \textcolor{black!40}{constraint logic programming} &  &  &  & \textcolor{black}{crew-scheduling} &  &  & \textcolor{black!40}{real-life} & \\
\href{../works/ZhuS02.pdf}{ZhuS02}~\cite{ZhuS02} \hyperref[detail:ZhuS02]{Details} A Meeting Scheduling System Based on Open Constraint Programming & 5 & \noindent{}\textbf{1.00} \textbf{1.00} 0.21 & \textcolor{blue}{scheduling}, \textcolor{blue}{resource}, \textcolor{black!40}{activity} & \textcolor{blue}{CLP}, \textcolor{black}{constraint programming}, \textcolor{black!40}{constraint logic programming}, \textcolor{black!40}{CP} & \textcolor{black}{distributed} &  &  & \textcolor{blue}{meeting scheduling} &  &  &  & \\
\href{../works/HebrardALLCMR22.pdf}{HebrardALLCMR22}~\cite{HebrardALLCMR22} \hyperref[detail:HebrardALLCMR22]{Details} An Efficient Approach to Data Transfer Scheduling for Long Range Space Exploration & 7 & \noindent{}\textcolor{black!50}{0.00} \textcolor{black!50}{0.00} \textcolor{black!50}{0.00} & \textcolor{blue}{order}, \textcolor{black}{scheduling}, \textcolor{black!40}{activity} &  & \textcolor{black!40}{Unsatisfiable} &  & \textcolor{black!40}{cumulative} & \textcolor{black}{round-robin}, \textcolor{black!40}{deep space} &  & \textcolor{black!40}{Claire} &  & \textcolor{black!40}{sweep}\\
\href{../works/Hunsberger08.pdf}{Hunsberger08}~\cite{Hunsberger08} \hyperref[detail:Hunsberger08]{Details} A Practical Temporal Constraint Management System for Real-Time Applications & 5 & \noindent{}\textcolor{black!50}{0.00} \textcolor{black!50}{0.00} 0.25 & \textcolor{blue}{scheduling}, \textcolor{black}{task}, \textcolor{black!40}{order} & \textcolor{blue}{propagation}, \textcolor{blue}{constraint propagation} & \textcolor{black}{multi-agent}, \textcolor{black!40}{distributed} &  & \textcolor{black}{cycle} &  &  &  & \textcolor{black!40}{real-world} & \\
\href{../works/ElhouraniDM07.pdf}{ElhouraniDM07}~\cite{ElhouraniDM07} \hyperref[detail:ElhouraniDM07]{Details} A Distributed Constraint Optimization Solution to the {P2P} Video Streaming Problem & 6 & \noindent{}\textcolor{black!50}{0.00} \textcolor{black!50}{0.00} \textcolor{black!50}{0.01} & \textcolor{black!40}{task} & \textcolor{black}{constraint optimization}, \textcolor{black}{constraint satisfaction}, \textcolor{black!40}{COP} & \textcolor{blue}{distributed}, \textcolor{black!40}{multi-agent}, \textcolor{black!40}{buffer-capacity} &  & \textcolor{blue}{cycle} &  &  & \textcolor{black!40}{OPL} &  & \\
\href{../works/AngelsmarkJ00.pdf}{AngelsmarkJ00}~\cite{AngelsmarkJ00} \hyperref[detail:AngelsmarkJ00]{Details} Some Observations on Durations, Scheduling and Allen's Algebra & 5 & \noindent{}\textcolor{black!50}{0.00} \textcolor{black!50}{0.00} \textcolor{black!50}{0.05} & \textcolor{blue}{scheduling}, \textcolor{black}{task}, \textcolor{black!40}{resource}, \textcolor{black!40}{job}, \textcolor{black!40}{order} & \textcolor{black!40}{propagation}, \textcolor{black!40}{constraint satisfaction}, \textcolor{black!40}{constraint propagation}, \textcolor{black!40}{CP} & \textcolor{black!40}{job-shop} &  &  &  &  &  &  & \\
\href{../works/QuSN06.pdf}{QuSN06}~\cite{QuSN06} \hyperref[detail:QuSN06]{Details} Using Constraint Programming to Achieve Optimal Prefetch Scheduling for Dependent Tasks on Run-Time Reconfigurable Devices & 4 & \noindent{}\textbf{2.00} \textbf{2.00} \textbf{2.21} & \textcolor{blue}{task}, \textcolor{blue}{scheduling}, \textcolor{blue}{resource} & \textcolor{blue}{CP}, \textcolor{blue}{constraint programming}, \textcolor{black!40}{CSP} & \textcolor{black!40}{distributed}, \textcolor{black!40}{precedence} &  & \textcolor{black!40}{circuit} &  &  & \textcolor{black}{SICStus} &  & \\
\href{../works/LiuJ06.pdf}{LiuJ06}~\cite{LiuJ06} \hyperref[detail:LiuJ06]{Details} {LP-TPOP:} Integrating Planning and Scheduling Through Constraint Programming & 5 & \noindent{}\textbf{1.00} \textbf{1.00} \textcolor{black!50}{0.01} & \textcolor{blue}{scheduling}, \textcolor{black}{order}, \textcolor{black!40}{task}, \textcolor{black!40}{resource} & \textcolor{black!40}{propagation} & \textcolor{black!40}{make-span}, \textcolor{black!40}{multi-objective} &  & \textcolor{black}{cycle}, \textcolor{black!40}{disjunctive}, \textcolor{black!40}{Disjunctive constraint} &  &  &  &  & \\
\href{../works/DincbasS91.pdf}{DincbasS91}~\cite{DincbasS91} \hyperref[detail:DincbasS91]{Details} Apache-a constraint based, automated stand allocation system & 13 & \noindent{}\textcolor{black!50}{0.00} \textcolor{black!50}{0.00} \textcolor{black!50}{0.12} & \textcolor{blue}{scheduling}, \textcolor{blue}{resource}, \textcolor{black!40}{task} & \textcolor{black}{constraint logic programming}, \textcolor{black!40}{constraint programming} & \textcolor{black}{re-scheduling}, \textcolor{black!40}{online scheduling} &  &  & \textcolor{blue}{airport}, \textcolor{black!40}{aircraft} &  & \textcolor{blue}{CHIP} &  & \\
\href{../works/Valdes87.pdf}{Valdes87}~\cite{Valdes87} \hyperref[detail:Valdes87]{Details} The Satisfiability of Temporal Constraint Networks & 5 & \noindent{}\textcolor{black!50}{0.00} \textcolor{black!50}{0.00} \textcolor{black!50}{0.05} & \textcolor{blue}{order}, \textcolor{black!40}{task} & \textcolor{black!40}{constraint satisfaction}, \textcolor{black!40}{constraint propagation}, \textcolor{black!40}{propagation} & \textcolor{black}{Unsatisfiable}, \textcolor{black!40}{precedence} &  & \textcolor{blue}{disjunctive}, \textcolor{black!40}{circuit}, \textcolor{black!40}{cycle} &  &  &  &  & \\
\href{../works/YoshikawaKNW94.pdf}{YoshikawaKNW94}~\cite{YoshikawaKNW94} \hyperref[detail:YoshikawaKNW94]{Details} A Constraint-Based Approach to High-School Timetabling Problems: {A} Case Study & 6 & \noindent{}\textcolor{black!50}{0.00} \textcolor{black!50}{0.00} \textcolor{black!50}{0.02} & \textcolor{blue}{scheduling}, \textcolor{blue}{order}, \textcolor{black!40}{job}, \textcolor{black!40}{task} & \textcolor{black!40}{CSP} &  &  &  & \textcolor{blue}{high school timetabling} &  &  & \textcolor{black}{real-world} & \textcolor{blue}{time-tabling}, \textcolor{black!40}{genetic algorithm}, \textcolor{black!40}{simulated annealing}\\
\href{../works/FortinZDF05.pdf}{FortinZDF05}~\cite{FortinZDF05} \hyperref[detail:FortinZDF05]{Details} Interval Analysis in Scheduling & 15 & \noindent{}\textcolor{black!50}{0.00} \textcolor{black!50}{0.00} \textcolor{black!50}{0.11} & \textcolor{blue}{task}, \textcolor{blue}{activity}, \textcolor{blue}{scheduling}, \textcolor{black}{order}, \textcolor{black!40}{resource} & \textcolor{black!40}{CP} & \textcolor{blue}{make-span}, \textcolor{black}{precedence}, \textcolor{black!40}{stochastic}, \textcolor{black!40}{temporal constraint reasoning} & \textcolor{black!40}{psplib} &  &  &  &  &  & \\
\href{../works/ChapadosJR11.pdf}{ChapadosJR11}~\cite{ChapadosJR11} \hyperref[detail:ChapadosJR11]{Details} Retail Store Workforce Scheduling by Expected Operating Income Maximization & 6 & \noindent{}\textcolor{black!50}{0.00} \textcolor{black!50}{0.00} \textcolor{black!50}{0.03} & \textcolor{black}{scheduling}, \textcolor{black}{activity}, \textcolor{black!40}{order}, \textcolor{black!40}{task} & \textcolor{black}{constraint programming} &  &  & \textcolor{black!40}{cycle}, \textcolor{black!40}{cumulative} &  & \textcolor{black!40}{retail industry} & \textcolor{black!40}{OPL} &  & \textcolor{black!40}{column generation}, \textcolor{black!40}{time-tabling}\\
\href{../works/ZhangLS12.pdf}{ZhangLS12}~\cite{ZhangLS12} \hyperref[detail:ZhangLS12]{Details} Model and Solution for Hot Strip Rolling Scheduling Problem Based on Constraint Programming Method & 4 & \noindent{}\textbf{1.00} \textbf{1.00} 0.97 & \textcolor{blue}{scheduling}, \textcolor{blue}{order} & \textcolor{blue}{CP}, \textcolor{blue}{constraint programming}, \textcolor{blue}{constraint satisfaction}, \textcolor{black}{CSP}, \textcolor{black}{propagation}, \textcolor{black!40}{constraint propagation} & \textcolor{black}{cmax} &  &  &  &  &  &  & \textcolor{black!40}{ant colony}, \textcolor{black!40}{time-tabling}\\
\href{../works/BandaSC11.pdf}{BandaSC11}~\cite{BandaSC11} \hyperref[detail:BandaSC11]{Details} Solving Talent Scheduling with Dynamic Programming & 18 & \noindent{}\textcolor{black!50}{0.00} \textcolor{black!50}{0.00} 0.79 & \textcolor{blue}{scheduling}, \textcolor{blue}{order}, \textcolor{black!40}{task} & \textcolor{black}{constraint programming}, \textcolor{black!40}{constraint optimization}, \textcolor{black!40}{CP} & \textcolor{black!40}{precedence} &  &  &  &  & \textcolor{black!40}{Ilog Solver} & \textcolor{blue}{benchmark}, \textcolor{black}{CSPlib}, \textcolor{black!40}{random instance} & \\
\href{../works/CrawfordB94.pdf}{CrawfordB94}~\cite{CrawfordB94} \hyperref[detail:CrawfordB94]{Details} Experimental Results on the Application of Satisfiability Algorithms to Scheduling Problems & 6 & \noindent{}\textcolor{black!50}{0.00} \textcolor{black!50}{0.00} 0.37 & \textcolor{blue}{scheduling}, \textcolor{blue}{order}, \textcolor{blue}{job}, \textcolor{blue}{resource}, \textcolor{black}{machine}, \textcolor{black!40}{task} & \textcolor{blue}{propagation} & \textcolor{black}{job-shop}, \textcolor{black!40}{Over-constrained}, \textcolor{black!40}{distributed} &  &  & \textcolor{black!40}{operating room} &  &  &  & \\
\href{../works/BensanaLV99.pdf}{BensanaLV99}~\cite{BensanaLV99} \hyperref[detail:BensanaLV99]{Details} Earth Observation Satellite Management & 7 & \noindent{}\textcolor{black!50}{0.00} \textcolor{black!50}{0.00} \textcolor{black!50}{0.07} & \textcolor{blue}{order} & \textcolor{black}{CSP}, \textcolor{black}{constraint satisfaction}, \textcolor{black!40}{constraint programming}, \textcolor{black!40}{constraint optimization} & \textcolor{black!40}{explanation} &  & \textcolor{black!40}{cycle} & \textcolor{blue}{satellite}, \textcolor{black}{earth observation} &  & \textcolor{black!40}{Ilog Solver}, \textcolor{black!40}{Cplex} & \textcolor{blue}{benchmark} & \\
\href{../works/Hooker17.pdf}{Hooker17}~\cite{Hooker17} \hyperref[detail:Hooker17]{Details} Job Sequencing Bounds from Decision Diagrams & 14 & \noindent{}\textcolor{black!50}{0.00} \textcolor{black!50}{0.00} 0.35 & \textcolor{blue}{job}, \textcolor{black!40}{order}, \textcolor{black!40}{scheduling}, \textcolor{black!40}{resource} & \textcolor{black}{CP}, \textcolor{black!40}{constraint programming} & \textcolor{black}{tardiness}, \textcolor{black!40}{Infeasible}, \textcolor{black!40}{due-date} &  & \textcolor{black!40}{circuit} &  &  &  & \textcolor{black!40}{benchmark}, \textcolor{black!40}{random instance} & \\
\href{../works/KletzanderM17.pdf}{KletzanderM17}~\cite{KletzanderM17} \hyperref[detail:KletzanderM17]{Details} A Multi-stage Simulated Annealing Algorithm for the Torpedo Scheduling Problem & 15 & \noindent{}\textcolor{black!50}{0.00} \textcolor{black!50}{0.00} \textcolor{black!50}{0.00} & \textcolor{blue}{scheduling}, \textcolor{blue}{order}, \textcolor{black!40}{machine}, \textcolor{black!40}{resource} &  & \textcolor{black!40}{transportation} & \textcolor{black!40}{parallel machine} &  & \textcolor{blue}{torpedo} & \textcolor{black!40}{steel industry} &  &  & \textcolor{blue}{simulated annealing}, \textcolor{black!40}{genetic algorithm}, \textcolor{black!40}{neural network}, \textcolor{black!40}{meta heuristic}\\
\href{../works/BofillGSV15.pdf}{BofillGSV15}~\cite{BofillGSV15} \hyperref[detail:BofillGSV15]{Details} MaxSAT-Based Scheduling of {B2B} Meetings & 9 & \noindent{}\textcolor{black!50}{0.00} \textcolor{black!50}{0.00} \textcolor{black!50}{0.16} & \textcolor{blue}{scheduling}, \textcolor{black}{order}, \textcolor{black!40}{machine} & \textcolor{blue}{CP}, \textcolor{black!40}{constraint programming} &  &  & \textcolor{black}{Cardinality constraint}, \textcolor{black!40}{Channeling constraint} & \textcolor{black!40}{meeting scheduling}, \textcolor{black!40}{high school timetabling} &  & \textcolor{black}{Cplex} & \textcolor{black!40}{industrial instance} & \textcolor{black}{time-tabling}\\
\href{../works/ZibranR11.pdf}{ZibranR11}~\cite{ZibranR11} \hyperref[detail:ZibranR11]{Details} Conflict-Aware Optimal Scheduling of Code Clone Refactoring: {A} Constraint Programming Approach & 4 & \noindent{}\textbf{1.00} \textbf{1.00} \textcolor{black!50}{0.17} & \textcolor{blue}{scheduling}, \textcolor{black}{order}, \textcolor{black}{activity} & \textcolor{black}{constraint programming}, \textcolor{black}{CP}, \textcolor{black!40}{constraint satisfaction} & \textcolor{black!40}{Infeasible} &  &  &  &  & \textcolor{black!40}{Cplex}, \textcolor{black!40}{OPL} &  & \textcolor{black!40}{simulated annealing}, \textcolor{black!40}{genetic algorithm}, \textcolor{black!40}{meta heuristic}\\
\end{longtable}
}



\clearpage
\section{Unmatched Concepts}

This section lists those concepts for which no matches were found. The most likely cause is a mistake in the regular expression used to find the concept, but it is also possible that some concept simply is not mentioned in any of the documents. 

{\scriptsize
\begin{longtable}{lp{10cm}rr}
\caption{Unmatched Concepts}\\ \toprule
Type & Name & CaseSensitive & Revision\\ \midrule
\endhead
\bottomrule
\endfoot
ProgLanguages & Julia &  & 0\\Industries & steel making industry &  & 0\\ApplicationAreas & datacentre &  & 0\\ApplicationAreas & day-ahead market &  & 0\\ApplicationAreas & deep space &  & 0\\ApplicationAreas & ship building &  & 0\\ApplicationAreas & vaccine &  & 0\\Classification & Modified Generalized Assignment Problem &  & 0\\Classification & PP-MS-MMRCPSP & Y & 1\\Classification & Pre-emptive Job-Shop scheduling Problem &  & 0\\Classification & Resource-constrained Project Scheduling Problem with Discounted Cashflow &  & 0\\Classification & SMSDP & Y & 1\\Classification & Steel-making and continuous casting &  & 0\\Concepts & Allen's algebra &  & 0\\Concepts & make to stock &  & 0\\\end{longtable}
}



\clearpage
\section{Works by Author}

\subsection{Works by J. Christopher Beck}
\label{sec:a89}
{\scriptsize
\begin{longtable}{>{\raggedright\arraybackslash}p{3cm}>{\raggedright\arraybackslash}p{6cm}>{\raggedright\arraybackslash}p{6.5cm}rrrp{2.5cm}rrrrr}
\rowcolor{white}\caption{Works from bibtex (Total 46)}\\ \toprule
\rowcolor{white}Key & Authors & Title & LC & Cite & Year & \shortstack{Conference\\/Journal} & Pages & \shortstack{Nr\\Cites} & \shortstack{Nr\\Refs} & b & c \\ \midrule\endhead
\bottomrule
\endfoot
LuoB22 \href{https://doi.org/10.1007/978-3-031-08011-1\_17}{LuoB22} & \hyperref[auth:a754]{Yiqing L. Luo}, \hyperref[auth:a89]{J. Christopher Beck} & Packing by Scheduling: Using Constraint Programming to Solve a Complex 2D Cutting Stock Problem & \href{works/LuoB22.pdf}{Yes} & \cite{LuoB22} & 2022 & CPAIOR 2022 & 17 & 0 & 28 & \ref{b:LuoB22} & \ref{c:LuoB22}\\
ZhangBB22 \href{https://ojs.aaai.org/index.php/ICAPS/article/view/19826}{ZhangBB22} & \hyperref[auth:a808]{J. Zhang}, \hyperref[auth:a809]{Giovanni Lo Bianco}, \hyperref[auth:a89]{J. Christopher Beck} & Solving Job-Shop Scheduling Problems with QUBO-Based Specialized Hardware & \href{works/ZhangBB22.pdf}{Yes} & \cite{ZhangBB22} & 2022 & ICAPS 2022 & 9 & 0 & 0 & \ref{b:ZhangBB22} & \ref{c:ZhangBB22}\\
TangB20 \href{https://doi.org/10.1007/978-3-030-58942-4\_28}{TangB20} & \hyperref[auth:a88]{Tanya Y. Tang}, \hyperref[auth:a89]{J. Christopher Beck} & {CP} and Hybrid Models for Two-Stage Batching and Scheduling & \href{works/TangB20.pdf}{Yes} & \cite{TangB20} & 2020 & CPAIOR 2020 & 16 & 6 & 12 & \ref{b:TangB20} & \ref{c:TangB20}\\
TranPZLDB18 \href{https://doi.org/10.1007/s10951-017-0537-x}{TranPZLDB18} & \hyperref[auth:a810]{Tony T. Tran}, \hyperref[auth:a811]{M. Padmanabhan}, \hyperref[auth:a812]{Peter Yun Zhang}, \hyperref[auth:a813]{H. Li}, \hyperref[auth:a814]{Douglas G. Down}, \hyperref[auth:a89]{J. Christopher Beck} & Multi-stage resource-aware scheduling for data centers with heterogeneous servers & \href{works/TranPZLDB18.pdf}{Yes} & \cite{TranPZLDB18} & 2018 & J. Sched. & 17 & 8 & 26 & \ref{b:TranPZLDB18} & \ref{c:TranPZLDB18}\\
CohenHB17 \href{https://doi.org/10.1007/978-3-319-66263-3\_10}{CohenHB17} & \hyperref[auth:a816]{E. Cohen}, \hyperref[auth:a817]{G. Huang}, \hyperref[auth:a89]{J. Christopher Beck} & {(I} Can Get) Satisfaction: Preference-Based Scheduling for Concert-Goers at Multi-venue Music Festivals & \href{works/CohenHB17.pdf}{Yes} & \cite{CohenHB17} & 2017 & SAT 2017 & 17 & 1 & 12 & \ref{b:CohenHB17} & \ref{c:CohenHB17}\\
TranVNB17 \href{https://doi.org/10.1613/jair.5306}{TranVNB17} & \hyperref[auth:a810]{Tony T. Tran}, \hyperref[auth:a815]{Tiago Stegun Vaquero}, \hyperref[auth:a209]{G. Nejat}, \hyperref[auth:a89]{J. Christopher Beck} & Robots in Retirement Homes: Applying Off-the-Shelf Planning and Scheduling to a Team of Assistive Robots & \href{works/TranVNB17.pdf}{Yes} & \cite{TranVNB17} & 2017 & J. Artif. Intell. Res. & 68 & 12 & 0 & \ref{b:TranVNB17} & \ref{c:TranVNB17}\\
TranVNB17a \href{https://doi.org/10.24963/ijcai.2017/726}{TranVNB17a} & \hyperref[auth:a810]{Tony T. Tran}, \hyperref[auth:a815]{Tiago Stegun Vaquero}, \hyperref[auth:a209]{G. Nejat}, \hyperref[auth:a89]{J. Christopher Beck} & Robots in Retirement Homes: Applying Off-the-Shelf Planning and Scheduling to a Team of Assistive Robots (Extended Abstract) & \href{works/TranVNB17a.pdf}{Yes} & \cite{TranVNB17a} & 2017 & IJCAI 2017 & 5 & 1 & 0 & \ref{b:TranVNB17a} & \ref{c:TranVNB17a}\\
BoothNB16 \href{https://doi.org/10.1007/978-3-319-44953-1\_34}{BoothNB16} & \hyperref[auth:a208]{Kyle E. C. Booth}, \hyperref[auth:a209]{G. Nejat}, \hyperref[auth:a89]{J. Christopher Beck} & A Constraint Programming Approach to Multi-Robot Task Allocation and Scheduling in Retirement Homes & \href{works/BoothNB16.pdf}{Yes} & \cite{BoothNB16} & 2016 & CP 2016 & 17 & 21 & 24 & \ref{b:BoothNB16} & \ref{c:BoothNB16}\\
KuB16 \href{https://doi.org/10.1016/j.cor.2016.04.006}{KuB16} & \hyperref[auth:a336]{W. Ku}, \hyperref[auth:a89]{J. Christopher Beck} & Mixed Integer Programming models for job shop scheduling: {A} computational analysis & No & \cite{KuB16} & 2016 & Comput. Oper. Res. & 9 & 119 & 17 & No & \ref{c:KuB16}\\
LuoVLBM16 \href{http://www.aaai.org/ocs/index.php/KR/KR16/paper/view/12909}{LuoVLBM16} & \hyperref[auth:a824]{R. Luo}, \hyperref[auth:a825]{Richard Anthony Valenzano}, \hyperref[auth:a826]{Y. Li}, \hyperref[auth:a89]{J. Christopher Beck}, \hyperref[auth:a827]{Sheila A. McIlraith} & Using Metric Temporal Logic to Specify Scheduling Problems & \href{works/LuoVLBM16.pdf}{Yes} & \cite{LuoVLBM16} & 2016 & KR 2016 & 4 & 0 & 0 & \ref{b:LuoVLBM16} & \ref{c:LuoVLBM16}\\
TranAB16 \href{https://doi.org/10.1287/ijoc.2015.0666}{TranAB16} & \hyperref[auth:a810]{Tony T. Tran}, \hyperref[auth:a818]{A. Araujo}, \hyperref[auth:a89]{J. Christopher Beck} & Decomposition Methods for the Parallel Machine Scheduling Problem with Setups & No & \cite{TranAB16} & 2016 & {INFORMS} J. Comput. & 13 & 72 & 28 & No & \ref{c:TranAB16}\\
TranDRFWOVB16 \href{https://doi.org/10.1609/socs.v7i1.18390}{TranDRFWOVB16} & \hyperref[auth:a810]{Tony T. Tran}, \hyperref[auth:a820]{M. Do}, \hyperref[auth:a821]{Eleanor Gilbert Rieffel}, \hyperref[auth:a383]{J. Frank}, \hyperref[auth:a819]{Z. Wang}, \hyperref[auth:a822]{B. O'Gorman}, \hyperref[auth:a823]{D. Venturelli}, \hyperref[auth:a89]{J. Christopher Beck} & A Hybrid Quantum-Classical Approach to Solving Scheduling Problems & \href{works/TranDRFWOVB16.pdf}{Yes} & \cite{TranDRFWOVB16} & 2016 & SOCS 2016 & 9 & 3 & 0 & \ref{b:TranDRFWOVB16} & \ref{c:TranDRFWOVB16}\\
TranWDRFOVB16 \href{http://www.aaai.org/ocs/index.php/WS/AAAIW16/paper/view/12664}{TranWDRFOVB16} & \hyperref[auth:a810]{Tony T. Tran}, \hyperref[auth:a819]{Z. Wang}, \hyperref[auth:a820]{M. Do}, \hyperref[auth:a821]{Eleanor Gilbert Rieffel}, \hyperref[auth:a383]{J. Frank}, \hyperref[auth:a822]{B. O'Gorman}, \hyperref[auth:a823]{D. Venturelli}, \hyperref[auth:a89]{J. Christopher Beck} & Explorations of Quantum-Classical Approaches to Scheduling a Mars Lander Activity Problem & \href{works/TranWDRFOVB16.pdf}{Yes} & \cite{TranWDRFOVB16} & 2016 & AAAI 2016 & 9 & 0 & 0 & \ref{b:TranWDRFOVB16} & \ref{c:TranWDRFOVB16}\\
BajestaniB15 \href{https://doi.org/10.1007/s10951-015-0416-2}{BajestaniB15} & \hyperref[auth:a828]{Maliheh Aramon Bajestani}, \hyperref[auth:a89]{J. Christopher Beck} & A two-stage coupled algorithm for an integrated maintenance planning and flowshop scheduling problem with deteriorating machines & \href{works/BajestaniB15.pdf}{Yes} & \cite{BajestaniB15} & 2015 & J. Sched. & 16 & 17 & 59 & \ref{b:BajestaniB15} & \ref{c:BajestaniB15}\\
KoschB14 \href{https://doi.org/10.1007/978-3-319-07046-9\_5}{KoschB14} & \hyperref[auth:a332]{S. Kosch}, \hyperref[auth:a89]{J. Christopher Beck} & A New {MIP} Model for Parallel-Batch Scheduling with Non-identical Job Sizes & \href{works/KoschB14.pdf}{Yes} & \cite{KoschB14} & 2014 & CPAIOR 2014 & 16 & 4 & 18 & \ref{b:KoschB14} & \ref{c:KoschB14}\\
LouieVNB14 \href{https://doi.org/10.1109/ICRA.2014.6907637}{LouieVNB14} & \hyperref[auth:a830]{Wing{-}Yue Geoffrey Louie}, \hyperref[auth:a815]{Tiago Stegun Vaquero}, \hyperref[auth:a209]{G. Nejat}, \hyperref[auth:a89]{J. Christopher Beck} & An autonomous assistive robot for planning, scheduling and facilitating multi-user activities & No & \cite{LouieVNB14} & 2014 & ICRA 2014 & 7 & 16 & 9 & No & \ref{c:LouieVNB14}\\
TerekhovTDB14 \href{https://doi.org/10.1613/jair.4278}{TerekhovTDB14} & \hyperref[auth:a829]{D. Terekhov}, \hyperref[auth:a810]{Tony T. Tran}, \hyperref[auth:a814]{Douglas G. Down}, \hyperref[auth:a89]{J. Christopher Beck} & Integrating Queueing Theory and Scheduling for Dynamic Scheduling Problems & \href{works/TerekhovTDB14.pdf}{Yes} & \cite{TerekhovTDB14} & 2014 & J. Artif. Intell. Res. & 38 & 12 & 0 & \ref{b:TerekhovTDB14} & \ref{c:TerekhovTDB14}\\
BajestaniB13 \href{https://doi.org/10.1613/jair.3902}{BajestaniB13} & \hyperref[auth:a828]{Maliheh Aramon Bajestani}, \hyperref[auth:a89]{J. Christopher Beck} & Scheduling a Dynamic Aircraft Repair Shop with Limited Repair Resources & \href{works/BajestaniB13.pdf}{Yes} & \cite{BajestaniB13} & 2013 & J. Artif. Intell. Res. & 36 & 14 & 0 & \ref{b:BajestaniB13} & \ref{c:BajestaniB13}\\
HeinzKB13 \href{https://doi.org/10.1007/978-3-642-38171-3\_2}{HeinzKB13} & \hyperref[auth:a133]{S. Heinz}, \hyperref[auth:a336]{W. Ku}, \hyperref[auth:a89]{J. Christopher Beck} & Recent Improvements Using Constraint Integer Programming for Resource Allocation and Scheduling & \href{works/HeinzKB13.pdf}{Yes} & \cite{HeinzKB13} & 2013 & CPAIOR 2013 & 16 & 9 & 15 & \ref{b:HeinzKB13} & \ref{c:HeinzKB13}\\
HeinzSB13 \href{https://doi.org/10.1007/s10601-012-9136-9}{HeinzSB13} & \hyperref[auth:a133]{S. Heinz}, \hyperref[auth:a134]{J. Schulz}, \hyperref[auth:a89]{J. Christopher Beck} & Using dual presolving reductions to reformulate cumulative constraints & \href{works/HeinzSB13.pdf}{Yes} & \cite{HeinzSB13} & 2013 & Constraints An Int. J. & 36 & 7 & 31 & \ref{b:HeinzSB13} & \ref{c:HeinzSB13}\\
TranTDB13 \href{http://www.aaai.org/ocs/index.php/ICAPS/ICAPS13/paper/view/6005}{TranTDB13} & \hyperref[auth:a810]{Tony T. Tran}, \hyperref[auth:a829]{D. Terekhov}, \hyperref[auth:a814]{Douglas G. Down}, \hyperref[auth:a89]{J. Christopher Beck} & Hybrid Queueing Theory and Scheduling Models for Dynamic Environments with Sequence-Dependent Setup Times & \href{works/TranTDB13.pdf}{Yes} & \cite{TranTDB13} & 2013 & ICAPS 2013 & 9 & 0 & 0 & \ref{b:TranTDB13} & \ref{c:TranTDB13}\\
HeinzB12 \href{https://doi.org/10.1007/978-3-642-29828-8\_14}{HeinzB12} & \hyperref[auth:a133]{S. Heinz}, \hyperref[auth:a89]{J. Christopher Beck} & Reconsidering Mixed Integer Programming and MIP-Based Hybrids for Scheduling & \href{works/HeinzB12.pdf}{Yes} & \cite{HeinzB12} & 2012 & CPAIOR 2012 & 17 & 8 & 21 & \ref{b:HeinzB12} & \ref{c:HeinzB12}\\
TerekhovDOB12 \href{https://doi.org/10.1016/j.cie.2012.02.006}{TerekhovDOB12} & \hyperref[auth:a829]{D. Terekhov}, \hyperref[auth:a831]{Mustafa K. Dogru}, \hyperref[auth:a832]{U. {\"{O}}zen}, \hyperref[auth:a89]{J. Christopher Beck} & Solving two-machine assembly scheduling problems with inventory constraints & No & \cite{TerekhovDOB12} & 2012 & Comput. Ind. Eng. & 15 & 8 & 48 & No & \ref{c:TerekhovDOB12}\\
TranB12 \href{https://doi.org/10.3233/978-1-61499-098-7-774}{TranB12} & \hyperref[auth:a810]{Tony T. Tran}, \hyperref[auth:a89]{J. Christopher Beck} & Logic-based Benders Decomposition for Alternative Resource Scheduling with Sequence Dependent Setups & \href{works/TranB12.pdf}{Yes} & \cite{TranB12} & 2012 & ECAI 2012 & 6 & 0 & 0 & \ref{b:TranB12} & \ref{c:TranB12}\\
BajestaniB11 \href{http://aaai.org/ocs/index.php/ICAPS/ICAPS11/paper/view/2680}{BajestaniB11} & \hyperref[auth:a828]{Maliheh Aramon Bajestani}, \hyperref[auth:a89]{J. Christopher Beck} & Scheduling an Aircraft Repair Shop & \href{works/BajestaniB11.pdf}{Yes} & \cite{BajestaniB11} & 2011 & ICAPS 2011 & 8 & 0 & 0 & \ref{b:BajestaniB11} & \ref{c:BajestaniB11}\\
BeckFW11 \href{https://doi.org/10.1287/ijoc.1100.0388}{BeckFW11} & \hyperref[auth:a89]{J. Christopher Beck}, \hyperref[auth:a833]{T. K. Feng}, \hyperref[auth:a364]{J. Watson} & Combining Constraint Programming and Local Search for Job-Shop Scheduling & \href{works/BeckFW11.pdf}{Yes} & \cite{BeckFW11} & 2011 & {INFORMS} J. Comput. & 14 & 43 & 23 & \ref{b:BeckFW11} & \ref{c:BeckFW11}\\
HeckmanB11 \href{https://doi.org/10.1007/s10951-009-0113-0}{HeckmanB11} & \hyperref[auth:a834]{I. Heckman}, \hyperref[auth:a89]{J. Christopher Beck} & Understanding the behavior of Solution-Guided Search for job-shop scheduling & \href{works/HeckmanB11.pdf}{Yes} & \cite{HeckmanB11} & 2011 & J. Sched. & 20 & 0 & 22 & \ref{b:HeckmanB11} & \ref{c:HeckmanB11}\\
KovacsB11 \href{https://doi.org/10.1007/s10601-009-9088-x}{KovacsB11} & \hyperref[auth:a146]{A. Kov{\'{a}}cs}, \hyperref[auth:a89]{J. Christopher Beck} & A global constraint for total weighted completion time for unary resources & \href{works/KovacsB11.pdf}{Yes} & \cite{KovacsB11} & 2011 & Constraints An Int. J. & 24 & 4 & 26 & \ref{b:KovacsB11} & \ref{c:KovacsB11}\\
BidotVLB09 \href{https://doi.org/10.1007/s10951-008-0080-x}{BidotVLB09} & \hyperref[auth:a835]{J. Bidot}, \hyperref[auth:a836]{T. Vidal}, \hyperref[auth:a118]{P. Laborie}, \hyperref[auth:a89]{J. Christopher Beck} & A theoretic and practical framework for scheduling in a stochastic environment & \href{works/BidotVLB09.pdf}{Yes} & \cite{BidotVLB09} & 2009 & J. Sched. & 30 & 58 & 20 & \ref{b:BidotVLB09} & \ref{c:BidotVLB09}\\
WuBB09 \href{https://doi.org/10.1016/j.cor.2008.08.008}{WuBB09} & \hyperref[auth:a276]{Christine Wei Wu}, \hyperref[auth:a222]{Kenneth N. Brown}, \hyperref[auth:a89]{J. Christopher Beck} & Scheduling with uncertain durations: Modeling beta-robust scheduling with constraints & No & \cite{WuBB09} & 2009 & Comput. Oper. Res. & 9 & 42 & 5 & No & \ref{c:WuBB09}\\
KovacsB08 \href{https://doi.org/10.1016/j.engappai.2008.03.004}{KovacsB08} & \hyperref[auth:a146]{A. Kov{\'{a}}cs}, \hyperref[auth:a89]{J. Christopher Beck} & A global constraint for total weighted completion time for cumulative resources & \href{works/KovacsB08.pdf}{Yes} & \cite{KovacsB08} & 2008 & Eng. Appl. Artif. Intell. & 7 & 5 & 14 & \ref{b:KovacsB08} & \ref{c:KovacsB08}\\
WatsonB08 \href{https://doi.org/10.1007/978-3-540-68155-7\_21}{WatsonB08} & \hyperref[auth:a364]{J. Watson}, \hyperref[auth:a89]{J. Christopher Beck} & A Hybrid Constraint Programming / Local Search Approach to the Job-Shop Scheduling Problem & \href{works/WatsonB08.pdf}{Yes} & \cite{WatsonB08} & 2008 & CPAIOR 2008 & 15 & 14 & 17 & \ref{b:WatsonB08} & \ref{c:WatsonB08}\\
Beck07 \href{https://doi.org/10.1613/jair.2169}{Beck07} & \hyperref[auth:a89]{J. Christopher Beck} & Solution-Guided Multi-Point Constructive Search for Job Shop Scheduling & \href{works/Beck07.pdf}{Yes} & \cite{Beck07} & 2007 & J. Artif. Intell. Res. & 29 & 34 & 0 & \ref{b:Beck07} & \ref{c:Beck07}\\
BeckW07 \href{https://doi.org/10.1613/jair.2080}{BeckW07} & \hyperref[auth:a89]{J. Christopher Beck}, \hyperref[auth:a837]{N. Wilson} & Proactive Algorithms for Job Shop Scheduling with Probabilistic Durations & \href{works/BeckW07.pdf}{Yes} & \cite{BeckW07} & 2007 & J. Artif. Intell. Res. & 50 & 27 & 0 & \ref{b:BeckW07} & \ref{c:BeckW07}\\
KovacsB07 \href{https://doi.org/10.1007/978-3-540-72397-4\_9}{KovacsB07} & \hyperref[auth:a146]{A. Kov{\'{a}}cs}, \hyperref[auth:a89]{J. Christopher Beck} & A Global Constraint for Total Weighted Completion Time & \href{works/KovacsB07.pdf}{Yes} & \cite{KovacsB07} & 2007 & CPAIOR 2007 & 15 & 2 & 12 & \ref{b:KovacsB07} & \ref{c:KovacsB07}\\
Beck06 \href{http://www.aaai.org/Library/ICAPS/2006/icaps06-028.php}{Beck06} & \hyperref[auth:a89]{J. Christopher Beck} & An Empirical Study of Multi-Point Constructive Search for Constraint-Based Scheduling & \href{works/Beck06.pdf}{Yes} & \cite{Beck06} & 2006 & ICAPS 2006 & 10 & 0 & 0 & \ref{b:Beck06} & \ref{c:Beck06}\\
BeckW05 \href{http://ijcai.org/Proceedings/05/Papers/0748.pdf}{BeckW05} & \hyperref[auth:a89]{J. Christopher Beck}, \hyperref[auth:a837]{N. Wilson} & Proactive Algorithms for Scheduling with Probabilistic Durations & \href{works/BeckW05.pdf}{Yes} & \cite{BeckW05} & 2005 & IJCAI 2005 & 6 & 0 & 0 & \ref{b:BeckW05} & \ref{c:BeckW05}\\
CarchraeBF05 \href{https://doi.org/10.1007/11564751\_80}{CarchraeBF05} & \hyperref[auth:a274]{T. Carchrae}, \hyperref[auth:a89]{J. Christopher Beck}, \hyperref[auth:a275]{Eugene C. Freuder} & Methods to Learn Abstract Scheduling Models & \href{works/CarchraeBF05.pdf}{Yes} & \cite{CarchraeBF05} & 2005 & CP 2005 & 1 & 0 & 0 & \ref{b:CarchraeBF05} & \ref{c:CarchraeBF05}\\
WuBB05 \href{https://doi.org/10.1007/11564751\_110}{WuBB05} & \hyperref[auth:a276]{Christine Wei Wu}, \hyperref[auth:a222]{Kenneth N. Brown}, \hyperref[auth:a89]{J. Christopher Beck} & Scheduling with Uncertain Start Dates & \href{works/WuBB05.pdf}{Yes} & \cite{WuBB05} & 2005 & CP 2005 & 1 & 0 & 0 & \ref{b:WuBB05} & \ref{c:WuBB05}\\
BeckW04 \href{}{BeckW04} & \hyperref[auth:a89]{J. Christopher Beck}, \hyperref[auth:a837]{N. Wilson} & Job Shop Scheduling with Probabilistic Durations & \href{works/BeckW04.pdf}{Yes} & \cite{BeckW04} & 2004 & ECAI 2004 & 5 & 0 & 0 & \ref{b:BeckW04} & \ref{c:BeckW04}\\
BeckPS03 \href{http://www.aaai.org/Library/ICAPS/2003/icaps03-027.php}{BeckPS03} & \hyperref[auth:a89]{J. Christopher Beck}, \hyperref[auth:a838]{P. Prosser}, \hyperref[auth:a839]{E. Selensky} & Vehicle Routing and Job Shop Scheduling: What's the Difference? & \href{works/BeckPS03.pdf}{Yes} & \cite{BeckPS03} & 2003 & ICAPS 2003 & 10 & 0 & 0 & \ref{b:BeckPS03} & \ref{c:BeckPS03}\\
BeckR03 \href{https://doi.org/10.1023/A:1021849405707}{BeckR03} & \hyperref[auth:a89]{J. Christopher Beck}, \hyperref[auth:a256]{P. Refalo} & A Hybrid Approach to Scheduling with Earliness and Tardiness Costs & \href{works/BeckR03.pdf}{Yes} & \cite{BeckR03} & 2003 & Ann. Oper. Res. & 23 & 29 & 0 & \ref{b:BeckR03} & \ref{c:BeckR03}\\
BeckF00 \href{https://doi.org/10.1016/S0004-3702(99)00099-5}{BeckF00} & \hyperref[auth:a89]{J. Christopher Beck}, \hyperref[auth:a304]{Mark S. Fox} & Dynamic problem structure analysis as a basis for constraint-directed scheduling heuristics & \href{works/BeckF00.pdf}{Yes} & \cite{BeckF00} & 2000 & Artif. Intell. & 51 & 24 & 19 & \ref{b:BeckF00} & \ref{c:BeckF00}\\
Beck99 \href{https://librarysearch.library.utoronto.ca/permalink/01UTORONTO\_INST/14bjeso/alma991106162342106196}{Beck99} & \hyperref[auth:a89]{J. Christopher Beck} & Texture measurements as a basis for heuristic commitment techniques in constraint-directed scheduling & \href{works/Beck99.pdf}{Yes} & \cite{Beck99} & 1999 & University of Toronto, Canada & 418 & 0 & 0 & \ref{b:Beck99} & \ref{c:Beck99}\\
BeckF98 \href{https://doi.org/10.1609/aimag.v19i4.1426}{BeckF98} & \hyperref[auth:a89]{J. Christopher Beck}, \hyperref[auth:a304]{Mark S. Fox} & A Generic Framework for Constraint-Directed Search and Scheduling & \href{works/BeckF98.pdf}{Yes} & \cite{BeckF98} & 1998 & {AI} Mag. & 30 & 0 & 0 & \ref{b:BeckF98} & \ref{c:BeckF98}\\
BeckDF97 \href{https://doi.org/10.1007/BFb0017455}{BeckDF97} & \hyperref[auth:a89]{J. Christopher Beck}, \hyperref[auth:a250]{Andrew J. Davenport}, \hyperref[auth:a304]{Mark S. Fox} & Five Pitfalls of Empirical Scheduling Research & \href{works/BeckDF97.pdf}{Yes} & \cite{BeckDF97} & 1997 & CP 1997 & 15 & 3 & 12 & \ref{b:BeckDF97} & \ref{c:BeckDF97}\\
\end{longtable}
}

\subsection{Works by Michela Milano}
\label{sec:a143}
{\scriptsize
\begin{longtable}{>{\raggedright\arraybackslash}p{3cm}>{\raggedright\arraybackslash}p{6cm}>{\raggedright\arraybackslash}p{6.5cm}rrrp{2.5cm}rrrrr}
\rowcolor{white}\caption{Works from bibtex (Total 24)}\\ \toprule
\rowcolor{white}Key & Authors & Title & LC & Cite & Year & \shortstack{Conference\\/Journal} & Pages & \shortstack{Nr\\Cites} & \shortstack{Nr\\Refs} & b & c \\ \midrule\endhead
\bottomrule
\endfoot
BorghesiBLMB18 \href{https://doi.org/10.1016/j.suscom.2018.05.007}{BorghesiBLMB18} & \hyperref[auth:a231]{A. Borghesi}, \hyperref[auth:a230]{A. Bartolini}, \hyperref[auth:a142]{M. Lombardi}, \hyperref[auth:a143]{M. Milano}, \hyperref[auth:a247]{L. Benini} & Scheduling-based power capping in high performance computing systems & \href{works/BorghesiBLMB18.pdf}{Yes} & \cite{BorghesiBLMB18} & 2018 & Sustain. Comput. Informatics Syst. & 13 & 11 & 22 & \ref{b:BorghesiBLMB18} & \ref{c:BorghesiBLMB18}\\
BonfiettiZLM16 \href{https://doi.org/10.1007/978-3-319-44953-1\_8}{BonfiettiZLM16} & \hyperref[auth:a203]{A. Bonfietti}, \hyperref[auth:a204]{A. Zanarini}, \hyperref[auth:a142]{M. Lombardi}, \hyperref[auth:a143]{M. Milano} & The Multirate Resource Constraint & \href{works/BonfiettiZLM16.pdf}{Yes} & \cite{BonfiettiZLM16} & 2016 & CP 2016 & 17 & 0 & 11 & \ref{b:BonfiettiZLM16} & \ref{c:BonfiettiZLM16}\\
BridiBLMB16 \href{https://doi.org/10.1109/TPDS.2016.2516997}{BridiBLMB16} & \hyperref[auth:a232]{T. Bridi}, \hyperref[auth:a230]{A. Bartolini}, \hyperref[auth:a142]{M. Lombardi}, \hyperref[auth:a143]{M. Milano}, \hyperref[auth:a247]{L. Benini} & A Constraint Programming Scheduler for Heterogeneous High-Performance Computing Machines & \href{works/BridiBLMB16.pdf}{Yes} & \cite{BridiBLMB16} & 2016 & {IEEE} Trans. Parallel Distributed Syst. & 14 & 17 & 22 & \ref{b:BridiBLMB16} & \ref{c:BridiBLMB16}\\
BridiLBBM16 \href{https://doi.org/10.3233/978-1-61499-672-9-1598}{BridiLBBM16} & \hyperref[auth:a232]{T. Bridi}, \hyperref[auth:a142]{M. Lombardi}, \hyperref[auth:a230]{A. Bartolini}, \hyperref[auth:a247]{L. Benini}, \hyperref[auth:a143]{M. Milano} & {DARDIS:} Distributed And Randomized DIspatching and Scheduling & \href{works/BridiLBBM16.pdf}{Yes} & \cite{BridiLBBM16} & 2016 & ECAI 2016 & 2 & 0 & 0 & \ref{b:BridiLBBM16} & \ref{c:BridiLBBM16}\\
LombardiBM15 \href{https://doi.org/10.1007/978-3-319-23219-5\_20}{LombardiBM15} & \hyperref[auth:a142]{M. Lombardi}, \hyperref[auth:a203]{A. Bonfietti}, \hyperref[auth:a143]{M. Milano} & Deterministic Estimation of the Expected Makespan of a {POS} Under Duration Uncertainty & \href{works/LombardiBM15.pdf}{Yes} & \cite{LombardiBM15} & 2015 & CP 2015 & 16 & 0 & 8 & \ref{b:LombardiBM15} & \ref{c:LombardiBM15}\\
BartoliniBBLM14 \href{https://doi.org/10.1007/978-3-319-10428-7\_55}{BartoliniBBLM14} & \hyperref[auth:a230]{A. Bartolini}, \hyperref[auth:a231]{A. Borghesi}, \hyperref[auth:a232]{T. Bridi}, \hyperref[auth:a142]{M. Lombardi}, \hyperref[auth:a143]{M. Milano} & Proactive Workload Dispatching on the {EURORA} Supercomputer & \href{works/BartoliniBBLM14.pdf}{Yes} & \cite{BartoliniBBLM14} & 2014 & CP 2014 & 16 & 12 & 3 & \ref{b:BartoliniBBLM14} & \ref{c:BartoliniBBLM14}\\
BonfiettiLBM14 \href{https://doi.org/10.1016/j.artint.2013.09.006}{BonfiettiLBM14} & \hyperref[auth:a203]{A. Bonfietti}, \hyperref[auth:a142]{M. Lombardi}, \hyperref[auth:a247]{L. Benini}, \hyperref[auth:a143]{M. Milano} & {CROSS} cyclic resource-constrained scheduling solver & \href{works/BonfiettiLBM14.pdf}{Yes} & \cite{BonfiettiLBM14} & 2014 & Artif. Intell. & 28 & 8 & 15 & \ref{b:BonfiettiLBM14} & \ref{c:BonfiettiLBM14}\\
BonfiettiLM14 \href{https://doi.org/10.1007/978-3-319-07046-9\_15}{BonfiettiLM14} & \hyperref[auth:a203]{A. Bonfietti}, \hyperref[auth:a142]{M. Lombardi}, \hyperref[auth:a143]{M. Milano} & Disregarding Duration Uncertainty in Partial Order Schedules? Yes, We Can! & \href{works/BonfiettiLM14.pdf}{Yes} & \cite{BonfiettiLM14} & 2014 & CPAIOR 2014 & 16 & 3 & 12 & \ref{b:BonfiettiLM14} & \ref{c:BonfiettiLM14}\\
BonfiettiLM13 \href{http://www.aaai.org/ocs/index.php/ICAPS/ICAPS13/paper/view/6050}{BonfiettiLM13} & \hyperref[auth:a203]{A. Bonfietti}, \hyperref[auth:a142]{M. Lombardi}, \hyperref[auth:a143]{M. Milano} & De-Cycling Cyclic Scheduling Problems & \href{works/BonfiettiLM13.pdf}{Yes} & \cite{BonfiettiLM13} & 2013 & ICAPS 2013 & 5 & 0 & 0 & \ref{b:BonfiettiLM13} & \ref{c:BonfiettiLM13}\\
LombardiM13 \href{http://www.aaai.org/ocs/index.php/ICAPS/ICAPS13/paper/view/6052}{LombardiM13} & \hyperref[auth:a142]{M. Lombardi}, \hyperref[auth:a143]{M. Milano} & A Min-Flow Algorithm for Minimal Critical Set Detection in Resource Constrained Project Scheduling & \href{works/LombardiM13.pdf}{Yes} & \cite{LombardiM13} & 2013 & ICAPS 2013 & 2 & 0 & 0 & \ref{b:LombardiM13} & \ref{c:LombardiM13}\\
BonfiettiLBM12 \href{https://doi.org/10.1007/978-3-642-29828-8\_6}{BonfiettiLBM12} & \hyperref[auth:a203]{A. Bonfietti}, \hyperref[auth:a142]{M. Lombardi}, \hyperref[auth:a247]{L. Benini}, \hyperref[auth:a143]{M. Milano} & Global Cyclic Cumulative Constraint & \href{works/BonfiettiLBM12.pdf}{Yes} & \cite{BonfiettiLBM12} & 2012 & CPAIOR 2012 & 16 & 2 & 11 & \ref{b:BonfiettiLBM12} & \ref{c:BonfiettiLBM12}\\
BonfiettiM12 \href{https://ceur-ws.org/Vol-926/paper2.pdf}{BonfiettiM12} & \hyperref[auth:a203]{A. Bonfietti}, \hyperref[auth:a143]{M. Milano} & A Constraint-based Approach to Cyclic Resource-Constrained Scheduling Problem & \href{works/BonfiettiM12.pdf}{Yes} & \cite{BonfiettiM12} & 2012 & DC SIAAI 2012 & 3 & 0 & 0 & \ref{b:BonfiettiM12} & \ref{c:BonfiettiM12}\\
LombardiM12 \href{https://doi.org/10.1007/s10601-011-9115-6}{LombardiM12} & \hyperref[auth:a142]{M. Lombardi}, \hyperref[auth:a143]{M. Milano} & Optimal methods for resource allocation and scheduling: a cross-disciplinary survey & \href{works/LombardiM12.pdf}{Yes} & \cite{LombardiM12} & 2012 & Constraints An Int. J. & 35 & 39 & 68 & \ref{b:LombardiM12} & \ref{c:LombardiM12}\\
LombardiM12a \href{https://doi.org/10.1016/j.artint.2011.12.001}{LombardiM12a} & \hyperref[auth:a142]{M. Lombardi}, \hyperref[auth:a143]{M. Milano} & A min-flow algorithm for Minimal Critical Set detection in Resource Constrained Project Scheduling & \href{works/LombardiM12a.pdf}{Yes} & \cite{LombardiM12a} & 2012 & Artif. Intell. & 10 & 3 & 13 & \ref{b:LombardiM12a} & \ref{c:LombardiM12a}\\
BeniniLMR11 \href{https://doi.org/10.1007/s10479-010-0718-x}{BeniniLMR11} & \hyperref[auth:a247]{L. Benini}, \hyperref[auth:a142]{M. Lombardi}, \hyperref[auth:a143]{M. Milano}, \hyperref[auth:a727]{M. Ruggiero} & Optimal resource allocation and scheduling for the {CELL} {BE} platform & \href{works/BeniniLMR11.pdf}{Yes} & \cite{BeniniLMR11} & 2011 & Ann. Oper. Res. & 27 & 18 & 16 & \ref{b:BeniniLMR11} & \ref{c:BeniniLMR11}\\
BonfiettiLBM11 \href{https://doi.org/10.1007/978-3-642-23786-7\_12}{BonfiettiLBM11} & \hyperref[auth:a203]{A. Bonfietti}, \hyperref[auth:a142]{M. Lombardi}, \hyperref[auth:a247]{L. Benini}, \hyperref[auth:a143]{M. Milano} & A Constraint Based Approach to Cyclic {RCPSP} & \href{works/BonfiettiLBM11.pdf}{Yes} & \cite{BonfiettiLBM11} & 2011 & CP 2011 & 15 & 3 & 14 & \ref{b:BonfiettiLBM11} & \ref{c:BonfiettiLBM11}\\
LombardiBMB11 \href{https://doi.org/10.1007/978-3-642-21311-3\_14}{LombardiBMB11} & \hyperref[auth:a142]{M. Lombardi}, \hyperref[auth:a203]{A. Bonfietti}, \hyperref[auth:a143]{M. Milano}, \hyperref[auth:a247]{L. Benini} & Precedence Constraint Posting for Cyclic Scheduling Problems & \href{works/LombardiBMB11.pdf}{Yes} & \cite{LombardiBMB11} & 2011 & CPAIOR 2011 & 17 & 1 & 13 & \ref{b:LombardiBMB11} & \ref{c:LombardiBMB11}\\
LombardiM10 \href{https://doi.org/10.1007/978-3-642-15396-9\_32}{LombardiM10} & \hyperref[auth:a142]{M. Lombardi}, \hyperref[auth:a143]{M. Milano} & Constraint Based Scheduling to Deal with Uncertain Durations and Self-Timed Execution & \href{works/LombardiM10.pdf}{Yes} & \cite{LombardiM10} & 2010 & CP 2010 & 15 & 1 & 11 & \ref{b:LombardiM10} & \ref{c:LombardiM10}\\
LombardiM10a \href{https://doi.org/10.1016/j.artint.2010.02.004}{LombardiM10a} & \hyperref[auth:a142]{M. Lombardi}, \hyperref[auth:a143]{M. Milano} & Allocation and scheduling of Conditional Task Graphs & \href{works/LombardiM10a.pdf}{Yes} & \cite{LombardiM10a} & 2010 & Artif. Intell. & 30 & 8 & 24 & \ref{b:LombardiM10a} & \ref{c:LombardiM10a}\\
LombardiM09 \href{https://doi.org/10.1007/978-3-642-04244-7\_45}{LombardiM09} & \hyperref[auth:a142]{M. Lombardi}, \hyperref[auth:a143]{M. Milano} & A Precedence Constraint Posting Approach for the {RCPSP} with Time Lags and Variable Durations & \href{works/LombardiM09.pdf}{Yes} & \cite{LombardiM09} & 2009 & CP 2009 & 15 & 7 & 12 & \ref{b:LombardiM09} & \ref{c:LombardiM09}\\
RuggieroBBMA09 \href{https://doi.org/10.1109/TCAD.2009.2013536}{RuggieroBBMA09} & \hyperref[auth:a727]{M. Ruggiero}, \hyperref[auth:a379]{D. Bertozzi}, \hyperref[auth:a247]{L. Benini}, \hyperref[auth:a143]{M. Milano}, \hyperref[auth:a728]{A. Andrei} & Reducing the Abstraction and Optimality Gaps in the Allocation and Scheduling for Variable Voltage/Frequency MPSoC Platforms & \href{works/RuggieroBBMA09.pdf}{Yes} & \cite{RuggieroBBMA09} & 2009 & {IEEE} Trans. Comput. Aided Des. Integr. Circuits Syst. & 14 & 9 & 27 & \ref{b:RuggieroBBMA09} & \ref{c:RuggieroBBMA09}\\
BeniniBGM06 \href{https://doi.org/10.1007/11757375\_6}{BeniniBGM06} & \hyperref[auth:a247]{L. Benini}, \hyperref[auth:a379]{D. Bertozzi}, \hyperref[auth:a380]{A. Guerri}, \hyperref[auth:a143]{M. Milano} & Allocation, Scheduling and Voltage Scaling on Energy Aware MPSoCs & \href{works/BeniniBGM06.pdf}{Yes} & \cite{BeniniBGM06} & 2006 & CPAIOR 2006 & 15 & 18 & 10 & \ref{b:BeniniBGM06} & \ref{c:BeniniBGM06}\\
LammaMM97 \href{https://doi.org/10.1016/S0954-1810(96)00002-7}{LammaMM97} & \hyperref[auth:a729]{E. Lamma}, \hyperref[auth:a730]{P. Mello}, \hyperref[auth:a143]{M. Milano} & A distributed constraint-based scheduler & \href{works/LammaMM97.pdf}{Yes} & \cite{LammaMM97} & 1997 & Artif. Intell. Eng. & 15 & 11 & 7 & \ref{b:LammaMM97} & \ref{c:LammaMM97}\\
BrusoniCLMMT96 \href{https://doi.org/10.1007/3-540-61286-6\_157}{BrusoniCLMMT96} & \hyperref[auth:a731]{V. Brusoni}, \hyperref[auth:a732]{L. Console}, \hyperref[auth:a729]{E. Lamma}, \hyperref[auth:a730]{P. Mello}, \hyperref[auth:a143]{M. Milano}, \hyperref[auth:a733]{P. Terenziani} & Resource-Based vs. Task-Based Approaches for Scheduling Problems & \href{works/BrusoniCLMMT96.pdf}{Yes} & \cite{BrusoniCLMMT96} & 1996 & ISMIS 1996 & 10 & 1 & 9 & \ref{b:BrusoniCLMMT96} & \ref{c:BrusoniCLMMT96}\\
\end{longtable}
}

\subsection{Works by Andreas Schutt}
\label{sec:a124}
{\scriptsize
\begin{longtable}{>{\raggedright\arraybackslash}p{3cm}>{\raggedright\arraybackslash}p{6cm}>{\raggedright\arraybackslash}p{6.5cm}rrrp{2.5cm}rrrrr}
\rowcolor{white}\caption{Works from bibtex (Total 24)}\\ \toprule
\rowcolor{white}Key & Authors & Title & LC & Cite & Year & \shortstack{Conference\\/Journal} & Pages & \shortstack{Nr\\Cites} & \shortstack{Nr\\Refs} & b & c \\ \midrule\endhead
\bottomrule
\endfoot
YangSS19 \href{https://doi.org/10.1007/978-3-030-19212-9\_42}{YangSS19} & \hyperref[auth:a311]{M. Yang}, \hyperref[auth:a124]{A. Schutt}, \hyperref[auth:a125]{Peter J. Stuckey} & Time Table Edge Finding with Energy Variables & \href{works/YangSS19.pdf}{Yes} & \cite{YangSS19} & 2019 & CPAIOR 2019 & 10 & 1 & 14 & \ref{b:YangSS19} & \ref{c:YangSS19}\\
GoldwaserS18 \href{https://doi.org/10.1613/jair.1.11268}{GoldwaserS18} & \hyperref[auth:a194]{A. Goldwaser}, \hyperref[auth:a124]{A. Schutt} & Optimal Torpedo Scheduling & \href{works/GoldwaserS18.pdf}{Yes} & \cite{GoldwaserS18} & 2018 & J. Artif. Intell. Res. & 32 & 8 & 0 & \ref{b:GoldwaserS18} & \ref{c:GoldwaserS18}\\
KreterSSZ18 \href{https://doi.org/10.1016/j.ejor.2017.10.014}{KreterSSZ18} & \hyperref[auth:a123]{S. Kreter}, \hyperref[auth:a124]{A. Schutt}, \hyperref[auth:a125]{Peter J. Stuckey}, \hyperref[auth:a803]{J. Zimmermann} & Mixed-integer linear programming and constraint programming formulations for solving resource availability cost problems & No & \cite{KreterSSZ18} & 2018 & Eur. J. Oper. Res. & 15 & 25 & 31 & No & \ref{c:KreterSSZ18}\\
MusliuSS18 \href{https://doi.org/10.1007/978-3-319-93031-2\_31}{MusliuSS18} & \hyperref[auth:a45]{N. Musliu}, \hyperref[auth:a124]{A. Schutt}, \hyperref[auth:a125]{Peter J. Stuckey} & Solver Independent Rotating Workforce Scheduling & \href{works/MusliuSS18.pdf}{Yes} & \cite{MusliuSS18} & 2018 & CPAIOR 2018 & 17 & 7 & 23 & \ref{b:MusliuSS18} & \ref{c:MusliuSS18}\\
GoldwaserS17 \href{https://doi.org/10.1007/978-3-319-66158-2\_22}{GoldwaserS17} & \hyperref[auth:a194]{A. Goldwaser}, \hyperref[auth:a124]{A. Schutt} & Optimal Torpedo Scheduling & \href{works/GoldwaserS17.pdf}{Yes} & \cite{GoldwaserS17} & 2017 & CP 2017 & 16 & 0 & 10 & \ref{b:GoldwaserS17} & \ref{c:GoldwaserS17}\\
KreterSS17 \href{https://doi.org/10.1007/s10601-016-9266-6}{KreterSS17} & \hyperref[auth:a123]{S. Kreter}, \hyperref[auth:a124]{A. Schutt}, \hyperref[auth:a125]{Peter J. Stuckey} & Using constraint programming for solving RCPSP/max-cal & \href{works/KreterSS17.pdf}{Yes} & \cite{KreterSS17} & 2017 & Constraints An Int. J. & 31 & 15 & 20 & \ref{b:KreterSS17} & \ref{c:KreterSS17}\\
YoungFS17 \href{https://doi.org/10.1007/978-3-319-66158-2\_20}{YoungFS17} & \hyperref[auth:a193]{Kenneth D. Young}, \hyperref[auth:a154]{T. Feydy}, \hyperref[auth:a124]{A. Schutt} & Constraint Programming Applied to the Multi-Skill Project Scheduling Problem & \href{works/YoungFS17.pdf}{Yes} & \cite{YoungFS17} & 2017 & CP 2017 & 10 & 6 & 21 & \ref{b:YoungFS17} & \ref{c:YoungFS17}\\
SchuttS16 \href{https://doi.org/10.1007/978-3-319-44953-1\_28}{SchuttS16} & \hyperref[auth:a124]{A. Schutt}, \hyperref[auth:a125]{Peter J. Stuckey} & Explaining Producer/Consumer Constraints & \href{works/SchuttS16.pdf}{Yes} & \cite{SchuttS16} & 2016 & CP 2016 & 17 & 3 & 23 & \ref{b:SchuttS16} & \ref{c:SchuttS16}\\
SzerediS16 \href{https://doi.org/10.1007/978-3-319-44953-1\_31}{SzerediS16} & \hyperref[auth:a205]{R. Szeredi}, \hyperref[auth:a124]{A. Schutt} & Modelling and Solving Multi-mode Resource-Constrained Project Scheduling & \href{works/SzerediS16.pdf}{Yes} & \cite{SzerediS16} & 2016 & CP 2016 & 10 & 9 & 14 & \ref{b:SzerediS16} & \ref{c:SzerediS16}\\
EvenSH15 \href{https://doi.org/10.1007/978-3-319-23219-5\_40}{EvenSH15} & \hyperref[auth:a219]{C. Even}, \hyperref[auth:a124]{A. Schutt}, \hyperref[auth:a148]{Pascal Van Hentenryck} & A Constraint Programming Approach for Non-preemptive Evacuation Scheduling & \href{works/EvenSH15.pdf}{Yes} & \cite{EvenSH15} & 2015 & CP 2015 & 18 & 3 & 12 & \ref{b:EvenSH15} & \ref{c:EvenSH15}\\
EvenSH15a \href{http://arxiv.org/abs/1505.02487}{EvenSH15a} & \hyperref[auth:a219]{C. Even}, \hyperref[auth:a124]{A. Schutt}, \hyperref[auth:a148]{Pascal Van Hentenryck} & A Constraint Programming Approach for Non-Preemptive Evacuation Scheduling & \href{works/EvenSH15a.pdf}{Yes} & \cite{EvenSH15a} & 2015 & CoRR & 16 & 0 & 0 & \ref{b:EvenSH15a} & \ref{c:EvenSH15a}\\
KreterSS15 \href{https://doi.org/10.1007/978-3-319-23219-5\_19}{KreterSS15} & \hyperref[auth:a123]{S. Kreter}, \hyperref[auth:a124]{A. Schutt}, \hyperref[auth:a125]{Peter J. Stuckey} & Modeling and Solving Project Scheduling with Calendars & \href{works/KreterSS15.pdf}{Yes} & \cite{KreterSS15} & 2015 & CP 2015 & 17 & 7 & 16 & \ref{b:KreterSS15} & \ref{c:KreterSS15}\\
ThiruvadyWGS14 \href{https://doi.org/10.1007/s10732-014-9260-3}{ThiruvadyWGS14} & \hyperref[auth:a400]{Dhananjay R. Thiruvady}, \hyperref[auth:a117]{M. Wallace}, \hyperref[auth:a341]{H. Gu}, \hyperref[auth:a124]{A. Schutt} & A Lagrangian relaxation and {ACO} hybrid for resource constrained project scheduling with discounted cash flows & \href{works/ThiruvadyWGS14.pdf}{Yes} & \cite{ThiruvadyWGS14} & 2014 & J. Heuristics & 34 & 19 & 18 & \ref{b:ThiruvadyWGS14} & \ref{c:ThiruvadyWGS14}\\
ChuGNSW13 \href{http://www.aaai.org/ocs/index.php/IJCAI/IJCAI13/paper/view/6878}{ChuGNSW13} & \hyperref[auth:a348]{G. Chu}, \hyperref[auth:a804]{S. Gaspers}, \hyperref[auth:a805]{N. Narodytska}, \hyperref[auth:a124]{A. Schutt}, \hyperref[auth:a278]{T. Walsh} & On the Complexity of Global Scheduling Constraints under Structural Restrictions & \href{works/ChuGNSW13.pdf}{Yes} & \cite{ChuGNSW13} & 2013 & IJCAI 2013 & 7 & 0 & 0 & \ref{b:ChuGNSW13} & \ref{c:ChuGNSW13}\\
GuSS13 \href{https://doi.org/10.1007/978-3-642-38171-3\_24}{GuSS13} & \hyperref[auth:a341]{H. Gu}, \hyperref[auth:a124]{A. Schutt}, \hyperref[auth:a125]{Peter J. Stuckey} & A Lagrangian Relaxation Based Forward-Backward Improvement Heuristic for Maximising the Net Present Value of Resource-Constrained Projects & \href{works/GuSS13.pdf}{Yes} & \cite{GuSS13} & 2013 & CPAIOR 2013 & 7 & 10 & 24 & \ref{b:GuSS13} & \ref{c:GuSS13}\\
SchuttFS13 \href{https://doi.org/10.1007/978-3-642-40627-0\_47}{SchuttFS13} & \hyperref[auth:a124]{A. Schutt}, \hyperref[auth:a154]{T. Feydy}, \hyperref[auth:a125]{Peter J. Stuckey} & Scheduling Optional Tasks with Explanation & \href{works/SchuttFS13.pdf}{Yes} & \cite{SchuttFS13} & 2013 & CP 2013 & 17 & 10 & 20 & \ref{b:SchuttFS13} & \ref{c:SchuttFS13}\\
SchuttFS13a \href{https://doi.org/10.1007/978-3-642-38171-3\_16}{SchuttFS13a} & \hyperref[auth:a124]{A. Schutt}, \hyperref[auth:a154]{T. Feydy}, \hyperref[auth:a125]{Peter J. Stuckey} & Explaining Time-Table-Edge-Finding Propagation for the Cumulative Resource Constraint & \href{works/SchuttFS13a.pdf}{Yes} & \cite{SchuttFS13a} & 2013 & CPAIOR 2013 & 17 & 20 & 27 & \ref{b:SchuttFS13a} & \ref{c:SchuttFS13a}\\
SchuttFSW13 \href{https://doi.org/10.1007/s10951-012-0285-x}{SchuttFSW13} & \hyperref[auth:a124]{A. Schutt}, \hyperref[auth:a154]{T. Feydy}, \hyperref[auth:a125]{Peter J. Stuckey}, \hyperref[auth:a155]{Mark G. Wallace} & Solving RCPSP/max by lazy clause generation & \href{works/SchuttFSW13.pdf}{Yes} & \cite{SchuttFSW13} & 2013 & J. Sched. & 17 & 43 & 23 & \ref{b:SchuttFSW13} & \ref{c:SchuttFSW13}\\
SchuttCSW12 \href{https://doi.org/10.1007/978-3-642-29828-8\_24}{SchuttCSW12} & \hyperref[auth:a124]{A. Schutt}, \hyperref[auth:a348]{G. Chu}, \hyperref[auth:a125]{Peter J. Stuckey}, \hyperref[auth:a155]{Mark G. Wallace} & Maximising the Net Present Value for Resource-Constrained Project Scheduling & \href{works/SchuttCSW12.pdf}{Yes} & \cite{SchuttCSW12} & 2012 & CPAIOR 2012 & 17 & 18 & 21 & \ref{b:SchuttCSW12} & \ref{c:SchuttCSW12}\\
SchuttFSW11 \href{https://doi.org/10.1007/s10601-010-9103-2}{SchuttFSW11} & \hyperref[auth:a124]{A. Schutt}, \hyperref[auth:a154]{T. Feydy}, \hyperref[auth:a125]{Peter J. Stuckey}, \hyperref[auth:a155]{Mark G. Wallace} & Explaining the cumulative propagator & \href{works/SchuttFSW11.pdf}{Yes} & \cite{SchuttFSW11} & 2011 & Constraints An Int. J. & 33 & 57 & 23 & \ref{b:SchuttFSW11} & \ref{c:SchuttFSW11}\\
SchuttW10 \href{https://doi.org/10.1007/978-3-642-15396-9\_36}{SchuttW10} & \hyperref[auth:a124]{A. Schutt}, \hyperref[auth:a51]{A. Wolf} & A New \emph{O}(\emph{n}\({}^{\mbox{2}}\)log\emph{n}) Not-First/Not-Last Pruning Algorithm for Cumulative Resource Constraints & \href{works/SchuttW10.pdf}{Yes} & \cite{SchuttW10} & 2010 & CP 2010 & 15 & 13 & 14 & \ref{b:SchuttW10} & \ref{c:SchuttW10}\\
abs-1009-0347 \href{http://arxiv.org/abs/1009.0347}{abs-1009-0347} & \hyperref[auth:a124]{A. Schutt}, \hyperref[auth:a154]{T. Feydy}, \hyperref[auth:a125]{Peter J. Stuckey}, \hyperref[auth:a155]{Mark G. Wallace} & Solving the Resource Constrained Project Scheduling Problem with Generalized Precedences by Lazy Clause Generation & \href{works/abs-1009-0347.pdf}{Yes} & \cite{abs-1009-0347} & 2010 & CoRR & 37 & 0 & 0 & \ref{b:abs-1009-0347} & \ref{c:abs-1009-0347}\\
SchuttFSW09 \href{https://doi.org/10.1007/978-3-642-04244-7\_58}{SchuttFSW09} & \hyperref[auth:a124]{A. Schutt}, \hyperref[auth:a154]{T. Feydy}, \hyperref[auth:a125]{Peter J. Stuckey}, \hyperref[auth:a117]{M. Wallace} & Why Cumulative Decomposition Is Not as Bad as It Sounds & \href{works/SchuttFSW09.pdf}{Yes} & \cite{SchuttFSW09} & 2009 & CP 2009 & 16 & 34 & 11 & \ref{b:SchuttFSW09} & \ref{c:SchuttFSW09}\\
SchuttWS05 \href{https://doi.org/10.1007/11963578\_6}{SchuttWS05} & \hyperref[auth:a124]{A. Schutt}, \hyperref[auth:a51]{A. Wolf}, \hyperref[auth:a720]{G. Schrader} & Not-First and Not-Last Detection for Cumulative Scheduling in \emph{O}(\emph{n}\({}^{\mbox{3}}\)log\emph{n}) & \href{works/SchuttWS05.pdf}{Yes} & \cite{SchuttWS05} & 2005 & INAP 2005 & 15 & 6 & 4 & \ref{b:SchuttWS05} & \ref{c:SchuttWS05}\\
\end{longtable}
}

\subsection{Works by Peter J. Stuckey}
\label{sec:a125}
{\scriptsize
\begin{longtable}{>{\raggedright\arraybackslash}p{3cm}>{\raggedright\arraybackslash}p{6cm}>{\raggedright\arraybackslash}p{6.5cm}rrrp{2.5cm}rrrrr}
\rowcolor{white}\caption{Works from bibtex (Total 23)}\\ \toprule
\rowcolor{white}Key & Authors & Title & LC & Cite & Year & \shortstack{Conference\\/Journal} & Pages & \shortstack{Nr\\Cites} & \shortstack{Nr\\Refs} & b & c \\ \midrule\endhead
\bottomrule
\endfoot
YangSS19 \href{https://doi.org/10.1007/978-3-030-19212-9\_42}{YangSS19} & \hyperref[auth:a311]{M. Yang}, \hyperref[auth:a124]{A. Schutt}, \hyperref[auth:a125]{Peter J. Stuckey} & Time Table Edge Finding with Energy Variables & \href{works/YangSS19.pdf}{Yes} & \cite{YangSS19} & 2019 & CPAIOR 2019 & 10 & 1 & 14 & \ref{b:YangSS19} & \ref{c:YangSS19}\\
DemirovicS18 \href{https://doi.org/10.1007/978-3-319-93031-2\_10}{DemirovicS18} & \hyperref[auth:a314]{E. Demirovic}, \hyperref[auth:a125]{Peter J. Stuckey} & Constraint Programming for High School Timetabling: {A} Scheduling-Based Model with Hot Starts & \href{works/DemirovicS18.pdf}{Yes} & \cite{DemirovicS18} & 2018 & CPAIOR 2018 & 18 & 4 & 16 & \ref{b:DemirovicS18} & \ref{c:DemirovicS18}\\
KreterSSZ18 \href{https://doi.org/10.1016/j.ejor.2017.10.014}{KreterSSZ18} & \hyperref[auth:a123]{S. Kreter}, \hyperref[auth:a124]{A. Schutt}, \hyperref[auth:a125]{Peter J. Stuckey}, \hyperref[auth:a803]{J. Zimmermann} & Mixed-integer linear programming and constraint programming formulations for solving resource availability cost problems & No & \cite{KreterSSZ18} & 2018 & Eur. J. Oper. Res. & 15 & 25 & 31 & No & \ref{c:KreterSSZ18}\\
MusliuSS18 \href{https://doi.org/10.1007/978-3-319-93031-2\_31}{MusliuSS18} & \hyperref[auth:a45]{N. Musliu}, \hyperref[auth:a124]{A. Schutt}, \hyperref[auth:a125]{Peter J. Stuckey} & Solver Independent Rotating Workforce Scheduling & \href{works/MusliuSS18.pdf}{Yes} & \cite{MusliuSS18} & 2018 & CPAIOR 2018 & 17 & 7 & 23 & \ref{b:MusliuSS18} & \ref{c:MusliuSS18}\\
KreterSS17 \href{https://doi.org/10.1007/s10601-016-9266-6}{KreterSS17} & \hyperref[auth:a123]{S. Kreter}, \hyperref[auth:a124]{A. Schutt}, \hyperref[auth:a125]{Peter J. Stuckey} & Using constraint programming for solving RCPSP/max-cal & \href{works/KreterSS17.pdf}{Yes} & \cite{KreterSS17} & 2017 & Constraints An Int. J. & 31 & 15 & 20 & \ref{b:KreterSS17} & \ref{c:KreterSS17}\\
BlomPS16 \href{https://doi.org/10.1287/mnsc.2015.2284}{BlomPS16} & \hyperref[auth:a806]{Michelle L. Blom}, \hyperref[auth:a327]{Adrian R. Pearce}, \hyperref[auth:a125]{Peter J. Stuckey} & A Decomposition-Based Algorithm for the Scheduling of Open-Pit Networks Over Multiple Time Periods & No & \cite{BlomPS16} & 2016 & Manag. Sci. & 26 & 20 & 36 & No & \ref{c:BlomPS16}\\
SchuttS16 \href{https://doi.org/10.1007/978-3-319-44953-1\_28}{SchuttS16} & \hyperref[auth:a124]{A. Schutt}, \hyperref[auth:a125]{Peter J. Stuckey} & Explaining Producer/Consumer Constraints & \href{works/SchuttS16.pdf}{Yes} & \cite{SchuttS16} & 2016 & CP 2016 & 17 & 3 & 23 & \ref{b:SchuttS16} & \ref{c:SchuttS16}\\
BurtLPS15 \href{https://doi.org/10.1007/978-3-319-18008-3\_7}{BurtLPS15} & \hyperref[auth:a325]{Christina N. Burt}, \hyperref[auth:a326]{N. Lipovetzky}, \hyperref[auth:a327]{Adrian R. Pearce}, \hyperref[auth:a125]{Peter J. Stuckey} & Scheduling with Fixed Maintenance, Shared Resources and Nonlinear Feedrate Constraints: {A} Mine Planning Case Study & \href{works/BurtLPS15.pdf}{Yes} & \cite{BurtLPS15} & 2015 & CPAIOR 2015 & 17 & 0 & 8 & \ref{b:BurtLPS15} & \ref{c:BurtLPS15}\\
KreterSS15 \href{https://doi.org/10.1007/978-3-319-23219-5\_19}{KreterSS15} & \hyperref[auth:a123]{S. Kreter}, \hyperref[auth:a124]{A. Schutt}, \hyperref[auth:a125]{Peter J. Stuckey} & Modeling and Solving Project Scheduling with Calendars & \href{works/KreterSS15.pdf}{Yes} & \cite{KreterSS15} & 2015 & CP 2015 & 17 & 7 & 16 & \ref{b:KreterSS15} & \ref{c:KreterSS15}\\
BlomBPS14 \href{https://doi.org/10.1287/ijoc.2013.0590}{BlomBPS14} & \hyperref[auth:a806]{Michelle L. Blom}, \hyperref[auth:a325]{Christina N. Burt}, \hyperref[auth:a327]{Adrian R. Pearce}, \hyperref[auth:a125]{Peter J. Stuckey} & A Decomposition-Based Heuristic for Collaborative Scheduling in a Network of Open-Pit Mines & No & \cite{BlomBPS14} & 2014 & {INFORMS} J. Comput. & 19 & 15 & 47 & No & \ref{c:BlomBPS14}\\
LipovetzkyBPS14 \href{http://www.aaai.org/ocs/index.php/ICAPS/ICAPS14/paper/view/7942}{LipovetzkyBPS14} & \hyperref[auth:a326]{N. Lipovetzky}, \hyperref[auth:a325]{Christina N. Burt}, \hyperref[auth:a327]{Adrian R. Pearce}, \hyperref[auth:a125]{Peter J. Stuckey} & Planning for Mining Operations with Time and Resource Constraints & \href{works/LipovetzkyBPS14.pdf}{Yes} & \cite{LipovetzkyBPS14} & 2014 & ICAPS 2014 & 9 & 0 & 0 & \ref{b:LipovetzkyBPS14} & \ref{c:LipovetzkyBPS14}\\
GuSS13 \href{https://doi.org/10.1007/978-3-642-38171-3\_24}{GuSS13} & \hyperref[auth:a341]{H. Gu}, \hyperref[auth:a124]{A. Schutt}, \hyperref[auth:a125]{Peter J. Stuckey} & A Lagrangian Relaxation Based Forward-Backward Improvement Heuristic for Maximising the Net Present Value of Resource-Constrained Projects & \href{works/GuSS13.pdf}{Yes} & \cite{GuSS13} & 2013 & CPAIOR 2013 & 7 & 10 & 24 & \ref{b:GuSS13} & \ref{c:GuSS13}\\
SchuttFS13 \href{https://doi.org/10.1007/978-3-642-40627-0\_47}{SchuttFS13} & \hyperref[auth:a124]{A. Schutt}, \hyperref[auth:a154]{T. Feydy}, \hyperref[auth:a125]{Peter J. Stuckey} & Scheduling Optional Tasks with Explanation & \href{works/SchuttFS13.pdf}{Yes} & \cite{SchuttFS13} & 2013 & CP 2013 & 17 & 10 & 20 & \ref{b:SchuttFS13} & \ref{c:SchuttFS13}\\
SchuttFS13a \href{https://doi.org/10.1007/978-3-642-38171-3\_16}{SchuttFS13a} & \hyperref[auth:a124]{A. Schutt}, \hyperref[auth:a154]{T. Feydy}, \hyperref[auth:a125]{Peter J. Stuckey} & Explaining Time-Table-Edge-Finding Propagation for the Cumulative Resource Constraint & \href{works/SchuttFS13a.pdf}{Yes} & \cite{SchuttFS13a} & 2013 & CPAIOR 2013 & 17 & 20 & 27 & \ref{b:SchuttFS13a} & \ref{c:SchuttFS13a}\\
SchuttFSW13 \href{https://doi.org/10.1007/s10951-012-0285-x}{SchuttFSW13} & \hyperref[auth:a124]{A. Schutt}, \hyperref[auth:a154]{T. Feydy}, \hyperref[auth:a125]{Peter J. Stuckey}, \hyperref[auth:a155]{Mark G. Wallace} & Solving RCPSP/max by lazy clause generation & \href{works/SchuttFSW13.pdf}{Yes} & \cite{SchuttFSW13} & 2013 & J. Sched. & 17 & 43 & 23 & \ref{b:SchuttFSW13} & \ref{c:SchuttFSW13}\\
GuSW12 \href{https://doi.org/10.1007/978-3-642-33558-7\_55}{GuSW12} & \hyperref[auth:a341]{H. Gu}, \hyperref[auth:a125]{Peter J. Stuckey}, \hyperref[auth:a155]{Mark G. Wallace} & Maximising the Net Present Value of Large Resource-Constrained Projects & \href{works/GuSW12.pdf}{Yes} & \cite{GuSW12} & 2012 & CP 2012 & 15 & 5 & 20 & \ref{b:GuSW12} & \ref{c:GuSW12}\\
SchuttCSW12 \href{https://doi.org/10.1007/978-3-642-29828-8\_24}{SchuttCSW12} & \hyperref[auth:a124]{A. Schutt}, \hyperref[auth:a348]{G. Chu}, \hyperref[auth:a125]{Peter J. Stuckey}, \hyperref[auth:a155]{Mark G. Wallace} & Maximising the Net Present Value for Resource-Constrained Project Scheduling & \href{works/SchuttCSW12.pdf}{Yes} & \cite{SchuttCSW12} & 2012 & CPAIOR 2012 & 17 & 18 & 21 & \ref{b:SchuttCSW12} & \ref{c:SchuttCSW12}\\
BandaSC11 \href{https://doi.org/10.1287/ijoc.1090.0378}{BandaSC11} & \hyperref[auth:a807]{Maria Garcia de la Banda}, \hyperref[auth:a125]{Peter J. Stuckey}, \hyperref[auth:a348]{G. Chu} & Solving Talent Scheduling with Dynamic Programming & No & \cite{BandaSC11} & 2011 & {INFORMS} J. Comput. & 18 & 24 & 17 & No & \ref{c:BandaSC11}\\
SchuttFSW11 \href{https://doi.org/10.1007/s10601-010-9103-2}{SchuttFSW11} & \hyperref[auth:a124]{A. Schutt}, \hyperref[auth:a154]{T. Feydy}, \hyperref[auth:a125]{Peter J. Stuckey}, \hyperref[auth:a155]{Mark G. Wallace} & Explaining the cumulative propagator & \href{works/SchuttFSW11.pdf}{Yes} & \cite{SchuttFSW11} & 2011 & Constraints An Int. J. & 33 & 57 & 23 & \ref{b:SchuttFSW11} & \ref{c:SchuttFSW11}\\
abs-1009-0347 \href{http://arxiv.org/abs/1009.0347}{abs-1009-0347} & \hyperref[auth:a124]{A. Schutt}, \hyperref[auth:a154]{T. Feydy}, \hyperref[auth:a125]{Peter J. Stuckey}, \hyperref[auth:a155]{Mark G. Wallace} & Solving the Resource Constrained Project Scheduling Problem with Generalized Precedences by Lazy Clause Generation & \href{works/abs-1009-0347.pdf}{Yes} & \cite{abs-1009-0347} & 2010 & CoRR & 37 & 0 & 0 & \ref{b:abs-1009-0347} & \ref{c:abs-1009-0347}\\
OhrimenkoSC09 \href{http://dx.doi.org/10.1007/s10601-008-9064-x}{OhrimenkoSC09} & \hyperref[auth:a875]{O. Ohrimenko}, \hyperref[auth:a125]{Peter J. Stuckey}, \hyperref[auth:a876]{M. Codish} & Propagation via lazy clause generation & \href{works/OhrimenkoSC09.pdf}{Yes} & \cite{OhrimenkoSC09} & 2009 & Constraints & 35 & 127 & 15 & \ref{b:OhrimenkoSC09} & \ref{c:OhrimenkoSC09}\\
SchuttFSW09 \href{https://doi.org/10.1007/978-3-642-04244-7\_58}{SchuttFSW09} & \hyperref[auth:a124]{A. Schutt}, \hyperref[auth:a154]{T. Feydy}, \hyperref[auth:a125]{Peter J. Stuckey}, \hyperref[auth:a117]{M. Wallace} & Why Cumulative Decomposition Is Not as Bad as It Sounds & \href{works/SchuttFSW09.pdf}{Yes} & \cite{SchuttFSW09} & 2009 & CP 2009 & 16 & 34 & 11 & \ref{b:SchuttFSW09} & \ref{c:SchuttFSW09}\\
NethercoteSBBDT07 \href{https://doi.org/10.1007/978-3-540-74970-7\_38}{NethercoteSBBDT07} & \hyperref[auth:a867]{N. Nethercote}, \hyperref[auth:a125]{Peter J. Stuckey}, \hyperref[auth:a868]{R. Becket}, \hyperref[auth:a869]{S. Brand}, \hyperref[auth:a870]{Gregory J. Duck}, \hyperref[auth:a871]{G. Tack} & MiniZinc: Towards a Standard {CP} Modelling Language & \href{works/NethercoteSBBDT07.pdf}{Yes} & \cite{NethercoteSBBDT07} & 2007 & CP 2007 & 15 & 344 & 5 & \ref{b:NethercoteSBBDT07} & \ref{c:NethercoteSBBDT07}\\
\end{longtable}
}

\subsection{Works by Michele Lombardi}
\label{sec:a142}
{\scriptsize
\begin{longtable}{>{\raggedright\arraybackslash}p{3cm}>{\raggedright\arraybackslash}p{6cm}>{\raggedright\arraybackslash}p{6.5cm}rrrp{2.5cm}rrrrr}
\rowcolor{white}\caption{Works from bibtex (Total 22)}\\ \toprule
\rowcolor{white}Key & Authors & Title & LC & Cite & Year & \shortstack{Conference\\/Journal} & Pages & \shortstack{Nr\\Cites} & \shortstack{Nr\\Refs} & b & c \\ \midrule\endhead
\bottomrule
\endfoot
BorghesiBLMB18 \href{https://doi.org/10.1016/j.suscom.2018.05.007}{BorghesiBLMB18} & \hyperref[auth:a231]{A. Borghesi}, \hyperref[auth:a230]{A. Bartolini}, \hyperref[auth:a142]{M. Lombardi}, \hyperref[auth:a143]{M. Milano}, \hyperref[auth:a247]{L. Benini} & Scheduling-based power capping in high performance computing systems & \href{works/BorghesiBLMB18.pdf}{Yes} & \cite{BorghesiBLMB18} & 2018 & Sustain. Comput. Informatics Syst. & 13 & 11 & 22 & \ref{b:BorghesiBLMB18} & \ref{c:BorghesiBLMB18}\\
CauwelaertLS18 \href{https://doi.org/10.1007/s10601-017-9277-y}{CauwelaertLS18} & \hyperref[auth:a206]{Sascha Van Cauwelaert}, \hyperref[auth:a142]{M. Lombardi}, \hyperref[auth:a147]{P. Schaus} & How efficient is a global constraint in practice? - {A} fair experimental framework & \href{works/CauwelaertLS18.pdf}{Yes} & \cite{CauwelaertLS18} & 2018 & Constraints An Int. J. & 36 & 2 & 39 & \ref{b:CauwelaertLS18} & \ref{c:CauwelaertLS18}\\
BonfiettiZLM16 \href{https://doi.org/10.1007/978-3-319-44953-1\_8}{BonfiettiZLM16} & \hyperref[auth:a203]{A. Bonfietti}, \hyperref[auth:a204]{A. Zanarini}, \hyperref[auth:a142]{M. Lombardi}, \hyperref[auth:a143]{M. Milano} & The Multirate Resource Constraint & \href{works/BonfiettiZLM16.pdf}{Yes} & \cite{BonfiettiZLM16} & 2016 & CP 2016 & 17 & 0 & 11 & \ref{b:BonfiettiZLM16} & \ref{c:BonfiettiZLM16}\\
BridiBLMB16 \href{https://doi.org/10.1109/TPDS.2016.2516997}{BridiBLMB16} & \hyperref[auth:a232]{T. Bridi}, \hyperref[auth:a230]{A. Bartolini}, \hyperref[auth:a142]{M. Lombardi}, \hyperref[auth:a143]{M. Milano}, \hyperref[auth:a247]{L. Benini} & A Constraint Programming Scheduler for Heterogeneous High-Performance Computing Machines & \href{works/BridiBLMB16.pdf}{Yes} & \cite{BridiBLMB16} & 2016 & {IEEE} Trans. Parallel Distributed Syst. & 14 & 17 & 22 & \ref{b:BridiBLMB16} & \ref{c:BridiBLMB16}\\
BridiLBBM16 \href{https://doi.org/10.3233/978-1-61499-672-9-1598}{BridiLBBM16} & \hyperref[auth:a232]{T. Bridi}, \hyperref[auth:a142]{M. Lombardi}, \hyperref[auth:a230]{A. Bartolini}, \hyperref[auth:a247]{L. Benini}, \hyperref[auth:a143]{M. Milano} & {DARDIS:} Distributed And Randomized DIspatching and Scheduling & \href{works/BridiLBBM16.pdf}{Yes} & \cite{BridiLBBM16} & 2016 & ECAI 2016 & 2 & 0 & 0 & \ref{b:BridiLBBM16} & \ref{c:BridiLBBM16}\\
LombardiBM15 \href{https://doi.org/10.1007/978-3-319-23219-5\_20}{LombardiBM15} & \hyperref[auth:a142]{M. Lombardi}, \hyperref[auth:a203]{A. Bonfietti}, \hyperref[auth:a143]{M. Milano} & Deterministic Estimation of the Expected Makespan of a {POS} Under Duration Uncertainty & \href{works/LombardiBM15.pdf}{Yes} & \cite{LombardiBM15} & 2015 & CP 2015 & 16 & 0 & 8 & \ref{b:LombardiBM15} & \ref{c:LombardiBM15}\\
BartoliniBBLM14 \href{https://doi.org/10.1007/978-3-319-10428-7\_55}{BartoliniBBLM14} & \hyperref[auth:a230]{A. Bartolini}, \hyperref[auth:a231]{A. Borghesi}, \hyperref[auth:a232]{T. Bridi}, \hyperref[auth:a142]{M. Lombardi}, \hyperref[auth:a143]{M. Milano} & Proactive Workload Dispatching on the {EURORA} Supercomputer & \href{works/BartoliniBBLM14.pdf}{Yes} & \cite{BartoliniBBLM14} & 2014 & CP 2014 & 16 & 12 & 3 & \ref{b:BartoliniBBLM14} & \ref{c:BartoliniBBLM14}\\
BonfiettiLBM14 \href{https://doi.org/10.1016/j.artint.2013.09.006}{BonfiettiLBM14} & \hyperref[auth:a203]{A. Bonfietti}, \hyperref[auth:a142]{M. Lombardi}, \hyperref[auth:a247]{L. Benini}, \hyperref[auth:a143]{M. Milano} & {CROSS} cyclic resource-constrained scheduling solver & \href{works/BonfiettiLBM14.pdf}{Yes} & \cite{BonfiettiLBM14} & 2014 & Artif. Intell. & 28 & 8 & 15 & \ref{b:BonfiettiLBM14} & \ref{c:BonfiettiLBM14}\\
BonfiettiLM14 \href{https://doi.org/10.1007/978-3-319-07046-9\_15}{BonfiettiLM14} & \hyperref[auth:a203]{A. Bonfietti}, \hyperref[auth:a142]{M. Lombardi}, \hyperref[auth:a143]{M. Milano} & Disregarding Duration Uncertainty in Partial Order Schedules? Yes, We Can! & \href{works/BonfiettiLM14.pdf}{Yes} & \cite{BonfiettiLM14} & 2014 & CPAIOR 2014 & 16 & 3 & 12 & \ref{b:BonfiettiLM14} & \ref{c:BonfiettiLM14}\\
BonfiettiLM13 \href{http://www.aaai.org/ocs/index.php/ICAPS/ICAPS13/paper/view/6050}{BonfiettiLM13} & \hyperref[auth:a203]{A. Bonfietti}, \hyperref[auth:a142]{M. Lombardi}, \hyperref[auth:a143]{M. Milano} & De-Cycling Cyclic Scheduling Problems & \href{works/BonfiettiLM13.pdf}{Yes} & \cite{BonfiettiLM13} & 2013 & ICAPS 2013 & 5 & 0 & 0 & \ref{b:BonfiettiLM13} & \ref{c:BonfiettiLM13}\\
LombardiM13 \href{http://www.aaai.org/ocs/index.php/ICAPS/ICAPS13/paper/view/6052}{LombardiM13} & \hyperref[auth:a142]{M. Lombardi}, \hyperref[auth:a143]{M. Milano} & A Min-Flow Algorithm for Minimal Critical Set Detection in Resource Constrained Project Scheduling & \href{works/LombardiM13.pdf}{Yes} & \cite{LombardiM13} & 2013 & ICAPS 2013 & 2 & 0 & 0 & \ref{b:LombardiM13} & \ref{c:LombardiM13}\\
BonfiettiLBM12 \href{https://doi.org/10.1007/978-3-642-29828-8\_6}{BonfiettiLBM12} & \hyperref[auth:a203]{A. Bonfietti}, \hyperref[auth:a142]{M. Lombardi}, \hyperref[auth:a247]{L. Benini}, \hyperref[auth:a143]{M. Milano} & Global Cyclic Cumulative Constraint & \href{works/BonfiettiLBM12.pdf}{Yes} & \cite{BonfiettiLBM12} & 2012 & CPAIOR 2012 & 16 & 2 & 11 & \ref{b:BonfiettiLBM12} & \ref{c:BonfiettiLBM12}\\
LombardiM12 \href{https://doi.org/10.1007/s10601-011-9115-6}{LombardiM12} & \hyperref[auth:a142]{M. Lombardi}, \hyperref[auth:a143]{M. Milano} & Optimal methods for resource allocation and scheduling: a cross-disciplinary survey & \href{works/LombardiM12.pdf}{Yes} & \cite{LombardiM12} & 2012 & Constraints An Int. J. & 35 & 39 & 68 & \ref{b:LombardiM12} & \ref{c:LombardiM12}\\
LombardiM12a \href{https://doi.org/10.1016/j.artint.2011.12.001}{LombardiM12a} & \hyperref[auth:a142]{M. Lombardi}, \hyperref[auth:a143]{M. Milano} & A min-flow algorithm for Minimal Critical Set detection in Resource Constrained Project Scheduling & \href{works/LombardiM12a.pdf}{Yes} & \cite{LombardiM12a} & 2012 & Artif. Intell. & 10 & 3 & 13 & \ref{b:LombardiM12a} & \ref{c:LombardiM12a}\\
BeniniLMR11 \href{https://doi.org/10.1007/s10479-010-0718-x}{BeniniLMR11} & \hyperref[auth:a247]{L. Benini}, \hyperref[auth:a142]{M. Lombardi}, \hyperref[auth:a143]{M. Milano}, \hyperref[auth:a727]{M. Ruggiero} & Optimal resource allocation and scheduling for the {CELL} {BE} platform & \href{works/BeniniLMR11.pdf}{Yes} & \cite{BeniniLMR11} & 2011 & Ann. Oper. Res. & 27 & 18 & 16 & \ref{b:BeniniLMR11} & \ref{c:BeniniLMR11}\\
BonfiettiLBM11 \href{https://doi.org/10.1007/978-3-642-23786-7\_12}{BonfiettiLBM11} & \hyperref[auth:a203]{A. Bonfietti}, \hyperref[auth:a142]{M. Lombardi}, \hyperref[auth:a247]{L. Benini}, \hyperref[auth:a143]{M. Milano} & A Constraint Based Approach to Cyclic {RCPSP} & \href{works/BonfiettiLBM11.pdf}{Yes} & \cite{BonfiettiLBM11} & 2011 & CP 2011 & 15 & 3 & 14 & \ref{b:BonfiettiLBM11} & \ref{c:BonfiettiLBM11}\\
LombardiBMB11 \href{https://doi.org/10.1007/978-3-642-21311-3\_14}{LombardiBMB11} & \hyperref[auth:a142]{M. Lombardi}, \hyperref[auth:a203]{A. Bonfietti}, \hyperref[auth:a143]{M. Milano}, \hyperref[auth:a247]{L. Benini} & Precedence Constraint Posting for Cyclic Scheduling Problems & \href{works/LombardiBMB11.pdf}{Yes} & \cite{LombardiBMB11} & 2011 & CPAIOR 2011 & 17 & 1 & 13 & \ref{b:LombardiBMB11} & \ref{c:LombardiBMB11}\\
Lombardi10 \href{http://amsdottorato.unibo.it/2961/}{Lombardi10} & \hyperref[auth:a142]{M. Lombardi} & Hybrid Methods for Resource Allocation and Scheduling Problems in Deterministic and Stochastic Environments & \href{works/Lombardi10.pdf}{Yes} & \cite{Lombardi10} & 2010 & University of Bologna, Italy & 175 & 0 & 0 & \ref{b:Lombardi10} & \ref{c:Lombardi10}\\
LombardiM10 \href{https://doi.org/10.1007/978-3-642-15396-9\_32}{LombardiM10} & \hyperref[auth:a142]{M. Lombardi}, \hyperref[auth:a143]{M. Milano} & Constraint Based Scheduling to Deal with Uncertain Durations and Self-Timed Execution & \href{works/LombardiM10.pdf}{Yes} & \cite{LombardiM10} & 2010 & CP 2010 & 15 & 1 & 11 & \ref{b:LombardiM10} & \ref{c:LombardiM10}\\
LombardiM10a \href{https://doi.org/10.1016/j.artint.2010.02.004}{LombardiM10a} & \hyperref[auth:a142]{M. Lombardi}, \hyperref[auth:a143]{M. Milano} & Allocation and scheduling of Conditional Task Graphs & \href{works/LombardiM10a.pdf}{Yes} & \cite{LombardiM10a} & 2010 & Artif. Intell. & 30 & 8 & 24 & \ref{b:LombardiM10a} & \ref{c:LombardiM10a}\\
LombardiM09 \href{https://doi.org/10.1007/978-3-642-04244-7\_45}{LombardiM09} & \hyperref[auth:a142]{M. Lombardi}, \hyperref[auth:a143]{M. Milano} & A Precedence Constraint Posting Approach for the {RCPSP} with Time Lags and Variable Durations & \href{works/LombardiM09.pdf}{Yes} & \cite{LombardiM09} & 2009 & CP 2009 & 15 & 7 & 12 & \ref{b:LombardiM09} & \ref{c:LombardiM09}\\
HoeveGSL07 \href{http://www.aaai.org/Library/AAAI/2007/aaai07-291.php}{HoeveGSL07} & \hyperref[auth:a651]{Willem Jan van Hoeve}, \hyperref[auth:a652]{Carla P. Gomes}, \hyperref[auth:a653]{B. Selman}, \hyperref[auth:a142]{M. Lombardi} & Optimal Multi-Agent Scheduling with Constraint Programming & \href{works/HoeveGSL07.pdf}{Yes} & \cite{HoeveGSL07} & 2007 & AAAI 2007 & 6 & 0 & 0 & \ref{b:HoeveGSL07} & \ref{c:HoeveGSL07}\\
\end{longtable}
}

\subsection{Works by Emmanuel Hebrard}
\label{sec:a1}
{\scriptsize
\begin{longtable}{>{\raggedright\arraybackslash}p{3cm}>{\raggedright\arraybackslash}p{6cm}>{\raggedright\arraybackslash}p{6.5cm}rrrp{2.5cm}rrrrr}
\rowcolor{white}\caption{Works from bibtex (Total 17)}\\ \toprule
\rowcolor{white}Key & Authors & Title & LC & Cite & Year & \shortstack{Conference\\/Journal} & Pages & \shortstack{Nr\\Cites} & \shortstack{Nr\\Refs} & b & c \\ \midrule\endhead
\bottomrule
\endfoot
JuvinHHL23 \href{https://doi.org/10.4230/LIPIcs.CP.2023.19}{JuvinHHL23} & \hyperref[auth:a0]{C. Juvin}, \hyperref[auth:a1]{E. Hebrard}, \hyperref[auth:a2]{L. Houssin}, \hyperref[auth:a3]{P. Lopez} & An Efficient Constraint Programming Approach to Preemptive Job Shop Scheduling & \href{works/JuvinHHL23.pdf}{Yes} & \cite{JuvinHHL23} & 2023 & CP 2023 & 16 & 0 & 0 & \ref{b:JuvinHHL23} & \ref{c:JuvinHHL23}\\
HebrardALLCMR22 \href{https://doi.org/10.24963/ijcai.2022/643}{HebrardALLCMR22} & \hyperref[auth:a1]{E. Hebrard}, \hyperref[auth:a6]{C. Artigues}, \hyperref[auth:a3]{P. Lopez}, \hyperref[auth:a796]{A. Lusson}, \hyperref[auth:a797]{Steve A. Chien}, \hyperref[auth:a798]{A. Maillard}, \hyperref[auth:a799]{Gregg R. Rabideau} & An Efficient Approach to Data Transfer Scheduling for Long Range Space Exploration & \href{works/HebrardALLCMR22.pdf}{Yes} & \cite{HebrardALLCMR22} & 2022 & IJCAI 2022 & 7 & 0 & 0 & \ref{b:HebrardALLCMR22} & \ref{c:HebrardALLCMR22}\\
AntuoriHHEN21 \href{https://doi.org/10.4230/LIPIcs.CP.2021.14}{AntuoriHHEN21} & \hyperref[auth:a53]{V. Antuori}, \hyperref[auth:a1]{E. Hebrard}, \hyperref[auth:a54]{M. Huguet}, \hyperref[auth:a55]{S. Essodaigui}, \hyperref[auth:a56]{A. Nguyen} & Combining Monte Carlo Tree Search and Depth First Search Methods for a Car Manufacturing Workshop Scheduling Problem & \href{works/AntuoriHHEN21.pdf}{Yes} & \cite{AntuoriHHEN21} & 2021 & CP 2021 & 16 & 0 & 0 & \ref{b:AntuoriHHEN21} & \ref{c:AntuoriHHEN21}\\
ArtiguesHQT21 \href{https://doi.org/10.5220/0010190101290136}{ArtiguesHQT21} & \hyperref[auth:a6]{C. Artigues}, \hyperref[auth:a1]{E. Hebrard}, \hyperref[auth:a800]{A. Quilliot}, \hyperref[auth:a801]{H. Toussaint} & Multi-Mode {RCPSP} with Safety Margin Maximization: Models and Algorithms & No & \cite{ArtiguesHQT21} & 2021 & ICORES 2021 & 8 & 0 & 0 & No & \ref{c:ArtiguesHQT21}\\
AntuoriHHEN20 \href{https://doi.org/10.1007/978-3-030-58475-7\_38}{AntuoriHHEN20} & \hyperref[auth:a53]{V. Antuori}, \hyperref[auth:a1]{E. Hebrard}, \hyperref[auth:a54]{M. Huguet}, \hyperref[auth:a55]{S. Essodaigui}, \hyperref[auth:a56]{A. Nguyen} & Leveraging Reinforcement Learning, Constraint Programming and Local Search: {A} Case Study in Car Manufacturing & \href{works/AntuoriHHEN20.pdf}{Yes} & \cite{AntuoriHHEN20} & 2020 & CP 2020 & 16 & 3 & 8 & \ref{b:AntuoriHHEN20} & \ref{c:AntuoriHHEN20}\\
GodetLHS20 \href{https://doi.org/10.1609/aaai.v34i02.5510}{GodetLHS20} & \hyperref[auth:a476]{A. Godet}, \hyperref[auth:a246]{X. Lorca}, \hyperref[auth:a1]{E. Hebrard}, \hyperref[auth:a126]{G. Simonin} & Using Approximation within Constraint Programming to Solve the Parallel Machine Scheduling Problem with Additional Unit Resources & \href{works/GodetLHS20.pdf}{Yes} & \cite{GodetLHS20} & 2020 & AAAI 2020 & 8 & 1 & 0 & \ref{b:GodetLHS20} & \ref{c:GodetLHS20}\\
HebrardHJMPV16 \href{https://doi.org/10.1016/j.dam.2016.07.003}{HebrardHJMPV16} & \hyperref[auth:a1]{E. Hebrard}, \hyperref[auth:a54]{M. Huguet}, \hyperref[auth:a802]{N. Jozefowiez}, \hyperref[auth:a798]{A. Maillard}, \hyperref[auth:a21]{C. Pralet}, \hyperref[auth:a174]{G. Verfaillie} & Approximation of the parallel machine scheduling problem with additional unit resources & \href{works/HebrardHJMPV16.pdf}{Yes} & \cite{HebrardHJMPV16} & 2016 & Discret. Appl. Math. & 10 & 9 & 8 & \ref{b:HebrardHJMPV16} & \ref{c:HebrardHJMPV16}\\
GrimesH15 \href{https://doi.org/10.1287/ijoc.2014.0625}{GrimesH15} & \hyperref[auth:a182]{D. Grimes}, \hyperref[auth:a1]{E. Hebrard} & Solving Variants of the Job Shop Scheduling Problem Through Conflict-Directed Search & No & \cite{GrimesH15} & 2015 & {INFORMS} J. Comput. & 17 & 12 & 41 & No & \ref{c:GrimesH15}\\
SialaAH15 \href{https://doi.org/10.1007/978-3-319-23219-5\_28}{SialaAH15} & \hyperref[auth:a129]{M. Siala}, \hyperref[auth:a6]{C. Artigues}, \hyperref[auth:a1]{E. Hebrard} & Two Clause Learning Approaches for Disjunctive Scheduling & \href{works/SialaAH15.pdf}{Yes} & \cite{SialaAH15} & 2015 & CP 2015 & 10 & 4 & 17 & \ref{b:SialaAH15} & \ref{c:SialaAH15}\\
SimoninAHL15 \href{https://doi.org/10.1007/s10601-014-9169-3}{SimoninAHL15} & \hyperref[auth:a126]{G. Simonin}, \hyperref[auth:a6]{C. Artigues}, \hyperref[auth:a1]{E. Hebrard}, \hyperref[auth:a3]{P. Lopez} & Scheduling scientific experiments for comet exploration & \href{works/SimoninAHL15.pdf}{Yes} & \cite{SimoninAHL15} & 2015 & Constraints An Int. J. & 23 & 4 & 5 & \ref{b:SimoninAHL15} & \ref{c:SimoninAHL15}\\
BessiereHMQW14 \href{https://doi.org/10.1007/978-3-319-07046-9\_23}{BessiereHMQW14} & \hyperref[auth:a333]{C. Bessiere}, \hyperref[auth:a1]{E. Hebrard}, \hyperref[auth:a334]{M. M{\'{e}}nard}, \hyperref[auth:a37]{C. Quimper}, \hyperref[auth:a278]{T. Walsh} & Buffered Resource Constraint: Algorithms and Complexity & \href{works/BessiereHMQW14.pdf}{Yes} & \cite{BessiereHMQW14} & 2014 & CPAIOR 2014 & 16 & 1 & 3 & \ref{b:BessiereHMQW14} & \ref{c:BessiereHMQW14}\\
BillautHL12 \href{https://doi.org/10.1007/978-3-642-29828-8\_5}{BillautHL12} & \hyperref[auth:a342]{J. Billaut}, \hyperref[auth:a1]{E. Hebrard}, \hyperref[auth:a3]{P. Lopez} & Complete Characterization of Near-Optimal Sequences for the Two-Machine Flow Shop Scheduling Problem & \href{works/BillautHL12.pdf}{Yes} & \cite{BillautHL12} & 2012 & CPAIOR 2012 & 15 & 1 & 19 & \ref{b:BillautHL12} & \ref{c:BillautHL12}\\
SimoninAHL12 \href{https://doi.org/10.1007/978-3-642-33558-7\_5}{SimoninAHL12} & \hyperref[auth:a126]{G. Simonin}, \hyperref[auth:a6]{C. Artigues}, \hyperref[auth:a1]{E. Hebrard}, \hyperref[auth:a3]{P. Lopez} & Scheduling Scientific Experiments on the Rosetta/Philae Mission & \href{works/SimoninAHL12.pdf}{Yes} & \cite{SimoninAHL12} & 2012 & CP 2012 & 15 & 3 & 8 & \ref{b:SimoninAHL12} & \ref{c:SimoninAHL12}\\
GrimesH11 \href{https://doi.org/10.1007/978-3-642-23786-7\_28}{GrimesH11} & \hyperref[auth:a182]{D. Grimes}, \hyperref[auth:a1]{E. Hebrard} & Models and Strategies for Variants of the Job Shop Scheduling Problem & \href{works/GrimesH11.pdf}{Yes} & \cite{GrimesH11} & 2011 & CP 2011 & 17 & 5 & 18 & \ref{b:GrimesH11} & \ref{c:GrimesH11}\\
GrimesH10 \href{https://doi.org/10.1007/978-3-642-13520-0\_19}{GrimesH10} & \hyperref[auth:a182]{D. Grimes}, \hyperref[auth:a1]{E. Hebrard} & Job Shop Scheduling with Setup Times and Maximal Time-Lags: {A} Simple Constraint Programming Approach & \href{works/GrimesH10.pdf}{Yes} & \cite{GrimesH10} & 2010 & CPAIOR 2010 & 15 & 13 & 20 & \ref{b:GrimesH10} & \ref{c:GrimesH10}\\
GrimesHM09 \href{https://doi.org/10.1007/978-3-642-04244-7\_33}{GrimesHM09} & \hyperref[auth:a182]{D. Grimes}, \hyperref[auth:a1]{E. Hebrard}, \hyperref[auth:a82]{A. Malapert} & Closing the Open Shop: Contradicting Conventional Wisdom & \href{works/GrimesHM09.pdf}{Yes} & \cite{GrimesHM09} & 2009 & CP 2009 & 9 & 15 & 12 & \ref{b:GrimesHM09} & \ref{c:GrimesHM09}\\
HebrardTW05 \href{https://doi.org/10.1007/11564751\_117}{HebrardTW05} & \hyperref[auth:a1]{E. Hebrard}, \hyperref[auth:a277]{P. Tyler}, \hyperref[auth:a278]{T. Walsh} & Computing Super-Schedules & \href{works/HebrardTW05.pdf}{Yes} & \cite{HebrardTW05} & 2005 & CP 2005 & 1 & 0 & 3 & \ref{b:HebrardTW05} & \ref{c:HebrardTW05}\\
\end{longtable}
}

\subsection{Works by John N. Hooker}
\label{sec:a161}
{\scriptsize
\begin{longtable}{>{\raggedright\arraybackslash}p{3cm}>{\raggedright\arraybackslash}p{6cm}>{\raggedright\arraybackslash}p{6.5cm}rrrp{2.5cm}rrrrr}
\rowcolor{white}\caption{Works from bibtex (Total 14)}\\ \toprule
\rowcolor{white}Key & Authors & Title & LC & Cite & Year & \shortstack{Conference\\/Journal} & Pages & \shortstack{Nr\\Cites} & \shortstack{Nr\\Refs} & b & c \\ \midrule\endhead
\bottomrule
\endfoot
Hooker19 \href{https://ideas.repec.org/h/spr/spochp/978-3-030-22788-3_1.html}{Hooker19} & \hyperref[auth:a161]{John N. Hooker} & {Logic-Based Benders Decomposition for Large-Scale Optimization} & No & \cite{Hooker19} & 2019 & {Large Scale Optimization in Supply Chains and Smart Manufacturing} & 26 & 8 & 0 & No & \ref{c:Hooker19}\\
HookerH18 \href{https://doi.org/10.1007/s10601-017-9280-3}{HookerH18} & \hyperref[auth:a161]{John N. Hooker}, \hyperref[auth:a651]{Willem Jan van Hoeve} & Constraint programming and operations research & \href{works/HookerH18.pdf}{Yes} & \cite{HookerH18} & 2018 & Constraints An Int. J. & 24 & 12 & 189 & \ref{b:HookerH18} & \ref{c:HookerH18}\\
Hooker17 \href{https://doi.org/10.1007/978-3-319-66158-2\_36}{Hooker17} & \hyperref[auth:a161]{John N. Hooker} & Job Sequencing Bounds from Decision Diagrams & \href{works/Hooker17.pdf}{Yes} & \cite{Hooker17} & 2017 & CP 2017 & 14 & 6 & 24 & \ref{b:Hooker17} & \ref{c:Hooker17}\\
HechingH16 \href{https://doi.org/10.1007/978-3-319-33954-2\_14}{HechingH16} & \hyperref[auth:a322]{Aliza R. Heching}, \hyperref[auth:a161]{John N. Hooker} & Scheduling Home Hospice Care with Logic-Based Benders Decomposition & \href{works/HechingH16.pdf}{Yes} & \cite{HechingH16} & 2016 & CPAIOR 2016 & 11 & 10 & 0 & \ref{b:HechingH16} & \ref{c:HechingH16}\\
CireCH13 \href{https://doi.org/10.1007/978-3-642-38171-3\_22}{CireCH13} & \hyperref[auth:a158]{Andr{\'{e}} A. Cir{\'{e}}}, \hyperref[auth:a340]{E. Coban}, \hyperref[auth:a161]{John N. Hooker} & Mixed Integer Programming vs. Logic-Based Benders Decomposition for Planning and Scheduling & \href{works/CireCH13.pdf}{Yes} & \cite{CireCH13} & 2013 & CPAIOR 2013 & 7 & 3 & 23 & \ref{b:CireCH13} & \ref{c:CireCH13}\\
CobanH10 \href{https://doi.org/10.1007/978-3-642-13520-0\_11}{CobanH10} & \hyperref[auth:a340]{E. Coban}, \hyperref[auth:a161]{John N. Hooker} & Single-Facility Scheduling over Long Time Horizons by Logic-Based Benders Decomposition & \href{works/CobanH10.pdf}{Yes} & \cite{CobanH10} & 2010 & CPAIOR 2010 & 5 & 9 & 9 & \ref{b:CobanH10} & \ref{c:CobanH10}\\
Hooker07 \href{http://dx.doi.org/10.1287/opre.1060.0371}{Hooker07} & \hyperref[auth:a161]{John N. Hooker} & Planning and Scheduling by Logic-Based Benders Decomposition & No & \cite{Hooker07} & 2007 & Operations Research & null & 181 & 19 & No & \ref{c:Hooker07}\\
Hooker06 \href{https://doi.org/10.1007/s10601-006-8060-2}{Hooker06} & \hyperref[auth:a161]{John N. Hooker} & An Integrated Method for Planning and Scheduling to Minimize Tardiness & \href{works/Hooker06.pdf}{Yes} & \cite{Hooker06} & 2006 & Constraints An Int. J. & 19 & 19 & 13 & \ref{b:Hooker06} & \ref{c:Hooker06}\\
Hooker05 \href{https://doi.org/10.1007/s10601-005-2812-2}{Hooker05} & \hyperref[auth:a161]{John N. Hooker} & A Hybrid Method for the Planning and Scheduling & \href{works/Hooker05.pdf}{Yes} & \cite{Hooker05} & 2005 & Constraints An Int. J. & 17 & 68 & 11 & \ref{b:Hooker05} & \ref{c:Hooker05}\\
Hooker05a \href{https://doi.org/10.1007/11564751\_25}{Hooker05a} & \hyperref[auth:a161]{John N. Hooker} & Planning and Scheduling to Minimize Tardiness & \href{works/Hooker05a.pdf}{Yes} & \cite{Hooker05a} & 2005 & CP 2005 & 14 & 30 & 10 & \ref{b:Hooker05a} & \ref{c:Hooker05a}\\
Hooker04 \href{https://doi.org/10.1007/978-3-540-30201-8\_24}{Hooker04} & \hyperref[auth:a161]{John N. Hooker} & A Hybrid Method for Planning and Scheduling & \href{works/Hooker04.pdf}{Yes} & \cite{Hooker04} & 2004 & CP 2004 & 12 & 39 & 9 & \ref{b:Hooker04} & \ref{c:Hooker04}\\
HookerO03 \href{http://dx.doi.org/10.1007/s10107-003-0375-9}{HookerO03} & \hyperref[auth:a161]{John N. Hooker}, \hyperref[auth:a866]{G. Ottosson} & Logic-based Benders decomposition & \href{works/HookerO03.pdf}{Yes} & \cite{HookerO03} & 2003 & Mathematical Programming & 28 & 317 & 0 & \ref{b:HookerO03} & \ref{c:HookerO03}\\
HookerY02 \href{https://doi.org/10.1007/3-540-46135-3\_46}{HookerY02} & \hyperref[auth:a161]{John N. Hooker}, \hyperref[auth:a293]{H. Yan} & A Relaxation of the Cumulative Constraint & \href{works/HookerY02.pdf}{Yes} & \cite{HookerY02} & 2002 & CP 2002 & 5 & 8 & 7 & \ref{b:HookerY02} & \ref{c:HookerY02}\\
Hooker00 \href{http://dx.doi.org/10.1002/9781118033036}{Hooker00} & \hyperref[auth:a161]{John N. Hooker} & Logic Based Methods for Optimization: Combining Optimization and Constraint Satisfaction & No & \cite{Hooker00} & 2000 & Book & null & 185 & 0 & No & \ref{c:Hooker00}\\
\end{longtable}
}

\subsection{Works by Helmut Simonis}
\label{sec:a17}
{\scriptsize
\begin{longtable}{>{\raggedright\arraybackslash}p{3cm}>{\raggedright\arraybackslash}p{6cm}>{\raggedright\arraybackslash}p{6.5cm}rrrp{2.5cm}rrrrr}
\rowcolor{white}\caption{Works from bibtex (Total 14)}\\ \toprule
\rowcolor{white}Key & Authors & Title & LC & Cite & Year & \shortstack{Conference\\/Journal} & Pages & \shortstack{Nr\\Cites} & \shortstack{Nr\\Refs} & b & c \\ \midrule\endhead
\bottomrule
\endfoot
ArmstrongGOS22 \href{https://doi.org/10.1007/978-3-031-08011-1\_1}{ArmstrongGOS22} & \hyperref[auth:a14]{E. Armstrong}, \hyperref[auth:a15]{M. Garraffa}, \hyperref[auth:a16]{B. O'Sullivan}, \hyperref[auth:a17]{H. Simonis} & A Two-Phase Hybrid Approach for the Hybrid Flexible Flowshop with Transportation Times & \href{works/ArmstrongGOS22.pdf}{Yes} & \cite{ArmstrongGOS22} & 2022 & CPAIOR 2022 & 13 & 0 & 14 & \ref{b:ArmstrongGOS22} & \ref{c:ArmstrongGOS22}\\
ArmstrongGOS21 \href{https://doi.org/10.4230/LIPIcs.CP.2021.16}{ArmstrongGOS21} & \hyperref[auth:a14]{E. Armstrong}, \hyperref[auth:a15]{M. Garraffa}, \hyperref[auth:a16]{B. O'Sullivan}, \hyperref[auth:a17]{H. Simonis} & The Hybrid Flexible Flowshop with Transportation Times & \href{works/ArmstrongGOS21.pdf}{Yes} & \cite{ArmstrongGOS21} & 2021 & CP 2021 & 18 & 1 & 0 & \ref{b:ArmstrongGOS21} & \ref{c:ArmstrongGOS21}\\
AntunesABDEGGOL20 \href{https://doi.org/10.1142/S0218213020600076}{AntunesABDEGGOL20} & \hyperref[auth:a893]{M. Antunes}, \hyperref[auth:a894]{V. Armant}, \hyperref[auth:a222]{Kenneth N. Brown}, \hyperref[auth:a895]{Daniel A. Desmond}, \hyperref[auth:a896]{G. Escamocher}, \hyperref[auth:a897]{A. George}, \hyperref[auth:a182]{D. Grimes}, \hyperref[auth:a898]{M. O'Keeffe}, \hyperref[auth:a899]{Y. Lin}, \hyperref[auth:a16]{B. O'Sullivan}, \hyperref[auth:a900]{C. Ozturk}, \hyperref[auth:a901]{L. Quesada}, \hyperref[auth:a129]{M. Siala}, \hyperref[auth:a17]{H. Simonis}, \hyperref[auth:a837]{N. Wilson} & Assigning and Scheduling Service Visits in a Mixed Urban/Rural Setting & No & \cite{AntunesABDEGGOL20} & 2020 & Int. J. Artif. Intell. Tools & 31 & 0 & 16 & No & \ref{c:AntunesABDEGGOL20}\\
AntunesABDEGGOL18 \href{https://doi.org/10.1109/ICTAI.2018.00027}{AntunesABDEGGOL18} & \hyperref[auth:a893]{M. Antunes}, \hyperref[auth:a894]{V. Armant}, \hyperref[auth:a222]{Kenneth N. Brown}, \hyperref[auth:a895]{Daniel A. Desmond}, \hyperref[auth:a896]{G. Escamocher}, \hyperref[auth:a897]{A. George}, \hyperref[auth:a182]{D. Grimes}, \hyperref[auth:a898]{M. O'Keeffe}, \hyperref[auth:a899]{Y. Lin}, \hyperref[auth:a16]{B. O'Sullivan}, \hyperref[auth:a900]{C. Ozturk}, \hyperref[auth:a901]{L. Quesada}, \hyperref[auth:a129]{M. Siala}, \hyperref[auth:a17]{H. Simonis}, \hyperref[auth:a837]{N. Wilson} & Assigning and Scheduling Service Visits in a Mixed Urban/Rural Setting & No & \cite{AntunesABDEGGOL18} & 2018 & ICTAI 2018 & 8 & 1 & 24 & No & \ref{c:AntunesABDEGGOL18}\\
HurleyOS16 \href{https://doi.org/10.1007/978-3-319-50137-6\_15}{HurleyOS16} & \hyperref[auth:a902]{B. Hurley}, \hyperref[auth:a16]{B. O'Sullivan}, \hyperref[auth:a17]{H. Simonis} & {ICON} Loop Energy Show Case & \href{works/HurleyOS16.pdf}{Yes} & \cite{HurleyOS16} & 2016 & Data Mining and Constraint Programming - Foundations of a Cross-Disciplinary Approach & 14 & 0 & 16 & \ref{b:HurleyOS16} & \ref{c:HurleyOS16}\\
GrimesIOS14 \href{https://doi.org/10.1016/j.suscom.2014.08.009}{GrimesIOS14} & \hyperref[auth:a182]{D. Grimes}, \hyperref[auth:a183]{G. Ifrim}, \hyperref[auth:a16]{B. O'Sullivan}, \hyperref[auth:a17]{H. Simonis} & Analyzing the impact of electricity price forecasting on energy cost-aware scheduling & \href{works/GrimesIOS14.pdf}{Yes} & \cite{GrimesIOS14} & 2014 & Sustain. Comput. Informatics Syst. & 16 & 6 & 7 & \ref{b:GrimesIOS14} & \ref{c:GrimesIOS14}\\
IfrimOS12 \href{https://doi.org/10.1007/978-3-642-33558-7\_68}{IfrimOS12} & \hyperref[auth:a183]{G. Ifrim}, \hyperref[auth:a16]{B. O'Sullivan}, \hyperref[auth:a17]{H. Simonis} & Properties of Energy-Price Forecasts for Scheduling & \href{works/IfrimOS12.pdf}{Yes} & \cite{IfrimOS12} & 2012 & CP 2012 & 16 & 6 & 20 & \ref{b:IfrimOS12} & \ref{c:IfrimOS12}\\
Simonis07 \href{https://doi.org/10.1007/s10601-006-9011-7}{Simonis07} & \hyperref[auth:a17]{H. Simonis} & Models for Global Constraint Applications & \href{works/Simonis07.pdf}{Yes} & \cite{Simonis07} & 2007 & Constraints An Int. J. & 30 & 10 & 17 & \ref{b:Simonis07} & \ref{c:Simonis07}\\
SimonisCK00 \href{https://doi.org/10.1109/5254.820326}{SimonisCK00} & \hyperref[auth:a17]{H. Simonis}, \hyperref[auth:a903]{P. Charlier}, \hyperref[auth:a904]{P. Kay} & Constraint Handling in an Integrated Transportation Problem & No & \cite{SimonisCK00} & 2000 & {IEEE} Intell. Syst. & 7 & 11 & 5 & No & \ref{c:SimonisCK00}\\
Simonis99 \href{https://doi.org/10.1007/3-540-45406-3\_6}{Simonis99} & \hyperref[auth:a17]{H. Simonis} & Building Industrial Applications with Constraint Programming & \href{works/Simonis99.pdf}{Yes} & \cite{Simonis99} & 1999 & CCL'99 1999 & 39 & 5 & 18 & \ref{b:Simonis99} & \ref{c:Simonis99}\\
Simonis95 \href{https://doi.org/10.1007/3-540-60299-2\_42}{Simonis95} & \hyperref[auth:a17]{H. Simonis} & The {CHIP} System and Its Applications & \href{works/Simonis95.pdf}{Yes} & \cite{Simonis95} & 1995 & CP 1995 & 4 & 7 & 3 & \ref{b:Simonis95} & \ref{c:Simonis95}\\
Simonis95a \href{https://doi.org/10.1007/3-540-60794-3\_11}{Simonis95a} & \hyperref[auth:a17]{H. Simonis} & Application Development with the {CHIP} System & \href{works/Simonis95a.pdf}{Yes} & \cite{Simonis95a} & 1995 & CONTESSA 1995 & 21 & 1 & 12 & \ref{b:Simonis95a} & \ref{c:Simonis95a}\\
SimonisC95 \href{https://doi.org/10.1007/3-540-60299-2\_27}{SimonisC95} & \hyperref[auth:a17]{H. Simonis}, \hyperref[auth:a305]{T. Cornelissens} & Modelling Producer/Consumer Constraints & \href{works/SimonisC95.pdf}{Yes} & \cite{SimonisC95} & 1995 & CP 1995 & 14 & 17 & 8 & \ref{b:SimonisC95} & \ref{c:SimonisC95}\\
DincbasSH90 \href{https://doi.org/10.1016/0743-1066(90)90052-7}{DincbasSH90} & \hyperref[auth:a726]{M. Dincbas}, \hyperref[auth:a17]{H. Simonis}, \hyperref[auth:a148]{Pascal Van Hentenryck} & Solving Large Combinatorial Problems in Logic Programming & \href{works/DincbasSH90.pdf}{Yes} & \cite{DincbasSH90} & 1990 & J. Log. Program. & 19 & 86 & 9 & \ref{b:DincbasSH90} & \ref{c:DincbasSH90}\\
\end{longtable}
}

\subsection{Works by Nicolas Beldiceanu}
\label{sec:a128}
{\scriptsize
\begin{longtable}{>{\raggedright\arraybackslash}p{3cm}>{\raggedright\arraybackslash}p{6cm}>{\raggedright\arraybackslash}p{6.5cm}rrrp{2.5cm}rrrrr}
\rowcolor{white}\caption{Works from bibtex (Total 13)}\\ \toprule
\rowcolor{white}Key & Authors & Title & LC & Cite & Year & \shortstack{Conference\\/Journal} & Pages & \shortstack{Nr\\Cites} & \shortstack{Nr\\Refs} & b & c \\ \midrule\endhead
\bottomrule
\endfoot
Madi-WambaLOBM17 \href{https://doi.org/10.1109/ICPADS.2017.00089}{Madi-WambaLOBM17} & \hyperref[auth:a323]{G. Madi{-}Wamba}, \hyperref[auth:a723]{Y. Li}, \hyperref[auth:a724]{A. Orgerie}, \hyperref[auth:a128]{N. Beldiceanu}, \hyperref[auth:a725]{J. Menaud} & Green Energy Aware Scheduling Problem in Virtualized Datacenters & \href{works/Madi-WambaLOBM17.pdf}{Yes} & \cite{Madi-WambaLOBM17} & 2017 & ICPADS 2017 & 8 & 1 & 8 & \ref{b:Madi-WambaLOBM17} & \ref{c:Madi-WambaLOBM17}\\
Madi-WambaB16 \href{https://doi.org/10.1007/978-3-319-33954-2\_18}{Madi-WambaB16} & \hyperref[auth:a323]{G. Madi{-}Wamba}, \hyperref[auth:a128]{N. Beldiceanu} & The TaskIntersection Constraint & \href{works/Madi-WambaB16.pdf}{Yes} & \cite{Madi-WambaB16} & 2016 & CPAIOR 2016 & 16 & 0 & 0 & \ref{b:Madi-WambaB16} & \ref{c:Madi-WambaB16}\\
LetortCB15 \href{https://doi.org/10.1007/s10601-014-9172-8}{LetortCB15} & \hyperref[auth:a127]{A. Letort}, \hyperref[auth:a91]{M. Carlsson}, \hyperref[auth:a128]{N. Beldiceanu} & Synchronized sweep algorithms for scalable scheduling constraints & \href{works/LetortCB15.pdf}{Yes} & \cite{LetortCB15} & 2015 & Constraints An Int. J. & 52 & 2 & 14 & \ref{b:LetortCB15} & \ref{c:LetortCB15}\\
LetortCB13 \href{https://doi.org/10.1007/978-3-642-38171-3\_10}{LetortCB13} & \hyperref[auth:a127]{A. Letort}, \hyperref[auth:a91]{M. Carlsson}, \hyperref[auth:a128]{N. Beldiceanu} & A Synchronized Sweep Algorithm for the \emph{k-dimensional cumulative} Constraint & \href{works/LetortCB13.pdf}{Yes} & \cite{LetortCB13} & 2013 & CPAIOR 2013 & 16 & 3 & 10 & \ref{b:LetortCB13} & \ref{c:LetortCB13}\\
LetortBC12 \href{https://doi.org/10.1007/978-3-642-33558-7\_33}{LetortBC12} & \hyperref[auth:a127]{A. Letort}, \hyperref[auth:a128]{N. Beldiceanu}, \hyperref[auth:a91]{M. Carlsson} & A Scalable Sweep Algorithm for the cumulative Constraint & \href{works/LetortBC12.pdf}{Yes} & \cite{LetortBC12} & 2012 & CP 2012 & 16 & 18 & 12 & \ref{b:LetortBC12} & \ref{c:LetortBC12}\\
BeldiceanuCDP11 \href{https://doi.org/10.1007/s10479-010-0731-0}{BeldiceanuCDP11} & \hyperref[auth:a128]{N. Beldiceanu}, \hyperref[auth:a91]{M. Carlsson}, \hyperref[auth:a245]{S. Demassey}, \hyperref[auth:a362]{E. Poder} & New filtering for the \emph{cumulative} constraint in the context of non-overlapping rectangles & \href{works/BeldiceanuCDP11.pdf}{Yes} & \cite{BeldiceanuCDP11} & 2011 & Ann. Oper. Res. & 24 & 8 & 8 & \ref{b:BeldiceanuCDP11} & \ref{c:BeldiceanuCDP11}\\
ClercqPBJ11 \href{https://doi.org/10.1007/978-3-642-23786-7\_20}{ClercqPBJ11} & \hyperref[auth:a248]{Alexis De Clercq}, \hyperref[auth:a226]{T. Petit}, \hyperref[auth:a128]{N. Beldiceanu}, \hyperref[auth:a249]{N. Jussien} & Filtering Algorithms for Discrete Cumulative Problems with Overloads of Resource & \href{works/ClercqPBJ11.pdf}{Yes} & \cite{ClercqPBJ11} & 2011 & CP 2011 & 16 & 3 & 11 & \ref{b:ClercqPBJ11} & \ref{c:ClercqPBJ11}\\
BeldiceanuCP08 \href{https://doi.org/10.1007/978-3-540-68155-7\_5}{BeldiceanuCP08} & \hyperref[auth:a128]{N. Beldiceanu}, \hyperref[auth:a91]{M. Carlsson}, \hyperref[auth:a362]{E. Poder} & New Filtering for the cumulative Constraint in the Context of Non-Overlapping Rectangles & \href{works/BeldiceanuCP08.pdf}{Yes} & \cite{BeldiceanuCP08} & 2008 & CPAIOR 2008 & 15 & 8 & 9 & \ref{b:BeldiceanuCP08} & \ref{c:BeldiceanuCP08}\\
PoderB08 \href{http://www.aaai.org/Library/ICAPS/2008/icaps08-033.php}{PoderB08} & \hyperref[auth:a362]{E. Poder}, \hyperref[auth:a128]{N. Beldiceanu} & Filtering for a Continuous Multi-Resources cumulative Constraint with Resource Consumption and Production & \href{works/PoderB08.pdf}{Yes} & \cite{PoderB08} & 2008 & ICAPS 2008 & 8 & 0 & 0 & \ref{b:PoderB08} & \ref{c:PoderB08}\\
BeldiceanuP07 \href{https://doi.org/10.1007/978-3-540-72397-4\_16}{BeldiceanuP07} & \hyperref[auth:a128]{N. Beldiceanu}, \hyperref[auth:a362]{E. Poder} & A Continuous Multi-resources \emph{cumulative} Constraint with Positive-Negative Resource Consumption-Production & \href{works/BeldiceanuP07.pdf}{Yes} & \cite{BeldiceanuP07} & 2007 & CPAIOR 2007 & 15 & 4 & 7 & \ref{b:BeldiceanuP07} & \ref{c:BeldiceanuP07}\\
PoderBS04 \href{https://doi.org/10.1016/S0377-2217(02)00756-7}{PoderBS04} & \hyperref[auth:a362]{E. Poder}, \hyperref[auth:a128]{N. Beldiceanu}, \hyperref[auth:a722]{E. Sanlaville} & Computing a lower approximation of the compulsory part of a task with varying duration and varying resource consumption & \href{works/PoderBS04.pdf}{Yes} & \cite{PoderBS04} & 2004 & Eur. J. Oper. Res. & 16 & 7 & 8 & \ref{b:PoderBS04} & \ref{c:PoderBS04}\\
BeldiceanuC02 \href{https://doi.org/10.1007/3-540-46135-3\_5}{BeldiceanuC02} & \hyperref[auth:a128]{N. Beldiceanu}, \hyperref[auth:a91]{M. Carlsson} & A New Multi-resource cumulatives Constraint with Negative Heights & \href{works/BeldiceanuC02.pdf}{Yes} & \cite{BeldiceanuC02} & 2002 & CP 2002 & 17 & 33 & 9 & \ref{b:BeldiceanuC02} & \ref{c:BeldiceanuC02}\\
AggounB93 \href{https://www.sciencedirect.com/science/article/pii/089571779390068A}{AggounB93} & \hyperref[auth:a734]{A. Aggoun}, \hyperref[auth:a128]{N. Beldiceanu} & Extending {CHIP} in order to solve complex scheduling and placement problems & \href{works/AggounB93.pdf}{Yes} & \cite{AggounB93} & 1993 & Mathematical and Computer Modelling & 17 & 187 & 11 & \ref{b:AggounB93} & \ref{c:AggounB93}\\
\end{longtable}
}

\subsection{Works by Pierre Lopez}
\label{sec:a3}
{\scriptsize
\begin{longtable}{>{\raggedright\arraybackslash}p{3cm}>{\raggedright\arraybackslash}p{6cm}>{\raggedright\arraybackslash}p{6.5cm}rrrp{2.5cm}rrrrr}
\rowcolor{white}\caption{Works from bibtex (Total 13)}\\ \toprule
\rowcolor{white}Key & Authors & Title & LC & Cite & Year & \shortstack{Conference\\/Journal} & Pages & \shortstack{Nr\\Cites} & \shortstack{Nr\\Refs} & b & c \\ \midrule\endhead
\bottomrule
\endfoot
JuvinHHL23 \href{https://doi.org/10.4230/LIPIcs.CP.2023.19}{JuvinHHL23} & \hyperref[auth:a0]{C. Juvin}, \hyperref[auth:a1]{E. Hebrard}, \hyperref[auth:a2]{L. Houssin}, \hyperref[auth:a3]{P. Lopez} & An Efficient Constraint Programming Approach to Preemptive Job Shop Scheduling & \href{works/JuvinHHL23.pdf}{Yes} & \cite{JuvinHHL23} & 2023 & CP 2023 & 16 & 0 & 0 & \ref{b:JuvinHHL23} & \ref{c:JuvinHHL23}\\
JuvinHL23 \href{https://doi.org/10.1007/978-3-031-33271-5\_23}{JuvinHL23} & \hyperref[auth:a0]{C. Juvin}, \hyperref[auth:a2]{L. Houssin}, \hyperref[auth:a3]{P. Lopez} & Constraint Programming for the Robust Two-Machine Flow-Shop Scheduling Problem with Budgeted Uncertainty & \href{works/JuvinHL23.pdf}{Yes} & \cite{JuvinHL23} & 2023 & CPAIOR 2023 & 16 & 0 & 11 & \ref{b:JuvinHL23} & \ref{c:JuvinHL23}\\
HebrardALLCMR22 \href{https://doi.org/10.24963/ijcai.2022/643}{HebrardALLCMR22} & \hyperref[auth:a1]{E. Hebrard}, \hyperref[auth:a6]{C. Artigues}, \hyperref[auth:a3]{P. Lopez}, \hyperref[auth:a796]{A. Lusson}, \hyperref[auth:a797]{Steve A. Chien}, \hyperref[auth:a798]{A. Maillard}, \hyperref[auth:a799]{Gregg R. Rabideau} & An Efficient Approach to Data Transfer Scheduling for Long Range Space Exploration & \href{works/HebrardALLCMR22.pdf}{Yes} & \cite{HebrardALLCMR22} & 2022 & IJCAI 2022 & 7 & 0 & 0 & \ref{b:HebrardALLCMR22} & \ref{c:HebrardALLCMR22}\\
Polo-MejiaALB20 \href{https://doi.org/10.1080/00207543.2019.1693654}{Polo-MejiaALB20} & \hyperref[auth:a522]{O. Polo{-}Mej{\'{\i}}a}, \hyperref[auth:a6]{C. Artigues}, \hyperref[auth:a3]{P. Lopez}, \hyperref[auth:a523]{V. Basini} & Mixed-integer/linear and constraint programming approaches for activity scheduling in a nuclear research facility & \href{works/Polo-MejiaALB20.pdf}{Yes} & \cite{Polo-MejiaALB20} & 2020 & Int. J. Prod. Res. & 18 & 8 & 23 & \ref{b:Polo-MejiaALB20} & \ref{c:Polo-MejiaALB20}\\
NattafAL17 \href{https://doi.org/10.1007/s10601-017-9271-4}{NattafAL17} & \hyperref[auth:a81]{M. Nattaf}, \hyperref[auth:a6]{C. Artigues}, \hyperref[auth:a3]{P. Lopez} & Cumulative scheduling with variable task profiles and concave piecewise linear processing rate functions & \href{works/NattafAL17.pdf}{Yes} & \cite{NattafAL17} & 2017 & Constraints An Int. J. & 18 & 5 & 10 & \ref{b:NattafAL17} & \ref{c:NattafAL17}\\
NattafAL15 \href{https://doi.org/10.1007/s10601-015-9192-z}{NattafAL15} & \hyperref[auth:a81]{M. Nattaf}, \hyperref[auth:a6]{C. Artigues}, \hyperref[auth:a3]{P. Lopez} & A hybrid exact method for a scheduling problem with a continuous resource and energy constraints & \href{works/NattafAL15.pdf}{Yes} & \cite{NattafAL15} & 2015 & Constraints An Int. J. & 21 & 14 & 13 & \ref{b:NattafAL15} & \ref{c:NattafAL15}\\
SimoninAHL15 \href{https://doi.org/10.1007/s10601-014-9169-3}{SimoninAHL15} & \hyperref[auth:a126]{G. Simonin}, \hyperref[auth:a6]{C. Artigues}, \hyperref[auth:a1]{E. Hebrard}, \hyperref[auth:a3]{P. Lopez} & Scheduling scientific experiments for comet exploration & \href{works/SimoninAHL15.pdf}{Yes} & \cite{SimoninAHL15} & 2015 & Constraints An Int. J. & 23 & 4 & 5 & \ref{b:SimoninAHL15} & \ref{c:SimoninAHL15}\\
BillautHL12 \href{https://doi.org/10.1007/978-3-642-29828-8\_5}{BillautHL12} & \hyperref[auth:a342]{J. Billaut}, \hyperref[auth:a1]{E. Hebrard}, \hyperref[auth:a3]{P. Lopez} & Complete Characterization of Near-Optimal Sequences for the Two-Machine Flow Shop Scheduling Problem & \href{works/BillautHL12.pdf}{Yes} & \cite{BillautHL12} & 2012 & CPAIOR 2012 & 15 & 1 & 19 & \ref{b:BillautHL12} & \ref{c:BillautHL12}\\
SimoninAHL12 \href{https://doi.org/10.1007/978-3-642-33558-7\_5}{SimoninAHL12} & \hyperref[auth:a126]{G. Simonin}, \hyperref[auth:a6]{C. Artigues}, \hyperref[auth:a1]{E. Hebrard}, \hyperref[auth:a3]{P. Lopez} & Scheduling Scientific Experiments on the Rosetta/Philae Mission & \href{works/SimoninAHL12.pdf}{Yes} & \cite{SimoninAHL12} & 2012 & CP 2012 & 15 & 3 & 8 & \ref{b:SimoninAHL12} & \ref{c:SimoninAHL12}\\
LahimerLH11 \href{https://doi.org/10.1007/978-3-642-21311-3\_12}{LahimerLH11} & \hyperref[auth:a353]{A. Lahimer}, \hyperref[auth:a3]{P. Lopez}, \hyperref[auth:a354]{M. Haouari} & Climbing Depth-Bounded Adjacent Discrepancy Search for Solving Hybrid Flow Shop Scheduling Problems with Multiprocessor Tasks & \href{works/LahimerLH11.pdf}{Yes} & \cite{LahimerLH11} & 2011 & CPAIOR 2011 & 14 & 3 & 15 & \ref{b:LahimerLH11} & \ref{c:LahimerLH11}\\
TrojetHL11 \href{https://doi.org/10.1016/j.cie.2010.08.014}{TrojetHL11} & \hyperref[auth:a715]{M. Trojet}, \hyperref[auth:a716]{F. H'Mida}, \hyperref[auth:a3]{P. Lopez} & Project scheduling under resource constraints: Application of the cumulative global constraint in a decision support framework & \href{works/TrojetHL11.pdf}{Yes} & \cite{TrojetHL11} & 2011 & Comput. Ind. Eng. & 7 & 11 & 17 & \ref{b:TrojetHL11} & \ref{c:TrojetHL11}\\
LopezAKYG00 \href{https://doi.org/10.1016/S0947-3580(00)71114-9}{LopezAKYG00} & \hyperref[auth:a3]{P. Lopez}, \hyperref[auth:a693]{H. Alla}, \hyperref[auth:a690]{O. Korbaa}, \hyperref[auth:a691]{P. Yim}, \hyperref[auth:a692]{J. Gentina} & Discussion on: 'Solving Transient Scheduling Problems with Constraint Programming' by O. Korbaa, P. Yim, and {J.-C.} Gentina & \href{works/LopezAKYG00.pdf}{Yes} & \cite{LopezAKYG00} & 2000 & Eur. J. Control & 4 & 0 & 0 & \ref{b:LopezAKYG00} & \ref{c:LopezAKYG00}\\
TorresL00 \href{http://dx.doi.org/10.1016/s0377-2217(99)00497-x}{TorresL00} & \hyperref[auth:a888]{P. Torres}, \hyperref[auth:a3]{P. Lopez} & On Not-First/Not-Last conditions in disjunctive scheduling & No & \cite{TorresL00} & 2000 & European Journal of Operational Research & null & 26 & 13 & No & \ref{c:TorresL00}\\
\end{longtable}
}

\subsection{Works by Christian Artigues}
\label{sec:a6}
{\scriptsize
\begin{longtable}{>{\raggedright\arraybackslash}p{3cm}>{\raggedright\arraybackslash}p{6cm}>{\raggedright\arraybackslash}p{6.5cm}rrrp{2.5cm}rrrrr}
\rowcolor{white}\caption{Works from bibtex (Total 12)}\\ \toprule
\rowcolor{white}Key & Authors & Title & LC & Cite & Year & \shortstack{Conference\\/Journal} & Pages & \shortstack{Nr\\Cites} & \shortstack{Nr\\Refs} & b & c \\ \midrule\endhead
\bottomrule
\endfoot
PovedaAA23 \href{https://doi.org/10.4230/LIPIcs.CP.2023.31}{PovedaAA23} & \hyperref[auth:a4]{G. Pov{\'{e}}da}, \hyperref[auth:a5]{N. {\'{A}}lvarez}, \hyperref[auth:a6]{C. Artigues} & Partially Preemptive Multi Skill/Mode Resource-Constrained Project Scheduling with Generalized Precedence Relations and Calendars & \href{works/PovedaAA23.pdf}{Yes} & \cite{PovedaAA23} & 2023 & CP 2023 & 21 & 0 & 0 & \ref{b:PovedaAA23} & \ref{c:PovedaAA23}\\
HebrardALLCMR22 \href{https://doi.org/10.24963/ijcai.2022/643}{HebrardALLCMR22} & \hyperref[auth:a1]{E. Hebrard}, \hyperref[auth:a6]{C. Artigues}, \hyperref[auth:a3]{P. Lopez}, \hyperref[auth:a796]{A. Lusson}, \hyperref[auth:a797]{Steve A. Chien}, \hyperref[auth:a798]{A. Maillard}, \hyperref[auth:a799]{Gregg R. Rabideau} & An Efficient Approach to Data Transfer Scheduling for Long Range Space Exploration & \href{works/HebrardALLCMR22.pdf}{Yes} & \cite{HebrardALLCMR22} & 2022 & IJCAI 2022 & 7 & 0 & 0 & \ref{b:HebrardALLCMR22} & \ref{c:HebrardALLCMR22}\\
PohlAK22 \href{https://doi.org/10.1016/j.ejor.2021.08.028}{PohlAK22} & \hyperref[auth:a444]{M. Pohl}, \hyperref[auth:a6]{C. Artigues}, \hyperref[auth:a445]{R. Kolisch} & Solving the time-discrete winter runway scheduling problem: {A} column generation and constraint programming approach & \href{works/PohlAK22.pdf}{Yes} & \cite{PohlAK22} & 2022 & Eur. J. Oper. Res. & 16 & 4 & 31 & \ref{b:PohlAK22} & \ref{c:PohlAK22}\\
ArtiguesHQT21 \href{https://doi.org/10.5220/0010190101290136}{ArtiguesHQT21} & \hyperref[auth:a6]{C. Artigues}, \hyperref[auth:a1]{E. Hebrard}, \hyperref[auth:a800]{A. Quilliot}, \hyperref[auth:a801]{H. Toussaint} & Multi-Mode {RCPSP} with Safety Margin Maximization: Models and Algorithms & No & \cite{ArtiguesHQT21} & 2021 & ICORES 2021 & 8 & 0 & 0 & No & \ref{c:ArtiguesHQT21}\\
Polo-MejiaALB20 \href{https://doi.org/10.1080/00207543.2019.1693654}{Polo-MejiaALB20} & \hyperref[auth:a522]{O. Polo{-}Mej{\'{\i}}a}, \hyperref[auth:a6]{C. Artigues}, \hyperref[auth:a3]{P. Lopez}, \hyperref[auth:a523]{V. Basini} & Mixed-integer/linear and constraint programming approaches for activity scheduling in a nuclear research facility & \href{works/Polo-MejiaALB20.pdf}{Yes} & \cite{Polo-MejiaALB20} & 2020 & Int. J. Prod. Res. & 18 & 8 & 23 & \ref{b:Polo-MejiaALB20} & \ref{c:Polo-MejiaALB20}\\
NattafAL17 \href{https://doi.org/10.1007/s10601-017-9271-4}{NattafAL17} & \hyperref[auth:a81]{M. Nattaf}, \hyperref[auth:a6]{C. Artigues}, \hyperref[auth:a3]{P. Lopez} & Cumulative scheduling with variable task profiles and concave piecewise linear processing rate functions & \href{works/NattafAL17.pdf}{Yes} & \cite{NattafAL17} & 2017 & Constraints An Int. J. & 18 & 5 & 10 & \ref{b:NattafAL17} & \ref{c:NattafAL17}\\
NattafAL15 \href{https://doi.org/10.1007/s10601-015-9192-z}{NattafAL15} & \hyperref[auth:a81]{M. Nattaf}, \hyperref[auth:a6]{C. Artigues}, \hyperref[auth:a3]{P. Lopez} & A hybrid exact method for a scheduling problem with a continuous resource and energy constraints & \href{works/NattafAL15.pdf}{Yes} & \cite{NattafAL15} & 2015 & Constraints An Int. J. & 21 & 14 & 13 & \ref{b:NattafAL15} & \ref{c:NattafAL15}\\
SialaAH15 \href{https://doi.org/10.1007/978-3-319-23219-5\_28}{SialaAH15} & \hyperref[auth:a129]{M. Siala}, \hyperref[auth:a6]{C. Artigues}, \hyperref[auth:a1]{E. Hebrard} & Two Clause Learning Approaches for Disjunctive Scheduling & \href{works/SialaAH15.pdf}{Yes} & \cite{SialaAH15} & 2015 & CP 2015 & 10 & 4 & 17 & \ref{b:SialaAH15} & \ref{c:SialaAH15}\\
SimoninAHL15 \href{https://doi.org/10.1007/s10601-014-9169-3}{SimoninAHL15} & \hyperref[auth:a126]{G. Simonin}, \hyperref[auth:a6]{C. Artigues}, \hyperref[auth:a1]{E. Hebrard}, \hyperref[auth:a3]{P. Lopez} & Scheduling scientific experiments for comet exploration & \href{works/SimoninAHL15.pdf}{Yes} & \cite{SimoninAHL15} & 2015 & Constraints An Int. J. & 23 & 4 & 5 & \ref{b:SimoninAHL15} & \ref{c:SimoninAHL15}\\
SimoninAHL12 \href{https://doi.org/10.1007/978-3-642-33558-7\_5}{SimoninAHL12} & \hyperref[auth:a126]{G. Simonin}, \hyperref[auth:a6]{C. Artigues}, \hyperref[auth:a1]{E. Hebrard}, \hyperref[auth:a3]{P. Lopez} & Scheduling Scientific Experiments on the Rosetta/Philae Mission & \href{works/SimoninAHL12.pdf}{Yes} & \cite{SimoninAHL12} & 2012 & CP 2012 & 15 & 3 & 8 & \ref{b:SimoninAHL12} & \ref{c:SimoninAHL12}\\
ArtiguesBF04 \href{https://doi.org/10.1007/978-3-540-24664-0\_3}{ArtiguesBF04} & \hyperref[auth:a6]{C. Artigues}, \hyperref[auth:a387]{S. Belmokhtar}, \hyperref[auth:a360]{D. Feillet} & A New Exact Solution Algorithm for the Job Shop Problem with Sequence-Dependent Setup Times & \href{works/ArtiguesBF04.pdf}{Yes} & \cite{ArtiguesBF04} & 2004 & CPAIOR 2004 & 13 & 16 & 9 & \ref{b:ArtiguesBF04} & \ref{c:ArtiguesBF04}\\
ArtiguesR00 \href{https://doi.org/10.1016/S0377-2217(99)00496-8}{ArtiguesR00} & \hyperref[auth:a6]{C. Artigues}, \hyperref[auth:a721]{F. Roubellat} & A polynomial activity insertion algorithm in a multi-resource schedule with cumulative constraints and multiple modes & \href{works/ArtiguesR00.pdf}{Yes} & \cite{ArtiguesR00} & 2000 & Eur. J. Oper. Res. & 20 & 84 & 3 & \ref{b:ArtiguesR00} & \ref{c:ArtiguesR00}\\
\end{longtable}
}

\subsection{Works by Pierre Schaus}
\label{sec:a147}
{\scriptsize
\begin{longtable}{>{\raggedright\arraybackslash}p{3cm}>{\raggedright\arraybackslash}p{6cm}>{\raggedright\arraybackslash}p{6.5cm}rrrp{2.5cm}rrrrr}
\rowcolor{white}\caption{Works from bibtex (Total 12)}\\ \toprule
\rowcolor{white}Key & Authors & Title & LC & Cite & Year & \shortstack{Conference\\/Journal} & Pages & \shortstack{Nr\\Cites} & \shortstack{Nr\\Refs} & b & c \\ \midrule\endhead
\bottomrule
\endfoot
CauwelaertDS20 \href{http://dx.doi.org/10.1007/s10951-019-00632-8}{CauwelaertDS20} & \hyperref[auth:a850]{Sasha Van Cauwelaert}, \hyperref[auth:a207]{C. Dejemeppe}, \hyperref[auth:a147]{P. Schaus} & An Efficient Filtering Algorithm for the Unary Resource Constraint with Transition Times and Optional Activities & \href{works/CauwelaertDS20.pdf}{Yes} & \cite{CauwelaertDS20} & 2020 & Journal of Scheduling & 19 & 2 & 21 & \ref{b:CauwelaertDS20} & \ref{c:CauwelaertDS20}\\
CappartTSR18 \href{https://doi.org/10.1007/978-3-319-98334-9\_32}{CappartTSR18} & \hyperref[auth:a42]{Q. Cappart}, \hyperref[auth:a849]{C. Thomas}, \hyperref[auth:a147]{P. Schaus}, \hyperref[auth:a331]{L. Rousseau} & A Constraint Programming Approach for Solving Patient Transportation Problems & \href{works/CappartTSR18.pdf}{Yes} & \cite{CappartTSR18} & 2018 & CP 2018 & 17 & 6 & 31 & \ref{b:CappartTSR18} & \ref{c:CappartTSR18}\\
CauwelaertLS18 \href{https://doi.org/10.1007/s10601-017-9277-y}{CauwelaertLS18} & \hyperref[auth:a206]{Sascha Van Cauwelaert}, \hyperref[auth:a142]{M. Lombardi}, \hyperref[auth:a147]{P. Schaus} & How efficient is a global constraint in practice? - {A} fair experimental framework & \href{works/CauwelaertLS18.pdf}{Yes} & \cite{CauwelaertLS18} & 2018 & Constraints An Int. J. & 36 & 2 & 39 & \ref{b:CauwelaertLS18} & \ref{c:CauwelaertLS18}\\
CappartS17 \href{https://doi.org/10.1007/978-3-319-59776-8\_26}{CappartS17} & \hyperref[auth:a42]{Q. Cappart}, \hyperref[auth:a147]{P. Schaus} & Rescheduling Railway Traffic on Real Time Situations Using Time-Interval Variables & \href{works/CappartS17.pdf}{Yes} & \cite{CappartS17} & 2017 & CPAIOR 2017 & 16 & 2 & 28 & \ref{b:CappartS17} & \ref{c:CappartS17}\\
CauwelaertDMS16 \href{https://doi.org/10.1007/978-3-319-44953-1\_33}{CauwelaertDMS16} & \hyperref[auth:a206]{Sascha Van Cauwelaert}, \hyperref[auth:a207]{C. Dejemeppe}, \hyperref[auth:a149]{J. Monette}, \hyperref[auth:a147]{P. Schaus} & Efficient Filtering for the Unary Resource with Family-Based Transition Times & \href{works/CauwelaertDMS16.pdf}{Yes} & \cite{CauwelaertDMS16} & 2016 & CP 2016 & 16 & 1 & 12 & \ref{b:CauwelaertDMS16} & \ref{c:CauwelaertDMS16}\\
DejemeppeCS15 \href{https://doi.org/10.1007/978-3-319-23219-5\_7}{DejemeppeCS15} & \hyperref[auth:a207]{C. Dejemeppe}, \hyperref[auth:a206]{Sascha Van Cauwelaert}, \hyperref[auth:a147]{P. Schaus} & The Unary Resource with Transition Times & \href{works/DejemeppeCS15.pdf}{Yes} & \cite{DejemeppeCS15} & 2015 & CP 2015 & 16 & 5 & 11 & \ref{b:DejemeppeCS15} & \ref{c:DejemeppeCS15}\\
GayHLS15 \href{https://doi.org/10.1007/978-3-319-23219-5\_10}{GayHLS15} & \hyperref[auth:a216]{S. Gay}, \hyperref[auth:a217]{R. Hartert}, \hyperref[auth:a218]{C. Lecoutre}, \hyperref[auth:a147]{P. Schaus} & Conflict Ordering Search for Scheduling Problems & \href{works/GayHLS15.pdf}{Yes} & \cite{GayHLS15} & 2015 & CP 2015 & 9 & 20 & 15 & \ref{b:GayHLS15} & \ref{c:GayHLS15}\\
GayHS15 \href{https://doi.org/10.1007/978-3-319-23219-5\_11}{GayHS15} & \hyperref[auth:a216]{S. Gay}, \hyperref[auth:a217]{R. Hartert}, \hyperref[auth:a147]{P. Schaus} & Simple and Scalable Time-Table Filtering for the Cumulative Constraint & \href{works/GayHS15.pdf}{Yes} & \cite{GayHS15} & 2015 & CP 2015 & 9 & 10 & 9 & \ref{b:GayHS15} & \ref{c:GayHS15}\\
GayHS15a \href{https://doi.org/10.1007/978-3-319-18008-3\_11}{GayHS15a} & \hyperref[auth:a216]{S. Gay}, \hyperref[auth:a217]{R. Hartert}, \hyperref[auth:a147]{P. Schaus} & Time-Table Disjunctive Reasoning for the Cumulative Constraint & \href{works/GayHS15a.pdf}{Yes} & \cite{GayHS15a} & 2015 & CPAIOR 2015 & 16 & 5 & 12 & \ref{b:GayHS15a} & \ref{c:GayHS15a}\\
GaySS14 \href{https://doi.org/10.1007/978-3-319-10428-7\_59}{GaySS14} & \hyperref[auth:a216]{S. Gay}, \hyperref[auth:a147]{P. Schaus}, \hyperref[auth:a239]{Vivian De Smedt} & Continuous Casting Scheduling with Constraint Programming & \href{works/GaySS14.pdf}{Yes} & \cite{GaySS14} & 2014 & CP 2014 & 15 & 7 & 11 & \ref{b:GaySS14} & \ref{c:GaySS14}\\
HoundjiSWD14 \href{https://doi.org/10.1007/978-3-319-10428-7\_29}{HoundjiSWD14} & \hyperref[auth:a228]{Vinas{\'{e}}tan Ratheil Houndji}, \hyperref[auth:a147]{P. Schaus}, \hyperref[auth:a229]{Laurence A. Wolsey}, \hyperref[auth:a151]{Y. Deville} & The StockingCost Constraint & \href{works/HoundjiSWD14.pdf}{Yes} & \cite{HoundjiSWD14} & 2014 & CP 2014 & 16 & 5 & 7 & \ref{b:HoundjiSWD14} & \ref{c:HoundjiSWD14}\\
SchausHMCMD11 \href{https://doi.org/10.1007/s10601-010-9100-5}{SchausHMCMD11} & \hyperref[auth:a147]{P. Schaus}, \hyperref[auth:a148]{Pascal Van Hentenryck}, \hyperref[auth:a149]{J. Monette}, \hyperref[auth:a150]{C. Coffrin}, \hyperref[auth:a32]{L. Michel}, \hyperref[auth:a151]{Y. Deville} & Solving Steel Mill Slab Problems with constraint-based techniques: CP, LNS, and {CBLS} & \href{works/SchausHMCMD11.pdf}{Yes} & \cite{SchausHMCMD11} & 2011 & Constraints An Int. J. & 23 & 14 & 5 & \ref{b:SchausHMCMD11} & \ref{c:SchausHMCMD11}\\
\end{longtable}
}

\subsection{Works by Roman Bart{\'{a}}k}
\label{sec:a152}
{\scriptsize
\begin{longtable}{>{\raggedright\arraybackslash}p{3cm}>{\raggedright\arraybackslash}p{6cm}>{\raggedright\arraybackslash}p{6.5cm}rrrp{2.5cm}rrrrr}
\rowcolor{white}\caption{Works from bibtex (Total 11)}\\ \toprule
\rowcolor{white}Key & Authors & Title & LC & Cite & Year & \shortstack{Conference\\/Journal} & Pages & \shortstack{Nr\\Cites} & \shortstack{Nr\\Refs} & b & c \\ \midrule\endhead
\bottomrule
\endfoot
SvancaraB22 \href{https://doi.org/10.5220/0010869700003116}{SvancaraB22} & \hyperref[auth:a787]{J. Svancara}, \hyperref[auth:a152]{R. Bart{\'{a}}k} & Tackling Train Routing via Multi-agent Pathfinding and Constraint-based Scheduling & \href{works/SvancaraB22.pdf}{Yes} & \cite{SvancaraB22} & 2022 & ICAART 2022 & 8 & 0 & 0 & \ref{b:SvancaraB22} & \ref{c:SvancaraB22}\\
JelinekB16 \href{https://doi.org/10.1007/978-3-319-28228-2\_1}{JelinekB16} & \hyperref[auth:a788]{J. Jel{\'{\i}}nek}, \hyperref[auth:a152]{R. Bart{\'{a}}k} & Using Constraint Logic Programming to Schedule Solar Array Operations on the International Space Station & \href{works/JelinekB16.pdf}{Yes} & \cite{JelinekB16} & 2016 & PADL 2016 & 10 & 0 & 5 & \ref{b:JelinekB16} & \ref{c:JelinekB16}\\
BartakV15 \href{}{BartakV15} & \hyperref[auth:a152]{R. Bart{\'{a}}k}, \hyperref[auth:a313]{M. Vlk} & Reactive Recovery from Machine Breakdown in Production Scheduling with Temporal Distance and Resource Constraints & \href{works/BartakV15.pdf}{Yes} & \cite{BartakV15} & 2015 & ICAART 2015 & 12 & 0 & 0 & \ref{b:BartakV15} & \ref{c:BartakV15}\\
Bartak14 \href{}{Bartak14} & \hyperref[auth:a152]{R. Bart{\'{a}}k} & Planning and Scheduling & No & \cite{Bartak14} & 2014 & Computing Handbook, Third Edition: Computer Science and Software Engineering & null & 0 & 0 & No & \ref{c:Bartak14}\\
BartakS11 \href{https://doi.org/10.1007/s10601-011-9109-4}{BartakS11} & \hyperref[auth:a152]{R. Bart{\'{a}}k}, \hyperref[auth:a153]{Miguel A. Salido} & Constraint satisfaction for planning and scheduling problems & \href{works/BartakS11.pdf}{Yes} & \cite{BartakS11} & 2011 & Constraints An Int. J. & 5 & 17 & 3 & \ref{b:BartakS11} & \ref{c:BartakS11}\\
BartakCS10 \href{https://doi.org/10.1007/s10479-008-0492-1}{BartakCS10} & \hyperref[auth:a152]{R. Bart{\'{a}}k}, \hyperref[auth:a162]{O. Cepek}, \hyperref[auth:a789]{P. Surynek} & Discovering implied constraints in precedence graphs with alternatives & \href{works/BartakCS10.pdf}{Yes} & \cite{BartakCS10} & 2010 & Ann. Oper. Res. & 31 & 2 & 9 & \ref{b:BartakCS10} & \ref{c:BartakCS10}\\
BartakSR10 \href{https://doi.org/10.1017/S0269888910000202}{BartakSR10} & \hyperref[auth:a152]{R. Bart{\'{a}}k}, \hyperref[auth:a153]{Miguel A. Salido}, \hyperref[auth:a318]{F. Rossi} & New trends in constraint satisfaction, planning, and scheduling: a survey & \href{works/BartakSR10.pdf}{Yes} & \cite{BartakSR10} & 2010 & Knowl. Eng. Rev. & 31 & 28 & 47 & \ref{b:BartakSR10} & \ref{c:BartakSR10}\\
VilimBC05 \href{https://doi.org/10.1007/s10601-005-2814-0}{VilimBC05} & \hyperref[auth:a121]{P. Vil{\'{\i}}m}, \hyperref[auth:a152]{R. Bart{\'{a}}k}, \hyperref[auth:a162]{O. Cepek} & Extension of \emph{O}(\emph{n} log \emph{n}) Filtering Algorithms for the Unary Resource Constraint to Optional Activities & \href{works/VilimBC05.pdf}{Yes} & \cite{VilimBC05} & 2005 & Constraints An Int. J. & 23 & 21 & 5 & \ref{b:VilimBC05} & \ref{c:VilimBC05}\\
VilimBC04 \href{https://doi.org/10.1007/978-3-540-30201-8\_8}{VilimBC04} & \hyperref[auth:a121]{P. Vil{\'{\i}}m}, \hyperref[auth:a152]{R. Bart{\'{a}}k}, \hyperref[auth:a162]{O. Cepek} & Unary Resource Constraint with Optional Activities & \href{works/VilimBC04.pdf}{Yes} & \cite{VilimBC04} & 2004 & CP 2004 & 15 & 13 & 4 & \ref{b:VilimBC04} & \ref{c:VilimBC04}\\
Bartak02 \href{https://doi.org/10.1007/3-540-46135-3\_39}{Bartak02} & \hyperref[auth:a152]{R. Bart{\'{a}}k} & Visopt ShopFloor: On the Edge of Planning and Scheduling & \href{works/Bartak02.pdf}{Yes} & \cite{Bartak02} & 2002 & CP 2002 & 16 & 6 & 4 & \ref{b:Bartak02} & \ref{c:Bartak02}\\
Bartak02a \href{https://doi.org/10.1007/3-540-36607-5\_14}{Bartak02a} & \hyperref[auth:a152]{R. Bart{\'{a}}k} & Visopt ShopFloor: Going Beyond Traditional Scheduling & \href{works/Bartak02a.pdf}{Yes} & \cite{Bartak02a} & 2002 & ERCIM/CologNet 2002 & 15 & 1 & 9 & \ref{b:Bartak02a} & \ref{c:Bartak02a}\\
\end{longtable}
}

\subsection{Works by Philippe Laborie}
\label{sec:a118}
{\scriptsize
\begin{longtable}{>{\raggedright\arraybackslash}p{3cm}>{\raggedright\arraybackslash}p{6cm}>{\raggedright\arraybackslash}p{6.5cm}rrrp{2.5cm}rrrrr}
\rowcolor{white}\caption{Works from bibtex (Total 11)}\\ \toprule
\rowcolor{white}Key & Authors & Title & LC & Cite & Year & \shortstack{Conference\\/Journal} & Pages & \shortstack{Nr\\Cites} & \shortstack{Nr\\Refs} & b & c \\ \midrule\endhead
\bottomrule
\endfoot
LunardiBLRV20 \href{https://doi.org/10.1016/j.cor.2020.105020}{LunardiBLRV20} & \hyperref[auth:a510]{Willian T. Lunardi}, \hyperref[auth:a511]{Ernesto G. Birgin}, \hyperref[auth:a118]{P. Laborie}, \hyperref[auth:a512]{D{\'{e}}bora P. Ronconi}, \hyperref[auth:a513]{H. Voos} & Mixed Integer linear programming and constraint programming models for the online printing shop scheduling problem & \href{works/LunardiBLRV20.pdf}{Yes} & \cite{LunardiBLRV20} & 2020 & Comput. Oper. Res. & 20 & 30 & 18 & \ref{b:LunardiBLRV20} & \ref{c:LunardiBLRV20}\\
Laborie18a \href{https://doi.org/10.1007/978-3-319-93031-2\_29}{Laborie18a} & \hyperref[auth:a118]{P. Laborie} & An Update on the Comparison of MIP, {CP} and Hybrid Approaches for Mixed Resource Allocation and Scheduling & \href{works/Laborie18a.pdf}{Yes} & \cite{Laborie18a} & 2018 & CPAIOR 2018 & 9 & 18 & 10 & \ref{b:Laborie18a} & \ref{c:Laborie18a}\\
LaborieRSV18 \href{https://doi.org/10.1007/s10601-018-9281-x}{LaborieRSV18} & \hyperref[auth:a118]{P. Laborie}, \hyperref[auth:a119]{J. Rogerie}, \hyperref[auth:a120]{P. Shaw}, \hyperref[auth:a121]{P. Vil{\'{\i}}m} & {IBM} {ILOG} {CP} optimizer for scheduling - 20+ years of scheduling with constraints at {IBM/ILOG} & \href{works/LaborieRSV18.pdf}{Yes} & \cite{LaborieRSV18} & 2018 & Constraints An Int. J. & 41 & 148 & 35 & \ref{b:LaborieRSV18} & \ref{c:LaborieRSV18}\\
MelgarejoLS15 \href{https://doi.org/10.1007/978-3-319-18008-3\_1}{MelgarejoLS15} & \hyperref[auth:a324]{P. Aguiar{-}Melgarejo}, \hyperref[auth:a118]{P. Laborie}, \hyperref[auth:a85]{C. Solnon} & A Time-Dependent No-Overlap Constraint: Application to Urban Delivery Problems & \href{works/MelgarejoLS15.pdf}{Yes} & \cite{MelgarejoLS15} & 2015 & CPAIOR 2015 & 17 & 14 & 17 & \ref{b:MelgarejoLS15} & \ref{c:MelgarejoLS15}\\
VilimLS15 \href{https://doi.org/10.1007/978-3-319-18008-3\_30}{VilimLS15} & \hyperref[auth:a121]{P. Vil{\'{\i}}m}, \hyperref[auth:a118]{P. Laborie}, \hyperref[auth:a120]{P. Shaw} & Failure-Directed Search for Constraint-Based Scheduling & \href{works/VilimLS15.pdf}{Yes} & \cite{VilimLS15} & 2015 & CPAIOR 2015 & 17 & 31 & 19 & \ref{b:VilimLS15} & \ref{c:VilimLS15}\\
BidotVLB09 \href{https://doi.org/10.1007/s10951-008-0080-x}{BidotVLB09} & \hyperref[auth:a835]{J. Bidot}, \hyperref[auth:a836]{T. Vidal}, \hyperref[auth:a118]{P. Laborie}, \hyperref[auth:a89]{J. Christopher Beck} & A theoretic and practical framework for scheduling in a stochastic environment & \href{works/BidotVLB09.pdf}{Yes} & \cite{BidotVLB09} & 2009 & J. Sched. & 30 & 58 & 20 & \ref{b:BidotVLB09} & \ref{c:BidotVLB09}\\
Laborie09 \href{https://doi.org/10.1007/978-3-642-01929-6\_12}{Laborie09} & \hyperref[auth:a118]{P. Laborie} & {IBM} {ILOG} {CP} Optimizer for Detailed Scheduling Illustrated on Three Problems & \href{works/Laborie09.pdf}{Yes} & \cite{Laborie09} & 2009 & CPAIOR 2009 & 15 & 53 & 2 & \ref{b:Laborie09} & \ref{c:Laborie09}\\
BaptisteLPN06 \href{https://doi.org/10.1016/S1574-6526(06)80026-X}{BaptisteLPN06} & \hyperref[auth:a163]{P. Baptiste}, \hyperref[auth:a118]{P. Laborie}, \hyperref[auth:a164]{Claude Le Pape}, \hyperref[auth:a666]{W. Nuijten} & Constraint-Based Scheduling and Planning & No & \cite{BaptisteLPN06} & 2006 & Handbook of Constraint Programming & 39 & 30 & 25 & No & \ref{c:BaptisteLPN06}\\
GodardLN05 \href{http://www.aaai.org/Library/ICAPS/2005/icaps05-009.php}{GodardLN05} & \hyperref[auth:a782]{D. Godard}, \hyperref[auth:a118]{P. Laborie}, \hyperref[auth:a666]{W. Nuijten} & Randomized Large Neighborhood Search for Cumulative Scheduling & \href{works/GodardLN05.pdf}{Yes} & \cite{GodardLN05} & 2005 & ICAPS 2005 & 9 & 0 & 0 & \ref{b:GodardLN05} & \ref{c:GodardLN05}\\
Laborie03 \href{http://dx.doi.org/10.1016/s0004-3702(02)00362-4}{Laborie03} & \hyperref[auth:a118]{P. Laborie} & Algorithms for propagating resource constraints in AI planning and scheduling: Existing approaches and new results & \href{works/Laborie03.pdf}{Yes} & \cite{Laborie03} & 2003 & Artificial Intelligence & 38 & 128 & 10 & \ref{b:Laborie03} & \ref{c:Laborie03}\\
FocacciLN00 \href{http://www.aaai.org/Library/AIPS/2000/aips00-010.php}{FocacciLN00} & \hyperref[auth:a784]{F. Focacci}, \hyperref[auth:a118]{P. Laborie}, \hyperref[auth:a666]{W. Nuijten} & Solving Scheduling Problems with Setup Times and Alternative Resources & \href{works/FocacciLN00.pdf}{Yes} & \cite{FocacciLN00} & 2000 & AIPS 2000 & 10 & 0 & 0 & \ref{b:FocacciLN00} & \ref{c:FocacciLN00}\\
\end{longtable}
}

\subsection{Works by Petr Vil{\'{\i}}m}
\label{sec:a121}
{\scriptsize
\begin{longtable}{>{\raggedright\arraybackslash}p{3cm}>{\raggedright\arraybackslash}p{6cm}>{\raggedright\arraybackslash}p{6.5cm}rrrp{2.5cm}rrrrr}
\rowcolor{white}\caption{Works from bibtex (Total 11)}\\ \toprule
\rowcolor{white}Key & Authors & Title & LC & Cite & Year & \shortstack{Conference\\/Journal} & Pages & \shortstack{Nr\\Cites} & \shortstack{Nr\\Refs} & b & c \\ \midrule\endhead
\bottomrule
\endfoot
LaborieRSV18 \href{https://doi.org/10.1007/s10601-018-9281-x}{LaborieRSV18} & \hyperref[auth:a118]{P. Laborie}, \hyperref[auth:a119]{J. Rogerie}, \hyperref[auth:a120]{P. Shaw}, \hyperref[auth:a121]{P. Vil{\'{\i}}m} & {IBM} {ILOG} {CP} optimizer for scheduling - 20+ years of scheduling with constraints at {IBM/ILOG} & \href{works/LaborieRSV18.pdf}{Yes} & \cite{LaborieRSV18} & 2018 & Constraints An Int. J. & 41 & 148 & 35 & \ref{b:LaborieRSV18} & \ref{c:LaborieRSV18}\\
VilimLS15 \href{https://doi.org/10.1007/978-3-319-18008-3\_30}{VilimLS15} & \hyperref[auth:a121]{P. Vil{\'{\i}}m}, \hyperref[auth:a118]{P. Laborie}, \hyperref[auth:a120]{P. Shaw} & Failure-Directed Search for Constraint-Based Scheduling & \href{works/VilimLS15.pdf}{Yes} & \cite{VilimLS15} & 2015 & CPAIOR 2015 & 17 & 31 & 19 & \ref{b:VilimLS15} & \ref{c:VilimLS15}\\
Vilim11 \href{https://doi.org/10.1007/978-3-642-21311-3\_22}{Vilim11} & \hyperref[auth:a121]{P. Vil{\'{\i}}m} & Timetable Edge Finding Filtering Algorithm for Discrete Cumulative Resources & \href{works/Vilim11.pdf}{Yes} & \cite{Vilim11} & 2011 & CPAIOR 2011 & 16 & 28 & 6 & \ref{b:Vilim11} & \ref{c:Vilim11}\\
Vilim09 \href{https://doi.org/10.1007/978-3-642-04244-7\_62}{Vilim09} & \hyperref[auth:a121]{P. Vil{\'{\i}}m} & Edge Finding Filtering Algorithm for Discrete Cumulative Resources in \emph{O}(\emph{kn} log \emph{n})\{{\textbackslash}mathcal O\}(kn \{{\textbackslash}rm log\} n) & \href{works/Vilim09.pdf}{Yes} & \cite{Vilim09} & 2009 & CP 2009 & 15 & 25 & 4 & \ref{b:Vilim09} & \ref{c:Vilim09}\\
Vilim09a \href{https://doi.org/10.1007/978-3-642-01929-6\_22}{Vilim09a} & \hyperref[auth:a121]{P. Vil{\'{\i}}m} & Max Energy Filtering Algorithm for Discrete Cumulative Resources & \href{works/Vilim09a.pdf}{Yes} & \cite{Vilim09a} & 2009 & CPAIOR 2009 & 15 & 13 & 4 & \ref{b:Vilim09a} & \ref{c:Vilim09a}\\
Vilim05 \href{https://doi.org/10.1007/11493853\_29}{Vilim05} & \hyperref[auth:a121]{P. Vil{\'{\i}}m} & Computing Explanations for the Unary Resource Constraint & \href{works/Vilim05.pdf}{Yes} & \cite{Vilim05} & 2005 & CPAIOR 2005 & 14 & 5 & 8 & \ref{b:Vilim05} & \ref{c:Vilim05}\\
VilimBC05 \href{https://doi.org/10.1007/s10601-005-2814-0}{VilimBC05} & \hyperref[auth:a121]{P. Vil{\'{\i}}m}, \hyperref[auth:a152]{R. Bart{\'{a}}k}, \hyperref[auth:a162]{O. Cepek} & Extension of \emph{O}(\emph{n} log \emph{n}) Filtering Algorithms for the Unary Resource Constraint to Optional Activities & \href{works/VilimBC05.pdf}{Yes} & \cite{VilimBC05} & 2005 & Constraints An Int. J. & 23 & 21 & 5 & \ref{b:VilimBC05} & \ref{c:VilimBC05}\\
Vilim04 \href{https://doi.org/10.1007/978-3-540-24664-0\_23}{Vilim04} & \hyperref[auth:a121]{P. Vil{\'{\i}}m} & O(n log n) Filtering Algorithms for Unary Resource Constraint & \href{works/Vilim04.pdf}{Yes} & \cite{Vilim04} & 2004 & CPAIOR 2004 & 13 & 22 & 5 & \ref{b:Vilim04} & \ref{c:Vilim04}\\
VilimBC04 \href{https://doi.org/10.1007/978-3-540-30201-8\_8}{VilimBC04} & \hyperref[auth:a121]{P. Vil{\'{\i}}m}, \hyperref[auth:a152]{R. Bart{\'{a}}k}, \hyperref[auth:a162]{O. Cepek} & Unary Resource Constraint with Optional Activities & \href{works/VilimBC04.pdf}{Yes} & \cite{VilimBC04} & 2004 & CP 2004 & 15 & 13 & 4 & \ref{b:VilimBC04} & \ref{c:VilimBC04}\\
Vilim03 \href{https://doi.org/10.1007/978-3-540-45193-8\_124}{Vilim03} & \hyperref[auth:a121]{P. Vil{\'{\i}}m} & Computing Explanations for Global Scheduling Constraints & \href{works/Vilim03.pdf}{Yes} & \cite{Vilim03} & 2003 & CP 2003 & 1 & 1 & 1 & \ref{b:Vilim03} & \ref{c:Vilim03}\\
Vilim02 \href{https://doi.org/10.1007/3-540-46135-3\_62}{Vilim02} & \hyperref[auth:a121]{P. Vil{\'{\i}}m} & Batch Processing with Sequence Dependent Setup Times & \href{works/Vilim02.pdf}{Yes} & \cite{Vilim02} & 2002 & CP 2002 & 1 & 6 & 1 & \ref{b:Vilim02} & \ref{c:Vilim02}\\
\end{longtable}
}

\subsection{Works by Luca Benini}
\label{sec:a247}
{\scriptsize
\begin{longtable}{>{\raggedright\arraybackslash}p{3cm}>{\raggedright\arraybackslash}p{6cm}>{\raggedright\arraybackslash}p{6.5cm}rrrp{2.5cm}rrrrr}
\rowcolor{white}\caption{Works from bibtex (Total 10)}\\ \toprule
\rowcolor{white}Key & Authors & Title & LC & Cite & Year & \shortstack{Conference\\/Journal} & Pages & \shortstack{Nr\\Cites} & \shortstack{Nr\\Refs} & b & c \\ \midrule\endhead
\bottomrule
\endfoot
BorghesiBLMB18 \href{https://doi.org/10.1016/j.suscom.2018.05.007}{BorghesiBLMB18} & \hyperref[auth:a231]{A. Borghesi}, \hyperref[auth:a230]{A. Bartolini}, \hyperref[auth:a142]{M. Lombardi}, \hyperref[auth:a143]{M. Milano}, \hyperref[auth:a247]{L. Benini} & Scheduling-based power capping in high performance computing systems & \href{works/BorghesiBLMB18.pdf}{Yes} & \cite{BorghesiBLMB18} & 2018 & Sustain. Comput. Informatics Syst. & 13 & 11 & 22 & \ref{b:BorghesiBLMB18} & \ref{c:BorghesiBLMB18}\\
BridiBLMB16 \href{https://doi.org/10.1109/TPDS.2016.2516997}{BridiBLMB16} & \hyperref[auth:a232]{T. Bridi}, \hyperref[auth:a230]{A. Bartolini}, \hyperref[auth:a142]{M. Lombardi}, \hyperref[auth:a143]{M. Milano}, \hyperref[auth:a247]{L. Benini} & A Constraint Programming Scheduler for Heterogeneous High-Performance Computing Machines & \href{works/BridiBLMB16.pdf}{Yes} & \cite{BridiBLMB16} & 2016 & {IEEE} Trans. Parallel Distributed Syst. & 14 & 17 & 22 & \ref{b:BridiBLMB16} & \ref{c:BridiBLMB16}\\
BridiLBBM16 \href{https://doi.org/10.3233/978-1-61499-672-9-1598}{BridiLBBM16} & \hyperref[auth:a232]{T. Bridi}, \hyperref[auth:a142]{M. Lombardi}, \hyperref[auth:a230]{A. Bartolini}, \hyperref[auth:a247]{L. Benini}, \hyperref[auth:a143]{M. Milano} & {DARDIS:} Distributed And Randomized DIspatching and Scheduling & \href{works/BridiLBBM16.pdf}{Yes} & \cite{BridiLBBM16} & 2016 & ECAI 2016 & 2 & 0 & 0 & \ref{b:BridiLBBM16} & \ref{c:BridiLBBM16}\\
BonfiettiLBM14 \href{https://doi.org/10.1016/j.artint.2013.09.006}{BonfiettiLBM14} & \hyperref[auth:a203]{A. Bonfietti}, \hyperref[auth:a142]{M. Lombardi}, \hyperref[auth:a247]{L. Benini}, \hyperref[auth:a143]{M. Milano} & {CROSS} cyclic resource-constrained scheduling solver & \href{works/BonfiettiLBM14.pdf}{Yes} & \cite{BonfiettiLBM14} & 2014 & Artif. Intell. & 28 & 8 & 15 & \ref{b:BonfiettiLBM14} & \ref{c:BonfiettiLBM14}\\
BonfiettiLBM12 \href{https://doi.org/10.1007/978-3-642-29828-8\_6}{BonfiettiLBM12} & \hyperref[auth:a203]{A. Bonfietti}, \hyperref[auth:a142]{M. Lombardi}, \hyperref[auth:a247]{L. Benini}, \hyperref[auth:a143]{M. Milano} & Global Cyclic Cumulative Constraint & \href{works/BonfiettiLBM12.pdf}{Yes} & \cite{BonfiettiLBM12} & 2012 & CPAIOR 2012 & 16 & 2 & 11 & \ref{b:BonfiettiLBM12} & \ref{c:BonfiettiLBM12}\\
BeniniLMR11 \href{https://doi.org/10.1007/s10479-010-0718-x}{BeniniLMR11} & \hyperref[auth:a247]{L. Benini}, \hyperref[auth:a142]{M. Lombardi}, \hyperref[auth:a143]{M. Milano}, \hyperref[auth:a727]{M. Ruggiero} & Optimal resource allocation and scheduling for the {CELL} {BE} platform & \href{works/BeniniLMR11.pdf}{Yes} & \cite{BeniniLMR11} & 2011 & Ann. Oper. Res. & 27 & 18 & 16 & \ref{b:BeniniLMR11} & \ref{c:BeniniLMR11}\\
BonfiettiLBM11 \href{https://doi.org/10.1007/978-3-642-23786-7\_12}{BonfiettiLBM11} & \hyperref[auth:a203]{A. Bonfietti}, \hyperref[auth:a142]{M. Lombardi}, \hyperref[auth:a247]{L. Benini}, \hyperref[auth:a143]{M. Milano} & A Constraint Based Approach to Cyclic {RCPSP} & \href{works/BonfiettiLBM11.pdf}{Yes} & \cite{BonfiettiLBM11} & 2011 & CP 2011 & 15 & 3 & 14 & \ref{b:BonfiettiLBM11} & \ref{c:BonfiettiLBM11}\\
LombardiBMB11 \href{https://doi.org/10.1007/978-3-642-21311-3\_14}{LombardiBMB11} & \hyperref[auth:a142]{M. Lombardi}, \hyperref[auth:a203]{A. Bonfietti}, \hyperref[auth:a143]{M. Milano}, \hyperref[auth:a247]{L. Benini} & Precedence Constraint Posting for Cyclic Scheduling Problems & \href{works/LombardiBMB11.pdf}{Yes} & \cite{LombardiBMB11} & 2011 & CPAIOR 2011 & 17 & 1 & 13 & \ref{b:LombardiBMB11} & \ref{c:LombardiBMB11}\\
RuggieroBBMA09 \href{https://doi.org/10.1109/TCAD.2009.2013536}{RuggieroBBMA09} & \hyperref[auth:a727]{M. Ruggiero}, \hyperref[auth:a379]{D. Bertozzi}, \hyperref[auth:a247]{L. Benini}, \hyperref[auth:a143]{M. Milano}, \hyperref[auth:a728]{A. Andrei} & Reducing the Abstraction and Optimality Gaps in the Allocation and Scheduling for Variable Voltage/Frequency MPSoC Platforms & \href{works/RuggieroBBMA09.pdf}{Yes} & \cite{RuggieroBBMA09} & 2009 & {IEEE} Trans. Comput. Aided Des. Integr. Circuits Syst. & 14 & 9 & 27 & \ref{b:RuggieroBBMA09} & \ref{c:RuggieroBBMA09}\\
BeniniBGM06 \href{https://doi.org/10.1007/11757375\_6}{BeniniBGM06} & \hyperref[auth:a247]{L. Benini}, \hyperref[auth:a379]{D. Bertozzi}, \hyperref[auth:a380]{A. Guerri}, \hyperref[auth:a143]{M. Milano} & Allocation, Scheduling and Voltage Scaling on Energy Aware MPSoCs & \href{works/BeniniBGM06.pdf}{Yes} & \cite{BeniniBGM06} & 2006 & CPAIOR 2006 & 15 & 18 & 10 & \ref{b:BeniniBGM06} & \ref{c:BeniniBGM06}\\
\end{longtable}
}

\subsection{Works by Alessio Bonfietti}
\label{sec:a203}
{\scriptsize
\begin{longtable}{>{\raggedright\arraybackslash}p{3cm}>{\raggedright\arraybackslash}p{6cm}>{\raggedright\arraybackslash}p{6.5cm}rrrp{2.5cm}rrrrr}
\rowcolor{white}\caption{Works from bibtex (Total 10)}\\ \toprule
\rowcolor{white}Key & Authors & Title & LC & Cite & Year & \shortstack{Conference\\/Journal} & Pages & \shortstack{Nr\\Cites} & \shortstack{Nr\\Refs} & b & c \\ \midrule\endhead
\bottomrule
\endfoot
Bonfietti16 \href{https://doi.org/10.3233/IA-160095}{Bonfietti16} & \hyperref[auth:a203]{A. Bonfietti} & A constraint programming scheduling solver for the MPOpt programming environment & \href{works/Bonfietti16.pdf}{Yes} & \cite{Bonfietti16} & 2016 & Intelligenza Artificiale & 13 & 0 & 19 & \ref{b:Bonfietti16} & \ref{c:Bonfietti16}\\
BonfiettiZLM16 \href{https://doi.org/10.1007/978-3-319-44953-1\_8}{BonfiettiZLM16} & \hyperref[auth:a203]{A. Bonfietti}, \hyperref[auth:a204]{A. Zanarini}, \hyperref[auth:a142]{M. Lombardi}, \hyperref[auth:a143]{M. Milano} & The Multirate Resource Constraint & \href{works/BonfiettiZLM16.pdf}{Yes} & \cite{BonfiettiZLM16} & 2016 & CP 2016 & 17 & 0 & 11 & \ref{b:BonfiettiZLM16} & \ref{c:BonfiettiZLM16}\\
LombardiBM15 \href{https://doi.org/10.1007/978-3-319-23219-5\_20}{LombardiBM15} & \hyperref[auth:a142]{M. Lombardi}, \hyperref[auth:a203]{A. Bonfietti}, \hyperref[auth:a143]{M. Milano} & Deterministic Estimation of the Expected Makespan of a {POS} Under Duration Uncertainty & \href{works/LombardiBM15.pdf}{Yes} & \cite{LombardiBM15} & 2015 & CP 2015 & 16 & 0 & 8 & \ref{b:LombardiBM15} & \ref{c:LombardiBM15}\\
BonfiettiLBM14 \href{https://doi.org/10.1016/j.artint.2013.09.006}{BonfiettiLBM14} & \hyperref[auth:a203]{A. Bonfietti}, \hyperref[auth:a142]{M. Lombardi}, \hyperref[auth:a247]{L. Benini}, \hyperref[auth:a143]{M. Milano} & {CROSS} cyclic resource-constrained scheduling solver & \href{works/BonfiettiLBM14.pdf}{Yes} & \cite{BonfiettiLBM14} & 2014 & Artif. Intell. & 28 & 8 & 15 & \ref{b:BonfiettiLBM14} & \ref{c:BonfiettiLBM14}\\
BonfiettiLM14 \href{https://doi.org/10.1007/978-3-319-07046-9\_15}{BonfiettiLM14} & \hyperref[auth:a203]{A. Bonfietti}, \hyperref[auth:a142]{M. Lombardi}, \hyperref[auth:a143]{M. Milano} & Disregarding Duration Uncertainty in Partial Order Schedules? Yes, We Can! & \href{works/BonfiettiLM14.pdf}{Yes} & \cite{BonfiettiLM14} & 2014 & CPAIOR 2014 & 16 & 3 & 12 & \ref{b:BonfiettiLM14} & \ref{c:BonfiettiLM14}\\
BonfiettiLM13 \href{http://www.aaai.org/ocs/index.php/ICAPS/ICAPS13/paper/view/6050}{BonfiettiLM13} & \hyperref[auth:a203]{A. Bonfietti}, \hyperref[auth:a142]{M. Lombardi}, \hyperref[auth:a143]{M. Milano} & De-Cycling Cyclic Scheduling Problems & \href{works/BonfiettiLM13.pdf}{Yes} & \cite{BonfiettiLM13} & 2013 & ICAPS 2013 & 5 & 0 & 0 & \ref{b:BonfiettiLM13} & \ref{c:BonfiettiLM13}\\
BonfiettiLBM12 \href{https://doi.org/10.1007/978-3-642-29828-8\_6}{BonfiettiLBM12} & \hyperref[auth:a203]{A. Bonfietti}, \hyperref[auth:a142]{M. Lombardi}, \hyperref[auth:a247]{L. Benini}, \hyperref[auth:a143]{M. Milano} & Global Cyclic Cumulative Constraint & \href{works/BonfiettiLBM12.pdf}{Yes} & \cite{BonfiettiLBM12} & 2012 & CPAIOR 2012 & 16 & 2 & 11 & \ref{b:BonfiettiLBM12} & \ref{c:BonfiettiLBM12}\\
BonfiettiM12 \href{https://ceur-ws.org/Vol-926/paper2.pdf}{BonfiettiM12} & \hyperref[auth:a203]{A. Bonfietti}, \hyperref[auth:a143]{M. Milano} & A Constraint-based Approach to Cyclic Resource-Constrained Scheduling Problem & \href{works/BonfiettiM12.pdf}{Yes} & \cite{BonfiettiM12} & 2012 & DC SIAAI 2012 & 3 & 0 & 0 & \ref{b:BonfiettiM12} & \ref{c:BonfiettiM12}\\
BonfiettiLBM11 \href{https://doi.org/10.1007/978-3-642-23786-7\_12}{BonfiettiLBM11} & \hyperref[auth:a203]{A. Bonfietti}, \hyperref[auth:a142]{M. Lombardi}, \hyperref[auth:a247]{L. Benini}, \hyperref[auth:a143]{M. Milano} & A Constraint Based Approach to Cyclic {RCPSP} & \href{works/BonfiettiLBM11.pdf}{Yes} & \cite{BonfiettiLBM11} & 2011 & CP 2011 & 15 & 3 & 14 & \ref{b:BonfiettiLBM11} & \ref{c:BonfiettiLBM11}\\
LombardiBMB11 \href{https://doi.org/10.1007/978-3-642-21311-3\_14}{LombardiBMB11} & \hyperref[auth:a142]{M. Lombardi}, \hyperref[auth:a203]{A. Bonfietti}, \hyperref[auth:a143]{M. Milano}, \hyperref[auth:a247]{L. Benini} & Precedence Constraint Posting for Cyclic Scheduling Problems & \href{works/LombardiBMB11.pdf}{Yes} & \cite{LombardiBMB11} & 2011 & CPAIOR 2011 & 17 & 1 & 13 & \ref{b:LombardiBMB11} & \ref{c:LombardiBMB11}\\
\end{longtable}
}

\subsection{Works by Pascal Van Hentenryck}
\label{sec:a148}
{\scriptsize
\begin{longtable}{>{\raggedright\arraybackslash}p{3cm}>{\raggedright\arraybackslash}p{6cm}>{\raggedright\arraybackslash}p{6.5cm}rrrp{2.5cm}rrrrr}
\rowcolor{white}\caption{Works from bibtex (Total 10)}\\ \toprule
\rowcolor{white}Key & Authors & Title & LC & Cite & Year & \shortstack{Conference\\/Journal} & Pages & \shortstack{Nr\\Cites} & \shortstack{Nr\\Refs} & b & c \\ \midrule\endhead
\bottomrule
\endfoot
FontaineMH16 \href{https://doi.org/10.1007/978-3-319-33954-2\_12}{FontaineMH16} & \hyperref[auth:a320]{D. Fontaine}, \hyperref[auth:a321]{Laurent D. Michel}, \hyperref[auth:a148]{Pascal Van Hentenryck} & Parallel Composition of Scheduling Solvers & \href{works/FontaineMH16.pdf}{Yes} & \cite{FontaineMH16} & 2016 & CPAIOR 2016 & 11 & 3 & 0 & \ref{b:FontaineMH16} & \ref{c:FontaineMH16}\\
EvenSH15 \href{https://doi.org/10.1007/978-3-319-23219-5\_40}{EvenSH15} & \hyperref[auth:a219]{C. Even}, \hyperref[auth:a124]{A. Schutt}, \hyperref[auth:a148]{Pascal Van Hentenryck} & A Constraint Programming Approach for Non-preemptive Evacuation Scheduling & \href{works/EvenSH15.pdf}{Yes} & \cite{EvenSH15} & 2015 & CP 2015 & 18 & 3 & 12 & \ref{b:EvenSH15} & \ref{c:EvenSH15}\\
EvenSH15a \href{http://arxiv.org/abs/1505.02487}{EvenSH15a} & \hyperref[auth:a219]{C. Even}, \hyperref[auth:a124]{A. Schutt}, \hyperref[auth:a148]{Pascal Van Hentenryck} & A Constraint Programming Approach for Non-Preemptive Evacuation Scheduling & \href{works/EvenSH15a.pdf}{Yes} & \cite{EvenSH15a} & 2015 & CoRR & 16 & 0 & 0 & \ref{b:EvenSH15a} & \ref{c:EvenSH15a}\\
SchausHMCMD11 \href{https://doi.org/10.1007/s10601-010-9100-5}{SchausHMCMD11} & \hyperref[auth:a147]{P. Schaus}, \hyperref[auth:a148]{Pascal Van Hentenryck}, \hyperref[auth:a149]{J. Monette}, \hyperref[auth:a150]{C. Coffrin}, \hyperref[auth:a32]{L. Michel}, \hyperref[auth:a151]{Y. Deville} & Solving Steel Mill Slab Problems with constraint-based techniques: CP, LNS, and {CBLS} & \href{works/SchausHMCMD11.pdf}{Yes} & \cite{SchausHMCMD11} & 2011 & Constraints An Int. J. & 23 & 14 & 5 & \ref{b:SchausHMCMD11} & \ref{c:SchausHMCMD11}\\
MonetteDH09 \href{http://aaai.org/ocs/index.php/ICAPS/ICAPS09/paper/view/712}{MonetteDH09} & \hyperref[auth:a149]{J. Monette}, \hyperref[auth:a151]{Y. Deville}, \hyperref[auth:a148]{Pascal Van Hentenryck} & Just-In-Time Scheduling with Constraint Programming & \href{works/MonetteDH09.pdf}{Yes} & \cite{MonetteDH09} & 2009 & ICAPS 2009 & 8 & 0 & 0 & \ref{b:MonetteDH09} & \ref{c:MonetteDH09}\\
DoomsH08 \href{https://doi.org/10.1007/978-3-540-68155-7\_8}{DoomsH08} & \hyperref[auth:a363]{G. Dooms}, \hyperref[auth:a148]{Pascal Van Hentenryck} & Gap Reduction Techniques for Online Stochastic Project Scheduling & \href{works/DoomsH08.pdf}{Yes} & \cite{DoomsH08} & 2008 & CPAIOR 2008 & 16 & 1 & 2 & \ref{b:DoomsH08} & \ref{c:DoomsH08}\\
HentenryckM08 \href{https://doi.org/10.1007/978-3-540-68155-7\_41}{HentenryckM08} & \hyperref[auth:a148]{Pascal Van Hentenryck}, \hyperref[auth:a32]{L. Michel} & The Steel Mill Slab Design Problem Revisited & \href{works/HentenryckM08.pdf}{Yes} & \cite{HentenryckM08} & 2008 & CPAIOR 2008 & 5 & 13 & 3 & \ref{b:HentenryckM08} & \ref{c:HentenryckM08}\\
MercierH08 \href{http://dx.doi.org/10.1287/ijoc.1070.0226}{MercierH08} & \hyperref[auth:a865]{L. Mercier}, \hyperref[auth:a148]{Pascal Van Hentenryck} & Edge Finding for Cumulative Scheduling & No & \cite{MercierH08} & 2008 & INFORMS Journal on Computing & null & 32 & 5 & No & \ref{c:MercierH08}\\
HentenryckM04 \href{https://doi.org/10.1007/978-3-540-24664-0\_22}{HentenryckM04} & \hyperref[auth:a148]{Pascal Van Hentenryck}, \hyperref[auth:a32]{L. Michel} & Scheduling Abstractions for Local Search & \href{works/HentenryckM04.pdf}{Yes} & \cite{HentenryckM04} & 2004 & CPAIOR 2004 & 16 & 12 & 14 & \ref{b:HentenryckM04} & \ref{c:HentenryckM04}\\
DincbasSH90 \href{https://doi.org/10.1016/0743-1066(90)90052-7}{DincbasSH90} & \hyperref[auth:a726]{M. Dincbas}, \hyperref[auth:a17]{H. Simonis}, \hyperref[auth:a148]{Pascal Van Hentenryck} & Solving Large Combinatorial Problems in Logic Programming & \href{works/DincbasSH90.pdf}{Yes} & \cite{DincbasSH90} & 1990 & J. Log. Program. & 19 & 86 & 9 & \ref{b:DincbasSH90} & \ref{c:DincbasSH90}\\
\end{longtable}
}

\subsection{Works by Philippe Baptiste}
\label{sec:a163}
{\scriptsize
\begin{longtable}{>{\raggedright\arraybackslash}p{3cm}>{\raggedright\arraybackslash}p{6cm}>{\raggedright\arraybackslash}p{6.5cm}rrrp{2.5cm}rrrrr}
\rowcolor{white}\caption{Works from bibtex (Total 9)}\\ \toprule
\rowcolor{white}Key & Authors & Title & LC & Cite & Year & \shortstack{Conference\\/Journal} & Pages & \shortstack{Nr\\Cites} & \shortstack{Nr\\Refs} & b & c \\ \midrule\endhead
\bottomrule
\endfoot
BaptisteB18 \href{https://doi.org/10.1016/j.dam.2017.05.001}{BaptisteB18} & \hyperref[auth:a163]{P. Baptiste}, \hyperref[auth:a714]{N. Bonifas} & Redundant cumulative constraints to compute preemptive bounds & \href{works/BaptisteB18.pdf}{Yes} & \cite{BaptisteB18} & 2018 & Discret. Appl. Math. & 10 & 3 & 13 & \ref{b:BaptisteB18} & \ref{c:BaptisteB18}\\
Baptiste09 \href{https://doi.org/10.1007/978-3-642-04244-7\_1}{Baptiste09} & \hyperref[auth:a163]{P. Baptiste} & Constraint-Based Schedulers, Do They Really Work? & \href{works/Baptiste09.pdf}{Yes} & \cite{Baptiste09} & 2009 & CP 2009 & 1 & 0 & 0 & \ref{b:Baptiste09} & \ref{c:Baptiste09}\\
BaptisteLPN06 \href{https://doi.org/10.1016/S1574-6526(06)80026-X}{BaptisteLPN06} & \hyperref[auth:a163]{P. Baptiste}, \hyperref[auth:a118]{P. Laborie}, \hyperref[auth:a164]{Claude Le Pape}, \hyperref[auth:a666]{W. Nuijten} & Constraint-Based Scheduling and Planning & No & \cite{BaptisteLPN06} & 2006 & Handbook of Constraint Programming & 39 & 30 & 25 & No & \ref{c:BaptisteLPN06}\\
ArtiouchineB05 \href{https://doi.org/10.1007/11564751\_8}{ArtiouchineB05} & \hyperref[auth:a264]{K. Artiouchine}, \hyperref[auth:a163]{P. Baptiste} & Inter-distance Constraint: An Extension of the All-Different Constraint for Scheduling Equal Length Jobs & \href{works/ArtiouchineB05.pdf}{Yes} & \cite{ArtiouchineB05} & 2005 & CP 2005 & 15 & 3 & 11 & \ref{b:ArtiouchineB05} & \ref{c:ArtiouchineB05}\\
BaptistePN01 \href{http://dx.doi.org/10.1007/978-1-4615-1479-4}{BaptistePN01} & \hyperref[auth:a163]{P. Baptiste}, \hyperref[auth:a164]{Claude Le Pape}, \hyperref[auth:a666]{W. Nuijten} & Constraint-Based Scheduling & No & \cite{BaptistePN01} & 2001 & Book & null & 296 & 0 & No & \ref{c:BaptistePN01}\\
BaptisteP00 \href{https://doi.org/10.1023/A:1009822502231}{BaptisteP00} & \hyperref[auth:a163]{P. Baptiste}, \hyperref[auth:a164]{Claude Le Pape} & Constraint Propagation and Decomposition Techniques for Highly Disjunctive and Highly Cumulative Project Scheduling Problems & \href{works/BaptisteP00.pdf}{Yes} & \cite{BaptisteP00} & 2000 & Constraints An Int. J. & 21 & 46 & 0 & \ref{b:BaptisteP00} & \ref{c:BaptisteP00}\\
PapaB98 \href{https://doi.org/10.1023/A:1009723704757}{PapaB98} & \hyperref[auth:a164]{Claude Le Pape}, \hyperref[auth:a163]{P. Baptiste} & Resource Constraints for Preemptive Job-shop Scheduling & \href{works/PapaB98.pdf}{Yes} & \cite{PapaB98} & 1998 & Constraints An Int. J. & 25 & 14 & 0 & \ref{b:PapaB98} & \ref{c:PapaB98}\\
BaptisteP97 \href{https://doi.org/10.1007/BFb0017454}{BaptisteP97} & \hyperref[auth:a163]{P. Baptiste}, \hyperref[auth:a164]{Claude Le Pape} & Constraint Propagation and Decomposition Techniques for Highly Disjunctive and Highly Cumulative Project Scheduling Problems & \href{works/BaptisteP97.pdf}{Yes} & \cite{BaptisteP97} & 1997 & CP 1997 & 15 & 8 & 10 & \ref{b:BaptisteP97} & \ref{c:BaptisteP97}\\
PapeB97 \href{}{PapeB97} & \hyperref[auth:a164]{Claude Le Pape}, \hyperref[auth:a163]{P. Baptiste} & A Constraint Programming Library for Preemptive and Non-Preemptive Scheduling & No & \cite{PapeB97} & 1997 & PACT 1997 & 20 & 0 & 0 & No & \ref{c:PapeB97}\\
\end{longtable}
}

\subsection{Works by Nysret Musliu}
\label{sec:a45}
{\scriptsize
\begin{longtable}{>{\raggedright\arraybackslash}p{3cm}>{\raggedright\arraybackslash}p{6cm}>{\raggedright\arraybackslash}p{6.5cm}rrrp{2.5cm}rrrrr}
\rowcolor{white}\caption{Works from bibtex (Total 9)}\\ \toprule
\rowcolor{white}Key & Authors & Title & LC & Cite & Year & \shortstack{Conference\\/Journal} & Pages & \shortstack{Nr\\Cites} & \shortstack{Nr\\Refs} & b & c \\ \midrule\endhead
\bottomrule
\endfoot
LacknerMMWW23 \href{https://doi.org/10.1007/s10601-023-09347-2}{LacknerMMWW23} & \hyperref[auth:a62]{M. Lackner}, \hyperref[auth:a63]{C. Mrkvicka}, \hyperref[auth:a45]{N. Musliu}, \hyperref[auth:a46]{D. Walkiewicz}, \hyperref[auth:a43]{F. Winter} & Exact methods for the Oven Scheduling Problem & \href{works/LacknerMMWW23.pdf}{Yes} & \cite{LacknerMMWW23} & 2023 & Constraints An Int. J. & 42 & 0 & 32 & \ref{b:LacknerMMWW23} & \ref{c:LacknerMMWW23}\\
WinterMMW22 \href{https://doi.org/10.4230/LIPIcs.CP.2022.41}{WinterMMW22} & \hyperref[auth:a43]{F. Winter}, \hyperref[auth:a44]{S. Meiswinkel}, \hyperref[auth:a45]{N. Musliu}, \hyperref[auth:a46]{D. Walkiewicz} & Modeling and Solving Parallel Machine Scheduling with Contamination Constraints in the Agricultural Industry & \href{works/WinterMMW22.pdf}{Yes} & \cite{WinterMMW22} & 2022 & CP 2022 & 18 & 0 & 0 & \ref{b:WinterMMW22} & \ref{c:WinterMMW22}\\
GeibingerKKMMW21 \href{https://doi.org/10.1007/978-3-030-78230-6\_29}{GeibingerKKMMW21} & \hyperref[auth:a77]{T. Geibinger}, \hyperref[auth:a78]{L. Kletzander}, \hyperref[auth:a79]{M. Krainz}, \hyperref[auth:a80]{F. Mischek}, \hyperref[auth:a45]{N. Musliu}, \hyperref[auth:a43]{F. Winter} & Physician Scheduling During a Pandemic & \href{works/GeibingerKKMMW21.pdf}{Yes} & \cite{GeibingerKKMMW21} & 2021 & CPAIOR 2021 & 10 & 0 & 6 & \ref{b:GeibingerKKMMW21} & \ref{c:GeibingerKKMMW21}\\
GeibingerMM21 \href{https://doi.org/10.1609/aaai.v35i7.16789}{GeibingerMM21} & \hyperref[auth:a77]{T. Geibinger}, \hyperref[auth:a80]{F. Mischek}, \hyperref[auth:a45]{N. Musliu} & Constraint Logic Programming for Real-World Test Laboratory Scheduling & \href{works/GeibingerMM21.pdf}{Yes} & \cite{GeibingerMM21} & 2021 & AAAI 2021 & 9 & 0 & 0 & \ref{b:GeibingerMM21} & \ref{c:GeibingerMM21}\\
LacknerMMWW21 \href{https://doi.org/10.4230/LIPIcs.CP.2021.37}{LacknerMMWW21} & \hyperref[auth:a62]{M. Lackner}, \hyperref[auth:a63]{C. Mrkvicka}, \hyperref[auth:a45]{N. Musliu}, \hyperref[auth:a46]{D. Walkiewicz}, \hyperref[auth:a43]{F. Winter} & Minimizing Cumulative Batch Processing Time for an Industrial Oven Scheduling Problem & \href{works/LacknerMMWW21.pdf}{Yes} & \cite{LacknerMMWW21} & 2021 & CP 2021 & 18 & 0 & 0 & \ref{b:LacknerMMWW21} & \ref{c:LacknerMMWW21}\\
GeibingerMM19 \href{https://doi.org/10.1007/978-3-030-19212-9\_20}{GeibingerMM19} & \hyperref[auth:a77]{T. Geibinger}, \hyperref[auth:a80]{F. Mischek}, \hyperref[auth:a45]{N. Musliu} & Investigating Constraint Programming for Real World Industrial Test Laboratory Scheduling & \href{works/GeibingerMM19.pdf}{Yes} & \cite{GeibingerMM19} & 2019 & CPAIOR 2019 & 16 & 6 & 15 & \ref{b:GeibingerMM19} & \ref{c:GeibingerMM19}\\
abs-1911-04766 \href{http://arxiv.org/abs/1911.04766}{abs-1911-04766} & \hyperref[auth:a77]{T. Geibinger}, \hyperref[auth:a80]{F. Mischek}, \hyperref[auth:a45]{N. Musliu} & Investigating Constraint Programming and Hybrid Methods for Real World Industrial Test Laboratory Scheduling & \href{works/abs-1911-04766.pdf}{Yes} & \cite{abs-1911-04766} & 2019 & CoRR & 16 & 0 & 0 & \ref{b:abs-1911-04766} & \ref{c:abs-1911-04766}\\
MusliuSS18 \href{https://doi.org/10.1007/978-3-319-93031-2\_31}{MusliuSS18} & \hyperref[auth:a45]{N. Musliu}, \hyperref[auth:a124]{A. Schutt}, \hyperref[auth:a125]{Peter J. Stuckey} & Solver Independent Rotating Workforce Scheduling & \href{works/MusliuSS18.pdf}{Yes} & \cite{MusliuSS18} & 2018 & CPAIOR 2018 & 17 & 7 & 23 & \ref{b:MusliuSS18} & \ref{c:MusliuSS18}\\
KletzanderM17 \href{https://doi.org/10.1007/978-3-319-59776-8\_28}{KletzanderM17} & \hyperref[auth:a78]{L. Kletzander}, \hyperref[auth:a45]{N. Musliu} & A Multi-stage Simulated Annealing Algorithm for the Torpedo Scheduling Problem & \href{works/KletzanderM17.pdf}{Yes} & \cite{KletzanderM17} & 2017 & CPAIOR 2017 & 15 & 1 & 9 & \ref{b:KletzanderM17} & \ref{c:KletzanderM17}\\
\end{longtable}
}

\subsection{Works by Claude{-}Guy Quimper}
\label{sec:a37}
{\scriptsize
\begin{longtable}{>{\raggedright\arraybackslash}p{3cm}>{\raggedright\arraybackslash}p{6cm}>{\raggedright\arraybackslash}p{6.5cm}rrrp{2.5cm}rrrrr}
\rowcolor{white}\caption{Works from bibtex (Total 9)}\\ \toprule
\rowcolor{white}Key & Authors & Title & LC & Cite & Year & \shortstack{Conference\\/Journal} & Pages & \shortstack{Nr\\Cites} & \shortstack{Nr\\Refs} & b & c \\ \midrule\endhead
\bottomrule
\endfoot
BoudreaultSLQ22 \href{https://doi.org/10.4230/LIPIcs.CP.2022.10}{BoudreaultSLQ22} & \hyperref[auth:a34]{R. Boudreault}, \hyperref[auth:a35]{V. Simard}, \hyperref[auth:a36]{D. Lafond}, \hyperref[auth:a37]{C. Quimper} & A Constraint Programming Approach to Ship Refit Project Scheduling & \href{works/BoudreaultSLQ22.pdf}{Yes} & \cite{BoudreaultSLQ22} & 2022 & CP 2022 & 16 & 0 & 0 & \ref{b:BoudreaultSLQ22} & \ref{c:BoudreaultSLQ22}\\
OuelletQ22 \href{https://doi.org/10.1007/978-3-031-08011-1\_21}{OuelletQ22} & \hyperref[auth:a52]{Y. Ouellet}, \hyperref[auth:a37]{C. Quimper} & A MinCumulative Resource Constraint & \href{works/OuelletQ22.pdf}{Yes} & \cite{OuelletQ22} & 2022 & CPAIOR 2022 & 17 & 1 & 22 & \ref{b:OuelletQ22} & \ref{c:OuelletQ22}\\
Mercier-AubinGQ20 \href{https://doi.org/10.1007/978-3-030-58942-4\_22}{Mercier-AubinGQ20} & \hyperref[auth:a86]{A. Mercier{-}Aubin}, \hyperref[auth:a87]{J. Gaudreault}, \hyperref[auth:a37]{C. Quimper} & Leveraging Constraint Scheduling: {A} Case Study to the Textile Industry & \href{works/Mercier-AubinGQ20.pdf}{Yes} & \cite{Mercier-AubinGQ20} & 2020 & CPAIOR 2020 & 13 & 2 & 13 & \ref{b:Mercier-AubinGQ20} & \ref{c:Mercier-AubinGQ20}\\
FahimiOQ18 \href{https://doi.org/10.1007/s10601-018-9282-9}{FahimiOQ18} & \hyperref[auth:a122]{H. Fahimi}, \hyperref[auth:a52]{Y. Ouellet}, \hyperref[auth:a37]{C. Quimper} & Linear-time filtering algorithms for the disjunctive constraint and a quadratic filtering algorithm for the cumulative not-first not-last & \href{works/FahimiOQ18.pdf}{Yes} & \cite{FahimiOQ18} & 2018 & Constraints An Int. J. & 22 & 2 & 20 & \ref{b:FahimiOQ18} & \ref{c:FahimiOQ18}\\
KameugneFGOQ18 \href{https://doi.org/10.1007/978-3-319-93031-2\_23}{KameugneFGOQ18} & \hyperref[auth:a10]{R. Kameugne}, \hyperref[auth:a11]{S{\'{e}}v{\'{e}}rine Betmbe Fetgo}, \hyperref[auth:a315]{V. Gingras}, \hyperref[auth:a52]{Y. Ouellet}, \hyperref[auth:a37]{C. Quimper} & Horizontally Elastic Not-First/Not-Last Filtering Algorithm for Cumulative Resource Constraint & \href{works/KameugneFGOQ18.pdf}{Yes} & \cite{KameugneFGOQ18} & 2018 & CPAIOR 2018 & 17 & 1 & 12 & \ref{b:KameugneFGOQ18} & \ref{c:KameugneFGOQ18}\\
OuelletQ18 \href{https://doi.org/10.1007/978-3-319-93031-2\_34}{OuelletQ18} & \hyperref[auth:a52]{Y. Ouellet}, \hyperref[auth:a37]{C. Quimper} & A O(n {\textbackslash}log {\^{}}2 n) Checker and O(n{\^{}}2 {\textbackslash}log n) Filtering Algorithm for the Energetic Reasoning & \href{works/OuelletQ18.pdf}{Yes} & \cite{OuelletQ18} & 2018 & CPAIOR 2018 & 18 & 6 & 16 & \ref{b:OuelletQ18} & \ref{c:OuelletQ18}\\
GingrasQ16 \href{http://www.ijcai.org/Abstract/16/440}{GingrasQ16} & \hyperref[auth:a315]{V. Gingras}, \hyperref[auth:a37]{C. Quimper} & Generalizing the Edge-Finder Rule for the Cumulative Constraint & \href{works/GingrasQ16.pdf}{Yes} & \cite{GingrasQ16} & 2016 & IJCAI 2016 & 7 & 0 & 0 & \ref{b:GingrasQ16} & \ref{c:GingrasQ16}\\
BessiereHMQW14 \href{https://doi.org/10.1007/978-3-319-07046-9\_23}{BessiereHMQW14} & \hyperref[auth:a333]{C. Bessiere}, \hyperref[auth:a1]{E. Hebrard}, \hyperref[auth:a334]{M. M{\'{e}}nard}, \hyperref[auth:a37]{C. Quimper}, \hyperref[auth:a278]{T. Walsh} & Buffered Resource Constraint: Algorithms and Complexity & \href{works/BessiereHMQW14.pdf}{Yes} & \cite{BessiereHMQW14} & 2014 & CPAIOR 2014 & 16 & 1 & 3 & \ref{b:BessiereHMQW14} & \ref{c:BessiereHMQW14}\\
OuelletQ13 \href{https://doi.org/10.1007/978-3-642-40627-0\_42}{OuelletQ13} & \hyperref[auth:a240]{P. Ouellet}, \hyperref[auth:a37]{C. Quimper} & Time-Table Extended-Edge-Finding for the Cumulative Constraint & \href{works/OuelletQ13.pdf}{Yes} & \cite{OuelletQ13} & 2013 & CP 2013 & 16 & 12 & 14 & \ref{b:OuelletQ13} & \ref{c:OuelletQ13}\\
\end{longtable}
}

\subsection{Works by Tony T. Tran}
\label{sec:a810}
{\scriptsize
\begin{longtable}{>{\raggedright\arraybackslash}p{3cm}>{\raggedright\arraybackslash}p{6cm}>{\raggedright\arraybackslash}p{6.5cm}rrrp{2.5cm}rrrrr}
\rowcolor{white}\caption{Works from bibtex (Total 9)}\\ \toprule
\rowcolor{white}Key & Authors & Title & LC & Cite & Year & \shortstack{Conference\\/Journal} & Pages & \shortstack{Nr\\Cites} & \shortstack{Nr\\Refs} & b & c \\ \midrule\endhead
\bottomrule
\endfoot
TranPZLDB18 \href{https://doi.org/10.1007/s10951-017-0537-x}{TranPZLDB18} & \hyperref[auth:a810]{Tony T. Tran}, \hyperref[auth:a811]{M. Padmanabhan}, \hyperref[auth:a812]{Peter Yun Zhang}, \hyperref[auth:a813]{H. Li}, \hyperref[auth:a814]{Douglas G. Down}, \hyperref[auth:a89]{J. Christopher Beck} & Multi-stage resource-aware scheduling for data centers with heterogeneous servers & \href{works/TranPZLDB18.pdf}{Yes} & \cite{TranPZLDB18} & 2018 & J. Sched. & 17 & 8 & 26 & \ref{b:TranPZLDB18} & \ref{c:TranPZLDB18}\\
TranVNB17 \href{https://doi.org/10.1613/jair.5306}{TranVNB17} & \hyperref[auth:a810]{Tony T. Tran}, \hyperref[auth:a815]{Tiago Stegun Vaquero}, \hyperref[auth:a209]{G. Nejat}, \hyperref[auth:a89]{J. Christopher Beck} & Robots in Retirement Homes: Applying Off-the-Shelf Planning and Scheduling to a Team of Assistive Robots & \href{works/TranVNB17.pdf}{Yes} & \cite{TranVNB17} & 2017 & J. Artif. Intell. Res. & 68 & 12 & 0 & \ref{b:TranVNB17} & \ref{c:TranVNB17}\\
TranVNB17a \href{https://doi.org/10.24963/ijcai.2017/726}{TranVNB17a} & \hyperref[auth:a810]{Tony T. Tran}, \hyperref[auth:a815]{Tiago Stegun Vaquero}, \hyperref[auth:a209]{G. Nejat}, \hyperref[auth:a89]{J. Christopher Beck} & Robots in Retirement Homes: Applying Off-the-Shelf Planning and Scheduling to a Team of Assistive Robots (Extended Abstract) & \href{works/TranVNB17a.pdf}{Yes} & \cite{TranVNB17a} & 2017 & IJCAI 2017 & 5 & 1 & 0 & \ref{b:TranVNB17a} & \ref{c:TranVNB17a}\\
TranAB16 \href{https://doi.org/10.1287/ijoc.2015.0666}{TranAB16} & \hyperref[auth:a810]{Tony T. Tran}, \hyperref[auth:a818]{A. Araujo}, \hyperref[auth:a89]{J. Christopher Beck} & Decomposition Methods for the Parallel Machine Scheduling Problem with Setups & No & \cite{TranAB16} & 2016 & {INFORMS} J. Comput. & 13 & 72 & 28 & No & \ref{c:TranAB16}\\
TranDRFWOVB16 \href{https://doi.org/10.1609/socs.v7i1.18390}{TranDRFWOVB16} & \hyperref[auth:a810]{Tony T. Tran}, \hyperref[auth:a820]{M. Do}, \hyperref[auth:a821]{Eleanor Gilbert Rieffel}, \hyperref[auth:a383]{J. Frank}, \hyperref[auth:a819]{Z. Wang}, \hyperref[auth:a822]{B. O'Gorman}, \hyperref[auth:a823]{D. Venturelli}, \hyperref[auth:a89]{J. Christopher Beck} & A Hybrid Quantum-Classical Approach to Solving Scheduling Problems & \href{works/TranDRFWOVB16.pdf}{Yes} & \cite{TranDRFWOVB16} & 2016 & SOCS 2016 & 9 & 3 & 0 & \ref{b:TranDRFWOVB16} & \ref{c:TranDRFWOVB16}\\
TranWDRFOVB16 \href{http://www.aaai.org/ocs/index.php/WS/AAAIW16/paper/view/12664}{TranWDRFOVB16} & \hyperref[auth:a810]{Tony T. Tran}, \hyperref[auth:a819]{Z. Wang}, \hyperref[auth:a820]{M. Do}, \hyperref[auth:a821]{Eleanor Gilbert Rieffel}, \hyperref[auth:a383]{J. Frank}, \hyperref[auth:a822]{B. O'Gorman}, \hyperref[auth:a823]{D. Venturelli}, \hyperref[auth:a89]{J. Christopher Beck} & Explorations of Quantum-Classical Approaches to Scheduling a Mars Lander Activity Problem & \href{works/TranWDRFOVB16.pdf}{Yes} & \cite{TranWDRFOVB16} & 2016 & AAAI 2016 & 9 & 0 & 0 & \ref{b:TranWDRFOVB16} & \ref{c:TranWDRFOVB16}\\
TerekhovTDB14 \href{https://doi.org/10.1613/jair.4278}{TerekhovTDB14} & \hyperref[auth:a829]{D. Terekhov}, \hyperref[auth:a810]{Tony T. Tran}, \hyperref[auth:a814]{Douglas G. Down}, \hyperref[auth:a89]{J. Christopher Beck} & Integrating Queueing Theory and Scheduling for Dynamic Scheduling Problems & \href{works/TerekhovTDB14.pdf}{Yes} & \cite{TerekhovTDB14} & 2014 & J. Artif. Intell. Res. & 38 & 12 & 0 & \ref{b:TerekhovTDB14} & \ref{c:TerekhovTDB14}\\
TranTDB13 \href{http://www.aaai.org/ocs/index.php/ICAPS/ICAPS13/paper/view/6005}{TranTDB13} & \hyperref[auth:a810]{Tony T. Tran}, \hyperref[auth:a829]{D. Terekhov}, \hyperref[auth:a814]{Douglas G. Down}, \hyperref[auth:a89]{J. Christopher Beck} & Hybrid Queueing Theory and Scheduling Models for Dynamic Environments with Sequence-Dependent Setup Times & \href{works/TranTDB13.pdf}{Yes} & \cite{TranTDB13} & 2013 & ICAPS 2013 & 9 & 0 & 0 & \ref{b:TranTDB13} & \ref{c:TranTDB13}\\
TranB12 \href{https://doi.org/10.3233/978-1-61499-098-7-774}{TranB12} & \hyperref[auth:a810]{Tony T. Tran}, \hyperref[auth:a89]{J. Christopher Beck} & Logic-based Benders Decomposition for Alternative Resource Scheduling with Sequence Dependent Setups & \href{works/TranB12.pdf}{Yes} & \cite{TranB12} & 2012 & ECAI 2012 & 6 & 0 & 0 & \ref{b:TranB12} & \ref{c:TranB12}\\
\end{longtable}
}

\subsection{Works by Mats Carlsson}
\label{sec:a91}
{\scriptsize
\begin{longtable}{>{\raggedright\arraybackslash}p{3cm}>{\raggedright\arraybackslash}p{6cm}>{\raggedright\arraybackslash}p{6.5cm}rrrp{2.5cm}rrrrr}
\rowcolor{white}\caption{Works from bibtex (Total 8)}\\ \toprule
\rowcolor{white}Key & Authors & Title & LC & Cite & Year & \shortstack{Conference\\/Journal} & Pages & \shortstack{Nr\\Cites} & \shortstack{Nr\\Refs} & b & c \\ \midrule\endhead
\bottomrule
\endfoot
WessenCS20 \href{https://doi.org/10.1007/978-3-030-58942-4\_33}{WessenCS20} & \hyperref[auth:a90]{J. Wess{\'{e}}n}, \hyperref[auth:a91]{M. Carlsson}, \hyperref[auth:a92]{C. Schulte} & Scheduling of Dual-Arm Multi-tool Assembly Robots and Workspace Layout Optimization & \href{works/WessenCS20.pdf}{Yes} & \cite{WessenCS20} & 2020 & CPAIOR 2020 & 10 & 2 & 11 & \ref{b:WessenCS20} & \ref{c:WessenCS20}\\
MossigeGSMC17 \href{https://doi.org/10.1007/978-3-319-66158-2\_25}{MossigeGSMC17} & \hyperref[auth:a199]{M. Mossige}, \hyperref[auth:a200]{A. Gotlieb}, \hyperref[auth:a201]{H. Spieker}, \hyperref[auth:a202]{H. Meling}, \hyperref[auth:a91]{M. Carlsson} & Time-Aware Test Case Execution Scheduling for Cyber-Physical Systems & \href{works/MossigeGSMC17.pdf}{Yes} & \cite{MossigeGSMC17} & 2017 & CP 2017 & 18 & 6 & 33 & \ref{b:MossigeGSMC17} & \ref{c:MossigeGSMC17}\\
LetortCB15 \href{https://doi.org/10.1007/s10601-014-9172-8}{LetortCB15} & \hyperref[auth:a127]{A. Letort}, \hyperref[auth:a91]{M. Carlsson}, \hyperref[auth:a128]{N. Beldiceanu} & Synchronized sweep algorithms for scalable scheduling constraints & \href{works/LetortCB15.pdf}{Yes} & \cite{LetortCB15} & 2015 & Constraints An Int. J. & 52 & 2 & 14 & \ref{b:LetortCB15} & \ref{c:LetortCB15}\\
LetortCB13 \href{https://doi.org/10.1007/978-3-642-38171-3\_10}{LetortCB13} & \hyperref[auth:a127]{A. Letort}, \hyperref[auth:a91]{M. Carlsson}, \hyperref[auth:a128]{N. Beldiceanu} & A Synchronized Sweep Algorithm for the \emph{k-dimensional cumulative} Constraint & \href{works/LetortCB13.pdf}{Yes} & \cite{LetortCB13} & 2013 & CPAIOR 2013 & 16 & 3 & 10 & \ref{b:LetortCB13} & \ref{c:LetortCB13}\\
LetortBC12 \href{https://doi.org/10.1007/978-3-642-33558-7\_33}{LetortBC12} & \hyperref[auth:a127]{A. Letort}, \hyperref[auth:a128]{N. Beldiceanu}, \hyperref[auth:a91]{M. Carlsson} & A Scalable Sweep Algorithm for the cumulative Constraint & \href{works/LetortBC12.pdf}{Yes} & \cite{LetortBC12} & 2012 & CP 2012 & 16 & 18 & 12 & \ref{b:LetortBC12} & \ref{c:LetortBC12}\\
BeldiceanuCDP11 \href{https://doi.org/10.1007/s10479-010-0731-0}{BeldiceanuCDP11} & \hyperref[auth:a128]{N. Beldiceanu}, \hyperref[auth:a91]{M. Carlsson}, \hyperref[auth:a245]{S. Demassey}, \hyperref[auth:a362]{E. Poder} & New filtering for the \emph{cumulative} constraint in the context of non-overlapping rectangles & \href{works/BeldiceanuCDP11.pdf}{Yes} & \cite{BeldiceanuCDP11} & 2011 & Ann. Oper. Res. & 24 & 8 & 8 & \ref{b:BeldiceanuCDP11} & \ref{c:BeldiceanuCDP11}\\
BeldiceanuCP08 \href{https://doi.org/10.1007/978-3-540-68155-7\_5}{BeldiceanuCP08} & \hyperref[auth:a128]{N. Beldiceanu}, \hyperref[auth:a91]{M. Carlsson}, \hyperref[auth:a362]{E. Poder} & New Filtering for the cumulative Constraint in the Context of Non-Overlapping Rectangles & \href{works/BeldiceanuCP08.pdf}{Yes} & \cite{BeldiceanuCP08} & 2008 & CPAIOR 2008 & 15 & 8 & 9 & \ref{b:BeldiceanuCP08} & \ref{c:BeldiceanuCP08}\\
BeldiceanuC02 \href{https://doi.org/10.1007/3-540-46135-3\_5}{BeldiceanuC02} & \hyperref[auth:a128]{N. Beldiceanu}, \hyperref[auth:a91]{M. Carlsson} & A New Multi-resource cumulatives Constraint with Negative Heights & \href{works/BeldiceanuC02.pdf}{Yes} & \cite{BeldiceanuC02} & 2002 & CP 2002 & 17 & 33 & 9 & \ref{b:BeldiceanuC02} & \ref{c:BeldiceanuC02}\\
\end{longtable}
}

\subsection{Works by Claude Le Pape}
\label{sec:a164}
{\scriptsize
\begin{longtable}{>{\raggedright\arraybackslash}p{3cm}>{\raggedright\arraybackslash}p{6cm}>{\raggedright\arraybackslash}p{6.5cm}rrrp{2.5cm}rrrrr}
\rowcolor{white}\caption{Works from bibtex (Total 8)}\\ \toprule
\rowcolor{white}Key & Authors & Title & LC & Cite & Year & \shortstack{Conference\\/Journal} & Pages & \shortstack{Nr\\Cites} & \shortstack{Nr\\Refs} & b & c \\ \midrule\endhead
\bottomrule
\endfoot
BaptisteLPN06 \href{https://doi.org/10.1016/S1574-6526(06)80026-X}{BaptisteLPN06} & \hyperref[auth:a163]{P. Baptiste}, \hyperref[auth:a118]{P. Laborie}, \hyperref[auth:a164]{Claude Le Pape}, \hyperref[auth:a666]{W. Nuijten} & Constraint-Based Scheduling and Planning & No & \cite{BaptisteLPN06} & 2006 & Handbook of Constraint Programming & 39 & 30 & 25 & No & \ref{c:BaptisteLPN06}\\
BaptistePN01 \href{http://dx.doi.org/10.1007/978-1-4615-1479-4}{BaptistePN01} & \hyperref[auth:a163]{P. Baptiste}, \hyperref[auth:a164]{Claude Le Pape}, \hyperref[auth:a666]{W. Nuijten} & Constraint-Based Scheduling & No & \cite{BaptistePN01} & 2001 & Book & null & 296 & 0 & No & \ref{c:BaptistePN01}\\
BaptisteP00 \href{https://doi.org/10.1023/A:1009822502231}{BaptisteP00} & \hyperref[auth:a163]{P. Baptiste}, \hyperref[auth:a164]{Claude Le Pape} & Constraint Propagation and Decomposition Techniques for Highly Disjunctive and Highly Cumulative Project Scheduling Problems & \href{works/BaptisteP00.pdf}{Yes} & \cite{BaptisteP00} & 2000 & Constraints An Int. J. & 21 & 46 & 0 & \ref{b:BaptisteP00} & \ref{c:BaptisteP00}\\
NuijtenP98 \href{https://doi.org/10.1023/A:1009687210594}{NuijtenP98} & \hyperref[auth:a666]{W. Nuijten}, \hyperref[auth:a164]{Claude Le Pape} & Constraint-Based Job Shop Scheduling with {\textbackslash}sc Ilog Scheduler & \href{works/NuijtenP98.pdf}{Yes} & \cite{NuijtenP98} & 1998 & J. Heuristics & 16 & 42 & 0 & \ref{b:NuijtenP98} & \ref{c:NuijtenP98}\\
PapaB98 \href{https://doi.org/10.1023/A:1009723704757}{PapaB98} & \hyperref[auth:a164]{Claude Le Pape}, \hyperref[auth:a163]{P. Baptiste} & Resource Constraints for Preemptive Job-shop Scheduling & \href{works/PapaB98.pdf}{Yes} & \cite{PapaB98} & 1998 & Constraints An Int. J. & 25 & 14 & 0 & \ref{b:PapaB98} & \ref{c:PapaB98}\\
BaptisteP97 \href{https://doi.org/10.1007/BFb0017454}{BaptisteP97} & \hyperref[auth:a163]{P. Baptiste}, \hyperref[auth:a164]{Claude Le Pape} & Constraint Propagation and Decomposition Techniques for Highly Disjunctive and Highly Cumulative Project Scheduling Problems & \href{works/BaptisteP97.pdf}{Yes} & \cite{BaptisteP97} & 1997 & CP 1997 & 15 & 8 & 10 & \ref{b:BaptisteP97} & \ref{c:BaptisteP97}\\
PapeB97 \href{}{PapeB97} & \hyperref[auth:a164]{Claude Le Pape}, \hyperref[auth:a163]{P. Baptiste} & A Constraint Programming Library for Preemptive and Non-Preemptive Scheduling & No & \cite{PapeB97} & 1997 & PACT 1997 & 20 & 0 & 0 & No & \ref{c:PapeB97}\\
Pape94 \href{http://dx.doi.org/10.1049/ise.1994.0009}{Pape94} & \hyperref[auth:a164]{Claude Le Pape} & Implementation of resource constraints in ILOG SCHEDULE: a library for the development of constraint-based scheduling systems & No & \cite{Pape94} & 1994 & Intelligent Systems Engineering & 1 & 98 & 0 & No & \ref{c:Pape94}\\
\end{longtable}
}

\subsection{Works by Mark Wallace}
\label{sec:a117}
{\scriptsize
\begin{longtable}{>{\raggedright\arraybackslash}p{3cm}>{\raggedright\arraybackslash}p{6cm}>{\raggedright\arraybackslash}p{6.5cm}rrrp{2.5cm}rrrrr}
\rowcolor{white}\caption{Works from bibtex (Total 8)}\\ \toprule
\rowcolor{white}Key & Authors & Title & LC & Cite & Year & \shortstack{Conference\\/Journal} & Pages & \shortstack{Nr\\Cites} & \shortstack{Nr\\Refs} & b & c \\ \midrule\endhead
\bottomrule
\endfoot
WallaceY20 \href{https://doi.org/10.1007/s10601-020-09316-z}{WallaceY20} & \hyperref[auth:a117]{M. Wallace}, \hyperref[auth:a19]{N. Yorke{-}Smith} & A new constraint programming model and solving for the cyclic hoist scheduling problem & \href{works/WallaceY20.pdf}{Yes} & \cite{WallaceY20} & 2020 & Constraints An Int. J. & 19 & 5 & 18 & \ref{b:WallaceY20} & \ref{c:WallaceY20}\\
He0GLW18 \href{https://doi.org/10.1007/978-3-319-98334-9\_42}{He0GLW18} & \hyperref[auth:a185]{S. He}, \hyperref[auth:a117]{M. Wallace}, \hyperref[auth:a186]{G. Gange}, \hyperref[auth:a187]{A. Liebman}, \hyperref[auth:a188]{C. Wilson} & A Fast and Scalable Algorithm for Scheduling Large Numbers of Devices Under Real-Time Pricing & \href{works/He0GLW18.pdf}{Yes} & \cite{He0GLW18} & 2018 & CP 2018 & 18 & 6 & 26 & \ref{b:He0GLW18} & \ref{c:He0GLW18}\\
ThiruvadyWGS14 \href{https://doi.org/10.1007/s10732-014-9260-3}{ThiruvadyWGS14} & \hyperref[auth:a400]{Dhananjay R. Thiruvady}, \hyperref[auth:a117]{M. Wallace}, \hyperref[auth:a341]{H. Gu}, \hyperref[auth:a124]{A. Schutt} & A Lagrangian relaxation and {ACO} hybrid for resource constrained project scheduling with discounted cash flows & \href{works/ThiruvadyWGS14.pdf}{Yes} & \cite{ThiruvadyWGS14} & 2014 & J. Heuristics & 34 & 19 & 18 & \ref{b:ThiruvadyWGS14} & \ref{c:ThiruvadyWGS14}\\
SchuttFSW09 \href{https://doi.org/10.1007/978-3-642-04244-7\_58}{SchuttFSW09} & \hyperref[auth:a124]{A. Schutt}, \hyperref[auth:a154]{T. Feydy}, \hyperref[auth:a125]{Peter J. Stuckey}, \hyperref[auth:a117]{M. Wallace} & Why Cumulative Decomposition Is Not as Bad as It Sounds & \href{works/SchuttFSW09.pdf}{Yes} & \cite{SchuttFSW09} & 2009 & CP 2009 & 16 & 34 & 11 & \ref{b:SchuttFSW09} & \ref{c:SchuttFSW09}\\
SakkoutW00 \href{https://doi.org/10.1023/A:1009856210543}{SakkoutW00} & \hyperref[auth:a167]{Hani El Sakkout}, \hyperref[auth:a117]{M. Wallace} & Probe Backtrack Search for Minimal Perturbation in Dynamic Scheduling & \href{works/SakkoutW00.pdf}{Yes} & \cite{SakkoutW00} & 2000 & Constraints An Int. J. & 30 & 73 & 0 & \ref{b:SakkoutW00} & \ref{c:SakkoutW00}\\
RodosekW98 \href{https://doi.org/10.1007/3-540-49481-2\_28}{RodosekW98} & \hyperref[auth:a299]{R. Rodosek}, \hyperref[auth:a117]{M. Wallace} & A Generic Model and Hybrid Algorithm for Hoist Scheduling Problems & \href{works/RodosekW98.pdf}{Yes} & \cite{RodosekW98} & 1998 & CP 1998 & 15 & 19 & 10 & \ref{b:RodosekW98} & \ref{c:RodosekW98}\\
Wallace96 \href{https://doi.org/10.1007/BF00143881}{Wallace96} & \hyperref[auth:a117]{M. Wallace} & Practical Applications of Constraint Programming & \href{works/Wallace96.pdf}{Yes} & \cite{Wallace96} & 1996 & Constraints An Int. J. & 30 & 87 & 55 & \ref{b:Wallace96} & \ref{c:Wallace96}\\
Wallace94 \href{}{Wallace94} & \hyperref[auth:a117]{M. Wallace} & Applying Constraints for Scheduling & No & \cite{Wallace94} & 1994 & Constraint Programming 1994 & 19 & 0 & 0 & No & \ref{c:Wallace94}\\
\end{longtable}
}

\subsection{Works by Thibaut Feydy}
\label{sec:a154}
{\scriptsize
\begin{longtable}{>{\raggedright\arraybackslash}p{3cm}>{\raggedright\arraybackslash}p{6cm}>{\raggedright\arraybackslash}p{6.5cm}rrrp{2.5cm}rrrrr}
\rowcolor{white}\caption{Works from bibtex (Total 7)}\\ \toprule
\rowcolor{white}Key & Authors & Title & LC & Cite & Year & \shortstack{Conference\\/Journal} & Pages & \shortstack{Nr\\Cites} & \shortstack{Nr\\Refs} & b & c \\ \midrule\endhead
\bottomrule
\endfoot
YoungFS17 \href{https://doi.org/10.1007/978-3-319-66158-2\_20}{YoungFS17} & \hyperref[auth:a193]{Kenneth D. Young}, \hyperref[auth:a154]{T. Feydy}, \hyperref[auth:a124]{A. Schutt} & Constraint Programming Applied to the Multi-Skill Project Scheduling Problem & \href{works/YoungFS17.pdf}{Yes} & \cite{YoungFS17} & 2017 & CP 2017 & 10 & 6 & 21 & \ref{b:YoungFS17} & \ref{c:YoungFS17}\\
SchuttFS13 \href{https://doi.org/10.1007/978-3-642-40627-0\_47}{SchuttFS13} & \hyperref[auth:a124]{A. Schutt}, \hyperref[auth:a154]{T. Feydy}, \hyperref[auth:a125]{Peter J. Stuckey} & Scheduling Optional Tasks with Explanation & \href{works/SchuttFS13.pdf}{Yes} & \cite{SchuttFS13} & 2013 & CP 2013 & 17 & 10 & 20 & \ref{b:SchuttFS13} & \ref{c:SchuttFS13}\\
SchuttFS13a \href{https://doi.org/10.1007/978-3-642-38171-3\_16}{SchuttFS13a} & \hyperref[auth:a124]{A. Schutt}, \hyperref[auth:a154]{T. Feydy}, \hyperref[auth:a125]{Peter J. Stuckey} & Explaining Time-Table-Edge-Finding Propagation for the Cumulative Resource Constraint & \href{works/SchuttFS13a.pdf}{Yes} & \cite{SchuttFS13a} & 2013 & CPAIOR 2013 & 17 & 20 & 27 & \ref{b:SchuttFS13a} & \ref{c:SchuttFS13a}\\
SchuttFSW13 \href{https://doi.org/10.1007/s10951-012-0285-x}{SchuttFSW13} & \hyperref[auth:a124]{A. Schutt}, \hyperref[auth:a154]{T. Feydy}, \hyperref[auth:a125]{Peter J. Stuckey}, \hyperref[auth:a155]{Mark G. Wallace} & Solving RCPSP/max by lazy clause generation & \href{works/SchuttFSW13.pdf}{Yes} & \cite{SchuttFSW13} & 2013 & J. Sched. & 17 & 43 & 23 & \ref{b:SchuttFSW13} & \ref{c:SchuttFSW13}\\
SchuttFSW11 \href{https://doi.org/10.1007/s10601-010-9103-2}{SchuttFSW11} & \hyperref[auth:a124]{A. Schutt}, \hyperref[auth:a154]{T. Feydy}, \hyperref[auth:a125]{Peter J. Stuckey}, \hyperref[auth:a155]{Mark G. Wallace} & Explaining the cumulative propagator & \href{works/SchuttFSW11.pdf}{Yes} & \cite{SchuttFSW11} & 2011 & Constraints An Int. J. & 33 & 57 & 23 & \ref{b:SchuttFSW11} & \ref{c:SchuttFSW11}\\
abs-1009-0347 \href{http://arxiv.org/abs/1009.0347}{abs-1009-0347} & \hyperref[auth:a124]{A. Schutt}, \hyperref[auth:a154]{T. Feydy}, \hyperref[auth:a125]{Peter J. Stuckey}, \hyperref[auth:a155]{Mark G. Wallace} & Solving the Resource Constrained Project Scheduling Problem with Generalized Precedences by Lazy Clause Generation & \href{works/abs-1009-0347.pdf}{Yes} & \cite{abs-1009-0347} & 2010 & CoRR & 37 & 0 & 0 & \ref{b:abs-1009-0347} & \ref{c:abs-1009-0347}\\
SchuttFSW09 \href{https://doi.org/10.1007/978-3-642-04244-7\_58}{SchuttFSW09} & \hyperref[auth:a124]{A. Schutt}, \hyperref[auth:a154]{T. Feydy}, \hyperref[auth:a125]{Peter J. Stuckey}, \hyperref[auth:a117]{M. Wallace} & Why Cumulative Decomposition Is Not as Bad as It Sounds & \href{works/SchuttFSW09.pdf}{Yes} & \cite{SchuttFSW09} & 2009 & CP 2009 & 16 & 34 & 11 & \ref{b:SchuttFSW09} & \ref{c:SchuttFSW09}\\
\end{longtable}
}

\subsection{Works by Diarmuid Grimes}
\label{sec:a182}
{\scriptsize
\begin{longtable}{>{\raggedright\arraybackslash}p{3cm}>{\raggedright\arraybackslash}p{6cm}>{\raggedright\arraybackslash}p{6.5cm}rrrp{2.5cm}rrrrr}
\rowcolor{white}\caption{Works from bibtex (Total 7)}\\ \toprule
\rowcolor{white}Key & Authors & Title & LC & Cite & Year & \shortstack{Conference\\/Journal} & Pages & \shortstack{Nr\\Cites} & \shortstack{Nr\\Refs} & b & c \\ \midrule\endhead
\bottomrule
\endfoot
AntunesABDEGGOL20 \href{https://doi.org/10.1142/S0218213020600076}{AntunesABDEGGOL20} & \hyperref[auth:a893]{M. Antunes}, \hyperref[auth:a894]{V. Armant}, \hyperref[auth:a222]{Kenneth N. Brown}, \hyperref[auth:a895]{Daniel A. Desmond}, \hyperref[auth:a896]{G. Escamocher}, \hyperref[auth:a897]{A. George}, \hyperref[auth:a182]{D. Grimes}, \hyperref[auth:a898]{M. O'Keeffe}, \hyperref[auth:a899]{Y. Lin}, \hyperref[auth:a16]{B. O'Sullivan}, \hyperref[auth:a900]{C. Ozturk}, \hyperref[auth:a901]{L. Quesada}, \hyperref[auth:a129]{M. Siala}, \hyperref[auth:a17]{H. Simonis}, \hyperref[auth:a837]{N. Wilson} & Assigning and Scheduling Service Visits in a Mixed Urban/Rural Setting & No & \cite{AntunesABDEGGOL20} & 2020 & Int. J. Artif. Intell. Tools & 31 & 0 & 16 & No & \ref{c:AntunesABDEGGOL20}\\
AntunesABDEGGOL18 \href{https://doi.org/10.1109/ICTAI.2018.00027}{AntunesABDEGGOL18} & \hyperref[auth:a893]{M. Antunes}, \hyperref[auth:a894]{V. Armant}, \hyperref[auth:a222]{Kenneth N. Brown}, \hyperref[auth:a895]{Daniel A. Desmond}, \hyperref[auth:a896]{G. Escamocher}, \hyperref[auth:a897]{A. George}, \hyperref[auth:a182]{D. Grimes}, \hyperref[auth:a898]{M. O'Keeffe}, \hyperref[auth:a899]{Y. Lin}, \hyperref[auth:a16]{B. O'Sullivan}, \hyperref[auth:a900]{C. Ozturk}, \hyperref[auth:a901]{L. Quesada}, \hyperref[auth:a129]{M. Siala}, \hyperref[auth:a17]{H. Simonis}, \hyperref[auth:a837]{N. Wilson} & Assigning and Scheduling Service Visits in a Mixed Urban/Rural Setting & No & \cite{AntunesABDEGGOL18} & 2018 & ICTAI 2018 & 8 & 1 & 24 & No & \ref{c:AntunesABDEGGOL18}\\
GrimesH15 \href{https://doi.org/10.1287/ijoc.2014.0625}{GrimesH15} & \hyperref[auth:a182]{D. Grimes}, \hyperref[auth:a1]{E. Hebrard} & Solving Variants of the Job Shop Scheduling Problem Through Conflict-Directed Search & No & \cite{GrimesH15} & 2015 & {INFORMS} J. Comput. & 17 & 12 & 41 & No & \ref{c:GrimesH15}\\
GrimesIOS14 \href{https://doi.org/10.1016/j.suscom.2014.08.009}{GrimesIOS14} & \hyperref[auth:a182]{D. Grimes}, \hyperref[auth:a183]{G. Ifrim}, \hyperref[auth:a16]{B. O'Sullivan}, \hyperref[auth:a17]{H. Simonis} & Analyzing the impact of electricity price forecasting on energy cost-aware scheduling & \href{works/GrimesIOS14.pdf}{Yes} & \cite{GrimesIOS14} & 2014 & Sustain. Comput. Informatics Syst. & 16 & 6 & 7 & \ref{b:GrimesIOS14} & \ref{c:GrimesIOS14}\\
GrimesH11 \href{https://doi.org/10.1007/978-3-642-23786-7\_28}{GrimesH11} & \hyperref[auth:a182]{D. Grimes}, \hyperref[auth:a1]{E. Hebrard} & Models and Strategies for Variants of the Job Shop Scheduling Problem & \href{works/GrimesH11.pdf}{Yes} & \cite{GrimesH11} & 2011 & CP 2011 & 17 & 5 & 18 & \ref{b:GrimesH11} & \ref{c:GrimesH11}\\
GrimesH10 \href{https://doi.org/10.1007/978-3-642-13520-0\_19}{GrimesH10} & \hyperref[auth:a182]{D. Grimes}, \hyperref[auth:a1]{E. Hebrard} & Job Shop Scheduling with Setup Times and Maximal Time-Lags: {A} Simple Constraint Programming Approach & \href{works/GrimesH10.pdf}{Yes} & \cite{GrimesH10} & 2010 & CPAIOR 2010 & 15 & 13 & 20 & \ref{b:GrimesH10} & \ref{c:GrimesH10}\\
GrimesHM09 \href{https://doi.org/10.1007/978-3-642-04244-7\_33}{GrimesHM09} & \hyperref[auth:a182]{D. Grimes}, \hyperref[auth:a1]{E. Hebrard}, \hyperref[auth:a82]{A. Malapert} & Closing the Open Shop: Contradicting Conventional Wisdom & \href{works/GrimesHM09.pdf}{Yes} & \cite{GrimesHM09} & 2009 & CP 2009 & 9 & 15 & 12 & \ref{b:GrimesHM09} & \ref{c:GrimesHM09}\\
\end{longtable}
}

\subsection{Works by Zdenek Hanz{\'{a}}lek}
\label{sec:a116}
{\scriptsize
\begin{longtable}{>{\raggedright\arraybackslash}p{3cm}>{\raggedright\arraybackslash}p{6cm}>{\raggedright\arraybackslash}p{6.5cm}rrrp{2.5cm}rrrrr}
\rowcolor{white}\caption{Works from bibtex (Total 7)}\\ \toprule
\rowcolor{white}Key & Authors & Title & LC & Cite & Year & \shortstack{Conference\\/Journal} & Pages & \shortstack{Nr\\Cites} & \shortstack{Nr\\Refs} & b & c \\ \midrule\endhead
\bottomrule
\endfoot
Mehdizadeh-Somarin23 \href{https://doi.org/10.1007/978-3-031-43670-3\_33}{Mehdizadeh-Somarin23} & \hyperref[auth:a433]{Z. Mehdizadeh{-}Somarin}, \hyperref[auth:a434]{R. Tavakkoli{-}Moghaddam}, \hyperref[auth:a435]{M. Rohaninejad}, \hyperref[auth:a116]{Z. Hanz{\'{a}}lek}, \hyperref[auth:a436]{Behdin Vahedi Nouri} & A Constraint Programming Model for a Reconfigurable Job Shop Scheduling Problem with Machine Availability & \href{works/Mehdizadeh-Somarin23.pdf}{Yes} & \cite{Mehdizadeh-Somarin23} & 2023 & APMS 2023 & 14 & 0 & 0 & \ref{b:Mehdizadeh-Somarin23} & \ref{c:Mehdizadeh-Somarin23}\\
abs-2305-19888 \href{https://doi.org/10.48550/arXiv.2305.19888}{abs-2305-19888} & \hyperref[auth:a437]{V. Heinz}, \hyperref[auth:a438]{A. Nov{\'{a}}k}, \hyperref[auth:a313]{M. Vlk}, \hyperref[auth:a116]{Z. Hanz{\'{a}}lek} & Constraint Programming and Constructive Heuristics for Parallel Machine Scheduling with Sequence-Dependent Setups and Common Servers & \href{works/abs-2305-19888.pdf}{Yes} & \cite{abs-2305-19888} & 2023 & CoRR & 42 & 0 & 0 & \ref{b:abs-2305-19888} & \ref{c:abs-2305-19888}\\
HeinzNVH22 \href{https://doi.org/10.1016/j.cie.2022.108586}{HeinzNVH22} & \hyperref[auth:a437]{V. Heinz}, \hyperref[auth:a438]{A. Nov{\'{a}}k}, \hyperref[auth:a313]{M. Vlk}, \hyperref[auth:a116]{Z. Hanz{\'{a}}lek} & Constraint Programming and constructive heuristics for parallel machine scheduling with sequence-dependent setups and common servers & \href{works/HeinzNVH22.pdf}{Yes} & \cite{HeinzNVH22} & 2022 & Comput. Ind. Eng. & 16 & 5 & 25 & \ref{b:HeinzNVH22} & \ref{c:HeinzNVH22}\\
VlkHT21 \href{https://doi.org/10.1016/j.cie.2021.107317}{VlkHT21} & \hyperref[auth:a313]{M. Vlk}, \hyperref[auth:a116]{Z. Hanz{\'{a}}lek}, \hyperref[auth:a480]{S. Tang} & Constraint programming approaches to joint routing and scheduling in time-sensitive networks & \href{works/VlkHT21.pdf}{Yes} & \cite{VlkHT21} & 2021 & Comput. Ind. Eng. & 14 & 7 & 22 & \ref{b:VlkHT21} & \ref{c:VlkHT21}\\
BenediktMH20 \href{https://doi.org/10.1007/s10601-020-09317-y}{BenediktMH20} & \hyperref[auth:a114]{O. Benedikt}, \hyperref[auth:a115]{I. M{\'{o}}dos}, \hyperref[auth:a116]{Z. Hanz{\'{a}}lek} & Power of pre-processing: production scheduling with variable energy pricing and power-saving states & \href{works/BenediktMH20.pdf}{Yes} & \cite{BenediktMH20} & 2020 & Constraints An Int. J. & 19 & 1 & 18 & \ref{b:BenediktMH20} & \ref{c:BenediktMH20}\\
BenediktSMVH18 \href{https://doi.org/10.1007/978-3-319-93031-2\_6}{BenediktSMVH18} & \hyperref[auth:a114]{O. Benedikt}, \hyperref[auth:a312]{P. Sucha}, \hyperref[auth:a115]{I. M{\'{o}}dos}, \hyperref[auth:a313]{M. Vlk}, \hyperref[auth:a116]{Z. Hanz{\'{a}}lek} & Energy-Aware Production Scheduling with Power-Saving Modes & \href{works/BenediktSMVH18.pdf}{Yes} & \cite{BenediktSMVH18} & 2018 & CPAIOR 2018 & 10 & 2 & 12 & \ref{b:BenediktSMVH18} & \ref{c:BenediktSMVH18}\\
KelbelH11 \href{https://doi.org/10.1007/s10845-009-0318-2}{KelbelH11} & \hyperref[auth:a627]{J. Kelbel}, \hyperref[auth:a116]{Z. Hanz{\'{a}}lek} & Solving production scheduling with earliness/tardiness penalties by constraint programming & \href{works/KelbelH11.pdf}{Yes} & \cite{KelbelH11} & 2011 & J. Intell. Manuf. & 10 & 12 & 14 & \ref{b:KelbelH11} & \ref{c:KelbelH11}\\
\end{longtable}
}

\subsection{Works by Andr{\'{a}}s Kov{\'{a}}cs}
\label{sec:a146}
{\scriptsize
\begin{longtable}{>{\raggedright\arraybackslash}p{3cm}>{\raggedright\arraybackslash}p{6cm}>{\raggedright\arraybackslash}p{6.5cm}rrrp{2.5cm}rrrrr}
\rowcolor{white}\caption{Works from bibtex (Total 7)}\\ \toprule
\rowcolor{white}Key & Authors & Title & LC & Cite & Year & \shortstack{Conference\\/Journal} & Pages & \shortstack{Nr\\Cites} & \shortstack{Nr\\Refs} & b & c \\ \midrule\endhead
\bottomrule
\endfoot
KovacsB11 \href{https://doi.org/10.1007/s10601-009-9088-x}{KovacsB11} & \hyperref[auth:a146]{A. Kov{\'{a}}cs}, \hyperref[auth:a89]{J. Christopher Beck} & A global constraint for total weighted completion time for unary resources & \href{works/KovacsB11.pdf}{Yes} & \cite{KovacsB11} & 2011 & Constraints An Int. J. & 24 & 4 & 26 & \ref{b:KovacsB11} & \ref{c:KovacsB11}\\
KovacsK11 \href{https://doi.org/10.1007/s10601-010-9102-3}{KovacsK11} & \hyperref[auth:a146]{A. Kov{\'{a}}cs}, \hyperref[auth:a156]{T. Kis} & Constraint programming approach to a bilevel scheduling problem & \href{works/KovacsK11.pdf}{Yes} & \cite{KovacsK11} & 2011 & Constraints An Int. J. & 24 & 3 & 24 & \ref{b:KovacsK11} & \ref{c:KovacsK11}\\
KovacsB08 \href{https://doi.org/10.1016/j.engappai.2008.03.004}{KovacsB08} & \hyperref[auth:a146]{A. Kov{\'{a}}cs}, \hyperref[auth:a89]{J. Christopher Beck} & A global constraint for total weighted completion time for cumulative resources & \href{works/KovacsB08.pdf}{Yes} & \cite{KovacsB08} & 2008 & Eng. Appl. Artif. Intell. & 7 & 5 & 14 & \ref{b:KovacsB08} & \ref{c:KovacsB08}\\
KovacsB07 \href{https://doi.org/10.1007/978-3-540-72397-4\_9}{KovacsB07} & \hyperref[auth:a146]{A. Kov{\'{a}}cs}, \hyperref[auth:a89]{J. Christopher Beck} & A Global Constraint for Total Weighted Completion Time & \href{works/KovacsB07.pdf}{Yes} & \cite{KovacsB07} & 2007 & CPAIOR 2007 & 15 & 2 & 12 & \ref{b:KovacsB07} & \ref{c:KovacsB07}\\
KovacsV06 \href{https://doi.org/10.1007/11757375\_13}{KovacsV06} & \hyperref[auth:a146]{A. Kov{\'{a}}cs}, \hyperref[auth:a280]{J. V{\'{a}}ncza} & Progressive Solutions: {A} Simple but Efficient Dominance Rule for Practical {RCPSP} & \href{works/KovacsV06.pdf}{Yes} & \cite{KovacsV06} & 2006 & CPAIOR 2006 & 13 & 2 & 7 & \ref{b:KovacsV06} & \ref{c:KovacsV06}\\
KovacsEKV05 \href{https://doi.org/10.1007/11564751\_118}{KovacsEKV05} & \hyperref[auth:a146]{A. Kov{\'{a}}cs}, \hyperref[auth:a279]{P. Egri}, \hyperref[auth:a156]{T. Kis}, \hyperref[auth:a280]{J. V{\'{a}}ncza} & Proterv-II: An Integrated Production Planning and Scheduling System & \href{works/KovacsEKV05.pdf}{Yes} & \cite{KovacsEKV05} & 2005 & CP 2005 & 1 & 2 & 3 & \ref{b:KovacsEKV05} & \ref{c:KovacsEKV05}\\
KovacsV04 \href{https://doi.org/10.1007/978-3-540-30201-8\_26}{KovacsV04} & \hyperref[auth:a146]{A. Kov{\'{a}}cs}, \hyperref[auth:a280]{J. V{\'{a}}ncza} & Completable Partial Solutions in Constraint Programming and Constraint-Based Scheduling & \href{works/KovacsV04.pdf}{Yes} & \cite{KovacsV04} & 2004 & CP 2004 & 15 & 3 & 12 & \ref{b:KovacsV04} & \ref{c:KovacsV04}\\
\end{longtable}
}

\subsection{Works by Barry O'Sullivan}
\label{sec:a16}
{\scriptsize
\begin{longtable}{>{\raggedright\arraybackslash}p{3cm}>{\raggedright\arraybackslash}p{6cm}>{\raggedright\arraybackslash}p{6.5cm}rrrp{2.5cm}rrrrr}
\rowcolor{white}\caption{Works from bibtex (Total 7)}\\ \toprule
\rowcolor{white}Key & Authors & Title & LC & Cite & Year & \shortstack{Conference\\/Journal} & Pages & \shortstack{Nr\\Cites} & \shortstack{Nr\\Refs} & b & c \\ \midrule\endhead
\bottomrule
\endfoot
ArmstrongGOS22 \href{https://doi.org/10.1007/978-3-031-08011-1\_1}{ArmstrongGOS22} & \hyperref[auth:a14]{E. Armstrong}, \hyperref[auth:a15]{M. Garraffa}, \hyperref[auth:a16]{B. O'Sullivan}, \hyperref[auth:a17]{H. Simonis} & A Two-Phase Hybrid Approach for the Hybrid Flexible Flowshop with Transportation Times & \href{works/ArmstrongGOS22.pdf}{Yes} & \cite{ArmstrongGOS22} & 2022 & CPAIOR 2022 & 13 & 0 & 14 & \ref{b:ArmstrongGOS22} & \ref{c:ArmstrongGOS22}\\
ArmstrongGOS21 \href{https://doi.org/10.4230/LIPIcs.CP.2021.16}{ArmstrongGOS21} & \hyperref[auth:a14]{E. Armstrong}, \hyperref[auth:a15]{M. Garraffa}, \hyperref[auth:a16]{B. O'Sullivan}, \hyperref[auth:a17]{H. Simonis} & The Hybrid Flexible Flowshop with Transportation Times & \href{works/ArmstrongGOS21.pdf}{Yes} & \cite{ArmstrongGOS21} & 2021 & CP 2021 & 18 & 1 & 0 & \ref{b:ArmstrongGOS21} & \ref{c:ArmstrongGOS21}\\
AntunesABDEGGOL20 \href{https://doi.org/10.1142/S0218213020600076}{AntunesABDEGGOL20} & \hyperref[auth:a893]{M. Antunes}, \hyperref[auth:a894]{V. Armant}, \hyperref[auth:a222]{Kenneth N. Brown}, \hyperref[auth:a895]{Daniel A. Desmond}, \hyperref[auth:a896]{G. Escamocher}, \hyperref[auth:a897]{A. George}, \hyperref[auth:a182]{D. Grimes}, \hyperref[auth:a898]{M. O'Keeffe}, \hyperref[auth:a899]{Y. Lin}, \hyperref[auth:a16]{B. O'Sullivan}, \hyperref[auth:a900]{C. Ozturk}, \hyperref[auth:a901]{L. Quesada}, \hyperref[auth:a129]{M. Siala}, \hyperref[auth:a17]{H. Simonis}, \hyperref[auth:a837]{N. Wilson} & Assigning and Scheduling Service Visits in a Mixed Urban/Rural Setting & No & \cite{AntunesABDEGGOL20} & 2020 & Int. J. Artif. Intell. Tools & 31 & 0 & 16 & No & \ref{c:AntunesABDEGGOL20}\\
AntunesABDEGGOL18 \href{https://doi.org/10.1109/ICTAI.2018.00027}{AntunesABDEGGOL18} & \hyperref[auth:a893]{M. Antunes}, \hyperref[auth:a894]{V. Armant}, \hyperref[auth:a222]{Kenneth N. Brown}, \hyperref[auth:a895]{Daniel A. Desmond}, \hyperref[auth:a896]{G. Escamocher}, \hyperref[auth:a897]{A. George}, \hyperref[auth:a182]{D. Grimes}, \hyperref[auth:a898]{M. O'Keeffe}, \hyperref[auth:a899]{Y. Lin}, \hyperref[auth:a16]{B. O'Sullivan}, \hyperref[auth:a900]{C. Ozturk}, \hyperref[auth:a901]{L. Quesada}, \hyperref[auth:a129]{M. Siala}, \hyperref[auth:a17]{H. Simonis}, \hyperref[auth:a837]{N. Wilson} & Assigning and Scheduling Service Visits in a Mixed Urban/Rural Setting & No & \cite{AntunesABDEGGOL18} & 2018 & ICTAI 2018 & 8 & 1 & 24 & No & \ref{c:AntunesABDEGGOL18}\\
HurleyOS16 \href{https://doi.org/10.1007/978-3-319-50137-6\_15}{HurleyOS16} & \hyperref[auth:a902]{B. Hurley}, \hyperref[auth:a16]{B. O'Sullivan}, \hyperref[auth:a17]{H. Simonis} & {ICON} Loop Energy Show Case & \href{works/HurleyOS16.pdf}{Yes} & \cite{HurleyOS16} & 2016 & Data Mining and Constraint Programming - Foundations of a Cross-Disciplinary Approach & 14 & 0 & 16 & \ref{b:HurleyOS16} & \ref{c:HurleyOS16}\\
GrimesIOS14 \href{https://doi.org/10.1016/j.suscom.2014.08.009}{GrimesIOS14} & \hyperref[auth:a182]{D. Grimes}, \hyperref[auth:a183]{G. Ifrim}, \hyperref[auth:a16]{B. O'Sullivan}, \hyperref[auth:a17]{H. Simonis} & Analyzing the impact of electricity price forecasting on energy cost-aware scheduling & \href{works/GrimesIOS14.pdf}{Yes} & \cite{GrimesIOS14} & 2014 & Sustain. Comput. Informatics Syst. & 16 & 6 & 7 & \ref{b:GrimesIOS14} & \ref{c:GrimesIOS14}\\
IfrimOS12 \href{https://doi.org/10.1007/978-3-642-33558-7\_68}{IfrimOS12} & \hyperref[auth:a183]{G. Ifrim}, \hyperref[auth:a16]{B. O'Sullivan}, \hyperref[auth:a17]{H. Simonis} & Properties of Energy-Price Forecasts for Scheduling & \href{works/IfrimOS12.pdf}{Yes} & \cite{IfrimOS12} & 2012 & CP 2012 & 16 & 6 & 20 & \ref{b:IfrimOS12} & \ref{c:IfrimOS12}\\
\end{longtable}
}

\subsection{Works by Gabriela P. Henning}
\label{sec:a596}
{\scriptsize
\begin{longtable}{>{\raggedright\arraybackslash}p{3cm}>{\raggedright\arraybackslash}p{6cm}>{\raggedright\arraybackslash}p{6.5cm}rrrp{2.5cm}rrrrr}
\rowcolor{white}\caption{Works from bibtex (Total 7)}\\ \toprule
\rowcolor{white}Key & Authors & Title & LC & Cite & Year & \shortstack{Conference\\/Journal} & Pages & \shortstack{Nr\\Cites} & \shortstack{Nr\\Refs} & b & c \\ \midrule\endhead
\bottomrule
\endfoot
NovaraNH16 \href{https://doi.org/10.1016/j.compchemeng.2016.04.030}{NovaraNH16} & \hyperref[auth:a595]{Franco M. Novara}, \hyperref[auth:a529]{Juan M. Novas}, \hyperref[auth:a596]{Gabriela P. Henning} & A novel constraint programming model for large-scale scheduling problems in multiproduct multistage batch plants: Limited resources and campaign-based operation & \href{works/NovaraNH16.pdf}{Yes} & \cite{NovaraNH16} & 2016 & Comput. Chem. Eng. & 17 & 18 & 31 & \ref{b:NovaraNH16} & \ref{c:NovaraNH16}\\
NovasH14 \href{https://doi.org/10.1016/j.eswa.2013.09.026}{NovasH14} & \hyperref[auth:a529]{Juan M. Novas}, \hyperref[auth:a596]{Gabriela P. Henning} & Integrated scheduling of resource-constrained flexible manufacturing systems using constraint programming & \href{works/NovasH14.pdf}{Yes} & \cite{NovasH14} & 2014 & Expert Syst. Appl. & 14 & 35 & 26 & \ref{b:NovasH14} & \ref{c:NovasH14}\\
NovasH12 \href{https://doi.org/10.1016/j.compchemeng.2012.01.005}{NovasH12} & \hyperref[auth:a529]{Juan M. Novas}, \hyperref[auth:a596]{Gabriela P. Henning} & A comprehensive constraint programming approach for the rolling horizon-based scheduling of automated wet-etch stations & \href{works/NovasH12.pdf}{Yes} & \cite{NovasH12} & 2012 & Comput. Chem. Eng. & 17 & 17 & 15 & \ref{b:NovasH12} & \ref{c:NovasH12}\\
NovasH10 \href{https://doi.org/10.1016/j.compchemeng.2010.07.011}{NovasH10} & \hyperref[auth:a529]{Juan M. Novas}, \hyperref[auth:a596]{Gabriela P. Henning} & Reactive scheduling framework based on domain knowledge and constraint programming & \href{works/NovasH10.pdf}{Yes} & \cite{NovasH10} & 2010 & Comput. Chem. Eng. & 20 & 48 & 19 & \ref{b:NovasH10} & \ref{c:NovasH10}\\
ZeballosQH10 \href{https://doi.org/10.1016/j.engappai.2009.07.002}{ZeballosQH10} & \hyperref[auth:a630]{L. Zeballos}, \hyperref[auth:a631]{O. Quiroga}, \hyperref[auth:a596]{Gabriela P. Henning} & A constraint programming model for the scheduling of flexible manufacturing systems with machine and tool limitations & \href{works/ZeballosQH10.pdf}{Yes} & \cite{ZeballosQH10} & 2010 & Eng. Appl. Artif. Intell. & 20 & 33 & 28 & \ref{b:ZeballosQH10} & \ref{c:ZeballosQH10}\\
QuirogaZH05 \href{https://doi.org/10.1109/ROBOT.2005.1570686}{QuirogaZH05} & \hyperref[auth:a631]{O. Quiroga}, \hyperref[auth:a630]{L. Zeballos}, \hyperref[auth:a596]{Gabriela P. Henning} & A Constraint Programming Approach to Tool Allocation and Resource Scheduling in {FMS} & \href{works/QuirogaZH05.pdf}{Yes} & \cite{QuirogaZH05} & 2005 & ICRA 2005 & 6 & 2 & 7 & \ref{b:QuirogaZH05} & \ref{c:QuirogaZH05}\\
ZeballosH05 \href{http://journal.iberamia.org/index.php/ia/article/view/452/article\%20\%281\%29.pdf}{ZeballosH05} & \hyperref[auth:a630]{L. Zeballos}, \hyperref[auth:a596]{Gabriela P. Henning} & A Constraint Programming Approach to {FMS} Scheduling. Consideration of Storage and Transportation Resources & \href{works/ZeballosH05.pdf}{Yes} & \cite{ZeballosH05} & 2005 & Inteligencia Artif. & 10 & 0 & 0 & \ref{b:ZeballosH05} & \ref{c:ZeballosH05}\\
\end{longtable}
}

\subsection{Works by Stefan Heinz}
\label{sec:a133}
{\scriptsize
\begin{longtable}{>{\raggedright\arraybackslash}p{3cm}>{\raggedright\arraybackslash}p{6cm}>{\raggedright\arraybackslash}p{6.5cm}rrrp{2.5cm}rrrrr}
\rowcolor{white}\caption{Works from bibtex (Total 6)}\\ \toprule
\rowcolor{white}Key & Authors & Title & LC & Cite & Year & \shortstack{Conference\\/Journal} & Pages & \shortstack{Nr\\Cites} & \shortstack{Nr\\Refs} & b & c \\ \midrule\endhead
\bottomrule
\endfoot
HeinzKB13 \href{https://doi.org/10.1007/978-3-642-38171-3\_2}{HeinzKB13} & \hyperref[auth:a133]{S. Heinz}, \hyperref[auth:a336]{W. Ku}, \hyperref[auth:a89]{J. Christopher Beck} & Recent Improvements Using Constraint Integer Programming for Resource Allocation and Scheduling & \href{works/HeinzKB13.pdf}{Yes} & \cite{HeinzKB13} & 2013 & CPAIOR 2013 & 16 & 9 & 15 & \ref{b:HeinzKB13} & \ref{c:HeinzKB13}\\
HeinzSB13 \href{https://doi.org/10.1007/s10601-012-9136-9}{HeinzSB13} & \hyperref[auth:a133]{S. Heinz}, \hyperref[auth:a134]{J. Schulz}, \hyperref[auth:a89]{J. Christopher Beck} & Using dual presolving reductions to reformulate cumulative constraints & \href{works/HeinzSB13.pdf}{Yes} & \cite{HeinzSB13} & 2013 & Constraints An Int. J. & 36 & 7 & 31 & \ref{b:HeinzSB13} & \ref{c:HeinzSB13}\\
HeinzB12 \href{https://doi.org/10.1007/978-3-642-29828-8\_14}{HeinzB12} & \hyperref[auth:a133]{S. Heinz}, \hyperref[auth:a89]{J. Christopher Beck} & Reconsidering Mixed Integer Programming and MIP-Based Hybrids for Scheduling & \href{works/HeinzB12.pdf}{Yes} & \cite{HeinzB12} & 2012 & CPAIOR 2012 & 17 & 8 & 21 & \ref{b:HeinzB12} & \ref{c:HeinzB12}\\
HeinzSSW12 \href{https://doi.org/10.1007/s10601-011-9113-8}{HeinzSSW12} & \hyperref[auth:a133]{S. Heinz}, \hyperref[auth:a139]{T. Schlechte}, \hyperref[auth:a140]{R. Stephan}, \hyperref[auth:a141]{M. Winkler} & Solving steel mill slab design problems & \href{works/HeinzSSW12.pdf}{Yes} & \cite{HeinzSSW12} & 2012 & Constraints An Int. J. & 12 & 10 & 9 & \ref{b:HeinzSSW12} & \ref{c:HeinzSSW12}\\
HeinzS11 \href{https://doi.org/10.1007/978-3-642-20662-7\_34}{HeinzS11} & \hyperref[auth:a133]{S. Heinz}, \hyperref[auth:a134]{J. Schulz} & Explanations for the Cumulative Constraint: An Experimental Study & \href{works/HeinzS11.pdf}{Yes} & \cite{HeinzS11} & 2011 & SEA 2011 & 10 & 5 & 12 & \ref{b:HeinzS11} & \ref{c:HeinzS11}\\
BertholdHLMS10 \href{https://doi.org/10.1007/978-3-642-13520-0\_34}{BertholdHLMS10} & \hyperref[auth:a355]{T. Berthold}, \hyperref[auth:a133]{S. Heinz}, \hyperref[auth:a356]{Marco E. L{\"{u}}bbecke}, \hyperref[auth:a357]{Rolf H. M{\"{o}}hring}, \hyperref[auth:a134]{J. Schulz} & A Constraint Integer Programming Approach for Resource-Constrained Project Scheduling & \href{works/BertholdHLMS10.pdf}{Yes} & \cite{BertholdHLMS10} & 2010 & CPAIOR 2010 & 5 & 28 & 10 & \ref{b:BertholdHLMS10} & \ref{c:BertholdHLMS10}\\
\end{longtable}
}

\subsection{Works by Wim Nuijten}
\label{sec:a666}
{\scriptsize
\begin{longtable}{>{\raggedright\arraybackslash}p{3cm}>{\raggedright\arraybackslash}p{6cm}>{\raggedright\arraybackslash}p{6.5cm}rrrp{2.5cm}rrrrr}
\rowcolor{white}\caption{Works from bibtex (Total 6)}\\ \toprule
\rowcolor{white}Key & Authors & Title & LC & Cite & Year & \shortstack{Conference\\/Journal} & Pages & \shortstack{Nr\\Cites} & \shortstack{Nr\\Refs} & b & c \\ \midrule\endhead
\bottomrule
\endfoot
BaptisteLPN06 \href{https://doi.org/10.1016/S1574-6526(06)80026-X}{BaptisteLPN06} & \hyperref[auth:a163]{P. Baptiste}, \hyperref[auth:a118]{P. Laborie}, \hyperref[auth:a164]{Claude Le Pape}, \hyperref[auth:a666]{W. Nuijten} & Constraint-Based Scheduling and Planning & No & \cite{BaptisteLPN06} & 2006 & Handbook of Constraint Programming & 39 & 30 & 25 & No & \ref{c:BaptisteLPN06}\\
GodardLN05 \href{http://www.aaai.org/Library/ICAPS/2005/icaps05-009.php}{GodardLN05} & \hyperref[auth:a782]{D. Godard}, \hyperref[auth:a118]{P. Laborie}, \hyperref[auth:a666]{W. Nuijten} & Randomized Large Neighborhood Search for Cumulative Scheduling & \href{works/GodardLN05.pdf}{Yes} & \cite{GodardLN05} & 2005 & ICAPS 2005 & 9 & 0 & 0 & \ref{b:GodardLN05} & \ref{c:GodardLN05}\\
BaptistePN01 \href{http://dx.doi.org/10.1007/978-1-4615-1479-4}{BaptistePN01} & \hyperref[auth:a163]{P. Baptiste}, \hyperref[auth:a164]{Claude Le Pape}, \hyperref[auth:a666]{W. Nuijten} & Constraint-Based Scheduling & No & \cite{BaptistePN01} & 2001 & Book & null & 296 & 0 & No & \ref{c:BaptistePN01}\\
FocacciLN00 \href{http://www.aaai.org/Library/AIPS/2000/aips00-010.php}{FocacciLN00} & \hyperref[auth:a784]{F. Focacci}, \hyperref[auth:a118]{P. Laborie}, \hyperref[auth:a666]{W. Nuijten} & Solving Scheduling Problems with Setup Times and Alternative Resources & \href{works/FocacciLN00.pdf}{Yes} & \cite{FocacciLN00} & 2000 & AIPS 2000 & 10 & 0 & 0 & \ref{b:FocacciLN00} & \ref{c:FocacciLN00}\\
SourdN00 \href{https://doi.org/10.1287/ijoc.12.4.341.11881}{SourdN00} & \hyperref[auth:a783]{F. Sourd}, \hyperref[auth:a666]{W. Nuijten} & Multiple-Machine Lower Bounds for Shop-Scheduling Problems & \href{works/SourdN00.pdf}{Yes} & \cite{SourdN00} & 2000 & {INFORMS} J. Comput. & 12 & 7 & 14 & \ref{b:SourdN00} & \ref{c:SourdN00}\\
NuijtenP98 \href{https://doi.org/10.1023/A:1009687210594}{NuijtenP98} & \hyperref[auth:a666]{W. Nuijten}, \hyperref[auth:a164]{Claude Le Pape} & Constraint-Based Job Shop Scheduling with {\textbackslash}sc Ilog Scheduler & \href{works/NuijtenP98.pdf}{Yes} & \cite{NuijtenP98} & 1998 & J. Heuristics & 16 & 42 & 0 & \ref{b:NuijtenP98} & \ref{c:NuijtenP98}\\
\end{longtable}
}

\subsection{Works by Emmanuel Poder}
\label{sec:a362}
{\scriptsize
\begin{longtable}{>{\raggedright\arraybackslash}p{3cm}>{\raggedright\arraybackslash}p{6cm}>{\raggedright\arraybackslash}p{6.5cm}rrrp{2.5cm}rrrrr}
\rowcolor{white}\caption{Works from bibtex (Total 6)}\\ \toprule
\rowcolor{white}Key & Authors & Title & LC & Cite & Year & \shortstack{Conference\\/Journal} & Pages & \shortstack{Nr\\Cites} & \shortstack{Nr\\Refs} & b & c \\ \midrule\endhead
\bottomrule
\endfoot
BeldiceanuCDP11 \href{https://doi.org/10.1007/s10479-010-0731-0}{BeldiceanuCDP11} & \hyperref[auth:a128]{N. Beldiceanu}, \hyperref[auth:a91]{M. Carlsson}, \hyperref[auth:a245]{S. Demassey}, \hyperref[auth:a362]{E. Poder} & New filtering for the \emph{cumulative} constraint in the context of non-overlapping rectangles & \href{works/BeldiceanuCDP11.pdf}{Yes} & \cite{BeldiceanuCDP11} & 2011 & Ann. Oper. Res. & 24 & 8 & 8 & \ref{b:BeldiceanuCDP11} & \ref{c:BeldiceanuCDP11}\\
abs-0907-0939 \href{http://arxiv.org/abs/0907.0939}{abs-0907-0939} & \hyperref[auth:a226]{T. Petit}, \hyperref[auth:a362]{E. Poder} & The Soft Cumulative Constraint & \href{works/abs-0907-0939.pdf}{Yes} & \cite{abs-0907-0939} & 2009 & CoRR & 12 & 0 & 0 & \ref{b:abs-0907-0939} & \ref{c:abs-0907-0939}\\
BeldiceanuCP08 \href{https://doi.org/10.1007/978-3-540-68155-7\_5}{BeldiceanuCP08} & \hyperref[auth:a128]{N. Beldiceanu}, \hyperref[auth:a91]{M. Carlsson}, \hyperref[auth:a362]{E. Poder} & New Filtering for the cumulative Constraint in the Context of Non-Overlapping Rectangles & \href{works/BeldiceanuCP08.pdf}{Yes} & \cite{BeldiceanuCP08} & 2008 & CPAIOR 2008 & 15 & 8 & 9 & \ref{b:BeldiceanuCP08} & \ref{c:BeldiceanuCP08}\\
PoderB08 \href{http://www.aaai.org/Library/ICAPS/2008/icaps08-033.php}{PoderB08} & \hyperref[auth:a362]{E. Poder}, \hyperref[auth:a128]{N. Beldiceanu} & Filtering for a Continuous Multi-Resources cumulative Constraint with Resource Consumption and Production & \href{works/PoderB08.pdf}{Yes} & \cite{PoderB08} & 2008 & ICAPS 2008 & 8 & 0 & 0 & \ref{b:PoderB08} & \ref{c:PoderB08}\\
BeldiceanuP07 \href{https://doi.org/10.1007/978-3-540-72397-4\_16}{BeldiceanuP07} & \hyperref[auth:a128]{N. Beldiceanu}, \hyperref[auth:a362]{E. Poder} & A Continuous Multi-resources \emph{cumulative} Constraint with Positive-Negative Resource Consumption-Production & \href{works/BeldiceanuP07.pdf}{Yes} & \cite{BeldiceanuP07} & 2007 & CPAIOR 2007 & 15 & 4 & 7 & \ref{b:BeldiceanuP07} & \ref{c:BeldiceanuP07}\\
PoderBS04 \href{https://doi.org/10.1016/S0377-2217(02)00756-7}{PoderBS04} & \hyperref[auth:a362]{E. Poder}, \hyperref[auth:a128]{N. Beldiceanu}, \hyperref[auth:a722]{E. Sanlaville} & Computing a lower approximation of the compulsory part of a task with varying duration and varying resource consumption & \href{works/PoderBS04.pdf}{Yes} & \cite{PoderBS04} & 2004 & Eur. J. Oper. Res. & 16 & 7 & 8 & \ref{b:PoderBS04} & \ref{c:PoderBS04}\\
\end{longtable}
}

\subsection{Works by Louis{-}Martin Rousseau}
\label{sec:a331}
{\scriptsize
\begin{longtable}{>{\raggedright\arraybackslash}p{3cm}>{\raggedright\arraybackslash}p{6cm}>{\raggedright\arraybackslash}p{6.5cm}rrrp{2.5cm}rrrrr}
\rowcolor{white}\caption{Works from bibtex (Total 6)}\\ \toprule
\rowcolor{white}Key & Authors & Title & LC & Cite & Year & \shortstack{Conference\\/Journal} & Pages & \shortstack{Nr\\Cites} & \shortstack{Nr\\Refs} & b & c \\ \midrule\endhead
\bottomrule
\endfoot
CappartTSR18 \href{https://doi.org/10.1007/978-3-319-98334-9\_32}{CappartTSR18} & \hyperref[auth:a42]{Q. Cappart}, \hyperref[auth:a849]{C. Thomas}, \hyperref[auth:a147]{P. Schaus}, \hyperref[auth:a331]{L. Rousseau} & A Constraint Programming Approach for Solving Patient Transportation Problems & \href{works/CappartTSR18.pdf}{Yes} & \cite{CappartTSR18} & 2018 & CP 2018 & 17 & 6 & 31 & \ref{b:CappartTSR18} & \ref{c:CappartTSR18}\\
DoulabiRP16 \href{https://doi.org/10.1287/ijoc.2015.0686}{DoulabiRP16} & \hyperref[auth:a335]{Seyed Hossein Hashemi Doulabi}, \hyperref[auth:a331]{L. Rousseau}, \hyperref[auth:a8]{G. Pesant} & A Constraint-Programming-Based Branch-and-Price-and-Cut Approach for Operating Room Planning and Scheduling & \href{works/DoulabiRP16.pdf}{Yes} & \cite{DoulabiRP16} & 2016 & {INFORMS} J. Comput. & 17 & 56 & 28 & \ref{b:DoulabiRP16} & \ref{c:DoulabiRP16}\\
PesantRR15 \href{https://doi.org/10.1007/978-3-319-18008-3\_21}{PesantRR15} & \hyperref[auth:a8]{G. Pesant}, \hyperref[auth:a330]{G. Rix}, \hyperref[auth:a331]{L. Rousseau} & A Comparative Study of {MIP} and {CP} Formulations for the {B2B} Scheduling Optimization Problem & \href{works/PesantRR15.pdf}{Yes} & \cite{PesantRR15} & 2015 & CPAIOR 2015 & 16 & 1 & 7 & \ref{b:PesantRR15} & \ref{c:PesantRR15}\\
DoulabiRP14 \href{https://doi.org/10.1007/978-3-319-07046-9\_32}{DoulabiRP14} & \hyperref[auth:a335]{Seyed Hossein Hashemi Doulabi}, \hyperref[auth:a331]{L. Rousseau}, \hyperref[auth:a8]{G. Pesant} & A Constraint Programming-Based Column Generation Approach for Operating Room Planning and Scheduling & \href{works/DoulabiRP14.pdf}{Yes} & \cite{DoulabiRP14} & 2014 & CPAIOR 2014 & 9 & 3 & 10 & \ref{b:DoulabiRP14} & \ref{c:DoulabiRP14}\\
ChapadosJR11 \href{https://doi.org/10.1007/978-3-642-21311-3\_7}{ChapadosJR11} & \hyperref[auth:a349]{N. Chapados}, \hyperref[auth:a350]{M. Joliveau}, \hyperref[auth:a331]{L. Rousseau} & Retail Store Workforce Scheduling by Expected Operating Income Maximization & \href{works/ChapadosJR11.pdf}{Yes} & \cite{ChapadosJR11} & 2011 & CPAIOR 2011 & 6 & 5 & 12 & \ref{b:ChapadosJR11} & \ref{c:ChapadosJR11}\\
HachemiGR11 \href{https://doi.org/10.1007/s10479-010-0698-x}{HachemiGR11} & \hyperref[auth:a623]{Nizar El Hachemi}, \hyperref[auth:a624]{M. Gendreau}, \hyperref[auth:a331]{L. Rousseau} & A hybrid constraint programming approach to the log-truck scheduling problem & \href{works/HachemiGR11.pdf}{Yes} & \cite{HachemiGR11} & 2011 & Ann. Oper. Res. & 16 & 32 & 19 & \ref{b:HachemiGR11} & \ref{c:HachemiGR11}\\
\end{longtable}
}

\subsection{Works by Cyrille Dejemeppe}
\label{sec:a207}
{\scriptsize
\begin{longtable}{>{\raggedright\arraybackslash}p{3cm}>{\raggedright\arraybackslash}p{6cm}>{\raggedright\arraybackslash}p{6.5cm}rrrp{2.5cm}rrrrr}
\rowcolor{white}\caption{Works from bibtex (Total 5)}\\ \toprule
\rowcolor{white}Key & Authors & Title & LC & Cite & Year & \shortstack{Conference\\/Journal} & Pages & \shortstack{Nr\\Cites} & \shortstack{Nr\\Refs} & b & c \\ \midrule\endhead
\bottomrule
\endfoot
CauwelaertDS20 \href{http://dx.doi.org/10.1007/s10951-019-00632-8}{CauwelaertDS20} & \hyperref[auth:a850]{Sasha Van Cauwelaert}, \hyperref[auth:a207]{C. Dejemeppe}, \hyperref[auth:a147]{P. Schaus} & An Efficient Filtering Algorithm for the Unary Resource Constraint with Transition Times and Optional Activities & \href{works/CauwelaertDS20.pdf}{Yes} & \cite{CauwelaertDS20} & 2020 & Journal of Scheduling & 19 & 2 & 21 & \ref{b:CauwelaertDS20} & \ref{c:CauwelaertDS20}\\
CauwelaertDMS16 \href{https://doi.org/10.1007/978-3-319-44953-1\_33}{CauwelaertDMS16} & \hyperref[auth:a206]{Sascha Van Cauwelaert}, \hyperref[auth:a207]{C. Dejemeppe}, \hyperref[auth:a149]{J. Monette}, \hyperref[auth:a147]{P. Schaus} & Efficient Filtering for the Unary Resource with Family-Based Transition Times & \href{works/CauwelaertDMS16.pdf}{Yes} & \cite{CauwelaertDMS16} & 2016 & CP 2016 & 16 & 1 & 12 & \ref{b:CauwelaertDMS16} & \ref{c:CauwelaertDMS16}\\
Dejemeppe16 \href{https://hdl.handle.net/2078.1/178078}{Dejemeppe16} & \hyperref[auth:a207]{C. Dejemeppe} & Constraint programming algorithms and models for scheduling applications & \href{works/Dejemeppe16.pdf}{Yes} & \cite{Dejemeppe16} & 2016 & Catholic University of Louvain, Louvain-la-Neuve, Belgium & 274 & 0 & 0 & \ref{b:Dejemeppe16} & \ref{c:Dejemeppe16}\\
DejemeppeCS15 \href{https://doi.org/10.1007/978-3-319-23219-5\_7}{DejemeppeCS15} & \hyperref[auth:a207]{C. Dejemeppe}, \hyperref[auth:a206]{Sascha Van Cauwelaert}, \hyperref[auth:a147]{P. Schaus} & The Unary Resource with Transition Times & \href{works/DejemeppeCS15.pdf}{Yes} & \cite{DejemeppeCS15} & 2015 & CP 2015 & 16 & 5 & 11 & \ref{b:DejemeppeCS15} & \ref{c:DejemeppeCS15}\\
DejemeppeD14 \href{https://doi.org/10.1007/978-3-319-07046-9\_20}{DejemeppeD14} & \hyperref[auth:a207]{C. Dejemeppe}, \hyperref[auth:a151]{Y. Deville} & Continuously Degrading Resource and Interval Dependent Activity Durations in Nuclear Medicine Patient Scheduling & \href{works/DejemeppeD14.pdf}{Yes} & \cite{DejemeppeD14} & 2014 & CPAIOR 2014 & 9 & 0 & 7 & \ref{b:DejemeppeD14} & \ref{c:DejemeppeD14}\\
\end{longtable}
}

\subsection{Works by Yves Deville}
\label{sec:a151}
{\scriptsize
\begin{longtable}{>{\raggedright\arraybackslash}p{3cm}>{\raggedright\arraybackslash}p{6cm}>{\raggedright\arraybackslash}p{6.5cm}rrrp{2.5cm}rrrrr}
\rowcolor{white}\caption{Works from bibtex (Total 5)}\\ \toprule
\rowcolor{white}Key & Authors & Title & LC & Cite & Year & \shortstack{Conference\\/Journal} & Pages & \shortstack{Nr\\Cites} & \shortstack{Nr\\Refs} & b & c \\ \midrule\endhead
\bottomrule
\endfoot
DejemeppeD14 \href{https://doi.org/10.1007/978-3-319-07046-9\_20}{DejemeppeD14} & \hyperref[auth:a207]{C. Dejemeppe}, \hyperref[auth:a151]{Y. Deville} & Continuously Degrading Resource and Interval Dependent Activity Durations in Nuclear Medicine Patient Scheduling & \href{works/DejemeppeD14.pdf}{Yes} & \cite{DejemeppeD14} & 2014 & CPAIOR 2014 & 9 & 0 & 7 & \ref{b:DejemeppeD14} & \ref{c:DejemeppeD14}\\
HoundjiSWD14 \href{https://doi.org/10.1007/978-3-319-10428-7\_29}{HoundjiSWD14} & \hyperref[auth:a228]{Vinas{\'{e}}tan Ratheil Houndji}, \hyperref[auth:a147]{P. Schaus}, \hyperref[auth:a229]{Laurence A. Wolsey}, \hyperref[auth:a151]{Y. Deville} & The StockingCost Constraint & \href{works/HoundjiSWD14.pdf}{Yes} & \cite{HoundjiSWD14} & 2014 & CP 2014 & 16 & 5 & 7 & \ref{b:HoundjiSWD14} & \ref{c:HoundjiSWD14}\\
SchausHMCMD11 \href{https://doi.org/10.1007/s10601-010-9100-5}{SchausHMCMD11} & \hyperref[auth:a147]{P. Schaus}, \hyperref[auth:a148]{Pascal Van Hentenryck}, \hyperref[auth:a149]{J. Monette}, \hyperref[auth:a150]{C. Coffrin}, \hyperref[auth:a32]{L. Michel}, \hyperref[auth:a151]{Y. Deville} & Solving Steel Mill Slab Problems with constraint-based techniques: CP, LNS, and {CBLS} & \href{works/SchausHMCMD11.pdf}{Yes} & \cite{SchausHMCMD11} & 2011 & Constraints An Int. J. & 23 & 14 & 5 & \ref{b:SchausHMCMD11} & \ref{c:SchausHMCMD11}\\
MonetteDH09 \href{http://aaai.org/ocs/index.php/ICAPS/ICAPS09/paper/view/712}{MonetteDH09} & \hyperref[auth:a149]{J. Monette}, \hyperref[auth:a151]{Y. Deville}, \hyperref[auth:a148]{Pascal Van Hentenryck} & Just-In-Time Scheduling with Constraint Programming & \href{works/MonetteDH09.pdf}{Yes} & \cite{MonetteDH09} & 2009 & ICAPS 2009 & 8 & 0 & 0 & \ref{b:MonetteDH09} & \ref{c:MonetteDH09}\\
MonetteDD07 \href{https://doi.org/10.1007/978-3-540-72397-4\_14}{MonetteDD07} & \hyperref[auth:a149]{J. Monette}, \hyperref[auth:a151]{Y. Deville}, \hyperref[auth:a372]{P. Dupont} & A Position-Based Propagator for the Open-Shop Problem & \href{works/MonetteDD07.pdf}{Yes} & \cite{MonetteDD07} & 2007 & CPAIOR 2007 & 14 & 0 & 12 & \ref{b:MonetteDD07} & \ref{c:MonetteDD07}\\
\end{longtable}
}

\subsection{Works by Mark G. Wallace}
\label{sec:a155}
{\scriptsize
\begin{longtable}{>{\raggedright\arraybackslash}p{3cm}>{\raggedright\arraybackslash}p{6cm}>{\raggedright\arraybackslash}p{6.5cm}rrrp{2.5cm}rrrrr}
\rowcolor{white}\caption{Works from bibtex (Total 5)}\\ \toprule
\rowcolor{white}Key & Authors & Title & LC & Cite & Year & \shortstack{Conference\\/Journal} & Pages & \shortstack{Nr\\Cites} & \shortstack{Nr\\Refs} & b & c \\ \midrule\endhead
\bottomrule
\endfoot
SchuttFSW13 \href{https://doi.org/10.1007/s10951-012-0285-x}{SchuttFSW13} & \hyperref[auth:a124]{A. Schutt}, \hyperref[auth:a154]{T. Feydy}, \hyperref[auth:a125]{Peter J. Stuckey}, \hyperref[auth:a155]{Mark G. Wallace} & Solving RCPSP/max by lazy clause generation & \href{works/SchuttFSW13.pdf}{Yes} & \cite{SchuttFSW13} & 2013 & J. Sched. & 17 & 43 & 23 & \ref{b:SchuttFSW13} & \ref{c:SchuttFSW13}\\
GuSW12 \href{https://doi.org/10.1007/978-3-642-33558-7\_55}{GuSW12} & \hyperref[auth:a341]{H. Gu}, \hyperref[auth:a125]{Peter J. Stuckey}, \hyperref[auth:a155]{Mark G. Wallace} & Maximising the Net Present Value of Large Resource-Constrained Projects & \href{works/GuSW12.pdf}{Yes} & \cite{GuSW12} & 2012 & CP 2012 & 15 & 5 & 20 & \ref{b:GuSW12} & \ref{c:GuSW12}\\
SchuttCSW12 \href{https://doi.org/10.1007/978-3-642-29828-8\_24}{SchuttCSW12} & \hyperref[auth:a124]{A. Schutt}, \hyperref[auth:a348]{G. Chu}, \hyperref[auth:a125]{Peter J. Stuckey}, \hyperref[auth:a155]{Mark G. Wallace} & Maximising the Net Present Value for Resource-Constrained Project Scheduling & \href{works/SchuttCSW12.pdf}{Yes} & \cite{SchuttCSW12} & 2012 & CPAIOR 2012 & 17 & 18 & 21 & \ref{b:SchuttCSW12} & \ref{c:SchuttCSW12}\\
SchuttFSW11 \href{https://doi.org/10.1007/s10601-010-9103-2}{SchuttFSW11} & \hyperref[auth:a124]{A. Schutt}, \hyperref[auth:a154]{T. Feydy}, \hyperref[auth:a125]{Peter J. Stuckey}, \hyperref[auth:a155]{Mark G. Wallace} & Explaining the cumulative propagator & \href{works/SchuttFSW11.pdf}{Yes} & \cite{SchuttFSW11} & 2011 & Constraints An Int. J. & 33 & 57 & 23 & \ref{b:SchuttFSW11} & \ref{c:SchuttFSW11}\\
abs-1009-0347 \href{http://arxiv.org/abs/1009.0347}{abs-1009-0347} & \hyperref[auth:a124]{A. Schutt}, \hyperref[auth:a154]{T. Feydy}, \hyperref[auth:a125]{Peter J. Stuckey}, \hyperref[auth:a155]{Mark G. Wallace} & Solving the Resource Constrained Project Scheduling Problem with Generalized Precedences by Lazy Clause Generation & \href{works/abs-1009-0347.pdf}{Yes} & \cite{abs-1009-0347} & 2010 & CoRR & 37 & 0 & 0 & \ref{b:abs-1009-0347} & \ref{c:abs-1009-0347}\\
\end{longtable}
}

\subsection{Works by Roger Kameugne}
\label{sec:a10}
{\scriptsize
\begin{longtable}{>{\raggedright\arraybackslash}p{3cm}>{\raggedright\arraybackslash}p{6cm}>{\raggedright\arraybackslash}p{6.5cm}rrrp{2.5cm}rrrrr}
\rowcolor{white}\caption{Works from bibtex (Total 5)}\\ \toprule
\rowcolor{white}Key & Authors & Title & LC & Cite & Year & \shortstack{Conference\\/Journal} & Pages & \shortstack{Nr\\Cites} & \shortstack{Nr\\Refs} & b & c \\ \midrule\endhead
\bottomrule
\endfoot
KameugneFND23 \href{https://doi.org/10.4230/LIPIcs.CP.2023.20}{KameugneFND23} & \hyperref[auth:a10]{R. Kameugne}, \hyperref[auth:a11]{S{\'{e}}v{\'{e}}rine Betmbe Fetgo}, \hyperref[auth:a12]{T. Noulamo}, \hyperref[auth:a13]{Cl{\'{e}}mentin Tayou Djam{\'{e}}gni} & Horizontally Elastic Edge Finder Rule for Cumulative Constraint Based on Slack and Density & \href{works/KameugneFND23.pdf}{Yes} & \cite{KameugneFND23} & 2023 & CP 2023 & 17 & 0 & 0 & \ref{b:KameugneFND23} & \ref{c:KameugneFND23}\\
KameugneFGOQ18 \href{https://doi.org/10.1007/978-3-319-93031-2\_23}{KameugneFGOQ18} & \hyperref[auth:a10]{R. Kameugne}, \hyperref[auth:a11]{S{\'{e}}v{\'{e}}rine Betmbe Fetgo}, \hyperref[auth:a315]{V. Gingras}, \hyperref[auth:a52]{Y. Ouellet}, \hyperref[auth:a37]{C. Quimper} & Horizontally Elastic Not-First/Not-Last Filtering Algorithm for Cumulative Resource Constraint & \href{works/KameugneFGOQ18.pdf}{Yes} & \cite{KameugneFGOQ18} & 2018 & CPAIOR 2018 & 17 & 1 & 12 & \ref{b:KameugneFGOQ18} & \ref{c:KameugneFGOQ18}\\
Kameugne15 \href{https://doi.org/10.1007/s10601-015-9227-5}{Kameugne15} & \hyperref[auth:a10]{R. Kameugne} & Propagation techniques of resource constraint for cumulative scheduling & \href{works/Kameugne15.pdf}{Yes} & \cite{Kameugne15} & 2015 & Constraints An Int. J. & 2 & 0 & 0 & \ref{b:Kameugne15} & \ref{c:Kameugne15}\\
KameugneFSN14 \href{https://doi.org/10.1007/s10601-013-9157-z}{KameugneFSN14} & \hyperref[auth:a10]{R. Kameugne}, \hyperref[auth:a130]{Laure Pauline Fotso}, \hyperref[auth:a131]{Joseph D. Scott}, \hyperref[auth:a132]{Y. Ngo{-}Kateu} & A quadratic edge-finding filtering algorithm for cumulative resource constraints & \href{works/KameugneFSN14.pdf}{Yes} & \cite{KameugneFSN14} & 2014 & Constraints An Int. J. & 27 & 6 & 10 & \ref{b:KameugneFSN14} & \ref{c:KameugneFSN14}\\
KameugneFSN11 \href{https://doi.org/10.1007/978-3-642-23786-7\_37}{KameugneFSN11} & \hyperref[auth:a10]{R. Kameugne}, \hyperref[auth:a130]{Laure Pauline Fotso}, \hyperref[auth:a131]{Joseph D. Scott}, \hyperref[auth:a132]{Y. Ngo{-}Kateu} & A Quadratic Edge-Finding Filtering Algorithm for Cumulative Resource Constraints & \href{works/KameugneFSN11.pdf}{Yes} & \cite{KameugneFSN11} & 2011 & CP 2011 & 15 & 7 & 9 & \ref{b:KameugneFSN11} & \ref{c:KameugneFSN11}\\
\end{longtable}
}

\subsection{Works by Juan M. Novas}
\label{sec:a529}
{\scriptsize
\begin{longtable}{>{\raggedright\arraybackslash}p{3cm}>{\raggedright\arraybackslash}p{6cm}>{\raggedright\arraybackslash}p{6.5cm}rrrp{2.5cm}rrrrr}
\rowcolor{white}\caption{Works from bibtex (Total 5)}\\ \toprule
\rowcolor{white}Key & Authors & Title & LC & Cite & Year & \shortstack{Conference\\/Journal} & Pages & \shortstack{Nr\\Cites} & \shortstack{Nr\\Refs} & b & c \\ \midrule\endhead
\bottomrule
\endfoot
Novas19 \href{https://doi.org/10.1016/j.cie.2019.07.011}{Novas19} & \hyperref[auth:a529]{Juan M. Novas} & Production scheduling and lot streaming at flexible job-shops environments using constraint programming & \href{works/Novas19.pdf}{Yes} & \cite{Novas19} & 2019 & Comput. Ind. Eng. & 13 & 30 & 29 & \ref{b:Novas19} & \ref{c:Novas19}\\
NovaraNH16 \href{https://doi.org/10.1016/j.compchemeng.2016.04.030}{NovaraNH16} & \hyperref[auth:a595]{Franco M. Novara}, \hyperref[auth:a529]{Juan M. Novas}, \hyperref[auth:a596]{Gabriela P. Henning} & A novel constraint programming model for large-scale scheduling problems in multiproduct multistage batch plants: Limited resources and campaign-based operation & \href{works/NovaraNH16.pdf}{Yes} & \cite{NovaraNH16} & 2016 & Comput. Chem. Eng. & 17 & 18 & 31 & \ref{b:NovaraNH16} & \ref{c:NovaraNH16}\\
NovasH14 \href{https://doi.org/10.1016/j.eswa.2013.09.026}{NovasH14} & \hyperref[auth:a529]{Juan M. Novas}, \hyperref[auth:a596]{Gabriela P. Henning} & Integrated scheduling of resource-constrained flexible manufacturing systems using constraint programming & \href{works/NovasH14.pdf}{Yes} & \cite{NovasH14} & 2014 & Expert Syst. Appl. & 14 & 35 & 26 & \ref{b:NovasH14} & \ref{c:NovasH14}\\
NovasH12 \href{https://doi.org/10.1016/j.compchemeng.2012.01.005}{NovasH12} & \hyperref[auth:a529]{Juan M. Novas}, \hyperref[auth:a596]{Gabriela P. Henning} & A comprehensive constraint programming approach for the rolling horizon-based scheduling of automated wet-etch stations & \href{works/NovasH12.pdf}{Yes} & \cite{NovasH12} & 2012 & Comput. Chem. Eng. & 17 & 17 & 15 & \ref{b:NovasH12} & \ref{c:NovasH12}\\
NovasH10 \href{https://doi.org/10.1016/j.compchemeng.2010.07.011}{NovasH10} & \hyperref[auth:a529]{Juan M. Novas}, \hyperref[auth:a596]{Gabriela P. Henning} & Reactive scheduling framework based on domain knowledge and constraint programming & \href{works/NovasH10.pdf}{Yes} & \cite{NovasH10} & 2010 & Comput. Chem. Eng. & 20 & 48 & 19 & \ref{b:NovasH10} & \ref{c:NovasH10}\\
\end{longtable}
}

\subsection{Works by Kenneth N. Brown}
\label{sec:a222}
{\scriptsize
\begin{longtable}{>{\raggedright\arraybackslash}p{3cm}>{\raggedright\arraybackslash}p{6cm}>{\raggedright\arraybackslash}p{6.5cm}rrrp{2.5cm}rrrrr}
\rowcolor{white}\caption{Works from bibtex (Total 5)}\\ \toprule
\rowcolor{white}Key & Authors & Title & LC & Cite & Year & \shortstack{Conference\\/Journal} & Pages & \shortstack{Nr\\Cites} & \shortstack{Nr\\Refs} & b & c \\ \midrule\endhead
\bottomrule
\endfoot
AntunesABDEGGOL20 \href{https://doi.org/10.1142/S0218213020600076}{AntunesABDEGGOL20} & \hyperref[auth:a893]{M. Antunes}, \hyperref[auth:a894]{V. Armant}, \hyperref[auth:a222]{Kenneth N. Brown}, \hyperref[auth:a895]{Daniel A. Desmond}, \hyperref[auth:a896]{G. Escamocher}, \hyperref[auth:a897]{A. George}, \hyperref[auth:a182]{D. Grimes}, \hyperref[auth:a898]{M. O'Keeffe}, \hyperref[auth:a899]{Y. Lin}, \hyperref[auth:a16]{B. O'Sullivan}, \hyperref[auth:a900]{C. Ozturk}, \hyperref[auth:a901]{L. Quesada}, \hyperref[auth:a129]{M. Siala}, \hyperref[auth:a17]{H. Simonis}, \hyperref[auth:a837]{N. Wilson} & Assigning and Scheduling Service Visits in a Mixed Urban/Rural Setting & No & \cite{AntunesABDEGGOL20} & 2020 & Int. J. Artif. Intell. Tools & 31 & 0 & 16 & No & \ref{c:AntunesABDEGGOL20}\\
AntunesABDEGGOL18 \href{https://doi.org/10.1109/ICTAI.2018.00027}{AntunesABDEGGOL18} & \hyperref[auth:a893]{M. Antunes}, \hyperref[auth:a894]{V. Armant}, \hyperref[auth:a222]{Kenneth N. Brown}, \hyperref[auth:a895]{Daniel A. Desmond}, \hyperref[auth:a896]{G. Escamocher}, \hyperref[auth:a897]{A. George}, \hyperref[auth:a182]{D. Grimes}, \hyperref[auth:a898]{M. O'Keeffe}, \hyperref[auth:a899]{Y. Lin}, \hyperref[auth:a16]{B. O'Sullivan}, \hyperref[auth:a900]{C. Ozturk}, \hyperref[auth:a901]{L. Quesada}, \hyperref[auth:a129]{M. Siala}, \hyperref[auth:a17]{H. Simonis}, \hyperref[auth:a837]{N. Wilson} & Assigning and Scheduling Service Visits in a Mixed Urban/Rural Setting & No & \cite{AntunesABDEGGOL18} & 2018 & ICTAI 2018 & 8 & 1 & 24 & No & \ref{c:AntunesABDEGGOL18}\\
MurphyMB15 \href{https://doi.org/10.1007/978-3-319-23219-5\_47}{MurphyMB15} & \hyperref[auth:a220]{Se{\'{a}}n {\'{O}}g Murphy}, \hyperref[auth:a221]{O. Manzano}, \hyperref[auth:a222]{Kenneth N. Brown} & Design and Evaluation of a Constraint-Based Energy Saving and Scheduling Recommender System & \href{works/MurphyMB15.pdf}{Yes} & \cite{MurphyMB15} & 2015 & CP 2015 & 17 & 1 & 20 & \ref{b:MurphyMB15} & \ref{c:MurphyMB15}\\
WuBB09 \href{https://doi.org/10.1016/j.cor.2008.08.008}{WuBB09} & \hyperref[auth:a276]{Christine Wei Wu}, \hyperref[auth:a222]{Kenneth N. Brown}, \hyperref[auth:a89]{J. Christopher Beck} & Scheduling with uncertain durations: Modeling beta-robust scheduling with constraints & No & \cite{WuBB09} & 2009 & Comput. Oper. Res. & 9 & 42 & 5 & No & \ref{c:WuBB09}\\
WuBB05 \href{https://doi.org/10.1007/11564751\_110}{WuBB05} & \hyperref[auth:a276]{Christine Wei Wu}, \hyperref[auth:a222]{Kenneth N. Brown}, \hyperref[auth:a89]{J. Christopher Beck} & Scheduling with Uncertain Start Dates & \href{works/WuBB05.pdf}{Yes} & \cite{WuBB05} & 2005 & CP 2005 & 1 & 0 & 0 & \ref{b:WuBB05} & \ref{c:WuBB05}\\
\end{longtable}
}

\subsection{Works by Mohamed Siala}
\label{sec:a129}
{\scriptsize
\begin{longtable}{>{\raggedright\arraybackslash}p{3cm}>{\raggedright\arraybackslash}p{6cm}>{\raggedright\arraybackslash}p{6.5cm}rrrp{2.5cm}rrrrr}
\rowcolor{white}\caption{Works from bibtex (Total 5)}\\ \toprule
\rowcolor{white}Key & Authors & Title & LC & Cite & Year & \shortstack{Conference\\/Journal} & Pages & \shortstack{Nr\\Cites} & \shortstack{Nr\\Refs} & b & c \\ \midrule\endhead
\bottomrule
\endfoot
AntunesABDEGGOL20 \href{https://doi.org/10.1142/S0218213020600076}{AntunesABDEGGOL20} & \hyperref[auth:a893]{M. Antunes}, \hyperref[auth:a894]{V. Armant}, \hyperref[auth:a222]{Kenneth N. Brown}, \hyperref[auth:a895]{Daniel A. Desmond}, \hyperref[auth:a896]{G. Escamocher}, \hyperref[auth:a897]{A. George}, \hyperref[auth:a182]{D. Grimes}, \hyperref[auth:a898]{M. O'Keeffe}, \hyperref[auth:a899]{Y. Lin}, \hyperref[auth:a16]{B. O'Sullivan}, \hyperref[auth:a900]{C. Ozturk}, \hyperref[auth:a901]{L. Quesada}, \hyperref[auth:a129]{M. Siala}, \hyperref[auth:a17]{H. Simonis}, \hyperref[auth:a837]{N. Wilson} & Assigning and Scheduling Service Visits in a Mixed Urban/Rural Setting & No & \cite{AntunesABDEGGOL20} & 2020 & Int. J. Artif. Intell. Tools & 31 & 0 & 16 & No & \ref{c:AntunesABDEGGOL20}\\
AntunesABDEGGOL18 \href{https://doi.org/10.1109/ICTAI.2018.00027}{AntunesABDEGGOL18} & \hyperref[auth:a893]{M. Antunes}, \hyperref[auth:a894]{V. Armant}, \hyperref[auth:a222]{Kenneth N. Brown}, \hyperref[auth:a895]{Daniel A. Desmond}, \hyperref[auth:a896]{G. Escamocher}, \hyperref[auth:a897]{A. George}, \hyperref[auth:a182]{D. Grimes}, \hyperref[auth:a898]{M. O'Keeffe}, \hyperref[auth:a899]{Y. Lin}, \hyperref[auth:a16]{B. O'Sullivan}, \hyperref[auth:a900]{C. Ozturk}, \hyperref[auth:a901]{L. Quesada}, \hyperref[auth:a129]{M. Siala}, \hyperref[auth:a17]{H. Simonis}, \hyperref[auth:a837]{N. Wilson} & Assigning and Scheduling Service Visits in a Mixed Urban/Rural Setting & No & \cite{AntunesABDEGGOL18} & 2018 & ICTAI 2018 & 8 & 1 & 24 & No & \ref{c:AntunesABDEGGOL18}\\
Siala15 \href{https://doi.org/10.1007/s10601-015-9213-y}{Siala15} & \hyperref[auth:a129]{M. Siala} & Search, propagation, and learning in sequencing and scheduling problems & \href{works/Siala15.pdf}{Yes} & \cite{Siala15} & 2015 & Constraints An Int. J. & 2 & 4 & 0 & \ref{b:Siala15} & \ref{c:Siala15}\\
Siala15a \href{https://tel.archives-ouvertes.fr/tel-01164291}{Siala15a} & \hyperref[auth:a129]{M. Siala} & Search, propagation, and learning in sequencing and scheduling problems. (Recherche, propagation et apprentissage dans les probl{\`{e}}mes de s{\'{e}}quencement et d'ordonnancement) & \href{works/Siala15a.pdf}{Yes} & \cite{Siala15a} & 2015 & {INSA} Toulouse, France & 199 & 0 & 0 & \ref{b:Siala15a} & \ref{c:Siala15a}\\
SialaAH15 \href{https://doi.org/10.1007/978-3-319-23219-5\_28}{SialaAH15} & \hyperref[auth:a129]{M. Siala}, \hyperref[auth:a6]{C. Artigues}, \hyperref[auth:a1]{E. Hebrard} & Two Clause Learning Approaches for Disjunctive Scheduling & \href{works/SialaAH15.pdf}{Yes} & \cite{SialaAH15} & 2015 & CP 2015 & 10 & 4 & 17 & \ref{b:SialaAH15} & \ref{c:SialaAH15}\\
\end{longtable}
}

\subsection{Works by Marek Vlk}
\label{sec:a313}
{\scriptsize
\begin{longtable}{>{\raggedright\arraybackslash}p{3cm}>{\raggedright\arraybackslash}p{6cm}>{\raggedright\arraybackslash}p{6.5cm}rrrp{2.5cm}rrrrr}
\rowcolor{white}\caption{Works from bibtex (Total 5)}\\ \toprule
\rowcolor{white}Key & Authors & Title & LC & Cite & Year & \shortstack{Conference\\/Journal} & Pages & \shortstack{Nr\\Cites} & \shortstack{Nr\\Refs} & b & c \\ \midrule\endhead
\bottomrule
\endfoot
abs-2305-19888 \href{https://doi.org/10.48550/arXiv.2305.19888}{abs-2305-19888} & \hyperref[auth:a437]{V. Heinz}, \hyperref[auth:a438]{A. Nov{\'{a}}k}, \hyperref[auth:a313]{M. Vlk}, \hyperref[auth:a116]{Z. Hanz{\'{a}}lek} & Constraint Programming and Constructive Heuristics for Parallel Machine Scheduling with Sequence-Dependent Setups and Common Servers & \href{works/abs-2305-19888.pdf}{Yes} & \cite{abs-2305-19888} & 2023 & CoRR & 42 & 0 & 0 & \ref{b:abs-2305-19888} & \ref{c:abs-2305-19888}\\
HeinzNVH22 \href{https://doi.org/10.1016/j.cie.2022.108586}{HeinzNVH22} & \hyperref[auth:a437]{V. Heinz}, \hyperref[auth:a438]{A. Nov{\'{a}}k}, \hyperref[auth:a313]{M. Vlk}, \hyperref[auth:a116]{Z. Hanz{\'{a}}lek} & Constraint Programming and constructive heuristics for parallel machine scheduling with sequence-dependent setups and common servers & \href{works/HeinzNVH22.pdf}{Yes} & \cite{HeinzNVH22} & 2022 & Comput. Ind. Eng. & 16 & 5 & 25 & \ref{b:HeinzNVH22} & \ref{c:HeinzNVH22}\\
VlkHT21 \href{https://doi.org/10.1016/j.cie.2021.107317}{VlkHT21} & \hyperref[auth:a313]{M. Vlk}, \hyperref[auth:a116]{Z. Hanz{\'{a}}lek}, \hyperref[auth:a480]{S. Tang} & Constraint programming approaches to joint routing and scheduling in time-sensitive networks & \href{works/VlkHT21.pdf}{Yes} & \cite{VlkHT21} & 2021 & Comput. Ind. Eng. & 14 & 7 & 22 & \ref{b:VlkHT21} & \ref{c:VlkHT21}\\
BenediktSMVH18 \href{https://doi.org/10.1007/978-3-319-93031-2\_6}{BenediktSMVH18} & \hyperref[auth:a114]{O. Benedikt}, \hyperref[auth:a312]{P. Sucha}, \hyperref[auth:a115]{I. M{\'{o}}dos}, \hyperref[auth:a313]{M. Vlk}, \hyperref[auth:a116]{Z. Hanz{\'{a}}lek} & Energy-Aware Production Scheduling with Power-Saving Modes & \href{works/BenediktSMVH18.pdf}{Yes} & \cite{BenediktSMVH18} & 2018 & CPAIOR 2018 & 10 & 2 & 12 & \ref{b:BenediktSMVH18} & \ref{c:BenediktSMVH18}\\
BartakV15 \href{}{BartakV15} & \hyperref[auth:a152]{R. Bart{\'{a}}k}, \hyperref[auth:a313]{M. Vlk} & Reactive Recovery from Machine Breakdown in Production Scheduling with Temporal Distance and Resource Constraints & \href{works/BartakV15.pdf}{Yes} & \cite{BartakV15} & 2015 & ICAART 2015 & 12 & 0 & 0 & \ref{b:BartakV15} & \ref{c:BartakV15}\\
\end{longtable}
}

\subsection{Works by Nic Wilson}
\label{sec:a837}
{\scriptsize
\begin{longtable}{>{\raggedright\arraybackslash}p{3cm}>{\raggedright\arraybackslash}p{6cm}>{\raggedright\arraybackslash}p{6.5cm}rrrp{2.5cm}rrrrr}
\rowcolor{white}\caption{Works from bibtex (Total 5)}\\ \toprule
\rowcolor{white}Key & Authors & Title & LC & Cite & Year & \shortstack{Conference\\/Journal} & Pages & \shortstack{Nr\\Cites} & \shortstack{Nr\\Refs} & b & c \\ \midrule\endhead
\bottomrule
\endfoot
AntunesABDEGGOL20 \href{https://doi.org/10.1142/S0218213020600076}{AntunesABDEGGOL20} & \hyperref[auth:a893]{M. Antunes}, \hyperref[auth:a894]{V. Armant}, \hyperref[auth:a222]{Kenneth N. Brown}, \hyperref[auth:a895]{Daniel A. Desmond}, \hyperref[auth:a896]{G. Escamocher}, \hyperref[auth:a897]{A. George}, \hyperref[auth:a182]{D. Grimes}, \hyperref[auth:a898]{M. O'Keeffe}, \hyperref[auth:a899]{Y. Lin}, \hyperref[auth:a16]{B. O'Sullivan}, \hyperref[auth:a900]{C. Ozturk}, \hyperref[auth:a901]{L. Quesada}, \hyperref[auth:a129]{M. Siala}, \hyperref[auth:a17]{H. Simonis}, \hyperref[auth:a837]{N. Wilson} & Assigning and Scheduling Service Visits in a Mixed Urban/Rural Setting & No & \cite{AntunesABDEGGOL20} & 2020 & Int. J. Artif. Intell. Tools & 31 & 0 & 16 & No & \ref{c:AntunesABDEGGOL20}\\
AntunesABDEGGOL18 \href{https://doi.org/10.1109/ICTAI.2018.00027}{AntunesABDEGGOL18} & \hyperref[auth:a893]{M. Antunes}, \hyperref[auth:a894]{V. Armant}, \hyperref[auth:a222]{Kenneth N. Brown}, \hyperref[auth:a895]{Daniel A. Desmond}, \hyperref[auth:a896]{G. Escamocher}, \hyperref[auth:a897]{A. George}, \hyperref[auth:a182]{D. Grimes}, \hyperref[auth:a898]{M. O'Keeffe}, \hyperref[auth:a899]{Y. Lin}, \hyperref[auth:a16]{B. O'Sullivan}, \hyperref[auth:a900]{C. Ozturk}, \hyperref[auth:a901]{L. Quesada}, \hyperref[auth:a129]{M. Siala}, \hyperref[auth:a17]{H. Simonis}, \hyperref[auth:a837]{N. Wilson} & Assigning and Scheduling Service Visits in a Mixed Urban/Rural Setting & No & \cite{AntunesABDEGGOL18} & 2018 & ICTAI 2018 & 8 & 1 & 24 & No & \ref{c:AntunesABDEGGOL18}\\
BeckW07 \href{https://doi.org/10.1613/jair.2080}{BeckW07} & \hyperref[auth:a89]{J. Christopher Beck}, \hyperref[auth:a837]{N. Wilson} & Proactive Algorithms for Job Shop Scheduling with Probabilistic Durations & \href{works/BeckW07.pdf}{Yes} & \cite{BeckW07} & 2007 & J. Artif. Intell. Res. & 50 & 27 & 0 & \ref{b:BeckW07} & \ref{c:BeckW07}\\
BeckW05 \href{http://ijcai.org/Proceedings/05/Papers/0748.pdf}{BeckW05} & \hyperref[auth:a89]{J. Christopher Beck}, \hyperref[auth:a837]{N. Wilson} & Proactive Algorithms for Scheduling with Probabilistic Durations & \href{works/BeckW05.pdf}{Yes} & \cite{BeckW05} & 2005 & IJCAI 2005 & 6 & 0 & 0 & \ref{b:BeckW05} & \ref{c:BeckW05}\\
BeckW04 \href{}{BeckW04} & \hyperref[auth:a89]{J. Christopher Beck}, \hyperref[auth:a837]{N. Wilson} & Job Shop Scheduling with Probabilistic Durations & \href{works/BeckW04.pdf}{Yes} & \cite{BeckW04} & 2004 & ECAI 2004 & 5 & 0 & 0 & \ref{b:BeckW04} & \ref{c:BeckW04}\\
\end{longtable}
}

\subsection{Works by Armin Wolf}
\label{sec:a51}
{\scriptsize
\begin{longtable}{>{\raggedright\arraybackslash}p{3cm}>{\raggedright\arraybackslash}p{6cm}>{\raggedright\arraybackslash}p{6.5cm}rrrp{2.5cm}rrrrr}
\rowcolor{white}\caption{Works from bibtex (Total 5)}\\ \toprule
\rowcolor{white}Key & Authors & Title & LC & Cite & Year & \shortstack{Conference\\/Journal} & Pages & \shortstack{Nr\\Cites} & \shortstack{Nr\\Refs} & b & c \\ \midrule\endhead
\bottomrule
\endfoot
GeitzGSSW22 \href{https://doi.org/10.1007/978-3-031-08011-1\_10}{GeitzGSSW22} & \hyperref[auth:a47]{M. Geitz}, \hyperref[auth:a48]{C. Grozea}, \hyperref[auth:a49]{W. Steigerwald}, \hyperref[auth:a50]{R. St{\"{o}}hr}, \hyperref[auth:a51]{A. Wolf} & Solving the Extended Job Shop Scheduling Problem with AGVs - Classical and Quantum Approaches & \href{works/GeitzGSSW22.pdf}{Yes} & \cite{GeitzGSSW22} & 2022 & CPAIOR 2022 & 18 & 0 & 24 & \ref{b:GeitzGSSW22} & \ref{c:GeitzGSSW22}\\
SchuttW10 \href{https://doi.org/10.1007/978-3-642-15396-9\_36}{SchuttW10} & \hyperref[auth:a124]{A. Schutt}, \hyperref[auth:a51]{A. Wolf} & A New \emph{O}(\emph{n}\({}^{\mbox{2}}\)log\emph{n}) Not-First/Not-Last Pruning Algorithm for Cumulative Resource Constraints & \href{works/SchuttW10.pdf}{Yes} & \cite{SchuttW10} & 2010 & CP 2010 & 15 & 13 & 14 & \ref{b:SchuttW10} & \ref{c:SchuttW10}\\
SchuttWS05 \href{https://doi.org/10.1007/11963578\_6}{SchuttWS05} & \hyperref[auth:a124]{A. Schutt}, \hyperref[auth:a51]{A. Wolf}, \hyperref[auth:a720]{G. Schrader} & Not-First and Not-Last Detection for Cumulative Scheduling in \emph{O}(\emph{n}\({}^{\mbox{3}}\)log\emph{n}) & \href{works/SchuttWS05.pdf}{Yes} & \cite{SchuttWS05} & 2005 & INAP 2005 & 15 & 6 & 4 & \ref{b:SchuttWS05} & \ref{c:SchuttWS05}\\
WolfS05 \href{https://doi.org/10.1007/11963578\_8}{WolfS05} & \hyperref[auth:a51]{A. Wolf}, \hyperref[auth:a720]{G. Schrader} & \emph{O}(\emph{n} log\emph{n}) Overload Checking for the Cumulative Constraint and Its Application & \href{works/WolfS05.pdf}{Yes} & \cite{WolfS05} & 2005 & INAP 2005 & 14 & 6 & 6 & \ref{b:WolfS05} & \ref{c:WolfS05}\\
Wolf03 \href{https://doi.org/10.1007/978-3-540-45193-8\_50}{Wolf03} & \hyperref[auth:a51]{A. Wolf} & Pruning while Sweeping over Task Intervals & \href{works/Wolf03.pdf}{Yes} & \cite{Wolf03} & 2003 & CP 2003 & 15 & 11 & 7 & \ref{b:Wolf03} & \ref{c:Wolf03}\\
\end{longtable}
}



\clearpage
\section{Other Works}

\clearpage
\subsection{Books from bibtex}
{\scriptsize
\begin{longtable}{>{\raggedright\arraybackslash}p{3cm}>{\raggedright\arraybackslash}p{4.5cm}>{\raggedright\arraybackslash}p{6.0cm}rrrp{2.5cm}rp{1cm}p{1cm}rr}
\rowcolor{white}\caption{BOOK (Total 3)}\\ \toprule
\rowcolor{white}\shortstack{Key\\Source} & Authors & Title (Colored by Open Access)& LC & Cite & Year & \shortstack{Conference\\/Journal\\/School} & Pages & \shortstack{Cites\\OC XR\\SC} & \shortstack{Refs\\OC\\XR} & b & c \\ \midrule\endhead
\bottomrule
\endfoot
\index{ArtiguesDN08}\rowlabel{a:ArtiguesDN08}ArtiguesDN08 \href{http://dx.doi.org/10.1002/9780470611227}{ArtiguesDN08} & \hyperref[auth:a930]{} & Resource Constrained Project Scheduling & No & \cite{ArtiguesDN08} & 2008 & Book & null & 63 0 0 & 0 0 & No & n/a\\
\index{BaptistePN01}\rowlabel{a:BaptistePN01}BaptistePN01 \href{http://dx.doi.org/10.1007/978-1-4615-1479-4}{BaptistePN01} & \hyperref[auth:a162]{P. Baptiste}, \hyperref[auth:a163]{C. L. Pape}, \hyperref[auth:a656]{W. Nuijten} & Constraint-Based Scheduling & No & \cite{BaptistePN01} & 2001 & Book & null & 296 302 0 & 0 0 & No & n/a\\
\index{Hooker00}\rowlabel{a:Hooker00}Hooker00 \href{http://dx.doi.org/10.1002/9781118033036}{Hooker00} & \hyperref[auth:a160]{J. N. Hooker} & Logic Based Methods for Optimization: Combining Optimization and Constraint Satisfaction & No & \cite{Hooker00} & 2000 & Book & null & 185 186 0 & 0 0 & No & n/a\\
\end{longtable}
}



\clearpage
\subsection{PhDThesis from bibtex}
{\scriptsize
\begin{longtable}{>{\raggedright\arraybackslash}p{3cm}>{\raggedright\arraybackslash}p{6cm}>{\raggedright\arraybackslash}p{6.5cm}rrrp{2.5cm}rrrrr}
\rowcolor{white}\caption{Works from bibtex (Total 27)}\\ \toprule
\rowcolor{white}Key & Authors & Title & LC & Cite & Year & \shortstack{Conference\\/Journal} & Pages & \shortstack{Nr\\Cites} & \shortstack{Nr\\Refs} & b & c \\ \midrule\endhead
\bottomrule
\endfoot
\rowlabel{a:Astrand21}Astrand21 \href{https://nbn-resolving.org/urn:nbn:se:kth:diva-294959}{Astrand21} & \hyperref[auth:a74]{M. {\AA}strand} & Short-term Underground Mine Scheduling: An Industrial Application of Constraint Programming & \href{works/Astrand21.pdf}{Yes} & \cite{Astrand21} & 2021 & Royal Institute of Technology, Stockholm, Sweden & 142 & 0 & 0 & \ref{b:Astrand21} & \ref{c:Astrand21}\\
\rowlabel{a:Godet21a}Godet21a \href{https://tel.archives-ouvertes.fr/tel-03681868}{Godet21a} & \hyperref[auth:a476]{A. Godet} & Sur le tri de t{\^{a}}ches pour r{\'{e}}soudre des probl{\`{e}}mes d'ordonnancement avec la programmation par contraintes. (On the use of tasks ordering to solve scheduling problems with constraint programming) & \href{works/Godet21a.pdf}{Yes} & \cite{Godet21a} & 2021 & {IMT} Atlantique Bretagne Pays de la Loire, Brest, France & 168 & 0 & 0 & \ref{b:Godet21a} & \ref{c:Godet21a}\\
\rowlabel{a:Groleaz21}Groleaz21 \href{https://hal.science/tel-03266690}{Groleaz21} & \hyperref[auth:a83]{L. Groleaz} & {The Group Cumulative Scheduling Problem} & \href{works/Groleaz21.pdf}{Yes} & \cite{Groleaz21} & 2021 & {Universit{\'e} de Lyon} & 153 & 0 & 0 & \ref{b:Groleaz21} & \ref{c:Groleaz21}\\
\rowlabel{a:Lemos21}Lemos21 \href{https://scholar.tecnico.ulisboa.pt/records/u5RPHM-pu_yoOLXJF7BHrgJx47D827b0xHb3}{Lemos21} & \hyperref[auth:a887]{Alexandre Duarte {de Almeida} Lemos} & Solving scheduling problems under disruptions & \href{works/Lemos21.pdf}{Yes} & \cite{Lemos21} & 2021 & UNIVERSIDADE DE LISBOA INSTITUTO SUPERIOR TÉCNICO & 188 & 0 & 0 & \ref{b:Lemos21} & \ref{c:Lemos21}\\
\rowlabel{a:Zahout21}Zahout21 \href{https://hal.science/tel-03606639}{Zahout21} & \hyperref[auth:a902]{B. Zahout} & {Algorithmes exacts et approch{\'e}s pour l'ordonnancement des travaux multiressources {\`a} intervalles fixes dans des syst{\`e}mes distribu{\'e}s : approche monocrit{\`e}re et multiagent} & \href{works/Zahout21.pdf}{Yes} & \cite{Zahout21} & 2021 & {Universit{\'e} de Tours - LIFAT} & 185 & 0 & 0 & \ref{b:Zahout21} & \ref{c:Zahout21}\\
\rowlabel{a:Lunardi20}Lunardi20 \href{http://orbilu.uni.lu/handle/10993/43893}{Lunardi20} & \hyperref[auth:a501]{Willian Tessaro Lunardi} & A Real-World Flexible Job Shop Scheduling Problem With Sequencing Flexibility: Mathematical Programming, Constraint Programming, and Metaheuristics & \href{works/Lunardi20.pdf}{Yes} & \cite{Lunardi20} & 2020 & University of Luxembourg, Luxembourg City, Luxembourg & 181 & 0 & 0 & \ref{b:Lunardi20} & \ref{c:Lunardi20}\\
\rowlabel{a:Caballero19}Caballero19 \href{https://www.tesisenred.net/handle/10803/667963#page=1}{Caballero19} & \hyperref[auth:a102]{Jordi Coll Caballero} & Scheduling Through Logic-Based Tools & \href{works/Caballero19.pdf}{Yes} & \cite{Caballero19} & 2019 & Universitat de Girona, Spain & 194 & 0 & 0 & \ref{b:Caballero19} & \ref{c:Caballero19}\\
\rowlabel{a:German18}German18 \href{https://theses.hal.science/tel-01896325}{German18} & \hyperref[auth:a903]{G. German} & {Constraint programming for lot-sizing problems} & \href{works/German18.pdf}{Yes} & \cite{German18} & 2018 & {Universit{\'e} Grenoble Alpes} & 112 & 0 & 0 & \ref{b:German18} & \ref{c:German18}\\
\rowlabel{a:Dejemeppe16}Dejemeppe16 \href{https://hdl.handle.net/2078.1/178078}{Dejemeppe16} & \hyperref[auth:a207]{C. Dejemeppe} & Constraint programming algorithms and models for scheduling applications & \href{works/Dejemeppe16.pdf}{Yes} & \cite{Dejemeppe16} & 2016 & Catholic University of Louvain, Louvain-la-Neuve, Belgium & 274 & 0 & 0 & \ref{b:Dejemeppe16} & \ref{c:Dejemeppe16}\\
\rowlabel{a:Fahimi16}Fahimi16 \href{http://cp2014.a4cp.org/sites/default/files/hamed_fahimi_-_efficient_algorithms_to_solve_scheduling_problems_with_a_variety_of_optimization_criteria.pdf}{Fahimi16} & \hyperref[auth:a122]{H. Fahimi} & Efficient algorithms to solve scheduling problems with a variety of optimization criteria & \href{works/Fahimi16.pdf}{Yes} & \cite{Fahimi16} & 2016 & Universit{\'{e}} Laval, Quebec, Canada & 120 & 0 & 0 & \ref{b:Fahimi16} & \ref{c:Fahimi16}\\
\rowlabel{a:Froger16}Froger16 \href{https://theses.hal.science/tel-01440836}{Froger16} & \hyperref[auth:a901]{A. Froger} & {Maintenance scheduling in the electricity industry : a particular focus on a problem rising in the onshore wind industry} & \href{works/Froger16.pdf}{Yes} & \cite{Froger16} & 2016 & {Universit{\'e} d'Angers} & 181 & 0 & 0 & \ref{b:Froger16} & \ref{c:Froger16}\\
\rowlabel{a:Nattaf16}Nattaf16 \href{https://laas.hal.science/tel-01417288}{Nattaf16} & \hyperref[auth:a81]{M. Nattaf} & {Ordonnancement sous contraintes d'{\'e}nergie} & \href{works/Nattaf16.pdf}{Yes} & \cite{Nattaf16} & 2016 & {UPS Toulouse - Universit{\'e} Toulouse 3 Paul Sabatier} & 199 & 0 & 0 & \ref{b:Nattaf16} & \ref{c:Nattaf16}\\
\rowlabel{a:Derrien15}Derrien15 \href{https://tel.archives-ouvertes.fr/tel-01242789}{Derrien15} & \hyperref[auth:a225]{A. Derrien} & Ordonnancement cumulatif en programmation par contraintes : caract{\'{e}}risation {\'{e}}nerg{\'{e}}tique des raisonnements et solutions robustes. (Cumulative scheduling in constraint programming : energetic characterization of reasoning and robust solutions) & \href{works/Derrien15.pdf}{Yes} & \cite{Derrien15} & 2015 & {\'{E}}cole des mines de Nantes, France & 113 & 0 & 0 & \ref{b:Derrien15} & \ref{c:Derrien15}\\
\rowlabel{a:Siala15a}Siala15a \href{https://tel.archives-ouvertes.fr/tel-01164291}{Siala15a} & \hyperref[auth:a129]{M. Siala} & Search, propagation, and learning in sequencing and scheduling problems. (Recherche, propagation et apprentissage dans les probl{\`{e}}mes de s{\'{e}}quencement et d'ordonnancement) & \href{works/Siala15a.pdf}{Yes} & \cite{Siala15a} & 2015 & {INSA} Toulouse, France & 199 & 0 & 0 & \ref{b:Siala15a} & \ref{c:Siala15a}\\
\rowlabel{a:Kameugne14}Kameugne14 \href{http://cp2013.a4cp.org/sites/default/files/roger_kameugne_-_propagation_techniques_of_resource_constraint_for_cumulative_scheduling.pdf}{Kameugne14} & \hyperref[auth:a10]{R. Kameugne} & Techniques de Propagation de la Contrainte de Ressource en Ordonnancement Cumulatif & \href{works/Kameugne14.pdf}{Yes} & \cite{Kameugne14} & 2014 & University of Yaounde I, Cameroon & 139 & 0 & 0 & \ref{b:Kameugne14} & \ref{c:Kameugne14}\\
\rowlabel{a:Letort13}Letort13 \href{https://theses.hal.science/tel-00932215}{Letort13} & \hyperref[auth:a127]{A. Letort} & {Passage {\`a} l'{\'e}chelle pour les contraintes d'ordonnancement multi-ressources} & \href{works/Letort13.pdf}{Yes} & \cite{Letort13} & 2013 & {Ecole des Mines de Nantes} & 132 & 0 & 0 & \ref{b:Letort13} & \ref{c:Letort13}\\
\rowlabel{a:Clercq12}Clercq12 \href{https://theses.hal.science/tel-00794323}{Clercq12} & \hyperref[auth:a900]{Alexis de Clercq} & {Ordonnancement cumulatif avec d{\'e}passements de capacit{\'e} : Contrainte globale et d{\'e}compositions} & \href{works/Clercq12.pdf}{Yes} & \cite{Clercq12} & 2012 & {Ecole des Mines de Nantes} & 196 & 0 & 0 & \ref{b:Clercq12} & \ref{c:Clercq12}\\
\rowlabel{a:Malapert11}Malapert11 \href{https://tel.archives-ouvertes.fr/tel-00630122}{Malapert11} & \hyperref[auth:a82]{A. Malapert} & Techniques d'ordonnancement d'atelier et de fourn{\'{e}}es bas{\'{e}}es sur la programmation par contraintes. (Shop and batch scheduling with constraints) & \href{works/Malapert11.pdf}{Yes} & \cite{Malapert11} & 2011 & {\'{E}}cole des mines de Nantes, France & 194 & 0 & 0 & \ref{b:Malapert11} & \ref{c:Malapert11}\\
\rowlabel{a:Menana11}Menana11 \href{https://tel.archives-ouvertes.fr/tel-00785838}{Menana11} & \hyperref[auth:a622]{J. Menana} & Automates et programmation par contraintes pour la planification de personnel. (Automata and Constraint Programming for Personnel Scheduling Problems) & \href{works/Menana11.pdf}{Yes} & \cite{Menana11} & 2011 & University of Nantes, France & 148 & 0 & 0 & \ref{b:Menana11} & \ref{c:Menana11}\\
\rowlabel{a:Schutt11}Schutt11 \href{https://www.a4cp.org/sites/default/files/andreas_schutt_-_improving_scheduling_by_learning.pdf}{Schutt11} & \hyperref[auth:a124]{A. Schutt} & Improving Scheduling by Learning & \href{works/Schutt11.pdf}{Yes} & \cite{Schutt11} & 2011 & University of Melbourne, Australia & 209 & 0 & 0 & \ref{b:Schutt11} & \ref{c:Schutt11}\\
\rowlabel{a:Lombardi10}Lombardi10 \href{http://amsdottorato.unibo.it/2961/}{Lombardi10} & \hyperref[auth:a142]{M. Lombardi} & Hybrid Methods for Resource Allocation and Scheduling Problems in Deterministic and Stochastic Environments & \href{works/Lombardi10.pdf}{Yes} & \cite{Lombardi10} & 2010 & University of Bologna, Italy & 175 & 0 & 0 & \ref{b:Lombardi10} & \ref{c:Lombardi10}\\
\rowlabel{a:Malik08}Malik08 \href{https://hdl.handle.net/10012/3612}{Malik08} & \hyperref[auth:a647]{Abid M. Malik} & Constraint Programming Techniques for Optimal Instruction Scheduling & \href{works/Malik08.pdf}{Yes} & \cite{Malik08} & 2008 & University of Waterloo, Ontario, Canada & 151 & 0 & 0 & \ref{b:Malik08} & \ref{c:Malik08}\\
\rowlabel{a:Demassey03}Demassey03 \href{https://tel.archives-ouvertes.fr/tel-00293564}{Demassey03} & \hyperref[auth:a245]{S. Demassey} & M{\'{e}}thodes hybrides de programmation par contraintes et programmation lin{\'{e}}aire pour le probl{\`{e}}me d'ordonnancement de projet {\`{a}} contraintes de ressources. (Hybrid Constraint Programming-Integer Linear Programming approaches for the Resource-Constrained Project Scheduling Problem) & \href{works/Demassey03.pdf}{Yes} & \cite{Demassey03} & 2003 & University of Avignon, France & 148 & 0 & 0 & \ref{b:Demassey03} & \ref{c:Demassey03}\\
\rowlabel{a:Elkhyari03}Elkhyari03 \href{https://theses.hal.science/tel-00008377}{Elkhyari03} & \hyperref[auth:a294]{A. Elkhyari} & {Outils d'aide {\`a} la d{\'e}cision pour des probl{\`e}mes d'ordonnancement dynamiques} & \href{works/Elkhyari03.pdf}{Yes} & \cite{Elkhyari03} & 2003 & {Universit{\'e} de Nantes} & 333 & 0 & 0 & \ref{b:Elkhyari03} & \ref{c:Elkhyari03}\\
\rowlabel{a:Baptiste02}Baptiste02 \href{https://theses.hal.science/tel-00124998}{Baptiste02} & \hyperref[auth:a163]{P. Baptiste} & {R{\'e}sultats de complexit{\'e} et programmation par contraintes pour l'ordonnancement} & \href{works/Baptiste02.pdf}{Yes} & \cite{Baptiste02} & 2002 & {Universit{\'e} de Technologie de Compi{\`e}gne} & 237 & 0 & 0 & \ref{b:Baptiste02} & \ref{c:Baptiste02}\\
\rowlabel{a:Layfield02}Layfield02 \href{http://etheses.whiterose.ac.uk/1301/}{Layfield02} & \hyperref[auth:a680]{Colin J. Layfield} & A constraint programming pre-processor for duty scheduling & \href{works/Layfield02.pdf}{Yes} & \cite{Layfield02} & 2002 & University of Leeds, {UK} & 230 & 0 & 0 & \ref{b:Layfield02} & \ref{c:Layfield02}\\
\rowlabel{a:Beck99}Beck99 \href{https://librarysearch.library.utoronto.ca/permalink/01UTORONTO\_INST/14bjeso/alma991106162342106196}{Beck99} & \hyperref[auth:a89]{J. Christopher Beck} & Texture measurements as a basis for heuristic commitment techniques in constraint-directed scheduling & \href{works/Beck99.pdf}{Yes} & \cite{Beck99} & 1999 & University of Toronto, Canada & 418 & 0 & 0 & \ref{b:Beck99} & \ref{c:Beck99}\\
\end{longtable}
}



\clearpage
{\scriptsize
\begin{longtable}{>{\raggedright\arraybackslash}p{3cm}r>{\raggedright\arraybackslash}p{4cm}p{1.5cm}p{2cm}p{1.5cm}p{1.5cm}p{1.5cm}p{1.5cm}p{2cm}p{1.5cm}rr}
\rowcolor{white}\caption{Automatically Extracted THESIS Properties (Requires Local Copy)}\\ \toprule
\rowcolor{white}Work & Pages & Concepts & Classification & Constraints & \shortstack{Prog\\Languages} & \shortstack{CP\\Systems} & Areas & Industries & Benchmarks & Algorithm & a & c\\ \midrule\endhead
\bottomrule
\endfoot
\rowlabel{b:Astrand21}\href{../works/Astrand21.pdf}{Astrand21}~\cite{Astrand21} & 142 & distributed, one-machine scheduling, due-date, job-shop, flow-shop, resource, transportation, net present value, open-shop, machine, job, re-scheduling, stochastic, precedence, order, inventory, two-stage scheduling, tardiness, activity, setup-time, preempt, breakdown, release-date, planned maintenance, periodic, scheduling, make-span, completion-time, multi-objective, task, unavailability, sequence dependent setup & RCPSP, parallel machine, Resource-constrained Project Scheduling Problem, HFS, single machine, Partial Order Schedule & cumulative, alldifferent, cycle, circuit, disjunctive, Disjunctive constraint, Reified constraint & C++, Julia & Cplex, OPL, Gecode & satellite, drone, agriculture, semiconductor, robot & mineral industry, mining industry, maritime industry, potash industry, shipping industry & real-world, generated instance, real-life, benchmark & time-tabling, not-first, large neighborhood search, not-last, meta heuristic, neural network, reinforcement learning, edge-finding, simulated annealing, genetic algorithm, NEH & \ref{a:Astrand21} & n/a\\
\rowlabel{b:Baptiste02}\href{../works/Baptiste02.pdf}{Baptiste02}~\cite{Baptiste02} & 237 & re-scheduling, resource, release-date, scheduling, Pareto, preempt, flow-time, task, job-shop, preemptive, machine, activity, make-span, reactive scheduling, flow-shop, job, completion-time, precedence, distributed, inventory, no preempt, setup-time, due-date, single-machine scheduling, open-shop, tardiness, order, lateness, earliness, one-machine scheduling, cmax, sequence dependent setup & Open Shop Scheduling Problem, PJSSP, HFS, single machine, RCPSP, OSSP, parallel machine, Resource-constrained Project Scheduling Problem, JSSP & cumulative, circuit, disjunctive, Cardinality constraint, Disjunctive constraint, alternative constraint, table constraint, Arithmetic constraint & Prolog, C++ & Choco Solver, Claire, Ilog Solver, OPL, CHIP, ECLiPSe, Ilog Scheduler, Z3 & hoist &  & real-life, generated instance, benchmark & genetic algorithm, column generation, not-first, Lagrangian relaxation, energetic reasoning, not-last, simulated annealing, edge-finding & \ref{a:Baptiste02} & n/a\\
\rowlabel{b:Beck99}\href{../works/Beck99.pdf}{Beck99}~\cite{Beck99} & 418 & stochastic, due-date, multi-agent, order, distributed, preempt, scheduling, inventory, preemptive, machine, release-date, job-shop, task, tardiness, activity, transportation, stock level, precedence, make-span, re-scheduling, resource, job, producer/consumer & single machine & cumulative, Disjunctive constraint, circuit, disjunctive & Prolog, C++ & Ilog Solver, CHIP, Ilog Scheduler, OPL & telescope, robot, evacuation, medical &  & benchmark, real-world & column generation, not-last, machine learning, edge-finding, meta heuristic, not-first, simulated annealing, genetic algorithm & \ref{a:Beck99} & n/a\\
\rowlabel{b:Caballero19}\href{../works/Caballero19.pdf}{Caballero19}~\cite{Caballero19} & 194 & resource, machine, setup-time, preempt, periodic, task, order, activity, distributed, precedence, release-date, cmax, make-span, preemptive, scheduling, completion-time & psplib, Resource-constrained Project Scheduling Problem, RCPSP & alldifferent, circuit, Cardinality constraint, cycle, Arithmetic constraint, cumulative & C++ & SCIP, CHIP, Z3, CPO, Chuffed, MiniZinc, OPL &  &  & benchmark, real-life, instance generator & lazy clause generation, energetic reasoning, GRASP, time-tabling, meta heuristic, edge-finding, bi-partite matching, conflict-driven clause learning & \ref{a:Caballero19} & n/a\\
\rowlabel{b:Clercq12}\href{../works/Clercq12.pdf}{Clercq12}~\cite{Clercq12} & 196 & task, order, machine, job, manpower, activity, job-shop, make-span, resource, scheduling, due-date & psplib & Cumulatives constraint, alldifferent, cumulative, disjunctive, SoftCumulativeSum, circuit, SoftCumulative & Prolog & ECLiPSe, SICStus, Choco Solver, CHIP, Gecode & patient &  & benchmark & not-last, energetic reasoning, edge-finding, sweep, time-tabling, not-first & \ref{a:Clercq12} & n/a\\
\rowlabel{b:Dejemeppe16}\href{../works/Dejemeppe16.pdf}{Dejemeppe16}~\cite{Dejemeppe16} & 274 & make-span, sequence dependent setup, open-shop, order, job, activity, Pareto, continuous-process, machine, preempt, release-date, flow-shop, batch process, multi-objective, energy efficiency, tardiness, preemptive, scheduling, completion-time, re-scheduling, resource, setup-time, earliness, due-date, no-wait, task, stochastic, job-shop, lateness, precedence, bi-objective & PTC, psplib, single machine, Resource-constrained Project Scheduling Problem, RCPSP & disjunctive, cumulative, Element constraint, Reified constraint, Cumulatives constraint, alldifferent, GCC constraint, cycle, circuit, Disjunctive constraint, Cardinality constraint, Regular constraint &  & Ilog Solver, OPL, Gecode, CHIP, OR-Tools, CPO & medical, patient, super-computer, nurse, physician, robot, container terminal & paper industry & benchmark, instance generator, generated instance, industrial partner, random instance, real-world, bitbucket & Lagrangian relaxation, simulated annealing, not-first, meta heuristic, ant colony, not-last, particle swarm, sweep, edge-finding, genetic algorithm, large neighborhood search & \ref{a:Dejemeppe16} & n/a\\
\rowlabel{b:Demassey03}\href{../works/Demassey03.pdf}{Demassey03}~\cite{Demassey03} & 148 & machine, job, precedence, Benders Decomposition, release-date, stochastic, Logic-Based Benders Decomposition, job-shop, preemptive, single-machine scheduling, open-shop, activity, flow-shop, order, resource, scheduling, preempt, task & single machine, Resource-constrained Project Scheduling Problem, CuSP, psplib, RCPSP, TCSP & circuit, cumulative, disjunctive, cycle & C++ & Cplex, Claire, Ilog Solver &  &  & benchmark & not-last, meta heuristic, edge-finding, time-tabling, column generation, not-first, Lagrangian relaxation & \ref{a:Demassey03} & n/a\\
\rowlabel{b:Derrien15}\href{../works/Derrien15.pdf}{Derrien15}~\cite{Derrien15} & 113 & scheduling, precedence, order, make-span, task, activity, preemptive, job-shop, resource, machine, job, stochastic, preempt, open-shop & psplib, CuSP & Disjunctive constraint, cumulative, alldifferent, circuit, disjunctive &  & Claire, Choco Solver & robot &  & benchmark & edge-finding, sweep, time-tabling, energetic reasoning & \ref{a:Derrien15} & n/a\\
\rowlabel{b:Elkhyari03}\href{../works/Elkhyari03.pdf}{Elkhyari03}~\cite{Elkhyari03} & 333 & scheduling, task, job-shop, preemptive, machine, activity, make-span, flow-shop, cmax, open-shop, tardiness, order, preempt, breakdown, re-scheduling, reactive scheduling, resource, job, precedence, release-date, periodic & RCPSP, CuSP, parallel machine, Resource-constrained Project Scheduling Problem, Temporal Constraint Satisfaction Problem, single machine & cycle, cumulative, disjunctive &  & CPO, Choco Solver, Claire &  &  & benchmark, Roadef & meta heuristic, time-tabling, mat heuristic, genetic algorithm & \ref{a:Elkhyari03} & n/a\\
\rowlabel{b:Fahimi16}\href{../works/Fahimi16.pdf}{Fahimi16}~\cite{Fahimi16} & 120 & reactive scheduling, completion-time, flow-shop, precedence, batch process, setup-time, due-date, task, open-shop, preemptive, order, make-span, stochastic, machine, job, periodic, activity, resource, lateness, job-shop, Logic-Based Benders Decomposition, transportation, sequence dependent setup, preempt, tardiness, scheduling, Benders Decomposition & single machine, CuSP, parallel machine, RCPSP & Disjunctive constraint, Cardinality constraint, Cumulatives constraint, alldifferent, cycle, AllDiff constraint, cumulative, alternative constraint, disjunctive & Java, C++ & Choco Solver, CHIP, Ilog Scheduler, Gecode & aircraft &  & benchmark, random instance, real-world, Roadef & time-tabling, not-first, not-last, energetic reasoning, edge-finding, max-flow, sweep & \ref{a:Fahimi16} & n/a\\
\rowlabel{b:Froger16}\href{../works/Froger16.pdf}{Froger16}~\cite{Froger16} & 181 & breakdown, preempt, distributed, resource, inventory, scheduling, multi-objective, Benders Decomposition, reactive scheduling, batch process, re-scheduling, task, preemptive, order, sustainability, stochastic, completion-time, machine, job, manpower, Pareto, release-date, Logic-Based Benders Decomposition, unavailability, transportation & single machine, CuSP, TMS & disjunctive, cycle, cumulative & Java & Gurobi, OZ, Choco Solver & train schedule, maintenance scheduling, satellite, energy-price, offshore & power industry, electricity industry, energy industry, wind industry & benchmark, real-life, real-world, industrial partner, instance generator, Roadef, generated instance & ant colony, particle swarm, genetic algorithm, neural network, large neighborhood search, Lagrangian relaxation, simulated annealing, column generation, max-flow, mat heuristic, meta heuristic & \ref{a:Froger16} & n/a\\
\rowlabel{b:German18}\href{../works/German18.pdf}{German18}~\cite{German18} & 112 & stock level, setup-time, job, task, activity, stochastic, earliness, machine, resource, job-shop, cmax, order, inventory, scheduling &  & Disjunctive constraint, Cardinality constraint, bin-packing, Balance constraint, cumulative, Among constraint, disjunctive & Prolog & Z3, SICStus, OPL, Choco Solver, Cplex & nurse &  & real-world, benchmark, real-life, CSPlib, generated instance &  & \ref{a:German18} & n/a\\
\rowlabel{b:Godet21a}\href{../works/Godet21a.pdf}{Godet21a}~\cite{Godet21a} & 168 & open-shop, release-date, make-span, transportation, machine, distributed, periodic, resource, lateness, job-shop, flow-shop, precedence, cmax, preempt, due-date, preemptive, order, scheduling, Benders Decomposition, completion-time, job, task, activity & single machine, RCPSP, parallel machine, JSSP, PMSP, Resource-constrained Project Scheduling Problem, psplib & AllDiff constraint, bin-packing, GeneralizedAllDiffPrec, disjunctive, BinPacking constraint, cumulative, AllDiffPrec constraint, Disjunctive constraint, Element constraint, alldifferent, Cardinality constraint, cycle &  & OR-Tools, OPL, Claire, Choco Solver, Chuffed, MiniZinc, CHIP & satellite, robot, railway, maintenance scheduling & electricity industry & real-life, github, generated instance, benchmark, random instance & sweep, lazy clause generation, meta heuristic, time-tabling, edge-finding & \ref{a:Godet21a} & n/a\\
\rowlabel{b:Groleaz21}\href{../works/Groleaz21.pdf}{Groleaz21}~\cite{Groleaz21} & 153 & inventory, tardiness, activity, setup-time, preempt, breakdown, release-date, earliness, periodic, single-machine scheduling, scheduling, make-span, completion-time, task, online scheduling, bi-objective, reactive scheduling, preemptive, sequence dependent setup, distributed, due-date, job-shop, flow-shop, resource, transportation, cmax, open-shop, machine, job, lateness, re-scheduling, stochastic, precedence, order & Resource-constrained Project Scheduling Problem, Open Shop Scheduling Problem, single machine, GCSP, RCPSP, OSP, parallel machine & circuit, disjunctive, Disjunctive constraint, span constraint, cumulative, cycle, noOverlap & Julia, Java & Choco Solver, Z3, OPL, OR-Tools, Gurobi, CPO, Gecode, SCIP, Cplex & dairy, robot, automotive, business process & food industry, agrifood industry, dairy industry & benchmark, real-life & mat heuristic, evolutionary computing, memetic algorithm, meta heuristic, swarm intelligence, neural network, column generation, edge-finding, machine learning, simulated annealing, genetic algorithm, not-first, ant colony, large neighborhood search, not-last & \ref{a:Groleaz21} & n/a\\
\rowlabel{b:Kameugne14}\href{../works/Kameugne14.pdf}{Kameugne14}~\cite{Kameugne14} & 139 & resource, job, scheduling, task, job-shop, preemptive, machine, make-span, flow-shop, completion-time, order, preempt & RCPSP, CuSP, parallel machine, Resource-constrained Project Scheduling Problem, psplib & circuit, Disjunctive constraint, Cumulatives constraint, Balance constraint, cumulative, disjunctive & Java, Prolog, C++ & Choco Solver, Claire, Gecode, CHIP, ECLiPSe, SICStus, Cplex, Mistral &  &  & Roadef & not-last, edge-finder, energetic reasoning, time-tabling, edge-finding, not-first & \ref{a:Kameugne14} & n/a\\
\rowlabel{b:Layfield02}\href{../works/Layfield02.pdf}{Layfield02}~\cite{Layfield02} & 230 &  &  &  & C  & OPL, OZ, Z3 &  &  &  &  & \ref{a:Layfield02} & n/a\\
\rowlabel{b:Lemos21}\href{../works/Lemos21.pdf}{Lemos21}~\cite{Lemos21} & 188 & transportation, precedence, job-shop, multi-objective, machine, re-scheduling, distributed, unavailability, multi-agent, bi-objective, task, job, stochastic, order, periodic, energy efficiency, Pareto, resource, scheduling & RCPSP & cycle, alldifferent, cumulative, Cardinality constraint & Java, C++, Python & OPL, Gurobi, Cplex & surgery, meeting scheduling, COVID, train schedule, high school timetabling, medical, crew-scheduling, railway & railway industry & real-world, github, real-life, benchmark, Roadef & machine learning, simulated annealing, large neighborhood search, meta heuristic, GRASP, reinforcement learning, genetic algorithm, conflict-driven clause learning, evolutionary computing, time-tabling & \ref{a:Lemos21} & n/a\\
\rowlabel{b:Letort13}\href{../works/Letort13.pdf}{Letort13}~\cite{Letort13} & 132 & machine, resource, job-shop, precedence, cmax, order, scheduling, job, task & psplib & bin-packing, alldifferent, cumulative, geost, Cumulatives constraint, disjunctive & Java, Prolog & SICStus, Claire, Choco Solver, CHIP & steel mill, datacenter &  & Roadef, CSPlib, benchmark & energetic reasoning, edge-finding, sweep, meta heuristic, not-first, time-tabling, large neighborhood search, not-last & \ref{a:Letort13} & n/a\\
\rowlabel{b:Lombardi10}\href{../works/Lombardi10.pdf}{Lombardi10}~\cite{Lombardi10} & 175 & re-scheduling, make-span, job, precedence, Benders Decomposition, release-date, periodic, stochastic, distributed, Logic-Based Benders Decomposition, setup-time, job-shop, preemptive, due-date, activity, completion-time, order, net present value, inventory, multi-objective, tardiness, resource, energy efficiency, scheduling, preempt, task, machine & single machine, SCC, Resource-constrained Project Scheduling Problem, CTW, RCPSP, TCSP & Disjunctive constraint, cycle, Balance constraint, AllDiff constraint, cumulative, disjunctive, table constraint, span constraint, bin-packing, circuit & C  & OPL, Cplex, Ilog Solver & aircraft, pipeline, semiconductor, business process, medical, automotive & semiconductor industry & generated instance, benchmark, real-world, instance generator, real-life & not-last, simulated annealing, lazy clause generation, meta heuristic, sweep, large neighborhood search, edge-finder, edge-finding, energetic reasoning, genetic algorithm, time-tabling, column generation, not-first, Lagrangian relaxation, machine learning & \ref{a:Lombardi10} & n/a\\
\rowlabel{b:Lunardi20}\href{../works/Lunardi20.pdf}{Lunardi20}~\cite{Lunardi20} & 181 & activity, setup-time, breakdown, Pareto, release-date, reactive scheduling, unavailability, scheduling, make-span, task, cmax, bi-objective, machine, job, lateness, re-scheduling, stochastic, no preempt, due-date, job-shop, batch process, preempt, flow-shop, resource, transportation, open-shop, precedence, order, completion-time, multi-objective, tardiness & FJS, parallel machine, single machine & cycle, noOverlap, endBeforeStart, alldifferent, disjunctive & Python & CPO, OPL, Cplex & robot, high performance computing & printing industry, glass industry & industrial partner, instance generator, benchmark, random instance, github, supplementary material, real-world, generated instance, real-life & mat heuristic, memetic algorithm, meta heuristic, machine learning, simulated annealing, genetic algorithm, particle swarm, ant colony, swarm intelligence, neural network, reinforcement learning & \ref{a:Lunardi20} & n/a\\
\rowlabel{b:Malapert11}\href{../works/Malapert11.pdf}{Malapert11}~\cite{Malapert11} & 194 & tardiness, order, lateness, preempt, cmax, multi-objective, batch process, transportation, resource, scheduling, flow-time, task, job-shop, preemptive, machine, activity, make-span, no-wait, flow-shop, job, completion-time, precedence, planned maintenance, inventory, setup-time, due-date, open-shop & Open Shop Scheduling Problem, single machine & cumulative, diffn, circuit, disjunctive, geost, cycle, alldifferent, Element constraint, bin-packing, Disjunctive constraint, Cumulatives constraint & Prolog, C++, Java & Mistral, Choco Solver, Claire, Gecode, ECLiPSe, SICStus, Cplex, OPL, CHIP, Ilog Scheduler & rectangle-packing, robot, semiconductor, maintenance scheduling, patient &  & real-world, industrial partner, generated instance, benchmark & edge-finding, particle swarm, genetic algorithm, column generation, not-first, ant colony, energetic reasoning, not-last, time-tabling, meta heuristic, sweep & \ref{a:Malapert11} & n/a\\
\rowlabel{b:Malik08}\href{../works/Malik08.pdf}{Malik08}~\cite{Malik08} & 151 & order, machine, completion-time, activity, distributed, precedence, breakdown, task, job, resource, make-span, cyclic scheduling, scheduling &  & alldifferent, Cardinality constraint, cycle &  &  & pipeline &  & real-life, benchmark & edge-finding, machine learning & \ref{a:Malik08} & n/a\\
\rowlabel{b:Menana11}\href{../works/Menana11.pdf}{Menana11}~\cite{Menana11} & 148 & machine, task, manpower, activity, distributed, resource, multi-objective, cyclic scheduling, precedence, scheduling &  & Regular constraint, alldifferent, Cardinality constraint & Prolog & Z3, CHIP, OPL, Claire, Choco Solver & nurse &  & Roadef, github, benchmark & large neighborhood search, Lagrangian relaxation, meta heuristic, time-tabling, genetic algorithm, column generation & \ref{a:Menana11} & n/a\\
\rowlabel{b:Nattaf16}\href{../works/Nattaf16.pdf}{Nattaf16}~\cite{Nattaf16} & 199 & preemptive, order, tardiness, inventory, scheduling, flow-shop, setup-time, job, task, make-span, machine, resource, job-shop, bi-objective, cmax, preempt, due-date & RCPSP, CECSP, Resource-constrained Project Scheduling Problem, psplib, single machine, CuSP, parallel machine & alldifferent, cumulative, disjunctive & C++ & Claire, Cplex & maintenance scheduling, robot & process industry & Roadef & genetic algorithm, column generation, energetic reasoning, edge-finding, sweep, mat heuristic & \ref{a:Nattaf16} & n/a\\
\rowlabel{b:Schutt11}\href{../works/Schutt11.pdf}{Schutt11}~\cite{Schutt11} & 209 & resource, job-shop, precedence, cmax, preempt, preemptive, order, tardiness, scheduling, completion-time, machine, setup-time, job, periodic, task, activity, open-shop, one-machine scheduling, release-date, make-span & RCPSP, Resource-constrained Project Scheduling Problem, Open Shop Scheduling Problem, psplib & disjunctive, Arithmetic constraint, UTVPI constraint, cumulative, circuit, bin-packing, Reified constraint, Disjunctive constraint, Element constraint, alldifferent, cycle, geost & Prolog, C++ & CHIP, SICStus, Ilog Scheduler, SCIP, ECLiPSe, Ilog Solver & rectangle-packing & carpet industry & benchmark, real-world, industrial instance, instance generator & sweep, ant colony, lazy clause generation, meta heuristic, edge-finder, time-tabling, not-first, simulated annealing, energetic reasoning, edge-finding, not-last & \ref{a:Schutt11} & n/a\\
\rowlabel{b:Siala15a}\href{../works/Siala15a.pdf}{Siala15a}~\cite{Siala15a} & 199 & job-shop, precedence, earliness, cmax, sequence dependent setup, due-date, order, tardiness, scheduling, setup-time, task, activity, open-shop, make-span, machine, job, periodic, resource & RCPSP, OSP, single machine, TMS & AtMostSeq, table constraint, Balance constraint, cumulative, circuit, Among constraint, AmongSeq constraint, disjunctive, Atmost constraint, Regular constraint, Disjunctive constraint, GCC constraint, Cardinality constraint, CardPath, MultiAtMostSeqCard, AtMostSeqCard, Reified constraint, alldifferent, cycle &  & CHIP, Ilog Solver, Mistral, OPL, Claire & automotive, rectangle-packing &  & github, benchmark, random instance, Roadef, real-world, CSPlib & conflict-driven clause learning, evolutionary computing, lazy clause generation, time-tabling, large neighborhood search, edge-finding, ant colony, GRASP, swarm intelligence & \ref{a:Siala15a} & n/a\\
\rowlabel{b:Zahout21}\href{../works/Zahout21.pdf}{Zahout21}~\cite{Zahout21} & 185 & completion-time, machine, job, activity, Pareto, online scheduling, release-date, make-span, multi-agent, distributed, resource, energy efficiency, multi-objective, job-shop, flow-shop, precedence, bi-objective, preempt, due-date, re-scheduling, task, preemptive, scheduling & CuSP, parallel machine, RCPSP, SCC, TCSP, single machine & cycle, cumulative, circuit, bin-packing &  & CPO, Cplex, Claire & datacenter, business process, high performance computing, crew-scheduling, satellite &  & benchmark & meta heuristic, reinforcement learning, GRASP, genetic algorithm, column generation & \ref{a:Zahout21} & n/a\\
\end{longtable}
}




\clearpage
\subsection{InBook from bibtex}
{\scriptsize
\begin{longtable}{>{\raggedright\arraybackslash}p{2.5cm}>{\raggedright\arraybackslash}p{4.5cm}>{\raggedright\arraybackslash}p{6.0cm}p{1.0cm}rr>{\raggedright\arraybackslash}p{2.0cm}r>{\raggedright\arraybackslash}p{1cm}p{1cm}p{1cm}p{1cm}}
\rowcolor{white}\caption{INBOOK (Total 46)}\\ \toprule
\rowcolor{white}\shortstack{Key\\Source} & Authors & Title (Colored by Open Access)& \shortstack{Details\\LC} & Cite & Year & \shortstack{Conference\\/Journal\\/School} & Pages & Relevance &\shortstack{Cites\\OC XR\\SC} & \shortstack{Refs\\OC\\XR} & \shortstack{Links\\Cites\\Refs}\\ \midrule\endhead
\bottomrule
\endfoot
\index{Marcolini2022}\rowlabel{a:Marcolini2022}Marcolini2022 \href{http://dx.doi.org/10.1016/b978-0-323-85159-6.50083-x}{Marcolini2022} & \hyperref[auth:a2042]{L. D. Marcolini}, \hyperref[auth:a586]{F. M. Novara}, \hyperref[auth:a587]{G. P. Henning} & Production scheduling in multiproduct multistage semicontinuous processes. A constraint programming approach & \cellcolor{red!30}\hyperref[detail:Marcolini2022]{Details} No & \cite{Marcolini2022} & 2022 & Computer Aided Chemical Engineering & null & \noindent{}\textbf{1.00} \textbf{1.00} n/a & 0 0 0 & 4 5 & 1 0 1\\
\index{Sitek2018}\rowlabel{a:Sitek2018}Sitek2018 \href{http://dx.doi.org/10.1007/978-3-319-90287-6_8}{Sitek2018} & \hyperref[auth:a1474]{P. Sitek}, \hyperref[auth:a534]{J. Wikarek} & An MP/CP-Based Hybrid Approach to Optimization of the Resource-Constrained Scheduling Problems & \hyperref[detail:Sitek2018]{Details} \href{../works/Sitek2018.pdf}{Yes} & \cite{Sitek2018} & 2018 & Transactions on Computational Collective Intelligence XXIX & 19 & \noindent{}\textbf{1.50} \textbf{1.50} \textbf{10.08} & 0 0 0 & 20 24 & 5 0 5\\
\index{Bgler2016}\rowlabel{a:Bgler2016}Bgler2016 \href{http://dx.doi.org/10.4018/978-1-4666-9619-8.ch006}{Bgler2016} & \hyperref[auth:a1542]{M. Bügler}, \hyperref[auth:a1543]{A. Borrmann} & \cellcolor{gold!20}Simulation Based Construction Project Schedule Optimization \hyperref[abs:Bgler2016]{Abstract} & \cellcolor{red!30}\hyperref[detail:Bgler2016]{Details} No & \cite{Bgler2016} & 2016 & Civil and Environmental Engineering & null & \noindent{}\textcolor{black!50}{0.00} \textcolor{black!50}{0.00} n/a & 0 0 0 & 56 71 & 7 0 7\\
\index{Bgler2016a}\rowlabel{a:Bgler2016a}Bgler2016a \href{http://dx.doi.org/10.4018/978-1-4666-8823-0.ch016}{Bgler2016a} & \hyperref[auth:a1542]{M. Bügler}, \hyperref[auth:a1543]{A. Borrmann} & Simulation Based Construction Project Schedule Optimization \hyperref[abs:Bgler2016a]{Abstract} & \cellcolor{red!30}\hyperref[detail:Bgler2016a]{Details} No & \cite{Bgler2016a} & 2016 & Advances in Systems Analysis, Software Engineering, and High Performance Computing & null & \noindent{}\textcolor{black!50}{0.00} \textcolor{black!50}{0.00} n/a & 0 0 1 & 58 71 & 7 0 7\\
\index{Novara2015}\rowlabel{a:Novara2015}Novara2015 \href{http://dx.doi.org/10.1016/b978-0-444-63576-1.50032-7}{Novara2015} & \hyperref[auth:a586]{F. M. Novara}, \hyperref[auth:a587]{G. P. Henning} & A Hybrid CP/MILP Approach for Big Size Scheduling Problems of Multiproduct, Multistage Batch Plants & \cellcolor{red!30}\hyperref[detail:Novara2015]{Details} No & \cite{Novara2015} & 2015 & 12th International Symposium on Process Systems Engineering and 25th European Symposium on Computer Aided Process Engineering & null & \noindent{}\textbf{1.00} \textbf{1.00} n/a & 3 3 4 & 3 4 & 2 0 2\\
\index{SchuttFSW15}\rowlabel{a:SchuttFSW15}SchuttFSW15 \href{https://doi.org/10.1007/978-3-319-05443-8_7}{SchuttFSW15} & \hyperref[auth:a124]{A. Schutt}, \hyperref[auth:a154]{T. Feydy}, \hyperref[auth:a125]{P. J. Stuckey}, \hyperref[auth:a117]{M. G. Wallace} & A Satisfiability Solving Approach \hyperref[abs:SchuttFSW15]{Abstract} & \cellcolor{red!30}\hyperref[detail:SchuttFSW15]{Details} No & \cite{SchuttFSW15} & 2015 & Handbook on Project Management and Scheduling Vol.1 & 26 & \noindent{}\textcolor{black!50}{0.00} \textbf{1.50} n/a & 3 4 6 & 28 41 & 24 3 21\\
\index{CestaOPS14}\rowlabel{a:CestaOPS14}CestaOPS14 \href{http://dx.doi.org/10.1007/978-3-319-05443-8_6}{CestaOPS14} & \hyperref[auth:a284]{A. Cesta}, \hyperref[auth:a282]{A. Oddi}, \hyperref[auth:a283]{N. Policella}, \hyperref[auth:a298]{S. F. Smith} & A Precedence Constraint Posting Approach & \cellcolor{red!30}\hyperref[detail:CestaOPS14]{Details} No & \cite{CestaOPS14} & 2014 & Handbook on Project Management and Scheduling Vol.1 & 21 & \noindent{}\textcolor{black!50}{0.00} \textcolor{black!50}{0.00} n/a & 2 2 3 & 17 40 & 11 0 11\\
\index{Gaspero2014}\rowlabel{a:Gaspero2014}Gaspero2014 \href{http://dx.doi.org/10.1007/978-3-319-07644-7_1}{Gaspero2014} & \hyperref[auth:a2040]{L. D. Gaspero}, \hyperref[auth:a2041]{T. Urli} & A CP/LNS Approach for Multi-day Homecare Scheduling Problems & \hyperref[detail:Gaspero2014]{Details} \href{../works/Gaspero2014.pdf}{Yes} & \cite{Gaspero2014} & 2014 & Hybrid Metaheuristics & 15 & \noindent{}\textbf{1.00} \textbf{1.00} \textbf{1.92} & 5 5 13 & 12 18 & 2 0 2\\
\index{GuSSWC14}\rowlabel{a:GuSSWC14}GuSSWC14 \href{http://dx.doi.org/10.1007/978-3-319-05443-8_14}{GuSSWC14} & \hyperref[auth:a336]{H. Gu}, \hyperref[auth:a124]{A. Schutt}, \hyperref[auth:a125]{P. J. Stuckey}, \hyperref[auth:a117]{M. G. Wallace}, \hyperref[auth:a343]{G. Chu} & Exact and Heuristic Methods for the Resource-Constrained Net Present Value Problem & \cellcolor{red!30}\hyperref[detail:GuSSWC14]{Details} No & \cite{GuSSWC14} & 2014 & Handbook on Project Management and Scheduling Vol.1 & 20 & \noindent{}\textcolor{black!50}{0.00} \textcolor{black!50}{0.00} n/a & 5 6 7 & 35 39 & 13 1 12\\
\index{Moukrim2014}\rowlabel{a:Moukrim2014}Moukrim2014 \href{http://dx.doi.org/10.1007/978-3-319-12631-9_6}{Moukrim2014} & \hyperref[auth:a1169]{A. Moukrim}, \hyperref[auth:a788]{A. Quilliot}, \hyperref[auth:a1698]{H. Toussaint} & Branch and Price for Preemptive and Non Preemptive RCPSP Based on Interval Orders on Precedence Graphs & \cellcolor{red!30}\hyperref[detail:Moukrim2014]{Details} No & \cite{Moukrim2014} & 2014 & Recent Advances in Computational Optimization & null & \noindent{}\textcolor{black!50}{0.00} \textcolor{black!50}{0.00} n/a & 1 1 1 & 15 25 & 5 0 5\\
\index{Ortiz-Bayliss2014}\rowlabel{a:Ortiz-Bayliss2014}Ortiz-Bayliss2014 \href{http://dx.doi.org/10.1007/978-3-319-01692-4_25}{Ortiz-Bayliss2014} & \hyperref[auth:a1778]{J. C. Ortiz-Bayliss}, \hyperref[auth:a1606]{H. Terashima-Marín}, \hyperref[auth:a1779]{S. E. Conant-Pablos} & \cellcolor{green!10}Branching Schemes and Variable Ordering Heuristics for Constraint Satisfaction Problems: Is There Something to Learn? & \cellcolor{red!30}\hyperref[detail:Ortiz-Bayliss2014]{Details} No & \cite{Ortiz-Bayliss2014} & 2014 & Nature Inspired Cooperative Strategies for Optimization (NICSO 2013) & null & \noindent{}0.50 0.50 n/a & 0 0 0 & 24 40 & 5 0 5\\
\index{Austrin2013}\rowlabel{a:Austrin2013}Austrin2013 \href{http://dx.doi.org/10.1007/978-3-642-40328-6_3}{Austrin2013} & \hyperref[auth:a1926]{P. Austrin}, \hyperref[auth:a1927]{R. Manokaran}, \hyperref[auth:a1928]{C. Wenner} & \cellcolor{green!10}On the NP-Hardness of Approximating Ordering Constraint Satisfaction Problems & \hyperref[detail:Austrin2013]{Details} \href{../works/Austrin2013.pdf}{Yes} & \cite{Austrin2013} & 2013 & Approximation, Randomization, and Combinatorial Optimization. Algorithms and Techniques & 16 & \noindent{}0.50 0.50 0.35 & 1 1 5 & 21 23 & 1 1 0\\
\index{Guimarans2013}\rowlabel{a:Guimarans2013}Guimarans2013 \href{http://dx.doi.org/10.4018/978-1-4666-2461-0.ch007}{Guimarans2013} & \hyperref[auth:a1837]{D. Guimarans}, \hyperref[auth:a1838]{R. Herrero}, \hyperref[auth:a1839]{J. J. Ramos}, \hyperref[auth:a1840]{S. Padrón} & Solving Vehicle Routing Problems Using Constraint Programming and Lagrangean Relaxation in a Metaheuristics Framework \hyperref[abs:Guimarans2013]{Abstract} & \cellcolor{red!30}\hyperref[detail:Guimarans2013]{Details} No & \cite{Guimarans2013} & 2013 & Management Innovations for Intelligent Supply Chains & null & \noindent{}\textcolor{black!50}{0.00} 0.50 n/a & 1 1 0 & 21 32 & 1 1 0\\
\index{Novara2013}\rowlabel{a:Novara2013}Novara2013 \href{http://dx.doi.org/10.1016/b978-0-444-63234-0.50099-3}{Novara2013} & \hyperref[auth:a586]{F. M. Novara}, \hyperref[auth:a523]{J. M. Novas}, \hyperref[auth:a587]{G. P. Henning} & A comprehensive CP approach for the scheduling of resource-constrained multiproduct multistage batch plants & \cellcolor{red!30}\hyperref[detail:Novara2013]{Details} No & \cite{Novara2013} & 2013 & Computer Aided Chemical Engineering & null & \noindent{}\textbf{1.50} \textbf{1.50} n/a & 3 3 6 & 4 5 & 2 1 1\\
\index{Talbi2013a}\rowlabel{a:Talbi2013a}Talbi2013a \href{http://dx.doi.org/10.1007/978-3-642-30671-6_1}{Talbi2013a} & \hyperref[auth:a1657]{E.-G. Talbi} & A Unified Taxonomy of Hybrid Metaheuristics with Mathematical Programming, Constraint Programming and Machine Learning & \cellcolor{red!30}\hyperref[detail:Talbi2013a]{Details} No & \cite{Talbi2013a} & 2013 & Hybrid Metaheuristics & null & \noindent{}0.50 0.50 n/a & 15 15 32 & 107 183 & 7 0 7\\
\index{Laborie2011}\rowlabel{a:Laborie2011}Laborie2011 \href{http://dx.doi.org/10.1007/978-3-642-23592-4_6}{Laborie2011} & \hyperref[auth:a118]{P. Laborie}, \hyperref[auth:a1673]{J. Rogerie}, \hyperref[auth:a120]{P. Shaw}, \hyperref[auth:a1674]{P. Vilím}, \hyperref[auth:a1675]{F. Katai} & Interval-Based Language for Modeling Scheduling Problems: An Extension to Constraint Programming & \cellcolor{red!30}\hyperref[detail:Laborie2011]{Details} No & \cite{Laborie2011} & 2011 & Algebraic Modeling Systems & null & \noindent{}\textbf{1.00} \textbf{1.00} n/a & 2 2 0 & 4 20 & 3 1 2\\
\index{Milano11}\rowlabel{a:Milano11}Milano11 \href{http://dx.doi.org/10.1002/9780470400531.eorms0473}{Milano11} & \hyperref[auth:a143]{M. Milano} & Constraint Programming Links with Math Programming & \cellcolor{red!30}\hyperref[detail:Milano11]{Details} No & \cite{Milano11} & 2011 & Wiley Encyclopedia of Operations Research and Management Science & null & \noindent{}\textcolor{black!50}{0.00} \textcolor{black!50}{0.00} n/a & 0 0 0 & 28 65 & 15 0 15\\
\index{Triska2011}\rowlabel{a:Triska2011}Triska2011 \href{http://dx.doi.org/10.1007/978-3-642-21332-8_12}{Triska2011} & \hyperref[auth:a1843]{M. Triska}, \hyperref[auth:a45]{N. Musliu} & A Constraint Programming Application for Rotating Workforce Scheduling & \cellcolor{red!30}\hyperref[detail:Triska2011]{Details} No & \cite{Triska2011} & 2011 & Developing Concepts in Applied Intelligence & null & \noindent{}\textbf{1.00} \textbf{1.00} n/a & 3 3 5 & 8 11 & 2 1 1\\
\index{CastroGR10}\rowlabel{a:CastroGR10}CastroGR10 \href{http://dx.doi.org/10.1007/978-1-4419-1644-0_4}{CastroGR10} & \hyperref[auth:a890]{P. M. Castro}, \hyperref[auth:a382]{I. E. Grossmann}, \hyperref[auth:a326]{L.-M. Rousseau} & Decomposition Techniques for Hybrid MILP/CP Models applied to Scheduling and Routing Problems & \cellcolor{red!30}\hyperref[detail:CastroGR10]{Details} No & \cite{CastroGR10} & 2010 & Hybrid Optimization & 33 & \noindent{}\textbf{1.00} \textbf{1.00} n/a & 0 0 4 & 67 88 & 30 0 30\\
\index{Hooker10}\rowlabel{a:Hooker10}Hooker10 \href{http://dx.doi.org/10.1007/978-1-4419-1644-0_2}{Hooker10} & \hyperref[auth:a160]{J. N. Hooker} & Hybrid Modeling & \cellcolor{red!30}\hyperref[detail:Hooker10]{Details} No & \cite{Hooker10} & 2010 & Hybrid Optimization & 52 & \noindent{}\textcolor{black!50}{0.00} \textcolor{black!50}{0.00} n/a & 9 9 12 & 39 65 & 12 0 12\\
\index{GongLMW09}\rowlabel{a:GongLMW09}GongLMW09 \href{http://dx.doi.org/10.1007/978-0-387-88617-6_11}{GongLMW09} & \hyperref[auth:a1232]{J. Gong}, \hyperref[auth:a1233]{E. E. Lee}, \hyperref[auth:a1234]{J. E. Mitchell}, \hyperref[auth:a1235]{W. A. Wallace} & Logic-based MultiObjective Optimization for Restoration Planning & \cellcolor{red!30}\hyperref[detail:GongLMW09]{Details} No & \cite{GongLMW09} & 2009 & Optimization and Logistics Challenges in the Enterprise & 20 & \noindent{}\textcolor{black!50}{0.00} \textcolor{black!50}{0.00} n/a & 14 14 30 & 13 20 & 10 1 9\\
\index{AggounMV08}\rowlabel{a:AggounMV08}AggounMV08 \href{http://dx.doi.org/10.1007/978-0-387-74759-0_396}{AggounMV08} & \hyperref[auth:a724]{A. Aggoun}, \hyperref[auth:a381]{C. T. Maravelias}, \hyperref[auth:a906]{A. Vazacopoulos} & Mixed Integer Programming/Constraint Programming Hybrid Methods & \cellcolor{red!30}\hyperref[detail:AggounMV08]{Details} No & \cite{AggounMV08} & 2008 & Encyclopedia of Optimization & 7 & \noindent{}\textcolor{black!50}{0.00} \textcolor{black!50}{0.00} n/a & 0 0 0 & 34 53 & 19 0 19\\
\index{2007}\rowlabel{a:2007}2007 \href{http://dx.doi.org/10.1007/978-3-540-32220-7_13}{2007} &  & Constraint Programming and Disjunctive Scheduling & \cellcolor{red!30}\hyperref[detail:2007]{Details} No & \cite{2007} & 2007 & International Handbook on Information Systems & null & \noindent{}\textbf{1.00} \textbf{1.00} n/a & 0 0 0 & 37 63 & 11 0 11\\
\index{Hooker06a}\rowlabel{a:Hooker06a}Hooker06a \href{http://dx.doi.org/10.1016/s1574-6526(06)80019-2}{Hooker06a} & \hyperref[auth:a160]{J. N. Hooker} & Operations Research Methods in Constraint Programming & \cellcolor{red!30}\hyperref[detail:Hooker06a]{Details} No & \cite{Hooker06a} & 2006 & Foundations of Artificial Intelligence & 44 & \noindent{}\textcolor{black!50}{0.00} \textcolor{black!50}{0.00} n/a & 11 11 13 & 44 133 & 16 5 11\\
\index{NeronABCDD06}\rowlabel{a:NeronABCDD06}NeronABCDD06 \href{http://dx.doi.org/10.1007/978-0-387-33768-5_7}{NeronABCDD06} & \hyperref[auth:a898]{E. Néron}, \hyperref[auth:a6]{C. Artigues}, \hyperref[auth:a162]{P. Baptiste}, \hyperref[auth:a844]{J. Carlier}, \hyperref[auth:a899]{J. Damay}, \hyperref[auth:a243]{S. Demassey}, \hyperref[auth:a118]{P. Laborie} & Lower Bounds for Resource Constrained Project Scheduling Problem & \cellcolor{red!30}\hyperref[detail:NeronABCDD06]{Details} No & \cite{NeronABCDD06} & 2006 & Perspectives in Modern Project Scheduling & 38 & \noindent{}\textcolor{black!50}{0.00} \textcolor{black!50}{0.00} n/a & 3 3 0 & 34 49 & 15 1 14\\
\index{Xing2006}\rowlabel{a:Xing2006}Xing2006 \href{http://dx.doi.org/10.1007/11760191_135}{Xing2006} & \hyperref[auth:a1983]{L.-N. Xing}, \hyperref[auth:a1984]{Y.-W. Chen}, \hyperref[auth:a1985]{X.-S. Shen} & A Constraint Satisfaction Adaptive Neural Network with Dynamic Model for Job-Shop Scheduling Problem & \hyperref[detail:Xing2006]{Details} \href{../works/Xing2006.pdf}{Yes} & \cite{Xing2006} & 2006 & Advances in Neural Networks - ISNN 2006 & 6 & \noindent{}\textbf{2.00} \textbf{2.00} \textcolor{black!50}{0.14} & 3 3 11 & 3 5 & 2 0 2\\
\index{Zeballos2006}\rowlabel{a:Zeballos2006}Zeballos2006 \href{http://dx.doi.org/10.1016/s1570-7946(06)80335-4}{Zeballos2006} & \hyperref[auth:a620]{L. J. Zeballos}, \hyperref[auth:a587]{G. P. Henning} & A CP method for the scheduling of multiproduct continuous plants with resource constraints & \cellcolor{red!30}\hyperref[detail:Zeballos2006]{Details} No & \cite{Zeballos2006} & 2006 & 16th European Symposium on Computer Aided Process Engineering and 9th International Symposium on Process Systems Engineering & null & \noindent{}\textbf{1.50} \textbf{1.50} n/a & 1 1 1 & 4 7 & 3 1 2\\
\index{Bartak2005}\rowlabel{a:Bartak2005}Bartak2005 \href{http://dx.doi.org/10.4018/978-1-59140-450-7.ch010}{Bartak2005} & \hyperref[auth:a1480]{R. Bartak} & Constraint Satisfaction for Planning and Scheduling \hyperref[abs:Bartak2005]{Abstract} & \cellcolor{red!30}\hyperref[detail:Bartak2005]{Details} No & \cite{Bartak2005} & 2005 & Intelligent Techniques for Planning & null & \noindent{}\textbf{1.00} \textbf{3.75} n/a & 3 3 0 & 0 0 & 2 2 0\\
\index{Vazacopoulos2005}\rowlabel{a:Vazacopoulos2005}Vazacopoulos2005 \href{http://dx.doi.org/10.1007/0-387-26281-4_12}{Vazacopoulos2005} & \hyperref[auth:a906]{A. Vazacopoulos}, \hyperref[auth:a1560]{N. Verma} & Hybrid MIP-CP Techniques to Solve a Multi-Machine Assignment and Scheduling Problem in Xpress-CP & \cellcolor{red!30}\hyperref[detail:Vazacopoulos2005]{Details} No & \cite{Vazacopoulos2005} & 2005 & Applied Optimization & null & \noindent{}\textbf{1.50} \textbf{1.50} n/a & 3 3 0 & 8 16 & 6 1 5\\
\index{WolfS05a}\rowlabel{a:WolfS05a}WolfS05a \href{http://dx.doi.org/10.1007/11415763_12}{WolfS05a} & \hyperref[auth:a51]{A. Wolf}, \hyperref[auth:a710]{H. Schlenker} & Realising the Alternative Resources Constraint & \hyperref[detail:WolfS05a]{Details} \href{../works/WolfS05a.pdf}{Yes} & \cite{WolfS05a} & 2005 & Applications of Declarative Programming and Knowledge Management & 15 & \noindent{}\textcolor{black!50}{0.00} \textcolor{black!50}{0.00} \textbf{1.44} & 5 5 5 & 6 8 & 8 3 5\\
\index{AggounV04}\rowlabel{a:AggounV04}AggounV04 \href{http://dx.doi.org/10.1007/978-3-540-24734-0_15}{AggounV04} & \hyperref[auth:a724]{A. Aggoun}, \hyperref[auth:a906]{A. Vazacopoulos} & Solving Sports Scheduling and Timetabling Problems with Constraint Programming & \cellcolor{red!30}\hyperref[detail:AggounV04]{Details} No & \cite{AggounV04} & 2004 & Economics, Management and Optimization in Sports & 22 & \noindent{}\textbf{1.00} \textbf{1.00} n/a & 7 7 0 & 4 18 & 7 4 3\\
\index{AjiliW04}\rowlabel{a:AjiliW04}AjiliW04 \href{http://dx.doi.org/10.1007/978-1-4419-8917-8_6}{AjiliW04} & \hyperref[auth:a948]{F. Ajili}, \hyperref[auth:a117]{M. G. Wallace} & Hybrid Problem Solving in ECLiPSe & \cellcolor{red!30}\hyperref[detail:AjiliW04]{Details} No & \cite{AjiliW04} & 2004 & Constraint and Integer Programming & 38 & \noindent{}\textcolor{black!50}{0.00} \textcolor{black!50}{0.00} n/a & 4 4 0 & 24 42 & 13 1 12\\
\index{DannaP04}\rowlabel{a:DannaP04}DannaP04 \href{http://dx.doi.org/10.1007/978-1-4419-8917-8_2}{DannaP04} & \hyperref[auth:a287]{E. Danna}, \hyperref[auth:a163]{C. L. Pape} & Two Generic Schemes for Efficient and Robust Cooperative Algorithms & \cellcolor{red!30}\hyperref[detail:DannaP04]{Details} No & \cite{DannaP04} & 2004 & Constraints and Integer Programming & 25 & \noindent{}\textcolor{black!50}{0.00} \textcolor{black!50}{0.00} n/a & 2 2 0 & 34 63 & 18 0 18\\
\index{DomdorfPH03}\rowlabel{a:DomdorfPH03}DomdorfPH03 \href{http://dx.doi.org/10.1007/978-3-642-18965-4_31}{DomdorfPH03} & \hyperref[auth:a958]{U. Domdorf}, \hyperref[auth:a437]{E. Pesch}, \hyperref[auth:a959]{T. P. Huy} & Machine Learning by Schedule Decomposition — Prospects for an Integration of AI and OR Techniques for Job Shop Scheduling & \cellcolor{red!30}\hyperref[detail:DomdorfPH03]{Details} No & \cite{DomdorfPH03} & 2003 & Advances in Evolutionary Computing & 26 & \noindent{}\textcolor{black!50}{0.00} \textcolor{black!50}{0.00} n/a & 0 0 0 & 57 96 & 14 0 14\\
\index{Roe2003}\rowlabel{a:Roe2003}Roe2003 \href{http://dx.doi.org/10.1016/s1570-7946(03)80608-9}{Roe2003} & \hyperref[auth:a1240]{B. Roe}, \hyperref[auth:a1242]{N. Shah}, \hyperref[auth:a1241]{L. G. Papageorgiou} & A hybrid CLP and MILP approach to batch process scheduling & \cellcolor{red!30}\hyperref[detail:Roe2003]{Details} No & \cite{Roe2003} & 2003 & Computer Aided Chemical Engineering & null & \noindent{}\textbf{1.00} \textbf{1.00} n/a & 0 1 2 & 2 5 & 1 0 1\\
\index{Timpe2003}\rowlabel{a:Timpe2003}Timpe2003 \href{http://dx.doi.org/10.1007/978-3-662-05607-3_5}{Timpe2003} & \hyperref[auth:a672]{C. Timpe} & Solving planning and scheduling problems with combined integer and constraint programming & \cellcolor{red!30}\hyperref[detail:Timpe2003]{Details} No & \cite{Timpe2003} & 2003 & Advanced Planning and Scheduling Solutions in Process Industry & null & \noindent{}\textbf{1.00} \textbf{1.00} n/a & 2 2 0 & 9 15 & 4 0 4\\
\index{Baptiste2001}\rowlabel{a:Baptiste2001}Baptiste2001 \href{http://dx.doi.org/10.1007/978-1-4615-1479-4_2}{Baptiste2001} & \hyperref[auth:a162]{P. Baptiste}, \hyperref[auth:a163]{C. L. Pape}, \hyperref[auth:a655]{W. Nuijten} & Propagation of the One-Machine Resource Constraint & \cellcolor{red!30}\hyperref[detail:Baptiste2001]{Details} No & \cite{Baptiste2001} & 2001 & International Series in Operations Research \  Management Science & null & \noindent{}0.50 0.50 n/a & 1 1 0 & 0 0 & 1 1 0\\
\index{Galipienso2001}\rowlabel{a:Galipienso2001}Galipienso2001 \href{http://dx.doi.org/10.1007/3-540-45517-5_63}{Galipienso2001} & \hyperref[auth:a1875]{M. I. A. Galipienso}, \hyperref[auth:a1876]{F. B. Sanchís} & A Mixed Closure-CSP Method to Solve Scheduling Problems & \hyperref[detail:Galipienso2001]{Details} \href{../works/Galipienso2001.pdf}{Yes} & \cite{Galipienso2001} & 2001 & Engineering of Intelligent Systems & 12 & \noindent{}\textbf{1.00} \textbf{1.00} \textbf{1.86} & 1 1 3 & 7 16 & 3 0 3\\
\index{Harjunkoski2001}\rowlabel{a:Harjunkoski2001}Harjunkoski2001 \href{http://dx.doi.org/10.1016/s1570-7946(01)80140-1}{Harjunkoski2001} & \hyperref[auth:a870]{I. Harjunkoski}, \hyperref[auth:a382]{I. E. Grossmann} & Combined MILP-constraint programming approach for the optimal scheduling of multistage batch processes & \cellcolor{red!30}\hyperref[detail:Harjunkoski2001]{Details} No & \cite{Harjunkoski2001} & 2001 & Computer Aided Chemical Engineering & null & \noindent{}\textbf{1.00} \textbf{1.00} n/a & 2 2 2 & 3 9 & 2 2 0\\
\index{Rgin2001}\rowlabel{a:Rgin2001}Rgin2001 \href{http://dx.doi.org/10.1090/dimacs/057/07}{Rgin2001} & \hyperref[auth:a1419]{J.-C. Régin} & Minimization of the number of breaks in sports scheduling problems using constraint programming & \cellcolor{red!30}\hyperref[detail:Rgin2001]{Details} No & \cite{Rgin2001} & 2001 & DIMACS Series in Discrete Mathematics and Theoretical Computer Science & 16 & \noindent{}\textbf{1.00} \textbf{1.00} n/a & 28 29 0 & 0 0 & 11 11 0\\
\index{DorndorfHP99}\rowlabel{a:DorndorfHP99}DorndorfHP99 \href{http://dx.doi.org/10.1007/978-1-4615-5533-9_10}{DorndorfHP99} & \hyperref[auth:a903]{U. Dorndorf}, \hyperref[auth:a904]{T. P. Huy}, \hyperref[auth:a437]{E. Pesch} & A Survey of Interval Capacity Consistency Tests for Time- and Resource-Constrained Scheduling & \cellcolor{red!30}\hyperref[detail:DorndorfHP99]{Details} No & \cite{DorndorfHP99} & 1999 & Project Scheduling & 26 & \noindent{}\textcolor{black!50}{0.00} \textcolor{black!50}{0.00} n/a & 18 18 0 & 20 40 & 18 5 13\\
\index{Jaffar1998}\rowlabel{a:Jaffar1998}Jaffar1998 \href{http://dx.doi.org/10.1093/oso/9780198537922.003.0012}{Jaffar1998} & \hyperref[auth:a1066]{J. Jaffar}, \hyperref[auth:a1067]{M. J. Maher} & Constraint Logic Programming: A Survey \hyperref[abs:Jaffar1998]{Abstract} & \cellcolor{red!30}\hyperref[detail:Jaffar1998]{Details} No & \cite{Jaffar1998} & 1998 & Handbook of Logic in Artificial Intelligence and Logic Programming: Volume 5: Logic Programming & null & \noindent{}\textcolor{black!50}{0.00} \textbf{1.00} n/a & 3 3 0 & 0 0 & 1 1 0\\
\index{Larrosa1998}\rowlabel{a:Larrosa1998}Larrosa1998 \href{http://dx.doi.org/10.1007/3-540-64574-8_390}{Larrosa1998} & \hyperref[auth:a1791]{J. Larrosa}, \hyperref[auth:a1792]{P. Meseguer} & \cellcolor{green!10}Generic CSP techniques for the job-shop problem & \cellcolor{red!30}\hyperref[detail:Larrosa1998]{Details} No & \cite{Larrosa1998} & 1998 & Tasks and Methods in Applied Artificial Intelligence & null & \noindent{}\textbf{1.00} \textbf{1.00} n/a & 2 2 4 & 3 13 & 2 0 2\\
\index{Mesghouni1997}\rowlabel{a:Mesghouni1997}Mesghouni1997 \href{http://dx.doi.org/10.1007/978-0-387-35086-8_12}{Mesghouni1997} & \hyperref[auth:a1906]{K. Mesghouni}, \hyperref[auth:a1457]{P. Pesin}, \hyperref[auth:a1907]{S. Hammadi}, \hyperref[auth:a1458]{C. Tahon}, \hyperref[auth:a1908]{P. Borne} & Genetic Algorithms — Constraint Logic Programming. Hybrid Method for Job Shop Scheduling & \cellcolor{red!30}\hyperref[detail:Mesghouni1997]{Details} No & \cite{Mesghouni1997} & 1997 & Re-engineering for Sustainable Industrial Production & null & \noindent{}\textbf{2.00} \textbf{2.00} n/a & 2 2 0 & 1 11 & 1 1 0\\
\index{Caseau1996}\rowlabel{a:Caseau1996}Caseau1996 \href{http://dx.doi.org/10.1007/3-540-61576-8_79}{Caseau1996} & \hyperref[auth:a301]{Y. Caseau}, \hyperref[auth:a1511]{F. Laburthe} & Improving branch and bound for Jobshop scheduling with constraint propagation & \cellcolor{red!30}\hyperref[detail:Caseau1996]{Details} No & \cite{Caseau1996} & 1996 & Combinatorics and Computer Science & null & \noindent{}\textbf{3.00} \textbf{3.00} n/a & 11 11 10 & 12 23 & 11 6 5\\
\index{Breitinger1994}\rowlabel{a:Breitinger1994}Breitinger1994 \href{http://dx.doi.org/10.1007/3-540-58402-1_20}{Breitinger1994} & \hyperref[auth:a694]{S. Breitinger}, \hyperref[auth:a695]{H. C. R. Lock} & Improving search for job-shop scheduling with CLP(FD) & \cellcolor{red!30}\hyperref[detail:Breitinger1994]{Details} No & \cite{Breitinger1994} & 1994 & Programming Language Implementation and Logic Programming & null & \noindent{}\textbf{2.00} \textbf{2.00} n/a & 3 3 3 & 6 17 & 2 1 1\\
\end{longtable}
}



\clearpage
\subsection{InCollection from bibtex}
{\scriptsize
\begin{longtable}{>{\raggedright\arraybackslash}p{3cm}>{\raggedright\arraybackslash}p{6cm}>{\raggedright\arraybackslash}p{6.5cm}rrrp{2.5cm}rrrrr}
\rowcolor{white}\caption{Works from bibtex (Total 7)}\\ \toprule
\rowcolor{white}Key & Authors & Title & LC & Cite & Year & \shortstack{Conference\\/Journal} & Pages & \shortstack{Nr\\Cites} & \shortstack{Nr\\Refs} & b & c \\ \midrule\endhead
\bottomrule
\endfoot
\rowlabel{a:BlazewiczEP19}BlazewiczEP19 \href{https://ideas.repec.org/h/spr/ihichp/978-3-319-99849-7_16.html}{BlazewiczEP19} & \hyperref[auth:a774]{J. Blazewicz}, \hyperref[auth:a775]{Klaus H. Ecker}, \hyperref[auth:a443]{E. Pesch}, \hyperref[auth:a776]{G. Schmidt}, \hyperref[auth:a777]{M. Sterna}, \hyperref[auth:a778]{J. Weglarz} & {Constraint Programming and Disjunctive Scheduling} & No & \cite{BlazewiczEP19} & 2019 & {Handbook on Scheduling} & 62 & 38 & 0 & No & \ref{c:BlazewiczEP19}\\
\rowlabel{a:Hooker19}Hooker19 \href{https://ideas.repec.org/h/spr/spochp/978-3-030-22788-3_1.html}{Hooker19} & \hyperref[auth:a161]{John N. Hooker} & {Logic-Based Benders Decomposition for Large-Scale Optimization} & No & \cite{Hooker19} & 2019 & {Large Scale Optimization in Supply Chains and Smart Manufacturing} & 26 & 8 & 0 & No & \ref{c:Hooker19}\\
\rowlabel{a:HurleyOS16}HurleyOS16 \href{https://doi.org/10.1007/978-3-319-50137-6\_15}{HurleyOS16} & \hyperref[auth:a902]{B. Hurley}, \hyperref[auth:a16]{B. O'Sullivan}, \hyperref[auth:a17]{H. Simonis} & {ICON} Loop Energy Show Case & \href{works/HurleyOS16.pdf}{Yes} & \cite{HurleyOS16} & 2016 & Data Mining and Constraint Programming - Foundations of a Cross-Disciplinary Approach & 14 & 0 & 16 & \ref{b:HurleyOS16} & \ref{c:HurleyOS16}\\
\rowlabel{a:Bartak14}Bartak14 \href{}{Bartak14} & \hyperref[auth:a152]{R. Bart{\'{a}}k} & Planning and Scheduling & No & \cite{Bartak14} & 2014 & Computing Handbook, Third Edition: Computer Science and Software Engineering & null & 0 & 0 & No & \ref{c:Bartak14}\\
\rowlabel{a:BaptisteLPN06}BaptisteLPN06 \href{https://doi.org/10.1016/S1574-6526(06)80026-X}{BaptisteLPN06} & \hyperref[auth:a163]{P. Baptiste}, \hyperref[auth:a118]{P. Laborie}, \hyperref[auth:a164]{Claude Le Pape}, \hyperref[auth:a666]{W. Nuijten} & Constraint-Based Scheduling and Planning & No & \cite{BaptisteLPN06} & 2006 & Handbook of Constraint Programming & 39 & 30 & 25 & No & \ref{c:BaptisteLPN06}\\
\rowlabel{a:KanetAG04}KanetAG04 \href{http://www.crcnetbase.com/doi/abs/10.1201/9780203489802.ch47}{KanetAG04} & \hyperref[auth:a672]{John J. Kanet}, \hyperref[auth:a673]{S. Ahire}, \hyperref[auth:a674]{Michael F. Gorman} & Constraint Programming for Scheduling & No & \cite{KanetAG04} & 2004 & Handbook of Scheduling - Algorithms, Models, and Performance Analysis & null & 0 & 0 & No & \ref{c:KanetAG04}\\
\rowlabel{a:BreitingerL95}BreitingerL95 \href{}{BreitingerL95} & \hyperref[auth:a705]{S. Breitinger}, \hyperref[auth:a706]{Hendrik C. R. Lock} & Using Constraint Logic Programming for Industrial Scheduling Problems & No & \cite{BreitingerL95} & 1995 & Logic Programming: Formal Methods and Practical Applications, Studies in Computer Science and Artificial Intelligence & 27 & 0 & 0 & No & \ref{c:BreitingerL95}\\
\end{longtable}
}



\clearpage
{\scriptsize
\begin{longtable}{>{\raggedright\arraybackslash}p{3cm}r>{\raggedright\arraybackslash}p{4cm}p{1.5cm}p{2cm}p{1.5cm}p{1.5cm}p{1.5cm}p{1.5cm}p{2cm}p{1.5cm}rr}
\rowcolor{white}\caption{Automatically Extracted INCOLLECTION Properties (Requires Local Copy)}\\ \toprule
\rowcolor{white}Work & Pages & Concepts & Classification & Constraints & \shortstack{Prog\\Languages} & \shortstack{CP\\Systems} & Areas & Industries & Benchmarks & Algorithm & a & c\\ \midrule\endhead
\bottomrule
\endfoot
\rowlabel{b:Hooker19}\href{../works/Hooker19.pdf}{Hooker19}~\cite{Hooker19} & 26 & constraint satisfaction, machine, job, task, one-machine scheduling, release-date, transportation, stochastic, constraint programming, job-shop, Logic-Based Benders Decomposition, constraint optimization, inventory, activity, CP, make-span, explanation, single-machine scheduling, constraint logic programming, distributed, resource, sequence dependent setup, due-date, order, tardiness, CLP, scheduling, multi-objective, Benders Decomposition & single machine, parallel machine & disjunctive, cycle, cumulative, circuit &  & OPL, MiniZinc & container terminal, yard crane, operating room, railway, maintenance scheduling, satellite, torpedo, patient, aircraft &  & industrial instance & mat heuristic, large neighborhood search, time-tabling, column generation, quadratic programming, meta heuristic & \ref{a:Hooker19} & n/a\\
\rowlabel{b:HurleyOS16}\href{../works/HurleyOS16.pdf}{HurleyOS16}~\cite{HurleyOS16} & 14 & resource, constraint programming, CP, order, re-scheduling, scheduling, task, machine, distributed &  & cumulative &  &  & datacentre, energy-price, super-computer, high performance computing &  & real-world, benchmark & machine learning & \ref{a:HurleyOS16} & n/a\\
\rowlabel{b:KanetAG04}\href{../works/KanetAG04.pdf}{KanetAG04}~\cite{KanetAG04} & 22 & scheduling, precedence, make-span, tardiness, task, COP, earliness, CSP, due-date, CP, machine, inventory, constraint programming, transportation, CLP, order, completion-time, activity, job-shop, resource, constraint logic programming, job, constraint satisfaction, setup-time, single-machine scheduling & single machine, parallel machine & alldifferent, Disjunctive constraint, disjunctive &  & ECLiPSe, Cplex, Ilog Solver, OPL & high school timetabling, patient &  &  & time-tabling, meta heuristic & \ref{a:KanetAG04} & n/a\\
\end{longtable}
}



\clearpage
\section{Papers by Conference Series}

\subsection{AAAI}

\index{AAAI}
{\scriptsize
\begin{longtable}{>{\raggedright\arraybackslash}p{3cm}>{\raggedright\arraybackslash}p{4.5cm}>{\raggedright\arraybackslash}p{6.0cm}rrrp{2.5cm}rp{1cm}p{1cm}rr}
\rowcolor{white}\caption{Papers in Conference Series AAAI (Total 51) (Total 51)}\\ \toprule
\rowcolor{white}\shortstack{Key\\Source} & Authors & Title (Colored by Open Access)& LC & Cite & Year & \shortstack{Conference\\/Journal\\/School} & Pages & \shortstack{Cites\\OC XR\\SC} & \shortstack{Refs\\OC\\XR} & b & c \\ \midrule\endhead
\bottomrule
\endfoot
FalqueALM24 \href{https://doi.org/10.1609/aaai.v38i21.30308}{FalqueALM24} & \hyperref[auth:a1369]{T. Falque}, \hyperref[auth:a1370]{G. Audemard}, \hyperref[auth:a213]{C. Lecoutre}, \hyperref[auth:a1371]{B. Mazure} & \cellcolor{gold!20}Check-In Desk Scheduling Optimisation at {CDG} International Airport & \href{../works/FalqueALM24.pdf}{Yes} & \cite{FalqueALM24} & 2024 & AAAI 2024 & 9 & 0 0 0 & 0 0 & \ref{b:FalqueALM24} & \ref{c:FalqueALM24}\\
LiLZDZW24 \href{https://doi.org/10.1609/aaai.v38i18.29998}{LiLZDZW24} & \hyperref[auth:a1363]{L. Li}, \hyperref[auth:a1364]{S. Liang}, \hyperref[auth:a1365]{Z. Zhu}, \hyperref[auth:a1366]{C. Ding}, \hyperref[auth:a1367]{H. Zha}, \hyperref[auth:a1368]{B. Wu} & \cellcolor{gold!20}Learning to Optimize Permutation Flow Shop Scheduling via Graph-Based Imitation Learning & \href{../works/LiLZDZW24.pdf}{Yes} & \cite{LiLZDZW24} & 2024 & AAAI 2024 & 9 & 0 0 0 & 0 0 & \ref{b:LiLZDZW24} & \ref{c:LiLZDZW24}\\
KotaryFH22 \href{https://doi.org/10.1609/aaai.v36i7.20685}{KotaryFH22} & \hyperref[auth:a1361]{J. Kotary}, \hyperref[auth:a1362]{F. Fioretto}, \hyperref[auth:a148]{P. V. Hentenryck} & Fast Approximations for Job Shop Scheduling: {A} Lagrangian Dual Deep Learning Method & \href{../works/KotaryFH22.pdf}{Yes} & \cite{KotaryFH22} & 2022 & AAAI 2022 & 8 & 0 2 0 & 0 0 & \ref{b:KotaryFH22} & \ref{c:KotaryFH22}\\
GeibingerMM21 \href{https://doi.org/10.1609/aaai.v35i7.16789}{GeibingerMM21} & \hyperref[auth:a77]{T. Geibinger}, \hyperref[auth:a80]{F. Mischek}, \hyperref[auth:a45]{N. Musliu} & Constraint Logic Programming for Real-World Test Laboratory Scheduling & \href{../works/GeibingerMM21.pdf}{Yes} & \cite{GeibingerMM21} & 2021 & AAAI 2021 & 9 & 0 1 0 & 0 0 & \ref{b:GeibingerMM21} & \ref{c:GeibingerMM21}\\
KletzanderMH21 \href{https://doi.org/10.1609/aaai.v35i13.17408}{KletzanderMH21} & \hyperref[auth:a78]{L. Kletzander}, \hyperref[auth:a45]{N. Musliu}, \hyperref[auth:a148]{P. V. Hentenryck} & Branch and Price for Bus Driver Scheduling with Complex Break Constraints & \href{../works/KletzanderMH21.pdf}{Yes} & \cite{KletzanderMH21} & 2021 & AAAI 2021 & 9 & 2 2 0 & 0 0 & \ref{b:KletzanderMH21} & n/a\\
GodetLHS20 \href{https://doi.org/10.1609/aaai.v34i02.5510}{GodetLHS20} & \hyperref[auth:a471]{A. Godet}, \hyperref[auth:a244]{X. Lorca}, \hyperref[auth:a1]{E. Hebrard}, \hyperref[auth:a126]{G. Simonin} & Using Approximation within Constraint Programming to Solve the Parallel Machine Scheduling Problem with Additional Unit Resources & \href{../works/GodetLHS20.pdf}{Yes} & \cite{GodetLHS20} & 2020 & AAAI 2020 & 8 & 1 1 0 & 0 0 & \ref{b:GodetLHS20} & \ref{c:GodetLHS20}\\
AgussurjaKL18 \href{https://doi.org/10.1609/aaai.v32i1.12086}{AgussurjaKL18} & \hyperref[auth:a1359]{L. Agussurja}, \hyperref[auth:a1360]{A. Kumar}, \hyperref[auth:a364]{H. C. Lau} & Resource-Constrained Scheduling for Maritime Traffic Management & \href{../works/AgussurjaKL18.pdf}{Yes} & \cite{AgussurjaKL18} & 2018 & AAAI 2018 & 8 & 4 4 0 & 0 0 & \ref{b:AgussurjaKL18} & n/a\\
KinsellaS0OS16 \href{https://doi.org/10.1609/aaai.v30i2.19079}{KinsellaS0OS16} & \hyperref[auth:a1357]{A. Kinsella}, \hyperref[auth:a1358]{A. F. Smeaton}, \hyperref[auth:a885]{B. Hurley}, \hyperref[auth:a16]{B. O'Sullivan}, \hyperref[auth:a17]{H. Simonis} & Optimizing Energy Costs in a Zinc and Lead Mine & \href{../works/KinsellaS0OS16.pdf}{Yes} & \cite{KinsellaS0OS16} & 2016 & AAAI 2016 & 6 & 1 2 0 & 0 0 & \ref{b:KinsellaS0OS16} & n/a\\
TranWDRFOVB16 \href{http://www.aaai.org/ocs/index.php/WS/AAAIW16/paper/view/12664}{TranWDRFOVB16} & \hyperref[auth:a799]{T. T. Tran}, \hyperref[auth:a808]{Z. Wang}, \hyperref[auth:a809]{M. Do}, \hyperref[auth:a810]{E. G. Rieffel}, \hyperref[auth:a379]{J. Frank}, \hyperref[auth:a811]{B. O'Gorman}, \hyperref[auth:a812]{D. Venturelli}, \hyperref[auth:a89]{J. C. Beck} & Explorations of Quantum-Classical Approaches to Scheduling a Mars Lander Activity Problem & \href{../works/TranWDRFOVB16.pdf}{Yes} & \cite{TranWDRFOVB16} & 2016 & AAAI 2016 & 9 & 0 0 0 & 0 0 & \ref{b:TranWDRFOVB16} & n/a\\
LimBTBB15a \href{https://doi.org/10.1609/aaai.v29i1.9236}{LimBTBB15a} & \hyperref[auth:a207]{B. Lim}, \hyperref[auth:a210]{M. van den Briel}, \hyperref[auth:a209]{S. Thi{\'{e}}baux}, \hyperref[auth:a1356]{S. Backhaus}, \hyperref[auth:a1355]{R. Bent} & HVAC-Aware Occupancy Scheduling & \href{../works/LimBTBB15a.pdf}{Yes} & \cite{LimBTBB15a} & 2015 & AAAI 2015 & 8 & 3 3 0 & 0 0 & \ref{b:LimBTBB15a} & n/a\\
ChunS14 \href{https://doi.org/10.1609/aaai.v28i2.19013}{ChunS14} & \hyperref[auth:a1323]{A. H. W. Chun}, \hyperref[auth:a1373]{T. Y. T. Suen} & Engineering Works Scheduling for Hong Kong's Rail Network & \href{../works/ChunS14.pdf}{Yes} & \cite{ChunS14} & 2014 & AAAI 2014 & 8 & 3 3 0 & 0 0 & \ref{b:ChunS14} & n/a\\
LudwigKRBMS14 \href{https://doi.org/10.1609/aaai.v28i2.19030}{LudwigKRBMS14} & \hyperref[auth:a1349]{J. Ludwig}, \hyperref[auth:a1350]{A. Kalton}, \hyperref[auth:a1351]{R. Richards}, \hyperref[auth:a1352]{B. Bautsch}, \hyperref[auth:a1353]{C. Markusic}, \hyperref[auth:a1354]{J. Schumacher} & A Schedule Optimization Tool for Destructive and Non-Destructive Vehicle Tests & \href{../works/LudwigKRBMS14.pdf}{Yes} & \cite{LudwigKRBMS14} & 2014 & AAAI 2014 & 6 & 1 1 0 & 0 0 & \ref{b:LudwigKRBMS14} & n/a\\
DoRZ08 \href{http://www.aaai.org/Library/AAAI/2008/aaai08-253.php}{DoRZ08} & \hyperref[auth:a1346]{M. B. Do}, \hyperref[auth:a1347]{W. Ruml}, \hyperref[auth:a1348]{R. Zhou} & On-line Planning and Scheduling: An Application to Controlling Modular Printers & \href{../works/DoRZ08.pdf}{Yes} & \cite{DoRZ08} & 2008 & AAAI 2008 & 5 & 0 0 0 & 0 0 & \ref{b:DoRZ08} & n/a\\
SchausD08 \href{http://www.aaai.org/Library/AAAI/2008/aaai08-058.php}{SchausD08} & \hyperref[auth:a147]{P. Schaus}, \hyperref[auth:a151]{Y. Deville} & A Global Constraint for Bin-Packing with Precedences: Application to the Assembly Line Balancing Problem & \href{../works/SchausD08.pdf}{Yes} & \cite{SchausD08} & 2008 & AAAI 2008 & 6 & 0 0 0 & 0 0 & \ref{b:SchausD08} & n/a\\
ElhouraniDM07 \href{http://www.aaai.org/Library/AAAI/2007/aaai07-213.php}{ElhouraniDM07} & \hyperref[auth:a1343]{T. Elhourani}, \hyperref[auth:a1344]{N. Denny}, \hyperref[auth:a1345]{M. M. Marefat} & A Distributed Constraint Optimization Solution to the {P2P} Video Streaming Problem & \href{../works/ElhouraniDM07.pdf}{Yes} & \cite{ElhouraniDM07} & 2007 & AAAI 2007 & 6 & 0 0 0 & 0 0 & \ref{b:ElhouraniDM07} & n/a\\
HoeveGSL07 \href{http://www.aaai.org/Library/AAAI/2007/aaai07-291.php}{HoeveGSL07} & \hyperref[auth:a206]{W.-J. van Hoeve}, \hyperref[auth:a642]{C. P. Gomes}, \hyperref[auth:a643]{B. Selman}, \hyperref[auth:a142]{M. Lombardi} & Optimal Multi-Agent Scheduling with Constraint Programming & \href{../works/HoeveGSL07.pdf}{Yes} & \cite{HoeveGSL07} & 2007 & AAAI 2007 & 6 & 0 0 0 & 0 0 & \ref{b:HoeveGSL07} & n/a\\
GomesHS06 \href{http://www.aaai.org/Library/Symposia/Spring/2006/ss06-04-024.php}{GomesHS06} & \hyperref[auth:a642]{C. P. Gomes}, \hyperref[auth:a206]{W.-J. van Hoeve}, \hyperref[auth:a643]{B. Selman} & Constraint Programming for Distributed Planning and Scheduling & \href{../works/GomesHS06.pdf}{Yes} & \cite{GomesHS06} & 2006 & AAAI 2006 & 2 & 0 0 0 & 0 0 & \ref{b:GomesHS06} & n/a\\
Johnston05 \href{}{Johnston05} & \hyperref[auth:a1342]{B. J. Clement}, \hyperref[auth:a1212]{M. D. Johnston} & The Deep Space Network Scheduling Problem & \href{../works/Johnston05.pdf}{Yes} & \cite{Johnston05} & 2005 & AAAI 2005 & 7 & 0 0 0 & 0 0 & \ref{b:Johnston05} & n/a\\
MoffittPP05 \href{http://www.aaai.org/Library/AAAI/2005/aaai05-188.php}{MoffittPP05} & \hyperref[auth:a771]{M. D. Moffitt}, \hyperref[auth:a772]{B. Peintner}, \hyperref[auth:a773]{M. E. Pollack} & Augmenting Disjunctive Temporal Problems with Finite-Domain Constraints & \href{../works/MoffittPP05.pdf}{Yes} & \cite{MoffittPP05} & 2005 & AAAI 2005 & 6 & 0 0 0 & 0 0 & \ref{b:MoffittPP05} & n/a\\
PolicellaWSO05 \href{http://www.aaai.org/Library/AAAI/2005/aaai05-190.php}{PolicellaWSO05} & \hyperref[auth:a283]{N. Policella}, \hyperref[auth:a1341]{X. Wang}, \hyperref[auth:a298]{S. F. Smith}, \hyperref[auth:a282]{A. Oddi} & Exploiting Temporal Flexibility to Obtain High Quality Schedules & \href{../works/PolicellaWSO05.pdf}{Yes} & \cite{PolicellaWSO05} & 2005 & AAAI 2005 & 6 & 0 0 0 & 0 0 & \ref{b:PolicellaWSO05} & n/a\\
BarbulescuWH04 \href{http://www.aaai.org/Library/AAAI/2004/aaai04-023.php}{BarbulescuWH04} & \hyperref[auth:a1315]{L. Barbulescu}, \hyperref[auth:a1317]{L. D. Whitley}, \hyperref[auth:a1316]{A. E. Howe} & Leap Before You Look: An Effective Strategy in an Oversubscribed Scheduling Problem & \href{../works/BarbulescuWH04.pdf}{Yes} & \cite{BarbulescuWH04} & 2004 & AAAI 2004 & 6 & 0 0 0 & 0 0 & \ref{b:BarbulescuWH04} & n/a\\
DilkinaH04 \href{}{DilkinaH04} & \hyperref[auth:a267]{B. N. Dilkina}, \hyperref[auth:a269]{W. S. Havens} & The {U.S.} National Football League Scheduling Problem & \href{../works/DilkinaH04.pdf}{Yes} & \cite{DilkinaH04} & 2004 & AAAI 2004 & 6 & 0 0 0 & 0 0 & \ref{b:DilkinaH04} & n/a\\
GlobusCLP04 \href{}{GlobusCLP04} & \hyperref[auth:a1337]{A. Globus}, \hyperref[auth:a1338]{J. Crawford}, \hyperref[auth:a1339]{J. D. Lohn}, \hyperref[auth:a1340]{A. Pryor} & A Comparison of Techniques for Scheduling Earth Observing Satellites & \href{../works/GlobusCLP04.pdf}{Yes} & \cite{GlobusCLP04} & 2004 & AAAI 2004 & 8 & 0 0 0 & 0 0 & \ref{b:GlobusCLP04} & n/a\\
FukunagaHFAMN02 \href{http://www.aaai.org/Library/AAAI/2002/aaai02-123.php}{FukunagaHFAMN02} & \hyperref[auth:a1328]{A. S. Fukunaga}, \hyperref[auth:a1329]{E. Hamilton}, \hyperref[auth:a1330]{J. Fama}, \hyperref[auth:a1331]{D. Andre}, \hyperref[auth:a1332]{O. Matan}, \hyperref[auth:a1333]{I. R. Nourbakhsh} & Staff Scheduling for Inbound Call Centers and Customer Contact Centers & \href{../works/FukunagaHFAMN02.pdf}{Yes} & \cite{FukunagaHFAMN02} & 2002 & AAAI 2002 & 8 & 0 0 0 & 0 0 & \ref{b:FukunagaHFAMN02} & n/a\\
LimAHO02a \href{http://www.aaai.org/Library/AAAI/2002/aaai02-175.php}{LimAHO02a} & \hyperref[auth:a279]{A. Lim}, \hyperref[auth:a1334]{J. C. Ang}, \hyperref[auth:a1335]{W.-K. Ho}, \hyperref[auth:a1336]{W.-C. Oon} & UTTSExam: {A} University Examination Timetable Scheduler & \href{../works/LimAHO02a.pdf}{Yes} & \cite{LimAHO02a} & 2002 & AAAI 2002 & 2 & 0 0 0 & 0 0 & \ref{b:LimAHO02a} & n/a\\
CestaOS00 \href{http://www.aaai.org/Library/AAAI/2000/aaai00-114.php}{CestaOS00} & \hyperref[auth:a284]{A. Cesta}, \hyperref[auth:a282]{A. Oddi}, \hyperref[auth:a298]{S. F. Smith} & Iterative Flattening: {A} Scalable Method for Solving Multi-Capacity Scheduling Problems & \href{../works/CestaOS00.pdf}{Yes} & \cite{CestaOS00} & 2000 & AAAI 2000 & 6 & 0 0 0 & 0 0 & \ref{b:CestaOS00} & n/a\\
Junker00 \href{http://www.aaai.org/Library/AAAI/2000/aaai00-139.php}{Junker00} & \hyperref[auth:a1327]{U. Junker} & Preference-Based Search for Scheduling & \href{../works/Junker00.pdf}{Yes} & \cite{Junker00} & 2000 & AAAI 2000 & 6 & 0 0 0 & 0 0 & \ref{b:Junker00} & n/a\\
AbdennadherS99 \href{http://www.aaai.org/Library/IAAI/1999/iaai99-118.php}{AbdennadherS99} & \hyperref[auth:a1318]{S. Abdennadher}, \hyperref[auth:a711]{H. Schlenker} & Nurse Scheduling using Constraint Logic Programming & \href{../works/AbdennadherS99.pdf}{Yes} & \cite{AbdennadherS99} & 1999 & AAAI 1999 & 6 & 0 0 0 & 0 0 & \ref{b:AbdennadherS99} & n/a\\
BeckF99 \href{http://www.aaai.org/Library/AAAI/1999/aaai99-097.php}{BeckF99} & \hyperref[auth:a89]{J. C. Beck}, \hyperref[auth:a302]{M. S. Fox} & Scheduling Alternative Activities & \href{../works/BeckF99.pdf}{Yes} & \cite{BeckF99} & 1999 & AAAI 1999 & 8 & 0 0 0 & 0 0 & \ref{b:BeckF99} & n/a\\
ChunCTY99 \href{http://www.aaai.org/Library/IAAI/1999/iaai99-111.php}{ChunCTY99} & \hyperref[auth:a1323]{A. H. W. Chun}, \hyperref[auth:a1324]{S. H. C. Chan}, \hyperref[auth:a1325]{F. M. F. Tsang}, \hyperref[auth:a1326]{D. W. M. Yeung} & {HKIA} {SAS:} {A} Constraint-Based Airport Stand Allocation System Developed with Software Components & \href{../works/ChunCTY99.pdf}{Yes} & \cite{ChunCTY99} & 1999 & AAAI 1999 & 8 & 0 0 0 & 0 0 & \ref{b:ChunCTY99} & n/a\\
JoLLH99 \href{http://www.aaai.org/Library/IAAI/1999/iaai99-114.php}{JoLLH99} & \hyperref[auth:a1319]{G. Jo}, \hyperref[auth:a1320]{K.-H. Lee}, \hyperref[auth:a1321]{H.-Y. Lee}, \hyperref[auth:a1322]{S.-H. Hyun} & Ramp Activity Expert System for Scheduling and Co-ordination at an Airport & \href{../works/JoLLH99.pdf}{Yes} & \cite{JoLLH99} & 1999 & AAAI 1999 & 6 & 0 0 0 & 0 0 & \ref{b:JoLLH99} & n/a\\
WatsonBHW99 \href{http://www.aaai.org/Library/AAAI/1999/aaai99-098.php}{WatsonBHW99} & \hyperref[auth:a360]{J.-P. Watson}, \hyperref[auth:a1315]{L. Barbulescu}, \hyperref[auth:a1316]{A. E. Howe}, \hyperref[auth:a1317]{L. D. Whitley} & Algorithm Performance and Problem Structure for Flow-shop Scheduling & \href{../works/WatsonBHW99.pdf}{Yes} & \cite{WatsonBHW99} & 1999 & AAAI 1999 & 8 & 0 0 0 & 0 0 & \ref{b:WatsonBHW99} & n/a\\
BeckDSF97 \href{http://www.aaai.org/Library/AAAI/1997/aaai97-037.php}{BeckDSF97} & \hyperref[auth:a89]{J. C. Beck}, \hyperref[auth:a248]{A. J. Davenport}, \hyperref[auth:a1288]{E. M. Sitarski}, \hyperref[auth:a302]{M. S. Fox} & Beyond Contention: Extending Texture-Based Scheduling Heuristics & \href{../works/BeckDSF97.pdf}{Yes} & \cite{BeckDSF97} & 1997 & AAAI 1997 & 8 & 0 0 0 & 0 0 & \ref{b:BeckDSF97} & n/a\\
BeckDSF97a \href{http://www.aaai.org/Library/AAAI/1997/aaai97-038.php}{BeckDSF97a} & \hyperref[auth:a89]{J. C. Beck}, \hyperref[auth:a248]{A. J. Davenport}, \hyperref[auth:a1288]{E. M. Sitarski}, \hyperref[auth:a302]{M. S. Fox} & Texture-Based Heuristics for Scheduling Revisited & \href{../works/BeckDSF97a.pdf}{Yes} & \cite{BeckDSF97a} & 1997 & AAAI 1997 & 8 & 0 0 0 & 0 0 & \ref{b:BeckDSF97a} & n/a\\
GetoorOFC97 \href{http://www.aaai.org/Library/AAAI/1997/aaai97-047.php}{GetoorOFC97} & \hyperref[auth:a1293]{L. Getoor}, \hyperref[auth:a852]{G. Ottosson}, \hyperref[auth:a1294]{M. P. J. Fromherz}, \hyperref[auth:a1295]{B. Carlson} & Effective Redundant Constraints for Online Scheduling & \href{../works/GetoorOFC97.pdf}{Yes} & \cite{GetoorOFC97} & 1997 & AAAI 1997 & 6 & 0 0 0 & 0 0 & \ref{b:GetoorOFC97} & n/a\\
LeeKLKKYHP97 \href{http://www.aaai.org/Library/IAAI/1997/iaai97-182.php}{LeeKLKKYHP97} & \hyperref[auth:a1303]{K. J. Lee}, \hyperref[auth:a1304]{H. W. Kim}, \hyperref[auth:a1305]{J. K. Lee}, \hyperref[auth:a1306]{T. H. Kim}, \hyperref[auth:a1307]{C. G. Kim}, \hyperref[auth:a1308]{M. K. Yoon}, \hyperref[auth:a1309]{E. J. Hwang}, \hyperref[auth:a1310]{H. J. Park} & Case and Constraint-Based Apartment Construction Project Planning System: FASTrak-APT & \href{../works/LeeKLKKYHP97.pdf}{Yes} & \cite{LeeKLKKYHP97} & 1997 & AAAI 1997 & 6 & 0 0 0 & 0 0 & \ref{b:LeeKLKKYHP97} & n/a\\
MorgadoM97 \href{http://www.aaai.org/Library/IAAI/1997/iaai97-186.php}{MorgadoM97} & \hyperref[auth:a1296]{E. M. Morgado}, \hyperref[auth:a1297]{J. P. Martins} & CREWS{\ }NS: Scheduling Train Crew in The Netherlands & \href{../works/MorgadoM97.pdf}{Yes} & \cite{MorgadoM97} & 1997 & AAAI 1997 & 10 & 0 0 0 & 0 0 & \ref{b:MorgadoM97} & n/a\\
MurphyRFSS97 \href{http://www.aaai.org/Library/IAAI/1997/iaai97-187.php}{MurphyRFSS97} & \hyperref[auth:a1298]{K. Murphy}, \hyperref[auth:a1299]{E. Ralston}, \hyperref[auth:a1300]{D. Friedlander}, \hyperref[auth:a1301]{R. Swab}, \hyperref[auth:a1302]{P. Steege} & The Scheduling of Rail at Union Pacific Railroad & \href{../works/MurphyRFSS97.pdf}{Yes} & \cite{MurphyRFSS97} & 1997 & AAAI 1997 & 10 & 0 0 0 & 0 0 & \ref{b:MurphyRFSS97} & n/a\\
MurthyRAW97 \href{}{MurthyRAW97} & \hyperref[auth:a1311]{S. S. Murthy}, \hyperref[auth:a1312]{J. Rachlin}, \hyperref[auth:a1313]{R. Akkiraju}, \hyperref[auth:a1314]{F. Y. Wu} & Agent-Based Cooperative Scheduling & No & \cite{MurthyRAW97} & 1997 & AAAI 1997 & 6 & 0 0 0 & 0 0 & No & n/a\\
OddiS97 \href{http://www.aaai.org/Library/AAAI/1997/aaai97-048.php}{OddiS97} & \hyperref[auth:a282]{A. Oddi}, \hyperref[auth:a298]{S. F. Smith} & Stochastic Procedures for Generating Feasible Schedules & \href{../works/OddiS97.pdf}{Yes} & \cite{OddiS97} & 1997 & AAAI 1997 & 7 & 0 0 0 & 0 0 & \ref{b:OddiS97} & n/a\\
RoweJCA96 \href{http://www.aaai.org/Library/IAAI/1996/iaai96-280.php}{RoweJCA96} & \hyperref[auth:a1284]{J. Rowe}, \hyperref[auth:a1285]{K. Jewers}, \hyperref[auth:a1286]{A. Codd}, \hyperref[auth:a1287]{A. Alcock} & Intelligent Retail Logistics Scheduling & \href{../works/RoweJCA96.pdf}{Yes} & \cite{RoweJCA96} & 1996 & AAAI 1996 & 9 & 0 0 0 & 0 0 & \ref{b:RoweJCA96} & n/a\\
CrawfordB94 \href{http://www.aaai.org/Library/AAAI/1994/aaai94-168.php}{CrawfordB94} & \hyperref[auth:a1278]{J. M. Crawford}, \hyperref[auth:a1279]{A. B. Baker} & Experimental Results on the Application of Satisfiability Algorithms to Scheduling Problems & \href{../works/CrawfordB94.pdf}{Yes} & \cite{CrawfordB94} & 1994 & AAAI 1994 & 6 & 0 0 0 & 0 0 & \ref{b:CrawfordB94} & n/a\\
Muscettola94 \href{http://www.aaai.org/Library/AAAI/1994/aaai94-170.php}{Muscettola94} & \hyperref[auth:a289]{N. Muscettola} & On the Utility of Bottleneck Reasoning for Scheduling & \href{../works/Muscettola94.pdf}{Yes} & \cite{Muscettola94} & 1994 & AAAI 1994 & 6 & 0 0 0 & 0 0 & \ref{b:Muscettola94} & n/a\\
YoshikawaKNW94 \href{http://www.aaai.org/Library/AAAI/1994/aaai94-171.php}{YoshikawaKNW94} & \hyperref[auth:a1280]{M. Yoshikawa}, \hyperref[auth:a1281]{K. Kaneko}, \hyperref[auth:a1282]{Y. Nomura}, \hyperref[auth:a1283]{M. Watanabe} & A Constraint-Based Approach to High-School Timetabling Problems: {A} Case Study & \href{../works/YoshikawaKNW94.pdf}{Yes} & \cite{YoshikawaKNW94} & 1994 & AAAI 1994 & 6 & 0 0 0 & 0 0 & \ref{b:YoshikawaKNW94} & n/a\\
SmithC93 \href{http://www.aaai.org/Library/AAAI/1993/aaai93-022.php}{SmithC93} & \hyperref[auth:a298]{S. F. Smith}, \hyperref[auth:a1277]{C.-C. Cheng} & Slack-Based Heuristics for Constraint Satisfaction Scheduling & \href{../works/SmithC93.pdf}{Yes} & \cite{SmithC93} & 1993 & AAAI 1993 & 6 & 0 0 0 & 0 0 & \ref{b:SmithC93} & n/a\\
Hamscher91 \href{http://www.aaai.org/Library/AAAI/1991/aaai91-079.php}{Hamscher91} & \hyperref[auth:a1276]{W. Hamscher} & {ACP:} Reason Maintenance and Inference Control for Constraint Propagation Over Intervals & \href{../works/Hamscher91.pdf}{Yes} & \cite{Hamscher91} & 1991 & AAAI 1991 & 6 & 0 0 0 & 0 0 & \ref{b:Hamscher91} & n/a\\
EskeyZ90 \href{http://www.aaai.org/Library/AAAI/1990/aaai90-136.php}{EskeyZ90} & \hyperref[auth:a1274]{M. Eskey}, \hyperref[auth:a1275]{M. Zweben} & Learning Search Control for Constraint-Based Scheduling & \href{../works/EskeyZ90.pdf}{Yes} & \cite{EskeyZ90} & 1990 & AAAI 1990 & 8 & 0 0 0 & 0 0 & \ref{b:EskeyZ90} & n/a\\
MintonJPL90 \href{http://www.aaai.org/Library/AAAI/1990/aaai90-003.php}{MintonJPL90} & \hyperref[auth:a1211]{S. Minton}, \hyperref[auth:a1212]{M. D. Johnston}, \hyperref[auth:a1213]{A. B. Philips}, \hyperref[auth:a1214]{P. Laird} & Solving Large-Scale Constraint-Satisfaction and Scheduling Problems Using a Heuristic Repair Method & \href{../works/MintonJPL90.pdf}{Yes} & \cite{MintonJPL90} & 1990 & AAAI 1990 & 8 & 0 0 0 & 0 0 & \ref{b:MintonJPL90} & n/a\\
Valdes87 \href{http://www.aaai.org/Library/AAAI/1987/aaai87-046.php}{Valdes87} & \hyperref[auth:a1273]{R. E. Vald{\'{e}}s-P{\'{e}}rez} & The Satisfiability of Temporal Constraint Networks & \href{../works/Valdes87.pdf}{Yes} & \cite{Valdes87} & 1987 & AAAI 1987 & 5 & 0 0 0 & 0 0 & \ref{b:Valdes87} & n/a\\
Rit86 \href{http://www.aaai.org/Library/AAAI/1986/aaai86-064.php}{Rit86} & \hyperref[auth:a1272]{J.-F. Rit} & Propagating Temporal Constraints for Scheduling & \href{../works/Rit86.pdf}{Yes} & \cite{Rit86} & 1986 & AAAI 1986 & 6 & 0 0 0 & 0 0 & \ref{b:Rit86} & n/a\\
FoxAS82 \href{http://www.aaai.org/Library/AAAI/1982/aaai82-037.php}{FoxAS82} & \hyperref[auth:a302]{M. S. Fox}, \hyperref[auth:a1006]{B. P. Allen}, \hyperref[auth:a1007]{G. Strohm} & Job-Shop Scheduling: An Investigation in Constraint-Directed Reasoning & \href{../works/FoxAS82.pdf}{Yes} & \cite{FoxAS82} & 1982 & AAAI 1982 & 4 & 0 0 0 & 0 0 & \ref{b:FoxAS82} & n/a\\
\end{longtable}
}

\subsection{ACIIDS}

\index{ACIIDS}
{\scriptsize
\begin{longtable}{>{\raggedright\arraybackslash}p{3cm}>{\raggedright\arraybackslash}p{4.5cm}>{\raggedright\arraybackslash}p{6.0cm}rrrp{2.5cm}rp{1cm}p{1cm}rr}
\rowcolor{white}\caption{Papers in Conference Series ACIIDS (Total 1) (Total 1)}\\ \toprule
\rowcolor{white}\shortstack{Key\\Source} & Authors & Title (Colored by Open Access)& LC & Cite & Year & \shortstack{Conference\\/Journal\\/School} & Pages & \shortstack{Cites\\OC XR\\SC} & \shortstack{Refs\\OC\\XR} & b & c \\ \midrule\endhead
\bottomrule
\endfoot
ArbaouiY18 \href{https://doi.org/10.1007/978-3-319-75420-8_67}{ArbaouiY18} & \hyperref[auth:a578]{T. Arbaoui}, \hyperref[auth:a455]{F. Yalaoui} & Solving the Unrelated Parallel Machine Scheduling Problem with Additional Resources Using Constraint Programming & \href{../works/ArbaouiY18.pdf}{Yes} & \cite{ArbaouiY18} & 2018 & ACIIDS 2018 & 10 & 2 2 8 & 14 15 & \ref{b:ArbaouiY18} & n/a\\
\end{longtable}
}

\subsection{AIAI}

\index{AIAI}
{\scriptsize
\begin{longtable}{>{\raggedright\arraybackslash}p{3cm}>{\raggedright\arraybackslash}p{4.5cm}>{\raggedright\arraybackslash}p{6.0cm}rrrp{2.5cm}rp{1cm}p{1cm}rr}
\rowcolor{white}\caption{Papers in Conference Series AIAI (Total 1) (Total 1)}\\ \toprule
\rowcolor{white}\shortstack{Key\\Source} & Authors & Title (Colored by Open Access)& LC & Cite & Year & \shortstack{Conference\\/Journal\\/School} & Pages & \shortstack{Cites\\OC XR\\SC} & \shortstack{Refs\\OC\\XR} & b & c \\ \midrule\endhead
\bottomrule
\endfoot
LiuLH19 \href{https://doi.org/10.1007/978-3-030-19823-7_19}{LiuLH19} & \hyperref[auth:a1391]{K. Liu}, \hyperref[auth:a1400]{S. L{\"{o}}ffler}, \hyperref[auth:a1393]{P. Hofstedt} & \cellcolor{green!10}Solving the Talent Scheduling Problem by Parallel Constraint Programming & \href{../works/LiuLH19.pdf}{Yes} & \cite{LiuLH19} & 2019 & AIAI 2019 & 9 & 1 1 1 & 5 11 & \ref{b:LiuLH19} & n/a\\
\end{longtable}
}

\subsection{AICCC}

\index{AICCC}
{\scriptsize
\begin{longtable}{>{\raggedright\arraybackslash}p{3cm}>{\raggedright\arraybackslash}p{4.5cm}>{\raggedright\arraybackslash}p{6.0cm}rrrp{2.5cm}rp{1cm}p{1cm}rr}
\rowcolor{white}\caption{Papers in Conference Series AICCC (Total 1) (Total 1)}\\ \toprule
\rowcolor{white}\shortstack{Key\\Source} & Authors & Title (Colored by Open Access)& LC & Cite & Year & \shortstack{Conference\\/Journal\\/School} & Pages & \shortstack{Cites\\OC XR\\SC} & \shortstack{Refs\\OC\\XR} & b & c \\ \midrule\endhead
\bottomrule
\endfoot
HoYCLLCLC18 \href{https://doi.org/10.1145/3299819.3299825}{HoYCLLCLC18} & \hyperref[auth:a579]{T.-W. Ho}, \hyperref[auth:a580]{J.-S. Yao}, \hyperref[auth:a581]{Y.-T. Chang}, \hyperref[auth:a582]{F. Lai}, \hyperref[auth:a583]{J.-F. Lai}, \hyperref[auth:a584]{S.-M. Chu}, \hyperref[auth:a585]{W.-C. Liao}, \hyperref[auth:a586]{H.-M. Chiu} & A Platform for Dynamic Optimal Nurse Scheduling Based on Integer Linear Programming along with Multiple Criteria Constraints & \href{../works/HoYCLLCLC18.pdf}{Yes} & \cite{HoYCLLCLC18} & 2018 & AICCC 2018 & 6 & 2 3 1 & 14 14 & \ref{b:HoYCLLCLC18} & n/a\\
\end{longtable}
}

\subsection{AIPS}

\index{AIPS}
{\scriptsize
\begin{longtable}{>{\raggedright\arraybackslash}p{3cm}>{\raggedright\arraybackslash}p{4.5cm}>{\raggedright\arraybackslash}p{6.0cm}rrrp{2.5cm}rp{1cm}p{1cm}rr}
\rowcolor{white}\caption{Papers in Conference Series AIPS (Total 1) (Total 1)}\\ \toprule
\rowcolor{white}\shortstack{Key\\Source} & Authors & Title (Colored by Open Access)& LC & Cite & Year & \shortstack{Conference\\/Journal\\/School} & Pages & \shortstack{Cites\\OC XR\\SC} & \shortstack{Refs\\OC\\XR} & b & c \\ \midrule\endhead
\bottomrule
\endfoot
FocacciLN00 \href{http://www.aaai.org/Library/AIPS/2000/aips00-010.php}{FocacciLN00} & \hyperref[auth:a776]{F. Focacci}, \hyperref[auth:a118]{P. Laborie}, \hyperref[auth:a656]{W. Nuijten} & Solving Scheduling Problems with Setup Times and Alternative Resources & \href{../works/FocacciLN00.pdf}{Yes} & \cite{FocacciLN00} & 2000 & AIPS 2000 & 10 & 0 0 0 & 0 0 & \ref{b:FocacciLN00} & n/a\\
\end{longtable}
}

\subsection{ANT}

\index{ANT}
{\scriptsize
\begin{longtable}{>{\raggedright\arraybackslash}p{3cm}>{\raggedright\arraybackslash}p{4.5cm}>{\raggedright\arraybackslash}p{6.0cm}rrrp{2.5cm}rp{1cm}p{1cm}rr}
\rowcolor{white}\caption{Papers in Conference Series ANT (Total 1) (Total 1)}\\ \toprule
\rowcolor{white}\shortstack{Key\\Source} & Authors & Title (Colored by Open Access)& LC & Cite & Year & \shortstack{Conference\\/Journal\\/School} & Pages & \shortstack{Cites\\OC XR\\SC} & \shortstack{Refs\\OC\\XR} & b & c \\ \midrule\endhead
\bottomrule
\endfoot
YuraszeckMC23 \href{https://doi.org/10.1016/j.procs.2023.03.130}{YuraszeckMC23} & \hyperref[auth:a405]{F. Yuraszeck}, \hyperref[auth:a424]{G. Mej{\'{\i}}a}, \hyperref[auth:a407]{D. Canut-de-Bon} & \cellcolor{gold!20}A competitive constraint programming approach for the group shop scheduling problem & \href{../works/YuraszeckMC23.pdf}{Yes} & \cite{YuraszeckMC23} & 2023 & ANT 2023 & 6 & 1 1 1 & 15 18 & \ref{b:YuraszeckMC23} & \ref{c:YuraszeckMC23}\\
\end{longtable}
}

\subsection{ANTS}

\index{ANTS}
{\scriptsize
\begin{longtable}{>{\raggedright\arraybackslash}p{3cm}>{\raggedright\arraybackslash}p{4.5cm}>{\raggedright\arraybackslash}p{6.0cm}rrrp{2.5cm}rp{1cm}p{1cm}rr}
\rowcolor{white}\caption{Papers in Conference Series ANTS (Total 1) (Total 1)}\\ \toprule
\rowcolor{white}\shortstack{Key\\Source} & Authors & Title (Colored by Open Access)& LC & Cite & Year & \shortstack{Conference\\/Journal\\/School} & Pages & \shortstack{Cites\\OC XR\\SC} & \shortstack{Refs\\OC\\XR} & b & c \\ \midrule\endhead
\bottomrule
\endfoot
MeyerE04 \href{https://doi.org/10.1007/978-3-540-28646-2_15}{MeyerE04} & \hyperref[auth:a637]{B. Meyer}, \hyperref[auth:a1412]{A. Ernst} & Integrating ACO and Constraint Propagation & \href{../works/MeyerE04.pdf}{Yes} & \cite{MeyerE04} & 2004 & ANTS 2004 & 12 & 37 37 45 & 14 17 & \ref{b:MeyerE04} & n/a\\
\end{longtable}
}

\subsection{APMS}

\index{APMS}
{\scriptsize
\begin{longtable}{>{\raggedright\arraybackslash}p{3cm}>{\raggedright\arraybackslash}p{4.5cm}>{\raggedright\arraybackslash}p{6.0cm}rrrp{2.5cm}rp{1cm}p{1cm}rr}
\rowcolor{white}\caption{Papers in Conference Series APMS (Total 1) (Total 1)}\\ \toprule
\rowcolor{white}\shortstack{Key\\Source} & Authors & Title (Colored by Open Access)& LC & Cite & Year & \shortstack{Conference\\/Journal\\/School} & Pages & \shortstack{Cites\\OC XR\\SC} & \shortstack{Refs\\OC\\XR} & b & c \\ \midrule\endhead
\bottomrule
\endfoot
Mehdizadeh-Somarin23 \href{https://doi.org/10.1007/978-3-031-43670-3_33}{Mehdizadeh-Somarin23} & \hyperref[auth:a429]{Z. Mehdizadeh-Somarin}, \hyperref[auth:a430]{R. Tavakkoli-Moghaddam}, \hyperref[auth:a431]{M. Rohaninejad}, \hyperref[auth:a116]{Z. Hanz{\'{a}}lek}, \hyperref[auth:a432]{B. V. Nouri} & A Constraint Programming Model for a Reconfigurable Job Shop Scheduling Problem with Machine Availability & \href{../works/Mehdizadeh-Somarin23.pdf}{Yes} & \cite{Mehdizadeh-Somarin23} & 2023 & APMS 2023 & 14 & 0 0 0 & 0 31 & \ref{b:Mehdizadeh-Somarin23} & \ref{c:Mehdizadeh-Somarin23}\\
\end{longtable}
}

\subsection{ASTAIR}

\index{ASTAIR}
{\scriptsize
\begin{longtable}{>{\raggedright\arraybackslash}p{3cm}>{\raggedright\arraybackslash}p{4.5cm}>{\raggedright\arraybackslash}p{6.0cm}rrrp{2.5cm}rp{1cm}p{1cm}rr}
\rowcolor{white}\caption{Papers in Conference Series ASTAIR (Total 1) (Total 1)}\\ \toprule
\rowcolor{white}\shortstack{Key\\Source} & Authors & Title (Colored by Open Access)& LC & Cite & Year & \shortstack{Conference\\/Journal\\/School} & Pages & \shortstack{Cites\\OC XR\\SC} & \shortstack{Refs\\OC\\XR} & b & c \\ \midrule\endhead
\bottomrule
\endfoot
DincbasS91 \href{}{DincbasS91} & \hyperref[auth:a717]{M. Dincbas}, \hyperref[auth:a17]{H. Simonis} & Apache-a constraint based, automated stand allocation system & \href{../works/DincbasS91.pdf}{Yes} & \cite{DincbasS91} & 1991 & ASTAIR 1991 & 13 & 0 0 0 & 0 0 & \ref{b:DincbasS91} & n/a\\
\end{longtable}
}

\subsection{ATMOS}

\index{ATMOS}
{\scriptsize
\begin{longtable}{>{\raggedright\arraybackslash}p{3cm}>{\raggedright\arraybackslash}p{4.5cm}>{\raggedright\arraybackslash}p{6.0cm}rrrp{2.5cm}rp{1cm}p{1cm}rr}
\rowcolor{white}\caption{Papers in Conference Series ATMOS (Total 1) (Total 1)}\\ \toprule
\rowcolor{white}\shortstack{Key\\Source} & Authors & Title (Colored by Open Access)& LC & Cite & Year & \shortstack{Conference\\/Journal\\/School} & Pages & \shortstack{Cites\\OC XR\\SC} & \shortstack{Refs\\OC\\XR} & b & c \\ \midrule\endhead
\bottomrule
\endfoot
AronssonBK09 \href{http://drops.dagstuhl.de/opus/volltexte/2009/2141}{AronssonBK09} & \hyperref[auth:a707]{M. Aronsson}, \hyperref[auth:a708]{M. Bohlin}, \hyperref[auth:a709]{P. Kreuger} & {MILP} formulations of cumulative constraints for railway scheduling - {A} comparative study & \href{../works/AronssonBK09.pdf}{Yes} & \cite{AronssonBK09} & 2009 & ATMOS 2009 & 13 & 0 0 0 & 0 0 & \ref{b:AronssonBK09} & n/a\\
\end{longtable}
}

\subsection{CANDAR}

\index{CANDAR}
{\scriptsize
\begin{longtable}{>{\raggedright\arraybackslash}p{3cm}>{\raggedright\arraybackslash}p{4.5cm}>{\raggedright\arraybackslash}p{6.0cm}rrrp{2.5cm}rp{1cm}p{1cm}rr}
\rowcolor{white}\caption{Papers in Conference Series CANDAR (Total 1) (Total 1)}\\ \toprule
\rowcolor{white}\shortstack{Key\\Source} & Authors & Title (Colored by Open Access)& LC & Cite & Year & \shortstack{Conference\\/Journal\\/School} & Pages & \shortstack{Cites\\OC XR\\SC} & \shortstack{Refs\\OC\\XR} & b & c \\ \midrule\endhead
\bottomrule
\endfoot
NishikawaSTT18 \href{https://doi.org/10.1109/CANDAR.2018.00025}{NishikawaSTT18} & \hyperref[auth:a531]{H. Nishikawa}, \hyperref[auth:a532]{K. Shimada}, \hyperref[auth:a533]{I. Taniguchi}, \hyperref[auth:a534]{H. Tomiyama} & Scheduling of Malleable Fork-Join Tasks with Constraint Programming & \href{../works/NishikawaSTT18.pdf}{Yes} & \cite{NishikawaSTT18} & 2018 & CANDAR 2018 & 6 & 2 2 2 & 14 21 & \ref{b:NishikawaSTT18} & n/a\\
\end{longtable}
}

\subsection{CAiSE}

\index{CAiSE}
{\scriptsize
\begin{longtable}{>{\raggedright\arraybackslash}p{3cm}>{\raggedright\arraybackslash}p{4.5cm}>{\raggedright\arraybackslash}p{6.0cm}rrrp{2.5cm}rp{1cm}p{1cm}rr}
\rowcolor{white}\caption{Papers in Conference Series CAiSE (Total 1) (Total 1)}\\ \toprule
\rowcolor{white}\shortstack{Key\\Source} & Authors & Title (Colored by Open Access)& LC & Cite & Year & \shortstack{Conference\\/Journal\\/School} & Pages & \shortstack{Cites\\OC XR\\SC} & \shortstack{Refs\\OC\\XR} & b & c \\ \midrule\endhead
\bottomrule
\endfoot
ZhuS02 \href{https://doi.org/10.1007/3-540-47961-9_69}{ZhuS02} & \hyperref[auth:a674]{K. Q. Zhu}, \hyperref[auth:a675]{A. E. Santosa} & A Meeting Scheduling System Based on Open Constraint Programming & \href{../works/ZhuS02.pdf}{Yes} & \cite{ZhuS02} & 2002 & CAiSE 2002 & 5 & 0 0 0 & 5 7 & \ref{b:ZhuS02} & n/a\\
\end{longtable}
}

\subsection{CCL'99}

\index{CCL'99}
{\scriptsize
\begin{longtable}{>{\raggedright\arraybackslash}p{3cm}>{\raggedright\arraybackslash}p{4.5cm}>{\raggedright\arraybackslash}p{6.0cm}rrrp{2.5cm}rp{1cm}p{1cm}rr}
\rowcolor{white}\caption{Papers in Conference Series CCL'99 (Total 1) (Total 1)}\\ \toprule
\rowcolor{white}\shortstack{Key\\Source} & Authors & Title (Colored by Open Access)& LC & Cite & Year & \shortstack{Conference\\/Journal\\/School} & Pages & \shortstack{Cites\\OC XR\\SC} & \shortstack{Refs\\OC\\XR} & b & c \\ \midrule\endhead
\bottomrule
\endfoot
Simonis99 \href{https://doi.org/10.1007/3-540-45406-3_6}{Simonis99} & \hyperref[auth:a17]{H. Simonis} & Building Industrial Applications with Constraint Programming & \href{../works/Simonis99.pdf}{Yes} & \cite{Simonis99} & 1999 & CCL'99 1999 & 39 & 5 5 0 & 18 63 & \ref{b:Simonis99} & n/a\\
\end{longtable}
}

\subsection{CIT}

\index{CIT}
{\scriptsize
\begin{longtable}{>{\raggedright\arraybackslash}p{3cm}>{\raggedright\arraybackslash}p{4.5cm}>{\raggedright\arraybackslash}p{6.0cm}rrrp{2.5cm}rp{1cm}p{1cm}rr}
\rowcolor{white}\caption{Papers in Conference Series CIT (Total 1) (Total 1)}\\ \toprule
\rowcolor{white}\shortstack{Key\\Source} & Authors & Title (Colored by Open Access)& LC & Cite & Year & \shortstack{Conference\\/Journal\\/School} & Pages & \shortstack{Cites\\OC XR\\SC} & \shortstack{Refs\\OC\\XR} & b & c \\ \midrule\endhead
\bottomrule
\endfoot
ZhangLS12 \href{https://doi.org/10.1109/CIT.2012.96}{ZhangLS12} & \hyperref[auth:a611]{X. Zhang}, \hyperref[auth:a612]{Z. Lv}, \hyperref[auth:a613]{X. Song} & Model and Solution for Hot Strip Rolling Scheduling Problem Based on Constraint Programming Method & \href{../works/ZhangLS12.pdf}{Yes} & \cite{ZhangLS12} & 2012 & CIT 2012 & 4 & 1 1 1 & 3 9 & \ref{b:ZhangLS12} & n/a\\
\end{longtable}
}

\subsection{CONTESSA}

\index{CONTESSA}
{\scriptsize
\begin{longtable}{>{\raggedright\arraybackslash}p{3cm}>{\raggedright\arraybackslash}p{4.5cm}>{\raggedright\arraybackslash}p{6.0cm}rrrp{2.5cm}rp{1cm}p{1cm}rr}
\rowcolor{white}\caption{Papers in Conference Series CONTESSA (Total 1) (Total 1)}\\ \toprule
\rowcolor{white}\shortstack{Key\\Source} & Authors & Title (Colored by Open Access)& LC & Cite & Year & \shortstack{Conference\\/Journal\\/School} & Pages & \shortstack{Cites\\OC XR\\SC} & \shortstack{Refs\\OC\\XR} & b & c \\ \midrule\endhead
\bottomrule
\endfoot
Simonis95a \href{https://doi.org/10.1007/3-540-60794-3_11}{Simonis95a} & \hyperref[auth:a17]{H. Simonis} & Application Development with the {CHIP} System & \href{../works/Simonis95a.pdf}{Yes} & \cite{Simonis95a} & 1995 & CONTESSA 1995 & 21 & 1 1 0 & 12 30 & \ref{b:Simonis95a} & n/a\\
\end{longtable}
}

\subsection{CP}

\index{CP}
{\scriptsize
\begin{longtable}{>{\raggedright\arraybackslash}p{3cm}>{\raggedright\arraybackslash}p{4.5cm}>{\raggedright\arraybackslash}p{6.0cm}rrrp{2.5cm}rp{1cm}p{1cm}rr}
\rowcolor{white}\caption{Papers in Conference Series CP (Total 132) (Total 132)}\\ \toprule
\rowcolor{white}\shortstack{Key\\Source} & Authors & Title (Colored by Open Access)& LC & Cite & Year & \shortstack{Conference\\/Journal\\/School} & Pages & \shortstack{Cites\\OC XR\\SC} & \shortstack{Refs\\OC\\XR} & b & c \\ \midrule\endhead
\bottomrule
\endfoot
AalianPG23 \href{https://doi.org/10.4230/LIPIcs.CP.2023.6}{AalianPG23} & \hyperref[auth:a7]{Y. Aalian}, \hyperref[auth:a8]{G. Pesant}, \hyperref[auth:a9]{M. Gamache} & Optimization of Short-Term Underground Mine Planning Using Constraint Programming & \href{../works/AalianPG23.pdf}{Yes} & \cite{AalianPG23} & 2023 & CP 2023 & 16 & 0 0 0 & 0 0 & \ref{b:AalianPG23} & \ref{c:AalianPG23}\\
JuvinHHL23 \href{https://doi.org/10.4230/LIPIcs.CP.2023.19}{JuvinHHL23} & \hyperref[auth:a0]{C. Juvin}, \hyperref[auth:a1]{E. Hebrard}, \hyperref[auth:a2]{L. Houssin}, \hyperref[auth:a3]{P. Lopez} & An Efficient Constraint Programming Approach to Preemptive Job Shop Scheduling & \href{../works/JuvinHHL23.pdf}{Yes} & \cite{JuvinHHL23} & 2023 & CP 2023 & 16 & 0 0 0 & 0 0 & \ref{b:JuvinHHL23} & \ref{c:JuvinHHL23}\\
KameugneFND23 \href{https://doi.org/10.4230/LIPIcs.CP.2023.20}{KameugneFND23} & \hyperref[auth:a10]{R. Kameugne}, \hyperref[auth:a11]{S. B. Fetgo}, \hyperref[auth:a12]{T. Noulamo}, \hyperref[auth:a13]{C. T. Djam{\'{e}}gni} & Horizontally Elastic Edge Finder Rule for Cumulative Constraint Based on Slack and Density & \href{../works/KameugneFND23.pdf}{Yes} & \cite{KameugneFND23} & 2023 & CP 2023 & 17 & 0 0 0 & 0 0 & \ref{b:KameugneFND23} & \ref{c:KameugneFND23}\\
PovedaAA23 \href{https://doi.org/10.4230/LIPIcs.CP.2023.31}{PovedaAA23} & \hyperref[auth:a4]{G. Pov{\'{e}}da}, \hyperref[auth:a5]{N. {\'{A}}lvarez}, \hyperref[auth:a6]{C. Artigues} & Partially Preemptive Multi Skill/Mode Resource-Constrained Project Scheduling with Generalized Precedence Relations and Calendars & \href{../works/PovedaAA23.pdf}{Yes} & \cite{PovedaAA23} & 2023 & CP 2023 & 21 & 0 0 0 & 0 0 & \ref{b:PovedaAA23} & \ref{c:PovedaAA23}\\
BoudreaultSLQ22 \href{https://doi.org/10.4230/LIPIcs.CP.2022.10}{BoudreaultSLQ22} & \hyperref[auth:a34]{R. Boudreault}, \hyperref[auth:a35]{V. Simard}, \hyperref[auth:a36]{D. Lafond}, \hyperref[auth:a37]{C.-G. Quimper} & A Constraint Programming Approach to Ship Refit Project Scheduling & \href{../works/BoudreaultSLQ22.pdf}{Yes} & \cite{BoudreaultSLQ22} & 2022 & CP 2022 & 16 & 0 0 0 & 0 0 & \ref{b:BoudreaultSLQ22} & \ref{c:BoudreaultSLQ22}\\
PopovicCGNC22 \href{https://doi.org/10.4230/LIPIcs.CP.2022.34}{PopovicCGNC22} & \hyperref[auth:a38]{L. Popovic}, \hyperref[auth:a39]{A. C{\^{o}}t{\'{e}}}, \hyperref[auth:a40]{M. Gaha}, \hyperref[auth:a41]{F. Nguewouo}, \hyperref[auth:a42]{Q. Cappart} & Scheduling the Equipment Maintenance of an Electric Power Transmission Network Using Constraint Programming & \href{../works/PopovicCGNC22.pdf}{Yes} & \cite{PopovicCGNC22} & 2022 & CP 2022 & 15 & 0 0 0 & 0 0 & \ref{b:PopovicCGNC22} & \ref{c:PopovicCGNC22}\\
WinterMMW22 \href{https://doi.org/10.4230/LIPIcs.CP.2022.41}{WinterMMW22} & \hyperref[auth:a43]{F. Winter}, \hyperref[auth:a44]{S. Meiswinkel}, \hyperref[auth:a45]{N. Musliu}, \hyperref[auth:a46]{D. Walkiewicz} & Modeling and Solving Parallel Machine Scheduling with Contamination Constraints in the Agricultural Industry & \href{../works/WinterMMW22.pdf}{Yes} & \cite{WinterMMW22} & 2022 & CP 2022 & 18 & 0 0 0 & 0 0 & \ref{b:WinterMMW22} & \ref{c:WinterMMW22}\\
AntuoriHHEN21 \href{https://doi.org/10.4230/LIPIcs.CP.2021.14}{AntuoriHHEN21} & \hyperref[auth:a53]{V. Antuori}, \hyperref[auth:a1]{E. Hebrard}, \hyperref[auth:a54]{M.-J. Huguet}, \hyperref[auth:a55]{S. Essodaigui}, \hyperref[auth:a56]{A. Nguyen} & Combining Monte Carlo Tree Search and Depth First Search Methods for a Car Manufacturing Workshop Scheduling Problem & \href{../works/AntuoriHHEN21.pdf}{Yes} & \cite{AntuoriHHEN21} & 2021 & CP 2021 & 16 & 0 0 1 & 0 0 & \ref{b:AntuoriHHEN21} & \ref{c:AntuoriHHEN21}\\
ArmstrongGOS21 \href{https://doi.org/10.4230/LIPIcs.CP.2021.16}{ArmstrongGOS21} & \hyperref[auth:a14]{E. Armstrong}, \hyperref[auth:a15]{M. Garraffa}, \hyperref[auth:a16]{B. O'Sullivan}, \hyperref[auth:a17]{H. Simonis} & The Hybrid Flexible Flowshop with Transportation Times & \href{../works/ArmstrongGOS21.pdf}{Yes} & \cite{ArmstrongGOS21} & 2021 & CP 2021 & 18 & 1 0 1 & 0 0 & \ref{b:ArmstrongGOS21} & \ref{c:ArmstrongGOS21}\\
KovacsTKSG21 \href{https://doi.org/10.4230/LIPIcs.CP.2021.36}{KovacsTKSG21} & \hyperref[auth:a57]{B. Kov{\'{a}}cs}, \hyperref[auth:a58]{P. Tassel}, \hyperref[auth:a59]{W. Kohlenbrein}, \hyperref[auth:a60]{P. Schrott-Kostwein}, \hyperref[auth:a61]{M. Gebser} & Utilizing Constraint Optimization for Industrial Machine Workload Balancing & \href{../works/KovacsTKSG21.pdf}{Yes} & \cite{KovacsTKSG21} & 2021 & CP 2021 & 17 & 0 0 4 & 0 0 & \ref{b:KovacsTKSG21} & \ref{c:KovacsTKSG21}\\
LacknerMMWW21 \href{https://doi.org/10.4230/LIPIcs.CP.2021.37}{LacknerMMWW21} & \hyperref[auth:a62]{M.-L. Lackner}, \hyperref[auth:a63]{C. Mrkvicka}, \hyperref[auth:a45]{N. Musliu}, \hyperref[auth:a46]{D. Walkiewicz}, \hyperref[auth:a43]{F. Winter} & Minimizing Cumulative Batch Processing Time for an Industrial Oven Scheduling Problem & \href{../works/LacknerMMWW21.pdf}{Yes} & \cite{LacknerMMWW21} & 2021 & CP 2021 & 18 & 0 0 3 & 0 0 & \ref{b:LacknerMMWW21} & \ref{c:LacknerMMWW21}\\
AntuoriHHEN20 \href{https://doi.org/10.1007/978-3-030-58475-7_38}{AntuoriHHEN20} & \hyperref[auth:a53]{V. Antuori}, \hyperref[auth:a1]{E. Hebrard}, \hyperref[auth:a54]{M.-J. Huguet}, \hyperref[auth:a55]{S. Essodaigui}, \hyperref[auth:a56]{A. Nguyen} & \cellcolor{green!10}Leveraging Reinforcement Learning, Constraint Programming and Local Search: {A} Case Study in Car Manufacturing & \href{../works/AntuoriHHEN20.pdf}{Yes} & \cite{AntuoriHHEN20} & 2020 & CP 2020 & 16 & 3 3 6 & 8 16 & \ref{b:AntuoriHHEN20} & \ref{c:AntuoriHHEN20}\\
GroleazNS20 \href{https://doi.org/10.1007/978-3-030-58475-7_36}{GroleazNS20} & \hyperref[auth:a83]{L. Groleaz}, \hyperref[auth:a84]{S. N. Ndiaye}, \hyperref[auth:a85]{C. Solnon} & \cellcolor{green!10}Solving the Group Cumulative Scheduling Problem with {CPO} and {ACO} & \href{../works/GroleazNS20.pdf}{Yes} & \cite{GroleazNS20} & 2020 & CP 2020 & 17 & 1 1 1 & 25 29 & \ref{b:GroleazNS20} & \ref{c:GroleazNS20}\\
NattafM20 \href{https://doi.org/10.1007/978-3-030-58475-7_27}{NattafM20} & \hyperref[auth:a81]{M. Nattaf}, \hyperref[auth:a82]{A. Malapert} & \cellcolor{green!10}Filtering Rules for Flow Time Minimization in a Parallel Machine Scheduling Problem & \href{../works/NattafM20.pdf}{Yes} & \cite{NattafM20} & 2020 & CP 2020 & 16 & 0 0 0 & 6 12 & \ref{b:NattafM20} & \ref{c:NattafM20}\\
ColT19 \href{https://doi.org/10.1007/978-3-030-30048-7_9}{ColT19} & \hyperref[auth:a93]{G. D. Col}, \hyperref[auth:a94]{E. C. Teppan} & Industrial Size Job Shop Scheduling Tackled by Present Day {CP} Solvers & \href{../works/ColT19.pdf}{Yes} & \cite{ColT19} & 2019 & CP 2019 & 17 & 11 12 17 & 12 20 & \ref{b:ColT19} & \ref{c:ColT19}\\
FrimodigS19 \href{https://doi.org/10.1007/978-3-030-30048-7_25}{FrimodigS19} & \hyperref[auth:a95]{S. Frimodig}, \hyperref[auth:a92]{C. Schulte} & Models for Radiation Therapy Patient Scheduling & \href{../works/FrimodigS19.pdf}{Yes} & \cite{FrimodigS19} & 2019 & CP 2019 & 17 & 3 4 4 & 26 32 & \ref{b:FrimodigS19} & \ref{c:FrimodigS19}\\
GalleguillosKSB19 \href{https://doi.org/10.1007/978-3-030-30048-7_26}{GalleguillosKSB19} & \hyperref[auth:a96]{C. Galleguillos}, \hyperref[auth:a97]{Z. Kiziltan}, \hyperref[auth:a98]{A. S{\^{\i}}rbu}, \hyperref[auth:a99]{{\"{O}}zalp Babaoglu} & \cellcolor{green!10}Constraint Programming-Based Job Dispatching for Modern {HPC} Applications & \href{../works/GalleguillosKSB19.pdf}{Yes} & \cite{GalleguillosKSB19} & 2019 & CP 2019 & 18 & 1 2 3 & 27 39 & \ref{b:GalleguillosKSB19} & \ref{c:GalleguillosKSB19}\\
MurinR19 \href{https://doi.org/10.1007/978-3-030-30048-7_27}{MurinR19} & \hyperref[auth:a100]{S. Mur{\'{\i}}n}, \hyperref[auth:a101]{H. Rudov{\'{a}}} & Scheduling of Mobile Robots Using Constraint Programming & \href{../works/MurinR19.pdf}{Yes} & \cite{MurinR19} & 2019 & CP 2019 & 16 & 2 2 2 & 22 26 & \ref{b:MurinR19} & \ref{c:MurinR19}\\
CappartTSR18 \href{https://doi.org/10.1007/978-3-319-98334-9_32}{CappartTSR18} & \hyperref[auth:a42]{Q. Cappart}, \hyperref[auth:a834]{C. Thomas}, \hyperref[auth:a147]{P. Schaus}, \hyperref[auth:a326]{L.-M. Rousseau} & A Constraint Programming Approach for Solving Patient Transportation Problems & \href{../works/CappartTSR18.pdf}{Yes} & \cite{CappartTSR18} & 2018 & CP 2018 & 17 & 6 6 11 & 31 37 & \ref{b:CappartTSR18} & \ref{c:CappartTSR18}\\
He0GLW18 \href{https://doi.org/10.1007/978-3-319-98334-9_42}{He0GLW18} & \hyperref[auth:a184]{S. He}, \hyperref[auth:a117]{M. G. Wallace}, \hyperref[auth:a185]{G. Gange}, \hyperref[auth:a186]{A. Liebman}, \hyperref[auth:a187]{C. Wilson} & A Fast and Scalable Algorithm for Scheduling Large Numbers of Devices Under Real-Time Pricing & \href{../works/He0GLW18.pdf}{Yes} & \cite{He0GLW18} & 2018 & CP 2018 & 18 & 6 6 10 & 26 35 & \ref{b:He0GLW18} & \ref{c:He0GLW18}\\
Tesch18 \href{https://doi.org/10.1007/978-3-319-98334-9_41}{Tesch18} & \hyperref[auth:a183]{A. Tesch} & Improving Energetic Propagations for Cumulative Scheduling & \href{../works/Tesch18.pdf}{Yes} & \cite{Tesch18} & 2018 & CP 2018 & 17 & 5 6 7 & 21 22 & \ref{b:Tesch18} & n/a\\
BofillCSV17 \href{https://doi.org/10.1007/978-3-319-66158-2_5}{BofillCSV17} & \hyperref[auth:a228]{M. Bofill}, \hyperref[auth:a1449]{J. Coll}, \hyperref[auth:a232]{J. Suy}, \hyperref[auth:a233]{M. Villaret} & An Efficient {SMT} Approach to Solve MRCPSP/max Instances with Tight Constraints on Resources & \href{../works/BofillCSV17.pdf}{Yes} & \cite{BofillCSV17} & 2017 & CP 2017 & 9 & 1 1 5 & 12 17 & \ref{b:BofillCSV17} & n/a\\
GoldwaserS17 \href{https://doi.org/10.1007/978-3-319-66158-2_22}{GoldwaserS17} & \hyperref[auth:a189]{A. Goldwaser}, \hyperref[auth:a124]{A. Schutt} & Optimal Torpedo Scheduling & \href{../works/GoldwaserS17.pdf}{Yes} & \cite{GoldwaserS17} & 2017 & CP 2017 & 16 & 0 0 2 & 10 14 & \ref{b:GoldwaserS17} & \ref{c:GoldwaserS17}\\
Hooker17 \href{https://doi.org/10.1007/978-3-319-66158-2_36}{Hooker17} & \hyperref[auth:a160]{J. N. Hooker} & Job Sequencing Bounds from Decision Diagrams & \href{../works/Hooker17.pdf}{Yes} & \cite{Hooker17} & 2017 & CP 2017 & 14 & 6 6 11 & 24 27 & \ref{b:Hooker17} & n/a\\
LiuCGM17 \href{https://doi.org/10.1007/978-3-319-66158-2_24}{LiuCGM17} & \hyperref[auth:a190]{T. Liu}, \hyperref[auth:a191]{R. D. Cosmo}, \hyperref[auth:a192]{M. Gabbrielli}, \hyperref[auth:a193]{J. Mauro} & \cellcolor{green!10}NightSplitter: {A} Scheduling Tool to Optimize (Sub)group Activities & \href{../works/LiuCGM17.pdf}{Yes} & \cite{LiuCGM17} & 2017 & CP 2017 & 17 & 0 0 0 & 15 31 & \ref{b:LiuCGM17} & \ref{c:LiuCGM17}\\
MossigeGSMC17 \href{https://doi.org/10.1007/978-3-319-66158-2_25}{MossigeGSMC17} & \hyperref[auth:a194]{M. Mossige}, \hyperref[auth:a195]{A. Gotlieb}, \hyperref[auth:a196]{H. Spieker}, \hyperref[auth:a197]{H. Meling}, \hyperref[auth:a91]{M. Carlsson} & \cellcolor{green!10}Time-Aware Test Case Execution Scheduling for Cyber-Physical Systems & \href{../works/MossigeGSMC17.pdf}{Yes} & \cite{MossigeGSMC17} & 2017 & CP 2017 & 18 & 6 7 8 & 33 39 & \ref{b:MossigeGSMC17} & n/a\\
Pralet17 \href{https://doi.org/10.1007/978-3-319-66158-2_16}{Pralet17} & \hyperref[auth:a21]{C. Pralet} & An Incomplete Constraint-Based System for Scheduling with Renewable Resources & \href{../works/Pralet17.pdf}{Yes} & \cite{Pralet17} & 2017 & CP 2017 & 19 & 1 1 2 & 30 37 & \ref{b:Pralet17} & n/a\\
YoungFS17 \href{https://doi.org/10.1007/978-3-319-66158-2_20}{YoungFS17} & \hyperref[auth:a188]{K. D. Young}, \hyperref[auth:a154]{T. Feydy}, \hyperref[auth:a124]{A. Schutt} & Constraint Programming Applied to the Multi-Skill Project Scheduling Problem & \href{../works/YoungFS17.pdf}{Yes} & \cite{YoungFS17} & 2017 & CP 2017 & 10 & 6 8 13 & 21 28 & \ref{b:YoungFS17} & \ref{c:YoungFS17}\\
BonfiettiZLM16 \href{https://doi.org/10.1007/978-3-319-44953-1_8}{BonfiettiZLM16} & \hyperref[auth:a198]{A. Bonfietti}, \hyperref[auth:a199]{A. Zanarini}, \hyperref[auth:a142]{M. Lombardi}, \hyperref[auth:a143]{M. Milano} & The Multirate Resource Constraint & \href{../works/BonfiettiZLM16.pdf}{Yes} & \cite{BonfiettiZLM16} & 2016 & CP 2016 & 17 & 0 0 0 & 11 18 & \ref{b:BonfiettiZLM16} & \ref{c:BonfiettiZLM16}\\
BoothNB16 \href{https://doi.org/10.1007/978-3-319-44953-1_34}{BoothNB16} & \hyperref[auth:a203]{K. E. C. Booth}, \hyperref[auth:a204]{G. Nejat}, \hyperref[auth:a89]{J. C. Beck} & \cellcolor{green!10}A Constraint Programming Approach to Multi-Robot Task Allocation and Scheduling in Retirement Homes & \href{../works/BoothNB16.pdf}{Yes} & \cite{BoothNB16} & 2016 & CP 2016 & 17 & 21 21 32 & 24 31 & \ref{b:BoothNB16} & n/a\\
CauwelaertDMS16 \href{https://doi.org/10.1007/978-3-319-44953-1_33}{CauwelaertDMS16} & \hyperref[auth:a201]{S. V. Cauwelaert}, \hyperref[auth:a202]{C. Dejemeppe}, \hyperref[auth:a149]{J.-N. Monette}, \hyperref[auth:a147]{P. Schaus} & Efficient Filtering for the Unary Resource with Family-Based Transition Times & \href{../works/CauwelaertDMS16.pdf}{Yes} & \cite{CauwelaertDMS16} & 2016 & CP 2016 & 16 & 1 1 2 & 12 20 & \ref{b:CauwelaertDMS16} & \ref{c:CauwelaertDMS16}\\
GilesH16 \href{https://doi.org/10.1007/978-3-319-44953-1_38}{GilesH16} & \hyperref[auth:a205]{K. Giles}, \hyperref[auth:a206]{W.-J. van Hoeve} & Solving a Supply-Delivery Scheduling Problem with Constraint Programming & \href{../works/GilesH16.pdf}{Yes} & \cite{GilesH16} & 2016 & CP 2016 & 16 & 2 2 2 & 6 8 & \ref{b:GilesH16} & n/a\\
LimHTB16 \href{https://doi.org/10.1007/978-3-319-44953-1_43}{LimHTB16} & \hyperref[auth:a207]{B. Lim}, \hyperref[auth:a208]{H. L. Hijazi}, \hyperref[auth:a209]{S. Thi{\'{e}}baux}, \hyperref[auth:a210]{M. van den Briel} & Online HVAC-Aware Occupancy Scheduling with Adaptive Temperature Control & \href{../works/LimHTB16.pdf}{Yes} & \cite{LimHTB16} & 2016 & CP 2016 & 18 & 2 2 7 & 23 32 & \ref{b:LimHTB16} & n/a\\
SchuttS16 \href{https://doi.org/10.1007/978-3-319-44953-1_28}{SchuttS16} & \hyperref[auth:a124]{A. Schutt}, \hyperref[auth:a125]{P. J. Stuckey} & Explaining Producer/Consumer Constraints & \href{../works/SchuttS16.pdf}{Yes} & \cite{SchuttS16} & 2016 & CP 2016 & 17 & 3 3 3 & 23 23 & \ref{b:SchuttS16} & n/a\\
SzerediS16 \href{https://doi.org/10.1007/978-3-319-44953-1_31}{SzerediS16} & \hyperref[auth:a200]{R. Szeredi}, \hyperref[auth:a124]{A. Schutt} & Modelling and Solving Multi-mode Resource-Constrained Project Scheduling & \href{../works/SzerediS16.pdf}{Yes} & \cite{SzerediS16} & 2016 & CP 2016 & 10 & 9 9 15 & 14 16 & \ref{b:SzerediS16} & n/a\\
Tesch16 \href{https://doi.org/10.1007/978-3-319-44953-1_32}{Tesch16} & \hyperref[auth:a183]{A. Tesch} & A Nearly Exact Propagation Algorithm for Energetic Reasoning in {\textbackslash}mathcal O(n{\^{}}2 {\textbackslash}log n) & \href{../works/Tesch16.pdf}{Yes} & \cite{Tesch16} & 2016 & CP 2016 & 27 & 4 5 4 & 14 18 & \ref{b:Tesch16} & n/a\\
DejemeppeCS15 \href{https://doi.org/10.1007/978-3-319-23219-5_7}{DejemeppeCS15} & \hyperref[auth:a202]{C. Dejemeppe}, \hyperref[auth:a201]{S. V. Cauwelaert}, \hyperref[auth:a147]{P. Schaus} & The Unary Resource with Transition Times & \href{../works/DejemeppeCS15.pdf}{Yes} & \cite{DejemeppeCS15} & 2015 & CP 2015 & 16 & 5 5 8 & 11 21 & \ref{b:DejemeppeCS15} & \ref{c:DejemeppeCS15}\\
EvenSH15 \href{https://doi.org/10.1007/978-3-319-23219-5_40}{EvenSH15} & \hyperref[auth:a214]{C. Even}, \hyperref[auth:a124]{A. Schutt}, \hyperref[auth:a148]{P. V. Hentenryck} & \cellcolor{green!10}A Constraint Programming Approach for Non-preemptive Evacuation Scheduling & \href{../works/EvenSH15.pdf}{Yes} & \cite{EvenSH15} & 2015 & CP 2015 & 18 & 3 2 6 & 12 14 & \ref{b:EvenSH15} & n/a\\
GayHLS15 \href{https://doi.org/10.1007/978-3-319-23219-5_10}{GayHLS15} & \hyperref[auth:a211]{S. Gay}, \hyperref[auth:a212]{R. Hartert}, \hyperref[auth:a213]{C. Lecoutre}, \hyperref[auth:a147]{P. Schaus} & \cellcolor{green!10}Conflict Ordering Search for Scheduling Problems & \href{../works/GayHLS15.pdf}{Yes} & \cite{GayHLS15} & 2015 & CP 2015 & 9 & 20 20 31 & 15 19 & \ref{b:GayHLS15} & \ref{c:GayHLS15}\\
GayHS15 \href{https://doi.org/10.1007/978-3-319-23219-5_11}{GayHS15} & \hyperref[auth:a211]{S. Gay}, \hyperref[auth:a212]{R. Hartert}, \hyperref[auth:a147]{P. Schaus} & \cellcolor{green!10}Simple and Scalable Time-Table Filtering for the Cumulative Constraint & \href{../works/GayHS15.pdf}{Yes} & \cite{GayHS15} & 2015 & CP 2015 & 9 & 10 10 17 & 9 15 & \ref{b:GayHS15} & \ref{c:GayHS15}\\
KreterSS15 \href{https://doi.org/10.1007/978-3-319-23219-5_19}{KreterSS15} & \hyperref[auth:a123]{S. Kreter}, \hyperref[auth:a124]{A. Schutt}, \hyperref[auth:a125]{P. J. Stuckey} & Modeling and Solving Project Scheduling with Calendars & \href{../works/KreterSS15.pdf}{Yes} & \cite{KreterSS15} & 2015 & CP 2015 & 17 & 7 7 9 & 16 23 & \ref{b:KreterSS15} & n/a\\
LombardiBM15 \href{https://doi.org/10.1007/978-3-319-23219-5_20}{LombardiBM15} & \hyperref[auth:a142]{M. Lombardi}, \hyperref[auth:a198]{A. Bonfietti}, \hyperref[auth:a143]{M. Milano} & Deterministic Estimation of the Expected Makespan of a {POS} Under Duration Uncertainty & \href{../works/LombardiBM15.pdf}{Yes} & \cite{LombardiBM15} & 2015 & CP 2015 & 16 & 0 0 0 & 8 15 & \ref{b:LombardiBM15} & n/a\\
MurphyMB15 \href{https://doi.org/10.1007/978-3-319-23219-5_47}{MurphyMB15} & \hyperref[auth:a215]{S. {\'{O}}g Murphy}, \hyperref[auth:a216]{O. Manzano}, \hyperref[auth:a217]{K. N. Brown} & Design and Evaluation of a Constraint-Based Energy Saving and Scheduling Recommender System & \href{../works/MurphyMB15.pdf}{Yes} & \cite{MurphyMB15} & 2015 & CP 2015 & 17 & 1 3 6 & 20 26 & \ref{b:MurphyMB15} & n/a\\
PraletLJ15 \href{https://doi.org/10.1007/978-3-319-23219-5_48}{PraletLJ15} & \hyperref[auth:a21]{C. Pralet}, \hyperref[auth:a218]{S. Lemai-Chenevier}, \hyperref[auth:a219]{J. Jaubert} & Scheduling Running Modes of Satellite Instruments Using Constraint-Based Local Search & \href{../works/PraletLJ15.pdf}{Yes} & \cite{PraletLJ15} & 2015 & CP 2015 & 16 & 0 0 0 & 8 11 & \ref{b:PraletLJ15} & n/a\\
SialaAH15 \href{https://doi.org/10.1007/978-3-319-23219-5_28}{SialaAH15} & \hyperref[auth:a129]{M. Siala}, \hyperref[auth:a6]{C. Artigues}, \hyperref[auth:a1]{E. Hebrard} & \cellcolor{green!10}Two Clause Learning Approaches for Disjunctive Scheduling & \href{../works/SialaAH15.pdf}{Yes} & \cite{SialaAH15} & 2015 & CP 2015 & 10 & 4 4 4 & 17 27 & \ref{b:SialaAH15} & \ref{c:SialaAH15}\\
AlesioNBG14 \href{https://doi.org/10.1007/978-3-319-10428-7_58}{AlesioNBG14} & \hyperref[auth:a234]{S. {Di Alesio}}, \hyperref[auth:a235]{S. Nejati}, \hyperref[auth:a236]{L. C. Briand}, \hyperref[auth:a195]{A. Gotlieb} & Worst-Case Scheduling of Software Tasks - {A} Constraint Optimization Model to Support Performance Testing & \href{../works/AlesioNBG14.pdf}{Yes} & \cite{AlesioNBG14} & 2014 & CP 2014 & 18 & 3 2 3 & 19 28 & \ref{b:AlesioNBG14} & n/a\\
BartoliniBBLM14 \href{https://doi.org/10.1007/978-3-319-10428-7_55}{BartoliniBBLM14} & \hyperref[auth:a225]{A. Bartolini}, \hyperref[auth:a226]{A. Borghesi}, \hyperref[auth:a227]{T. Bridi}, \hyperref[auth:a142]{M. Lombardi}, \hyperref[auth:a143]{M. Milano} & Proactive Workload Dispatching on the {EURORA} Supercomputer & \href{../works/BartoliniBBLM14.pdf}{Yes} & \cite{BartoliniBBLM14} & 2014 & CP 2014 & 16 & 12 12 16 & 3 9 & \ref{b:BartoliniBBLM14} & n/a\\
BofillEGPSV14 \href{https://doi.org/10.1007/978-3-319-10428-7_56}{BofillEGPSV14} & \hyperref[auth:a228]{M. Bofill}, \hyperref[auth:a229]{J. Espasa}, \hyperref[auth:a230]{M. Garcia}, \hyperref[auth:a231]{M. Palah{\'{\i}}}, \hyperref[auth:a232]{J. Suy}, \hyperref[auth:a233]{M. Villaret} & Scheduling {B2B} Meetings & \href{../works/BofillEGPSV14.pdf}{Yes} & \cite{BofillEGPSV14} & 2014 & CP 2014 & 16 & 3 4 10 & 10 17 & \ref{b:BofillEGPSV14} & n/a\\
DerrienP14 \href{https://doi.org/10.1007/978-3-319-10428-7_22}{DerrienP14} & \hyperref[auth:a220]{A. Derrien}, \hyperref[auth:a221]{T. Petit} & \cellcolor{green!10}A New Characterization of Relevant Intervals for Energetic Reasoning & \href{../works/DerrienP14.pdf}{Yes} & \cite{DerrienP14} & 2014 & CP 2014 & 9 & 14 15 14 & 0 0 & \ref{b:DerrienP14} & n/a\\
DerrienPZ14 \href{https://doi.org/10.1007/978-3-319-10428-7_23}{DerrienPZ14} & \hyperref[auth:a220]{A. Derrien}, \hyperref[auth:a221]{T. Petit}, \hyperref[auth:a222]{S. Zampelli} & \cellcolor{green!10}A Declarative Paradigm for Robust Cumulative Scheduling & \href{../works/DerrienPZ14.pdf}{Yes} & \cite{DerrienPZ14} & 2014 & CP 2014 & 9 & 3 3 3 & 10 17 & \ref{b:DerrienPZ14} & n/a\\
GaySS14 \href{https://doi.org/10.1007/978-3-319-10428-7_59}{GaySS14} & \hyperref[auth:a211]{S. Gay}, \hyperref[auth:a147]{P. Schaus}, \hyperref[auth:a237]{V. D. Smedt} & \cellcolor{green!10}Continuous Casting Scheduling with Constraint Programming & \href{../works/GaySS14.pdf}{Yes} & \cite{GaySS14} & 2014 & CP 2014 & 15 & 7 9 11 & 11 19 & \ref{b:GaySS14} & n/a\\
HoundjiSWD14 \href{https://doi.org/10.1007/978-3-319-10428-7_29}{HoundjiSWD14} & \hyperref[auth:a223]{V. R. Houndji}, \hyperref[auth:a147]{P. Schaus}, \hyperref[auth:a224]{L. A. Wolsey}, \hyperref[auth:a151]{Y. Deville} & \cellcolor{green!10}The StockingCost Constraint & \href{../works/HoundjiSWD14.pdf}{Yes} & \cite{HoundjiSWD14} & 2014 & CP 2014 & 16 & 5 5 8 & 7 13 & \ref{b:HoundjiSWD14} & \ref{c:HoundjiSWD14}\\
OuelletQ13 \href{https://doi.org/10.1007/978-3-642-40627-0_42}{OuelletQ13} & \hyperref[auth:a238]{P. Ouellet}, \hyperref[auth:a37]{C.-G. Quimper} & Time-Table Extended-Edge-Finding for the Cumulative Constraint & \href{../works/OuelletQ13.pdf}{Yes} & \cite{OuelletQ13} & 2013 & CP 2013 & 16 & 12 12 18 & 14 16 & \ref{b:OuelletQ13} & n/a\\
SchuttFS13 \href{https://doi.org/10.1007/978-3-642-40627-0_47}{SchuttFS13} & \hyperref[auth:a124]{A. Schutt}, \hyperref[auth:a154]{T. Feydy}, \hyperref[auth:a125]{P. J. Stuckey} & Scheduling Optional Tasks with Explanation & \href{../works/SchuttFS13.pdf}{Yes} & \cite{SchuttFS13} & 2013 & CP 2013 & 17 & 10 10 13 & 20 32 & \ref{b:SchuttFS13} & n/a\\
ZampelliVSDR13 \href{https://doi.org/10.1007/978-3-642-40627-0_64}{ZampelliVSDR13} & \hyperref[auth:a222]{S. Zampelli}, \hyperref[auth:a1207]{Y. Vergados}, \hyperref[auth:a1208]{R. V. Schaeren}, \hyperref[auth:a1209]{W. Dullaert}, \hyperref[auth:a1210]{B. Raa} & The Berth Allocation and Quay Crane Assignment Problem Using a {CP} Approach & \href{../works/ZampelliVSDR13.pdf}{Yes} & \cite{ZampelliVSDR13} & 2013 & CP 2013 & 17 & 20 21 31 & 19 23 & \ref{b:ZampelliVSDR13} & n/a\\
GuSW12 \href{https://doi.org/10.1007/978-3-642-33558-7_55}{GuSW12} & \hyperref[auth:a336]{H. Gu}, \hyperref[auth:a125]{P. J. Stuckey}, \hyperref[auth:a117]{M. G. Wallace} & Maximising the Net Present Value of Large Resource-Constrained Projects & \href{../works/GuSW12.pdf}{Yes} & \cite{GuSW12} & 2012 & CP 2012 & 15 & 5 5 5 & 20 25 & \ref{b:GuSW12} & n/a\\
IfrimOS12 \href{https://doi.org/10.1007/978-3-642-33558-7_68}{IfrimOS12} & \hyperref[auth:a182]{G. Ifrim}, \hyperref[auth:a16]{B. O'Sullivan}, \hyperref[auth:a17]{H. Simonis} & Properties of Energy-Price Forecasts for Scheduling & \href{../works/IfrimOS12.pdf}{Yes} & \cite{IfrimOS12} & 2012 & CP 2012 & 16 & 6 5 13 & 20 29 & \ref{b:IfrimOS12} & n/a\\
LetortBC12 \href{https://doi.org/10.1007/978-3-642-33558-7_33}{LetortBC12} & \hyperref[auth:a127]{A. Letort}, \hyperref[auth:a128]{N. Beldiceanu}, \hyperref[auth:a91]{M. Carlsson} & A Scalable Sweep Algorithm for the cumulative Constraint & \href{../works/LetortBC12.pdf}{Yes} & \cite{LetortBC12} & 2012 & CP 2012 & 16 & 18 19 27 & 12 21 & \ref{b:LetortBC12} & n/a\\
LozanoCDS12 \href{https://doi.org/10.1007/978-3-642-33558-7_54}{LozanoCDS12} & \hyperref[auth:a1226]{R. C. Lozano}, \hyperref[auth:a91]{M. Carlsson}, \hyperref[auth:a1227]{F. Drejhammar}, \hyperref[auth:a92]{C. Schulte} & \cellcolor{green!10}Constraint-Based Register Allocation and Instruction Scheduling & \href{../works/LozanoCDS12.pdf}{Yes} & \cite{LozanoCDS12} & 2012 & CP 2012 & 17 & 21 17 25 & 30 36 & \ref{b:LozanoCDS12} & n/a\\
SerraNM12 \href{https://doi.org/10.1007/978-3-642-33558-7_59}{SerraNM12} & \hyperref[auth:a239]{T. Serra}, \hyperref[auth:a240]{G. Nishioka}, \hyperref[auth:a241]{F. J. M. Marcellino} & The Offshore Resources Scheduling Problem: Detailing a Constraint Programming Approach & \href{../works/SerraNM12.pdf}{Yes} & \cite{SerraNM12} & 2012 & CP 2012 & 17 & 0 0 2 & 8 24 & \ref{b:SerraNM12} & n/a\\
SimoninAHL12 \href{https://doi.org/10.1007/978-3-642-33558-7_5}{SimoninAHL12} & \hyperref[auth:a126]{G. Simonin}, \hyperref[auth:a6]{C. Artigues}, \hyperref[auth:a1]{E. Hebrard}, \hyperref[auth:a3]{P. Lopez} & \cellcolor{green!10}Scheduling Scientific Experiments on the Rosetta/Philae Mission & \href{../works/SimoninAHL12.pdf}{Yes} & \cite{SimoninAHL12} & 2012 & CP 2012 & 15 & 3 3 4 & 8 10 & \ref{b:SimoninAHL12} & \ref{c:SimoninAHL12}\\
BonfiettiLBM11 \href{https://doi.org/10.1007/978-3-642-23786-7_12}{BonfiettiLBM11} & \hyperref[auth:a198]{A. Bonfietti}, \hyperref[auth:a142]{M. Lombardi}, \hyperref[auth:a245]{L. Benini}, \hyperref[auth:a143]{M. Milano} & A Constraint Based Approach to Cyclic {RCPSP} & \href{../works/BonfiettiLBM11.pdf}{Yes} & \cite{BonfiettiLBM11} & 2011 & CP 2011 & 15 & 3 3 4 & 14 24 & \ref{b:BonfiettiLBM11} & n/a\\
ClercqPBJ11 \href{https://doi.org/10.1007/978-3-642-23786-7_20}{ClercqPBJ11} & \hyperref[auth:a246]{A. D. Clercq}, \hyperref[auth:a221]{T. Petit}, \hyperref[auth:a128]{N. Beldiceanu}, \hyperref[auth:a247]{N. Jussien} & Filtering Algorithms for Discrete Cumulative Problems with Overloads of Resource & \href{../works/ClercqPBJ11.pdf}{Yes} & \cite{ClercqPBJ11} & 2011 & CP 2011 & 16 & 3 3 4 & 11 13 & \ref{b:ClercqPBJ11} & n/a\\
GrimesH11 \href{https://doi.org/10.1007/978-3-642-23786-7_28}{GrimesH11} & \hyperref[auth:a181]{D. Grimes}, \hyperref[auth:a1]{E. Hebrard} & \cellcolor{green!10}Models and Strategies for Variants of the Job Shop Scheduling Problem & \href{../works/GrimesH11.pdf}{Yes} & \cite{GrimesH11} & 2011 & CP 2011 & 17 & 5 5 11 & 18 27 & \ref{b:GrimesH11} & n/a\\
HermenierDL11 \href{https://doi.org/10.1007/978-3-642-23786-7_5}{HermenierDL11} & \hyperref[auth:a242]{F. Hermenier}, \hyperref[auth:a243]{S. Demassey}, \hyperref[auth:a244]{X. Lorca} & Bin Repacking Scheduling in Virtualized Datacenters & \href{../works/HermenierDL11.pdf}{Yes} & \cite{HermenierDL11} & 2011 & CP 2011 & 15 & 28 26 40 & 5 9 & \ref{b:HermenierDL11} & n/a\\
KameugneFSN11 \href{https://doi.org/10.1007/978-3-642-23786-7_37}{KameugneFSN11} & \hyperref[auth:a10]{R. Kameugne}, \hyperref[auth:a130]{L. P. Fotso}, \hyperref[auth:a131]{J. D. Scott}, \hyperref[auth:a132]{Y. Ngo-Kateu} & A Quadratic Edge-Finding Filtering Algorithm for Cumulative Resource Constraints & \href{../works/KameugneFSN11.pdf}{Yes} & \cite{KameugneFSN11} & 2011 & CP 2011 & 15 & 7 8 7 & 9 14 & \ref{b:KameugneFSN11} & n/a\\
Beck10 \href{https://doi.org/10.1007/978-3-642-15396-9_10}{Beck10} & \hyperref[auth:a89]{J. C. Beck} & \cellcolor{green!10}Checking-Up on Branch-and-Check & \href{../works/Beck10.pdf}{Yes} & \cite{Beck10} & 2010 & CP 2010 & 15 & 19 21 45 & 11 12 & \ref{b:Beck10} & n/a\\
LombardiM10 \href{https://doi.org/10.1007/978-3-642-15396-9_32}{LombardiM10} & \hyperref[auth:a142]{M. Lombardi}, \hyperref[auth:a143]{M. Milano} & Constraint Based Scheduling to Deal with Uncertain Durations and Self-Timed Execution & \href{../works/LombardiM10.pdf}{Yes} & \cite{LombardiM10} & 2010 & CP 2010 & 15 & 1 1 0 & 11 17 & \ref{b:LombardiM10} & n/a\\
SchuttW10 \href{https://doi.org/10.1007/978-3-642-15396-9_36}{SchuttW10} & \hyperref[auth:a124]{A. Schutt}, \hyperref[auth:a51]{A. Wolf} & A New \emph{O}(\emph{n}\({}^{\mbox{2}}\)log\emph{n}) Not-First/Not-Last Pruning Algorithm for Cumulative Resource Constraints & \href{../works/SchuttW10.pdf}{Yes} & \cite{SchuttW10} & 2010 & CP 2010 & 15 & 13 14 16 & 14 19 & \ref{b:SchuttW10} & n/a\\
Baptiste09 \href{https://doi.org/10.1007/978-3-642-04244-7_1}{Baptiste09} & \hyperref[auth:a162]{P. Baptiste} & Constraint-Based Schedulers, Do They Really Work? & \href{../works/Baptiste09.pdf}{Yes} & \cite{Baptiste09} & 2009 & CP 2009 & 1 & 0 0 0 & 0 0 & \ref{b:Baptiste09} & n/a\\
GrimesHM09 \href{https://doi.org/10.1007/978-3-642-04244-7_33}{GrimesHM09} & \hyperref[auth:a181]{D. Grimes}, \hyperref[auth:a1]{E. Hebrard}, \hyperref[auth:a82]{A. Malapert} & Closing the Open Shop: Contradicting Conventional Wisdom & \href{../works/GrimesHM09.pdf}{Yes} & \cite{GrimesHM09} & 2009 & CP 2009 & 9 & 15 15 18 & 12 23 & \ref{b:GrimesHM09} & n/a\\
LombardiM09 \href{https://doi.org/10.1007/978-3-642-04244-7_45}{LombardiM09} & \hyperref[auth:a142]{M. Lombardi}, \hyperref[auth:a143]{M. Milano} & A Precedence Constraint Posting Approach for the {RCPSP} with Time Lags and Variable Durations & \href{../works/LombardiM09.pdf}{Yes} & \cite{LombardiM09} & 2009 & CP 2009 & 15 & 7 7 15 & 12 20 & \ref{b:LombardiM09} & n/a\\
SchuttFSW09 \href{https://doi.org/10.1007/978-3-642-04244-7_58}{SchuttFSW09} & \hyperref[auth:a124]{A. Schutt}, \hyperref[auth:a154]{T. Feydy}, \hyperref[auth:a125]{P. J. Stuckey}, \hyperref[auth:a117]{M. G. Wallace} & Why Cumulative Decomposition Is Not as Bad as It Sounds & \href{../works/SchuttFSW09.pdf}{Yes} & \cite{SchuttFSW09} & 2009 & CP 2009 & 16 & 34 34 44 & 11 18 & \ref{b:SchuttFSW09} & n/a\\
Vilim09 \href{https://doi.org/10.1007/978-3-642-04244-7_62}{Vilim09} & \hyperref[auth:a121]{P. Vil{\'{\i}}m} & Edge Finding Filtering Algorithm for Discrete Cumulative Resources in \emph{O}(\emph{kn} log \emph{n})\{{\textbackslash}mathcal O\}(kn \{{\textbackslash}rm log\} n) & \href{../works/Vilim09.pdf}{Yes} & \cite{Vilim09} & 2009 & CP 2009 & 15 & 25 26 34 & 4 7 & \ref{b:Vilim09} & n/a\\
BeniniLMR08 \href{http://dx.doi.org/10.1007/978-3-540-85958-1_2}{BeniniLMR08} & \hyperref[auth:a245]{L. Benini}, \hyperref[auth:a142]{M. Lombardi}, \hyperref[auth:a143]{M. Milano}, \hyperref[auth:a718]{M. Ruggiero} & A Constraint Programming Approach for Allocation and Scheduling on the CELL Broadband Engine & \href{../works/BeniniLMR08.pdf}{Yes} & \cite{BeniniLMR08} & 2008 & CP 2008 & 15 & 7 6 7 & 23 34 & \ref{b:BeniniLMR08} & n/a\\
MouraSCL08 \href{https://doi.org/10.1007/978-3-540-85958-1_3}{MouraSCL08} & \hyperref[auth:a159]{A. V. Moura}, \hyperref[auth:a170]{C. C. de Souza}, \hyperref[auth:a157]{A. A. Cir{\'{e}}}, \hyperref[auth:a156]{T. M. T. Lopes} & Planning and Scheduling the Operation of a Very Large Oil Pipeline Network & \href{../works/MouraSCL08.pdf}{Yes} & \cite{MouraSCL08} & 2008 & CP 2008 & 16 & 11 11 22 & 10 14 & \ref{b:MouraSCL08} & n/a\\
DavenportKRSH07 \href{https://doi.org/10.1007/978-3-540-74970-7_7}{DavenportKRSH07} & \hyperref[auth:a248]{A. J. Davenport}, \hyperref[auth:a249]{J. Kalagnanam}, \hyperref[auth:a250]{C. Reddy}, \hyperref[auth:a251]{S. Siegel}, \hyperref[auth:a252]{J. Hou} & \cellcolor{gold!20}An Application of Constraint Programming to Generating Detailed Operations Schedules for Steel Manufacturing & \href{../works/DavenportKRSH07.pdf}{Yes} & \cite{DavenportKRSH07} & 2007 & CP 2007 & 13 & 1 1 5 & 2 13 & \ref{b:DavenportKRSH07} & n/a\\
GarganiR07 \href{https://doi.org/10.1007/978-3-540-74970-7_8}{GarganiR07} & \hyperref[auth:a253]{A. Gargani}, \hyperref[auth:a254]{P. Refalo} & An Efficient Model and Strategy for the Steel Mill Slab Design Problem & \href{../works/GarganiR07.pdf}{Yes} & \cite{GarganiR07} & 2007 & CP 2007 & 13 & 17 18 28 & 5 12 & \ref{b:GarganiR07} & n/a\\
KrogtLPHJ07 \href{https://doi.org/10.1007/978-3-540-74970-7_10}{KrogtLPHJ07} & \hyperref[auth:a255]{R. van der Krogt}, \hyperref[auth:a178]{J. Little}, \hyperref[auth:a256]{K. Pulliam}, \hyperref[auth:a257]{S. Hanhilammi}, \hyperref[auth:a258]{Y. Jin} & Scheduling for Cellular Manufacturing & \href{../works/KrogtLPHJ07.pdf}{Yes} & \cite{KrogtLPHJ07} & 2007 & CP 2007 & 13 & 2 2 3 & 3 9 & \ref{b:KrogtLPHJ07} & n/a\\
KhemmoudjPB06 \href{https://doi.org/10.1007/11889205_21}{KhemmoudjPB06} & \hyperref[auth:a259]{M. O. I. Khemmoudj}, \hyperref[auth:a260]{M. Porcheron}, \hyperref[auth:a261]{H. Bennaceur} & When Constraint Programming and Local Search Solve the Scheduling Problem of Electricit{\'{e}} de France Nuclear Power Plant Outages & \href{../works/KhemmoudjPB06.pdf}{Yes} & \cite{KhemmoudjPB06} & 2006 & CP 2006 & 13 & 8 8 15 & 8 12 & \ref{b:KhemmoudjPB06} & n/a\\
AbrilSB05 \href{https://doi.org/10.1007/11564751_75}{AbrilSB05} & \hyperref[auth:a270]{M. Abril}, \hyperref[auth:a153]{M. A. Salido}, \hyperref[auth:a271]{F. Barber} & \cellcolor{green!10}Distributed Constraints for Large-Scale Scheduling Problems & \href{../works/AbrilSB05.pdf}{Yes} & \cite{AbrilSB05} & 2005 & CP 2005 & 1 & 0 0 0 & 0 2 & \ref{b:AbrilSB05} & n/a\\
ArtiouchineB05 \href{https://doi.org/10.1007/11564751_8}{ArtiouchineB05} & \hyperref[auth:a262]{K. Artiouchine}, \hyperref[auth:a162]{P. Baptiste} & \cellcolor{green!10}Inter-distance Constraint: An Extension of the All-Different Constraint for Scheduling Equal Length Jobs & \href{../works/ArtiouchineB05.pdf}{Yes} & \cite{ArtiouchineB05} & 2005 & CP 2005 & 15 & 3 3 9 & 11 24 & \ref{b:ArtiouchineB05} & n/a\\
BeniniBGM05 \href{https://doi.org/10.1007/11564751_11}{BeniniBGM05} & \hyperref[auth:a245]{L. Benini}, \hyperref[auth:a375]{D. Bertozzi}, \hyperref[auth:a376]{A. Guerri}, \hyperref[auth:a143]{M. Milano} & \cellcolor{green!10}Allocation and Scheduling for MPSoCs via Decomposition and No-Good Generation & \href{../works/BeniniBGM05.pdf}{Yes} & \cite{BeniniBGM05} & 2005 & CP 2005 & 15 & 25 25 41 & 21 29 & \ref{b:BeniniBGM05} & n/a\\
CambazardJ05 \href{https://doi.org/10.1007/11564751_58}{CambazardJ05} & \hyperref[auth:a999]{H. Cambazard}, \hyperref[auth:a247]{N. Jussien} & Integrating Benders Decomposition Within Constraint Programming & \href{../works/CambazardJ05.pdf}{Yes} & \cite{CambazardJ05} & 2005 & CP 2005 & 5 & 6 6 7 & 8 8 & \ref{b:CambazardJ05} & n/a\\
CarchraeBF05 \href{https://doi.org/10.1007/11564751_80}{CarchraeBF05} & \hyperref[auth:a272]{T. Carchrae}, \hyperref[auth:a89]{J. C. Beck}, \hyperref[auth:a273]{E. C. Freuder} & \cellcolor{green!10}Methods to Learn Abstract Scheduling Models & \href{../works/CarchraeBF05.pdf}{Yes} & \cite{CarchraeBF05} & 2005 & CP 2005 & 1 & 0 0 0 & 0 0 & \ref{b:CarchraeBF05} & n/a\\
DilkinaDH05 \href{https://doi.org/10.1007/11564751_60}{DilkinaDH05} & \hyperref[auth:a267]{B. N. Dilkina}, \hyperref[auth:a268]{L. Duan}, \hyperref[auth:a269]{W. S. Havens} & Extending Systematic Local Search for Job Shop Scheduling Problems & \href{../works/DilkinaDH05.pdf}{Yes} & \cite{DilkinaDH05} & 2005 & CP 2005 & 5 & 2 2 4 & 7 8 & \ref{b:DilkinaDH05} & n/a\\
FortinZDF05 \href{https://doi.org/10.1007/11564751_19}{FortinZDF05} & \hyperref[auth:a263]{J. Fortin}, \hyperref[auth:a264]{P. Zielinski}, \hyperref[auth:a265]{D. Dubois}, \hyperref[auth:a266]{H. Fargier} & Interval Analysis in Scheduling & \href{../works/FortinZDF05.pdf}{Yes} & \cite{FortinZDF05} & 2005 & CP 2005 & 15 & 13 12 24 & 11 15 & \ref{b:FortinZDF05} & n/a\\
HebrardTW05 \href{https://doi.org/10.1007/11564751_117}{HebrardTW05} & \hyperref[auth:a1]{E. Hebrard}, \hyperref[auth:a275]{P. Tyler}, \hyperref[auth:a276]{T. Walsh} & Computing Super-Schedules & \href{../works/HebrardTW05.pdf}{Yes} & \cite{HebrardTW05} & 2005 & CP 2005 & 1 & 0 0 1 & 3 5 & \ref{b:HebrardTW05} & n/a\\
Hooker05a \href{https://doi.org/10.1007/11564751_25}{Hooker05a} & \hyperref[auth:a160]{J. N. Hooker} & \cellcolor{green!10}Planning and Scheduling to Minimize Tardiness & \href{../works/Hooker05a.pdf}{Yes} & \cite{Hooker05a} & 2005 & CP 2005 & 14 & 30 31 35 & 10 12 & \ref{b:Hooker05a} & n/a\\
KovacsEKV05 \href{https://doi.org/10.1007/11564751_118}{KovacsEKV05} & \hyperref[auth:a146]{A. Kov{\'{a}}cs}, \hyperref[auth:a277]{P. Egri}, \hyperref[auth:a155]{T. Kis}, \hyperref[auth:a278]{J. V{\'{a}}ncza} & Proterv-II: An Integrated Production Planning and Scheduling System & \href{../works/KovacsEKV05.pdf}{Yes} & \cite{KovacsEKV05} & 2005 & CP 2005 & 1 & 2 2 1 & 3 3 & \ref{b:KovacsEKV05} & n/a\\
Perron05 \href{https://doi.org/10.1007/11564751_67}{Perron05} & \hyperref[auth:a288]{L. Perron} & Alternate modeling in sport scheduling & \href{../works/Perron05.pdf}{Yes} & \cite{Perron05} & 2005 & CP 2005 & 5 & 1 1 2 & 3 9 & \ref{b:Perron05} & n/a\\
WuBB05 \href{https://doi.org/10.1007/11564751_110}{WuBB05} & \hyperref[auth:a274]{C. W. Wu}, \hyperref[auth:a217]{K. N. Brown}, \hyperref[auth:a89]{J. C. Beck} & Scheduling with Uncertain Start Dates & \href{../works/WuBB05.pdf}{Yes} & \cite{WuBB05} & 2005 & CP 2005 & 1 & 0 0 0 & 0 2 & \ref{b:WuBB05} & n/a\\
CambazardHDJT04 \href{https://doi.org/10.1007/978-3-540-30201-8_14}{CambazardHDJT04} & \hyperref[auth:a999]{H. Cambazard}, \hyperref[auth:a1061]{P.-E. Hladik}, \hyperref[auth:a1062]{A.-M. D{\'{e}}planche}, \hyperref[auth:a247]{N. Jussien}, \hyperref[auth:a1063]{Y. Trinquet} & Decomposition and Learning for a Hard Real Time Task Allocation Problem & \href{../works/CambazardHDJT04.pdf}{Yes} & \cite{CambazardHDJT04} & 2004 & CP 2004 & 15 & 33 33 45 & 13 23 & \ref{b:CambazardHDJT04} & n/a\\
Hooker04 \href{https://doi.org/10.1007/978-3-540-30201-8_24}{Hooker04} & \hyperref[auth:a160]{J. N. Hooker} & \cellcolor{green!10}A Hybrid Method for Planning and Scheduling & \href{../works/Hooker04.pdf}{Yes} & \cite{Hooker04} & 2004 & CP 2004 & 12 & 39 40 46 & 9 10 & \ref{b:Hooker04} & n/a\\
KovacsV04 \href{https://doi.org/10.1007/978-3-540-30201-8_26}{KovacsV04} & \hyperref[auth:a146]{A. Kov{\'{a}}cs}, \hyperref[auth:a278]{J. V{\'{a}}ncza} & \cellcolor{green!10}Completable Partial Solutions in Constraint Programming and Constraint-Based Scheduling & \href{../works/KovacsV04.pdf}{Yes} & \cite{KovacsV04} & 2004 & CP 2004 & 15 & 3 3 4 & 12 21 & \ref{b:KovacsV04} & n/a\\
LimRX04 \href{https://doi.org/10.1007/978-3-540-30201-8_59}{LimRX04} & \hyperref[auth:a279]{A. Lim}, \hyperref[auth:a280]{B. Rodrigues}, \hyperref[auth:a281]{Z. Xu} & Solving the Crane Scheduling Problem Using Intelligent Search Schemes & \href{../works/LimRX04.pdf}{Yes} & \cite{LimRX04} & 2004 & CP 2004 & 5 & 5 6 8 & 6 9 & \ref{b:LimRX04} & n/a\\
PerronSF04 \href{https://doi.org/10.1007/978-3-540-30201-8_35}{PerronSF04} & \hyperref[auth:a288]{L. Perron}, \hyperref[auth:a120]{P. Shaw}, \hyperref[auth:a1072]{V. Furnon} & Propagation Guided Large Neighborhood Search & \href{../works/PerronSF04.pdf}{Yes} & \cite{PerronSF04} & 2004 & CP 2004 & 14 & 34 34 67 & 8 16 & \ref{b:PerronSF04} & n/a\\
VilimBC04 \href{https://doi.org/10.1007/978-3-540-30201-8_8}{VilimBC04} & \hyperref[auth:a121]{P. Vil{\'{\i}}m}, \hyperref[auth:a152]{R. Bart{\'{a}}k}, \hyperref[auth:a161]{O. Cepek} & Unary Resource Constraint with Optional Activities & \href{../works/VilimBC04.pdf}{Yes} & \cite{VilimBC04} & 2004 & CP 2004 & 15 & 13 12 17 & 4 11 & \ref{b:VilimBC04} & n/a\\
BourdaisGP03 \href{https://doi.org/10.1007/978-3-540-45193-8_11}{BourdaisGP03} & \hyperref[auth:a1205]{S. Bourdais}, \hyperref[auth:a1206]{P. Galinier}, \hyperref[auth:a8]{G. Pesant} & {HIBISCUS:} {A} Constraint Programming Application to Staff Scheduling in Health Care & \href{../works/BourdaisGP03.pdf}{Yes} & \cite{BourdaisGP03} & 2003 & CP 2003 & 15 & 29 30 51 & 5 19 & \ref{b:BourdaisGP03} & n/a\\
DannaP03 \href{https://doi.org/10.1007/978-3-540-45193-8_59}{DannaP03} & \hyperref[auth:a287]{E. Danna}, \hyperref[auth:a288]{L. Perron} & Structured vs. Unstructured Large Neighborhood Search: {A} Case Study on Job-Shop Scheduling Problems with Earliness and Tardiness Costs & \href{../works/DannaP03.pdf}{Yes} & \cite{DannaP03} & 2003 & CP 2003 & 5 & 21 20 34 & 3 9 & \ref{b:DannaP03} & n/a\\
Kumar03 \href{https://doi.org/10.1007/978-3-540-45193-8_45}{Kumar03} & \hyperref[auth:a286]{T. K. S. Kumar} & Incremental Computation of Resource-Envelopes in Producer-Consumer Models & \href{../works/Kumar03.pdf}{Yes} & \cite{Kumar03} & 2003 & CP 2003 & 15 & 4 4 7 & 2 6 & \ref{b:Kumar03} & n/a\\
OddiPCC03 \href{https://doi.org/10.1007/978-3-540-45193-8_39}{OddiPCC03} & \hyperref[auth:a282]{A. Oddi}, \hyperref[auth:a283]{N. Policella}, \hyperref[auth:a284]{A. Cesta}, \hyperref[auth:a285]{G. Cortellessa} & Generating High Quality Schedules for a Spacecraft Memory Downlink Problem & \href{../works/OddiPCC03.pdf}{Yes} & \cite{OddiPCC03} & 2003 & CP 2003 & 15 & 8 7 14 & 6 12 & \ref{b:OddiPCC03} & n/a\\
Vilim03 \href{https://doi.org/10.1007/978-3-540-45193-8_124}{Vilim03} & \hyperref[auth:a121]{P. Vil{\'{\i}}m} & Computing Explanations for Global Scheduling Constraints & \href{../works/Vilim03.pdf}{Yes} & \cite{Vilim03} & 2003 & CP 2003 & 1 & 1 1 1 & 1 4 & \ref{b:Vilim03} & n/a\\
Wolf03 \href{https://doi.org/10.1007/978-3-540-45193-8_50}{Wolf03} & \hyperref[auth:a51]{A. Wolf} & Pruning while Sweeping over Task Intervals & \href{../works/Wolf03.pdf}{Yes} & \cite{Wolf03} & 2003 & CP 2003 & 15 & 11 11 10 & 7 13 & \ref{b:Wolf03} & n/a\\
Bartak02 \href{https://doi.org/10.1007/3-540-46135-3_39}{Bartak02} & \hyperref[auth:a152]{R. Bart{\'{a}}k} & \cellcolor{green!10}Visopt ShopFloor: On the Edge of Planning and Scheduling & \href{../works/Bartak02.pdf}{Yes} & \cite{Bartak02} & 2002 & CP 2002 & 16 & 6 6 11 & 4 20 & \ref{b:Bartak02} & n/a\\
BeldiceanuC02 \href{https://doi.org/10.1007/3-540-46135-3_5}{BeldiceanuC02} & \hyperref[auth:a128]{N. Beldiceanu}, \hyperref[auth:a91]{M. Carlsson} & \cellcolor{green!10}A New Multi-resource cumulatives Constraint with Negative Heights & \href{../works/BeldiceanuC02.pdf}{Yes} & \cite{BeldiceanuC02} & 2002 & CP 2002 & 17 & 33 33 48 & 9 20 & \ref{b:BeldiceanuC02} & n/a\\
BenoistGR02 \href{https://doi.org/10.1007/3-540-46135-3_40}{BenoistGR02} & \hyperref[auth:a1164]{T. Benoist}, \hyperref[auth:a1165]{E. Gaudin}, \hyperref[auth:a1166]{B. Rottembourg} & Constraint Programming Contribution to Benders Decomposition: {A} Case Study & \href{../works/BenoistGR02.pdf}{Yes} & \cite{BenoistGR02} & 2002 & CP 2002 & 15 & 13 13 19 & 11 19 & \ref{b:BenoistGR02} & n/a\\
ElkhyariGJ02 \href{https://doi.org/10.1007/3-540-46135-3_49}{ElkhyariGJ02} & \hyperref[auth:a292]{A. Elkhyari}, \hyperref[auth:a293]{C. Gu{\'{e}}ret}, \hyperref[auth:a247]{N. Jussien} & Conflict-Based Repair Techniques for Solving Dynamic Scheduling Problems & \href{../works/ElkhyariGJ02.pdf}{Yes} & \cite{ElkhyariGJ02} & 2002 & CP 2002 & 6 & 1 1 0 & 6 12 & \ref{b:ElkhyariGJ02} & n/a\\
HookerY02 \href{https://doi.org/10.1007/3-540-46135-3_46}{HookerY02} & \hyperref[auth:a160]{J. N. Hooker}, \hyperref[auth:a291]{H. Yan} & A Relaxation of the Cumulative Constraint & \href{../works/HookerY02.pdf}{Yes} & \cite{HookerY02} & 2002 & CP 2002 & 5 & 8 7 13 & 7 19 & \ref{b:HookerY02} & n/a\\
KamarainenS02 \href{https://doi.org/10.1007/3-540-46135-3_11}{KamarainenS02} & \hyperref[auth:a290]{O. Kamarainen}, \hyperref[auth:a166]{H. E. Sakkout} & Local Probing Applied to Scheduling & \href{../works/KamarainenS02.pdf}{Yes} & \cite{KamarainenS02} & 2002 & CP 2002 & 17 & 9 10 16 & 13 28 & \ref{b:KamarainenS02} & n/a\\
Muscettola02 \href{https://doi.org/10.1007/3-540-46135-3_10}{Muscettola02} & \hyperref[auth:a289]{N. Muscettola} & \cellcolor{green!10}Computing the Envelope for Stepwise-Constant Resource Allocations & \href{../works/Muscettola02.pdf}{Yes} & \cite{Muscettola02} & 2002 & CP 2002 & 16 & 14 14 26 & 4 15 & \ref{b:Muscettola02} & n/a\\
Vilim02 \href{https://doi.org/10.1007/3-540-46135-3_62}{Vilim02} & \hyperref[auth:a121]{P. Vil{\'{\i}}m} & Batch Processing with Sequence Dependent Setup Times & \href{../works/Vilim02.pdf}{Yes} & \cite{Vilim02} & 2002 & CP 2002 & 1 & 6 5 1 & 1 3 & \ref{b:Vilim02} & n/a\\
BeldiceanuC01 \href{https://doi.org/10.1007/3-540-45578-7_26}{BeldiceanuC01} & \hyperref[auth:a128]{N. Beldiceanu}, \hyperref[auth:a91]{M. Carlsson} & \cellcolor{green!10}Sweep as a Generic Pruning Technique Applied to the Non-overlapping Rectangles Constraint & \href{../works/BeldiceanuC01.pdf}{Yes} & \cite{BeldiceanuC01} & 2001 & CP 2001 & 15 & 34 33 45 & 0 13 & \ref{b:BeldiceanuC01} & n/a\\
EreminW01 \href{https://doi.org/10.1007/3-540-45578-7_1}{EreminW01} & \hyperref[auth:a1049]{A. Eremin}, \hyperref[auth:a117]{M. G. Wallace} & Hybrid Benders Decomposition Algorithms in Constraint Logic Programming & \href{../works/EreminW01.pdf}{Yes} & \cite{EreminW01} & 2001 & CP 2001 & 15 & 27 27 30 & 7 13 & \ref{b:EreminW01} & n/a\\
Thorsteinsson01 \href{https://doi.org/10.1007/3-540-45578-7_2}{Thorsteinsson01} & \hyperref[auth:a874]{E. S. Thorsteinsson} & Branch-and-Check: {A} Hybrid Framework Integrating Mixed Integer Programming and Constraint Logic Programming & \href{../works/Thorsteinsson01.pdf}{Yes} & \cite{Thorsteinsson01} & 2001 & CP 2001 & 15 & 67 68 97 & 12 25 & \ref{b:Thorsteinsson01} & n/a\\
VanczaM01 \href{https://doi.org/10.1007/3-540-45578-7_60}{VanczaM01} & \hyperref[auth:a278]{J. V{\'{a}}ncza}, \hyperref[auth:a294]{A. M{\'{a}}rkus} & A Constraint Engine for Manufacturing Process Planning & \href{../works/VanczaM01.pdf}{Yes} & \cite{VanczaM01} & 2001 & CP 2001 & 15 & 2 2 4 & 19 29 & \ref{b:VanczaM01} & n/a\\
VerfaillieL01 \href{https://doi.org/10.1007/3-540-45578-7_55}{VerfaillieL01} & \hyperref[auth:a173]{G. Verfaillie}, \hyperref[auth:a172]{M. Lema{\^{\i}}tre} & Selecting and Scheduling Observations for Agile Satellites: Some Lessons from the Constraint Reasoning Community Point of View & \href{../works/VerfaillieL01.pdf}{Yes} & \cite{VerfaillieL01} & 2001 & CP 2001 & 15 & 11 11 20 & 6 15 & \ref{b:VerfaillieL01} & n/a\\
AngelsmarkJ00 \href{https://doi.org/10.1007/3-540-45349-0_35}{AngelsmarkJ00} & \hyperref[auth:a295]{O. Angelsmark}, \hyperref[auth:a296]{P. Jonsson} & Some Observations on Durations, Scheduling and Allen's Algebra & \href{../works/AngelsmarkJ00.pdf}{Yes} & \cite{AngelsmarkJ00} & 2000 & CP 2000 & 5 & 1 1 6 & 9 17 & \ref{b:AngelsmarkJ00} & n/a\\
Refalo00 \href{https://doi.org/10.1007/3-540-45349-0_27}{Refalo00} & \hyperref[auth:a254]{P. Refalo} & Linear Formulation of Constraint Programming Models and Hybrid Solvers & \href{../works/Refalo00.pdf}{Yes} & \cite{Refalo00} & 2000 & CP 2000 & 15 & 35 37 49 & 11 22 & \ref{b:Refalo00} & n/a\\
CestaOS98 \href{https://doi.org/10.1007/3-540-49481-2_36}{CestaOS98} & \hyperref[auth:a284]{A. Cesta}, \hyperref[auth:a282]{A. Oddi}, \hyperref[auth:a298]{S. F. Smith} & Scheduling Multi-capacitated Resources Under Complex Temporal Constraints & \href{../works/CestaOS98.pdf}{Yes} & \cite{CestaOS98} & 1998 & CP 1998 & 1 & 5 5 4 & 0 3 & \ref{b:CestaOS98} & n/a\\
FrostD98 \href{https://doi.org/10.1007/3-540-49481-2_40}{FrostD98} & \hyperref[auth:a299]{D. Frost}, \hyperref[auth:a300]{R. Dechter} & Optimizing with Constraints: {A} Case Study in Scheduling Maintenance of Electric Power Units & \href{../works/FrostD98.pdf}{Yes} & \cite{FrostD98} & 1998 & CP 1998 & 1 & 10 10 11 & 2 3 & \ref{b:FrostD98} & n/a\\
RodosekW98 \href{https://doi.org/10.1007/3-540-49481-2_28}{RodosekW98} & \hyperref[auth:a297]{R. Rodosek}, \hyperref[auth:a117]{M. G. Wallace} & A Generic Model and Hybrid Algorithm for Hoist Scheduling Problems & \href{../works/RodosekW98.pdf}{Yes} & \cite{RodosekW98} & 1998 & CP 1998 & 15 & 19 20 32 & 10 23 & \ref{b:RodosekW98} & n/a\\
BaptisteP97 \href{https://doi.org/10.1007/BFb0017454}{BaptisteP97} & \hyperref[auth:a162]{P. Baptiste}, \hyperref[auth:a163]{C. L. Pape} & Constraint Propagation and Decomposition Techniques for Highly Disjunctive and Highly Cumulative Project Scheduling Problems & \href{../works/BaptisteP97.pdf}{Yes} & \cite{BaptisteP97} & 1997 & CP 1997 & 15 & 8 9 14 & 10 22 & \ref{b:BaptisteP97} & n/a\\
BeckDF97 \href{https://doi.org/10.1007/BFb0017455}{BeckDF97} & \hyperref[auth:a89]{J. C. Beck}, \hyperref[auth:a248]{A. J. Davenport}, \hyperref[auth:a302]{M. S. Fox} & Five Pitfalls of Empirical Scheduling Research & \href{../works/BeckDF97.pdf}{Yes} & \cite{BeckDF97} & 1997 & CP 1997 & 15 & 3 3 6 & 12 41 & \ref{b:BeckDF97} & n/a\\
Caseau97 \href{https://doi.org/10.1007/BFb0017437}{Caseau97} & \hyperref[auth:a301]{Y. Caseau} & Using Constraint Propagation for Complex Scheduling Problems: Managing Size, Complex Resources and Travel & \href{../works/Caseau97.pdf}{Yes} & \cite{Caseau97} & 1997 & CP 1997 & 4 & 0 1 0 & 0 0 & \ref{b:Caseau97} & n/a\\
Colombani96 \href{https://doi.org/10.1007/3-540-61551-2_72}{Colombani96} & \hyperref[auth:a168]{Y. Colombani} & Constraint Programming: an Efficient and Practical Approach to Solving the Job-Shop Problem & \href{../works/Colombani96.pdf}{Yes} & \cite{Colombani96} & 1996 & CP 1996 & 15 & 4 4 10 & 5 12 & \ref{b:Colombani96} & n/a\\
Zhou96 \href{https://doi.org/10.1007/3-540-61551-2_97}{Zhou96} & \hyperref[auth:a176]{J. Zhou} & A Constraint Program for Solving the Job-Shop Problem & \href{../works/Zhou96.pdf}{Yes} & \cite{Zhou96} & 1996 & CP 1996 & 15 & 10 10 8 & 7 17 & \ref{b:Zhou96} & n/a\\
Goltz95 \href{https://doi.org/10.1007/3-540-60299-2_33}{Goltz95} & \hyperref[auth:a304]{H.-J. Goltz} & Reducing Domains for Search in {CLP(FD)} and Its Application to Job-Shop Scheduling & \href{../works/Goltz95.pdf}{Yes} & \cite{Goltz95} & 1995 & CP 1995 & 14 & 7 7 8 & 7 14 & \ref{b:Goltz95} & n/a\\
Puget95 \href{https://doi.org/10.1007/3-540-60299-2_43}{Puget95} & \hyperref[auth:a305]{J.-F. Puget} & Applications of Constraint Programming & \href{../works/Puget95.pdf}{Yes} & \cite{Puget95} & 1995 & CP 1995 & 4 & 6 6 11 & 2 11 & \ref{b:Puget95} & n/a\\
Simonis95 \href{https://doi.org/10.1007/3-540-60299-2_42}{Simonis95} & \hyperref[auth:a17]{H. Simonis} & The {CHIP} System and Its Applications & \href{../works/Simonis95.pdf}{Yes} & \cite{Simonis95} & 1995 & CP 1995 & 4 & 7 7 13 & 3 7 & \ref{b:Simonis95} & n/a\\
SimonisC95 \href{https://doi.org/10.1007/3-540-60299-2_27}{SimonisC95} & \hyperref[auth:a17]{H. Simonis}, \hyperref[auth:a303]{T. Cornelissens} & Modelling Producer/Consumer Constraints & \href{../works/SimonisC95.pdf}{Yes} & \cite{SimonisC95} & 1995 & CP 1995 & 14 & 17 16 21 & 8 23 & \ref{b:SimonisC95} & n/a\\
Touraivane95 \href{https://doi.org/10.1007/3-540-60299-2_41}{Touraivane95} & \hyperref[auth:a306]{Toura{\"{\i}}vane} & Constraint Programming and Industrial Applications & \href{../works/Touraivane95.pdf}{Yes} & \cite{Touraivane95} & 1995 & CP 1995 & 3 & 2 0 1 & 1 0 & \ref{b:Touraivane95} & n/a\\
\end{longtable}
}

\subsection{CPAIOR}

\index{CPAIOR}
{\scriptsize
\begin{longtable}{>{\raggedright\arraybackslash}p{3cm}>{\raggedright\arraybackslash}p{4.5cm}>{\raggedright\arraybackslash}p{6.0cm}rrrp{2.5cm}rp{1cm}p{1cm}rr}
\rowcolor{white}\caption{Papers in Conference Series CPAIOR (Total 99) (Total 99)}\\ \toprule
\rowcolor{white}\shortstack{Key\\Source} & Authors & Title (Colored by Open Access)& LC & Cite & Year & \shortstack{Conference\\/Journal\\/School} & Pages & \shortstack{Cites\\OC XR\\SC} & \shortstack{Refs\\OC\\XR} & b & c \\ \midrule\endhead
\bottomrule
\endfoot
EfthymiouY23 \href{https://doi.org/10.1007/978-3-031-33271-5_16}{EfthymiouY23} & \hyperref[auth:a18]{N. Efthymiou}, \hyperref[auth:a19]{N. Yorke-Smith} & \cellcolor{green!10}Predicting the Optimal Period for Cyclic Hoist Scheduling Problems & \href{../works/EfthymiouY23.pdf}{Yes} & \cite{EfthymiouY23} & 2023 & CPAIOR 2023 & 16 & 0 0 1 & 23 26 & \ref{b:EfthymiouY23} & \ref{c:EfthymiouY23}\\
JuvinHL23 \href{https://doi.org/10.1007/978-3-031-33271-5_23}{JuvinHL23} & \hyperref[auth:a0]{C. Juvin}, \hyperref[auth:a2]{L. Houssin}, \hyperref[auth:a3]{P. Lopez} & \cellcolor{green!10}Constraint Programming for the Robust Two-Machine Flow-Shop Scheduling Problem with Budgeted Uncertainty & \href{../works/JuvinHL23.pdf}{Yes} & \cite{JuvinHL23} & 2023 & CPAIOR 2023 & 16 & 0 1 0 & 11 12 & \ref{b:JuvinHL23} & \ref{c:JuvinHL23}\\
KimCMLLP23 \href{https://doi.org/10.1007/978-3-031-33271-5_31}{KimCMLLP23} & \hyperref[auth:a23]{D. Kim}, \hyperref[auth:a24]{Y. Choi}, \hyperref[auth:a25]{K. Moon}, \hyperref[auth:a26]{M. Lee}, \hyperref[auth:a27]{K. Lee}, \hyperref[auth:a28]{M. L. Pinedo} & Iterated Greedy Constraint Programming for Scheduling Steelmaking Continuous Casting & \href{../works/KimCMLLP23.pdf}{Yes} & \cite{KimCMLLP23} & 2023 & CPAIOR 2023 & 16 & 0 0 0 & 13 14 & \ref{b:KimCMLLP23} & \ref{c:KimCMLLP23}\\
SquillaciPR23 \href{https://doi.org/10.1007/978-3-031-33271-5_29}{SquillaciPR23} & \hyperref[auth:a20]{S. Squillaci}, \hyperref[auth:a21]{C. Pralet}, \hyperref[auth:a22]{S. Roussel} & Scheduling Complex Observation Requests for a Constellation of Satellites: Large Neighborhood Search Approaches & \href{../works/SquillaciPR23.pdf}{Yes} & \cite{SquillaciPR23} & 2023 & CPAIOR 2023 & 17 & 0 0 0 & 19 23 & \ref{b:SquillaciPR23} & \ref{c:SquillaciPR23}\\
TardivoDFMP23 \href{https://doi.org/10.1007/978-3-031-33271-5_22}{TardivoDFMP23} & \hyperref[auth:a29]{F. Tardivo}, \hyperref[auth:a30]{A. Dovier}, \hyperref[auth:a31]{A. Formisano}, \hyperref[auth:a32]{L. Michel}, \hyperref[auth:a33]{E. Pontelli} & Constraint Propagation on {GPU:} {A} Case Study for the Cumulative Constraint & \href{../works/TardivoDFMP23.pdf}{Yes} & \cite{TardivoDFMP23} & 2023 & CPAIOR 2023 & 18 & 0 0 0 & 30 48 & \ref{b:TardivoDFMP23} & \ref{c:TardivoDFMP23}\\
ArmstrongGOS22 \href{https://doi.org/10.1007/978-3-031-08011-1_1}{ArmstrongGOS22} & \hyperref[auth:a14]{E. Armstrong}, \hyperref[auth:a15]{M. Garraffa}, \hyperref[auth:a16]{B. O'Sullivan}, \hyperref[auth:a17]{H. Simonis} & \cellcolor{green!10}A Two-Phase Hybrid Approach for the Hybrid Flexible Flowshop with Transportation Times & \href{../works/ArmstrongGOS22.pdf}{Yes} & \cite{ArmstrongGOS22} & 2022 & CPAIOR 2022 & 13 & 0 0 0 & 14 15 & \ref{b:ArmstrongGOS22} & \ref{c:ArmstrongGOS22}\\
GeitzGSSW22 \href{https://doi.org/10.1007/978-3-031-08011-1_10}{GeitzGSSW22} & \hyperref[auth:a47]{M. Geitz}, \hyperref[auth:a48]{C. Grozea}, \hyperref[auth:a49]{W. Steigerwald}, \hyperref[auth:a50]{R. St{\"{o}}hr}, \hyperref[auth:a51]{A. Wolf} & \cellcolor{green!10}Solving the Extended Job Shop Scheduling Problem with AGVs - Classical and Quantum Approaches & \href{../works/GeitzGSSW22.pdf}{Yes} & \cite{GeitzGSSW22} & 2022 & CPAIOR 2022 & 18 & 0 1 3 & 24 36 & \ref{b:GeitzGSSW22} & \ref{c:GeitzGSSW22}\\
LuoB22 \href{https://doi.org/10.1007/978-3-031-08011-1_17}{LuoB22} & \hyperref[auth:a745]{Y. L. Luo}, \hyperref[auth:a89]{J. C. Beck} & Packing by Scheduling: Using Constraint Programming to Solve a Complex 2D Cutting Stock Problem & \href{../works/LuoB22.pdf}{Yes} & \cite{LuoB22} & 2022 & CPAIOR 2022 & 17 & 0 0 1 & 28 34 & \ref{b:LuoB22} & \ref{c:LuoB22}\\
OuelletQ22 \href{https://doi.org/10.1007/978-3-031-08011-1_21}{OuelletQ22} & \hyperref[auth:a52]{Y. Ouellet}, \hyperref[auth:a37]{C.-G. Quimper} & A MinCumulative Resource Constraint & \href{../works/OuelletQ22.pdf}{Yes} & \cite{OuelletQ22} & 2022 & CPAIOR 2022 & 17 & 1 1 0 & 22 27 & \ref{b:OuelletQ22} & \ref{c:OuelletQ22}\\
Astrand0F21 \href{https://doi.org/10.1007/978-3-030-78230-6_23}{Astrand0F21} & \hyperref[auth:a74]{M. {\AA}strand}, \hyperref[auth:a75]{M. Johansson}, \hyperref[auth:a76]{H. R. Feyzmahdavian} & Short-Term Scheduling of Production Fleets in Underground Mines Using CP-Based {LNS} & \href{../works/Astrand0F21.pdf}{Yes} & \cite{Astrand0F21} & 2021 & CPAIOR 2021 & 18 & 2 2 2 & 25 31 & \ref{b:Astrand0F21} & \ref{c:Astrand0F21}\\
GeibingerKKMMW21 \href{https://doi.org/10.1007/978-3-030-78230-6_29}{GeibingerKKMMW21} & \hyperref[auth:a77]{T. Geibinger}, \hyperref[auth:a78]{L. Kletzander}, \hyperref[auth:a79]{M. Krainz}, \hyperref[auth:a80]{F. Mischek}, \hyperref[auth:a45]{N. Musliu}, \hyperref[auth:a43]{F. Winter} & Physician Scheduling During a Pandemic & \href{../works/GeibingerKKMMW21.pdf}{Yes} & \cite{GeibingerKKMMW21} & 2021 & CPAIOR 2021 & 10 & 0 0 0 & 6 13 & \ref{b:GeibingerKKMMW21} & \ref{c:GeibingerKKMMW21}\\
HanenKP21 \href{https://doi.org/10.1007/978-3-030-78230-6_14}{HanenKP21} & \hyperref[auth:a71]{C. Hanen}, \hyperref[auth:a72]{A. M. Kordon}, \hyperref[auth:a73]{T. Pedersen} & Two Deadline Reduction Algorithms for Scheduling Dependent Tasks on Parallel Processors & \href{../works/HanenKP21.pdf}{Yes} & \cite{HanenKP21} & 2021 & CPAIOR 2021 & 17 & 1 2 0 & 24 29 & \ref{b:HanenKP21} & \ref{c:HanenKP21}\\
HillTV21 \href{https://doi.org/10.1007/978-3-030-78230-6_2}{HillTV21} & \hyperref[auth:a64]{A. Hill}, \hyperref[auth:a65]{J. Ticktin}, \hyperref[auth:a66]{T. W. M. Vossen} & A Computational Study of Constraint Programming Approaches for Resource-Constrained Project Scheduling with Autonomous Learning Effects & \href{../works/HillTV21.pdf}{Yes} & \cite{HillTV21} & 2021 & CPAIOR 2021 & 19 & 0 0 3 & 38 42 & \ref{b:HillTV21} & \ref{c:HillTV21}\\
KlankeBYE21 \href{https://doi.org/10.1007/978-3-030-78230-6_9}{KlankeBYE21} & \hyperref[auth:a67]{C. Klanke}, \hyperref[auth:a68]{D. R. Bleidorn}, \hyperref[auth:a69]{V. Yfantis}, \hyperref[auth:a70]{S. Engell} & Combining Constraint Programming and Temporal Decomposition Approaches - Scheduling of an Industrial Formulation Plant & \href{../works/KlankeBYE21.pdf}{Yes} & \cite{KlankeBYE21} & 2021 & CPAIOR 2021 & 16 & 3 3 2 & 13 14 & \ref{b:KlankeBYE21} & \ref{c:KlankeBYE21}\\
GokGSTO20 \href{https://doi.org/10.1007/978-3-030-58942-4_15}{GokGSTO20} & \hyperref[auth:a1015]{Y. S. G\"{o}k}, \hyperref[auth:a1013]{D. Guimarans}, \hyperref[auth:a125]{P. J. Stuckey}, \hyperref[auth:a1012]{M. Tomasella}, \hyperref[auth:a135]{C. {\"{O}}zt{\"{u}}rk} & Robust Resource Planning for Aircraft Ground Operations & \href{../works/GokGSTO20.pdf}{Yes} & \cite{GokGSTO20} & 2020 & CPAIOR 2020 & 17 & 2 3 8 & 14 23 & \ref{b:GokGSTO20} & n/a\\
Mercier-AubinGQ20 \href{https://doi.org/10.1007/978-3-030-58942-4_22}{Mercier-AubinGQ20} & \hyperref[auth:a86]{A. Mercier-Aubin}, \hyperref[auth:a87]{J. Gaudreault}, \hyperref[auth:a37]{C.-G. Quimper} & Leveraging Constraint Scheduling: {A} Case Study to the Textile Industry & \href{../works/Mercier-AubinGQ20.pdf}{Yes} & \cite{Mercier-AubinGQ20} & 2020 & CPAIOR 2020 & 13 & 2 2 2 & 13 17 & \ref{b:Mercier-AubinGQ20} & \ref{c:Mercier-AubinGQ20}\\
TangB20 \href{https://doi.org/10.1007/978-3-030-58942-4_28}{TangB20} & \hyperref[auth:a88]{T. Y. Tang}, \hyperref[auth:a89]{J. C. Beck} & {CP} and Hybrid Models for Two-Stage Batching and Scheduling & \href{../works/TangB20.pdf}{Yes} & \cite{TangB20} & 2020 & CPAIOR 2020 & 16 & 6 6 6 & 12 14 & \ref{b:TangB20} & \ref{c:TangB20}\\
ThomasKS20 \href{https://doi.org/10.1007/978-3-030-58942-4_30}{ThomasKS20} & \hyperref[auth:a834]{C. Thomas}, \hyperref[auth:a10]{R. Kameugne}, \hyperref[auth:a147]{P. Schaus} & Insertion Sequence Variables for Hybrid Routing and Scheduling Problems & \href{../works/ThomasKS20.pdf}{Yes} & \cite{ThomasKS20} & 2020 & CPAIOR 2020 & 18 & 0 0 2 & 16 28 & \ref{b:ThomasKS20} & \ref{c:ThomasKS20}\\
WessenCS20 \href{https://doi.org/10.1007/978-3-030-58942-4_33}{WessenCS20} & \hyperref[auth:a90]{J. Wess{\'{e}}n}, \hyperref[auth:a91]{M. Carlsson}, \hyperref[auth:a92]{C. Schulte} & \cellcolor{green!10}Scheduling of Dual-Arm Multi-tool Assembly Robots and Workspace Layout Optimization & \href{../works/WessenCS20.pdf}{Yes} & \cite{WessenCS20} & 2020 & CPAIOR 2020 & 10 & 2 2 2 & 11 19 & \ref{b:WessenCS20} & \ref{c:WessenCS20}\\
BogaerdtW19 \href{https://doi.org/10.1007/978-3-030-19212-9_38}{BogaerdtW19} & \hyperref[auth:a307]{P. van den Bogaerdt}, \hyperref[auth:a308]{M. de Weerdt} & \cellcolor{green!10}Lower Bounds for Uniform Machine Scheduling Using Decision Diagrams & \href{../works/BogaerdtW19.pdf}{Yes} & \cite{BogaerdtW19} & 2019 & CPAIOR 2019 & 16 & 1 1 2 & 16 20 & \ref{b:BogaerdtW19} & \ref{c:BogaerdtW19}\\
GeibingerMM19 \href{https://doi.org/10.1007/978-3-030-19212-9_20}{GeibingerMM19} & \hyperref[auth:a77]{T. Geibinger}, \hyperref[auth:a80]{F. Mischek}, \hyperref[auth:a45]{N. Musliu} & Investigating Constraint Programming for Real World Industrial Test Laboratory Scheduling & \href{../works/GeibingerMM19.pdf}{Yes} & \cite{GeibingerMM19} & 2019 & CPAIOR 2019 & 16 & 6 9 10 & 15 23 & \ref{b:GeibingerMM19} & n/a\\
MalapertN19 \href{https://doi.org/10.1007/978-3-030-19212-9_28}{MalapertN19} & \hyperref[auth:a82]{A. Malapert}, \hyperref[auth:a81]{M. Nattaf} & \cellcolor{green!10}A New CP-Approach for a Parallel Machine Scheduling Problem with Time Constraints on Machine Qualifications & \href{../works/MalapertN19.pdf}{Yes} & \cite{MalapertN19} & 2019 & CPAIOR 2019 & 17 & 1 1 1 & 7 16 & \ref{b:MalapertN19} & n/a\\
YangSS19 \href{https://doi.org/10.1007/978-3-030-19212-9_42}{YangSS19} & \hyperref[auth:a309]{M. Yang}, \hyperref[auth:a124]{A. Schutt}, \hyperref[auth:a125]{P. J. Stuckey} & Time Table Edge Finding with Energy Variables & \href{../works/YangSS19.pdf}{Yes} & \cite{YangSS19} & 2019 & CPAIOR 2019 & 10 & 1 1 1 & 14 18 & \ref{b:YangSS19} & n/a\\
AstrandJZ18 \href{https://doi.org/10.1007/978-3-319-93031-2_44}{AstrandJZ18} & \hyperref[auth:a74]{M. {\AA}strand}, \hyperref[auth:a75]{M. Johansson}, \hyperref[auth:a199]{A. Zanarini} & Fleet Scheduling in Underground Mines Using Constraint Programming & \href{../works/AstrandJZ18.pdf}{Yes} & \cite{AstrandJZ18} & 2018 & CPAIOR 2018 & 9 & 9 9 14 & 10 14 & \ref{b:AstrandJZ18} & n/a\\
BenediktSMVH18 \href{https://doi.org/10.1007/978-3-319-93031-2_6}{BenediktSMVH18} & \hyperref[auth:a114]{O. Benedikt}, \hyperref[auth:a310]{P. Sucha}, \hyperref[auth:a115]{I. M{\'{o}}dos}, \hyperref[auth:a311]{M. Vlk}, \hyperref[auth:a116]{Z. Hanz{\'{a}}lek} & Energy-Aware Production Scheduling with Power-Saving Modes & \href{../works/BenediktSMVH18.pdf}{Yes} & \cite{BenediktSMVH18} & 2018 & CPAIOR 2018 & 10 & 2 2 4 & 12 13 & \ref{b:BenediktSMVH18} & \ref{c:BenediktSMVH18}\\
DemirovicS18 \href{https://doi.org/10.1007/978-3-319-93031-2_10}{DemirovicS18} & \hyperref[auth:a312]{E. Demirovic}, \hyperref[auth:a125]{P. J. Stuckey} & Constraint Programming for High School Timetabling: {A} Scheduling-Based Model with Hot Starts & \href{../works/DemirovicS18.pdf}{Yes} & \cite{DemirovicS18} & 2018 & CPAIOR 2018 & 18 & 4 4 8 & 16 34 & \ref{b:DemirovicS18} & n/a\\
KameugneFGOQ18 \href{https://doi.org/10.1007/978-3-319-93031-2_23}{KameugneFGOQ18} & \hyperref[auth:a10]{R. Kameugne}, \hyperref[auth:a11]{S. B. Fetgo}, \hyperref[auth:a313]{V. Gingras}, \hyperref[auth:a52]{Y. Ouellet}, \hyperref[auth:a37]{C.-G. Quimper} & Horizontally Elastic Not-First/Not-Last Filtering Algorithm for Cumulative Resource Constraint & \href{../works/KameugneFGOQ18.pdf}{Yes} & \cite{KameugneFGOQ18} & 2018 & CPAIOR 2018 & 17 & 1 2 2 & 12 17 & \ref{b:KameugneFGOQ18} & n/a\\
Laborie18a \href{https://doi.org/10.1007/978-3-319-93031-2_29}{Laborie18a} & \hyperref[auth:a118]{P. Laborie} & An Update on the Comparison of MIP, {CP} and Hybrid Approaches for Mixed Resource Allocation and Scheduling & \href{../works/Laborie18a.pdf}{Yes} & \cite{Laborie18a} & 2018 & CPAIOR 2018 & 9 & 18 19 25 & 10 15 & \ref{b:Laborie18a} & n/a\\
MusliuSS18 \href{https://doi.org/10.1007/978-3-319-93031-2_31}{MusliuSS18} & \hyperref[auth:a45]{N. Musliu}, \hyperref[auth:a124]{A. Schutt}, \hyperref[auth:a125]{P. J. Stuckey} & Solver Independent Rotating Workforce Scheduling & \href{../works/MusliuSS18.pdf}{Yes} & \cite{MusliuSS18} & 2018 & CPAIOR 2018 & 17 & 7 7 10 & 23 28 & \ref{b:MusliuSS18} & n/a\\
OuelletQ18 \href{https://doi.org/10.1007/978-3-319-93031-2_34}{OuelletQ18} & \hyperref[auth:a52]{Y. Ouellet}, \hyperref[auth:a37]{C.-G. Quimper} & A O(n {\textbackslash}log {\^{}}2 n) Checker and O(n{\^{}}2 {\textbackslash}log n) Filtering Algorithm for the Energetic Reasoning & \href{../works/OuelletQ18.pdf}{Yes} & \cite{OuelletQ18} & 2018 & CPAIOR 2018 & 18 & 6 8 8 & 16 23 & \ref{b:OuelletQ18} & n/a\\
CappartS17 \href{https://doi.org/10.1007/978-3-319-59776-8_26}{CappartS17} & \hyperref[auth:a42]{Q. Cappart}, \hyperref[auth:a147]{P. Schaus} & Rescheduling Railway Traffic on Real Time Situations Using Time-Interval Variables & \href{../works/CappartS17.pdf}{Yes} & \cite{CappartS17} & 2017 & CPAIOR 2017 & 16 & 2 2 3 & 28 34 & \ref{b:CappartS17} & \ref{c:CappartS17}\\
GelainPRVW17 \href{https://doi.org/10.1007/978-3-319-59776-8_32}{GelainPRVW17} & \hyperref[auth:a314]{M. Gelain}, \hyperref[auth:a315]{M. S. Pini}, \hyperref[auth:a316]{F. Rossi}, \hyperref[auth:a317]{K. B. Venable}, \hyperref[auth:a276]{T. Walsh} & A Local Search Approach for Incomplete Soft Constraint Problems: Experimental Results on Meeting Scheduling Problems & \href{../works/GelainPRVW17.pdf}{Yes} & \cite{GelainPRVW17} & 2017 & CPAIOR 2017 & 16 & 1 2 2 & 5 16 & \ref{b:GelainPRVW17} & n/a\\
KletzanderM17 \href{https://doi.org/10.1007/978-3-319-59776-8_28}{KletzanderM17} & \hyperref[auth:a78]{L. Kletzander}, \hyperref[auth:a45]{N. Musliu} & A Multi-stage Simulated Annealing Algorithm for the Torpedo Scheduling Problem & \href{../works/KletzanderM17.pdf}{Yes} & \cite{KletzanderM17} & 2017 & CPAIOR 2017 & 15 & 1 1 6 & 9 14 & \ref{b:KletzanderM17} & n/a\\
FontaineMH16 \href{https://doi.org/10.1007/978-3-319-33954-2_12}{FontaineMH16} & \hyperref[auth:a318]{D. Fontaine}, \hyperref[auth:a32]{L. Michel}, \hyperref[auth:a148]{P. V. Hentenryck} & Parallel Composition of Scheduling Solvers & \href{../works/FontaineMH16.pdf}{Yes} & \cite{FontaineMH16} & 2016 & CPAIOR 2016 & 11 & 3 3 3 & 0 0 & \ref{b:FontaineMH16} & n/a\\
HechingH16 \href{https://doi.org/10.1007/978-3-319-33954-2_14}{HechingH16} & \hyperref[auth:a319]{A. R. Heching}, \hyperref[auth:a160]{J. N. Hooker} & Scheduling Home Hospice Care with Logic-Based Benders Decomposition & \href{../works/HechingH16.pdf}{Yes} & \cite{HechingH16} & 2016 & CPAIOR 2016 & 11 & 10 11 18 & 0 0 & \ref{b:HechingH16} & n/a\\
Madi-WambaB16 \href{https://doi.org/10.1007/978-3-319-33954-2_18}{Madi-WambaB16} & \hyperref[auth:a320]{G. Madi-Wamba}, \hyperref[auth:a128]{N. Beldiceanu} & The TaskIntersection Constraint & \href{../works/Madi-WambaB16.pdf}{Yes} & \cite{Madi-WambaB16} & 2016 & CPAIOR 2016 & 16 & 0 0 2 & 0 0 & \ref{b:Madi-WambaB16} & n/a\\
BofillGSV15 \href{https://doi.org/10.1007/978-3-319-18008-3_5}{BofillGSV15} & \hyperref[auth:a228]{M. Bofill}, \hyperref[auth:a230]{M. Garcia}, \hyperref[auth:a232]{J. Suy}, \hyperref[auth:a233]{M. Villaret} & \cellcolor{green!10}MaxSAT-Based Scheduling of {B2B} Meetings & \href{../works/BofillGSV15.pdf}{Yes} & \cite{BofillGSV15} & 2015 & CPAIOR 2015 & 9 & 7 8 22 & 8 12 & \ref{b:BofillGSV15} & n/a\\
BurtLPS15 \href{https://doi.org/10.1007/978-3-319-18008-3_7}{BurtLPS15} & \hyperref[auth:a322]{C. N. Burt}, \hyperref[auth:a323]{N. Lipovetzky}, \hyperref[auth:a324]{A. R. Pearce}, \hyperref[auth:a125]{P. J. Stuckey} & \cellcolor{green!10}Scheduling with Fixed Maintenance, Shared Resources and Nonlinear Feedrate Constraints: {A} Mine Planning Case Study & \href{../works/BurtLPS15.pdf}{Yes} & \cite{BurtLPS15} & 2015 & CPAIOR 2015 & 17 & 0 0 3 & 8 16 & \ref{b:BurtLPS15} & n/a\\
CauwelaertLS15 \href{https://doi.org/10.1007/978-3-319-18008-3_29}{CauwelaertLS15} & \hyperref[auth:a201]{S. V. Cauwelaert}, \hyperref[auth:a142]{M. Lombardi}, \hyperref[auth:a147]{P. Schaus} & \cellcolor{green!10}Understanding the Potential of Propagators & \href{../works/CauwelaertLS15.pdf}{Yes} & \cite{CauwelaertLS15} & 2015 & CPAIOR 2015 & 10 & 12 11 9 & 0 21 & \ref{b:CauwelaertLS15} & \ref{c:CauwelaertLS15}\\
GayHS15a \href{https://doi.org/10.1007/978-3-319-18008-3_11}{GayHS15a} & \hyperref[auth:a211]{S. Gay}, \hyperref[auth:a212]{R. Hartert}, \hyperref[auth:a147]{P. Schaus} & \cellcolor{green!10}Time-Table Disjunctive Reasoning for the Cumulative Constraint & \href{../works/GayHS15a.pdf}{Yes} & \cite{GayHS15a} & 2015 & CPAIOR 2015 & 16 & 5 5 6 & 12 16 & \ref{b:GayHS15a} & \ref{c:GayHS15a}\\
LimBTBB15 \href{https://doi.org/10.1007/978-3-319-18008-3_17}{LimBTBB15} & \hyperref[auth:a207]{B. Lim}, \hyperref[auth:a210]{M. van den Briel}, \hyperref[auth:a209]{S. Thi{\'{e}}baux}, \hyperref[auth:a1355]{R. Bent}, \hyperref[auth:a1356]{S. Backhaus} & \cellcolor{green!10}Large Neighborhood Search for Energy Aware Meeting Scheduling in Smart Buildings & \href{../works/LimBTBB15.pdf}{Yes} & \cite{LimBTBB15} & 2015 & CPAIOR 2015 & 15 & 4 4 5 & 18 24 & \ref{b:LimBTBB15} & n/a\\
MelgarejoLS15 \href{https://doi.org/10.1007/978-3-319-18008-3_1}{MelgarejoLS15} & \hyperref[auth:a321]{P. Aguiar-Melgarejo}, \hyperref[auth:a118]{P. Laborie}, \hyperref[auth:a85]{C. Solnon} & \cellcolor{green!10}A Time-Dependent No-Overlap Constraint: Application to Urban Delivery Problems & \href{../works/MelgarejoLS15.pdf}{Yes} & \cite{MelgarejoLS15} & 2015 & CPAIOR 2015 & 17 & 14 14 22 & 17 30 & \ref{b:MelgarejoLS15} & n/a\\
PesantRR15 \href{https://doi.org/10.1007/978-3-319-18008-3_21}{PesantRR15} & \hyperref[auth:a8]{G. Pesant}, \hyperref[auth:a325]{G. Rix}, \hyperref[auth:a326]{L.-M. Rousseau} & A Comparative Study of {MIP} and {CP} Formulations for the {B2B} Scheduling Optimization Problem & \href{../works/PesantRR15.pdf}{Yes} & \cite{PesantRR15} & 2015 & CPAIOR 2015 & 16 & 1 1 6 & 7 9 & \ref{b:PesantRR15} & n/a\\
VilimLS15 \href{https://doi.org/10.1007/978-3-319-18008-3_30}{VilimLS15} & \hyperref[auth:a121]{P. Vil{\'{\i}}m}, \hyperref[auth:a118]{P. Laborie}, \hyperref[auth:a120]{P. Shaw} & Failure-Directed Search for Constraint-Based Scheduling & \href{../works/VilimLS15.pdf}{Yes} & \cite{VilimLS15} & 2015 & CPAIOR 2015 & 17 & 31 31 55 & 19 38 & \ref{b:VilimLS15} & n/a\\
BessiereHMQW14 \href{https://doi.org/10.1007/978-3-319-07046-9_23}{BessiereHMQW14} & \hyperref[auth:a328]{C. Bessiere}, \hyperref[auth:a1]{E. Hebrard}, \hyperref[auth:a329]{M.-A. M{\'{e}}nard}, \hyperref[auth:a37]{C.-G. Quimper}, \hyperref[auth:a276]{T. Walsh} & \cellcolor{green!10}Buffered Resource Constraint: Algorithms and Complexity & \href{../works/BessiereHMQW14.pdf}{Yes} & \cite{BessiereHMQW14} & 2014 & CPAIOR 2014 & 16 & 1 1 1 & 3 7 & \ref{b:BessiereHMQW14} & n/a\\
BonfiettiLM14 \href{https://doi.org/10.1007/978-3-319-07046-9_15}{BonfiettiLM14} & \hyperref[auth:a198]{A. Bonfietti}, \hyperref[auth:a142]{M. Lombardi}, \hyperref[auth:a143]{M. Milano} & Disregarding Duration Uncertainty in Partial Order Schedules? Yes, We Can! & \href{../works/BonfiettiLM14.pdf}{Yes} & \cite{BonfiettiLM14} & 2014 & CPAIOR 2014 & 16 & 3 3 9 & 12 18 & \ref{b:BonfiettiLM14} & n/a\\
DejemeppeD14 \href{https://doi.org/10.1007/978-3-319-07046-9_20}{DejemeppeD14} & \hyperref[auth:a202]{C. Dejemeppe}, \hyperref[auth:a151]{Y. Deville} & \cellcolor{green!10}Continuously Degrading Resource and Interval Dependent Activity Durations in Nuclear Medicine Patient Scheduling & \href{../works/DejemeppeD14.pdf}{Yes} & \cite{DejemeppeD14} & 2014 & CPAIOR 2014 & 9 & 0 0 0 & 7 8 & \ref{b:DejemeppeD14} & \ref{c:DejemeppeD14}\\
DoulabiRP14 \href{https://doi.org/10.1007/978-3-319-07046-9_32}{DoulabiRP14} & \hyperref[auth:a330]{S. H. H. Doulabi}, \hyperref[auth:a326]{L.-M. Rousseau}, \hyperref[auth:a8]{G. Pesant} & A Constraint Programming-Based Column Generation Approach for Operating Room Planning and Scheduling & \href{../works/DoulabiRP14.pdf}{Yes} & \cite{DoulabiRP14} & 2014 & CPAIOR 2014 & 9 & 3 3 10 & 10 10 & \ref{b:DoulabiRP14} & n/a\\
KoschB14 \href{https://doi.org/10.1007/978-3-319-07046-9_5}{KoschB14} & \hyperref[auth:a327]{S. Kosch}, \hyperref[auth:a89]{J. C. Beck} & A New {MIP} Model for Parallel-Batch Scheduling with Non-identical Job Sizes & \href{../works/KoschB14.pdf}{Yes} & \cite{KoschB14} & 2014 & CPAIOR 2014 & 16 & 4 4 6 & 18 25 & \ref{b:KoschB14} & n/a\\
LarsonJC14 \href{https://doi.org/10.1007/978-3-319-07046-9_11}{LarsonJC14} & \hyperref[auth:a1413]{J. Larson}, \hyperref[auth:a75]{M. Johansson}, \hyperref[auth:a91]{M. Carlsson} & An Integrated Constraint Programming Approach to Scheduling Sports Leagues with Divisional and Round-Robin Tournaments & \href{../works/LarsonJC14.pdf}{Yes} & \cite{LarsonJC14} & 2014 & CPAIOR 2014 & 15 & 1 2 4 & 16 21 & \ref{b:LarsonJC14} & n/a\\
CireCH13 \href{https://doi.org/10.1007/978-3-642-38171-3_22}{CireCH13} & \hyperref[auth:a157]{A. A. Cir{\'{e}}}, \hyperref[auth:a335]{E. Coban}, \hyperref[auth:a160]{J. N. Hooker} & Mixed Integer Programming vs. Logic-Based Benders Decomposition for Planning and Scheduling & \href{../works/CireCH13.pdf}{Yes} & \cite{CireCH13} & 2013 & CPAIOR 2013 & 7 & 3 3 13 & 23 28 & \ref{b:CireCH13} & \ref{c:CireCH13}\\
GuSS13 \href{https://doi.org/10.1007/978-3-642-38171-3_24}{GuSS13} & \hyperref[auth:a336]{H. Gu}, \hyperref[auth:a124]{A. Schutt}, \hyperref[auth:a125]{P. J. Stuckey} & A Lagrangian Relaxation Based Forward-Backward Improvement Heuristic for Maximising the Net Present Value of Resource-Constrained Projects & \href{../works/GuSS13.pdf}{Yes} & \cite{GuSS13} & 2013 & CPAIOR 2013 & 7 & 10 10 14 & 24 25 & \ref{b:GuSS13} & \ref{c:GuSS13}\\
HeinzKB13 \href{https://doi.org/10.1007/978-3-642-38171-3_2}{HeinzKB13} & \hyperref[auth:a133]{S. Heinz}, \hyperref[auth:a331]{W.-Y. Ku}, \hyperref[auth:a89]{J. C. Beck} & Recent Improvements Using Constraint Integer Programming for Resource Allocation and Scheduling & \href{../works/HeinzKB13.pdf}{Yes} & \cite{HeinzKB13} & 2013 & CPAIOR 2013 & 16 & 9 9 17 & 15 20 & \ref{b:HeinzKB13} & n/a\\
KelarevaTK13 \href{https://doi.org/10.1007/978-3-642-38171-3_8}{KelarevaTK13} & \hyperref[auth:a332]{E. Kelareva}, \hyperref[auth:a333]{K. Tierney}, \hyperref[auth:a334]{P. Kilby} & {CP} Methods for Scheduling and Routing with Time-Dependent Task Costs & \href{../works/KelarevaTK13.pdf}{Yes} & \cite{KelarevaTK13} & 2013 & CPAIOR 2013 & 17 & 16 15 13 & 28 39 & \ref{b:KelarevaTK13} & \ref{c:KelarevaTK13}\\
LetortCB13 \href{https://doi.org/10.1007/978-3-642-38171-3_10}{LetortCB13} & \hyperref[auth:a127]{A. Letort}, \hyperref[auth:a91]{M. Carlsson}, \hyperref[auth:a128]{N. Beldiceanu} & A Synchronized Sweep Algorithm for the \emph{k-dimensional cumulative} Constraint & \href{../works/LetortCB13.pdf}{Yes} & \cite{LetortCB13} & 2013 & CPAIOR 2013 & 16 & 3 3 4 & 10 16 & \ref{b:LetortCB13} & \ref{c:LetortCB13}\\
SchuttFS13a \href{https://doi.org/10.1007/978-3-642-38171-3_16}{SchuttFS13a} & \hyperref[auth:a124]{A. Schutt}, \hyperref[auth:a154]{T. Feydy}, \hyperref[auth:a125]{P. J. Stuckey} & \cellcolor{green!10}Explaining Time-Table-Edge-Finding Propagation for the Cumulative Resource Constraint & \href{../works/SchuttFS13a.pdf}{Yes} & \cite{SchuttFS13a} & 2013 & CPAIOR 2013 & 17 & 20 20 33 & 27 35 & \ref{b:SchuttFS13a} & \ref{c:SchuttFS13a}\\
BillautHL12 \href{https://doi.org/10.1007/978-3-642-29828-8_5}{BillautHL12} & \hyperref[auth:a337]{J.-C. Billaut}, \hyperref[auth:a1]{E. Hebrard}, \hyperref[auth:a3]{P. Lopez} & \cellcolor{green!10}Complete Characterization of Near-Optimal Sequences for the Two-Machine Flow Shop Scheduling Problem & \href{../works/BillautHL12.pdf}{Yes} & \cite{BillautHL12} & 2012 & CPAIOR 2012 & 15 & 1 0 2 & 19 28 & \ref{b:BillautHL12} & n/a\\
BonfiettiLBM12 \href{https://doi.org/10.1007/978-3-642-29828-8_6}{BonfiettiLBM12} & \hyperref[auth:a198]{A. Bonfietti}, \hyperref[auth:a142]{M. Lombardi}, \hyperref[auth:a245]{L. Benini}, \hyperref[auth:a143]{M. Milano} & Global Cyclic Cumulative Constraint & \href{../works/BonfiettiLBM12.pdf}{Yes} & \cite{BonfiettiLBM12} & 2012 & CPAIOR 2012 & 16 & 2 2 4 & 11 18 & \ref{b:BonfiettiLBM12} & n/a\\
HeinzB12 \href{https://doi.org/10.1007/978-3-642-29828-8_14}{HeinzB12} & \hyperref[auth:a133]{S. Heinz}, \hyperref[auth:a89]{J. C. Beck} & Reconsidering Mixed Integer Programming and MIP-Based Hybrids for Scheduling & \href{../works/HeinzB12.pdf}{Yes} & \cite{HeinzB12} & 2012 & CPAIOR 2012 & 17 & 8 7 12 & 21 28 & \ref{b:HeinzB12} & n/a\\
RendlPHPR12 \href{https://doi.org/10.1007/978-3-642-29828-8_22}{RendlPHPR12} & \hyperref[auth:a338]{A. Rendl}, \hyperref[auth:a339]{M. Prandtstetter}, \hyperref[auth:a340]{G. Hiermann}, \hyperref[auth:a341]{J. Puchinger}, \hyperref[auth:a342]{G. R. Raidl} & \cellcolor{green!10}Hybrid Heuristics for Multimodal Homecare Scheduling & \href{../works/RendlPHPR12.pdf}{Yes} & \cite{RendlPHPR12} & 2012 & CPAIOR 2012 & 17 & 14 14 24 & 14 22 & \ref{b:RendlPHPR12} & n/a\\
SchuttCSW12 \href{https://doi.org/10.1007/978-3-642-29828-8_24}{SchuttCSW12} & \hyperref[auth:a124]{A. Schutt}, \hyperref[auth:a343]{G. Chu}, \hyperref[auth:a125]{P. J. Stuckey}, \hyperref[auth:a117]{M. G. Wallace} & Maximising the Net Present Value for Resource-Constrained Project Scheduling & \href{../works/SchuttCSW12.pdf}{Yes} & \cite{SchuttCSW12} & 2012 & CPAIOR 2012 & 17 & 18 19 24 & 21 25 & \ref{b:SchuttCSW12} & n/a\\
ChapadosJR11 \href{https://doi.org/10.1007/978-3-642-21311-3_7}{ChapadosJR11} & \hyperref[auth:a344]{N. Chapados}, \hyperref[auth:a345]{M. Joliveau}, \hyperref[auth:a326]{L.-M. Rousseau} & Retail Store Workforce Scheduling by Expected Operating Income Maximization & \href{../works/ChapadosJR11.pdf}{Yes} & \cite{ChapadosJR11} & 2011 & CPAIOR 2011 & 6 & 5 5 12 & 12 13 & \ref{b:ChapadosJR11} & n/a\\
EdisO11 \href{https://doi.org/10.1007/978-3-642-21311-3_10}{EdisO11} & \hyperref[auth:a346]{E. B. Edis}, \hyperref[auth:a347]{C. Oguz} & Parallel Machine Scheduling with Additional Resources: {A} Lagrangian-Based Constraint Programming Approach & \href{../works/EdisO11.pdf}{Yes} & \cite{EdisO11} & 2011 & CPAIOR 2011 & 7 & 5 5 15 & 16 21 & \ref{b:EdisO11} & n/a\\
LahimerLH11 \href{https://doi.org/10.1007/978-3-642-21311-3_12}{LahimerLH11} & \hyperref[auth:a349]{A. Lahimer}, \hyperref[auth:a3]{P. Lopez}, \hyperref[auth:a350]{M. Haouari} & \cellcolor{green!10}Climbing Depth-Bounded Adjacent Discrepancy Search for Solving Hybrid Flow Shop Scheduling Problems with Multiprocessor Tasks & \href{../works/LahimerLH11.pdf}{Yes} & \cite{LahimerLH11} & 2011 & CPAIOR 2011 & 14 & 3 3 5 & 15 23 & \ref{b:LahimerLH11} & n/a\\
LombardiBMB11 \href{https://doi.org/10.1007/978-3-642-21311-3_14}{LombardiBMB11} & \hyperref[auth:a142]{M. Lombardi}, \hyperref[auth:a198]{A. Bonfietti}, \hyperref[auth:a143]{M. Milano}, \hyperref[auth:a245]{L. Benini} & Precedence Constraint Posting for Cyclic Scheduling Problems & \href{../works/LombardiBMB11.pdf}{Yes} & \cite{LombardiBMB11} & 2011 & CPAIOR 2011 & 17 & 1 1 3 & 13 26 & \ref{b:LombardiBMB11} & n/a\\
Vilim11 \href{https://doi.org/10.1007/978-3-642-21311-3_22}{Vilim11} & \hyperref[auth:a121]{P. Vil{\'{\i}}m} & Timetable Edge Finding Filtering Algorithm for Discrete Cumulative Resources & \href{../works/Vilim11.pdf}{Yes} & \cite{Vilim11} & 2011 & CPAIOR 2011 & 16 & 28 29 46 & 6 11 & \ref{b:Vilim11} & n/a\\
BertholdHLMS10 \href{https://doi.org/10.1007/978-3-642-13520-0_34}{BertholdHLMS10} & \hyperref[auth:a351]{T. Berthold}, \hyperref[auth:a133]{S. Heinz}, \hyperref[auth:a352]{M. E. L{\"{u}}bbecke}, \hyperref[auth:a353]{R. H. M{\"{o}}hring}, \hyperref[auth:a134]{J. Schulz} & A Constraint Integer Programming Approach for Resource-Constrained Project Scheduling & \href{../works/BertholdHLMS10.pdf}{Yes} & \cite{BertholdHLMS10} & 2010 & CPAIOR 2010 & 5 & 28 27 34 & 10 13 & \ref{b:BertholdHLMS10} & n/a\\
CobanH10 \href{https://doi.org/10.1007/978-3-642-13520-0_11}{CobanH10} & \hyperref[auth:a335]{E. Coban}, \hyperref[auth:a160]{J. N. Hooker} & \cellcolor{green!10}Single-Facility Scheduling over Long Time Horizons by Logic-Based Benders Decomposition & \href{../works/CobanH10.pdf}{Yes} & \cite{CobanH10} & 2010 & CPAIOR 2010 & 5 & 9 9 10 & 9 11 & \ref{b:CobanH10} & n/a\\
Davenport10 \href{https://doi.org/10.1007/978-3-642-13520-0_12}{Davenport10} & \hyperref[auth:a248]{A. J. Davenport} & \cellcolor{green!10}Integrated Maintenance Scheduling for Semiconductor Manufacturing & \href{../works/Davenport10.pdf}{Yes} & \cite{Davenport10} & 2010 & CPAIOR 2010 & 5 & 9 9 13 & 2 2 & \ref{b:Davenport10} & n/a\\
GrimesH10 \href{https://doi.org/10.1007/978-3-642-13520-0_19}{GrimesH10} & \hyperref[auth:a181]{D. Grimes}, \hyperref[auth:a1]{E. Hebrard} & \cellcolor{green!10}Job Shop Scheduling with Setup Times and Maximal Time-Lags: {A} Simple Constraint Programming Approach & \href{../works/GrimesH10.pdf}{Yes} & \cite{GrimesH10} & 2010 & CPAIOR 2010 & 15 & 13 13 20 & 20 29 & \ref{b:GrimesH10} & n/a\\
Acuna-AgostMFG09 \href{https://doi.org/10.1007/978-3-642-01929-6_24}{Acuna-AgostMFG09} & \hyperref[auth:a354]{R. Acuna-Agost}, \hyperref[auth:a355]{P. Michelon}, \hyperref[auth:a356]{D. Feillet}, \hyperref[auth:a357]{S. Gueye} & Constraint Programming and Mixed Integer Linear Programming for Rescheduling Trains under Disrupted Operations & \href{../works/Acuna-AgostMFG09.pdf}{Yes} & \cite{Acuna-AgostMFG09} & 2009 & CPAIOR 2009 & 2 & 3 3 5 & 2 5 & \ref{b:Acuna-AgostMFG09} & n/a\\
Laborie09 \href{https://doi.org/10.1007/978-3-642-01929-6_12}{Laborie09} & \hyperref[auth:a118]{P. Laborie} & {IBM} {ILOG} {CP} Optimizer for Detailed Scheduling Illustrated on Three Problems & \href{../works/Laborie09.pdf}{Yes} & \cite{Laborie09} & 2009 & CPAIOR 2009 & 15 & 53 52 91 & 2 9 & \ref{b:Laborie09} & n/a\\
Vilim09a \href{https://doi.org/10.1007/978-3-642-01929-6_22}{Vilim09a} & \hyperref[auth:a121]{P. Vil{\'{\i}}m} & Max Energy Filtering Algorithm for Discrete Cumulative Resources & \href{../works/Vilim09a.pdf}{Yes} & \cite{Vilim09a} & 2009 & CPAIOR 2009 & 15 & 13 13 18 & 4 7 & \ref{b:Vilim09a} & n/a\\
AchterbergBKW08 \href{https://doi.org/10.1007/978-3-540-68155-7_4}{AchterbergBKW08} & \hyperref[auth:a1045]{T. Achterberg}, \hyperref[auth:a351]{T. Berthold}, \hyperref[auth:a1168]{T. Koch}, \hyperref[auth:a1169]{K. Wolter} & Constraint Integer Programming: {A} New Approach to Integrate {CP} and {MIP} & \href{../works/AchterbergBKW08.pdf}{Yes} & \cite{AchterbergBKW08} & 2008 & CPAIOR 2008 & 15 & 80 80 125 & 25 43 & \ref{b:AchterbergBKW08} & n/a\\
BarlattCG08 \href{https://doi.org/10.1007/978-3-540-68155-7_24}{BarlattCG08} & \hyperref[auth:a361]{A. Barlatt}, \hyperref[auth:a362]{A. M. Cohn}, \hyperref[auth:a363]{O. Y. Gusikhin} & A Hybrid Approach for Solving Shift-Selection and Task-Sequencing Problems & \href{../works/BarlattCG08.pdf}{Yes} & \cite{BarlattCG08} & 2008 & CPAIOR 2008 & 5 & 1 1 1 & 9 10 & \ref{b:BarlattCG08} & n/a\\
BeldiceanuCP08 \href{https://doi.org/10.1007/978-3-540-68155-7_5}{BeldiceanuCP08} & \hyperref[auth:a128]{N. Beldiceanu}, \hyperref[auth:a91]{M. Carlsson}, \hyperref[auth:a358]{E. Poder} & New Filtering for the cumulative Constraint in the Context of Non-Overlapping Rectangles & \href{../works/BeldiceanuCP08.pdf}{Yes} & \cite{BeldiceanuCP08} & 2008 & CPAIOR 2008 & 15 & 8 8 18 & 9 13 & \ref{b:BeldiceanuCP08} & n/a\\
BeniniLMMR08 \href{https://doi.org/10.1007/978-3-540-68155-7_6}{BeniniLMMR08} & \hyperref[auth:a245]{L. Benini}, \hyperref[auth:a142]{M. Lombardi}, \hyperref[auth:a1153]{M. Mantovani}, \hyperref[auth:a143]{M. Milano}, \hyperref[auth:a718]{M. Ruggiero} & Multi-stage Benders Decomposition for Optimizing Multicore Architectures & \href{../works/BeniniLMMR08.pdf}{Yes} & \cite{BeniniLMMR08} & 2008 & CPAIOR 2008 & 15 & 12 12 17 & 13 17 & \ref{b:BeniniLMMR08} & n/a\\
DoomsH08 \href{https://doi.org/10.1007/978-3-540-68155-7_8}{DoomsH08} & \hyperref[auth:a359]{G. Dooms}, \hyperref[auth:a148]{P. V. Hentenryck} & Gap Reduction Techniques for Online Stochastic Project Scheduling & \href{../works/DoomsH08.pdf}{Yes} & \cite{DoomsH08} & 2008 & CPAIOR 2008 & 16 & 1 1 0 & 2 8 & \ref{b:DoomsH08} & n/a\\
HentenryckM08 \href{https://doi.org/10.1007/978-3-540-68155-7_41}{HentenryckM08} & \hyperref[auth:a148]{P. V. Hentenryck}, \hyperref[auth:a32]{L. Michel} & The Steel Mill Slab Design Problem Revisited & \href{../works/HentenryckM08.pdf}{Yes} & \cite{HentenryckM08} & 2008 & CPAIOR 2008 & 5 & 13 14 23 & 3 7 & \ref{b:HentenryckM08} & n/a\\
LauLN08 \href{https://doi.org/10.1007/978-3-540-68155-7_33}{LauLN08} & \hyperref[auth:a364]{H. C. Lau}, \hyperref[auth:a365]{K. W. Lye}, \hyperref[auth:a366]{V. B. Nguyen} & A Combinatorial Auction Framework for Solving Decentralized Scheduling Problems (Extended Abstract) & \href{../works/LauLN08.pdf}{Yes} & \cite{LauLN08} & 2008 & CPAIOR 2008 & 5 & 0 0 0 & 4 5 & \ref{b:LauLN08} & n/a\\
WatsonB08 \href{https://doi.org/10.1007/978-3-540-68155-7_21}{WatsonB08} & \hyperref[auth:a360]{J.-P. Watson}, \hyperref[auth:a89]{J. C. Beck} & \cellcolor{green!10}A Hybrid Constraint Programming / Local Search Approach to the Job-Shop Scheduling Problem & \href{../works/WatsonB08.pdf}{Yes} & \cite{WatsonB08} & 2008 & CPAIOR 2008 & 15 & 14 14 24 & 17 25 & \ref{b:WatsonB08} & n/a\\
AkkerDH07 \href{https://doi.org/10.1007/978-3-540-72397-4_27}{AkkerDH07} & \hyperref[auth:a372]{J. M. van den Akker}, \hyperref[auth:a373]{G. Diepen}, \hyperref[auth:a374]{J. A. Hoogeveen} & \cellcolor{green!10}A Column Generation Based Destructive Lower Bound for Resource Constrained Project Scheduling Problems & \href{../works/AkkerDH07.pdf}{Yes} & \cite{AkkerDH07} & 2007 & CPAIOR 2007 & 15 & 2 2 4 & 8 10 & \ref{b:AkkerDH07} & n/a\\
BeldiceanuP07 \href{https://doi.org/10.1007/978-3-540-72397-4_16}{BeldiceanuP07} & \hyperref[auth:a128]{N. Beldiceanu}, \hyperref[auth:a358]{E. Poder} & A Continuous Multi-resources \emph{cumulative} Constraint with Positive-Negative Resource Consumption-Production & \href{../works/BeldiceanuP07.pdf}{Yes} & \cite{BeldiceanuP07} & 2007 & CPAIOR 2007 & 15 & 4 4 6 & 7 12 & \ref{b:BeldiceanuP07} & n/a\\
KeriK07 \href{https://doi.org/10.1007/978-3-540-72397-4_10}{KeriK07} & \hyperref[auth:a367]{A. K{\'{e}}ri}, \hyperref[auth:a155]{T. Kis} & Computing Tight Time Windows for {RCPSPWET} with the Primal-Dual Method & \href{../works/KeriK07.pdf}{Yes} & \cite{KeriK07} & 2007 & CPAIOR 2007 & 14 & 1 1 2 & 13 21 & \ref{b:KeriK07} & n/a\\
KovacsB07 \href{https://doi.org/10.1007/978-3-540-72397-4_9}{KovacsB07} & \hyperref[auth:a146]{A. Kov{\'{a}}cs}, \hyperref[auth:a89]{J. C. Beck} & \cellcolor{green!10}A Global Constraint for Total Weighted Completion Time & \href{../works/KovacsB07.pdf}{Yes} & \cite{KovacsB07} & 2007 & CPAIOR 2007 & 15 & 2 2 4 & 12 18 & \ref{b:KovacsB07} & n/a\\
MonetteDD07 \href{https://doi.org/10.1007/978-3-540-72397-4_14}{MonetteDD07} & \hyperref[auth:a149]{J.-N. Monette}, \hyperref[auth:a151]{Y. Deville}, \hyperref[auth:a368]{P. Dupont} & \cellcolor{green!10}A Position-Based Propagator for the Open-Shop Problem & \href{../works/MonetteDD07.pdf}{Yes} & \cite{MonetteDD07} & 2007 & CPAIOR 2007 & 14 & 0 0 1 & 12 15 & \ref{b:MonetteDD07} & n/a\\
RossiTHP07 \href{https://doi.org/10.1007/978-3-540-72397-4_17}{RossiTHP07} & \hyperref[auth:a369]{R. Rossi}, \hyperref[auth:a370]{A. Tarim}, \hyperref[auth:a137]{B. Hnich}, \hyperref[auth:a371]{S. D. Prestwich} & \cellcolor{green!10}Replenishment Planning for Stochastic Inventory Systems with Shortage Cost & \href{../works/RossiTHP07.pdf}{Yes} & \cite{RossiTHP07} & 2007 & CPAIOR 2007 & 15 & 6 6 9 & 10 17 & \ref{b:RossiTHP07} & n/a\\
BeniniBGM06 \href{https://doi.org/10.1007/11757375_6}{BeniniBGM06} & \hyperref[auth:a245]{L. Benini}, \hyperref[auth:a375]{D. Bertozzi}, \hyperref[auth:a376]{A. Guerri}, \hyperref[auth:a143]{M. Milano} & \cellcolor{green!10}Allocation, Scheduling and Voltage Scaling on Energy Aware MPSoCs & \href{../works/BeniniBGM06.pdf}{Yes} & \cite{BeniniBGM06} & 2006 & CPAIOR 2006 & 15 & 18 19 19 & 10 12 & \ref{b:BeniniBGM06} & n/a\\
KovacsV06 \href{https://doi.org/10.1007/11757375_13}{KovacsV06} & \hyperref[auth:a146]{A. Kov{\'{a}}cs}, \hyperref[auth:a278]{J. V{\'{a}}ncza} & \cellcolor{green!10}Progressive Solutions: {A} Simple but Efficient Dominance Rule for Practical {RCPSP} & \href{../works/KovacsV06.pdf}{Yes} & \cite{KovacsV06} & 2006 & CPAIOR 2006 & 13 & 2 2 3 & 7 13 & \ref{b:KovacsV06} & n/a\\
RasmussenT06 \href{https://doi.org/10.1007/11757375_15}{RasmussenT06} & \hyperref[auth:a1404]{R. V. Rasmussen}, \hyperref[auth:a1390]{M. A. Trick} & The Timetable Constrained Distance Minimization Problem & \href{../works/RasmussenT06.pdf}{Yes} & \cite{RasmussenT06} & 2006 & CPAIOR 2006 & 15 & 10 12 19 & 14 16 & \ref{b:RasmussenT06} & n/a\\
ChuX05 \href{https://doi.org/10.1007/11493853_10}{ChuX05} & \hyperref[auth:a377]{Y. Chu}, \hyperref[auth:a378]{Q. Xia} & A Hybrid Algorithm for a Class of Resource Constrained Scheduling Problems & \href{../works/ChuX05.pdf}{Yes} & \cite{ChuX05} & 2005 & CPAIOR 2005 & 15 & 13 13 21 & 13 15 & \ref{b:ChuX05} & n/a\\
FrankK05 \href{https://doi.org/10.1007/11493853_15}{FrankK05} & \hyperref[auth:a379]{J. Frank}, \hyperref[auth:a380]{E. K{\"{u}}rkl{\"{u}}} & Mixed Discrete and Continuous Algorithms for Scheduling Airborne Astronomy Observations & \href{../works/FrankK05.pdf}{Yes} & \cite{FrankK05} & 2005 & CPAIOR 2005 & 18 & 4 4 4 & 4 15 & \ref{b:FrankK05} & n/a\\
Hooker05b \href{https://doi.org/10.1007/11493853_19}{Hooker05b} & \hyperref[auth:a160]{J. N. Hooker} & \cellcolor{green!10}A Search-Infer-and-Relax Framework for Integrating Solution Methods & \href{../works/Hooker05b.pdf}{Yes} & \cite{Hooker05b} & 2005 & CPAIOR 2005 & 15 & 7 6 12 & 19 26 & \ref{b:Hooker05b} & n/a\\
Vilim05 \href{https://doi.org/10.1007/11493853_29}{Vilim05} & \hyperref[auth:a121]{P. Vil{\'{\i}}m} & Computing Explanations for the Unary Resource Constraint & \href{../works/Vilim05.pdf}{Yes} & \cite{Vilim05} & 2005 & CPAIOR 2005 & 14 & 5 5 6 & 8 11 & \ref{b:Vilim05} & n/a\\
ArtiguesBF04 \href{https://doi.org/10.1007/978-3-540-24664-0_3}{ArtiguesBF04} & \hyperref[auth:a6]{C. Artigues}, \hyperref[auth:a383]{S. Belmokhtar}, \hyperref[auth:a356]{D. Feillet} & A New Exact Solution Algorithm for the Job Shop Problem with Sequence-Dependent Setup Times & \href{../works/ArtiguesBF04.pdf}{Yes} & \cite{ArtiguesBF04} & 2004 & CPAIOR 2004 & 13 & 16 17 17 & 9 16 & \ref{b:ArtiguesBF04} & n/a\\
HentenryckM04 \href{https://doi.org/10.1007/978-3-540-24664-0_22}{HentenryckM04} & \hyperref[auth:a148]{P. V. Hentenryck}, \hyperref[auth:a32]{L. Michel} & \cellcolor{green!10}Scheduling Abstractions for Local Search & \href{../works/HentenryckM04.pdf}{Yes} & \cite{HentenryckM04} & 2004 & CPAIOR 2004 & 16 & 12 12 16 & 14 21 & \ref{b:HentenryckM04} & n/a\\
MaraveliasG04 \href{https://doi.org/10.1007/978-3-540-24664-0_1}{MaraveliasG04} & \hyperref[auth:a381]{C. T. Maravelias}, \hyperref[auth:a382]{I. E. Grossmann} & Using {MILP} and {CP} for the Scheduling of Batch Chemical Processes & \href{../works/MaraveliasG04.pdf}{Yes} & \cite{MaraveliasG04} & 2004 & CPAIOR 2004 & 20 & 15 14 23 & 15 23 & \ref{b:MaraveliasG04} & n/a\\
Sadykov04 \href{https://doi.org/10.1007/978-3-540-24664-0_31}{Sadykov04} & \hyperref[auth:a384]{R. Sadykov} & A Hybrid Branch-And-Cut Algorithm for the One-Machine Scheduling Problem & \href{../works/Sadykov04.pdf}{Yes} & \cite{Sadykov04} & 2004 & CPAIOR 2004 & 7 & 11 11 10 & 7 10 & \ref{b:Sadykov04} & n/a\\
Vilim04 \href{https://doi.org/10.1007/978-3-540-24664-0_23}{Vilim04} & \hyperref[auth:a121]{P. Vil{\'{\i}}m} & O(n log n) Filtering Algorithms for Unary Resource Constraint & \href{../works/Vilim04.pdf}{Yes} & \cite{Vilim04} & 2004 & CPAIOR 2004 & 13 & 22 22 32 & 5 11 & \ref{b:Vilim04} & n/a\\
\end{longtable}
}

\subsection{CSCLP}

\index{CSCLP}
{\scriptsize
\begin{longtable}{>{\raggedright\arraybackslash}p{3cm}>{\raggedright\arraybackslash}p{4.5cm}>{\raggedright\arraybackslash}p{6.0cm}rrrp{2.5cm}rp{1cm}p{1cm}rr}
\rowcolor{white}\caption{Papers in Conference Series CSCLP (Total 4) (Total 4)}\\ \toprule
\rowcolor{white}\shortstack{Key\\Source} & Authors & Title (Colored by Open Access)& LC & Cite & Year & \shortstack{Conference\\/Journal\\/School} & Pages & \shortstack{Cites\\OC XR\\SC} & \shortstack{Refs\\OC\\XR} & b & c \\ \midrule\endhead
\bottomrule
\endfoot
SimonisH11 \href{http://dx.doi.org/10.1007/978-3-642-19486-3_5}{SimonisH11} & \hyperref[auth:a17]{H. Simonis}, \hyperref[auth:a906]{T. Hadzic} & A Resource Cost Aware Cumulative & \href{../works/SimonisH11.pdf}{Yes} & \cite{SimonisH11} & 2011 & CSCLP 2011 & 14 & 3 3 7 & 9 10 & \ref{b:SimonisH11} & n/a\\
Wolf11 \href{http://dx.doi.org/10.1007/978-3-642-19486-3_8}{Wolf11} & \hyperref[auth:a51]{A. Wolf} & Constraint-Based Modeling and Scheduling of Clinical Pathways & \href{../works/Wolf11.pdf}{Yes} & \cite{Wolf11} & 2011 & CSCLP 2011 & 17 & 5 5 6 & 19 26 & \ref{b:Wolf11} & n/a\\
Wallace06 \href{http://dx.doi.org/10.1007/978-3-540-73817-6_1}{Wallace06} & \hyperref[auth:a117]{M. G. Wallace} & Hybrid Algorithms in Constraint Programming & \href{../works/Wallace06.pdf}{Yes} & \cite{Wallace06} & 2006 & CSCLP 2006 & 32 & 1 1 3 & 35 63 & \ref{b:Wallace06} & n/a\\
Wolf05 \href{http://dx.doi.org/10.1007/11402763_15}{Wolf05} & \hyperref[auth:a51]{A. Wolf} & Better Propagation for Non-preemptive Single-Resource Constraint Problems & \href{../works/Wolf05.pdf}{Yes} & \cite{Wolf05} & 2005 & CSCLP 2005 & 15 & 4 4 4 & 8 12 & \ref{b:Wolf05} & n/a\\
\end{longtable}
}

\subsection{CSE}

\index{CSE}
{\scriptsize
\begin{longtable}{>{\raggedright\arraybackslash}p{3cm}>{\raggedright\arraybackslash}p{4.5cm}>{\raggedright\arraybackslash}p{6.0cm}rrrp{2.5cm}rp{1cm}p{1cm}rr}
\rowcolor{white}\caption{Papers in Conference Series CSE (Total 1) (Total 1)}\\ \toprule
\rowcolor{white}\shortstack{Key\\Source} & Authors & Title (Colored by Open Access)& LC & Cite & Year & \shortstack{Conference\\/Journal\\/School} & Pages & \shortstack{Cites\\OC XR\\SC} & \shortstack{Refs\\OC\\XR} & b & c \\ \midrule\endhead
\bottomrule
\endfoot
MouraSCL08a \href{https://doi.org/10.1109/CSE.2008.24}{MouraSCL08a} & \hyperref[auth:a159]{A. V. Moura}, \hyperref[auth:a170]{C. C. de Souza}, \hyperref[auth:a157]{A. A. Cir{\'{e}}}, \hyperref[auth:a156]{T. M. T. Lopes} & Heuristics and Constraint Programming Hybridizations for a Real Pipeline Planning and Scheduling Problem & \href{../works/MouraSCL08a.pdf}{Yes} & \cite{MouraSCL08a} & 2008 & CSE 2008 & 8 & 5 5 10 & 14 21 & \ref{b:MouraSCL08a} & n/a\\
\end{longtable}
}

\subsection{Canadian AI}

\index{Canadian AI}
{\scriptsize
\begin{longtable}{>{\raggedright\arraybackslash}p{3cm}>{\raggedright\arraybackslash}p{4.5cm}>{\raggedright\arraybackslash}p{6.0cm}rrrp{2.5cm}rp{1cm}p{1cm}rr}
\rowcolor{white}\caption{Papers in Conference Series Canadian AI (Total 1) (Total 1)}\\ \toprule
\rowcolor{white}\shortstack{Key\\Source} & Authors & Title (Colored by Open Access)& LC & Cite & Year & \shortstack{Conference\\/Journal\\/School} & Pages & \shortstack{Cites\\OC XR\\SC} & \shortstack{Refs\\OC\\XR} & b & c \\ \midrule\endhead
\bottomrule
\endfoot
TanT18 \href{http://dx.doi.org/10.1007/978-3-319-89656-4_5}{TanT18} & \hyperref[auth:a909]{Y. Tan}, \hyperref[auth:a818]{D. Terekhov} & Logic-Based Benders Decomposition for Two-Stage Flexible Flow Shop Scheduling with Unrelated Parallel Machines & \href{../works/TanT18.pdf}{Yes} & \cite{TanT18} & 2018 & Canadian AI 2018 & 12 & 1 1 2 & 23 23 & \ref{b:TanT18} & n/a\\
\end{longtable}
}

\subsection{Chinese Control and Decision Conference}

\index{Chinese Control and Decision Conference}
{\scriptsize
\begin{longtable}{>{\raggedright\arraybackslash}p{3cm}>{\raggedright\arraybackslash}p{4.5cm}>{\raggedright\arraybackslash}p{6.0cm}rrrp{2.5cm}rp{1cm}p{1cm}rr}
\rowcolor{white}\caption{Papers in Conference Series Chinese Control and Decision Conference (Total 1) (Total 1)}\\ \toprule
\rowcolor{white}\shortstack{Key\\Source} & Authors & Title (Colored by Open Access)& LC & Cite & Year & \shortstack{Conference\\/Journal\\/School} & Pages & \shortstack{Cites\\OC XR\\SC} & \shortstack{Refs\\OC\\XR} & b & c \\ \midrule\endhead
\bottomrule
\endfoot
TanSD10 \href{http://dx.doi.org/10.1109/ccdc.2010.5499100}{TanSD10} & \hyperref[auth:a1184]{Y. Tan}, \hyperref[auth:a465]{S. Liu}, \hyperref[auth:a1220]{D. Wang} & A constraint programming-based branch and bound algorithm for job shop problems & \href{../works/TanSD10.pdf}{Yes} & \cite{TanSD10} & 2010 & Chinese Control and Decision Conference 2010 & 6 & 1 0 4 & 11 0 & \ref{b:TanSD10} & n/a\\
\end{longtable}
}

\subsection{CoDIT}

\index{CoDIT}
{\scriptsize
\begin{longtable}{>{\raggedright\arraybackslash}p{3cm}>{\raggedright\arraybackslash}p{4.5cm}>{\raggedright\arraybackslash}p{6.0cm}rrrp{2.5cm}rp{1cm}p{1cm}rr}
\rowcolor{white}\caption{Papers in Conference Series CoDIT (Total 1) (Total 1)}\\ \toprule
\rowcolor{white}\shortstack{Key\\Source} & Authors & Title (Colored by Open Access)& LC & Cite & Year & \shortstack{Conference\\/Journal\\/School} & Pages & \shortstack{Cites\\OC XR\\SC} & \shortstack{Refs\\OC\\XR} & b & c \\ \midrule\endhead
\bottomrule
\endfoot
OujanaAYB22 \href{https://doi.org/10.1109/CoDIT55151.2022.9803972}{OujanaAYB22} & \hyperref[auth:a453]{S. Oujana}, \hyperref[auth:a454]{L. Amodeo}, \hyperref[auth:a455]{F. Yalaoui}, \hyperref[auth:a456]{D. Brodart} & Solving a realistic hybrid and flexible flow shop scheduling problem through constraint programming: industrial case in a packaging company & \href{../works/OujanaAYB22.pdf}{Yes} & \cite{OujanaAYB22} & 2022 & CoDIT 2022 & 6 & 1 1 2 & 21 24 & \ref{b:OujanaAYB22} & \ref{c:OujanaAYB22}\\
\end{longtable}
}

\subsection{Constraint Programming}

\index{Constraint Programming}
{\scriptsize
\begin{longtable}{>{\raggedright\arraybackslash}p{3cm}>{\raggedright\arraybackslash}p{4.5cm}>{\raggedright\arraybackslash}p{6.0cm}rrrp{2.5cm}rp{1cm}p{1cm}rr}
\rowcolor{white}\caption{Papers in Conference Series Constraint Programming (Total 1) (Total 1)}\\ \toprule
\rowcolor{white}\shortstack{Key\\Source} & Authors & Title (Colored by Open Access)& LC & Cite & Year & \shortstack{Conference\\/Journal\\/School} & Pages & \shortstack{Cites\\OC XR\\SC} & \shortstack{Refs\\OC\\XR} & b & c \\ \midrule\endhead
\bottomrule
\endfoot
Wallace94 \href{}{Wallace94} & \hyperref[auth:a117]{M. G. Wallace} & Applying Constraints for Scheduling & No & \cite{Wallace94} & 1994 & Constraint Programming 1994 & 19 & 0 0 0 & 0 0 & No & n/a\\
\end{longtable}
}

\subsection{DC SIAAI}

\index{DC SIAAI}
{\scriptsize
\begin{longtable}{>{\raggedright\arraybackslash}p{3cm}>{\raggedright\arraybackslash}p{4.5cm}>{\raggedright\arraybackslash}p{6.0cm}rrrp{2.5cm}rp{1cm}p{1cm}rr}
\rowcolor{white}\caption{Papers in Conference Series DC SIAAI (Total 1) (Total 1)}\\ \toprule
\rowcolor{white}\shortstack{Key\\Source} & Authors & Title (Colored by Open Access)& LC & Cite & Year & \shortstack{Conference\\/Journal\\/School} & Pages & \shortstack{Cites\\OC XR\\SC} & \shortstack{Refs\\OC\\XR} & b & c \\ \midrule\endhead
\bottomrule
\endfoot
BonfiettiM12 \href{https://ceur-ws.org/Vol-926/paper2.pdf}{BonfiettiM12} & \hyperref[auth:a198]{A. Bonfietti}, \hyperref[auth:a143]{M. Milano} & A Constraint-based Approach to Cyclic Resource-Constrained Scheduling Problem & \href{../works/BonfiettiM12.pdf}{Yes} & \cite{BonfiettiM12} & 2012 & DC SIAAI 2012 & 3 & 0 0 0 & 0 0 & \ref{b:BonfiettiM12} & n/a\\
\end{longtable}
}

\subsection{DIMACS}

\index{DIMACS}
{\scriptsize
\begin{longtable}{>{\raggedright\arraybackslash}p{3cm}>{\raggedright\arraybackslash}p{4.5cm}>{\raggedright\arraybackslash}p{6.0cm}rrrp{2.5cm}rp{1cm}p{1cm}rr}
\rowcolor{white}\caption{Papers in Conference Series DIMACS (Total 1) (Total 1)}\\ \toprule
\rowcolor{white}\shortstack{Key\\Source} & Authors & Title (Colored by Open Access)& LC & Cite & Year & \shortstack{Conference\\/Journal\\/School} & Pages & \shortstack{Cites\\OC XR\\SC} & \shortstack{Refs\\OC\\XR} & b & c \\ \midrule\endhead
\bottomrule
\endfoot
PembertonG98 \href{https://doi.org/10.1090/dimacs/057/06}{PembertonG98} & \hyperref[auth:a684]{J. C. Pemberton}, \hyperref[auth:a685]{F. G. III} & A constraint-based approach to satellite scheduling & \href{../works/PembertonG98.pdf}{Yes} & \cite{PembertonG98} & 1998 & DIMACS 1998 & 14 & 26 0 0 & 0 0 & \ref{b:PembertonG98} & n/a\\
\end{longtable}
}

\subsection{DSD}

\index{DSD}
{\scriptsize
\begin{longtable}{>{\raggedright\arraybackslash}p{3cm}>{\raggedright\arraybackslash}p{4.5cm}>{\raggedright\arraybackslash}p{6.0cm}rrrp{2.5cm}rp{1cm}p{1cm}rr}
\rowcolor{white}\caption{Papers in Conference Series DSD (Total 1) (Total 1)}\\ \toprule
\rowcolor{white}\shortstack{Key\\Source} & Authors & Title (Colored by Open Access)& LC & Cite & Year & \shortstack{Conference\\/Journal\\/School} & Pages & \shortstack{Cites\\OC XR\\SC} & \shortstack{Refs\\OC\\XR} & b & c \\ \midrule\endhead
\bottomrule
\endfoot
WolinskiKG04 \href{https://doi.org/10.1109/DSD.2004.1333291}{WolinskiKG04} & \hyperref[auth:a659]{C. Wolinski}, \hyperref[auth:a660]{K. Kuchcinski}, \hyperref[auth:a661]{M. B. Gokhale} & A Constraints Programming Approach to Communication Scheduling on SoPC Architectures & \href{../works/WolinskiKG04.pdf}{Yes} & \cite{WolinskiKG04} & 2004 & DSD 2004 & 8 & 0 0 1 & 9 14 & \ref{b:WolinskiKG04} & n/a\\
\end{longtable}
}

\subsection{ECAI}

\index{ECAI}
{\scriptsize
\begin{longtable}{>{\raggedright\arraybackslash}p{3cm}>{\raggedright\arraybackslash}p{4.5cm}>{\raggedright\arraybackslash}p{6.0cm}rrrp{2.5cm}rp{1cm}p{1cm}rr}
\rowcolor{white}\caption{Papers in Conference Series ECAI (Total 18) (Total 18)}\\ \toprule
\rowcolor{white}\shortstack{Key\\Source} & Authors & Title (Colored by Open Access)& LC & Cite & Year & \shortstack{Conference\\/Journal\\/School} & Pages & \shortstack{Cites\\OC XR\\SC} & \shortstack{Refs\\OC\\XR} & b & c \\ \midrule\endhead
\bottomrule
\endfoot
Bit-Monnot23 \href{https://doi.org/10.3233/FAIA230278}{Bit-Monnot23} & \hyperref[auth:a392]{A. Bit-Monnot} & \cellcolor{gold!20}Enhancing Hybrid {CP-SAT} Search for Disjunctive Scheduling & \href{../works/Bit-Monnot23.pdf}{Yes} & \cite{Bit-Monnot23} & 2023 & ECAI 2023 & 8 & 0 0 0 & 0 0 & \ref{b:Bit-Monnot23} & \ref{c:Bit-Monnot23}\\
WangB20 \href{https://doi.org/10.3233/FAIA200114}{WangB20} & \hyperref[auth:a393]{R. Wang}, \hyperref[auth:a394]{N. Barnier} & Global Propagation of Transition Cost for Fixed Job Scheduling & \href{../works/WangB20.pdf}{Yes} & \cite{WangB20} & 2020 & ECAI 2020 & 8 & 0 0 0 & 0 0 & \ref{b:WangB20} & \ref{c:WangB20}\\
BridiLBBM16 \href{https://doi.org/10.3233/978-1-61499-672-9-1598}{BridiLBBM16} & \hyperref[auth:a227]{T. Bridi}, \hyperref[auth:a142]{M. Lombardi}, \hyperref[auth:a225]{A. Bartolini}, \hyperref[auth:a245]{L. Benini}, \hyperref[auth:a143]{M. Milano} & {DARDIS:} Distributed And Randomized DIspatching and Scheduling & \href{../works/BridiLBBM16.pdf}{Yes} & \cite{BridiLBBM16} & 2016 & ECAI 2016 & 2 & 0 0 0 & 0 0 & \ref{b:BridiLBBM16} & n/a\\
TranB12 \href{https://doi.org/10.3233/978-1-61499-098-7-774}{TranB12} & \hyperref[auth:a799]{T. T. Tran}, \hyperref[auth:a89]{J. C. Beck} & Logic-based Benders Decomposition for Alternative Resource Scheduling with Sequence Dependent Setups & \href{../works/TranB12.pdf}{Yes} & \cite{TranB12} & 2012 & ECAI 2012 & 6 & 0 0 30 & 0 0 & \ref{b:TranB12} & n/a\\
OddiRC10 \href{https://doi.org/10.3233/978-1-60750-606-5-967}{OddiRC10} & \hyperref[auth:a282]{A. Oddi}, \hyperref[auth:a1271]{R. Rasconi}, \hyperref[auth:a284]{A. Cesta} & Project Scheduling as a Disjunctive Temporal Problem & \href{../works/OddiRC10.pdf}{Yes} & \cite{OddiRC10} & 2010 & ECAI 2010 & 2 & 0 0 2 & 0 0 & \ref{b:OddiRC10} & n/a\\
Hunsberger08 \href{https://doi.org/10.3233/978-1-58603-891-5-553}{Hunsberger08} & \hyperref[auth:a1270]{L. Hunsberger} & A Practical Temporal Constraint Management System for Real-Time Applications & \href{../works/Hunsberger08.pdf}{Yes} & \cite{Hunsberger08} & 2008 & ECAI 2008 & 5 & 0 0 1 & 0 0 & \ref{b:Hunsberger08} & n/a\\
BeckW04 \href{}{BeckW04} & \hyperref[auth:a89]{J. C. Beck}, \hyperref[auth:a826]{N. Wilson} & Job Shop Scheduling with Probabilistic Durations & \href{../works/BeckW04.pdf}{Yes} & \cite{BeckW04} & 2004 & ECAI 2004 & 5 & 0 0 0 & 0 0 & \ref{b:BeckW04} & n/a\\
TsurutaS00 \href{}{TsurutaS00} & \hyperref[auth:a1267]{T. Tsuruta}, \hyperref[auth:a1268]{T. Shintani} & Scheduling Meetings Using Distributed Valued Constraint Satisfaction Algorithm & No & \cite{TsurutaS00} & 2000 & ECAI 2000 & 5 & 0 0 0 & 0 0 & No & n/a\\
WallaceF00 \href{}{WallaceF00} & \hyperref[auth:a1269]{R. J. Wallace}, \hyperref[auth:a273]{E. C. Freuder} & Dispatchability Conditions for Schedules with Consumable Resources & \href{../works/WallaceF00.pdf}{Yes} & \cite{WallaceF00} & 2000 & ECAI 2000 & 7 & 0 0 0 & 0 0 & \ref{b:WallaceF00} & n/a\\
SakkoutRW98 \href{}{SakkoutRW98} & \hyperref[auth:a166]{H. E. Sakkout}, \hyperref[auth:a1266]{T. Richards}, \hyperref[auth:a117]{M. G. Wallace} & Minimal Perturbance in Dynamic Scheduling & No & \cite{SakkoutRW98} & 1998 & ECAI 1998 & 5 & 0 0 0 & 0 0 & No & n/a\\
PapeB96 \href{}{PapeB96} & \hyperref[auth:a163]{C. L. Pape}, \hyperref[auth:a162]{P. Baptiste} & Constraint Propagation Techniques for Disjunctive Scheduling: The Preemptive Case & No & \cite{PapeB96} & 1996 & ECAI 1996 & 5 & 0 0 0 & 0 0 & No & n/a\\
Schaerf96 \href{}{Schaerf96} & \hyperref[auth:a1262]{A. Schaerf} & Scheduling Sport Tournaments using Constraint Logic Programming & No & \cite{Schaerf96} & 1996 & ECAI 1996 & 5 & 0 0 0 & 0 0 & No & n/a\\
StidsenKM96 \href{}{StidsenKM96} & \hyperref[auth:a1263]{T. R. Stidsen}, \hyperref[auth:a1264]{L. V. Kragelund}, \hyperref[auth:a1265]{O. Mateescu} & Jobshop Scheduling in a Shipyard & No & \cite{StidsenKM96} & 1996 & ECAI 1996 & 8 & 0 0 0 & 0 0 & No & n/a\\
NuijtenA94 \href{}{NuijtenA94} & \hyperref[auth:a656]{W. Nuijten}, \hyperref[auth:a777]{E. H. L. Aarts} & Constraint Satisfaction for Multiple Capacitated Job Shop Scheduling & \href{../works/NuijtenA94.pdf}{Yes} & \cite{NuijtenA94} & 1994 & ECAI 1994 & 5 & 0 0 0 & 0 0 & \ref{b:NuijtenA94} & n/a\\
NuijtenA94a \href{}{NuijtenA94a} & \hyperref[auth:a1257]{W. P. M. Nuijten}, \hyperref[auth:a777]{E. H. L. Aarts} & Constraint Satisfaction for Multiple Capacitated Job Shop Scheduling & No & \cite{NuijtenA94a} & 1994 & ECAI 1994 & 5 & 0 0 0 & 0 0 & No & n/a\\
Rodosek94 \href{}{Rodosek94} & \hyperref[auth:a297]{R. Rodosek} & Combining Constraint Network and Causal Theory to Solve Scheduling Problems from a {CSP} Perspective & No & \cite{Rodosek94} & 1994 & ECAI 1994 & 5 & 0 0 0 & 0 0 & No & n/a\\
YeGMH94 \href{}{YeGMH94} & \hyperref[auth:a1258]{P. Ye}, \hyperref[auth:a1259]{D. Glass}, \hyperref[auth:a1260]{M. F. McTear}, \hyperref[auth:a1261]{J. G. Hughes} & Job Cost and Constraint Relaxation for Scheduling Problem Solving in the {CLP} Paradigm & No & \cite{YeGMH94} & 1994 & ECAI 1994 & 5 & 0 0 0 & 0 0 & No & n/a\\
FoxS90 \href{}{FoxS90} & \hyperref[auth:a302]{M. S. Fox}, \hyperref[auth:a1044]{N. M. Sadeh} & Why is Scheduling Difficult? {A} {CSP} Perspective & \href{../works/FoxS90.pdf}{Yes} & \cite{FoxS90} & 1990 & ECAI 1990 & 14 & 0 0 0 & 0 0 & \ref{b:FoxS90} & n/a\\
\end{longtable}
}

\subsection{ECC}

\index{ECC}
{\scriptsize
\begin{longtable}{>{\raggedright\arraybackslash}p{3cm}>{\raggedright\arraybackslash}p{4.5cm}>{\raggedright\arraybackslash}p{6.0cm}rrrp{2.5cm}rp{1cm}p{1cm}rr}
\rowcolor{white}\caption{Papers in Conference Series ECC (Total 1) (Total 1)}\\ \toprule
\rowcolor{white}\shortstack{Key\\Source} & Authors & Title (Colored by Open Access)& LC & Cite & Year & \shortstack{Conference\\/Journal\\/School} & Pages & \shortstack{Cites\\OC XR\\SC} & \shortstack{Refs\\OC\\XR} & b & c \\ \midrule\endhead
\bottomrule
\endfoot
KorbaaYG99 \href{https://doi.org/10.23919/ECC.1999.7099947}{KorbaaYG99} & \hyperref[auth:a680]{O. Korbaa}, \hyperref[auth:a681]{P. Yim}, \hyperref[auth:a682]{J.-C. Gentina} & Solving transient scheduling problem for cyclic production using timed Petri nets and constraint programming & \href{../works/KorbaaYG99.pdf}{Yes} & \cite{KorbaaYG99} & 1999 & ECC 1999 & 8 & 1 1 1 & 0 0 & \ref{b:KorbaaYG99} & n/a\\
\end{longtable}
}

\subsection{ERCIM/CologNet}

\index{ERCIM/CologNet}
{\scriptsize
\begin{longtable}{>{\raggedright\arraybackslash}p{3cm}>{\raggedright\arraybackslash}p{4.5cm}>{\raggedright\arraybackslash}p{6.0cm}rrrp{2.5cm}rp{1cm}p{1cm}rr}
\rowcolor{white}\caption{Papers in Conference Series ERCIM/CologNet (Total 1) (Total 1)}\\ \toprule
\rowcolor{white}\shortstack{Key\\Source} & Authors & Title (Colored by Open Access)& LC & Cite & Year & \shortstack{Conference\\/Journal\\/School} & Pages & \shortstack{Cites\\OC XR\\SC} & \shortstack{Refs\\OC\\XR} & b & c \\ \midrule\endhead
\bottomrule
\endfoot
Bartak02a \href{https://doi.org/10.1007/3-540-36607-5_14}{Bartak02a} & \hyperref[auth:a152]{R. Bart{\'{a}}k} & \cellcolor{green!10}Visopt ShopFloor: Going Beyond Traditional Scheduling & \href{../works/Bartak02a.pdf}{Yes} & \cite{Bartak02a} & 2002 & ERCIM/CologNet 2002 & 15 & 1 1 1 & 9 22 & \ref{b:Bartak02a} & n/a\\
\end{longtable}
}

\subsection{ESCAPE}

\index{ESCAPE}
{\scriptsize
\begin{longtable}{>{\raggedright\arraybackslash}p{3cm}>{\raggedright\arraybackslash}p{4.5cm}>{\raggedright\arraybackslash}p{6.0cm}rrrp{2.5cm}rp{1cm}p{1cm}rr}
\rowcolor{white}\caption{Papers in Conference Series ESCAPE (Total 1) (Total 1)}\\ \toprule
\rowcolor{white}\shortstack{Key\\Source} & Authors & Title (Colored by Open Access)& LC & Cite & Year & \shortstack{Conference\\/Journal\\/School} & Pages & \shortstack{Cites\\OC XR\\SC} & \shortstack{Refs\\OC\\XR} & b & c \\ \midrule\endhead
\bottomrule
\endfoot
MagataoAN05 \href{https://www.sciencedirect.com/science/article/pii/S1570794605800136}{MagataoAN05} & \hyperref[auth:a1470]{L. Magatao}, \hyperref[auth:a1471]{L. Arruda}, \hyperref[auth:a1472]{F. Neves-Jr} & Using CLP and MILP for scheduling commodities in a pipeline & No & \cite{MagataoAN05} & 2005 & ESCAPE 2005 & 6 & 7 7 12 & 2 7 & No & n/a\\
\end{longtable}
}

\subsection{EUROCAST}

\index{EUROCAST}
{\scriptsize
\begin{longtable}{>{\raggedright\arraybackslash}p{3cm}>{\raggedright\arraybackslash}p{4.5cm}>{\raggedright\arraybackslash}p{6.0cm}rrrp{2.5cm}rp{1cm}p{1cm}rr}
\rowcolor{white}\caption{Papers in Conference Series EUROCAST (Total 1) (Total 1)}\\ \toprule
\rowcolor{white}\shortstack{Key\\Source} & Authors & Title (Colored by Open Access)& LC & Cite & Year & \shortstack{Conference\\/Journal\\/School} & Pages & \shortstack{Cites\\OC XR\\SC} & \shortstack{Refs\\OC\\XR} & b & c \\ \midrule\endhead
\bottomrule
\endfoot
FrohnerTR19 \href{https://doi.org/10.1007/978-3-030-45093-9_34}{FrohnerTR19} & \hyperref[auth:a537]{N. Frohner}, \hyperref[auth:a538]{S. Teuschl}, \hyperref[auth:a342]{G. R. Raidl} & Casual Employee Scheduling with Constraint Programming and Metaheuristics & \href{../works/FrohnerTR19.pdf}{Yes} & \cite{FrohnerTR19} & 2019 & EUROCAST 2019 & 9 & 0 0 0 & 6 7 & \ref{b:FrohnerTR19} & n/a\\
\end{longtable}
}

\subsection{EUROMICRO}

\index{EUROMICRO}
{\scriptsize
\begin{longtable}{>{\raggedright\arraybackslash}p{3cm}>{\raggedright\arraybackslash}p{4.5cm}>{\raggedright\arraybackslash}p{6.0cm}rrrp{2.5cm}rp{1cm}p{1cm}rr}
\rowcolor{white}\caption{Papers in Conference Series EUROMICRO (Total 1) (Total 1)}\\ \toprule
\rowcolor{white}\shortstack{Key\\Source} & Authors & Title (Colored by Open Access)& LC & Cite & Year & \shortstack{Conference\\/Journal\\/School} & Pages & \shortstack{Cites\\OC XR\\SC} & \shortstack{Refs\\OC\\XR} & b & c \\ \midrule\endhead
\bottomrule
\endfoot
GruianK98 \href{https://doi.org/10.1109/EURMIC.1998.711781}{GruianK98} & \hyperref[auth:a686]{F. Gruian}, \hyperref[auth:a660]{K. Kuchcinski} & Operation Binding and Scheduling for Low Power Using Constraint Logic Programming & \href{../works/GruianK98.pdf}{Yes} & \cite{GruianK98} & 1998 & EUROMICRO 1998 & 8 & 5 5 8 & 10 16 & \ref{b:GruianK98} & n/a\\
\end{longtable}
}

\subsection{FSKD}

\index{FSKD}
{\scriptsize
\begin{longtable}{>{\raggedright\arraybackslash}p{3cm}>{\raggedright\arraybackslash}p{4.5cm}>{\raggedright\arraybackslash}p{6.0cm}rrrp{2.5cm}rp{1cm}p{1cm}rr}
\rowcolor{white}\caption{Papers in Conference Series FSKD (Total 1) (Total 1)}\\ \toprule
\rowcolor{white}\shortstack{Key\\Source} & Authors & Title (Colored by Open Access)& LC & Cite & Year & \shortstack{Conference\\/Journal\\/School} & Pages & \shortstack{Cites\\OC XR\\SC} & \shortstack{Refs\\OC\\XR} & b & c \\ \midrule\endhead
\bottomrule
\endfoot
ZhouGL15 \href{https://doi.org/10.1109/FSKD.2015.7382064}{ZhouGL15} & \hyperref[auth:a599]{J. Zhou}, \hyperref[auth:a600]{Y. Guo}, \hyperref[auth:a601]{G. Li} & On complex hybrid flexible flowshop scheduling problems based on constraint programming & \href{../works/ZhouGL15.pdf}{Yes} & \cite{ZhouGL15} & 2015 & FSKD 2015 & 5 & 0 0 2 & 16 0 & \ref{b:ZhouGL15} & n/a\\
\end{longtable}
}

\subsection{FUZZ-IEEE}

\index{FUZZ-IEEE}
{\scriptsize
\begin{longtable}{>{\raggedright\arraybackslash}p{3cm}>{\raggedright\arraybackslash}p{4.5cm}>{\raggedright\arraybackslash}p{6.0cm}rrrp{2.5cm}rp{1cm}p{1cm}rr}
\rowcolor{white}\caption{Papers in Conference Series FUZZ-IEEE (Total 1) (Total 1)}\\ \toprule
\rowcolor{white}\shortstack{Key\\Source} & Authors & Title (Colored by Open Access)& LC & Cite & Year & \shortstack{Conference\\/Journal\\/School} & Pages & \shortstack{Cites\\OC XR\\SC} & \shortstack{Refs\\OC\\XR} & b & c \\ \midrule\endhead
\bottomrule
\endfoot
Tom19 \href{https://doi.org/10.1109/FUZZ-IEEE.2019.8859029}{Tom19} & \hyperref[auth:a539]{M. Tom} & Fuzzy Multi-Constraint Programming Model for Weekly Meals Scheduling & \href{../works/Tom19.pdf}{Yes} & \cite{Tom19} & 2019 & FUZZ-IEEE 2019 & 6 & 0 0 0 & 21 24 & \ref{b:Tom19} & n/a\\
\end{longtable}
}

\subsection{Fog-IoT}

\index{Fog-IoT}
{\scriptsize
\begin{longtable}{>{\raggedright\arraybackslash}p{3cm}>{\raggedright\arraybackslash}p{4.5cm}>{\raggedright\arraybackslash}p{6.0cm}rrrp{2.5cm}rp{1cm}p{1cm}rr}
\rowcolor{white}\caption{Papers in Conference Series Fog-IoT (Total 1) (Total 1)}\\ \toprule
\rowcolor{white}\shortstack{Key\\Source} & Authors & Title (Colored by Open Access)& LC & Cite & Year & \shortstack{Conference\\/Journal\\/School} & Pages & \shortstack{Cites\\OC XR\\SC} & \shortstack{Refs\\OC\\XR} & b & c \\ \midrule\endhead
\bottomrule
\endfoot
BarzegaranZP20 \href{https://doi.org/10.4230/OASIcs.Fog-IoT.2020.3}{BarzegaranZP20} & \hyperref[auth:a521]{M. Barzegaran}, \hyperref[auth:a522]{B. Zarrin}, \hyperref[auth:a523]{P. Pop} & Quality-Of-Control-Aware Scheduling of Communication in TSN-Based Fog Computing Platforms Using Constraint Programming & \href{../works/BarzegaranZP20.pdf}{Yes} & \cite{BarzegaranZP20} & 2020 & Fog-IoT 2020 & 9 & 0 0 0 & 0 0 & \ref{b:BarzegaranZP20} & \ref{c:BarzegaranZP20}\\
\end{longtable}
}

\subsection{GECCO}

\index{GECCO}
{\scriptsize
\begin{longtable}{>{\raggedright\arraybackslash}p{3cm}>{\raggedright\arraybackslash}p{4.5cm}>{\raggedright\arraybackslash}p{6.0cm}rrrp{2.5cm}rp{1cm}p{1cm}rr}
\rowcolor{white}\caption{Papers in Conference Series GECCO (Total 1) (Total 1)}\\ \toprule
\rowcolor{white}\shortstack{Key\\Source} & Authors & Title (Colored by Open Access)& LC & Cite & Year & \shortstack{Conference\\/Journal\\/School} & Pages & \shortstack{Cites\\OC XR\\SC} & \shortstack{Refs\\OC\\XR} & b & c \\ \midrule\endhead
\bottomrule
\endfoot
GroleazNS20a \href{https://doi.org/10.1145/3377930.3389818}{GroleazNS20a} & \hyperref[auth:a83]{L. Groleaz}, \hyperref[auth:a84]{S. N. Ndiaye}, \hyperref[auth:a85]{C. Solnon} & \cellcolor{green!10}{ACO} with automatic parameter selection for a scheduling problem with a group cumulative constraint & \href{../works/GroleazNS20a.pdf}{Yes} & \cite{GroleazNS20a} & 2020 & GECCO 2020 & 9 & 3 3 3 & 28 31 & \ref{b:GroleazNS20a} & \ref{c:GroleazNS20a}\\
\end{longtable}
}

\subsection{GOR}

\index{GOR}
{\scriptsize
\begin{longtable}{>{\raggedright\arraybackslash}p{3cm}>{\raggedright\arraybackslash}p{4.5cm}>{\raggedright\arraybackslash}p{6.0cm}rrrp{2.5cm}rp{1cm}p{1cm}rr}
\rowcolor{white}\caption{Papers in Conference Series GOR (Total 2) (Total 2)}\\ \toprule
\rowcolor{white}\shortstack{Key\\Source} & Authors & Title (Colored by Open Access)& LC & Cite & Year & \shortstack{Conference\\/Journal\\/School} & Pages & \shortstack{Cites\\OC XR\\SC} & \shortstack{Refs\\OC\\XR} & b & c \\ \midrule\endhead
\bottomrule
\endfoot
FriedrichFMRSST14 \href{https://doi.org/10.1007/978-3-319-28697-6_23}{FriedrichFMRSST14} & \hyperref[auth:a602]{G. Friedrich}, \hyperref[auth:a603]{M. Fr{\"{u}}hst{\"{u}}ck}, \hyperref[auth:a604]{V. Mersheeva}, \hyperref[auth:a605]{A. Ryabokon}, \hyperref[auth:a606]{M. Sander}, \hyperref[auth:a607]{A. Starzacher}, \hyperref[auth:a608]{E. Teppan} & Representing Production Scheduling with Constraint Answer Set Programming & No & \cite{FriedrichFMRSST14} & 2014 & GOR 2014 & 7 & 3 3 0 & 2 6 & No & n/a\\
Limtanyakul07 \href{https://doi.org/10.1007/978-3-540-77903-2_65}{Limtanyakul07} & \hyperref[auth:a144]{K. Limtanyakul} & Scheduling of Tests on Vehicle Prototypes Using Constraint and Integer Programming & \href{../works/Limtanyakul07.pdf}{Yes} & \cite{Limtanyakul07} & 2007 & GOR 2007 & 6 & 2 2 0 & 3 6 & \ref{b:Limtanyakul07} & n/a\\
\end{longtable}
}

\subsection{GreenCom}

\index{GreenCom}
{\scriptsize
\begin{longtable}{>{\raggedright\arraybackslash}p{3cm}>{\raggedright\arraybackslash}p{4.5cm}>{\raggedright\arraybackslash}p{6.0cm}rrrp{2.5cm}rp{1cm}p{1cm}rr}
\rowcolor{white}\caption{Papers in Conference Series GreenCom (Total 1) (Total 1)}\\ \toprule
\rowcolor{white}\shortstack{Key\\Source} & Authors & Title (Colored by Open Access)& LC & Cite & Year & \shortstack{Conference\\/Journal\\/School} & Pages & \shortstack{Cites\\OC XR\\SC} & \shortstack{Refs\\OC\\XR} & b & c \\ \midrule\endhead
\bottomrule
\endfoot
SunLYL10 \href{https://doi.org/10.1109/GreenCom-CPSCom.2010.111}{SunLYL10} & \hyperref[auth:a623]{Z. Sun}, \hyperref[auth:a624]{H. Li}, \hyperref[auth:a625]{M. Yao}, \hyperref[auth:a626]{N. Li} & Scheduling Optimization Techniques for FlexRay Using Constraint-Programming & \href{../works/SunLYL10.pdf}{Yes} & \cite{SunLYL10} & 2010 & GreenCom 2010 & 6 & 4 4 3 & 8 17 & \ref{b:SunLYL10} & n/a\\
\end{longtable}
}

\subsection{HM}

\index{HM}
{\scriptsize
\begin{longtable}{>{\raggedright\arraybackslash}p{3cm}>{\raggedright\arraybackslash}p{4.5cm}>{\raggedright\arraybackslash}p{6.0cm}rrrp{2.5cm}rp{1cm}p{1cm}rr}
\rowcolor{white}\caption{Papers in Conference Series HM (Total 1) (Total 1)}\\ \toprule
\rowcolor{white}\shortstack{Key\\Source} & Authors & Title (Colored by Open Access)& LC & Cite & Year & \shortstack{Conference\\/Journal\\/School} & Pages & \shortstack{Cites\\OC XR\\SC} & \shortstack{Refs\\OC\\XR} & b & c \\ \midrule\endhead
\bottomrule
\endfoot
ThiruvadyBME09 \href{https://doi.org/10.1007/978-3-642-04918-7_3}{ThiruvadyBME09} & \hyperref[auth:a396]{D. R. Thiruvady}, \hyperref[auth:a636]{C. Blum}, \hyperref[auth:a637]{B. Meyer}, \hyperref[auth:a469]{A. T. Ernst} & Hybridizing Beam-ACO with Constraint Programming for Single Machine Job Scheduling & \href{../works/ThiruvadyBME09.pdf}{Yes} & \cite{ThiruvadyBME09} & 2009 & HM 2009 & 15 & 13 12 15 & 12 19 & \ref{b:ThiruvadyBME09} & n/a\\
\end{longtable}
}

\subsection{ICAART}

\index{ICAART}
{\scriptsize
\begin{longtable}{>{\raggedright\arraybackslash}p{3cm}>{\raggedright\arraybackslash}p{4.5cm}>{\raggedright\arraybackslash}p{6.0cm}rrrp{2.5cm}rp{1cm}p{1cm}rr}
\rowcolor{white}\caption{Papers in Conference Series ICAART (Total 5) (Total 5)}\\ \toprule
\rowcolor{white}\shortstack{Key\\Source} & Authors & Title (Colored by Open Access)& LC & Cite & Year & \shortstack{Conference\\/Journal\\/School} & Pages & \shortstack{Cites\\OC XR\\SC} & \shortstack{Refs\\OC\\XR} & b & c \\ \midrule\endhead
\bottomrule
\endfoot
SvancaraB22 \href{https://doi.org/10.5220/0010869700003116}{SvancaraB22} & \hyperref[auth:a778]{J. Svancara}, \hyperref[auth:a152]{R. Bart{\'{a}}k} & \cellcolor{gold!20}Tackling Train Routing via Multi-agent Pathfinding and Constraint-based Scheduling & \href{../works/SvancaraB22.pdf}{Yes} & \cite{SvancaraB22} & 2022 & ICAART 2022 & 8 & 0 0 0 & 0 0 & \ref{b:SvancaraB22} & n/a\\
Teppan22 \href{https://doi.org/10.5220/0010849900003116}{Teppan22} & \hyperref[auth:a94]{E. C. Teppan} & \cellcolor{gold!20}Types of Flexible Job Shop Scheduling: {A} Constraint Programming Experiment & \href{../works/Teppan22.pdf}{Yes} & \cite{Teppan22} & 2022 & ICAART 2022 & 8 & 0 1 4 & 0 0 & \ref{b:Teppan22} & \ref{c:Teppan22}\\
TouatBT22 \href{http://dx.doi.org/10.5220/0010800700003116}{TouatBT22} & \hyperref[auth:a457]{M. Touat}, \hyperref[auth:a458]{B. Benhamou}, \hyperref[auth:a459]{F. B.-S. Tayeb} & \cellcolor{gold!20}A Constraint Programming Model for the Scheduling Problem with Flexible Maintenance under Human Resource Constraints & \href{../works/TouatBT22.pdf}{Yes} & \cite{TouatBT22} & 2022 & ICAART 2022 & 8 & 0 0 0 & 0 0 & \ref{b:TouatBT22} & \ref{c:TouatBT22}\\
LiuLH19a \href{http://dx.doi.org/10.5220/0007252300290039}{LiuLH19a} & \hyperref[auth:a1391]{K. Liu}, \hyperref[auth:a1392]{S. Loeffler}, \hyperref[auth:a1393]{P. Hofstedt} & \cellcolor{gold!20}Solving the Social Golfers Problems by Constraint Programming in Sequential and Parallel & \href{../works/LiuLH19a.pdf}{Yes} & \cite{LiuLH19a} & 2019 & ICAART 2019 & 11 & 3 3 4 & 0 0 & \ref{b:LiuLH19a} & n/a\\
BartakV15 \href{http://dx.doi.org/10.5220/0005215701190130 }{BartakV15} & \hyperref[auth:a152]{R. Bart{\'{a}}k}, \hyperref[auth:a311]{M. Vlk} & \cellcolor{gold!20}Reactive Recovery from Machine Breakdown in Production Scheduling with Temporal Distance and Resource Constraints & \href{../works/BartakV15.pdf}{Yes} & \cite{BartakV15} & 2015 & ICAART 2015 & 12 & 0 0 1 & 0 0 & \ref{b:BartakV15} & n/a\\
\end{longtable}
}

\subsection{ICAPS}

\index{ICAPS}
{\scriptsize
\begin{longtable}{>{\raggedright\arraybackslash}p{3cm}>{\raggedright\arraybackslash}p{4.5cm}>{\raggedright\arraybackslash}p{6.0cm}rrrp{2.5cm}rp{1cm}p{1cm}rr}
\rowcolor{white}\caption{Papers in Conference Series ICAPS (Total 18) (Total 18)}\\ \toprule
\rowcolor{white}\shortstack{Key\\Source} & Authors & Title (Colored by Open Access)& LC & Cite & Year & \shortstack{Conference\\/Journal\\/School} & Pages & \shortstack{Cites\\OC XR\\SC} & \shortstack{Refs\\OC\\XR} & b & c \\ \midrule\endhead
\bottomrule
\endfoot
TasselGS23 \href{https://doi.org/10.1609/icaps.v33i1.27243}{TasselGS23} & \hyperref[auth:a58]{P. Tassel}, \hyperref[auth:a61]{M. Gebser}, \hyperref[auth:a423]{K. Schekotihin} & \cellcolor{gold!20}An End-to-End Reinforcement Learning Approach for Job-Shop Scheduling Problems Based on Constraint Programming & \href{../works/TasselGS23.pdf}{Yes} & \cite{TasselGS23} & 2023 & ICAPS 2023 & 9 & 0 1 2 & 0 0 & \ref{b:TasselGS23} & \ref{c:TasselGS23}\\
ZhangBB22 \href{https://ojs.aaai.org/index.php/ICAPS/article/view/19826}{ZhangBB22} & \hyperref[auth:a797]{J. Zhang}, \hyperref[auth:a798]{G. L. Bianco}, \hyperref[auth:a89]{J. C. Beck} & \cellcolor{gold!20}Solving Job-Shop Scheduling Problems with QUBO-Based Specialized Hardware & \href{../works/ZhangBB22.pdf}{Yes} & \cite{ZhangBB22} & 2022 & ICAPS 2022 & 9 & 1 2 2 & 0 0 & \ref{b:ZhangBB22} & n/a\\
KletzanderM20 \href{https://ojs.aaai.org/index.php/ICAPS/article/view/6688}{KletzanderM20} & \hyperref[auth:a78]{L. Kletzander}, \hyperref[auth:a45]{N. Musliu} & Solving Large Real-Life Bus Driver Scheduling Problems with Complex Break Constraints & \href{../works/KletzanderM20.pdf}{Yes} & \cite{KletzanderM20} & 2020 & ICAPS 2020 & 10 & 0 0 0 & 0 0 & \ref{b:KletzanderM20} & n/a\\
SenderovichBB19 \href{https://ojs.aaai.org/index.php/ICAPS/article/view/3504}{SenderovichBB19} & \hyperref[auth:a1372]{A. Senderovich}, \hyperref[auth:a203]{K. E. C. Booth}, \hyperref[auth:a89]{J. C. Beck} & Learning Scheduling Models from Event Data & \href{../works/SenderovichBB19.pdf}{Yes} & \cite{SenderovichBB19} & 2019 & ICAPS 2019 & 9 & 0 0 0 & 0 0 & \ref{b:SenderovichBB19} & \ref{c:SenderovichBB19}\\
RiahiNS018 \href{https://aaai.org/ocs/index.php/ICAPS/ICAPS18/paper/view/17755}{RiahiNS018} & \hyperref[auth:a388]{V. Riahi}, \hyperref[auth:a389]{M. A. H. Newton}, \hyperref[auth:a390]{K. Su}, \hyperref[auth:a391]{A. Sattar} & Local Search for Flowshops with Setup Times and Blocking Constraints & \href{../works/RiahiNS018.pdf}{Yes} & \cite{RiahiNS018} & 2018 & ICAPS 2018 & 9 & 4 4 0 & 0 0 & \ref{b:RiahiNS018} & n/a\\
FrankDT16 \href{http://www.aaai.org/ocs/index.php/ICAPS/ICAPS16/paper/view/13072}{FrankDT16} & \hyperref[auth:a379]{J. Frank}, \hyperref[auth:a809]{M. Do}, \hyperref[auth:a799]{T. T. Tran} & Scheduling Ocean Color Observations for a GEO-Stationary Satellite & \href{../works/FrankDT16.pdf}{Yes} & \cite{FrankDT16} & 2016 & ICAPS 2016 & 9 & 4 5 0 & 0 0 & \ref{b:FrankDT16} & n/a\\
LipovetzkyBPS14 \href{http://www.aaai.org/ocs/index.php/ICAPS/ICAPS14/paper/view/7942}{LipovetzkyBPS14} & \hyperref[auth:a323]{N. Lipovetzky}, \hyperref[auth:a322]{C. N. Burt}, \hyperref[auth:a324]{A. R. Pearce}, \hyperref[auth:a125]{P. J. Stuckey} & Planning for Mining Operations with Time and Resource Constraints & \href{../works/LipovetzkyBPS14.pdf}{Yes} & \cite{LipovetzkyBPS14} & 2014 & ICAPS 2014 & 9 & 5 5 0 & 0 0 & \ref{b:LipovetzkyBPS14} & n/a\\
BonfiettiLM13 \href{http://www.aaai.org/ocs/index.php/ICAPS/ICAPS13/paper/view/6050}{BonfiettiLM13} & \hyperref[auth:a198]{A. Bonfietti}, \hyperref[auth:a142]{M. Lombardi}, \hyperref[auth:a143]{M. Milano} & De-Cycling Cyclic Scheduling Problems & \href{../works/BonfiettiLM13.pdf}{Yes} & \cite{BonfiettiLM13} & 2013 & ICAPS 2013 & 5 & 1 1 0 & 0 0 & \ref{b:BonfiettiLM13} & n/a\\
LombardiM13 \href{http://www.aaai.org/ocs/index.php/ICAPS/ICAPS13/paper/view/6052}{LombardiM13} & \hyperref[auth:a142]{M. Lombardi}, \hyperref[auth:a143]{M. Milano} & A Min-Flow Algorithm for Minimal Critical Set Detection in Resource Constrained Project Scheduling & \href{../works/LombardiM13.pdf}{Yes} & \cite{LombardiM13} & 2013 & ICAPS 2013 & 2 & 3 0 0 & 13 0 & \ref{b:LombardiM13} & n/a\\
MalapertCGJLR13 \href{http://www.aaai.org/ocs/index.php/ICAPS/ICAPS13/paper/view/6016}{MalapertCGJLR13} & \hyperref[auth:a82]{A. Malapert}, \hyperref[auth:a999]{H. Cambazard}, \hyperref[auth:a293]{C. Gu{\'{e}}ret}, \hyperref[auth:a247]{N. Jussien}, \hyperref[auth:a645]{A. Langevin}, \hyperref[auth:a326]{L.-M. Rousseau} & An Optimal Constraint Programming Approach to the Open-Shop Problem & \href{../works/MalapertCGJLR13.pdf}{Yes} & \cite{MalapertCGJLR13} & 2013 & ICAPS 2013 & 2 & 0 0 0 & 0 0 & \ref{b:MalapertCGJLR13} & n/a\\
TranTDB13 \href{http://www.aaai.org/ocs/index.php/ICAPS/ICAPS13/paper/view/6005}{TranTDB13} & \hyperref[auth:a799]{T. T. Tran}, \hyperref[auth:a818]{D. Terekhov}, \hyperref[auth:a803]{D. G. Down}, \hyperref[auth:a89]{J. C. Beck} & Hybrid Queueing Theory and Scheduling Models for Dynamic Environments with Sequence-Dependent Setup Times & \href{../works/TranTDB13.pdf}{Yes} & \cite{TranTDB13} & 2013 & ICAPS 2013 & 9 & 2 2 0 & 0 0 & \ref{b:TranTDB13} & n/a\\
BajestaniB11 \href{http://aaai.org/ocs/index.php/ICAPS/ICAPS11/paper/view/2680}{BajestaniB11} & \hyperref[auth:a817]{M. A. Bajestani}, \hyperref[auth:a89]{J. C. Beck} & Scheduling an Aircraft Repair Shop & \href{../works/BajestaniB11.pdf}{Yes} & \cite{BajestaniB11} & 2011 & ICAPS 2011 & 8 & 2 2 0 & 0 0 & \ref{b:BajestaniB11} & n/a\\
MonetteDH09 \href{http://aaai.org/ocs/index.php/ICAPS/ICAPS09/paper/view/712}{MonetteDH09} & \hyperref[auth:a149]{J.-N. Monette}, \hyperref[auth:a151]{Y. Deville}, \hyperref[auth:a148]{P. V. Hentenryck} & Just-In-Time Scheduling with Constraint Programming & \href{../works/MonetteDH09.pdf}{Yes} & \cite{MonetteDH09} & 2009 & ICAPS 2009 & 8 & 9 10 0 & 0 0 & \ref{b:MonetteDH09} & n/a\\
PoderB08 \href{http://www.aaai.org/Library/ICAPS/2008/icaps08-033.php}{PoderB08} & \hyperref[auth:a358]{E. Poder}, \hyperref[auth:a128]{N. Beldiceanu} & Filtering for a Continuous Multi-Resources cumulative Constraint with Resource Consumption and Production & \href{../works/PoderB08.pdf}{Yes} & \cite{PoderB08} & 2008 & ICAPS 2008 & 8 & 0 0 0 & 0 0 & \ref{b:PoderB08} & n/a\\
Beck06 \href{http://www.aaai.org/Library/ICAPS/2006/icaps06-028.php}{Beck06} & \hyperref[auth:a89]{J. C. Beck} & An Empirical Study of Multi-Point Constructive Search for Constraint-Based Scheduling & \href{../works/Beck06.pdf}{Yes} & \cite{Beck06} & 2006 & ICAPS 2006 & 10 & 0 0 0 & 0 0 & \ref{b:Beck06} & n/a\\
GodardLN05 \href{http://www.aaai.org/Library/ICAPS/2005/icaps05-009.php}{GodardLN05} & \hyperref[auth:a774]{D. Godard}, \hyperref[auth:a118]{P. Laborie}, \hyperref[auth:a656]{W. Nuijten} & Randomized Large Neighborhood Search for Cumulative Scheduling & \href{../works/GodardLN05.pdf}{Yes} & \cite{GodardLN05} & 2005 & ICAPS 2005 & 9 & 0 0 0 & 0 0 & \ref{b:GodardLN05} & n/a\\
BeckPS03 \href{http://www.aaai.org/Library/ICAPS/2003/icaps03-027.php}{BeckPS03} & \hyperref[auth:a89]{J. C. Beck}, \hyperref[auth:a827]{P. Prosser}, \hyperref[auth:a828]{E. Selensky} & Vehicle Routing and Job Shop Scheduling: What's the Difference? & \href{../works/BeckPS03.pdf}{Yes} & \cite{BeckPS03} & 2003 & ICAPS 2003 & 10 & 0 0 0 & 0 0 & \ref{b:BeckPS03} & n/a\\
FrankK03 \href{http://www.aaai.org/Library/ICAPS/2003/icaps03-023.php}{FrankK03} & \hyperref[auth:a379]{J. Frank}, \hyperref[auth:a380]{E. K{\"{u}}rkl{\"{u}}} & SOFIA's Choice: Scheduling Observations for an Airborne Observatory & \href{../works/FrankK03.pdf}{Yes} & \cite{FrankK03} & 2003 & ICAPS 2003 & 10 & 0 0 0 & 0 0 & \ref{b:FrankK03} & n/a\\
\end{longtable}
}

\subsection{ICCL}

\index{ICCL}
{\scriptsize
\begin{longtable}{>{\raggedright\arraybackslash}p{3cm}>{\raggedright\arraybackslash}p{4.5cm}>{\raggedright\arraybackslash}p{6.0cm}rrrp{2.5cm}rp{1cm}p{1cm}rr}
\rowcolor{white}\caption{Papers in Conference Series ICCL (Total 1) (Total 1)}\\ \toprule
\rowcolor{white}\shortstack{Key\\Source} & Authors & Title (Colored by Open Access)& LC & Cite & Year & \shortstack{Conference\\/Journal\\/School} & Pages & \shortstack{Cites\\OC XR\\SC} & \shortstack{Refs\\OC\\XR} & b & c \\ \midrule\endhead
\bottomrule
\endfoot
BenderWS21 \href{https://doi.org/10.1007/978-3-030-87672-2_37}{BenderWS21} & \hyperref[auth:a493]{T. Bender}, \hyperref[auth:a494]{D. Wittwer}, \hyperref[auth:a495]{T. Schmidt} & Applying Constraint Programming to the Multi-mode Scheduling Problem in Harvest Logistics & \href{../works/BenderWS21.pdf}{Yes} & \cite{BenderWS21} & 2021 & ICCL 2021 & 16 & 1 1 2 & 16 20 & \ref{b:BenderWS21} & \ref{c:BenderWS21}\\
\end{longtable}
}

\subsection{ICMSAO}

\index{ICMSAO}
{\scriptsize
\begin{longtable}{>{\raggedright\arraybackslash}p{3cm}>{\raggedright\arraybackslash}p{4.5cm}>{\raggedright\arraybackslash}p{6.0cm}rrrp{2.5cm}rp{1cm}p{1cm}rr}
\rowcolor{white}\caption{Papers in Conference Series ICMSAO (Total 1) (Total 1)}\\ \toprule
\rowcolor{white}\shortstack{Key\\Source} & Authors & Title (Colored by Open Access)& LC & Cite & Year & \shortstack{Conference\\/Journal\\/School} & Pages & \shortstack{Cites\\OC XR\\SC} & \shortstack{Refs\\OC\\XR} & b & c \\ \midrule\endhead
\bottomrule
\endfoot
HamdiL13 \href{http://dx.doi.org/10.1109/icmsao.2013.6552689}{HamdiL13} & \hyperref[auth:a1232]{I. Hamdi}, \hyperref[auth:a1233]{T. Loukil} & Logic-based Benders decomposition to solve the permutation flowshop scheduling problem with time lags & \href{../works/HamdiL13.pdf}{Yes} & \cite{HamdiL13} & 2013 & ICMSAO 2013 & 7 & 2 2 4 & 11 13 & \ref{b:HamdiL13} & n/a\\
\end{longtable}
}

\subsection{ICNC}

\index{ICNC}
{\scriptsize
\begin{longtable}{>{\raggedright\arraybackslash}p{3cm}>{\raggedright\arraybackslash}p{4.5cm}>{\raggedright\arraybackslash}p{6.0cm}rrrp{2.5cm}rp{1cm}p{1cm}rr}
\rowcolor{white}\caption{Papers in Conference Series ICNC (Total 1) (Total 1)}\\ \toprule
\rowcolor{white}\shortstack{Key\\Source} & Authors & Title (Colored by Open Access)& LC & Cite & Year & \shortstack{Conference\\/Journal\\/School} & Pages & \shortstack{Cites\\OC XR\\SC} & \shortstack{Refs\\OC\\XR} & b & c \\ \midrule\endhead
\bottomrule
\endfoot
MakMS10 \href{https://doi.org/10.1109/ICNC.2010.5583494}{MakMS10} & \hyperref[auth:a627]{K.-L. Mak}, \hyperref[auth:a628]{J. Ma}, \hyperref[auth:a629]{W. Su} & A constraint programming approach for production scheduling of multi-period virtual cellular manufacturing systems & \href{../works/MakMS10.pdf}{Yes} & \cite{MakMS10} & 2010 & ICNC 2010 & 5 & 1 0 1 & 3 5 & \ref{b:MakMS10} & n/a\\
\end{longtable}
}

\subsection{ICNSC}

\index{ICNSC}
{\scriptsize
\begin{longtable}{>{\raggedright\arraybackslash}p{3cm}>{\raggedright\arraybackslash}p{4.5cm}>{\raggedright\arraybackslash}p{6.0cm}rrrp{2.5cm}rp{1cm}p{1cm}rr}
\rowcolor{white}\caption{Papers in Conference Series ICNSC (Total 2) (Total 2)}\\ \toprule
\rowcolor{white}\shortstack{Key\\Source} & Authors & Title (Colored by Open Access)& LC & Cite & Year & \shortstack{Conference\\/Journal\\/School} & Pages & \shortstack{Cites\\OC XR\\SC} & \shortstack{Refs\\OC\\XR} & b & c \\ \midrule\endhead
\bottomrule
\endfoot
LiFJZLL22 \href{https://doi.org/10.1109/ICNSC55942.2022.10004158}{LiFJZLL22} & \hyperref[auth:a460]{X. Li}, \hyperref[auth:a461]{J. Fu}, \hyperref[auth:a462]{Z. Jia}, \hyperref[auth:a463]{Z. Zhao}, \hyperref[auth:a464]{S. Li}, \hyperref[auth:a465]{S. Liu} & Constraint Programming for a Novel Integrated Optimization of Blocking Job Shop Scheduling and Variable-Speed Transfer Robot Assignment & \href{../works/LiFJZLL22.pdf}{Yes} & \cite{LiFJZLL22} & 2022 & ICNSC 2022 & 6 & 0 1 1 & 31 34 & \ref{b:LiFJZLL22} & \ref{c:LiFJZLL22}\\
ZhangJZL22 \href{https://doi.org/10.1109/ICNSC55942.2022.10004154}{ZhangJZL22} & \hyperref[auth:a466]{H. Zhang}, \hyperref[auth:a467]{Y. Ji}, \hyperref[auth:a463]{Z. Zhao}, \hyperref[auth:a465]{S. Liu} & Constraint Programming for Modeling and Solving a Hybrid Flow Shop Scheduling Problem & \href{../works/ZhangJZL22.pdf}{Yes} & \cite{ZhangJZL22} & 2022 & ICNSC 2022 & 6 & 0 1 1 & 21 24 & \ref{b:ZhangJZL22} & \ref{c:ZhangJZL22}\\
\end{longtable}
}

\subsection{ICORES}

\index{ICORES}
{\scriptsize
\begin{longtable}{>{\raggedright\arraybackslash}p{3cm}>{\raggedright\arraybackslash}p{4.5cm}>{\raggedright\arraybackslash}p{6.0cm}rrrp{2.5cm}rp{1cm}p{1cm}rr}
\rowcolor{white}\caption{Papers in Conference Series ICORES (Total 2) (Total 2)}\\ \toprule
\rowcolor{white}\shortstack{Key\\Source} & Authors & Title (Colored by Open Access)& LC & Cite & Year & \shortstack{Conference\\/Journal\\/School} & Pages & \shortstack{Cites\\OC XR\\SC} & \shortstack{Refs\\OC\\XR} & b & c \\ \midrule\endhead
\bottomrule
\endfoot
BonninMNE24 \href{https://doi.org/10.5220/0012310200003639}{BonninMNE24} & \hyperref[auth:a1008]{C. Bonnin}, \hyperref[auth:a82]{A. Malapert}, \hyperref[auth:a81]{M. Nattaf}, \hyperref[auth:a1009]{M.-L. Espinouse} & \cellcolor{gold!20}Toward a Global Constraint for Minimizing the Flowtime & \href{../works/BonninMNE24.pdf}{Yes} & \cite{BonninMNE24} & 2024 & ICORES 2024 & 12 & 0 0 0 & 0 0 & \ref{b:BonninMNE24} & n/a\\
ArtiguesHQT21 \href{https://doi.org/10.5220/0010190101290136}{ArtiguesHQT21} & \hyperref[auth:a6]{C. Artigues}, \hyperref[auth:a1]{E. Hebrard}, \hyperref[auth:a789]{A. Quilliot}, \hyperref[auth:a790]{H. Toussaint} & \cellcolor{gold!20}Multi-Mode {RCPSP} with Safety Margin Maximization: Models and Algorithms & \href{../works/ArtiguesHQT21.pdf}{Yes} & \cite{ArtiguesHQT21} & 2021 & ICORES 2021 & 8 & 0 0 0 & 0 0 & \ref{b:ArtiguesHQT21} & n/a\\
\end{longtable}
}

\subsection{ICPADS}

\index{ICPADS}
{\scriptsize
\begin{longtable}{>{\raggedright\arraybackslash}p{3cm}>{\raggedright\arraybackslash}p{4.5cm}>{\raggedright\arraybackslash}p{6.0cm}rrrp{2.5cm}rp{1cm}p{1cm}rr}
\rowcolor{white}\caption{Papers in Conference Series ICPADS (Total 1) (Total 1)}\\ \toprule
\rowcolor{white}\shortstack{Key\\Source} & Authors & Title (Colored by Open Access)& LC & Cite & Year & \shortstack{Conference\\/Journal\\/School} & Pages & \shortstack{Cites\\OC XR\\SC} & \shortstack{Refs\\OC\\XR} & b & c \\ \midrule\endhead
\bottomrule
\endfoot
Madi-WambaLOBM17 \href{https://doi.org/10.1109/ICPADS.2017.00089}{Madi-WambaLOBM17} & \hyperref[auth:a320]{G. Madi-Wamba}, \hyperref[auth:a714]{Y. Li}, \hyperref[auth:a715]{A.-C. Orgerie}, \hyperref[auth:a128]{N. Beldiceanu}, \hyperref[auth:a716]{J.-M. Menaud} & \cellcolor{green!10}Green Energy Aware Scheduling Problem in Virtualized Datacenters & \href{../works/Madi-WambaLOBM17.pdf}{Yes} & \cite{Madi-WambaLOBM17} & 2017 & ICPADS 2017 & 8 & 1 1 1 & 8 18 & \ref{b:Madi-WambaLOBM17} & n/a\\
\end{longtable}
}

\subsection{ICPC}

\index{ICPC}
{\scriptsize
\begin{longtable}{>{\raggedright\arraybackslash}p{3cm}>{\raggedright\arraybackslash}p{4.5cm}>{\raggedright\arraybackslash}p{6.0cm}rrrp{2.5cm}rp{1cm}p{1cm}rr}
\rowcolor{white}\caption{Papers in Conference Series ICPC (Total 1) (Total 1)}\\ \toprule
\rowcolor{white}\shortstack{Key\\Source} & Authors & Title (Colored by Open Access)& LC & Cite & Year & \shortstack{Conference\\/Journal\\/School} & Pages & \shortstack{Cites\\OC XR\\SC} & \shortstack{Refs\\OC\\XR} & b & c \\ \midrule\endhead
\bottomrule
\endfoot
ZibranR11 \href{https://doi.org/10.1109/ICPC.2011.45}{ZibranR11} & \hyperref[auth:a619]{M. F. Zibran}, \hyperref[auth:a620]{C. K. Roy} & Conflict-Aware Optimal Scheduling of Code Clone Refactoring: {A} Constraint Programming Approach & \href{../works/ZibranR11.pdf}{Yes} & \cite{ZibranR11} & 2011 & ICPC 2011 & 4 & 17 16 19 & 18 24 & \ref{b:ZibranR11} & n/a\\
\end{longtable}
}

\subsection{ICRA}

\index{ICRA}
{\scriptsize
\begin{longtable}{>{\raggedright\arraybackslash}p{3cm}>{\raggedright\arraybackslash}p{4.5cm}>{\raggedright\arraybackslash}p{6.0cm}rrrp{2.5cm}rp{1cm}p{1cm}rr}
\rowcolor{white}\caption{Papers in Conference Series ICRA (Total 4) (Total 4)}\\ \toprule
\rowcolor{white}\shortstack{Key\\Source} & Authors & Title (Colored by Open Access)& LC & Cite & Year & \shortstack{Conference\\/Journal\\/School} & Pages & \shortstack{Cites\\OC XR\\SC} & \shortstack{Refs\\OC\\XR} & b & c \\ \midrule\endhead
\bottomrule
\endfoot
BehrensLM19 \href{https://doi.org/10.1109/ICRA.2019.8794022}{BehrensLM19} & \hyperref[auth:a540]{J. K. Behrens}, \hyperref[auth:a541]{R. Lange}, \hyperref[auth:a542]{M. Mansouri} & \cellcolor{green!10}A Constraint Programming Approach to Simultaneous Task Allocation and Motion Scheduling for Industrial Dual-Arm Manipulation Tasks & \href{../works/BehrensLM19.pdf}{Yes} & \cite{BehrensLM19} & 2019 & ICRA 2019 & 7 & 12 17 27 & 18 27 & \ref{b:BehrensLM19} & \ref{c:BehrensLM19}\\
LouieVNB14 \href{https://doi.org/10.1109/ICRA.2014.6907637}{LouieVNB14} & \hyperref[auth:a819]{W.-Y. G. Louie}, \hyperref[auth:a804]{T. S. Vaquero}, \hyperref[auth:a204]{G. Nejat}, \hyperref[auth:a89]{J. C. Beck} & An autonomous assistive robot for planning, scheduling and facilitating multi-user activities & \href{../works/LouieVNB14.pdf}{Yes} & \cite{LouieVNB14} & 2014 & ICRA 2014 & 7 & 16 16 28 & 9 21 & \ref{b:LouieVNB14} & n/a\\
QuirogaZH05 \href{https://doi.org/10.1109/ROBOT.2005.1570686}{QuirogaZH05} & \hyperref[auth:a622]{O. Quiroga}, \hyperref[auth:a621]{L. J. Zeballos}, \hyperref[auth:a588]{G. P. Henning} & A Constraint Programming Approach to Tool Allocation and Resource Scheduling in {FMS} & \href{../works/QuirogaZH05.pdf}{Yes} & \cite{QuirogaZH05} & 2005 & ICRA 2005 & 6 & 2 1 2 & 7 11 & \ref{b:QuirogaZH05} & n/a\\
BaptisteLV92 \href{https://doi.org/10.1109/ROBOT.1992.220195}{BaptisteLV92} & \hyperref[auth:a693]{P. Baptiste}, \hyperref[auth:a694]{B. Legeard}, \hyperref[auth:a692]{C. Varnier} & Hoist scheduling problem: an approach based on constraint logic programming & \href{../works/BaptisteLV92.pdf}{Yes} & \cite{BaptisteLV92} & 1992 & ICRA 1992 & 6 & 13 11 0 & 6 14 & \ref{b:BaptisteLV92} & n/a\\
\end{longtable}
}

\subsection{ICROMA}

\index{ICROMA}
{\scriptsize
\begin{longtable}{>{\raggedright\arraybackslash}p{3cm}>{\raggedright\arraybackslash}p{4.5cm}>{\raggedright\arraybackslash}p{6.0cm}rrrp{2.5cm}rp{1cm}p{1cm}rr}
\rowcolor{white}\caption{Papers in Conference Series ICROMA (Total 2) (Total 2)}\\ \toprule
\rowcolor{white}\shortstack{Key\\Source} & Authors & Title (Colored by Open Access)& LC & Cite & Year & \shortstack{Conference\\/Journal\\/School} & Pages & \shortstack{Cites\\OC XR\\SC} & \shortstack{Refs\\OC\\XR} & b & c \\ \midrule\endhead
\bottomrule
\endfoot
RodriguezS09 \href{}{RodriguezS09} & \hyperref[auth:a781]{J. Rodriguez}, \hyperref[auth:a1018]{S. Sobieraj} & A study of an incremental texture-based heuristic for the train routing and scheduling problem & \href{../works/RodriguezS09.pdf}{Yes} & \cite{RodriguezS09} & 2009 & ICROMA 2009 & 14 & 0 0 0 & 0 0 & \ref{b:RodriguezS09} & n/a\\
Rodriguez07b \href{}{Rodriguez07b} & \hyperref[auth:a781]{J. Rodriguez} & A study of the use of state resources in a constraint-based model for routing and scheduling trains & \href{../works/Rodriguez07b.pdf}{Yes} & \cite{Rodriguez07b} & 2007 & ICROMA 2007 & 14 & 0 0 0 & 0 0 & \ref{b:Rodriguez07b} & n/a\\
\end{longtable}
}

\subsection{ICTAI}

\index{ICTAI}
{\scriptsize
\begin{longtable}{>{\raggedright\arraybackslash}p{3cm}>{\raggedright\arraybackslash}p{4.5cm}>{\raggedright\arraybackslash}p{6.0cm}rrrp{2.5cm}rp{1cm}p{1cm}rr}
\rowcolor{white}\caption{Papers in Conference Series ICTAI (Total 3) (Total 3)}\\ \toprule
\rowcolor{white}\shortstack{Key\\Source} & Authors & Title (Colored by Open Access)& LC & Cite & Year & \shortstack{Conference\\/Journal\\/School} & Pages & \shortstack{Cites\\OC XR\\SC} & \shortstack{Refs\\OC\\XR} & b & c \\ \midrule\endhead
\bottomrule
\endfoot
PerezGSL23 \href{https://doi.org/10.1109/ICTAI59109.2023.00108}{PerezGSL23} & \hyperref[auth:a425]{G. Perez}, \hyperref[auth:a426]{G. Glorian}, \hyperref[auth:a427]{W. Suijlen}, \hyperref[auth:a428]{A. Lallouet} & A Constraint Programming Model for Scheduling the Unloading of Trains in Ports & \href{../works/PerezGSL23.pdf}{Yes} & \cite{PerezGSL23} & 2023 & ICTAI 2023 & 7 & 0 0 0 & 0 19 & \ref{b:PerezGSL23} & \ref{c:PerezGSL23}\\
WangB23 \href{https://doi.org/10.1109/ICTAI59109.2023.00062}{WangB23} & \hyperref[auth:a393]{R. Wang}, \hyperref[auth:a394]{N. Barnier} & Dynamic All-Different and Maximal Cliques Constraints for Fixed Job Scheduling & \href{../works/WangB23.pdf}{Yes} & \cite{WangB23} & 2023 & ICTAI 2023 & 8 & 0 0 0 & 0 19 & \ref{b:WangB23} & \ref{c:WangB23}\\
AntunesABD18 \href{https://doi.org/10.1109/ICTAI.2018.00027}{AntunesABD18} & \hyperref[auth:a877]{M. Antunes}, \hyperref[auth:a878]{V. Armant}, \hyperref[auth:a217]{K. N. Brown}, \hyperref[auth:a879]{D. A. Desmond}, \hyperref[auth:a880]{G. Escamocher}, \hyperref[auth:a881]{A.-M. George}, \hyperref[auth:a181]{D. Grimes}, \hyperref[auth:a882]{M. O'Keeffe}, \hyperref[auth:a883]{Y. Lin}, \hyperref[auth:a16]{B. O'Sullivan}, \hyperref[auth:a135]{C. {\"{O}}zt{\"{u}}rk}, \hyperref[auth:a884]{L. Quesada}, \hyperref[auth:a129]{M. Siala}, \hyperref[auth:a17]{H. Simonis}, \hyperref[auth:a826]{N. Wilson} & Assigning and Scheduling Service Visits in a Mixed Urban/Rural Setting & \href{../works/AntunesABD18.pdf}{Yes} & \cite{AntunesABD18} & 2018 & ICTAI 2018 & 8 & 1 1 3 & 24 29 & \ref{b:AntunesABD18} & n/a\\
\end{longtable}
}

\subsection{IDC}

\index{IDC}
{\scriptsize
\begin{longtable}{>{\raggedright\arraybackslash}p{3cm}>{\raggedright\arraybackslash}p{4.5cm}>{\raggedright\arraybackslash}p{6.0cm}rrrp{2.5cm}rp{1cm}p{1cm}rr}
\rowcolor{white}\caption{Papers in Conference Series IDC (Total 1) (Total 1)}\\ \toprule
\rowcolor{white}\shortstack{Key\\Source} & Authors & Title (Colored by Open Access)& LC & Cite & Year & \shortstack{Conference\\/Journal\\/School} & Pages & \shortstack{Cites\\OC XR\\SC} & \shortstack{Refs\\OC\\XR} & b & c \\ \midrule\endhead
\bottomrule
\endfoot
BadicaBIL19 \href{https://doi.org/10.1007/978-3-030-32258-8_17}{BadicaBIL19} & \hyperref[auth:a497]{A. Badica}, \hyperref[auth:a498]{C. Badica}, \hyperref[auth:a499]{M. Ivanovic}, \hyperref[auth:a543]{D. Logofatu} & Exploring the Space of Block Structured Scheduling Processes Using Constraint Logic Programming & \href{../works/BadicaBIL19.pdf}{Yes} & \cite{BadicaBIL19} & 2019 & IDC 2019 & 11 & 2 2 3 & 6 9 & \ref{b:BadicaBIL19} & \ref{c:BadicaBIL19}\\
\end{longtable}
}

\subsection{IEEE International Conference on Automation and Logistics}

\index{IEEE International Conference on Automation and Logistics}
{\scriptsize
\begin{longtable}{>{\raggedright\arraybackslash}p{3cm}>{\raggedright\arraybackslash}p{4.5cm}>{\raggedright\arraybackslash}p{6.0cm}rrrp{2.5cm}rp{1cm}p{1cm}rr}
\rowcolor{white}\caption{Papers in Conference Series IEEE International Conference on Automation and Logistics (Total 1) (Total 1)}\\ \toprule
\rowcolor{white}\shortstack{Key\\Source} & Authors & Title (Colored by Open Access)& LC & Cite & Year & \shortstack{Conference\\/Journal\\/School} & Pages & \shortstack{Cites\\OC XR\\SC} & \shortstack{Refs\\OC\\XR} & b & c \\ \midrule\endhead
\bottomrule
\endfoot
RenT09 \href{http://dx.doi.org/10.1109/ical.2009.5262795}{RenT09} & \hyperref[auth:a1250]{H. Ren}, \hyperref[auth:a1197]{L. Tang} & An improved hybrid MILP/CP algorithm framework for the job-shop scheduling & \href{../works/RenT09.pdf}{Yes} & \cite{RenT09} & 2009 & IEEE International Conference on Automation and Logistics 2009 & 5 & 2 3 4 & 12 14 & \ref{b:RenT09} & n/a\\
\end{longtable}
}

\subsection{IESM}

\index{IESM}
{\scriptsize
\begin{longtable}{>{\raggedright\arraybackslash}p{3cm}>{\raggedright\arraybackslash}p{4.5cm}>{\raggedright\arraybackslash}p{6.0cm}rrrp{2.5cm}rp{1cm}p{1cm}rr}
\rowcolor{white}\caption{Papers in Conference Series IESM (Total 1) (Total 1)}\\ \toprule
\rowcolor{white}\shortstack{Key\\Source} & Authors & Title (Colored by Open Access)& LC & Cite & Year & \shortstack{Conference\\/Journal\\/School} & Pages & \shortstack{Cites\\OC XR\\SC} & \shortstack{Refs\\OC\\XR} & b & c \\ \midrule\endhead
\bottomrule
\endfoot
MeskensDHG11 \href{}{MeskensDHG11} & \hyperref[auth:a597]{N. Meskens}, \hyperref[auth:a598]{D. Duvivier}, \hyperref[auth:a1374]{A. Hanset}, \hyperref[auth:a1375]{D. Gossart} & Multi-objective Constraint programming for Scheduling Operating Theatres & \href{../works/MeskensDHG11.pdf}{Yes} & \cite{MeskensDHG11} & 2011 & IESM 2011 & 10 & 0 0 0 & 0 0 & \ref{b:MeskensDHG11} & n/a\\
\end{longtable}
}

\subsection{IJCAI}

\index{IJCAI}
{\scriptsize
\begin{longtable}{>{\raggedright\arraybackslash}p{3cm}>{\raggedright\arraybackslash}p{4.5cm}>{\raggedright\arraybackslash}p{6.0cm}rrrp{2.5cm}rp{1cm}p{1cm}rr}
\rowcolor{white}\caption{Papers in Conference Series IJCAI (Total 29) (Total 29)}\\ \toprule
\rowcolor{white}\shortstack{Key\\Source} & Authors & Title (Colored by Open Access)& LC & Cite & Year & \shortstack{Conference\\/Journal\\/School} & Pages & \shortstack{Cites\\OC XR\\SC} & \shortstack{Refs\\OC\\XR} & b & c \\ \midrule\endhead
\bottomrule
\endfoot
BofillCGGPSV23 \href{https://doi.org/10.24963/ijcai.2023/768}{BofillCGGPSV23} & \hyperref[auth:a228]{M. Bofill}, \hyperref[auth:a1449]{J. Coll}, \hyperref[auth:a230]{M. Garcia}, \hyperref[auth:a1453]{J. Gir{\'{a}}ldez-Cru}, \hyperref[auth:a8]{G. Pesant}, \hyperref[auth:a232]{J. Suy}, \hyperref[auth:a233]{M. Villaret} & Constraint Solving Approaches to the Business-to-Business Meeting Scheduling Problem (Extended Abstract) & \href{../works/BofillCGGPSV23.pdf}{Yes} & \cite{BofillCGGPSV23} & 2023 & IJCAI 2023 & 2 & 0 0 0 & 0 0 & \ref{b:BofillCGGPSV23} & n/a\\
IklassovMR023 \href{https://doi.org/10.24963/ijcai.2023/594}{IklassovMR023} & \hyperref[auth:a1454]{Z. Iklassov}, \hyperref[auth:a1455]{D. Medvedev}, \hyperref[auth:a1456]{Ruben Solozabal Ochoa de Retana}, \hyperref[auth:a1457]{M. Tak{\'{a}}c} & On the Study of Curriculum Learning for Inferring Dispatching Policies on the Job Shop Scheduling & \href{../works/IklassovMR023.pdf}{Yes} & \cite{IklassovMR023} & 2023 & IJCAI 2023 & 9 & 0 0 0 & 0 0 & \ref{b:IklassovMR023} & \ref{c:IklassovMR023}\\
HebrardALLCMR22 \href{https://doi.org/10.24963/ijcai.2022/643}{HebrardALLCMR22} & \hyperref[auth:a1]{E. Hebrard}, \hyperref[auth:a6]{C. Artigues}, \hyperref[auth:a3]{P. Lopez}, \hyperref[auth:a785]{A. Lusson}, \hyperref[auth:a786]{S. A. Chien}, \hyperref[auth:a787]{A. Maillard}, \hyperref[auth:a788]{G. R. Rabideau} & An Efficient Approach to Data Transfer Scheduling for Long Range Space Exploration & \href{../works/HebrardALLCMR22.pdf}{Yes} & \cite{HebrardALLCMR22} & 2022 & IJCAI 2022 & 7 & 0 0 0 & 0 0 & \ref{b:HebrardALLCMR22} & n/a\\
Tassel22 \href{https://doi.org/10.24963/ijcai.2022/841}{Tassel22} & \hyperref[auth:a58]{P. Tassel} & Adaptive Artificial Intelligence Scheduling Methods for Large-Scale, Stochastic, Industrial Applications & \href{../works/Tassel22.pdf}{Yes} & \cite{Tassel22} & 2022 & IJCAI 2022 & 2 & 0 0 0 & 0 0 & \ref{b:Tassel22} & \ref{c:Tassel22}\\
BhatnagarKL19 \href{https://doi.org/10.24963/ijcai.2019/803}{BhatnagarKL19} & \hyperref[auth:a1452]{S. Bhatnagar}, \hyperref[auth:a1360]{A. Kumar}, \hyperref[auth:a364]{H. C. Lau} & Decision Making for Improving Maritime Traffic Safety Using Constraint Programming & \href{../works/BhatnagarKL19.pdf}{Yes} & \cite{BhatnagarKL19} & 2019 & IJCAI 2019 & 7 & 1 1 0 & 0 0 & \ref{b:BhatnagarKL19} & n/a\\
PachecoPR19 \href{https://doi.org/10.24963/ijcai.2019/161}{PachecoPR19} & \hyperref[auth:a1451]{A. Pacheco}, \hyperref[auth:a21]{C. Pralet}, \hyperref[auth:a22]{S. Roussel} & Constraint-Based Scheduling with Complex Setup Operations: An Iterative Two-Layer Approach & \href{../works/PachecoPR19.pdf}{Yes} & \cite{PachecoPR19} & 2019 & IJCAI 2019 & 7 & 1 1 0 & 0 0 & \ref{b:PachecoPR19} & n/a\\
BofillCSV17a \href{https://doi.org/10.24963/ijcai.2017/78}{BofillCSV17a} & \hyperref[auth:a228]{M. Bofill}, \hyperref[auth:a1449]{J. Coll}, \hyperref[auth:a232]{J. Suy}, \hyperref[auth:a233]{M. Villaret} & Compact MDDs for Pseudo-Boolean Constraints with At-Most-One Relations in Resource-Constrained Scheduling Problems & \href{../works/BofillCSV17a.pdf}{Yes} & \cite{BofillCSV17a} & 2017 & IJCAI 2017 & 8 & 6 7 0 & 0 0 & \ref{b:BofillCSV17a} & n/a\\
ErkingerM17 \href{https://doi.org/10.24963/ijcai.2017/86}{ErkingerM17} & \hyperref[auth:a1450]{C. Erkinger}, \hyperref[auth:a45]{N. Musliu} & Personnel Scheduling as Satisfiability Modulo Theories & \href{../works/ErkingerM17.pdf}{Yes} & \cite{ErkingerM17} & 2017 & IJCAI 2017 & 8 & 4 4 0 & 0 0 & \ref{b:ErkingerM17} & n/a\\
TranVNB17a \href{https://doi.org/10.24963/ijcai.2017/726}{TranVNB17a} & \hyperref[auth:a799]{T. T. Tran}, \hyperref[auth:a804]{T. S. Vaquero}, \hyperref[auth:a204]{G. Nejat}, \hyperref[auth:a89]{J. C. Beck} & Robots in Retirement Homes: Applying Off-the-Shelf Planning and Scheduling to a Team of Assistive Robots (Extended Abstract) & \href{../works/TranVNB17a.pdf}{Yes} & \cite{TranVNB17a} & 2017 & IJCAI 2017 & 5 & 1 1 0 & 0 0 & \ref{b:TranVNB17a} & n/a\\
CatusseCBL16 \href{http://www.ijcai.org/Abstract/16/434}{CatusseCBL16} & \hyperref[auth:a998]{N. Catusse}, \hyperref[auth:a999]{H. Cambazard}, \hyperref[auth:a1000]{N. Brauner}, \hyperref[auth:a979]{P. Lemaire}, \hyperref[auth:a1001]{B. Penz}, \hyperref[auth:a1002]{A.-M. Lagrange}, \hyperref[auth:a1003]{P. Rubini} & A Branch-and-Price Algorithm for Scheduling Observations on a Telescope & \href{../works/CatusseCBL16.pdf}{Yes} & \cite{CatusseCBL16} & 2016 & IJCAI 2016 & 7 & 0 0 0 & 0 0 & \ref{b:CatusseCBL16} & n/a\\
GingrasQ16 \href{http://www.ijcai.org/Abstract/16/440}{GingrasQ16} & \hyperref[auth:a313]{V. Gingras}, \hyperref[auth:a37]{C.-G. Quimper} & Generalizing the Edge-Finder Rule for the Cumulative Constraint & \href{../works/GingrasQ16.pdf}{Yes} & \cite{GingrasQ16} & 2016 & IJCAI 2016 & 7 & 0 0 0 & 0 0 & \ref{b:GingrasQ16} & n/a\\
Maillard15 \href{http://ijcai.org/Abstract/15/637}{Maillard15} & \hyperref[auth:a787]{A. Maillard} & Flexible Scheduling for an Agile Earth-Observing Satelllite & \href{../works/Maillard15.pdf}{Yes} & \cite{Maillard15} & 2015 & IJCAI 2015 & 2 & 0 0 0 & 0 0 & \ref{b:Maillard15} & n/a\\
ChuGNSW13 \href{http://www.aaai.org/ocs/index.php/IJCAI/IJCAI13/paper/view/6878}{ChuGNSW13} & \hyperref[auth:a343]{G. Chu}, \hyperref[auth:a793]{S. Gaspers}, \hyperref[auth:a794]{N. Narodytska}, \hyperref[auth:a124]{A. Schutt}, \hyperref[auth:a276]{T. Walsh} & On the Complexity of Global Scheduling Constraints under Structural Restrictions & \href{../works/ChuGNSW13.pdf}{Yes} & \cite{ChuGNSW13} & 2013 & IJCAI 2013 & 7 & 0 0 0 & 0 0 & \ref{b:ChuGNSW13} & n/a\\
OddiRCS11 \href{https://doi.org/10.5591/978-1-57735-516-8/IJCAI11-332}{OddiRCS11} & \hyperref[auth:a282]{A. Oddi}, \hyperref[auth:a1271]{R. Rasconi}, \hyperref[auth:a284]{A. Cesta}, \hyperref[auth:a298]{S. F. Smith} & Iterative Flattening Search for the Flexible Job Shop Scheduling Problem & \href{../works/OddiRCS11.pdf}{Yes} & \cite{OddiRCS11} & 2011 & IJCAI 2011 & 6 & 0 0 0 & 0 0 & \ref{b:OddiRCS11} & n/a\\
PacinoH11 \href{https://doi.org/10.5591/978-1-57735-516-8/IJCAI11-333}{PacinoH11} & \hyperref[auth:a1448]{D. Pacino}, \hyperref[auth:a148]{P. V. Hentenryck} & Large Neighborhood Search and Adaptive Randomized Decompositions for Flexible Jobshop Scheduling & \href{../works/PacinoH11.pdf}{Yes} & \cite{PacinoH11} & 2011 & IJCAI 2011 & 6 & 0 0 0 & 0 0 & \ref{b:PacinoH11} & n/a\\
BidotVLB07 \href{http://ijcai.org/Proceedings/07/Papers/007.pdf}{BidotVLB07} & \hyperref[auth:a824]{J. Bidot}, \hyperref[auth:a825]{T. Vidal}, \hyperref[auth:a118]{P. Laborie}, \hyperref[auth:a89]{J. C. Beck} & A General Framework for Scheduling in a Stochastic Environment & \href{../works/BidotVLB07.pdf}{Yes} & \cite{BidotVLB07} & 2007 & IJCAI 2007 & 6 & 0 0 0 & 0 0 & \ref{b:BidotVLB07} & n/a\\
KusterJF07 \href{http://ijcai.org/Proceedings/07/Papers/316.pdf}{KusterJF07} & \hyperref[auth:a1446]{J. Kuster}, \hyperref[auth:a1447]{D. Jannach}, \hyperref[auth:a602]{G. Friedrich} & Handling Alternative Activities in Resource-Constrained Project Scheduling Problems & \href{../works/KusterJF07.pdf}{Yes} & \cite{KusterJF07} & 2007 & IJCAI 2007 & 6 & 0 0 0 & 0 0 & \ref{b:KusterJF07} & n/a\\
SultanikMR07 \href{http://ijcai.org/Proceedings/07/Papers/247.pdf}{SultanikMR07} & \hyperref[auth:a1443]{E. Sultanik}, \hyperref[auth:a1444]{P. J. Modi}, \hyperref[auth:a1445]{W. C. Regli} & On Modeling Multiagent Task Scheduling as a Distributed Constraint Optimization Problem & \href{../works/SultanikMR07.pdf}{Yes} & \cite{SultanikMR07} & 2007 & IJCAI 2007 & 6 & 0 0 0 & 0 0 & \ref{b:SultanikMR07} & n/a\\
BeckW05 \href{http://ijcai.org/Proceedings/05/Papers/0748.pdf}{BeckW05} & \hyperref[auth:a89]{J. C. Beck}, \hyperref[auth:a826]{N. Wilson} & Proactive Algorithms for Scheduling with Probabilistic Durations & \href{../works/BeckW05.pdf}{Yes} & \cite{BeckW05} & 2005 & IJCAI 2005 & 6 & 0 0 0 & 0 0 & \ref{b:BeckW05} & n/a\\
BeniniBGM05a \href{http://ijcai.org/Proceedings/05/Papers/post-0368.pdf}{BeniniBGM05a} & \hyperref[auth:a245]{L. Benini}, \hyperref[auth:a375]{D. Bertozzi}, \hyperref[auth:a376]{A. Guerri}, \hyperref[auth:a143]{M. Milano} & Allocation and Scheduling for MPSoCs via decomposition and no-good generation & \href{../works/BeniniBGM05a.pdf}{Yes} & \cite{BeniniBGM05a} & 2005 & IJCAI 2005 & 2 & 0 0 0 & 0 0 & \ref{b:BeniniBGM05a} & n/a\\
Laborie05 \href{http://ijcai.org/Proceedings/05/Papers/0571.pdf}{Laborie05} & \hyperref[auth:a118]{P. Laborie} & Complete MCS-Based Search: Application to Resource Constrained Project Scheduling & \href{../works/Laborie05.pdf}{Yes} & \cite{Laborie05} & 2005 & IJCAI 2005 & 6 & 0 0 0 & 0 0 & \ref{b:Laborie05} & n/a\\
Musliu05 \href{http://ijcai.org/Proceedings/05/Papers/post-0448.pdf}{Musliu05} & \hyperref[auth:a45]{N. Musliu} & Combination of Local Search Strategies for Rotating Workforce Scheduling Problem & \href{../works/Musliu05.pdf}{Yes} & \cite{Musliu05} & 2005 & IJCAI 2005 & 2 & 0 0 0 & 0 0 & \ref{b:Musliu05} & n/a\\
CestaOF99 \href{http://ijcai.org/Proceedings/99-2/Papers/051.pdf}{CestaOF99} & \hyperref[auth:a284]{A. Cesta}, \hyperref[auth:a282]{A. Oddi}, \hyperref[auth:a298]{S. F. Smith} & An Iterative Sampling Procedure for Resource Constrained Project Scheduling with Time Windows & \href{../works/CestaOF99.pdf}{Yes} & \cite{CestaOF99} & 1999 & IJCAI 1999 & 12 & 0 0 0 & 0 0 & \ref{b:CestaOF99} & n/a\\
DraperJCJ99 \href{http://ijcai.org/Proceedings/99-2/Papers/050.pdf}{DraperJCJ99} & \hyperref[auth:a1440]{D. Draper}, \hyperref[auth:a1042]{A. K. J{\'{o}}nsson}, \hyperref[auth:a1441]{D. P. Clements}, \hyperref[auth:a1442]{D. Joslin} & Cyclic Scheduling & \href{../works/DraperJCJ99.pdf}{Yes} & \cite{DraperJCJ99} & 1999 & IJCAI 1999 & 6 & 0 0 0 & 0 0 & \ref{b:DraperJCJ99} & n/a\\
Schaerf97 \href{http://ijcai.org/Proceedings/97-2/Papers/067.pdf}{Schaerf97} & \hyperref[auth:a1262]{A. Schaerf} & Combining Local Search and Look-Ahead for Scheduling and Constraint Satisfaction Problems & \href{../works/Schaerf97.pdf}{Yes} & \cite{Schaerf97} & 1997 & IJCAI 1997 & 6 & 0 0 0 & 0 0 & \ref{b:Schaerf97} & n/a\\
BaptisteP95 \href{http://ijcai.org/Proceedings/95-1/Papers/079.pdf}{BaptisteP95} & \hyperref[auth:a162]{P. Baptiste}, \hyperref[auth:a163]{C. L. Pape} & A Theoretical and Experimental Comparison of Constraint Propagation Techniques for Disjunctive Scheduling & \href{../works/BaptisteP95.pdf}{Yes} & \cite{BaptisteP95} & 1995 & IJCAI 1995 & 7 & 0 0 0 & 0 0 & \ref{b:BaptisteP95} & n/a\\
FeldmanG89 \href{http://ijcai.org/Proceedings/89-2/Papers/026.pdf}{FeldmanG89} & \hyperref[auth:a1436]{R. Feldman}, \hyperref[auth:a1437]{M. C. Golumbic} & Constraint Satisfiability Algorithms for Interactive Student Scheduling & \href{../works/FeldmanG89.pdf}{Yes} & \cite{FeldmanG89} & 1989 & IJCAI 1989 & 7 & 0 0 0 & 0 0 & \ref{b:FeldmanG89} & n/a\\
KengY89 \href{http://ijcai.org/Proceedings/89-2/Papers/024.pdf}{KengY89} & \hyperref[auth:a1438]{N. Keng}, \hyperref[auth:a1439]{D. Y. Y. Yun} & A Planning/Scheduling Methodology for the Constrained Resource Problem & \href{../works/KengY89.pdf}{Yes} & \cite{KengY89} & 1989 & IJCAI 1989 & 6 & 0 0 0 & 0 0 & \ref{b:KengY89} & n/a\\
Prosser89 \href{http://ijcai.org/Proceedings/89-2/Papers/025.pdf}{Prosser89} & \hyperref[auth:a827]{P. Prosser} & A Reactive Scheduling Agent & \href{../works/Prosser89.pdf}{Yes} & \cite{Prosser89} & 1989 & IJCAI 1989 & 6 & 0 0 0 & 0 0 & \ref{b:Prosser89} & n/a\\
\end{longtable}
}

\subsection{ILPS}

\index{ILPS}
{\scriptsize
\begin{longtable}{>{\raggedright\arraybackslash}p{3cm}>{\raggedright\arraybackslash}p{4.5cm}>{\raggedright\arraybackslash}p{6.0cm}rrrp{2.5cm}rp{1cm}p{1cm}rr}
\rowcolor{white}\caption{Papers in Conference Series ILPS (Total 1) (Total 1)}\\ \toprule
\rowcolor{white}\shortstack{Key\\Source} & Authors & Title (Colored by Open Access)& LC & Cite & Year & \shortstack{Conference\\/Journal\\/School} & Pages & \shortstack{Cites\\OC XR\\SC} & \shortstack{Refs\\OC\\XR} & b & c \\ \midrule\endhead
\bottomrule
\endfoot
JourdanFRD94 \href{}{JourdanFRD94} & \hyperref[auth:a697]{J. Jourdan}, \hyperref[auth:a698]{F. Fages}, \hyperref[auth:a699]{D. Rozzonelli}, \hyperref[auth:a700]{A. Demeure} & Data Alignment and Task Scheduling On Parallel Machines Using Concurrent Constraint Model-based Programming & No & \cite{JourdanFRD94} & 1994 & ILPS 1994 & 1 & 0 0 0 & 0 0 & No & n/a\\
\end{longtable}
}

\subsection{INAP}

\index{INAP}
{\scriptsize
\begin{longtable}{>{\raggedright\arraybackslash}p{3cm}>{\raggedright\arraybackslash}p{4.5cm}>{\raggedright\arraybackslash}p{6.0cm}rrrp{2.5cm}rp{1cm}p{1cm}rr}
\rowcolor{white}\caption{Papers in Conference Series INAP (Total 4) (Total 4)}\\ \toprule
\rowcolor{white}\shortstack{Key\\Source} & Authors & Title (Colored by Open Access)& LC & Cite & Year & \shortstack{Conference\\/Journal\\/School} & Pages & \shortstack{Cites\\OC XR\\SC} & \shortstack{Refs\\OC\\XR} & b & c \\ \midrule\endhead
\bottomrule
\endfoot
Wolf09 \href{http://dx.doi.org/10.1007/978-3-642-00675-3_2}{Wolf09} & \hyperref[auth:a51]{A. Wolf}, \hyperref[auth:a710]{G. Schrader} & Linear Weighted-Task-Sum – Scheduling Prioritized Tasks on a Single Resource & \href{../works/Wolf09.pdf}{Yes} & \cite{Wolf09} & 2009 & INAP 2009 & 17 & 1 1 1 & 12 15 & \ref{b:Wolf09} & n/a\\
Geske05 \href{https://doi.org/10.1007/11963578_10}{Geske05} & \hyperref[auth:a657]{U. Geske} & Railway Scheduling with Declarative Constraint Programming & \href{../works/Geske05.pdf}{Yes} & \cite{Geske05} & 2005 & INAP 2005 & 18 & 2 2 6 & 3 17 & \ref{b:Geske05} & n/a\\
SchuttWS05 \href{https://doi.org/10.1007/11963578_6}{SchuttWS05} & \hyperref[auth:a124]{A. Schutt}, \hyperref[auth:a51]{A. Wolf}, \hyperref[auth:a710]{G. Schrader} & Not-First and Not-Last Detection for Cumulative Scheduling in \emph{O}(\emph{n}\({}^{\mbox{3}}\)log\emph{n}) & \href{../works/SchuttWS05.pdf}{Yes} & \cite{SchuttWS05} & 2005 & INAP 2005 & 15 & 6 6 9 & 4 11 & \ref{b:SchuttWS05} & n/a\\
WolfS05 \href{https://doi.org/10.1007/11963578_8}{WolfS05} & \hyperref[auth:a51]{A. Wolf}, \hyperref[auth:a710]{G. Schrader} & \emph{O}(\emph{n} log\emph{n}) Overload Checking for the Cumulative Constraint and Its Application & \href{../works/WolfS05.pdf}{Yes} & \cite{WolfS05} & 2005 & INAP 2005 & 14 & 6 7 12 & 6 10 & \ref{b:WolfS05} & n/a\\
\end{longtable}
}

\subsection{IPDPS}

\index{IPDPS}
{\scriptsize
\begin{longtable}{>{\raggedright\arraybackslash}p{3cm}>{\raggedright\arraybackslash}p{4.5cm}>{\raggedright\arraybackslash}p{6.0cm}rrrp{2.5cm}rp{1cm}p{1cm}rr}
\rowcolor{white}\caption{Papers in Conference Series IPDPS (Total 1) (Total 1)}\\ \toprule
\rowcolor{white}\shortstack{Key\\Source} & Authors & Title (Colored by Open Access)& LC & Cite & Year & \shortstack{Conference\\/Journal\\/School} & Pages & \shortstack{Cites\\OC XR\\SC} & \shortstack{Refs\\OC\\XR} & b & c \\ \midrule\endhead
\bottomrule
\endfoot
JungblutK22 \href{https://doi.org/10.1109/IPDPSW55747.2022.00025}{JungblutK22} & \hyperref[auth:a740]{P. Jungblut}, \hyperref[auth:a741]{D. Kranzlm{\"{u}}ller} & Optimal Schedules for High-Level Programming Environments on FPGAs with Constraint Programming & \href{../works/JungblutK22.pdf}{Yes} & \cite{JungblutK22} & 2022 & IPDPS 2022 & 4 & 0 1 1 & 0 0 & \ref{b:JungblutK22} & \ref{c:JungblutK22}\\
\end{longtable}
}

\subsection{ISCA}

\index{ISCA}
{\scriptsize
\begin{longtable}{>{\raggedright\arraybackslash}p{3cm}>{\raggedright\arraybackslash}p{4.5cm}>{\raggedright\arraybackslash}p{6.0cm}rrrp{2.5cm}rp{1cm}p{1cm}rr}
\rowcolor{white}\caption{Papers in Conference Series ISCA (Total 1) (Total 1)}\\ \toprule
\rowcolor{white}\shortstack{Key\\Source} & Authors & Title (Colored by Open Access)& LC & Cite & Year & \shortstack{Conference\\/Journal\\/School} & Pages & \shortstack{Cites\\OC XR\\SC} & \shortstack{Refs\\OC\\XR} & b & c \\ \midrule\endhead
\bottomrule
\endfoot
VillaverdeP04 \href{}{VillaverdeP04} & \hyperref[auth:a658]{K. Villaverde}, \hyperref[auth:a33]{E. Pontelli} & An Investigation of Scheduling in Distributed Constraint Logic Programming & No & \cite{VillaverdeP04} & 2004 & ISCA 2004 & 6 & 0 0 0 & 0 0 & No & n/a\\
\end{longtable}
}

\subsection{ISMIS}

\index{ISMIS}
{\scriptsize
\begin{longtable}{>{\raggedright\arraybackslash}p{3cm}>{\raggedright\arraybackslash}p{4.5cm}>{\raggedright\arraybackslash}p{6.0cm}rrrp{2.5cm}rp{1cm}p{1cm}rr}
\rowcolor{white}\caption{Papers in Conference Series ISMIS (Total 1) (Total 1)}\\ \toprule
\rowcolor{white}\shortstack{Key\\Source} & Authors & Title (Colored by Open Access)& LC & Cite & Year & \shortstack{Conference\\/Journal\\/School} & Pages & \shortstack{Cites\\OC XR\\SC} & \shortstack{Refs\\OC\\XR} & b & c \\ \midrule\endhead
\bottomrule
\endfoot
BrusoniCLMMT96 \href{https://doi.org/10.1007/3-540-61286-6_157}{BrusoniCLMMT96} & \hyperref[auth:a722]{V. Brusoni}, \hyperref[auth:a723]{L. Console}, \hyperref[auth:a720]{E. Lamma}, \hyperref[auth:a721]{P. Mello}, \hyperref[auth:a143]{M. Milano}, \hyperref[auth:a724]{P. Terenziani} & Resource-Based vs. Task-Based Approaches for Scheduling Problems & \href{../works/BrusoniCLMMT96.pdf}{Yes} & \cite{BrusoniCLMMT96} & 1996 & ISMIS 1996 & 10 & 1 1 4 & 9 26 & \ref{b:BrusoniCLMMT96} & n/a\\
\end{longtable}
}

\subsection{KES}

\index{KES}
{\scriptsize
\begin{longtable}{>{\raggedright\arraybackslash}p{3cm}>{\raggedright\arraybackslash}p{4.5cm}>{\raggedright\arraybackslash}p{6.0cm}rrrp{2.5cm}rp{1cm}p{1cm}rr}
\rowcolor{white}\caption{Papers in Conference Series KES (Total 1) (Total 1)}\\ \toprule
\rowcolor{white}\shortstack{Key\\Source} & Authors & Title (Colored by Open Access)& LC & Cite & Year & \shortstack{Conference\\/Journal\\/School} & Pages & \shortstack{Cites\\OC XR\\SC} & \shortstack{Refs\\OC\\XR} & b & c \\ \midrule\endhead
\bottomrule
\endfoot
ValleMGT03 \href{https://doi.org/10.1007/978-3-540-45226-3_180}{ValleMGT03} & \hyperref[auth:a666]{C. D. Valle}, \hyperref[auth:a667]{A. A. M{\'{a}}rquez}, \hyperref[auth:a668]{R. M. Gasca}, \hyperref[auth:a669]{M. Toro} & On Selecting and Scheduling Assembly Plans Using Constraint Programming & \href{../works/ValleMGT03.pdf}{Yes} & \cite{ValleMGT03} & 2003 & KES 2003 & 8 & 7 7 6 & 7 11 & \ref{b:ValleMGT03} & n/a\\
\end{longtable}
}

\subsection{KR}

\index{KR}
{\scriptsize
\begin{longtable}{>{\raggedright\arraybackslash}p{3cm}>{\raggedright\arraybackslash}p{4.5cm}>{\raggedright\arraybackslash}p{6.0cm}rrrp{2.5cm}rp{1cm}p{1cm}rr}
\rowcolor{white}\caption{Papers in Conference Series KR (Total 1) (Total 1)}\\ \toprule
\rowcolor{white}\shortstack{Key\\Source} & Authors & Title (Colored by Open Access)& LC & Cite & Year & \shortstack{Conference\\/Journal\\/School} & Pages & \shortstack{Cites\\OC XR\\SC} & \shortstack{Refs\\OC\\XR} & b & c \\ \midrule\endhead
\bottomrule
\endfoot
LuoVLBM16 \href{http://www.aaai.org/ocs/index.php/KR/KR16/paper/view/12909}{LuoVLBM16} & \hyperref[auth:a813]{R. Luo}, \hyperref[auth:a814]{R. A. Valenzano}, \hyperref[auth:a815]{Y. Li}, \hyperref[auth:a89]{J. C. Beck}, \hyperref[auth:a816]{S. A. McIlraith} & Using Metric Temporal Logic to Specify Scheduling Problems & \href{../works/LuoVLBM16.pdf}{Yes} & \cite{LuoVLBM16} & 2016 & KR 2016 & 4 & 0 0 0 & 0 0 & \ref{b:LuoVLBM16} & n/a\\
\end{longtable}
}

\subsection{LION}

\index{LION}
{\scriptsize
\begin{longtable}{>{\raggedright\arraybackslash}p{3cm}>{\raggedright\arraybackslash}p{4.5cm}>{\raggedright\arraybackslash}p{6.0cm}rrrp{2.5cm}rp{1cm}p{1cm}rr}
\rowcolor{white}\caption{Papers in Conference Series LION (Total 1) (Total 1)}\\ \toprule
\rowcolor{white}\shortstack{Key\\Source} & Authors & Title (Colored by Open Access)& LC & Cite & Year & \shortstack{Conference\\/Journal\\/School} & Pages & \shortstack{Cites\\OC XR\\SC} & \shortstack{Refs\\OC\\XR} & b & c \\ \midrule\endhead
\bottomrule
\endfoot
AmadiniGM16 \href{http://dx.doi.org/10.1007/978-3-319-50349-3_16}{AmadiniGM16} & \hyperref[auth:a910]{R. Amadini}, \hyperref[auth:a192]{M. Gabbrielli}, \hyperref[auth:a193]{J. Mauro} & \cellcolor{green!10}Parallelizing Constraint Solvers for Hard RCPSP Instances & \href{../works/AmadiniGM16.pdf}{Yes} & \cite{AmadiniGM16} & 2016 & LION 2016 & 7 & 2 2 2 & 16 20 & \ref{b:AmadiniGM16} & \ref{c:AmadiniGM16}\\
\end{longtable}
}

\subsection{LPNMR}

\index{LPNMR}
{\scriptsize
\begin{longtable}{>{\raggedright\arraybackslash}p{3cm}>{\raggedright\arraybackslash}p{4.5cm}>{\raggedright\arraybackslash}p{6.0cm}rrrp{2.5cm}rp{1cm}p{1cm}rr}
\rowcolor{white}\caption{Papers in Conference Series LPNMR (Total 1) (Total 1)}\\ \toprule
\rowcolor{white}\shortstack{Key\\Source} & Authors & Title (Colored by Open Access)& LC & Cite & Year & \shortstack{Conference\\/Journal\\/School} & Pages & \shortstack{Cites\\OC XR\\SC} & \shortstack{Refs\\OC\\XR} & b & c \\ \midrule\endhead
\bottomrule
\endfoot
Balduccini11 \href{https://doi.org/10.1007/978-3-642-20895-9_33}{Balduccini11} & \hyperref[auth:a1043]{M. Balduccini} & Industrial-Size Scheduling with {ASP+CP} & \href{../works/Balduccini11.pdf}{Yes} & \cite{Balduccini11} & 2011 & LPNMR 2011 & 13 & 20 22 38 & 9 13 & \ref{b:Balduccini11} & n/a\\
\end{longtable}
}

\subsection{MIKE}

\index{MIKE}
{\scriptsize
\begin{longtable}{>{\raggedright\arraybackslash}p{3cm}>{\raggedright\arraybackslash}p{4.5cm}>{\raggedright\arraybackslash}p{6.0cm}rrrp{2.5cm}rp{1cm}p{1cm}rr}
\rowcolor{white}\caption{Papers in Conference Series MIKE (Total 1) (Total 1)}\\ \toprule
\rowcolor{white}\shortstack{Key\\Source} & Authors & Title (Colored by Open Access)& LC & Cite & Year & \shortstack{Conference\\/Journal\\/School} & Pages & \shortstack{Cites\\OC XR\\SC} & \shortstack{Refs\\OC\\XR} & b & c \\ \midrule\endhead
\bottomrule
\endfoot
LiuLH18 \href{http://dx.doi.org/10.1007/978-3-030-05918-7_7}{LiuLH18} & \hyperref[auth:a1391]{K. Liu}, \hyperref[auth:a1392]{S. Loeffler}, \hyperref[auth:a1393]{P. Hofstedt} & Solving the Traveling Tournament Problem with Predefined Venues by Parallel Constraint Programming & \href{../works/LiuLH18.pdf}{Yes} & \cite{LiuLH18} & 2018 & MIKE 2018 & 16 & 2 2 1 & 7 12 & \ref{b:LiuLH18} & n/a\\
\end{longtable}
}

\subsection{MISTA}

\index{MISTA}
{\scriptsize
\begin{longtable}{>{\raggedright\arraybackslash}p{3cm}>{\raggedright\arraybackslash}p{4.5cm}>{\raggedright\arraybackslash}p{6.0cm}rrrp{2.5cm}rp{1cm}p{1cm}rr}
\rowcolor{white}\caption{Papers in Conference Series MISTA (Total 1) (Total 1)}\\ \toprule
\rowcolor{white}\shortstack{Key\\Source} & Authors & Title (Colored by Open Access)& LC & Cite & Year & \shortstack{Conference\\/Journal\\/School} & Pages & \shortstack{Cites\\OC XR\\SC} & \shortstack{Refs\\OC\\XR} & b & c \\ \midrule\endhead
\bottomrule
\endfoot
OddiPCC05 \href{http://dx.doi.org/10.1007/0-387-27744-7_7}{OddiPCC05} & \hyperref[auth:a282]{A. Oddi}, \hyperref[auth:a283]{N. Policella}, \hyperref[auth:a284]{A. Cesta}, \hyperref[auth:a285]{G. Cortellessa} & Constraint-Based Random Search for Solving Spacecraft Downlink Scheduling Problems & No & \cite{OddiPCC05} & 2005 & MISTA 2005 & 28 & 3 3 0 & 12 19 & No & n/a\\
\end{longtable}
}

\subsection{Operations Research Proceedings}

\index{Operations Research Proceedings}
{\scriptsize
\begin{longtable}{>{\raggedright\arraybackslash}p{3cm}>{\raggedright\arraybackslash}p{4.5cm}>{\raggedright\arraybackslash}p{6.0cm}rrrp{2.5cm}rp{1cm}p{1cm}rr}
\rowcolor{white}\caption{Papers in Conference Series Operations Research Proceedings (Total 1) (Total 1)}\\ \toprule
\rowcolor{white}\shortstack{Key\\Source} & Authors & Title (Colored by Open Access)& LC & Cite & Year & \shortstack{Conference\\/Journal\\/School} & Pages & \shortstack{Cites\\OC XR\\SC} & \shortstack{Refs\\OC\\XR} & b & c \\ \midrule\endhead
\bottomrule
\endfoot
DorndorfPH99 \href{http://dx.doi.org/10.1007/978-3-642-58409-1_35}{DorndorfPH99} & \hyperref[auth:a904]{U. Dorndorf}, \hyperref[auth:a438]{E. Pesch}, \hyperref[auth:a905]{T. P. Huy} & Recent Developments in Scheduling & No & \cite{DorndorfPH99} & 1999 & Operations Research Proceedings 1999 & 13 & 0 0 0 & 34 61 & No & n/a\\
\end{longtable}
}

\subsection{PACT}

\index{PACT}
{\scriptsize
\begin{longtable}{>{\raggedright\arraybackslash}p{3cm}>{\raggedright\arraybackslash}p{4.5cm}>{\raggedright\arraybackslash}p{6.0cm}rrrp{2.5cm}rp{1cm}p{1cm}rr}
\rowcolor{white}\caption{Papers in Conference Series PACT (Total 2) (Total 2)}\\ \toprule
\rowcolor{white}\shortstack{Key\\Source} & Authors & Title (Colored by Open Access)& LC & Cite & Year & \shortstack{Conference\\/Journal\\/School} & Pages & \shortstack{Cites\\OC XR\\SC} & \shortstack{Refs\\OC\\XR} & b & c \\ \midrule\endhead
\bottomrule
\endfoot
BoucherBVBL97 \href{}{BoucherBVBL97} & \hyperref[auth:a690]{E. Boucher}, \hyperref[auth:a691]{A. Bachelu}, \hyperref[auth:a692]{C. Varnier}, \hyperref[auth:a693]{P. Baptiste}, \hyperref[auth:a694]{B. Legeard} & Multi-criteria Comparison Between Algorithmic, Constraint Logic and Specific Constraint Programming on a Real Schedulingt Problem & No & \cite{BoucherBVBL97} & 1997 & PACT 1997 & 18 & 0 0 0 & 0 0 & No & n/a\\
PapeB97 \href{}{PapeB97} & \hyperref[auth:a163]{C. L. Pape}, \hyperref[auth:a162]{P. Baptiste} & A Constraint Programming Library for Preemptive and Non-Preemptive Scheduling & \href{../works/PapeB97.pdf}{Yes} & \cite{PapeB97} & 1997 & PACT 1997 & 20 & 0 0 0 & 0 0 & \ref{b:PapeB97} & n/a\\
\end{longtable}
}

\subsection{PADL}

\index{PADL}
{\scriptsize
\begin{longtable}{>{\raggedright\arraybackslash}p{3cm}>{\raggedright\arraybackslash}p{4.5cm}>{\raggedright\arraybackslash}p{6.0cm}rrrp{2.5cm}rp{1cm}p{1cm}rr}
\rowcolor{white}\caption{Papers in Conference Series PADL (Total 2) (Total 2)}\\ \toprule
\rowcolor{white}\shortstack{Key\\Source} & Authors & Title (Colored by Open Access)& LC & Cite & Year & \shortstack{Conference\\/Journal\\/School} & Pages & \shortstack{Cites\\OC XR\\SC} & \shortstack{Refs\\OC\\XR} & b & c \\ \midrule\endhead
\bottomrule
\endfoot
JelinekB16 \href{https://doi.org/10.1007/978-3-319-28228-2_1}{JelinekB16} & \hyperref[auth:a779]{J. Jel{\'{\i}}nek}, \hyperref[auth:a152]{R. Bart{\'{a}}k} & Using Constraint Logic Programming to Schedule Solar Array Operations on the International Space Station & \href{../works/JelinekB16.pdf}{Yes} & \cite{JelinekB16} & 2016 & PADL 2016 & 10 & 0 0 0 & 5 8 & \ref{b:JelinekB16} & n/a\\
CarlssonKA99 \href{https://doi.org/10.1007/3-540-49201-1\_23}{CarlssonKA99} & \hyperref[auth:a91]{M. Carlsson}, \hyperref[auth:a709]{P. Kreuger}, \hyperref[auth:a1414]{E. {\AA}str{\"{o}}m} & Constraint-Based Resource Allocation and Scheduling in Steel Manufacturing & \href{../works/CarlssonKA99.pdf}{Yes} & \cite{CarlssonKA99} & 1999 & PADL 1999 & 15 & 1 1 0 & 3 5 & \ref{b:CarlssonKA99} & n/a\\
\end{longtable}
}

\subsection{PATAT}

\index{PATAT}
{\scriptsize
\begin{longtable}{>{\raggedright\arraybackslash}p{3cm}>{\raggedright\arraybackslash}p{4.5cm}>{\raggedright\arraybackslash}p{6.0cm}rrrp{2.5cm}rp{1cm}p{1cm}rr}
\rowcolor{white}\caption{Papers in Conference Series PATAT (Total 3) (Total 3)}\\ \toprule
\rowcolor{white}\shortstack{Key\\Source} & Authors & Title (Colored by Open Access)& LC & Cite & Year & \shortstack{Conference\\/Journal\\/School} & Pages & \shortstack{Cites\\OC XR\\SC} & \shortstack{Refs\\OC\\XR} & b & c \\ \midrule\endhead
\bottomrule
\endfoot
Trick03 \href{https://doi.org/10.1007/978-3-540-45157-0_4}{Trick03} & \hyperref[auth:a1390]{M. A. Trick} & Integer and constraint programming approaches for round-robin tournament scheduling & \href{../works/Trick03.pdf}{Yes} & \cite{Trick03} & 2003 & PATAT 2003 & 15 & 22 24 39 & 15 25 & \ref{b:Trick03} & n/a\\
EastonNT02 \href{https://doi.org/10.1007/978-3-540-45157-0\_6}{EastonNT02} & \hyperref[auth:a1432]{K. Easton}, \hyperref[auth:a1433]{G. L. Nemhauser}, \hyperref[auth:a1390]{M. A. Trick} & Solving the Travelling Tournament Problem: {A} Combined Integer Programming and Constraint Programming Approach & \href{../works/EastonNT02.pdf}{Yes} & \cite{EastonNT02} & 2002 & PATAT 2002 & 13 & 48 50 0 & 8 11 & \ref{b:EastonNT02} & n/a\\
ElkhyariGJ02a \href{https://doi.org/10.1007/978-3-540-45157-0_3}{ElkhyariGJ02a} & \hyperref[auth:a292]{A. Elkhyari}, \hyperref[auth:a293]{C. Gu{\'{e}}ret}, \hyperref[auth:a247]{N. Jussien} & \cellcolor{green!10}Solving Dynamic Resource Constraint Project Scheduling Problems Using New Constraint Programming Tools & \href{../works/ElkhyariGJ02a.pdf}{Yes} & \cite{ElkhyariGJ02a} & 2002 & PATAT 2002 & 24 & 9 9 14 & 20 32 & \ref{b:ElkhyariGJ02a} & n/a\\
\end{longtable}
}

\subsection{PLILP}

\index{PLILP}
{\scriptsize
\begin{longtable}{>{\raggedright\arraybackslash}p{3cm}>{\raggedright\arraybackslash}p{4.5cm}>{\raggedright\arraybackslash}p{6.0cm}rrrp{2.5cm}rp{1cm}p{1cm}rr}
\rowcolor{white}\caption{Papers in Conference Series PLILP (Total 1) (Total 1)}\\ \toprule
\rowcolor{white}\shortstack{Key\\Source} & Authors & Title (Colored by Open Access)& LC & Cite & Year & \shortstack{Conference\\/Journal\\/School} & Pages & \shortstack{Cites\\OC XR\\SC} & \shortstack{Refs\\OC\\XR} & b & c \\ \midrule\endhead
\bottomrule
\endfoot
ErtlK91 \href{https://doi.org/10.1007/3-540-54444-5_89}{ErtlK91} & \hyperref[auth:a702]{M. A. Ertl}, \hyperref[auth:a703]{A. Krall} & \cellcolor{green!10}Optimal Instruction Scheduling using Constraint Logic Programming & \href{../works/ErtlK91.pdf}{Yes} & \cite{ErtlK91} & 1991 & PLILP 1991 & 12 & 14 14 13 & 14 19 & \ref{b:ErtlK91} & n/a\\
\end{longtable}
}

\subsection{PRICAI}

\index{PRICAI}
{\scriptsize
\begin{longtable}{>{\raggedright\arraybackslash}p{3cm}>{\raggedright\arraybackslash}p{4.5cm}>{\raggedright\arraybackslash}p{6.0cm}rrrp{2.5cm}rp{1cm}p{1cm}rr}
\rowcolor{white}\caption{Papers in Conference Series PRICAI (Total 1) (Total 1)}\\ \toprule
\rowcolor{white}\shortstack{Key\\Source} & Authors & Title (Colored by Open Access)& LC & Cite & Year & \shortstack{Conference\\/Journal\\/School} & Pages & \shortstack{Cites\\OC XR\\SC} & \shortstack{Refs\\OC\\XR} & b & c \\ \midrule\endhead
\bottomrule
\endfoot
LiuJ06 \href{https://doi.org/10.1007/11801603_92}{LiuJ06} & \hyperref[auth:a654]{Y. Liu}, \hyperref[auth:a655]{Y. Jiang} & {LP-TPOP:} Integrating Planning and Scheduling Through Constraint Programming & \href{../works/LiuJ06.pdf}{Yes} & \cite{LiuJ06} & 2006 & PRICAI 2006 & 5 & 0 0 1 & 0 0 & \ref{b:LiuJ06} & n/a\\
\end{longtable}
}

\subsection{PSE}

\index{PSE}
{\scriptsize
\begin{longtable}{>{\raggedright\arraybackslash}p{3cm}>{\raggedright\arraybackslash}p{4.5cm}>{\raggedright\arraybackslash}p{6.0cm}rrrp{2.5cm}rp{1cm}p{1cm}rr}
\rowcolor{white}\caption{Papers in Conference Series PSE (Total 1) (Total 1)}\\ \toprule
\rowcolor{white}\shortstack{Key\\Source} & Authors & Title (Colored by Open Access)& LC & Cite & Year & \shortstack{Conference\\/Journal\\/School} & Pages & \shortstack{Cites\\OC XR\\SC} & \shortstack{Refs\\OC\\XR} & b & c \\ \midrule\endhead
\bottomrule
\endfoot
FelizariAL09 \href{https://www.sciencedirect.com/science/article/pii/S1570794605800136}{FelizariAL09} & \hyperref[auth:a1463]{L. C. Felizari}, \hyperref[auth:a1464]{Lucia V. R. de Arruda}, \hyperref[auth:a1465]{R. Lueders}, \hyperref[auth:a1466]{S. L. Stebel} & Sequencing Batches in a Real-World Pipeline Network Using Constraint Programming & No & \cite{FelizariAL09} & 2009 & PSE 2009 & 6 & 7 7 12 & 2 7 & No & n/a\\
\end{longtable}
}

\subsection{RAAD}

\index{RAAD}
{\scriptsize
\begin{longtable}{>{\raggedright\arraybackslash}p{3cm}>{\raggedright\arraybackslash}p{4.5cm}>{\raggedright\arraybackslash}p{6.0cm}rrrp{2.5cm}rp{1cm}p{1cm}rr}
\rowcolor{white}\caption{Papers in Conference Series RAAD (Total 1) (Total 1)}\\ \toprule
\rowcolor{white}\shortstack{Key\\Source} & Authors & Title (Colored by Open Access)& LC & Cite & Year & \shortstack{Conference\\/Journal\\/School} & Pages & \shortstack{Cites\\OC XR\\SC} & \shortstack{Refs\\OC\\XR} & b & c \\ \midrule\endhead
\bottomrule
\endfoot
ParkUJR19 \href{https://doi.org/10.1007/978-3-030-19648-6_15}{ParkUJR19} & \hyperref[auth:a544]{H. Park}, \hyperref[auth:a545]{J. Um}, \hyperref[auth:a546]{J.-Y. Jung}, \hyperref[auth:a547]{M. Ruskowski} & Developing a Production Scheduling System for Modular Factory Using Constraint Programming & \href{../works/ParkUJR19.pdf}{Yes} & \cite{ParkUJR19} & 2019 & RAAD 2019 & 8 & 1 1 3 & 3 6 & \ref{b:ParkUJR19} & n/a\\
\end{longtable}
}

\subsection{RAST}

\index{RAST}
{\scriptsize
\begin{longtable}{>{\raggedright\arraybackslash}p{3cm}>{\raggedright\arraybackslash}p{4.5cm}>{\raggedright\arraybackslash}p{6.0cm}rrrp{2.5cm}rp{1cm}p{1cm}rr}
\rowcolor{white}\caption{Papers in Conference Series RAST (Total 1) (Total 1)}\\ \toprule
\rowcolor{white}\shortstack{Key\\Source} & Authors & Title (Colored by Open Access)& LC & Cite & Year & \shortstack{Conference\\/Journal\\/School} & Pages & \shortstack{Cites\\OC XR\\SC} & \shortstack{Refs\\OC\\XR} & b & c \\ \midrule\endhead
\bottomrule
\endfoot
KucukY19 \href{https://api.semanticscholar.org/CorpusID:198146161}{KucukY19} & \hyperref[auth:a762]{M. K{\"u}ç{\"u}k}, \hyperref[auth:a421]{S. T. Yildiz} & A Constraint Programming Approach for Agile Earth Observation Satellite Scheduling Problem & \href{../works/KucukY19.pdf}{Yes} & \cite{KucukY19} & 2019 & RAST 2019 & 5 & 2 3 7 & 17 23 & \ref{b:KucukY19} & n/a\\
\end{longtable}
}

\subsection{SAT}

\index{SAT}
{\scriptsize
\begin{longtable}{>{\raggedright\arraybackslash}p{3cm}>{\raggedright\arraybackslash}p{4.5cm}>{\raggedright\arraybackslash}p{6.0cm}rrrp{2.5cm}rp{1cm}p{1cm}rr}
\rowcolor{white}\caption{Papers in Conference Series SAT (Total 1) (Total 1)}\\ \toprule
\rowcolor{white}\shortstack{Key\\Source} & Authors & Title (Colored by Open Access)& LC & Cite & Year & \shortstack{Conference\\/Journal\\/School} & Pages & \shortstack{Cites\\OC XR\\SC} & \shortstack{Refs\\OC\\XR} & b & c \\ \midrule\endhead
\bottomrule
\endfoot
CohenHB17 \href{https://doi.org/10.1007/978-3-319-66263-3_10}{CohenHB17} & \hyperref[auth:a805]{E. Cohen}, \hyperref[auth:a806]{G. Huang}, \hyperref[auth:a89]{J. C. Beck} & {(I} Can Get) Satisfaction: Preference-Based Scheduling for Concert-Goers at Multi-venue Music Festivals & \href{../works/CohenHB17.pdf}{Yes} & \cite{CohenHB17} & 2017 & SAT 2017 & 17 & 1 1 4 & 12 26 & \ref{b:CohenHB17} & n/a\\
\end{longtable}
}

\subsection{SCAM}

\index{SCAM}
{\scriptsize
\begin{longtable}{>{\raggedright\arraybackslash}p{3cm}>{\raggedright\arraybackslash}p{4.5cm}>{\raggedright\arraybackslash}p{6.0cm}rrrp{2.5cm}rp{1cm}p{1cm}rr}
\rowcolor{white}\caption{Papers in Conference Series SCAM (Total 1) (Total 1)}\\ \toprule
\rowcolor{white}\shortstack{Key\\Source} & Authors & Title (Colored by Open Access)& LC & Cite & Year & \shortstack{Conference\\/Journal\\/School} & Pages & \shortstack{Cites\\OC XR\\SC} & \shortstack{Refs\\OC\\XR} & b & c \\ \midrule\endhead
\bottomrule
\endfoot
ZibranR11a \href{https://doi.org/10.1109/SCAM.2011.21}{ZibranR11a} & \hyperref[auth:a619]{M. F. Zibran}, \hyperref[auth:a620]{C. K. Roy} & A Constraint Programming Approach to Conflict-Aware Optimal Scheduling of Prioritized Code Clone Refactoring & \href{../works/ZibranR11a.pdf}{Yes} & \cite{ZibranR11a} & 2011 & SCAM 2011 & 10 & 26 26 33 & 27 35 & \ref{b:ZibranR11a} & n/a\\
\end{longtable}
}

\subsection{SEA}

\index{SEA}
{\scriptsize
\begin{longtable}{>{\raggedright\arraybackslash}p{3cm}>{\raggedright\arraybackslash}p{4.5cm}>{\raggedright\arraybackslash}p{6.0cm}rrrp{2.5cm}rp{1cm}p{1cm}rr}
\rowcolor{white}\caption{Papers in Conference Series SEA (Total 1) (Total 1)}\\ \toprule
\rowcolor{white}\shortstack{Key\\Source} & Authors & Title (Colored by Open Access)& LC & Cite & Year & \shortstack{Conference\\/Journal\\/School} & Pages & \shortstack{Cites\\OC XR\\SC} & \shortstack{Refs\\OC\\XR} & b & c \\ \midrule\endhead
\bottomrule
\endfoot
HeinzS11 \href{https://doi.org/10.1007/978-3-642-20662-7_34}{HeinzS11} & \hyperref[auth:a133]{S. Heinz}, \hyperref[auth:a134]{J. Schulz} & Explanations for the Cumulative Constraint: An Experimental Study & \href{../works/HeinzS11.pdf}{Yes} & \cite{HeinzS11} & 2011 & SEA 2011 & 10 & 5 6 5 & 12 14 & \ref{b:HeinzS11} & n/a\\
\end{longtable}
}

\subsection{SOCS}

\index{SOCS}
{\scriptsize
\begin{longtable}{>{\raggedright\arraybackslash}p{3cm}>{\raggedright\arraybackslash}p{4.5cm}>{\raggedright\arraybackslash}p{6.0cm}rrrp{2.5cm}rp{1cm}p{1cm}rr}
\rowcolor{white}\caption{Papers in Conference Series SOCS (Total 1) (Total 1)}\\ \toprule
\rowcolor{white}\shortstack{Key\\Source} & Authors & Title (Colored by Open Access)& LC & Cite & Year & \shortstack{Conference\\/Journal\\/School} & Pages & \shortstack{Cites\\OC XR\\SC} & \shortstack{Refs\\OC\\XR} & b & c \\ \midrule\endhead
\bottomrule
\endfoot
TranDRFWOVB16 \href{https://doi.org/10.1609/socs.v7i1.18390}{TranDRFWOVB16} & \hyperref[auth:a799]{T. T. Tran}, \hyperref[auth:a809]{M. Do}, \hyperref[auth:a810]{E. G. Rieffel}, \hyperref[auth:a379]{J. Frank}, \hyperref[auth:a808]{Z. Wang}, \hyperref[auth:a811]{B. O'Gorman}, \hyperref[auth:a812]{D. Venturelli}, \hyperref[auth:a89]{J. C. Beck} & A Hybrid Quantum-Classical Approach to Solving Scheduling Problems & \href{../works/TranDRFWOVB16.pdf}{Yes} & \cite{TranDRFWOVB16} & 2016 & SOCS 2016 & 9 & 3 9 0 & 0 0 & \ref{b:TranDRFWOVB16} & n/a\\
\end{longtable}
}

\subsection{SoC}

\index{SoC}
{\scriptsize
\begin{longtable}{>{\raggedright\arraybackslash}p{3cm}>{\raggedright\arraybackslash}p{4.5cm}>{\raggedright\arraybackslash}p{6.0cm}rrrp{2.5cm}rp{1cm}p{1cm}rr}
\rowcolor{white}\caption{Papers in Conference Series SoC (Total 1) (Total 1)}\\ \toprule
\rowcolor{white}\shortstack{Key\\Source} & Authors & Title (Colored by Open Access)& LC & Cite & Year & \shortstack{Conference\\/Journal\\/School} & Pages & \shortstack{Cites\\OC XR\\SC} & \shortstack{Refs\\OC\\XR} & b & c \\ \midrule\endhead
\bottomrule
\endfoot
QuSN06 \href{https://doi.org/10.1109/ISSOC.2006.321973}{QuSN06} & \hyperref[auth:a651]{Y. Qu}, \hyperref[auth:a652]{J.-P. Soininen}, \hyperref[auth:a653]{J. Nurmi} & Using Constraint Programming to Achieve Optimal Prefetch Scheduling for Dependent Tasks on Run-Time Reconfigurable Devices & \href{../works/QuSN06.pdf}{Yes} & \cite{QuSN06} & 2006 & SoC 2006 & 4 & 2 2 5 & 5 14 & \ref{b:QuSN06} & n/a\\
\end{longtable}
}

\subsection{TENCON}

\index{TENCON}
{\scriptsize
\begin{longtable}{>{\raggedright\arraybackslash}p{3cm}>{\raggedright\arraybackslash}p{4.5cm}>{\raggedright\arraybackslash}p{6.0cm}rrrp{2.5cm}rp{1cm}p{1cm}rr}
\rowcolor{white}\caption{Papers in Conference Series TENCON (Total 1) (Total 1)}\\ \toprule
\rowcolor{white}\shortstack{Key\\Source} & Authors & Title (Colored by Open Access)& LC & Cite & Year & \shortstack{Conference\\/Journal\\/School} & Pages & \shortstack{Cites\\OC XR\\SC} & \shortstack{Refs\\OC\\XR} & b & c \\ \midrule\endhead
\bottomrule
\endfoot
NishikawaSTT18a \href{https://doi.org/10.1109/TENCON.2018.8650168}{NishikawaSTT18a} & \hyperref[auth:a531]{H. Nishikawa}, \hyperref[auth:a532]{K. Shimada}, \hyperref[auth:a533]{I. Taniguchi}, \hyperref[auth:a534]{H. Tomiyama} & Scheduling of Malleable Tasks Based on Constraint Programming & \href{../works/NishikawaSTT18a.pdf}{Yes} & \cite{NishikawaSTT18a} & 2018 & TENCON 2018 & 6 & 1 1 1 & 9 16 & \ref{b:NishikawaSTT18a} & n/a\\
\end{longtable}
}



\clearpage
\section{Articles by Journal}

\subsection{4OR}

\index{4OR}
{\scriptsize
\begin{longtable}{>{\raggedright\arraybackslash}p{2.5cm}>{\raggedright\arraybackslash}p{4.5cm}>{\raggedright\arraybackslash}p{6.0cm}p{1.0cm}rr>{\raggedright\arraybackslash}p{2.0cm}r>{\raggedright\arraybackslash}p{1cm}p{1cm}p{1cm}p{1cm}}
\rowcolor{white}\caption{Articles in Journal 4OR (Total 2)}\\ \toprule
\rowcolor{white}\shortstack{Key\\Source} & Authors & Title (Colored by Open Access)& \shortstack{Details\\LC} & Cite & Year & \shortstack{Conference\\/Journal\\/School} & Pages & Relevance &\shortstack{Cites\\OC XR\\SC} & \shortstack{Refs\\OC\\XR} & \shortstack{Links\\Cites\\Refs}\\ \midrule\endhead
\bottomrule
\endfoot
Talbi2013 \href{http://dx.doi.org/10.1007/s10288-013-0242-3}{Talbi2013} & \hyperref[auth:a1659]{E.-G. Talbi} & Combining metaheuristics with mathematical programming, constraint programming and machine learning & \hyperref[detail:Talbi2013]{Details} No & \cite{Talbi2013} & 2013 & 4OR & null & \noindent{}0.50 0.50 n/a & 15 15 22 & 90 150 & 9 1 8\\
MilanoW06 \href{http://dx.doi.org/10.1007/s10288-006-0019-z}{MilanoW06} & \hyperref[auth:a143]{M. Milano}, \hyperref[auth:a117]{M. G. Wallace} & Integrating operations research in constraint programming & \hyperref[detail:MilanoW06]{Details} \href{../works/MilanoW06.pdf}{Yes} & \cite{MilanoW06} & 2006 & 4OR & 45 & \noindent{}\textcolor{black!50}{0.00} \textcolor{black!50}{0.00} \textbf{47.13} & 18 18 22 & 46 67 & 20 4 16\\
\end{longtable}
}

\subsection{ACM Computing Surveys}

\index{ACM Computing Surveys}
{\scriptsize
\begin{longtable}{>{\raggedright\arraybackslash}p{2.5cm}>{\raggedright\arraybackslash}p{4.5cm}>{\raggedright\arraybackslash}p{6.0cm}p{1.0cm}rr>{\raggedright\arraybackslash}p{2.0cm}r>{\raggedright\arraybackslash}p{1cm}p{1cm}p{1cm}p{1cm}}
\rowcolor{white}\caption{Articles in Journal ACM Computing Surveys (Total 2)}\\ \toprule
\rowcolor{white}\shortstack{Key\\Source} & Authors & Title (Colored by Open Access)& \shortstack{Details\\LC} & Cite & Year & \shortstack{Conference\\/Journal\\/School} & Pages & Relevance &\shortstack{Cites\\OC XR\\SC} & \shortstack{Refs\\OC\\XR} & \shortstack{Links\\Cites\\Refs}\\ \midrule\endhead
\bottomrule
\endfoot
Lozano2019a \href{http://dx.doi.org/10.1145/3200920}{Lozano2019a} & \hyperref[auth:a1522]{R. C. Lozano}, \hyperref[auth:a92]{C. Schulte} & \cellcolor{green!10}Survey on Combinatorial Register Allocation and Instruction Scheduling \hyperref[abs:Lozano2019a]{Abstract} & \hyperref[detail:Lozano2019a]{Details} No & \cite{Lozano2019a} & 2019 & ACM Computing Surveys & null & \noindent{}\textcolor{black!50}{0.00} \textbf{2.50} n/a & 5 7 8 & 97 154 & 9 0 9\\
Smith-Miles2009 \href{http://dx.doi.org/10.1145/1456650.1456656}{Smith-Miles2009} & \hyperref[auth:a1742]{K. A. Smith-Miles} & Cross-disciplinary perspectives on meta-learning for algorithm selection \hyperref[abs:Smith-Miles2009]{Abstract} & \hyperref[detail:Smith-Miles2009]{Details} No & \cite{Smith-Miles2009} & 2009 & ACM Computing Surveys & null & \noindent{}\textcolor{black!50}{0.00} \textbf{1.50} n/a & 298 307 395 & 46 99 & 3 3 0\\
\end{longtable}
}

\subsection{ACM SIGART Bulletin}

\index{ACM SIGART Bulletin}
{\scriptsize
\begin{longtable}{>{\raggedright\arraybackslash}p{2.5cm}>{\raggedright\arraybackslash}p{4.5cm}>{\raggedright\arraybackslash}p{6.0cm}p{1.0cm}rr>{\raggedright\arraybackslash}p{2.0cm}r>{\raggedright\arraybackslash}p{1cm}p{1cm}p{1cm}p{1cm}}
\rowcolor{white}\caption{Articles in Journal ACM SIGART Bulletin (Total 1)}\\ \toprule
\rowcolor{white}\shortstack{Key\\Source} & Authors & Title (Colored by Open Access)& \shortstack{Details\\LC} & Cite & Year & \shortstack{Conference\\/Journal\\/School} & Pages & Relevance &\shortstack{Cites\\OC XR\\SC} & \shortstack{Refs\\OC\\XR} & \shortstack{Links\\Cites\\Refs}\\ \midrule\endhead
\bottomrule
\endfoot
Barber1993 \href{http://dx.doi.org/10.1145/152947.152955}{Barber1993} & \hyperref[auth:a1959]{F. A. Barber} & A metric time-point and duration-based temporal model \hyperref[abs:Barber1993]{Abstract} & \hyperref[detail:Barber1993]{Details} No & \cite{Barber1993} & 1993 & ACM SIGART Bulletin & null & \noindent{}\textcolor{black!50}{0.00} 0.50 n/a & 13 13 0 & 9 26 & 2 0 2\\
\end{longtable}
}

\subsection{ACM SIGPLAN Notices}

\index{ACM SIGPLAN Notices}
{\scriptsize
\begin{longtable}{>{\raggedright\arraybackslash}p{2.5cm}>{\raggedright\arraybackslash}p{4.5cm}>{\raggedright\arraybackslash}p{6.0cm}p{1.0cm}rr>{\raggedright\arraybackslash}p{2.0cm}r>{\raggedright\arraybackslash}p{1cm}p{1cm}p{1cm}p{1cm}}
\rowcolor{white}\caption{Articles in Journal ACM SIGPLAN Notices (Total 2)}\\ \toprule
\rowcolor{white}\shortstack{Key\\Source} & Authors & Title (Colored by Open Access)& \shortstack{Details\\LC} & Cite & Year & \shortstack{Conference\\/Journal\\/School} & Pages & Relevance &\shortstack{Cites\\OC XR\\SC} & \shortstack{Refs\\OC\\XR} & \shortstack{Links\\Cites\\Refs}\\ \midrule\endhead
\bottomrule
\endfoot
Lozano2014 \href{http://dx.doi.org/10.1145/2666357.2597815}{Lozano2014} & \hyperref[auth:a1522]{R. C. Lozano}, \hyperref[auth:a91]{M. Carlsson}, \hyperref[auth:a1523]{G. H. Blindell}, \hyperref[auth:a92]{C. Schulte} & Combinatorial spill code optimization and ultimate coalescing \hyperref[abs:Lozano2014]{Abstract} & \hyperref[detail:Lozano2014]{Details} No & \cite{Lozano2014} & 2014 & ACM SIGPLAN Notices & null & \noindent{}\textcolor{black!50}{0.00} \textbf{1.00} n/a & 3 2 0 & 12 22 & 1 0 1\\
Nowatzki2013 \href{http://dx.doi.org/10.1145/2499370.2462163}{Nowatzki2013} & \hyperref[auth:a1631]{T. Nowatzki}, \hyperref[auth:a1632]{M. Sartin-Tarm}, \hyperref[auth:a1633]{L. D. Carli}, \hyperref[auth:a1634]{K. Sankaralingam}, \hyperref[auth:a1635]{C. Estan}, \hyperref[auth:a1636]{B. Robatmili} & A general constraint-centric scheduling framework for spatial architectures \hyperref[abs:Nowatzki2013]{Abstract} & \hyperref[detail:Nowatzki2013]{Details} No & \cite{Nowatzki2013} & 2013 & ACM SIGPLAN Notices & null & \noindent{}\textcolor{black!50}{0.00} \textbf{1.50} n/a & 22 21 35 & 36 51 & 3 1 2\\
\end{longtable}
}

\subsection{ACM Transactions on Architecture and Code Optimization}

\index{ACM Transactions on Architecture and Code Optimization}
{\scriptsize
\begin{longtable}{>{\raggedright\arraybackslash}p{2.5cm}>{\raggedright\arraybackslash}p{4.5cm}>{\raggedright\arraybackslash}p{6.0cm}p{1.0cm}rr>{\raggedright\arraybackslash}p{2.0cm}r>{\raggedright\arraybackslash}p{1cm}p{1cm}p{1cm}p{1cm}}
\rowcolor{white}\caption{Articles in Journal ACM Transactions on Architecture and Code Optimization (Total 2)}\\ \toprule
\rowcolor{white}\shortstack{Key\\Source} & Authors & Title (Colored by Open Access)& \shortstack{Details\\LC} & Cite & Year & \shortstack{Conference\\/Journal\\/School} & Pages & Relevance &\shortstack{Cites\\OC XR\\SC} & \shortstack{Refs\\OC\\XR} & \shortstack{Links\\Cites\\Refs}\\ \midrule\endhead
\bottomrule
\endfoot
Rieber2021 \href{http://dx.doi.org/10.1145/3487922}{Rieber2021} & \hyperref[auth:a1890]{D. Rieber}, \hyperref[auth:a1891]{A. Acosta}, \hyperref[auth:a1892]{H. Fröning} & \cellcolor{gold!20}Joint Program and Layout Transformations to Enable Convolutional Operators on Specialized Hardware Based on Constraint Programming \hyperref[abs:Rieber2021]{Abstract} & \hyperref[detail:Rieber2021]{Details} No & \cite{Rieber2021} & 2021 & ACM Transactions on Architecture and Code Optimization & null & \noindent{}\textcolor{black!50}{0.00} \textbf{2.00} n/a & 0 0 0 & 22 38 & 2 0 2\\
Shobaki2013 \href{http://dx.doi.org/10.1145/2512432}{Shobaki2013} & \hyperref[auth:a1784]{G. Shobaki}, \hyperref[auth:a1785]{M. Shawabkeh}, \hyperref[auth:a1786]{N. E. A. Rmaileh} & \cellcolor{gold!20}Preallocation instruction scheduling with register pressure minimization using a combinatorial optimization approach \hyperref[abs:Shobaki2013]{Abstract} & \hyperref[detail:Shobaki2013]{Details} No & \cite{Shobaki2013} & 2013 & ACM Transactions on Architecture and Code Optimization & null & \noindent{}\textcolor{black!50}{0.00} \textcolor{black!50}{0.00} n/a & 16 18 21 & 11 19 & 2 2 0\\
\end{longtable}
}

\subsection{ACM Transactions on Computational Logic}

\index{ACM Transactions on Computational Logic}
{\scriptsize
\begin{longtable}{>{\raggedright\arraybackslash}p{2.5cm}>{\raggedright\arraybackslash}p{4.5cm}>{\raggedright\arraybackslash}p{6.0cm}p{1.0cm}rr>{\raggedright\arraybackslash}p{2.0cm}r>{\raggedright\arraybackslash}p{1cm}p{1cm}p{1cm}p{1cm}}
\rowcolor{white}\caption{Articles in Journal ACM Transactions on Computational Logic (Total 3)}\\ \toprule
\rowcolor{white}\shortstack{Key\\Source} & Authors & Title (Colored by Open Access)& \shortstack{Details\\LC} & Cite & Year & \shortstack{Conference\\/Journal\\/School} & Pages & Relevance &\shortstack{Cites\\OC XR\\SC} & \shortstack{Refs\\OC\\XR} & \shortstack{Links\\Cites\\Refs}\\ \midrule\endhead
\bottomrule
\endfoot
Choi2007 \href{http://dx.doi.org/10.1145/1276920.1276925}{Choi2007} & \hyperref[auth:a1816]{C. W. Choi}, \hyperref[auth:a1817]{J. H. M. Lee}, \hyperref[auth:a1818]{P. J. Stuckey} & \cellcolor{green!10}Removing propagation redundant constraints in redundant modeling \hyperref[abs:Choi2007]{Abstract} & \hyperref[detail:Choi2007]{Details} No & \cite{Choi2007} & 2007 & ACM Transactions on Computational Logic & null & \noindent{}\textcolor{black!50}{0.00} \textbf{1.25} n/a & 8 8 12 & 10 32 & 4 1 3\\
Michel2004 \href{http://dx.doi.org/10.1145/976706.976714}{Michel2004} & \hyperref[auth:a32]{L. Michel}, \hyperref[auth:a148]{P. V. Hentenryck} & A decomposition-based implementation of search strategies \hyperref[abs:Michel2004]{Abstract} & \hyperref[detail:Michel2004]{Details} No & \cite{Michel2004} & 2004 & ACM Transactions on Computational Logic & null & \noindent{}\textcolor{black!50}{0.00} \textbf{2.00} n/a & 10 10 11 & 12 34 & 7 1 6\\
Hentenryck2000 \href{http://dx.doi.org/10.1145/359496.359529}{Hentenryck2000} & \hyperref[auth:a148]{P. V. Hentenryck}, \hyperref[auth:a288]{L. Perron}, \hyperref[auth:a1653]{J.-F. Puget} & Search and strategies in OPL \hyperref[abs:Hentenryck2000]{Abstract} & \hyperref[detail:Hentenryck2000]{Details} No & \cite{Hentenryck2000} & 2000 & ACM Transactions on Computational Logic & null & \noindent{}\textcolor{black!50}{0.00} \textbf{1.00} n/a & 44 43 48 & 14 33 & 9 7 2\\
\end{longtable}
}

\subsection{ACM Transactions on Design Automation of Electronic Systems}

\index{ACM Transactions on Design Automation of Electronic Systems}
{\scriptsize
\begin{longtable}{>{\raggedright\arraybackslash}p{2.5cm}>{\raggedright\arraybackslash}p{4.5cm}>{\raggedright\arraybackslash}p{6.0cm}p{1.0cm}rr>{\raggedright\arraybackslash}p{2.0cm}r>{\raggedright\arraybackslash}p{1cm}p{1cm}p{1cm}p{1cm}}
\rowcolor{white}\caption{Articles in Journal ACM Transactions on Design Automation of Electronic Systems (Total 3)}\\ \toprule
\rowcolor{white}\shortstack{Key\\Source} & Authors & Title (Colored by Open Access)& \shortstack{Details\\LC} & Cite & Year & \shortstack{Conference\\/Journal\\/School} & Pages & Relevance &\shortstack{Cites\\OC XR\\SC} & \shortstack{Refs\\OC\\XR} & \shortstack{Links\\Cites\\Refs}\\ \midrule\endhead
\bottomrule
\endfoot
EmeretlisTAV17 \href{http://dx.doi.org/10.1145/3133219}{EmeretlisTAV17} & \hyperref[auth:a1227]{A. Emeretlis}, \hyperref[auth:a1228]{G. Theodoridis}, \hyperref[auth:a1229]{P. Alefragis}, \hyperref[auth:a1230]{N. Voros} & Static Mapping of Applications on Heterogeneous Multi-Core Platforms Combining Logic-Based Benders Decomposition with Integer Linear Programming & \hyperref[detail:EmeretlisTAV17]{Details} \href{../works/EmeretlisTAV17.pdf}{Yes} & \cite{EmeretlisTAV17} & 2017 & ACM Transactions on Design Automation of Electronic Systems & 24 & \noindent{}\textcolor{black!50}{0.00} \textcolor{black!50}{0.00} \textbf{5.91} & 4 6 9 & 42 48 & 11 1 10\\
Tseng2008 \href{http://dx.doi.org/10.1145/1367045.1367061}{Tseng2008} & \hyperref[auth:a1682]{I.-L. Tseng}, \hyperref[auth:a1683]{A. Postula} & Partitioning parameterized 45-degree polygons with constraint programming \hyperref[abs:Tseng2008]{Abstract} & \hyperref[detail:Tseng2008]{Details} No & \cite{Tseng2008} & 2008 & ACM Transactions on Design Automation of Electronic Systems & null & \noindent{}\textcolor{black!50}{0.00} 0.50 n/a & 6 6 9 & 18 30 & 2 0 2\\
Kuchcinski03 \href{http://dx.doi.org/10.1145/785411.785416}{Kuchcinski03} & \hyperref[auth:a660]{K. Kuchcinski} & Constraints-driven scheduling and resource assignment & \hyperref[detail:Kuchcinski03]{Details} \href{../works/Kuchcinski03.pdf}{Yes} & \cite{Kuchcinski03} & 2003 & ACM Transactions on Design Automation of Electronic Systems & 29 & \noindent{}\textcolor{black!50}{0.00} \textcolor{black!50}{0.00} \textbf{11.59} & 105 105 116 & 15 42 & 13 11 2\\
\end{longtable}
}

\subsection{ACM Transactions on Intelligent Systems and Technology}

\index{ACM Transactions on Intelligent Systems and Technology}
{\scriptsize
\begin{longtable}{>{\raggedright\arraybackslash}p{2.5cm}>{\raggedright\arraybackslash}p{4.5cm}>{\raggedright\arraybackslash}p{6.0cm}p{1.0cm}rr>{\raggedright\arraybackslash}p{2.0cm}r>{\raggedright\arraybackslash}p{1cm}p{1cm}p{1cm}p{1cm}}
\rowcolor{white}\caption{Articles in Journal ACM Transactions on Intelligent Systems and Technology (Total 2)}\\ \toprule
\rowcolor{white}\shortstack{Key\\Source} & Authors & Title (Colored by Open Access)& \shortstack{Details\\LC} & Cite & Year & \shortstack{Conference\\/Journal\\/School} & Pages & Relevance &\shortstack{Cites\\OC XR\\SC} & \shortstack{Refs\\OC\\XR} & \shortstack{Links\\Cites\\Refs}\\ \midrule\endhead
\bottomrule
\endfoot
Danzinger2023 \href{http://dx.doi.org/10.1145/3546871}{Danzinger2023} & \hyperref[auth:a1484]{P. Danzinger}, \hyperref[auth:a77]{T. Geibinger}, \hyperref[auth:a1485]{D. Janneau}, \hyperref[auth:a80]{F. Mischek}, \hyperref[auth:a45]{N. Musliu}, \hyperref[auth:a1486]{C. Poschalko} & A System for Automated Industrial Test Laboratory Scheduling \hyperref[abs:Danzinger2023]{Abstract} & \hyperref[detail:Danzinger2023]{Details} No & \cite{Danzinger2023} & 2023 & ACM Transactions on Intelligent Systems and Technology & null & \noindent{}\textcolor{black!50}{0.00} \textbf{5.00} n/a & 0 1 1 & 19 26 & 10 0 10\\
Refanidis2010 \href{http://dx.doi.org/10.1145/1869397.1869401}{Refanidis2010} & \hyperref[auth:a1546]{I. Refanidis}, \hyperref[auth:a19]{N. Yorke-Smith} & A constraint-based approach to scheduling an individual's activities \hyperref[abs:Refanidis2010]{Abstract} & \hyperref[detail:Refanidis2010]{Details} No & \cite{Refanidis2010} & 2010 & ACM Transactions on Intelligent Systems and Technology & null & \noindent{}\textcolor{black!50}{0.00} \textbf{2.00} n/a & 11 11 20 & 9 26 & 2 0 2\\
\end{longtable}
}

\subsection{ACM Transactions on Programming Languages and Systems}

\index{ACM Transactions on Programming Languages and Systems}
{\scriptsize
\begin{longtable}{>{\raggedright\arraybackslash}p{2.5cm}>{\raggedright\arraybackslash}p{4.5cm}>{\raggedright\arraybackslash}p{6.0cm}p{1.0cm}rr>{\raggedright\arraybackslash}p{2.0cm}r>{\raggedright\arraybackslash}p{1cm}p{1cm}p{1cm}p{1cm}}
\rowcolor{white}\caption{Articles in Journal ACM Transactions on Programming Languages and Systems (Total 1)}\\ \toprule
\rowcolor{white}\shortstack{Key\\Source} & Authors & Title (Colored by Open Access)& \shortstack{Details\\LC} & Cite & Year & \shortstack{Conference\\/Journal\\/School} & Pages & Relevance &\shortstack{Cites\\OC XR\\SC} & \shortstack{Refs\\OC\\XR} & \shortstack{Links\\Cites\\Refs}\\ \midrule\endhead
\bottomrule
\endfoot
Lozano2019 \href{http://dx.doi.org/10.1145/3332373}{Lozano2019} & \hyperref[auth:a1522]{R. C. Lozano}, \hyperref[auth:a91]{M. Carlsson}, \hyperref[auth:a1523]{G. H. Blindell}, \hyperref[auth:a92]{C. Schulte} & \cellcolor{green!10}Combinatorial Register Allocation and Instruction Scheduling \hyperref[abs:Lozano2019]{Abstract} & \hyperref[detail:Lozano2019]{Details} No & \cite{Lozano2019} & 2019 & ACM Transactions on Programming Languages and Systems & null & \noindent{}\textcolor{black!50}{0.00} \textbf{1.50} n/a & 10 13 16 & 56 100 & 8 0 8\\
\end{longtable}
}

\subsection{ACM Transactions on Reconfigurable Technology and Systems}

\index{ACM Transactions on Reconfigurable Technology and Systems}
{\scriptsize
\begin{longtable}{>{\raggedright\arraybackslash}p{2.5cm}>{\raggedright\arraybackslash}p{4.5cm}>{\raggedright\arraybackslash}p{6.0cm}p{1.0cm}rr>{\raggedright\arraybackslash}p{2.0cm}r>{\raggedright\arraybackslash}p{1cm}p{1cm}p{1cm}p{1cm}}
\rowcolor{white}\caption{Articles in Journal ACM Transactions on Reconfigurable Technology and Systems (Total 1)}\\ \toprule
\rowcolor{white}\shortstack{Key\\Source} & Authors & Title (Colored by Open Access)& \shortstack{Details\\LC} & Cite & Year & \shortstack{Conference\\/Journal\\/School} & Pages & Relevance &\shortstack{Cites\\OC XR\\SC} & \shortstack{Refs\\OC\\XR} & \shortstack{Links\\Cites\\Refs}\\ \midrule\endhead
\bottomrule
\endfoot
Martin2012 \href{http://dx.doi.org/10.1145/2209285.2209289}{Martin2012} & \hyperref[auth:a1578]{K. Martin}, \hyperref[auth:a659]{C. Wolinski}, \hyperref[auth:a660]{K. Kuchcinski}, \hyperref[auth:a1579]{A. Floch}, \hyperref[auth:a1532]{F. Charot} & Constraint Programming Approach to Reconfigurable Processor Extension Generation and Application Compilation \hyperref[abs:Martin2012]{Abstract} & \hyperref[detail:Martin2012]{Details} No & \cite{Martin2012} & 2012 & ACM Transactions on Reconfigurable Technology and Systems & null & \noindent{}\textcolor{black!50}{0.00} \textbf{2.00} n/a & 15 16 22 & 30 47 & 4 2 2\\
\end{longtable}
}

\subsection{ACM Transactions on Software Engineering and Methodology}

\index{ACM Transactions on Software Engineering and Methodology}
{\scriptsize
\begin{longtable}{>{\raggedright\arraybackslash}p{2.5cm}>{\raggedright\arraybackslash}p{4.5cm}>{\raggedright\arraybackslash}p{6.0cm}p{1.0cm}rr>{\raggedright\arraybackslash}p{2.0cm}r>{\raggedright\arraybackslash}p{1cm}p{1cm}p{1cm}p{1cm}}
\rowcolor{white}\caption{Articles in Journal ACM Transactions on Software Engineering and Methodology (Total 1)}\\ \toprule
\rowcolor{white}\shortstack{Key\\Source} & Authors & Title (Colored by Open Access)& \shortstack{Details\\LC} & Cite & Year & \shortstack{Conference\\/Journal\\/School} & Pages & Relevance &\shortstack{Cites\\OC XR\\SC} & \shortstack{Refs\\OC\\XR} & \shortstack{Links\\Cites\\Refs}\\ \midrule\endhead
\bottomrule
\endfoot
AlesioBNG15 \href{http://dx.doi.org/10.1145/2818640}{AlesioBNG15} & \hyperref[auth:a1223]{S. D. Alesio}, \hyperref[auth:a236]{L. C. Briand}, \hyperref[auth:a235]{S. Nejati}, \hyperref[auth:a195]{A. Gotlieb} & \cellcolor{green!10}Combining Genetic Algorithms and Constraint Programming to Support Stress Testing of Task Deadlines & \hyperref[detail:AlesioBNG15]{Details} \href{../works/AlesioBNG15.pdf}{Yes} & \cite{AlesioBNG15} & 2015 & ACM Transactions on Software Engineering and Methodology & 37 & \noindent{}\textbf{1.00} \textbf{1.00} \textbf{119.50} & 13 14 17 & 51 59 & 10 0 10\\
\end{longtable}
}

\subsection{AI}

\index{AI}
{\scriptsize
\begin{longtable}{>{\raggedright\arraybackslash}p{2.5cm}>{\raggedright\arraybackslash}p{4.5cm}>{\raggedright\arraybackslash}p{6.0cm}p{1.0cm}rr>{\raggedright\arraybackslash}p{2.0cm}r>{\raggedright\arraybackslash}p{1cm}p{1cm}p{1cm}p{1cm}}
\rowcolor{white}\caption{Articles in Journal AI (Total 1)}\\ \toprule
\rowcolor{white}\shortstack{Key\\Source} & Authors & Title (Colored by Open Access)& \shortstack{Details\\LC} & Cite & Year & \shortstack{Conference\\/Journal\\/School} & Pages & Relevance &\shortstack{Cites\\OC XR\\SC} & \shortstack{Refs\\OC\\XR} & \shortstack{Links\\Cites\\Refs}\\ \midrule\endhead
\bottomrule
\endfoot
Spieker2021 \href{http://dx.doi.org/10.3390/ai2040033}{Spieker2021} & \hyperref[auth:a196]{H. Spieker}, \hyperref[auth:a195]{A. Gotlieb} & \cellcolor{gold!20}Predictive Machine Learning of Objective Boundaries for Solving COPs \hyperref[abs:Spieker2021]{Abstract} & \hyperref[detail:Spieker2021]{Details} No & \cite{Spieker2021} & 2021 & AI & null & \noindent{}\textcolor{black!50}{0.00} \textbf{2.00} n/a & 0 0 0 & 51 75 & 4 0 4\\
\end{longtable}
}

\subsection{AI Magazine}

\index{AI Magazine}
{\scriptsize
\begin{longtable}{>{\raggedright\arraybackslash}p{2.5cm}>{\raggedright\arraybackslash}p{4.5cm}>{\raggedright\arraybackslash}p{6.0cm}p{1.0cm}rr>{\raggedright\arraybackslash}p{2.0cm}r>{\raggedright\arraybackslash}p{1cm}p{1cm}p{1cm}p{1cm}}
\rowcolor{white}\caption{Articles in Journal AI Magazine (Total 1)}\\ \toprule
\rowcolor{white}\shortstack{Key\\Source} & Authors & Title (Colored by Open Access)& \shortstack{Details\\LC} & Cite & Year & \shortstack{Conference\\/Journal\\/School} & Pages & Relevance &\shortstack{Cites\\OC XR\\SC} & \shortstack{Refs\\OC\\XR} & \shortstack{Links\\Cites\\Refs}\\ \midrule\endhead
\bottomrule
\endfoot
Chun2011 \href{http://dx.doi.org/10.1609/aimag.v32i2.2346}{Chun2011} & \hyperref[auth:a1322]{A. H. W. Chun} & \cellcolor{gold!20}Optimizing Limousine Service with AI \hyperref[abs:Chun2011]{Abstract} & \hyperref[detail:Chun2011]{Details} No & \cite{Chun2011} & 2011 & AI Magazine & null & \noindent{}\textcolor{black!50}{0.00} \textbf{1.50} n/a & 1 1 1 & 15 30 & 2 0 2\\
\end{longtable}
}

\subsection{AIChE Journal}

\index{AIChE Journal}
{\scriptsize
\begin{longtable}{>{\raggedright\arraybackslash}p{2.5cm}>{\raggedright\arraybackslash}p{4.5cm}>{\raggedright\arraybackslash}p{6.0cm}p{1.0cm}rr>{\raggedright\arraybackslash}p{2.0cm}r>{\raggedright\arraybackslash}p{1cm}p{1cm}p{1cm}p{1cm}}
\rowcolor{white}\caption{Articles in Journal AIChE Journal (Total 1)}\\ \toprule
\rowcolor{white}\shortstack{Key\\Source} & Authors & Title (Colored by Open Access)& \shortstack{Details\\LC} & Cite & Year & \shortstack{Conference\\/Journal\\/School} & Pages & Relevance &\shortstack{Cites\\OC XR\\SC} & \shortstack{Refs\\OC\\XR} & \shortstack{Links\\Cites\\Refs}\\ \midrule\endhead
\bottomrule
\endfoot
Velez2013 \href{http://dx.doi.org/10.1002/aic.14021}{Velez2013} & \hyperref[auth:a1480]{S. Velez}, \hyperref[auth:a1481]{A. Sundaramoorthy}, \hyperref[auth:a381]{C. T. Maravelias} & Valid Inequalities Based on Demand Propagation for Chemical Production Scheduling MIP Models \hyperref[abs:Velez2013]{Abstract} & \hyperref[detail:Velez2013]{Details} No & \cite{Velez2013} & 2013 & AIChE Journal & null & \noindent{}0.50 \textbf{3.75} n/a & 38 39 39 & 43 50 & 7 4 3\\
\end{longtable}
}

\subsection{APPLIED MATHEMATICAL MODELLING}

\index{APPLIED MATHEMATICAL MODELLING}
{\scriptsize
\begin{longtable}{>{\raggedright\arraybackslash}p{2.5cm}>{\raggedright\arraybackslash}p{4.5cm}>{\raggedright\arraybackslash}p{6.0cm}p{1.0cm}rr>{\raggedright\arraybackslash}p{2.0cm}r>{\raggedright\arraybackslash}p{1cm}p{1cm}p{1cm}p{1cm}}
\rowcolor{white}\caption{Articles in Journal APPLIED MATHEMATICAL MODELLING (Total 1)}\\ \toprule
\rowcolor{white}\shortstack{Key\\Source} & Authors & Title (Colored by Open Access)& \shortstack{Details\\LC} & Cite & Year & \shortstack{Conference\\/Journal\\/School} & Pages & Relevance &\shortstack{Cites\\OC XR\\SC} & \shortstack{Refs\\OC\\XR} & \shortstack{Links\\Cites\\Refs}\\ \midrule\endhead
\bottomrule
\endfoot
GhandehariK22 \href{http://dx.doi.org/10.1016/j.apm.2022.01.001}{GhandehariK22} & \hyperref[auth:a1461]{N. Ghandehari}, \hyperref[auth:a760]{K. Kianfar} & Mixed-integer linear programming, constraint programming and column generation approaches for operating room planning under block strategy \hyperref[abs:GhandehariK22]{Abstract} & \hyperref[detail:GhandehariK22]{Details} \href{../works/GhandehariK22.pdf}{Yes} & \cite{GhandehariK22} & 2022 & APPLIED MATHEMATICAL MODELLING & 16 & \noindent{}\textcolor{black!50}{0.00} \textcolor{black!50}{0.00} \textbf{8.19} & 4 4 4 & 46 55 & 6 1 5\\
\end{longtable}
}

\subsection{Acta Automatica Sinica}

\index{Acta Automatica Sinica}
{\scriptsize
\begin{longtable}{>{\raggedright\arraybackslash}p{2.5cm}>{\raggedright\arraybackslash}p{4.5cm}>{\raggedright\arraybackslash}p{6.0cm}p{1.0cm}rr>{\raggedright\arraybackslash}p{2.0cm}r>{\raggedright\arraybackslash}p{1cm}p{1cm}p{1cm}p{1cm}}
\rowcolor{white}\caption{Articles in Journal Acta Automatica Sinica (Total 1)}\\ \toprule
\rowcolor{white}\shortstack{Key\\Source} & Authors & Title (Colored by Open Access)& \shortstack{Details\\LC} & Cite & Year & \shortstack{Conference\\/Journal\\/School} & Pages & Relevance &\shortstack{Cites\\OC XR\\SC} & \shortstack{Refs\\OC\\XR} & \shortstack{Links\\Cites\\Refs}\\ \midrule\endhead
\bottomrule
\endfoot
LiuGT10 \href{http://dx.doi.org/10.3724/sp.j.1004.2010.00603}{LiuGT10} & \hyperref[auth:a1220]{S.-X. Liu}, \hyperref[auth:a1221]{Z. Guo}, \hyperref[auth:a1222]{J.-F. Tang} & Constraint Propagation for Cumulative Scheduling Problems with Precedences: Constraint Propagation for Cumulative Scheduling Problems with Precedences & \hyperref[detail:LiuGT10]{Details} No & \cite{LiuGT10} & 2010 & \cellcolor{red!20}Acta Automatica Sinica & 7 & \noindent{}\textbf{1.50} \textbf{1.50} n/a & 2 2 9 & 15 20 & 11 0 11\\
\end{longtable}
}

\subsection{Advanced Materials Research}

\index{Advanced Materials Research}
{\scriptsize
\begin{longtable}{>{\raggedright\arraybackslash}p{2.5cm}>{\raggedright\arraybackslash}p{4.5cm}>{\raggedright\arraybackslash}p{6.0cm}p{1.0cm}rr>{\raggedright\arraybackslash}p{2.0cm}r>{\raggedright\arraybackslash}p{1cm}p{1cm}p{1cm}p{1cm}}
\rowcolor{white}\caption{Articles in Journal Advanced Materials Research (Total 1)}\\ \toprule
\rowcolor{white}\shortstack{Key\\Source} & Authors & Title (Colored by Open Access)& \shortstack{Details\\LC} & Cite & Year & \shortstack{Conference\\/Journal\\/School} & Pages & Relevance &\shortstack{Cites\\OC XR\\SC} & \shortstack{Refs\\OC\\XR} & \shortstack{Links\\Cites\\Refs}\\ \midrule\endhead
\bottomrule
\endfoot
ShangGuan2012 \href{http://dx.doi.org/10.4028/www.scientific.net/amr.443-444.724}{ShangGuan2012} & \hyperref[auth:a1983]{C. X. ShangGuan}, \hyperref[auth:a1984]{J. T. Li}, \hyperref[auth:a1985]{R. F. Shi} & Rescheduling of Parallel Machines under Machine Failures \hyperref[abs:ShangGuan2012]{Abstract} & \hyperref[detail:ShangGuan2012]{Details} No & \cite{ShangGuan2012} & 2012 & Advanced Materials Research & null & \noindent{}\textcolor{black!50}{0.00} \textbf{2.50} n/a & 5 5 6 & 7 15 & 1 0 1\\
\end{longtable}
}

\subsection{Advances in Mechanical Engineering}

\index{Advances in Mechanical Engineering}
{\scriptsize
\begin{longtable}{>{\raggedright\arraybackslash}p{2.5cm}>{\raggedright\arraybackslash}p{4.5cm}>{\raggedright\arraybackslash}p{6.0cm}p{1.0cm}rr>{\raggedright\arraybackslash}p{2.0cm}r>{\raggedright\arraybackslash}p{1cm}p{1cm}p{1cm}p{1cm}}
\rowcolor{white}\caption{Articles in Journal Advances in Mechanical Engineering (Total 1)}\\ \toprule
\rowcolor{white}\shortstack{Key\\Source} & Authors & Title (Colored by Open Access)& \shortstack{Details\\LC} & Cite & Year & \shortstack{Conference\\/Journal\\/School} & Pages & Relevance &\shortstack{Cites\\OC XR\\SC} & \shortstack{Refs\\OC\\XR} & \shortstack{Links\\Cites\\Refs}\\ \midrule\endhead
\bottomrule
\endfoot
Li2014b \href{http://dx.doi.org/10.1155/2014/169097}{Li2014b} & \hyperref[auth:a2002]{Y. Li}, \hyperref[auth:a2003]{W. Zhao}, \hyperref[auth:a2017]{Y. Ma}, \hyperref[auth:a2018]{L. Hu} & \cellcolor{gold!20}Scheduling of Changes in Complex Engineering Design Process via Genetic Algorithm and Elementary Effects Method \hyperref[abs:Li2014b]{Abstract} & \hyperref[detail:Li2014b]{Details} No & \cite{Li2014b} & 2014 & Advances in Mechanical Engineering & null & \noindent{}\textcolor{black!50}{0.00} \textbf{1.25} n/a & 0 0 0 & 22 26 & 1 0 1\\
\end{longtable}
}

\subsection{Advances in Science and Technology Research Journal}

\index{Advances in Science and Technology Research Journal}
{\scriptsize
\begin{longtable}{>{\raggedright\arraybackslash}p{2.5cm}>{\raggedright\arraybackslash}p{4.5cm}>{\raggedright\arraybackslash}p{6.0cm}p{1.0cm}rr>{\raggedright\arraybackslash}p{2.0cm}r>{\raggedright\arraybackslash}p{1cm}p{1cm}p{1cm}p{1cm}}
\rowcolor{white}\caption{Articles in Journal Advances in Science and Technology Research Journal (Total 1)}\\ \toprule
\rowcolor{white}\shortstack{Key\\Source} & Authors & Title (Colored by Open Access)& \shortstack{Details\\LC} & Cite & Year & \shortstack{Conference\\/Journal\\/School} & Pages & Relevance &\shortstack{Cites\\OC XR\\SC} & \shortstack{Refs\\OC\\XR} & \shortstack{Links\\Cites\\Refs}\\ \midrule\endhead
\bottomrule
\endfoot
CzerniachowskaWZ23 \href{https://doi.org/10.12913/22998624/166588}{CzerniachowskaWZ23} & \hyperref[auth:a732]{K. Czerniachowska}, \hyperref[auth:a733]{R. Wichniarek}, \hyperref[auth:a734]{K. Żywicki} & \cellcolor{gold!20}Constraint Programming for Flexible Flow Shop Scheduling Problem with Repeated Jobs and Repeated Operations \hyperref[abs:CzerniachowskaWZ23]{Abstract} & \hyperref[detail:CzerniachowskaWZ23]{Details} \href{../works/CzerniachowskaWZ23.pdf}{Yes} & \cite{CzerniachowskaWZ23} & 2023 & Advances in Science and Technology Research Journal & 14 & \noindent{}\textbf{2.00} \textbf{2.00} \textbf{8.41} & 0 0 0 & 0 0 & 0 0 0\\
\end{longtable}
}

\subsection{Algorithms}

\index{Algorithms}
{\scriptsize
\begin{longtable}{>{\raggedright\arraybackslash}p{2.5cm}>{\raggedright\arraybackslash}p{4.5cm}>{\raggedright\arraybackslash}p{6.0cm}p{1.0cm}rr>{\raggedright\arraybackslash}p{2.0cm}r>{\raggedright\arraybackslash}p{1cm}p{1cm}p{1cm}p{1cm}}
\rowcolor{white}\caption{Articles in Journal Algorithms (Total 4)}\\ \toprule
\rowcolor{white}\shortstack{Key\\Source} & Authors & Title (Colored by Open Access)& \shortstack{Details\\LC} & Cite & Year & \shortstack{Conference\\/Journal\\/School} & Pages & Relevance &\shortstack{Cites\\OC XR\\SC} & \shortstack{Refs\\OC\\XR} & \shortstack{Links\\Cites\\Refs}\\ \midrule\endhead
\bottomrule
\endfoot
MontemanniD23 \href{https://doi.org/10.3390/a16010040}{MontemanniD23} & \hyperref[auth:a410]{R. Montemanni}, \hyperref[auth:a411]{M. Dell'Amico} & \cellcolor{gold!20}Solving the Parallel Drone Scheduling Traveling Salesman Problem via Constraint Programming & \hyperref[detail:MontemanniD23]{Details} \href{../works/MontemanniD23.pdf}{Yes} & \cite{MontemanniD23} & 2023 & Algorithms & 13 & \noindent{}\textbf{1.00} \textbf{1.00} \textbf{1.09} & 2 7 8 & 18 21 & 1 1 0\\
Gembarski2022 \href{http://dx.doi.org/10.3390/a15090318}{Gembarski2022} & \hyperref[auth:a1991]{P. C. Gembarski} & \cellcolor{gold!20}Joining Constraint Satisfaction Problems and Configurable CAD Product Models: A Step-by-Step Implementation Guide \hyperref[abs:Gembarski2022]{Abstract} & \hyperref[detail:Gembarski2022]{Details} No & \cite{Gembarski2022} & 2022 & Algorithms & null & \noindent{}\textcolor{black!50}{0.00} \textbf{2.00} n/a & 1 2 3 & 19 32 & 2 0 2\\
Valouxis2022 \href{http://dx.doi.org/10.3390/a15120450}{Valouxis2022} & \hyperref[auth:a1507]{C. Valouxis}, \hyperref[auth:a1508]{C. Gogos}, \hyperref[auth:a1509]{A. Dimitsas}, \hyperref[auth:a1510]{P. Potikas}, \hyperref[auth:a1511]{A. Vittas} & \cellcolor{gold!20}A Hybrid Exact–Local Search Approach for One-Machine Scheduling with Time-Dependent Capacity \hyperref[abs:Valouxis2022]{Abstract} & \hyperref[detail:Valouxis2022]{Details} No & \cite{Valouxis2022} & 2022 & Algorithms & null & \noindent{}\textcolor{black!50}{0.00} \textbf{3.50} n/a & 2 3 3 & 24 28 & 3 0 3\\
Kizilay2019 \href{http://dx.doi.org/10.3390/a12050100}{Kizilay2019} & \hyperref[auth:a1380]{D. Kizilay}, \hyperref[auth:a1973]{M. F. Tasgetiren}, \hyperref[auth:a1974]{Q.-K. Pan}, \hyperref[auth:a1975]{L. Gao} & \cellcolor{gold!20}A Variable Block Insertion Heuristic for Solving Permutation Flow Shop Scheduling Problem with Makespan Criterion \hyperref[abs:Kizilay2019]{Abstract} & \hyperref[detail:Kizilay2019]{Details} No & \cite{Kizilay2019} & 2019 & Algorithms & null & \noindent{}\textcolor{black!50}{0.00} \textbf{5.00} n/a & 20 21 22 & 36 42 & 2 0 2\\
\end{longtable}
}

\subsection{Analytics}

\index{Analytics}
{\scriptsize
\begin{longtable}{>{\raggedright\arraybackslash}p{2.5cm}>{\raggedright\arraybackslash}p{4.5cm}>{\raggedright\arraybackslash}p{6.0cm}p{1.0cm}rr>{\raggedright\arraybackslash}p{2.0cm}r>{\raggedright\arraybackslash}p{1cm}p{1cm}p{1cm}p{1cm}}
\rowcolor{white}\caption{Articles in Journal Analytics (Total 1)}\\ \toprule
\rowcolor{white}\shortstack{Key\\Source} & Authors & Title (Colored by Open Access)& \shortstack{Details\\LC} & Cite & Year & \shortstack{Conference\\/Journal\\/School} & Pages & Relevance &\shortstack{Cites\\OC XR\\SC} & \shortstack{Refs\\OC\\XR} & \shortstack{Links\\Cites\\Refs}\\ \midrule\endhead
\bottomrule
\endfoot
Lyons2023 \href{http://dx.doi.org/10.3390/analytics2030036}{Lyons2023} & \hyperref[auth:a1524]{J. S. F. Lyons}, \hyperref[auth:a836]{M. A. Begen}, \hyperref[auth:a1525]{P. C. Bell} & Surgery Scheduling and Perioperative Care: Smoothing and Visualizing Elective Surgery and Recovery Patient Flow \hyperref[abs:Lyons2023]{Abstract} & \hyperref[detail:Lyons2023]{Details} No & \cite{Lyons2023} & 2023 & Analytics & null & \noindent{}\textcolor{black!50}{0.00} \textbf{3.00} n/a & 0 0 0 & 23 29 & 4 0 4\\
\end{longtable}
}

\subsection{Annals of Mathematics and Artificial Intelligence}

\index{Annals of Mathematics and Artificial Intelligence}
{\scriptsize
\begin{longtable}{>{\raggedright\arraybackslash}p{2.5cm}>{\raggedright\arraybackslash}p{4.5cm}>{\raggedright\arraybackslash}p{6.0cm}p{1.0cm}rr>{\raggedright\arraybackslash}p{2.0cm}r>{\raggedright\arraybackslash}p{1cm}p{1cm}p{1cm}p{1cm}}
\rowcolor{white}\caption{Articles in Journal Annals of Mathematics and Artificial Intelligence (Total 1)}\\ \toprule
\rowcolor{white}\shortstack{Key\\Source} & Authors & Title (Colored by Open Access)& \shortstack{Details\\LC} & Cite & Year & \shortstack{Conference\\/Journal\\/School} & Pages & Relevance &\shortstack{Cites\\OC XR\\SC} & \shortstack{Refs\\OC\\XR} & \shortstack{Links\\Cites\\Refs}\\ \midrule\endhead
\bottomrule
\endfoot
Doolaard2022 \href{http://dx.doi.org/10.1007/s10472-022-09816-z}{Doolaard2022} & \hyperref[auth:a1900]{F. Doolaard}, \hyperref[auth:a19]{N. Yorke-Smith} & \cellcolor{gold!20}Online learning of variable ordering heuristics for constraint optimisation problems \hyperref[abs:Doolaard2022]{Abstract} & \hyperref[detail:Doolaard2022]{Details} No & \cite{Doolaard2022} & 2022 & Annals of Mathematics and Artificial Intelligence & null & \noindent{}0.50 0.50 n/a & 0 0 1 & 17 29 & 7 0 7\\
\end{longtable}
}

\subsection{Annals of Operations Research}

\index{Annals of Operations Research}
{\scriptsize
\begin{longtable}{>{\raggedright\arraybackslash}p{2.5cm}>{\raggedright\arraybackslash}p{4.5cm}>{\raggedright\arraybackslash}p{6.0cm}p{1.0cm}rr>{\raggedright\arraybackslash}p{2.0cm}r>{\raggedright\arraybackslash}p{1cm}p{1cm}p{1cm}p{1cm}}
\rowcolor{white}\caption{Articles in Journal Annals of Operations Research (Total 22)}\\ \toprule
\rowcolor{white}\shortstack{Key\\Source} & Authors & Title (Colored by Open Access)& \shortstack{Details\\LC} & Cite & Year & \shortstack{Conference\\/Journal\\/School} & Pages & Relevance &\shortstack{Cites\\OC XR\\SC} & \shortstack{Refs\\OC\\XR} & \shortstack{Links\\Cites\\Refs}\\ \midrule\endhead
\bottomrule
\endfoot
GokPTGO23 \href{https://ideas.repec.org/a/spr/annopr/v320y2023i2d10.1007_s10479-022-04547-0.html}{GokPTGO23} & \hyperref[auth:a1009]{Y. S. G{\"{o}}k}, \hyperref[auth:a1010]{S. Padr{\'{o}}n}, \hyperref[auth:a1011]{M. Tomasella}, \hyperref[auth:a1012]{D. Guimarans}, \hyperref[auth:a135]{C. {\"{O}}zt{\"{u}}rk} & {Constraint-based robust planning and scheduling of airport apron operations through simheuristics} & \hyperref[detail:GokPTGO23]{Details} \href{../works/GokPTGO23.pdf}{Yes} & \cite{GokPTGO23} & 2023 & Annals of Operations Research & 36 & \noindent{}\textcolor{black!50}{0.00} \textcolor{black!50}{0.00} \textbf{5.61} & 0 0 0 & 0 0 & 0 0 0\\
EmdeZD22 \href{http://dx.doi.org/10.1007/s10479-022-04891-1}{EmdeZD22} & \hyperref[auth:a956]{S. Emde}, \hyperref[auth:a957]{S. Zehtabian}, \hyperref[auth:a958]{Y. Disser} & Point-to-point and milk run delivery scheduling: models, complexity results, and algorithms based on Benders decomposition & \hyperref[detail:EmdeZD22]{Details} \href{../works/EmdeZD22.pdf}{Yes} & \cite{EmdeZD22} & 2022 & Annals of Operations Research & 30 & \noindent{}\textcolor{black!50}{0.00} \textcolor{black!50}{0.00} \textbf{3.40} & 0 0 0 & 52 59 & 11 0 11\\
Mischek2021a \href{http://dx.doi.org/10.1007/s10479-021-04007-1}{Mischek2021a} & \hyperref[auth:a80]{F. Mischek}, \hyperref[auth:a45]{N. Musliu} & \cellcolor{gold!20}A local search framework for industrial test laboratory scheduling \hyperref[abs:Mischek2021a]{Abstract} & \hyperref[detail:Mischek2021a]{Details} No & \cite{Mischek2021a} & 2021 & Annals of Operations Research & null & \noindent{}\textcolor{black!50}{0.00} \textbf{2.50} n/a & 3 5 5 & 28 35 & 10 2 8\\
Talbi2015 \href{http://dx.doi.org/10.1007/s10479-015-2034-y}{Talbi2015} & \hyperref[auth:a1659]{E.-G. Talbi} & Combining metaheuristics with mathematical programming, constraint programming and machine learning & \hyperref[detail:Talbi2015]{Details} No & \cite{Talbi2015} & 2015 & Annals of Operations Research & null & \noindent{}0.50 0.50 n/a & 84 90 87 & 88 151 & 11 2 9\\
Han2014 \href{http://dx.doi.org/10.1007/s10479-014-1619-1}{Han2014} & \hyperref[auth:a1664]{A. F. Han}, \hyperref[auth:a1665]{E. C. Li} & A constraint programming-based approach to the crew scheduling problem of the Taipei mass rapid transit system & \hyperref[detail:Han2014]{Details} No & \cite{Han2014} & 2014 & Annals of Operations Research & null & \noindent{}\textbf{1.00} \textbf{1.00} n/a & 13 15 15 & 21 28 & 9 2 7\\
GuyonLPR12 \href{http://dx.doi.org/10.1007/s10479-012-1132-3}{GuyonLPR12} & \hyperref[auth:a977]{O. Guyon}, \hyperref[auth:a978]{P. Lemaire}, \hyperref[auth:a846]{E. Pinson}, \hyperref[auth:a979]{D. Rivreau} & \cellcolor{green!10}Solving an integrated job-shop problem with human resource constraints & \hyperref[detail:GuyonLPR12]{Details} \href{../works/GuyonLPR12.pdf}{Yes} & \cite{GuyonLPR12} & 2012 & Annals of Operations Research & 25 & \noindent{}\textcolor{black!50}{0.00} \textcolor{black!50}{0.00} \textbf{1.65} & 32 33 40 & 25 38 & 14 6 8\\
MenciaSV12 \href{http://dx.doi.org/10.1007/s10479-012-1296-x}{MenciaSV12} & \hyperref[auth:a918]{C. Mencía}, \hyperref[auth:a919]{M. R. Sierra}, \hyperref[auth:a920]{R. Varela} & Depth-first heuristic search for the job shop scheduling problem & \hyperref[detail:MenciaSV12]{Details} \href{../works/MenciaSV12.pdf}{Yes} & \cite{MenciaSV12} & 2012 & Annals of Operations Research & 32 & \noindent{}\textcolor{black!50}{0.00} \textcolor{black!50}{0.00} \textbf{3.53} & 16 17 18 & 40 57 & 13 0 13\\
Pinto2012 \href{http://dx.doi.org/10.1007/s10479-012-1256-5}{Pinto2012} & \hyperref[auth:a1598]{G. Pinto}, \hyperref[auth:a1599]{Y. T. Ben-Dov}, \hyperref[auth:a1600]{G. Rabinowitz} & Formulating and solving a multi-mode resource-collaboration and constrained scheduling problem (MRCCSP) & \hyperref[detail:Pinto2012]{Details} No & \cite{Pinto2012} & 2012 & Annals of Operations Research & null & \noindent{}\textcolor{black!50}{0.00} \textcolor{black!50}{0.00} n/a & 5 6 8 & 41 50 & 6 1 5\\
BeldiceanuCDP11 \href{https://doi.org/10.1007/s10479-010-0731-0}{BeldiceanuCDP11} & \hyperref[auth:a128]{N. Beldiceanu}, \hyperref[auth:a91]{M. Carlsson}, \hyperref[auth:a243]{S. Demassey}, \hyperref[auth:a358]{E. Poder} & New filtering for the \emph{cumulative} constraint in the context of non-overlapping rectangles & \hyperref[detail:BeldiceanuCDP11]{Details} \href{../works/BeldiceanuCDP11.pdf}{Yes} & \cite{BeldiceanuCDP11} & 2011 & Annals of Operations Research & 24 & \noindent{}\textcolor{black!50}{0.00} \textcolor{black!50}{0.00} \textbf{2.13} & 8 8 9 & 8 17 & 6 2 4\\
BeniniLMR11 \href{https://doi.org/10.1007/s10479-010-0718-x}{BeniniLMR11} & \hyperref[auth:a245]{L. Benini}, \hyperref[auth:a142]{M. Lombardi}, \hyperref[auth:a143]{M. Milano}, \hyperref[auth:a718]{M. Ruggiero} & Optimal resource allocation and scheduling for the {CELL} {BE} platform & \hyperref[detail:BeniniLMR11]{Details} \href{../works/BeniniLMR11.pdf}{Yes} & \cite{BeniniLMR11} & 2011 & Annals of Operations Research & 27 & \noindent{}\textcolor{black!50}{0.00} \textcolor{black!50}{0.00} \textbf{15.38} & 18 17 17 & 16 29 & 15 4 11\\
CobanH11 \href{http://dx.doi.org/10.1007/s10479-011-1031-z}{CobanH11} & \hyperref[auth:a335]{E. Coban}, \hyperref[auth:a160]{J. N. Hooker} & \cellcolor{green!10}Single-facility scheduling by logic-based Benders decomposition & \hyperref[detail:CobanH11]{Details} \href{../works/CobanH11.pdf}{Yes} & \cite{CobanH11} & 2011 & Annals of Operations Research & 28 & \noindent{}\textcolor{black!50}{0.00} \textcolor{black!50}{0.00} \textbf{17.60} & 14 15 17 & 37 44 & 29 8 21\\
HachemiGR11 \href{https://doi.org/10.1007/s10479-010-0698-x}{HachemiGR11} & \hyperref[auth:a615]{N. E. Hachemi}, \hyperref[auth:a616]{M. Gendreau}, \hyperref[auth:a326]{L.-M. Rousseau} & A hybrid constraint programming approach to the log-truck scheduling problem & \hyperref[detail:HachemiGR11]{Details} \href{../works/HachemiGR11.pdf}{Yes} & \cite{HachemiGR11} & 2011 & Annals of Operations Research & 16 & \noindent{}\textbf{1.00} \textbf{1.00} \textbf{2.82} & 32 34 38 & 19 30 & 8 3 5\\
BartakCS10 \href{https://doi.org/10.1007/s10479-008-0492-1}{BartakCS10} & \hyperref[auth:a152]{R. Bart{\'{a}}k}, \hyperref[auth:a161]{O. Cepek}, \hyperref[auth:a780]{P. Surynek} & Discovering implied constraints in precedence graphs with alternatives & \hyperref[detail:BartakCS10]{Details} \href{../works/BartakCS10.pdf}{Yes} & \cite{BartakCS10} & 2010 & Annals of Operations Research & 31 & \noindent{}\textcolor{black!50}{0.00} \textcolor{black!50}{0.00} 0.91 & 2 2 2 & 9 21 & 3 1 2\\
MilanoW09 \href{http://dx.doi.org/10.1007/s10479-009-0654-9}{MilanoW09} & \hyperref[auth:a143]{M. Milano}, \hyperref[auth:a117]{M. G. Wallace} & Integrating Operations Research in Constraint Programming & \hyperref[detail:MilanoW09]{Details} \href{../works/MilanoW09.pdf}{Yes} & \cite{MilanoW09} & 2009 & Annals of Operations Research & 40 & \noindent{}\textcolor{black!50}{0.00} \textcolor{black!50}{0.00} \textbf{44.83} & 34 35 41 & 46 77 & 25 6 19\\
RasmussenT09 \href{http://dx.doi.org/10.1007/s10479-008-0384-4}{RasmussenT09} & \hyperref[auth:a1403]{R. V. Rasmussen}, \hyperref[auth:a1389]{M. A. Trick} & \cellcolor{green!10}The timetable constrained distance minimization problem \hyperref[abs:RasmussenT09]{Abstract} & \hyperref[detail:RasmussenT09]{Details} \href{../works/RasmussenT09.pdf}{Yes} & \cite{RasmussenT09} & 2009 & Annals of Operations Research & 15 & \noindent{}\textcolor{black!50}{0.00} \textbf{1.00} \textbf{1.89} & 8 9 9 & 15 25 & 7 1 6\\
LiessM08 \href{https://doi.org/10.1007/s10479-007-0188-y}{LiessM08} & \hyperref[auth:a639]{O. Liess}, \hyperref[auth:a355]{P. Michelon} & A constraint programming approach for the resource-constrained project scheduling problem & \hyperref[detail:LiessM08]{Details} \href{../works/LiessM08.pdf}{Yes} & \cite{LiessM08} & 2008 & Annals of Operations Research & 12 & \noindent{}\textbf{1.50} \textbf{1.50} \textbf{2.13} & 22 25 28 & 14 17 & 17 9 8\\
ArtiguesF07 \href{http://dx.doi.org/10.1007/s10479-007-0283-0}{ArtiguesF07} & \hyperref[auth:a6]{C. Artigues}, \hyperref[auth:a356]{D. Feillet} & \cellcolor{green!10}A branch and bound method for the job-shop problem with sequence-dependent setup times & \hyperref[detail:ArtiguesF07]{Details} \href{../works/ArtiguesF07.pdf}{Yes} & \cite{ArtiguesF07} & 2007 & Annals of Operations Research & 25 & \noindent{}\textcolor{black!50}{0.00} \textcolor{black!50}{0.00} \textbf{6.23} & 49 49 66 & 32 46 & 19 11 8\\
BeckR03 \href{https://doi.org/10.1023/A:1021849405707}{BeckR03} & \hyperref[auth:a89]{J. C. Beck}, \hyperref[auth:a254]{P. Refalo} & A Hybrid Approach to Scheduling with Earliness and Tardiness Costs & \hyperref[detail:BeckR03]{Details} \href{../works/BeckR03.pdf}{Yes} & \cite{BeckR03} & 2003 & Annals of Operations Research & 23 & \noindent{}\textcolor{black!50}{0.00} \textcolor{black!50}{0.00} \textbf{15.68} & 29 0 45 & 0 0 & 10 10 0\\
MartinPY01 \href{https://doi.org/10.1023/A:1016067230126}{MartinPY01} & \hyperref[auth:a676]{F. Martin}, \hyperref[auth:a677]{A. Pinkney}, \hyperref[auth:a678]{X. Yu} & Cane Railway Scheduling via Constraint Logic Programming: Labelling Order and Constraints in a Real-Life Application & \hyperref[detail:MartinPY01]{Details} \href{../works/MartinPY01.pdf}{Yes} & \cite{MartinPY01} & 2001 & Annals of Operations Research & 17 & \noindent{}\textbf{1.50} \textbf{1.50} \textbf{5.69} & 11 0 14 & 0 0 & 0 0 0\\
Mason01 \href{https://doi.org/10.1023/A:1016023415105}{Mason01} & \hyperref[auth:a679]{A. J. Mason} & Elastic Constraint Branching, the Wedelin/Carmen Lagrangian Heuristic and Integer Programming for Personnel Scheduling & \hyperref[detail:Mason01]{Details} \href{../works/Mason01.pdf}{Yes} & \cite{Mason01} & 2001 & Annals of Operations Research & 38 & \noindent{}\textcolor{black!50}{0.00} \textcolor{black!50}{0.00} \textcolor{black!50}{0.00} & 5 0 6 & 0 0 & 0 0 0\\
BaptistePN99 \href{http://dx.doi.org/10.1023/a:1018995000688}{BaptistePN99} & \hyperref[auth:a162]{P. Baptiste}, \hyperref[auth:a163]{C. L. Pape}, \hyperref[auth:a656]{W. Nuijten} & Satisfiability tests and time-bound adjustments for cumulative scheduling problems & \hyperref[detail:BaptistePN99]{Details} \href{../works/BaptistePN99.pdf}{Yes} & \cite{BaptistePN99} & 1999 & Annals of Operations Research & 29 & \noindent{}\textcolor{black!50}{0.00} \textcolor{black!50}{0.00} \textbf{1.74} & 72 0 85 & 0 0 & 19 19 0\\
RodosekWH99 \href{http://dx.doi.org/10.1023/a:1018904229454}{RodosekWH99} & \hyperref[auth:a297]{R. Rodosek}, \hyperref[auth:a117]{M. G. Wallace}, \hyperref[auth:a1030]{M. Hajian} & A new approach to integrating mixed integer programming and constraint logic programming & \hyperref[detail:RodosekWH99]{Details} \href{../works/RodosekWH99.pdf}{Yes} & \cite{RodosekWH99} & 1999 & Annals of Operations Research & 25 & \noindent{}\textcolor{black!50}{0.00} \textcolor{black!50}{0.00} \textbf{8.11} & 53 0 67 & 0 0 & 23 23 0\\
\end{longtable}
}

\subsection{Annual Review of Chemical and Biomolecular Engineering}

\index{Annual Review of Chemical and Biomolecular Engineering}
{\scriptsize
\begin{longtable}{>{\raggedright\arraybackslash}p{2.5cm}>{\raggedright\arraybackslash}p{4.5cm}>{\raggedright\arraybackslash}p{6.0cm}p{1.0cm}rr>{\raggedright\arraybackslash}p{2.0cm}r>{\raggedright\arraybackslash}p{1cm}p{1cm}p{1cm}p{1cm}}
\rowcolor{white}\caption{Articles in Journal Annual Review of Chemical and Biomolecular Engineering (Total 1)}\\ \toprule
\rowcolor{white}\shortstack{Key\\Source} & Authors & Title (Colored by Open Access)& \shortstack{Details\\LC} & Cite & Year & \shortstack{Conference\\/Journal\\/School} & Pages & Relevance &\shortstack{Cites\\OC XR\\SC} & \shortstack{Refs\\OC\\XR} & \shortstack{Links\\Cites\\Refs}\\ \midrule\endhead
\bottomrule
\endfoot
Velez2014 \href{http://dx.doi.org/10.1146/annurev-chembioeng-060713-035859}{Velez2014} & \hyperref[auth:a1480]{S. Velez}, \hyperref[auth:a381]{C. T. Maravelias} & \cellcolor{gold!20}Advances in Mixed-Integer Programming Methods for Chemical Production Scheduling \hyperref[abs:Velez2014]{Abstract} & \hyperref[detail:Velez2014]{Details} No & \cite{Velez2014} & 2014 & Annual Review of Chemical and Biomolecular Engineering & null & \noindent{}\textcolor{black!50}{0.00} \textbf{1.50} n/a & 28 30 25 & 112 121 & 12 1 11\\
\end{longtable}
}

\subsection{Applied Mathematical Modelling}

\index{Applied Mathematical Modelling}
{\scriptsize
\begin{longtable}{>{\raggedright\arraybackslash}p{2.5cm}>{\raggedright\arraybackslash}p{4.5cm}>{\raggedright\arraybackslash}p{6.0cm}p{1.0cm}rr>{\raggedright\arraybackslash}p{2.0cm}r>{\raggedright\arraybackslash}p{1cm}p{1cm}p{1cm}p{1cm}}
\rowcolor{white}\caption{Articles in Journal Applied Mathematical Modelling (Total 1)}\\ \toprule
\rowcolor{white}\shortstack{Key\\Source} & Authors & Title (Colored by Open Access)& \shortstack{Details\\LC} & Cite & Year & \shortstack{Conference\\/Journal\\/School} & Pages & Relevance &\shortstack{Cites\\OC XR\\SC} & \shortstack{Refs\\OC\\XR} & \shortstack{Links\\Cites\\Refs}\\ \midrule\endhead
\bottomrule
\endfoot
Sahraeian2015 \href{http://dx.doi.org/10.1016/j.apm.2014.07.032}{Sahraeian2015} & \hyperref[auth:a1863]{R. Sahraeian}, \hyperref[auth:a1864]{M. Namakshenas} & \cellcolor{gold!20}On the optimal modeling and evaluation of job shops with a total weighted tardiness objective: Constraint programming vs. mixed integer programming & \hyperref[detail:Sahraeian2015]{Details} No & \cite{Sahraeian2015} & 2015 & Applied Mathematical Modelling & null & \noindent{}\textbf{1.00} \textbf{1.00} n/a & 2 3 8 & 11 18 & 1 1 0\\
\end{longtable}
}

\subsection{Applied Mechanics and Materials}

\index{Applied Mechanics and Materials}
{\scriptsize
\begin{longtable}{>{\raggedright\arraybackslash}p{2.5cm}>{\raggedright\arraybackslash}p{4.5cm}>{\raggedright\arraybackslash}p{6.0cm}p{1.0cm}rr>{\raggedright\arraybackslash}p{2.0cm}r>{\raggedright\arraybackslash}p{1cm}p{1cm}p{1cm}p{1cm}}
\rowcolor{white}\caption{Articles in Journal Applied Mechanics and Materials (Total 3)}\\ \toprule
\rowcolor{white}\shortstack{Key\\Source} & Authors & Title (Colored by Open Access)& \shortstack{Details\\LC} & Cite & Year & \shortstack{Conference\\/Journal\\/School} & Pages & Relevance &\shortstack{Cites\\OC XR\\SC} & \shortstack{Refs\\OC\\XR} & \shortstack{Links\\Cites\\Refs}\\ \midrule\endhead
\bottomrule
\endfoot
Bzdyra2015 \href{http://dx.doi.org/10.4028/www.scientific.net/amm.791.70}{Bzdyra2015} & \hyperref[auth:a1813]{K. Bzdyra}, \hyperref[auth:a630]{G. Bocewicz}, \hyperref[auth:a1814]{Z. Banaszak} & Mass Customized Projects Portfolio Scheduling - Imprecise Operations Time Approach \hyperref[abs:Bzdyra2015]{Abstract} & \hyperref[detail:Bzdyra2015]{Details} No & \cite{Bzdyra2015} & 2015 & Applied Mechanics and Materials & null & \noindent{}\textcolor{black!50}{0.00} \textbf{2.50} n/a & 5 5 0 & 10 14 & 2 1 1\\
Li2014 \href{http://dx.doi.org/10.4028/www.scientific.net/amm.681.265}{Li2014} & \hyperref[auth:a1492]{Y. Li}, \hyperref[auth:a1493]{Z. R. Xiao} & A Constraint-Based Approach for Multi-Skilled Project Scheduling Problem \hyperref[abs:Li2014]{Abstract} & \hyperref[detail:Li2014]{Details} No & \cite{Li2014} & 2014 & Applied Mechanics and Materials & null & \noindent{}\textcolor{black!50}{0.00} \textbf{5.00} n/a & 0 2 2 & 4 5 & 4 0 4\\
Wang2013 \href{http://dx.doi.org/10.4028/www.scientific.net/amm.357-360.2720}{Wang2013} & \hyperref[auth:a1903]{H. S. Wang}, \hyperref[auth:a1904]{S. S. Liu} & Road Inspection Scheduling Model Using Constraint Programming \hyperref[abs:Wang2013]{Abstract} & \hyperref[detail:Wang2013]{Details} No & \cite{Wang2013} & 2013 & Applied Mechanics and Materials & null & \noindent{}\textbf{1.00} \textbf{3.00} n/a & 0 0 0 & 6 7 & 2 0 2\\
\end{longtable}
}

\subsection{Applied Sciences}

\index{Applied Sciences}
{\scriptsize
\begin{longtable}{>{\raggedright\arraybackslash}p{2.5cm}>{\raggedright\arraybackslash}p{4.5cm}>{\raggedright\arraybackslash}p{6.0cm}p{1.0cm}rr>{\raggedright\arraybackslash}p{2.0cm}r>{\raggedright\arraybackslash}p{1cm}p{1cm}p{1cm}p{1cm}}
\rowcolor{white}\caption{Articles in Journal Applied Sciences (Total 11)}\\ \toprule
\rowcolor{white}\shortstack{Key\\Source} & Authors & Title (Colored by Open Access)& \shortstack{Details\\LC} & Cite & Year & \shortstack{Conference\\/Journal\\/School} & Pages & Relevance &\shortstack{Cites\\OC XR\\SC} & \shortstack{Refs\\OC\\XR} & \shortstack{Links\\Cites\\Refs}\\ \midrule\endhead
\bottomrule
\endfoot
Bocewicz2023 \href{http://dx.doi.org/10.3390/app13127165}{Bocewicz2023} & \hyperref[auth:a630]{G. Bocewicz}, \hyperref[auth:a1997]{E. Szwarc}, \hyperref[auth:a2016]{A. Thibbotuwawa}, \hyperref[auth:a1814]{Z. Banaszak} & \cellcolor{gold!20}Project Portfolio Planning Taking into Account the Effect of Loss of Competences of Project Team Members \hyperref[abs:Bocewicz2023]{Abstract} & \hyperref[detail:Bocewicz2023]{Details} No & \cite{Bocewicz2023} & 2023 & Applied Sciences & null & \noindent{}\textcolor{black!50}{0.00} \textbf{1.50} n/a & 0 0 0 & 41 47 & 1 0 1\\
Oujana2023 \href{http://dx.doi.org/10.3390/app13106003}{Oujana2023} & \hyperref[auth:a453]{S. Oujana}, \hyperref[auth:a454]{L. Amodeo}, \hyperref[auth:a455]{F. Yalaoui}, \hyperref[auth:a1477]{D. Brodart} & \cellcolor{gold!20}Mixed-Integer Linear Programming, Constraint Programming and a Novel Dedicated Heuristic for Production Scheduling in a Packaging Plant \hyperref[abs:Oujana2023]{Abstract} & \hyperref[detail:Oujana2023]{Details} No & \cite{Oujana2023} & 2023 & Applied Sciences & null & \noindent{}\textbf{1.00} \textbf{7.01} n/a & 3 3 3 & 55 57 & 4 1 3\\
Schweitzer2023 \href{http://dx.doi.org/10.3390/app13020806}{Schweitzer2023} & \hyperref[auth:a1592]{F. Schweitzer}, \hyperref[auth:a1593]{G. Bitsch}, \hyperref[auth:a1594]{L. Louw} & \cellcolor{gold!20}Choosing Solution Strategies for Scheduling Automated Guided Vehicles in Production Using Machine Learning \hyperref[abs:Schweitzer2023]{Abstract} & \hyperref[detail:Schweitzer2023]{Details} No & \cite{Schweitzer2023} & 2023 & Applied Sciences & null & \noindent{}\textcolor{black!50}{0.00} \textbf{1.50} n/a & 2 5 5 & 49 68 & 7 0 7\\
Tayyab2023 \href{http://dx.doi.org/10.3390/app13063616}{Tayyab2023} & \hyperref[auth:a1640]{A. Tayyab}, \hyperref[auth:a1641]{S. Ullah}, \hyperref[auth:a1642]{T. Mahmood}, \hyperref[auth:a1643]{Y. Y. Ghadi}, \hyperref[auth:a1644]{B. Latif}, \hyperref[auth:a1645]{H. Aljuaid} & \cellcolor{gold!20}Modeling of Multi-Level Planning of Shifting Bottleneck Resources Integrated with Downstream Wards in a Hospital \hyperref[abs:Tayyab2023]{Abstract} & \hyperref[detail:Tayyab2023]{Details} No & \cite{Tayyab2023} & 2023 & Applied Sciences & null & \noindent{}\textcolor{black!50}{0.00} \textcolor{black!50}{0.00} n/a & 1 0 1 & 53 57 & 3 0 3\\
Feng2022 \href{http://dx.doi.org/10.3390/app12189062}{Feng2022} & \hyperref[auth:a1738]{C. Feng}, \hyperref[auth:a1739]{S. Hu}, \hyperref[auth:a1740]{Y. Ma}, \hyperref[auth:a1741]{Z. Li} & \cellcolor{gold!20}A Project Scheduling Game Equilibrium Problem Based on Dynamic Resource Supply \hyperref[abs:Feng2022]{Abstract} & \hyperref[detail:Feng2022]{Details} No & \cite{Feng2022} & 2022 & Applied Sciences & null & \noindent{}\textcolor{black!50}{0.00} \textcolor{black!50}{0.00} n/a & 1 1 2 & 27 36 & 3 0 3\\
Relich2022 \href{http://dx.doi.org/10.3390/app12041921}{Relich2022} & \hyperref[auth:a1646]{M. Relich}, \hyperref[auth:a1705]{I. Nielsen}, \hyperref[auth:a1815]{A. Gola} & \cellcolor{gold!20}Reducing the Total Product Cost at the Product Design Stage \hyperref[abs:Relich2022]{Abstract} & \hyperref[detail:Relich2022]{Details} No & \cite{Relich2022} & 2022 & Applied Sciences & null & \noindent{}\textcolor{black!50}{0.00} \textbf{2.00} n/a & 7 12 13 & 42 55 & 2 0 2\\
Grzegorz2021 \href{http://dx.doi.org/10.3390/app11198898}{Grzegorz2021} & \hyperref[auth:a2062]{R. Grzegorz}, \hyperref[auth:a2063]{B. Grzegorz}, \hyperref[auth:a2064]{D. Bogdan}, \hyperref[auth:a2065]{B. Zbigniew} & \cellcolor{gold!20}Reactive Planning-Driven Approach to Online UAVs Mission Rerouting and Rescheduling \hyperref[abs:Grzegorz2021]{Abstract} & \hyperref[detail:Grzegorz2021]{Details} No & \cite{Grzegorz2021} & 2021 & Applied Sciences & null & \noindent{}\textcolor{black!50}{0.00} \textbf{1.50} n/a & 1 3 2 & 30 36 & 1 0 1\\
Liu2021 \href{http://dx.doi.org/10.3390/app11041447}{Liu2021} & \hyperref[auth:a1244]{S.-S. Liu}, \hyperref[auth:a1489]{M. F. A. Arifin}, \hyperref[auth:a1490]{W. T. Chen}, \hyperref[auth:a1491]{Y.-H. Huang} & \cellcolor{gold!20}Emergency Repair Scheduling Model for Road Network Integrating Rescheduling Feature \hyperref[abs:Liu2021]{Abstract} & \hyperref[detail:Liu2021]{Details} No & \cite{Liu2021} & 2021 & Applied Sciences & null & \noindent{}\textcolor{black!50}{0.00} \textbf{4.00} n/a & 2 2 2 & 52 53 & 6 0 6\\
Ortiz-Bayliss2021 \href{http://dx.doi.org/10.3390/app11062749}{Ortiz-Bayliss2021} & \hyperref[auth:a1603]{J. C. Ortiz-Bayliss}, \hyperref[auth:a1604]{I. Amaya}, \hyperref[auth:a1605]{J. M. Cruz-Duarte}, \hyperref[auth:a1606]{A. E. Gutierrez-Rodriguez}, \hyperref[auth:a1607]{S. E. Conant-Pablos}, \hyperref[auth:a1608]{H. Terashima-Marín} & \cellcolor{gold!20}A General Framework Based on Machine Learning for Algorithm Selection in Constraint Satisfaction Problems \hyperref[abs:Ortiz-Bayliss2021]{Abstract} & \hyperref[detail:Ortiz-Bayliss2021]{Details} No & \cite{Ortiz-Bayliss2021} & 2021 & Applied Sciences & null & \noindent{}0.50 \textbf{1.50} n/a & 2 4 4 & 37 59 & 7 0 7\\
Liu2020 \href{http://dx.doi.org/10.3390/app10248887}{Liu2020} & \hyperref[auth:a1244]{S.-S. Liu}, \hyperref[auth:a1494]{H.-Y. Huang}, \hyperref[auth:a1495]{N. R. D. Kumala} & \cellcolor{gold!20}Two-Stage Optimization Model for Life Cycle Maintenance Scheduling of Bridge Infrastructure \hyperref[abs:Liu2020]{Abstract} & \hyperref[detail:Liu2020]{Details} No & \cite{Liu2020} & 2020 & Applied Sciences & null & \noindent{}\textcolor{black!50}{0.00} \textbf{4.00} n/a & 3 3 3 & 54 60 & 7 2 5\\
Relich2020 \href{http://dx.doi.org/10.3390/app10186330}{Relich2020} & \hyperref[auth:a1646]{M. Relich}, \hyperref[auth:a1647]{A. Świć} & \cellcolor{gold!20}Parametric Estimation and Constraint Programming-Based Planning and Simulation of Production Cost of a New Product \hyperref[abs:Relich2020]{Abstract} & \hyperref[detail:Relich2020]{Details} No & \cite{Relich2020} & 2020 & Applied Sciences & null & \noindent{}\textcolor{black!50}{0.00} \textbf{1.00} n/a & 14 15 16 & 24 40 & 5 2 3\\
\end{longtable}
}

\subsection{Artif. Intell. Eng.}

\index{Artif. Intell. Eng.}
{\scriptsize
\begin{longtable}{>{\raggedright\arraybackslash}p{2.5cm}>{\raggedright\arraybackslash}p{4.5cm}>{\raggedright\arraybackslash}p{6.0cm}p{1.0cm}rr>{\raggedright\arraybackslash}p{2.0cm}r>{\raggedright\arraybackslash}p{1cm}p{1cm}p{1cm}p{1cm}}
\rowcolor{white}\caption{Articles in Journal Artif. Intell. Eng. (Total 1)}\\ \toprule
\rowcolor{white}\shortstack{Key\\Source} & Authors & Title (Colored by Open Access)& \shortstack{Details\\LC} & Cite & Year & \shortstack{Conference\\/Journal\\/School} & Pages & Relevance &\shortstack{Cites\\OC XR\\SC} & \shortstack{Refs\\OC\\XR} & \shortstack{Links\\Cites\\Refs}\\ \midrule\endhead
\bottomrule
\endfoot
LammaMM97 \href{https://doi.org/10.1016/S0954-1810(96)00002-7}{LammaMM97} & \hyperref[auth:a720]{E. Lamma}, \hyperref[auth:a721]{P. Mello}, \hyperref[auth:a143]{M. Milano} & A distributed constraint-based scheduler & \hyperref[detail:LammaMM97]{Details} \href{../works/LammaMM97.pdf}{Yes} & \cite{LammaMM97} & 1997 & Artif. Intell. Eng. & 15 & \noindent{}\textcolor{black!50}{0.00} \textcolor{black!50}{0.00} \textbf{9.45} & 11 11 13 & 7 28 & 3 2 1\\
\end{longtable}
}

\subsection{Artif. Intell. Rev.}

\index{Artif. Intell. Rev.}
{\scriptsize
\begin{longtable}{>{\raggedright\arraybackslash}p{2.5cm}>{\raggedright\arraybackslash}p{4.5cm}>{\raggedright\arraybackslash}p{6.0cm}p{1.0cm}rr>{\raggedright\arraybackslash}p{2.0cm}r>{\raggedright\arraybackslash}p{1cm}p{1cm}p{1cm}p{1cm}}
\rowcolor{white}\caption{Articles in Journal Artif. Intell. Rev. (Total 1)}\\ \toprule
\rowcolor{white}\shortstack{Key\\Source} & Authors & Title (Colored by Open Access)& \shortstack{Details\\LC} & Cite & Year & \shortstack{Conference\\/Journal\\/School} & Pages & Relevance &\shortstack{Cites\\OC XR\\SC} & \shortstack{Refs\\OC\\XR} & \shortstack{Links\\Cites\\Refs}\\ \midrule\endhead
\bottomrule
\endfoot
ZarandiASC20 \href{https://doi.org/10.1007/s10462-018-9667-6}{ZarandiASC20} & \hyperref[auth:a829]{M. H. F. Zarandi}, \hyperref[auth:a830]{A. A. S. Asl}, \hyperref[auth:a831]{S. Sotudian}, \hyperref[auth:a832]{O. Castillo} & A state of the art review of intelligent scheduling & \hyperref[detail:ZarandiASC20]{Details} \href{../works/ZarandiASC20.pdf}{Yes} & \cite{ZarandiASC20} & 2020 & Artif. Intell. Rev. & 93 & \noindent{}\textcolor{black!50}{0.00} \textcolor{black!50}{0.00} \textbf{440.67} & 55 64 66 & 445 538 & 66 3 63\\
\end{longtable}
}

\subsection{Artificial Intelligence}

\index{Artificial Intelligence}
{\scriptsize
\begin{longtable}{>{\raggedright\arraybackslash}p{2.5cm}>{\raggedright\arraybackslash}p{4.5cm}>{\raggedright\arraybackslash}p{6.0cm}p{1.0cm}rr>{\raggedright\arraybackslash}p{2.0cm}r>{\raggedright\arraybackslash}p{1cm}p{1cm}p{1cm}p{1cm}}
\rowcolor{white}\caption{Articles in Journal Artificial Intelligence (Total 13)}\\ \toprule
\rowcolor{white}\shortstack{Key\\Source} & Authors & Title (Colored by Open Access)& \shortstack{Details\\LC} & Cite & Year & \shortstack{Conference\\/Journal\\/School} & Pages & Relevance &\shortstack{Cites\\OC XR\\SC} & \shortstack{Refs\\OC\\XR} & \shortstack{Links\\Cites\\Refs}\\ \midrule\endhead
\bottomrule
\endfoot
BonfiettiLBM14 \href{https://doi.org/10.1016/j.artint.2013.09.006}{BonfiettiLBM14} & \hyperref[auth:a198]{A. Bonfietti}, \hyperref[auth:a142]{M. Lombardi}, \hyperref[auth:a245]{L. Benini}, \hyperref[auth:a143]{M. Milano} & \cellcolor{gold!20}{CROSS} cyclic resource-constrained scheduling solver & \hyperref[detail:BonfiettiLBM14]{Details} \href{../works/BonfiettiLBM14.pdf}{Yes} & \cite{BonfiettiLBM14} & 2014 & Artificial Intelligence & 28 & \noindent{}\textcolor{black!50}{0.00} \textcolor{black!50}{0.00} \textbf{5.15} & 8 9 8 & 15 35 & 4 3 1\\
LombardiM12a \href{https://doi.org/10.1016/j.artint.2011.12.001}{LombardiM12a} & \hyperref[auth:a142]{M. Lombardi}, \hyperref[auth:a143]{M. Milano} & \cellcolor{gold!20}A min-flow algorithm for Minimal Critical Set detection in Resource Constrained Project Scheduling & \hyperref[detail:LombardiM12a]{Details} \href{../works/LombardiM12a.pdf}{Yes} & \cite{LombardiM12a} & 2012 & Artificial Intelligence & 10 & \noindent{}\textcolor{black!50}{0.00} \textcolor{black!50}{0.00} 0.37 & 3 3 15 & 13 21 & 8 1 7\\
LombardiM10a \href{https://doi.org/10.1016/j.artint.2010.02.004}{LombardiM10a} & \hyperref[auth:a142]{M. Lombardi}, \hyperref[auth:a143]{M. Milano} & \cellcolor{gold!20}Allocation and scheduling of Conditional Task Graphs & \hyperref[detail:LombardiM10a]{Details} \href{../works/LombardiM10a.pdf}{Yes} & \cite{LombardiM10a} & 2010 & Artificial Intelligence & 30 & \noindent{}\textcolor{black!50}{0.00} \textcolor{black!50}{0.00} \textbf{10.65} & 8 8 13 & 24 41 & 11 2 9\\
Frisch2006 \href{http://dx.doi.org/10.1016/j.artint.2006.03.002}{Frisch2006} & \hyperref[auth:a1666]{A. M. Frisch}, \hyperref[auth:a137]{B. Hnich}, \hyperref[auth:a97]{Z. Kiziltan}, \hyperref[auth:a1667]{I. Miguel}, \hyperref[auth:a276]{T. Walsh} & \cellcolor{gold!20}Propagation algorithms for lexicographic ordering constraints & \hyperref[detail:Frisch2006]{Details} No & \cite{Frisch2006} & 2006 & Artificial Intelligence & null & \noindent{}0.25 0.25 n/a & 23 23 31 & 9 29 & 5 2 3\\
Laborie03 \href{http://dx.doi.org/10.1016/s0004-3702(02)00362-4}{Laborie03} & \hyperref[auth:a118]{P. Laborie} & \cellcolor{gold!20}Algorithms for propagating resource constraints in AI planning and scheduling: Existing approaches and new results & \hyperref[detail:Laborie03]{Details} \href{../works/Laborie03.pdf}{Yes} & \cite{Laborie03} & 2003 & Artificial Intelligence & 38 & \noindent{}\textcolor{black!50}{0.00} \textcolor{black!50}{0.00} \textbf{8.42} & 128 129 175 & 10 31 & 48 43 5\\
JussienL02 \href{http://dx.doi.org/10.1016/s0004-3702(02)00221-7}{JussienL02} & \hyperref[auth:a247]{N. Jussien}, \hyperref[auth:a1072]{O. Lhomme} & \cellcolor{gold!20}Local search with constraint propagation and conflict-based heuristics & \hyperref[detail:JussienL02]{Details} \href{../works/JussienL02.pdf}{Yes} & \cite{JussienL02} & 2002 & Artificial Intelligence & 25 & \noindent{}\textcolor{black!50}{0.00} \textcolor{black!50}{0.00} \textbf{4.25} & 88 88 108 & 16 54 & 15 8 7\\
BeckF00 \href{https://doi.org/10.1016/S0004-3702(99)00099-5}{BeckF00} & \hyperref[auth:a89]{J. C. Beck}, \hyperref[auth:a302]{M. S. Fox} & \cellcolor{gold!20}Dynamic problem structure analysis as a basis for constraint-directed scheduling heuristics & \hyperref[detail:BeckF00]{Details} \href{../works/BeckF00.pdf}{Yes} & \cite{BeckF00} & 2000 & Artificial Intelligence & 51 & \noindent{}\textcolor{black!50}{0.00} \textcolor{black!50}{0.00} \textbf{13.43} & 24 24 36 & 19 76 & 19 10 9\\
BeckF00a \href{http://dx.doi.org/10.1016/s0004-3702(00)00035-7}{BeckF00a} & \hyperref[auth:a89]{J. C. Beck}, \hyperref[auth:a302]{M. S. Fox} & \cellcolor{gold!20}Constraint-directed techniques for scheduling alternative activities & \hyperref[detail:BeckF00a]{Details} \href{../works/BeckF00a.pdf}{Yes} & \cite{BeckF00a} & 2000 & Artificial Intelligence & 40 & \noindent{}\textcolor{black!50}{0.00} \textcolor{black!50}{0.00} \textbf{17.71} & 48 48 60 & 10 44 & 14 9 5\\
Dorndorf2000 \href{http://dx.doi.org/10.1016/s0004-3702(00)00040-0}{Dorndorf2000} & \hyperref[auth:a904]{U. Dorndorf}, \hyperref[auth:a438]{E. Pesch}, \hyperref[auth:a1046]{T. Phan-Huy} & \cellcolor{gold!20}Constraint propagation techniques for the disjunctive scheduling problem & \hyperref[detail:Dorndorf2000]{Details} \href{../works/Dorndorf2000.pdf}{Yes} & \cite{Dorndorf2000} & 2000 & Artificial Intelligence & 52 & \noindent{}\textbf{1.50} \textbf{1.50} \textbf{18.08} & 47 47 51 & 33 62 & 25 15 10\\
SadehF96 \href{http://dx.doi.org/10.1016/0004-3702(95)00098-4}{SadehF96} & \hyperref[auth:a1043]{N. M. Sadeh}, \hyperref[auth:a302]{M. S. Fox} & \cellcolor{gold!20}Variable and value ordering heuristics for the job shop scheduling constraint satisfaction problem & \hyperref[detail:SadehF96]{Details} \href{../works/SadehF96.pdf}{Yes} & \cite{SadehF96} & 1996 & Artificial Intelligence & 41 & \noindent{}\textbf{2.50} \textbf{2.50} \textbf{22.27} & 95 97 131 & 17 56 & 19 16 3\\
Sadeh1995 \href{http://dx.doi.org/10.1016/0004-3702(95)00078-s}{Sadeh1995} & \hyperref[auth:a1581]{N. Sadeh}, \hyperref[auth:a1582]{K. Sycara}, \hyperref[auth:a1583]{Y. Xiong} & \cellcolor{gold!20}Backtracking techniques for the job shop scheduling constraint satisfaction problem & \hyperref[detail:Sadeh1995]{Details} No & \cite{Sadeh1995} & 1995 & Artificial Intelligence & null & \noindent{}\textbf{2.00} \textbf{2.00} n/a & 21 20 31 & 12 28 & 5 5 0\\
MintonJPL92 \href{http://dx.doi.org/10.1016/0004-3702(92)90007-k}{MintonJPL92} & \hyperref[auth:a1210]{S. Minton}, \hyperref[auth:a1211]{M. D. Johnston}, \hyperref[auth:a1212]{A. B. Philips}, \hyperref[auth:a1213]{P. Laird} & \cellcolor{green!10}Minimizing conflicts: a heuristic repair method for constraint satisfaction and scheduling problems & \hyperref[detail:MintonJPL92]{Details} \href{../works/MintonJPL92.pdf}{Yes} & \cite{MintonJPL92} & 1992 & Artificial Intelligence & 45 & \noindent{}\textbf{1.00} \textbf{1.00} \textbf{1.68} & 437 440 525 & 13 46 & 18 18 0\\
Davis87 \href{http://dx.doi.org/10.1016/0004-3702(87)90091-9}{Davis87} & \hyperref[auth:a1215]{E. Davis} & \cellcolor{gold!20}Constraint propagation with interval labels & \hyperref[detail:Davis87]{Details} \href{../works/Davis87.pdf}{Yes} & \cite{Davis87} & 1987 & Artificial Intelligence & 51 & \noindent{}\textcolor{black!50}{0.00} \textcolor{black!50}{0.00} \textbf{2.34} & 308 312 332 & 21 51 & 12 11 1\\
\end{longtable}
}

\subsection{Artificial Intelligence for Engineering Design, Analysis and Manufacturing}

\index{Artificial Intelligence for Engineering Design, Analysis and Manufacturing}
{\scriptsize
\begin{longtable}{>{\raggedright\arraybackslash}p{2.5cm}>{\raggedright\arraybackslash}p{4.5cm}>{\raggedright\arraybackslash}p{6.0cm}p{1.0cm}rr>{\raggedright\arraybackslash}p{2.0cm}r>{\raggedright\arraybackslash}p{1cm}p{1cm}p{1cm}p{1cm}}
\rowcolor{white}\caption{Articles in Journal Artificial Intelligence for Engineering Design, Analysis and Manufacturing (Total 1)}\\ \toprule
\rowcolor{white}\shortstack{Key\\Source} & Authors & Title (Colored by Open Access)& \shortstack{Details\\LC} & Cite & Year & \shortstack{Conference\\/Journal\\/School} & Pages & Relevance &\shortstack{Cites\\OC XR\\SC} & \shortstack{Refs\\OC\\XR} & \shortstack{Links\\Cites\\Refs}\\ \midrule\endhead
\bottomrule
\endfoot
Baykan1997 \href{http://dx.doi.org/10.1017/s0890060400003206}{Baykan1997} & \hyperref[auth:a1689]{C. A. Baykan}, \hyperref[auth:a302]{M. S. Fox} & \cellcolor{green!10}Spatial synthesis by disjunctive constraint satisfaction \hyperref[abs:Baykan1997]{Abstract} & \hyperref[detail:Baykan1997]{Details} No & \cite{Baykan1997} & 1997 & Artificial Intelligence for Engineering Design, Analysis and Manufacturing & null & \noindent{}\textcolor{black!50}{0.00} 0.50 n/a & 11 11 19 & 12 35 & 2 0 2\\
\end{longtable}
}

\subsection{Asia-Pacific Journal of Operational Research}

\index{Asia-Pacific Journal of Operational Research}
{\scriptsize
\begin{longtable}{>{\raggedright\arraybackslash}p{2.5cm}>{\raggedright\arraybackslash}p{4.5cm}>{\raggedright\arraybackslash}p{6.0cm}p{1.0cm}rr>{\raggedright\arraybackslash}p{2.0cm}r>{\raggedright\arraybackslash}p{1cm}p{1cm}p{1cm}p{1cm}}
\rowcolor{white}\caption{Articles in Journal Asia-Pacific Journal of Operational Research (Total 1)}\\ \toprule
\rowcolor{white}\shortstack{Key\\Source} & Authors & Title (Colored by Open Access)& \shortstack{Details\\LC} & Cite & Year & \shortstack{Conference\\/Journal\\/School} & Pages & Relevance &\shortstack{Cites\\OC XR\\SC} & \shortstack{Refs\\OC\\XR} & \shortstack{Links\\Cites\\Refs}\\ \midrule\endhead
\bottomrule
\endfoot
Juan2014 \href{http://dx.doi.org/10.1142/s0217595914500419}{Juan2014} & \hyperref[auth:a1981]{Y.-C. Juan}, \hyperref[auth:a1982]{Y.-R. Peng} & A Constraint Satisfaction Coordination Approach for Distributed Supply Chain Production Planning \hyperref[abs:Juan2014]{Abstract} & \hyperref[detail:Juan2014]{Details} No & \cite{Juan2014} & 2014 & Asia-Pacific Journal of Operational Research & null & \noindent{}\textcolor{black!50}{0.00} \textbf{3.00} n/a & 0 0 0 & 14 20 & 1 0 1\\
\end{longtable}
}

\subsection{Assembly Automation}

\index{Assembly Automation}
{\scriptsize
\begin{longtable}{>{\raggedright\arraybackslash}p{2.5cm}>{\raggedright\arraybackslash}p{4.5cm}>{\raggedright\arraybackslash}p{6.0cm}p{1.0cm}rr>{\raggedright\arraybackslash}p{2.0cm}r>{\raggedright\arraybackslash}p{1cm}p{1cm}p{1cm}p{1cm}}
\rowcolor{white}\caption{Articles in Journal Assembly Automation (Total 2)}\\ \toprule
\rowcolor{white}\shortstack{Key\\Source} & Authors & Title (Colored by Open Access)& \shortstack{Details\\LC} & Cite & Year & \shortstack{Conference\\/Journal\\/School} & Pages & Relevance &\shortstack{Cites\\OC XR\\SC} & \shortstack{Refs\\OC\\XR} & \shortstack{Links\\Cites\\Refs}\\ \midrule\endhead
\bottomrule
\endfoot
Michels2022 \href{http://dx.doi.org/10.1108/aa-10-2021-0140}{Michels2022} & \hyperref[auth:a1551]{A. S. Michels}, \hyperref[auth:a1552]{A. M. Costa} & Mixed-integer linear programming models for the type-II resource-constrained assembly line balancing problem \hyperref[abs:Michels2022]{Abstract} & \hyperref[detail:Michels2022]{Details} No & \cite{Michels2022} & 2022 & Assembly Automation & null & \noindent{}\textcolor{black!50}{0.00} \textbf{1.50} n/a & 2 2 1 & 26 35 & 4 1 3\\
PinarbasiAY19 \href{http://dx.doi.org/10.1108/aa-12-2018-0262}{PinarbasiAY19} & \hyperref[auth:a413]{M. Pinarbasi}, \hyperref[auth:a1423]{H. M. Alakas}, \hyperref[auth:a1424]{M. Yuzukirmizi} & A constraint programming approach to type-2 assembly line balancing problem with assignment restrictions & \hyperref[detail:PinarbasiAY19]{Details} \href{../works/PinarbasiAY19.pdf}{Yes} & \cite{PinarbasiAY19} & 2019 & Assembly Automation & 14 & \noindent{}\textcolor{black!50}{0.00} \textcolor{black!50}{0.00} \textbf{26.65} & 16 18 0 & 41 68 & 12 7 5\\
\end{longtable}
}

\subsection{Automation in Construction}

\index{Automation in Construction}
{\scriptsize
\begin{longtable}{>{\raggedright\arraybackslash}p{2.5cm}>{\raggedright\arraybackslash}p{4.5cm}>{\raggedright\arraybackslash}p{6.0cm}p{1.0cm}rr>{\raggedright\arraybackslash}p{2.0cm}r>{\raggedright\arraybackslash}p{1cm}p{1cm}p{1cm}p{1cm}}
\rowcolor{white}\caption{Articles in Journal Automation in Construction (Total 4)}\\ \toprule
\rowcolor{white}\shortstack{Key\\Source} & Authors & Title (Colored by Open Access)& \shortstack{Details\\LC} & Cite & Year & \shortstack{Conference\\/Journal\\/School} & Pages & Relevance &\shortstack{Cites\\OC XR\\SC} & \shortstack{Refs\\OC\\XR} & \shortstack{Links\\Cites\\Refs}\\ \midrule\endhead
\bottomrule
\endfoot
ZouZ20 \href{https://api.semanticscholar.org/CorpusID:208840808}{ZouZ20} & \hyperref[auth:a756]{X. Zou}, \hyperref[auth:a757]{L. Zhang} & A constraint programming approach for scheduling repetitive projects with atypical activities considering soft logic & \hyperref[detail:ZouZ20]{Details} \href{../works/ZouZ20.pdf}{Yes} & \cite{ZouZ20} & 2020 & Automation in Construction & 10 & \noindent{}\textbf{1.00} \textbf{1.00} \textbf{5.88} & 18 21 19 & 48 52 & 7 3 4\\
Dolabi2014 \href{http://dx.doi.org/10.1016/j.autcon.2014.09.003}{Dolabi2014} & \hyperref[auth:a1748]{H. R. Z. Dolabi}, \hyperref[auth:a1749]{A. Afshar}, \hyperref[auth:a1750]{R. Abbasnia} & CPM/LOB Scheduling Method for Project Deadline Constraint Satisfaction & \hyperref[detail:Dolabi2014]{Details} No & \cite{Dolabi2014} & 2014 & Automation in Construction & null & \noindent{}\textbf{1.00} \textbf{1.00} n/a & 33 38 41 & 30 31 & 5 4 1\\
Tang2014 \href{http://dx.doi.org/10.1016/j.autcon.2013.09.008}{Tang2014} & \hyperref[auth:a555]{Y. Tang}, \hyperref[auth:a556]{R. Liu}, \hyperref[auth:a558]{Q. Sun} & Schedule control model for linear projects based on linear scheduling method and constraint programming & \hyperref[detail:Tang2014]{Details} No & \cite{Tang2014} & 2014 & Automation in Construction & null & \noindent{}\textbf{1.00} \textbf{1.00} n/a & 32 35 42 & 23 39 & 9 5 4\\
LiuW11 \href{http://dx.doi.org/10.1016/j.autcon.2011.04.012}{LiuW11} & \hyperref[auth:a1244]{S.-S. Liu}, \hyperref[auth:a1245]{C.-J. Wang} & Optimizing project selection and scheduling problems with time-dependent resource constraints & \hyperref[detail:LiuW11]{Details} \href{../works/LiuW11.pdf}{Yes} & \cite{LiuW11} & 2011 & Automation in Construction & 10 & \noindent{}\textcolor{black!50}{0.00} \textcolor{black!50}{0.00} \textbf{11.58} & 57 59 71 & 35 48 & 15 5 10\\
\end{longtable}
}

\subsection{Buildings}

\index{Buildings}
{\scriptsize
\begin{longtable}{>{\raggedright\arraybackslash}p{2.5cm}>{\raggedright\arraybackslash}p{4.5cm}>{\raggedright\arraybackslash}p{6.0cm}p{1.0cm}rr>{\raggedright\arraybackslash}p{2.0cm}r>{\raggedright\arraybackslash}p{1cm}p{1cm}p{1cm}p{1cm}}
\rowcolor{white}\caption{Articles in Journal Buildings (Total 1)}\\ \toprule
\rowcolor{white}\shortstack{Key\\Source} & Authors & Title (Colored by Open Access)& \shortstack{Details\\LC} & Cite & Year & \shortstack{Conference\\/Journal\\/School} & Pages & Relevance &\shortstack{Cites\\OC XR\\SC} & \shortstack{Refs\\OC\\XR} & \shortstack{Links\\Cites\\Refs}\\ \midrule\endhead
\bottomrule
\endfoot
Liu2023 \href{http://dx.doi.org/10.3390/buildings13071867}{Liu2023} & \hyperref[auth:a1244]{S.-S. Liu}, \hyperref[auth:a1718]{P. Utami}, \hyperref[auth:a1719]{A. Budiwirawan}, \hyperref[auth:a1489]{M. F. A. Arifin}, \hyperref[auth:a1720]{F. S. Perdana} & \cellcolor{gold!20}Optimization Model of Maintenance Scheduling Problem for Heritage Buildings with Constraint Programming \hyperref[abs:Liu2023]{Abstract} & \hyperref[detail:Liu2023]{Details} No & \cite{Liu2023} & 2023 & Buildings & null & \noindent{}\textbf{1.00} \textbf{4.01} n/a & 0 1 1 & 48 55 & 2 0 2\\
\end{longtable}
}

\subsection{Business Research}

\index{Business Research}
{\scriptsize
\begin{longtable}{>{\raggedright\arraybackslash}p{2.5cm}>{\raggedright\arraybackslash}p{4.5cm}>{\raggedright\arraybackslash}p{6.0cm}p{1.0cm}rr>{\raggedright\arraybackslash}p{2.0cm}r>{\raggedright\arraybackslash}p{1cm}p{1cm}p{1cm}p{1cm}}
\rowcolor{white}\caption{Articles in Journal Business Research (Total 1)}\\ \toprule
\rowcolor{white}\shortstack{Key\\Source} & Authors & Title (Colored by Open Access)& \shortstack{Details\\LC} & Cite & Year & \shortstack{Conference\\/Journal\\/School} & Pages & Relevance &\shortstack{Cites\\OC XR\\SC} & \shortstack{Refs\\OC\\XR} & \shortstack{Links\\Cites\\Refs}\\ \midrule\endhead
\bottomrule
\endfoot
Schwarz2019 \href{http://dx.doi.org/10.1007/s40685-019-00102-z}{Schwarz2019} & \hyperref[auth:a2013]{K. Schwarz}, \hyperref[auth:a2014]{M. Römer}, \hyperref[auth:a2015]{T. Mellouli} & \cellcolor{gold!20}A data-driven hierarchical MILP approach for scheduling clinical pathways: a real-world case study from a German university hospital \hyperref[abs:Schwarz2019]{Abstract} & \hyperref[detail:Schwarz2019]{Details} No & \cite{Schwarz2019} & 2019 & Business Research & null & \noindent{}\textcolor{black!50}{0.00} \textbf{1.50} n/a & 2 2 3 & 50 58 & 1 0 1\\
\end{longtable}
}

\subsection{CMC-COMPUTERS MATERIALS \& CONTINUA}

\index{CMC-COMPUTERS MATERIALS \& CONTINUA}
{\scriptsize
\begin{longtable}{>{\raggedright\arraybackslash}p{2.5cm}>{\raggedright\arraybackslash}p{4.5cm}>{\raggedright\arraybackslash}p{6.0cm}p{1.0cm}rr>{\raggedright\arraybackslash}p{2.0cm}r>{\raggedright\arraybackslash}p{1cm}p{1cm}p{1cm}p{1cm}}
\rowcolor{white}\caption{Articles in Journal CMC-COMPUTERS MATERIALS \  CONTINUA (Total 1)}\\ \toprule
\rowcolor{white}\shortstack{Key\\Source} & Authors & Title (Colored by Open Access)& \shortstack{Details\\LC} & Cite & Year & \shortstack{Conference\\/Journal\\/School} & Pages & Relevance &\shortstack{Cites\\OC XR\\SC} & \shortstack{Refs\\OC\\XR} & \shortstack{Links\\Cites\\Refs}\\ \midrule\endhead
\bottomrule
\endfoot
NaqviAIAAA22 \href{http://dx.doi.org/10.32604/cmc.2022.019653}{NaqviAIAAA22} & \hyperref[auth:a1393]{S. R. Naqvi}, \hyperref[auth:a1394]{A. Ahmad}, \hyperref[auth:a1395]{S. M. R. Islam}, \hyperref[auth:a1396]{T. Akram}, \hyperref[auth:a1397]{M. Abdullah-Al-Wadud}, \hyperref[auth:a1398]{A. Alamri} & \cellcolor{gold!20}Towards Prevention of Sportsmen Burnout: Formal Analysis of Sub-Optimal Tournament Scheduling \hyperref[abs:NaqviAIAAA22]{Abstract} & \hyperref[detail:NaqviAIAAA22]{Details} \href{../works/NaqviAIAAA22.pdf}{Yes} & \cite{NaqviAIAAA22} & 2022 & CMC-COMPUTERS MATERIALS \  CONTINUA & 18 & \noindent{}\textcolor{black!50}{0.00} \textcolor{black!50}{0.00} 0.34 & 0 0 0 & 22 26 & 2 0 2\\
\end{longtable}
}

\subsection{Canadian Journal of Civil Engineering}

\index{Canadian Journal of Civil Engineering}
{\scriptsize
\begin{longtable}{>{\raggedright\arraybackslash}p{2.5cm}>{\raggedright\arraybackslash}p{4.5cm}>{\raggedright\arraybackslash}p{6.0cm}p{1.0cm}rr>{\raggedright\arraybackslash}p{2.0cm}r>{\raggedright\arraybackslash}p{1cm}p{1cm}p{1cm}p{1cm}}
\rowcolor{white}\caption{Articles in Journal Canadian Journal of Civil Engineering (Total 1)}\\ \toprule
\rowcolor{white}\shortstack{Key\\Source} & Authors & Title (Colored by Open Access)& \shortstack{Details\\LC} & Cite & Year & \shortstack{Conference\\/Journal\\/School} & Pages & Relevance &\shortstack{Cites\\OC XR\\SC} & \shortstack{Refs\\OC\\XR} & \shortstack{Links\\Cites\\Refs}\\ \midrule\endhead
\bottomrule
\endfoot
Abuwarda2019 \href{http://dx.doi.org/10.1139/cjce-2018-0544}{Abuwarda2019} & \hyperref[auth:a1520]{Z. Abuwarda}, \hyperref[auth:a1521]{T. Hegazy} & \cellcolor{gold!20}Multi-dimensional optimization model for schedule fast-tracking without over-stressing construction workers \hyperref[abs:Abuwarda2019]{Abstract} & \hyperref[detail:Abuwarda2019]{Details} No & \cite{Abuwarda2019} & 2019 & Canadian Journal of Civil Engineering & null & \noindent{}\textcolor{black!50}{0.00} \textbf{2.00} n/a & 4 5 5 & 46 48 & 6 0 6\\
\end{longtable}
}

\subsection{Central Eur. J. Oper. Res.}

\index{Central Eur. J. Oper. Res.}
{\scriptsize
\begin{longtable}{>{\raggedright\arraybackslash}p{2.5cm}>{\raggedright\arraybackslash}p{4.5cm}>{\raggedright\arraybackslash}p{6.0cm}p{1.0cm}rr>{\raggedright\arraybackslash}p{2.0cm}r>{\raggedright\arraybackslash}p{1cm}p{1cm}p{1cm}p{1cm}}
\rowcolor{white}\caption{Articles in Journal Central Eur. J. Oper. Res. (Total 1)}\\ \toprule
\rowcolor{white}\shortstack{Key\\Source} & Authors & Title (Colored by Open Access)& \shortstack{Details\\LC} & Cite & Year & \shortstack{Conference\\/Journal\\/School} & Pages & Relevance &\shortstack{Cites\\OC XR\\SC} & \shortstack{Refs\\OC\\XR} & \shortstack{Links\\Cites\\Refs}\\ \midrule\endhead
\bottomrule
\endfoot
GurPAE23 \href{https://doi.org/10.1007/s10100-022-00835-z}{GurPAE23} & \hyperref[auth:a412]{S. G{\"{u}}r}, \hyperref[auth:a413]{M. Pinarbasi}, \hyperref[auth:a414]{H. M. Alakas}, \hyperref[auth:a415]{T. Eren} & Operating room scheduling with surgical team: a new approach with constraint programming and goal programming & \hyperref[detail:GurPAE23]{Details} \href{../works/GurPAE23.pdf}{Yes} & \cite{GurPAE23} & 2023 & Central Eur. J. Oper. Res. & 25 & \noindent{}\textbf{1.00} \textbf{1.00} \textbf{4.04} & 1 5 3 & 40 46 & 4 1 3\\
\end{longtable}
}

\subsection{Central European Journal of Operations Research}

\index{Central European Journal of Operations Research}
{\scriptsize
\begin{longtable}{>{\raggedright\arraybackslash}p{2.5cm}>{\raggedright\arraybackslash}p{4.5cm}>{\raggedright\arraybackslash}p{6.0cm}p{1.0cm}rr>{\raggedright\arraybackslash}p{2.0cm}r>{\raggedright\arraybackslash}p{1cm}p{1cm}p{1cm}p{1cm}}
\rowcolor{white}\caption{Articles in Journal Central European Journal of Operations Research (Total 1)}\\ \toprule
\rowcolor{white}\shortstack{Key\\Source} & Authors & Title (Colored by Open Access)& \shortstack{Details\\LC} & Cite & Year & \shortstack{Conference\\/Journal\\/School} & Pages & Relevance &\shortstack{Cites\\OC XR\\SC} & \shortstack{Refs\\OC\\XR} & \shortstack{Links\\Cites\\Refs}\\ \midrule\endhead
\bottomrule
\endfoot
Dimny2023 \href{http://dx.doi.org/10.1007/s10100-023-00885-x}{Dimny2023} & \hyperref[auth:a1487]{I. Dimény}, \hyperref[auth:a1488]{T. Koltai} & \cellcolor{gold!20}Comparison of MILP and CP models for balancing partially automated assembly lines \hyperref[abs:Dimny2023]{Abstract} & \hyperref[detail:Dimny2023]{Details} No & \cite{Dimny2023} & 2023 & Central European Journal of Operations Research & null & \noindent{}\textcolor{black!50}{0.00} \textbf{5.00} n/a & 0 0 0 & 35 37 & 3 0 3\\
\end{longtable}
}

\subsection{CoRR}

\index{CoRR}
{\scriptsize
\begin{longtable}{>{\raggedright\arraybackslash}p{2.5cm}>{\raggedright\arraybackslash}p{4.5cm}>{\raggedright\arraybackslash}p{6.0cm}p{1.0cm}rr>{\raggedright\arraybackslash}p{2.0cm}r>{\raggedright\arraybackslash}p{1cm}p{1cm}p{1cm}p{1cm}}
\rowcolor{white}\caption{Articles in Journal CoRR (Total 13)}\\ \toprule
\rowcolor{white}\shortstack{Key\\Source} & Authors & Title (Colored by Open Access)& \shortstack{Details\\LC} & Cite & Year & \shortstack{Conference\\/Journal\\/School} & Pages & Relevance &\shortstack{Cites\\OC XR\\SC} & \shortstack{Refs\\OC\\XR} & \shortstack{Links\\Cites\\Refs}\\ \midrule\endhead
\bottomrule
\endfoot
abs-2402-00459 \href{https://doi.org/10.48550/arXiv.2402.00459}{abs-2402-00459} & \hyperref[auth:a395]{S. Nguyen}, \hyperref[auth:a396]{D. R. Thiruvady}, \hyperref[auth:a397]{Y. Sun}, \hyperref[auth:a398]{M. Zhang} & Genetic-based Constraint Programming for Resource Constrained Job Scheduling & \hyperref[detail:abs-2402-00459]{Details} \href{../works/abs-2402-00459.pdf}{Yes} & \cite{abs-2402-00459} & 2024 & CoRR & 21 & \noindent{}\textbf{2.50} \textbf{2.50} \textbf{10.36} & 0 0 0 & 0 0 & 0 0 0\\
abs-2305-19888 \href{https://doi.org/10.48550/arXiv.2305.19888}{abs-2305-19888} & \hyperref[auth:a433]{V. Heinz}, \hyperref[auth:a434]{A. Nov{\'{a}}k}, \hyperref[auth:a311]{M. Vlk}, \hyperref[auth:a116]{Z. Hanz{\'{a}}lek} & Constraint Programming and Constructive Heuristics for Parallel Machine Scheduling with Sequence-Dependent Setups and Common Servers & \hyperref[detail:abs-2305-19888]{Details} \href{../works/abs-2305-19888.pdf}{Yes} & \cite{abs-2305-19888} & 2023 & CoRR & 42 & \noindent{}\textbf{1.50} \textbf{1.50} \textbf{41.88} & 0 0 0 & 0 0 & 0 0 0\\
abs-2306-05747 \href{https://doi.org/10.48550/arXiv.2306.05747}{abs-2306-05747} & \hyperref[auth:a58]{P. Tassel}, \hyperref[auth:a61]{M. Gebser}, \hyperref[auth:a423]{K. Schekotihin} & An End-to-End Reinforcement Learning Approach for Job-Shop Scheduling Problems Based on Constraint Programming & \hyperref[detail:abs-2306-05747]{Details} \href{../works/abs-2306-05747.pdf}{Yes} & \cite{abs-2306-05747} & 2023 & CoRR & 9 & \noindent{}\textbf{2.00} \textbf{2.00} \textbf{12.09} & 0 0 0 & 0 0 & 0 0 0\\
abs-2312-13682 \href{https://doi.org/10.48550/arXiv.2312.13682}{abs-2312-13682} & \hyperref[auth:a425]{G. Perez}, \hyperref[auth:a426]{G. Glorian}, \hyperref[auth:a427]{W. Suijlen}, \hyperref[auth:a428]{A. Lallouet} & A Constraint Programming Model for Scheduling the Unloading of Trains in Ports: Extended & \hyperref[detail:abs-2312-13682]{Details} \href{../works/abs-2312-13682.pdf}{Yes} & \cite{abs-2312-13682} & 2023 & CoRR & 20 & \noindent{}\textbf{1.00} \textbf{1.00} 0.94 & 0 0 0 & 0 0 & 0 0 0\\
abs-2211-14492 \href{https://doi.org/10.48550/arXiv.2211.14492}{abs-2211-14492} & \hyperref[auth:a397]{Y. Sun}, \hyperref[auth:a395]{S. Nguyen}, \hyperref[auth:a396]{D. R. Thiruvady}, \hyperref[auth:a468]{X. Li}, \hyperref[auth:a469]{A. T. Ernst}, \hyperref[auth:a470]{U. Aickelin} & Enhancing Constraint Programming via Supervised Learning for Job Shop Scheduling & \hyperref[detail:abs-2211-14492]{Details} \href{../works/abs-2211-14492.pdf}{Yes} & \cite{abs-2211-14492} & 2022 & CoRR & 17 & \noindent{}\textbf{2.00} \textbf{2.00} \textbf{13.19} & 0 0 0 & 0 0 & 0 0 0\\
abs-2102-08778 \href{https://arxiv.org/abs/2102.08778}{abs-2102-08778} & \hyperref[auth:a93]{G. D. Col}, \hyperref[auth:a608]{E. Teppan} & Large-Scale Benchmarks for the Job Shop Scheduling Problem & \hyperref[detail:abs-2102-08778]{Details} \href{../works/abs-2102-08778.pdf}{Yes} & \cite{abs-2102-08778} & 2021 & CoRR & 10 & \noindent{}\textcolor{black!50}{0.00} \textcolor{black!50}{0.00} \textbf{1.00} & 0 0 0 & 0 0 & 0 0 0\\
abs-1901-07914 \href{http://arxiv.org/abs/1901.07914}{abs-1901-07914} & \hyperref[auth:a540]{J. K. Behrens}, \hyperref[auth:a541]{R. Lange}, \hyperref[auth:a542]{M. Mansouri} & A Constraint Programming Approach to Simultaneous Task Allocation and Motion Scheduling for Industrial Dual-Arm Manipulation Tasks & \hyperref[detail:abs-1901-07914]{Details} \href{../works/abs-1901-07914.pdf}{Yes} & \cite{abs-1901-07914} & 2019 & CoRR & 8 & \noindent{}\textbf{2.00} \textbf{2.00} \textbf{4.13} & 0 0 0 & 0 0 & 0 0 0\\
abs-1902-01193 \href{http://arxiv.org/abs/1902.01193}{abs-1902-01193} & \hyperref[auth:a548]{O. M. Alade}, \hyperref[auth:a549]{A. O. Amusat} & Solving Nurse Scheduling Problem Using Constraint Programming Technique & \hyperref[detail:abs-1902-01193]{Details} \href{../works/abs-1902-01193.pdf}{Yes} & \cite{abs-1902-01193} & 2019 & CoRR & 9 & \noindent{}\textbf{1.00} \textbf{1.00} \textbf{1.99} & 0 0 0 & 0 0 & 0 0 0\\
abs-1902-09244 \href{http://arxiv.org/abs/1902.09244}{abs-1902-09244} & \hyperref[auth:a550]{V. A. Hauder}, \hyperref[auth:a551]{A. Beham}, \hyperref[auth:a552]{S. Raggl}, \hyperref[auth:a553]{S. N. Parragh}, \hyperref[auth:a554]{M. Affenzeller} & On constraint programming for a new flexible project scheduling problem with resource constraints & \hyperref[detail:abs-1902-09244]{Details} \href{../works/abs-1902-09244.pdf}{Yes} & \cite{abs-1902-09244} & 2019 & CoRR & 62 & \noindent{}\textbf{1.50} \textbf{1.50} \textbf{350.76} & 0 0 0 & 0 0 & 0 0 0\\
abs-1911-04766 \href{http://arxiv.org/abs/1911.04766}{abs-1911-04766} & \hyperref[auth:a77]{T. Geibinger}, \hyperref[auth:a80]{F. Mischek}, \hyperref[auth:a45]{N. Musliu} & Investigating Constraint Programming and Hybrid Methods for Real World Industrial Test Laboratory Scheduling & \hyperref[detail:abs-1911-04766]{Details} \href{../works/abs-1911-04766.pdf}{Yes} & \cite{abs-1911-04766} & 2019 & CoRR & 16 & \noindent{}\textbf{1.00} \textbf{1.00} \textbf{15.95} & 0 0 0 & 0 0 & 0 0 0\\
EvenSH15a \href{http://arxiv.org/abs/1505.02487}{EvenSH15a} & \hyperref[auth:a214]{C. Even}, \hyperref[auth:a124]{A. Schutt}, \hyperref[auth:a148]{P. V. Hentenryck} & A Constraint Programming Approach for Non-Preemptive Evacuation Scheduling & \hyperref[detail:EvenSH15a]{Details} \href{../works/EvenSH15a.pdf}{Yes} & \cite{EvenSH15a} & 2015 & CoRR & 16 & \noindent{}\textbf{1.00} \textbf{1.00} 0.42 & 0 0 0 & 0 0 & 0 0 0\\
abs-1009-0347 \href{http://arxiv.org/abs/1009.0347}{abs-1009-0347} & \hyperref[auth:a124]{A. Schutt}, \hyperref[auth:a154]{T. Feydy}, \hyperref[auth:a125]{P. J. Stuckey}, \hyperref[auth:a117]{M. G. Wallace} & Solving the Resource Constrained Project Scheduling Problem with Generalized Precedences by Lazy Clause Generation & \hyperref[detail:abs-1009-0347]{Details} \href{../works/abs-1009-0347.pdf}{Yes} & \cite{abs-1009-0347} & 2010 & CoRR & 37 & \noindent{}\textcolor{black!50}{0.00} \textcolor{black!50}{0.00} \textbf{4.70} & 0 0 0 & 0 0 & 0 0 0\\
abs-0907-0939 \href{http://arxiv.org/abs/0907.0939}{abs-0907-0939} & \hyperref[auth:a221]{T. Petit}, \hyperref[auth:a358]{E. Poder} & The Soft Cumulative Constraint & \hyperref[detail:abs-0907-0939]{Details} \href{../works/abs-0907-0939.pdf}{Yes} & \cite{abs-0907-0939} & 2009 & CoRR & 12 & \noindent{}\textcolor{black!50}{0.00} \textcolor{black!50}{0.00} 0.37 & 0 0 0 & 0 0 & 0 0 0\\
\end{longtable}
}

\subsection{Communications - Scientific letters of the University of Zilina}

\index{Communications - Scientific letters of the University of Zilina}
{\scriptsize
\begin{longtable}{>{\raggedright\arraybackslash}p{2.5cm}>{\raggedright\arraybackslash}p{4.5cm}>{\raggedright\arraybackslash}p{6.0cm}p{1.0cm}rr>{\raggedright\arraybackslash}p{2.0cm}r>{\raggedright\arraybackslash}p{1cm}p{1cm}p{1cm}p{1cm}}
\rowcolor{white}\caption{Articles in Journal Communications - Scientific letters of the University of Zilina (Total 1)}\\ \toprule
\rowcolor{white}\shortstack{Key\\Source} & Authors & Title (Colored by Open Access)& \shortstack{Details\\LC} & Cite & Year & \shortstack{Conference\\/Journal\\/School} & Pages & Relevance &\shortstack{Cites\\OC XR\\SC} & \shortstack{Refs\\OC\\XR} & \shortstack{Links\\Cites\\Refs}\\ \midrule\endhead
\bottomrule
\endfoot
Janosikova2013 \href{http://dx.doi.org/10.26552/com.c.2013.1.39-43}{Janosikova2013} & \hyperref[auth:a2038]{L. Janosikova}, \hyperref[auth:a2039]{T. Hreben} & Mathematical Programming vs. Constraint Programming for Scheduling Problems & \hyperref[detail:Janosikova2013]{Details} No & \cite{Janosikova2013} & 2013 & Communications - Scientific letters of the University of Zilina & null & \noindent{}\textbf{1.00} \textbf{1.00} n/a & 0 0 0 & 4 7 & 3 0 3\\
\end{longtable}
}

\subsection{Comput. Aided Civ. Infrastructure Eng.}

\index{Comput. Aided Civ. Infrastructure Eng.}
{\scriptsize
\begin{longtable}{>{\raggedright\arraybackslash}p{2.5cm}>{\raggedright\arraybackslash}p{4.5cm}>{\raggedright\arraybackslash}p{6.0cm}p{1.0cm}rr>{\raggedright\arraybackslash}p{2.0cm}r>{\raggedright\arraybackslash}p{1cm}p{1cm}p{1cm}p{1cm}}
\rowcolor{white}\caption{Articles in Journal Comput. Aided Civ. Infrastructure Eng. (Total 1)}\\ \toprule
\rowcolor{white}\shortstack{Key\\Source} & Authors & Title (Colored by Open Access)& \shortstack{Details\\LC} & Cite & Year & \shortstack{Conference\\/Journal\\/School} & Pages & Relevance &\shortstack{Cites\\OC XR\\SC} & \shortstack{Refs\\OC\\XR} & \shortstack{Links\\Cites\\Refs}\\ \midrule\endhead
\bottomrule
\endfoot
TangLWSK18 \href{https://doi.org/10.1111/mice.12277}{TangLWSK18} & \hyperref[auth:a555]{Y. Tang}, \hyperref[auth:a556]{R. Liu}, \hyperref[auth:a557]{F. Wang}, \hyperref[auth:a558]{Q. Sun}, \hyperref[auth:a559]{A. A. Kandil} & Scheduling Optimization of Linear Schedule with Constraint Programming & \hyperref[detail:TangLWSK18]{Details} \href{../works/TangLWSK18.pdf}{Yes} & \cite{TangLWSK18} & 2018 & Comput. Aided Civ. Infrastructure Eng. & 28 & \noindent{}\textbf{1.00} \textbf{1.00} \textbf{24.56} & 24 32 30 & 76 88 & 12 5 7\\
\end{longtable}
}

\subsection{Comput. J.}

\index{Comput. J.}
{\scriptsize
\begin{longtable}{>{\raggedright\arraybackslash}p{2.5cm}>{\raggedright\arraybackslash}p{4.5cm}>{\raggedright\arraybackslash}p{6.0cm}p{1.0cm}rr>{\raggedright\arraybackslash}p{2.0cm}r>{\raggedright\arraybackslash}p{1cm}p{1cm}p{1cm}p{1cm}}
\rowcolor{white}\caption{Articles in Journal Comput. J. (Total 1)}\\ \toprule
\rowcolor{white}\shortstack{Key\\Source} & Authors & Title (Colored by Open Access)& \shortstack{Details\\LC} & Cite & Year & \shortstack{Conference\\/Journal\\/School} & Pages & Relevance &\shortstack{Cites\\OC XR\\SC} & \shortstack{Refs\\OC\\XR} & \shortstack{Links\\Cites\\Refs}\\ \midrule\endhead
\bottomrule
\endfoot
Tay92 \href{}{Tay92} & \hyperref[auth:a701]{D. B. H. Tay} & {COPS:} {A} Constraint Programming Approach to Resource-Limited Project Scheduling & \hyperref[detail:Tay92]{Details} No & \cite{Tay92} & 1992 & Comput. J. & null & \noindent{}\textbf{1.50} \textbf{1.50} n/a & 0 0 0 & 0 0 & 0 0 0\\
\end{longtable}
}

\subsection{Computation}

\index{Computation}
{\scriptsize
\begin{longtable}{>{\raggedright\arraybackslash}p{2.5cm}>{\raggedright\arraybackslash}p{4.5cm}>{\raggedright\arraybackslash}p{6.0cm}p{1.0cm}rr>{\raggedright\arraybackslash}p{2.0cm}r>{\raggedright\arraybackslash}p{1cm}p{1cm}p{1cm}p{1cm}}
\rowcolor{white}\caption{Articles in Journal Computation (Total 1)}\\ \toprule
\rowcolor{white}\shortstack{Key\\Source} & Authors & Title (Colored by Open Access)& \shortstack{Details\\LC} & Cite & Year & \shortstack{Conference\\/Journal\\/School} & Pages & Relevance &\shortstack{Cites\\OC XR\\SC} & \shortstack{Refs\\OC\\XR} & \shortstack{Links\\Cites\\Refs}\\ \midrule\endhead
\bottomrule
\endfoot
Hajji2023 \href{http://dx.doi.org/10.3390/computation11070137}{Hajji2023} & \hyperref[auth:a1537]{M. K. Hajji}, \hyperref[auth:a1538]{H. Hadda}, \hyperref[auth:a1539]{N. Dridi} & \cellcolor{gold!20}Makespan Minimization for the Two-Stage Hybrid Flow Shop Problem with Dedicated Machines: A Comprehensive Study of Exact and Heuristic Approaches \hyperref[abs:Hajji2023]{Abstract} & \hyperref[detail:Hajji2023]{Details} No & \cite{Hajji2023} & 2023 & Computation & null & \noindent{}\textcolor{black!50}{0.00} \textbf{2.50} n/a & 0 0 0 & 26 42 & 4 0 4\\
\end{longtable}
}

\subsection{Computational Intelligence and Neuroscience}

\index{Computational Intelligence and Neuroscience}
{\scriptsize
\begin{longtable}{>{\raggedright\arraybackslash}p{2.5cm}>{\raggedright\arraybackslash}p{4.5cm}>{\raggedright\arraybackslash}p{6.0cm}p{1.0cm}rr>{\raggedright\arraybackslash}p{2.0cm}r>{\raggedright\arraybackslash}p{1cm}p{1cm}p{1cm}p{1cm}}
\rowcolor{white}\caption{Articles in Journal Computational Intelligence and Neuroscience (Total 2)}\\ \toprule
\rowcolor{white}\shortstack{Key\\Source} & Authors & Title (Colored by Open Access)& \shortstack{Details\\LC} & Cite & Year & \shortstack{Conference\\/Journal\\/School} & Pages & Relevance &\shortstack{Cites\\OC XR\\SC} & \shortstack{Refs\\OC\\XR} & \shortstack{Links\\Cites\\Refs}\\ \midrule\endhead
\bottomrule
\endfoot
Ortiz-Bayliss2018 \href{http://dx.doi.org/10.1155/2018/6103726}{Ortiz-Bayliss2018} & \hyperref[auth:a1781]{J. C. Ortiz-Bayliss}, \hyperref[auth:a1604]{I. Amaya}, \hyperref[auth:a1782]{S. E. Conant-Pablos}, \hyperref[auth:a1608]{H. Terashima-Marín} & \cellcolor{gold!20}Exploring the Impact of Early Decisions in Variable Ordering for Constraint Satisfaction Problems \hyperref[abs:Ortiz-Bayliss2018]{Abstract} & \hyperref[detail:Ortiz-Bayliss2018]{Details} No & \cite{Ortiz-Bayliss2018} & 2018 & Computational Intelligence and Neuroscience & null & \noindent{}0.50 0.50 n/a & 1 1 2 & 26 29 & 5 1 4\\
Moreno-Scott2016 \href{http://dx.doi.org/10.1155/2016/7349070}{Moreno-Scott2016} & \hyperref[auth:a1783]{J. H. Moreno-Scott}, \hyperref[auth:a1781]{J. C. Ortiz-Bayliss}, \hyperref[auth:a1608]{H. Terashima-Marín}, \hyperref[auth:a1782]{S. E. Conant-Pablos} & \cellcolor{gold!20}Experimental Matching of Instances to Heuristics for Constraint Satisfaction Problems \hyperref[abs:Moreno-Scott2016]{Abstract} & \hyperref[detail:Moreno-Scott2016]{Details} No & \cite{Moreno-Scott2016} & 2016 & Computational Intelligence and Neuroscience & null & \noindent{}\textcolor{black!50}{0.00} 0.50 n/a & 4 4 3 & 29 29 & 4 1 3\\
\end{longtable}
}

\subsection{Computer-Aided Civil and Infrastructure Engineering}

\index{Computer-Aided Civil and Infrastructure Engineering}
{\scriptsize
\begin{longtable}{>{\raggedright\arraybackslash}p{2.5cm}>{\raggedright\arraybackslash}p{4.5cm}>{\raggedright\arraybackslash}p{6.0cm}p{1.0cm}rr>{\raggedright\arraybackslash}p{2.0cm}r>{\raggedright\arraybackslash}p{1cm}p{1cm}p{1cm}p{1cm}}
\rowcolor{white}\caption{Articles in Journal Computer-Aided Civil and Infrastructure Engineering (Total 2)}\\ \toprule
\rowcolor{white}\shortstack{Key\\Source} & Authors & Title (Colored by Open Access)& \shortstack{Details\\LC} & Cite & Year & \shortstack{Conference\\/Journal\\/School} & Pages & Relevance &\shortstack{Cites\\OC XR\\SC} & \shortstack{Refs\\OC\\XR} & \shortstack{Links\\Cites\\Refs}\\ \midrule\endhead
\bottomrule
\endfoot
GarcaNieves2018 \href{http://dx.doi.org/10.1111/mice.12356}{GarcaNieves2018} & \hyperref[auth:a1724]{J. D. García‐Nieves}, \hyperref[auth:a1725]{J. L. Ponz‐Tienda}, \hyperref[auth:a1726]{A. Salcedo‐Bernal}, \hyperref[auth:a1727]{E. Pellicer} & \cellcolor{green!10}The Multimode Resource‐Constrained Project Scheduling Problem for Repetitive Activities in Construction Projects \hyperref[abs:GarcaNieves2018]{Abstract} & \hyperref[detail:GarcaNieves2018]{Details} No & \cite{GarcaNieves2018} & 2018 & Computer-Aided Civil and Infrastructure Engineering & null & \noindent{}\textcolor{black!50}{0.00} \textcolor{black!50}{0.00} n/a & 33 43 46 & 36 44 & 8 4 4\\
Tang2018 \href{http://dx.doi.org/10.1111/mice.12383}{Tang2018} & \hyperref[auth:a555]{Y. Tang}, \hyperref[auth:a558]{Q. Sun}, \hyperref[auth:a556]{R. Liu}, \hyperref[auth:a557]{F. Wang} & Resource Leveling Based on Line of Balance and Constraint Programming \hyperref[abs:Tang2018]{Abstract} & \hyperref[detail:Tang2018]{Details} No & \cite{Tang2018} & 2018 & Computer-Aided Civil and Infrastructure Engineering & null & \noindent{}0.50 \textbf{1.50} n/a & 18 21 23 & 80 88 & 11 2 9\\
\end{longtable}
}

\subsection{Computers \& Chemical Engineering}

\index{Computers \& Chemical Engineering}
{\scriptsize
\begin{longtable}{>{\raggedright\arraybackslash}p{2.5cm}>{\raggedright\arraybackslash}p{4.5cm}>{\raggedright\arraybackslash}p{6.0cm}p{1.0cm}rr>{\raggedright\arraybackslash}p{2.0cm}r>{\raggedright\arraybackslash}p{1cm}p{1cm}p{1cm}p{1cm}}
\rowcolor{white}\caption{Articles in Journal Computers \  Chemical Engineering (Total 14)}\\ \toprule
\rowcolor{white}\shortstack{Key\\Source} & Authors & Title (Colored by Open Access)& \shortstack{Details\\LC} & Cite & Year & \shortstack{Conference\\/Journal\\/School} & Pages & Relevance &\shortstack{Cites\\OC XR\\SC} & \shortstack{Refs\\OC\\XR} & \shortstack{Links\\Cites\\Refs}\\ \midrule\endhead
\bottomrule
\endfoot
AwadMDMT22 \href{http://dx.doi.org/10.1016/j.compchemeng.2021.107565}{AwadMDMT22} & \hyperref[auth:a1171]{M. Awad}, \hyperref[auth:a1172]{K. Mulrennan}, \hyperref[auth:a1173]{J. Donovan}, \hyperref[auth:a1174]{R. Macpherson}, \hyperref[auth:a1175]{D. Tormey} & A constraint programming model for makespan minimisation in batch manufacturing pharmaceutical facilities & \hyperref[detail:AwadMDMT22]{Details} \href{../works/AwadMDMT22.pdf}{Yes} & \cite{AwadMDMT22} & 2022 & Computers \  Chemical Engineering & 22 & \noindent{}\textcolor{black!50}{0.00} \textcolor{black!50}{0.00} \textbf{20.73} & 3 6 6 & 41 53 & 10 2 8\\
Misra2022 \href{http://dx.doi.org/10.1016/j.compchemeng.2022.107895}{Misra2022} & \hyperref[auth:a1802]{S. Misra}, \hyperref[auth:a1803]{L. R. Buttazoni}, \hyperref[auth:a1804]{V. Avadiappan}, \hyperref[auth:a1805]{H. J. Lee}, \hyperref[auth:a1806]{M. Yang}, \hyperref[auth:a381]{C. T. Maravelias} & CProS: A web-based application for chemical production scheduling & \hyperref[detail:Misra2022]{Details} No & \cite{Misra2022} & 2022 & Computers \  Chemical Engineering & null & \noindent{}\textcolor{black!50}{0.00} \textcolor{black!50}{0.00} n/a & 2 4 4 & 16 17 & 2 0 2\\
EscobetPQPRA19 \href{https://doi.org/10.1016/j.compchemeng.2018.08.040}{EscobetPQPRA19} & \hyperref[auth:a525]{T. Escobet}, \hyperref[auth:a526]{V. Puig}, \hyperref[auth:a527]{J. Quevedo}, \hyperref[auth:a528]{P. Pal{\`{a}}-Sch{\"{o}}nw{\"{a}}lder}, \hyperref[auth:a529]{J. Romera}, \hyperref[auth:a530]{W. Adelman} & \cellcolor{green!10}Optimal batch scheduling of a multiproduct dairy process using a combined optimization/constraint programming approach & \hyperref[detail:EscobetPQPRA19]{Details} \href{../works/EscobetPQPRA19.pdf}{Yes} & \cite{EscobetPQPRA19} & 2019 & Computers \  Chemical Engineering & 10 & \noindent{}\textbf{1.00} \textbf{1.00} \textbf{7.68} & 17 17 17 & 18 25 & 6 0 6\\
NovaraNH16 \href{https://doi.org/10.1016/j.compchemeng.2016.04.030}{NovaraNH16} & \hyperref[auth:a587]{F. M. Novara}, \hyperref[auth:a524]{J. M. Novas}, \hyperref[auth:a588]{G. P. Henning} & A novel constraint programming model for large-scale scheduling problems in multiproduct multistage batch plants: Limited resources and campaign-based operation & \hyperref[detail:NovaraNH16]{Details} \href{../works/NovaraNH16.pdf}{Yes} & \cite{NovaraNH16} & 2016 & Computers \  Chemical Engineering & 17 & \noindent{}\textbf{1.50} \textbf{1.50} \textbf{22.47} & 18 17 19 & 31 40 & 12 8 4\\
HarjunkoskiMBC14 \href{http://dx.doi.org/10.1016/j.compchemeng.2013.12.001}{HarjunkoskiMBC14} & \hyperref[auth:a871]{I. Harjunkoski}, \hyperref[auth:a381]{C. T. Maravelias}, \hyperref[auth:a937]{P. Bongers}, \hyperref[auth:a891]{P. M. Castro}, \hyperref[auth:a70]{S. Engell}, \hyperref[auth:a382]{I. E. Grossmann}, \hyperref[auth:a160]{J. N. Hooker}, \hyperref[auth:a938]{C. Méndez}, \hyperref[auth:a939]{G. Sand}, \hyperref[auth:a940]{J. Wassick} & \cellcolor{green!10}Scope for industrial applications of production scheduling models and solution methods & \hyperref[detail:HarjunkoskiMBC14]{Details} \href{../works/HarjunkoskiMBC14.pdf}{Yes} & \cite{HarjunkoskiMBC14} & 2014 & Computers \  Chemical Engineering & 33 & \noindent{}\textcolor{black!50}{0.00} \textcolor{black!50}{0.00} \textbf{41.04} & 381 393 418 & 176 229 & 28 10 18\\
NovasH12 \href{https://doi.org/10.1016/j.compchemeng.2012.01.005}{NovasH12} & \hyperref[auth:a524]{J. M. Novas}, \hyperref[auth:a588]{G. P. Henning} & A comprehensive constraint programming approach for the rolling horizon-based scheduling of automated wet-etch stations & \hyperref[detail:NovasH12]{Details} \href{../works/NovasH12.pdf}{Yes} & \cite{NovasH12} & 2012 & Computers \  Chemical Engineering & 17 & \noindent{}\textbf{1.00} \textbf{1.00} \textbf{11.52} & 17 17 22 & 15 23 & 5 3 2\\
ZeballosNH11 \href{http://dx.doi.org/10.1016/j.compchemeng.2011.01.043}{ZeballosNH11} & \hyperref[auth:a621]{L. J. Zeballos}, \hyperref[auth:a524]{J. M. Novas}, \hyperref[auth:a588]{G. P. Henning} & A CP formulation for scheduling multiproduct multistage batch plants & \hyperref[detail:ZeballosNH11]{Details} \href{../works/ZeballosNH11.pdf}{Yes} & \cite{ZeballosNH11} & 2011 & Computers \  Chemical Engineering & 17 & \noindent{}\textbf{1.00} \textbf{1.00} \textbf{28.37} & 26 26 28 & 29 39 & 16 7 9\\
NovasH10 \href{https://doi.org/10.1016/j.compchemeng.2010.07.011}{NovasH10} & \hyperref[auth:a524]{J. M. Novas}, \hyperref[auth:a588]{G. P. Henning} & \cellcolor{green!10}Reactive scheduling framework based on domain knowledge and constraint programming & \hyperref[detail:NovasH10]{Details} \href{../works/NovasH10.pdf}{Yes} & \cite{NovasH10} & 2010 & Computers \  Chemical Engineering & 20 & \noindent{}\textbf{1.00} \textbf{1.00} \textbf{22.75} & 48 49 50 & 19 29 & 4 4 0\\
RoePS05 \href{http://dx.doi.org/10.1016/j.compchemeng.2005.02.024}{RoePS05} & \hyperref[auth:a1241]{B. Roe}, \hyperref[auth:a1242]{L. G. Papageorgiou}, \hyperref[auth:a1243]{N. Shah} & A hybrid MILP/CLP algorithm for multipurpose batch process scheduling & \hyperref[detail:RoePS05]{Details} \href{../works/RoePS05.pdf}{Yes} & \cite{RoePS05} & 2005 & Computers \  Chemical Engineering & 15 & \noindent{}\textbf{1.00} \textbf{1.00} \textbf{16.82} & 48 47 46 & 15 23 & 13 6 7\\
MaraveliasCG04 \href{http://dx.doi.org/10.1016/j.compchemeng.2004.03.016}{MaraveliasCG04} & \hyperref[auth:a381]{C. T. Maravelias}, \hyperref[auth:a382]{I. E. Grossmann} & A hybrid MILP/CP decomposition approach for the continuous time scheduling of multipurpose batch plants & \hyperref[detail:MaraveliasCG04]{Details} \href{../works/MaraveliasCG04.pdf}{Yes} & \cite{MaraveliasCG04} & 2004 & Computers \  Chemical Engineering & 29 & \noindent{}\textbf{1.00} \textbf{1.00} \textbf{49.17} & 116 119 130 & 24 29 & 29 23 6\\
HarjunkoskiG02 \href{http://dx.doi.org/10.1016/s0098-1354(02)00100-x}{HarjunkoskiG02} & \hyperref[auth:a871]{I. Harjunkoski}, \hyperref[auth:a382]{I. E. Grossmann} & Decomposition techniques for multistage scheduling problems using mixed-integer and constraint programming methods & \hyperref[detail:HarjunkoskiG02]{Details} \href{../works/HarjunkoskiG02.pdf}{Yes} & \cite{HarjunkoskiG02} & 2002 & Computers \  Chemical Engineering & 20 & \noindent{}\textbf{1.00} \textbf{1.00} \textbf{20.38} & 169 173 192 & 11 25 & 42 39 3\\
HarjunkoskiJG00 \href{http://dx.doi.org/10.1016/s0098-1354(00)00470-1}{HarjunkoskiJG00} & \hyperref[auth:a871]{I. Harjunkoski}, \hyperref[auth:a844]{V. Jain}, \hyperref[auth:a1160]{I. E. Grossman} & Hybrid mixed-integer/constraint logic programming strategies for solving scheduling and combinatorial optimization problems & \hyperref[detail:HarjunkoskiJG00]{Details} \href{../works/HarjunkoskiJG00.pdf}{Yes} & \cite{HarjunkoskiJG00} & 2000 & Computers \  Chemical Engineering & 7 & \noindent{}\textbf{1.00} \textbf{1.00} \textbf{1.64} & 44 44 49 & 3 9 & 15 15 0\\
Huang2000 \href{http://dx.doi.org/10.1016/s0098-1354(00)00483-x}{Huang2000} & \hyperref[auth:a1648]{W. Huang}, \hyperref[auth:a1649]{P. W. H. Chung} & Scheduling of pipeless batch plants using constraint satisfaction techniques & \hyperref[detail:Huang2000]{Details} No & \cite{Huang2000} & 2000 & Computers \  Chemical Engineering & null & \noindent{}\textbf{1.00} \textbf{1.00} n/a & 15 15 16 & 2 5 & 2 2 0\\
PintoG97 \href{https://www.sciencedirect.com/science/article/pii/S0098135496003183}{PintoG97} & \hyperref[auth:a1255]{J. M. Pinto}, \hyperref[auth:a382]{I. E. Grossmann} & A logic-based approach to scheduling problems with resource constraints & \hyperref[detail:PintoG97]{Details} \href{../works/PintoG97.pdf}{Yes} & \cite{PintoG97} & 1997 & Computers \  Chemical Engineering & 18 & \noindent{}\textcolor{black!50}{0.00} \textcolor{black!50}{0.00} \textcolor{black!50}{0.00} & 56 56 60 & 12 24 & 3 3 0\\
\end{longtable}
}

\subsection{Computers \& Industrial Engineering}

\index{Computers \& Industrial Engineering}
{\scriptsize
\begin{longtable}{>{\raggedright\arraybackslash}p{2.5cm}>{\raggedright\arraybackslash}p{4.5cm}>{\raggedright\arraybackslash}p{6.0cm}p{1.0cm}rr>{\raggedright\arraybackslash}p{2.0cm}r>{\raggedright\arraybackslash}p{1cm}p{1cm}p{1cm}p{1cm}}
\rowcolor{white}\caption{Articles in Journal Computers \  Industrial Engineering (Total 20)}\\ \toprule
\rowcolor{white}\shortstack{Key\\Source} & Authors & Title (Colored by Open Access)& \shortstack{Details\\LC} & Cite & Year & \shortstack{Conference\\/Journal\\/School} & Pages & Relevance &\shortstack{Cites\\OC XR\\SC} & \shortstack{Refs\\OC\\XR} & \shortstack{Links\\Cites\\Refs}\\ \midrule\endhead
\bottomrule
\endfoot
Adelgren2023 \href{http://dx.doi.org/10.1016/j.cie.2023.109330}{Adelgren2023} & \hyperref[auth:a967]{N. Adelgren}, \hyperref[auth:a381]{C. T. Maravelias} & On the utility of production scheduling formulations including record keeping variables & \hyperref[detail:Adelgren2023]{Details} \href{../works/Adelgren2023.pdf}{Yes} & \cite{Adelgren2023} & 2023 & Computers \  Industrial Engineering & 12 & \noindent{}\textcolor{black!50}{0.00} \textcolor{black!50}{0.00} \textbf{1.69} & 0 1 1 & 43 52 & 11 0 11\\
AfsarVPG23 \href{http://dx.doi.org/10.1016/j.cie.2023.109454}{AfsarVPG23} & \hyperref[auth:a961]{S. Afsar}, \hyperref[auth:a962]{C. R. Vela}, \hyperref[auth:a963]{J. J. Palacios}, \hyperref[auth:a964]{I. González-Rodríguez} & \cellcolor{gold!20}Mathematical models and benchmarking for the fuzzy job shop scheduling problem & \hyperref[detail:AfsarVPG23]{Details} \href{../works/AfsarVPG23.pdf}{Yes} & \cite{AfsarVPG23} & 2023 & Computers \  Industrial Engineering & 14 & \noindent{}\textcolor{black!50}{0.00} \textcolor{black!50}{0.00} \textbf{22.09} & 0 0 0 & 50 66 & 7 0 7\\
AlfieriGPS23 \href{https://www.sciencedirect.com/science/article/pii/S0360835223000074}{AlfieriGPS23} & \hyperref[auth:a729]{A. Alfieri}, \hyperref[auth:a15]{M. Garraffa}, \hyperref[auth:a730]{E. Pastore}, \hyperref[auth:a731]{F. Salassa} & \cellcolor{gold!20}Permutation flowshop problems minimizing core waiting time and core idle time \hyperref[abs:AlfieriGPS23]{Abstract} & \hyperref[detail:AlfieriGPS23]{Details} \href{../works/AlfieriGPS23.pdf}{Yes} & \cite{AlfieriGPS23} & 2023 & Computers \  Industrial Engineering & 13 & \noindent{}\textcolor{black!50}{0.00} \textbf{2.00} \textbf{5.78} & 0 2 3 & 37 45 & 2 0 2\\
AbreuN22 \href{https://doi.org/10.1016/j.cie.2022.108128}{AbreuN22} & \hyperref[auth:a418]{L. R. de Abreu}, \hyperref[auth:a419]{M. S. Nagano} & A new hybridization of adaptive large neighborhood search with constraint programming for open shop scheduling with sequence-dependent setup times & \hyperref[detail:AbreuN22]{Details} \href{../works/AbreuN22.pdf}{Yes} & \cite{AbreuN22} & 2022 & Computers \  Industrial Engineering & 20 & \noindent{}\textbf{1.00} \textbf{1.00} \textbf{57.71} & 10 14 13 & 56 74 & 7 2 5\\
HeinzNVH22 \href{https://doi.org/10.1016/j.cie.2022.108586}{HeinzNVH22} & \hyperref[auth:a433]{V. Heinz}, \hyperref[auth:a434]{A. Nov{\'{a}}k}, \hyperref[auth:a311]{M. Vlk}, \hyperref[auth:a116]{Z. Hanz{\'{a}}lek} & \cellcolor{green!10}Constraint Programming and constructive heuristics for parallel machine scheduling with sequence-dependent setups and common servers & \hyperref[detail:HeinzNVH22]{Details} \href{../works/HeinzNVH22.pdf}{Yes} & \cite{HeinzNVH22} & 2022 & Computers \  Industrial Engineering & 16 & \noindent{}\textbf{1.50} \textbf{1.50} \textbf{40.68} & 5 7 8 & 25 31 & 7 3 4\\
VlkHT21 \href{https://doi.org/10.1016/j.cie.2021.107317}{VlkHT21} & \hyperref[auth:a311]{M. Vlk}, \hyperref[auth:a116]{Z. Hanz{\'{a}}lek}, \hyperref[auth:a475]{S. Tang} & Constraint programming approaches to joint routing and scheduling in time-sensitive networks & \hyperref[detail:VlkHT21]{Details} \href{../works/VlkHT21.pdf}{Yes} & \cite{VlkHT21} & 2021 & Computers \  Industrial Engineering & 14 & \noindent{}\textbf{1.00} \textbf{1.00} \textbf{5.93} & 7 18 20 & 22 36 & 3 0 3\\
FachiniA20 \href{http://dx.doi.org/10.1016/j.cie.2020.106641}{FachiniA20} & \hyperref[auth:a1023]{R. F. Fachini}, \hyperref[auth:a1024]{V. A. Armentano} & Logic-based Benders decomposition for the heterogeneous fixed fleet vehicle routing problem with time windows & \hyperref[detail:FachiniA20]{Details} \href{../works/FachiniA20.pdf}{Yes} & \cite{FachiniA20} & 2020 & Computers \  Industrial Engineering & 18 & \noindent{}\textcolor{black!50}{0.00} \textcolor{black!50}{0.00} \textcolor{black!50}{0.20} & 25 26 35 & 55 68 & 12 2 10\\
HauderBRPA20 \href{http://dx.doi.org/10.1016/j.cie.2020.106857}{HauderBRPA20} & \hyperref[auth:a550]{V. A. Hauder}, \hyperref[auth:a551]{A. Beham}, \hyperref[auth:a552]{S. Raggl}, \hyperref[auth:a553]{S. N. Parragh}, \hyperref[auth:a554]{M. Affenzeller} & \cellcolor{green!10}Resource-constrained multi-project scheduling with activity and time flexibility & \hyperref[detail:HauderBRPA20]{Details} \href{../works/HauderBRPA20.pdf}{Yes} & \cite{HauderBRPA20} & 2020 & Computers \  Industrial Engineering & 14 & \noindent{}\textcolor{black!50}{0.00} \textcolor{black!50}{0.00} \textbf{25.41} & 14 19 27 & 46 56 & 17 3 14\\
MengZRZL20 \href{https://doi.org/10.1016/j.cie.2020.106347}{MengZRZL20} & \hyperref[auth:a500]{L. Meng}, \hyperref[auth:a501]{C. Zhang}, \hyperref[auth:a502]{Y. Ren}, \hyperref[auth:a503]{B. Zhang}, \hyperref[auth:a504]{C. Lv} & Mixed-integer linear programming and constraint programming formulations for solving distributed flexible job shop scheduling problem & \hyperref[detail:MengZRZL20]{Details} \href{../works/MengZRZL20.pdf}{Yes} & \cite{MengZRZL20} & 2020 & Computers \  Industrial Engineering & 13 & \noindent{}\textbf{2.00} \textbf{2.00} \textbf{36.82} & 100 133 152 & 62 69 & 26 16 10\\
Novas19 \href{https://doi.org/10.1016/j.cie.2019.07.011}{Novas19} & \hyperref[auth:a524]{J. M. Novas} & Production scheduling and lot streaming at flexible job-shops environments using constraint programming & \hyperref[detail:Novas19]{Details} \href{../works/Novas19.pdf}{Yes} & \cite{Novas19} & 2019 & Computers \  Industrial Engineering & 13 & \noindent{}\textbf{2.00} \textbf{2.00} \textbf{8.25} & 30 35 43 & 29 40 & 10 4 6\\
GedikKEK18 \href{https://doi.org/10.1016/j.cie.2018.05.014}{GedikKEK18} & \hyperref[auth:a560]{R. Gedik}, \hyperref[auth:a561]{D. Kalathia}, \hyperref[auth:a562]{G. Egilmez}, \hyperref[auth:a563]{E. Kirac} & A constraint programming approach for solving unrelated parallel machine scheduling problem & \hyperref[detail:GedikKEK18]{Details} \href{../works/GedikKEK18.pdf}{Yes} & \cite{GedikKEK18} & 2018 & Computers \  Industrial Engineering & 11 & \noindent{}\textbf{1.50} \textbf{1.50} \textbf{24.08} & 43 49 47 & 22 31 & 23 20 3\\
GedikKBR17 \href{http://dx.doi.org/10.1016/j.cie.2017.03.017}{GedikKBR17} & \hyperref[auth:a560]{R. Gedik}, \hyperref[auth:a563]{E. Kirac}, \hyperref[auth:a1155]{A. B. Milburn}, \hyperref[auth:a1156]{C. Rainwater} & \cellcolor{green!10}A constraint programming approach for the team orienteering problem with time windows & \hyperref[detail:GedikKBR17]{Details} \href{../works/GedikKBR17.pdf}{Yes} & \cite{GedikKBR17} & 2017 & Computers \  Industrial Engineering & 18 & \noindent{}\textcolor{black!50}{0.00} \textcolor{black!50}{0.00} \textbf{3.60} & 20 23 26 & 32 47 & 14 7 7\\
HamC16 \href{http://dx.doi.org/10.1016/j.cie.2016.11.001}{HamC16} & \hyperref[auth:a770]{A. M. Ham}, \hyperref[auth:a875]{E. Cakici} & Flexible job shop scheduling problem with parallel batch processing machines: MIP and CP approaches & \hyperref[detail:HamC16]{Details} \href{../works/HamC16.pdf}{Yes} & \cite{HamC16} & 2016 & Computers \  Industrial Engineering & 6 & \noindent{}\textbf{2.50} \textbf{2.50} \textbf{8.71} & 50 55 55 & 26 35 & 19 16 3\\
SuCC13 \href{http://dx.doi.org/10.1016/j.cie.2013.02.021}{SuCC13} & \hyperref[auth:a1400]{L.-H. Su}, \hyperref[auth:a1401]{Y. Chiu}, \hyperref[auth:a1402]{T. C. E. Cheng} & Sports tournament scheduling to determine the required number of venues subject to the minimum timeslots under given formats \hyperref[abs:SuCC13]{Abstract} & \hyperref[detail:SuCC13]{Details} \href{../works/SuCC13.pdf}{Yes} & \cite{SuCC13} & 2013 & Computers \  Industrial Engineering & 7 & \noindent{}\textcolor{black!50}{0.00} \textbf{1.00} 0.27 & 2 2 4 & 15 16 & 4 0 4\\
TerekhovDOB12 \href{https://doi.org/10.1016/j.cie.2012.02.006}{TerekhovDOB12} & \hyperref[auth:a818]{D. Terekhov}, \hyperref[auth:a820]{M. K. Dogru}, \hyperref[auth:a821]{U. {\"{O}}zen}, \hyperref[auth:a89]{J. C. Beck} & Solving two-machine assembly scheduling problems with inventory constraints & \hyperref[detail:TerekhovDOB12]{Details} \href{../works/TerekhovDOB12.pdf}{Yes} & \cite{TerekhovDOB12} & 2012 & Computers \  Industrial Engineering & 15 & \noindent{}\textcolor{black!50}{0.00} \textcolor{black!50}{0.00} \textbf{9.01} & 8 9 16 & 48 59 & 9 2 7\\
TrojetHL11 \href{https://doi.org/10.1016/j.cie.2010.08.014}{TrojetHL11} & \hyperref[auth:a705]{M. Trojet}, \hyperref[auth:a706]{F. H'Mida}, \hyperref[auth:a3]{P. Lopez} & \cellcolor{green!10}Project scheduling under resource constraints: Application of the cumulative global constraint in a decision support framework & \hyperref[detail:TrojetHL11]{Details} \href{../works/TrojetHL11.pdf}{Yes} & \cite{TrojetHL11} & 2011 & Computers \  Industrial Engineering & 7 & \noindent{}\textcolor{black!50}{0.00} \textcolor{black!50}{0.00} \textbf{5.05} & 11 13 12 & 17 32 & 7 3 4\\
Hindi2004 \href{http://dx.doi.org/10.1016/j.cie.2004.03.002}{Hindi2004} & \hyperref[auth:a1826]{K. S. Hindi}, \hyperref[auth:a1827]{K. Fleszar} & A constraint propagation heuristic for the single-hoist, multiple-products scheduling problem & \hyperref[detail:Hindi2004]{Details} No & \cite{Hindi2004} & 2004 & Computers \  Industrial Engineering & null & \noindent{}\textbf{1.50} \textbf{1.50} n/a & 29 30 34 & 8 12 & 1 1 0\\
Kim2004 \href{http://dx.doi.org/10.1016/j.cie.2003.12.017}{Kim2004} & \hyperref[auth:a2029]{K. H. Kim}, \hyperref[auth:a2030]{K. W. Kim}, \hyperref[auth:a2031]{H. Hwang}, \hyperref[auth:a2032]{C. S. Ko} & Operator-scheduling using a constraint satisfaction technique in port container terminals & \hyperref[detail:Kim2004]{Details} No & \cite{Kim2004} & 2004 & Computers \  Industrial Engineering & null & \noindent{}\textbf{1.00} \textbf{1.00} n/a & 22 23 29 & 9 9 & 1 0 1\\
YunG02 \href{http://dx.doi.org/10.1016/s0360-8352(02)00065-7}{YunG02} & \hyperref[auth:a1472]{Y.-S. Yun}, \hyperref[auth:a1473]{M. Gen} & Advanced scheduling problem using constraint programming techniques in SCM environment & \hyperref[detail:YunG02]{Details} No & \cite{YunG02} & 2002 & Computers \  Industrial Engineering & 17 & \noindent{}\textbf{1.00} \textbf{1.00} n/a & 19 20 27 & 6 19 & 5 5 0\\
TrentesauxPT01 \href{https://www.sciencedirect.com/science/article/pii/S0360835200000784}{TrentesauxPT01} & \hyperref[auth:a1457]{D. Trentesaux}, \hyperref[auth:a1458]{P. Pesin}, \hyperref[auth:a1459]{C. Tahon} & Comparison of constraint logic programming and distributed problem solving: a case study for interactive, efficient and practicable job-shop scheduling \hyperref[abs:TrentesauxPT01]{Abstract} & \hyperref[detail:TrentesauxPT01]{Details} \href{../works/TrentesauxPT01.pdf}{Yes} & \cite{TrentesauxPT01} & 2001 & Computers \  Industrial Engineering & 25 & \noindent{}\textbf{2.00} \textbf{5.00} \textbf{22.74} & 7 7 9 & 9 26 & 3 1 2\\
\end{longtable}
}

\subsection{Computers \& Operations Research}

\index{Computers \& Operations Research}
{\scriptsize
\begin{longtable}{>{\raggedright\arraybackslash}p{2.5cm}>{\raggedright\arraybackslash}p{4.5cm}>{\raggedright\arraybackslash}p{6.0cm}p{1.0cm}rr>{\raggedright\arraybackslash}p{2.0cm}r>{\raggedright\arraybackslash}p{1cm}p{1cm}p{1cm}p{1cm}}
\rowcolor{white}\caption{Articles in Journal Computers \  Operations Research (Total 22)}\\ \toprule
\rowcolor{white}\shortstack{Key\\Source} & Authors & Title (Colored by Open Access)& \shortstack{Details\\LC} & Cite & Year & \shortstack{Conference\\/Journal\\/School} & Pages & Relevance &\shortstack{Cites\\OC XR\\SC} & \shortstack{Refs\\OC\\XR} & \shortstack{Links\\Cites\\Refs}\\ \midrule\endhead
\bottomrule
\endfoot
AbreuPNF23 \href{https://www.sciencedirect.com/science/article/pii/S0305054823002502}{AbreuPNF23} & \hyperref[auth:a386]{L. R. Abreu}, \hyperref[auth:a385]{B. A. Prata}, \hyperref[auth:a387]{M. S. Nagano}, \hyperref[auth:a833]{J. M. Framinan} & A constraint programming-based iterated greedy algorithm for the open shop with sequence-dependent processing times and makespan minimization \hyperref[abs:AbreuPNF23]{Abstract} & \hyperref[detail:AbreuPNF23]{Details} \href{../works/AbreuPNF23.pdf}{Yes} & \cite{AbreuPNF23} & 2023 & Computers \  Operations Research & 12 & \noindent{}\textcolor{black!50}{0.00} \textbf{5.00} \textbf{28.79} & 0 3 3 & 46 68 & 15 0 15\\
JuvinHL23a \href{http://dx.doi.org/10.1016/j.cor.2023.106156}{JuvinHL23a} & \hyperref[auth:a0]{C. Juvin}, \hyperref[auth:a2]{L. Houssin}, \hyperref[auth:a3]{P. Lopez} & \cellcolor{green!10}Logic-based Benders decomposition for the preemptive flexible job-shop scheduling problem & \hyperref[detail:JuvinHL23a]{Details} \href{../works/JuvinHL23a.pdf}{Yes} & \cite{JuvinHL23a} & 2023 & Computers \  Operations Research & 17 & \noindent{}\textcolor{black!50}{0.00} \textcolor{black!50}{0.00} \textbf{9.23} & 0 3 4 & 40 53 & 15 0 15\\
PenzDN23 \href{https://doi.org/10.1016/j.cor.2022.106092}{PenzDN23} & \hyperref[auth:a992]{L. Penz}, \hyperref[auth:a993]{S. Dauz{\`{e}}re-P{\'{e}}r{\`{e}}s}, \hyperref[auth:a81]{M. Nattaf} & \cellcolor{gold!20}Minimizing the sum of completion times on a single machine with health index and flexible maintenance operations & \hyperref[detail:PenzDN23]{Details} \href{../works/PenzDN23.pdf}{Yes} & \cite{PenzDN23} & 2023 & Computers \  Operations Research & 13 & \noindent{}\textcolor{black!50}{0.00} \textcolor{black!50}{0.00} \textcolor{black!50}{0.00} & 0 3 1 & 34 36 & 1 0 1\\
NaderiBZ22a \href{http://dx.doi.org/10.1016/j.cor.2022.105728}{NaderiBZ22a} & \hyperref[auth:a726]{B. Naderi}, \hyperref[auth:a836]{M. A. Begen}, \hyperref[auth:a838]{G. S. Zaric} & Type-2 integrated process-planning and scheduling problem: Reformulation and solution algorithms & \hyperref[detail:NaderiBZ22a]{Details} \href{../works/NaderiBZ22a.pdf}{Yes} & \cite{NaderiBZ22a} & 2022 & Computers \  Operations Research & 19 & \noindent{}\textcolor{black!50}{0.00} \textcolor{black!50}{0.00} \textbf{22.24} & 3 4 4 & 44 54 & 17 2 15\\
Edis21 \href{http://dx.doi.org/10.1016/j.cor.2020.105111}{Edis21} & \hyperref[auth:a346]{E. B. Edis} & Constraint programming approaches to disassembly line balancing problem with sequencing decisions & \hyperref[detail:Edis21]{Details} \href{../works/Edis21.pdf}{Yes} & \cite{Edis21} & 2021 & Computers \  Operations Research & 20 & \noindent{}\textcolor{black!50}{0.00} \textcolor{black!50}{0.00} \textbf{60.40} & 13 19 20 & 48 53 & 10 2 8\\
FanXG21 \href{https://doi.org/10.1016/j.cor.2021.105401}{FanXG21} & \hyperref[auth:a476]{H. Fan}, \hyperref[auth:a477]{H. Xiong}, \hyperref[auth:a478]{M. Goh} & Genetic programming-based hyper-heuristic approach for solving dynamic job shop scheduling problem with extended technical precedence constraints & \hyperref[detail:FanXG21]{Details} \href{../works/FanXG21.pdf}{Yes} & \cite{FanXG21} & 2021 & Computers \  Operations Research & 15 & \noindent{}\textcolor{black!50}{0.00} \textcolor{black!50}{0.00} \textcolor{black!50}{0.00} & 18 27 30 & 57 68 & 1 0 1\\
ZhangYW21 \href{https://doi.org/10.1016/j.cor.2021.105282}{ZhangYW21} & \hyperref[auth:a479]{L. Zhang}, \hyperref[auth:a480]{C. Yu}, \hyperref[auth:a481]{T. N. Wong} & A graph-based constraint programming approach for the integrated process planning and scheduling problem & \hyperref[detail:ZhangYW21]{Details} \href{../works/ZhangYW21.pdf}{Yes} & \cite{ZhangYW21} & 2021 & Computers \  Operations Research & 10 & \noindent{}\textbf{1.00} \textbf{1.00} \textbf{9.87} & 6 7 8 & 35 41 & 14 3 11\\
AbidinK20 \href{http://dx.doi.org/10.1016/j.cor.2020.105069}{AbidinK20} & \hyperref[auth:a1381]{Z. A. Cil}, \hyperref[auth:a1380]{D. Kizilay} & Constraint programming model for multi-manned assembly line balancing problem & \hyperref[detail:AbidinK20]{Details} \href{../works/AbidinK20.pdf}{Yes} & \cite{AbidinK20} & 2020 & Computers \  Operations Research & 14 & \noindent{}\textcolor{black!50}{0.00} \textcolor{black!50}{0.00} \textbf{13.74} & 11 14 0 & 27 35 & 6 1 5\\
AstrandJZ20 \href{https://doi.org/10.1016/j.cor.2020.105036}{AstrandJZ20} & \hyperref[auth:a74]{M. {\AA}strand}, \hyperref[auth:a75]{M. Johansson}, \hyperref[auth:a199]{A. Zanarini} & Underground mine scheduling of mobile machines using Constraint Programming and Large Neighborhood Search & \hyperref[detail:AstrandJZ20]{Details} \href{../works/AstrandJZ20.pdf}{Yes} & \cite{AstrandJZ20} & 2020 & Computers \  Operations Research & 13 & \noindent{}\textbf{1.50} \textbf{1.50} \textbf{22.04} & 16 19 19 & 24 53 & 9 1 8\\
LunardiBLRV20 \href{https://doi.org/10.1016/j.cor.2020.105020}{LunardiBLRV20} & \hyperref[auth:a505]{W. T. Lunardi}, \hyperref[auth:a506]{E. G. Birgin}, \hyperref[auth:a118]{P. Laborie}, \hyperref[auth:a507]{D. P. Ronconi}, \hyperref[auth:a508]{H. Voos} & \cellcolor{green!10}Mixed Integer linear programming and constraint programming models for the online printing shop scheduling problem & \hyperref[detail:LunardiBLRV20]{Details} \href{../works/LunardiBLRV20.pdf}{Yes} & \cite{LunardiBLRV20} & 2020 & Computers \  Operations Research & 20 & \noindent{}\textbf{1.00} \textbf{1.00} \textbf{46.75} & 30 36 39 & 18 24 & 16 13 3\\
NattafDYW19 \href{https://doi.org/10.1016/j.cor.2019.03.004}{NattafDYW19} & \hyperref[auth:a81]{M. Nattaf}, \hyperref[auth:a993]{S. Dauz{\`{e}}re-P{\'{e}}r{\`{e}}s}, \hyperref[auth:a994]{C. Yugma}, \hyperref[auth:a995]{C.-H. Wu} & \cellcolor{gold!20}Parallel machine scheduling with time constraints on machine qualifications & \hyperref[detail:NattafDYW19]{Details} \href{../works/NattafDYW19.pdf}{Yes} & \cite{NattafDYW19} & 2019 & Computers \  Operations Research & 16 & \noindent{}\textcolor{black!50}{0.00} \textcolor{black!50}{0.00} \textbf{21.82} & 14 22 22 & 21 29 & 6 3 3\\
KuB16 \href{https://doi.org/10.1016/j.cor.2016.04.006}{KuB16} & \hyperref[auth:a331]{W.-Y. Ku}, \hyperref[auth:a89]{J. C. Beck} & \cellcolor{green!10}Mixed Integer Programming models for job shop scheduling: {A} computational analysis & \hyperref[detail:KuB16]{Details} \href{../works/KuB16.pdf}{Yes} & \cite{KuB16} & 2016 & Computers \  Operations Research & 9 & \noindent{}\textcolor{black!50}{0.00} \textcolor{black!50}{0.00} \textbf{5.53} & 119 132 141 & 17 25 & 25 19 6\\
ZengM12 \href{http://dx.doi.org/10.1016/j.cor.2011.10.004}{ZengM12} & \hyperref[auth:a1404]{L. Zeng}, \hyperref[auth:a1405]{S. Mizuno} & On the separation in 2-period double round robin tournaments with minimum breaks \hyperref[abs:ZengM12]{Abstract} & \hyperref[detail:ZengM12]{Details} \href{../works/ZengM12.pdf}{Yes} & \cite{ZengM12} & 2012 & Computers \  Operations Research & 9 & \noindent{}\textcolor{black!50}{0.00} \textbf{1.00} \textbf{2.60} & 3 3 4 & 18 25 & 5 0 5\\
Deblaere2011 \href{http://dx.doi.org/10.1016/j.cor.2010.01.001}{Deblaere2011} & \hyperref[auth:a1775]{F. Deblaere}, \hyperref[auth:a1090]{E. Demeulemeester}, \hyperref[auth:a1102]{W. Herroelen} & Reactive scheduling in the multi-mode RCPSP & \hyperref[detail:Deblaere2011]{Details} No & \cite{Deblaere2011} & 2011 & Computers \  Operations Research & null & \noindent{}\textcolor{black!50}{0.00} \textcolor{black!50}{0.00} n/a & 77 85 103 & 33 41 & 8 5 3\\
TopalogluO11 \href{https://doi.org/10.1016/j.cor.2010.04.018}{TopalogluO11} & \hyperref[auth:a617]{S. Topaloglu}, \hyperref[auth:a348]{I. Ozkarahan} & A constraint programming-based solution approach for medical resident scheduling problems & \hyperref[detail:TopalogluO11]{Details} \href{../works/TopalogluO11.pdf}{Yes} & \cite{TopalogluO11} & 2011 & Computers \  Operations Research & 10 & \noindent{}\textbf{1.00} \textbf{1.00} \textbf{2.44} & 46 47 59 & 24 32 & 14 11 3\\
KendallKRU10 \href{http://dx.doi.org/10.1016/j.cor.2009.05.013}{KendallKRU10} & \hyperref[auth:a1387]{G. Kendall}, \hyperref[auth:a1166]{S. Knust}, \hyperref[auth:a1386]{C. C. Ribeiro}, \hyperref[auth:a1388]{S. Urrutia} & Scheduling in sports: An annotated bibliography \hyperref[abs:KendallKRU10]{Abstract} & \hyperref[detail:KendallKRU10]{Details} \href{../works/KendallKRU10.pdf}{Yes} & \cite{KendallKRU10} & 2010 & Computers \  Operations Research & 19 & \noindent{}\textcolor{black!50}{0.00} \textcolor{black!50}{0.00} \textbf{9.51} & 181 186 220 & 0 0 & 6 6 0\\
WuBB09 \href{https://doi.org/10.1016/j.cor.2008.08.008}{WuBB09} & \hyperref[auth:a274]{C. W. Wu}, \hyperref[auth:a217]{K. N. Brown}, \hyperref[auth:a89]{J. C. Beck} & Scheduling with uncertain durations: Modeling beta-robust scheduling with constraints & \hyperref[detail:WuBB09]{Details} \href{../works/WuBB09.pdf}{Yes} & \cite{WuBB09} & 2009 & Computers \  Operations Research & 9 & \noindent{}\textcolor{black!50}{0.00} \textcolor{black!50}{0.00} \textbf{3.11} & 42 42 48 & 5 16 & 3 2 1\\
ClautiauxJCM08 \href{http://dx.doi.org/10.1016/j.cor.2006.05.012}{ClautiauxJCM08} & \hyperref[auth:a1169]{F. Clautiaux}, \hyperref[auth:a929]{A. Jouglet}, \hyperref[auth:a845]{J. Carlier}, \hyperref[auth:a1170]{A. Moukrim} & A new constraint programming approach for the orthogonal packing problem & \hyperref[detail:ClautiauxJCM08]{Details} \href{../works/ClautiauxJCM08.pdf}{Yes} & \cite{ClautiauxJCM08} & 2008 & Computers \  Operations Research & 16 & \noindent{}\textcolor{black!50}{0.00} \textcolor{black!50}{0.00} \textbf{2.75} & 64 65 70 & 14 26 & 9 6 3\\
CorreaLR07 \href{http://dx.doi.org/10.1016/j.cor.2005.07.004}{CorreaLR07} & \hyperref[auth:a948]{A. I. Corr{\'{e}}a}, \hyperref[auth:a645]{A. Langevin}, \hyperref[auth:a326]{L.-M. Rousseau} & Scheduling and routing of automated guided vehicles: A hybrid approach & \hyperref[detail:CorreaLR07]{Details} \href{../works/CorreaLR07.pdf}{Yes} & \cite{CorreaLR07} & 2007 & Computers \  Operations Research & 20 & \noindent{}\textcolor{black!50}{0.00} \textcolor{black!50}{0.00} \textbf{9.26} & 106 114 137 & 20 28 & 13 4 9\\
BockmayrP06 \href{http://dx.doi.org/10.1016/j.cor.2005.01.010}{BockmayrP06} & \hyperref[auth:a908]{A. Bockmayr}, \hyperref[auth:a1178]{N. Pisaruk} & Detecting infeasibility and generating cuts for mixed integer programming using constraint programming & \hyperref[detail:BockmayrP06]{Details} \href{../works/BockmayrP06.pdf}{Yes} & \cite{BockmayrP06} & 2006 & Computers \  Operations Research & 10 & \noindent{}\textcolor{black!50}{0.00} \textcolor{black!50}{0.00} \textbf{2.66} & 12 12 10 & 7 10 & 9 4 5\\
Gronkvist06 \href{http://dx.doi.org/10.1016/j.cor.2005.01.017}{Gronkvist06} & \hyperref[auth:a1214]{M. Gr\"{o}nkvist} & Accelerating column generation for aircraft scheduling using constraint propagation & \hyperref[detail:Gronkvist06]{Details} \href{../works/Gronkvist06.pdf}{Yes} & \cite{Gronkvist06} & 2006 & Computers \  Operations Research & 17 & \noindent{}\textbf{1.50} \textbf{1.50} \textbf{2.29} & 28 28 36 & 15 30 & 7 5 2\\
RussellU06 \href{http://dx.doi.org/10.1016/j.cor.2004.09.029}{RussellU06} & \hyperref[auth:a1433]{R. A. Russell}, \hyperref[auth:a1434]{T. L. Urban} & A constraint programming approach to the multiple-venue,  sport-scheduling problem & \hyperref[detail:RussellU06]{Details} \href{../works/RussellU06.pdf}{Yes} & \cite{RussellU06} & 2006 & Computers \  Operations Research & 12 & \noindent{}\textbf{1.00} \textbf{1.00} \textbf{2.33} & 22 22 0 & 16 22 & 11 6 5\\
\end{longtable}
}

\subsection{Computers in Railways VIII}

\index{Computers in Railways VIII}
{\scriptsize
\begin{longtable}{>{\raggedright\arraybackslash}p{2.5cm}>{\raggedright\arraybackslash}p{4.5cm}>{\raggedright\arraybackslash}p{6.0cm}p{1.0cm}rr>{\raggedright\arraybackslash}p{2.0cm}r>{\raggedright\arraybackslash}p{1cm}p{1cm}p{1cm}p{1cm}}
\rowcolor{white}\caption{Articles in Journal Computers in Railways VIII (Total 1)}\\ \toprule
\rowcolor{white}\shortstack{Key\\Source} & Authors & Title (Colored by Open Access)& \shortstack{Details\\LC} & Cite & Year & \shortstack{Conference\\/Journal\\/School} & Pages & Relevance &\shortstack{Cites\\OC XR\\SC} & \shortstack{Refs\\OC\\XR} & \shortstack{Links\\Cites\\Refs}\\ \midrule\endhead
\bottomrule
\endfoot
RodriguezDG02 \href{}{RodriguezDG02} & \hyperref[auth:a781]{J. Rodriguez}, \hyperref[auth:a782]{X. Delorme}, \hyperref[auth:a783]{X. Gandibleux} & Railway infrastructure saturation using constraint programming approach & \hyperref[detail:RodriguezDG02]{Details} \href{../works/RodriguezDG02.pdf}{Yes} & \cite{RodriguezDG02} & 2002 & Computers in Railways VIII & 10 & \noindent{}\textcolor{black!50}{0.00} \textcolor{black!50}{0.00} 0.38 & 0 0 0 & 0 0 & 0 0 0\\
\end{longtable}
}

\subsection{Concurrency and Computation: Practice and Experience}

\index{Concurrency and Computation: Practice and Experience}
{\scriptsize
\begin{longtable}{>{\raggedright\arraybackslash}p{2.5cm}>{\raggedright\arraybackslash}p{4.5cm}>{\raggedright\arraybackslash}p{6.0cm}p{1.0cm}rr>{\raggedright\arraybackslash}p{2.0cm}r>{\raggedright\arraybackslash}p{1cm}p{1cm}p{1cm}p{1cm}}
\rowcolor{white}\caption{Articles in Journal Concurrency and Computation: Practice and Experience (Total 1)}\\ \toprule
\rowcolor{white}\shortstack{Key\\Source} & Authors & Title (Colored by Open Access)& \shortstack{Details\\LC} & Cite & Year & \shortstack{Conference\\/Journal\\/School} & Pages & Relevance &\shortstack{Cites\\OC XR\\SC} & \shortstack{Refs\\OC\\XR} & \shortstack{Links\\Cites\\Refs}\\ \midrule\endhead
\bottomrule
\endfoot
Menouer2016 \href{http://dx.doi.org/10.1002/cpe.3840}{Menouer2016} & \hyperref[auth:a1976]{T. Menouer}, \hyperref[auth:a1977]{N. Sukhija}, \hyperref[auth:a1978]{L. C. Bertrand} & A learning Portfolio solver for optimizing the performance of constraint programming problems on multi‐core computing systems \hyperref[abs:Menouer2016]{Abstract} & \hyperref[detail:Menouer2016]{Details} No & \cite{Menouer2016} & 2016 & Concurrency and Computation: Practice and Experience & null & \noindent{}\textcolor{black!50}{0.00} \textbf{4.50} n/a & 2 2 1 & 17 38 & 1 0 1\\
\end{longtable}
}

\subsection{Concurrent Engineering}

\index{Concurrent Engineering}
{\scriptsize
\begin{longtable}{>{\raggedright\arraybackslash}p{2.5cm}>{\raggedright\arraybackslash}p{4.5cm}>{\raggedright\arraybackslash}p{6.0cm}p{1.0cm}rr>{\raggedright\arraybackslash}p{2.0cm}r>{\raggedright\arraybackslash}p{1cm}p{1cm}p{1cm}p{1cm}}
\rowcolor{white}\caption{Articles in Journal Concurrent Engineering (Total 1)}\\ \toprule
\rowcolor{white}\shortstack{Key\\Source} & Authors & Title (Colored by Open Access)& \shortstack{Details\\LC} & Cite & Year & \shortstack{Conference\\/Journal\\/School} & Pages & Relevance &\shortstack{Cites\\OC XR\\SC} & \shortstack{Refs\\OC\\XR} & \shortstack{Links\\Cites\\Refs}\\ \midrule\endhead
\bottomrule
\endfoot
Li2014a \href{http://dx.doi.org/10.1177/1063293x14553809}{Li2014a} & \hyperref[auth:a2002]{Y. Li}, \hyperref[auth:a2003]{W. Zhao} & An integrated change propagation scheduling approach for product design \hyperref[abs:Li2014a]{Abstract} & \hyperref[detail:Li2014a]{Details} No & \cite{Li2014a} & 2014 & Concurrent Engineering & null & \noindent{}0.50 \textbf{1.50} n/a & 25 28 36 & 31 37 & 1 0 1\\
\end{longtable}
}

\subsection{Constraints An Int. J.}

\index{Constraints An Int. J.}
{\scriptsize
\begin{longtable}{>{\raggedright\arraybackslash}p{2.5cm}>{\raggedright\arraybackslash}p{4.5cm}>{\raggedright\arraybackslash}p{6.0cm}p{1.0cm}rr>{\raggedright\arraybackslash}p{2.0cm}r>{\raggedright\arraybackslash}p{1cm}p{1cm}p{1cm}p{1cm}}
\rowcolor{white}\caption{Articles in Journal Constraints An Int. J. (Total 47)}\\ \toprule
\rowcolor{white}\shortstack{Key\\Source} & Authors & Title (Colored by Open Access)& \shortstack{Details\\LC} & Cite & Year & \shortstack{Conference\\/Journal\\/School} & Pages & Relevance &\shortstack{Cites\\OC XR\\SC} & \shortstack{Refs\\OC\\XR} & \shortstack{Links\\Cites\\Refs}\\ \midrule\endhead
\bottomrule
\endfoot
Caballero23 \href{https://doi.org/10.1007/s10601-023-09357-0}{Caballero23} & \hyperref[auth:a102]{J. C. Caballero} & Scheduling through logic-based tools & \hyperref[detail:Caballero23]{Details} \href{../works/Caballero23.pdf}{Yes} & \cite{Caballero23} & 2023 & Constraints An Int. J. & 1 & \noindent{}\textcolor{black!50}{0.00} \textcolor{black!50}{0.00} \textcolor{black!50}{0.00} & 0 0 0 & 0 0 & 0 0 0\\
LacknerMMWW23 \href{https://doi.org/10.1007/s10601-023-09347-2}{LacknerMMWW23} & \hyperref[auth:a62]{M.-L. Lackner}, \hyperref[auth:a63]{C. Mrkvicka}, \hyperref[auth:a45]{N. Musliu}, \hyperref[auth:a46]{D. Walkiewicz}, \hyperref[auth:a43]{F. Winter} & \cellcolor{gold!20}Exact methods for the Oven Scheduling Problem & \hyperref[detail:LacknerMMWW23]{Details} \href{../works/LacknerMMWW23.pdf}{Yes} & \cite{LacknerMMWW23} & 2023 & Constraints An Int. J. & 42 & \noindent{}\textcolor{black!50}{0.00} \textcolor{black!50}{0.00} \textbf{46.45} & 0 1 0 & 32 38 & 8 0 8\\
WessenCSFPM23 \href{https://doi.org/10.1007/s10601-023-09345-4}{WessenCSFPM23} & \hyperref[auth:a90]{J. Wess{\'{e}}n}, \hyperref[auth:a91]{M. Carlsson}, \hyperref[auth:a92]{C. Schulte}, \hyperref[auth:a1416]{P. Flener}, \hyperref[auth:a1417]{F. Pecora}, \hyperref[auth:a1418]{M. Matskin} & \cellcolor{gold!20}A constraint programming model for the scheduling and workspace layout design of a dual-arm multi-tool assembly robot & \hyperref[detail:WessenCSFPM23]{Details} \href{../works/WessenCSFPM23.pdf}{Yes} & \cite{WessenCSFPM23} & 2023 & Constraints An Int. J. & 34 & \noindent{}\textbf{1.00} \textbf{1.00} \textbf{6.97} & 0 0 0 & 38 50 & 6 0 6\\
CampeauG22 \href{https://doi.org/10.1007/s10601-022-09337-w}{CampeauG22} & \hyperref[auth:a103]{L.-P. Campeau}, \hyperref[auth:a9]{M. Gamache} & Short- and medium-term optimization of underground mine planning using constraint programming & \hyperref[detail:CampeauG22]{Details} \href{../works/CampeauG22.pdf}{Yes} & \cite{CampeauG22} & 2022 & Constraints An Int. J. & 18 & \noindent{}\textcolor{black!50}{0.00} \textcolor{black!50}{0.00} \textbf{4.09} & 0 0 1 & 22 26 & 5 0 5\\
KoehlerBFFHPSSS21 \href{https://doi.org/10.1007/s10601-021-09321-w}{KoehlerBFFHPSSS21} & \hyperref[auth:a104]{J. Koehler}, \hyperref[auth:a105]{J. B{\"{u}}rgler}, \hyperref[auth:a106]{U. Fontana}, \hyperref[auth:a107]{E. Fux}, \hyperref[auth:a108]{F. A. Herzog}, \hyperref[auth:a109]{M. Pouly}, \hyperref[auth:a110]{S. Saller}, \hyperref[auth:a111]{A. Salyaeva}, \hyperref[auth:a112]{P. Scheiblechner}, \hyperref[auth:a113]{K. Waelti} & \cellcolor{gold!20}Cable tree wiring - benchmarking solvers on a real-world scheduling problem with a variety of precedence constraints & \hyperref[detail:KoehlerBFFHPSSS21]{Details} \href{../works/KoehlerBFFHPSSS21.pdf}{Yes} & \cite{KoehlerBFFHPSSS21} & 2021 & Constraints An Int. J. & 51 & \noindent{}\textcolor{black!50}{0.00} \textcolor{black!50}{0.00} \textbf{16.91} & 2 3 2 & 52 66 & 6 0 6\\
BenediktMH20 \href{https://doi.org/10.1007/s10601-020-09317-y}{BenediktMH20} & \hyperref[auth:a114]{O. Benedikt}, \hyperref[auth:a115]{I. M{\'{o}}dos}, \hyperref[auth:a116]{Z. Hanz{\'{a}}lek} & \cellcolor{green!10}Power of pre-processing: production scheduling with variable energy pricing and power-saving states & \hyperref[detail:BenediktMH20]{Details} \href{../works/BenediktMH20.pdf}{Yes} & \cite{BenediktMH20} & 2020 & Constraints An Int. J. & 19 & \noindent{}\textcolor{black!50}{0.00} \textcolor{black!50}{0.00} \textbf{8.72} & 1 2 2 & 18 18 & 3 1 2\\
WallaceY20 \href{https://doi.org/10.1007/s10601-020-09316-z}{WallaceY20} & \hyperref[auth:a117]{M. G. Wallace}, \hyperref[auth:a19]{N. Yorke-Smith} & \cellcolor{gold!20}A new constraint programming model and solving for the cyclic hoist scheduling problem & \hyperref[detail:WallaceY20]{Details} \href{../works/WallaceY20.pdf}{Yes} & \cite{WallaceY20} & 2020 & Constraints An Int. J. & 19 & \noindent{}\textbf{1.00} \textbf{1.00} \textbf{4.76} & 5 6 5 & 18 23 & 6 3 3\\
HoundjiSW19 \href{https://doi.org/10.1007/s10601-018-9300-y}{HoundjiSW19} & \hyperref[auth:a223]{V. R. Houndji}, \hyperref[auth:a147]{P. Schaus}, \hyperref[auth:a224]{L. A. Wolsey} & The item dependent stockingcost constraint & \hyperref[detail:HoundjiSW19]{Details} \href{../works/HoundjiSW19.pdf}{Yes} & \cite{HoundjiSW19} & 2019 & Constraints An Int. J. & 27 & \noindent{}\textcolor{black!50}{0.00} \textcolor{black!50}{0.00} \textbf{2.91} & 0 0 0 & 17 28 & 5 0 5\\
CauwelaertLS18 \href{https://doi.org/10.1007/s10601-017-9277-y}{CauwelaertLS18} & \hyperref[auth:a201]{S. V. Cauwelaert}, \hyperref[auth:a142]{M. Lombardi}, \hyperref[auth:a147]{P. Schaus} & How efficient is a global constraint in practice? - {A} fair experimental framework & \hyperref[detail:CauwelaertLS18]{Details} \href{../works/CauwelaertLS18.pdf}{Yes} & \cite{CauwelaertLS18} & 2018 & Constraints An Int. J. & 36 & \noindent{}\textcolor{black!50}{0.00} \textcolor{black!50}{0.00} \textbf{3.12} & 2 1 1 & 39 61 & 14 1 13\\
FahimiOQ18 \href{https://doi.org/10.1007/s10601-018-9282-9}{FahimiOQ18} & \hyperref[auth:a122]{H. Fahimi}, \hyperref[auth:a52]{Y. Ouellet}, \hyperref[auth:a37]{C.-G. Quimper} & Linear-time filtering algorithms for the disjunctive constraint and a quadratic filtering algorithm for the cumulative not-first not-last & \hyperref[detail:FahimiOQ18]{Details} \href{../works/FahimiOQ18.pdf}{Yes} & \cite{FahimiOQ18} & 2018 & Constraints An Int. J. & 22 & \noindent{}\textcolor{black!50}{0.00} \textcolor{black!50}{0.00} \textbf{6.07} & 2 2 7 & 20 36 & 19 2 17\\
LaborieRSV18 \href{https://doi.org/10.1007/s10601-018-9281-x}{LaborieRSV18} & \hyperref[auth:a118]{P. Laborie}, \hyperref[auth:a119]{J. Rogerie}, \hyperref[auth:a120]{P. Shaw}, \hyperref[auth:a121]{P. Vil{\'{\i}}m} & {IBM} {ILOG} {CP} optimizer for scheduling - 20+ years of scheduling with constraints at {IBM/ILOG} & \hyperref[detail:LaborieRSV18]{Details} \href{../works/LaborieRSV18.pdf}{Yes} & \cite{LaborieRSV18} & 2018 & Constraints An Int. J. & 41 & \noindent{}\textbf{1.00} \textbf{1.00} \textbf{54.37} & 148 178 203 & 35 54 & 92 69 23\\
HookerH17 \href{http://dx.doi.org/10.1007/s10601-017-9280-3}{HookerH17} & \hyperref[auth:a160]{J. N. Hooker}, \hyperref[auth:a206]{W.-J. van Hoeve} & Constraint programming and operations research & \hyperref[detail:HookerH17]{Details} \href{../works/HookerH17.pdf}{Yes} & \cite{HookerH17} & 2017 & Constraints An Int. J. & 24 & \noindent{}\textcolor{black!50}{0.00} \textcolor{black!50}{0.00} \textbf{36.80} & 12 13 10 & 189 255 & 61 2 59\\
KreterSS17 \href{https://doi.org/10.1007/s10601-016-9266-6}{KreterSS17} & \hyperref[auth:a123]{S. Kreter}, \hyperref[auth:a124]{A. Schutt}, \hyperref[auth:a125]{P. J. Stuckey} & Using constraint programming for solving RCPSP/max-cal & \hyperref[detail:KreterSS17]{Details} \href{../works/KreterSS17.pdf}{Yes} & \cite{KreterSS17} & 2017 & Constraints An Int. J. & 31 & \noindent{}\textcolor{black!50}{0.00} \textcolor{black!50}{0.00} \textbf{8.01} & 15 18 21 & 20 27 & 23 12 11\\
NattafAL17 \href{https://doi.org/10.1007/s10601-017-9271-4}{NattafAL17} & \hyperref[auth:a81]{M. Nattaf}, \hyperref[auth:a6]{C. Artigues}, \hyperref[auth:a3]{P. Lopez} & \cellcolor{green!10}Cumulative scheduling with variable task profiles and concave piecewise linear processing rate functions & \hyperref[detail:NattafAL17]{Details} \href{../works/NattafAL17.pdf}{Yes} & \cite{NattafAL17} & 2017 & Constraints An Int. J. & 18 & \noindent{}\textcolor{black!50}{0.00} \textcolor{black!50}{0.00} \textbf{1.71} & 5 5 7 & 10 16 & 8 3 5\\
Kameugne15 \href{https://doi.org/10.1007/s10601-015-9227-5}{Kameugne15} & \hyperref[auth:a10]{R. Kameugne} & Propagation techniques of resource constraint for cumulative scheduling & \hyperref[detail:Kameugne15]{Details} \href{../works/Kameugne15.pdf}{Yes} & \cite{Kameugne15} & 2015 & Constraints An Int. J. & 2 & \noindent{}0.75 0.75 \textcolor{black!50}{0.07} & 0 0 0 & 0 0 & 0 0 0\\
LetortCB15 \href{https://doi.org/10.1007/s10601-014-9172-8}{LetortCB15} & \hyperref[auth:a127]{A. Letort}, \hyperref[auth:a91]{M. Carlsson}, \hyperref[auth:a128]{N. Beldiceanu} & \cellcolor{green!10}Synchronized sweep algorithms for scalable scheduling constraints & \hyperref[detail:LetortCB15]{Details} \href{../works/LetortCB15.pdf}{Yes} & \cite{LetortCB15} & 2015 & Constraints An Int. J. & 52 & \noindent{}\textcolor{black!50}{0.00} \textcolor{black!50}{0.00} \textbf{18.50} & 2 2 4 & 14 28 & 12 0 12\\
NattafAL15 \href{https://doi.org/10.1007/s10601-015-9192-z}{NattafAL15} & \hyperref[auth:a81]{M. Nattaf}, \hyperref[auth:a6]{C. Artigues}, \hyperref[auth:a3]{P. Lopez} & \cellcolor{green!10}A hybrid exact method for a scheduling problem with a continuous resource and energy constraints & \hyperref[detail:NattafAL15]{Details} \href{../works/NattafAL15.pdf}{Yes} & \cite{NattafAL15} & 2015 & Constraints An Int. J. & 21 & \noindent{}\textcolor{black!50}{0.00} \textcolor{black!50}{0.00} 0.96 & 14 15 15 & 13 18 & 7 3 4\\
Siala15 \href{https://doi.org/10.1007/s10601-015-9213-y}{Siala15} & \hyperref[auth:a129]{M. Siala} & Search, propagation, and learning in sequencing and scheduling problems & \hyperref[detail:Siala15]{Details} \href{../works/Siala15.pdf}{Yes} & \cite{Siala15} & 2015 & Constraints An Int. J. & 2 & \noindent{}0.50 0.50 \textbf{1.43} & 4 3 0 & 0 0 & 0 0 0\\
SimoninAHL15 \href{https://doi.org/10.1007/s10601-014-9169-3}{SimoninAHL15} & \hyperref[auth:a126]{G. Simonin}, \hyperref[auth:a6]{C. Artigues}, \hyperref[auth:a1]{E. Hebrard}, \hyperref[auth:a3]{P. Lopez} & \cellcolor{green!10}Scheduling scientific experiments for comet exploration & \hyperref[detail:SimoninAHL15]{Details} \href{../works/SimoninAHL15.pdf}{Yes} & \cite{SimoninAHL15} & 2015 & Constraints An Int. J. & 23 & \noindent{}\textcolor{black!50}{0.00} \textcolor{black!50}{0.00} 0.87 & 4 4 6 & 5 8 & 2 0 2\\
KameugneFSN14 \href{https://doi.org/10.1007/s10601-013-9157-z}{KameugneFSN14} & \hyperref[auth:a10]{R. Kameugne}, \hyperref[auth:a130]{L. P. Fotso}, \hyperref[auth:a131]{J. D. Scott}, \hyperref[auth:a132]{Y. Ngo-Kateu} & A quadratic edge-finding filtering algorithm for cumulative resource constraints & \hyperref[detail:KameugneFSN14]{Details} \href{../works/KameugneFSN14.pdf}{Yes} & \cite{KameugneFSN14} & 2014 & Constraints An Int. J. & 27 & \noindent{}\textcolor{black!50}{0.00} \textcolor{black!50}{0.00} \textbf{5.63} & 6 6 9 & 10 20 & 16 6 10\\
HeinzSB13 \href{https://doi.org/10.1007/s10601-012-9136-9}{HeinzSB13} & \hyperref[auth:a133]{S. Heinz}, \hyperref[auth:a134]{J. Schulz}, \hyperref[auth:a89]{J. C. Beck} & Using dual presolving reductions to reformulate cumulative constraints & \hyperref[detail:HeinzSB13]{Details} \href{../works/HeinzSB13.pdf}{Yes} & \cite{HeinzSB13} & 2013 & Constraints An Int. J. & 36 & \noindent{}\textcolor{black!50}{0.00} \textcolor{black!50}{0.00} \textbf{11.84} & 7 7 9 & 31 41 & 15 2 13\\
OzturkTHO13 \href{https://doi.org/10.1007/s10601-013-9142-6}{OzturkTHO13} & \hyperref[auth:a135]{C. {\"{O}}zt{\"{u}}rk}, \hyperref[auth:a136]{S. Tunali}, \hyperref[auth:a137]{B. Hnich}, \hyperref[auth:a138]{A. {\"{O}}rnek} & Balancing and scheduling of flexible mixed model assembly lines & \hyperref[detail:OzturkTHO13]{Details} \href{../works/OzturkTHO13.pdf}{Yes} & \cite{OzturkTHO13} & 2013 & Constraints An Int. J. & 36 & \noindent{}\textcolor{black!50}{0.00} \textcolor{black!50}{0.00} \textbf{54.08} & 31 31 34 & 44 62 & 16 6 10\\
HeinzSSW12 \href{https://doi.org/10.1007/s10601-011-9113-8}{HeinzSSW12} & \hyperref[auth:a133]{S. Heinz}, \hyperref[auth:a139]{T. Schlechte}, \hyperref[auth:a140]{R. Stephan}, \hyperref[auth:a141]{M. Winkler} & Solving steel mill slab design problems & \hyperref[detail:HeinzSSW12]{Details} \href{../works/HeinzSSW12.pdf}{Yes} & \cite{HeinzSSW12} & 2012 & Constraints An Int. J. & 12 & \noindent{}\textcolor{black!50}{0.00} \textcolor{black!50}{0.00} \textcolor{black!50}{0.14} & 10 11 12 & 9 16 & 4 1 3\\
LimtanyakulS12 \href{https://doi.org/10.1007/s10601-012-9118-y}{LimtanyakulS12} & \hyperref[auth:a144]{K. Limtanyakul}, \hyperref[auth:a145]{U. Schwiegelshohn} & Improvements of constraint programming and hybrid methods for scheduling of tests on vehicle prototypes & \hyperref[detail:LimtanyakulS12]{Details} \href{../works/LimtanyakulS12.pdf}{Yes} & \cite{LimtanyakulS12} & 2012 & Constraints An Int. J. & 32 & \noindent{}\textbf{1.00} \textbf{1.00} \textbf{25.90} & 4 4 5 & 16 27 & 6 1 5\\
LombardiM12 \href{https://doi.org/10.1007/s10601-011-9115-6}{LombardiM12} & \hyperref[auth:a142]{M. Lombardi}, \hyperref[auth:a143]{M. Milano} & Optimal methods for resource allocation and scheduling: a cross-disciplinary survey & \hyperref[detail:LombardiM12]{Details} \href{../works/LombardiM12.pdf}{Yes} & \cite{LombardiM12} & 2012 & Constraints An Int. J. & 35 & \noindent{}\textcolor{black!50}{0.00} \textcolor{black!50}{0.00} \textbf{43.11} & 39 39 47 & 68 94 & 41 5 36\\
BartakS11 \href{https://doi.org/10.1007/s10601-011-9109-4}{BartakS11} & \hyperref[auth:a152]{R. Bart{\'{a}}k}, \hyperref[auth:a153]{M. A. Salido} & \cellcolor{green!10}Constraint satisfaction for planning and scheduling problems & \hyperref[detail:BartakS11]{Details} \href{../works/BartakS11.pdf}{Yes} & \cite{BartakS11} & 2011 & Constraints An Int. J. & 5 & \noindent{}\textbf{1.00} \textbf{1.00} \textbf{1.58} & 17 18 21 & 3 7 & 3 2 1\\
KovacsB11 \href{https://doi.org/10.1007/s10601-009-9088-x}{KovacsB11} & \hyperref[auth:a146]{A. Kov{\'{a}}cs}, \hyperref[auth:a89]{J. C. Beck} & A global constraint for total weighted completion time for unary resources & \hyperref[detail:KovacsB11]{Details} \href{../works/KovacsB11.pdf}{Yes} & \cite{KovacsB11} & 2011 & Constraints An Int. J. & 24 & \noindent{}\textcolor{black!50}{0.00} \textcolor{black!50}{0.00} \textbf{8.15} & 4 4 9 & 26 36 & 5 1 4\\
KovacsK11 \href{https://doi.org/10.1007/s10601-010-9102-3}{KovacsK11} & \hyperref[auth:a146]{A. Kov{\'{a}}cs}, \hyperref[auth:a155]{T. Kis} & Constraint programming approach to a bilevel scheduling problem & \hyperref[detail:KovacsK11]{Details} \href{../works/KovacsK11.pdf}{Yes} & \cite{KovacsK11} & 2011 & Constraints An Int. J. & 24 & \noindent{}\textbf{1.00} \textbf{1.00} \textbf{5.37} & 3 4 5 & 24 37 & 3 0 3\\
SchausHMCMD11 \href{https://doi.org/10.1007/s10601-010-9100-5}{SchausHMCMD11} & \hyperref[auth:a147]{P. Schaus}, \hyperref[auth:a148]{P. V. Hentenryck}, \hyperref[auth:a149]{J.-N. Monette}, \hyperref[auth:a150]{C. Coffrin}, \hyperref[auth:a32]{L. Michel}, \hyperref[auth:a151]{Y. Deville} & \cellcolor{green!10}Solving Steel Mill Slab Problems with constraint-based techniques: CP, LNS, and {CBLS} & \hyperref[detail:SchausHMCMD11]{Details} \href{../works/SchausHMCMD11.pdf}{Yes} & \cite{SchausHMCMD11} & 2011 & Constraints An Int. J. & 23 & \noindent{}\textcolor{black!50}{0.00} \textcolor{black!50}{0.00} \textbf{3.20} & 14 16 19 & 5 12 & 5 2 3\\
SchuttFSW11 \href{https://doi.org/10.1007/s10601-010-9103-2}{SchuttFSW11} & \hyperref[auth:a124]{A. Schutt}, \hyperref[auth:a154]{T. Feydy}, \hyperref[auth:a125]{P. J. Stuckey}, \hyperref[auth:a117]{M. G. Wallace} & Explaining the cumulative propagator & \hyperref[detail:SchuttFSW11]{Details} \href{../works/SchuttFSW11.pdf}{Yes} & \cite{SchuttFSW11} & 2011 & Constraints An Int. J. & 33 & \noindent{}\textcolor{black!50}{0.00} \textcolor{black!50}{0.00} \textbf{14.77} & 57 61 65 & 23 39 & 48 34 14\\
LopesCSM10 \href{https://doi.org/10.1007/s10601-009-9086-z}{LopesCSM10} & \hyperref[auth:a156]{T. M. T. Lopes}, \hyperref[auth:a157]{A. A. Cir{\'{e}}}, \hyperref[auth:a158]{C. C. de Souza}, \hyperref[auth:a159]{A. V. Moura} & A hybrid model for a multiproduct pipeline planning and scheduling problem & \hyperref[detail:LopesCSM10]{Details} \href{../works/LopesCSM10.pdf}{Yes} & \cite{LopesCSM10} & 2010 & Constraints An Int. J. & 39 & \noindent{}\textcolor{black!50}{0.00} \textcolor{black!50}{0.00} \textbf{11.10} & 31 31 35 & 18 31 & 4 0 4\\
OhrimenkoSC09 \href{http://dx.doi.org/10.1007/s10601-008-9064-x}{OhrimenkoSC09} & \hyperref[auth:a861]{O. Ohrimenko}, \hyperref[auth:a125]{P. J. Stuckey}, \hyperref[auth:a862]{M. Codish} & Propagation via lazy clause generation & \hyperref[detail:OhrimenkoSC09]{Details} \href{../works/OhrimenkoSC09.pdf}{Yes} & \cite{OhrimenkoSC09} & 2009 & Constraints An Int. J. & 35 & \noindent{}\textcolor{black!50}{0.00} \textcolor{black!50}{0.00} \textbf{3.19} & 127 128 198 & 15 35 & 33 31 2\\
Simonis07 \href{https://doi.org/10.1007/s10601-006-9011-7}{Simonis07} & \hyperref[auth:a17]{H. Simonis} & Models for Global Constraint Applications & \hyperref[detail:Simonis07]{Details} \href{../works/Simonis07.pdf}{Yes} & \cite{Simonis07} & 2007 & Constraints An Int. J. & 30 & \noindent{}\textcolor{black!50}{0.00} \textcolor{black!50}{0.00} \textbf{10.93} & 10 11 19 & 17 74 & 14 3 11\\
Hooker06 \href{https://doi.org/10.1007/s10601-006-8060-2}{Hooker06} & \hyperref[auth:a160]{J. N. Hooker} & \cellcolor{green!10}An Integrated Method for Planning and Scheduling to Minimize Tardiness & \hyperref[detail:Hooker06]{Details} \href{../works/Hooker06.pdf}{Yes} & \cite{Hooker06} & 2006 & Constraints An Int. J. & 19 & \noindent{}\textcolor{black!50}{0.00} \textcolor{black!50}{0.00} \textbf{8.82} & 19 20 27 & 13 20 & 24 16 8\\
Hooker05 \href{https://doi.org/10.1007/s10601-005-2812-2}{Hooker05} & \hyperref[auth:a160]{J. N. Hooker} & \cellcolor{green!10}A Hybrid Method for the Planning and Scheduling & \hyperref[detail:Hooker05]{Details} \href{../works/Hooker05.pdf}{Yes} & \cite{Hooker05} & 2005 & Constraints An Int. J. & 17 & \noindent{}\textcolor{black!50}{0.00} \textcolor{black!50}{0.00} \textbf{7.71} & 68 69 87 & 11 18 & 40 30 10\\
VilimBC05 \href{https://doi.org/10.1007/s10601-005-2814-0}{VilimBC05} & \hyperref[auth:a121]{P. Vil{\'{\i}}m}, \hyperref[auth:a152]{R. Bart{\'{a}}k}, \hyperref[auth:a161]{O. Cepek} & Extension of \emph{O}(\emph{n} log \emph{n}) Filtering Algorithms for the Unary Resource Constraint to Optional Activities & \hyperref[detail:VilimBC05]{Details} \href{../works/VilimBC05.pdf}{Yes} & \cite{VilimBC05} & 2005 & Constraints An Int. J. & 23 & \noindent{}\textcolor{black!50}{0.00} \textcolor{black!50}{0.00} \textbf{2.28} & 21 21 32 & 5 16 & 15 12 3\\
BaptisteP00 \href{https://doi.org/10.1023/A:1009822502231}{BaptisteP00} & \hyperref[auth:a162]{P. Baptiste}, \hyperref[auth:a163]{C. L. Pape} & Constraint Propagation and Decomposition Techniques for Highly Disjunctive and Highly Cumulative Project Scheduling Problems & \hyperref[detail:BaptisteP00]{Details} \href{../works/BaptisteP00.pdf}{Yes} & \cite{BaptisteP00} & 2000 & Constraints An Int. J. & 21 & \noindent{}\textbf{1.50} \textbf{1.50} \textbf{14.93} & 46 0 62 & 0 0 & 29 29 0\\
HeipckeCCS00 \href{https://doi.org/10.1023/A:1009860311452}{HeipckeCCS00} & \hyperref[auth:a167]{S. Heipcke}, \hyperref[auth:a168]{Y. Colombani}, \hyperref[auth:a169]{C. C. B. Cavalcante}, \hyperref[auth:a170]{C. C. de Souza} & Scheduling under Labour Resource Constraints & \hyperref[detail:HeipckeCCS00]{Details} \href{../works/HeipckeCCS00.pdf}{Yes} & \cite{HeipckeCCS00} & 2000 & Constraints An Int. J. & 8 & \noindent{}\textcolor{black!50}{0.00} \textcolor{black!50}{0.00} \textbf{2.08} & 5 0 5 & 0 0 & 1 1 0\\
SakkoutW00 \href{https://doi.org/10.1023/A:1009856210543}{SakkoutW00} & \hyperref[auth:a166]{H. E. Sakkout}, \hyperref[auth:a117]{M. G. Wallace} & Probe Backtrack Search for Minimal Perturbation in Dynamic Scheduling & \hyperref[detail:SakkoutW00]{Details} \href{../works/SakkoutW00.pdf}{Yes} & \cite{SakkoutW00} & 2000 & Constraints An Int. J. & 30 & \noindent{}\textcolor{black!50}{0.00} \textcolor{black!50}{0.00} \textbf{7.62} & 73 0 105 & 0 0 & 18 18 0\\
SchildW00 \href{https://doi.org/10.1023/A:1009804226473}{SchildW00} & \hyperref[auth:a164]{K. Schild}, \hyperref[auth:a165]{J. W{\"{u}}rtz} & Scheduling of Time-Triggered Real-Time Systems & \hyperref[detail:SchildW00]{Details} \href{../works/SchildW00.pdf}{Yes} & \cite{SchildW00} & 2000 & Constraints An Int. J. & 23 & \noindent{}\textcolor{black!50}{0.00} \textcolor{black!50}{0.00} \textbf{2.74} & 23 0 32 & 0 0 & 1 1 0\\
BensanaLV99 \href{https://doi.org/10.1023/A:1026488509554}{BensanaLV99} & \hyperref[auth:a171]{E. Bensana}, \hyperref[auth:a172]{M. Lema{\^{\i}}tre}, \hyperref[auth:a173]{G. Verfaillie} & Earth Observation Satellite Management & \hyperref[detail:BensanaLV99]{Details} \href{../works/BensanaLV99.pdf}{Yes} & \cite{BensanaLV99} & 1999 & Constraints An Int. J. & 7 & \noindent{}\textcolor{black!50}{0.00} \textcolor{black!50}{0.00} \textcolor{black!50}{0.07} & 99 0 150 & 0 0 & 4 4 0\\
BelhadjiI98 \href{https://doi.org/10.1023/A:1009777711218}{BelhadjiI98} & \hyperref[auth:a174]{S. Belhadji}, \hyperref[auth:a175]{A. Isli} & Temporal Constraint Satisfaction Techniques in Job Shop Scheduling Problem Solving & \hyperref[detail:BelhadjiI98]{Details} \href{../works/BelhadjiI98.pdf}{Yes} & \cite{BelhadjiI98} & 1998 & Constraints An Int. J. & 9 & \noindent{}\textbf{2.00} \textbf{2.00} \textbf{1.70} & 3 0 5 & 0 0 & 1 1 0\\
PapaB98 \href{https://doi.org/10.1023/A:1009723704757}{PapaB98} & \hyperref[auth:a163]{C. L. Pape}, \hyperref[auth:a162]{P. Baptiste} & Resource Constraints for Preemptive Job-shop Scheduling & \hyperref[detail:PapaB98]{Details} \href{../works/PapaB98.pdf}{Yes} & \cite{PapaB98} & 1998 & Constraints An Int. J. & 25 & \noindent{}\textcolor{black!50}{0.00} \textcolor{black!50}{0.00} \textbf{10.97} & 14 0 19 & 0 0 & 4 4 0\\
Darby-DowmanLMZ97 \href{https://doi.org/10.1007/BF00137871}{Darby-DowmanLMZ97} & \hyperref[auth:a177]{K. Darby-Dowman}, \hyperref[auth:a178]{J. Little}, \hyperref[auth:a179]{G. Mitra}, \hyperref[auth:a180]{M. Zaffalon} & \cellcolor{green!10}Constraint Logic Programming and Integer Programming Approaches and Their Collaboration in Solving an Assignment Scheduling Problem & \hyperref[detail:Darby-DowmanLMZ97]{Details} \href{../works/Darby-DowmanLMZ97.pdf}{Yes} & \cite{Darby-DowmanLMZ97} & 1997 & Constraints An Int. J. & 20 & \noindent{}\textbf{1.00} \textbf{1.00} \textbf{14.30} & 28 28 32 & 5 22 & 12 12 0\\
Zhou97 \href{https://doi.org/10.1023/A:1009757726572}{Zhou97} & \hyperref[auth:a176]{J. Zhou} & A Permutation-Based Approach for Solving the Job-Shop Problem & \hyperref[detail:Zhou97]{Details} \href{../works/Zhou97.pdf}{Yes} & \cite{Zhou97} & 1997 & Constraints An Int. J. & 29 & \noindent{}\textcolor{black!50}{0.00} \textcolor{black!50}{0.00} \textbf{8.24} & 14 0 16 & 0 0 & 5 5 0\\
SmithBHW96 \href{http://dx.doi.org/10.1007/bf00143880}{SmithBHW96} & \hyperref[auth:a1053]{B. M. Smith}, \hyperref[auth:a1051]{S. C. Brailsford}, \hyperref[auth:a1179]{P. M. Hubbard}, \hyperref[auth:a1180]{H. P. Williams} & The progressive party problem: Integer linear programming and constraint programming compared & \hyperref[detail:SmithBHW96]{Details} \href{../works/SmithBHW96.pdf}{Yes} & \cite{SmithBHW96} & 1996 & Constraints An Int. J. & 20 & \noindent{}\textcolor{black!50}{0.00} \textcolor{black!50}{0.00} \textbf{1.29} & 56 57 61 & 4 9 & 14 14 0\\
Wallace96 \href{https://doi.org/10.1007/BF00143881}{Wallace96} & \hyperref[auth:a117]{M. G. Wallace} & Practical Applications of Constraint Programming & \hyperref[detail:Wallace96]{Details} \href{../works/Wallace96.pdf}{Yes} & \cite{Wallace96} & 1996 & Constraints An Int. J. & 30 & \noindent{}\textcolor{black!50}{0.00} \textcolor{black!50}{0.00} \textbf{16.50} & 87 89 138 & 55 143 & 20 12 8\\
\end{longtable}
}

\subsection{Control Engineering Practice}

\index{Control Engineering Practice}
{\scriptsize
\begin{longtable}{>{\raggedright\arraybackslash}p{2.5cm}>{\raggedright\arraybackslash}p{4.5cm}>{\raggedright\arraybackslash}p{6.0cm}p{1.0cm}rr>{\raggedright\arraybackslash}p{2.0cm}r>{\raggedright\arraybackslash}p{1cm}p{1cm}p{1cm}p{1cm}}
\rowcolor{white}\caption{Articles in Journal Control Engineering Practice (Total 1)}\\ \toprule
\rowcolor{white}\shortstack{Key\\Source} & Authors & Title (Colored by Open Access)& \shortstack{Details\\LC} & Cite & Year & \shortstack{Conference\\/Journal\\/School} & Pages & Relevance &\shortstack{Cites\\OC XR\\SC} & \shortstack{Refs\\OC\\XR} & \shortstack{Links\\Cites\\Refs}\\ \midrule\endhead
\bottomrule
\endfoot
MarliereSPR23 \href{https://www.sciencedirect.com/science/article/pii/S0967066122002611}{MarliereSPR23} & \hyperref[auth:a1018]{G. Marlière}, \hyperref[auth:a1019]{S. {Sobieraj Richard}}, \hyperref[auth:a1020]{P. Pellegrini}, \hyperref[auth:a781]{J. Rodriguez} & \cellcolor{green!10}A conditional time-intervals formulation of the real-time Railway Traffic Management Problem & \hyperref[detail:MarliereSPR23]{Details} \href{../works/MarliereSPR23.pdf}{Yes} & \cite{MarliereSPR23} & 2023 & Control Engineering Practice & 22 & \noindent{}\textcolor{black!50}{0.00} \textcolor{black!50}{0.00} \textbf{12.23} & 1 3 4 & 75 101 & 6 0 6\\
\end{longtable}
}

\subsection{DECISION SUPPORT SYSTEMS}

\index{DECISION SUPPORT SYSTEMS}
{\scriptsize
\begin{longtable}{>{\raggedright\arraybackslash}p{2.5cm}>{\raggedright\arraybackslash}p{4.5cm}>{\raggedright\arraybackslash}p{6.0cm}p{1.0cm}rr>{\raggedright\arraybackslash}p{2.0cm}r>{\raggedright\arraybackslash}p{1cm}p{1cm}p{1cm}p{1cm}}
\rowcolor{white}\caption{Articles in Journal DECISION SUPPORT SYSTEMS (Total 1)}\\ \toprule
\rowcolor{white}\shortstack{Key\\Source} & Authors & Title (Colored by Open Access)& \shortstack{Details\\LC} & Cite & Year & \shortstack{Conference\\/Journal\\/School} & Pages & Relevance &\shortstack{Cites\\OC XR\\SC} & \shortstack{Refs\\OC\\XR} & \shortstack{Links\\Cites\\Refs}\\ \midrule\endhead
\bottomrule
\endfoot
MeskensDL13 \href{http://dx.doi.org/10.1016/j.dss.2012.10.019}{MeskensDL13} & \hyperref[auth:a597]{N. Meskens}, \hyperref[auth:a598]{D. Duvivier}, \hyperref[auth:a1460]{A. Lianset} & Multi-objective operating room scheduling considering desiderata of the surgical team \hyperref[abs:MeskensDL13]{Abstract} & \hyperref[detail:MeskensDL13]{Details} \href{../works/MeskensDL13.pdf}{Yes} & \cite{MeskensDL13} & 2013 & DECISION SUPPORT SYSTEMS & 10 & \noindent{}\textcolor{black!50}{0.00} \textbf{1.00} \textbf{1.50} & 102 102 116 & 31 39 & 5 5 0\\
\end{longtable}
}

\subsection{Decision Making in Manufacturing and Services}

\index{Decision Making in Manufacturing and Services}
{\scriptsize
\begin{longtable}{>{\raggedright\arraybackslash}p{2.5cm}>{\raggedright\arraybackslash}p{4.5cm}>{\raggedright\arraybackslash}p{6.0cm}p{1.0cm}rr>{\raggedright\arraybackslash}p{2.0cm}r>{\raggedright\arraybackslash}p{1cm}p{1cm}p{1cm}p{1cm}}
\rowcolor{white}\caption{Articles in Journal Decision Making in Manufacturing and Services (Total 1)}\\ \toprule
\rowcolor{white}\shortstack{Key\\Source} & Authors & Title (Colored by Open Access)& \shortstack{Details\\LC} & Cite & Year & \shortstack{Conference\\/Journal\\/School} & Pages & Relevance &\shortstack{Cites\\OC XR\\SC} & \shortstack{Refs\\OC\\XR} & \shortstack{Links\\Cites\\Refs}\\ \midrule\endhead
\bottomrule
\endfoot
Banaszak2008 \href{http://dx.doi.org/10.7494/dmms.2008.2.2.5}{Banaszak2008} & \hyperref[auth:a1814]{Z. Banaszak}, \hyperref[auth:a630]{G. Bocewicz}, \hyperref[auth:a631]{I. Bach} & CP-driven Production Process Planning in Multiproject Environment \hyperref[abs:Banaszak2008]{Abstract} & \hyperref[detail:Banaszak2008]{Details} No & \cite{Banaszak2008} & 2008 & Decision Making in Manufacturing and Services & null & \noindent{}\textcolor{black!50}{0.00} \textbf{4.00} n/a & 4 4 0 & 0 0 & 1 1 0\\
\end{longtable}
}

\subsection{Discrete Applied Mathematics}

\index{Discrete Applied Mathematics}
{\scriptsize
\begin{longtable}{>{\raggedright\arraybackslash}p{2.5cm}>{\raggedright\arraybackslash}p{4.5cm}>{\raggedright\arraybackslash}p{6.0cm}p{1.0cm}rr>{\raggedright\arraybackslash}p{2.0cm}r>{\raggedright\arraybackslash}p{1cm}p{1cm}p{1cm}p{1cm}}
\rowcolor{white}\caption{Articles in Journal Discrete Applied Mathematics (Total 5)}\\ \toprule
\rowcolor{white}\shortstack{Key\\Source} & Authors & Title (Colored by Open Access)& \shortstack{Details\\LC} & Cite & Year & \shortstack{Conference\\/Journal\\/School} & Pages & Relevance &\shortstack{Cites\\OC XR\\SC} & \shortstack{Refs\\OC\\XR} & \shortstack{Links\\Cites\\Refs}\\ \midrule\endhead
\bottomrule
\endfoot
NattafHKAL19 \href{https://doi.org/10.1016/j.dam.2018.11.008}{NattafHKAL19} & \hyperref[auth:a81]{M. Nattaf}, \hyperref[auth:a996]{M. Horv{\'{a}}th}, \hyperref[auth:a155]{T. Kis}, \hyperref[auth:a6]{C. Artigues}, \hyperref[auth:a3]{P. Lopez} & \cellcolor{gold!20}Polyhedral results and valid inequalities for the continuous energy-constrained scheduling problem & \hyperref[detail:NattafHKAL19]{Details} \href{../works/NattafHKAL19.pdf}{Yes} & \cite{NattafHKAL19} & 2019 & Discrete Applied Mathematics & 16 & \noindent{}\textcolor{black!50}{0.00} \textcolor{black!50}{0.00} 0.49 & 5 6 5 & 12 17 & 5 0 5\\
BaptisteB18 \href{https://doi.org/10.1016/j.dam.2017.05.001}{BaptisteB18} & \hyperref[auth:a162]{P. Baptiste}, \hyperref[auth:a704]{N. Bonifas} & \cellcolor{gold!20}Redundant cumulative constraints to compute preemptive bounds & \hyperref[detail:BaptisteB18]{Details} \href{../works/BaptisteB18.pdf}{Yes} & \cite{BaptisteB18} & 2018 & Discrete Applied Mathematics & 10 & \noindent{}\textcolor{black!50}{0.00} \textcolor{black!50}{0.00} \textbf{2.63} & 3 4 4 & 13 19 & 8 2 6\\
HebrardHJMPV16 \href{https://doi.org/10.1016/j.dam.2016.07.003}{HebrardHJMPV16} & \hyperref[auth:a1]{E. Hebrard}, \hyperref[auth:a54]{M.-J. Huguet}, \hyperref[auth:a791]{N. Jozefowiez}, \hyperref[auth:a787]{A. Maillard}, \hyperref[auth:a21]{C. Pralet}, \hyperref[auth:a173]{G. Verfaillie} & \cellcolor{gold!20}Approximation of the parallel machine scheduling problem with additional unit resources & \hyperref[detail:HebrardHJMPV16]{Details} \href{../works/HebrardHJMPV16.pdf}{Yes} & \cite{HebrardHJMPV16} & 2016 & Discrete Applied Mathematics & 10 & \noindent{}\textcolor{black!50}{0.00} \textcolor{black!50}{0.00} \textcolor{black!50}{0.00} & 9 10 12 & 8 8 & 1 0 1\\
Brucker2002 \href{http://dx.doi.org/10.1016/s0166-218x(01)00342-0}{Brucker2002} & \hyperref[auth:a847]{P. Brucker} & \cellcolor{gold!20}Scheduling and constraint propagation & \hyperref[detail:Brucker2002]{Details} No & \cite{Brucker2002} & 2002 & Discrete Applied Mathematics & null & \noindent{}\textbf{1.50} \textbf{1.50} n/a & 40 40 49 & 33 48 & 17 9 8\\
HookerO99 \href{http://dx.doi.org/10.1016/s0166-218x(99)00100-6}{HookerO99} & \hyperref[auth:a160]{J. N. Hooker}, \hyperref[auth:a1153]{M. Osorio} & \cellcolor{gold!20}Mixed logical-linear programming & \hyperref[detail:HookerO99]{Details} \href{../works/HookerO99.pdf}{Yes} & \cite{HookerO99} & 1999 & Discrete Applied Mathematics & 48 & \noindent{}\textcolor{black!50}{0.00} \textcolor{black!50}{0.00} \textbf{1.61} & 92 95 111 & 48 75 & 19 19 0\\
\end{longtable}
}

\subsection{Discrete Dynamics in Nature and Society}

\index{Discrete Dynamics in Nature and Society}
{\scriptsize
\begin{longtable}{>{\raggedright\arraybackslash}p{2.5cm}>{\raggedright\arraybackslash}p{4.5cm}>{\raggedright\arraybackslash}p{6.0cm}p{1.0cm}rr>{\raggedright\arraybackslash}p{2.0cm}r>{\raggedright\arraybackslash}p{1cm}p{1cm}p{1cm}p{1cm}}
\rowcolor{white}\caption{Articles in Journal Discrete Dynamics in Nature and Society (Total 2)}\\ \toprule
\rowcolor{white}\shortstack{Key\\Source} & Authors & Title (Colored by Open Access)& \shortstack{Details\\LC} & Cite & Year & \shortstack{Conference\\/Journal\\/School} & Pages & Relevance &\shortstack{Cites\\OC XR\\SC} & \shortstack{Refs\\OC\\XR} & \shortstack{Links\\Cites\\Refs}\\ \midrule\endhead
\bottomrule
\endfoot
Ren2016 \href{http://dx.doi.org/10.1155/2016/5201937}{Ren2016} & \hyperref[auth:a1249]{H. Ren}, \hyperref[auth:a1611]{S. Sun} & \cellcolor{gold!20}A Hybrid IP/GA Approach to the Parallel Production Lines Scheduling Problem \hyperref[abs:Ren2016]{Abstract} & \hyperref[detail:Ren2016]{Details} No & \cite{Ren2016} & 2016 & Discrete Dynamics in Nature and Society & null & \noindent{}\textcolor{black!50}{0.00} \textbf{2.00} n/a & 0 0 0 & 32 37 & 5 0 5\\
PengLC14 \href{http://dx.doi.org/10.1155/2014/917685}{PengLC14} & \hyperref[auth:a915]{Y. Peng}, \hyperref[auth:a1385]{D. Lu}, \hyperref[auth:a913]{Y. Chen} & \cellcolor{gold!20}A Constraint Programming Method for Advanced Planning and Scheduling System with Multilevel Structured Products & \hyperref[detail:PengLC14]{Details} \href{../works/PengLC14.pdf}{Yes} & \cite{PengLC14} & 2014 & Discrete Dynamics in Nature and Society & 7 & \noindent{}\textbf{1.00} \textbf{1.00} \textbf{8.94} & 5 4 9 & 13 17 & 7 1 6\\
\end{longtable}
}

\subsection{Discrete Optimization}

\index{Discrete Optimization}
{\scriptsize
\begin{longtable}{>{\raggedright\arraybackslash}p{2.5cm}>{\raggedright\arraybackslash}p{4.5cm}>{\raggedright\arraybackslash}p{6.0cm}p{1.0cm}rr>{\raggedright\arraybackslash}p{2.0cm}r>{\raggedright\arraybackslash}p{1cm}p{1cm}p{1cm}p{1cm}}
\rowcolor{white}\caption{Articles in Journal Discrete Optimization (Total 1)}\\ \toprule
\rowcolor{white}\shortstack{Key\\Source} & Authors & Title (Colored by Open Access)& \shortstack{Details\\LC} & Cite & Year & \shortstack{Conference\\/Journal\\/School} & Pages & Relevance &\shortstack{Cites\\OC XR\\SC} & \shortstack{Refs\\OC\\XR} & \shortstack{Links\\Cites\\Refs}\\ \midrule\endhead
\bottomrule
\endfoot
MercierH07 \href{http://dx.doi.org/10.1016/j.disopt.2007.01.001}{MercierH07} & \hyperref[auth:a851]{L. Mercier}, \hyperref[auth:a148]{P. V. Hentenryck} & \cellcolor{gold!20}Strong polynomiality of resource constraint propagation & \hyperref[detail:MercierH07]{Details} \href{../works/MercierH07.pdf}{Yes} & \cite{MercierH07} & 2007 & Discrete Optimization & 27 & \noindent{}0.75 0.75 \textbf{11.75} & 5 5 7 & 8 17 & 11 4 7\\
\end{longtable}
}

\subsection{EURO Journal on Computational Optimization}

\index{EURO Journal on Computational Optimization}
{\scriptsize
\begin{longtable}{>{\raggedright\arraybackslash}p{2.5cm}>{\raggedright\arraybackslash}p{4.5cm}>{\raggedright\arraybackslash}p{6.0cm}p{1.0cm}rr>{\raggedright\arraybackslash}p{2.0cm}r>{\raggedright\arraybackslash}p{1cm}p{1cm}p{1cm}p{1cm}}
\rowcolor{white}\caption{Articles in Journal EURO Journal on Computational Optimization (Total 1)}\\ \toprule
\rowcolor{white}\shortstack{Key\\Source} & Authors & Title (Colored by Open Access)& \shortstack{Details\\LC} & Cite & Year & \shortstack{Conference\\/Journal\\/School} & Pages & Relevance &\shortstack{Cites\\OC XR\\SC} & \shortstack{Refs\\OC\\XR} & \shortstack{Links\\Cites\\Refs}\\ \midrule\endhead
\bottomrule
\endfoot
Kelareva2014 \href{http://dx.doi.org/10.1007/s13675-014-0022-7}{Kelareva2014} & \hyperref[auth:a332]{E. Kelareva}, \hyperref[auth:a333]{K. Tierney}, \hyperref[auth:a334]{P. Kilby} & \cellcolor{gold!20}CP methods for scheduling and routing with time-dependent task costs & \hyperref[detail:Kelareva2014]{Details} No & \cite{Kelareva2014} & 2014 & EURO Journal on Computational Optimization & null & \noindent{}\textcolor{black!50}{0.00} \textcolor{black!50}{0.00} n/a & 10 12 15 & 49 71 & 8 1 7\\
\end{longtable}
}

\subsection{Eksploatacja i Niezawodność – Maintenance and Reliability}

\index{Eksploatacja i Niezawodność – Maintenance and Reliability}
{\scriptsize
\begin{longtable}{>{\raggedright\arraybackslash}p{2.5cm}>{\raggedright\arraybackslash}p{4.5cm}>{\raggedright\arraybackslash}p{6.0cm}p{1.0cm}rr>{\raggedright\arraybackslash}p{2.0cm}r>{\raggedright\arraybackslash}p{1cm}p{1cm}p{1cm}p{1cm}}
\rowcolor{white}\caption{Articles in Journal Eksploatacja i Niezawodność – Maintenance and Reliability (Total 1)}\\ \toprule
\rowcolor{white}\shortstack{Key\\Source} & Authors & Title (Colored by Open Access)& \shortstack{Details\\LC} & Cite & Year & \shortstack{Conference\\/Journal\\/School} & Pages & Relevance &\shortstack{Cites\\OC XR\\SC} & \shortstack{Refs\\OC\\XR} & \shortstack{Links\\Cites\\Refs}\\ \midrule\endhead
\bottomrule
\endfoot
Bocewicz2021 \href{http://dx.doi.org/10.17531/ein.2021.1.13}{Bocewicz2021} & \hyperref[auth:a630]{G. Bocewicz}, \hyperref[auth:a1997]{E. Szwarc}, \hyperref[auth:a535]{J. Wikarek}, \hyperref[auth:a1527]{P. Nielsen}, \hyperref[auth:a1814]{Z. Banaszak} & \cellcolor{gold!20}A competency-driven staff assignment approach to improving employee scheduling robustness \hyperref[abs:Bocewicz2021]{Abstract} & \hyperref[detail:Bocewicz2021]{Details} No & \cite{Bocewicz2021} & 2021 & Eksploatacja i Niezawodność – Maintenance and Reliability & null & \noindent{}\textcolor{black!50}{0.00} \textbf{2.00} n/a & 5 9 9 & 38 45 & 2 0 2\\
\end{longtable}
}

\subsection{Electronic Notes in Discrete Mathematics}

\index{Electronic Notes in Discrete Mathematics}
{\scriptsize
\begin{longtable}{>{\raggedright\arraybackslash}p{2.5cm}>{\raggedright\arraybackslash}p{4.5cm}>{\raggedright\arraybackslash}p{6.0cm}p{1.0cm}rr>{\raggedright\arraybackslash}p{2.0cm}r>{\raggedright\arraybackslash}p{1cm}p{1cm}p{1cm}p{1cm}}
\rowcolor{white}\caption{Articles in Journal Electronic Notes in Discrete Mathematics (Total 1)}\\ \toprule
\rowcolor{white}\shortstack{Key\\Source} & Authors & Title (Colored by Open Access)& \shortstack{Details\\LC} & Cite & Year & \shortstack{Conference\\/Journal\\/School} & Pages & Relevance &\shortstack{Cites\\OC XR\\SC} & \shortstack{Refs\\OC\\XR} & \shortstack{Links\\Cites\\Refs}\\ \midrule\endhead
\bottomrule
\endfoot
OzturkTHO10 \href{https://www.sciencedirect.com/science/article/pii/S1571065310000107}{OzturkTHO10} & \hyperref[auth:a135]{C. {\"{O}}zt{\"{u}}rk}, \hyperref[auth:a136]{S. Tunali}, \hyperref[auth:a137]{B. Hnich}, \hyperref[auth:a138]{A. {\"{O}}rnek} & Simultaneous Balancing and Scheduling of Flexible Mixed Model Assembly Lines with Sequence-Dependent Setup Times & \hyperref[detail:OzturkTHO10]{Details} \href{../works/OzturkTHO10.pdf}{Yes} & \cite{OzturkTHO10} & 2010 & Electronic Notes in Discrete Mathematics & 8 & \noindent{}\textcolor{black!50}{0.00} \textcolor{black!50}{0.00} \textbf{1.83} & 15 15 17 & 1 3 & 1 1 0\\
\end{longtable}
}

\subsection{Electronic Proceedings in Theoretical Computer Science}

\index{Electronic Proceedings in Theoretical Computer Science}
{\scriptsize
\begin{longtable}{>{\raggedright\arraybackslash}p{2.5cm}>{\raggedright\arraybackslash}p{4.5cm}>{\raggedright\arraybackslash}p{6.0cm}p{1.0cm}rr>{\raggedright\arraybackslash}p{2.0cm}r>{\raggedright\arraybackslash}p{1cm}p{1cm}p{1cm}p{1cm}}
\rowcolor{white}\caption{Articles in Journal Electronic Proceedings in Theoretical Computer Science (Total 1)}\\ \toprule
\rowcolor{white}\shortstack{Key\\Source} & Authors & Title (Colored by Open Access)& \shortstack{Details\\LC} & Cite & Year & \shortstack{Conference\\/Journal\\/School} & Pages & Relevance &\shortstack{Cites\\OC XR\\SC} & \shortstack{Refs\\OC\\XR} & \shortstack{Links\\Cites\\Refs}\\ \midrule\endhead
\bottomrule
\endfoot
ColT2019a \href{http://dx.doi.org/10.4204/eptcs.306.30}{ColT2019a} & \hyperref[auth:a93]{G. D. Col}, \hyperref[auth:a608]{E. Teppan} & \cellcolor{gold!20}Google vs IBM: A Constraint Solving Challenge on the Job-Shop Scheduling Problem & \hyperref[detail:ColT2019a]{Details} \href{../works/ColT2019a.pdf}{Yes} & \cite{ColT2019a} & 2019 & Electronic Proceedings in Theoretical Computer Science & 7 & \noindent{}\textcolor{black!50}{0.00} \textcolor{black!50}{0.00} \textbf{4.42} & 10 13 11 & 10 18 & 10 2 8\\
\end{longtable}
}

\subsection{Eng. Appl. Artif. Intell.}

\index{Eng. Appl. Artif. Intell.}
{\scriptsize
\begin{longtable}{>{\raggedright\arraybackslash}p{2.5cm}>{\raggedright\arraybackslash}p{4.5cm}>{\raggedright\arraybackslash}p{6.0cm}p{1.0cm}rr>{\raggedright\arraybackslash}p{2.0cm}r>{\raggedright\arraybackslash}p{1cm}p{1cm}p{1cm}p{1cm}}
\rowcolor{white}\caption{Articles in Journal Eng. Appl. Artif. Intell. (Total 3)}\\ \toprule
\rowcolor{white}\shortstack{Key\\Source} & Authors & Title (Colored by Open Access)& \shortstack{Details\\LC} & Cite & Year & \shortstack{Conference\\/Journal\\/School} & Pages & Relevance &\shortstack{Cites\\OC XR\\SC} & \shortstack{Refs\\OC\\XR} & \shortstack{Links\\Cites\\Refs}\\ \midrule\endhead
\bottomrule
\endfoot
ZeballosQH10 \href{https://doi.org/10.1016/j.engappai.2009.07.002}{ZeballosQH10} & \hyperref[auth:a621]{L. J. Zeballos}, \hyperref[auth:a622]{O. Quiroga}, \hyperref[auth:a588]{G. P. Henning} & A constraint programming model for the scheduling of flexible manufacturing systems with machine and tool limitations & \hyperref[detail:ZeballosQH10]{Details} \href{../works/ZeballosQH10.pdf}{Yes} & \cite{ZeballosQH10} & 2010 & Eng. Appl. Artif. Intell. & 20 & \noindent{}\textbf{1.50} \textbf{1.50} \textbf{13.56} & 33 33 43 & 28 41 & 16 10 6\\
GarridoOS08 \href{https://doi.org/10.1016/j.engappai.2008.03.009}{GarridoOS08} & \hyperref[auth:a633]{A. Garrido}, \hyperref[auth:a635]{E. Onaindia}, \hyperref[auth:a640]{{\'{O}}scar Sapena} & Planning and scheduling in an e-learning environment. {A} constraint-programming-based approach & \hyperref[detail:GarridoOS08]{Details} \href{../works/GarridoOS08.pdf}{Yes} & \cite{GarridoOS08} & 2008 & Eng. Appl. Artif. Intell. & 11 & \noindent{}\textcolor{black!50}{0.00} \textcolor{black!50}{0.00} \textbf{15.57} & 22 22 28 & 7 24 & 4 3 1\\
KovacsB08 \href{https://doi.org/10.1016/j.engappai.2008.03.004}{KovacsB08} & \hyperref[auth:a146]{A. Kov{\'{a}}cs}, \hyperref[auth:a89]{J. C. Beck} & \cellcolor{green!10}A global constraint for total weighted completion time for cumulative resources & \hyperref[detail:KovacsB08]{Details} \href{../works/KovacsB08.pdf}{Yes} & \cite{KovacsB08} & 2008 & Eng. Appl. Artif. Intell. & 7 & \noindent{}\textcolor{black!50}{0.00} \textcolor{black!50}{0.00} \textbf{2.37} & 5 5 5 & 14 20 & 5 1 4\\
\end{longtable}
}

\subsection{Engineering Applications of Artificial Intelligence}

\index{Engineering Applications of Artificial Intelligence}
{\scriptsize
\begin{longtable}{>{\raggedright\arraybackslash}p{2.5cm}>{\raggedright\arraybackslash}p{4.5cm}>{\raggedright\arraybackslash}p{6.0cm}p{1.0cm}rr>{\raggedright\arraybackslash}p{2.0cm}r>{\raggedright\arraybackslash}p{1cm}p{1cm}p{1cm}p{1cm}}
\rowcolor{white}\caption{Articles in Journal Engineering Applications of Artificial Intelligence (Total 4)}\\ \toprule
\rowcolor{white}\shortstack{Key\\Source} & Authors & Title (Colored by Open Access)& \shortstack{Details\\LC} & Cite & Year & \shortstack{Conference\\/Journal\\/School} & Pages & Relevance &\shortstack{Cites\\OC XR\\SC} & \shortstack{Refs\\OC\\XR} & \shortstack{Links\\Cites\\Refs}\\ \midrule\endhead
\bottomrule
\endfoot
Song2022 \href{http://dx.doi.org/10.1016/j.engappai.2021.104603}{Song2022} & \hyperref[auth:a1874]{W. Song}, \hyperref[auth:a1875]{Z. Cao}, \hyperref[auth:a1876]{J. Zhang}, \hyperref[auth:a1877]{C. Xu}, \hyperref[auth:a279]{A. Lim} & \cellcolor{green!10}Learning variable ordering heuristics for solving Constraint Satisfaction Problems & \hyperref[detail:Song2022]{Details} No & \cite{Song2022} & 2022 & Engineering Applications of Artificial Intelligence & null & \noindent{}0.50 0.50 n/a & 12 15 20 & 22 55 & 3 1 2\\
Artigues2011 \href{http://dx.doi.org/10.1016/j.engappai.2010.07.008}{Artigues2011} & \hyperref[auth:a6]{C. Artigues}, \hyperref[auth:a1199]{M.-J. Huguet}, \hyperref[auth:a3]{P. Lopez} & \cellcolor{green!10}Generalized disjunctive constraint propagation for solving the job shop problem with time lags & \hyperref[detail:Artigues2011]{Details} No & \cite{Artigues2011} & 2011 & Engineering Applications of Artificial Intelligence & null & \noindent{}\textbf{1.50} \textbf{1.50} n/a & 22 22 28 & 16 25 & 7 2 5\\
Salido2008 \href{http://dx.doi.org/10.1016/j.engappai.2008.03.007}{Salido2008} & \hyperref[auth:a153]{M. A. Salido}, \hyperref[auth:a633]{A. Garrido}, \hyperref[auth:a1063]{R. Barták} & Introduction: Special issue on constraint satisfaction techniques for planning and scheduling problems & \hyperref[detail:Salido2008]{Details} No & \cite{Salido2008} & 2008 & Engineering Applications of Artificial Intelligence & null & \noindent{}\textbf{1.00} \textbf{1.00} n/a & 12 13 16 & 6 7 & 6 4 2\\
Salido2008a \href{http://dx.doi.org/10.1016/j.engappai.2008.03.006}{Salido2008a} & \hyperref[auth:a153]{M. A. Salido}, \hyperref[auth:a1941]{A. Giret} & \cellcolor{green!10}Feasible distributed CSP models for scheduling problems & \hyperref[detail:Salido2008a]{Details} No & \cite{Salido2008a} & 2008 & Engineering Applications of Artificial Intelligence & null & \noindent{}\textbf{1.00} \textbf{1.00} n/a & 8 8 10 & 14 25 & 1 1 0\\
\end{longtable}
}

\subsection{Engineering Optimization}

\index{Engineering Optimization}
{\scriptsize
\begin{longtable}{>{\raggedright\arraybackslash}p{2.5cm}>{\raggedright\arraybackslash}p{4.5cm}>{\raggedright\arraybackslash}p{6.0cm}p{1.0cm}rr>{\raggedright\arraybackslash}p{2.0cm}r>{\raggedright\arraybackslash}p{1cm}p{1cm}p{1cm}p{1cm}}
\rowcolor{white}\caption{Articles in Journal Engineering Optimization (Total 6)}\\ \toprule
\rowcolor{white}\shortstack{Key\\Source} & Authors & Title (Colored by Open Access)& \shortstack{Details\\LC} & Cite & Year & \shortstack{Conference\\/Journal\\/School} & Pages & Relevance &\shortstack{Cites\\OC XR\\SC} & \shortstack{Refs\\OC\\XR} & \shortstack{Links\\Cites\\Refs}\\ \midrule\endhead
\bottomrule
\endfoot
AbreuAPNM21 \href{http://dx.doi.org/10.1080/0305215x.2021.1957101}{AbreuAPNM21} & \hyperref[auth:a418]{L. R. de Abreu}, \hyperref[auth:a747]{K. A. G. Araújo}, \hyperref[auth:a748]{B. de A. Prata}, \hyperref[auth:a419]{M. S. Nagano}, \hyperref[auth:a749]{J. V. Moccellin} & A new variable neighbourhood search with a constraint programming search strategy for the open shop scheduling problem with operation repetitions & \hyperref[detail:AbreuAPNM21]{Details} \href{../works/AbreuAPNM21.pdf}{Yes} & \cite{AbreuAPNM21} & 2021 & \cellcolor{red!20}Engineering Optimization & 20 & \noindent{}\textbf{1.00} \textbf{1.00} \textbf{32.82} & 5 4 7 & 50 58 & 11 2 9\\
Pinarbasi21 \href{http://dx.doi.org/10.1080/0305215x.2021.1921171}{Pinarbasi21} & \hyperref[auth:a1384]{M. Pınarbaşı} & New mathematical and constraint programming models for U-type assembly line balancing problems with assignment restrictions & \hyperref[detail:Pinarbasi21]{Details} No & \cite{Pinarbasi21} & 2021 & \cellcolor{red!20}Engineering Optimization & 16 & \noindent{}\textcolor{black!50}{0.00} \textcolor{black!50}{0.00} n/a & 3 6 0 & 46 50 & 8 2 6\\
GuoHLW20 \href{http://dx.doi.org/10.1080/0305215x.2019.1699919}{GuoHLW20} & \hyperref[auth:a931]{P. Guo}, \hyperref[auth:a932]{X. He}, \hyperref[auth:a933]{Y. Luan}, \hyperref[auth:a934]{Y. Wang} & Logic-based Benders decomposition for gantry crane scheduling with transferring position constraints in a rail–road container terminal & \hyperref[detail:GuoHLW20]{Details} No & \cite{GuoHLW20} & 2020 & \cellcolor{red!20}Engineering Optimization & 21 & \noindent{}\textcolor{black!50}{0.00} \textcolor{black!50}{0.00} n/a & 8 10 8 & 31 34 & 12 0 12\\
KizilayC20 \href{http://dx.doi.org/10.1080/0305215x.2020.1786081}{KizilayC20} & \hyperref[auth:a1380]{D. Kizilay}, \hyperref[auth:a1381]{Z. A. Cil} & Constraint programming approach for multi-objective two-sided assembly line balancing problem with multi-operator stations & \hyperref[detail:KizilayC20]{Details} No & \cite{KizilayC20} & 2020 & \cellcolor{red!20}Engineering Optimization & 16 & \noindent{}\textcolor{black!50}{0.00} \textcolor{black!50}{0.00} n/a & 11 12 14 & 38 39 & 12 6 6\\
PinarbasiA20 \href{http://dx.doi.org/10.1080/0305215x.2020.1716746}{PinarbasiA20} & \hyperref[auth:a1384]{M. Pınarbaşı}, \hyperref[auth:a764]{H. M. Alakaş} & Balancing stochastic type-II assembly lines: chance-constrained mixed integer and constraint programming models & \hyperref[detail:PinarbasiA20]{Details} No & \cite{PinarbasiA20} & 2020 & \cellcolor{red!20}Engineering Optimization & 18 & \noindent{}\textcolor{black!50}{0.00} \textcolor{black!50}{0.00} n/a & 7 9 11 & 32 36 & 10 3 7\\
EdisO11a \href{http://dx.doi.org/10.1080/03052151003759117}{EdisO11a} & \hyperref[auth:a346]{E. B. Edis}, \hyperref[auth:a348]{I. Ozkarahan} & A combined integer/constraint programming approach to a resource-constrained parallel machine scheduling problem with machine eligibility restrictions & \hyperref[detail:EdisO11a]{Details} No & \cite{EdisO11a} & 2011 & \cellcolor{red!20}Engineering Optimization & 23 & \noindent{}\textbf{2.00} \textbf{2.00} n/a & 43 46 51 & 37 53 & 28 13 15\\
\end{longtable}
}

\subsection{Engineering, Construction and Architectural Management}

\index{Engineering, Construction and Architectural Management}
{\scriptsize
\begin{longtable}{>{\raggedright\arraybackslash}p{2.5cm}>{\raggedright\arraybackslash}p{4.5cm}>{\raggedright\arraybackslash}p{6.0cm}p{1.0cm}rr>{\raggedright\arraybackslash}p{2.0cm}r>{\raggedright\arraybackslash}p{1cm}p{1cm}p{1cm}p{1cm}}
\rowcolor{white}\caption{Articles in Journal Engineering, Construction and Architectural Management (Total 1)}\\ \toprule
\rowcolor{white}\shortstack{Key\\Source} & Authors & Title (Colored by Open Access)& \shortstack{Details\\LC} & Cite & Year & \shortstack{Conference\\/Journal\\/School} & Pages & Relevance &\shortstack{Cites\\OC XR\\SC} & \shortstack{Refs\\OC\\XR} & \shortstack{Links\\Cites\\Refs}\\ \midrule\endhead
\bottomrule
\endfoot
Zou2021 \href{http://dx.doi.org/10.1108/ecam-10-2020-0843}{Zou2021} & \hyperref[auth:a756]{X. Zou}, \hyperref[auth:a757]{L. Zhang}, \hyperref[auth:a1483]{Q. Zhang} & Time-cost optimization in repetitive project scheduling with limited resources \hyperref[abs:Zou2021]{Abstract} & \hyperref[detail:Zou2021]{Details} No & \cite{Zou2021} & 2021 & Engineering, Construction and Architectural Management & null & \noindent{}\textcolor{black!50}{0.00} \textbf{4.00} n/a & 3 9 12 & 27 43 & 6 0 6\\
\end{longtable}
}

\subsection{Engineering, Technology \& Applied Science Research}

\index{Engineering, Technology \& Applied Science Research}
{\scriptsize
\begin{longtable}{>{\raggedright\arraybackslash}p{2.5cm}>{\raggedright\arraybackslash}p{4.5cm}>{\raggedright\arraybackslash}p{6.0cm}p{1.0cm}rr>{\raggedright\arraybackslash}p{2.0cm}r>{\raggedright\arraybackslash}p{1cm}p{1cm}p{1cm}p{1cm}}
\rowcolor{white}\caption{Articles in Journal Engineering, Technology \  Applied Science Research (Total 1)}\\ \toprule
\rowcolor{white}\shortstack{Key\\Source} & Authors & Title (Colored by Open Access)& \shortstack{Details\\LC} & Cite & Year & \shortstack{Conference\\/Journal\\/School} & Pages & Relevance &\shortstack{Cites\\OC XR\\SC} & \shortstack{Refs\\OC\\XR} & \shortstack{Links\\Cites\\Refs}\\ \midrule\endhead
\bottomrule
\endfoot
Abdul-Niby2016 \href{http://dx.doi.org/10.48084/etasr.627}{Abdul-Niby2016} & \hyperref[auth:a1855]{M. Abdul-Niby}, \hyperref[auth:a1856]{M. Alameen}, \hyperref[auth:a1857]{A. Salhieh}, \hyperref[auth:a1858]{A. Radhi} & Improved Genetic and Simulating Annealing Algorithms to Solve the Traveling Salesman Problem Using Constraint Programming \hyperref[abs:Abdul-Niby2016]{Abstract} & \hyperref[detail:Abdul-Niby2016]{Details} No & \cite{Abdul-Niby2016} & 2016 & Engineering, Technology \  Applied Science Research & null & \noindent{}\textcolor{black!50}{0.00} 0.50 n/a & 3 4 0 & 7 8 & 1 1 0\\
\end{longtable}
}

\subsection{Eur. J. Control}

\index{Eur. J. Control}
{\scriptsize
\begin{longtable}{>{\raggedright\arraybackslash}p{2.5cm}>{\raggedright\arraybackslash}p{4.5cm}>{\raggedright\arraybackslash}p{6.0cm}p{1.0cm}rr>{\raggedright\arraybackslash}p{2.0cm}r>{\raggedright\arraybackslash}p{1cm}p{1cm}p{1cm}p{1cm}}
\rowcolor{white}\caption{Articles in Journal Eur. J. Control (Total 2)}\\ \toprule
\rowcolor{white}\shortstack{Key\\Source} & Authors & Title (Colored by Open Access)& \shortstack{Details\\LC} & Cite & Year & \shortstack{Conference\\/Journal\\/School} & Pages & Relevance &\shortstack{Cites\\OC XR\\SC} & \shortstack{Refs\\OC\\XR} & \shortstack{Links\\Cites\\Refs}\\ \midrule\endhead
\bottomrule
\endfoot
KorbaaYG00 \href{https://doi.org/10.1016/S0947-3580(00)71113-7}{KorbaaYG00} & \hyperref[auth:a680]{O. Korbaa}, \hyperref[auth:a681]{P. Yim}, \hyperref[auth:a682]{J.-C. Gentina} & Solving Transient Scheduling Problems with Constraint Programming & \hyperref[detail:KorbaaYG00]{Details} \href{../works/KorbaaYG00.pdf}{Yes} & \cite{KorbaaYG00} & 2000 & Eur. J. Control & 10 & \noindent{}\textbf{1.00} \textbf{1.00} \textcolor{black!50}{0.00} & 7 7 9 & 4 15 & 2 2 0\\
LopezAKYG00 \href{https://doi.org/10.1016/S0947-3580(00)71114-9}{LopezAKYG00} & \hyperref[auth:a3]{P. Lopez}, \hyperref[auth:a683]{H. Alla}, \hyperref[auth:a680]{O. Korbaa}, \hyperref[auth:a681]{P. Yim}, \hyperref[auth:a682]{J.-C. Gentina} & Discussion on: 'Solving Transient Scheduling Problems with Constraint Programming' by O. Korbaa, P. Yim, and {J.-C.} Gentina & \hyperref[detail:LopezAKYG00]{Details} \href{../works/LopezAKYG00.pdf}{Yes} & \cite{LopezAKYG00} & 2000 & Eur. J. Control & 4 & \noindent{}\textbf{1.00} \textbf{1.00} \textcolor{black!50}{0.00} & 0 0 0 & 0 1 & 0 0 0\\
\end{longtable}
}

\subsection{European J. of Industrial Engineering}

\index{European J. of Industrial Engineering}
{\scriptsize
\begin{longtable}{>{\raggedright\arraybackslash}p{2.5cm}>{\raggedright\arraybackslash}p{4.5cm}>{\raggedright\arraybackslash}p{6.0cm}p{1.0cm}rr>{\raggedright\arraybackslash}p{2.0cm}r>{\raggedright\arraybackslash}p{1cm}p{1cm}p{1cm}p{1cm}}
\rowcolor{white}\caption{Articles in Journal European J. of Industrial Engineering (Total 1)}\\ \toprule
\rowcolor{white}\shortstack{Key\\Source} & Authors & Title (Colored by Open Access)& \shortstack{Details\\LC} & Cite & Year & \shortstack{Conference\\/Journal\\/School} & Pages & Relevance &\shortstack{Cites\\OC XR\\SC} & \shortstack{Refs\\OC\\XR} & \shortstack{Links\\Cites\\Refs}\\ \midrule\endhead
\bottomrule
\endfoot
Hat2011 \href{http://dx.doi.org/10.1504/ejie.2011.042742}{Hat2011} & \hyperref[auth:a1162]{A. Haït}, \hyperref[auth:a6]{C. Artigues} & \cellcolor{green!10}A hybrid CP/MILP method for scheduling with energy costs & \hyperref[detail:Hat2011]{Details} No & \cite{Hat2011} & 2011 & European J. of Industrial Engineering & null & \noindent{}\textbf{1.00} \textbf{1.00} n/a & 20 20 25 & 0 0 & 1 1 0\\
\end{longtable}
}

\subsection{European Journal of Operational Research}

\index{European Journal of Operational Research}
{\scriptsize
\begin{longtable}{>{\raggedright\arraybackslash}p{2.5cm}>{\raggedright\arraybackslash}p{4.5cm}>{\raggedright\arraybackslash}p{6.0cm}p{1.0cm}rr>{\raggedright\arraybackslash}p{2.0cm}r>{\raggedright\arraybackslash}p{1cm}p{1cm}p{1cm}p{1cm}}
\rowcolor{white}\caption{Articles in Journal European Journal of Operational Research (Total 32)}\\ \toprule
\rowcolor{white}\shortstack{Key\\Source} & Authors & Title (Colored by Open Access)& \shortstack{Details\\LC} & Cite & Year & \shortstack{Conference\\/Journal\\/School} & Pages & Relevance &\shortstack{Cites\\OC XR\\SC} & \shortstack{Refs\\OC\\XR} & \shortstack{Links\\Cites\\Refs}\\ \midrule\endhead
\bottomrule
\endfoot
ForbesHJST24 \href{http://dx.doi.org/10.1016/j.ejor.2023.07.032}{ForbesHJST24} & \hyperref[auth:a983]{M. Forbes}, \hyperref[auth:a984]{M. Harris}, \hyperref[auth:a985]{H. Jansen}, \hyperref[auth:a986]{F. van der Schoot}, \hyperref[auth:a987]{T. Taimre} & \cellcolor{gold!20}Combining optimisation and simulation using logic-based Benders decomposition & \hyperref[detail:ForbesHJST24]{Details} \href{../works/ForbesHJST24.pdf}{Yes} & \cite{ForbesHJST24} & 2024 & European Journal of Operational Research & 15 & \noindent{}\textcolor{black!50}{0.00} \textcolor{black!50}{0.00} \textbf{4.07} & 0 0 0 & 26 37 & 9 0 9\\
GuoZ23 \href{http://dx.doi.org/10.1016/j.ejor.2023.06.006}{GuoZ23} & \hyperref[auth:a943]{P. Guo}, \hyperref[auth:a944]{J. Zhu} & Capacity reservation for humanitarian relief: A logic-based Benders decomposition method with subgradient cut & \hyperref[detail:GuoZ23]{Details} \href{../works/GuoZ23.pdf}{Yes} & \cite{GuoZ23} & 2023 & European Journal of Operational Research & 29 & \noindent{}\textcolor{black!50}{0.00} \textcolor{black!50}{0.00} 0.20 & 0 1 1 & 112 145 & 18 0 18\\
MullerMKP22 \href{https://doi.org/10.1016/j.ejor.2022.01.034}{MullerMKP22} & \hyperref[auth:a435]{D. M{\"{u}}ller}, \hyperref[auth:a436]{M. G. M{\"{u}}ller}, \hyperref[auth:a437]{D. Kress}, \hyperref[auth:a438]{E. Pesch} & An algorithm selection approach for the flexible job shop scheduling problem: Choosing constraint programming solvers through machine learning & \hyperref[detail:MullerMKP22]{Details} \href{../works/MullerMKP22.pdf}{Yes} & \cite{MullerMKP22} & 2022 & European Journal of Operational Research & 18 & \noindent{}\textbf{2.50} \textbf{2.50} \textbf{12.25} & 17 19 20 & 59 93 & 16 3 13\\
PohlAK22 \href{https://doi.org/10.1016/j.ejor.2021.08.028}{PohlAK22} & \hyperref[auth:a439]{M. Pohl}, \hyperref[auth:a6]{C. Artigues}, \hyperref[auth:a440]{R. Kolisch} & \cellcolor{green!10}Solving the time-discrete winter runway scheduling problem: {A} column generation and constraint programming approach & \hyperref[detail:PohlAK22]{Details} \href{../works/PohlAK22.pdf}{Yes} & \cite{PohlAK22} & 2022 & European Journal of Operational Research & 16 & \noindent{}\textbf{1.00} \textbf{1.00} \textbf{5.76} & 4 4 4 & 31 44 & 4 1 3\\
RoshanaeiN21 \href{http://dx.doi.org/10.1016/j.ejor.2020.12.004}{RoshanaeiN21} & \hyperref[auth:a728]{V. Roshanaei}, \hyperref[auth:a726]{B. Naderi} & Solving integrated operating room planning and scheduling: Logic-based Benders decomposition versus Branch-Price-and-Cut & \hyperref[detail:RoshanaeiN21]{Details} \href{../works/RoshanaeiN21.pdf}{Yes} & \cite{RoshanaeiN21} & 2021 & European Journal of Operational Research & 14 & \noindent{}\textcolor{black!50}{0.00} \textcolor{black!50}{0.00} \textbf{6.68} & 15 15 15 & 44 51 & 17 7 10\\
CarlierPSJ20 \href{http://dx.doi.org/10.1016/j.ejor.2020.03.079}{CarlierPSJ20} & \hyperref[auth:a845]{J. Carlier}, \hyperref[auth:a846]{E. Pinson}, \hyperref[auth:a1239]{A. Sahli}, \hyperref[auth:a1240]{A. Jouglet} & \cellcolor{gold!20}An O(n2) algorithm for time-bound adjustments for the cumulative scheduling problem & \hyperref[detail:CarlierPSJ20]{Details} \href{../works/CarlierPSJ20.pdf}{Yes} & \cite{CarlierPSJ20} & 2020 & European Journal of Operational Research & 9 & \noindent{}\textcolor{black!50}{0.00} \textcolor{black!50}{0.00} 0.60 & 6 7 8 & 10 19 & 11 4 7\\
MejiaY20 \href{https://doi.org/10.1016/j.ejor.2020.02.010}{MejiaY20} & \hyperref[auth:a424]{G. Mej{\'{\i}}a}, \hyperref[auth:a405]{F. Yuraszeck} & A self-tuning variable neighborhood search algorithm and an effective decoding scheme for open shop scheduling problems with travel/setup times & \hyperref[detail:MejiaY20]{Details} \href{../works/MejiaY20.pdf}{Yes} & \cite{MejiaY20} & 2020 & European Journal of Operational Research & 13 & \noindent{}\textcolor{black!50}{0.00} \textcolor{black!50}{0.00} \textbf{12.71} & 24 29 34 & 45 50 & 6 4 2\\
QinDCS20 \href{https://doi.org/10.1016/j.ejor.2020.02.021}{QinDCS20} & \hyperref[auth:a509]{T. Qin}, \hyperref[auth:a510]{Y. Du}, \hyperref[auth:a511]{J. H. Chen}, \hyperref[auth:a512]{M. Sha} & Combining mixed integer programming and constraint programming to solve the integrated scheduling problem of container handling operations of a single vessel & \hyperref[detail:QinDCS20]{Details} \href{../works/QinDCS20.pdf}{Yes} & \cite{QinDCS20} & 2020 & European Journal of Operational Research & 18 & \noindent{}\textbf{1.00} \textbf{1.00} \textbf{20.41} & 27 30 31 & 30 35 & 9 6 3\\
ArkhipovBL19 \href{http://dx.doi.org/10.1016/j.ejor.2018.11.005}{ArkhipovBL19} & \hyperref[auth:a924]{D. Arkhipov}, \hyperref[auth:a925]{O. Battaïa}, \hyperref[auth:a926]{A. Lazarev} & \cellcolor{gold!20}An efficient pseudo-polynomial algorithm for finding a lower bound on the makespan for the Resource Constrained Project Scheduling Problem & \hyperref[detail:ArkhipovBL19]{Details} \href{../works/ArkhipovBL19.pdf}{Yes} & \cite{ArkhipovBL19} & 2019 & European Journal of Operational Research & 10 & \noindent{}\textcolor{black!50}{0.00} \textcolor{black!50}{0.00} \textbf{2.65} & 12 13 14 & 24 40 & 12 1 11\\
SunTB19 \href{http://dx.doi.org/10.1016/j.ejor.2018.08.009}{SunTB19} & \hyperref[auth:a1195]{D. Sun}, \hyperref[auth:a1196]{L. Tang}, \hyperref[auth:a1197]{R. Baldacci} & \cellcolor{green!10}A Benders decomposition-based framework for solving quay crane scheduling problems & \hyperref[detail:SunTB19]{Details} \href{../works/SunTB19.pdf}{Yes} & \cite{SunTB19} & 2019 & European Journal of Operational Research & 12 & \noindent{}\textcolor{black!50}{0.00} \textcolor{black!50}{0.00} \textcolor{black!50}{0.06} & 31 34 37 & 29 31 & 9 3 6\\
KreterSSZ18 \href{https://doi.org/10.1016/j.ejor.2017.10.014}{KreterSSZ18} & \hyperref[auth:a123]{S. Kreter}, \hyperref[auth:a124]{A. Schutt}, \hyperref[auth:a125]{P. J. Stuckey}, \hyperref[auth:a792]{J. Zimmermann} & Mixed-integer linear programming and constraint programming formulations for solving resource availability cost problems & \hyperref[detail:KreterSSZ18]{Details} \href{../works/KreterSSZ18.pdf}{Yes} & \cite{KreterSSZ18} & 2018 & European Journal of Operational Research & 15 & \noindent{}0.50 0.50 \textbf{18.78} & 25 29 26 & 31 54 & 22 14 8\\
PourDERB18 \href{https://doi.org/10.1016/j.ejor.2017.08.033}{PourDERB18} & \hyperref[auth:a564]{S. M. Pour}, \hyperref[auth:a565]{J. H. Drake}, \hyperref[auth:a566]{L. S. Ejlertsen}, \hyperref[auth:a567]{K. M. Rasmussen}, \hyperref[auth:a568]{E. K. Burke} & \cellcolor{gold!20}A hybrid Constraint Programming/Mixed Integer Programming framework for the preventive signaling maintenance crew scheduling problem & \hyperref[detail:PourDERB18]{Details} \href{../works/PourDERB18.pdf}{Yes} & \cite{PourDERB18} & 2018 & European Journal of Operational Research & 12 & \noindent{}\textbf{1.00} \textbf{1.00} \textbf{30.35} & 41 47 48 & 13 25 & 16 14 2\\
CarlssonJL17 \href{https://doi.org/10.1016/j.ejor.2016.11.033}{CarlssonJL17} & \hyperref[auth:a91]{M. Carlsson}, \hyperref[auth:a75]{M. Johansson}, \hyperref[auth:a1412]{J. Larson} & \cellcolor{gold!20}Scheduling double round-robin tournaments with divisional play using constraint programming & \hyperref[detail:CarlssonJL17]{Details} \href{../works/CarlssonJL17.pdf}{Yes} & \cite{CarlssonJL17} & 2017 & European Journal of Operational Research & 11 & \noindent{}\textbf{1.00} \textbf{1.00} \textbf{3.77} & 10 11 12 & 14 42 & 6 1 5\\
RoshanaeiLAU17 \href{http://dx.doi.org/10.1016/j.ejor.2016.08.024}{RoshanaeiLAU17} & \hyperref[auth:a728]{V. Roshanaei}, \hyperref[auth:a927]{C. Luong}, \hyperref[auth:a895]{D. M. Aleman}, \hyperref[auth:a896]{D. R. Urbach} & Propagating logic-based Benders' decomposition approaches for distributed operating room scheduling & \hyperref[detail:RoshanaeiLAU17]{Details} \href{../works/RoshanaeiLAU17.pdf}{Yes} & \cite{RoshanaeiLAU17} & 2017 & European Journal of Operational Research & 17 & \noindent{}\textcolor{black!50}{0.00} \textcolor{black!50}{0.00} \textbf{2.97} & 61 66 65 & 46 53 & 24 14 10\\
RiiseML16 \href{http://dx.doi.org/10.1016/j.ejor.2016.06.015}{RiiseML16} & \hyperref[auth:a1064]{A. Riise}, \hyperref[auth:a1065]{C. Mannino}, \hyperref[auth:a1066]{L. Lamorgese} & Recursive logic-based Benders' decomposition for multi-mode outpatient scheduling & \hyperref[detail:RiiseML16]{Details} \href{../works/RiiseML16.pdf}{Yes} & \cite{RiiseML16} & 2016 & European Journal of Operational Research & 10 & \noindent{}\textcolor{black!50}{0.00} \textcolor{black!50}{0.00} 0.72 & 27 27 26 & 29 41 & 14 8 6\\
GoelSHFS15 \href{https://doi.org/10.1016/j.ejor.2014.09.048}{GoelSHFS15} & \hyperref[auth:a592]{V. Goel}, \hyperref[auth:a593]{M. Slusky}, \hyperref[auth:a206]{W.-J. van Hoeve}, \hyperref[auth:a594]{K. C. Furman}, \hyperref[auth:a595]{Y. Shao} & Constraint programming for {LNG} ship scheduling and inventory management & \hyperref[detail:GoelSHFS15]{Details} \href{../works/GoelSHFS15.pdf}{Yes} & \cite{GoelSHFS15} & 2015 & European Journal of Operational Research & 12 & \noindent{}\textbf{1.00} \textbf{1.00} \textbf{8.63} & 48 53 54 & 4 8 & 9 9 0\\
WangMD15 \href{https://doi.org/10.1016/j.ejor.2015.06.008}{WangMD15} & \hyperref[auth:a596]{T. Wang}, \hyperref[auth:a597]{N. Meskens}, \hyperref[auth:a598]{D. Duvivier} & Scheduling operating theatres: Mixed integer programming vs. constraint programming & \hyperref[detail:WangMD15]{Details} \href{../works/WangMD15.pdf}{Yes} & \cite{WangMD15} & 2015 & European Journal of Operational Research & 13 & \noindent{}\textbf{1.00} \textbf{1.00} \textbf{6.87} & 36 38 38 & 33 49 & 21 15 6\\
MalapertGR12 \href{http://dx.doi.org/10.1016/j.ejor.2012.04.008}{MalapertGR12} & \hyperref[auth:a82]{A. Malapert}, \hyperref[auth:a1375]{C. Guéret}, \hyperref[auth:a326]{L.-M. Rousseau} & \cellcolor{green!10}A constraint programming approach for a batch processing problem with non-identical job sizes & \hyperref[detail:MalapertGR12]{Details} \href{../works/MalapertGR12.pdf}{Yes} & \cite{MalapertGR12} & 2012 & European Journal of Operational Research & 13 & \noindent{}\textbf{1.00} \textbf{1.00} \textbf{10.74} & 43 44 50 & 24 41 & 7 7 0\\
Acuna-Agost2011 \href{http://dx.doi.org/10.1016/j.ejor.2011.05.047}{Acuna-Agost2011} & \hyperref[auth:a354]{R. Acuna-Agost}, \hyperref[auth:a355]{P. Michelon}, \hyperref[auth:a356]{D. Feillet}, \hyperref[auth:a357]{S. Gueye} & SAPI: Statistical Analysis of Propagation of Incidents. A new approach for rescheduling trains after disruptions & \hyperref[detail:Acuna-Agost2011]{Details} No & \cite{Acuna-Agost2011} & 2011 & European Journal of Operational Research & null & \noindent{}0.50 0.50 n/a & 36 37 45 & 26 52 & 3 0 3\\
Coelho2011 \href{http://dx.doi.org/10.1016/j.ejor.2011.03.019}{Coelho2011} & \hyperref[auth:a1555]{J. Coelho}, \hyperref[auth:a1556]{M. Vanhoucke} & \cellcolor{green!10}Multi-mode resource-constrained project scheduling using RCPSP and SAT solvers & \hyperref[detail:Coelho2011]{Details} No & \cite{Coelho2011} & 2011 & European Journal of Operational Research & null & \noindent{}\textcolor{black!50}{0.00} \textcolor{black!50}{0.00} n/a & 82 89 102 & 48 58 & 15 9 6\\
RasmussenT07 \href{http://dx.doi.org/10.1016/j.ejor.2005.10.063}{RasmussenT07} & \hyperref[auth:a1403]{R. V. Rasmussen}, \hyperref[auth:a1389]{M. A. Trick} & A Benders approach for the constrained minimum break problem \hyperref[abs:RasmussenT07]{Abstract} & \hyperref[detail:RasmussenT07]{Details} \href{../works/RasmussenT07.pdf}{Yes} & \cite{RasmussenT07} & 2007 & European Journal of Operational Research & 16 & \noindent{}\textcolor{black!50}{0.00} \textcolor{black!50}{0.00} 0.52 & 60 62 71 & 16 27 & 21 14 7\\
KhayatLR06 \href{https://doi.org/10.1016/j.ejor.2005.02.077}{KhayatLR06} & \hyperref[auth:a644]{G. E. Khayat}, \hyperref[auth:a645]{A. Langevin}, \hyperref[auth:a646]{D. Riopel} & Integrated production and material handling scheduling using mathematical programming and constraint programming & \hyperref[detail:KhayatLR06]{Details} \href{../works/KhayatLR06.pdf}{Yes} & \cite{KhayatLR06} & 2006 & European Journal of Operational Research & 15 & \noindent{}\textbf{1.00} \textbf{1.00} \textbf{19.97} & 84 89 96 & 14 26 & 14 12 2\\
HenzMT04 \href{http://dx.doi.org/10.1016/s0377-2217(03)00101-2}{HenzMT04} & \hyperref[auth:a1419]{M. Henz}, \hyperref[auth:a1421]{T. M\"{u}ller}, \hyperref[auth:a1422]{S. Thiel} & Global constraints for round robin tournament scheduling & \hyperref[detail:HenzMT04]{Details} \href{../works/HenzMT04.pdf}{Yes} & \cite{HenzMT04} & 2004 & European Journal of Operational Research & 10 & \noindent{}\textcolor{black!50}{0.00} \textcolor{black!50}{0.00} 0.69 & 44 47 0 & 8 24 & 12 10 2\\
PoderBS04 \href{https://doi.org/10.1016/S0377-2217(02)00756-7}{PoderBS04} & \hyperref[auth:a358]{E. Poder}, \hyperref[auth:a128]{N. Beldiceanu}, \hyperref[auth:a713]{E. Sanlaville} & Computing a lower approximation of the compulsory part of a task with varying duration and varying resource consumption & \hyperref[detail:PoderBS04]{Details} \href{../works/PoderBS04.pdf}{Yes} & \cite{PoderBS04} & 2004 & European Journal of Operational Research & 16 & \noindent{}\textcolor{black!50}{0.00} \textcolor{black!50}{0.00} \textbf{4.99} & 7 7 10 & 8 18 & 7 2 5\\
ArtiguesR00 \href{https://doi.org/10.1016/S0377-2217(99)00496-8}{ArtiguesR00} & \hyperref[auth:a6]{C. Artigues}, \hyperref[auth:a712]{F. Roubellat} & A polynomial activity insertion algorithm in a multi-resource schedule with cumulative constraints and multiple modes & \hyperref[detail:ArtiguesR00]{Details} \href{../works/ArtiguesR00.pdf}{Yes} & \cite{ArtiguesR00} & 2000 & European Journal of Operational Research & 20 & \noindent{}\textcolor{black!50}{0.00} \textcolor{black!50}{0.00} \textcolor{black!50}{0.00} & 84 85 86 & 3 8 & 3 3 0\\
BruckerK00 \href{http://dx.doi.org/10.1016/s0377-2217(99)00489-0}{BruckerK00} & \hyperref[auth:a847]{P. Brucker}, \hyperref[auth:a1166]{S. Knust} & A linear programming and constraint propagation-based lower bound for the RCPSP & \hyperref[detail:BruckerK00]{Details} \href{../works/BruckerK00.pdf}{Yes} & \cite{BruckerK00} & 2000 & European Journal of Operational Research & 8 & \noindent{}\textcolor{black!50}{0.00} \textcolor{black!50}{0.00} \textbf{1.12} & 66 67 80 & 8 12 & 14 13 1\\
TorresL00 \href{http://dx.doi.org/10.1016/s0377-2217(99)00497-x}{TorresL00} & \hyperref[auth:a873]{P. Torres}, \hyperref[auth:a3]{P. Lopez} & \cellcolor{green!10}On Not-First/Not-Last conditions in disjunctive scheduling & \hyperref[detail:TorresL00]{Details} \href{../works/TorresL00.pdf}{Yes} & \cite{TorresL00} & 2000 & European Journal of Operational Research & 12 & \noindent{}\textcolor{black!50}{0.00} \textcolor{black!50}{0.00} \textbf{7.36} & 26 26 26 & 13 27 & 26 20 6\\
JainM99 \href{http://dx.doi.org/10.1016/s0377-2217(98)00113-1}{JainM99} & \hyperref[auth:a954]{A. Jain}, \hyperref[auth:a955]{S. Meeran} & Deterministic job-shop scheduling: Past, present and future & \hyperref[detail:JainM99]{Details} \href{../works/JainM99.pdf}{Yes} & \cite{JainM99} & 1999 & European Journal of Operational Research & 45 & \noindent{}\textcolor{black!50}{0.00} \textcolor{black!50}{0.00} \textbf{7.96} & 490 503 630 & 150 262 & 26 10 16\\
PesantGPR99 \href{http://dx.doi.org/10.1016/s0377-2217(98)00248-3}{PesantGPR99} & \hyperref[auth:a8]{G. Pesant}, \hyperref[auth:a616]{M. Gendreau}, \hyperref[auth:a1202]{J.-Y. Potvin}, \hyperref[auth:a1203]{J.-M. Rousseau} & On the flexibility of constraint programming models: From single to multiple time windows for the traveling salesman problem & \hyperref[detail:PesantGPR99]{Details} \href{../works/PesantGPR99.pdf}{Yes} & \cite{PesantGPR99} & 1999 & European Journal of Operational Research & 11 & \noindent{}\textcolor{black!50}{0.00} \textcolor{black!50}{0.00} 0.65 & 26 28 32 & 18 24 & 7 4 3\\
Psarras1997 \href{http://dx.doi.org/10.1016/s0377-2217(96)00114-2}{Psarras1997} & \hyperref[auth:a2040]{J. Psarras}, \hyperref[auth:a2041]{E. Stefanitsis}, \hyperref[auth:a2042]{N. Christodoulou} & Combination of local search and CLP in the vehicle-fleet scheduling problem & \hyperref[detail:Psarras1997]{Details} No & \cite{Psarras1997} & 1997 & European Journal of Operational Research & null & \noindent{}\textbf{1.00} \textbf{1.00} n/a & 1 1 3 & 11 23 & 1 0 1\\
BlazewiczDP96 \href{http://dx.doi.org/10.1016/0377-2217(95)00362-2}{BlazewiczDP96} & \hyperref[auth:a975]{J. Błażewicz}, \hyperref[auth:a976]{W. Domschke}, \hyperref[auth:a438]{E. Pesch} & The job shop scheduling problem: Conventional and new solution techniques & \hyperref[detail:BlazewiczDP96]{Details} \href{../works/BlazewiczDP96.pdf}{Yes} & \cite{BlazewiczDP96} & 1996 & European Journal of Operational Research & 33 & \noindent{}\textcolor{black!50}{0.00} \textcolor{black!50}{0.00} \textbf{14.88} & 344 357 412 & 127 224 & 33 20 13\\
NuijtenA96 \href{http://dx.doi.org/10.1016/0377-2217(95)00354-1}{NuijtenA96} & \hyperref[auth:a656]{W. Nuijten}, \hyperref[auth:a777]{E. H. L. Aarts} & A computational study of constraint satisfaction for multiple capacitated job shop scheduling & \hyperref[detail:NuijtenA96]{Details} \href{../works/NuijtenA96.pdf}{Yes} & \cite{NuijtenA96} & 1996 & European Journal of Operational Research & 16 & \noindent{}\textbf{2.00} \textbf{2.00} \textbf{2.88} & 65 65 90 & 6 21 & 27 23 4\\
\end{longtable}
}

\subsection{Expert Systems}

\index{Expert Systems}
{\scriptsize
\begin{longtable}{>{\raggedright\arraybackslash}p{2.5cm}>{\raggedright\arraybackslash}p{4.5cm}>{\raggedright\arraybackslash}p{6.0cm}p{1.0cm}rr>{\raggedright\arraybackslash}p{2.0cm}r>{\raggedright\arraybackslash}p{1cm}p{1cm}p{1cm}p{1cm}}
\rowcolor{white}\caption{Articles in Journal Expert Systems (Total 1)}\\ \toprule
\rowcolor{white}\shortstack{Key\\Source} & Authors & Title (Colored by Open Access)& \shortstack{Details\\LC} & Cite & Year & \shortstack{Conference\\/Journal\\/School} & Pages & Relevance &\shortstack{Cites\\OC XR\\SC} & \shortstack{Refs\\OC\\XR} & \shortstack{Links\\Cites\\Refs}\\ \midrule\endhead
\bottomrule
\endfoot
Ramos2021 \href{http://dx.doi.org/10.1111/exsy.12830}{Ramos2021} & \hyperref[auth:a1731]{A. S. Ramos}, \hyperref[auth:a1736]{E. Olivares‐Benitez}, \hyperref[auth:a1737]{P. A. Miranda‐Gonzalez} & Multi‐start iterated local search metaheuristic for the multi‐mode resource‐constrained project scheduling problem \hyperref[abs:Ramos2021]{Abstract} & \hyperref[detail:Ramos2021]{Details} No & \cite{Ramos2021} & 2021 & Expert Systems & null & \noindent{}\textcolor{black!50}{0.00} \textcolor{black!50}{0.00} n/a & 3 3 4 & 52 56 & 8 1 7\\
\end{longtable}
}

\subsection{Expert Systems with Applications}

\index{Expert Systems with Applications}
{\scriptsize
\begin{longtable}{>{\raggedright\arraybackslash}p{2.5cm}>{\raggedright\arraybackslash}p{4.5cm}>{\raggedright\arraybackslash}p{6.0cm}p{1.0cm}rr>{\raggedright\arraybackslash}p{2.0cm}r>{\raggedright\arraybackslash}p{1cm}p{1cm}p{1cm}p{1cm}}
\rowcolor{white}\caption{Articles in Journal Expert Systems with Applications (Total 4)}\\ \toprule
\rowcolor{white}\shortstack{Key\\Source} & Authors & Title (Colored by Open Access)& \shortstack{Details\\LC} & Cite & Year & \shortstack{Conference\\/Journal\\/School} & Pages & Relevance &\shortstack{Cites\\OC XR\\SC} & \shortstack{Refs\\OC\\XR} & \shortstack{Links\\Cites\\Refs}\\ \midrule\endhead
\bottomrule
\endfoot
CilKLO22 \href{http://dx.doi.org/10.1016/j.eswa.2022.117529}{CilKLO22} & \hyperref[auth:a1381]{Z. A. Cil}, \hyperref[auth:a1380]{D. Kizilay}, \hyperref[auth:a1382]{Z. Li}, \hyperref[auth:a1383]{H. \"{O}ztop} & Two-sided disassembly line balancing problem with sequence-dependent setup time: A constraint programming model and artificial bee colony algorithm & \hyperref[detail:CilKLO22]{Details} \href{../works/CilKLO22.pdf}{Yes} & \cite{CilKLO22} & 2022 & Expert Systems with Applications & 19 & \noindent{}\textcolor{black!50}{0.00} \textcolor{black!50}{0.00} \textbf{39.52} & 5 11 10 & 44 52 & 10 1 9\\
NovasH14 \href{https://doi.org/10.1016/j.eswa.2013.09.026}{NovasH14} & \hyperref[auth:a524]{J. M. Novas}, \hyperref[auth:a588]{G. P. Henning} & \cellcolor{green!10}Integrated scheduling of resource-constrained flexible manufacturing systems using constraint programming & \hyperref[detail:NovasH14]{Details} \href{../works/NovasH14.pdf}{Yes} & \cite{NovasH14} & 2014 & Expert Systems with Applications & 14 & \noindent{}\textbf{1.50} \textbf{1.50} \textbf{23.96} & 35 37 44 & 26 30 & 18 6 12\\
Filho2012 \href{http://dx.doi.org/10.1016/j.eswa.2011.07.027}{Filho2012} & \hyperref[auth:a1949]{Cicero Ferreira Fernandes Costa Filho}, \hyperref[auth:a1950]{D. A. R. Rocha}, \hyperref[auth:a1951]{M. G. F. Costa}, \hyperref[auth:a1952]{Wagner Coelho de Albuquerque Pereira} & Using Constraint Satisfaction Problem approach to solve human resource allocation problems in cooperative health services & \hyperref[detail:Filho2012]{Details} No & \cite{Filho2012} & 2012 & Expert Systems with Applications & null & \noindent{}0.50 0.50 n/a & 17 19 24 & 6 10 & 1 0 1\\
TopalogluSS12 \href{http://dx.doi.org/10.1016/j.eswa.2011.09.038}{TopalogluSS12} & \hyperref[auth:a617]{S. Topaloglu}, \hyperref[auth:a1378]{L. Salum}, \hyperref[auth:a1379]{A. A. Supciller} & Rule-based modeling and constraint programming based solution of the assembly line balancing problem & \hyperref[detail:TopalogluSS12]{Details} \href{../works/TopalogluSS12.pdf}{Yes} & \cite{TopalogluSS12} & 2012 & Expert Systems with Applications & 10 & \noindent{}\textcolor{black!50}{0.00} \textcolor{black!50}{0.00} \textbf{6.33} & 31 32 38 & 34 43 & 17 15 2\\
\end{longtable}
}

\subsection{Foundations of Management}

\index{Foundations of Management}
{\scriptsize
\begin{longtable}{>{\raggedright\arraybackslash}p{2.5cm}>{\raggedright\arraybackslash}p{4.5cm}>{\raggedright\arraybackslash}p{6.0cm}p{1.0cm}rr>{\raggedright\arraybackslash}p{2.0cm}r>{\raggedright\arraybackslash}p{1cm}p{1cm}p{1cm}p{1cm}}
\rowcolor{white}\caption{Articles in Journal Foundations of Management (Total 1)}\\ \toprule
\rowcolor{white}\shortstack{Key\\Source} & Authors & Title (Colored by Open Access)& \shortstack{Details\\LC} & Cite & Year & \shortstack{Conference\\/Journal\\/School} & Pages & Relevance &\shortstack{Cites\\OC XR\\SC} & \shortstack{Refs\\OC\\XR} & \shortstack{Links\\Cites\\Refs}\\ \midrule\endhead
\bottomrule
\endfoot
Banaszak2014 \href{http://dx.doi.org/10.1515/fman-2015-0014}{Banaszak2014} & \hyperref[auth:a1814]{Z. Banaszak}, \hyperref[auth:a630]{G. Bocewicz} & Declarative Modeling for Production Order Portfolio Scheduling \hyperref[abs:Banaszak2014]{Abstract} & \hyperref[detail:Banaszak2014]{Details} No & \cite{Banaszak2014} & 2014 & Foundations of Management & null & \noindent{}\textcolor{black!50}{0.00} \textbf{2.50} n/a & 8 8 0 & 0 0 & 1 1 0\\
\end{longtable}
}

\subsection{Frontiers of Mechanical Engineering in China}

\index{Frontiers of Mechanical Engineering in China}
{\scriptsize
\begin{longtable}{>{\raggedright\arraybackslash}p{2.5cm}>{\raggedright\arraybackslash}p{4.5cm}>{\raggedright\arraybackslash}p{6.0cm}p{1.0cm}rr>{\raggedright\arraybackslash}p{2.0cm}r>{\raggedright\arraybackslash}p{1cm}p{1cm}p{1cm}p{1cm}}
\rowcolor{white}\caption{Articles in Journal Frontiers of Mechanical Engineering in China (Total 1)}\\ \toprule
\rowcolor{white}\shortstack{Key\\Source} & Authors & Title (Colored by Open Access)& \shortstack{Details\\LC} & Cite & Year & \shortstack{Conference\\/Journal\\/School} & Pages & Relevance &\shortstack{Cites\\OC XR\\SC} & \shortstack{Refs\\OC\\XR} & \shortstack{Links\\Cites\\Refs}\\ \midrule\endhead
\bottomrule
\endfoot
ChenGPSH10 \href{http://dx.doi.org/10.1007/s11465-010-0106-x}{ChenGPSH10} & \hyperref[auth:a913]{Y. Chen}, \hyperref[auth:a914]{Z. Guan}, \hyperref[auth:a915]{Y. Peng}, \hyperref[auth:a916]{X. Shao}, \hyperref[auth:a917]{M. Hasseb} & Technology and system of constraint programming for industry production scheduling — Part I: A brief survey and potential directions & \hyperref[detail:ChenGPSH10]{Details} \href{../works/ChenGPSH10.pdf}{Yes} & \cite{ChenGPSH10} & 2010 & Frontiers of Mechanical Engineering in China & 10 & \noindent{}\textbf{1.00} \textbf{1.00} \textbf{20.87} & 2 2 4 & 32 50 & 18 2 16\\
\end{longtable}
}

\subsection{Gazi University Journal of Science}

\index{Gazi University Journal of Science}
{\scriptsize
\begin{longtable}{>{\raggedright\arraybackslash}p{2.5cm}>{\raggedright\arraybackslash}p{4.5cm}>{\raggedright\arraybackslash}p{6.0cm}p{1.0cm}rr>{\raggedright\arraybackslash}p{2.0cm}r>{\raggedright\arraybackslash}p{1cm}p{1cm}p{1cm}p{1cm}}
\rowcolor{white}\caption{Articles in Journal Gazi University Journal of Science (Total 1)}\\ \toprule
\rowcolor{white}\shortstack{Key\\Source} & Authors & Title (Colored by Open Access)& \shortstack{Details\\LC} & Cite & Year & \shortstack{Conference\\/Journal\\/School} & Pages & Relevance &\shortstack{Cites\\OC XR\\SC} & \shortstack{Refs\\OC\\XR} & \shortstack{Links\\Cites\\Refs}\\ \midrule\endhead
\bottomrule
\endfoot
Gokgur2022 \href{http://dx.doi.org/10.35378/gujs.681151}{Gokgur2022} & \hyperref[auth:a1612]{B. Gokgur}, \hyperref[auth:a1613]{S. Özpeyni̇rci̇} & \cellcolor{gold!20}Minimization of Number of Tool Switching Instants in Automated Manufacturing Systems \hyperref[abs:Gokgur2022]{Abstract} & \hyperref[detail:Gokgur2022]{Details} No & \cite{Gokgur2022} & 2022 & Gazi University Journal of Science & null & \noindent{}\textcolor{black!50}{0.00} \textbf{1.50} n/a & 0 0 0 & 29 30 & 6 0 6\\
\end{longtable}
}

\subsection{Human Factors and Ergonomics in Manufacturing \& Service Industries}

\index{Human Factors and Ergonomics in Manufacturing \& Service Industries}
{\scriptsize
\begin{longtable}{>{\raggedright\arraybackslash}p{2.5cm}>{\raggedright\arraybackslash}p{4.5cm}>{\raggedright\arraybackslash}p{6.0cm}p{1.0cm}rr>{\raggedright\arraybackslash}p{2.0cm}r>{\raggedright\arraybackslash}p{1cm}p{1cm}p{1cm}p{1cm}}
\rowcolor{white}\caption{Articles in Journal Human Factors and Ergonomics in Manufacturing \  Service Industries (Total 1)}\\ \toprule
\rowcolor{white}\shortstack{Key\\Source} & Authors & Title (Colored by Open Access)& \shortstack{Details\\LC} & Cite & Year & \shortstack{Conference\\/Journal\\/School} & Pages & Relevance &\shortstack{Cites\\OC XR\\SC} & \shortstack{Refs\\OC\\XR} & \shortstack{Links\\Cites\\Refs}\\ \midrule\endhead
\bottomrule
\endfoot
Hoc2012 \href{http://dx.doi.org/10.1002/hfm.20359}{Hoc2012} & \hyperref[auth:a2009]{J. Hoc}, \hyperref[auth:a2010]{C. Guerin}, \hyperref[auth:a2011]{N. Mebarki} & The Nature of Expertise in Scheduling: The Case of Timetabling \hyperref[abs:Hoc2012]{Abstract} & \hyperref[detail:Hoc2012]{Details} No & \cite{Hoc2012} & 2012 & Human Factors and Ergonomics in Manufacturing \  Service Industries & null & \noindent{}\textcolor{black!50}{0.00} \textbf{1.50} n/a & 7 7 8 & 29 45 & 1 0 1\\
\end{longtable}
}

\subsection{IEEE Access}

\index{IEEE Access}
{\scriptsize
\begin{longtable}{>{\raggedright\arraybackslash}p{2.5cm}>{\raggedright\arraybackslash}p{4.5cm}>{\raggedright\arraybackslash}p{6.0cm}p{1.0cm}rr>{\raggedright\arraybackslash}p{2.0cm}r>{\raggedright\arraybackslash}p{1cm}p{1cm}p{1cm}p{1cm}}
\rowcolor{white}\caption{Articles in Journal IEEE Access (Total 1)}\\ \toprule
\rowcolor{white}\shortstack{Key\\Source} & Authors & Title (Colored by Open Access)& \shortstack{Details\\LC} & Cite & Year & \shortstack{Conference\\/Journal\\/School} & Pages & Relevance &\shortstack{Cites\\OC XR\\SC} & \shortstack{Refs\\OC\\XR} & \shortstack{Links\\Cites\\Refs}\\ \midrule\endhead
\bottomrule
\endfoot
Li2018 \href{http://dx.doi.org/10.1109/access.2018.2859618}{Li2018} & \hyperref[auth:a1796]{H. Li}, \hyperref[auth:a1801]{Z. Li} & \cellcolor{gold!20}A Novel Strategy of Combining Variable Ordering Heuristics for Constraint Satisfaction Problems & \hyperref[detail:Li2018]{Details} No & \cite{Li2018} & 2018 & IEEE Access & null & \noindent{}0.50 0.50 n/a & 4 4 4 & 17 35 & 5 0 5\\
\end{longtable}
}

\subsection{IEEE Engineering in Medicine and Biology Magazine}

\index{IEEE Engineering in Medicine and Biology Magazine}
{\scriptsize
\begin{longtable}{>{\raggedright\arraybackslash}p{2.5cm}>{\raggedright\arraybackslash}p{4.5cm}>{\raggedright\arraybackslash}p{6.0cm}p{1.0cm}rr>{\raggedright\arraybackslash}p{2.0cm}r>{\raggedright\arraybackslash}p{1cm}p{1cm}p{1cm}p{1cm}}
\rowcolor{white}\caption{Articles in Journal IEEE Engineering in Medicine and Biology Magazine (Total 1)}\\ \toprule
\rowcolor{white}\shortstack{Key\\Source} & Authors & Title (Colored by Open Access)& \shortstack{Details\\LC} & Cite & Year & \shortstack{Conference\\/Journal\\/School} & Pages & Relevance &\shortstack{Cites\\OC XR\\SC} & \shortstack{Refs\\OC\\XR} & \shortstack{Links\\Cites\\Refs}\\ \midrule\endhead
\bottomrule
\endfoot
WeilHFP95 \href{http://dx.doi.org/10.1109/51.395324}{WeilHFP95} & \hyperref[auth:a1191]{G. Weil}, \hyperref[auth:a1192]{K. Heus}, \hyperref[auth:a1193]{P. Francois}, \hyperref[auth:a1194]{M. Poujade} & Constraint programming for nurse scheduling & \hyperref[detail:WeilHFP95]{Details} \href{../works/WeilHFP95.pdf}{Yes} & \cite{WeilHFP95} & 1995 & IEEE Engineering in Medicine and Biology Magazine & 6 & \noindent{}\textbf{1.00} \textbf{1.00} 0.68 & 56 56 68 & 9 21 & 10 9 1\\
\end{longtable}
}

\subsection{IEEE LATIN AMERICA TRANSACTIONS}

\index{IEEE LATIN AMERICA TRANSACTIONS}
{\scriptsize
\begin{longtable}{>{\raggedright\arraybackslash}p{2.5cm}>{\raggedright\arraybackslash}p{4.5cm}>{\raggedright\arraybackslash}p{6.0cm}p{1.0cm}rr>{\raggedright\arraybackslash}p{2.0cm}r>{\raggedright\arraybackslash}p{1cm}p{1cm}p{1cm}p{1cm}}
\rowcolor{white}\caption{Articles in Journal IEEE LATIN AMERICA TRANSACTIONS (Total 1)}\\ \toprule
\rowcolor{white}\shortstack{Key\\Source} & Authors & Title (Colored by Open Access)& \shortstack{Details\\LC} & Cite & Year & \shortstack{Conference\\/Journal\\/School} & Pages & Relevance &\shortstack{Cites\\OC XR\\SC} & \shortstack{Refs\\OC\\XR} & \shortstack{Links\\Cites\\Refs}\\ \midrule\endhead
\bottomrule
\endfoot
KonowalenkoMM19 \href{http://dx.doi.org/10.1109/tla.2019.8932340}{KonowalenkoMM19} & \hyperref[auth:a1466]{E. Konowalenko}, \hyperref[auth:a1467]{W. Meira}, \hyperref[auth:a1468]{L. Magatao} & Constraint Logic Programming Applied to Sequencing Tasks in a Pipeline Network \hyperref[abs:KonowalenkoMM19]{Abstract} & \hyperref[detail:KonowalenkoMM19]{Details} \href{../works/KonowalenkoMM19.pdf}{Yes} & \cite{KonowalenkoMM19} & 2019 & IEEE LATIN AMERICA TRANSACTIONS & 9 & \noindent{}\textbf{1.00} \textbf{4.00} 0.57 & 0 0 0 & 0 0 & 0 0 0\\
\end{longtable}
}

\subsection{IEEE Robotics and Automation Letters}

\index{IEEE Robotics and Automation Letters}
{\scriptsize
\begin{longtable}{>{\raggedright\arraybackslash}p{2.5cm}>{\raggedright\arraybackslash}p{4.5cm}>{\raggedright\arraybackslash}p{6.0cm}p{1.0cm}rr>{\raggedright\arraybackslash}p{2.0cm}r>{\raggedright\arraybackslash}p{1cm}p{1cm}p{1cm}p{1cm}}
\rowcolor{white}\caption{Articles in Journal IEEE Robotics and Automation Letters (Total 2)}\\ \toprule
\rowcolor{white}\shortstack{Key\\Source} & Authors & Title (Colored by Open Access)& \shortstack{Details\\LC} & Cite & Year & \shortstack{Conference\\/Journal\\/School} & Pages & Relevance &\shortstack{Cites\\OC XR\\SC} & \shortstack{Refs\\OC\\XR} & \shortstack{Links\\Cites\\Refs}\\ \midrule\endhead
\bottomrule
\endfoot
HamP21 \href{http://dx.doi.org/10.1109/lra.2021.3056069}{HamP21} & \hyperref[auth:a750]{A. Ham}, \hyperref[auth:a751]{M.-J. Park} & Human–Robot Task Allocation and Scheduling: Boeing 777 Case Study & \hyperref[detail:HamP21]{Details} \href{../works/HamP21.pdf}{Yes} & \cite{HamP21} & 2021 & IEEE Robotics and Automation Letters & 8 & \noindent{}\textcolor{black!50}{0.00} \textcolor{black!50}{0.00} \textbf{6.63} & 13 16 17 & 26 30 & 12 2 10\\
BoothTNB16 \href{http://dx.doi.org/10.1109/lra.2016.2522096}{BoothTNB16} & \hyperref[auth:a203]{K. E. C. Booth}, \hyperref[auth:a799]{T. T. Tran}, \hyperref[auth:a204]{G. Nejat}, \hyperref[auth:a89]{J. C. Beck} & \cellcolor{green!10}Mixed-Integer and Constraint Programming Techniques for Mobile Robot Task Planning & \hyperref[detail:BoothTNB16]{Details} \href{../works/BoothTNB16.pdf}{Yes} & \cite{BoothTNB16} & 2016 & IEEE Robotics and Automation Letters & 8 & \noindent{}\textbf{1.00} \textbf{1.00} \textbf{11.60} & 27 28 35 & 21 34 & 14 5 9\\
\end{longtable}
}

\subsection{IEEE Transactions on Automation Science and Engineering}

\index{IEEE Transactions on Automation Science and Engineering}
{\scriptsize
\begin{longtable}{>{\raggedright\arraybackslash}p{2.5cm}>{\raggedright\arraybackslash}p{4.5cm}>{\raggedright\arraybackslash}p{6.0cm}p{1.0cm}rr>{\raggedright\arraybackslash}p{2.0cm}r>{\raggedright\arraybackslash}p{1cm}p{1cm}p{1cm}p{1cm}}
\rowcolor{white}\caption{Articles in Journal IEEE Transactions on Automation Science and Engineering (Total 2)}\\ \toprule
\rowcolor{white}\shortstack{Key\\Source} & Authors & Title (Colored by Open Access)& \shortstack{Details\\LC} & Cite & Year & \shortstack{Conference\\/Journal\\/School} & Pages & Relevance &\shortstack{Cites\\OC XR\\SC} & \shortstack{Refs\\OC\\XR} & \shortstack{Links\\Cites\\Refs}\\ \midrule\endhead
\bottomrule
\endfoot
Ham20a \href{http://dx.doi.org/10.1109/tase.2019.2952523}{Ham20a} & \hyperref[auth:a750]{A. Ham} & Drone-Based Material Transfer System in a Robotic Mobile Fulfillment Center & \hyperref[detail:Ham20a]{Details} \href{../works/Ham20a.pdf}{Yes} & \cite{Ham20a} & 2020 & IEEE Transactions on Automation Science and Engineering & 9 & \noindent{}\textcolor{black!50}{0.00} \textcolor{black!50}{0.00} \textbf{9.90} & 15 19 19 & 27 41 & 8 1 7\\
TanZWGQ19 \href{http://dx.doi.org/10.1109/tase.2019.2894093}{TanZWGQ19} & \hyperref[auth:a1183]{Y. Tan}, \hyperref[auth:a1184]{M. Zhou}, \hyperref[auth:a1185]{Y. Wang}, \hyperref[auth:a1186]{X. Guo}, \hyperref[auth:a1187]{L. Qi} & A Hybrid MIP–CP Approach to Multistage Scheduling Problem in Continuous Casting and Hot-Rolling Processes & \hyperref[detail:TanZWGQ19]{Details} \href{../works/TanZWGQ19.pdf}{Yes} & \cite{TanZWGQ19} & 2019 & IEEE Transactions on Automation Science and Engineering & 10 & \noindent{}\textbf{1.00} \textbf{1.00} \textbf{7.38} & 40 45 43 & 40 44 & 7 0 7\\
\end{longtable}
}

\subsection{IEEE Transactions on Computers}

\index{IEEE Transactions on Computers}
{\scriptsize
\begin{longtable}{>{\raggedright\arraybackslash}p{2.5cm}>{\raggedright\arraybackslash}p{4.5cm}>{\raggedright\arraybackslash}p{6.0cm}p{1.0cm}rr>{\raggedright\arraybackslash}p{2.0cm}r>{\raggedright\arraybackslash}p{1cm}p{1cm}p{1cm}p{1cm}}
\rowcolor{white}\caption{Articles in Journal IEEE Transactions on Computers (Total 1)}\\ \toprule
\rowcolor{white}\shortstack{Key\\Source} & Authors & Title (Colored by Open Access)& \shortstack{Details\\LC} & Cite & Year & \shortstack{Conference\\/Journal\\/School} & Pages & Relevance &\shortstack{Cites\\OC XR\\SC} & \shortstack{Refs\\OC\\XR} & \shortstack{Links\\Cites\\Refs}\\ \midrule\endhead
\bottomrule
\endfoot
LombardiMB13 \href{http://dx.doi.org/10.1109/tc.2011.203}{LombardiMB13} & \hyperref[auth:a142]{M. Lombardi}, \hyperref[auth:a143]{M. Milano}, \hyperref[auth:a245]{L. Benini} & Robust Scheduling of Task Graphs under Execution Time Uncertainty & \hyperref[detail:LombardiMB13]{Details} \href{../works/LombardiMB13.pdf}{Yes} & \cite{LombardiMB13} & 2013 & IEEE Transactions on Computers & 14 & \noindent{}\textcolor{black!50}{0.00} \textcolor{black!50}{0.00} \textbf{10.63} & 28 28 36 & 29 44 & 10 2 8\\
\end{longtable}
}

\subsection{IEEE Transactions on Neural Networks}

\index{IEEE Transactions on Neural Networks}
{\scriptsize
\begin{longtable}{>{\raggedright\arraybackslash}p{2.5cm}>{\raggedright\arraybackslash}p{4.5cm}>{\raggedright\arraybackslash}p{6.0cm}p{1.0cm}rr>{\raggedright\arraybackslash}p{2.0cm}r>{\raggedright\arraybackslash}p{1cm}p{1cm}p{1cm}p{1cm}}
\rowcolor{white}\caption{Articles in Journal IEEE Transactions on Neural Networks (Total 1)}\\ \toprule
\rowcolor{white}\shortstack{Key\\Source} & Authors & Title (Colored by Open Access)& \shortstack{Details\\LC} & Cite & Year & \shortstack{Conference\\/Journal\\/School} & Pages & Relevance &\shortstack{Cites\\OC XR\\SC} & \shortstack{Refs\\OC\\XR} & \shortstack{Links\\Cites\\Refs}\\ \midrule\endhead
\bottomrule
\endfoot
Yang2000 \href{http://dx.doi.org/10.1109/72.839016}{Yang2000} & \hyperref[auth:a1912]{S. Yang}, \hyperref[auth:a1824]{D. Wang} & \cellcolor{green!10}Constraint satisfaction adaptive neural network and heuristics combined approaches for generalized job-shop scheduling & \hyperref[detail:Yang2000]{Details} No & \cite{Yang2000} & 2000 & IEEE Transactions on Neural Networks & null & \noindent{}\textbf{2.00} \textbf{2.00} n/a & 37 0 48 & 10 0 & 4 4 0\\
\end{longtable}
}

\subsection{IEEE Transactions on Robotics}

\index{IEEE Transactions on Robotics}
{\scriptsize
\begin{longtable}{>{\raggedright\arraybackslash}p{2.5cm}>{\raggedright\arraybackslash}p{4.5cm}>{\raggedright\arraybackslash}p{6.0cm}p{1.0cm}rr>{\raggedright\arraybackslash}p{2.0cm}r>{\raggedright\arraybackslash}p{1cm}p{1cm}p{1cm}p{1cm}}
\rowcolor{white}\caption{Articles in Journal IEEE Transactions on Robotics (Total 1)}\\ \toprule
\rowcolor{white}\shortstack{Key\\Source} & Authors & Title (Colored by Open Access)& \shortstack{Details\\LC} & Cite & Year & \shortstack{Conference\\/Journal\\/School} & Pages & Relevance &\shortstack{Cites\\OC XR\\SC} & \shortstack{Refs\\OC\\XR} & \shortstack{Links\\Cites\\Refs}\\ \midrule\endhead
\bottomrule
\endfoot
GombolayWS18 \href{http://dx.doi.org/10.1109/tro.2018.2795034}{GombolayWS18} & \hyperref[auth:a921]{M. C. Gombolay}, \hyperref[auth:a922]{R. J. Wilcox}, \hyperref[auth:a923]{J. A. Shah} & \cellcolor{gold!20}Fast Scheduling of Robot Teams Performing Tasks With Temporospatial Constraints & \hyperref[detail:GombolayWS18]{Details} \href{../works/GombolayWS18.pdf}{Yes} & \cite{GombolayWS18} & 2018 & IEEE Transactions on Robotics & 20 & \noindent{}\textcolor{black!50}{0.00} \textcolor{black!50}{0.00} \textbf{11.87} & 71 80 79 & 75 89 & 12 1 11\\
\end{longtable}
}

\subsection{IEEE Transactions on Semiconductor Manufacturing}

\index{IEEE Transactions on Semiconductor Manufacturing}
{\scriptsize
\begin{longtable}{>{\raggedright\arraybackslash}p{2.5cm}>{\raggedright\arraybackslash}p{4.5cm}>{\raggedright\arraybackslash}p{6.0cm}p{1.0cm}rr>{\raggedright\arraybackslash}p{2.0cm}r>{\raggedright\arraybackslash}p{1cm}p{1cm}p{1cm}p{1cm}}
\rowcolor{white}\caption{Articles in Journal IEEE Transactions on Semiconductor Manufacturing (Total 2)}\\ \toprule
\rowcolor{white}\shortstack{Key\\Source} & Authors & Title (Colored by Open Access)& \shortstack{Details\\LC} & Cite & Year & \shortstack{Conference\\/Journal\\/School} & Pages & Relevance &\shortstack{Cites\\OC XR\\SC} & \shortstack{Refs\\OC\\XR} & \shortstack{Links\\Cites\\Refs}\\ \midrule\endhead
\bottomrule
\endfoot
Ham18a \href{http://dx.doi.org/10.1109/tsm.2017.2768899}{Ham18a} & \hyperref[auth:a750]{A. Ham} & Scheduling of Dual Resource Constrained Lithography Production: Using CP and MIP/CP & \hyperref[detail:Ham18a]{Details} \href{../works/Ham18a.pdf}{Yes} & \cite{Ham18a} & 2018 & IEEE Transactions on Semiconductor Manufacturing & 10 & \noindent{}\textbf{1.50} \textbf{1.50} \textbf{17.72} & 20 24 28 & 21 28 & 16 5 11\\
HamFC17 \href{http://dx.doi.org/10.1109/tsm.2017.2740340}{HamFC17} & \hyperref[auth:a750]{A. Ham}, \hyperref[auth:a1201]{J. W. Fowler}, \hyperref[auth:a875]{E. Cakici} & Constraint Programming Approach for Scheduling Jobs With Release Times, Non-Identical Sizes, and Incompatible Families on Parallel Batching Machines & \hyperref[detail:HamFC17]{Details} \href{../works/HamFC17.pdf}{Yes} & \cite{HamFC17} & 2017 & IEEE Transactions on Semiconductor Manufacturing & 8 & \noindent{}\textbf{2.50} \textbf{2.50} \textbf{15.73} & 21 24 25 & 28 33 & 8 3 5\\
\end{longtable}
}

\subsection{IET Collaborative Intelligent Manufacturing}

\index{IET Collaborative Intelligent Manufacturing}
{\scriptsize
\begin{longtable}{>{\raggedright\arraybackslash}p{2.5cm}>{\raggedright\arraybackslash}p{4.5cm}>{\raggedright\arraybackslash}p{6.0cm}p{1.0cm}rr>{\raggedright\arraybackslash}p{2.0cm}r>{\raggedright\arraybackslash}p{1cm}p{1cm}p{1cm}p{1cm}}
\rowcolor{white}\caption{Articles in Journal IET Collaborative Intelligent Manufacturing (Total 1)}\\ \toprule
\rowcolor{white}\shortstack{Key\\Source} & Authors & Title (Colored by Open Access)& \shortstack{Details\\LC} & Cite & Year & \shortstack{Conference\\/Journal\\/School} & Pages & Relevance &\shortstack{Cites\\OC XR\\SC} & \shortstack{Refs\\OC\\XR} & \shortstack{Links\\Cites\\Refs}\\ \midrule\endhead
\bottomrule
\endfoot
MengLZB21 \href{http://dx.doi.org/10.1049/cim2.12005}{MengLZB21} & \hyperref[auth:a500]{L. Meng}, \hyperref[auth:a1157]{C. Lu}, \hyperref[auth:a503]{B. Zhang}, \hyperref[auth:a502]{Y. Ren}, \hyperref[auth:a504]{C. Lv}, \hyperref[auth:a1158]{H. Sang}, \hyperref[auth:a1159]{J. Li}, \hyperref[auth:a501]{C. Zhang} & \cellcolor{gold!20}Constraint programing for solving four complex flexible shop scheduling problems & \hyperref[detail:MengLZB21]{Details} \href{../works/MengLZB21.pdf}{Yes} & \cite{MengLZB21} & 2021 & IET Collaborative Intelligent Manufacturing & 14 & \noindent{}\textcolor{black!50}{0.00} \textcolor{black!50}{0.00} \textbf{28.91} & 5 8 8 & 39 44 & 12 4 8\\
\end{longtable}
}

\subsection{IFAC Proceedings Volumes}

\index{IFAC Proceedings Volumes}
{\scriptsize
\begin{longtable}{>{\raggedright\arraybackslash}p{2.5cm}>{\raggedright\arraybackslash}p{4.5cm}>{\raggedright\arraybackslash}p{6.0cm}p{1.0cm}rr>{\raggedright\arraybackslash}p{2.0cm}r>{\raggedright\arraybackslash}p{1cm}p{1cm}p{1cm}p{1cm}}
\rowcolor{white}\caption{Articles in Journal IFAC Proceedings Volumes (Total 6)}\\ \toprule
\rowcolor{white}\shortstack{Key\\Source} & Authors & Title (Colored by Open Access)& \shortstack{Details\\LC} & Cite & Year & \shortstack{Conference\\/Journal\\/School} & Pages & Relevance &\shortstack{Cites\\OC XR\\SC} & \shortstack{Refs\\OC\\XR} & \shortstack{Links\\Cites\\Refs}\\ \midrule\endhead
\bottomrule
\endfoot
Pessoa2013 \href{http://dx.doi.org/10.3182/20130522-3-br-4036.00069}{Pessoa2013} & \hyperref[auth:a1669]{M. A. O. Pessoa}, \hyperref[auth:a1670]{R. A. E. Montesco}, \hyperref[auth:a1671]{F. Junqueira}, \hyperref[auth:a1672]{Diolino Jose dos Santos Filho}, \hyperref[auth:a1673]{P. E. Miyagi} & \cellcolor{gold!20}Advanced Planning and Scheduling Systems based on Time Windows and Constraint Programming & \hyperref[detail:Pessoa2013]{Details} No & \cite{Pessoa2013} & 2013 & IFAC Proceedings Volumes & null & \noindent{}\textbf{1.00} \textbf{1.00} n/a & 4 4 4 & 13 21 & 2 1 1\\
OzturkTHO12 \href{https://www.sciencedirect.com/science/article/pii/S1474667016331858}{OzturkTHO12} & \hyperref[auth:a1015]{C. {\"{O}}zt{\"{u}}rk}, \hyperref[auth:a1016]{S. Tunalı}, \hyperref[auth:a137]{B. Hnich}, \hyperref[auth:a138]{A. {\"{O}}rnek} & A Constraint Programming Model for Balancing and Scheduling of Flexible Mixed Model Assembly Lines with Parallel Stations & \hyperref[detail:OzturkTHO12]{Details} \href{../works/OzturkTHO12.pdf}{Yes} & \cite{OzturkTHO12} & 2012 & IFAC Proceedings Volumes & 6 & \noindent{}\textbf{1.00} \textbf{1.00} \textbf{7.24} & 5 4 5 & 5 10 & 7 5 2\\
Bidot2006 \href{http://dx.doi.org/10.3182/20060517-3-fr-2903.00313}{Bidot2006} & \hyperref[auth:a824]{J. Bidot}, \hyperref[auth:a118]{P. Laborie}, \hyperref[auth:a89]{J. C. Beck}, \hyperref[auth:a825]{T. Vidal} & \cellcolor{gold!20}Using constraint programming and simulation for execution monitoring and progressive scheduling & \hyperref[detail:Bidot2006]{Details} No & \cite{Bidot2006} & 2006 & IFAC Proceedings Volumes & null & \noindent{}\textbf{1.00} \textbf{1.00} n/a & 2 2 4 & 2 6 & 1 1 0\\
Elkhyari2006 \href{http://dx.doi.org/10.3182/20060517-3-fr-2903.00358}{Elkhyari2006} & \hyperref[auth:a292]{A. Elkhyari}, \hyperref[auth:a2069]{C. Combes} & Using constraint programming for solving dynamic scheduling in the endoscopy unit & \hyperref[detail:Elkhyari2006]{Details} No & \cite{Elkhyari2006} & 2006 & IFAC Proceedings Volumes & null & \noindent{}\textbf{1.00} \textbf{1.00} n/a & 1 1 1 & 1 8 & 1 1 0\\
Trilling2006 \href{http://dx.doi.org/10.3182/20060517-3-fr-2903.00340}{Trilling2006} & \hyperref[auth:a1656]{L. Trilling}, \hyperref[auth:a1657]{A. Guinet}, \hyperref[auth:a1658]{D. L. Magny} & Nurse scheduling using integer linear programming and constraint programming & \hyperref[detail:Trilling2006]{Details} No & \cite{Trilling2006} & 2006 & IFAC Proceedings Volumes & null & \noindent{}\textbf{1.00} \textbf{1.00} n/a & 25 25 0 & 11 13 & 4 3 1\\
Kovcs2003 \href{http://dx.doi.org/10.1016/s1474-6670(17)37762-5}{Kovcs2003} & \hyperref[auth:a1880]{A. Kovács}, \hyperref[auth:a1881]{J. Váncza}, \hyperref[auth:a1882]{B. Kádár}, \hyperref[auth:a1883]{L. Monostori}, \hyperref[auth:a1884]{A. Pfeiffer} & Real-Life Scheduling Using Constraint Programming and Simulation & \hyperref[detail:Kovcs2003]{Details} No & \cite{Kovcs2003} & 2003 & IFAC Proceedings Volumes & null & \noindent{}\textbf{1.00} \textbf{1.00} n/a & 2 2 6 & 8 13 & 2 0 2\\
\end{longtable}
}

\subsection{INFORMS Journal on Computing}

\index{INFORMS Journal on Computing}
{\scriptsize
\begin{longtable}{>{\raggedright\arraybackslash}p{2.5cm}>{\raggedright\arraybackslash}p{4.5cm}>{\raggedright\arraybackslash}p{6.0cm}p{1.0cm}rr>{\raggedright\arraybackslash}p{2.0cm}r>{\raggedright\arraybackslash}p{1cm}p{1cm}p{1cm}p{1cm}}
\rowcolor{white}\caption{Articles in Journal INFORMS Journal on Computing (Total 31)}\\ \toprule
\rowcolor{white}\shortstack{Key\\Source} & Authors & Title (Colored by Open Access)& \shortstack{Details\\LC} & Cite & Year & \shortstack{Conference\\/Journal\\/School} & Pages & Relevance &\shortstack{Cites\\OC XR\\SC} & \shortstack{Refs\\OC\\XR} & \shortstack{Links\\Cites\\Refs}\\ \midrule\endhead
\bottomrule
\endfoot
FahimiQ23 \href{http://dx.doi.org/10.1287/ijoc.2021.0138}{FahimiQ23} & \hyperref[auth:a122]{H. Fahimi}, \hyperref[auth:a37]{C.-G. Quimper} & Overload-Checking and Edge-Finding for Robust Cumulative Scheduling & \hyperref[detail:FahimiQ23]{Details} No & \cite{FahimiQ23} & 2023 & \cellcolor{red!20}INFORMS Journal on Computing & 20 & \noindent{}\textcolor{black!50}{0.00} \textcolor{black!50}{0.00} n/a & 0 0 0 & 16 21 & 8 0 8\\
NaderiRR23 \href{https://doi.org/10.1287/ijoc.2023.1287}{NaderiRR23} & \hyperref[auth:a726]{B. Naderi}, \hyperref[auth:a727]{R. Ruiz}, \hyperref[auth:a728]{V. Roshanaei} & Mixed-Integer Programming vs. Constraint Programming for Shop Scheduling Problems: New Results and Outlook & \hyperref[detail:NaderiRR23]{Details} \href{../works/NaderiRR23.pdf}{Yes} & \cite{NaderiRR23} & 2023 & \cellcolor{red!20}INFORMS Journal on Computing & 27 & \noindent{}\textbf{1.00} \textbf{1.00} \textbf{184.97} & 2 7 7 & 50 55 & 22 1 21\\
ElciOH22 \href{http://dx.doi.org/10.1287/ijoc.2022.1184}{ElciOH22} & \hyperref[auth:a930]{\"{O}zg\"{u}n El\c{c}i}, \hyperref[auth:a160]{J. N. Hooker} & \cellcolor{green!10}Stochastic Planning and Scheduling with Logic-Based Benders Decomposition & \hyperref[detail:ElciOH22]{Details} \href{../works/ElciOH22.pdf}{Yes} & \cite{ElciOH22} & 2022 & \cellcolor{red!20}INFORMS Journal on Computing & 15 & \noindent{}\textcolor{black!50}{0.00} \textcolor{black!50}{0.00} \textbf{3.20} & 2 4 6 & 34 36 & 16 2 14\\
HillBCGN22 \href{http://dx.doi.org/10.1287/ijoc.2022.1222}{HillBCGN22} & \hyperref[auth:a64]{A. Hill}, \hyperref[auth:a971]{A. J. Brickey}, \hyperref[auth:a972]{I. Cipriano}, \hyperref[auth:a973]{M. Goycoolea}, \hyperref[auth:a974]{A. Newman} & Optimization Strategies for Resource-Constrained Project Scheduling Problems in Underground Mining & \hyperref[detail:HillBCGN22]{Details} No & \cite{HillBCGN22} & 2022 & \cellcolor{red!20}INFORMS Journal on Computing & 17 & \noindent{}\textcolor{black!50}{0.00} \textcolor{black!50}{0.00} n/a & 0 2 2 & 53 58 & 10 0 10\\
MartnezAJ22 \href{http://dx.doi.org/10.1287/ijoc.2021.1079}{MartnezAJ22} & \hyperref[auth:a935]{K. P. Martínez}, \hyperref[auth:a936]{Y. Adulyasak}, \hyperref[auth:a841]{R. Jans} & Logic-Based Benders Decomposition for Integrated Process Configuration and Production Planning Problems & \hyperref[detail:MartnezAJ22]{Details} No & \cite{MartnezAJ22} & 2022 & \cellcolor{red!20}INFORMS Journal on Computing & 15 & \noindent{}\textcolor{black!50}{0.00} \textcolor{black!50}{0.00} n/a & 1 2 2 & 29 29 & 16 1 15\\
Zohali2022 \href{http://dx.doi.org/10.1287/ijoc.2020.1015}{Zohali2022} & \hyperref[auth:a1526]{H. Zohali}, \hyperref[auth:a726]{B. Naderi}, \hyperref[auth:a728]{V. Roshanaei} & Solving the Type-2 Assembly Line Balancing with Setups Using Logic-Based Benders Decomposition \hyperref[abs:Zohali2022]{Abstract} & \hyperref[detail:Zohali2022]{Details} No & \cite{Zohali2022} & 2022 & \cellcolor{red!20}INFORMS Journal on Computing & null & \noindent{}\textcolor{black!50}{0.00} \textcolor{black!50}{0.00} n/a & 11 12 12 & 52 53 & 12 2 10\\
RoshanaeiLAU17a \href{http://dx.doi.org/10.1287/ijoc.2017.0745}{RoshanaeiLAU17a} & \hyperref[auth:a728]{V. Roshanaei}, \hyperref[auth:a927]{C. Luong}, \hyperref[auth:a895]{D. M. Aleman}, \hyperref[auth:a896]{D. R. Urbach} & Collaborative Operating Room Planning and Scheduling & \hyperref[detail:RoshanaeiLAU17a]{Details} No & \cite{RoshanaeiLAU17a} & 2017 & \cellcolor{red!20}INFORMS Journal on Computing & 23 & \noindent{}\textcolor{black!50}{0.00} \textcolor{black!50}{0.00} n/a & 54 55 55 & 42 47 & 18 10 8\\
DoulabiRP16 \href{https://doi.org/10.1287/ijoc.2015.0686}{DoulabiRP16} & \hyperref[auth:a330]{S. H. H. Doulabi}, \hyperref[auth:a326]{L.-M. Rousseau}, \hyperref[auth:a8]{G. Pesant} & A Constraint-Programming-Based Branch-and-Price-and-Cut Approach for Operating Room Planning and Scheduling & \hyperref[detail:DoulabiRP16]{Details} \href{../works/DoulabiRP16.pdf}{Yes} & \cite{DoulabiRP16} & 2016 & \cellcolor{red!20}INFORMS Journal on Computing & 17 & \noindent{}\textcolor{black!50}{0.00} \textcolor{black!50}{0.00} \textbf{3.92} & 56 63 64 & 28 32 & 14 12 2\\
TranAB16 \href{https://doi.org/10.1287/ijoc.2015.0666}{TranAB16} & \hyperref[auth:a799]{T. T. Tran}, \hyperref[auth:a807]{A. Araujo}, \hyperref[auth:a89]{J. C. Beck} & Decomposition Methods for the Parallel Machine Scheduling Problem with Setups & \hyperref[detail:TranAB16]{Details} \href{../works/TranAB16.pdf}{Yes} & \cite{TranAB16} & 2016 & \cellcolor{red!20}INFORMS Journal on Computing & 13 & \noindent{}\textcolor{black!50}{0.00} \textcolor{black!50}{0.00} \textbf{3.31} & 72 75 80 & 28 36 & 31 17 14\\
GrimesH15 \href{https://doi.org/10.1287/ijoc.2014.0625}{GrimesH15} & \hyperref[auth:a181]{D. Grimes}, \hyperref[auth:a1]{E. Hebrard} & \cellcolor{green!10}Solving Variants of the Job Shop Scheduling Problem Through Conflict-Directed Search & \hyperref[detail:GrimesH15]{Details} \href{../works/GrimesH15.pdf}{Yes} & \cite{GrimesH15} & 2015 & \cellcolor{red!20}INFORMS Journal on Computing & 17 & \noindent{}\textcolor{black!50}{0.00} \textcolor{black!50}{0.00} \textbf{35.48} & 12 13 16 & 41 66 & 28 5 23\\
BlomBPS14 \href{https://doi.org/10.1287/ijoc.2013.0590}{BlomBPS14} & \hyperref[auth:a795]{M. L. Blom}, \hyperref[auth:a322]{C. N. Burt}, \hyperref[auth:a324]{A. R. Pearce}, \hyperref[auth:a125]{P. J. Stuckey} & A Decomposition-Based Heuristic for Collaborative Scheduling in a Network of Open-Pit Mines & \hyperref[detail:BlomBPS14]{Details} \href{../works/BlomBPS14.pdf}{Yes} & \cite{BlomBPS14} & 2014 & \cellcolor{red!20}INFORMS Journal on Computing & 19 & \noindent{}\textcolor{black!50}{0.00} \textcolor{black!50}{0.00} \textcolor{black!50}{0.00} & 15 15 16 & 47 55 & 2 1 1\\
Clautiaux2013 \href{http://dx.doi.org/10.1287/ijoc.1110.0478}{Clautiaux2013} & \hyperref[auth:a1686]{F. Clautiaux}, \hyperref[auth:a929]{A. Jouglet}, \hyperref[auth:a1170]{A. Moukrim} & A New Graph-Theoretical Model for the Guillotine-Cutting Problem \hyperref[abs:Clautiaux2013]{Abstract} & \hyperref[detail:Clautiaux2013]{Details} No & \cite{Clautiaux2013} & 2013 & \cellcolor{red!20}INFORMS Journal on Computing & null & \noindent{}\textcolor{black!50}{0.00} 0.50 n/a & 5 5 7 & 20 22 & 2 0 2\\
Berbeglia2012 \href{http://dx.doi.org/10.1287/ijoc.1110.0454}{Berbeglia2012} & \hyperref[auth:a1847]{G. Berbeglia}, \hyperref[auth:a1848]{J.-F. Cordeau}, \hyperref[auth:a1074]{G. Laporte} & A Hybrid Tabu Search and Constraint Programming Algorithm for the Dynamic Dial-a-Ride Problem \hyperref[abs:Berbeglia2012]{Abstract} & \hyperref[detail:Berbeglia2012]{Details} No & \cite{Berbeglia2012} & 2012 & \cellcolor{red!20}INFORMS Journal on Computing & null & \noindent{}\textcolor{black!50}{0.00} \textbf{1.00} n/a & 65 68 78 & 24 26 & 1 1 0\\
Eirinakis2012 \href{http://dx.doi.org/10.1287/ijoc.1110.0449}{Eirinakis2012} & \hyperref[auth:a1916]{P. Eirinakis}, \hyperref[auth:a1917]{D. Magos}, \hyperref[auth:a1918]{I. Mourtos}, \hyperref[auth:a1919]{P. Miliotis} & Finding All Stable Pairs and Solutions to the Many-to-Many Stable Matching Problem \hyperref[abs:Eirinakis2012]{Abstract} & \hyperref[detail:Eirinakis2012]{Details} No & \cite{Eirinakis2012} & 2012 & \cellcolor{red!20}INFORMS Journal on Computing & null & \noindent{}\textcolor{black!50}{0.00} \textbf{2.00} n/a & 11 11 13 & 25 30 & 1 0 1\\
MalapertCGJLR12 \href{https://doi.org/10.1287/ijoc.1100.0446}{MalapertCGJLR12} & \hyperref[auth:a82]{A. Malapert}, \hyperref[auth:a998]{H. Cambazard}, \hyperref[auth:a293]{C. Gu{\'{e}}ret}, \hyperref[auth:a247]{N. Jussien}, \hyperref[auth:a645]{A. Langevin}, \hyperref[auth:a326]{L.-M. Rousseau} & \cellcolor{green!10}An Optimal Constraint Programming Approach to the Open-Shop Problem & \hyperref[detail:MalapertCGJLR12]{Details} \href{../works/MalapertCGJLR12.pdf}{Yes} & \cite{MalapertCGJLR12} & 2012 & \cellcolor{red!20}INFORMS Journal on Computing & 17 & \noindent{}\textcolor{black!50}{0.00} \textcolor{black!50}{0.00} \textbf{12.50} & 23 24 25 & 21 31 & 20 15 5\\
ZarandiB12 \href{http://dx.doi.org/10.1287/ijoc.1110.0458}{ZarandiB12} & \hyperref[auth:a945]{M. M. Fazel-Zarandi}, \hyperref[auth:a89]{J. C. Beck} & Using Logic-Based Benders Decomposition to Solve the Capacity- and Distance-Constrained Plant Location Problem & \hyperref[detail:ZarandiB12]{Details} No & \cite{ZarandiB12} & 2012 & \cellcolor{red!20}INFORMS Journal on Computing & 12 & \noindent{}\textcolor{black!50}{0.00} \textcolor{black!50}{0.00} n/a & 38 38 42 & 57 61 & 28 18 10\\
BandaSC11 \href{https://doi.org/10.1287/ijoc.1090.0378}{BandaSC11} & \hyperref[auth:a796]{Maria Garcia de la Banda}, \hyperref[auth:a125]{P. J. Stuckey}, \hyperref[auth:a343]{G. Chu} & Solving Talent Scheduling with Dynamic Programming & \hyperref[detail:BandaSC11]{Details} \href{../works/BandaSC11.pdf}{Yes} & \cite{BandaSC11} & 2011 & \cellcolor{red!20}INFORMS Journal on Computing & 18 & \noindent{}\textcolor{black!50}{0.00} \textcolor{black!50}{0.00} 0.79 & 24 25 28 & 17 18 & 1 1 0\\
BeckFW11 \href{https://doi.org/10.1287/ijoc.1100.0388}{BeckFW11} & \hyperref[auth:a89]{J. C. Beck}, \hyperref[auth:a822]{T. K. Feng}, \hyperref[auth:a360]{J.-P. Watson} & Combining Constraint Programming and Local Search for Job-Shop Scheduling & \hyperref[detail:BeckFW11]{Details} \href{../works/BeckFW11.pdf}{Yes} & \cite{BeckFW11} & 2011 & \cellcolor{red!20}INFORMS Journal on Computing & 14 & \noindent{}\textbf{2.00} \textbf{2.00} \textbf{4.48} & 43 46 59 & 23 33 & 19 9 10\\
Jans09 \href{http://dx.doi.org/10.1287/ijoc.1080.0283}{Jans09} & \hyperref[auth:a841]{R. Jans} & \cellcolor{green!10}Solving Lot-Sizing Problems on Parallel Identical Machines Using Symmetry-Breaking Constraints & \hyperref[detail:Jans09]{Details} \href{../works/Jans09.pdf}{Yes} & \cite{Jans09} & 2009 & \cellcolor{red!20}INFORMS Journal on Computing & 14 & \noindent{}\textcolor{black!50}{0.00} \textcolor{black!50}{0.00} \textcolor{black!50}{0.00} & 59 60 61 & 73 77 & 15 0 15\\
Michel2009 \href{http://dx.doi.org/10.1287/ijoc.1080.0313}{Michel2009} & \hyperref[auth:a32]{L. Michel}, \hyperref[auth:a1807]{A. See}, \hyperref[auth:a148]{P. V. Hentenryck} & Transparent Parallelization of Constraint Programming \hyperref[abs:Michel2009]{Abstract} & \hyperref[detail:Michel2009]{Details} No & \cite{Michel2009} & 2009 & \cellcolor{red!20}INFORMS Journal on Computing & null & \noindent{}\textcolor{black!50}{0.00} 0.50 n/a & 28 28 39 & 15 24 & 5 1 4\\
MercierH08 \href{http://dx.doi.org/10.1287/ijoc.1070.0226}{MercierH08} & \hyperref[auth:a851]{L. Mercier}, \hyperref[auth:a148]{P. V. Hentenryck} & Edge Finding for Cumulative Scheduling & \hyperref[detail:MercierH08]{Details} \href{../works/MercierH08.pdf}{Yes} & \cite{MercierH08} & 2008 & \cellcolor{red!20}INFORMS Journal on Computing & 11 & \noindent{}\textcolor{black!50}{0.00} \textcolor{black!50}{0.00} 0.71 & 32 33 0 & 5 8 & 31 26 5\\
SadykovW06 \href{https://doi.org/10.1287/ijoc.1040.0110}{SadykovW06} & \hyperref[auth:a384]{R. Sadykov}, \hyperref[auth:a224]{L. A. Wolsey} & Integer Programming and Constraint Programming in Solving a Multimachine Assignment Scheduling Problem with Deadlines and Release Dates & \hyperref[detail:SadykovW06]{Details} \href{../works/SadykovW06.pdf}{Yes} & \cite{SadykovW06} & 2006 & \cellcolor{red!20}INFORMS Journal on Computing & 9 & \noindent{}\textbf{1.50} \textbf{1.50} \textbf{8.63} & 45 46 38 & 6 9 & 18 15 3\\
Zhu2006 \href{http://dx.doi.org/10.1287/ijoc.1040.0121}{Zhu2006} & \hyperref[auth:a1528]{G. Zhu}, \hyperref[auth:a1529]{J. F. Bard}, \hyperref[auth:a1530]{G. Yu} & A Branch-and-Cut Procedure for the Multimode Resource-Constrained Project-Scheduling Problem \hyperref[abs:Zhu2006]{Abstract} & \hyperref[detail:Zhu2006]{Details} No & \cite{Zhu2006} & 2006 & \cellcolor{red!20}INFORMS Journal on Computing & null & \noindent{}\textcolor{black!50}{0.00} \textcolor{black!50}{0.00} n/a & 78 85 118 & 23 23 & 19 15 4\\
DemasseyAM05 \href{http://dx.doi.org/10.1287/ijoc.1030.0043}{DemasseyAM05} & \hyperref[auth:a243]{S. Demassey}, \hyperref[auth:a6]{C. Artigues}, \hyperref[auth:a355]{P. Michelon} & \cellcolor{green!10}Constraint-Propagation-Based Cutting Planes: An Application to the Resource-Constrained Project Scheduling Problem & \hyperref[detail:DemasseyAM05]{Details} \href{../works/DemasseyAM05.pdf}{Yes} & \cite{DemasseyAM05} & 2005 & \cellcolor{red!20}INFORMS Journal on Computing & 14 & \noindent{}\textbf{2.25} \textbf{2.25} \textbf{12.96} & 43 43 51 & 25 30 & 18 6 12\\
Hooker02 \href{http://dx.doi.org/10.1287/ijoc.14.4.295.2828}{Hooker02} & \hyperref[auth:a160]{J. N. Hooker} & Logic, Optimization, and Constraint Programming & \hyperref[detail:Hooker02]{Details} No & \cite{Hooker02} & 2002 & \cellcolor{red!20}INFORMS Journal on Computing & 27 & \noindent{}\textcolor{black!50}{0.00} \textcolor{black!50}{0.00} n/a & 94 93 0 & 84 149 & 40 22 18\\
MilanoORT02 \href{http://dx.doi.org/10.1287/ijoc.14.4.387.2830}{MilanoORT02} & \hyperref[auth:a143]{M. Milano}, \hyperref[auth:a852]{G. Ottosson}, \hyperref[auth:a254]{P. Refalo}, \hyperref[auth:a874]{E. S. Thorsteinsson} & The Role of Integer Programming Techniques in Constraint Programming's Global Constraints & \hyperref[detail:MilanoORT02]{Details} No & \cite{MilanoORT02} & 2002 & \cellcolor{red!20}INFORMS Journal on Computing & 16 & \noindent{}\textcolor{black!50}{0.00} \textcolor{black!50}{0.00} n/a & 14 14 0 & 31 60 & 18 4 14\\
JainG01 \href{http://dx.doi.org/10.1287/ijoc.13.4.258.9733}{JainG01} & \hyperref[auth:a844]{V. Jain}, \hyperref[auth:a382]{I. E. Grossmann} & Algorithms for Hybrid MILP/CP Models for a Class of Optimization Problems & \hyperref[detail:JainG01]{Details} \href{../works/JainG01.pdf}{Yes} & \cite{JainG01} & 2001 & \cellcolor{red!20}INFORMS Journal on Computing & 19 & \noindent{}\textcolor{black!50}{0.00} \textcolor{black!50}{0.00} \textbf{29.84} & 279 284 321 & 23 38 & 101 89 12\\
SourdN00 \href{https://doi.org/10.1287/ijoc.12.4.341.11881}{SourdN00} & \hyperref[auth:a775]{F. Sourd}, \hyperref[auth:a656]{W. Nuijten} & Multiple-Machine Lower Bounds for Shop-Scheduling Problems & \hyperref[detail:SourdN00]{Details} \href{../works/SourdN00.pdf}{Yes} & \cite{SourdN00} & 2000 & \cellcolor{red!20}INFORMS Journal on Computing & 12 & \noindent{}\textcolor{black!50}{0.00} \textcolor{black!50}{0.00} 0.32 & 7 7 8 & 14 23 & 13 2 11\\
BockmayrK98 \href{http://dx.doi.org/10.1287/ijoc.10.3.287}{BockmayrK98} & \hyperref[auth:a908]{A. Bockmayr}, \hyperref[auth:a1045]{T. Kasper} & Branch and Infer: A Unifying Framework for Integer and Finite Domain Constraint Programming & \hyperref[detail:BockmayrK98]{Details} No & \cite{BockmayrK98} & 1998 & \cellcolor{red!20}INFORMS Journal on Computing & 14 & \noindent{}\textcolor{black!50}{0.00} \textcolor{black!50}{0.00} n/a & 79 79 92 & 27 42 & 32 28 4\\
DarbyDowmanL98 \href{http://dx.doi.org/10.1287/ijoc.10.3.276}{DarbyDowmanL98} & \hyperref[auth:a177]{K. Darby-Dowman}, \hyperref[auth:a178]{J. Little} & Properties of Some Combinatorial Optimization Problems and Their Effect on the Performance of Integer Programming and Constraint Logic Programming & \hyperref[detail:DarbyDowmanL98]{Details} No & \cite{DarbyDowmanL98} & 1998 & \cellcolor{red!20}INFORMS Journal on Computing & 11 & \noindent{}\textcolor{black!50}{0.00} \textcolor{black!50}{0.00} n/a & 28 28 35 & 6 13 & 14 13 1\\
PeschT96 \href{http://dx.doi.org/10.1287/ijoc.8.2.144}{PeschT96} & \hyperref[auth:a438]{E. Pesch}, \hyperref[auth:a1216]{U. A. W. Tetzlaff} & Constraint Propagation Based Scheduling of Job Shops & \hyperref[detail:PeschT96]{Details} No & \cite{PeschT96} & 1996 & \cellcolor{red!20}INFORMS Journal on Computing & 14 & \noindent{}\textbf{3.00} \textbf{3.00} n/a & 22 23 17 & 0 0 & 7 7 0\\
\end{longtable}
}

\subsection{INFORMS Journal on Optimization}

\index{INFORMS Journal on Optimization}
{\scriptsize
\begin{longtable}{>{\raggedright\arraybackslash}p{2.5cm}>{\raggedright\arraybackslash}p{4.5cm}>{\raggedright\arraybackslash}p{6.0cm}p{1.0cm}rr>{\raggedright\arraybackslash}p{2.0cm}r>{\raggedright\arraybackslash}p{1cm}p{1cm}p{1cm}p{1cm}}
\rowcolor{white}\caption{Articles in Journal INFORMS Journal on Optimization (Total 1)}\\ \toprule
\rowcolor{white}\shortstack{Key\\Source} & Authors & Title (Colored by Open Access)& \shortstack{Details\\LC} & Cite & Year & \shortstack{Conference\\/Journal\\/School} & Pages & Relevance &\shortstack{Cites\\OC XR\\SC} & \shortstack{Refs\\OC\\XR} & \shortstack{Links\\Cites\\Refs}\\ \midrule\endhead
\bottomrule
\endfoot
NaderiR22 \href{http://dx.doi.org/10.1287/ijoo.2021.0056}{NaderiR22} & \hyperref[auth:a726]{B. Naderi}, \hyperref[auth:a728]{V. Roshanaei} & Critical-Path-Search Logic-Based Benders Decomposition Approaches for Flexible Job Shop Scheduling & \hyperref[detail:NaderiR22]{Details} No & \cite{NaderiR22} & 2022 & \cellcolor{red!20}INFORMS Journal on Optimization & 28 & \noindent{}\textcolor{black!50}{0.00} \textcolor{black!50}{0.00} n/a & 5 7 0 & 49 52 & 14 3 11\\
\end{longtable}
}

\subsection{INTERNATIONAL TRANSACTIONS IN OPERATIONAL RESEARCH}

\index{INTERNATIONAL TRANSACTIONS IN OPERATIONAL RESEARCH}
{\scriptsize
\begin{longtable}{>{\raggedright\arraybackslash}p{2.5cm}>{\raggedright\arraybackslash}p{4.5cm}>{\raggedright\arraybackslash}p{6.0cm}p{1.0cm}rr>{\raggedright\arraybackslash}p{2.0cm}r>{\raggedright\arraybackslash}p{1cm}p{1cm}p{1cm}p{1cm}}
\rowcolor{white}\caption{Articles in Journal INTERNATIONAL TRANSACTIONS IN OPERATIONAL RESEARCH (Total 1)}\\ \toprule
\rowcolor{white}\shortstack{Key\\Source} & Authors & Title (Colored by Open Access)& \shortstack{Details\\LC} & Cite & Year & \shortstack{Conference\\/Journal\\/School} & Pages & Relevance &\shortstack{Cites\\OC XR\\SC} & \shortstack{Refs\\OC\\XR} & \shortstack{Links\\Cites\\Refs}\\ \midrule\endhead
\bottomrule
\endfoot
Ribeiro12 \href{http://dx.doi.org/10.1111/j.1475-3995.2011.00819.x}{Ribeiro12} & \hyperref[auth:a1386]{C. C. Ribeiro} & Sports scheduling: Problems and applications \hyperref[abs:Ribeiro12]{Abstract} & \hyperref[detail:Ribeiro12]{Details} \href{../works/Ribeiro12.pdf}{Yes} & \cite{Ribeiro12} & 2012 & INTERNATIONAL TRANSACTIONS IN OPERATIONAL RESEARCH & 26 & \noindent{}\textcolor{black!50}{0.00} \textbf{1.00} \textbf{1.64} & 47 52 54 & 59 92 & 9 1 8\\
\end{longtable}
}

\subsection{ISIJ International}

\index{ISIJ International}
{\scriptsize
\begin{longtable}{>{\raggedright\arraybackslash}p{2.5cm}>{\raggedright\arraybackslash}p{4.5cm}>{\raggedright\arraybackslash}p{6.0cm}p{1.0cm}rr>{\raggedright\arraybackslash}p{2.0cm}r>{\raggedright\arraybackslash}p{1cm}p{1cm}p{1cm}p{1cm}}
\rowcolor{white}\caption{Articles in Journal ISIJ International (Total 1)}\\ \toprule
\rowcolor{white}\shortstack{Key\\Source} & Authors & Title (Colored by Open Access)& \shortstack{Details\\LC} & Cite & Year & \shortstack{Conference\\/Journal\\/School} & Pages & Relevance &\shortstack{Cites\\OC XR\\SC} & \shortstack{Refs\\OC\\XR} & \shortstack{Links\\Cites\\Refs}\\ \midrule\endhead
\bottomrule
\endfoot
Gao2018 \href{http://dx.doi.org/10.2355/isijinternational.isijint-2018-305}{Gao2018} & \hyperref[auth:a1712]{C. Gao}, \hyperref[auth:a1713]{D. Qu} & \cellcolor{gold!20}A Modelling and a New Hybrid MILP/CP Decomposition Method for Parallel Continuous Galvanizing Line Scheduling Problem & \hyperref[detail:Gao2018]{Details} No & \cite{Gao2018} & 2018 & ISIJ International & null & \noindent{}\textbf{1.00} \textbf{1.00} n/a & 1 2 2 & 7 16 & 4 0 4\\
\end{longtable}
}

\subsection{Independent Journal of Management \& Production}

\index{Independent Journal of Management \& Production}
{\scriptsize
\begin{longtable}{>{\raggedright\arraybackslash}p{2.5cm}>{\raggedright\arraybackslash}p{4.5cm}>{\raggedright\arraybackslash}p{6.0cm}p{1.0cm}rr>{\raggedright\arraybackslash}p{2.0cm}r>{\raggedright\arraybackslash}p{1cm}p{1cm}p{1cm}p{1cm}}
\rowcolor{white}\caption{Articles in Journal Independent Journal of Management \  Production (Total 1)}\\ \toprule
\rowcolor{white}\shortstack{Key\\Source} & Authors & Title (Colored by Open Access)& \shortstack{Details\\LC} & Cite & Year & \shortstack{Conference\\/Journal\\/School} & Pages & Relevance &\shortstack{Cites\\OC XR\\SC} & \shortstack{Refs\\OC\\XR} & \shortstack{Links\\Cites\\Refs}\\ \midrule\endhead
\bottomrule
\endfoot
Oliveira2015 \href{http://dx.doi.org/10.14807/ijmp.v6i1.262}{Oliveira2015} & \hyperref[auth:a1568]{Renata Melo e Silva de Oliveira}, \hyperref[auth:a1569]{Maria Sofia F. Oliveira de Castro Ribeiro} & Comparing Mixed \  Integer Programming vs. Constraint Programming by solving Job-Shop Scheduling Problems & \hyperref[detail:Oliveira2015]{Details} No & \cite{Oliveira2015} & 2015 & Independent Journal of Management \  Production & null & \noindent{}\textbf{2.00} \textbf{2.00} n/a & 2 1 0 & 0 0 & 2 2 0\\
\end{longtable}
}

\subsection{Indian Journal of Pure and Applied Mathematics}

\index{Indian Journal of Pure and Applied Mathematics}
{\scriptsize
\begin{longtable}{>{\raggedright\arraybackslash}p{2.5cm}>{\raggedright\arraybackslash}p{4.5cm}>{\raggedright\arraybackslash}p{6.0cm}p{1.0cm}rr>{\raggedright\arraybackslash}p{2.0cm}r>{\raggedright\arraybackslash}p{1cm}p{1cm}p{1cm}p{1cm}}
\rowcolor{white}\caption{Articles in Journal Indian Journal of Pure and Applied Mathematics (Total 1)}\\ \toprule
\rowcolor{white}\shortstack{Key\\Source} & Authors & Title (Colored by Open Access)& \shortstack{Details\\LC} & Cite & Year & \shortstack{Conference\\/Journal\\/School} & Pages & Relevance &\shortstack{Cites\\OC XR\\SC} & \shortstack{Refs\\OC\\XR} & \shortstack{Links\\Cites\\Refs}\\ \midrule\endhead
\bottomrule
\endfoot
KameugneF13 \href{http://dx.doi.org/10.1007/s13226-013-0005-z}{KameugneF13} & \hyperref[auth:a10]{R. Kameugne}, \hyperref[auth:a130]{L. P. Fotso} & A cumulative not-first/not-last filtering algorithm in O(n 2log(n)) & \hyperref[detail:KameugneF13]{Details} \href{../works/KameugneF13.pdf}{Yes} & \cite{KameugneF13} & 2013 & Indian Journal of Pure and Applied Mathematics & 21 & \noindent{}\textcolor{black!50}{0.00} \textcolor{black!50}{0.00} \textcolor{black!50}{0.00} & 6 8 8 & 4 19 & 10 6 4\\
\end{longtable}
}

\subsection{Industrial Management \& Data Systems}

\index{Industrial Management \& Data Systems}
{\scriptsize
\begin{longtable}{>{\raggedright\arraybackslash}p{2.5cm}>{\raggedright\arraybackslash}p{4.5cm}>{\raggedright\arraybackslash}p{6.0cm}p{1.0cm}rr>{\raggedright\arraybackslash}p{2.0cm}r>{\raggedright\arraybackslash}p{1cm}p{1cm}p{1cm}p{1cm}}
\rowcolor{white}\caption{Articles in Journal Industrial Management \  Data Systems (Total 1)}\\ \toprule
\rowcolor{white}\shortstack{Key\\Source} & Authors & Title (Colored by Open Access)& \shortstack{Details\\LC} & Cite & Year & \shortstack{Conference\\/Journal\\/School} & Pages & Relevance &\shortstack{Cites\\OC XR\\SC} & \shortstack{Refs\\OC\\XR} & \shortstack{Links\\Cites\\Refs}\\ \midrule\endhead
\bottomrule
\endfoot
Sitek2017 \href{http://dx.doi.org/10.1108/imds-10-2016-0465}{Sitek2017} & \hyperref[auth:a536]{P. Sitek}, \hyperref[auth:a535]{J. Wikarek}, \hyperref[auth:a1527]{P. Nielsen} & \cellcolor{gold!20}A constraint-driven approach to food supply chain management \hyperref[abs:Sitek2017]{Abstract} & \hyperref[detail:Sitek2017]{Details} No & \cite{Sitek2017} & 2017 & Industrial Management \  Data Systems & null & \noindent{}\textcolor{black!50}{0.00} \textbf{1.50} n/a & 24 0 26 & 26 0 & 4 1 3\\
\end{longtable}
}

\subsection{Industrial \& Engineering Chemistry Research}

\index{Industrial \& Engineering Chemistry Research}
{\scriptsize
\begin{longtable}{>{\raggedright\arraybackslash}p{2.5cm}>{\raggedright\arraybackslash}p{4.5cm}>{\raggedright\arraybackslash}p{6.0cm}p{1.0cm}rr>{\raggedright\arraybackslash}p{2.0cm}r>{\raggedright\arraybackslash}p{1cm}p{1cm}p{1cm}p{1cm}}
\rowcolor{white}\caption{Articles in Journal Industrial \  Engineering Chemistry Research (Total 2)}\\ \toprule
\rowcolor{white}\shortstack{Key\\Source} & Authors & Title (Colored by Open Access)& \shortstack{Details\\LC} & Cite & Year & \shortstack{Conference\\/Journal\\/School} & Pages & Relevance &\shortstack{Cites\\OC XR\\SC} & \shortstack{Refs\\OC\\XR} & \shortstack{Links\\Cites\\Refs}\\ \midrule\endhead
\bottomrule
\endfoot
ZeballosCM10 \href{http://dx.doi.org/10.1021/ie1016199}{ZeballosCM10} & \hyperref[auth:a621]{L. J. Zeballos}, \hyperref[auth:a891]{P. M. Castro}, \hyperref[auth:a1190]{C. A. Méndez} & \cellcolor{green!10}Integrated Constraint Programming Scheduling Approach for Automated Wet-Etch Stations in Semiconductor Manufacturing & \hyperref[detail:ZeballosCM10]{Details} \href{../works/ZeballosCM10.pdf}{Yes} & \cite{ZeballosCM10} & 2010 & Industrial \  Engineering Chemistry Research & 11 & \noindent{}\textbf{1.00} \textbf{1.00} \textbf{15.92} & 22 23 29 & 30 39 & 10 3 7\\
ZeballosM09 \href{http://dx.doi.org/10.1021/ie901176n}{ZeballosM09} & \hyperref[auth:a621]{L. J. Zeballos}, \hyperref[auth:a1190]{C. A. Méndez} & \cellcolor{green!10}An Integrated CP-Based Approach for Scheduling of Processing and Transport Units in Pipeless Plants & \hyperref[detail:ZeballosM09]{Details} \href{../works/ZeballosM09.pdf}{Yes} & \cite{ZeballosM09} & 2009 & Industrial \  Engineering Chemistry Research & 13 & \noindent{}\textbf{1.00} \textbf{1.00} \textbf{14.79} & 7 7 7 & 14 23 & 9 1 8\\
\end{longtable}
}

\subsection{Inf. Comput.}

\index{Inf. Comput.}
{\scriptsize
\begin{longtable}{>{\raggedright\arraybackslash}p{2.5cm}>{\raggedright\arraybackslash}p{4.5cm}>{\raggedright\arraybackslash}p{6.0cm}p{1.0cm}rr>{\raggedright\arraybackslash}p{2.0cm}r>{\raggedright\arraybackslash}p{1cm}p{1cm}p{1cm}p{1cm}}
\rowcolor{white}\caption{Articles in Journal Inf. Comput. (Total 1)}\\ \toprule
\rowcolor{white}\shortstack{Key\\Source} & Authors & Title (Colored by Open Access)& \shortstack{Details\\LC} & Cite & Year & \shortstack{Conference\\/Journal\\/School} & Pages & Relevance &\shortstack{Cites\\OC XR\\SC} & \shortstack{Refs\\OC\\XR} & \shortstack{Links\\Cites\\Refs}\\ \midrule\endhead
\bottomrule
\endfoot
FalaschiGMP97 \href{https://doi.org/10.1006/inco.1997.2638}{FalaschiGMP97} & \hyperref[auth:a687]{M. Falaschi}, \hyperref[auth:a192]{M. Gabbrielli}, \hyperref[auth:a688]{K. Marriott}, \hyperref[auth:a689]{C. Palamidessi} & \cellcolor{gold!20}Constraint Logic Programming with Dynamic Scheduling: {A} Semantics Based on Closure Operators & \hyperref[detail:FalaschiGMP97]{Details} \href{../works/FalaschiGMP97.pdf}{Yes} & \cite{FalaschiGMP97} & 1997 & Inf. Comput. & 27 & \noindent{}\textbf{1.00} \textbf{1.00} \textbf{1.33} & 10 10 12 & 9 15 & 0 0 0\\
\end{longtable}
}

\subsection{Int. J. Adv. Intell. Paradigms}

\index{Int. J. Adv. Intell. Paradigms}
{\scriptsize
\begin{longtable}{>{\raggedright\arraybackslash}p{2.5cm}>{\raggedright\arraybackslash}p{4.5cm}>{\raggedright\arraybackslash}p{6.0cm}p{1.0cm}rr>{\raggedright\arraybackslash}p{2.0cm}r>{\raggedright\arraybackslash}p{1cm}p{1cm}p{1cm}p{1cm}}
\rowcolor{white}\caption{Articles in Journal Int. J. Adv. Intell. Paradigms (Total 1)}\\ \toprule
\rowcolor{white}\shortstack{Key\\Source} & Authors & Title (Colored by Open Access)& \shortstack{Details\\LC} & Cite & Year & \shortstack{Conference\\/Journal\\/School} & Pages & Relevance &\shortstack{Cites\\OC XR\\SC} & \shortstack{Refs\\OC\\XR} & \shortstack{Links\\Cites\\Refs}\\ \midrule\endhead
\bottomrule
\endfoot
AlizdehS20 \href{https://doi.org/10.1504/IJAIP.2020.106687}{AlizdehS20} & \hyperref[auth:a513]{S. Alizdeh}, \hyperref[auth:a514]{S. Saeidi} & Fuzzy project scheduling with critical path including risk and resource constraints using linear programming & \hyperref[detail:AlizdehS20]{Details} No & \cite{AlizdehS20} & 2020 & \cellcolor{red!20}Int. J. Adv. Intell. Paradigms & 14 & \noindent{}\textcolor{black!50}{0.00} \textcolor{black!50}{0.00} n/a & 1 1 3 & 0 0 & 0 0 0\\
\end{longtable}
}

\subsection{Int. J. Artif. Intell. Tools}

\index{Int. J. Artif. Intell. Tools}
{\scriptsize
\begin{longtable}{>{\raggedright\arraybackslash}p{2.5cm}>{\raggedright\arraybackslash}p{4.5cm}>{\raggedright\arraybackslash}p{6.0cm}p{1.0cm}rr>{\raggedright\arraybackslash}p{2.0cm}r>{\raggedright\arraybackslash}p{1cm}p{1cm}p{1cm}p{1cm}}
\rowcolor{white}\caption{Articles in Journal Int. J. Artif. Intell. Tools (Total 2)}\\ \toprule
\rowcolor{white}\shortstack{Key\\Source} & Authors & Title (Colored by Open Access)& \shortstack{Details\\LC} & Cite & Year & \shortstack{Conference\\/Journal\\/School} & Pages & Relevance &\shortstack{Cites\\OC XR\\SC} & \shortstack{Refs\\OC\\XR} & \shortstack{Links\\Cites\\Refs}\\ \midrule\endhead
\bottomrule
\endfoot
AntunesABD20 \href{https://doi.org/10.1142/S0218213020600076}{AntunesABD20} & \hyperref[auth:a877]{M. Antunes}, \hyperref[auth:a878]{V. Armant}, \hyperref[auth:a217]{K. N. Brown}, \hyperref[auth:a879]{D. A. Desmond}, \hyperref[auth:a880]{G. Escamocher}, \hyperref[auth:a881]{A.-M. George}, \hyperref[auth:a181]{D. Grimes}, \hyperref[auth:a882]{M. O'Keeffe}, \hyperref[auth:a883]{Y. Lin}, \hyperref[auth:a16]{B. O'Sullivan}, \hyperref[auth:a135]{C. {\"{O}}zt{\"{u}}rk}, \hyperref[auth:a884]{L. Quesada}, \hyperref[auth:a129]{M. Siala}, \hyperref[auth:a17]{H. Simonis}, \hyperref[auth:a826]{N. Wilson} & \cellcolor{green!10}Assigning and Scheduling Service Visits in a Mixed Urban/Rural Setting & \hyperref[detail:AntunesABD20]{Details} \href{../works/AntunesABD20.pdf}{Yes} & \cite{AntunesABD20} & 2020 & Int. J. Artif. Intell. Tools & 31 & \noindent{}\textcolor{black!50}{0.00} \textcolor{black!50}{0.00} 0.63 & 0 0 1 & 16 18 & 0 0 0\\
MalikMB08 \href{https://doi.org/10.1142/S0218213008003765}{MalikMB08} & \hyperref[auth:a638]{A. M. Malik}, \hyperref[auth:a641]{J. McInnes}, \hyperref[auth:a610]{P. van Beek} & Optimal Basic Block Instruction Scheduling for Multiple-Issue Processors Using Constraint Programming & \hyperref[detail:MalikMB08]{Details} \href{../works/MalikMB08.pdf}{Yes} & \cite{MalikMB08} & 2008 & Int. J. Artif. Intell. Tools & 18 & \noindent{}\textbf{1.00} \textbf{1.00} \textbf{2.02} & 15 14 14 & 8 12 & 6 5 1\\
\end{longtable}
}

\subsection{Int. J. Comput. Integr. Manuf.}

\index{Int. J. Comput. Integr. Manuf.}
{\scriptsize
\begin{longtable}{>{\raggedright\arraybackslash}p{2.5cm}>{\raggedright\arraybackslash}p{4.5cm}>{\raggedright\arraybackslash}p{6.0cm}p{1.0cm}rr>{\raggedright\arraybackslash}p{2.0cm}r>{\raggedright\arraybackslash}p{1cm}p{1cm}p{1cm}p{1cm}}
\rowcolor{white}\caption{Articles in Journal Int. J. Comput. Integr. Manuf. (Total 1)}\\ \toprule
\rowcolor{white}\shortstack{Key\\Source} & Authors & Title (Colored by Open Access)& \shortstack{Details\\LC} & Cite & Year & \shortstack{Conference\\/Journal\\/School} & Pages & Relevance &\shortstack{Cites\\OC XR\\SC} & \shortstack{Refs\\OC\\XR} & \shortstack{Links\\Cites\\Refs}\\ \midrule\endhead
\bottomrule
\endfoot
MokhtarzadehTNF20 \href{https://doi.org/10.1080/0951192X.2020.1736713}{MokhtarzadehTNF20} & \hyperref[auth:a515]{M. Mokhtarzadeh}, \hyperref[auth:a430]{R. Tavakkoli-Moghaddam}, \hyperref[auth:a432]{B. V. Nouri}, \hyperref[auth:a516]{A. Farsi} & Scheduling of human-robot collaboration in assembly of printed circuit boards: a constraint programming approach & \hyperref[detail:MokhtarzadehTNF20]{Details} \href{../works/MokhtarzadehTNF20.pdf}{Yes} & \cite{MokhtarzadehTNF20} & 2020 & Int. J. Comput. Integr. Manuf. & 14 & \noindent{}\textbf{1.00} \textbf{1.00} \textbf{15.70} & 25 29 30 & 32 34 & 10 3 7\\
\end{longtable}
}

\subsection{Int. J. Electron. Secur. Digit. Forensics}

\index{Int. J. Electron. Secur. Digit. Forensics}
{\scriptsize
\begin{longtable}{>{\raggedright\arraybackslash}p{2.5cm}>{\raggedright\arraybackslash}p{4.5cm}>{\raggedright\arraybackslash}p{6.0cm}p{1.0cm}rr>{\raggedright\arraybackslash}p{2.0cm}r>{\raggedright\arraybackslash}p{1cm}p{1cm}p{1cm}p{1cm}}
\rowcolor{white}\caption{Articles in Journal Int. J. Electron. Secur. Digit. Forensics (Total 1)}\\ \toprule
\rowcolor{white}\shortstack{Key\\Source} & Authors & Title (Colored by Open Access)& \shortstack{Details\\LC} & Cite & Year & \shortstack{Conference\\/Journal\\/School} & Pages & Relevance &\shortstack{Cites\\OC XR\\SC} & \shortstack{Refs\\OC\\XR} & \shortstack{Links\\Cites\\Refs}\\ \midrule\endhead
\bottomrule
\endfoot
ShaikhK23 \href{https://doi.org/10.1504/IJESDF.2023.10045616}{ShaikhK23} & \hyperref[auth:a416]{A. A. Shaikh}, \hyperref[auth:a417]{A. A. Khan} & Management of electronic ledger: a constraint programming approach for solving curricula scheduling problems & \hyperref[detail:ShaikhK23]{Details} \href{../works/ShaikhK23.pdf}{Yes} & \cite{ShaikhK23} & 2023 & Int. J. Electron. Secur. Digit. Forensics & 12 & \noindent{}\textbf{1.00} \textbf{1.00} 0.56 & 0 0 0 & 0 0 & 0 0 0\\
\end{longtable}
}

\subsection{Int. J. Intell. Inf. Database Syst.}

\index{Int. J. Intell. Inf. Database Syst.}
{\scriptsize
\begin{longtable}{>{\raggedright\arraybackslash}p{2.5cm}>{\raggedright\arraybackslash}p{4.5cm}>{\raggedright\arraybackslash}p{6.0cm}p{1.0cm}rr>{\raggedright\arraybackslash}p{2.0cm}r>{\raggedright\arraybackslash}p{1cm}p{1cm}p{1cm}p{1cm}}
\rowcolor{white}\caption{Articles in Journal Int. J. Intell. Inf. Database Syst. (Total 1)}\\ \toprule
\rowcolor{white}\shortstack{Key\\Source} & Authors & Title (Colored by Open Access)& \shortstack{Details\\LC} & Cite & Year & \shortstack{Conference\\/Journal\\/School} & Pages & Relevance &\shortstack{Cites\\OC XR\\SC} & \shortstack{Refs\\OC\\XR} & \shortstack{Links\\Cites\\Refs}\\ \midrule\endhead
\bottomrule
\endfoot
BocewiczBB09 \href{https://doi.org/10.1504/IJIIDS.2009.023038}{BocewiczBB09} & \hyperref[auth:a630]{G. Bocewicz}, \hyperref[auth:a631]{I. Bach}, \hyperref[auth:a632]{Z. A. Banaszak} & Logic-algebraic method based and constraints programming driven approach to AGVs scheduling & \hyperref[detail:BocewiczBB09]{Details} \href{../works/BocewiczBB09.pdf}{Yes} & \cite{BocewiczBB09} & 2009 & Int. J. Intell. Inf. Database Syst. & 19 & \noindent{}\textcolor{black!50}{0.00} \textcolor{black!50}{0.00} 0.91 & 0 0 1 & 0 0 & 0 0 0\\
\end{longtable}
}

\subsection{Int. J. Netw. Comput.}

\index{Int. J. Netw. Comput.}
{\scriptsize
\begin{longtable}{>{\raggedright\arraybackslash}p{2.5cm}>{\raggedright\arraybackslash}p{4.5cm}>{\raggedright\arraybackslash}p{6.0cm}p{1.0cm}rr>{\raggedright\arraybackslash}p{2.0cm}r>{\raggedright\arraybackslash}p{1cm}p{1cm}p{1cm}p{1cm}}
\rowcolor{white}\caption{Articles in Journal Int. J. Netw. Comput. (Total 1)}\\ \toprule
\rowcolor{white}\shortstack{Key\\Source} & Authors & Title (Colored by Open Access)& \shortstack{Details\\LC} & Cite & Year & \shortstack{Conference\\/Journal\\/School} & Pages & Relevance &\shortstack{Cites\\OC XR\\SC} & \shortstack{Refs\\OC\\XR} & \shortstack{Links\\Cites\\Refs}\\ \midrule\endhead
\bottomrule
\endfoot
NishikawaSTT19 \href{http://www.ijnc.org/index.php/ijnc/article/view/201}{NishikawaSTT19} & \hyperref[auth:a531]{H. Nishikawa}, \hyperref[auth:a532]{K. Shimada}, \hyperref[auth:a533]{I. Taniguchi}, \hyperref[auth:a534]{H. Tomiyama} & A Constraint Programming Approach to Scheduling of Malleable Tasks & \hyperref[detail:NishikawaSTT19]{Details} \href{../works/NishikawaSTT19.pdf}{Yes} & \cite{NishikawaSTT19} & 2019 & Int. J. Netw. Comput. & 16 & \noindent{}\textbf{2.00} \textbf{2.00} \textbf{51.27} & 3 3 0 & 20 30 & 5 0 5\\
\end{longtable}
}

\subsection{Int. J. Parallel Emergent Distributed Syst.}

\index{Int. J. Parallel Emergent Distributed Syst.}
{\scriptsize
\begin{longtable}{>{\raggedright\arraybackslash}p{2.5cm}>{\raggedright\arraybackslash}p{4.5cm}>{\raggedright\arraybackslash}p{6.0cm}p{1.0cm}rr>{\raggedright\arraybackslash}p{2.0cm}r>{\raggedright\arraybackslash}p{1cm}p{1cm}p{1cm}p{1cm}}
\rowcolor{white}\caption{Articles in Journal Int. J. Parallel Emergent Distributed Syst. (Total 1)}\\ \toprule
\rowcolor{white}\shortstack{Key\\Source} & Authors & Title (Colored by Open Access)& \shortstack{Details\\LC} & Cite & Year & \shortstack{Conference\\/Journal\\/School} & Pages & Relevance &\shortstack{Cites\\OC XR\\SC} & \shortstack{Refs\\OC\\XR} & \shortstack{Links\\Cites\\Refs}\\ \midrule\endhead
\bottomrule
\endfoot
SureshMOK06 \href{https://doi.org/10.1080/17445760600567842}{SureshMOK06} & \hyperref[auth:a647]{S. Sundaram}, \hyperref[auth:a648]{V. Mani}, \hyperref[auth:a649]{S. N. Omkar}, \hyperref[auth:a650]{H. J. Kim} & Divisible load scheduling in distributed system with buffer constraints: genetic algorithm and linear programming approach & \hyperref[detail:SureshMOK06]{Details} \href{../works/SureshMOK06.pdf}{Yes} & \cite{SureshMOK06} & 2006 & Int. J. Parallel Emergent Distributed Syst. & 19 & \noindent{}\textcolor{black!50}{0.00} \textcolor{black!50}{0.00} \textcolor{black!50}{0.00} & 12 12 13 & 23 39 & 0 0 0\\
\end{longtable}
}

\subsection{Integrated Manufacturing Systems}

\index{Integrated Manufacturing Systems}
{\scriptsize
\begin{longtable}{>{\raggedright\arraybackslash}p{2.5cm}>{\raggedright\arraybackslash}p{4.5cm}>{\raggedright\arraybackslash}p{6.0cm}p{1.0cm}rr>{\raggedright\arraybackslash}p{2.0cm}r>{\raggedright\arraybackslash}p{1cm}p{1cm}p{1cm}p{1cm}}
\rowcolor{white}\caption{Articles in Journal Integrated Manufacturing Systems (Total 1)}\\ \toprule
\rowcolor{white}\shortstack{Key\\Source} & Authors & Title (Colored by Open Access)& \shortstack{Details\\LC} & Cite & Year & \shortstack{Conference\\/Journal\\/School} & Pages & Relevance &\shortstack{Cites\\OC XR\\SC} & \shortstack{Refs\\OC\\XR} & \shortstack{Links\\Cites\\Refs}\\ \midrule\endhead
\bottomrule
\endfoot
Priore2003 \href{http://dx.doi.org/10.1108/09576060310459456}{Priore2003} & \hyperref[auth:a1819]{P. Priore}, \hyperref[auth:a1820]{D. de la Fuente}, \hyperref[auth:a1821]{R. Pino}, \hyperref[auth:a1822]{J. Puente} & Dynamic scheduling of flexible manufacturing systems using neural networks and inductive learning \hyperref[abs:Priore2003]{Abstract} & \hyperref[detail:Priore2003]{Details} No & \cite{Priore2003} & 2003 & Integrated Manufacturing Systems & null & \noindent{}\textcolor{black!50}{0.00} \textbf{1.25} n/a & 15 15 14 & 23 26 & 1 1 0\\
\end{longtable}
}

\subsection{Inteligencia Artif.}

\index{Inteligencia Artif.}
{\scriptsize
\begin{longtable}{>{\raggedright\arraybackslash}p{2.5cm}>{\raggedright\arraybackslash}p{4.5cm}>{\raggedright\arraybackslash}p{6.0cm}p{1.0cm}rr>{\raggedright\arraybackslash}p{2.0cm}r>{\raggedright\arraybackslash}p{1cm}p{1cm}p{1cm}p{1cm}}
\rowcolor{white}\caption{Articles in Journal Inteligencia Artif. (Total 1)}\\ \toprule
\rowcolor{white}\shortstack{Key\\Source} & Authors & Title (Colored by Open Access)& \shortstack{Details\\LC} & Cite & Year & \shortstack{Conference\\/Journal\\/School} & Pages & Relevance &\shortstack{Cites\\OC XR\\SC} & \shortstack{Refs\\OC\\XR} & \shortstack{Links\\Cites\\Refs}\\ \midrule\endhead
\bottomrule
\endfoot
ZeballosH05 \href{http://journal.iberamia.org/index.php/ia/article/view/452/article\%20\%281\%29.pdf}{ZeballosH05} & \hyperref[auth:a621]{L. J. Zeballos}, \hyperref[auth:a588]{G. P. Henning} & \cellcolor{green!10}A Constraint Programming Approach to {FMS} Scheduling. Consideration of Storage and Transportation Resources & \hyperref[detail:ZeballosH05]{Details} \href{../works/ZeballosH05.pdf}{Yes} & \cite{ZeballosH05} & 2005 & Inteligencia Artif. & 10 & \noindent{}\textbf{1.50} \textbf{1.50} \textbf{8.59} & 0 0 0 & 0 0 & 0 0 0\\
\end{longtable}
}

\subsection{Inteligencia Artificial}

\index{Inteligencia Artificial}
{\scriptsize
\begin{longtable}{>{\raggedright\arraybackslash}p{2.5cm}>{\raggedright\arraybackslash}p{4.5cm}>{\raggedright\arraybackslash}p{6.0cm}p{1.0cm}rr>{\raggedright\arraybackslash}p{2.0cm}r>{\raggedright\arraybackslash}p{1cm}p{1cm}p{1cm}p{1cm}}
\rowcolor{white}\caption{Articles in Journal Inteligencia Artificial (Total 1)}\\ \toprule
\rowcolor{white}\shortstack{Key\\Source} & Authors & Title (Colored by Open Access)& \shortstack{Details\\LC} & Cite & Year & \shortstack{Conference\\/Journal\\/School} & Pages & Relevance &\shortstack{Cites\\OC XR\\SC} & \shortstack{Refs\\OC\\XR} & \shortstack{Links\\Cites\\Refs}\\ \midrule\endhead
\bottomrule
\endfoot
Cox2019 \href{http://dx.doi.org/10.4114/intartif.vol22iss63pp1-15}{Cox2019} & \hyperref[auth:a1920]{J. L. Cox}, \hyperref[auth:a1921]{S. Lucci}, \hyperref[auth:a1922]{T. Pay} & \cellcolor{gold!20}Effects of Dynamic Variable - Value Ordering  Heuristics on the Search Space of Sudoku Modeled as a Constraint Satisfaction Problem \hyperref[abs:Cox2019]{Abstract} & \hyperref[detail:Cox2019]{Details} No & \cite{Cox2019} & 2019 & Inteligencia Artificial & null & \noindent{}0.50 0.50 n/a & 1 1 2 & 0 0 & 1 1 0\\
\end{longtable}
}

\subsection{Intelligent Systems Engineering}

\index{Intelligent Systems Engineering}
{\scriptsize
\begin{longtable}{>{\raggedright\arraybackslash}p{2.5cm}>{\raggedright\arraybackslash}p{4.5cm}>{\raggedright\arraybackslash}p{6.0cm}p{1.0cm}rr>{\raggedright\arraybackslash}p{2.0cm}r>{\raggedright\arraybackslash}p{1cm}p{1cm}p{1cm}p{1cm}}
\rowcolor{white}\caption{Articles in Journal Intelligent Systems Engineering (Total 1)}\\ \toprule
\rowcolor{white}\shortstack{Key\\Source} & Authors & Title (Colored by Open Access)& \shortstack{Details\\LC} & Cite & Year & \shortstack{Conference\\/Journal\\/School} & Pages & Relevance &\shortstack{Cites\\OC XR\\SC} & \shortstack{Refs\\OC\\XR} & \shortstack{Links\\Cites\\Refs}\\ \midrule\endhead
\bottomrule
\endfoot
Pape94 \href{http://dx.doi.org/10.1049/ise.1994.0009}{Pape94} & \hyperref[auth:a163]{C. L. Pape} & Implementation of resource constraints in ILOG SCHEDULE: a library for the development of constraint-based scheduling systems & \hyperref[detail:Pape94]{Details} \href{../works/Pape94.pdf}{Yes} & \cite{Pape94} & 1994 & Intelligent Systems Engineering & 34 & \noindent{}\textcolor{black!50}{0.00} \textcolor{black!50}{0.00} \textbf{12.56} & 98 98 103 & 0 53 & 38 38 0\\
\end{longtable}
}

\subsection{Intelligenza Artificiale}

\index{Intelligenza Artificiale}
{\scriptsize
\begin{longtable}{>{\raggedright\arraybackslash}p{2.5cm}>{\raggedright\arraybackslash}p{4.5cm}>{\raggedright\arraybackslash}p{6.0cm}p{1.0cm}rr>{\raggedright\arraybackslash}p{2.0cm}r>{\raggedright\arraybackslash}p{1cm}p{1cm}p{1cm}p{1cm}}
\rowcolor{white}\caption{Articles in Journal Intelligenza Artificiale (Total 1)}\\ \toprule
\rowcolor{white}\shortstack{Key\\Source} & Authors & Title (Colored by Open Access)& \shortstack{Details\\LC} & Cite & Year & \shortstack{Conference\\/Journal\\/School} & Pages & Relevance &\shortstack{Cites\\OC XR\\SC} & \shortstack{Refs\\OC\\XR} & \shortstack{Links\\Cites\\Refs}\\ \midrule\endhead
\bottomrule
\endfoot
Bonfietti16 \href{https://doi.org/10.3233/IA-160095}{Bonfietti16} & \hyperref[auth:a198]{A. Bonfietti} & A constraint programming scheduling solver for the MPOpt programming environment & \hyperref[detail:Bonfietti16]{Details} \href{../works/Bonfietti16.pdf}{Yes} & \cite{Bonfietti16} & 2016 & Intelligenza Artificiale & 13 & \noindent{}\textbf{1.00} \textbf{1.00} \textcolor{black!50}{0.13} & 0 0 0 & 19 37 & 1 0 1\\
\end{longtable}
}

\subsection{Interfaces}

\index{Interfaces}
{\scriptsize
\begin{longtable}{>{\raggedright\arraybackslash}p{2.5cm}>{\raggedright\arraybackslash}p{4.5cm}>{\raggedright\arraybackslash}p{6.0cm}p{1.0cm}rr>{\raggedright\arraybackslash}p{2.0cm}r>{\raggedright\arraybackslash}p{1cm}p{1cm}p{1cm}p{1cm}}
\rowcolor{white}\caption{Articles in Journal Interfaces (Total 2)}\\ \toprule
\rowcolor{white}\shortstack{Key\\Source} & Authors & Title (Colored by Open Access)& \shortstack{Details\\LC} & Cite & Year & \shortstack{Conference\\/Journal\\/School} & Pages & Relevance &\shortstack{Cites\\OC XR\\SC} & \shortstack{Refs\\OC\\XR} & \shortstack{Links\\Cites\\Refs}\\ \midrule\endhead
\bottomrule
\endfoot
Lambert2014 \href{http://dx.doi.org/10.1287/inte.2013.0731}{Lambert2014} & \hyperref[auth:a1558]{W. B. Lambert}, \hyperref[auth:a1559]{A. Brickey}, \hyperref[auth:a1560]{A. M. Newman}, \hyperref[auth:a1561]{K. Eurek} & Open-Pit Block-Sequencing Formulations: A Tutorial \hyperref[abs:Lambert2014]{Abstract} & \hyperref[detail:Lambert2014]{Details} No & \cite{Lambert2014} & 2014 & \cellcolor{red!20}Interfaces & null & \noindent{}\textcolor{black!50}{0.00} \textcolor{black!50}{0.00} n/a & 37 40 48 & 13 22 & 2 2 0\\
Fisher1985 \href{http://dx.doi.org/10.1287/inte.15.2.10}{Fisher1985} & \hyperref[auth:a1772]{M. L. Fisher} & An Applications Oriented Guide to Lagrangian Relaxation \hyperref[abs:Fisher1985]{Abstract} & \hyperref[detail:Fisher1985]{Details} No & \cite{Fisher1985} & 1985 & \cellcolor{red!20}Interfaces & null & \noindent{}\textcolor{black!50}{0.00} \textcolor{black!50}{0.00} n/a & 462 473 517 & 0 0 & 2 2 0\\
\end{longtable}
}

\subsection{International Journal of Advanced Robotic Systems}

\index{International Journal of Advanced Robotic Systems}
{\scriptsize
\begin{longtable}{>{\raggedright\arraybackslash}p{2.5cm}>{\raggedright\arraybackslash}p{4.5cm}>{\raggedright\arraybackslash}p{6.0cm}p{1.0cm}rr>{\raggedright\arraybackslash}p{2.0cm}r>{\raggedright\arraybackslash}p{1cm}p{1cm}p{1cm}p{1cm}}
\rowcolor{white}\caption{Articles in Journal International Journal of Advanced Robotic Systems (Total 1)}\\ \toprule
\rowcolor{white}\shortstack{Key\\Source} & Authors & Title (Colored by Open Access)& \shortstack{Details\\LC} & Cite & Year & \shortstack{Conference\\/Journal\\/School} & Pages & Relevance &\shortstack{Cites\\OC XR\\SC} & \shortstack{Refs\\OC\\XR} & \shortstack{Links\\Cites\\Refs}\\ \midrule\endhead
\bottomrule
\endfoot
Zhang2013 \href{http://dx.doi.org/10.5772/55956}{Zhang2013} & \hyperref[auth:a1517]{R. Zhang} & \cellcolor{gold!20}A Simulated Annealing-Based Heuristic Algorithm for Job Shop Scheduling to Minimize Lateness \hyperref[abs:Zhang2013]{Abstract} & \hyperref[detail:Zhang2013]{Details} No & \cite{Zhang2013} & 2013 & International Journal of Advanced Robotic Systems & null & \noindent{}\textcolor{black!50}{0.00} \textbf{3.00} n/a & 6 8 13 & 26 27 & 4 1 3\\
\end{longtable}
}

\subsection{International Journal of Applied Evolutionary Computation}

\index{International Journal of Applied Evolutionary Computation}
{\scriptsize
\begin{longtable}{>{\raggedright\arraybackslash}p{2.5cm}>{\raggedright\arraybackslash}p{4.5cm}>{\raggedright\arraybackslash}p{6.0cm}p{1.0cm}rr>{\raggedright\arraybackslash}p{2.0cm}r>{\raggedright\arraybackslash}p{1cm}p{1cm}p{1cm}p{1cm}}
\rowcolor{white}\caption{Articles in Journal International Journal of Applied Evolutionary Computation (Total 1)}\\ \toprule
\rowcolor{white}\shortstack{Key\\Source} & Authors & Title (Colored by Open Access)& \shortstack{Details\\LC} & Cite & Year & \shortstack{Conference\\/Journal\\/School} & Pages & Relevance &\shortstack{Cites\\OC XR\\SC} & \shortstack{Refs\\OC\\XR} & \shortstack{Links\\Cites\\Refs}\\ \midrule\endhead
\bottomrule
\endfoot
Zoulfaghari2013 \href{http://dx.doi.org/10.4018/jaec.2013040103}{Zoulfaghari2013} & \hyperref[auth:a1758]{H. Zoulfaghari}, \hyperref[auth:a1759]{J. Nematian}, \hyperref[auth:a1760]{N. Mahmoudi}, \hyperref[auth:a1761]{M. Khodabandeh} & A New Genetic Algorithm for the RCPSP in Large Scale \hyperref[abs:Zoulfaghari2013]{Abstract} & \hyperref[detail:Zoulfaghari2013]{Details} No & \cite{Zoulfaghari2013} & 2013 & International Journal of Applied Evolutionary Computation & null & \noindent{}\textcolor{black!50}{0.00} \textcolor{black!50}{0.00} n/a & 5 5 0 & 38 45 & 5 0 5\\
\end{longtable}
}

\subsection{International Journal of Applied Management Science}

\index{International Journal of Applied Management Science}
{\scriptsize
\begin{longtable}{>{\raggedright\arraybackslash}p{2.5cm}>{\raggedright\arraybackslash}p{4.5cm}>{\raggedright\arraybackslash}p{6.0cm}p{1.0cm}rr>{\raggedright\arraybackslash}p{2.0cm}r>{\raggedright\arraybackslash}p{1cm}p{1cm}p{1cm}p{1cm}}
\rowcolor{white}\caption{Articles in Journal International Journal of Applied Management Science (Total 1)}\\ \toprule
\rowcolor{white}\shortstack{Key\\Source} & Authors & Title (Colored by Open Access)& \shortstack{Details\\LC} & Cite & Year & \shortstack{Conference\\/Journal\\/School} & Pages & Relevance &\shortstack{Cites\\OC XR\\SC} & \shortstack{Refs\\OC\\XR} & \shortstack{Links\\Cites\\Refs}\\ \midrule\endhead
\bottomrule
\endfoot
FallahiAC20 \href{https://api.semanticscholar.org/CorpusID:213449737}{FallahiAC20} & \hyperref[auth:a753]{A. E. Fallahi}, \hyperref[auth:a754]{E. Y. Anass}, \hyperref[auth:a755]{M. Cherkaoui} & Tabu search and constraint programming-based approach for a real scheduling and routing problem & \hyperref[detail:FallahiAC20]{Details} \href{../works/FallahiAC20.pdf}{Yes} & \cite{FallahiAC20} & 2020 & International Journal of Applied Management Science & 18 & \noindent{}\textbf{1.00} \textbf{1.00} \textbf{5.04} & 0 0 0 & 0 0 & 0 0 0\\
\end{longtable}
}

\subsection{International Journal of Applied Metaheuristic Computing}

\index{International Journal of Applied Metaheuristic Computing}
{\scriptsize
\begin{longtable}{>{\raggedright\arraybackslash}p{2.5cm}>{\raggedright\arraybackslash}p{4.5cm}>{\raggedright\arraybackslash}p{6.0cm}p{1.0cm}rr>{\raggedright\arraybackslash}p{2.0cm}r>{\raggedright\arraybackslash}p{1cm}p{1cm}p{1cm}p{1cm}}
\rowcolor{white}\caption{Articles in Journal International Journal of Applied Metaheuristic Computing (Total 1)}\\ \toprule
\rowcolor{white}\shortstack{Key\\Source} & Authors & Title (Colored by Open Access)& \shortstack{Details\\LC} & Cite & Year & \shortstack{Conference\\/Journal\\/School} & Pages & Relevance &\shortstack{Cites\\OC XR\\SC} & \shortstack{Refs\\OC\\XR} & \shortstack{Links\\Cites\\Refs}\\ \midrule\endhead
\bottomrule
\endfoot
Mnif2020 \href{http://dx.doi.org/10.4018/ijamc.2020040107}{Mnif2020} & \hyperref[auth:a1964]{M. G. Mnif}, \hyperref[auth:a1965]{S. Bouamama} & Multi-Layer Distributed Constraint Satisfaction for Multi-criteria Optimization Problem \hyperref[abs:Mnif2020]{Abstract} & \hyperref[detail:Mnif2020]{Details} No & \cite{Mnif2020} & 2020 & International Journal of Applied Metaheuristic Computing & null & \noindent{}\textcolor{black!50}{0.00} \textbf{6.01} n/a & 2 2 2 & 11 19 & 1 0 1\\
\end{longtable}
}

\subsection{International Journal of Embedded and Real-Time Communication Systems}

\index{International Journal of Embedded and Real-Time Communication Systems}
{\scriptsize
\begin{longtable}{>{\raggedright\arraybackslash}p{2.5cm}>{\raggedright\arraybackslash}p{4.5cm}>{\raggedright\arraybackslash}p{6.0cm}p{1.0cm}rr>{\raggedright\arraybackslash}p{2.0cm}r>{\raggedright\arraybackslash}p{1cm}p{1cm}p{1cm}p{1cm}}
\rowcolor{white}\caption{Articles in Journal International Journal of Embedded and Real-Time Communication Systems (Total 1)}\\ \toprule
\rowcolor{white}\shortstack{Key\\Source} & Authors & Title (Colored by Open Access)& \shortstack{Details\\LC} & Cite & Year & \shortstack{Conference\\/Journal\\/School} & Pages & Relevance &\shortstack{Cites\\OC XR\\SC} & \shortstack{Refs\\OC\\XR} & \shortstack{Links\\Cites\\Refs}\\ \midrule\endhead
\bottomrule
\endfoot
Raffin2012 \href{http://dx.doi.org/10.4018/jertcs.2012010101}{Raffin2012} & \hyperref[auth:a1531]{E. Raffin}, \hyperref[auth:a659]{C. Wolinski}, \hyperref[auth:a1532]{F. Charot}, \hyperref[auth:a1533]{E. Casseau}, \hyperref[auth:a1534]{A. Floc’h}, \hyperref[auth:a660]{K. Kuchcinski}, \hyperref[auth:a1535]{S. Chevobbe}, \hyperref[auth:a1536]{S. Guyetant} & \cellcolor{green!10}Scheduling, Binding and Routing System for a Run-Time Reconfigurable Operator Based Multimedia Architecture \hyperref[abs:Raffin2012]{Abstract} & \hyperref[detail:Raffin2012]{Details} No & \cite{Raffin2012} & 2012 & International Journal of Embedded and Real-Time Communication Systems & null & \noindent{}\textcolor{black!50}{0.00} \textbf{2.50} n/a & 0 0 0 & 25 33 & 2 0 2\\
\end{longtable}
}

\subsection{International Journal of Foundations of Computer Science}

\index{International Journal of Foundations of Computer Science}
{\scriptsize
\begin{longtable}{>{\raggedright\arraybackslash}p{2.5cm}>{\raggedright\arraybackslash}p{4.5cm}>{\raggedright\arraybackslash}p{6.0cm}p{1.0cm}rr>{\raggedright\arraybackslash}p{2.0cm}r>{\raggedright\arraybackslash}p{1cm}p{1cm}p{1cm}p{1cm}}
\rowcolor{white}\caption{Articles in Journal International Journal of Foundations of Computer Science (Total 1)}\\ \toprule
\rowcolor{white}\shortstack{Key\\Source} & Authors & Title (Colored by Open Access)& \shortstack{Details\\LC} & Cite & Year & \shortstack{Conference\\/Journal\\/School} & Pages & Relevance &\shortstack{Cites\\OC XR\\SC} & \shortstack{Refs\\OC\\XR} & \shortstack{Links\\Cites\\Refs}\\ \midrule\endhead
\bottomrule
\endfoot
Hofe2001 \href{http://dx.doi.org/10.1142/s0129054101000710}{Hofe2001} & \hyperref[auth:a2012]{H. M. A. Hofe} & Solving rostering tasks by generic methods for constraint optimization \hyperref[abs:Hofe2001]{Abstract} & \hyperref[detail:Hofe2001]{Details} No & \cite{Hofe2001} & 2001 & International Journal of Foundations of Computer Science & null & \noindent{}\textbf{1.00} \textbf{1.50} n/a & 3 3 6 & 3 4 & 1 0 1\\
\end{longtable}
}

\subsection{International Journal of Industrial Engineering: Theory, Applications and Practice}

\index{International Journal of Industrial Engineering: Theory, Applications and Practice}
{\scriptsize
\begin{longtable}{>{\raggedright\arraybackslash}p{2.5cm}>{\raggedright\arraybackslash}p{4.5cm}>{\raggedright\arraybackslash}p{6.0cm}p{1.0cm}rr>{\raggedright\arraybackslash}p{2.0cm}r>{\raggedright\arraybackslash}p{1cm}p{1cm}p{1cm}p{1cm}}
\rowcolor{white}\caption{Articles in Journal International Journal of Industrial Engineering: Theory, Applications and Practice (Total 1)}\\ \toprule
\rowcolor{white}\shortstack{Key\\Source} & Authors & Title (Colored by Open Access)& \shortstack{Details\\LC} & Cite & Year & \shortstack{Conference\\/Journal\\/School} & Pages & Relevance &\shortstack{Cites\\OC XR\\SC} & \shortstack{Refs\\OC\\XR} & \shortstack{Links\\Cites\\Refs}\\ \midrule\endhead
\bottomrule
\endfoot
OrnekO16 \href{https://journals.sfu.ca/ijietap/index.php/ijie/article/view/1930}{OrnekO16} & \hyperref[auth:a138]{A. {\"{O}}rnek}, \hyperref[auth:a135]{C. {\"{O}}zt{\"{u}}rk} & Optimisation and Constraint Based Heuristic Methods for Advanced Planning and Scheduling Systems & \hyperref[detail:OrnekO16]{Details} \href{../works/OrnekO16.pdf}{Yes} & \cite{OrnekO16} & 2016 & International Journal of Industrial Engineering: Theory, Applications and Practice & 25 & \noindent{}\textcolor{black!50}{0.00} \textcolor{black!50}{0.00} \textbf{21.72} & 0 0 0 & 0 0 & 0 0 0\\
\end{longtable}
}

\subsection{International Journal of Management Science and Engineering Management}

\index{International Journal of Management Science and Engineering Management}
{\scriptsize
\begin{longtable}{>{\raggedright\arraybackslash}p{2.5cm}>{\raggedright\arraybackslash}p{4.5cm}>{\raggedright\arraybackslash}p{6.0cm}p{1.0cm}rr>{\raggedright\arraybackslash}p{2.0cm}r>{\raggedright\arraybackslash}p{1cm}p{1cm}p{1cm}p{1cm}}
\rowcolor{white}\caption{Articles in Journal International Journal of Management Science and Engineering Management (Total 2)}\\ \toprule
\rowcolor{white}\shortstack{Key\\Source} & Authors & Title (Colored by Open Access)& \shortstack{Details\\LC} & Cite & Year & \shortstack{Conference\\/Journal\\/School} & Pages & Relevance &\shortstack{Cites\\OC XR\\SC} & \shortstack{Refs\\OC\\XR} & \shortstack{Links\\Cites\\Refs}\\ \midrule\endhead
\bottomrule
\endfoot
FarsiTM22 \href{https://api.semanticscholar.org/CorpusID:250301745}{FarsiTM22} & \hyperref[auth:a516]{A. Farsi}, \hyperref[auth:a739]{S. A. Torabi}, \hyperref[auth:a515]{M. Mokhtarzadeh} & Integrated surgery scheduling by constraint programming and meta-heuristics & \hyperref[detail:FarsiTM22]{Details} \href{../works/FarsiTM22.pdf}{Yes} & \cite{FarsiTM22} & 2022 & \cellcolor{red!20}International Journal of Management Science and Engineering Management & 14 & \noindent{}\textbf{1.00} \textbf{1.00} \textbf{4.92} & 5 5 8 & 47 50 & 6 1 5\\
RabbaniMM21 \href{http://dx.doi.org/10.1080/17509653.2021.1905096}{RabbaniMM21} & \hyperref[auth:a1246]{M. Rabbani}, \hyperref[auth:a515]{M. Mokhtarzadeh}, \hyperref[auth:a1247]{N. Manavizadeh} & A constraint programming approach and a hybrid of genetic and K-means algorithms to solve the p-hub location-allocation problems & \hyperref[detail:RabbaniMM21]{Details} No & \cite{RabbaniMM21} & 2021 & \cellcolor{red!20}International Journal of Management Science and Engineering Management & 11 & \noindent{}\textcolor{black!50}{0.00} \textcolor{black!50}{0.00} n/a & 4 4 9 & 44 46 & 9 1 8\\
\end{longtable}
}

\subsection{International Journal of Operations \& Production Management}

\index{International Journal of Operations \& Production Management}
{\scriptsize
\begin{longtable}{>{\raggedright\arraybackslash}p{2.5cm}>{\raggedright\arraybackslash}p{4.5cm}>{\raggedright\arraybackslash}p{6.0cm}p{1.0cm}rr>{\raggedright\arraybackslash}p{2.0cm}r>{\raggedright\arraybackslash}p{1cm}p{1cm}p{1cm}p{1cm}}
\rowcolor{white}\caption{Articles in Journal International Journal of Operations \  Production Management (Total 1)}\\ \toprule
\rowcolor{white}\shortstack{Key\\Source} & Authors & Title (Colored by Open Access)& \shortstack{Details\\LC} & Cite & Year & \shortstack{Conference\\/Journal\\/School} & Pages & Relevance &\shortstack{Cites\\OC XR\\SC} & \shortstack{Refs\\OC\\XR} & \shortstack{Links\\Cites\\Refs}\\ \midrule\endhead
\bottomrule
\endfoot
Icmeli1993 \href{http://dx.doi.org/10.1108/01443579310046454}{Icmeli1993} & \hyperref[auth:a1553]{O. Icmeli}, \hyperref[auth:a1554]{S. S. Erenguc}, \hyperref[auth:a1723]{C. J. Zappe} & Project Scheduling Problems: A Survey \hyperref[abs:Icmeli1993]{Abstract} & \hyperref[detail:Icmeli1993]{Details} No & \cite{Icmeli1993} & 1993 & International Journal of Operations \  Production Management & null & \noindent{}\textcolor{black!50}{0.00} \textcolor{black!50}{0.00} n/a & 97 99 0 & 39 56 & 5 3 2\\
\end{longtable}
}

\subsection{International Journal of Production Economics}

\index{International Journal of Production Economics}
{\scriptsize
\begin{longtable}{>{\raggedright\arraybackslash}p{2.5cm}>{\raggedright\arraybackslash}p{4.5cm}>{\raggedright\arraybackslash}p{6.0cm}p{1.0cm}rr>{\raggedright\arraybackslash}p{2.0cm}r>{\raggedright\arraybackslash}p{1cm}p{1cm}p{1cm}p{1cm}}
\rowcolor{white}\caption{Articles in Journal International Journal of Production Economics (Total 2)}\\ \toprule
\rowcolor{white}\shortstack{Key\\Source} & Authors & Title (Colored by Open Access)& \shortstack{Details\\LC} & Cite & Year & \shortstack{Conference\\/Journal\\/School} & Pages & Relevance &\shortstack{Cites\\OC XR\\SC} & \shortstack{Refs\\OC\\XR} & \shortstack{Links\\Cites\\Refs}\\ \midrule\endhead
\bottomrule
\endfoot
RoshanaeiBAUB20 \href{http://dx.doi.org/10.1016/j.ijpe.2019.07.006}{RoshanaeiBAUB20} & \hyperref[auth:a728]{V. Roshanaei}, \hyperref[auth:a203]{K. E. C. Booth}, \hyperref[auth:a895]{D. M. Aleman}, \hyperref[auth:a896]{D. R. Urbach}, \hyperref[auth:a89]{J. C. Beck} & Branch-and-check methods for multi-level operating room planning and scheduling & \hyperref[detail:RoshanaeiBAUB20]{Details} \href{../works/RoshanaeiBAUB20.pdf}{Yes} & \cite{RoshanaeiBAUB20} & 2020 & International Journal of Production Economics & 19 & \noindent{}\textcolor{black!50}{0.00} \textcolor{black!50}{0.00} \textbf{2.56} & 24 29 29 & 43 56 & 20 9 11\\
ArtiguesLH13 \href{http://dx.doi.org/10.1016/j.ijpe.2010.09.030}{ArtiguesLH13} & \hyperref[auth:a6]{C. Artigues}, \hyperref[auth:a3]{P. Lopez}, \hyperref[auth:a1162]{A. Haït} & \cellcolor{green!10}The energy scheduling problem: Industrial case-study and constraint propagation techniques & \hyperref[detail:ArtiguesLH13]{Details} \href{../works/ArtiguesLH13.pdf}{Yes} & \cite{ArtiguesLH13} & 2013 & International Journal of Production Economics & 11 & \noindent{}\textbf{1.50} \textbf{1.50} \textbf{5.79} & 76 82 83 & 16 25 & 11 6 5\\
\end{longtable}
}

\subsection{International Journal of Production Research}

\index{International Journal of Production Research}
{\scriptsize
\begin{longtable}{>{\raggedright\arraybackslash}p{2.5cm}>{\raggedright\arraybackslash}p{4.5cm}>{\raggedright\arraybackslash}p{6.0cm}p{1.0cm}rr>{\raggedright\arraybackslash}p{2.0cm}r>{\raggedright\arraybackslash}p{1cm}p{1cm}p{1cm}p{1cm}}
\rowcolor{white}\caption{Articles in Journal International Journal of Production Research (Total 13)}\\ \toprule
\rowcolor{white}\shortstack{Key\\Source} & Authors & Title (Colored by Open Access)& \shortstack{Details\\LC} & Cite & Year & \shortstack{Conference\\/Journal\\/School} & Pages & Relevance &\shortstack{Cites\\OC XR\\SC} & \shortstack{Refs\\OC\\XR} & \shortstack{Links\\Cites\\Refs}\\ \midrule\endhead
\bottomrule
\endfoot
AbreuNP23 \href{https://doi.org/10.1080/00207543.2022.2154404}{AbreuNP23} & \hyperref[auth:a418]{L. R. de Abreu}, \hyperref[auth:a419]{M. S. Nagano}, \hyperref[auth:a385]{B. A. Prata} & A new two-stage constraint programming approach for open shop scheduling problem with machine blocking & \hyperref[detail:AbreuNP23]{Details} \href{../works/AbreuNP23.pdf}{Yes} & \cite{AbreuNP23} & 2023 & \cellcolor{red!20}International Journal of Production Research & 20 & \noindent{}\textbf{1.50} \textbf{1.50} \textbf{44.59} & 1 2 0 & 47 54 & 12 1 11\\
Ahmadi-Javid2023 \href{http://dx.doi.org/10.1080/00207543.2023.2230489}{Ahmadi-Javid2023} & \hyperref[auth:a1762]{A. Ahmadi-Javid}, \hyperref[auth:a1763]{M. Haghi}, \hyperref[auth:a1764]{P. Hooshangi-Tabrizi} & Integrated job-shop scheduling in an FMS with heterogeneous transporters: MILP formulation, constraint programming, and branch-and-bound & \hyperref[detail:Ahmadi-Javid2023]{Details} No & \cite{Ahmadi-Javid2023} & 2023 & \cellcolor{red!20}International Journal of Production Research & null & \noindent{}\textbf{2.00} \textbf{2.00} n/a & 0 0 0 & 66 74 & 3 0 3\\
GunerGSKD23 \href{http://dx.doi.org/10.1080/00207543.2023.2226772}{GunerGSKD23} & \hyperref[auth:a1426]{F. G\"{u}ner}, \hyperref[auth:a1427]{A. K. G\"{o}r\"{u}r}, \hyperref[auth:a1428]{B. Satır}, \hyperref[auth:a1429]{L. Kandiller}, \hyperref[auth:a1430]{J. H. Drake} & A constraint programming approach to a real-world workforce scheduling problem for multi-manned assembly lines with sequence-dependent setup times & \hyperref[detail:GunerGSKD23]{Details} No & \cite{GunerGSKD23} & 2023 & \cellcolor{red!20}International Journal of Production Research & 18 & \noindent{}\textbf{1.00} \textbf{1.00} n/a & 0 3 0 & 43 46 & 7 0 7\\
NouriMHD23 \href{http://dx.doi.org/10.1080/00207543.2023.2173503}{NouriMHD23} & \hyperref[auth:a737]{B. Vahedi-Nouri}, \hyperref[auth:a430]{R. Tavakkoli-Moghaddam}, \hyperref[auth:a946]{Z. Hanzálek}, \hyperref[auth:a947]{A. Dolgui} & Production scheduling in a reconfigurable manufacturing system benefiting from human-robot collaboration & \hyperref[detail:NouriMHD23]{Details} No & \cite{NouriMHD23} & 2023 & \cellcolor{red!20}International Journal of Production Research & 17 & \noindent{}\textcolor{black!50}{0.00} \textcolor{black!50}{0.00} n/a & 2 6 5 & 44 49 & 11 0 11\\
BourreauGGLT22 \href{https://doi.org/10.1080/00207543.2020.1856436}{BourreauGGLT22} & \hyperref[auth:a441]{E. Bourreau}, \hyperref[auth:a442]{T. Garaix}, \hyperref[auth:a443]{M. Gondran}, \hyperref[auth:a444]{P. Lacomme}, \hyperref[auth:a445]{N. Tchernev} & \cellcolor{green!10}A constraint-programming based decomposition method for the Generalised Workforce Scheduling and Routing Problem {(GWSRP)} & \hyperref[detail:BourreauGGLT22]{Details} \href{../works/BourreauGGLT22.pdf}{Yes} & \cite{BourreauGGLT22} & 2022 & \cellcolor{red!20}International Journal of Production Research & 19 & \noindent{}\textcolor{black!50}{0.00} \textcolor{black!50}{0.00} \textbf{3.25} & 4 6 6 & 44 50 & 2 1 1\\
ShiYXQ22 \href{https://doi.org/10.1080/00207543.2021.1963496}{ShiYXQ22} & \hyperref[auth:a446]{G. Shi}, \hyperref[auth:a447]{Z. Yang}, \hyperref[auth:a448]{Y. Xu}, \hyperref[auth:a449]{Y. Quan} & Solving the integrated process planning and scheduling problem using an enhanced constraint programming-based approach & \hyperref[detail:ShiYXQ22]{Details} No & \cite{ShiYXQ22} & 2022 & \cellcolor{red!20}International Journal of Production Research & 18 & \noindent{}\textbf{1.00} \textbf{1.00} n/a & 2 3 3 & 45 53 & 5 1 4\\
YunusogluY22 \href{https://doi.org/10.1080/00207543.2021.1885068}{YunusogluY22} & \hyperref[auth:a450]{P. Yunusoglu}, \hyperref[auth:a421]{S. T. Yildiz} & Constraint programming approach for multi-resource-constrained unrelated parallel machine scheduling problem with sequence-dependent setup times & \hyperref[detail:YunusogluY22]{Details} \href{../works/YunusogluY22.pdf}{Yes} & \cite{YunusogluY22} & 2022 & \cellcolor{red!20}International Journal of Production Research & 18 & \noindent{}\textbf{2.00} \textbf{2.00} \textbf{69.33} & 20 36 40 & 58 64 & 16 6 10\\
CarlierSJP21 \href{http://dx.doi.org/10.1080/00207543.2021.1923853}{CarlierSJP21} & \hyperref[auth:a845]{J. Carlier}, \hyperref[auth:a928]{A. Sahli}, \hyperref[auth:a929]{A. Jouglet}, \hyperref[auth:a846]{E. Pinson} & A faster checker of the energetic reasoning for the cumulative scheduling problem & \hyperref[detail:CarlierSJP21]{Details} No & \cite{CarlierSJP21} & 2021 & \cellcolor{red!20}International Journal of Production Research & 16 & \noindent{}\textcolor{black!50}{0.00} \textcolor{black!50}{0.00} n/a & 3 6 4 & 26 29 & 12 2 10\\
Ham20 \href{http://dx.doi.org/10.1080/00207543.2019.1709671}{Ham20} & \hyperref[auth:a750]{A. Ham} & Transfer-robot task scheduling in job shop & \hyperref[detail:Ham20]{Details} No & \cite{Ham20} & 2020 & \cellcolor{red!20}International Journal of Production Research & 11 & \noindent{}\textcolor{black!50}{0.00} \textcolor{black!50}{0.00} n/a & 27 19 37 & 27 41 & 12 5 7\\
Polo-MejiaALB20 \href{https://doi.org/10.1080/00207543.2019.1693654}{Polo-MejiaALB20} & \hyperref[auth:a517]{O. Polo-Mej{\'{\i}}a}, \hyperref[auth:a6]{C. Artigues}, \hyperref[auth:a3]{P. Lopez}, \hyperref[auth:a518]{V. Basini} & \cellcolor{green!10}Mixed-integer/linear and constraint programming approaches for activity scheduling in a nuclear research facility & \hyperref[detail:Polo-MejiaALB20]{Details} \href{../works/Polo-MejiaALB20.pdf}{Yes} & \cite{Polo-MejiaALB20} & 2020 & \cellcolor{red!20}International Journal of Production Research & 18 & \noindent{}\textbf{1.50} \textbf{1.50} \textbf{14.35} & 8 10 11 & 23 36 & 8 2 6\\
WariZ19 \href{http://dx.doi.org/10.1080/00207543.2019.1571250}{WariZ19} & \hyperref[auth:a839]{E. Wari}, \hyperref[auth:a840]{W. Zhu} & A Constraint Programming model for food processing industry: a case for an ice cream processing facility & \hyperref[detail:WariZ19]{Details} No & \cite{WariZ19} & 2019 & \cellcolor{red!20}International Journal of Production Research & 17 & \noindent{}\textcolor{black!50}{0.00} \textcolor{black!50}{0.00} n/a & 11 11 12 & 42 54 & 18 3 15\\
GokgurHO18 \href{https://doi.org/10.1080/00207543.2017.1421781}{GokgurHO18} & \hyperref[auth:a569]{B. G{\"{o}}kg{\"{u}}r}, \hyperref[auth:a137]{B. Hnich}, \hyperref[auth:a570]{S. {\"{O}}zpeynirci} & Parallel machine scheduling with tool loading: a constraint programming approach & \hyperref[detail:GokgurHO18]{Details} \href{../works/GokgurHO18.pdf}{Yes} & \cite{GokgurHO18} & 2018 & \cellcolor{red!20}International Journal of Production Research & 17 & \noindent{}\textbf{1.50} \textbf{1.50} \textbf{73.63} & 31 40 51 & 43 62 & 23 8 15\\
Bourdeaudhuy2011 \href{http://dx.doi.org/10.1080/00207543.2010.519113}{Bourdeaudhuy2011} & \hyperref[auth:a1650]{T. Bourdeaud'huy}, \hyperref[auth:a1651]{O. Belkahla}, \hyperref[auth:a681]{P. Yim}, \hyperref[auth:a680]{O. Korbaa}, \hyperref[auth:a1652]{K. Ghedira} & Transient inter-production scheduling based on Petri nets and constraint programming & \hyperref[detail:Bourdeaudhuy2011]{Details} No & \cite{Bourdeaudhuy2011} & 2011 & \cellcolor{red!20}International Journal of Production Research & null & \noindent{}\textbf{1.00} \textbf{1.00} n/a & 5 5 6 & 12 32 & 2 1 1\\
\end{longtable}
}

\subsection{International Journal of Systems Science: Operations \& Logistics}

\index{International Journal of Systems Science: Operations \& Logistics}
{\scriptsize
\begin{longtable}{>{\raggedright\arraybackslash}p{2.5cm}>{\raggedright\arraybackslash}p{4.5cm}>{\raggedright\arraybackslash}p{6.0cm}p{1.0cm}rr>{\raggedright\arraybackslash}p{2.0cm}r>{\raggedright\arraybackslash}p{1cm}p{1cm}p{1cm}p{1cm}}
\rowcolor{white}\caption{Articles in Journal International Journal of Systems Science: Operations \  Logistics (Total 1)}\\ \toprule
\rowcolor{white}\shortstack{Key\\Source} & Authors & Title (Colored by Open Access)& \shortstack{Details\\LC} & Cite & Year & \shortstack{Conference\\/Journal\\/School} & Pages & Relevance &\shortstack{Cites\\OC XR\\SC} & \shortstack{Refs\\OC\\XR} & \shortstack{Links\\Cites\\Refs}\\ \midrule\endhead
\bottomrule
\endfoot
GhasemiMH23 \href{http://dx.doi.org/10.1080/23302674.2023.2224509}{GhasemiMH23} & \hyperref[auth:a981]{S. Ghasemi}, \hyperref[auth:a430]{R. Tavakkoli-Moghaddam}, \hyperref[auth:a982]{M. Hamid} & Operating room scheduling by emphasising human factors and dynamic decision-making styles: a constraint programming method & \hyperref[detail:GhasemiMH23]{Details} No & \cite{GhasemiMH23} & 2023 & \cellcolor{red!20}International Journal of Systems Science: Operations \  Logistics & null & \noindent{}\textbf{1.00} \textbf{1.00} n/a & 0 0 1 & 104 130 & 16 0 16\\
\end{longtable}
}

\subsection{International Journal on Artificial Intelligence Tools}

\index{International Journal on Artificial Intelligence Tools}
{\scriptsize
\begin{longtable}{>{\raggedright\arraybackslash}p{2.5cm}>{\raggedright\arraybackslash}p{4.5cm}>{\raggedright\arraybackslash}p{6.0cm}p{1.0cm}rr>{\raggedright\arraybackslash}p{2.0cm}r>{\raggedright\arraybackslash}p{1cm}p{1cm}p{1cm}p{1cm}}
\rowcolor{white}\caption{Articles in Journal International Journal on Artificial Intelligence Tools (Total 6)}\\ \toprule
\rowcolor{white}\shortstack{Key\\Source} & Authors & Title (Colored by Open Access)& \shortstack{Details\\LC} & Cite & Year & \shortstack{Conference\\/Journal\\/School} & Pages & Relevance &\shortstack{Cites\\OC XR\\SC} & \shortstack{Refs\\OC\\XR} & \shortstack{Links\\Cites\\Refs}\\ \midrule\endhead
\bottomrule
\endfoot
Dasygenis2018 \href{http://dx.doi.org/10.1142/s0218213018600023}{Dasygenis2018} & \hyperref[auth:a2000]{M. Dasygenis}, \hyperref[auth:a2001]{K. Stergiou} & Methods for Parallelizing Constraint Propagation through the Use of Strong Local Consistencies \hyperref[abs:Dasygenis2018]{Abstract} & \hyperref[detail:Dasygenis2018]{Details} No & \cite{Dasygenis2018} & 2018 & International Journal on Artificial Intelligence Tools & null & \noindent{}\textcolor{black!50}{0.00} \textbf{1.75} n/a & 1 1 2 & 12 30 & 2 0 2\\
Petrovic2008 \href{http://dx.doi.org/10.1142/s0218213008004023}{Petrovic2008} & \hyperref[auth:a1861]{S. Petrovic}, \hyperref[auth:a1862]{S. L. Epstein} & \cellcolor{green!10}Random subsets support learning a mixture of heuristics \hyperref[abs:Petrovic2008]{Abstract} & \hyperref[detail:Petrovic2008]{Details} No & \cite{Petrovic2008} & 2008 & International Journal on Artificial Intelligence Tools & null & \noindent{}\textcolor{black!50}{0.00} 0.50 n/a & 11 11 11 & 5 12 & 1 1 0\\
Wallace2008 \href{http://dx.doi.org/10.1142/s0218213008004199}{Wallace2008} & \hyperref[auth:a1268]{R. J. Wallace} & Determining the principles underlying performance variation in csp heuristics \hyperref[abs:Wallace2008]{Abstract} & \hyperref[detail:Wallace2008]{Details} No & \cite{Wallace2008} & 2008 & International Journal on Artificial Intelligence Tools & null & \noindent{}\textcolor{black!50}{0.00} 0.25 n/a & 3 3 4 & 3 8 & 1 1 0\\
Wu2008 \href{http://dx.doi.org/10.1142/s0218213008004187}{Wu2008} & \hyperref[auth:a2060]{H. Wu}, \hyperref[auth:a2061]{P. V. Beek} & Portfolios with deadlines for backtracking search \hyperref[abs:Wu2008]{Abstract} & \hyperref[detail:Wu2008]{Details} No & \cite{Wu2008} & 2008 & International Journal on Artificial Intelligence Tools & null & \noindent{}\textcolor{black!50}{0.00} \textbf{1.50} n/a & 5 5 6 & 13 14 & 2 1 1\\
Lallouet2007 \href{http://dx.doi.org/10.1142/s0218213007003503}{Lallouet2007} & \hyperref[auth:a428]{A. Lallouet}, \hyperref[auth:a1935]{A. Legtchenko} & Building consistencies for partially defined constraints with decision trees and neural networks \hyperref[abs:Lallouet2007]{Abstract} & \hyperref[detail:Lallouet2007]{Details} No & \cite{Lallouet2007} & 2007 & International Journal on Artificial Intelligence Tools & null & \noindent{}\textcolor{black!50}{0.00} 0.25 n/a & 3 3 6 & 4 12 & 2 1 1\\
Schiex1994 \href{http://dx.doi.org/10.1142/s0218213094000108}{Schiex1994} & \hyperref[auth:a1721]{T. Schiex}, \hyperref[auth:a1722]{G. Verfaillie} & Nogood recording for static and dynamic constraint satisfaction problems \hyperref[abs:Schiex1994]{Abstract} & \hyperref[detail:Schiex1994]{Details} No & \cite{Schiex1994} & 1994 & International Journal on Artificial Intelligence Tools & null & \noindent{}\textcolor{black!50}{0.00} \textbf{4.00} n/a & 65 66 0 & 0 0 & 2 2 0\\
\end{longtable}
}

\subsection{International Transactions in Operational Research}

\index{International Transactions in Operational Research}
{\scriptsize
\begin{longtable}{>{\raggedright\arraybackslash}p{2.5cm}>{\raggedright\arraybackslash}p{4.5cm}>{\raggedright\arraybackslash}p{6.0cm}p{1.0cm}rr>{\raggedright\arraybackslash}p{2.0cm}r>{\raggedright\arraybackslash}p{1cm}p{1cm}p{1cm}p{1cm}}
\rowcolor{white}\caption{Articles in Journal International Transactions in Operational Research (Total 1)}\\ \toprule
\rowcolor{white}\shortstack{Key\\Source} & Authors & Title (Colored by Open Access)& \shortstack{Details\\LC} & Cite & Year & \shortstack{Conference\\/Journal\\/School} & Pages & Relevance &\shortstack{Cites\\OC XR\\SC} & \shortstack{Refs\\OC\\XR} & \shortstack{Links\\Cites\\Refs}\\ \midrule\endhead
\bottomrule
\endfoot
Chakrabortty2019 \href{http://dx.doi.org/10.1111/itor.12644}{Chakrabortty2019} & \hyperref[auth:a1614]{R. K. Chakrabortty}, \hyperref[auth:a1615]{A. Abbasi}, \hyperref[auth:a1616]{M. J. Ryan} & \cellcolor{gold!20}Multi‐mode resource‐constrained project scheduling using modified variable neighborhood search heuristic \hyperref[abs:Chakrabortty2019]{Abstract} & \hyperref[detail:Chakrabortty2019]{Details} No & \cite{Chakrabortty2019} & 2019 & International Transactions in Operational Research & null & \noindent{}\textcolor{black!50}{0.00} \textcolor{black!50}{0.00} n/a & 29 37 41 & 81 85 & 10 3 7\\
\end{longtable}
}

\subsection{J. Artif. Intell. Res.}

\index{J. Artif. Intell. Res.}
{\scriptsize
\begin{longtable}{>{\raggedright\arraybackslash}p{2.5cm}>{\raggedright\arraybackslash}p{4.5cm}>{\raggedright\arraybackslash}p{6.0cm}p{1.0cm}rr>{\raggedright\arraybackslash}p{2.0cm}r>{\raggedright\arraybackslash}p{1cm}p{1cm}p{1cm}p{1cm}}
\rowcolor{white}\caption{Articles in Journal J. Artif. Intell. Res. (Total 6)}\\ \toprule
\rowcolor{white}\shortstack{Key\\Source} & Authors & Title (Colored by Open Access)& \shortstack{Details\\LC} & Cite & Year & \shortstack{Conference\\/Journal\\/School} & Pages & Relevance &\shortstack{Cites\\OC XR\\SC} & \shortstack{Refs\\OC\\XR} & \shortstack{Links\\Cites\\Refs}\\ \midrule\endhead
\bottomrule
\endfoot
GoldwaserS18 \href{https://doi.org/10.1613/jair.1.11268}{GoldwaserS18} & \hyperref[auth:a189]{A. Goldwaser}, \hyperref[auth:a124]{A. Schutt} & \cellcolor{gold!20}Optimal Torpedo Scheduling & \hyperref[detail:GoldwaserS18]{Details} \href{../works/GoldwaserS18.pdf}{Yes} & \cite{GoldwaserS18} & 2018 & J. Artif. Intell. Res. & 32 & \noindent{}\textcolor{black!50}{0.00} \textcolor{black!50}{0.00} \textbf{3.31} & 8 8 9 & 0 0 & 1 1 0\\
TranVNB17 \href{https://doi.org/10.1613/jair.5306}{TranVNB17} & \hyperref[auth:a799]{T. T. Tran}, \hyperref[auth:a804]{T. S. Vaquero}, \hyperref[auth:a204]{G. Nejat}, \hyperref[auth:a89]{J. C. Beck} & \cellcolor{gold!20}Robots in Retirement Homes: Applying Off-the-Shelf Planning and Scheduling to a Team of Assistive Robots & \hyperref[detail:TranVNB17]{Details} \href{../works/TranVNB17.pdf}{Yes} & \cite{TranVNB17} & 2017 & J. Artif. Intell. Res. & 68 & \noindent{}\textcolor{black!50}{0.00} \textcolor{black!50}{0.00} \textbf{61.11} & 12 12 21 & 0 0 & 2 2 0\\
TerekhovTDB14 \href{https://doi.org/10.1613/jair.4278}{TerekhovTDB14} & \hyperref[auth:a818]{D. Terekhov}, \hyperref[auth:a799]{T. T. Tran}, \hyperref[auth:a803]{D. G. Down}, \hyperref[auth:a89]{J. C. Beck} & \cellcolor{gold!20}Integrating Queueing Theory and Scheduling for Dynamic Scheduling Problems & \hyperref[detail:TerekhovTDB14]{Details} \href{../works/TerekhovTDB14.pdf}{Yes} & \cite{TerekhovTDB14} & 2014 & J. Artif. Intell. Res. & 38 & \noindent{}\textcolor{black!50}{0.00} \textcolor{black!50}{0.00} \textbf{2.01} & 12 13 17 & 0 0 & 1 1 0\\
BajestaniB13 \href{https://doi.org/10.1613/jair.3902}{BajestaniB13} & \hyperref[auth:a817]{M. A. Bajestani}, \hyperref[auth:a89]{J. C. Beck} & \cellcolor{gold!20}Scheduling a Dynamic Aircraft Repair Shop with Limited Repair Resources & \hyperref[detail:BajestaniB13]{Details} \href{../works/BajestaniB13.pdf}{Yes} & \cite{BajestaniB13} & 2013 & J. Artif. Intell. Res. & 36 & \noindent{}\textcolor{black!50}{0.00} \textcolor{black!50}{0.00} \textbf{33.71} & 14 15 20 & 0 0 & 4 4 0\\
Beck07 \href{https://doi.org/10.1613/jair.2169}{Beck07} & \hyperref[auth:a89]{J. C. Beck} & \cellcolor{gold!20}Solution-Guided Multi-Point Constructive Search for Job Shop Scheduling & \hyperref[detail:Beck07]{Details} \href{../works/Beck07.pdf}{Yes} & \cite{Beck07} & 2007 & J. Artif. Intell. Res. & 29 & \noindent{}\textcolor{black!50}{0.00} \textcolor{black!50}{0.00} \textbf{2.01} & 34 34 57 & 0 0 & 15 15 0\\
BeckW07 \href{https://doi.org/10.1613/jair.2080}{BeckW07} & \hyperref[auth:a89]{J. C. Beck}, \hyperref[auth:a826]{N. Wilson} & \cellcolor{gold!20}Proactive Algorithms for Job Shop Scheduling with Probabilistic Durations & \hyperref[detail:BeckW07]{Details} \href{../works/BeckW07.pdf}{Yes} & \cite{BeckW07} & 2007 & J. Artif. Intell. Res. & 50 & \noindent{}\textcolor{black!50}{0.00} \textcolor{black!50}{0.00} \textbf{8.34} & 27 31 61 & 0 0 & 4 4 0\\
\end{longtable}
}

\subsection{J. Heuristics}

\index{J. Heuristics}
{\scriptsize
\begin{longtable}{>{\raggedright\arraybackslash}p{2.5cm}>{\raggedright\arraybackslash}p{4.5cm}>{\raggedright\arraybackslash}p{6.0cm}p{1.0cm}rr>{\raggedright\arraybackslash}p{2.0cm}r>{\raggedright\arraybackslash}p{1cm}p{1cm}p{1cm}p{1cm}}
\rowcolor{white}\caption{Articles in Journal J. Heuristics (Total 2)}\\ \toprule
\rowcolor{white}\shortstack{Key\\Source} & Authors & Title (Colored by Open Access)& \shortstack{Details\\LC} & Cite & Year & \shortstack{Conference\\/Journal\\/School} & Pages & Relevance &\shortstack{Cites\\OC XR\\SC} & \shortstack{Refs\\OC\\XR} & \shortstack{Links\\Cites\\Refs}\\ \midrule\endhead
\bottomrule
\endfoot
ThiruvadyWGS14 \href{https://doi.org/10.1007/s10732-014-9260-3}{ThiruvadyWGS14} & \hyperref[auth:a396]{D. R. Thiruvady}, \hyperref[auth:a117]{M. G. Wallace}, \hyperref[auth:a336]{H. Gu}, \hyperref[auth:a124]{A. Schutt} & \cellcolor{green!10}A Lagrangian relaxation and {ACO} hybrid for resource constrained project scheduling with discounted cash flows & \hyperref[detail:ThiruvadyWGS14]{Details} \href{../works/ThiruvadyWGS14.pdf}{Yes} & \cite{ThiruvadyWGS14} & 2014 & J. Heuristics & 34 & \noindent{}\textcolor{black!50}{0.00} \textcolor{black!50}{0.00} \textbf{1.79} & 19 20 19 & 18 24 & 4 1 3\\
NuijtenP98 \href{https://doi.org/10.1023/A:1009687210594}{NuijtenP98} & \hyperref[auth:a656]{W. Nuijten}, \hyperref[auth:a163]{C. L. Pape} & Constraint-Based Job Shop Scheduling with {\textbackslash}sc Ilog Scheduler & \hyperref[detail:NuijtenP98]{Details} \href{../works/NuijtenP98.pdf}{Yes} & \cite{NuijtenP98} & 1998 & J. Heuristics & 16 & \noindent{}\textcolor{black!50}{0.00} \textcolor{black!50}{0.00} \textbf{10.57} & 42 0 50 & 0 0 & 24 24 0\\
\end{longtable}
}

\subsection{J. Supercomput.}

\index{J. Supercomput.}
{\scriptsize
\begin{longtable}{>{\raggedright\arraybackslash}p{2.5cm}>{\raggedright\arraybackslash}p{4.5cm}>{\raggedright\arraybackslash}p{6.0cm}p{1.0cm}rr>{\raggedright\arraybackslash}p{2.0cm}r>{\raggedright\arraybackslash}p{1cm}p{1cm}p{1cm}p{1cm}}
\rowcolor{white}\caption{Articles in Journal J. Supercomput. (Total 1)}\\ \toprule
\rowcolor{white}\shortstack{Key\\Source} & Authors & Title (Colored by Open Access)& \shortstack{Details\\LC} & Cite & Year & \shortstack{Conference\\/Journal\\/School} & Pages & Relevance &\shortstack{Cites\\OC XR\\SC} & \shortstack{Refs\\OC\\XR} & \shortstack{Links\\Cites\\Refs}\\ \midrule\endhead
\bottomrule
\endfoot
PandeyS21a \href{https://doi.org/10.1007/s11227-020-03516-3}{PandeyS21a} & \hyperref[auth:a491]{V. Pandey}, \hyperref[auth:a492]{P. Saini} & Constraint programming versus heuristic approach to MapReduce scheduling problem in Hadoop {YARN} for energy minimization & \hyperref[detail:PandeyS21a]{Details} \href{../works/PandeyS21a.pdf}{Yes} & \cite{PandeyS21a} & 2021 & J. Supercomput. & 29 & \noindent{}\textbf{1.00} \textbf{1.00} \textbf{47.79} & 3 3 3 & 32 41 & 8 0 8\\
\end{longtable}
}

\subsection{J. Syst. Archit.}

\index{J. Syst. Archit.}
{\scriptsize
\begin{longtable}{>{\raggedright\arraybackslash}p{2.5cm}>{\raggedright\arraybackslash}p{4.5cm}>{\raggedright\arraybackslash}p{6.0cm}p{1.0cm}rr>{\raggedright\arraybackslash}p{2.0cm}r>{\raggedright\arraybackslash}p{1cm}p{1cm}p{1cm}p{1cm}}
\rowcolor{white}\caption{Articles in Journal J. Syst. Archit. (Total 1)}\\ \toprule
\rowcolor{white}\shortstack{Key\\Source} & Authors & Title (Colored by Open Access)& \shortstack{Details\\LC} & Cite & Year & \shortstack{Conference\\/Journal\\/School} & Pages & Relevance &\shortstack{Cites\\OC XR\\SC} & \shortstack{Refs\\OC\\XR} & \shortstack{Links\\Cites\\Refs}\\ \midrule\endhead
\bottomrule
\endfoot
KuchcinskiW03 \href{https://doi.org/10.1016/S1383-7621(03)00075-4}{KuchcinskiW03} & \hyperref[auth:a660]{K. Kuchcinski}, \hyperref[auth:a659]{C. Wolinski} & Global approach to assignment and scheduling of complex behaviors based on {HCDG} and constraint programming & \hyperref[detail:KuchcinskiW03]{Details} \href{../works/KuchcinskiW03.pdf}{Yes} & \cite{KuchcinskiW03} & 2003 & J. Syst. Archit. & 15 & \noindent{}\textbf{1.00} \textbf{1.00} \textbf{1.08} & 19 19 22 & 18 23 & 7 6 1\\
\end{longtable}
}

\subsection{JOURNAL OF CIVIL ENGINEERING AND MANAGEMENT}

\index{JOURNAL OF CIVIL ENGINEERING AND MANAGEMENT}
{\scriptsize
\begin{longtable}{>{\raggedright\arraybackslash}p{2.5cm}>{\raggedright\arraybackslash}p{4.5cm}>{\raggedright\arraybackslash}p{6.0cm}p{1.0cm}rr>{\raggedright\arraybackslash}p{2.0cm}r>{\raggedright\arraybackslash}p{1cm}p{1cm}p{1cm}p{1cm}}
\rowcolor{white}\caption{Articles in Journal JOURNAL OF CIVIL ENGINEERING AND MANAGEMENT (Total 1)}\\ \toprule
\rowcolor{white}\shortstack{Key\\Source} & Authors & Title (Colored by Open Access)& \shortstack{Details\\LC} & Cite & Year & \shortstack{Conference\\/Journal\\/School} & Pages & Relevance &\shortstack{Cites\\OC XR\\SC} & \shortstack{Refs\\OC\\XR} & \shortstack{Links\\Cites\\Refs}\\ \midrule\endhead
\bottomrule
\endfoot
Tomczak2022 \href{http://dx.doi.org/10.3846/jcem.2022.16943}{Tomczak2022} & \hyperref[auth:a1768]{M. Tomczak}, \hyperref[auth:a1769]{P. Jaśkowski} & \cellcolor{gold!20}Scheduling repetitive construction projects: structured literature review \hyperref[abs:Tomczak2022]{Abstract} & \hyperref[detail:Tomczak2022]{Details} No & \cite{Tomczak2022} & 2022 & JOURNAL OF CIVIL ENGINEERING AND MANAGEMENT & null & \noindent{}\textcolor{black!50}{0.00} \textbf{1.00} n/a & 3 5 5 & 191 197 & 6 0 6\\
\end{longtable}
}

\subsection{Journal Europ{\'e}en des Syst{\`e}mes Automatis{\'e}s}

\index{Journal Europ{\'e}en des Syst{\`e}mes Automatis{\'e}s}
{\scriptsize
\begin{longtable}{>{\raggedright\arraybackslash}p{2.5cm}>{\raggedright\arraybackslash}p{4.5cm}>{\raggedright\arraybackslash}p{6.0cm}p{1.0cm}rr>{\raggedright\arraybackslash}p{2.0cm}r>{\raggedright\arraybackslash}p{1cm}p{1cm}p{1cm}p{1cm}}
\rowcolor{white}\caption{Articles in Journal Journal Europ{\'e}en des Syst{\`e}mes Automatis{\'e}s (Total 1)}\\ \toprule
\rowcolor{white}\shortstack{Key\\Source} & Authors & Title (Colored by Open Access)& \shortstack{Details\\LC} & Cite & Year & \shortstack{Conference\\/Journal\\/School} & Pages & Relevance &\shortstack{Cites\\OC XR\\SC} & \shortstack{Refs\\OC\\XR} & \shortstack{Links\\Cites\\Refs}\\ \midrule\endhead
\bottomrule
\endfoot
Bedhief21 \href{https://api.semanticscholar.org/CorpusID:240611192}{Bedhief21} & \hyperref[auth:a746]{A. O. Bedhief} & \cellcolor{gold!20}Comparing Mixed-Integer Programming and Constraint Programming Models for the Hybrid Flow Shop Scheduling Problem with Dedicated Machines & \hyperref[detail:Bedhief21]{Details} \href{../works/Bedhief21.pdf}{Yes} & \cite{Bedhief21} & 2021 & Journal Europ{\'e}en des Syst{\`e}mes Automatis{\'e}s & 7 & \noindent{}\textbf{1.50} \textbf{1.50} \textbf{8.02} & 0 0 2 & 0 0 & 0 0 0\\
\end{longtable}
}

\subsection{Journal of Advanced Transportation}

\index{Journal of Advanced Transportation}
{\scriptsize
\begin{longtable}{>{\raggedright\arraybackslash}p{2.5cm}>{\raggedright\arraybackslash}p{4.5cm}>{\raggedright\arraybackslash}p{6.0cm}p{1.0cm}rr>{\raggedright\arraybackslash}p{2.0cm}r>{\raggedright\arraybackslash}p{1cm}p{1cm}p{1cm}p{1cm}}
\rowcolor{white}\caption{Articles in Journal Journal of Advanced Transportation (Total 1)}\\ \toprule
\rowcolor{white}\shortstack{Key\\Source} & Authors & Title (Colored by Open Access)& \shortstack{Details\\LC} & Cite & Year & \shortstack{Conference\\/Journal\\/School} & Pages & Relevance &\shortstack{Cites\\OC XR\\SC} & \shortstack{Refs\\OC\\XR} & \shortstack{Links\\Cites\\Refs}\\ \midrule\endhead
\bottomrule
\endfoot
Wang2021 \href{http://dx.doi.org/10.1155/2021/5531063}{Wang2021} & \hyperref[auth:a1968]{L. Wang}, \hyperref[auth:a1969]{W. Ma}, \hyperref[auth:a1970]{L. Wang}, \hyperref[auth:a1971]{Y. Ren}, \hyperref[auth:a1972]{C. Yu} & \cellcolor{gold!20}Enabling In-Depot Automated Routing and Recharging Scheduling for Automated Electric Bus Transit Systems \hyperref[abs:Wang2021]{Abstract} & \hyperref[detail:Wang2021]{Details} No & \cite{Wang2021} & 2021 & Journal of Advanced Transportation & null & \noindent{}\textcolor{black!50}{0.00} \textbf{5.00} n/a & 1 4 3 & 34 39 & 3 0 3\\
\end{longtable}
}

\subsection{Journal of Applied Mathematics}

\index{Journal of Applied Mathematics}
{\scriptsize
\begin{longtable}{>{\raggedright\arraybackslash}p{2.5cm}>{\raggedright\arraybackslash}p{4.5cm}>{\raggedright\arraybackslash}p{6.0cm}p{1.0cm}rr>{\raggedright\arraybackslash}p{2.0cm}r>{\raggedright\arraybackslash}p{1cm}p{1cm}p{1cm}p{1cm}}
\rowcolor{white}\caption{Articles in Journal Journal of Applied Mathematics (Total 1)}\\ \toprule
\rowcolor{white}\shortstack{Key\\Source} & Authors & Title (Colored by Open Access)& \shortstack{Details\\LC} & Cite & Year & \shortstack{Conference\\/Journal\\/School} & Pages & Relevance &\shortstack{Cites\\OC XR\\SC} & \shortstack{Refs\\OC\\XR} & \shortstack{Links\\Cites\\Refs}\\ \midrule\endhead
\bottomrule
\endfoot
GomesM17 \href{http://dx.doi.org/10.1155/2017/9452762}{GomesM17} & \hyperref[auth:a965]{F. R. A. Gomes}, \hyperref[auth:a966]{G. R. Mateus} & \cellcolor{gold!20}Improved Combinatorial Benders Decomposition for a Scheduling Problem with Unrelated Parallel Machines & \hyperref[detail:GomesM17]{Details} \href{../works/GomesM17.pdf}{Yes} & \cite{GomesM17} & 2017 & Journal of Applied Mathematics & 10 & \noindent{}\textcolor{black!50}{0.00} \textcolor{black!50}{0.00} 0.31 & 1 1 3 & 43 44 & 9 0 9\\
\end{longtable}
}

\subsection{Journal of Artificial Intelligence Research}

\index{Journal of Artificial Intelligence Research}
{\scriptsize
\begin{longtable}{>{\raggedright\arraybackslash}p{2.5cm}>{\raggedright\arraybackslash}p{4.5cm}>{\raggedright\arraybackslash}p{6.0cm}p{1.0cm}rr>{\raggedright\arraybackslash}p{2.0cm}r>{\raggedright\arraybackslash}p{1cm}p{1cm}p{1cm}p{1cm}}
\rowcolor{white}\caption{Articles in Journal Journal of Artificial Intelligence Research (Total 6)}\\ \toprule
\rowcolor{white}\shortstack{Key\\Source} & Authors & Title (Colored by Open Access)& \shortstack{Details\\LC} & Cite & Year & \shortstack{Conference\\/Journal\\/School} & Pages & Relevance &\shortstack{Cites\\OC XR\\SC} & \shortstack{Refs\\OC\\XR} & \shortstack{Links\\Cites\\Refs}\\ \midrule\endhead
\bottomrule
\endfoot
Geiger2019 \href{http://dx.doi.org/10.1613/jair.1.11303}{Geiger2019} & \hyperref[auth:a1829]{M. J. Geiger}, \hyperref[auth:a78]{L. Kletzander}, \hyperref[auth:a45]{N. Musliu} & \cellcolor{gold!20}Solving the Torpedo Scheduling Problem \hyperref[abs:Geiger2019]{Abstract} & \hyperref[detail:Geiger2019]{Details} No & \cite{Geiger2019} & 2019 & Journal of Artificial Intelligence Research & null & \noindent{}\textcolor{black!50}{0.00} \textbf{1.50} n/a & 4 6 6 & 0 0 & 1 1 0\\
Lindauer2015 \href{http://dx.doi.org/10.1613/jair.4726}{Lindauer2015} & \hyperref[auth:a1942]{M. Lindauer}, \hyperref[auth:a1943]{H. H. Hoos}, \hyperref[auth:a1944]{F. Hutter}, \hyperref[auth:a1945]{T. Schaub} & \cellcolor{gold!20}AutoFolio: An Automatically Configured Algorithm Selector \hyperref[abs:Lindauer2015]{Abstract} & \hyperref[detail:Lindauer2015]{Details} No & \cite{Lindauer2015} & 2015 & Journal of Artificial Intelligence Research & null & \noindent{}\textcolor{black!50}{0.00} 0.50 n/a & 53 58 84 & 0 0 & 1 1 0\\
Bergman2014 \href{http://dx.doi.org/10.1613/jair.4199}{Bergman2014} & \hyperref[auth:a1514]{D. Bergman}, \hyperref[auth:a1515]{A. A. Cire}, \hyperref[auth:a1516]{W. V. Hoeve} & \cellcolor{gold!20}MDD Propagation for Sequence Constraints \hyperref[abs:Bergman2014]{Abstract} & \hyperref[detail:Bergman2014]{Details} No & \cite{Bergman2014} & 2014 & Journal of Artificial Intelligence Research & null & \noindent{}\textcolor{black!50}{0.00} \textbf{2.25} n/a & 14 14 19 & 0 0 & 3 3 0\\
Pesant2012 \href{http://dx.doi.org/10.1613/jair.3463}{Pesant2012} & \hyperref[auth:a1586]{G. Pesant}, \hyperref[auth:a1587]{C. Quimper}, \hyperref[auth:a1588]{A. Zanarini} & \cellcolor{gold!20}Counting-Based Search: Branching Heuristics for Constraint Satisfaction Problems \hyperref[abs:Pesant2012]{Abstract} & \hyperref[detail:Pesant2012]{Details} No & \cite{Pesant2012} & 2012 & Journal of Artificial Intelligence Research & null & \noindent{}\textcolor{black!50}{0.00} \textbf{2.00} n/a & 32 32 51 & 0 0 & 6 6 0\\
Younes2003 \href{http://dx.doi.org/10.1613/jair.1136}{Younes2003} & \hyperref[auth:a1844]{H. L. S. Younes}, \hyperref[auth:a1845]{R. G. Simmons} & \cellcolor{gold!20}VHPOP: Versatile Heuristic Partial Order Planner \hyperref[abs:Younes2003]{Abstract} & \hyperref[detail:Younes2003]{Details} No & \cite{Younes2003} & 2003 & Journal of Artificial Intelligence Research & null & \noindent{}\textcolor{black!50}{0.00} 0.50 n/a & 54 55 128 & 0 0 & 1 1 0\\
Kambhampati2000 \href{http://dx.doi.org/10.1613/jair.655}{Kambhampati2000} & \hyperref[auth:a1915]{S. Kambhampati} & \cellcolor{gold!20}Planning Graph as a (Dynamic) CSP: Exploiting EBL, DDB and other CSP Search Techniques in Graphplan \hyperref[abs:Kambhampati2000]{Abstract} & \hyperref[detail:Kambhampati2000]{Details} No & \cite{Kambhampati2000} & 2000 & Journal of Artificial Intelligence Research & null & \noindent{}\textcolor{black!50}{0.00} \textbf{1.00} n/a & 31 35 58 & 0 0 & 1 1 0\\
\end{longtable}
}

\subsection{Journal of Computer Science and Technology}

\index{Journal of Computer Science and Technology}
{\scriptsize
\begin{longtable}{>{\raggedright\arraybackslash}p{2.5cm}>{\raggedright\arraybackslash}p{4.5cm}>{\raggedright\arraybackslash}p{6.0cm}p{1.0cm}rr>{\raggedright\arraybackslash}p{2.0cm}r>{\raggedright\arraybackslash}p{1cm}p{1cm}p{1cm}p{1cm}}
\rowcolor{white}\caption{Articles in Journal Journal of Computer Science and Technology (Total 1)}\\ \toprule
\rowcolor{white}\shortstack{Key\\Source} & Authors & Title (Colored by Open Access)& \shortstack{Details\\LC} & Cite & Year & \shortstack{Conference\\/Journal\\/School} & Pages & Relevance &\shortstack{Cites\\OC XR\\SC} & \shortstack{Refs\\OC\\XR} & \shortstack{Links\\Cites\\Refs}\\ \midrule\endhead
\bottomrule
\endfoot
Yan2003 \href{http://dx.doi.org/10.1007/bf02948893}{Yan2003} & \hyperref[auth:a2033]{J. Yan}, \hyperref[auth:a2034]{C. Wu} & A constraint satisfaction neural network and heuristic combined approach for concurrent activities scheduling & \hyperref[detail:Yan2003]{Details} No & \cite{Yan2003} & 2003 & Journal of Computer Science and Technology & null & \noindent{}\textbf{1.00} \textbf{1.00} n/a & 3 3 1 & 7 12 & 1 0 1\\
\end{longtable}
}

\subsection{Journal of Construction Engineering and Management}

\index{Journal of Construction Engineering and Management}
{\scriptsize
\begin{longtable}{>{\raggedright\arraybackslash}p{2.5cm}>{\raggedright\arraybackslash}p{4.5cm}>{\raggedright\arraybackslash}p{6.0cm}p{1.0cm}rr>{\raggedright\arraybackslash}p{2.0cm}r>{\raggedright\arraybackslash}p{1cm}p{1cm}p{1cm}p{1cm}}
\rowcolor{white}\caption{Articles in Journal Journal of Construction Engineering and Management (Total 5)}\\ \toprule
\rowcolor{white}\shortstack{Key\\Source} & Authors & Title (Colored by Open Access)& \shortstack{Details\\LC} & Cite & Year & \shortstack{Conference\\/Journal\\/School} & Pages & Relevance &\shortstack{Cites\\OC XR\\SC} & \shortstack{Refs\\OC\\XR} & \shortstack{Links\\Cites\\Refs}\\ \midrule\endhead
\bottomrule
\endfoot
Kong2021 \href{http://dx.doi.org/10.1061/(asce)co.1943-7862.0002192}{Kong2021} & \hyperref[auth:a1706]{F. Kong}, \hyperref[auth:a1707]{J. Guo}, \hyperref[auth:a1708]{X. Lv} & Project Resource Input Optimization Problem with Combined Time Constraints Based on Node Network Diagram and Constraint Programming & \hyperref[detail:Kong2021]{Details} No & \cite{Kong2021} & 2021 & Journal of Construction Engineering and Management & null & \noindent{}0.50 0.50 n/a & 1 1 1 & 31 32 & 6 0 6\\
Kong2020 \href{http://dx.doi.org/10.1061/(asce)co.1943-7862.0001929}{Kong2020} & \hyperref[auth:a1706]{F. Kong}, \hyperref[auth:a1780]{D. Dou} & RCPSP with Combined Precedence Relations and Resource Calendars & \hyperref[detail:Kong2020]{Details} No & \cite{Kong2020} & 2020 & Journal of Construction Engineering and Management & null & \noindent{}\textcolor{black!50}{0.00} \textcolor{black!50}{0.00} n/a & 5 6 6 & 39 44 & 5 1 4\\
Ammar2013 \href{http://dx.doi.org/10.1061/(asce)co.1943-7862.0000569}{Ammar2013} & \hyperref[auth:a1779]{M. A. Ammar} & LOB and CPM Integrated Method for Scheduling Repetitive Projects & \hyperref[detail:Ammar2013]{Details} No & \cite{Ammar2013} & 2013 & Journal of Construction Engineering and Management & null & \noindent{}\textcolor{black!50}{0.00} \textcolor{black!50}{0.00} n/a & 59 60 65 & 20 22 & 5 5 0\\
Lorterapong2013 \href{http://dx.doi.org/10.1061/(asce)co.1943-7862.0000582}{Lorterapong2013} & \hyperref[auth:a1792]{P. Lorterapong}, \hyperref[auth:a1793]{M. Ussavadilokrit} & Construction Scheduling Using the Constraint Satisfaction Problem Method & \hyperref[detail:Lorterapong2013]{Details} No & \cite{Lorterapong2013} & 2013 & Journal of Construction Engineering and Management & null & \noindent{}\textbf{1.00} \textbf{1.00} n/a & 14 15 16 & 18 25 & 3 0 3\\
Chan2002 \href{http://dx.doi.org/10.1061/(asce)0733-9364(2002)128:6(513)}{Chan2002} & \hyperref[auth:a1662]{W. T. Chan}, \hyperref[auth:a1663]{H. Hu} & Constraint Programming Approach to Precast Production Scheduling & \hyperref[detail:Chan2002]{Details} No & \cite{Chan2002} & 2002 & Journal of Construction Engineering and Management & null & \noindent{}\textbf{1.00} \textbf{1.00} n/a & 65 69 86 & 9 19 & 10 8 2\\
\end{longtable}
}

\subsection{Journal of Decision Systems}

\index{Journal of Decision Systems}
{\scriptsize
\begin{longtable}{>{\raggedright\arraybackslash}p{2.5cm}>{\raggedright\arraybackslash}p{4.5cm}>{\raggedright\arraybackslash}p{6.0cm}p{1.0cm}rr>{\raggedright\arraybackslash}p{2.0cm}r>{\raggedright\arraybackslash}p{1cm}p{1cm}p{1cm}p{1cm}}
\rowcolor{white}\caption{Articles in Journal Journal of Decision Systems (Total 1)}\\ \toprule
\rowcolor{white}\shortstack{Key\\Source} & Authors & Title (Colored by Open Access)& \shortstack{Details\\LC} & Cite & Year & \shortstack{Conference\\/Journal\\/School} & Pages & Relevance &\shortstack{Cites\\OC XR\\SC} & \shortstack{Refs\\OC\\XR} & \shortstack{Links\\Cites\\Refs}\\ \midrule\endhead
\bottomrule
\endfoot
Paredis1992 \href{http://dx.doi.org/10.1080/12460125.1992.10511509}{Paredis1992} & \hyperref[auth:a1998]{J. Paredis}, \hyperref[auth:a1999]{T. van Rij} & Simulation and Constraint Programming as Support Methodologies for Job Shop Scheduling & \hyperref[detail:Paredis1992]{Details} No & \cite{Paredis1992} & 1992 & Journal of Decision Systems & null & \noindent{}\textbf{2.00} \textbf{2.00} n/a & 0 2 5 & 3 15 & 1 0 1\\
\end{longtable}
}

\subsection{Journal of Heuristics}

\index{Journal of Heuristics}
{\scriptsize
\begin{longtable}{>{\raggedright\arraybackslash}p{2.5cm}>{\raggedright\arraybackslash}p{4.5cm}>{\raggedright\arraybackslash}p{6.0cm}p{1.0cm}rr>{\raggedright\arraybackslash}p{2.0cm}r>{\raggedright\arraybackslash}p{1cm}p{1cm}p{1cm}p{1cm}}
\rowcolor{white}\caption{Articles in Journal Journal of Heuristics (Total 2)}\\ \toprule
\rowcolor{white}\shortstack{Key\\Source} & Authors & Title (Colored by Open Access)& \shortstack{Details\\LC} & Cite & Year & \shortstack{Conference\\/Journal\\/School} & Pages & Relevance &\shortstack{Cites\\OC XR\\SC} & \shortstack{Refs\\OC\\XR} & \shortstack{Links\\Cites\\Refs}\\ \midrule\endhead
\bottomrule
\endfoot
Li2020 \href{http://dx.doi.org/10.1007/s10732-019-09434-9}{Li2020} & \hyperref[auth:a1796]{H. Li}, \hyperref[auth:a1811]{G. Feng}, \hyperref[auth:a1812]{M. Yin} & On combining variable ordering heuristics for constraint satisfaction problems & \hyperref[detail:Li2020]{Details} No & \cite{Li2020} & 2020 & Journal of Heuristics & null & \noindent{}0.50 0.50 n/a & 2 2 3 & 28 36 & 7 1 6\\
Li2015 \href{http://dx.doi.org/10.1007/s10732-015-9305-2}{Li2015} & \hyperref[auth:a1796]{H. Li}, \hyperref[auth:a1797]{Y. Liang}, \hyperref[auth:a1798]{N. Zhang}, \hyperref[auth:a1799]{J. Guo}, \hyperref[auth:a1800]{D. Xu}, \hyperref[auth:a1801]{Z. Li} & Improving degree-based variable ordering heuristics for solving constraint satisfaction problems & \hyperref[detail:Li2015]{Details} No & \cite{Li2015} & 2015 & Journal of Heuristics & null & \noindent{}0.50 0.50 n/a & 6 6 12 & 14 25 & 7 3 4\\
\end{longtable}
}

\subsection{Journal of Intelligent Manufacturing}

\index{Journal of Intelligent Manufacturing}
{\scriptsize
\begin{longtable}{>{\raggedright\arraybackslash}p{2.5cm}>{\raggedright\arraybackslash}p{4.5cm}>{\raggedright\arraybackslash}p{6.0cm}p{1.0cm}rr>{\raggedright\arraybackslash}p{2.0cm}r>{\raggedright\arraybackslash}p{1cm}p{1cm}p{1cm}p{1cm}}
\rowcolor{white}\caption{Articles in Journal Journal of Intelligent Manufacturing (Total 6)}\\ \toprule
\rowcolor{white}\shortstack{Key\\Source} & Authors & Title (Colored by Open Access)& \shortstack{Details\\LC} & Cite & Year & \shortstack{Conference\\/Journal\\/School} & Pages & Relevance &\shortstack{Cites\\OC XR\\SC} & \shortstack{Refs\\OC\\XR} & \shortstack{Links\\Cites\\Refs}\\ \midrule\endhead
\bottomrule
\endfoot
He2019 \href{http://dx.doi.org/10.1007/s10845-019-01518-4}{He2019} & \hyperref[auth:a1547]{L. He}, \hyperref[auth:a308]{M. de Weerdt}, \hyperref[auth:a19]{N. Yorke-Smith} & \cellcolor{gold!20}Time/sequence-dependent scheduling: the design and evaluation of a general purpose tabu-based adaptive large neighbourhood search algorithm \hyperref[abs:He2019]{Abstract} & \hyperref[detail:He2019]{Details} No & \cite{He2019} & 2019 & Journal of Intelligent Manufacturing & null & \noindent{}\textcolor{black!50}{0.00} \textbf{1.50} n/a & 24 33 36 & 34 43 & 3 2 1\\
ZarandiKS16 \href{https://doi.org/10.1007/s10845-013-0860-9}{ZarandiKS16} & \hyperref[auth:a589]{M. H. F. Zarandi}, \hyperref[auth:a590]{H. Khorshidian}, \hyperref[auth:a591]{M. A. Shirazi} & A constraint programming model for the scheduling of {JIT} cross-docking systems with preemption & \hyperref[detail:ZarandiKS16]{Details} \href{../works/ZarandiKS16.pdf}{Yes} & \cite{ZarandiKS16} & 2016 & Journal of Intelligent Manufacturing & 17 & \noindent{}\textbf{1.00} \textbf{1.00} \textbf{5.09} & 28 29 31 & 14 22 & 9 4 5\\
MenciaSV13 \href{http://dx.doi.org/10.1007/s10845-012-0726-6}{MenciaSV13} & \hyperref[auth:a918]{C. Mencía}, \hyperref[auth:a919]{M. R. Sierra}, \hyperref[auth:a920]{R. Varela} & Intensified iterative deepening A* with application to job shop scheduling & \hyperref[detail:MenciaSV13]{Details} \href{../works/MenciaSV13.pdf}{Yes} & \cite{MenciaSV13} & 2013 & Journal of Intelligent Manufacturing & 11 & \noindent{}\textcolor{black!50}{0.00} \textcolor{black!50}{0.00} \textbf{6.60} & 9 9 12 & 43 55 & 10 0 10\\
KelbelH11 \href{https://doi.org/10.1007/s10845-009-0318-2}{KelbelH11} & \hyperref[auth:a618]{J. Kelbel}, \hyperref[auth:a116]{Z. Hanz{\'{a}}lek} & Solving production scheduling with earliness/tardiness penalties by constraint programming & \hyperref[detail:KelbelH11]{Details} \href{../works/KelbelH11.pdf}{Yes} & \cite{KelbelH11} & 2011 & Journal of Intelligent Manufacturing & 10 & \noindent{}\textbf{1.00} \textbf{1.00} \textbf{11.84} & 12 16 18 & 14 25 & 13 5 8\\
Salido10 \href{https://doi.org/10.1007/s10845-008-0188-z}{Salido10} & \hyperref[auth:a153]{M. A. Salido} & \cellcolor{gold!20}Introduction to planning, scheduling and constraint satisfaction \hyperref[abs:Salido10]{Abstract} & \hyperref[detail:Salido10]{Details} \href{../works/Salido10.pdf}{Yes} & \cite{Salido10} & 2010 & Journal of Intelligent Manufacturing & 4 & \noindent{}\textbf{1.00} \textbf{4.50} \textbf{1.68} & 22 22 17 & 1 3 & 1 0 1\\
BartakSR08 \href{http://dx.doi.org/10.1007/s10845-008-0203-4}{BartakSR08} & \hyperref[auth:a1063]{R. Barták}, \hyperref[auth:a153]{M. A. Salido}, \hyperref[auth:a316]{F. Rossi} & \cellcolor{green!10}Constraint satisfaction techniques in planning and scheduling & \hyperref[detail:BartakSR08]{Details} \href{../works/BartakSR08.pdf}{Yes} & \cite{BartakSR08} & 2008 & Journal of Intelligent Manufacturing & 11 & \noindent{}\textbf{1.00} \textbf{1.00} \textbf{12.88} & 54 57 76 & 21 51 & 16 7 9\\
\end{longtable}
}

\subsection{Journal of Intelligent \& Fuzzy Systems}

\index{Journal of Intelligent \& Fuzzy Systems}
{\scriptsize
\begin{longtable}{>{\raggedright\arraybackslash}p{2.5cm}>{\raggedright\arraybackslash}p{4.5cm}>{\raggedright\arraybackslash}p{6.0cm}p{1.0cm}rr>{\raggedright\arraybackslash}p{2.0cm}r>{\raggedright\arraybackslash}p{1cm}p{1cm}p{1cm}p{1cm}}
\rowcolor{white}\caption{Articles in Journal Journal of Intelligent \  Fuzzy Systems (Total 2)}\\ \toprule
\rowcolor{white}\shortstack{Key\\Source} & Authors & Title (Colored by Open Access)& \shortstack{Details\\LC} & Cite & Year & \shortstack{Conference\\/Journal\\/School} & Pages & Relevance &\shortstack{Cites\\OC XR\\SC} & \shortstack{Refs\\OC\\XR} & \shortstack{Links\\Cites\\Refs}\\ \midrule\endhead
\bottomrule
\endfoot
Xia2021 \href{http://dx.doi.org/10.3233/jifs-189721}{Xia2021} & \hyperref[auth:a1540]{Y. Xia}, \hyperref[auth:a1541]{Z. Xie}, \hyperref[auth:a1542]{Y. Xin}, \hyperref[auth:a1543]{X. Zhang} & A multi-shop integrated scheduling algorithm with fixed output constraint \hyperref[abs:Xia2021]{Abstract} & \hyperref[detail:Xia2021]{Details} No & \cite{Xia2021} & 2021 & Journal of Intelligent \  Fuzzy Systems & null & \noindent{}\textcolor{black!50}{0.00} \textbf{2.50} n/a & 4 6 7 & 17 20 & 2 0 2\\
Wikarek2019 \href{http://dx.doi.org/10.3233/jifs-179364}{Wikarek2019} & \hyperref[auth:a1476]{J. Wikarek}, \hyperref[auth:a1475]{P. Sitek}, \hyperref[auth:a630]{G. Bocewicz} & Resource constrained portfolio scheduling problem (RCPoSP): A hybrid approach & \hyperref[detail:Wikarek2019]{Details} No & \cite{Wikarek2019} & 2019 & Journal of Intelligent \  Fuzzy Systems & null & \noindent{}\textcolor{black!50}{0.00} \textcolor{black!50}{0.00} n/a & 0 0 0 & 14 22 & 4 0 4\\
\end{longtable}
}

\subsection{Journal of Manufacturing Systems}

\index{Journal of Manufacturing Systems}
{\scriptsize
\begin{longtable}{>{\raggedright\arraybackslash}p{2.5cm}>{\raggedright\arraybackslash}p{4.5cm}>{\raggedright\arraybackslash}p{6.0cm}p{1.0cm}rr>{\raggedright\arraybackslash}p{2.0cm}r>{\raggedright\arraybackslash}p{1cm}p{1cm}p{1cm}p{1cm}}
\rowcolor{white}\caption{Articles in Journal Journal of Manufacturing Systems (Total 1)}\\ \toprule
\rowcolor{white}\shortstack{Key\\Source} & Authors & Title (Colored by Open Access)& \shortstack{Details\\LC} & Cite & Year & \shortstack{Conference\\/Journal\\/School} & Pages & Relevance &\shortstack{Cites\\OC XR\\SC} & \shortstack{Refs\\OC\\XR} & \shortstack{Links\\Cites\\Refs}\\ \midrule\endhead
\bottomrule
\endfoot
OzturkTHO15 \href{https://www.sciencedirect.com/science/article/pii/S0278612515000527}{OzturkTHO15} & \hyperref[auth:a135]{C. {\"{O}}zt{\"{u}}rk}, \hyperref[auth:a1016]{S. Tunalı}, \hyperref[auth:a137]{B. Hnich}, \hyperref[auth:a138]{A. {\"{O}}rnek} & Cyclic scheduling of flexible mixed model assembly lines with parallel stations & \hyperref[detail:OzturkTHO15]{Details} \href{../works/OzturkTHO15.pdf}{Yes} & \cite{OzturkTHO15} & 2015 & Journal of Manufacturing Systems & 12 & \noindent{}\textcolor{black!50}{0.00} \textcolor{black!50}{0.00} \textbf{14.33} & 27 28 31 & 17 32 & 9 6 3\\
\end{longtable}
}

\subsection{Journal of Marine Science and Engineering}

\index{Journal of Marine Science and Engineering}
{\scriptsize
\begin{longtable}{>{\raggedright\arraybackslash}p{2.5cm}>{\raggedright\arraybackslash}p{4.5cm}>{\raggedright\arraybackslash}p{6.0cm}p{1.0cm}rr>{\raggedright\arraybackslash}p{2.0cm}r>{\raggedright\arraybackslash}p{1cm}p{1cm}p{1cm}p{1cm}}
\rowcolor{white}\caption{Articles in Journal Journal of Marine Science and Engineering (Total 1)}\\ \toprule
\rowcolor{white}\shortstack{Key\\Source} & Authors & Title (Colored by Open Access)& \shortstack{Details\\LC} & Cite & Year & \shortstack{Conference\\/Journal\\/School} & Pages & Relevance &\shortstack{Cites\\OC XR\\SC} & \shortstack{Refs\\OC\\XR} & \shortstack{Links\\Cites\\Refs}\\ \midrule\endhead
\bottomrule
\endfoot
LuZZYW24 \href{https://www.mdpi.com/2077-1312/12/1/124}{LuZZYW24} & \hyperref[auth:a1250]{X. Lu}, \hyperref[auth:a1251]{Y. Zhang}, \hyperref[auth:a1252]{L. Zheng}, \hyperref[auth:a1253]{C. Yang}, \hyperref[auth:a1254]{J. Wang} & \cellcolor{gold!20}Integrated Inbound and Outbound Scheduling for Coal Port: Constraint Programming and Adaptive Local Search & \hyperref[detail:LuZZYW24]{Details} \href{../works/LuZZYW24.pdf}{Yes} & \cite{LuZZYW24} & 2024 & Journal of Marine Science and Engineering & 36 & \noindent{}\textbf{1.00} \textbf{1.00} \textbf{71.08} & 0 0 0 & 0 57 & 0 0 0\\
\end{longtable}
}

\subsection{Journal of Mathematical Modelling and Algorithms}

\index{Journal of Mathematical Modelling and Algorithms}
{\scriptsize
\begin{longtable}{>{\raggedright\arraybackslash}p{2.5cm}>{\raggedright\arraybackslash}p{4.5cm}>{\raggedright\arraybackslash}p{6.0cm}p{1.0cm}rr>{\raggedright\arraybackslash}p{2.0cm}r>{\raggedright\arraybackslash}p{1cm}p{1cm}p{1cm}p{1cm}}
\rowcolor{white}\caption{Articles in Journal Journal of Mathematical Modelling and Algorithms (Total 1)}\\ \toprule
\rowcolor{white}\shortstack{Key\\Source} & Authors & Title (Colored by Open Access)& \shortstack{Details\\LC} & Cite & Year & \shortstack{Conference\\/Journal\\/School} & Pages & Relevance &\shortstack{Cites\\OC XR\\SC} & \shortstack{Refs\\OC\\XR} & \shortstack{Links\\Cites\\Refs}\\ \midrule\endhead
\bottomrule
\endfoot
CarchraeB09 \href{http://dx.doi.org/10.1007/s10852-008-9100-2}{CarchraeB09} & \hyperref[auth:a272]{T. Carchrae}, \hyperref[auth:a89]{J. C. Beck} & Principles for the Design of Large Neighborhood Search & \hyperref[detail:CarchraeB09]{Details} \href{../works/CarchraeB09.pdf}{Yes} & \cite{CarchraeB09} & 2009 & Journal of Mathematical Modelling and Algorithms & 26 & \noindent{}\textcolor{black!50}{0.00} \textcolor{black!50}{0.00} \textbf{5.14} & 16 17 25 & 19 29 & 17 5 12\\
\end{longtable}
}

\subsection{Journal of Modelling in Management}

\index{Journal of Modelling in Management}
{\scriptsize
\begin{longtable}{>{\raggedright\arraybackslash}p{2.5cm}>{\raggedright\arraybackslash}p{4.5cm}>{\raggedright\arraybackslash}p{6.0cm}p{1.0cm}rr>{\raggedright\arraybackslash}p{2.0cm}r>{\raggedright\arraybackslash}p{1cm}p{1cm}p{1cm}p{1cm}}
\rowcolor{white}\caption{Articles in Journal Journal of Modelling in Management (Total 1)}\\ \toprule
\rowcolor{white}\shortstack{Key\\Source} & Authors & Title (Colored by Open Access)& \shortstack{Details\\LC} & Cite & Year & \shortstack{Conference\\/Journal\\/School} & Pages & Relevance &\shortstack{Cites\\OC XR\\SC} & \shortstack{Refs\\OC\\XR} & \shortstack{Links\\Cites\\Refs}\\ \midrule\endhead
\bottomrule
\endfoot
Hosseinian2019 \href{http://dx.doi.org/10.1108/jm2-07-2018-0098}{Hosseinian2019} & \hyperref[auth:a1573]{A. H. Hosseinian}, \hyperref[auth:a1574]{V. Baradaran}, \hyperref[auth:a1575]{M. Bashiri} & Modeling of the time-dependent multi-skilled RCPSP considering learning effect \hyperref[abs:Hosseinian2019]{Abstract} & \hyperref[detail:Hosseinian2019]{Details} No & \cite{Hosseinian2019} & 2019 & Journal of Modelling in Management & null & \noindent{}\textcolor{black!50}{0.00} \textcolor{black!50}{0.00} n/a & 19 27 31 & 44 53 & 7 4 3\\
\end{longtable}
}

\subsection{Journal of Physics: Conference Series}

\index{Journal of Physics: Conference Series}
{\scriptsize
\begin{longtable}{>{\raggedright\arraybackslash}p{2.5cm}>{\raggedright\arraybackslash}p{4.5cm}>{\raggedright\arraybackslash}p{6.0cm}p{1.0cm}rr>{\raggedright\arraybackslash}p{2.0cm}r>{\raggedright\arraybackslash}p{1cm}p{1cm}p{1cm}p{1cm}}
\rowcolor{white}\caption{Articles in Journal Journal of Physics: Conference Series (Total 1)}\\ \toprule
\rowcolor{white}\shortstack{Key\\Source} & Authors & Title (Colored by Open Access)& \shortstack{Details\\LC} & Cite & Year & \shortstack{Conference\\/Journal\\/School} & Pages & Relevance &\shortstack{Cites\\OC XR\\SC} & \shortstack{Refs\\OC\\XR} & \shortstack{Links\\Cites\\Refs}\\ \midrule\endhead
\bottomrule
\endfoot
Zuenko2021 \href{http://dx.doi.org/10.1088/1742-6596/2060/1/012021}{Zuenko2021} & \hyperref[auth:a1994]{A. Zuenko}, \hyperref[auth:a1995]{Y. Oleynik}, \hyperref[auth:a1996]{R. Makedonov} & \cellcolor{gold!20}A method for solving the open-pit mine production scheduling problem using the constraint programming paradigm \hyperref[abs:Zuenko2021]{Abstract} & \hyperref[detail:Zuenko2021]{Details} No & \cite{Zuenko2021} & 2021 & Journal of Physics: Conference Series & null & \noindent{}\textbf{1.00} \textbf{2.00} n/a & 1 1 1 & 4 17 & 1 0 1\\
\end{longtable}
}

\subsection{Journal of Renewable and Sustainable Energy}

\index{Journal of Renewable and Sustainable Energy}
{\scriptsize
\begin{longtable}{>{\raggedright\arraybackslash}p{2.5cm}>{\raggedright\arraybackslash}p{4.5cm}>{\raggedright\arraybackslash}p{6.0cm}p{1.0cm}rr>{\raggedright\arraybackslash}p{2.0cm}r>{\raggedright\arraybackslash}p{1cm}p{1cm}p{1cm}p{1cm}}
\rowcolor{white}\caption{Articles in Journal Journal of Renewable and Sustainable Energy (Total 1)}\\ \toprule
\rowcolor{white}\shortstack{Key\\Source} & Authors & Title (Colored by Open Access)& \shortstack{Details\\LC} & Cite & Year & \shortstack{Conference\\/Journal\\/School} & Pages & Relevance &\shortstack{Cites\\OC XR\\SC} & \shortstack{Refs\\OC\\XR} & \shortstack{Links\\Cites\\Refs}\\ \midrule\endhead
\bottomrule
\endfoot
Zhang2019 \href{http://dx.doi.org/10.1063/1.5053623}{Zhang2019} & \hyperref[auth:a1745]{Z. Zhang}, \hyperref[auth:a1746]{M. Liu}, \hyperref[auth:a1747]{X. Song} & A bi-level fuzzy random model for multi-mode resource-constrained project scheduling problem of photovoltaic power plant \hyperref[abs:Zhang2019]{Abstract} & \hyperref[detail:Zhang2019]{Details} No & \cite{Zhang2019} & 2019 & Journal of Renewable and Sustainable Energy & null & \noindent{}\textcolor{black!50}{0.00} \textcolor{black!50}{0.00} n/a & 4 5 5 & 55 60 & 2 1 1\\
\end{longtable}
}

\subsection{Journal of Scheduling}

\index{Journal of Scheduling}
{\scriptsize
\begin{longtable}{>{\raggedright\arraybackslash}p{2.5cm}>{\raggedright\arraybackslash}p{4.5cm}>{\raggedright\arraybackslash}p{6.0cm}p{1.0cm}rr>{\raggedright\arraybackslash}p{2.0cm}r>{\raggedright\arraybackslash}p{1cm}p{1cm}p{1cm}p{1cm}}
\rowcolor{white}\caption{Articles in Journal Journal of Scheduling (Total 20)}\\ \toprule
\rowcolor{white}\shortstack{Key\\Source} & Authors & Title (Colored by Open Access)& \shortstack{Details\\LC} & Cite & Year & \shortstack{Conference\\/Journal\\/School} & Pages & Relevance &\shortstack{Cites\\OC XR\\SC} & \shortstack{Refs\\OC\\XR} & \shortstack{Links\\Cites\\Refs}\\ \midrule\endhead
\bottomrule
\endfoot
Braune2022 \href{http://dx.doi.org/10.1007/s10951-022-00750-w}{Braune2022} & \hyperref[auth:a1512]{R. Braune} & \cellcolor{gold!20}Packing-based branch-and-bound for discrete malleable task scheduling \hyperref[abs:Braune2022]{Abstract} & \hyperref[detail:Braune2022]{Details} No & \cite{Braune2022} & 2022 & Journal of Scheduling & null & \noindent{}\textcolor{black!50}{0.00} \textbf{3.00} n/a & 1 2 2 & 34 40 & 6 0 6\\
BulckG22 \href{http://dx.doi.org/10.1007/s10951-021-00717-3}{BulckG22} & \hyperref[auth:a1409]{D. V. Bulck}, \hyperref[auth:a1410]{D. Goossens} & \cellcolor{green!10}Optimizing rest times and differences in games played: an iterative two-phase approach \hyperref[abs:BulckG22]{Abstract} & \hyperref[detail:BulckG22]{Details} \href{../works/BulckG22.pdf}{Yes} & \cite{BulckG22} & 2022 & Journal of Scheduling & 11 & \noindent{}\textcolor{black!50}{0.00} \textcolor{black!50}{0.00} 0.33 & 2 3 3 & 19 22 & 4 0 4\\
HubnerGSV21 \href{https://doi.org/10.1007/s10951-021-00682-x}{HubnerGSV21} & \hyperref[auth:a482]{F. H{\"{u}}bner}, \hyperref[auth:a483]{P. Gerhards}, \hyperref[auth:a484]{C. St{\"{u}}rck}, \hyperref[auth:a485]{R. Volk} & \cellcolor{gold!20}Solving the nuclear dismantling project scheduling problem by combining mixed-integer and constraint programming techniques and metaheuristics & \hyperref[detail:HubnerGSV21]{Details} \href{../works/HubnerGSV21.pdf}{Yes} & \cite{HubnerGSV21} & 2021 & Journal of Scheduling & 22 & \noindent{}\textbf{1.00} \textbf{1.00} \textbf{13.02} & 0 2 1 & 37 46 & 4 0 4\\
Mischek2021 \href{http://dx.doi.org/10.1007/s10951-021-00699-2}{Mischek2021} & \hyperref[auth:a80]{F. Mischek}, \hyperref[auth:a45]{N. Musliu}, \hyperref[auth:a1261]{A. Schaerf} & \cellcolor{gold!20}Local search approaches for the test laboratory scheduling problem with variable task grouping \hyperref[abs:Mischek2021]{Abstract} & \hyperref[detail:Mischek2021]{Details} No & \cite{Mischek2021} & 2021 & Journal of Scheduling & null & \noindent{}\textcolor{black!50}{0.00} \textcolor{black!50}{0.00} n/a & 1 3 2 & 32 38 & 9 1 8\\
CauwelaertDS20 \href{http://dx.doi.org/10.1007/s10951-019-00632-8}{CauwelaertDS20} & \hyperref[auth:a835]{S. V. Cauwelaert}, \hyperref[auth:a202]{C. Dejemeppe}, \hyperref[auth:a147]{P. Schaus} & An Efficient Filtering Algorithm for the Unary Resource Constraint with Transition Times and Optional Activities & \hyperref[detail:CauwelaertDS20]{Details} \href{../works/CauwelaertDS20.pdf}{Yes} & \cite{CauwelaertDS20} & 2020 & Journal of Scheduling & 19 & \noindent{}\textcolor{black!50}{0.00} \textcolor{black!50}{0.00} \textbf{5.18} & 2 2 2 & 21 36 & 16 1 15\\
Tesch2020 \href{http://dx.doi.org/10.1007/s10951-020-00647-6}{Tesch2020} & \hyperref[auth:a183]{A. Tesch} & \cellcolor{gold!20}A polyhedral study of event-based models for the resource-constrained project scheduling problem \hyperref[abs:Tesch2020]{Abstract} & \hyperref[detail:Tesch2020]{Details} No & \cite{Tesch2020} & 2020 & Journal of Scheduling & null & \noindent{}\textcolor{black!50}{0.00} \textcolor{black!50}{0.00} n/a & 6 7 7 & 29 42 & 12 1 11\\
TranPZLDB18 \href{https://doi.org/10.1007/s10951-017-0537-x}{TranPZLDB18} & \hyperref[auth:a799]{T. T. Tran}, \hyperref[auth:a800]{M. Padmanabhan}, \hyperref[auth:a801]{P. Y. Zhang}, \hyperref[auth:a802]{H. Li}, \hyperref[auth:a803]{D. G. Down}, \hyperref[auth:a89]{J. C. Beck} & \cellcolor{green!10}Multi-stage resource-aware scheduling for data centers with heterogeneous servers & \hyperref[detail:TranPZLDB18]{Details} \href{../works/TranPZLDB18.pdf}{Yes} & \cite{TranPZLDB18} & 2018 & Journal of Scheduling & 17 & \noindent{}\textcolor{black!50}{0.00} \textcolor{black!50}{0.00} \textcolor{black!50}{0.00} & 8 9 9 & 26 29 & 1 0 1\\
BajestaniB15 \href{https://doi.org/10.1007/s10951-015-0416-2}{BajestaniB15} & \hyperref[auth:a817]{M. A. Bajestani}, \hyperref[auth:a89]{J. C. Beck} & A two-stage coupled algorithm for an integrated maintenance planning and flowshop scheduling problem with deteriorating machines & \hyperref[detail:BajestaniB15]{Details} \href{../works/BajestaniB15.pdf}{Yes} & \cite{BajestaniB15} & 2015 & Journal of Scheduling & 16 & \noindent{}\textcolor{black!50}{0.00} \textcolor{black!50}{0.00} \textbf{1.05} & 17 18 20 & 59 69 & 8 1 7\\
ArtiguesL14 \href{http://dx.doi.org/10.1007/s10951-014-0404-y}{ArtiguesL14} & \hyperref[auth:a6]{C. Artigues}, \hyperref[auth:a3]{P. Lopez} & \cellcolor{green!10}Energetic reasoning for energy-constrained scheduling with a continuous resource & \hyperref[detail:ArtiguesL14]{Details} \href{../works/ArtiguesL14.pdf}{Yes} & \cite{ArtiguesL14} & 2014 & Journal of Scheduling & 17 & \noindent{}\textcolor{black!50}{0.00} \textcolor{black!50}{0.00} \textbf{1.14} & 11 13 14 & 19 26 & 16 8 8\\
LaborieR14 \href{http://dx.doi.org/10.1007/s10951-014-0408-7}{LaborieR14} & \hyperref[auth:a118]{P. Laborie}, \hyperref[auth:a1069]{J. Rogerie} & Temporal linear relaxation in IBM ILOG CP Optimizer & \hyperref[detail:LaborieR14]{Details} \href{../works/LaborieR14.pdf}{Yes} & \cite{LaborieR14} & 2014 & Journal of Scheduling & 10 & \noindent{}\textcolor{black!50}{0.00} \textcolor{black!50}{0.00} \textbf{4.83} & 17 19 22 & 13 26 & 16 12 4\\
SchuttFSW13 \href{https://doi.org/10.1007/s10951-012-0285-x}{SchuttFSW13} & \hyperref[auth:a124]{A. Schutt}, \hyperref[auth:a154]{T. Feydy}, \hyperref[auth:a125]{P. J. Stuckey}, \hyperref[auth:a117]{M. G. Wallace} & \cellcolor{green!10}Solving RCPSP/max by lazy clause generation & \hyperref[detail:SchuttFSW13]{Details} \href{../works/SchuttFSW13.pdf}{Yes} & \cite{SchuttFSW13} & 2013 & Journal of Scheduling & 17 & \noindent{}\textcolor{black!50}{0.00} \textcolor{black!50}{0.00} \textbf{5.21} & 43 45 57 & 23 38 & 36 25 11\\
HeckmanB11 \href{https://doi.org/10.1007/s10951-009-0113-0}{HeckmanB11} & \hyperref[auth:a823]{I. Heckman}, \hyperref[auth:a89]{J. C. Beck} & Understanding the behavior of Solution-Guided Search for job-shop scheduling & \hyperref[detail:HeckmanB11]{Details} \href{../works/HeckmanB11.pdf}{Yes} & \cite{HeckmanB11} & 2011 & Journal of Scheduling & 20 & \noindent{}\textcolor{black!50}{0.00} \textcolor{black!50}{0.00} \textbf{3.92} & 0 1 3 & 22 39 & 7 0 7\\
LombardiMRB10 \href{http://dx.doi.org/10.1007/s10951-010-0184-y}{LombardiMRB10} & \hyperref[auth:a142]{M. Lombardi}, \hyperref[auth:a143]{M. Milano}, \hyperref[auth:a718]{M. Ruggiero}, \hyperref[auth:a245]{L. Benini} & Stochastic allocation and scheduling for conditional task graphs in multi-processor systems-on-chip & \hyperref[detail:LombardiMRB10]{Details} \href{../works/LombardiMRB10.pdf}{Yes} & \cite{LombardiMRB10} & 2010 & Journal of Scheduling & 31 & \noindent{}\textcolor{black!50}{0.00} \textcolor{black!50}{0.00} \textbf{17.06} & 24 24 30 & 41 55 & 19 5 14\\
Magato2010 \href{http://dx.doi.org/10.1007/s10951-010-0186-9}{Magato2010} & \hyperref[auth:a1808]{L. Magatão}, \hyperref[auth:a1809]{L. V. R. Arruda}, \hyperref[auth:a1810]{F. Neves-Jr} & A combined CLP-MILP approach for scheduling commodities in a pipeline & \hyperref[detail:Magato2010]{Details} No & \cite{Magato2010} & 2010 & Journal of Scheduling & null & \noindent{}\textbf{1.00} \textbf{1.00} n/a & 27 27 29 & 21 38 & 3 0 3\\
BidotVLB09 \href{https://doi.org/10.1007/s10951-008-0080-x}{BidotVLB09} & \hyperref[auth:a824]{J. Bidot}, \hyperref[auth:a825]{T. Vidal}, \hyperref[auth:a118]{P. Laborie}, \hyperref[auth:a89]{J. C. Beck} & A theoretic and practical framework for scheduling in a stochastic environment & \hyperref[detail:BidotVLB09]{Details} \href{../works/BidotVLB09.pdf}{Yes} & \cite{BidotVLB09} & 2009 & Journal of Scheduling & 30 & \noindent{}\textcolor{black!50}{0.00} \textcolor{black!50}{0.00} \textbf{9.21} & 58 60 76 & 20 49 & 8 1 7\\
GarridoAO09 \href{https://doi.org/10.1007/s10951-008-0083-7}{GarridoAO09} & \hyperref[auth:a633]{A. Garrido}, \hyperref[auth:a634]{M. Arang{\'{u}}}, \hyperref[auth:a635]{E. Onaindia} & A constraint programming formulation for planning: from plan scheduling to plan generation & \hyperref[detail:GarridoAO09]{Details} \href{../works/GarridoAO09.pdf}{Yes} & \cite{GarridoAO09} & 2009 & Journal of Scheduling & 30 & \noindent{}\textbf{1.00} \textbf{1.00} \textbf{20.10} & 5 5 9 & 14 37 & 4 1 3\\
Yang2009 \href{http://dx.doi.org/10.1007/s10951-009-0106-z}{Yang2009} & \hyperref[auth:a1823]{S. Yang}, \hyperref[auth:a1824]{D. Wang}, \hyperref[auth:a1825]{T. Chai}, \hyperref[auth:a1387]{G. Kendall} & \cellcolor{green!10}An improved constraint satisfaction adaptive neural network for job-shop scheduling & \hyperref[detail:Yang2009]{Details} No & \cite{Yang2009} & 2009 & Journal of Scheduling & null & \noindent{}\textbf{2.00} \textbf{2.00} n/a & 15 15 20 & 25 40 & 4 1 3\\
LiW08 \href{http://dx.doi.org/10.1007/s10951-008-0079-3}{LiW08} & \hyperref[auth:a952]{H. Li}, \hyperref[auth:a953]{K. Womer} & Scheduling projects with multi-skilled personnel by a hybrid MILP/CP benders decomposition algorithm & \hyperref[detail:LiW08]{Details} \href{../works/LiW08.pdf}{Yes} & \cite{LiW08} & 2008 & Journal of Scheduling & 18 & \noindent{}\textbf{1.00} \textbf{1.00} \textbf{23.20} & 113 123 144 & 31 52 & 25 10 15\\
Tsang03 \href{https://doi.org/10.1023/A:1024016929283}{Tsang03} & \hyperref[auth:a665]{E. P. K. Tsang} & Constraint Based Scheduling: Applying Constraint Programming to Scheduling Problems & \hyperref[detail:Tsang03]{Details} \href{../works/Tsang03.pdf}{Yes} & \cite{Tsang03} & 2003 & Journal of Scheduling & 2 & \noindent{}\textbf{1.00} \textbf{1.00} 0.36 & 1 0 0 & 0 0 & 0 0 0\\
BeckDDF98 \href{http://dx.doi.org/10.1002/(sici)1099-1425(199808)1:2<89::aid-jos9>3.0.co;2-h}{BeckDDF98} & \hyperref[auth:a89]{J. C. Beck}, \hyperref[auth:a248]{A. J. Davenport}, \hyperref[auth:a1218]{E. D. Davis}, \hyperref[auth:a302]{M. S. Fox} & The ODO project: toward a unified basis for constraint-directed scheduling & \hyperref[detail:BeckDDF98]{Details} \href{../works/BeckDDF98.pdf}{Yes} & \cite{BeckDDF98} & 1998 & Journal of Scheduling & 37 & \noindent{}\textcolor{black!50}{0.00} \textcolor{black!50}{0.00} \textbf{22.70} & 9 8 0 & 0 0 & 5 5 0\\
\end{longtable}
}

\subsection{Journal of Systems and Software}

\index{Journal of Systems and Software}
{\scriptsize
\begin{longtable}{>{\raggedright\arraybackslash}p{2.5cm}>{\raggedright\arraybackslash}p{4.5cm}>{\raggedright\arraybackslash}p{6.0cm}p{1.0cm}rr>{\raggedright\arraybackslash}p{2.0cm}r>{\raggedright\arraybackslash}p{1cm}p{1cm}p{1cm}p{1cm}}
\rowcolor{white}\caption{Articles in Journal Journal of Systems and Software (Total 1)}\\ \toprule
\rowcolor{white}\shortstack{Key\\Source} & Authors & Title (Colored by Open Access)& \shortstack{Details\\LC} & Cite & Year & \shortstack{Conference\\/Journal\\/School} & Pages & Relevance &\shortstack{Cites\\OC XR\\SC} & \shortstack{Refs\\OC\\XR} & \shortstack{Links\\Cites\\Refs}\\ \midrule\endhead
\bottomrule
\endfoot
HladikCDJ08 \href{http://dx.doi.org/10.1016/j.jss.2007.02.032}{HladikCDJ08} & \hyperref[auth:a1060]{P.-E. Hladik}, \hyperref[auth:a998]{H. Cambazard}, \hyperref[auth:a1161]{A.-M. Déplanche}, \hyperref[auth:a247]{N. Jussien} & \cellcolor{green!10}Solving a real-time allocation problem with constraint programming & \hyperref[detail:HladikCDJ08]{Details} \href{../works/HladikCDJ08.pdf}{Yes} & \cite{HladikCDJ08} & 2008 & Journal of Systems and Software & 18 & \noindent{}\textcolor{black!50}{0.00} \textcolor{black!50}{0.00} \textbf{9.26} & 36 37 48 & 27 66 & 16 8 8\\
\end{longtable}
}

\subsection{Journal of Transport Literature}

\index{Journal of Transport Literature}
{\scriptsize
\begin{longtable}{>{\raggedright\arraybackslash}p{2.5cm}>{\raggedright\arraybackslash}p{4.5cm}>{\raggedright\arraybackslash}p{6.0cm}p{1.0cm}rr>{\raggedright\arraybackslash}p{2.0cm}r>{\raggedright\arraybackslash}p{1cm}p{1cm}p{1cm}p{1cm}}
\rowcolor{white}\caption{Articles in Journal Journal of Transport Literature (Total 1)}\\ \toprule
\rowcolor{white}\shortstack{Key\\Source} & Authors & Title (Colored by Open Access)& \shortstack{Details\\LC} & Cite & Year & \shortstack{Conference\\/Journal\\/School} & Pages & Relevance &\shortstack{Cites\\OC XR\\SC} & \shortstack{Refs\\OC\\XR} & \shortstack{Links\\Cites\\Refs}\\ \midrule\endhead
\bottomrule
\endfoot
Silva2014 \href{http://dx.doi.org/10.1590/2238-1031.jtl.v8n4a9}{Silva2014} & \hyperref[auth:a1888]{G. P. Silva}, \hyperref[auth:a1889]{Allexandre Fortes da Silva Reis} & A study of different metaheuristics to solve the urban transit crew scheduling problem \hyperref[abs:Silva2014]{Abstract} & \hyperref[detail:Silva2014]{Details} No & \cite{Silva2014} & 2014 & Journal of Transport Literature & null & \noindent{}\textcolor{black!50}{0.00} \textbf{2.00} n/a & 2 2 0 & 10 22 & 1 1 0\\
\end{longtable}
}

\subsection{Journal of the Operational Research Society}

\index{Journal of the Operational Research Society}
{\scriptsize
\begin{longtable}{>{\raggedright\arraybackslash}p{2.5cm}>{\raggedright\arraybackslash}p{4.5cm}>{\raggedright\arraybackslash}p{6.0cm}p{1.0cm}rr>{\raggedright\arraybackslash}p{2.0cm}r>{\raggedright\arraybackslash}p{1cm}p{1cm}p{1cm}p{1cm}}
\rowcolor{white}\caption{Articles in Journal Journal of the Operational Research Society (Total 3)}\\ \toprule
\rowcolor{white}\shortstack{Key\\Source} & Authors & Title (Colored by Open Access)& \shortstack{Details\\LC} & Cite & Year & \shortstack{Conference\\/Journal\\/School} & Pages & Relevance &\shortstack{Cites\\OC XR\\SC} & \shortstack{Refs\\OC\\XR} & \shortstack{Links\\Cites\\Refs}\\ \midrule\endhead
\bottomrule
\endfoot
Tapkan2022 \href{http://dx.doi.org/10.1080/01605682.2022.2125843}{Tapkan2022} & \hyperref[auth:a1787]{P. Tapkan}, \hyperref[auth:a1788]{S. Kulluk}, \hyperref[auth:a1789]{L. Özbakır}, \hyperref[auth:a1790]{F. Bahar}, \hyperref[auth:a1791]{B. Gülmez} & A constraint programming based column generation approach for crew scheduling: A case study for the Kayseri railway & \hyperref[detail:Tapkan2022]{Details} No & \cite{Tapkan2022} & 2022 & \cellcolor{red!20}Journal of the Operational Research Society & null & \noindent{}\textbf{1.00} \textbf{1.00} n/a & 0 1 1 & 32 38 & 4 0 4\\
EdwardsBSE19 \href{http://dx.doi.org/10.1080/01605682.2019.1595192}{EdwardsBSE19} & \hyperref[auth:a892]{S. J. Edwards}, \hyperref[auth:a893]{D. Baatar}, \hyperref[auth:a894]{K. Smith-Miles}, \hyperref[auth:a469]{A. T. Ernst} & Symmetry breaking of identical projects in the high-multiplicity RCPSP/max & \hyperref[detail:EdwardsBSE19]{Details} No & \cite{EdwardsBSE19} & 2019 & \cellcolor{red!20}Journal of the Operational Research Society & 22 & \noindent{}\textcolor{black!50}{0.00} \textcolor{black!50}{0.00} n/a & 3 3 3 & 40 51 & 17 1 16\\
LorigeonBB02 \href{https://doi.org/10.1057/palgrave.jors.2601421}{LorigeonBB02} & \hyperref[auth:a671]{T. Lorigeon}, \hyperref[auth:a337]{J.-C. Billaut}, \hyperref[auth:a672]{J.-L. Bouquard} & A dynamic programming algorithm for scheduling jobs in a two-machine open shop with an availability constraint & \hyperref[detail:LorigeonBB02]{Details} \href{../works/LorigeonBB02.pdf}{Yes} & \cite{LorigeonBB02} & 2002 & \cellcolor{red!20}Journal of the Operational Research Society & 8 & \noindent{}\textcolor{black!50}{0.00} \textcolor{black!50}{0.00} \textcolor{black!50}{0.00} & 22 23 25 & 0 0 & 0 0 0\\
\end{longtable}
}

\subsection{Knowl. Eng. Rev.}

\index{Knowl. Eng. Rev.}
{\scriptsize
\begin{longtable}{>{\raggedright\arraybackslash}p{2.5cm}>{\raggedright\arraybackslash}p{4.5cm}>{\raggedright\arraybackslash}p{6.0cm}p{1.0cm}rr>{\raggedright\arraybackslash}p{2.0cm}r>{\raggedright\arraybackslash}p{1cm}p{1cm}p{1cm}p{1cm}}
\rowcolor{white}\caption{Articles in Journal Knowl. Eng. Rev. (Total 1)}\\ \toprule
\rowcolor{white}\shortstack{Key\\Source} & Authors & Title (Colored by Open Access)& \shortstack{Details\\LC} & Cite & Year & \shortstack{Conference\\/Journal\\/School} & Pages & Relevance &\shortstack{Cites\\OC XR\\SC} & \shortstack{Refs\\OC\\XR} & \shortstack{Links\\Cites\\Refs}\\ \midrule\endhead
\bottomrule
\endfoot
BartakSR10 \href{https://doi.org/10.1017/S0269888910000202}{BartakSR10} & \hyperref[auth:a152]{R. Bart{\'{a}}k}, \hyperref[auth:a153]{M. A. Salido}, \hyperref[auth:a316]{F. Rossi} & \cellcolor{green!10}New trends in constraint satisfaction, planning, and scheduling: a survey & \hyperref[detail:BartakSR10]{Details} \href{../works/BartakSR10.pdf}{Yes} & \cite{BartakSR10} & 2010 & Knowl. Eng. Rev. & 31 & \noindent{}\textbf{1.00} \textbf{1.00} \textbf{59.34} & 28 29 31 & 47 88 & 18 4 14\\
\end{longtable}
}

\subsection{Kybernetes}

\index{Kybernetes}
{\scriptsize
\begin{longtable}{>{\raggedright\arraybackslash}p{2.5cm}>{\raggedright\arraybackslash}p{4.5cm}>{\raggedright\arraybackslash}p{6.0cm}p{1.0cm}rr>{\raggedright\arraybackslash}p{2.0cm}r>{\raggedright\arraybackslash}p{1cm}p{1cm}p{1cm}p{1cm}}
\rowcolor{white}\caption{Articles in Journal Kybernetes (Total 3)}\\ \toprule
\rowcolor{white}\shortstack{Key\\Source} & Authors & Title (Colored by Open Access)& \shortstack{Details\\LC} & Cite & Year & \shortstack{Conference\\/Journal\\/School} & Pages & Relevance &\shortstack{Cites\\OC XR\\SC} & \shortstack{Refs\\OC\\XR} & \shortstack{Links\\Cites\\Refs}\\ \midrule\endhead
\bottomrule
\endfoot
Xu2023 \href{http://dx.doi.org/10.1108/k-09-2022-1339}{Xu2023} & \hyperref[auth:a1619]{J. Xu}, \hyperref[auth:a1620]{S. Bai} & A reactive scheduling approach for the resource-constrained project scheduling problem with dynamic resource disruption \hyperref[abs:Xu2023]{Abstract} & \hyperref[detail:Xu2023]{Details} No & \cite{Xu2023} & 2023 & Kybernetes & null & \noindent{}\textcolor{black!50}{0.00} \textcolor{black!50}{0.00} n/a & 1 2 2 & 42 51 & 7 1 6\\
Daneshamooz2021 \href{http://dx.doi.org/10.1108/k-08-2020-0521}{Daneshamooz2021} & \hyperref[auth:a1728]{F. Daneshamooz}, \hyperref[auth:a1729]{P. Fattahi}, \hyperref[auth:a1730]{S. M. H. Hosseini} & Mathematical modeling and two efficient branch and bound algorithms for job shop scheduling problem followed by an assembly stage \hyperref[abs:Daneshamooz2021]{Abstract} & \hyperref[detail:Daneshamooz2021]{Details} No & \cite{Daneshamooz2021} & 2021 & Kybernetes & null & \noindent{}\textcolor{black!50}{0.00} \textcolor{black!50}{0.00} n/a & 6 7 8 & 21 33 & 2 1 1\\
Bocewicz2009 \href{http://dx.doi.org/10.1108/03684920910976989}{Bocewicz2009} & \hyperref[auth:a630]{G. Bocewicz}, \hyperref[auth:a631]{I. Bach}, \hyperref[auth:a1913]{R. Wójcik} & Production flow prototyping subject to imprecise activity specification \hyperref[abs:Bocewicz2009]{Abstract} & \hyperref[detail:Bocewicz2009]{Details} No & \cite{Bocewicz2009} & 2009 & Kybernetes & null & \noindent{}\textcolor{black!50}{0.00} \textbf{13.51} n/a & 2 3 3 & 11 17 & 1 0 1\\
\end{longtable}
}

\subsection{Manag. Sci.}

\index{Manag. Sci.}
{\scriptsize
\begin{longtable}{>{\raggedright\arraybackslash}p{2.5cm}>{\raggedright\arraybackslash}p{4.5cm}>{\raggedright\arraybackslash}p{6.0cm}p{1.0cm}rr>{\raggedright\arraybackslash}p{2.0cm}r>{\raggedright\arraybackslash}p{1cm}p{1cm}p{1cm}p{1cm}}
\rowcolor{white}\caption{Articles in Journal Manag. Sci. (Total 1)}\\ \toprule
\rowcolor{white}\shortstack{Key\\Source} & Authors & Title (Colored by Open Access)& \shortstack{Details\\LC} & Cite & Year & \shortstack{Conference\\/Journal\\/School} & Pages & Relevance &\shortstack{Cites\\OC XR\\SC} & \shortstack{Refs\\OC\\XR} & \shortstack{Links\\Cites\\Refs}\\ \midrule\endhead
\bottomrule
\endfoot
BlomPS16 \href{https://doi.org/10.1287/mnsc.2015.2284}{BlomPS16} & \hyperref[auth:a795]{M. L. Blom}, \hyperref[auth:a324]{A. R. Pearce}, \hyperref[auth:a125]{P. J. Stuckey} & A Decomposition-Based Algorithm for the Scheduling of Open-Pit Networks Over Multiple Time Periods & \hyperref[detail:BlomPS16]{Details} \href{../works/BlomPS16.pdf}{Yes} & \cite{BlomPS16} & 2016 & Manag. Sci. & 26 & \noindent{}\textcolor{black!50}{0.00} \textcolor{black!50}{0.00} \textcolor{black!50}{0.00} & 20 23 25 & 36 46 & 2 0 2\\
\end{longtable}
}

\subsection{Management Science}

\index{Management Science}
{\scriptsize
\begin{longtable}{>{\raggedright\arraybackslash}p{2.5cm}>{\raggedright\arraybackslash}p{4.5cm}>{\raggedright\arraybackslash}p{6.0cm}p{1.0cm}rr>{\raggedright\arraybackslash}p{2.0cm}r>{\raggedright\arraybackslash}p{1cm}p{1cm}p{1cm}p{1cm}}
\rowcolor{white}\caption{Articles in Journal Management Science (Total 6)}\\ \toprule
\rowcolor{white}\shortstack{Key\\Source} & Authors & Title (Colored by Open Access)& \shortstack{Details\\LC} & Cite & Year & \shortstack{Conference\\/Journal\\/School} & Pages & Relevance &\shortstack{Cites\\OC XR\\SC} & \shortstack{Refs\\OC\\XR} & \shortstack{Links\\Cites\\Refs}\\ \midrule\endhead
\bottomrule
\endfoot
Dorndorf2000a \href{http://dx.doi.org/10.1287/mnsc.46.10.1365.12272}{Dorndorf2000a} & \hyperref[auth:a904]{U. Dorndorf}, \hyperref[auth:a438]{E. Pesch}, \hyperref[auth:a1046]{T. Phan-Huy} & A Time-Oriented Branch-and-Bound Algorithm for Resource-Constrained Project Scheduling with Generalised Precedence Constraints \hyperref[abs:Dorndorf2000a]{Abstract} & \hyperref[detail:Dorndorf2000a]{Details} No & \cite{Dorndorf2000a} & 2000 & Management Science & null & \noindent{}\textcolor{black!50}{0.00} \textbf{3.00} n/a & 79 80 87 & 20 28 & 18 9 9\\
Demeulemeester1997 \href{http://dx.doi.org/10.1287/mnsc.43.11.1485}{Demeulemeester1997} & \hyperref[auth:a1584]{E. L. Demeulemeester}, \hyperref[auth:a1585]{W. S. Herroelen} & \cellcolor{green!10}New Benchmark Results for the Resource-Constrained Project Scheduling Problem \hyperref[abs:Demeulemeester1997]{Abstract} & \hyperref[detail:Demeulemeester1997]{Details} No & \cite{Demeulemeester1997} & 1997 & Management Science & null & \noindent{}\textcolor{black!50}{0.00} \textcolor{black!50}{0.00} n/a & 158 161 183 & 0 0 & 14 14 0\\
Icmeli1996 \href{http://dx.doi.org/10.1287/mnsc.42.10.1395}{Icmeli1996} & \hyperref[auth:a1553]{O. Icmeli}, \hyperref[auth:a1554]{S. S. Erenguc} & A Branch and Bound Procedure for the Resource Constrained Project Scheduling Problem with Discounted Cash Flows \hyperref[abs:Icmeli1996]{Abstract} & \hyperref[detail:Icmeli1996]{Details} No & \cite{Icmeli1996} & 1996 & Management Science & null & \noindent{}\textcolor{black!50}{0.00} \textcolor{black!50}{0.00} n/a & 65 67 78 & 0 0 & 8 8 0\\
Demeulemeester1992 \href{http://dx.doi.org/10.1287/mnsc.38.12.1803}{Demeulemeester1992} & \hyperref[auth:a1090]{E. Demeulemeester}, \hyperref[auth:a1102]{W. Herroelen} & A Branch-and-Bound Procedure for the Multiple Resource-Constrained Project Scheduling Problem \hyperref[abs:Demeulemeester1992]{Abstract} & \hyperref[detail:Demeulemeester1992]{Details} No & \cite{Demeulemeester1992} & 1992 & Management Science & null & \noindent{}\textcolor{black!50}{0.00} \textcolor{black!50}{0.00} n/a & 380 387 0 & 0 0 & 22 22 0\\
Elmaghraby1992 \href{http://dx.doi.org/10.1287/mnsc.38.9.1245}{Elmaghraby1992} & \hyperref[auth:a1773]{S. E. Elmaghraby}, \hyperref[auth:a1774]{J. Kamburowski} & The Analysis of Activity Networks Under Generalized Precedence Relations (GPRs) \hyperref[abs:Elmaghraby1992]{Abstract} & \hyperref[detail:Elmaghraby1992]{Details} No & \cite{Elmaghraby1992} & 1992 & Management Science & null & \noindent{}\textcolor{black!50}{0.00} \textcolor{black!50}{0.00} n/a & 117 121 0 & 0 0 & 6 6 0\\
Talbot1978 \href{http://dx.doi.org/10.1287/mnsc.24.11.1163}{Talbot1978} & \hyperref[auth:a1497]{F. B. Talbot}, \hyperref[auth:a1498]{J. H. Patterson} & An Efficient Integer Programming Algorithm with Network Cuts for Solving Resource-Constrained Scheduling Problems \hyperref[abs:Talbot1978]{Abstract} & \hyperref[detail:Talbot1978]{Details} No & \cite{Talbot1978} & 1978 & Management Science & null & \noindent{}\textcolor{black!50}{0.00} \textcolor{black!50}{0.00} n/a & 149 152 155 & 0 0 & 8 8 0\\
\end{longtable}
}

\subsection{Management and Production Engineering Review}

\index{Management and Production Engineering Review}
{\scriptsize
\begin{longtable}{>{\raggedright\arraybackslash}p{2.5cm}>{\raggedright\arraybackslash}p{4.5cm}>{\raggedright\arraybackslash}p{6.0cm}p{1.0cm}rr>{\raggedright\arraybackslash}p{2.0cm}r>{\raggedright\arraybackslash}p{1cm}p{1cm}p{1cm}p{1cm}}
\rowcolor{white}\caption{Articles in Journal Management and Production Engineering Review (Total 1)}\\ \toprule
\rowcolor{white}\shortstack{Key\\Source} & Authors & Title (Colored by Open Access)& \shortstack{Details\\LC} & Cite & Year & \shortstack{Conference\\/Journal\\/School} & Pages & Relevance &\shortstack{Cites\\OC XR\\SC} & \shortstack{Refs\\OC\\XR} & \shortstack{Links\\Cites\\Refs}\\ \midrule\endhead
\bottomrule
\endfoot
Boek2016 \href{http://dx.doi.org/10.1515/mper-2016-0003}{Boek2016} & \hyperref[auth:a1885]{A. Bożek}, \hyperref[auth:a1886]{M. Wysocki} & \cellcolor{gold!20}Off-Line and Dynamic Production Scheduling – A Comparative Case Study \hyperref[abs:Boek2016]{Abstract} & \hyperref[detail:Boek2016]{Details} No & \cite{Boek2016} & 2016 & Management and Production Engineering Review & null & \noindent{}\textcolor{black!50}{0.00} \textbf{5.00} n/a & 5 5 8 & 10 23 & 1 1 0\\
\end{longtable}
}

\subsection{Marine Science and Technology Bulletin}

\index{Marine Science and Technology Bulletin}
{\scriptsize
\begin{longtable}{>{\raggedright\arraybackslash}p{2.5cm}>{\raggedright\arraybackslash}p{4.5cm}>{\raggedright\arraybackslash}p{6.0cm}p{1.0cm}rr>{\raggedright\arraybackslash}p{2.0cm}r>{\raggedright\arraybackslash}p{1cm}p{1cm}p{1cm}p{1cm}}
\rowcolor{white}\caption{Articles in Journal Marine Science and Technology Bulletin (Total 1)}\\ \toprule
\rowcolor{white}\shortstack{Key\\Source} & Authors & Title (Colored by Open Access)& \shortstack{Details\\LC} & Cite & Year & \shortstack{Conference\\/Journal\\/School} & Pages & Relevance &\shortstack{Cites\\OC XR\\SC} & \shortstack{Refs\\OC\\XR} & \shortstack{Links\\Cites\\Refs}\\ \midrule\endhead
\bottomrule
\endfoot
Akan2023 \href{http://dx.doi.org/10.33714/masteb.1324266}{Akan2023} & \hyperref[auth:a1751]{E. Akan}, \hyperref[auth:a1752]{G. Alkan} & Optimizing Shipbuilding Production Project Scheduling Under Resource Constraints Using Genetic Algorithms and Fuzzy Sets \hyperref[abs:Akan2023]{Abstract} & \hyperref[detail:Akan2023]{Details} No & \cite{Akan2023} & 2023 & Marine Science and Technology Bulletin & null & \noindent{}\textcolor{black!50}{0.00} \textcolor{black!50}{0.00} n/a & 0 0 0 & 132 148 & 6 0 6\\
\end{longtable}
}

\subsection{Mathematical Problems in Engineering}

\index{Mathematical Problems in Engineering}
{\scriptsize
\begin{longtable}{>{\raggedright\arraybackslash}p{2.5cm}>{\raggedright\arraybackslash}p{4.5cm}>{\raggedright\arraybackslash}p{6.0cm}p{1.0cm}rr>{\raggedright\arraybackslash}p{2.0cm}r>{\raggedright\arraybackslash}p{1cm}p{1cm}p{1cm}p{1cm}}
\rowcolor{white}\caption{Articles in Journal Mathematical Problems in Engineering (Total 7)}\\ \toprule
\rowcolor{white}\shortstack{Key\\Source} & Authors & Title (Colored by Open Access)& \shortstack{Details\\LC} & Cite & Year & \shortstack{Conference\\/Journal\\/School} & Pages & Relevance &\shortstack{Cites\\OC XR\\SC} & \shortstack{Refs\\OC\\XR} & \shortstack{Links\\Cites\\Refs}\\ \midrule\endhead
\bottomrule
\endfoot
HamPK21 \href{https://api.semanticscholar.org/CorpusID:237898414}{HamPK21} & \hyperref[auth:a750]{A. Ham}, \hyperref[auth:a751]{M.-J. Park}, \hyperref[auth:a752]{K. M. Kim} & \cellcolor{gold!20}Energy-Aware Flexible Job Shop Scheduling Using Mixed Integer Programming and Constraint Programming & \hyperref[detail:HamPK21]{Details} \href{../works/HamPK21.pdf}{Yes} & \cite{HamPK21} & 2021 & Mathematical Problems in Engineering & 12 & \noindent{}\textbf{2.00} \textbf{2.00} \textbf{10.25} & 6 9 11 & 46 51 & 3 1 2\\
Trker2018 \href{http://dx.doi.org/10.1155/2018/7870849}{Trker2018} & \hyperref[auth:a1714]{T. Türker}, \hyperref[auth:a1715]{A. Demiriz} & \cellcolor{gold!20}An Integrated Approach for Shift Scheduling and Rostering Problems with Break Times for Inbound Call Centers \hyperref[abs:Trker2018]{Abstract} & \hyperref[detail:Trker2018]{Details} No & \cite{Trker2018} & 2018 & Mathematical Problems in Engineering & null & \noindent{}\textcolor{black!50}{0.00} \textbf{5.00} n/a & 7 8 6 & 63 72 & 4 0 4\\
Yvars2018 \href{http://dx.doi.org/10.1155/2018/6861429}{Yvars2018} & \hyperref[auth:a1979]{P.-A. Yvars}, \hyperref[auth:a1980]{L. Zimmer} & \cellcolor{gold!20}System Sizing with a Model-Based Approach: Application to the Optimization of a Power Transmission System \hyperref[abs:Yvars2018]{Abstract} & \hyperref[detail:Yvars2018]{Details} No & \cite{Yvars2018} & 2018 & Mathematical Problems in Engineering & null & \noindent{}\textcolor{black!50}{0.00} \textbf{4.50} n/a & 2 3 4 & 16 25 & 2 0 2\\
Morillo2017 \href{http://dx.doi.org/10.1155/2017/4627856}{Morillo2017} & \hyperref[auth:a1735]{D. Morillo}, \hyperref[auth:a271]{F. Barber}, \hyperref[auth:a153]{M. A. Salido} & \cellcolor{gold!20}Mode-Based versus Activity-Based Search for a Nonredundant Resolution of the Multimode Resource-Constrained Project Scheduling Problem \hyperref[abs:Morillo2017]{Abstract} & \hyperref[detail:Morillo2017]{Details} No & \cite{Morillo2017} & 2017 & Mathematical Problems in Engineering & null & \noindent{}\textcolor{black!50}{0.00} \textcolor{black!50}{0.00} n/a & 2 3 3 & 33 37 & 4 0 4\\
Soto2015 \href{http://dx.doi.org/10.1155/2015/580785}{Soto2015} & \hyperref[auth:a1830]{R. Soto}, \hyperref[auth:a1831]{B. Crawford}, \hyperref[auth:a1832]{W. Palma}, \hyperref[auth:a1833]{E. Monfroy}, \hyperref[auth:a1834]{R. Olivares}, \hyperref[auth:a1835]{C. Castro}, \hyperref[auth:a1836]{F. Paredes} & \cellcolor{gold!20}Top-kBased Adaptive Enumeration in Constraint Programming \hyperref[abs:Soto2015]{Abstract} & \hyperref[detail:Soto2015]{Details} No & \cite{Soto2015} & 2015 & Mathematical Problems in Engineering & null & \noindent{}\textcolor{black!50}{0.00} \textbf{1.00} n/a & 8 7 8 & 17 22 & 2 1 1\\
Chaleshtarti2014 \href{http://dx.doi.org/10.1155/2014/634649}{Chaleshtarti2014} & \hyperref[auth:a1755]{A. S. Chaleshtarti}, \hyperref[auth:a1756]{S. Shadrokh}, \hyperref[auth:a1757]{Y. Fathi} & \cellcolor{gold!20}Branch and Bound Algorithms for Resource Constrained Project Scheduling Problem Subject to Nonrenewable Resources with Prescheduled Procurement \hyperref[abs:Chaleshtarti2014]{Abstract} & \hyperref[detail:Chaleshtarti2014]{Details} No & \cite{Chaleshtarti2014} & 2014 & Mathematical Problems in Engineering & null & \noindent{}\textcolor{black!50}{0.00} \textcolor{black!50}{0.00} n/a & 3 3 5 & 26 33 & 7 0 7\\
Bocewicz2013 \href{http://dx.doi.org/10.1155/2013/407096}{Bocewicz2013} & \hyperref[auth:a630]{G. Bocewicz}, \hyperref[auth:a1913]{R. Wójcik}, \hyperref[auth:a632]{Z. A. Banaszak}, \hyperref[auth:a1914]{P. Pawlewski} & \cellcolor{gold!20}Multimodal Processes Rescheduling: Cyclic Steady States Space Approach \hyperref[abs:Bocewicz2013]{Abstract} & \hyperref[detail:Bocewicz2013]{Details} No & \cite{Bocewicz2013} & 2013 & Mathematical Problems in Engineering & null & \noindent{}\textcolor{black!50}{0.00} \textbf{1.50} n/a & 13 14 20 & 18 24 & 1 1 0\\
\end{longtable}
}

\subsection{Mathematical Programming}

\index{Mathematical Programming}
{\scriptsize
\begin{longtable}{>{\raggedright\arraybackslash}p{2.5cm}>{\raggedright\arraybackslash}p{4.5cm}>{\raggedright\arraybackslash}p{6.0cm}p{1.0cm}rr>{\raggedright\arraybackslash}p{2.0cm}r>{\raggedright\arraybackslash}p{1cm}p{1cm}p{1cm}p{1cm}}
\rowcolor{white}\caption{Articles in Journal Mathematical Programming (Total 1)}\\ \toprule
\rowcolor{white}\shortstack{Key\\Source} & Authors & Title (Colored by Open Access)& \shortstack{Details\\LC} & Cite & Year & \shortstack{Conference\\/Journal\\/School} & Pages & Relevance &\shortstack{Cites\\OC XR\\SC} & \shortstack{Refs\\OC\\XR} & \shortstack{Links\\Cites\\Refs}\\ \midrule\endhead
\bottomrule
\endfoot
HookerO03 \href{http://dx.doi.org/10.1007/s10107-003-0375-9}{HookerO03} & \hyperref[auth:a160]{J. N. Hooker}, \hyperref[auth:a852]{G. Ottosson} & \cellcolor{green!10}Logic-based Benders decomposition & \hyperref[detail:HookerO03]{Details} \href{../works/HookerO03.pdf}{Yes} & \cite{HookerO03} & 2003 & Mathematical Programming & 28 & \noindent{}\textcolor{black!50}{0.00} \textcolor{black!50}{0.00} 0.61 & 317 333 371 & 0 0 & 78 78 0\\
\end{longtable}
}

\subsection{Mathematical Structures in Computer Science}

\index{Mathematical Structures in Computer Science}
{\scriptsize
\begin{longtable}{>{\raggedright\arraybackslash}p{2.5cm}>{\raggedright\arraybackslash}p{4.5cm}>{\raggedright\arraybackslash}p{6.0cm}p{1.0cm}rr>{\raggedright\arraybackslash}p{2.0cm}r>{\raggedright\arraybackslash}p{1cm}p{1cm}p{1cm}p{1cm}}
\rowcolor{white}\caption{Articles in Journal Mathematical Structures in Computer Science (Total 1)}\\ \toprule
\rowcolor{white}\shortstack{Key\\Source} & Authors & Title (Colored by Open Access)& \shortstack{Details\\LC} & Cite & Year & \shortstack{Conference\\/Journal\\/School} & Pages & Relevance &\shortstack{Cites\\OC XR\\SC} & \shortstack{Refs\\OC\\XR} & \shortstack{Links\\Cites\\Refs}\\ \midrule\endhead
\bottomrule
\endfoot
Larrosa2002 \href{http://dx.doi.org/10.1017/s0960129501003577}{Larrosa2002} & \hyperref[auth:a1794]{J. Larrosa}, \hyperref[auth:a1854]{G. Valiente} & Constraint satisfaction algorithms for graph  pattern matching \hyperref[abs:Larrosa2002]{Abstract} & \hyperref[detail:Larrosa2002]{Details} No & \cite{Larrosa2002} & 2002 & Mathematical Structures in Computer Science & null & \noindent{}\textcolor{black!50}{0.00} 0.50 n/a & 82 81 97 & 0 0 & 3 3 0\\
\end{longtable}
}

\subsection{Mathematical and Computer Modelling}

\index{Mathematical and Computer Modelling}
{\scriptsize
\begin{longtable}{>{\raggedright\arraybackslash}p{2.5cm}>{\raggedright\arraybackslash}p{4.5cm}>{\raggedright\arraybackslash}p{6.0cm}p{1.0cm}rr>{\raggedright\arraybackslash}p{2.0cm}r>{\raggedright\arraybackslash}p{1cm}p{1cm}p{1cm}p{1cm}}
\rowcolor{white}\caption{Articles in Journal Mathematical and Computer Modelling (Total 2)}\\ \toprule
\rowcolor{white}\shortstack{Key\\Source} & Authors & Title (Colored by Open Access)& \shortstack{Details\\LC} & Cite & Year & \shortstack{Conference\\/Journal\\/School} & Pages & Relevance &\shortstack{Cites\\OC XR\\SC} & \shortstack{Refs\\OC\\XR} & \shortstack{Links\\Cites\\Refs}\\ \midrule\endhead
\bottomrule
\endfoot
BeldiceanuC94 \href{https://www.sciencedirect.com/science/article/pii/0895717794901279}{BeldiceanuC94} & \hyperref[auth:a128]{N. Beldiceanu}, \hyperref[auth:a784]{E. Contejean} & \cellcolor{gold!20}Introducing Global Constraints in {CHIP} \hyperref[abs:BeldiceanuC94]{Abstract} & \hyperref[detail:BeldiceanuC94]{Details} \href{../works/BeldiceanuC94.pdf}{Yes} & \cite{BeldiceanuC94} & 1994 & Mathematical and Computer Modelling & 27 & \noindent{}\textcolor{black!50}{0.00} \textbf{1.00} \textbf{1.72} & 167 169 223 & 8 21 & 37 34 3\\
AggounB93 \href{https://www.sciencedirect.com/science/article/pii/089571779390068A}{AggounB93} & \hyperref[auth:a725]{A. Aggoun}, \hyperref[auth:a128]{N. Beldiceanu} & \cellcolor{gold!20}Extending {CHIP} in order to solve complex scheduling and placement problems \hyperref[abs:AggounB93]{Abstract} & \hyperref[detail:AggounB93]{Details} \href{../works/AggounB93.pdf}{Yes} & \cite{AggounB93} & 1993 & Mathematical and Computer Modelling & 17 & \noindent{}\textcolor{black!50}{0.00} \textbf{3.00} \textbf{3.18} & 187 191 214 & 11 36 & 91 89 2\\
\end{longtable}
}

\subsection{Mathematics}

\index{Mathematics}
{\scriptsize
\begin{longtable}{>{\raggedright\arraybackslash}p{2.5cm}>{\raggedright\arraybackslash}p{4.5cm}>{\raggedright\arraybackslash}p{6.0cm}p{1.0cm}rr>{\raggedright\arraybackslash}p{2.0cm}r>{\raggedright\arraybackslash}p{1cm}p{1cm}p{1cm}p{1cm}}
\rowcolor{white}\caption{Articles in Journal Mathematics (Total 5)}\\ \toprule
\rowcolor{white}\shortstack{Key\\Source} & Authors & Title (Colored by Open Access)& \shortstack{Details\\LC} & Cite & Year & \shortstack{Conference\\/Journal\\/School} & Pages & Relevance &\shortstack{Cites\\OC XR\\SC} & \shortstack{Refs\\OC\\XR} & \shortstack{Links\\Cites\\Refs}\\ \midrule\endhead
\bottomrule
\endfoot
Ramos2023 \href{http://dx.doi.org/10.3390/math11020337}{Ramos2023} & \hyperref[auth:a1731]{A. S. Ramos}, \hyperref[auth:a1732]{P. A. Miranda-Gonzalez}, \hyperref[auth:a1733]{S. Nucamendi-Guillén}, \hyperref[auth:a1734]{E. Olivares-Benitez} & \cellcolor{gold!20}A Formulation for the Stochastic Multi-Mode Resource-Constrained Project Scheduling Problem Solved with a Multi-Start Iterated Local Search Metaheuristic \hyperref[abs:Ramos2023]{Abstract} & \hyperref[detail:Ramos2023]{Details} No & \cite{Ramos2023} & 2023 & Mathematics & null & \noindent{}\textcolor{black!50}{0.00} \textcolor{black!50}{0.00} n/a & 0 1 1 & 57 64 & 7 0 7\\
YuraszeckMPV22 \href{http://dx.doi.org/10.3390/math10030329}{YuraszeckMPV22} & \hyperref[auth:a405]{F. Yuraszeck}, \hyperref[auth:a742]{G. Mejía}, \hyperref[auth:a743]{J. Pereira}, \hyperref[auth:a744]{M. Vilà} & \cellcolor{gold!20}A Novel Constraint Programming Decomposition Approach for the Total Flow Time Fixed Group Shop Scheduling Problem & \hyperref[detail:YuraszeckMPV22]{Details} \href{../works/YuraszeckMPV22.pdf}{Yes} & \cite{YuraszeckMPV22} & 2022 & Mathematics & 26 & \noindent{}\textbf{1.00} \textbf{1.00} \textbf{43.40} & 6 9 9 & 29 37 & 13 1 12\\
Liu2021a \href{http://dx.doi.org/10.3390/math9192492}{Liu2021a} & \hyperref[auth:a1244]{S.-S. Liu}, \hyperref[auth:a1719]{A. Budiwirawan}, \hyperref[auth:a1489]{M. F. A. Arifin} & \cellcolor{gold!20}Non-Sequential Linear Construction Project Scheduling Model for Minimizing Idle Equipment Using Constraint Programming (CP) \hyperref[abs:Liu2021a]{Abstract} & \hyperref[detail:Liu2021a]{Details} No & \cite{Liu2021a} & 2021 & Mathematics & null & \noindent{}\textbf{2.00} \textbf{3.00} n/a & 0 1 1 & 49 52 & 5 0 5\\
GurEA19 \href{https://api.semanticscholar.org/CorpusID:88492001}{GurEA19} & \hyperref[auth:a763]{Şeyda G{\"u}r}, \hyperref[auth:a415]{T. Eren}, \hyperref[auth:a764]{H. M. Alakaş} & \cellcolor{gold!20}Surgical Operation Scheduling with Goal Programming and Constraint Programming: A Case Study & \hyperref[detail:GurEA19]{Details} \href{../works/GurEA19.pdf}{Yes} & \cite{GurEA19} & 2019 & Mathematics & 24 & \noindent{}\textbf{1.00} \textbf{1.00} \textbf{4.44} & 19 21 19 & 30 49 & 4 4 0\\
Ozder2019 \href{http://dx.doi.org/10.3390/math7020192}{Ozder2019} & \hyperref[auth:a1753]{E. Özder}, \hyperref[auth:a1754]{E. Özcan}, \hyperref[auth:a415]{T. Eren} & \cellcolor{gold!20}Staff Task-Based Shift Scheduling Solution with an ANP and Goal Programming Method in a Natural Gas Combined Cycle Power Plant \hyperref[abs:Ozder2019]{Abstract} & \hyperref[detail:Ozder2019]{Details} No & \cite{Ozder2019} & 2019 & Mathematics & null & \noindent{}\textcolor{black!50}{0.00} \textcolor{black!50}{0.00} n/a & 24 26 18 & 58 75 & 2 0 2\\
\end{longtable}
}

\subsection{Naval Research Logistics (NRL)}

\index{Naval Research Logistics (NRL)}
{\scriptsize
\begin{longtable}{>{\raggedright\arraybackslash}p{2.5cm}>{\raggedright\arraybackslash}p{4.5cm}>{\raggedright\arraybackslash}p{6.0cm}p{1.0cm}rr>{\raggedright\arraybackslash}p{2.0cm}r>{\raggedright\arraybackslash}p{1cm}p{1cm}p{1cm}p{1cm}}
\rowcolor{white}\caption{Articles in Journal Naval Research Logistics (NRL) (Total 2)}\\ \toprule
\rowcolor{white}\shortstack{Key\\Source} & Authors & Title (Colored by Open Access)& \shortstack{Details\\LC} & Cite & Year & \shortstack{Conference\\/Journal\\/School} & Pages & Relevance &\shortstack{Cites\\OC XR\\SC} & \shortstack{Refs\\OC\\XR} & \shortstack{Links\\Cites\\Refs}\\ \midrule\endhead
\bottomrule
\endfoot
Moccia2005 \href{http://dx.doi.org/10.1002/nav.20121}{Moccia2005} & \hyperref[auth:a1589]{L. Moccia}, \hyperref[auth:a1590]{J. Cordeau}, \hyperref[auth:a1591]{M. Gaudioso}, \hyperref[auth:a1074]{G. Laporte} & A branch‐and‐cut algorithm for the quay crane scheduling problem in a container terminal \hyperref[abs:Moccia2005]{Abstract} & \hyperref[detail:Moccia2005]{Details} No & \cite{Moccia2005} & 2005 & Naval Research Logistics (NRL) & null & \noindent{}\textcolor{black!50}{0.00} \textcolor{black!50}{0.00} n/a & 140 147 168 & 13 20 & 4 3 1\\
Lim2004 \href{http://dx.doi.org/10.1002/nav.10123}{Lim2004} & \hyperref[auth:a279]{A. Lim}, \hyperref[auth:a280]{B. Rodrigues}, \hyperref[auth:a1743]{F. Xiao}, \hyperref[auth:a1744]{Y. Zhu} & Crane scheduling with spatial constraints \hyperref[abs:Lim2004]{Abstract} & \hyperref[detail:Lim2004]{Details} No & \cite{Lim2004} & 2004 & Naval Research Logistics (NRL) & null & \noindent{}\textcolor{black!50}{0.00} \textcolor{black!50}{0.00} n/a & 103 103 121 & 8 15 & 2 2 0\\
\end{longtable}
}

\subsection{Networks}

\index{Networks}
{\scriptsize
\begin{longtable}{>{\raggedright\arraybackslash}p{2.5cm}>{\raggedright\arraybackslash}p{4.5cm}>{\raggedright\arraybackslash}p{6.0cm}p{1.0cm}rr>{\raggedright\arraybackslash}p{2.0cm}r>{\raggedright\arraybackslash}p{1cm}p{1cm}p{1cm}p{1cm}}
\rowcolor{white}\caption{Articles in Journal Networks (Total 3)}\\ \toprule
\rowcolor{white}\shortstack{Key\\Source} & Authors & Title (Colored by Open Access)& \shortstack{Details\\LC} & Cite & Year & \shortstack{Conference\\/Journal\\/School} & Pages & Relevance &\shortstack{Cites\\OC XR\\SC} & \shortstack{Refs\\OC\\XR} & \shortstack{Links\\Cites\\Refs}\\ \midrule\endhead
\bottomrule
\endfoot
Mladenovic2015 \href{http://dx.doi.org/10.1002/net.21625}{Mladenovic2015} & \hyperref[auth:a1621]{S. Mladenovic}, \hyperref[auth:a1622]{S. Veskovic}, \hyperref[auth:a1623]{I. Branovic}, \hyperref[auth:a1624]{S. Jankovic}, \hyperref[auth:a1625]{S. Acimovic} & Heuristic Based Real‐Time Train Rescheduling System \hyperref[abs:Mladenovic2015]{Abstract} & \hyperref[detail:Mladenovic2015]{Details} No & \cite{Mladenovic2015} & 2015 & Networks & null & \noindent{}\textcolor{black!50}{0.00} \textbf{2.00} n/a & 2 2 3 & 21 30 & 4 0 4\\
Capone2009 \href{http://dx.doi.org/10.1002/net.20367}{Capone2009} & \hyperref[auth:a1563]{A. Capone}, \hyperref[auth:a1564]{G. Carello}, \hyperref[auth:a1565]{I. Filippini}, \hyperref[auth:a1566]{S. Gualandi}, \hyperref[auth:a1567]{F. Malucelli} & Solving a resource allocation problem in wireless mesh networks: A comparison between a CP‐based and a classical column generation \hyperref[abs:Capone2009]{Abstract} & \hyperref[detail:Capone2009]{Details} No & \cite{Capone2009} & 2009 & Networks & null & \noindent{}0.50 \textbf{1.00} n/a & 3 14 17 & 16 22 & 3 1 2\\
Ouaja2004 \href{http://dx.doi.org/10.1002/net.10110}{Ouaja2004} & \hyperref[auth:a1548]{W. Ouaja}, \hyperref[auth:a1549]{B. Richards} & \cellcolor{gold!20}A hybrid multicommodity routing algorithm for traffic engineering \hyperref[abs:Ouaja2004]{Abstract} & \hyperref[detail:Ouaja2004]{Details} No & \cite{Ouaja2004} & 2004 & Networks & null & \noindent{}\textcolor{black!50}{0.00} \textbf{1.50} n/a & 14 14 20 & 10 31 & 3 1 2\\
\end{longtable}
}

\subsection{OPERATIONS RESEARCH LETTERS}

\index{OPERATIONS RESEARCH LETTERS}
{\scriptsize
\begin{longtable}{>{\raggedright\arraybackslash}p{2.5cm}>{\raggedright\arraybackslash}p{4.5cm}>{\raggedright\arraybackslash}p{6.0cm}p{1.0cm}rr>{\raggedright\arraybackslash}p{2.0cm}r>{\raggedright\arraybackslash}p{1cm}p{1cm}p{1cm}p{1cm}}
\rowcolor{white}\caption{Articles in Journal OPERATIONS RESEARCH LETTERS (Total 1)}\\ \toprule
\rowcolor{white}\shortstack{Key\\Source} & Authors & Title (Colored by Open Access)& \shortstack{Details\\LC} & Cite & Year & \shortstack{Conference\\/Journal\\/School} & Pages & Relevance &\shortstack{Cites\\OC XR\\SC} & \shortstack{Refs\\OC\\XR} & \shortstack{Links\\Cites\\Refs}\\ \midrule\endhead
\bottomrule
\endfoot
ElfJR03 \href{http://dx.doi.org/10.1016/s0167-6377(03)00025-7}{ElfJR03} & \hyperref[auth:a1406]{M. Elf}, \hyperref[auth:a1407]{M. Jünger}, \hyperref[auth:a1408]{G. Rinaldi} & \cellcolor{green!10}Minimizing breaks by maximizing cuts \hyperref[abs:ElfJR03]{Abstract} & \hyperref[detail:ElfJR03]{Details} \href{../works/ElfJR03.pdf}{Yes} & \cite{ElfJR03} & 2003 & OPERATIONS RESEARCH LETTERS & 7 & \noindent{}\textcolor{black!50}{0.00} \textbf{1.00} \textcolor{black!50}{0.09} & 41 41 45 & 7 10 & 8 6 2\\
\end{longtable}
}

\subsection{OR Spectrum}

\index{OR Spectrum}
{\scriptsize
\begin{longtable}{>{\raggedright\arraybackslash}p{2.5cm}>{\raggedright\arraybackslash}p{4.5cm}>{\raggedright\arraybackslash}p{6.0cm}p{1.0cm}rr>{\raggedright\arraybackslash}p{2.0cm}r>{\raggedright\arraybackslash}p{1cm}p{1cm}p{1cm}p{1cm}}
\rowcolor{white}\caption{Articles in Journal OR Spectrum (Total 2)}\\ \toprule
\rowcolor{white}\shortstack{Key\\Source} & Authors & Title (Colored by Open Access)& \shortstack{Details\\LC} & Cite & Year & \shortstack{Conference\\/Journal\\/School} & Pages & Relevance &\shortstack{Cites\\OC XR\\SC} & \shortstack{Refs\\OC\\XR} & \shortstack{Links\\Cites\\Refs}\\ \midrule\endhead
\bottomrule
\endfoot
Watermeyer2020 \href{http://dx.doi.org/10.1007/s00291-020-00583-z}{Watermeyer2020} & \hyperref[auth:a1770]{K. Watermeyer}, \hyperref[auth:a1771]{J. Zimmermann} & \cellcolor{gold!20}A branch-and-bound procedure for the resource-constrained project scheduling problem with partially renewable resources and general temporal constraints \hyperref[abs:Watermeyer2020]{Abstract} & \hyperref[detail:Watermeyer2020]{Details} No & \cite{Watermeyer2020} & 2020 & OR Spectrum & null & \noindent{}\textcolor{black!50}{0.00} \textcolor{black!50}{0.00} n/a & 12 17 21 & 26 36 & 7 1 6\\
Benda2019 \href{http://dx.doi.org/10.1007/s00291-019-00567-8}{Benda2019} & \hyperref[auth:a1966]{F. Benda}, \hyperref[auth:a1512]{R. Braune}, \hyperref[auth:a1967]{K. F. Doerner}, \hyperref[auth:a951]{R. F. Hartl} & \cellcolor{gold!20}A machine learning approach for flow shop scheduling problems with alternative resources, sequence-dependent setup times, and blocking \hyperref[abs:Benda2019]{Abstract} & \hyperref[detail:Benda2019]{Details} No & \cite{Benda2019} & 2019 & OR Spectrum & null & \noindent{}\textcolor{black!50}{0.00} \textbf{6.01} n/a & 12 14 18 & 14 22 & 0 0 0\\
\end{longtable}
}

\subsection{Omega}

\index{Omega}
{\scriptsize
\begin{longtable}{>{\raggedright\arraybackslash}p{2.5cm}>{\raggedright\arraybackslash}p{4.5cm}>{\raggedright\arraybackslash}p{6.0cm}p{1.0cm}rr>{\raggedright\arraybackslash}p{2.0cm}r>{\raggedright\arraybackslash}p{1cm}p{1cm}p{1cm}p{1cm}}
\rowcolor{white}\caption{Articles in Journal Omega (Total 5)}\\ \toprule
\rowcolor{white}\shortstack{Key\\Source} & Authors & Title (Colored by Open Access)& \shortstack{Details\\LC} & Cite & Year & \shortstack{Conference\\/Journal\\/School} & Pages & Relevance &\shortstack{Cites\\OC XR\\SC} & \shortstack{Refs\\OC\\XR} & \shortstack{Links\\Cites\\Refs}\\ \midrule\endhead
\bottomrule
\endfoot
Fatemi-AnarakiTFV23 \href{http://dx.doi.org/10.1016/j.omega.2022.102770}{Fatemi-AnarakiTFV23} & \hyperref[auth:a735]{S. Fatemi-Anaraki}, \hyperref[auth:a430]{R. Tavakkoli-Moghaddam}, \hyperref[auth:a736]{M. Foumani}, \hyperref[auth:a737]{B. Vahedi-Nouri} & Scheduling of Multi-Robot Job Shop Systems in Dynamic Environments: Mixed-Integer Linear Programming and Constraint Programming Approaches & \hyperref[detail:Fatemi-AnarakiTFV23]{Details} \href{../works/Fatemi-AnarakiTFV23.pdf}{Yes} & \cite{Fatemi-AnarakiTFV23} & 2023 & Omega & 15 & \noindent{}\textbf{2.00} \textbf{2.00} \textbf{26.16} & 7 14 16 & 60 66 & 15 0 15\\
NaderiBZR23 \href{http://dx.doi.org/10.1016/j.omega.2022.102805}{NaderiBZR23} & \hyperref[auth:a726]{B. Naderi}, \hyperref[auth:a836]{M. A. Begen}, \hyperref[auth:a838]{G. S. Zaric}, \hyperref[auth:a728]{V. Roshanaei} & A novel and efficient exact technique for integrated staffing, assignment, routing, and scheduling of home care services under uncertainty & \hyperref[detail:NaderiBZR23]{Details} \href{../works/NaderiBZR23.pdf}{Yes} & \cite{NaderiBZR23} & 2023 & Omega & 15 & \noindent{}\textcolor{black!50}{0.00} \textcolor{black!50}{0.00} \textbf{1.01} & 4 6 6 & 64 80 & 12 0 12\\
ZhuSZW23 \href{http://dx.doi.org/10.1016/j.omega.2022.102823}{ZhuSZW23} & \hyperref[auth:a988]{X. Zhu}, \hyperref[auth:a989]{J. Son}, \hyperref[auth:a990]{X. Zhang}, \hyperref[auth:a991]{J. Wu} & Constraint programming and logic-based Benders decomposition for the integrated process planning and scheduling problem & \hyperref[detail:ZhuSZW23]{Details} \href{../works/ZhuSZW23.pdf}{Yes} & \cite{ZhuSZW23} & 2023 & Omega & 22 & \noindent{}\textbf{1.00} \textbf{1.00} \textbf{34.84} & 1 1 1 & 36 50 & 13 0 13\\
BukchinR18 \href{http://dx.doi.org/10.1016/j.omega.2017.06.008}{BukchinR18} & \hyperref[auth:a1181]{Y. Bukchin}, \hyperref[auth:a1182]{T. Raviv} & Constraint programming for solving various assembly line balancing problems & \hyperref[detail:BukchinR18]{Details} \href{../works/BukchinR18.pdf}{Yes} & \cite{BukchinR18} & 2018 & Omega & 12 & \noindent{}\textcolor{black!50}{0.00} \textcolor{black!50}{0.00} \textbf{21.89} & 66 68 81 & 29 43 & 23 22 1\\
Wang2007 \href{http://dx.doi.org/10.1016/j.omega.2005.06.001}{Wang2007} & \hyperref[auth:a1936]{S. M. Wang}, \hyperref[auth:a1937]{J. C. Chen}, \hyperref[auth:a1938]{K.-J. Wang} & Resource portfolio planning of make-to-stock products using a constraint programming-based genetic algorithm & \hyperref[detail:Wang2007]{Details} No & \cite{Wang2007} & 2007 & Omega & null & \noindent{}0.50 0.50 n/a & 26 28 32 & 13 17 & 1 1 0\\
\end{longtable}
}

\subsection{Oper. Res. Forum}

\index{Oper. Res. Forum}
{\scriptsize
\begin{longtable}{>{\raggedright\arraybackslash}p{2.5cm}>{\raggedright\arraybackslash}p{4.5cm}>{\raggedright\arraybackslash}p{6.0cm}p{1.0cm}rr>{\raggedright\arraybackslash}p{2.0cm}r>{\raggedright\arraybackslash}p{1cm}p{1cm}p{1cm}p{1cm}}
\rowcolor{white}\caption{Articles in Journal Oper. Res. Forum (Total 3)}\\ \toprule
\rowcolor{white}\shortstack{Key\\Source} & Authors & Title (Colored by Open Access)& \shortstack{Details\\LC} & Cite & Year & \shortstack{Conference\\/Journal\\/School} & Pages & Relevance &\shortstack{Cites\\OC XR\\SC} & \shortstack{Refs\\OC\\XR} & \shortstack{Links\\Cites\\Refs}\\ \midrule\endhead
\bottomrule
\endfoot
FrimodigECM23 \href{https://doi.org/10.1007/s43069-023-00251-2}{FrimodigECM23} & \hyperref[auth:a95]{S. Frimodig}, \hyperref[auth:a1414]{P. Enqvist}, \hyperref[auth:a91]{M. Carlsson}, \hyperref[auth:a1415]{C. Mercier} & \cellcolor{gold!20}Comparing Optimization Methods for Radiation Therapy Patient Scheduling using Different Objectives & \hyperref[detail:FrimodigECM23]{Details} \href{../works/FrimodigECM23.pdf}{Yes} & \cite{FrimodigECM23} & 2023 & Oper. Res. Forum & 38 & \noindent{}\textcolor{black!50}{0.00} \textcolor{black!50}{0.00} \textbf{11.72} & 0 0 0 & 0 56 & 0 0 0\\
FetgoD22 \href{https://doi.org/10.1007/s43069-022-00172-6}{FetgoD22} & \hyperref[auth:a11]{S. B. Fetgo}, \hyperref[auth:a13]{C. T. Djam{\'{e}}gni} & \cellcolor{green!10}Horizontally Elastic Edge-Finder Algorithm for Cumulative Resource Constraint Revisited & \hyperref[detail:FetgoD22]{Details} \href{../works/FetgoD22.pdf}{Yes} & \cite{FetgoD22} & 2022 & Oper. Res. Forum & 32 & \noindent{}\textcolor{black!50}{0.00} \textcolor{black!50}{0.00} \textbf{12.41} & 0 0 1 & 20 29 & 16 0 16\\
SacramentoSP20 \href{https://doi.org/10.1007/s43069-020-00036-x}{SacramentoSP20} & \hyperref[auth:a519]{D. Sacramento}, \hyperref[auth:a85]{C. Solnon}, \hyperref[auth:a520]{D. Pisinger} & \cellcolor{gold!20}Constraint Programming and Local Search Heuristic: a Matheuristic Approach for Routing and Scheduling Feeder Vessels in Multi-terminal Ports & \hyperref[detail:SacramentoSP20]{Details} \href{../works/SacramentoSP20.pdf}{Yes} & \cite{SacramentoSP20} & 2020 & Oper. Res. Forum & 33 & \noindent{}\textbf{1.00} \textbf{1.00} \textbf{14.30} & 2 4 5 & 38 46 & 13 2 11\\
\end{longtable}
}

\subsection{Operational Research}

\index{Operational Research}
{\scriptsize
\begin{longtable}{>{\raggedright\arraybackslash}p{2.5cm}>{\raggedright\arraybackslash}p{4.5cm}>{\raggedright\arraybackslash}p{6.0cm}p{1.0cm}rr>{\raggedright\arraybackslash}p{2.0cm}r>{\raggedright\arraybackslash}p{1cm}p{1cm}p{1cm}p{1cm}}
\rowcolor{white}\caption{Articles in Journal Operational Research (Total 1)}\\ \toprule
\rowcolor{white}\shortstack{Key\\Source} & Authors & Title (Colored by Open Access)& \shortstack{Details\\LC} & Cite & Year & \shortstack{Conference\\/Journal\\/School} & Pages & Relevance &\shortstack{Cites\\OC XR\\SC} & \shortstack{Refs\\OC\\XR} & \shortstack{Links\\Cites\\Refs}\\ \midrule\endhead
\bottomrule
\endfoot
OrnekOS20 \href{https://ideas.repec.org/a/spr/operea/v22y2022i1d10.1007_s12351-020-00563-9.html}{OrnekOS20} & \hyperref[auth:a138]{A. {\"{O}}rnek}, \hyperref[auth:a135]{C. {\"{O}}zt{\"{u}}rk}, \hyperref[auth:a1013]{I. Sugut} & {Integer and constraint programming model formulations for flight-gate assignment problem} & \hyperref[detail:OrnekOS20]{Details} \href{../works/OrnekOS20.pdf}{Yes} & \cite{OrnekOS20} & 2022 & Operational Research & 29 & \noindent{}\textcolor{black!50}{0.00} \textcolor{black!50}{0.00} \textbf{1.65} & 0 0 0 & 0 0 & 0 0 0\\
\end{longtable}
}

\subsection{Operations Research}

\index{Operations Research}
{\scriptsize
\begin{longtable}{>{\raggedright\arraybackslash}p{2.5cm}>{\raggedright\arraybackslash}p{4.5cm}>{\raggedright\arraybackslash}p{6.0cm}p{1.0cm}rr>{\raggedright\arraybackslash}p{2.0cm}r>{\raggedright\arraybackslash}p{1cm}p{1cm}p{1cm}p{1cm}}
\rowcolor{white}\caption{Articles in Journal Operations Research (Total 2)}\\ \toprule
\rowcolor{white}\shortstack{Key\\Source} & Authors & Title (Colored by Open Access)& \shortstack{Details\\LC} & Cite & Year & \shortstack{Conference\\/Journal\\/School} & Pages & Relevance &\shortstack{Cites\\OC XR\\SC} & \shortstack{Refs\\OC\\XR} & \shortstack{Links\\Cites\\Refs}\\ \midrule\endhead
\bottomrule
\endfoot
Hooker07 \href{http://dx.doi.org/10.1287/opre.1060.0371}{Hooker07} & \hyperref[auth:a160]{J. N. Hooker} & Planning and Scheduling by Logic-Based Benders Decomposition & \hyperref[detail:Hooker07]{Details} \href{../works/Hooker07.pdf}{Yes} & \cite{Hooker07} & 2007 & \cellcolor{red!20}Operations Research & 15 & \noindent{}\textcolor{black!50}{0.00} \textcolor{black!50}{0.00} \textbf{14.07} & 181 197 205 & 19 20 & 66 52 14\\
Henz01 \href{http://dx.doi.org/10.1287/opre.49.1.163.11193}{Henz01} & \hyperref[auth:a1419]{M. Henz} & Scheduling a Major College Basketball Conference—Revisited & \hyperref[detail:Henz01]{Details} No & \cite{Henz01} & 2001 & \cellcolor{red!20}Operations Research & 6 & \noindent{}\textcolor{black!50}{0.00} \textcolor{black!50}{0.00} n/a & 65 68 0 & 9 16 & 18 14 4\\
\end{longtable}
}

\subsection{Operations Research Perspectives}

\index{Operations Research Perspectives}
{\scriptsize
\begin{longtable}{>{\raggedright\arraybackslash}p{2.5cm}>{\raggedright\arraybackslash}p{4.5cm}>{\raggedright\arraybackslash}p{6.0cm}p{1.0cm}rr>{\raggedright\arraybackslash}p{2.0cm}r>{\raggedright\arraybackslash}p{1cm}p{1cm}p{1cm}p{1cm}}
\rowcolor{white}\caption{Articles in Journal Operations Research Perspectives (Total 2)}\\ \toprule
\rowcolor{white}\shortstack{Key\\Source} & Authors & Title (Colored by Open Access)& \shortstack{Details\\LC} & Cite & Year & \shortstack{Conference\\/Journal\\/School} & Pages & Relevance &\shortstack{Cites\\OC XR\\SC} & \shortstack{Refs\\OC\\XR} & \shortstack{Links\\Cites\\Refs}\\ \midrule\endhead
\bottomrule
\endfoot
ColT22 \href{http://dx.doi.org/10.1016/j.orp.2022.100249}{ColT22} & \hyperref[auth:a93]{G. D. Col}, \hyperref[auth:a738]{E. C. Teppan} & \cellcolor{gold!20}Industrial-size job shop scheduling with constraint programming & \hyperref[detail:ColT22]{Details} \href{../works/ColT22.pdf}{Yes} & \cite{ColT22} & 2022 & Operations Research Perspectives & 19 & \noindent{}\textbf{2.00} \textbf{2.00} \textbf{33.77} & 3 8 13 & 55 99 & 14 0 14\\
SchnellH17 \href{http://dx.doi.org/10.1016/j.orp.2017.01.002}{SchnellH17} & \hyperref[auth:a950]{A. Schnell}, \hyperref[auth:a951]{R. F. Hartl} & \cellcolor{gold!20}On the generalization of constraint programming and boolean satisfiability solving techniques to schedule a resource-constrained project consisting of multi-mode jobs & \hyperref[detail:SchnellH17]{Details} \href{../works/SchnellH17.pdf}{Yes} & \cite{SchnellH17} & 2017 & Operations Research Perspectives & 11 & \noindent{}\textbf{1.50} \textbf{1.50} \textbf{9.59} & 12 18 21 & 20 37 & 13 4 9\\
\end{longtable}
}

\subsection{Operations Research for Health Care}

\index{Operations Research for Health Care}
{\scriptsize
\begin{longtable}{>{\raggedright\arraybackslash}p{2.5cm}>{\raggedright\arraybackslash}p{4.5cm}>{\raggedright\arraybackslash}p{6.0cm}p{1.0cm}rr>{\raggedright\arraybackslash}p{2.0cm}r>{\raggedright\arraybackslash}p{1cm}p{1cm}p{1cm}p{1cm}}
\rowcolor{white}\caption{Articles in Journal Operations Research for Health Care (Total 1)}\\ \toprule
\rowcolor{white}\shortstack{Key\\Source} & Authors & Title (Colored by Open Access)& \shortstack{Details\\LC} & Cite & Year & \shortstack{Conference\\/Journal\\/School} & Pages & Relevance &\shortstack{Cites\\OC XR\\SC} & \shortstack{Refs\\OC\\XR} & \shortstack{Links\\Cites\\Refs}\\ \midrule\endhead
\bottomrule
\endfoot
ZhaoL14 \href{http://dx.doi.org/10.1016/j.orhc.2014.05.003}{ZhaoL14} & \hyperref[auth:a1376]{Z. Zhao}, \hyperref[auth:a1377]{X. Li} & Scheduling elective surgeries with sequence-dependent setup times to multiple operating rooms using constraint programming & \hyperref[detail:ZhaoL14]{Details} \href{../works/ZhaoL14.pdf}{Yes} & \cite{ZhaoL14} & 2014 & Operations Research for Health Care & 8 & \noindent{}\textbf{1.00} \textbf{1.00} \textbf{7.14} & 40 40 50 & 23 34 & 6 5 1\\
\end{longtable}
}

\subsection{Operations research for health care}

\index{Operations research for health care}
{\scriptsize
\begin{longtable}{>{\raggedright\arraybackslash}p{2.5cm}>{\raggedright\arraybackslash}p{4.5cm}>{\raggedright\arraybackslash}p{6.0cm}p{1.0cm}rr>{\raggedright\arraybackslash}p{2.0cm}r>{\raggedright\arraybackslash}p{1cm}p{1cm}p{1cm}p{1cm}}
\rowcolor{white}\caption{Articles in Journal Operations research for health care (Total 1)}\\ \toprule
\rowcolor{white}\shortstack{Key\\Source} & Authors & Title (Colored by Open Access)& \shortstack{Details\\LC} & Cite & Year & \shortstack{Conference\\/Journal\\/School} & Pages & Relevance &\shortstack{Cites\\OC XR\\SC} & \shortstack{Refs\\OC\\XR} & \shortstack{Links\\Cites\\Refs}\\ \midrule\endhead
\bottomrule
\endfoot
YounespourAKE19 \href{https://api.semanticscholar.org/CorpusID:208103305}{YounespourAKE19} & \hyperref[auth:a758]{M. Younespour}, \hyperref[auth:a759]{A. Atighehchian}, \hyperref[auth:a760]{K. Kianfar}, \hyperref[auth:a761]{E. T. Esfahani} & Using mixed integer programming and constraint programming for operating rooms scheduling with modified block strategy & \hyperref[detail:YounespourAKE19]{Details} \href{../works/YounespourAKE19.pdf}{Yes} & \cite{YounespourAKE19} & 2019 & Operations research for health care & 11 & \noindent{}\textbf{1.00} \textbf{1.00} \textbf{5.94} & 7 7 11 & 15 26 & 8 4 4\\
\end{longtable}
}

\subsection{PLOS ONE}

\index{PLOS ONE}
{\scriptsize
\begin{longtable}{>{\raggedright\arraybackslash}p{2.5cm}>{\raggedright\arraybackslash}p{4.5cm}>{\raggedright\arraybackslash}p{6.0cm}p{1.0cm}rr>{\raggedright\arraybackslash}p{2.0cm}r>{\raggedright\arraybackslash}p{1cm}p{1cm}p{1cm}p{1cm}}
\rowcolor{white}\caption{Articles in Journal PLOS ONE (Total 1)}\\ \toprule
\rowcolor{white}\shortstack{Key\\Source} & Authors & Title (Colored by Open Access)& \shortstack{Details\\LC} & Cite & Year & \shortstack{Conference\\/Journal\\/School} & Pages & Relevance &\shortstack{Cites\\OC XR\\SC} & \shortstack{Refs\\OC\\XR} & \shortstack{Links\\Cites\\Refs}\\ \midrule\endhead
\bottomrule
\endfoot
Kuramata2022 \href{http://dx.doi.org/10.1371/journal.pone.0266846}{Kuramata2022} & \hyperref[auth:a1690]{M. Kuramata}, \hyperref[auth:a1691]{R. Katsuki}, \hyperref[auth:a1692]{K. Nakata} & \cellcolor{gold!20}Solving large break minimization problems in a mirrored double round-robin tournament using quantum annealing \hyperref[abs:Kuramata2022]{Abstract} & \hyperref[detail:Kuramata2022]{Details} No & \cite{Kuramata2022} & 2022 & PLOS ONE & null & \noindent{}\textcolor{black!50}{0.00} \textcolor{black!50}{0.00} n/a & 0 0 0 & 25 36 & 4 0 4\\
\end{longtable}
}

\subsection{Pattern Recognition Letters}

\index{Pattern Recognition Letters}
{\scriptsize
\begin{longtable}{>{\raggedright\arraybackslash}p{2.5cm}>{\raggedright\arraybackslash}p{4.5cm}>{\raggedright\arraybackslash}p{6.0cm}p{1.0cm}rr>{\raggedright\arraybackslash}p{2.0cm}r>{\raggedright\arraybackslash}p{1cm}p{1cm}p{1cm}p{1cm}}
\rowcolor{white}\caption{Articles in Journal Pattern Recognition Letters (Total 1)}\\ \toprule
\rowcolor{white}\shortstack{Key\\Source} & Authors & Title (Colored by Open Access)& \shortstack{Details\\LC} & Cite & Year & \shortstack{Conference\\/Journal\\/School} & Pages & Relevance &\shortstack{Cites\\OC XR\\SC} & \shortstack{Refs\\OC\\XR} & \shortstack{Links\\Cites\\Refs}\\ \midrule\endhead
\bottomrule
\endfoot
Ortiz-Bayliss2013 \href{http://dx.doi.org/10.1016/j.patrec.2012.09.009}{Ortiz-Bayliss2013} & \hyperref[auth:a1781]{J. C. Ortiz-Bayliss}, \hyperref[auth:a1608]{H. Terashima-Marín}, \hyperref[auth:a1782]{S. E. Conant-Pablos} & Learning vector quantization for variable ordering in constraint satisfaction problems & \hyperref[detail:Ortiz-Bayliss2013]{Details} No & \cite{Ortiz-Bayliss2013} & 2013 & Pattern Recognition Letters & null & \noindent{}0.50 0.50 n/a & 22 22 26 & 15 50 & 6 4 2\\
\end{longtable}
}

\subsection{Pesquisa Operacional}

\index{Pesquisa Operacional}
{\scriptsize
\begin{longtable}{>{\raggedright\arraybackslash}p{2.5cm}>{\raggedright\arraybackslash}p{4.5cm}>{\raggedright\arraybackslash}p{6.0cm}p{1.0cm}rr>{\raggedright\arraybackslash}p{2.0cm}r>{\raggedright\arraybackslash}p{1cm}p{1cm}p{1cm}p{1cm}}
\rowcolor{white}\caption{Articles in Journal Pesquisa Operacional (Total 1)}\\ \toprule
\rowcolor{white}\shortstack{Key\\Source} & Authors & Title (Colored by Open Access)& \shortstack{Details\\LC} & Cite & Year & \shortstack{Conference\\/Journal\\/School} & Pages & Relevance &\shortstack{Cites\\OC XR\\SC} & \shortstack{Refs\\OC\\XR} & \shortstack{Links\\Cites\\Refs}\\ \midrule\endhead
\bottomrule
\endfoot
Magato2008 \href{http://dx.doi.org/10.1590/s0101-74382008000300007}{Magato2008} & \hyperref[auth:a1637]{L. Magatão}, \hyperref[auth:a1638]{Lúcia Valéria Ramos de Arruda}, \hyperref[auth:a1639]{F. Neves-Jr} & \cellcolor{gold!20}Um modelo híbrido (CLP-MILP) para scheduling de operações em polidutos \hyperref[abs:Magato2008]{Abstract} & \hyperref[detail:Magato2008]{Details} No & \cite{Magato2008} & 2008 & Pesquisa Operacional & null & \noindent{}\textbf{1.00} \textbf{2.00} n/a & 1 1 4 & 24 42 & 4 0 4\\
\end{longtable}
}

\subsection{Procedia CIRP}

\index{Procedia CIRP}
{\scriptsize
\begin{longtable}{>{\raggedright\arraybackslash}p{2.5cm}>{\raggedright\arraybackslash}p{4.5cm}>{\raggedright\arraybackslash}p{6.0cm}p{1.0cm}rr>{\raggedright\arraybackslash}p{2.0cm}r>{\raggedright\arraybackslash}p{1cm}p{1cm}p{1cm}p{1cm}}
\rowcolor{white}\caption{Articles in Journal Procedia CIRP (Total 1)}\\ \toprule
\rowcolor{white}\shortstack{Key\\Source} & Authors & Title (Colored by Open Access)& \shortstack{Details\\LC} & Cite & Year & \shortstack{Conference\\/Journal\\/School} & Pages & Relevance &\shortstack{Cites\\OC XR\\SC} & \shortstack{Refs\\OC\\XR} & \shortstack{Links\\Cites\\Refs}\\ \midrule\endhead
\bottomrule
\endfoot
Teschemacher2016 \href{http://dx.doi.org/10.1016/j.procir.2015.12.071}{Teschemacher2016} & \hyperref[auth:a1905]{U. Teschemacher}, \hyperref[auth:a1906]{G. Reinhart} & \cellcolor{gold!20}Enhancing Constraint Propagation in ACO-based Schedulers for Solving the Job Shop Scheduling Problem & \hyperref[detail:Teschemacher2016]{Details} No & \cite{Teschemacher2016} & 2016 & Procedia CIRP & null & \noindent{}\textbf{3.00} \textbf{3.00} n/a & 5 5 8 & 4 8 & 1 0 1\\
\end{longtable}
}

\subsection{Proceedings of the AAAI Conference on Artificial Intelligence}

\index{Proceedings of the AAAI Conference on Artificial Intelligence}
{\scriptsize
\begin{longtable}{>{\raggedright\arraybackslash}p{2.5cm}>{\raggedright\arraybackslash}p{4.5cm}>{\raggedright\arraybackslash}p{6.0cm}p{1.0cm}rr>{\raggedright\arraybackslash}p{2.0cm}r>{\raggedright\arraybackslash}p{1cm}p{1cm}p{1cm}p{1cm}}
\rowcolor{white}\caption{Articles in Journal Proceedings of the AAAI Conference on Artificial Intelligence (Total 1)}\\ \toprule
\rowcolor{white}\shortstack{Key\\Source} & Authors & Title (Colored by Open Access)& \shortstack{Details\\LC} & Cite & Year & \shortstack{Conference\\/Journal\\/School} & Pages & Relevance &\shortstack{Cites\\OC XR\\SC} & \shortstack{Refs\\OC\\XR} & \shortstack{Links\\Cites\\Refs}\\ \midrule\endhead
\bottomrule
\endfoot
Ouellet2022 \href{http://dx.doi.org/10.1609/aaai.v36i4.20296}{Ouellet2022} & \hyperref[auth:a52]{Y. Ouellet}, \hyperref[auth:a37]{C.-G. Quimper} & The SoftCumulative Constraint with Quadratic Penalty \hyperref[abs:Ouellet2022]{Abstract} & \hyperref[detail:Ouellet2022]{Details} No & \cite{Ouellet2022} & 2022 & Proceedings of the AAAI Conference on Artificial Intelligence & null & \noindent{}\textcolor{black!50}{0.00} \textbf{1.50} n/a & 1 0 0 & 0 0 & 1 1 0\\
\end{longtable}
}

\subsection{Proceedings of the Design Society: International Conference on Engineering Design}

\index{Proceedings of the Design Society: International Conference on Engineering Design}
{\scriptsize
\begin{longtable}{>{\raggedright\arraybackslash}p{2.5cm}>{\raggedright\arraybackslash}p{4.5cm}>{\raggedright\arraybackslash}p{6.0cm}p{1.0cm}rr>{\raggedright\arraybackslash}p{2.0cm}r>{\raggedright\arraybackslash}p{1cm}p{1cm}p{1cm}p{1cm}}
\rowcolor{white}\caption{Articles in Journal Proceedings of the Design Society: International Conference on Engineering Design (Total 1)}\\ \toprule
\rowcolor{white}\shortstack{Key\\Source} & Authors & Title (Colored by Open Access)& \shortstack{Details\\LC} & Cite & Year & \shortstack{Conference\\/Journal\\/School} & Pages & Relevance &\shortstack{Cites\\OC XR\\SC} & \shortstack{Refs\\OC\\XR} & \shortstack{Links\\Cites\\Refs}\\ \midrule\endhead
\bottomrule
\endfoot
Xidias2019 \href{http://dx.doi.org/10.1017/dsi.2019.292}{Xidias2019} & \hyperref[auth:a1989]{E. Xidias}, \hyperref[auth:a1990]{P. Azariadis} & \cellcolor{gold!20}Energy Efficient Motion Design and Task Scheduling for an Autonomous Vehicle \hyperref[abs:Xidias2019]{Abstract} & \hyperref[detail:Xidias2019]{Details} No & \cite{Xidias2019} & 2019 & Proceedings of the Design Society: International Conference on Engineering Design & null & \noindent{}\textcolor{black!50}{0.00} \textbf{2.00} n/a & 1 2 3 & 16 19 & 1 0 1\\
\end{longtable}
}

\subsection{Proceedings of the Institution of Mechanical Engineers, Part O: Journal of Risk and Reliability}

\index{Proceedings of the Institution of Mechanical Engineers, Part O: Journal of Risk and Reliability}
{\scriptsize
\begin{longtable}{>{\raggedright\arraybackslash}p{2.5cm}>{\raggedright\arraybackslash}p{4.5cm}>{\raggedright\arraybackslash}p{6.0cm}p{1.0cm}rr>{\raggedright\arraybackslash}p{2.0cm}r>{\raggedright\arraybackslash}p{1cm}p{1cm}p{1cm}p{1cm}}
\rowcolor{white}\caption{Articles in Journal Proceedings of the Institution of Mechanical Engineers, Part O: Journal of Risk and Reliability (Total 1)}\\ \toprule
\rowcolor{white}\shortstack{Key\\Source} & Authors & Title (Colored by Open Access)& \shortstack{Details\\LC} & Cite & Year & \shortstack{Conference\\/Journal\\/School} & Pages & Relevance &\shortstack{Cites\\OC XR\\SC} & \shortstack{Refs\\OC\\XR} & \shortstack{Links\\Cites\\Refs}\\ \midrule\endhead
\bottomrule
\endfoot
Mutha2017 \href{http://dx.doi.org/10.1177/1748006x17744380}{Mutha2017} & \hyperref[auth:a1957]{C. Mutha}, \hyperref[auth:a1958]{C. Smidts} & Basis for non-propagation domains, their transformations and their impact on software reliability \hyperref[abs:Mutha2017]{Abstract} & \hyperref[detail:Mutha2017]{Details} No & \cite{Mutha2017} & 2017 & Proceedings of the Institution of Mechanical Engineers, Part O: Journal of Risk and Reliability & null & \noindent{}\textcolor{black!50}{0.00} 0.25 n/a & 0 0 0 & 19 33 & 1 0 1\\
\end{longtable}
}

\subsection{Proceedings of the International Conference on Automated Planning and Scheduling}

\index{Proceedings of the International Conference on Automated Planning and Scheduling}
{\scriptsize
\begin{longtable}{>{\raggedright\arraybackslash}p{2.5cm}>{\raggedright\arraybackslash}p{4.5cm}>{\raggedright\arraybackslash}p{6.0cm}p{1.0cm}rr>{\raggedright\arraybackslash}p{2.0cm}r>{\raggedright\arraybackslash}p{1cm}p{1cm}p{1cm}p{1cm}}
\rowcolor{white}\caption{Articles in Journal Proceedings of the International Conference on Automated Planning and Scheduling (Total 5)}\\ \toprule
\rowcolor{white}\shortstack{Key\\Source} & Authors & Title (Colored by Open Access)& \shortstack{Details\\LC} & Cite & Year & \shortstack{Conference\\/Journal\\/School} & Pages & Relevance &\shortstack{Cites\\OC XR\\SC} & \shortstack{Refs\\OC\\XR} & \shortstack{Links\\Cites\\Refs}\\ \midrule\endhead
\bottomrule
\endfoot
Danzinger2020 \href{http://dx.doi.org/10.1609/icaps.v30i1.6681}{Danzinger2020} & \hyperref[auth:a1484]{P. Danzinger}, \hyperref[auth:a77]{T. Geibinger}, \hyperref[auth:a80]{F. Mischek}, \hyperref[auth:a45]{N. Musliu} & Solving the Test Laboratory Scheduling Problem with Variable Task Grouping \hyperref[abs:Danzinger2020]{Abstract} & \hyperref[detail:Danzinger2020]{Details} No & \cite{Danzinger2020} & 2020 & Proceedings of the International Conference on Automated Planning and Scheduling & null & \noindent{}\textcolor{black!50}{0.00} \textbf{3.50} n/a & 1 2 0 & 0 0 & 1 1 0\\
Gonzlez2017 \href{http://dx.doi.org/10.1609/icaps.v27i1.13809}{Gonzlez2017} & \hyperref[auth:a1828]{M. Ángel González}, \hyperref[auth:a282]{A. Oddi}, \hyperref[auth:a1270]{R. Rasconi} & Multi-Objective Optimization in a Job Shop with Energy Costs through Hybrid Evolutionary Techniques \hyperref[abs:Gonzlez2017]{Abstract} & \hyperref[detail:Gonzlez2017]{Details} No & \cite{Gonzlez2017} & 2017 & Proceedings of the International Conference on Automated Planning and Scheduling & null & \noindent{}\textcolor{black!50}{0.00} \textbf{2.00} n/a & 10 12 0 & 0 0 & 1 1 0\\
Laborie2017 \href{http://dx.doi.org/10.1609/icaps.v27i1.13844}{Laborie2017} & \hyperref[auth:a118]{P. Laborie}, \hyperref[auth:a1550]{B. Messaoudi} & New Results for the GEO-CAPE Observation Scheduling Problem \hyperref[abs:Laborie2017]{Abstract} & \hyperref[detail:Laborie2017]{Details} No & \cite{Laborie2017} & 2017 & Proceedings of the International Conference on Automated Planning and Scheduling & null & \noindent{}\textcolor{black!50}{0.00} \textbf{2.00} n/a & 2 2 0 & 0 0 & 2 2 0\\
Levine2014 \href{http://dx.doi.org/10.1609/icaps.v24i1.13672}{Levine2014} & \hyperref[auth:a1927]{S. Levine}, \hyperref[auth:a1928]{B. Williams} & Concurrent Plan Recognition and Execution for Human-Robot Teams \hyperref[abs:Levine2014]{Abstract} & \hyperref[detail:Levine2014]{Details} No & \cite{Levine2014} & 2014 & Proceedings of the International Conference on Automated Planning and Scheduling & null & \noindent{}\textcolor{black!50}{0.00} 0.50 n/a & 17 20 0 & 0 0 & 1 1 0\\
Kelareva2012 \href{http://dx.doi.org/10.1609/icaps.v22i1.13494}{Kelareva2012} & \hyperref[auth:a332]{E. Kelareva}, \hyperref[auth:a855]{S. Brand}, \hyperref[auth:a334]{P. Kilby}, \hyperref[auth:a1518]{S. Thiebaux}, \hyperref[auth:a1519]{M. Wallace} & CP and MIP Methods for Ship Scheduling with Time-Varying Draft \hyperref[abs:Kelareva2012]{Abstract} & \hyperref[detail:Kelareva2012]{Details} No & \cite{Kelareva2012} & 2012 & Proceedings of the International Conference on Automated Planning and Scheduling & null & \noindent{}\textcolor{black!50}{0.00} \textbf{3.00} n/a & 11 14 0 & 0 0 & 5 5 0\\
\end{longtable}
}

\subsection{Proceedings of the VLDB Endowment}

\index{Proceedings of the VLDB Endowment}
{\scriptsize
\begin{longtable}{>{\raggedright\arraybackslash}p{2.5cm}>{\raggedright\arraybackslash}p{4.5cm}>{\raggedright\arraybackslash}p{6.0cm}p{1.0cm}rr>{\raggedright\arraybackslash}p{2.0cm}r>{\raggedright\arraybackslash}p{1cm}p{1cm}p{1cm}p{1cm}}
\rowcolor{white}\caption{Articles in Journal Proceedings of the VLDB Endowment (Total 1)}\\ \toprule
\rowcolor{white}\shortstack{Key\\Source} & Authors & Title (Colored by Open Access)& \shortstack{Details\\LC} & Cite & Year & \shortstack{Conference\\/Journal\\/School} & Pages & Relevance &\shortstack{Cites\\OC XR\\SC} & \shortstack{Refs\\OC\\XR} & \shortstack{Links\\Cites\\Refs}\\ \midrule\endhead
\bottomrule
\endfoot
Zou2012 \href{http://dx.doi.org/10.14778/2535568.2448945}{Zou2012} & \hyperref[auth:a2054]{T. Zou}, \hyperref[auth:a2055]{R. L. Bras}, \hyperref[auth:a2056]{M. V. Salles}, \hyperref[auth:a2057]{A. Demers}, \hyperref[auth:a2058]{J. Gehrke} & ClouDiA \hyperref[abs:Zou2012]{Abstract} & \hyperref[detail:Zou2012]{Details} No & \cite{Zou2012} & 2012 & Proceedings of the VLDB Endowment & null & \noindent{}\textcolor{black!50}{0.00} \textbf{1.00} n/a & 17 17 14 & 26 69 & 1 0 1\\
\end{longtable}
}

\subsection{Production and Operations Management}

\index{Production and Operations Management}
{\scriptsize
\begin{longtable}{>{\raggedright\arraybackslash}p{2.5cm}>{\raggedright\arraybackslash}p{4.5cm}>{\raggedright\arraybackslash}p{6.0cm}p{1.0cm}rr>{\raggedright\arraybackslash}p{2.0cm}r>{\raggedright\arraybackslash}p{1cm}p{1cm}p{1cm}p{1cm}}
\rowcolor{white}\caption{Articles in Journal Production and Operations Management (Total 3)}\\ \toprule
\rowcolor{white}\shortstack{Key\\Source} & Authors & Title (Colored by Open Access)& \shortstack{Details\\LC} & Cite & Year & \shortstack{Conference\\/Journal\\/School} & Pages & Relevance &\shortstack{Cites\\OC XR\\SC} & \shortstack{Refs\\OC\\XR} & \shortstack{Links\\Cites\\Refs}\\ \midrule\endhead
\bottomrule
\endfoot
Kasapidis2023 \href{http://dx.doi.org/10.1111/poms.13977}{Kasapidis2023} & \hyperref[auth:a1503]{G. A. Kasapidis}, \hyperref[auth:a1716]{S. Dauzère‐Pérès}, \hyperref[auth:a1504]{D. C. Paraskevopoulos}, \hyperref[auth:a1505]{P. P. Repoussis}, \hyperref[auth:a1506]{C. D. Tarantilis} & \cellcolor{gold!20}On the multiresource flexible job‐shop scheduling problem with arbitrary precedence graphs \hyperref[abs:Kasapidis2023]{Abstract} & \hyperref[detail:Kasapidis2023]{Details} No & \cite{Kasapidis2023} & 2023 & \cellcolor{red!20}Production and Operations Management & null & \noindent{}\textcolor{black!50}{0.00} \textbf{5.00} n/a & 1 4 4 & 13 13 & 2 0 2\\
Kasapidis2021 \href{http://dx.doi.org/10.1111/poms.13501}{Kasapidis2021} & \hyperref[auth:a1503]{G. A. Kasapidis}, \hyperref[auth:a1504]{D. C. Paraskevopoulos}, \hyperref[auth:a1505]{P. P. Repoussis}, \hyperref[auth:a1506]{C. D. Tarantilis} & \cellcolor{green!10}Flexible Job Shop Scheduling Problems with Arbitrary Precedence Graphs \hyperref[abs:Kasapidis2021]{Abstract} & \hyperref[detail:Kasapidis2021]{Details} No & \cite{Kasapidis2021} & 2021 & \cellcolor{red!20}Production and Operations Management & null & \noindent{}\textcolor{black!50}{0.00} \textbf{3.00} n/a & 7 12 10 & 36 43 & 6 1 5\\
NaderiRBAU21 \href{http://dx.doi.org/10.1111/poms.13397}{NaderiRBAU21} & \hyperref[auth:a726]{B. Naderi}, \hyperref[auth:a728]{V. Roshanaei}, \hyperref[auth:a836]{M. A. Begen}, \hyperref[auth:a895]{D. M. Aleman}, \hyperref[auth:a896]{D. R. Urbach} & Increased Surgical Capacity without Additional Resources: Generalized Operating Room Planning and Scheduling & \hyperref[detail:NaderiRBAU21]{Details} No & \cite{NaderiRBAU21} & 2021 & \cellcolor{red!20}Production and Operations Management & 28 & \noindent{}\textcolor{black!50}{0.00} \textcolor{black!50}{0.00} n/a & 22 23 23 & 61 66 & 22 5 17\\
\end{longtable}
}

\subsection{Projectics / Proyéctica / Projectique}

\index{Projectics / Proyéctica / Projectique}
{\scriptsize
\begin{longtable}{>{\raggedright\arraybackslash}p{2.5cm}>{\raggedright\arraybackslash}p{4.5cm}>{\raggedright\arraybackslash}p{6.0cm}p{1.0cm}rr>{\raggedright\arraybackslash}p{2.0cm}r>{\raggedright\arraybackslash}p{1cm}p{1cm}p{1cm}p{1cm}}
\rowcolor{white}\caption{Articles in Journal Projectics / Proyéctica / Projectique (Total 1)}\\ \toprule
\rowcolor{white}\shortstack{Key\\Source} & Authors & Title (Colored by Open Access)& \shortstack{Details\\LC} & Cite & Year & \shortstack{Conference\\/Journal\\/School} & Pages & Relevance &\shortstack{Cites\\OC XR\\SC} & \shortstack{Refs\\OC\\XR} & \shortstack{Links\\Cites\\Refs}\\ \midrule\endhead
\bottomrule
\endfoot
Lizarralde2011 \href{http://dx.doi.org/10.3917/proj.007.0089}{Lizarralde2011} & \hyperref[auth:a1478]{I. Lizarralde}, \hyperref[auth:a1248]{P. Esquirol}, \hyperref[auth:a1479]{A. Rivière} & A decision support system to schedule design activities with interdependency and resource constraints \hyperref[abs:Lizarralde2011]{Abstract} & \hyperref[detail:Lizarralde2011]{Details} No & \cite{Lizarralde2011} & 2011 & Projectics / Proyéctica / Projectique & null & \noindent{}\textcolor{black!50}{0.00} \textbf{4.50} n/a & 1 1 0 & 12 34 & 7 1 6\\
\end{longtable}
}

\subsection{RAIRO - Operations Research}

\index{RAIRO - Operations Research}
{\scriptsize
\begin{longtable}{>{\raggedright\arraybackslash}p{2.5cm}>{\raggedright\arraybackslash}p{4.5cm}>{\raggedright\arraybackslash}p{6.0cm}p{1.0cm}rr>{\raggedright\arraybackslash}p{2.0cm}r>{\raggedright\arraybackslash}p{1cm}p{1cm}p{1cm}p{1cm}}
\rowcolor{white}\caption{Articles in Journal RAIRO - Operations Research (Total 3)}\\ \toprule
\rowcolor{white}\shortstack{Key\\Source} & Authors & Title (Colored by Open Access)& \shortstack{Details\\LC} & Cite & Year & \shortstack{Conference\\/Journal\\/School} & Pages & Relevance &\shortstack{Cites\\OC XR\\SC} & \shortstack{Refs\\OC\\XR} & \shortstack{Links\\Cites\\Refs}\\ \midrule\endhead
\bottomrule
\endfoot
Hosseinian2021 \href{http://dx.doi.org/10.1051/ro/2021087}{Hosseinian2021} & \hyperref[auth:a1573]{A. H. Hosseinian}, \hyperref[auth:a1574]{V. Baradaran} & \cellcolor{gold!20}A multi-objective multi-agent optimization algorithm for the multi-skill resource-constrained project scheduling problem with transfer times \hyperref[abs:Hosseinian2021]{Abstract} & \hyperref[detail:Hosseinian2021]{Details} No & \cite{Hosseinian2021} & 2021 & RAIRO - Operations Research & null & \noindent{}\textcolor{black!50}{0.00} \textcolor{black!50}{0.00} n/a & 3 10 10 & 65 77 & 5 0 5\\
Sahli2021 \href{http://dx.doi.org/10.1051/ro/2021164}{Sahli2021} & \hyperref[auth:a928]{A. Sahli}, \hyperref[auth:a845]{J. Carlier}, \hyperref[auth:a1170]{A. Moukrim} & \cellcolor{gold!20}Polynomial algorithms for some scheduling problems with one nonrenewable resource \hyperref[abs:Sahli2021]{Abstract} & \hyperref[detail:Sahli2021]{Details} No & \cite{Sahli2021} & 2021 & RAIRO - Operations Research & null & \noindent{}\textcolor{black!50}{0.00} \textcolor{black!50}{0.00} n/a & 0 0 0 & 26 27 & 6 0 6\\
Richard1998 \href{http://dx.doi.org/10.1051/ro/1998320201251}{Richard1998} & \hyperref[auth:a1684]{P. Richard}, \hyperref[auth:a1685]{C. Proust} & \cellcolor{gold!20}Solving scheduling problems using Petri nets and constraint logic programming & \hyperref[detail:Richard1998]{Details} No & \cite{Richard1998} & 1998 & RAIRO - Operations Research & null & \noindent{}\textbf{1.00} \textbf{1.00} n/a & 4 4 6 & 15 43 & 3 0 3\\
\end{longtable}
}

\subsection{Results in Control and Optimization}

\index{Results in Control and Optimization}
{\scriptsize
\begin{longtable}{>{\raggedright\arraybackslash}p{2.5cm}>{\raggedright\arraybackslash}p{4.5cm}>{\raggedright\arraybackslash}p{6.0cm}p{1.0cm}rr>{\raggedright\arraybackslash}p{2.0cm}r>{\raggedright\arraybackslash}p{1cm}p{1cm}p{1cm}p{1cm}}
\rowcolor{white}\caption{Articles in Journal Results in Control and Optimization (Total 1)}\\ \toprule
\rowcolor{white}\shortstack{Key\\Source} & Authors & Title (Colored by Open Access)& \shortstack{Details\\LC} & Cite & Year & \shortstack{Conference\\/Journal\\/School} & Pages & Relevance &\shortstack{Cites\\OC XR\\SC} & \shortstack{Refs\\OC\\XR} & \shortstack{Links\\Cites\\Refs}\\ \midrule\endhead
\bottomrule
\endfoot
PrataAN23 \href{https://www.sciencedirect.com/science/article/pii/S2666720723001522}{PrataAN23} & \hyperref[auth:a385]{B. A. Prata}, \hyperref[auth:a386]{L. R. Abreu}, \hyperref[auth:a387]{M. S. Nagano} & \cellcolor{gold!20}Applications of constraint programming in production scheduling problems: A descriptive bibliometric analysis \hyperref[abs:PrataAN23]{Abstract} & \hyperref[detail:PrataAN23]{Details} \href{../works/PrataAN23.pdf}{Yes} & \cite{PrataAN23} & 2024 & Results in Control and Optimization & 17 & \noindent{}\textbf{1.00} \textbf{1.00} \textbf{54.10} & 0 0 0 & 0 149 & 0 0 0\\
\end{longtable}
}

\subsection{Robotics and Computer-Integrated Manufacturing}

\index{Robotics and Computer-Integrated Manufacturing}
{\scriptsize
\begin{longtable}{>{\raggedright\arraybackslash}p{2.5cm}>{\raggedright\arraybackslash}p{4.5cm}>{\raggedright\arraybackslash}p{6.0cm}p{1.0cm}rr>{\raggedright\arraybackslash}p{2.0cm}r>{\raggedright\arraybackslash}p{1cm}p{1cm}p{1cm}p{1cm}}
\rowcolor{white}\caption{Articles in Journal Robotics and Computer-Integrated Manufacturing (Total 1)}\\ \toprule
\rowcolor{white}\shortstack{Key\\Source} & Authors & Title (Colored by Open Access)& \shortstack{Details\\LC} & Cite & Year & \shortstack{Conference\\/Journal\\/School} & Pages & Relevance &\shortstack{Cites\\OC XR\\SC} & \shortstack{Refs\\OC\\XR} & \shortstack{Links\\Cites\\Refs}\\ \midrule\endhead
\bottomrule
\endfoot
Zeballos10 \href{http://dx.doi.org/10.1016/j.rcim.2010.04.005}{Zeballos10} & \hyperref[auth:a621]{L. J. Zeballos} & A constraint programming approach to tool allocation and production scheduling in flexible manufacturing systems & \hyperref[detail:Zeballos10]{Details} \href{../works/Zeballos10.pdf}{Yes} & \cite{Zeballos10} & 2010 & Robotics and Computer-Integrated Manufacturing & 19 & \noindent{}\textbf{1.00} \textbf{1.00} \textbf{16.15} & 41 42 51 & 16 23 & 13 7 6\\
\end{longtable}
}

\subsection{SIMULATION}

\index{SIMULATION}
{\scriptsize
\begin{longtable}{>{\raggedright\arraybackslash}p{2.5cm}>{\raggedright\arraybackslash}p{4.5cm}>{\raggedright\arraybackslash}p{6.0cm}p{1.0cm}rr>{\raggedright\arraybackslash}p{2.0cm}r>{\raggedright\arraybackslash}p{1cm}p{1cm}p{1cm}p{1cm}}
\rowcolor{white}\caption{Articles in Journal SIMULATION (Total 1)}\\ \toprule
\rowcolor{white}\shortstack{Key\\Source} & Authors & Title (Colored by Open Access)& \shortstack{Details\\LC} & Cite & Year & \shortstack{Conference\\/Journal\\/School} & Pages & Relevance &\shortstack{Cites\\OC XR\\SC} & \shortstack{Refs\\OC\\XR} & \shortstack{Links\\Cites\\Refs}\\ \midrule\endhead
\bottomrule
\endfoot
Biswas2010 \href{http://dx.doi.org/10.1177/0037549710373601}{Biswas2010} & \hyperref[auth:a2019]{M. Biswas}, \hyperref[auth:a2020]{M. R. Frater}, \hyperref[auth:a2021]{M. Ryan} & Determination of hub port attenuation satisfying radio link path losses for hardware emulation \hyperref[abs:Biswas2010]{Abstract} & \hyperref[detail:Biswas2010]{Details} No & \cite{Biswas2010} & 2010 & SIMULATION & null & \noindent{}\textcolor{black!50}{0.00} \textbf{1.00} n/a & 0 1 1 & 2 9 & 1 0 1\\
\end{longtable}
}

\subsection{SN Computer Science}

\index{SN Computer Science}
{\scriptsize
\begin{longtable}{>{\raggedright\arraybackslash}p{2.5cm}>{\raggedright\arraybackslash}p{4.5cm}>{\raggedright\arraybackslash}p{6.0cm}p{1.0cm}rr>{\raggedright\arraybackslash}p{2.0cm}r>{\raggedright\arraybackslash}p{1cm}p{1cm}p{1cm}p{1cm}}
\rowcolor{white}\caption{Articles in Journal SN Computer Science (Total 1)}\\ \toprule
\rowcolor{white}\shortstack{Key\\Source} & Authors & Title (Colored by Open Access)& \shortstack{Details\\LC} & Cite & Year & \shortstack{Conference\\/Journal\\/School} & Pages & Relevance &\shortstack{Cites\\OC XR\\SC} & \shortstack{Refs\\OC\\XR} & \shortstack{Links\\Cites\\Refs}\\ \midrule\endhead
\bottomrule
\endfoot
EtminaniesfahaniGNMS22 \href{http://dx.doi.org/10.1007/s42979-022-01487-1}{EtminaniesfahaniGNMS22} & \hyperref[auth:a901]{A. Etminaniesfahani}, \hyperref[auth:a336]{H. Gu}, \hyperref[auth:a902]{L. M. Naeni}, \hyperref[auth:a903]{A. Salehipour} & A Forward–Backward Relax-and-Solve Algorithm for the Resource-Constrained Project Scheduling Problem & \hyperref[detail:EtminaniesfahaniGNMS22]{Details} \href{../works/EtminaniesfahaniGNMS22.pdf}{Yes} & \cite{EtminaniesfahaniGNMS22} & 2022 & SN Computer Science & 10 & \noindent{}\textcolor{black!50}{0.00} \textcolor{black!50}{0.00} \textbf{8.63} & 0 1 2 & 57 66 & 17 0 17\\
\end{longtable}
}

\subsection{SSRN Electronic Journal}

\index{SSRN Electronic Journal}
{\scriptsize
\begin{longtable}{>{\raggedright\arraybackslash}p{2.5cm}>{\raggedright\arraybackslash}p{4.5cm}>{\raggedright\arraybackslash}p{6.0cm}p{1.0cm}rr>{\raggedright\arraybackslash}p{2.0cm}r>{\raggedright\arraybackslash}p{1cm}p{1cm}p{1cm}p{1cm}}
\rowcolor{white}\caption{Articles in Journal SSRN Electronic Journal (Total 4)}\\ \toprule
\rowcolor{white}\shortstack{Key\\Source} & Authors & Title (Colored by Open Access)& \shortstack{Details\\LC} & Cite & Year & \shortstack{Conference\\/Journal\\/School} & Pages & Relevance &\shortstack{Cites\\OC XR\\SC} & \shortstack{Refs\\OC\\XR} & \shortstack{Links\\Cites\\Refs}\\ \midrule\endhead
\bottomrule
\endfoot
NaderiBZ23 \href{http://dx.doi.org/10.2139/ssrn.4494381}{NaderiBZ23} & \hyperref[auth:a726]{B. Naderi}, \hyperref[auth:a836]{M. A. Begen}, \hyperref[auth:a837]{G. Zhang} & Integrated Order Acceptance and Resource Decisions Under Uncertainty: Robust and Stochastic Approaches & \hyperref[detail:NaderiBZ23]{Details} \href{../works/NaderiBZ23.pdf}{Yes} & \cite{NaderiBZ23} & 2023 & SSRN Electronic Journal & 32 & \noindent{}\textcolor{black!50}{0.00} \textcolor{black!50}{0.00} \textbf{10.49} & 0 0 0 & 46 56 & 12 0 12\\
JuvinHL22 \href{http://dx.doi.org/10.2139/ssrn.4068164}{JuvinHL22} & \hyperref[auth:a0]{C. Juvin}, \hyperref[auth:a2]{L. Houssin}, \hyperref[auth:a3]{P. Lopez} & Logic-Based Benders Decomposition for the Preemptive Flexible Job-Shop Scheduling Problem & \hyperref[detail:JuvinHL22]{Details} \href{../works/JuvinHL22.pdf}{Yes} & \cite{JuvinHL22} & 2022 & SSRN Electronic Journal & 32 & \noindent{}\textcolor{black!50}{0.00} \textcolor{black!50}{0.00} \textbf{12.00} & 0 0 0 & 29 40 & 12 0 12\\
NaderiBZ22 \href{http://dx.doi.org/10.2139/ssrn.4140716}{NaderiBZ22} & \hyperref[auth:a726]{B. Naderi}, \hyperref[auth:a836]{M. A. Begen}, \hyperref[auth:a837]{G. Zhang} & Integrated Order Acceptance and Resource Decisions Under Uncertainty: Robust and Stochastic Approaches & \hyperref[detail:NaderiBZ22]{Details} \href{../works/NaderiBZ22.pdf}{Yes} & \cite{NaderiBZ22} & 2022 & SSRN Electronic Journal & 29 & \noindent{}\textcolor{black!50}{0.00} \textcolor{black!50}{0.00} \textbf{9.27} & 0 0 0 & 44 51 & 11 0 11\\
Sadykov2003 \href{http://dx.doi.org/10.2139/ssrn.988640}{Sadykov2003} & \hyperref[auth:a384]{R. Sadykov}, \hyperref[auth:a224]{L. A. Wolsey} & Integer Programming and Constraint Programming in Solving a Multi-Machine Assignment Scheduling Problem With Deadlines and Release Dates & \hyperref[detail:Sadykov2003]{Details} No & \cite{Sadykov2003} & 2003 & SSRN Electronic Journal & null & \noindent{}\textbf{1.50} \textbf{1.50} n/a & 3 3 0 & 8 11 & 5 1 4\\
\end{longtable}
}

\subsection{Science}

\index{Science}
{\scriptsize
\begin{longtable}{>{\raggedright\arraybackslash}p{2.5cm}>{\raggedright\arraybackslash}p{4.5cm}>{\raggedright\arraybackslash}p{6.0cm}p{1.0cm}rr>{\raggedright\arraybackslash}p{2.0cm}r>{\raggedright\arraybackslash}p{1cm}p{1cm}p{1cm}p{1cm}}
\rowcolor{white}\caption{Articles in Journal Science (Total 1)}\\ \toprule
\rowcolor{white}\shortstack{Key\\Source} & Authors & Title (Colored by Open Access)& \shortstack{Details\\LC} & Cite & Year & \shortstack{Conference\\/Journal\\/School} & Pages & Relevance &\shortstack{Cites\\OC XR\\SC} & \shortstack{Refs\\OC\\XR} & \shortstack{Links\\Cites\\Refs}\\ \midrule\endhead
\bottomrule
\endfoot
Clearwater1991 \href{http://dx.doi.org/10.1126/science.254.5035.1181}{Clearwater1991} & \hyperref[auth:a1776]{S. H. Clearwater}, \hyperref[auth:a1777]{B. A. Huberman}, \hyperref[auth:a1778]{T. Hogg} & Cooperative Solution of Constraint Satisfaction Problems \hyperref[abs:Clearwater1991]{Abstract} & \hyperref[detail:Clearwater1991]{Details} No & \cite{Clearwater1991} & 1991 & Science & null & \noindent{}\textcolor{black!50}{0.00} \textbf{1.00} n/a & 91 91 93 & 4 8 & 3 3 0\\
\end{longtable}
}

\subsection{Scientific Programming}

\index{Scientific Programming}
{\scriptsize
\begin{longtable}{>{\raggedright\arraybackslash}p{2.5cm}>{\raggedright\arraybackslash}p{4.5cm}>{\raggedright\arraybackslash}p{6.0cm}p{1.0cm}rr>{\raggedright\arraybackslash}p{2.0cm}r>{\raggedright\arraybackslash}p{1cm}p{1cm}p{1cm}p{1cm}}
\rowcolor{white}\caption{Articles in Journal Scientific Programming (Total 1)}\\ \toprule
\rowcolor{white}\shortstack{Key\\Source} & Authors & Title (Colored by Open Access)& \shortstack{Details\\LC} & Cite & Year & \shortstack{Conference\\/Journal\\/School} & Pages & Relevance &\shortstack{Cites\\OC XR\\SC} & \shortstack{Refs\\OC\\XR} & \shortstack{Links\\Cites\\Refs}\\ \midrule\endhead
\bottomrule
\endfoot
Sitek2016 \href{http://dx.doi.org/10.1155/2016/5102616}{Sitek2016} & \hyperref[auth:a1475]{P. Sitek}, \hyperref[auth:a1476]{J. Wikarek} & \cellcolor{gold!20}A Hybrid Programming Framework for Modeling and Solving Constraint Satisfaction and Optimization Problems \hyperref[abs:Sitek2016]{Abstract} & \hyperref[detail:Sitek2016]{Details} No & \cite{Sitek2016} & 2016 & Scientific Programming & null & \noindent{}\textcolor{black!50}{0.00} \textbf{9.01} n/a & 40 39 57 & 11 15 & 9 3 6\\
\end{longtable}
}

\subsection{Soft Computing}

\index{Soft Computing}
{\scriptsize
\begin{longtable}{>{\raggedright\arraybackslash}p{2.5cm}>{\raggedright\arraybackslash}p{4.5cm}>{\raggedright\arraybackslash}p{6.0cm}p{1.0cm}rr>{\raggedright\arraybackslash}p{2.0cm}r>{\raggedright\arraybackslash}p{1cm}p{1cm}p{1cm}p{1cm}}
\rowcolor{white}\caption{Articles in Journal Soft Computing (Total 5)}\\ \toprule
\rowcolor{white}\shortstack{Key\\Source} & Authors & Title (Colored by Open Access)& \shortstack{Details\\LC} & Cite & Year & \shortstack{Conference\\/Journal\\/School} & Pages & Relevance &\shortstack{Cites\\OC XR\\SC} & \shortstack{Refs\\OC\\XR} & \shortstack{Links\\Cites\\Refs}\\ \midrule\endhead
\bottomrule
\endfoot
AlakaP23 \href{http://dx.doi.org/10.1007/s00500-023-09105-9}{AlakaP23} & \hyperref[auth:a764]{H. M. Alakaş}, \hyperref[auth:a1384]{M. Pınarbaşı} & \cellcolor{green!10}Balancing of cost-oriented U-type general resource-constrained assembly line: new constraint programming models & \hyperref[detail:AlakaP23]{Details} \href{../works/AlakaP23.pdf}{Yes} & \cite{AlakaP23} & 2023 & Soft Computing & 14 & \noindent{}0.50 0.50 \textbf{23.23} & 0 0 0 & 35 42 & 14 0 14\\
IsikYA23 \href{https://doi.org/10.1007/s00500-023-09086-9}{IsikYA23} & \hyperref[auth:a420]{E. E. Isik}, \hyperref[auth:a421]{S. T. Yildiz}, \hyperref[auth:a422]{{\"{O}}zge S. Akpunar} & Constraint programming models for the hybrid flow shop scheduling problem and its extensions & \hyperref[detail:IsikYA23]{Details} \href{../works/IsikYA23.pdf}{Yes} & \cite{IsikYA23} & 2023 & Soft Computing & 28 & \noindent{}\textbf{1.00} \textbf{1.00} \textbf{80.43} & 0 2 2 & 127 141 & 12 0 12\\
SubulanC22 \href{https://doi.org/10.1007/s00500-021-06399-5}{SubulanC22} & \hyperref[auth:a451]{K. Subulan}, \hyperref[auth:a452]{G. {\c{C}}akir} & Constraint programming-based transformation approach for a mixed fuzzy-stochastic resource investment project scheduling problem & \hyperref[detail:SubulanC22]{Details} \href{../works/SubulanC22.pdf}{Yes} & \cite{SubulanC22} & 2022 & Soft Computing & 38 & \noindent{}\textbf{1.50} \textbf{1.50} \textbf{44.71} & 5 7 7 & 86 107 & 5 0 5\\
Alaka21 \href{http://dx.doi.org/10.1007/s00500-021-05602-x}{Alaka21} & \hyperref[auth:a764]{H. M. Alakaş} & General resource-constrained assembly line balancing problem: conjunction normal form based constraint programming models & \hyperref[detail:Alaka21]{Details} \href{../works/Alaka21.pdf}{Yes} & \cite{Alaka21} & 2021 & Soft Computing & 11 & \noindent{}0.50 0.50 \textbf{19.49} & 7 9 9 & 20 27 & 11 2 9\\
AlakaPY19 \href{http://dx.doi.org/10.1007/s00500-019-04294-8}{AlakaPY19} & \hyperref[auth:a764]{H. M. Alakaş}, \hyperref[auth:a1384]{M. Pınarbaşı}, \hyperref[auth:a1425]{M. Y\"{u}z\"{u}kırmızı} & Constraint programming model for resource-constrained assembly line balancing problem & \hyperref[detail:AlakaPY19]{Details} \href{../works/AlakaPY19.pdf}{Yes} & \cite{AlakaPY19} & 2019 & Soft Computing & 9 & \noindent{}0.50 0.50 \textbf{14.19} & 15 17 0 & 14 23 & 11 6 5\\
\end{longtable}
}

\subsection{Software: Practice and Experience}

\index{Software: Practice and Experience}
{\scriptsize
\begin{longtable}{>{\raggedright\arraybackslash}p{2.5cm}>{\raggedright\arraybackslash}p{4.5cm}>{\raggedright\arraybackslash}p{6.0cm}p{1.0cm}rr>{\raggedright\arraybackslash}p{2.0cm}r>{\raggedright\arraybackslash}p{1cm}p{1cm}p{1cm}p{1cm}}
\rowcolor{white}\caption{Articles in Journal Software: Practice and Experience (Total 1)}\\ \toprule
\rowcolor{white}\shortstack{Key\\Source} & Authors & Title (Colored by Open Access)& \shortstack{Details\\LC} & Cite & Year & \shortstack{Conference\\/Journal\\/School} & Pages & Relevance &\shortstack{Cites\\OC XR\\SC} & \shortstack{Refs\\OC\\XR} & \shortstack{Links\\Cites\\Refs}\\ \midrule\endhead
\bottomrule
\endfoot
BosiM2001 \href{http://dx.doi.org/10.1002/1097-024x(200101)31:1<17::aid-spe355>3.0.co;2-l}{BosiM2001} & \hyperref[auth:a1224]{F. Bosi}, \hyperref[auth:a143]{M. Milano} & Enhancing CLP branch and bound techniques for scheduling problems & \hyperref[detail:BosiM2001]{Details} \href{../works/BosiM2001.pdf}{Yes} & \cite{BosiM2001} & 2001 & Software: Practice and Experience & 26 & \noindent{}\textbf{1.00} \textbf{1.00} \textbf{65.58} & 3 3 0 & 12 41 & 9 0 9\\
\end{longtable}
}

\subsection{Sustain. Comput. Informatics Syst.}

\index{Sustain. Comput. Informatics Syst.}
{\scriptsize
\begin{longtable}{>{\raggedright\arraybackslash}p{2.5cm}>{\raggedright\arraybackslash}p{4.5cm}>{\raggedright\arraybackslash}p{6.0cm}p{1.0cm}rr>{\raggedright\arraybackslash}p{2.0cm}r>{\raggedright\arraybackslash}p{1cm}p{1cm}p{1cm}p{1cm}}
\rowcolor{white}\caption{Articles in Journal Sustain. Comput. Informatics Syst. (Total 2)}\\ \toprule
\rowcolor{white}\shortstack{Key\\Source} & Authors & Title (Colored by Open Access)& \shortstack{Details\\LC} & Cite & Year & \shortstack{Conference\\/Journal\\/School} & Pages & Relevance &\shortstack{Cites\\OC XR\\SC} & \shortstack{Refs\\OC\\XR} & \shortstack{Links\\Cites\\Refs}\\ \midrule\endhead
\bottomrule
\endfoot
BorghesiBLMB18 \href{https://doi.org/10.1016/j.suscom.2018.05.007}{BorghesiBLMB18} & \hyperref[auth:a226]{A. Borghesi}, \hyperref[auth:a225]{A. Bartolini}, \hyperref[auth:a142]{M. Lombardi}, \hyperref[auth:a143]{M. Milano}, \hyperref[auth:a245]{L. Benini} & \cellcolor{green!10}Scheduling-based power capping in high performance computing systems & \hyperref[detail:BorghesiBLMB18]{Details} \href{../works/BorghesiBLMB18.pdf}{Yes} & \cite{BorghesiBLMB18} & 2018 & Sustain. Comput. Informatics Syst. & 13 & \noindent{}\textcolor{black!50}{0.00} \textcolor{black!50}{0.00} \textbf{9.54} & 11 12 19 & 22 66 & 2 0 2\\
GrimesIOS14 \href{https://doi.org/10.1016/j.suscom.2014.08.009}{GrimesIOS14} & \hyperref[auth:a181]{D. Grimes}, \hyperref[auth:a182]{G. Ifrim}, \hyperref[auth:a16]{B. O'Sullivan}, \hyperref[auth:a17]{H. Simonis} & \cellcolor{green!10}Analyzing the impact of electricity price forecasting on energy cost-aware scheduling & \hyperref[detail:GrimesIOS14]{Details} \href{../works/GrimesIOS14.pdf}{Yes} & \cite{GrimesIOS14} & 2014 & Sustain. Comput. Informatics Syst. & 16 & \noindent{}\textcolor{black!50}{0.00} \textcolor{black!50}{0.00} 0.85 & 6 6 19 & 7 28 & 1 0 1\\
\end{longtable}
}

\subsection{Sustainability}

\index{Sustainability}
{\scriptsize
\begin{longtable}{>{\raggedright\arraybackslash}p{2.5cm}>{\raggedright\arraybackslash}p{4.5cm}>{\raggedright\arraybackslash}p{6.0cm}p{1.0cm}rr>{\raggedright\arraybackslash}p{2.0cm}r>{\raggedright\arraybackslash}p{1cm}p{1cm}p{1cm}p{1cm}}
\rowcolor{white}\caption{Articles in Journal Sustainability (Total 2)}\\ \toprule
\rowcolor{white}\shortstack{Key\\Source} & Authors & Title (Colored by Open Access)& \shortstack{Details\\LC} & Cite & Year & \shortstack{Conference\\/Journal\\/School} & Pages & Relevance &\shortstack{Cites\\OC XR\\SC} & \shortstack{Refs\\OC\\XR} & \shortstack{Links\\Cites\\Refs}\\ \midrule\endhead
\bottomrule
\endfoot
Relich2023 \href{http://dx.doi.org/10.3390/su15097667}{Relich2023} & \hyperref[auth:a1646]{M. Relich} & \cellcolor{gold!20}Predictive and Prescriptive Analytics in Identifying Opportunities for Improving Sustainable Manufacturing \hyperref[abs:Relich2023]{Abstract} & \hyperref[detail:Relich2023]{Details} No & \cite{Relich2023} & 2023 & Sustainability & null & \noindent{}\textcolor{black!50}{0.00} \textbf{3.00} n/a & 0 1 1 & 66 75 & 1 0 1\\
Radzki2021 \href{http://dx.doi.org/10.3390/su13095228}{Radzki2021} & \hyperref[auth:a2007]{G. Radzki}, \hyperref[auth:a1705]{I. Nielsen}, \hyperref[auth:a2008]{P. Golińska-Dawson}, \hyperref[auth:a630]{G. Bocewicz}, \hyperref[auth:a1814]{Z. Banaszak} & \cellcolor{gold!20}Reactive UAV Fleet's Mission Planning in Highly Dynamic and Unpredictable Environments \hyperref[abs:Radzki2021]{Abstract} & \hyperref[detail:Radzki2021]{Details} No & \cite{Radzki2021} & 2021 & Sustainability & null & \noindent{}\textcolor{black!50}{0.00} \textbf{1.50} n/a & 13 17 15 & 50 60 & 2 1 1\\
\end{longtable}
}

\subsection{Swarm and Evolutionary Computation}

\index{Swarm and Evolutionary Computation}
{\scriptsize
\begin{longtable}{>{\raggedright\arraybackslash}p{2.5cm}>{\raggedright\arraybackslash}p{4.5cm}>{\raggedright\arraybackslash}p{6.0cm}p{1.0cm}rr>{\raggedright\arraybackslash}p{2.0cm}r>{\raggedright\arraybackslash}p{1cm}p{1cm}p{1cm}p{1cm}}
\rowcolor{white}\caption{Articles in Journal Swarm and Evolutionary Computation (Total 1)}\\ \toprule
\rowcolor{white}\shortstack{Key\\Source} & Authors & Title (Colored by Open Access)& \shortstack{Details\\LC} & Cite & Year & \shortstack{Conference\\/Journal\\/School} & Pages & Relevance &\shortstack{Cites\\OC XR\\SC} & \shortstack{Refs\\OC\\XR} & \shortstack{Links\\Cites\\Refs}\\ \midrule\endhead
\bottomrule
\endfoot
MengGRZSC22 \href{http://dx.doi.org/10.1016/j.swevo.2022.101058}{MengGRZSC22} & \hyperref[auth:a500]{L. Meng}, \hyperref[auth:a1176]{K. Gao}, \hyperref[auth:a502]{Y. Ren}, \hyperref[auth:a503]{B. Zhang}, \hyperref[auth:a1158]{H. Sang}, \hyperref[auth:a1177]{Z. Chaoyong} & Novel MILP and CP models for distributed hybrid flowshop scheduling problem with sequence-dependent setup times & \hyperref[detail:MengGRZSC22]{Details} \href{../works/MengGRZSC22.pdf}{Yes} & \cite{MengGRZSC22} & 2022 & Swarm and Evolutionary Computation & 13 & \noindent{}\textbf{1.00} \textbf{1.00} \textbf{24.78} & 38 56 62 & 37 42 & 9 3 6\\
\end{longtable}
}

\subsection{Symmetry}

\index{Symmetry}
{\scriptsize
\begin{longtable}{>{\raggedright\arraybackslash}p{2.5cm}>{\raggedright\arraybackslash}p{4.5cm}>{\raggedright\arraybackslash}p{6.0cm}p{1.0cm}rr>{\raggedright\arraybackslash}p{2.0cm}r>{\raggedright\arraybackslash}p{1cm}p{1cm}p{1cm}p{1cm}}
\rowcolor{white}\caption{Articles in Journal Symmetry (Total 1)}\\ \toprule
\rowcolor{white}\shortstack{Key\\Source} & Authors & Title (Colored by Open Access)& \shortstack{Details\\LC} & Cite & Year & \shortstack{Conference\\/Journal\\/School} & Pages & Relevance &\shortstack{Cites\\OC XR\\SC} & \shortstack{Refs\\OC\\XR} & \shortstack{Links\\Cites\\Refs}\\ \midrule\endhead
\bottomrule
\endfoot
Liu2021b \href{http://dx.doi.org/10.3390/sym13030364}{Liu2021b} & \hyperref[auth:a1244]{S.-S. Liu}, \hyperref[auth:a1719]{A. Budiwirawan}, \hyperref[auth:a1489]{M. F. A. Arifin}, \hyperref[auth:a1490]{W. T. Chen}, \hyperref[auth:a1491]{Y.-H. Huang} & \cellcolor{gold!20}Optimization Model for the Pavement Pothole Repair Problem Considering Consumable Resources \hyperref[abs:Liu2021b]{Abstract} & \hyperref[detail:Liu2021b]{Details} No & \cite{Liu2021b} & 2021 & Symmetry & null & \noindent{}\textcolor{black!50}{0.00} \textbf{6.01} n/a & 2 3 3 & 0 0 & 1 1 0\\
\end{longtable}
}

\subsection{Tehnika}

\index{Tehnika}
{\scriptsize
\begin{longtable}{>{\raggedright\arraybackslash}p{2.5cm}>{\raggedright\arraybackslash}p{4.5cm}>{\raggedright\arraybackslash}p{6.0cm}p{1.0cm}rr>{\raggedright\arraybackslash}p{2.0cm}r>{\raggedright\arraybackslash}p{1cm}p{1cm}p{1cm}p{1cm}}
\rowcolor{white}\caption{Articles in Journal Tehnika (Total 1)}\\ \toprule
\rowcolor{white}\shortstack{Key\\Source} & Authors & Title (Colored by Open Access)& \shortstack{Details\\LC} & Cite & Year & \shortstack{Conference\\/Journal\\/School} & Pages & Relevance &\shortstack{Cites\\OC XR\\SC} & \shortstack{Refs\\OC\\XR} & \shortstack{Links\\Cites\\Refs}\\ \midrule\endhead
\bottomrule
\endfoot
Strak2021 \href{http://dx.doi.org/10.5937/tehnika2102239s}{Strak2021} & \hyperref[auth:a2027]{M. Strak}, \hyperref[auth:a2028]{R. Lečić} & Organization of work with clients in the COVID-19 emergency conditions using constraint programming \hyperref[abs:Strak2021]{Abstract} & \hyperref[detail:Strak2021]{Details} No & \cite{Strak2021} & 2021 & Tehnika & null & \noindent{}\textcolor{black!50}{0.00} \textbf{1.00} n/a & 0 0 0 & 6 19 & 1 0 1\\
\end{longtable}
}

\subsection{The International Journal of Advanced Manufacturing Technology}

\index{The International Journal of Advanced Manufacturing Technology}
{\scriptsize
\begin{longtable}{>{\raggedright\arraybackslash}p{2.5cm}>{\raggedright\arraybackslash}p{4.5cm}>{\raggedright\arraybackslash}p{6.0cm}p{1.0cm}rr>{\raggedright\arraybackslash}p{2.0cm}r>{\raggedright\arraybackslash}p{1cm}p{1cm}p{1cm}p{1cm}}
\rowcolor{white}\caption{Articles in Journal The International Journal of Advanced Manufacturing Technology (Total 1)}\\ \toprule
\rowcolor{white}\shortstack{Key\\Source} & Authors & Title (Colored by Open Access)& \shortstack{Details\\LC} & Cite & Year & \shortstack{Conference\\/Journal\\/School} & Pages & Relevance &\shortstack{Cites\\OC XR\\SC} & \shortstack{Refs\\OC\\XR} & \shortstack{Links\\Cites\\Refs}\\ \midrule\endhead
\bottomrule
\endfoot
Caricato2020 \href{http://dx.doi.org/10.1007/s00170-020-06176-y}{Caricato2020} & \hyperref[auth:a1499]{P. Caricato}, \hyperref[auth:a1500]{A. Grieco}, \hyperref[auth:a1501]{A. Arigliano}, \hyperref[auth:a1502]{L. Rondone} & \cellcolor{gold!20}Workforce influence on manufacturing machines schedules \hyperref[abs:Caricato2020]{Abstract} & \hyperref[detail:Caricato2020]{Details} No & \cite{Caricato2020} & 2020 & The International Journal of Advanced Manufacturing Technology & null & \noindent{}\textcolor{black!50}{0.00} \textbf{3.00} n/a & 3 4 3 & 20 27 & 3 1 2\\
\end{longtable}
}

\subsection{The Journal of Logic Programming}

\index{The Journal of Logic Programming}
{\scriptsize
\begin{longtable}{>{\raggedright\arraybackslash}p{2.5cm}>{\raggedright\arraybackslash}p{4.5cm}>{\raggedright\arraybackslash}p{6.0cm}p{1.0cm}rr>{\raggedright\arraybackslash}p{2.0cm}r>{\raggedright\arraybackslash}p{1cm}p{1cm}p{1cm}p{1cm}}
\rowcolor{white}\caption{Articles in Journal The Journal of Logic Programming (Total 1)}\\ \toprule
\rowcolor{white}\shortstack{Key\\Source} & Authors & Title (Colored by Open Access)& \shortstack{Details\\LC} & Cite & Year & \shortstack{Conference\\/Journal\\/School} & Pages & Relevance &\shortstack{Cites\\OC XR\\SC} & \shortstack{Refs\\OC\\XR} & \shortstack{Links\\Cites\\Refs}\\ \midrule\endhead
\bottomrule
\endfoot
DincbasSH90 \href{https://doi.org/10.1016/0743-1066(90)90052-7}{DincbasSH90} & \hyperref[auth:a717]{M. Dincbas}, \hyperref[auth:a17]{H. Simonis}, \hyperref[auth:a148]{P. V. Hentenryck} & \cellcolor{gold!20}Solving Large Combinatorial Problems in Logic Programming & \hyperref[detail:DincbasSH90]{Details} \href{../works/DincbasSH90.pdf}{Yes} & \cite{DincbasSH90} & 1990 & The Journal of Logic Programming & 19 & \noindent{}\textcolor{black!50}{0.00} \textcolor{black!50}{0.00} \textbf{1.34} & 86 85 99 & 9 28 & 17 15 2\\
\end{longtable}
}

\subsection{The Journal of Supercomputing}

\index{The Journal of Supercomputing}
{\scriptsize
\begin{longtable}{>{\raggedright\arraybackslash}p{2.5cm}>{\raggedright\arraybackslash}p{4.5cm}>{\raggedright\arraybackslash}p{6.0cm}p{1.0cm}rr>{\raggedright\arraybackslash}p{2.0cm}r>{\raggedright\arraybackslash}p{1cm}p{1cm}p{1cm}p{1cm}}
\rowcolor{white}\caption{Articles in Journal The Journal of Supercomputing (Total 1)}\\ \toprule
\rowcolor{white}\shortstack{Key\\Source} & Authors & Title (Colored by Open Access)& \shortstack{Details\\LC} & Cite & Year & \shortstack{Conference\\/Journal\\/School} & Pages & Relevance &\shortstack{Cites\\OC XR\\SC} & \shortstack{Refs\\OC\\XR} & \shortstack{Links\\Cites\\Refs}\\ \midrule\endhead
\bottomrule
\endfoot
Gao2022 \href{http://dx.doi.org/10.1007/s11227-022-04943-0}{Gao2022} & \hyperref[auth:a1837]{J. Gao}, \hyperref[auth:a1838]{X. Zhu}, \hyperref[auth:a1839]{R. Zhang} & Optimization of parallel test task scheduling with constraint satisfaction & \hyperref[detail:Gao2022]{Details} No & \cite{Gao2022} & 2022 & The Journal of Supercomputing & null & \noindent{}\textbf{2.00} \textbf{2.00} n/a & 2 4 6 & 32 32 & 2 0 2\\
\end{longtable}
}

\subsection{The Knowledge Engineering Review}

\index{The Knowledge Engineering Review}
{\scriptsize
\begin{longtable}{>{\raggedright\arraybackslash}p{2.5cm}>{\raggedright\arraybackslash}p{4.5cm}>{\raggedright\arraybackslash}p{6.0cm}p{1.0cm}rr>{\raggedright\arraybackslash}p{2.0cm}r>{\raggedright\arraybackslash}p{1cm}p{1cm}p{1cm}p{1cm}}
\rowcolor{white}\caption{Articles in Journal The Knowledge Engineering Review (Total 6)}\\ \toprule
\rowcolor{white}\shortstack{Key\\Source} & Authors & Title (Colored by Open Access)& \shortstack{Details\\LC} & Cite & Year & \shortstack{Conference\\/Journal\\/School} & Pages & Relevance &\shortstack{Cites\\OC XR\\SC} & \shortstack{Refs\\OC\\XR} & \shortstack{Links\\Cites\\Refs}\\ \midrule\endhead
\bottomrule
\endfoot
CireCH16 \href{http://dx.doi.org/10.1017/s0269888916000254}{CireCH16} & \hyperref[auth:a157]{A. A. Cir{\'{e}}}, \hyperref[auth:a335]{E. Coban}, \hyperref[auth:a160]{J. N. Hooker} & \cellcolor{green!10}Logic-based Benders decomposition for planning and scheduling: a computational analysis & \hyperref[detail:CireCH16]{Details} \href{../works/CireCH16.pdf}{Yes} & \cite{CireCH16} & 2016 & The Knowledge Engineering Review & 12 & \noindent{}\textcolor{black!50}{0.00} \textcolor{black!50}{0.00} \textbf{3.51} & 15 17 13 & 21 30 & 25 8 17\\
Junker2012 \href{http://dx.doi.org/10.1017/s0269888912000240}{Junker2012} & \hyperref[auth:a1326]{U. Junker} & Air traffic flow management with heuristic repair \hyperref[abs:Junker2012]{Abstract} & \hyperref[detail:Junker2012]{Details} No & \cite{Junker2012} & 2012 & The Knowledge Engineering Review & null & \noindent{}\textcolor{black!50}{0.00} 0.50 n/a & 3 3 4 & 5 12 & 1 0 1\\
Verfaillie2010 \href{http://dx.doi.org/10.1017/s0269888910000172}{Verfaillie2010} & \hyperref[auth:a1722]{G. Verfaillie}, \hyperref[auth:a1897]{C. Pralet}, \hyperref[auth:a2052]{M. Lemaître} & How to model planning and scheduling problems using constraint networks on timelines \hyperref[abs:Verfaillie2010]{Abstract} & \hyperref[detail:Verfaillie2010]{Details} No & \cite{Verfaillie2010} & 2010 & The Knowledge Engineering Review & null & \noindent{}\textcolor{black!50}{0.00} \textbf{1.00} n/a & 18 0 28 & 9 0 & 1 0 1\\
Caseau2001 \href{http://dx.doi.org/10.1017/s0269888901000078}{Caseau2001} & \hyperref[auth:a301]{Y. Caseau}, \hyperref[auth:a1513]{F. Laburthe}, \hyperref[auth:a163]{C. L. Pape}, \hyperref[auth:a1576]{B. Rottembourg} & Combining local and global search in a constraint programming environment \hyperref[abs:Caseau2001]{Abstract} & \hyperref[detail:Caseau2001]{Details} No & \cite{Caseau2001} & 2001 & The Knowledge Engineering Review & null & \noindent{}\textcolor{black!50}{0.00} \textbf{2.00} n/a & 10 11 12 & 0 0 & 4 4 0\\
Farias2001 \href{http://dx.doi.org/10.1017/s0269888901000030}{Farias2001} & \hyperref[auth:a1932]{I. R. D. Farias}, \hyperref[auth:a1933]{E. L. Johnson}, \hyperref[auth:a1934]{G. L. Nemhauser} & Branch-and-cut for combinatorial optimization problems without auxiliary binary variables \hyperref[abs:Farias2001]{Abstract} & \hyperref[detail:Farias2001]{Details} No & \cite{Farias2001} & 2001 & The Knowledge Engineering Review & null & \noindent{}\textcolor{black!50}{0.00} \textbf{1.00} n/a & 33 33 35 & 0 0 & 2 2 0\\
HookerOTK00 \href{http://dx.doi.org/10.1017/s0269888900001077}{HookerOTK00} & \hyperref[auth:a160]{J. N. Hooker}, \hyperref[auth:a852]{G. Ottosson}, \hyperref[auth:a1188]{E. S. Thorsteinsson}, \hyperref[auth:a1189]{H.-J. Kim} & \cellcolor{green!10}A scheme for unifying optimization and constraint satisfaction methods & \hyperref[detail:HookerOTK00]{Details} \href{../works/HookerOTK00.pdf}{Yes} & \cite{HookerOTK00} & 2000 & The Knowledge Engineering Review & 20 & \noindent{}\textcolor{black!50}{0.00} \textcolor{black!50}{0.00} \textbf{1.64} & 30 30 44 & 0 0 & 11 11 0\\
\end{longtable}
}

\subsection{The Scientific World Journal}

\index{The Scientific World Journal}
{\scriptsize
\begin{longtable}{>{\raggedright\arraybackslash}p{2.5cm}>{\raggedright\arraybackslash}p{4.5cm}>{\raggedright\arraybackslash}p{6.0cm}p{1.0cm}rr>{\raggedright\arraybackslash}p{2.0cm}r>{\raggedright\arraybackslash}p{1cm}p{1cm}p{1cm}p{1cm}}
\rowcolor{white}\caption{Articles in Journal The Scientific World Journal (Total 1)}\\ \toprule
\rowcolor{white}\shortstack{Key\\Source} & Authors & Title (Colored by Open Access)& \shortstack{Details\\LC} & Cite & Year & \shortstack{Conference\\/Journal\\/School} & Pages & Relevance &\shortstack{Cites\\OC XR\\SC} & \shortstack{Refs\\OC\\XR} & \shortstack{Links\\Cites\\Refs}\\ \midrule\endhead
\bottomrule
\endfoot
Wang2014 \href{http://dx.doi.org/10.1155/2014/271280}{Wang2014} & \hyperref[auth:a2022]{H. Wang}, \hyperref[auth:a2023]{X. Lu}, \hyperref[auth:a2024]{X. Zhang}, \hyperref[auth:a2025]{Q. Wang}, \hyperref[auth:a2026]{Y. Deng} & \cellcolor{gold!20}A Bio-Inspired Method for the Constrained Shortest Path Problem \hyperref[abs:Wang2014]{Abstract} & \hyperref[detail:Wang2014]{Details} No & \cite{Wang2014} & 2014 & The Scientific World Journal & null & \noindent{}\textcolor{black!50}{0.00} \textbf{1.00} n/a & 10 11 19 & 57 62 & 1 0 1\\
\end{longtable}
}

\subsection{Theory and Practice of Logic Programming}

\index{Theory and Practice of Logic Programming}
{\scriptsize
\begin{longtable}{>{\raggedright\arraybackslash}p{2.5cm}>{\raggedright\arraybackslash}p{4.5cm}>{\raggedright\arraybackslash}p{6.0cm}p{1.0cm}rr>{\raggedright\arraybackslash}p{2.0cm}r>{\raggedright\arraybackslash}p{1cm}p{1cm}p{1cm}p{1cm}}
\rowcolor{white}\caption{Articles in Journal Theory and Practice of Logic Programming (Total 5)}\\ \toprule
\rowcolor{white}\shortstack{Key\\Source} & Authors & Title (Colored by Open Access)& \shortstack{Details\\LC} & Cite & Year & \shortstack{Conference\\/Journal\\/School} & Pages & Relevance &\shortstack{Cites\\OC XR\\SC} & \shortstack{Refs\\OC\\XR} & \shortstack{Links\\Cites\\Refs}\\ \midrule\endhead
\bottomrule
\endfoot
Eiter2023 \href{http://dx.doi.org/10.1017/s1471068423000017}{Eiter2023} & \hyperref[auth:a1960]{T. Eiter}, \hyperref[auth:a77]{T. Geibinger}, \hyperref[auth:a45]{N. Musliu}, \hyperref[auth:a1961]{J. Oetsch}, \hyperref[auth:a1962]{P. Skočovský}, \hyperref[auth:a1963]{D. Stepanova} & \cellcolor{gold!20}Answer-Set Programming for Lexicographical Makespan Optimisation in Parallel Machine Scheduling \hyperref[abs:Eiter2023]{Abstract} & \hyperref[detail:Eiter2023]{Details} No & \cite{Eiter2023} & 2023 & Theory and Practice of Logic Programming & null & \noindent{}\textcolor{black!50}{0.00} \textbf{6.01} n/a & 0 1 0 & 27 34 & 3 0 3\\
El-Kholany2022 \href{http://dx.doi.org/10.1017/s1471068422000217}{El-Kholany2022} & \hyperref[auth:a1496]{M. M. S. El-Kholany}, \hyperref[auth:a61]{M. Gebser}, \hyperref[auth:a423]{K. Schekotihin} & \cellcolor{gold!20}Problem Decomposition and Multi-shot ASP Solving for Job-shop Scheduling \hyperref[abs:El-Kholany2022]{Abstract} & \hyperref[detail:El-Kholany2022]{Details} No & \cite{El-Kholany2022} & 2022 & Theory and Practice of Logic Programming & null & \noindent{}\textcolor{black!50}{0.00} \textbf{4.01} n/a & 6 8 7 & 28 37 & 7 0 7\\
Balduccini2017 \href{http://dx.doi.org/10.1017/s1471068417000102}{Balduccini2017} & \hyperref[auth:a1042]{M. Balduccini}, \hyperref[auth:a2051]{Y. Lierler} & \cellcolor{green!10}Constraint answer set solver EZCSP and why integration schemas matter \hyperref[abs:Balduccini2017]{Abstract} & \hyperref[detail:Balduccini2017]{Details} No & \cite{Balduccini2017} & 2017 & Theory and Practice of Logic Programming & null & \noindent{}\textcolor{black!50}{0.00} \textbf{1.00} n/a & 20 21 32 & 34 52 & 1 0 1\\
Amadini2014 \href{http://dx.doi.org/10.1017/s1471068414000179}{Amadini2014} & \hyperref[auth:a910]{R. Amadini}, \hyperref[auth:a192]{M. Gabbrielli}, \hyperref[auth:a193]{J. Mauro} & \cellcolor{green!10}SUNNY: a Lazy Portfolio Approach for Constraint Solving \hyperref[abs:Amadini2014]{Abstract} & \hyperref[detail:Amadini2014]{Details} No & \cite{Amadini2014} & 2014 & Theory and Practice of Logic Programming & null & \noindent{}\textcolor{black!50}{0.00} \textbf{1.50} n/a & 19 20 42 & 10 33 & 2 0 2\\
Apt2001 \href{http://dx.doi.org/10.1017/s1471068401000072}{Apt2001} & \hyperref[auth:a1887]{K. R. Apt}, \hyperref[auth:a1833]{E. Monfroy} & \cellcolor{green!10}Constraint programming viewed as rule-based programming \hyperref[abs:Apt2001]{Abstract} & \hyperref[detail:Apt2001]{Details} No & \cite{Apt2001} & 2001 & Theory and Practice of Logic Programming & null & \noindent{}\textcolor{black!50}{0.00} \textbf{1.75} n/a & 18 17 30 & 0 0 & 2 2 0\\
\end{longtable}
}

\subsection{Transactions of the Institute of Measurement and Control}

\index{Transactions of the Institute of Measurement and Control}
{\scriptsize
\begin{longtable}{>{\raggedright\arraybackslash}p{2.5cm}>{\raggedright\arraybackslash}p{4.5cm}>{\raggedright\arraybackslash}p{6.0cm}p{1.0cm}rr>{\raggedright\arraybackslash}p{2.0cm}r>{\raggedright\arraybackslash}p{1cm}p{1cm}p{1cm}p{1cm}}
\rowcolor{white}\caption{Articles in Journal Transactions of the Institute of Measurement and Control (Total 1)}\\ \toprule
\rowcolor{white}\shortstack{Key\\Source} & Authors & Title (Colored by Open Access)& \shortstack{Details\\LC} & Cite & Year & \shortstack{Conference\\/Journal\\/School} & Pages & Relevance &\shortstack{Cites\\OC XR\\SC} & \shortstack{Refs\\OC\\XR} & \shortstack{Links\\Cites\\Refs}\\ \midrule\endhead
\bottomrule
\endfoot
Bartk2010 \href{http://dx.doi.org/10.1177/0142331208100099}{Bartk2010} & \hyperref[auth:a1063]{R. Barták}, \hyperref[auth:a1557]{O. Čepek} & \cellcolor{green!10}Incremental propagation rules for a precedence graph with optional activities and time windows \hyperref[abs:Bartk2010]{Abstract} & \hyperref[detail:Bartk2010]{Details} No & \cite{Bartk2010} & 2010 & Transactions of the Institute of Measurement and Control & null & \noindent{}\textcolor{black!50}{0.00} 0.75 n/a & 5 5 6 & 6 15 & 6 2 4\\
\end{longtable}
}

\subsection{Transportation Research Part B: Methodological}

\index{Transportation Research Part B: Methodological}
{\scriptsize
\begin{longtable}{>{\raggedright\arraybackslash}p{2.5cm}>{\raggedright\arraybackslash}p{4.5cm}>{\raggedright\arraybackslash}p{6.0cm}p{1.0cm}rr>{\raggedright\arraybackslash}p{2.0cm}r>{\raggedright\arraybackslash}p{1cm}p{1cm}p{1cm}p{1cm}}
\rowcolor{white}\caption{Articles in Journal Transportation Research Part B: Methodological (Total 1)}\\ \toprule
\rowcolor{white}\shortstack{Key\\Source} & Authors & Title (Colored by Open Access)& \shortstack{Details\\LC} & Cite & Year & \shortstack{Conference\\/Journal\\/School} & Pages & Relevance &\shortstack{Cites\\OC XR\\SC} & \shortstack{Refs\\OC\\XR} & \shortstack{Links\\Cites\\Refs}\\ \midrule\endhead
\bottomrule
\endfoot
Rodriguez07 \href{https://www.sciencedirect.com/science/article/pii/S0191261506000233}{Rodriguez07} & \hyperref[auth:a781]{J. Rodriguez} & A constraint programming model for real-time train scheduling at junctions \hyperref[abs:Rodriguez07]{Abstract} & \hyperref[detail:Rodriguez07]{Details} \href{../works/Rodriguez07.pdf}{Yes} & \cite{Rodriguez07} & 2007 & Transportation Research Part B: Methodological & 15 & \noindent{}\textbf{1.00} \textbf{1.50} \textbf{6.88} & 117 121 141 & 6 14 & 11 9 2\\
\end{longtable}
}

\subsection{Transportation Research Part C: Emerging Technologies}

\index{Transportation Research Part C: Emerging Technologies}
{\scriptsize
\begin{longtable}{>{\raggedright\arraybackslash}p{2.5cm}>{\raggedright\arraybackslash}p{4.5cm}>{\raggedright\arraybackslash}p{6.0cm}p{1.0cm}rr>{\raggedright\arraybackslash}p{2.0cm}r>{\raggedright\arraybackslash}p{1cm}p{1cm}p{1cm}p{1cm}}
\rowcolor{white}\caption{Articles in Journal Transportation Research Part C: Emerging Technologies (Total 1)}\\ \toprule
\rowcolor{white}\shortstack{Key\\Source} & Authors & Title (Colored by Open Access)& \shortstack{Details\\LC} & Cite & Year & \shortstack{Conference\\/Journal\\/School} & Pages & Relevance &\shortstack{Cites\\OC XR\\SC} & \shortstack{Refs\\OC\\XR} & \shortstack{Links\\Cites\\Refs}\\ \midrule\endhead
\bottomrule
\endfoot
Ham18 \href{http://dx.doi.org/10.1016/j.trc.2018.03.025}{Ham18} & \hyperref[auth:a770]{A. M. Ham} & Integrated scheduling of m-truck, m-drone, and m-depot constrained by time-window, drop-pickup, and m-visit using constraint programming & \hyperref[detail:Ham18]{Details} \href{../works/Ham18.pdf}{Yes} & \cite{Ham18} & 2018 & Transportation Research Part C: Emerging Technologies & 14 & \noindent{}\textbf{1.00} \textbf{1.00} \textbf{12.98} & 164 192 197 & 14 30 & 11 7 4\\
\end{longtable}
}

\subsection{Transportation Research Part E: Logistics and Transportation Review}

\index{Transportation Research Part E: Logistics and Transportation Review}
{\scriptsize
\begin{longtable}{>{\raggedright\arraybackslash}p{2.5cm}>{\raggedright\arraybackslash}p{4.5cm}>{\raggedright\arraybackslash}p{6.0cm}p{1.0cm}rr>{\raggedright\arraybackslash}p{2.0cm}r>{\raggedright\arraybackslash}p{1cm}p{1cm}p{1cm}p{1cm}}
\rowcolor{white}\caption{Articles in Journal Transportation Research Part E: Logistics and Transportation Review (Total 3)}\\ \toprule
\rowcolor{white}\shortstack{Key\\Source} & Authors & Title (Colored by Open Access)& \shortstack{Details\\LC} & Cite & Year & \shortstack{Conference\\/Journal\\/School} & Pages & Relevance &\shortstack{Cites\\OC XR\\SC} & \shortstack{Refs\\OC\\XR} & \shortstack{Links\\Cites\\Refs}\\ \midrule\endhead
\bottomrule
\endfoot
UnsalO19 \href{http://dx.doi.org/10.1016/j.tre.2019.03.018}{UnsalO19} & \hyperref[auth:a1217]{O. Unsal}, \hyperref[auth:a347]{C. Oguz} & An exact algorithm for integrated planning of operations in dry bulk terminals & \hyperref[detail:UnsalO19]{Details} \href{../works/UnsalO19.pdf}{Yes} & \cite{UnsalO19} & 2019 & Transportation Research Part E: Logistics and Transportation Review & 19 & \noindent{}\textcolor{black!50}{0.00} \textcolor{black!50}{0.00} \textbf{3.21} & 44 52 54 & 27 34 & 12 4 8\\
QinDS16 \href{http://dx.doi.org/10.1016/j.tre.2016.01.007}{QinDS16} & \hyperref[auth:a509]{T. Qin}, \hyperref[auth:a510]{Y. Du}, \hyperref[auth:a512]{M. Sha} & Evaluating the solution performance of IP and CP for berth allocation with time-varying water depth & \hyperref[detail:QinDS16]{Details} \href{../works/QinDS16.pdf}{Yes} & \cite{QinDS16} & 2016 & Transportation Research Part E: Logistics and Transportation Review & 19 & \noindent{}\textcolor{black!50}{0.00} \textcolor{black!50}{0.00} \textbf{14.66} & 17 18 21 & 40 49 & 14 3 11\\
UnsalO13 \href{http://dx.doi.org/10.1016/j.tre.2013.08.006}{UnsalO13} & \hyperref[auth:a1217]{O. Unsal}, \hyperref[auth:a347]{C. Oguz} & Constraint programming approach to quay crane scheduling problem & \hyperref[detail:UnsalO13]{Details} \href{../works/UnsalO13.pdf}{Yes} & \cite{UnsalO13} & 2013 & Transportation Research Part E: Logistics and Transportation Review & 15 & \noindent{}\textbf{1.00} \textbf{1.00} \textbf{15.56} & 44 45 54 & 25 34 & 10 6 4\\
\end{longtable}
}

\subsection{Transportation Research Record: Journal of the Transportation Research Board}

\index{Transportation Research Record: Journal of the Transportation Research Board}
{\scriptsize
\begin{longtable}{>{\raggedright\arraybackslash}p{2.5cm}>{\raggedright\arraybackslash}p{4.5cm}>{\raggedright\arraybackslash}p{6.0cm}p{1.0cm}rr>{\raggedright\arraybackslash}p{2.0cm}r>{\raggedright\arraybackslash}p{1cm}p{1cm}p{1cm}p{1cm}}
\rowcolor{white}\caption{Articles in Journal Transportation Research Record: Journal of the Transportation Research Board (Total 3)}\\ \toprule
\rowcolor{white}\shortstack{Key\\Source} & Authors & Title (Colored by Open Access)& \shortstack{Details\\LC} & Cite & Year & \shortstack{Conference\\/Journal\\/School} & Pages & Relevance &\shortstack{Cites\\OC XR\\SC} & \shortstack{Refs\\OC\\XR} & \shortstack{Links\\Cites\\Refs}\\ \midrule\endhead
\bottomrule
\endfoot
Chen2021 \href{http://dx.doi.org/10.1177/03611981211036368}{Chen2021} & \hyperref[auth:a1626]{G.-H. Chen}, \hyperref[auth:a1627]{J.-C. Jong}, \hyperref[auth:a1628]{A. F.-W. Han} & \cellcolor{gold!20}Applying Constraint Programming and Integer Programming to Solve the Crew Scheduling Problem for Railroad Systems: Model Formulation and a Case Study \hyperref[abs:Chen2021]{Abstract} & \hyperref[detail:Chen2021]{Details} No & \cite{Chen2021} & 2021 & Transportation Research Record: Journal of the Transportation Research Board & null & \noindent{}\textbf{1.00} \textbf{2.00} n/a & 1 4 4 & 20 25 & 6 0 6\\
Li2016 \href{http://dx.doi.org/10.3141/2549-01}{Li2016} & \hyperref[auth:a2066]{S. Li}, \hyperref[auth:a2067]{R. R. Negenborn}, \hyperref[auth:a2068]{G. Lodewijks} & Approach Integrating Mixed-Integer Programming and Constraint Programming for Planning Rotations of Inland Vessels in a Large Seaport \hyperref[abs:Li2016]{Abstract} & \hyperref[detail:Li2016]{Details} No & \cite{Li2016} & 2016 & Transportation Research Record: Journal of the Transportation Research Board & null & \noindent{}\textcolor{black!50}{0.00} \textbf{1.00} n/a & 3 2 3 & 5 8 & 2 1 1\\
Wang2015 \href{http://dx.doi.org/10.3141/2482-15}{Wang2015} & \hyperref[auth:a1710]{S. Wang}, \hyperref[auth:a1711]{E. Y. Chou} & Cross-Asset Transportation Project Coordination with Integer Programming and Constraint Programming \hyperref[abs:Wang2015]{Abstract} & \hyperref[detail:Wang2015]{Details} No & \cite{Wang2015} & 2015 & Transportation Research Record: Journal of the Transportation Research Board & null & \noindent{}\textcolor{black!50}{0.00} \textbf{1.00} n/a & 1 1 2 & 9 11 & 4 0 4\\
\end{longtable}
}

\subsection{Transportation Science}

\index{Transportation Science}
{\scriptsize
\begin{longtable}{>{\raggedright\arraybackslash}p{2.5cm}>{\raggedright\arraybackslash}p{4.5cm}>{\raggedright\arraybackslash}p{6.0cm}p{1.0cm}rr>{\raggedright\arraybackslash}p{2.0cm}r>{\raggedright\arraybackslash}p{1cm}p{1cm}p{1cm}p{1cm}}
\rowcolor{white}\caption{Articles in Journal Transportation Science (Total 3)}\\ \toprule
\rowcolor{white}\shortstack{Key\\Source} & Authors & Title (Colored by Open Access)& \shortstack{Details\\LC} & Cite & Year & \shortstack{Conference\\/Journal\\/School} & Pages & Relevance &\shortstack{Cites\\OC XR\\SC} & \shortstack{Refs\\OC\\XR} & \shortstack{Links\\Cites\\Refs}\\ \midrule\endhead
\bottomrule
\endfoot
BalochG20 \href{http://dx.doi.org/10.1287/trsc.2019.0928}{BalochG20} & \hyperref[auth:a1237]{G. Baloch}, \hyperref[auth:a1238]{F. Gzara} & Strategic Network Design for Parcel Delivery with Drones Under Competition & \hyperref[detail:BalochG20]{Details} No & \cite{BalochG20} & 2020 & \cellcolor{red!20}Transportation Science & 25 & \noindent{}\textcolor{black!50}{0.00} \textcolor{black!50}{0.00} n/a & 25 32 33 & 46 52 & 9 2 7\\
HechingHK19 \href{http://dx.doi.org/10.1287/trsc.2018.0830}{HechingHK19} & \hyperref[auth:a1021]{A. Heching}, \hyperref[auth:a160]{J. N. Hooker}, \hyperref[auth:a1022]{R. Kimura} & \cellcolor{gold!20}A Logic-Based Benders Approach to Home Healthcare Delivery & \hyperref[detail:HechingHK19]{Details} No & \cite{HechingHK19} & 2019 & \cellcolor{red!20}Transportation Science & 13 & \noindent{}\textcolor{black!50}{0.00} \textcolor{black!50}{0.00} n/a & 35 42 37 & 29 32 & 20 8 12\\
Yunes2005 \href{http://dx.doi.org/10.1287/trsc.1030.0078}{Yunes2005} & \hyperref[auth:a942]{T. H. Yunes}, \hyperref[auth:a1580]{A. V. Moura}, \hyperref[auth:a170]{C. C. de Souza} & Hybrid Column Generation Approaches for Urban Transit Crew Management Problems \hyperref[abs:Yunes2005]{Abstract} & \hyperref[detail:Yunes2005]{Details} No & \cite{Yunes2005} & 2005 & \cellcolor{red!20}Transportation Science & null & \noindent{}\textcolor{black!50}{0.00} \textbf{2.00} n/a & 42 42 44 & 17 31 & 13 11 2\\
\end{longtable}
}

\subsection{Vietnam. J. Comput. Sci.}

\index{Vietnam. J. Comput. Sci.}
{\scriptsize
\begin{longtable}{>{\raggedright\arraybackslash}p{2.5cm}>{\raggedright\arraybackslash}p{4.5cm}>{\raggedright\arraybackslash}p{6.0cm}p{1.0cm}rr>{\raggedright\arraybackslash}p{2.0cm}r>{\raggedright\arraybackslash}p{1cm}p{1cm}p{1cm}p{1cm}}
\rowcolor{white}\caption{Articles in Journal Vietnam. J. Comput. Sci. (Total 1)}\\ \toprule
\rowcolor{white}\shortstack{Key\\Source} & Authors & Title (Colored by Open Access)& \shortstack{Details\\LC} & Cite & Year & \shortstack{Conference\\/Journal\\/School} & Pages & Relevance &\shortstack{Cites\\OC XR\\SC} & \shortstack{Refs\\OC\\XR} & \shortstack{Links\\Cites\\Refs}\\ \midrule\endhead
\bottomrule
\endfoot
WikarekS19 \href{https://doi.org/10.1142/S2196888819500027}{WikarekS19} & \hyperref[auth:a535]{J. Wikarek}, \hyperref[auth:a536]{P. Sitek} & \cellcolor{gold!20}A Constraint-Based Declarative Programming Framework for Scheduling and Resource Allocation Problems & \hyperref[detail:WikarekS19]{Details} \href{../works/WikarekS19.pdf}{Yes} & \cite{WikarekS19} & 2019 & Vietnam. J. Comput. Sci. & 22 & \noindent{}\textcolor{black!50}{0.00} \textcolor{black!50}{0.00} \textbf{4.30} & 0 0 0 & 11 16 & 6 0 6\\
\end{longtable}
}

\subsection{Yugoslav Journal of Operations Research}

\index{Yugoslav Journal of Operations Research}
{\scriptsize
\begin{longtable}{>{\raggedright\arraybackslash}p{2.5cm}>{\raggedright\arraybackslash}p{4.5cm}>{\raggedright\arraybackslash}p{6.0cm}p{1.0cm}rr>{\raggedright\arraybackslash}p{2.0cm}r>{\raggedright\arraybackslash}p{1cm}p{1cm}p{1cm}p{1cm}}
\rowcolor{white}\caption{Articles in Journal Yugoslav Journal of Operations Research (Total 1)}\\ \toprule
\rowcolor{white}\shortstack{Key\\Source} & Authors & Title (Colored by Open Access)& \shortstack{Details\\LC} & Cite & Year & \shortstack{Conference\\/Journal\\/School} & Pages & Relevance &\shortstack{Cites\\OC XR\\SC} & \shortstack{Refs\\OC\\XR} & \shortstack{Links\\Cites\\Refs}\\ \midrule\endhead
\bottomrule
\endfoot
Mladenovic2007 \href{http://dx.doi.org/10.2298/yjor0701009m}{Mladenovic2007} & \hyperref[auth:a1621]{S. Mladenovic}, \hyperref[auth:a1717]{M. Cangalovic} & \cellcolor{gold!20}Heuristic approach to train rescheduling \hyperref[abs:Mladenovic2007]{Abstract} & \hyperref[detail:Mladenovic2007]{Details} No & \cite{Mladenovic2007} & 2007 & Yugoslav Journal of Operations Research & null & \noindent{}\textcolor{black!50}{0.00} \textbf{3.00} n/a & 7 6 9 & 3 3 & 2 2 0\\
\end{longtable}
}

\subsection{{ACM} Trans. Embed. Comput. Syst.}

\index{{ACM} Trans. Embed. Comput. Syst.}
{\scriptsize
\begin{longtable}{>{\raggedright\arraybackslash}p{2.5cm}>{\raggedright\arraybackslash}p{4.5cm}>{\raggedright\arraybackslash}p{6.0cm}p{1.0cm}rr>{\raggedright\arraybackslash}p{2.0cm}r>{\raggedright\arraybackslash}p{1cm}p{1cm}p{1cm}p{1cm}}
\rowcolor{white}\caption{Articles in Journal {ACM} Trans. Embed. Comput. Syst. (Total 1)}\\ \toprule
\rowcolor{white}\shortstack{Key\\Source} & Authors & Title (Colored by Open Access)& \shortstack{Details\\LC} & Cite & Year & \shortstack{Conference\\/Journal\\/School} & Pages & Relevance &\shortstack{Cites\\OC XR\\SC} & \shortstack{Refs\\OC\\XR} & \shortstack{Links\\Cites\\Refs}\\ \midrule\endhead
\bottomrule
\endfoot
BegB13 \href{http://doi.acm.org/10.1145/2512470}{BegB13} & \hyperref[auth:a609]{M. O. Beg}, \hyperref[auth:a610]{P. van Beek} & \cellcolor{gold!20}A constraint programming approach for integrated spatial and temporal scheduling for clustered architectures & \hyperref[detail:BegB13]{Details} \href{../works/BegB13.pdf}{Yes} & \cite{BegB13} & 2013 & {ACM} Trans. Embed. Comput. Syst. & 23 & \noindent{}\textbf{1.00} \textbf{1.00} \textbf{4.22} & 1 1 1 & 28 46 & 4 1 3\\
\end{longtable}
}

\subsection{{ACM} Trans. Intell. Syst. Technol.}

\index{{ACM} Trans. Intell. Syst. Technol.}
{\scriptsize
\begin{longtable}{>{\raggedright\arraybackslash}p{2.5cm}>{\raggedright\arraybackslash}p{4.5cm}>{\raggedright\arraybackslash}p{6.0cm}p{1.0cm}rr>{\raggedright\arraybackslash}p{2.0cm}r>{\raggedright\arraybackslash}p{1cm}p{1cm}p{1cm}p{1cm}}
\rowcolor{white}\caption{Articles in Journal {ACM} Trans. Intell. Syst. Technol. (Total 1)}\\ \toprule
\rowcolor{white}\shortstack{Key\\Source} & Authors & Title (Colored by Open Access)& \shortstack{Details\\LC} & Cite & Year & \shortstack{Conference\\/Journal\\/School} & Pages & Relevance &\shortstack{Cites\\OC XR\\SC} & \shortstack{Refs\\OC\\XR} & \shortstack{Links\\Cites\\Refs}\\ \midrule\endhead
\bottomrule
\endfoot
ReddyFIBKAJ11 \href{https://doi.org/10.1145/1989734.1989745}{ReddyFIBKAJ11} & \hyperref[auth:a1037]{S. Y. Reddy}, \hyperref[auth:a379]{J. Frank}, \hyperref[auth:a1038]{M. Iatauro}, \hyperref[auth:a1039]{M. E. Boyce}, \hyperref[auth:a380]{E. K{\"{u}}rkl{\"{u}}}, \hyperref[auth:a1040]{M. Ai-Chang}, \hyperref[auth:a1041]{A. K. J{\'{o}}nsson} & Planning solar array operations on the international space station & \hyperref[detail:ReddyFIBKAJ11]{Details} \href{../works/ReddyFIBKAJ11.pdf}{Yes} & \cite{ReddyFIBKAJ11} & 2011 & {ACM} Trans. Intell. Syst. Technol. & 24 & \noindent{}\textcolor{black!50}{0.00} \textcolor{black!50}{0.00} 0.59 & 3 3 11 & 8 22 & 1 1 0\\
\end{longtable}
}

\subsection{{AI} Commun.}

\index{{AI} Commun.}
{\scriptsize
\begin{longtable}{>{\raggedright\arraybackslash}p{2.5cm}>{\raggedright\arraybackslash}p{4.5cm}>{\raggedright\arraybackslash}p{6.0cm}p{1.0cm}rr>{\raggedright\arraybackslash}p{2.0cm}r>{\raggedright\arraybackslash}p{1cm}p{1cm}p{1cm}p{1cm}}
\rowcolor{white}\caption{Articles in Journal {AI} Commun. (Total 1)}\\ \toprule
\rowcolor{white}\shortstack{Key\\Source} & Authors & Title (Colored by Open Access)& \shortstack{Details\\LC} & Cite & Year & \shortstack{Conference\\/Journal\\/School} & Pages & Relevance &\shortstack{Cites\\OC XR\\SC} & \shortstack{Refs\\OC\\XR} & \shortstack{Links\\Cites\\Refs}\\ \midrule\endhead
\bottomrule
\endfoot
BadicaBI20 \href{https://doi.org/10.3233/AIC-200650}{BadicaBI20} & \hyperref[auth:a497]{A. Badica}, \hyperref[auth:a498]{C. Badica}, \hyperref[auth:a499]{M. Ivanovic} & Block structured scheduling using constraint logic programming & \hyperref[detail:BadicaBI20]{Details} \href{../works/BadicaBI20.pdf}{Yes} & \cite{BadicaBI20} & 2020 & {AI} Commun. & 17 & \noindent{}\textbf{1.00} \textbf{1.00} \textbf{9.81} & 2 4 4 & 28 31 & 3 0 3\\
\end{longtable}
}

\subsection{{AI} Mag.}

\index{{AI} Mag.}
{\scriptsize
\begin{longtable}{>{\raggedright\arraybackslash}p{2.5cm}>{\raggedright\arraybackslash}p{4.5cm}>{\raggedright\arraybackslash}p{6.0cm}p{1.0cm}rr>{\raggedright\arraybackslash}p{2.0cm}r>{\raggedright\arraybackslash}p{1cm}p{1cm}p{1cm}p{1cm}}
\rowcolor{white}\caption{Articles in Journal {AI} Mag. (Total 1)}\\ \toprule
\rowcolor{white}\shortstack{Key\\Source} & Authors & Title (Colored by Open Access)& \shortstack{Details\\LC} & Cite & Year & \shortstack{Conference\\/Journal\\/School} & Pages & Relevance &\shortstack{Cites\\OC XR\\SC} & \shortstack{Refs\\OC\\XR} & \shortstack{Links\\Cites\\Refs}\\ \midrule\endhead
\bottomrule
\endfoot
BeckF98 \href{https://doi.org/10.1609/aimag.v19i4.1426}{BeckF98} & \hyperref[auth:a89]{J. C. Beck}, \hyperref[auth:a302]{M. S. Fox} & A Generic Framework for Constraint-Directed Search and Scheduling & \hyperref[detail:BeckF98]{Details} \href{../works/BeckF98.pdf}{Yes} & \cite{BeckF98} & 1998 & {AI} Mag. & 30 & \noindent{}\textcolor{black!50}{0.00} \textcolor{black!50}{0.00} \textbf{9.67} & 0 0 0 & 0 0 & 0 0 0\\
\end{longtable}
}

\subsection{{EURO} J. Comput. Optim.}

\index{{EURO} J. Comput. Optim.}
{\scriptsize
\begin{longtable}{>{\raggedright\arraybackslash}p{2.5cm}>{\raggedright\arraybackslash}p{4.5cm}>{\raggedright\arraybackslash}p{6.0cm}p{1.0cm}rr>{\raggedright\arraybackslash}p{2.0cm}r>{\raggedright\arraybackslash}p{1cm}p{1cm}p{1cm}p{1cm}}
\rowcolor{white}\caption{Articles in Journal {EURO} J. Comput. Optim. (Total 1)}\\ \toprule
\rowcolor{white}\shortstack{Key\\Source} & Authors & Title (Colored by Open Access)& \shortstack{Details\\LC} & Cite & Year & \shortstack{Conference\\/Journal\\/School} & Pages & Relevance &\shortstack{Cites\\OC XR\\SC} & \shortstack{Refs\\OC\\XR} & \shortstack{Links\\Cites\\Refs}\\ \midrule\endhead
\bottomrule
\endfoot
MontemanniD23a \href{https://doi.org/10.1016/j.ejco.2023.100078}{MontemanniD23a} & \hyperref[auth:a410]{R. Montemanni}, \hyperref[auth:a411]{M. Dell'Amico} & \cellcolor{gold!20}Constraint programming models for the parallel drone scheduling vehicle routing problem & \hyperref[detail:MontemanniD23a]{Details} \href{../works/MontemanniD23a.pdf}{Yes} & \cite{MontemanniD23a} & 2023 & {EURO} J. Comput. Optim. & 20 & \noindent{}\textbf{1.00} \textbf{1.00} 0.64 & 0 1 1 & 14 19 & 1 0 1\\
\end{longtable}
}

\subsection{{IBM} J. Res. Dev.}

\index{{IBM} J. Res. Dev.}
{\scriptsize
\begin{longtable}{>{\raggedright\arraybackslash}p{2.5cm}>{\raggedright\arraybackslash}p{4.5cm}>{\raggedright\arraybackslash}p{6.0cm}p{1.0cm}rr>{\raggedright\arraybackslash}p{2.0cm}r>{\raggedright\arraybackslash}p{1cm}p{1cm}p{1cm}p{1cm}}
\rowcolor{white}\caption{Articles in Journal {IBM} J. Res. Dev. (Total 1)}\\ \toprule
\rowcolor{white}\shortstack{Key\\Source} & Authors & Title (Colored by Open Access)& \shortstack{Details\\LC} & Cite & Year & \shortstack{Conference\\/Journal\\/School} & Pages & Relevance &\shortstack{Cites\\OC XR\\SC} & \shortstack{Refs\\OC\\XR} & \shortstack{Links\\Cites\\Refs}\\ \midrule\endhead
\bottomrule
\endfoot
OkanoDTRYA04 \href{https://doi.org/10.1147/rd.485.0811}{OkanoDTRYA04} & \hyperref[auth:a1288]{H. Okano}, \hyperref[auth:a248]{A. J. Davenport}, \hyperref[auth:a1289]{M. Trumbo}, \hyperref[auth:a250]{C. Reddy}, \hyperref[auth:a1290]{K. Yoda}, \hyperref[auth:a1291]{M. Amano} & Finishing Line Scheduling in the steel industry & \hyperref[detail:OkanoDTRYA04]{Details} No & \cite{OkanoDTRYA04} & 2004 & {IBM} J. Res. Dev. & 20 & \noindent{}\textcolor{black!50}{0.00} \textcolor{black!50}{0.00} n/a & 19 20 26 & 0 0 & 0 0 0\\
\end{longtable}
}

\subsection{{IEEE} Access}

\index{{IEEE} Access}
{\scriptsize
\begin{longtable}{>{\raggedright\arraybackslash}p{2.5cm}>{\raggedright\arraybackslash}p{4.5cm}>{\raggedright\arraybackslash}p{6.0cm}p{1.0cm}rr>{\raggedright\arraybackslash}p{2.0cm}r>{\raggedright\arraybackslash}p{1cm}p{1cm}p{1cm}p{1cm}}
\rowcolor{white}\caption{Articles in Journal {IEEE} Access (Total 3)}\\ \toprule
\rowcolor{white}\shortstack{Key\\Source} & Authors & Title (Colored by Open Access)& \shortstack{Details\\LC} & Cite & Year & \shortstack{Conference\\/Journal\\/School} & Pages & Relevance &\shortstack{Cites\\OC XR\\SC} & \shortstack{Refs\\OC\\XR} & \shortstack{Links\\Cites\\Refs}\\ \midrule\endhead
\bottomrule
\endfoot
AkramNHRSA23 \href{https://doi.org/10.1109/ACCESS.2023.3343409}{AkramNHRSA23} & \hyperref[auth:a399]{B. O. Akram}, \hyperref[auth:a400]{N. K. Noordin}, \hyperref[auth:a401]{F. Hashim}, \hyperref[auth:a402]{M. F. A. Rasid}, \hyperref[auth:a403]{M. I. Salman}, \hyperref[auth:a404]{A. M. Abdulghani} & \cellcolor{gold!20}Joint Scheduling and Routing Optimization for Deterministic Hybrid Traffic in Time-Sensitive Networks Using Constraint Programming & \hyperref[detail:AkramNHRSA23]{Details} \href{../works/AkramNHRSA23.pdf}{Yes} & \cite{AkramNHRSA23} & 2023 & {IEEE} Access & 16 & \noindent{}\textbf{1.00} \textbf{1.00} \textbf{7.96} & 0 0 0 & 0 37 & 0 0 0\\
YuraszeckMCCR23 \href{https://doi.org/10.1109/ACCESS.2023.3345793}{YuraszeckMCCR23} & \hyperref[auth:a405]{F. Yuraszeck}, \hyperref[auth:a406]{E. Montero}, \hyperref[auth:a407]{D. Canut-de-Bon}, \hyperref[auth:a408]{N. Cuneo}, \hyperref[auth:a409]{M. Rojel} & \cellcolor{gold!20}A Constraint Programming Formulation of the Multi-Mode Resource-Constrained Project Scheduling Problem for the Flexible Job Shop Scheduling Problem & \hyperref[detail:YuraszeckMCCR23]{Details} \href{../works/YuraszeckMCCR23.pdf}{Yes} & \cite{YuraszeckMCCR23} & 2023 & {IEEE} Access & 11 & \noindent{}\textbf{2.50} \textbf{2.50} \textbf{8.55} & 0 0 0 & 0 29 & 0 0 0\\
AbohashimaEG21 \href{https://doi.org/10.1109/ACCESS.2021.3112600}{AbohashimaEG21} & \hyperref[auth:a472]{H. Abohashima}, \hyperref[auth:a473]{A. B. Eltawil}, \hyperref[auth:a474]{M. S. Gheith} & \cellcolor{gold!20}A Mathematical Programming Model and a Firefly-Based Heuristic for Real-Time Traffic Signal Scheduling With Physical Constraints & \hyperref[detail:AbohashimaEG21]{Details} \href{../works/AbohashimaEG21.pdf}{Yes} & \cite{AbohashimaEG21} & 2021 & {IEEE} Access & 14 & \noindent{}\textcolor{black!50}{0.00} \textcolor{black!50}{0.00} \textcolor{black!50}{0.00} & 1 3 3 & 25 27 & 0 0 0\\
\end{longtable}
}

\subsection{{IEEE} Intell. Syst.}

\index{{IEEE} Intell. Syst.}
{\scriptsize
\begin{longtable}{>{\raggedright\arraybackslash}p{2.5cm}>{\raggedright\arraybackslash}p{4.5cm}>{\raggedright\arraybackslash}p{6.0cm}p{1.0cm}rr>{\raggedright\arraybackslash}p{2.0cm}r>{\raggedright\arraybackslash}p{1cm}p{1cm}p{1cm}p{1cm}}
\rowcolor{white}\caption{Articles in Journal {IEEE} Intell. Syst. (Total 1)}\\ \toprule
\rowcolor{white}\shortstack{Key\\Source} & Authors & Title (Colored by Open Access)& \shortstack{Details\\LC} & Cite & Year & \shortstack{Conference\\/Journal\\/School} & Pages & Relevance &\shortstack{Cites\\OC XR\\SC} & \shortstack{Refs\\OC\\XR} & \shortstack{Links\\Cites\\Refs}\\ \midrule\endhead
\bottomrule
\endfoot
SimonisCK00 \href{https://doi.org/10.1109/5254.820326}{SimonisCK00} & \hyperref[auth:a17]{H. Simonis}, \hyperref[auth:a886]{P. Charlier}, \hyperref[auth:a887]{P. Kay} & Constraint Handling in an Integrated Transportation Problem & \hyperref[detail:SimonisCK00]{Details} \href{../works/SimonisCK00.pdf}{Yes} & \cite{SimonisCK00} & 2000 & {IEEE} Intell. Syst. & 7 & \noindent{}\textcolor{black!50}{0.00} \textcolor{black!50}{0.00} 0.48 & 11 11 6 & 5 14 & 10 5 5\\
\end{longtable}
}

\subsection{{IEEE} Trans Autom. Sci. Eng.}

\index{{IEEE} Trans Autom. Sci. Eng.}
{\scriptsize
\begin{longtable}{>{\raggedright\arraybackslash}p{2.5cm}>{\raggedright\arraybackslash}p{4.5cm}>{\raggedright\arraybackslash}p{6.0cm}p{1.0cm}rr>{\raggedright\arraybackslash}p{2.0cm}r>{\raggedright\arraybackslash}p{1cm}p{1cm}p{1cm}p{1cm}}
\rowcolor{white}\caption{Articles in Journal {IEEE} Trans Autom. Sci. Eng. (Total 1)}\\ \toprule
\rowcolor{white}\shortstack{Key\\Source} & Authors & Title (Colored by Open Access)& \shortstack{Details\\LC} & Cite & Year & \shortstack{Conference\\/Journal\\/School} & Pages & Relevance &\shortstack{Cites\\OC XR\\SC} & \shortstack{Refs\\OC\\XR} & \shortstack{Links\\Cites\\Refs}\\ \midrule\endhead
\bottomrule
\endfoot
QinWSLS21 \href{https://doi.org/10.1109/TASE.2019.2947398}{QinWSLS21} & \hyperref[auth:a486]{M. Qin}, \hyperref[auth:a487]{R. Wang}, \hyperref[auth:a488]{Z. Shi}, \hyperref[auth:a489]{L. Liu}, \hyperref[auth:a490]{L. Shi} & A Genetic Programming-Based Scheduling Approach for Hybrid Flow Shop With a Batch Processor and Waiting Time Constraint & \hyperref[detail:QinWSLS21]{Details} \href{../works/QinWSLS21.pdf}{Yes} & \cite{QinWSLS21} & 2021 & {IEEE} Trans Autom. Sci. Eng. & 12 & \noindent{}\textcolor{black!50}{0.00} \textcolor{black!50}{0.00} \textcolor{black!50}{0.00} & 12 19 0 & 30 30 & 1 0 1\\
\end{longtable}
}

\subsection{{IEEE} Trans. Comput. Aided Des. Integr. Circuits Syst.}

\index{{IEEE} Trans. Comput. Aided Des. Integr. Circuits Syst.}
{\scriptsize
\begin{longtable}{>{\raggedright\arraybackslash}p{2.5cm}>{\raggedright\arraybackslash}p{4.5cm}>{\raggedright\arraybackslash}p{6.0cm}p{1.0cm}rr>{\raggedright\arraybackslash}p{2.0cm}r>{\raggedright\arraybackslash}p{1cm}p{1cm}p{1cm}p{1cm}}
\rowcolor{white}\caption{Articles in Journal {IEEE} Trans. Comput. Aided Des. Integr. Circuits Syst. (Total 1)}\\ \toprule
\rowcolor{white}\shortstack{Key\\Source} & Authors & Title (Colored by Open Access)& \shortstack{Details\\LC} & Cite & Year & \shortstack{Conference\\/Journal\\/School} & Pages & Relevance &\shortstack{Cites\\OC XR\\SC} & \shortstack{Refs\\OC\\XR} & \shortstack{Links\\Cites\\Refs}\\ \midrule\endhead
\bottomrule
\endfoot
RuggieroBBMA09 \href{https://doi.org/10.1109/TCAD.2009.2013536}{RuggieroBBMA09} & \hyperref[auth:a718]{M. Ruggiero}, \hyperref[auth:a375]{D. Bertozzi}, \hyperref[auth:a245]{L. Benini}, \hyperref[auth:a143]{M. Milano}, \hyperref[auth:a719]{A. Andrei} & \cellcolor{green!10}Reducing the Abstraction and Optimality Gaps in the Allocation and Scheduling for Variable Voltage/Frequency MPSoC Platforms & \hyperref[detail:RuggieroBBMA09]{Details} \href{../works/RuggieroBBMA09.pdf}{Yes} & \cite{RuggieroBBMA09} & 2009 & {IEEE} Trans. Comput. Aided Des. Integr. Circuits Syst. & 14 & \noindent{}\textcolor{black!50}{0.00} \textcolor{black!50}{0.00} \textbf{5.02} & 9 9 7 & 27 37 & 5 0 5\\
\end{longtable}
}

\subsection{{IEEE} Trans. Engineering Management}

\index{{IEEE} Trans. Engineering Management}
{\scriptsize
\begin{longtable}{>{\raggedright\arraybackslash}p{2.5cm}>{\raggedright\arraybackslash}p{4.5cm}>{\raggedright\arraybackslash}p{6.0cm}p{1.0cm}rr>{\raggedright\arraybackslash}p{2.0cm}r>{\raggedright\arraybackslash}p{1cm}p{1cm}p{1cm}p{1cm}}
\rowcolor{white}\caption{Articles in Journal {IEEE} Trans. Engineering Management (Total 1)}\\ \toprule
\rowcolor{white}\shortstack{Key\\Source} & Authors & Title (Colored by Open Access)& \shortstack{Details\\LC} & Cite & Year & \shortstack{Conference\\/Journal\\/School} & Pages & Relevance &\shortstack{Cites\\OC XR\\SC} & \shortstack{Refs\\OC\\XR} & \shortstack{Links\\Cites\\Refs}\\ \midrule\endhead
\bottomrule
\endfoot
ZhangW18 \href{https://doi.org/10.1109/TEM.2017.2785774}{ZhangW18} & \hyperref[auth:a571]{S. Zhang}, \hyperref[auth:a572]{S. Wang} & Flexible Assembly Job-Shop Scheduling With Sequence-Dependent Setup Times and Part Sharing in a Dynamic Environment: Constraint Programming Model, Mixed-Integer Programming Model, and Dispatching Rules & \hyperref[detail:ZhangW18]{Details} \href{../works/ZhangW18.pdf}{Yes} & \cite{ZhangW18} & 2018 & {IEEE} Trans. Engineering Management & 18 & \noindent{}\textbf{2.00} \textbf{2.00} \textbf{9.42} & 49 56 60 & 28 32 & 4 2 2\\
\end{longtable}
}

\subsection{{IEEE} Trans. Parallel Distributed Syst.}

\index{{IEEE} Trans. Parallel Distributed Syst.}
{\scriptsize
\begin{longtable}{>{\raggedright\arraybackslash}p{2.5cm}>{\raggedright\arraybackslash}p{4.5cm}>{\raggedright\arraybackslash}p{6.0cm}p{1.0cm}rr>{\raggedright\arraybackslash}p{2.0cm}r>{\raggedright\arraybackslash}p{1cm}p{1cm}p{1cm}p{1cm}}
\rowcolor{white}\caption{Articles in Journal {IEEE} Trans. Parallel Distributed Syst. (Total 1)}\\ \toprule
\rowcolor{white}\shortstack{Key\\Source} & Authors & Title (Colored by Open Access)& \shortstack{Details\\LC} & Cite & Year & \shortstack{Conference\\/Journal\\/School} & Pages & Relevance &\shortstack{Cites\\OC XR\\SC} & \shortstack{Refs\\OC\\XR} & \shortstack{Links\\Cites\\Refs}\\ \midrule\endhead
\bottomrule
\endfoot
BridiBLMB16 \href{https://doi.org/10.1109/TPDS.2016.2516997}{BridiBLMB16} & \hyperref[auth:a227]{T. Bridi}, \hyperref[auth:a225]{A. Bartolini}, \hyperref[auth:a142]{M. Lombardi}, \hyperref[auth:a143]{M. Milano}, \hyperref[auth:a245]{L. Benini} & \cellcolor{green!10}A Constraint Programming Scheduler for Heterogeneous High-Performance Computing Machines & \hyperref[detail:BridiBLMB16]{Details} \href{../works/BridiBLMB16.pdf}{Yes} & \cite{BridiBLMB16} & 2016 & {IEEE} Trans. Parallel Distributed Syst. & 14 & \noindent{}0.50 0.50 \textbf{20.89} & 17 18 21 & 22 34 & 4 2 2\\
\end{longtable}
}

\subsection{{IEEE} Trans. Syst. Man Cybern. Syst.}

\index{{IEEE} Trans. Syst. Man Cybern. Syst.}
{\scriptsize
\begin{longtable}{>{\raggedright\arraybackslash}p{2.5cm}>{\raggedright\arraybackslash}p{4.5cm}>{\raggedright\arraybackslash}p{6.0cm}p{1.0cm}rr>{\raggedright\arraybackslash}p{2.0cm}r>{\raggedright\arraybackslash}p{1cm}p{1cm}p{1cm}p{1cm}}
\rowcolor{white}\caption{Articles in Journal {IEEE} Trans. Syst. Man Cybern. Syst. (Total 1)}\\ \toprule
\rowcolor{white}\shortstack{Key\\Source} & Authors & Title (Colored by Open Access)& \shortstack{Details\\LC} & Cite & Year & \shortstack{Conference\\/Journal\\/School} & Pages & Relevance &\shortstack{Cites\\OC XR\\SC} & \shortstack{Refs\\OC\\XR} & \shortstack{Links\\Cites\\Refs}\\ \midrule\endhead
\bottomrule
\endfoot
ShinBBHO18 \href{https://doi.org/10.1109/TSMC.2017.2681623}{ShinBBHO18} & \hyperref[auth:a573]{S. Y. Shin}, \hyperref[auth:a574]{Y. Brun}, \hyperref[auth:a575]{H. Balasubramanian}, \hyperref[auth:a576]{P. L. Henneman}, \hyperref[auth:a577]{L. J. Osterweil} & \cellcolor{gold!20}Discrete-Event Simulation and Integer Linear Programming for Constraint-Aware Resource Scheduling & \hyperref[detail:ShinBBHO18]{Details} \href{../works/ShinBBHO18.pdf}{Yes} & \cite{ShinBBHO18} & 2018 & {IEEE} Trans. Syst. Man Cybern. Syst. & 16 & \noindent{}\textcolor{black!50}{0.00} \textcolor{black!50}{0.00} \textcolor{black!50}{0.00} & 9 9 12 & 31 39 & 0 0 0\\
\end{longtable}
}

\subsection{{OR} Spectrum}

\index{{OR} Spectrum}
{\scriptsize
\begin{longtable}{>{\raggedright\arraybackslash}p{2.5cm}>{\raggedright\arraybackslash}p{4.5cm}>{\raggedright\arraybackslash}p{6.0cm}p{1.0cm}rr>{\raggedright\arraybackslash}p{2.0cm}r>{\raggedright\arraybackslash}p{1cm}p{1cm}p{1cm}p{1cm}}
\rowcolor{white}\caption{Articles in Journal {OR} Spectrum (Total 3)}\\ \toprule
\rowcolor{white}\shortstack{Key\\Source} & Authors & Title (Colored by Open Access)& \shortstack{Details\\LC} & Cite & Year & \shortstack{Conference\\/Journal\\/School} & Pages & Relevance &\shortstack{Cites\\OC XR\\SC} & \shortstack{Refs\\OC\\XR} & \shortstack{Links\\Cites\\Refs}\\ \midrule\endhead
\bottomrule
\endfoot
NattafALR16 \href{https://doi.org/10.1007/s00291-015-0423-x}{NattafALR16} & \hyperref[auth:a81]{M. Nattaf}, \hyperref[auth:a6]{C. Artigues}, \hyperref[auth:a3]{P. Lopez}, \hyperref[auth:a979]{D. Rivreau} & \cellcolor{green!10}Energetic reasoning and mixed-integer linear programming for scheduling with a continuous resource and linear efficiency functions & \hyperref[detail:NattafALR16]{Details} \href{../works/NattafALR16.pdf}{Yes} & \cite{NattafALR16} & 2016 & {OR} Spectrum & 34 & \noindent{}\textcolor{black!50}{0.00} \textcolor{black!50}{0.00} \textbf{1.68} & 10 10 10 & 15 19 & 6 1 5\\
SchnellH15 \href{http://dx.doi.org/10.1007/s00291-015-0419-6}{SchnellH15} & \hyperref[auth:a950]{A. Schnell}, \hyperref[auth:a951]{R. F. Hartl} & On the efficient modeling and solution of the multi-mode resource-constrained project scheduling problem with generalized precedence relations & \hyperref[detail:SchnellH15]{Details} \href{../works/SchnellH15.pdf}{Yes} & \cite{SchnellH15} & 2015 & {OR} Spectrum & 21 & \noindent{}\textcolor{black!50}{0.00} \textcolor{black!50}{0.00} \textbf{2.11} & 24 27 31 & 20 30 & 19 8 11\\
Timpe02 \href{https://doi.org/10.1007/s00291-002-0107-1}{Timpe02} & \hyperref[auth:a673]{C. Timpe} & Solving planning and scheduling problems with combined integer and constraint programming & \hyperref[detail:Timpe02]{Details} \href{../works/Timpe02.pdf}{Yes} & \cite{Timpe02} & 2002 & {OR} Spectrum & 18 & \noindent{}\textbf{1.00} \textbf{1.00} \textbf{7.47} & 42 42 54 & 0 0 & 20 20 0\\
\end{longtable}
}



\clearpage
\section{Background Works}

{\scriptsize
\begin{longtable}{>{\raggedright\arraybackslash}p{3cm}>{\raggedright\arraybackslash}p{6cm}>{\raggedright\arraybackslash}p{6.5cm}rrrp{2.5cm}rrrrr}
\rowcolor{white}\caption{Works from bibtex (Total 23)}\\ \toprule
\rowcolor{white}\shortstack{Key\\Source} & Authors & Title & LC & Cite & Year & \shortstack{Conference\\/Journal\\/School} & Pages & \shortstack{Nr\\Cites} & \shortstack{Nr\\Refs} & b & c \\ \midrule\endhead
\bottomrule
\endfoot
HartmannB22 \href{http://dx.doi.org/10.1016/j.ejor.2021.05.004}{HartmannB22} & S. Hartmann, D. Briskorn & An updated survey of variants and extensions of the resource-constrained project scheduling problem & \href{../works/HartmannB22.pdf}{Yes} & \cite{HartmannB22} & 2022 & European Journal of Operational Research & 14 & 55 & 196 & No & n/a\\
LamGSHD20 \href{http://dx.doi.org/10.1007/s43069-020-00023-2}{LamGSHD20} & E. Lam, \hyperref[auth:a187]{G. Gange}, \hyperref[auth:a126]{Peter J. Stuckey}, \hyperref[auth:a149]{Pascal Van Hentenryck}, Jip J. Dekker & Nutmeg: a MIP and CP Hybrid Solver Using Branch-and-Check & \href{../works/LamGSHD20.pdf}{Yes} & \cite{LamGSHD20} & 2020 & SN Operations Research Forum & 27 & 7 & 28 & No & n/a\\
RahmanianiCGR17 \href{http://dx.doi.org/10.1016/j.ejor.2016.12.005}{RahmanianiCGR17} & R. Rahmaniani, Teodor Gabriel Crainic, \hyperref[auth:a626]{M. Gendreau}, W. Rei & The Benders decomposition algorithm: A literature review & \href{../works/RahmanianiCGR17.pdf}{Yes} & \cite{RahmanianiCGR17} & 2017 & European Journal of Operational Research & 17 & 386 & 113 & No & n/a\\
HartmannB10 \href{http://dx.doi.org/10.1016/j.ejor.2009.11.005}{HartmannB10} & S. Hartmann, D. Briskorn & A survey of variants and extensions of the resource-constrained project scheduling problem & \href{../works/HartmannB10.pdf}{Yes} & \cite{HartmannB10} & 2010 & European Journal of Operational Research & 14 & 577 & 177 & No & n/a\\
YunesAH10 \href{http://dx.doi.org/10.1287/opre.1090.0733}{YunesAH10} & T. Yunes, Ionuţ D. Aron, \hyperref[auth:a162]{John N. Hooker} & An Integrated Solver for Optimization Problems & \href{../works/YunesAH10.pdf}{Yes} & \cite{YunesAH10} & 2010 & Operations Research & 16 & 25 & 38 & No & n/a\\
NethercoteSBBDT07 \href{https://doi.org/10.1007/978-3-540-74970-7\_38}{NethercoteSBBDT07} & N. Nethercote, \hyperref[auth:a126]{Peter J. Stuckey}, R. Becket, S. Brand, Gregory J. Duck, G. Tack & MiniZinc: Towards a Standard {CP} Modelling Language & \href{../works/NethercoteSBBDT07.pdf}{Yes} & \cite{NethercoteSBBDT07} & 2007 & CP 2007 & 15 & 344 & 5 & No & n/a\\
KolischH06 \href{http://dx.doi.org/10.1016/j.ejor.2005.01.065}{KolischH06} & \hyperref[auth:a447]{R. Kolisch}, S. Hartmann & Experimental investigation of heuristics for resource-constrained project scheduling: An update & \href{../works/KolischH06.pdf}{Yes} & \cite{KolischH06} & 2006 & European Journal of Operational Research & 15 & 503 & 62 & No & n/a\\
BockmayrH05 \href{http://dx.doi.org/10.1016/s0927-0507(05)12010-6}{BockmayrH05} & A. Bockmayr, \hyperref[auth:a162]{John N. Hooker} & Constraint Programming & \href{../works/BockmayrH05.pdf}{Yes} & \cite{BockmayrH05} & 2005 & Handbooks in Operations Research and Management Science & 42 & 12 & 52 & No & n/a\\
AronHY2004 \href{http://dx.doi.org/10.1007/978-3-540-24664-0_2}{AronHY2004} & I. Aron, \hyperref[auth:a162]{John N. Hooker}, Tallys H. Yunes & SIMPL: A System for Integrating Optimization Techniques & \href{../works/AronHY2004.pdf}{Yes} & \cite{AronHY2004} & 2004 & CPAIOR 2004 & 16 & 16 & 23 & No & n/a\\
BruckerDMNP99 \href{http://dx.doi.org/10.1016/s0377-2217(98)00204-5}{BruckerDMNP99} & P. Brucker, A. Drexl, R. M\"{o}hring, K. Neumann, \hyperref[auth:a445]{E. Pesch} & Resource-constrained project scheduling: Notation,  classification,  models,  and methods & \href{../works/BruckerDMNP99.pdf}{Yes} & \cite{BruckerDMNP99} & 1999 & European Journal of Operational Research & 39 & 990 & 137 & No & n/a\\
Shaw98 \href{https://doi.org/10.1007/3-540-49481-2\_30}{Shaw98} & \hyperref[auth:a120]{P. Shaw} & Using Constraint Programming and Local Search Methods to Solve Vehicle Routing Problems & \href{../works/Shaw98.pdf}{Yes} & \cite{Shaw98} & 1998 & CP 1998 & 15 & 630 & 11 & No & n/a\\
KolischS97 \href{http://dx.doi.org/10.1016/s0377-2217(96)00170-1}{KolischS97} & \hyperref[auth:a447]{R. Kolisch}, A. Sprecher & PSPLIB - A project scheduling problem library & \href{../works/KolischS97.pdf}{Yes} & \cite{KolischS97} & 1997 & European Journal of Operational Research & 12 & 840 & 18 & No & n/a\\
CarlierP94 \href{http://dx.doi.org/10.1016/0377-2217(94)90379-4}{CarlierP94} & \hyperref[auth:a857]{J. Carlier}, \hyperref[auth:a858]{E. Pinson} & Adjustment of heads and tails for the job-shop problem & \href{../works/CarlierP94.pdf}{Yes} & \cite{CarlierP94} & 1994 & European Journal of Operational Research & 16 & 151 & 10 & No & n/a\\
Taillard93 \href{http://dx.doi.org/10.1016/0377-2217(93)90182-m}{Taillard93} & E. Taillard & Benchmarks for basic scheduling problems & \href{../works/Taillard93.pdf}{Yes} & \cite{Taillard93} & 1993 & European Journal of Operational Research & 8 & 1568 & 6 & No & n/a\\
ApplegateC91 \href{http://dx.doi.org/10.1287/ijoc.3.2.149}{ApplegateC91} & D. Applegate, W. Cook & A Computational Study of the Job-Shop Scheduling Problem & \href{../works/ApplegateC91.pdf}{Yes} & \cite{ApplegateC91} & 1991 & ORSA Journal on Computing & 8 & 536 & 0 & No & n/a\\
DechterMP91 \href{http://dx.doi.org/10.1016/0004-3702(91)90006-6}{DechterMP91} & \hyperref[auth:a303]{R. Dechter}, I. Meiri, J. Pearl & Temporal constraint networks & \href{../works/DechterMP91.pdf}{Yes} & \cite{DechterMP91} & 1991 & Artificial Intelligence & 35 & 879 & 28 & No & n/a\\
CarlierP90 \href{http://dx.doi.org/10.1007/bf03543071}{CarlierP90} & \hyperref[auth:a857]{J. Carlier}, \hyperref[auth:a858]{E. Pinson} & A practical use of Jackson's preemptive schedule for solving the job shop problem & \href{../works/CarlierP90.pdf}{Yes} & \cite{CarlierP90} & 1990 & Annals of Operations Research & 19 & 112 & 11 & No & n/a\\
CarlierP89 \href{http://dx.doi.org/10.1287/mnsc.35.2.164}{CarlierP89} & \hyperref[auth:a857]{J. Carlier}, \hyperref[auth:a858]{E. Pinson} & An Algorithm for Solving the Job-Shop Problem & \href{../works/CarlierP89.pdf}{Yes} & \cite{CarlierP89} & 1989 & Management Science & 14 & 516 & 0 & No & n/a\\
AdamsBZ88 \href{http://dx.doi.org/10.1287/mnsc.34.3.391}{AdamsBZ88} & J. Adams, E. Balas, D. Zawack & The Shifting Bottleneck Procedure for Job Shop Scheduling & \href{../works/AdamsBZ88.pdf}{Yes} & \cite{AdamsBZ88} & 1988 & Management Science & 12 & 1054 & 0 & No & n/a\\
DincbasHSAGB88 \href{}{DincbasHSAGB88} & \hyperref[auth:a726]{M. Dincbas}, \hyperref[auth:a149]{Pascal Van Hentenryck}, \hyperref[auth:a17]{H. Simonis}, \hyperref[auth:a734]{A. Aggoun}, T. Graf, F. Berthier & The Constraint Logic Programming Language {CHIP} & \href{../works/DincbasHSAGB88.pdf}{Yes} & \cite{DincbasHSAGB88} & 1988 & FGCS 1988 & 10 & 0 & 0 & No & n/a\\
BlazewiczLK83 \href{https://doi.org/10.1016/0166-218X(83)90012-4}{BlazewiczLK83} & \hyperref[auth:a775]{J. Blazewicz}, Jan Karel Lenstra, A. H. G. Rinnooy Kan & Scheduling subject to resource constraints: classification and complexity & \href{../works/BlazewiczLK83.pdf}{Yes} & \cite{BlazewiczLK83} & 1983 & Discret. Appl. Math. & 14 & 947 & 6 & No & n/a\\
Lauriere78 \href{http://dx.doi.org/10.1016/0004-3702(78)90029-2}{Lauriere78} & J. Lauriere & A language and a program for stating and solving combinatorial problems & No & \cite{Lauriere78} & 1978 & Artificial Intelligence & null & 149 & 14 & No & n/a\\
Benders62 \href{http://dx.doi.org/10.1007/bf01386316}{Benders62} & Jacques F. Benders & Partitioning procedures for solving mixed-variables programming problems & \href{../works/Benders62.pdf}{Yes} & \cite{Benders62} & 1962 & Numerische Mathematik & 15 & 2583 & 6 & No & n/a\\
\end{longtable}
}



\clearpage
\section{Most Similar Works}

The following table shows the most similar works for each work. The first line shows the five works most similar by references and citations, the second line shows the similarity by Euclidean distance of the feature vectors. The next two entries are based on the dot product similarity, resp. the cosine similarity. Note that the first two entries will have low values for similar items, as they are based on distance. The last two entries use high values for similar items, as they use a similarity value, not a distance.

Fewer than five entries are shown if no similar entries are found. Colour coding is based on percentiles (80, 90, 95, 98, 99, 99.5 percentiles for similarity, 20, 10, 5, 2, 1, 0.5 percentiles for distance) of all similarity values found.

{\scriptsize
\begin{longtable}{rlllll}
\caption{Most Similar Works}\\ \toprule
Work & 1 & 2 & 3 & 4 & 5 \\ \midrule\endhead
\bottomrule
\endfoot
\index{AalianPG23}\href{../works/AalianPG23.pdf}{AalianPG23} R\&C\\
Euclid& \cellcolor{yellow!20}\href{../works/AstrandJZ18.pdf}{AstrandJZ18} (0.28)& \cellcolor{yellow!20}\href{../works/PerezGSL23.pdf}{PerezGSL23} (0.28)& \cellcolor{yellow!20}\href{../works/abs-2312-13682.pdf}{abs-2312-13682} (0.28)& \cellcolor{green!20}\href{../works/BenderWS21.pdf}{BenderWS21} (0.30)& \cellcolor{green!20}\href{../works/BockmayrP06.pdf}{BockmayrP06} (0.30)\\
Dot& \cellcolor{red!40}\href{../works/Dejemeppe16.pdf}{Dejemeppe16} (113.00)& \cellcolor{red!40}\href{../works/LaborieRSV18.pdf}{LaborieRSV18} (112.00)& \cellcolor{red!40}\href{../works/Groleaz21.pdf}{Groleaz21} (110.00)& \cellcolor{red!40}\href{../works/AwadMDMT22.pdf}{AwadMDMT22} (107.00)& \cellcolor{red!40}\href{../works/Astrand21.pdf}{Astrand21} (106.00)\\
Cosine& \cellcolor{red!40}\href{../works/AstrandJZ18.pdf}{AstrandJZ18} (0.74)& \cellcolor{red!40}\href{../works/PerezGSL23.pdf}{PerezGSL23} (0.74)& \cellcolor{red!40}\href{../works/abs-2312-13682.pdf}{abs-2312-13682} (0.74)& \cellcolor{red!40}\href{../works/BridiBLMB16.pdf}{BridiBLMB16} (0.71)& \cellcolor{red!40}\href{../works/BenderWS21.pdf}{BenderWS21} (0.70)\\
\index{AbdennadherS99}\href{../works/AbdennadherS99.pdf}{AbdennadherS99} R\&C\\
Euclid& \cellcolor{red!40}\href{../works/FalaschiGMP97.pdf}{FalaschiGMP97} (0.24)& \cellcolor{red!40}\href{../works/Touraivane95.pdf}{Touraivane95} (0.24)& \cellcolor{red!20}\href{../works/FeldmanG89.pdf}{FeldmanG89} (0.25)& \cellcolor{red!20}\href{../works/SmithBHW96.pdf}{SmithBHW96} (0.26)& \cellcolor{yellow!20}\href{../works/JelinekB16.pdf}{JelinekB16} (0.27)\\
Dot& \cellcolor{red!40}\href{../works/Wallace96.pdf}{Wallace96} (66.00)& \cellcolor{red!40}\href{../works/ZarandiASC20.pdf}{ZarandiASC20} (63.00)& \cellcolor{red!40}\href{../works/LammaMM97.pdf}{LammaMM97} (60.00)& \cellcolor{red!40}\href{../works/Simonis07.pdf}{Simonis07} (59.00)& \cellcolor{red!40}\href{../works/Baptiste02.pdf}{Baptiste02} (59.00)\\
Cosine& \cellcolor{red!40}\href{../works/FalaschiGMP97.pdf}{FalaschiGMP97} (0.68)& \cellcolor{red!40}\href{../works/Touraivane95.pdf}{Touraivane95} (0.66)& \cellcolor{red!40}\href{../works/Bartak02.pdf}{Bartak02} (0.65)& \cellcolor{red!40}\href{../works/LammaMM97.pdf}{LammaMM97} (0.64)& \cellcolor{red!40}\href{../works/WeilHFP95.pdf}{WeilHFP95} (0.62)\\
\index{AbidinK20}\href{../works/AbidinK20.pdf}{AbidinK20} R\&C& \cellcolor{red!40}KizilayC20 (0.82)& \cellcolor{red!20}PinarbasiA20 (0.87)& \cellcolor{yellow!20}Pinarbasi21 (0.91)& \cellcolor{yellow!20}\href{../works/Alaka21.pdf}{Alaka21} (0.92)& \cellcolor{yellow!20}\href{../works/OzturkTHO13.pdf}{OzturkTHO13} (0.92)\\
Euclid& \cellcolor{red!20}\href{../works/TopalogluSS12.pdf}{TopalogluSS12} (0.25)& \cellcolor{yellow!20}\href{../works/LozanoCDS12.pdf}{LozanoCDS12} (0.28)& \cellcolor{green!20}\href{../works/BukchinR18.pdf}{BukchinR18} (0.29)& \cellcolor{green!20}\href{../works/NishikawaSTT18.pdf}{NishikawaSTT18} (0.29)& \cellcolor{green!20}\href{../works/AlakaPY19.pdf}{AlakaPY19} (0.29)\\
Dot& \cellcolor{red!40}\href{../works/Dejemeppe16.pdf}{Dejemeppe16} (117.00)& \cellcolor{red!40}\href{../works/ZarandiASC20.pdf}{ZarandiASC20} (115.00)& \cellcolor{red!40}\href{../works/Groleaz21.pdf}{Groleaz21} (113.00)& \cellcolor{red!40}\href{../works/Lunardi20.pdf}{Lunardi20} (112.00)& \cellcolor{red!40}\href{../works/IsikYA23.pdf}{IsikYA23} (108.00)\\
Cosine& \cellcolor{red!40}\href{../works/TopalogluSS12.pdf}{TopalogluSS12} (0.81)& \cellcolor{red!40}\href{../works/CilKLO22.pdf}{CilKLO22} (0.78)& \cellcolor{red!40}\href{../works/PinarbasiAY19.pdf}{PinarbasiAY19} (0.74)& \cellcolor{red!40}\href{../works/Edis21.pdf}{Edis21} (0.74)& \cellcolor{red!40}\href{../works/LozanoCDS12.pdf}{LozanoCDS12} (0.73)\\
\index{AbohashimaEG21}\href{../works/AbohashimaEG21.pdf}{AbohashimaEG21} R\&C\\
Euclid& \cellcolor{red!20}\href{../works/LiuJ06.pdf}{LiuJ06} (0.25)& \cellcolor{yellow!20}\href{../works/Hunsberger08.pdf}{Hunsberger08} (0.27)& \cellcolor{yellow!20}\href{../works/Baptiste09.pdf}{Baptiste09} (0.27)& \cellcolor{yellow!20}\href{../works/CarchraeBF05.pdf}{CarchraeBF05} (0.27)& \cellcolor{yellow!20}\href{../works/KovacsEKV05.pdf}{KovacsEKV05} (0.28)\\
Dot& \cellcolor{red!40}\href{../works/Lunardi20.pdf}{Lunardi20} (77.00)& \cellcolor{red!40}\href{../works/ZarandiASC20.pdf}{ZarandiASC20} (73.00)& \cellcolor{red!40}\href{../works/Groleaz21.pdf}{Groleaz21} (72.00)& \cellcolor{red!40}\href{../works/IsikYA23.pdf}{IsikYA23} (65.00)& \cellcolor{red!40}\href{../works/Froger16.pdf}{Froger16} (62.00)\\
Cosine& \cellcolor{red!40}\href{../works/CilKLO22.pdf}{CilKLO22} (0.55)& \cellcolor{red!40}\href{../works/LiuJ06.pdf}{LiuJ06} (0.54)& \cellcolor{red!40}\href{../works/TranAB16.pdf}{TranAB16} (0.54)& \cellcolor{red!40}\href{../works/ParkUJR19.pdf}{ParkUJR19} (0.54)& \cellcolor{red!40}\href{../works/GroleazNS20a.pdf}{GroleazNS20a} (0.53)\\
\index{AbreuAPNM21}\href{../works/AbreuAPNM21.pdf}{AbreuAPNM21} R\&C& \cellcolor{red!40}\href{../works/AbreuN22.pdf}{AbreuN22} (0.75)& \cellcolor{red!40}\href{../works/MejiaY20.pdf}{MejiaY20} (0.77)& \cellcolor{red!40}\href{../works/AbreuNP23.pdf}{AbreuNP23} (0.82)& \cellcolor{red!40}\href{../works/MalapertCGJLR12.pdf}{MalapertCGJLR12} (0.83)& \cellcolor{red!40}\href{../works/AbreuPNF23.pdf}{AbreuPNF23} (0.83)\\
Euclid& \cellcolor{yellow!20}\href{../works/AbreuN22.pdf}{AbreuN22} (0.26)& \cellcolor{green!20}\href{../works/AbreuPNF23.pdf}{AbreuPNF23} (0.29)& \cellcolor{green!20}\href{../works/AbreuNP23.pdf}{AbreuNP23} (0.29)& \cellcolor{black!20}\href{../works/MejiaY20.pdf}{MejiaY20} (0.34)& \href{../works/BillautHL12.pdf}{BillautHL12} (0.38)\\
Dot& \cellcolor{red!40}\href{../works/ZarandiASC20.pdf}{ZarandiASC20} (208.00)& \cellcolor{red!40}\href{../works/Groleaz21.pdf}{Groleaz21} (194.00)& \cellcolor{red!40}\href{../works/AbreuN22.pdf}{AbreuN22} (192.00)& \cellcolor{red!40}\href{../works/AbreuPNF23.pdf}{AbreuPNF23} (185.00)& \cellcolor{red!40}\href{../works/AbreuNP23.pdf}{AbreuNP23} (184.00)\\
Cosine& \cellcolor{red!40}\href{../works/AbreuN22.pdf}{AbreuN22} (0.89)& \cellcolor{red!40}\href{../works/AbreuPNF23.pdf}{AbreuPNF23} (0.87)& \cellcolor{red!40}\href{../works/AbreuNP23.pdf}{AbreuNP23} (0.87)& \cellcolor{red!40}\href{../works/MejiaY20.pdf}{MejiaY20} (0.80)& \cellcolor{red!40}\href{../works/YuraszeckMPV22.pdf}{YuraszeckMPV22} (0.75)\\
\index{AbreuN22}\href{../works/AbreuN22.pdf}{AbreuN22} R\&C& \cellcolor{red!40}\href{../works/AbreuNP23.pdf}{AbreuNP23} (0.73)& \cellcolor{red!40}\href{../works/AbreuAPNM21.pdf}{AbreuAPNM21} (0.75)& \cellcolor{red!40}\href{../works/MejiaY20.pdf}{MejiaY20} (0.79)& \cellcolor{red!40}\href{../works/AbreuPNF23.pdf}{AbreuPNF23} (0.79)& \cellcolor{red!40}\href{../works/KelbelH11.pdf}{KelbelH11} (0.85)\\
Euclid& \cellcolor{yellow!20}\href{../works/AbreuAPNM21.pdf}{AbreuAPNM21} (0.26)& \cellcolor{green!20}\href{../works/AbreuPNF23.pdf}{AbreuPNF23} (0.29)& \cellcolor{green!20}\href{../works/AbreuNP23.pdf}{AbreuNP23} (0.30)& \href{../works/MejiaY20.pdf}{MejiaY20} (0.38)& \href{../works/GedikKEK18.pdf}{GedikKEK18} (0.41)\\
Dot& \cellcolor{red!40}\href{../works/ZarandiASC20.pdf}{ZarandiASC20} (229.00)& \cellcolor{red!40}\href{../works/Groleaz21.pdf}{Groleaz21} (214.00)& \cellcolor{red!40}\href{../works/AbreuPNF23.pdf}{AbreuPNF23} (205.00)& \cellcolor{red!40}\href{../works/AbreuNP23.pdf}{AbreuNP23} (202.00)& \cellcolor{red!40}\href{../works/Lunardi20.pdf}{Lunardi20} (200.00)\\
Cosine& \cellcolor{red!40}\href{../works/AbreuAPNM21.pdf}{AbreuAPNM21} (0.89)& \cellcolor{red!40}\href{../works/AbreuPNF23.pdf}{AbreuPNF23} (0.87)& \cellcolor{red!40}\href{../works/AbreuNP23.pdf}{AbreuNP23} (0.86)& \cellcolor{red!40}\href{../works/MejiaY20.pdf}{MejiaY20} (0.78)& \cellcolor{red!40}\href{../works/MengZRZL20.pdf}{MengZRZL20} (0.75)\\
\index{AbreuNP23}\href{../works/AbreuNP23.pdf}{AbreuNP23} R\&C& \cellcolor{red!40}\href{../works/AbreuN22.pdf}{AbreuN22} (0.73)& \cellcolor{red!40}\href{../works/AwadMDMT22.pdf}{AwadMDMT22} (0.75)& \cellcolor{red!40}\href{../works/AbreuPNF23.pdf}{AbreuPNF23} (0.78)& \cellcolor{red!40}\href{../works/AbreuAPNM21.pdf}{AbreuAPNM21} (0.82)& \cellcolor{red!40}\href{../works/HeinzNVH22.pdf}{HeinzNVH22} (0.83)\\
Euclid& \cellcolor{green!20}\href{../works/AbreuAPNM21.pdf}{AbreuAPNM21} (0.29)& \cellcolor{green!20}\href{../works/AbreuN22.pdf}{AbreuN22} (0.30)& \cellcolor{blue!20}\href{../works/AbreuPNF23.pdf}{AbreuPNF23} (0.33)& \href{../works/MejiaY20.pdf}{MejiaY20} (0.38)& \href{../works/MengZRZL20.pdf}{MengZRZL20} (0.40)\\
Dot& \cellcolor{red!40}\href{../works/ZarandiASC20.pdf}{ZarandiASC20} (202.00)& \cellcolor{red!40}\href{../works/AbreuN22.pdf}{AbreuN22} (202.00)& \cellcolor{red!40}\href{../works/AbreuPNF23.pdf}{AbreuPNF23} (195.00)& \cellcolor{red!40}\href{../works/Groleaz21.pdf}{Groleaz21} (193.00)& \cellcolor{red!40}\href{../works/MengZRZL20.pdf}{MengZRZL20} (187.00)\\
Cosine& \cellcolor{red!40}\href{../works/AbreuAPNM21.pdf}{AbreuAPNM21} (0.87)& \cellcolor{red!40}\href{../works/AbreuN22.pdf}{AbreuN22} (0.86)& \cellcolor{red!40}\href{../works/AbreuPNF23.pdf}{AbreuPNF23} (0.84)& \cellcolor{red!40}\href{../works/MejiaY20.pdf}{MejiaY20} (0.78)& \cellcolor{red!40}\href{../works/MengZRZL20.pdf}{MengZRZL20} (0.78)\\
\index{AbreuPNF23}\href{../works/AbreuPNF23.pdf}{AbreuPNF23} R\&C& \cellcolor{red!40}\href{../works/AbreuNP23.pdf}{AbreuNP23} (0.78)& \cellcolor{red!40}\href{../works/AbreuN22.pdf}{AbreuN22} (0.79)& \cellcolor{red!40}\href{../works/AbreuAPNM21.pdf}{AbreuAPNM21} (0.83)& \cellcolor{yellow!20}\href{../works/MejiaY20.pdf}{MejiaY20} (0.92)& \cellcolor{green!20}\href{../works/YuraszeckMPV22.pdf}{YuraszeckMPV22} (0.95)\\
Euclid& \cellcolor{green!20}\href{../works/AbreuAPNM21.pdf}{AbreuAPNM21} (0.29)& \cellcolor{green!20}\href{../works/AbreuN22.pdf}{AbreuN22} (0.29)& \cellcolor{blue!20}\href{../works/AbreuNP23.pdf}{AbreuNP23} (0.33)& \href{../works/MejiaY20.pdf}{MejiaY20} (0.38)& \href{../works/OujanaAYB22.pdf}{OujanaAYB22} (0.40)\\
Dot& \cellcolor{red!40}\href{../works/ZarandiASC20.pdf}{ZarandiASC20} (231.00)& \cellcolor{red!40}\href{../works/Groleaz21.pdf}{Groleaz21} (214.00)& \cellcolor{red!40}\href{../works/AbreuN22.pdf}{AbreuN22} (205.00)& \cellcolor{red!40}\href{../works/Lunardi20.pdf}{Lunardi20} (198.00)& \cellcolor{red!40}\href{../works/PrataAN23.pdf}{PrataAN23} (196.00)\\
Cosine& \cellcolor{red!40}\href{../works/AbreuN22.pdf}{AbreuN22} (0.87)& \cellcolor{red!40}\href{../works/AbreuAPNM21.pdf}{AbreuAPNM21} (0.87)& \cellcolor{red!40}\href{../works/AbreuNP23.pdf}{AbreuNP23} (0.84)& \cellcolor{red!40}\href{../works/MejiaY20.pdf}{MejiaY20} (0.78)& \cellcolor{red!40}\href{../works/PrataAN23.pdf}{PrataAN23} (0.77)\\
\index{AbrilSB05}\href{../works/AbrilSB05.pdf}{AbrilSB05} R\&C\\
Euclid& \cellcolor{red!40}\href{../works/Baptiste09.pdf}{Baptiste09} (0.12)& \cellcolor{red!40}\href{../works/FrostD98.pdf}{FrostD98} (0.13)& \cellcolor{red!40}\href{../works/CarchraeBF05.pdf}{CarchraeBF05} (0.14)& \cellcolor{red!40}\href{../works/MaraveliasG04.pdf}{MaraveliasG04} (0.14)& \cellcolor{red!40}\href{../works/FeldmanG89.pdf}{FeldmanG89} (0.14)\\
Dot& \cellcolor{red!40}\href{../works/ZarandiASC20.pdf}{ZarandiASC20} (23.00)& \cellcolor{red!40}\href{../works/Lemos21.pdf}{Lemos21} (21.00)& \cellcolor{red!40}\href{../works/LammaMM97.pdf}{LammaMM97} (20.00)& \cellcolor{red!40}\href{../works/BartakSR10.pdf}{BartakSR10} (20.00)& \cellcolor{red!40}\href{../works/Beck99.pdf}{Beck99} (19.00)\\
Cosine& \cellcolor{red!40}\href{../works/FeldmanG89.pdf}{FeldmanG89} (0.72)& \cellcolor{red!40}\href{../works/BocewiczBB09.pdf}{BocewiczBB09} (0.60)& \cellcolor{red!40}\href{../works/BartakS11.pdf}{BartakS11} (0.59)& \cellcolor{red!40}\href{../works/GelainPRVW17.pdf}{GelainPRVW17} (0.57)& \cellcolor{red!40}\href{../works/WolinskiKG04.pdf}{WolinskiKG04} (0.57)\\
\index{AchterbergBKW08}\href{../works/AchterbergBKW08.pdf}{AchterbergBKW08} R\&C& \cellcolor{red!40}\href{../works/MilanoW09.pdf}{MilanoW09} (0.81)& \cellcolor{red!20}\href{../works/Hooker05b.pdf}{Hooker05b} (0.87)& \cellcolor{red!20}\href{../works/JainG01.pdf}{JainG01} (0.88)& \cellcolor{red!20}AggounMV08 (0.90)& \cellcolor{yellow!20}Hooker06a (0.92)\\
Euclid& \cellcolor{yellow!20}\href{../works/Davis87.pdf}{Davis87} (0.26)& \cellcolor{yellow!20}\href{../works/ZibranR11.pdf}{ZibranR11} (0.27)& \cellcolor{yellow!20}\href{../works/BofillGSV15.pdf}{BofillGSV15} (0.27)& \cellcolor{yellow!20}\href{../works/CireCH13.pdf}{CireCH13} (0.28)& \cellcolor{yellow!20}\href{../works/Hooker05b.pdf}{Hooker05b} (0.28)\\
Dot& \cellcolor{red!40}\href{../works/Malapert11.pdf}{Malapert11} (62.00)& \cellcolor{red!40}\href{../works/MilanoW09.pdf}{MilanoW09} (61.00)& \cellcolor{red!40}\href{../works/Groleaz21.pdf}{Groleaz21} (59.00)& \cellcolor{red!40}\href{../works/ColT22.pdf}{ColT22} (58.00)& \cellcolor{red!40}\href{../works/HookerH17.pdf}{HookerH17} (58.00)\\
Cosine& \cellcolor{red!40}\href{../works/CireCH13.pdf}{CireCH13} (0.63)& \cellcolor{red!40}\href{../works/Thorsteinsson01.pdf}{Thorsteinsson01} (0.62)& \cellcolor{red!40}\href{../works/NishikawaSTT18.pdf}{NishikawaSTT18} (0.57)& \cellcolor{red!40}\href{../works/HarjunkoskiJG00.pdf}{HarjunkoskiJG00} (0.57)& \cellcolor{red!40}\href{../works/BukchinR18.pdf}{BukchinR18} (0.57)\\
\index{Acuna-AgostMFG09}\href{../works/Acuna-AgostMFG09.pdf}{Acuna-AgostMFG09} R\&C& \cellcolor{black!20}\href{../works/MarliereSPR23.pdf}{MarliereSPR23} (0.99)& \cellcolor{black!20}\href{../works/Rodriguez07.pdf}{Rodriguez07} (0.99)\\
Euclid& \cellcolor{red!40}\href{../works/Baptiste09.pdf}{Baptiste09} (0.16)& \cellcolor{red!40}\href{../works/AbrilSB05.pdf}{AbrilSB05} (0.18)& \cellcolor{red!40}\href{../works/CarchraeBF05.pdf}{CarchraeBF05} (0.19)& \cellcolor{red!40}\href{../works/Caballero23.pdf}{Caballero23} (0.20)& \cellcolor{red!40}\href{../works/FrostD98.pdf}{FrostD98} (0.20)\\
Dot& \cellcolor{red!40}\href{../works/Lemos21.pdf}{Lemos21} (40.00)& \cellcolor{red!40}\href{../works/MarliereSPR23.pdf}{MarliereSPR23} (39.00)& \cellcolor{red!40}\href{../works/CappartS17.pdf}{CappartS17} (36.00)& \cellcolor{red!40}\href{../works/ZarandiASC20.pdf}{ZarandiASC20} (35.00)& \cellcolor{red!40}\href{../works/PourDERB18.pdf}{PourDERB18} (33.00)\\
Cosine& \cellcolor{red!40}\href{../works/RodriguezDG02.pdf}{RodriguezDG02} (0.65)& \cellcolor{red!40}\href{../works/Baptiste09.pdf}{Baptiste09} (0.65)& \cellcolor{red!40}\href{../works/MartinPY01.pdf}{MartinPY01} (0.62)& \cellcolor{red!40}\href{../works/CappartS17.pdf}{CappartS17} (0.61)& \cellcolor{red!40}\href{../works/BarzegaranZP20.pdf}{BarzegaranZP20} (0.59)\\
\index{Adelgren2023}\href{../works/Adelgren2023.pdf}{Adelgren2023} R\&C& \cellcolor{green!20}\href{../works/AwadMDMT22.pdf}{AwadMDMT22} (0.94)& \cellcolor{green!20}\href{../works/RoePS05.pdf}{RoePS05} (0.95)& \cellcolor{green!20}\href{../works/MaraveliasCG04.pdf}{MaraveliasCG04} (0.96)& \cellcolor{green!20}\href{../works/SadykovW06.pdf}{SadykovW06} (0.96)& \cellcolor{blue!20}\href{../works/Beck10.pdf}{Beck10} (0.96)\\
Euclid& \cellcolor{blue!20}\href{../works/FontaineMH16.pdf}{FontaineMH16} (0.33)& \cellcolor{black!20}\href{../works/BenediktSMVH18.pdf}{BenediktSMVH18} (0.34)& \cellcolor{black!20}\href{../works/Jans09.pdf}{Jans09} (0.35)& \cellcolor{black!20}\href{../works/CatusseCBL16.pdf}{CatusseCBL16} (0.35)& \cellcolor{black!20}\href{../works/NishikawaSTT19.pdf}{NishikawaSTT19} (0.35)\\
Dot& \cellcolor{red!40}\href{../works/Groleaz21.pdf}{Groleaz21} (148.00)& \cellcolor{red!40}\href{../works/ZarandiASC20.pdf}{ZarandiASC20} (138.00)& \cellcolor{red!40}\href{../works/Astrand21.pdf}{Astrand21} (129.00)& \cellcolor{red!40}\href{../works/Baptiste02.pdf}{Baptiste02} (129.00)& \cellcolor{red!40}\href{../works/NaderiRR23.pdf}{NaderiRR23} (127.00)\\
Cosine& \cellcolor{red!40}\href{../works/HeinzNVH22.pdf}{HeinzNVH22} (0.70)& \cellcolor{red!40}\href{../works/FontaineMH16.pdf}{FontaineMH16} (0.70)& \cellcolor{red!40}\href{../works/Ham18a.pdf}{Ham18a} (0.70)& \cellcolor{red!40}\href{../works/RoshanaeiLAU17.pdf}{RoshanaeiLAU17} (0.69)& \cellcolor{red!40}\href{../works/HamFC17.pdf}{HamFC17} (0.67)\\
\index{AfsarVPG23}\href{../works/AfsarVPG23.pdf}{AfsarVPG23} R\&C& \cellcolor{yellow!20}\href{../works/ColT19.pdf}{ColT19} (0.92)& \cellcolor{green!20}\href{../works/ColT2019a.pdf}{ColT2019a} (0.93)& \cellcolor{green!20}\href{../works/ColT22.pdf}{ColT22} (0.95)& \cellcolor{green!20}\href{../works/Fatemi-AnarakiTFV23.pdf}{Fatemi-AnarakiTFV23} (0.95)& \cellcolor{green!20}\href{../works/KuB16.pdf}{KuB16} (0.96)\\
Euclid& \cellcolor{black!20}\href{../works/ZhangYW21.pdf}{ZhangYW21} (0.36)& \cellcolor{black!20}\href{../works/Beck07.pdf}{Beck07} (0.36)& \cellcolor{black!20}\href{../works/LiFJZLL22.pdf}{LiFJZLL22} (0.37)& \cellcolor{black!20}\href{../works/KhayatLR06.pdf}{KhayatLR06} (0.37)& \href{../works/BeckPS03.pdf}{BeckPS03} (0.38)\\
Dot& \cellcolor{red!40}\href{../works/Lunardi20.pdf}{Lunardi20} (193.00)& \cellcolor{red!40}\href{../works/ZarandiASC20.pdf}{ZarandiASC20} (189.00)& \cellcolor{red!40}\href{../works/Groleaz21.pdf}{Groleaz21} (178.00)& \cellcolor{red!40}\href{../works/Astrand21.pdf}{Astrand21} (175.00)& \cellcolor{red!40}\href{../works/Dejemeppe16.pdf}{Dejemeppe16} (165.00)\\
Cosine& \cellcolor{red!40}\href{../works/ZhangYW21.pdf}{ZhangYW21} (0.74)& \cellcolor{red!40}\href{../works/Beck07.pdf}{Beck07} (0.74)& \cellcolor{red!40}\href{../works/Lunardi20.pdf}{Lunardi20} (0.73)& \cellcolor{red!40}\href{../works/LiFJZLL22.pdf}{LiFJZLL22} (0.72)& \cellcolor{red!40}\href{../works/CzerniachowskaWZ23.pdf}{CzerniachowskaWZ23} (0.72)\\
\index{AggounB93}\href{../works/AggounB93.pdf}{AggounB93} R\&C& \cellcolor{red!20}\href{../works/BeldiceanuC94.pdf}{BeldiceanuC94} (0.87)& \cellcolor{red!20}\href{../works/PoderBS04.pdf}{PoderBS04} (0.89)& \cellcolor{yellow!20}\href{../works/DincbasSH90.pdf}{DincbasSH90} (0.92)& \cellcolor{yellow!20}\href{../works/BeldiceanuCP08.pdf}{BeldiceanuCP08} (0.92)& \cellcolor{yellow!20}\href{../works/Vilim11.pdf}{Vilim11} (0.93)\\
Euclid& \cellcolor{green!20}\href{../works/Goltz95.pdf}{Goltz95} (0.31)& \cellcolor{blue!20}\href{../works/DincbasSH90.pdf}{DincbasSH90} (0.33)& \cellcolor{black!20}\href{../works/BeldiceanuCP08.pdf}{BeldiceanuCP08} (0.37)& \cellcolor{black!20}\href{../works/Zhou96.pdf}{Zhou96} (0.37)& \href{../works/Rit86.pdf}{Rit86} (0.38)\\
Dot& \cellcolor{red!40}\href{../works/Malapert11.pdf}{Malapert11} (133.00)& \cellcolor{red!40}\href{../works/Baptiste02.pdf}{Baptiste02} (132.00)& \cellcolor{red!40}\href{../works/Dejemeppe16.pdf}{Dejemeppe16} (122.00)& \cellcolor{red!40}\href{../works/Beck99.pdf}{Beck99} (120.00)& \cellcolor{red!40}\href{../works/LaborieRSV18.pdf}{LaborieRSV18} (118.00)\\
Cosine& \cellcolor{red!40}\href{../works/Goltz95.pdf}{Goltz95} (0.77)& \cellcolor{red!40}\href{../works/DincbasSH90.pdf}{DincbasSH90} (0.74)& \cellcolor{red!40}\href{../works/TrentesauxPT01.pdf}{TrentesauxPT01} (0.68)& \cellcolor{red!40}\href{../works/RoePS05.pdf}{RoePS05} (0.68)& \cellcolor{red!40}\href{../works/Zhou96.pdf}{Zhou96} (0.66)\\
\index{AggounMV08}AggounMV08 R\&C& \cellcolor{red!40}\href{../works/CobanH11.pdf}{CobanH11} (0.86)& \cellcolor{red!20}\href{../works/Hooker05b.pdf}{Hooker05b} (0.89)& \cellcolor{red!20}\href{../works/Thorsteinsson01.pdf}{Thorsteinsson01} (0.89)& \cellcolor{red!20}\href{../works/LiW08.pdf}{LiW08} (0.89)& \cellcolor{red!20}GongLMW09 (0.89)\\
Euclid\\
Dot\\
Cosine\\
\index{AggounV04}AggounV04 R\&C& \cellcolor{red!40}\href{../works/Simonis95.pdf}{Simonis95} (0.71)& \cellcolor{red!40}\href{../works/Geske05.pdf}{Geske05} (0.75)& \cellcolor{red!40}\href{../works/SimonisCK00.pdf}{SimonisCK00} (0.78)& \cellcolor{red!40}\href{../works/SimonisC95.pdf}{SimonisC95} (0.83)& \cellcolor{red!40}\href{../works/BeldiceanuCDP11.pdf}{BeldiceanuCDP11} (0.83)\\
Euclid\\
Dot\\
Cosine\\
\index{AgussurjaKL18}\href{../works/AgussurjaKL18.pdf}{AgussurjaKL18} R\&C& \cellcolor{red!40}\href{../works/HamdiL13.pdf}{HamdiL13} (0.83)& \cellcolor{red!40}\href{../works/CireCH13.pdf}{CireCH13} (0.86)& \cellcolor{yellow!20}\href{../works/TerekhovDOB12.pdf}{TerekhovDOB12} (0.92)& \cellcolor{yellow!20}GuoHLW20 (0.92)& \cellcolor{yellow!20}\href{../works/CobanH10.pdf}{CobanH10} (0.92)\\
Euclid& \cellcolor{red!40}\href{../works/BhatnagarKL19.pdf}{BhatnagarKL19} (0.20)& \cellcolor{yellow!20}\href{../works/LombardiM13.pdf}{LombardiM13} (0.27)& \cellcolor{yellow!20}\href{../works/BonfiettiM12.pdf}{BonfiettiM12} (0.28)& \cellcolor{green!20}\href{../works/Caballero23.pdf}{Caballero23} (0.29)& \cellcolor{green!20}\href{../works/BofillCSV17a.pdf}{BofillCSV17a} (0.29)\\
Dot& \cellcolor{red!40}\href{../works/Lombardi10.pdf}{Lombardi10} (96.00)& \cellcolor{red!40}\href{../works/LombardiM12.pdf}{LombardiM12} (94.00)& \cellcolor{red!40}\href{../works/ZarandiASC20.pdf}{ZarandiASC20} (92.00)& \cellcolor{red!40}\href{../works/LaborieRSV18.pdf}{LaborieRSV18} (82.00)& \cellcolor{red!40}\href{../works/LiW08.pdf}{LiW08} (80.00)\\
Cosine& \cellcolor{red!40}\href{../works/BhatnagarKL19.pdf}{BhatnagarKL19} (0.83)& \cellcolor{red!40}\href{../works/TranVNB17.pdf}{TranVNB17} (0.68)& \cellcolor{red!40}\href{../works/BofillCSV17a.pdf}{BofillCSV17a} (0.68)& \cellcolor{red!40}\href{../works/CampeauG22.pdf}{CampeauG22} (0.67)& \cellcolor{red!40}\href{../works/KusterJF07.pdf}{KusterJF07} (0.66)\\
\index{AjiliW04}AjiliW04 R\&C& \cellcolor{red!40}\href{../works/Wallace06.pdf}{Wallace06} (0.85)& \cellcolor{red!20}\href{../works/BockmayrP06.pdf}{BockmayrP06} (0.87)& \cellcolor{red!20}DannaP04 (0.88)& \cellcolor{red!20}\href{../works/Refalo00.pdf}{Refalo00} (0.89)& \cellcolor{red!20}MilanoORT02 (0.89)\\
Euclid\\
Dot\\
Cosine\\
\index{AkkerDH07}\href{../works/AkkerDH07.pdf}{AkkerDH07} R\&C& \cellcolor{red!20}\href{../works/LombardiM10.pdf}{LombardiM10} (0.89)& \cellcolor{red!20}\href{../works/LiW08.pdf}{LiW08} (0.90)& \cellcolor{red!20}\href{../works/Davenport10.pdf}{Davenport10} (0.90)& \cellcolor{red!20}OddiPCC05 (0.90)& \cellcolor{yellow!20}\href{../works/BaptisteB18.pdf}{BaptisteB18} (0.90)\\
Euclid& \cellcolor{green!20}\href{../works/Limtanyakul07.pdf}{Limtanyakul07} (0.31)& \cellcolor{green!20}\href{../works/Sadykov04.pdf}{Sadykov04} (0.31)& \cellcolor{blue!20}\href{../works/BenediktSMVH18.pdf}{BenediktSMVH18} (0.32)& \cellcolor{blue!20}\href{../works/SadykovW06.pdf}{SadykovW06} (0.33)& \cellcolor{blue!20}\href{../works/HeipckeCCS00.pdf}{HeipckeCCS00} (0.33)\\
Dot& \cellcolor{red!40}\href{../works/Groleaz21.pdf}{Groleaz21} (138.00)& \cellcolor{red!40}\href{../works/ZarandiASC20.pdf}{ZarandiASC20} (133.00)& \cellcolor{red!40}\href{../works/Baptiste02.pdf}{Baptiste02} (133.00)& \cellcolor{red!40}\href{../works/Dejemeppe16.pdf}{Dejemeppe16} (118.00)& \cellcolor{red!40}\href{../works/Godet21a.pdf}{Godet21a} (115.00)\\
Cosine& \cellcolor{red!40}\href{../works/Limtanyakul07.pdf}{Limtanyakul07} (0.71)& \cellcolor{red!40}\href{../works/Sadykov04.pdf}{Sadykov04} (0.71)& \cellcolor{red!40}\href{../works/HanenKP21.pdf}{HanenKP21} (0.69)& \cellcolor{red!40}\href{../works/HeipckeCCS00.pdf}{HeipckeCCS00} (0.69)& \cellcolor{red!40}\href{../works/SadykovW06.pdf}{SadykovW06} (0.69)\\
\index{AkramNHRSA23}\href{../works/AkramNHRSA23.pdf}{AkramNHRSA23} R\&C\\
Euclid& \cellcolor{red!40}\href{../works/GomesHS06.pdf}{GomesHS06} (0.23)& \cellcolor{red!40}\href{../works/SunLYL10.pdf}{SunLYL10} (0.24)& \cellcolor{red!40}\href{../works/QuSN06.pdf}{QuSN06} (0.24)& \cellcolor{red!40}\href{../works/BockmayrP06.pdf}{BockmayrP06} (0.24)& \cellcolor{red!20}\href{../works/JungblutK22.pdf}{JungblutK22} (0.25)\\
Dot& \cellcolor{red!40}\href{../works/Groleaz21.pdf}{Groleaz21} (83.00)& \cellcolor{red!40}\href{../works/ZarandiASC20.pdf}{ZarandiASC20} (80.00)& \cellcolor{red!40}\href{../works/Dejemeppe16.pdf}{Dejemeppe16} (75.00)& \cellcolor{red!40}\href{../works/Astrand21.pdf}{Astrand21} (73.00)& \cellcolor{red!40}\href{../works/Lombardi10.pdf}{Lombardi10} (73.00)\\
Cosine& \cellcolor{red!40}\href{../works/BockmayrP06.pdf}{BockmayrP06} (0.71)& \cellcolor{red!40}\href{../works/AlakaPY19.pdf}{AlakaPY19} (0.69)& \cellcolor{red!40}\href{../works/GomesHS06.pdf}{GomesHS06} (0.69)& \cellcolor{red!40}\href{../works/Alaka21.pdf}{Alaka21} (0.69)& \cellcolor{red!40}\href{../works/BukchinR18.pdf}{BukchinR18} (0.68)\\
\index{Alaka21}\href{../works/Alaka21.pdf}{Alaka21} R\&C& \cellcolor{red!40}\href{../works/AlakaPY19.pdf}{AlakaPY19} (0.41)& \cellcolor{red!40}\href{../works/AlakaP23.pdf}{AlakaP23} (0.69)& \cellcolor{red!40}PinarbasiA20 (0.74)& \cellcolor{red!40}Pinarbasi21 (0.75)& \cellcolor{red!40}\href{../works/PinarbasiAY19.pdf}{PinarbasiAY19} (0.83)\\
Euclid& \cellcolor{red!40}\href{../works/AlakaPY19.pdf}{AlakaPY19} (0.10)& \cellcolor{red!40}\href{../works/AlakaP23.pdf}{AlakaP23} (0.15)& \cellcolor{red!40}\href{../works/VanczaM01.pdf}{VanczaM01} (0.20)& \cellcolor{red!40}\href{../works/PinarbasiAY19.pdf}{PinarbasiAY19} (0.23)& \cellcolor{red!20}\href{../works/LozanoCDS12.pdf}{LozanoCDS12} (0.25)\\
Dot& \cellcolor{red!40}\href{../works/Astrand21.pdf}{Astrand21} (98.00)& \cellcolor{red!40}\href{../works/ZarandiASC20.pdf}{ZarandiASC20} (97.00)& \cellcolor{red!40}\href{../works/Lunardi20.pdf}{Lunardi20} (94.00)& \cellcolor{red!40}\href{../works/Groleaz21.pdf}{Groleaz21} (94.00)& \cellcolor{red!40}\href{../works/Dejemeppe16.pdf}{Dejemeppe16} (91.00)\\
Cosine& \cellcolor{red!40}\href{../works/AlakaPY19.pdf}{AlakaPY19} (0.96)& \cellcolor{red!40}\href{../works/AlakaP23.pdf}{AlakaP23} (0.91)& \cellcolor{red!40}\href{../works/PinarbasiAY19.pdf}{PinarbasiAY19} (0.85)& \cellcolor{red!40}\href{../works/VanczaM01.pdf}{VanczaM01} (0.84)& \cellcolor{red!40}\href{../works/CilKLO22.pdf}{CilKLO22} (0.76)\\
\index{AlakaP23}\href{../works/AlakaP23.pdf}{AlakaP23} R\&C& \cellcolor{red!40}\href{../works/Alaka21.pdf}{Alaka21} (0.69)& \cellcolor{red!40}\href{../works/AlakaPY19.pdf}{AlakaPY19} (0.76)& \cellcolor{red!40}Pinarbasi21 (0.85)& \cellcolor{red!20}PinarbasiA20 (0.87)& \cellcolor{yellow!20}\href{../works/CilKLO22.pdf}{CilKLO22} (0.91)\\
Euclid& \cellcolor{red!40}\href{../works/AlakaPY19.pdf}{AlakaPY19} (0.15)& \cellcolor{red!40}\href{../works/Alaka21.pdf}{Alaka21} (0.15)& \cellcolor{red!40}\href{../works/PinarbasiAY19.pdf}{PinarbasiAY19} (0.23)& \cellcolor{red!20}\href{../works/VanczaM01.pdf}{VanczaM01} (0.25)& \cellcolor{red!20}\href{../works/NishikawaSTT18a.pdf}{NishikawaSTT18a} (0.26)\\
Dot& \cellcolor{red!40}\href{../works/Astrand21.pdf}{Astrand21} (107.00)& \cellcolor{red!40}\href{../works/ZarandiASC20.pdf}{ZarandiASC20} (106.00)& \cellcolor{red!40}\href{../works/Lunardi20.pdf}{Lunardi20} (103.00)& \cellcolor{red!40}\href{../works/Groleaz21.pdf}{Groleaz21} (100.00)& \cellcolor{red!40}\href{../works/Edis21.pdf}{Edis21} (98.00)\\
Cosine& \cellcolor{red!40}\href{../works/AlakaPY19.pdf}{AlakaPY19} (0.92)& \cellcolor{red!40}\href{../works/Alaka21.pdf}{Alaka21} (0.91)& \cellcolor{red!40}\href{../works/PinarbasiAY19.pdf}{PinarbasiAY19} (0.85)& \cellcolor{red!40}\href{../works/VanczaM01.pdf}{VanczaM01} (0.77)& \cellcolor{red!40}\href{../works/CilKLO22.pdf}{CilKLO22} (0.77)\\
\index{AlakaPY19}\href{../works/AlakaPY19.pdf}{AlakaPY19} R\&C& \cellcolor{red!40}\href{../works/Alaka21.pdf}{Alaka21} (0.41)& \cellcolor{red!40}\href{../works/PinarbasiAY19.pdf}{PinarbasiAY19} (0.62)& \cellcolor{red!40}PinarbasiA20 (0.71)& \cellcolor{red!40}\href{../works/AlakaP23.pdf}{AlakaP23} (0.76)& \cellcolor{red!40}KizilayC20 (0.83)\\
Euclid& \cellcolor{red!40}\href{../works/Alaka21.pdf}{Alaka21} (0.10)& \cellcolor{red!40}\href{../works/AlakaP23.pdf}{AlakaP23} (0.15)& \cellcolor{red!40}\href{../works/VanczaM01.pdf}{VanczaM01} (0.22)& \cellcolor{red!40}\href{../works/LozanoCDS12.pdf}{LozanoCDS12} (0.23)& \cellcolor{red!40}\href{../works/QuSN06.pdf}{QuSN06} (0.23)\\
Dot& \cellcolor{red!40}\href{../works/Astrand21.pdf}{Astrand21} (86.00)& \cellcolor{red!40}\href{../works/Groleaz21.pdf}{Groleaz21} (85.00)& \cellcolor{red!40}\href{../works/Lunardi20.pdf}{Lunardi20} (83.00)& \cellcolor{red!40}\href{../works/ZarandiASC20.pdf}{ZarandiASC20} (82.00)& \cellcolor{red!40}\href{../works/CilKLO22.pdf}{CilKLO22} (81.00)\\
Cosine& \cellcolor{red!40}\href{../works/Alaka21.pdf}{Alaka21} (0.96)& \cellcolor{red!40}\href{../works/AlakaP23.pdf}{AlakaP23} (0.92)& \cellcolor{red!40}\href{../works/PinarbasiAY19.pdf}{PinarbasiAY19} (0.84)& \cellcolor{red!40}\href{../works/VanczaM01.pdf}{VanczaM01} (0.79)& \cellcolor{red!40}\href{../works/LozanoCDS12.pdf}{LozanoCDS12} (0.75)\\
\index{AlesioBNG15}\href{../works/AlesioBNG15.pdf}{AlesioBNG15} R\&C& \cellcolor{red!40}\href{../works/AlesioNBG14.pdf}{AlesioNBG14} (0.62)& \cellcolor{green!20}\href{../works/GrimesH15.pdf}{GrimesH15} (0.96)& \cellcolor{blue!20}\href{../works/CambazardHDJT04.pdf}{CambazardHDJT04} (0.96)& \cellcolor{blue!20}\href{../works/GilesH16.pdf}{GilesH16} (0.96)& \cellcolor{blue!20}\href{../works/DejemeppeD14.pdf}{DejemeppeD14} (0.97)\\
Euclid& \cellcolor{red!20}\href{../works/AlesioNBG14.pdf}{AlesioNBG14} (0.25)& \cellcolor{blue!20}\href{../works/CambazardHDJT04.pdf}{CambazardHDJT04} (0.32)& \cellcolor{blue!20}\href{../works/AkramNHRSA23.pdf}{AkramNHRSA23} (0.33)& \cellcolor{black!20}\href{../works/NishikawaSTT19.pdf}{NishikawaSTT19} (0.34)& \cellcolor{black!20}\href{../works/ZibranR11a.pdf}{ZibranR11a} (0.34)\\
Dot& \cellcolor{red!40}\href{../works/ZarandiASC20.pdf}{ZarandiASC20} (131.00)& \cellcolor{red!40}\href{../works/Groleaz21.pdf}{Groleaz21} (127.00)& \cellcolor{red!40}\href{../works/Lombardi10.pdf}{Lombardi10} (117.00)& \cellcolor{red!40}\href{../works/Dejemeppe16.pdf}{Dejemeppe16} (116.00)& \cellcolor{red!40}\href{../works/Astrand21.pdf}{Astrand21} (112.00)\\
Cosine& \cellcolor{red!40}\href{../works/AlesioNBG14.pdf}{AlesioNBG14} (0.83)& \cellcolor{red!40}\href{../works/CambazardHDJT04.pdf}{CambazardHDJT04} (0.71)& \cellcolor{red!40}\href{../works/HladikCDJ08.pdf}{HladikCDJ08} (0.67)& \cellcolor{red!40}\href{../works/NishikawaSTT19.pdf}{NishikawaSTT19} (0.66)& \cellcolor{red!40}\href{../works/AkramNHRSA23.pdf}{AkramNHRSA23} (0.66)\\
\index{AlesioNBG14}\href{../works/AlesioNBG14.pdf}{AlesioNBG14} R\&C& \cellcolor{red!40}\href{../works/AlesioBNG15.pdf}{AlesioBNG15} (0.62)& \cellcolor{yellow!20}\href{../works/CambazardHDJT04.pdf}{CambazardHDJT04} (0.91)& \cellcolor{yellow!20}\href{../works/GilesH16.pdf}{GilesH16} (0.92)& \cellcolor{yellow!20}\href{../works/MalapertCGJLR12.pdf}{MalapertCGJLR12} (0.92)& \cellcolor{yellow!20}\href{../works/HladikCDJ08.pdf}{HladikCDJ08} (0.93)\\
Euclid& \cellcolor{red!20}\href{../works/AlesioBNG15.pdf}{AlesioBNG15} (0.25)& \cellcolor{green!20}\href{../works/Caseau97.pdf}{Caseau97} (0.31)& \cellcolor{blue!20}\href{../works/CambazardHDJT04.pdf}{CambazardHDJT04} (0.32)& \cellcolor{blue!20}\href{../works/WolfS05.pdf}{WolfS05} (0.32)& \cellcolor{blue!20}\href{../works/SimoninAHL12.pdf}{SimoninAHL12} (0.32)\\
Dot& \cellcolor{red!40}\href{../works/Dejemeppe16.pdf}{Dejemeppe16} (122.00)& \cellcolor{red!40}\href{../works/Groleaz21.pdf}{Groleaz21} (122.00)& \cellcolor{red!40}\href{../works/Godet21a.pdf}{Godet21a} (113.00)& \cellcolor{red!40}\href{../works/ZarandiASC20.pdf}{ZarandiASC20} (112.00)& \cellcolor{red!40}\href{../works/Lombardi10.pdf}{Lombardi10} (109.00)\\
Cosine& \cellcolor{red!40}\href{../works/AlesioBNG15.pdf}{AlesioBNG15} (0.83)& \cellcolor{red!40}\href{../works/CambazardHDJT04.pdf}{CambazardHDJT04} (0.70)& \cellcolor{red!40}\href{../works/KhayatLR06.pdf}{KhayatLR06} (0.67)& \cellcolor{red!40}\href{../works/PengLC14.pdf}{PengLC14} (0.66)& \cellcolor{red!40}\href{../works/PembertonG98.pdf}{PembertonG98} (0.66)\\
\index{AlfieriGPS23}\href{../works/AlfieriGPS23.pdf}{AlfieriGPS23} R\&C& \cellcolor{green!20}\href{../works/ColT2019a.pdf}{ColT2019a} (0.96)& \cellcolor{green!20}\href{../works/ColT19.pdf}{ColT19} (0.96)& \cellcolor{blue!20}\href{../works/NaderiRR23.pdf}{NaderiRR23} (0.97)& \cellcolor{blue!20}\href{../works/ParkUJR19.pdf}{ParkUJR19} (0.98)& \cellcolor{blue!20}\href{../works/HauderBRPA20.pdf}{HauderBRPA20} (0.98)\\
Euclid& \cellcolor{black!20}\href{../works/LiFJZLL22.pdf}{LiFJZLL22} (0.37)& \cellcolor{black!20}\href{../works/BogaerdtW19.pdf}{BogaerdtW19} (0.37)& \cellcolor{black!20}\href{../works/Beck06.pdf}{Beck06} (0.37)& \href{../works/ZhangJZL22.pdf}{ZhangJZL22} (0.38)& \href{../works/ParkUJR19.pdf}{ParkUJR19} (0.38)\\
Dot& \cellcolor{red!40}\href{../works/ZarandiASC20.pdf}{ZarandiASC20} (189.00)& \cellcolor{red!40}\href{../works/Groleaz21.pdf}{Groleaz21} (170.00)& \cellcolor{red!40}\href{../works/PrataAN23.pdf}{PrataAN23} (167.00)& \cellcolor{red!40}\href{../works/IsikYA23.pdf}{IsikYA23} (158.00)& \cellcolor{red!40}\href{../works/NaderiRR23.pdf}{NaderiRR23} (158.00)\\
Cosine& \cellcolor{red!40}\href{../works/PrataAN23.pdf}{PrataAN23} (0.74)& \cellcolor{red!40}\href{../works/AbreuAPNM21.pdf}{AbreuAPNM21} (0.73)& \cellcolor{red!40}\href{../works/AbreuPNF23.pdf}{AbreuPNF23} (0.72)& \cellcolor{red!40}\href{../works/LiFJZLL22.pdf}{LiFJZLL22} (0.72)& \cellcolor{red!40}\href{../works/MengZRZL20.pdf}{MengZRZL20} (0.71)\\
\index{AlizdehS20}AlizdehS20 R\&C\\
Euclid\\
Dot\\
Cosine\\
\index{AmadiniGM16}\href{../works/AmadiniGM16.pdf}{AmadiniGM16} R\&C& \cellcolor{red!40}\href{../works/SzerediS16.pdf}{SzerediS16} (0.83)& \cellcolor{red!20}\href{../works/KreterSS17.pdf}{KreterSS17} (0.86)& \cellcolor{red!20}\href{../works/SchnellH15.pdf}{SchnellH15} (0.86)& \cellcolor{red!20}\href{../works/YoungFS17.pdf}{YoungFS17} (0.86)& \cellcolor{red!20}\href{../works/SchuttS16.pdf}{SchuttS16} (0.87)\\
Euclid& \cellcolor{red!40}\href{../works/LombardiM13.pdf}{LombardiM13} (0.24)& \cellcolor{red!20}\href{../works/Caballero23.pdf}{Caballero23} (0.26)& \cellcolor{red!20}\href{../works/BofillCSV17.pdf}{BofillCSV17} (0.26)& \cellcolor{yellow!20}\href{../works/OddiRC10.pdf}{OddiRC10} (0.27)& \cellcolor{yellow!20}\href{../works/LombardiM12a.pdf}{LombardiM12a} (0.27)\\
Dot& \cellcolor{red!40}\href{../works/Godet21a.pdf}{Godet21a} (98.00)& \cellcolor{red!40}\href{../works/BoudreaultSLQ22.pdf}{BoudreaultSLQ22} (91.00)& \cellcolor{red!40}\href{../works/Groleaz21.pdf}{Groleaz21} (87.00)& \cellcolor{red!40}\href{../works/PovedaAA23.pdf}{PovedaAA23} (84.00)& \cellcolor{red!40}\href{../works/Caballero19.pdf}{Caballero19} (84.00)\\
Cosine& \cellcolor{red!40}\href{../works/SzerediS16.pdf}{SzerediS16} (0.77)& \cellcolor{red!40}\href{../works/BofillCSV17.pdf}{BofillCSV17} (0.73)& \cellcolor{red!40}\href{../works/BoudreaultSLQ22.pdf}{BoudreaultSLQ22} (0.72)& \cellcolor{red!40}\href{../works/LombardiM13.pdf}{LombardiM13} (0.72)& \cellcolor{red!40}\href{../works/LombardiM12a.pdf}{LombardiM12a} (0.71)\\
\index{AngelsmarkJ00}\href{../works/AngelsmarkJ00.pdf}{AngelsmarkJ00} R\&C& \cellcolor{red!20}\href{../works/BrusoniCLMMT96.pdf}{BrusoniCLMMT96} (0.89)& \cellcolor{yellow!20}EsquirolLH2008 (0.92)& \cellcolor{yellow!20}\href{../works/Muscettola02.pdf}{Muscettola02} (0.92)& \cellcolor{yellow!20}\href{../works/Wolf11.pdf}{Wolf11} (0.93)& \cellcolor{green!20}\href{../works/PraletLJ15.pdf}{PraletLJ15} (0.94)\\
Euclid& \cellcolor{red!40}\href{../works/CarchraeBF05.pdf}{CarchraeBF05} (0.10)& \cellcolor{red!40}\href{../works/HebrardTW05.pdf}{HebrardTW05} (0.12)& \cellcolor{red!40}\href{../works/Baptiste09.pdf}{Baptiste09} (0.13)& \cellcolor{red!40}\href{../works/KovacsEKV05.pdf}{KovacsEKV05} (0.13)& \cellcolor{red!40}\href{../works/BarlattCG08.pdf}{BarlattCG08} (0.14)\\
Dot& \cellcolor{red!40}\href{../works/BartakSR10.pdf}{BartakSR10} (33.00)& \cellcolor{red!40}\href{../works/Beck99.pdf}{Beck99} (33.00)& \cellcolor{red!40}\href{../works/ZarandiASC20.pdf}{ZarandiASC20} (33.00)& \cellcolor{red!40}\href{../works/Siala15a.pdf}{Siala15a} (33.00)& \cellcolor{red!40}\href{../works/Baptiste02.pdf}{Baptiste02} (33.00)\\
Cosine& \cellcolor{red!40}\href{../works/CarchraeBF05.pdf}{CarchraeBF05} (0.80)& \cellcolor{red!40}\href{../works/Caseau97.pdf}{Caseau97} (0.78)& \cellcolor{red!40}\href{../works/BarlattCG08.pdf}{BarlattCG08} (0.77)& \cellcolor{red!40}\href{../works/Rit86.pdf}{Rit86} (0.76)& \cellcolor{red!40}\href{../works/BarbulescuWH04.pdf}{BarbulescuWH04} (0.76)\\
\index{AntunesABD18}\href{../works/AntunesABD18.pdf}{AntunesABD18} R\&C& \cellcolor{red!40}\href{../works/AntunesABD20.pdf}{AntunesABD20} (0.60)& \cellcolor{blue!20}\href{../works/FrohnerTR19.pdf}{FrohnerTR19} (0.97)& \cellcolor{blue!20}\href{../works/ArmstrongGOS22.pdf}{ArmstrongGOS22} (0.97)& \cellcolor{blue!20}\href{../works/HoYCLLCLC18.pdf}{HoYCLLCLC18} (0.97)& \cellcolor{blue!20}\href{../works/Ham18.pdf}{Ham18} (0.97)\\
Euclid& \cellcolor{red!40}\href{../works/AntunesABD20.pdf}{AntunesABD20} (0.15)& \cellcolor{red!20}\href{../works/ZibranR11.pdf}{ZibranR11} (0.26)& \cellcolor{yellow!20}\href{../works/ChapadosJR11.pdf}{ChapadosJR11} (0.27)& \cellcolor{yellow!20}\href{../works/ZibranR11a.pdf}{ZibranR11a} (0.27)& \cellcolor{yellow!20}\href{../works/TranVNB17a.pdf}{TranVNB17a} (0.27)\\
Dot& \cellcolor{red!40}\href{../works/Groleaz21.pdf}{Groleaz21} (85.00)& \cellcolor{red!40}\href{../works/HarjunkoskiMBC14.pdf}{HarjunkoskiMBC14} (83.00)& \cellcolor{red!40}\href{../works/Dejemeppe16.pdf}{Dejemeppe16} (82.00)& \cellcolor{red!40}\href{../works/ZarandiASC20.pdf}{ZarandiASC20} (82.00)& \cellcolor{red!40}\href{../works/Lombardi10.pdf}{Lombardi10} (82.00)\\
Cosine& \cellcolor{red!40}\href{../works/AntunesABD20.pdf}{AntunesABD20} (0.91)& \cellcolor{red!40}\href{../works/BoothNB16.pdf}{BoothNB16} (0.65)& \cellcolor{red!40}\href{../works/ZibranR11.pdf}{ZibranR11} (0.65)& \cellcolor{red!40}\href{../works/TranVNB17a.pdf}{TranVNB17a} (0.64)& \cellcolor{red!40}\href{../works/GrimesIOS14.pdf}{GrimesIOS14} (0.64)\\
\index{AntunesABD20}\href{../works/AntunesABD20.pdf}{AntunesABD20} R\&C& \cellcolor{red!40}\href{../works/AntunesABD18.pdf}{AntunesABD18} (0.60)& \cellcolor{blue!20}\href{../works/HoYCLLCLC18.pdf}{HoYCLLCLC18} (0.97)& \cellcolor{blue!20}\href{../works/Ham18.pdf}{Ham18} (0.97)& \cellcolor{blue!20}\href{../works/MusliuSS18.pdf}{MusliuSS18} (0.97)& \cellcolor{blue!20}HechingHK19 (0.98)\\
Euclid& \cellcolor{red!40}\href{../works/AntunesABD18.pdf}{AntunesABD18} (0.15)& \cellcolor{green!20}\href{../works/ZibranR11.pdf}{ZibranR11} (0.30)& \cellcolor{green!20}\href{../works/GomesHS06.pdf}{GomesHS06} (0.30)& \cellcolor{green!20}\href{../works/ChapadosJR11.pdf}{ChapadosJR11} (0.31)& \cellcolor{green!20}\href{../works/FortinZDF05.pdf}{FortinZDF05} (0.31)\\
Dot& \cellcolor{red!40}\href{../works/ZarandiASC20.pdf}{ZarandiASC20} (95.00)& \cellcolor{red!40}\href{../works/Groleaz21.pdf}{Groleaz21} (93.00)& \cellcolor{red!40}\href{../works/Malapert11.pdf}{Malapert11} (90.00)& \cellcolor{red!40}\href{../works/HarjunkoskiMBC14.pdf}{HarjunkoskiMBC14} (90.00)& \cellcolor{red!40}\href{../works/Dejemeppe16.pdf}{Dejemeppe16} (88.00)\\
Cosine& \cellcolor{red!40}\href{../works/AntunesABD18.pdf}{AntunesABD18} (0.91)& \cellcolor{red!40}\href{../works/NishikawaSTT18a.pdf}{NishikawaSTT18a} (0.60)& \cellcolor{red!40}\href{../works/NovasH10.pdf}{NovasH10} (0.59)& \cellcolor{red!40}\href{../works/BoothNB16.pdf}{BoothNB16} (0.59)& \cellcolor{red!40}\href{../works/BockmayrP06.pdf}{BockmayrP06} (0.58)\\
\index{AntuoriHHEN20}\href{../works/AntuoriHHEN20.pdf}{AntuoriHHEN20} R\&C& \cellcolor{red!20}\href{../works/DavenportKRSH07.pdf}{DavenportKRSH07} (0.90)& \cellcolor{yellow!20}\href{../works/WatsonB08.pdf}{WatsonB08} (0.92)& \cellcolor{yellow!20}\href{../works/SimoninAHL15.pdf}{SimoninAHL15} (0.92)& \cellcolor{yellow!20}\href{../works/CarchraeB09.pdf}{CarchraeB09} (0.93)& \cellcolor{green!20}\href{../works/BeldiceanuP07.pdf}{BeldiceanuP07} (0.93)\\
Euclid& \cellcolor{green!20}\href{../works/AntuoriHHEN21.pdf}{AntuoriHHEN21} (0.29)& \cellcolor{black!20}\href{../works/AlakaPY19.pdf}{AlakaPY19} (0.36)& \cellcolor{black!20}\href{../works/Alaka21.pdf}{Alaka21} (0.37)& \cellcolor{black!20}\href{../works/BockmayrP06.pdf}{BockmayrP06} (0.37)& \href{../works/BukchinR18.pdf}{BukchinR18} (0.37)\\
Dot& \cellcolor{red!40}\href{../works/Groleaz21.pdf}{Groleaz21} (139.00)& \cellcolor{red!40}\href{../works/Dejemeppe16.pdf}{Dejemeppe16} (134.00)& \cellcolor{red!40}\href{../works/ZarandiASC20.pdf}{ZarandiASC20} (130.00)& \cellcolor{red!40}\href{../works/Lombardi10.pdf}{Lombardi10} (127.00)& \cellcolor{red!40}\href{../works/Astrand21.pdf}{Astrand21} (119.00)\\
Cosine& \cellcolor{red!40}\href{../works/AntuoriHHEN21.pdf}{AntuoriHHEN21} (0.79)& \cellcolor{red!40}\href{../works/MonetteDH09.pdf}{MonetteDH09} (0.67)& \cellcolor{red!40}\href{../works/AlakaPY19.pdf}{AlakaPY19} (0.65)& \cellcolor{red!40}\href{../works/Alaka21.pdf}{Alaka21} (0.64)& \cellcolor{red!40}\href{../works/HeipckeCCS00.pdf}{HeipckeCCS00} (0.64)\\
\index{AntuoriHHEN21}\href{../works/AntuoriHHEN21.pdf}{AntuoriHHEN21} R\&C\\
Euclid& \cellcolor{green!20}\href{../works/AntuoriHHEN20.pdf}{AntuoriHHEN20} (0.29)& \cellcolor{blue!20}\href{../works/BockmayrP06.pdf}{BockmayrP06} (0.32)& \cellcolor{blue!20}\href{../works/BukchinR18.pdf}{BukchinR18} (0.33)& \cellcolor{blue!20}\href{../works/LozanoCDS12.pdf}{LozanoCDS12} (0.33)& \cellcolor{blue!20}\href{../works/Colombani96.pdf}{Colombani96} (0.33)\\
Dot& \cellcolor{red!40}\href{../works/Groleaz21.pdf}{Groleaz21} (121.00)& \cellcolor{red!40}\href{../works/ZarandiASC20.pdf}{ZarandiASC20} (112.00)& \cellcolor{red!40}\href{../works/AntuoriHHEN20.pdf}{AntuoriHHEN20} (106.00)& \cellcolor{red!40}\href{../works/Astrand21.pdf}{Astrand21} (103.00)& \cellcolor{red!40}\href{../works/Malapert11.pdf}{Malapert11} (103.00)\\
Cosine& \cellcolor{red!40}\href{../works/AntuoriHHEN20.pdf}{AntuoriHHEN20} (0.79)& \cellcolor{red!40}\href{../works/MonetteDH09.pdf}{MonetteDH09} (0.66)& \cellcolor{red!40}\href{../works/Colombani96.pdf}{Colombani96} (0.65)& \cellcolor{red!40}\href{../works/BockmayrP06.pdf}{BockmayrP06} (0.65)& \cellcolor{red!40}\href{../works/BukchinR18.pdf}{BukchinR18} (0.64)\\
\index{ArbaouiY18}\href{../works/ArbaouiY18.pdf}{ArbaouiY18} R\&C& \cellcolor{red!40}\href{../works/EdisO11.pdf}{EdisO11} (0.82)& \cellcolor{red!20}\href{../works/NattafALR16.pdf}{NattafALR16} (0.88)& \cellcolor{yellow!20}\href{../works/PraletLJ15.pdf}{PraletLJ15} (0.91)& \cellcolor{yellow!20}\href{../works/PandeyS21a.pdf}{PandeyS21a} (0.91)& \cellcolor{yellow!20}\href{../works/Ham18a.pdf}{Ham18a} (0.93)\\
Euclid& \cellcolor{green!20}\href{../works/EdisO11.pdf}{EdisO11} (0.29)& \cellcolor{green!20}\href{../works/Ham18a.pdf}{Ham18a} (0.30)& \cellcolor{green!20}\href{../works/abs-2305-19888.pdf}{abs-2305-19888} (0.30)& \cellcolor{green!20}\href{../works/BenediktSMVH18.pdf}{BenediktSMVH18} (0.31)& \cellcolor{green!20}\href{../works/NattafDYW19.pdf}{NattafDYW19} (0.31)\\
Dot& \cellcolor{red!40}\href{../works/Groleaz21.pdf}{Groleaz21} (126.00)& \cellcolor{red!40}\href{../works/Lunardi20.pdf}{Lunardi20} (120.00)& \cellcolor{red!40}\href{../works/NaderiRR23.pdf}{NaderiRR23} (120.00)& \cellcolor{red!40}\href{../works/IsikYA23.pdf}{IsikYA23} (116.00)& \cellcolor{red!40}\href{../works/YunusogluY22.pdf}{YunusogluY22} (115.00)\\
Cosine& \cellcolor{red!40}\href{../works/GedikKEK18.pdf}{GedikKEK18} (0.79)& \cellcolor{red!40}\href{../works/abs-2305-19888.pdf}{abs-2305-19888} (0.78)& \cellcolor{red!40}\href{../works/Ham18a.pdf}{Ham18a} (0.77)& \cellcolor{red!40}\href{../works/HeinzNVH22.pdf}{HeinzNVH22} (0.75)& \cellcolor{red!40}\href{../works/NattafDYW19.pdf}{NattafDYW19} (0.74)\\
\index{Arkhipov19}Arkhipov19 R\&C\\
Euclid\\
Dot\\
Cosine\\
\index{ArkhipovBL19}\href{../works/ArkhipovBL19.pdf}{ArkhipovBL19} R\&C& \cellcolor{red!40}NeronABCDD06 (0.67)& \cellcolor{red!40}\href{../works/DemasseyAM05.pdf}{DemasseyAM05} (0.80)& \cellcolor{red!40}\href{../works/CarlierPSJ20.pdf}{CarlierPSJ20} (0.80)& \cellcolor{red!40}CarlierSJP21 (0.83)& \cellcolor{red!40}\href{../works/LiessM08.pdf}{LiessM08} (0.84)\\
Euclid& \cellcolor{yellow!20}\href{../works/HeipckeCCS00.pdf}{HeipckeCCS00} (0.28)& \cellcolor{green!20}\href{../works/VilimLS15.pdf}{VilimLS15} (0.29)& \cellcolor{green!20}\href{../works/HillTV21.pdf}{HillTV21} (0.30)& \cellcolor{green!20}\href{../works/KovacsV06.pdf}{KovacsV06} (0.30)& \cellcolor{green!20}\href{../works/LiessM08.pdf}{LiessM08} (0.30)\\
Dot& \cellcolor{red!40}\href{../works/Schutt11.pdf}{Schutt11} (160.00)& \cellcolor{red!40}\href{../works/Godet21a.pdf}{Godet21a} (159.00)& \cellcolor{red!40}\href{../works/Groleaz21.pdf}{Groleaz21} (158.00)& \cellcolor{red!40}\href{../works/Baptiste02.pdf}{Baptiste02} (156.00)& \cellcolor{red!40}\href{../works/Dejemeppe16.pdf}{Dejemeppe16} (148.00)\\
Cosine& \cellcolor{red!40}\href{../works/VilimLS15.pdf}{VilimLS15} (0.81)& \cellcolor{red!40}\href{../works/HeipckeCCS00.pdf}{HeipckeCCS00} (0.80)& \cellcolor{red!40}\href{../works/HillTV21.pdf}{HillTV21} (0.79)& \cellcolor{red!40}\href{../works/BaptisteB18.pdf}{BaptisteB18} (0.78)& \cellcolor{red!40}\href{../works/DemasseyAM05.pdf}{DemasseyAM05} (0.78)\\
\index{ArmstrongGOS21}\href{../works/ArmstrongGOS21.pdf}{ArmstrongGOS21} R\&C& \cellcolor{red!40}\href{../works/TangB20.pdf}{TangB20} (0.86)& \cellcolor{green!20}\href{../works/RendlPHPR12.pdf}{RendlPHPR12} (0.93)& \cellcolor{black!20}\href{../works/LaborieRSV18.pdf}{LaborieRSV18} (0.99)\\
Euclid& \href{../works/JuvinHL23.pdf}{JuvinHL23} (0.47)& \href{../works/ArmstrongGOS22.pdf}{ArmstrongGOS22} (0.47)& \href{../works/CzerniachowskaWZ23.pdf}{CzerniachowskaWZ23} (0.48)& \href{../works/ZhouGL15.pdf}{ZhouGL15} (0.48)& \href{../works/LiFJZLL22.pdf}{LiFJZLL22} (0.48)\\
Dot& \cellcolor{red!40}\href{../works/Lunardi20.pdf}{Lunardi20} (193.00)& \cellcolor{red!40}\href{../works/Malapert11.pdf}{Malapert11} (186.00)& \cellcolor{red!40}\href{../works/Groleaz21.pdf}{Groleaz21} (180.00)& \cellcolor{red!40}\href{../works/ZarandiASC20.pdf}{ZarandiASC20} (176.00)& \cellcolor{red!40}\href{../works/Astrand21.pdf}{Astrand21} (169.00)\\
Cosine& \cellcolor{red!40}\href{../works/ArmstrongGOS22.pdf}{ArmstrongGOS22} (0.66)& \cellcolor{red!40}\href{../works/JuvinHL23.pdf}{JuvinHL23} (0.65)& \cellcolor{red!40}\href{../works/CzerniachowskaWZ23.pdf}{CzerniachowskaWZ23} (0.65)& \cellcolor{red!40}\href{../works/ZhouGL15.pdf}{ZhouGL15} (0.64)& \cellcolor{red!40}\href{../works/LiFJZLL22.pdf}{LiFJZLL22} (0.63)\\
\index{ArmstrongGOS22}\href{../works/ArmstrongGOS22.pdf}{ArmstrongGOS22} R\&C& \cellcolor{yellow!20}\href{../works/GroleazNS20.pdf}{GroleazNS20} (0.92)& \cellcolor{green!20}\href{../works/ParkUJR19.pdf}{ParkUJR19} (0.94)& \cellcolor{green!20}\href{../works/Laborie18a.pdf}{Laborie18a} (0.96)& \cellcolor{green!20}\href{../works/ColT2019a.pdf}{ColT2019a} (0.96)& \cellcolor{green!20}\href{../works/ColT19.pdf}{ColT19} (0.96)\\
Euclid& \cellcolor{black!20}\href{../works/ZhouGL15.pdf}{ZhouGL15} (0.35)& \cellcolor{black!20}\href{../works/LiLZDZW24.pdf}{LiLZDZW24} (0.36)& \cellcolor{black!20}\href{../works/JuvinHL23.pdf}{JuvinHL23} (0.36)& \cellcolor{black!20}\href{../works/ZhangJZL22.pdf}{ZhangJZL22} (0.36)& \cellcolor{black!20}\href{../works/PerezGSL23.pdf}{PerezGSL23} (0.37)\\
Dot& \cellcolor{red!40}\href{../works/ArmstrongGOS21.pdf}{ArmstrongGOS21} (128.00)& \cellcolor{red!40}\href{../works/IsikYA23.pdf}{IsikYA23} (120.00)& \cellcolor{red!40}\href{../works/Astrand21.pdf}{Astrand21} (118.00)& \cellcolor{red!40}\href{../works/Lunardi20.pdf}{Lunardi20} (118.00)& \cellcolor{red!40}\href{../works/Groleaz21.pdf}{Groleaz21} (115.00)\\
Cosine& \cellcolor{red!40}\href{../works/ZhouGL15.pdf}{ZhouGL15} (0.71)& \cellcolor{red!40}\href{../works/ZhangJZL22.pdf}{ZhangJZL22} (0.67)& \cellcolor{red!40}\href{../works/ArmstrongGOS21.pdf}{ArmstrongGOS21} (0.66)& \cellcolor{red!40}\href{../works/LiLZDZW24.pdf}{LiLZDZW24} (0.65)& \cellcolor{red!40}\href{../works/JuvinHL23.pdf}{JuvinHL23} (0.65)\\
\index{AronssonBK09}\href{../works/AronssonBK09.pdf}{AronssonBK09} R\&C\\
Euclid& \cellcolor{yellow!20}\href{../works/HebrardTW05.pdf}{HebrardTW05} (0.27)& \cellcolor{yellow!20}\href{../works/Acuna-AgostMFG09.pdf}{Acuna-AgostMFG09} (0.27)& \cellcolor{yellow!20}\href{../works/Hooker17.pdf}{Hooker17} (0.27)& \cellcolor{yellow!20}\href{../works/Caseau97.pdf}{Caseau97} (0.27)& \cellcolor{yellow!20}\href{../works/AngelsmarkJ00.pdf}{AngelsmarkJ00} (0.28)\\
Dot& \cellcolor{red!40}\href{../works/Malapert11.pdf}{Malapert11} (72.00)& \cellcolor{red!40}\href{../works/LaborieRSV18.pdf}{LaborieRSV18} (69.00)& \cellcolor{red!40}\href{../works/Lemos21.pdf}{Lemos21} (67.00)& \cellcolor{red!40}\href{../works/Lunardi20.pdf}{Lunardi20} (59.00)& \cellcolor{red!40}\href{../works/ColT22.pdf}{ColT22} (59.00)\\
Cosine& \cellcolor{red!40}\href{../works/Geske05.pdf}{Geske05} (0.61)& \cellcolor{red!40}\href{../works/Rodriguez07.pdf}{Rodriguez07} (0.61)& \cellcolor{red!40}\href{../works/PourDERB18.pdf}{PourDERB18} (0.60)& \cellcolor{red!40}\href{../works/WolfS05.pdf}{WolfS05} (0.59)& \cellcolor{red!40}\href{../works/Colombani96.pdf}{Colombani96} (0.58)\\
\index{ArtiguesBF04}\href{../works/ArtiguesBF04.pdf}{ArtiguesBF04} R\&C& \cellcolor{red!40}\href{../works/ArtiguesF07.pdf}{ArtiguesF07} (0.71)& \cellcolor{red!40}\href{../works/GrimesH10.pdf}{GrimesH10} (0.76)& \cellcolor{red!40}\href{../works/CarlssonKA99.pdf}{CarlssonKA99} (0.83)& \cellcolor{red!40}\href{../works/Wolf03.pdf}{Wolf03} (0.84)& \cellcolor{red!40}\href{../works/DejemeppeCS15.pdf}{DejemeppeCS15} (0.86)\\
Euclid& \cellcolor{red!40}\href{../works/ArtiguesF07.pdf}{ArtiguesF07} (0.18)& \cellcolor{red!20}\href{../works/TanSD10.pdf}{TanSD10} (0.26)& \cellcolor{yellow!20}\href{../works/FocacciLN00.pdf}{FocacciLN00} (0.27)& \cellcolor{green!20}\href{../works/CauwelaertDMS16.pdf}{CauwelaertDMS16} (0.29)& \cellcolor{green!20}\href{../works/MenciaSV13.pdf}{MenciaSV13} (0.29)\\
Dot& \cellcolor{red!40}\href{../works/Baptiste02.pdf}{Baptiste02} (142.00)& \cellcolor{red!40}\href{../works/Groleaz21.pdf}{Groleaz21} (137.00)& \cellcolor{red!40}\href{../works/Malapert11.pdf}{Malapert11} (133.00)& \cellcolor{red!40}\href{../works/GrimesH15.pdf}{GrimesH15} (131.00)& \cellcolor{red!40}\href{../works/Dejemeppe16.pdf}{Dejemeppe16} (128.00)\\
Cosine& \cellcolor{red!40}\href{../works/ArtiguesF07.pdf}{ArtiguesF07} (0.92)& \cellcolor{red!40}\href{../works/FocacciLN00.pdf}{FocacciLN00} (0.82)& \cellcolor{red!40}\href{../works/TanSD10.pdf}{TanSD10} (0.82)& \cellcolor{red!40}\href{../works/OrnekO16.pdf}{OrnekO16} (0.79)& \cellcolor{red!40}\href{../works/MenciaSV13.pdf}{MenciaSV13} (0.79)\\
\index{ArtiguesDN08}ArtiguesDN08 R\&C& \cellcolor{green!20}\href{../works/BertholdHLMS10.pdf}{BertholdHLMS10} (0.95)& \cellcolor{green!20}\href{../works/SchuttFSW11.pdf}{SchuttFSW11} (0.96)& \cellcolor{green!20}\href{../works/LiW08.pdf}{LiW08} (0.96)& \cellcolor{blue!20}\href{../works/AkkerDH07.pdf}{AkkerDH07} (0.97)& \cellcolor{blue!20}\href{../works/DemasseyAM05.pdf}{DemasseyAM05} (0.97)\\
Euclid\\
Dot\\
Cosine\\
\index{ArtiguesF07}\href{../works/ArtiguesF07.pdf}{ArtiguesF07} R\&C& \cellcolor{red!40}\href{../works/ArtiguesBF04.pdf}{ArtiguesBF04} (0.71)& \cellcolor{red!20}\href{../works/GrimesH10.pdf}{GrimesH10} (0.86)& \cellcolor{red!20}\href{../works/SourdN00.pdf}{SourdN00} (0.87)& \cellcolor{red!20}DorndorfPH99 (0.88)& \cellcolor{red!20}\href{../works/GrimesHM09.pdf}{GrimesHM09} (0.88)\\
Euclid& \cellcolor{red!40}\href{../works/ArtiguesBF04.pdf}{ArtiguesBF04} (0.18)& \cellcolor{yellow!20}\href{../works/TanSD10.pdf}{TanSD10} (0.26)& \cellcolor{green!20}\href{../works/FocacciLN00.pdf}{FocacciLN00} (0.29)& \cellcolor{green!20}\href{../works/MenciaSV13.pdf}{MenciaSV13} (0.29)& \cellcolor{green!20}\href{../works/CauwelaertDMS16.pdf}{CauwelaertDMS16} (0.30)\\
Dot& \cellcolor{red!40}\href{../works/Groleaz21.pdf}{Groleaz21} (157.00)& \cellcolor{red!40}\href{../works/Baptiste02.pdf}{Baptiste02} (153.00)& \cellcolor{red!40}\href{../works/Malapert11.pdf}{Malapert11} (146.00)& \cellcolor{red!40}\href{../works/Dejemeppe16.pdf}{Dejemeppe16} (145.00)& \cellcolor{red!40}\href{../works/ZarandiASC20.pdf}{ZarandiASC20} (145.00)\\
Cosine& \cellcolor{red!40}\href{../works/ArtiguesBF04.pdf}{ArtiguesBF04} (0.92)& \cellcolor{red!40}\href{../works/TanSD10.pdf}{TanSD10} (0.82)& \cellcolor{red!40}\href{../works/FocacciLN00.pdf}{FocacciLN00} (0.81)& \cellcolor{red!40}\href{../works/MenciaSV12.pdf}{MenciaSV12} (0.80)& \cellcolor{red!40}\href{../works/MenciaSV13.pdf}{MenciaSV13} (0.79)\\
\index{ArtiguesHQT21}\href{../works/ArtiguesHQT21.pdf}{ArtiguesHQT21} R\&C\\
Euclid& \cellcolor{green!20}\href{../works/CarlierPSJ20.pdf}{CarlierPSJ20} (0.29)& \cellcolor{green!20}\href{../works/Caballero23.pdf}{Caballero23} (0.29)& \cellcolor{green!20}\href{../works/BertholdHLMS10.pdf}{BertholdHLMS10} (0.29)& \cellcolor{green!20}\href{../works/CrawfordB94.pdf}{CrawfordB94} (0.30)& \cellcolor{green!20}\href{../works/Caseau97.pdf}{Caseau97} (0.31)\\
Dot& \cellcolor{red!40}\href{../works/ZarandiASC20.pdf}{ZarandiASC20} (91.00)& \cellcolor{red!40}\href{../works/Lombardi10.pdf}{Lombardi10} (88.00)& \cellcolor{red!40}\href{../works/Groleaz21.pdf}{Groleaz21} (86.00)& \cellcolor{red!40}\href{../works/Baptiste02.pdf}{Baptiste02} (86.00)& \cellcolor{red!40}\href{../works/BaptistePN99.pdf}{BaptistePN99} (86.00)\\
Cosine& \cellcolor{red!40}\href{../works/CarlierPSJ20.pdf}{CarlierPSJ20} (0.70)& \cellcolor{red!40}\href{../works/BaptistePN99.pdf}{BaptistePN99} (0.69)& \cellcolor{red!40}\href{../works/HanenKP21.pdf}{HanenKP21} (0.68)& \cellcolor{red!40}\href{../works/BertholdHLMS10.pdf}{BertholdHLMS10} (0.65)& \cellcolor{red!40}\href{../works/NattafHKAL19.pdf}{NattafHKAL19} (0.64)\\
\index{ArtiguesL14}\href{../works/ArtiguesL14.pdf}{ArtiguesL14} R\&C& \cellcolor{red!40}\href{../works/NattafAL15.pdf}{NattafAL15} (0.75)& \cellcolor{red!40}\href{../works/NattafAL17.pdf}{NattafAL17} (0.78)& \cellcolor{red!40}\href{../works/DerrienP14.pdf}{DerrienP14} (0.80)& \cellcolor{red!40}\href{../works/ArtiguesLH13.pdf}{ArtiguesLH13} (0.81)& \cellcolor{red!20}\href{../works/Tesch18.pdf}{Tesch18} (0.86)\\
Euclid& \cellcolor{red!40}\href{../works/NattafAL15.pdf}{NattafAL15} (0.20)& \cellcolor{red!40}\href{../works/NattafALR16.pdf}{NattafALR16} (0.22)& \cellcolor{red!40}\href{../works/NattafHKAL19.pdf}{NattafHKAL19} (0.23)& \cellcolor{yellow!20}\href{../works/NattafAL17.pdf}{NattafAL17} (0.26)& \cellcolor{yellow!20}\href{../works/CarlierPSJ20.pdf}{CarlierPSJ20} (0.27)\\
Dot& \cellcolor{red!40}\href{../works/Baptiste02.pdf}{Baptiste02} (112.00)& \cellcolor{red!40}\href{../works/Lombardi10.pdf}{Lombardi10} (111.00)& \cellcolor{red!40}\href{../works/Fahimi16.pdf}{Fahimi16} (108.00)& \cellcolor{red!40}\href{../works/NattafAL15.pdf}{NattafAL15} (104.00)& \cellcolor{red!40}\href{../works/NattafALR16.pdf}{NattafALR16} (102.00)\\
Cosine& \cellcolor{red!40}\href{../works/NattafAL15.pdf}{NattafAL15} (0.89)& \cellcolor{red!40}\href{../works/NattafALR16.pdf}{NattafALR16} (0.86)& \cellcolor{red!40}\href{../works/NattafHKAL19.pdf}{NattafHKAL19} (0.83)& \cellcolor{red!40}\href{../works/NattafAL17.pdf}{NattafAL17} (0.77)& \cellcolor{red!40}\href{../works/CarlierPSJ20.pdf}{CarlierPSJ20} (0.76)\\
\index{ArtiguesLH13}\href{../works/ArtiguesLH13.pdf}{ArtiguesLH13} R\&C& \cellcolor{red!40}\href{../works/ArtiguesL14.pdf}{ArtiguesL14} (0.81)& \cellcolor{red!40}\href{../works/NattafAL15.pdf}{NattafAL15} (0.85)& \cellcolor{red!20}\href{../works/NattafALR16.pdf}{NattafALR16} (0.90)& \cellcolor{green!20}\href{../works/NishikawaSTT18.pdf}{NishikawaSTT18} (0.93)& \cellcolor{green!20}\href{../works/NattafHKAL19.pdf}{NattafHKAL19} (0.94)\\
Euclid& \cellcolor{blue!20}\href{../works/HanenKP21.pdf}{HanenKP21} (0.33)& \cellcolor{blue!20}\href{../works/Hooker05a.pdf}{Hooker05a} (0.34)& \cellcolor{blue!20}\href{../works/JainG01.pdf}{JainG01} (0.34)& \cellcolor{black!20}\href{../works/Hooker06.pdf}{Hooker06} (0.34)& \cellcolor{black!20}\href{../works/Limtanyakul07.pdf}{Limtanyakul07} (0.35)\\
Dot& \cellcolor{red!40}\href{../works/Baptiste02.pdf}{Baptiste02} (175.00)& \cellcolor{red!40}\href{../works/Lombardi10.pdf}{Lombardi10} (163.00)& \cellcolor{red!40}\href{../works/Groleaz21.pdf}{Groleaz21} (160.00)& \cellcolor{red!40}\href{../works/ZarandiASC20.pdf}{ZarandiASC20} (156.00)& \cellcolor{red!40}\href{../works/Dejemeppe16.pdf}{Dejemeppe16} (154.00)\\
Cosine& \cellcolor{red!40}\href{../works/HanenKP21.pdf}{HanenKP21} (0.76)& \cellcolor{red!40}\href{../works/JainG01.pdf}{JainG01} (0.75)& \cellcolor{red!40}\href{../works/Hooker05a.pdf}{Hooker05a} (0.75)& \cellcolor{red!40}\href{../works/Limtanyakul07.pdf}{Limtanyakul07} (0.74)& \cellcolor{red!40}\href{../works/Hooker06.pdf}{Hooker06} (0.74)\\
\index{ArtiguesR00}\href{../works/ArtiguesR00.pdf}{ArtiguesR00} R\&C& \cellcolor{red!20}\href{../works/DilkinaDH05.pdf}{DilkinaDH05} (0.90)& \cellcolor{green!20}\href{../works/GeibingerMM19.pdf}{GeibingerMM19} (0.94)& \cellcolor{green!20}\href{../works/HeckmanB11.pdf}{HeckmanB11} (0.96)& \cellcolor{green!20}\href{../works/BruckerK00.pdf}{BruckerK00} (0.96)& \cellcolor{blue!20}EsquirolLH2008 (0.97)\\
Euclid& \cellcolor{blue!20}\href{../works/HentenryckM04.pdf}{HentenryckM04} (0.32)& \cellcolor{blue!20}\href{../works/OzturkTHO12.pdf}{OzturkTHO12} (0.33)& \cellcolor{blue!20}\href{../works/CestaOF99.pdf}{CestaOF99} (0.33)& \cellcolor{black!20}\href{../works/BhatnagarKL19.pdf}{BhatnagarKL19} (0.34)& \cellcolor{black!20}\href{../works/Laborie05.pdf}{Laborie05} (0.35)\\
Dot& \cellcolor{red!40}\href{../works/Groleaz21.pdf}{Groleaz21} (156.00)& \cellcolor{red!40}\href{../works/Baptiste02.pdf}{Baptiste02} (151.00)& \cellcolor{red!40}\href{../works/Dejemeppe16.pdf}{Dejemeppe16} (143.00)& \cellcolor{red!40}\href{../works/ZarandiASC20.pdf}{ZarandiASC20} (142.00)& \cellcolor{red!40}\href{../works/Malapert11.pdf}{Malapert11} (135.00)\\
Cosine& \cellcolor{red!40}\href{../works/HentenryckM04.pdf}{HentenryckM04} (0.74)& \cellcolor{red!40}\href{../works/OzturkTHO12.pdf}{OzturkTHO12} (0.71)& \cellcolor{red!40}\href{../works/CestaOF99.pdf}{CestaOF99} (0.71)& \cellcolor{red!40}\href{../works/Pralet17.pdf}{Pralet17} (0.71)& \cellcolor{red!40}\href{../works/Laborie05.pdf}{Laborie05} (0.71)\\
\index{ArtiouchineB05}\href{../works/ArtiouchineB05.pdf}{ArtiouchineB05} R\&C& \cellcolor{red!40}\href{../works/Vilim05.pdf}{Vilim05} (0.66)& \cellcolor{red!40}\href{../works/Wolf05.pdf}{Wolf05} (0.70)& \cellcolor{red!40}\href{../works/MonetteDD07.pdf}{MonetteDD07} (0.74)& \cellcolor{red!40}\href{../works/VilimBC04.pdf}{VilimBC04} (0.80)& \cellcolor{red!40}\href{../works/Wolf03.pdf}{Wolf03} (0.82)\\
Euclid& \cellcolor{green!20}\href{../works/BeckF99.pdf}{BeckF99} (0.29)& \cellcolor{green!20}\href{../works/VilimBC04.pdf}{VilimBC04} (0.30)& \cellcolor{green!20}\href{../works/VilimBC05.pdf}{VilimBC05} (0.31)& \cellcolor{green!20}\href{../works/MonetteDD07.pdf}{MonetteDD07} (0.31)& \cellcolor{blue!20}\href{../works/CauwelaertDMS16.pdf}{CauwelaertDMS16} (0.32)\\
Dot& \cellcolor{red!40}\href{../works/Baptiste02.pdf}{Baptiste02} (146.00)& \cellcolor{red!40}\href{../works/Fahimi16.pdf}{Fahimi16} (139.00)& \cellcolor{red!40}\href{../works/Dejemeppe16.pdf}{Dejemeppe16} (134.00)& \cellcolor{red!40}\href{../works/Malapert11.pdf}{Malapert11} (127.00)& \cellcolor{red!40}\href{../works/Lombardi10.pdf}{Lombardi10} (126.00)\\
Cosine& \cellcolor{red!40}\href{../works/VilimBC05.pdf}{VilimBC05} (0.74)& \cellcolor{red!40}\href{../works/BeckF99.pdf}{BeckF99} (0.74)& \cellcolor{red!40}\href{../works/MonetteDD07.pdf}{MonetteDD07} (0.74)& \cellcolor{red!40}\href{../works/GokgurHO18.pdf}{GokgurHO18} (0.74)& \cellcolor{red!40}\href{../works/VilimBC04.pdf}{VilimBC04} (0.74)\\
\index{Astrand0F21}\href{../works/Astrand0F21.pdf}{Astrand0F21} R\&C& \cellcolor{red!40}\href{../works/AstrandJZ18.pdf}{AstrandJZ18} (0.76)& \cellcolor{red!40}\href{../works/AstrandJZ20.pdf}{AstrandJZ20} (0.83)& \cellcolor{yellow!20}\href{../works/SialaAH15.pdf}{SialaAH15} (0.90)& \cellcolor{green!20}\href{../works/AntuoriHHEN20.pdf}{AntuoriHHEN20} (0.94)& \cellcolor{green!20}\href{../works/GaySS14.pdf}{GaySS14} (0.94)\\
Euclid& \cellcolor{yellow!20}\href{../works/AstrandJZ20.pdf}{AstrandJZ20} (0.28)& \cellcolor{green!20}\href{../works/PacinoH11.pdf}{PacinoH11} (0.29)& \cellcolor{green!20}\href{../works/AstrandJZ18.pdf}{AstrandJZ18} (0.31)& \cellcolor{green!20}\href{../works/CarchraeB09.pdf}{CarchraeB09} (0.31)& \cellcolor{green!20}\href{../works/VanczaM01.pdf}{VanczaM01} (0.31)\\
Dot& \cellcolor{red!40}\href{../works/Astrand21.pdf}{Astrand21} (157.00)& \cellcolor{red!40}\href{../works/Dejemeppe16.pdf}{Dejemeppe16} (137.00)& \cellcolor{red!40}\href{../works/LaborieRSV18.pdf}{LaborieRSV18} (133.00)& \cellcolor{red!40}\href{../works/Groleaz21.pdf}{Groleaz21} (133.00)& \cellcolor{red!40}\href{../works/AstrandJZ20.pdf}{AstrandJZ20} (131.00)\\
Cosine& \cellcolor{red!40}\href{../works/AstrandJZ20.pdf}{AstrandJZ20} (0.83)& \cellcolor{red!40}\href{../works/PacinoH11.pdf}{PacinoH11} (0.77)& \cellcolor{red!40}\href{../works/CarchraeB09.pdf}{CarchraeB09} (0.75)& \cellcolor{red!40}\href{../works/KhayatLR06.pdf}{KhayatLR06} (0.75)& \cellcolor{red!40}\href{../works/OzturkTHO15.pdf}{OzturkTHO15} (0.74)\\
\index{Astrand21}\href{../works/Astrand21.pdf}{Astrand21} R\&C\\
Euclid& \href{../works/AstrandJZ20.pdf}{AstrandJZ20} (0.41)& \href{../works/Astrand0F21.pdf}{Astrand0F21} (0.52)& \href{../works/JainM99.pdf}{JainM99} (0.52)& \href{../works/BartakSR08.pdf}{BartakSR08} (0.53)& \href{../works/AfsarVPG23.pdf}{AfsarVPG23} (0.54)\\
Dot& \cellcolor{red!40}\href{../works/ZarandiASC20.pdf}{ZarandiASC20} (306.00)& \cellcolor{red!40}\href{../works/Groleaz21.pdf}{Groleaz21} (293.00)& \cellcolor{red!40}\href{../works/Dejemeppe16.pdf}{Dejemeppe16} (262.00)& \cellcolor{red!40}\href{../works/Baptiste02.pdf}{Baptiste02} (262.00)& \cellcolor{red!40}\href{../works/Lunardi20.pdf}{Lunardi20} (258.00)\\
Cosine& \cellcolor{red!40}\href{../works/AstrandJZ20.pdf}{AstrandJZ20} (0.83)& \cellcolor{red!40}\href{../works/Lunardi20.pdf}{Lunardi20} (0.70)& \cellcolor{red!40}\href{../works/Astrand0F21.pdf}{Astrand0F21} (0.70)& \cellcolor{red!40}\href{../works/JainM99.pdf}{JainM99} (0.70)& \cellcolor{red!40}\href{../works/Groleaz21.pdf}{Groleaz21} (0.69)\\
\index{AstrandJZ18}\href{../works/AstrandJZ18.pdf}{AstrandJZ18} R\&C& \cellcolor{red!40}\href{../works/AstrandJZ20.pdf}{AstrandJZ20} (0.62)& \cellcolor{red!40}\href{../works/Astrand0F21.pdf}{Astrand0F21} (0.76)& \cellcolor{yellow!20}\href{../works/GrimesHM09.pdf}{GrimesHM09} (0.91)& \cellcolor{yellow!20}\href{../works/Davenport10.pdf}{Davenport10} (0.92)& \cellcolor{yellow!20}\href{../works/Limtanyakul07.pdf}{Limtanyakul07} (0.92)\\
Euclid& \cellcolor{red!40}\href{../works/BockmayrP06.pdf}{BockmayrP06} (0.23)& \cellcolor{red!20}\href{../works/TranVNB17a.pdf}{TranVNB17a} (0.24)& \cellcolor{red!20}\href{../works/ChapadosJR11.pdf}{ChapadosJR11} (0.26)& \cellcolor{red!20}\href{../works/RoweJCA96.pdf}{RoweJCA96} (0.26)& \cellcolor{red!20}\href{../works/BoothNB16.pdf}{BoothNB16} (0.26)\\
Dot& \cellcolor{red!40}\href{../works/Astrand21.pdf}{Astrand21} (101.00)& \cellcolor{red!40}\href{../works/Malapert11.pdf}{Malapert11} (97.00)& \cellcolor{red!40}\href{../works/Dejemeppe16.pdf}{Dejemeppe16} (96.00)& \cellcolor{red!40}\href{../works/ZarandiASC20.pdf}{ZarandiASC20} (94.00)& \cellcolor{red!40}\href{../works/Fahimi16.pdf}{Fahimi16} (94.00)\\
Cosine& \cellcolor{red!40}\href{../works/BockmayrP06.pdf}{BockmayrP06} (0.77)& \cellcolor{red!40}\href{../works/ChapadosJR11.pdf}{ChapadosJR11} (0.74)& \cellcolor{red!40}\href{../works/AalianPG23.pdf}{AalianPG23} (0.74)& \cellcolor{red!40}\href{../works/TranVNB17a.pdf}{TranVNB17a} (0.74)& \cellcolor{red!40}\href{../works/BoothNB16.pdf}{BoothNB16} (0.73)\\
\index{AstrandJZ20}\href{../works/AstrandJZ20.pdf}{AstrandJZ20} R\&C& \cellcolor{red!40}\href{../works/AstrandJZ18.pdf}{AstrandJZ18} (0.62)& \cellcolor{red!40}\href{../works/Astrand0F21.pdf}{Astrand0F21} (0.83)& \cellcolor{red!20}\href{../works/CampeauG22.pdf}{CampeauG22} (0.89)& \cellcolor{yellow!20}\href{../works/GroleazNS20.pdf}{GroleazNS20} (0.92)& \cellcolor{yellow!20}\href{../works/BlomBPS14.pdf}{BlomBPS14} (0.93)\\
Euclid& \cellcolor{yellow!20}\href{../works/Astrand0F21.pdf}{Astrand0F21} (0.28)& \cellcolor{blue!20}\href{../works/PacinoH11.pdf}{PacinoH11} (0.33)& \cellcolor{blue!20}\href{../works/BeckPS03.pdf}{BeckPS03} (0.34)& \cellcolor{black!20}\href{../works/HeckmanB11.pdf}{HeckmanB11} (0.34)& \cellcolor{black!20}\href{../works/Beck07.pdf}{Beck07} (0.34)\\
Dot& \cellcolor{red!40}\href{../works/Astrand21.pdf}{Astrand21} (210.00)& \cellcolor{red!40}\href{../works/Groleaz21.pdf}{Groleaz21} (186.00)& \cellcolor{red!40}\href{../works/ZarandiASC20.pdf}{ZarandiASC20} (178.00)& \cellcolor{red!40}\href{../works/Dejemeppe16.pdf}{Dejemeppe16} (176.00)& \cellcolor{red!40}\href{../works/Lunardi20.pdf}{Lunardi20} (166.00)\\
Cosine& \cellcolor{red!40}\href{../works/Astrand0F21.pdf}{Astrand0F21} (0.83)& \cellcolor{red!40}\href{../works/Astrand21.pdf}{Astrand21} (0.83)& \cellcolor{red!40}\href{../works/PacinoH11.pdf}{PacinoH11} (0.76)& \cellcolor{red!40}\href{../works/BeckPS03.pdf}{BeckPS03} (0.75)& \cellcolor{red!40}\href{../works/HeckmanB11.pdf}{HeckmanB11} (0.74)\\
\index{AwadMDMT22}\href{../works/AwadMDMT22.pdf}{AwadMDMT22} R\&C& \cellcolor{red!40}\href{../works/AbreuNP23.pdf}{AbreuNP23} (0.75)& \cellcolor{red!40}\href{../works/KlankeBYE21.pdf}{KlankeBYE21} (0.78)& \cellcolor{red!40}\href{../works/HeinzNVH22.pdf}{HeinzNVH22} (0.86)& \cellcolor{yellow!20}\href{../works/AbreuN22.pdf}{AbreuN22} (0.91)& \cellcolor{yellow!20}\href{../works/EscobetPQPRA19.pdf}{EscobetPQPRA19} (0.92)\\
Euclid& \href{../works/Novas19.pdf}{Novas19} (0.38)& \href{../works/NovaraNH16.pdf}{NovaraNH16} (0.40)& \href{../works/OrnekO16.pdf}{OrnekO16} (0.40)& \href{../works/OzturkTHO12.pdf}{OzturkTHO12} (0.40)& \href{../works/ZeballosNH11.pdf}{ZeballosNH11} (0.41)\\
Dot& \cellcolor{red!40}\href{../works/Groleaz21.pdf}{Groleaz21} (235.00)& \cellcolor{red!40}\href{../works/ZarandiASC20.pdf}{ZarandiASC20} (225.00)& \cellcolor{red!40}\href{../works/Dejemeppe16.pdf}{Dejemeppe16} (216.00)& \cellcolor{red!40}\href{../works/Malapert11.pdf}{Malapert11} (207.00)& \cellcolor{red!40}\href{../works/Baptiste02.pdf}{Baptiste02} (205.00)\\
Cosine& \cellcolor{red!40}\href{../works/Novas19.pdf}{Novas19} (0.78)& \cellcolor{red!40}\href{../works/NovaraNH16.pdf}{NovaraNH16} (0.75)& \cellcolor{red!40}\href{../works/OrnekO16.pdf}{OrnekO16} (0.74)& \cellcolor{red!40}\href{../works/ZeballosNH11.pdf}{ZeballosNH11} (0.73)& \cellcolor{red!40}\href{../works/OzturkTHO12.pdf}{OzturkTHO12} (0.73)\\
\index{BadicaBI20}\href{../works/BadicaBI20.pdf}{BadicaBI20} R\&C& \cellcolor{red!40}\href{../works/BadicaBIL19.pdf}{BadicaBIL19} (0.57)& \cellcolor{blue!20}\href{../works/KameugneF13.pdf}{KameugneF13} (0.97)& \cellcolor{blue!20}\href{../works/LombardiBM15.pdf}{LombardiBM15} (0.97)& \cellcolor{blue!20}\href{../works/BruckerK00.pdf}{BruckerK00} (0.97)& \cellcolor{blue!20}\href{../works/KameugneFSN11.pdf}{KameugneFSN11} (0.97)\\
Euclid& \cellcolor{green!20}\href{../works/BadicaBIL19.pdf}{BadicaBIL19} (0.30)& \cellcolor{black!20}\href{../works/LombardiBM15.pdf}{LombardiBM15} (0.35)& \cellcolor{black!20}\href{../works/FortinZDF05.pdf}{FortinZDF05} (0.35)& \cellcolor{black!20}\href{../works/BofillCSV17a.pdf}{BofillCSV17a} (0.36)& \cellcolor{black!20}\href{../works/LombardiM12a.pdf}{LombardiM12a} (0.37)\\
Dot& \cellcolor{red!40}\href{../works/Schutt11.pdf}{Schutt11} (130.00)& \cellcolor{red!40}\href{../works/Malapert11.pdf}{Malapert11} (125.00)& \cellcolor{red!40}\href{../works/ZarandiASC20.pdf}{ZarandiASC20} (119.00)& \cellcolor{red!40}\href{../works/Lombardi10.pdf}{Lombardi10} (119.00)& \cellcolor{red!40}\href{../works/Astrand21.pdf}{Astrand21} (115.00)\\
Cosine& \cellcolor{red!40}\href{../works/BadicaBIL19.pdf}{BadicaBIL19} (0.77)& \cellcolor{red!40}\href{../works/LombardiBM15.pdf}{LombardiBM15} (0.67)& \cellcolor{red!40}\href{../works/LiW08.pdf}{LiW08} (0.65)& \cellcolor{red!40}\href{../works/FortinZDF05.pdf}{FortinZDF05} (0.65)& \cellcolor{red!40}\href{../works/WikarekS19.pdf}{WikarekS19} (0.64)\\
\index{BadicaBIL19}\href{../works/BadicaBIL19.pdf}{BadicaBIL19} R\&C& \cellcolor{red!40}\href{../works/BadicaBI20.pdf}{BadicaBI20} (0.57)& \cellcolor{black!20}\href{../works/ZarandiASC20.pdf}{ZarandiASC20} (0.98)\\
Euclid& \cellcolor{yellow!20}\href{../works/FalaschiGMP97.pdf}{FalaschiGMP97} (0.26)& \cellcolor{yellow!20}\href{../works/WallaceF00.pdf}{WallaceF00} (0.26)& \cellcolor{yellow!20}\href{../works/ChapadosJR11.pdf}{ChapadosJR11} (0.27)& \cellcolor{yellow!20}\href{../works/FukunagaHFAMN02.pdf}{FukunagaHFAMN02} (0.27)& \cellcolor{yellow!20}\href{../works/ZibranR11.pdf}{ZibranR11} (0.27)\\
Dot& \cellcolor{red!40}\href{../works/BadicaBI20.pdf}{BadicaBI20} (77.00)& \cellcolor{red!40}\href{../works/Malapert11.pdf}{Malapert11} (68.00)& \cellcolor{red!40}\href{../works/ZarandiASC20.pdf}{ZarandiASC20} (59.00)& \cellcolor{red!40}\href{../works/Schutt11.pdf}{Schutt11} (59.00)& \cellcolor{red!40}\href{../works/TrentesauxPT01.pdf}{TrentesauxPT01} (58.00)\\
Cosine& \cellcolor{red!40}\href{../works/BadicaBI20.pdf}{BadicaBI20} (0.77)& \cellcolor{red!40}\href{../works/AstrandJZ18.pdf}{AstrandJZ18} (0.59)& \cellcolor{red!40}\href{../works/FalaschiGMP97.pdf}{FalaschiGMP97} (0.58)& \cellcolor{red!40}\href{../works/GruianK98.pdf}{GruianK98} (0.57)& \cellcolor{red!40}\href{../works/ZibranR11a.pdf}{ZibranR11a} (0.57)\\
\index{BajestaniB11}\href{../works/BajestaniB11.pdf}{BajestaniB11} R\&C& \cellcolor{blue!20}\href{../works/WuBB09.pdf}{WuBB09} (0.98)& \cellcolor{black!20}\href{../works/BidotVLB09.pdf}{BidotVLB09} (0.98)& \cellcolor{black!20}\href{../works/Hooker05.pdf}{Hooker05} (0.99)\\
Euclid& \cellcolor{yellow!20}\href{../works/Beck10.pdf}{Beck10} (0.27)& \cellcolor{yellow!20}\href{../works/CireCH13.pdf}{CireCH13} (0.28)& \cellcolor{yellow!20}\href{../works/HookerO03.pdf}{HookerO03} (0.28)& \cellcolor{green!20}\href{../works/ChuX05.pdf}{ChuX05} (0.29)& \cellcolor{green!20}\href{../works/HookerY02.pdf}{HookerY02} (0.29)\\
Dot& \cellcolor{red!40}\href{../works/BajestaniB13.pdf}{BajestaniB13} (123.00)& \cellcolor{red!40}\href{../works/ZarandiASC20.pdf}{ZarandiASC20} (101.00)& \cellcolor{red!40}\href{../works/Lombardi10.pdf}{Lombardi10} (101.00)& \cellcolor{red!40}\href{../works/Groleaz21.pdf}{Groleaz21} (99.00)& \cellcolor{red!40}\href{../works/Hooker19.pdf}{Hooker19} (97.00)\\
Cosine& \cellcolor{red!40}\href{../works/BajestaniB13.pdf}{BajestaniB13} (0.85)& \cellcolor{red!40}\href{../works/Beck10.pdf}{Beck10} (0.76)& \cellcolor{red!40}\href{../works/CobanH11.pdf}{CobanH11} (0.72)& \cellcolor{red!40}\href{../works/CireCH13.pdf}{CireCH13} (0.72)& \cellcolor{red!40}\href{../works/HamdiL13.pdf}{HamdiL13} (0.72)\\
\index{BajestaniB13}\href{../works/BajestaniB13.pdf}{BajestaniB13} R\&C& \cellcolor{red!20}\href{../works/HamdiL13.pdf}{HamdiL13} (0.88)& \cellcolor{red!20}\href{../works/Beck10.pdf}{Beck10} (0.88)& \cellcolor{red!20}\href{../works/CireCH13.pdf}{CireCH13} (0.88)& \cellcolor{red!20}\href{../works/ChuX05.pdf}{ChuX05} (0.89)& \cellcolor{yellow!20}\href{../works/Sadykov04.pdf}{Sadykov04} (0.92)\\
Euclid& \cellcolor{green!20}\href{../works/BajestaniB11.pdf}{BajestaniB11} (0.30)& \href{../works/CobanH11.pdf}{CobanH11} (0.39)& \href{../works/Beck10.pdf}{Beck10} (0.40)& \href{../works/ElciOH22.pdf}{ElciOH22} (0.40)& \href{../works/HeinzKB13.pdf}{HeinzKB13} (0.40)\\
Dot& \cellcolor{red!40}\href{../works/Groleaz21.pdf}{Groleaz21} (187.00)& \cellcolor{red!40}\href{../works/ZarandiASC20.pdf}{ZarandiASC20} (180.00)& \cellcolor{red!40}\href{../works/Lombardi10.pdf}{Lombardi10} (162.00)& \cellcolor{red!40}\href{../works/Baptiste02.pdf}{Baptiste02} (152.00)& \cellcolor{red!40}\href{../works/LombardiM12.pdf}{LombardiM12} (149.00)\\
Cosine& \cellcolor{red!40}\href{../works/BajestaniB11.pdf}{BajestaniB11} (0.85)& \cellcolor{red!40}\href{../works/CobanH11.pdf}{CobanH11} (0.70)& \cellcolor{red!40}\href{../works/ElciOH22.pdf}{ElciOH22} (0.69)& \cellcolor{red!40}\href{../works/Hooker19.pdf}{Hooker19} (0.68)& \cellcolor{red!40}\href{../works/BajestaniB15.pdf}{BajestaniB15} (0.68)\\
\index{BajestaniB15}\href{../works/BajestaniB15.pdf}{BajestaniB15} R\&C& \cellcolor{red!20}\href{../works/GuyonLPR12.pdf}{GuyonLPR12} (0.88)& \cellcolor{red!20}\href{../works/CireCH13.pdf}{CireCH13} (0.90)& \cellcolor{yellow!20}\href{../works/CobanH10.pdf}{CobanH10} (0.92)& \cellcolor{yellow!20}\href{../works/CireCH16.pdf}{CireCH16} (0.92)& \cellcolor{yellow!20}\href{../works/TranAB16.pdf}{TranAB16} (0.92)\\
Euclid& \cellcolor{blue!20}\href{../works/PenzDN23.pdf}{PenzDN23} (0.34)& \cellcolor{black!20}\href{../works/BajestaniB11.pdf}{BajestaniB11} (0.35)& \cellcolor{black!20}\href{../works/HamdiL13.pdf}{HamdiL13} (0.35)& \cellcolor{black!20}\href{../works/Beck10.pdf}{Beck10} (0.36)& \href{../works/ChuX05.pdf}{ChuX05} (0.39)\\
Dot& \cellcolor{red!40}\href{../works/ZarandiASC20.pdf}{ZarandiASC20} (169.00)& \cellcolor{red!40}\href{../works/Groleaz21.pdf}{Groleaz21} (154.00)& \cellcolor{red!40}\href{../works/Astrand21.pdf}{Astrand21} (145.00)& \cellcolor{red!40}\href{../works/Lombardi10.pdf}{Lombardi10} (144.00)& \cellcolor{red!40}\href{../works/Lunardi20.pdf}{Lunardi20} (136.00)\\
Cosine& \cellcolor{red!40}\href{../works/PenzDN23.pdf}{PenzDN23} (0.75)& \cellcolor{red!40}\href{../works/HamdiL13.pdf}{HamdiL13} (0.71)& \cellcolor{red!40}\href{../works/BajestaniB11.pdf}{BajestaniB11} (0.71)& \cellcolor{red!40}\href{../works/BajestaniB13.pdf}{BajestaniB13} (0.68)& \cellcolor{red!40}\href{../works/Beck10.pdf}{Beck10} (0.68)\\
\index{Balduccini11}\href{../works/Balduccini11.pdf}{Balduccini11} R\&C& \cellcolor{red!20}\href{../works/ColT19.pdf}{ColT19} (0.87)& \cellcolor{yellow!20}\href{../works/Simonis95.pdf}{Simonis95} (0.92)& \cellcolor{yellow!20}\href{../works/SchuttWS05.pdf}{SchuttWS05} (0.92)& \cellcolor{yellow!20}AggounV04 (0.92)& \cellcolor{yellow!20}\href{../works/SimonisCK00.pdf}{SimonisCK00} (0.93)\\
Euclid& \cellcolor{red!20}\href{../works/ChuGNSW13.pdf}{ChuGNSW13} (0.25)& \cellcolor{red!20}\href{../works/Hooker17.pdf}{Hooker17} (0.25)& \cellcolor{red!20}\href{../works/SmithC93.pdf}{SmithC93} (0.25)& \cellcolor{yellow!20}\href{../works/HookerY02.pdf}{HookerY02} (0.27)& \cellcolor{yellow!20}\href{../works/GelainPRVW17.pdf}{GelainPRVW17} (0.27)\\
Dot& \cellcolor{red!40}\href{../works/Baptiste02.pdf}{Baptiste02} (94.00)& \cellcolor{red!40}\href{../works/Lombardi10.pdf}{Lombardi10} (88.00)& \cellcolor{red!40}\href{../works/Malapert11.pdf}{Malapert11} (88.00)& \cellcolor{red!40}\href{../works/Groleaz21.pdf}{Groleaz21} (88.00)& \cellcolor{red!40}\href{../works/Dejemeppe16.pdf}{Dejemeppe16} (87.00)\\
Cosine& \cellcolor{red!40}\href{../works/SmithC93.pdf}{SmithC93} (0.74)& \cellcolor{red!40}\href{../works/ChuGNSW13.pdf}{ChuGNSW13} (0.73)& \cellcolor{red!40}\href{../works/Madi-WambaLOBM17.pdf}{Madi-WambaLOBM17} (0.70)& \cellcolor{red!40}\href{../works/KovacsV04.pdf}{KovacsV04} (0.69)& \cellcolor{red!40}\href{../works/Hooker17.pdf}{Hooker17} (0.69)\\
\index{BalochG20}BalochG20 R\&C& \cellcolor{red!20}ZarandiB12 (0.86)& \cellcolor{red!20}MartnezAJ22 (0.87)& \cellcolor{red!20}\href{../works/TranAB16.pdf}{TranAB16} (0.88)& \cellcolor{red!20}HechingHK19 (0.88)& \cellcolor{red!20}\href{../works/ElciOH22.pdf}{ElciOH22} (0.90)\\
Euclid\\
Dot\\
Cosine\\
\index{BandaSC11}\href{../works/BandaSC11.pdf}{BandaSC11} R\&C& \cellcolor{yellow!20}\href{../works/LiuLH19.pdf}{LiuLH19} (0.91)& \cellcolor{green!20}\href{../works/GarganiR07.pdf}{GarganiR07} (0.95)& \cellcolor{blue!20}\href{../works/DoulabiRP14.pdf}{DoulabiRP14} (0.96)& \cellcolor{blue!20}\href{../works/DoulabiRP16.pdf}{DoulabiRP16} (0.98)& \cellcolor{blue!20}\href{../works/GarridoOS08.pdf}{GarridoOS08} (0.98)\\
Euclid& \cellcolor{red!40}\href{../works/FeldmanG89.pdf}{FeldmanG89} (0.19)& \cellcolor{red!40}\href{../works/LiuLH19.pdf}{LiuLH19} (0.21)& \cellcolor{red!40}\href{../works/CarchraeBF05.pdf}{CarchraeBF05} (0.21)& \cellcolor{red!40}\href{../works/FrostD98.pdf}{FrostD98} (0.21)& \cellcolor{red!40}\href{../works/GelainPRVW17.pdf}{GelainPRVW17} (0.21)\\
Dot& \cellcolor{red!40}\href{../works/Siala15a.pdf}{Siala15a} (56.00)& \cellcolor{red!40}\href{../works/Dejemeppe16.pdf}{Dejemeppe16} (55.00)& \cellcolor{red!40}\href{../works/Godet21a.pdf}{Godet21a} (51.00)& \cellcolor{red!40}\href{../works/Beck99.pdf}{Beck99} (50.00)& \cellcolor{red!40}\href{../works/LaborieRSV18.pdf}{LaborieRSV18} (49.00)\\
Cosine& \cellcolor{red!40}\href{../works/LiuLH19.pdf}{LiuLH19} (0.77)& \cellcolor{red!40}\href{../works/GelainPRVW17.pdf}{GelainPRVW17} (0.68)& \cellcolor{red!40}\href{../works/HenzMT04.pdf}{HenzMT04} (0.68)& \cellcolor{red!40}\href{../works/FeldmanG89.pdf}{FeldmanG89} (0.68)& \cellcolor{red!40}\href{../works/ErkingerM17.pdf}{ErkingerM17} (0.66)\\
\index{Baptiste02}\href{../works/Baptiste02.pdf}{Baptiste02} R\&C\\
Euclid& \href{../works/BartakSR10.pdf}{BartakSR10} (0.53)& \href{../works/PapaB98.pdf}{PapaB98} (0.53)& \href{../works/GokgurHO18.pdf}{GokgurHO18} (0.53)& \href{../works/BartakSR08.pdf}{BartakSR08} (0.57)& \href{../works/Fahimi16.pdf}{Fahimi16} (0.57)\\
Dot& \cellcolor{red!40}\href{../works/ZarandiASC20.pdf}{ZarandiASC20} (331.00)& \cellcolor{red!40}\href{../works/Groleaz21.pdf}{Groleaz21} (318.00)& \cellcolor{red!40}\href{../works/Dejemeppe16.pdf}{Dejemeppe16} (299.00)& \cellcolor{red!40}\href{../works/Malapert11.pdf}{Malapert11} (297.00)& \cellcolor{red!40}\href{../works/Lombardi10.pdf}{Lombardi10} (284.00)\\
Cosine& \cellcolor{red!40}\href{../works/GokgurHO18.pdf}{GokgurHO18} (0.75)& \cellcolor{red!40}\href{../works/PapaB98.pdf}{PapaB98} (0.75)& \cellcolor{red!40}\href{../works/BartakSR10.pdf}{BartakSR10} (0.75)& \cellcolor{red!40}\href{../works/BartakSR08.pdf}{BartakSR08} (0.72)& \cellcolor{red!40}\href{../works/Fahimi16.pdf}{Fahimi16} (0.72)\\
\index{Baptiste09}\href{../works/Baptiste09.pdf}{Baptiste09} R\&C\\
Euclid& \cellcolor{red!40}\href{../works/CarchraeBF05.pdf}{CarchraeBF05} (0.09)& \cellcolor{red!40}\href{../works/Caballero23.pdf}{Caballero23} (0.12)& \cellcolor{red!40}\href{../works/AbrilSB05.pdf}{AbrilSB05} (0.12)& \cellcolor{red!40}\href{../works/AngelsmarkJ00.pdf}{AngelsmarkJ00} (0.13)& \cellcolor{red!40}\href{../works/FrostD98.pdf}{FrostD98} (0.13)\\
Dot& \cellcolor{red!40}\href{../works/JuvinHHL23.pdf}{JuvinHHL23} (12.00)& \cellcolor{red!40}\href{../works/PovedaAA23.pdf}{PovedaAA23} (12.00)& \cellcolor{red!40}\href{../works/KameugneFND23.pdf}{KameugneFND23} (12.00)& \cellcolor{red!40}\href{../works/EfthymiouY23.pdf}{EfthymiouY23} (12.00)& \cellcolor{red!40}\href{../works/SquillaciPR23.pdf}{SquillaciPR23} (12.00)\\
Cosine& \cellcolor{red!40}\href{../works/CarchraeBF05.pdf}{CarchraeBF05} (0.76)& \cellcolor{red!40}\href{../works/Caballero23.pdf}{Caballero23} (0.75)& \cellcolor{red!40}\href{../works/ZibranR11.pdf}{ZibranR11} (0.70)& \cellcolor{red!40}\href{../works/BofillCGGPSV23.pdf}{BofillCGGPSV23} (0.68)& \cellcolor{red!40}\href{../works/Hunsberger08.pdf}{Hunsberger08} (0.66)\\
\index{BaptisteB18}\href{../works/BaptisteB18.pdf}{BaptisteB18} R\&C& \cellcolor{red!40}\href{../works/OuelletQ13.pdf}{OuelletQ13} (0.72)& \cellcolor{red!40}\href{../works/GroleazNS20a.pdf}{GroleazNS20a} (0.81)& \cellcolor{red!40}\href{../works/KameugneF13.pdf}{KameugneF13} (0.82)& \cellcolor{red!40}\href{../works/KameugneFSN14.pdf}{KameugneFSN14} (0.83)& \cellcolor{red!20}\href{../works/OuelletQ18.pdf}{OuelletQ18} (0.86)\\
Euclid& \cellcolor{yellow!20}\href{../works/LiessM08.pdf}{LiessM08} (0.28)& \cellcolor{yellow!20}\href{../works/Caseau97.pdf}{Caseau97} (0.28)& \cellcolor{green!20}\href{../works/PoderBS04.pdf}{PoderBS04} (0.30)& \cellcolor{green!20}\href{../works/ArkhipovBL19.pdf}{ArkhipovBL19} (0.30)& \cellcolor{green!20}\href{../works/BaptisteP97.pdf}{BaptisteP97} (0.30)\\
Dot& \cellcolor{red!40}\href{../works/Schutt11.pdf}{Schutt11} (134.00)& \cellcolor{red!40}\href{../works/Godet21a.pdf}{Godet21a} (130.00)& \cellcolor{red!40}\href{../works/Lombardi10.pdf}{Lombardi10} (125.00)& \cellcolor{red!40}\href{../works/Baptiste02.pdf}{Baptiste02} (124.00)& \cellcolor{red!40}\href{../works/Dejemeppe16.pdf}{Dejemeppe16} (120.00)\\
Cosine& \cellcolor{red!40}\href{../works/ArkhipovBL19.pdf}{ArkhipovBL19} (0.78)& \cellcolor{red!40}\href{../works/BaptisteP97.pdf}{BaptisteP97} (0.78)& \cellcolor{red!40}\href{../works/LiessM08.pdf}{LiessM08} (0.77)& \cellcolor{red!40}\href{../works/BaptisteP00.pdf}{BaptisteP00} (0.75)& \cellcolor{red!40}\href{../works/Caseau97.pdf}{Caseau97} (0.74)\\
\index{BaptisteLPN06}BaptisteLPN06 R\&C& \cellcolor{red!40}\href{../works/Laborie03.pdf}{Laborie03} (0.78)& \cellcolor{red!40}\href{../works/GrimesH15.pdf}{GrimesH15} (0.83)& \cellcolor{red!40}DorndorfHP99 (0.85)& \cellcolor{red!20}\href{../works/TanSD10.pdf}{TanSD10} (0.86)& \cellcolor{red!20}DannaP04 (0.86)\\
Euclid\\
Dot\\
Cosine\\
\index{BaptisteLV92}\href{../works/BaptisteLV92.pdf}{BaptisteLV92} R\&C& \cellcolor{red!40}\href{../works/RodosekW98.pdf}{RodosekW98} (0.78)& \cellcolor{red!20}\href{../works/Simonis95a.pdf}{Simonis95a} (0.89)& \cellcolor{yellow!20}\href{../works/Simonis99.pdf}{Simonis99} (0.92)& \cellcolor{yellow!20}\href{../works/Zhou96.pdf}{Zhou96} (0.92)& \cellcolor{yellow!20}\href{../works/Goltz95.pdf}{Goltz95} (0.92)\\
Euclid\\
Dot\\
Cosine\\
\index{BaptisteP00}\href{../works/BaptisteP00.pdf}{BaptisteP00} R\&C& \cellcolor{red!20}\href{../works/SchuttW10.pdf}{SchuttW10} (0.90)& \cellcolor{red!20}\href{../works/Vilim09.pdf}{Vilim09} (0.90)& \cellcolor{yellow!20}\href{../works/BruckerK00.pdf}{BruckerK00} (0.90)& \cellcolor{yellow!20}\href{../works/DemasseyAM05.pdf}{DemasseyAM05} (0.92)& \cellcolor{yellow!20}\href{../works/KameugneF13.pdf}{KameugneF13} (0.92)\\
Euclid& \cellcolor{red!40}\href{../works/BaptisteP97.pdf}{BaptisteP97} (0.13)& \cellcolor{yellow!20}\href{../works/PapeB97.pdf}{PapeB97} (0.27)& \cellcolor{green!20}\href{../works/BaptistePN99.pdf}{BaptistePN99} (0.29)& \cellcolor{green!20}\href{../works/DemasseyAM05.pdf}{DemasseyAM05} (0.30)& \cellcolor{blue!20}\href{../works/LiessM08.pdf}{LiessM08} (0.32)\\
Dot& \cellcolor{red!40}\href{../works/Baptiste02.pdf}{Baptiste02} (176.00)& \cellcolor{red!40}\href{../works/Lombardi10.pdf}{Lombardi10} (165.00)& \cellcolor{red!40}\href{../works/PapeB97.pdf}{PapeB97} (159.00)& \cellcolor{red!40}\href{../works/Godet21a.pdf}{Godet21a} (158.00)& \cellcolor{red!40}\href{../works/Malapert11.pdf}{Malapert11} (158.00)\\
Cosine& \cellcolor{red!40}\href{../works/BaptisteP97.pdf}{BaptisteP97} (0.96)& \cellcolor{red!40}\href{../works/PapeB97.pdf}{PapeB97} (0.87)& \cellcolor{red!40}\href{../works/BaptistePN99.pdf}{BaptistePN99} (0.82)& \cellcolor{red!40}\href{../works/DemasseyAM05.pdf}{DemasseyAM05} (0.80)& \cellcolor{red!40}\href{../works/PapaB98.pdf}{PapaB98} (0.78)\\
\index{BaptisteP95}\href{../works/BaptisteP95.pdf}{BaptisteP95} R\&C\\
Euclid& \cellcolor{green!20}\href{../works/BeckF00a.pdf}{BeckF00a} (0.31)& \cellcolor{green!20}\href{../works/BeckF99.pdf}{BeckF99} (0.31)& \cellcolor{blue!20}\href{../works/CauwelaertDMS16.pdf}{CauwelaertDMS16} (0.32)& \cellcolor{blue!20}\href{../works/OzturkTHO12.pdf}{OzturkTHO12} (0.32)& \cellcolor{blue!20}\href{../works/BartakSR08.pdf}{BartakSR08} (0.33)\\
Dot& \cellcolor{red!40}\href{../works/Baptiste02.pdf}{Baptiste02} (156.00)& \cellcolor{red!40}\href{../works/Fahimi16.pdf}{Fahimi16} (150.00)& \cellcolor{red!40}\href{../works/Lombardi10.pdf}{Lombardi10} (144.00)& \cellcolor{red!40}\href{../works/PapaB98.pdf}{PapaB98} (138.00)& \cellcolor{red!40}\href{../works/Schutt11.pdf}{Schutt11} (138.00)\\
Cosine& \cellcolor{red!40}\href{../works/PapaB98.pdf}{PapaB98} (0.79)& \cellcolor{red!40}\href{../works/BeckF00a.pdf}{BeckF00a} (0.76)& \cellcolor{red!40}\href{../works/GokgurHO18.pdf}{GokgurHO18} (0.75)& \cellcolor{red!40}\href{../works/BartakSR08.pdf}{BartakSR08} (0.75)& \cellcolor{red!40}\href{../works/ChenGPSH10.pdf}{ChenGPSH10} (0.75)\\
\index{BaptisteP97}\href{../works/BaptisteP97.pdf}{BaptisteP97} R\&C& \cellcolor{red!40}DorndorfHP99 (0.83)& \cellcolor{red!40}\href{../works/KameugneFGOQ18.pdf}{KameugneFGOQ18} (0.84)& \cellcolor{red!40}\href{../works/DemasseyAM05.pdf}{DemasseyAM05} (0.86)& \cellcolor{red!20}\href{../works/Colombani96.pdf}{Colombani96} (0.87)& \cellcolor{red!20}\href{../works/NuijtenA96.pdf}{NuijtenA96} (0.88)\\
Euclid& \cellcolor{red!40}\href{../works/BaptisteP00.pdf}{BaptisteP00} (0.13)& \cellcolor{red!40}\href{../works/DemasseyAM05.pdf}{DemasseyAM05} (0.24)& \cellcolor{red!20}\href{../works/BaptistePN99.pdf}{BaptistePN99} (0.26)& \cellcolor{yellow!20}\href{../works/LiessM08.pdf}{LiessM08} (0.27)& \cellcolor{green!20}\href{../works/BofillCSV17a.pdf}{BofillCSV17a} (0.29)\\
Dot& \cellcolor{red!40}\href{../works/Baptiste02.pdf}{Baptiste02} (172.00)& \cellcolor{red!40}\href{../works/Lombardi10.pdf}{Lombardi10} (162.00)& \cellcolor{red!40}\href{../works/Godet21a.pdf}{Godet21a} (160.00)& \cellcolor{red!40}\href{../works/Schutt11.pdf}{Schutt11} (157.00)& \cellcolor{red!40}\href{../works/Dejemeppe16.pdf}{Dejemeppe16} (155.00)\\
Cosine& \cellcolor{red!40}\href{../works/BaptisteP00.pdf}{BaptisteP00} (0.96)& \cellcolor{red!40}\href{../works/DemasseyAM05.pdf}{DemasseyAM05} (0.86)& \cellcolor{red!40}\href{../works/BaptistePN99.pdf}{BaptistePN99} (0.86)& \cellcolor{red!40}\href{../works/PapeB97.pdf}{PapeB97} (0.83)& \cellcolor{red!40}\href{../works/LiessM08.pdf}{LiessM08} (0.82)\\
\index{BaptistePN01}BaptistePN01 R\&C& \cellcolor{green!20}\href{../works/Laborie03.pdf}{Laborie03} (0.94)& \cellcolor{green!20}\href{../works/AggounB93.pdf}{AggounB93} (0.94)& \cellcolor{green!20}\href{../works/Vilim04.pdf}{Vilim04} (0.95)& \cellcolor{green!20}\href{../works/JainG01.pdf}{JainG01} (0.95)& \cellcolor{green!20}\href{../works/MercierH08.pdf}{MercierH08} (0.95)\\
Euclid\\
Dot\\
Cosine\\
\index{BaptistePN99}\href{../works/BaptistePN99.pdf}{BaptistePN99} R\&C& \cellcolor{yellow!20}\href{../works/BruckerK00.pdf}{BruckerK00} (0.91)& \cellcolor{green!20}\href{../works/Vilim11.pdf}{Vilim11} (0.95)& \cellcolor{green!20}\href{../works/ArtiguesL14.pdf}{ArtiguesL14} (0.95)& \cellcolor{green!20}\href{../works/MercierH08.pdf}{MercierH08} (0.95)& \cellcolor{green!20}\href{../works/SchuttW10.pdf}{SchuttW10} (0.95)\\
Euclid& \cellcolor{red!20}\href{../works/BaptisteP97.pdf}{BaptisteP97} (0.26)& \cellcolor{green!20}\href{../works/DemasseyAM05.pdf}{DemasseyAM05} (0.29)& \cellcolor{green!20}\href{../works/BaptisteP00.pdf}{BaptisteP00} (0.29)& \cellcolor{green!20}\href{../works/CarlierPSJ20.pdf}{CarlierPSJ20} (0.30)& \cellcolor{green!20}\href{../works/LiessM08.pdf}{LiessM08} (0.31)\\
Dot& \cellcolor{red!40}\href{../works/Baptiste02.pdf}{Baptiste02} (184.00)& \cellcolor{red!40}\href{../works/Lombardi10.pdf}{Lombardi10} (162.00)& \cellcolor{red!40}\href{../works/Groleaz21.pdf}{Groleaz21} (161.00)& \cellcolor{red!40}\href{../works/Godet21a.pdf}{Godet21a} (160.00)& \cellcolor{red!40}\href{../works/Dejemeppe16.pdf}{Dejemeppe16} (156.00)\\
Cosine& \cellcolor{red!40}\href{../works/BaptisteP97.pdf}{BaptisteP97} (0.86)& \cellcolor{red!40}\href{../works/BaptisteP00.pdf}{BaptisteP00} (0.82)& \cellcolor{red!40}\href{../works/DemasseyAM05.pdf}{DemasseyAM05} (0.82)& \cellcolor{red!40}\href{../works/CarlierPSJ20.pdf}{CarlierPSJ20} (0.79)& \cellcolor{red!40}\href{../works/HanenKP21.pdf}{HanenKP21} (0.77)\\
\index{BarbulescuWH04}\href{../works/BarbulescuWH04.pdf}{BarbulescuWH04} R\&C\\
Euclid& \cellcolor{red!40}\href{../works/AngelsmarkJ00.pdf}{AngelsmarkJ00} (0.23)& \cellcolor{red!20}\href{../works/GlobusCLP04.pdf}{GlobusCLP04} (0.25)& \cellcolor{red!20}\href{../works/CrawfordB94.pdf}{CrawfordB94} (0.26)& \cellcolor{red!20}\href{../works/LudwigKRBMS14.pdf}{LudwigKRBMS14} (0.26)& \cellcolor{red!20}\href{../works/Caseau97.pdf}{Caseau97} (0.26)\\
Dot& \cellcolor{red!40}\href{../works/ZarandiASC20.pdf}{ZarandiASC20} (84.00)& \cellcolor{red!40}\href{../works/Astrand21.pdf}{Astrand21} (79.00)& \cellcolor{red!40}\href{../works/Beck99.pdf}{Beck99} (78.00)& \cellcolor{red!40}\href{../works/Godet21a.pdf}{Godet21a} (76.00)& \cellcolor{red!40}\href{../works/Lombardi10.pdf}{Lombardi10} (75.00)\\
Cosine& \cellcolor{red!40}\href{../works/AngelsmarkJ00.pdf}{AngelsmarkJ00} (0.76)& \cellcolor{red!40}\href{../works/GlobusCLP04.pdf}{GlobusCLP04} (0.76)& \cellcolor{red!40}\href{../works/JussienL02.pdf}{JussienL02} (0.72)& \cellcolor{red!40}\href{../works/Salido10.pdf}{Salido10} (0.70)& \cellcolor{red!40}\href{../works/MurphyRFSS97.pdf}{MurphyRFSS97} (0.67)\\
\index{BarlattCG08}\href{../works/BarlattCG08.pdf}{BarlattCG08} R\&C& \cellcolor{green!20}\href{../works/ElkhyariGJ02.pdf}{ElkhyariGJ02} (0.93)& \cellcolor{green!20}\href{../works/NuijtenA96.pdf}{NuijtenA96} (0.93)& \cellcolor{green!20}\href{../works/HebrardHJMPV16.pdf}{HebrardHJMPV16} (0.94)& \cellcolor{green!20}\href{../works/BertholdHLMS10.pdf}{BertholdHLMS10} (0.95)& \cellcolor{green!20}\href{../works/WikarekS19.pdf}{WikarekS19} (0.95)\\
Euclid& \cellcolor{red!40}\href{../works/AngelsmarkJ00.pdf}{AngelsmarkJ00} (0.14)& \cellcolor{red!40}\href{../works/CarchraeBF05.pdf}{CarchraeBF05} (0.17)& \cellcolor{red!40}\href{../works/KovacsEKV05.pdf}{KovacsEKV05} (0.18)& \cellcolor{red!40}\href{../works/Baptiste09.pdf}{Baptiste09} (0.19)& \cellcolor{red!40}\href{../works/HebrardTW05.pdf}{HebrardTW05} (0.19)\\
Dot& \cellcolor{red!40}\href{../works/Lunardi20.pdf}{Lunardi20} (45.00)& \cellcolor{red!40}\href{../works/CzerniachowskaWZ23.pdf}{CzerniachowskaWZ23} (44.00)& \cellcolor{red!40}\href{../works/Groleaz21.pdf}{Groleaz21} (44.00)& \cellcolor{red!40}\href{../works/Astrand21.pdf}{Astrand21} (44.00)& \cellcolor{red!40}\href{../works/ColT22.pdf}{ColT22} (43.00)\\
Cosine& \cellcolor{red!40}\href{../works/AngelsmarkJ00.pdf}{AngelsmarkJ00} (0.77)& \cellcolor{red!40}\href{../works/LudwigKRBMS14.pdf}{LudwigKRBMS14} (0.66)& \cellcolor{red!40}\href{../works/BeniniBGM05a.pdf}{BeniniBGM05a} (0.65)& \cellcolor{red!40}\href{../works/Caseau97.pdf}{Caseau97} (0.63)& \cellcolor{red!40}\href{../works/SunLYL10.pdf}{SunLYL10} (0.62)\\
\index{Bartak02}\href{../works/Bartak02.pdf}{Bartak02} R\&C& \cellcolor{red!40}\href{../works/Bartak02a.pdf}{Bartak02a} (0.69)& \cellcolor{green!20}\href{../works/MonetteDD07.pdf}{MonetteDD07} (0.94)& \cellcolor{green!20}\href{../works/Zhou97.pdf}{Zhou97} (0.95)& \cellcolor{blue!20}\href{../works/VilimBC05.pdf}{VilimBC05} (0.96)& \cellcolor{blue!20}\href{../works/YuraszeckMPV22.pdf}{YuraszeckMPV22} (0.97)\\
Euclid& \cellcolor{red!40}\href{../works/Bartak02a.pdf}{Bartak02a} (0.23)& \cellcolor{red!20}\href{../works/BrusoniCLMMT96.pdf}{BrusoniCLMMT96} (0.26)& \cellcolor{yellow!20}\href{../works/KovacsV04.pdf}{KovacsV04} (0.27)& \cellcolor{yellow!20}\href{../works/GarridoOS08.pdf}{GarridoOS08} (0.28)& \cellcolor{yellow!20}\href{../works/ChuGNSW13.pdf}{ChuGNSW13} (0.28)\\
Dot& \cellcolor{red!40}\href{../works/Baptiste02.pdf}{Baptiste02} (113.00)& \cellcolor{red!40}\href{../works/Malapert11.pdf}{Malapert11} (110.00)& \cellcolor{red!40}\href{../works/Fahimi16.pdf}{Fahimi16} (106.00)& \cellcolor{red!40}\href{../works/Siala15a.pdf}{Siala15a} (101.00)& \cellcolor{red!40}\href{../works/ZarandiASC20.pdf}{ZarandiASC20} (100.00)\\
Cosine& \cellcolor{red!40}\href{../works/Bartak02a.pdf}{Bartak02a} (0.81)& \cellcolor{red!40}\href{../works/BeckF00a.pdf}{BeckF00a} (0.75)& \cellcolor{red!40}\href{../works/BrusoniCLMMT96.pdf}{BrusoniCLMMT96} (0.75)& \cellcolor{red!40}\href{../works/KovacsV04.pdf}{KovacsV04} (0.74)& \cellcolor{red!40}\href{../works/BartakSR08.pdf}{BartakSR08} (0.72)\\
\index{Bartak02a}\href{../works/Bartak02a.pdf}{Bartak02a} R\&C& \cellcolor{red!40}\href{../works/Bartak02.pdf}{Bartak02} (0.69)& \cellcolor{yellow!20}\href{../works/BartakS11.pdf}{BartakS11} (0.92)& \cellcolor{yellow!20}\href{../works/VilimBC04.pdf}{VilimBC04} (0.92)& \cellcolor{yellow!20}\href{../works/VilimBC05.pdf}{VilimBC05} (0.93)& \cellcolor{green!20}\href{../works/MonetteDD07.pdf}{MonetteDD07} (0.95)\\
Euclid& \cellcolor{red!40}\href{../works/KovacsV04.pdf}{KovacsV04} (0.20)& \cellcolor{red!40}\href{../works/Bartak02.pdf}{Bartak02} (0.23)& \cellcolor{red!40}\href{../works/BartakCS10.pdf}{BartakCS10} (0.24)& \cellcolor{red!20}\href{../works/ChuGNSW13.pdf}{ChuGNSW13} (0.24)& \cellcolor{red!20}\href{../works/ValleMGT03.pdf}{ValleMGT03} (0.25)\\
Dot& \cellcolor{red!40}\href{../works/Baptiste02.pdf}{Baptiste02} (126.00)& \cellcolor{red!40}\href{../works/Lombardi10.pdf}{Lombardi10} (112.00)& \cellcolor{red!40}\href{../works/Fahimi16.pdf}{Fahimi16} (112.00)& \cellcolor{red!40}\href{../works/Beck99.pdf}{Beck99} (111.00)& \cellcolor{red!40}\href{../works/Malapert11.pdf}{Malapert11} (111.00)\\
Cosine& \cellcolor{red!40}\href{../works/KovacsV04.pdf}{KovacsV04} (0.86)& \cellcolor{red!40}\href{../works/Bartak02.pdf}{Bartak02} (0.81)& \cellcolor{red!40}\href{../works/BartakSR08.pdf}{BartakSR08} (0.80)& \cellcolor{red!40}\href{../works/BartakCS10.pdf}{BartakCS10} (0.79)& \cellcolor{red!40}\href{../works/ChuGNSW13.pdf}{ChuGNSW13} (0.77)\\
\index{Bartak14}Bartak14 R\&C\\
Euclid\\
Dot\\
Cosine\\
\index{BartakCS10}\href{../works/BartakCS10.pdf}{BartakCS10} R\&C& \cellcolor{red!40}\href{../works/LombardiM10a.pdf}{LombardiM10a} (0.81)& \cellcolor{red!40}\href{../works/CestaOS98.pdf}{CestaOS98} (0.86)& \cellcolor{red!20}\href{../works/LombardiM09.pdf}{LombardiM09} (0.89)& \cellcolor{red!20}\href{../works/BeldiceanuCDP11.pdf}{BeldiceanuCDP11} (0.90)& \cellcolor{yellow!20}\href{../works/CobanH10.pdf}{CobanH10} (0.91)\\
Euclid& \cellcolor{red!40}\href{../works/Bartak02a.pdf}{Bartak02a} (0.24)& \cellcolor{red!20}\href{../works/OddiRC10.pdf}{OddiRC10} (0.25)& \cellcolor{red!20}\href{../works/LombardiM13.pdf}{LombardiM13} (0.26)& \cellcolor{red!20}\href{../works/Junker00.pdf}{Junker00} (0.26)& \cellcolor{yellow!20}\href{../works/WallaceF00.pdf}{WallaceF00} (0.26)\\
Dot& \cellcolor{red!40}\href{../works/Baptiste02.pdf}{Baptiste02} (101.00)& \cellcolor{red!40}\href{../works/Astrand21.pdf}{Astrand21} (95.00)& \cellcolor{red!40}\href{../works/ZarandiASC20.pdf}{ZarandiASC20} (93.00)& \cellcolor{red!40}\href{../works/Godet21a.pdf}{Godet21a} (91.00)& \cellcolor{red!40}\href{../works/Lombardi10.pdf}{Lombardi10} (91.00)\\
Cosine& \cellcolor{red!40}\href{../works/Bartak02a.pdf}{Bartak02a} (0.79)& \cellcolor{red!40}\href{../works/CestaOF99.pdf}{CestaOF99} (0.74)& \cellcolor{red!40}\href{../works/OddiRC10.pdf}{OddiRC10} (0.73)& \cellcolor{red!40}\href{../works/Junker00.pdf}{Junker00} (0.73)& \cellcolor{red!40}\href{../works/KovacsV04.pdf}{KovacsV04} (0.72)\\
\index{BartakS11}\href{../works/BartakS11.pdf}{BartakS11} R\&C& \cellcolor{red!40}\href{../works/Salido10.pdf}{Salido10} (0.75)& \cellcolor{red!40}\href{../works/VilimBC04.pdf}{VilimBC04} (0.86)& \cellcolor{red!20}\href{../works/VilimBC05.pdf}{VilimBC05} (0.88)& \cellcolor{red!20}\href{../works/Colombani96.pdf}{Colombani96} (0.88)& \cellcolor{red!20}\href{../works/Rodriguez07.pdf}{Rodriguez07} (0.89)\\
Euclid& \cellcolor{red!20}\href{../works/Salido10.pdf}{Salido10} (0.24)& \cellcolor{red!20}\href{../works/GelainPRVW17.pdf}{GelainPRVW17} (0.25)& \cellcolor{red!20}\href{../works/SmithBHW96.pdf}{SmithBHW96} (0.25)& \cellcolor{red!20}\href{../works/FukunagaHFAMN02.pdf}{FukunagaHFAMN02} (0.25)& \cellcolor{red!20}\href{../works/DilkinaH04.pdf}{DilkinaH04} (0.26)\\
Dot& \cellcolor{red!40}\href{../works/Beck99.pdf}{Beck99} (89.00)& \cellcolor{red!40}\href{../works/Godet21a.pdf}{Godet21a} (86.00)& \cellcolor{red!40}\href{../works/Lombardi10.pdf}{Lombardi10} (86.00)& \cellcolor{red!40}\href{../works/Lemos21.pdf}{Lemos21} (86.00)& \cellcolor{red!40}\href{../works/Dejemeppe16.pdf}{Dejemeppe16} (83.00)\\
Cosine& \cellcolor{red!40}\href{../works/Salido10.pdf}{Salido10} (0.76)& \cellcolor{red!40}\href{../works/GelainPRVW17.pdf}{GelainPRVW17} (0.70)& \cellcolor{red!40}\href{../works/LiuLH19.pdf}{LiuLH19} (0.70)& \cellcolor{red!40}\href{../works/SmithBHW96.pdf}{SmithBHW96} (0.69)& \cellcolor{red!40}\href{../works/KovacsV04.pdf}{KovacsV04} (0.69)\\
\index{BartakSR08}\href{../works/BartakSR08.pdf}{BartakSR08} R\&C& \cellcolor{red!40}\href{../works/BartakSR10.pdf}{BartakSR10} (0.74)& \cellcolor{red!20}\href{../works/GarridoAO09.pdf}{GarridoAO09} (0.89)& \cellcolor{red!20}\href{../works/Laborie03.pdf}{Laborie03} (0.89)& \cellcolor{red!20}BaptisteLPN06 (0.89)& \cellcolor{yellow!20}BriandHHL08 (0.91)\\
Euclid& \cellcolor{yellow!20}\href{../works/MalapertCGJLR13.pdf}{MalapertCGJLR13} (0.27)& \cellcolor{yellow!20}\href{../works/KovacsV04.pdf}{KovacsV04} (0.28)& \cellcolor{yellow!20}\href{../works/Bartak02a.pdf}{Bartak02a} (0.28)& \cellcolor{green!20}\href{../works/TorresL00.pdf}{TorresL00} (0.29)& \cellcolor{green!20}\href{../works/BeckF00a.pdf}{BeckF00a} (0.30)\\
Dot& \cellcolor{red!40}\href{../works/Baptiste02.pdf}{Baptiste02} (186.00)& \cellcolor{red!40}\href{../works/BartakSR10.pdf}{BartakSR10} (174.00)& \cellcolor{red!40}\href{../works/Dejemeppe16.pdf}{Dejemeppe16} (168.00)& \cellcolor{red!40}\href{../works/Groleaz21.pdf}{Groleaz21} (168.00)& \cellcolor{red!40}\href{../works/Malapert11.pdf}{Malapert11} (166.00)\\
Cosine& \cellcolor{red!40}\href{../works/BartakSR10.pdf}{BartakSR10} (0.88)& \cellcolor{red!40}\href{../works/GokgurHO18.pdf}{GokgurHO18} (0.83)& \cellcolor{red!40}\href{../works/MalapertCGJLR13.pdf}{MalapertCGJLR13} (0.81)& \cellcolor{red!40}\href{../works/TorresL00.pdf}{TorresL00} (0.81)& \cellcolor{red!40}\href{../works/ChenGPSH10.pdf}{ChenGPSH10} (0.80)\\
\index{BartakSR10}\href{../works/BartakSR10.pdf}{BartakSR10} R\&C& \cellcolor{red!40}\href{../works/BartakSR08.pdf}{BartakSR08} (0.74)& \cellcolor{red!20}\href{../works/SadehF96.pdf}{SadehF96} (0.90)& \cellcolor{yellow!20}BaptisteLPN06 (0.92)& \cellcolor{yellow!20}EsquirolLH2008 (0.92)& \cellcolor{yellow!20}\href{../works/BeckF00.pdf}{BeckF00} (0.92)\\
Euclid& \cellcolor{green!20}\href{../works/BartakSR08.pdf}{BartakSR08} (0.30)& \cellcolor{black!20}\href{../works/GokgurHO18.pdf}{GokgurHO18} (0.37)& \href{../works/BeckF00.pdf}{BeckF00} (0.40)& \href{../works/FahimiOQ18.pdf}{FahimiOQ18} (0.41)& \href{../works/BeckDDF98.pdf}{BeckDDF98} (0.41)\\
Dot& \cellcolor{red!40}\href{../works/Baptiste02.pdf}{Baptiste02} (254.00)& \cellcolor{red!40}\href{../works/Groleaz21.pdf}{Groleaz21} (234.00)& \cellcolor{red!40}\href{../works/ZarandiASC20.pdf}{ZarandiASC20} (233.00)& \cellcolor{red!40}\href{../works/Dejemeppe16.pdf}{Dejemeppe16} (232.00)& \cellcolor{red!40}\href{../works/Malapert11.pdf}{Malapert11} (223.00)\\
Cosine& \cellcolor{red!40}\href{../works/BartakSR08.pdf}{BartakSR08} (0.88)& \cellcolor{red!40}\href{../works/GokgurHO18.pdf}{GokgurHO18} (0.80)& \cellcolor{red!40}\href{../works/BeckF00.pdf}{BeckF00} (0.76)& \cellcolor{red!40}\href{../works/FahimiOQ18.pdf}{FahimiOQ18} (0.76)& \cellcolor{red!40}\href{../works/BeckDDF98.pdf}{BeckDDF98} (0.76)\\
\index{BartakV15}\href{../works/BartakV15.pdf}{BartakV15} R\&C\\
Euclid& \cellcolor{red!20}\href{../works/BidotVLB07.pdf}{BidotVLB07} (0.26)& \cellcolor{yellow!20}\href{../works/LuoVLBM16.pdf}{LuoVLBM16} (0.27)& \cellcolor{yellow!20}\href{../works/WallaceF00.pdf}{WallaceF00} (0.27)& \cellcolor{yellow!20}\href{../works/FoxAS82.pdf}{FoxAS82} (0.28)& \cellcolor{green!20}\href{../works/CrawfordB94.pdf}{CrawfordB94} (0.29)\\
Dot& \cellcolor{red!40}\href{../works/ZarandiASC20.pdf}{ZarandiASC20} (98.00)& \cellcolor{red!40}\href{../works/Groleaz21.pdf}{Groleaz21} (89.00)& \cellcolor{red!40}\href{../works/Astrand21.pdf}{Astrand21} (89.00)& \cellcolor{red!40}\href{../works/Baptiste02.pdf}{Baptiste02} (89.00)& \cellcolor{red!40}\href{../works/BidotVLB09.pdf}{BidotVLB09} (87.00)\\
Cosine& \cellcolor{red!40}\href{../works/BidotVLB07.pdf}{BidotVLB07} (0.72)& \cellcolor{red!40}\href{../works/BidotVLB09.pdf}{BidotVLB09} (0.70)& \cellcolor{red!40}\href{../works/BeckPS03.pdf}{BeckPS03} (0.69)& \cellcolor{red!40}\href{../works/NovasH10.pdf}{NovasH10} (0.68)& \cellcolor{red!40}\href{../works/BartakCS10.pdf}{BartakCS10} (0.67)\\
\index{BartoliniBBLM14}\href{../works/BartoliniBBLM14.pdf}{BartoliniBBLM14} R\&C& \cellcolor{red!40}\href{../works/GilesH16.pdf}{GilesH16} (0.78)& \cellcolor{red!20}\href{../works/BridiBLMB16.pdf}{BridiBLMB16} (0.90)& \cellcolor{yellow!20}\href{../works/BorghesiBLMB18.pdf}{BorghesiBLMB18} (0.92)& \cellcolor{yellow!20}BaptisteLPN06 (0.93)& \cellcolor{green!20}\href{../works/CohenHB17.pdf}{CohenHB17} (0.93)\\
Euclid& \cellcolor{red!40}\href{../works/GalleguillosKSB19.pdf}{GalleguillosKSB19} (0.23)& \cellcolor{red!20}\href{../works/Limtanyakul07.pdf}{Limtanyakul07} (0.26)& \cellcolor{yellow!20}\href{../works/Hooker17.pdf}{Hooker17} (0.27)& \cellcolor{yellow!20}\href{../works/BonfiettiM12.pdf}{BonfiettiM12} (0.27)& \cellcolor{yellow!20}\href{../works/BridiLBBM16.pdf}{BridiLBBM16} (0.28)\\
Dot& \cellcolor{red!40}\href{../works/LaborieRSV18.pdf}{LaborieRSV18} (85.00)& \cellcolor{red!40}\href{../works/Dejemeppe16.pdf}{Dejemeppe16} (85.00)& \cellcolor{red!40}\href{../works/Lombardi10.pdf}{Lombardi10} (83.00)& \cellcolor{red!40}\href{../works/Hooker19.pdf}{Hooker19} (79.00)& \cellcolor{red!40}\href{../works/Groleaz21.pdf}{Groleaz21} (79.00)\\
Cosine& \cellcolor{red!40}\href{../works/GalleguillosKSB19.pdf}{GalleguillosKSB19} (0.78)& \cellcolor{red!40}\href{../works/BridiBLMB16.pdf}{BridiBLMB16} (0.72)& \cellcolor{red!40}\href{../works/ZeballosH05.pdf}{ZeballosH05} (0.71)& \cellcolor{red!40}\href{../works/BorghesiBLMB18.pdf}{BorghesiBLMB18} (0.70)& \cellcolor{red!40}\href{../works/Beck06.pdf}{Beck06} (0.69)\\
\index{BarzegaranZP20}\href{../works/BarzegaranZP20.pdf}{BarzegaranZP20} R\&C\\
Euclid& \cellcolor{red!40}\href{../works/ZibranR11.pdf}{ZibranR11} (0.23)& \cellcolor{red!20}\href{../works/GomesHS06.pdf}{GomesHS06} (0.24)& \cellcolor{red!20}\href{../works/ZibranR11a.pdf}{ZibranR11a} (0.25)& \cellcolor{red!20}\href{../works/PesantGPR99.pdf}{PesantGPR99} (0.25)& \cellcolor{red!20}\href{../works/FrankDT16.pdf}{FrankDT16} (0.26)\\
Dot& \cellcolor{red!40}\href{../works/Groleaz21.pdf}{Groleaz21} (78.00)& \cellcolor{red!40}\href{../works/ZarandiASC20.pdf}{ZarandiASC20} (75.00)& \cellcolor{red!40}\href{../works/Astrand21.pdf}{Astrand21} (72.00)& \cellcolor{red!40}\href{../works/Froger16.pdf}{Froger16} (71.00)& \cellcolor{red!40}\href{../works/HarjunkoskiMBC14.pdf}{HarjunkoskiMBC14} (69.00)\\
Cosine& \cellcolor{red!40}\href{../works/FallahiAC20.pdf}{FallahiAC20} (0.71)& \cellcolor{red!40}\href{../works/BoothNB16.pdf}{BoothNB16} (0.70)& \cellcolor{red!40}\href{../works/ZibranR11.pdf}{ZibranR11} (0.69)& \cellcolor{red!40}\href{../works/MontemanniD23a.pdf}{MontemanniD23a} (0.67)& \cellcolor{red!40}\href{../works/ZibranR11a.pdf}{ZibranR11a} (0.67)\\
\index{Beck06}\href{../works/Beck06.pdf}{Beck06} R\&C\\
Euclid& \cellcolor{red!40}\href{../works/WatsonB08.pdf}{WatsonB08} (0.21)& \cellcolor{red!40}\href{../works/Beck07.pdf}{Beck07} (0.22)& \cellcolor{red!40}\href{../works/HeckmanB11.pdf}{HeckmanB11} (0.23)& \cellcolor{red!40}\href{../works/CarchraeB09.pdf}{CarchraeB09} (0.23)& \cellcolor{red!20}\href{../works/BeckW05.pdf}{BeckW05} (0.24)\\
Dot& \cellcolor{red!40}\href{../works/ZarandiASC20.pdf}{ZarandiASC20} (107.00)& \cellcolor{red!40}\href{../works/Groleaz21.pdf}{Groleaz21} (105.00)& \cellcolor{red!40}\href{../works/GrimesH15.pdf}{GrimesH15} (103.00)& \cellcolor{red!40}\href{../works/Lunardi20.pdf}{Lunardi20} (100.00)& \cellcolor{red!40}\href{../works/Dejemeppe16.pdf}{Dejemeppe16} (100.00)\\
Cosine& \cellcolor{red!40}\href{../works/Beck07.pdf}{Beck07} (0.86)& \cellcolor{red!40}\href{../works/WatsonB08.pdf}{WatsonB08} (0.86)& \cellcolor{red!40}\href{../works/HeckmanB11.pdf}{HeckmanB11} (0.83)& \cellcolor{red!40}\href{../works/CarchraeB09.pdf}{CarchraeB09} (0.83)& \cellcolor{red!40}\href{../works/BeckFW11.pdf}{BeckFW11} (0.81)\\
\index{Beck07}\href{../works/Beck07.pdf}{Beck07} R\&C& \cellcolor{red!40}\href{../works/WatsonB08.pdf}{WatsonB08} (0.83)& \cellcolor{red!20}\href{../works/BeckF00.pdf}{BeckF00} (0.90)& \cellcolor{yellow!20}\href{../works/GrimesHM09.pdf}{GrimesHM09} (0.92)& \cellcolor{green!20}\href{../works/BeckFW11.pdf}{BeckFW11} (0.94)& \cellcolor{green!20}\href{../works/GrimesH10.pdf}{GrimesH10} (0.94)\\
Euclid& \cellcolor{red!40}\href{../works/HeckmanB11.pdf}{HeckmanB11} (0.22)& \cellcolor{red!40}\href{../works/Beck06.pdf}{Beck06} (0.22)& \cellcolor{red!40}\href{../works/WatsonB08.pdf}{WatsonB08} (0.23)& \cellcolor{red!20}\href{../works/HentenryckM04.pdf}{HentenryckM04} (0.24)& \cellcolor{red!20}\href{../works/BeckW05.pdf}{BeckW05} (0.26)\\
Dot& \cellcolor{red!40}\href{../works/Groleaz21.pdf}{Groleaz21} (130.00)& \cellcolor{red!40}\href{../works/Astrand21.pdf}{Astrand21} (128.00)& \cellcolor{red!40}\href{../works/ZarandiASC20.pdf}{ZarandiASC20} (125.00)& \cellcolor{red!40}\href{../works/Lombardi10.pdf}{Lombardi10} (125.00)& \cellcolor{red!40}\href{../works/Schutt11.pdf}{Schutt11} (124.00)\\
Cosine& \cellcolor{red!40}\href{../works/HeckmanB11.pdf}{HeckmanB11} (0.87)& \cellcolor{red!40}\href{../works/Beck06.pdf}{Beck06} (0.86)& \cellcolor{red!40}\href{../works/WatsonB08.pdf}{WatsonB08} (0.85)& \cellcolor{red!40}\href{../works/HentenryckM04.pdf}{HentenryckM04} (0.83)& \cellcolor{red!40}\href{../works/BeckFW11.pdf}{BeckFW11} (0.81)\\
\index{Beck10}\href{../works/Beck10.pdf}{Beck10} R\&C& \cellcolor{red!40}\href{../works/Hooker05.pdf}{Hooker05} (0.68)& \cellcolor{red!40}\href{../works/CobanH11.pdf}{CobanH11} (0.71)& \cellcolor{red!40}\href{../works/CireCH16.pdf}{CireCH16} (0.72)& \cellcolor{red!40}\href{../works/ChuX05.pdf}{ChuX05} (0.74)& \cellcolor{red!40}\href{../works/Hooker05a.pdf}{Hooker05a} (0.75)\\
Euclid& \cellcolor{red!40}\href{../works/HeinzKB13.pdf}{HeinzKB13} (0.23)& \cellcolor{red!40}\href{../works/HookerY02.pdf}{HookerY02} (0.23)& \cellcolor{red!40}\href{../works/ChuX05.pdf}{ChuX05} (0.24)& \cellcolor{red!40}\href{../works/CireCH13.pdf}{CireCH13} (0.24)& \cellcolor{red!20}\href{../works/HookerO03.pdf}{HookerO03} (0.25)\\
Dot& \cellcolor{red!40}\href{../works/Lombardi10.pdf}{Lombardi10} (100.00)& \cellcolor{red!40}\href{../works/BajestaniB13.pdf}{BajestaniB13} (96.00)& \cellcolor{red!40}\href{../works/JuvinHL23a.pdf}{JuvinHL23a} (93.00)& \cellcolor{red!40}\href{../works/MilanoW09.pdf}{MilanoW09} (92.00)& \cellcolor{red!40}\href{../works/TerekhovDOB12.pdf}{TerekhovDOB12} (91.00)\\
Cosine& \cellcolor{red!40}\href{../works/HeinzKB13.pdf}{HeinzKB13} (0.81)& \cellcolor{red!40}\href{../works/ChuX05.pdf}{ChuX05} (0.78)& \cellcolor{red!40}\href{../works/HookerY02.pdf}{HookerY02} (0.78)& \cellcolor{red!40}\href{../works/CireCH13.pdf}{CireCH13} (0.78)& \cellcolor{red!40}\href{../works/HookerO03.pdf}{HookerO03} (0.76)\\
\index{Beck99}\href{../works/Beck99.pdf}{Beck99} R\&C\\
Euclid& \cellcolor{black!20}\href{../works/BeckF98.pdf}{BeckF98} (0.36)& \cellcolor{black!20}\href{../works/BeckDDF98.pdf}{BeckDDF98} (0.37)& \href{../works/BeckF00.pdf}{BeckF00} (0.38)& \href{../works/BeckF00a.pdf}{BeckF00a} (0.42)& \href{../works/GokgurHO18.pdf}{GokgurHO18} (0.44)\\
Dot& \cellcolor{red!40}\href{../works/Dejemeppe16.pdf}{Dejemeppe16} (251.00)& \cellcolor{red!40}\href{../works/Lombardi10.pdf}{Lombardi10} (244.00)& \cellcolor{red!40}\href{../works/Baptiste02.pdf}{Baptiste02} (241.00)& \cellcolor{red!40}\href{../works/ZarandiASC20.pdf}{ZarandiASC20} (240.00)& \cellcolor{red!40}\href{../works/Malapert11.pdf}{Malapert11} (231.00)\\
Cosine& \cellcolor{red!40}\href{../works/BeckF98.pdf}{BeckF98} (0.84)& \cellcolor{red!40}\href{../works/BeckDDF98.pdf}{BeckDDF98} (0.83)& \cellcolor{red!40}\href{../works/BeckF00.pdf}{BeckF00} (0.82)& \cellcolor{red!40}\href{../works/BeckF00a.pdf}{BeckF00a} (0.78)& \cellcolor{red!40}\href{../works/BartakSR10.pdf}{BartakSR10} (0.75)\\
\index{BeckDDF98}\href{../works/BeckDDF98.pdf}{BeckDDF98} R\&C& \cellcolor{green!20}\href{../works/BeckF00.pdf}{BeckF00} (0.94)& \cellcolor{green!20}\href{../works/HeinzB12.pdf}{HeinzB12} (0.94)& \cellcolor{green!20}\href{../works/HeinzKB13.pdf}{HeinzKB13} (0.94)& \cellcolor{green!20}\href{../works/CobanH10.pdf}{CobanH10} (0.94)& \cellcolor{green!20}\href{../works/Muscettola02.pdf}{Muscettola02} (0.96)\\
Euclid& \cellcolor{green!20}\href{../works/BeckF98.pdf}{BeckF98} (0.31)& \cellcolor{black!20}\href{../works/BidotVLB09.pdf}{BidotVLB09} (0.34)& \cellcolor{black!20}\href{../works/BeckPS03.pdf}{BeckPS03} (0.35)& \cellcolor{black!20}\href{../works/BeckDSF97.pdf}{BeckDSF97} (0.35)& \cellcolor{black!20}\href{../works/BeckF00.pdf}{BeckF00} (0.36)\\
Dot& \cellcolor{red!40}\href{../works/ZarandiASC20.pdf}{ZarandiASC20} (224.00)& \cellcolor{red!40}\href{../works/Beck99.pdf}{Beck99} (212.00)& \cellcolor{red!40}\href{../works/Dejemeppe16.pdf}{Dejemeppe16} (203.00)& \cellcolor{red!40}\href{../works/Baptiste02.pdf}{Baptiste02} (199.00)& \cellcolor{red!40}\href{../works/Groleaz21.pdf}{Groleaz21} (196.00)\\
Cosine& \cellcolor{red!40}\href{../works/BeckF98.pdf}{BeckF98} (0.84)& \cellcolor{red!40}\href{../works/Beck99.pdf}{Beck99} (0.83)& \cellcolor{red!40}\href{../works/BidotVLB09.pdf}{BidotVLB09} (0.80)& \cellcolor{red!40}\href{../works/BeckDSF97.pdf}{BeckDSF97} (0.79)& \cellcolor{red!40}\href{../works/BeckPS03.pdf}{BeckPS03} (0.78)\\
\index{BeckDF97}\href{../works/BeckDF97.pdf}{BeckDF97} R\&C& \cellcolor{red!40}\href{../works/BeckF00.pdf}{BeckF00} (0.77)& \cellcolor{red!20}\href{../works/CarlssonKA99.pdf}{CarlssonKA99} (0.87)& \cellcolor{red!20}\href{../works/GrimesHM09.pdf}{GrimesHM09} (0.88)& \cellcolor{red!20}\href{../works/TorresL00.pdf}{TorresL00} (0.88)& \cellcolor{red!20}BaptisteLPN06 (0.89)\\
Euclid& \cellcolor{red!40}\href{../works/BeckDSF97.pdf}{BeckDSF97} (0.19)& \cellcolor{red!40}\href{../works/BeckDSF97a.pdf}{BeckDSF97a} (0.20)& \cellcolor{yellow!20}\href{../works/BeckF00.pdf}{BeckF00} (0.27)& \cellcolor{yellow!20}\href{../works/CestaOS00.pdf}{CestaOS00} (0.27)& \cellcolor{yellow!20}\href{../works/SadehF96.pdf}{SadehF96} (0.28)\\
Dot& \cellcolor{red!40}\href{../works/Beck99.pdf}{Beck99} (133.00)& \cellcolor{red!40}\href{../works/Dejemeppe16.pdf}{Dejemeppe16} (120.00)& \cellcolor{red!40}\href{../works/Lombardi10.pdf}{Lombardi10} (117.00)& \cellcolor{red!40}\href{../works/Baptiste02.pdf}{Baptiste02} (117.00)& \cellcolor{red!40}\href{../works/BeckF00.pdf}{BeckF00} (114.00)\\
Cosine& \cellcolor{red!40}\href{../works/BeckDSF97.pdf}{BeckDSF97} (0.88)& \cellcolor{red!40}\href{../works/BeckDSF97a.pdf}{BeckDSF97a} (0.87)& \cellcolor{red!40}\href{../works/BeckF00.pdf}{BeckF00} (0.84)& \cellcolor{red!40}\href{../works/BeckF98.pdf}{BeckF98} (0.80)& \cellcolor{red!40}\href{../works/SadehF96.pdf}{SadehF96} (0.80)\\
\index{BeckDSF97}\href{../works/BeckDSF97.pdf}{BeckDSF97} R\&C\\
Euclid& \cellcolor{red!40}\href{../works/BeckDSF97a.pdf}{BeckDSF97a} (0.19)& \cellcolor{red!40}\href{../works/BeckDF97.pdf}{BeckDF97} (0.19)& \cellcolor{red!20}\href{../works/BeckF00a.pdf}{BeckF00a} (0.24)& \cellcolor{red!20}\href{../works/BeckF99.pdf}{BeckF99} (0.25)& \cellcolor{red!20}\href{../works/VilimBC04.pdf}{VilimBC04} (0.26)\\
Dot& \cellcolor{red!40}\href{../works/Beck99.pdf}{Beck99} (129.00)& \cellcolor{red!40}\href{../works/BeckDDF98.pdf}{BeckDDF98} (118.00)& \cellcolor{red!40}\href{../works/Baptiste02.pdf}{Baptiste02} (117.00)& \cellcolor{red!40}\href{../works/Dejemeppe16.pdf}{Dejemeppe16} (116.00)& \cellcolor{red!40}\href{../works/Lombardi10.pdf}{Lombardi10} (113.00)\\
Cosine& \cellcolor{red!40}\href{../works/BeckDSF97a.pdf}{BeckDSF97a} (0.88)& \cellcolor{red!40}\href{../works/BeckDF97.pdf}{BeckDF97} (0.88)& \cellcolor{red!40}\href{../works/BeckF00.pdf}{BeckF00} (0.84)& \cellcolor{red!40}\href{../works/BeckF00a.pdf}{BeckF00a} (0.82)& \cellcolor{red!40}\href{../works/BeckDDF98.pdf}{BeckDDF98} (0.79)\\
\index{BeckDSF97a}\href{../works/BeckDSF97a.pdf}{BeckDSF97a} R\&C\\
Euclid& \cellcolor{red!40}\href{../works/BeckDSF97.pdf}{BeckDSF97} (0.19)& \cellcolor{red!40}\href{../works/BeckDF97.pdf}{BeckDF97} (0.20)& \cellcolor{red!40}\href{../works/BeckW05.pdf}{BeckW05} (0.22)& \cellcolor{red!40}\href{../works/BeckF00a.pdf}{BeckF00a} (0.23)& \cellcolor{red!40}\href{../works/BeckF99.pdf}{BeckF99} (0.23)\\
Dot& \cellcolor{red!40}\href{../works/Beck99.pdf}{Beck99} (120.00)& \cellcolor{red!40}\href{../works/Dejemeppe16.pdf}{Dejemeppe16} (117.00)& \cellcolor{red!40}\href{../works/Baptiste02.pdf}{Baptiste02} (112.00)& \cellcolor{red!40}\href{../works/BeckF00.pdf}{BeckF00} (111.00)& \cellcolor{red!40}\href{../works/Fahimi16.pdf}{Fahimi16} (111.00)\\
Cosine& \cellcolor{red!40}\href{../works/BeckDSF97.pdf}{BeckDSF97} (0.88)& \cellcolor{red!40}\href{../works/BeckDF97.pdf}{BeckDF97} (0.87)& \cellcolor{red!40}\href{../works/BeckF00a.pdf}{BeckF00a} (0.85)& \cellcolor{red!40}\href{../works/BeckF00.pdf}{BeckF00} (0.84)& \cellcolor{red!40}\href{../works/HeckmanB11.pdf}{HeckmanB11} (0.83)\\
\index{BeckF00}\href{../works/BeckF00.pdf}{BeckF00} R\&C& \cellcolor{red!40}\href{../works/BeckDF97.pdf}{BeckDF97} (0.77)& \cellcolor{red!40}\href{../works/BeckF00a.pdf}{BeckF00a} (0.78)& \cellcolor{red!40}\href{../works/Colombani96.pdf}{Colombani96} (0.83)& \cellcolor{red!40}\href{../works/WatsonB08.pdf}{WatsonB08} (0.84)& \cellcolor{red!40}\href{../works/Zhou96.pdf}{Zhou96} (0.85)\\
Euclid& \cellcolor{red!40}\href{../works/BeckF00a.pdf}{BeckF00a} (0.23)& \cellcolor{yellow!20}\href{../works/BeckDF97.pdf}{BeckDF97} (0.27)& \cellcolor{yellow!20}\href{../works/BeckDSF97a.pdf}{BeckDSF97a} (0.27)& \cellcolor{yellow!20}\href{../works/BeckDSF97.pdf}{BeckDSF97} (0.27)& \cellcolor{green!20}\href{../works/BeckF99.pdf}{BeckF99} (0.29)\\
Dot& \cellcolor{red!40}\href{../works/Beck99.pdf}{Beck99} (184.00)& \cellcolor{red!40}\href{../works/Baptiste02.pdf}{Baptiste02} (175.00)& \cellcolor{red!40}\href{../works/Dejemeppe16.pdf}{Dejemeppe16} (172.00)& \cellcolor{red!40}\href{../works/Fahimi16.pdf}{Fahimi16} (171.00)& \cellcolor{red!40}\href{../works/Lombardi10.pdf}{Lombardi10} (170.00)\\
Cosine& \cellcolor{red!40}\href{../works/BeckF00a.pdf}{BeckF00a} (0.89)& \cellcolor{red!40}\href{../works/BeckDF97.pdf}{BeckDF97} (0.84)& \cellcolor{red!40}\href{../works/BeckDSF97a.pdf}{BeckDSF97a} (0.84)& \cellcolor{red!40}\href{../works/BeckDSF97.pdf}{BeckDSF97} (0.84)& \cellcolor{red!40}\href{../works/BeckF99.pdf}{BeckF99} (0.82)\\
\index{BeckF00a}\href{../works/BeckF00a.pdf}{BeckF00a} R\&C& \cellcolor{red!40}\href{../works/BeckF00.pdf}{BeckF00} (0.78)& \cellcolor{red!40}\href{../works/HebrardTW05.pdf}{HebrardTW05} (0.85)& \cellcolor{red!40}\href{../works/CarlssonKA99.pdf}{CarlssonKA99} (0.85)& \cellcolor{red!40}\href{../works/VilimBC05.pdf}{VilimBC05} (0.86)& \cellcolor{red!20}\href{../works/Colombani96.pdf}{Colombani96} (0.87)\\
Euclid& \cellcolor{red!40}\href{../works/BeckF99.pdf}{BeckF99} (0.18)& \cellcolor{red!40}\href{../works/BeckF00.pdf}{BeckF00} (0.23)& \cellcolor{red!40}\href{../works/BeckDSF97a.pdf}{BeckDSF97a} (0.23)& \cellcolor{red!20}\href{../works/BeckDSF97.pdf}{BeckDSF97} (0.24)& \cellcolor{red!20}\href{../works/Vilim05.pdf}{Vilim05} (0.26)\\
Dot& \cellcolor{red!40}\href{../works/Beck99.pdf}{Beck99} (150.00)& \cellcolor{red!40}\href{../works/Baptiste02.pdf}{Baptiste02} (140.00)& \cellcolor{red!40}\href{../works/Lombardi10.pdf}{Lombardi10} (138.00)& \cellcolor{red!40}\href{../works/Dejemeppe16.pdf}{Dejemeppe16} (136.00)& \cellcolor{red!40}\href{../works/Fahimi16.pdf}{Fahimi16} (135.00)\\
Cosine& \cellcolor{red!40}\href{../works/BeckF99.pdf}{BeckF99} (0.91)& \cellcolor{red!40}\href{../works/BeckF00.pdf}{BeckF00} (0.89)& \cellcolor{red!40}\href{../works/BeckDSF97a.pdf}{BeckDSF97a} (0.85)& \cellcolor{red!40}\href{../works/BeckDSF97.pdf}{BeckDSF97} (0.82)& \cellcolor{red!40}\href{../works/Vilim05.pdf}{Vilim05} (0.81)\\
\index{BeckF98}\href{../works/BeckF98.pdf}{BeckF98} R\&C\\
Euclid& \cellcolor{green!20}\href{../works/BeckDF97.pdf}{BeckDF97} (0.31)& \cellcolor{green!20}\href{../works/BeckDDF98.pdf}{BeckDDF98} (0.31)& \cellcolor{blue!20}\href{../works/BeckDSF97a.pdf}{BeckDSF97a} (0.32)& \cellcolor{blue!20}\href{../works/SadehF96.pdf}{SadehF96} (0.32)& \cellcolor{blue!20}\href{../works/BeckDSF97.pdf}{BeckDSF97} (0.32)\\
Dot& \cellcolor{red!40}\href{../works/Beck99.pdf}{Beck99} (196.00)& \cellcolor{red!40}\href{../works/ZarandiASC20.pdf}{ZarandiASC20} (184.00)& \cellcolor{red!40}\href{../works/Dejemeppe16.pdf}{Dejemeppe16} (180.00)& \cellcolor{red!40}\href{../works/Groleaz21.pdf}{Groleaz21} (178.00)& \cellcolor{red!40}\href{../works/Lombardi10.pdf}{Lombardi10} (174.00)\\
Cosine& \cellcolor{red!40}\href{../works/Beck99.pdf}{Beck99} (0.84)& \cellcolor{red!40}\href{../works/BeckDDF98.pdf}{BeckDDF98} (0.84)& \cellcolor{red!40}\href{../works/BeckDF97.pdf}{BeckDF97} (0.80)& \cellcolor{red!40}\href{../works/BeckF00.pdf}{BeckF00} (0.80)& \cellcolor{red!40}\href{../works/SadehF96.pdf}{SadehF96} (0.79)\\
\index{BeckF99}\href{../works/BeckF99.pdf}{BeckF99} R\&C\\
Euclid& \cellcolor{red!40}\href{../works/BeckF00a.pdf}{BeckF00a} (0.18)& \cellcolor{red!40}\href{../works/VilimBC04.pdf}{VilimBC04} (0.23)& \cellcolor{red!40}\href{../works/BeckDSF97a.pdf}{BeckDSF97a} (0.23)& \cellcolor{red!40}\href{../works/Vilim09.pdf}{Vilim09} (0.23)& \cellcolor{red!40}\href{../works/BeckW05.pdf}{BeckW05} (0.24)\\
Dot& \cellcolor{red!40}\href{../works/Beck99.pdf}{Beck99} (107.00)& \cellcolor{red!40}\href{../works/Dejemeppe16.pdf}{Dejemeppe16} (102.00)& \cellcolor{red!40}\href{../works/Schutt11.pdf}{Schutt11} (101.00)& \cellcolor{red!40}\href{../works/Fahimi16.pdf}{Fahimi16} (100.00)& \cellcolor{red!40}\href{../works/Malapert11.pdf}{Malapert11} (99.00)\\
Cosine& \cellcolor{red!40}\href{../works/BeckF00a.pdf}{BeckF00a} (0.91)& \cellcolor{red!40}\href{../works/BeckF00.pdf}{BeckF00} (0.82)& \cellcolor{red!40}\href{../works/BeckDSF97a.pdf}{BeckDSF97a} (0.81)& \cellcolor{red!40}\href{../works/VilimBC04.pdf}{VilimBC04} (0.80)& \cellcolor{red!40}\href{../works/VilimBC05.pdf}{VilimBC05} (0.80)\\
\index{BeckFW11}\href{../works/BeckFW11.pdf}{BeckFW11} R\&C& \cellcolor{red!40}\href{../works/WatsonB08.pdf}{WatsonB08} (0.61)& \cellcolor{red!40}\href{../works/GrimesHM09.pdf}{GrimesHM09} (0.74)& \cellcolor{red!40}\href{../works/CarchraeB09.pdf}{CarchraeB09} (0.80)& \cellcolor{red!40}\href{../works/HeckmanB11.pdf}{HeckmanB11} (0.80)& \cellcolor{red!40}\href{../works/MenciaSV13.pdf}{MenciaSV13} (0.85)\\
Euclid& \cellcolor{red!40}\href{../works/WatsonB08.pdf}{WatsonB08} (0.16)& \cellcolor{red!40}\href{../works/CarchraeB09.pdf}{CarchraeB09} (0.23)& \cellcolor{red!20}\href{../works/TanSD10.pdf}{TanSD10} (0.24)& \cellcolor{red!20}\href{../works/Beck06.pdf}{Beck06} (0.25)& \cellcolor{yellow!20}\href{../works/Beck07.pdf}{Beck07} (0.26)\\
Dot& \cellcolor{red!40}\href{../works/Groleaz21.pdf}{Groleaz21} (130.00)& \cellcolor{red!40}\href{../works/ZarandiASC20.pdf}{ZarandiASC20} (124.00)& \cellcolor{red!40}\href{../works/Schutt11.pdf}{Schutt11} (124.00)& \cellcolor{red!40}\href{../works/Lunardi20.pdf}{Lunardi20} (123.00)& \cellcolor{red!40}\href{../works/Beck99.pdf}{Beck99} (123.00)\\
Cosine& \cellcolor{red!40}\href{../works/WatsonB08.pdf}{WatsonB08} (0.93)& \cellcolor{red!40}\href{../works/CarchraeB09.pdf}{CarchraeB09} (0.85)& \cellcolor{red!40}\href{../works/TanSD10.pdf}{TanSD10} (0.83)& \cellcolor{red!40}\href{../works/Beck06.pdf}{Beck06} (0.81)& \cellcolor{red!40}\href{../works/Beck07.pdf}{Beck07} (0.81)\\
\index{BeckPS03}\href{../works/BeckPS03.pdf}{BeckPS03} R\&C\\
Euclid& \cellcolor{red!20}\href{../works/HeckmanB11.pdf}{HeckmanB11} (0.24)& \cellcolor{red!20}\href{../works/KovacsV06.pdf}{KovacsV06} (0.25)& \cellcolor{red!20}\href{../works/HentenryckM04.pdf}{HentenryckM04} (0.25)& \cellcolor{red!20}\href{../works/KhayatLR06.pdf}{KhayatLR06} (0.26)& \cellcolor{red!20}\href{../works/ZeballosH05.pdf}{ZeballosH05} (0.26)\\
Dot& \cellcolor{red!40}\href{../works/ZarandiASC20.pdf}{ZarandiASC20} (154.00)& \cellcolor{red!40}\href{../works/Dejemeppe16.pdf}{Dejemeppe16} (144.00)& \cellcolor{red!40}\href{../works/Lombardi10.pdf}{Lombardi10} (140.00)& \cellcolor{red!40}\href{../works/Baptiste02.pdf}{Baptiste02} (140.00)& \cellcolor{red!40}\href{../works/Astrand21.pdf}{Astrand21} (139.00)\\
Cosine& \cellcolor{red!40}\href{../works/HeckmanB11.pdf}{HeckmanB11} (0.84)& \cellcolor{red!40}\href{../works/ZeballosH05.pdf}{ZeballosH05} (0.82)& \cellcolor{red!40}\href{../works/HentenryckM04.pdf}{HentenryckM04} (0.82)& \cellcolor{red!40}\href{../works/KovacsV06.pdf}{KovacsV06} (0.82)& \cellcolor{red!40}\href{../works/BeckW07.pdf}{BeckW07} (0.82)\\
\index{BeckR03}\href{../works/BeckR03.pdf}{BeckR03} R\&C& \cellcolor{yellow!20}\href{../works/KelbelH11.pdf}{KelbelH11} (0.90)& \cellcolor{yellow!20}\href{../works/DannaP03.pdf}{DannaP03} (0.92)& \cellcolor{yellow!20}\href{../works/KamarainenS02.pdf}{KamarainenS02} (0.92)& \cellcolor{green!20}\href{../works/GrimesH10.pdf}{GrimesH10} (0.95)& \cellcolor{green!20}\href{../works/GrimesHM09.pdf}{GrimesHM09} (0.95)\\
Euclid& \cellcolor{yellow!20}\href{../works/DannaP03.pdf}{DannaP03} (0.27)& \cellcolor{green!20}\href{../works/HeckmanB11.pdf}{HeckmanB11} (0.29)& \cellcolor{green!20}\href{../works/MonetteDH09.pdf}{MonetteDH09} (0.30)& \cellcolor{green!20}\href{../works/BeckPS03.pdf}{BeckPS03} (0.30)& \cellcolor{green!20}\href{../works/HentenryckM04.pdf}{HentenryckM04} (0.31)\\
Dot& \cellcolor{red!40}\href{../works/ZarandiASC20.pdf}{ZarandiASC20} (153.00)& \cellcolor{red!40}\href{../works/Groleaz21.pdf}{Groleaz21} (152.00)& \cellcolor{red!40}\href{../works/Baptiste02.pdf}{Baptiste02} (150.00)& \cellcolor{red!40}\href{../works/Dejemeppe16.pdf}{Dejemeppe16} (139.00)& \cellcolor{red!40}\href{../works/Lombardi10.pdf}{Lombardi10} (138.00)\\
Cosine& \cellcolor{red!40}\href{../works/DannaP03.pdf}{DannaP03} (0.80)& \cellcolor{red!40}\href{../works/MonetteDH09.pdf}{MonetteDH09} (0.79)& \cellcolor{red!40}\href{../works/HeckmanB11.pdf}{HeckmanB11} (0.78)& \cellcolor{red!40}\href{../works/BeckPS03.pdf}{BeckPS03} (0.77)& \cellcolor{red!40}\href{../works/GrimesH11.pdf}{GrimesH11} (0.77)\\
\index{BeckW04}\href{../works/BeckW04.pdf}{BeckW04} R\&C\\
Euclid& \cellcolor{red!40}\href{../works/BeckW05.pdf}{BeckW05} (0.19)& \cellcolor{red!20}\href{../works/CrawfordB94.pdf}{CrawfordB94} (0.24)& \cellcolor{red!20}\href{../works/BonfiettiLM14.pdf}{BonfiettiLM14} (0.25)& \cellcolor{red!20}\href{../works/OddiS97.pdf}{OddiS97} (0.25)& \cellcolor{red!20}\href{../works/LauLN08.pdf}{LauLN08} (0.25)\\
Dot& \cellcolor{red!40}\href{../works/ZarandiASC20.pdf}{ZarandiASC20} (99.00)& \cellcolor{red!40}\href{../works/Astrand21.pdf}{Astrand21} (93.00)& \cellcolor{red!40}\href{../works/Beck99.pdf}{Beck99} (92.00)& \cellcolor{red!40}\href{../works/BeckW07.pdf}{BeckW07} (91.00)& \cellcolor{red!40}\href{../works/Baptiste02.pdf}{Baptiste02} (88.00)\\
Cosine& \cellcolor{red!40}\href{../works/BeckW05.pdf}{BeckW05} (0.85)& \cellcolor{red!40}\href{../works/BonfiettiLM14.pdf}{BonfiettiLM14} (0.79)& \cellcolor{red!40}\href{../works/BeckW07.pdf}{BeckW07} (0.78)& \cellcolor{red!40}\href{../works/BeckDSF97a.pdf}{BeckDSF97a} (0.77)& \cellcolor{red!40}\href{../works/OddiS97.pdf}{OddiS97} (0.75)\\
\index{BeckW05}\href{../works/BeckW05.pdf}{BeckW05} R\&C\\
Euclid& \cellcolor{red!40}\href{../works/BeckW04.pdf}{BeckW04} (0.19)& \cellcolor{red!40}\href{../works/BeckDSF97a.pdf}{BeckDSF97a} (0.22)& \cellcolor{red!40}\href{../works/DoomsH08.pdf}{DoomsH08} (0.23)& \cellcolor{red!40}\href{../works/OddiS97.pdf}{OddiS97} (0.23)& \cellcolor{red!40}\href{../works/BidotVLB07.pdf}{BidotVLB07} (0.24)\\
Dot& \cellcolor{red!40}\href{../works/Lombardi10.pdf}{Lombardi10} (82.00)& \cellcolor{red!40}\href{../works/Beck99.pdf}{Beck99} (81.00)& \cellcolor{red!40}\href{../works/BeckW07.pdf}{BeckW07} (80.00)& \cellcolor{red!40}\href{../works/ZarandiASC20.pdf}{ZarandiASC20} (80.00)& \cellcolor{red!40}\href{../works/Siala15a.pdf}{Siala15a} (79.00)\\
Cosine& \cellcolor{red!40}\href{../works/BeckW04.pdf}{BeckW04} (0.85)& \cellcolor{red!40}\href{../works/BeckDSF97a.pdf}{BeckDSF97a} (0.83)& \cellcolor{red!40}\href{../works/HeckmanB11.pdf}{HeckmanB11} (0.80)& \cellcolor{red!40}\href{../works/Beck07.pdf}{Beck07} (0.78)& \cellcolor{red!40}\href{../works/Beck06.pdf}{Beck06} (0.77)\\
\index{BeckW07}\href{../works/BeckW07.pdf}{BeckW07} R\&C& \cellcolor{green!20}\href{../works/WuBB09.pdf}{WuBB09} (0.94)& \cellcolor{blue!20}\href{../works/BonfiettiLM14.pdf}{BonfiettiLM14} (0.97)& \cellcolor{blue!20}\href{../works/RossiTHP07.pdf}{RossiTHP07} (0.97)& \cellcolor{blue!20}\href{../works/LombardiM09.pdf}{LombardiM09} (0.97)& \cellcolor{blue!20}\href{../works/Muscettola02.pdf}{Muscettola02} (0.98)\\
Euclid& \cellcolor{yellow!20}\href{../works/BeckPS03.pdf}{BeckPS03} (0.28)& \cellcolor{green!20}\href{../works/BonfiettiLM14.pdf}{BonfiettiLM14} (0.29)& \cellcolor{green!20}\href{../works/LombardiBM15.pdf}{LombardiBM15} (0.30)& \cellcolor{green!20}\href{../works/BeckW04.pdf}{BeckW04} (0.30)& \cellcolor{green!20}\href{../works/KovacsV06.pdf}{KovacsV06} (0.31)\\
Dot& \cellcolor{red!40}\href{../works/ZarandiASC20.pdf}{ZarandiASC20} (168.00)& \cellcolor{red!40}\href{../works/Lombardi10.pdf}{Lombardi10} (154.00)& \cellcolor{red!40}\href{../works/Baptiste02.pdf}{Baptiste02} (148.00)& \cellcolor{red!40}\href{../works/Groleaz21.pdf}{Groleaz21} (147.00)& \cellcolor{red!40}\href{../works/LaborieRSV18.pdf}{LaborieRSV18} (141.00)\\
Cosine& \cellcolor{red!40}\href{../works/BeckPS03.pdf}{BeckPS03} (0.82)& \cellcolor{red!40}\href{../works/BonfiettiLM14.pdf}{BonfiettiLM14} (0.79)& \cellcolor{red!40}\href{../works/LombardiBM15.pdf}{LombardiBM15} (0.78)& \cellcolor{red!40}\href{../works/BeckW04.pdf}{BeckW04} (0.78)& \cellcolor{red!40}\href{../works/DemasseyAM05.pdf}{DemasseyAM05} (0.77)\\
\index{Bedhief21}\href{../works/Bedhief21.pdf}{Bedhief21} R\&C\\
Euclid& \cellcolor{green!20}\href{../works/JuvinHL23.pdf}{JuvinHL23} (0.31)& \cellcolor{blue!20}\href{../works/ZhangJZL22.pdf}{ZhangJZL22} (0.33)& \cellcolor{black!20}\href{../works/HamdiL13.pdf}{HamdiL13} (0.34)& \cellcolor{black!20}\href{../works/BillautHL12.pdf}{BillautHL12} (0.34)& \cellcolor{black!20}\href{../works/EdisO11.pdf}{EdisO11} (0.35)\\
Dot& \cellcolor{red!40}\href{../works/Groleaz21.pdf}{Groleaz21} (151.00)& \cellcolor{red!40}\href{../works/Lunardi20.pdf}{Lunardi20} (144.00)& \cellcolor{red!40}\href{../works/ZarandiASC20.pdf}{ZarandiASC20} (144.00)& \cellcolor{red!40}\href{../works/IsikYA23.pdf}{IsikYA23} (143.00)& \cellcolor{red!40}\href{../works/Astrand21.pdf}{Astrand21} (138.00)\\
Cosine& \cellcolor{red!40}\href{../works/JuvinHL23.pdf}{JuvinHL23} (0.74)& \cellcolor{red!40}\href{../works/ZhangJZL22.pdf}{ZhangJZL22} (0.73)& \cellcolor{red!40}\href{../works/MengZRZL20.pdf}{MengZRZL20} (0.71)& \cellcolor{red!40}\href{../works/HamdiL13.pdf}{HamdiL13} (0.70)& \cellcolor{red!40}\href{../works/Novas19.pdf}{Novas19} (0.70)\\
\index{BegB13}\href{../works/BegB13.pdf}{BegB13} R\&C& \cellcolor{red!20}\href{../works/MalikMB08.pdf}{MalikMB08} (0.88)& \cellcolor{blue!20}\href{../works/LozanoCDS12.pdf}{LozanoCDS12} (0.97)& \cellcolor{blue!20}\href{../works/EreminW01.pdf}{EreminW01} (0.97)& \cellcolor{blue!20}\href{../works/BeldiceanuC01.pdf}{BeldiceanuC01} (0.97)& \cellcolor{blue!20}\href{../works/CambazardJ05.pdf}{CambazardJ05} (0.97)\\
Euclid& \cellcolor{red!40}\href{../works/MalikMB08.pdf}{MalikMB08} (0.19)& \cellcolor{red!20}\href{../works/Malik08.pdf}{Malik08} (0.25)& \cellcolor{red!20}\href{../works/KuchcinskiW03.pdf}{KuchcinskiW03} (0.25)& \cellcolor{red!20}\href{../works/ErtlK91.pdf}{ErtlK91} (0.26)& \cellcolor{red!20}\href{../works/LozanoCDS12.pdf}{LozanoCDS12} (0.26)\\
Dot& \cellcolor{red!40}\href{../works/Malik08.pdf}{Malik08} (81.00)& \cellcolor{red!40}\href{../works/Lombardi10.pdf}{Lombardi10} (81.00)& \cellcolor{red!40}\href{../works/Groleaz21.pdf}{Groleaz21} (80.00)& \cellcolor{red!40}\href{../works/Froger16.pdf}{Froger16} (80.00)& \cellcolor{red!40}\href{../works/Kuchcinski03.pdf}{Kuchcinski03} (76.00)\\
Cosine& \cellcolor{red!40}\href{../works/MalikMB08.pdf}{MalikMB08} (0.83)& \cellcolor{red!40}\href{../works/Malik08.pdf}{Malik08} (0.81)& \cellcolor{red!40}\href{../works/KuchcinskiW03.pdf}{KuchcinskiW03} (0.73)& \cellcolor{red!40}\href{../works/LozanoCDS12.pdf}{LozanoCDS12} (0.71)& \cellcolor{red!40}\href{../works/ErtlK91.pdf}{ErtlK91} (0.69)\\
\index{BehrensLM19}\href{../works/BehrensLM19.pdf}{BehrensLM19} R\&C& \cellcolor{red!40}\href{../works/WessenCS20.pdf}{WessenCS20} (0.79)& \cellcolor{yellow!20}\href{../works/WessenCSFPM23.pdf}{WessenCSFPM23} (0.93)& \cellcolor{green!20}\href{../works/CarchraeB09.pdf}{CarchraeB09} (0.94)& \cellcolor{green!20}\href{../works/BoothNB16.pdf}{BoothNB16} (0.94)& \cellcolor{green!20}\href{../works/WallaceY20.pdf}{WallaceY20} (0.94)\\
Euclid& \cellcolor{red!40}\href{../works/abs-1901-07914.pdf}{abs-1901-07914} (0.09)& \cellcolor{red!40}\href{../works/WessenCS20.pdf}{WessenCS20} (0.24)& \cellcolor{red!20}\href{../works/ValleMGT03.pdf}{ValleMGT03} (0.25)& \cellcolor{red!20}\href{../works/JungblutK22.pdf}{JungblutK22} (0.26)& \cellcolor{yellow!20}\href{../works/BoothNB16.pdf}{BoothNB16} (0.26)\\
Dot& \cellcolor{red!40}\href{../works/ZarandiASC20.pdf}{ZarandiASC20} (98.00)& \cellcolor{red!40}\href{../works/BartakSR10.pdf}{BartakSR10} (91.00)& \cellcolor{red!40}\href{../works/abs-1901-07914.pdf}{abs-1901-07914} (89.00)& \cellcolor{red!40}\href{../works/Dejemeppe16.pdf}{Dejemeppe16} (88.00)& \cellcolor{red!40}\href{../works/GombolayWS18.pdf}{GombolayWS18} (88.00)\\
Cosine& \cellcolor{red!40}\href{../works/abs-1901-07914.pdf}{abs-1901-07914} (0.97)& \cellcolor{red!40}\href{../works/WessenCS20.pdf}{WessenCS20} (0.77)& \cellcolor{red!40}\href{../works/ValleMGT03.pdf}{ValleMGT03} (0.74)& \cellcolor{red!40}\href{../works/BoothNB16.pdf}{BoothNB16} (0.72)& \cellcolor{red!40}\href{../works/MokhtarzadehTNF20.pdf}{MokhtarzadehTNF20} (0.72)\\
\index{BeldiceanuC01}\href{../works/BeldiceanuC01.pdf}{BeldiceanuC01} R\&C& \cellcolor{red!20}\href{../works/Wolf03.pdf}{Wolf03} (0.89)& \cellcolor{yellow!20}\href{../works/BeldiceanuCP08.pdf}{BeldiceanuCP08} (0.93)& \cellcolor{yellow!20}\href{../works/Vilim04.pdf}{Vilim04} (0.93)& \cellcolor{green!20}\href{../works/VilimBC04.pdf}{VilimBC04} (0.94)& \cellcolor{green!20}\href{../works/BeldiceanuC02.pdf}{BeldiceanuC02} (0.94)\\
Euclid& \cellcolor{red!40}\href{../works/HebrardALLCMR22.pdf}{HebrardALLCMR22} (0.21)& \cellcolor{red!40}\href{../works/HebrardTW05.pdf}{HebrardTW05} (0.22)& \cellcolor{red!40}\href{../works/Davis87.pdf}{Davis87} (0.22)& \cellcolor{red!40}\href{../works/AngelsmarkJ00.pdf}{AngelsmarkJ00} (0.22)& \cellcolor{red!40}\href{../works/CarchraeBF05.pdf}{CarchraeBF05} (0.22)\\
Dot& \cellcolor{red!40}\href{../works/Malapert11.pdf}{Malapert11} (52.00)& \cellcolor{red!40}\href{../works/Schutt11.pdf}{Schutt11} (48.00)& \cellcolor{red!40}\href{../works/LetortCB15.pdf}{LetortCB15} (45.00)& \cellcolor{red!40}\href{../works/Godet21a.pdf}{Godet21a} (45.00)& \cellcolor{red!40}\href{../works/Dejemeppe16.pdf}{Dejemeppe16} (44.00)\\
Cosine& \cellcolor{red!40}\href{../works/BeldiceanuC02.pdf}{BeldiceanuC02} (0.64)& \cellcolor{red!40}\href{../works/PoderB08.pdf}{PoderB08} (0.63)& \cellcolor{red!40}\href{../works/WolfS05.pdf}{WolfS05} (0.62)& \cellcolor{red!40}\href{../works/ClercqPBJ11.pdf}{ClercqPBJ11} (0.61)& \cellcolor{red!40}\href{../works/BeldiceanuCP08.pdf}{BeldiceanuCP08} (0.61)\\
\index{BeldiceanuC02}\href{../works/BeldiceanuC02.pdf}{BeldiceanuC02} R\&C& \cellcolor{red!40}\href{../works/LetortBC12.pdf}{LetortBC12} (0.75)& \cellcolor{red!40}\href{../works/WolfS05a.pdf}{WolfS05a} (0.77)& \cellcolor{red!40}\href{../works/Wolf03.pdf}{Wolf03} (0.81)& \cellcolor{red!40}\href{../works/Vilim09.pdf}{Vilim09} (0.82)& \cellcolor{red!40}\href{../works/SimoninAHL12.pdf}{SimoninAHL12} (0.82)\\
Euclid& \cellcolor{red!40}\href{../works/PoderB08.pdf}{PoderB08} (0.22)& \cellcolor{red!40}\href{../works/BeldiceanuP07.pdf}{BeldiceanuP07} (0.24)& \cellcolor{red!20}\href{../works/WolfS05.pdf}{WolfS05} (0.25)& \cellcolor{red!20}\href{../works/SimonisH11.pdf}{SimonisH11} (0.25)& \cellcolor{red!20}\href{../works/PoderBS04.pdf}{PoderBS04} (0.26)\\
Dot& \cellcolor{red!40}\href{../works/Malapert11.pdf}{Malapert11} (87.00)& \cellcolor{red!40}\href{../works/Simonis07.pdf}{Simonis07} (77.00)& \cellcolor{red!40}\href{../works/Dejemeppe16.pdf}{Dejemeppe16} (77.00)& \cellcolor{red!40}\href{../works/Schutt11.pdf}{Schutt11} (77.00)& \cellcolor{red!40}\href{../works/Beck99.pdf}{Beck99} (77.00)\\
Cosine& \cellcolor{red!40}\href{../works/PoderB08.pdf}{PoderB08} (0.80)& \cellcolor{red!40}\href{../works/BeldiceanuP07.pdf}{BeldiceanuP07} (0.75)& \cellcolor{red!40}\href{../works/WolfS05.pdf}{WolfS05} (0.75)& \cellcolor{red!40}\href{../works/SimonisH11.pdf}{SimonisH11} (0.73)& \cellcolor{red!40}\href{../works/PoderBS04.pdf}{PoderBS04} (0.73)\\
\index{BeldiceanuC94}\href{../works/BeldiceanuC94.pdf}{BeldiceanuC94} R\&C& \cellcolor{red!40}\href{../works/SimonisC95.pdf}{SimonisC95} (0.85)& \cellcolor{red!40}\href{../works/Goltz95.pdf}{Goltz95} (0.86)& \cellcolor{red!20}\href{../works/AggounB93.pdf}{AggounB93} (0.87)& \cellcolor{red!20}\href{../works/BrusoniCLMMT96.pdf}{BrusoniCLMMT96} (0.88)& \cellcolor{red!20}\href{../works/Simonis95.pdf}{Simonis95} (0.89)\\
Euclid& \cellcolor{green!20}\href{../works/Simonis95.pdf}{Simonis95} (0.31)& \cellcolor{blue!20}\href{../works/Simonis95a.pdf}{Simonis95a} (0.33)& \cellcolor{blue!20}\href{../works/SimonisCK00.pdf}{SimonisCK00} (0.33)& \cellcolor{blue!20}\href{../works/ErtlK91.pdf}{ErtlK91} (0.33)& \cellcolor{blue!20}\href{../works/GruianK98.pdf}{GruianK98} (0.34)\\
Dot& \cellcolor{red!40}\href{../works/Simonis07.pdf}{Simonis07} (108.00)& \cellcolor{red!40}\href{../works/Simonis99.pdf}{Simonis99} (107.00)& \cellcolor{red!40}\href{../works/Malapert11.pdf}{Malapert11} (104.00)& \cellcolor{red!40}\href{../works/Dejemeppe16.pdf}{Dejemeppe16} (101.00)& \cellcolor{red!40}\href{../works/Lombardi10.pdf}{Lombardi10} (101.00)\\
Cosine& \cellcolor{red!40}\href{../works/Simonis95.pdf}{Simonis95} (0.73)& \cellcolor{red!40}\href{../works/Simonis95a.pdf}{Simonis95a} (0.72)& \cellcolor{red!40}\href{../works/SimonisCK00.pdf}{SimonisCK00} (0.69)& \cellcolor{red!40}\href{../works/GruianK98.pdf}{GruianK98} (0.68)& \cellcolor{red!40}\href{../works/ErtlK91.pdf}{ErtlK91} (0.67)\\
\index{BeldiceanuCDP11}\href{../works/BeldiceanuCDP11.pdf}{BeldiceanuCDP11} R\&C& \cellcolor{red!40}\href{../works/BeldiceanuCP08.pdf}{BeldiceanuCP08} (0.53)& \cellcolor{red!40}\href{../works/Simonis95.pdf}{Simonis95} (0.82)& \cellcolor{red!40}AggounV04 (0.83)& \cellcolor{red!40}\href{../works/SimonisCK00.pdf}{SimonisCK00} (0.85)& \cellcolor{red!20}\href{../works/Vilim09a.pdf}{Vilim09a} (0.87)\\
Euclid& \cellcolor{red!40}\href{../works/BeldiceanuCP08.pdf}{BeldiceanuCP08} (0.24)& \cellcolor{green!20}\href{../works/BeldiceanuP07.pdf}{BeldiceanuP07} (0.30)& \cellcolor{green!20}\href{../works/WolfS05.pdf}{WolfS05} (0.30)& \cellcolor{blue!20}\href{../works/ClautiauxJCM08.pdf}{ClautiauxJCM08} (0.32)& \cellcolor{blue!20}\href{../works/PoderB08.pdf}{PoderB08} (0.32)\\
Dot& \cellcolor{red!40}\href{../works/Malapert11.pdf}{Malapert11} (116.00)& \cellcolor{red!40}\href{../works/Schutt11.pdf}{Schutt11} (112.00)& \cellcolor{red!40}\href{../works/Godet21a.pdf}{Godet21a} (108.00)& \cellcolor{red!40}\href{../works/Baptiste02.pdf}{Baptiste02} (108.00)& \cellcolor{red!40}\href{../works/PapeB97.pdf}{PapeB97} (104.00)\\
Cosine& \cellcolor{red!40}\href{../works/BeldiceanuCP08.pdf}{BeldiceanuCP08} (0.83)& \cellcolor{red!40}\href{../works/ClautiauxJCM08.pdf}{ClautiauxJCM08} (0.72)& \cellcolor{red!40}\href{../works/BeldiceanuP07.pdf}{BeldiceanuP07} (0.72)& \cellcolor{red!40}\href{../works/WolfS05.pdf}{WolfS05} (0.71)& \cellcolor{red!40}\href{../works/EvenSH15.pdf}{EvenSH15} (0.68)\\
\index{BeldiceanuCP08}\href{../works/BeldiceanuCP08.pdf}{BeldiceanuCP08} R\&C& \cellcolor{red!40}\href{../works/BeldiceanuCDP11.pdf}{BeldiceanuCDP11} (0.53)& \cellcolor{red!40}\href{../works/Vilim09a.pdf}{Vilim09a} (0.78)& \cellcolor{red!40}\href{../works/LetortBC12.pdf}{LetortBC12} (0.83)& \cellcolor{red!40}\href{../works/Simonis95.pdf}{Simonis95} (0.83)& \cellcolor{red!40}\href{../works/SimonisC95.pdf}{SimonisC95} (0.84)\\
Euclid& \cellcolor{red!40}\href{../works/BeldiceanuP07.pdf}{BeldiceanuP07} (0.22)& \cellcolor{red!40}\href{../works/PoderB08.pdf}{PoderB08} (0.24)& \cellcolor{red!40}\href{../works/WolfS05.pdf}{WolfS05} (0.24)& \cellcolor{red!40}\href{../works/BeldiceanuCDP11.pdf}{BeldiceanuCDP11} (0.24)& \cellcolor{red!20}\href{../works/Caseau97.pdf}{Caseau97} (0.25)\\
Dot& \cellcolor{red!40}\href{../works/BeldiceanuCDP11.pdf}{BeldiceanuCDP11} (84.00)& \cellcolor{red!40}\href{../works/Malapert11.pdf}{Malapert11} (82.00)& \cellcolor{red!40}\href{../works/Schutt11.pdf}{Schutt11} (80.00)& \cellcolor{red!40}\href{../works/LaborieRSV18.pdf}{LaborieRSV18} (76.00)& \cellcolor{red!40}\href{../works/SchuttFSW11.pdf}{SchuttFSW11} (75.00)\\
Cosine& \cellcolor{red!40}\href{../works/BeldiceanuCDP11.pdf}{BeldiceanuCDP11} (0.83)& \cellcolor{red!40}\href{../works/BeldiceanuP07.pdf}{BeldiceanuP07} (0.77)& \cellcolor{red!40}\href{../works/WolfS05.pdf}{WolfS05} (0.74)& \cellcolor{red!40}\href{../works/SchuttWS05.pdf}{SchuttWS05} (0.73)& \cellcolor{red!40}\href{../works/PoderB08.pdf}{PoderB08} (0.73)\\
\index{BeldiceanuP07}\href{../works/BeldiceanuP07.pdf}{BeldiceanuP07} R\&C& \cellcolor{red!40}\href{../works/SchuttS16.pdf}{SchuttS16} (0.79)& \cellcolor{red!40}\href{../works/WolfS05a.pdf}{WolfS05a} (0.85)& \cellcolor{red!40}\href{../works/LombardiM09.pdf}{LombardiM09} (0.86)& \cellcolor{red!20}\href{../works/Wolf05.pdf}{Wolf05} (0.87)& \cellcolor{red!20}\href{../works/DavenportKRSH07.pdf}{DavenportKRSH07} (0.89)\\
Euclid& \cellcolor{red!40}\href{../works/PoderB08.pdf}{PoderB08} (0.13)& \cellcolor{red!40}\href{../works/WolfS05.pdf}{WolfS05} (0.14)& \cellcolor{red!40}\href{../works/SimonisH11.pdf}{SimonisH11} (0.20)& \cellcolor{red!40}\href{../works/Caseau97.pdf}{Caseau97} (0.21)& \cellcolor{red!40}\href{../works/BeniniBGM05a.pdf}{BeniniBGM05a} (0.21)\\
Dot& \cellcolor{red!40}\href{../works/Malapert11.pdf}{Malapert11} (72.00)& \cellcolor{red!40}\href{../works/Lombardi10.pdf}{Lombardi10} (71.00)& \cellcolor{red!40}\href{../works/FahimiOQ18.pdf}{FahimiOQ18} (70.00)& \cellcolor{red!40}\href{../works/Godet21a.pdf}{Godet21a} (70.00)& \cellcolor{red!40}\href{../works/Dejemeppe16.pdf}{Dejemeppe16} (70.00)\\
Cosine& \cellcolor{red!40}\href{../works/PoderB08.pdf}{PoderB08} (0.91)& \cellcolor{red!40}\href{../works/WolfS05.pdf}{WolfS05} (0.90)& \cellcolor{red!40}\href{../works/SimonisH11.pdf}{SimonisH11} (0.80)& \cellcolor{red!40}\href{../works/GayHS15.pdf}{GayHS15} (0.79)& \cellcolor{red!40}\href{../works/SimoninAHL15.pdf}{SimoninAHL15} (0.79)\\
\index{BelhadjiI98}\href{../works/BelhadjiI98.pdf}{BelhadjiI98} R\&C& \cellcolor{black!20}\href{../works/MintonJPL92.pdf}{MintonJPL92} (1.00)\\
Euclid& \cellcolor{yellow!20}\href{../works/KengY89.pdf}{KengY89} (0.28)& \cellcolor{green!20}\href{../works/Zhou96.pdf}{Zhou96} (0.30)& \cellcolor{green!20}\href{../works/TorresL00.pdf}{TorresL00} (0.31)& \cellcolor{green!20}\href{../works/Rit86.pdf}{Rit86} (0.31)& \cellcolor{green!20}\href{../works/Prosser89.pdf}{Prosser89} (0.31)\\
Dot& \cellcolor{red!40}\href{../works/Baptiste02.pdf}{Baptiste02} (123.00)& \cellcolor{red!40}\href{../works/BartakSR10.pdf}{BartakSR10} (121.00)& \cellcolor{red!40}\href{../works/Godet21a.pdf}{Godet21a} (118.00)& \cellcolor{red!40}\href{../works/Lombardi10.pdf}{Lombardi10} (114.00)& \cellcolor{red!40}\href{../works/ZarandiASC20.pdf}{ZarandiASC20} (109.00)\\
Cosine& \cellcolor{red!40}\href{../works/TorresL00.pdf}{TorresL00} (0.75)& \cellcolor{red!40}\href{../works/KengY89.pdf}{KengY89} (0.73)& \cellcolor{red!40}\href{../works/BartakSR08.pdf}{BartakSR08} (0.73)& \cellcolor{red!40}\href{../works/Zhou96.pdf}{Zhou96} (0.72)& \cellcolor{red!40}\href{../works/FoxS90.pdf}{FoxS90} (0.72)\\
\index{BenderWS21}\href{../works/BenderWS21.pdf}{BenderWS21} R\&C& \cellcolor{red!20}\href{../works/Polo-MejiaALB20.pdf}{Polo-MejiaALB20} (0.89)& \cellcolor{yellow!20}Ham20 (0.93)& \cellcolor{yellow!20}\href{../works/Ham20a.pdf}{Ham20a} (0.93)& \cellcolor{green!20}\href{../works/LunardiBLRV20.pdf}{LunardiBLRV20} (0.94)& \cellcolor{green!20}\href{../works/CauwelaertDS20.pdf}{CauwelaertDS20} (0.95)\\
Euclid& \cellcolor{red!20}\href{../works/LiessM08.pdf}{LiessM08} (0.25)& \cellcolor{red!20}\href{../works/BofillCSV17a.pdf}{BofillCSV17a} (0.26)& \cellcolor{red!20}\href{../works/BhatnagarKL19.pdf}{BhatnagarKL19} (0.26)& \cellcolor{yellow!20}\href{../works/BridiLBBM16.pdf}{BridiLBBM16} (0.27)& \cellcolor{yellow!20}\href{../works/AstrandJZ18.pdf}{AstrandJZ18} (0.27)\\
Dot& \cellcolor{red!40}\href{../works/ZarandiASC20.pdf}{ZarandiASC20} (103.00)& \cellcolor{red!40}\href{../works/Lombardi10.pdf}{Lombardi10} (100.00)& \cellcolor{red!40}\href{../works/Groleaz21.pdf}{Groleaz21} (100.00)& \cellcolor{red!40}\href{../works/Schutt11.pdf}{Schutt11} (100.00)& \cellcolor{red!40}\href{../works/Godet21a.pdf}{Godet21a} (99.00)\\
Cosine& \cellcolor{red!40}\href{../works/LiessM08.pdf}{LiessM08} (0.80)& \cellcolor{red!40}\href{../works/HeipckeCCS00.pdf}{HeipckeCCS00} (0.74)& \cellcolor{red!40}\href{../works/BofillCSV17a.pdf}{BofillCSV17a} (0.74)& \cellcolor{red!40}\href{../works/BeckW07.pdf}{BeckW07} (0.74)& \cellcolor{red!40}\href{../works/LombardiBM15.pdf}{LombardiBM15} (0.73)\\
\index{BenediktMH20}\href{../works/BenediktMH20.pdf}{BenediktMH20} R\&C& \cellcolor{red!40}\href{../works/HeinzNVH22.pdf}{HeinzNVH22} (0.81)& \cellcolor{red!20}\href{../works/YounespourAKE19.pdf}{YounespourAKE19} (0.88)& \cellcolor{yellow!20}\href{../works/HamP21.pdf}{HamP21} (0.91)& \cellcolor{yellow!20}\href{../works/HamPK21.pdf}{HamPK21} (0.92)& \cellcolor{green!20}\href{../works/BenediktSMVH18.pdf}{BenediktSMVH18} (0.93)\\
Euclid& \cellcolor{red!40}\href{../works/BenediktSMVH18.pdf}{BenediktSMVH18} (0.23)& \cellcolor{green!20}\href{../works/HebrardTW05.pdf}{HebrardTW05} (0.29)& \cellcolor{green!20}\href{../works/HeipckeCCS00.pdf}{HeipckeCCS00} (0.30)& \cellcolor{green!20}\href{../works/Hooker17.pdf}{Hooker17} (0.30)& \cellcolor{green!20}\href{../works/BogaerdtW19.pdf}{BogaerdtW19} (0.30)\\
Dot& \cellcolor{red!40}\href{../works/Groleaz21.pdf}{Groleaz21} (96.00)& \cellcolor{red!40}\href{../works/ZarandiASC20.pdf}{ZarandiASC20} (91.00)& \cellcolor{red!40}\href{../works/PrataAN23.pdf}{PrataAN23} (88.00)& \cellcolor{red!40}\href{../works/Lunardi20.pdf}{Lunardi20} (88.00)& \cellcolor{red!40}\href{../works/HamPK21.pdf}{HamPK21} (86.00)\\
Cosine& \cellcolor{red!40}\href{../works/BenediktSMVH18.pdf}{BenediktSMVH18} (0.78)& \cellcolor{red!40}\href{../works/HeipckeCCS00.pdf}{HeipckeCCS00} (0.68)& \cellcolor{red!40}\href{../works/PandeyS21a.pdf}{PandeyS21a} (0.66)& \cellcolor{red!40}\href{../works/HamPK21.pdf}{HamPK21} (0.66)& \cellcolor{red!40}\href{../works/BogaerdtW19.pdf}{BogaerdtW19} (0.65)\\
\index{BenediktSMVH18}\href{../works/BenediktSMVH18.pdf}{BenediktSMVH18} R\&C& \cellcolor{green!20}\href{../works/BenediktMH20.pdf}{BenediktMH20} (0.93)& \cellcolor{green!20}\href{../works/OddiPCC03.pdf}{OddiPCC03} (0.94)& \cellcolor{green!20}\href{../works/DejemeppeCS15.pdf}{DejemeppeCS15} (0.96)& \cellcolor{green!20}\href{../works/HamdiL13.pdf}{HamdiL13} (0.96)& \cellcolor{green!20}OddiPCC05 (0.96)\\
Euclid& \cellcolor{red!40}\href{../works/BenediktMH20.pdf}{BenediktMH20} (0.23)& \cellcolor{red!40}\href{../works/HebrardTW05.pdf}{HebrardTW05} (0.24)& \cellcolor{red!20}\href{../works/Sadykov04.pdf}{Sadykov04} (0.25)& \cellcolor{red!20}\href{../works/Limtanyakul07.pdf}{Limtanyakul07} (0.25)& \cellcolor{red!20}\href{../works/Hooker17.pdf}{Hooker17} (0.25)\\
Dot& \cellcolor{red!40}\href{../works/Groleaz21.pdf}{Groleaz21} (83.00)& \cellcolor{red!40}\href{../works/ZarandiASC20.pdf}{ZarandiASC20} (80.00)& \cellcolor{red!40}\href{../works/Lunardi20.pdf}{Lunardi20} (75.00)& \cellcolor{red!40}\href{../works/PrataAN23.pdf}{PrataAN23} (74.00)& \cellcolor{red!40}\href{../works/Baptiste02.pdf}{Baptiste02} (73.00)\\
Cosine& \cellcolor{red!40}\href{../works/BenediktMH20.pdf}{BenediktMH20} (0.78)& \cellcolor{red!40}\href{../works/SadykovW06.pdf}{SadykovW06} (0.72)& \cellcolor{red!40}\href{../works/Sadykov04.pdf}{Sadykov04} (0.70)& \cellcolor{red!40}\href{../works/CatusseCBL16.pdf}{CatusseCBL16} (0.70)& \cellcolor{red!40}\href{../works/EdisO11.pdf}{EdisO11} (0.69)\\
\index{BeniniBGM05}\href{../works/BeniniBGM05.pdf}{BeniniBGM05} R\&C& \cellcolor{red!40}\href{../works/Hooker05a.pdf}{Hooker05a} (0.62)& \cellcolor{red!40}\href{../works/Hooker05.pdf}{Hooker05} (0.70)& \cellcolor{red!40}\href{../works/BeniniBGM06.pdf}{BeniniBGM06} (0.71)& \cellcolor{red!40}\href{../works/Hooker04.pdf}{Hooker04} (0.72)& \cellcolor{red!40}\href{../works/BeniniLMR08.pdf}{BeniniLMR08} (0.73)\\
Euclid& \cellcolor{red!40}\href{../works/BeniniLMR08.pdf}{BeniniLMR08} (0.21)& \cellcolor{red!20}\href{../works/BeniniBGM06.pdf}{BeniniBGM06} (0.24)& \cellcolor{yellow!20}\href{../works/RuggieroBBMA09.pdf}{RuggieroBBMA09} (0.27)& \cellcolor{green!20}\href{../works/BeniniLMMR08.pdf}{BeniniLMMR08} (0.29)& \cellcolor{green!20}\href{../works/BeniniLMR11.pdf}{BeniniLMR11} (0.29)\\
Dot& \cellcolor{red!40}\href{../works/Lombardi10.pdf}{Lombardi10} (145.00)& \cellcolor{red!40}\href{../works/Groleaz21.pdf}{Groleaz21} (121.00)& \cellcolor{red!40}\href{../works/ZarandiASC20.pdf}{ZarandiASC20} (120.00)& \cellcolor{red!40}\href{../works/HarjunkoskiMBC14.pdf}{HarjunkoskiMBC14} (119.00)& \cellcolor{red!40}\href{../works/LombardiMRB10.pdf}{LombardiMRB10} (115.00)\\
Cosine& \cellcolor{red!40}\href{../works/BeniniLMR08.pdf}{BeniniLMR08} (0.88)& \cellcolor{red!40}\href{../works/BeniniBGM06.pdf}{BeniniBGM06} (0.83)& \cellcolor{red!40}\href{../works/RuggieroBBMA09.pdf}{RuggieroBBMA09} (0.81)& \cellcolor{red!40}\href{../works/BeniniLMR11.pdf}{BeniniLMR11} (0.78)& \cellcolor{red!40}\href{../works/LombardiMRB10.pdf}{LombardiMRB10} (0.77)\\
\index{BeniniBGM05a}\href{../works/BeniniBGM05a.pdf}{BeniniBGM05a} R\&C\\
Euclid& \cellcolor{red!40}\href{../works/QuSN06.pdf}{QuSN06} (0.18)& \cellcolor{red!40}\href{../works/AngelsmarkJ00.pdf}{AngelsmarkJ00} (0.19)& \cellcolor{red!40}\href{../works/LombardiM13.pdf}{LombardiM13} (0.19)& \cellcolor{red!40}\href{../works/FortinZDF05.pdf}{FortinZDF05} (0.20)& \cellcolor{red!40}\href{../works/BonfiettiM12.pdf}{BonfiettiM12} (0.20)\\
Dot& \cellcolor{red!40}\href{../works/Lombardi10.pdf}{Lombardi10} (67.00)& \cellcolor{red!40}\href{../works/LombardiMRB10.pdf}{LombardiMRB10} (63.00)& \cellcolor{red!40}\href{../works/LaborieRSV18.pdf}{LaborieRSV18} (61.00)& \cellcolor{red!40}\href{../works/BeniniBGM05.pdf}{BeniniBGM05} (61.00)& \cellcolor{red!40}\href{../works/Malapert11.pdf}{Malapert11} (59.00)\\
Cosine& \cellcolor{red!40}\href{../works/BeniniBGM06.pdf}{BeniniBGM06} (0.80)& \cellcolor{red!40}\href{../works/NishikawaSTT18.pdf}{NishikawaSTT18} (0.80)& \cellcolor{red!40}\href{../works/NishikawaSTT18a.pdf}{NishikawaSTT18a} (0.80)& \cellcolor{red!40}\href{../works/QuSN06.pdf}{QuSN06} (0.79)& \cellcolor{red!40}\href{../works/NishikawaSTT19.pdf}{NishikawaSTT19} (0.79)\\
\index{BeniniBGM06}\href{../works/BeniniBGM06.pdf}{BeniniBGM06} R\&C& \cellcolor{red!40}\href{../works/Hooker05a.pdf}{Hooker05a} (0.67)& \cellcolor{red!40}\href{../works/CambazardJ05.pdf}{CambazardJ05} (0.68)& \cellcolor{red!40}\href{../works/Hooker04.pdf}{Hooker04} (0.68)& \cellcolor{red!40}\href{../works/Hooker05.pdf}{Hooker05} (0.70)& \cellcolor{red!40}\href{../works/BeniniBGM05.pdf}{BeniniBGM05} (0.71)\\
Euclid& \cellcolor{red!40}\href{../works/BeniniBGM05a.pdf}{BeniniBGM05a} (0.24)& \cellcolor{red!20}\href{../works/BeniniBGM05.pdf}{BeniniBGM05} (0.24)& \cellcolor{red!20}\href{../works/BeniniLMR08.pdf}{BeniniLMR08} (0.25)& \cellcolor{red!20}\href{../works/BeniniLMMR08.pdf}{BeniniLMMR08} (0.25)& \cellcolor{red!20}\href{../works/LozanoCDS12.pdf}{LozanoCDS12} (0.25)\\
Dot& \cellcolor{red!40}\href{../works/Lombardi10.pdf}{Lombardi10} (114.00)& \cellcolor{red!40}\href{../works/LombardiMRB10.pdf}{LombardiMRB10} (106.00)& \cellcolor{red!40}\href{../works/LaborieRSV18.pdf}{LaborieRSV18} (99.00)& \cellcolor{red!40}\href{../works/LombardiM12.pdf}{LombardiM12} (99.00)& \cellcolor{red!40}\href{../works/RuggieroBBMA09.pdf}{RuggieroBBMA09} (99.00)\\
Cosine& \cellcolor{red!40}\href{../works/BeniniBGM05.pdf}{BeniniBGM05} (0.83)& \cellcolor{red!40}\href{../works/RuggieroBBMA09.pdf}{RuggieroBBMA09} (0.81)& \cellcolor{red!40}\href{../works/BeniniBGM05a.pdf}{BeniniBGM05a} (0.80)& \cellcolor{red!40}\href{../works/BeniniLMR08.pdf}{BeniniLMR08} (0.80)& \cellcolor{red!40}\href{../works/LombardiMRB10.pdf}{LombardiMRB10} (0.80)\\
\index{BeniniLMMR08}\href{../works/BeniniLMMR08.pdf}{BeniniLMMR08} R\&C& \cellcolor{red!40}\href{../works/BeniniLMR11.pdf}{BeniniLMR11} (0.62)& \cellcolor{red!40}\href{../works/BeniniLMR08.pdf}{BeniniLMR08} (0.67)& \cellcolor{red!40}\href{../works/CobanH11.pdf}{CobanH11} (0.71)& \cellcolor{red!40}\href{../works/BeniniBGM06.pdf}{BeniniBGM06} (0.72)& \cellcolor{red!40}\href{../works/CireCH13.pdf}{CireCH13} (0.73)\\
Euclid& \cellcolor{red!40}\href{../works/BeniniLMR11.pdf}{BeniniLMR11} (0.17)& \cellcolor{red!40}\href{../works/FortinZDF05.pdf}{FortinZDF05} (0.23)& \cellcolor{red!40}\href{../works/BeniniLMR08.pdf}{BeniniLMR08} (0.24)& \cellcolor{red!40}\href{../works/Hooker04.pdf}{Hooker04} (0.24)& \cellcolor{red!20}\href{../works/CireCH16.pdf}{CireCH16} (0.25)\\
Dot& \cellcolor{red!40}\href{../works/Lombardi10.pdf}{Lombardi10} (111.00)& \cellcolor{red!40}\href{../works/BeniniLMR11.pdf}{BeniniLMR11} (101.00)& \cellcolor{red!40}\href{../works/LaborieRSV18.pdf}{LaborieRSV18} (96.00)& \cellcolor{red!40}\href{../works/Groleaz21.pdf}{Groleaz21} (93.00)& \cellcolor{red!40}\href{../works/Hooker05.pdf}{Hooker05} (92.00)\\
Cosine& \cellcolor{red!40}\href{../works/BeniniLMR11.pdf}{BeniniLMR11} (0.92)& \cellcolor{red!40}\href{../works/Hooker04.pdf}{Hooker04} (0.80)& \cellcolor{red!40}\href{../works/BeniniLMR08.pdf}{BeniniLMR08} (0.80)& \cellcolor{red!40}\href{../works/Hooker05.pdf}{Hooker05} (0.79)& \cellcolor{red!40}\href{../works/BeniniBGM06.pdf}{BeniniBGM06} (0.79)\\
\index{BeniniLMR08}\href{../works/BeniniLMR08.pdf}{BeniniLMR08} R\&C& \cellcolor{red!40}\href{../works/BeniniLMMR08.pdf}{BeniniLMMR08} (0.67)& \cellcolor{red!40}\href{../works/BeniniLMR11.pdf}{BeniniLMR11} (0.72)& \cellcolor{red!40}\href{../works/BeniniBGM05.pdf}{BeniniBGM05} (0.73)& \cellcolor{red!40}\href{../works/BeniniBGM06.pdf}{BeniniBGM06} (0.77)& \cellcolor{red!40}\href{../works/LombardiMRB10.pdf}{LombardiMRB10} (0.80)\\
Euclid& \cellcolor{red!40}\href{../works/BeniniBGM05.pdf}{BeniniBGM05} (0.21)& \cellcolor{red!40}\href{../works/BeniniLMR11.pdf}{BeniniLMR11} (0.23)& \cellcolor{red!40}\href{../works/BeniniLMMR08.pdf}{BeniniLMMR08} (0.24)& \cellcolor{red!20}\href{../works/BeniniBGM05a.pdf}{BeniniBGM05a} (0.24)& \cellcolor{red!20}\href{../works/BeniniBGM06.pdf}{BeniniBGM06} (0.25)\\
Dot& \cellcolor{red!40}\href{../works/Lombardi10.pdf}{Lombardi10} (128.00)& \cellcolor{red!40}\href{../works/ZarandiASC20.pdf}{ZarandiASC20} (105.00)& \cellcolor{red!40}\href{../works/BeniniBGM05.pdf}{BeniniBGM05} (103.00)& \cellcolor{red!40}\href{../works/LombardiMRB10.pdf}{LombardiMRB10} (102.00)& \cellcolor{red!40}\href{../works/Beck99.pdf}{Beck99} (101.00)\\
Cosine& \cellcolor{red!40}\href{../works/BeniniBGM05.pdf}{BeniniBGM05} (0.88)& \cellcolor{red!40}\href{../works/BeniniLMR11.pdf}{BeniniLMR11} (0.85)& \cellcolor{red!40}\href{../works/BeniniLMMR08.pdf}{BeniniLMMR08} (0.80)& \cellcolor{red!40}\href{../works/BeniniBGM06.pdf}{BeniniBGM06} (0.80)& \cellcolor{red!40}\href{../works/BeniniBGM05a.pdf}{BeniniBGM05a} (0.78)\\
\index{BeniniLMR11}\href{../works/BeniniLMR11.pdf}{BeniniLMR11} R\&C& \cellcolor{red!40}\href{../works/BeniniLMMR08.pdf}{BeniniLMMR08} (0.62)& \cellcolor{red!40}\href{../works/BeniniLMR08.pdf}{BeniniLMR08} (0.72)& \cellcolor{red!40}\href{../works/BeniniBGM06.pdf}{BeniniBGM06} (0.77)& \cellcolor{red!40}\href{../works/CambazardHDJT04.pdf}{CambazardHDJT04} (0.80)& \cellcolor{red!40}\href{../works/CireCH13.pdf}{CireCH13} (0.80)\\
Euclid& \cellcolor{red!40}\href{../works/BeniniLMMR08.pdf}{BeniniLMMR08} (0.17)& \cellcolor{red!40}\href{../works/BeniniLMR08.pdf}{BeniniLMR08} (0.23)& \cellcolor{yellow!20}\href{../works/Hooker04.pdf}{Hooker04} (0.27)& \cellcolor{yellow!20}\href{../works/BeniniBGM06.pdf}{BeniniBGM06} (0.27)& \cellcolor{green!20}\href{../works/EmeretlisTAV17.pdf}{EmeretlisTAV17} (0.29)\\
Dot& \cellcolor{red!40}\href{../works/Lombardi10.pdf}{Lombardi10} (154.00)& \cellcolor{red!40}\href{../works/Groleaz21.pdf}{Groleaz21} (122.00)& \cellcolor{red!40}\href{../works/Baptiste02.pdf}{Baptiste02} (121.00)& \cellcolor{red!40}\href{../works/LombardiM12.pdf}{LombardiM12} (120.00)& \cellcolor{red!40}\href{../works/Beck99.pdf}{Beck99} (120.00)\\
Cosine& \cellcolor{red!40}\href{../works/BeniniLMMR08.pdf}{BeniniLMMR08} (0.92)& \cellcolor{red!40}\href{../works/BeniniLMR08.pdf}{BeniniLMR08} (0.85)& \cellcolor{red!40}\href{../works/Hooker04.pdf}{Hooker04} (0.80)& \cellcolor{red!40}\href{../works/Hooker05.pdf}{Hooker05} (0.78)& \cellcolor{red!40}\href{../works/BeniniBGM06.pdf}{BeniniBGM06} (0.78)\\
\index{BenoistGR02}\href{../works/BenoistGR02.pdf}{BenoistGR02} R\&C& \cellcolor{red!40}\href{../works/Hooker04.pdf}{Hooker04} (0.77)& \cellcolor{red!40}\href{../works/EreminW01.pdf}{EreminW01} (0.79)& \cellcolor{red!40}\href{../works/CambazardHDJT04.pdf}{CambazardHDJT04} (0.83)& \cellcolor{red!40}\href{../works/Hooker05a.pdf}{Hooker05a} (0.83)& \cellcolor{red!40}\href{../works/Hooker05.pdf}{Hooker05} (0.84)\\
Euclid& \cellcolor{red!40}\href{../works/CambazardJ05.pdf}{CambazardJ05} (0.23)& \cellcolor{red!40}\href{../works/Baptiste09.pdf}{Baptiste09} (0.24)& \cellcolor{red!20}\href{../works/ZibranR11.pdf}{ZibranR11} (0.24)& \cellcolor{red!20}\href{../works/Hooker05b.pdf}{Hooker05b} (0.25)& \cellcolor{red!20}\href{../works/HebrardALLCMR22.pdf}{HebrardALLCMR22} (0.25)\\
Dot& \cellcolor{red!40}\href{../works/Froger16.pdf}{Froger16} (63.00)& \cellcolor{red!40}\href{../works/Fahimi16.pdf}{Fahimi16} (59.00)& \cellcolor{red!40}\href{../works/Godet21a.pdf}{Godet21a} (58.00)& \cellcolor{red!40}\href{../works/HookerH17.pdf}{HookerH17} (58.00)& \cellcolor{red!40}\href{../works/Lombardi10.pdf}{Lombardi10} (56.00)\\
Cosine& \cellcolor{red!40}\href{../works/HookerOTK00.pdf}{HookerOTK00} (0.63)& \cellcolor{red!40}\href{../works/CorreaLR07.pdf}{CorreaLR07} (0.62)& \cellcolor{red!40}\href{../works/HladikCDJ08.pdf}{HladikCDJ08} (0.62)& \cellcolor{red!40}\href{../works/CambazardJ05.pdf}{CambazardJ05} (0.62)& \cellcolor{red!40}\href{../works/Hooker05b.pdf}{Hooker05b} (0.61)\\
\index{BensanaLV99}\href{../works/BensanaLV99.pdf}{BensanaLV99} R\&C& \cellcolor{green!20}\href{../works/VerfaillieL01.pdf}{VerfaillieL01} (0.94)& \cellcolor{blue!20}\href{../works/OddiPCC03.pdf}{OddiPCC03} (0.97)& \cellcolor{black!20}\href{../works/Hooker07.pdf}{Hooker07} (0.99)& \cellcolor{black!20}\href{../works/Beck10.pdf}{Beck10} (0.99)& \cellcolor{black!20}\href{../works/PerronSF04.pdf}{PerronSF04} (0.99)\\
Euclid& \cellcolor{red!40}\href{../works/BandaSC11.pdf}{BandaSC11} (0.23)& \cellcolor{red!40}\href{../works/SmithBHW96.pdf}{SmithBHW96} (0.24)& \cellcolor{red!40}\href{../works/Davis87.pdf}{Davis87} (0.24)& \cellcolor{red!40}\href{../works/FrostD98.pdf}{FrostD98} (0.24)& \cellcolor{red!40}\href{../works/FeldmanG89.pdf}{FeldmanG89} (0.24)\\
Dot& \cellcolor{red!40}\href{../works/Godet21a.pdf}{Godet21a} (52.00)& \cellcolor{red!40}\href{../works/Lemos21.pdf}{Lemos21} (44.00)& \cellcolor{red!40}\href{../works/Dejemeppe16.pdf}{Dejemeppe16} (42.00)& \cellcolor{red!40}\href{../works/Siala15a.pdf}{Siala15a} (42.00)& \cellcolor{red!40}\href{../works/LaborieRSV18.pdf}{LaborieRSV18} (41.00)\\
Cosine& \cellcolor{red!40}\href{../works/VerfaillieL01.pdf}{VerfaillieL01} (0.61)& \cellcolor{red!40}\href{../works/LiuLH19.pdf}{LiuLH19} (0.60)& \cellcolor{red!40}\href{../works/BandaSC11.pdf}{BandaSC11} (0.59)& \cellcolor{red!40}\href{../works/SmithBHW96.pdf}{SmithBHW96} (0.54)& \cellcolor{red!40}\href{../works/HenzMT04.pdf}{HenzMT04} (0.51)\\
\index{BertholdHLMS10}\href{../works/BertholdHLMS10.pdf}{BertholdHLMS10} R\&C& \cellcolor{red!40}\href{../works/HeinzS11.pdf}{HeinzS11} (0.65)& \cellcolor{red!40}\href{../works/SchuttW10.pdf}{SchuttW10} (0.83)& \cellcolor{red!40}\href{../works/SchuttFSW11.pdf}{SchuttFSW11} (0.84)& \cellcolor{red!40}\href{../works/SchuttFSW13.pdf}{SchuttFSW13} (0.84)& \cellcolor{red!40}\href{../works/HeinzSB13.pdf}{HeinzSB13} (0.85)\\
Euclid& \cellcolor{red!40}\href{../works/HookerY02.pdf}{HookerY02} (0.23)& \cellcolor{red!40}\href{../works/HeinzS11.pdf}{HeinzS11} (0.23)& \cellcolor{red!20}\href{../works/Caballero23.pdf}{Caballero23} (0.25)& \cellcolor{red!20}\href{../works/BonfiettiM12.pdf}{BonfiettiM12} (0.26)& \cellcolor{red!20}\href{../works/LombardiM13.pdf}{LombardiM13} (0.26)\\
Dot& \cellcolor{red!40}\href{../works/Schutt11.pdf}{Schutt11} (88.00)& \cellcolor{red!40}\href{../works/HeinzSB13.pdf}{HeinzSB13} (87.00)& \cellcolor{red!40}\href{../works/Groleaz21.pdf}{Groleaz21} (81.00)& \cellcolor{red!40}\href{../works/Godet21a.pdf}{Godet21a} (79.00)& \cellcolor{red!40}\href{../works/Lombardi10.pdf}{Lombardi10} (79.00)\\
Cosine& \cellcolor{red!40}\href{../works/HeinzS11.pdf}{HeinzS11} (0.79)& \cellcolor{red!40}\href{../works/SchnellH17.pdf}{SchnellH17} (0.77)& \cellcolor{red!40}\href{../works/HeinzSB13.pdf}{HeinzSB13} (0.77)& \cellcolor{red!40}\href{../works/HookerY02.pdf}{HookerY02} (0.74)& \cellcolor{red!40}\href{../works/ArkhipovBL19.pdf}{ArkhipovBL19} (0.73)\\
\index{BessiereHMQW14}\href{../works/BessiereHMQW14.pdf}{BessiereHMQW14} R\&C& \cellcolor{red!20}\href{../works/HoundjiSWD14.pdf}{HoundjiSWD14} (0.90)& \cellcolor{red!20}\href{../works/PesantRR15.pdf}{PesantRR15} (0.90)& \cellcolor{yellow!20}\href{../works/PerronSF04.pdf}{PerronSF04} (0.91)& \cellcolor{green!20}\href{../works/GayHLS15.pdf}{GayHLS15} (0.94)& \cellcolor{green!20}\href{../works/HoundjiSW19.pdf}{HoundjiSW19} (0.95)\\
Euclid& \cellcolor{red!20}\href{../works/ErtlK91.pdf}{ErtlK91} (0.25)& \cellcolor{red!20}\href{../works/RoweJCA96.pdf}{RoweJCA96} (0.26)& \cellcolor{red!20}\href{../works/Davis87.pdf}{Davis87} (0.26)& \cellcolor{red!20}\href{../works/LudwigKRBMS14.pdf}{LudwigKRBMS14} (0.26)& \cellcolor{yellow!20}\href{../works/LiuJ06.pdf}{LiuJ06} (0.27)\\
Dot& \cellcolor{red!40}\href{../works/Dejemeppe16.pdf}{Dejemeppe16} (80.00)& \cellcolor{red!40}\href{../works/Godet21a.pdf}{Godet21a} (78.00)& \cellcolor{red!40}\href{../works/Malapert11.pdf}{Malapert11} (75.00)& \cellcolor{red!40}\href{../works/Siala15a.pdf}{Siala15a} (73.00)& \cellcolor{red!40}\href{../works/Astrand21.pdf}{Astrand21} (73.00)\\
Cosine& \cellcolor{red!40}\href{../works/ErtlK91.pdf}{ErtlK91} (0.68)& \cellcolor{red!40}\href{../works/RoweJCA96.pdf}{RoweJCA96} (0.66)& \cellcolor{red!40}\href{../works/Malik08.pdf}{Malik08} (0.66)& \cellcolor{red!40}\href{../works/BockmayrP06.pdf}{BockmayrP06} (0.66)& \cellcolor{red!40}\href{../works/KengY89.pdf}{KengY89} (0.66)\\
\index{BhatnagarKL19}\href{../works/BhatnagarKL19.pdf}{BhatnagarKL19} R\&C& \cellcolor{red!40}\href{../works/FrimodigS19.pdf}{FrimodigS19} (0.75)& \cellcolor{red!40}\href{../works/GeibingerMM19.pdf}{GeibingerMM19} (0.86)\\
Euclid& \cellcolor{red!40}\href{../works/LombardiM13.pdf}{LombardiM13} (0.18)& \cellcolor{red!40}\href{../works/AgussurjaKL18.pdf}{AgussurjaKL18} (0.20)& \cellcolor{red!40}\href{../works/BofillCSV17a.pdf}{BofillCSV17a} (0.21)& \cellcolor{red!40}\href{../works/BonfiettiM12.pdf}{BonfiettiM12} (0.21)& \cellcolor{red!40}\href{../works/OddiRC10.pdf}{OddiRC10} (0.22)\\
Dot& \cellcolor{red!40}\href{../works/ZarandiASC20.pdf}{ZarandiASC20} (88.00)& \cellcolor{red!40}\href{../works/Lombardi10.pdf}{Lombardi10} (86.00)& \cellcolor{red!40}\href{../works/LombardiM12.pdf}{LombardiM12} (83.00)& \cellcolor{red!40}\href{../works/Dejemeppe16.pdf}{Dejemeppe16} (83.00)& \cellcolor{red!40}\href{../works/Baptiste02.pdf}{Baptiste02} (83.00)\\
Cosine& \cellcolor{red!40}\href{../works/AgussurjaKL18.pdf}{AgussurjaKL18} (0.83)& \cellcolor{red!40}\href{../works/LombardiM13.pdf}{LombardiM13} (0.83)& \cellcolor{red!40}\href{../works/BofillCSV17a.pdf}{BofillCSV17a} (0.81)& \cellcolor{red!40}\href{../works/CampeauG22.pdf}{CampeauG22} (0.80)& \cellcolor{red!40}\href{../works/LombardiM10.pdf}{LombardiM10} (0.79)\\
\index{BidotVLB07}\href{../works/BidotVLB07.pdf}{BidotVLB07} R\&C\\
Euclid& \cellcolor{red!40}\href{../works/OddiS97.pdf}{OddiS97} (0.24)& \cellcolor{red!40}\href{../works/BeckW05.pdf}{BeckW05} (0.24)& \cellcolor{red!20}\href{../works/FortinZDF05.pdf}{FortinZDF05} (0.25)& \cellcolor{red!20}\href{../works/Muscettola94.pdf}{Muscettola94} (0.25)& \cellcolor{red!20}\href{../works/DoomsH08.pdf}{DoomsH08} (0.25)\\
Dot& \cellcolor{red!40}\href{../works/ZarandiASC20.pdf}{ZarandiASC20} (100.00)& \cellcolor{red!40}\href{../works/Astrand21.pdf}{Astrand21} (98.00)& \cellcolor{red!40}\href{../works/BidotVLB09.pdf}{BidotVLB09} (94.00)& \cellcolor{red!40}\href{../works/Lombardi10.pdf}{Lombardi10} (90.00)& \cellcolor{red!40}\href{../works/Groleaz21.pdf}{Groleaz21} (90.00)\\
Cosine& \cellcolor{red!40}\href{../works/BidotVLB09.pdf}{BidotVLB09} (0.79)& \cellcolor{red!40}\href{../works/BeckPS03.pdf}{BeckPS03} (0.76)& \cellcolor{red!40}\href{../works/Muscettola94.pdf}{Muscettola94} (0.75)& \cellcolor{red!40}\href{../works/OddiS97.pdf}{OddiS97} (0.75)& \cellcolor{red!40}\href{../works/BeckW05.pdf}{BeckW05} (0.74)\\
\index{BidotVLB09}\href{../works/BidotVLB09.pdf}{BidotVLB09} R\&C& \cellcolor{yellow!20}\href{../works/Kumar03.pdf}{Kumar03} (0.91)& \cellcolor{yellow!20}CestaOPS14 (0.92)& \cellcolor{yellow!20}\href{../works/LombardiM09.pdf}{LombardiM09} (0.92)& \cellcolor{yellow!20}\href{../works/SourdN00.pdf}{SourdN00} (0.93)& \cellcolor{yellow!20}\href{../works/Wolf03.pdf}{Wolf03} (0.93)\\
Euclid& \cellcolor{green!20}\href{../works/BeckPS03.pdf}{BeckPS03} (0.30)& \cellcolor{blue!20}\href{../works/BidotVLB07.pdf}{BidotVLB07} (0.32)& \cellcolor{black!20}\href{../works/BeckDDF98.pdf}{BeckDDF98} (0.34)& \cellcolor{black!20}\href{../works/BeckW07.pdf}{BeckW07} (0.34)& \cellcolor{black!20}\href{../works/NovasH10.pdf}{NovasH10} (0.35)\\
Dot& \cellcolor{red!40}\href{../works/ZarandiASC20.pdf}{ZarandiASC20} (186.00)& \cellcolor{red!40}\href{../works/Groleaz21.pdf}{Groleaz21} (165.00)& \cellcolor{red!40}\href{../works/LaborieRSV18.pdf}{LaborieRSV18} (164.00)& \cellcolor{red!40}\href{../works/Astrand21.pdf}{Astrand21} (163.00)& \cellcolor{red!40}\href{../works/Baptiste02.pdf}{Baptiste02} (162.00)\\
Cosine& \cellcolor{red!40}\href{../works/BeckPS03.pdf}{BeckPS03} (0.81)& \cellcolor{red!40}\href{../works/BeckDDF98.pdf}{BeckDDF98} (0.80)& \cellcolor{red!40}\href{../works/BidotVLB07.pdf}{BidotVLB07} (0.79)& \cellcolor{red!40}\href{../works/BeckW07.pdf}{BeckW07} (0.76)& \cellcolor{red!40}\href{../works/NovasH10.pdf}{NovasH10} (0.75)\\
\index{BillautHL12}\href{../works/BillautHL12.pdf}{BillautHL12} R\&C& \cellcolor{yellow!20}\href{../works/GrimesH11.pdf}{GrimesH11} (0.92)& \cellcolor{yellow!20}\href{../works/GrimesH10.pdf}{GrimesH10} (0.92)& \cellcolor{yellow!20}BriandHHL08 (0.93)& \cellcolor{green!20}\href{../works/GrimesH15.pdf}{GrimesH15} (0.93)& \cellcolor{green!20}\href{../works/SialaAH15.pdf}{SialaAH15} (0.94)\\
Euclid& \cellcolor{yellow!20}\href{../works/JuvinHL23.pdf}{JuvinHL23} (0.27)& \cellcolor{yellow!20}\href{../works/DilkinaDH05.pdf}{DilkinaDH05} (0.28)& \cellcolor{green!20}\href{../works/ArtiguesBF04.pdf}{ArtiguesBF04} (0.29)& \cellcolor{green!20}\href{../works/ParkUJR19.pdf}{ParkUJR19} (0.29)& \cellcolor{green!20}\href{../works/Colombani96.pdf}{Colombani96} (0.29)\\
Dot& \cellcolor{red!40}\href{../works/Groleaz21.pdf}{Groleaz21} (128.00)& \cellcolor{red!40}\href{../works/Lunardi20.pdf}{Lunardi20} (123.00)& \cellcolor{red!40}\href{../works/Malapert11.pdf}{Malapert11} (121.00)& \cellcolor{red!40}\href{../works/ZarandiASC20.pdf}{ZarandiASC20} (117.00)& \cellcolor{red!40}\href{../works/Baptiste02.pdf}{Baptiste02} (115.00)\\
Cosine& \cellcolor{red!40}\href{../works/JuvinHL23.pdf}{JuvinHL23} (0.76)& \cellcolor{red!40}\href{../works/ArtiguesBF04.pdf}{ArtiguesBF04} (0.76)& \cellcolor{red!40}\href{../works/ParkUJR19.pdf}{ParkUJR19} (0.75)& \cellcolor{red!40}\href{../works/Mehdizadeh-Somarin23.pdf}{Mehdizadeh-Somarin23} (0.74)& \cellcolor{red!40}\href{../works/GuyonLPR12.pdf}{GuyonLPR12} (0.74)\\
\index{Bit-Monnot23}\href{../works/Bit-Monnot23.pdf}{Bit-Monnot23} R\&C\\
Euclid& \cellcolor{green!20}\href{../works/SialaAH15.pdf}{SialaAH15} (0.29)& \cellcolor{green!20}\href{../works/GrimesHM09.pdf}{GrimesHM09} (0.30)& \cellcolor{blue!20}\href{../works/Vilim05.pdf}{Vilim05} (0.33)& \cellcolor{blue!20}\href{../works/Beck07.pdf}{Beck07} (0.33)& \cellcolor{blue!20}\href{../works/MalapertCGJLR13.pdf}{MalapertCGJLR13} (0.34)\\
Dot& \cellcolor{red!40}\href{../works/Siala15a.pdf}{Siala15a} (165.00)& \cellcolor{red!40}\href{../works/Groleaz21.pdf}{Groleaz21} (158.00)& \cellcolor{red!40}\href{../works/Godet21a.pdf}{Godet21a} (150.00)& \cellcolor{red!40}\href{../works/GrimesH15.pdf}{GrimesH15} (149.00)& \cellcolor{red!40}\href{../works/Dejemeppe16.pdf}{Dejemeppe16} (146.00)\\
Cosine& \cellcolor{red!40}\href{../works/SialaAH15.pdf}{SialaAH15} (0.81)& \cellcolor{red!40}\href{../works/GrimesHM09.pdf}{GrimesHM09} (0.79)& \cellcolor{red!40}\href{../works/MalapertCGJLR12.pdf}{MalapertCGJLR12} (0.75)& \cellcolor{red!40}\href{../works/SchuttFS13.pdf}{SchuttFS13} (0.74)& \cellcolor{red!40}\href{../works/Beck07.pdf}{Beck07} (0.74)\\
\index{BlazewiczDP96}\href{../works/BlazewiczDP96.pdf}{BlazewiczDP96} R\&C& \cellcolor{red!40}\href{../works/JainM99.pdf}{JainM99} (0.66)& \cellcolor{red!40}DomdorfPH03 (0.80)& \cellcolor{red!20}DorndorfPH99 (0.88)& \cellcolor{red!20}\href{../works/ColT22.pdf}{ColT22} (0.89)& \cellcolor{red!20}\href{../works/Dorndorf2000.pdf}{Dorndorf2000} (0.90)\\
Euclid& \cellcolor{black!20}\href{../works/JainM99.pdf}{JainM99} (0.37)& \href{../works/SourdN00.pdf}{SourdN00} (0.40)& \href{../works/MenciaSV12.pdf}{MenciaSV12} (0.40)& \href{../works/MenciaSV13.pdf}{MenciaSV13} (0.40)& \href{../works/BeckF98.pdf}{BeckF98} (0.41)\\
Dot& \cellcolor{red!40}\href{../works/ZarandiASC20.pdf}{ZarandiASC20} (235.00)& \cellcolor{red!40}\href{../works/Groleaz21.pdf}{Groleaz21} (224.00)& \cellcolor{red!40}\href{../works/Baptiste02.pdf}{Baptiste02} (222.00)& \cellcolor{red!40}\href{../works/Dejemeppe16.pdf}{Dejemeppe16} (209.00)& \cellcolor{red!40}\href{../works/Astrand21.pdf}{Astrand21} (194.00)\\
Cosine& \cellcolor{red!40}\href{../works/JainM99.pdf}{JainM99} (0.80)& \cellcolor{red!40}\href{../works/MenciaSV12.pdf}{MenciaSV12} (0.74)& \cellcolor{red!40}\href{../works/SourdN00.pdf}{SourdN00} (0.74)& \cellcolor{red!40}\href{../works/BartakSR10.pdf}{BartakSR10} (0.73)& \cellcolor{red!40}\href{../works/MenciaSV13.pdf}{MenciaSV13} (0.73)\\
\index{BlazewiczEP19}BlazewiczEP19 R\&C& \cellcolor{black!20}BaptisteLPN06 (0.99)& \cellcolor{black!20}\href{../works/LaborieRSV18.pdf}{LaborieRSV18} (0.99)& \cellcolor{black!20}\href{../works/HarjunkoskiMBC14.pdf}{HarjunkoskiMBC14} (1.00)\\
Euclid\\
Dot\\
Cosine\\
\index{BlomBPS14}\href{../works/BlomBPS14.pdf}{BlomBPS14} R\&C& \cellcolor{red!40}\href{../works/BlomPS16.pdf}{BlomPS16} (0.54)& \cellcolor{yellow!20}\href{../works/AstrandJZ20.pdf}{AstrandJZ20} (0.93)& \cellcolor{green!20}\href{../works/CampeauG22.pdf}{CampeauG22} (0.96)& \cellcolor{green!20}\href{../works/AstrandJZ18.pdf}{AstrandJZ18} (0.96)& \cellcolor{blue!20}HillBCGN22 (0.98)\\
Euclid& \cellcolor{red!40}\href{../works/BlomPS16.pdf}{BlomPS16} (0.22)& \cellcolor{red!20}\href{../works/LudwigKRBMS14.pdf}{LudwigKRBMS14} (0.26)& \cellcolor{yellow!20}\href{../works/LipovetzkyBPS14.pdf}{LipovetzkyBPS14} (0.27)& \cellcolor{yellow!20}\href{../works/FukunagaHFAMN02.pdf}{FukunagaHFAMN02} (0.27)& \cellcolor{yellow!20}\href{../works/Rit86.pdf}{Rit86} (0.27)\\
Dot& \cellcolor{red!40}\href{../works/HarjunkoskiMBC14.pdf}{HarjunkoskiMBC14} (72.00)& \cellcolor{red!40}\href{../works/LaborieRSV18.pdf}{LaborieRSV18} (63.00)& \cellcolor{red!40}\href{../works/Astrand21.pdf}{Astrand21} (61.00)& \cellcolor{red!40}\href{../works/SacramentoSP20.pdf}{SacramentoSP20} (60.00)& \cellcolor{red!40}\href{../works/NaderiRR23.pdf}{NaderiRR23} (60.00)\\
Cosine& \cellcolor{red!40}\href{../works/BlomPS16.pdf}{BlomPS16} (0.76)& \cellcolor{red!40}\href{../works/LipovetzkyBPS14.pdf}{LipovetzkyBPS14} (0.71)& \cellcolor{red!40}\href{../works/LudwigKRBMS14.pdf}{LudwigKRBMS14} (0.63)& \cellcolor{red!40}\href{../works/NishikawaSTT18.pdf}{NishikawaSTT18} (0.62)& \cellcolor{red!40}\href{../works/HoeveGSL07.pdf}{HoeveGSL07} (0.61)\\
\index{BlomPS16}\href{../works/BlomPS16.pdf}{BlomPS16} R\&C& \cellcolor{red!40}\href{../works/BlomBPS14.pdf}{BlomBPS14} (0.54)& \cellcolor{green!20}HillBCGN22 (0.96)& \cellcolor{green!20}\href{../works/AstrandJZ20.pdf}{AstrandJZ20} (0.96)& \cellcolor{blue!20}\href{../works/CampeauG22.pdf}{CampeauG22} (0.97)& \cellcolor{blue!20}\href{../works/AstrandJZ18.pdf}{AstrandJZ18} (0.97)\\
Euclid& \cellcolor{red!40}\href{../works/BlomBPS14.pdf}{BlomBPS14} (0.22)& \cellcolor{red!40}\href{../works/BeniniBGM05a.pdf}{BeniniBGM05a} (0.24)& \cellcolor{red!20}\href{../works/Rit86.pdf}{Rit86} (0.24)& \cellcolor{red!20}\href{../works/FukunagaHFAMN02.pdf}{FukunagaHFAMN02} (0.24)& \cellcolor{red!20}\href{../works/LudwigKRBMS14.pdf}{LudwigKRBMS14} (0.25)\\
Dot& \cellcolor{red!40}\href{../works/HarjunkoskiMBC14.pdf}{HarjunkoskiMBC14} (78.00)& \cellcolor{red!40}\href{../works/ZarandiASC20.pdf}{ZarandiASC20} (71.00)& \cellcolor{red!40}\href{../works/Beck99.pdf}{Beck99} (68.00)& \cellcolor{red!40}\href{../works/Lombardi10.pdf}{Lombardi10} (67.00)& \cellcolor{red!40}\href{../works/LaborieRSV18.pdf}{LaborieRSV18} (64.00)\\
Cosine& \cellcolor{red!40}\href{../works/BlomBPS14.pdf}{BlomBPS14} (0.76)& \cellcolor{red!40}\href{../works/BeniniBGM05a.pdf}{BeniniBGM05a} (0.67)& \cellcolor{red!40}\href{../works/NishikawaSTT19.pdf}{NishikawaSTT19} (0.66)& \cellcolor{red!40}\href{../works/BridiLBBM16.pdf}{BridiLBBM16} (0.65)& \cellcolor{red!40}\href{../works/HoeveGSL07.pdf}{HoeveGSL07} (0.64)\\
\index{BocewiczBB09}\href{../works/BocewiczBB09.pdf}{BocewiczBB09} R\&C\\
Euclid& \cellcolor{red!20}\href{../works/Salido10.pdf}{Salido10} (0.25)& \cellcolor{yellow!20}\href{../works/LudwigKRBMS14.pdf}{LudwigKRBMS14} (0.27)& \cellcolor{yellow!20}\href{../works/WolinskiKG04.pdf}{WolinskiKG04} (0.27)& \cellcolor{yellow!20}\href{../works/ValleMGT03.pdf}{ValleMGT03} (0.27)& \cellcolor{yellow!20}\href{../works/KengY89.pdf}{KengY89} (0.27)\\
Dot& \cellcolor{red!40}\href{../works/ZarandiASC20.pdf}{ZarandiASC20} (107.00)& \cellcolor{red!40}\href{../works/Groleaz21.pdf}{Groleaz21} (95.00)& \cellcolor{red!40}\href{../works/Astrand21.pdf}{Astrand21} (95.00)& \cellcolor{red!40}\href{../works/Lombardi10.pdf}{Lombardi10} (94.00)& \cellcolor{red!40}\href{../works/Malapert11.pdf}{Malapert11} (94.00)\\
Cosine& \cellcolor{red!40}\href{../works/Salido10.pdf}{Salido10} (0.75)& \cellcolor{red!40}\href{../works/ValleMGT03.pdf}{ValleMGT03} (0.71)& \cellcolor{red!40}\href{../works/KorbaaYG99.pdf}{KorbaaYG99} (0.70)& \cellcolor{red!40}\href{../works/KengY89.pdf}{KengY89} (0.69)& \cellcolor{red!40}\href{../works/WolinskiKG04.pdf}{WolinskiKG04} (0.69)\\
\index{BockmayrK98}BockmayrK98 R\&C& \cellcolor{red!40}\href{../works/RodosekWH99.pdf}{RodosekWH99} (0.80)& \cellcolor{red!40}\href{../works/HookerO99.pdf}{HookerO99} (0.82)& \cellcolor{red!20}\href{../works/JainG01.pdf}{JainG01} (0.88)& \cellcolor{red!20}\href{../works/EreminW01.pdf}{EreminW01} (0.89)& \cellcolor{red!20}\href{../works/Simonis99.pdf}{Simonis99} (0.89)\\
Euclid\\
Dot\\
Cosine\\
\index{BockmayrP06}\href{../works/BockmayrP06.pdf}{BockmayrP06} R\&C& \cellcolor{red!40}\href{../works/HookerY02.pdf}{HookerY02} (0.79)& \cellcolor{red!40}\href{../works/SadykovW06.pdf}{SadykovW06} (0.82)& \cellcolor{red!40}\href{../works/RoePS05.pdf}{RoePS05} (0.84)& \cellcolor{red!20}\href{../works/CireCH13.pdf}{CireCH13} (0.87)& \cellcolor{red!20}\href{../works/CambazardJ05.pdf}{CambazardJ05} (0.87)\\
Euclid& \cellcolor{red!40}\href{../works/WolfS05.pdf}{WolfS05} (0.20)& \cellcolor{red!40}\href{../works/Limtanyakul07.pdf}{Limtanyakul07} (0.21)& \cellcolor{red!40}\href{../works/MurphyMB15.pdf}{MurphyMB15} (0.22)& \cellcolor{red!40}\href{../works/PoderBS04.pdf}{PoderBS04} (0.22)& \cellcolor{red!40}\href{../works/BeldiceanuP07.pdf}{BeldiceanuP07} (0.22)\\
Dot& \cellcolor{red!40}\href{../works/Dejemeppe16.pdf}{Dejemeppe16} (84.00)& \cellcolor{red!40}\href{../works/Simonis07.pdf}{Simonis07} (82.00)& \cellcolor{red!40}\href{../works/Lombardi10.pdf}{Lombardi10} (81.00)& \cellcolor{red!40}\href{../works/Groleaz21.pdf}{Groleaz21} (81.00)& \cellcolor{red!40}\href{../works/Baptiste02.pdf}{Baptiste02} (81.00)\\
Cosine& \cellcolor{red!40}\href{../works/WolfS05.pdf}{WolfS05} (0.81)& \cellcolor{red!40}\href{../works/PoderBS04.pdf}{PoderBS04} (0.79)& \cellcolor{red!40}\href{../works/Colombani96.pdf}{Colombani96} (0.79)& \cellcolor{red!40}\href{../works/Limtanyakul07.pdf}{Limtanyakul07} (0.77)& \cellcolor{red!40}\href{../works/MurphyMB15.pdf}{MurphyMB15} (0.77)\\
\index{BofillCGGPSV23}\href{../works/BofillCGGPSV23.pdf}{BofillCGGPSV23} R\&C\\
Euclid& \cellcolor{red!40}\href{../works/Baptiste09.pdf}{Baptiste09} (0.18)& \cellcolor{red!40}\href{../works/CarchraeBF05.pdf}{CarchraeBF05} (0.19)& \cellcolor{red!40}\href{../works/PesantRR15.pdf}{PesantRR15} (0.19)& \cellcolor{red!40}\href{../works/Hunsberger08.pdf}{Hunsberger08} (0.20)& \cellcolor{red!40}\href{../works/AngelsmarkJ00.pdf}{AngelsmarkJ00} (0.20)\\
Dot& \cellcolor{red!40}\href{../works/GeibingerMM19.pdf}{GeibingerMM19} (43.00)& \cellcolor{red!40}\href{../works/abs-1911-04766.pdf}{abs-1911-04766} (43.00)& \cellcolor{red!40}\href{../works/KoehlerBFFHPSSS21.pdf}{KoehlerBFFHPSSS21} (41.00)& \cellcolor{red!40}\href{../works/ColT22.pdf}{ColT22} (40.00)& \cellcolor{red!40}\href{../works/Lemos21.pdf}{Lemos21} (40.00)\\
Cosine& \cellcolor{red!40}\href{../works/PesantRR15.pdf}{PesantRR15} (0.72)& \cellcolor{red!40}\href{../works/BofillEGPSV14.pdf}{BofillEGPSV14} (0.69)& \cellcolor{red!40}\href{../works/Baptiste09.pdf}{Baptiste09} (0.68)& \cellcolor{red!40}\href{../works/BofillGSV15.pdf}{BofillGSV15} (0.64)& \cellcolor{red!40}\href{../works/ZhuS02.pdf}{ZhuS02} (0.63)\\
\index{BofillCSV17}\href{../works/BofillCSV17.pdf}{BofillCSV17} R\&C& \cellcolor{red!40}\href{../works/BofillCSV17a.pdf}{BofillCSV17a} (0.86)& \cellcolor{red!20}\href{../works/GeibingerMM19.pdf}{GeibingerMM19} (0.89)& \cellcolor{yellow!20}\href{../works/LombardiM13.pdf}{LombardiM13} (0.92)& \cellcolor{yellow!20}\href{../works/LombardiM12a.pdf}{LombardiM12a} (0.92)& \cellcolor{yellow!20}\href{../works/LiessM08.pdf}{LiessM08} (0.92)\\
Euclid& \cellcolor{red!40}\href{../works/BofillCSV17a.pdf}{BofillCSV17a} (0.17)& \cellcolor{red!40}\href{../works/LombardiM12a.pdf}{LombardiM12a} (0.19)& \cellcolor{red!40}\href{../works/LombardiM13.pdf}{LombardiM13} (0.20)& \cellcolor{red!40}\href{../works/OddiRC10.pdf}{OddiRC10} (0.22)& \cellcolor{red!40}\href{../works/BhatnagarKL19.pdf}{BhatnagarKL19} (0.23)\\
Dot& \cellcolor{red!40}\href{../works/Caballero19.pdf}{Caballero19} (111.00)& \cellcolor{red!40}\href{../works/Schutt11.pdf}{Schutt11} (107.00)& \cellcolor{red!40}\href{../works/Godet21a.pdf}{Godet21a} (102.00)& \cellcolor{red!40}\href{../works/Baptiste02.pdf}{Baptiste02} (100.00)& \cellcolor{red!40}\href{../works/Lombardi10.pdf}{Lombardi10} (98.00)\\
Cosine& \cellcolor{red!40}\href{../works/BofillCSV17a.pdf}{BofillCSV17a} (0.88)& \cellcolor{red!40}\href{../works/LombardiM12a.pdf}{LombardiM12a} (0.87)& \cellcolor{red!40}\href{../works/LombardiM13.pdf}{LombardiM13} (0.84)& \cellcolor{red!40}\href{../works/SchnellH15.pdf}{SchnellH15} (0.84)& \cellcolor{red!40}\href{../works/SzerediS16.pdf}{SzerediS16} (0.81)\\
\index{BofillCSV17a}\href{../works/BofillCSV17a.pdf}{BofillCSV17a} R\&C& \cellcolor{red!40}\href{../works/BofillCSV17.pdf}{BofillCSV17} (0.86)& \cellcolor{red!20}\href{../works/Kumar03.pdf}{Kumar03} (0.90)& \cellcolor{green!20}\href{../works/SzerediS16.pdf}{SzerediS16} (0.93)& \cellcolor{green!20}\href{../works/SchnellH15.pdf}{SchnellH15} (0.93)& \cellcolor{blue!20}\href{../works/VilimLS15.pdf}{VilimLS15} (0.97)\\
Euclid& \cellcolor{red!40}\href{../works/BofillCSV17.pdf}{BofillCSV17} (0.17)& \cellcolor{red!40}\href{../works/LombardiM13.pdf}{LombardiM13} (0.19)& \cellcolor{red!40}\href{../works/LombardiM12a.pdf}{LombardiM12a} (0.20)& \cellcolor{red!40}\href{../works/BhatnagarKL19.pdf}{BhatnagarKL19} (0.21)& \cellcolor{red!40}\href{../works/LiessM08.pdf}{LiessM08} (0.21)\\
Dot& \cellcolor{red!40}\href{../works/Schutt11.pdf}{Schutt11} (111.00)& \cellcolor{red!40}\href{../works/Caballero19.pdf}{Caballero19} (104.00)& \cellcolor{red!40}\href{../works/Godet21a.pdf}{Godet21a} (102.00)& \cellcolor{red!40}\href{../works/Lombardi10.pdf}{Lombardi10} (102.00)& \cellcolor{red!40}\href{../works/Dejemeppe16.pdf}{Dejemeppe16} (100.00)\\
Cosine& \cellcolor{red!40}\href{../works/BofillCSV17.pdf}{BofillCSV17} (0.88)& \cellcolor{red!40}\href{../works/LombardiM13.pdf}{LombardiM13} (0.85)& \cellcolor{red!40}\href{../works/LombardiM12a.pdf}{LombardiM12a} (0.85)& \cellcolor{red!40}\href{../works/LiessM08.pdf}{LiessM08} (0.85)& \cellcolor{red!40}\href{../works/YoungFS17.pdf}{YoungFS17} (0.82)\\
\index{BofillEGPSV14}\href{../works/BofillEGPSV14.pdf}{BofillEGPSV14} R\&C& \cellcolor{red!40}\href{../works/PesantRR15.pdf}{PesantRR15} (0.75)& \cellcolor{red!40}\href{../works/BofillGSV15.pdf}{BofillGSV15} (0.79)& \cellcolor{yellow!20}\href{../works/SzerediS16.pdf}{SzerediS16} (0.92)& \cellcolor{yellow!20}\href{../works/KelarevaTK13.pdf}{KelarevaTK13} (0.92)& \cellcolor{yellow!20}\href{../works/KreterSS15.pdf}{KreterSS15} (0.92)\\
Euclid& \cellcolor{red!40}\href{../works/BofillCGGPSV23.pdf}{BofillCGGPSV23} (0.24)& \cellcolor{red!20}\href{../works/BofillGSV15.pdf}{BofillGSV15} (0.24)& \cellcolor{yellow!20}\href{../works/FeldmanG89.pdf}{FeldmanG89} (0.27)& \cellcolor{yellow!20}\href{../works/GelainPRVW17.pdf}{GelainPRVW17} (0.28)& \cellcolor{yellow!20}\href{../works/FrostD98.pdf}{FrostD98} (0.28)\\
Dot& \cellcolor{red!40}\href{../works/Caballero19.pdf}{Caballero19} (61.00)& \cellcolor{red!40}\href{../works/Godet21a.pdf}{Godet21a} (60.00)& \cellcolor{red!40}\href{../works/KoehlerBFFHPSSS21.pdf}{KoehlerBFFHPSSS21} (59.00)& \cellcolor{red!40}\href{../works/ColT22.pdf}{ColT22} (59.00)& \cellcolor{red!40}\href{../works/Lemos21.pdf}{Lemos21} (58.00)\\
Cosine& \cellcolor{red!40}\href{../works/BofillCGGPSV23.pdf}{BofillCGGPSV23} (0.69)& \cellcolor{red!40}\href{../works/BofillGSV15.pdf}{BofillGSV15} (0.69)& \cellcolor{red!40}\href{../works/FeldmanG89.pdf}{FeldmanG89} (0.60)& \cellcolor{red!40}\href{../works/GelainPRVW17.pdf}{GelainPRVW17} (0.60)& \cellcolor{red!40}\href{../works/LiuCGM17.pdf}{LiuCGM17} (0.59)\\
\index{BofillGSV15}\href{../works/BofillGSV15.pdf}{BofillGSV15} R\&C& \cellcolor{red!40}\href{../works/BofillEGPSV14.pdf}{BofillEGPSV14} (0.79)& \cellcolor{red!40}\href{../works/PesantRR15.pdf}{PesantRR15} (0.81)\\
Euclid& \cellcolor{red!40}\href{../works/Baptiste09.pdf}{Baptiste09} (0.19)& \cellcolor{red!40}\href{../works/ZibranR11.pdf}{ZibranR11} (0.20)& \cellcolor{red!40}\href{../works/CarchraeBF05.pdf}{CarchraeBF05} (0.20)& \cellcolor{red!40}\href{../works/Perron05.pdf}{Perron05} (0.21)& \cellcolor{red!40}\href{../works/BofillCGGPSV23.pdf}{BofillCGGPSV23} (0.21)\\
Dot& \cellcolor{red!40}\href{../works/Lemos21.pdf}{Lemos21} (48.00)& \cellcolor{red!40}\href{../works/KanetAG04.pdf}{KanetAG04} (45.00)& \cellcolor{red!40}\href{../works/Malapert11.pdf}{Malapert11} (44.00)& \cellcolor{red!40}\href{../works/Siala15a.pdf}{Siala15a} (44.00)& \cellcolor{red!40}\href{../works/Fahimi16.pdf}{Fahimi16} (43.00)\\
Cosine& \cellcolor{red!40}\href{../works/BofillEGPSV14.pdf}{BofillEGPSV14} (0.69)& \cellcolor{red!40}\href{../works/ZibranR11.pdf}{ZibranR11} (0.66)& \cellcolor{red!40}\href{../works/Baptiste09.pdf}{Baptiste09} (0.65)& \cellcolor{red!40}\href{../works/ZhangLS12.pdf}{ZhangLS12} (0.64)& \cellcolor{red!40}\href{../works/BofillCGGPSV23.pdf}{BofillCGGPSV23} (0.64)\\
\index{BogaerdtW19}\href{../works/BogaerdtW19.pdf}{BogaerdtW19} R\&C& \cellcolor{red!40}\href{../works/Hooker17.pdf}{Hooker17} (0.78)& \cellcolor{yellow!20}\href{../works/HamdiL13.pdf}{HamdiL13} (0.93)& \cellcolor{green!20}\href{../works/GroleazNS20.pdf}{GroleazNS20} (0.95)& \cellcolor{green!20}\href{../works/GroleazNS20a.pdf}{GroleazNS20a} (0.95)& \cellcolor{green!20}\href{../works/OddiPCC03.pdf}{OddiPCC03} (0.95)\\
Euclid& \cellcolor{yellow!20}\href{../works/BenediktSMVH18.pdf}{BenediktSMVH18} (0.28)& \cellcolor{green!20}\href{../works/Hooker17.pdf}{Hooker17} (0.30)& \cellcolor{green!20}\href{../works/SmithC93.pdf}{SmithC93} (0.30)& \cellcolor{green!20}\href{../works/Jans09.pdf}{Jans09} (0.30)& \cellcolor{green!20}\href{../works/HebrardTW05.pdf}{HebrardTW05} (0.30)\\
Dot& \cellcolor{red!40}\href{../works/Groleaz21.pdf}{Groleaz21} (104.00)& \cellcolor{red!40}\href{../works/ZarandiASC20.pdf}{ZarandiASC20} (101.00)& \cellcolor{red!40}\href{../works/Baptiste02.pdf}{Baptiste02} (96.00)& \cellcolor{red!40}\href{../works/Lunardi20.pdf}{Lunardi20} (93.00)& \cellcolor{red!40}\href{../works/PrataAN23.pdf}{PrataAN23} (91.00)\\
Cosine& \cellcolor{red!40}\href{../works/GomesM17.pdf}{GomesM17} (0.71)& \cellcolor{red!40}\href{../works/AlfieriGPS23.pdf}{AlfieriGPS23} (0.70)& \cellcolor{red!40}\href{../works/PenzDN23.pdf}{PenzDN23} (0.70)& \cellcolor{red!40}\href{../works/NattafM20.pdf}{NattafM20} (0.67)& \cellcolor{red!40}\href{../works/BillautHL12.pdf}{BillautHL12} (0.67)\\
\index{Bonfietti16}\href{../works/Bonfietti16.pdf}{Bonfietti16} R\&C& \cellcolor{yellow!20}\href{../works/BonfiettiLBM11.pdf}{BonfiettiLBM11} (0.91)& \cellcolor{green!20}\href{../works/BonfiettiLBM12.pdf}{BonfiettiLBM12} (0.93)& \cellcolor{green!20}\href{../works/BonfiettiLBM14.pdf}{BonfiettiLBM14} (0.94)& \cellcolor{blue!20}\href{../works/LombardiBMB11.pdf}{LombardiBMB11} (0.97)& \cellcolor{blue!20}\href{../works/BeniniLMR11.pdf}{BeniniLMR11} (0.97)\\
Euclid& \cellcolor{red!40}\href{../works/BeniniBGM05a.pdf}{BeniniBGM05a} (0.21)& \cellcolor{red!40}\href{../works/BonfiettiM12.pdf}{BonfiettiM12} (0.22)& \cellcolor{red!40}\href{../works/BonfiettiLBM11.pdf}{BonfiettiLBM11} (0.22)& \cellcolor{red!40}\href{../works/LombardiM13.pdf}{LombardiM13} (0.23)& \cellcolor{red!40}\href{../works/Rit86.pdf}{Rit86} (0.23)\\
Dot& \cellcolor{red!40}\href{../works/Lombardi10.pdf}{Lombardi10} (81.00)& \cellcolor{red!40}\href{../works/Fahimi16.pdf}{Fahimi16} (76.00)& \cellcolor{red!40}\href{../works/LaborieRSV18.pdf}{LaborieRSV18} (74.00)& \cellcolor{red!40}\href{../works/Godet21a.pdf}{Godet21a} (73.00)& \cellcolor{red!40}\href{../works/Schutt11.pdf}{Schutt11} (73.00)\\
Cosine& \cellcolor{red!40}\href{../works/BonfiettiLBM11.pdf}{BonfiettiLBM11} (0.78)& \cellcolor{red!40}\href{../works/SimoninAHL12.pdf}{SimoninAHL12} (0.76)& \cellcolor{red!40}\href{../works/BeniniBGM05a.pdf}{BeniniBGM05a} (0.75)& \cellcolor{red!40}\href{../works/BeniniLMR08.pdf}{BeniniLMR08} (0.75)& \cellcolor{red!40}\href{../works/LombardiBMB11.pdf}{LombardiBMB11} (0.74)\\
\index{BonfiettiLBM11}\href{../works/BonfiettiLBM11.pdf}{BonfiettiLBM11} R\&C& \cellcolor{red!40}\href{../works/LombardiBMB11.pdf}{LombardiBMB11} (0.60)& \cellcolor{red!40}\href{../works/BonfiettiLBM12.pdf}{BonfiettiLBM12} (0.76)& \cellcolor{red!40}\href{../works/BonfiettiLBM14.pdf}{BonfiettiLBM14} (0.79)& \cellcolor{yellow!20}\href{../works/Bonfietti16.pdf}{Bonfietti16} (0.91)& \cellcolor{green!20}\href{../works/Muscettola02.pdf}{Muscettola02} (0.94)\\
Euclid& \cellcolor{red!40}\href{../works/LombardiBMB11.pdf}{LombardiBMB11} (0.17)& \cellcolor{red!40}\href{../works/BonfiettiM12.pdf}{BonfiettiM12} (0.18)& \cellcolor{red!40}\href{../works/BonfiettiLBM12.pdf}{BonfiettiLBM12} (0.19)& \cellcolor{red!40}\href{../works/Bonfietti16.pdf}{Bonfietti16} (0.22)& \cellcolor{red!40}\href{../works/BonfiettiLM13.pdf}{BonfiettiLM13} (0.23)\\
Dot& \cellcolor{red!40}\href{../works/BonfiettiLBM14.pdf}{BonfiettiLBM14} (97.00)& \cellcolor{red!40}\href{../works/Dejemeppe16.pdf}{Dejemeppe16} (88.00)& \cellcolor{red!40}\href{../works/Lombardi10.pdf}{Lombardi10} (88.00)& \cellcolor{red!40}\href{../works/Schutt11.pdf}{Schutt11} (88.00)& \cellcolor{red!40}\href{../works/Astrand21.pdf}{Astrand21} (87.00)\\
Cosine& \cellcolor{red!40}\href{../works/LombardiBMB11.pdf}{LombardiBMB11} (0.90)& \cellcolor{red!40}\href{../works/BonfiettiM12.pdf}{BonfiettiM12} (0.87)& \cellcolor{red!40}\href{../works/BonfiettiLBM12.pdf}{BonfiettiLBM12} (0.86)& \cellcolor{red!40}\href{../works/BonfiettiLBM14.pdf}{BonfiettiLBM14} (0.85)& \cellcolor{red!40}\href{../works/BonfiettiLM13.pdf}{BonfiettiLM13} (0.80)\\
\index{BonfiettiLBM12}\href{../works/BonfiettiLBM12.pdf}{BonfiettiLBM12} R\&C& \cellcolor{red!40}\href{../works/BonfiettiLBM11.pdf}{BonfiettiLBM11} (0.76)& \cellcolor{red!40}\href{../works/BonfiettiLBM14.pdf}{BonfiettiLBM14} (0.81)& \cellcolor{red!20}\href{../works/BeniniLMR08.pdf}{BeniniLMR08} (0.89)& \cellcolor{yellow!20}\href{../works/LombardiBMB11.pdf}{LombardiBMB11} (0.92)& \cellcolor{yellow!20}\href{../works/Davenport10.pdf}{Davenport10} (0.92)\\
Euclid& \cellcolor{red!40}\href{../works/BonfiettiLM13.pdf}{BonfiettiLM13} (0.19)& \cellcolor{red!40}\href{../works/BonfiettiLBM11.pdf}{BonfiettiLBM11} (0.19)& \cellcolor{red!40}\href{../works/LombardiBMB11.pdf}{LombardiBMB11} (0.20)& \cellcolor{red!40}\href{../works/BonfiettiM12.pdf}{BonfiettiM12} (0.24)& \cellcolor{red!20}\href{../works/BonfiettiZLM16.pdf}{BonfiettiZLM16} (0.25)\\
Dot& \cellcolor{red!40}\href{../works/BonfiettiLBM14.pdf}{BonfiettiLBM14} (104.00)& \cellcolor{red!40}\href{../works/Lombardi10.pdf}{Lombardi10} (99.00)& \cellcolor{red!40}\href{../works/Fahimi16.pdf}{Fahimi16} (96.00)& \cellcolor{red!40}\href{../works/Godet21a.pdf}{Godet21a} (95.00)& \cellcolor{red!40}\href{../works/Groleaz21.pdf}{Groleaz21} (95.00)\\
Cosine& \cellcolor{red!40}\href{../works/BonfiettiLM13.pdf}{BonfiettiLM13} (0.87)& \cellcolor{red!40}\href{../works/LombardiBMB11.pdf}{LombardiBMB11} (0.86)& \cellcolor{red!40}\href{../works/BonfiettiLBM11.pdf}{BonfiettiLBM11} (0.86)& \cellcolor{red!40}\href{../works/BonfiettiLBM14.pdf}{BonfiettiLBM14} (0.85)& \cellcolor{red!40}\href{../works/BonfiettiZLM16.pdf}{BonfiettiZLM16} (0.79)\\
\index{BonfiettiLBM14}\href{../works/BonfiettiLBM14.pdf}{BonfiettiLBM14} R\&C& \cellcolor{red!40}\href{../works/BonfiettiLBM11.pdf}{BonfiettiLBM11} (0.79)& \cellcolor{red!40}\href{../works/BonfiettiLBM12.pdf}{BonfiettiLBM12} (0.81)& \cellcolor{red!20}\href{../works/BonfiettiLM13.pdf}{BonfiettiLM13} (0.89)& \cellcolor{red!20}\href{../works/LombardiBMB11.pdf}{LombardiBMB11} (0.89)& \cellcolor{green!20}\href{../works/OzturkTHO15.pdf}{OzturkTHO15} (0.94)\\
Euclid& \cellcolor{red!20}\href{../works/BonfiettiLBM11.pdf}{BonfiettiLBM11} (0.26)& \cellcolor{red!20}\href{../works/BonfiettiLBM12.pdf}{BonfiettiLBM12} (0.26)& \cellcolor{yellow!20}\href{../works/LombardiBMB11.pdf}{LombardiBMB11} (0.27)& \cellcolor{green!20}\href{../works/BonfiettiLM13.pdf}{BonfiettiLM13} (0.29)& \cellcolor{green!20}\href{../works/BonfiettiZLM16.pdf}{BonfiettiZLM16} (0.29)\\
Dot& \cellcolor{red!40}\href{../works/Lombardi10.pdf}{Lombardi10} (152.00)& \cellcolor{red!40}\href{../works/Dejemeppe16.pdf}{Dejemeppe16} (146.00)& \cellcolor{red!40}\href{../works/Fahimi16.pdf}{Fahimi16} (142.00)& \cellcolor{red!40}\href{../works/Godet21a.pdf}{Godet21a} (141.00)& \cellcolor{red!40}\href{../works/Schutt11.pdf}{Schutt11} (141.00)\\
Cosine& \cellcolor{red!40}\href{../works/BonfiettiLBM11.pdf}{BonfiettiLBM11} (0.85)& \cellcolor{red!40}\href{../works/BonfiettiLBM12.pdf}{BonfiettiLBM12} (0.85)& \cellcolor{red!40}\href{../works/LombardiBMB11.pdf}{LombardiBMB11} (0.82)& \cellcolor{red!40}\href{../works/BonfiettiLM13.pdf}{BonfiettiLM13} (0.79)& \cellcolor{red!40}\href{../works/BonfiettiZLM16.pdf}{BonfiettiZLM16} (0.79)\\
\index{BonfiettiLM13}\href{../works/BonfiettiLM13.pdf}{BonfiettiLM13} R\&C& \cellcolor{red!20}\href{../works/BonfiettiLBM14.pdf}{BonfiettiLBM14} (0.89)& \cellcolor{green!20}\href{../works/LetortBC12.pdf}{LetortBC12} (0.95)& \cellcolor{green!20}\href{../works/Vilim04.pdf}{Vilim04} (0.96)& \cellcolor{blue!20}\href{../works/MercierH08.pdf}{MercierH08} (0.97)\\
Euclid& \cellcolor{red!40}\href{../works/BonfiettiLBM12.pdf}{BonfiettiLBM12} (0.19)& \cellcolor{red!40}\href{../works/LombardiBMB11.pdf}{LombardiBMB11} (0.22)& \cellcolor{red!40}\href{../works/LombardiM13.pdf}{LombardiM13} (0.23)& \cellcolor{red!40}\href{../works/BonfiettiM12.pdf}{BonfiettiM12} (0.23)& \cellcolor{red!40}\href{../works/BonfiettiLBM11.pdf}{BonfiettiLBM11} (0.23)\\
Dot& \cellcolor{red!40}\href{../works/Lombardi10.pdf}{Lombardi10} (101.00)& \cellcolor{red!40}\href{../works/Groleaz21.pdf}{Groleaz21} (98.00)& \cellcolor{red!40}\href{../works/BonfiettiLBM14.pdf}{BonfiettiLBM14} (97.00)& \cellcolor{red!40}\href{../works/Godet21a.pdf}{Godet21a} (94.00)& \cellcolor{red!40}\href{../works/LaborieRSV18.pdf}{LaborieRSV18} (91.00)\\
Cosine& \cellcolor{red!40}\href{../works/BonfiettiLBM12.pdf}{BonfiettiLBM12} (0.87)& \cellcolor{red!40}\href{../works/LombardiBMB11.pdf}{LombardiBMB11} (0.84)& \cellcolor{red!40}\href{../works/BonfiettiM12.pdf}{BonfiettiM12} (0.80)& \cellcolor{red!40}\href{../works/BonfiettiLBM11.pdf}{BonfiettiLBM11} (0.80)& \cellcolor{red!40}\href{../works/LombardiM13.pdf}{LombardiM13} (0.79)\\
\index{BonfiettiLM14}\href{../works/BonfiettiLM14.pdf}{BonfiettiLM14} R\&C& \cellcolor{red!20}\href{../works/LombardiMB13.pdf}{LombardiMB13} (0.89)& \cellcolor{red!20}\href{../works/LombardiBM15.pdf}{LombardiBM15} (0.90)& \cellcolor{yellow!20}\href{../works/WuBB09.pdf}{WuBB09} (0.92)& \cellcolor{yellow!20}\href{../works/LombardiBMB11.pdf}{LombardiBMB11} (0.92)& \cellcolor{yellow!20}\href{../works/LombardiM13.pdf}{LombardiM13} (0.92)\\
Euclid& \cellcolor{red!40}\href{../works/LombardiBM15.pdf}{LombardiBM15} (0.18)& \cellcolor{red!20}\href{../works/BeckW04.pdf}{BeckW04} (0.25)& \cellcolor{red!20}\href{../works/LombardiM12a.pdf}{LombardiM12a} (0.26)& \cellcolor{yellow!20}\href{../works/CestaOF99.pdf}{CestaOF99} (0.27)& \cellcolor{yellow!20}\href{../works/FortinZDF05.pdf}{FortinZDF05} (0.27)\\
Dot& \cellcolor{red!40}\href{../works/Lombardi10.pdf}{Lombardi10} (114.00)& \cellcolor{red!40}\href{../works/ZarandiASC20.pdf}{ZarandiASC20} (113.00)& \cellcolor{red!40}\href{../works/LaborieRSV18.pdf}{LaborieRSV18} (111.00)& \cellcolor{red!40}\href{../works/Beck99.pdf}{Beck99} (111.00)& \cellcolor{red!40}\href{../works/Groleaz21.pdf}{Groleaz21} (110.00)\\
Cosine& \cellcolor{red!40}\href{../works/LombardiBM15.pdf}{LombardiBM15} (0.89)& \cellcolor{red!40}\href{../works/BeckW07.pdf}{BeckW07} (0.79)& \cellcolor{red!40}\href{../works/BeckW04.pdf}{BeckW04} (0.79)& \cellcolor{red!40}\href{../works/LombardiM12a.pdf}{LombardiM12a} (0.78)& \cellcolor{red!40}\href{../works/CestaOF99.pdf}{CestaOF99} (0.77)\\
\index{BonfiettiM12}\href{../works/BonfiettiM12.pdf}{BonfiettiM12} R\&C\\
Euclid& \cellcolor{red!40}\href{../works/KovacsEKV05.pdf}{KovacsEKV05} (0.17)& \cellcolor{red!40}\href{../works/Caballero23.pdf}{Caballero23} (0.17)& \cellcolor{red!40}\href{../works/CestaOS98.pdf}{CestaOS98} (0.18)& \cellcolor{red!40}\href{../works/BonfiettiLBM11.pdf}{BonfiettiLBM11} (0.18)& \cellcolor{red!40}\href{../works/LombardiM13.pdf}{LombardiM13} (0.18)\\
Dot& \cellcolor{red!40}\href{../works/BonfiettiLBM14.pdf}{BonfiettiLBM14} (60.00)& \cellcolor{red!40}\href{../works/Lombardi10.pdf}{Lombardi10} (56.00)& \cellcolor{red!40}\href{../works/Schutt11.pdf}{Schutt11} (56.00)& \cellcolor{red!40}\href{../works/LaborieRSV18.pdf}{LaborieRSV18} (55.00)& \cellcolor{red!40}\href{../works/Fahimi16.pdf}{Fahimi16} (55.00)\\
Cosine& \cellcolor{red!40}\href{../works/BonfiettiLBM11.pdf}{BonfiettiLBM11} (0.87)& \cellcolor{red!40}\href{../works/BonfiettiLM13.pdf}{BonfiettiLM13} (0.80)& \cellcolor{red!40}\href{../works/BonfiettiLBM12.pdf}{BonfiettiLBM12} (0.78)& \cellcolor{red!40}\href{../works/LombardiBMB11.pdf}{LombardiBMB11} (0.76)& \cellcolor{red!40}\href{../works/LombardiM13.pdf}{LombardiM13} (0.75)\\
\index{BonfiettiZLM16}\href{../works/BonfiettiZLM16.pdf}{BonfiettiZLM16} R\&C& \cellcolor{red!40}\href{../works/OuelletQ13.pdf}{OuelletQ13} (0.84)& \cellcolor{red!40}\href{../works/LetortCB13.pdf}{LetortCB13} (0.86)& \cellcolor{red!20}\href{../works/Vilim09.pdf}{Vilim09} (0.87)& \cellcolor{red!20}\href{../works/Vilim09a.pdf}{Vilim09a} (0.87)& \cellcolor{red!20}\href{../works/KameugneF13.pdf}{KameugneF13} (0.87)\\
Euclid& \cellcolor{red!20}\href{../works/BonfiettiLBM12.pdf}{BonfiettiLBM12} (0.25)& \cellcolor{red!20}\href{../works/BonfiettiLM13.pdf}{BonfiettiLM13} (0.26)& \cellcolor{yellow!20}\href{../works/LombardiBMB11.pdf}{LombardiBMB11} (0.26)& \cellcolor{yellow!20}\href{../works/SimoninAHL12.pdf}{SimoninAHL12} (0.27)& \cellcolor{yellow!20}\href{../works/BonfiettiLBM11.pdf}{BonfiettiLBM11} (0.27)\\
Dot& \cellcolor{red!40}\href{../works/Lombardi10.pdf}{Lombardi10} (105.00)& \cellcolor{red!40}\href{../works/Schutt11.pdf}{Schutt11} (102.00)& \cellcolor{red!40}\href{../works/BonfiettiLBM14.pdf}{BonfiettiLBM14} (102.00)& \cellcolor{red!40}\href{../works/Godet21a.pdf}{Godet21a} (96.00)& \cellcolor{red!40}\href{../works/Fahimi16.pdf}{Fahimi16} (96.00)\\
Cosine& \cellcolor{red!40}\href{../works/BonfiettiLBM14.pdf}{BonfiettiLBM14} (0.79)& \cellcolor{red!40}\href{../works/BonfiettiLBM12.pdf}{BonfiettiLBM12} (0.79)& \cellcolor{red!40}\href{../works/LombardiBMB11.pdf}{LombardiBMB11} (0.77)& \cellcolor{red!40}\href{../works/BonfiettiLM13.pdf}{BonfiettiLM13} (0.77)& \cellcolor{red!40}\href{../works/DerrienPZ14.pdf}{DerrienPZ14} (0.75)\\
\index{BonninMNE24}\href{../works/BonninMNE24.pdf}{BonninMNE24} R\&C\\
Euclid& \cellcolor{black!20}\href{../works/KovacsB11.pdf}{KovacsB11} (0.35)& \cellcolor{black!20}\href{../works/BartakSR08.pdf}{BartakSR08} (0.36)& \cellcolor{black!20}\href{../works/HeipckeCCS00.pdf}{HeipckeCCS00} (0.36)& \cellcolor{black!20}\href{../works/OzturkTHO12.pdf}{OzturkTHO12} (0.36)& \cellcolor{black!20}\href{../works/SourdN00.pdf}{SourdN00} (0.37)\\
Dot& \cellcolor{red!40}\href{../works/Baptiste02.pdf}{Baptiste02} (190.00)& \cellcolor{red!40}\href{../works/Groleaz21.pdf}{Groleaz21} (180.00)& \cellcolor{red!40}\href{../works/Malapert11.pdf}{Malapert11} (177.00)& \cellcolor{red!40}\href{../works/ZarandiASC20.pdf}{ZarandiASC20} (170.00)& \cellcolor{red!40}\href{../works/Dejemeppe16.pdf}{Dejemeppe16} (166.00)\\
Cosine& \cellcolor{red!40}\href{../works/KovacsB11.pdf}{KovacsB11} (0.77)& \cellcolor{red!40}\href{../works/BartakSR08.pdf}{BartakSR08} (0.76)& \cellcolor{red!40}\href{../works/SourdN00.pdf}{SourdN00} (0.74)& \cellcolor{red!40}\href{../works/HeipckeCCS00.pdf}{HeipckeCCS00} (0.74)& \cellcolor{red!40}\href{../works/OzturkTHO12.pdf}{OzturkTHO12} (0.74)\\
\index{BoothNB16}\href{../works/BoothNB16.pdf}{BoothNB16} R\&C& \cellcolor{red!40}\href{../works/BoothTNB16.pdf}{BoothTNB16} (0.85)& \cellcolor{yellow!20}\href{../works/Laborie18a.pdf}{Laborie18a} (0.92)& \cellcolor{yellow!20}\href{../works/GilesH16.pdf}{GilesH16} (0.92)& \cellcolor{yellow!20}\href{../works/KovacsV04.pdf}{KovacsV04} (0.93)& \cellcolor{yellow!20}\href{../works/LaborieRSV18.pdf}{LaborieRSV18} (0.93)\\
Euclid& \cellcolor{red!40}\href{../works/TranVNB17.pdf}{TranVNB17} (0.19)& \cellcolor{red!40}\href{../works/TranVNB17a.pdf}{TranVNB17a} (0.19)& \cellcolor{red!40}\href{../works/NishikawaSTT19.pdf}{NishikawaSTT19} (0.23)& \cellcolor{red!40}\href{../works/NishikawaSTT18a.pdf}{NishikawaSTT18a} (0.23)& \cellcolor{red!40}\href{../works/NishikawaSTT18.pdf}{NishikawaSTT18} (0.23)\\
Dot& \cellcolor{red!40}\href{../works/LaborieRSV18.pdf}{LaborieRSV18} (99.00)& \cellcolor{red!40}\href{../works/Lombardi10.pdf}{Lombardi10} (94.00)& \cellcolor{red!40}\href{../works/Astrand21.pdf}{Astrand21} (93.00)& \cellcolor{red!40}\href{../works/Beck99.pdf}{Beck99} (93.00)& \cellcolor{red!40}\href{../works/Dejemeppe16.pdf}{Dejemeppe16} (92.00)\\
Cosine& \cellcolor{red!40}\href{../works/TranVNB17.pdf}{TranVNB17} (0.88)& \cellcolor{red!40}\href{../works/TranVNB17a.pdf}{TranVNB17a} (0.84)& \cellcolor{red!40}\href{../works/NishikawaSTT19.pdf}{NishikawaSTT19} (0.81)& \cellcolor{red!40}\href{../works/NishikawaSTT18a.pdf}{NishikawaSTT18a} (0.79)& \cellcolor{red!40}\href{../works/BoothTNB16.pdf}{BoothTNB16} (0.79)\\
\index{BoothTNB16}\href{../works/BoothTNB16.pdf}{BoothTNB16} R\&C& \cellcolor{red!40}\href{../works/BoothNB16.pdf}{BoothNB16} (0.85)& \cellcolor{red!20}\href{../works/Laborie18a.pdf}{Laborie18a} (0.87)& \cellcolor{red!20}\href{../works/GilesH16.pdf}{GilesH16} (0.89)& \cellcolor{yellow!20}\href{../works/LouieVNB14.pdf}{LouieVNB14} (0.91)& \cellcolor{yellow!20}\href{../works/BeckF00.pdf}{BeckF00} (0.93)\\
Euclid& \cellcolor{red!20}\href{../works/BoothNB16.pdf}{BoothNB16} (0.25)& \cellcolor{red!20}\href{../works/TranVNB17a.pdf}{TranVNB17a} (0.25)& \cellcolor{green!20}\href{../works/BockmayrP06.pdf}{BockmayrP06} (0.29)& \cellcolor{green!20}\href{../works/TranVNB17.pdf}{TranVNB17} (0.30)& \cellcolor{green!20}\href{../works/CauwelaertDMS16.pdf}{CauwelaertDMS16} (0.30)\\
Dot& \cellcolor{red!40}\href{../works/Astrand21.pdf}{Astrand21} (115.00)& \cellcolor{red!40}\href{../works/LaborieRSV18.pdf}{LaborieRSV18} (112.00)& \cellcolor{red!40}\href{../works/Dejemeppe16.pdf}{Dejemeppe16} (112.00)& \cellcolor{red!40}\href{../works/Groleaz21.pdf}{Groleaz21} (112.00)& \cellcolor{red!40}\href{../works/Lunardi20.pdf}{Lunardi20} (111.00)\\
Cosine& \cellcolor{red!40}\href{../works/BoothNB16.pdf}{BoothNB16} (0.79)& \cellcolor{red!40}\href{../works/TranVNB17a.pdf}{TranVNB17a} (0.78)& \cellcolor{red!40}\href{../works/Ham18a.pdf}{Ham18a} (0.73)& \cellcolor{red!40}\href{../works/TranVNB17.pdf}{TranVNB17} (0.72)& \cellcolor{red!40}\href{../works/MurinR19.pdf}{MurinR19} (0.71)\\
\index{BorghesiBLMB18}\href{../works/BorghesiBLMB18.pdf}{BorghesiBLMB18} R\&C& \cellcolor{yellow!20}\href{../works/BartoliniBBLM14.pdf}{BartoliniBBLM14} (0.92)& \cellcolor{green!20}\href{../works/GalleguillosKSB19.pdf}{GalleguillosKSB19} (0.94)& \cellcolor{green!20}\href{../works/BridiBLMB16.pdf}{BridiBLMB16} (0.95)& \cellcolor{blue!20}\href{../works/GilesH16.pdf}{GilesH16} (0.96)& \cellcolor{blue!20}\href{../works/MelgarejoLS15.pdf}{MelgarejoLS15} (0.97)\\
Euclid& \cellcolor{yellow!20}\href{../works/BridiBLMB16.pdf}{BridiBLMB16} (0.28)& \cellcolor{green!20}\href{../works/GalleguillosKSB19.pdf}{GalleguillosKSB19} (0.30)& \cellcolor{green!20}\href{../works/HurleyOS16.pdf}{HurleyOS16} (0.30)& \cellcolor{green!20}\href{../works/AstrandJZ18.pdf}{AstrandJZ18} (0.30)& \cellcolor{green!20}\href{../works/HeipckeCCS00.pdf}{HeipckeCCS00} (0.31)\\
Dot& \cellcolor{red!40}\href{../works/Groleaz21.pdf}{Groleaz21} (115.00)& \cellcolor{red!40}\href{../works/Dejemeppe16.pdf}{Dejemeppe16} (114.00)& \cellcolor{red!40}\href{../works/Lombardi10.pdf}{Lombardi10} (114.00)& \cellcolor{red!40}\href{../works/Godet21a.pdf}{Godet21a} (113.00)& \cellcolor{red!40}\href{../works/ZarandiASC20.pdf}{ZarandiASC20} (112.00)\\
Cosine& \cellcolor{red!40}\href{../works/BridiBLMB16.pdf}{BridiBLMB16} (0.80)& \cellcolor{red!40}\href{../works/GalleguillosKSB19.pdf}{GalleguillosKSB19} (0.74)& \cellcolor{red!40}\href{../works/HurleyOS16.pdf}{HurleyOS16} (0.74)& \cellcolor{red!40}\href{../works/AstrandJZ18.pdf}{AstrandJZ18} (0.73)& \cellcolor{red!40}\href{../works/HeipckeCCS00.pdf}{HeipckeCCS00} (0.73)\\
\index{BosiM2001}\href{../works/BosiM2001.pdf}{BosiM2001} R\&C& \cellcolor{red!20}\href{../works/Simonis07.pdf}{Simonis07} (0.86)& \cellcolor{red!20}\href{../works/Simonis99.pdf}{Simonis99} (0.87)& \cellcolor{red!20}\href{../works/Simonis95a.pdf}{Simonis95a} (0.88)& \cellcolor{red!20}\href{../works/SimonisCK00.pdf}{SimonisCK00} (0.88)& \cellcolor{red!20}\href{../works/Goltz95.pdf}{Goltz95} (0.89)\\
Euclid& \cellcolor{black!20}\href{../works/Goltz95.pdf}{Goltz95} (0.34)& \cellcolor{black!20}\href{../works/WikarekS19.pdf}{WikarekS19} (0.34)& \cellcolor{black!20}\href{../works/KovacsV04.pdf}{KovacsV04} (0.37)& \cellcolor{black!20}\href{../works/Colombani96.pdf}{Colombani96} (0.37)& \href{../works/KovacsV06.pdf}{KovacsV06} (0.37)\\
Dot& \cellcolor{red!40}\href{../works/Baptiste02.pdf}{Baptiste02} (164.00)& \cellcolor{red!40}\href{../works/ZarandiASC20.pdf}{ZarandiASC20} (152.00)& \cellcolor{red!40}\href{../works/Beck99.pdf}{Beck99} (150.00)& \cellcolor{red!40}\href{../works/Malapert11.pdf}{Malapert11} (149.00)& \cellcolor{red!40}\href{../works/Groleaz21.pdf}{Groleaz21} (147.00)\\
Cosine& \cellcolor{red!40}\href{../works/WikarekS19.pdf}{WikarekS19} (0.75)& \cellcolor{red!40}\href{../works/Goltz95.pdf}{Goltz95} (0.74)& \cellcolor{red!40}\href{../works/TrentesauxPT01.pdf}{TrentesauxPT01} (0.70)& \cellcolor{red!40}\href{../works/KovacsV04.pdf}{KovacsV04} (0.70)& \cellcolor{red!40}\href{../works/Colombani96.pdf}{Colombani96} (0.70)\\
\index{BoucherBVBL97}BoucherBVBL97 R\&C\\
Euclid\\
Dot\\
Cosine\\
\index{BoudreaultSLQ22}\href{../works/BoudreaultSLQ22.pdf}{BoudreaultSLQ22} R\&C\\
Euclid& \cellcolor{green!20}\href{../works/YoungFS17.pdf}{YoungFS17} (0.31)& \cellcolor{blue!20}\href{../works/SzerediS16.pdf}{SzerediS16} (0.34)& \cellcolor{black!20}\href{../works/LiessM08.pdf}{LiessM08} (0.35)& \cellcolor{black!20}\href{../works/PovedaAA23.pdf}{PovedaAA23} (0.36)& \cellcolor{black!20}\href{../works/abs-1009-0347.pdf}{abs-1009-0347} (0.36)\\
Dot& \cellcolor{red!40}\href{../works/Godet21a.pdf}{Godet21a} (184.00)& \cellcolor{red!40}\href{../works/Dejemeppe16.pdf}{Dejemeppe16} (166.00)& \cellcolor{red!40}\href{../works/Schutt11.pdf}{Schutt11} (166.00)& \cellcolor{red!40}\href{../works/Caballero19.pdf}{Caballero19} (155.00)& \cellcolor{red!40}\href{../works/Lombardi10.pdf}{Lombardi10} (154.00)\\
Cosine& \cellcolor{red!40}\href{../works/YoungFS17.pdf}{YoungFS17} (0.81)& \cellcolor{red!40}\href{../works/SzerediS16.pdf}{SzerediS16} (0.77)& \cellcolor{red!40}\href{../works/PovedaAA23.pdf}{PovedaAA23} (0.77)& \cellcolor{red!40}\href{../works/LiessM08.pdf}{LiessM08} (0.74)& \cellcolor{red!40}\href{../works/SchuttFSW11.pdf}{SchuttFSW11} (0.74)\\
\index{BourdaisGP03}\href{../works/BourdaisGP03.pdf}{BourdaisGP03} R\&C& \cellcolor{red!20}\href{../works/WeilHFP95.pdf}{WeilHFP95} (0.88)& \cellcolor{green!20}\href{../works/OzturkTHO12.pdf}{OzturkTHO12} (0.94)& \cellcolor{green!20}\href{../works/ZeballosQH10.pdf}{ZeballosQH10} (0.95)& \cellcolor{green!20}\href{../works/Simonis07.pdf}{Simonis07} (0.95)& \cellcolor{green!20}\href{../works/NovasH12.pdf}{NovasH12} (0.96)\\
Euclid& \cellcolor{red!40}\href{../works/abs-1902-01193.pdf}{abs-1902-01193} (0.23)& \cellcolor{red!40}\href{../works/HoYCLLCLC18.pdf}{HoYCLLCLC18} (0.23)& \cellcolor{red!40}\href{../works/PesantGPR99.pdf}{PesantGPR99} (0.24)& \cellcolor{red!20}\href{../works/BandaSC11.pdf}{BandaSC11} (0.25)& \cellcolor{red!20}\href{../works/AngelsmarkJ00.pdf}{AngelsmarkJ00} (0.25)\\
Dot& \cellcolor{red!40}\href{../works/Dejemeppe16.pdf}{Dejemeppe16} (74.00)& \cellcolor{red!40}\href{../works/Simonis07.pdf}{Simonis07} (66.00)& \cellcolor{red!40}\href{../works/ZarandiASC20.pdf}{ZarandiASC20} (66.00)& \cellcolor{red!40}\href{../works/HookerH17.pdf}{HookerH17} (64.00)& \cellcolor{red!40}\href{../works/Beck99.pdf}{Beck99} (64.00)\\
Cosine& \cellcolor{red!40}\href{../works/abs-1902-01193.pdf}{abs-1902-01193} (0.73)& \cellcolor{red!40}\href{../works/HoYCLLCLC18.pdf}{HoYCLLCLC18} (0.70)& \cellcolor{red!40}\href{../works/PesantGPR99.pdf}{PesantGPR99} (0.69)& \cellcolor{red!40}\href{../works/NishikawaSTT18a.pdf}{NishikawaSTT18a} (0.68)& \cellcolor{red!40}\href{../works/WeilHFP95.pdf}{WeilHFP95} (0.67)\\
\index{BourreauGGLT22}\href{../works/BourreauGGLT22.pdf}{BourreauGGLT22} R\&C& \cellcolor{green!20}KizilayC20 (0.93)& \cellcolor{green!20}\href{../works/YunusogluY22.pdf}{YunusogluY22} (0.96)& \cellcolor{blue!20}\href{../works/NaderiBZR23.pdf}{NaderiBZR23} (0.96)& \cellcolor{blue!20}\href{../works/RendlPHPR12.pdf}{RendlPHPR12} (0.97)& \cellcolor{blue!20}\href{../works/GokgurHO18.pdf}{GokgurHO18} (0.97)\\
Euclid& \cellcolor{black!20}\href{../works/Puget95.pdf}{Puget95} (0.35)& \cellcolor{black!20}\href{../works/KletzanderMH21.pdf}{KletzanderMH21} (0.35)& \cellcolor{black!20}\href{../works/MelgarejoLS15.pdf}{MelgarejoLS15} (0.36)& \cellcolor{black!20}\href{../works/MouraSCL08a.pdf}{MouraSCL08a} (0.36)& \cellcolor{black!20}\href{../works/NishikawaSTT18.pdf}{NishikawaSTT18} (0.37)\\
Dot& \cellcolor{red!40}\href{../works/Malapert11.pdf}{Malapert11} (118.00)& \cellcolor{red!40}\href{../works/Groleaz21.pdf}{Groleaz21} (110.00)& \cellcolor{red!40}\href{../works/Astrand21.pdf}{Astrand21} (105.00)& \cellcolor{red!40}\href{../works/Froger16.pdf}{Froger16} (104.00)& \cellcolor{red!40}\href{../works/Lunardi20.pdf}{Lunardi20} (102.00)\\
Cosine& \cellcolor{red!40}\href{../works/Wallace06.pdf}{Wallace06} (0.65)& \cellcolor{red!40}\href{../works/MelgarejoLS15.pdf}{MelgarejoLS15} (0.63)& \cellcolor{red!40}\href{../works/KletzanderMH21.pdf}{KletzanderMH21} (0.62)& \cellcolor{red!40}\href{../works/MouraSCL08a.pdf}{MouraSCL08a} (0.61)& \cellcolor{red!40}\href{../works/EreminW01.pdf}{EreminW01} (0.61)\\
\index{BreitingerL95}BreitingerL95 R\&C\\
Euclid\\
Dot\\
Cosine\\
\index{BriandHHL08}BriandHHL08 R\&C& \cellcolor{red!40}LiuGT10 (0.84)& \cellcolor{red!40}\href{../works/GrimesHM09.pdf}{GrimesHM09} (0.85)& \cellcolor{red!20}\href{../works/TanSD10.pdf}{TanSD10} (0.88)& \cellcolor{red!20}EsquirolLH2008 (0.88)& \cellcolor{red!20}\href{../works/GrimesH10.pdf}{GrimesH10} (0.88)\\
Euclid\\
Dot\\
Cosine\\
\index{BridiBLMB16}\href{../works/BridiBLMB16.pdf}{BridiBLMB16} R\&C& \cellcolor{red!20}\href{../works/BartoliniBBLM14.pdf}{BartoliniBBLM14} (0.90)& \cellcolor{green!20}\href{../works/NishikawaSTT18a.pdf}{NishikawaSTT18a} (0.94)& \cellcolor{green!20}\href{../works/NishikawaSTT19.pdf}{NishikawaSTT19} (0.95)& \cellcolor{green!20}\href{../works/BorghesiBLMB18.pdf}{BorghesiBLMB18} (0.95)& \cellcolor{green!20}BaptisteLPN06 (0.96)\\
Euclid& \cellcolor{yellow!20}\href{../works/BorghesiBLMB18.pdf}{BorghesiBLMB18} (0.28)& \cellcolor{green!20}\href{../works/BartoliniBBLM14.pdf}{BartoliniBBLM14} (0.31)& \cellcolor{green!20}\href{../works/GalleguillosKSB19.pdf}{GalleguillosKSB19} (0.31)& \cellcolor{blue!20}\href{../works/AalianPG23.pdf}{AalianPG23} (0.33)& \cellcolor{blue!20}\href{../works/KhemmoudjPB06.pdf}{KhemmoudjPB06} (0.34)\\
Dot& \cellcolor{red!40}\href{../works/Lombardi10.pdf}{Lombardi10} (128.00)& \cellcolor{red!40}\href{../works/Dejemeppe16.pdf}{Dejemeppe16} (126.00)& \cellcolor{red!40}\href{../works/ZarandiASC20.pdf}{ZarandiASC20} (124.00)& \cellcolor{red!40}\href{../works/Groleaz21.pdf}{Groleaz21} (119.00)& \cellcolor{red!40}\href{../works/Astrand21.pdf}{Astrand21} (115.00)\\
Cosine& \cellcolor{red!40}\href{../works/BorghesiBLMB18.pdf}{BorghesiBLMB18} (0.80)& \cellcolor{red!40}\href{../works/BartoliniBBLM14.pdf}{BartoliniBBLM14} (0.72)& \cellcolor{red!40}\href{../works/GalleguillosKSB19.pdf}{GalleguillosKSB19} (0.71)& \cellcolor{red!40}\href{../works/AalianPG23.pdf}{AalianPG23} (0.71)& \cellcolor{red!40}\href{../works/HentenryckM04.pdf}{HentenryckM04} (0.67)\\
\index{BridiLBBM16}\href{../works/BridiLBBM16.pdf}{BridiLBBM16} R\&C\\
Euclid& \cellcolor{red!40}\href{../works/CrawfordB94.pdf}{CrawfordB94} (0.21)& \cellcolor{red!40}\href{../works/LuoVLBM16.pdf}{LuoVLBM16} (0.22)& \cellcolor{red!40}\href{../works/FukunagaHFAMN02.pdf}{FukunagaHFAMN02} (0.22)& \cellcolor{red!40}\href{../works/WallaceF00.pdf}{WallaceF00} (0.23)& \cellcolor{red!40}\href{../works/BonfiettiM12.pdf}{BonfiettiM12} (0.23)\\
Dot& \cellcolor{red!40}\href{../works/ZarandiASC20.pdf}{ZarandiASC20} (70.00)& \cellcolor{red!40}\href{../works/Lombardi10.pdf}{Lombardi10} (69.00)& \cellcolor{red!40}\href{../works/Beck99.pdf}{Beck99} (69.00)& \cellcolor{red!40}\href{../works/TrentesauxPT01.pdf}{TrentesauxPT01} (68.00)& \cellcolor{red!40}\href{../works/Godet21a.pdf}{Godet21a} (67.00)\\
Cosine& \cellcolor{red!40}\href{../works/CrawfordB94.pdf}{CrawfordB94} (0.74)& \cellcolor{red!40}\href{../works/LuoVLBM16.pdf}{LuoVLBM16} (0.73)& \cellcolor{red!40}\href{../works/TranPZLDB18.pdf}{TranPZLDB18} (0.72)& \cellcolor{red!40}\href{../works/BeckW04.pdf}{BeckW04} (0.72)& \cellcolor{red!40}\href{../works/BorghesiBLMB18.pdf}{BorghesiBLMB18} (0.70)\\
\index{BruckerK00}\href{../works/BruckerK00.pdf}{BruckerK00} R\&C& \cellcolor{red!40}\href{../works/DemasseyAM05.pdf}{DemasseyAM05} (0.73)& \cellcolor{red!40}\href{../works/LiessM08.pdf}{LiessM08} (0.80)& \cellcolor{red!40}\href{../works/ElkhyariGJ02a.pdf}{ElkhyariGJ02a} (0.84)& \cellcolor{red!40}\href{../works/ElkhyariGJ02.pdf}{ElkhyariGJ02} (0.86)& \cellcolor{red!20}\href{../works/BaptisteP97.pdf}{BaptisteP97} (0.89)\\
Euclid& \cellcolor{red!40}\href{../works/BofillCSV17.pdf}{BofillCSV17} (0.24)& \cellcolor{red!20}\href{../works/BofillCSV17a.pdf}{BofillCSV17a} (0.25)& \cellcolor{yellow!20}\href{../works/LombardiM13.pdf}{LombardiM13} (0.27)& \cellcolor{green!20}\href{../works/LombardiM12a.pdf}{LombardiM12a} (0.29)& \cellcolor{green!20}\href{../works/LombardiM09.pdf}{LombardiM09} (0.29)\\
Dot& \cellcolor{red!40}\href{../works/Groleaz21.pdf}{Groleaz21} (107.00)& \cellcolor{red!40}\href{../works/Baptiste02.pdf}{Baptiste02} (107.00)& \cellcolor{red!40}\href{../works/Schutt11.pdf}{Schutt11} (106.00)& \cellcolor{red!40}\href{../works/Godet21a.pdf}{Godet21a} (105.00)& \cellcolor{red!40}\href{../works/Polo-MejiaALB20.pdf}{Polo-MejiaALB20} (102.00)\\
Cosine& \cellcolor{red!40}\href{../works/BofillCSV17.pdf}{BofillCSV17} (0.80)& \cellcolor{red!40}\href{../works/BofillCSV17a.pdf}{BofillCSV17a} (0.78)& \cellcolor{red!40}\href{../works/LombardiM13.pdf}{LombardiM13} (0.75)& \cellcolor{red!40}\href{../works/Polo-MejiaALB20.pdf}{Polo-MejiaALB20} (0.73)& \cellcolor{red!40}\href{../works/LombardiM12a.pdf}{LombardiM12a} (0.73)\\
\index{BrusoniCLMMT96}\href{../works/BrusoniCLMMT96.pdf}{BrusoniCLMMT96} R\&C& \cellcolor{red!20}\href{../works/Goltz95.pdf}{Goltz95} (0.88)& \cellcolor{red!20}\href{../works/BeldiceanuC94.pdf}{BeldiceanuC94} (0.88)& \cellcolor{red!20}\href{../works/AngelsmarkJ00.pdf}{AngelsmarkJ00} (0.89)& \cellcolor{yellow!20}\href{../works/Simonis95a.pdf}{Simonis95a} (0.90)& \cellcolor{yellow!20}\href{../works/Muscettola02.pdf}{Muscettola02} (0.92)\\
Euclid& \cellcolor{red!40}\href{../works/LammaMM97.pdf}{LammaMM97} (0.23)& \cellcolor{red!20}\href{../works/Bartak02.pdf}{Bartak02} (0.26)& \cellcolor{yellow!20}\href{../works/Rit86.pdf}{Rit86} (0.27)& \cellcolor{yellow!20}\href{../works/Junker00.pdf}{Junker00} (0.28)& \cellcolor{green!20}\href{../works/LudwigKRBMS14.pdf}{LudwigKRBMS14} (0.30)\\
Dot& \cellcolor{red!40}\href{../works/ZarandiASC20.pdf}{ZarandiASC20} (97.00)& \cellcolor{red!40}\href{../works/LammaMM97.pdf}{LammaMM97} (90.00)& \cellcolor{red!40}\href{../works/Baptiste02.pdf}{Baptiste02} (89.00)& \cellcolor{red!40}\href{../works/Malapert11.pdf}{Malapert11} (88.00)& \cellcolor{red!40}\href{../works/Beck99.pdf}{Beck99} (85.00)\\
Cosine& \cellcolor{red!40}\href{../works/LammaMM97.pdf}{LammaMM97} (0.83)& \cellcolor{red!40}\href{../works/Bartak02.pdf}{Bartak02} (0.75)& \cellcolor{red!40}\href{../works/FoxS90.pdf}{FoxS90} (0.70)& \cellcolor{red!40}\href{../works/Junker00.pdf}{Junker00} (0.69)& \cellcolor{red!40}\href{../works/Rit86.pdf}{Rit86} (0.67)\\
\index{BukchinR18}\href{../works/BukchinR18.pdf}{BukchinR18} R\&C& \cellcolor{red!40}\href{../works/TopalogluSS12.pdf}{TopalogluSS12} (0.80)& \cellcolor{red!20}\href{../works/PinarbasiAY19.pdf}{PinarbasiAY19} (0.86)& \cellcolor{red!20}\href{../works/AlakaPY19.pdf}{AlakaPY19} (0.88)& \cellcolor{red!20}PinarbasiA20 (0.90)& \cellcolor{yellow!20}KizilayC20 (0.91)\\
Euclid& \cellcolor{red!20}\href{../works/AlakaPY19.pdf}{AlakaPY19} (0.24)& \cellcolor{red!20}\href{../works/NishikawaSTT18.pdf}{NishikawaSTT18} (0.25)& \cellcolor{red!20}\href{../works/LozanoCDS12.pdf}{LozanoCDS12} (0.25)& \cellcolor{red!20}\href{../works/NishikawaSTT18a.pdf}{NishikawaSTT18a} (0.26)& \cellcolor{red!20}\href{../works/AkramNHRSA23.pdf}{AkramNHRSA23} (0.26)\\
Dot& \cellcolor{red!40}\href{../works/Groleaz21.pdf}{Groleaz21} (91.00)& \cellcolor{red!40}\href{../works/Malapert11.pdf}{Malapert11} (87.00)& \cellcolor{red!40}\href{../works/Lunardi20.pdf}{Lunardi20} (84.00)& \cellcolor{red!40}\href{../works/Astrand21.pdf}{Astrand21} (83.00)& \cellcolor{red!40}\href{../works/ZarandiASC20.pdf}{ZarandiASC20} (82.00)\\
Cosine& \cellcolor{red!40}\href{../works/NishikawaSTT18.pdf}{NishikawaSTT18} (0.75)& \cellcolor{red!40}\href{../works/HeipckeCCS00.pdf}{HeipckeCCS00} (0.74)& \cellcolor{red!40}\href{../works/Ham20a.pdf}{Ham20a} (0.74)& \cellcolor{red!40}\href{../works/AlakaPY19.pdf}{AlakaPY19} (0.74)& \cellcolor{red!40}\href{../works/PinarbasiAY19.pdf}{PinarbasiAY19} (0.73)\\
\index{BulckG22}\href{../works/BulckG22.pdf}{BulckG22} R\&C& \cellcolor{red!20}\href{../works/Ribeiro12.pdf}{Ribeiro12} (0.89)& \cellcolor{yellow!20}\href{../works/SuCC13.pdf}{SuCC13} (0.91)& \cellcolor{yellow!20}\href{../works/RasmussenT07.pdf}{RasmussenT07} (0.93)& \cellcolor{yellow!20}\href{../works/NaqviAIAAA22.pdf}{NaqviAIAAA22} (0.93)& \cellcolor{green!20}\href{../works/RasmussenT06.pdf}{RasmussenT06} (0.94)\\
Euclid& \cellcolor{red!40}\href{../works/RasmussenT07.pdf}{RasmussenT07} (0.20)& \cellcolor{red!40}\href{../works/RasmussenT06.pdf}{RasmussenT06} (0.21)& \cellcolor{red!40}\href{../works/EastonNT02.pdf}{EastonNT02} (0.21)& \cellcolor{red!40}\href{../works/SuCC13.pdf}{SuCC13} (0.22)& \cellcolor{red!40}\href{../works/ZengM12.pdf}{ZengM12} (0.23)\\
Dot& \cellcolor{red!40}\href{../works/KendallKRU10.pdf}{KendallKRU10} (85.00)& \cellcolor{red!40}\href{../works/Ribeiro12.pdf}{Ribeiro12} (80.00)& \cellcolor{red!40}\href{../works/RasmussenT07.pdf}{RasmussenT07} (69.00)& \cellcolor{red!40}\href{../works/RasmussenT09.pdf}{RasmussenT09} (68.00)& \cellcolor{red!40}\href{../works/ZarandiASC20.pdf}{ZarandiASC20} (65.00)\\
Cosine& \cellcolor{red!40}\href{../works/RasmussenT07.pdf}{RasmussenT07} (0.83)& \cellcolor{red!40}\href{../works/Ribeiro12.pdf}{Ribeiro12} (0.81)& \cellcolor{red!40}\href{../works/RasmussenT06.pdf}{RasmussenT06} (0.80)& \cellcolor{red!40}\href{../works/EastonNT02.pdf}{EastonNT02} (0.78)& \cellcolor{red!40}\href{../works/RasmussenT09.pdf}{RasmussenT09} (0.78)\\
\index{BurtLPS15}\href{../works/BurtLPS15.pdf}{BurtLPS15} R\&C& \cellcolor{green!20}\href{../works/KelarevaTK13.pdf}{KelarevaTK13} (0.94)& \cellcolor{green!20}\href{../works/BofillEGPSV14.pdf}{BofillEGPSV14} (0.94)& \cellcolor{green!20}\href{../works/ColT19.pdf}{ColT19} (0.95)& \cellcolor{green!20}\href{../works/Mercier-AubinGQ20.pdf}{Mercier-AubinGQ20} (0.95)& \cellcolor{green!20}\href{../works/SzerediS16.pdf}{SzerediS16} (0.95)\\
Euclid& \cellcolor{green!20}\href{../works/KovacsV06.pdf}{KovacsV06} (0.29)& \cellcolor{green!20}\href{../works/HeipckeCCS00.pdf}{HeipckeCCS00} (0.30)& \cellcolor{green!20}\href{../works/LozanoCDS12.pdf}{LozanoCDS12} (0.30)& \cellcolor{green!20}\href{../works/LipovetzkyBPS14.pdf}{LipovetzkyBPS14} (0.31)& \cellcolor{green!20}\href{../works/BockmayrP06.pdf}{BockmayrP06} (0.31)\\
Dot& \cellcolor{red!40}\href{../works/Groleaz21.pdf}{Groleaz21} (132.00)& \cellcolor{red!40}\href{../works/Astrand21.pdf}{Astrand21} (122.00)& \cellcolor{red!40}\href{../works/Dejemeppe16.pdf}{Dejemeppe16} (117.00)& \cellcolor{red!40}\href{../works/Fahimi16.pdf}{Fahimi16} (117.00)& \cellcolor{red!40}\href{../works/Lombardi10.pdf}{Lombardi10} (116.00)\\
Cosine& \cellcolor{red!40}\href{../works/KovacsV06.pdf}{KovacsV06} (0.74)& \cellcolor{red!40}\href{../works/HeipckeCCS00.pdf}{HeipckeCCS00} (0.74)& \cellcolor{red!40}\href{../works/LipovetzkyBPS14.pdf}{LipovetzkyBPS14} (0.71)& \cellcolor{red!40}\href{../works/LozanoCDS12.pdf}{LozanoCDS12} (0.71)& \cellcolor{red!40}\href{../works/CarchraeB09.pdf}{CarchraeB09} (0.71)\\
\index{Caballero19}\href{../works/Caballero19.pdf}{Caballero19} R\&C\\
Euclid& \cellcolor{black!20}\href{../works/SchuttFS13a.pdf}{SchuttFS13a} (0.37)& \cellcolor{black!20}\href{../works/abs-1009-0347.pdf}{abs-1009-0347} (0.37)& \href{../works/SchuttFSW13.pdf}{SchuttFSW13} (0.38)& \href{../works/SchuttFSW11.pdf}{SchuttFSW11} (0.40)& \href{../works/SchnellH15.pdf}{SchnellH15} (0.40)\\
Dot& \cellcolor{red!40}\href{../works/Schutt11.pdf}{Schutt11} (213.00)& \cellcolor{red!40}\href{../works/Godet21a.pdf}{Godet21a} (208.00)& \cellcolor{red!40}\href{../works/Lombardi10.pdf}{Lombardi10} (199.00)& \cellcolor{red!40}\href{../works/Dejemeppe16.pdf}{Dejemeppe16} (194.00)& \cellcolor{red!40}\href{../works/Baptiste02.pdf}{Baptiste02} (192.00)\\
Cosine& \cellcolor{red!40}\href{../works/SchuttFS13a.pdf}{SchuttFS13a} (0.79)& \cellcolor{red!40}\href{../works/abs-1009-0347.pdf}{abs-1009-0347} (0.79)& \cellcolor{red!40}\href{../works/SchuttFSW13.pdf}{SchuttFSW13} (0.77)& \cellcolor{red!40}\href{../works/SchuttFSW11.pdf}{SchuttFSW11} (0.76)& \cellcolor{red!40}\href{../works/BofillCSV17.pdf}{BofillCSV17} (0.75)\\
\index{Caballero23}\href{../works/Caballero23.pdf}{Caballero23} R\&C\\
Euclid& \cellcolor{red!40}\href{../works/Baptiste09.pdf}{Baptiste09} (0.12)& \cellcolor{red!40}\href{../works/KovacsEKV05.pdf}{KovacsEKV05} (0.13)& \cellcolor{red!40}\href{../works/CestaOS98.pdf}{CestaOS98} (0.13)& \cellcolor{red!40}\href{../works/CarchraeBF05.pdf}{CarchraeBF05} (0.14)& \cellcolor{red!40}\href{../works/AngelsmarkJ00.pdf}{AngelsmarkJ00} (0.15)\\
Dot& \cellcolor{red!40}\href{../works/PovedaAA23.pdf}{PovedaAA23} (30.00)& \cellcolor{red!40}\href{../works/HillTV21.pdf}{HillTV21} (30.00)& \cellcolor{red!40}\href{../works/HeinzSB13.pdf}{HeinzSB13} (30.00)& \cellcolor{red!40}\href{../works/LombardiM12.pdf}{LombardiM12} (30.00)& \cellcolor{red!40}\href{../works/SubulanC22.pdf}{SubulanC22} (30.00)\\
Cosine& \cellcolor{red!40}\href{../works/LombardiM13.pdf}{LombardiM13} (0.78)& \cellcolor{red!40}\href{../works/Baptiste09.pdf}{Baptiste09} (0.75)& \cellcolor{red!40}\href{../works/KovacsEKV05.pdf}{KovacsEKV05} (0.74)& \cellcolor{red!40}\href{../works/CestaOS98.pdf}{CestaOS98} (0.72)& \cellcolor{red!40}\href{../works/BhatnagarKL19.pdf}{BhatnagarKL19} (0.71)\\
\index{CambazardHDJT04}\href{../works/CambazardHDJT04.pdf}{CambazardHDJT04} R\&C& \cellcolor{red!40}\href{../works/Hooker04.pdf}{Hooker04} (0.68)& \cellcolor{red!40}\href{../works/Hooker05a.pdf}{Hooker05a} (0.70)& \cellcolor{red!40}\href{../works/HladikCDJ08.pdf}{HladikCDJ08} (0.70)& \cellcolor{red!40}\href{../works/CambazardJ05.pdf}{CambazardJ05} (0.71)& \cellcolor{red!40}\href{../works/Hooker05.pdf}{Hooker05} (0.75)\\
Euclid& \cellcolor{red!40}\href{../works/HladikCDJ08.pdf}{HladikCDJ08} (0.20)& \cellcolor{green!20}\href{../works/BeniniLMR08.pdf}{BeniniLMR08} (0.29)& \cellcolor{green!20}\href{../works/AkramNHRSA23.pdf}{AkramNHRSA23} (0.30)& \cellcolor{green!20}\href{../works/EmeretlisTAV17.pdf}{EmeretlisTAV17} (0.31)& \cellcolor{green!20}\href{../works/NishikawaSTT19.pdf}{NishikawaSTT19} (0.31)\\
Dot& \cellcolor{red!40}\href{../works/Lombardi10.pdf}{Lombardi10} (124.00)& \cellcolor{red!40}\href{../works/ZarandiASC20.pdf}{ZarandiASC20} (112.00)& \cellcolor{red!40}\href{../works/HladikCDJ08.pdf}{HladikCDJ08} (112.00)& \cellcolor{red!40}\href{../works/Godet21a.pdf}{Godet21a} (106.00)& \cellcolor{red!40}\href{../works/Groleaz21.pdf}{Groleaz21} (106.00)\\
Cosine& \cellcolor{red!40}\href{../works/HladikCDJ08.pdf}{HladikCDJ08} (0.89)& \cellcolor{red!40}\href{../works/BeniniLMR08.pdf}{BeniniLMR08} (0.74)& \cellcolor{red!40}\href{../works/EmeretlisTAV17.pdf}{EmeretlisTAV17} (0.73)& \cellcolor{red!40}\href{../works/AlesioBNG15.pdf}{AlesioBNG15} (0.71)& \cellcolor{red!40}\href{../works/BeniniBGM05.pdf}{BeniniBGM05} (0.71)\\
\index{CambazardJ05}\href{../works/CambazardJ05.pdf}{CambazardJ05} R\&C& \cellcolor{red!40}\href{../works/Hooker05a.pdf}{Hooker05a} (0.58)& \cellcolor{red!40}\href{../works/Hooker05.pdf}{Hooker05} (0.63)& \cellcolor{red!40}\href{../works/BeniniBGM06.pdf}{BeniniBGM06} (0.68)& \cellcolor{red!40}\href{../works/Hooker04.pdf}{Hooker04} (0.68)& \cellcolor{red!40}\href{../works/CambazardHDJT04.pdf}{CambazardHDJT04} (0.71)\\
Euclid& \cellcolor{red!40}\href{../works/Vilim03.pdf}{Vilim03} (0.20)& \cellcolor{red!40}\href{../works/Baptiste09.pdf}{Baptiste09} (0.21)& \cellcolor{red!40}\href{../works/AbrilSB05.pdf}{AbrilSB05} (0.23)& \cellcolor{red!40}\href{../works/BenoistGR02.pdf}{BenoistGR02} (0.23)& \cellcolor{red!40}\href{../works/Hooker05b.pdf}{Hooker05b} (0.23)\\
Dot& \cellcolor{red!40}\href{../works/HookerH17.pdf}{HookerH17} (45.00)& \cellcolor{red!40}\href{../works/Lombardi10.pdf}{Lombardi10} (45.00)& \cellcolor{red!40}\href{../works/HladikCDJ08.pdf}{HladikCDJ08} (44.00)& \cellcolor{red!40}\href{../works/Hooker05.pdf}{Hooker05} (43.00)& \cellcolor{red!40}\href{../works/Froger16.pdf}{Froger16} (41.00)\\
Cosine& \cellcolor{red!40}\href{../works/CambazardHDJT04.pdf}{CambazardHDJT04} (0.63)& \cellcolor{red!40}\href{../works/HladikCDJ08.pdf}{HladikCDJ08} (0.62)& \cellcolor{red!40}\href{../works/BenoistGR02.pdf}{BenoistGR02} (0.62)& \cellcolor{red!40}\href{../works/Hooker05b.pdf}{Hooker05b} (0.62)& \cellcolor{red!40}\href{../works/BeniniLMMR08.pdf}{BeniniLMMR08} (0.61)\\
\index{CampeauG22}\href{../works/CampeauG22.pdf}{CampeauG22} R\&C& \cellcolor{red!20}\href{../works/AstrandJZ20.pdf}{AstrandJZ20} (0.89)& \cellcolor{green!20}EdwardsBSE19 (0.94)& \cellcolor{green!20}\href{../works/HauderBRPA20.pdf}{HauderBRPA20} (0.94)& \cellcolor{green!20}\href{../works/GuSW12.pdf}{GuSW12} (0.95)& \cellcolor{green!20}\href{../works/SchnellH15.pdf}{SchnellH15} (0.95)\\
Euclid& \cellcolor{red!20}\href{../works/BhatnagarKL19.pdf}{BhatnagarKL19} (0.24)& \cellcolor{red!20}\href{../works/BofillCSV17a.pdf}{BofillCSV17a} (0.26)& \cellcolor{red!20}\href{../works/LombardiM09.pdf}{LombardiM09} (0.26)& \cellcolor{yellow!20}\href{../works/LombardiM10.pdf}{LombardiM10} (0.26)& \cellcolor{yellow!20}\href{../works/NishikawaSTT18a.pdf}{NishikawaSTT18a} (0.27)\\
Dot& \cellcolor{red!40}\href{../works/Lombardi10.pdf}{Lombardi10} (120.00)& \cellcolor{red!40}\href{../works/LaborieRSV18.pdf}{LaborieRSV18} (119.00)& \cellcolor{red!40}\href{../works/Groleaz21.pdf}{Groleaz21} (115.00)& \cellcolor{red!40}\href{../works/Schutt11.pdf}{Schutt11} (113.00)& \cellcolor{red!40}\href{../works/Astrand21.pdf}{Astrand21} (109.00)\\
Cosine& \cellcolor{red!40}\href{../works/BhatnagarKL19.pdf}{BhatnagarKL19} (0.80)& \cellcolor{red!40}\href{../works/BofillCSV17a.pdf}{BofillCSV17a} (0.78)& \cellcolor{red!40}\href{../works/LombardiM10.pdf}{LombardiM10} (0.78)& \cellcolor{red!40}\href{../works/LombardiM09.pdf}{LombardiM09} (0.78)& \cellcolor{red!40}\href{../works/NishikawaSTT18a.pdf}{NishikawaSTT18a} (0.76)\\
\index{CappartS17}\href{../works/CappartS17.pdf}{CappartS17} R\&C& \cellcolor{red!40}\href{../works/GilesH16.pdf}{GilesH16} (0.72)& \cellcolor{red!40}\href{../works/FrankDT16.pdf}{FrankDT16} (0.83)& \cellcolor{red!20}\href{../works/TangB20.pdf}{TangB20} (0.88)& \cellcolor{red!20}\href{../works/LaborieR14.pdf}{LaborieR14} (0.89)& \cellcolor{yellow!20}\href{../works/GedikKBR17.pdf}{GedikKBR17} (0.91)\\
Euclid& \cellcolor{green!20}\href{../works/Rodriguez07b.pdf}{Rodriguez07b} (0.29)& \cellcolor{green!20}\href{../works/RodriguezS09.pdf}{RodriguezS09} (0.29)& \cellcolor{green!20}\href{../works/RodriguezDG02.pdf}{RodriguezDG02} (0.30)& \cellcolor{green!20}\href{../works/MorgadoM97.pdf}{MorgadoM97} (0.30)& \cellcolor{green!20}\href{../works/BidotVLB07.pdf}{BidotVLB07} (0.31)\\
Dot& \cellcolor{red!40}\href{../works/ZarandiASC20.pdf}{ZarandiASC20} (116.00)& \cellcolor{red!40}\href{../works/Dejemeppe16.pdf}{Dejemeppe16} (108.00)& \cellcolor{red!40}\href{../works/Groleaz21.pdf}{Groleaz21} (107.00)& \cellcolor{red!40}\href{../works/LaborieRSV18.pdf}{LaborieRSV18} (106.00)& \cellcolor{red!40}\href{../works/Lombardi10.pdf}{Lombardi10} (106.00)\\
Cosine& \cellcolor{red!40}\href{../works/Rodriguez07b.pdf}{Rodriguez07b} (0.75)& \cellcolor{red!40}\href{../works/RodriguezS09.pdf}{RodriguezS09} (0.75)& \cellcolor{red!40}\href{../works/MorgadoM97.pdf}{MorgadoM97} (0.72)& \cellcolor{red!40}\href{../works/MarliereSPR23.pdf}{MarliereSPR23} (0.71)& \cellcolor{red!40}\href{../works/Rodriguez07.pdf}{Rodriguez07} (0.70)\\
\index{CappartTSR18}\href{../works/CappartTSR18.pdf}{CappartTSR18} R\&C& \cellcolor{red!40}\href{../works/CauwelaertLS18.pdf}{CauwelaertLS18} (0.82)& \cellcolor{red!40}\href{../works/DejemeppeCS15.pdf}{DejemeppeCS15} (0.82)& \cellcolor{red!40}\href{../works/ThomasKS20.pdf}{ThomasKS20} (0.85)& \cellcolor{red!20}\href{../works/GayHLS15.pdf}{GayHLS15} (0.87)& \cellcolor{red!20}\href{../works/Laborie18a.pdf}{Laborie18a} (0.89)\\
Euclid& \cellcolor{red!40}\href{../works/ThomasKS20.pdf}{ThomasKS20} (0.22)& \cellcolor{blue!20}\href{../works/GoelSHFS15.pdf}{GoelSHFS15} (0.33)& \cellcolor{blue!20}\href{../works/DejemeppeD14.pdf}{DejemeppeD14} (0.33)& \cellcolor{blue!20}\href{../works/ZibranR11a.pdf}{ZibranR11a} (0.33)& \cellcolor{black!20}\href{../works/BoothNB16.pdf}{BoothNB16} (0.34)\\
Dot& \cellcolor{red!40}\href{../works/Dejemeppe16.pdf}{Dejemeppe16} (116.00)& \cellcolor{red!40}\href{../works/LaborieRSV18.pdf}{LaborieRSV18} (111.00)& \cellcolor{red!40}\href{../works/Groleaz21.pdf}{Groleaz21} (101.00)& \cellcolor{red!40}\href{../works/Lombardi10.pdf}{Lombardi10} (96.00)& \cellcolor{red!40}\href{../works/ColT22.pdf}{ColT22} (95.00)\\
Cosine& \cellcolor{red!40}\href{../works/ThomasKS20.pdf}{ThomasKS20} (0.86)& \cellcolor{red!40}\href{../works/GoelSHFS15.pdf}{GoelSHFS15} (0.71)& \cellcolor{red!40}\href{../works/DejemeppeD14.pdf}{DejemeppeD14} (0.68)& \cellcolor{red!40}\href{../works/TranVNB17.pdf}{TranVNB17} (0.65)& \cellcolor{red!40}\href{../works/GaySS14.pdf}{GaySS14} (0.65)\\
\index{CarchraeB09}\href{../works/CarchraeB09.pdf}{CarchraeB09} R\&C& \cellcolor{red!40}\href{../works/PerronSF04.pdf}{PerronSF04} (0.79)& \cellcolor{red!40}\href{../works/BeckFW11.pdf}{BeckFW11} (0.80)& \cellcolor{red!40}\href{../works/GrimesH15.pdf}{GrimesH15} (0.81)& \cellcolor{red!40}\href{../works/DannaP03.pdf}{DannaP03} (0.84)& \cellcolor{red!40}\href{../works/SchausHMCMD11.pdf}{SchausHMCMD11} (0.86)\\
Euclid& \cellcolor{red!40}\href{../works/WatsonB08.pdf}{WatsonB08} (0.23)& \cellcolor{red!40}\href{../works/BeckFW11.pdf}{BeckFW11} (0.23)& \cellcolor{red!40}\href{../works/Beck06.pdf}{Beck06} (0.23)& \cellcolor{red!20}\href{../works/FontaineMH16.pdf}{FontaineMH16} (0.26)& \cellcolor{red!20}\href{../works/KovacsV06.pdf}{KovacsV06} (0.26)\\
Dot& \cellcolor{red!40}\href{../works/Dejemeppe16.pdf}{Dejemeppe16} (129.00)& \cellcolor{red!40}\href{../works/LaborieRSV18.pdf}{LaborieRSV18} (126.00)& \cellcolor{red!40}\href{../works/Groleaz21.pdf}{Groleaz21} (126.00)& \cellcolor{red!40}\href{../works/ColT22.pdf}{ColT22} (119.00)& \cellcolor{red!40}\href{../works/GrimesH15.pdf}{GrimesH15} (119.00)\\
Cosine& \cellcolor{red!40}\href{../works/BeckFW11.pdf}{BeckFW11} (0.85)& \cellcolor{red!40}\href{../works/WatsonB08.pdf}{WatsonB08} (0.85)& \cellcolor{red!40}\href{../works/Beck06.pdf}{Beck06} (0.83)& \cellcolor{red!40}\href{../works/FontaineMH16.pdf}{FontaineMH16} (0.81)& \cellcolor{red!40}\href{../works/KovacsV06.pdf}{KovacsV06} (0.80)\\
\index{CarchraeBF05}\href{../works/CarchraeBF05.pdf}{CarchraeBF05} R\&C\\
Euclid& \cellcolor{red!40}\href{../works/Baptiste09.pdf}{Baptiste09} (0.09)& \cellcolor{red!40}\href{../works/AngelsmarkJ00.pdf}{AngelsmarkJ00} (0.10)& \cellcolor{red!40}\href{../works/LiuJ06.pdf}{LiuJ06} (0.13)& \cellcolor{red!40}\href{../works/FrostD98.pdf}{FrostD98} (0.13)& \cellcolor{red!40}\href{../works/Caballero23.pdf}{Caballero23} (0.14)\\
Dot& \cellcolor{red!40}\href{../works/LombardiBM15.pdf}{LombardiBM15} (22.00)& \cellcolor{red!40}\href{../works/Astrand0F21.pdf}{Astrand0F21} (22.00)& \cellcolor{red!40}\href{../works/Astrand21.pdf}{Astrand21} (22.00)& \cellcolor{red!40}\href{../works/JuvinHHL23.pdf}{JuvinHHL23} (21.00)& \cellcolor{red!40}\href{../works/PovedaAA23.pdf}{PovedaAA23} (21.00)\\
Cosine& \cellcolor{red!40}\href{../works/AngelsmarkJ00.pdf}{AngelsmarkJ00} (0.80)& \cellcolor{red!40}\href{../works/Baptiste09.pdf}{Baptiste09} (0.76)& \cellcolor{red!40}\href{../works/LiuJ06.pdf}{LiuJ06} (0.72)& \cellcolor{red!40}\href{../works/FortinZDF05.pdf}{FortinZDF05} (0.71)& \cellcolor{red!40}\href{../works/Hunsberger08.pdf}{Hunsberger08} (0.69)\\
\index{CarlierPSJ20}\href{../works/CarlierPSJ20.pdf}{CarlierPSJ20} R\&C& \cellcolor{red!40}CarlierSJP21 (0.67)& \cellcolor{red!40}\href{../works/Tesch18.pdf}{Tesch18} (0.70)& \cellcolor{red!40}\href{../works/OuelletQ18.pdf}{OuelletQ18} (0.71)& \cellcolor{red!40}\href{../works/ArkhipovBL19.pdf}{ArkhipovBL19} (0.80)& \cellcolor{red!40}\href{../works/YangSS19.pdf}{YangSS19} (0.82)\\
Euclid& \cellcolor{red!40}\href{../works/HanenKP21.pdf}{HanenKP21} (0.22)& \cellcolor{red!20}\href{../works/Caseau97.pdf}{Caseau97} (0.26)& \cellcolor{yellow!20}\href{../works/ArtiguesL14.pdf}{ArtiguesL14} (0.27)& \cellcolor{yellow!20}\href{../works/NattafAL15.pdf}{NattafAL15} (0.27)& \cellcolor{yellow!20}\href{../works/Limtanyakul07.pdf}{Limtanyakul07} (0.27)\\
Dot& \cellcolor{red!40}\href{../works/Baptiste02.pdf}{Baptiste02} (111.00)& \cellcolor{red!40}\href{../works/Lombardi10.pdf}{Lombardi10} (107.00)& \cellcolor{red!40}\href{../works/HanenKP21.pdf}{HanenKP21} (102.00)& \cellcolor{red!40}\href{../works/Fahimi16.pdf}{Fahimi16} (102.00)& \cellcolor{red!40}\href{../works/BaptistePN99.pdf}{BaptistePN99} (102.00)\\
Cosine& \cellcolor{red!40}\href{../works/HanenKP21.pdf}{HanenKP21} (0.87)& \cellcolor{red!40}\href{../works/BaptistePN99.pdf}{BaptistePN99} (0.79)& \cellcolor{red!40}\href{../works/NattafAL15.pdf}{NattafAL15} (0.77)& \cellcolor{red!40}\href{../works/Tesch18.pdf}{Tesch18} (0.77)& \cellcolor{red!40}\href{../works/ArtiguesL14.pdf}{ArtiguesL14} (0.76)\\
\index{CarlierSJP21}CarlierSJP21 R\&C& \cellcolor{red!40}\href{../works/CarlierPSJ20.pdf}{CarlierPSJ20} (0.67)& \cellcolor{red!40}\href{../works/Tesch16.pdf}{Tesch16} (0.76)& \cellcolor{red!40}\href{../works/FahimiOQ18.pdf}{FahimiOQ18} (0.78)& \cellcolor{red!40}EdwardsBSE19 (0.79)& \cellcolor{red!40}\href{../works/OuelletQ18.pdf}{OuelletQ18} (0.79)\\
Euclid\\
Dot\\
Cosine\\
\index{CarlssonJL17}\href{../works/CarlssonJL17.pdf}{CarlssonJL17} R\&C& \cellcolor{red!40}\href{../works/LarsonJC14.pdf}{LarsonJC14} (0.64)& \cellcolor{red!40}\href{../works/SuCC13.pdf}{SuCC13} (0.74)& \cellcolor{red!40}\href{../works/RasmussenT06.pdf}{RasmussenT06} (0.77)& \cellcolor{red!40}\href{../works/ZengM12.pdf}{ZengM12} (0.80)& \cellcolor{red!40}\href{../works/RasmussenT09.pdf}{RasmussenT09} (0.83)\\
Euclid& \cellcolor{red!20}\href{../works/LarsonJC14.pdf}{LarsonJC14} (0.25)& \cellcolor{blue!20}\href{../works/ZengM12.pdf}{ZengM12} (0.32)& \cellcolor{blue!20}\href{../works/LiuLH18.pdf}{LiuLH18} (0.33)& \cellcolor{blue!20}\href{../works/RussellU06.pdf}{RussellU06} (0.33)& \cellcolor{blue!20}\href{../works/Perron05.pdf}{Perron05} (0.34)\\
Dot& \cellcolor{red!40}\href{../works/Godet21a.pdf}{Godet21a} (107.00)& \cellcolor{red!40}\href{../works/Dejemeppe16.pdf}{Dejemeppe16} (93.00)& \cellcolor{red!40}\href{../works/LarsonJC14.pdf}{LarsonJC14} (93.00)& \cellcolor{red!40}\href{../works/Siala15a.pdf}{Siala15a} (90.00)& \cellcolor{red!40}\href{../works/HookerH17.pdf}{HookerH17} (84.00)\\
Cosine& \cellcolor{red!40}\href{../works/LarsonJC14.pdf}{LarsonJC14} (0.83)& \cellcolor{red!40}\href{../works/RussellU06.pdf}{RussellU06} (0.68)& \cellcolor{red!40}\href{../works/ZengM12.pdf}{ZengM12} (0.68)& \cellcolor{red!40}\href{../works/LiuLH18.pdf}{LiuLH18} (0.67)& \cellcolor{red!40}\href{../works/Perron05.pdf}{Perron05} (0.65)\\
\index{CarlssonKA99}\href{../works/CarlssonKA99.pdf}{CarlssonKA99} R\&C& \cellcolor{red!40}\href{../works/ArtiguesBF04.pdf}{ArtiguesBF04} (0.83)& \cellcolor{red!40}\href{../works/BeckF00a.pdf}{BeckF00a} (0.85)& \cellcolor{red!20}\href{../works/BeckDF97.pdf}{BeckDF97} (0.87)& \cellcolor{red!20}\href{../works/Colombani96.pdf}{Colombani96} (0.88)& \cellcolor{red!20}\href{../works/Wolf03.pdf}{Wolf03} (0.90)\\
Euclid& \cellcolor{red!40}\href{../works/LiuJ06.pdf}{LiuJ06} (0.23)& \cellcolor{red!40}\href{../works/KengY89.pdf}{KengY89} (0.23)& \cellcolor{red!40}\href{../works/LudwigKRBMS14.pdf}{LudwigKRBMS14} (0.24)& \cellcolor{red!20}\href{../works/FoxAS82.pdf}{FoxAS82} (0.25)& \cellcolor{red!20}\href{../works/CrawfordB94.pdf}{CrawfordB94} (0.25)\\
Dot& \cellcolor{red!40}\href{../works/Malapert11.pdf}{Malapert11} (90.00)& \cellcolor{red!40}\href{../works/ZarandiASC20.pdf}{ZarandiASC20} (87.00)& \cellcolor{red!40}\href{../works/Groleaz21.pdf}{Groleaz21} (87.00)& \cellcolor{red!40}\href{../works/Lunardi20.pdf}{Lunardi20} (86.00)& \cellcolor{red!40}\href{../works/Dejemeppe16.pdf}{Dejemeppe16} (84.00)\\
Cosine& \cellcolor{red!40}\href{../works/KengY89.pdf}{KengY89} (0.73)& \cellcolor{red!40}\href{../works/DraperJCJ99.pdf}{DraperJCJ99} (0.71)& \cellcolor{red!40}\href{../works/LiuJ06.pdf}{LiuJ06} (0.70)& \cellcolor{red!40}\href{../works/LudwigKRBMS14.pdf}{LudwigKRBMS14} (0.68)& \cellcolor{red!40}\href{../works/BockmayrP06.pdf}{BockmayrP06} (0.68)\\
\index{Caseau97}\href{../works/Caseau97.pdf}{Caseau97} R\&C\\
Euclid& \cellcolor{red!40}\href{../works/AngelsmarkJ00.pdf}{AngelsmarkJ00} (0.18)& \cellcolor{red!40}\href{../works/Puget95.pdf}{Puget95} (0.20)& \cellcolor{red!40}\href{../works/CrawfordB94.pdf}{CrawfordB94} (0.20)& \cellcolor{red!40}\href{../works/WolfS05.pdf}{WolfS05} (0.21)& \cellcolor{red!40}\href{../works/Rit86.pdf}{Rit86} (0.21)\\
Dot& \cellcolor{red!40}\href{../works/Dejemeppe16.pdf}{Dejemeppe16} (77.00)& \cellcolor{red!40}\href{../works/Fahimi16.pdf}{Fahimi16} (75.00)& \cellcolor{red!40}\href{../works/Beck99.pdf}{Beck99} (75.00)& \cellcolor{red!40}\href{../works/Baptiste02.pdf}{Baptiste02} (74.00)& \cellcolor{red!40}\href{../works/FahimiOQ18.pdf}{FahimiOQ18} (73.00)\\
Cosine& \cellcolor{red!40}\href{../works/KovacsV04.pdf}{KovacsV04} (0.78)& \cellcolor{red!40}\href{../works/AngelsmarkJ00.pdf}{AngelsmarkJ00} (0.78)& \cellcolor{red!40}\href{../works/KovacsV06.pdf}{KovacsV06} (0.77)& \cellcolor{red!40}\href{../works/WolfS05.pdf}{WolfS05} (0.77)& \cellcolor{red!40}\href{../works/HeipckeCCS00.pdf}{HeipckeCCS00} (0.76)\\
\index{CastroGR10}CastroGR10 R\&C& \cellcolor{red!40}\href{../works/JainG01.pdf}{JainG01} (0.81)& \cellcolor{red!20}\href{../works/HookerH17.pdf}{HookerH17} (0.88)& \cellcolor{red!20}\href{../works/CobanH11.pdf}{CobanH11} (0.88)& \cellcolor{red!20}Hooker06a (0.89)& \cellcolor{yellow!20}\href{../works/CobanH10.pdf}{CobanH10} (0.91)\\
Euclid\\
Dot\\
Cosine\\
\index{CatusseCBL16}\href{../works/CatusseCBL16.pdf}{CatusseCBL16} R\&C\\
Euclid& \cellcolor{red!20}\href{../works/BockmayrP06.pdf}{BockmayrP06} (0.25)& \cellcolor{red!20}\href{../works/Sadykov04.pdf}{Sadykov04} (0.26)& \cellcolor{red!20}\href{../works/BenediktSMVH18.pdf}{BenediktSMVH18} (0.26)& \cellcolor{red!20}\href{../works/Limtanyakul07.pdf}{Limtanyakul07} (0.26)& \cellcolor{yellow!20}\href{../works/Colombani96.pdf}{Colombani96} (0.27)\\
Dot& \cellcolor{red!40}\href{../works/Groleaz21.pdf}{Groleaz21} (97.00)& \cellcolor{red!40}\href{../works/ZarandiASC20.pdf}{ZarandiASC20} (86.00)& \cellcolor{red!40}\href{../works/MilanoW06.pdf}{MilanoW06} (84.00)& \cellcolor{red!40}\href{../works/Baptiste02.pdf}{Baptiste02} (83.00)& \cellcolor{red!40}\href{../works/MilanoW09.pdf}{MilanoW09} (82.00)\\
Cosine& \cellcolor{red!40}\href{../works/Colombani96.pdf}{Colombani96} (0.73)& \cellcolor{red!40}\href{../works/JainG01.pdf}{JainG01} (0.72)& \cellcolor{red!40}\href{../works/BockmayrP06.pdf}{BockmayrP06} (0.72)& \cellcolor{red!40}\href{../works/Sadykov04.pdf}{Sadykov04} (0.70)& \cellcolor{red!40}\href{../works/BenediktSMVH18.pdf}{BenediktSMVH18} (0.70)\\
\index{CauwelaertDMS16}\href{../works/CauwelaertDMS16.pdf}{CauwelaertDMS16} R\&C& \cellcolor{red!40}\href{../works/CauwelaertDS20.pdf}{CauwelaertDS20} (0.67)& \cellcolor{red!40}\href{../works/DejemeppeCS15.pdf}{DejemeppeCS15} (0.74)& \cellcolor{red!40}\href{../works/GrimesH15.pdf}{GrimesH15} (0.85)& \cellcolor{red!20}\href{../works/GrimesH10.pdf}{GrimesH10} (0.87)& \cellcolor{red!20}\href{../works/MurinR19.pdf}{MurinR19} (0.88)\\
Euclid& \cellcolor{red!40}\href{../works/CauwelaertDS20.pdf}{CauwelaertDS20} (0.17)& \cellcolor{red!40}\href{../works/DejemeppeCS15.pdf}{DejemeppeCS15} (0.19)& \cellcolor{red!40}\href{../works/VilimBC04.pdf}{VilimBC04} (0.20)& \cellcolor{red!40}\href{../works/VilimBC05.pdf}{VilimBC05} (0.21)& \cellcolor{red!40}\href{../works/Vilim04.pdf}{Vilim04} (0.21)\\
Dot& \cellcolor{red!40}\href{../works/Dejemeppe16.pdf}{Dejemeppe16} (131.00)& \cellcolor{red!40}\href{../works/Baptiste02.pdf}{Baptiste02} (129.00)& \cellcolor{red!40}\href{../works/Malapert11.pdf}{Malapert11} (124.00)& \cellcolor{red!40}\href{../works/Fahimi16.pdf}{Fahimi16} (123.00)& \cellcolor{red!40}\href{../works/Schutt11.pdf}{Schutt11} (117.00)\\
Cosine& \cellcolor{red!40}\href{../works/CauwelaertDS20.pdf}{CauwelaertDS20} (0.93)& \cellcolor{red!40}\href{../works/DejemeppeCS15.pdf}{DejemeppeCS15} (0.91)& \cellcolor{red!40}\href{../works/VilimBC05.pdf}{VilimBC05} (0.87)& \cellcolor{red!40}\href{../works/VilimBC04.pdf}{VilimBC04} (0.86)& \cellcolor{red!40}\href{../works/Vilim04.pdf}{Vilim04} (0.84)\\
\index{CauwelaertDS20}\href{../works/CauwelaertDS20.pdf}{CauwelaertDS20} R\&C& \cellcolor{red!40}\href{../works/CauwelaertDMS16.pdf}{CauwelaertDMS16} (0.67)& \cellcolor{red!40}\href{../works/DejemeppeCS15.pdf}{DejemeppeCS15} (0.75)& \cellcolor{red!40}\href{../works/SchuttS16.pdf}{SchuttS16} (0.80)& \cellcolor{red!40}\href{../works/WolfS05a.pdf}{WolfS05a} (0.82)& \cellcolor{red!40}\href{../works/MurinR19.pdf}{MurinR19} (0.86)\\
Euclid& \cellcolor{red!40}\href{../works/CauwelaertDMS16.pdf}{CauwelaertDMS16} (0.17)& \cellcolor{red!40}\href{../works/DejemeppeCS15.pdf}{DejemeppeCS15} (0.19)& \cellcolor{red!40}\href{../works/VilimBC05.pdf}{VilimBC05} (0.21)& \cellcolor{red!20}\href{../works/Vilim04.pdf}{Vilim04} (0.25)& \cellcolor{red!20}\href{../works/VilimBC04.pdf}{VilimBC04} (0.25)\\
Dot& \cellcolor{red!40}\href{../works/Dejemeppe16.pdf}{Dejemeppe16} (170.00)& \cellcolor{red!40}\href{../works/Malapert11.pdf}{Malapert11} (157.00)& \cellcolor{red!40}\href{../works/Baptiste02.pdf}{Baptiste02} (157.00)& \cellcolor{red!40}\href{../works/Fahimi16.pdf}{Fahimi16} (151.00)& \cellcolor{red!40}\href{../works/Schutt11.pdf}{Schutt11} (146.00)\\
Cosine& \cellcolor{red!40}\href{../works/CauwelaertDMS16.pdf}{CauwelaertDMS16} (0.93)& \cellcolor{red!40}\href{../works/DejemeppeCS15.pdf}{DejemeppeCS15} (0.91)& \cellcolor{red!40}\href{../works/VilimBC05.pdf}{VilimBC05} (0.89)& \cellcolor{red!40}\href{../works/Vilim04.pdf}{Vilim04} (0.86)& \cellcolor{red!40}\href{../works/VilimBC04.pdf}{VilimBC04} (0.84)\\
\index{CauwelaertLS15}\href{../works/CauwelaertLS15.pdf}{CauwelaertLS15} R\&C& \cellcolor{red!40}\href{../works/GayHLS15.pdf}{GayHLS15} (0.81)& \cellcolor{red!40}\href{../works/DejemeppeCS15.pdf}{DejemeppeCS15} (0.82)& \cellcolor{red!40}\href{../works/HoundjiSWD14.pdf}{HoundjiSWD14} (0.82)& \cellcolor{red!40}\href{../works/GaySS14.pdf}{GaySS14} (0.84)& \cellcolor{red!20}\href{../works/SimonisH11.pdf}{SimonisH11} (0.87)\\
Euclid& \cellcolor{red!40}\href{../works/CauwelaertLS18.pdf}{CauwelaertLS18} (0.23)& \cellcolor{yellow!20}\href{../works/BandaSC11.pdf}{BandaSC11} (0.26)& \cellcolor{yellow!20}\href{../works/WolfS05.pdf}{WolfS05} (0.27)& \cellcolor{yellow!20}\href{../works/OuelletQ18.pdf}{OuelletQ18} (0.28)& \cellcolor{yellow!20}\href{../works/DerrienP14.pdf}{DerrienP14} (0.28)\\
Dot& \cellcolor{red!40}\href{../works/CauwelaertLS18.pdf}{CauwelaertLS18} (86.00)& \cellcolor{red!40}\href{../works/Lombardi10.pdf}{Lombardi10} (82.00)& \cellcolor{red!40}\href{../works/Godet21a.pdf}{Godet21a} (81.00)& \cellcolor{red!40}\href{../works/Schutt11.pdf}{Schutt11} (80.00)& \cellcolor{red!40}\href{../works/Fahimi16.pdf}{Fahimi16} (77.00)\\
Cosine& \cellcolor{red!40}\href{../works/CauwelaertLS18.pdf}{CauwelaertLS18} (0.86)& \cellcolor{red!40}\href{../works/ClautiauxJCM08.pdf}{ClautiauxJCM08} (0.73)& \cellcolor{red!40}\href{../works/OuelletQ18.pdf}{OuelletQ18} (0.73)& \cellcolor{red!40}\href{../works/HeinzS11.pdf}{HeinzS11} (0.70)& \cellcolor{red!40}\href{../works/LetortCB13.pdf}{LetortCB13} (0.69)\\
\index{CauwelaertLS18}\href{../works/CauwelaertLS18.pdf}{CauwelaertLS18} R\&C& \cellcolor{red!40}\href{../works/CappartTSR18.pdf}{CappartTSR18} (0.82)& \cellcolor{red!20}\href{../works/KameugneFGOQ18.pdf}{KameugneFGOQ18} (0.86)& \cellcolor{red!20}\href{../works/GayHS15a.pdf}{GayHS15a} (0.88)& \cellcolor{red!20}\href{../works/FetgoD22.pdf}{FetgoD22} (0.90)& \cellcolor{yellow!20}\href{../works/OuelletQ18.pdf}{OuelletQ18} (0.91)\\
Euclid& \cellcolor{red!40}\href{../works/CauwelaertLS15.pdf}{CauwelaertLS15} (0.23)& \cellcolor{blue!20}\href{../works/Vilim11.pdf}{Vilim11} (0.32)& \cellcolor{blue!20}\href{../works/GayHS15a.pdf}{GayHS15a} (0.32)& \cellcolor{blue!20}\href{../works/OuelletQ18.pdf}{OuelletQ18} (0.33)& \cellcolor{blue!20}\href{../works/HeinzS11.pdf}{HeinzS11} (0.33)\\
Dot& \cellcolor{red!40}\href{../works/Malapert11.pdf}{Malapert11} (135.00)& \cellcolor{red!40}\href{../works/Schutt11.pdf}{Schutt11} (135.00)& \cellcolor{red!40}\href{../works/Dejemeppe16.pdf}{Dejemeppe16} (130.00)& \cellcolor{red!40}\href{../works/Fahimi16.pdf}{Fahimi16} (129.00)& \cellcolor{red!40}\href{../works/Lombardi10.pdf}{Lombardi10} (127.00)\\
Cosine& \cellcolor{red!40}\href{../works/CauwelaertLS15.pdf}{CauwelaertLS15} (0.86)& \cellcolor{red!40}\href{../works/GayHS15a.pdf}{GayHS15a} (0.72)& \cellcolor{red!40}\href{../works/ClautiauxJCM08.pdf}{ClautiauxJCM08} (0.71)& \cellcolor{red!40}\href{../works/Vilim11.pdf}{Vilim11} (0.71)& \cellcolor{red!40}\href{../works/OuelletQ18.pdf}{OuelletQ18} (0.70)\\
\index{CestaOF99}\href{../works/CestaOF99.pdf}{CestaOF99} R\&C\\
Euclid& \cellcolor{red!40}\href{../works/OddiRC10.pdf}{OddiRC10} (0.23)& \cellcolor{red!40}\href{../works/CestaOS00.pdf}{CestaOS00} (0.23)& \cellcolor{red!40}\href{../works/LombardiM12a.pdf}{LombardiM12a} (0.24)& \cellcolor{red!20}\href{../works/LombardiM10.pdf}{LombardiM10} (0.24)& \cellcolor{red!20}\href{../works/SadehF96.pdf}{SadehF96} (0.26)\\
Dot& \cellcolor{red!40}\href{../works/Dejemeppe16.pdf}{Dejemeppe16} (123.00)& \cellcolor{red!40}\href{../works/Lombardi10.pdf}{Lombardi10} (123.00)& \cellcolor{red!40}\href{../works/Groleaz21.pdf}{Groleaz21} (123.00)& \cellcolor{red!40}\href{../works/Baptiste02.pdf}{Baptiste02} (123.00)& \cellcolor{red!40}\href{../works/Godet21a.pdf}{Godet21a} (120.00)\\
Cosine& \cellcolor{red!40}\href{../works/CestaOS00.pdf}{CestaOS00} (0.83)& \cellcolor{red!40}\href{../works/OddiRC10.pdf}{OddiRC10} (0.83)& \cellcolor{red!40}\href{../works/SadehF96.pdf}{SadehF96} (0.82)& \cellcolor{red!40}\href{../works/LombardiM12a.pdf}{LombardiM12a} (0.82)& \cellcolor{red!40}\href{../works/LombardiM10.pdf}{LombardiM10} (0.81)\\
\index{CestaOPS14}CestaOPS14 R\&C& \cellcolor{red!20}\href{../works/CarchraeB09.pdf}{CarchraeB09} (0.86)& \cellcolor{red!20}OddiPCC05 (0.86)& \cellcolor{red!20}\href{../works/LombardiM13.pdf}{LombardiM13} (0.87)& \cellcolor{red!20}\href{../works/LombardiM12a.pdf}{LombardiM12a} (0.87)& \cellcolor{red!20}\href{../works/LombardiMB13.pdf}{LombardiMB13} (0.88)\\
Euclid\\
Dot\\
Cosine\\
\index{CestaOS00}\href{../works/CestaOS00.pdf}{CestaOS00} R\&C\\
Euclid& \cellcolor{red!40}\href{../works/CestaOF99.pdf}{CestaOF99} (0.23)& \cellcolor{red!20}\href{../works/BeckDSF97a.pdf}{BeckDSF97a} (0.25)& \cellcolor{yellow!20}\href{../works/OddiS97.pdf}{OddiS97} (0.26)& \cellcolor{yellow!20}\href{../works/Muscettola94.pdf}{Muscettola94} (0.27)& \cellcolor{yellow!20}\href{../works/GodardLN05.pdf}{GodardLN05} (0.27)\\
Dot& \cellcolor{red!40}\href{../works/ZarandiASC20.pdf}{ZarandiASC20} (108.00)& \cellcolor{red!40}\href{../works/Baptiste02.pdf}{Baptiste02} (108.00)& \cellcolor{red!40}\href{../works/Godet21a.pdf}{Godet21a} (107.00)& \cellcolor{red!40}\href{../works/Dejemeppe16.pdf}{Dejemeppe16} (103.00)& \cellcolor{red!40}\href{../works/Beck99.pdf}{Beck99} (103.00)\\
Cosine& \cellcolor{red!40}\href{../works/CestaOF99.pdf}{CestaOF99} (0.83)& \cellcolor{red!40}\href{../works/SadehF96.pdf}{SadehF96} (0.80)& \cellcolor{red!40}\href{../works/BeckDSF97a.pdf}{BeckDSF97a} (0.79)& \cellcolor{red!40}\href{../works/OddiRCS11.pdf}{OddiRCS11} (0.77)& \cellcolor{red!40}\href{../works/GodardLN05.pdf}{GodardLN05} (0.77)\\
\index{CestaOS98}\href{../works/CestaOS98.pdf}{CestaOS98} R\&C& \cellcolor{red!40}\href{../works/LombardiM09.pdf}{LombardiM09} (0.83)& \cellcolor{red!40}\href{../works/BartakCS10.pdf}{BartakCS10} (0.86)& \cellcolor{yellow!20}\href{../works/LombardiM10a.pdf}{LombardiM10a} (0.92)& \cellcolor{yellow!20}\href{../works/BeldiceanuCDP11.pdf}{BeldiceanuCDP11} (0.92)& \cellcolor{yellow!20}\href{../works/CobanH10.pdf}{CobanH10} (0.93)\\
Euclid& \cellcolor{red!40}\href{../works/KovacsEKV05.pdf}{KovacsEKV05} (0.12)& \cellcolor{red!40}\href{../works/WuBB05.pdf}{WuBB05} (0.13)& \cellcolor{red!40}\href{../works/Caballero23.pdf}{Caballero23} (0.13)& \cellcolor{red!40}\href{../works/Baptiste09.pdf}{Baptiste09} (0.14)& \cellcolor{red!40}\href{../works/AngelsmarkJ00.pdf}{AngelsmarkJ00} (0.14)\\
Dot& \cellcolor{red!40}\href{../works/ZarandiASC20.pdf}{ZarandiASC20} (27.00)& \cellcolor{red!40}\href{../works/SadehF96.pdf}{SadehF96} (27.00)& \cellcolor{red!40}\href{../works/BartakSR10.pdf}{BartakSR10} (27.00)& \cellcolor{red!40}\href{../works/BeckF98.pdf}{BeckF98} (27.00)& \cellcolor{red!40}\href{../works/Dejemeppe16.pdf}{Dejemeppe16} (26.00)\\
Cosine& \cellcolor{red!40}\href{../works/KovacsEKV05.pdf}{KovacsEKV05} (0.74)& \cellcolor{red!40}\href{../works/Caballero23.pdf}{Caballero23} (0.72)& \cellcolor{red!40}\href{../works/Maillard15.pdf}{Maillard15} (0.70)& \cellcolor{red!40}\href{../works/WuBB05.pdf}{WuBB05} (0.69)& \cellcolor{red!40}\href{../works/BonfiettiM12.pdf}{BonfiettiM12} (0.68)\\
\index{ChapadosJR11}\href{../works/ChapadosJR11.pdf}{ChapadosJR11} R\&C& \cellcolor{blue!20}\href{../works/OuelletQ22.pdf}{OuelletQ22} (0.97)& \cellcolor{blue!20}\href{../works/MusliuSS18.pdf}{MusliuSS18} (0.97)& \cellcolor{blue!20}\href{../works/TopalogluO11.pdf}{TopalogluO11} (0.97)& \cellcolor{blue!20}\href{../works/GuyonLPR12.pdf}{GuyonLPR12} (0.97)& \cellcolor{black!20}\href{../works/MilanoW09.pdf}{MilanoW09} (0.98)\\
Euclid& \cellcolor{red!40}\href{../works/Baptiste09.pdf}{Baptiste09} (0.16)& \cellcolor{red!40}\href{../works/CarchraeBF05.pdf}{CarchraeBF05} (0.16)& \cellcolor{red!40}\href{../works/ZibranR11.pdf}{ZibranR11} (0.16)& \cellcolor{red!40}\href{../works/AngelsmarkJ00.pdf}{AngelsmarkJ00} (0.17)& \cellcolor{red!40}\href{../works/Tsang03.pdf}{Tsang03} (0.17)\\
Dot& \cellcolor{red!40}\href{../works/Dejemeppe16.pdf}{Dejemeppe16} (36.00)& \cellcolor{red!40}\href{../works/Malapert11.pdf}{Malapert11} (36.00)& \cellcolor{red!40}\href{../works/Godet21a.pdf}{Godet21a} (35.00)& \cellcolor{red!40}\href{../works/Lombardi10.pdf}{Lombardi10} (35.00)& \cellcolor{red!40}\href{../works/AwadMDMT22.pdf}{AwadMDMT22} (35.00)\\
Cosine& \cellcolor{red!40}\href{../works/ZibranR11a.pdf}{ZibranR11a} (0.78)& \cellcolor{red!40}\href{../works/AstrandJZ18.pdf}{AstrandJZ18} (0.74)& \cellcolor{red!40}\href{../works/ZibranR11.pdf}{ZibranR11} (0.72)& \cellcolor{red!40}\href{../works/TranVNB17a.pdf}{TranVNB17a} (0.69)& \cellcolor{red!40}\href{../works/HachemiGR11.pdf}{HachemiGR11} (0.67)\\
\index{ChenGPSH10}\href{../works/ChenGPSH10.pdf}{ChenGPSH10} R\&C& \cellcolor{red!20}\href{../works/HeckmanB11.pdf}{HeckmanB11} (0.89)& \cellcolor{red!20}\href{../works/BeckFW11.pdf}{BeckFW11} (0.89)& \cellcolor{red!20}\href{../works/WatsonB08.pdf}{WatsonB08} (0.90)& \cellcolor{yellow!20}\href{../works/GrimesHM09.pdf}{GrimesHM09} (0.91)& \cellcolor{yellow!20}\href{../works/BeckF00a.pdf}{BeckF00a} (0.93)\\
Euclid& \cellcolor{green!20}\href{../works/BartakSR08.pdf}{BartakSR08} (0.31)& \cellcolor{blue!20}\href{../works/PengLC14.pdf}{PengLC14} (0.32)& \cellcolor{blue!20}\href{../works/NuijtenP98.pdf}{NuijtenP98} (0.33)& \cellcolor{blue!20}\href{../works/Bartak02a.pdf}{Bartak02a} (0.34)& \cellcolor{black!20}\href{../works/BaptisteP95.pdf}{BaptisteP95} (0.34)\\
Dot& \cellcolor{red!40}\href{../works/Baptiste02.pdf}{Baptiste02} (178.00)& \cellcolor{red!40}\href{../works/Dejemeppe16.pdf}{Dejemeppe16} (171.00)& \cellcolor{red!40}\href{../works/Godet21a.pdf}{Godet21a} (170.00)& \cellcolor{red!40}\href{../works/Malapert11.pdf}{Malapert11} (168.00)& \cellcolor{red!40}\href{../works/Fahimi16.pdf}{Fahimi16} (167.00)\\
Cosine& \cellcolor{red!40}\href{../works/BartakSR08.pdf}{BartakSR08} (0.80)& \cellcolor{red!40}\href{../works/PengLC14.pdf}{PengLC14} (0.77)& \cellcolor{red!40}\href{../works/NuijtenP98.pdf}{NuijtenP98} (0.77)& \cellcolor{red!40}\href{../works/Dorndorf2000.pdf}{Dorndorf2000} (0.75)& \cellcolor{red!40}\href{../works/BaptisteP95.pdf}{BaptisteP95} (0.75)\\
\index{ChuGNSW13}\href{../works/ChuGNSW13.pdf}{ChuGNSW13} R\&C\\
Euclid& \cellcolor{red!40}\href{../works/PoderBS04.pdf}{PoderBS04} (0.23)& \cellcolor{red!40}\href{../works/KovacsV04.pdf}{KovacsV04} (0.23)& \cellcolor{red!40}\href{../works/WolfS05.pdf}{WolfS05} (0.23)& \cellcolor{red!40}\href{../works/Caseau97.pdf}{Caseau97} (0.24)& \cellcolor{red!20}\href{../works/Bartak02a.pdf}{Bartak02a} (0.24)\\
Dot& \cellcolor{red!40}\href{../works/Dejemeppe16.pdf}{Dejemeppe16} (98.00)& \cellcolor{red!40}\href{../works/Fahimi16.pdf}{Fahimi16} (97.00)& \cellcolor{red!40}\href{../works/Beck99.pdf}{Beck99} (94.00)& \cellcolor{red!40}\href{../works/Baptiste02.pdf}{Baptiste02} (94.00)& \cellcolor{red!40}\href{../works/Malapert11.pdf}{Malapert11} (92.00)\\
Cosine& \cellcolor{red!40}\href{../works/KovacsV04.pdf}{KovacsV04} (0.80)& \cellcolor{red!40}\href{../works/TrojetHL11.pdf}{TrojetHL11} (0.79)& \cellcolor{red!40}\href{../works/PoderBS04.pdf}{PoderBS04} (0.78)& \cellcolor{red!40}\href{../works/Madi-WambaLOBM17.pdf}{Madi-WambaLOBM17} (0.77)& \cellcolor{red!40}\href{../works/Bartak02a.pdf}{Bartak02a} (0.77)\\
\index{ChuX05}\href{../works/ChuX05.pdf}{ChuX05} R\&C& \cellcolor{red!40}\href{../works/Hooker05.pdf}{Hooker05} (0.58)& \cellcolor{red!40}\href{../works/Hooker06.pdf}{Hooker06} (0.64)& \cellcolor{red!40}\href{../works/Hooker05a.pdf}{Hooker05a} (0.65)& \cellcolor{red!40}\href{../works/CireCH13.pdf}{CireCH13} (0.68)& \cellcolor{red!40}\href{../works/Hooker04.pdf}{Hooker04} (0.69)\\
Euclid& \cellcolor{red!40}\href{../works/HookerY02.pdf}{HookerY02} (0.21)& \cellcolor{red!40}\href{../works/Limtanyakul07.pdf}{Limtanyakul07} (0.24)& \cellcolor{red!40}\href{../works/Beck10.pdf}{Beck10} (0.24)& \cellcolor{red!40}\href{../works/HeinzKB13.pdf}{HeinzKB13} (0.24)& \cellcolor{red!20}\href{../works/HookerO03.pdf}{HookerO03} (0.25)\\
Dot& \cellcolor{red!40}\href{../works/Lombardi10.pdf}{Lombardi10} (94.00)& \cellcolor{red!40}\href{../works/Baptiste02.pdf}{Baptiste02} (89.00)& \cellcolor{red!40}\href{../works/MilanoW09.pdf}{MilanoW09} (87.00)& \cellcolor{red!40}\href{../works/MilanoW06.pdf}{MilanoW06} (86.00)& \cellcolor{red!40}\href{../works/Groleaz21.pdf}{Groleaz21} (85.00)\\
Cosine& \cellcolor{red!40}\href{../works/HookerY02.pdf}{HookerY02} (0.81)& \cellcolor{red!40}\href{../works/Beck10.pdf}{Beck10} (0.78)& \cellcolor{red!40}\href{../works/HamdiL13.pdf}{HamdiL13} (0.78)& \cellcolor{red!40}\href{../works/HeinzKB13.pdf}{HeinzKB13} (0.77)& \cellcolor{red!40}\href{../works/Limtanyakul07.pdf}{Limtanyakul07} (0.74)\\
\index{ChunCTY99}\href{../works/ChunCTY99.pdf}{ChunCTY99} R\&C\\
Euclid& \cellcolor{green!20}\href{../works/Puget95.pdf}{Puget95} (0.31)& \cellcolor{blue!20}\href{../works/BocewiczBB09.pdf}{BocewiczBB09} (0.32)& \cellcolor{blue!20}\href{../works/GelainPRVW17.pdf}{GelainPRVW17} (0.32)& \cellcolor{blue!20}\href{../works/Gronkvist06.pdf}{Gronkvist06} (0.32)& \cellcolor{blue!20}\href{../works/TranDRFWOVB16.pdf}{TranDRFWOVB16} (0.32)\\
Dot& \cellcolor{red!40}\href{../works/ZarandiASC20.pdf}{ZarandiASC20} (88.00)& \cellcolor{red!40}\href{../works/Fahimi16.pdf}{Fahimi16} (80.00)& \cellcolor{red!40}\href{../works/Lombardi10.pdf}{Lombardi10} (78.00)& \cellcolor{red!40}\href{../works/Beck99.pdf}{Beck99} (78.00)& \cellcolor{red!40}\href{../works/Wallace96.pdf}{Wallace96} (76.00)\\
Cosine& \cellcolor{red!40}\href{../works/Gronkvist06.pdf}{Gronkvist06} (0.64)& \cellcolor{red!40}\href{../works/BocewiczBB09.pdf}{BocewiczBB09} (0.63)& \cellcolor{red!40}\href{../works/JoLLH99.pdf}{JoLLH99} (0.60)& \cellcolor{red!40}\href{../works/Wallace96.pdf}{Wallace96} (0.60)& \cellcolor{red!40}\href{../works/TranDRFWOVB16.pdf}{TranDRFWOVB16} (0.60)\\
\index{ChunS14}\href{../works/ChunS14.pdf}{ChunS14} R\&C\\
Euclid& \cellcolor{green!20}\href{../works/Prosser89.pdf}{Prosser89} (0.29)& \cellcolor{green!20}\href{../works/Puget95.pdf}{Puget95} (0.29)& \cellcolor{green!20}\href{../works/AngelsmarkJ00.pdf}{AngelsmarkJ00} (0.29)& \cellcolor{green!20}\href{../works/CestaOS98.pdf}{CestaOS98} (0.29)& \cellcolor{green!20}\href{../works/DincbasS91.pdf}{DincbasS91} (0.29)\\
Dot& \cellcolor{red!40}\href{../works/ZarandiASC20.pdf}{ZarandiASC20} (80.00)& \cellcolor{red!40}\href{../works/Lemos21.pdf}{Lemos21} (77.00)& \cellcolor{red!40}\href{../works/Froger16.pdf}{Froger16} (65.00)& \cellcolor{red!40}\href{../works/Baptiste02.pdf}{Baptiste02} (65.00)& \cellcolor{red!40}\href{../works/Lombardi10.pdf}{Lombardi10} (64.00)\\
Cosine& \cellcolor{red!40}\href{../works/Prosser89.pdf}{Prosser89} (0.64)& \cellcolor{red!40}\href{../works/MorgadoM97.pdf}{MorgadoM97} (0.60)& \cellcolor{red!40}\href{../works/TranDRFWOVB16.pdf}{TranDRFWOVB16} (0.59)& \cellcolor{red!40}\href{../works/Madi-WambaLOBM17.pdf}{Madi-WambaLOBM17} (0.59)& \cellcolor{red!40}\href{../works/Puget95.pdf}{Puget95} (0.58)\\
\index{CilKLO22}\href{../works/CilKLO22.pdf}{CilKLO22} R\&C& \cellcolor{red!40}\href{../works/Edis21.pdf}{Edis21} (0.78)& \cellcolor{red!40}\href{../works/NaderiBZ22a.pdf}{NaderiBZ22a} (0.85)& \cellcolor{red!20}\href{../works/PohlAK22.pdf}{PohlAK22} (0.89)& \cellcolor{red!20}\href{../works/ZhangYW21.pdf}{ZhangYW21} (0.90)& \cellcolor{yellow!20}GunerGSKD23 (0.91)\\
Euclid& \cellcolor{green!20}\href{../works/PinarbasiAY19.pdf}{PinarbasiAY19} (0.30)& \cellcolor{green!20}\href{../works/AbidinK20.pdf}{AbidinK20} (0.31)& \cellcolor{green!20}\href{../works/TopalogluSS12.pdf}{TopalogluSS12} (0.31)& \cellcolor{green!20}\href{../works/AlakaP23.pdf}{AlakaP23} (0.31)& \cellcolor{blue!20}\href{../works/Alaka21.pdf}{Alaka21} (0.32)\\
Dot& \cellcolor{red!40}\href{../works/Lunardi20.pdf}{Lunardi20} (168.00)& \cellcolor{red!40}\href{../works/ZarandiASC20.pdf}{ZarandiASC20} (167.00)& \cellcolor{red!40}\href{../works/Dejemeppe16.pdf}{Dejemeppe16} (154.00)& \cellcolor{red!40}\href{../works/Groleaz21.pdf}{Groleaz21} (153.00)& \cellcolor{red!40}\href{../works/Astrand21.pdf}{Astrand21} (153.00)\\
Cosine& \cellcolor{red!40}\href{../works/PinarbasiAY19.pdf}{PinarbasiAY19} (0.79)& \cellcolor{red!40}\href{../works/Edis21.pdf}{Edis21} (0.78)& \cellcolor{red!40}\href{../works/AbidinK20.pdf}{AbidinK20} (0.78)& \cellcolor{red!40}\href{../works/TopalogluSS12.pdf}{TopalogluSS12} (0.77)& \cellcolor{red!40}\href{../works/AlakaP23.pdf}{AlakaP23} (0.77)\\
\index{CireCH13}\href{../works/CireCH13.pdf}{CireCH13} R\&C& \cellcolor{red!40}\href{../works/HamdiL13.pdf}{HamdiL13} (0.42)& \cellcolor{red!40}\href{../works/CireCH16.pdf}{CireCH16} (0.58)& \cellcolor{red!40}\href{../works/CobanH11.pdf}{CobanH11} (0.58)& \cellcolor{red!40}\href{../works/Hooker06.pdf}{Hooker06} (0.63)& \cellcolor{red!40}\href{../works/Hooker07.pdf}{Hooker07} (0.66)\\
Euclid& \cellcolor{red!40}\href{../works/HeinzKB13.pdf}{HeinzKB13} (0.18)& \cellcolor{red!40}\href{../works/HookerO03.pdf}{HookerO03} (0.21)& \cellcolor{red!40}\href{../works/CireCH16.pdf}{CireCH16} (0.22)& \cellcolor{red!40}\href{../works/CobanH10.pdf}{CobanH10} (0.23)& \cellcolor{red!40}\href{../works/HookerY02.pdf}{HookerY02} (0.23)\\
Dot& \cellcolor{red!40}\href{../works/Lombardi10.pdf}{Lombardi10} (92.00)& \cellcolor{red!40}\href{../works/HookerH17.pdf}{HookerH17} (86.00)& \cellcolor{red!40}\href{../works/MilanoW09.pdf}{MilanoW09} (86.00)& \cellcolor{red!40}\href{../works/Hooker05.pdf}{Hooker05} (83.00)& \cellcolor{red!40}\href{../works/Hooker07.pdf}{Hooker07} (83.00)\\
Cosine& \cellcolor{red!40}\href{../works/HeinzKB13.pdf}{HeinzKB13} (0.86)& \cellcolor{red!40}\href{../works/HookerO03.pdf}{HookerO03} (0.81)& \cellcolor{red!40}\href{../works/CireCH16.pdf}{CireCH16} (0.80)& \cellcolor{red!40}\href{../works/HeinzB12.pdf}{HeinzB12} (0.79)& \cellcolor{red!40}\href{../works/Hooker07.pdf}{Hooker07} (0.78)\\
\index{CireCH16}\href{../works/CireCH16.pdf}{CireCH16} R\&C& \cellcolor{red!40}\href{../works/CobanH11.pdf}{CobanH11} (0.55)& \cellcolor{red!40}\href{../works/CireCH13.pdf}{CireCH13} (0.58)& \cellcolor{red!40}\href{../works/Hooker07.pdf}{Hooker07} (0.68)& \cellcolor{red!40}\href{../works/CobanH10.pdf}{CobanH10} (0.69)& \cellcolor{red!40}\href{../works/Beck10.pdf}{Beck10} (0.72)\\
Euclid& \cellcolor{red!40}\href{../works/CireCH13.pdf}{CireCH13} (0.22)& \cellcolor{red!40}\href{../works/Hooker04.pdf}{Hooker04} (0.22)& \cellcolor{red!20}\href{../works/Hooker05a.pdf}{Hooker05a} (0.24)& \cellcolor{red!20}\href{../works/HeinzKB13.pdf}{HeinzKB13} (0.25)& \cellcolor{red!20}\href{../works/BeniniLMMR08.pdf}{BeniniLMMR08} (0.25)\\
Dot& \cellcolor{red!40}\href{../works/Hooker07.pdf}{Hooker07} (91.00)& \cellcolor{red!40}\href{../works/Lombardi10.pdf}{Lombardi10} (87.00)& \cellcolor{red!40}\href{../works/Hooker05.pdf}{Hooker05} (85.00)& \cellcolor{red!40}\href{../works/Hooker19.pdf}{Hooker19} (85.00)& \cellcolor{red!40}\href{../works/LaborieRSV18.pdf}{LaborieRSV18} (83.00)\\
Cosine& \cellcolor{red!40}\href{../works/Hooker04.pdf}{Hooker04} (0.83)& \cellcolor{red!40}\href{../works/Hooker07.pdf}{Hooker07} (0.82)& \cellcolor{red!40}\href{../works/CireCH13.pdf}{CireCH13} (0.80)& \cellcolor{red!40}\href{../works/Hooker06.pdf}{Hooker06} (0.79)& \cellcolor{red!40}\href{../works/BeniniLMMR08.pdf}{BeniniLMMR08} (0.77)\\
\index{ClautiauxJCM08}\href{../works/ClautiauxJCM08.pdf}{ClautiauxJCM08} R\&C& \cellcolor{yellow!20}\href{../works/MercierH08.pdf}{MercierH08} (0.91)& \cellcolor{yellow!20}\href{../works/LetortCB13.pdf}{LetortCB13} (0.92)& \cellcolor{yellow!20}\href{../works/BonfiettiZLM16.pdf}{BonfiettiZLM16} (0.92)& \cellcolor{yellow!20}\href{../works/GrimesHM09.pdf}{GrimesHM09} (0.92)& \cellcolor{yellow!20}\href{../works/BeldiceanuCDP11.pdf}{BeldiceanuCDP11} (0.93)\\
Euclid& \cellcolor{yellow!20}\href{../works/WolfS05.pdf}{WolfS05} (0.27)& \cellcolor{green!20}\href{../works/CauwelaertLS15.pdf}{CauwelaertLS15} (0.29)& \cellcolor{green!20}\href{../works/BeldiceanuP07.pdf}{BeldiceanuP07} (0.29)& \cellcolor{green!20}\href{../works/PoderB08.pdf}{PoderB08} (0.29)& \cellcolor{green!20}\href{../works/KovacsB08.pdf}{KovacsB08} (0.30)\\
Dot& \cellcolor{red!40}\href{../works/Malapert11.pdf}{Malapert11} (116.00)& \cellcolor{red!40}\href{../works/Schutt11.pdf}{Schutt11} (115.00)& \cellcolor{red!40}\href{../works/Godet21a.pdf}{Godet21a} (111.00)& \cellcolor{red!40}\href{../works/Lombardi10.pdf}{Lombardi10} (111.00)& \cellcolor{red!40}\href{../works/Baptiste02.pdf}{Baptiste02} (111.00)\\
Cosine& \cellcolor{red!40}\href{../works/WolfS05.pdf}{WolfS05} (0.77)& \cellcolor{red!40}\href{../works/CauwelaertLS15.pdf}{CauwelaertLS15} (0.73)& \cellcolor{red!40}\href{../works/KovacsB08.pdf}{KovacsB08} (0.73)& \cellcolor{red!40}\href{../works/BeldiceanuP07.pdf}{BeldiceanuP07} (0.72)& \cellcolor{red!40}\href{../works/BeldiceanuCDP11.pdf}{BeldiceanuCDP11} (0.72)\\
\index{Clercq12}\href{../works/Clercq12.pdf}{Clercq12} R\&C\\
Euclid& \cellcolor{blue!20}\href{../works/Derrien15.pdf}{Derrien15} (0.33)& \href{../works/BeckF00a.pdf}{BeckF00a} (0.37)& \href{../works/Letort13.pdf}{Letort13} (0.39)& \href{../works/BeckF99.pdf}{BeckF99} (0.39)& \href{../works/VilimBC05.pdf}{VilimBC05} (0.39)\\
Dot& \cellcolor{red!40}\href{../works/Malapert11.pdf}{Malapert11} (175.00)& \cellcolor{red!40}\href{../works/Fahimi16.pdf}{Fahimi16} (164.00)& \cellcolor{red!40}\href{../works/Dejemeppe16.pdf}{Dejemeppe16} (163.00)& \cellcolor{red!40}\href{../works/Schutt11.pdf}{Schutt11} (162.00)& \cellcolor{red!40}\href{../works/Kameugne14.pdf}{Kameugne14} (148.00)\\
Cosine& \cellcolor{red!40}\href{../works/Derrien15.pdf}{Derrien15} (0.79)& \cellcolor{red!40}\href{../works/Kameugne14.pdf}{Kameugne14} (0.73)& \cellcolor{red!40}\href{../works/Letort13.pdf}{Letort13} (0.73)& \cellcolor{red!40}\href{../works/BeckF00a.pdf}{BeckF00a} (0.70)& \cellcolor{red!40}\href{../works/GayHS15a.pdf}{GayHS15a} (0.67)\\
\index{ClercqPBJ11}\href{../works/ClercqPBJ11.pdf}{ClercqPBJ11} R\&C& \cellcolor{red!20}\href{../works/LetortBC12.pdf}{LetortBC12} (0.87)& \cellcolor{red!20}\href{../works/GayHS15a.pdf}{GayHS15a} (0.87)& \cellcolor{red!20}\href{../works/LetortCB15.pdf}{LetortCB15} (0.88)& \cellcolor{red!20}\href{../works/OuelletQ13.pdf}{OuelletQ13} (0.88)& \cellcolor{red!20}\href{../works/GayHS15.pdf}{GayHS15} (0.90)\\
Euclid& \cellcolor{yellow!20}\href{../works/Vilim09.pdf}{Vilim09} (0.28)& \cellcolor{green!20}\href{../works/Vilim04.pdf}{Vilim04} (0.29)& \cellcolor{green!20}\href{../works/DerrienP14.pdf}{DerrienP14} (0.30)& \cellcolor{green!20}\href{../works/PoderB08.pdf}{PoderB08} (0.30)& \cellcolor{green!20}\href{../works/DerrienPZ14.pdf}{DerrienPZ14} (0.30)\\
Dot& \cellcolor{red!40}\href{../works/Fahimi16.pdf}{Fahimi16} (103.00)& \cellcolor{red!40}\href{../works/Malapert11.pdf}{Malapert11} (102.00)& \cellcolor{red!40}\href{../works/Dejemeppe16.pdf}{Dejemeppe16} (94.00)& \cellcolor{red!40}\href{../works/Godet21a.pdf}{Godet21a} (92.00)& \cellcolor{red!40}\href{../works/Schutt11.pdf}{Schutt11} (91.00)\\
Cosine& \cellcolor{red!40}\href{../works/abs-0907-0939.pdf}{abs-0907-0939} (0.73)& \cellcolor{red!40}\href{../works/Vilim09.pdf}{Vilim09} (0.71)& \cellcolor{red!40}\href{../works/DerrienPZ14.pdf}{DerrienPZ14} (0.71)& \cellcolor{red!40}\href{../works/Vilim04.pdf}{Vilim04} (0.69)& \cellcolor{red!40}\href{../works/OuelletQ22.pdf}{OuelletQ22} (0.69)\\
\index{CobanH10}\href{../works/CobanH10.pdf}{CobanH10} R\&C& \cellcolor{red!40}\href{../works/Hooker06.pdf}{Hooker06} (0.67)& \cellcolor{red!40}\href{../works/CireCH13.pdf}{CireCH13} (0.67)& \cellcolor{red!40}\href{../works/CireCH16.pdf}{CireCH16} (0.69)& \cellcolor{red!40}\href{../works/HamdiL13.pdf}{HamdiL13} (0.71)& \cellcolor{red!40}\href{../works/Hooker05a.pdf}{Hooker05a} (0.71)\\
Euclid& \cellcolor{red!40}\href{../works/CireCH13.pdf}{CireCH13} (0.23)& \cellcolor{red!20}\href{../works/Hooker17.pdf}{Hooker17} (0.25)& \cellcolor{red!20}\href{../works/CobanH11.pdf}{CobanH11} (0.26)& \cellcolor{yellow!20}\href{../works/HookerO03.pdf}{HookerO03} (0.28)& \cellcolor{yellow!20}\href{../works/HebrardTW05.pdf}{HebrardTW05} (0.28)\\
Dot& \cellcolor{red!40}\href{../works/Lombardi10.pdf}{Lombardi10} (87.00)& \cellcolor{red!40}\href{../works/LombardiM12.pdf}{LombardiM12} (78.00)& \cellcolor{red!40}\href{../works/CobanH11.pdf}{CobanH11} (78.00)& \cellcolor{red!40}\href{../works/NaderiRR23.pdf}{NaderiRR23} (76.00)& \cellcolor{red!40}\href{../works/Groleaz21.pdf}{Groleaz21} (75.00)\\
Cosine& \cellcolor{red!40}\href{../works/CobanH11.pdf}{CobanH11} (0.80)& \cellcolor{red!40}\href{../works/CireCH13.pdf}{CireCH13} (0.77)& \cellcolor{red!40}\href{../works/EmeretlisTAV17.pdf}{EmeretlisTAV17} (0.72)& \cellcolor{red!40}\href{../works/Hooker07.pdf}{Hooker07} (0.69)& \cellcolor{red!40}\href{../works/ElciOH22.pdf}{ElciOH22} (0.68)\\
\index{CobanH11}\href{../works/CobanH11.pdf}{CobanH11} R\&C& \cellcolor{red!40}\href{../works/CireCH16.pdf}{CireCH16} (0.55)& \cellcolor{red!40}\href{../works/CireCH13.pdf}{CireCH13} (0.58)& \cellcolor{red!40}\href{../works/Hooker06.pdf}{Hooker06} (0.63)& \cellcolor{red!40}\href{../works/Hooker07.pdf}{Hooker07} (0.67)& \cellcolor{red!40}\href{../works/BeniniLMMR08.pdf}{BeniniLMMR08} (0.71)\\
Euclid& \cellcolor{red!20}\href{../works/HamdiL13.pdf}{HamdiL13} (0.26)& \cellcolor{red!20}\href{../works/CobanH10.pdf}{CobanH10} (0.26)& \cellcolor{yellow!20}\href{../works/Hooker06.pdf}{Hooker06} (0.27)& \cellcolor{yellow!20}\href{../works/ElciOH22.pdf}{ElciOH22} (0.27)& \cellcolor{yellow!20}\href{../works/Hooker05a.pdf}{Hooker05a} (0.28)\\
Dot& \cellcolor{red!40}\href{../works/Lombardi10.pdf}{Lombardi10} (135.00)& \cellcolor{red!40}\href{../works/ZarandiASC20.pdf}{ZarandiASC20} (129.00)& \cellcolor{red!40}\href{../works/Groleaz21.pdf}{Groleaz21} (129.00)& \cellcolor{red!40}\href{../works/Hooker19.pdf}{Hooker19} (119.00)& \cellcolor{red!40}\href{../works/Baptiste02.pdf}{Baptiste02} (118.00)\\
Cosine& \cellcolor{red!40}\href{../works/HamdiL13.pdf}{HamdiL13} (0.82)& \cellcolor{red!40}\href{../works/CobanH10.pdf}{CobanH10} (0.80)& \cellcolor{red!40}\href{../works/ElciOH22.pdf}{ElciOH22} (0.80)& \cellcolor{red!40}\href{../works/Hooker06.pdf}{Hooker06} (0.79)& \cellcolor{red!40}\href{../works/Hooker07.pdf}{Hooker07} (0.78)\\
\index{CohenHB17}\href{../works/CohenHB17.pdf}{CohenHB17} R\&C& \cellcolor{green!20}\href{../works/BartoliniBBLM14.pdf}{BartoliniBBLM14} (0.93)& \cellcolor{green!20}\href{../works/GilesH16.pdf}{GilesH16} (0.94)& \cellcolor{green!20}\href{../works/NishikawaSTT18.pdf}{NishikawaSTT18} (0.96)& \cellcolor{blue!20}\href{../works/ArtiguesLH13.pdf}{ArtiguesLH13} (0.96)& \cellcolor{blue!20}\href{../works/SquillaciPR23.pdf}{SquillaciPR23} (0.97)\\
Euclid& \cellcolor{red!40}\href{../works/ZibranR11.pdf}{ZibranR11} (0.22)& \cellcolor{red!40}\href{../works/ZibranR11a.pdf}{ZibranR11a} (0.24)& \cellcolor{red!40}\href{../works/ChapadosJR11.pdf}{ChapadosJR11} (0.24)& \cellcolor{red!20}\href{../works/AngelsmarkJ00.pdf}{AngelsmarkJ00} (0.24)& \cellcolor{red!20}\href{../works/CarchraeBF05.pdf}{CarchraeBF05} (0.25)\\
Dot& \cellcolor{red!40}\href{../works/Groleaz21.pdf}{Groleaz21} (64.00)& \cellcolor{red!40}\href{../works/ZarandiASC20.pdf}{ZarandiASC20} (61.00)& \cellcolor{red!40}\href{../works/Lemos21.pdf}{Lemos21} (60.00)& \cellcolor{red!40}\href{../works/Lunardi20.pdf}{Lunardi20} (59.00)& \cellcolor{red!40}\href{../works/ColT22.pdf}{ColT22} (59.00)\\
Cosine& \cellcolor{red!40}\href{../works/ZibranR11.pdf}{ZibranR11} (0.68)& \cellcolor{red!40}\href{../works/ZibranR11a.pdf}{ZibranR11a} (0.67)& \cellcolor{red!40}\href{../works/BoothNB16.pdf}{BoothNB16} (0.65)& \cellcolor{red!40}\href{../works/BockmayrP06.pdf}{BockmayrP06} (0.64)& \cellcolor{red!40}\href{../works/AstrandJZ18.pdf}{AstrandJZ18} (0.63)\\
\index{ColT19}\href{../works/ColT19.pdf}{ColT19} R\&C& \cellcolor{red!40}\href{../works/ColT2019a.pdf}{ColT2019a} (0.54)& \cellcolor{red!40}\href{../works/VilimLS15.pdf}{VilimLS15} (0.82)& \cellcolor{red!40}\href{../works/Laborie18a.pdf}{Laborie18a} (0.83)& \cellcolor{red!40}\href{../works/Wolf03.pdf}{Wolf03} (0.84)& \cellcolor{red!20}\href{../works/BeckF00.pdf}{BeckF00} (0.87)\\
Euclid& \cellcolor{red!40}\href{../works/ColT2019a.pdf}{ColT2019a} (0.17)& \cellcolor{red!40}\href{../works/abs-2102-08778.pdf}{abs-2102-08778} (0.22)& \cellcolor{green!20}\href{../works/NuijtenA96.pdf}{NuijtenA96} (0.31)& \cellcolor{green!20}\href{../works/PacinoH11.pdf}{PacinoH11} (0.31)& \cellcolor{blue!20}\href{../works/NuijtenA94.pdf}{NuijtenA94} (0.32)\\
Dot& \cellcolor{red!40}\href{../works/ColT22.pdf}{ColT22} (135.00)& \cellcolor{red!40}\href{../works/Groleaz21.pdf}{Groleaz21} (124.00)& \cellcolor{red!40}\href{../works/JuvinHHL23.pdf}{JuvinHHL23} (113.00)& \cellcolor{red!40}\href{../works/Dejemeppe16.pdf}{Dejemeppe16} (113.00)& \cellcolor{red!40}\href{../works/Godet21a.pdf}{Godet21a} (111.00)\\
Cosine& \cellcolor{red!40}\href{../works/ColT2019a.pdf}{ColT2019a} (0.92)& \cellcolor{red!40}\href{../works/abs-2102-08778.pdf}{abs-2102-08778} (0.86)& \cellcolor{red!40}\href{../works/TasselGS23.pdf}{TasselGS23} (0.74)& \cellcolor{red!40}\href{../works/abs-2306-05747.pdf}{abs-2306-05747} (0.74)& \cellcolor{red!40}\href{../works/NuijtenA96.pdf}{NuijtenA96} (0.73)\\
\index{ColT2019a}\href{../works/ColT2019a.pdf}{ColT2019a} R\&C& \cellcolor{red!40}\href{../works/ColT19.pdf}{ColT19} (0.54)& \cellcolor{red!40}\href{../works/Wolf03.pdf}{Wolf03} (0.82)& \cellcolor{red!40}\href{../works/MenciaSV12.pdf}{MenciaSV12} (0.84)& \cellcolor{red!40}\href{../works/Laborie18a.pdf}{Laborie18a} (0.85)& \cellcolor{red!20}\href{../works/VilimLS15.pdf}{VilimLS15} (0.86)\\
Euclid& \cellcolor{red!40}\href{../works/ColT19.pdf}{ColT19} (0.17)& \cellcolor{red!40}\href{../works/abs-2102-08778.pdf}{abs-2102-08778} (0.23)& \cellcolor{green!20}\href{../works/CarchraeB09.pdf}{CarchraeB09} (0.30)& \cellcolor{green!20}\href{../works/WatsonB08.pdf}{WatsonB08} (0.31)& \cellcolor{green!20}\href{../works/PacinoH11.pdf}{PacinoH11} (0.31)\\
Dot& \cellcolor{red!40}\href{../works/ColT22.pdf}{ColT22} (125.00)& \cellcolor{red!40}\href{../works/Groleaz21.pdf}{Groleaz21} (119.00)& \cellcolor{red!40}\href{../works/Dejemeppe16.pdf}{Dejemeppe16} (111.00)& \cellcolor{red!40}\href{../works/ColT19.pdf}{ColT19} (109.00)& \cellcolor{red!40}\href{../works/LacknerMMWW23.pdf}{LacknerMMWW23} (107.00)\\
Cosine& \cellcolor{red!40}\href{../works/ColT19.pdf}{ColT19} (0.92)& \cellcolor{red!40}\href{../works/abs-2102-08778.pdf}{abs-2102-08778} (0.83)& \cellcolor{red!40}\href{../works/CarchraeB09.pdf}{CarchraeB09} (0.73)& \cellcolor{red!40}\href{../works/Bit-Monnot23.pdf}{Bit-Monnot23} (0.71)& \cellcolor{red!40}\href{../works/ColT22.pdf}{ColT22} (0.70)\\
\index{ColT22}\href{../works/ColT22.pdf}{ColT22} R\&C& \cellcolor{red!20}DomdorfPH03 (0.88)& \cellcolor{red!20}\href{../works/ColT19.pdf}{ColT19} (0.88)& \cellcolor{red!20}DorndorfPH99 (0.89)& \cellcolor{red!20}\href{../works/BlazewiczDP96.pdf}{BlazewiczDP96} (0.89)& \cellcolor{red!20}\href{../works/ColT2019a.pdf}{ColT2019a} (0.89)\\
Euclid& \href{../works/OujanaAYB22.pdf}{OujanaAYB22} (0.44)& \href{../works/ColT19.pdf}{ColT19} (0.45)& \href{../works/Novas19.pdf}{Novas19} (0.45)& \href{../works/ColT2019a.pdf}{ColT2019a} (0.46)& \href{../works/abs-2102-08778.pdf}{abs-2102-08778} (0.46)\\
Dot& \cellcolor{red!40}\href{../works/Groleaz21.pdf}{Groleaz21} (250.00)& \cellcolor{red!40}\href{../works/ZarandiASC20.pdf}{ZarandiASC20} (249.00)& \cellcolor{red!40}\href{../works/Dejemeppe16.pdf}{Dejemeppe16} (231.00)& \cellcolor{red!40}\href{../works/Lunardi20.pdf}{Lunardi20} (229.00)& \cellcolor{red!40}\href{../works/Baptiste02.pdf}{Baptiste02} (227.00)\\
Cosine& \cellcolor{red!40}\href{../works/OujanaAYB22.pdf}{OujanaAYB22} (0.73)& \cellcolor{red!40}\href{../works/Novas19.pdf}{Novas19} (0.72)& \cellcolor{red!40}\href{../works/ColT19.pdf}{ColT19} (0.72)& \cellcolor{red!40}\href{../works/Lunardi20.pdf}{Lunardi20} (0.71)& \cellcolor{red!40}\href{../works/ColT2019a.pdf}{ColT2019a} (0.70)\\
\index{Colombani96}\href{../works/Colombani96.pdf}{Colombani96} R\&C& \cellcolor{red!40}\href{../works/Zhou96.pdf}{Zhou96} (0.60)& \cellcolor{red!40}\href{../works/Goltz95.pdf}{Goltz95} (0.75)& \cellcolor{red!40}\href{../works/Wolf03.pdf}{Wolf03} (0.77)& \cellcolor{red!40}\href{../works/Rodriguez07.pdf}{Rodriguez07} (0.82)& \cellcolor{red!40}\href{../works/Vilim04.pdf}{Vilim04} (0.82)\\
Euclid& \cellcolor{red!40}\href{../works/Zhou96.pdf}{Zhou96} (0.22)& \cellcolor{red!40}\href{../works/BockmayrP06.pdf}{BockmayrP06} (0.23)& \cellcolor{red!40}\href{../works/Limtanyakul07.pdf}{Limtanyakul07} (0.24)& \cellcolor{red!20}\href{../works/Sadykov04.pdf}{Sadykov04} (0.25)& \cellcolor{red!20}\href{../works/HarjunkoskiG02.pdf}{HarjunkoskiG02} (0.26)\\
Dot& \cellcolor{red!40}\href{../works/Baptiste02.pdf}{Baptiste02} (112.00)& \cellcolor{red!40}\href{../works/ZarandiASC20.pdf}{ZarandiASC20} (110.00)& \cellcolor{red!40}\href{../works/Beck99.pdf}{Beck99} (110.00)& \cellcolor{red!40}\href{../works/Groleaz21.pdf}{Groleaz21} (108.00)& \cellcolor{red!40}\href{../works/BartakSR10.pdf}{BartakSR10} (108.00)\\
Cosine& \cellcolor{red!40}\href{../works/Zhou96.pdf}{Zhou96} (0.84)& \cellcolor{red!40}\href{../works/Zhou97.pdf}{Zhou97} (0.80)& \cellcolor{red!40}\href{../works/HarjunkoskiG02.pdf}{HarjunkoskiG02} (0.79)& \cellcolor{red!40}\href{../works/BockmayrP06.pdf}{BockmayrP06} (0.79)& \cellcolor{red!40}\href{../works/Limtanyakul07.pdf}{Limtanyakul07} (0.78)\\
\index{CorreaLR07}\href{../works/CorreaLR07.pdf}{CorreaLR07} R\&C& \cellcolor{red!40}\href{../works/CambazardJ05.pdf}{CambazardJ05} (0.82)& \cellcolor{red!40}\href{../works/Hooker07.pdf}{Hooker07} (0.83)& \cellcolor{red!40}\href{../works/Hooker05.pdf}{Hooker05} (0.83)& \cellcolor{red!40}\href{../works/BeniniBGM06.pdf}{BeniniBGM06} (0.83)& \cellcolor{red!20}\href{../works/Hooker04.pdf}{Hooker04} (0.86)\\
Euclid& \cellcolor{green!20}\href{../works/Hooker04.pdf}{Hooker04} (0.29)& \cellcolor{green!20}\href{../works/BeniniLMMR08.pdf}{BeniniLMMR08} (0.29)& \cellcolor{green!20}\href{../works/UnsalO19.pdf}{UnsalO19} (0.30)& \cellcolor{green!20}\href{../works/CireCH13.pdf}{CireCH13} (0.30)& \cellcolor{green!20}\href{../works/Thorsteinsson01.pdf}{Thorsteinsson01} (0.30)\\
Dot& \cellcolor{red!40}\href{../works/Lombardi10.pdf}{Lombardi10} (101.00)& \cellcolor{red!40}\href{../works/LaborieRSV18.pdf}{LaborieRSV18} (99.00)& \cellcolor{red!40}\href{../works/QinDS16.pdf}{QinDS16} (97.00)& \cellcolor{red!40}\href{../works/ZarandiASC20.pdf}{ZarandiASC20} (95.00)& \cellcolor{red!40}\href{../works/Godet21a.pdf}{Godet21a} (94.00)\\
Cosine& \cellcolor{red!40}\href{../works/QinDS16.pdf}{QinDS16} (0.75)& \cellcolor{red!40}\href{../works/UnsalO19.pdf}{UnsalO19} (0.74)& \cellcolor{red!40}\href{../works/Hooker04.pdf}{Hooker04} (0.73)& \cellcolor{red!40}\href{../works/BeniniLMR11.pdf}{BeniniLMR11} (0.73)& \cellcolor{red!40}\href{../works/Hooker07.pdf}{Hooker07} (0.71)\\
\index{CrawfordB94}\href{../works/CrawfordB94.pdf}{CrawfordB94} R\&C\\
Euclid& \cellcolor{red!40}\href{../works/LauLN08.pdf}{LauLN08} (0.14)& \cellcolor{red!40}\href{../works/FoxAS82.pdf}{FoxAS82} (0.16)& \cellcolor{red!40}\href{../works/KengY89.pdf}{KengY89} (0.17)& \cellcolor{red!40}\href{../works/HebrardTW05.pdf}{HebrardTW05} (0.18)& \cellcolor{red!40}\href{../works/AngelsmarkJ00.pdf}{AngelsmarkJ00} (0.18)\\
Dot& \cellcolor{red!40}\href{../works/ZarandiASC20.pdf}{ZarandiASC20} (56.00)& \cellcolor{red!40}\href{../works/PrataAN23.pdf}{PrataAN23} (54.00)& \cellcolor{red!40}\href{../works/TerekhovTDB14.pdf}{TerekhovTDB14} (54.00)& \cellcolor{red!40}\href{../works/GombolayWS18.pdf}{GombolayWS18} (54.00)& \cellcolor{red!40}\href{../works/TrentesauxPT01.pdf}{TrentesauxPT01} (54.00)\\
Cosine& \cellcolor{red!40}\href{../works/LauLN08.pdf}{LauLN08} (0.87)& \cellcolor{red!40}\href{../works/KengY89.pdf}{KengY89} (0.84)& \cellcolor{red!40}\href{../works/FoxAS82.pdf}{FoxAS82} (0.83)& \cellcolor{red!40}\href{../works/DoRZ08.pdf}{DoRZ08} (0.77)& \cellcolor{red!40}\href{../works/LuoVLBM16.pdf}{LuoVLBM16} (0.75)\\
\index{CzerniachowskaWZ23}\href{../works/CzerniachowskaWZ23.pdf}{CzerniachowskaWZ23} R\&C\\
Euclid& \cellcolor{blue!20}\href{../works/JuvinHL23.pdf}{JuvinHL23} (0.33)& \cellcolor{blue!20}\href{../works/TanZWGQ19.pdf}{TanZWGQ19} (0.33)& \cellcolor{blue!20}\href{../works/LiFJZLL22.pdf}{LiFJZLL22} (0.33)& \cellcolor{blue!20}\href{../works/NovasH14.pdf}{NovasH14} (0.33)& \cellcolor{black!20}\href{../works/MurinR19.pdf}{MurinR19} (0.34)\\
Dot& \cellcolor{red!40}\href{../works/Lunardi20.pdf}{Lunardi20} (180.00)& \cellcolor{red!40}\href{../works/ZarandiASC20.pdf}{ZarandiASC20} (165.00)& \cellcolor{red!40}\href{../works/Groleaz21.pdf}{Groleaz21} (163.00)& \cellcolor{red!40}\href{../works/Astrand21.pdf}{Astrand21} (156.00)& \cellcolor{red!40}\href{../works/ColT22.pdf}{ColT22} (154.00)\\
Cosine& \cellcolor{red!40}\href{../works/TanZWGQ19.pdf}{TanZWGQ19} (0.76)& \cellcolor{red!40}\href{../works/JuvinHL23.pdf}{JuvinHL23} (0.76)& \cellcolor{red!40}\href{../works/LiFJZLL22.pdf}{LiFJZLL22} (0.76)& \cellcolor{red!40}\href{../works/NovasH14.pdf}{NovasH14} (0.75)& \cellcolor{red!40}\href{../works/MurinR19.pdf}{MurinR19} (0.75)\\
\index{DannaP03}\href{../works/DannaP03.pdf}{DannaP03} R\&C& \cellcolor{red!40}\href{../works/PerronSF04.pdf}{PerronSF04} (0.67)& \cellcolor{red!40}\href{../works/Laborie09.pdf}{Laborie09} (0.79)& \cellcolor{red!40}\href{../works/CarchraeB09.pdf}{CarchraeB09} (0.84)& \cellcolor{red!20}\href{../works/SchausHMCMD11.pdf}{SchausHMCMD11} (0.88)& \cellcolor{red!20}\href{../works/GarganiR07.pdf}{GarganiR07} (0.88)\\
Euclid& \cellcolor{red!40}\href{../works/PerronSF04.pdf}{PerronSF04} (0.21)& \cellcolor{yellow!20}\href{../works/BeckR03.pdf}{BeckR03} (0.27)& \cellcolor{green!20}\href{../works/CarchraeB09.pdf}{CarchraeB09} (0.29)& \cellcolor{green!20}\href{../works/Puget95.pdf}{Puget95} (0.29)& \cellcolor{green!20}\href{../works/GodardLN05.pdf}{GodardLN05} (0.29)\\
Dot& \cellcolor{red!40}\href{../works/LaborieRSV18.pdf}{LaborieRSV18} (112.00)& \cellcolor{red!40}\href{../works/Groleaz21.pdf}{Groleaz21} (108.00)& \cellcolor{red!40}\href{../works/Dejemeppe16.pdf}{Dejemeppe16} (107.00)& \cellcolor{red!40}\href{../works/GrimesH11.pdf}{GrimesH11} (102.00)& \cellcolor{red!40}\href{../works/GrimesH15.pdf}{GrimesH15} (98.00)\\
Cosine& \cellcolor{red!40}\href{../works/PerronSF04.pdf}{PerronSF04} (0.85)& \cellcolor{red!40}\href{../works/BeckR03.pdf}{BeckR03} (0.80)& \cellcolor{red!40}\href{../works/MonetteDH09.pdf}{MonetteDH09} (0.77)& \cellcolor{red!40}\href{../works/CarchraeB09.pdf}{CarchraeB09} (0.75)& \cellcolor{red!40}\href{../works/GodardLN05.pdf}{GodardLN05} (0.74)\\
\index{DannaP04}DannaP04 R\&C& \cellcolor{red!20}BaptisteLPN06 (0.86)& \cellcolor{red!20}\href{../works/SourdN00.pdf}{SourdN00} (0.88)& \cellcolor{red!20}AjiliW04 (0.88)& \cellcolor{yellow!20}\href{../works/PerronSF04.pdf}{PerronSF04} (0.90)& \cellcolor{yellow!20}\href{../works/CarchraeB09.pdf}{CarchraeB09} (0.91)\\
Euclid\\
Dot\\
Cosine\\
\index{Darby-DowmanLMZ97}\href{../works/Darby-DowmanLMZ97.pdf}{Darby-DowmanLMZ97} R\&C& \cellcolor{red!40}\href{../works/SmithBHW96.pdf}{SmithBHW96} (0.69)& \cellcolor{red!40}DarbyDowmanL98 (0.80)& \cellcolor{red!20}\href{../works/RodosekWH99.pdf}{RodosekWH99} (0.89)& \cellcolor{yellow!20}\href{../works/NuijtenA96.pdf}{NuijtenA96} (0.91)& \cellcolor{yellow!20}BockmayrK98 (0.92)\\
Euclid& \cellcolor{green!20}\href{../works/RodosekWH99.pdf}{RodosekWH99} (0.29)& \cellcolor{blue!20}\href{../works/HookerO99.pdf}{HookerO99} (0.34)& \cellcolor{black!20}\href{../works/Bartak02.pdf}{Bartak02} (0.34)& \cellcolor{black!20}\href{../works/HarjunkoskiJG00.pdf}{HarjunkoskiJG00} (0.35)& \cellcolor{black!20}\href{../works/BockmayrP06.pdf}{BockmayrP06} (0.35)\\
Dot& \cellcolor{red!40}\href{../works/Malapert11.pdf}{Malapert11} (109.00)& \cellcolor{red!40}\href{../works/Baptiste02.pdf}{Baptiste02} (98.00)& \cellcolor{red!40}\href{../works/Lombardi10.pdf}{Lombardi10} (97.00)& \cellcolor{red!40}\href{../works/Groleaz21.pdf}{Groleaz21} (94.00)& \cellcolor{red!40}\href{../works/LaborieRSV18.pdf}{LaborieRSV18} (90.00)\\
Cosine& \cellcolor{red!40}\href{../works/RodosekWH99.pdf}{RodosekWH99} (0.75)& \cellcolor{red!40}\href{../works/HookerO99.pdf}{HookerO99} (0.69)& \cellcolor{red!40}\href{../works/BadicaBI20.pdf}{BadicaBI20} (0.64)& \cellcolor{red!40}\href{../works/Bartak02.pdf}{Bartak02} (0.63)& \cellcolor{red!40}\href{../works/RodosekW98.pdf}{RodosekW98} (0.62)\\
\index{DarbyDowmanL98}DarbyDowmanL98 R\&C& \cellcolor{red!40}\href{../works/Darby-DowmanLMZ97.pdf}{Darby-DowmanLMZ97} (0.80)& \cellcolor{red!40}\href{../works/SmithBHW96.pdf}{SmithBHW96} (0.82)& \cellcolor{yellow!20}\href{../works/RodosekWH99.pdf}{RodosekWH99} (0.91)& \cellcolor{yellow!20}\href{../works/RodosekW98.pdf}{RodosekW98} (0.92)& \cellcolor{yellow!20}\href{../works/EdisO11.pdf}{EdisO11} (0.92)\\
Euclid\\
Dot\\
Cosine\\
\index{Davenport10}\href{../works/Davenport10.pdf}{Davenport10} R\&C& \cellcolor{red!40}\href{../works/Limtanyakul07.pdf}{Limtanyakul07} (0.80)& \cellcolor{red!40}\href{../works/Vilim09.pdf}{Vilim09} (0.83)& \cellcolor{red!40}\href{../works/Vilim09a.pdf}{Vilim09a} (0.83)& \cellcolor{red!40}\href{../works/KameugneF13.pdf}{KameugneF13} (0.83)& \cellcolor{red!40}\href{../works/VilimBC05.pdf}{VilimBC05} (0.86)\\
Euclid& \cellcolor{red!20}\href{../works/FrostD98.pdf}{FrostD98} (0.24)& \cellcolor{red!20}\href{../works/CarchraeBF05.pdf}{CarchraeBF05} (0.26)& \cellcolor{red!20}\href{../works/KovacsEKV05.pdf}{KovacsEKV05} (0.26)& \cellcolor{red!20}\href{../works/AngelsmarkJ00.pdf}{AngelsmarkJ00} (0.26)& \cellcolor{red!20}\href{../works/CestaOS98.pdf}{CestaOS98} (0.26)\\
Dot& \cellcolor{red!40}\href{../works/ZarandiASC20.pdf}{ZarandiASC20} (63.00)& \cellcolor{red!40}\href{../works/Groleaz21.pdf}{Groleaz21} (63.00)& \cellcolor{red!40}\href{../works/KelbelH11.pdf}{KelbelH11} (55.00)& \cellcolor{red!40}\href{../works/TerekhovDOB12.pdf}{TerekhovDOB12} (55.00)& \cellcolor{red!40}\href{../works/Lombardi10.pdf}{Lombardi10} (55.00)\\
Cosine& \cellcolor{red!40}\href{../works/KeriK07.pdf}{KeriK07} (0.60)& \cellcolor{red!40}\href{../works/Hooker05a.pdf}{Hooker05a} (0.57)& \cellcolor{red!40}\href{../works/BeckR03.pdf}{BeckR03} (0.56)& \cellcolor{red!40}\href{../works/KrogtLPHJ07.pdf}{KrogtLPHJ07} (0.56)& \cellcolor{red!40}\href{../works/Hooker06.pdf}{Hooker06} (0.55)\\
\index{DavenportKRSH07}\href{../works/DavenportKRSH07.pdf}{DavenportKRSH07} R\&C& \cellcolor{red!40}\href{../works/SimoninAHL15.pdf}{SimoninAHL15} (0.86)& \cellcolor{red!20}\href{../works/BeldiceanuP07.pdf}{BeldiceanuP07} (0.89)& \cellcolor{red!20}\href{../works/SimoninAHL12.pdf}{SimoninAHL12} (0.90)& \cellcolor{red!20}\href{../works/SerraNM12.pdf}{SerraNM12} (0.90)& \cellcolor{red!20}\href{../works/AntuoriHHEN20.pdf}{AntuoriHHEN20} (0.90)\\
Euclid& \cellcolor{red!20}\href{../works/CauwelaertDMS16.pdf}{CauwelaertDMS16} (0.25)& \cellcolor{yellow!20}\href{../works/MakMS10.pdf}{MakMS10} (0.27)& \cellcolor{yellow!20}\href{../works/Vilim04.pdf}{Vilim04} (0.28)& \cellcolor{green!20}\href{../works/GilesH16.pdf}{GilesH16} (0.29)& \cellcolor{green!20}\href{../works/HarjunkoskiG02.pdf}{HarjunkoskiG02} (0.29)\\
Dot& \cellcolor{red!40}\href{../works/Astrand21.pdf}{Astrand21} (109.00)& \cellcolor{red!40}\href{../works/Baptiste02.pdf}{Baptiste02} (104.00)& \cellcolor{red!40}\href{../works/Dejemeppe16.pdf}{Dejemeppe16} (103.00)& \cellcolor{red!40}\href{../works/ZarandiASC20.pdf}{ZarandiASC20} (103.00)& \cellcolor{red!40}\href{../works/Malapert11.pdf}{Malapert11} (103.00)\\
Cosine& \cellcolor{red!40}\href{../works/CauwelaertDMS16.pdf}{CauwelaertDMS16} (0.80)& \cellcolor{red!40}\href{../works/FocacciLN00.pdf}{FocacciLN00} (0.76)& \cellcolor{red!40}\href{../works/CauwelaertDS20.pdf}{CauwelaertDS20} (0.76)& \cellcolor{red!40}\href{../works/Pralet17.pdf}{Pralet17} (0.75)& \cellcolor{red!40}\href{../works/HarjunkoskiG02.pdf}{HarjunkoskiG02} (0.74)\\
\index{Davis87}\href{../works/Davis87.pdf}{Davis87} R\&C& \cellcolor{red!20}\href{../works/DincbasSH90.pdf}{DincbasSH90} (0.89)& \cellcolor{yellow!20}\href{../works/BeckF00.pdf}{BeckF00} (0.93)& \cellcolor{green!20}\href{../works/Dorndorf2000.pdf}{Dorndorf2000} (0.94)& \cellcolor{green!20}\href{../works/SadehF96.pdf}{SadehF96} (0.94)& \cellcolor{green!20}\href{../works/Salido10.pdf}{Salido10} (0.95)\\
Euclid& \cellcolor{red!40}\href{../works/Valdes87.pdf}{Valdes87} (0.12)& \cellcolor{red!40}\href{../works/FrostD98.pdf}{FrostD98} (0.15)& \cellcolor{red!40}\href{../works/HebrardTW05.pdf}{HebrardTW05} (0.15)& \cellcolor{red!40}\href{../works/LiuJ06.pdf}{LiuJ06} (0.15)& \cellcolor{red!40}\href{../works/AngelsmarkJ00.pdf}{AngelsmarkJ00} (0.15)\\
Dot& \cellcolor{red!40}\href{../works/Astrand21.pdf}{Astrand21} (36.00)& \cellcolor{red!40}\href{../works/Lombardi10.pdf}{Lombardi10} (35.00)& \cellcolor{red!40}\href{../works/KoehlerBFFHPSSS21.pdf}{KoehlerBFFHPSSS21} (34.00)& \cellcolor{red!40}\href{../works/Dejemeppe16.pdf}{Dejemeppe16} (34.00)& \cellcolor{red!40}\href{../works/Malapert11.pdf}{Malapert11} (34.00)\\
Cosine& \cellcolor{red!40}\href{../works/Valdes87.pdf}{Valdes87} (0.77)& \cellcolor{red!40}\href{../works/MurphyMB15.pdf}{MurphyMB15} (0.68)& \cellcolor{red!40}\href{../works/RoweJCA96.pdf}{RoweJCA96} (0.67)& \cellcolor{red!40}\href{../works/BukchinR18.pdf}{BukchinR18} (0.64)& \cellcolor{red!40}\href{../works/AstrandJZ18.pdf}{AstrandJZ18} (0.64)\\
\index{Dejemeppe16}\href{../works/Dejemeppe16.pdf}{Dejemeppe16} R\&C\\
Euclid& \href{../works/GrimesH15.pdf}{GrimesH15} (0.62)& \href{../works/Fahimi16.pdf}{Fahimi16} (0.63)& \href{../works/DejemeppeCS15.pdf}{DejemeppeCS15} (0.63)& \href{../works/BartakSR10.pdf}{BartakSR10} (0.64)& \href{../works/Beck99.pdf}{Beck99} (0.64)\\
Dot& \cellcolor{red!40}\href{../works/ZarandiASC20.pdf}{ZarandiASC20} (335.00)& \cellcolor{red!40}\href{../works/Groleaz21.pdf}{Groleaz21} (324.00)& \cellcolor{red!40}\href{../works/Malapert11.pdf}{Malapert11} (307.00)& \cellcolor{red!40}\href{../works/Baptiste02.pdf}{Baptiste02} (299.00)& \cellcolor{red!40}\href{../works/Lombardi10.pdf}{Lombardi10} (279.00)\\
Cosine& \cellcolor{red!40}\href{../works/GrimesH15.pdf}{GrimesH15} (0.68)& \cellcolor{red!40}\href{../works/Fahimi16.pdf}{Fahimi16} (0.68)& \cellcolor{red!40}\href{../works/DejemeppeCS15.pdf}{DejemeppeCS15} (0.67)& \cellcolor{red!40}\href{../works/Malapert11.pdf}{Malapert11} (0.66)& \cellcolor{red!40}\href{../works/Groleaz21.pdf}{Groleaz21} (0.66)\\
\index{DejemeppeCS15}\href{../works/DejemeppeCS15.pdf}{DejemeppeCS15} R\&C& \cellcolor{red!40}\href{../works/CauwelaertDMS16.pdf}{CauwelaertDMS16} (0.74)& \cellcolor{red!40}\href{../works/CauwelaertDS20.pdf}{CauwelaertDS20} (0.75)& \cellcolor{red!40}\href{../works/GayHS15a.pdf}{GayHS15a} (0.76)& \cellcolor{red!40}\href{../works/GrimesH10.pdf}{GrimesH10} (0.77)& \cellcolor{red!40}\href{../works/GaySS14.pdf}{GaySS14} (0.79)\\
Euclid& \cellcolor{red!40}\href{../works/CauwelaertDMS16.pdf}{CauwelaertDMS16} (0.19)& \cellcolor{red!40}\href{../works/CauwelaertDS20.pdf}{CauwelaertDS20} (0.19)& \cellcolor{red!40}\href{../works/VilimBC05.pdf}{VilimBC05} (0.22)& \cellcolor{red!40}\href{../works/VilimBC04.pdf}{VilimBC04} (0.23)& \cellcolor{red!20}\href{../works/Vilim04.pdf}{Vilim04} (0.25)\\
Dot& \cellcolor{red!40}\href{../works/Dejemeppe16.pdf}{Dejemeppe16} (173.00)& \cellcolor{red!40}\href{../works/Baptiste02.pdf}{Baptiste02} (158.00)& \cellcolor{red!40}\href{../works/Malapert11.pdf}{Malapert11} (152.00)& \cellcolor{red!40}\href{../works/Groleaz21.pdf}{Groleaz21} (152.00)& \cellcolor{red!40}\href{../works/Fahimi16.pdf}{Fahimi16} (147.00)\\
Cosine& \cellcolor{red!40}\href{../works/CauwelaertDS20.pdf}{CauwelaertDS20} (0.91)& \cellcolor{red!40}\href{../works/CauwelaertDMS16.pdf}{CauwelaertDMS16} (0.91)& \cellcolor{red!40}\href{../works/VilimBC05.pdf}{VilimBC05} (0.88)& \cellcolor{red!40}\href{../works/VilimBC04.pdf}{VilimBC04} (0.86)& \cellcolor{red!40}\href{../works/Vilim04.pdf}{Vilim04} (0.83)\\
\index{DejemeppeD14}\href{../works/DejemeppeD14.pdf}{DejemeppeD14} R\&C& \cellcolor{red!20}\href{../works/DannaP03.pdf}{DannaP03} (0.90)& \cellcolor{yellow!20}\href{../works/SchausHMCMD11.pdf}{SchausHMCMD11} (0.92)& \cellcolor{yellow!20}\href{../works/GarganiR07.pdf}{GarganiR07} (0.92)& \cellcolor{yellow!20}\href{../works/VerfaillieL01.pdf}{VerfaillieL01} (0.92)& \cellcolor{yellow!20}\href{../works/PesantRR15.pdf}{PesantRR15} (0.93)\\
Euclid& \cellcolor{yellow!20}\href{../works/BeckW05.pdf}{BeckW05} (0.28)& \cellcolor{green!20}\href{../works/BhatnagarKL19.pdf}{BhatnagarKL19} (0.29)& \cellcolor{green!20}\href{../works/BonfiettiM12.pdf}{BonfiettiM12} (0.30)& \cellcolor{green!20}\href{../works/Caseau97.pdf}{Caseau97} (0.30)& \cellcolor{green!20}\href{../works/GalleguillosKSB19.pdf}{GalleguillosKSB19} (0.31)\\
Dot& \cellcolor{red!40}\href{../works/Dejemeppe16.pdf}{Dejemeppe16} (123.00)& \cellcolor{red!40}\href{../works/ZarandiASC20.pdf}{ZarandiASC20} (94.00)& \cellcolor{red!40}\href{../works/Malapert11.pdf}{Malapert11} (90.00)& \cellcolor{red!40}\href{../works/LaborieRSV18.pdf}{LaborieRSV18} (87.00)& \cellcolor{red!40}\href{../works/Lombardi10.pdf}{Lombardi10} (86.00)\\
Cosine& \cellcolor{red!40}\href{../works/GodardLN05.pdf}{GodardLN05} (0.71)& \cellcolor{red!40}\href{../works/BeckW05.pdf}{BeckW05} (0.70)& \cellcolor{red!40}\href{../works/Pralet17.pdf}{Pralet17} (0.68)& \cellcolor{red!40}\href{../works/CappartTSR18.pdf}{CappartTSR18} (0.68)& \cellcolor{red!40}\href{../works/SenderovichBB19.pdf}{SenderovichBB19} (0.68)\\
\index{Demassey03}\href{../works/Demassey03.pdf}{Demassey03} R\&C\\
Euclid& \href{../works/DemasseyAM05.pdf}{DemasseyAM05} (0.39)& \href{../works/Kameugne14.pdf}{Kameugne14} (0.39)& \href{../works/BaptistePN99.pdf}{BaptistePN99} (0.40)& \href{../works/BartakSR08.pdf}{BartakSR08} (0.42)& \href{../works/Derrien15.pdf}{Derrien15} (0.42)\\
Dot& \cellcolor{red!40}\href{../works/Baptiste02.pdf}{Baptiste02} (198.00)& \cellcolor{red!40}\href{../works/Schutt11.pdf}{Schutt11} (191.00)& \cellcolor{red!40}\href{../works/Malapert11.pdf}{Malapert11} (190.00)& \cellcolor{red!40}\href{../works/Fahimi16.pdf}{Fahimi16} (187.00)& \cellcolor{red!40}\href{../works/Lombardi10.pdf}{Lombardi10} (184.00)\\
Cosine& \cellcolor{red!40}\href{../works/Kameugne14.pdf}{Kameugne14} (0.77)& \cellcolor{red!40}\href{../works/DemasseyAM05.pdf}{DemasseyAM05} (0.74)& \cellcolor{red!40}\href{../works/BaptistePN99.pdf}{BaptistePN99} (0.72)& \cellcolor{red!40}\href{../works/Elkhyari03.pdf}{Elkhyari03} (0.71)& \cellcolor{red!40}\href{../works/BartakSR08.pdf}{BartakSR08} (0.70)\\
\index{DemasseyAM05}\href{../works/DemasseyAM05.pdf}{DemasseyAM05} R\&C& \cellcolor{red!40}NeronABCDD06 (0.71)& \cellcolor{red!40}\href{../works/BruckerK00.pdf}{BruckerK00} (0.73)& \cellcolor{red!40}\href{../works/LiessM08.pdf}{LiessM08} (0.77)& \cellcolor{red!40}\href{../works/ArkhipovBL19.pdf}{ArkhipovBL19} (0.80)& \cellcolor{red!40}DorndorfHP99 (0.83)\\
Euclid& \cellcolor{red!40}\href{../works/LiessM08.pdf}{LiessM08} (0.23)& \cellcolor{red!40}\href{../works/BaptisteP97.pdf}{BaptisteP97} (0.24)& \cellcolor{yellow!20}\href{../works/BofillCSV17a.pdf}{BofillCSV17a} (0.26)& \cellcolor{yellow!20}\href{../works/Laborie05.pdf}{Laborie05} (0.27)& \cellcolor{yellow!20}\href{../works/CestaOF99.pdf}{CestaOF99} (0.27)\\
Dot& \cellcolor{red!40}\href{../works/Baptiste02.pdf}{Baptiste02} (161.00)& \cellcolor{red!40}\href{../works/Lombardi10.pdf}{Lombardi10} (159.00)& \cellcolor{red!40}\href{../works/Schutt11.pdf}{Schutt11} (156.00)& \cellcolor{red!40}\href{../works/Dejemeppe16.pdf}{Dejemeppe16} (150.00)& \cellcolor{red!40}\href{../works/Groleaz21.pdf}{Groleaz21} (149.00)\\
Cosine& \cellcolor{red!40}\href{../works/LiessM08.pdf}{LiessM08} (0.86)& \cellcolor{red!40}\href{../works/BaptisteP97.pdf}{BaptisteP97} (0.86)& \cellcolor{red!40}\href{../works/Laborie05.pdf}{Laborie05} (0.82)& \cellcolor{red!40}\href{../works/BaptistePN99.pdf}{BaptistePN99} (0.82)& \cellcolor{red!40}\href{../works/VilimLS15.pdf}{VilimLS15} (0.81)\\
\index{DemirovicS18}\href{../works/DemirovicS18.pdf}{DemirovicS18} R\&C& \cellcolor{green!20}\href{../works/KreterSS15.pdf}{KreterSS15} (0.94)& \cellcolor{green!20}\href{../works/KreterSS17.pdf}{KreterSS17} (0.94)& \cellcolor{green!20}\href{../works/MusliuSS18.pdf}{MusliuSS18} (0.95)& \cellcolor{green!20}\href{../works/FrimodigS19.pdf}{FrimodigS19} (0.95)& \cellcolor{green!20}\href{../works/FrohnerTR19.pdf}{FrohnerTR19} (0.95)\\
Euclid& \cellcolor{green!20}\href{../works/ShaikhK23.pdf}{ShaikhK23} (0.29)& \cellcolor{green!20}\href{../works/PesantGPR99.pdf}{PesantGPR99} (0.30)& \cellcolor{green!20}\href{../works/Puget95.pdf}{Puget95} (0.30)& \cellcolor{green!20}\href{../works/QuSN06.pdf}{QuSN06} (0.30)& \cellcolor{green!20}\href{../works/BeniniBGM05a.pdf}{BeniniBGM05a} (0.30)\\
Dot& \cellcolor{red!40}\href{../works/Astrand21.pdf}{Astrand21} (88.00)& \cellcolor{red!40}\href{../works/Lemos21.pdf}{Lemos21} (87.00)& \cellcolor{red!40}\href{../works/Fahimi16.pdf}{Fahimi16} (86.00)& \cellcolor{red!40}\href{../works/Groleaz21.pdf}{Groleaz21} (85.00)& \cellcolor{red!40}\href{../works/Godet21a.pdf}{Godet21a} (84.00)\\
Cosine& \cellcolor{red!40}\href{../works/ElkhyariGJ02a.pdf}{ElkhyariGJ02a} (0.68)& \cellcolor{red!40}\href{../works/ShaikhK23.pdf}{ShaikhK23} (0.66)& \cellcolor{red!40}\href{../works/NishikawaSTT18.pdf}{NishikawaSTT18} (0.65)& \cellcolor{red!40}\href{../works/NishikawaSTT19.pdf}{NishikawaSTT19} (0.65)& \cellcolor{red!40}\href{../works/HeipckeCCS00.pdf}{HeipckeCCS00} (0.65)\\
\index{Derrien15}\href{../works/Derrien15.pdf}{Derrien15} R\&C\\
Euclid& \cellcolor{blue!20}\href{../works/Clercq12.pdf}{Clercq12} (0.33)& \cellcolor{black!20}\href{../works/DerrienP14.pdf}{DerrienP14} (0.36)& \cellcolor{black!20}\href{../works/KovacsV04.pdf}{KovacsV04} (0.36)& \cellcolor{black!20}\href{../works/Vilim11.pdf}{Vilim11} (0.36)& \cellcolor{black!20}\href{../works/Letort13.pdf}{Letort13} (0.36)\\
Dot& \cellcolor{red!40}\href{../works/Fahimi16.pdf}{Fahimi16} (165.00)& \cellcolor{red!40}\href{../works/Malapert11.pdf}{Malapert11} (161.00)& \cellcolor{red!40}\href{../works/Godet21a.pdf}{Godet21a} (158.00)& \cellcolor{red!40}\href{../works/Dejemeppe16.pdf}{Dejemeppe16} (150.00)& \cellcolor{red!40}\href{../works/Schutt11.pdf}{Schutt11} (149.00)\\
Cosine& \cellcolor{red!40}\href{../works/Clercq12.pdf}{Clercq12} (0.79)& \cellcolor{red!40}\href{../works/Letort13.pdf}{Letort13} (0.75)& \cellcolor{red!40}\href{../works/Fahimi16.pdf}{Fahimi16} (0.71)& \cellcolor{red!40}\href{../works/Kameugne14.pdf}{Kameugne14} (0.71)& \cellcolor{red!40}\href{../works/FahimiOQ18.pdf}{FahimiOQ18} (0.71)\\
\index{DerrienP14}\href{../works/DerrienP14.pdf}{DerrienP14} R\&C& \cellcolor{red!40}\href{../works/SchuttW10.pdf}{SchuttW10} (0.78)& \cellcolor{red!40}\href{../works/LetortBC12.pdf}{LetortBC12} (0.78)& \cellcolor{red!40}\href{../works/Tesch18.pdf}{Tesch18} (0.79)& \cellcolor{red!40}\href{../works/GayHS15.pdf}{GayHS15} (0.79)& \cellcolor{red!40}\href{../works/KameugneFSN14.pdf}{KameugneFSN14} (0.80)\\
Euclid& \cellcolor{red!20}\href{../works/Tesch16.pdf}{Tesch16} (0.26)& \cellcolor{red!20}\href{../works/Vilim11.pdf}{Vilim11} (0.26)& \cellcolor{red!20}\href{../works/WolfS05.pdf}{WolfS05} (0.26)& \cellcolor{yellow!20}\href{../works/Vilim09.pdf}{Vilim09} (0.26)& \cellcolor{yellow!20}\href{../works/ZibranR11.pdf}{ZibranR11} (0.26)\\
Dot& \cellcolor{red!40}\href{../works/Fahimi16.pdf}{Fahimi16} (83.00)& \cellcolor{red!40}\href{../works/Schutt11.pdf}{Schutt11} (80.00)& \cellcolor{red!40}\href{../works/SchuttFS13a.pdf}{SchuttFS13a} (78.00)& \cellcolor{red!40}\href{../works/Malapert11.pdf}{Malapert11} (76.00)& \cellcolor{red!40}\href{../works/Derrien15.pdf}{Derrien15} (76.00)\\
Cosine& \cellcolor{red!40}\href{../works/Vilim11.pdf}{Vilim11} (0.77)& \cellcolor{red!40}\href{../works/Tesch16.pdf}{Tesch16} (0.75)& \cellcolor{red!40}\href{../works/DerrienPZ14.pdf}{DerrienPZ14} (0.72)& \cellcolor{red!40}\href{../works/Vilim09.pdf}{Vilim09} (0.69)& \cellcolor{red!40}\href{../works/OuelletQ18.pdf}{OuelletQ18} (0.69)\\
\index{DerrienPZ14}\href{../works/DerrienPZ14.pdf}{DerrienPZ14} R\&C& \cellcolor{red!40}\href{../works/FahimiOQ18.pdf}{FahimiOQ18} (0.73)& \cellcolor{red!40}\href{../works/Mercier-AubinGQ20.pdf}{Mercier-AubinGQ20} (0.80)& \cellcolor{red!40}\href{../works/GayHS15.pdf}{GayHS15} (0.82)& \cellcolor{red!20}\href{../works/KameugneFSN14.pdf}{KameugneFSN14} (0.89)& \cellcolor{red!20}\href{../works/Madi-WambaLOBM17.pdf}{Madi-WambaLOBM17} (0.89)\\
Euclid& \cellcolor{yellow!20}\href{../works/BhatnagarKL19.pdf}{BhatnagarKL19} (0.27)& \cellcolor{yellow!20}\href{../works/BonfiettiZLM16.pdf}{BonfiettiZLM16} (0.28)& \cellcolor{yellow!20}\href{../works/DerrienP14.pdf}{DerrienP14} (0.28)& \cellcolor{green!20}\href{../works/Tesch16.pdf}{Tesch16} (0.29)& \cellcolor{green!20}\href{../works/BofillCSV17.pdf}{BofillCSV17} (0.29)\\
Dot& \cellcolor{red!40}\href{../works/Fahimi16.pdf}{Fahimi16} (105.00)& \cellcolor{red!40}\href{../works/Malapert11.pdf}{Malapert11} (101.00)& \cellcolor{red!40}\href{../works/Dejemeppe16.pdf}{Dejemeppe16} (101.00)& \cellcolor{red!40}\href{../works/LaborieRSV18.pdf}{LaborieRSV18} (99.00)& \cellcolor{red!40}\href{../works/Godet21a.pdf}{Godet21a} (98.00)\\
Cosine& \cellcolor{red!40}\href{../works/BonfiettiZLM16.pdf}{BonfiettiZLM16} (0.75)& \cellcolor{red!40}\href{../works/Tesch16.pdf}{Tesch16} (0.72)& \cellcolor{red!40}\href{../works/DerrienP14.pdf}{DerrienP14} (0.72)& \cellcolor{red!40}\href{../works/BhatnagarKL19.pdf}{BhatnagarKL19} (0.71)& \cellcolor{red!40}\href{../works/ClercqPBJ11.pdf}{ClercqPBJ11} (0.71)\\
\index{DilkinaDH05}\href{../works/DilkinaDH05.pdf}{DilkinaDH05} R\&C& \cellcolor{red!20}\href{../works/ArtiguesR00.pdf}{ArtiguesR00} (0.90)& \cellcolor{yellow!20}\href{../works/ElkhyariGJ02.pdf}{ElkhyariGJ02} (0.92)& \cellcolor{green!20}\href{../works/ArtiguesBF04.pdf}{ArtiguesBF04} (0.94)& \cellcolor{green!20}\href{../works/GrimesHM09.pdf}{GrimesHM09} (0.95)& \cellcolor{green!20}\href{../works/TorresL00.pdf}{TorresL00} (0.95)\\
Euclid& \cellcolor{red!40}\href{../works/HebrardTW05.pdf}{HebrardTW05} (0.23)& \cellcolor{red!40}\href{../works/CrawfordB94.pdf}{CrawfordB94} (0.24)& \cellcolor{red!20}\href{../works/FoxAS82.pdf}{FoxAS82} (0.25)& \cellcolor{red!20}\href{../works/DoRZ08.pdf}{DoRZ08} (0.25)& \cellcolor{red!20}\href{../works/OddiS97.pdf}{OddiS97} (0.25)\\
Dot& \cellcolor{red!40}\href{../works/Dejemeppe16.pdf}{Dejemeppe16} (75.00)& \cellcolor{red!40}\href{../works/ZarandiASC20.pdf}{ZarandiASC20} (74.00)& \cellcolor{red!40}\href{../works/Groleaz21.pdf}{Groleaz21} (74.00)& \cellcolor{red!40}\href{../works/Beck99.pdf}{Beck99} (72.00)& \cellcolor{red!40}\href{../works/BeckDDF98.pdf}{BeckDDF98} (72.00)\\
Cosine& \cellcolor{red!40}\href{../works/OddiS97.pdf}{OddiS97} (0.71)& \cellcolor{red!40}\href{../works/SmithC93.pdf}{SmithC93} (0.70)& \cellcolor{red!40}\href{../works/HebrardTW05.pdf}{HebrardTW05} (0.70)& \cellcolor{red!40}\href{../works/BillautHL12.pdf}{BillautHL12} (0.70)& \cellcolor{red!40}\href{../works/PengLC14.pdf}{PengLC14} (0.69)\\
\index{DilkinaH04}\href{../works/DilkinaH04.pdf}{DilkinaH04} R\&C\\
Euclid& \cellcolor{red!40}\href{../works/Tsang03.pdf}{Tsang03} (0.19)& \cellcolor{red!40}\href{../works/FeldmanG89.pdf}{FeldmanG89} (0.21)& \cellcolor{red!40}\href{../works/GelainPRVW17.pdf}{GelainPRVW17} (0.22)& \cellcolor{red!40}\href{../works/AbrilSB05.pdf}{AbrilSB05} (0.22)& \cellcolor{red!40}\href{../works/Vilim03.pdf}{Vilim03} (0.23)\\
Dot& \cellcolor{red!40}\href{../works/ZarandiASC20.pdf}{ZarandiASC20} (53.00)& \cellcolor{red!40}\href{../works/Lemos21.pdf}{Lemos21} (49.00)& \cellcolor{red!40}\href{../works/HookerH17.pdf}{HookerH17} (47.00)& \cellcolor{red!40}\href{../works/Lombardi10.pdf}{Lombardi10} (47.00)& \cellcolor{red!40}\href{../works/KendallKRU10.pdf}{KendallKRU10} (46.00)\\
Cosine& \cellcolor{red!40}\href{../works/GelainPRVW17.pdf}{GelainPRVW17} (0.67)& \cellcolor{red!40}\href{../works/BartakS11.pdf}{BartakS11} (0.67)& \cellcolor{red!40}\href{../works/Tsang03.pdf}{Tsang03} (0.65)& \cellcolor{red!40}\href{../works/HenzMT04.pdf}{HenzMT04} (0.65)& \cellcolor{red!40}\href{../works/LiuLH19.pdf}{LiuLH19} (0.63)\\
\index{DincbasS91}\href{../works/DincbasS91.pdf}{DincbasS91} R\&C\\
Euclid& \cellcolor{red!40}\href{../works/CestaOS98.pdf}{CestaOS98} (0.23)& \cellcolor{red!40}\href{../works/KovacsEKV05.pdf}{KovacsEKV05} (0.23)& \cellcolor{red!40}\href{../works/Simonis95.pdf}{Simonis95} (0.24)& \cellcolor{red!40}\href{../works/AngelsmarkJ00.pdf}{AngelsmarkJ00} (0.24)& \cellcolor{red!40}\href{../works/Caballero23.pdf}{Caballero23} (0.24)\\
Dot& \cellcolor{red!40}\href{../works/Simonis07.pdf}{Simonis07} (52.00)& \cellcolor{red!40}\href{../works/Simonis99.pdf}{Simonis99} (48.00)& \cellcolor{red!40}\href{../works/Wallace96.pdf}{Wallace96} (43.00)& \cellcolor{red!40}\href{../works/ZarandiASC20.pdf}{ZarandiASC20} (41.00)& \cellcolor{red!40}\href{../works/Simonis95a.pdf}{Simonis95a} (39.00)\\
Cosine& \cellcolor{red!40}\href{../works/Simonis95.pdf}{Simonis95} (0.63)& \cellcolor{red!40}\href{../works/TranDRFWOVB16.pdf}{TranDRFWOVB16} (0.61)& \cellcolor{red!40}\href{../works/JoLLH99.pdf}{JoLLH99} (0.59)& \cellcolor{red!40}\href{../works/ChunS14.pdf}{ChunS14} (0.56)& \cellcolor{red!40}\href{../works/GruianK98.pdf}{GruianK98} (0.55)\\
\index{DincbasSH90}\href{../works/DincbasSH90.pdf}{DincbasSH90} R\&C& \cellcolor{red!40}\href{../works/Wallace96.pdf}{Wallace96} (0.86)& \cellcolor{red!40}\href{../works/NuijtenA96.pdf}{NuijtenA96} (0.86)& \cellcolor{red!20}\href{../works/Davis87.pdf}{Davis87} (0.89)& \cellcolor{red!20}\href{../works/BeckF00.pdf}{BeckF00} (0.89)& \cellcolor{red!20}\href{../works/Salido10.pdf}{Salido10} (0.90)\\
Euclid& \cellcolor{red!20}\href{../works/Rit86.pdf}{Rit86} (0.26)& \cellcolor{yellow!20}\href{../works/Simonis95.pdf}{Simonis95} (0.27)& \cellcolor{yellow!20}\href{../works/LammaMM97.pdf}{LammaMM97} (0.28)& \cellcolor{green!20}\href{../works/Zhou96.pdf}{Zhou96} (0.30)& \cellcolor{green!20}\href{../works/BrusoniCLMMT96.pdf}{BrusoniCLMMT96} (0.30)\\
Dot& \cellcolor{red!40}\href{../works/Baptiste02.pdf}{Baptiste02} (94.00)& \cellcolor{red!40}\href{../works/Malapert11.pdf}{Malapert11} (92.00)& \cellcolor{red!40}\href{../works/AggounB93.pdf}{AggounB93} (91.00)& \cellcolor{red!40}\href{../works/Simonis99.pdf}{Simonis99} (86.00)& \cellcolor{red!40}\href{../works/RoePS05.pdf}{RoePS05} (86.00)\\
Cosine& \cellcolor{red!40}\href{../works/LammaMM97.pdf}{LammaMM97} (0.76)& \cellcolor{red!40}\href{../works/AggounB93.pdf}{AggounB93} (0.74)& \cellcolor{red!40}\href{../works/Rit86.pdf}{Rit86} (0.73)& \cellcolor{red!40}\href{../works/Simonis95a.pdf}{Simonis95a} (0.71)& \cellcolor{red!40}\href{../works/Simonis95.pdf}{Simonis95} (0.69)\\
\index{DoRZ08}\href{../works/DoRZ08.pdf}{DoRZ08} R\&C\\
Euclid& \cellcolor{red!40}\href{../works/CrawfordB94.pdf}{CrawfordB94} (0.20)& \cellcolor{red!40}\href{../works/LauLN08.pdf}{LauLN08} (0.22)& \cellcolor{red!40}\href{../works/FoxAS82.pdf}{FoxAS82} (0.23)& \cellcolor{red!40}\href{../works/FukunagaHFAMN02.pdf}{FukunagaHFAMN02} (0.23)& \cellcolor{red!20}\href{../works/DilkinaDH05.pdf}{DilkinaDH05} (0.25)\\
Dot& \cellcolor{red!40}\href{../works/Lunardi20.pdf}{Lunardi20} (71.00)& \cellcolor{red!40}\href{../works/ZarandiASC20.pdf}{ZarandiASC20} (70.00)& \cellcolor{red!40}\href{../works/Astrand21.pdf}{Astrand21} (70.00)& \cellcolor{red!40}\href{../works/GombolayWS18.pdf}{GombolayWS18} (69.00)& \cellcolor{red!40}\href{../works/PrataAN23.pdf}{PrataAN23} (68.00)\\
Cosine& \cellcolor{red!40}\href{../works/CrawfordB94.pdf}{CrawfordB94} (0.77)& \cellcolor{red!40}\href{../works/LauLN08.pdf}{LauLN08} (0.73)& \cellcolor{red!40}\href{../works/HebrardHJMPV16.pdf}{HebrardHJMPV16} (0.70)& \cellcolor{red!40}\href{../works/FoxAS82.pdf}{FoxAS82} (0.70)& \cellcolor{red!40}\href{../works/GetoorOFC97.pdf}{GetoorOFC97} (0.70)\\
\index{DomdorfPH03}DomdorfPH03 R\&C& \cellcolor{red!40}\href{../works/BlazewiczDP96.pdf}{BlazewiczDP96} (0.80)& \cellcolor{red!40}\href{../works/Dorndorf2000.pdf}{Dorndorf2000} (0.82)& \cellcolor{red!20}\href{../works/JainM99.pdf}{JainM99} (0.86)& \cellcolor{red!20}DorndorfHP99 (0.87)& \cellcolor{red!20}\href{../works/SourdN00.pdf}{SourdN00} (0.87)\\
Euclid\\
Dot\\
Cosine\\
\index{DoomsH08}\href{../works/DoomsH08.pdf}{DoomsH08} R\&C\\
Euclid& \cellcolor{red!40}\href{../works/BonfiettiM12.pdf}{BonfiettiM12} (0.22)& \cellcolor{red!40}\href{../works/BeckW05.pdf}{BeckW05} (0.23)& \cellcolor{red!40}\href{../works/LombardiM13.pdf}{LombardiM13} (0.24)& \cellcolor{red!40}\href{../works/Caballero23.pdf}{Caballero23} (0.24)& \cellcolor{red!40}\href{../works/WallaceF00.pdf}{WallaceF00} (0.24)\\
Dot& \cellcolor{red!40}\href{../works/Lombardi10.pdf}{Lombardi10} (80.00)& \cellcolor{red!40}\href{../works/ZarandiASC20.pdf}{ZarandiASC20} (79.00)& \cellcolor{red!40}\href{../works/BeckW07.pdf}{BeckW07} (76.00)& \cellcolor{red!40}\href{../works/Groleaz21.pdf}{Groleaz21} (75.00)& \cellcolor{red!40}\href{../works/LaborieRSV18.pdf}{LaborieRSV18} (73.00)\\
Cosine& \cellcolor{red!40}\href{../works/LombardiM09.pdf}{LombardiM09} (0.75)& \cellcolor{red!40}\href{../works/BeckW05.pdf}{BeckW05} (0.74)& \cellcolor{red!40}\href{../works/BeckW07.pdf}{BeckW07} (0.73)& \cellcolor{red!40}\href{../works/BonfiettiM12.pdf}{BonfiettiM12} (0.72)& \cellcolor{red!40}\href{../works/LombardiBM15.pdf}{LombardiBM15} (0.72)\\
\index{Dorndorf2000}\href{../works/Dorndorf2000.pdf}{Dorndorf2000} R\&C& \cellcolor{red!40}DorndorfHP99 (0.74)& \cellcolor{red!40}DomdorfPH03 (0.82)& \cellcolor{red!40}DorndorfPH99 (0.85)& \cellcolor{red!40}\href{../works/SourdN00.pdf}{SourdN00} (0.85)& \cellcolor{red!40}\href{../works/BeckF00.pdf}{BeckF00} (0.86)\\
Euclid& \cellcolor{black!20}\href{../works/BartakSR08.pdf}{BartakSR08} (0.36)& \cellcolor{black!20}\href{../works/PengLC14.pdf}{PengLC14} (0.36)& \cellcolor{black!20}\href{../works/BaptisteP95.pdf}{BaptisteP95} (0.37)& \cellcolor{black!20}\href{../works/ChenGPSH10.pdf}{ChenGPSH10} (0.37)& \href{../works/SourdN00.pdf}{SourdN00} (0.38)\\
Dot& \cellcolor{red!40}\href{../works/Baptiste02.pdf}{Baptiste02} (198.00)& \cellcolor{red!40}\href{../works/Groleaz21.pdf}{Groleaz21} (192.00)& \cellcolor{red!40}\href{../works/Malapert11.pdf}{Malapert11} (181.00)& \cellcolor{red!40}\href{../works/Godet21a.pdf}{Godet21a} (180.00)& \cellcolor{red!40}\href{../works/Fahimi16.pdf}{Fahimi16} (178.00)\\
Cosine& \cellcolor{red!40}\href{../works/BartakSR08.pdf}{BartakSR08} (0.75)& \cellcolor{red!40}\href{../works/ChenGPSH10.pdf}{ChenGPSH10} (0.75)& \cellcolor{red!40}\href{../works/PengLC14.pdf}{PengLC14} (0.75)& \cellcolor{red!40}\href{../works/BaptisteP95.pdf}{BaptisteP95} (0.74)& \cellcolor{red!40}\href{../works/SourdN00.pdf}{SourdN00} (0.73)\\
\index{DorndorfHP99}DorndorfHP99 R\&C& \cellcolor{red!40}DorndorfPH99 (0.70)& \cellcolor{red!40}\href{../works/Dorndorf2000.pdf}{Dorndorf2000} (0.74)& \cellcolor{red!40}\href{../works/SourdN00.pdf}{SourdN00} (0.76)& \cellcolor{red!40}\href{../works/MonetteDD07.pdf}{MonetteDD07} (0.81)& \cellcolor{red!40}\href{../works/DemasseyAM05.pdf}{DemasseyAM05} (0.83)\\
Euclid\\
Dot\\
Cosine\\
\index{DorndorfPH99}DorndorfPH99 R\&C& \cellcolor{red!40}DorndorfHP99 (0.70)& \cellcolor{red!40}\href{../works/SourdN00.pdf}{SourdN00} (0.81)& \cellcolor{red!40}\href{../works/Dorndorf2000.pdf}{Dorndorf2000} (0.85)& \cellcolor{red!20}\href{../works/ArtiguesF07.pdf}{ArtiguesF07} (0.88)& \cellcolor{red!20}\href{../works/JainM99.pdf}{JainM99} (0.88)\\
Euclid\\
Dot\\
Cosine\\
\index{DoulabiRP14}\href{../works/DoulabiRP14.pdf}{DoulabiRP14} R\&C& \cellcolor{red!40}\href{../works/DoulabiRP16.pdf}{DoulabiRP16} (0.74)& \cellcolor{red!40}RoshanaeiLAU17a (0.85)& \cellcolor{red!40}\href{../works/WangMD15.pdf}{WangMD15} (0.86)& \cellcolor{red!20}\href{../works/MeskensDL13.pdf}{MeskensDL13} (0.87)& \cellcolor{red!20}\href{../works/RiiseML16.pdf}{RiiseML16} (0.87)\\
Euclid& \cellcolor{red!20}\href{../works/DoulabiRP16.pdf}{DoulabiRP16} (0.26)& \cellcolor{yellow!20}\href{../works/ChapadosJR11.pdf}{ChapadosJR11} (0.27)& \cellcolor{yellow!20}\href{../works/GhandehariK22.pdf}{GhandehariK22} (0.28)& \cellcolor{green!20}\href{../works/ZibranR11.pdf}{ZibranR11} (0.29)& \cellcolor{green!20}\href{../works/KovacsEKV05.pdf}{KovacsEKV05} (0.30)\\
Dot& \cellcolor{red!40}\href{../works/RiiseML16.pdf}{RiiseML16} (72.00)& \cellcolor{red!40}\href{../works/ZarandiASC20.pdf}{ZarandiASC20} (71.00)& \cellcolor{red!40}\href{../works/RoshanaeiLAU17.pdf}{RoshanaeiLAU17} (70.00)& \cellcolor{red!40}\href{../works/GhandehariK22.pdf}{GhandehariK22} (70.00)& \cellcolor{red!40}\href{../works/RoshanaeiN21.pdf}{RoshanaeiN21} (68.00)\\
Cosine& \cellcolor{red!40}\href{../works/DoulabiRP16.pdf}{DoulabiRP16} (0.75)& \cellcolor{red!40}\href{../works/GhandehariK22.pdf}{GhandehariK22} (0.74)& \cellcolor{red!40}\href{../works/RiiseML16.pdf}{RiiseML16} (0.73)& \cellcolor{red!40}\href{../works/WangMD15.pdf}{WangMD15} (0.65)& \cellcolor{red!40}\href{../works/GurEA19.pdf}{GurEA19} (0.64)\\
\index{DoulabiRP16}\href{../works/DoulabiRP16.pdf}{DoulabiRP16} R\&C& \cellcolor{red!40}\href{../works/RoshanaeiLAU17.pdf}{RoshanaeiLAU17} (0.70)& \cellcolor{red!40}RoshanaeiLAU17a (0.70)& \cellcolor{red!40}\href{../works/DoulabiRP14.pdf}{DoulabiRP14} (0.74)& \cellcolor{red!40}\href{../works/WangMD15.pdf}{WangMD15} (0.74)& \cellcolor{red!40}\href{../works/RoshanaeiBAUB20.pdf}{RoshanaeiBAUB20} (0.77)\\
Euclid& \cellcolor{red!20}\href{../works/GhandehariK22.pdf}{GhandehariK22} (0.25)& \cellcolor{red!20}\href{../works/GurEA19.pdf}{GurEA19} (0.25)& \cellcolor{red!20}\href{../works/DoulabiRP14.pdf}{DoulabiRP14} (0.26)& \cellcolor{yellow!20}\href{../works/GurPAE23.pdf}{GurPAE23} (0.26)& \cellcolor{green!20}\href{../works/ZhaoL14.pdf}{ZhaoL14} (0.30)\\
Dot& \cellcolor{red!40}\href{../works/ZarandiASC20.pdf}{ZarandiASC20} (99.00)& \cellcolor{red!40}\href{../works/RoshanaeiLAU17.pdf}{RoshanaeiLAU17} (99.00)& \cellcolor{red!40}\href{../works/RoshanaeiBAUB20.pdf}{RoshanaeiBAUB20} (92.00)& \cellcolor{red!40}\href{../works/ZhaoL14.pdf}{ZhaoL14} (90.00)& \cellcolor{red!40}\href{../works/GhandehariK22.pdf}{GhandehariK22} (90.00)\\
Cosine& \cellcolor{red!40}\href{../works/GhandehariK22.pdf}{GhandehariK22} (0.81)& \cellcolor{red!40}\href{../works/GurEA19.pdf}{GurEA19} (0.80)& \cellcolor{red!40}\href{../works/GurPAE23.pdf}{GurPAE23} (0.78)& \cellcolor{red!40}\href{../works/RoshanaeiLAU17.pdf}{RoshanaeiLAU17} (0.77)& \cellcolor{red!40}\href{../works/DoulabiRP14.pdf}{DoulabiRP14} (0.75)\\
\index{DraperJCJ99}\href{../works/DraperJCJ99.pdf}{DraperJCJ99} R\&C\\
Euclid& \cellcolor{yellow!20}\href{../works/BonfiettiLBM11.pdf}{BonfiettiLBM11} (0.27)& \cellcolor{yellow!20}\href{../works/CarlssonKA99.pdf}{CarlssonKA99} (0.28)& \cellcolor{green!20}\href{../works/LombardiBMB11.pdf}{LombardiBMB11} (0.29)& \cellcolor{green!20}\href{../works/BeckPS03.pdf}{BeckPS03} (0.29)& \cellcolor{green!20}\href{../works/KengY89.pdf}{KengY89} (0.30)\\
Dot& \cellcolor{red!40}\href{../works/Astrand21.pdf}{Astrand21} (106.00)& \cellcolor{red!40}\href{../works/ZarandiASC20.pdf}{ZarandiASC20} (104.00)& \cellcolor{red!40}\href{../works/Dejemeppe16.pdf}{Dejemeppe16} (101.00)& \cellcolor{red!40}\href{../works/Lombardi10.pdf}{Lombardi10} (100.00)& \cellcolor{red!40}\href{../works/Groleaz21.pdf}{Groleaz21} (99.00)\\
Cosine& \cellcolor{red!40}\href{../works/BeckPS03.pdf}{BeckPS03} (0.75)& \cellcolor{red!40}\href{../works/BonfiettiLBM11.pdf}{BonfiettiLBM11} (0.74)& \cellcolor{red!40}\href{../works/LombardiBMB11.pdf}{LombardiBMB11} (0.72)& \cellcolor{red!40}\href{../works/CarlssonKA99.pdf}{CarlssonKA99} (0.71)& \cellcolor{red!40}\href{../works/WessenCSFPM23.pdf}{WessenCSFPM23} (0.71)\\
\index{EastonNT02}\href{../works/EastonNT02.pdf}{EastonNT02} R\&C& \cellcolor{red!40}\href{../works/RasmussenT06.pdf}{RasmussenT06} (0.80)& \cellcolor{red!40}\href{../works/Trick03.pdf}{Trick03} (0.83)& \cellcolor{red!40}\href{../works/RasmussenT09.pdf}{RasmussenT09} (0.85)& \cellcolor{red!20}\href{../works/HenzMT04.pdf}{HenzMT04} (0.88)& \cellcolor{yellow!20}\href{../works/RussellU06.pdf}{RussellU06} (0.90)\\
Euclid& \cellcolor{red!40}\href{../works/SuCC13.pdf}{SuCC13} (0.16)& \cellcolor{red!40}\href{../works/RasmussenT06.pdf}{RasmussenT06} (0.16)& \cellcolor{red!40}\href{../works/ZengM12.pdf}{ZengM12} (0.18)& \cellcolor{red!40}\href{../works/Trick03.pdf}{Trick03} (0.19)& \cellcolor{red!40}\href{../works/ElfJR03.pdf}{ElfJR03} (0.19)\\
Dot& \cellcolor{red!40}\href{../works/KendallKRU10.pdf}{KendallKRU10} (66.00)& \cellcolor{red!40}\href{../works/Ribeiro12.pdf}{Ribeiro12} (64.00)& \cellcolor{red!40}\href{../works/RasmussenT09.pdf}{RasmussenT09} (61.00)& \cellcolor{red!40}\href{../works/ZengM12.pdf}{ZengM12} (56.00)& \cellcolor{red!40}\href{../works/RasmussenT07.pdf}{RasmussenT07} (56.00)\\
Cosine& \cellcolor{red!40}\href{../works/RasmussenT06.pdf}{RasmussenT06} (0.86)& \cellcolor{red!40}\href{../works/ZengM12.pdf}{ZengM12} (0.85)& \cellcolor{red!40}\href{../works/RasmussenT09.pdf}{RasmussenT09} (0.84)& \cellcolor{red!40}\href{../works/SuCC13.pdf}{SuCC13} (0.84)& \cellcolor{red!40}\href{../works/RasmussenT07.pdf}{RasmussenT07} (0.82)\\
\index{Edis21}\href{../works/Edis21.pdf}{Edis21} R\&C& \cellcolor{red!40}\href{../works/CilKLO22.pdf}{CilKLO22} (0.78)& \cellcolor{red!20}\href{../works/AlakaPY19.pdf}{AlakaPY19} (0.88)& \cellcolor{red!20}KizilayC20 (0.89)& \cellcolor{red!20}\href{../works/PinarbasiAY19.pdf}{PinarbasiAY19} (0.90)& \cellcolor{yellow!20}\href{../works/MengLZB21.pdf}{MengLZB21} (0.92)\\
Euclid& \cellcolor{blue!20}\href{../works/TopalogluSS12.pdf}{TopalogluSS12} (0.32)& \cellcolor{blue!20}\href{../works/PinarbasiAY19.pdf}{PinarbasiAY19} (0.33)& \cellcolor{blue!20}\href{../works/CilKLO22.pdf}{CilKLO22} (0.33)& \cellcolor{black!20}\href{../works/AlakaP23.pdf}{AlakaP23} (0.34)& \cellcolor{black!20}\href{../works/AbidinK20.pdf}{AbidinK20} (0.35)\\
Dot& \cellcolor{red!40}\href{../works/ZarandiASC20.pdf}{ZarandiASC20} (171.00)& \cellcolor{red!40}\href{../works/Dejemeppe16.pdf}{Dejemeppe16} (160.00)& \cellcolor{red!40}\href{../works/Groleaz21.pdf}{Groleaz21} (158.00)& \cellcolor{red!40}\href{../works/Lunardi20.pdf}{Lunardi20} (144.00)& \cellcolor{red!40}\href{../works/Astrand21.pdf}{Astrand21} (138.00)\\
Cosine& \cellcolor{red!40}\href{../works/TopalogluSS12.pdf}{TopalogluSS12} (0.79)& \cellcolor{red!40}\href{../works/CilKLO22.pdf}{CilKLO22} (0.78)& \cellcolor{red!40}\href{../works/PinarbasiAY19.pdf}{PinarbasiAY19} (0.76)& \cellcolor{red!40}\href{../works/AlakaP23.pdf}{AlakaP23} (0.74)& \cellcolor{red!40}\href{../works/AbidinK20.pdf}{AbidinK20} (0.74)\\
\index{EdisO11}\href{../works/EdisO11.pdf}{EdisO11} R\&C& \cellcolor{red!40}EdisO11a (0.78)& \cellcolor{red!40}\href{../works/HamdiL13.pdf}{HamdiL13} (0.82)& \cellcolor{red!40}\href{../works/ArbaouiY18.pdf}{ArbaouiY18} (0.82)& \cellcolor{red!40}\href{../works/ZhangLS12.pdf}{ZhangLS12} (0.83)& \cellcolor{red!40}\href{../works/CireCH13.pdf}{CireCH13} (0.85)\\
Euclid& \cellcolor{yellow!20}\href{../works/Limtanyakul07.pdf}{Limtanyakul07} (0.27)& \cellcolor{yellow!20}\href{../works/BenediktSMVH18.pdf}{BenediktSMVH18} (0.28)& \cellcolor{yellow!20}\href{../works/HookerY02.pdf}{HookerY02} (0.28)& \cellcolor{yellow!20}\href{../works/ChuX05.pdf}{ChuX05} (0.28)& \cellcolor{green!20}\href{../works/ArbaouiY18.pdf}{ArbaouiY18} (0.29)\\
Dot& \cellcolor{red!40}\href{../works/Groleaz21.pdf}{Groleaz21} (114.00)& \cellcolor{red!40}\href{../works/ZarandiASC20.pdf}{ZarandiASC20} (111.00)& \cellcolor{red!40}\href{../works/YunusogluY22.pdf}{YunusogluY22} (106.00)& \cellcolor{red!40}\href{../works/Baptiste02.pdf}{Baptiste02} (106.00)& \cellcolor{red!40}\href{../works/AwadMDMT22.pdf}{AwadMDMT22} (106.00)\\
Cosine& \cellcolor{red!40}\href{../works/ArbaouiY18.pdf}{ArbaouiY18} (0.74)& \cellcolor{red!40}\href{../works/QinDS16.pdf}{QinDS16} (0.73)& \cellcolor{red!40}\href{../works/JainG01.pdf}{JainG01} (0.72)& \cellcolor{red!40}\href{../works/HamdiL13.pdf}{HamdiL13} (0.72)& \cellcolor{red!40}\href{../works/Ham18a.pdf}{Ham18a} (0.71)\\
\index{EdisO11a}EdisO11a R\&C& \cellcolor{red!40}\href{../works/EdisO11.pdf}{EdisO11} (0.78)& \cellcolor{red!20}\href{../works/JainG01.pdf}{JainG01} (0.90)& \cellcolor{red!20}GongLMW09 (0.90)& \cellcolor{yellow!20}\href{../works/Hooker07.pdf}{Hooker07} (0.91)& \cellcolor{yellow!20}\href{../works/CobanH10.pdf}{CobanH10} (0.91)\\
Euclid\\
Dot\\
Cosine\\
\index{EdwardsBSE19}EdwardsBSE19 R\&C& \cellcolor{red!40}CarlierSJP21 (0.79)& \cellcolor{red!20}\href{../works/SchnellH15.pdf}{SchnellH15} (0.87)& \cellcolor{red!20}\href{../works/HillTV21.pdf}{HillTV21} (0.87)& \cellcolor{red!20}\href{../works/KreterSSZ18.pdf}{KreterSSZ18} (0.87)& \cellcolor{red!20}\href{../works/CarlierPSJ20.pdf}{CarlierPSJ20} (0.89)\\
Euclid\\
Dot\\
Cosine\\
\index{EfthymiouY23}\href{../works/EfthymiouY23.pdf}{EfthymiouY23} R\&C& \cellcolor{yellow!20}\href{../works/WallaceY20.pdf}{WallaceY20} (0.90)& \cellcolor{blue!20}\href{../works/BaptisteLV92.pdf}{BaptisteLV92} (0.97)& \cellcolor{blue!20}\href{../works/AntuoriHHEN20.pdf}{AntuoriHHEN20} (0.97)& \cellcolor{blue!20}\href{../works/RodosekW98.pdf}{RodosekW98} (0.97)& \cellcolor{blue!20}\href{../works/BenediktSMVH18.pdf}{BenediktSMVH18} (0.97)\\
Euclid& \cellcolor{blue!20}\href{../works/IklassovMR023.pdf}{IklassovMR023} (0.32)& \cellcolor{blue!20}\href{../works/GalleguillosKSB19.pdf}{GalleguillosKSB19} (0.33)& \cellcolor{black!20}\href{../works/KotaryFH22.pdf}{KotaryFH22} (0.34)& \cellcolor{black!20}\href{../works/Tassel22.pdf}{Tassel22} (0.34)& \cellcolor{black!20}\href{../works/LiLZDZW24.pdf}{LiLZDZW24} (0.35)\\
Dot& \cellcolor{red!40}\href{../works/Groleaz21.pdf}{Groleaz21} (116.00)& \cellcolor{red!40}\href{../works/ZarandiASC20.pdf}{ZarandiASC20} (108.00)& \cellcolor{red!40}\href{../works/Dejemeppe16.pdf}{Dejemeppe16} (108.00)& \cellcolor{red!40}\href{../works/Godet21a.pdf}{Godet21a} (103.00)& \cellcolor{red!40}\href{../works/abs-2211-14492.pdf}{abs-2211-14492} (102.00)\\
Cosine& \cellcolor{red!40}\href{../works/IklassovMR023.pdf}{IklassovMR023} (0.69)& \cellcolor{red!40}\href{../works/KotaryFH22.pdf}{KotaryFH22} (0.67)& \cellcolor{red!40}\href{../works/CarchraeB09.pdf}{CarchraeB09} (0.66)& \cellcolor{red!40}\href{../works/abs-2211-14492.pdf}{abs-2211-14492} (0.66)& \cellcolor{red!40}\href{../works/TasselGS23.pdf}{TasselGS23} (0.65)\\
\index{ElciOH22}\href{../works/ElciOH22.pdf}{ElciOH22} R\&C& \cellcolor{red!40}MartnezAJ22 (0.54)& \cellcolor{red!40}NaderiR22 (0.67)& \cellcolor{red!40}HechingHK19 (0.85)& \cellcolor{red!40}\href{../works/ForbesHJST24.pdf}{ForbesHJST24} (0.85)& \cellcolor{red!20}\href{../works/Hooker05.pdf}{Hooker05} (0.87)\\
Euclid& \cellcolor{yellow!20}\href{../works/CobanH11.pdf}{CobanH11} (0.27)& \cellcolor{yellow!20}\href{../works/CireCH13.pdf}{CireCH13} (0.28)& \cellcolor{green!20}\href{../works/Beck10.pdf}{Beck10} (0.29)& \cellcolor{green!20}\href{../works/CireCH16.pdf}{CireCH16} (0.29)& \cellcolor{green!20}\href{../works/Hooker04.pdf}{Hooker04} (0.30)\\
Dot& \cellcolor{red!40}\href{../works/Hooker19.pdf}{Hooker19} (123.00)& \cellcolor{red!40}\href{../works/Lombardi10.pdf}{Lombardi10} (121.00)& \cellcolor{red!40}\href{../works/Groleaz21.pdf}{Groleaz21} (119.00)& \cellcolor{red!40}\href{../works/NaderiRR23.pdf}{NaderiRR23} (118.00)& \cellcolor{red!40}\href{../works/ZarandiASC20.pdf}{ZarandiASC20} (118.00)\\
Cosine& \cellcolor{red!40}\href{../works/CobanH11.pdf}{CobanH11} (0.80)& \cellcolor{red!40}\href{../works/Hooker19.pdf}{Hooker19} (0.79)& \cellcolor{red!40}\href{../works/CireCH13.pdf}{CireCH13} (0.77)& \cellcolor{red!40}\href{../works/Beck10.pdf}{Beck10} (0.76)& \cellcolor{red!40}\href{../works/ForbesHJST24.pdf}{ForbesHJST24} (0.75)\\
\index{ElfJR03}\href{../works/ElfJR03.pdf}{ElfJR03} R\&C& \cellcolor{red!40}\href{../works/HenzMT04.pdf}{HenzMT04} (0.80)& \cellcolor{red!40}\href{../works/RasmussenT07.pdf}{RasmussenT07} (0.81)& \cellcolor{red!40}Henz01 (0.83)& \cellcolor{red!20}Rgin2001 (0.87)& \cellcolor{red!20}\href{../works/Perron05.pdf}{Perron05} (0.90)\\
Euclid& \cellcolor{red!40}\href{../works/SuCC13.pdf}{SuCC13} (0.19)& \cellcolor{red!40}\href{../works/EastonNT02.pdf}{EastonNT02} (0.19)& \cellcolor{red!40}\href{../works/Perron05.pdf}{Perron05} (0.20)& \cellcolor{red!40}\href{../works/Trick03.pdf}{Trick03} (0.21)& \cellcolor{red!40}\href{../works/RasmussenT06.pdf}{RasmussenT06} (0.21)\\
Dot& \cellcolor{red!40}\href{../works/KendallKRU10.pdf}{KendallKRU10} (59.00)& \cellcolor{red!40}\href{../works/Ribeiro12.pdf}{Ribeiro12} (48.00)& \cellcolor{red!40}\href{../works/ZarandiASC20.pdf}{ZarandiASC20} (47.00)& \cellcolor{red!40}\href{../works/RasmussenT09.pdf}{RasmussenT09} (47.00)& \cellcolor{red!40}\href{../works/Lunardi20.pdf}{Lunardi20} (46.00)\\
Cosine& \cellcolor{red!40}\href{../works/ZengM12.pdf}{ZengM12} (0.76)& \cellcolor{red!40}\href{../works/Trick03.pdf}{Trick03} (0.76)& \cellcolor{red!40}\href{../works/RasmussenT06.pdf}{RasmussenT06} (0.75)& \cellcolor{red!40}\href{../works/SuCC13.pdf}{SuCC13} (0.75)& \cellcolor{red!40}\href{../works/EastonNT02.pdf}{EastonNT02} (0.73)\\
\index{ElhouraniDM07}\href{../works/ElhouraniDM07.pdf}{ElhouraniDM07} R\&C\\
Euclid& \cellcolor{red!20}\href{../works/AbrilSB05.pdf}{AbrilSB05} (0.24)& \cellcolor{red!20}\href{../works/Hunsberger08.pdf}{Hunsberger08} (0.25)& \cellcolor{red!20}\href{../works/MaraveliasG04.pdf}{MaraveliasG04} (0.25)& \cellcolor{red!20}\href{../works/KameugneF13.pdf}{KameugneF13} (0.25)& \cellcolor{red!20}\href{../works/SultanikMR07.pdf}{SultanikMR07} (0.25)\\
Dot& \cellcolor{red!40}\href{../works/SultanikMR07.pdf}{SultanikMR07} (36.00)& \cellcolor{red!40}\href{../works/Godet21a.pdf}{Godet21a} (35.00)& \cellcolor{red!40}\href{../works/Lemos21.pdf}{Lemos21} (34.00)& \cellcolor{red!40}\href{../works/Dejemeppe16.pdf}{Dejemeppe16} (34.00)& \cellcolor{red!40}\href{../works/ZarandiASC20.pdf}{ZarandiASC20} (29.00)\\
Cosine& \cellcolor{red!40}\href{../works/SultanikMR07.pdf}{SultanikMR07} (0.66)& \cellcolor{red!40}\href{../works/SunLYL10.pdf}{SunLYL10} (0.43)& \cellcolor{red!40}\href{../works/BartakS11.pdf}{BartakS11} (0.41)& \cellcolor{red!40}\href{../works/Hunsberger08.pdf}{Hunsberger08} (0.39)& \cellcolor{red!40}\href{../works/Salido10.pdf}{Salido10} (0.38)\\
\index{Elkhyari03}\href{../works/Elkhyari03.pdf}{Elkhyari03} R\&C\\
Euclid& \href{../works/MalapertCGJLR13.pdf}{MalapertCGJLR13} (0.39)& \href{../works/VilimLS15.pdf}{VilimLS15} (0.40)& \href{../works/ElkhyariGJ02a.pdf}{ElkhyariGJ02a} (0.42)& \href{../works/KovacsV04.pdf}{KovacsV04} (0.42)& \href{../works/BartakSR08.pdf}{BartakSR08} (0.42)\\
Dot& \cellcolor{red!40}\href{../works/Groleaz21.pdf}{Groleaz21} (217.00)& \cellcolor{red!40}\href{../works/ZarandiASC20.pdf}{ZarandiASC20} (201.00)& \cellcolor{red!40}\href{../works/Godet21a.pdf}{Godet21a} (198.00)& \cellcolor{red!40}\href{../works/Baptiste02.pdf}{Baptiste02} (198.00)& \cellcolor{red!40}\href{../works/Malapert11.pdf}{Malapert11} (182.00)\\
Cosine& \cellcolor{red!40}\href{../works/MalapertCGJLR13.pdf}{MalapertCGJLR13} (0.73)& \cellcolor{red!40}\href{../works/VilimLS15.pdf}{VilimLS15} (0.72)& \cellcolor{red!40}\href{../works/Demassey03.pdf}{Demassey03} (0.71)& \cellcolor{red!40}\href{../works/BartakSR08.pdf}{BartakSR08} (0.68)& \cellcolor{red!40}\href{../works/ElkhyariGJ02a.pdf}{ElkhyariGJ02a} (0.68)\\
\index{ElkhyariGJ02}\href{../works/ElkhyariGJ02.pdf}{ElkhyariGJ02} R\&C& \cellcolor{red!40}\href{../works/ElkhyariGJ02a.pdf}{ElkhyariGJ02a} (0.77)& \cellcolor{red!40}\href{../works/Wolf05.pdf}{Wolf05} (0.80)& \cellcolor{red!40}\href{../works/BruckerK00.pdf}{BruckerK00} (0.86)& \cellcolor{red!20}\href{../works/BertholdHLMS10.pdf}{BertholdHLMS10} (0.88)& \cellcolor{red!20}\href{../works/SchuttFSW11.pdf}{SchuttFSW11} (0.90)\\
Euclid& \cellcolor{red!40}\href{../works/WallaceF00.pdf}{WallaceF00} (0.22)& \cellcolor{red!40}\href{../works/LombardiM13.pdf}{LombardiM13} (0.23)& \cellcolor{red!40}\href{../works/BonfiettiM12.pdf}{BonfiettiM12} (0.23)& \cellcolor{red!40}\href{../works/BeniniBGM05a.pdf}{BeniniBGM05a} (0.23)& \cellcolor{red!40}\href{../works/BhatnagarKL19.pdf}{BhatnagarKL19} (0.23)\\
Dot& \cellcolor{red!40}\href{../works/Lombardi10.pdf}{Lombardi10} (91.00)& \cellcolor{red!40}\href{../works/Godet21a.pdf}{Godet21a} (86.00)& \cellcolor{red!40}\href{../works/Schutt11.pdf}{Schutt11} (85.00)& \cellcolor{red!40}\href{../works/Baptiste02.pdf}{Baptiste02} (84.00)& \cellcolor{red!40}\href{../works/ZarandiASC20.pdf}{ZarandiASC20} (83.00)\\
Cosine& \cellcolor{red!40}\href{../works/ElkhyariGJ02a.pdf}{ElkhyariGJ02a} (0.77)& \cellcolor{red!40}\href{../works/LiessM08.pdf}{LiessM08} (0.75)& \cellcolor{red!40}\href{../works/BhatnagarKL19.pdf}{BhatnagarKL19} (0.73)& \cellcolor{red!40}\href{../works/WallaceF00.pdf}{WallaceF00} (0.72)& \cellcolor{red!40}\href{../works/BeniniLMMR08.pdf}{BeniniLMMR08} (0.72)\\
\index{ElkhyariGJ02a}\href{../works/ElkhyariGJ02a.pdf}{ElkhyariGJ02a} R\&C& \cellcolor{red!40}\href{../works/ElkhyariGJ02.pdf}{ElkhyariGJ02} (0.77)& \cellcolor{red!40}\href{../works/BruckerK00.pdf}{BruckerK00} (0.84)& \cellcolor{red!20}\href{../works/PoderBS04.pdf}{PoderBS04} (0.87)& \cellcolor{red!20}\href{../works/DemasseyAM05.pdf}{DemasseyAM05} (0.87)& \cellcolor{red!20}\href{../works/LiessM08.pdf}{LiessM08} (0.88)\\
Euclid& \cellcolor{yellow!20}\href{../works/ElkhyariGJ02.pdf}{ElkhyariGJ02} (0.28)& \cellcolor{green!20}\href{../works/OddiRC10.pdf}{OddiRC10} (0.30)& \cellcolor{green!20}\href{../works/LombardiM10.pdf}{LombardiM10} (0.30)& \cellcolor{green!20}\href{../works/BofillCSV17a.pdf}{BofillCSV17a} (0.31)& \cellcolor{green!20}\href{../works/LombardiM13.pdf}{LombardiM13} (0.31)\\
Dot& \cellcolor{red!40}\href{../works/Godet21a.pdf}{Godet21a} (136.00)& \cellcolor{red!40}\href{../works/Lombardi10.pdf}{Lombardi10} (132.00)& \cellcolor{red!40}\href{../works/Baptiste02.pdf}{Baptiste02} (131.00)& \cellcolor{red!40}\href{../works/ZarandiASC20.pdf}{ZarandiASC20} (127.00)& \cellcolor{red!40}\href{../works/Schutt11.pdf}{Schutt11} (126.00)\\
Cosine& \cellcolor{red!40}\href{../works/ElkhyariGJ02.pdf}{ElkhyariGJ02} (0.77)& \cellcolor{red!40}\href{../works/LombardiM10.pdf}{LombardiM10} (0.73)& \cellcolor{red!40}\href{../works/OddiRC10.pdf}{OddiRC10} (0.73)& \cellcolor{red!40}\href{../works/LiessM08.pdf}{LiessM08} (0.72)& \cellcolor{red!40}\href{../works/BofillCSV17a.pdf}{BofillCSV17a} (0.72)\\
\index{EmdeZD22}\href{../works/EmdeZD22.pdf}{EmdeZD22} R\&C& \cellcolor{yellow!20}HechingHK19 (0.91)& \cellcolor{yellow!20}\href{../works/TranAB16.pdf}{TranAB16} (0.93)& \cellcolor{yellow!20}\href{../works/ElciOH22.pdf}{ElciOH22} (0.93)& \cellcolor{yellow!20}\href{../works/CobanH11.pdf}{CobanH11} (0.93)& \cellcolor{green!20}\href{../works/CireCH13.pdf}{CireCH13} (0.93)\\
Euclid& \cellcolor{black!20}\href{../works/Beck10.pdf}{Beck10} (0.34)& \cellcolor{black!20}\href{../works/CireCH13.pdf}{CireCH13} (0.36)& \cellcolor{black!20}\href{../works/HamdiL13.pdf}{HamdiL13} (0.36)& \cellcolor{black!20}\href{../works/CobanH11.pdf}{CobanH11} (0.37)& \cellcolor{black!20}\href{../works/HookerO03.pdf}{HookerO03} (0.37)\\
Dot& \cellcolor{red!40}\href{../works/ZarandiASC20.pdf}{ZarandiASC20} (151.00)& \cellcolor{red!40}\href{../works/Groleaz21.pdf}{Groleaz21} (137.00)& \cellcolor{red!40}\href{../works/Lombardi10.pdf}{Lombardi10} (134.00)& \cellcolor{red!40}\href{../works/NaderiRR23.pdf}{NaderiRR23} (131.00)& \cellcolor{red!40}\href{../works/PrataAN23.pdf}{PrataAN23} (128.00)\\
Cosine& \cellcolor{red!40}\href{../works/Beck10.pdf}{Beck10} (0.71)& \cellcolor{red!40}\href{../works/NaderiBZ22.pdf}{NaderiBZ22} (0.69)& \cellcolor{red!40}\href{../works/NaderiBZ23.pdf}{NaderiBZ23} (0.69)& \cellcolor{red!40}\href{../works/CobanH11.pdf}{CobanH11} (0.69)& \cellcolor{red!40}\href{../works/HamdiL13.pdf}{HamdiL13} (0.69)\\
\index{EmeretlisTAV17}\href{../works/EmeretlisTAV17.pdf}{EmeretlisTAV17} R\&C& \cellcolor{red!40}\href{../works/TanT18.pdf}{TanT18} (0.77)& \cellcolor{red!40}\href{../works/CobanH11.pdf}{CobanH11} (0.86)& \cellcolor{red!20}\href{../works/CireCH16.pdf}{CireCH16} (0.87)& \cellcolor{yellow!20}\href{../works/BeniniLMR11.pdf}{BeniniLMR11} (0.92)& \cellcolor{yellow!20}\href{../works/Beck10.pdf}{Beck10} (0.92)\\
Euclid& \cellcolor{green!20}\href{../works/CobanH11.pdf}{CobanH11} (0.29)& \cellcolor{green!20}\href{../works/BeniniLMMR08.pdf}{BeniniLMMR08} (0.29)& \cellcolor{green!20}\href{../works/BeniniLMR11.pdf}{BeniniLMR11} (0.29)& \cellcolor{green!20}\href{../works/Hooker05.pdf}{Hooker05} (0.29)& \cellcolor{green!20}\href{../works/CireCH13.pdf}{CireCH13} (0.29)\\
Dot& \cellcolor{red!40}\href{../works/Lombardi10.pdf}{Lombardi10} (152.00)& \cellcolor{red!40}\href{../works/Groleaz21.pdf}{Groleaz21} (137.00)& \cellcolor{red!40}\href{../works/ZarandiASC20.pdf}{ZarandiASC20} (134.00)& \cellcolor{red!40}\href{../works/Astrand21.pdf}{Astrand21} (127.00)& \cellcolor{red!40}\href{../works/Dejemeppe16.pdf}{Dejemeppe16} (127.00)\\
Cosine& \cellcolor{red!40}\href{../works/Hooker05.pdf}{Hooker05} (0.79)& \cellcolor{red!40}\href{../works/CobanH11.pdf}{CobanH11} (0.78)& \cellcolor{red!40}\href{../works/BeniniLMR11.pdf}{BeniniLMR11} (0.78)& \cellcolor{red!40}\href{../works/NaderiBZ22a.pdf}{NaderiBZ22a} (0.77)& \cellcolor{red!40}\href{../works/BeniniBGM05.pdf}{BeniniBGM05} (0.76)\\
\index{EreminW01}\href{../works/EreminW01.pdf}{EreminW01} R\&C& \cellcolor{red!40}\href{../works/Thorsteinsson01.pdf}{Thorsteinsson01} (0.79)& \cellcolor{red!40}\href{../works/BenoistGR02.pdf}{BenoistGR02} (0.79)& \cellcolor{red!40}\href{../works/Hooker04.pdf}{Hooker04} (0.83)& \cellcolor{red!20}\href{../works/ChuX05.pdf}{ChuX05} (0.88)& \cellcolor{red!20}\href{../works/CambazardHDJT04.pdf}{CambazardHDJT04} (0.88)\\
Euclid& \cellcolor{green!20}\href{../works/CireCH16.pdf}{CireCH16} (0.29)& \cellcolor{green!20}\href{../works/LozanoCDS12.pdf}{LozanoCDS12} (0.30)& \cellcolor{green!20}\href{../works/BourdaisGP03.pdf}{BourdaisGP03} (0.30)& \cellcolor{green!20}\href{../works/NishikawaSTT18.pdf}{NishikawaSTT18} (0.30)& \cellcolor{green!20}\href{../works/Wallace06.pdf}{Wallace06} (0.30)\\
Dot& \cellcolor{red!40}\href{../works/MilanoW09.pdf}{MilanoW09} (98.00)& \cellcolor{red!40}\href{../works/MilanoW06.pdf}{MilanoW06} (97.00)& \cellcolor{red!40}\href{../works/Froger16.pdf}{Froger16} (93.00)& \cellcolor{red!40}\href{../works/HookerH17.pdf}{HookerH17} (90.00)& \cellcolor{red!40}\href{../works/Wallace06.pdf}{Wallace06} (88.00)\\
Cosine& \cellcolor{red!40}\href{../works/Wallace06.pdf}{Wallace06} (0.74)& \cellcolor{red!40}\href{../works/CireCH16.pdf}{CireCH16} (0.69)& \cellcolor{red!40}\href{../works/BeniniBGM06.pdf}{BeniniBGM06} (0.68)& \cellcolor{red!40}\href{../works/NishikawaSTT18.pdf}{NishikawaSTT18} (0.68)& \cellcolor{red!40}\href{../works/CorreaLR07.pdf}{CorreaLR07} (0.67)\\
\index{ErkingerM17}\href{../works/ErkingerM17.pdf}{ErkingerM17} R\&C& \cellcolor{red!40}\href{../works/MusliuSS18.pdf}{MusliuSS18} (0.82)& \cellcolor{black!20}\href{../works/ZarandiASC20.pdf}{ZarandiASC20} (0.98)& \cellcolor{black!20}\href{../works/DoulabiRP16.pdf}{DoulabiRP16} (0.98)\\
Euclid& \cellcolor{red!20}\href{../works/BandaSC11.pdf}{BandaSC11} (0.24)& \cellcolor{red!20}\href{../works/ZhangLS12.pdf}{ZhangLS12} (0.25)& \cellcolor{red!20}\href{../works/ZibranR11.pdf}{ZibranR11} (0.26)& \cellcolor{red!20}\href{../works/BourdaisGP03.pdf}{BourdaisGP03} (0.26)& \cellcolor{red!20}\href{../works/GelainPRVW17.pdf}{GelainPRVW17} (0.26)\\
Dot& \cellcolor{red!40}\href{../works/Godet21a.pdf}{Godet21a} (68.00)& \cellcolor{red!40}\href{../works/Baptiste02.pdf}{Baptiste02} (66.00)& \cellcolor{red!40}\href{../works/Caballero19.pdf}{Caballero19} (65.00)& \cellcolor{red!40}\href{../works/German18.pdf}{German18} (64.00)& \cellcolor{red!40}\href{../works/Wallace06.pdf}{Wallace06} (63.00)\\
Cosine& \cellcolor{red!40}\href{../works/MusliuSS18.pdf}{MusliuSS18} (0.68)& \cellcolor{red!40}\href{../works/LiuLH19.pdf}{LiuLH19} (0.68)& \cellcolor{red!40}\href{../works/BourdaisGP03.pdf}{BourdaisGP03} (0.67)& \cellcolor{red!40}\href{../works/ZhangLS12.pdf}{ZhangLS12} (0.66)& \cellcolor{red!40}\href{../works/BandaSC11.pdf}{BandaSC11} (0.66)\\
\index{ErtlK91}\href{../works/ErtlK91.pdf}{ErtlK91} R\&C& \cellcolor{red!20}\href{../works/MalikMB08.pdf}{MalikMB08} (0.89)& \cellcolor{green!20}\href{../works/Goltz95.pdf}{Goltz95} (0.95)& \cellcolor{green!20}\href{../works/AggounB93.pdf}{AggounB93} (0.96)& \cellcolor{green!20}\href{../works/TrentesauxPT01.pdf}{TrentesauxPT01} (0.96)& \cellcolor{green!20}\href{../works/RodosekW98.pdf}{RodosekW98} (0.96)\\
Euclid& \cellcolor{red!40}\href{../works/MalikMB08.pdf}{MalikMB08} (0.22)& \cellcolor{red!40}\href{../works/LiuJ06.pdf}{LiuJ06} (0.22)& \cellcolor{red!40}\href{../works/RoweJCA96.pdf}{RoweJCA96} (0.23)& \cellcolor{red!40}\href{../works/Davis87.pdf}{Davis87} (0.23)& \cellcolor{red!40}\href{../works/FukunagaHFAMN02.pdf}{FukunagaHFAMN02} (0.24)\\
Dot& \cellcolor{red!40}\href{../works/Lombardi10.pdf}{Lombardi10} (62.00)& \cellcolor{red!40}\href{../works/BeldiceanuC94.pdf}{BeldiceanuC94} (61.00)& \cellcolor{red!40}\href{../works/HarjunkoskiMBC14.pdf}{HarjunkoskiMBC14} (58.00)& \cellcolor{red!40}\href{../works/Malik08.pdf}{Malik08} (57.00)& \cellcolor{red!40}\href{../works/Schutt11.pdf}{Schutt11} (57.00)\\
Cosine& \cellcolor{red!40}\href{../works/MalikMB08.pdf}{MalikMB08} (0.75)& \cellcolor{red!40}\href{../works/BegB13.pdf}{BegB13} (0.69)& \cellcolor{red!40}\href{../works/BessiereHMQW14.pdf}{BessiereHMQW14} (0.68)& \cellcolor{red!40}\href{../works/Malik08.pdf}{Malik08} (0.68)& \cellcolor{red!40}\href{../works/RoweJCA96.pdf}{RoweJCA96} (0.67)\\
\index{EscobetPQPRA19}\href{../works/EscobetPQPRA19.pdf}{EscobetPQPRA19} R\&C& \cellcolor{red!20}\href{../works/KlankeBYE21.pdf}{KlankeBYE21} (0.89)& \cellcolor{yellow!20}\href{../works/AwadMDMT22.pdf}{AwadMDMT22} (0.92)& \cellcolor{yellow!20}\href{../works/NovaraNH16.pdf}{NovaraNH16} (0.93)& \cellcolor{green!20}\href{../works/Novas19.pdf}{Novas19} (0.94)& \cellcolor{green!20}\href{../works/OujanaAYB22.pdf}{OujanaAYB22} (0.95)\\
Euclid& \cellcolor{green!20}\href{../works/Colombani96.pdf}{Colombani96} (0.30)& \cellcolor{blue!20}\href{../works/HarjunkoskiG02.pdf}{HarjunkoskiG02} (0.32)& \cellcolor{blue!20}\href{../works/BockmayrP06.pdf}{BockmayrP06} (0.32)& \cellcolor{blue!20}\href{../works/Limtanyakul07.pdf}{Limtanyakul07} (0.32)& \cellcolor{blue!20}\href{../works/QuirogaZH05.pdf}{QuirogaZH05} (0.33)\\
Dot& \cellcolor{red!40}\href{../works/HarjunkoskiMBC14.pdf}{HarjunkoskiMBC14} (124.00)& \cellcolor{red!40}\href{../works/Groleaz21.pdf}{Groleaz21} (123.00)& \cellcolor{red!40}\href{../works/ZarandiASC20.pdf}{ZarandiASC20} (122.00)& \cellcolor{red!40}\href{../works/Dejemeppe16.pdf}{Dejemeppe16} (117.00)& \cellcolor{red!40}\href{../works/LaborieRSV18.pdf}{LaborieRSV18} (110.00)\\
Cosine& \cellcolor{red!40}\href{../works/Colombani96.pdf}{Colombani96} (0.74)& \cellcolor{red!40}\href{../works/HarjunkoskiG02.pdf}{HarjunkoskiG02} (0.72)& \cellcolor{red!40}\href{../works/JainG01.pdf}{JainG01} (0.72)& \cellcolor{red!40}\href{../works/KhayatLR06.pdf}{KhayatLR06} (0.70)& \cellcolor{red!40}\href{../works/QuirogaZH05.pdf}{QuirogaZH05} (0.70)\\
\index{EskeyZ90}\href{../works/EskeyZ90.pdf}{EskeyZ90} R\&C\\
Euclid& \cellcolor{red!20}\href{../works/WallaceF00.pdf}{WallaceF00} (0.26)& \cellcolor{yellow!20}\href{../works/LuoVLBM16.pdf}{LuoVLBM16} (0.26)& \cellcolor{yellow!20}\href{../works/FukunagaHFAMN02.pdf}{FukunagaHFAMN02} (0.26)& \cellcolor{yellow!20}\href{../works/HoYCLLCLC18.pdf}{HoYCLLCLC18} (0.27)& \cellcolor{yellow!20}\href{../works/ElkhyariGJ02.pdf}{ElkhyariGJ02} (0.27)\\
Dot& \cellcolor{red!40}\href{../works/Lombardi10.pdf}{Lombardi10} (86.00)& \cellcolor{red!40}\href{../works/Beck99.pdf}{Beck99} (86.00)& \cellcolor{red!40}\href{../works/ZarandiASC20.pdf}{ZarandiASC20} (81.00)& \cellcolor{red!40}\href{../works/Schutt11.pdf}{Schutt11} (75.00)& \cellcolor{red!40}\href{../works/Dejemeppe16.pdf}{Dejemeppe16} (75.00)\\
Cosine& \cellcolor{red!40}\href{../works/HurleyOS16.pdf}{HurleyOS16} (0.68)& \cellcolor{red!40}\href{../works/ElkhyariGJ02.pdf}{ElkhyariGJ02} (0.66)& \cellcolor{red!40}\href{../works/BeckPS03.pdf}{BeckPS03} (0.66)& \cellcolor{red!40}\href{../works/MorgadoM97.pdf}{MorgadoM97} (0.66)& \cellcolor{red!40}\href{../works/MurphyMB15.pdf}{MurphyMB15} (0.66)\\
\index{EsquirolLH2008}EsquirolLH2008 R\&C& \cellcolor{red!20}\href{../works/BeckF00.pdf}{BeckF00} (0.87)& \cellcolor{red!20}BriandHHL08 (0.88)& \cellcolor{red!20}LiuGT10 (0.88)& \cellcolor{red!20}\href{../works/ArtiouchineB05.pdf}{ArtiouchineB05} (0.90)& \cellcolor{yellow!20}\href{../works/KeriK07.pdf}{KeriK07} (0.90)\\
Euclid\\
Dot\\
Cosine\\
\index{EtminaniesfahaniGNMS22}\href{../works/EtminaniesfahaniGNMS22.pdf}{EtminaniesfahaniGNMS22} R\&C& \cellcolor{green!20}GuSSWC14 (0.93)& \cellcolor{green!20}\href{../works/SchuttFSW11.pdf}{SchuttFSW11} (0.94)& \cellcolor{green!20}\href{../works/AmadiniGM16.pdf}{AmadiniGM16} (0.95)& \cellcolor{green!20}EdwardsBSE19 (0.95)& \cellcolor{green!20}\href{../works/GuSS13.pdf}{GuSS13} (0.95)\\
Euclid& \cellcolor{black!20}\href{../works/VilimLS15.pdf}{VilimLS15} (0.34)& \cellcolor{black!20}\href{../works/ZhangYW21.pdf}{ZhangYW21} (0.35)& \cellcolor{black!20}\href{../works/LahimerLH11.pdf}{LahimerLH11} (0.35)& \cellcolor{black!20}\href{../works/LiessM08.pdf}{LiessM08} (0.35)& \cellcolor{black!20}\href{../works/HillTV21.pdf}{HillTV21} (0.36)\\
Dot& \cellcolor{red!40}\href{../works/ZarandiASC20.pdf}{ZarandiASC20} (177.00)& \cellcolor{red!40}\href{../works/Groleaz21.pdf}{Groleaz21} (177.00)& \cellcolor{red!40}\href{../works/Lunardi20.pdf}{Lunardi20} (157.00)& \cellcolor{red!40}\href{../works/Dejemeppe16.pdf}{Dejemeppe16} (148.00)& \cellcolor{red!40}\href{../works/Lombardi10.pdf}{Lombardi10} (145.00)\\
Cosine& \cellcolor{red!40}\href{../works/VilimLS15.pdf}{VilimLS15} (0.77)& \cellcolor{red!40}\href{../works/YuraszeckMCCR23.pdf}{YuraszeckMCCR23} (0.75)& \cellcolor{red!40}\href{../works/ZhangYW21.pdf}{ZhangYW21} (0.74)& \cellcolor{red!40}\href{../works/LiessM08.pdf}{LiessM08} (0.73)& \cellcolor{red!40}\href{../works/LahimerLH11.pdf}{LahimerLH11} (0.73)\\
\index{EvenSH15}\href{../works/EvenSH15.pdf}{EvenSH15} R\&C& \cellcolor{red!40}\href{../works/ZhangLS12.pdf}{ZhangLS12} (0.75)& \cellcolor{red!40}\href{../works/QuirogaZH05.pdf}{QuirogaZH05} (0.80)& \cellcolor{red!40}\href{../works/Geske05.pdf}{Geske05} (0.80)& \cellcolor{red!40}\href{../works/KovacsV04.pdf}{KovacsV04} (0.83)& \cellcolor{red!40}\href{../works/LimtanyakulS12.pdf}{LimtanyakulS12} (0.86)\\
Euclid& \cellcolor{red!40}\href{../works/EvenSH15a.pdf}{EvenSH15a} (0.09)& \cellcolor{yellow!20}\href{../works/WolfS05.pdf}{WolfS05} (0.27)& \cellcolor{green!20}\href{../works/BeldiceanuP07.pdf}{BeldiceanuP07} (0.29)& \cellcolor{green!20}\href{../works/BockmayrP06.pdf}{BockmayrP06} (0.31)& \cellcolor{green!20}\href{../works/PoderB08.pdf}{PoderB08} (0.31)\\
Dot& \cellcolor{red!40}\href{../works/EvenSH15a.pdf}{EvenSH15a} (120.00)& \cellcolor{red!40}\href{../works/Lombardi10.pdf}{Lombardi10} (114.00)& \cellcolor{red!40}\href{../works/ZarandiASC20.pdf}{ZarandiASC20} (110.00)& \cellcolor{red!40}\href{../works/Fahimi16.pdf}{Fahimi16} (108.00)& \cellcolor{red!40}\href{../works/Godet21a.pdf}{Godet21a} (107.00)\\
Cosine& \cellcolor{red!40}\href{../works/EvenSH15a.pdf}{EvenSH15a} (0.98)& \cellcolor{red!40}\href{../works/WolfS05.pdf}{WolfS05} (0.78)& \cellcolor{red!40}\href{../works/BeldiceanuP07.pdf}{BeldiceanuP07} (0.73)& \cellcolor{red!40}\href{../works/YangSS19.pdf}{YangSS19} (0.68)& \cellcolor{red!40}\href{../works/ZampelliVSDR13.pdf}{ZampelliVSDR13} (0.68)\\
\index{EvenSH15a}\href{../works/EvenSH15a.pdf}{EvenSH15a} R\&C\\
Euclid& \cellcolor{red!40}\href{../works/EvenSH15.pdf}{EvenSH15} (0.09)& \cellcolor{yellow!20}\href{../works/WolfS05.pdf}{WolfS05} (0.28)& \cellcolor{green!20}\href{../works/BeldiceanuP07.pdf}{BeldiceanuP07} (0.29)& \cellcolor{green!20}\href{../works/PoderB08.pdf}{PoderB08} (0.31)& \cellcolor{blue!20}\href{../works/MurphyMB15.pdf}{MurphyMB15} (0.32)\\
Dot& \cellcolor{red!40}\href{../works/EvenSH15.pdf}{EvenSH15} (120.00)& \cellcolor{red!40}\href{../works/Lombardi10.pdf}{Lombardi10} (113.00)& \cellcolor{red!40}\href{../works/ZarandiASC20.pdf}{ZarandiASC20} (110.00)& \cellcolor{red!40}\href{../works/Dejemeppe16.pdf}{Dejemeppe16} (109.00)& \cellcolor{red!40}\href{../works/Malapert11.pdf}{Malapert11} (108.00)\\
Cosine& \cellcolor{red!40}\href{../works/EvenSH15.pdf}{EvenSH15} (0.98)& \cellcolor{red!40}\href{../works/WolfS05.pdf}{WolfS05} (0.75)& \cellcolor{red!40}\href{../works/BeldiceanuP07.pdf}{BeldiceanuP07} (0.72)& \cellcolor{red!40}\href{../works/YangSS19.pdf}{YangSS19} (0.67)& \cellcolor{red!40}\href{../works/PoderB08.pdf}{PoderB08} (0.67)\\
\index{FachiniA20}\href{../works/FachiniA20.pdf}{FachiniA20} R\&C& \cellcolor{red!40}MartnezAJ22 (0.84)& \cellcolor{red!20}\href{../works/ElciOH22.pdf}{ElciOH22} (0.88)& \cellcolor{red!20}ZarandiB12 (0.90)& \cellcolor{yellow!20}HechingHK19 (0.92)& \cellcolor{yellow!20}NaderiR22 (0.93)\\
Euclid& \cellcolor{black!20}\href{../works/MontemanniD23.pdf}{MontemanniD23} (0.36)& \cellcolor{black!20}\href{../works/FallahiAC20.pdf}{FallahiAC20} (0.37)& \href{../works/BarzegaranZP20.pdf}{BarzegaranZP20} (0.38)& \href{../works/MontemanniD23a.pdf}{MontemanniD23a} (0.39)& \href{../works/EreminW01.pdf}{EreminW01} (0.39)\\
Dot& \cellcolor{red!40}\href{../works/Froger16.pdf}{Froger16} (122.00)& \cellcolor{red!40}\href{../works/Groleaz21.pdf}{Groleaz21} (102.00)& \cellcolor{red!40}\href{../works/Lunardi20.pdf}{Lunardi20} (96.00)& \cellcolor{red!40}\href{../works/Lemos21.pdf}{Lemos21} (96.00)& \cellcolor{red!40}\href{../works/HarjunkoskiMBC14.pdf}{HarjunkoskiMBC14} (96.00)\\
Cosine& \cellcolor{red!40}\href{../works/MontemanniD23.pdf}{MontemanniD23} (0.63)& \cellcolor{red!40}\href{../works/FallahiAC20.pdf}{FallahiAC20} (0.62)& \cellcolor{red!40}\href{../works/GoldwaserS18.pdf}{GoldwaserS18} (0.61)& \cellcolor{red!40}\href{../works/NaderiBZR23.pdf}{NaderiBZR23} (0.59)& \cellcolor{red!40}\href{../works/EreminW01.pdf}{EreminW01} (0.59)\\
\index{Fahimi16}\href{../works/Fahimi16.pdf}{Fahimi16} R\&C\\
Euclid& \href{../works/FahimiOQ18.pdf}{FahimiOQ18} (0.40)& \href{../works/GokgurHO18.pdf}{GokgurHO18} (0.47)& \href{../works/BartakSR08.pdf}{BartakSR08} (0.48)& \href{../works/BartakSR10.pdf}{BartakSR10} (0.48)& \href{../works/BeckF00.pdf}{BeckF00} (0.49)\\
Dot& \cellcolor{red!40}\href{../works/Malapert11.pdf}{Malapert11} (280.00)& \cellcolor{red!40}\href{../works/Baptiste02.pdf}{Baptiste02} (279.00)& \cellcolor{red!40}\href{../works/Dejemeppe16.pdf}{Dejemeppe16} (274.00)& \cellcolor{red!40}\href{../works/Groleaz21.pdf}{Groleaz21} (255.00)& \cellcolor{red!40}\href{../works/Lombardi10.pdf}{Lombardi10} (253.00)\\
Cosine& \cellcolor{red!40}\href{../works/FahimiOQ18.pdf}{FahimiOQ18} (0.82)& \cellcolor{red!40}\href{../works/GokgurHO18.pdf}{GokgurHO18} (0.74)& \cellcolor{red!40}\href{../works/BartakSR10.pdf}{BartakSR10} (0.74)& \cellcolor{red!40}\href{../works/BartakSR08.pdf}{BartakSR08} (0.74)& \cellcolor{red!40}\href{../works/Malapert11.pdf}{Malapert11} (0.73)\\
\index{FahimiOQ18}\href{../works/FahimiOQ18.pdf}{FahimiOQ18} R\&C& \cellcolor{red!40}\href{../works/KameugneFSN14.pdf}{KameugneFSN14} (0.65)& \cellcolor{red!40}\href{../works/Tesch16.pdf}{Tesch16} (0.72)& \cellcolor{red!40}\href{../works/DerrienPZ14.pdf}{DerrienPZ14} (0.73)& \cellcolor{red!40}\href{../works/OuelletQ18.pdf}{OuelletQ18} (0.74)& \cellcolor{red!40}\href{../works/OuelletQ13.pdf}{OuelletQ13} (0.76)\\
Euclid& \cellcolor{black!20}\href{../works/VilimBC05.pdf}{VilimBC05} (0.36)& \cellcolor{black!20}\href{../works/MonetteDD07.pdf}{MonetteDD07} (0.36)& \cellcolor{black!20}\href{../works/OuelletQ13.pdf}{OuelletQ13} (0.36)& \cellcolor{black!20}\href{../works/Wolf03.pdf}{Wolf03} (0.36)& \href{../works/VilimBC04.pdf}{VilimBC04} (0.38)\\
Dot& \cellcolor{red!40}\href{../works/Fahimi16.pdf}{Fahimi16} (226.00)& \cellcolor{red!40}\href{../works/Dejemeppe16.pdf}{Dejemeppe16} (213.00)& \cellcolor{red!40}\href{../works/Malapert11.pdf}{Malapert11} (211.00)& \cellcolor{red!40}\href{../works/Baptiste02.pdf}{Baptiste02} (211.00)& \cellcolor{red!40}\href{../works/Schutt11.pdf}{Schutt11} (209.00)\\
Cosine& \cellcolor{red!40}\href{../works/Fahimi16.pdf}{Fahimi16} (0.82)& \cellcolor{red!40}\href{../works/VilimBC05.pdf}{VilimBC05} (0.78)& \cellcolor{red!40}\href{../works/MonetteDD07.pdf}{MonetteDD07} (0.78)& \cellcolor{red!40}\href{../works/OuelletQ13.pdf}{OuelletQ13} (0.78)& \cellcolor{red!40}\href{../works/Wolf03.pdf}{Wolf03} (0.78)\\
\index{FahimiQ23}FahimiQ23 R\&C& \cellcolor{red!40}\href{../works/KameugneFGOQ18.pdf}{KameugneFGOQ18} (0.86)& \cellcolor{red!20}\href{../works/Tesch16.pdf}{Tesch16} (0.87)& \cellcolor{red!20}\href{../works/Tesch18.pdf}{Tesch18} (0.89)& \cellcolor{red!20}\href{../works/GayHS15a.pdf}{GayHS15a} (0.89)& \cellcolor{red!20}\href{../works/OuelletQ13.pdf}{OuelletQ13} (0.90)\\
Euclid\\
Dot\\
Cosine\\
\index{FalaschiGMP97}\href{../works/FalaschiGMP97.pdf}{FalaschiGMP97} R\&C& \cellcolor{green!20}\href{../works/BaptisteLV92.pdf}{BaptisteLV92} (0.93)& \cellcolor{green!20}\href{../works/Goltz95.pdf}{Goltz95} (0.94)& \cellcolor{green!20}\href{../works/BrusoniCLMMT96.pdf}{BrusoniCLMMT96} (0.94)& \cellcolor{green!20}\href{../works/Simonis95a.pdf}{Simonis95a} (0.95)& \cellcolor{green!20}\href{../works/BeniniLMMR08.pdf}{BeniniLMMR08} (0.95)\\
Euclid& \cellcolor{red!40}\href{../works/Touraivane95.pdf}{Touraivane95} (0.13)& \cellcolor{red!40}\href{../works/JelinekB16.pdf}{JelinekB16} (0.22)& \cellcolor{red!40}\href{../works/CarchraeBF05.pdf}{CarchraeBF05} (0.23)& \cellcolor{red!40}\href{../works/FrostD98.pdf}{FrostD98} (0.23)& \cellcolor{red!40}\href{../works/FeldmanG89.pdf}{FeldmanG89} (0.23)\\
Dot& \cellcolor{red!40}\href{../works/Wallace96.pdf}{Wallace96} (48.00)& \cellcolor{red!40}\href{../works/Simonis99.pdf}{Simonis99} (48.00)& \cellcolor{red!40}\href{../works/TrentesauxPT01.pdf}{TrentesauxPT01} (46.00)& \cellcolor{red!40}\href{../works/Simonis95a.pdf}{Simonis95a} (44.00)& \cellcolor{red!40}\href{../works/MartinPY01.pdf}{MartinPY01} (43.00)\\
Cosine& \cellcolor{red!40}\href{../works/Touraivane95.pdf}{Touraivane95} (0.87)& \cellcolor{red!40}\href{../works/AbdennadherS99.pdf}{AbdennadherS99} (0.68)& \cellcolor{red!40}\href{../works/MartinPY01.pdf}{MartinPY01} (0.66)& \cellcolor{red!40}\href{../works/PesantGPR99.pdf}{PesantGPR99} (0.62)& \cellcolor{red!40}\href{../works/JelinekB16.pdf}{JelinekB16} (0.61)\\
\index{FallahiAC20}\href{../works/FallahiAC20.pdf}{FallahiAC20} R\&C\\
Euclid& \cellcolor{yellow!20}\href{../works/BarzegaranZP20.pdf}{BarzegaranZP20} (0.27)& \cellcolor{green!20}\href{../works/GelainPRVW17.pdf}{GelainPRVW17} (0.30)& \cellcolor{green!20}\href{../works/LiuLH19.pdf}{LiuLH19} (0.30)& \cellcolor{green!20}\href{../works/abs-1902-01193.pdf}{abs-1902-01193} (0.30)& \cellcolor{green!20}\href{../works/ZhangLS12.pdf}{ZhangLS12} (0.30)\\
Dot& \cellcolor{red!40}\href{../works/ZarandiASC20.pdf}{ZarandiASC20} (101.00)& \cellcolor{red!40}\href{../works/Groleaz21.pdf}{Groleaz21} (94.00)& \cellcolor{red!40}\href{../works/Dejemeppe16.pdf}{Dejemeppe16} (93.00)& \cellcolor{red!40}\href{../works/Lemos21.pdf}{Lemos21} (90.00)& \cellcolor{red!40}\href{../works/Froger16.pdf}{Froger16} (90.00)\\
Cosine& \cellcolor{red!40}\href{../works/BarzegaranZP20.pdf}{BarzegaranZP20} (0.71)& \cellcolor{red!40}\href{../works/Wallace06.pdf}{Wallace06} (0.66)& \cellcolor{red!40}\href{../works/LiuLH19.pdf}{LiuLH19} (0.66)& \cellcolor{red!40}\href{../works/KletzanderMH21.pdf}{KletzanderMH21} (0.66)& \cellcolor{red!40}\href{../works/BocewiczBB09.pdf}{BocewiczBB09} (0.65)\\
\index{FalqueALM24}\href{../works/FalqueALM24.pdf}{FalqueALM24} R\&C\\
Euclid& \cellcolor{black!20}\href{../works/LiuCGM17.pdf}{LiuCGM17} (0.34)& \cellcolor{black!20}\href{../works/Puget95.pdf}{Puget95} (0.34)& \cellcolor{black!20}\href{../works/TranDRFWOVB16.pdf}{TranDRFWOVB16} (0.35)& \cellcolor{black!20}\href{../works/QuSN06.pdf}{QuSN06} (0.36)& \cellcolor{black!20}\href{../works/AngelsmarkJ00.pdf}{AngelsmarkJ00} (0.36)\\
Dot& \cellcolor{red!40}\href{../works/Lemos21.pdf}{Lemos21} (99.00)& \cellcolor{red!40}\href{../works/Godet21a.pdf}{Godet21a} (95.00)& \cellcolor{red!40}\href{../works/KoehlerBFFHPSSS21.pdf}{KoehlerBFFHPSSS21} (83.00)& \cellcolor{red!40}\href{../works/Malapert11.pdf}{Malapert11} (83.00)& \cellcolor{red!40}\href{../works/Dejemeppe16.pdf}{Dejemeppe16} (83.00)\\
Cosine& \cellcolor{red!40}\href{../works/LiuCGM17.pdf}{LiuCGM17} (0.61)& \cellcolor{red!40}\href{../works/CarlssonJL17.pdf}{CarlssonJL17} (0.60)& \cellcolor{red!40}\href{../works/GarridoAO09.pdf}{GarridoAO09} (0.59)& \cellcolor{red!40}\href{../works/TranDRFWOVB16.pdf}{TranDRFWOVB16} (0.57)& \cellcolor{red!40}\href{../works/WangB23.pdf}{WangB23} (0.57)\\
\index{FanXG21}\href{../works/FanXG21.pdf}{FanXG21} R\&C& \cellcolor{yellow!20}\href{../works/QinWSLS21.pdf}{QinWSLS21} (0.93)& \cellcolor{blue!20}\href{../works/ColT2019a.pdf}{ColT2019a} (0.96)& \cellcolor{blue!20}\href{../works/MenciaSV12.pdf}{MenciaSV12} (0.97)& \cellcolor{blue!20}\href{../works/ColT22.pdf}{ColT22} (0.97)& \cellcolor{blue!20}DomdorfPH03 (0.97)\\
Euclid& \cellcolor{black!20}\href{../works/FoxAS82.pdf}{FoxAS82} (0.37)& \href{../works/ZhangW18.pdf}{ZhangW18} (0.37)& \href{../works/IklassovMR023.pdf}{IklassovMR023} (0.38)& \href{../works/LiFJZLL22.pdf}{LiFJZLL22} (0.38)& \href{../works/SmithC93.pdf}{SmithC93} (0.38)\\
Dot& \cellcolor{red!40}\href{../works/ZarandiASC20.pdf}{ZarandiASC20} (186.00)& \cellcolor{red!40}\href{../works/Groleaz21.pdf}{Groleaz21} (161.00)& \cellcolor{red!40}\href{../works/Lunardi20.pdf}{Lunardi20} (157.00)& \cellcolor{red!40}\href{../works/Dejemeppe16.pdf}{Dejemeppe16} (138.00)& \cellcolor{red!40}\href{../works/Baptiste02.pdf}{Baptiste02} (138.00)\\
Cosine& \cellcolor{red!40}\href{../works/ZhangW18.pdf}{ZhangW18} (0.72)& \cellcolor{red!40}\href{../works/AlfieriGPS23.pdf}{AlfieriGPS23} (0.70)& \cellcolor{red!40}\href{../works/OujanaAYB22.pdf}{OujanaAYB22} (0.66)& \cellcolor{red!40}\href{../works/LiFJZLL22.pdf}{LiFJZLL22} (0.66)& \cellcolor{red!40}\href{../works/Lunardi20.pdf}{Lunardi20} (0.65)\\
\index{FarsiTM22}\href{../works/FarsiTM22.pdf}{FarsiTM22} R\&C& \cellcolor{red!40}\href{../works/GhandehariK22.pdf}{GhandehariK22} (0.86)& \cellcolor{red!20}\href{../works/YounespourAKE19.pdf}{YounespourAKE19} (0.87)& \cellcolor{red!20}\href{../works/MengLZB21.pdf}{MengLZB21} (0.90)& \cellcolor{yellow!20}\href{../works/GurPAE23.pdf}{GurPAE23} (0.92)& \cellcolor{yellow!20}GhasemiMH23 (0.93)\\
Euclid& \cellcolor{black!20}\href{../works/GurPAE23.pdf}{GurPAE23} (0.36)& \cellcolor{black!20}\href{../works/MeskensDHG11.pdf}{MeskensDHG11} (0.37)& \cellcolor{black!20}\href{../works/MeskensDL13.pdf}{MeskensDL13} (0.37)& \href{../works/GurEA19.pdf}{GurEA19} (0.37)& \href{../works/GhandehariK22.pdf}{GhandehariK22} (0.38)\\
Dot& \cellcolor{red!40}\href{../works/ZarandiASC20.pdf}{ZarandiASC20} (152.00)& \cellcolor{red!40}\href{../works/Dejemeppe16.pdf}{Dejemeppe16} (129.00)& \cellcolor{red!40}\href{../works/Lunardi20.pdf}{Lunardi20} (108.00)& \cellcolor{red!40}\href{../works/Groleaz21.pdf}{Groleaz21} (107.00)& \cellcolor{red!40}\href{../works/NaderiBZ22a.pdf}{NaderiBZ22a} (102.00)\\
Cosine& \cellcolor{red!40}\href{../works/GurPAE23.pdf}{GurPAE23} (0.69)& \cellcolor{red!40}\href{../works/MeskensDL13.pdf}{MeskensDL13} (0.68)& \cellcolor{red!40}\href{../works/MeskensDHG11.pdf}{MeskensDHG11} (0.66)& \cellcolor{red!40}\href{../works/WangMD15.pdf}{WangMD15} (0.65)& \cellcolor{red!40}\href{../works/GurEA19.pdf}{GurEA19} (0.65)\\
\index{Fatemi-AnarakiTFV23}\href{../works/Fatemi-AnarakiTFV23.pdf}{Fatemi-AnarakiTFV23} R\&C& \cellcolor{red!40}NouriMHD23 (0.83)& \cellcolor{green!20}\href{../works/LaborieRSV18.pdf}{LaborieRSV18} (0.94)& \cellcolor{green!20}\href{../works/LiFJZLL22.pdf}{LiFJZLL22} (0.95)& \cellcolor{green!20}\href{../works/NaderiRR23.pdf}{NaderiRR23} (0.95)& \cellcolor{green!20}\href{../works/MengLZB21.pdf}{MengLZB21} (0.95)\\
Euclid& \cellcolor{black!20}\href{../works/Ham18a.pdf}{Ham18a} (0.35)& \cellcolor{black!20}\href{../works/MokhtarzadehTNF20.pdf}{MokhtarzadehTNF20} (0.36)& \cellcolor{black!20}\href{../works/Ham18.pdf}{Ham18} (0.37)& \href{../works/Mehdizadeh-Somarin23.pdf}{Mehdizadeh-Somarin23} (0.38)& \href{../works/HamP21.pdf}{HamP21} (0.38)\\
Dot& \cellcolor{red!40}\href{../works/ZarandiASC20.pdf}{ZarandiASC20} (174.00)& \cellcolor{red!40}\href{../works/Lunardi20.pdf}{Lunardi20} (162.00)& \cellcolor{red!40}\href{../works/Groleaz21.pdf}{Groleaz21} (161.00)& \cellcolor{red!40}\href{../works/Astrand21.pdf}{Astrand21} (153.00)& \cellcolor{red!40}\href{../works/Malapert11.pdf}{Malapert11} (144.00)\\
Cosine& \cellcolor{red!40}\href{../works/Ham18a.pdf}{Ham18a} (0.74)& \cellcolor{red!40}\href{../works/MokhtarzadehTNF20.pdf}{MokhtarzadehTNF20} (0.72)& \cellcolor{red!40}\href{../works/Ham18.pdf}{Ham18} (0.70)& \cellcolor{red!40}\href{../works/CzerniachowskaWZ23.pdf}{CzerniachowskaWZ23} (0.70)& \cellcolor{red!40}\href{../works/Mehdizadeh-Somarin23.pdf}{Mehdizadeh-Somarin23} (0.69)\\
\index{FeldmanG89}\href{../works/FeldmanG89.pdf}{FeldmanG89} R\&C\\
Euclid& \cellcolor{red!40}\href{../works/FrostD98.pdf}{FrostD98} (0.13)& \cellcolor{red!40}\href{../works/AbrilSB05.pdf}{AbrilSB05} (0.14)& \cellcolor{red!40}\href{../works/CarchraeBF05.pdf}{CarchraeBF05} (0.16)& \cellcolor{red!40}\href{../works/LiuJ06.pdf}{LiuJ06} (0.17)& \cellcolor{red!40}\href{../works/AngelsmarkJ00.pdf}{AngelsmarkJ00} (0.17)\\
Dot& \cellcolor{red!40}\href{../works/SakkoutW00.pdf}{SakkoutW00} (30.00)& \cellcolor{red!40}\href{../works/Godet21a.pdf}{Godet21a} (30.00)& \cellcolor{red!40}\href{../works/ZarandiKS16.pdf}{ZarandiKS16} (30.00)& \cellcolor{red!40}\href{../works/KanetAG04.pdf}{KanetAG04} (30.00)& \cellcolor{red!40}\href{../works/TrojetHL11.pdf}{TrojetHL11} (30.00)\\
Cosine& \cellcolor{red!40}\href{../works/FrostD98.pdf}{FrostD98} (0.76)& \cellcolor{red!40}\href{../works/AbrilSB05.pdf}{AbrilSB05} (0.72)& \cellcolor{red!40}\href{../works/ZhangLS12.pdf}{ZhangLS12} (0.69)& \cellcolor{red!40}\href{../works/GelainPRVW17.pdf}{GelainPRVW17} (0.68)& \cellcolor{red!40}\href{../works/BandaSC11.pdf}{BandaSC11} (0.68)\\
\index{FelizariAL09}FelizariAL09 R\&C& \cellcolor{red!40}MagataoAN05 (0.00)& \cellcolor{red!40}\href{../works/MouraSCL08a.pdf}{MouraSCL08a} (0.77)& \cellcolor{red!20}\href{../works/HookerY02.pdf}{HookerY02} (0.89)& \cellcolor{yellow!20}\href{../works/LopesCSM10.pdf}{LopesCSM10} (0.92)& \cellcolor{green!20}\href{../works/MouraSCL08.pdf}{MouraSCL08} (0.94)\\
Euclid\\
Dot\\
Cosine\\
\index{FetgoD22}\href{../works/FetgoD22.pdf}{FetgoD22} R\&C& \cellcolor{red!40}\href{../works/KameugneFGOQ18.pdf}{KameugneFGOQ18} (0.72)& \cellcolor{red!40}\href{../works/Tesch16.pdf}{Tesch16} (0.76)& \cellcolor{red!40}\href{../works/GayHS15a.pdf}{GayHS15a} (0.78)& \cellcolor{red!40}\href{../works/YangSS19.pdf}{YangSS19} (0.79)& \cellcolor{red!40}\href{../works/Tesch18.pdf}{Tesch18} (0.80)\\
Euclid& \cellcolor{red!40}\href{../works/KameugneFND23.pdf}{KameugneFND23} (0.23)& \cellcolor{red!20}\href{../works/KameugneFGOQ18.pdf}{KameugneFGOQ18} (0.26)& \cellcolor{yellow!20}\href{../works/OuelletQ13.pdf}{OuelletQ13} (0.28)& \cellcolor{green!20}\href{../works/GingrasQ16.pdf}{GingrasQ16} (0.30)& \cellcolor{black!20}\href{../works/KameugneFSN14.pdf}{KameugneFSN14} (0.35)\\
Dot& \cellcolor{red!40}\href{../works/Fahimi16.pdf}{Fahimi16} (161.00)& \cellcolor{red!40}\href{../works/Lombardi10.pdf}{Lombardi10} (159.00)& \cellcolor{red!40}\href{../works/Schutt11.pdf}{Schutt11} (159.00)& \cellcolor{red!40}\href{../works/Baptiste02.pdf}{Baptiste02} (159.00)& \cellcolor{red!40}\href{../works/KameugneFND23.pdf}{KameugneFND23} (155.00)\\
Cosine& \cellcolor{red!40}\href{../works/KameugneFND23.pdf}{KameugneFND23} (0.90)& \cellcolor{red!40}\href{../works/KameugneFGOQ18.pdf}{KameugneFGOQ18} (0.86)& \cellcolor{red!40}\href{../works/OuelletQ13.pdf}{OuelletQ13} (0.84)& \cellcolor{red!40}\href{../works/GingrasQ16.pdf}{GingrasQ16} (0.81)& \cellcolor{red!40}\href{../works/KameugneFSN14.pdf}{KameugneFSN14} (0.75)\\
\index{FocacciLN00}\href{../works/FocacciLN00.pdf}{FocacciLN00} R\&C\\
Euclid& \cellcolor{yellow!20}\href{../works/CauwelaertDMS16.pdf}{CauwelaertDMS16} (0.27)& \cellcolor{yellow!20}\href{../works/ArtiguesBF04.pdf}{ArtiguesBF04} (0.27)& \cellcolor{green!20}\href{../works/ArtiguesF07.pdf}{ArtiguesF07} (0.29)& \cellcolor{green!20}\href{../works/VilimBC05.pdf}{VilimBC05} (0.30)& \cellcolor{green!20}\href{../works/CauwelaertDS20.pdf}{CauwelaertDS20} (0.30)\\
Dot& \cellcolor{red!40}\href{../works/Baptiste02.pdf}{Baptiste02} (160.00)& \cellcolor{red!40}\href{../works/Dejemeppe16.pdf}{Dejemeppe16} (151.00)& \cellcolor{red!40}\href{../works/Malapert11.pdf}{Malapert11} (150.00)& \cellcolor{red!40}\href{../works/GrimesH15.pdf}{GrimesH15} (148.00)& \cellcolor{red!40}\href{../works/Fahimi16.pdf}{Fahimi16} (147.00)\\
Cosine& \cellcolor{red!40}\href{../works/ArtiguesBF04.pdf}{ArtiguesBF04} (0.82)& \cellcolor{red!40}\href{../works/CauwelaertDMS16.pdf}{CauwelaertDMS16} (0.81)& \cellcolor{red!40}\href{../works/ArtiguesF07.pdf}{ArtiguesF07} (0.81)& \cellcolor{red!40}\href{../works/CauwelaertDS20.pdf}{CauwelaertDS20} (0.80)& \cellcolor{red!40}\href{../works/GrimesH10.pdf}{GrimesH10} (0.79)\\
\index{FontaineMH16}\href{../works/FontaineMH16.pdf}{FontaineMH16} R\&C& \cellcolor{red!20}\href{../works/YounespourAKE19.pdf}{YounespourAKE19} (0.90)& \cellcolor{green!20}\href{../works/HermenierDL11.pdf}{HermenierDL11} (0.94)& \cellcolor{green!20}\href{../works/RoshanaeiN21.pdf}{RoshanaeiN21} (0.94)& \cellcolor{green!20}\href{../works/Laborie18a.pdf}{Laborie18a} (0.95)& \cellcolor{green!20}NaderiRBAU21 (0.96)\\
Euclid& \cellcolor{red!20}\href{../works/CarchraeB09.pdf}{CarchraeB09} (0.26)& \cellcolor{yellow!20}\href{../works/SialaAH15.pdf}{SialaAH15} (0.27)& \cellcolor{yellow!20}\href{../works/Colombani96.pdf}{Colombani96} (0.27)& \cellcolor{yellow!20}\href{../works/HeipckeCCS00.pdf}{HeipckeCCS00} (0.28)& \cellcolor{yellow!20}\href{../works/Puget95.pdf}{Puget95} (0.28)\\
Dot& \cellcolor{red!40}\href{../works/Groleaz21.pdf}{Groleaz21} (120.00)& \cellcolor{red!40}\href{../works/Lombardi10.pdf}{Lombardi10} (113.00)& \cellcolor{red!40}\href{../works/Astrand21.pdf}{Astrand21} (112.00)& \cellcolor{red!40}\href{../works/NaderiRR23.pdf}{NaderiRR23} (110.00)& \cellcolor{red!40}\href{../works/Baptiste02.pdf}{Baptiste02} (110.00)\\
Cosine& \cellcolor{red!40}\href{../works/CarchraeB09.pdf}{CarchraeB09} (0.81)& \cellcolor{red!40}\href{../works/SialaAH15.pdf}{SialaAH15} (0.78)& \cellcolor{red!40}\href{../works/ZhangBB22.pdf}{ZhangBB22} (0.77)& \cellcolor{red!40}\href{../works/Colombani96.pdf}{Colombani96} (0.76)& \cellcolor{red!40}\href{../works/HeipckeCCS00.pdf}{HeipckeCCS00} (0.76)\\
\index{ForbesHJST24}\href{../works/ForbesHJST24.pdf}{ForbesHJST24} R\&C& \cellcolor{red!40}\href{../works/ElciOH22.pdf}{ElciOH22} (0.85)& \cellcolor{yellow!20}\href{../works/CambazardJ05.pdf}{CambazardJ05} (0.91)& \cellcolor{yellow!20}\href{../works/Hooker04.pdf}{Hooker04} (0.91)& \cellcolor{yellow!20}\href{../works/Hooker05a.pdf}{Hooker05a} (0.92)& \cellcolor{yellow!20}\href{../works/BeniniBGM06.pdf}{BeniniBGM06} (0.92)\\
Euclid& \cellcolor{green!20}\href{../works/HookerY02.pdf}{HookerY02} (0.30)& \cellcolor{green!20}\href{../works/CireCH13.pdf}{CireCH13} (0.30)& \cellcolor{green!20}\href{../works/ElciOH22.pdf}{ElciOH22} (0.31)& \cellcolor{green!20}\href{../works/HookerO03.pdf}{HookerO03} (0.31)& \cellcolor{green!20}\href{../works/CireCH16.pdf}{CireCH16} (0.31)\\
Dot& \cellcolor{red!40}\href{../works/Lombardi10.pdf}{Lombardi10} (120.00)& \cellcolor{red!40}\href{../works/Froger16.pdf}{Froger16} (117.00)& \cellcolor{red!40}\href{../works/ZarandiASC20.pdf}{ZarandiASC20} (112.00)& \cellcolor{red!40}\href{../works/Groleaz21.pdf}{Groleaz21} (110.00)& \cellcolor{red!40}\href{../works/NaderiRR23.pdf}{NaderiRR23} (109.00)\\
Cosine& \cellcolor{red!40}\href{../works/ElciOH22.pdf}{ElciOH22} (0.75)& \cellcolor{red!40}\href{../works/HookerY02.pdf}{HookerY02} (0.73)& \cellcolor{red!40}\href{../works/CobanH11.pdf}{CobanH11} (0.72)& \cellcolor{red!40}\href{../works/EmeretlisTAV17.pdf}{EmeretlisTAV17} (0.72)& \cellcolor{red!40}\href{../works/CireCH13.pdf}{CireCH13} (0.71)\\
\index{FortinZDF05}\href{../works/FortinZDF05.pdf}{FortinZDF05} R\&C& \cellcolor{red!20}\href{../works/LombardiBM15.pdf}{LombardiBM15} (0.89)& \cellcolor{yellow!20}\href{../works/LombardiM13.pdf}{LombardiM13} (0.92)& \cellcolor{yellow!20}\href{../works/LombardiM12a.pdf}{LombardiM12a} (0.92)& \cellcolor{yellow!20}LiuGT10 (0.92)& \cellcolor{green!20}\href{../works/Muscettola02.pdf}{Muscettola02} (0.93)\\
Euclid& \cellcolor{red!40}\href{../works/LombardiM13.pdf}{LombardiM13} (0.18)& \cellcolor{red!40}\href{../works/LeeKLKKYHP97.pdf}{LeeKLKKYHP97} (0.19)& \cellcolor{red!40}\href{../works/BeniniBGM05a.pdf}{BeniniBGM05a} (0.20)& \cellcolor{red!40}\href{../works/CarchraeBF05.pdf}{CarchraeBF05} (0.21)& \cellcolor{red!40}\href{../works/WallaceF00.pdf}{WallaceF00} (0.22)\\
Dot& \cellcolor{red!40}\href{../works/LaborieRSV18.pdf}{LaborieRSV18} (63.00)& \cellcolor{red!40}\href{../works/SchuttFSW11.pdf}{SchuttFSW11} (63.00)& \cellcolor{red!40}\href{../works/LombardiM10a.pdf}{LombardiM10a} (63.00)& \cellcolor{red!40}\href{../works/ZarandiASC20.pdf}{ZarandiASC20} (63.00)& \cellcolor{red!40}\href{../works/Lombardi10.pdf}{Lombardi10} (63.00)\\
Cosine& \cellcolor{red!40}\href{../works/LombardiM09.pdf}{LombardiM09} (0.81)& \cellcolor{red!40}\href{../works/LombardiM13.pdf}{LombardiM13} (0.79)& \cellcolor{red!40}\href{../works/LombardiBM15.pdf}{LombardiBM15} (0.78)& \cellcolor{red!40}\href{../works/BeniniLMMR08.pdf}{BeniniLMMR08} (0.78)& \cellcolor{red!40}\href{../works/LombardiBMB11.pdf}{LombardiBMB11} (0.76)\\
\index{FoxAS82}\href{../works/FoxAS82.pdf}{FoxAS82} R\&C\\
Euclid& \cellcolor{red!40}\href{../works/CrawfordB94.pdf}{CrawfordB94} (0.16)& \cellcolor{red!40}\href{../works/LauLN08.pdf}{LauLN08} (0.20)& \cellcolor{red!40}\href{../works/KengY89.pdf}{KengY89} (0.21)& \cellcolor{red!40}\href{../works/HebrardTW05.pdf}{HebrardTW05} (0.21)& \cellcolor{red!40}\href{../works/FukunagaHFAMN02.pdf}{FukunagaHFAMN02} (0.22)\\
Dot& \cellcolor{red!40}\href{../works/ZarandiASC20.pdf}{ZarandiASC20} (72.00)& \cellcolor{red!40}\href{../works/BidotVLB09.pdf}{BidotVLB09} (69.00)& \cellcolor{red!40}\href{../works/Groleaz21.pdf}{Groleaz21} (68.00)& \cellcolor{red!40}\href{../works/BartakSR10.pdf}{BartakSR10} (68.00)& \cellcolor{red!40}\href{../works/Beck99.pdf}{Beck99} (68.00)\\
Cosine& \cellcolor{red!40}\href{../works/CrawfordB94.pdf}{CrawfordB94} (0.83)& \cellcolor{red!40}\href{../works/SmithC93.pdf}{SmithC93} (0.78)& \cellcolor{red!40}\href{../works/KengY89.pdf}{KengY89} (0.77)& \cellcolor{red!40}\href{../works/Prosser89.pdf}{Prosser89} (0.76)& \cellcolor{red!40}\href{../works/Colombani96.pdf}{Colombani96} (0.73)\\
\index{FoxS90}\href{../works/FoxS90.pdf}{FoxS90} R\&C\\
Euclid& \cellcolor{green!20}\href{../works/BeckPS03.pdf}{BeckPS03} (0.30)& \cellcolor{green!20}\href{../works/SmithC93.pdf}{SmithC93} (0.30)& \cellcolor{green!20}\href{../works/Junker00.pdf}{Junker00} (0.30)& \cellcolor{green!20}\href{../works/Prosser89.pdf}{Prosser89} (0.31)& \cellcolor{green!20}\href{../works/SadehF96.pdf}{SadehF96} (0.31)\\
Dot& \cellcolor{red!40}\href{../works/ZarandiASC20.pdf}{ZarandiASC20} (156.00)& \cellcolor{red!40}\href{../works/Baptiste02.pdf}{Baptiste02} (153.00)& \cellcolor{red!40}\href{../works/Dejemeppe16.pdf}{Dejemeppe16} (145.00)& \cellcolor{red!40}\href{../works/Groleaz21.pdf}{Groleaz21} (141.00)& \cellcolor{red!40}\href{../works/Malapert11.pdf}{Malapert11} (139.00)\\
Cosine& \cellcolor{red!40}\href{../works/BeckDDF98.pdf}{BeckDDF98} (0.77)& \cellcolor{red!40}\href{../works/SadehF96.pdf}{SadehF96} (0.76)& \cellcolor{red!40}\href{../works/BeckPS03.pdf}{BeckPS03} (0.76)& \cellcolor{red!40}\href{../works/BeckF00.pdf}{BeckF00} (0.74)& \cellcolor{red!40}\href{../works/SmithC93.pdf}{SmithC93} (0.73)\\
\index{FrankDT16}\href{../works/FrankDT16.pdf}{FrankDT16} R\&C& \cellcolor{red!40}\href{../works/GilesH16.pdf}{GilesH16} (0.83)& \cellcolor{red!40}\href{../works/CappartS17.pdf}{CappartS17} (0.83)& \cellcolor{yellow!20}\href{../works/GayHS15.pdf}{GayHS15} (0.93)& \cellcolor{green!20}\href{../works/LaborieR14.pdf}{LaborieR14} (0.95)& \cellcolor{green!20}\href{../works/QinDS16.pdf}{QinDS16} (0.95)\\
Euclid& \cellcolor{red!40}\href{../works/AngelsmarkJ00.pdf}{AngelsmarkJ00} (0.23)& \cellcolor{red!40}\href{../works/KucukY19.pdf}{KucukY19} (0.23)& \cellcolor{red!40}\href{../works/Baptiste09.pdf}{Baptiste09} (0.23)& \cellcolor{red!40}\href{../works/CarchraeBF05.pdf}{CarchraeBF05} (0.23)& \cellcolor{red!40}\href{../works/KovacsEKV05.pdf}{KovacsEKV05} (0.24)\\
Dot& \cellcolor{red!40}\href{../works/LaborieRSV18.pdf}{LaborieRSV18} (67.00)& \cellcolor{red!40}\href{../works/Groleaz21.pdf}{Groleaz21} (62.00)& \cellcolor{red!40}\href{../works/Astrand21.pdf}{Astrand21} (59.00)& \cellcolor{red!40}\href{../works/Lunardi20.pdf}{Lunardi20} (58.00)& \cellcolor{red!40}\href{../works/ZarandiASC20.pdf}{ZarandiASC20} (56.00)\\
Cosine& \cellcolor{red!40}\href{../works/KucukY19.pdf}{KucukY19} (0.73)& \cellcolor{red!40}\href{../works/PraletLJ15.pdf}{PraletLJ15} (0.69)& \cellcolor{red!40}\href{../works/BoothNB16.pdf}{BoothNB16} (0.67)& \cellcolor{red!40}\href{../works/NishikawaSTT18a.pdf}{NishikawaSTT18a} (0.67)& \cellcolor{red!40}\href{../works/VerfaillieL01.pdf}{VerfaillieL01} (0.66)\\
\index{FrankK03}\href{../works/FrankK03.pdf}{FrankK03} R\&C\\
Euclid& \cellcolor{red!40}\href{../works/FrankK05.pdf}{FrankK05} (0.21)& \cellcolor{red!20}\href{../works/CarchraeBF05.pdf}{CarchraeBF05} (0.24)& \cellcolor{red!20}\href{../works/FrostD98.pdf}{FrostD98} (0.25)& \cellcolor{red!20}\href{../works/FeldmanG89.pdf}{FeldmanG89} (0.25)& \cellcolor{red!20}\href{../works/LiuJ06.pdf}{LiuJ06} (0.25)\\
Dot& \cellcolor{red!40}\href{../works/FrankK05.pdf}{FrankK05} (54.00)& \cellcolor{red!40}\href{../works/GokPTGO23.pdf}{GokPTGO23} (42.00)& \cellcolor{red!40}\href{../works/PohlAK22.pdf}{PohlAK22} (39.00)& \cellcolor{red!40}\href{../works/TranDRFWOVB16.pdf}{TranDRFWOVB16} (38.00)& \cellcolor{red!40}\href{../works/Simonis99.pdf}{Simonis99} (37.00)\\
Cosine& \cellcolor{red!40}\href{../works/FrankK05.pdf}{FrankK05} (0.81)& \cellcolor{red!40}\href{../works/TranDRFWOVB16.pdf}{TranDRFWOVB16} (0.59)& \cellcolor{red!40}\href{../works/JoLLH99.pdf}{JoLLH99} (0.53)& \cellcolor{red!40}\href{../works/Gronkvist06.pdf}{Gronkvist06} (0.50)& \cellcolor{red!40}\href{../works/OrnekOS20.pdf}{OrnekOS20} (0.49)\\
\index{FrankK05}\href{../works/FrankK05.pdf}{FrankK05} R\&C& \cellcolor{green!20}\href{../works/BenoistGR02.pdf}{BenoistGR02} (0.94)& \cellcolor{green!20}\href{../works/LiW08.pdf}{LiW08} (0.94)& \cellcolor{blue!20}\href{../works/EreminW01.pdf}{EreminW01} (0.97)& \cellcolor{black!20}\href{../works/BruckerK00.pdf}{BruckerK00} (0.99)& \cellcolor{black!20}BockmayrK98 (0.99)\\
Euclid& \cellcolor{red!40}\href{../works/FrankK03.pdf}{FrankK03} (0.21)& \cellcolor{yellow!20}\href{../works/TranDRFWOVB16.pdf}{TranDRFWOVB16} (0.27)& \cellcolor{green!20}\href{../works/AngelsmarkJ00.pdf}{AngelsmarkJ00} (0.30)& \cellcolor{green!20}\href{../works/Johnston05.pdf}{Johnston05} (0.30)& \cellcolor{green!20}\href{../works/Hunsberger08.pdf}{Hunsberger08} (0.31)\\
Dot& \cellcolor{red!40}\href{../works/ZarandiASC20.pdf}{ZarandiASC20} (73.00)& \cellcolor{red!40}\href{../works/GokPTGO23.pdf}{GokPTGO23} (70.00)& \cellcolor{red!40}\href{../works/Fahimi16.pdf}{Fahimi16} (70.00)& \cellcolor{red!40}\href{../works/Lombardi10.pdf}{Lombardi10} (69.00)& \cellcolor{red!40}\href{../works/LaborieRSV18.pdf}{LaborieRSV18} (67.00)\\
Cosine& \cellcolor{red!40}\href{../works/FrankK03.pdf}{FrankK03} (0.81)& \cellcolor{red!40}\href{../works/TranDRFWOVB16.pdf}{TranDRFWOVB16} (0.70)& \cellcolor{red!40}\href{../works/Johnston05.pdf}{Johnston05} (0.63)& \cellcolor{red!40}\href{../works/CatusseCBL16.pdf}{CatusseCBL16} (0.61)& \cellcolor{red!40}\href{../works/BarbulescuWH04.pdf}{BarbulescuWH04} (0.60)\\
\index{FriedrichFMRSST14}FriedrichFMRSST14 R\&C& \cellcolor{yellow!20}\href{../works/DannaP03.pdf}{DannaP03} (0.92)& \cellcolor{yellow!20}\href{../works/ColT2019a.pdf}{ColT2019a} (0.92)& \cellcolor{yellow!20}\href{../works/ColT19.pdf}{ColT19} (0.93)& \cellcolor{green!20}\href{../works/BartakSR10.pdf}{BartakSR10} (0.94)& \cellcolor{green!20}\href{../works/Balduccini11.pdf}{Balduccini11} (0.96)\\
Euclid\\
Dot\\
Cosine\\
\index{FrimodigECM23}\href{../works/FrimodigECM23.pdf}{FrimodigECM23} R\&C\\
Euclid& \cellcolor{red!20}\href{../works/FrimodigS19.pdf}{FrimodigS19} (0.25)& \cellcolor{green!20}\href{../works/DoulabiRP16.pdf}{DoulabiRP16} (0.31)& \cellcolor{blue!20}\href{../works/TopalogluO11.pdf}{TopalogluO11} (0.32)& \cellcolor{blue!20}\href{../works/GhandehariK22.pdf}{GhandehariK22} (0.32)& \cellcolor{black!20}\href{../works/DoulabiRP14.pdf}{DoulabiRP14} (0.34)\\
Dot& \cellcolor{red!40}\href{../works/Dejemeppe16.pdf}{Dejemeppe16} (115.00)& \cellcolor{red!40}\href{../works/ZarandiASC20.pdf}{ZarandiASC20} (106.00)& \cellcolor{red!40}\href{../works/RoshanaeiLAU17.pdf}{RoshanaeiLAU17} (101.00)& \cellcolor{red!40}\href{../works/Froger16.pdf}{Froger16} (98.00)& \cellcolor{red!40}\href{../works/Lombardi10.pdf}{Lombardi10} (97.00)\\
Cosine& \cellcolor{red!40}\href{../works/FrimodigS19.pdf}{FrimodigS19} (0.82)& \cellcolor{red!40}\href{../works/DoulabiRP16.pdf}{DoulabiRP16} (0.72)& \cellcolor{red!40}\href{../works/GhandehariK22.pdf}{GhandehariK22} (0.72)& \cellcolor{red!40}\href{../works/TopalogluO11.pdf}{TopalogluO11} (0.71)& \cellcolor{red!40}\href{../works/RoshanaeiLAU17.pdf}{RoshanaeiLAU17} (0.70)\\
\index{FrimodigS19}\href{../works/FrimodigS19.pdf}{FrimodigS19} R\&C& \cellcolor{red!40}\href{../works/BhatnagarKL19.pdf}{BhatnagarKL19} (0.75)& \cellcolor{red!40}\href{../works/GeibingerMM19.pdf}{GeibingerMM19} (0.78)& \cellcolor{yellow!20}\href{../works/LetortBC12.pdf}{LetortBC12} (0.92)& \cellcolor{green!20}\href{../works/SchausHMCMD11.pdf}{SchausHMCMD11} (0.94)& \cellcolor{green!20}\href{../works/GarganiR07.pdf}{GarganiR07} (0.94)\\
Euclid& \cellcolor{red!20}\href{../works/FrimodigECM23.pdf}{FrimodigECM23} (0.25)& \cellcolor{green!20}\href{../works/BourdaisGP03.pdf}{BourdaisGP03} (0.31)& \cellcolor{blue!20}\href{../works/LauLN08.pdf}{LauLN08} (0.32)& \cellcolor{blue!20}\href{../works/HoYCLLCLC18.pdf}{HoYCLLCLC18} (0.32)& \cellcolor{blue!20}\href{../works/MurphyMB15.pdf}{MurphyMB15} (0.32)\\
Dot& \cellcolor{red!40}\href{../works/Dejemeppe16.pdf}{Dejemeppe16} (97.00)& \cellcolor{red!40}\href{../works/FrimodigECM23.pdf}{FrimodigECM23} (94.00)& \cellcolor{red!40}\href{../works/Lombardi10.pdf}{Lombardi10} (86.00)& \cellcolor{red!40}\href{../works/LaborieRSV18.pdf}{LaborieRSV18} (85.00)& \cellcolor{red!40}\href{../works/SenderovichBB19.pdf}{SenderovichBB19} (83.00)\\
Cosine& \cellcolor{red!40}\href{../works/FrimodigECM23.pdf}{FrimodigECM23} (0.82)& \cellcolor{red!40}\href{../works/SenderovichBB19.pdf}{SenderovichBB19} (0.67)& \cellcolor{red!40}\href{../works/NaderiBZR23.pdf}{NaderiBZR23} (0.64)& \cellcolor{red!40}\href{../works/RoshanaeiLAU17.pdf}{RoshanaeiLAU17} (0.64)& \cellcolor{red!40}\href{../works/RoshanaeiBAUB20.pdf}{RoshanaeiBAUB20} (0.63)\\
\index{Froger16}\href{../works/Froger16.pdf}{Froger16} R\&C\\
Euclid& \href{../works/LuZZYW24.pdf}{LuZZYW24} (0.54)& \href{../works/EmeretlisTAV17.pdf}{EmeretlisTAV17} (0.60)& \href{../works/CilKLO22.pdf}{CilKLO22} (0.60)& \href{../works/Hooker19.pdf}{Hooker19} (0.60)& \href{../works/PinarbasiAY19.pdf}{PinarbasiAY19} (0.60)\\
Dot& \cellcolor{red!40}\href{../works/ZarandiASC20.pdf}{ZarandiASC20} (227.00)& \cellcolor{red!40}\href{../works/Groleaz21.pdf}{Groleaz21} (209.00)& \cellcolor{red!40}\href{../works/Astrand21.pdf}{Astrand21} (194.00)& \cellcolor{red!40}\href{../works/Lunardi20.pdf}{Lunardi20} (192.00)& \cellcolor{red!40}\href{../works/Lombardi10.pdf}{Lombardi10} (189.00)\\
Cosine& \cellcolor{red!40}\href{../works/LuZZYW24.pdf}{LuZZYW24} (0.66)& \cellcolor{red!40}\href{../works/Hooker19.pdf}{Hooker19} (0.57)& \cellcolor{red!40}\href{../works/CilKLO22.pdf}{CilKLO22} (0.56)& \cellcolor{red!40}\href{../works/Lemos21.pdf}{Lemos21} (0.56)& \cellcolor{red!40}\href{../works/EmeretlisTAV17.pdf}{EmeretlisTAV17} (0.56)\\
\index{FrohnerTR19}\href{../works/FrohnerTR19.pdf}{FrohnerTR19} R\&C& \cellcolor{red!20}\href{../works/MusliuSS18.pdf}{MusliuSS18} (0.90)& \cellcolor{red!20}\href{../works/HoYCLLCLC18.pdf}{HoYCLLCLC18} (0.90)& \cellcolor{green!20}\href{../works/BofillEGPSV14.pdf}{BofillEGPSV14} (0.94)& \cellcolor{green!20}\href{../works/ThiruvadyBME09.pdf}{ThiruvadyBME09} (0.94)& \cellcolor{green!20}\href{../works/ColT19.pdf}{ColT19} (0.94)\\
Euclid& \cellcolor{green!20}\href{../works/ZibranR11.pdf}{ZibranR11} (0.29)& \cellcolor{green!20}\href{../works/BofillCGGPSV23.pdf}{BofillCGGPSV23} (0.29)& \cellcolor{green!20}\href{../works/ZhangLS12.pdf}{ZhangLS12} (0.30)& \cellcolor{green!20}\href{../works/GeibingerKKMMW21.pdf}{GeibingerKKMMW21} (0.31)& \cellcolor{green!20}\href{../works/BarzegaranZP20.pdf}{BarzegaranZP20} (0.31)\\
Dot& \cellcolor{red!40}\href{../works/ZarandiASC20.pdf}{ZarandiASC20} (73.00)& \cellcolor{red!40}\href{../works/KoehlerBFFHPSSS21.pdf}{KoehlerBFFHPSSS21} (71.00)& \cellcolor{red!40}\href{../works/Lemos21.pdf}{Lemos21} (71.00)& \cellcolor{red!40}\href{../works/Dejemeppe16.pdf}{Dejemeppe16} (70.00)& \cellcolor{red!40}\href{../works/Groleaz21.pdf}{Groleaz21} (67.00)\\
Cosine& \cellcolor{red!40}\href{../works/GeibingerKKMMW21.pdf}{GeibingerKKMMW21} (0.59)& \cellcolor{red!40}\href{../works/BofillCGGPSV23.pdf}{BofillCGGPSV23} (0.58)& \cellcolor{red!40}\href{../works/ZibranR11.pdf}{ZibranR11} (0.58)& \cellcolor{red!40}\href{../works/BarzegaranZP20.pdf}{BarzegaranZP20} (0.58)& \cellcolor{red!40}\href{../works/ZhangLS12.pdf}{ZhangLS12} (0.58)\\
\index{FrostD98}\href{../works/FrostD98.pdf}{FrostD98} R\&C& \cellcolor{green!20}\href{../works/KhemmoudjPB06.pdf}{KhemmoudjPB06} (0.94)\\
Euclid& \cellcolor{red!40}\href{../works/Baptiste09.pdf}{Baptiste09} (0.13)& \cellcolor{red!40}\href{../works/AbrilSB05.pdf}{AbrilSB05} (0.13)& \cellcolor{red!40}\href{../works/CarchraeBF05.pdf}{CarchraeBF05} (0.13)& \cellcolor{red!40}\href{../works/FeldmanG89.pdf}{FeldmanG89} (0.13)& \cellcolor{red!40}\href{../works/AngelsmarkJ00.pdf}{AngelsmarkJ00} (0.15)\\
Dot& \cellcolor{red!40}\href{../works/Froger16.pdf}{Froger16} (26.00)& \cellcolor{red!40}\href{../works/Malapert11.pdf}{Malapert11} (25.00)& \cellcolor{red!40}\href{../works/Godet21a.pdf}{Godet21a} (23.00)& \cellcolor{red!40}\href{../works/ZarandiASC20.pdf}{ZarandiASC20} (23.00)& \cellcolor{red!40}\href{../works/BajestaniB13.pdf}{BajestaniB13} (23.00)\\
Cosine& \cellcolor{red!40}\href{../works/FeldmanG89.pdf}{FeldmanG89} (0.76)& \cellcolor{red!40}\href{../works/ZhangLS12.pdf}{ZhangLS12} (0.67)& \cellcolor{red!40}\href{../works/CarchraeBF05.pdf}{CarchraeBF05} (0.60)& \cellcolor{red!40}\href{../works/GelainPRVW17.pdf}{GelainPRVW17} (0.59)& \cellcolor{red!40}\href{../works/ZibranR11.pdf}{ZibranR11} (0.58)\\
\index{FukunagaHFAMN02}\href{../works/FukunagaHFAMN02.pdf}{FukunagaHFAMN02} R\&C\\
Euclid& \cellcolor{red!40}\href{../works/LudwigKRBMS14.pdf}{LudwigKRBMS14} (0.17)& \cellcolor{red!40}\href{../works/AngelsmarkJ00.pdf}{AngelsmarkJ00} (0.17)& \cellcolor{red!40}\href{../works/LiuJ06.pdf}{LiuJ06} (0.18)& \cellcolor{red!40}\href{../works/WallaceF00.pdf}{WallaceF00} (0.18)& \cellcolor{red!40}\href{../works/Davis87.pdf}{Davis87} (0.18)\\
Dot& \cellcolor{red!40}\href{../works/ZarandiASC20.pdf}{ZarandiASC20} (53.00)& \cellcolor{red!40}\href{../works/Beck99.pdf}{Beck99} (51.00)& \cellcolor{red!40}\href{../works/Lombardi10.pdf}{Lombardi10} (48.00)& \cellcolor{red!40}\href{../works/Astrand21.pdf}{Astrand21} (48.00)& \cellcolor{red!40}\href{../works/BeckDDF98.pdf}{BeckDDF98} (48.00)\\
Cosine& \cellcolor{red!40}\href{../works/LudwigKRBMS14.pdf}{LudwigKRBMS14} (0.75)& \cellcolor{red!40}\href{../works/LouieVNB14.pdf}{LouieVNB14} (0.74)& \cellcolor{red!40}\href{../works/CrawfordB94.pdf}{CrawfordB94} (0.71)& \cellcolor{red!40}\href{../works/BridiLBBM16.pdf}{BridiLBBM16} (0.70)& \cellcolor{red!40}\href{../works/WallaceF00.pdf}{WallaceF00} (0.68)\\
\index{GalleguillosKSB19}\href{../works/GalleguillosKSB19.pdf}{GalleguillosKSB19} R\&C& \cellcolor{green!20}\href{../works/BorghesiBLMB18.pdf}{BorghesiBLMB18} (0.94)& \cellcolor{green!20}\href{../works/HurleyOS16.pdf}{HurleyOS16} (0.95)& \cellcolor{green!20}\href{../works/MurinR19.pdf}{MurinR19} (0.96)& \cellcolor{green!20}\href{../works/BridiBLMB16.pdf}{BridiBLMB16} (0.96)& \cellcolor{blue!20}\href{../works/ParkUJR19.pdf}{ParkUJR19} (0.97)\\
Euclid& \cellcolor{red!40}\href{../works/BartoliniBBLM14.pdf}{BartoliniBBLM14} (0.23)& \cellcolor{red!20}\href{../works/Limtanyakul07.pdf}{Limtanyakul07} (0.25)& \cellcolor{yellow!20}\href{../works/HookerY02.pdf}{HookerY02} (0.27)& \cellcolor{yellow!20}\href{../works/DoRZ08.pdf}{DoRZ08} (0.28)& \cellcolor{yellow!20}\href{../works/HurleyOS16.pdf}{HurleyOS16} (0.28)\\
Dot& \cellcolor{red!40}\href{../works/ZarandiASC20.pdf}{ZarandiASC20} (93.00)& \cellcolor{red!40}\href{../works/Groleaz21.pdf}{Groleaz21} (91.00)& \cellcolor{red!40}\href{../works/Lombardi10.pdf}{Lombardi10} (86.00)& \cellcolor{red!40}\href{../works/LaborieRSV18.pdf}{LaborieRSV18} (85.00)& \cellcolor{red!40}\href{../works/Godet21a.pdf}{Godet21a} (85.00)\\
Cosine& \cellcolor{red!40}\href{../works/BartoliniBBLM14.pdf}{BartoliniBBLM14} (0.78)& \cellcolor{red!40}\href{../works/BorghesiBLMB18.pdf}{BorghesiBLMB18} (0.74)& \cellcolor{red!40}\href{../works/Limtanyakul07.pdf}{Limtanyakul07} (0.72)& \cellcolor{red!40}\href{../works/BridiBLMB16.pdf}{BridiBLMB16} (0.71)& \cellcolor{red!40}\href{../works/CarchraeB09.pdf}{CarchraeB09} (0.70)\\
\index{GarganiR07}\href{../works/GarganiR07.pdf}{GarganiR07} R\&C& \cellcolor{red!40}\href{../works/HentenryckM08.pdf}{HentenryckM08} (0.52)& \cellcolor{red!40}\href{../works/HeinzSSW12.pdf}{HeinzSSW12} (0.60)& \cellcolor{red!40}\href{../works/SchausHMCMD11.pdf}{SchausHMCMD11} (0.67)& \cellcolor{red!20}\href{../works/DannaP03.pdf}{DannaP03} (0.88)& \cellcolor{red!20}\href{../works/LetortBC12.pdf}{LetortBC12} (0.88)\\
Euclid& \cellcolor{red!40}\href{../works/HentenryckM08.pdf}{HentenryckM08} (0.20)& \cellcolor{red!40}\href{../works/HeinzSSW12.pdf}{HeinzSSW12} (0.21)& \cellcolor{yellow!20}\href{../works/SmithBHW96.pdf}{SmithBHW96} (0.27)& \cellcolor{yellow!20}\href{../works/SchausHMCMD11.pdf}{SchausHMCMD11} (0.27)& \cellcolor{green!20}\href{../works/BandaSC11.pdf}{BandaSC11} (0.30)\\
Dot& \cellcolor{red!40}\href{../works/SchausHMCMD11.pdf}{SchausHMCMD11} (64.00)& \cellcolor{red!40}\href{../works/German18.pdf}{German18} (59.00)& \cellcolor{red!40}\href{../works/GaySS14.pdf}{GaySS14} (58.00)& \cellcolor{red!40}\href{../works/Malapert11.pdf}{Malapert11} (55.00)& \cellcolor{red!40}\href{../works/LaborieRSV18.pdf}{LaborieRSV18} (54.00)\\
Cosine& \cellcolor{red!40}\href{../works/HentenryckM08.pdf}{HentenryckM08} (0.79)& \cellcolor{red!40}\href{../works/HeinzSSW12.pdf}{HeinzSSW12} (0.78)& \cellcolor{red!40}\href{../works/SchausHMCMD11.pdf}{SchausHMCMD11} (0.72)& \cellcolor{red!40}\href{../works/SmithBHW96.pdf}{SmithBHW96} (0.59)& \cellcolor{red!40}\href{../works/LiuLH19a.pdf}{LiuLH19a} (0.59)\\
\index{GarridoAO09}\href{../works/GarridoAO09.pdf}{GarridoAO09} R\&C& \cellcolor{red!40}\href{../works/GarridoOS08.pdf}{GarridoOS08} (0.74)& \cellcolor{red!40}\href{../works/ZhangLS12.pdf}{ZhangLS12} (0.83)& \cellcolor{red!40}\href{../works/QuirogaZH05.pdf}{QuirogaZH05} (0.86)& \cellcolor{red!40}\href{../works/Geske05.pdf}{Geske05} (0.86)& \cellcolor{red!20}BaptisteLPN06 (0.87)\\
Euclid& \cellcolor{red!40}\href{../works/GarridoOS08.pdf}{GarridoOS08} (0.21)& \cellcolor{red!20}\href{../works/KovacsV04.pdf}{KovacsV04} (0.26)& \cellcolor{yellow!20}\href{../works/LiuLH19.pdf}{LiuLH19} (0.28)& \cellcolor{yellow!20}\href{../works/GayHLS15.pdf}{GayHLS15} (0.28)& \cellcolor{yellow!20}\href{../works/ChuGNSW13.pdf}{ChuGNSW13} (0.28)\\
Dot& \cellcolor{red!40}\href{../works/Godet21a.pdf}{Godet21a} (106.00)& \cellcolor{red!40}\href{../works/Fahimi16.pdf}{Fahimi16} (105.00)& \cellcolor{red!40}\href{../works/ZarandiASC20.pdf}{ZarandiASC20} (102.00)& \cellcolor{red!40}\href{../works/Malapert11.pdf}{Malapert11} (102.00)& \cellcolor{red!40}\href{../works/Groleaz21.pdf}{Groleaz21} (101.00)\\
Cosine& \cellcolor{red!40}\href{../works/GarridoOS08.pdf}{GarridoOS08} (0.85)& \cellcolor{red!40}\href{../works/KovacsV04.pdf}{KovacsV04} (0.78)& \cellcolor{red!40}\href{../works/JussienL02.pdf}{JussienL02} (0.72)& \cellcolor{red!40}\href{../works/LiuLH19.pdf}{LiuLH19} (0.72)& \cellcolor{red!40}\href{../works/GayHLS15.pdf}{GayHLS15} (0.71)\\
\index{GarridoOS08}\href{../works/GarridoOS08.pdf}{GarridoOS08} R\&C& \cellcolor{red!40}\href{../works/GarridoAO09.pdf}{GarridoAO09} (0.74)& \cellcolor{green!20}\href{../works/ZhangLS12.pdf}{ZhangLS12} (0.96)& \cellcolor{green!20}\href{../works/QuirogaZH05.pdf}{QuirogaZH05} (0.96)& \cellcolor{green!20}\href{../works/Geske05.pdf}{Geske05} (0.96)& \cellcolor{green!20}\href{../works/KovacsV04.pdf}{KovacsV04} (0.96)\\
Euclid& \cellcolor{red!40}\href{../works/GarridoAO09.pdf}{GarridoAO09} (0.21)& \cellcolor{yellow!20}\href{../works/WallaceF00.pdf}{WallaceF00} (0.26)& \cellcolor{yellow!20}\href{../works/LiuLH19.pdf}{LiuLH19} (0.26)& \cellcolor{yellow!20}\href{../works/SmithBHW96.pdf}{SmithBHW96} (0.27)& \cellcolor{yellow!20}\href{../works/BartakS11.pdf}{BartakS11} (0.28)\\
Dot& \cellcolor{red!40}\href{../works/Godet21a.pdf}{Godet21a} (93.00)& \cellcolor{red!40}\href{../works/Fahimi16.pdf}{Fahimi16} (93.00)& \cellcolor{red!40}\href{../works/Malapert11.pdf}{Malapert11} (90.00)& \cellcolor{red!40}\href{../works/Lombardi10.pdf}{Lombardi10} (88.00)& \cellcolor{red!40}\href{../works/Beck99.pdf}{Beck99} (87.00)\\
Cosine& \cellcolor{red!40}\href{../works/GarridoAO09.pdf}{GarridoAO09} (0.85)& \cellcolor{red!40}\href{../works/ZeballosM09.pdf}{ZeballosM09} (0.71)& \cellcolor{red!40}\href{../works/Bartak02.pdf}{Bartak02} (0.71)& \cellcolor{red!40}\href{../works/GetoorOFC97.pdf}{GetoorOFC97} (0.71)& \cellcolor{red!40}\href{../works/LiuLH19.pdf}{LiuLH19} (0.71)\\
\index{GayHLS15}\href{../works/GayHLS15.pdf}{GayHLS15} R\&C& \cellcolor{red!40}\href{../works/VilimLS15.pdf}{VilimLS15} (0.77)& \cellcolor{red!40}\href{../works/GayHS15.pdf}{GayHS15} (0.78)& \cellcolor{red!40}\href{../works/CauwelaertLS15.pdf}{CauwelaertLS15} (0.81)& \cellcolor{red!40}\href{../works/HoundjiSWD14.pdf}{HoundjiSWD14} (0.84)& \cellcolor{red!40}\href{../works/DejemeppeCS15.pdf}{DejemeppeCS15} (0.84)\\
Euclid& \cellcolor{red!20}\href{../works/LombardiM13.pdf}{LombardiM13} (0.24)& \cellcolor{red!20}\href{../works/LombardiM10.pdf}{LombardiM10} (0.25)& \cellcolor{red!20}\href{../works/ChuGNSW13.pdf}{ChuGNSW13} (0.25)& \cellcolor{red!20}\href{../works/FortinZDF05.pdf}{FortinZDF05} (0.26)& \cellcolor{red!20}\href{../works/OddiRC10.pdf}{OddiRC10} (0.26)\\
Dot& \cellcolor{red!40}\href{../works/Godet21a.pdf}{Godet21a} (92.00)& \cellcolor{red!40}\href{../works/Siala15a.pdf}{Siala15a} (92.00)& \cellcolor{red!40}\href{../works/Groleaz21.pdf}{Groleaz21} (91.00)& \cellcolor{red!40}\href{../works/Schutt11.pdf}{Schutt11} (91.00)& \cellcolor{red!40}\href{../works/Caballero19.pdf}{Caballero19} (90.00)\\
Cosine& \cellcolor{red!40}\href{../works/LombardiM10.pdf}{LombardiM10} (0.77)& \cellcolor{red!40}\href{../works/TrojetHL11.pdf}{TrojetHL11} (0.74)& \cellcolor{red!40}\href{../works/CestaOF99.pdf}{CestaOF99} (0.73)& \cellcolor{red!40}\href{../works/KovacsV04.pdf}{KovacsV04} (0.73)& \cellcolor{red!40}\href{../works/ChuGNSW13.pdf}{ChuGNSW13} (0.72)\\
\index{GayHS15}\href{../works/GayHS15.pdf}{GayHS15} R\&C& \cellcolor{red!40}\href{../works/GayHS15a.pdf}{GayHS15a} (0.56)& \cellcolor{red!40}\href{../works/OuelletQ13.pdf}{OuelletQ13} (0.60)& \cellcolor{red!40}\href{../works/LetortBC12.pdf}{LetortBC12} (0.61)& \cellcolor{red!40}\href{../works/OuelletQ18.pdf}{OuelletQ18} (0.72)& \cellcolor{red!40}\href{../works/Tesch16.pdf}{Tesch16} (0.73)\\
Euclid& \cellcolor{red!40}\href{../works/BeldiceanuP07.pdf}{BeldiceanuP07} (0.22)& \cellcolor{red!20}\href{../works/WolfS05.pdf}{WolfS05} (0.26)& \cellcolor{red!20}\href{../works/PoderB08.pdf}{PoderB08} (0.26)& \cellcolor{yellow!20}\href{../works/SimonisH11.pdf}{SimonisH11} (0.28)& \cellcolor{green!20}\href{../works/SimoninAHL15.pdf}{SimoninAHL15} (0.29)\\
Dot& \cellcolor{red!40}\href{../works/Fahimi16.pdf}{Fahimi16} (91.00)& \cellcolor{red!40}\href{../works/Malapert11.pdf}{Malapert11} (88.00)& \cellcolor{red!40}\href{../works/Godet21a.pdf}{Godet21a} (85.00)& \cellcolor{red!40}\href{../works/Groleaz21.pdf}{Groleaz21} (82.00)& \cellcolor{red!40}\href{../works/FahimiOQ18.pdf}{FahimiOQ18} (82.00)\\
Cosine& \cellcolor{red!40}\href{../works/BeldiceanuP07.pdf}{BeldiceanuP07} (0.79)& \cellcolor{red!40}\href{../works/WolfS05.pdf}{WolfS05} (0.72)& \cellcolor{red!40}\href{../works/PoderB08.pdf}{PoderB08} (0.70)& \cellcolor{red!40}\href{../works/SimoninAHL15.pdf}{SimoninAHL15} (0.69)& \cellcolor{red!40}\href{../works/SimonisH11.pdf}{SimonisH11} (0.67)\\
\index{GayHS15a}\href{../works/GayHS15a.pdf}{GayHS15a} R\&C& \cellcolor{red!40}\href{../works/GayHS15.pdf}{GayHS15} (0.56)& \cellcolor{red!40}\href{../works/OuelletQ13.pdf}{OuelletQ13} (0.61)& \cellcolor{red!40}\href{../works/KameugneFGOQ18.pdf}{KameugneFGOQ18} (0.67)& \cellcolor{red!40}\href{../works/Tesch16.pdf}{Tesch16} (0.73)& \cellcolor{red!40}\href{../works/LetortBC12.pdf}{LetortBC12} (0.75)\\
Euclid& \cellcolor{red!40}\href{../works/OuelletQ18.pdf}{OuelletQ18} (0.23)& \cellcolor{red!20}\href{../works/Vilim11.pdf}{Vilim11} (0.25)& \cellcolor{green!20}\href{../works/OuelletQ22.pdf}{OuelletQ22} (0.30)& \cellcolor{green!20}\href{../works/OuelletQ13.pdf}{OuelletQ13} (0.30)& \cellcolor{green!20}\href{../works/SchuttW10.pdf}{SchuttW10} (0.30)\\
Dot& \cellcolor{red!40}\href{../works/Schutt11.pdf}{Schutt11} (132.00)& \cellcolor{red!40}\href{../works/Fahimi16.pdf}{Fahimi16} (130.00)& \cellcolor{red!40}\href{../works/FahimiOQ18.pdf}{FahimiOQ18} (122.00)& \cellcolor{red!40}\href{../works/Kameugne14.pdf}{Kameugne14} (121.00)& \cellcolor{red!40}\href{../works/Malapert11.pdf}{Malapert11} (118.00)\\
Cosine& \cellcolor{red!40}\href{../works/OuelletQ18.pdf}{OuelletQ18} (0.85)& \cellcolor{red!40}\href{../works/Vilim11.pdf}{Vilim11} (0.82)& \cellcolor{red!40}\href{../works/TardivoDFMP23.pdf}{TardivoDFMP23} (0.77)& \cellcolor{red!40}\href{../works/OuelletQ22.pdf}{OuelletQ22} (0.76)& \cellcolor{red!40}\href{../works/OuelletQ13.pdf}{OuelletQ13} (0.76)\\
\index{GaySS14}\href{../works/GaySS14.pdf}{GaySS14} R\&C& \cellcolor{red!40}\href{../works/DejemeppeCS15.pdf}{DejemeppeCS15} (0.79)& \cellcolor{red!40}\href{../works/SchausHMCMD11.pdf}{SchausHMCMD11} (0.83)& \cellcolor{red!40}\href{../works/HoundjiSWD14.pdf}{HoundjiSWD14} (0.83)& \cellcolor{red!40}\href{../works/Vilim09a.pdf}{Vilim09a} (0.83)& \cellcolor{red!40}\href{../works/CauwelaertLS15.pdf}{CauwelaertLS15} (0.84)\\
Euclid& \cellcolor{green!20}\href{../works/CauwelaertDMS16.pdf}{CauwelaertDMS16} (0.30)& \cellcolor{blue!20}\href{../works/PacinoH11.pdf}{PacinoH11} (0.32)& \cellcolor{blue!20}\href{../works/VilimBC04.pdf}{VilimBC04} (0.32)& \cellcolor{blue!20}\href{../works/DavenportKRSH07.pdf}{DavenportKRSH07} (0.32)& \cellcolor{blue!20}\href{../works/PerezGSL23.pdf}{PerezGSL23} (0.32)\\
Dot& \cellcolor{red!40}\href{../works/Dejemeppe16.pdf}{Dejemeppe16} (135.00)& \cellcolor{red!40}\href{../works/Astrand21.pdf}{Astrand21} (130.00)& \cellcolor{red!40}\href{../works/Groleaz21.pdf}{Groleaz21} (128.00)& \cellcolor{red!40}\href{../works/Lombardi10.pdf}{Lombardi10} (124.00)& \cellcolor{red!40}\href{../works/Baptiste02.pdf}{Baptiste02} (124.00)\\
Cosine& \cellcolor{red!40}\href{../works/CauwelaertDMS16.pdf}{CauwelaertDMS16} (0.75)& \cellcolor{red!40}\href{../works/Pralet17.pdf}{Pralet17} (0.73)& \cellcolor{red!40}\href{../works/CauwelaertDS20.pdf}{CauwelaertDS20} (0.71)& \cellcolor{red!40}\href{../works/BartakSR08.pdf}{BartakSR08} (0.71)& \cellcolor{red!40}\href{../works/PacinoH11.pdf}{PacinoH11} (0.71)\\
\index{GedikKBR17}\href{../works/GedikKBR17.pdf}{GedikKBR17} R\&C& \cellcolor{red!40}\href{../works/GedikKEK18.pdf}{GedikKEK18} (0.85)& \cellcolor{yellow!20}\href{../works/CappartS17.pdf}{CappartS17} (0.91)& \cellcolor{yellow!20}\href{../works/MengLZB21.pdf}{MengLZB21} (0.93)& \cellcolor{green!20}GongLMW09 (0.93)& \cellcolor{green!20}CastroGR10 (0.94)\\
Euclid& \cellcolor{green!20}\href{../works/PerezGSL23.pdf}{PerezGSL23} (0.29)& \cellcolor{green!20}\href{../works/abs-2312-13682.pdf}{abs-2312-13682} (0.29)& \cellcolor{green!20}\href{../works/BockmayrP06.pdf}{BockmayrP06} (0.29)& \cellcolor{green!20}\href{../works/AkramNHRSA23.pdf}{AkramNHRSA23} (0.30)& \cellcolor{green!20}\href{../works/MelgarejoLS15.pdf}{MelgarejoLS15} (0.30)\\
Dot& \cellcolor{red!40}\href{../works/Groleaz21.pdf}{Groleaz21} (113.00)& \cellcolor{red!40}\href{../works/IsikYA23.pdf}{IsikYA23} (104.00)& \cellcolor{red!40}\href{../works/LaborieRSV18.pdf}{LaborieRSV18} (101.00)& \cellcolor{red!40}\href{../works/Dejemeppe16.pdf}{Dejemeppe16} (100.00)& \cellcolor{red!40}\href{../works/Lunardi20.pdf}{Lunardi20} (99.00)\\
Cosine& \cellcolor{red!40}\href{../works/GedikKEK18.pdf}{GedikKEK18} (0.74)& \cellcolor{red!40}\href{../works/PerezGSL23.pdf}{PerezGSL23} (0.68)& \cellcolor{red!40}\href{../works/abs-2312-13682.pdf}{abs-2312-13682} (0.68)& \cellcolor{red!40}\href{../works/MelgarejoLS15.pdf}{MelgarejoLS15} (0.68)& \cellcolor{red!40}\href{../works/NishikawaSTT19.pdf}{NishikawaSTT19} (0.67)\\
\index{GedikKEK18}\href{../works/GedikKEK18.pdf}{GedikKEK18} R\&C& \cellcolor{red!40}\href{../works/GedikKBR17.pdf}{GedikKBR17} (0.85)& \cellcolor{red!20}\href{../works/MengZRZL20.pdf}{MengZRZL20} (0.87)& \cellcolor{yellow!20}\href{../works/GomesM17.pdf}{GomesM17} (0.91)& \cellcolor{yellow!20}\href{../works/KelbelH11.pdf}{KelbelH11} (0.91)& \cellcolor{yellow!20}\href{../works/QinDCS20.pdf}{QinDCS20} (0.91)\\
Euclid& \cellcolor{green!20}\href{../works/TranAB16.pdf}{TranAB16} (0.31)& \cellcolor{green!20}\href{../works/ArbaouiY18.pdf}{ArbaouiY18} (0.31)& \cellcolor{blue!20}\href{../works/TranB12.pdf}{TranB12} (0.32)& \cellcolor{blue!20}\href{../works/abs-2305-19888.pdf}{abs-2305-19888} (0.34)& \cellcolor{black!20}\href{../works/GedikKBR17.pdf}{GedikKBR17} (0.34)\\
Dot& \cellcolor{red!40}\href{../works/Groleaz21.pdf}{Groleaz21} (189.00)& \cellcolor{red!40}\href{../works/ZarandiASC20.pdf}{ZarandiASC20} (171.00)& \cellcolor{red!40}\href{../works/IsikYA23.pdf}{IsikYA23} (166.00)& \cellcolor{red!40}\href{../works/Lunardi20.pdf}{Lunardi20} (163.00)& \cellcolor{red!40}\href{../works/NaderiRR23.pdf}{NaderiRR23} (163.00)\\
Cosine& \cellcolor{red!40}\href{../works/TranAB16.pdf}{TranAB16} (0.82)& \cellcolor{red!40}\href{../works/TranB12.pdf}{TranB12} (0.80)& \cellcolor{red!40}\href{../works/ArbaouiY18.pdf}{ArbaouiY18} (0.79)& \cellcolor{red!40}\href{../works/abs-2305-19888.pdf}{abs-2305-19888} (0.77)& \cellcolor{red!40}\href{../works/YunusogluY22.pdf}{YunusogluY22} (0.75)\\
\index{GeibingerKKMMW21}\href{../works/GeibingerKKMMW21.pdf}{GeibingerKKMMW21} R\&C& \cellcolor{yellow!20}\href{../works/Simonis07.pdf}{Simonis07} (0.91)& \cellcolor{green!20}\href{../works/NishikawaSTT18a.pdf}{NishikawaSTT18a} (0.93)& \cellcolor{green!20}\href{../works/Wolf11.pdf}{Wolf11} (0.96)& \cellcolor{blue!20}\href{../works/NovasH14.pdf}{NovasH14} (0.97)& \cellcolor{blue!20}Hooker10 (0.98)\\
Euclid& \cellcolor{green!20}\href{../works/BofillCGGPSV23.pdf}{BofillCGGPSV23} (0.29)& \cellcolor{green!20}\href{../works/Baptiste09.pdf}{Baptiste09} (0.29)& \cellcolor{green!20}\href{../works/BofillGSV15.pdf}{BofillGSV15} (0.30)& \cellcolor{green!20}\href{../works/ZibranR11.pdf}{ZibranR11} (0.30)& \cellcolor{green!20}\href{../works/CarchraeBF05.pdf}{CarchraeBF05} (0.31)\\
Dot& \cellcolor{red!40}\href{../works/RoshanaeiLAU17.pdf}{RoshanaeiLAU17} (54.00)& \cellcolor{red!40}\href{../works/Dejemeppe16.pdf}{Dejemeppe16} (53.00)& \cellcolor{red!40}\href{../works/KoehlerBFFHPSSS21.pdf}{KoehlerBFFHPSSS21} (52.00)& \cellcolor{red!40}\href{../works/GuoZ23.pdf}{GuoZ23} (52.00)& \cellcolor{red!40}\href{../works/Lemos21.pdf}{Lemos21} (50.00)\\
Cosine& \cellcolor{red!40}\href{../works/FrohnerTR19.pdf}{FrohnerTR19} (0.59)& \cellcolor{red!40}\href{../works/BourdaisGP03.pdf}{BourdaisGP03} (0.52)& \cellcolor{red!40}\href{../works/BofillCGGPSV23.pdf}{BofillCGGPSV23} (0.52)& \cellcolor{red!40}\href{../works/HechingH16.pdf}{HechingH16} (0.50)& \cellcolor{red!40}\href{../works/BofillGSV15.pdf}{BofillGSV15} (0.50)\\
\index{GeibingerMM19}\href{../works/GeibingerMM19.pdf}{GeibingerMM19} R\&C& \cellcolor{red!40}\href{../works/YoungFS17.pdf}{YoungFS17} (0.75)& \cellcolor{red!40}\href{../works/FrimodigS19.pdf}{FrimodigS19} (0.78)& \cellcolor{red!40}\href{../works/SzerediS16.pdf}{SzerediS16} (0.80)& \cellcolor{red!40}\href{../works/BhatnagarKL19.pdf}{BhatnagarKL19} (0.86)& \cellcolor{red!20}\href{../works/BofillCSV17.pdf}{BofillCSV17} (0.89)\\
Euclid& \cellcolor{red!40}\href{../works/abs-1911-04766.pdf}{abs-1911-04766} (0.23)& \cellcolor{blue!20}\href{../works/CampeauG22.pdf}{CampeauG22} (0.32)& \cellcolor{black!20}\href{../works/BhatnagarKL19.pdf}{BhatnagarKL19} (0.34)& \cellcolor{black!20}\href{../works/GeibingerMM21.pdf}{GeibingerMM21} (0.34)& \cellcolor{black!20}\href{../works/HeipckeCCS00.pdf}{HeipckeCCS00} (0.34)\\
Dot& \cellcolor{red!40}\href{../works/abs-1911-04766.pdf}{abs-1911-04766} (168.00)& \cellcolor{red!40}\href{../works/Groleaz21.pdf}{Groleaz21} (151.00)& \cellcolor{red!40}\href{../works/LaborieRSV18.pdf}{LaborieRSV18} (148.00)& \cellcolor{red!40}\href{../works/Dejemeppe16.pdf}{Dejemeppe16} (145.00)& \cellcolor{red!40}\href{../works/Lombardi10.pdf}{Lombardi10} (142.00)\\
Cosine& \cellcolor{red!40}\href{../works/abs-1911-04766.pdf}{abs-1911-04766} (0.91)& \cellcolor{red!40}\href{../works/CampeauG22.pdf}{CampeauG22} (0.76)& \cellcolor{red!40}\href{../works/GeibingerMM21.pdf}{GeibingerMM21} (0.73)& \cellcolor{red!40}\href{../works/BhatnagarKL19.pdf}{BhatnagarKL19} (0.72)& \cellcolor{red!40}\href{../works/HeipckeCCS00.pdf}{HeipckeCCS00} (0.71)\\
\index{GeibingerMM21}\href{../works/GeibingerMM21.pdf}{GeibingerMM21} R\&C\\
Euclid& \cellcolor{green!20}\href{../works/HeipckeCCS00.pdf}{HeipckeCCS00} (0.30)& \cellcolor{blue!20}\href{../works/abs-1911-04766.pdf}{abs-1911-04766} (0.32)& \cellcolor{blue!20}\href{../works/Laborie18a.pdf}{Laborie18a} (0.33)& \cellcolor{black!20}\href{../works/GeibingerMM19.pdf}{GeibingerMM19} (0.34)& \cellcolor{black!20}\href{../works/BhatnagarKL19.pdf}{BhatnagarKL19} (0.35)\\
Dot& \cellcolor{red!40}\href{../works/abs-1911-04766.pdf}{abs-1911-04766} (138.00)& \cellcolor{red!40}\href{../works/Lombardi10.pdf}{Lombardi10} (135.00)& \cellcolor{red!40}\href{../works/Dejemeppe16.pdf}{Dejemeppe16} (134.00)& \cellcolor{red!40}\href{../works/ZarandiASC20.pdf}{ZarandiASC20} (131.00)& \cellcolor{red!40}\href{../works/Groleaz21.pdf}{Groleaz21} (131.00)\\
Cosine& \cellcolor{red!40}\href{../works/abs-1911-04766.pdf}{abs-1911-04766} (0.81)& \cellcolor{red!40}\href{../works/HeipckeCCS00.pdf}{HeipckeCCS00} (0.75)& \cellcolor{red!40}\href{../works/GeibingerMM19.pdf}{GeibingerMM19} (0.73)& \cellcolor{red!40}\href{../works/PovedaAA23.pdf}{PovedaAA23} (0.72)& \cellcolor{red!40}\href{../works/VilimLS15.pdf}{VilimLS15} (0.70)\\
\index{GeitzGSSW22}\href{../works/GeitzGSSW22.pdf}{GeitzGSSW22} R\&C& \cellcolor{red!20}\href{../works/FahimiOQ18.pdf}{FahimiOQ18} (0.86)& \cellcolor{red!20}\href{../works/WolfS05.pdf}{WolfS05} (0.87)& \cellcolor{red!20}\href{../works/DejemeppeCS15.pdf}{DejemeppeCS15} (0.89)& \cellcolor{red!20}\href{../works/Wolf09.pdf}{Wolf09} (0.89)& \cellcolor{red!20}\href{../works/WolfS05a.pdf}{WolfS05a} (0.90)\\
Euclid& \cellcolor{blue!20}\href{../works/KovacsV06.pdf}{KovacsV06} (0.32)& \cellcolor{blue!20}\href{../works/CauwelaertDMS16.pdf}{CauwelaertDMS16} (0.32)& \cellcolor{blue!20}\href{../works/NuijtenP98.pdf}{NuijtenP98} (0.33)& \cellcolor{blue!20}\href{../works/KhayatLR06.pdf}{KhayatLR06} (0.33)& \cellcolor{blue!20}\href{../works/PengLC14.pdf}{PengLC14} (0.33)\\
Dot& \cellcolor{red!40}\href{../works/ZarandiASC20.pdf}{ZarandiASC20} (165.00)& \cellcolor{red!40}\href{../works/Baptiste02.pdf}{Baptiste02} (152.00)& \cellcolor{red!40}\href{../works/Groleaz21.pdf}{Groleaz21} (150.00)& \cellcolor{red!40}\href{../works/Dejemeppe16.pdf}{Dejemeppe16} (149.00)& \cellcolor{red!40}\href{../works/Malapert11.pdf}{Malapert11} (143.00)\\
Cosine& \cellcolor{red!40}\href{../works/NuijtenP98.pdf}{NuijtenP98} (0.76)& \cellcolor{red!40}\href{../works/KovacsV06.pdf}{KovacsV06} (0.74)& \cellcolor{red!40}\href{../works/TorresL00.pdf}{TorresL00} (0.74)& \cellcolor{red!40}\href{../works/PengLC14.pdf}{PengLC14} (0.74)& \cellcolor{red!40}\href{../works/CauwelaertDMS16.pdf}{CauwelaertDMS16} (0.73)\\
\index{GelainPRVW17}\href{../works/GelainPRVW17.pdf}{GelainPRVW17} R\&C& \cellcolor{red!40}\href{../works/LimBTBB15.pdf}{LimBTBB15} (0.80)& \cellcolor{black!20}\href{../works/BartakSR10.pdf}{BartakSR10} (0.98)\\
Euclid& \cellcolor{red!40}\href{../works/LiuLH19.pdf}{LiuLH19} (0.20)& \cellcolor{red!40}\href{../works/FeldmanG89.pdf}{FeldmanG89} (0.20)& \cellcolor{red!40}\href{../works/ZhangLS12.pdf}{ZhangLS12} (0.21)& \cellcolor{red!40}\href{../works/BandaSC11.pdf}{BandaSC11} (0.21)& \cellcolor{red!40}\href{../works/DilkinaH04.pdf}{DilkinaH04} (0.22)\\
Dot& \cellcolor{red!40}\href{../works/Lemos21.pdf}{Lemos21} (58.00)& \cellcolor{red!40}\href{../works/Siala15a.pdf}{Siala15a} (55.00)& \cellcolor{red!40}\href{../works/BartakSR10.pdf}{BartakSR10} (55.00)& \cellcolor{red!40}\href{../works/Baptiste02.pdf}{Baptiste02} (55.00)& \cellcolor{red!40}\href{../works/Godet21a.pdf}{Godet21a} (54.00)\\
Cosine& \cellcolor{red!40}\href{../works/LiuLH19.pdf}{LiuLH19} (0.80)& \cellcolor{red!40}\href{../works/ZhangLS12.pdf}{ZhangLS12} (0.73)& \cellcolor{red!40}\href{../works/BartakS11.pdf}{BartakS11} (0.70)& \cellcolor{red!40}\href{../works/ZhuS02.pdf}{ZhuS02} (0.69)& \cellcolor{red!40}\href{../works/BandaSC11.pdf}{BandaSC11} (0.68)\\
\index{German18}\href{../works/German18.pdf}{German18} R\&C\\
Euclid& \cellcolor{black!20}\href{../works/LiuLH19.pdf}{LiuLH19} (0.37)& \cellcolor{black!20}\href{../works/MalapertCGJLR13.pdf}{MalapertCGJLR13} (0.37)& \cellcolor{black!20}\href{../works/TanSD10.pdf}{TanSD10} (0.37)& \href{../works/Bartak02a.pdf}{Bartak02a} (0.37)& \href{../works/SakkoutW00.pdf}{SakkoutW00} (0.38)\\
Dot& \cellcolor{red!40}\href{../works/Malapert11.pdf}{Malapert11} (149.00)& \cellcolor{red!40}\href{../works/Godet21a.pdf}{Godet21a} (145.00)& \cellcolor{red!40}\href{../works/Siala15a.pdf}{Siala15a} (144.00)& \cellcolor{red!40}\href{../works/Groleaz21.pdf}{Groleaz21} (144.00)& \cellcolor{red!40}\href{../works/Baptiste02.pdf}{Baptiste02} (144.00)\\
Cosine& \cellcolor{red!40}\href{../works/SakkoutW00.pdf}{SakkoutW00} (0.69)& \cellcolor{red!40}\href{../works/VilimLS15.pdf}{VilimLS15} (0.68)& \cellcolor{red!40}\href{../works/BartakSR08.pdf}{BartakSR08} (0.67)& \cellcolor{red!40}\href{../works/TanSD10.pdf}{TanSD10} (0.67)& \cellcolor{red!40}\href{../works/GrimesH10.pdf}{GrimesH10} (0.67)\\
\index{Geske05}\href{../works/Geske05.pdf}{Geske05} R\&C& \cellcolor{red!40}\href{../works/ZhangLS12.pdf}{ZhangLS12} (0.67)& \cellcolor{red!40}AggounV04 (0.75)& \cellcolor{red!40}\href{../works/QuirogaZH05.pdf}{QuirogaZH05} (0.75)& \cellcolor{red!40}\href{../works/SimonisCK00.pdf}{SimonisCK00} (0.75)& \cellcolor{red!40}\href{../works/KovacsV04.pdf}{KovacsV04} (0.80)\\
Euclid& \cellcolor{green!20}\href{../works/BridiLBBM16.pdf}{BridiLBBM16} (0.31)& \cellcolor{blue!20}\href{../works/WolfS05.pdf}{WolfS05} (0.32)& \cellcolor{blue!20}\href{../works/BockmayrP06.pdf}{BockmayrP06} (0.32)& \cellcolor{blue!20}\href{../works/Madi-WambaLOBM17.pdf}{Madi-WambaLOBM17} (0.33)& \cellcolor{blue!20}\href{../works/CappartS17.pdf}{CappartS17} (0.33)\\
Dot& \cellcolor{red!40}\href{../works/ZarandiASC20.pdf}{ZarandiASC20} (114.00)& \cellcolor{red!40}\href{../works/Baptiste02.pdf}{Baptiste02} (111.00)& \cellcolor{red!40}\href{../works/Beck99.pdf}{Beck99} (107.00)& \cellcolor{red!40}\href{../works/Malapert11.pdf}{Malapert11} (105.00)& \cellcolor{red!40}\href{../works/BartakSR10.pdf}{BartakSR10} (102.00)\\
Cosine& \cellcolor{red!40}\href{../works/Madi-WambaLOBM17.pdf}{Madi-WambaLOBM17} (0.70)& \cellcolor{red!40}\href{../works/CappartS17.pdf}{CappartS17} (0.69)& \cellcolor{red!40}\href{../works/WuBB09.pdf}{WuBB09} (0.68)& \cellcolor{red!40}\href{../works/LammaMM97.pdf}{LammaMM97} (0.66)& \cellcolor{red!40}\href{../works/BridiLBBM16.pdf}{BridiLBBM16} (0.66)\\
\index{GetoorOFC97}\href{../works/GetoorOFC97.pdf}{GetoorOFC97} R\&C\\
Euclid& \cellcolor{yellow!20}\href{../works/KengY89.pdf}{KengY89} (0.27)& \cellcolor{yellow!20}\href{../works/CrawfordB94.pdf}{CrawfordB94} (0.27)& \cellcolor{yellow!20}\href{../works/KovacsV04.pdf}{KovacsV04} (0.27)& \cellcolor{yellow!20}\href{../works/Prosser89.pdf}{Prosser89} (0.27)& \cellcolor{yellow!20}\href{../works/DoRZ08.pdf}{DoRZ08} (0.27)\\
Dot& \cellcolor{red!40}\href{../works/ZarandiASC20.pdf}{ZarandiASC20} (100.00)& \cellcolor{red!40}\href{../works/Godet21a.pdf}{Godet21a} (97.00)& \cellcolor{red!40}\href{../works/Groleaz21.pdf}{Groleaz21} (97.00)& \cellcolor{red!40}\href{../works/Dejemeppe16.pdf}{Dejemeppe16} (97.00)& \cellcolor{red!40}\href{../works/Beck99.pdf}{Beck99} (97.00)\\
Cosine& \cellcolor{red!40}\href{../works/KovacsV04.pdf}{KovacsV04} (0.74)& \cellcolor{red!40}\href{../works/BeckPS03.pdf}{BeckPS03} (0.73)& \cellcolor{red!40}\href{../works/KengY89.pdf}{KengY89} (0.72)& \cellcolor{red!40}\href{../works/Bartak02a.pdf}{Bartak02a} (0.72)& \cellcolor{red!40}\href{../works/Prosser89.pdf}{Prosser89} (0.71)\\
\index{GhandehariK22}\href{../works/GhandehariK22.pdf}{GhandehariK22} R\&C& \cellcolor{red!40}\href{../works/RoshanaeiBAUB20.pdf}{RoshanaeiBAUB20} (0.85)& \cellcolor{red!40}\href{../works/FarsiTM22.pdf}{FarsiTM22} (0.86)& \cellcolor{red!40}\href{../works/YounespourAKE19.pdf}{YounespourAKE19} (0.86)& \cellcolor{red!20}\href{../works/ZhaoL14.pdf}{ZhaoL14} (0.87)& \cellcolor{red!20}\href{../works/MengLZB21.pdf}{MengLZB21} (0.88)\\
Euclid& \cellcolor{red!20}\href{../works/DoulabiRP16.pdf}{DoulabiRP16} (0.25)& \cellcolor{yellow!20}\href{../works/GurEA19.pdf}{GurEA19} (0.28)& \cellcolor{yellow!20}\href{../works/DoulabiRP14.pdf}{DoulabiRP14} (0.28)& \cellcolor{green!20}\href{../works/ZhaoL14.pdf}{ZhaoL14} (0.29)& \cellcolor{green!20}\href{../works/GurPAE23.pdf}{GurPAE23} (0.30)\\
Dot& \cellcolor{red!40}\href{../works/ZarandiASC20.pdf}{ZarandiASC20} (126.00)& \cellcolor{red!40}\href{../works/Dejemeppe16.pdf}{Dejemeppe16} (113.00)& \cellcolor{red!40}\href{../works/Groleaz21.pdf}{Groleaz21} (112.00)& \cellcolor{red!40}\href{../works/RoshanaeiBAUB20.pdf}{RoshanaeiBAUB20} (104.00)& \cellcolor{red!40}\href{../works/Lunardi20.pdf}{Lunardi20} (101.00)\\
Cosine& \cellcolor{red!40}\href{../works/DoulabiRP16.pdf}{DoulabiRP16} (0.81)& \cellcolor{red!40}\href{../works/ZhaoL14.pdf}{ZhaoL14} (0.78)& \cellcolor{red!40}\href{../works/GurEA19.pdf}{GurEA19} (0.76)& \cellcolor{red!40}\href{../works/RoshanaeiBAUB20.pdf}{RoshanaeiBAUB20} (0.76)& \cellcolor{red!40}\href{../works/RiiseML16.pdf}{RiiseML16} (0.75)\\
\index{GhasemiMH23}GhasemiMH23 R\&C& \cellcolor{yellow!20}\href{../works/GhandehariK22.pdf}{GhandehariK22} (0.91)& \cellcolor{yellow!20}\href{../works/GurPAE23.pdf}{GurPAE23} (0.92)& \cellcolor{yellow!20}\href{../works/RoshanaeiBAUB20.pdf}{RoshanaeiBAUB20} (0.93)& \cellcolor{yellow!20}\href{../works/FarsiTM22.pdf}{FarsiTM22} (0.93)& \cellcolor{yellow!20}NaderiRBAU21 (0.93)\\
Euclid\\
Dot\\
Cosine\\
\index{GilesH16}\href{../works/GilesH16.pdf}{GilesH16} R\&C& \cellcolor{red!40}\href{../works/CappartS17.pdf}{CappartS17} (0.72)& \cellcolor{red!40}\href{../works/BartoliniBBLM14.pdf}{BartoliniBBLM14} (0.78)& \cellcolor{red!40}\href{../works/FrankDT16.pdf}{FrankDT16} (0.83)& \cellcolor{red!20}\href{../works/Davenport10.pdf}{Davenport10} (0.88)& \cellcolor{red!20}\href{../works/Limtanyakul07.pdf}{Limtanyakul07} (0.89)\\
Euclid& \cellcolor{red!40}\href{../works/BeniniBGM05a.pdf}{BeniniBGM05a} (0.22)& \cellcolor{red!40}\href{../works/TranVNB17a.pdf}{TranVNB17a} (0.23)& \cellcolor{red!20}\href{../works/ZibranR11a.pdf}{ZibranR11a} (0.25)& \cellcolor{red!20}\href{../works/WallaceF00.pdf}{WallaceF00} (0.25)& \cellcolor{red!20}\href{../works/QuSN06.pdf}{QuSN06} (0.25)\\
Dot& \cellcolor{red!40}\href{../works/Astrand21.pdf}{Astrand21} (81.00)& \cellcolor{red!40}\href{../works/LaborieRSV18.pdf}{LaborieRSV18} (79.00)& \cellcolor{red!40}\href{../works/Beck99.pdf}{Beck99} (79.00)& \cellcolor{red!40}\href{../works/GoelSHFS15.pdf}{GoelSHFS15} (78.00)& \cellcolor{red!40}\href{../works/HarjunkoskiMBC14.pdf}{HarjunkoskiMBC14} (78.00)\\
Cosine& \cellcolor{red!40}\href{../works/GoelSHFS15.pdf}{GoelSHFS15} (0.78)& \cellcolor{red!40}\href{../works/BeniniBGM05a.pdf}{BeniniBGM05a} (0.76)& \cellcolor{red!40}\href{../works/TranVNB17a.pdf}{TranVNB17a} (0.74)& \cellcolor{red!40}\href{../works/MaraveliasCG04.pdf}{MaraveliasCG04} (0.74)& \cellcolor{red!40}\href{../works/BoothNB16.pdf}{BoothNB16} (0.72)\\
\index{GingrasQ16}\href{../works/GingrasQ16.pdf}{GingrasQ16} R\&C\\
Euclid& \cellcolor{red!40}\href{../works/OuelletQ13.pdf}{OuelletQ13} (0.22)& \cellcolor{yellow!20}\href{../works/KameugneFND23.pdf}{KameugneFND23} (0.27)& \cellcolor{yellow!20}\href{../works/KameugneFSN14.pdf}{KameugneFSN14} (0.27)& \cellcolor{yellow!20}\href{../works/KameugneFSN11.pdf}{KameugneFSN11} (0.28)& \cellcolor{green!20}\href{../works/WolfS05.pdf}{WolfS05} (0.29)\\
Dot& \cellcolor{red!40}\href{../works/Schutt11.pdf}{Schutt11} (115.00)& \cellcolor{red!40}\href{../works/FetgoD22.pdf}{FetgoD22} (114.00)& \cellcolor{red!40}\href{../works/KameugneFND23.pdf}{KameugneFND23} (113.00)& \cellcolor{red!40}\href{../works/Lombardi10.pdf}{Lombardi10} (111.00)& \cellcolor{red!40}\href{../works/Fahimi16.pdf}{Fahimi16} (111.00)\\
Cosine& \cellcolor{red!40}\href{../works/OuelletQ13.pdf}{OuelletQ13} (0.86)& \cellcolor{red!40}\href{../works/KameugneFND23.pdf}{KameugneFND23} (0.84)& \cellcolor{red!40}\href{../works/KameugneFSN14.pdf}{KameugneFSN14} (0.82)& \cellcolor{red!40}\href{../works/FetgoD22.pdf}{FetgoD22} (0.81)& \cellcolor{red!40}\href{../works/KameugneFSN11.pdf}{KameugneFSN11} (0.76)\\
\index{GlobusCLP04}\href{../works/GlobusCLP04.pdf}{GlobusCLP04} R\&C\\
Euclid& \cellcolor{red!20}\href{../works/BarbulescuWH04.pdf}{BarbulescuWH04} (0.25)& \cellcolor{green!20}\href{../works/Maillard15.pdf}{Maillard15} (0.29)& \cellcolor{green!20}\href{../works/VerfaillieL01.pdf}{VerfaillieL01} (0.30)& \cellcolor{green!20}\href{../works/CrawfordB94.pdf}{CrawfordB94} (0.30)& \cellcolor{green!20}\href{../works/LudwigKRBMS14.pdf}{LudwigKRBMS14} (0.31)\\
Dot& \cellcolor{red!40}\href{../works/ZarandiASC20.pdf}{ZarandiASC20} (96.00)& \cellcolor{red!40}\href{../works/Lunardi20.pdf}{Lunardi20} (89.00)& \cellcolor{red!40}\href{../works/Astrand21.pdf}{Astrand21} (87.00)& \cellcolor{red!40}\href{../works/Froger16.pdf}{Froger16} (80.00)& \cellcolor{red!40}\href{../works/Groleaz21.pdf}{Groleaz21} (76.00)\\
Cosine& \cellcolor{red!40}\href{../works/BarbulescuWH04.pdf}{BarbulescuWH04} (0.76)& \cellcolor{red!40}\href{../works/VerfaillieL01.pdf}{VerfaillieL01} (0.65)& \cellcolor{red!40}\href{../works/Maillard15.pdf}{Maillard15} (0.62)& \cellcolor{red!40}\href{../works/CilKLO22.pdf}{CilKLO22} (0.60)& \cellcolor{red!40}\href{../works/CrawfordB94.pdf}{CrawfordB94} (0.60)\\
\index{GodardLN05}\href{../works/GodardLN05.pdf}{GodardLN05} R\&C\\
Euclid& \cellcolor{red!40}\href{../works/HentenryckM04.pdf}{HentenryckM04} (0.23)& \cellcolor{red!20}\href{../works/PacinoH11.pdf}{PacinoH11} (0.25)& \cellcolor{yellow!20}\href{../works/VilimBC04.pdf}{VilimBC04} (0.27)& \cellcolor{yellow!20}\href{../works/CarchraeB09.pdf}{CarchraeB09} (0.27)& \cellcolor{yellow!20}\href{../works/CestaOS00.pdf}{CestaOS00} (0.27)\\
Dot& \cellcolor{red!40}\href{../works/Baptiste02.pdf}{Baptiste02} (132.00)& \cellcolor{red!40}\href{../works/Dejemeppe16.pdf}{Dejemeppe16} (126.00)& \cellcolor{red!40}\href{../works/LaborieRSV18.pdf}{LaborieRSV18} (124.00)& \cellcolor{red!40}\href{../works/Lombardi10.pdf}{Lombardi10} (122.00)& \cellcolor{red!40}\href{../works/Godet21a.pdf}{Godet21a} (118.00)\\
Cosine& \cellcolor{red!40}\href{../works/HentenryckM04.pdf}{HentenryckM04} (0.85)& \cellcolor{red!40}\href{../works/Pralet17.pdf}{Pralet17} (0.81)& \cellcolor{red!40}\href{../works/PacinoH11.pdf}{PacinoH11} (0.81)& \cellcolor{red!40}\href{../works/CarchraeB09.pdf}{CarchraeB09} (0.79)& \cellcolor{red!40}\href{../works/VilimBC04.pdf}{VilimBC04} (0.78)\\
\index{Godet21a}\href{../works/Godet21a.pdf}{Godet21a} R\&C\\
Euclid& \href{../works/GodetLHS20.pdf}{GodetLHS20} (0.53)& \href{../works/BoudreaultSLQ22.pdf}{BoudreaultSLQ22} (0.56)& \href{../works/Elkhyari03.pdf}{Elkhyari03} (0.56)& \href{../works/Caballero19.pdf}{Caballero19} (0.57)& \href{../works/ChenGPSH10.pdf}{ChenGPSH10} (0.57)\\
Dot& \cellcolor{red!40}\href{../works/Baptiste02.pdf}{Baptiste02} (267.00)& \cellcolor{red!40}\href{../works/Groleaz21.pdf}{Groleaz21} (263.00)& \cellcolor{red!40}\href{../works/Malapert11.pdf}{Malapert11} (262.00)& \cellcolor{red!40}\href{../works/Dejemeppe16.pdf}{Dejemeppe16} (262.00)& \cellcolor{red!40}\href{../works/ZarandiASC20.pdf}{ZarandiASC20} (256.00)\\
Cosine& \cellcolor{red!40}\href{../works/GodetLHS20.pdf}{GodetLHS20} (0.71)& \cellcolor{red!40}\href{../works/Caballero19.pdf}{Caballero19} (0.67)& \cellcolor{red!40}\href{../works/Elkhyari03.pdf}{Elkhyari03} (0.67)& \cellcolor{red!40}\href{../works/BoudreaultSLQ22.pdf}{BoudreaultSLQ22} (0.67)& \cellcolor{red!40}\href{../works/Fahimi16.pdf}{Fahimi16} (0.66)\\
\index{GodetLHS20}\href{../works/GodetLHS20.pdf}{GodetLHS20} R\&C& \cellcolor{yellow!20}\href{../works/HookerH17.pdf}{HookerH17} (0.92)\\
Euclid& \cellcolor{black!20}\href{../works/HebrardHJMPV16.pdf}{HebrardHJMPV16} (0.37)& \href{../works/SialaAH15.pdf}{SialaAH15} (0.38)& \href{../works/TanSD10.pdf}{TanSD10} (0.38)& \href{../works/HeipckeCCS00.pdf}{HeipckeCCS00} (0.38)& \href{../works/LahimerLH11.pdf}{LahimerLH11} (0.38)\\
Dot& \cellcolor{red!40}\href{../works/Godet21a.pdf}{Godet21a} (188.00)& \cellcolor{red!40}\href{../works/Baptiste02.pdf}{Baptiste02} (148.00)& \cellcolor{red!40}\href{../works/Malapert11.pdf}{Malapert11} (142.00)& \cellcolor{red!40}\href{../works/Groleaz21.pdf}{Groleaz21} (142.00)& \cellcolor{red!40}\href{../works/NaderiRR23.pdf}{NaderiRR23} (137.00)\\
Cosine& \cellcolor{red!40}\href{../works/Godet21a.pdf}{Godet21a} (0.71)& \cellcolor{red!40}\href{../works/HebrardHJMPV16.pdf}{HebrardHJMPV16} (0.69)& \cellcolor{red!40}\href{../works/SialaAH15.pdf}{SialaAH15} (0.68)& \cellcolor{red!40}\href{../works/GedikKEK18.pdf}{GedikKEK18} (0.68)& \cellcolor{red!40}\href{../works/GrimesH10.pdf}{GrimesH10} (0.68)\\
\index{GoelSHFS15}\href{../works/GoelSHFS15.pdf}{GoelSHFS15} R\&C& \cellcolor{red!20}\href{../works/GilesH16.pdf}{GilesH16} (0.90)& \cellcolor{green!20}\href{../works/UnsalO13.pdf}{UnsalO13} (0.96)& \cellcolor{green!20}\href{../works/BoothNB16.pdf}{BoothNB16} (0.96)& \cellcolor{green!20}\href{../works/KreterSSZ18.pdf}{KreterSSZ18} (0.96)& \cellcolor{blue!20}\href{../works/OzturkTHO12.pdf}{OzturkTHO12} (0.96)\\
Euclid& \cellcolor{yellow!20}\href{../works/GilesH16.pdf}{GilesH16} (0.26)& \cellcolor{green!20}\href{../works/BoothNB16.pdf}{BoothNB16} (0.31)& \cellcolor{blue!20}\href{../works/SimoninAHL15.pdf}{SimoninAHL15} (0.32)& \cellcolor{blue!20}\href{../works/WolfS05.pdf}{WolfS05} (0.33)& \cellcolor{blue!20}\href{../works/CappartTSR18.pdf}{CappartTSR18} (0.33)\\
Dot& \cellcolor{red!40}\href{../works/LaborieRSV18.pdf}{LaborieRSV18} (126.00)& \cellcolor{red!40}\href{../works/Groleaz21.pdf}{Groleaz21} (110.00)& \cellcolor{red!40}\href{../works/Astrand21.pdf}{Astrand21} (104.00)& \cellcolor{red!40}\href{../works/Lunardi20.pdf}{Lunardi20} (101.00)& \cellcolor{red!40}\href{../works/Malapert11.pdf}{Malapert11} (99.00)\\
Cosine& \cellcolor{red!40}\href{../works/GilesH16.pdf}{GilesH16} (0.78)& \cellcolor{red!40}\href{../works/CappartTSR18.pdf}{CappartTSR18} (0.71)& \cellcolor{red!40}\href{../works/BoothNB16.pdf}{BoothNB16} (0.70)& \cellcolor{red!40}\href{../works/SerraNM12.pdf}{SerraNM12} (0.69)& \cellcolor{red!40}\href{../works/TranVNB17.pdf}{TranVNB17} (0.68)\\
\index{GokGSTO20}\href{../works/GokGSTO20.pdf}{GokGSTO20} R\&C& \cellcolor{red!20}\href{../works/MusliuSS18.pdf}{MusliuSS18} (0.86)& \cellcolor{green!20}\href{../works/AmadiniGM16.pdf}{AmadiniGM16} (0.93)& \cellcolor{green!20}\href{../works/ElkhyariGJ02.pdf}{ElkhyariGJ02} (0.95)& \cellcolor{green!20}\href{../works/FrohnerTR19.pdf}{FrohnerTR19} (0.95)& \cellcolor{green!20}\href{../works/HebrardHJMPV16.pdf}{HebrardHJMPV16} (0.95)\\
Euclid& \cellcolor{black!20}\href{../works/TranDRFWOVB16.pdf}{TranDRFWOVB16} (0.35)& \cellcolor{black!20}\href{../works/KusterJF07.pdf}{KusterJF07} (0.36)& \cellcolor{black!20}\href{../works/LombardiM13.pdf}{LombardiM13} (0.36)& \cellcolor{black!20}\href{../works/GokPTGO23.pdf}{GokPTGO23} (0.37)& \cellcolor{black!20}\href{../works/BhatnagarKL19.pdf}{BhatnagarKL19} (0.37)\\
Dot& \cellcolor{red!40}\href{../works/Groleaz21.pdf}{Groleaz21} (121.00)& \cellcolor{red!40}\href{../works/ZarandiASC20.pdf}{ZarandiASC20} (120.00)& \cellcolor{red!40}\href{../works/Lombardi10.pdf}{Lombardi10} (115.00)& \cellcolor{red!40}\href{../works/LaborieRSV18.pdf}{LaborieRSV18} (114.00)& \cellcolor{red!40}\href{../works/GokPTGO23.pdf}{GokPTGO23} (114.00)\\
Cosine& \cellcolor{red!40}\href{../works/GokPTGO23.pdf}{GokPTGO23} (0.72)& \cellcolor{red!40}\href{../works/KusterJF07.pdf}{KusterJF07} (0.65)& \cellcolor{red!40}\href{../works/TranDRFWOVB16.pdf}{TranDRFWOVB16} (0.64)& \cellcolor{red!40}\href{../works/WangB23.pdf}{WangB23} (0.60)& \cellcolor{red!40}\href{../works/LombardiM10.pdf}{LombardiM10} (0.60)\\
\index{GokPTGO23}\href{../works/GokPTGO23.pdf}{GokPTGO23} R\&C\\
Euclid& \cellcolor{black!20}\href{../works/BeckPS03.pdf}{BeckPS03} (0.37)& \cellcolor{black!20}\href{../works/GokGSTO20.pdf}{GokGSTO20} (0.37)& \href{../works/BhatnagarKL19.pdf}{BhatnagarKL19} (0.38)& \href{../works/LombardiM09.pdf}{LombardiM09} (0.38)& \href{../works/TranDRFWOVB16.pdf}{TranDRFWOVB16} (0.38)\\
Dot& \cellcolor{red!40}\href{../works/ZarandiASC20.pdf}{ZarandiASC20} (195.00)& \cellcolor{red!40}\href{../works/Lombardi10.pdf}{Lombardi10} (167.00)& \cellcolor{red!40}\href{../works/Groleaz21.pdf}{Groleaz21} (164.00)& \cellcolor{red!40}\href{../works/Astrand21.pdf}{Astrand21} (164.00)& \cellcolor{red!40}\href{../works/Dejemeppe16.pdf}{Dejemeppe16} (163.00)\\
Cosine& \cellcolor{red!40}\href{../works/LombardiM12.pdf}{LombardiM12} (0.73)& \cellcolor{red!40}\href{../works/GokGSTO20.pdf}{GokGSTO20} (0.72)& \cellcolor{red!40}\href{../works/BeckPS03.pdf}{BeckPS03} (0.71)& \cellcolor{red!40}\href{../works/abs-1902-09244.pdf}{abs-1902-09244} (0.68)& \cellcolor{red!40}\href{../works/HauderBRPA20.pdf}{HauderBRPA20} (0.68)\\
\index{GokgurHO18}\href{../works/GokgurHO18.pdf}{GokgurHO18} R\&C& \cellcolor{red!20}\href{../works/MercierH07.pdf}{MercierH07} (0.86)& \cellcolor{yellow!20}DorndorfHP99 (0.92)& \cellcolor{yellow!20}\href{../works/Vilim05.pdf}{Vilim05} (0.92)& \cellcolor{yellow!20}\href{../works/YunusogluY22.pdf}{YunusogluY22} (0.93)& \cellcolor{green!20}\href{../works/ArbaouiY18.pdf}{ArbaouiY18} (0.93)\\
Euclid& \cellcolor{green!20}\href{../works/BartakSR08.pdf}{BartakSR08} (0.30)& \cellcolor{blue!20}\href{../works/BeckF00.pdf}{BeckF00} (0.33)& \cellcolor{black!20}\href{../works/BeckF00a.pdf}{BeckF00a} (0.35)& \cellcolor{black!20}\href{../works/OrnekO16.pdf}{OrnekO16} (0.35)& \cellcolor{black!20}\href{../works/BaptisteP97.pdf}{BaptisteP97} (0.36)\\
Dot& \cellcolor{red!40}\href{../works/Baptiste02.pdf}{Baptiste02} (223.00)& \cellcolor{red!40}\href{../works/Dejemeppe16.pdf}{Dejemeppe16} (199.00)& \cellcolor{red!40}\href{../works/Malapert11.pdf}{Malapert11} (195.00)& \cellcolor{red!40}\href{../works/Fahimi16.pdf}{Fahimi16} (193.00)& \cellcolor{red!40}\href{../works/Lombardi10.pdf}{Lombardi10} (192.00)\\
Cosine& \cellcolor{red!40}\href{../works/BartakSR08.pdf}{BartakSR08} (0.83)& \cellcolor{red!40}\href{../works/BartakSR10.pdf}{BartakSR10} (0.80)& \cellcolor{red!40}\href{../works/BeckF00.pdf}{BeckF00} (0.79)& \cellcolor{red!40}\href{../works/OrnekO16.pdf}{OrnekO16} (0.77)& \cellcolor{red!40}\href{../works/PapaB98.pdf}{PapaB98} (0.77)\\
\index{GoldwaserS17}\href{../works/GoldwaserS17.pdf}{GoldwaserS17} R\&C& \cellcolor{red!40}\href{../works/KletzanderM17.pdf}{KletzanderM17} (0.74)& \cellcolor{green!20}\href{../works/SchnellH17.pdf}{SchnellH17} (0.93)& \cellcolor{green!20}\href{../works/CambazardJ05.pdf}{CambazardJ05} (0.94)& \cellcolor{green!20}\href{../works/Hooker04.pdf}{Hooker04} (0.95)& \cellcolor{green!20}\href{../works/CobanH10.pdf}{CobanH10} (0.95)\\
Euclid& \cellcolor{red!40}\href{../works/GoldwaserS18.pdf}{GoldwaserS18} (0.23)& \cellcolor{red!20}\href{../works/KletzanderM17.pdf}{KletzanderM17} (0.25)& \cellcolor{yellow!20}\href{../works/HoundjiSWD14.pdf}{HoundjiSWD14} (0.28)& \cellcolor{yellow!20}\href{../works/BockmayrP06.pdf}{BockmayrP06} (0.28)& \cellcolor{green!20}\href{../works/FoxAS82.pdf}{FoxAS82} (0.29)\\
Dot& \cellcolor{red!40}\href{../works/GoldwaserS18.pdf}{GoldwaserS18} (88.00)& \cellcolor{red!40}\href{../works/Froger16.pdf}{Froger16} (72.00)& \cellcolor{red!40}\href{../works/MilanoW09.pdf}{MilanoW09} (71.00)& \cellcolor{red!40}\href{../works/NaderiRR23.pdf}{NaderiRR23} (69.00)& \cellcolor{red!40}\href{../works/Lombardi10.pdf}{Lombardi10} (69.00)\\
Cosine& \cellcolor{red!40}\href{../works/GoldwaserS18.pdf}{GoldwaserS18} (0.86)& \cellcolor{red!40}\href{../works/KletzanderM17.pdf}{KletzanderM17} (0.68)& \cellcolor{red!40}\href{../works/BockmayrP06.pdf}{BockmayrP06} (0.64)& \cellcolor{red!40}\href{../works/BajestaniB11.pdf}{BajestaniB11} (0.63)& \cellcolor{red!40}\href{../works/Hooker06.pdf}{Hooker06} (0.63)\\
\index{GoldwaserS18}\href{../works/GoldwaserS18.pdf}{GoldwaserS18} R\&C& \cellcolor{red!20}\href{../works/HamdiL13.pdf}{HamdiL13} (0.90)& \cellcolor{yellow!20}\href{../works/CireCH13.pdf}{CireCH13} (0.91)& \cellcolor{green!20}\href{../works/TerekhovDOB12.pdf}{TerekhovDOB12} (0.94)& \cellcolor{green!20}\href{../works/CobanH10.pdf}{CobanH10} (0.94)& \cellcolor{green!20}\href{../works/Sadykov04.pdf}{Sadykov04} (0.95)\\
Euclid& \cellcolor{red!40}\href{../works/GoldwaserS17.pdf}{GoldwaserS17} (0.23)& \cellcolor{blue!20}\href{../works/BockmayrP06.pdf}{BockmayrP06} (0.33)& \cellcolor{blue!20}\href{../works/BeldiceanuP07.pdf}{BeldiceanuP07} (0.34)& \cellcolor{blue!20}\href{../works/LozanoCDS12.pdf}{LozanoCDS12} (0.34)& \cellcolor{black!20}\href{../works/Hooker05a.pdf}{Hooker05a} (0.35)\\
Dot& \cellcolor{red!40}\href{../works/Godet21a.pdf}{Godet21a} (108.00)& \cellcolor{red!40}\href{../works/Froger16.pdf}{Froger16} (104.00)& \cellcolor{red!40}\href{../works/Lombardi10.pdf}{Lombardi10} (102.00)& \cellcolor{red!40}\href{../works/MilanoW09.pdf}{MilanoW09} (100.00)& \cellcolor{red!40}\href{../works/Fahimi16.pdf}{Fahimi16} (99.00)\\
Cosine& \cellcolor{red!40}\href{../works/GoldwaserS17.pdf}{GoldwaserS17} (0.86)& \cellcolor{red!40}\href{../works/Hooker06.pdf}{Hooker06} (0.66)& \cellcolor{red!40}\href{../works/ForbesHJST24.pdf}{ForbesHJST24} (0.66)& \cellcolor{red!40}\href{../works/BockmayrP06.pdf}{BockmayrP06} (0.66)& \cellcolor{red!40}\href{../works/EreminW01.pdf}{EreminW01} (0.65)\\
\index{Goltz95}\href{../works/Goltz95.pdf}{Goltz95} R\&C& \cellcolor{red!40}\href{../works/Colombani96.pdf}{Colombani96} (0.75)& \cellcolor{red!40}\href{../works/Simonis95a.pdf}{Simonis95a} (0.79)& \cellcolor{red!40}\href{../works/Wolf03.pdf}{Wolf03} (0.80)& \cellcolor{red!40}\href{../works/Simonis99.pdf}{Simonis99} (0.84)& \cellcolor{red!40}\href{../works/Zhou96.pdf}{Zhou96} (0.86)\\
Euclid& \cellcolor{yellow!20}\href{../works/CrawfordB94.pdf}{CrawfordB94} (0.28)& \cellcolor{green!20}\href{../works/Simonis95.pdf}{Simonis95} (0.29)& \cellcolor{green!20}\href{../works/SimonisC95.pdf}{SimonisC95} (0.29)& \cellcolor{green!20}\href{../works/Caseau97.pdf}{Caseau97} (0.29)& \cellcolor{green!20}\href{../works/FoxAS82.pdf}{FoxAS82} (0.29)\\
Dot& \cellcolor{red!40}\href{../works/TrentesauxPT01.pdf}{TrentesauxPT01} (105.00)& \cellcolor{red!40}\href{../works/Baptiste02.pdf}{Baptiste02} (102.00)& \cellcolor{red!40}\href{../works/BosiM2001.pdf}{BosiM2001} (102.00)& \cellcolor{red!40}\href{../works/Malapert11.pdf}{Malapert11} (100.00)& \cellcolor{red!40}\href{../works/AggounB93.pdf}{AggounB93} (99.00)\\
Cosine& \cellcolor{red!40}\href{../works/TrentesauxPT01.pdf}{TrentesauxPT01} (0.77)& \cellcolor{red!40}\href{../works/AggounB93.pdf}{AggounB93} (0.77)& \cellcolor{red!40}\href{../works/SimonisC95.pdf}{SimonisC95} (0.75)& \cellcolor{red!40}\href{../works/BosiM2001.pdf}{BosiM2001} (0.74)& \cellcolor{red!40}\href{../works/Zhou97.pdf}{Zhou97} (0.71)\\
\index{GombolayWS18}\href{../works/GombolayWS18.pdf}{GombolayWS18} R\&C& \cellcolor{green!20}\href{../works/HamP21.pdf}{HamP21} (0.94)& \cellcolor{green!20}\href{../works/Hooker07.pdf}{Hooker07} (0.95)& \cellcolor{green!20}\href{../works/LiW08.pdf}{LiW08} (0.95)& \cellcolor{green!20}\href{../works/CireCH16.pdf}{CireCH16} (0.95)& \cellcolor{green!20}\href{../works/Hooker04.pdf}{Hooker04} (0.95)\\
Euclid& \href{../works/ZhangYW21.pdf}{ZhangYW21} (0.39)& \href{../works/NaderiBZ22a.pdf}{NaderiBZ22a} (0.39)& \href{../works/BeckF98.pdf}{BeckF98} (0.41)& \href{../works/HeinzNVH22.pdf}{HeinzNVH22} (0.42)& \href{../works/BeckPS03.pdf}{BeckPS03} (0.42)\\
Dot& \cellcolor{red!40}\href{../works/ZarandiASC20.pdf}{ZarandiASC20} (204.00)& \cellcolor{red!40}\href{../works/Groleaz21.pdf}{Groleaz21} (190.00)& \cellcolor{red!40}\href{../works/Astrand21.pdf}{Astrand21} (177.00)& \cellcolor{red!40}\href{../works/Lombardi10.pdf}{Lombardi10} (170.00)& \cellcolor{red!40}\href{../works/Dejemeppe16.pdf}{Dejemeppe16} (170.00)\\
Cosine& \cellcolor{red!40}\href{../works/NaderiBZ22a.pdf}{NaderiBZ22a} (0.74)& \cellcolor{red!40}\href{../works/ZhangYW21.pdf}{ZhangYW21} (0.72)& \cellcolor{red!40}\href{../works/BeckF98.pdf}{BeckF98} (0.71)& \cellcolor{red!40}\href{../works/JainM99.pdf}{JainM99} (0.68)& \cellcolor{red!40}\href{../works/HeinzNVH22.pdf}{HeinzNVH22} (0.68)\\
\index{GomesHS06}\href{../works/GomesHS06.pdf}{GomesHS06} R\&C\\
Euclid& \cellcolor{red!40}\href{../works/Hunsberger08.pdf}{Hunsberger08} (0.19)& \cellcolor{red!40}\href{../works/SunLYL10.pdf}{SunLYL10} (0.19)& \cellcolor{red!40}\href{../works/QuSN06.pdf}{QuSN06} (0.19)& \cellcolor{red!40}\href{../works/BeniniBGM05a.pdf}{BeniniBGM05a} (0.21)& \cellcolor{red!40}\href{../works/AngelsmarkJ00.pdf}{AngelsmarkJ00} (0.21)\\
Dot& \cellcolor{red!40}\href{../works/BartakSR10.pdf}{BartakSR10} (50.00)& \cellcolor{red!40}\href{../works/ZarandiASC20.pdf}{ZarandiASC20} (49.00)& \cellcolor{red!40}\href{../works/LombardiMRB10.pdf}{LombardiMRB10} (47.00)& \cellcolor{red!40}\href{../works/Beck99.pdf}{Beck99} (47.00)& \cellcolor{red!40}\href{../works/PrataAN23.pdf}{PrataAN23} (46.00)\\
Cosine& \cellcolor{red!40}\href{../works/QuSN06.pdf}{QuSN06} (0.74)& \cellcolor{red!40}\href{../works/HoeveGSL07.pdf}{HoeveGSL07} (0.73)& \cellcolor{red!40}\href{../works/SunLYL10.pdf}{SunLYL10} (0.72)& \cellcolor{red!40}\href{../works/PesantGPR99.pdf}{PesantGPR99} (0.70)& \cellcolor{red!40}\href{../works/AkramNHRSA23.pdf}{AkramNHRSA23} (0.69)\\
\index{GomesM17}\href{../works/GomesM17.pdf}{GomesM17} R\&C& \cellcolor{red!20}\href{../works/CireCH16.pdf}{CireCH16} (0.89)& \cellcolor{yellow!20}\href{../works/CobanH10.pdf}{CobanH10} (0.90)& \cellcolor{yellow!20}\href{../works/GedikKEK18.pdf}{GedikKEK18} (0.91)& \cellcolor{yellow!20}\href{../works/CireCH13.pdf}{CireCH13} (0.91)& \cellcolor{yellow!20}\href{../works/TranAB16.pdf}{TranAB16} (0.92)\\
Euclid& \cellcolor{green!20}\href{../works/TranB12.pdf}{TranB12} (0.29)& \cellcolor{green!20}\href{../works/TranAB16.pdf}{TranAB16} (0.31)& \cellcolor{blue!20}\href{../works/BogaerdtW19.pdf}{BogaerdtW19} (0.32)& \cellcolor{black!20}\href{../works/ParkUJR19.pdf}{ParkUJR19} (0.35)& \cellcolor{black!20}\href{../works/ThiruvadyBME09.pdf}{ThiruvadyBME09} (0.37)\\
Dot& \cellcolor{red!40}\href{../works/ZarandiASC20.pdf}{ZarandiASC20} (146.00)& \cellcolor{red!40}\href{../works/Groleaz21.pdf}{Groleaz21} (139.00)& \cellcolor{red!40}\href{../works/TranAB16.pdf}{TranAB16} (132.00)& \cellcolor{red!40}\href{../works/NaderiRR23.pdf}{NaderiRR23} (128.00)& \cellcolor{red!40}\href{../works/Lunardi20.pdf}{Lunardi20} (126.00)\\
Cosine& \cellcolor{red!40}\href{../works/TranB12.pdf}{TranB12} (0.81)& \cellcolor{red!40}\href{../works/TranAB16.pdf}{TranAB16} (0.81)& \cellcolor{red!40}\href{../works/BogaerdtW19.pdf}{BogaerdtW19} (0.71)& \cellcolor{red!40}\href{../works/GedikKEK18.pdf}{GedikKEK18} (0.70)& \cellcolor{red!40}\href{../works/NaderiBZ22.pdf}{NaderiBZ22} (0.69)\\
\index{GongLMW09}GongLMW09 R\&C& \cellcolor{red!40}\href{../works/CobanH10.pdf}{CobanH10} (0.73)& \cellcolor{red!40}\href{../works/HamdiL13.pdf}{HamdiL13} (0.77)& \cellcolor{red!40}\href{../works/CireCH13.pdf}{CireCH13} (0.80)& \cellcolor{red!40}\href{../works/CobanH11.pdf}{CobanH11} (0.84)& \cellcolor{red!40}\href{../works/Beck10.pdf}{Beck10} (0.84)\\
Euclid\\
Dot\\
Cosine\\
\index{GrimesH10}\href{../works/GrimesH10.pdf}{GrimesH10} R\&C& \cellcolor{red!40}\href{../works/GrimesH11.pdf}{GrimesH11} (0.65)& \cellcolor{red!40}\href{../works/GrimesHM09.pdf}{GrimesHM09} (0.70)& \cellcolor{red!40}\href{../works/GrimesH15.pdf}{GrimesH15} (0.72)& \cellcolor{red!40}\href{../works/ArtiguesBF04.pdf}{ArtiguesBF04} (0.76)& \cellcolor{red!40}\href{../works/DejemeppeCS15.pdf}{DejemeppeCS15} (0.77)\\
Euclid& \cellcolor{green!20}\href{../works/MalapertCGJLR13.pdf}{MalapertCGJLR13} (0.30)& \cellcolor{green!20}\href{../works/ArtiguesBF04.pdf}{ArtiguesBF04} (0.30)& \cellcolor{green!20}\href{../works/SialaAH15.pdf}{SialaAH15} (0.31)& \cellcolor{green!20}\href{../works/FocacciLN00.pdf}{FocacciLN00} (0.31)& \cellcolor{blue!20}\href{../works/ArtiguesF07.pdf}{ArtiguesF07} (0.32)\\
Dot& \cellcolor{red!40}\href{../works/Malapert11.pdf}{Malapert11} (176.00)& \cellcolor{red!40}\href{../works/GrimesH15.pdf}{GrimesH15} (171.00)& \cellcolor{red!40}\href{../works/Groleaz21.pdf}{Groleaz21} (168.00)& \cellcolor{red!40}\href{../works/Baptiste02.pdf}{Baptiste02} (165.00)& \cellcolor{red!40}\href{../works/ZarandiASC20.pdf}{ZarandiASC20} (164.00)\\
Cosine& \cellcolor{red!40}\href{../works/GrimesH15.pdf}{GrimesH15} (0.81)& \cellcolor{red!40}\href{../works/ArtiguesBF04.pdf}{ArtiguesBF04} (0.79)& \cellcolor{red!40}\href{../works/FocacciLN00.pdf}{FocacciLN00} (0.79)& \cellcolor{red!40}\href{../works/GrimesH11.pdf}{GrimesH11} (0.79)& \cellcolor{red!40}\href{../works/MalapertCGJLR13.pdf}{MalapertCGJLR13} (0.79)\\
\index{GrimesH11}\href{../works/GrimesH11.pdf}{GrimesH11} R\&C& \cellcolor{red!40}\href{../works/GrimesH10.pdf}{GrimesH10} (0.65)& \cellcolor{red!40}\href{../works/GrimesH15.pdf}{GrimesH15} (0.77)& \cellcolor{red!40}\href{../works/GrimesHM09.pdf}{GrimesHM09} (0.83)& \cellcolor{red!20}\href{../works/DannaP03.pdf}{DannaP03} (0.88)& \cellcolor{red!20}\href{../works/Laborie09.pdf}{Laborie09} (0.88)\\
Euclid& \cellcolor{blue!20}\href{../works/GrimesH10.pdf}{GrimesH10} (0.33)& \cellcolor{black!20}\href{../works/MonetteDH09.pdf}{MonetteDH09} (0.34)& \cellcolor{black!20}\href{../works/BeckR03.pdf}{BeckR03} (0.34)& \cellcolor{black!20}\href{../works/VilimLS15.pdf}{VilimLS15} (0.35)& \cellcolor{black!20}\href{../works/CarchraeB09.pdf}{CarchraeB09} (0.35)\\
Dot& \cellcolor{red!40}\href{../works/Groleaz21.pdf}{Groleaz21} (196.00)& \cellcolor{red!40}\href{../works/GrimesH15.pdf}{GrimesH15} (189.00)& \cellcolor{red!40}\href{../works/Baptiste02.pdf}{Baptiste02} (186.00)& \cellcolor{red!40}\href{../works/Dejemeppe16.pdf}{Dejemeppe16} (185.00)& \cellcolor{red!40}\href{../works/Malapert11.pdf}{Malapert11} (184.00)\\
Cosine& \cellcolor{red!40}\href{../works/GrimesH15.pdf}{GrimesH15} (0.82)& \cellcolor{red!40}\href{../works/GrimesH10.pdf}{GrimesH10} (0.79)& \cellcolor{red!40}\href{../works/MonetteDH09.pdf}{MonetteDH09} (0.77)& \cellcolor{red!40}\href{../works/KelbelH11.pdf}{KelbelH11} (0.77)& \cellcolor{red!40}\href{../works/BeckR03.pdf}{BeckR03} (0.77)\\
\index{GrimesH15}\href{../works/GrimesH15.pdf}{GrimesH15} R\&C& \cellcolor{red!40}\href{../works/GrimesHM09.pdf}{GrimesHM09} (0.70)& \cellcolor{red!40}\href{../works/GrimesH10.pdf}{GrimesH10} (0.72)& \cellcolor{red!40}\href{../works/GrimesH11.pdf}{GrimesH11} (0.77)& \cellcolor{red!40}\href{../works/CarchraeB09.pdf}{CarchraeB09} (0.81)& \cellcolor{red!40}BaptisteLPN06 (0.83)\\
Euclid& \cellcolor{black!20}\href{../works/GrimesH11.pdf}{GrimesH11} (0.36)& \cellcolor{black!20}\href{../works/GrimesH10.pdf}{GrimesH10} (0.37)& \href{../works/MalapertCGJLR12.pdf}{MalapertCGJLR12} (0.40)& \href{../works/GrimesHM09.pdf}{GrimesHM09} (0.42)& \href{../works/BartakSR08.pdf}{BartakSR08} (0.43)\\
Dot& \cellcolor{red!40}\href{../works/Groleaz21.pdf}{Groleaz21} (259.00)& \cellcolor{red!40}\href{../works/Dejemeppe16.pdf}{Dejemeppe16} (247.00)& \cellcolor{red!40}\href{../works/ZarandiASC20.pdf}{ZarandiASC20} (245.00)& \cellcolor{red!40}\href{../works/Malapert11.pdf}{Malapert11} (236.00)& \cellcolor{red!40}\href{../works/Baptiste02.pdf}{Baptiste02} (235.00)\\
Cosine& \cellcolor{red!40}\href{../works/GrimesH11.pdf}{GrimesH11} (0.82)& \cellcolor{red!40}\href{../works/GrimesH10.pdf}{GrimesH10} (0.81)& \cellcolor{red!40}\href{../works/MalapertCGJLR12.pdf}{MalapertCGJLR12} (0.77)& \cellcolor{red!40}\href{../works/NaderiRR23.pdf}{NaderiRR23} (0.76)& \cellcolor{red!40}\href{../works/GrimesHM09.pdf}{GrimesHM09} (0.75)\\
\index{GrimesHM09}\href{../works/GrimesHM09.pdf}{GrimesHM09} R\&C& \cellcolor{red!40}\href{../works/WatsonB08.pdf}{WatsonB08} (0.62)& \cellcolor{red!40}\href{../works/GrimesH10.pdf}{GrimesH10} (0.70)& \cellcolor{red!40}\href{../works/GrimesH15.pdf}{GrimesH15} (0.70)& \cellcolor{red!40}\href{../works/BeckFW11.pdf}{BeckFW11} (0.74)& \cellcolor{red!40}\href{../works/MalapertCGJLR12.pdf}{MalapertCGJLR12} (0.78)\\
Euclid& \cellcolor{yellow!20}\href{../works/MalapertCGJLR13.pdf}{MalapertCGJLR13} (0.28)& \cellcolor{green!20}\href{../works/SialaAH15.pdf}{SialaAH15} (0.30)& \cellcolor{green!20}\href{../works/Bit-Monnot23.pdf}{Bit-Monnot23} (0.30)& \cellcolor{blue!20}\href{../works/MonetteDD07.pdf}{MonetteDD07} (0.32)& \cellcolor{blue!20}\href{../works/MalapertCGJLR12.pdf}{MalapertCGJLR12} (0.32)\\
Dot& \cellcolor{red!40}\href{../works/Malapert11.pdf}{Malapert11} (160.00)& \cellcolor{red!40}\href{../works/GrimesH15.pdf}{GrimesH15} (149.00)& \cellcolor{red!40}\href{../works/Siala15a.pdf}{Siala15a} (142.00)& \cellcolor{red!40}\href{../works/Fahimi16.pdf}{Fahimi16} (142.00)& \cellcolor{red!40}\href{../works/Groleaz21.pdf}{Groleaz21} (136.00)\\
Cosine& \cellcolor{red!40}\href{../works/MalapertCGJLR12.pdf}{MalapertCGJLR12} (0.80)& \cellcolor{red!40}\href{../works/MalapertCGJLR13.pdf}{MalapertCGJLR13} (0.80)& \cellcolor{red!40}\href{../works/Bit-Monnot23.pdf}{Bit-Monnot23} (0.79)& \cellcolor{red!40}\href{../works/SialaAH15.pdf}{SialaAH15} (0.77)& \cellcolor{red!40}\href{../works/GrimesH10.pdf}{GrimesH10} (0.75)\\
\index{GrimesIOS14}\href{../works/GrimesIOS14.pdf}{GrimesIOS14} R\&C& \cellcolor{red!40}\href{../works/IfrimOS12.pdf}{IfrimOS12} (0.81)\\
Euclid& \cellcolor{red!20}\href{../works/IfrimOS12.pdf}{IfrimOS12} (0.26)& \cellcolor{green!20}\href{../works/HurleyOS16.pdf}{HurleyOS16} (0.29)& \cellcolor{green!20}\href{../works/BockmayrP06.pdf}{BockmayrP06} (0.31)& \cellcolor{blue!20}\href{../works/KinsellaS0OS16.pdf}{KinsellaS0OS16} (0.33)& \cellcolor{blue!20}\href{../works/AntunesABD18.pdf}{AntunesABD18} (0.33)\\
Dot& \cellcolor{red!40}\href{../works/ZarandiASC20.pdf}{ZarandiASC20} (107.00)& \cellcolor{red!40}\href{../works/Groleaz21.pdf}{Groleaz21} (106.00)& \cellcolor{red!40}\href{../works/Lombardi10.pdf}{Lombardi10} (104.00)& \cellcolor{red!40}\href{../works/Astrand21.pdf}{Astrand21} (99.00)& \cellcolor{red!40}\href{../works/Baptiste02.pdf}{Baptiste02} (98.00)\\
Cosine& \cellcolor{red!40}\href{../works/IfrimOS12.pdf}{IfrimOS12} (0.78)& \cellcolor{red!40}\href{../works/HurleyOS16.pdf}{HurleyOS16} (0.71)& \cellcolor{red!40}\href{../works/BockmayrP06.pdf}{BockmayrP06} (0.67)& \cellcolor{red!40}\href{../works/TranPZLDB18.pdf}{TranPZLDB18} (0.65)& \cellcolor{red!40}\href{../works/QuirogaZH05.pdf}{QuirogaZH05} (0.64)\\
\index{Groleaz21}\href{../works/Groleaz21.pdf}{Groleaz21} R\&C\\
Euclid& \href{../works/GrimesH15.pdf}{GrimesH15} (0.60)& \href{../works/NaderiRR23.pdf}{NaderiRR23} (0.61)& \href{../works/AwadMDMT22.pdf}{AwadMDMT22} (0.62)& \href{../works/PrataAN23.pdf}{PrataAN23} (0.63)& \href{../works/Astrand21.pdf}{Astrand21} (0.63)\\
Dot& \cellcolor{red!40}\href{../works/ZarandiASC20.pdf}{ZarandiASC20} (368.00)& \cellcolor{red!40}\href{../works/Dejemeppe16.pdf}{Dejemeppe16} (324.00)& \cellcolor{red!40}\href{../works/Baptiste02.pdf}{Baptiste02} (318.00)& \cellcolor{red!40}\href{../works/Lunardi20.pdf}{Lunardi20} (293.00)& \cellcolor{red!40}\href{../works/Astrand21.pdf}{Astrand21} (293.00)\\
Cosine& \cellcolor{red!40}\href{../works/GrimesH15.pdf}{GrimesH15} (0.70)& \cellcolor{red!40}\href{../works/NaderiRR23.pdf}{NaderiRR23} (0.70)& \cellcolor{red!40}\href{../works/Astrand21.pdf}{Astrand21} (0.69)& \cellcolor{red!40}\href{../works/Lunardi20.pdf}{Lunardi20} (0.68)& \cellcolor{red!40}\href{../works/Baptiste02.pdf}{Baptiste02} (0.68)\\
\index{GroleazNS20}\href{../works/GroleazNS20.pdf}{GroleazNS20} R\&C& \cellcolor{red!40}\href{../works/GroleazNS20a.pdf}{GroleazNS20a} (0.79)& \cellcolor{yellow!20}\href{../works/FetgoD22.pdf}{FetgoD22} (0.91)& \cellcolor{yellow!20}\href{../works/GayHS15.pdf}{GayHS15} (0.91)& \cellcolor{yellow!20}\href{../works/LetortBC12.pdf}{LetortBC12} (0.92)& \cellcolor{yellow!20}\href{../works/GayHS15a.pdf}{GayHS15a} (0.92)\\
Euclid& \cellcolor{green!20}\href{../works/GroleazNS20a.pdf}{GroleazNS20a} (0.30)& \cellcolor{black!20}\href{../works/Balduccini11.pdf}{Balduccini11} (0.35)& \cellcolor{black!20}\href{../works/SmithC93.pdf}{SmithC93} (0.35)& \cellcolor{black!20}\href{../works/HeipckeCCS00.pdf}{HeipckeCCS00} (0.36)& \cellcolor{black!20}\href{../works/PengLC14.pdf}{PengLC14} (0.36)\\
Dot& \cellcolor{red!40}\href{../works/Groleaz21.pdf}{Groleaz21} (184.00)& \cellcolor{red!40}\href{../works/Dejemeppe16.pdf}{Dejemeppe16} (158.00)& \cellcolor{red!40}\href{../works/ZarandiASC20.pdf}{ZarandiASC20} (139.00)& \cellcolor{red!40}\href{../works/Lombardi10.pdf}{Lombardi10} (139.00)& \cellcolor{red!40}\href{../works/Baptiste02.pdf}{Baptiste02} (134.00)\\
Cosine& \cellcolor{red!40}\href{../works/GroleazNS20a.pdf}{GroleazNS20a} (0.81)& \cellcolor{red!40}\href{../works/PengLC14.pdf}{PengLC14} (0.69)& \cellcolor{red!40}\href{../works/GaySS14.pdf}{GaySS14} (0.69)& \cellcolor{red!40}\href{../works/HeipckeCCS00.pdf}{HeipckeCCS00} (0.68)& \cellcolor{red!40}\href{../works/Balduccini11.pdf}{Balduccini11} (0.68)\\
\index{GroleazNS20a}\href{../works/GroleazNS20a.pdf}{GroleazNS20a} R\&C& \cellcolor{red!40}\href{../works/GroleazNS20.pdf}{GroleazNS20} (0.79)& \cellcolor{red!40}\href{../works/BaptisteB18.pdf}{BaptisteB18} (0.81)& \cellcolor{yellow!20}\href{../works/ColT19.pdf}{ColT19} (0.90)& \cellcolor{yellow!20}\href{../works/OuelletQ13.pdf}{OuelletQ13} (0.91)& \cellcolor{green!20}\href{../works/PoderBS04.pdf}{PoderBS04} (0.94)\\
Euclid& \cellcolor{green!20}\href{../works/GroleazNS20.pdf}{GroleazNS20} (0.30)& \cellcolor{black!20}\href{../works/GedikKBR17.pdf}{GedikKBR17} (0.37)& \cellcolor{black!20}\href{../works/Limtanyakul07.pdf}{Limtanyakul07} (0.37)& \href{../works/HeipckeCCS00.pdf}{HeipckeCCS00} (0.38)& \href{../works/Laborie09.pdf}{Laborie09} (0.38)\\
Dot& \cellcolor{red!40}\href{../works/Groleaz21.pdf}{Groleaz21} (172.00)& \cellcolor{red!40}\href{../works/Dejemeppe16.pdf}{Dejemeppe16} (147.00)& \cellcolor{red!40}\href{../works/ZarandiASC20.pdf}{ZarandiASC20} (145.00)& \cellcolor{red!40}\href{../works/Lunardi20.pdf}{Lunardi20} (138.00)& \cellcolor{red!40}\href{../works/NaderiRR23.pdf}{NaderiRR23} (129.00)\\
Cosine& \cellcolor{red!40}\href{../works/GroleazNS20.pdf}{GroleazNS20} (0.81)& \cellcolor{red!40}\href{../works/Laborie09.pdf}{Laborie09} (0.66)& \cellcolor{red!40}\href{../works/GedikKEK18.pdf}{GedikKEK18} (0.65)& \cellcolor{red!40}\href{../works/MeyerE04.pdf}{MeyerE04} (0.64)& \cellcolor{red!40}\href{../works/AwadMDMT22.pdf}{AwadMDMT22} (0.64)\\
\index{Gronkvist06}\href{../works/Gronkvist06.pdf}{Gronkvist06} R\&C& \cellcolor{yellow!20}\href{../works/MilanoW06.pdf}{MilanoW06} (0.90)& \cellcolor{yellow!20}\href{../works/BosiM2001.pdf}{BosiM2001} (0.93)& \cellcolor{yellow!20}\href{../works/EastonNT02.pdf}{EastonNT02} (0.93)& \cellcolor{yellow!20}Milano11 (0.93)& \cellcolor{green!20}Hooker06a (0.94)\\
Euclid& \cellcolor{green!20}\href{../works/ZibranR11a.pdf}{ZibranR11a} (0.29)& \cellcolor{green!20}\href{../works/ZibranR11.pdf}{ZibranR11} (0.30)& \cellcolor{green!20}\href{../works/Puget95.pdf}{Puget95} (0.30)& \cellcolor{green!20}\href{../works/ZhangLS12.pdf}{ZhangLS12} (0.31)& \cellcolor{green!20}\href{../works/WallaceF00.pdf}{WallaceF00} (0.31)\\
Dot& \cellcolor{red!40}\href{../works/ZarandiASC20.pdf}{ZarandiASC20} (90.00)& \cellcolor{red!40}\href{../works/Fahimi16.pdf}{Fahimi16} (84.00)& \cellcolor{red!40}\href{../works/Dejemeppe16.pdf}{Dejemeppe16} (80.00)& \cellcolor{red!40}\href{../works/Simonis07.pdf}{Simonis07} (78.00)& \cellcolor{red!40}\href{../works/Lombardi10.pdf}{Lombardi10} (78.00)\\
Cosine& \cellcolor{red!40}\href{../works/ZibranR11a.pdf}{ZibranR11a} (0.65)& \cellcolor{red!40}\href{../works/ChunCTY99.pdf}{ChunCTY99} (0.64)& \cellcolor{red!40}\href{../works/Mason01.pdf}{Mason01} (0.64)& \cellcolor{red!40}\href{../works/HachemiGR11.pdf}{HachemiGR11} (0.63)& \cellcolor{red!40}\href{../works/ZibranR11.pdf}{ZibranR11} (0.62)\\
\index{GruianK98}\href{../works/GruianK98.pdf}{GruianK98} R\&C& \cellcolor{blue!20}\href{../works/KuchcinskiW03.pdf}{KuchcinskiW03} (0.96)& \cellcolor{blue!20}\href{../works/LombardiMRB10.pdf}{LombardiMRB10} (0.98)\\
Euclid& \cellcolor{yellow!20}\href{../works/Simonis95.pdf}{Simonis95} (0.27)& \cellcolor{green!20}\href{../works/KuchcinskiW03.pdf}{KuchcinskiW03} (0.29)& \cellcolor{green!20}\href{../works/Simonis95a.pdf}{Simonis95a} (0.29)& \cellcolor{green!20}\href{../works/WallaceF00.pdf}{WallaceF00} (0.30)& \cellcolor{green!20}\href{../works/ErtlK91.pdf}{ErtlK91} (0.31)\\
Dot& \cellcolor{red!40}\href{../works/Simonis99.pdf}{Simonis99} (90.00)& \cellcolor{red!40}\href{../works/Kuchcinski03.pdf}{Kuchcinski03} (87.00)& \cellcolor{red!40}\href{../works/Simonis07.pdf}{Simonis07} (85.00)& \cellcolor{red!40}\href{../works/Malapert11.pdf}{Malapert11} (85.00)& \cellcolor{red!40}\href{../works/Wallace96.pdf}{Wallace96} (82.00)\\
Cosine& \cellcolor{red!40}\href{../works/Simonis95a.pdf}{Simonis95a} (0.73)& \cellcolor{red!40}\href{../works/Simonis95.pdf}{Simonis95} (0.70)& \cellcolor{red!40}\href{../works/BeldiceanuC94.pdf}{BeldiceanuC94} (0.68)& \cellcolor{red!40}\href{../works/Simonis99.pdf}{Simonis99} (0.68)& \cellcolor{red!40}\href{../works/KuchcinskiW03.pdf}{KuchcinskiW03} (0.66)\\
\index{GuSS13}\href{../works/GuSS13.pdf}{GuSS13} R\&C& \cellcolor{red!40}GuSSWC14 (0.51)& \cellcolor{red!40}\href{../works/GuSW12.pdf}{GuSW12} (0.60)& \cellcolor{red!40}\href{../works/SchuttCSW12.pdf}{SchuttCSW12} (0.66)& \cellcolor{red!40}\href{../works/ThiruvadyWGS14.pdf}{ThiruvadyWGS14} (0.74)& \cellcolor{red!40}SchuttFSW15 (0.83)\\
Euclid& \cellcolor{red!40}\href{../works/GuSW12.pdf}{GuSW12} (0.22)& \cellcolor{red!20}\href{../works/SchuttCSW12.pdf}{SchuttCSW12} (0.25)& \cellcolor{green!20}\href{../works/ThiruvadyWGS14.pdf}{ThiruvadyWGS14} (0.29)& \cellcolor{green!20}\href{../works/LombardiM13.pdf}{LombardiM13} (0.30)& \cellcolor{green!20}\href{../works/BofillCSV17a.pdf}{BofillCSV17a} (0.30)\\
Dot& \cellcolor{red!40}\href{../works/ZarandiASC20.pdf}{ZarandiASC20} (94.00)& \cellcolor{red!40}\href{../works/Lombardi10.pdf}{Lombardi10} (93.00)& \cellcolor{red!40}\href{../works/Schutt11.pdf}{Schutt11} (92.00)& \cellcolor{red!40}\href{../works/Caballero19.pdf}{Caballero19} (86.00)& \cellcolor{red!40}\href{../works/Godet21a.pdf}{Godet21a} (85.00)\\
Cosine& \cellcolor{red!40}\href{../works/GuSW12.pdf}{GuSW12} (0.81)& \cellcolor{red!40}\href{../works/SchuttCSW12.pdf}{SchuttCSW12} (0.78)& \cellcolor{red!40}\href{../works/ThiruvadyWGS14.pdf}{ThiruvadyWGS14} (0.73)& \cellcolor{red!40}\href{../works/SchnellH15.pdf}{SchnellH15} (0.66)& \cellcolor{red!40}\href{../works/BofillCSV17a.pdf}{BofillCSV17a} (0.66)\\
\index{GuSSWC14}GuSSWC14 R\&C& \cellcolor{red!40}\href{../works/GuSS13.pdf}{GuSS13} (0.51)& \cellcolor{red!40}\href{../works/GuSW12.pdf}{GuSW12} (0.59)& \cellcolor{red!40}\href{../works/SchuttCSW12.pdf}{SchuttCSW12} (0.72)& \cellcolor{red!40}SchuttFSW15 (0.80)& \cellcolor{red!40}\href{../works/ThiruvadyWGS14.pdf}{ThiruvadyWGS14} (0.85)\\
Euclid\\
Dot\\
Cosine\\
\index{GuSW12}\href{../works/GuSW12.pdf}{GuSW12} R\&C& \cellcolor{red!40}\href{../works/SchuttCSW12.pdf}{SchuttCSW12} (0.59)& \cellcolor{red!40}GuSSWC14 (0.59)& \cellcolor{red!40}\href{../works/GuSS13.pdf}{GuSS13} (0.60)& \cellcolor{red!40}NeronABCDD06 (0.84)& \cellcolor{red!20}\href{../works/ThiruvadyWGS14.pdf}{ThiruvadyWGS14} (0.88)\\
Euclid& \cellcolor{red!40}\href{../works/GuSS13.pdf}{GuSS13} (0.22)& \cellcolor{red!40}\href{../works/SchuttCSW12.pdf}{SchuttCSW12} (0.23)& \cellcolor{red!20}\href{../works/BofillCSV17.pdf}{BofillCSV17} (0.24)& \cellcolor{red!20}\href{../works/LombardiM13.pdf}{LombardiM13} (0.25)& \cellcolor{yellow!20}\href{../works/LeeKLKKYHP97.pdf}{LeeKLKKYHP97} (0.26)\\
Dot& \cellcolor{red!40}\href{../works/ZarandiASC20.pdf}{ZarandiASC20} (86.00)& \cellcolor{red!40}\href{../works/Lombardi10.pdf}{Lombardi10} (84.00)& \cellcolor{red!40}\href{../works/Schutt11.pdf}{Schutt11} (84.00)& \cellcolor{red!40}\href{../works/Godet21a.pdf}{Godet21a} (83.00)& \cellcolor{red!40}\href{../works/Dejemeppe16.pdf}{Dejemeppe16} (82.00)\\
Cosine& \cellcolor{red!40}\href{../works/SchuttCSW12.pdf}{SchuttCSW12} (0.81)& \cellcolor{red!40}\href{../works/GuSS13.pdf}{GuSS13} (0.81)& \cellcolor{red!40}\href{../works/BofillCSV17.pdf}{BofillCSV17} (0.76)& \cellcolor{red!40}\href{../works/ThiruvadyWGS14.pdf}{ThiruvadyWGS14} (0.72)& \cellcolor{red!40}\href{../works/LombardiM13.pdf}{LombardiM13} (0.71)\\
\index{GunerGSKD23}GunerGSKD23 R\&C& \cellcolor{yellow!20}\href{../works/CilKLO22.pdf}{CilKLO22} (0.91)& \cellcolor{yellow!20}\href{../works/OzturkTHO13.pdf}{OzturkTHO13} (0.93)& \cellcolor{green!20}\href{../works/AbidinK20.pdf}{AbidinK20} (0.94)& \cellcolor{green!20}Pinarbasi21 (0.94)& \cellcolor{green!20}PinarbasiA20 (0.95)\\
Euclid\\
Dot\\
Cosine\\
\index{GuoHLW20}GuoHLW20 R\&C& \cellcolor{red!20}MartnezAJ22 (0.88)& \cellcolor{red!20}\href{../works/CireCH16.pdf}{CireCH16} (0.90)& \cellcolor{yellow!20}\href{../works/SunTB19.pdf}{SunTB19} (0.91)& \cellcolor{yellow!20}\href{../works/UnsalO19.pdf}{UnsalO19} (0.91)& \cellcolor{yellow!20}\href{../works/AgussurjaKL18.pdf}{AgussurjaKL18} (0.92)\\
Euclid\\
Dot\\
Cosine\\
\index{GuoZ23}\href{../works/GuoZ23.pdf}{GuoZ23} R\&C& \cellcolor{yellow!20}MartnezAJ22 (0.91)& \cellcolor{yellow!20}\href{../works/ElciOH22.pdf}{ElciOH22} (0.92)& \cellcolor{yellow!20}NaderiRBAU21 (0.92)& \cellcolor{green!20}\href{../works/RoshanaeiBAUB20.pdf}{RoshanaeiBAUB20} (0.94)& \cellcolor{green!20}\href{../works/NaderiBZR23.pdf}{NaderiBZR23} (0.94)\\
Euclid& \href{../works/ElciOH22.pdf}{ElciOH22} (0.42)& \href{../works/RoshanaeiLAU17.pdf}{RoshanaeiLAU17} (0.42)& \href{../works/CireCH13.pdf}{CireCH13} (0.43)& \href{../works/ForbesHJST24.pdf}{ForbesHJST24} (0.43)& \href{../works/NaderiBZR23.pdf}{NaderiBZR23} (0.43)\\
Dot& \cellcolor{red!40}\href{../works/ZarandiASC20.pdf}{ZarandiASC20} (133.00)& \cellcolor{red!40}\href{../works/HarjunkoskiMBC14.pdf}{HarjunkoskiMBC14} (126.00)& \cellcolor{red!40}\href{../works/Groleaz21.pdf}{Groleaz21} (123.00)& \cellcolor{red!40}\href{../works/LaborieRSV18.pdf}{LaborieRSV18} (122.00)& \cellcolor{red!40}\href{../works/Lunardi20.pdf}{Lunardi20} (122.00)\\
Cosine& \cellcolor{red!40}\href{../works/RoshanaeiLAU17.pdf}{RoshanaeiLAU17} (0.66)& \cellcolor{red!40}\href{../works/NaderiBZR23.pdf}{NaderiBZR23} (0.64)& \cellcolor{red!40}\href{../works/ElciOH22.pdf}{ElciOH22} (0.64)& \cellcolor{red!40}\href{../works/Hooker19.pdf}{Hooker19} (0.62)& \cellcolor{red!40}\href{../works/ForbesHJST24.pdf}{ForbesHJST24} (0.62)\\
\index{GurEA19}\href{../works/GurEA19.pdf}{GurEA19} R\&C& \cellcolor{red!40}\href{../works/GurPAE23.pdf}{GurPAE23} (0.85)& \cellcolor{red!20}\href{../works/WangMD15.pdf}{WangMD15} (0.88)& \cellcolor{red!20}\href{../works/GhandehariK22.pdf}{GhandehariK22} (0.88)& \cellcolor{yellow!20}\href{../works/ZhaoL14.pdf}{ZhaoL14} (0.91)& \cellcolor{yellow!20}\href{../works/RoshanaeiBAUB20.pdf}{RoshanaeiBAUB20} (0.91)\\
Euclid& \cellcolor{red!40}\href{../works/GurPAE23.pdf}{GurPAE23} (0.24)& \cellcolor{red!20}\href{../works/DoulabiRP16.pdf}{DoulabiRP16} (0.25)& \cellcolor{yellow!20}\href{../works/MeskensDHG11.pdf}{MeskensDHG11} (0.28)& \cellcolor{yellow!20}\href{../works/GhandehariK22.pdf}{GhandehariK22} (0.28)& \cellcolor{green!20}\href{../works/RiiseML16.pdf}{RiiseML16} (0.29)\\
Dot& \cellcolor{red!40}\href{../works/ZarandiASC20.pdf}{ZarandiASC20} (100.00)& \cellcolor{red!40}\href{../works/WangMD15.pdf}{WangMD15} (95.00)& \cellcolor{red!40}\href{../works/RoshanaeiLAU17.pdf}{RoshanaeiLAU17} (93.00)& \cellcolor{red!40}\href{../works/GurPAE23.pdf}{GurPAE23} (92.00)& \cellcolor{red!40}\href{../works/RoshanaeiBAUB20.pdf}{RoshanaeiBAUB20} (92.00)\\
Cosine& \cellcolor{red!40}\href{../works/GurPAE23.pdf}{GurPAE23} (0.83)& \cellcolor{red!40}\href{../works/DoulabiRP16.pdf}{DoulabiRP16} (0.80)& \cellcolor{red!40}\href{../works/WangMD15.pdf}{WangMD15} (0.77)& \cellcolor{red!40}\href{../works/RiiseML16.pdf}{RiiseML16} (0.77)& \cellcolor{red!40}\href{../works/GhandehariK22.pdf}{GhandehariK22} (0.76)\\
\index{GurPAE23}\href{../works/GurPAE23.pdf}{GurPAE23} R\&C& \cellcolor{red!40}\href{../works/GurEA19.pdf}{GurEA19} (0.85)& \cellcolor{yellow!20}\href{../works/FarsiTM22.pdf}{FarsiTM22} (0.92)& \cellcolor{yellow!20}GhasemiMH23 (0.92)& \cellcolor{green!20}PinarbasiA20 (0.94)& \cellcolor{green!20}\href{../works/Alaka21.pdf}{Alaka21} (0.95)\\
Euclid& \cellcolor{red!40}\href{../works/GurEA19.pdf}{GurEA19} (0.24)& \cellcolor{yellow!20}\href{../works/DoulabiRP16.pdf}{DoulabiRP16} (0.26)& \cellcolor{yellow!20}\href{../works/MeskensDHG11.pdf}{MeskensDHG11} (0.27)& \cellcolor{yellow!20}\href{../works/MeskensDL13.pdf}{MeskensDL13} (0.28)& \cellcolor{green!20}\href{../works/WangMD15.pdf}{WangMD15} (0.29)\\
Dot& \cellcolor{red!40}\href{../works/ZarandiASC20.pdf}{ZarandiASC20} (109.00)& \cellcolor{red!40}\href{../works/WangMD15.pdf}{WangMD15} (98.00)& \cellcolor{red!40}\href{../works/MeskensDL13.pdf}{MeskensDL13} (98.00)& \cellcolor{red!40}\href{../works/RoshanaeiLAU17.pdf}{RoshanaeiLAU17} (94.00)& \cellcolor{red!40}\href{../works/FarsiTM22.pdf}{FarsiTM22} (92.00)\\
Cosine& \cellcolor{red!40}\href{../works/GurEA19.pdf}{GurEA19} (0.83)& \cellcolor{red!40}\href{../works/MeskensDL13.pdf}{MeskensDL13} (0.79)& \cellcolor{red!40}\href{../works/DoulabiRP16.pdf}{DoulabiRP16} (0.78)& \cellcolor{red!40}\href{../works/WangMD15.pdf}{WangMD15} (0.78)& \cellcolor{red!40}\href{../works/MeskensDHG11.pdf}{MeskensDHG11} (0.76)\\
\index{GuyonLPR12}\href{../works/GuyonLPR12.pdf}{GuyonLPR12} R\&C& \cellcolor{red!40}\href{../works/MilanoW09.pdf}{MilanoW09} (0.84)& \cellcolor{red!20}\href{../works/BajestaniB15.pdf}{BajestaniB15} (0.88)& \cellcolor{red!20}\href{../works/CobanH10.pdf}{CobanH10} (0.89)& \cellcolor{red!20}\href{../works/CireCH13.pdf}{CireCH13} (0.89)& \cellcolor{red!20}\href{../works/CireCH16.pdf}{CireCH16} (0.89)\\
Euclid& \cellcolor{green!20}\href{../works/KhayatLR06.pdf}{KhayatLR06} (0.31)& \cellcolor{blue!20}\href{../works/TanSD10.pdf}{TanSD10} (0.32)& \cellcolor{blue!20}\href{../works/BillautHL12.pdf}{BillautHL12} (0.32)& \cellcolor{blue!20}\href{../works/HeipckeCCS00.pdf}{HeipckeCCS00} (0.32)& \cellcolor{blue!20}\href{../works/TanT18.pdf}{TanT18} (0.33)\\
Dot& \cellcolor{red!40}\href{../works/Godet21a.pdf}{Godet21a} (155.00)& \cellcolor{red!40}\href{../works/Baptiste02.pdf}{Baptiste02} (152.00)& \cellcolor{red!40}\href{../works/Groleaz21.pdf}{Groleaz21} (151.00)& \cellcolor{red!40}\href{../works/ZarandiASC20.pdf}{ZarandiASC20} (148.00)& \cellcolor{red!40}\href{../works/Malapert11.pdf}{Malapert11} (147.00)\\
Cosine& \cellcolor{red!40}\href{../works/KhayatLR06.pdf}{KhayatLR06} (0.76)& \cellcolor{red!40}\href{../works/JuvinHL22.pdf}{JuvinHL22} (0.76)& \cellcolor{red!40}\href{../works/TanSD10.pdf}{TanSD10} (0.75)& \cellcolor{red!40}\href{../works/BillautHL12.pdf}{BillautHL12} (0.74)& \cellcolor{red!40}\href{../works/HeipckeCCS00.pdf}{HeipckeCCS00} (0.74)\\
\index{HachemiGR11}\href{../works/HachemiGR11.pdf}{HachemiGR11} R\&C& \cellcolor{yellow!20}\href{../works/Limtanyakul07.pdf}{Limtanyakul07} (0.91)& \cellcolor{yellow!20}EdisO11a (0.93)& \cellcolor{green!20}\href{../works/Beck10.pdf}{Beck10} (0.93)& \cellcolor{green!20}\href{../works/ChuX05.pdf}{ChuX05} (0.94)& \cellcolor{green!20}\href{../works/Wallace06.pdf}{Wallace06} (0.94)\\
Euclid& \cellcolor{yellow!20}\href{../works/ZeballosM09.pdf}{ZeballosM09} (0.28)& \cellcolor{green!20}\href{../works/ZibranR11a.pdf}{ZibranR11a} (0.31)& \cellcolor{green!20}\href{../works/BeniniLMMR08.pdf}{BeniniLMMR08} (0.31)& \cellcolor{blue!20}\href{../works/LozanoCDS12.pdf}{LozanoCDS12} (0.32)& \cellcolor{blue!20}\href{../works/CorreaLR07.pdf}{CorreaLR07} (0.32)\\
Dot& \cellcolor{red!40}\href{../works/Dejemeppe16.pdf}{Dejemeppe16} (110.00)& \cellcolor{red!40}\href{../works/Malapert11.pdf}{Malapert11} (107.00)& \cellcolor{red!40}\href{../works/ZarandiASC20.pdf}{ZarandiASC20} (106.00)& \cellcolor{red!40}\href{../works/MilanoW09.pdf}{MilanoW09} (104.00)& \cellcolor{red!40}\href{../works/LaborieRSV18.pdf}{LaborieRSV18} (103.00)\\
Cosine& \cellcolor{red!40}\href{../works/ZeballosM09.pdf}{ZeballosM09} (0.77)& \cellcolor{red!40}\href{../works/QinDS16.pdf}{QinDS16} (0.73)& \cellcolor{red!40}\href{../works/NovasH12.pdf}{NovasH12} (0.72)& \cellcolor{red!40}\href{../works/MaraveliasCG04.pdf}{MaraveliasCG04} (0.71)& \cellcolor{red!40}\href{../works/NovasH14.pdf}{NovasH14} (0.71)\\
\index{Ham18}\href{../works/Ham18.pdf}{Ham18} R\&C& \cellcolor{red!20}\href{../works/Ham18a.pdf}{Ham18a} (0.87)& \cellcolor{green!20}Ham20 (0.93)& \cellcolor{green!20}\href{../works/Ham20a.pdf}{Ham20a} (0.93)& \cellcolor{green!20}\href{../works/OujanaAYB22.pdf}{OujanaAYB22} (0.94)& \cellcolor{green!20}\href{../works/SadykovW06.pdf}{SadykovW06} (0.95)\\
Euclid& \cellcolor{red!20}\href{../works/Ham20a.pdf}{Ham20a} (0.25)& \cellcolor{yellow!20}\href{../works/Ham18a.pdf}{Ham18a} (0.28)& \cellcolor{green!20}\href{../works/MakMS10.pdf}{MakMS10} (0.30)& \cellcolor{green!20}\href{../works/HamP21.pdf}{HamP21} (0.31)& \cellcolor{blue!20}\href{../works/JuvinHL23.pdf}{JuvinHL23} (0.32)\\
Dot& \cellcolor{red!40}\href{../works/ZarandiASC20.pdf}{ZarandiASC20} (137.00)& \cellcolor{red!40}\href{../works/Groleaz21.pdf}{Groleaz21} (137.00)& \cellcolor{red!40}\href{../works/Lunardi20.pdf}{Lunardi20} (134.00)& \cellcolor{red!40}\href{../works/Malapert11.pdf}{Malapert11} (126.00)& \cellcolor{red!40}\href{../works/Astrand21.pdf}{Astrand21} (125.00)\\
Cosine& \cellcolor{red!40}\href{../works/Ham20a.pdf}{Ham20a} (0.83)& \cellcolor{red!40}\href{../works/Ham18a.pdf}{Ham18a} (0.80)& \cellcolor{red!40}\href{../works/HamP21.pdf}{HamP21} (0.74)& \cellcolor{red!40}\href{../works/ZeballosH05.pdf}{ZeballosH05} (0.72)& \cellcolor{red!40}\href{../works/NovasH14.pdf}{NovasH14} (0.72)\\
\index{Ham18a}\href{../works/Ham18a.pdf}{Ham18a} R\&C& \cellcolor{red!40}\href{../works/NattafDYW19.pdf}{NattafDYW19} (0.84)& \cellcolor{red!20}\href{../works/Ham18.pdf}{Ham18} (0.87)& \cellcolor{red!20}\href{../works/HamFC17.pdf}{HamFC17} (0.89)& \cellcolor{red!20}\href{../works/LaborieRSV18.pdf}{LaborieRSV18} (0.89)& \cellcolor{yellow!20}Ham20 (0.91)\\
Euclid& \cellcolor{yellow!20}\href{../works/Ham18.pdf}{Ham18} (0.28)& \cellcolor{yellow!20}\href{../works/NattafDYW19.pdf}{NattafDYW19} (0.28)& \cellcolor{green!20}\href{../works/ArbaouiY18.pdf}{ArbaouiY18} (0.30)& \cellcolor{green!20}\href{../works/HamFC17.pdf}{HamFC17} (0.30)& \cellcolor{green!20}\href{../works/Ham20a.pdf}{Ham20a} (0.30)\\
Dot& \cellcolor{red!40}\href{../works/Groleaz21.pdf}{Groleaz21} (151.00)& \cellcolor{red!40}\href{../works/ZarandiASC20.pdf}{ZarandiASC20} (147.00)& \cellcolor{red!40}\href{../works/Lunardi20.pdf}{Lunardi20} (146.00)& \cellcolor{red!40}\href{../works/Astrand21.pdf}{Astrand21} (137.00)& \cellcolor{red!40}\href{../works/Malapert11.pdf}{Malapert11} (134.00)\\
Cosine& \cellcolor{red!40}\href{../works/Ham18.pdf}{Ham18} (0.80)& \cellcolor{red!40}\href{../works/NattafDYW19.pdf}{NattafDYW19} (0.80)& \cellcolor{red!40}\href{../works/HamFC17.pdf}{HamFC17} (0.77)& \cellcolor{red!40}\href{../works/ArbaouiY18.pdf}{ArbaouiY18} (0.77)& \cellcolor{red!40}\href{../works/abs-2305-19888.pdf}{abs-2305-19888} (0.77)\\
\index{Ham20}Ham20 R\&C& \cellcolor{red!40}\href{../works/Ham20a.pdf}{Ham20a} (0.00)& \cellcolor{yellow!20}\href{../works/HamP21.pdf}{HamP21} (0.91)& \cellcolor{yellow!20}\href{../works/Ham18a.pdf}{Ham18a} (0.91)& \cellcolor{yellow!20}\href{../works/Laborie18a.pdf}{Laborie18a} (0.92)& \cellcolor{yellow!20}\href{../works/LaborieRSV18.pdf}{LaborieRSV18} (0.93)\\
Euclid\\
Dot\\
Cosine\\
\index{Ham20a}\href{../works/Ham20a.pdf}{Ham20a} R\&C& \cellcolor{red!40}Ham20 (0.00)& \cellcolor{yellow!20}\href{../works/HamP21.pdf}{HamP21} (0.91)& \cellcolor{yellow!20}\href{../works/Ham18a.pdf}{Ham18a} (0.91)& \cellcolor{yellow!20}\href{../works/Laborie18a.pdf}{Laborie18a} (0.92)& \cellcolor{yellow!20}\href{../works/LaborieRSV18.pdf}{LaborieRSV18} (0.93)\\
Euclid& \cellcolor{red!20}\href{../works/Ham18.pdf}{Ham18} (0.25)& \cellcolor{yellow!20}\href{../works/HamP21.pdf}{HamP21} (0.27)& \cellcolor{green!20}\href{../works/BukchinR18.pdf}{BukchinR18} (0.29)& \cellcolor{green!20}\href{../works/Ham18a.pdf}{Ham18a} (0.30)& \cellcolor{green!20}\href{../works/HeipckeCCS00.pdf}{HeipckeCCS00} (0.30)\\
Dot& \cellcolor{red!40}\href{../works/Groleaz21.pdf}{Groleaz21} (127.00)& \cellcolor{red!40}\href{../works/Lunardi20.pdf}{Lunardi20} (126.00)& \cellcolor{red!40}\href{../works/ZarandiASC20.pdf}{ZarandiASC20} (126.00)& \cellcolor{red!40}\href{../works/Astrand21.pdf}{Astrand21} (126.00)& \cellcolor{red!40}\href{../works/LaborieRSV18.pdf}{LaborieRSV18} (124.00)\\
Cosine& \cellcolor{red!40}\href{../works/Ham18.pdf}{Ham18} (0.83)& \cellcolor{red!40}\href{../works/HamP21.pdf}{HamP21} (0.81)& \cellcolor{red!40}\href{../works/Ham18a.pdf}{Ham18a} (0.76)& \cellcolor{red!40}\href{../works/HamFC17.pdf}{HamFC17} (0.75)& \cellcolor{red!40}\href{../works/BukchinR18.pdf}{BukchinR18} (0.74)\\
\index{HamC16}\href{../works/HamC16.pdf}{HamC16} R\&C& \cellcolor{red!20}\href{../works/LunardiBLRV20.pdf}{LunardiBLRV20} (0.88)& \cellcolor{yellow!20}\href{../works/HamFC17.pdf}{HamFC17} (0.91)& \cellcolor{yellow!20}\href{../works/ZhangW18.pdf}{ZhangW18} (0.92)& \cellcolor{yellow!20}\href{../works/LaborieRSV18.pdf}{LaborieRSV18} (0.92)& \cellcolor{yellow!20}\href{../works/MengZRZL20.pdf}{MengZRZL20} (0.93)\\
Euclid& \cellcolor{green!20}\href{../works/ArtiguesF07.pdf}{ArtiguesF07} (0.30)& \cellcolor{green!20}\href{../works/ArtiguesBF04.pdf}{ArtiguesBF04} (0.30)& \cellcolor{green!20}\href{../works/BillautHL12.pdf}{BillautHL12} (0.31)& \cellcolor{green!20}\href{../works/Ham18a.pdf}{Ham18a} (0.31)& \cellcolor{blue!20}\href{../works/Teppan22.pdf}{Teppan22} (0.32)\\
Dot& \cellcolor{red!40}\href{../works/ZarandiASC20.pdf}{ZarandiASC20} (145.00)& \cellcolor{red!40}\href{../works/Lunardi20.pdf}{Lunardi20} (136.00)& \cellcolor{red!40}\href{../works/MengZRZL20.pdf}{MengZRZL20} (131.00)& \cellcolor{red!40}\href{../works/Groleaz21.pdf}{Groleaz21} (127.00)& \cellcolor{red!40}\href{../works/Malapert11.pdf}{Malapert11} (123.00)\\
Cosine& \cellcolor{red!40}\href{../works/ArtiguesF07.pdf}{ArtiguesF07} (0.78)& \cellcolor{red!40}\href{../works/HamPK21.pdf}{HamPK21} (0.75)& \cellcolor{red!40}\href{../works/ArtiguesBF04.pdf}{ArtiguesBF04} (0.75)& \cellcolor{red!40}\href{../works/Ham18a.pdf}{Ham18a} (0.75)& \cellcolor{red!40}\href{../works/MengZRZL20.pdf}{MengZRZL20} (0.72)\\
\index{HamFC17}\href{../works/HamFC17.pdf}{HamFC17} R\&C& \cellcolor{red!20}\href{../works/TangB20.pdf}{TangB20} (0.88)& \cellcolor{red!20}\href{../works/Ham18a.pdf}{Ham18a} (0.89)& \cellcolor{yellow!20}\href{../works/HamC16.pdf}{HamC16} (0.91)& \cellcolor{yellow!20}\href{../works/LacknerMMWW23.pdf}{LacknerMMWW23} (0.92)& \cellcolor{yellow!20}\href{../works/MalapertGR12.pdf}{MalapertGR12} (0.92)\\
Euclid& \cellcolor{green!20}\href{../works/Ham18a.pdf}{Ham18a} (0.30)& \cellcolor{green!20}\href{../works/Ham20a.pdf}{Ham20a} (0.30)& \cellcolor{blue!20}\href{../works/KoschB14.pdf}{KoschB14} (0.32)& \cellcolor{blue!20}\href{../works/TanSD10.pdf}{TanSD10} (0.32)& \cellcolor{blue!20}\href{../works/HamC16.pdf}{HamC16} (0.33)\\
Dot& \cellcolor{red!40}\href{../works/ZarandiASC20.pdf}{ZarandiASC20} (142.00)& \cellcolor{red!40}\href{../works/Groleaz21.pdf}{Groleaz21} (133.00)& \cellcolor{red!40}\href{../works/Malapert11.pdf}{Malapert11} (128.00)& \cellcolor{red!40}\href{../works/AwadMDMT22.pdf}{AwadMDMT22} (128.00)& \cellcolor{red!40}\href{../works/Astrand21.pdf}{Astrand21} (122.00)\\
Cosine& \cellcolor{red!40}\href{../works/Ham18a.pdf}{Ham18a} (0.77)& \cellcolor{red!40}\href{../works/KoschB14.pdf}{KoschB14} (0.75)& \cellcolor{red!40}\href{../works/Ham20a.pdf}{Ham20a} (0.75)& \cellcolor{red!40}\href{../works/AwadMDMT22.pdf}{AwadMDMT22} (0.72)& \cellcolor{red!40}\href{../works/TanSD10.pdf}{TanSD10} (0.72)\\
\index{HamP21}\href{../works/HamP21.pdf}{HamP21} R\&C& \cellcolor{red!40}\href{../works/LaborieRSV18.pdf}{LaborieRSV18} (0.85)& \cellcolor{red!20}\href{../works/HeinzNVH22.pdf}{HeinzNVH22} (0.87)& \cellcolor{red!20}\href{../works/LunardiBLRV20.pdf}{LunardiBLRV20} (0.89)& \cellcolor{red!20}\href{../works/ColT2019a.pdf}{ColT2019a} (0.89)& \cellcolor{red!20}\href{../works/MurinR19.pdf}{MurinR19} (0.89)\\
Euclid& \cellcolor{yellow!20}\href{../works/Ham20a.pdf}{Ham20a} (0.27)& \cellcolor{green!20}\href{../works/Teppan22.pdf}{Teppan22} (0.29)& \cellcolor{green!20}\href{../works/HeipckeCCS00.pdf}{HeipckeCCS00} (0.30)& \cellcolor{green!20}\href{../works/LiFJZLL22.pdf}{LiFJZLL22} (0.31)& \cellcolor{green!20}\href{../works/NishikawaSTT18.pdf}{NishikawaSTT18} (0.31)\\
Dot& \cellcolor{red!40}\href{../works/Lunardi20.pdf}{Lunardi20} (140.00)& \cellcolor{red!40}\href{../works/ZarandiASC20.pdf}{ZarandiASC20} (133.00)& \cellcolor{red!40}\href{../works/Groleaz21.pdf}{Groleaz21} (128.00)& \cellcolor{red!40}\href{../works/LaborieRSV18.pdf}{LaborieRSV18} (127.00)& \cellcolor{red!40}\href{../works/ColT22.pdf}{ColT22} (127.00)\\
Cosine& \cellcolor{red!40}\href{../works/Ham20a.pdf}{Ham20a} (0.81)& \cellcolor{red!40}\href{../works/Teppan22.pdf}{Teppan22} (0.78)& \cellcolor{red!40}\href{../works/LiFJZLL22.pdf}{LiFJZLL22} (0.75)& \cellcolor{red!40}\href{../works/CarchraeB09.pdf}{CarchraeB09} (0.74)& \cellcolor{red!40}\href{../works/HeipckeCCS00.pdf}{HeipckeCCS00} (0.74)\\
\index{HamPK21}\href{../works/HamPK21.pdf}{HamPK21} R\&C& \cellcolor{red!40}\href{../works/MengLZB21.pdf}{MengLZB21} (0.85)& \cellcolor{red!20}\href{../works/HoYCLLCLC18.pdf}{HoYCLLCLC18} (0.88)& \cellcolor{red!20}\href{../works/MengZRZL20.pdf}{MengZRZL20} (0.89)& \cellcolor{yellow!20}\href{../works/ZhangYW21.pdf}{ZhangYW21} (0.90)& \cellcolor{yellow!20}\href{../works/LunardiBLRV20.pdf}{LunardiBLRV20} (0.92)\\
Euclid& \cellcolor{green!20}\href{../works/ZhangYW21.pdf}{ZhangYW21} (0.31)& \cellcolor{black!20}\href{../works/HamC16.pdf}{HamC16} (0.34)& \cellcolor{black!20}\href{../works/BillautHL12.pdf}{BillautHL12} (0.35)& \cellcolor{black!20}\href{../works/MengGRZSC22.pdf}{MengGRZSC22} (0.36)& \cellcolor{black!20}\href{../works/NaderiBZ22a.pdf}{NaderiBZ22a} (0.37)\\
Dot& \cellcolor{red!40}\href{../works/ZarandiASC20.pdf}{ZarandiASC20} (195.00)& \cellcolor{red!40}\href{../works/Lunardi20.pdf}{Lunardi20} (185.00)& \cellcolor{red!40}\href{../works/Groleaz21.pdf}{Groleaz21} (179.00)& \cellcolor{red!40}\href{../works/IsikYA23.pdf}{IsikYA23} (169.00)& \cellcolor{red!40}\href{../works/MengZRZL20.pdf}{MengZRZL20} (167.00)\\
Cosine& \cellcolor{red!40}\href{../works/ZhangYW21.pdf}{ZhangYW21} (0.79)& \cellcolor{red!40}\href{../works/MengZRZL20.pdf}{MengZRZL20} (0.78)& \cellcolor{red!40}\href{../works/MengGRZSC22.pdf}{MengGRZSC22} (0.77)& \cellcolor{red!40}\href{../works/HamC16.pdf}{HamC16} (0.75)& \cellcolor{red!40}\href{../works/NaderiBZ22a.pdf}{NaderiBZ22a} (0.75)\\
\index{HamdiL13}\href{../works/HamdiL13.pdf}{HamdiL13} R\&C& \cellcolor{red!40}\href{../works/CireCH13.pdf}{CireCH13} (0.42)& \cellcolor{red!40}\href{../works/CobanH10.pdf}{CobanH10} (0.71)& \cellcolor{red!40}\href{../works/BeniniLMMR08.pdf}{BeniniLMMR08} (0.73)& \cellcolor{red!40}\href{../works/ChuX05.pdf}{ChuX05} (0.74)& \cellcolor{red!40}\href{../works/CobanH11.pdf}{CobanH11} (0.75)\\
Euclid& \cellcolor{red!20}\href{../works/CobanH11.pdf}{CobanH11} (0.26)& \cellcolor{yellow!20}\href{../works/HookerO03.pdf}{HookerO03} (0.26)& \cellcolor{yellow!20}\href{../works/ChuX05.pdf}{ChuX05} (0.27)& \cellcolor{yellow!20}\href{../works/Hooker06.pdf}{Hooker06} (0.27)& \cellcolor{yellow!20}\href{../works/Hooker05a.pdf}{Hooker05a} (0.27)\\
Dot& \cellcolor{red!40}\href{../works/Groleaz21.pdf}{Groleaz21} (127.00)& \cellcolor{red!40}\href{../works/Baptiste02.pdf}{Baptiste02} (126.00)& \cellcolor{red!40}\href{../works/Lombardi10.pdf}{Lombardi10} (121.00)& \cellcolor{red!40}\href{../works/ZarandiASC20.pdf}{ZarandiASC20} (119.00)& \cellcolor{red!40}\href{../works/PrataAN23.pdf}{PrataAN23} (118.00)\\
Cosine& \cellcolor{red!40}\href{../works/CobanH11.pdf}{CobanH11} (0.82)& \cellcolor{red!40}\href{../works/Hooker06.pdf}{Hooker06} (0.79)& \cellcolor{red!40}\href{../works/HookerO03.pdf}{HookerO03} (0.78)& \cellcolor{red!40}\href{../works/ChuX05.pdf}{ChuX05} (0.78)& \cellcolor{red!40}\href{../works/Hooker05a.pdf}{Hooker05a} (0.76)\\
\index{Hamscher91}\href{../works/Hamscher91.pdf}{Hamscher91} R\&C\\
Euclid& \cellcolor{red!40}\href{../works/Baptiste09.pdf}{Baptiste09} (0.17)& \cellcolor{red!40}\href{../works/AbrilSB05.pdf}{AbrilSB05} (0.20)& \cellcolor{red!40}\href{../works/Layfield02.pdf}{Layfield02} (0.20)& \cellcolor{red!40}\href{../works/MaraveliasG04.pdf}{MaraveliasG04} (0.21)& \cellcolor{red!40}\href{../works/Davis87.pdf}{Davis87} (0.21)\\
Dot& \cellcolor{red!40}\href{../works/Froger16.pdf}{Froger16} (26.00)& \cellcolor{red!40}\href{../works/PopovicCGNC22.pdf}{PopovicCGNC22} (22.00)& \cellcolor{red!40}\href{../works/Timpe02.pdf}{Timpe02} (21.00)& \cellcolor{red!40}\href{../works/TerekhovDOB12.pdf}{TerekhovDOB12} (21.00)& \cellcolor{red!40}\href{../works/Jans09.pdf}{Jans09} (21.00)\\
Cosine& \cellcolor{red!40}\href{../works/Jans09.pdf}{Jans09} (0.44)& \cellcolor{red!40}\href{../works/RossiTHP07.pdf}{RossiTHP07} (0.44)& \cellcolor{red!40}\href{../works/PopovicCGNC22.pdf}{PopovicCGNC22} (0.44)& \cellcolor{red!40}\href{../works/HebrardALLCMR22.pdf}{HebrardALLCMR22} (0.42)& \cellcolor{red!40}\href{../works/Baptiste09.pdf}{Baptiste09} (0.42)\\
\index{HanenKP21}\href{../works/HanenKP21.pdf}{HanenKP21} R\&C& \cellcolor{red!40}\href{../works/Tesch18.pdf}{Tesch18} (0.79)& \cellcolor{red!40}\href{../works/CarlierPSJ20.pdf}{CarlierPSJ20} (0.83)& \cellcolor{red!40}\href{../works/OuelletQ18.pdf}{OuelletQ18} (0.83)& \cellcolor{red!20}CarlierSJP21 (0.90)& \cellcolor{yellow!20}\href{../works/FetgoD22.pdf}{FetgoD22} (0.91)\\
Euclid& \cellcolor{red!40}\href{../works/CarlierPSJ20.pdf}{CarlierPSJ20} (0.22)& \cellcolor{yellow!20}\href{../works/HeipckeCCS00.pdf}{HeipckeCCS00} (0.28)& \cellcolor{green!20}\href{../works/Limtanyakul07.pdf}{Limtanyakul07} (0.30)& \cellcolor{green!20}\href{../works/Tesch18.pdf}{Tesch18} (0.30)& \cellcolor{green!20}\href{../works/PoderBS04.pdf}{PoderBS04} (0.31)\\
Dot& \cellcolor{red!40}\href{../works/Baptiste02.pdf}{Baptiste02} (154.00)& \cellcolor{red!40}\href{../works/Lombardi10.pdf}{Lombardi10} (146.00)& \cellcolor{red!40}\href{../works/Groleaz21.pdf}{Groleaz21} (140.00)& \cellcolor{red!40}\href{../works/Dejemeppe16.pdf}{Dejemeppe16} (137.00)& \cellcolor{red!40}\href{../works/Fahimi16.pdf}{Fahimi16} (136.00)\\
Cosine& \cellcolor{red!40}\href{../works/CarlierPSJ20.pdf}{CarlierPSJ20} (0.87)& \cellcolor{red!40}\href{../works/HeipckeCCS00.pdf}{HeipckeCCS00} (0.78)& \cellcolor{red!40}\href{../works/Tesch18.pdf}{Tesch18} (0.78)& \cellcolor{red!40}\href{../works/BaptistePN99.pdf}{BaptistePN99} (0.77)& \cellcolor{red!40}\href{../works/ArkhipovBL19.pdf}{ArkhipovBL19} (0.77)\\
\index{HarjunkoskiG02}\href{../works/HarjunkoskiG02.pdf}{HarjunkoskiG02} R\&C& \cellcolor{red!40}\href{../works/JainG01.pdf}{JainG01} (0.74)& \cellcolor{red!40}\href{../works/RoePS05.pdf}{RoePS05} (0.75)& \cellcolor{red!40}\href{../works/MaraveliasCG04.pdf}{MaraveliasCG04} (0.77)& \cellcolor{red!40}\href{../works/HarjunkoskiJG00.pdf}{HarjunkoskiJG00} (0.81)& \cellcolor{red!20}\href{../works/Thorsteinsson01.pdf}{Thorsteinsson01} (0.87)\\
Euclid& \cellcolor{red!40}\href{../works/JainG01.pdf}{JainG01} (0.23)& \cellcolor{red!20}\href{../works/Colombani96.pdf}{Colombani96} (0.26)& \cellcolor{yellow!20}\href{../works/Limtanyakul07.pdf}{Limtanyakul07} (0.27)& \cellcolor{yellow!20}\href{../works/KrogtLPHJ07.pdf}{KrogtLPHJ07} (0.28)& \cellcolor{green!20}\href{../works/DavenportKRSH07.pdf}{DavenportKRSH07} (0.29)\\
Dot& \cellcolor{red!40}\href{../works/Baptiste02.pdf}{Baptiste02} (122.00)& \cellcolor{red!40}\href{../works/ZarandiASC20.pdf}{ZarandiASC20} (116.00)& \cellcolor{red!40}\href{../works/Dejemeppe16.pdf}{Dejemeppe16} (116.00)& \cellcolor{red!40}\href{../works/Malapert11.pdf}{Malapert11} (114.00)& \cellcolor{red!40}\href{../works/Beck99.pdf}{Beck99} (112.00)\\
Cosine& \cellcolor{red!40}\href{../works/JainG01.pdf}{JainG01} (0.85)& \cellcolor{red!40}\href{../works/Colombani96.pdf}{Colombani96} (0.79)& \cellcolor{red!40}\href{../works/KhayatLR06.pdf}{KhayatLR06} (0.75)& \cellcolor{red!40}\href{../works/DavenportKRSH07.pdf}{DavenportKRSH07} (0.74)& \cellcolor{red!40}\href{../works/Limtanyakul07.pdf}{Limtanyakul07} (0.74)\\
\index{HarjunkoskiJG00}\href{../works/HarjunkoskiJG00.pdf}{HarjunkoskiJG00} R\&C& \cellcolor{red!40}\href{../works/HarjunkoskiG02.pdf}{HarjunkoskiG02} (0.81)& \cellcolor{red!20}\href{../works/JainG01.pdf}{JainG01} (0.87)& \cellcolor{red!20}BockmayrK98 (0.89)& \cellcolor{yellow!20}\href{../works/PintoG97.pdf}{PintoG97} (0.92)& \cellcolor{yellow!20}\href{../works/RoePS05.pdf}{RoePS05} (0.92)\\
Euclid& \cellcolor{red!20}\href{../works/RenT09.pdf}{RenT09} (0.26)& \cellcolor{yellow!20}\href{../works/HebrardTW05.pdf}{HebrardTW05} (0.27)& \cellcolor{yellow!20}\href{../works/BofillGSV15.pdf}{BofillGSV15} (0.28)& \cellcolor{green!20}\href{../works/Davis87.pdf}{Davis87} (0.29)& \cellcolor{green!20}\href{../works/PesantGPR99.pdf}{PesantGPR99} (0.29)\\
Dot& \cellcolor{red!40}\href{../works/HarjunkoskiMBC14.pdf}{HarjunkoskiMBC14} (74.00)& \cellcolor{red!40}\href{../works/KanetAG04.pdf}{KanetAG04} (73.00)& \cellcolor{red!40}\href{../works/JainG01.pdf}{JainG01} (73.00)& \cellcolor{red!40}\href{../works/Malapert11.pdf}{Malapert11} (73.00)& \cellcolor{red!40}\href{../works/HookerO99.pdf}{HookerO99} (73.00)\\
Cosine& \cellcolor{red!40}\href{../works/HookerO99.pdf}{HookerO99} (0.72)& \cellcolor{red!40}\href{../works/RenT09.pdf}{RenT09} (0.72)& \cellcolor{red!40}\href{../works/JainG01.pdf}{JainG01} (0.71)& \cellcolor{red!40}\href{../works/HarjunkoskiG02.pdf}{HarjunkoskiG02} (0.68)& \cellcolor{red!40}\href{../works/HookerOTK00.pdf}{HookerOTK00} (0.63)\\
\index{HarjunkoskiMBC14}\href{../works/HarjunkoskiMBC14.pdf}{HarjunkoskiMBC14} R\&C& \cellcolor{yellow!20}\href{../works/NovaraNH16.pdf}{NovaraNH16} (0.90)& \cellcolor{yellow!20}\href{../works/ZeballosNH11.pdf}{ZeballosNH11} (0.91)& \cellcolor{yellow!20}\href{../works/MaraveliasCG04.pdf}{MaraveliasCG04} (0.92)& \cellcolor{yellow!20}\href{../works/NovasH10.pdf}{NovasH10} (0.92)& \cellcolor{yellow!20}\href{../works/ZeballosCM10.pdf}{ZeballosCM10} (0.93)\\
Euclid& \href{../works/NovasH10.pdf}{NovasH10} (0.48)& \href{../works/BidotVLB09.pdf}{BidotVLB09} (0.49)& \href{../works/LopesCSM10.pdf}{LopesCSM10} (0.50)& \href{../works/EscobetPQPRA19.pdf}{EscobetPQPRA19} (0.50)& \href{../works/BeckDDF98.pdf}{BeckDDF98} (0.50)\\
Dot& \cellcolor{red!40}\href{../works/ZarandiASC20.pdf}{ZarandiASC20} (231.00)& \cellcolor{red!40}\href{../works/Groleaz21.pdf}{Groleaz21} (231.00)& \cellcolor{red!40}\href{../works/Astrand21.pdf}{Astrand21} (204.00)& \cellcolor{red!40}\href{../works/Lombardi10.pdf}{Lombardi10} (200.00)& \cellcolor{red!40}\href{../works/Dejemeppe16.pdf}{Dejemeppe16} (199.00)\\
Cosine& \cellcolor{red!40}\href{../works/NovasH10.pdf}{NovasH10} (0.68)& \cellcolor{red!40}\href{../works/BeckDDF98.pdf}{BeckDDF98} (0.66)& \cellcolor{red!40}\href{../works/BidotVLB09.pdf}{BidotVLB09} (0.66)& \cellcolor{red!40}\href{../works/LopesCSM10.pdf}{LopesCSM10} (0.64)& \cellcolor{red!40}\href{../works/EscobetPQPRA19.pdf}{EscobetPQPRA19} (0.64)\\
\index{HauderBRPA20}\href{../works/HauderBRPA20.pdf}{HauderBRPA20} R\&C& \cellcolor{red!20}\href{../works/SchnellH17.pdf}{SchnellH17} (0.89)& \cellcolor{yellow!20}\href{../works/SubulanC22.pdf}{SubulanC22} (0.92)& \cellcolor{yellow!20}EdwardsBSE19 (0.92)& \cellcolor{yellow!20}\href{../works/KreterSSZ18.pdf}{KreterSSZ18} (0.92)& \cellcolor{yellow!20}\href{../works/SchnellH15.pdf}{SchnellH15} (0.93)\\
Euclid& \cellcolor{red!40}\href{../works/abs-1902-09244.pdf}{abs-1902-09244} (0.11)& \cellcolor{black!20}\href{../works/BeckPS03.pdf}{BeckPS03} (0.35)& \href{../works/MonetteDH09.pdf}{MonetteDH09} (0.38)& \href{../works/LaborieR14.pdf}{LaborieR14} (0.38)& \href{../works/BeckR03.pdf}{BeckR03} (0.38)\\
Dot& \cellcolor{red!40}\href{../works/ZarandiASC20.pdf}{ZarandiASC20} (202.00)& \cellcolor{red!40}\href{../works/abs-1902-09244.pdf}{abs-1902-09244} (200.00)& \cellcolor{red!40}\href{../works/Dejemeppe16.pdf}{Dejemeppe16} (184.00)& \cellcolor{red!40}\href{../works/Groleaz21.pdf}{Groleaz21} (183.00)& \cellcolor{red!40}\href{../works/LaborieRSV18.pdf}{LaborieRSV18} (175.00)\\
Cosine& \cellcolor{red!40}\href{../works/abs-1902-09244.pdf}{abs-1902-09244} (0.98)& \cellcolor{red!40}\href{../works/BeckPS03.pdf}{BeckPS03} (0.76)& \cellcolor{red!40}\href{../works/YuraszeckMCCR23.pdf}{YuraszeckMCCR23} (0.73)& \cellcolor{red!40}\href{../works/MonetteDH09.pdf}{MonetteDH09} (0.72)& \cellcolor{red!40}\href{../works/BeckR03.pdf}{BeckR03} (0.72)\\
\index{He0GLW18}\href{../works/He0GLW18.pdf}{He0GLW18} R\&C& \cellcolor{blue!20}\href{../works/MurphyMB15.pdf}{MurphyMB15} (0.98)\\
Euclid& \cellcolor{red!40}\href{../works/Hunsberger08.pdf}{Hunsberger08} (0.22)& \cellcolor{red!40}\href{../works/AbrilSB05.pdf}{AbrilSB05} (0.23)& \cellcolor{red!40}\href{../works/Baptiste09.pdf}{Baptiste09} (0.24)& \cellcolor{red!40}\href{../works/CarchraeBF05.pdf}{CarchraeBF05} (0.24)& \cellcolor{red!20}\href{../works/FukunagaHFAMN02.pdf}{FukunagaHFAMN02} (0.24)\\
Dot& \cellcolor{red!40}\href{../works/ZarandiASC20.pdf}{ZarandiASC20} (52.00)& \cellcolor{red!40}\href{../works/GombolayWS18.pdf}{GombolayWS18} (50.00)& \cellcolor{red!40}\href{../works/Lemos21.pdf}{Lemos21} (45.00)& \cellcolor{red!40}\href{../works/Zahout21.pdf}{Zahout21} (42.00)& \cellcolor{red!40}\href{../works/Beck99.pdf}{Beck99} (42.00)\\
Cosine& \cellcolor{red!40}\href{../works/Hunsberger08.pdf}{Hunsberger08} (0.60)& \cellcolor{red!40}\href{../works/HoeveGSL07.pdf}{HoeveGSL07} (0.59)& \cellcolor{red!40}\href{../works/SultanikMR07.pdf}{SultanikMR07} (0.57)& \cellcolor{red!40}\href{../works/LimHTB16.pdf}{LimHTB16} (0.52)& \cellcolor{red!40}\href{../works/FukunagaHFAMN02.pdf}{FukunagaHFAMN02} (0.51)\\
\index{HebrardALLCMR22}\href{../works/HebrardALLCMR22.pdf}{HebrardALLCMR22} R\&C\\
Euclid& \cellcolor{red!40}\href{../works/Baptiste09.pdf}{Baptiste09} (0.17)& \cellcolor{red!40}\href{../works/CarchraeBF05.pdf}{CarchraeBF05} (0.18)& \cellcolor{red!40}\href{../works/FrostD98.pdf}{FrostD98} (0.18)& \cellcolor{red!40}\href{../works/ZibranR11.pdf}{ZibranR11} (0.18)& \cellcolor{red!40}\href{../works/Davis87.pdf}{Davis87} (0.18)\\
Dot& \cellcolor{red!40}\href{../works/Malapert11.pdf}{Malapert11} (36.00)& \cellcolor{red!40}\href{../works/PapeB97.pdf}{PapeB97} (33.00)& \cellcolor{red!40}\href{../works/BaptisteP00.pdf}{BaptisteP00} (33.00)& \cellcolor{red!40}\href{../works/DerrienPZ14.pdf}{DerrienPZ14} (33.00)& \cellcolor{red!40}\href{../works/ClautiauxJCM08.pdf}{ClautiauxJCM08} (33.00)\\
Cosine& \cellcolor{red!40}\href{../works/ZibranR11.pdf}{ZibranR11} (0.67)& \cellcolor{red!40}\href{../works/ZibranR11a.pdf}{ZibranR11a} (0.63)& \cellcolor{red!40}\href{../works/WolfS05.pdf}{WolfS05} (0.63)& \cellcolor{red!40}\href{../works/Baptiste09.pdf}{Baptiste09} (0.62)& \cellcolor{red!40}\href{../works/PoderB08.pdf}{PoderB08} (0.62)\\
\index{HebrardHJMPV16}\href{../works/HebrardHJMPV16.pdf}{HebrardHJMPV16} R\&C& \cellcolor{yellow!20}\href{../works/ElkhyariGJ02.pdf}{ElkhyariGJ02} (0.93)& \cellcolor{green!20}\href{../works/BarlattCG08.pdf}{BarlattCG08} (0.94)& \cellcolor{green!20}\href{../works/BertholdHLMS10.pdf}{BertholdHLMS10} (0.94)& \cellcolor{green!20}\href{../works/WikarekS19.pdf}{WikarekS19} (0.95)& \cellcolor{green!20}\href{../works/SchuttW10.pdf}{SchuttW10} (0.95)\\
Euclid& \cellcolor{yellow!20}\href{../works/DoRZ08.pdf}{DoRZ08} (0.27)& \cellcolor{yellow!20}\href{../works/CrawfordB94.pdf}{CrawfordB94} (0.28)& \cellcolor{green!20}\href{../works/LauLN08.pdf}{LauLN08} (0.31)& \cellcolor{green!20}\href{../works/LahimerLH11.pdf}{LahimerLH11} (0.31)& \cellcolor{green!20}\href{../works/BridiLBBM16.pdf}{BridiLBBM16} (0.31)\\
Dot& \cellcolor{red!40}\href{../works/Godet21a.pdf}{Godet21a} (92.00)& \cellcolor{red!40}\href{../works/GodetLHS20.pdf}{GodetLHS20} (90.00)& \cellcolor{red!40}\href{../works/Groleaz21.pdf}{Groleaz21} (90.00)& \cellcolor{red!40}\href{../works/Baptiste02.pdf}{Baptiste02} (88.00)& \cellcolor{red!40}\href{../works/Lunardi20.pdf}{Lunardi20} (87.00)\\
Cosine& \cellcolor{red!40}\href{../works/DoRZ08.pdf}{DoRZ08} (0.70)& \cellcolor{red!40}\href{../works/GodetLHS20.pdf}{GodetLHS20} (0.69)& \cellcolor{red!40}\href{../works/Ham18a.pdf}{Ham18a} (0.67)& \cellcolor{red!40}\href{../works/CrawfordB94.pdf}{CrawfordB94} (0.66)& \cellcolor{red!40}\href{../works/ParkUJR19.pdf}{ParkUJR19} (0.65)\\
\index{HebrardTW05}\href{../works/HebrardTW05.pdf}{HebrardTW05} R\&C& \cellcolor{red!40}\href{../works/Puget95.pdf}{Puget95} (0.80)& \cellcolor{red!40}\href{../works/BeckF00a.pdf}{BeckF00a} (0.85)& \cellcolor{red!40}\href{../works/VilimBC04.pdf}{VilimBC04} (0.86)& \cellcolor{red!20}\href{../works/VilimBC05.pdf}{VilimBC05} (0.88)& \cellcolor{red!20}\href{../works/Vilim04.pdf}{Vilim04} (0.88)\\
Euclid& \cellcolor{red!40}\href{../works/AngelsmarkJ00.pdf}{AngelsmarkJ00} (0.12)& \cellcolor{red!40}\href{../works/Baptiste09.pdf}{Baptiste09} (0.14)& \cellcolor{red!40}\href{../works/KovacsEKV05.pdf}{KovacsEKV05} (0.14)& \cellcolor{red!40}\href{../works/CarchraeBF05.pdf}{CarchraeBF05} (0.15)& \cellcolor{red!40}\href{../works/Davis87.pdf}{Davis87} (0.15)\\
Dot& \cellcolor{red!40}\href{../works/abs-2211-14492.pdf}{abs-2211-14492} (33.00)& \cellcolor{red!40}\href{../works/KanetAG04.pdf}{KanetAG04} (33.00)& \cellcolor{red!40}\href{../works/HeckmanB11.pdf}{HeckmanB11} (33.00)& \cellcolor{red!40}\href{../works/ZarandiASC20.pdf}{ZarandiASC20} (33.00)& \cellcolor{red!40}\href{../works/Siala15a.pdf}{Siala15a} (33.00)\\
Cosine& \cellcolor{red!40}\href{../works/Colombani96.pdf}{Colombani96} (0.75)& \cellcolor{red!40}\href{../works/CrawfordB94.pdf}{CrawfordB94} (0.74)& \cellcolor{red!40}\href{../works/Beck06.pdf}{Beck06} (0.74)& \cellcolor{red!40}\href{../works/RenT09.pdf}{RenT09} (0.74)& \cellcolor{red!40}\href{../works/WatsonB08.pdf}{WatsonB08} (0.73)\\
\index{HechingH16}\href{../works/HechingH16.pdf}{HechingH16} R\&C& \cellcolor{red!40}\href{../works/HamdiL13.pdf}{HamdiL13} (0.83)& \cellcolor{red!40}\href{../works/CireCH13.pdf}{CireCH13} (0.85)& \cellcolor{red!20}\href{../works/Beck10.pdf}{Beck10} (0.90)& \cellcolor{yellow!20}\href{../works/Sadykov04.pdf}{Sadykov04} (0.90)& \cellcolor{yellow!20}\href{../works/BeniniLMMR08.pdf}{BeniniLMMR08} (0.91)\\
Euclid& \cellcolor{red!20}\href{../works/CireCH13.pdf}{CireCH13} (0.24)& \cellcolor{red!20}\href{../works/CireCH16.pdf}{CireCH16} (0.26)& \cellcolor{yellow!20}\href{../works/CambazardJ05.pdf}{CambazardJ05} (0.28)& \cellcolor{yellow!20}\href{../works/HookerO03.pdf}{HookerO03} (0.28)& \cellcolor{green!20}\href{../works/Hooker05b.pdf}{Hooker05b} (0.29)\\
Dot& \cellcolor{red!40}\href{../works/Lombardi10.pdf}{Lombardi10} (73.00)& \cellcolor{red!40}\href{../works/NaderiBZR23.pdf}{NaderiBZR23} (73.00)& \cellcolor{red!40}\href{../works/RoshanaeiBAUB20.pdf}{RoshanaeiBAUB20} (68.00)& \cellcolor{red!40}\href{../works/Dejemeppe16.pdf}{Dejemeppe16} (68.00)& \cellcolor{red!40}\href{../works/RoshanaeiN21.pdf}{RoshanaeiN21} (66.00)\\
Cosine& \cellcolor{red!40}\href{../works/CireCH13.pdf}{CireCH13} (0.74)& \cellcolor{red!40}\href{../works/CireCH16.pdf}{CireCH16} (0.72)& \cellcolor{red!40}\href{../works/NaderiBZR23.pdf}{NaderiBZR23} (0.66)& \cellcolor{red!40}\href{../works/HookerO03.pdf}{HookerO03} (0.64)& \cellcolor{red!40}\href{../works/CobanH11.pdf}{CobanH11} (0.64)\\
\index{HechingHK19}HechingHK19 R\&C& \cellcolor{red!40}\href{../works/CireCH16.pdf}{CireCH16} (0.74)& \cellcolor{red!40}ZarandiB12 (0.74)& \cellcolor{red!40}\href{../works/TranAB16.pdf}{TranAB16} (0.75)& \cellcolor{red!40}\href{../works/Beck10.pdf}{Beck10} (0.81)& \cellcolor{red!40}\href{../works/Hooker07.pdf}{Hooker07} (0.83)\\
Euclid\\
Dot\\
Cosine\\
\index{HeckmanB11}\href{../works/HeckmanB11.pdf}{HeckmanB11} R\&C& \cellcolor{red!40}\href{../works/BeckFW11.pdf}{BeckFW11} (0.80)& \cellcolor{red!40}\href{../works/WatsonB08.pdf}{WatsonB08} (0.82)& \cellcolor{red!20}\href{../works/GrimesHM09.pdf}{GrimesHM09} (0.88)& \cellcolor{red!20}\href{../works/ChenGPSH10.pdf}{ChenGPSH10} (0.89)& \cellcolor{yellow!20}\href{../works/MenciaSV12.pdf}{MenciaSV12} (0.92)\\
Euclid& \cellcolor{red!40}\href{../works/Beck07.pdf}{Beck07} (0.22)& \cellcolor{red!40}\href{../works/Beck06.pdf}{Beck06} (0.23)& \cellcolor{red!40}\href{../works/WatsonB08.pdf}{WatsonB08} (0.24)& \cellcolor{red!40}\href{../works/BeckDSF97a.pdf}{BeckDSF97a} (0.24)& \cellcolor{red!20}\href{../works/BeckPS03.pdf}{BeckPS03} (0.24)\\
Dot& \cellcolor{red!40}\href{../works/ZarandiASC20.pdf}{ZarandiASC20} (134.00)& \cellcolor{red!40}\href{../works/Beck99.pdf}{Beck99} (134.00)& \cellcolor{red!40}\href{../works/Dejemeppe16.pdf}{Dejemeppe16} (133.00)& \cellcolor{red!40}\href{../works/Lombardi10.pdf}{Lombardi10} (128.00)& \cellcolor{red!40}\href{../works/Baptiste02.pdf}{Baptiste02} (128.00)\\
Cosine& \cellcolor{red!40}\href{../works/Beck07.pdf}{Beck07} (0.87)& \cellcolor{red!40}\href{../works/BeckPS03.pdf}{BeckPS03} (0.84)& \cellcolor{red!40}\href{../works/Beck06.pdf}{Beck06} (0.83)& \cellcolor{red!40}\href{../works/WatsonB08.pdf}{WatsonB08} (0.83)& \cellcolor{red!40}\href{../works/BeckDSF97a.pdf}{BeckDSF97a} (0.83)\\
\index{HeinzB12}\href{../works/HeinzB12.pdf}{HeinzB12} R\&C& \cellcolor{red!40}\href{../works/HeinzKB13.pdf}{HeinzKB13} (0.58)& \cellcolor{red!40}\href{../works/HeinzS11.pdf}{HeinzS11} (0.69)& \cellcolor{red!40}\href{../works/CobanH10.pdf}{CobanH10} (0.82)& \cellcolor{red!40}\href{../works/CireCH16.pdf}{CireCH16} (0.84)& \cellcolor{red!20}\href{../works/Hooker05a.pdf}{Hooker05a} (0.86)\\
Euclid& \cellcolor{red!40}\href{../works/HeinzKB13.pdf}{HeinzKB13} (0.20)& \cellcolor{red!20}\href{../works/CireCH13.pdf}{CireCH13} (0.26)& \cellcolor{green!20}\href{../works/HookerY02.pdf}{HookerY02} (0.29)& \cellcolor{green!20}\href{../works/ChuX05.pdf}{ChuX05} (0.29)& \cellcolor{green!20}\href{../works/Beck10.pdf}{Beck10} (0.31)\\
Dot& \cellcolor{red!40}\href{../works/Lombardi10.pdf}{Lombardi10} (113.00)& \cellcolor{red!40}\href{../works/MilanoW09.pdf}{MilanoW09} (106.00)& \cellcolor{red!40}\href{../works/Groleaz21.pdf}{Groleaz21} (106.00)& \cellcolor{red!40}\href{../works/LaborieRSV18.pdf}{LaborieRSV18} (103.00)& \cellcolor{red!40}\href{../works/HookerH17.pdf}{HookerH17} (103.00)\\
Cosine& \cellcolor{red!40}\href{../works/HeinzKB13.pdf}{HeinzKB13} (0.88)& \cellcolor{red!40}\href{../works/CireCH13.pdf}{CireCH13} (0.79)& \cellcolor{red!40}\href{../works/HookerY02.pdf}{HookerY02} (0.73)& \cellcolor{red!40}\href{../works/ChuX05.pdf}{ChuX05} (0.73)& \cellcolor{red!40}\href{../works/Hooker05.pdf}{Hooker05} (0.73)\\
\index{HeinzKB13}\href{../works/HeinzKB13.pdf}{HeinzKB13} R\&C& \cellcolor{red!40}\href{../works/HeinzB12.pdf}{HeinzB12} (0.58)& \cellcolor{red!40}\href{../works/CireCH16.pdf}{CireCH16} (0.79)& \cellcolor{red!40}\href{../works/CireCH13.pdf}{CireCH13} (0.84)& \cellcolor{red!40}\href{../works/HeinzSB13.pdf}{HeinzSB13} (0.85)& \cellcolor{red!20}\href{../works/TranAB16.pdf}{TranAB16} (0.87)\\
Euclid& \cellcolor{red!40}\href{../works/CireCH13.pdf}{CireCH13} (0.18)& \cellcolor{red!40}\href{../works/HeinzB12.pdf}{HeinzB12} (0.20)& \cellcolor{red!40}\href{../works/HookerY02.pdf}{HookerY02} (0.21)& \cellcolor{red!40}\href{../works/Beck10.pdf}{Beck10} (0.23)& \cellcolor{red!40}\href{../works/ChuX05.pdf}{ChuX05} (0.24)\\
Dot& \cellcolor{red!40}\href{../works/MilanoW09.pdf}{MilanoW09} (92.00)& \cellcolor{red!40}\href{../works/HookerH17.pdf}{HookerH17} (91.00)& \cellcolor{red!40}\href{../works/Lombardi10.pdf}{Lombardi10} (91.00)& \cellcolor{red!40}\href{../works/HeinzB12.pdf}{HeinzB12} (91.00)& \cellcolor{red!40}\href{../works/BajestaniB13.pdf}{BajestaniB13} (89.00)\\
Cosine& \cellcolor{red!40}\href{../works/HeinzB12.pdf}{HeinzB12} (0.88)& \cellcolor{red!40}\href{../works/CireCH13.pdf}{CireCH13} (0.86)& \cellcolor{red!40}\href{../works/HookerY02.pdf}{HookerY02} (0.82)& \cellcolor{red!40}\href{../works/Beck10.pdf}{Beck10} (0.81)& \cellcolor{red!40}\href{../works/ChuX05.pdf}{ChuX05} (0.77)\\
\index{HeinzNVH22}\href{../works/HeinzNVH22.pdf}{HeinzNVH22} R\&C& \cellcolor{red!40}\href{../works/BenediktMH20.pdf}{BenediktMH20} (0.81)& \cellcolor{red!40}\href{../works/MurinR19.pdf}{MurinR19} (0.81)& \cellcolor{red!40}\href{../works/AbreuNP23.pdf}{AbreuNP23} (0.83)& \cellcolor{red!40}\href{../works/AwadMDMT22.pdf}{AwadMDMT22} (0.86)& \cellcolor{red!20}\href{../works/HamP21.pdf}{HamP21} (0.87)\\
Euclid& \cellcolor{red!40}\href{../works/abs-2305-19888.pdf}{abs-2305-19888} (0.13)& \cellcolor{green!20}\href{../works/ArbaouiY18.pdf}{ArbaouiY18} (0.31)& \cellcolor{blue!20}\href{../works/BenderWS21.pdf}{BenderWS21} (0.32)& \cellcolor{blue!20}\href{../works/EdisO11.pdf}{EdisO11} (0.33)& \cellcolor{blue!20}\href{../works/JuvinHL23.pdf}{JuvinHL23} (0.34)\\
Dot& \cellcolor{red!40}\href{../works/ZarandiASC20.pdf}{ZarandiASC20} (154.00)& \cellcolor{red!40}\href{../works/Lunardi20.pdf}{Lunardi20} (153.00)& \cellcolor{red!40}\href{../works/Groleaz21.pdf}{Groleaz21} (153.00)& \cellcolor{red!40}\href{../works/Astrand21.pdf}{Astrand21} (147.00)& \cellcolor{red!40}\href{../works/IsikYA23.pdf}{IsikYA23} (145.00)\\
Cosine& \cellcolor{red!40}\href{../works/abs-2305-19888.pdf}{abs-2305-19888} (0.96)& \cellcolor{red!40}\href{../works/ArbaouiY18.pdf}{ArbaouiY18} (0.75)& \cellcolor{red!40}\href{../works/GedikKEK18.pdf}{GedikKEK18} (0.73)& \cellcolor{red!40}\href{../works/OujanaAYB22.pdf}{OujanaAYB22} (0.72)& \cellcolor{red!40}\href{../works/BenderWS21.pdf}{BenderWS21} (0.72)\\
\index{HeinzS11}\href{../works/HeinzS11.pdf}{HeinzS11} R\&C& \cellcolor{red!40}\href{../works/BertholdHLMS10.pdf}{BertholdHLMS10} (0.65)& \cellcolor{red!40}\href{../works/HeinzB12.pdf}{HeinzB12} (0.69)& \cellcolor{red!40}\href{../works/HeinzSB13.pdf}{HeinzSB13} (0.73)& \cellcolor{red!40}\href{../works/SchuttFS13a.pdf}{SchuttFS13a} (0.83)& \cellcolor{red!40}\href{../works/SchuttW10.pdf}{SchuttW10} (0.85)\\
Euclid& \cellcolor{red!40}\href{../works/BertholdHLMS10.pdf}{BertholdHLMS10} (0.23)& \cellcolor{yellow!20}\href{../works/Tesch16.pdf}{Tesch16} (0.27)& \cellcolor{yellow!20}\href{../works/CauwelaertLS15.pdf}{CauwelaertLS15} (0.28)& \cellcolor{green!20}\href{../works/OuelletQ18.pdf}{OuelletQ18} (0.29)& \cellcolor{green!20}\href{../works/HookerY02.pdf}{HookerY02} (0.31)\\
Dot& \cellcolor{red!40}\href{../works/Schutt11.pdf}{Schutt11} (106.00)& \cellcolor{red!40}\href{../works/Lombardi10.pdf}{Lombardi10} (102.00)& \cellcolor{red!40}\href{../works/Godet21a.pdf}{Godet21a} (96.00)& \cellcolor{red!40}\href{../works/Fahimi16.pdf}{Fahimi16} (92.00)& \cellcolor{red!40}\href{../works/HeinzSB13.pdf}{HeinzSB13} (91.00)\\
Cosine& \cellcolor{red!40}\href{../works/BertholdHLMS10.pdf}{BertholdHLMS10} (0.79)& \cellcolor{red!40}\href{../works/Tesch16.pdf}{Tesch16} (0.75)& \cellcolor{red!40}\href{../works/OuelletQ18.pdf}{OuelletQ18} (0.74)& \cellcolor{red!40}\href{../works/HeinzSB13.pdf}{HeinzSB13} (0.70)& \cellcolor{red!40}\href{../works/CauwelaertLS15.pdf}{CauwelaertLS15} (0.70)\\
\index{HeinzSB13}\href{../works/HeinzSB13.pdf}{HeinzSB13} R\&C& \cellcolor{red!40}\href{../works/HeinzS11.pdf}{HeinzS11} (0.73)& \cellcolor{red!40}\href{../works/SchuttFS13a.pdf}{SchuttFS13a} (0.84)& \cellcolor{red!40}\href{../works/BertholdHLMS10.pdf}{BertholdHLMS10} (0.85)& \cellcolor{red!40}\href{../works/HeinzKB13.pdf}{HeinzKB13} (0.85)& \cellcolor{red!20}\href{../works/HeinzB12.pdf}{HeinzB12} (0.88)\\
Euclid& \cellcolor{blue!20}\href{../works/BertholdHLMS10.pdf}{BertholdHLMS10} (0.32)& \cellcolor{black!20}\href{../works/HeipckeCCS00.pdf}{HeipckeCCS00} (0.35)& \cellcolor{black!20}\href{../works/HeinzS11.pdf}{HeinzS11} (0.36)& \cellcolor{black!20}\href{../works/SchnellH17.pdf}{SchnellH17} (0.36)& \cellcolor{black!20}\href{../works/DemasseyAM05.pdf}{DemasseyAM05} (0.37)\\
Dot& \cellcolor{red!40}\href{../works/Groleaz21.pdf}{Groleaz21} (157.00)& \cellcolor{red!40}\href{../works/Godet21a.pdf}{Godet21a} (156.00)& \cellcolor{red!40}\href{../works/Dejemeppe16.pdf}{Dejemeppe16} (147.00)& \cellcolor{red!40}\href{../works/Schutt11.pdf}{Schutt11} (145.00)& \cellcolor{red!40}\href{../works/Lombardi10.pdf}{Lombardi10} (143.00)\\
Cosine& \cellcolor{red!40}\href{../works/BertholdHLMS10.pdf}{BertholdHLMS10} (0.77)& \cellcolor{red!40}\href{../works/HeipckeCCS00.pdf}{HeipckeCCS00} (0.72)& \cellcolor{red!40}\href{../works/SchnellH17.pdf}{SchnellH17} (0.71)& \cellcolor{red!40}\href{../works/ArkhipovBL19.pdf}{ArkhipovBL19} (0.71)& \cellcolor{red!40}\href{../works/DemasseyAM05.pdf}{DemasseyAM05} (0.71)\\
\index{HeinzSSW12}\href{../works/HeinzSSW12.pdf}{HeinzSSW12} R\&C& \cellcolor{red!40}\href{../works/GarganiR07.pdf}{GarganiR07} (0.60)& \cellcolor{red!40}\href{../works/SchausHMCMD11.pdf}{SchausHMCMD11} (0.69)& \cellcolor{red!40}\href{../works/HentenryckM08.pdf}{HentenryckM08} (0.70)& \cellcolor{red!20}\href{../works/GaySS14.pdf}{GaySS14} (0.90)& \cellcolor{yellow!20}\href{../works/LetortCB13.pdf}{LetortCB13} (0.92)\\
Euclid& \cellcolor{red!40}\href{../works/HentenryckM08.pdf}{HentenryckM08} (0.19)& \cellcolor{red!40}\href{../works/GarganiR07.pdf}{GarganiR07} (0.21)& \cellcolor{yellow!20}\href{../works/Davis87.pdf}{Davis87} (0.26)& \cellcolor{yellow!20}\href{../works/ZibranR11.pdf}{ZibranR11} (0.27)& \cellcolor{yellow!20}\href{../works/ChapadosJR11.pdf}{ChapadosJR11} (0.27)\\
Dot& \cellcolor{red!40}\href{../works/SchausHMCMD11.pdf}{SchausHMCMD11} (52.00)& \cellcolor{red!40}\href{../works/GarganiR07.pdf}{GarganiR07} (51.00)& \cellcolor{red!40}\href{../works/Letort13.pdf}{Letort13} (46.00)& \cellcolor{red!40}\href{../works/Lombardi10.pdf}{Lombardi10} (45.00)& \cellcolor{red!40}\href{../works/Malapert11.pdf}{Malapert11} (44.00)\\
Cosine& \cellcolor{red!40}\href{../works/GarganiR07.pdf}{GarganiR07} (0.78)& \cellcolor{red!40}\href{../works/HentenryckM08.pdf}{HentenryckM08} (0.74)& \cellcolor{red!40}\href{../works/SchausHMCMD11.pdf}{SchausHMCMD11} (0.67)& \cellcolor{red!40}\href{../works/LiuLH19a.pdf}{LiuLH19a} (0.52)& \cellcolor{red!40}\href{../works/SchausD08.pdf}{SchausD08} (0.50)\\
\index{HeipckeCCS00}\href{../works/HeipckeCCS00.pdf}{HeipckeCCS00} R\&C& \cellcolor{green!20}\href{../works/SimonisC95.pdf}{SimonisC95} (0.95)& \cellcolor{black!20}\href{../works/BruckerK00.pdf}{BruckerK00} (0.99)& \cellcolor{black!20}\href{../works/AggounB93.pdf}{AggounB93} (0.99)\\
Euclid& \cellcolor{red!40}\href{../works/KovacsV06.pdf}{KovacsV06} (0.21)& \cellcolor{red!40}\href{../works/LiessM08.pdf}{LiessM08} (0.23)& \cellcolor{red!40}\href{../works/KovacsV04.pdf}{KovacsV04} (0.24)& \cellcolor{red!20}\href{../works/Limtanyakul07.pdf}{Limtanyakul07} (0.26)& \cellcolor{red!20}\href{../works/KhayatLR06.pdf}{KhayatLR06} (0.26)\\
Dot& \cellcolor{red!40}\href{../works/Baptiste02.pdf}{Baptiste02} (131.00)& \cellcolor{red!40}\href{../works/Groleaz21.pdf}{Groleaz21} (129.00)& \cellcolor{red!40}\href{../works/Lombardi10.pdf}{Lombardi10} (128.00)& \cellcolor{red!40}\href{../works/Godet21a.pdf}{Godet21a} (128.00)& \cellcolor{red!40}\href{../works/Dejemeppe16.pdf}{Dejemeppe16} (126.00)\\
Cosine& \cellcolor{red!40}\href{../works/KovacsV06.pdf}{KovacsV06} (0.85)& \cellcolor{red!40}\href{../works/LiessM08.pdf}{LiessM08} (0.83)& \cellcolor{red!40}\href{../works/VilimLS15.pdf}{VilimLS15} (0.81)& \cellcolor{red!40}\href{../works/KovacsV04.pdf}{KovacsV04} (0.81)& \cellcolor{red!40}\href{../works/KhayatLR06.pdf}{KhayatLR06} (0.80)\\
\index{HentenryckM04}\href{../works/HentenryckM04.pdf}{HentenryckM04} R\&C& \cellcolor{yellow!20}\href{../works/KhemmoudjPB06.pdf}{KhemmoudjPB06} (0.91)& \cellcolor{yellow!20}\href{../works/TanSD10.pdf}{TanSD10} (0.92)& \cellcolor{yellow!20}\href{../works/ArtiguesBF04.pdf}{ArtiguesBF04} (0.92)& \cellcolor{green!20}\href{../works/WatsonB08.pdf}{WatsonB08} (0.94)& \cellcolor{green!20}CestaOPS14 (0.94)\\
Euclid& \cellcolor{red!40}\href{../works/GodardLN05.pdf}{GodardLN05} (0.23)& \cellcolor{red!40}\href{../works/PacinoH11.pdf}{PacinoH11} (0.23)& \cellcolor{red!20}\href{../works/Beck07.pdf}{Beck07} (0.24)& \cellcolor{red!20}\href{../works/BeckPS03.pdf}{BeckPS03} (0.25)& \cellcolor{red!20}\href{../works/VilimBC04.pdf}{VilimBC04} (0.26)\\
Dot& \cellcolor{red!40}\href{../works/Dejemeppe16.pdf}{Dejemeppe16} (135.00)& \cellcolor{red!40}\href{../works/Groleaz21.pdf}{Groleaz21} (133.00)& \cellcolor{red!40}\href{../works/Baptiste02.pdf}{Baptiste02} (133.00)& \cellcolor{red!40}\href{../works/Malapert11.pdf}{Malapert11} (131.00)& \cellcolor{red!40}\href{../works/Fahimi16.pdf}{Fahimi16} (130.00)\\
Cosine& \cellcolor{red!40}\href{../works/GodardLN05.pdf}{GodardLN05} (0.85)& \cellcolor{red!40}\href{../works/Beck07.pdf}{Beck07} (0.83)& \cellcolor{red!40}\href{../works/PacinoH11.pdf}{PacinoH11} (0.83)& \cellcolor{red!40}\href{../works/BeckPS03.pdf}{BeckPS03} (0.82)& \cellcolor{red!40}\href{../works/MonetteDH09.pdf}{MonetteDH09} (0.80)\\
\index{HentenryckM08}\href{../works/HentenryckM08.pdf}{HentenryckM08} R\&C& \cellcolor{red!40}\href{../works/GarganiR07.pdf}{GarganiR07} (0.52)& \cellcolor{red!40}\href{../works/SchausHMCMD11.pdf}{SchausHMCMD11} (0.64)& \cellcolor{red!40}\href{../works/HeinzSSW12.pdf}{HeinzSSW12} (0.70)& \cellcolor{yellow!20}\href{../works/GaySS14.pdf}{GaySS14} (0.93)& \cellcolor{yellow!20}\href{../works/Beck10.pdf}{Beck10} (0.93)\\
Euclid& \cellcolor{red!40}\href{../works/HeinzSSW12.pdf}{HeinzSSW12} (0.19)& \cellcolor{red!40}\href{../works/GarganiR07.pdf}{GarganiR07} (0.20)& \cellcolor{red!40}\href{../works/AbrilSB05.pdf}{AbrilSB05} (0.22)& \cellcolor{red!40}\href{../works/FrostD98.pdf}{FrostD98} (0.23)& \cellcolor{red!40}\href{../works/SmithBHW96.pdf}{SmithBHW96} (0.23)\\
Dot& \cellcolor{red!40}\href{../works/SchausHMCMD11.pdf}{SchausHMCMD11} (48.00)& \cellcolor{red!40}\href{../works/GaySS14.pdf}{GaySS14} (46.00)& \cellcolor{red!40}\href{../works/GarganiR07.pdf}{GarganiR07} (44.00)& \cellcolor{red!40}\href{../works/HeinzSSW12.pdf}{HeinzSSW12} (36.00)& \cellcolor{red!40}\href{../works/Dejemeppe16.pdf}{Dejemeppe16} (36.00)\\
Cosine& \cellcolor{red!40}\href{../works/GarganiR07.pdf}{GarganiR07} (0.79)& \cellcolor{red!40}\href{../works/HeinzSSW12.pdf}{HeinzSSW12} (0.74)& \cellcolor{red!40}\href{../works/SchausHMCMD11.pdf}{SchausHMCMD11} (0.73)& \cellcolor{red!40}\href{../works/GaySS14.pdf}{GaySS14} (0.62)& \cellcolor{red!40}\href{../works/LiuLH19a.pdf}{LiuLH19a} (0.55)\\
\index{Henz01}Henz01 R\&C& \cellcolor{red!40}\href{../works/HenzMT04.pdf}{HenzMT04} (0.56)& \cellcolor{red!40}\href{../works/RasmussenT07.pdf}{RasmussenT07} (0.73)& \cellcolor{red!40}\href{../works/Trick03.pdf}{Trick03} (0.73)& \cellcolor{red!40}\href{../works/RussellU06.pdf}{RussellU06} (0.79)& \cellcolor{red!40}\href{../works/ElfJR03.pdf}{ElfJR03} (0.83)\\
Euclid\\
Dot\\
Cosine\\
\index{HenzMT04}\href{../works/HenzMT04.pdf}{HenzMT04} R\&C& \cellcolor{red!40}Henz01 (0.56)& \cellcolor{red!40}\href{../works/Trick03.pdf}{Trick03} (0.69)& \cellcolor{red!40}\href{../works/RussellU06.pdf}{RussellU06} (0.71)& \cellcolor{red!40}\href{../works/RasmussenT07.pdf}{RasmussenT07} (0.73)& \cellcolor{red!40}\href{../works/Perron05.pdf}{Perron05} (0.80)\\
Euclid& \cellcolor{red!40}\href{../works/EastonNT02.pdf}{EastonNT02} (0.21)& \cellcolor{red!40}\href{../works/BandaSC11.pdf}{BandaSC11} (0.24)& \cellcolor{red!20}\href{../works/SuCC13.pdf}{SuCC13} (0.24)& \cellcolor{red!20}\href{../works/DilkinaH04.pdf}{DilkinaH04} (0.25)& \cellcolor{red!20}\href{../works/LiuLH19.pdf}{LiuLH19} (0.25)\\
Dot& \cellcolor{red!40}\href{../works/KendallKRU10.pdf}{KendallKRU10} (69.00)& \cellcolor{red!40}\href{../works/ZarandiASC20.pdf}{ZarandiASC20} (64.00)& \cellcolor{red!40}\href{../works/RussellU06.pdf}{RussellU06} (63.00)& \cellcolor{red!40}\href{../works/Lemos21.pdf}{Lemos21} (61.00)& \cellcolor{red!40}\href{../works/Siala15a.pdf}{Siala15a} (59.00)\\
Cosine& \cellcolor{red!40}\href{../works/EastonNT02.pdf}{EastonNT02} (0.76)& \cellcolor{red!40}\href{../works/RussellU06.pdf}{RussellU06} (0.74)& \cellcolor{red!40}\href{../works/LiuLH19.pdf}{LiuLH19} (0.70)& \cellcolor{red!40}\href{../works/BulckG22.pdf}{BulckG22} (0.68)& \cellcolor{red!40}\href{../works/SuCC13.pdf}{SuCC13} (0.68)\\
\index{HermenierDL11}\href{../works/HermenierDL11.pdf}{HermenierDL11} R\&C& \cellcolor{red!20}\href{../works/Simonis95.pdf}{Simonis95} (0.88)& \cellcolor{yellow!20}\href{../works/PoderBS04.pdf}{PoderBS04} (0.92)& \cellcolor{yellow!20}\href{../works/SimonisH11.pdf}{SimonisH11} (0.93)& \cellcolor{green!20}\href{../works/FontaineMH16.pdf}{FontaineMH16} (0.94)& \cellcolor{green!20}\href{../works/Simonis95a.pdf}{Simonis95a} (0.94)\\
Euclid& \cellcolor{red!20}\href{../works/BeniniBGM05a.pdf}{BeniniBGM05a} (0.25)& \cellcolor{red!20}\href{../works/WolfS05.pdf}{WolfS05} (0.26)& \cellcolor{red!20}\href{../works/BockmayrP06.pdf}{BockmayrP06} (0.26)& \cellcolor{yellow!20}\href{../works/PoderBS04.pdf}{PoderBS04} (0.27)& \cellcolor{yellow!20}\href{../works/LozanoCDS12.pdf}{LozanoCDS12} (0.27)\\
Dot& \cellcolor{red!40}\href{../works/Godet21a.pdf}{Godet21a} (88.00)& \cellcolor{red!40}\href{../works/Lombardi10.pdf}{Lombardi10} (86.00)& \cellcolor{red!40}\href{../works/Malapert11.pdf}{Malapert11} (86.00)& \cellcolor{red!40}\href{../works/Groleaz21.pdf}{Groleaz21} (83.00)& \cellcolor{red!40}\href{../works/Schutt11.pdf}{Schutt11} (82.00)\\
Cosine& \cellcolor{red!40}\href{../works/BeniniBGM05a.pdf}{BeniniBGM05a} (0.72)& \cellcolor{red!40}\href{../works/PoderBS04.pdf}{PoderBS04} (0.71)& \cellcolor{red!40}\href{../works/WolfS05.pdf}{WolfS05} (0.71)& \cellcolor{red!40}\href{../works/BockmayrP06.pdf}{BockmayrP06} (0.71)& \cellcolor{red!40}\href{../works/LozanoCDS12.pdf}{LozanoCDS12} (0.69)\\
\index{HillBCGN22}HillBCGN22 R\&C& \cellcolor{red!20}\href{../works/GuSS13.pdf}{GuSS13} (0.90)& \cellcolor{yellow!20}EdwardsBSE19 (0.90)& \cellcolor{yellow!20}GuSSWC14 (0.91)& \cellcolor{yellow!20}\href{../works/ThiruvadyWGS14.pdf}{ThiruvadyWGS14} (0.93)& \cellcolor{yellow!20}\href{../works/GuSW12.pdf}{GuSW12} (0.93)\\
Euclid\\
Dot\\
Cosine\\
\index{HillTV21}\href{../works/HillTV21.pdf}{HillTV21} R\&C& \cellcolor{red!20}EdwardsBSE19 (0.87)& \cellcolor{red!20}\href{../works/SchnellH15.pdf}{SchnellH15} (0.88)& \cellcolor{yellow!20}\href{../works/SzerediS16.pdf}{SzerediS16} (0.92)& \cellcolor{yellow!20}\href{../works/HauderBRPA20.pdf}{HauderBRPA20} (0.93)& \cellcolor{yellow!20}\href{../works/SchuttFS13.pdf}{SchuttFS13} (0.93)\\
Euclid& \cellcolor{yellow!20}\href{../works/HeipckeCCS00.pdf}{HeipckeCCS00} (0.28)& \cellcolor{green!20}\href{../works/DemasseyAM05.pdf}{DemasseyAM05} (0.29)& \cellcolor{green!20}\href{../works/KovacsV06.pdf}{KovacsV06} (0.29)& \cellcolor{green!20}\href{../works/LiessM08.pdf}{LiessM08} (0.29)& \cellcolor{green!20}\href{../works/BofillCSV17a.pdf}{BofillCSV17a} (0.29)\\
Dot& \cellcolor{red!40}\href{../works/ZarandiASC20.pdf}{ZarandiASC20} (151.00)& \cellcolor{red!40}\href{../works/Groleaz21.pdf}{Groleaz21} (145.00)& \cellcolor{red!40}\href{../works/Lombardi10.pdf}{Lombardi10} (141.00)& \cellcolor{red!40}\href{../works/Schutt11.pdf}{Schutt11} (137.00)& \cellcolor{red!40}\href{../works/Godet21a.pdf}{Godet21a} (137.00)\\
Cosine& \cellcolor{red!40}\href{../works/DemasseyAM05.pdf}{DemasseyAM05} (0.79)& \cellcolor{red!40}\href{../works/ArkhipovBL19.pdf}{ArkhipovBL19} (0.79)& \cellcolor{red!40}\href{../works/HeipckeCCS00.pdf}{HeipckeCCS00} (0.78)& \cellcolor{red!40}\href{../works/LiessM08.pdf}{LiessM08} (0.77)& \cellcolor{red!40}\href{../works/KovacsV06.pdf}{KovacsV06} (0.76)\\
\index{HladikCDJ08}\href{../works/HladikCDJ08.pdf}{HladikCDJ08} R\&C& \cellcolor{red!40}\href{../works/CambazardHDJT04.pdf}{CambazardHDJT04} (0.70)& \cellcolor{red!40}\href{../works/CambazardJ05.pdf}{CambazardJ05} (0.83)& \cellcolor{red!40}\href{../works/CireCH13.pdf}{CireCH13} (0.85)& \cellcolor{red!40}\href{../works/Hooker05.pdf}{Hooker05} (0.85)& \cellcolor{red!40}\href{../works/Hooker05a.pdf}{Hooker05a} (0.85)\\
Euclid& \cellcolor{red!40}\href{../works/CambazardHDJT04.pdf}{CambazardHDJT04} (0.20)& \cellcolor{green!20}\href{../works/BeniniLMR08.pdf}{BeniniLMR08} (0.31)& \cellcolor{blue!20}\href{../works/EmeretlisTAV17.pdf}{EmeretlisTAV17} (0.32)& \cellcolor{blue!20}\href{../works/BeniniBGM05.pdf}{BeniniBGM05} (0.32)& \cellcolor{blue!20}\href{../works/LozanoCDS12.pdf}{LozanoCDS12} (0.33)\\
Dot& \cellcolor{red!40}\href{../works/Lombardi10.pdf}{Lombardi10} (135.00)& \cellcolor{red!40}\href{../works/ZarandiASC20.pdf}{ZarandiASC20} (129.00)& \cellcolor{red!40}\href{../works/Froger16.pdf}{Froger16} (116.00)& \cellcolor{red!40}\href{../works/Godet21a.pdf}{Godet21a} (114.00)& \cellcolor{red!40}\href{../works/Beck99.pdf}{Beck99} (112.00)\\
Cosine& \cellcolor{red!40}\href{../works/CambazardHDJT04.pdf}{CambazardHDJT04} (0.89)& \cellcolor{red!40}\href{../works/EmeretlisTAV17.pdf}{EmeretlisTAV17} (0.74)& \cellcolor{red!40}\href{../works/BeniniLMR08.pdf}{BeniniLMR08} (0.74)& \cellcolor{red!40}\href{../works/BeniniBGM05.pdf}{BeniniBGM05} (0.73)& \cellcolor{red!40}\href{../works/BeniniLMR11.pdf}{BeniniLMR11} (0.69)\\
\index{HoYCLLCLC18}\href{../works/HoYCLLCLC18.pdf}{HoYCLLCLC18} R\&C& \cellcolor{red!20}\href{../works/HamPK21.pdf}{HamPK21} (0.88)& \cellcolor{red!20}\href{../works/FrohnerTR19.pdf}{FrohnerTR19} (0.90)& \cellcolor{green!20}\href{../works/WatsonB08.pdf}{WatsonB08} (0.94)& \cellcolor{green!20}\href{../works/MusliuSS18.pdf}{MusliuSS18} (0.95)& \cellcolor{blue!20}\href{../works/RendlPHPR12.pdf}{RendlPHPR12} (0.96)\\
Euclid& \cellcolor{red!40}\href{../works/LudwigKRBMS14.pdf}{LudwigKRBMS14} (0.22)& \cellcolor{red!40}\href{../works/FukunagaHFAMN02.pdf}{FukunagaHFAMN02} (0.22)& \cellcolor{red!40}\href{../works/AngelsmarkJ00.pdf}{AngelsmarkJ00} (0.22)& \cellcolor{red!40}\href{../works/BourdaisGP03.pdf}{BourdaisGP03} (0.23)& \cellcolor{red!40}\href{../works/CrawfordB94.pdf}{CrawfordB94} (0.23)\\
Dot& \cellcolor{red!40}\href{../works/ZarandiASC20.pdf}{ZarandiASC20} (69.00)& \cellcolor{red!40}\href{../works/Dejemeppe16.pdf}{Dejemeppe16} (60.00)& \cellcolor{red!40}\href{../works/Lombardi10.pdf}{Lombardi10} (59.00)& \cellcolor{red!40}\href{../works/Astrand21.pdf}{Astrand21} (58.00)& \cellcolor{red!40}\href{../works/RoshanaeiLAU17.pdf}{RoshanaeiLAU17} (57.00)\\
Cosine& \cellcolor{red!40}\href{../works/ShinBBHO18.pdf}{ShinBBHO18} (0.76)& \cellcolor{red!40}\href{../works/BourdaisGP03.pdf}{BourdaisGP03} (0.70)& \cellcolor{red!40}\href{../works/LudwigKRBMS14.pdf}{LudwigKRBMS14} (0.69)& \cellcolor{red!40}\href{../works/FukunagaHFAMN02.pdf}{FukunagaHFAMN02} (0.66)& \cellcolor{red!40}\href{../works/IfrimOS12.pdf}{IfrimOS12} (0.66)\\
\index{HoeveGSL07}\href{../works/HoeveGSL07.pdf}{HoeveGSL07} R\&C\\
Euclid& \cellcolor{red!40}\href{../works/GomesHS06.pdf}{GomesHS06} (0.24)& \cellcolor{red!20}\href{../works/NishikawaSTT18.pdf}{NishikawaSTT18} (0.24)& \cellcolor{red!20}\href{../works/NishikawaSTT18a.pdf}{NishikawaSTT18a} (0.24)& \cellcolor{red!20}\href{../works/QuSN06.pdf}{QuSN06} (0.24)& \cellcolor{red!20}\href{../works/BeniniBGM05a.pdf}{BeniniBGM05a} (0.25)\\
Dot& \cellcolor{red!40}\href{../works/Beck99.pdf}{Beck99} (93.00)& \cellcolor{red!40}\href{../works/ZarandiASC20.pdf}{ZarandiASC20} (89.00)& \cellcolor{red!40}\href{../works/BartakSR10.pdf}{BartakSR10} (88.00)& \cellcolor{red!40}\href{../works/GombolayWS18.pdf}{GombolayWS18} (86.00)& \cellcolor{red!40}\href{../works/LaborieRSV18.pdf}{LaborieRSV18} (85.00)\\
Cosine& \cellcolor{red!40}\href{../works/NishikawaSTT19.pdf}{NishikawaSTT19} (0.76)& \cellcolor{red!40}\href{../works/NishikawaSTT18.pdf}{NishikawaSTT18} (0.76)& \cellcolor{red!40}\href{../works/NishikawaSTT18a.pdf}{NishikawaSTT18a} (0.76)& \cellcolor{red!40}\href{../works/GomesHS06.pdf}{GomesHS06} (0.73)& \cellcolor{red!40}\href{../works/Pape94.pdf}{Pape94} (0.72)\\
\index{Hooker00}Hooker00 R\&C& \cellcolor{red!20}\href{../works/JainG01.pdf}{JainG01} (0.89)& \cellcolor{yellow!20}\href{../works/HookerO99.pdf}{HookerO99} (0.91)& \cellcolor{yellow!20}\href{../works/Hooker07.pdf}{Hooker07} (0.91)& \cellcolor{yellow!20}\href{../works/Thorsteinsson01.pdf}{Thorsteinsson01} (0.91)& \cellcolor{yellow!20}BockmayrK98 (0.91)\\
Euclid\\
Dot\\
Cosine\\
\index{Hooker02}Hooker02 R\&C& \cellcolor{red!40}\href{../works/HookerO99.pdf}{HookerO99} (0.72)& \cellcolor{red!40}\href{../works/JainG01.pdf}{JainG01} (0.80)& \cellcolor{red!40}Hooker06a (0.84)& \cellcolor{red!20}\href{../works/Thorsteinsson01.pdf}{Thorsteinsson01} (0.86)& \cellcolor{red!20}\href{../works/Hooker05b.pdf}{Hooker05b} (0.86)\\
Euclid\\
Dot\\
Cosine\\
\index{Hooker04}\href{../works/Hooker04.pdf}{Hooker04} R\&C& \cellcolor{red!40}\href{../works/Hooker05a.pdf}{Hooker05a} (0.44)& \cellcolor{red!40}\href{../works/Hooker05.pdf}{Hooker05} (0.55)& \cellcolor{red!40}\href{../works/Hooker07.pdf}{Hooker07} (0.67)& \cellcolor{red!40}\href{../works/Hooker06.pdf}{Hooker06} (0.68)& \cellcolor{red!40}\href{../works/CambazardHDJT04.pdf}{CambazardHDJT04} (0.68)\\
Euclid& \cellcolor{red!40}\href{../works/Hooker07.pdf}{Hooker07} (0.16)& \cellcolor{red!40}\href{../works/Hooker05.pdf}{Hooker05} (0.20)& \cellcolor{red!40}\href{../works/Hooker05a.pdf}{Hooker05a} (0.20)& \cellcolor{red!40}\href{../works/Hooker06.pdf}{Hooker06} (0.20)& \cellcolor{red!40}\href{../works/CireCH16.pdf}{CireCH16} (0.22)\\
Dot& \cellcolor{red!40}\href{../works/Hooker07.pdf}{Hooker07} (117.00)& \cellcolor{red!40}\href{../works/Lombardi10.pdf}{Lombardi10} (117.00)& \cellcolor{red!40}\href{../works/Hooker05.pdf}{Hooker05} (113.00)& \cellcolor{red!40}\href{../works/LaborieRSV18.pdf}{LaborieRSV18} (102.00)& \cellcolor{red!40}\href{../works/Groleaz21.pdf}{Groleaz21} (101.00)\\
Cosine& \cellcolor{red!40}\href{../works/Hooker07.pdf}{Hooker07} (0.94)& \cellcolor{red!40}\href{../works/Hooker05.pdf}{Hooker05} (0.90)& \cellcolor{red!40}\href{../works/Hooker06.pdf}{Hooker06} (0.88)& \cellcolor{red!40}\href{../works/Hooker05a.pdf}{Hooker05a} (0.87)& \cellcolor{red!40}\href{../works/CireCH16.pdf}{CireCH16} (0.83)\\
\index{Hooker05}\href{../works/Hooker05.pdf}{Hooker05} R\&C& \cellcolor{red!40}\href{../works/Hooker05a.pdf}{Hooker05a} (0.49)& \cellcolor{red!40}\href{../works/Hooker04.pdf}{Hooker04} (0.55)& \cellcolor{red!40}\href{../works/Hooker07.pdf}{Hooker07} (0.57)& \cellcolor{red!40}\href{../works/ChuX05.pdf}{ChuX05} (0.58)& \cellcolor{red!40}\href{../works/Hooker06.pdf}{Hooker06} (0.61)\\
Euclid& \cellcolor{red!40}\href{../works/Hooker07.pdf}{Hooker07} (0.19)& \cellcolor{red!40}\href{../works/Hooker04.pdf}{Hooker04} (0.20)& \cellcolor{red!40}\href{../works/Hooker06.pdf}{Hooker06} (0.23)& \cellcolor{red!20}\href{../works/Hooker05a.pdf}{Hooker05a} (0.25)& \cellcolor{yellow!20}\href{../works/BeniniLMMR08.pdf}{BeniniLMMR08} (0.28)\\
Dot& \cellcolor{red!40}\href{../works/Lombardi10.pdf}{Lombardi10} (149.00)& \cellcolor{red!40}\href{../works/Hooker07.pdf}{Hooker07} (132.00)& \cellcolor{red!40}\href{../works/Beck99.pdf}{Beck99} (129.00)& \cellcolor{red!40}\href{../works/Baptiste02.pdf}{Baptiste02} (128.00)& \cellcolor{red!40}\href{../works/LaborieRSV18.pdf}{LaborieRSV18} (127.00)\\
Cosine& \cellcolor{red!40}\href{../works/Hooker07.pdf}{Hooker07} (0.92)& \cellcolor{red!40}\href{../works/Hooker04.pdf}{Hooker04} (0.90)& \cellcolor{red!40}\href{../works/Hooker06.pdf}{Hooker06} (0.87)& \cellcolor{red!40}\href{../works/Hooker05a.pdf}{Hooker05a} (0.84)& \cellcolor{red!40}\href{../works/BeniniLMMR08.pdf}{BeniniLMMR08} (0.79)\\
\index{Hooker05a}\href{../works/Hooker05a.pdf}{Hooker05a} R\&C& \cellcolor{red!40}\href{../works/Hooker04.pdf}{Hooker04} (0.44)& \cellcolor{red!40}\href{../works/Hooker05.pdf}{Hooker05} (0.49)& \cellcolor{red!40}\href{../works/CambazardJ05.pdf}{CambazardJ05} (0.58)& \cellcolor{red!40}\href{../works/BeniniBGM05.pdf}{BeniniBGM05} (0.62)& \cellcolor{red!40}\href{../works/Hooker06.pdf}{Hooker06} (0.64)\\
Euclid& \cellcolor{red!40}\href{../works/Hooker06.pdf}{Hooker06} (0.11)& \cellcolor{red!40}\href{../works/Hooker07.pdf}{Hooker07} (0.19)& \cellcolor{red!40}\href{../works/Hooker04.pdf}{Hooker04} (0.20)& \cellcolor{red!40}\href{../works/BockmayrP06.pdf}{BockmayrP06} (0.24)& \cellcolor{red!20}\href{../works/CireCH16.pdf}{CireCH16} (0.24)\\
Dot& \cellcolor{red!40}\href{../works/Lombardi10.pdf}{Lombardi10} (106.00)& \cellcolor{red!40}\href{../works/Hooker07.pdf}{Hooker07} (104.00)& \cellcolor{red!40}\href{../works/Hooker06.pdf}{Hooker06} (100.00)& \cellcolor{red!40}\href{../works/Groleaz21.pdf}{Groleaz21} (100.00)& \cellcolor{red!40}\href{../works/Baptiste02.pdf}{Baptiste02} (100.00)\\
Cosine& \cellcolor{red!40}\href{../works/Hooker06.pdf}{Hooker06} (0.97)& \cellcolor{red!40}\href{../works/Hooker07.pdf}{Hooker07} (0.92)& \cellcolor{red!40}\href{../works/Hooker04.pdf}{Hooker04} (0.87)& \cellcolor{red!40}\href{../works/Hooker05.pdf}{Hooker05} (0.84)& \cellcolor{red!40}\href{../works/CobanH11.pdf}{CobanH11} (0.77)\\
\index{Hooker05b}\href{../works/Hooker05b.pdf}{Hooker05b} R\&C& \cellcolor{red!40}\href{../works/Hooker05.pdf}{Hooker05} (0.67)& \cellcolor{red!40}\href{../works/Hooker05a.pdf}{Hooker05a} (0.70)& \cellcolor{red!40}\href{../works/Hooker04.pdf}{Hooker04} (0.73)& \cellcolor{red!40}\href{../works/CambazardJ05.pdf}{CambazardJ05} (0.74)& \cellcolor{red!40}\href{../works/Hooker07.pdf}{Hooker07} (0.76)\\
Euclid& \cellcolor{red!40}\href{../works/ZibranR11.pdf}{ZibranR11} (0.23)& \cellcolor{red!40}\href{../works/CambazardJ05.pdf}{CambazardJ05} (0.23)& \cellcolor{red!40}\href{../works/ZhangLS12.pdf}{ZhangLS12} (0.24)& \cellcolor{red!20}\href{../works/Davis87.pdf}{Davis87} (0.24)& \cellcolor{red!20}\href{../works/BenoistGR02.pdf}{BenoistGR02} (0.25)\\
Dot& \cellcolor{red!40}\href{../works/Lombardi10.pdf}{Lombardi10} (56.00)& \cellcolor{red!40}\href{../works/Froger16.pdf}{Froger16} (55.00)& \cellcolor{red!40}\href{../works/Lemos21.pdf}{Lemos21} (53.00)& \cellcolor{red!40}\href{../works/KendallKRU10.pdf}{KendallKRU10} (53.00)& \cellcolor{red!40}\href{../works/HookerH17.pdf}{HookerH17} (52.00)\\
Cosine& \cellcolor{red!40}\href{../works/ZhangLS12.pdf}{ZhangLS12} (0.65)& \cellcolor{red!40}\href{../works/HookerOTK00.pdf}{HookerOTK00} (0.63)& \cellcolor{red!40}\href{../works/Thorsteinsson01.pdf}{Thorsteinsson01} (0.63)& \cellcolor{red!40}\href{../works/CireCH13.pdf}{CireCH13} (0.62)& \cellcolor{red!40}\href{../works/ZibranR11.pdf}{ZibranR11} (0.62)\\
\index{Hooker06}\href{../works/Hooker06.pdf}{Hooker06} R\&C& \cellcolor{red!40}\href{../works/Hooker07.pdf}{Hooker07} (0.57)& \cellcolor{red!40}\href{../works/Hooker05.pdf}{Hooker05} (0.61)& \cellcolor{red!40}\href{../works/CobanH11.pdf}{CobanH11} (0.63)& \cellcolor{red!40}\href{../works/CireCH13.pdf}{CireCH13} (0.63)& \cellcolor{red!40}\href{../works/Hooker05a.pdf}{Hooker05a} (0.64)\\
Euclid& \cellcolor{red!40}\href{../works/Hooker05a.pdf}{Hooker05a} (0.11)& \cellcolor{red!40}\href{../works/Hooker07.pdf}{Hooker07} (0.16)& \cellcolor{red!40}\href{../works/Hooker04.pdf}{Hooker04} (0.20)& \cellcolor{red!40}\href{../works/Hooker05.pdf}{Hooker05} (0.23)& \cellcolor{red!20}\href{../works/CireCH16.pdf}{CireCH16} (0.26)\\
Dot& \cellcolor{red!40}\href{../works/Lombardi10.pdf}{Lombardi10} (128.00)& \cellcolor{red!40}\href{../works/Hooker07.pdf}{Hooker07} (123.00)& \cellcolor{red!40}\href{../works/Baptiste02.pdf}{Baptiste02} (118.00)& \cellcolor{red!40}\href{../works/Dejemeppe16.pdf}{Dejemeppe16} (117.00)& \cellcolor{red!40}\href{../works/Groleaz21.pdf}{Groleaz21} (116.00)\\
Cosine& \cellcolor{red!40}\href{../works/Hooker05a.pdf}{Hooker05a} (0.97)& \cellcolor{red!40}\href{../works/Hooker07.pdf}{Hooker07} (0.94)& \cellcolor{red!40}\href{../works/Hooker04.pdf}{Hooker04} (0.88)& \cellcolor{red!40}\href{../works/Hooker05.pdf}{Hooker05} (0.87)& \cellcolor{red!40}\href{../works/CobanH11.pdf}{CobanH11} (0.79)\\
\index{Hooker06a}Hooker06a R\&C& \cellcolor{red!40}\href{../works/Hooker05b.pdf}{Hooker05b} (0.83)& \cellcolor{red!40}Hooker02 (0.84)& \cellcolor{red!20}\href{../works/Hooker07.pdf}{Hooker07} (0.88)& \cellcolor{red!20}CastroGR10 (0.89)& \cellcolor{red!20}\href{../works/CireCH16.pdf}{CireCH16} (0.89)\\
Euclid\\
Dot\\
Cosine\\
\index{Hooker07}\href{../works/Hooker07.pdf}{Hooker07} R\&C& \cellcolor{red!40}\href{../works/Hooker06.pdf}{Hooker06} (0.57)& \cellcolor{red!40}\href{../works/Hooker05.pdf}{Hooker05} (0.57)& \cellcolor{red!40}\href{../works/Hooker05a.pdf}{Hooker05a} (0.64)& \cellcolor{red!40}\href{../works/CireCH13.pdf}{CireCH13} (0.66)& \cellcolor{red!40}\href{../works/Hooker04.pdf}{Hooker04} (0.67)\\
Euclid& \cellcolor{red!40}\href{../works/Hooker06.pdf}{Hooker06} (0.16)& \cellcolor{red!40}\href{../works/Hooker04.pdf}{Hooker04} (0.16)& \cellcolor{red!40}\href{../works/Hooker05.pdf}{Hooker05} (0.19)& \cellcolor{red!40}\href{../works/Hooker05a.pdf}{Hooker05a} (0.19)& \cellcolor{red!20}\href{../works/CireCH16.pdf}{CireCH16} (0.26)\\
Dot& \cellcolor{red!40}\href{../works/Lombardi10.pdf}{Lombardi10} (145.00)& \cellcolor{red!40}\href{../works/Hooker05.pdf}{Hooker05} (132.00)& \cellcolor{red!40}\href{../works/LaborieRSV18.pdf}{LaborieRSV18} (131.00)& \cellcolor{red!40}\href{../works/Baptiste02.pdf}{Baptiste02} (130.00)& \cellcolor{red!40}\href{../works/Groleaz21.pdf}{Groleaz21} (129.00)\\
Cosine& \cellcolor{red!40}\href{../works/Hooker04.pdf}{Hooker04} (0.94)& \cellcolor{red!40}\href{../works/Hooker06.pdf}{Hooker06} (0.94)& \cellcolor{red!40}\href{../works/Hooker05a.pdf}{Hooker05a} (0.92)& \cellcolor{red!40}\href{../works/Hooker05.pdf}{Hooker05} (0.92)& \cellcolor{red!40}\href{../works/CireCH16.pdf}{CireCH16} (0.82)\\
\index{Hooker10}Hooker10 R\&C& \cellcolor{red!20}\href{../works/CireCH16.pdf}{CireCH16} (0.90)& \cellcolor{yellow!20}\href{../works/CireCH13.pdf}{CireCH13} (0.90)& \cellcolor{yellow!20}\href{../works/Simonis07.pdf}{Simonis07} (0.91)& \cellcolor{yellow!20}\href{../works/Hooker05a.pdf}{Hooker05a} (0.91)& \cellcolor{yellow!20}\href{../works/Hooker05.pdf}{Hooker05} (0.92)\\
Euclid\\
Dot\\
Cosine\\
\index{Hooker17}\href{../works/Hooker17.pdf}{Hooker17} R\&C& \cellcolor{red!40}\href{../works/BogaerdtW19.pdf}{BogaerdtW19} (0.78)& \cellcolor{green!20}\href{../works/HechingH16.pdf}{HechingH16} (0.94)& \cellcolor{green!20}\href{../works/HookerH17.pdf}{HookerH17} (0.94)& \cellcolor{green!20}\href{../works/TranVNB17.pdf}{TranVNB17} (0.94)& \cellcolor{green!20}\href{../works/CireCH16.pdf}{CireCH16} (0.95)\\
Euclid& \cellcolor{red!40}\href{../works/HebrardTW05.pdf}{HebrardTW05} (0.17)& \cellcolor{red!40}\href{../works/Baptiste09.pdf}{Baptiste09} (0.18)& \cellcolor{red!40}\href{../works/CestaOS98.pdf}{CestaOS98} (0.19)& \cellcolor{red!40}\href{../works/WuBB05.pdf}{WuBB05} (0.19)& \cellcolor{red!40}\href{../works/KovacsEKV05.pdf}{KovacsEKV05} (0.20)\\
Dot& \cellcolor{red!40}\href{../works/Dejemeppe16.pdf}{Dejemeppe16} (46.00)& \cellcolor{red!40}\href{../works/Groleaz21.pdf}{Groleaz21} (44.00)& \cellcolor{red!40}\href{../works/KelbelH11.pdf}{KelbelH11} (43.00)& \cellcolor{red!40}\href{../works/Lombardi10.pdf}{Lombardi10} (43.00)& \cellcolor{red!40}\href{../works/abs-1902-09244.pdf}{abs-1902-09244} (42.00)\\
Cosine& \cellcolor{red!40}\href{../works/Balduccini11.pdf}{Balduccini11} (0.69)& \cellcolor{red!40}\href{../works/CobanH10.pdf}{CobanH10} (0.66)& \cellcolor{red!40}\href{../works/SmithC93.pdf}{SmithC93} (0.65)& \cellcolor{red!40}\href{../works/DannaP03.pdf}{DannaP03} (0.65)& \cellcolor{red!40}\href{../works/PerronSF04.pdf}{PerronSF04} (0.64)\\
\index{Hooker19}\href{../works/Hooker19.pdf}{Hooker19} R\&C\\
Euclid& \cellcolor{blue!20}\href{../works/ElciOH22.pdf}{ElciOH22} (0.32)& \cellcolor{blue!20}\href{../works/CobanH11.pdf}{CobanH11} (0.33)& \cellcolor{black!20}\href{../works/Hooker06.pdf}{Hooker06} (0.37)& \cellcolor{black!20}\href{../works/Hooker07.pdf}{Hooker07} (0.37)& \cellcolor{black!20}\href{../works/Hooker05a.pdf}{Hooker05a} (0.37)\\
Dot& \cellcolor{red!40}\href{../works/ZarandiASC20.pdf}{ZarandiASC20} (175.00)& \cellcolor{red!40}\href{../works/Lombardi10.pdf}{Lombardi10} (162.00)& \cellcolor{red!40}\href{../works/Groleaz21.pdf}{Groleaz21} (155.00)& \cellcolor{red!40}\href{../works/LaborieRSV18.pdf}{LaborieRSV18} (150.00)& \cellcolor{red!40}\href{../works/HookerH17.pdf}{HookerH17} (149.00)\\
Cosine& \cellcolor{red!40}\href{../works/ElciOH22.pdf}{ElciOH22} (0.79)& \cellcolor{red!40}\href{../works/CobanH11.pdf}{CobanH11} (0.77)& \cellcolor{red!40}\href{../works/Hooker07.pdf}{Hooker07} (0.72)& \cellcolor{red!40}\href{../works/Hooker06.pdf}{Hooker06} (0.72)& \cellcolor{red!40}\href{../works/MilanoW06.pdf}{MilanoW06} (0.71)\\
\index{HookerH17}\href{../works/HookerH17.pdf}{HookerH17} R\&C& \cellcolor{red!20}\href{../works/CireCH16.pdf}{CireCH16} (0.88)& \cellcolor{red!20}CastroGR10 (0.88)& \cellcolor{yellow!20}\href{../works/CobanH11.pdf}{CobanH11} (0.91)& \cellcolor{yellow!20}\href{../works/CireCH13.pdf}{CireCH13} (0.92)& \cellcolor{yellow!20}Hooker06a (0.92)\\
Euclid& \href{../works/HookerOTK00.pdf}{HookerOTK00} (0.49)& \href{../works/Hooker19.pdf}{Hooker19} (0.49)& \href{../works/MilanoW06.pdf}{MilanoW06} (0.50)& \href{../works/MilanoW09.pdf}{MilanoW09} (0.50)& \href{../works/FontaineMH16.pdf}{FontaineMH16} (0.50)\\
Dot& \cellcolor{red!40}\href{../works/Dejemeppe16.pdf}{Dejemeppe16} (204.00)& \cellcolor{red!40}\href{../works/Lombardi10.pdf}{Lombardi10} (203.00)& \cellcolor{red!40}\href{../works/ZarandiASC20.pdf}{ZarandiASC20} (202.00)& \cellcolor{red!40}\href{../works/Baptiste02.pdf}{Baptiste02} (196.00)& \cellcolor{red!40}\href{../works/Fahimi16.pdf}{Fahimi16} (183.00)\\
Cosine& \cellcolor{red!40}\href{../works/MilanoW09.pdf}{MilanoW09} (0.67)& \cellcolor{red!40}\href{../works/MilanoW06.pdf}{MilanoW06} (0.66)& \cellcolor{red!40}\href{../works/Hooker19.pdf}{Hooker19} (0.65)& \cellcolor{red!40}\href{../works/HookerOTK00.pdf}{HookerOTK00} (0.65)& \cellcolor{red!40}\href{../works/Demassey03.pdf}{Demassey03} (0.64)\\
\index{HookerO03}\href{../works/HookerO03.pdf}{HookerO03} R\&C& \cellcolor{red!40}\href{../works/Hooker07.pdf}{Hooker07} (0.80)& \cellcolor{red!20}\href{../works/JainG01.pdf}{JainG01} (0.88)& \cellcolor{red!20}\href{../works/Thorsteinsson01.pdf}{Thorsteinsson01} (0.88)& \cellcolor{yellow!20}ZarandiB12 (0.91)& \cellcolor{yellow!20}\href{../works/RoshanaeiLAU17.pdf}{RoshanaeiLAU17} (0.92)\\
Euclid& \cellcolor{red!40}\href{../works/Thorsteinsson01.pdf}{Thorsteinsson01} (0.21)& \cellcolor{red!40}\href{../works/HookerY02.pdf}{HookerY02} (0.21)& \cellcolor{red!40}\href{../works/CireCH13.pdf}{CireCH13} (0.21)& \cellcolor{red!20}\href{../works/Beck10.pdf}{Beck10} (0.25)& \cellcolor{red!20}\href{../works/ChuX05.pdf}{ChuX05} (0.25)\\
Dot& \cellcolor{red!40}\href{../works/Lombardi10.pdf}{Lombardi10} (90.00)& \cellcolor{red!40}\href{../works/HookerH17.pdf}{HookerH17} (90.00)& \cellcolor{red!40}\href{../works/MilanoW09.pdf}{MilanoW09} (83.00)& \cellcolor{red!40}\href{../works/MilanoW06.pdf}{MilanoW06} (80.00)& \cellcolor{red!40}\href{../works/Hooker19.pdf}{Hooker19} (80.00)\\
Cosine& \cellcolor{red!40}\href{../works/Thorsteinsson01.pdf}{Thorsteinsson01} (0.83)& \cellcolor{red!40}\href{../works/CireCH13.pdf}{CireCH13} (0.81)& \cellcolor{red!40}\href{../works/HookerY02.pdf}{HookerY02} (0.79)& \cellcolor{red!40}\href{../works/HamdiL13.pdf}{HamdiL13} (0.78)& \cellcolor{red!40}\href{../works/Hooker06.pdf}{Hooker06} (0.77)\\
\index{HookerO99}\href{../works/HookerO99.pdf}{HookerO99} R\&C& \cellcolor{red!40}Hooker02 (0.72)& \cellcolor{red!40}BockmayrK98 (0.82)& \cellcolor{red!20}\href{../works/RodosekWH99.pdf}{RodosekWH99} (0.87)& \cellcolor{red!20}\href{../works/HookerOTK00.pdf}{HookerOTK00} (0.89)& \cellcolor{red!20}\href{../works/JainG01.pdf}{JainG01} (0.90)\\
Euclid& \cellcolor{green!20}\href{../works/RodosekWH99.pdf}{RodosekWH99} (0.30)& \cellcolor{green!20}\href{../works/HookerOTK00.pdf}{HookerOTK00} (0.30)& \cellcolor{green!20}\href{../works/HarjunkoskiJG00.pdf}{HarjunkoskiJG00} (0.30)& \cellcolor{green!20}\href{../works/FontaineMH16.pdf}{FontaineMH16} (0.31)& \cellcolor{blue!20}\href{../works/CireCH13.pdf}{CireCH13} (0.32)\\
Dot& \cellcolor{red!40}\href{../works/Malapert11.pdf}{Malapert11} (114.00)& \cellcolor{red!40}\href{../works/Lombardi10.pdf}{Lombardi10} (109.00)& \cellcolor{red!40}\href{../works/Baptiste02.pdf}{Baptiste02} (109.00)& \cellcolor{red!40}\href{../works/HookerH17.pdf}{HookerH17} (106.00)& \cellcolor{red!40}\href{../works/Siala15a.pdf}{Siala15a} (103.00)\\
Cosine& \cellcolor{red!40}\href{../works/Zhou97.pdf}{Zhou97} (0.74)& \cellcolor{red!40}\href{../works/RodosekWH99.pdf}{RodosekWH99} (0.74)& \cellcolor{red!40}\href{../works/HookerOTK00.pdf}{HookerOTK00} (0.73)& \cellcolor{red!40}\href{../works/HarjunkoskiJG00.pdf}{HarjunkoskiJG00} (0.72)& \cellcolor{red!40}\href{../works/FontaineMH16.pdf}{FontaineMH16} (0.72)\\
\index{HookerOTK00}\href{../works/HookerOTK00.pdf}{HookerOTK00} R\&C& \cellcolor{red!20}BockmayrK98 (0.89)& \cellcolor{red!20}\href{../works/HookerO99.pdf}{HookerO99} (0.89)& \cellcolor{yellow!20}\href{../works/BenoistGR02.pdf}{BenoistGR02} (0.91)& \cellcolor{yellow!20}\href{../works/Thorsteinsson01.pdf}{Thorsteinsson01} (0.93)& \cellcolor{green!20}Hooker02 (0.94)\\
Euclid& \cellcolor{yellow!20}\href{../works/HookerY02.pdf}{HookerY02} (0.27)& \cellcolor{yellow!20}\href{../works/HookerO03.pdf}{HookerO03} (0.27)& \cellcolor{yellow!20}\href{../works/Thorsteinsson01.pdf}{Thorsteinsson01} (0.28)& \cellcolor{yellow!20}\href{../works/Puget95.pdf}{Puget95} (0.28)& \cellcolor{yellow!20}\href{../works/CireCH13.pdf}{CireCH13} (0.28)\\
Dot& \cellcolor{red!40}\href{../works/HookerH17.pdf}{HookerH17} (103.00)& \cellcolor{red!40}\href{../works/Lombardi10.pdf}{Lombardi10} (93.00)& \cellcolor{red!40}\href{../works/Dejemeppe16.pdf}{Dejemeppe16} (89.00)& \cellcolor{red!40}\href{../works/Godet21a.pdf}{Godet21a} (86.00)& \cellcolor{red!40}\href{../works/Baptiste02.pdf}{Baptiste02} (85.00)\\
Cosine& \cellcolor{red!40}\href{../works/HookerO99.pdf}{HookerO99} (0.73)& \cellcolor{red!40}\href{../works/Thorsteinsson01.pdf}{Thorsteinsson01} (0.71)& \cellcolor{red!40}\href{../works/HookerO03.pdf}{HookerO03} (0.70)& \cellcolor{red!40}\href{../works/JainG01.pdf}{JainG01} (0.69)& \cellcolor{red!40}\href{../works/HookerY02.pdf}{HookerY02} (0.69)\\
\index{HookerY02}\href{../works/HookerY02.pdf}{HookerY02} R\&C& \cellcolor{red!40}\href{../works/Hooker05b.pdf}{Hooker05b} (0.78)& \cellcolor{red!40}\href{../works/BockmayrP06.pdf}{BockmayrP06} (0.79)& \cellcolor{red!40}\href{../works/Thorsteinsson01.pdf}{Thorsteinsson01} (0.82)& \cellcolor{red!40}\href{../works/Hooker04.pdf}{Hooker04} (0.85)& \cellcolor{red!40}MilanoORT02 (0.86)\\
Euclid& \cellcolor{red!40}\href{../works/HeinzKB13.pdf}{HeinzKB13} (0.21)& \cellcolor{red!40}\href{../works/ChuX05.pdf}{ChuX05} (0.21)& \cellcolor{red!40}\href{../works/HookerO03.pdf}{HookerO03} (0.21)& \cellcolor{red!40}\href{../works/Limtanyakul07.pdf}{Limtanyakul07} (0.22)& \cellcolor{red!40}\href{../works/BertholdHLMS10.pdf}{BertholdHLMS10} (0.23)\\
Dot& \cellcolor{red!40}\href{../works/HookerH17.pdf}{HookerH17} (83.00)& \cellcolor{red!40}\href{../works/Lombardi10.pdf}{Lombardi10} (80.00)& \cellcolor{red!40}\href{../works/Hooker19.pdf}{Hooker19} (75.00)& \cellcolor{red!40}\href{../works/Demassey03.pdf}{Demassey03} (74.00)& \cellcolor{red!40}\href{../works/Froger16.pdf}{Froger16} (74.00)\\
Cosine& \cellcolor{red!40}\href{../works/HeinzKB13.pdf}{HeinzKB13} (0.82)& \cellcolor{red!40}\href{../works/ChuX05.pdf}{ChuX05} (0.81)& \cellcolor{red!40}\href{../works/HookerO03.pdf}{HookerO03} (0.79)& \cellcolor{red!40}\href{../works/Beck10.pdf}{Beck10} (0.78)& \cellcolor{red!40}\href{../works/BertholdHLMS10.pdf}{BertholdHLMS10} (0.74)\\
\index{HoundjiSW19}\href{../works/HoundjiSW19.pdf}{HoundjiSW19} R\&C& \cellcolor{red!40}\href{../works/HoundjiSWD14.pdf}{HoundjiSWD14} (0.79)& \cellcolor{red!20}\href{../works/CauwelaertDMS16.pdf}{CauwelaertDMS16} (0.90)& \cellcolor{yellow!20}\href{../works/PesantRR15.pdf}{PesantRR15} (0.92)& \cellcolor{green!20}\href{../works/CauwelaertLS18.pdf}{CauwelaertLS18} (0.95)& \cellcolor{green!20}\href{../works/CauwelaertDS20.pdf}{CauwelaertDS20} (0.95)\\
Euclid& \cellcolor{red!40}\href{../works/HoundjiSWD14.pdf}{HoundjiSWD14} (0.20)& \cellcolor{red!20}\href{../works/BockmayrP06.pdf}{BockmayrP06} (0.26)& \cellcolor{green!20}\href{../works/Limtanyakul07.pdf}{Limtanyakul07} (0.30)& \cellcolor{green!20}\href{../works/KrogtLPHJ07.pdf}{KrogtLPHJ07} (0.31)& \cellcolor{green!20}\href{../works/BofillGSV15.pdf}{BofillGSV15} (0.31)\\
Dot& \cellcolor{red!40}\href{../works/Dejemeppe16.pdf}{Dejemeppe16} (98.00)& \cellcolor{red!40}\href{../works/MilanoW09.pdf}{MilanoW09} (91.00)& \cellcolor{red!40}\href{../works/MilanoW06.pdf}{MilanoW06} (91.00)& \cellcolor{red!40}\href{../works/Simonis07.pdf}{Simonis07} (84.00)& \cellcolor{red!40}\href{../works/Malapert11.pdf}{Malapert11} (82.00)\\
Cosine& \cellcolor{red!40}\href{../works/HoundjiSWD14.pdf}{HoundjiSWD14} (0.84)& \cellcolor{red!40}\href{../works/BockmayrP06.pdf}{BockmayrP06} (0.72)& \cellcolor{red!40}\href{../works/MilanoW06.pdf}{MilanoW06} (0.65)& \cellcolor{red!40}\href{../works/MilanoW09.pdf}{MilanoW09} (0.63)& \cellcolor{red!40}\href{../works/CatusseCBL16.pdf}{CatusseCBL16} (0.61)\\
\index{HoundjiSWD14}\href{../works/HoundjiSWD14.pdf}{HoundjiSWD14} R\&C& \cellcolor{red!40}\href{../works/SimonisH11.pdf}{SimonisH11} (0.75)& \cellcolor{red!40}\href{../works/HoundjiSW19.pdf}{HoundjiSW19} (0.79)& \cellcolor{red!40}\href{../works/CauwelaertLS15.pdf}{CauwelaertLS15} (0.82)& \cellcolor{red!40}\href{../works/GaySS14.pdf}{GaySS14} (0.83)& \cellcolor{red!40}\href{../works/GayHLS15.pdf}{GayHLS15} (0.84)\\
Euclid& \cellcolor{red!40}\href{../works/HoundjiSW19.pdf}{HoundjiSW19} (0.20)& \cellcolor{red!40}\href{../works/FoxAS82.pdf}{FoxAS82} (0.24)& \cellcolor{red!20}\href{../works/Davis87.pdf}{Davis87} (0.25)& \cellcolor{yellow!20}\href{../works/KrogtLPHJ07.pdf}{KrogtLPHJ07} (0.26)& \cellcolor{yellow!20}\href{../works/BofillGSV15.pdf}{BofillGSV15} (0.26)\\
Dot& \cellcolor{red!40}\href{../works/Dejemeppe16.pdf}{Dejemeppe16} (72.00)& \cellcolor{red!40}\href{../works/HoundjiSW19.pdf}{HoundjiSW19} (61.00)& \cellcolor{red!40}\href{../works/MilanoW09.pdf}{MilanoW09} (58.00)& \cellcolor{red!40}\href{../works/MilanoW06.pdf}{MilanoW06} (58.00)& \cellcolor{red!40}\href{../works/BajestaniB13.pdf}{BajestaniB13} (58.00)\\
Cosine& \cellcolor{red!40}\href{../works/HoundjiSW19.pdf}{HoundjiSW19} (0.84)& \cellcolor{red!40}\href{../works/FoxAS82.pdf}{FoxAS82} (0.64)& \cellcolor{red!40}\href{../works/KrogtLPHJ07.pdf}{KrogtLPHJ07} (0.63)& \cellcolor{red!40}\href{../works/BockmayrP06.pdf}{BockmayrP06} (0.63)& \cellcolor{red!40}\href{../works/GoldwaserS17.pdf}{GoldwaserS17} (0.61)\\
\index{HubnerGSV21}\href{../works/HubnerGSV21.pdf}{HubnerGSV21} R\&C& \cellcolor{green!20}\href{../works/LimtanyakulS12.pdf}{LimtanyakulS12} (0.94)& \cellcolor{green!20}\href{../works/SchnellH17.pdf}{SchnellH17} (0.95)& \cellcolor{green!20}\href{../works/HauderBRPA20.pdf}{HauderBRPA20} (0.95)& \cellcolor{green!20}\href{../works/KreterSSZ18.pdf}{KreterSSZ18} (0.96)& \cellcolor{green!20}\href{../works/LombardiM09.pdf}{LombardiM09} (0.96)\\
Euclid& \cellcolor{blue!20}\href{../works/CampeauG22.pdf}{CampeauG22} (0.32)& \cellcolor{blue!20}\href{../works/BofillCSV17a.pdf}{BofillCSV17a} (0.33)& \cellcolor{black!20}\href{../works/LiessM08.pdf}{LiessM08} (0.34)& \cellcolor{black!20}\href{../works/PinarbasiAY19.pdf}{PinarbasiAY19} (0.35)& \cellcolor{black!20}\href{../works/HeipckeCCS00.pdf}{HeipckeCCS00} (0.36)\\
Dot& \cellcolor{red!40}\href{../works/ZarandiASC20.pdf}{ZarandiASC20} (167.00)& \cellcolor{red!40}\href{../works/Groleaz21.pdf}{Groleaz21} (167.00)& \cellcolor{red!40}\href{../works/Lombardi10.pdf}{Lombardi10} (151.00)& \cellcolor{red!40}\href{../works/Astrand21.pdf}{Astrand21} (151.00)& \cellcolor{red!40}\href{../works/Dejemeppe16.pdf}{Dejemeppe16} (147.00)\\
Cosine& \cellcolor{red!40}\href{../works/SubulanC22.pdf}{SubulanC22} (0.77)& \cellcolor{red!40}\href{../works/CampeauG22.pdf}{CampeauG22} (0.76)& \cellcolor{red!40}\href{../works/BofillCSV17a.pdf}{BofillCSV17a} (0.74)& \cellcolor{red!40}\href{../works/LiessM08.pdf}{LiessM08} (0.72)& \cellcolor{red!40}\href{../works/abs-1902-09244.pdf}{abs-1902-09244} (0.72)\\
\index{Hunsberger08}\href{../works/Hunsberger08.pdf}{Hunsberger08} R\&C\\
Euclid& \cellcolor{red!40}\href{../works/CarchraeBF05.pdf}{CarchraeBF05} (0.15)& \cellcolor{red!40}\href{../works/AngelsmarkJ00.pdf}{AngelsmarkJ00} (0.15)& \cellcolor{red!40}\href{../works/Baptiste09.pdf}{Baptiste09} (0.16)& \cellcolor{red!40}\href{../works/LiuJ06.pdf}{LiuJ06} (0.16)& \cellcolor{red!40}\href{../works/SunLYL10.pdf}{SunLYL10} (0.17)\\
Dot& \cellcolor{red!40}\href{../works/ZarandiASC20.pdf}{ZarandiASC20} (39.00)& \cellcolor{red!40}\href{../works/Lemos21.pdf}{Lemos21} (36.00)& \cellcolor{red!40}\href{../works/SultanikMR07.pdf}{SultanikMR07} (36.00)& \cellcolor{red!40}\href{../works/GombolayWS18.pdf}{GombolayWS18} (36.00)& \cellcolor{red!40}\href{../works/WessenCSFPM23.pdf}{WessenCSFPM23} (35.00)\\
Cosine& \cellcolor{red!40}\href{../works/SunLYL10.pdf}{SunLYL10} (0.79)& \cellcolor{red!40}\href{../works/SultanikMR07.pdf}{SultanikMR07} (0.76)& \cellcolor{red!40}\href{../works/CarchraeBF05.pdf}{CarchraeBF05} (0.69)& \cellcolor{red!40}\href{../works/RoweJCA96.pdf}{RoweJCA96} (0.69)& \cellcolor{red!40}\href{../works/AngelsmarkJ00.pdf}{AngelsmarkJ00} (0.68)\\
\index{HurleyOS16}\href{../works/HurleyOS16.pdf}{HurleyOS16} R\&C& \cellcolor{green!20}\href{../works/GalleguillosKSB19.pdf}{GalleguillosKSB19} (0.95)& \cellcolor{blue!20}\href{../works/IfrimOS12.pdf}{IfrimOS12} (0.97)& \cellcolor{blue!20}\href{../works/BridiBLMB16.pdf}{BridiBLMB16} (0.97)& \cellcolor{blue!20}\href{../works/LimHTB16.pdf}{LimHTB16} (0.97)\\
Euclid& \cellcolor{red!20}\href{../works/BockmayrP06.pdf}{BockmayrP06} (0.24)& \cellcolor{red!20}\href{../works/IfrimOS12.pdf}{IfrimOS12} (0.25)& \cellcolor{red!20}\href{../works/BeniniBGM05a.pdf}{BeniniBGM05a} (0.26)& \cellcolor{yellow!20}\href{../works/QuSN06.pdf}{QuSN06} (0.27)& \cellcolor{yellow!20}\href{../works/AkramNHRSA23.pdf}{AkramNHRSA23} (0.27)\\
Dot& \cellcolor{red!40}\href{../works/Groleaz21.pdf}{Groleaz21} (79.00)& \cellcolor{red!40}\href{../works/ZarandiASC20.pdf}{ZarandiASC20} (77.00)& \cellcolor{red!40}\href{../works/BorghesiBLMB18.pdf}{BorghesiBLMB18} (77.00)& \cellcolor{red!40}\href{../works/Lemos21.pdf}{Lemos21} (76.00)& \cellcolor{red!40}\href{../works/Beck99.pdf}{Beck99} (76.00)\\
Cosine& \cellcolor{red!40}\href{../works/BorghesiBLMB18.pdf}{BorghesiBLMB18} (0.74)& \cellcolor{red!40}\href{../works/BockmayrP06.pdf}{BockmayrP06} (0.73)& \cellcolor{red!40}\href{../works/IfrimOS12.pdf}{IfrimOS12} (0.73)& \cellcolor{red!40}\href{../works/GrimesIOS14.pdf}{GrimesIOS14} (0.71)& \cellcolor{red!40}\href{../works/Madi-WambaLOBM17.pdf}{Madi-WambaLOBM17} (0.68)\\
\index{IfrimOS12}\href{../works/IfrimOS12.pdf}{IfrimOS12} R\&C& \cellcolor{red!40}\href{../works/GrimesIOS14.pdf}{GrimesIOS14} (0.81)& \cellcolor{red!20}\href{../works/LimBTBB15a.pdf}{LimBTBB15a} (0.89)& \cellcolor{red!20}\href{../works/LimBTBB15.pdf}{LimBTBB15} (0.90)& \cellcolor{green!20}\href{../works/BartoliniBBLM14.pdf}{BartoliniBBLM14} (0.94)& \cellcolor{blue!20}\href{../works/HurleyOS16.pdf}{HurleyOS16} (0.97)\\
Euclid& \cellcolor{red!40}\href{../works/KinsellaS0OS16.pdf}{KinsellaS0OS16} (0.23)& \cellcolor{red!20}\href{../works/HurleyOS16.pdf}{HurleyOS16} (0.25)& \cellcolor{red!20}\href{../works/FoxAS82.pdf}{FoxAS82} (0.25)& \cellcolor{red!20}\href{../works/HoYCLLCLC18.pdf}{HoYCLLCLC18} (0.25)& \cellcolor{red!20}\href{../works/GrimesIOS14.pdf}{GrimesIOS14} (0.26)\\
Dot& \cellcolor{red!40}\href{../works/ZarandiASC20.pdf}{ZarandiASC20} (78.00)& \cellcolor{red!40}\href{../works/Groleaz21.pdf}{Groleaz21} (78.00)& \cellcolor{red!40}\href{../works/Froger16.pdf}{Froger16} (72.00)& \cellcolor{red!40}\href{../works/GrimesIOS14.pdf}{GrimesIOS14} (71.00)& \cellcolor{red!40}\href{../works/Astrand21.pdf}{Astrand21} (71.00)\\
Cosine& \cellcolor{red!40}\href{../works/GrimesIOS14.pdf}{GrimesIOS14} (0.78)& \cellcolor{red!40}\href{../works/HurleyOS16.pdf}{HurleyOS16} (0.73)& \cellcolor{red!40}\href{../works/TranPZLDB18.pdf}{TranPZLDB18} (0.71)& \cellcolor{red!40}\href{../works/KinsellaS0OS16.pdf}{KinsellaS0OS16} (0.68)& \cellcolor{red!40}\href{../works/FoxAS82.pdf}{FoxAS82} (0.66)\\
\index{IklassovMR023}\href{../works/IklassovMR023.pdf}{IklassovMR023} R\&C\\
Euclid& \cellcolor{red!20}\href{../works/KotaryFH22.pdf}{KotaryFH22} (0.26)& \cellcolor{yellow!20}\href{../works/Tassel22.pdf}{Tassel22} (0.27)& \cellcolor{yellow!20}\href{../works/LiLZDZW24.pdf}{LiLZDZW24} (0.27)& \cellcolor{green!20}\href{../works/DilkinaDH05.pdf}{DilkinaDH05} (0.29)& \cellcolor{green!20}\href{../works/CarchraeB09.pdf}{CarchraeB09} (0.29)\\
Dot& \cellcolor{red!40}\href{../works/ZarandiASC20.pdf}{ZarandiASC20} (126.00)& \cellcolor{red!40}\href{../works/Groleaz21.pdf}{Groleaz21} (118.00)& \cellcolor{red!40}\href{../works/abs-2211-14492.pdf}{abs-2211-14492} (109.00)& \cellcolor{red!40}\href{../works/Astrand21.pdf}{Astrand21} (109.00)& \cellcolor{red!40}\href{../works/Dejemeppe16.pdf}{Dejemeppe16} (109.00)\\
Cosine& \cellcolor{red!40}\href{../works/KotaryFH22.pdf}{KotaryFH22} (0.81)& \cellcolor{red!40}\href{../works/TasselGS23.pdf}{TasselGS23} (0.77)& \cellcolor{red!40}\href{../works/abs-2306-05747.pdf}{abs-2306-05747} (0.77)& \cellcolor{red!40}\href{../works/abs-2211-14492.pdf}{abs-2211-14492} (0.75)& \cellcolor{red!40}\href{../works/BeckFW11.pdf}{BeckFW11} (0.75)\\
\index{IsikYA23}\href{../works/IsikYA23.pdf}{IsikYA23} R\&C& \cellcolor{yellow!20}\href{../works/MengLZB21.pdf}{MengLZB21} (0.93)& \cellcolor{green!20}\href{../works/OujanaAYB22.pdf}{OujanaAYB22} (0.95)& \cellcolor{green!20}\href{../works/MengGRZSC22.pdf}{MengGRZSC22} (0.96)& \cellcolor{blue!20}WariZ19 (0.96)& \cellcolor{blue!20}\href{../works/NaderiRR23.pdf}{NaderiRR23} (0.97)\\
Euclid& \href{../works/MengZRZL20.pdf}{MengZRZL20} (0.38)& \href{../works/YunusogluY22.pdf}{YunusogluY22} (0.41)& \href{../works/Novas19.pdf}{Novas19} (0.42)& \href{../works/MengLZB21.pdf}{MengLZB21} (0.44)& \href{../works/ZhangJZL22.pdf}{ZhangJZL22} (0.45)\\
Dot& \cellcolor{red!40}\href{../works/ZarandiASC20.pdf}{ZarandiASC20} (276.00)& \cellcolor{red!40}\href{../works/Lunardi20.pdf}{Lunardi20} (246.00)& \cellcolor{red!40}\href{../works/Groleaz21.pdf}{Groleaz21} (245.00)& \cellcolor{red!40}\href{../works/Dejemeppe16.pdf}{Dejemeppe16} (228.00)& \cellcolor{red!40}\href{../works/MengZRZL20.pdf}{MengZRZL20} (225.00)\\
Cosine& \cellcolor{red!40}\href{../works/MengZRZL20.pdf}{MengZRZL20} (0.82)& \cellcolor{red!40}\href{../works/YunusogluY22.pdf}{YunusogluY22} (0.80)& \cellcolor{red!40}\href{../works/Novas19.pdf}{Novas19} (0.77)& \cellcolor{red!40}\href{../works/MengLZB21.pdf}{MengLZB21} (0.75)& \cellcolor{red!40}\href{../works/Lunardi20.pdf}{Lunardi20} (0.74)\\
\index{JainG01}\href{../works/JainG01.pdf}{JainG01} R\&C& \cellcolor{red!40}\href{../works/HarjunkoskiG02.pdf}{HarjunkoskiG02} (0.74)& \cellcolor{red!40}\href{../works/Thorsteinsson01.pdf}{Thorsteinsson01} (0.74)& \cellcolor{red!40}Hooker02 (0.80)& \cellcolor{red!40}CastroGR10 (0.81)& \cellcolor{red!40}\href{../works/MaraveliasCG04.pdf}{MaraveliasCG04} (0.84)\\
Euclid& \cellcolor{red!40}\href{../works/HarjunkoskiG02.pdf}{HarjunkoskiG02} (0.23)& \cellcolor{green!20}\href{../works/Colombani96.pdf}{Colombani96} (0.29)& \cellcolor{green!20}\href{../works/ZeballosM09.pdf}{ZeballosM09} (0.30)& \cellcolor{green!20}\href{../works/Hooker05a.pdf}{Hooker05a} (0.30)& \cellcolor{green!20}\href{../works/Thorsteinsson01.pdf}{Thorsteinsson01} (0.30)\\
Dot& \cellcolor{red!40}\href{../works/Baptiste02.pdf}{Baptiste02} (132.00)& \cellcolor{red!40}\href{../works/ZarandiASC20.pdf}{ZarandiASC20} (123.00)& \cellcolor{red!40}\href{../works/Groleaz21.pdf}{Groleaz21} (117.00)& \cellcolor{red!40}\href{../works/MilanoW06.pdf}{MilanoW06} (116.00)& \cellcolor{red!40}\href{../works/Malapert11.pdf}{Malapert11} (115.00)\\
Cosine& \cellcolor{red!40}\href{../works/HarjunkoskiG02.pdf}{HarjunkoskiG02} (0.85)& \cellcolor{red!40}\href{../works/ArtiguesLH13.pdf}{ArtiguesLH13} (0.75)& \cellcolor{red!40}\href{../works/ZeballosM09.pdf}{ZeballosM09} (0.75)& \cellcolor{red!40}\href{../works/Colombani96.pdf}{Colombani96} (0.75)& \cellcolor{red!40}\href{../works/Hooker06.pdf}{Hooker06} (0.75)\\
\index{JainM99}\href{../works/JainM99.pdf}{JainM99} R\&C& \cellcolor{red!40}\href{../works/BlazewiczDP96.pdf}{BlazewiczDP96} (0.66)& \cellcolor{red!20}DomdorfPH03 (0.86)& \cellcolor{red!20}DorndorfPH99 (0.88)& \cellcolor{red!20}\href{../works/ColT22.pdf}{ColT22} (0.89)& \cellcolor{yellow!20}DorndorfHP99 (0.93)\\
Euclid& \cellcolor{black!20}\href{../works/BlazewiczDP96.pdf}{BlazewiczDP96} (0.37)& \href{../works/MenciaSV13.pdf}{MenciaSV13} (0.38)& \href{../works/MalapertCGJLR13.pdf}{MalapertCGJLR13} (0.39)& \href{../works/KotaryFH22.pdf}{KotaryFH22} (0.40)& \href{../works/GrimesH10.pdf}{GrimesH10} (0.40)\\
Dot& \cellcolor{red!40}\href{../works/ZarandiASC20.pdf}{ZarandiASC20} (234.00)& \cellcolor{red!40}\href{../works/Groleaz21.pdf}{Groleaz21} (212.00)& \cellcolor{red!40}\href{../works/Astrand21.pdf}{Astrand21} (193.00)& \cellcolor{red!40}\href{../works/Baptiste02.pdf}{Baptiste02} (192.00)& \cellcolor{red!40}\href{../works/Lunardi20.pdf}{Lunardi20} (186.00)\\
Cosine& \cellcolor{red!40}\href{../works/BlazewiczDP96.pdf}{BlazewiczDP96} (0.80)& \cellcolor{red!40}\href{../works/MenciaSV13.pdf}{MenciaSV13} (0.73)& \cellcolor{red!40}\href{../works/BartakSR10.pdf}{BartakSR10} (0.72)& \cellcolor{red!40}\href{../works/BeckF98.pdf}{BeckF98} (0.72)& \cellcolor{red!40}\href{../works/GrimesH10.pdf}{GrimesH10} (0.71)\\
\index{Jans09}\href{../works/Jans09.pdf}{Jans09} R\&C& \cellcolor{green!20}CastroGR10 (0.94)& \cellcolor{green!20}\href{../works/ChenGPSH10.pdf}{ChenGPSH10} (0.94)& \cellcolor{green!20}\href{../works/CobanH11.pdf}{CobanH11} (0.95)& \cellcolor{green!20}\href{../works/CireCH16.pdf}{CireCH16} (0.95)& \cellcolor{green!20}\href{../works/CobanH10.pdf}{CobanH10} (0.95)\\
Euclid& \cellcolor{yellow!20}\href{../works/LauLN08.pdf}{LauLN08} (0.27)& \cellcolor{green!20}\href{../works/BenediktSMVH18.pdf}{BenediktSMVH18} (0.29)& \cellcolor{green!20}\href{../works/BogaerdtW19.pdf}{BogaerdtW19} (0.30)& \cellcolor{green!20}\href{../works/CrawfordB94.pdf}{CrawfordB94} (0.30)& \cellcolor{green!20}\href{../works/DavenportKRSH07.pdf}{DavenportKRSH07} (0.30)\\
Dot& \cellcolor{red!40}\href{../works/ZarandiASC20.pdf}{ZarandiASC20} (92.00)& \cellcolor{red!40}\href{../works/Groleaz21.pdf}{Groleaz21} (91.00)& \cellcolor{red!40}\href{../works/Astrand21.pdf}{Astrand21} (87.00)& \cellcolor{red!40}\href{../works/PrataAN23.pdf}{PrataAN23} (83.00)& \cellcolor{red!40}\href{../works/Baptiste02.pdf}{Baptiste02} (80.00)\\
Cosine& \cellcolor{red!40}\href{../works/DavenportKRSH07.pdf}{DavenportKRSH07} (0.67)& \cellcolor{red!40}\href{../works/ArbaouiY18.pdf}{ArbaouiY18} (0.66)& \cellcolor{red!40}\href{../works/LauLN08.pdf}{LauLN08} (0.66)& \cellcolor{red!40}\href{../works/BogaerdtW19.pdf}{BogaerdtW19} (0.66)& \cellcolor{red!40}\href{../works/BenediktSMVH18.pdf}{BenediktSMVH18} (0.65)\\
\index{JelinekB16}\href{../works/JelinekB16.pdf}{JelinekB16} R\&C& \cellcolor{yellow!20}\href{../works/ReddyFIBKAJ11.pdf}{ReddyFIBKAJ11} (0.92)& \cellcolor{green!20}\href{../works/BoothTNB16.pdf}{BoothTNB16} (0.96)& \cellcolor{blue!20}BaptisteLPN06 (0.97)& \cellcolor{black!20}\href{../works/HarjunkoskiMBC14.pdf}{HarjunkoskiMBC14} (0.99)\\
Euclid& \cellcolor{red!40}\href{../works/AngelsmarkJ00.pdf}{AngelsmarkJ00} (0.19)& \cellcolor{red!40}\href{../works/CarchraeBF05.pdf}{CarchraeBF05} (0.19)& \cellcolor{red!40}\href{../works/FrostD98.pdf}{FrostD98} (0.19)& \cellcolor{red!40}\href{../works/FeldmanG89.pdf}{FeldmanG89} (0.19)& \cellcolor{red!40}\href{../works/Davis87.pdf}{Davis87} (0.19)\\
Dot& \cellcolor{red!40}\href{../works/Baptiste02.pdf}{Baptiste02} (47.00)& \cellcolor{red!40}\href{../works/Lombardi10.pdf}{Lombardi10} (43.00)& \cellcolor{red!40}\href{../works/Simonis99.pdf}{Simonis99} (41.00)& \cellcolor{red!40}\href{../works/ZarandiASC20.pdf}{ZarandiASC20} (38.00)& \cellcolor{red!40}\href{../works/Schutt11.pdf}{Schutt11} (38.00)\\
Cosine& \cellcolor{red!40}\href{../works/FeldmanG89.pdf}{FeldmanG89} (0.62)& \cellcolor{red!40}\href{../works/FalaschiGMP97.pdf}{FalaschiGMP97} (0.61)& \cellcolor{red!40}\href{../works/AngelsmarkJ00.pdf}{AngelsmarkJ00} (0.60)& \cellcolor{red!40}\href{../works/FukunagaHFAMN02.pdf}{FukunagaHFAMN02} (0.60)& \cellcolor{red!40}\href{../works/Schaerf97.pdf}{Schaerf97} (0.59)\\
\index{JoLLH99}\href{../works/JoLLH99.pdf}{JoLLH99} R\&C\\
Euclid& \cellcolor{red!20}\href{../works/TranDRFWOVB16.pdf}{TranDRFWOVB16} (0.25)& \cellcolor{green!20}\href{../works/DincbasS91.pdf}{DincbasS91} (0.30)& \cellcolor{green!20}\href{../works/AngelsmarkJ00.pdf}{AngelsmarkJ00} (0.30)& \cellcolor{green!20}\href{../works/WallaceF00.pdf}{WallaceF00} (0.31)& \cellcolor{green!20}\href{../works/Rit86.pdf}{Rit86} (0.31)\\
Dot& \cellcolor{red!40}\href{../works/ZarandiASC20.pdf}{ZarandiASC20} (91.00)& \cellcolor{red!40}\href{../works/Fahimi16.pdf}{Fahimi16} (79.00)& \cellcolor{red!40}\href{../works/Simonis07.pdf}{Simonis07} (77.00)& \cellcolor{red!40}\href{../works/Simonis99.pdf}{Simonis99} (75.00)& \cellcolor{red!40}\href{../works/Baptiste02.pdf}{Baptiste02} (75.00)\\
Cosine& \cellcolor{red!40}\href{../works/TranDRFWOVB16.pdf}{TranDRFWOVB16} (0.75)& \cellcolor{red!40}\href{../works/BartakV15.pdf}{BartakV15} (0.61)& \cellcolor{red!40}\href{../works/ChunCTY99.pdf}{ChunCTY99} (0.60)& \cellcolor{red!40}\href{../works/Prosser89.pdf}{Prosser89} (0.60)& \cellcolor{red!40}\href{../works/Wallace96.pdf}{Wallace96} (0.60)\\
\index{Johnston05}\href{../works/Johnston05.pdf}{Johnston05} R\&C\\
Euclid& \cellcolor{red!20}\href{../works/WallaceF00.pdf}{WallaceF00} (0.26)& \cellcolor{yellow!20}\href{../works/Maillard15.pdf}{Maillard15} (0.27)& \cellcolor{yellow!20}\href{../works/LudwigKRBMS14.pdf}{LudwigKRBMS14} (0.28)& \cellcolor{yellow!20}\href{../works/FukunagaHFAMN02.pdf}{FukunagaHFAMN02} (0.28)& \cellcolor{green!20}\href{../works/AngelsmarkJ00.pdf}{AngelsmarkJ00} (0.29)\\
Dot& \cellcolor{red!40}\href{../works/ZarandiASC20.pdf}{ZarandiASC20} (70.00)& \cellcolor{red!40}\href{../works/LaborieRSV18.pdf}{LaborieRSV18} (68.00)& \cellcolor{red!40}\href{../works/Astrand21.pdf}{Astrand21} (68.00)& \cellcolor{red!40}\href{../works/Lombardi10.pdf}{Lombardi10} (66.00)& \cellcolor{red!40}\href{../works/HarjunkoskiMBC14.pdf}{HarjunkoskiMBC14} (64.00)\\
Cosine& \cellcolor{red!40}\href{../works/WallaceF00.pdf}{WallaceF00} (0.66)& \cellcolor{red!40}\href{../works/PembertonG98.pdf}{PembertonG98} (0.65)& \cellcolor{red!40}\href{../works/SimoninAHL12.pdf}{SimoninAHL12} (0.65)& \cellcolor{red!40}\href{../works/ReddyFIBKAJ11.pdf}{ReddyFIBKAJ11} (0.63)& \cellcolor{red!40}\href{../works/FrankK05.pdf}{FrankK05} (0.63)\\
\index{JourdanFRD94}JourdanFRD94 R\&C\\
Euclid\\
Dot\\
Cosine\\
\index{JungblutK22}\href{../works/JungblutK22.pdf}{JungblutK22} R\&C\\
Euclid& \cellcolor{red!40}\href{../works/GomesHS06.pdf}{GomesHS06} (0.24)& \cellcolor{red!40}\href{../works/FukunagaHFAMN02.pdf}{FukunagaHFAMN02} (0.24)& \cellcolor{red!40}\href{../works/AngelsmarkJ00.pdf}{AngelsmarkJ00} (0.24)& \cellcolor{red!20}\href{../works/CarchraeBF05.pdf}{CarchraeBF05} (0.24)& \cellcolor{red!20}\href{../works/BockmayrP06.pdf}{BockmayrP06} (0.24)\\
Dot& \cellcolor{red!40}\href{../works/Godet21a.pdf}{Godet21a} (70.00)& \cellcolor{red!40}\href{../works/ColT22.pdf}{ColT22} (69.00)& \cellcolor{red!40}\href{../works/KoehlerBFFHPSSS21.pdf}{KoehlerBFFHPSSS21} (64.00)& \cellcolor{red!40}\href{../works/Lombardi10.pdf}{Lombardi10} (63.00)& \cellcolor{red!40}\href{../works/Beck99.pdf}{Beck99} (63.00)\\
Cosine& \cellcolor{red!40}\href{../works/abs-1901-07914.pdf}{abs-1901-07914} (0.74)& \cellcolor{red!40}\href{../works/BehrensLM19.pdf}{BehrensLM19} (0.70)& \cellcolor{red!40}\href{../works/BockmayrP06.pdf}{BockmayrP06} (0.69)& \cellcolor{red!40}\href{../works/AkramNHRSA23.pdf}{AkramNHRSA23} (0.66)& \cellcolor{red!40}\href{../works/BridiLBBM16.pdf}{BridiLBBM16} (0.63)\\
\index{Junker00}\href{../works/Junker00.pdf}{Junker00} R\&C\\
Euclid& \cellcolor{red!20}\href{../works/Muscettola94.pdf}{Muscettola94} (0.26)& \cellcolor{red!20}\href{../works/OddiS97.pdf}{OddiS97} (0.26)& \cellcolor{red!20}\href{../works/BartakCS10.pdf}{BartakCS10} (0.26)& \cellcolor{red!20}\href{../works/LeeKLKKYHP97.pdf}{LeeKLKKYHP97} (0.26)& \cellcolor{red!20}\href{../works/BeckDSF97a.pdf}{BeckDSF97a} (0.26)\\
Dot& \cellcolor{red!40}\href{../works/Beck99.pdf}{Beck99} (88.00)& \cellcolor{red!40}\href{../works/Baptiste02.pdf}{Baptiste02} (88.00)& \cellcolor{red!40}\href{../works/ZarandiASC20.pdf}{ZarandiASC20} (86.00)& \cellcolor{red!40}\href{../works/Lombardi10.pdf}{Lombardi10} (86.00)& \cellcolor{red!40}\href{../works/Fahimi16.pdf}{Fahimi16} (85.00)\\
Cosine& \cellcolor{red!40}\href{../works/BeckDSF97a.pdf}{BeckDSF97a} (0.76)& \cellcolor{red!40}\href{../works/BeckDSF97.pdf}{BeckDSF97} (0.75)& \cellcolor{red!40}\href{../works/BeckPS03.pdf}{BeckPS03} (0.74)& \cellcolor{red!40}\href{../works/SadehF96.pdf}{SadehF96} (0.74)& \cellcolor{red!40}\href{../works/Muscettola94.pdf}{Muscettola94} (0.74)\\
\index{JussienL02}\href{../works/JussienL02.pdf}{JussienL02} R\&C& \cellcolor{red!40}\href{../works/MalapertCGJLR12.pdf}{MalapertCGJLR12} (0.86)& \cellcolor{red!40}\href{../works/Colombani96.pdf}{Colombani96} (0.86)& \cellcolor{red!20}\href{../works/Zhou96.pdf}{Zhou96} (0.87)& \cellcolor{red!20}\href{../works/BeckF00.pdf}{BeckF00} (0.88)& \cellcolor{red!20}\href{../works/Dorndorf2000.pdf}{Dorndorf2000} (0.90)\\
Euclid& \cellcolor{yellow!20}\href{../works/MalapertCGJLR13.pdf}{MalapertCGJLR13} (0.28)& \cellcolor{green!20}\href{../works/KovacsV04.pdf}{KovacsV04} (0.30)& \cellcolor{green!20}\href{../works/BarbulescuWH04.pdf}{BarbulescuWH04} (0.30)& \cellcolor{green!20}\href{../works/GarridoAO09.pdf}{GarridoAO09} (0.31)& \cellcolor{blue!20}\href{../works/Bartak02.pdf}{Bartak02} (0.32)\\
Dot& \cellcolor{red!40}\href{../works/ZarandiASC20.pdf}{ZarandiASC20} (142.00)& \cellcolor{red!40}\href{../works/Godet21a.pdf}{Godet21a} (128.00)& \cellcolor{red!40}\href{../works/Groleaz21.pdf}{Groleaz21} (125.00)& \cellcolor{red!40}\href{../works/Baptiste02.pdf}{Baptiste02} (125.00)& \cellcolor{red!40}\href{../works/Lombardi10.pdf}{Lombardi10} (124.00)\\
Cosine& \cellcolor{red!40}\href{../works/MalapertCGJLR13.pdf}{MalapertCGJLR13} (0.77)& \cellcolor{red!40}\href{../works/KovacsV04.pdf}{KovacsV04} (0.74)& \cellcolor{red!40}\href{../works/PengLC14.pdf}{PengLC14} (0.72)& \cellcolor{red!40}\href{../works/GarridoAO09.pdf}{GarridoAO09} (0.72)& \cellcolor{red!40}\href{../works/BarbulescuWH04.pdf}{BarbulescuWH04} (0.72)\\
\index{JuvinHHL23}\href{../works/JuvinHHL23.pdf}{JuvinHHL23} R\&C\\
Euclid& \href{../works/MenciaSV13.pdf}{MenciaSV13} (0.38)& \href{../works/JuvinHL22.pdf}{JuvinHL22} (0.39)& \href{../works/TorresL00.pdf}{TorresL00} (0.39)& \href{../works/JuvinHL23a.pdf}{JuvinHL23a} (0.40)& \href{../works/TanSD10.pdf}{TanSD10} (0.40)\\
Dot& \cellcolor{red!40}\href{../works/Baptiste02.pdf}{Baptiste02} (203.00)& \cellcolor{red!40}\href{../works/Godet21a.pdf}{Godet21a} (198.00)& \cellcolor{red!40}\href{../works/Malapert11.pdf}{Malapert11} (197.00)& \cellcolor{red!40}\href{../works/Groleaz21.pdf}{Groleaz21} (195.00)& \cellcolor{red!40}\href{../works/Dejemeppe16.pdf}{Dejemeppe16} (190.00)\\
Cosine& \cellcolor{red!40}\href{../works/JuvinHL22.pdf}{JuvinHL22} (0.77)& \cellcolor{red!40}\href{../works/MenciaSV13.pdf}{MenciaSV13} (0.77)& \cellcolor{red!40}\href{../works/JuvinHL23a.pdf}{JuvinHL23a} (0.75)& \cellcolor{red!40}\href{../works/TorresL00.pdf}{TorresL00} (0.74)& \cellcolor{red!40}\href{../works/PapaB98.pdf}{PapaB98} (0.74)\\
\index{JuvinHL22}\href{../works/JuvinHL22.pdf}{JuvinHL22} R\&C& \cellcolor{red!40}NaderiR22 (0.86)& \cellcolor{red!20}\href{../works/SourdN00.pdf}{SourdN00} (0.88)& \cellcolor{yellow!20}\href{../works/MullerMKP22.pdf}{MullerMKP22} (0.91)& \cellcolor{yellow!20}\href{../works/NaderiRR23.pdf}{NaderiRR23} (0.91)& \cellcolor{yellow!20}\href{../works/Wolf03.pdf}{Wolf03} (0.92)\\
Euclid& \cellcolor{red!40}\href{../works/JuvinHL23a.pdf}{JuvinHL23a} (0.23)& \cellcolor{black!20}\href{../works/TanT18.pdf}{TanT18} (0.35)& \cellcolor{black!20}\href{../works/GuyonLPR12.pdf}{GuyonLPR12} (0.36)& \cellcolor{black!20}\href{../works/NaderiBZ22a.pdf}{NaderiBZ22a} (0.37)& \cellcolor{black!20}\href{../works/MenciaSV13.pdf}{MenciaSV13} (0.37)\\
Dot& \cellcolor{red!40}\href{../works/Groleaz21.pdf}{Groleaz21} (196.00)& \cellcolor{red!40}\href{../works/Baptiste02.pdf}{Baptiste02} (192.00)& \cellcolor{red!40}\href{../works/ZarandiASC20.pdf}{ZarandiASC20} (189.00)& \cellcolor{red!40}\href{../works/JuvinHL23a.pdf}{JuvinHL23a} (189.00)& \cellcolor{red!40}\href{../works/NaderiRR23.pdf}{NaderiRR23} (179.00)\\
Cosine& \cellcolor{red!40}\href{../works/JuvinHL23a.pdf}{JuvinHL23a} (0.91)& \cellcolor{red!40}\href{../works/NaderiBZ22a.pdf}{NaderiBZ22a} (0.77)& \cellcolor{red!40}\href{../works/JuvinHHL23.pdf}{JuvinHHL23} (0.77)& \cellcolor{red!40}\href{../works/TanT18.pdf}{TanT18} (0.77)& \cellcolor{red!40}\href{../works/GuyonLPR12.pdf}{GuyonLPR12} (0.76)\\
\index{JuvinHL23}\href{../works/JuvinHL23.pdf}{JuvinHL23} R\&C& \cellcolor{blue!20}\href{../works/KuB16.pdf}{KuB16} (0.96)& \cellcolor{blue!20}\href{../works/BillautHL12.pdf}{BillautHL12} (0.97)& \cellcolor{blue!20}\href{../works/NaderiRR23.pdf}{NaderiRR23} (0.97)& \cellcolor{blue!20}\href{../works/GrimesH10.pdf}{GrimesH10} (0.97)& \cellcolor{blue!20}BriandHHL08 (0.97)\\
Euclid& \cellcolor{red!20}\href{../works/ParkUJR19.pdf}{ParkUJR19} (0.26)& \cellcolor{yellow!20}\href{../works/BillautHL12.pdf}{BillautHL12} (0.27)& \cellcolor{yellow!20}\href{../works/LiFJZLL22.pdf}{LiFJZLL22} (0.27)& \cellcolor{yellow!20}\href{../works/WessenCS20.pdf}{WessenCS20} (0.28)& \cellcolor{green!20}\href{../works/HamdiL13.pdf}{HamdiL13} (0.29)\\
Dot& \cellcolor{red!40}\href{../works/Lunardi20.pdf}{Lunardi20} (122.00)& \cellcolor{red!40}\href{../works/Groleaz21.pdf}{Groleaz21} (120.00)& \cellcolor{red!40}\href{../works/ZarandiASC20.pdf}{ZarandiASC20} (111.00)& \cellcolor{red!40}\href{../works/ColT22.pdf}{ColT22} (109.00)& \cellcolor{red!40}\href{../works/Astrand21.pdf}{Astrand21} (108.00)\\
Cosine& \cellcolor{red!40}\href{../works/ParkUJR19.pdf}{ParkUJR19} (0.80)& \cellcolor{red!40}\href{../works/LiFJZLL22.pdf}{LiFJZLL22} (0.79)& \cellcolor{red!40}\href{../works/BillautHL12.pdf}{BillautHL12} (0.76)& \cellcolor{red!40}\href{../works/CzerniachowskaWZ23.pdf}{CzerniachowskaWZ23} (0.76)& \cellcolor{red!40}\href{../works/HamdiL13.pdf}{HamdiL13} (0.75)\\
\index{JuvinHL23a}\href{../works/JuvinHL23a.pdf}{JuvinHL23a} R\&C& \cellcolor{red!40}NaderiR22 (0.84)& \cellcolor{red!20}\href{../works/SourdN00.pdf}{SourdN00} (0.89)& \cellcolor{yellow!20}\href{../works/NaderiRR23.pdf}{NaderiRR23} (0.91)& \cellcolor{yellow!20}\href{../works/MullerMKP22.pdf}{MullerMKP22} (0.92)& \cellcolor{yellow!20}\href{../works/ColT22.pdf}{ColT22} (0.93)\\
Euclid& \cellcolor{red!40}\href{../works/JuvinHL22.pdf}{JuvinHL22} (0.23)& \cellcolor{black!20}\href{../works/NaderiBZ22a.pdf}{NaderiBZ22a} (0.36)& \href{../works/Teppan22.pdf}{Teppan22} (0.38)& \href{../works/GuyonLPR12.pdf}{GuyonLPR12} (0.38)& \href{../works/MenciaSV13.pdf}{MenciaSV13} (0.38)\\
Dot& \cellcolor{red!40}\href{../works/ZarandiASC20.pdf}{ZarandiASC20} (192.00)& \cellcolor{red!40}\href{../works/Groleaz21.pdf}{Groleaz21} (190.00)& \cellcolor{red!40}\href{../works/JuvinHL22.pdf}{JuvinHL22} (189.00)& \cellcolor{red!40}\href{../works/Baptiste02.pdf}{Baptiste02} (177.00)& \cellcolor{red!40}\href{../works/Lombardi10.pdf}{Lombardi10} (174.00)\\
Cosine& \cellcolor{red!40}\href{../works/JuvinHL22.pdf}{JuvinHL22} (0.91)& \cellcolor{red!40}\href{../works/NaderiBZ22a.pdf}{NaderiBZ22a} (0.78)& \cellcolor{red!40}\href{../works/JuvinHHL23.pdf}{JuvinHHL23} (0.75)& \cellcolor{red!40}\href{../works/GuyonLPR12.pdf}{GuyonLPR12} (0.73)& \cellcolor{red!40}\href{../works/Teppan22.pdf}{Teppan22} (0.72)\\
\index{KamarainenS02}\href{../works/KamarainenS02.pdf}{KamarainenS02} R\&C& \cellcolor{red!40}\href{../works/KhemmoudjPB06.pdf}{KhemmoudjPB06} (0.86)& \cellcolor{red!20}\href{../works/Wallace06.pdf}{Wallace06} (0.88)& \cellcolor{red!20}\href{../works/EreminW01.pdf}{EreminW01} (0.89)& \cellcolor{red!20}CestaOPS14 (0.90)& \cellcolor{yellow!20}\href{../works/MeyerE04.pdf}{MeyerE04} (0.90)\\
Euclid& \cellcolor{red!20}\href{../works/ZibranR11.pdf}{ZibranR11} (0.26)& \cellcolor{red!20}\href{../works/ZibranR11a.pdf}{ZibranR11a} (0.26)& \cellcolor{yellow!20}\href{../works/MakMS10.pdf}{MakMS10} (0.27)& \cellcolor{green!20}\href{../works/ZhangLS12.pdf}{ZhangLS12} (0.29)& \cellcolor{green!20}\href{../works/WallaceF00.pdf}{WallaceF00} (0.29)\\
Dot& \cellcolor{red!40}\href{../works/ZarandiASC20.pdf}{ZarandiASC20} (100.00)& \cellcolor{red!40}\href{../works/Beck99.pdf}{Beck99} (93.00)& \cellcolor{red!40}\href{../works/Dejemeppe16.pdf}{Dejemeppe16} (90.00)& \cellcolor{red!40}\href{../works/Lombardi10.pdf}{Lombardi10} (88.00)& \cellcolor{red!40}\href{../works/Astrand21.pdf}{Astrand21} (88.00)\\
Cosine& \cellcolor{red!40}\href{../works/SakkoutW00.pdf}{SakkoutW00} (0.75)& \cellcolor{red!40}\href{../works/HeckmanB11.pdf}{HeckmanB11} (0.72)& \cellcolor{red!40}\href{../works/MakMS10.pdf}{MakMS10} (0.70)& \cellcolor{red!40}\href{../works/ZibranR11.pdf}{ZibranR11} (0.68)& \cellcolor{red!40}\href{../works/ZibranR11a.pdf}{ZibranR11a} (0.68)\\
\index{Kameugne14}\href{../works/Kameugne14.pdf}{Kameugne14} R\&C\\
Euclid& \cellcolor{black!20}\href{../works/KameugneFSN14.pdf}{KameugneFSN14} (0.37)& \href{../works/Demassey03.pdf}{Demassey03} (0.39)& \href{../works/KameugneFND23.pdf}{KameugneFND23} (0.39)& \href{../works/GayHS15a.pdf}{GayHS15a} (0.40)& \href{../works/Clercq12.pdf}{Clercq12} (0.40)\\
Dot& \cellcolor{red!40}\href{../works/Malapert11.pdf}{Malapert11} (197.00)& \cellcolor{red!40}\href{../works/Fahimi16.pdf}{Fahimi16} (197.00)& \cellcolor{red!40}\href{../works/Baptiste02.pdf}{Baptiste02} (188.00)& \cellcolor{red!40}\href{../works/Schutt11.pdf}{Schutt11} (186.00)& \cellcolor{red!40}\href{../works/Demassey03.pdf}{Demassey03} (173.00)\\
Cosine& \cellcolor{red!40}\href{../works/Demassey03.pdf}{Demassey03} (0.77)& \cellcolor{red!40}\href{../works/KameugneFSN14.pdf}{KameugneFSN14} (0.76)& \cellcolor{red!40}\href{../works/KameugneFND23.pdf}{KameugneFND23} (0.74)& \cellcolor{red!40}\href{../works/Clercq12.pdf}{Clercq12} (0.73)& \cellcolor{red!40}\href{../works/Fahimi16.pdf}{Fahimi16} (0.72)\\
\index{Kameugne15}\href{../works/Kameugne15.pdf}{Kameugne15} R\&C\\
Euclid& \cellcolor{red!40}\href{../works/Caseau97.pdf}{Caseau97} (0.22)& \cellcolor{red!40}\href{../works/Vilim09a.pdf}{Vilim09a} (0.23)& \cellcolor{red!40}\href{../works/Vilim02.pdf}{Vilim02} (0.23)& \cellcolor{red!40}\href{../works/KameugneF13.pdf}{KameugneF13} (0.23)& \cellcolor{red!40}\href{../works/CestaOS98.pdf}{CestaOS98} (0.23)\\
Dot& \cellcolor{red!40}\href{../works/Schutt11.pdf}{Schutt11} (63.00)& \cellcolor{red!40}\href{../works/Dejemeppe16.pdf}{Dejemeppe16} (63.00)& \cellcolor{red!40}\href{../works/Baptiste02.pdf}{Baptiste02} (63.00)& \cellcolor{red!40}\href{../works/Fahimi16.pdf}{Fahimi16} (63.00)& \cellcolor{red!40}\href{../works/FahimiOQ18.pdf}{FahimiOQ18} (62.00)\\
Cosine& \cellcolor{red!40}\href{../works/Vilim09a.pdf}{Vilim09a} (0.71)& \cellcolor{red!40}\href{../works/GayHS15a.pdf}{GayHS15a} (0.70)& \cellcolor{red!40}\href{../works/Vilim09.pdf}{Vilim09} (0.70)& \cellcolor{red!40}\href{../works/OuelletQ13.pdf}{OuelletQ13} (0.68)& \cellcolor{red!40}\href{../works/Caseau97.pdf}{Caseau97} (0.67)\\
\index{KameugneF13}\href{../works/KameugneF13.pdf}{KameugneF13} R\&C& \cellcolor{red!40}\href{../works/OuelletQ18.pdf}{OuelletQ18} (0.60)& \cellcolor{red!40}\href{../works/OuelletQ13.pdf}{OuelletQ13} (0.61)& \cellcolor{red!40}\href{../works/KameugneFSN11.pdf}{KameugneFSN11} (0.62)& \cellcolor{red!40}\href{../works/SchuttW10.pdf}{SchuttW10} (0.62)& \cellcolor{red!40}\href{../works/Vilim09a.pdf}{Vilim09a} (0.64)\\
Euclid& \cellcolor{red!40}\href{../works/MaraveliasG04.pdf}{MaraveliasG04} (0.13)& \cellcolor{red!40}\href{../works/Baptiste09.pdf}{Baptiste09} (0.15)& \cellcolor{red!40}\href{../works/AbrilSB05.pdf}{AbrilSB05} (0.15)& \cellcolor{red!40}\href{../works/CarchraeBF05.pdf}{CarchraeBF05} (0.15)& \cellcolor{red!40}\href{../works/AngelsmarkJ00.pdf}{AngelsmarkJ00} (0.16)\\
Dot& \cellcolor{red!40}\href{../works/MercierH07.pdf}{MercierH07} (18.00)& \cellcolor{red!40}\href{../works/Baptiste02.pdf}{Baptiste02} (18.00)& \cellcolor{red!40}\href{../works/KameugneFSN14.pdf}{KameugneFSN14} (17.00)& \cellcolor{red!40}\href{../works/Lombardi10.pdf}{Lombardi10} (17.00)& \cellcolor{red!40}\href{../works/Dejemeppe16.pdf}{Dejemeppe16} (17.00)\\
Cosine& \cellcolor{red!40}\href{../works/PoderB08.pdf}{PoderB08} (0.58)& \cellcolor{red!40}\href{../works/SchuttWS05.pdf}{SchuttWS05} (0.55)& \cellcolor{red!40}\href{../works/ChuGNSW13.pdf}{ChuGNSW13} (0.53)& \cellcolor{red!40}\href{../works/BeldiceanuP07.pdf}{BeldiceanuP07} (0.53)& \cellcolor{red!40}\href{../works/CarlierPSJ20.pdf}{CarlierPSJ20} (0.53)\\
\index{KameugneFGOQ18}\href{../works/KameugneFGOQ18.pdf}{KameugneFGOQ18} R\&C& \cellcolor{red!40}\href{../works/Tesch16.pdf}{Tesch16} (0.61)& \cellcolor{red!40}\href{../works/OuelletQ18.pdf}{OuelletQ18} (0.64)& \cellcolor{red!40}\href{../works/GayHS15a.pdf}{GayHS15a} (0.67)& \cellcolor{red!40}\href{../works/Tesch18.pdf}{Tesch18} (0.68)& \cellcolor{red!40}\href{../works/FetgoD22.pdf}{FetgoD22} (0.72)\\
Euclid& \cellcolor{red!40}\href{../works/KameugneFND23.pdf}{KameugneFND23} (0.23)& \cellcolor{red!20}\href{../works/FetgoD22.pdf}{FetgoD22} (0.26)& \cellcolor{yellow!20}\href{../works/OuelletQ13.pdf}{OuelletQ13} (0.27)& \cellcolor{green!20}\href{../works/SchuttW10.pdf}{SchuttW10} (0.29)& \cellcolor{green!20}\href{../works/OuelletQ18.pdf}{OuelletQ18} (0.30)\\
Dot& \cellcolor{red!40}\href{../works/Baptiste02.pdf}{Baptiste02} (144.00)& \cellcolor{red!40}\href{../works/Fahimi16.pdf}{Fahimi16} (142.00)& \cellcolor{red!40}\href{../works/Schutt11.pdf}{Schutt11} (138.00)& \cellcolor{red!40}\href{../works/FetgoD22.pdf}{FetgoD22} (138.00)& \cellcolor{red!40}\href{../works/KameugneFND23.pdf}{KameugneFND23} (136.00)\\
Cosine& \cellcolor{red!40}\href{../works/KameugneFND23.pdf}{KameugneFND23} (0.89)& \cellcolor{red!40}\href{../works/FetgoD22.pdf}{FetgoD22} (0.86)& \cellcolor{red!40}\href{../works/OuelletQ13.pdf}{OuelletQ13} (0.82)& \cellcolor{red!40}\href{../works/SchuttW10.pdf}{SchuttW10} (0.78)& \cellcolor{red!40}\href{../works/OuelletQ18.pdf}{OuelletQ18} (0.77)\\
\index{KameugneFND23}\href{../works/KameugneFND23.pdf}{KameugneFND23} R\&C\\
Euclid& \cellcolor{red!40}\href{../works/OuelletQ13.pdf}{OuelletQ13} (0.22)& \cellcolor{red!40}\href{../works/KameugneFGOQ18.pdf}{KameugneFGOQ18} (0.23)& \cellcolor{red!40}\href{../works/FetgoD22.pdf}{FetgoD22} (0.23)& \cellcolor{yellow!20}\href{../works/GingrasQ16.pdf}{GingrasQ16} (0.27)& \cellcolor{yellow!20}\href{../works/KameugneFSN14.pdf}{KameugneFSN14} (0.27)\\
Dot& \cellcolor{red!40}\href{../works/Schutt11.pdf}{Schutt11} (163.00)& \cellcolor{red!40}\href{../works/FetgoD22.pdf}{FetgoD22} (155.00)& \cellcolor{red!40}\href{../works/Baptiste02.pdf}{Baptiste02} (154.00)& \cellcolor{red!40}\href{../works/Fahimi16.pdf}{Fahimi16} (152.00)& \cellcolor{red!40}\href{../works/Malapert11.pdf}{Malapert11} (149.00)\\
Cosine& \cellcolor{red!40}\href{../works/FetgoD22.pdf}{FetgoD22} (0.90)& \cellcolor{red!40}\href{../works/OuelletQ13.pdf}{OuelletQ13} (0.89)& \cellcolor{red!40}\href{../works/KameugneFGOQ18.pdf}{KameugneFGOQ18} (0.89)& \cellcolor{red!40}\href{../works/GingrasQ16.pdf}{GingrasQ16} (0.84)& \cellcolor{red!40}\href{../works/KameugneFSN14.pdf}{KameugneFSN14} (0.84)\\
\index{KameugneFSN11}\href{../works/KameugneFSN11.pdf}{KameugneFSN11} R\&C& \cellcolor{red!40}\href{../works/Vilim09.pdf}{Vilim09} (0.58)& \cellcolor{red!40}\href{../works/KameugneF13.pdf}{KameugneF13} (0.62)& \cellcolor{red!40}\href{../works/SchuttW10.pdf}{SchuttW10} (0.63)& \cellcolor{red!40}\href{../works/KameugneFSN14.pdf}{KameugneFSN14} (0.68)& \cellcolor{red!40}\href{../works/OuelletQ13.pdf}{OuelletQ13} (0.69)\\
Euclid& \cellcolor{red!40}\href{../works/KameugneFSN14.pdf}{KameugneFSN14} (0.20)& \cellcolor{red!20}\href{../works/OuelletQ13.pdf}{OuelletQ13} (0.26)& \cellcolor{red!20}\href{../works/SchuttW10.pdf}{SchuttW10} (0.26)& \cellcolor{yellow!20}\href{../works/OuelletQ18.pdf}{OuelletQ18} (0.28)& \cellcolor{yellow!20}\href{../works/GingrasQ16.pdf}{GingrasQ16} (0.28)\\
Dot& \cellcolor{red!40}\href{../works/Fahimi16.pdf}{Fahimi16} (128.00)& \cellcolor{red!40}\href{../works/Schutt11.pdf}{Schutt11} (127.00)& \cellcolor{red!40}\href{../works/Baptiste02.pdf}{Baptiste02} (126.00)& \cellcolor{red!40}\href{../works/Dejemeppe16.pdf}{Dejemeppe16} (123.00)& \cellcolor{red!40}\href{../works/Lombardi10.pdf}{Lombardi10} (121.00)\\
Cosine& \cellcolor{red!40}\href{../works/KameugneFSN14.pdf}{KameugneFSN14} (0.90)& \cellcolor{red!40}\href{../works/OuelletQ13.pdf}{OuelletQ13} (0.82)& \cellcolor{red!40}\href{../works/SchuttW10.pdf}{SchuttW10} (0.81)& \cellcolor{red!40}\href{../works/KameugneFND23.pdf}{KameugneFND23} (0.78)& \cellcolor{red!40}\href{../works/OuelletQ18.pdf}{OuelletQ18} (0.76)\\
\index{KameugneFSN14}\href{../works/KameugneFSN14.pdf}{KameugneFSN14} R\&C& \cellcolor{red!40}\href{../works/OuelletQ13.pdf}{OuelletQ13} (0.54)& \cellcolor{red!40}\href{../works/OuelletQ18.pdf}{OuelletQ18} (0.56)& \cellcolor{red!40}\href{../works/SchuttW10.pdf}{SchuttW10} (0.63)& \cellcolor{red!40}\href{../works/FahimiOQ18.pdf}{FahimiOQ18} (0.65)& \cellcolor{red!40}\href{../works/LetortBC12.pdf}{LetortBC12} (0.66)\\
Euclid& \cellcolor{red!40}\href{../works/KameugneFSN11.pdf}{KameugneFSN11} (0.20)& \cellcolor{yellow!20}\href{../works/GingrasQ16.pdf}{GingrasQ16} (0.27)& \cellcolor{yellow!20}\href{../works/KameugneFND23.pdf}{KameugneFND23} (0.27)& \cellcolor{yellow!20}\href{../works/OuelletQ13.pdf}{OuelletQ13} (0.28)& \cellcolor{yellow!20}\href{../works/SchuttW10.pdf}{SchuttW10} (0.28)\\
Dot& \cellcolor{red!40}\href{../works/Schutt11.pdf}{Schutt11} (148.00)& \cellcolor{red!40}\href{../works/Fahimi16.pdf}{Fahimi16} (145.00)& \cellcolor{red!40}\href{../works/Lombardi10.pdf}{Lombardi10} (141.00)& \cellcolor{red!40}\href{../works/Baptiste02.pdf}{Baptiste02} (141.00)& \cellcolor{red!40}\href{../works/Kameugne14.pdf}{Kameugne14} (140.00)\\
Cosine& \cellcolor{red!40}\href{../works/KameugneFSN11.pdf}{KameugneFSN11} (0.90)& \cellcolor{red!40}\href{../works/KameugneFND23.pdf}{KameugneFND23} (0.84)& \cellcolor{red!40}\href{../works/GingrasQ16.pdf}{GingrasQ16} (0.82)& \cellcolor{red!40}\href{../works/OuelletQ13.pdf}{OuelletQ13} (0.81)& \cellcolor{red!40}\href{../works/SchuttW10.pdf}{SchuttW10} (0.80)\\
\index{KanetAG04}\href{../works/KanetAG04.pdf}{KanetAG04} R\&C\\
Euclid& \href{../works/JainG01.pdf}{JainG01} (0.38)& \href{../works/ZeballosH05.pdf}{ZeballosH05} (0.38)& \href{../works/BeckR03.pdf}{BeckR03} (0.39)& \href{../works/FoxS90.pdf}{FoxS90} (0.39)& \href{../works/BartakSR08.pdf}{BartakSR08} (0.39)\\
Dot& \cellcolor{red!40}\href{../works/ZarandiASC20.pdf}{ZarandiASC20} (183.00)& \cellcolor{red!40}\href{../works/Baptiste02.pdf}{Baptiste02} (178.00)& \cellcolor{red!40}\href{../works/Malapert11.pdf}{Malapert11} (177.00)& \cellcolor{red!40}\href{../works/Dejemeppe16.pdf}{Dejemeppe16} (174.00)& \cellcolor{red!40}\href{../works/Groleaz21.pdf}{Groleaz21} (170.00)\\
Cosine& \cellcolor{red!40}\href{../works/TerekhovDOB12.pdf}{TerekhovDOB12} (0.71)& \cellcolor{red!40}\href{../works/KelbelH11.pdf}{KelbelH11} (0.71)& \cellcolor{red!40}\href{../works/JainG01.pdf}{JainG01} (0.71)& \cellcolor{red!40}\href{../works/ZeballosH05.pdf}{ZeballosH05} (0.71)& \cellcolor{red!40}\href{../works/BeckR03.pdf}{BeckR03} (0.71)\\
\index{KelarevaTK13}\href{../works/KelarevaTK13.pdf}{KelarevaTK13} R\&C& \cellcolor{yellow!20}\href{../works/SzerediS16.pdf}{SzerediS16} (0.90)& \cellcolor{yellow!20}\href{../works/SchuttCSW12.pdf}{SchuttCSW12} (0.91)& \cellcolor{yellow!20}\href{../works/SchuttFS13.pdf}{SchuttFS13} (0.92)& \cellcolor{yellow!20}\href{../works/BofillEGPSV14.pdf}{BofillEGPSV14} (0.92)& \cellcolor{yellow!20}\href{../works/GuSS13.pdf}{GuSS13} (0.92)\\
Euclid& \cellcolor{black!20}\href{../works/FrankDT16.pdf}{FrankDT16} (0.37)& \href{../works/GilesH16.pdf}{GilesH16} (0.37)& \href{../works/NishikawaSTT18a.pdf}{NishikawaSTT18a} (0.38)& \href{../works/KucukY19.pdf}{KucukY19} (0.38)& \href{../works/BeniniBGM05a.pdf}{BeniniBGM05a} (0.38)\\
Dot& \cellcolor{red!40}\href{../works/Godet21a.pdf}{Godet21a} (103.00)& \cellcolor{red!40}\href{../works/LaborieRSV18.pdf}{LaborieRSV18} (101.00)& \cellcolor{red!40}\href{../works/ZarandiASC20.pdf}{ZarandiASC20} (99.00)& \cellcolor{red!40}\href{../works/Astrand21.pdf}{Astrand21} (99.00)& \cellcolor{red!40}\href{../works/Lunardi20.pdf}{Lunardi20} (94.00)\\
Cosine& \cellcolor{red!40}\href{../works/TranVNB17.pdf}{TranVNB17} (0.60)& \cellcolor{red!40}\href{../works/GilesH16.pdf}{GilesH16} (0.59)& \cellcolor{red!40}\href{../works/FrankDT16.pdf}{FrankDT16} (0.59)& \cellcolor{red!40}\href{../works/NishikawaSTT18a.pdf}{NishikawaSTT18a} (0.58)& \cellcolor{red!40}\href{../works/PraletLJ15.pdf}{PraletLJ15} (0.57)\\
\index{KelbelH11}\href{../works/KelbelH11.pdf}{KelbelH11} R\&C& \cellcolor{red!40}\href{../works/AbreuN22.pdf}{AbreuN22} (0.85)& \cellcolor{red!20}\href{../works/TanSD10.pdf}{TanSD10} (0.88)& \cellcolor{red!20}\href{../works/GrimesHM09.pdf}{GrimesHM09} (0.89)& \cellcolor{red!20}BaptisteLPN06 (0.90)& \cellcolor{yellow!20}\href{../works/BeckR03.pdf}{BeckR03} (0.90)\\
Euclid& \cellcolor{green!20}\href{../works/MonetteDH09.pdf}{MonetteDH09} (0.31)& \cellcolor{black!20}\href{../works/GodardLN05.pdf}{GodardLN05} (0.36)& \cellcolor{black!20}\href{../works/GrimesH11.pdf}{GrimesH11} (0.36)& \cellcolor{black!20}\href{../works/Hooker06.pdf}{Hooker06} (0.36)& \cellcolor{black!20}\href{../works/PengLC14.pdf}{PengLC14} (0.36)\\
Dot& \cellcolor{red!40}\href{../works/Dejemeppe16.pdf}{Dejemeppe16} (191.00)& \cellcolor{red!40}\href{../works/Baptiste02.pdf}{Baptiste02} (187.00)& \cellcolor{red!40}\href{../works/Groleaz21.pdf}{Groleaz21} (182.00)& \cellcolor{red!40}\href{../works/Lombardi10.pdf}{Lombardi10} (176.00)& \cellcolor{red!40}\href{../works/ZarandiASC20.pdf}{ZarandiASC20} (170.00)\\
Cosine& \cellcolor{red!40}\href{../works/MonetteDH09.pdf}{MonetteDH09} (0.81)& \cellcolor{red!40}\href{../works/GrimesH11.pdf}{GrimesH11} (0.77)& \cellcolor{red!40}\href{../works/GodardLN05.pdf}{GodardLN05} (0.74)& \cellcolor{red!40}\href{../works/BeckR03.pdf}{BeckR03} (0.74)& \cellcolor{red!40}\href{../works/Hooker06.pdf}{Hooker06} (0.73)\\
\index{KendallKRU10}\href{../works/KendallKRU10.pdf}{KendallKRU10} R\&C& \cellcolor{red!40}\href{../works/Ribeiro12.pdf}{Ribeiro12} (0.86)& \cellcolor{yellow!20}\href{../works/EastonNT02.pdf}{EastonNT02} (0.92)& \cellcolor{yellow!20}\href{../works/ElfJR03.pdf}{ElfJR03} (0.93)& \cellcolor{green!20}\href{../works/RasmussenT07.pdf}{RasmussenT07} (0.93)& \cellcolor{green!20}\href{../works/HenzMT04.pdf}{HenzMT04} (0.95)\\
Euclid& \cellcolor{yellow!20}\href{../works/Ribeiro12.pdf}{Ribeiro12} (0.27)& \cellcolor{blue!20}\href{../works/RasmussenT09.pdf}{RasmussenT09} (0.34)& \cellcolor{black!20}\href{../works/ZengM12.pdf}{ZengM12} (0.37)& \href{../works/RasmussenT07.pdf}{RasmussenT07} (0.37)& \href{../works/BulckG22.pdf}{BulckG22} (0.38)\\
Dot& \cellcolor{red!40}\href{../works/ZarandiASC20.pdf}{ZarandiASC20} (176.00)& \cellcolor{red!40}\href{../works/Lemos21.pdf}{Lemos21} (150.00)& \cellcolor{red!40}\href{../works/Astrand21.pdf}{Astrand21} (136.00)& \cellcolor{red!40}\href{../works/Ribeiro12.pdf}{Ribeiro12} (134.00)& \cellcolor{red!40}\href{../works/Lunardi20.pdf}{Lunardi20} (133.00)\\
Cosine& \cellcolor{red!40}\href{../works/Ribeiro12.pdf}{Ribeiro12} (0.86)& \cellcolor{red!40}\href{../works/RasmussenT09.pdf}{RasmussenT09} (0.77)& \cellcolor{red!40}\href{../works/ZengM12.pdf}{ZengM12} (0.72)& \cellcolor{red!40}\href{../works/RasmussenT07.pdf}{RasmussenT07} (0.71)& \cellcolor{red!40}\href{../works/BulckG22.pdf}{BulckG22} (0.70)\\
\index{KengY89}\href{../works/KengY89.pdf}{KengY89} R\&C\\
Euclid& \cellcolor{red!40}\href{../works/CrawfordB94.pdf}{CrawfordB94} (0.17)& \cellcolor{red!40}\href{../works/FoxAS82.pdf}{FoxAS82} (0.21)& \cellcolor{red!40}\href{../works/LauLN08.pdf}{LauLN08} (0.23)& \cellcolor{red!40}\href{../works/Prosser89.pdf}{Prosser89} (0.23)& \cellcolor{red!40}\href{../works/Caseau97.pdf}{Caseau97} (0.23)\\
Dot& \cellcolor{red!40}\href{../works/Dejemeppe16.pdf}{Dejemeppe16} (78.00)& \cellcolor{red!40}\href{../works/Groleaz21.pdf}{Groleaz21} (78.00)& \cellcolor{red!40}\href{../works/Malapert11.pdf}{Malapert11} (77.00)& \cellcolor{red!40}\href{../works/ZarandiASC20.pdf}{ZarandiASC20} (76.00)& \cellcolor{red!40}\href{../works/Siala15a.pdf}{Siala15a} (74.00)\\
Cosine& \cellcolor{red!40}\href{../works/CrawfordB94.pdf}{CrawfordB94} (0.84)& \cellcolor{red!40}\href{../works/FoxAS82.pdf}{FoxAS82} (0.77)& \cellcolor{red!40}\href{../works/Prosser89.pdf}{Prosser89} (0.76)& \cellcolor{red!40}\href{../works/BelhadjiI98.pdf}{BelhadjiI98} (0.73)& \cellcolor{red!40}\href{../works/Salido10.pdf}{Salido10} (0.73)\\
\index{KeriK07}\href{../works/KeriK07.pdf}{KeriK07} R\&C& \cellcolor{yellow!20}EsquirolLH2008 (0.90)& \cellcolor{yellow!20}\href{../works/SchuttCSW12.pdf}{SchuttCSW12} (0.91)& \cellcolor{yellow!20}\href{../works/BaptisteP97.pdf}{BaptisteP97} (0.91)& \cellcolor{yellow!20}\href{../works/KovacsV04.pdf}{KovacsV04} (0.92)& \cellcolor{yellow!20}\href{../works/LimtanyakulS12.pdf}{LimtanyakulS12} (0.93)\\
Euclid& \cellcolor{red!40}\href{../works/LombardiM13.pdf}{LombardiM13} (0.22)& \cellcolor{red!40}\href{../works/Muscettola02.pdf}{Muscettola02} (0.24)& \cellcolor{red!20}\href{../works/BofillCSV17a.pdf}{BofillCSV17a} (0.25)& \cellcolor{red!20}\href{../works/WallaceF00.pdf}{WallaceF00} (0.26)& \cellcolor{red!20}\href{../works/BonfiettiM12.pdf}{BonfiettiM12} (0.26)\\
Dot& \cellcolor{red!40}\href{../works/ZarandiASC20.pdf}{ZarandiASC20} (90.00)& \cellcolor{red!40}\href{../works/Baptiste02.pdf}{Baptiste02} (90.00)& \cellcolor{red!40}\href{../works/Dejemeppe16.pdf}{Dejemeppe16} (89.00)& \cellcolor{red!40}\href{../works/LaborieRSV18.pdf}{LaborieRSV18} (87.00)& \cellcolor{red!40}\href{../works/Groleaz21.pdf}{Groleaz21} (87.00)\\
Cosine& \cellcolor{red!40}\href{../works/LombardiM13.pdf}{LombardiM13} (0.76)& \cellcolor{red!40}\href{../works/KusterJF07.pdf}{KusterJF07} (0.74)& \cellcolor{red!40}\href{../works/BofillCSV17a.pdf}{BofillCSV17a} (0.73)& \cellcolor{red!40}\href{../works/Muscettola02.pdf}{Muscettola02} (0.72)& \cellcolor{red!40}\href{../works/BofillCSV17.pdf}{BofillCSV17} (0.71)\\
\index{KhayatLR06}\href{../works/KhayatLR06.pdf}{KhayatLR06} R\&C& \cellcolor{red!40}\href{../works/ZeballosQH10.pdf}{ZeballosQH10} (0.84)& \cellcolor{yellow!20}\href{../works/NovasH14.pdf}{NovasH14} (0.91)& \cellcolor{yellow!20}\href{../works/CorreaLR07.pdf}{CorreaLR07} (0.92)& \cellcolor{green!20}\href{../works/Zeballos10.pdf}{Zeballos10} (0.93)& \cellcolor{green!20}\href{../works/HarjunkoskiG02.pdf}{HarjunkoskiG02} (0.94)\\
Euclid& \cellcolor{red!20}\href{../works/ZhangYW21.pdf}{ZhangYW21} (0.26)& \cellcolor{red!20}\href{../works/HeipckeCCS00.pdf}{HeipckeCCS00} (0.26)& \cellcolor{red!20}\href{../works/KovacsV06.pdf}{KovacsV06} (0.26)& \cellcolor{red!20}\href{../works/BeckPS03.pdf}{BeckPS03} (0.26)& \cellcolor{red!20}\href{../works/NovasH14.pdf}{NovasH14} (0.26)\\
Dot& \cellcolor{red!40}\href{../works/ZarandiASC20.pdf}{ZarandiASC20} (134.00)& \cellcolor{red!40}\href{../works/Baptiste02.pdf}{Baptiste02} (134.00)& \cellcolor{red!40}\href{../works/Dejemeppe16.pdf}{Dejemeppe16} (132.00)& \cellcolor{red!40}\href{../works/Groleaz21.pdf}{Groleaz21} (129.00)& \cellcolor{red!40}\href{../works/Lunardi20.pdf}{Lunardi20} (128.00)\\
Cosine& \cellcolor{red!40}\href{../works/ZhangYW21.pdf}{ZhangYW21} (0.82)& \cellcolor{red!40}\href{../works/BeckPS03.pdf}{BeckPS03} (0.81)& \cellcolor{red!40}\href{../works/NovasH14.pdf}{NovasH14} (0.81)& \cellcolor{red!40}\href{../works/PengLC14.pdf}{PengLC14} (0.81)& \cellcolor{red!40}\href{../works/HeipckeCCS00.pdf}{HeipckeCCS00} (0.80)\\
\index{KhemmoudjPB06}\href{../works/KhemmoudjPB06.pdf}{KhemmoudjPB06} R\&C& \cellcolor{red!40}\href{../works/KamarainenS02.pdf}{KamarainenS02} (0.86)& \cellcolor{yellow!20}\href{../works/HentenryckM04.pdf}{HentenryckM04} (0.91)& \cellcolor{green!20}\href{../works/DejemeppeD14.pdf}{DejemeppeD14} (0.93)& \cellcolor{green!20}\href{../works/PerronSF04.pdf}{PerronSF04} (0.94)& \cellcolor{green!20}\href{../works/KletzanderM17.pdf}{KletzanderM17} (0.94)\\
Euclid& \cellcolor{red!40}\href{../works/WolfS05.pdf}{WolfS05} (0.23)& \cellcolor{red!20}\href{../works/LozanoCDS12.pdf}{LozanoCDS12} (0.24)& \cellcolor{red!20}\href{../works/BockmayrP06.pdf}{BockmayrP06} (0.25)& \cellcolor{yellow!20}\href{../works/KuchcinskiW03.pdf}{KuchcinskiW03} (0.26)& \cellcolor{yellow!20}\href{../works/BeldiceanuP07.pdf}{BeldiceanuP07} (0.26)\\
Dot& \cellcolor{red!40}\href{../works/Lemos21.pdf}{Lemos21} (76.00)& \cellcolor{red!40}\href{../works/Astrand21.pdf}{Astrand21} (73.00)& \cellcolor{red!40}\href{../works/Froger16.pdf}{Froger16} (72.00)& \cellcolor{red!40}\href{../works/Dejemeppe16.pdf}{Dejemeppe16} (69.00)& \cellcolor{red!40}\href{../works/Malapert11.pdf}{Malapert11} (68.00)\\
Cosine& \cellcolor{red!40}\href{../works/WolfS05.pdf}{WolfS05} (0.74)& \cellcolor{red!40}\href{../works/LozanoCDS12.pdf}{LozanoCDS12} (0.72)& \cellcolor{red!40}\href{../works/BockmayrP06.pdf}{BockmayrP06} (0.71)& \cellcolor{red!40}\href{../works/BartakS11.pdf}{BartakS11} (0.68)& \cellcolor{red!40}\href{../works/SerraNM12.pdf}{SerraNM12} (0.68)\\
\index{KimCMLLP23}\href{../works/KimCMLLP23.pdf}{KimCMLLP23} R\&C& \cellcolor{red!20}\href{../works/GaySS14.pdf}{GaySS14} (0.88)& \cellcolor{green!20}\href{../works/KoschB14.pdf}{KoschB14} (0.94)& \cellcolor{green!20}\href{../works/DannaP03.pdf}{DannaP03} (0.94)& \cellcolor{green!20}\href{../works/OujanaAYB22.pdf}{OujanaAYB22} (0.94)& \cellcolor{green!20}\href{../works/SacramentoSP20.pdf}{SacramentoSP20} (0.94)\\
Euclid& \cellcolor{blue!20}\href{../works/DannaP03.pdf}{DannaP03} (0.32)& \cellcolor{blue!20}\href{../works/PerronSF04.pdf}{PerronSF04} (0.32)& \cellcolor{black!20}\href{../works/CarchraeB09.pdf}{CarchraeB09} (0.34)& \cellcolor{black!20}\href{../works/MonetteDH09.pdf}{MonetteDH09} (0.34)& \cellcolor{black!20}\href{../works/Beck06.pdf}{Beck06} (0.35)\\
Dot& \cellcolor{red!40}\href{../works/Groleaz21.pdf}{Groleaz21} (134.00)& \cellcolor{red!40}\href{../works/ZarandiASC20.pdf}{ZarandiASC20} (123.00)& \cellcolor{red!40}\href{../works/Dejemeppe16.pdf}{Dejemeppe16} (120.00)& \cellcolor{red!40}\href{../works/Lunardi20.pdf}{Lunardi20} (119.00)& \cellcolor{red!40}\href{../works/MengZRZL20.pdf}{MengZRZL20} (119.00)\\
Cosine& \cellcolor{red!40}\href{../works/MonetteDH09.pdf}{MonetteDH09} (0.70)& \cellcolor{red!40}\href{../works/DannaP03.pdf}{DannaP03} (0.70)& \cellcolor{red!40}\href{../works/AbreuPNF23.pdf}{AbreuPNF23} (0.68)& \cellcolor{red!40}\href{../works/PerronSF04.pdf}{PerronSF04} (0.68)& \cellcolor{red!40}\href{../works/CarchraeB09.pdf}{CarchraeB09} (0.68)\\
\index{KinsellaS0OS16}\href{../works/KinsellaS0OS16.pdf}{KinsellaS0OS16} R\&C\\
Euclid& \cellcolor{red!40}\href{../works/CarchraeBF05.pdf}{CarchraeBF05} (0.20)& \cellcolor{red!40}\href{../works/AngelsmarkJ00.pdf}{AngelsmarkJ00} (0.20)& \cellcolor{red!40}\href{../works/Hunsberger08.pdf}{Hunsberger08} (0.20)& \cellcolor{red!40}\href{../works/Baptiste09.pdf}{Baptiste09} (0.21)& \cellcolor{red!40}\href{../works/LiuJ06.pdf}{LiuJ06} (0.22)\\
Dot& \cellcolor{red!40}\href{../works/Astrand21.pdf}{Astrand21} (47.00)& \cellcolor{red!40}\href{../works/HarjunkoskiMBC14.pdf}{HarjunkoskiMBC14} (46.00)& \cellcolor{red!40}\href{../works/Lombardi10.pdf}{Lombardi10} (44.00)& \cellcolor{red!40}\href{../works/Lemos21.pdf}{Lemos21} (43.00)& \cellcolor{red!40}\href{../works/Froger16.pdf}{Froger16} (43.00)\\
Cosine& \cellcolor{red!40}\href{../works/IfrimOS12.pdf}{IfrimOS12} (0.68)& \cellcolor{red!40}\href{../works/GrimesIOS14.pdf}{GrimesIOS14} (0.61)& \cellcolor{red!40}\href{../works/Hunsberger08.pdf}{Hunsberger08} (0.59)& \cellcolor{red!40}\href{../works/HurleyOS16.pdf}{HurleyOS16} (0.59)& \cellcolor{red!40}\href{../works/HoYCLLCLC18.pdf}{HoYCLLCLC18} (0.57)\\
\index{KizilayC20}KizilayC20 R\&C& \cellcolor{red!40}Pinarbasi21 (0.79)& \cellcolor{red!40}\href{../works/AbidinK20.pdf}{AbidinK20} (0.82)& \cellcolor{red!40}\href{../works/PinarbasiAY19.pdf}{PinarbasiAY19} (0.83)& \cellcolor{red!40}\href{../works/AlakaPY19.pdf}{AlakaPY19} (0.83)& \cellcolor{red!40}PinarbasiA20 (0.83)\\
Euclid\\
Dot\\
Cosine\\
\index{KlankeBYE21}\href{../works/KlankeBYE21.pdf}{KlankeBYE21} R\&C& \cellcolor{red!40}\href{../works/AwadMDMT22.pdf}{AwadMDMT22} (0.78)& \cellcolor{red!20}\href{../works/EscobetPQPRA19.pdf}{EscobetPQPRA19} (0.89)& \cellcolor{green!20}\href{../works/Simonis95.pdf}{Simonis95} (0.94)& \cellcolor{green!20}\href{../works/HermenierDL11.pdf}{HermenierDL11} (0.94)& \cellcolor{green!20}\href{../works/OddiPCC03.pdf}{OddiPCC03} (0.95)\\
Euclid& \cellcolor{green!20}\href{../works/Vilim09a.pdf}{Vilim09a} (0.29)& \cellcolor{green!20}\href{../works/BockmayrP06.pdf}{BockmayrP06} (0.30)& \cellcolor{green!20}\href{../works/BenderWS21.pdf}{BenderWS21} (0.30)& \cellcolor{green!20}\href{../works/BeniniBGM05a.pdf}{BeniniBGM05a} (0.30)& \cellcolor{green!20}\href{../works/FortinZDF05.pdf}{FortinZDF05} (0.30)\\
Dot& \cellcolor{red!40}\href{../works/Groleaz21.pdf}{Groleaz21} (121.00)& \cellcolor{red!40}\href{../works/Baptiste02.pdf}{Baptiste02} (115.00)& \cellcolor{red!40}\href{../works/Lombardi10.pdf}{Lombardi10} (114.00)& \cellcolor{red!40}\href{../works/Dejemeppe16.pdf}{Dejemeppe16} (112.00)& \cellcolor{red!40}\href{../works/Malapert11.pdf}{Malapert11} (110.00)\\
Cosine& \cellcolor{red!40}\href{../works/OzturkTHO15.pdf}{OzturkTHO15} (0.71)& \cellcolor{red!40}\href{../works/QuirogaZH05.pdf}{QuirogaZH05} (0.71)& \cellcolor{red!40}\href{../works/BoothTNB16.pdf}{BoothTNB16} (0.71)& \cellcolor{red!40}\href{../works/BenderWS21.pdf}{BenderWS21} (0.70)& \cellcolor{red!40}\href{../works/OzturkTHO12.pdf}{OzturkTHO12} (0.70)\\
\index{KletzanderM17}\href{../works/KletzanderM17.pdf}{KletzanderM17} R\&C& \cellcolor{red!40}\href{../works/GoldwaserS17.pdf}{GoldwaserS17} (0.74)& \cellcolor{yellow!20}\href{../works/WuBB09.pdf}{WuBB09} (0.93)& \cellcolor{green!20}\href{../works/KhemmoudjPB06.pdf}{KhemmoudjPB06} (0.94)& \cellcolor{green!20}\href{../works/AggounB93.pdf}{AggounB93} (0.95)& \cellcolor{green!20}\href{../works/KamarainenS02.pdf}{KamarainenS02} (0.95)\\
Euclid& \cellcolor{red!40}\href{../works/ZibranR11.pdf}{ZibranR11} (0.22)& \cellcolor{red!40}\href{../works/CarchraeBF05.pdf}{CarchraeBF05} (0.22)& \cellcolor{red!40}\href{../works/LimRX04.pdf}{LimRX04} (0.23)& \cellcolor{red!40}\href{../works/Baptiste09.pdf}{Baptiste09} (0.23)& \cellcolor{red!40}\href{../works/FrostD98.pdf}{FrostD98} (0.23)\\
Dot& \cellcolor{red!40}\href{../works/ZarandiASC20.pdf}{ZarandiASC20} (53.00)& \cellcolor{red!40}\href{../works/IsikYA23.pdf}{IsikYA23} (50.00)& \cellcolor{red!40}\href{../works/Lunardi20.pdf}{Lunardi20} (49.00)& \cellcolor{red!40}\href{../works/Froger16.pdf}{Froger16} (49.00)& \cellcolor{red!40}\href{../works/SacramentoSP20.pdf}{SacramentoSP20} (48.00)\\
Cosine& \cellcolor{red!40}\href{../works/GoldwaserS17.pdf}{GoldwaserS17} (0.68)& \cellcolor{red!40}\href{../works/LimRX04.pdf}{LimRX04} (0.64)& \cellcolor{red!40}\href{../works/ZibranR11.pdf}{ZibranR11} (0.62)& \cellcolor{red!40}\href{../works/KletzanderM20.pdf}{KletzanderM20} (0.58)& \cellcolor{red!40}\href{../works/FukunagaHFAMN02.pdf}{FukunagaHFAMN02} (0.58)\\
\index{KletzanderM20}\href{../works/KletzanderM20.pdf}{KletzanderM20} R\&C\\
Euclid& \cellcolor{red!40}\href{../works/KletzanderMH21.pdf}{KletzanderMH21} (0.21)& \cellcolor{green!20}\href{../works/KletzanderM17.pdf}{KletzanderM17} (0.29)& \cellcolor{green!20}\href{../works/LimRX04.pdf}{LimRX04} (0.31)& \cellcolor{green!20}\href{../works/Musliu05.pdf}{Musliu05} (0.31)& \cellcolor{blue!20}\href{../works/LimAHO02a.pdf}{LimAHO02a} (0.32)\\
Dot& \cellcolor{red!40}\href{../works/Froger16.pdf}{Froger16} (83.00)& \cellcolor{red!40}\href{../works/Lemos21.pdf}{Lemos21} (82.00)& \cellcolor{red!40}\href{../works/KendallKRU10.pdf}{KendallKRU10} (79.00)& \cellcolor{red!40}\href{../works/ZarandiASC20.pdf}{ZarandiASC20} (78.00)& \cellcolor{red!40}\href{../works/IsikYA23.pdf}{IsikYA23} (76.00)\\
Cosine& \cellcolor{red!40}\href{../works/KletzanderMH21.pdf}{KletzanderMH21} (0.82)& \cellcolor{red!40}\href{../works/Ribeiro12.pdf}{Ribeiro12} (0.63)& \cellcolor{red!40}\href{../works/KendallKRU10.pdf}{KendallKRU10} (0.62)& \cellcolor{red!40}\href{../works/KletzanderM17.pdf}{KletzanderM17} (0.58)& \cellcolor{red!40}\href{../works/PourDERB18.pdf}{PourDERB18} (0.56)\\
\index{KletzanderMH21}\href{../works/KletzanderMH21.pdf}{KletzanderMH21} R\&C\\
Euclid& \cellcolor{red!40}\href{../works/KletzanderM20.pdf}{KletzanderM20} (0.21)& \cellcolor{yellow!20}\href{../works/FeldmanG89.pdf}{FeldmanG89} (0.28)& \cellcolor{green!20}\href{../works/CestaOS98.pdf}{CestaOS98} (0.30)& \cellcolor{green!20}\href{../works/GelainPRVW17.pdf}{GelainPRVW17} (0.30)& \cellcolor{green!20}\href{../works/WallaceF00.pdf}{WallaceF00} (0.30)\\
Dot& \cellcolor{red!40}\href{../works/Lemos21.pdf}{Lemos21} (88.00)& \cellcolor{red!40}\href{../works/ZarandiASC20.pdf}{ZarandiASC20} (83.00)& \cellcolor{red!40}\href{../works/KendallKRU10.pdf}{KendallKRU10} (77.00)& \cellcolor{red!40}\href{../works/Froger16.pdf}{Froger16} (75.00)& \cellcolor{red!40}\href{../works/Groleaz21.pdf}{Groleaz21} (71.00)\\
Cosine& \cellcolor{red!40}\href{../works/KletzanderM20.pdf}{KletzanderM20} (0.82)& \cellcolor{red!40}\href{../works/FallahiAC20.pdf}{FallahiAC20} (0.66)& \cellcolor{red!40}\href{../works/BourreauGGLT22.pdf}{BourreauGGLT22} (0.62)& \cellcolor{red!40}\href{../works/KendallKRU10.pdf}{KendallKRU10} (0.62)& \cellcolor{red!40}\href{../works/LiuLH19.pdf}{LiuLH19} (0.60)\\
\index{KoehlerBFFHPSSS21}\href{../works/KoehlerBFFHPSSS21.pdf}{KoehlerBFFHPSSS21} R\&C& \cellcolor{green!20}\href{../works/ColT19.pdf}{ColT19} (0.95)& \cellcolor{blue!20}\href{../works/MelgarejoLS15.pdf}{MelgarejoLS15} (0.97)& \cellcolor{blue!20}\href{../works/TanT18.pdf}{TanT18} (0.97)& \cellcolor{black!20}\href{../works/WuBB09.pdf}{WuBB09} (0.98)& \cellcolor{black!20}\href{../works/FrohnerTR19.pdf}{FrohnerTR19} (0.98)\\
Euclid& \href{../works/MelgarejoLS15.pdf}{MelgarejoLS15} (0.51)& \href{../works/KuB16.pdf}{KuB16} (0.52)& \href{../works/WallaceY20.pdf}{WallaceY20} (0.52)& \href{../works/BukchinR18.pdf}{BukchinR18} (0.52)& \href{../works/abs-1901-07914.pdf}{abs-1901-07914} (0.53)\\
Dot& \cellcolor{red!40}\href{../works/Groleaz21.pdf}{Groleaz21} (182.00)& \cellcolor{red!40}\href{../works/Dejemeppe16.pdf}{Dejemeppe16} (179.00)& \cellcolor{red!40}\href{../works/Godet21a.pdf}{Godet21a} (173.00)& \cellcolor{red!40}\href{../works/Malapert11.pdf}{Malapert11} (172.00)& \cellcolor{red!40}\href{../works/ColT22.pdf}{ColT22} (166.00)\\
Cosine& \cellcolor{red!40}\href{../works/MullerMKP22.pdf}{MullerMKP22} (0.60)& \cellcolor{red!40}\href{../works/ColT22.pdf}{ColT22} (0.60)& \cellcolor{red!40}\href{../works/WallaceY20.pdf}{WallaceY20} (0.59)& \cellcolor{red!40}\href{../works/MelgarejoLS15.pdf}{MelgarejoLS15} (0.59)& \cellcolor{red!40}\href{../works/KovacsTKSG21.pdf}{KovacsTKSG21} (0.57)\\
\index{KonowalenkoMM19}\href{../works/KonowalenkoMM19.pdf}{KonowalenkoMM19} R\&C\\
Euclid& \cellcolor{red!20}\href{../works/BarlattCG08.pdf}{BarlattCG08} (0.26)& \cellcolor{yellow!20}\href{../works/BeniniBGM05a.pdf}{BeniniBGM05a} (0.27)& \cellcolor{yellow!20}\href{../works/Baptiste09.pdf}{Baptiste09} (0.28)& \cellcolor{yellow!20}\href{../works/AngelsmarkJ00.pdf}{AngelsmarkJ00} (0.28)& \cellcolor{yellow!20}\href{../works/CarchraeBF05.pdf}{CarchraeBF05} (0.28)\\
Dot& \cellcolor{red!40}\href{../works/ZarandiASC20.pdf}{ZarandiASC20} (62.00)& \cellcolor{red!40}\href{../works/HarjunkoskiMBC14.pdf}{HarjunkoskiMBC14} (61.00)& \cellcolor{red!40}\href{../works/Groleaz21.pdf}{Groleaz21} (58.00)& \cellcolor{red!40}\href{../works/ColT22.pdf}{ColT22} (57.00)& \cellcolor{red!40}\href{../works/Simonis07.pdf}{Simonis07} (56.00)\\
Cosine& \cellcolor{red!40}\href{../works/BeniniBGM05a.pdf}{BeniniBGM05a} (0.60)& \cellcolor{red!40}\href{../works/BarlattCG08.pdf}{BarlattCG08} (0.60)& \cellcolor{red!40}\href{../works/RodosekWH99.pdf}{RodosekWH99} (0.57)& \cellcolor{red!40}\href{../works/Darby-DowmanLMZ97.pdf}{Darby-DowmanLMZ97} (0.56)& \cellcolor{red!40}\href{../works/RenT09.pdf}{RenT09} (0.53)\\
\index{KorbaaYG00}\href{../works/KorbaaYG00.pdf}{KorbaaYG00} R\&C& \cellcolor{yellow!20}\href{../works/OzturkTHO12.pdf}{OzturkTHO12} (0.92)& \cellcolor{green!20}\href{../works/NovasH12.pdf}{NovasH12} (0.96)& \cellcolor{blue!20}\href{../works/LorigeonBB02.pdf}{LorigeonBB02} (0.97)& \cellcolor{blue!20}\href{../works/GokgurHO18.pdf}{GokgurHO18} (0.97)& \cellcolor{blue!20}\href{../works/HachemiGR11.pdf}{HachemiGR11} (0.97)\\
Euclid\\
Dot\\
Cosine\\
\index{KorbaaYG99}\href{../works/KorbaaYG99.pdf}{KorbaaYG99} R\&C\\
Euclid& \cellcolor{yellow!20}\href{../works/AstrandJZ18.pdf}{AstrandJZ18} (0.28)& \cellcolor{yellow!20}\href{../works/BocewiczBB09.pdf}{BocewiczBB09} (0.28)& \cellcolor{green!20}\href{../works/KengY89.pdf}{KengY89} (0.29)& \cellcolor{green!20}\href{../works/ErtlK91.pdf}{ErtlK91} (0.29)& \cellcolor{green!20}\href{../works/GetoorOFC97.pdf}{GetoorOFC97} (0.29)\\
Dot& \cellcolor{red!40}\href{../works/Groleaz21.pdf}{Groleaz21} (95.00)& \cellcolor{red!40}\href{../works/Astrand21.pdf}{Astrand21} (93.00)& \cellcolor{red!40}\href{../works/ZarandiASC20.pdf}{ZarandiASC20} (92.00)& \cellcolor{red!40}\href{../works/Lombardi10.pdf}{Lombardi10} (91.00)& \cellcolor{red!40}\href{../works/Malapert11.pdf}{Malapert11} (90.00)\\
Cosine& \cellcolor{red!40}\href{../works/BonfiettiLBM14.pdf}{BonfiettiLBM14} (0.73)& \cellcolor{red!40}\href{../works/AstrandJZ18.pdf}{AstrandJZ18} (0.70)& \cellcolor{red!40}\href{../works/BocewiczBB09.pdf}{BocewiczBB09} (0.70)& \cellcolor{red!40}\href{../works/GetoorOFC97.pdf}{GetoorOFC97} (0.69)& \cellcolor{red!40}\href{../works/BonfiettiLBM12.pdf}{BonfiettiLBM12} (0.68)\\
\index{KoschB14}\href{../works/KoschB14.pdf}{KoschB14} R\&C& \cellcolor{red!40}\href{../works/MalapertGR12.pdf}{MalapertGR12} (0.82)& \cellcolor{red!40}\href{../works/TangB20.pdf}{TangB20} (0.83)& \cellcolor{red!20}\href{../works/LetortCB13.pdf}{LetortCB13} (0.89)& \cellcolor{red!20}\href{../works/KameugneFSN14.pdf}{KameugneFSN14} (0.89)& \cellcolor{red!20}\href{../works/Beck10.pdf}{Beck10} (0.90)\\
Euclid& \cellcolor{red!20}\href{../works/MalapertGR12.pdf}{MalapertGR12} (0.26)& \cellcolor{blue!20}\href{../works/HamFC17.pdf}{HamFC17} (0.32)& \cellcolor{blue!20}\href{../works/Beck10.pdf}{Beck10} (0.32)& \cellcolor{blue!20}\href{../works/ChuX05.pdf}{ChuX05} (0.33)& \cellcolor{blue!20}\href{../works/WatsonB08.pdf}{WatsonB08} (0.34)\\
Dot& \cellcolor{red!40}\href{../works/Groleaz21.pdf}{Groleaz21} (148.00)& \cellcolor{red!40}\href{../works/Malapert11.pdf}{Malapert11} (147.00)& \cellcolor{red!40}\href{../works/ZarandiASC20.pdf}{ZarandiASC20} (137.00)& \cellcolor{red!40}\href{../works/Baptiste02.pdf}{Baptiste02} (131.00)& \cellcolor{red!40}\href{../works/AwadMDMT22.pdf}{AwadMDMT22} (130.00)\\
Cosine& \cellcolor{red!40}\href{../works/MalapertGR12.pdf}{MalapertGR12} (0.85)& \cellcolor{red!40}\href{../works/HamFC17.pdf}{HamFC17} (0.75)& \cellcolor{red!40}\href{../works/Beck10.pdf}{Beck10} (0.72)& \cellcolor{red!40}\href{../works/ChuX05.pdf}{ChuX05} (0.71)& \cellcolor{red!40}\href{../works/Ham18a.pdf}{Ham18a} (0.70)\\
\index{KotaryFH22}\href{../works/KotaryFH22.pdf}{KotaryFH22} R\&C\\
Euclid& \cellcolor{red!20}\href{../works/IklassovMR023.pdf}{IklassovMR023} (0.26)& \cellcolor{green!20}\href{../works/KovacsV06.pdf}{KovacsV06} (0.30)& \cellcolor{green!20}\href{../works/HeipckeCCS00.pdf}{HeipckeCCS00} (0.30)& \cellcolor{green!20}\href{../works/CarchraeB09.pdf}{CarchraeB09} (0.30)& \cellcolor{green!20}\href{../works/FontaineMH16.pdf}{FontaineMH16} (0.31)\\
Dot& \cellcolor{red!40}\href{../works/ZarandiASC20.pdf}{ZarandiASC20} (121.00)& \cellcolor{red!40}\href{../works/Groleaz21.pdf}{Groleaz21} (120.00)& \cellcolor{red!40}\href{../works/JainM99.pdf}{JainM99} (116.00)& \cellcolor{red!40}\href{../works/Lombardi10.pdf}{Lombardi10} (114.00)& \cellcolor{red!40}\href{../works/Astrand21.pdf}{Astrand21} (112.00)\\
Cosine& \cellcolor{red!40}\href{../works/IklassovMR023.pdf}{IklassovMR023} (0.81)& \cellcolor{red!40}\href{../works/CarchraeB09.pdf}{CarchraeB09} (0.75)& \cellcolor{red!40}\href{../works/HeipckeCCS00.pdf}{HeipckeCCS00} (0.74)& \cellcolor{red!40}\href{../works/KovacsV06.pdf}{KovacsV06} (0.74)& \cellcolor{red!40}\href{../works/BeckPS03.pdf}{BeckPS03} (0.73)\\
\index{KovacsB07}\href{../works/KovacsB07.pdf}{KovacsB07} R\&C& \cellcolor{red!40}\href{../works/KovacsB11.pdf}{KovacsB11} (0.68)& \cellcolor{red!20}\href{../works/KovacsB08.pdf}{KovacsB08} (0.88)& \cellcolor{green!20}\href{../works/KovacsV06.pdf}{KovacsV06} (0.95)& \cellcolor{green!20}\href{../works/KovacsV04.pdf}{KovacsV04} (0.96)& \cellcolor{blue!20}\href{../works/BeldiceanuC01.pdf}{BeldiceanuC01} (0.97)\\
Euclid& \cellcolor{red!40}\href{../works/KovacsB11.pdf}{KovacsB11} (0.20)& \cellcolor{red!20}\href{../works/KovacsB08.pdf}{KovacsB08} (0.26)& \cellcolor{black!20}\href{../works/HanenKP21.pdf}{HanenKP21} (0.34)& \cellcolor{black!20}\href{../works/Vilim09.pdf}{Vilim09} (0.35)& \cellcolor{black!20}\href{../works/Wolf05.pdf}{Wolf05} (0.35)\\
Dot& \cellcolor{red!40}\href{../works/KovacsB11.pdf}{KovacsB11} (146.00)& \cellcolor{red!40}\href{../works/Baptiste02.pdf}{Baptiste02} (146.00)& \cellcolor{red!40}\href{../works/Groleaz21.pdf}{Groleaz21} (139.00)& \cellcolor{red!40}\href{../works/ZarandiASC20.pdf}{ZarandiASC20} (134.00)& \cellcolor{red!40}\href{../works/Dejemeppe16.pdf}{Dejemeppe16} (132.00)\\
Cosine& \cellcolor{red!40}\href{../works/KovacsB11.pdf}{KovacsB11} (0.92)& \cellcolor{red!40}\href{../works/KovacsB08.pdf}{KovacsB08} (0.82)& \cellcolor{red!40}\href{../works/BonninMNE24.pdf}{BonninMNE24} (0.73)& \cellcolor{red!40}\href{../works/ArtiguesLH13.pdf}{ArtiguesLH13} (0.71)& \cellcolor{red!40}\href{../works/TerekhovDOB12.pdf}{TerekhovDOB12} (0.71)\\
\index{KovacsB08}\href{../works/KovacsB08.pdf}{KovacsB08} R\&C& \cellcolor{red!40}\href{../works/KovacsB11.pdf}{KovacsB11} (0.80)& \cellcolor{red!20}\href{../works/KovacsB07.pdf}{KovacsB07} (0.88)& \cellcolor{yellow!20}LiuGT10 (0.93)& \cellcolor{green!20}\href{../works/TrojetHL11.pdf}{TrojetHL11} (0.94)& \cellcolor{green!20}\href{../works/Muscettola02.pdf}{Muscettola02} (0.94)\\
Euclid& \cellcolor{red!20}\href{../works/KovacsB07.pdf}{KovacsB07} (0.26)& \cellcolor{yellow!20}\href{../works/Vilim09a.pdf}{Vilim09a} (0.27)& \cellcolor{green!20}\href{../works/Vilim09.pdf}{Vilim09} (0.29)& \cellcolor{green!20}\href{../works/WolfS05.pdf}{WolfS05} (0.30)& \cellcolor{green!20}\href{../works/KovacsB11.pdf}{KovacsB11} (0.30)\\
Dot& \cellcolor{red!40}\href{../works/Dejemeppe16.pdf}{Dejemeppe16} (123.00)& \cellcolor{red!40}\href{../works/Lombardi10.pdf}{Lombardi10} (120.00)& \cellcolor{red!40}\href{../works/KovacsB11.pdf}{KovacsB11} (118.00)& \cellcolor{red!40}\href{../works/Malapert11.pdf}{Malapert11} (118.00)& \cellcolor{red!40}\href{../works/Baptiste02.pdf}{Baptiste02} (118.00)\\
Cosine& \cellcolor{red!40}\href{../works/KovacsB11.pdf}{KovacsB11} (0.82)& \cellcolor{red!40}\href{../works/KovacsB07.pdf}{KovacsB07} (0.82)& \cellcolor{red!40}\href{../works/Vilim09a.pdf}{Vilim09a} (0.75)& \cellcolor{red!40}\href{../works/ClautiauxJCM08.pdf}{ClautiauxJCM08} (0.73)& \cellcolor{red!40}\href{../works/Vilim09.pdf}{Vilim09} (0.72)\\
\index{KovacsB11}\href{../works/KovacsB11.pdf}{KovacsB11} R\&C& \cellcolor{red!40}\href{../works/KovacsB07.pdf}{KovacsB07} (0.68)& \cellcolor{red!40}\href{../works/KovacsB08.pdf}{KovacsB08} (0.80)& \cellcolor{yellow!20}\href{../works/MonetteDH09.pdf}{MonetteDH09} (0.92)& \cellcolor{green!20}WariZ19 (0.93)& \cellcolor{green!20}\href{../works/SchausHMCMD11.pdf}{SchausHMCMD11} (0.94)\\
Euclid& \cellcolor{red!40}\href{../works/KovacsB07.pdf}{KovacsB07} (0.20)& \cellcolor{green!20}\href{../works/KovacsB08.pdf}{KovacsB08} (0.30)& \cellcolor{black!20}\href{../works/BonninMNE24.pdf}{BonninMNE24} (0.35)& \cellcolor{black!20}\href{../works/MonetteDH09.pdf}{MonetteDH09} (0.36)& \cellcolor{black!20}\href{../works/TerekhovDOB12.pdf}{TerekhovDOB12} (0.37)\\
Dot& \cellcolor{red!40}\href{../works/Baptiste02.pdf}{Baptiste02} (190.00)& \cellcolor{red!40}\href{../works/Groleaz21.pdf}{Groleaz21} (182.00)& \cellcolor{red!40}\href{../works/ZarandiASC20.pdf}{ZarandiASC20} (181.00)& \cellcolor{red!40}\href{../works/Dejemeppe16.pdf}{Dejemeppe16} (173.00)& \cellcolor{red!40}\href{../works/Lombardi10.pdf}{Lombardi10} (160.00)\\
Cosine& \cellcolor{red!40}\href{../works/KovacsB07.pdf}{KovacsB07} (0.92)& \cellcolor{red!40}\href{../works/KovacsB08.pdf}{KovacsB08} (0.82)& \cellcolor{red!40}\href{../works/BonninMNE24.pdf}{BonninMNE24} (0.77)& \cellcolor{red!40}\href{../works/TerekhovDOB12.pdf}{TerekhovDOB12} (0.76)& \cellcolor{red!40}\href{../works/GokgurHO18.pdf}{GokgurHO18} (0.74)\\
\index{KovacsEKV05}\href{../works/KovacsEKV05.pdf}{KovacsEKV05} R\&C& \cellcolor{red!40}\href{../works/KovacsV06.pdf}{KovacsV06} (0.80)& \cellcolor{green!20}\href{../works/KovacsV04.pdf}{KovacsV04} (0.93)& \cellcolor{green!20}\href{../works/NattafHKAL19.pdf}{NattafHKAL19} (0.93)& \cellcolor{green!20}\href{../works/NattafAL15.pdf}{NattafAL15} (0.94)& \cellcolor{green!20}\href{../works/NattafALR16.pdf}{NattafALR16} (0.94)\\
Euclid& \cellcolor{red!40}\href{../works/CestaOS98.pdf}{CestaOS98} (0.12)& \cellcolor{red!40}\href{../works/Caballero23.pdf}{Caballero23} (0.13)& \cellcolor{red!40}\href{../works/AngelsmarkJ00.pdf}{AngelsmarkJ00} (0.13)& \cellcolor{red!40}\href{../works/Baptiste09.pdf}{Baptiste09} (0.14)& \cellcolor{red!40}\href{../works/HebrardTW05.pdf}{HebrardTW05} (0.14)\\
Dot& \cellcolor{red!40}\href{../works/Astrand21.pdf}{Astrand21} (38.00)& \cellcolor{red!40}\href{../works/Baptiste02.pdf}{Baptiste02} (38.00)& \cellcolor{red!40}\href{../works/ZarandiASC20.pdf}{ZarandiASC20} (37.00)& \cellcolor{red!40}\href{../works/Groleaz21.pdf}{Groleaz21} (36.00)& \cellcolor{red!40}\href{../works/Lunardi20.pdf}{Lunardi20} (35.00)\\
Cosine& \cellcolor{red!40}\href{../works/Caballero23.pdf}{Caballero23} (0.74)& \cellcolor{red!40}\href{../works/CestaOS98.pdf}{CestaOS98} (0.74)& \cellcolor{red!40}\href{../works/BonfiettiM12.pdf}{BonfiettiM12} (0.72)& \cellcolor{red!40}\href{../works/AngelsmarkJ00.pdf}{AngelsmarkJ00} (0.70)& \cellcolor{red!40}\href{../works/Tsang03.pdf}{Tsang03} (0.68)\\
\index{KovacsK11}\href{../works/KovacsK11.pdf}{KovacsK11} R\&C& \cellcolor{green!20}\href{../works/Sadykov04.pdf}{Sadykov04} (0.94)& \cellcolor{green!20}\href{../works/Hooker05.pdf}{Hooker05} (0.94)& \cellcolor{green!20}\href{../works/Beck10.pdf}{Beck10} (0.94)& \cellcolor{green!20}OddiPCC05 (0.94)& \cellcolor{green!20}\href{../works/ChuX05.pdf}{ChuX05} (0.95)\\
Euclid& \cellcolor{green!20}\href{../works/PengLC14.pdf}{PengLC14} (0.30)& \cellcolor{blue!20}\href{../works/BockmayrP06.pdf}{BockmayrP06} (0.33)& \cellcolor{blue!20}\href{../works/JuvinHL23.pdf}{JuvinHL23} (0.34)& \cellcolor{black!20}\href{../works/BocewiczBB09.pdf}{BocewiczBB09} (0.34)& \cellcolor{black!20}\href{../works/ValleMGT03.pdf}{ValleMGT03} (0.35)\\
Dot& \cellcolor{red!40}\href{../works/Groleaz21.pdf}{Groleaz21} (139.00)& \cellcolor{red!40}\href{../works/Dejemeppe16.pdf}{Dejemeppe16} (138.00)& \cellcolor{red!40}\href{../works/ZarandiASC20.pdf}{ZarandiASC20} (134.00)& \cellcolor{red!40}\href{../works/Malapert11.pdf}{Malapert11} (119.00)& \cellcolor{red!40}\href{../works/Baptiste02.pdf}{Baptiste02} (119.00)\\
Cosine& \cellcolor{red!40}\href{../works/PengLC14.pdf}{PengLC14} (0.77)& \cellcolor{red!40}\href{../works/PinarbasiAY19.pdf}{PinarbasiAY19} (0.68)& \cellcolor{red!40}\href{../works/BartakSR08.pdf}{BartakSR08} (0.67)& \cellcolor{red!40}\href{../works/HamdiL13.pdf}{HamdiL13} (0.67)& \cellcolor{red!40}\href{../works/BockmayrP06.pdf}{BockmayrP06} (0.67)\\
\index{KovacsTKSG21}\href{../works/KovacsTKSG21.pdf}{KovacsTKSG21} R\&C\\
Euclid& \cellcolor{blue!20}\href{../works/HeipckeCCS00.pdf}{HeipckeCCS00} (0.34)& \cellcolor{black!20}\href{../works/Tassel22.pdf}{Tassel22} (0.35)& \cellcolor{black!20}\href{../works/KovacsV06.pdf}{KovacsV06} (0.35)& \cellcolor{black!20}\href{../works/Caseau97.pdf}{Caseau97} (0.36)& \cellcolor{black!20}\href{../works/CarchraeB09.pdf}{CarchraeB09} (0.36)\\
Dot& \cellcolor{red!40}\href{../works/Groleaz21.pdf}{Groleaz21} (170.00)& \cellcolor{red!40}\href{../works/Dejemeppe16.pdf}{Dejemeppe16} (145.00)& \cellcolor{red!40}\href{../works/ZarandiASC20.pdf}{ZarandiASC20} (144.00)& \cellcolor{red!40}\href{../works/Baptiste02.pdf}{Baptiste02} (135.00)& \cellcolor{red!40}\href{../works/Lunardi20.pdf}{Lunardi20} (133.00)\\
Cosine& \cellcolor{red!40}\href{../works/abs-2211-14492.pdf}{abs-2211-14492} (0.73)& \cellcolor{red!40}\href{../works/HeipckeCCS00.pdf}{HeipckeCCS00} (0.73)& \cellcolor{red!40}\href{../works/abs-2402-00459.pdf}{abs-2402-00459} (0.71)& \cellcolor{red!40}\href{../works/Tassel22.pdf}{Tassel22} (0.70)& \cellcolor{red!40}\href{../works/CarchraeB09.pdf}{CarchraeB09} (0.70)\\
\index{KovacsV04}\href{../works/KovacsV04.pdf}{KovacsV04} R\&C& \cellcolor{red!40}\href{../works/ZhangLS12.pdf}{ZhangLS12} (0.75)& \cellcolor{red!40}\href{../works/QuirogaZH05.pdf}{QuirogaZH05} (0.80)& \cellcolor{red!40}\href{../works/Geske05.pdf}{Geske05} (0.80)& \cellcolor{red!40}\href{../works/LimtanyakulS12.pdf}{LimtanyakulS12} (0.82)& \cellcolor{red!40}\href{../works/EvenSH15.pdf}{EvenSH15} (0.83)\\
Euclid& \cellcolor{red!40}\href{../works/Bartak02a.pdf}{Bartak02a} (0.20)& \cellcolor{red!40}\href{../works/KovacsV06.pdf}{KovacsV06} (0.23)& \cellcolor{red!40}\href{../works/ChuGNSW13.pdf}{ChuGNSW13} (0.23)& \cellcolor{red!40}\href{../works/HeipckeCCS00.pdf}{HeipckeCCS00} (0.24)& \cellcolor{red!40}\href{../works/Caseau97.pdf}{Caseau97} (0.24)\\
Dot& \cellcolor{red!40}\href{../works/Baptiste02.pdf}{Baptiste02} (123.00)& \cellcolor{red!40}\href{../works/Godet21a.pdf}{Godet21a} (118.00)& \cellcolor{red!40}\href{../works/Lombardi10.pdf}{Lombardi10} (117.00)& \cellcolor{red!40}\href{../works/Dejemeppe16.pdf}{Dejemeppe16} (116.00)& \cellcolor{red!40}\href{../works/Beck99.pdf}{Beck99} (116.00)\\
Cosine& \cellcolor{red!40}\href{../works/Bartak02a.pdf}{Bartak02a} (0.86)& \cellcolor{red!40}\href{../works/KovacsV06.pdf}{KovacsV06} (0.82)& \cellcolor{red!40}\href{../works/HeipckeCCS00.pdf}{HeipckeCCS00} (0.81)& \cellcolor{red!40}\href{../works/BartakSR08.pdf}{BartakSR08} (0.80)& \cellcolor{red!40}\href{../works/ChuGNSW13.pdf}{ChuGNSW13} (0.80)\\
\index{KovacsV06}\href{../works/KovacsV06.pdf}{KovacsV06} R\&C& \cellcolor{red!40}\href{../works/Vilim03.pdf}{Vilim03} (0.67)& \cellcolor{red!40}\href{../works/SchuttS16.pdf}{SchuttS16} (0.80)& \cellcolor{red!40}\href{../works/KovacsEKV05.pdf}{KovacsEKV05} (0.80)& \cellcolor{red!20}\href{../works/KovacsV04.pdf}{KovacsV04} (0.89)& \cellcolor{green!20}\href{../works/KreterSS17.pdf}{KreterSS17} (0.94)\\
Euclid& \cellcolor{red!40}\href{../works/HeipckeCCS00.pdf}{HeipckeCCS00} (0.21)& \cellcolor{red!40}\href{../works/KovacsV04.pdf}{KovacsV04} (0.23)& \cellcolor{red!40}\href{../works/Caseau97.pdf}{Caseau97} (0.24)& \cellcolor{red!20}\href{../works/BeckPS03.pdf}{BeckPS03} (0.25)& \cellcolor{red!20}\href{../works/CarchraeB09.pdf}{CarchraeB09} (0.26)\\
Dot& \cellcolor{red!40}\href{../works/Groleaz21.pdf}{Groleaz21} (116.00)& \cellcolor{red!40}\href{../works/Dejemeppe16.pdf}{Dejemeppe16} (115.00)& \cellcolor{red!40}\href{../works/Baptiste02.pdf}{Baptiste02} (114.00)& \cellcolor{red!40}\href{../works/LaborieRSV18.pdf}{LaborieRSV18} (112.00)& \cellcolor{red!40}\href{../works/Lombardi10.pdf}{Lombardi10} (107.00)\\
Cosine& \cellcolor{red!40}\href{../works/HeipckeCCS00.pdf}{HeipckeCCS00} (0.85)& \cellcolor{red!40}\href{../works/KovacsV04.pdf}{KovacsV04} (0.82)& \cellcolor{red!40}\href{../works/BeckPS03.pdf}{BeckPS03} (0.82)& \cellcolor{red!40}\href{../works/CarchraeB09.pdf}{CarchraeB09} (0.80)& \cellcolor{red!40}\href{../works/KhayatLR06.pdf}{KhayatLR06} (0.79)\\
\index{KreterSS15}\href{../works/KreterSS15.pdf}{KreterSS15} R\&C& \cellcolor{red!40}\href{../works/KreterSS17.pdf}{KreterSS17} (0.56)& \cellcolor{red!40}\href{../works/SzerediS16.pdf}{SzerediS16} (0.68)& \cellcolor{red!40}\href{../works/SchuttFS13.pdf}{SchuttFS13} (0.68)& \cellcolor{red!40}\href{../works/SchuttCSW12.pdf}{SchuttCSW12} (0.70)& \cellcolor{red!40}SchuttFSW15 (0.76)\\
Euclid& \cellcolor{red!20}\href{../works/KreterSS17.pdf}{KreterSS17} (0.26)& \cellcolor{yellow!20}\href{../works/KreterSSZ18.pdf}{KreterSSZ18} (0.28)& \cellcolor{green!20}\href{../works/SzerediS16.pdf}{SzerediS16} (0.29)& \cellcolor{green!20}\href{../works/YoungFS17.pdf}{YoungFS17} (0.30)& \cellcolor{green!20}\href{../works/SchuttS16.pdf}{SchuttS16} (0.31)\\
Dot& \cellcolor{red!40}\href{../works/KreterSS17.pdf}{KreterSS17} (138.00)& \cellcolor{red!40}\href{../works/Godet21a.pdf}{Godet21a} (134.00)& \cellcolor{red!40}\href{../works/Schutt11.pdf}{Schutt11} (123.00)& \cellcolor{red!40}\href{../works/KreterSSZ18.pdf}{KreterSSZ18} (121.00)& \cellcolor{red!40}\href{../works/Caballero19.pdf}{Caballero19} (117.00)\\
Cosine& \cellcolor{red!40}\href{../works/KreterSS17.pdf}{KreterSS17} (0.87)& \cellcolor{red!40}\href{../works/KreterSSZ18.pdf}{KreterSSZ18} (0.82)& \cellcolor{red!40}\href{../works/SzerediS16.pdf}{SzerediS16} (0.77)& \cellcolor{red!40}\href{../works/YoungFS17.pdf}{YoungFS17} (0.77)& \cellcolor{red!40}\href{../works/abs-1009-0347.pdf}{abs-1009-0347} (0.75)\\
\index{KreterSS17}\href{../works/KreterSS17.pdf}{KreterSS17} R\&C& \cellcolor{red!40}\href{../works/KreterSS15.pdf}{KreterSS15} (0.56)& \cellcolor{red!40}\href{../works/KreterSSZ18.pdf}{KreterSSZ18} (0.79)& \cellcolor{red!40}SchuttFSW15 (0.80)& \cellcolor{red!40}\href{../works/YoungFS17.pdf}{YoungFS17} (0.81)& \cellcolor{red!40}\href{../works/SchuttCSW12.pdf}{SchuttCSW12} (0.82)\\
Euclid& \cellcolor{red!20}\href{../works/KreterSS15.pdf}{KreterSS15} (0.26)& \cellcolor{green!20}\href{../works/KreterSSZ18.pdf}{KreterSSZ18} (0.29)& \cellcolor{black!20}\href{../works/YoungFS17.pdf}{YoungFS17} (0.35)& \cellcolor{black!20}\href{../works/SzerediS16.pdf}{SzerediS16} (0.35)& \cellcolor{black!20}\href{../works/abs-1009-0347.pdf}{abs-1009-0347} (0.37)\\
Dot& \cellcolor{red!40}\href{../works/Schutt11.pdf}{Schutt11} (157.00)& \cellcolor{red!40}\href{../works/Godet21a.pdf}{Godet21a} (156.00)& \cellcolor{red!40}\href{../works/Caballero19.pdf}{Caballero19} (148.00)& \cellcolor{red!40}\href{../works/KreterSSZ18.pdf}{KreterSSZ18} (147.00)& \cellcolor{red!40}\href{../works/Groleaz21.pdf}{Groleaz21} (145.00)\\
Cosine& \cellcolor{red!40}\href{../works/KreterSS15.pdf}{KreterSS15} (0.87)& \cellcolor{red!40}\href{../works/KreterSSZ18.pdf}{KreterSSZ18} (0.84)& \cellcolor{red!40}\href{../works/YoungFS17.pdf}{YoungFS17} (0.75)& \cellcolor{red!40}\href{../works/SzerediS16.pdf}{SzerediS16} (0.74)& \cellcolor{red!40}\href{../works/abs-1009-0347.pdf}{abs-1009-0347} (0.72)\\
\index{KreterSSZ18}\href{../works/KreterSSZ18.pdf}{KreterSSZ18} R\&C& \cellcolor{red!40}\href{../works/KreterSS17.pdf}{KreterSS17} (0.79)& \cellcolor{red!40}\href{../works/SchnellH15.pdf}{SchnellH15} (0.80)& \cellcolor{red!40}\href{../works/SchnellH17.pdf}{SchnellH17} (0.81)& \cellcolor{red!40}\href{../works/KreterSS15.pdf}{KreterSS15} (0.84)& \cellcolor{red!20}EdwardsBSE19 (0.87)\\
Euclid& \cellcolor{yellow!20}\href{../works/KreterSS15.pdf}{KreterSS15} (0.28)& \cellcolor{green!20}\href{../works/KreterSS17.pdf}{KreterSS17} (0.29)& \cellcolor{green!20}\href{../works/SzerediS16.pdf}{SzerediS16} (0.30)& \cellcolor{green!20}\href{../works/YoungFS17.pdf}{YoungFS17} (0.30)& \cellcolor{green!20}\href{../works/SchuttFSW13.pdf}{SchuttFSW13} (0.30)\\
Dot& \cellcolor{red!40}\href{../works/KreterSS17.pdf}{KreterSS17} (147.00)& \cellcolor{red!40}\href{../works/Schutt11.pdf}{Schutt11} (142.00)& \cellcolor{red!40}\href{../works/Godet21a.pdf}{Godet21a} (140.00)& \cellcolor{red!40}\href{../works/Lombardi10.pdf}{Lombardi10} (135.00)& \cellcolor{red!40}\href{../works/Caballero19.pdf}{Caballero19} (135.00)\\
Cosine& \cellcolor{red!40}\href{../works/KreterSS17.pdf}{KreterSS17} (0.84)& \cellcolor{red!40}\href{../works/KreterSS15.pdf}{KreterSS15} (0.82)& \cellcolor{red!40}\href{../works/SchuttFSW13.pdf}{SchuttFSW13} (0.80)& \cellcolor{red!40}\href{../works/SzerediS16.pdf}{SzerediS16} (0.79)& \cellcolor{red!40}\href{../works/YoungFS17.pdf}{YoungFS17} (0.79)\\
\index{KrogtLPHJ07}\href{../works/KrogtLPHJ07.pdf}{KrogtLPHJ07} R\&C& \cellcolor{yellow!20}\href{../works/Wolf05.pdf}{Wolf05} (0.91)& \cellcolor{yellow!20}\href{../works/TanSD10.pdf}{TanSD10} (0.93)& \cellcolor{green!20}\href{../works/MonetteDD07.pdf}{MonetteDD07} (0.93)& \cellcolor{green!20}\href{../works/BosiM2001.pdf}{BosiM2001} (0.93)& \cellcolor{green!20}\href{../works/SourdN00.pdf}{SourdN00} (0.94)\\
Euclid& \cellcolor{red!20}\href{../works/ZibranR11a.pdf}{ZibranR11a} (0.25)& \cellcolor{red!20}\href{../works/WallaceF00.pdf}{WallaceF00} (0.25)& \cellcolor{red!20}\href{../works/FoxAS82.pdf}{FoxAS82} (0.25)& \cellcolor{red!20}\href{../works/ZibranR11.pdf}{ZibranR11} (0.26)& \cellcolor{red!20}\href{../works/RodriguezDG02.pdf}{RodriguezDG02} (0.26)\\
Dot& \cellcolor{red!40}\href{../works/ZarandiASC20.pdf}{ZarandiASC20} (84.00)& \cellcolor{red!40}\href{../works/Lombardi10.pdf}{Lombardi10} (79.00)& \cellcolor{red!40}\href{../works/HarjunkoskiMBC14.pdf}{HarjunkoskiMBC14} (78.00)& \cellcolor{red!40}\href{../works/Dejemeppe16.pdf}{Dejemeppe16} (77.00)& \cellcolor{red!40}\href{../works/Beck99.pdf}{Beck99} (77.00)\\
Cosine& \cellcolor{red!40}\href{../works/HarjunkoskiG02.pdf}{HarjunkoskiG02} (0.73)& \cellcolor{red!40}\href{../works/ZeballosM09.pdf}{ZeballosM09} (0.70)& \cellcolor{red!40}\href{../works/JainG01.pdf}{JainG01} (0.70)& \cellcolor{red!40}\href{../works/PolicellaWSO05.pdf}{PolicellaWSO05} (0.68)& \cellcolor{red!40}\href{../works/MaraveliasCG04.pdf}{MaraveliasCG04} (0.68)\\
\index{KuB16}\href{../works/KuB16.pdf}{KuB16} R\&C& \cellcolor{red!20}\href{../works/ColT2019a.pdf}{ColT2019a} (0.87)& \cellcolor{red!20}\href{../works/ColT19.pdf}{ColT19} (0.89)& \cellcolor{red!20}\href{../works/ColT22.pdf}{ColT22} (0.89)& \cellcolor{red!20}\href{../works/GrimesHM09.pdf}{GrimesHM09} (0.90)& \cellcolor{yellow!20}\href{../works/WatsonB08.pdf}{WatsonB08} (0.90)\\
Euclid& \cellcolor{green!20}\href{../works/TanSD10.pdf}{TanSD10} (0.30)& \cellcolor{green!20}\href{../works/WatsonB08.pdf}{WatsonB08} (0.30)& \cellcolor{green!20}\href{../works/ArtiguesBF04.pdf}{ArtiguesBF04} (0.31)& \cellcolor{blue!20}\href{../works/FontaineMH16.pdf}{FontaineMH16} (0.32)& \cellcolor{blue!20}\href{../works/Beck07.pdf}{Beck07} (0.32)\\
Dot& \cellcolor{red!40}\href{../works/Groleaz21.pdf}{Groleaz21} (129.00)& \cellcolor{red!40}\href{../works/NaderiRR23.pdf}{NaderiRR23} (127.00)& \cellcolor{red!40}\href{../works/Malapert11.pdf}{Malapert11} (116.00)& \cellcolor{red!40}\href{../works/GrimesH11.pdf}{GrimesH11} (111.00)& \cellcolor{red!40}\href{../works/GrimesH15.pdf}{GrimesH15} (110.00)\\
Cosine& \cellcolor{red!40}\href{../works/TanSD10.pdf}{TanSD10} (0.75)& \cellcolor{red!40}\href{../works/ArtiguesBF04.pdf}{ArtiguesBF04} (0.74)& \cellcolor{red!40}\href{../works/ZhangBB22.pdf}{ZhangBB22} (0.73)& \cellcolor{red!40}\href{../works/WatsonB08.pdf}{WatsonB08} (0.72)& \cellcolor{red!40}\href{../works/Beck07.pdf}{Beck07} (0.71)\\
\index{Kuchcinski03}\href{../works/Kuchcinski03.pdf}{Kuchcinski03} R\&C& \cellcolor{red!20}\href{../works/KuchcinskiW03.pdf}{KuchcinskiW03} (0.89)& \cellcolor{red!20}\href{../works/Colombani96.pdf}{Colombani96} (0.90)& \cellcolor{yellow!20}\href{../works/Goltz95.pdf}{Goltz95} (0.91)& \cellcolor{green!20}\href{../works/BeldiceanuC94.pdf}{BeldiceanuC94} (0.94)& \cellcolor{green!20}DorndorfHP99 (0.94)\\
Euclid& \href{../works/KuchcinskiW03.pdf}{KuchcinskiW03} (0.38)& \href{../works/Goltz95.pdf}{Goltz95} (0.39)& \href{../works/GruianK98.pdf}{GruianK98} (0.39)& \href{../works/ChuGNSW13.pdf}{ChuGNSW13} (0.40)& \href{../works/LozanoCDS12.pdf}{LozanoCDS12} (0.40)\\
Dot& \cellcolor{red!40}\href{../works/Malapert11.pdf}{Malapert11} (156.00)& \cellcolor{red!40}\href{../works/Dejemeppe16.pdf}{Dejemeppe16} (146.00)& \cellcolor{red!40}\href{../works/Schutt11.pdf}{Schutt11} (144.00)& \cellcolor{red!40}\href{../works/Lombardi10.pdf}{Lombardi10} (143.00)& \cellcolor{red!40}\href{../works/Groleaz21.pdf}{Groleaz21} (139.00)\\
Cosine& \cellcolor{red!40}\href{../works/KuchcinskiW03.pdf}{KuchcinskiW03} (0.68)& \cellcolor{red!40}\href{../works/Goltz95.pdf}{Goltz95} (0.66)& \cellcolor{red!40}\href{../works/GruianK98.pdf}{GruianK98} (0.65)& \cellcolor{red!40}\href{../works/TrojetHL11.pdf}{TrojetHL11} (0.65)& \cellcolor{red!40}\href{../works/ChuGNSW13.pdf}{ChuGNSW13} (0.65)\\
\index{KuchcinskiW03}\href{../works/KuchcinskiW03.pdf}{KuchcinskiW03} R\&C& \cellcolor{red!20}\href{../works/Kuchcinski03.pdf}{Kuchcinski03} (0.89)& \cellcolor{yellow!20}\href{../works/BeniniLMR08.pdf}{BeniniLMR08} (0.92)& \cellcolor{green!20}\href{../works/ZhangLS12.pdf}{ZhangLS12} (0.95)& \cellcolor{green!20}\href{../works/QuirogaZH05.pdf}{QuirogaZH05} (0.95)& \cellcolor{green!20}\href{../works/Geske05.pdf}{Geske05} (0.95)\\
Euclid& \cellcolor{red!40}\href{../works/MalikMB08.pdf}{MalikMB08} (0.21)& \cellcolor{red!40}\href{../works/WolinskiKG04.pdf}{WolinskiKG04} (0.22)& \cellcolor{red!40}\href{../works/LozanoCDS12.pdf}{LozanoCDS12} (0.23)& \cellcolor{red!20}\href{../works/LiuLH19.pdf}{LiuLH19} (0.25)& \cellcolor{red!20}\href{../works/BegB13.pdf}{BegB13} (0.25)\\
Dot& \cellcolor{red!40}\href{../works/Kuchcinski03.pdf}{Kuchcinski03} (77.00)& \cellcolor{red!40}\href{../works/Caballero19.pdf}{Caballero19} (74.00)& \cellcolor{red!40}\href{../works/Malapert11.pdf}{Malapert11} (73.00)& \cellcolor{red!40}\href{../works/Groleaz21.pdf}{Groleaz21} (73.00)& \cellcolor{red!40}\href{../works/Lombardi10.pdf}{Lombardi10} (72.00)\\
Cosine& \cellcolor{red!40}\href{../works/MalikMB08.pdf}{MalikMB08} (0.77)& \cellcolor{red!40}\href{../works/WolinskiKG04.pdf}{WolinskiKG04} (0.77)& \cellcolor{red!40}\href{../works/LozanoCDS12.pdf}{LozanoCDS12} (0.74)& \cellcolor{red!40}\href{../works/BegB13.pdf}{BegB13} (0.73)& \cellcolor{red!40}\href{../works/Malik08.pdf}{Malik08} (0.71)\\
\index{KucukY19}\href{../works/KucukY19.pdf}{KucukY19} R\&C& \cellcolor{blue!20}\href{../works/SquillaciPR23.pdf}{SquillaciPR23} (0.97)\\
Euclid& \cellcolor{red!40}\href{../works/FrankDT16.pdf}{FrankDT16} (0.23)& \cellcolor{red!40}\href{../works/VerfaillieL01.pdf}{VerfaillieL01} (0.23)& \cellcolor{red!20}\href{../works/GomesHS06.pdf}{GomesHS06} (0.25)& \cellcolor{yellow!20}\href{../works/SquillaciPR23.pdf}{SquillaciPR23} (0.26)& \cellcolor{yellow!20}\href{../works/SunLYL10.pdf}{SunLYL10} (0.26)\\
Dot& \cellcolor{red!40}\href{../works/Groleaz21.pdf}{Groleaz21} (81.00)& \cellcolor{red!40}\href{../works/ZarandiASC20.pdf}{ZarandiASC20} (78.00)& \cellcolor{red!40}\href{../works/Astrand21.pdf}{Astrand21} (77.00)& \cellcolor{red!40}\href{../works/Froger16.pdf}{Froger16} (75.00)& \cellcolor{red!40}\href{../works/LaborieRSV18.pdf}{LaborieRSV18} (73.00)\\
Cosine& \cellcolor{red!40}\href{../works/VerfaillieL01.pdf}{VerfaillieL01} (0.75)& \cellcolor{red!40}\href{../works/SquillaciPR23.pdf}{SquillaciPR23} (0.73)& \cellcolor{red!40}\href{../works/FrankDT16.pdf}{FrankDT16} (0.73)& \cellcolor{red!40}\href{../works/PraletLJ15.pdf}{PraletLJ15} (0.67)& \cellcolor{red!40}\href{../works/AkramNHRSA23.pdf}{AkramNHRSA23} (0.66)\\
\index{Kumar03}\href{../works/Kumar03.pdf}{Kumar03} R\&C& \cellcolor{red!40}\href{../works/LombardiM09.pdf}{LombardiM09} (0.84)& \cellcolor{red!20}\href{../works/Muscettola02.pdf}{Muscettola02} (0.89)& \cellcolor{red!20}\href{../works/BeldiceanuP07.pdf}{BeldiceanuP07} (0.89)& \cellcolor{red!20}\href{../works/BofillCSV17a.pdf}{BofillCSV17a} (0.90)& \cellcolor{yellow!20}\href{../works/LouieVNB14.pdf}{LouieVNB14} (0.91)\\
Euclid& \cellcolor{red!40}\href{../works/LiuJ06.pdf}{LiuJ06} (0.20)& \cellcolor{red!40}\href{../works/CestaOS98.pdf}{CestaOS98} (0.20)& \cellcolor{red!40}\href{../works/Caballero23.pdf}{Caballero23} (0.20)& \cellcolor{red!40}\href{../works/WallaceF00.pdf}{WallaceF00} (0.21)& \cellcolor{red!40}\href{../works/KovacsEKV05.pdf}{KovacsEKV05} (0.21)\\
Dot& \cellcolor{red!40}\href{../works/LombardiMRB10.pdf}{LombardiMRB10} (42.00)& \cellcolor{red!40}\href{../works/Caballero19.pdf}{Caballero19} (40.00)& \cellcolor{red!40}\href{../works/Simonis07.pdf}{Simonis07} (39.00)& \cellcolor{red!40}\href{../works/RuggieroBBMA09.pdf}{RuggieroBBMA09} (39.00)& \cellcolor{red!40}\href{../works/Malapert11.pdf}{Malapert11} (39.00)\\
Cosine& \cellcolor{red!40}\href{../works/LozanoCDS12.pdf}{LozanoCDS12} (0.64)& \cellcolor{red!40}\href{../works/WolinskiKG04.pdf}{WolinskiKG04} (0.63)& \cellcolor{red!40}\href{../works/KuchcinskiW03.pdf}{KuchcinskiW03} (0.63)& \cellcolor{red!40}\href{../works/LiuJ06.pdf}{LiuJ06} (0.63)& \cellcolor{red!40}\href{../works/CestaOS98.pdf}{CestaOS98} (0.62)\\
\index{KusterJF07}\href{../works/KusterJF07.pdf}{KusterJF07} R\&C\\
Euclid& \cellcolor{red!20}\href{../works/BofillCSV17a.pdf}{BofillCSV17a} (0.25)& \cellcolor{red!20}\href{../works/BhatnagarKL19.pdf}{BhatnagarKL19} (0.26)& \cellcolor{red!20}\href{../works/LombardiM13.pdf}{LombardiM13} (0.26)& \cellcolor{yellow!20}\href{../works/KeriK07.pdf}{KeriK07} (0.27)& \cellcolor{yellow!20}\href{../works/BofillCSV17.pdf}{BofillCSV17} (0.28)\\
Dot& \cellcolor{red!40}\href{../works/ZarandiASC20.pdf}{ZarandiASC20} (121.00)& \cellcolor{red!40}\href{../works/Groleaz21.pdf}{Groleaz21} (109.00)& \cellcolor{red!40}\href{../works/Baptiste02.pdf}{Baptiste02} (106.00)& \cellcolor{red!40}\href{../works/LombardiM12.pdf}{LombardiM12} (105.00)& \cellcolor{red!40}\href{../works/Lombardi10.pdf}{Lombardi10} (105.00)\\
Cosine& \cellcolor{red!40}\href{../works/BofillCSV17a.pdf}{BofillCSV17a} (0.78)& \cellcolor{red!40}\href{../works/BhatnagarKL19.pdf}{BhatnagarKL19} (0.76)& \cellcolor{red!40}\href{../works/LombardiM13.pdf}{LombardiM13} (0.75)& \cellcolor{red!40}\href{../works/KeriK07.pdf}{KeriK07} (0.74)& \cellcolor{red!40}\href{../works/BofillCSV17.pdf}{BofillCSV17} (0.72)\\
\index{Laborie03}\href{../works/Laborie03.pdf}{Laborie03} R\&C& \cellcolor{red!40}BaptisteLPN06 (0.78)& \cellcolor{red!40}\href{../works/NuijtenA96.pdf}{NuijtenA96} (0.85)& \cellcolor{red!20}\href{../works/MercierH07.pdf}{MercierH07} (0.88)& \cellcolor{red!20}\href{../works/BartakSR08.pdf}{BartakSR08} (0.89)& \cellcolor{red!20}\href{../works/BeckF00.pdf}{BeckF00} (0.89)\\
Euclid& \cellcolor{blue!20}\href{../works/BeckDSF97.pdf}{BeckDSF97} (0.34)& \cellcolor{black!20}\href{../works/VilimBC04.pdf}{VilimBC04} (0.34)& \cellcolor{black!20}\href{../works/VilimBC05.pdf}{VilimBC05} (0.35)& \cellcolor{black!20}\href{../works/BaptisteP00.pdf}{BaptisteP00} (0.35)& \cellcolor{black!20}\href{../works/DemasseyAM05.pdf}{DemasseyAM05} (0.35)\\
Dot& \cellcolor{red!40}\href{../works/Baptiste02.pdf}{Baptiste02} (175.00)& \cellcolor{red!40}\href{../works/Malapert11.pdf}{Malapert11} (164.00)& \cellcolor{red!40}\href{../works/Fahimi16.pdf}{Fahimi16} (164.00)& \cellcolor{red!40}\href{../works/Lombardi10.pdf}{Lombardi10} (163.00)& \cellcolor{red!40}\href{../works/Schutt11.pdf}{Schutt11} (156.00)\\
Cosine& \cellcolor{red!40}\href{../works/BaptisteP00.pdf}{BaptisteP00} (0.75)& \cellcolor{red!40}\href{../works/BeckDSF97.pdf}{BeckDSF97} (0.74)& \cellcolor{red!40}\href{../works/BaptisteP97.pdf}{BaptisteP97} (0.74)& \cellcolor{red!40}\href{../works/DemasseyAM05.pdf}{DemasseyAM05} (0.74)& \cellcolor{red!40}\href{../works/VilimBC05.pdf}{VilimBC05} (0.73)\\
\index{Laborie05}\href{../works/Laborie05.pdf}{Laborie05} R\&C\\
Euclid& \cellcolor{yellow!20}\href{../works/DemasseyAM05.pdf}{DemasseyAM05} (0.27)& \cellcolor{yellow!20}\href{../works/VilimLS15.pdf}{VilimLS15} (0.28)& \cellcolor{yellow!20}\href{../works/BofillCSV17a.pdf}{BofillCSV17a} (0.28)& \cellcolor{green!20}\href{../works/HentenryckM04.pdf}{HentenryckM04} (0.29)& \cellcolor{green!20}\href{../works/CestaOF99.pdf}{CestaOF99} (0.29)\\
Dot& \cellcolor{red!40}\href{../works/Schutt11.pdf}{Schutt11} (148.00)& \cellcolor{red!40}\href{../works/Godet21a.pdf}{Godet21a} (142.00)& \cellcolor{red!40}\href{../works/Baptiste02.pdf}{Baptiste02} (140.00)& \cellcolor{red!40}\href{../works/Groleaz21.pdf}{Groleaz21} (139.00)& \cellcolor{red!40}\href{../works/Siala15a.pdf}{Siala15a} (133.00)\\
Cosine& \cellcolor{red!40}\href{../works/DemasseyAM05.pdf}{DemasseyAM05} (0.82)& \cellcolor{red!40}\href{../works/VilimLS15.pdf}{VilimLS15} (0.82)& \cellcolor{red!40}\href{../works/Pralet17.pdf}{Pralet17} (0.79)& \cellcolor{red!40}\href{../works/HentenryckM04.pdf}{HentenryckM04} (0.79)& \cellcolor{red!40}\href{../works/CestaOF99.pdf}{CestaOF99} (0.78)\\
\index{Laborie09}\href{../works/Laborie09.pdf}{Laborie09} R\&C& \cellcolor{red!40}\href{../works/DannaP03.pdf}{DannaP03} (0.79)& \cellcolor{red!20}\href{../works/PerronSF04.pdf}{PerronSF04} (0.88)& \cellcolor{red!20}\href{../works/GrimesH11.pdf}{GrimesH11} (0.88)& \cellcolor{green!20}\href{../works/Mercier-AubinGQ20.pdf}{Mercier-AubinGQ20} (0.93)& \cellcolor{green!20}\href{../works/CarchraeB09.pdf}{CarchraeB09} (0.94)\\
Euclid& \cellcolor{blue!20}\href{../works/LaborieR14.pdf}{LaborieR14} (0.32)& \cellcolor{blue!20}\href{../works/MonetteDH09.pdf}{MonetteDH09} (0.33)& \cellcolor{blue!20}\href{../works/Hooker05a.pdf}{Hooker05a} (0.33)& \cellcolor{blue!20}\href{../works/PraletLJ15.pdf}{PraletLJ15} (0.34)& \cellcolor{blue!20}\href{../works/Limtanyakul07.pdf}{Limtanyakul07} (0.34)\\
Dot& \cellcolor{red!40}\href{../works/LaborieRSV18.pdf}{LaborieRSV18} (152.00)& \cellcolor{red!40}\href{../works/Groleaz21.pdf}{Groleaz21} (149.00)& \cellcolor{red!40}\href{../works/Dejemeppe16.pdf}{Dejemeppe16} (148.00)& \cellcolor{red!40}\href{../works/ZarandiASC20.pdf}{ZarandiASC20} (138.00)& \cellcolor{red!40}\href{../works/Baptiste02.pdf}{Baptiste02} (132.00)\\
Cosine& \cellcolor{red!40}\href{../works/LaborieR14.pdf}{LaborieR14} (0.75)& \cellcolor{red!40}\href{../works/MonetteDH09.pdf}{MonetteDH09} (0.74)& \cellcolor{red!40}\href{../works/LaborieRSV18.pdf}{LaborieRSV18} (0.69)& \cellcolor{red!40}\href{../works/KhayatLR06.pdf}{KhayatLR06} (0.69)& \cellcolor{red!40}\href{../works/Hooker05a.pdf}{Hooker05a} (0.69)\\
\index{Laborie18a}\href{../works/Laborie18a.pdf}{Laborie18a} R\&C& \cellcolor{red!40}\href{../works/ColT19.pdf}{ColT19} (0.83)& \cellcolor{red!40}\href{../works/ColT2019a.pdf}{ColT2019a} (0.85)& \cellcolor{red!40}\href{../works/LaborieRSV18.pdf}{LaborieRSV18} (0.85)& \cellcolor{red!20}\href{../works/BoothTNB16.pdf}{BoothTNB16} (0.87)& \cellcolor{red!20}\href{../works/CappartTSR18.pdf}{CappartTSR18} (0.89)\\
Euclid& \cellcolor{red!40}\href{../works/Limtanyakul07.pdf}{Limtanyakul07} (0.24)& \cellcolor{red!20}\href{../works/HookerY02.pdf}{HookerY02} (0.26)& \cellcolor{yellow!20}\href{../works/KovacsEKV05.pdf}{KovacsEKV05} (0.27)& \cellcolor{yellow!20}\href{../works/WuBB05.pdf}{WuBB05} (0.28)& \cellcolor{yellow!20}\href{../works/HeipckeCCS00.pdf}{HeipckeCCS00} (0.28)\\
Dot& \cellcolor{red!40}\href{../works/LaborieRSV18.pdf}{LaborieRSV18} (94.00)& \cellcolor{red!40}\href{../works/Lombardi10.pdf}{Lombardi10} (92.00)& \cellcolor{red!40}\href{../works/Dejemeppe16.pdf}{Dejemeppe16} (91.00)& \cellcolor{red!40}\href{../works/Groleaz21.pdf}{Groleaz21} (86.00)& \cellcolor{red!40}\href{../works/ColT22.pdf}{ColT22} (84.00)\\
Cosine& \cellcolor{red!40}\href{../works/Limtanyakul07.pdf}{Limtanyakul07} (0.74)& \cellcolor{red!40}\href{../works/HeipckeCCS00.pdf}{HeipckeCCS00} (0.73)& \cellcolor{red!40}\href{../works/Hooker06.pdf}{Hooker06} (0.72)& \cellcolor{red!40}\href{../works/LimtanyakulS12.pdf}{LimtanyakulS12} (0.70)& \cellcolor{red!40}\href{../works/HookerY02.pdf}{HookerY02} (0.70)\\
\index{LaborieR14}\href{../works/LaborieR14.pdf}{LaborieR14} R\&C& \cellcolor{red!20}\href{../works/VilimLS15.pdf}{VilimLS15} (0.88)& \cellcolor{red!20}\href{../works/ColT19.pdf}{ColT19} (0.89)& \cellcolor{red!20}\href{../works/CappartS17.pdf}{CappartS17} (0.89)& \cellcolor{yellow!20}\href{../works/LaborieRSV18.pdf}{LaborieRSV18} (0.91)& \cellcolor{yellow!20}\href{../works/Beck10.pdf}{Beck10} (0.93)\\
Euclid& \cellcolor{green!20}\href{../works/HeipckeCCS00.pdf}{HeipckeCCS00} (0.29)& \cellcolor{green!20}\href{../works/MonetteDH09.pdf}{MonetteDH09} (0.29)& \cellcolor{green!20}\href{../works/KovacsV06.pdf}{KovacsV06} (0.31)& \cellcolor{blue!20}\href{../works/Laborie09.pdf}{Laborie09} (0.32)& \cellcolor{blue!20}\href{../works/HentenryckM04.pdf}{HentenryckM04} (0.32)\\
Dot& \cellcolor{red!40}\href{../works/LaborieRSV18.pdf}{LaborieRSV18} (161.00)& \cellcolor{red!40}\href{../works/Groleaz21.pdf}{Groleaz21} (160.00)& \cellcolor{red!40}\href{../works/ZarandiASC20.pdf}{ZarandiASC20} (145.00)& \cellcolor{red!40}\href{../works/Baptiste02.pdf}{Baptiste02} (139.00)& \cellcolor{red!40}\href{../works/Lombardi10.pdf}{Lombardi10} (138.00)\\
Cosine& \cellcolor{red!40}\href{../works/MonetteDH09.pdf}{MonetteDH09} (0.80)& \cellcolor{red!40}\href{../works/HeipckeCCS00.pdf}{HeipckeCCS00} (0.78)& \cellcolor{red!40}\href{../works/Laborie09.pdf}{Laborie09} (0.75)& \cellcolor{red!40}\href{../works/TouatBT22.pdf}{TouatBT22} (0.74)& \cellcolor{red!40}\href{../works/KovacsV06.pdf}{KovacsV06} (0.74)\\
\index{LaborieRSV18}\href{../works/LaborieRSV18.pdf}{LaborieRSV18} R\&C& \cellcolor{red!40}\href{../works/HamP21.pdf}{HamP21} (0.85)& \cellcolor{red!40}\href{../works/Laborie18a.pdf}{Laborie18a} (0.85)& \cellcolor{red!20}\href{../works/VilimLS15.pdf}{VilimLS15} (0.87)& \cellcolor{red!20}\href{../works/Ham18a.pdf}{Ham18a} (0.89)& \cellcolor{yellow!20}\href{../works/LuoB22.pdf}{LuoB22} (0.90)\\
Euclid& \href{../works/LaborieR14.pdf}{LaborieR14} (0.49)& \href{../works/Laborie09.pdf}{Laborie09} (0.51)& \href{../works/NovaraNH16.pdf}{NovaraNH16} (0.52)& \href{../works/VilimLS15.pdf}{VilimLS15} (0.52)& \href{../works/HauderBRPA20.pdf}{HauderBRPA20} (0.54)\\
Dot& \cellcolor{red!40}\href{../works/Groleaz21.pdf}{Groleaz21} (262.00)& \cellcolor{red!40}\href{../works/Dejemeppe16.pdf}{Dejemeppe16} (240.00)& \cellcolor{red!40}\href{../works/ZarandiASC20.pdf}{ZarandiASC20} (239.00)& \cellcolor{red!40}\href{../works/Astrand21.pdf}{Astrand21} (226.00)& \cellcolor{red!40}\href{../works/Lombardi10.pdf}{Lombardi10} (219.00)\\
Cosine& \cellcolor{red!40}\href{../works/LaborieR14.pdf}{LaborieR14} (0.73)& \cellcolor{red!40}\href{../works/Laborie09.pdf}{Laborie09} (0.69)& \cellcolor{red!40}\href{../works/NovaraNH16.pdf}{NovaraNH16} (0.68)& \cellcolor{red!40}\href{../works/VilimLS15.pdf}{VilimLS15} (0.67)& \cellcolor{red!40}\href{../works/LombardiM12.pdf}{LombardiM12} (0.66)\\
\index{LacknerMMWW21}\href{../works/LacknerMMWW21.pdf}{LacknerMMWW21} R\&C\\
Euclid& \cellcolor{red!40}\href{../works/LacknerMMWW23.pdf}{LacknerMMWW23} (0.22)& \href{../works/ColT2019a.pdf}{ColT2019a} (0.46)& \href{../works/MeyerE04.pdf}{MeyerE04} (0.49)& \href{../works/GroleazNS20a.pdf}{GroleazNS20a} (0.49)& \href{../works/ColT19.pdf}{ColT19} (0.49)\\
Dot& \cellcolor{red!40}\href{../works/LacknerMMWW23.pdf}{LacknerMMWW23} (240.00)& \cellcolor{red!40}\href{../works/Groleaz21.pdf}{Groleaz21} (191.00)& \cellcolor{red!40}\href{../works/Dejemeppe16.pdf}{Dejemeppe16} (174.00)& \cellcolor{red!40}\href{../works/ColT22.pdf}{ColT22} (169.00)& \cellcolor{red!40}\href{../works/Lunardi20.pdf}{Lunardi20} (160.00)\\
Cosine& \cellcolor{red!40}\href{../works/LacknerMMWW23.pdf}{LacknerMMWW23} (0.94)& \cellcolor{red!40}\href{../works/ColT22.pdf}{ColT22} (0.65)& \cellcolor{red!40}\href{../works/ColT2019a.pdf}{ColT2019a} (0.62)& \cellcolor{red!40}\href{../works/WinterMMW22.pdf}{WinterMMW22} (0.60)& \cellcolor{red!40}\href{../works/GroleazNS20a.pdf}{GroleazNS20a} (0.59)\\
\index{LacknerMMWW23}\href{../works/LacknerMMWW23.pdf}{LacknerMMWW23} R\&C& \cellcolor{red!20}\href{../works/MalapertGR12.pdf}{MalapertGR12} (0.89)& \cellcolor{yellow!20}\href{../works/TangB20.pdf}{TangB20} (0.91)& \cellcolor{yellow!20}\href{../works/HamFC17.pdf}{HamFC17} (0.92)& \cellcolor{green!20}\href{../works/KoschB14.pdf}{KoschB14} (0.94)& \cellcolor{green!20}\href{../works/ColT19.pdf}{ColT19} (0.95)\\
Euclid& \cellcolor{red!40}\href{../works/LacknerMMWW21.pdf}{LacknerMMWW21} (0.22)& \href{../works/ColT2019a.pdf}{ColT2019a} (0.50)& \href{../works/MeyerE04.pdf}{MeyerE04} (0.52)& \href{../works/MalapertGR12.pdf}{MalapertGR12} (0.52)& \href{../works/ColT22.pdf}{ColT22} (0.52)\\
Dot& \cellcolor{red!40}\href{../works/LacknerMMWW21.pdf}{LacknerMMWW21} (240.00)& \cellcolor{red!40}\href{../works/Groleaz21.pdf}{Groleaz21} (209.00)& \cellcolor{red!40}\href{../works/Dejemeppe16.pdf}{Dejemeppe16} (199.00)& \cellcolor{red!40}\href{../works/ColT22.pdf}{ColT22} (185.00)& \cellcolor{red!40}\href{../works/Lunardi20.pdf}{Lunardi20} (180.00)\\
Cosine& \cellcolor{red!40}\href{../works/LacknerMMWW21.pdf}{LacknerMMWW21} (0.94)& \cellcolor{red!40}\href{../works/ColT22.pdf}{ColT22} (0.66)& \cellcolor{red!40}\href{../works/WinterMMW22.pdf}{WinterMMW22} (0.62)& \cellcolor{red!40}\href{../works/ColT2019a.pdf}{ColT2019a} (0.61)& \cellcolor{red!40}\href{../works/AwadMDMT22.pdf}{AwadMDMT22} (0.59)\\
\index{LahimerLH11}\href{../works/LahimerLH11.pdf}{LahimerLH11} R\&C& \cellcolor{blue!20}\href{../works/GuSS13.pdf}{GuSS13} (0.97)& \cellcolor{blue!20}\href{../works/MalapertGR12.pdf}{MalapertGR12} (0.98)& \cellcolor{blue!20}GuSSWC14 (0.98)& \cellcolor{black!20}\href{../works/GomesM17.pdf}{GomesM17} (0.98)& \cellcolor{black!20}\href{../works/BruckerK00.pdf}{BruckerK00} (0.99)\\
Euclid& \cellcolor{yellow!20}\href{../works/HeipckeCCS00.pdf}{HeipckeCCS00} (0.27)& \cellcolor{yellow!20}\href{../works/MakMS10.pdf}{MakMS10} (0.28)& \cellcolor{green!20}\href{../works/MalapertCGJLR13.pdf}{MalapertCGJLR13} (0.29)& \cellcolor{green!20}\href{../works/Limtanyakul07.pdf}{Limtanyakul07} (0.29)& \cellcolor{green!20}\href{../works/Caseau97.pdf}{Caseau97} (0.29)\\
Dot& \cellcolor{red!40}\href{../works/Groleaz21.pdf}{Groleaz21} (116.00)& \cellcolor{red!40}\href{../works/ZarandiASC20.pdf}{ZarandiASC20} (111.00)& \cellcolor{red!40}\href{../works/Baptiste02.pdf}{Baptiste02} (111.00)& \cellcolor{red!40}\href{../works/Godet21a.pdf}{Godet21a} (106.00)& \cellcolor{red!40}\href{../works/Malapert11.pdf}{Malapert11} (105.00)\\
Cosine& \cellcolor{red!40}\href{../works/HeipckeCCS00.pdf}{HeipckeCCS00} (0.76)& \cellcolor{red!40}\href{../works/ZhangYW21.pdf}{ZhangYW21} (0.74)& \cellcolor{red!40}\href{../works/KhayatLR06.pdf}{KhayatLR06} (0.73)& \cellcolor{red!40}\href{../works/EtminaniesfahaniGNMS22.pdf}{EtminaniesfahaniGNMS22} (0.73)& \cellcolor{red!40}\href{../works/VilimLS15.pdf}{VilimLS15} (0.73)\\
\index{LammaMM97}\href{../works/LammaMM97.pdf}{LammaMM97} R\&C& \cellcolor{red!20}\href{../works/Simonis95a.pdf}{Simonis95a} (0.89)& \cellcolor{yellow!20}\href{../works/Simonis99.pdf}{Simonis99} (0.92)& \cellcolor{yellow!20}\href{../works/BaptisteLV92.pdf}{BaptisteLV92} (0.92)& \cellcolor{yellow!20}\href{../works/Zhou96.pdf}{Zhou96} (0.93)& \cellcolor{yellow!20}\href{../works/Goltz95.pdf}{Goltz95} (0.93)\\
Euclid& \cellcolor{red!40}\href{../works/BrusoniCLMMT96.pdf}{BrusoniCLMMT96} (0.23)& \cellcolor{yellow!20}\href{../works/DincbasSH90.pdf}{DincbasSH90} (0.28)& \cellcolor{green!20}\href{../works/Bartak02.pdf}{Bartak02} (0.31)& \cellcolor{blue!20}\href{../works/Simonis95.pdf}{Simonis95} (0.32)& \cellcolor{blue!20}\href{../works/Simonis95a.pdf}{Simonis95a} (0.32)\\
Dot& \cellcolor{red!40}\href{../works/ZarandiASC20.pdf}{ZarandiASC20} (110.00)& \cellcolor{red!40}\href{../works/Wallace96.pdf}{Wallace96} (103.00)& \cellcolor{red!40}\href{../works/Baptiste02.pdf}{Baptiste02} (103.00)& \cellcolor{red!40}\href{../works/Malapert11.pdf}{Malapert11} (101.00)& \cellcolor{red!40}\href{../works/TrentesauxPT01.pdf}{TrentesauxPT01} (97.00)\\
Cosine& \cellcolor{red!40}\href{../works/BrusoniCLMMT96.pdf}{BrusoniCLMMT96} (0.83)& \cellcolor{red!40}\href{../works/DincbasSH90.pdf}{DincbasSH90} (0.76)& \cellcolor{red!40}\href{../works/Wallace96.pdf}{Wallace96} (0.72)& \cellcolor{red!40}\href{../works/Simonis95a.pdf}{Simonis95a} (0.71)& \cellcolor{red!40}\href{../works/Bartak02.pdf}{Bartak02} (0.70)\\
\index{LarsonJC14}\href{../works/LarsonJC14.pdf}{LarsonJC14} R\&C& \cellcolor{red!40}\href{../works/CarlssonJL17.pdf}{CarlssonJL17} (0.64)& \cellcolor{red!40}\href{../works/RasmussenT06.pdf}{RasmussenT06} (0.80)& \cellcolor{red!40}\href{../works/RasmussenT09.pdf}{RasmussenT09} (0.81)& \cellcolor{red!20}Trick11 (0.86)& \cellcolor{red!20}\href{../works/SuCC13.pdf}{SuCC13} (0.87)\\
Euclid& \cellcolor{red!40}\href{../works/ZengM12.pdf}{ZengM12} (0.23)& \cellcolor{red!20}\href{../works/RasmussenT06.pdf}{RasmussenT06} (0.25)& \cellcolor{red!20}\href{../works/CarlssonJL17.pdf}{CarlssonJL17} (0.25)& \cellcolor{red!20}\href{../works/EastonNT02.pdf}{EastonNT02} (0.25)& \cellcolor{red!20}\href{../works/RasmussenT07.pdf}{RasmussenT07} (0.26)\\
Dot& \cellcolor{red!40}\href{../works/CarlssonJL17.pdf}{CarlssonJL17} (93.00)& \cellcolor{red!40}\href{../works/KendallKRU10.pdf}{KendallKRU10} (76.00)& \cellcolor{red!40}\href{../works/HookerH17.pdf}{HookerH17} (73.00)& \cellcolor{red!40}\href{../works/RussellU06.pdf}{RussellU06} (72.00)& \cellcolor{red!40}\href{../works/Ribeiro12.pdf}{Ribeiro12} (71.00)\\
Cosine& \cellcolor{red!40}\href{../works/CarlssonJL17.pdf}{CarlssonJL17} (0.83)& \cellcolor{red!40}\href{../works/ZengM12.pdf}{ZengM12} (0.80)& \cellcolor{red!40}\href{../works/RasmussenT06.pdf}{RasmussenT06} (0.76)& \cellcolor{red!40}\href{../works/RasmussenT07.pdf}{RasmussenT07} (0.75)& \cellcolor{red!40}\href{../works/EastonNT02.pdf}{EastonNT02} (0.74)\\
\index{LauLN08}\href{../works/LauLN08.pdf}{LauLN08} R\&C& \cellcolor{green!20}\href{../works/GuSW12.pdf}{GuSW12} (0.96)& \cellcolor{blue!20}GuSSWC14 (0.97)\\
Euclid& \cellcolor{red!40}\href{../works/CrawfordB94.pdf}{CrawfordB94} (0.14)& \cellcolor{red!40}\href{../works/HebrardTW05.pdf}{HebrardTW05} (0.19)& \cellcolor{red!40}\href{../works/FoxAS82.pdf}{FoxAS82} (0.20)& \cellcolor{red!40}\href{../works/KovacsEKV05.pdf}{KovacsEKV05} (0.21)& \cellcolor{red!40}\href{../works/AngelsmarkJ00.pdf}{AngelsmarkJ00} (0.21)\\
Dot& \cellcolor{red!40}\href{../works/ZarandiASC20.pdf}{ZarandiASC20} (65.00)& \cellcolor{red!40}\href{../works/Astrand21.pdf}{Astrand21} (65.00)& \cellcolor{red!40}\href{../works/Beck99.pdf}{Beck99} (64.00)& \cellcolor{red!40}\href{../works/HarjunkoskiMBC14.pdf}{HarjunkoskiMBC14} (62.00)& \cellcolor{red!40}\href{../works/AbreuN22.pdf}{AbreuN22} (61.00)\\
Cosine& \cellcolor{red!40}\href{../works/CrawfordB94.pdf}{CrawfordB94} (0.87)& \cellcolor{red!40}\href{../works/FoxAS82.pdf}{FoxAS82} (0.73)& \cellcolor{red!40}\href{../works/DoRZ08.pdf}{DoRZ08} (0.73)& \cellcolor{red!40}\href{../works/Beck06.pdf}{Beck06} (0.72)& \cellcolor{red!40}\href{../works/HebrardTW05.pdf}{HebrardTW05} (0.72)\\
\index{Layfield02}\href{../works/Layfield02.pdf}{Layfield02} R\&C\\
Euclid& \cellcolor{red!40}\href{../works/MaraveliasG04.pdf}{MaraveliasG04} (0.14)& \cellcolor{red!40}\href{../works/Baptiste09.pdf}{Baptiste09} (0.16)& \cellcolor{red!40}\href{../works/AbrilSB05.pdf}{AbrilSB05} (0.19)& \cellcolor{red!40}\href{../works/Hamscher91.pdf}{Hamscher91} (0.20)& \cellcolor{red!40}\href{../works/CarchraeBF05.pdf}{CarchraeBF05} (0.21)\\
Dot& \cellcolor{red!40}\href{../works/ZeballosNH11.pdf}{ZeballosNH11} (18.00)& \cellcolor{red!40}\href{../works/KoehlerBFFHPSSS21.pdf}{KoehlerBFFHPSSS21} (18.00)& \cellcolor{red!40}\href{../works/Menana11.pdf}{Menana11} (14.00)& \cellcolor{red!40}\href{../works/ZeballosCM10.pdf}{ZeballosCM10} (14.00)& \cellcolor{red!40}\href{../works/TangLWSK18.pdf}{TangLWSK18} (13.00)\\
Cosine& \cellcolor{red!40}\href{../works/MaraveliasG04.pdf}{MaraveliasG04} (0.65)& \cellcolor{red!40}\href{../works/Baptiste09.pdf}{Baptiste09} (0.46)& \cellcolor{red!40}\href{../works/Hamscher91.pdf}{Hamscher91} (0.39)& \cellcolor{red!40}\href{../works/ZibranR11.pdf}{ZibranR11} (0.36)& \cellcolor{red!40}\href{../works/Hooker17.pdf}{Hooker17} (0.36)\\
\index{LeeKLKKYHP97}\href{../works/LeeKLKKYHP97.pdf}{LeeKLKKYHP97} R\&C\\
Euclid& \cellcolor{red!40}\href{../works/FortinZDF05.pdf}{FortinZDF05} (0.19)& \cellcolor{red!40}\href{../works/LombardiM13.pdf}{LombardiM13} (0.20)& \cellcolor{red!40}\href{../works/WallaceF00.pdf}{WallaceF00} (0.22)& \cellcolor{red!40}\href{../works/ChapadosJR11.pdf}{ChapadosJR11} (0.23)& \cellcolor{red!40}\href{../works/Bonfietti16.pdf}{Bonfietti16} (0.23)\\
Dot& \cellcolor{red!40}\href{../works/LaborieRSV18.pdf}{LaborieRSV18} (64.00)& \cellcolor{red!40}\href{../works/ZarandiASC20.pdf}{ZarandiASC20} (63.00)& \cellcolor{red!40}\href{../works/Dejemeppe16.pdf}{Dejemeppe16} (62.00)& \cellcolor{red!40}\href{../works/Astrand21.pdf}{Astrand21} (61.00)& \cellcolor{red!40}\href{../works/Lunardi20.pdf}{Lunardi20} (59.00)\\
Cosine& \cellcolor{red!40}\href{../works/FortinZDF05.pdf}{FortinZDF05} (0.76)& \cellcolor{red!40}\href{../works/LombardiBMB11.pdf}{LombardiBMB11} (0.72)& \cellcolor{red!40}\href{../works/BeckPS03.pdf}{BeckPS03} (0.71)& \cellcolor{red!40}\href{../works/BonfiettiLBM11.pdf}{BonfiettiLBM11} (0.71)& \cellcolor{red!40}\href{../works/LombardiM13.pdf}{LombardiM13} (0.70)\\
\index{Lemos21}\href{../works/Lemos21.pdf}{Lemos21} R\&C\\
Euclid& \href{../works/PourDERB18.pdf}{PourDERB18} (0.52)& \href{../works/Salido10.pdf}{Salido10} (0.54)& \href{../works/KendallKRU10.pdf}{KendallKRU10} (0.54)& \href{../works/PinarbasiAY19.pdf}{PinarbasiAY19} (0.54)& \href{../works/KovacsTKSG21.pdf}{KovacsTKSG21} (0.55)\\
Dot& \cellcolor{red!40}\href{../works/ZarandiASC20.pdf}{ZarandiASC20} (221.00)& \cellcolor{red!40}\href{../works/Dejemeppe16.pdf}{Dejemeppe16} (199.00)& \cellcolor{red!40}\href{../works/Groleaz21.pdf}{Groleaz21} (190.00)& \cellcolor{red!40}\href{../works/Astrand21.pdf}{Astrand21} (188.00)& \cellcolor{red!40}\href{../works/Froger16.pdf}{Froger16} (186.00)\\
Cosine& \cellcolor{red!40}\href{../works/PourDERB18.pdf}{PourDERB18} (0.64)& \cellcolor{red!40}\href{../works/KendallKRU10.pdf}{KendallKRU10} (0.61)& \cellcolor{red!40}\href{../works/LuZZYW24.pdf}{LuZZYW24} (0.60)& \cellcolor{red!40}\href{../works/Salido10.pdf}{Salido10} (0.60)& \cellcolor{red!40}\href{../works/KovacsTKSG21.pdf}{KovacsTKSG21} (0.59)\\
\index{Letort13}\href{../works/Letort13.pdf}{Letort13} R\&C\\
Euclid& \cellcolor{green!20}\href{../works/LetortCB15.pdf}{LetortCB15} (0.31)& \cellcolor{blue!20}\href{../works/LetortBC12.pdf}{LetortBC12} (0.33)& \cellcolor{blue!20}\href{../works/LetortCB13.pdf}{LetortCB13} (0.34)& \cellcolor{black!20}\href{../works/Derrien15.pdf}{Derrien15} (0.36)& \href{../works/GayHS15a.pdf}{GayHS15a} (0.38)\\
Dot& \cellcolor{red!40}\href{../works/Malapert11.pdf}{Malapert11} (181.00)& \cellcolor{red!40}\href{../works/Schutt11.pdf}{Schutt11} (163.00)& \cellcolor{red!40}\href{../works/Fahimi16.pdf}{Fahimi16} (160.00)& \cellcolor{red!40}\href{../works/Kameugne14.pdf}{Kameugne14} (149.00)& \cellcolor{red!40}\href{../works/Godet21a.pdf}{Godet21a} (145.00)\\
Cosine& \cellcolor{red!40}\href{../works/LetortCB15.pdf}{LetortCB15} (0.82)& \cellcolor{red!40}\href{../works/LetortBC12.pdf}{LetortBC12} (0.78)& \cellcolor{red!40}\href{../works/LetortCB13.pdf}{LetortCB13} (0.77)& \cellcolor{red!40}\href{../works/Derrien15.pdf}{Derrien15} (0.75)& \cellcolor{red!40}\href{../works/Clercq12.pdf}{Clercq12} (0.73)\\
\index{LetortBC12}\href{../works/LetortBC12.pdf}{LetortBC12} R\&C& \cellcolor{red!40}\href{../works/OuelletQ13.pdf}{OuelletQ13} (0.57)& \cellcolor{red!40}\href{../works/GayHS15.pdf}{GayHS15} (0.61)& \cellcolor{red!40}\href{../works/KameugneFSN14.pdf}{KameugneFSN14} (0.66)& \cellcolor{red!40}\href{../works/Vilim11.pdf}{Vilim11} (0.67)& \cellcolor{red!40}\href{../works/LetortCB13.pdf}{LetortCB13} (0.68)\\
Euclid& \cellcolor{red!40}\href{../works/LetortCB13.pdf}{LetortCB13} (0.14)& \cellcolor{red!40}\href{../works/LetortCB15.pdf}{LetortCB15} (0.18)& \cellcolor{green!20}\href{../works/WolfS05.pdf}{WolfS05} (0.30)& \cellcolor{green!20}\href{../works/BeldiceanuP07.pdf}{BeldiceanuP07} (0.30)& \cellcolor{green!20}\href{../works/BeldiceanuC02.pdf}{BeldiceanuC02} (0.31)\\
Dot& \cellcolor{red!40}\href{../works/Malapert11.pdf}{Malapert11} (121.00)& \cellcolor{red!40}\href{../works/LetortCB15.pdf}{LetortCB15} (120.00)& \cellcolor{red!40}\href{../works/Letort13.pdf}{Letort13} (119.00)& \cellcolor{red!40}\href{../works/LetortCB13.pdf}{LetortCB13} (110.00)& \cellcolor{red!40}\href{../works/Schutt11.pdf}{Schutt11} (104.00)\\
Cosine& \cellcolor{red!40}\href{../works/LetortCB13.pdf}{LetortCB13} (0.94)& \cellcolor{red!40}\href{../works/LetortCB15.pdf}{LetortCB15} (0.92)& \cellcolor{red!40}\href{../works/Letort13.pdf}{Letort13} (0.78)& \cellcolor{red!40}\href{../works/BeldiceanuC02.pdf}{BeldiceanuC02} (0.69)& \cellcolor{red!40}\href{../works/WolfS05.pdf}{WolfS05} (0.69)\\
\index{LetortCB13}\href{../works/LetortCB13.pdf}{LetortCB13} R\&C& \cellcolor{red!40}\href{../works/LetortCB15.pdf}{LetortCB15} (0.58)& \cellcolor{red!40}\href{../works/LetortBC12.pdf}{LetortBC12} (0.68)& \cellcolor{red!40}\href{../works/KameugneFSN11.pdf}{KameugneFSN11} (0.74)& \cellcolor{red!40}\href{../works/OuelletQ13.pdf}{OuelletQ13} (0.75)& \cellcolor{red!40}\href{../works/KameugneFSN14.pdf}{KameugneFSN14} (0.80)\\
Euclid& \cellcolor{red!40}\href{../works/LetortBC12.pdf}{LetortBC12} (0.14)& \cellcolor{red!40}\href{../works/LetortCB15.pdf}{LetortCB15} (0.17)& \cellcolor{green!20}\href{../works/BeldiceanuP07.pdf}{BeldiceanuP07} (0.30)& \cellcolor{green!20}\href{../works/CauwelaertLS15.pdf}{CauwelaertLS15} (0.30)& \cellcolor{green!20}\href{../works/WolfS05.pdf}{WolfS05} (0.30)\\
Dot& \cellcolor{red!40}\href{../works/Malapert11.pdf}{Malapert11} (123.00)& \cellcolor{red!40}\href{../works/LetortCB15.pdf}{LetortCB15} (120.00)& \cellcolor{red!40}\href{../works/Letort13.pdf}{Letort13} (117.00)& \cellcolor{red!40}\href{../works/Schutt11.pdf}{Schutt11} (112.00)& \cellcolor{red!40}\href{../works/LetortBC12.pdf}{LetortBC12} (110.00)\\
Cosine& \cellcolor{red!40}\href{../works/LetortBC12.pdf}{LetortBC12} (0.94)& \cellcolor{red!40}\href{../works/LetortCB15.pdf}{LetortCB15} (0.93)& \cellcolor{red!40}\href{../works/Letort13.pdf}{Letort13} (0.77)& \cellcolor{red!40}\href{../works/GingrasQ16.pdf}{GingrasQ16} (0.71)& \cellcolor{red!40}\href{../works/BeldiceanuP07.pdf}{BeldiceanuP07} (0.70)\\
\index{LetortCB15}\href{../works/LetortCB15.pdf}{LetortCB15} R\&C& \cellcolor{red!40}\href{../works/LetortCB13.pdf}{LetortCB13} (0.58)& \cellcolor{red!40}\href{../works/LetortBC12.pdf}{LetortBC12} (0.69)& \cellcolor{red!40}\href{../works/OuelletQ13.pdf}{OuelletQ13} (0.71)& \cellcolor{red!40}\href{../works/GayHS15a.pdf}{GayHS15a} (0.77)& \cellcolor{red!40}\href{../works/KameugneFSN14.pdf}{KameugneFSN14} (0.79)\\
Euclid& \cellcolor{red!40}\href{../works/LetortCB13.pdf}{LetortCB13} (0.17)& \cellcolor{red!40}\href{../works/LetortBC12.pdf}{LetortBC12} (0.18)& \cellcolor{green!20}\href{../works/Letort13.pdf}{Letort13} (0.31)& \cellcolor{blue!20}\href{../works/GingrasQ16.pdf}{GingrasQ16} (0.33)& \cellcolor{black!20}\href{../works/DerrienP14.pdf}{DerrienP14} (0.34)\\
Dot& \cellcolor{red!40}\href{../works/Malapert11.pdf}{Malapert11} (141.00)& \cellcolor{red!40}\href{../works/Letort13.pdf}{Letort13} (138.00)& \cellcolor{red!40}\href{../works/Schutt11.pdf}{Schutt11} (131.00)& \cellcolor{red!40}\href{../works/Godet21a.pdf}{Godet21a} (124.00)& \cellcolor{red!40}\href{../works/LetortBC12.pdf}{LetortBC12} (120.00)\\
Cosine& \cellcolor{red!40}\href{../works/LetortCB13.pdf}{LetortCB13} (0.93)& \cellcolor{red!40}\href{../works/LetortBC12.pdf}{LetortBC12} (0.92)& \cellcolor{red!40}\href{../works/Letort13.pdf}{Letort13} (0.82)& \cellcolor{red!40}\href{../works/GingrasQ16.pdf}{GingrasQ16} (0.70)& \cellcolor{red!40}\href{../works/OuelletQ13.pdf}{OuelletQ13} (0.68)\\
\index{LiFJZLL22}\href{../works/LiFJZLL22.pdf}{LiFJZLL22} R\&C& \cellcolor{green!20}\href{../works/Fatemi-AnarakiTFV23.pdf}{Fatemi-AnarakiTFV23} (0.95)& \cellcolor{green!20}\href{../works/AbreuNP23.pdf}{AbreuNP23} (0.96)& \cellcolor{blue!20}\href{../works/MurinR19.pdf}{MurinR19} (0.96)& \cellcolor{blue!20}\href{../works/ParkUJR19.pdf}{ParkUJR19} (0.97)& \cellcolor{blue!20}\href{../works/HamPK21.pdf}{HamPK21} (0.97)\\
Euclid& \cellcolor{yellow!20}\href{../works/JuvinHL23.pdf}{JuvinHL23} (0.27)& \cellcolor{green!20}\href{../works/HamP21.pdf}{HamP21} (0.31)& \cellcolor{green!20}\href{../works/NovasH14.pdf}{NovasH14} (0.31)& \cellcolor{blue!20}\href{../works/Teppan22.pdf}{Teppan22} (0.32)& \cellcolor{blue!20}\href{../works/IklassovMR023.pdf}{IklassovMR023} (0.32)\\
Dot& \cellcolor{red!40}\href{../works/ZarandiASC20.pdf}{ZarandiASC20} (154.00)& \cellcolor{red!40}\href{../works/Lunardi20.pdf}{Lunardi20} (137.00)& \cellcolor{red!40}\href{../works/PrataAN23.pdf}{PrataAN23} (135.00)& \cellcolor{red!40}\href{../works/IsikYA23.pdf}{IsikYA23} (135.00)& \cellcolor{red!40}\href{../works/Groleaz21.pdf}{Groleaz21} (133.00)\\
Cosine& \cellcolor{red!40}\href{../works/JuvinHL23.pdf}{JuvinHL23} (0.79)& \cellcolor{red!40}\href{../works/CzerniachowskaWZ23.pdf}{CzerniachowskaWZ23} (0.76)& \cellcolor{red!40}\href{../works/HamP21.pdf}{HamP21} (0.75)& \cellcolor{red!40}\href{../works/OujanaAYB22.pdf}{OujanaAYB22} (0.74)& \cellcolor{red!40}\href{../works/NovasH14.pdf}{NovasH14} (0.74)\\
\index{LiLZDZW24}\href{../works/LiLZDZW24.pdf}{LiLZDZW24} R\&C\\
Euclid& \cellcolor{yellow!20}\href{../works/WatsonBHW99.pdf}{WatsonBHW99} (0.26)& \cellcolor{yellow!20}\href{../works/IklassovMR023.pdf}{IklassovMR023} (0.27)& \cellcolor{green!20}\href{../works/Tassel22.pdf}{Tassel22} (0.30)& \cellcolor{green!20}\href{../works/LauLN08.pdf}{LauLN08} (0.31)& \cellcolor{green!20}\href{../works/DoRZ08.pdf}{DoRZ08} (0.31)\\
Dot& \cellcolor{red!40}\href{../works/Astrand21.pdf}{Astrand21} (91.00)& \cellcolor{red!40}\href{../works/ZarandiASC20.pdf}{ZarandiASC20} (89.00)& \cellcolor{red!40}\href{../works/IsikYA23.pdf}{IsikYA23} (89.00)& \cellcolor{red!40}\href{../works/abs-2211-14492.pdf}{abs-2211-14492} (88.00)& \cellcolor{red!40}\href{../works/Groleaz21.pdf}{Groleaz21} (86.00)\\
Cosine& \cellcolor{red!40}\href{../works/WatsonBHW99.pdf}{WatsonBHW99} (0.75)& \cellcolor{red!40}\href{../works/IklassovMR023.pdf}{IklassovMR023} (0.74)& \cellcolor{red!40}\href{../works/TasselGS23.pdf}{TasselGS23} (0.67)& \cellcolor{red!40}\href{../works/abs-2306-05747.pdf}{abs-2306-05747} (0.67)& \cellcolor{red!40}\href{../works/LiFJZLL22.pdf}{LiFJZLL22} (0.67)\\
\index{LiW08}\href{../works/LiW08.pdf}{LiW08} R\&C& \cellcolor{red!20}\href{../works/Hooker05.pdf}{Hooker05} (0.88)& \cellcolor{red!20}\href{../works/Hooker05b.pdf}{Hooker05b} (0.88)& \cellcolor{red!20}AggounMV08 (0.89)& \cellcolor{red!20}\href{../works/Thorsteinsson01.pdf}{Thorsteinsson01} (0.90)& \cellcolor{red!20}\href{../works/AkkerDH07.pdf}{AkkerDH07} (0.90)\\
Euclid& \cellcolor{green!20}\href{../works/KhayatLR06.pdf}{KhayatLR06} (0.31)& \cellcolor{blue!20}\href{../works/ZeballosM09.pdf}{ZeballosM09} (0.32)& \cellcolor{blue!20}\href{../works/BeckPS03.pdf}{BeckPS03} (0.33)& \cellcolor{blue!20}\href{../works/QuirogaZH05.pdf}{QuirogaZH05} (0.34)& \cellcolor{black!20}\href{../works/ZeballosQH10.pdf}{ZeballosQH10} (0.34)\\
Dot& \cellcolor{red!40}\href{../works/Lombardi10.pdf}{Lombardi10} (163.00)& \cellcolor{red!40}\href{../works/ZarandiASC20.pdf}{ZarandiASC20} (161.00)& \cellcolor{red!40}\href{../works/Dejemeppe16.pdf}{Dejemeppe16} (157.00)& \cellcolor{red!40}\href{../works/Baptiste02.pdf}{Baptiste02} (156.00)& \cellcolor{red!40}\href{../works/Schutt11.pdf}{Schutt11} (147.00)\\
Cosine& \cellcolor{red!40}\href{../works/KhayatLR06.pdf}{KhayatLR06} (0.78)& \cellcolor{red!40}\href{../works/ZeballosM09.pdf}{ZeballosM09} (0.76)& \cellcolor{red!40}\href{../works/BeckPS03.pdf}{BeckPS03} (0.76)& \cellcolor{red!40}\href{../works/ZeballosQH10.pdf}{ZeballosQH10} (0.75)& \cellcolor{red!40}\href{../works/ZeballosNH11.pdf}{ZeballosNH11} (0.73)\\
\index{LiessM08}\href{../works/LiessM08.pdf}{LiessM08} R\&C& \cellcolor{red!40}\href{../works/DemasseyAM05.pdf}{DemasseyAM05} (0.77)& \cellcolor{red!40}NeronABCDD06 (0.79)& \cellcolor{red!40}\href{../works/BruckerK00.pdf}{BruckerK00} (0.80)& \cellcolor{red!40}\href{../works/ArkhipovBL19.pdf}{ArkhipovBL19} (0.84)& \cellcolor{red!20}\href{../works/KameugneFSN14.pdf}{KameugneFSN14} (0.88)\\
Euclid& \cellcolor{red!40}\href{../works/BofillCSV17a.pdf}{BofillCSV17a} (0.21)& \cellcolor{red!40}\href{../works/DemasseyAM05.pdf}{DemasseyAM05} (0.23)& \cellcolor{red!40}\href{../works/HeipckeCCS00.pdf}{HeipckeCCS00} (0.23)& \cellcolor{red!20}\href{../works/BenderWS21.pdf}{BenderWS21} (0.25)& \cellcolor{red!20}\href{../works/BofillCSV17.pdf}{BofillCSV17} (0.25)\\
Dot& \cellcolor{red!40}\href{../works/Schutt11.pdf}{Schutt11} (136.00)& \cellcolor{red!40}\href{../works/Baptiste02.pdf}{Baptiste02} (131.00)& \cellcolor{red!40}\href{../works/Lombardi10.pdf}{Lombardi10} (129.00)& \cellcolor{red!40}\href{../works/Godet21a.pdf}{Godet21a} (129.00)& \cellcolor{red!40}\href{../works/Dejemeppe16.pdf}{Dejemeppe16} (125.00)\\
Cosine& \cellcolor{red!40}\href{../works/DemasseyAM05.pdf}{DemasseyAM05} (0.86)& \cellcolor{red!40}\href{../works/BofillCSV17a.pdf}{BofillCSV17a} (0.85)& \cellcolor{red!40}\href{../works/HeipckeCCS00.pdf}{HeipckeCCS00} (0.83)& \cellcolor{red!40}\href{../works/VilimLS15.pdf}{VilimLS15} (0.83)& \cellcolor{red!40}\href{../works/BaptisteP97.pdf}{BaptisteP97} (0.82)\\
\index{LimAHO02a}\href{../works/LimAHO02a.pdf}{LimAHO02a} R\&C\\
Euclid& \cellcolor{red!40}\href{../works/CarchraeBF05.pdf}{CarchraeBF05} (0.16)& \cellcolor{red!40}\href{../works/AngelsmarkJ00.pdf}{AngelsmarkJ00} (0.17)& \cellcolor{red!40}\href{../works/Tsang03.pdf}{Tsang03} (0.18)& \cellcolor{red!40}\href{../works/Baptiste09.pdf}{Baptiste09} (0.18)& \cellcolor{red!40}\href{../works/FrostD98.pdf}{FrostD98} (0.18)\\
Dot& \cellcolor{red!40}\href{../works/Lemos21.pdf}{Lemos21} (38.00)& \cellcolor{red!40}\href{../works/ZarandiASC20.pdf}{ZarandiASC20} (37.00)& \cellcolor{red!40}\href{../works/Astrand21.pdf}{Astrand21} (37.00)& \cellcolor{red!40}\href{../works/KendallKRU10.pdf}{KendallKRU10} (34.00)& \cellcolor{red!40}\href{../works/Lombardi10.pdf}{Lombardi10} (32.00)\\
Cosine& \cellcolor{red!40}\href{../works/YoshikawaKNW94.pdf}{YoshikawaKNW94} (0.71)& \cellcolor{red!40}\href{../works/Schaerf97.pdf}{Schaerf97} (0.70)& \cellcolor{red!40}\href{../works/FeldmanG89.pdf}{FeldmanG89} (0.59)& \cellcolor{red!40}\href{../works/AngelsmarkJ00.pdf}{AngelsmarkJ00} (0.58)& \cellcolor{red!40}\href{../works/ShaikhK23.pdf}{ShaikhK23} (0.58)\\
\index{LimBTBB15}\href{../works/LimBTBB15.pdf}{LimBTBB15} R\&C& \cellcolor{red!40}\href{../works/LimHTB16.pdf}{LimHTB16} (0.78)& \cellcolor{red!40}\href{../works/GelainPRVW17.pdf}{GelainPRVW17} (0.80)& \cellcolor{red!40}\href{../works/LimBTBB15a.pdf}{LimBTBB15a} (0.86)& \cellcolor{red!20}\href{../works/IfrimOS12.pdf}{IfrimOS12} (0.90)& \cellcolor{yellow!20}\href{../works/DannaP03.pdf}{DannaP03} (0.90)\\
Euclid& \cellcolor{red!40}\href{../works/LimHTB16.pdf}{LimHTB16} (0.22)& \cellcolor{red!20}\href{../works/LimBTBB15a.pdf}{LimBTBB15a} (0.25)& \cellcolor{yellow!20}\href{../works/BofillCGGPSV23.pdf}{BofillCGGPSV23} (0.27)& \cellcolor{yellow!20}\href{../works/ZibranR11.pdf}{ZibranR11} (0.28)& \cellcolor{yellow!20}\href{../works/BofillGSV15.pdf}{BofillGSV15} (0.28)\\
Dot& \cellcolor{red!40}\href{../works/HarjunkoskiMBC14.pdf}{HarjunkoskiMBC14} (69.00)& \cellcolor{red!40}\href{../works/Dejemeppe16.pdf}{Dejemeppe16} (68.00)& \cellcolor{red!40}\href{../works/Groleaz21.pdf}{Groleaz21} (68.00)& \cellcolor{red!40}\href{../works/ZarandiASC20.pdf}{ZarandiASC20} (67.00)& \cellcolor{red!40}\href{../works/LaborieRSV18.pdf}{LaborieRSV18} (67.00)\\
Cosine& \cellcolor{red!40}\href{../works/LimHTB16.pdf}{LimHTB16} (0.77)& \cellcolor{red!40}\href{../works/LimBTBB15a.pdf}{LimBTBB15a} (0.69)& \cellcolor{red!40}\href{../works/CarchraeB09.pdf}{CarchraeB09} (0.59)& \cellcolor{red!40}\href{../works/PerronSF04.pdf}{PerronSF04} (0.58)& \cellcolor{red!40}\href{../works/BofillCGGPSV23.pdf}{BofillCGGPSV23} (0.58)\\
\index{LimBTBB15a}\href{../works/LimBTBB15a.pdf}{LimBTBB15a} R\&C& \cellcolor{red!40}\href{../works/LimBTBB15.pdf}{LimBTBB15} (0.86)& \cellcolor{red!20}\href{../works/IfrimOS12.pdf}{IfrimOS12} (0.89)& \cellcolor{green!20}\href{../works/RendlPHPR12.pdf}{RendlPHPR12} (0.94)& \cellcolor{green!20}\href{../works/DannaP03.pdf}{DannaP03} (0.96)\\
Euclid& \cellcolor{red!20}\href{../works/LimBTBB15.pdf}{LimBTBB15} (0.25)& \cellcolor{yellow!20}\href{../works/LimHTB16.pdf}{LimHTB16} (0.26)& \cellcolor{yellow!20}\href{../works/BofillCGGPSV23.pdf}{BofillCGGPSV23} (0.27)& \cellcolor{yellow!20}\href{../works/Baptiste09.pdf}{Baptiste09} (0.28)& \cellcolor{yellow!20}\href{../works/ChapadosJR11.pdf}{ChapadosJR11} (0.28)\\
Dot& \cellcolor{red!40}\href{../works/Lemos21.pdf}{Lemos21} (54.00)& \cellcolor{red!40}\href{../works/ZarandiASC20.pdf}{ZarandiASC20} (51.00)& \cellcolor{red!40}\href{../works/LimBTBB15.pdf}{LimBTBB15} (48.00)& \cellcolor{red!40}\href{../works/LimHTB16.pdf}{LimHTB16} (47.00)& \cellcolor{red!40}\href{../works/PrataAN23.pdf}{PrataAN23} (46.00)\\
Cosine& \cellcolor{red!40}\href{../works/LimBTBB15.pdf}{LimBTBB15} (0.69)& \cellcolor{red!40}\href{../works/LimHTB16.pdf}{LimHTB16} (0.66)& \cellcolor{red!40}\href{../works/BofillCGGPSV23.pdf}{BofillCGGPSV23} (0.53)& \cellcolor{red!40}\href{../works/PesantRR15.pdf}{PesantRR15} (0.52)& \cellcolor{red!40}\href{../works/ZhuS02.pdf}{ZhuS02} (0.50)\\
\index{LimHTB16}\href{../works/LimHTB16.pdf}{LimHTB16} R\&C& \cellcolor{red!40}\href{../works/LimBTBB15.pdf}{LimBTBB15} (0.78)& \cellcolor{green!20}\href{../works/MurphyMB15.pdf}{MurphyMB15} (0.95)& \cellcolor{green!20}\href{../works/DannaP03.pdf}{DannaP03} (0.96)& \cellcolor{blue!20}\href{../works/SchausHMCMD11.pdf}{SchausHMCMD11} (0.96)& \cellcolor{blue!20}\href{../works/GarganiR07.pdf}{GarganiR07} (0.96)\\
Euclid& \cellcolor{red!40}\href{../works/LimBTBB15.pdf}{LimBTBB15} (0.22)& \cellcolor{red!20}\href{../works/ZibranR11.pdf}{ZibranR11} (0.26)& \cellcolor{yellow!20}\href{../works/LimBTBB15a.pdf}{LimBTBB15a} (0.26)& \cellcolor{yellow!20}\href{../works/PesantRR15.pdf}{PesantRR15} (0.27)& \cellcolor{yellow!20}\href{../works/ZibranR11a.pdf}{ZibranR11a} (0.27)\\
Dot& \cellcolor{red!40}\href{../works/ZarandiASC20.pdf}{ZarandiASC20} (70.00)& \cellcolor{red!40}\href{../works/Lombardi10.pdf}{Lombardi10} (67.00)& \cellcolor{red!40}\href{../works/Astrand21.pdf}{Astrand21} (66.00)& \cellcolor{red!40}\href{../works/Beck99.pdf}{Beck99} (64.00)& \cellcolor{red!40}\href{../works/Dejemeppe16.pdf}{Dejemeppe16} (62.00)\\
Cosine& \cellcolor{red!40}\href{../works/LimBTBB15.pdf}{LimBTBB15} (0.77)& \cellcolor{red!40}\href{../works/LimBTBB15a.pdf}{LimBTBB15a} (0.66)& \cellcolor{red!40}\href{../works/GrimesIOS14.pdf}{GrimesIOS14} (0.63)& \cellcolor{red!40}\href{../works/ZibranR11a.pdf}{ZibranR11a} (0.62)& \cellcolor{red!40}\href{../works/ZibranR11.pdf}{ZibranR11} (0.62)\\
\index{LimRX04}\href{../works/LimRX04.pdf}{LimRX04} R\&C& \cellcolor{yellow!20}\href{../works/ZampelliVSDR13.pdf}{ZampelliVSDR13} (0.92)& \cellcolor{green!20}\href{../works/UnsalO13.pdf}{UnsalO13} (0.94)& \cellcolor{green!20}\href{../works/GuSW12.pdf}{GuSW12} (0.96)& \cellcolor{blue!20}\href{../works/ZarandiKS16.pdf}{ZarandiKS16} (0.97)& \cellcolor{blue!20}\href{../works/LombardiMB13.pdf}{LombardiMB13} (0.97)\\
Euclid& \cellcolor{red!40}\href{../works/HebrardTW05.pdf}{HebrardTW05} (0.22)& \cellcolor{red!40}\href{../works/KletzanderM17.pdf}{KletzanderM17} (0.23)& \cellcolor{red!40}\href{../works/Davis87.pdf}{Davis87} (0.23)& \cellcolor{red!40}\href{../works/AngelsmarkJ00.pdf}{AngelsmarkJ00} (0.24)& \cellcolor{red!40}\href{../works/CarchraeBF05.pdf}{CarchraeBF05} (0.24)\\
Dot& \cellcolor{red!40}\href{../works/ZarandiASC20.pdf}{ZarandiASC20} (65.00)& \cellcolor{red!40}\href{../works/Froger16.pdf}{Froger16} (63.00)& \cellcolor{red!40}\href{../works/SacramentoSP20.pdf}{SacramentoSP20} (62.00)& \cellcolor{red!40}\href{../works/Lunardi20.pdf}{Lunardi20} (59.00)& \cellcolor{red!40}\href{../works/Astrand21.pdf}{Astrand21} (58.00)\\
Cosine& \cellcolor{red!40}\href{../works/KletzanderM17.pdf}{KletzanderM17} (0.64)& \cellcolor{red!40}\href{../works/NattafDYW19.pdf}{NattafDYW19} (0.61)& \cellcolor{red!40}\href{../works/HebrardTW05.pdf}{HebrardTW05} (0.61)& \cellcolor{red!40}\href{../works/RendlPHPR12.pdf}{RendlPHPR12} (0.60)& \cellcolor{red!40}\href{../works/WatsonB08.pdf}{WatsonB08} (0.60)\\
\index{Limtanyakul07}\href{../works/Limtanyakul07.pdf}{Limtanyakul07} R\&C& \cellcolor{red!40}\href{../works/LimtanyakulS12.pdf}{LimtanyakulS12} (0.73)& \cellcolor{red!40}\href{../works/Davenport10.pdf}{Davenport10} (0.80)& \cellcolor{red!40}\href{../works/Vilim09.pdf}{Vilim09} (0.86)& \cellcolor{red!40}\href{../works/Vilim09a.pdf}{Vilim09a} (0.86)& \cellcolor{red!40}\href{../works/Beck10.pdf}{Beck10} (0.86)\\
Euclid& \cellcolor{red!40}\href{../works/BockmayrP06.pdf}{BockmayrP06} (0.21)& \cellcolor{red!40}\href{../works/Sadykov04.pdf}{Sadykov04} (0.22)& \cellcolor{red!40}\href{../works/HookerY02.pdf}{HookerY02} (0.22)& \cellcolor{red!40}\href{../works/MakMS10.pdf}{MakMS10} (0.23)& \cellcolor{red!40}\href{../works/Hooker17.pdf}{Hooker17} (0.24)\\
Dot& \cellcolor{red!40}\href{../works/Baptiste02.pdf}{Baptiste02} (86.00)& \cellcolor{red!40}\href{../works/BartakSR10.pdf}{BartakSR10} (83.00)& \cellcolor{red!40}\href{../works/Lombardi10.pdf}{Lombardi10} (81.00)& \cellcolor{red!40}\href{../works/Dejemeppe16.pdf}{Dejemeppe16} (81.00)& \cellcolor{red!40}\href{../works/Groleaz21.pdf}{Groleaz21} (81.00)\\
Cosine& \cellcolor{red!40}\href{../works/Colombani96.pdf}{Colombani96} (0.78)& \cellcolor{red!40}\href{../works/BockmayrP06.pdf}{BockmayrP06} (0.77)& \cellcolor{red!40}\href{../works/HeipckeCCS00.pdf}{HeipckeCCS00} (0.76)& \cellcolor{red!40}\href{../works/Sadykov04.pdf}{Sadykov04} (0.76)& \cellcolor{red!40}\href{../works/LimtanyakulS12.pdf}{LimtanyakulS12} (0.75)\\
\index{LimtanyakulS12}\href{../works/LimtanyakulS12.pdf}{LimtanyakulS12} R\&C& \cellcolor{red!40}\href{../works/Limtanyakul07.pdf}{Limtanyakul07} (0.73)& \cellcolor{red!40}\href{../works/QuirogaZH05.pdf}{QuirogaZH05} (0.79)& \cellcolor{red!40}\href{../works/ZhangLS12.pdf}{ZhangLS12} (0.80)& \cellcolor{red!40}\href{../works/KovacsV04.pdf}{KovacsV04} (0.82)& \cellcolor{red!40}\href{../works/Geske05.pdf}{Geske05} (0.83)\\
Euclid& \cellcolor{green!20}\href{../works/Limtanyakul07.pdf}{Limtanyakul07} (0.31)& \cellcolor{blue!20}\href{../works/Hooker06.pdf}{Hooker06} (0.33)& \cellcolor{blue!20}\href{../works/Laborie18a.pdf}{Laborie18a} (0.33)& \cellcolor{black!20}\href{../works/ChuX05.pdf}{ChuX05} (0.34)& \cellcolor{black!20}\href{../works/Hooker05a.pdf}{Hooker05a} (0.35)\\
Dot& \cellcolor{red!40}\href{../works/Lombardi10.pdf}{Lombardi10} (157.00)& \cellcolor{red!40}\href{../works/Baptiste02.pdf}{Baptiste02} (149.00)& \cellcolor{red!40}\href{../works/Dejemeppe16.pdf}{Dejemeppe16} (144.00)& \cellcolor{red!40}\href{../works/Groleaz21.pdf}{Groleaz21} (140.00)& \cellcolor{red!40}\href{../works/Malapert11.pdf}{Malapert11} (134.00)\\
Cosine& \cellcolor{red!40}\href{../works/Limtanyakul07.pdf}{Limtanyakul07} (0.75)& \cellcolor{red!40}\href{../works/Hooker06.pdf}{Hooker06} (0.73)& \cellcolor{red!40}\href{../works/Laborie18a.pdf}{Laborie18a} (0.70)& \cellcolor{red!40}\href{../works/MercierH07.pdf}{MercierH07} (0.69)& \cellcolor{red!40}\href{../works/Hooker07.pdf}{Hooker07} (0.69)\\
\index{LipovetzkyBPS14}\href{../works/LipovetzkyBPS14.pdf}{LipovetzkyBPS14} R\&C\\
Euclid& \cellcolor{red!40}\href{../works/NishikawaSTT18.pdf}{NishikawaSTT18} (0.24)& \cellcolor{red!20}\href{../works/NishikawaSTT18a.pdf}{NishikawaSTT18a} (0.24)& \cellcolor{yellow!20}\href{../works/BlomBPS14.pdf}{BlomBPS14} (0.27)& \cellcolor{yellow!20}\href{../works/BeniniBGM05a.pdf}{BeniniBGM05a} (0.27)& \cellcolor{yellow!20}\href{../works/NishikawaSTT19.pdf}{NishikawaSTT19} (0.27)\\
Dot& \cellcolor{red!40}\href{../works/Lunardi20.pdf}{Lunardi20} (92.00)& \cellcolor{red!40}\href{../works/LaborieRSV18.pdf}{LaborieRSV18} (92.00)& \cellcolor{red!40}\href{../works/Groleaz21.pdf}{Groleaz21} (89.00)& \cellcolor{red!40}\href{../works/Malapert11.pdf}{Malapert11} (87.00)& \cellcolor{red!40}\href{../works/Astrand21.pdf}{Astrand21} (86.00)\\
Cosine& \cellcolor{red!40}\href{../works/NishikawaSTT18.pdf}{NishikawaSTT18} (0.78)& \cellcolor{red!40}\href{../works/NishikawaSTT18a.pdf}{NishikawaSTT18a} (0.77)& \cellcolor{red!40}\href{../works/NishikawaSTT19.pdf}{NishikawaSTT19} (0.73)& \cellcolor{red!40}\href{../works/BurtLPS15.pdf}{BurtLPS15} (0.71)& \cellcolor{red!40}\href{../works/BlomBPS14.pdf}{BlomBPS14} (0.71)\\
\index{LiuCGM17}\href{../works/LiuCGM17.pdf}{LiuCGM17} R\&C& \cellcolor{green!20}\href{../works/Salido10.pdf}{Salido10} (0.94)& \cellcolor{green!20}\href{../works/BartakS11.pdf}{BartakS11} (0.94)& \cellcolor{green!20}\href{../works/Colombani96.pdf}{Colombani96} (0.95)& \cellcolor{green!20}\href{../works/Rodriguez07.pdf}{Rodriguez07} (0.95)& \cellcolor{green!20}\href{../works/NuijtenA96.pdf}{NuijtenA96} (0.95)\\
Euclid& \cellcolor{yellow!20}\href{../works/ZibranR11.pdf}{ZibranR11} (0.27)& \cellcolor{yellow!20}\href{../works/ZibranR11a.pdf}{ZibranR11a} (0.27)& \cellcolor{green!20}\href{../works/ZhangLS12.pdf}{ZhangLS12} (0.29)& \cellcolor{green!20}\href{../works/ChapadosJR11.pdf}{ChapadosJR11} (0.30)& \cellcolor{green!20}\href{../works/CohenHB17.pdf}{CohenHB17} (0.30)\\
Dot& \cellcolor{red!40}\href{../works/Godet21a.pdf}{Godet21a} (97.00)& \cellcolor{red!40}\href{../works/Dejemeppe16.pdf}{Dejemeppe16} (88.00)& \cellcolor{red!40}\href{../works/Beck99.pdf}{Beck99} (86.00)& \cellcolor{red!40}\href{../works/ColT22.pdf}{ColT22} (84.00)& \cellcolor{red!40}\href{../works/ZarandiASC20.pdf}{ZarandiASC20} (82.00)\\
Cosine& \cellcolor{red!40}\href{../works/ZibranR11a.pdf}{ZibranR11a} (0.69)& \cellcolor{red!40}\href{../works/ZibranR11.pdf}{ZibranR11} (0.69)& \cellcolor{red!40}\href{../works/ZhangLS12.pdf}{ZhangLS12} (0.65)& \cellcolor{red!40}\href{../works/FallahiAC20.pdf}{FallahiAC20} (0.64)& \cellcolor{red!40}\href{../works/ChapadosJR11.pdf}{ChapadosJR11} (0.62)\\
\index{LiuGT10}LiuGT10 R\&C& \cellcolor{red!40}BriandHHL08 (0.84)& \cellcolor{red!40}\href{../works/KameugneF13.pdf}{KameugneF13} (0.84)& \cellcolor{red!40}\href{../works/TanSD10.pdf}{TanSD10} (0.85)& \cellcolor{red!20}\href{../works/ArkhipovBL19.pdf}{ArkhipovBL19} (0.87)& \cellcolor{red!20}\href{../works/KameugneFSN11.pdf}{KameugneFSN11} (0.88)\\
Euclid\\
Dot\\
Cosine\\
\index{LiuJ06}\href{../works/LiuJ06.pdf}{LiuJ06} R\&C\\
Euclid& \cellcolor{red!40}\href{../works/CarchraeBF05.pdf}{CarchraeBF05} (0.13)& \cellcolor{red!40}\href{../works/AngelsmarkJ00.pdf}{AngelsmarkJ00} (0.14)& \cellcolor{red!40}\href{../works/Davis87.pdf}{Davis87} (0.15)& \cellcolor{red!40}\href{../works/Hunsberger08.pdf}{Hunsberger08} (0.16)& \cellcolor{red!40}\href{../works/Rit86.pdf}{Rit86} (0.17)\\
Dot& \cellcolor{red!40}\href{../works/Malapert11.pdf}{Malapert11} (37.00)& \cellcolor{red!40}\href{../works/Dejemeppe16.pdf}{Dejemeppe16} (36.00)& \cellcolor{red!40}\href{../works/Astrand21.pdf}{Astrand21} (35.00)& \cellcolor{red!40}\href{../works/Lombardi10.pdf}{Lombardi10} (34.00)& \cellcolor{red!40}\href{../works/Schutt11.pdf}{Schutt11} (34.00)\\
Cosine& \cellcolor{red!40}\href{../works/Rit86.pdf}{Rit86} (0.73)& \cellcolor{red!40}\href{../works/CarchraeBF05.pdf}{CarchraeBF05} (0.72)& \cellcolor{red!40}\href{../works/LudwigKRBMS14.pdf}{LudwigKRBMS14} (0.71)& \cellcolor{red!40}\href{../works/CarlssonKA99.pdf}{CarlssonKA99} (0.70)& \cellcolor{red!40}\href{../works/AngelsmarkJ00.pdf}{AngelsmarkJ00} (0.67)\\
\index{LiuLH18}\href{../works/LiuLH18.pdf}{LiuLH18} R\&C& \cellcolor{red!40}\href{../works/LiuLH19a.pdf}{LiuLH19a} (0.80)& \cellcolor{yellow!20}\href{../works/LiuLH19.pdf}{LiuLH19} (0.92)& \cellcolor{yellow!20}\href{../works/ZengM12.pdf}{ZengM12} (0.92)& \cellcolor{green!20}\href{../works/Balduccini11.pdf}{Balduccini11} (0.94)& \cellcolor{green!20}\href{../works/CarlssonJL17.pdf}{CarlssonJL17} (0.95)\\
Euclid& \cellcolor{red!20}\href{../works/ZengM12.pdf}{ZengM12} (0.26)& \cellcolor{red!20}\href{../works/EastonNT02.pdf}{EastonNT02} (0.26)& \cellcolor{yellow!20}\href{../works/ElfJR03.pdf}{ElfJR03} (0.27)& \cellcolor{yellow!20}\href{../works/Perron05.pdf}{Perron05} (0.28)& \cellcolor{yellow!20}\href{../works/NaqviAIAAA22.pdf}{NaqviAIAAA22} (0.28)\\
Dot& \cellcolor{red!40}\href{../works/KendallKRU10.pdf}{KendallKRU10} (77.00)& \cellcolor{red!40}\href{../works/CarlssonJL17.pdf}{CarlssonJL17} (72.00)& \cellcolor{red!40}\href{../works/ZarandiASC20.pdf}{ZarandiASC20} (65.00)& \cellcolor{red!40}\href{../works/Godet21a.pdf}{Godet21a} (63.00)& \cellcolor{red!40}\href{../works/ZengM12.pdf}{ZengM12} (62.00)\\
Cosine& \cellcolor{red!40}\href{../works/ZengM12.pdf}{ZengM12} (0.73)& \cellcolor{red!40}\href{../works/EastonNT02.pdf}{EastonNT02} (0.70)& \cellcolor{red!40}\href{../works/CarlssonJL17.pdf}{CarlssonJL17} (0.67)& \cellcolor{red!40}\href{../works/NaqviAIAAA22.pdf}{NaqviAIAAA22} (0.66)& \cellcolor{red!40}\href{../works/RasmussenT06.pdf}{RasmussenT06} (0.65)\\
\index{LiuLH19}\href{../works/LiuLH19.pdf}{LiuLH19} R\&C& \cellcolor{yellow!20}\href{../works/BandaSC11.pdf}{BandaSC11} (0.91)& \cellcolor{yellow!20}\href{../works/LiuLH18.pdf}{LiuLH18} (0.92)& \cellcolor{yellow!20}\href{../works/PoderBS04.pdf}{PoderBS04} (0.92)& \cellcolor{yellow!20}\href{../works/DincbasSH90.pdf}{DincbasSH90} (0.93)& \cellcolor{green!20}\href{../works/AggounB93.pdf}{AggounB93} (0.94)\\
Euclid& \cellcolor{red!40}\href{../works/GelainPRVW17.pdf}{GelainPRVW17} (0.20)& \cellcolor{red!40}\href{../works/BandaSC11.pdf}{BandaSC11} (0.21)& \cellcolor{red!40}\href{../works/ZhangLS12.pdf}{ZhangLS12} (0.23)& \cellcolor{red!20}\href{../works/KuchcinskiW03.pdf}{KuchcinskiW03} (0.25)& \cellcolor{red!20}\href{../works/HenzMT04.pdf}{HenzMT04} (0.25)\\
Dot& \cellcolor{red!40}\href{../works/Godet21a.pdf}{Godet21a} (79.00)& \cellcolor{red!40}\href{../works/Lemos21.pdf}{Lemos21} (75.00)& \cellcolor{red!40}\href{../works/Fahimi16.pdf}{Fahimi16} (75.00)& \cellcolor{red!40}\href{../works/Dejemeppe16.pdf}{Dejemeppe16} (73.00)& \cellcolor{red!40}\href{../works/German18.pdf}{German18} (72.00)\\
Cosine& \cellcolor{red!40}\href{../works/GelainPRVW17.pdf}{GelainPRVW17} (0.80)& \cellcolor{red!40}\href{../works/BandaSC11.pdf}{BandaSC11} (0.77)& \cellcolor{red!40}\href{../works/GarridoAO09.pdf}{GarridoAO09} (0.72)& \cellcolor{red!40}\href{../works/ZhangLS12.pdf}{ZhangLS12} (0.72)& \cellcolor{red!40}\href{../works/LiuLH19a.pdf}{LiuLH19a} (0.71)\\
\index{LiuLH19a}\href{../works/LiuLH19a.pdf}{LiuLH19a} R\&C& \cellcolor{red!40}\href{../works/LiuLH18.pdf}{LiuLH18} (0.80)& \cellcolor{blue!20}\href{../works/EastonNT02.pdf}{EastonNT02} (0.98)\\
Euclid& \cellcolor{red!20}\href{../works/ZhangLS12.pdf}{ZhangLS12} (0.26)& \cellcolor{red!20}\href{../works/LiuLH19.pdf}{LiuLH19} (0.26)& \cellcolor{yellow!20}\href{../works/BandaSC11.pdf}{BandaSC11} (0.27)& \cellcolor{yellow!20}\href{../works/GelainPRVW17.pdf}{GelainPRVW17} (0.28)& \cellcolor{green!20}\href{../works/SmithBHW96.pdf}{SmithBHW96} (0.29)\\
Dot& \cellcolor{red!40}\href{../works/Siala15a.pdf}{Siala15a} (81.00)& \cellcolor{red!40}\href{../works/Godet21a.pdf}{Godet21a} (79.00)& \cellcolor{red!40}\href{../works/Schutt11.pdf}{Schutt11} (73.00)& \cellcolor{red!40}\href{../works/Dejemeppe16.pdf}{Dejemeppe16} (73.00)& \cellcolor{red!40}\href{../works/Malapert11.pdf}{Malapert11} (72.00)\\
Cosine& \cellcolor{red!40}\href{../works/LiuLH19.pdf}{LiuLH19} (0.71)& \cellcolor{red!40}\href{../works/ZhangLS12.pdf}{ZhangLS12} (0.69)& \cellcolor{red!40}\href{../works/BandaSC11.pdf}{BandaSC11} (0.65)& \cellcolor{red!40}\href{../works/LiuLH18.pdf}{LiuLH18} (0.65)& \cellcolor{red!40}\href{../works/RussellU06.pdf}{RussellU06} (0.64)\\
\index{LiuW11}\href{../works/LiuW11.pdf}{LiuW11} R\&C& \cellcolor{yellow!20}\href{../works/NovasH14.pdf}{NovasH14} (0.92)& \cellcolor{yellow!20}\href{../works/TangLWSK18.pdf}{TangLWSK18} (0.93)& \cellcolor{green!20}\href{../works/Zeballos10.pdf}{Zeballos10} (0.94)& \cellcolor{green!20}\href{../works/LiessM08.pdf}{LiessM08} (0.95)& \cellcolor{green!20}\href{../works/ZarandiKS16.pdf}{ZarandiKS16} (0.96)\\
Euclid& \cellcolor{green!20}\href{../works/MakMS10.pdf}{MakMS10} (0.29)& \cellcolor{green!20}\href{../works/VanczaM01.pdf}{VanczaM01} (0.31)& \cellcolor{green!20}\href{../works/BockmayrP06.pdf}{BockmayrP06} (0.31)& \cellcolor{green!20}\href{../works/TrojetHL11.pdf}{TrojetHL11} (0.31)& \cellcolor{green!20}\href{../works/AkramNHRSA23.pdf}{AkramNHRSA23} (0.31)\\
Dot& \cellcolor{red!40}\href{../works/ZarandiASC20.pdf}{ZarandiASC20} (152.00)& \cellcolor{red!40}\href{../works/Groleaz21.pdf}{Groleaz21} (139.00)& \cellcolor{red!40}\href{../works/Lombardi10.pdf}{Lombardi10} (136.00)& \cellcolor{red!40}\href{../works/Astrand21.pdf}{Astrand21} (124.00)& \cellcolor{red!40}\href{../works/Dejemeppe16.pdf}{Dejemeppe16} (124.00)\\
Cosine& \cellcolor{red!40}\href{../works/TrojetHL11.pdf}{TrojetHL11} (0.74)& \cellcolor{red!40}\href{../works/MakMS10.pdf}{MakMS10} (0.73)& \cellcolor{red!40}\href{../works/PinarbasiAY19.pdf}{PinarbasiAY19} (0.71)& \cellcolor{red!40}\href{../works/LombardiMB13.pdf}{LombardiMB13} (0.70)& \cellcolor{red!40}\href{../works/VanczaM01.pdf}{VanczaM01} (0.70)\\
\index{Lombardi10}\href{../works/Lombardi10.pdf}{Lombardi10} R\&C\\
Euclid& \href{../works/LombardiM12.pdf}{LombardiM12} (0.50)& \href{../works/LombardiM10a.pdf}{LombardiM10a} (0.50)& \href{../works/LombardiMRB10.pdf}{LombardiMRB10} (0.53)& \href{../works/Beck99.pdf}{Beck99} (0.55)& \href{../works/LombardiMB13.pdf}{LombardiMB13} (0.55)\\
Dot& \cellcolor{red!40}\href{../works/Groleaz21.pdf}{Groleaz21} (289.00)& \cellcolor{red!40}\href{../works/Baptiste02.pdf}{Baptiste02} (284.00)& \cellcolor{red!40}\href{../works/ZarandiASC20.pdf}{ZarandiASC20} (279.00)& \cellcolor{red!40}\href{../works/Dejemeppe16.pdf}{Dejemeppe16} (279.00)& \cellcolor{red!40}\href{../works/Schutt11.pdf}{Schutt11} (261.00)\\
Cosine& \cellcolor{red!40}\href{../works/LombardiM10a.pdf}{LombardiM10a} (0.76)& \cellcolor{red!40}\href{../works/LombardiM12.pdf}{LombardiM12} (0.76)& \cellcolor{red!40}\href{../works/LombardiMRB10.pdf}{LombardiMRB10} (0.72)& \cellcolor{red!40}\href{../works/Schutt11.pdf}{Schutt11} (0.71)& \cellcolor{red!40}\href{../works/Beck99.pdf}{Beck99} (0.71)\\
\index{LombardiBM15}\href{../works/LombardiBM15.pdf}{LombardiBM15} R\&C& \cellcolor{red!40}\href{../works/KameugneF13.pdf}{KameugneF13} (0.83)& \cellcolor{red!20}\href{../works/KameugneFSN11.pdf}{KameugneFSN11} (0.88)& \cellcolor{red!20}\href{../works/KameugneFSN14.pdf}{KameugneFSN14} (0.89)& \cellcolor{red!20}\href{../works/LetortCB13.pdf}{LetortCB13} (0.89)& \cellcolor{red!20}\href{../works/LombardiM10.pdf}{LombardiM10} (0.89)\\
Euclid& \cellcolor{red!40}\href{../works/BonfiettiLM14.pdf}{BonfiettiLM14} (0.18)& \cellcolor{red!40}\href{../works/LombardiM12a.pdf}{LombardiM12a} (0.19)& \cellcolor{red!40}\href{../works/LombardiM09.pdf}{LombardiM09} (0.23)& \cellcolor{red!20}\href{../works/FortinZDF05.pdf}{FortinZDF05} (0.24)& \cellcolor{red!20}\href{../works/LombardiM10.pdf}{LombardiM10} (0.25)\\
Dot& \cellcolor{red!40}\href{../works/ZarandiASC20.pdf}{ZarandiASC20} (118.00)& \cellcolor{red!40}\href{../works/Lombardi10.pdf}{Lombardi10} (115.00)& \cellcolor{red!40}\href{../works/Astrand21.pdf}{Astrand21} (112.00)& \cellcolor{red!40}\href{../works/LaborieRSV18.pdf}{LaborieRSV18} (109.00)& \cellcolor{red!40}\href{../works/Godet21a.pdf}{Godet21a} (109.00)\\
Cosine& \cellcolor{red!40}\href{../works/BonfiettiLM14.pdf}{BonfiettiLM14} (0.89)& \cellcolor{red!40}\href{../works/LombardiM12a.pdf}{LombardiM12a} (0.88)& \cellcolor{red!40}\href{../works/LombardiM09.pdf}{LombardiM09} (0.82)& \cellcolor{red!40}\href{../works/LombardiM10.pdf}{LombardiM10} (0.79)& \cellcolor{red!40}\href{../works/BeckW07.pdf}{BeckW07} (0.78)\\
\index{LombardiBMB11}\href{../works/LombardiBMB11.pdf}{LombardiBMB11} R\&C& \cellcolor{red!40}\href{../works/BonfiettiLBM11.pdf}{BonfiettiLBM11} (0.60)& \cellcolor{red!40}\href{../works/LombardiM13.pdf}{LombardiM13} (0.85)& \cellcolor{red!40}\href{../works/LombardiM12a.pdf}{LombardiM12a} (0.85)& \cellcolor{red!20}\href{../works/BonfiettiLBM14.pdf}{BonfiettiLBM14} (0.89)& \cellcolor{yellow!20}\href{../works/BonfiettiLBM12.pdf}{BonfiettiLBM12} (0.92)\\
Euclid& \cellcolor{red!40}\href{../works/BonfiettiLBM11.pdf}{BonfiettiLBM11} (0.17)& \cellcolor{red!40}\href{../works/BonfiettiLBM12.pdf}{BonfiettiLBM12} (0.20)& \cellcolor{red!40}\href{../works/BonfiettiLM13.pdf}{BonfiettiLM13} (0.22)& \cellcolor{red!20}\href{../works/FortinZDF05.pdf}{FortinZDF05} (0.25)& \cellcolor{red!20}\href{../works/BonfiettiM12.pdf}{BonfiettiM12} (0.26)\\
Dot& \cellcolor{red!40}\href{../works/BonfiettiLBM14.pdf}{BonfiettiLBM14} (105.00)& \cellcolor{red!40}\href{../works/Lombardi10.pdf}{Lombardi10} (105.00)& \cellcolor{red!40}\href{../works/Astrand21.pdf}{Astrand21} (103.00)& \cellcolor{red!40}\href{../works/Groleaz21.pdf}{Groleaz21} (103.00)& \cellcolor{red!40}\href{../works/Schutt11.pdf}{Schutt11} (100.00)\\
Cosine& \cellcolor{red!40}\href{../works/BonfiettiLBM11.pdf}{BonfiettiLBM11} (0.90)& \cellcolor{red!40}\href{../works/BonfiettiLBM12.pdf}{BonfiettiLBM12} (0.86)& \cellcolor{red!40}\href{../works/BonfiettiLM13.pdf}{BonfiettiLM13} (0.84)& \cellcolor{red!40}\href{../works/BonfiettiLBM14.pdf}{BonfiettiLBM14} (0.82)& \cellcolor{red!40}\href{../works/BonfiettiZLM16.pdf}{BonfiettiZLM16} (0.77)\\
\index{LombardiM09}\href{../works/LombardiM09.pdf}{LombardiM09} R\&C& \cellcolor{red!40}\href{../works/LombardiM13.pdf}{LombardiM13} (0.80)& \cellcolor{red!40}\href{../works/LombardiM12a.pdf}{LombardiM12a} (0.80)& \cellcolor{red!40}\href{../works/LombardiM10.pdf}{LombardiM10} (0.83)& \cellcolor{red!40}\href{../works/CestaOS98.pdf}{CestaOS98} (0.83)& \cellcolor{red!40}\href{../works/Kumar03.pdf}{Kumar03} (0.84)\\
Euclid& \cellcolor{red!40}\href{../works/LombardiM10.pdf}{LombardiM10} (0.21)& \cellcolor{red!40}\href{../works/LombardiM13.pdf}{LombardiM13} (0.22)& \cellcolor{red!40}\href{../works/FortinZDF05.pdf}{FortinZDF05} (0.22)& \cellcolor{red!40}\href{../works/LombardiBM15.pdf}{LombardiBM15} (0.23)& \cellcolor{red!40}\href{../works/LombardiM12a.pdf}{LombardiM12a} (0.23)\\
Dot& \cellcolor{red!40}\href{../works/Lombardi10.pdf}{Lombardi10} (108.00)& \cellcolor{red!40}\href{../works/ZarandiASC20.pdf}{ZarandiASC20} (104.00)& \cellcolor{red!40}\href{../works/SubulanC22.pdf}{SubulanC22} (98.00)& \cellcolor{red!40}\href{../works/Schutt11.pdf}{Schutt11} (97.00)& \cellcolor{red!40}\href{../works/Dejemeppe16.pdf}{Dejemeppe16} (96.00)\\
Cosine& \cellcolor{red!40}\href{../works/LombardiM10.pdf}{LombardiM10} (0.85)& \cellcolor{red!40}\href{../works/LombardiM13.pdf}{LombardiM13} (0.82)& \cellcolor{red!40}\href{../works/LombardiBM15.pdf}{LombardiBM15} (0.82)& \cellcolor{red!40}\href{../works/FortinZDF05.pdf}{FortinZDF05} (0.81)& \cellcolor{red!40}\href{../works/LombardiM12a.pdf}{LombardiM12a} (0.81)\\
\index{LombardiM10}\href{../works/LombardiM10.pdf}{LombardiM10} R\&C& \cellcolor{red!40}\href{../works/LombardiMB13.pdf}{LombardiMB13} (0.80)& \cellcolor{red!40}\href{../works/LombardiM09.pdf}{LombardiM09} (0.83)& \cellcolor{red!20}\href{../works/LombardiM13.pdf}{LombardiM13} (0.88)& \cellcolor{red!20}\href{../works/LombardiM12a.pdf}{LombardiM12a} (0.88)& \cellcolor{red!20}\href{../works/AmadiniGM16.pdf}{AmadiniGM16} (0.89)\\
Euclid& \cellcolor{red!40}\href{../works/LombardiM09.pdf}{LombardiM09} (0.21)& \cellcolor{red!40}\href{../works/LombardiM13.pdf}{LombardiM13} (0.23)& \cellcolor{red!40}\href{../works/OddiRC10.pdf}{OddiRC10} (0.23)& \cellcolor{red!40}\href{../works/BhatnagarKL19.pdf}{BhatnagarKL19} (0.24)& \cellcolor{red!40}\href{../works/LombardiM12a.pdf}{LombardiM12a} (0.24)\\
Dot& \cellcolor{red!40}\href{../works/Lombardi10.pdf}{Lombardi10} (116.00)& \cellcolor{red!40}\href{../works/Baptiste02.pdf}{Baptiste02} (107.00)& \cellcolor{red!40}\href{../works/Godet21a.pdf}{Godet21a} (106.00)& \cellcolor{red!40}\href{../works/Dejemeppe16.pdf}{Dejemeppe16} (106.00)& \cellcolor{red!40}\href{../works/Groleaz21.pdf}{Groleaz21} (105.00)\\
Cosine& \cellcolor{red!40}\href{../works/LombardiM09.pdf}{LombardiM09} (0.85)& \cellcolor{red!40}\href{../works/LombardiM13.pdf}{LombardiM13} (0.82)& \cellcolor{red!40}\href{../works/CestaOF99.pdf}{CestaOF99} (0.81)& \cellcolor{red!40}\href{../works/OddiRC10.pdf}{OddiRC10} (0.80)& \cellcolor{red!40}\href{../works/LombardiM12a.pdf}{LombardiM12a} (0.80)\\
\index{LombardiM10a}\href{../works/LombardiM10a.pdf}{LombardiM10a} R\&C& \cellcolor{red!40}\href{../works/BartakCS10.pdf}{BartakCS10} (0.81)& \cellcolor{red!20}\href{../works/LombardiMRB10.pdf}{LombardiMRB10} (0.86)& \cellcolor{red!20}\href{../works/BeniniLMR11.pdf}{BeniniLMR11} (0.86)& \cellcolor{yellow!20}\href{../works/LombardiM09.pdf}{LombardiM09} (0.91)& \cellcolor{yellow!20}\href{../works/LombardiBM15.pdf}{LombardiBM15} (0.91)\\
Euclid& \cellcolor{blue!20}\href{../works/LombardiMRB10.pdf}{LombardiMRB10} (0.32)& \cellcolor{black!20}\href{../works/LombardiM10.pdf}{LombardiM10} (0.35)& \cellcolor{black!20}\href{../works/OzturkTHO12.pdf}{OzturkTHO12} (0.36)& \cellcolor{black!20}\href{../works/BartakSR08.pdf}{BartakSR08} (0.36)& \cellcolor{black!20}\href{../works/BeniniLMR11.pdf}{BeniniLMR11} (0.37)\\
Dot& \cellcolor{red!40}\href{../works/Lombardi10.pdf}{Lombardi10} (203.00)& \cellcolor{red!40}\href{../works/Groleaz21.pdf}{Groleaz21} (169.00)& \cellcolor{red!40}\href{../works/Baptiste02.pdf}{Baptiste02} (168.00)& \cellcolor{red!40}\href{../works/Dejemeppe16.pdf}{Dejemeppe16} (156.00)& \cellcolor{red!40}\href{../works/ZarandiASC20.pdf}{ZarandiASC20} (155.00)\\
Cosine& \cellcolor{red!40}\href{../works/LombardiMRB10.pdf}{LombardiMRB10} (0.80)& \cellcolor{red!40}\href{../works/Lombardi10.pdf}{Lombardi10} (0.76)& \cellcolor{red!40}\href{../works/BartakSR08.pdf}{BartakSR08} (0.73)& \cellcolor{red!40}\href{../works/LombardiM10.pdf}{LombardiM10} (0.73)& \cellcolor{red!40}\href{../works/OzturkTHO12.pdf}{OzturkTHO12} (0.71)\\
\index{LombardiM12}\href{../works/LombardiM12.pdf}{LombardiM12} R\&C& \cellcolor{red!40}\href{../works/CireCH13.pdf}{CireCH13} (0.86)& \cellcolor{red!40}\href{../works/CobanH11.pdf}{CobanH11} (0.86)& \cellcolor{red!20}\href{../works/CireCH16.pdf}{CireCH16} (0.87)& \cellcolor{red!20}\href{../works/Hooker07.pdf}{Hooker07} (0.89)& \cellcolor{yellow!20}\href{../works/Hooker05.pdf}{Hooker05} (0.90)\\
Euclid& \href{../works/BaptisteP97.pdf}{BaptisteP97} (0.41)& \href{../works/GokPTGO23.pdf}{GokPTGO23} (0.42)& \href{../works/Mercier-AubinGQ20.pdf}{Mercier-AubinGQ20} (0.42)& \href{../works/LiessM08.pdf}{LiessM08} (0.42)& \href{../works/DemasseyAM05.pdf}{DemasseyAM05} (0.43)\\
Dot& \cellcolor{red!40}\href{../works/Lombardi10.pdf}{Lombardi10} (240.00)& \cellcolor{red!40}\href{../works/Groleaz21.pdf}{Groleaz21} (233.00)& \cellcolor{red!40}\href{../works/ZarandiASC20.pdf}{ZarandiASC20} (229.00)& \cellcolor{red!40}\href{../works/Dejemeppe16.pdf}{Dejemeppe16} (225.00)& \cellcolor{red!40}\href{../works/Baptiste02.pdf}{Baptiste02} (222.00)\\
Cosine& \cellcolor{red!40}\href{../works/Lombardi10.pdf}{Lombardi10} (0.76)& \cellcolor{red!40}\href{../works/BaptisteP97.pdf}{BaptisteP97} (0.73)& \cellcolor{red!40}\href{../works/GokPTGO23.pdf}{GokPTGO23} (0.73)& \cellcolor{red!40}\href{../works/Mercier-AubinGQ20.pdf}{Mercier-AubinGQ20} (0.72)& \cellcolor{red!40}\href{../works/LiessM08.pdf}{LiessM08} (0.71)\\
\index{LombardiM12a}\href{../works/LombardiM12a.pdf}{LombardiM12a} R\&C& \cellcolor{red!40}\href{../works/LombardiM13.pdf}{LombardiM13} (0.00)& \cellcolor{red!40}\href{../works/LombardiM09.pdf}{LombardiM09} (0.80)& \cellcolor{red!40}\href{../works/SchuttFSW13.pdf}{SchuttFSW13} (0.85)& \cellcolor{red!40}\href{../works/LombardiBMB11.pdf}{LombardiBMB11} (0.85)& \cellcolor{red!40}\href{../works/LombardiMB13.pdf}{LombardiMB13} (0.86)\\
Euclid& \cellcolor{red!40}\href{../works/BofillCSV17.pdf}{BofillCSV17} (0.19)& \cellcolor{red!40}\href{../works/OddiRC10.pdf}{OddiRC10} (0.19)& \cellcolor{red!40}\href{../works/LombardiBM15.pdf}{LombardiBM15} (0.19)& \cellcolor{red!40}\href{../works/BofillCSV17a.pdf}{BofillCSV17a} (0.20)& \cellcolor{red!40}\href{../works/LombardiM13.pdf}{LombardiM13} (0.21)\\
Dot& \cellcolor{red!40}\href{../works/Schutt11.pdf}{Schutt11} (101.00)& \cellcolor{red!40}\href{../works/Lombardi10.pdf}{Lombardi10} (100.00)& \cellcolor{red!40}\href{../works/Caballero19.pdf}{Caballero19} (97.00)& \cellcolor{red!40}\href{../works/Godet21a.pdf}{Godet21a} (96.00)& \cellcolor{red!40}\href{../works/ZarandiASC20.pdf}{ZarandiASC20} (94.00)\\
Cosine& \cellcolor{red!40}\href{../works/LombardiBM15.pdf}{LombardiBM15} (0.88)& \cellcolor{red!40}\href{../works/BofillCSV17.pdf}{BofillCSV17} (0.87)& \cellcolor{red!40}\href{../works/OddiRC10.pdf}{OddiRC10} (0.86)& \cellcolor{red!40}\href{../works/BofillCSV17a.pdf}{BofillCSV17a} (0.85)& \cellcolor{red!40}\href{../works/LombardiM13.pdf}{LombardiM13} (0.84)\\
\index{LombardiM13}\href{../works/LombardiM13.pdf}{LombardiM13} R\&C& \cellcolor{red!40}\href{../works/LombardiM12a.pdf}{LombardiM12a} (0.00)& \cellcolor{red!40}\href{../works/LombardiM09.pdf}{LombardiM09} (0.80)& \cellcolor{red!40}\href{../works/SchuttFSW13.pdf}{SchuttFSW13} (0.85)& \cellcolor{red!40}\href{../works/LombardiBMB11.pdf}{LombardiBMB11} (0.85)& \cellcolor{red!40}\href{../works/LombardiMB13.pdf}{LombardiMB13} (0.86)\\
Euclid& \cellcolor{red!40}\href{../works/OddiRC10.pdf}{OddiRC10} (0.16)& \cellcolor{red!40}\href{../works/Caballero23.pdf}{Caballero23} (0.17)& \cellcolor{red!40}\href{../works/BhatnagarKL19.pdf}{BhatnagarKL19} (0.18)& \cellcolor{red!40}\href{../works/FortinZDF05.pdf}{FortinZDF05} (0.18)& \cellcolor{red!40}\href{../works/BonfiettiM12.pdf}{BonfiettiM12} (0.18)\\
Dot& \cellcolor{red!40}\href{../works/Caballero19.pdf}{Caballero19} (65.00)& \cellcolor{red!40}\href{../works/Schutt11.pdf}{Schutt11} (64.00)& \cellcolor{red!40}\href{../works/Godet21a.pdf}{Godet21a} (64.00)& \cellcolor{red!40}\href{../works/ZarandiASC20.pdf}{ZarandiASC20} (63.00)& \cellcolor{red!40}\href{../works/Lombardi10.pdf}{Lombardi10} (63.00)\\
Cosine& \cellcolor{red!40}\href{../works/OddiRC10.pdf}{OddiRC10} (0.88)& \cellcolor{red!40}\href{../works/BofillCSV17a.pdf}{BofillCSV17a} (0.85)& \cellcolor{red!40}\href{../works/LombardiM12a.pdf}{LombardiM12a} (0.84)& \cellcolor{red!40}\href{../works/BofillCSV17.pdf}{BofillCSV17} (0.84)& \cellcolor{red!40}\href{../works/BhatnagarKL19.pdf}{BhatnagarKL19} (0.83)\\
\index{LombardiMB13}\href{../works/LombardiMB13.pdf}{LombardiMB13} R\&C& \cellcolor{red!40}\href{../works/LombardiM10.pdf}{LombardiM10} (0.80)& \cellcolor{red!40}\href{../works/LombardiM09.pdf}{LombardiM09} (0.85)& \cellcolor{red!40}\href{../works/LombardiM13.pdf}{LombardiM13} (0.86)& \cellcolor{red!40}\href{../works/LombardiM12a.pdf}{LombardiM12a} (0.86)& \cellcolor{red!20}CestaOPS14 (0.88)\\
Euclid& \cellcolor{green!20}\href{../works/LombardiM09.pdf}{LombardiM09} (0.29)& \cellcolor{green!20}\href{../works/LombardiMRB10.pdf}{LombardiMRB10} (0.30)& \cellcolor{green!20}\href{../works/LombardiM10.pdf}{LombardiM10} (0.30)& \cellcolor{green!20}\href{../works/NishikawaSTT19.pdf}{NishikawaSTT19} (0.31)& \cellcolor{blue!20}\href{../works/Vilim09a.pdf}{Vilim09a} (0.32)\\
Dot& \cellcolor{red!40}\href{../works/Lombardi10.pdf}{Lombardi10} (166.00)& \cellcolor{red!40}\href{../works/Groleaz21.pdf}{Groleaz21} (150.00)& \cellcolor{red!40}\href{../works/Dejemeppe16.pdf}{Dejemeppe16} (147.00)& \cellcolor{red!40}\href{../works/ZarandiASC20.pdf}{ZarandiASC20} (144.00)& \cellcolor{red!40}\href{../works/Schutt11.pdf}{Schutt11} (137.00)\\
Cosine& \cellcolor{red!40}\href{../works/LombardiMRB10.pdf}{LombardiMRB10} (0.80)& \cellcolor{red!40}\href{../works/LombardiM09.pdf}{LombardiM09} (0.77)& \cellcolor{red!40}\href{../works/LombardiM10.pdf}{LombardiM10} (0.75)& \cellcolor{red!40}\href{../works/NishikawaSTT19.pdf}{NishikawaSTT19} (0.74)& \cellcolor{red!40}\href{../works/Vilim09a.pdf}{Vilim09a} (0.72)\\
\index{LombardiMRB10}\href{../works/LombardiMRB10.pdf}{LombardiMRB10} R\&C& \cellcolor{red!40}\href{../works/BeniniBGM05.pdf}{BeniniBGM05} (0.78)& \cellcolor{red!40}\href{../works/BeniniLMR08.pdf}{BeniniLMR08} (0.80)& \cellcolor{red!40}\href{../works/BeniniLMMR08.pdf}{BeniniLMMR08} (0.81)& \cellcolor{red!40}\href{../works/CireCH13.pdf}{CireCH13} (0.82)& \cellcolor{red!40}\href{../works/BeniniLMR11.pdf}{BeniniLMR11} (0.82)\\
Euclid& \cellcolor{green!20}\href{../works/BeniniBGM06.pdf}{BeniniBGM06} (0.30)& \cellcolor{green!20}\href{../works/LombardiMB13.pdf}{LombardiMB13} (0.30)& \cellcolor{green!20}\href{../works/BeniniLMR08.pdf}{BeniniLMR08} (0.31)& \cellcolor{blue!20}\href{../works/LombardiM10a.pdf}{LombardiM10a} (0.32)& \cellcolor{blue!20}\href{../works/BeniniBGM05.pdf}{BeniniBGM05} (0.32)\\
Dot& \cellcolor{red!40}\href{../works/Lombardi10.pdf}{Lombardi10} (187.00)& \cellcolor{red!40}\href{../works/ZarandiASC20.pdf}{ZarandiASC20} (152.00)& \cellcolor{red!40}\href{../works/Groleaz21.pdf}{Groleaz21} (149.00)& \cellcolor{red!40}\href{../works/Baptiste02.pdf}{Baptiste02} (146.00)& \cellcolor{red!40}\href{../works/Astrand21.pdf}{Astrand21} (145.00)\\
Cosine& \cellcolor{red!40}\href{../works/LombardiMB13.pdf}{LombardiMB13} (0.80)& \cellcolor{red!40}\href{../works/LombardiM10a.pdf}{LombardiM10a} (0.80)& \cellcolor{red!40}\href{../works/BeniniBGM06.pdf}{BeniniBGM06} (0.80)& \cellcolor{red!40}\href{../works/BeniniLMR08.pdf}{BeniniLMR08} (0.78)& \cellcolor{red!40}\href{../works/BeniniBGM05.pdf}{BeniniBGM05} (0.77)\\
\index{LopesCSM10}\href{../works/LopesCSM10.pdf}{LopesCSM10} R\&C& \cellcolor{red!40}\href{../works/MouraSCL08.pdf}{MouraSCL08} (0.52)& \cellcolor{red!40}\href{../works/MouraSCL08a.pdf}{MouraSCL08a} (0.60)& \cellcolor{yellow!20}MagataoAN05 (0.92)& \cellcolor{yellow!20}FelizariAL09 (0.92)& \cellcolor{blue!20}Henz01 (0.96)\\
Euclid& \cellcolor{red!20}\href{../works/MouraSCL08.pdf}{MouraSCL08} (0.24)& \cellcolor{yellow!20}\href{../works/MouraSCL08a.pdf}{MouraSCL08a} (0.26)& \cellcolor{blue!20}\href{../works/GilesH16.pdf}{GilesH16} (0.33)& \cellcolor{black!20}\href{../works/BeniniBGM05a.pdf}{BeniniBGM05a} (0.34)& \cellcolor{black!20}\href{../works/PesantGPR99.pdf}{PesantGPR99} (0.35)\\
Dot& \cellcolor{red!40}\href{../works/HarjunkoskiMBC14.pdf}{HarjunkoskiMBC14} (125.00)& \cellcolor{red!40}\href{../works/Beck99.pdf}{Beck99} (123.00)& \cellcolor{red!40}\href{../works/ZarandiASC20.pdf}{ZarandiASC20} (120.00)& \cellcolor{red!40}\href{../works/Astrand21.pdf}{Astrand21} (120.00)& \cellcolor{red!40}\href{../works/Lombardi10.pdf}{Lombardi10} (119.00)\\
Cosine& \cellcolor{red!40}\href{../works/MouraSCL08.pdf}{MouraSCL08} (0.83)& \cellcolor{red!40}\href{../works/MouraSCL08a.pdf}{MouraSCL08a} (0.80)& \cellcolor{red!40}\href{../works/GilesH16.pdf}{GilesH16} (0.68)& \cellcolor{red!40}\href{../works/GoelSHFS15.pdf}{GoelSHFS15} (0.65)& \cellcolor{red!40}\href{../works/ZeballosM09.pdf}{ZeballosM09} (0.65)\\
\index{LopezAKYG00}\href{../works/LopezAKYG00.pdf}{LopezAKYG00} R\&C\\
Euclid\\
Dot\\
Cosine\\
\index{LorigeonBB02}\href{../works/LorigeonBB02.pdf}{LorigeonBB02} R\&C& \cellcolor{green!20}\href{../works/NuijtenA96.pdf}{NuijtenA96} (0.95)& \cellcolor{blue!20}\href{../works/KorbaaYG00.pdf}{KorbaaYG00} (0.97)& \cellcolor{blue!20}\href{../works/ArtiguesF07.pdf}{ArtiguesF07} (0.97)& \cellcolor{blue!20}\href{../works/MalapertCGJLR12.pdf}{MalapertCGJLR12} (0.98)& \cellcolor{blue!20}\href{../works/MejiaY20.pdf}{MejiaY20} (0.98)\\
Euclid& \cellcolor{green!20}\href{../works/JuvinHL23.pdf}{JuvinHL23} (0.30)& \cellcolor{green!20}\href{../works/BillautHL12.pdf}{BillautHL12} (0.30)& \cellcolor{blue!20}\href{../works/QinWSLS21.pdf}{QinWSLS21} (0.32)& \cellcolor{blue!20}\href{../works/ParkUJR19.pdf}{ParkUJR19} (0.32)& \cellcolor{blue!20}\href{../works/HebrardHJMPV16.pdf}{HebrardHJMPV16} (0.33)\\
Dot& \cellcolor{red!40}\href{../works/Lunardi20.pdf}{Lunardi20} (117.00)& \cellcolor{red!40}\href{../works/Groleaz21.pdf}{Groleaz21} (113.00)& \cellcolor{red!40}\href{../works/Malapert11.pdf}{Malapert11} (110.00)& \cellcolor{red!40}\href{../works/AbreuNP23.pdf}{AbreuNP23} (109.00)& \cellcolor{red!40}\href{../works/Baptiste02.pdf}{Baptiste02} (108.00)\\
Cosine& \cellcolor{red!40}\href{../works/JuvinHL23.pdf}{JuvinHL23} (0.71)& \cellcolor{red!40}\href{../works/QinWSLS21.pdf}{QinWSLS21} (0.70)& \cellcolor{red!40}\href{../works/BillautHL12.pdf}{BillautHL12} (0.70)& \cellcolor{red!40}\href{../works/ParkUJR19.pdf}{ParkUJR19} (0.69)& \cellcolor{red!40}\href{../works/AbreuNP23.pdf}{AbreuNP23} (0.68)\\
\index{LouieVNB14}\href{../works/LouieVNB14.pdf}{LouieVNB14} R\&C& \cellcolor{yellow!20}\href{../works/Kumar03.pdf}{Kumar03} (0.91)& \cellcolor{yellow!20}\href{../works/BoothTNB16.pdf}{BoothTNB16} (0.91)& \cellcolor{green!20}\href{../works/ReddyFIBKAJ11.pdf}{ReddyFIBKAJ11} (0.94)& \cellcolor{green!20}\href{../works/Laborie03.pdf}{Laborie03} (0.95)& \cellcolor{green!20}\href{../works/BidotVLB09.pdf}{BidotVLB09} (0.95)\\
Euclid& \cellcolor{red!40}\href{../works/FukunagaHFAMN02.pdf}{FukunagaHFAMN02} (0.19)& \cellcolor{red!40}\href{../works/Davis87.pdf}{Davis87} (0.23)& \cellcolor{red!40}\href{../works/TranWDRFOVB16.pdf}{TranWDRFOVB16} (0.23)& \cellcolor{red!40}\href{../works/WallaceF00.pdf}{WallaceF00} (0.23)& \cellcolor{red!40}\href{../works/ChapadosJR11.pdf}{ChapadosJR11} (0.23)\\
Dot& \cellcolor{red!40}\href{../works/Dejemeppe16.pdf}{Dejemeppe16} (66.00)& \cellcolor{red!40}\href{../works/LaborieRSV18.pdf}{LaborieRSV18} (65.00)& \cellcolor{red!40}\href{../works/ZarandiASC20.pdf}{ZarandiASC20} (63.00)& \cellcolor{red!40}\href{../works/Astrand21.pdf}{Astrand21} (62.00)& \cellcolor{red!40}\href{../works/ZeballosCM10.pdf}{ZeballosCM10} (61.00)\\
Cosine& \cellcolor{red!40}\href{../works/FukunagaHFAMN02.pdf}{FukunagaHFAMN02} (0.74)& \cellcolor{red!40}\href{../works/TranWDRFOVB16.pdf}{TranWDRFOVB16} (0.70)& \cellcolor{red!40}\href{../works/NovasH14.pdf}{NovasH14} (0.69)& \cellcolor{red!40}\href{../works/NovasH12.pdf}{NovasH12} (0.66)& \cellcolor{red!40}\href{../works/AstrandJZ18.pdf}{AstrandJZ18} (0.66)\\
\index{LozanoCDS12}\href{../works/LozanoCDS12.pdf}{LozanoCDS12} R\&C& \cellcolor{green!20}\href{../works/MalikMB08.pdf}{MalikMB08} (0.95)& \cellcolor{green!20}\href{../works/SimoninAHL12.pdf}{SimoninAHL12} (0.95)& \cellcolor{green!20}\href{../works/BeldiceanuC02.pdf}{BeldiceanuC02} (0.95)& \cellcolor{green!20}\href{../works/Kuchcinski03.pdf}{Kuchcinski03} (0.95)& \cellcolor{green!20}\href{../works/LetortCB15.pdf}{LetortCB15} (0.95)\\
Euclid& \cellcolor{red!40}\href{../works/BeniniBGM05a.pdf}{BeniniBGM05a} (0.23)& \cellcolor{red!40}\href{../works/KuchcinskiW03.pdf}{KuchcinskiW03} (0.23)& \cellcolor{red!40}\href{../works/BockmayrP06.pdf}{BockmayrP06} (0.23)& \cellcolor{red!40}\href{../works/AlakaPY19.pdf}{AlakaPY19} (0.23)& \cellcolor{red!40}\href{../works/MalikMB08.pdf}{MalikMB08} (0.23)\\
Dot& \cellcolor{red!40}\href{../works/Malapert11.pdf}{Malapert11} (85.00)& \cellcolor{red!40}\href{../works/Dejemeppe16.pdf}{Dejemeppe16} (82.00)& \cellcolor{red!40}\href{../works/Groleaz21.pdf}{Groleaz21} (81.00)& \cellcolor{red!40}\href{../works/Lombardi10.pdf}{Lombardi10} (80.00)& \cellcolor{red!40}\href{../works/HarjunkoskiMBC14.pdf}{HarjunkoskiMBC14} (80.00)\\
Cosine& \cellcolor{red!40}\href{../works/BeniniBGM06.pdf}{BeniniBGM06} (0.77)& \cellcolor{red!40}\href{../works/AlakaPY19.pdf}{AlakaPY19} (0.75)& \cellcolor{red!40}\href{../works/BockmayrP06.pdf}{BockmayrP06} (0.74)& \cellcolor{red!40}\href{../works/KuchcinskiW03.pdf}{KuchcinskiW03} (0.74)& \cellcolor{red!40}\href{../works/Alaka21.pdf}{Alaka21} (0.74)\\
\index{LuZZYW24}\href{../works/LuZZYW24.pdf}{LuZZYW24} R\&C\\
Euclid& \href{../works/AbidinK20.pdf}{AbidinK20} (0.41)& \href{../works/CilKLO22.pdf}{CilKLO22} (0.42)& \href{../works/PinarbasiAY19.pdf}{PinarbasiAY19} (0.43)& \href{../works/Alaka21.pdf}{Alaka21} (0.43)& \href{../works/ZouZ20.pdf}{ZouZ20} (0.44)\\
Dot& \cellcolor{red!40}\href{../works/ZarandiASC20.pdf}{ZarandiASC20} (196.00)& \cellcolor{red!40}\href{../works/Froger16.pdf}{Froger16} (180.00)& \cellcolor{red!40}\href{../works/Astrand21.pdf}{Astrand21} (169.00)& \cellcolor{red!40}\href{../works/Lunardi20.pdf}{Lunardi20} (163.00)& \cellcolor{red!40}\href{../works/Groleaz21.pdf}{Groleaz21} (162.00)\\
Cosine& \cellcolor{red!40}\href{../works/CilKLO22.pdf}{CilKLO22} (0.68)& \cellcolor{red!40}\href{../works/AbidinK20.pdf}{AbidinK20} (0.66)& \cellcolor{red!40}\href{../works/Froger16.pdf}{Froger16} (0.66)& \cellcolor{red!40}\href{../works/SacramentoSP20.pdf}{SacramentoSP20} (0.66)& \cellcolor{red!40}\href{../works/PinarbasiAY19.pdf}{PinarbasiAY19} (0.64)\\
\index{LudwigKRBMS14}\href{../works/LudwigKRBMS14.pdf}{LudwigKRBMS14} R\&C\\
Euclid& \cellcolor{red!40}\href{../works/FukunagaHFAMN02.pdf}{FukunagaHFAMN02} (0.17)& \cellcolor{red!40}\href{../works/AngelsmarkJ00.pdf}{AngelsmarkJ00} (0.18)& \cellcolor{red!40}\href{../works/LiuJ06.pdf}{LiuJ06} (0.18)& \cellcolor{red!40}\href{../works/Rit86.pdf}{Rit86} (0.18)& \cellcolor{red!40}\href{../works/BarlattCG08.pdf}{BarlattCG08} (0.20)\\
Dot& \cellcolor{red!40}\href{../works/Astrand21.pdf}{Astrand21} (52.00)& \cellcolor{red!40}\href{../works/Lunardi20.pdf}{Lunardi20} (50.00)& \cellcolor{red!40}\href{../works/Lombardi10.pdf}{Lombardi10} (50.00)& \cellcolor{red!40}\href{../works/ZarandiASC20.pdf}{ZarandiASC20} (49.00)& \cellcolor{red!40}\href{../works/Malapert11.pdf}{Malapert11} (49.00)\\
Cosine& \cellcolor{red!40}\href{../works/FukunagaHFAMN02.pdf}{FukunagaHFAMN02} (0.75)& \cellcolor{red!40}\href{../works/SimoninAHL15.pdf}{SimoninAHL15} (0.73)& \cellcolor{red!40}\href{../works/Rit86.pdf}{Rit86} (0.73)& \cellcolor{red!40}\href{../works/SimoninAHL12.pdf}{SimoninAHL12} (0.73)& \cellcolor{red!40}\href{../works/AngelsmarkJ00.pdf}{AngelsmarkJ00} (0.72)\\
\index{Lunardi20}\href{../works/Lunardi20.pdf}{Lunardi20} R\&C\\
Euclid& \href{../works/YunusogluY22.pdf}{YunusogluY22} (0.50)& \href{../works/AfsarVPG23.pdf}{AfsarVPG23} (0.51)& \href{../works/LunardiBLRV20.pdf}{LunardiBLRV20} (0.51)& \href{../works/MengZRZL20.pdf}{MengZRZL20} (0.51)& \href{../works/OujanaAYB22.pdf}{OujanaAYB22} (0.51)\\
Dot& \cellcolor{red!40}\href{../works/ZarandiASC20.pdf}{ZarandiASC20} (322.00)& \cellcolor{red!40}\href{../works/Groleaz21.pdf}{Groleaz21} (293.00)& \cellcolor{red!40}\href{../works/Dejemeppe16.pdf}{Dejemeppe16} (260.00)& \cellcolor{red!40}\href{../works/Astrand21.pdf}{Astrand21} (258.00)& \cellcolor{red!40}\href{../works/IsikYA23.pdf}{IsikYA23} (246.00)\\
Cosine& \cellcolor{red!40}\href{../works/IsikYA23.pdf}{IsikYA23} (0.74)& \cellcolor{red!40}\href{../works/YunusogluY22.pdf}{YunusogluY22} (0.73)& \cellcolor{red!40}\href{../works/MengZRZL20.pdf}{MengZRZL20} (0.73)& \cellcolor{red!40}\href{../works/AfsarVPG23.pdf}{AfsarVPG23} (0.73)& \cellcolor{red!40}\href{../works/LunardiBLRV20.pdf}{LunardiBLRV20} (0.72)\\
\index{LunardiBLRV20}\href{../works/LunardiBLRV20.pdf}{LunardiBLRV20} R\&C& \cellcolor{red!20}\href{../works/HamC16.pdf}{HamC16} (0.88)& \cellcolor{red!20}NaderiR22 (0.88)& \cellcolor{red!20}\href{../works/HamP21.pdf}{HamP21} (0.89)& \cellcolor{red!20}\href{../works/MengZRZL20.pdf}{MengZRZL20} (0.89)& \cellcolor{red!20}\href{../works/HeinzNVH22.pdf}{HeinzNVH22} (0.89)\\
Euclid& \cellcolor{black!20}\href{../works/MurinR19.pdf}{MurinR19} (0.35)& \cellcolor{black!20}\href{../works/ZhangW18.pdf}{ZhangW18} (0.36)& \cellcolor{black!20}\href{../works/JuvinHL23.pdf}{JuvinHL23} (0.36)& \cellcolor{black!20}\href{../works/CauwelaertDMS16.pdf}{CauwelaertDMS16} (0.36)& \cellcolor{black!20}\href{../works/BillautHL12.pdf}{BillautHL12} (0.37)\\
Dot& \cellcolor{red!40}\href{../works/Lunardi20.pdf}{Lunardi20} (184.00)& \cellcolor{red!40}\href{../works/Groleaz21.pdf}{Groleaz21} (155.00)& \cellcolor{red!40}\href{../works/NaderiRR23.pdf}{NaderiRR23} (150.00)& \cellcolor{red!40}\href{../works/ZarandiASC20.pdf}{ZarandiASC20} (148.00)& \cellcolor{red!40}\href{../works/Astrand21.pdf}{Astrand21} (147.00)\\
Cosine& \cellcolor{red!40}\href{../works/ZhangW18.pdf}{ZhangW18} (0.76)& \cellcolor{red!40}\href{../works/MurinR19.pdf}{MurinR19} (0.74)& \cellcolor{red!40}\href{../works/HamPK21.pdf}{HamPK21} (0.73)& \cellcolor{red!40}\href{../works/Lunardi20.pdf}{Lunardi20} (0.72)& \cellcolor{red!40}\href{../works/Novas19.pdf}{Novas19} (0.71)\\
\index{LuoB22}\href{../works/LuoB22.pdf}{LuoB22} R\&C& \cellcolor{yellow!20}\href{../works/LaborieRSV18.pdf}{LaborieRSV18} (0.90)& \cellcolor{yellow!20}\href{../works/ColT19.pdf}{ColT19} (0.93)& \cellcolor{green!20}\href{../works/BeldiceanuCDP11.pdf}{BeldiceanuCDP11} (0.94)& \cellcolor{green!20}\href{../works/BeldiceanuCP08.pdf}{BeldiceanuCP08} (0.95)& \cellcolor{green!20}\href{../works/Laborie18a.pdf}{Laborie18a} (0.95)\\
Euclid& \cellcolor{black!20}\href{../works/HookerY02.pdf}{HookerY02} (0.35)& \cellcolor{black!20}\href{../works/Balduccini11.pdf}{Balduccini11} (0.35)& \cellcolor{black!20}\href{../works/PerezGSL23.pdf}{PerezGSL23} (0.36)& \cellcolor{black!20}\href{../works/abs-2312-13682.pdf}{abs-2312-13682} (0.36)& \cellcolor{black!20}\href{../works/KhemmoudjPB06.pdf}{KhemmoudjPB06} (0.36)\\
Dot& \cellcolor{red!40}\href{../works/Malapert11.pdf}{Malapert11} (106.00)& \cellcolor{red!40}\href{../works/LaborieRSV18.pdf}{LaborieRSV18} (99.00)& \cellcolor{red!40}\href{../works/Godet21a.pdf}{Godet21a} (99.00)& \cellcolor{red!40}\href{../works/Dejemeppe16.pdf}{Dejemeppe16} (99.00)& \cellcolor{red!40}\href{../works/Lombardi10.pdf}{Lombardi10} (99.00)\\
Cosine& \cellcolor{red!40}\href{../works/Madi-WambaLOBM17.pdf}{Madi-WambaLOBM17} (0.64)& \cellcolor{red!40}\href{../works/Balduccini11.pdf}{Balduccini11} (0.61)& \cellcolor{red!40}\href{../works/HookerY02.pdf}{HookerY02} (0.60)& \cellcolor{red!40}\href{../works/BonfiettiLBM14.pdf}{BonfiettiLBM14} (0.59)& \cellcolor{red!40}\href{../works/PerezGSL23.pdf}{PerezGSL23} (0.59)\\
\index{LuoVLBM16}\href{../works/LuoVLBM16.pdf}{LuoVLBM16} R\&C\\
Euclid& \cellcolor{red!40}\href{../works/CrawfordB94.pdf}{CrawfordB94} (0.20)& \cellcolor{red!40}\href{../works/BridiLBBM16.pdf}{BridiLBBM16} (0.22)& \cellcolor{red!40}\href{../works/WallaceF00.pdf}{WallaceF00} (0.23)& \cellcolor{red!40}\href{../works/LauLN08.pdf}{LauLN08} (0.23)& \cellcolor{red!40}\href{../works/FoxAS82.pdf}{FoxAS82} (0.23)\\
Dot& \cellcolor{red!40}\href{../works/ZarandiASC20.pdf}{ZarandiASC20} (69.00)& \cellcolor{red!40}\href{../works/Siala15a.pdf}{Siala15a} (62.00)& \cellcolor{red!40}\href{../works/SenderovichBB19.pdf}{SenderovichBB19} (61.00)& \cellcolor{red!40}\href{../works/Godet21a.pdf}{Godet21a} (61.00)& \cellcolor{red!40}\href{../works/Dejemeppe16.pdf}{Dejemeppe16} (61.00)\\
Cosine& \cellcolor{red!40}\href{../works/CrawfordB94.pdf}{CrawfordB94} (0.75)& \cellcolor{red!40}\href{../works/BridiLBBM16.pdf}{BridiLBBM16} (0.73)& \cellcolor{red!40}\href{../works/Vilim05.pdf}{Vilim05} (0.69)& \cellcolor{red!40}\href{../works/FoxAS82.pdf}{FoxAS82} (0.68)& \cellcolor{red!40}\href{../works/BeckPS03.pdf}{BeckPS03} (0.68)\\
\index{Madi-WambaB16}\href{../works/Madi-WambaB16.pdf}{Madi-WambaB16} R\&C\\
Euclid& \cellcolor{red!40}\href{../works/LudwigKRBMS14.pdf}{LudwigKRBMS14} (0.23)& \cellcolor{red!40}\href{../works/Caseau97.pdf}{Caseau97} (0.23)& \cellcolor{red!40}\href{../works/AngelsmarkJ00.pdf}{AngelsmarkJ00} (0.24)& \cellcolor{red!20}\href{../works/Rit86.pdf}{Rit86} (0.25)& \cellcolor{red!20}\href{../works/ChuGNSW13.pdf}{ChuGNSW13} (0.26)\\
Dot& \cellcolor{red!40}\href{../works/Godet21a.pdf}{Godet21a} (75.00)& \cellcolor{red!40}\href{../works/Dejemeppe16.pdf}{Dejemeppe16} (74.00)& \cellcolor{red!40}\href{../works/Malapert11.pdf}{Malapert11} (73.00)& \cellcolor{red!40}\href{../works/Beck99.pdf}{Beck99} (72.00)& \cellcolor{red!40}\href{../works/Fahimi16.pdf}{Fahimi16} (71.00)\\
Cosine& \cellcolor{red!40}\href{../works/Caseau97.pdf}{Caseau97} (0.70)& \cellcolor{red!40}\href{../works/LudwigKRBMS14.pdf}{LudwigKRBMS14} (0.69)& \cellcolor{red!40}\href{../works/ChuGNSW13.pdf}{ChuGNSW13} (0.69)& \cellcolor{red!40}\href{../works/KovacsV04.pdf}{KovacsV04} (0.69)& \cellcolor{red!40}\href{../works/AngelsmarkJ00.pdf}{AngelsmarkJ00} (0.69)\\
\index{Madi-WambaLOBM17}\href{../works/Madi-WambaLOBM17.pdf}{Madi-WambaLOBM17} R\&C& \cellcolor{red!20}\href{../works/GayHS15.pdf}{GayHS15} (0.88)& \cellcolor{red!20}\href{../works/DerrienPZ14.pdf}{DerrienPZ14} (0.89)& \cellcolor{red!20}\href{../works/GayHS15a.pdf}{GayHS15a} (0.90)& \cellcolor{yellow!20}\href{../works/OuelletQ13.pdf}{OuelletQ13} (0.91)& \cellcolor{yellow!20}\href{../works/LetortCB15.pdf}{LetortCB15} (0.91)\\
Euclid& \cellcolor{yellow!20}\href{../works/ChuGNSW13.pdf}{ChuGNSW13} (0.27)& \cellcolor{yellow!20}\href{../works/KovacsV04.pdf}{KovacsV04} (0.28)& \cellcolor{yellow!20}\href{../works/Bartak02a.pdf}{Bartak02a} (0.28)& \cellcolor{green!20}\href{../works/MurphyMB15.pdf}{MurphyMB15} (0.29)& \cellcolor{green!20}\href{../works/BoothNB16.pdf}{BoothNB16} (0.29)\\
Dot& \cellcolor{red!40}\href{../works/Beck99.pdf}{Beck99} (119.00)& \cellcolor{red!40}\href{../works/Lombardi10.pdf}{Lombardi10} (118.00)& \cellcolor{red!40}\href{../works/Baptiste02.pdf}{Baptiste02} (117.00)& \cellcolor{red!40}\href{../works/Malapert11.pdf}{Malapert11} (115.00)& \cellcolor{red!40}\href{../works/Dejemeppe16.pdf}{Dejemeppe16} (114.00)\\
Cosine& \cellcolor{red!40}\href{../works/ChuGNSW13.pdf}{ChuGNSW13} (0.77)& \cellcolor{red!40}\href{../works/KovacsV04.pdf}{KovacsV04} (0.76)& \cellcolor{red!40}\href{../works/Bartak02a.pdf}{Bartak02a} (0.76)& \cellcolor{red!40}\href{../works/TrojetHL11.pdf}{TrojetHL11} (0.74)& \cellcolor{red!40}\href{../works/MurphyMB15.pdf}{MurphyMB15} (0.73)\\
\index{MagataoAN05}MagataoAN05 R\&C& \cellcolor{red!40}FelizariAL09 (0.00)& \cellcolor{red!40}\href{../works/MouraSCL08a.pdf}{MouraSCL08a} (0.77)& \cellcolor{red!20}\href{../works/HookerY02.pdf}{HookerY02} (0.89)& \cellcolor{yellow!20}\href{../works/LopesCSM10.pdf}{LopesCSM10} (0.92)& \cellcolor{green!20}\href{../works/MouraSCL08.pdf}{MouraSCL08} (0.94)\\
Euclid\\
Dot\\
Cosine\\
\index{Maillard15}\href{../works/Maillard15.pdf}{Maillard15} R\&C\\
Euclid& \cellcolor{red!40}\href{../works/CestaOS98.pdf}{CestaOS98} (0.16)& \cellcolor{red!40}\href{../works/Caballero23.pdf}{Caballero23} (0.17)& \cellcolor{red!40}\href{../works/KovacsEKV05.pdf}{KovacsEKV05} (0.17)& \cellcolor{red!40}\href{../works/AngelsmarkJ00.pdf}{AngelsmarkJ00} (0.19)& \cellcolor{red!40}\href{../works/WuBB05.pdf}{WuBB05} (0.19)\\
Dot& \cellcolor{red!40}\href{../works/LaborieRSV18.pdf}{LaborieRSV18} (44.00)& \cellcolor{red!40}\href{../works/Godet21a.pdf}{Godet21a} (41.00)& \cellcolor{red!40}\href{../works/ZarandiASC20.pdf}{ZarandiASC20} (38.00)& \cellcolor{red!40}\href{../works/Astrand21.pdf}{Astrand21} (37.00)& \cellcolor{red!40}\href{../works/Pralet17.pdf}{Pralet17} (36.00)\\
Cosine& \cellcolor{red!40}\href{../works/CestaOS98.pdf}{CestaOS98} (0.70)& \cellcolor{red!40}\href{../works/Caballero23.pdf}{Caballero23} (0.68)& \cellcolor{red!40}\href{../works/KovacsEKV05.pdf}{KovacsEKV05} (0.66)& \cellcolor{red!40}\href{../works/BonfiettiM12.pdf}{BonfiettiM12} (0.66)& \cellcolor{red!40}\href{../works/DoomsH08.pdf}{DoomsH08} (0.62)\\
\index{MakMS10}\href{../works/MakMS10.pdf}{MakMS10} R\&C\\
Euclid& \cellcolor{red!40}\href{../works/Limtanyakul07.pdf}{Limtanyakul07} (0.23)& \cellcolor{red!20}\href{../works/Sadykov04.pdf}{Sadykov04} (0.25)& \cellcolor{yellow!20}\href{../works/LauLN08.pdf}{LauLN08} (0.26)& \cellcolor{yellow!20}\href{../works/VanczaM01.pdf}{VanczaM01} (0.27)& \cellcolor{yellow!20}\href{../works/Alaka21.pdf}{Alaka21} (0.27)\\
Dot& \cellcolor{red!40}\href{../works/ZarandiASC20.pdf}{ZarandiASC20} (97.00)& \cellcolor{red!40}\href{../works/Beck99.pdf}{Beck99} (95.00)& \cellcolor{red!40}\href{../works/Astrand21.pdf}{Astrand21} (94.00)& \cellcolor{red!40}\href{../works/Groleaz21.pdf}{Groleaz21} (90.00)& \cellcolor{red!40}\href{../works/BeckDDF98.pdf}{BeckDDF98} (90.00)\\
Cosine& \cellcolor{red!40}\href{../works/Limtanyakul07.pdf}{Limtanyakul07} (0.75)& \cellcolor{red!40}\href{../works/ZeballosH05.pdf}{ZeballosH05} (0.75)& \cellcolor{red!40}\href{../works/KhayatLR06.pdf}{KhayatLR06} (0.74)& \cellcolor{red!40}\href{../works/PinarbasiAY19.pdf}{PinarbasiAY19} (0.74)& \cellcolor{red!40}\href{../works/NovasH14.pdf}{NovasH14} (0.74)\\
\index{Malapert11}\href{../works/Malapert11.pdf}{Malapert11} R\&C\\
Euclid& \href{../works/Fahimi16.pdf}{Fahimi16} (0.56)& \href{../works/MalapertCGJLR12.pdf}{MalapertCGJLR12} (0.58)& \href{../works/GrimesH15.pdf}{GrimesH15} (0.59)& \href{../works/FahimiOQ18.pdf}{FahimiOQ18} (0.60)& \href{../works/GokgurHO18.pdf}{GokgurHO18} (0.60)\\
Dot& \cellcolor{red!40}\href{../works/Dejemeppe16.pdf}{Dejemeppe16} (307.00)& \cellcolor{red!40}\href{../works/Baptiste02.pdf}{Baptiste02} (297.00)& \cellcolor{red!40}\href{../works/Groleaz21.pdf}{Groleaz21} (292.00)& \cellcolor{red!40}\href{../works/Fahimi16.pdf}{Fahimi16} (280.00)& \cellcolor{red!40}\href{../works/ZarandiASC20.pdf}{ZarandiASC20} (278.00)\\
Cosine& \cellcolor{red!40}\href{../works/Fahimi16.pdf}{Fahimi16} (0.73)& \cellcolor{red!40}\href{../works/MalapertCGJLR12.pdf}{MalapertCGJLR12} (0.69)& \cellcolor{red!40}\href{../works/GrimesH15.pdf}{GrimesH15} (0.68)& \cellcolor{red!40}\href{../works/Baptiste02.pdf}{Baptiste02} (0.67)& \cellcolor{red!40}\href{../works/Schutt11.pdf}{Schutt11} (0.67)\\
\index{MalapertCGJLR12}\href{../works/MalapertCGJLR12.pdf}{MalapertCGJLR12} R\&C& \cellcolor{red!40}\href{../works/MejiaY20.pdf}{MejiaY20} (0.77)& \cellcolor{red!40}\href{../works/GrimesHM09.pdf}{GrimesHM09} (0.78)& \cellcolor{red!40}\href{../works/AbreuAPNM21.pdf}{AbreuAPNM21} (0.83)& \cellcolor{red!40}\href{../works/JussienL02.pdf}{JussienL02} (0.86)& \cellcolor{red!20}\href{../works/MonetteDD07.pdf}{MonetteDD07} (0.88)\\
Euclid& \cellcolor{green!20}\href{../works/MalapertCGJLR13.pdf}{MalapertCGJLR13} (0.29)& \cellcolor{blue!20}\href{../works/GrimesHM09.pdf}{GrimesHM09} (0.32)& \cellcolor{blue!20}\href{../works/TanSD10.pdf}{TanSD10} (0.34)& \cellcolor{blue!20}\href{../works/GrimesH10.pdf}{GrimesH10} (0.34)& \cellcolor{black!20}\href{../works/MonetteDD07.pdf}{MonetteDD07} (0.34)\\
Dot& \cellcolor{red!40}\href{../works/Groleaz21.pdf}{Groleaz21} (202.00)& \cellcolor{red!40}\href{../works/Malapert11.pdf}{Malapert11} (199.00)& \cellcolor{red!40}\href{../works/ZarandiASC20.pdf}{ZarandiASC20} (178.00)& \cellcolor{red!40}\href{../works/Dejemeppe16.pdf}{Dejemeppe16} (178.00)& \cellcolor{red!40}\href{../works/GrimesH15.pdf}{GrimesH15} (176.00)\\
Cosine& \cellcolor{red!40}\href{../works/MalapertCGJLR13.pdf}{MalapertCGJLR13} (0.84)& \cellcolor{red!40}\href{../works/GrimesHM09.pdf}{GrimesHM09} (0.80)& \cellcolor{red!40}\href{../works/GrimesH10.pdf}{GrimesH10} (0.78)& \cellcolor{red!40}\href{../works/GrimesH15.pdf}{GrimesH15} (0.77)& \cellcolor{red!40}\href{../works/TanSD10.pdf}{TanSD10} (0.77)\\
\index{MalapertCGJLR13}\href{../works/MalapertCGJLR13.pdf}{MalapertCGJLR13} R\&C\\
Euclid& \cellcolor{red!20}\href{../works/KovacsV04.pdf}{KovacsV04} (0.25)& \cellcolor{yellow!20}\href{../works/HeipckeCCS00.pdf}{HeipckeCCS00} (0.26)& \cellcolor{yellow!20}\href{../works/Caseau97.pdf}{Caseau97} (0.26)& \cellcolor{yellow!20}\href{../works/ChuGNSW13.pdf}{ChuGNSW13} (0.27)& \cellcolor{yellow!20}\href{../works/Bartak02a.pdf}{Bartak02a} (0.27)\\
Dot& \cellcolor{red!40}\href{../works/Groleaz21.pdf}{Groleaz21} (130.00)& \cellcolor{red!40}\href{../works/Malapert11.pdf}{Malapert11} (128.00)& \cellcolor{red!40}\href{../works/ZarandiASC20.pdf}{ZarandiASC20} (125.00)& \cellcolor{red!40}\href{../works/Baptiste02.pdf}{Baptiste02} (122.00)& \cellcolor{red!40}\href{../works/Godet21a.pdf}{Godet21a} (116.00)\\
Cosine& \cellcolor{red!40}\href{../works/MalapertCGJLR12.pdf}{MalapertCGJLR12} (0.84)& \cellcolor{red!40}\href{../works/BartakSR08.pdf}{BartakSR08} (0.81)& \cellcolor{red!40}\href{../works/GrimesHM09.pdf}{GrimesHM09} (0.80)& \cellcolor{red!40}\href{../works/GrimesH10.pdf}{GrimesH10} (0.79)& \cellcolor{red!40}\href{../works/KovacsV04.pdf}{KovacsV04} (0.78)\\
\index{MalapertGR12}\href{../works/MalapertGR12.pdf}{MalapertGR12} R\&C& \cellcolor{red!40}\href{../works/KoschB14.pdf}{KoschB14} (0.82)& \cellcolor{red!20}\href{../works/TangB20.pdf}{TangB20} (0.86)& \cellcolor{red!20}\href{../works/LacknerMMWW23.pdf}{LacknerMMWW23} (0.89)& \cellcolor{yellow!20}\href{../works/HamFC17.pdf}{HamFC17} (0.92)& \cellcolor{green!20}\href{../works/ZeballosNH11.pdf}{ZeballosNH11} (0.96)\\
Euclid& \cellcolor{red!20}\href{../works/KoschB14.pdf}{KoschB14} (0.26)& \cellcolor{black!20}\href{../works/Sadykov04.pdf}{Sadykov04} (0.34)& \cellcolor{black!20}\href{../works/HamFC17.pdf}{HamFC17} (0.35)& \cellcolor{black!20}\href{../works/PengLC14.pdf}{PengLC14} (0.36)& \cellcolor{black!20}\href{../works/Limtanyakul07.pdf}{Limtanyakul07} (0.37)\\
Dot& \cellcolor{red!40}\href{../works/Malapert11.pdf}{Malapert11} (157.00)& \cellcolor{red!40}\href{../works/ZarandiASC20.pdf}{ZarandiASC20} (155.00)& \cellcolor{red!40}\href{../works/Groleaz21.pdf}{Groleaz21} (153.00)& \cellcolor{red!40}\href{../works/Dejemeppe16.pdf}{Dejemeppe16} (142.00)& \cellcolor{red!40}\href{../works/Baptiste02.pdf}{Baptiste02} (142.00)\\
Cosine& \cellcolor{red!40}\href{../works/KoschB14.pdf}{KoschB14} (0.85)& \cellcolor{red!40}\href{../works/HamFC17.pdf}{HamFC17} (0.70)& \cellcolor{red!40}\href{../works/Sadykov04.pdf}{Sadykov04} (0.69)& \cellcolor{red!40}\href{../works/PengLC14.pdf}{PengLC14} (0.68)& \cellcolor{red!40}\href{../works/Bedhief21.pdf}{Bedhief21} (0.68)\\
\index{MalapertN19}\href{../works/MalapertN19.pdf}{MalapertN19} R\&C& \cellcolor{red!40}\href{../works/NattafDYW19.pdf}{NattafDYW19} (0.75)& \cellcolor{yellow!20}\href{../works/NattafM20.pdf}{NattafM20} (0.92)& \cellcolor{green!20}\href{../works/TangB20.pdf}{TangB20} (0.95)& \cellcolor{blue!20}\href{../works/GokgurHO18.pdf}{GokgurHO18} (0.98)& \cellcolor{black!20}\href{../works/EmdeZD22.pdf}{EmdeZD22} (0.98)\\
Euclid& \cellcolor{red!40}\href{../works/NattafM20.pdf}{NattafM20} (0.23)& \cellcolor{blue!20}\href{../works/Ham18a.pdf}{Ham18a} (0.33)& \cellcolor{blue!20}\href{../works/NattafDYW19.pdf}{NattafDYW19} (0.34)& \cellcolor{black!20}\href{../works/ArbaouiY18.pdf}{ArbaouiY18} (0.36)& \cellcolor{black!20}\href{../works/GedikKEK18.pdf}{GedikKEK18} (0.36)\\
Dot& \cellcolor{red!40}\href{../works/Groleaz21.pdf}{Groleaz21} (153.00)& \cellcolor{red!40}\href{../works/NaderiRR23.pdf}{NaderiRR23} (152.00)& \cellcolor{red!40}\href{../works/YunusogluY22.pdf}{YunusogluY22} (135.00)& \cellcolor{red!40}\href{../works/ZarandiASC20.pdf}{ZarandiASC20} (135.00)& \cellcolor{red!40}\href{../works/Lunardi20.pdf}{Lunardi20} (134.00)\\
Cosine& \cellcolor{red!40}\href{../works/NattafM20.pdf}{NattafM20} (0.89)& \cellcolor{red!40}\href{../works/Ham18a.pdf}{Ham18a} (0.76)& \cellcolor{red!40}\href{../works/NattafDYW19.pdf}{NattafDYW19} (0.76)& \cellcolor{red!40}\href{../works/GedikKEK18.pdf}{GedikKEK18} (0.74)& \cellcolor{red!40}\href{../works/ArbaouiY18.pdf}{ArbaouiY18} (0.71)\\
\index{Malik08}\href{../works/Malik08.pdf}{Malik08} R\&C\\
Euclid& \cellcolor{red!20}\href{../works/MalikMB08.pdf}{MalikMB08} (0.25)& \cellcolor{red!20}\href{../works/BegB13.pdf}{BegB13} (0.25)& \cellcolor{green!20}\href{../works/KuchcinskiW03.pdf}{KuchcinskiW03} (0.29)& \cellcolor{green!20}\href{../works/OddiPCC03.pdf}{OddiPCC03} (0.30)& \cellcolor{green!20}\href{../works/ErtlK91.pdf}{ErtlK91} (0.30)\\
Dot& \cellcolor{red!40}\href{../works/Dejemeppe16.pdf}{Dejemeppe16} (121.00)& \cellcolor{red!40}\href{../works/Lombardi10.pdf}{Lombardi10} (117.00)& \cellcolor{red!40}\href{../works/Fahimi16.pdf}{Fahimi16} (113.00)& \cellcolor{red!40}\href{../works/Caballero19.pdf}{Caballero19} (112.00)& \cellcolor{red!40}\href{../works/Groleaz21.pdf}{Groleaz21} (111.00)\\
Cosine& \cellcolor{red!40}\href{../works/MalikMB08.pdf}{MalikMB08} (0.81)& \cellcolor{red!40}\href{../works/BegB13.pdf}{BegB13} (0.81)& \cellcolor{red!40}\href{../works/OddiPCC03.pdf}{OddiPCC03} (0.71)& \cellcolor{red!40}\href{../works/KuchcinskiW03.pdf}{KuchcinskiW03} (0.71)& \cellcolor{red!40}\href{../works/BonfiettiLBM14.pdf}{BonfiettiLBM14} (0.70)\\
\index{MalikMB08}\href{../works/MalikMB08.pdf}{MalikMB08} R\&C& \cellcolor{red!20}\href{../works/BegB13.pdf}{BegB13} (0.88)& \cellcolor{red!20}\href{../works/ErtlK91.pdf}{ErtlK91} (0.89)& \cellcolor{red!20}\href{../works/KovacsV04.pdf}{KovacsV04} (0.89)& \cellcolor{red!20}\href{../works/Davenport10.pdf}{Davenport10} (0.90)& \cellcolor{yellow!20}\href{../works/LimtanyakulS12.pdf}{LimtanyakulS12} (0.91)\\
Euclid& \cellcolor{red!40}\href{../works/BegB13.pdf}{BegB13} (0.19)& \cellcolor{red!40}\href{../works/KuchcinskiW03.pdf}{KuchcinskiW03} (0.21)& \cellcolor{red!40}\href{../works/ErtlK91.pdf}{ErtlK91} (0.22)& \cellcolor{red!40}\href{../works/LozanoCDS12.pdf}{LozanoCDS12} (0.23)& \cellcolor{red!20}\href{../works/Malik08.pdf}{Malik08} (0.25)\\
Dot& \cellcolor{red!40}\href{../works/Malik08.pdf}{Malik08} (72.00)& \cellcolor{red!40}\href{../works/Caballero19.pdf}{Caballero19} (71.00)& \cellcolor{red!40}\href{../works/Dejemeppe16.pdf}{Dejemeppe16} (69.00)& \cellcolor{red!40}\href{../works/Kuchcinski03.pdf}{Kuchcinski03} (69.00)& \cellcolor{red!40}\href{../works/Siala15a.pdf}{Siala15a} (66.00)\\
Cosine& \cellcolor{red!40}\href{../works/BegB13.pdf}{BegB13} (0.83)& \cellcolor{red!40}\href{../works/Malik08.pdf}{Malik08} (0.81)& \cellcolor{red!40}\href{../works/KuchcinskiW03.pdf}{KuchcinskiW03} (0.77)& \cellcolor{red!40}\href{../works/ErtlK91.pdf}{ErtlK91} (0.75)& \cellcolor{red!40}\href{../works/LozanoCDS12.pdf}{LozanoCDS12} (0.73)\\
\index{MaraveliasCG04}\href{../works/MaraveliasCG04.pdf}{MaraveliasCG04} R\&C& \cellcolor{red!40}\href{../works/RoePS05.pdf}{RoePS05} (0.68)& \cellcolor{red!40}\href{../works/HarjunkoskiG02.pdf}{HarjunkoskiG02} (0.77)& \cellcolor{red!40}\href{../works/MaraveliasG04.pdf}{MaraveliasG04} (0.77)& \cellcolor{red!40}\href{../works/JainG01.pdf}{JainG01} (0.84)& \cellcolor{red!40}\href{../works/ZeballosNH11.pdf}{ZeballosNH11} (0.85)\\
Euclid& \cellcolor{red!20}\href{../works/ZeballosM09.pdf}{ZeballosM09} (0.26)& \cellcolor{yellow!20}\href{../works/GilesH16.pdf}{GilesH16} (0.28)& \cellcolor{green!20}\href{../works/AstrandJZ18.pdf}{AstrandJZ18} (0.30)& \cellcolor{green!20}\href{../works/ZibranR11a.pdf}{ZibranR11a} (0.30)& \cellcolor{green!20}\href{../works/QuirogaZH05.pdf}{QuirogaZH05} (0.30)\\
Dot& \cellcolor{red!40}\href{../works/Dejemeppe16.pdf}{Dejemeppe16} (113.00)& \cellcolor{red!40}\href{../works/Malapert11.pdf}{Malapert11} (111.00)& \cellcolor{red!40}\href{../works/ZarandiASC20.pdf}{ZarandiASC20} (110.00)& \cellcolor{red!40}\href{../works/Astrand21.pdf}{Astrand21} (110.00)& \cellcolor{red!40}\href{../works/HarjunkoskiMBC14.pdf}{HarjunkoskiMBC14} (110.00)\\
Cosine& \cellcolor{red!40}\href{../works/ZeballosM09.pdf}{ZeballosM09} (0.79)& \cellcolor{red!40}\href{../works/QuirogaZH05.pdf}{QuirogaZH05} (0.74)& \cellcolor{red!40}\href{../works/ZeballosNH11.pdf}{ZeballosNH11} (0.74)& \cellcolor{red!40}\href{../works/GilesH16.pdf}{GilesH16} (0.74)& \cellcolor{red!40}\href{../works/AstrandJZ18.pdf}{AstrandJZ18} (0.72)\\
\index{MaraveliasG04}\href{../works/MaraveliasG04.pdf}{MaraveliasG04} R\&C& \cellcolor{red!40}\href{../works/MaraveliasCG04.pdf}{MaraveliasCG04} (0.77)& \cellcolor{red!40}\href{../works/HamdiL13.pdf}{HamdiL13} (0.81)& \cellcolor{red!40}\href{../works/CobanH10.pdf}{CobanH10} (0.83)& \cellcolor{red!40}\href{../works/ChuX05.pdf}{ChuX05} (0.86)& \cellcolor{red!20}\href{../works/CireCH13.pdf}{CireCH13} (0.86)\\
Euclid& \cellcolor{red!40}\href{../works/Baptiste09.pdf}{Baptiste09} (0.13)& \cellcolor{red!40}\href{../works/KameugneF13.pdf}{KameugneF13} (0.13)& \cellcolor{red!40}\href{../works/Layfield02.pdf}{Layfield02} (0.14)& \cellcolor{red!40}\href{../works/AbrilSB05.pdf}{AbrilSB05} (0.14)& \cellcolor{red!40}\href{../works/CarchraeBF05.pdf}{CarchraeBF05} (0.16)\\
Dot& \cellcolor{red!40}\href{../works/Layfield02.pdf}{Layfield02} (6.00)& \cellcolor{red!40}\href{../works/ZeballosNH11.pdf}{ZeballosNH11} (4.00)& \cellcolor{red!40}\href{../works/BeldiceanuC94.pdf}{BeldiceanuC94} (4.00)& \cellcolor{red!40}\href{../works/KorbaaYG99.pdf}{KorbaaYG99} (2.00)& \cellcolor{red!40}\href{../works/Froger16.pdf}{Froger16} (2.00)\\
Cosine& \cellcolor{red!40}\href{../works/Layfield02.pdf}{Layfield02} (0.65)& \cellcolor{red!40}\href{../works/BeldiceanuC94.pdf}{BeldiceanuC94} (0.17)& \cellcolor{red!40}\href{../works/ZeballosNH11.pdf}{ZeballosNH11} (0.16)& \cellcolor{red!40}\href{../works/LudwigKRBMS14.pdf}{LudwigKRBMS14} (0.15)& \cellcolor{red!40}\href{../works/KorbaaYG99.pdf}{KorbaaYG99} (0.10)\\
\index{MarliereSPR23}\href{../works/MarliereSPR23.pdf}{MarliereSPR23} R\&C& \cellcolor{green!20}\href{../works/CappartS17.pdf}{CappartS17} (0.96)& \cellcolor{blue!20}\href{../works/Laborie18a.pdf}{Laborie18a} (0.98)& \cellcolor{blue!20}\href{../works/ColT2019a.pdf}{ColT2019a} (0.98)& \cellcolor{blue!20}\href{../works/ColT19.pdf}{ColT19} (0.98)& \cellcolor{blue!20}\href{../works/VlkHT21.pdf}{VlkHT21} (0.98)\\
Euclid& \cellcolor{black!20}\href{../works/CappartS17.pdf}{CappartS17} (0.35)& \cellcolor{black!20}\href{../works/Rodriguez07b.pdf}{Rodriguez07b} (0.35)& \cellcolor{black!20}\href{../works/Rodriguez07.pdf}{Rodriguez07} (0.36)& \cellcolor{black!20}\href{../works/RodriguezS09.pdf}{RodriguezS09} (0.36)& \href{../works/NishikawaSTT19.pdf}{NishikawaSTT19} (0.38)\\
Dot& \cellcolor{red!40}\href{../works/ZarandiASC20.pdf}{ZarandiASC20} (144.00)& \cellcolor{red!40}\href{../works/Lombardi10.pdf}{Lombardi10} (133.00)& \cellcolor{red!40}\href{../works/Lemos21.pdf}{Lemos21} (131.00)& \cellcolor{red!40}\href{../works/LaborieRSV18.pdf}{LaborieRSV18} (130.00)& \cellcolor{red!40}\href{../works/Astrand21.pdf}{Astrand21} (127.00)\\
Cosine& \cellcolor{red!40}\href{../works/CappartS17.pdf}{CappartS17} (0.71)& \cellcolor{red!40}\href{../works/Rodriguez07b.pdf}{Rodriguez07b} (0.70)& \cellcolor{red!40}\href{../works/Rodriguez07.pdf}{Rodriguez07} (0.70)& \cellcolor{red!40}\href{../works/RodriguezS09.pdf}{RodriguezS09} (0.69)& \cellcolor{red!40}\href{../works/NishikawaSTT19.pdf}{NishikawaSTT19} (0.65)\\
\index{MartinPY01}\href{../works/MartinPY01.pdf}{MartinPY01} R\&C& \cellcolor{blue!20}\href{../works/RussellU06.pdf}{RussellU06} (0.97)& \cellcolor{blue!20}\href{../works/NovasH14.pdf}{NovasH14} (0.98)& \cellcolor{black!20}\href{../works/HenzMT04.pdf}{HenzMT04} (0.98)& \cellcolor{black!20}\href{../works/GoelSHFS15.pdf}{GoelSHFS15} (0.98)& \cellcolor{black!20}\href{../works/ArtiguesF07.pdf}{ArtiguesF07} (0.98)\\
Euclid& \cellcolor{yellow!20}\href{../works/Touraivane95.pdf}{Touraivane95} (0.27)& \cellcolor{yellow!20}\href{../works/PesantGPR99.pdf}{PesantGPR99} (0.27)& \cellcolor{yellow!20}\href{../works/FalaschiGMP97.pdf}{FalaschiGMP97} (0.27)& \cellcolor{green!20}\href{../works/Acuna-AgostMFG09.pdf}{Acuna-AgostMFG09} (0.29)& \cellcolor{green!20}\href{../works/JelinekB16.pdf}{JelinekB16} (0.30)\\
Dot& \cellcolor{red!40}\href{../works/ZarandiASC20.pdf}{ZarandiASC20} (85.00)& \cellcolor{red!40}\href{../works/Wallace96.pdf}{Wallace96} (79.00)& \cellcolor{red!40}\href{../works/Simonis99.pdf}{Simonis99} (77.00)& \cellcolor{red!40}\href{../works/Lemos21.pdf}{Lemos21} (77.00)& \cellcolor{red!40}\href{../works/Baptiste02.pdf}{Baptiste02} (77.00)\\
Cosine& \cellcolor{red!40}\href{../works/Touraivane95.pdf}{Touraivane95} (0.69)& \cellcolor{red!40}\href{../works/PesantGPR99.pdf}{PesantGPR99} (0.68)& \cellcolor{red!40}\href{../works/LammaMM97.pdf}{LammaMM97} (0.67)& \cellcolor{red!40}\href{../works/FalaschiGMP97.pdf}{FalaschiGMP97} (0.66)& \cellcolor{red!40}\href{../works/Wallace96.pdf}{Wallace96} (0.64)\\
\index{MartnezAJ22}MartnezAJ22 R\&C& \cellcolor{red!40}\href{../works/ElciOH22.pdf}{ElciOH22} (0.54)& \cellcolor{red!40}NaderiR22 (0.73)& \cellcolor{red!40}\href{../works/RoshanaeiN21.pdf}{RoshanaeiN21} (0.80)& \cellcolor{red!40}\href{../works/RoshanaeiBAUB20.pdf}{RoshanaeiBAUB20} (0.82)& \cellcolor{red!40}\href{../works/FachiniA20.pdf}{FachiniA20} (0.84)\\
Euclid\\
Dot\\
Cosine\\
\index{Mason01}\href{../works/Mason01.pdf}{Mason01} R\&C& \cellcolor{blue!20}EdisO11a (0.98)\\
Euclid& \cellcolor{green!20}\href{../works/ZibranR11.pdf}{ZibranR11} (0.30)& \cellcolor{green!20}\href{../works/Acuna-AgostMFG09.pdf}{Acuna-AgostMFG09} (0.31)& \cellcolor{green!20}\href{../works/ChapadosJR11.pdf}{ChapadosJR11} (0.31)& \cellcolor{green!20}\href{../works/HebrardALLCMR22.pdf}{HebrardALLCMR22} (0.31)& \cellcolor{green!20}\href{../works/Gronkvist06.pdf}{Gronkvist06} (0.31)\\
Dot& \cellcolor{red!40}\href{../works/ZarandiASC20.pdf}{ZarandiASC20} (74.00)& \cellcolor{red!40}\href{../works/MilanoW09.pdf}{MilanoW09} (68.00)& \cellcolor{red!40}\href{../works/MilanoW06.pdf}{MilanoW06} (68.00)& \cellcolor{red!40}\href{../works/QinDS16.pdf}{QinDS16} (63.00)& \cellcolor{red!40}\href{../works/BourreauGGLT22.pdf}{BourreauGGLT22} (63.00)\\
Cosine& \cellcolor{red!40}\href{../works/Gronkvist06.pdf}{Gronkvist06} (0.64)& \cellcolor{red!40}\href{../works/BourreauGGLT22.pdf}{BourreauGGLT22} (0.58)& \cellcolor{red!40}\href{../works/ZibranR11.pdf}{ZibranR11} (0.56)& \cellcolor{red!40}\href{../works/HachemiGR11.pdf}{HachemiGR11} (0.55)& \cellcolor{red!40}\href{../works/EreminW01.pdf}{EreminW01} (0.55)\\
\index{Mehdizadeh-Somarin23}\href{../works/Mehdizadeh-Somarin23.pdf}{Mehdizadeh-Somarin23} R\&C\\
Euclid& \cellcolor{green!20}\href{../works/JuvinHL23.pdf}{JuvinHL23} (0.30)& \cellcolor{green!20}\href{../works/BillautHL12.pdf}{BillautHL12} (0.30)& \cellcolor{blue!20}\href{../works/TanT18.pdf}{TanT18} (0.33)& \cellcolor{blue!20}\href{../works/WatsonB08.pdf}{WatsonB08} (0.34)& \cellcolor{blue!20}\href{../works/LiFJZLL22.pdf}{LiFJZLL22} (0.34)\\
Dot& \cellcolor{red!40}\href{../works/ZarandiASC20.pdf}{ZarandiASC20} (154.00)& \cellcolor{red!40}\href{../works/Groleaz21.pdf}{Groleaz21} (153.00)& \cellcolor{red!40}\href{../works/Lunardi20.pdf}{Lunardi20} (143.00)& \cellcolor{red!40}\href{../works/Astrand21.pdf}{Astrand21} (136.00)& \cellcolor{red!40}\href{../works/Baptiste02.pdf}{Baptiste02} (125.00)\\
Cosine& \cellcolor{red!40}\href{../works/JuvinHL23.pdf}{JuvinHL23} (0.74)& \cellcolor{red!40}\href{../works/BillautHL12.pdf}{BillautHL12} (0.74)& \cellcolor{red!40}\href{../works/TerekhovTDB14.pdf}{TerekhovTDB14} (0.73)& \cellcolor{red!40}\href{../works/TanT18.pdf}{TanT18} (0.72)& \cellcolor{red!40}\href{../works/ZhangW18.pdf}{ZhangW18} (0.71)\\
\index{MejiaY20}\href{../works/MejiaY20.pdf}{MejiaY20} R\&C& \cellcolor{red!40}\href{../works/MalapertCGJLR12.pdf}{MalapertCGJLR12} (0.77)& \cellcolor{red!40}\href{../works/AbreuAPNM21.pdf}{AbreuAPNM21} (0.77)& \cellcolor{red!40}\href{../works/AbreuN22.pdf}{AbreuN22} (0.79)& \cellcolor{red!40}\href{../works/AbreuNP23.pdf}{AbreuNP23} (0.85)& \cellcolor{yellow!20}\href{../works/JussienL02.pdf}{JussienL02} (0.91)\\
Euclid& \cellcolor{black!20}\href{../works/AbreuAPNM21.pdf}{AbreuAPNM21} (0.34)& \href{../works/AbreuPNF23.pdf}{AbreuPNF23} (0.38)& \href{../works/AbreuNP23.pdf}{AbreuNP23} (0.38)& \href{../works/AbreuN22.pdf}{AbreuN22} (0.38)& \href{../works/AfsarVPG23.pdf}{AfsarVPG23} (0.41)\\
Dot& \cellcolor{red!40}\href{../works/ZarandiASC20.pdf}{ZarandiASC20} (204.00)& \cellcolor{red!40}\href{../works/Lunardi20.pdf}{Lunardi20} (190.00)& \cellcolor{red!40}\href{../works/Groleaz21.pdf}{Groleaz21} (188.00)& \cellcolor{red!40}\href{../works/IsikYA23.pdf}{IsikYA23} (174.00)& \cellcolor{red!40}\href{../works/AbreuN22.pdf}{AbreuN22} (174.00)\\
Cosine& \cellcolor{red!40}\href{../works/AbreuAPNM21.pdf}{AbreuAPNM21} (0.80)& \cellcolor{red!40}\href{../works/AbreuN22.pdf}{AbreuN22} (0.78)& \cellcolor{red!40}\href{../works/AbreuPNF23.pdf}{AbreuPNF23} (0.78)& \cellcolor{red!40}\href{../works/AbreuNP23.pdf}{AbreuNP23} (0.78)& \cellcolor{red!40}\href{../works/MengZRZL20.pdf}{MengZRZL20} (0.74)\\
\index{MelgarejoLS15}\href{../works/MelgarejoLS15.pdf}{MelgarejoLS15} R\&C& \cellcolor{green!20}\href{../works/BartoliniBBLM14.pdf}{BartoliniBBLM14} (0.95)& \cellcolor{green!20}\href{../works/MusliuSS18.pdf}{MusliuSS18} (0.95)& \cellcolor{green!20}\href{../works/GilesH16.pdf}{GilesH16} (0.96)& \cellcolor{blue!20}\href{../works/KoehlerBFFHPSSS21.pdf}{KoehlerBFFHPSSS21} (0.97)& \cellcolor{blue!20}\href{../works/BartakSR08.pdf}{BartakSR08} (0.97)\\
Euclid& \cellcolor{green!20}\href{../works/GedikKBR17.pdf}{GedikKBR17} (0.30)& \cellcolor{green!20}\href{../works/LipovetzkyBPS14.pdf}{LipovetzkyBPS14} (0.31)& \cellcolor{green!20}\href{../works/Rit86.pdf}{Rit86} (0.31)& \cellcolor{blue!20}\href{../works/BandaSC11.pdf}{BandaSC11} (0.32)& \cellcolor{blue!20}\href{../works/BofillCGGPSV23.pdf}{BofillCGGPSV23} (0.32)\\
Dot& \cellcolor{red!40}\href{../works/Groleaz21.pdf}{Groleaz21} (105.00)& \cellcolor{red!40}\href{../works/Lunardi20.pdf}{Lunardi20} (102.00)& \cellcolor{red!40}\href{../works/Astrand21.pdf}{Astrand21} (99.00)& \cellcolor{red!40}\href{../works/LaborieRSV18.pdf}{LaborieRSV18} (98.00)& \cellcolor{red!40}\href{../works/Malapert11.pdf}{Malapert11} (97.00)\\
Cosine& \cellcolor{red!40}\href{../works/GedikKBR17.pdf}{GedikKBR17} (0.68)& \cellcolor{red!40}\href{../works/LipovetzkyBPS14.pdf}{LipovetzkyBPS14} (0.65)& \cellcolor{red!40}\href{../works/LaborieR14.pdf}{LaborieR14} (0.65)& \cellcolor{red!40}\href{../works/HeipckeCCS00.pdf}{HeipckeCCS00} (0.64)& \cellcolor{red!40}\href{../works/GeibingerMM21.pdf}{GeibingerMM21} (0.64)\\
\index{Menana11}\href{../works/Menana11.pdf}{Menana11} R\&C\\
Euclid& \cellcolor{blue!20}\href{../works/abs-1902-01193.pdf}{abs-1902-01193} (0.34)& \href{../works/BenoistGR02.pdf}{BenoistGR02} (0.38)& \href{../works/BofillGSV15.pdf}{BofillGSV15} (0.38)& \href{../works/ErkingerM17.pdf}{ErkingerM17} (0.38)& \href{../works/LiuLH19.pdf}{LiuLH19} (0.39)\\
Dot& \cellcolor{red!40}\href{../works/Malapert11.pdf}{Malapert11} (109.00)& \cellcolor{red!40}\href{../works/Dejemeppe16.pdf}{Dejemeppe16} (106.00)& \cellcolor{red!40}\href{../works/Siala15a.pdf}{Siala15a} (103.00)& \cellcolor{red!40}\href{../works/ZarandiASC20.pdf}{ZarandiASC20} (102.00)& \cellcolor{red!40}\href{../works/Godet21a.pdf}{Godet21a} (101.00)\\
Cosine& \cellcolor{red!40}\href{../works/abs-1902-01193.pdf}{abs-1902-01193} (0.65)& \cellcolor{red!40}\href{../works/Derrien15.pdf}{Derrien15} (0.54)& \cellcolor{red!40}\href{../works/ErkingerM17.pdf}{ErkingerM17} (0.53)& \cellcolor{red!40}\href{../works/German18.pdf}{German18} (0.53)& \cellcolor{red!40}\href{../works/PinarbasiAY19.pdf}{PinarbasiAY19} (0.53)\\
\index{MenciaSV12}\href{../works/MenciaSV12.pdf}{MenciaSV12} R\&C& \cellcolor{red!40}\href{../works/MenciaSV13.pdf}{MenciaSV13} (0.63)& \cellcolor{red!40}\href{../works/ColT2019a.pdf}{ColT2019a} (0.84)& \cellcolor{red!40}\href{../works/WatsonB08.pdf}{WatsonB08} (0.86)& \cellcolor{red!20}\href{../works/BeckFW11.pdf}{BeckFW11} (0.87)& \cellcolor{red!20}DomdorfPH03 (0.88)\\
Euclid& \cellcolor{red!40}\href{../works/MenciaSV13.pdf}{MenciaSV13} (0.20)& \cellcolor{green!20}\href{../works/SourdN00.pdf}{SourdN00} (0.29)& \cellcolor{green!20}\href{../works/TanSD10.pdf}{TanSD10} (0.30)& \cellcolor{green!20}\href{../works/ArtiguesF07.pdf}{ArtiguesF07} (0.30)& \cellcolor{blue!20}\href{../works/ArtiguesBF04.pdf}{ArtiguesBF04} (0.32)\\
Dot& \cellcolor{red!40}\href{../works/Baptiste02.pdf}{Baptiste02} (174.00)& \cellcolor{red!40}\href{../works/ZarandiASC20.pdf}{ZarandiASC20} (171.00)& \cellcolor{red!40}\href{../works/Groleaz21.pdf}{Groleaz21} (164.00)& \cellcolor{red!40}\href{../works/Dejemeppe16.pdf}{Dejemeppe16} (158.00)& \cellcolor{red!40}\href{../works/Malapert11.pdf}{Malapert11} (151.00)\\
Cosine& \cellcolor{red!40}\href{../works/MenciaSV13.pdf}{MenciaSV13} (0.91)& \cellcolor{red!40}\href{../works/SourdN00.pdf}{SourdN00} (0.82)& \cellcolor{red!40}\href{../works/ArtiguesF07.pdf}{ArtiguesF07} (0.80)& \cellcolor{red!40}\href{../works/TanSD10.pdf}{TanSD10} (0.79)& \cellcolor{red!40}\href{../works/NuijtenP98.pdf}{NuijtenP98} (0.78)\\
\index{MenciaSV13}\href{../works/MenciaSV13.pdf}{MenciaSV13} R\&C& \cellcolor{red!40}\href{../works/MenciaSV12.pdf}{MenciaSV12} (0.63)& \cellcolor{red!40}\href{../works/BeckFW11.pdf}{BeckFW11} (0.85)& \cellcolor{red!20}\href{../works/WatsonB08.pdf}{WatsonB08} (0.87)& \cellcolor{red!20}\href{../works/GrimesHM09.pdf}{GrimesHM09} (0.89)& \cellcolor{yellow!20}DomdorfPH03 (0.91)\\
Euclid& \cellcolor{red!40}\href{../works/MenciaSV12.pdf}{MenciaSV12} (0.20)& \cellcolor{yellow!20}\href{../works/TanSD10.pdf}{TanSD10} (0.27)& \cellcolor{yellow!20}\href{../works/SourdN00.pdf}{SourdN00} (0.28)& \cellcolor{green!20}\href{../works/ArtiguesBF04.pdf}{ArtiguesBF04} (0.29)& \cellcolor{green!20}\href{../works/ArtiguesF07.pdf}{ArtiguesF07} (0.29)\\
Dot& \cellcolor{red!40}\href{../works/Baptiste02.pdf}{Baptiste02} (161.00)& \cellcolor{red!40}\href{../works/ZarandiASC20.pdf}{ZarandiASC20} (154.00)& \cellcolor{red!40}\href{../works/Dejemeppe16.pdf}{Dejemeppe16} (150.00)& \cellcolor{red!40}\href{../works/Groleaz21.pdf}{Groleaz21} (148.00)& \cellcolor{red!40}\href{../works/Malapert11.pdf}{Malapert11} (145.00)\\
Cosine& \cellcolor{red!40}\href{../works/MenciaSV12.pdf}{MenciaSV12} (0.91)& \cellcolor{red!40}\href{../works/SourdN00.pdf}{SourdN00} (0.82)& \cellcolor{red!40}\href{../works/TanSD10.pdf}{TanSD10} (0.80)& \cellcolor{red!40}\href{../works/ArtiguesBF04.pdf}{ArtiguesBF04} (0.79)& \cellcolor{red!40}\href{../works/ArtiguesF07.pdf}{ArtiguesF07} (0.79)\\
\index{MengGRZSC22}\href{../works/MengGRZSC22.pdf}{MengGRZSC22} R\&C& \cellcolor{red!40}\href{../works/MengZRZL20.pdf}{MengZRZL20} (0.72)& \cellcolor{red!20}\href{../works/MengLZB21.pdf}{MengLZB21} (0.90)& \cellcolor{yellow!20}\href{../works/OujanaAYB22.pdf}{OujanaAYB22} (0.91)& \cellcolor{green!20}\href{../works/YunusogluY22.pdf}{YunusogluY22} (0.94)& \cellcolor{green!20}\href{../works/AbreuN22.pdf}{AbreuN22} (0.95)\\
Euclid& \cellcolor{blue!20}\href{../works/MengZRZL20.pdf}{MengZRZL20} (0.33)& \cellcolor{black!20}\href{../works/ZhangJZL22.pdf}{ZhangJZL22} (0.35)& \cellcolor{black!20}\href{../works/HamPK21.pdf}{HamPK21} (0.36)& \cellcolor{black!20}\href{../works/LiFJZLL22.pdf}{LiFJZLL22} (0.36)& \cellcolor{black!20}\href{../works/Novas19.pdf}{Novas19} (0.37)\\
Dot& \cellcolor{red!40}\href{../works/ZarandiASC20.pdf}{ZarandiASC20} (184.00)& \cellcolor{red!40}\href{../works/MengZRZL20.pdf}{MengZRZL20} (183.00)& \cellcolor{red!40}\href{../works/IsikYA23.pdf}{IsikYA23} (172.00)& \cellcolor{red!40}\href{../works/NaderiRR23.pdf}{NaderiRR23} (168.00)& \cellcolor{red!40}\href{../works/Lunardi20.pdf}{Lunardi20} (165.00)\\
Cosine& \cellcolor{red!40}\href{../works/MengZRZL20.pdf}{MengZRZL20} (0.83)& \cellcolor{red!40}\href{../works/HamPK21.pdf}{HamPK21} (0.77)& \cellcolor{red!40}\href{../works/Novas19.pdf}{Novas19} (0.76)& \cellcolor{red!40}\href{../works/ZhangJZL22.pdf}{ZhangJZL22} (0.75)& \cellcolor{red!40}\href{../works/MengLZB21.pdf}{MengLZB21} (0.74)\\
\index{MengLZB21}\href{../works/MengLZB21.pdf}{MengLZB21} R\&C& \cellcolor{red!40}\href{../works/NaderiRR23.pdf}{NaderiRR23} (0.81)& \cellcolor{red!40}\href{../works/SacramentoSP20.pdf}{SacramentoSP20} (0.83)& \cellcolor{red!40}\href{../works/MengZRZL20.pdf}{MengZRZL20} (0.84)& \cellcolor{red!40}\href{../works/HamPK21.pdf}{HamPK21} (0.85)& \cellcolor{red!20}\href{../works/GhandehariK22.pdf}{GhandehariK22} (0.88)\\
Euclid& \cellcolor{black!20}\href{../works/MengZRZL20.pdf}{MengZRZL20} (0.34)& \cellcolor{black!20}\href{../works/ZhangJZL22.pdf}{ZhangJZL22} (0.36)& \cellcolor{black!20}\href{../works/Novas19.pdf}{Novas19} (0.37)& \href{../works/MengGRZSC22.pdf}{MengGRZSC22} (0.39)& \href{../works/HamC16.pdf}{HamC16} (0.41)\\
Dot& \cellcolor{red!40}\href{../works/ZarandiASC20.pdf}{ZarandiASC20} (193.00)& \cellcolor{red!40}\href{../works/MengZRZL20.pdf}{MengZRZL20} (189.00)& \cellcolor{red!40}\href{../works/IsikYA23.pdf}{IsikYA23} (187.00)& \cellcolor{red!40}\href{../works/Lunardi20.pdf}{Lunardi20} (177.00)& \cellcolor{red!40}\href{../works/Groleaz21.pdf}{Groleaz21} (176.00)\\
Cosine& \cellcolor{red!40}\href{../works/MengZRZL20.pdf}{MengZRZL20} (0.83)& \cellcolor{red!40}\href{../works/Novas19.pdf}{Novas19} (0.77)& \cellcolor{red!40}\href{../works/IsikYA23.pdf}{IsikYA23} (0.75)& \cellcolor{red!40}\href{../works/ZhangJZL22.pdf}{ZhangJZL22} (0.75)& \cellcolor{red!40}\href{../works/MengGRZSC22.pdf}{MengGRZSC22} (0.74)\\
\index{MengZRZL20}\href{../works/MengZRZL20.pdf}{MengZRZL20} R\&C& \cellcolor{red!40}\href{../works/MengGRZSC22.pdf}{MengGRZSC22} (0.72)& \cellcolor{red!40}\href{../works/MengLZB21.pdf}{MengLZB21} (0.84)& \cellcolor{red!20}\href{../works/GedikKEK18.pdf}{GedikKEK18} (0.87)& \cellcolor{red!20}\href{../works/LunardiBLRV20.pdf}{LunardiBLRV20} (0.89)& \cellcolor{red!20}\href{../works/HamPK21.pdf}{HamPK21} (0.89)\\
Euclid& \cellcolor{blue!20}\href{../works/MengGRZSC22.pdf}{MengGRZSC22} (0.33)& \cellcolor{black!20}\href{../works/MengLZB21.pdf}{MengLZB21} (0.34)& \cellcolor{black!20}\href{../works/Novas19.pdf}{Novas19} (0.37)& \href{../works/HamPK21.pdf}{HamPK21} (0.38)& \href{../works/IsikYA23.pdf}{IsikYA23} (0.38)\\
Dot& \cellcolor{red!40}\href{../works/ZarandiASC20.pdf}{ZarandiASC20} (241.00)& \cellcolor{red!40}\href{../works/IsikYA23.pdf}{IsikYA23} (225.00)& \cellcolor{red!40}\href{../works/Lunardi20.pdf}{Lunardi20} (223.00)& \cellcolor{red!40}\href{../works/Groleaz21.pdf}{Groleaz21} (223.00)& \cellcolor{red!40}\href{../works/NaderiRR23.pdf}{NaderiRR23} (216.00)\\
Cosine& \cellcolor{red!40}\href{../works/MengGRZSC22.pdf}{MengGRZSC22} (0.83)& \cellcolor{red!40}\href{../works/MengLZB21.pdf}{MengLZB21} (0.83)& \cellcolor{red!40}\href{../works/IsikYA23.pdf}{IsikYA23} (0.82)& \cellcolor{red!40}\href{../works/Novas19.pdf}{Novas19} (0.80)& \cellcolor{red!40}\href{../works/HamPK21.pdf}{HamPK21} (0.78)\\
\index{Mercier-AubinGQ20}\href{../works/Mercier-AubinGQ20.pdf}{Mercier-AubinGQ20} R\&C& \cellcolor{red!40}\href{../works/DerrienPZ14.pdf}{DerrienPZ14} (0.80)& \cellcolor{red!20}\href{../works/LombardiM09.pdf}{LombardiM09} (0.89)& \cellcolor{yellow!20}\href{../works/SzerediS16.pdf}{SzerediS16} (0.93)& \cellcolor{green!20}\href{../works/Laborie09.pdf}{Laborie09} (0.93)& \cellcolor{green!20}\href{../works/GrimesH11.pdf}{GrimesH11} (0.94)\\
Euclid& \cellcolor{black!20}\href{../works/Hooker06.pdf}{Hooker06} (0.34)& \cellcolor{black!20}\href{../works/Hooker05a.pdf}{Hooker05a} (0.35)& \cellcolor{black!20}\href{../works/MonetteDH09.pdf}{MonetteDH09} (0.35)& \href{../works/PengLC14.pdf}{PengLC14} (0.38)& \href{../works/KovacsV06.pdf}{KovacsV06} (0.38)\\
Dot& \cellcolor{red!40}\href{../works/Dejemeppe16.pdf}{Dejemeppe16} (172.00)& \cellcolor{red!40}\href{../works/Groleaz21.pdf}{Groleaz21} (164.00)& \cellcolor{red!40}\href{../works/Lombardi10.pdf}{Lombardi10} (161.00)& \cellcolor{red!40}\href{../works/Baptiste02.pdf}{Baptiste02} (156.00)& \cellcolor{red!40}\href{../works/ZarandiASC20.pdf}{ZarandiASC20} (150.00)\\
Cosine& \cellcolor{red!40}\href{../works/MonetteDH09.pdf}{MonetteDH09} (0.73)& \cellcolor{red!40}\href{../works/Hooker06.pdf}{Hooker06} (0.73)& \cellcolor{red!40}\href{../works/LombardiM12.pdf}{LombardiM12} (0.72)& \cellcolor{red!40}\href{../works/Hooker05a.pdf}{Hooker05a} (0.72)& \cellcolor{red!40}\href{../works/KelbelH11.pdf}{KelbelH11} (0.70)\\
\index{MercierH07}\href{../works/MercierH07.pdf}{MercierH07} R\&C& \cellcolor{red!40}\href{../works/MercierH08.pdf}{MercierH08} (0.72)& \cellcolor{red!40}\href{../works/VilimBC04.pdf}{VilimBC04} (0.78)& \cellcolor{red!40}\href{../works/Puget95.pdf}{Puget95} (0.80)& \cellcolor{red!40}\href{../works/Vilim05.pdf}{Vilim05} (0.81)& \cellcolor{red!40}\href{../works/SchuttWS05.pdf}{SchuttWS05} (0.83)\\
Euclid& \cellcolor{blue!20}\href{../works/MercierH08.pdf}{MercierH08} (0.32)& \cellcolor{blue!20}\href{../works/VilimBC05.pdf}{VilimBC05} (0.32)& \cellcolor{blue!20}\href{../works/VilimBC04.pdf}{VilimBC04} (0.32)& \cellcolor{blue!20}\href{../works/TorresL00.pdf}{TorresL00} (0.33)& \cellcolor{blue!20}\href{../works/Zhou96.pdf}{Zhou96} (0.33)\\
Dot& \cellcolor{red!40}\href{../works/Baptiste02.pdf}{Baptiste02} (170.00)& \cellcolor{red!40}\href{../works/Lombardi10.pdf}{Lombardi10} (162.00)& \cellcolor{red!40}\href{../works/Fahimi16.pdf}{Fahimi16} (158.00)& \cellcolor{red!40}\href{../works/Beck99.pdf}{Beck99} (154.00)& \cellcolor{red!40}\href{../works/Schutt11.pdf}{Schutt11} (152.00)\\
Cosine& \cellcolor{red!40}\href{../works/MercierH08.pdf}{MercierH08} (0.77)& \cellcolor{red!40}\href{../works/VilimBC05.pdf}{VilimBC05} (0.76)& \cellcolor{red!40}\href{../works/TorresL00.pdf}{TorresL00} (0.76)& \cellcolor{red!40}\href{../works/VilimBC04.pdf}{VilimBC04} (0.75)& \cellcolor{red!40}\href{../works/Zhou97.pdf}{Zhou97} (0.74)\\
\index{MercierH08}\href{../works/MercierH08.pdf}{MercierH08} R\&C& \cellcolor{red!40}\href{../works/Vilim09.pdf}{Vilim09} (0.68)& \cellcolor{red!40}\href{../works/MercierH07.pdf}{MercierH07} (0.72)& \cellcolor{red!40}\href{../works/SchuttW10.pdf}{SchuttW10} (0.72)& \cellcolor{red!40}\href{../works/SchuttWS05.pdf}{SchuttWS05} (0.76)& \cellcolor{red!40}\href{../works/KameugneFSN14.pdf}{KameugneFSN14} (0.76)\\
Euclid& \cellcolor{red!40}\href{../works/SchuttWS05.pdf}{SchuttWS05} (0.24)& \cellcolor{red!20}\href{../works/BeldiceanuP07.pdf}{BeldiceanuP07} (0.24)& \cellcolor{red!20}\href{../works/Caseau97.pdf}{Caseau97} (0.26)& \cellcolor{yellow!20}\href{../works/PoderB08.pdf}{PoderB08} (0.26)& \cellcolor{yellow!20}\href{../works/Kameugne15.pdf}{Kameugne15} (0.27)\\
Dot& \cellcolor{red!40}\href{../works/Lombardi10.pdf}{Lombardi10} (95.00)& \cellcolor{red!40}\href{../works/Baptiste02.pdf}{Baptiste02} (95.00)& \cellcolor{red!40}\href{../works/MercierH07.pdf}{MercierH07} (92.00)& \cellcolor{red!40}\href{../works/Dejemeppe16.pdf}{Dejemeppe16} (88.00)& \cellcolor{red!40}\href{../works/BartakSR10.pdf}{BartakSR10} (88.00)\\
Cosine& \cellcolor{red!40}\href{../works/SchuttWS05.pdf}{SchuttWS05} (0.78)& \cellcolor{red!40}\href{../works/MercierH07.pdf}{MercierH07} (0.77)& \cellcolor{red!40}\href{../works/BeldiceanuP07.pdf}{BeldiceanuP07} (0.74)& \cellcolor{red!40}\href{../works/KameugneFSN14.pdf}{KameugneFSN14} (0.72)& \cellcolor{red!40}\href{../works/Caseau97.pdf}{Caseau97} (0.70)\\
\index{MeskensDHG11}\href{../works/MeskensDHG11.pdf}{MeskensDHG11} R\&C\\
Euclid& \cellcolor{red!40}\href{../works/MeskensDL13.pdf}{MeskensDL13} (0.19)& \cellcolor{red!40}\href{../works/WangMD15.pdf}{WangMD15} (0.23)& \cellcolor{yellow!20}\href{../works/GurPAE23.pdf}{GurPAE23} (0.27)& \cellcolor{yellow!20}\href{../works/GurEA19.pdf}{GurEA19} (0.28)& \cellcolor{green!20}\href{../works/DoulabiRP16.pdf}{DoulabiRP16} (0.30)\\
Dot& \cellcolor{red!40}\href{../works/MeskensDL13.pdf}{MeskensDL13} (108.00)& \cellcolor{red!40}\href{../works/WangMD15.pdf}{WangMD15} (104.00)& \cellcolor{red!40}\href{../works/YounespourAKE19.pdf}{YounespourAKE19} (95.00)& \cellcolor{red!40}\href{../works/ZarandiASC20.pdf}{ZarandiASC20} (95.00)& \cellcolor{red!40}\href{../works/FarsiTM22.pdf}{FarsiTM22} (85.00)\\
Cosine& \cellcolor{red!40}\href{../works/MeskensDL13.pdf}{MeskensDL13} (0.90)& \cellcolor{red!40}\href{../works/WangMD15.pdf}{WangMD15} (0.85)& \cellcolor{red!40}\href{../works/GurPAE23.pdf}{GurPAE23} (0.76)& \cellcolor{red!40}\href{../works/GurEA19.pdf}{GurEA19} (0.74)& \cellcolor{red!40}\href{../works/YounespourAKE19.pdf}{YounespourAKE19} (0.74)\\
\index{MeskensDL13}\href{../works/MeskensDL13.pdf}{MeskensDL13} R\&C& \cellcolor{red!40}\href{../works/WangMD15.pdf}{WangMD15} (0.63)& \cellcolor{red!40}\href{../works/ZhaoL14.pdf}{ZhaoL14} (0.73)& \cellcolor{red!40}RoshanaeiLAU17a (0.81)& \cellcolor{red!40}\href{../works/RoshanaeiLAU17.pdf}{RoshanaeiLAU17} (0.81)& \cellcolor{red!40}\href{../works/DoulabiRP16.pdf}{DoulabiRP16} (0.85)\\
Euclid& \cellcolor{red!40}\href{../works/MeskensDHG11.pdf}{MeskensDHG11} (0.19)& \cellcolor{red!40}\href{../works/WangMD15.pdf}{WangMD15} (0.24)& \cellcolor{yellow!20}\href{../works/GurPAE23.pdf}{GurPAE23} (0.28)& \cellcolor{green!20}\href{../works/GurEA19.pdf}{GurEA19} (0.30)& \cellcolor{blue!20}\href{../works/RiiseML16.pdf}{RiiseML16} (0.33)\\
Dot& \cellcolor{red!40}\href{../works/ZarandiASC20.pdf}{ZarandiASC20} (127.00)& \cellcolor{red!40}\href{../works/WangMD15.pdf}{WangMD15} (118.00)& \cellcolor{red!40}\href{../works/MeskensDHG11.pdf}{MeskensDHG11} (108.00)& \cellcolor{red!40}\href{../works/Dejemeppe16.pdf}{Dejemeppe16} (106.00)& \cellcolor{red!40}\href{../works/YounespourAKE19.pdf}{YounespourAKE19} (104.00)\\
Cosine& \cellcolor{red!40}\href{../works/MeskensDHG11.pdf}{MeskensDHG11} (0.90)& \cellcolor{red!40}\href{../works/WangMD15.pdf}{WangMD15} (0.86)& \cellcolor{red!40}\href{../works/GurPAE23.pdf}{GurPAE23} (0.79)& \cellcolor{red!40}\href{../works/GurEA19.pdf}{GurEA19} (0.76)& \cellcolor{red!40}\href{../works/YounespourAKE19.pdf}{YounespourAKE19} (0.72)\\
\index{MeyerE04}\href{../works/MeyerE04.pdf}{MeyerE04} R\&C& \cellcolor{red!40}\href{../works/ThiruvadyBME09.pdf}{ThiruvadyBME09} (0.80)& \cellcolor{yellow!20}\href{../works/KamarainenS02.pdf}{KamarainenS02} (0.90)& \cellcolor{yellow!20}\href{../works/Wallace06.pdf}{Wallace06} (0.93)& \cellcolor{green!20}\href{../works/ThiruvadyWGS14.pdf}{ThiruvadyWGS14} (0.94)& \cellcolor{green!20}\href{../works/Simonis99.pdf}{Simonis99} (0.94)\\
Euclid& \cellcolor{yellow!20}\href{../works/ThiruvadyBME09.pdf}{ThiruvadyBME09} (0.27)& \cellcolor{blue!20}\href{../works/CarlssonKA99.pdf}{CarlssonKA99} (0.33)& \cellcolor{blue!20}\href{../works/JuvinHL23.pdf}{JuvinHL23} (0.34)& \cellcolor{black!20}\href{../works/PesantGPR99.pdf}{PesantGPR99} (0.34)& \cellcolor{black!20}\href{../works/YuraszeckMC23.pdf}{YuraszeckMC23} (0.34)\\
Dot& \cellcolor{red!40}\href{../works/ZarandiASC20.pdf}{ZarandiASC20} (141.00)& \cellcolor{red!40}\href{../works/Groleaz21.pdf}{Groleaz21} (137.00)& \cellcolor{red!40}\href{../works/Lunardi20.pdf}{Lunardi20} (130.00)& \cellcolor{red!40}\href{../works/Malapert11.pdf}{Malapert11} (123.00)& \cellcolor{red!40}\href{../works/IsikYA23.pdf}{IsikYA23} (122.00)\\
Cosine& \cellcolor{red!40}\href{../works/ThiruvadyBME09.pdf}{ThiruvadyBME09} (0.78)& \cellcolor{red!40}\href{../works/GedikKEK18.pdf}{GedikKEK18} (0.69)& \cellcolor{red!40}\href{../works/LiFJZLL22.pdf}{LiFJZLL22} (0.67)& \cellcolor{red!40}\href{../works/AfsarVPG23.pdf}{AfsarVPG23} (0.67)& \cellcolor{red!40}\href{../works/JuvinHL23.pdf}{JuvinHL23} (0.67)\\
\index{Milano11}Milano11 R\&C& \cellcolor{red!40}\href{../works/MilanoW06.pdf}{MilanoW06} (0.84)& \cellcolor{red!40}\href{../works/MilanoW09.pdf}{MilanoW09} (0.85)& \cellcolor{yellow!20}\href{../works/ChuX05.pdf}{ChuX05} (0.90)& \cellcolor{yellow!20}\href{../works/Wallace06.pdf}{Wallace06} (0.90)& \cellcolor{yellow!20}\href{../works/EreminW01.pdf}{EreminW01} (0.91)\\
Euclid\\
Dot\\
Cosine\\
\index{MilanoORT02}MilanoORT02 R\&C& \cellcolor{red!40}\href{../works/Simonis99.pdf}{Simonis99} (0.85)& \cellcolor{red!40}\href{../works/HookerY02.pdf}{HookerY02} (0.86)& \cellcolor{red!20}\href{../works/Hooker05b.pdf}{Hooker05b} (0.86)& \cellcolor{red!20}\href{../works/Refalo00.pdf}{Refalo00} (0.87)& \cellcolor{red!20}\href{../works/MonetteDD07.pdf}{MonetteDD07} (0.88)\\
Euclid\\
Dot\\
Cosine\\
\index{MilanoW06}\href{../works/MilanoW06.pdf}{MilanoW06} R\&C& \cellcolor{red!40}\href{../works/MilanoW09.pdf}{MilanoW09} (0.61)& \cellcolor{red!40}Milano11 (0.84)& \cellcolor{red!20}\href{../works/Wallace06.pdf}{Wallace06} (0.88)& \cellcolor{red!20}\href{../works/ChuX05.pdf}{ChuX05} (0.88)& \cellcolor{yellow!20}\href{../works/Gronkvist06.pdf}{Gronkvist06} (0.90)\\
Euclid& \cellcolor{red!40}\href{../works/MilanoW09.pdf}{MilanoW09} (0.14)& \href{../works/JainG01.pdf}{JainG01} (0.40)& \href{../works/Hooker19.pdf}{Hooker19} (0.41)& \href{../works/Hooker06.pdf}{Hooker06} (0.42)& \href{../works/Hooker05a.pdf}{Hooker05a} (0.42)\\
Dot& \cellcolor{red!40}\href{../works/MilanoW09.pdf}{MilanoW09} (217.00)& \cellcolor{red!40}\href{../works/ZarandiASC20.pdf}{ZarandiASC20} (179.00)& \cellcolor{red!40}\href{../works/Dejemeppe16.pdf}{Dejemeppe16} (177.00)& \cellcolor{red!40}\href{../works/Groleaz21.pdf}{Groleaz21} (174.00)& \cellcolor{red!40}\href{../works/Lombardi10.pdf}{Lombardi10} (169.00)\\
Cosine& \cellcolor{red!40}\href{../works/MilanoW09.pdf}{MilanoW09} (0.97)& \cellcolor{red!40}\href{../works/Hooker19.pdf}{Hooker19} (0.71)& \cellcolor{red!40}\href{../works/JainG01.pdf}{JainG01} (0.70)& \cellcolor{red!40}\href{../works/QinDS16.pdf}{QinDS16} (0.67)& \cellcolor{red!40}\href{../works/Hooker06.pdf}{Hooker06} (0.67)\\
\index{MilanoW09}\href{../works/MilanoW09.pdf}{MilanoW09} R\&C& \cellcolor{red!40}\href{../works/MilanoW06.pdf}{MilanoW06} (0.61)& \cellcolor{red!40}\href{../works/AchterbergBKW08.pdf}{AchterbergBKW08} (0.81)& \cellcolor{red!40}\href{../works/GuyonLPR12.pdf}{GuyonLPR12} (0.84)& \cellcolor{red!40}Milano11 (0.85)& \cellcolor{red!20}\href{../works/Wallace06.pdf}{Wallace06} (0.90)\\
Euclid& \cellcolor{red!40}\href{../works/MilanoW06.pdf}{MilanoW06} (0.14)& \href{../works/Hooker19.pdf}{Hooker19} (0.43)& \href{../works/Hooker06.pdf}{Hooker06} (0.44)& \href{../works/JainG01.pdf}{JainG01} (0.44)& \href{../works/HeinzKB13.pdf}{HeinzKB13} (0.44)\\
Dot& \cellcolor{red!40}\href{../works/MilanoW06.pdf}{MilanoW06} (217.00)& \cellcolor{red!40}\href{../works/Dejemeppe16.pdf}{Dejemeppe16} (178.00)& \cellcolor{red!40}\href{../works/ZarandiASC20.pdf}{ZarandiASC20} (176.00)& \cellcolor{red!40}\href{../works/Groleaz21.pdf}{Groleaz21} (174.00)& \cellcolor{red!40}\href{../works/Lombardi10.pdf}{Lombardi10} (172.00)\\
Cosine& \cellcolor{red!40}\href{../works/MilanoW06.pdf}{MilanoW06} (0.97)& \cellcolor{red!40}\href{../works/Hooker19.pdf}{Hooker19} (0.69)& \cellcolor{red!40}\href{../works/HookerH17.pdf}{HookerH17} (0.67)& \cellcolor{red!40}\href{../works/QinDS16.pdf}{QinDS16} (0.66)& \cellcolor{red!40}\href{../works/Hooker06.pdf}{Hooker06} (0.65)\\
\index{MintonJPL90}\href{../works/MintonJPL90.pdf}{MintonJPL90} R\&C\\
Euclid& \cellcolor{red!40}\href{../works/MintonJPL92.pdf}{MintonJPL92} (0.20)& \cellcolor{green!20}\href{../works/FeldmanG89.pdf}{FeldmanG89} (0.29)& \cellcolor{green!20}\href{../works/FukunagaHFAMN02.pdf}{FukunagaHFAMN02} (0.30)& \cellcolor{green!20}\href{../works/LudwigKRBMS14.pdf}{LudwigKRBMS14} (0.30)& \cellcolor{green!20}\href{../works/AngelsmarkJ00.pdf}{AngelsmarkJ00} (0.30)\\
Dot& \cellcolor{red!40}\href{../works/ZarandiASC20.pdf}{ZarandiASC20} (93.00)& \cellcolor{red!40}\href{../works/MintonJPL92.pdf}{MintonJPL92} (82.00)& \cellcolor{red!40}\href{../works/Beck99.pdf}{Beck99} (82.00)& \cellcolor{red!40}\href{../works/Lombardi10.pdf}{Lombardi10} (80.00)& \cellcolor{red!40}\href{../works/Astrand21.pdf}{Astrand21} (78.00)\\
Cosine& \cellcolor{red!40}\href{../works/MintonJPL92.pdf}{MintonJPL92} (0.86)& \cellcolor{red!40}\href{../works/Prosser89.pdf}{Prosser89} (0.62)& \cellcolor{red!40}\href{../works/EskeyZ90.pdf}{EskeyZ90} (0.62)& \cellcolor{red!40}\href{../works/Salido10.pdf}{Salido10} (0.61)& \cellcolor{red!40}\href{../works/ReddyFIBKAJ11.pdf}{ReddyFIBKAJ11} (0.60)\\
\index{MintonJPL92}\href{../works/MintonJPL92.pdf}{MintonJPL92} R\&C& \cellcolor{red!20}\href{../works/SadehF96.pdf}{SadehF96} (0.89)& \cellcolor{green!20}\href{../works/SmithBHW96.pdf}{SmithBHW96} (0.94)& \cellcolor{green!20}\href{../works/NuijtenA96.pdf}{NuijtenA96} (0.94)& \cellcolor{green!20}\href{../works/BartakSR08.pdf}{BartakSR08} (0.94)& \cellcolor{green!20}\href{../works/Darby-DowmanLMZ97.pdf}{Darby-DowmanLMZ97} (0.94)\\
Euclid& \cellcolor{red!40}\href{../works/MintonJPL90.pdf}{MintonJPL90} (0.20)& \cellcolor{yellow!20}\href{../works/Prosser89.pdf}{Prosser89} (0.27)& \cellcolor{yellow!20}\href{../works/FukunagaHFAMN02.pdf}{FukunagaHFAMN02} (0.28)& \cellcolor{yellow!20}\href{../works/KengY89.pdf}{KengY89} (0.28)& \cellcolor{green!20}\href{../works/Salido10.pdf}{Salido10} (0.30)\\
Dot& \cellcolor{red!40}\href{../works/ZarandiASC20.pdf}{ZarandiASC20} (105.00)& \cellcolor{red!40}\href{../works/Beck99.pdf}{Beck99} (98.00)& \cellcolor{red!40}\href{../works/Lombardi10.pdf}{Lombardi10} (97.00)& \cellcolor{red!40}\href{../works/Groleaz21.pdf}{Groleaz21} (95.00)& \cellcolor{red!40}\href{../works/Astrand21.pdf}{Astrand21} (94.00)\\
Cosine& \cellcolor{red!40}\href{../works/MintonJPL90.pdf}{MintonJPL90} (0.86)& \cellcolor{red!40}\href{../works/Prosser89.pdf}{Prosser89} (0.72)& \cellcolor{red!40}\href{../works/KengY89.pdf}{KengY89} (0.69)& \cellcolor{red!40}\href{../works/Salido10.pdf}{Salido10} (0.68)& \cellcolor{red!40}\href{../works/FukunagaHFAMN02.pdf}{FukunagaHFAMN02} (0.68)\\
\index{MoffittPP05}\href{../works/MoffittPP05.pdf}{MoffittPP05} R\&C\\
Euclid& \cellcolor{red!40}\href{../works/Valdes87.pdf}{Valdes87} (0.20)& \cellcolor{red!40}\href{../works/FeldmanG89.pdf}{FeldmanG89} (0.20)& \cellcolor{red!40}\href{../works/LiuJ06.pdf}{LiuJ06} (0.21)& \cellcolor{red!40}\href{../works/Davis87.pdf}{Davis87} (0.21)& \cellcolor{red!40}\href{../works/FrostD98.pdf}{FrostD98} (0.22)\\
Dot& \cellcolor{red!40}\href{../works/Siala15a.pdf}{Siala15a} (57.00)& \cellcolor{red!40}\href{../works/Godet21a.pdf}{Godet21a} (56.00)& \cellcolor{red!40}\href{../works/Malapert11.pdf}{Malapert11} (56.00)& \cellcolor{red!40}\href{../works/Baptiste02.pdf}{Baptiste02} (56.00)& \cellcolor{red!40}\href{../works/PapaB98.pdf}{PapaB98} (51.00)\\
Cosine& \cellcolor{red!40}\href{../works/OddiRC10.pdf}{OddiRC10} (0.62)& \cellcolor{red!40}\href{../works/BartakCS10.pdf}{BartakCS10} (0.61)& \cellcolor{red!40}\href{../works/Bartak02.pdf}{Bartak02} (0.60)& \cellcolor{red!40}\href{../works/FeldmanG89.pdf}{FeldmanG89} (0.60)& \cellcolor{red!40}\href{../works/Valdes87.pdf}{Valdes87} (0.59)\\
\index{MokhtarzadehTNF20}\href{../works/MokhtarzadehTNF20.pdf}{MokhtarzadehTNF20} R\&C& \cellcolor{yellow!20}RabbaniMM21 (0.93)& \cellcolor{yellow!20}\href{../works/HamP21.pdf}{HamP21} (0.93)& \cellcolor{green!20}PinarbasiA20 (0.94)& \cellcolor{green!20}\href{../works/YunusogluY22.pdf}{YunusogluY22} (0.94)& \cellcolor{green!20}\href{../works/Edis21.pdf}{Edis21} (0.95)\\
Euclid& \cellcolor{yellow!20}\href{../works/WessenCS20.pdf}{WessenCS20} (0.28)& \cellcolor{green!20}\href{../works/BehrensLM19.pdf}{BehrensLM19} (0.30)& \cellcolor{green!20}\href{../works/abs-1901-07914.pdf}{abs-1901-07914} (0.31)& \cellcolor{blue!20}\href{../works/NishikawaSTT19.pdf}{NishikawaSTT19} (0.33)& \cellcolor{blue!20}\href{../works/NishikawaSTT18.pdf}{NishikawaSTT18} (0.33)\\
Dot& \cellcolor{red!40}\href{../works/ZarandiASC20.pdf}{ZarandiASC20} (139.00)& \cellcolor{red!40}\href{../works/Astrand21.pdf}{Astrand21} (127.00)& \cellcolor{red!40}\href{../works/IsikYA23.pdf}{IsikYA23} (126.00)& \cellcolor{red!40}\href{../works/Lunardi20.pdf}{Lunardi20} (125.00)& \cellcolor{red!40}\href{../works/PrataAN23.pdf}{PrataAN23} (118.00)\\
Cosine& \cellcolor{red!40}\href{../works/WessenCS20.pdf}{WessenCS20} (0.76)& \cellcolor{red!40}\href{../works/BehrensLM19.pdf}{BehrensLM19} (0.72)& \cellcolor{red!40}\href{../works/Fatemi-AnarakiTFV23.pdf}{Fatemi-AnarakiTFV23} (0.72)& \cellcolor{red!40}\href{../works/abs-1901-07914.pdf}{abs-1901-07914} (0.71)& \cellcolor{red!40}\href{../works/HeinzNVH22.pdf}{HeinzNVH22} (0.71)\\
\index{MonetteDD07}\href{../works/MonetteDD07.pdf}{MonetteDD07} R\&C& \cellcolor{red!40}\href{../works/ArtiouchineB05.pdf}{ArtiouchineB05} (0.74)& \cellcolor{red!40}\href{../works/Vilim05.pdf}{Vilim05} (0.80)& \cellcolor{red!40}\href{../works/Wolf05.pdf}{Wolf05} (0.80)& \cellcolor{red!40}\href{../works/SourdN00.pdf}{SourdN00} (0.81)& \cellcolor{red!40}DorndorfHP99 (0.81)\\
Euclid& \cellcolor{red!40}\href{../works/Wolf03.pdf}{Wolf03} (0.24)& \cellcolor{yellow!20}\href{../works/Wolf05.pdf}{Wolf05} (0.27)& \cellcolor{green!20}\href{../works/TanSD10.pdf}{TanSD10} (0.29)& \cellcolor{green!20}\href{../works/MalapertCGJLR13.pdf}{MalapertCGJLR13} (0.30)& \cellcolor{green!20}\href{../works/Vilim05.pdf}{Vilim05} (0.30)\\
Dot& \cellcolor{red!40}\href{../works/Fahimi16.pdf}{Fahimi16} (144.00)& \cellcolor{red!40}\href{../works/Baptiste02.pdf}{Baptiste02} (143.00)& \cellcolor{red!40}\href{../works/Schutt11.pdf}{Schutt11} (137.00)& \cellcolor{red!40}\href{../works/Dejemeppe16.pdf}{Dejemeppe16} (136.00)& \cellcolor{red!40}\href{../works/Malapert11.pdf}{Malapert11} (136.00)\\
Cosine& \cellcolor{red!40}\href{../works/Wolf03.pdf}{Wolf03} (0.86)& \cellcolor{red!40}\href{../works/Wolf05.pdf}{Wolf05} (0.81)& \cellcolor{red!40}\href{../works/FahimiOQ18.pdf}{FahimiOQ18} (0.78)& \cellcolor{red!40}\href{../works/TanSD10.pdf}{TanSD10} (0.76)& \cellcolor{red!40}\href{../works/MalapertCGJLR12.pdf}{MalapertCGJLR12} (0.76)\\
\index{MonetteDH09}\href{../works/MonetteDH09.pdf}{MonetteDH09} R\&C& \cellcolor{red!20}\href{../works/ZhangLS12.pdf}{ZhangLS12} (0.90)& \cellcolor{yellow!20}\href{../works/KelbelH11.pdf}{KelbelH11} (0.90)& \cellcolor{yellow!20}\href{../works/QuirogaZH05.pdf}{QuirogaZH05} (0.91)& \cellcolor{yellow!20}\href{../works/Geske05.pdf}{Geske05} (0.91)& \cellcolor{yellow!20}\href{../works/SchausHMCMD11.pdf}{SchausHMCMD11} (0.91)\\
Euclid& \cellcolor{yellow!20}\href{../works/HeipckeCCS00.pdf}{HeipckeCCS00} (0.28)& \cellcolor{yellow!20}\href{../works/HentenryckM04.pdf}{HentenryckM04} (0.28)& \cellcolor{yellow!20}\href{../works/BeckPS03.pdf}{BeckPS03} (0.28)& \cellcolor{green!20}\href{../works/PacinoH11.pdf}{PacinoH11} (0.29)& \cellcolor{green!20}\href{../works/LaborieR14.pdf}{LaborieR14} (0.29)\\
Dot& \cellcolor{red!40}\href{../works/Dejemeppe16.pdf}{Dejemeppe16} (166.00)& \cellcolor{red!40}\href{../works/Groleaz21.pdf}{Groleaz21} (157.00)& \cellcolor{red!40}\href{../works/Baptiste02.pdf}{Baptiste02} (152.00)& \cellcolor{red!40}\href{../works/ZarandiASC20.pdf}{ZarandiASC20} (148.00)& \cellcolor{red!40}\href{../works/Lombardi10.pdf}{Lombardi10} (148.00)\\
Cosine& \cellcolor{red!40}\href{../works/KelbelH11.pdf}{KelbelH11} (0.81)& \cellcolor{red!40}\href{../works/BeckPS03.pdf}{BeckPS03} (0.81)& \cellcolor{red!40}\href{../works/HeipckeCCS00.pdf}{HeipckeCCS00} (0.80)& \cellcolor{red!40}\href{../works/HentenryckM04.pdf}{HentenryckM04} (0.80)& \cellcolor{red!40}\href{../works/LaborieR14.pdf}{LaborieR14} (0.80)\\
\index{MontemanniD23}\href{../works/MontemanniD23.pdf}{MontemanniD23} R\&C& \cellcolor{red!40}\href{../works/MontemanniD23a.pdf}{MontemanniD23a} (0.63)& \cellcolor{blue!20}\href{../works/Ham18.pdf}{Ham18} (0.97)& \cellcolor{black!20}\href{../works/CauwelaertLS18.pdf}{CauwelaertLS18} (0.98)& \cellcolor{black!20}BalochG20 (0.98)\\
Euclid& \cellcolor{red!40}\href{../works/MontemanniD23a.pdf}{MontemanniD23a} (0.19)& \cellcolor{yellow!20}\href{../works/GomesHS06.pdf}{GomesHS06} (0.26)& \cellcolor{yellow!20}\href{../works/BarzegaranZP20.pdf}{BarzegaranZP20} (0.27)& \cellcolor{yellow!20}\href{../works/QuSN06.pdf}{QuSN06} (0.27)& \cellcolor{yellow!20}\href{../works/AkramNHRSA23.pdf}{AkramNHRSA23} (0.27)\\
Dot& \cellcolor{red!40}\href{../works/Groleaz21.pdf}{Groleaz21} (71.00)& \cellcolor{red!40}\href{../works/abs-2402-00459.pdf}{abs-2402-00459} (69.00)& \cellcolor{red!40}\href{../works/Lunardi20.pdf}{Lunardi20} (69.00)& \cellcolor{red!40}\href{../works/ZarandiASC20.pdf}{ZarandiASC20} (68.00)& \cellcolor{red!40}\href{../works/FachiniA20.pdf}{FachiniA20} (64.00)\\
Cosine& \cellcolor{red!40}\href{../works/MontemanniD23a.pdf}{MontemanniD23a} (0.82)& \cellcolor{red!40}\href{../works/BarzegaranZP20.pdf}{BarzegaranZP20} (0.65)& \cellcolor{red!40}\href{../works/AkramNHRSA23.pdf}{AkramNHRSA23} (0.64)& \cellcolor{red!40}\href{../works/FachiniA20.pdf}{FachiniA20} (0.63)& \cellcolor{red!40}\href{../works/QuSN06.pdf}{QuSN06} (0.61)\\
\index{MontemanniD23a}\href{../works/MontemanniD23a.pdf}{MontemanniD23a} R\&C& \cellcolor{red!40}\href{../works/MontemanniD23.pdf}{MontemanniD23} (0.63)& \cellcolor{blue!20}\href{../works/Ham18.pdf}{Ham18} (0.96)& \cellcolor{black!20}\href{../works/CauwelaertLS18.pdf}{CauwelaertLS18} (0.98)& \cellcolor{black!20}BalochG20 (0.98)\\
Euclid& \cellcolor{red!40}\href{../works/MontemanniD23.pdf}{MontemanniD23} (0.19)& \cellcolor{red!20}\href{../works/BarzegaranZP20.pdf}{BarzegaranZP20} (0.26)& \cellcolor{yellow!20}\href{../works/ZibranR11.pdf}{ZibranR11} (0.27)& \cellcolor{yellow!20}\href{../works/AkramNHRSA23.pdf}{AkramNHRSA23} (0.28)& \cellcolor{yellow!20}\href{../works/GomesHS06.pdf}{GomesHS06} (0.28)\\
Dot& \cellcolor{red!40}\href{../works/Groleaz21.pdf}{Groleaz21} (72.00)& \cellcolor{red!40}\href{../works/Froger16.pdf}{Froger16} (65.00)& \cellcolor{red!40}\href{../works/Schutt11.pdf}{Schutt11} (64.00)& \cellcolor{red!40}\href{../works/Astrand21.pdf}{Astrand21} (63.00)& \cellcolor{red!40}\href{../works/Lunardi20.pdf}{Lunardi20} (63.00)\\
Cosine& \cellcolor{red!40}\href{../works/MontemanniD23.pdf}{MontemanniD23} (0.82)& \cellcolor{red!40}\href{../works/BarzegaranZP20.pdf}{BarzegaranZP20} (0.67)& \cellcolor{red!40}\href{../works/AkramNHRSA23.pdf}{AkramNHRSA23} (0.62)& \cellcolor{red!40}\href{../works/Rodriguez07.pdf}{Rodriguez07} (0.60)& \cellcolor{red!40}\href{../works/ZibranR11.pdf}{ZibranR11} (0.59)\\
\index{MorgadoM97}\href{../works/MorgadoM97.pdf}{MorgadoM97} R\&C\\
Euclid& \cellcolor{green!20}\href{../works/WallaceF00.pdf}{WallaceF00} (0.29)& \cellcolor{green!20}\href{../works/CrawfordB94.pdf}{CrawfordB94} (0.29)& \cellcolor{green!20}\href{../works/BridiLBBM16.pdf}{BridiLBBM16} (0.29)& \cellcolor{green!20}\href{../works/LuoVLBM16.pdf}{LuoVLBM16} (0.29)& \cellcolor{green!20}\href{../works/LouieVNB14.pdf}{LouieVNB14} (0.30)\\
Dot& \cellcolor{red!40}\href{../works/ZarandiASC20.pdf}{ZarandiASC20} (105.00)& \cellcolor{red!40}\href{../works/LaborieRSV18.pdf}{LaborieRSV18} (91.00)& \cellcolor{red!40}\href{../works/Beck99.pdf}{Beck99} (90.00)& \cellcolor{red!40}\href{../works/Astrand21.pdf}{Astrand21} (85.00)& \cellcolor{red!40}\href{../works/Godet21a.pdf}{Godet21a} (85.00)\\
Cosine& \cellcolor{red!40}\href{../works/CappartS17.pdf}{CappartS17} (0.72)& \cellcolor{red!40}\href{../works/Pape94.pdf}{Pape94} (0.69)& \cellcolor{red!40}\href{../works/BeckPS03.pdf}{BeckPS03} (0.69)& \cellcolor{red!40}\href{../works/Rodriguez07.pdf}{Rodriguez07} (0.68)& \cellcolor{red!40}\href{../works/BridiLBBM16.pdf}{BridiLBBM16} (0.66)\\
\index{MossigeGSMC17}\href{../works/MossigeGSMC17.pdf}{MossigeGSMC17} R\&C& \cellcolor{red!40}\href{../works/SzerediS16.pdf}{SzerediS16} (0.83)& \cellcolor{red!20}\href{../works/AmadiniGM16.pdf}{AmadiniGM16} (0.88)& \cellcolor{yellow!20}SchuttFSW15 (0.90)& \cellcolor{yellow!20}\href{../works/BeldiceanuC02.pdf}{BeldiceanuC02} (0.90)& \cellcolor{yellow!20}\href{../works/BeldiceanuCDP11.pdf}{BeldiceanuCDP11} (0.90)\\
Euclid& \cellcolor{blue!20}\href{../works/HeipckeCCS00.pdf}{HeipckeCCS00} (0.33)& \cellcolor{black!20}\href{../works/KovacsV06.pdf}{KovacsV06} (0.34)& \cellcolor{black!20}\href{../works/TrojetHL11.pdf}{TrojetHL11} (0.36)& \cellcolor{black!20}\href{../works/SchuttFS13.pdf}{SchuttFS13} (0.36)& \cellcolor{black!20}\href{../works/VilimLS15.pdf}{VilimLS15} (0.36)\\
Dot& \cellcolor{red!40}\href{../works/Dejemeppe16.pdf}{Dejemeppe16} (171.00)& \cellcolor{red!40}\href{../works/ZarandiASC20.pdf}{ZarandiASC20} (164.00)& \cellcolor{red!40}\href{../works/Schutt11.pdf}{Schutt11} (162.00)& \cellcolor{red!40}\href{../works/Godet21a.pdf}{Godet21a} (161.00)& \cellcolor{red!40}\href{../works/Groleaz21.pdf}{Groleaz21} (157.00)\\
Cosine& \cellcolor{red!40}\href{../works/HeipckeCCS00.pdf}{HeipckeCCS00} (0.75)& \cellcolor{red!40}\href{../works/VilimLS15.pdf}{VilimLS15} (0.74)& \cellcolor{red!40}\href{../works/KovacsV06.pdf}{KovacsV06} (0.74)& \cellcolor{red!40}\href{../works/SchuttFS13.pdf}{SchuttFS13} (0.73)& \cellcolor{red!40}\href{../works/TrojetHL11.pdf}{TrojetHL11} (0.73)\\
\index{MouraSCL08}\href{../works/MouraSCL08.pdf}{MouraSCL08} R\&C& \cellcolor{red!40}\href{../works/LopesCSM10.pdf}{LopesCSM10} (0.52)& \cellcolor{red!40}\href{../works/MouraSCL08a.pdf}{MouraSCL08a} (0.60)& \cellcolor{green!20}FelizariAL09 (0.94)& \cellcolor{green!20}MagataoAN05 (0.94)& \cellcolor{green!20}Henz01 (0.95)\\
Euclid& \cellcolor{red!20}\href{../works/LopesCSM10.pdf}{LopesCSM10} (0.24)& \cellcolor{red!20}\href{../works/Muscettola02.pdf}{Muscettola02} (0.25)& \cellcolor{red!20}\href{../works/BeniniBGM05a.pdf}{BeniniBGM05a} (0.26)& \cellcolor{red!20}\href{../works/WallaceF00.pdf}{WallaceF00} (0.26)& \cellcolor{red!20}\href{../works/MouraSCL08a.pdf}{MouraSCL08a} (0.26)\\
Dot& \cellcolor{red!40}\href{../works/LopesCSM10.pdf}{LopesCSM10} (87.00)& \cellcolor{red!40}\href{../works/Lombardi10.pdf}{Lombardi10} (86.00)& \cellcolor{red!40}\href{../works/Malapert11.pdf}{Malapert11} (79.00)& \cellcolor{red!40}\href{../works/ZarandiASC20.pdf}{ZarandiASC20} (78.00)& \cellcolor{red!40}\href{../works/Astrand21.pdf}{Astrand21} (77.00)\\
Cosine& \cellcolor{red!40}\href{../works/LopesCSM10.pdf}{LopesCSM10} (0.83)& \cellcolor{red!40}\href{../works/MouraSCL08a.pdf}{MouraSCL08a} (0.72)& \cellcolor{red!40}\href{../works/Muscettola02.pdf}{Muscettola02} (0.70)& \cellcolor{red!40}\href{../works/BofillCSV17a.pdf}{BofillCSV17a} (0.70)& \cellcolor{red!40}\href{../works/BeniniBGM05a.pdf}{BeniniBGM05a} (0.68)\\
\index{MouraSCL08a}\href{../works/MouraSCL08a.pdf}{MouraSCL08a} R\&C& \cellcolor{red!40}\href{../works/LopesCSM10.pdf}{LopesCSM10} (0.60)& \cellcolor{red!40}\href{../works/MouraSCL08.pdf}{MouraSCL08} (0.60)& \cellcolor{red!40}FelizariAL09 (0.77)& \cellcolor{red!40}MagataoAN05 (0.77)& \cellcolor{green!20}Henz01 (0.96)\\
Euclid& \cellcolor{red!20}\href{../works/MouraSCL08.pdf}{MouraSCL08} (0.26)& \cellcolor{yellow!20}\href{../works/LopesCSM10.pdf}{LopesCSM10} (0.26)& \cellcolor{yellow!20}\href{../works/PesantGPR99.pdf}{PesantGPR99} (0.28)& \cellcolor{green!20}\href{../works/ZibranR11.pdf}{ZibranR11} (0.29)& \cellcolor{green!20}\href{../works/BeniniBGM05a.pdf}{BeniniBGM05a} (0.30)\\
Dot& \cellcolor{red!40}\href{../works/Beck99.pdf}{Beck99} (92.00)& \cellcolor{red!40}\href{../works/HarjunkoskiMBC14.pdf}{HarjunkoskiMBC14} (90.00)& \cellcolor{red!40}\href{../works/Lombardi10.pdf}{Lombardi10} (87.00)& \cellcolor{red!40}\href{../works/LopesCSM10.pdf}{LopesCSM10} (85.00)& \cellcolor{red!40}\href{../works/Astrand21.pdf}{Astrand21} (85.00)\\
Cosine& \cellcolor{red!40}\href{../works/LopesCSM10.pdf}{LopesCSM10} (0.80)& \cellcolor{red!40}\href{../works/MouraSCL08.pdf}{MouraSCL08} (0.72)& \cellcolor{red!40}\href{../works/PesantGPR99.pdf}{PesantGPR99} (0.64)& \cellcolor{red!40}\href{../works/GilesH16.pdf}{GilesH16} (0.62)& \cellcolor{red!40}\href{../works/BourreauGGLT22.pdf}{BourreauGGLT22} (0.61)\\
\index{MullerMKP22}\href{../works/MullerMKP22.pdf}{MullerMKP22} R\&C& \cellcolor{yellow!20}\href{../works/JuvinHL22.pdf}{JuvinHL22} (0.91)& \cellcolor{yellow!20}\href{../works/JuvinHL23a.pdf}{JuvinHL23a} (0.92)& \cellcolor{yellow!20}DomdorfPH03 (0.92)& \cellcolor{yellow!20}DorndorfPH99 (0.92)& \cellcolor{yellow!20}NaderiR22 (0.93)\\
Euclid& \href{../works/IklassovMR023.pdf}{IklassovMR023} (0.40)& \href{../works/HamP21.pdf}{HamP21} (0.41)& \href{../works/abs-2211-14492.pdf}{abs-2211-14492} (0.41)& \href{../works/Teppan22.pdf}{Teppan22} (0.42)& \href{../works/KotaryFH22.pdf}{KotaryFH22} (0.42)\\
Dot& \cellcolor{red!40}\href{../works/ZarandiASC20.pdf}{ZarandiASC20} (185.00)& \cellcolor{red!40}\href{../works/Groleaz21.pdf}{Groleaz21} (183.00)& \cellcolor{red!40}\href{../works/Lunardi20.pdf}{Lunardi20} (170.00)& \cellcolor{red!40}\href{../works/Dejemeppe16.pdf}{Dejemeppe16} (161.00)& \cellcolor{red!40}\href{../works/ColT22.pdf}{ColT22} (160.00)\\
Cosine& \cellcolor{red!40}\href{../works/abs-2211-14492.pdf}{abs-2211-14492} (0.72)& \cellcolor{red!40}\href{../works/IklassovMR023.pdf}{IklassovMR023} (0.70)& \cellcolor{red!40}\href{../works/HamP21.pdf}{HamP21} (0.69)& \cellcolor{red!40}\href{../works/TasselGS23.pdf}{TasselGS23} (0.67)& \cellcolor{red!40}\href{../works/abs-2306-05747.pdf}{abs-2306-05747} (0.67)\\
\index{MurinR19}\href{../works/MurinR19.pdf}{MurinR19} R\&C& \cellcolor{red!40}\href{../works/HeinzNVH22.pdf}{HeinzNVH22} (0.81)& \cellcolor{red!40}\href{../works/CauwelaertDS20.pdf}{CauwelaertDS20} (0.86)& \cellcolor{red!20}\href{../works/ThomasKS20.pdf}{ThomasKS20} (0.87)& \cellcolor{red!20}\href{../works/CauwelaertDMS16.pdf}{CauwelaertDMS16} (0.88)& \cellcolor{red!20}\href{../works/CappartTSR18.pdf}{CappartTSR18} (0.89)\\
Euclid& \cellcolor{green!20}\href{../works/NovasH14.pdf}{NovasH14} (0.30)& \cellcolor{green!20}\href{../works/KhayatLR06.pdf}{KhayatLR06} (0.31)& \cellcolor{blue!20}\href{../works/CauwelaertDMS16.pdf}{CauwelaertDMS16} (0.32)& \cellcolor{blue!20}\href{../works/DavenportKRSH07.pdf}{DavenportKRSH07} (0.32)& \cellcolor{blue!20}\href{../works/Ham20a.pdf}{Ham20a} (0.33)\\
Dot& \cellcolor{red!40}\href{../works/Lunardi20.pdf}{Lunardi20} (150.00)& \cellcolor{red!40}\href{../works/LaborieRSV18.pdf}{LaborieRSV18} (149.00)& \cellcolor{red!40}\href{../works/Astrand21.pdf}{Astrand21} (139.00)& \cellcolor{red!40}\href{../works/Groleaz21.pdf}{Groleaz21} (134.00)& \cellcolor{red!40}\href{../works/ZarandiASC20.pdf}{ZarandiASC20} (132.00)\\
Cosine& \cellcolor{red!40}\href{../works/NovasH14.pdf}{NovasH14} (0.78)& \cellcolor{red!40}\href{../works/KhayatLR06.pdf}{KhayatLR06} (0.76)& \cellcolor{red!40}\href{../works/CzerniachowskaWZ23.pdf}{CzerniachowskaWZ23} (0.75)& \cellcolor{red!40}\href{../works/LunardiBLRV20.pdf}{LunardiBLRV20} (0.74)& \cellcolor{red!40}\href{../works/CauwelaertDMS16.pdf}{CauwelaertDMS16} (0.74)\\
\index{MurphyMB15}\href{../works/MurphyMB15.pdf}{MurphyMB15} R\&C& \cellcolor{green!20}\href{../works/CarlierPSJ20.pdf}{CarlierPSJ20} (0.93)& \cellcolor{green!20}DorndorfHP99 (0.95)& \cellcolor{green!20}\href{../works/LimHTB16.pdf}{LimHTB16} (0.95)& \cellcolor{green!20}\href{../works/Simonis95.pdf}{Simonis95} (0.96)& \cellcolor{green!20}\href{../works/Geske05.pdf}{Geske05} (0.96)\\
Euclid& \cellcolor{red!40}\href{../works/BockmayrP06.pdf}{BockmayrP06} (0.22)& \cellcolor{red!40}\href{../works/Davis87.pdf}{Davis87} (0.23)& \cellcolor{red!40}\href{../works/WolfS05.pdf}{WolfS05} (0.23)& \cellcolor{red!40}\href{../works/FukunagaHFAMN02.pdf}{FukunagaHFAMN02} (0.23)& \cellcolor{red!40}\href{../works/PoderB08.pdf}{PoderB08} (0.24)\\
Dot& \cellcolor{red!40}\href{../works/Lombardi10.pdf}{Lombardi10} (87.00)& \cellcolor{red!40}\href{../works/Fahimi16.pdf}{Fahimi16} (81.00)& \cellcolor{red!40}\href{../works/Dejemeppe16.pdf}{Dejemeppe16} (79.00)& \cellcolor{red!40}\href{../works/Malapert11.pdf}{Malapert11} (78.00)& \cellcolor{red!40}\href{../works/Astrand21.pdf}{Astrand21} (77.00)\\
Cosine& \cellcolor{red!40}\href{../works/BockmayrP06.pdf}{BockmayrP06} (0.77)& \cellcolor{red!40}\href{../works/Madi-WambaLOBM17.pdf}{Madi-WambaLOBM17} (0.73)& \cellcolor{red!40}\href{../works/WolfS05.pdf}{WolfS05} (0.73)& \cellcolor{red!40}\href{../works/BoothNB16.pdf}{BoothNB16} (0.73)& \cellcolor{red!40}\href{../works/AstrandJZ18.pdf}{AstrandJZ18} (0.71)\\
\index{MurphyRFSS97}\href{../works/MurphyRFSS97.pdf}{MurphyRFSS97} R\&C\\
Euclid& \cellcolor{red!40}\href{../works/FukunagaHFAMN02.pdf}{FukunagaHFAMN02} (0.23)& \cellcolor{red!40}\href{../works/AngelsmarkJ00.pdf}{AngelsmarkJ00} (0.24)& \cellcolor{red!40}\href{../works/LudwigKRBMS14.pdf}{LudwigKRBMS14} (0.24)& \cellcolor{red!20}\href{../works/WallaceF00.pdf}{WallaceF00} (0.24)& \cellcolor{red!20}\href{../works/LiuJ06.pdf}{LiuJ06} (0.24)\\
Dot& \cellcolor{red!40}\href{../works/ZarandiASC20.pdf}{ZarandiASC20} (68.00)& \cellcolor{red!40}\href{../works/Beck99.pdf}{Beck99} (63.00)& \cellcolor{red!40}\href{../works/Astrand21.pdf}{Astrand21} (62.00)& \cellcolor{red!40}\href{../works/LuZZYW24.pdf}{LuZZYW24} (60.00)& \cellcolor{red!40}\href{../works/Novas19.pdf}{Novas19} (57.00)\\
Cosine& \cellcolor{red!40}\href{../works/BarbulescuWH04.pdf}{BarbulescuWH04} (0.67)& \cellcolor{red!40}\href{../works/LudwigKRBMS14.pdf}{LudwigKRBMS14} (0.64)& \cellcolor{red!40}\href{../works/FukunagaHFAMN02.pdf}{FukunagaHFAMN02} (0.64)& \cellcolor{red!40}\href{../works/GilesH16.pdf}{GilesH16} (0.63)& \cellcolor{red!40}\href{../works/BrusoniCLMMT96.pdf}{BrusoniCLMMT96} (0.62)\\
\index{MurthyRAW97}MurthyRAW97 R\&C\\
Euclid\\
Dot\\
Cosine\\
\index{Muscettola02}\href{../works/Muscettola02.pdf}{Muscettola02} R\&C& \cellcolor{red!20}\href{../works/Kumar03.pdf}{Kumar03} (0.89)& \cellcolor{red!20}\href{../works/Laborie03.pdf}{Laborie03} (0.90)& \cellcolor{yellow!20}\href{../works/LombardiM09.pdf}{LombardiM09} (0.90)& \cellcolor{yellow!20}CestaOPS14 (0.90)& \cellcolor{yellow!20}\href{../works/PraletLJ15.pdf}{PraletLJ15} (0.92)\\
Euclid& \cellcolor{red!40}\href{../works/LombardiM13.pdf}{LombardiM13} (0.21)& \cellcolor{red!40}\href{../works/WallaceF00.pdf}{WallaceF00} (0.22)& \cellcolor{red!40}\href{../works/Muscettola94.pdf}{Muscettola94} (0.23)& \cellcolor{red!40}\href{../works/BonfiettiM12.pdf}{BonfiettiM12} (0.23)& \cellcolor{red!40}\href{../works/LeeKLKKYHP97.pdf}{LeeKLKKYHP97} (0.23)\\
Dot& \cellcolor{red!40}\href{../works/Laborie03.pdf}{Laborie03} (67.00)& \cellcolor{red!40}\href{../works/Siala15a.pdf}{Siala15a} (66.00)& \cellcolor{red!40}\href{../works/Malapert11.pdf}{Malapert11} (66.00)& \cellcolor{red!40}\href{../works/ZarandiASC20.pdf}{ZarandiASC20} (63.00)& \cellcolor{red!40}\href{../works/OddiRCS11.pdf}{OddiRCS11} (63.00)\\
Cosine& \cellcolor{red!40}\href{../works/Muscettola94.pdf}{Muscettola94} (0.78)& \cellcolor{red!40}\href{../works/LombardiM13.pdf}{LombardiM13} (0.74)& \cellcolor{red!40}\href{../works/KeriK07.pdf}{KeriK07} (0.72)& \cellcolor{red!40}\href{../works/Vilim04.pdf}{Vilim04} (0.71)& \cellcolor{red!40}\href{../works/MouraSCL08.pdf}{MouraSCL08} (0.70)\\
\index{Muscettola94}\href{../works/Muscettola94.pdf}{Muscettola94} R\&C\\
Euclid& \cellcolor{red!40}\href{../works/OddiS97.pdf}{OddiS97} (0.20)& \cellcolor{red!40}\href{../works/Muscettola02.pdf}{Muscettola02} (0.23)& \cellcolor{red!20}\href{../works/BidotVLB07.pdf}{BidotVLB07} (0.25)& \cellcolor{red!20}\href{../works/Junker00.pdf}{Junker00} (0.26)& \cellcolor{red!20}\href{../works/SmithC93.pdf}{SmithC93} (0.26)\\
Dot& \cellcolor{red!40}\href{../works/ZarandiASC20.pdf}{ZarandiASC20} (97.00)& \cellcolor{red!40}\href{../works/Astrand21.pdf}{Astrand21} (97.00)& \cellcolor{red!40}\href{../works/Dejemeppe16.pdf}{Dejemeppe16} (96.00)& \cellcolor{red!40}\href{../works/Lombardi10.pdf}{Lombardi10} (92.00)& \cellcolor{red!40}\href{../works/Groleaz21.pdf}{Groleaz21} (91.00)\\
Cosine& \cellcolor{red!40}\href{../works/OddiS97.pdf}{OddiS97} (0.84)& \cellcolor{red!40}\href{../works/SadehF96.pdf}{SadehF96} (0.78)& \cellcolor{red!40}\href{../works/Muscettola02.pdf}{Muscettola02} (0.78)& \cellcolor{red!40}\href{../works/BidotVLB07.pdf}{BidotVLB07} (0.75)& \cellcolor{red!40}\href{../works/CestaOS00.pdf}{CestaOS00} (0.74)\\
\index{Musliu05}\href{../works/Musliu05.pdf}{Musliu05} R\&C\\
Euclid& \cellcolor{red!40}\href{../works/Baptiste09.pdf}{Baptiste09} (0.19)& \cellcolor{red!40}\href{../works/CarchraeBF05.pdf}{CarchraeBF05} (0.19)& \cellcolor{red!40}\href{../works/Tsang03.pdf}{Tsang03} (0.20)& \cellcolor{red!40}\href{../works/KovacsEKV05.pdf}{KovacsEKV05} (0.20)& \cellcolor{red!40}\href{../works/AngelsmarkJ00.pdf}{AngelsmarkJ00} (0.20)\\
Dot& \cellcolor{red!40}\href{../works/MusliuSS18.pdf}{MusliuSS18} (30.00)& \cellcolor{red!40}\href{../works/JainM99.pdf}{JainM99} (28.00)& \cellcolor{red!40}\href{../works/WinterMMW22.pdf}{WinterMMW22} (27.00)& \cellcolor{red!40}\href{../works/Wallace06.pdf}{Wallace06} (27.00)& \cellcolor{red!40}\href{../works/Astrand21.pdf}{Astrand21} (27.00)\\
Cosine& \cellcolor{red!40}\href{../works/ErkingerM17.pdf}{ErkingerM17} (0.58)& \cellcolor{red!40}\href{../works/MusliuSS18.pdf}{MusliuSS18} (0.54)& \cellcolor{red!40}\href{../works/Tsang03.pdf}{Tsang03} (0.51)& \cellcolor{red!40}\href{../works/BartakCS10.pdf}{BartakCS10} (0.47)& \cellcolor{red!40}\href{../works/KletzanderM20.pdf}{KletzanderM20} (0.46)\\
\index{MusliuSS18}\href{../works/MusliuSS18.pdf}{MusliuSS18} R\&C& \cellcolor{red!40}\href{../works/ErkingerM17.pdf}{ErkingerM17} (0.82)& \cellcolor{red!20}\href{../works/GokGSTO20.pdf}{GokGSTO20} (0.86)& \cellcolor{red!20}\href{../works/FrohnerTR19.pdf}{FrohnerTR19} (0.90)& \cellcolor{red!20}\href{../works/GeibingerMM19.pdf}{GeibingerMM19} (0.90)& \cellcolor{green!20}\href{../works/HoYCLLCLC18.pdf}{HoYCLLCLC18} (0.95)\\
Euclid& \cellcolor{green!20}\href{../works/ErkingerM17.pdf}{ErkingerM17} (0.29)& \cellcolor{green!20}\href{../works/PesantRR15.pdf}{PesantRR15} (0.31)& \cellcolor{blue!20}\href{../works/ZibranR11.pdf}{ZibranR11} (0.33)& \cellcolor{blue!20}\href{../works/ChapadosJR11.pdf}{ChapadosJR11} (0.33)& \cellcolor{blue!20}\href{../works/JungblutK22.pdf}{JungblutK22} (0.33)\\
Dot& \cellcolor{red!40}\href{../works/Dejemeppe16.pdf}{Dejemeppe16} (82.00)& \cellcolor{red!40}\href{../works/Siala15a.pdf}{Siala15a} (80.00)& \cellcolor{red!40}\href{../works/Godet21a.pdf}{Godet21a} (77.00)& \cellcolor{red!40}\href{../works/Caballero19.pdf}{Caballero19} (76.00)& \cellcolor{red!40}\href{../works/abs-1911-04766.pdf}{abs-1911-04766} (74.00)\\
Cosine& \cellcolor{red!40}\href{../works/ErkingerM17.pdf}{ErkingerM17} (0.68)& \cellcolor{red!40}\href{../works/AstrandJZ18.pdf}{AstrandJZ18} (0.60)& \cellcolor{red!40}\href{../works/PesantRR15.pdf}{PesantRR15} (0.60)& \cellcolor{red!40}\href{../works/BorghesiBLMB18.pdf}{BorghesiBLMB18} (0.59)& \cellcolor{red!40}\href{../works/BourdaisGP03.pdf}{BourdaisGP03} (0.57)\\
\index{NaderiBZ22}\href{../works/NaderiBZ22.pdf}{NaderiBZ22} R\&C& \cellcolor{red!20}\href{../works/NaderiRR23.pdf}{NaderiRR23} (0.86)& \cellcolor{yellow!20}\href{../works/NaderiBZ22a.pdf}{NaderiBZ22a} (0.92)& \cellcolor{yellow!20}NaderiRBAU21 (0.92)& \cellcolor{green!20}\href{../works/NaderiBZR23.pdf}{NaderiBZR23} (0.94)& \cellcolor{green!20}\href{../works/RoshanaeiN21.pdf}{RoshanaeiN21} (0.94)\\
Euclid& \cellcolor{red!40}\href{../works/NaderiBZ23.pdf}{NaderiBZ23} (0.09)& \cellcolor{blue!20}\href{../works/RoshanaeiN21.pdf}{RoshanaeiN21} (0.33)& \cellcolor{black!20}\href{../works/ElciOH22.pdf}{ElciOH22} (0.35)& \cellcolor{black!20}\href{../works/HamdiL13.pdf}{HamdiL13} (0.36)& \href{../works/TanZWGQ19.pdf}{TanZWGQ19} (0.38)\\
Dot& \cellcolor{red!40}\href{../works/NaderiBZ23.pdf}{NaderiBZ23} (189.00)& \cellcolor{red!40}\href{../works/NaderiRR23.pdf}{NaderiRR23} (185.00)& \cellcolor{red!40}\href{../works/Groleaz21.pdf}{Groleaz21} (182.00)& \cellcolor{red!40}\href{../works/ZarandiASC20.pdf}{ZarandiASC20} (174.00)& \cellcolor{red!40}\href{../works/Lunardi20.pdf}{Lunardi20} (169.00)\\
Cosine& \cellcolor{red!40}\href{../works/NaderiBZ23.pdf}{NaderiBZ23} (0.99)& \cellcolor{red!40}\href{../works/RoshanaeiN21.pdf}{RoshanaeiN21} (0.79)& \cellcolor{red!40}\href{../works/NaderiRR23.pdf}{NaderiRR23} (0.76)& \cellcolor{red!40}\href{../works/ElciOH22.pdf}{ElciOH22} (0.74)& \cellcolor{red!40}\href{../works/HamdiL13.pdf}{HamdiL13} (0.72)\\
\index{NaderiBZ22a}\href{../works/NaderiBZ22a.pdf}{NaderiBZ22a} R\&C& \cellcolor{red!40}\href{../works/ZhangYW21.pdf}{ZhangYW21} (0.79)& \cellcolor{red!40}\href{../works/CilKLO22.pdf}{CilKLO22} (0.85)& \cellcolor{red!40}ShiYXQ22 (0.85)& \cellcolor{red!40}\href{../works/PohlAK22.pdf}{PohlAK22} (0.86)& \cellcolor{red!20}\href{../works/ZhuSZW23.pdf}{ZhuSZW23} (0.88)\\
Euclid& \cellcolor{green!20}\href{../works/ZhangYW21.pdf}{ZhangYW21} (0.30)& \cellcolor{blue!20}\href{../works/EmeretlisTAV17.pdf}{EmeretlisTAV17} (0.34)& \cellcolor{blue!20}\href{../works/ZhuSZW23.pdf}{ZhuSZW23} (0.34)& \cellcolor{black!20}\href{../works/TanT18.pdf}{TanT18} (0.35)& \cellcolor{black!20}\href{../works/JuvinHL23a.pdf}{JuvinHL23a} (0.36)\\
Dot& \cellcolor{red!40}\href{../works/ZarandiASC20.pdf}{ZarandiASC20} (186.00)& \cellcolor{red!40}\href{../works/Groleaz21.pdf}{Groleaz21} (185.00)& \cellcolor{red!40}\href{../works/NaderiRR23.pdf}{NaderiRR23} (179.00)& \cellcolor{red!40}\href{../works/Lunardi20.pdf}{Lunardi20} (176.00)& \cellcolor{red!40}\href{../works/IsikYA23.pdf}{IsikYA23} (163.00)\\
Cosine& \cellcolor{red!40}\href{../works/ZhangYW21.pdf}{ZhangYW21} (0.82)& \cellcolor{red!40}\href{../works/ZhuSZW23.pdf}{ZhuSZW23} (0.80)& \cellcolor{red!40}\href{../works/JuvinHL23a.pdf}{JuvinHL23a} (0.78)& \cellcolor{red!40}\href{../works/JuvinHL22.pdf}{JuvinHL22} (0.77)& \cellcolor{red!40}\href{../works/EmeretlisTAV17.pdf}{EmeretlisTAV17} (0.77)\\
\index{NaderiBZ23}\href{../works/NaderiBZ23.pdf}{NaderiBZ23} R\&C& \cellcolor{red!40}\href{../works/NaderiRR23.pdf}{NaderiRR23} (0.85)& \cellcolor{yellow!20}\href{../works/NaderiBZ22a.pdf}{NaderiBZ22a} (0.91)& \cellcolor{yellow!20}\href{../works/NaderiBZR23.pdf}{NaderiBZR23} (0.92)& \cellcolor{yellow!20}NaderiRBAU21 (0.93)& \cellcolor{green!20}\href{../works/RoshanaeiN21.pdf}{RoshanaeiN21} (0.94)\\
Euclid& \cellcolor{red!40}\href{../works/NaderiBZ22.pdf}{NaderiBZ22} (0.09)& \cellcolor{blue!20}\href{../works/RoshanaeiN21.pdf}{RoshanaeiN21} (0.33)& \cellcolor{black!20}\href{../works/ElciOH22.pdf}{ElciOH22} (0.37)& \href{../works/HamdiL13.pdf}{HamdiL13} (0.38)& \href{../works/RoshanaeiBAUB20.pdf}{RoshanaeiBAUB20} (0.39)\\
Dot& \cellcolor{red!40}\href{../works/NaderiBZ22.pdf}{NaderiBZ22} (189.00)& \cellcolor{red!40}\href{../works/NaderiRR23.pdf}{NaderiRR23} (187.00)& \cellcolor{red!40}\href{../works/Groleaz21.pdf}{Groleaz21} (183.00)& \cellcolor{red!40}\href{../works/ZarandiASC20.pdf}{ZarandiASC20} (175.00)& \cellcolor{red!40}\href{../works/Lunardi20.pdf}{Lunardi20} (171.00)\\
Cosine& \cellcolor{red!40}\href{../works/NaderiBZ22.pdf}{NaderiBZ22} (0.99)& \cellcolor{red!40}\href{../works/RoshanaeiN21.pdf}{RoshanaeiN21} (0.79)& \cellcolor{red!40}\href{../works/NaderiRR23.pdf}{NaderiRR23} (0.75)& \cellcolor{red!40}\href{../works/ElciOH22.pdf}{ElciOH22} (0.73)& \cellcolor{red!40}\href{../works/RoshanaeiBAUB20.pdf}{RoshanaeiBAUB20} (0.71)\\
\index{NaderiBZR23}\href{../works/NaderiBZR23.pdf}{NaderiBZR23} R\&C& \cellcolor{red!20}NaderiRBAU21 (0.87)& \cellcolor{red!20}HechingHK19 (0.88)& \cellcolor{yellow!20}MartnezAJ22 (0.90)& \cellcolor{yellow!20}\href{../works/RoshanaeiBAUB20.pdf}{RoshanaeiBAUB20} (0.92)& \cellcolor{yellow!20}\href{../works/NaderiBZ23.pdf}{NaderiBZ23} (0.92)\\
Euclid& \cellcolor{blue!20}\href{../works/RoshanaeiBAUB20.pdf}{RoshanaeiBAUB20} (0.32)& \cellcolor{black!20}\href{../works/RoshanaeiLAU17.pdf}{RoshanaeiLAU17} (0.34)& \cellcolor{black!20}\href{../works/HechingH16.pdf}{HechingH16} (0.36)& \cellcolor{black!20}\href{../works/RiiseML16.pdf}{RiiseML16} (0.36)& \href{../works/FrimodigS19.pdf}{FrimodigS19} (0.38)\\
Dot& \cellcolor{red!40}\href{../works/ZarandiASC20.pdf}{ZarandiASC20} (132.00)& \cellcolor{red!40}\href{../works/RoshanaeiBAUB20.pdf}{RoshanaeiBAUB20} (123.00)& \cellcolor{red!40}\href{../works/Groleaz21.pdf}{Groleaz21} (122.00)& \cellcolor{red!40}\href{../works/NaderiBZ23.pdf}{NaderiBZ23} (121.00)& \cellcolor{red!40}\href{../works/RoshanaeiLAU17.pdf}{RoshanaeiLAU17} (120.00)\\
Cosine& \cellcolor{red!40}\href{../works/RoshanaeiBAUB20.pdf}{RoshanaeiBAUB20} (0.78)& \cellcolor{red!40}\href{../works/RoshanaeiLAU17.pdf}{RoshanaeiLAU17} (0.75)& \cellcolor{red!40}\href{../works/RiiseML16.pdf}{RiiseML16} (0.69)& \cellcolor{red!40}\href{../works/NaderiBZ23.pdf}{NaderiBZ23} (0.69)& \cellcolor{red!40}\href{../works/NaderiBZ22.pdf}{NaderiBZ22} (0.67)\\
\index{NaderiR22}NaderiR22 R\&C& \cellcolor{red!40}\href{../works/ElciOH22.pdf}{ElciOH22} (0.67)& \cellcolor{red!40}MartnezAJ22 (0.73)& \cellcolor{red!40}\href{../works/JuvinHL23a.pdf}{JuvinHL23a} (0.84)& \cellcolor{red!40}\href{../works/RoshanaeiN21.pdf}{RoshanaeiN21} (0.85)& \cellcolor{red!40}\href{../works/JuvinHL22.pdf}{JuvinHL22} (0.86)\\
Euclid\\
Dot\\
Cosine\\
\index{NaderiRBAU21}NaderiRBAU21 R\&C& \cellcolor{red!40}\href{../works/RoshanaeiN21.pdf}{RoshanaeiN21} (0.64)& \cellcolor{red!40}\href{../works/RoshanaeiBAUB20.pdf}{RoshanaeiBAUB20} (0.67)& \cellcolor{red!40}\href{../works/RoshanaeiLAU17.pdf}{RoshanaeiLAU17} (0.78)& \cellcolor{red!40}RoshanaeiLAU17a (0.80)& \cellcolor{red!40}MartnezAJ22 (0.84)\\
Euclid\\
Dot\\
Cosine\\
\index{NaderiRR23}\href{../works/NaderiRR23.pdf}{NaderiRR23} R\&C& \cellcolor{red!40}\href{../works/MengLZB21.pdf}{MengLZB21} (0.81)& \cellcolor{red!40}\href{../works/NaderiBZ23.pdf}{NaderiBZ23} (0.85)& \cellcolor{red!20}\href{../works/NaderiBZ22.pdf}{NaderiBZ22} (0.86)& \cellcolor{red!20}NaderiRBAU21 (0.88)& \cellcolor{yellow!20}\href{../works/RoshanaeiN21.pdf}{RoshanaeiN21} (0.90)\\
Euclid& \href{../works/NaderiBZ22.pdf}{NaderiBZ22} (0.45)& \href{../works/MengZRZL20.pdf}{MengZRZL20} (0.45)& \href{../works/NaderiBZ23.pdf}{NaderiBZ23} (0.45)& \href{../works/GrimesH15.pdf}{GrimesH15} (0.46)& \href{../works/NaderiBZ22a.pdf}{NaderiBZ22a} (0.47)\\
Dot& \cellcolor{red!40}\href{../works/Groleaz21.pdf}{Groleaz21} (279.00)& \cellcolor{red!40}\href{../works/ZarandiASC20.pdf}{ZarandiASC20} (242.00)& \cellcolor{red!40}\href{../works/Lunardi20.pdf}{Lunardi20} (227.00)& \cellcolor{red!40}\href{../works/GrimesH15.pdf}{GrimesH15} (224.00)& \cellcolor{red!40}\href{../works/Baptiste02.pdf}{Baptiste02} (222.00)\\
Cosine& \cellcolor{red!40}\href{../works/MengZRZL20.pdf}{MengZRZL20} (0.76)& \cellcolor{red!40}\href{../works/GrimesH15.pdf}{GrimesH15} (0.76)& \cellcolor{red!40}\href{../works/NaderiBZ22.pdf}{NaderiBZ22} (0.76)& \cellcolor{red!40}\href{../works/NaderiBZ23.pdf}{NaderiBZ23} (0.75)& \cellcolor{red!40}\href{../works/NaderiBZ22a.pdf}{NaderiBZ22a} (0.73)\\
\index{NaqviAIAAA22}\href{../works/NaqviAIAAA22.pdf}{NaqviAIAAA22} R\&C& \cellcolor{yellow!20}\href{../works/BulckG22.pdf}{BulckG22} (0.93)& \cellcolor{green!20}Trick11 (0.95)& \cellcolor{blue!20}\href{../works/Ribeiro12.pdf}{Ribeiro12} (0.96)& \cellcolor{blue!20}\href{../works/LiuLH18.pdf}{LiuLH18} (0.97)& \cellcolor{blue!20}\href{../works/RasmussenT06.pdf}{RasmussenT06} (0.97)\\
Euclid& \cellcolor{red!40}\href{../works/ZengM12.pdf}{ZengM12} (0.21)& \cellcolor{red!40}\href{../works/RasmussenT06.pdf}{RasmussenT06} (0.21)& \cellcolor{red!40}\href{../works/EastonNT02.pdf}{EastonNT02} (0.21)& \cellcolor{red!40}\href{../works/SuCC13.pdf}{SuCC13} (0.22)& \cellcolor{red!40}\href{../works/ElfJR03.pdf}{ElfJR03} (0.22)\\
Dot& \cellcolor{red!40}\href{../works/KendallKRU10.pdf}{KendallKRU10} (81.00)& \cellcolor{red!40}\href{../works/Ribeiro12.pdf}{Ribeiro12} (72.00)& \cellcolor{red!40}\href{../works/ZarandiASC20.pdf}{ZarandiASC20} (71.00)& \cellcolor{red!40}\href{../works/RasmussenT09.pdf}{RasmussenT09} (66.00)& \cellcolor{red!40}\href{../works/Lemos21.pdf}{Lemos21} (64.00)\\
Cosine& \cellcolor{red!40}\href{../works/ZengM12.pdf}{ZengM12} (0.80)& \cellcolor{red!40}\href{../works/RasmussenT09.pdf}{RasmussenT09} (0.79)& \cellcolor{red!40}\href{../works/RasmussenT06.pdf}{RasmussenT06} (0.79)& \cellcolor{red!40}\href{../works/EastonNT02.pdf}{EastonNT02} (0.77)& \cellcolor{red!40}\href{../works/Ribeiro12.pdf}{Ribeiro12} (0.76)\\
\index{Nattaf16}\href{../works/Nattaf16.pdf}{Nattaf16} R\&C\\
Euclid& \href{../works/NattafAL15.pdf}{NattafAL15} (0.43)& \href{../works/NattafAL17.pdf}{NattafAL17} (0.43)& \href{../works/DemasseyAM05.pdf}{DemasseyAM05} (0.44)& \href{../works/ArbaouiY18.pdf}{ArbaouiY18} (0.44)& \href{../works/BaptistePN99.pdf}{BaptistePN99} (0.44)\\
Dot& \cellcolor{red!40}\href{../works/Groleaz21.pdf}{Groleaz21} (186.00)& \cellcolor{red!40}\href{../works/Baptiste02.pdf}{Baptiste02} (186.00)& \cellcolor{red!40}\href{../works/Malapert11.pdf}{Malapert11} (175.00)& \cellcolor{red!40}\href{../works/ZarandiASC20.pdf}{ZarandiASC20} (162.00)& \cellcolor{red!40}\href{../works/Dejemeppe16.pdf}{Dejemeppe16} (157.00)\\
Cosine& \cellcolor{red!40}\href{../works/Demassey03.pdf}{Demassey03} (0.68)& \cellcolor{red!40}\href{../works/GokgurHO18.pdf}{GokgurHO18} (0.67)& \cellcolor{red!40}\href{../works/Elkhyari03.pdf}{Elkhyari03} (0.66)& \cellcolor{red!40}\href{../works/Kameugne14.pdf}{Kameugne14} (0.66)& \cellcolor{red!40}\href{../works/NattafAL15.pdf}{NattafAL15} (0.66)\\
\index{NattafAL15}\href{../works/NattafAL15.pdf}{NattafAL15} R\&C& \cellcolor{red!40}\href{../works/NattafALR16.pdf}{NattafALR16} (0.29)& \cellcolor{red!40}\href{../works/NattafHKAL19.pdf}{NattafHKAL19} (0.67)& \cellcolor{red!40}\href{../works/ArtiguesL14.pdf}{ArtiguesL14} (0.75)& \cellcolor{red!40}\href{../works/NattafAL17.pdf}{NattafAL17} (0.77)& \cellcolor{red!40}\href{../works/ArtiguesLH13.pdf}{ArtiguesLH13} (0.85)\\
Euclid& \cellcolor{red!40}\href{../works/NattafALR16.pdf}{NattafALR16} (0.10)& \cellcolor{red!40}\href{../works/ArtiguesL14.pdf}{ArtiguesL14} (0.20)& \cellcolor{red!40}\href{../works/NattafHKAL19.pdf}{NattafHKAL19} (0.23)& \cellcolor{yellow!20}\href{../works/CarlierPSJ20.pdf}{CarlierPSJ20} (0.27)& \cellcolor{yellow!20}\href{../works/NattafAL17.pdf}{NattafAL17} (0.28)\\
Dot& \cellcolor{red!40}\href{../works/NattafALR16.pdf}{NattafALR16} (121.00)& \cellcolor{red!40}\href{../works/Lombardi10.pdf}{Lombardi10} (113.00)& \cellcolor{red!40}\href{../works/Baptiste02.pdf}{Baptiste02} (109.00)& \cellcolor{red!40}\href{../works/Nattaf16.pdf}{Nattaf16} (108.00)& \cellcolor{red!40}\href{../works/Fahimi16.pdf}{Fahimi16} (108.00)\\
Cosine& \cellcolor{red!40}\href{../works/NattafALR16.pdf}{NattafALR16} (0.97)& \cellcolor{red!40}\href{../works/ArtiguesL14.pdf}{ArtiguesL14} (0.89)& \cellcolor{red!40}\href{../works/NattafHKAL19.pdf}{NattafHKAL19} (0.84)& \cellcolor{red!40}\href{../works/CarlierPSJ20.pdf}{CarlierPSJ20} (0.77)& \cellcolor{red!40}\href{../works/NattafAL17.pdf}{NattafAL17} (0.76)\\
\index{NattafAL17}\href{../works/NattafAL17.pdf}{NattafAL17} R\&C& \cellcolor{red!40}\href{../works/NattafALR16.pdf}{NattafALR16} (0.73)& \cellcolor{red!40}\href{../works/NattafAL15.pdf}{NattafAL15} (0.77)& \cellcolor{red!40}\href{../works/ArtiguesL14.pdf}{ArtiguesL14} (0.78)& \cellcolor{red!40}\href{../works/NattafHKAL19.pdf}{NattafHKAL19} (0.82)& \cellcolor{red!40}\href{../works/LetortCB15.pdf}{LetortCB15} (0.86)\\
Euclid& \cellcolor{yellow!20}\href{../works/ArtiguesL14.pdf}{ArtiguesL14} (0.26)& \cellcolor{yellow!20}\href{../works/Caseau97.pdf}{Caseau97} (0.27)& \cellcolor{yellow!20}\href{../works/NattafAL15.pdf}{NattafAL15} (0.28)& \cellcolor{yellow!20}\href{../works/ChuGNSW13.pdf}{ChuGNSW13} (0.28)& \cellcolor{yellow!20}\href{../works/WolfS05.pdf}{WolfS05} (0.28)\\
Dot& \cellcolor{red!40}\href{../works/Lombardi10.pdf}{Lombardi10} (104.00)& \cellcolor{red!40}\href{../works/Baptiste02.pdf}{Baptiste02} (100.00)& \cellcolor{red!40}\href{../works/Fahimi16.pdf}{Fahimi16} (99.00)& \cellcolor{red!40}\href{../works/Malapert11.pdf}{Malapert11} (95.00)& \cellcolor{red!40}\href{../works/Nattaf16.pdf}{Nattaf16} (93.00)\\
Cosine& \cellcolor{red!40}\href{../works/ArtiguesL14.pdf}{ArtiguesL14} (0.77)& \cellcolor{red!40}\href{../works/NattafAL15.pdf}{NattafAL15} (0.76)& \cellcolor{red!40}\href{../works/NattafALR16.pdf}{NattafALR16} (0.74)& \cellcolor{red!40}\href{../works/NattafHKAL19.pdf}{NattafHKAL19} (0.73)& \cellcolor{red!40}\href{../works/ChuGNSW13.pdf}{ChuGNSW13} (0.70)\\
\index{NattafALR16}\href{../works/NattafALR16.pdf}{NattafALR16} R\&C& \cellcolor{red!40}\href{../works/NattafAL15.pdf}{NattafAL15} (0.29)& \cellcolor{red!40}\href{../works/NattafHKAL19.pdf}{NattafHKAL19} (0.64)& \cellcolor{red!40}\href{../works/NattafAL17.pdf}{NattafAL17} (0.73)& \cellcolor{red!20}\href{../works/ArtiguesL14.pdf}{ArtiguesL14} (0.86)& \cellcolor{red!20}CarlierSJP21 (0.87)\\
Euclid& \cellcolor{red!40}\href{../works/NattafAL15.pdf}{NattafAL15} (0.10)& \cellcolor{red!40}\href{../works/ArtiguesL14.pdf}{ArtiguesL14} (0.22)& \cellcolor{red!20}\href{../works/NattafHKAL19.pdf}{NattafHKAL19} (0.25)& \cellcolor{green!20}\href{../works/CarlierPSJ20.pdf}{CarlierPSJ20} (0.29)& \cellcolor{green!20}\href{../works/NattafAL17.pdf}{NattafAL17} (0.30)\\
Dot& \cellcolor{red!40}\href{../works/NattafAL15.pdf}{NattafAL15} (121.00)& \cellcolor{red!40}\href{../works/Lombardi10.pdf}{Lombardi10} (117.00)& \cellcolor{red!40}\href{../works/Baptiste02.pdf}{Baptiste02} (111.00)& \cellcolor{red!40}\href{../works/Fahimi16.pdf}{Fahimi16} (110.00)& \cellcolor{red!40}\href{../works/Godet21a.pdf}{Godet21a} (107.00)\\
Cosine& \cellcolor{red!40}\href{../works/NattafAL15.pdf}{NattafAL15} (0.97)& \cellcolor{red!40}\href{../works/ArtiguesL14.pdf}{ArtiguesL14} (0.86)& \cellcolor{red!40}\href{../works/NattafHKAL19.pdf}{NattafHKAL19} (0.83)& \cellcolor{red!40}\href{../works/CarlierPSJ20.pdf}{CarlierPSJ20} (0.74)& \cellcolor{red!40}\href{../works/NattafAL17.pdf}{NattafAL17} (0.74)\\
\index{NattafDYW19}\href{../works/NattafDYW19.pdf}{NattafDYW19} R\&C& \cellcolor{red!40}\href{../works/MalapertN19.pdf}{MalapertN19} (0.75)& \cellcolor{red!40}\href{../works/Ham18a.pdf}{Ham18a} (0.84)& \cellcolor{yellow!20}\href{../works/ZeballosM09.pdf}{ZeballosM09} (0.91)& \cellcolor{yellow!20}\href{../works/Zeballos10.pdf}{Zeballos10} (0.92)& \cellcolor{yellow!20}\href{../works/QuirogaZH05.pdf}{QuirogaZH05} (0.93)\\
Euclid& \cellcolor{yellow!20}\href{../works/Ham18a.pdf}{Ham18a} (0.28)& \cellcolor{green!20}\href{../works/ArbaouiY18.pdf}{ArbaouiY18} (0.31)& \cellcolor{blue!20}\href{../works/NattafM20.pdf}{NattafM20} (0.33)& \cellcolor{blue!20}\href{../works/WatsonB08.pdf}{WatsonB08} (0.33)& \cellcolor{blue!20}\href{../works/MalapertN19.pdf}{MalapertN19} (0.34)\\
Dot& \cellcolor{red!40}\href{../works/ZarandiASC20.pdf}{ZarandiASC20} (139.00)& \cellcolor{red!40}\href{../works/Groleaz21.pdf}{Groleaz21} (134.00)& \cellcolor{red!40}\href{../works/Lunardi20.pdf}{Lunardi20} (129.00)& \cellcolor{red!40}\href{../works/IsikYA23.pdf}{IsikYA23} (121.00)& \cellcolor{red!40}\href{../works/Astrand21.pdf}{Astrand21} (121.00)\\
Cosine& \cellcolor{red!40}\href{../works/Ham18a.pdf}{Ham18a} (0.80)& \cellcolor{red!40}\href{../works/MalapertN19.pdf}{MalapertN19} (0.76)& \cellcolor{red!40}\href{../works/ArbaouiY18.pdf}{ArbaouiY18} (0.74)& \cellcolor{red!40}\href{../works/GedikKEK18.pdf}{GedikKEK18} (0.74)& \cellcolor{red!40}\href{../works/NattafM20.pdf}{NattafM20} (0.72)\\
\index{NattafHKAL19}\href{../works/NattafHKAL19.pdf}{NattafHKAL19} R\&C& \cellcolor{red!40}\href{../works/NattafALR16.pdf}{NattafALR16} (0.64)& \cellcolor{red!40}\href{../works/NattafAL15.pdf}{NattafAL15} (0.67)& \cellcolor{red!40}\href{../works/NattafAL17.pdf}{NattafAL17} (0.82)& \cellcolor{red!40}\href{../works/CarlierPSJ20.pdf}{CarlierPSJ20} (0.82)& \cellcolor{red!40}CarlierSJP21 (0.82)\\
Euclid& \cellcolor{red!40}\href{../works/NattafAL15.pdf}{NattafAL15} (0.23)& \cellcolor{red!40}\href{../works/ArtiguesL14.pdf}{ArtiguesL14} (0.23)& \cellcolor{red!20}\href{../works/NattafALR16.pdf}{NattafALR16} (0.25)& \cellcolor{yellow!20}\href{../works/PoderB08.pdf}{PoderB08} (0.28)& \cellcolor{green!20}\href{../works/NattafAL17.pdf}{NattafAL17} (0.29)\\
Dot& \cellcolor{red!40}\href{../works/Baptiste02.pdf}{Baptiste02} (118.00)& \cellcolor{red!40}\href{../works/Lombardi10.pdf}{Lombardi10} (117.00)& \cellcolor{red!40}\href{../works/Godet21a.pdf}{Godet21a} (109.00)& \cellcolor{red!40}\href{../works/Groleaz21.pdf}{Groleaz21} (109.00)& \cellcolor{red!40}\href{../works/ZarandiASC20.pdf}{ZarandiASC20} (106.00)\\
Cosine& \cellcolor{red!40}\href{../works/NattafAL15.pdf}{NattafAL15} (0.84)& \cellcolor{red!40}\href{../works/ArtiguesL14.pdf}{ArtiguesL14} (0.83)& \cellcolor{red!40}\href{../works/NattafALR16.pdf}{NattafALR16} (0.83)& \cellcolor{red!40}\href{../works/NattafAL17.pdf}{NattafAL17} (0.73)& \cellcolor{red!40}\href{../works/PoderB08.pdf}{PoderB08} (0.72)\\
\index{NattafM20}\href{../works/NattafM20.pdf}{NattafM20} R\&C& \cellcolor{yellow!20}\href{../works/MalapertN19.pdf}{MalapertN19} (0.92)& \cellcolor{blue!20}\href{../works/NattafDYW19.pdf}{NattafDYW19} (0.96)& \cellcolor{blue!20}\href{../works/LombardiM10a.pdf}{LombardiM10a} (0.97)& \cellcolor{black!20}\href{../works/YunusogluY22.pdf}{YunusogluY22} (0.98)& \cellcolor{black!20}\href{../works/JainM99.pdf}{JainM99} (0.99)\\
Euclid& \cellcolor{red!40}\href{../works/MalapertN19.pdf}{MalapertN19} (0.23)& \cellcolor{blue!20}\href{../works/BogaerdtW19.pdf}{BogaerdtW19} (0.33)& \cellcolor{blue!20}\href{../works/NattafDYW19.pdf}{NattafDYW19} (0.33)& \cellcolor{blue!20}\href{../works/ArbaouiY18.pdf}{ArbaouiY18} (0.33)& \cellcolor{black!20}\href{../works/BenediktSMVH18.pdf}{BenediktSMVH18} (0.34)\\
Dot& \cellcolor{red!40}\href{../works/MalapertN19.pdf}{MalapertN19} (132.00)& \cellcolor{red!40}\href{../works/NaderiRR23.pdf}{NaderiRR23} (121.00)& \cellcolor{red!40}\href{../works/Groleaz21.pdf}{Groleaz21} (119.00)& \cellcolor{red!40}\href{../works/ZarandiASC20.pdf}{ZarandiASC20} (108.00)& \cellcolor{red!40}\href{../works/YunusogluY22.pdf}{YunusogluY22} (107.00)\\
Cosine& \cellcolor{red!40}\href{../works/MalapertN19.pdf}{MalapertN19} (0.89)& \cellcolor{red!40}\href{../works/NattafDYW19.pdf}{NattafDYW19} (0.72)& \cellcolor{red!40}\href{../works/ArbaouiY18.pdf}{ArbaouiY18} (0.69)& \cellcolor{red!40}\href{../works/Ham18a.pdf}{Ham18a} (0.68)& \cellcolor{red!40}\href{../works/GedikKEK18.pdf}{GedikKEK18} (0.67)\\
\index{NeronABCDD06}NeronABCDD06 R\&C& \cellcolor{red!40}\href{../works/ArkhipovBL19.pdf}{ArkhipovBL19} (0.67)& \cellcolor{red!40}\href{../works/DemasseyAM05.pdf}{DemasseyAM05} (0.71)& \cellcolor{red!40}\href{../works/LiessM08.pdf}{LiessM08} (0.79)& \cellcolor{red!40}\href{../works/GuSW12.pdf}{GuSW12} (0.84)& \cellcolor{red!20}DorndorfHP99 (0.87)\\
Euclid\\
Dot\\
Cosine\\
\index{NishikawaSTT18}\href{../works/NishikawaSTT18.pdf}{NishikawaSTT18} R\&C& \cellcolor{red!40}\href{../works/NishikawaSTT19.pdf}{NishikawaSTT19} (0.65)& \cellcolor{red!40}\href{../works/NishikawaSTT18a.pdf}{NishikawaSTT18a} (0.65)& \cellcolor{red!20}\href{../works/GilesH16.pdf}{GilesH16} (0.90)& \cellcolor{green!20}\href{../works/ArtiguesLH13.pdf}{ArtiguesLH13} (0.93)& \cellcolor{green!20}\href{../works/Davenport10.pdf}{Davenport10} (0.94)\\
Euclid& \cellcolor{red!40}\href{../works/NishikawaSTT18a.pdf}{NishikawaSTT18a} (0.08)& \cellcolor{red!40}\href{../works/NishikawaSTT19.pdf}{NishikawaSTT19} (0.14)& \cellcolor{red!40}\href{../works/BeniniBGM05a.pdf}{BeniniBGM05a} (0.22)& \cellcolor{red!40}\href{../works/BoothNB16.pdf}{BoothNB16} (0.23)& \cellcolor{red!40}\href{../works/LipovetzkyBPS14.pdf}{LipovetzkyBPS14} (0.24)\\
Dot& \cellcolor{red!40}\href{../works/LaborieRSV18.pdf}{LaborieRSV18} (99.00)& \cellcolor{red!40}\href{../works/Lunardi20.pdf}{Lunardi20} (95.00)& \cellcolor{red!40}\href{../works/Astrand21.pdf}{Astrand21} (89.00)& \cellcolor{red!40}\href{../works/NishikawaSTT19.pdf}{NishikawaSTT19} (88.00)& \cellcolor{red!40}\href{../works/Beck99.pdf}{Beck99} (88.00)\\
Cosine& \cellcolor{red!40}\href{../works/NishikawaSTT18a.pdf}{NishikawaSTT18a} (0.98)& \cellcolor{red!40}\href{../works/NishikawaSTT19.pdf}{NishikawaSTT19} (0.93)& \cellcolor{red!40}\href{../works/BeniniBGM05a.pdf}{BeniniBGM05a} (0.80)& \cellcolor{red!40}\href{../works/ZouZ20.pdf}{ZouZ20} (0.80)& \cellcolor{red!40}\href{../works/LipovetzkyBPS14.pdf}{LipovetzkyBPS14} (0.78)\\
\index{NishikawaSTT18a}\href{../works/NishikawaSTT18a.pdf}{NishikawaSTT18a} R\&C& \cellcolor{red!40}\href{../works/NishikawaSTT18.pdf}{NishikawaSTT18} (0.65)& \cellcolor{red!40}\href{../works/NishikawaSTT19.pdf}{NishikawaSTT19} (0.72)& \cellcolor{green!20}\href{../works/GeibingerKKMMW21.pdf}{GeibingerKKMMW21} (0.93)& \cellcolor{green!20}\href{../works/SerraNM12.pdf}{SerraNM12} (0.94)& \cellcolor{green!20}\href{../works/BridiBLMB16.pdf}{BridiBLMB16} (0.94)\\
Euclid& \cellcolor{red!40}\href{../works/NishikawaSTT18.pdf}{NishikawaSTT18} (0.08)& \cellcolor{red!40}\href{../works/NishikawaSTT19.pdf}{NishikawaSTT19} (0.15)& \cellcolor{red!40}\href{../works/BeniniBGM05a.pdf}{BeniniBGM05a} (0.22)& \cellcolor{red!40}\href{../works/BoothNB16.pdf}{BoothNB16} (0.23)& \cellcolor{red!20}\href{../works/VanczaM01.pdf}{VanczaM01} (0.24)\\
Dot& \cellcolor{red!40}\href{../works/LaborieRSV18.pdf}{LaborieRSV18} (97.00)& \cellcolor{red!40}\href{../works/Lunardi20.pdf}{Lunardi20} (96.00)& \cellcolor{red!40}\href{../works/Astrand21.pdf}{Astrand21} (93.00)& \cellcolor{red!40}\href{../works/ZarandiASC20.pdf}{ZarandiASC20} (91.00)& \cellcolor{red!40}\href{../works/Groleaz21.pdf}{Groleaz21} (89.00)\\
Cosine& \cellcolor{red!40}\href{../works/NishikawaSTT18.pdf}{NishikawaSTT18} (0.98)& \cellcolor{red!40}\href{../works/NishikawaSTT19.pdf}{NishikawaSTT19} (0.92)& \cellcolor{red!40}\href{../works/BeniniBGM05a.pdf}{BeniniBGM05a} (0.80)& \cellcolor{red!40}\href{../works/BoothNB16.pdf}{BoothNB16} (0.79)& \cellcolor{red!40}\href{../works/ZouZ20.pdf}{ZouZ20} (0.79)\\
\index{NishikawaSTT19}\href{../works/NishikawaSTT19.pdf}{NishikawaSTT19} R\&C& \cellcolor{red!40}\href{../works/NishikawaSTT18.pdf}{NishikawaSTT18} (0.65)& \cellcolor{red!40}\href{../works/NishikawaSTT18a.pdf}{NishikawaSTT18a} (0.72)& \cellcolor{yellow!20}\href{../works/HeinzKB13.pdf}{HeinzKB13} (0.92)& \cellcolor{green!20}\href{../works/ArtiguesLH13.pdf}{ArtiguesLH13} (0.94)& \cellcolor{green!20}\href{../works/BridiBLMB16.pdf}{BridiBLMB16} (0.95)\\
Euclid& \cellcolor{red!40}\href{../works/NishikawaSTT18.pdf}{NishikawaSTT18} (0.14)& \cellcolor{red!40}\href{../works/NishikawaSTT18a.pdf}{NishikawaSTT18a} (0.15)& \cellcolor{red!40}\href{../works/BoothNB16.pdf}{BoothNB16} (0.23)& \cellcolor{red!40}\href{../works/BeniniBGM05a.pdf}{BeniniBGM05a} (0.24)& \cellcolor{red!20}\href{../works/HoeveGSL07.pdf}{HoeveGSL07} (0.25)\\
Dot& \cellcolor{red!40}\href{../works/Groleaz21.pdf}{Groleaz21} (113.00)& \cellcolor{red!40}\href{../works/LaborieRSV18.pdf}{LaborieRSV18} (112.00)& \cellcolor{red!40}\href{../works/Lombardi10.pdf}{Lombardi10} (107.00)& \cellcolor{red!40}\href{../works/Astrand21.pdf}{Astrand21} (106.00)& \cellcolor{red!40}\href{../works/Beck99.pdf}{Beck99} (106.00)\\
Cosine& \cellcolor{red!40}\href{../works/NishikawaSTT18.pdf}{NishikawaSTT18} (0.93)& \cellcolor{red!40}\href{../works/NishikawaSTT18a.pdf}{NishikawaSTT18a} (0.92)& \cellcolor{red!40}\href{../works/BoothNB16.pdf}{BoothNB16} (0.81)& \cellcolor{red!40}\href{../works/BeniniBGM05a.pdf}{BeniniBGM05a} (0.79)& \cellcolor{red!40}\href{../works/ZouZ20.pdf}{ZouZ20} (0.77)\\
\index{NouriMHD23}NouriMHD23 R\&C& \cellcolor{red!40}\href{../works/Fatemi-AnarakiTFV23.pdf}{Fatemi-AnarakiTFV23} (0.83)& \cellcolor{green!20}\href{../works/AbreuPNF23.pdf}{AbreuPNF23} (0.96)& \cellcolor{blue!20}\href{../works/MengLZB21.pdf}{MengLZB21} (0.96)& \cellcolor{blue!20}GunerGSKD23 (0.97)& \cellcolor{blue!20}\href{../works/LunardiBLRV20.pdf}{LunardiBLRV20} (0.97)\\
Euclid\\
Dot\\
Cosine\\
\index{NovaraNH16}\href{../works/NovaraNH16.pdf}{NovaraNH16} R\&C& \cellcolor{red!40}\href{../works/ZeballosNH11.pdf}{ZeballosNH11} (0.63)& \cellcolor{red!40}\href{../works/ZeballosCM10.pdf}{ZeballosCM10} (0.84)& \cellcolor{yellow!20}\href{../works/HarjunkoskiMBC14.pdf}{HarjunkoskiMBC14} (0.90)& \cellcolor{yellow!20}\href{../works/NovasH14.pdf}{NovasH14} (0.91)& \cellcolor{yellow!20}\href{../works/ZeballosQH10.pdf}{ZeballosQH10} (0.93)\\
Euclid& \cellcolor{blue!20}\href{../works/ZeballosNH11.pdf}{ZeballosNH11} (0.32)& \cellcolor{black!20}\href{../works/Hooker05a.pdf}{Hooker05a} (0.35)& \cellcolor{black!20}\href{../works/Hooker06.pdf}{Hooker06} (0.35)& \cellcolor{black!20}\href{../works/NovasH10.pdf}{NovasH10} (0.36)& \cellcolor{black!20}\href{../works/Hooker07.pdf}{Hooker07} (0.36)\\
Dot& \cellcolor{red!40}\href{../works/LaborieRSV18.pdf}{LaborieRSV18} (169.00)& \cellcolor{red!40}\href{../works/Groleaz21.pdf}{Groleaz21} (157.00)& \cellcolor{red!40}\href{../works/AwadMDMT22.pdf}{AwadMDMT22} (155.00)& \cellcolor{red!40}\href{../works/Dejemeppe16.pdf}{Dejemeppe16} (154.00)& \cellcolor{red!40}\href{../works/Baptiste02.pdf}{Baptiste02} (150.00)\\
Cosine& \cellcolor{red!40}\href{../works/ZeballosNH11.pdf}{ZeballosNH11} (0.80)& \cellcolor{red!40}\href{../works/AwadMDMT22.pdf}{AwadMDMT22} (0.75)& \cellcolor{red!40}\href{../works/NovasH10.pdf}{NovasH10} (0.74)& \cellcolor{red!40}\href{../works/Hooker07.pdf}{Hooker07} (0.73)& \cellcolor{red!40}\href{../works/Hooker05a.pdf}{Hooker05a} (0.73)\\
\index{Novas19}\href{../works/Novas19.pdf}{Novas19} R\&C& \cellcolor{yellow!20}\href{../works/LunardiBLRV20.pdf}{LunardiBLRV20} (0.91)& \cellcolor{yellow!20}\href{../works/ZhangW18.pdf}{ZhangW18} (0.91)& \cellcolor{green!20}\href{../works/EscobetPQPRA19.pdf}{EscobetPQPRA19} (0.94)& \cellcolor{green!20}\href{../works/HamC16.pdf}{HamC16} (0.94)& \cellcolor{green!20}\href{../works/MengZRZL20.pdf}{MengZRZL20} (0.95)\\
Euclid& \cellcolor{black!20}\href{../works/KhayatLR06.pdf}{KhayatLR06} (0.35)& \cellcolor{black!20}\href{../works/MengZRZL20.pdf}{MengZRZL20} (0.37)& \cellcolor{black!20}\href{../works/MengLZB21.pdf}{MengLZB21} (0.37)& \cellcolor{black!20}\href{../works/OujanaAYB22.pdf}{OujanaAYB22} (0.37)& \cellcolor{black!20}\href{../works/MengGRZSC22.pdf}{MengGRZSC22} (0.37)\\
Dot& \cellcolor{red!40}\href{../works/ZarandiASC20.pdf}{ZarandiASC20} (220.00)& \cellcolor{red!40}\href{../works/Groleaz21.pdf}{Groleaz21} (201.00)& \cellcolor{red!40}\href{../works/Lunardi20.pdf}{Lunardi20} (200.00)& \cellcolor{red!40}\href{../works/IsikYA23.pdf}{IsikYA23} (194.00)& \cellcolor{red!40}\href{../works/Dejemeppe16.pdf}{Dejemeppe16} (189.00)\\
Cosine& \cellcolor{red!40}\href{../works/MengZRZL20.pdf}{MengZRZL20} (0.80)& \cellcolor{red!40}\href{../works/AwadMDMT22.pdf}{AwadMDMT22} (0.78)& \cellcolor{red!40}\href{../works/IsikYA23.pdf}{IsikYA23} (0.77)& \cellcolor{red!40}\href{../works/MengLZB21.pdf}{MengLZB21} (0.77)& \cellcolor{red!40}\href{../works/KhayatLR06.pdf}{KhayatLR06} (0.77)\\
\index{NovasH10}\href{../works/NovasH10.pdf}{NovasH10} R\&C& \cellcolor{red!20}\href{../works/NovasH12.pdf}{NovasH12} (0.89)& \cellcolor{red!20}\href{../works/ZeballosCM10.pdf}{ZeballosCM10} (0.90)& \cellcolor{yellow!20}\href{../works/ZeballosNH11.pdf}{ZeballosNH11} (0.91)& \cellcolor{yellow!20}\href{../works/RoePS05.pdf}{RoePS05} (0.92)& \cellcolor{yellow!20}\href{../works/HarjunkoskiMBC14.pdf}{HarjunkoskiMBC14} (0.92)\\
Euclid& \cellcolor{green!20}\href{../works/QuirogaZH05.pdf}{QuirogaZH05} (0.30)& \cellcolor{green!20}\href{../works/ZeballosH05.pdf}{ZeballosH05} (0.31)& \cellcolor{blue!20}\href{../works/ZeballosM09.pdf}{ZeballosM09} (0.32)& \cellcolor{blue!20}\href{../works/ZeballosNH11.pdf}{ZeballosNH11} (0.32)& \cellcolor{blue!20}\href{../works/Zeballos10.pdf}{Zeballos10} (0.33)\\
Dot& \cellcolor{red!40}\href{../works/ZarandiASC20.pdf}{ZarandiASC20} (167.00)& \cellcolor{red!40}\href{../works/Groleaz21.pdf}{Groleaz21} (152.00)& \cellcolor{red!40}\href{../works/HarjunkoskiMBC14.pdf}{HarjunkoskiMBC14} (142.00)& \cellcolor{red!40}\href{../works/Dejemeppe16.pdf}{Dejemeppe16} (141.00)& \cellcolor{red!40}\href{../works/Baptiste02.pdf}{Baptiste02} (139.00)\\
Cosine& \cellcolor{red!40}\href{../works/ZeballosNH11.pdf}{ZeballosNH11} (0.78)& \cellcolor{red!40}\href{../works/ZeballosH05.pdf}{ZeballosH05} (0.77)& \cellcolor{red!40}\href{../works/QuirogaZH05.pdf}{QuirogaZH05} (0.77)& \cellcolor{red!40}\href{../works/Zeballos10.pdf}{Zeballos10} (0.75)& \cellcolor{red!40}\href{../works/BidotVLB09.pdf}{BidotVLB09} (0.75)\\
\index{NovasH12}\href{../works/NovasH12.pdf}{NovasH12} R\&C& \cellcolor{red!40}\href{../works/ZeballosCM10.pdf}{ZeballosCM10} (0.78)& \cellcolor{red!20}\href{../works/NovasH10.pdf}{NovasH10} (0.89)& \cellcolor{yellow!20}\href{../works/OzturkTHO12.pdf}{OzturkTHO12} (0.91)& \cellcolor{yellow!20}\href{../works/ZeballosNH11.pdf}{ZeballosNH11} (0.93)& \cellcolor{green!20}\href{../works/ZeballosQH10.pdf}{ZeballosQH10} (0.94)\\
Euclid& \cellcolor{red!20}\href{../works/NovasH14.pdf}{NovasH14} (0.24)& \cellcolor{yellow!20}\href{../works/ZeballosH05.pdf}{ZeballosH05} (0.27)& \cellcolor{yellow!20}\href{../works/ZeballosCM10.pdf}{ZeballosCM10} (0.27)& \cellcolor{yellow!20}\href{../works/ZeballosM09.pdf}{ZeballosM09} (0.27)& \cellcolor{blue!20}\href{../works/OzturkTHO10.pdf}{OzturkTHO10} (0.32)\\
Dot& \cellcolor{red!40}\href{../works/ZarandiASC20.pdf}{ZarandiASC20} (137.00)& \cellcolor{red!40}\href{../works/LaborieRSV18.pdf}{LaborieRSV18} (128.00)& \cellcolor{red!40}\href{../works/ZeballosCM10.pdf}{ZeballosCM10} (124.00)& \cellcolor{red!40}\href{../works/Astrand21.pdf}{Astrand21} (121.00)& \cellcolor{red!40}\href{../works/Malapert11.pdf}{Malapert11} (120.00)\\
Cosine& \cellcolor{red!40}\href{../works/NovasH14.pdf}{NovasH14} (0.85)& \cellcolor{red!40}\href{../works/ZeballosCM10.pdf}{ZeballosCM10} (0.84)& \cellcolor{red!40}\href{../works/ZeballosH05.pdf}{ZeballosH05} (0.83)& \cellcolor{red!40}\href{../works/ZeballosM09.pdf}{ZeballosM09} (0.80)& \cellcolor{red!40}\href{../works/Zeballos10.pdf}{Zeballos10} (0.75)\\
\index{NovasH14}\href{../works/NovasH14.pdf}{NovasH14} R\&C& \cellcolor{red!40}\href{../works/ZeballosQH10.pdf}{ZeballosQH10} (0.71)& \cellcolor{red!40}\href{../works/Zeballos10.pdf}{Zeballos10} (0.72)& \cellcolor{yellow!20}\href{../works/KhayatLR06.pdf}{KhayatLR06} (0.91)& \cellcolor{yellow!20}\href{../works/ZarandiKS16.pdf}{ZarandiKS16} (0.91)& \cellcolor{yellow!20}\href{../works/NovaraNH16.pdf}{NovaraNH16} (0.91)\\
Euclid& \cellcolor{red!40}\href{../works/ZeballosH05.pdf}{ZeballosH05} (0.22)& \cellcolor{red!20}\href{../works/ZeballosM09.pdf}{ZeballosM09} (0.24)& \cellcolor{red!20}\href{../works/NovasH12.pdf}{NovasH12} (0.24)& \cellcolor{red!20}\href{../works/KhayatLR06.pdf}{KhayatLR06} (0.26)& \cellcolor{yellow!20}\href{../works/QuirogaZH05.pdf}{QuirogaZH05} (0.28)\\
Dot& \cellcolor{red!40}\href{../works/ZarandiASC20.pdf}{ZarandiASC20} (142.00)& \cellcolor{red!40}\href{../works/Lunardi20.pdf}{Lunardi20} (138.00)& \cellcolor{red!40}\href{../works/Astrand21.pdf}{Astrand21} (135.00)& \cellcolor{red!40}\href{../works/Dejemeppe16.pdf}{Dejemeppe16} (127.00)& \cellcolor{red!40}\href{../works/LaborieRSV18.pdf}{LaborieRSV18} (125.00)\\
Cosine& \cellcolor{red!40}\href{../works/ZeballosH05.pdf}{ZeballosH05} (0.87)& \cellcolor{red!40}\href{../works/NovasH12.pdf}{NovasH12} (0.85)& \cellcolor{red!40}\href{../works/ZeballosM09.pdf}{ZeballosM09} (0.83)& \cellcolor{red!40}\href{../works/KhayatLR06.pdf}{KhayatLR06} (0.81)& \cellcolor{red!40}\href{../works/ZeballosCM10.pdf}{ZeballosCM10} (0.81)\\
\index{NuijtenA94}\href{../works/NuijtenA94.pdf}{NuijtenA94} R\&C\\
Euclid& \cellcolor{red!40}\href{../works/NuijtenA96.pdf}{NuijtenA96} (0.08)& \cellcolor{red!20}\href{../works/NuijtenP98.pdf}{NuijtenP98} (0.25)& \cellcolor{green!20}\href{../works/TorresL00.pdf}{TorresL00} (0.29)& \cellcolor{green!20}\href{../works/CestaOS00.pdf}{CestaOS00} (0.29)& \cellcolor{green!20}\href{../works/DilkinaDH05.pdf}{DilkinaDH05} (0.30)\\
Dot& \cellcolor{red!40}\href{../works/Baptiste02.pdf}{Baptiste02} (124.00)& \cellcolor{red!40}\href{../works/ZarandiASC20.pdf}{ZarandiASC20} (118.00)& \cellcolor{red!40}\href{../works/JuvinHHL23.pdf}{JuvinHHL23} (115.00)& \cellcolor{red!40}\href{../works/Fahimi16.pdf}{Fahimi16} (114.00)& \cellcolor{red!40}\href{../works/NuijtenP98.pdf}{NuijtenP98} (113.00)\\
Cosine& \cellcolor{red!40}\href{../works/NuijtenA96.pdf}{NuijtenA96} (0.98)& \cellcolor{red!40}\href{../works/NuijtenP98.pdf}{NuijtenP98} (0.85)& \cellcolor{red!40}\href{../works/TorresL00.pdf}{TorresL00} (0.78)& \cellcolor{red!40}\href{../works/SourdN00.pdf}{SourdN00} (0.75)& \cellcolor{red!40}\href{../works/MenciaSV13.pdf}{MenciaSV13} (0.74)\\
\index{NuijtenA94a}NuijtenA94a R\&C\\
Euclid\\
Dot\\
Cosine\\
\index{NuijtenA96}\href{../works/NuijtenA96.pdf}{NuijtenA96} R\&C& \cellcolor{red!40}\href{../works/Rodriguez07.pdf}{Rodriguez07} (0.83)& \cellcolor{red!40}\href{../works/Salido10.pdf}{Salido10} (0.85)& \cellcolor{red!40}\href{../works/Laborie03.pdf}{Laborie03} (0.85)& \cellcolor{red!40}\href{../works/DincbasSH90.pdf}{DincbasSH90} (0.86)& \cellcolor{red!20}DorndorfHP99 (0.87)\\
Euclid& \cellcolor{red!40}\href{../works/NuijtenA94.pdf}{NuijtenA94} (0.08)& \cellcolor{red!40}\href{../works/NuijtenP98.pdf}{NuijtenP98} (0.24)& \cellcolor{green!20}\href{../works/TorresL00.pdf}{TorresL00} (0.29)& \cellcolor{green!20}\href{../works/TanSD10.pdf}{TanSD10} (0.29)& \cellcolor{green!20}\href{../works/CestaOS00.pdf}{CestaOS00} (0.29)\\
Dot& \cellcolor{red!40}\href{../works/Baptiste02.pdf}{Baptiste02} (128.00)& \cellcolor{red!40}\href{../works/ZarandiASC20.pdf}{ZarandiASC20} (124.00)& \cellcolor{red!40}\href{../works/JuvinHHL23.pdf}{JuvinHHL23} (118.00)& \cellcolor{red!40}\href{../works/Godet21a.pdf}{Godet21a} (118.00)& \cellcolor{red!40}\href{../works/BartakSR10.pdf}{BartakSR10} (118.00)\\
Cosine& \cellcolor{red!40}\href{../works/NuijtenA94.pdf}{NuijtenA94} (0.98)& \cellcolor{red!40}\href{../works/NuijtenP98.pdf}{NuijtenP98} (0.87)& \cellcolor{red!40}\href{../works/TorresL00.pdf}{TorresL00} (0.78)& \cellcolor{red!40}\href{../works/SourdN00.pdf}{SourdN00} (0.78)& \cellcolor{red!40}\href{../works/MenciaSV13.pdf}{MenciaSV13} (0.76)\\
\index{NuijtenP98}\href{../works/NuijtenP98.pdf}{NuijtenP98} R\&C& \cellcolor{yellow!20}\href{../works/NuijtenA96.pdf}{NuijtenA96} (0.93)& \cellcolor{green!20}\href{../works/SourdN00.pdf}{SourdN00} (0.94)& \cellcolor{green!20}\href{../works/KamarainenS02.pdf}{KamarainenS02} (0.94)& \cellcolor{blue!20}DorndorfHP99 (0.97)& \cellcolor{blue!20}\href{../works/Vilim04.pdf}{Vilim04} (0.97)\\
Euclid& \cellcolor{red!40}\href{../works/NuijtenA96.pdf}{NuijtenA96} (0.24)& \cellcolor{red!20}\href{../works/NuijtenA94.pdf}{NuijtenA94} (0.25)& \cellcolor{green!20}\href{../works/BartakSR08.pdf}{BartakSR08} (0.30)& \cellcolor{green!20}\href{../works/SourdN00.pdf}{SourdN00} (0.31)& \cellcolor{green!20}\href{../works/MenciaSV13.pdf}{MenciaSV13} (0.31)\\
Dot& \cellcolor{red!40}\href{../works/Baptiste02.pdf}{Baptiste02} (174.00)& \cellcolor{red!40}\href{../works/ZarandiASC20.pdf}{ZarandiASC20} (165.00)& \cellcolor{red!40}\href{../works/Godet21a.pdf}{Godet21a} (147.00)& \cellcolor{red!40}\href{../works/BartakSR10.pdf}{BartakSR10} (147.00)& \cellcolor{red!40}\href{../works/Fahimi16.pdf}{Fahimi16} (146.00)\\
Cosine& \cellcolor{red!40}\href{../works/NuijtenA96.pdf}{NuijtenA96} (0.87)& \cellcolor{red!40}\href{../works/NuijtenA94.pdf}{NuijtenA94} (0.85)& \cellcolor{red!40}\href{../works/BartakSR08.pdf}{BartakSR08} (0.80)& \cellcolor{red!40}\href{../works/SourdN00.pdf}{SourdN00} (0.79)& \cellcolor{red!40}\href{../works/MenciaSV13.pdf}{MenciaSV13} (0.78)\\
\index{OddiPCC03}\href{../works/OddiPCC03.pdf}{OddiPCC03} R\&C& \cellcolor{red!40}OddiPCC05 (0.72)& \cellcolor{red!40}\href{../works/VerfaillieL01.pdf}{VerfaillieL01} (0.81)& \cellcolor{yellow!20}\href{../works/YuraszeckMC23.pdf}{YuraszeckMC23} (0.90)& \cellcolor{green!20}\href{../works/WikarekS19.pdf}{WikarekS19} (0.94)& \cellcolor{green!20}\href{../works/HamdiL13.pdf}{HamdiL13} (0.94)\\
Euclid& \cellcolor{yellow!20}\href{../works/WallaceF00.pdf}{WallaceF00} (0.27)& \cellcolor{yellow!20}\href{../works/ReddyFIBKAJ11.pdf}{ReddyFIBKAJ11} (0.27)& \cellcolor{yellow!20}\href{../works/FukunagaHFAMN02.pdf}{FukunagaHFAMN02} (0.28)& \cellcolor{green!20}\href{../works/AstrandJZ18.pdf}{AstrandJZ18} (0.29)& \cellcolor{green!20}\href{../works/KuchcinskiW03.pdf}{KuchcinskiW03} (0.29)\\
Dot& \cellcolor{red!40}\href{../works/ZarandiASC20.pdf}{ZarandiASC20} (104.00)& \cellcolor{red!40}\href{../works/Groleaz21.pdf}{Groleaz21} (101.00)& \cellcolor{red!40}\href{../works/Astrand21.pdf}{Astrand21} (97.00)& \cellcolor{red!40}\href{../works/Godet21a.pdf}{Godet21a} (94.00)& \cellcolor{red!40}\href{../works/Dejemeppe16.pdf}{Dejemeppe16} (94.00)\\
Cosine& \cellcolor{red!40}\href{../works/ReddyFIBKAJ11.pdf}{ReddyFIBKAJ11} (0.72)& \cellcolor{red!40}\href{../works/Malik08.pdf}{Malik08} (0.71)& \cellcolor{red!40}\href{../works/AstrandJZ18.pdf}{AstrandJZ18} (0.68)& \cellcolor{red!40}\href{../works/OddiRCS11.pdf}{OddiRCS11} (0.67)& \cellcolor{red!40}\href{../works/BonfiettiLBM14.pdf}{BonfiettiLBM14} (0.67)\\
\index{OddiPCC05}OddiPCC05 R\&C& \cellcolor{red!40}\href{../works/OddiPCC03.pdf}{OddiPCC03} (0.72)& \cellcolor{red!20}CestaOPS14 (0.86)& \cellcolor{red!20}\href{../works/MercierH08.pdf}{MercierH08} (0.88)& \cellcolor{red!20}\href{../works/SimoninAHL12.pdf}{SimoninAHL12} (0.90)& \cellcolor{red!20}\href{../works/AkkerDH07.pdf}{AkkerDH07} (0.90)\\
Euclid\\
Dot\\
Cosine\\
\index{OddiRC10}\href{../works/OddiRC10.pdf}{OddiRC10} R\&C\\
Euclid& \cellcolor{red!40}\href{../works/LombardiM13.pdf}{LombardiM13} (0.16)& \cellcolor{red!40}\href{../works/LombardiM12a.pdf}{LombardiM12a} (0.19)& \cellcolor{red!40}\href{../works/BofillCSV17a.pdf}{BofillCSV17a} (0.22)& \cellcolor{red!40}\href{../works/BofillCSV17.pdf}{BofillCSV17} (0.22)& \cellcolor{red!40}\href{../works/BhatnagarKL19.pdf}{BhatnagarKL19} (0.22)\\
Dot& \cellcolor{red!40}\href{../works/Lombardi10.pdf}{Lombardi10} (89.00)& \cellcolor{red!40}\href{../works/Godet21a.pdf}{Godet21a} (88.00)& \cellcolor{red!40}\href{../works/Baptiste02.pdf}{Baptiste02} (88.00)& \cellcolor{red!40}\href{../works/Schutt11.pdf}{Schutt11} (86.00)& \cellcolor{red!40}\href{../works/ZarandiASC20.pdf}{ZarandiASC20} (82.00)\\
Cosine& \cellcolor{red!40}\href{../works/LombardiM13.pdf}{LombardiM13} (0.88)& \cellcolor{red!40}\href{../works/LombardiM12a.pdf}{LombardiM12a} (0.86)& \cellcolor{red!40}\href{../works/CestaOF99.pdf}{CestaOF99} (0.83)& \cellcolor{red!40}\href{../works/BofillCSV17a.pdf}{BofillCSV17a} (0.80)& \cellcolor{red!40}\href{../works/BofillCSV17.pdf}{BofillCSV17} (0.80)\\
\index{OddiRCS11}\href{../works/OddiRCS11.pdf}{OddiRCS11} R\&C\\
Euclid& \cellcolor{green!20}\href{../works/CestaOS00.pdf}{CestaOS00} (0.30)& \cellcolor{green!20}\href{../works/PacinoH11.pdf}{PacinoH11} (0.31)& \cellcolor{blue!20}\href{../works/HentenryckM04.pdf}{HentenryckM04} (0.32)& \cellcolor{blue!20}\href{../works/GodardLN05.pdf}{GodardLN05} (0.32)& \cellcolor{blue!20}\href{../works/YuraszeckMCCR23.pdf}{YuraszeckMCCR23} (0.32)\\
Dot& \cellcolor{red!40}\href{../works/ZarandiASC20.pdf}{ZarandiASC20} (154.00)& \cellcolor{red!40}\href{../works/Groleaz21.pdf}{Groleaz21} (152.00)& \cellcolor{red!40}\href{../works/Malapert11.pdf}{Malapert11} (150.00)& \cellcolor{red!40}\href{../works/Godet21a.pdf}{Godet21a} (145.00)& \cellcolor{red!40}\href{../works/Dejemeppe16.pdf}{Dejemeppe16} (145.00)\\
Cosine& \cellcolor{red!40}\href{../works/YuraszeckMCCR23.pdf}{YuraszeckMCCR23} (0.79)& \cellcolor{red!40}\href{../works/CestaOS00.pdf}{CestaOS00} (0.77)& \cellcolor{red!40}\href{../works/MalapertCGJLR12.pdf}{MalapertCGJLR12} (0.76)& \cellcolor{red!40}\href{../works/PacinoH11.pdf}{PacinoH11} (0.76)& \cellcolor{red!40}\href{../works/MenciaSV13.pdf}{MenciaSV13} (0.76)\\
\index{OddiS97}\href{../works/OddiS97.pdf}{OddiS97} R\&C\\
Euclid& \cellcolor{red!40}\href{../works/SmithC93.pdf}{SmithC93} (0.18)& \cellcolor{red!40}\href{../works/Muscettola94.pdf}{Muscettola94} (0.20)& \cellcolor{red!40}\href{../works/BeckW05.pdf}{BeckW05} (0.23)& \cellcolor{red!40}\href{../works/BidotVLB07.pdf}{BidotVLB07} (0.24)& \cellcolor{red!20}\href{../works/DilkinaDH05.pdf}{DilkinaDH05} (0.25)\\
Dot& \cellcolor{red!40}\href{../works/ZarandiASC20.pdf}{ZarandiASC20} (94.00)& \cellcolor{red!40}\href{../works/Groleaz21.pdf}{Groleaz21} (92.00)& \cellcolor{red!40}\href{../works/SadehF96.pdf}{SadehF96} (90.00)& \cellcolor{red!40}\href{../works/Astrand21.pdf}{Astrand21} (88.00)& \cellcolor{red!40}\href{../works/Dejemeppe16.pdf}{Dejemeppe16} (88.00)\\
Cosine& \cellcolor{red!40}\href{../works/SmithC93.pdf}{SmithC93} (0.86)& \cellcolor{red!40}\href{../works/Muscettola94.pdf}{Muscettola94} (0.84)& \cellcolor{red!40}\href{../works/SadehF96.pdf}{SadehF96} (0.83)& \cellcolor{red!40}\href{../works/BeckW05.pdf}{BeckW05} (0.76)& \cellcolor{red!40}\href{../works/CestaOF99.pdf}{CestaOF99} (0.76)\\
\index{OhrimenkoSC09}\href{../works/OhrimenkoSC09.pdf}{OhrimenkoSC09} R\&C& \cellcolor{red!40}\href{../works/SchuttFSW11.pdf}{SchuttFSW11} (0.81)& \cellcolor{red!40}\href{../works/SchuttFSW13.pdf}{SchuttFSW13} (0.85)& \cellcolor{red!20}\href{../works/SchuttCSW12.pdf}{SchuttCSW12} (0.87)& \cellcolor{yellow!20}\href{../works/SchuttFSW09.pdf}{SchuttFSW09} (0.91)& \cellcolor{yellow!20}\href{../works/KreterSS15.pdf}{KreterSS15} (0.93)\\
Euclid& \cellcolor{blue!20}\href{../works/ZhangLS12.pdf}{ZhangLS12} (0.33)& \cellcolor{blue!20}\href{../works/LiuLH19.pdf}{LiuLH19} (0.33)& \cellcolor{blue!20}\href{../works/BandaSC11.pdf}{BandaSC11} (0.33)& \cellcolor{blue!20}\href{../works/GelainPRVW17.pdf}{GelainPRVW17} (0.34)& \cellcolor{blue!20}\href{../works/MalapertCGJLR13.pdf}{MalapertCGJLR13} (0.34)\\
Dot& \cellcolor{red!40}\href{../works/Siala15a.pdf}{Siala15a} (118.00)& \cellcolor{red!40}\href{../works/Fahimi16.pdf}{Fahimi16} (113.00)& \cellcolor{red!40}\href{../works/Schutt11.pdf}{Schutt11} (110.00)& \cellcolor{red!40}\href{../works/Godet21a.pdf}{Godet21a} (110.00)& \cellcolor{red!40}\href{../works/Dejemeppe16.pdf}{Dejemeppe16} (105.00)\\
Cosine& \cellcolor{red!40}\href{../works/Bit-Monnot23.pdf}{Bit-Monnot23} (0.67)& \cellcolor{red!40}\href{../works/JussienL02.pdf}{JussienL02} (0.63)& \cellcolor{red!40}\href{../works/MalapertCGJLR13.pdf}{MalapertCGJLR13} (0.63)& \cellcolor{red!40}\href{../works/LiuLH19a.pdf}{LiuLH19a} (0.62)& \cellcolor{red!40}\href{../works/LiuLH19.pdf}{LiuLH19} (0.62)\\
\index{OkanoDTRYA04}OkanoDTRYA04 R\&C\\
Euclid\\
Dot\\
Cosine\\
\index{OrnekO16}\href{../works/OrnekO16.pdf}{OrnekO16} R\&C\\
Euclid& \cellcolor{green!20}\href{../works/OzturkTHO12.pdf}{OzturkTHO12} (0.29)& \cellcolor{green!20}\href{../works/ArtiguesBF04.pdf}{ArtiguesBF04} (0.30)& \cellcolor{green!20}\href{../works/TanSD10.pdf}{TanSD10} (0.30)& \cellcolor{green!20}\href{../works/KhayatLR06.pdf}{KhayatLR06} (0.30)& \cellcolor{green!20}\href{../works/CauwelaertDMS16.pdf}{CauwelaertDMS16} (0.30)\\
Dot& \cellcolor{red!40}\href{../works/Baptiste02.pdf}{Baptiste02} (171.00)& \cellcolor{red!40}\href{../works/ZarandiASC20.pdf}{ZarandiASC20} (168.00)& \cellcolor{red!40}\href{../works/Dejemeppe16.pdf}{Dejemeppe16} (168.00)& \cellcolor{red!40}\href{../works/Groleaz21.pdf}{Groleaz21} (165.00)& \cellcolor{red!40}\href{../works/Lombardi10.pdf}{Lombardi10} (157.00)\\
Cosine& \cellcolor{red!40}\href{../works/OzturkTHO12.pdf}{OzturkTHO12} (0.81)& \cellcolor{red!40}\href{../works/ArtiguesBF04.pdf}{ArtiguesBF04} (0.79)& \cellcolor{red!40}\href{../works/KhayatLR06.pdf}{KhayatLR06} (0.79)& \cellcolor{red!40}\href{../works/TanSD10.pdf}{TanSD10} (0.79)& \cellcolor{red!40}\href{../works/ArtiguesF07.pdf}{ArtiguesF07} (0.78)\\
\index{OrnekOS20}\href{../works/OrnekOS20.pdf}{OrnekOS20} R\&C\\
Euclid& \cellcolor{blue!20}\href{../works/ZibranR11a.pdf}{ZibranR11a} (0.33)& \cellcolor{blue!20}\href{../works/BofillGSV15.pdf}{BofillGSV15} (0.34)& \cellcolor{black!20}\href{../works/KucukY19.pdf}{KucukY19} (0.34)& \cellcolor{black!20}\href{../works/ZibranR11.pdf}{ZibranR11} (0.34)& \cellcolor{black!20}\href{../works/Puget95.pdf}{Puget95} (0.35)\\
Dot& \cellcolor{red!40}\href{../works/ZarandiASC20.pdf}{ZarandiASC20} (108.00)& \cellcolor{red!40}\href{../works/Astrand21.pdf}{Astrand21} (97.00)& \cellcolor{red!40}\href{../works/Lunardi20.pdf}{Lunardi20} (96.00)& \cellcolor{red!40}\href{../works/Lombardi10.pdf}{Lombardi10} (93.00)& \cellcolor{red!40}\href{../works/Lemos21.pdf}{Lemos21} (93.00)\\
Cosine& \cellcolor{red!40}\href{../works/GokPTGO23.pdf}{GokPTGO23} (0.62)& \cellcolor{red!40}\href{../works/DemirovicS18.pdf}{DemirovicS18} (0.60)& \cellcolor{red!40}\href{../works/ZibranR11a.pdf}{ZibranR11a} (0.60)& \cellcolor{red!40}\href{../works/ZouZ20.pdf}{ZouZ20} (0.59)& \cellcolor{red!40}\href{../works/AlesioBNG15.pdf}{AlesioBNG15} (0.59)\\
\index{OuelletQ13}\href{../works/OuelletQ13.pdf}{OuelletQ13} R\&C& \cellcolor{red!40}\href{../works/KameugneFSN14.pdf}{KameugneFSN14} (0.54)& \cellcolor{red!40}\href{../works/LetortBC12.pdf}{LetortBC12} (0.57)& \cellcolor{red!40}\href{../works/GayHS15.pdf}{GayHS15} (0.60)& \cellcolor{red!40}\href{../works/OuelletQ18.pdf}{OuelletQ18} (0.60)& \cellcolor{red!40}\href{../works/KameugneF13.pdf}{KameugneF13} (0.61)\\
Euclid& \cellcolor{red!40}\href{../works/KameugneFND23.pdf}{KameugneFND23} (0.22)& \cellcolor{red!40}\href{../works/GingrasQ16.pdf}{GingrasQ16} (0.22)& \cellcolor{red!40}\href{../works/OuelletQ18.pdf}{OuelletQ18} (0.24)& \cellcolor{red!20}\href{../works/KameugneFSN11.pdf}{KameugneFSN11} (0.26)& \cellcolor{yellow!20}\href{../works/KameugneFGOQ18.pdf}{KameugneFGOQ18} (0.27)\\
Dot& \cellcolor{red!40}\href{../works/Fahimi16.pdf}{Fahimi16} (142.00)& \cellcolor{red!40}\href{../works/FahimiOQ18.pdf}{FahimiOQ18} (133.00)& \cellcolor{red!40}\href{../works/Schutt11.pdf}{Schutt11} (132.00)& \cellcolor{red!40}\href{../works/KameugneFND23.pdf}{KameugneFND23} (131.00)& \cellcolor{red!40}\href{../works/FetgoD22.pdf}{FetgoD22} (128.00)\\
Cosine& \cellcolor{red!40}\href{../works/KameugneFND23.pdf}{KameugneFND23} (0.89)& \cellcolor{red!40}\href{../works/GingrasQ16.pdf}{GingrasQ16} (0.86)& \cellcolor{red!40}\href{../works/OuelletQ18.pdf}{OuelletQ18} (0.84)& \cellcolor{red!40}\href{../works/FetgoD22.pdf}{FetgoD22} (0.84)& \cellcolor{red!40}\href{../works/KameugneFSN11.pdf}{KameugneFSN11} (0.82)\\
\index{OuelletQ18}\href{../works/OuelletQ18.pdf}{OuelletQ18} R\&C& \cellcolor{red!40}\href{../works/Tesch18.pdf}{Tesch18} (0.38)& \cellcolor{red!40}\href{../works/KameugneFSN14.pdf}{KameugneFSN14} (0.56)& \cellcolor{red!40}\href{../works/OuelletQ13.pdf}{OuelletQ13} (0.60)& \cellcolor{red!40}\href{../works/KameugneF13.pdf}{KameugneF13} (0.60)& \cellcolor{red!40}\href{../works/Tesch16.pdf}{Tesch16} (0.63)\\
Euclid& \cellcolor{red!40}\href{../works/GayHS15a.pdf}{GayHS15a} (0.23)& \cellcolor{red!40}\href{../works/OuelletQ13.pdf}{OuelletQ13} (0.24)& \cellcolor{red!20}\href{../works/Vilim11.pdf}{Vilim11} (0.24)& \cellcolor{yellow!20}\href{../works/CauwelaertLS15.pdf}{CauwelaertLS15} (0.28)& \cellcolor{yellow!20}\href{../works/KameugneFSN11.pdf}{KameugneFSN11} (0.28)\\
Dot& \cellcolor{red!40}\href{../works/Schutt11.pdf}{Schutt11} (119.00)& \cellcolor{red!40}\href{../works/Fahimi16.pdf}{Fahimi16} (118.00)& \cellcolor{red!40}\href{../works/KameugneFND23.pdf}{KameugneFND23} (110.00)& \cellcolor{red!40}\href{../works/FahimiOQ18.pdf}{FahimiOQ18} (110.00)& \cellcolor{red!40}\href{../works/SchuttFS13a.pdf}{SchuttFS13a} (106.00)\\
Cosine& \cellcolor{red!40}\href{../works/GayHS15a.pdf}{GayHS15a} (0.85)& \cellcolor{red!40}\href{../works/OuelletQ13.pdf}{OuelletQ13} (0.84)& \cellcolor{red!40}\href{../works/Vilim11.pdf}{Vilim11} (0.81)& \cellcolor{red!40}\href{../works/KameugneFND23.pdf}{KameugneFND23} (0.81)& \cellcolor{red!40}\href{../works/SchuttFS13a.pdf}{SchuttFS13a} (0.77)\\
\index{OuelletQ22}\href{../works/OuelletQ22.pdf}{OuelletQ22} R\&C& \cellcolor{red!40}\href{../works/OuelletQ18.pdf}{OuelletQ18} (0.82)& \cellcolor{red!40}\href{../works/OuelletQ13.pdf}{OuelletQ13} (0.83)& \cellcolor{red!40}SchuttFSW15 (0.84)& \cellcolor{red!40}\href{../works/GayHS15a.pdf}{GayHS15a} (0.85)& \cellcolor{red!40}\href{../works/FetgoD22.pdf}{FetgoD22} (0.86)\\
Euclid& \cellcolor{yellow!20}\href{../works/Vilim11.pdf}{Vilim11} (0.28)& \cellcolor{green!20}\href{../works/OuelletQ18.pdf}{OuelletQ18} (0.29)& \cellcolor{green!20}\href{../works/OuelletQ13.pdf}{OuelletQ13} (0.29)& \cellcolor{green!20}\href{../works/Vilim09a.pdf}{Vilim09a} (0.30)& \cellcolor{green!20}\href{../works/GayHS15a.pdf}{GayHS15a} (0.30)\\
Dot& \cellcolor{red!40}\href{../works/Fahimi16.pdf}{Fahimi16} (131.00)& \cellcolor{red!40}\href{../works/Schutt11.pdf}{Schutt11} (119.00)& \cellcolor{red!40}\href{../works/Dejemeppe16.pdf}{Dejemeppe16} (117.00)& \cellcolor{red!40}\href{../works/FahimiOQ18.pdf}{FahimiOQ18} (115.00)& \cellcolor{red!40}\href{../works/Baptiste02.pdf}{Baptiste02} (111.00)\\
Cosine& \cellcolor{red!40}\href{../works/OuelletQ13.pdf}{OuelletQ13} (0.78)& \cellcolor{red!40}\href{../works/Vilim11.pdf}{Vilim11} (0.77)& \cellcolor{red!40}\href{../works/OuelletQ18.pdf}{OuelletQ18} (0.77)& \cellcolor{red!40}\href{../works/GayHS15a.pdf}{GayHS15a} (0.76)& \cellcolor{red!40}\href{../works/Vilim09a.pdf}{Vilim09a} (0.72)\\
\index{OujanaAYB22}\href{../works/OujanaAYB22.pdf}{OujanaAYB22} R\&C& \cellcolor{red!20}\href{../works/MengLZB21.pdf}{MengLZB21} (0.90)& \cellcolor{yellow!20}\href{../works/MengGRZSC22.pdf}{MengGRZSC22} (0.91)& \cellcolor{yellow!20}\href{../works/MengZRZL20.pdf}{MengZRZL20} (0.92)& \cellcolor{green!20}\href{../works/YunusogluY22.pdf}{YunusogluY22} (0.94)& \cellcolor{green!20}\href{../works/YuraszeckMPV22.pdf}{YuraszeckMPV22} (0.94)\\
Euclid& \cellcolor{black!20}\href{../works/LiFJZLL22.pdf}{LiFJZLL22} (0.35)& \cellcolor{black!20}\href{../works/ZhouGL15.pdf}{ZhouGL15} (0.36)& \cellcolor{black!20}\href{../works/ParkUJR19.pdf}{ParkUJR19} (0.37)& \cellcolor{black!20}\href{../works/Teppan22.pdf}{Teppan22} (0.37)& \cellcolor{black!20}\href{../works/HeinzNVH22.pdf}{HeinzNVH22} (0.37)\\
Dot& \cellcolor{red!40}\href{../works/ZarandiASC20.pdf}{ZarandiASC20} (197.00)& \cellcolor{red!40}\href{../works/Lunardi20.pdf}{Lunardi20} (191.00)& \cellcolor{red!40}\href{../works/Groleaz21.pdf}{Groleaz21} (185.00)& \cellcolor{red!40}\href{../works/Astrand21.pdf}{Astrand21} (172.00)& \cellcolor{red!40}\href{../works/IsikYA23.pdf}{IsikYA23} (171.00)\\
Cosine& \cellcolor{red!40}\href{../works/Novas19.pdf}{Novas19} (0.76)& \cellcolor{red!40}\href{../works/MengZRZL20.pdf}{MengZRZL20} (0.75)& \cellcolor{red!40}\href{../works/LiFJZLL22.pdf}{LiFJZLL22} (0.74)& \cellcolor{red!40}\href{../works/PrataAN23.pdf}{PrataAN23} (0.74)& \cellcolor{red!40}\href{../works/AbreuPNF23.pdf}{AbreuPNF23} (0.74)\\
\index{OzturkTHO10}\href{../works/OzturkTHO10.pdf}{OzturkTHO10} R\&C& \cellcolor{green!20}\href{../works/OzturkTHO13.pdf}{OzturkTHO13} (0.93)& \cellcolor{green!20}\href{../works/OzturkTHO12.pdf}{OzturkTHO12} (0.95)& \cellcolor{green!20}\href{../works/ValleMGT03.pdf}{ValleMGT03} (0.95)& \cellcolor{green!20}\href{../works/TerekhovDOB12.pdf}{TerekhovDOB12} (0.96)& \cellcolor{green!20}\href{../works/TopalogluSS12.pdf}{TopalogluSS12} (0.96)\\
Euclid& \cellcolor{yellow!20}\href{../works/OzturkTHO12.pdf}{OzturkTHO12} (0.27)& \cellcolor{yellow!20}\href{../works/BeniniBGM05a.pdf}{BeniniBGM05a} (0.27)& \cellcolor{yellow!20}\href{../works/LeeKLKKYHP97.pdf}{LeeKLKKYHP97} (0.28)& \cellcolor{yellow!20}\href{../works/FortinZDF05.pdf}{FortinZDF05} (0.28)& \cellcolor{yellow!20}\href{../works/KhayatLR06.pdf}{KhayatLR06} (0.28)\\
Dot& \cellcolor{red!40}\href{../works/OzturkTHO13.pdf}{OzturkTHO13} (111.00)& \cellcolor{red!40}\href{../works/Baptiste02.pdf}{Baptiste02} (105.00)& \cellcolor{red!40}\href{../works/Malapert11.pdf}{Malapert11} (103.00)& \cellcolor{red!40}\href{../works/Lunardi20.pdf}{Lunardi20} (102.00)& \cellcolor{red!40}\href{../works/Dejemeppe16.pdf}{Dejemeppe16} (101.00)\\
Cosine& \cellcolor{red!40}\href{../works/OzturkTHO13.pdf}{OzturkTHO13} (0.81)& \cellcolor{red!40}\href{../works/OzturkTHO12.pdf}{OzturkTHO12} (0.77)& \cellcolor{red!40}\href{../works/OzturkTHO15.pdf}{OzturkTHO15} (0.76)& \cellcolor{red!40}\href{../works/KhayatLR06.pdf}{KhayatLR06} (0.75)& \cellcolor{red!40}\href{../works/ZeballosH05.pdf}{ZeballosH05} (0.74)\\
\index{OzturkTHO12}\href{../works/OzturkTHO12.pdf}{OzturkTHO12} R\&C& \cellcolor{red!40}\href{../works/ZhangLS12.pdf}{ZhangLS12} (0.83)& \cellcolor{red!40}\href{../works/QuirogaZH05.pdf}{QuirogaZH05} (0.86)& \cellcolor{red!40}\href{../works/Geske05.pdf}{Geske05} (0.86)& \cellcolor{red!20}\href{../works/EvenSH15.pdf}{EvenSH15} (0.88)& \cellcolor{red!20}\href{../works/KovacsV04.pdf}{KovacsV04} (0.88)\\
Euclid& \cellcolor{red!40}\href{../works/OzturkTHO15.pdf}{OzturkTHO15} (0.19)& \cellcolor{yellow!20}\href{../works/CauwelaertDMS16.pdf}{CauwelaertDMS16} (0.27)& \cellcolor{yellow!20}\href{../works/OzturkTHO10.pdf}{OzturkTHO10} (0.27)& \cellcolor{yellow!20}\href{../works/KhayatLR06.pdf}{KhayatLR06} (0.28)& \cellcolor{yellow!20}\href{../works/HeipckeCCS00.pdf}{HeipckeCCS00} (0.28)\\
Dot& \cellcolor{red!40}\href{../works/Dejemeppe16.pdf}{Dejemeppe16} (135.00)& \cellcolor{red!40}\href{../works/Schutt11.pdf}{Schutt11} (130.00)& \cellcolor{red!40}\href{../works/Lombardi10.pdf}{Lombardi10} (130.00)& \cellcolor{red!40}\href{../works/Malapert11.pdf}{Malapert11} (129.00)& \cellcolor{red!40}\href{../works/Baptiste02.pdf}{Baptiste02} (129.00)\\
Cosine& \cellcolor{red!40}\href{../works/OzturkTHO15.pdf}{OzturkTHO15} (0.92)& \cellcolor{red!40}\href{../works/OzturkTHO13.pdf}{OzturkTHO13} (0.82)& \cellcolor{red!40}\href{../works/OrnekO16.pdf}{OrnekO16} (0.81)& \cellcolor{red!40}\href{../works/DejemeppeCS15.pdf}{DejemeppeCS15} (0.79)& \cellcolor{red!40}\href{../works/Wolf05.pdf}{Wolf05} (0.78)\\
\index{OzturkTHO13}\href{../works/OzturkTHO13.pdf}{OzturkTHO13} R\&C& \cellcolor{red!40}\href{../works/OzturkTHO15.pdf}{OzturkTHO15} (0.71)& \cellcolor{red!40}\href{../works/TopalogluSS12.pdf}{TopalogluSS12} (0.85)& \cellcolor{red!20}\href{../works/PinarbasiAY19.pdf}{PinarbasiAY19} (0.88)& \cellcolor{red!20}\href{../works/ValleMGT03.pdf}{ValleMGT03} (0.89)& \cellcolor{yellow!20}PinarbasiA20 (0.91)\\
Euclid& \cellcolor{yellow!20}\href{../works/OzturkTHO15.pdf}{OzturkTHO15} (0.27)& \cellcolor{green!20}\href{../works/OzturkTHO12.pdf}{OzturkTHO12} (0.30)& \cellcolor{green!20}\href{../works/OzturkTHO10.pdf}{OzturkTHO10} (0.31)& \cellcolor{blue!20}\href{../works/KhayatLR06.pdf}{KhayatLR06} (0.34)& \cellcolor{blue!20}\href{../works/OrnekO16.pdf}{OrnekO16} (0.34)\\
Dot& \cellcolor{red!40}\href{../works/Baptiste02.pdf}{Baptiste02} (182.00)& \cellcolor{red!40}\href{../works/Malapert11.pdf}{Malapert11} (172.00)& \cellcolor{red!40}\href{../works/Dejemeppe16.pdf}{Dejemeppe16} (171.00)& \cellcolor{red!40}\href{../works/Groleaz21.pdf}{Groleaz21} (169.00)& \cellcolor{red!40}\href{../works/Lombardi10.pdf}{Lombardi10} (168.00)\\
Cosine& \cellcolor{red!40}\href{../works/OzturkTHO15.pdf}{OzturkTHO15} (0.86)& \cellcolor{red!40}\href{../works/OzturkTHO12.pdf}{OzturkTHO12} (0.82)& \cellcolor{red!40}\href{../works/OzturkTHO10.pdf}{OzturkTHO10} (0.81)& \cellcolor{red!40}\href{../works/OrnekO16.pdf}{OrnekO16} (0.78)& \cellcolor{red!40}\href{../works/BartakSR08.pdf}{BartakSR08} (0.77)\\
\index{OzturkTHO15}\href{../works/OzturkTHO15.pdf}{OzturkTHO15} R\&C& \cellcolor{red!40}\href{../works/OzturkTHO13.pdf}{OzturkTHO13} (0.71)& \cellcolor{red!20}\href{../works/ValleMGT03.pdf}{ValleMGT03} (0.88)& \cellcolor{red!20}\href{../works/AlakaPY19.pdf}{AlakaPY19} (0.90)& \cellcolor{red!20}\href{../works/TopalogluSS12.pdf}{TopalogluSS12} (0.90)& \cellcolor{yellow!20}\href{../works/PinarbasiAY19.pdf}{PinarbasiAY19} (0.91)\\
Euclid& \cellcolor{red!40}\href{../works/OzturkTHO12.pdf}{OzturkTHO12} (0.19)& \cellcolor{yellow!20}\href{../works/OzturkTHO13.pdf}{OzturkTHO13} (0.27)& \cellcolor{green!20}\href{../works/OzturkTHO10.pdf}{OzturkTHO10} (0.31)& \cellcolor{blue!20}\href{../works/HeipckeCCS00.pdf}{HeipckeCCS00} (0.32)& \cellcolor{blue!20}\href{../works/KhayatLR06.pdf}{KhayatLR06} (0.32)\\
Dot& \cellcolor{red!40}\href{../works/Lombardi10.pdf}{Lombardi10} (153.00)& \cellcolor{red!40}\href{../works/Groleaz21.pdf}{Groleaz21} (151.00)& \cellcolor{red!40}\href{../works/Astrand21.pdf}{Astrand21} (150.00)& \cellcolor{red!40}\href{../works/Dejemeppe16.pdf}{Dejemeppe16} (150.00)& \cellcolor{red!40}\href{../works/OzturkTHO13.pdf}{OzturkTHO13} (148.00)\\
Cosine& \cellcolor{red!40}\href{../works/OzturkTHO12.pdf}{OzturkTHO12} (0.92)& \cellcolor{red!40}\href{../works/OzturkTHO13.pdf}{OzturkTHO13} (0.86)& \cellcolor{red!40}\href{../works/OrnekO16.pdf}{OrnekO16} (0.77)& \cellcolor{red!40}\href{../works/OzturkTHO10.pdf}{OzturkTHO10} (0.76)& \cellcolor{red!40}\href{../works/KhayatLR06.pdf}{KhayatLR06} (0.75)\\
\index{PachecoPR19}\href{../works/PachecoPR19.pdf}{PachecoPR19} R\&C\\
Euclid& \cellcolor{yellow!20}\href{../works/KovacsV06.pdf}{KovacsV06} (0.27)& \cellcolor{yellow!20}\href{../works/WessenCS20.pdf}{WessenCS20} (0.28)& \cellcolor{yellow!20}\href{../works/FortinZDF05.pdf}{FortinZDF05} (0.28)& \cellcolor{yellow!20}\href{../works/ValleMGT03.pdf}{ValleMGT03} (0.28)& \cellcolor{yellow!20}\href{../works/BoothNB16.pdf}{BoothNB16} (0.28)\\
Dot& \cellcolor{red!40}\href{../works/Dejemeppe16.pdf}{Dejemeppe16} (115.00)& \cellcolor{red!40}\href{../works/Groleaz21.pdf}{Groleaz21} (113.00)& \cellcolor{red!40}\href{../works/LaborieRSV18.pdf}{LaborieRSV18} (112.00)& \cellcolor{red!40}\href{../works/Astrand21.pdf}{Astrand21} (112.00)& \cellcolor{red!40}\href{../works/ZarandiASC20.pdf}{ZarandiASC20} (111.00)\\
Cosine& \cellcolor{red!40}\href{../works/KovacsV06.pdf}{KovacsV06} (0.74)& \cellcolor{red!40}\href{../works/BeckPS03.pdf}{BeckPS03} (0.73)& \cellcolor{red!40}\href{../works/BeniniLMMR08.pdf}{BeniniLMMR08} (0.72)& \cellcolor{red!40}\href{../works/KovacsV04.pdf}{KovacsV04} (0.71)& \cellcolor{red!40}\href{../works/CestaOF99.pdf}{CestaOF99} (0.71)\\
\index{PacinoH11}\href{../works/PacinoH11.pdf}{PacinoH11} R\&C\\
Euclid& \cellcolor{red!40}\href{../works/VilimBC04.pdf}{VilimBC04} (0.23)& \cellcolor{red!40}\href{../works/HentenryckM04.pdf}{HentenryckM04} (0.23)& \cellcolor{red!20}\href{../works/GodardLN05.pdf}{GodardLN05} (0.25)& \cellcolor{yellow!20}\href{../works/Vilim05.pdf}{Vilim05} (0.26)& \cellcolor{yellow!20}\href{../works/VilimBC05.pdf}{VilimBC05} (0.27)\\
Dot& \cellcolor{red!40}\href{../works/Dejemeppe16.pdf}{Dejemeppe16} (119.00)& \cellcolor{red!40}\href{../works/Astrand21.pdf}{Astrand21} (114.00)& \cellcolor{red!40}\href{../works/LaborieRSV18.pdf}{LaborieRSV18} (108.00)& \cellcolor{red!40}\href{../works/Malapert11.pdf}{Malapert11} (108.00)& \cellcolor{red!40}\href{../works/Groleaz21.pdf}{Groleaz21} (108.00)\\
Cosine& \cellcolor{red!40}\href{../works/HentenryckM04.pdf}{HentenryckM04} (0.83)& \cellcolor{red!40}\href{../works/VilimBC04.pdf}{VilimBC04} (0.82)& \cellcolor{red!40}\href{../works/GodardLN05.pdf}{GodardLN05} (0.81)& \cellcolor{red!40}\href{../works/VilimBC05.pdf}{VilimBC05} (0.79)& \cellcolor{red!40}\href{../works/CarchraeB09.pdf}{CarchraeB09} (0.78)\\
\index{PandeyS21a}\href{../works/PandeyS21a.pdf}{PandeyS21a} R\&C& \cellcolor{yellow!20}\href{../works/ArbaouiY18.pdf}{ArbaouiY18} (0.91)& \cellcolor{yellow!20}\href{../works/EdisO11.pdf}{EdisO11} (0.92)& \cellcolor{green!20}EdisO11a (0.94)& \cellcolor{green!20}\href{../works/YunusogluY22.pdf}{YunusogluY22} (0.94)& \cellcolor{green!20}\href{../works/JainG01.pdf}{JainG01} (0.95)\\
Euclid& \cellcolor{black!20}\href{../works/HeipckeCCS00.pdf}{HeipckeCCS00} (0.34)& \cellcolor{black!20}\href{../works/NishikawaSTT19.pdf}{NishikawaSTT19} (0.35)& \cellcolor{black!20}\href{../works/KhayatLR06.pdf}{KhayatLR06} (0.35)& \cellcolor{black!20}\href{../works/EdisO11.pdf}{EdisO11} (0.35)& \cellcolor{black!20}\href{../works/JainG01.pdf}{JainG01} (0.35)\\
Dot& \cellcolor{red!40}\href{../works/ZarandiASC20.pdf}{ZarandiASC20} (141.00)& \cellcolor{red!40}\href{../works/Groleaz21.pdf}{Groleaz21} (140.00)& \cellcolor{red!40}\href{../works/Baptiste02.pdf}{Baptiste02} (134.00)& \cellcolor{red!40}\href{../works/LaborieRSV18.pdf}{LaborieRSV18} (130.00)& \cellcolor{red!40}\href{../works/Godet21a.pdf}{Godet21a} (130.00)\\
Cosine& \cellcolor{red!40}\href{../works/KhayatLR06.pdf}{KhayatLR06} (0.71)& \cellcolor{red!40}\href{../works/JainG01.pdf}{JainG01} (0.70)& \cellcolor{red!40}\href{../works/HeipckeCCS00.pdf}{HeipckeCCS00} (0.70)& \cellcolor{red!40}\href{../works/TouatBT22.pdf}{TouatBT22} (0.70)& \cellcolor{red!40}\href{../works/ArtiguesLH13.pdf}{ArtiguesLH13} (0.70)\\
\index{PapaB98}\href{../works/PapaB98.pdf}{PapaB98} R\&C& \cellcolor{yellow!20}\href{../works/Polo-MejiaALB20.pdf}{Polo-MejiaALB20} (0.91)& \cellcolor{green!20}\href{../works/ArtiouchineB05.pdf}{ArtiouchineB05} (0.94)& \cellcolor{green!20}NaderiR22 (0.95)& \cellcolor{green!20}\href{../works/PoderBS04.pdf}{PoderBS04} (0.95)& \cellcolor{green!20}\href{../works/ElkhyariGJ02a.pdf}{ElkhyariGJ02a} (0.96)\\
Euclid& \cellcolor{blue!20}\href{../works/PapeB97.pdf}{PapeB97} (0.32)& \cellcolor{black!20}\href{../works/BaptisteP95.pdf}{BaptisteP95} (0.34)& \cellcolor{black!20}\href{../works/BaptisteP00.pdf}{BaptisteP00} (0.35)& \cellcolor{black!20}\href{../works/BaptisteP97.pdf}{BaptisteP97} (0.36)& \href{../works/GokgurHO18.pdf}{GokgurHO18} (0.37)\\
Dot& \cellcolor{red!40}\href{../works/Baptiste02.pdf}{Baptiste02} (230.00)& \cellcolor{red!40}\href{../works/Godet21a.pdf}{Godet21a} (188.00)& \cellcolor{red!40}\href{../works/Lombardi10.pdf}{Lombardi10} (188.00)& \cellcolor{red!40}\href{../works/Fahimi16.pdf}{Fahimi16} (182.00)& \cellcolor{red!40}\href{../works/Malapert11.pdf}{Malapert11} (180.00)\\
Cosine& \cellcolor{red!40}\href{../works/PapeB97.pdf}{PapeB97} (0.83)& \cellcolor{red!40}\href{../works/BaptisteP95.pdf}{BaptisteP95} (0.79)& \cellcolor{red!40}\href{../works/BaptisteP00.pdf}{BaptisteP00} (0.78)& \cellcolor{red!40}\href{../works/BaptisteP97.pdf}{BaptisteP97} (0.77)& \cellcolor{red!40}\href{../works/GokgurHO18.pdf}{GokgurHO18} (0.77)\\
\index{Pape94}\href{../works/Pape94.pdf}{Pape94} R\&C& \cellcolor{yellow!20}\href{../works/NuijtenA96.pdf}{NuijtenA96} (0.91)& \cellcolor{green!20}\href{../works/Laborie03.pdf}{Laborie03} (0.93)& \cellcolor{green!20}\href{../works/BeckF00.pdf}{BeckF00} (0.94)& \cellcolor{green!20}\href{../works/AggounB93.pdf}{AggounB93} (0.95)& \cellcolor{green!20}\href{../works/WatsonB08.pdf}{WatsonB08} (0.96)\\
Euclid& \cellcolor{green!20}\href{../works/Colombani96.pdf}{Colombani96} (0.29)& \cellcolor{green!20}\href{../works/HoeveGSL07.pdf}{HoeveGSL07} (0.30)& \cellcolor{green!20}\href{../works/LiessM08.pdf}{LiessM08} (0.30)& \cellcolor{green!20}\href{../works/HeipckeCCS00.pdf}{HeipckeCCS00} (0.31)& \cellcolor{green!20}\href{../works/DavenportKRSH07.pdf}{DavenportKRSH07} (0.31)\\
Dot& \cellcolor{red!40}\href{../works/Beck99.pdf}{Beck99} (128.00)& \cellcolor{red!40}\href{../works/Baptiste02.pdf}{Baptiste02} (127.00)& \cellcolor{red!40}\href{../works/ZarandiASC20.pdf}{ZarandiASC20} (125.00)& \cellcolor{red!40}\href{../works/Lombardi10.pdf}{Lombardi10} (122.00)& \cellcolor{red!40}\href{../works/Astrand21.pdf}{Astrand21} (120.00)\\
Cosine& \cellcolor{red!40}\href{../works/TrentesauxPT01.pdf}{TrentesauxPT01} (0.76)& \cellcolor{red!40}\href{../works/Colombani96.pdf}{Colombani96} (0.75)& \cellcolor{red!40}\href{../works/LiessM08.pdf}{LiessM08} (0.74)& \cellcolor{red!40}\href{../works/HeipckeCCS00.pdf}{HeipckeCCS00} (0.73)& \cellcolor{red!40}\href{../works/DavenportKRSH07.pdf}{DavenportKRSH07} (0.73)\\
\index{PapeB96}PapeB96 R\&C\\
Euclid\\
Dot\\
Cosine\\
\index{PapeB97}\href{../works/PapeB97.pdf}{PapeB97} R\&C\\
Euclid& \cellcolor{yellow!20}\href{../works/BaptisteP00.pdf}{BaptisteP00} (0.27)& \cellcolor{green!20}\href{../works/BaptisteP97.pdf}{BaptisteP97} (0.30)& \cellcolor{blue!20}\href{../works/PapaB98.pdf}{PapaB98} (0.32)& \cellcolor{black!20}\href{../works/BaptistePN99.pdf}{BaptistePN99} (0.35)& \cellcolor{black!20}\href{../works/TanSD10.pdf}{TanSD10} (0.36)\\
Dot& \cellcolor{red!40}\href{../works/Baptiste02.pdf}{Baptiste02} (189.00)& \cellcolor{red!40}\href{../works/Malapert11.pdf}{Malapert11} (177.00)& \cellcolor{red!40}\href{../works/PapaB98.pdf}{PapaB98} (172.00)& \cellcolor{red!40}\href{../works/Schutt11.pdf}{Schutt11} (169.00)& \cellcolor{red!40}\href{../works/Dejemeppe16.pdf}{Dejemeppe16} (169.00)\\
Cosine& \cellcolor{red!40}\href{../works/BaptisteP00.pdf}{BaptisteP00} (0.87)& \cellcolor{red!40}\href{../works/PapaB98.pdf}{PapaB98} (0.83)& \cellcolor{red!40}\href{../works/BaptisteP97.pdf}{BaptisteP97} (0.83)& \cellcolor{red!40}\href{../works/BaptistePN99.pdf}{BaptistePN99} (0.77)& \cellcolor{red!40}\href{../works/GokgurHO18.pdf}{GokgurHO18} (0.75)\\
\index{ParkUJR19}\href{../works/ParkUJR19.pdf}{ParkUJR19} R\&C& \cellcolor{yellow!20}\href{../works/CobanH10.pdf}{CobanH10} (0.92)& \cellcolor{yellow!20}\href{../works/Laborie18a.pdf}{Laborie18a} (0.92)& \cellcolor{yellow!20}\href{../works/ColT2019a.pdf}{ColT2019a} (0.92)& \cellcolor{yellow!20}\href{../works/HamdiL13.pdf}{HamdiL13} (0.93)& \cellcolor{green!20}\href{../works/RenT09.pdf}{RenT09} (0.93)\\
Euclid& \cellcolor{red!20}\href{../works/JuvinHL23.pdf}{JuvinHL23} (0.26)& \cellcolor{green!20}\href{../works/BillautHL12.pdf}{BillautHL12} (0.29)& \cellcolor{green!20}\href{../works/Beck06.pdf}{Beck06} (0.30)& \cellcolor{blue!20}\href{../works/LorigeonBB02.pdf}{LorigeonBB02} (0.32)& \cellcolor{blue!20}\href{../works/HamdiL13.pdf}{HamdiL13} (0.32)\\
Dot& \cellcolor{red!40}\href{../works/Groleaz21.pdf}{Groleaz21} (145.00)& \cellcolor{red!40}\href{../works/ZarandiASC20.pdf}{ZarandiASC20} (143.00)& \cellcolor{red!40}\href{../works/Lunardi20.pdf}{Lunardi20} (141.00)& \cellcolor{red!40}\href{../works/PrataAN23.pdf}{PrataAN23} (128.00)& \cellcolor{red!40}\href{../works/Astrand21.pdf}{Astrand21} (128.00)\\
Cosine& \cellcolor{red!40}\href{../works/JuvinHL23.pdf}{JuvinHL23} (0.80)& \cellcolor{red!40}\href{../works/TerekhovTDB14.pdf}{TerekhovTDB14} (0.77)& \cellcolor{red!40}\href{../works/BillautHL12.pdf}{BillautHL12} (0.75)& \cellcolor{red!40}\href{../works/ZhouGL15.pdf}{ZhouGL15} (0.72)& \cellcolor{red!40}\href{../works/OujanaAYB22.pdf}{OujanaAYB22} (0.72)\\
\index{PembertonG98}\href{../works/PembertonG98.pdf}{PembertonG98} R\&C& \cellcolor{black!20}\href{../works/ZhangW18.pdf}{ZhangW18} (0.99)& \cellcolor{black!20}\href{../works/BartakSR08.pdf}{BartakSR08} (0.99)& \cellcolor{black!20}\href{../works/MintonJPL92.pdf}{MintonJPL92} (1.00)& \cellcolor{black!20}\href{../works/JainM99.pdf}{JainM99} (1.00)\\
Euclid& \cellcolor{yellow!20}\href{../works/KengY89.pdf}{KengY89} (0.28)& \cellcolor{green!20}\href{../works/Caseau97.pdf}{Caseau97} (0.29)& \cellcolor{green!20}\href{../works/CrawfordB94.pdf}{CrawfordB94} (0.30)& \cellcolor{green!20}\href{../works/SimoninAHL12.pdf}{SimoninAHL12} (0.30)& \cellcolor{green!20}\href{../works/Salido10.pdf}{Salido10} (0.30)\\
Dot& \cellcolor{red!40}\href{../works/ZarandiASC20.pdf}{ZarandiASC20} (103.00)& \cellcolor{red!40}\href{../works/Godet21a.pdf}{Godet21a} (101.00)& \cellcolor{red!40}\href{../works/Lombardi10.pdf}{Lombardi10} (101.00)& \cellcolor{red!40}\href{../works/Fahimi16.pdf}{Fahimi16} (98.00)& \cellcolor{red!40}\href{../works/Dejemeppe16.pdf}{Dejemeppe16} (96.00)\\
Cosine& \cellcolor{red!40}\href{../works/KengY89.pdf}{KengY89} (0.70)& \cellcolor{red!40}\href{../works/SimoninAHL12.pdf}{SimoninAHL12} (0.67)& \cellcolor{red!40}\href{../works/Salido10.pdf}{Salido10} (0.67)& \cellcolor{red!40}\href{../works/BaptisteP95.pdf}{BaptisteP95} (0.67)& \cellcolor{red!40}\href{../works/Caseau97.pdf}{Caseau97} (0.66)\\
\index{PengLC14}\href{../works/PengLC14.pdf}{PengLC14} R\&C& \cellcolor{yellow!20}WariZ19 (0.92)& \cellcolor{yellow!20}\href{../works/BeldiceanuCDP11.pdf}{BeldiceanuCDP11} (0.92)& \cellcolor{yellow!20}\href{../works/ZarandiKS16.pdf}{ZarandiKS16} (0.93)& \cellcolor{green!20}\href{../works/NovasH14.pdf}{NovasH14} (0.95)& \cellcolor{green!20}\href{../works/NovaraNH16.pdf}{NovaraNH16} (0.96)\\
Euclid& \cellcolor{yellow!20}\href{../works/KhayatLR06.pdf}{KhayatLR06} (0.26)& \cellcolor{yellow!20}\href{../works/HeipckeCCS00.pdf}{HeipckeCCS00} (0.27)& \cellcolor{yellow!20}\href{../works/KovacsV06.pdf}{KovacsV06} (0.27)& \cellcolor{yellow!20}\href{../works/ValleMGT03.pdf}{ValleMGT03} (0.27)& \cellcolor{yellow!20}\href{../works/KovacsV04.pdf}{KovacsV04} (0.27)\\
Dot& \cellcolor{red!40}\href{../works/Groleaz21.pdf}{Groleaz21} (155.00)& \cellcolor{red!40}\href{../works/Dejemeppe16.pdf}{Dejemeppe16} (152.00)& \cellcolor{red!40}\href{../works/ZarandiASC20.pdf}{ZarandiASC20} (148.00)& \cellcolor{red!40}\href{../works/Baptiste02.pdf}{Baptiste02} (142.00)& \cellcolor{red!40}\href{../works/Beck99.pdf}{Beck99} (134.00)\\
Cosine& \cellcolor{red!40}\href{../works/KhayatLR06.pdf}{KhayatLR06} (0.81)& \cellcolor{red!40}\href{../works/HeipckeCCS00.pdf}{HeipckeCCS00} (0.80)& \cellcolor{red!40}\href{../works/KovacsV06.pdf}{KovacsV06} (0.79)& \cellcolor{red!40}\href{../works/KovacsV04.pdf}{KovacsV04} (0.78)& \cellcolor{red!40}\href{../works/ValleMGT03.pdf}{ValleMGT03} (0.78)\\
\index{PenzDN23}\href{../works/PenzDN23.pdf}{PenzDN23} R\&C& \cellcolor{green!20}\href{../works/BajestaniB15.pdf}{BajestaniB15} (0.96)& \cellcolor{blue!20}\href{../works/NattafDYW19.pdf}{NattafDYW19} (0.96)& \cellcolor{blue!20}\href{../works/Sadykov04.pdf}{Sadykov04} (0.98)& \cellcolor{black!20}\href{../works/Ham18a.pdf}{Ham18a} (0.98)& \cellcolor{black!20}\href{../works/KovacsK11.pdf}{KovacsK11} (0.98)\\
Euclid& \cellcolor{blue!20}\href{../works/BogaerdtW19.pdf}{BogaerdtW19} (0.33)& \cellcolor{blue!20}\href{../works/BajestaniB15.pdf}{BajestaniB15} (0.34)& \cellcolor{black!20}\href{../works/WuBB09.pdf}{WuBB09} (0.36)& \cellcolor{black!20}\href{../works/ThiruvadyBME09.pdf}{ThiruvadyBME09} (0.36)& \cellcolor{black!20}\href{../works/NattafDYW19.pdf}{NattafDYW19} (0.36)\\
Dot& \cellcolor{red!40}\href{../works/ZarandiASC20.pdf}{ZarandiASC20} (165.00)& \cellcolor{red!40}\href{../works/Groleaz21.pdf}{Groleaz21} (142.00)& \cellcolor{red!40}\href{../works/Baptiste02.pdf}{Baptiste02} (128.00)& \cellcolor{red!40}\href{../works/Astrand21.pdf}{Astrand21} (125.00)& \cellcolor{red!40}\href{../works/PrataAN23.pdf}{PrataAN23} (124.00)\\
Cosine& \cellcolor{red!40}\href{../works/BajestaniB15.pdf}{BajestaniB15} (0.75)& \cellcolor{red!40}\href{../works/BogaerdtW19.pdf}{BogaerdtW19} (0.70)& \cellcolor{red!40}\href{../works/KovacsB11.pdf}{KovacsB11} (0.69)& \cellcolor{red!40}\href{../works/NattafDYW19.pdf}{NattafDYW19} (0.68)& \cellcolor{red!40}\href{../works/WuBB09.pdf}{WuBB09} (0.67)\\
\index{PerezGSL23}\href{../works/PerezGSL23.pdf}{PerezGSL23} R\&C\\
Euclid& \cellcolor{red!40}\href{../works/abs-2312-13682.pdf}{abs-2312-13682} (0.05)& \cellcolor{red!40}\href{../works/BockmayrP06.pdf}{BockmayrP06} (0.23)& \cellcolor{yellow!20}\href{../works/ZibranR11.pdf}{ZibranR11} (0.27)& \cellcolor{yellow!20}\href{../works/ZibranR11a.pdf}{ZibranR11a} (0.27)& \cellcolor{yellow!20}\href{../works/Limtanyakul07.pdf}{Limtanyakul07} (0.27)\\
Dot& \cellcolor{red!40}\href{../works/Dejemeppe16.pdf}{Dejemeppe16} (93.00)& \cellcolor{red!40}\href{../works/Astrand21.pdf}{Astrand21} (91.00)& \cellcolor{red!40}\href{../works/Lombardi10.pdf}{Lombardi10} (89.00)& \cellcolor{red!40}\href{../works/ZarandiASC20.pdf}{ZarandiASC20} (86.00)& \cellcolor{red!40}\href{../works/LaborieRSV18.pdf}{LaborieRSV18} (86.00)\\
Cosine& \cellcolor{red!40}\href{../works/abs-2312-13682.pdf}{abs-2312-13682} (0.99)& \cellcolor{red!40}\href{../works/BockmayrP06.pdf}{BockmayrP06} (0.76)& \cellcolor{red!40}\href{../works/AalianPG23.pdf}{AalianPG23} (0.74)& \cellcolor{red!40}\href{../works/CarchraeB09.pdf}{CarchraeB09} (0.71)& \cellcolor{red!40}\href{../works/QuirogaZH05.pdf}{QuirogaZH05} (0.70)\\
\index{Perron05}\href{../works/Perron05.pdf}{Perron05} R\&C& \cellcolor{red!40}\href{../works/RasmussenT06.pdf}{RasmussenT06} (0.73)& \cellcolor{red!40}\href{../works/HenzMT04.pdf}{HenzMT04} (0.80)& \cellcolor{red!40}\href{../works/RasmussenT09.pdf}{RasmussenT09} (0.83)& \cellcolor{red!40}\href{../works/RasmussenT07.pdf}{RasmussenT07} (0.84)& \cellcolor{red!40}\href{../works/Trick03.pdf}{Trick03} (0.85)\\
Euclid& \cellcolor{red!40}\href{../works/Baptiste09.pdf}{Baptiste09} (0.19)& \cellcolor{red!40}\href{../works/CarchraeBF05.pdf}{CarchraeBF05} (0.19)& \cellcolor{red!40}\href{../works/SuCC13.pdf}{SuCC13} (0.19)& \cellcolor{red!40}\href{../works/ElfJR03.pdf}{ElfJR03} (0.20)& \cellcolor{red!40}\href{../works/Tsang03.pdf}{Tsang03} (0.21)\\
Dot& \cellcolor{red!40}\href{../works/CarlssonJL17.pdf}{CarlssonJL17} (43.00)& \cellcolor{red!40}\href{../works/LarsonJC14.pdf}{LarsonJC14} (39.00)& \cellcolor{red!40}\href{../works/RussellU06.pdf}{RussellU06} (38.00)& \cellcolor{red!40}\href{../works/KendallKRU10.pdf}{KendallKRU10} (38.00)& \cellcolor{red!40}\href{../works/RasmussenT09.pdf}{RasmussenT09} (36.00)\\
Cosine& \cellcolor{red!40}\href{../works/SuCC13.pdf}{SuCC13} (0.74)& \cellcolor{red!40}\href{../works/RasmussenT06.pdf}{RasmussenT06} (0.69)& \cellcolor{red!40}\href{../works/LarsonJC14.pdf}{LarsonJC14} (0.69)& \cellcolor{red!40}\href{../works/EastonNT02.pdf}{EastonNT02} (0.67)& \cellcolor{red!40}\href{../works/Trick03.pdf}{Trick03} (0.66)\\
\index{PerronSF04}\href{../works/PerronSF04.pdf}{PerronSF04} R\&C& \cellcolor{red!40}\href{../works/DannaP03.pdf}{DannaP03} (0.67)& \cellcolor{red!40}\href{../works/CarchraeB09.pdf}{CarchraeB09} (0.79)& \cellcolor{red!40}\href{../works/SchausHMCMD11.pdf}{SchausHMCMD11} (0.82)& \cellcolor{red!20}\href{../works/Wolf03.pdf}{Wolf03} (0.87)& \cellcolor{red!20}\href{../works/PesantRR15.pdf}{PesantRR15} (0.87)\\
Euclid& \cellcolor{red!40}\href{../works/DannaP03.pdf}{DannaP03} (0.21)& \cellcolor{red!40}\href{../works/HebrardTW05.pdf}{HebrardTW05} (0.24)& \cellcolor{red!20}\href{../works/Hooker17.pdf}{Hooker17} (0.25)& \cellcolor{yellow!20}\href{../works/KovacsEKV05.pdf}{KovacsEKV05} (0.27)& \cellcolor{yellow!20}\href{../works/Puget95.pdf}{Puget95} (0.27)\\
Dot& \cellcolor{red!40}\href{../works/Groleaz21.pdf}{Groleaz21} (86.00)& \cellcolor{red!40}\href{../works/LaborieRSV18.pdf}{LaborieRSV18} (82.00)& \cellcolor{red!40}\href{../works/Dejemeppe16.pdf}{Dejemeppe16} (82.00)& \cellcolor{red!40}\href{../works/KelbelH11.pdf}{KelbelH11} (78.00)& \cellcolor{red!40}\href{../works/DannaP03.pdf}{DannaP03} (74.00)\\
Cosine& \cellcolor{red!40}\href{../works/DannaP03.pdf}{DannaP03} (0.85)& \cellcolor{red!40}\href{../works/HebrardTW05.pdf}{HebrardTW05} (0.73)& \cellcolor{red!40}\href{../works/CarchraeB09.pdf}{CarchraeB09} (0.70)& \cellcolor{red!40}\href{../works/MonetteDH09.pdf}{MonetteDH09} (0.68)& \cellcolor{red!40}\href{../works/KimCMLLP23.pdf}{KimCMLLP23} (0.68)\\
\index{PesantGPR99}\href{../works/PesantGPR99.pdf}{PesantGPR99} R\&C& \cellcolor{yellow!20}\href{../works/MercierH07.pdf}{MercierH07} (0.92)& \cellcolor{green!20}DorndorfHP99 (0.95)& \cellcolor{green!20}\href{../works/Puget95.pdf}{Puget95} (0.95)& \cellcolor{green!20}\href{../works/Geske05.pdf}{Geske05} (0.95)& \cellcolor{green!20}CastroGR10 (0.95)\\
Euclid& \cellcolor{red!40}\href{../works/GomesHS06.pdf}{GomesHS06} (0.22)& \cellcolor{red!40}\href{../works/ZibranR11.pdf}{ZibranR11} (0.23)& \cellcolor{red!40}\href{../works/ZibranR11a.pdf}{ZibranR11a} (0.23)& \cellcolor{red!40}\href{../works/Touraivane95.pdf}{Touraivane95} (0.23)& \cellcolor{red!40}\href{../works/BofillGSV15.pdf}{BofillGSV15} (0.23)\\
Dot& \cellcolor{red!40}\href{../works/ZarandiASC20.pdf}{ZarandiASC20} (71.00)& \cellcolor{red!40}\href{../works/Astrand21.pdf}{Astrand21} (70.00)& \cellcolor{red!40}\href{../works/Baptiste02.pdf}{Baptiste02} (69.00)& \cellcolor{red!40}\href{../works/Beck99.pdf}{Beck99} (68.00)& \cellcolor{red!40}\href{../works/Malapert11.pdf}{Malapert11} (66.00)\\
Cosine& \cellcolor{red!40}\href{../works/GomesHS06.pdf}{GomesHS06} (0.70)& \cellcolor{red!40}\href{../works/BourdaisGP03.pdf}{BourdaisGP03} (0.69)& \cellcolor{red!40}\href{../works/ZibranR11a.pdf}{ZibranR11a} (0.68)& \cellcolor{red!40}\href{../works/MartinPY01.pdf}{MartinPY01} (0.68)& \cellcolor{red!40}\href{../works/ShaikhK23.pdf}{ShaikhK23} (0.68)\\
\index{PesantRR15}\href{../works/PesantRR15.pdf}{PesantRR15} R\&C& \cellcolor{red!40}\href{../works/BofillEGPSV14.pdf}{BofillEGPSV14} (0.75)& \cellcolor{red!40}\href{../works/BofillGSV15.pdf}{BofillGSV15} (0.81)& \cellcolor{red!20}\href{../works/PerronSF04.pdf}{PerronSF04} (0.87)& \cellcolor{red!20}\href{../works/DannaP03.pdf}{DannaP03} (0.90)& \cellcolor{red!20}\href{../works/BessiereHMQW14.pdf}{BessiereHMQW14} (0.90)\\
Euclid& \cellcolor{red!40}\href{../works/BofillCGGPSV23.pdf}{BofillCGGPSV23} (0.19)& \cellcolor{red!40}\href{../works/ZibranR11.pdf}{ZibranR11} (0.22)& \cellcolor{red!40}\href{../works/Baptiste09.pdf}{Baptiste09} (0.23)& \cellcolor{red!40}\href{../works/CarchraeBF05.pdf}{CarchraeBF05} (0.23)& \cellcolor{red!40}\href{../works/BofillGSV15.pdf}{BofillGSV15} (0.23)\\
Dot& \cellcolor{red!40}\href{../works/Siala15a.pdf}{Siala15a} (51.00)& \cellcolor{red!40}\href{../works/Dejemeppe16.pdf}{Dejemeppe16} (46.00)& \cellcolor{red!40}\href{../works/MusliuSS18.pdf}{MusliuSS18} (44.00)& \cellcolor{red!40}\href{../works/Lemos21.pdf}{Lemos21} (43.00)& \cellcolor{red!40}\href{../works/HookerH17.pdf}{HookerH17} (42.00)\\
Cosine& \cellcolor{red!40}\href{../works/BofillCGGPSV23.pdf}{BofillCGGPSV23} (0.72)& \cellcolor{red!40}\href{../works/ZibranR11.pdf}{ZibranR11} (0.62)& \cellcolor{red!40}\href{../works/GelainPRVW17.pdf}{GelainPRVW17} (0.62)& \cellcolor{red!40}\href{../works/LimHTB16.pdf}{LimHTB16} (0.61)& \cellcolor{red!40}\href{../works/BofillGSV15.pdf}{BofillGSV15} (0.60)\\
\index{PeschT96}PeschT96 R\&C& \cellcolor{green!20}DorndorfHP99 (0.95)& \cellcolor{green!20}\href{../works/NuijtenA96.pdf}{NuijtenA96} (0.95)& \cellcolor{green!20}\href{../works/Dorndorf2000.pdf}{Dorndorf2000} (0.96)& \cellcolor{green!20}\href{../works/KovacsEKV05.pdf}{KovacsEKV05} (0.96)& \cellcolor{blue!20}\href{../works/ColT2019a.pdf}{ColT2019a} (0.97)\\
Euclid\\
Dot\\
Cosine\\
\index{Pinarbasi21}Pinarbasi21 R\&C& \cellcolor{red!40}\href{../works/PinarbasiAY19.pdf}{PinarbasiAY19} (0.65)& \cellcolor{red!40}\href{../works/Alaka21.pdf}{Alaka21} (0.75)& \cellcolor{red!40}KizilayC20 (0.79)& \cellcolor{red!40}PinarbasiA20 (0.80)& \cellcolor{red!40}\href{../works/SacramentoSP20.pdf}{SacramentoSP20} (0.80)\\
Euclid\\
Dot\\
Cosine\\
\index{PinarbasiA20}PinarbasiA20 R\&C& \cellcolor{red!40}\href{../works/AlakaPY19.pdf}{AlakaPY19} (0.71)& \cellcolor{red!40}\href{../works/Alaka21.pdf}{Alaka21} (0.74)& \cellcolor{red!40}\href{../works/PinarbasiAY19.pdf}{PinarbasiAY19} (0.78)& \cellcolor{red!40}Pinarbasi21 (0.80)& \cellcolor{red!40}KizilayC20 (0.83)\\
Euclid\\
Dot\\
Cosine\\
\index{PinarbasiAY19}\href{../works/PinarbasiAY19.pdf}{PinarbasiAY19} R\&C& \cellcolor{red!40}\href{../works/AlakaPY19.pdf}{AlakaPY19} (0.62)& \cellcolor{red!40}Pinarbasi21 (0.65)& \cellcolor{red!40}PinarbasiA20 (0.78)& \cellcolor{red!40}\href{../works/TopalogluSS12.pdf}{TopalogluSS12} (0.82)& \cellcolor{red!40}\href{../works/ValleMGT03.pdf}{ValleMGT03} (0.83)\\
Euclid& \cellcolor{red!40}\href{../works/Alaka21.pdf}{Alaka21} (0.23)& \cellcolor{red!40}\href{../works/AlakaP23.pdf}{AlakaP23} (0.23)& \cellcolor{red!40}\href{../works/AlakaPY19.pdf}{AlakaPY19} (0.23)& \cellcolor{yellow!20}\href{../works/ValleMGT03.pdf}{ValleMGT03} (0.28)& \cellcolor{yellow!20}\href{../works/VanczaM01.pdf}{VanczaM01} (0.28)\\
Dot& \cellcolor{red!40}\href{../works/ZarandiASC20.pdf}{ZarandiASC20} (138.00)& \cellcolor{red!40}\href{../works/Groleaz21.pdf}{Groleaz21} (138.00)& \cellcolor{red!40}\href{../works/Astrand21.pdf}{Astrand21} (136.00)& \cellcolor{red!40}\href{../works/Dejemeppe16.pdf}{Dejemeppe16} (133.00)& \cellcolor{red!40}\href{../works/Malapert11.pdf}{Malapert11} (126.00)\\
Cosine& \cellcolor{red!40}\href{../works/AlakaP23.pdf}{AlakaP23} (0.85)& \cellcolor{red!40}\href{../works/Alaka21.pdf}{Alaka21} (0.85)& \cellcolor{red!40}\href{../works/AlakaPY19.pdf}{AlakaPY19} (0.84)& \cellcolor{red!40}\href{../works/CilKLO22.pdf}{CilKLO22} (0.79)& \cellcolor{red!40}\href{../works/PengLC14.pdf}{PengLC14} (0.77)\\
\index{PintoG97}\href{../works/PintoG97.pdf}{PintoG97} R\&C& \cellcolor{red!20}\href{../works/MaraveliasCG04.pdf}{MaraveliasCG04} (0.89)& \cellcolor{yellow!20}\href{../works/RoePS05.pdf}{RoePS05} (0.91)& \cellcolor{yellow!20}\href{../works/HarjunkoskiJG00.pdf}{HarjunkoskiJG00} (0.92)& \cellcolor{green!20}\href{../works/HarjunkoskiG02.pdf}{HarjunkoskiG02} (0.94)& \cellcolor{green!20}\href{../works/HookerO99.pdf}{HookerO99} (0.95)\\
Euclid& \cellcolor{yellow!20}\href{../works/Rit86.pdf}{Rit86} (0.28)& \cellcolor{green!20}\href{../works/SchuttWS05.pdf}{SchuttWS05} (0.30)& \cellcolor{green!20}\href{../works/BeldiceanuP07.pdf}{BeldiceanuP07} (0.30)& \cellcolor{green!20}\href{../works/BockmayrP06.pdf}{BockmayrP06} (0.30)& \cellcolor{green!20}\href{../works/AngelsmarkJ00.pdf}{AngelsmarkJ00} (0.31)\\
Dot& \cellcolor{red!40}\href{../works/ZarandiASC20.pdf}{ZarandiASC20} (89.00)& \cellcolor{red!40}\href{../works/Dejemeppe16.pdf}{Dejemeppe16} (89.00)& \cellcolor{red!40}\href{../works/Groleaz21.pdf}{Groleaz21} (89.00)& \cellcolor{red!40}\href{../works/Baptiste02.pdf}{Baptiste02} (89.00)& \cellcolor{red!40}\href{../works/LaborieRSV18.pdf}{LaborieRSV18} (88.00)\\
Cosine& \cellcolor{red!40}\href{../works/Rit86.pdf}{Rit86} (0.66)& \cellcolor{red!40}\href{../works/SchuttWS05.pdf}{SchuttWS05} (0.66)& \cellcolor{red!40}\href{../works/ZeballosNH11.pdf}{ZeballosNH11} (0.64)& \cellcolor{red!40}\href{../works/NovaraNH16.pdf}{NovaraNH16} (0.64)& \cellcolor{red!40}\href{../works/MaraveliasCG04.pdf}{MaraveliasCG04} (0.63)\\
\index{PoderB08}\href{../works/PoderB08.pdf}{PoderB08} R\&C\\
Euclid& \cellcolor{red!40}\href{../works/BeldiceanuP07.pdf}{BeldiceanuP07} (0.13)& \cellcolor{red!40}\href{../works/WolfS05.pdf}{WolfS05} (0.15)& \cellcolor{red!40}\href{../works/PoderBS04.pdf}{PoderBS04} (0.19)& \cellcolor{red!40}\href{../works/Vilim09a.pdf}{Vilim09a} (0.19)& \cellcolor{red!40}\href{../works/SimonisH11.pdf}{SimonisH11} (0.20)\\
Dot& \cellcolor{red!40}\href{../works/Dejemeppe16.pdf}{Dejemeppe16} (71.00)& \cellcolor{red!40}\href{../works/Beck99.pdf}{Beck99} (71.00)& \cellcolor{red!40}\href{../works/Godet21a.pdf}{Godet21a} (70.00)& \cellcolor{red!40}\href{../works/Lombardi10.pdf}{Lombardi10} (70.00)& \cellcolor{red!40}\href{../works/Malapert11.pdf}{Malapert11} (70.00)\\
Cosine& \cellcolor{red!40}\href{../works/BeldiceanuP07.pdf}{BeldiceanuP07} (0.91)& \cellcolor{red!40}\href{../works/WolfS05.pdf}{WolfS05} (0.89)& \cellcolor{red!40}\href{../works/PoderBS04.pdf}{PoderBS04} (0.85)& \cellcolor{red!40}\href{../works/Vilim09a.pdf}{Vilim09a} (0.81)& \cellcolor{red!40}\href{../works/SimonisH11.pdf}{SimonisH11} (0.80)\\
\index{PoderBS04}\href{../works/PoderBS04.pdf}{PoderBS04} R\&C& \cellcolor{red!40}\href{../works/Simonis95.pdf}{Simonis95} (0.82)& \cellcolor{red!20}\href{../works/ElkhyariGJ02a.pdf}{ElkhyariGJ02a} (0.87)& \cellcolor{red!20}\href{../works/BeldiceanuCDP11.pdf}{BeldiceanuCDP11} (0.88)& \cellcolor{red!20}\href{../works/SimonisH11.pdf}{SimonisH11} (0.88)& \cellcolor{red!20}\href{../works/BeldiceanuCP08.pdf}{BeldiceanuCP08} (0.88)\\
Euclid& \cellcolor{red!40}\href{../works/PoderB08.pdf}{PoderB08} (0.19)& \cellcolor{red!40}\href{../works/WolfS05.pdf}{WolfS05} (0.21)& \cellcolor{red!40}\href{../works/BockmayrP06.pdf}{BockmayrP06} (0.22)& \cellcolor{red!40}\href{../works/BeldiceanuP07.pdf}{BeldiceanuP07} (0.23)& \cellcolor{red!40}\href{../works/ChuGNSW13.pdf}{ChuGNSW13} (0.23)\\
Dot& \cellcolor{red!40}\href{../works/Beck99.pdf}{Beck99} (98.00)& \cellcolor{red!40}\href{../works/Baptiste02.pdf}{Baptiste02} (93.00)& \cellcolor{red!40}\href{../works/Godet21a.pdf}{Godet21a} (90.00)& \cellcolor{red!40}\href{../works/Dejemeppe16.pdf}{Dejemeppe16} (89.00)& \cellcolor{red!40}\href{../works/Lombardi10.pdf}{Lombardi10} (89.00)\\
Cosine& \cellcolor{red!40}\href{../works/PoderB08.pdf}{PoderB08} (0.85)& \cellcolor{red!40}\href{../works/WolfS05.pdf}{WolfS05} (0.81)& \cellcolor{red!40}\href{../works/BockmayrP06.pdf}{BockmayrP06} (0.79)& \cellcolor{red!40}\href{../works/ChuGNSW13.pdf}{ChuGNSW13} (0.78)& \cellcolor{red!40}\href{../works/BeldiceanuP07.pdf}{BeldiceanuP07} (0.77)\\
\index{PohlAK22}\href{../works/PohlAK22.pdf}{PohlAK22} R\&C& \cellcolor{red!40}\href{../works/NaderiBZ22a.pdf}{NaderiBZ22a} (0.86)& \cellcolor{red!20}\href{../works/CilKLO22.pdf}{CilKLO22} (0.89)& \cellcolor{red!20}\href{../works/ZhangYW21.pdf}{ZhangYW21} (0.90)& \cellcolor{green!20}\href{../works/GrimesH15.pdf}{GrimesH15} (0.94)& \cellcolor{blue!20}\href{../works/MalapertCGJLR12.pdf}{MalapertCGJLR12} (0.96)\\
Euclid& \cellcolor{black!20}\href{../works/GedikKBR17.pdf}{GedikKBR17} (0.37)& \href{../works/Gronkvist06.pdf}{Gronkvist06} (0.38)& \href{../works/KimCMLLP23.pdf}{KimCMLLP23} (0.38)& \href{../works/MelgarejoLS15.pdf}{MelgarejoLS15} (0.39)& \href{../works/Balduccini11.pdf}{Balduccini11} (0.39)\\
Dot& \cellcolor{red!40}\href{../works/Groleaz21.pdf}{Groleaz21} (147.00)& \cellcolor{red!40}\href{../works/ZarandiASC20.pdf}{ZarandiASC20} (136.00)& \cellcolor{red!40}\href{../works/LaborieRSV18.pdf}{LaborieRSV18} (125.00)& \cellcolor{red!40}\href{../works/Dejemeppe16.pdf}{Dejemeppe16} (122.00)& \cellcolor{red!40}\href{../works/Baptiste02.pdf}{Baptiste02} (120.00)\\
Cosine& \cellcolor{red!40}\href{../works/GokPTGO23.pdf}{GokPTGO23} (0.66)& \cellcolor{red!40}\href{../works/LaborieR14.pdf}{LaborieR14} (0.64)& \cellcolor{red!40}\href{../works/GedikKBR17.pdf}{GedikKBR17} (0.64)& \cellcolor{red!40}\href{../works/KimCMLLP23.pdf}{KimCMLLP23} (0.64)& \cellcolor{red!40}\href{../works/Gronkvist06.pdf}{Gronkvist06} (0.62)\\
\index{PolicellaWSO05}\href{../works/PolicellaWSO05.pdf}{PolicellaWSO05} R\&C\\
Euclid& \cellcolor{yellow!20}\href{../works/LombardiM13.pdf}{LombardiM13} (0.27)& \cellcolor{yellow!20}\href{../works/KrogtLPHJ07.pdf}{KrogtLPHJ07} (0.28)& \cellcolor{yellow!20}\href{../works/NishikawaSTT18.pdf}{NishikawaSTT18} (0.28)& \cellcolor{yellow!20}\href{../works/BofillCSV17a.pdf}{BofillCSV17a} (0.28)& \cellcolor{green!20}\href{../works/NishikawaSTT19.pdf}{NishikawaSTT19} (0.29)\\
Dot& \cellcolor{red!40}\href{../works/Beck99.pdf}{Beck99} (96.00)& \cellcolor{red!40}\href{../works/ZarandiASC20.pdf}{ZarandiASC20} (91.00)& \cellcolor{red!40}\href{../works/Dejemeppe16.pdf}{Dejemeppe16} (90.00)& \cellcolor{red!40}\href{../works/Lombardi10.pdf}{Lombardi10} (88.00)& \cellcolor{red!40}\href{../works/BartakSR10.pdf}{BartakSR10} (87.00)\\
Cosine& \cellcolor{red!40}\href{../works/NishikawaSTT19.pdf}{NishikawaSTT19} (0.71)& \cellcolor{red!40}\href{../works/NishikawaSTT18.pdf}{NishikawaSTT18} (0.70)& \cellcolor{red!40}\href{../works/BofillCSV17a.pdf}{BofillCSV17a} (0.69)& \cellcolor{red!40}\href{../works/NishikawaSTT18a.pdf}{NishikawaSTT18a} (0.69)& \cellcolor{red!40}\href{../works/KrogtLPHJ07.pdf}{KrogtLPHJ07} (0.68)\\
\index{Polo-MejiaALB20}\href{../works/Polo-MejiaALB20.pdf}{Polo-MejiaALB20} R\&C& \cellcolor{red!40}\href{../works/YoungFS17.pdf}{YoungFS17} (0.86)& \cellcolor{red!20}\href{../works/BenderWS21.pdf}{BenderWS21} (0.89)& \cellcolor{yellow!20}\href{../works/PapaB98.pdf}{PapaB98} (0.91)& \cellcolor{yellow!20}NaderiR22 (0.92)& \cellcolor{yellow!20}\href{../works/GilesH16.pdf}{GilesH16} (0.93)\\
Euclid& \cellcolor{blue!20}\href{../works/LiessM08.pdf}{LiessM08} (0.33)& \cellcolor{blue!20}\href{../works/HanenKP21.pdf}{HanenKP21} (0.33)& \cellcolor{blue!20}\href{../works/BaptisteP97.pdf}{BaptisteP97} (0.33)& \cellcolor{black!20}\href{../works/ArkhipovBL19.pdf}{ArkhipovBL19} (0.35)& \cellcolor{black!20}\href{../works/BruckerK00.pdf}{BruckerK00} (0.35)\\
Dot& \cellcolor{red!40}\href{../works/Groleaz21.pdf}{Groleaz21} (183.00)& \cellcolor{red!40}\href{../works/Baptiste02.pdf}{Baptiste02} (173.00)& \cellcolor{red!40}\href{../works/Lombardi10.pdf}{Lombardi10} (167.00)& \cellcolor{red!40}\href{../works/Godet21a.pdf}{Godet21a} (163.00)& \cellcolor{red!40}\href{../works/Dejemeppe16.pdf}{Dejemeppe16} (163.00)\\
Cosine& \cellcolor{red!40}\href{../works/BaptisteP97.pdf}{BaptisteP97} (0.77)& \cellcolor{red!40}\href{../works/HanenKP21.pdf}{HanenKP21} (0.76)& \cellcolor{red!40}\href{../works/LiessM08.pdf}{LiessM08} (0.76)& \cellcolor{red!40}\href{../works/BaptistePN99.pdf}{BaptistePN99} (0.75)& \cellcolor{red!40}\href{../works/ArkhipovBL19.pdf}{ArkhipovBL19} (0.75)\\
\index{PopovicCGNC22}\href{../works/PopovicCGNC22.pdf}{PopovicCGNC22} R\&C\\
Euclid& \cellcolor{green!20}\href{../works/BoothNB16.pdf}{BoothNB16} (0.31)& \cellcolor{green!20}\href{../works/CohenHB17.pdf}{CohenHB17} (0.31)& \cellcolor{blue!20}\href{../works/ZibranR11.pdf}{ZibranR11} (0.32)& \cellcolor{blue!20}\href{../works/BeniniBGM05a.pdf}{BeniniBGM05a} (0.32)& \cellcolor{blue!20}\href{../works/ZibranR11a.pdf}{ZibranR11a} (0.32)\\
Dot& \cellcolor{red!40}\href{../works/LaborieRSV18.pdf}{LaborieRSV18} (86.00)& \cellcolor{red!40}\href{../works/Froger16.pdf}{Froger16} (85.00)& \cellcolor{red!40}\href{../works/Malapert11.pdf}{Malapert11} (84.00)& \cellcolor{red!40}\href{../works/Groleaz21.pdf}{Groleaz21} (83.00)& \cellcolor{red!40}\href{../works/Lunardi20.pdf}{Lunardi20} (80.00)\\
Cosine& \cellcolor{red!40}\href{../works/GoelSHFS15.pdf}{GoelSHFS15} (0.65)& \cellcolor{red!40}\href{../works/BoothNB16.pdf}{BoothNB16} (0.64)& \cellcolor{red!40}\href{../works/AalianPG23.pdf}{AalianPG23} (0.61)& \cellcolor{red!40}\href{../works/QinDS16.pdf}{QinDS16} (0.61)& \cellcolor{red!40}\href{../works/KlankeBYE21.pdf}{KlankeBYE21} (0.61)\\
\index{PourDERB18}\href{../works/PourDERB18.pdf}{PourDERB18} R\&C& \cellcolor{yellow!20}\href{../works/KreterSSZ18.pdf}{KreterSSZ18} (0.91)& \cellcolor{yellow!20}\href{../works/Geske05.pdf}{Geske05} (0.91)& \cellcolor{yellow!20}AggounV04 (0.92)& \cellcolor{yellow!20}\href{../works/GedikKEK18.pdf}{GedikKEK18} (0.93)& \cellcolor{green!20}\href{../works/BukchinR18.pdf}{BukchinR18} (0.93)\\
Euclid& \cellcolor{blue!20}\href{../works/AronssonBK09.pdf}{AronssonBK09} (0.34)& \cellcolor{black!20}\href{../works/Salido10.pdf}{Salido10} (0.34)& \cellcolor{black!20}\href{../works/ZhangLS12.pdf}{ZhangLS12} (0.35)& \cellcolor{black!20}\href{../works/BandaSC11.pdf}{BandaSC11} (0.35)& \cellcolor{black!20}\href{../works/KletzanderMH21.pdf}{KletzanderMH21} (0.35)\\
Dot& \cellcolor{red!40}\href{../works/Lemos21.pdf}{Lemos21} (123.00)& \cellcolor{red!40}\href{../works/ZarandiASC20.pdf}{ZarandiASC20} (113.00)& \cellcolor{red!40}\href{../works/Dejemeppe16.pdf}{Dejemeppe16} (106.00)& \cellcolor{red!40}\href{../works/Godet21a.pdf}{Godet21a} (99.00)& \cellcolor{red!40}\href{../works/Groleaz21.pdf}{Groleaz21} (99.00)\\
Cosine& \cellcolor{red!40}\href{../works/PinarbasiAY19.pdf}{PinarbasiAY19} (0.65)& \cellcolor{red!40}\href{../works/TangLWSK18.pdf}{TangLWSK18} (0.64)& \cellcolor{red!40}\href{../works/PengLC14.pdf}{PengLC14} (0.64)& \cellcolor{red!40}\href{../works/Lemos21.pdf}{Lemos21} (0.64)& \cellcolor{red!40}\href{../works/KovacsK11.pdf}{KovacsK11} (0.63)\\
\index{PovedaAA23}\href{../works/PovedaAA23.pdf}{PovedaAA23} R\&C\\
Euclid& \cellcolor{green!20}\href{../works/YoungFS17.pdf}{YoungFS17} (0.29)& \cellcolor{black!20}\href{../works/SzerediS16.pdf}{SzerediS16} (0.34)& \cellcolor{black!20}\href{../works/BoudreaultSLQ22.pdf}{BoudreaultSLQ22} (0.36)& \cellcolor{black!20}\href{../works/LiessM08.pdf}{LiessM08} (0.37)& \cellcolor{black!20}\href{../works/BaptisteP97.pdf}{BaptisteP97} (0.37)\\
Dot& \cellcolor{red!40}\href{../works/Godet21a.pdf}{Godet21a} (171.00)& \cellcolor{red!40}\href{../works/Lombardi10.pdf}{Lombardi10} (168.00)& \cellcolor{red!40}\href{../works/Schutt11.pdf}{Schutt11} (160.00)& \cellcolor{red!40}\href{../works/Groleaz21.pdf}{Groleaz21} (159.00)& \cellcolor{red!40}\href{../works/Dejemeppe16.pdf}{Dejemeppe16} (157.00)\\
Cosine& \cellcolor{red!40}\href{../works/YoungFS17.pdf}{YoungFS17} (0.84)& \cellcolor{red!40}\href{../works/SzerediS16.pdf}{SzerediS16} (0.77)& \cellcolor{red!40}\href{../works/BoudreaultSLQ22.pdf}{BoudreaultSLQ22} (0.77)& \cellcolor{red!40}\href{../works/abs-1911-04766.pdf}{abs-1911-04766} (0.75)& \cellcolor{red!40}\href{../works/BaptisteP97.pdf}{BaptisteP97} (0.74)\\
\index{Pralet17}\href{../works/Pralet17.pdf}{Pralet17} R\&C& \cellcolor{red!20}DorndorfHP99 (0.90)& \cellcolor{yellow!20}\href{../works/CauwelaertDS20.pdf}{CauwelaertDS20} (0.90)& \cellcolor{yellow!20}\href{../works/DejemeppeCS15.pdf}{DejemeppeCS15} (0.90)& \cellcolor{yellow!20}\href{../works/CauwelaertDMS16.pdf}{CauwelaertDMS16} (0.90)& \cellcolor{yellow!20}\href{../works/GrimesH10.pdf}{GrimesH10} (0.92)\\
Euclid& \cellcolor{yellow!20}\href{../works/GodardLN05.pdf}{GodardLN05} (0.28)& \cellcolor{green!20}\href{../works/HentenryckM04.pdf}{HentenryckM04} (0.29)& \cellcolor{green!20}\href{../works/DemasseyAM05.pdf}{DemasseyAM05} (0.30)& \cellcolor{green!20}\href{../works/Laborie05.pdf}{Laborie05} (0.30)& \cellcolor{green!20}\href{../works/VilimLS15.pdf}{VilimLS15} (0.30)\\
Dot& \cellcolor{red!40}\href{../works/Groleaz21.pdf}{Groleaz21} (158.00)& \cellcolor{red!40}\href{../works/Baptiste02.pdf}{Baptiste02} (155.00)& \cellcolor{red!40}\href{../works/Dejemeppe16.pdf}{Dejemeppe16} (153.00)& \cellcolor{red!40}\href{../works/LaborieRSV18.pdf}{LaborieRSV18} (151.00)& \cellcolor{red!40}\href{../works/ZarandiASC20.pdf}{ZarandiASC20} (150.00)\\
Cosine& \cellcolor{red!40}\href{../works/GodardLN05.pdf}{GodardLN05} (0.81)& \cellcolor{red!40}\href{../works/VilimLS15.pdf}{VilimLS15} (0.80)& \cellcolor{red!40}\href{../works/DemasseyAM05.pdf}{DemasseyAM05} (0.80)& \cellcolor{red!40}\href{../works/HentenryckM04.pdf}{HentenryckM04} (0.79)& \cellcolor{red!40}\href{../works/Laborie05.pdf}{Laborie05} (0.79)\\
\index{PraletLJ15}\href{../works/PraletLJ15.pdf}{PraletLJ15} R\&C& \cellcolor{red!20}CestaOPS14 (0.88)& \cellcolor{red!20}\href{../works/LetortCB13.pdf}{LetortCB13} (0.89)& \cellcolor{red!20}\href{../works/GaySS14.pdf}{GaySS14} (0.89)& \cellcolor{red!20}\href{../works/Davenport10.pdf}{Davenport10} (0.90)& \cellcolor{yellow!20}\href{../works/LetortCB15.pdf}{LetortCB15} (0.91)\\
Euclid& \cellcolor{yellow!20}\href{../works/NishikawaSTT18.pdf}{NishikawaSTT18} (0.27)& \cellcolor{yellow!20}\href{../works/NishikawaSTT18a.pdf}{NishikawaSTT18a} (0.27)& \cellcolor{yellow!20}\href{../works/FrankDT16.pdf}{FrankDT16} (0.28)& \cellcolor{green!20}\href{../works/BeniniBGM05a.pdf}{BeniniBGM05a} (0.29)& \cellcolor{green!20}\href{../works/AlakaPY19.pdf}{AlakaPY19} (0.29)\\
Dot& \cellcolor{red!40}\href{../works/LaborieRSV18.pdf}{LaborieRSV18} (115.00)& \cellcolor{red!40}\href{../works/Groleaz21.pdf}{Groleaz21} (108.00)& \cellcolor{red!40}\href{../works/Lunardi20.pdf}{Lunardi20} (104.00)& \cellcolor{red!40}\href{../works/Dejemeppe16.pdf}{Dejemeppe16} (98.00)& \cellcolor{red!40}\href{../works/ColT22.pdf}{ColT22} (98.00)\\
Cosine& \cellcolor{red!40}\href{../works/NishikawaSTT18.pdf}{NishikawaSTT18} (0.73)& \cellcolor{red!40}\href{../works/NishikawaSTT18a.pdf}{NishikawaSTT18a} (0.73)& \cellcolor{red!40}\href{../works/CampeauG22.pdf}{CampeauG22} (0.71)& \cellcolor{red!40}\href{../works/NishikawaSTT19.pdf}{NishikawaSTT19} (0.69)& \cellcolor{red!40}\href{../works/FrankDT16.pdf}{FrankDT16} (0.69)\\
\index{PrataAN23}\href{../works/PrataAN23.pdf}{PrataAN23} R\&C\\
Euclid& \href{../works/AbreuPNF23.pdf}{AbreuPNF23} (0.42)& \href{../works/AbreuAPNM21.pdf}{AbreuAPNM21} (0.43)& \href{../works/OujanaAYB22.pdf}{OujanaAYB22} (0.43)& \href{../works/AlfieriGPS23.pdf}{AlfieriGPS23} (0.43)& \href{../works/TerekhovDOB12.pdf}{TerekhovDOB12} (0.44)\\
Dot& \cellcolor{red!40}\href{../works/ZarandiASC20.pdf}{ZarandiASC20} (292.00)& \cellcolor{red!40}\href{../works/Groleaz21.pdf}{Groleaz21} (253.00)& \cellcolor{red!40}\href{../works/Dejemeppe16.pdf}{Dejemeppe16} (232.00)& \cellcolor{red!40}\href{../works/Baptiste02.pdf}{Baptiste02} (227.00)& \cellcolor{red!40}\href{../works/Lunardi20.pdf}{Lunardi20} (212.00)\\
Cosine& \cellcolor{red!40}\href{../works/AbreuPNF23.pdf}{AbreuPNF23} (0.77)& \cellcolor{red!40}\href{../works/AbreuAPNM21.pdf}{AbreuAPNM21} (0.75)& \cellcolor{red!40}\href{../works/OujanaAYB22.pdf}{OujanaAYB22} (0.74)& \cellcolor{red!40}\href{../works/AlfieriGPS23.pdf}{AlfieriGPS23} (0.74)& \cellcolor{red!40}\href{../works/AbreuN22.pdf}{AbreuN22} (0.74)\\
\index{Prosser89}\href{../works/Prosser89.pdf}{Prosser89} R\&C\\
Euclid& \cellcolor{red!40}\href{../works/FoxAS82.pdf}{FoxAS82} (0.22)& \cellcolor{red!40}\href{../works/KengY89.pdf}{KengY89} (0.23)& \cellcolor{red!40}\href{../works/CrawfordB94.pdf}{CrawfordB94} (0.23)& \cellcolor{red!20}\href{../works/FukunagaHFAMN02.pdf}{FukunagaHFAMN02} (0.25)& \cellcolor{red!20}\href{../works/Salido10.pdf}{Salido10} (0.26)\\
Dot& \cellcolor{red!40}\href{../works/ZarandiASC20.pdf}{ZarandiASC20} (96.00)& \cellcolor{red!40}\href{../works/Beck99.pdf}{Beck99} (86.00)& \cellcolor{red!40}\href{../works/BeckDDF98.pdf}{BeckDDF98} (86.00)& \cellcolor{red!40}\href{../works/Lombardi10.pdf}{Lombardi10} (84.00)& \cellcolor{red!40}\href{../works/Groleaz21.pdf}{Groleaz21} (84.00)\\
Cosine& \cellcolor{red!40}\href{../works/KengY89.pdf}{KengY89} (0.76)& \cellcolor{red!40}\href{../works/FoxAS82.pdf}{FoxAS82} (0.76)& \cellcolor{red!40}\href{../works/Salido10.pdf}{Salido10} (0.73)& \cellcolor{red!40}\href{../works/CrawfordB94.pdf}{CrawfordB94} (0.73)& \cellcolor{red!40}\href{../works/FoxS90.pdf}{FoxS90} (0.72)\\
\index{Puget95}\href{../works/Puget95.pdf}{Puget95} R\&C& \cellcolor{red!40}\href{../works/HebrardTW05.pdf}{HebrardTW05} (0.80)& \cellcolor{red!40}\href{../works/MercierH07.pdf}{MercierH07} (0.80)& \cellcolor{red!40}\href{../works/VilimBC04.pdf}{VilimBC04} (0.83)& \cellcolor{red!40}\href{../works/VilimBC05.pdf}{VilimBC05} (0.86)& \cellcolor{red!40}\href{../works/Vilim04.pdf}{Vilim04} (0.86)\\
Euclid& \cellcolor{red!40}\href{../works/Caseau97.pdf}{Caseau97} (0.20)& \cellcolor{red!40}\href{../works/KovacsEKV05.pdf}{KovacsEKV05} (0.21)& \cellcolor{red!40}\href{../works/HebrardTW05.pdf}{HebrardTW05} (0.22)& \cellcolor{red!40}\href{../works/AngelsmarkJ00.pdf}{AngelsmarkJ00} (0.22)& \cellcolor{red!40}\href{../works/CrawfordB94.pdf}{CrawfordB94} (0.22)\\
Dot& \cellcolor{red!40}\href{../works/Malapert11.pdf}{Malapert11} (71.00)& \cellcolor{red!40}\href{../works/ZarandiASC20.pdf}{ZarandiASC20} (70.00)& \cellcolor{red!40}\href{../works/LaborieRSV18.pdf}{LaborieRSV18} (69.00)& \cellcolor{red!40}\href{../works/Siala15a.pdf}{Siala15a} (68.00)& \cellcolor{red!40}\href{../works/Godet21a.pdf}{Godet21a} (68.00)\\
Cosine& \cellcolor{red!40}\href{../works/Caseau97.pdf}{Caseau97} (0.75)& \cellcolor{red!40}\href{../works/FontaineMH16.pdf}{FontaineMH16} (0.72)& \cellcolor{red!40}\href{../works/TranDRFWOVB16.pdf}{TranDRFWOVB16} (0.70)& \cellcolor{red!40}\href{../works/CrawfordB94.pdf}{CrawfordB94} (0.69)& \cellcolor{red!40}\href{../works/ZeballosM09.pdf}{ZeballosM09} (0.68)\\
\index{QinDCS20}\href{../works/QinDCS20.pdf}{QinDCS20} R\&C& \cellcolor{red!40}\href{../works/SunTB19.pdf}{SunTB19} (0.71)& \cellcolor{red!40}\href{../works/UnsalO13.pdf}{UnsalO13} (0.77)& \cellcolor{yellow!20}\href{../works/UnsalO19.pdf}{UnsalO19} (0.91)& \cellcolor{yellow!20}\href{../works/GedikKEK18.pdf}{GedikKEK18} (0.91)& \cellcolor{yellow!20}GuoHLW20 (0.92)\\
Euclid& \cellcolor{black!20}\href{../works/Ham18.pdf}{Ham18} (0.37)& \href{../works/SunTB19.pdf}{SunTB19} (0.37)& \href{../works/Ham18a.pdf}{Ham18a} (0.37)& \href{../works/MurinR19.pdf}{MurinR19} (0.38)& \href{../works/QinDS16.pdf}{QinDS16} (0.38)\\
Dot& \cellcolor{red!40}\href{../works/Lunardi20.pdf}{Lunardi20} (158.00)& \cellcolor{red!40}\href{../works/Groleaz21.pdf}{Groleaz21} (152.00)& \cellcolor{red!40}\href{../works/ZarandiASC20.pdf}{ZarandiASC20} (139.00)& \cellcolor{red!40}\href{../works/LaborieRSV18.pdf}{LaborieRSV18} (138.00)& \cellcolor{red!40}\href{../works/NaderiRR23.pdf}{NaderiRR23} (137.00)\\
Cosine& \cellcolor{red!40}\href{../works/Ham18a.pdf}{Ham18a} (0.70)& \cellcolor{red!40}\href{../works/MurinR19.pdf}{MurinR19} (0.70)& \cellcolor{red!40}\href{../works/QinDS16.pdf}{QinDS16} (0.70)& \cellcolor{red!40}\href{../works/Ham18.pdf}{Ham18} (0.70)& \cellcolor{red!40}\href{../works/SunTB19.pdf}{SunTB19} (0.70)\\
\index{QinDS16}\href{../works/QinDS16.pdf}{QinDS16} R\&C& \cellcolor{red!20}\href{../works/UnsalO19.pdf}{UnsalO19} (0.86)& \cellcolor{yellow!20}\href{../works/ZampelliVSDR13.pdf}{ZampelliVSDR13} (0.91)& \cellcolor{yellow!20}\href{../works/CappartS17.pdf}{CappartS17} (0.92)& \cellcolor{yellow!20}\href{../works/CobanH11.pdf}{CobanH11} (0.93)& \cellcolor{yellow!20}\href{../works/Laborie18a.pdf}{Laborie18a} (0.93)\\
Euclid& \cellcolor{green!20}\href{../works/UnsalO19.pdf}{UnsalO19} (0.31)& \cellcolor{blue!20}\href{../works/CorreaLR07.pdf}{CorreaLR07} (0.32)& \cellcolor{blue!20}\href{../works/ZeballosM09.pdf}{ZeballosM09} (0.33)& \cellcolor{blue!20}\href{../works/EdisO11.pdf}{EdisO11} (0.33)& \cellcolor{blue!20}\href{../works/NovasH14.pdf}{NovasH14} (0.33)\\
Dot& \cellcolor{red!40}\href{../works/ZarandiASC20.pdf}{ZarandiASC20} (137.00)& \cellcolor{red!40}\href{../works/Groleaz21.pdf}{Groleaz21} (136.00)& \cellcolor{red!40}\href{../works/Lunardi20.pdf}{Lunardi20} (132.00)& \cellcolor{red!40}\href{../works/LaborieRSV18.pdf}{LaborieRSV18} (131.00)& \cellcolor{red!40}\href{../works/MilanoW09.pdf}{MilanoW09} (126.00)\\
Cosine& \cellcolor{red!40}\href{../works/UnsalO19.pdf}{UnsalO19} (0.77)& \cellcolor{red!40}\href{../works/CorreaLR07.pdf}{CorreaLR07} (0.75)& \cellcolor{red!40}\href{../works/NovasH14.pdf}{NovasH14} (0.74)& \cellcolor{red!40}\href{../works/ZeballosM09.pdf}{ZeballosM09} (0.73)& \cellcolor{red!40}\href{../works/EdisO11.pdf}{EdisO11} (0.73)\\
\index{QinWSLS21}\href{../works/QinWSLS21.pdf}{QinWSLS21} R\&C& \cellcolor{yellow!20}\href{../works/FanXG21.pdf}{FanXG21} (0.93)& \cellcolor{green!20}\href{../works/Ham18a.pdf}{Ham18a} (0.95)& \cellcolor{green!20}\href{../works/TanZWGQ19.pdf}{TanZWGQ19} (0.96)& \cellcolor{green!20}\href{../works/NattafDYW19.pdf}{NattafDYW19} (0.96)& \cellcolor{blue!20}\href{../works/LacknerMMWW23.pdf}{LacknerMMWW23} (0.97)\\
Euclid& \cellcolor{blue!20}\href{../works/LorigeonBB02.pdf}{LorigeonBB02} (0.32)& \cellcolor{blue!20}\href{../works/HamC16.pdf}{HamC16} (0.33)& \cellcolor{blue!20}\href{../works/BillautHL12.pdf}{BillautHL12} (0.33)& \cellcolor{blue!20}\href{../works/Beck06.pdf}{Beck06} (0.33)& \cellcolor{blue!20}\href{../works/JuvinHL23.pdf}{JuvinHL23} (0.34)\\
Dot& \cellcolor{red!40}\href{../works/ZarandiASC20.pdf}{ZarandiASC20} (149.00)& \cellcolor{red!40}\href{../works/Lunardi20.pdf}{Lunardi20} (129.00)& \cellcolor{red!40}\href{../works/Groleaz21.pdf}{Groleaz21} (126.00)& \cellcolor{red!40}\href{../works/Malapert11.pdf}{Malapert11} (125.00)& \cellcolor{red!40}\href{../works/IsikYA23.pdf}{IsikYA23} (115.00)\\
Cosine& \cellcolor{red!40}\href{../works/HamC16.pdf}{HamC16} (0.72)& \cellcolor{red!40}\href{../works/LorigeonBB02.pdf}{LorigeonBB02} (0.70)& \cellcolor{red!40}\href{../works/HamFC17.pdf}{HamFC17} (0.69)& \cellcolor{red!40}\href{../works/BillautHL12.pdf}{BillautHL12} (0.68)& \cellcolor{red!40}\href{../works/HamPK21.pdf}{HamPK21} (0.67)\\
\index{QuSN06}\href{../works/QuSN06.pdf}{QuSN06} R\&C& \cellcolor{yellow!20}\href{../works/WolinskiKG04.pdf}{WolinskiKG04} (0.93)& \cellcolor{green!20}\href{../works/KoschB14.pdf}{KoschB14} (0.96)& \cellcolor{green!20}\href{../works/KuchcinskiW03.pdf}{KuchcinskiW03} (0.96)& \cellcolor{green!20}\href{../works/NishikawaSTT19.pdf}{NishikawaSTT19} (0.96)& \cellcolor{blue!20}\href{../works/LombardiM10a.pdf}{LombardiM10a} (0.97)\\
Euclid& \cellcolor{red!40}\href{../works/BeniniBGM05a.pdf}{BeniniBGM05a} (0.18)& \cellcolor{red!40}\href{../works/GomesHS06.pdf}{GomesHS06} (0.19)& \cellcolor{red!40}\href{../works/KovacsEKV05.pdf}{KovacsEKV05} (0.21)& \cellcolor{red!40}\href{../works/AngelsmarkJ00.pdf}{AngelsmarkJ00} (0.21)& \cellcolor{red!40}\href{../works/CestaOS98.pdf}{CestaOS98} (0.22)\\
Dot& \cellcolor{red!40}\href{../works/Malapert11.pdf}{Malapert11} (59.00)& \cellcolor{red!40}\href{../works/Beck99.pdf}{Beck99} (56.00)& \cellcolor{red!40}\href{../works/Lombardi10.pdf}{Lombardi10} (56.00)& \cellcolor{red!40}\href{../works/Letort13.pdf}{Letort13} (56.00)& \cellcolor{red!40}\href{../works/TrojetHL11.pdf}{TrojetHL11} (55.00)\\
Cosine& \cellcolor{red!40}\href{../works/BeniniBGM05a.pdf}{BeniniBGM05a} (0.79)& \cellcolor{red!40}\href{../works/GomesHS06.pdf}{GomesHS06} (0.74)& \cellcolor{red!40}\href{../works/HoeveGSL07.pdf}{HoeveGSL07} (0.72)& \cellcolor{red!40}\href{../works/AlakaPY19.pdf}{AlakaPY19} (0.71)& \cellcolor{red!40}\href{../works/NishikawaSTT18.pdf}{NishikawaSTT18} (0.71)\\
\index{QuirogaZH05}\href{../works/QuirogaZH05.pdf}{QuirogaZH05} R\&C& \cellcolor{red!40}\href{../works/ZhangLS12.pdf}{ZhangLS12} (0.67)& \cellcolor{red!40}\href{../works/Geske05.pdf}{Geske05} (0.75)& \cellcolor{red!40}\href{../works/Zeballos10.pdf}{Zeballos10} (0.76)& \cellcolor{red!40}\href{../works/LimtanyakulS12.pdf}{LimtanyakulS12} (0.79)& \cellcolor{red!40}\href{../works/EvenSH15.pdf}{EvenSH15} (0.80)\\
Euclid& \cellcolor{red!40}\href{../works/Zeballos10.pdf}{Zeballos10} (0.19)& \cellcolor{red!40}\href{../works/ZeballosQH10.pdf}{ZeballosQH10} (0.23)& \cellcolor{red!40}\href{../works/ZeballosH05.pdf}{ZeballosH05} (0.23)& \cellcolor{red!20}\href{../works/ZeballosM09.pdf}{ZeballosM09} (0.25)& \cellcolor{yellow!20}\href{../works/BockmayrP06.pdf}{BockmayrP06} (0.27)\\
Dot& \cellcolor{red!40}\href{../works/ZarandiASC20.pdf}{ZarandiASC20} (134.00)& \cellcolor{red!40}\href{../works/Dejemeppe16.pdf}{Dejemeppe16} (126.00)& \cellcolor{red!40}\href{../works/Baptiste02.pdf}{Baptiste02} (124.00)& \cellcolor{red!40}\href{../works/Zeballos10.pdf}{Zeballos10} (118.00)& \cellcolor{red!40}\href{../works/ZeballosQH10.pdf}{ZeballosQH10} (117.00)\\
Cosine& \cellcolor{red!40}\href{../works/Zeballos10.pdf}{Zeballos10} (0.92)& \cellcolor{red!40}\href{../works/ZeballosQH10.pdf}{ZeballosQH10} (0.88)& \cellcolor{red!40}\href{../works/ZeballosH05.pdf}{ZeballosH05} (0.86)& \cellcolor{red!40}\href{../works/ZeballosM09.pdf}{ZeballosM09} (0.81)& \cellcolor{red!40}\href{../works/NovasH14.pdf}{NovasH14} (0.78)\\
\index{RabbaniMM21}RabbaniMM21 R\&C& \cellcolor{yellow!20}\href{../works/MokhtarzadehTNF20.pdf}{MokhtarzadehTNF20} (0.93)& \cellcolor{green!20}\href{../works/RoshanaeiN21.pdf}{RoshanaeiN21} (0.95)& \cellcolor{green!20}\href{../works/SacramentoSP20.pdf}{SacramentoSP20} (0.95)& \cellcolor{green!20}\href{../works/DejemeppeD14.pdf}{DejemeppeD14} (0.96)& \cellcolor{blue!20}\href{../works/AlakaPY19.pdf}{AlakaPY19} (0.97)\\
Euclid\\
Dot\\
Cosine\\
\index{RasmussenT06}\href{../works/RasmussenT06.pdf}{RasmussenT06} R\&C& \cellcolor{red!40}\href{../works/RasmussenT09.pdf}{RasmussenT09} (0.55)& \cellcolor{red!40}\href{../works/RasmussenT07.pdf}{RasmussenT07} (0.69)& \cellcolor{red!40}\href{../works/Trick03.pdf}{Trick03} (0.73)& \cellcolor{red!40}\href{../works/Perron05.pdf}{Perron05} (0.73)& \cellcolor{red!40}\href{../works/RussellU06.pdf}{RussellU06} (0.77)\\
Euclid& \cellcolor{red!40}\href{../works/SuCC13.pdf}{SuCC13} (0.15)& \cellcolor{red!40}\href{../works/EastonNT02.pdf}{EastonNT02} (0.16)& \cellcolor{red!40}\href{../works/RasmussenT09.pdf}{RasmussenT09} (0.18)& \cellcolor{red!40}\href{../works/RasmussenT07.pdf}{RasmussenT07} (0.19)& \cellcolor{red!40}\href{../works/Trick03.pdf}{Trick03} (0.19)\\
Dot& \cellcolor{red!40}\href{../works/KendallKRU10.pdf}{KendallKRU10} (78.00)& \cellcolor{red!40}\href{../works/RasmussenT09.pdf}{RasmussenT09} (73.00)& \cellcolor{red!40}\href{../works/Ribeiro12.pdf}{Ribeiro12} (70.00)& \cellcolor{red!40}\href{../works/RasmussenT07.pdf}{RasmussenT07} (67.00)& \cellcolor{red!40}\href{../works/ZarandiASC20.pdf}{ZarandiASC20} (64.00)\\
Cosine& \cellcolor{red!40}\href{../works/RasmussenT09.pdf}{RasmussenT09} (0.88)& \cellcolor{red!40}\href{../works/SuCC13.pdf}{SuCC13} (0.88)& \cellcolor{red!40}\href{../works/EastonNT02.pdf}{EastonNT02} (0.86)& \cellcolor{red!40}\href{../works/RasmussenT07.pdf}{RasmussenT07} (0.85)& \cellcolor{red!40}\href{../works/ZengM12.pdf}{ZengM12} (0.83)\\
\index{RasmussenT07}\href{../works/RasmussenT07.pdf}{RasmussenT07} R\&C& \cellcolor{red!40}\href{../works/RasmussenT09.pdf}{RasmussenT09} (0.68)& \cellcolor{red!40}\href{../works/RasmussenT06.pdf}{RasmussenT06} (0.69)& \cellcolor{red!40}\href{../works/HenzMT04.pdf}{HenzMT04} (0.73)& \cellcolor{red!40}Henz01 (0.73)& \cellcolor{red!40}\href{../works/Trick03.pdf}{Trick03} (0.73)\\
Euclid& \cellcolor{red!40}\href{../works/RasmussenT09.pdf}{RasmussenT09} (0.19)& \cellcolor{red!40}\href{../works/RasmussenT06.pdf}{RasmussenT06} (0.19)& \cellcolor{red!40}\href{../works/ZengM12.pdf}{ZengM12} (0.19)& \cellcolor{red!40}\href{../works/BulckG22.pdf}{BulckG22} (0.20)& \cellcolor{red!40}\href{../works/SuCC13.pdf}{SuCC13} (0.21)\\
Dot& \cellcolor{red!40}\href{../works/KendallKRU10.pdf}{KendallKRU10} (93.00)& \cellcolor{red!40}\href{../works/Ribeiro12.pdf}{Ribeiro12} (86.00)& \cellcolor{red!40}\href{../works/RasmussenT09.pdf}{RasmussenT09} (82.00)& \cellcolor{red!40}\href{../works/ZengM12.pdf}{ZengM12} (72.00)& \cellcolor{red!40}\href{../works/LarsonJC14.pdf}{LarsonJC14} (70.00)\\
Cosine& \cellcolor{red!40}\href{../works/RasmussenT09.pdf}{RasmussenT09} (0.87)& \cellcolor{red!40}\href{../works/RasmussenT06.pdf}{RasmussenT06} (0.85)& \cellcolor{red!40}\href{../works/ZengM12.pdf}{ZengM12} (0.85)& \cellcolor{red!40}\href{../works/BulckG22.pdf}{BulckG22} (0.83)& \cellcolor{red!40}\href{../works/EastonNT02.pdf}{EastonNT02} (0.82)\\
\index{RasmussenT09}\href{../works/RasmussenT09.pdf}{RasmussenT09} R\&C& \cellcolor{red!40}\href{../works/RasmussenT06.pdf}{RasmussenT06} (0.55)& \cellcolor{red!40}\href{../works/RasmussenT07.pdf}{RasmussenT07} (0.68)& \cellcolor{red!40}\href{../works/ZengM12.pdf}{ZengM12} (0.76)& \cellcolor{red!40}Trick11 (0.78)& \cellcolor{red!40}\href{../works/Trick03.pdf}{Trick03} (0.80)\\
Euclid& \cellcolor{red!40}\href{../works/ZengM12.pdf}{ZengM12} (0.18)& \cellcolor{red!40}\href{../works/RasmussenT06.pdf}{RasmussenT06} (0.18)& \cellcolor{red!40}\href{../works/RasmussenT07.pdf}{RasmussenT07} (0.19)& \cellcolor{red!40}\href{../works/EastonNT02.pdf}{EastonNT02} (0.21)& \cellcolor{red!40}\href{../works/Ribeiro12.pdf}{Ribeiro12} (0.21)\\
Dot& \cellcolor{red!40}\href{../works/KendallKRU10.pdf}{KendallKRU10} (107.00)& \cellcolor{red!40}\href{../works/Ribeiro12.pdf}{Ribeiro12} (97.00)& \cellcolor{red!40}\href{../works/RasmussenT07.pdf}{RasmussenT07} (82.00)& \cellcolor{red!40}\href{../works/ZarandiASC20.pdf}{ZarandiASC20} (80.00)& \cellcolor{red!40}\href{../works/ZengM12.pdf}{ZengM12} (79.00)\\
Cosine& \cellcolor{red!40}\href{../works/RasmussenT06.pdf}{RasmussenT06} (0.88)& \cellcolor{red!40}\href{../works/ZengM12.pdf}{ZengM12} (0.88)& \cellcolor{red!40}\href{../works/RasmussenT07.pdf}{RasmussenT07} (0.87)& \cellcolor{red!40}\href{../works/Ribeiro12.pdf}{Ribeiro12} (0.87)& \cellcolor{red!40}\href{../works/EastonNT02.pdf}{EastonNT02} (0.84)\\
\index{ReddyFIBKAJ11}\href{../works/ReddyFIBKAJ11.pdf}{ReddyFIBKAJ11} R\&C& \cellcolor{yellow!20}\href{../works/SimoninAHL15.pdf}{SimoninAHL15} (0.92)& \cellcolor{yellow!20}\href{../works/JelinekB16.pdf}{JelinekB16} (0.92)& \cellcolor{yellow!20}\href{../works/Rodriguez07.pdf}{Rodriguez07} (0.93)& \cellcolor{green!20}\href{../works/SimoninAHL12.pdf}{SimoninAHL12} (0.94)& \cellcolor{green!20}\href{../works/LouieVNB14.pdf}{LouieVNB14} (0.94)\\
Euclid& \cellcolor{red!40}\href{../works/WallaceF00.pdf}{WallaceF00} (0.24)& \cellcolor{yellow!20}\href{../works/OddiPCC03.pdf}{OddiPCC03} (0.27)& \cellcolor{yellow!20}\href{../works/FukunagaHFAMN02.pdf}{FukunagaHFAMN02} (0.28)& \cellcolor{yellow!20}\href{../works/BartakCS10.pdf}{BartakCS10} (0.28)& \cellcolor{green!20}\href{../works/LombardiM13.pdf}{LombardiM13} (0.29)\\
Dot& \cellcolor{red!40}\href{../works/Godet21a.pdf}{Godet21a} (90.00)& \cellcolor{red!40}\href{../works/Lombardi10.pdf}{Lombardi10} (88.00)& \cellcolor{red!40}\href{../works/ZarandiASC20.pdf}{ZarandiASC20} (85.00)& \cellcolor{red!40}\href{../works/Baptiste02.pdf}{Baptiste02} (82.00)& \cellcolor{red!40}\href{../works/Dejemeppe16.pdf}{Dejemeppe16} (81.00)\\
Cosine& \cellcolor{red!40}\href{../works/WallaceF00.pdf}{WallaceF00} (0.75)& \cellcolor{red!40}\href{../works/OddiPCC03.pdf}{OddiPCC03} (0.72)& \cellcolor{red!40}\href{../works/BartakCS10.pdf}{BartakCS10} (0.68)& \cellcolor{red!40}\href{../works/BocewiczBB09.pdf}{BocewiczBB09} (0.66)& \cellcolor{red!40}\href{../works/BarbulescuWH04.pdf}{BarbulescuWH04} (0.66)\\
\index{Refalo00}\href{../works/Refalo00.pdf}{Refalo00} R\&C& \cellcolor{red!20}MilanoORT02 (0.87)& \cellcolor{red!20}\href{../works/Thorsteinsson01.pdf}{Thorsteinsson01} (0.87)& \cellcolor{red!20}\href{../works/HookerY02.pdf}{HookerY02} (0.87)& \cellcolor{red!20}\href{../works/Hooker05b.pdf}{Hooker05b} (0.89)& \cellcolor{red!20}AjiliW04 (0.89)\\
Euclid& \cellcolor{red!40}\href{../works/Davis87.pdf}{Davis87} (0.24)& \cellcolor{red!20}\href{../works/Valdes87.pdf}{Valdes87} (0.24)& \cellcolor{red!20}\href{../works/ZibranR11.pdf}{ZibranR11} (0.26)& \cellcolor{red!20}\href{../works/ChapadosJR11.pdf}{ChapadosJR11} (0.26)& \cellcolor{yellow!20}\href{../works/ZhangLS12.pdf}{ZhangLS12} (0.27)\\
Dot& \cellcolor{red!40}\href{../works/Siala15a.pdf}{Siala15a} (54.00)& \cellcolor{red!40}\href{../works/Malapert11.pdf}{Malapert11} (53.00)& \cellcolor{red!40}\href{../works/Schutt11.pdf}{Schutt11} (52.00)& \cellcolor{red!40}\href{../works/Dejemeppe16.pdf}{Dejemeppe16} (51.00)& \cellcolor{red!40}\href{../works/Wallace96.pdf}{Wallace96} (48.00)\\
Cosine& \cellcolor{red!40}\href{../works/HookerOTK00.pdf}{HookerOTK00} (0.61)& \cellcolor{red!40}\href{../works/ZhangLS12.pdf}{ZhangLS12} (0.55)& \cellcolor{red!40}\href{../works/HookerO99.pdf}{HookerO99} (0.54)& \cellcolor{red!40}\href{../works/Davis87.pdf}{Davis87} (0.53)& \cellcolor{red!40}\href{../works/BukchinR18.pdf}{BukchinR18} (0.53)\\
\index{RenT09}\href{../works/RenT09.pdf}{RenT09} R\&C& \cellcolor{red!20}\href{../works/Thorsteinsson01.pdf}{Thorsteinsson01} (0.88)& \cellcolor{red!20}\href{../works/ZeballosM09.pdf}{ZeballosM09} (0.88)& \cellcolor{red!20}\href{../works/RoePS05.pdf}{RoePS05} (0.89)& \cellcolor{yellow!20}\href{../works/HarjunkoskiG02.pdf}{HarjunkoskiG02} (0.91)& \cellcolor{yellow!20}GongLMW09 (0.92)\\
Euclid& \cellcolor{red!20}\href{../works/HebrardTW05.pdf}{HebrardTW05} (0.24)& \cellcolor{red!20}\href{../works/HarjunkoskiJG00.pdf}{HarjunkoskiJG00} (0.26)& \cellcolor{green!20}\href{../works/BofillGSV15.pdf}{BofillGSV15} (0.29)& \cellcolor{green!20}\href{../works/LauLN08.pdf}{LauLN08} (0.29)& \cellcolor{green!20}\href{../works/KovacsEKV05.pdf}{KovacsEKV05} (0.29)\\
Dot& \cellcolor{red!40}\href{../works/Malapert11.pdf}{Malapert11} (87.00)& \cellcolor{red!40}\href{../works/HarjunkoskiMBC14.pdf}{HarjunkoskiMBC14} (77.00)& \cellcolor{red!40}\href{../works/ColT22.pdf}{ColT22} (75.00)& \cellcolor{red!40}\href{../works/ZarandiASC20.pdf}{ZarandiASC20} (74.00)& \cellcolor{red!40}\href{../works/LaborieRSV18.pdf}{LaborieRSV18} (74.00)\\
Cosine& \cellcolor{red!40}\href{../works/HebrardTW05.pdf}{HebrardTW05} (0.74)& \cellcolor{red!40}\href{../works/HarjunkoskiJG00.pdf}{HarjunkoskiJG00} (0.72)& \cellcolor{red!40}\href{../works/Colombani96.pdf}{Colombani96} (0.68)& \cellcolor{red!40}\href{../works/HarjunkoskiG02.pdf}{HarjunkoskiG02} (0.68)& \cellcolor{red!40}\href{../works/WatsonB08.pdf}{WatsonB08} (0.67)\\
\index{RendlPHPR12}\href{../works/RendlPHPR12.pdf}{RendlPHPR12} R\&C& \cellcolor{green!20}\href{../works/ArmstrongGOS21.pdf}{ArmstrongGOS21} (0.93)& \cellcolor{green!20}\href{../works/LimBTBB15a.pdf}{LimBTBB15a} (0.94)& \cellcolor{green!20}\href{../works/TangB20.pdf}{TangB20} (0.95)& \cellcolor{green!20}\href{../works/FrohnerTR19.pdf}{FrohnerTR19} (0.95)& \cellcolor{green!20}\href{../works/HechingH16.pdf}{HechingH16} (0.96)\\
Euclid& \cellcolor{blue!20}\href{../works/HoYCLLCLC18.pdf}{HoYCLLCLC18} (0.32)& \cellcolor{blue!20}\href{../works/LimRX04.pdf}{LimRX04} (0.33)& \cellcolor{blue!20}\href{../works/BarzegaranZP20.pdf}{BarzegaranZP20} (0.33)& \cellcolor{blue!20}\href{../works/ZibranR11.pdf}{ZibranR11} (0.34)& \cellcolor{black!20}\href{../works/KletzanderM17.pdf}{KletzanderM17} (0.34)\\
Dot& \cellcolor{red!40}\href{../works/ZarandiASC20.pdf}{ZarandiASC20} (109.00)& \cellcolor{red!40}\href{../works/Dejemeppe16.pdf}{Dejemeppe16} (99.00)& \cellcolor{red!40}\href{../works/Lemos21.pdf}{Lemos21} (97.00)& \cellcolor{red!40}\href{../works/Astrand21.pdf}{Astrand21} (94.00)& \cellcolor{red!40}\href{../works/Froger16.pdf}{Froger16} (93.00)\\
Cosine& \cellcolor{red!40}\href{../works/HoYCLLCLC18.pdf}{HoYCLLCLC18} (0.62)& \cellcolor{red!40}\href{../works/GedikKBR17.pdf}{GedikKBR17} (0.61)& \cellcolor{red!40}\href{../works/BarzegaranZP20.pdf}{BarzegaranZP20} (0.60)& \cellcolor{red!40}\href{../works/LimRX04.pdf}{LimRX04} (0.60)& \cellcolor{red!40}\href{../works/WatsonB08.pdf}{WatsonB08} (0.59)\\
\index{Rgin2001}Rgin2001 R\&C& \cellcolor{red!20}\href{../works/ElfJR03.pdf}{ElfJR03} (0.87)& \cellcolor{red!20}Henz01 (0.88)& \cellcolor{red!20}\href{../works/HenzMT04.pdf}{HenzMT04} (0.89)& \cellcolor{red!20}\href{../works/RasmussenT06.pdf}{RasmussenT06} (0.89)& \cellcolor{yellow!20}\href{../works/RasmussenT07.pdf}{RasmussenT07} (0.91)\\
Euclid\\
Dot\\
Cosine\\
\index{RiahiNS018}\href{../works/RiahiNS018.pdf}{RiahiNS018} R\&C\\
Euclid& \cellcolor{black!20}\href{../works/WatsonBHW99.pdf}{WatsonBHW99} (0.34)& \cellcolor{black!20}\href{../works/LiLZDZW24.pdf}{LiLZDZW24} (0.35)& \cellcolor{black!20}\href{../works/DoRZ08.pdf}{DoRZ08} (0.36)& \cellcolor{black!20}\href{../works/JuvinHL23.pdf}{JuvinHL23} (0.37)& \cellcolor{black!20}\href{../works/LiFJZLL22.pdf}{LiFJZLL22} (0.37)\\
Dot& \cellcolor{red!40}\href{../works/AbreuNP23.pdf}{AbreuNP23} (114.00)& \cellcolor{red!40}\href{../works/IsikYA23.pdf}{IsikYA23} (112.00)& \cellcolor{red!40}\href{../works/ZarandiASC20.pdf}{ZarandiASC20} (107.00)& \cellcolor{red!40}\href{../works/Astrand21.pdf}{Astrand21} (106.00)& \cellcolor{red!40}\href{../works/AbreuPNF23.pdf}{AbreuPNF23} (100.00)\\
Cosine& \cellcolor{red!40}\href{../works/AbreuNP23.pdf}{AbreuNP23} (0.67)& \cellcolor{red!40}\href{../works/WatsonBHW99.pdf}{WatsonBHW99} (0.65)& \cellcolor{red!40}\href{../works/LiFJZLL22.pdf}{LiFJZLL22} (0.63)& \cellcolor{red!40}\href{../works/LiLZDZW24.pdf}{LiLZDZW24} (0.62)& \cellcolor{red!40}\href{../works/AlfieriGPS23.pdf}{AlfieriGPS23} (0.62)\\
\index{Ribeiro12}\href{../works/Ribeiro12.pdf}{Ribeiro12} R\&C& \cellcolor{red!40}Trick11 (0.79)& \cellcolor{red!40}\href{../works/RasmussenT07.pdf}{RasmussenT07} (0.82)& \cellcolor{red!40}\href{../works/ZengM12.pdf}{ZengM12} (0.84)& \cellcolor{red!40}\href{../works/KendallKRU10.pdf}{KendallKRU10} (0.86)& \cellcolor{red!20}\href{../works/Trick03.pdf}{Trick03} (0.86)\\
Euclid& \cellcolor{red!40}\href{../works/RasmussenT09.pdf}{RasmussenT09} (0.21)& \cellcolor{red!20}\href{../works/RasmussenT07.pdf}{RasmussenT07} (0.25)& \cellcolor{red!20}\href{../works/BulckG22.pdf}{BulckG22} (0.25)& \cellcolor{red!20}\href{../works/ZengM12.pdf}{ZengM12} (0.26)& \cellcolor{yellow!20}\href{../works/KendallKRU10.pdf}{KendallKRU10} (0.27)\\
Dot& \cellcolor{red!40}\href{../works/KendallKRU10.pdf}{KendallKRU10} (134.00)& \cellcolor{red!40}\href{../works/ZarandiASC20.pdf}{ZarandiASC20} (102.00)& \cellcolor{red!40}\href{../works/RasmussenT09.pdf}{RasmussenT09} (97.00)& \cellcolor{red!40}\href{../works/Lemos21.pdf}{Lemos21} (90.00)& \cellcolor{red!40}\href{../works/RasmussenT07.pdf}{RasmussenT07} (86.00)\\
Cosine& \cellcolor{red!40}\href{../works/RasmussenT09.pdf}{RasmussenT09} (0.87)& \cellcolor{red!40}\href{../works/KendallKRU10.pdf}{KendallKRU10} (0.86)& \cellcolor{red!40}\href{../works/BulckG22.pdf}{BulckG22} (0.81)& \cellcolor{red!40}\href{../works/RasmussenT07.pdf}{RasmussenT07} (0.81)& \cellcolor{red!40}\href{../works/ZengM12.pdf}{ZengM12} (0.79)\\
\index{RiiseML16}\href{../works/RiiseML16.pdf}{RiiseML16} R\&C& \cellcolor{red!40}\href{../works/RoshanaeiLAU17.pdf}{RoshanaeiLAU17} (0.69)& \cellcolor{red!40}RoshanaeiLAU17a (0.75)& \cellcolor{red!40}\href{../works/RoshanaeiBAUB20.pdf}{RoshanaeiBAUB20} (0.75)& \cellcolor{red!40}ZarandiB12 (0.82)& \cellcolor{red!40}\href{../works/RoshanaeiN21.pdf}{RoshanaeiN21} (0.82)\\
Euclid& \cellcolor{green!20}\href{../works/GurEA19.pdf}{GurEA19} (0.29)& \cellcolor{green!20}\href{../works/DoulabiRP14.pdf}{DoulabiRP14} (0.30)& \cellcolor{green!20}\href{../works/GhandehariK22.pdf}{GhandehariK22} (0.30)& \cellcolor{green!20}\href{../works/DoulabiRP16.pdf}{DoulabiRP16} (0.31)& \cellcolor{blue!20}\href{../works/WangMD15.pdf}{WangMD15} (0.33)\\
Dot& \cellcolor{red!40}\href{../works/ZarandiASC20.pdf}{ZarandiASC20} (121.00)& \cellcolor{red!40}\href{../works/Dejemeppe16.pdf}{Dejemeppe16} (106.00)& \cellcolor{red!40}\href{../works/RoshanaeiBAUB20.pdf}{RoshanaeiBAUB20} (105.00)& \cellcolor{red!40}\href{../works/Lombardi10.pdf}{Lombardi10} (105.00)& \cellcolor{red!40}\href{../works/Astrand21.pdf}{Astrand21} (103.00)\\
Cosine& \cellcolor{red!40}\href{../works/GurEA19.pdf}{GurEA19} (0.77)& \cellcolor{red!40}\href{../works/GhandehariK22.pdf}{GhandehariK22} (0.75)& \cellcolor{red!40}\href{../works/RoshanaeiBAUB20.pdf}{RoshanaeiBAUB20} (0.74)& \cellcolor{red!40}\href{../works/DoulabiRP14.pdf}{DoulabiRP14} (0.73)& \cellcolor{red!40}\href{../works/WangMD15.pdf}{WangMD15} (0.73)\\
\index{Rit86}\href{../works/Rit86.pdf}{Rit86} R\&C\\
Euclid& \cellcolor{red!40}\href{../works/AngelsmarkJ00.pdf}{AngelsmarkJ00} (0.16)& \cellcolor{red!40}\href{../works/LiuJ06.pdf}{LiuJ06} (0.17)& \cellcolor{red!40}\href{../works/LudwigKRBMS14.pdf}{LudwigKRBMS14} (0.18)& \cellcolor{red!40}\href{../works/Valdes87.pdf}{Valdes87} (0.19)& \cellcolor{red!40}\href{../works/CarchraeBF05.pdf}{CarchraeBF05} (0.19)\\
Dot& \cellcolor{red!40}\href{../works/Malapert11.pdf}{Malapert11} (54.00)& \cellcolor{red!40}\href{../works/Baptiste02.pdf}{Baptiste02} (54.00)& \cellcolor{red!40}\href{../works/Siala15.pdf}{Siala15} (53.00)& \cellcolor{red!40}\href{../works/Siala15a.pdf}{Siala15a} (53.00)& \cellcolor{red!40}\href{../works/Godet21a.pdf}{Godet21a} (53.00)\\
Cosine& \cellcolor{red!40}\href{../works/AngelsmarkJ00.pdf}{AngelsmarkJ00} (0.76)& \cellcolor{red!40}\href{../works/LudwigKRBMS14.pdf}{LudwigKRBMS14} (0.73)& \cellcolor{red!40}\href{../works/DincbasSH90.pdf}{DincbasSH90} (0.73)& \cellcolor{red!40}\href{../works/LiuJ06.pdf}{LiuJ06} (0.73)& \cellcolor{red!40}\href{../works/BeldiceanuCP08.pdf}{BeldiceanuCP08} (0.69)\\
\index{Rodosek94}Rodosek94 R\&C\\
Euclid\\
Dot\\
Cosine\\
\index{RodosekW98}\href{../works/RodosekW98.pdf}{RodosekW98} R\&C& \cellcolor{red!40}\href{../works/BaptisteLV92.pdf}{BaptisteLV92} (0.78)& \cellcolor{yellow!20}\href{../works/BosiM2001.pdf}{BosiM2001} (0.91)& \cellcolor{yellow!20}\href{../works/EreminW01.pdf}{EreminW01} (0.91)& \cellcolor{yellow!20}DarbyDowmanL98 (0.92)& \cellcolor{yellow!20}\href{../works/EdisO11.pdf}{EdisO11} (0.92)\\
Euclid& \cellcolor{yellow!20}\href{../works/RodosekWH99.pdf}{RodosekWH99} (0.26)& \cellcolor{blue!20}\href{../works/HookerO99.pdf}{HookerO99} (0.33)& \cellcolor{black!20}\href{../works/DincbasSH90.pdf}{DincbasSH90} (0.34)& \cellcolor{black!20}\href{../works/Rit86.pdf}{Rit86} (0.36)& \cellcolor{black!20}\href{../works/LiuJ06.pdf}{LiuJ06} (0.37)\\
Dot& \cellcolor{red!40}\href{../works/Malapert11.pdf}{Malapert11} (117.00)& \cellcolor{red!40}\href{../works/Baptiste02.pdf}{Baptiste02} (100.00)& \cellcolor{red!40}\href{../works/Schutt11.pdf}{Schutt11} (97.00)& \cellcolor{red!40}\href{../works/Simonis99.pdf}{Simonis99} (95.00)& \cellcolor{red!40}\href{../works/Siala15a.pdf}{Siala15a} (92.00)\\
Cosine& \cellcolor{red!40}\href{../works/RodosekWH99.pdf}{RodosekWH99} (0.80)& \cellcolor{red!40}\href{../works/HookerO99.pdf}{HookerO99} (0.70)& \cellcolor{red!40}\href{../works/DincbasSH90.pdf}{DincbasSH90} (0.64)& \cellcolor{red!40}\href{../works/Wallace96.pdf}{Wallace96} (0.62)& \cellcolor{red!40}\href{../works/Darby-DowmanLMZ97.pdf}{Darby-DowmanLMZ97} (0.62)\\
\index{RodosekWH99}\href{../works/RodosekWH99.pdf}{RodosekWH99} R\&C& \cellcolor{red!40}BockmayrK98 (0.80)& \cellcolor{red!20}\href{../works/HookerO99.pdf}{HookerO99} (0.87)& \cellcolor{red!20}\href{../works/Darby-DowmanLMZ97.pdf}{Darby-DowmanLMZ97} (0.89)& \cellcolor{yellow!20}\href{../works/SmithBHW96.pdf}{SmithBHW96} (0.91)& \cellcolor{yellow!20}DarbyDowmanL98 (0.91)\\
Euclid& \cellcolor{yellow!20}\href{../works/RodosekW98.pdf}{RodosekW98} (0.26)& \cellcolor{green!20}\href{../works/Darby-DowmanLMZ97.pdf}{Darby-DowmanLMZ97} (0.29)& \cellcolor{green!20}\href{../works/HookerO99.pdf}{HookerO99} (0.30)& \cellcolor{blue!20}\href{../works/Touraivane95.pdf}{Touraivane95} (0.32)& \cellcolor{blue!20}\href{../works/HarjunkoskiJG00.pdf}{HarjunkoskiJG00} (0.33)\\
Dot& \cellcolor{red!40}\href{../works/Malapert11.pdf}{Malapert11} (95.00)& \cellcolor{red!40}\href{../works/RodosekW98.pdf}{RodosekW98} (92.00)& \cellcolor{red!40}\href{../works/Darby-DowmanLMZ97.pdf}{Darby-DowmanLMZ97} (86.00)& \cellcolor{red!40}\href{../works/HookerO99.pdf}{HookerO99} (86.00)& \cellcolor{red!40}\href{../works/Baptiste02.pdf}{Baptiste02} (84.00)\\
Cosine& \cellcolor{red!40}\href{../works/RodosekW98.pdf}{RodosekW98} (0.80)& \cellcolor{red!40}\href{../works/Darby-DowmanLMZ97.pdf}{Darby-DowmanLMZ97} (0.75)& \cellcolor{red!40}\href{../works/HookerO99.pdf}{HookerO99} (0.74)& \cellcolor{red!40}\href{../works/HarjunkoskiJG00.pdf}{HarjunkoskiJG00} (0.60)& \cellcolor{red!40}\href{../works/EreminW01.pdf}{EreminW01} (0.59)\\
\index{Rodriguez07}\href{../works/Rodriguez07.pdf}{Rodriguez07} R\&C& \cellcolor{red!40}\href{../works/Colombani96.pdf}{Colombani96} (0.82)& \cellcolor{red!40}\href{../works/NuijtenA96.pdf}{NuijtenA96} (0.83)& \cellcolor{red!40}\href{../works/Zhou96.pdf}{Zhou96} (0.85)& \cellcolor{red!40}\href{../works/Salido10.pdf}{Salido10} (0.86)& \cellcolor{red!20}\href{../works/Puget95.pdf}{Puget95} (0.88)\\
Euclid& \cellcolor{red!40}\href{../works/Rodriguez07b.pdf}{Rodriguez07b} (0.22)& \cellcolor{red!40}\href{../works/RodriguezS09.pdf}{RodriguezS09} (0.24)& \cellcolor{red!20}\href{../works/RodriguezDG02.pdf}{RodriguezDG02} (0.25)& \cellcolor{green!20}\href{../works/Puget95.pdf}{Puget95} (0.30)& \cellcolor{green!20}\href{../works/GilesH16.pdf}{GilesH16} (0.31)\\
Dot& \cellcolor{red!40}\href{../works/Godet21a.pdf}{Godet21a} (108.00)& \cellcolor{red!40}\href{../works/Malapert11.pdf}{Malapert11} (107.00)& \cellcolor{red!40}\href{../works/ZarandiASC20.pdf}{ZarandiASC20} (106.00)& \cellcolor{red!40}\href{../works/Lombardi10.pdf}{Lombardi10} (106.00)& \cellcolor{red!40}\href{../works/LaborieRSV18.pdf}{LaborieRSV18} (103.00)\\
Cosine& \cellcolor{red!40}\href{../works/Rodriguez07b.pdf}{Rodriguez07b} (0.85)& \cellcolor{red!40}\href{../works/RodriguezS09.pdf}{RodriguezS09} (0.83)& \cellcolor{red!40}\href{../works/RodriguezDG02.pdf}{RodriguezDG02} (0.80)& \cellcolor{red!40}\href{../works/ZeballosM09.pdf}{ZeballosM09} (0.70)& \cellcolor{red!40}\href{../works/MarliereSPR23.pdf}{MarliereSPR23} (0.70)\\
\index{Rodriguez07b}\href{../works/Rodriguez07b.pdf}{Rodriguez07b} R\&C\\
Euclid& \cellcolor{red!40}\href{../works/RodriguezS09.pdf}{RodriguezS09} (0.12)& \cellcolor{red!40}\href{../works/Rodriguez07.pdf}{Rodriguez07} (0.22)& \cellcolor{red!40}\href{../works/RodriguezDG02.pdf}{RodriguezDG02} (0.23)& \cellcolor{green!20}\href{../works/CappartS17.pdf}{CappartS17} (0.29)& \cellcolor{green!20}\href{../works/Puget95.pdf}{Puget95} (0.30)\\
Dot& \cellcolor{red!40}\href{../works/RodriguezS09.pdf}{RodriguezS09} (107.00)& \cellcolor{red!40}\href{../works/Malapert11.pdf}{Malapert11} (100.00)& \cellcolor{red!40}\href{../works/ZarandiASC20.pdf}{ZarandiASC20} (98.00)& \cellcolor{red!40}\href{../works/Baptiste02.pdf}{Baptiste02} (98.00)& \cellcolor{red!40}\href{../works/Rodriguez07.pdf}{Rodriguez07} (95.00)\\
Cosine& \cellcolor{red!40}\href{../works/RodriguezS09.pdf}{RodriguezS09} (0.96)& \cellcolor{red!40}\href{../works/Rodriguez07.pdf}{Rodriguez07} (0.85)& \cellcolor{red!40}\href{../works/RodriguezDG02.pdf}{RodriguezDG02} (0.82)& \cellcolor{red!40}\href{../works/CappartS17.pdf}{CappartS17} (0.75)& \cellcolor{red!40}\href{../works/MarliereSPR23.pdf}{MarliereSPR23} (0.70)\\
\index{RodriguezDG02}\href{../works/RodriguezDG02.pdf}{RodriguezDG02} R\&C\\
Euclid& \cellcolor{red!40}\href{../works/Acuna-AgostMFG09.pdf}{Acuna-AgostMFG09} (0.22)& \cellcolor{red!40}\href{../works/ZibranR11.pdf}{ZibranR11} (0.23)& \cellcolor{red!40}\href{../works/Rodriguez07b.pdf}{Rodriguez07b} (0.23)& \cellcolor{red!40}\href{../works/ZibranR11a.pdf}{ZibranR11a} (0.23)& \cellcolor{red!40}\href{../works/WallaceF00.pdf}{WallaceF00} (0.23)\\
Dot& \cellcolor{red!40}\href{../works/RodriguezS09.pdf}{RodriguezS09} (65.00)& \cellcolor{red!40}\href{../works/Rodriguez07.pdf}{Rodriguez07} (65.00)& \cellcolor{red!40}\href{../works/Rodriguez07b.pdf}{Rodriguez07b} (64.00)& \cellcolor{red!40}\href{../works/MarliereSPR23.pdf}{MarliereSPR23} (62.00)& \cellcolor{red!40}\href{../works/Lombardi10.pdf}{Lombardi10} (59.00)\\
Cosine& \cellcolor{red!40}\href{../works/Rodriguez07b.pdf}{Rodriguez07b} (0.82)& \cellcolor{red!40}\href{../works/RodriguezS09.pdf}{RodriguezS09} (0.80)& \cellcolor{red!40}\href{../works/Rodriguez07.pdf}{Rodriguez07} (0.80)& \cellcolor{red!40}\href{../works/CappartS17.pdf}{CappartS17} (0.69)& \cellcolor{red!40}\href{../works/ZibranR11a.pdf}{ZibranR11a} (0.67)\\
\index{RodriguezS09}\href{../works/RodriguezS09.pdf}{RodriguezS09} R\&C\\
Euclid& \cellcolor{red!40}\href{../works/Rodriguez07b.pdf}{Rodriguez07b} (0.12)& \cellcolor{red!40}\href{../works/Rodriguez07.pdf}{Rodriguez07} (0.24)& \cellcolor{red!20}\href{../works/RodriguezDG02.pdf}{RodriguezDG02} (0.25)& \cellcolor{green!20}\href{../works/CappartS17.pdf}{CappartS17} (0.29)& \cellcolor{green!20}\href{../works/Puget95.pdf}{Puget95} (0.31)\\
Dot& \cellcolor{red!40}\href{../works/Malapert11.pdf}{Malapert11} (109.00)& \cellcolor{red!40}\href{../works/Rodriguez07b.pdf}{Rodriguez07b} (107.00)& \cellcolor{red!40}\href{../works/Lombardi10.pdf}{Lombardi10} (106.00)& \cellcolor{red!40}\href{../works/ZarandiASC20.pdf}{ZarandiASC20} (105.00)& \cellcolor{red!40}\href{../works/Baptiste02.pdf}{Baptiste02} (102.00)\\
Cosine& \cellcolor{red!40}\href{../works/Rodriguez07b.pdf}{Rodriguez07b} (0.96)& \cellcolor{red!40}\href{../works/Rodriguez07.pdf}{Rodriguez07} (0.83)& \cellcolor{red!40}\href{../works/RodriguezDG02.pdf}{RodriguezDG02} (0.80)& \cellcolor{red!40}\href{../works/CappartS17.pdf}{CappartS17} (0.75)& \cellcolor{red!40}\href{../works/MarliereSPR23.pdf}{MarliereSPR23} (0.69)\\
\index{RoePS05}\href{../works/RoePS05.pdf}{RoePS05} R\&C& \cellcolor{red!40}\href{../works/MaraveliasCG04.pdf}{MaraveliasCG04} (0.68)& \cellcolor{red!40}\href{../works/HarjunkoskiG02.pdf}{HarjunkoskiG02} (0.75)& \cellcolor{red!40}\href{../works/BockmayrP06.pdf}{BockmayrP06} (0.84)& \cellcolor{red!40}\href{../works/ZeballosM09.pdf}{ZeballosM09} (0.84)& \cellcolor{red!20}\href{../works/JainG01.pdf}{JainG01} (0.87)\\
Euclid& \cellcolor{black!20}\href{../works/WikarekS19.pdf}{WikarekS19} (0.35)& \cellcolor{black!20}\href{../works/Goltz95.pdf}{Goltz95} (0.37)& \href{../works/DincbasSH90.pdf}{DincbasSH90} (0.38)& \href{../works/Bartak02.pdf}{Bartak02} (0.38)& \href{../works/Zhou96.pdf}{Zhou96} (0.38)\\
Dot& \cellcolor{red!40}\href{../works/Malapert11.pdf}{Malapert11} (156.00)& \cellcolor{red!40}\href{../works/Baptiste02.pdf}{Baptiste02} (150.00)& \cellcolor{red!40}\href{../works/ZarandiASC20.pdf}{ZarandiASC20} (143.00)& \cellcolor{red!40}\href{../works/Fahimi16.pdf}{Fahimi16} (135.00)& \cellcolor{red!40}\href{../works/HarjunkoskiMBC14.pdf}{HarjunkoskiMBC14} (132.00)\\
Cosine& \cellcolor{red!40}\href{../works/WikarekS19.pdf}{WikarekS19} (0.72)& \cellcolor{red!40}\href{../works/Goltz95.pdf}{Goltz95} (0.68)& \cellcolor{red!40}\href{../works/TrentesauxPT01.pdf}{TrentesauxPT01} (0.68)& \cellcolor{red!40}\href{../works/AggounB93.pdf}{AggounB93} (0.68)& \cellcolor{red!40}\href{../works/GrimesH10.pdf}{GrimesH10} (0.67)\\
\index{RoshanaeiBAUB20}\href{../works/RoshanaeiBAUB20.pdf}{RoshanaeiBAUB20} R\&C& \cellcolor{red!40}\href{../works/RoshanaeiN21.pdf}{RoshanaeiN21} (0.55)& \cellcolor{red!40}RoshanaeiLAU17a (0.66)& \cellcolor{red!40}NaderiRBAU21 (0.67)& \cellcolor{red!40}\href{../works/RoshanaeiLAU17.pdf}{RoshanaeiLAU17} (0.69)& \cellcolor{red!40}\href{../works/RiiseML16.pdf}{RiiseML16} (0.75)\\
Euclid& \cellcolor{blue!20}\href{../works/RoshanaeiN21.pdf}{RoshanaeiN21} (0.32)& \cellcolor{blue!20}\href{../works/GhandehariK22.pdf}{GhandehariK22} (0.32)& \cellcolor{blue!20}\href{../works/NaderiBZR23.pdf}{NaderiBZR23} (0.32)& \cellcolor{blue!20}\href{../works/DoulabiRP16.pdf}{DoulabiRP16} (0.33)& \cellcolor{blue!20}\href{../works/RoshanaeiLAU17.pdf}{RoshanaeiLAU17} (0.33)\\
Dot& \cellcolor{red!40}\href{../works/ZarandiASC20.pdf}{ZarandiASC20} (131.00)& \cellcolor{red!40}\href{../works/Groleaz21.pdf}{Groleaz21} (129.00)& \cellcolor{red!40}\href{../works/NaderiRR23.pdf}{NaderiRR23} (127.00)& \cellcolor{red!40}\href{../works/RoshanaeiN21.pdf}{RoshanaeiN21} (126.00)& \cellcolor{red!40}\href{../works/NaderiBZ23.pdf}{NaderiBZ23} (124.00)\\
Cosine& \cellcolor{red!40}\href{../works/RoshanaeiN21.pdf}{RoshanaeiN21} (0.79)& \cellcolor{red!40}\href{../works/NaderiBZR23.pdf}{NaderiBZR23} (0.78)& \cellcolor{red!40}\href{../works/RoshanaeiLAU17.pdf}{RoshanaeiLAU17} (0.76)& \cellcolor{red!40}\href{../works/GhandehariK22.pdf}{GhandehariK22} (0.76)& \cellcolor{red!40}\href{../works/RiiseML16.pdf}{RiiseML16} (0.74)\\
\index{RoshanaeiLAU17}\href{../works/RoshanaeiLAU17.pdf}{RoshanaeiLAU17} R\&C& \cellcolor{red!40}RoshanaeiLAU17a (0.56)& \cellcolor{red!40}\href{../works/RoshanaeiBAUB20.pdf}{RoshanaeiBAUB20} (0.69)& \cellcolor{red!40}\href{../works/RiiseML16.pdf}{RiiseML16} (0.69)& \cellcolor{red!40}\href{../works/DoulabiRP16.pdf}{DoulabiRP16} (0.70)& \cellcolor{red!40}ZarandiB12 (0.75)\\
Euclid& \cellcolor{green!20}\href{../works/DoulabiRP16.pdf}{DoulabiRP16} (0.31)& \cellcolor{blue!20}\href{../works/RoshanaeiBAUB20.pdf}{RoshanaeiBAUB20} (0.33)& \cellcolor{blue!20}\href{../works/GhandehariK22.pdf}{GhandehariK22} (0.34)& \cellcolor{black!20}\href{../works/NaderiBZR23.pdf}{NaderiBZR23} (0.34)& \cellcolor{black!20}\href{../works/GurEA19.pdf}{GurEA19} (0.35)\\
Dot& \cellcolor{red!40}\href{../works/ZarandiASC20.pdf}{ZarandiASC20} (149.00)& \cellcolor{red!40}\href{../works/Groleaz21.pdf}{Groleaz21} (134.00)& \cellcolor{red!40}\href{../works/NaderiRR23.pdf}{NaderiRR23} (127.00)& \cellcolor{red!40}\href{../works/NaderiBZ23.pdf}{NaderiBZ23} (123.00)& \cellcolor{red!40}\href{../works/RoshanaeiBAUB20.pdf}{RoshanaeiBAUB20} (121.00)\\
Cosine& \cellcolor{red!40}\href{../works/DoulabiRP16.pdf}{DoulabiRP16} (0.77)& \cellcolor{red!40}\href{../works/RoshanaeiBAUB20.pdf}{RoshanaeiBAUB20} (0.76)& \cellcolor{red!40}\href{../works/NaderiBZR23.pdf}{NaderiBZR23} (0.75)& \cellcolor{red!40}\href{../works/GhandehariK22.pdf}{GhandehariK22} (0.73)& \cellcolor{red!40}\href{../works/ElciOH22.pdf}{ElciOH22} (0.71)\\
\index{RoshanaeiLAU17a}RoshanaeiLAU17a R\&C& \cellcolor{red!40}\href{../works/RoshanaeiLAU17.pdf}{RoshanaeiLAU17} (0.56)& \cellcolor{red!40}\href{../works/RoshanaeiBAUB20.pdf}{RoshanaeiBAUB20} (0.66)& \cellcolor{red!40}\href{../works/DoulabiRP16.pdf}{DoulabiRP16} (0.70)& \cellcolor{red!40}\href{../works/RiiseML16.pdf}{RiiseML16} (0.75)& \cellcolor{red!40}\href{../works/RoshanaeiN21.pdf}{RoshanaeiN21} (0.77)\\
Euclid\\
Dot\\
Cosine\\
\index{RoshanaeiN21}\href{../works/RoshanaeiN21.pdf}{RoshanaeiN21} R\&C& \cellcolor{red!40}\href{../works/RoshanaeiBAUB20.pdf}{RoshanaeiBAUB20} (0.55)& \cellcolor{red!40}NaderiRBAU21 (0.64)& \cellcolor{red!40}\href{../works/RoshanaeiLAU17.pdf}{RoshanaeiLAU17} (0.77)& \cellcolor{red!40}RoshanaeiLAU17a (0.77)& \cellcolor{red!40}MartnezAJ22 (0.80)\\
Euclid& \cellcolor{blue!20}\href{../works/RoshanaeiBAUB20.pdf}{RoshanaeiBAUB20} (0.32)& \cellcolor{blue!20}\href{../works/NaderiBZ23.pdf}{NaderiBZ23} (0.33)& \cellcolor{blue!20}\href{../works/NaderiBZ22.pdf}{NaderiBZ22} (0.33)& \cellcolor{black!20}\href{../works/GurEA19.pdf}{GurEA19} (0.36)& \href{../works/GhandehariK22.pdf}{GhandehariK22} (0.37)\\
Dot& \cellcolor{red!40}\href{../works/NaderiRR23.pdf}{NaderiRR23} (142.00)& \cellcolor{red!40}\href{../works/NaderiBZ23.pdf}{NaderiBZ23} (142.00)& \cellcolor{red!40}\href{../works/NaderiBZ22.pdf}{NaderiBZ22} (138.00)& \cellcolor{red!40}\href{../works/Groleaz21.pdf}{Groleaz21} (136.00)& \cellcolor{red!40}\href{../works/RoshanaeiBAUB20.pdf}{RoshanaeiBAUB20} (126.00)\\
Cosine& \cellcolor{red!40}\href{../works/NaderiBZ23.pdf}{NaderiBZ23} (0.79)& \cellcolor{red!40}\href{../works/NaderiBZ22.pdf}{NaderiBZ22} (0.79)& \cellcolor{red!40}\href{../works/RoshanaeiBAUB20.pdf}{RoshanaeiBAUB20} (0.79)& \cellcolor{red!40}\href{../works/RoshanaeiLAU17.pdf}{RoshanaeiLAU17} (0.69)& \cellcolor{red!40}\href{../works/GurEA19.pdf}{GurEA19} (0.69)\\
\index{RossiTHP07}\href{../works/RossiTHP07.pdf}{RossiTHP07} R\&C& \cellcolor{green!20}\href{../works/HoundjiSWD14.pdf}{HoundjiSWD14} (0.94)& \cellcolor{green!20}\href{../works/DejemeppeD14.pdf}{DejemeppeD14} (0.94)& \cellcolor{green!20}\href{../works/SerraNM12.pdf}{SerraNM12} (0.94)& \cellcolor{green!20}\href{../works/KhemmoudjPB06.pdf}{KhemmoudjPB06} (0.94)& \cellcolor{green!20}\href{../works/NishikawaSTT18a.pdf}{NishikawaSTT18a} (0.95)\\
Euclid& \cellcolor{yellow!20}\href{../works/SunLYL10.pdf}{SunLYL10} (0.27)& \cellcolor{yellow!20}\href{../works/KhemmoudjPB06.pdf}{KhemmoudjPB06} (0.28)& \cellcolor{yellow!20}\href{../works/GomesHS06.pdf}{GomesHS06} (0.28)& \cellcolor{yellow!20}\href{../works/ChapadosJR11.pdf}{ChapadosJR11} (0.28)& \cellcolor{yellow!20}\href{../works/Davis87.pdf}{Davis87} (0.28)\\
Dot& \cellcolor{red!40}\href{../works/Astrand21.pdf}{Astrand21} (64.00)& \cellcolor{red!40}\href{../works/HarjunkoskiMBC14.pdf}{HarjunkoskiMBC14} (64.00)& \cellcolor{red!40}\href{../works/ZarandiASC20.pdf}{ZarandiASC20} (63.00)& \cellcolor{red!40}\href{../works/Groleaz21.pdf}{Groleaz21} (62.00)& \cellcolor{red!40}\href{../works/Lombardi10.pdf}{Lombardi10} (61.00)\\
Cosine& \cellcolor{red!40}\href{../works/KhemmoudjPB06.pdf}{KhemmoudjPB06} (0.62)& \cellcolor{red!40}\href{../works/LozanoCDS12.pdf}{LozanoCDS12} (0.60)& \cellcolor{red!40}\href{../works/SunLYL10.pdf}{SunLYL10} (0.58)& \cellcolor{red!40}\href{../works/LombardiMRB10.pdf}{LombardiMRB10} (0.57)& \cellcolor{red!40}\href{../works/BridiBLMB16.pdf}{BridiBLMB16} (0.57)\\
\index{RoweJCA96}\href{../works/RoweJCA96.pdf}{RoweJCA96} R\&C\\
Euclid& \cellcolor{red!40}\href{../works/Hunsberger08.pdf}{Hunsberger08} (0.21)& \cellcolor{red!40}\href{../works/LiuJ06.pdf}{LiuJ06} (0.22)& \cellcolor{red!40}\href{../works/Davis87.pdf}{Davis87} (0.22)& \cellcolor{red!40}\href{../works/SunLYL10.pdf}{SunLYL10} (0.23)& \cellcolor{red!40}\href{../works/FukunagaHFAMN02.pdf}{FukunagaHFAMN02} (0.23)\\
Dot& \cellcolor{red!40}\href{../works/Lombardi10.pdf}{Lombardi10} (70.00)& \cellcolor{red!40}\href{../works/Astrand21.pdf}{Astrand21} (62.00)& \cellcolor{red!40}\href{../works/Groleaz21.pdf}{Groleaz21} (62.00)& \cellcolor{red!40}\href{../works/ZarandiASC20.pdf}{ZarandiASC20} (61.00)& \cellcolor{red!40}\href{../works/Schutt11.pdf}{Schutt11} (60.00)\\
Cosine& \cellcolor{red!40}\href{../works/AstrandJZ18.pdf}{AstrandJZ18} (0.71)& \cellcolor{red!40}\href{../works/Hunsberger08.pdf}{Hunsberger08} (0.69)& \cellcolor{red!40}\href{../works/LozanoCDS12.pdf}{LozanoCDS12} (0.68)& \cellcolor{red!40}\href{../works/BorghesiBLMB18.pdf}{BorghesiBLMB18} (0.68)& \cellcolor{red!40}\href{../works/SunLYL10.pdf}{SunLYL10} (0.67)\\
\index{RuggieroBBMA09}\href{../works/RuggieroBBMA09.pdf}{RuggieroBBMA09} R\&C& \cellcolor{red!40}\href{../works/BeniniBGM06.pdf}{BeniniBGM06} (0.84)& \cellcolor{red!20}\href{../works/Hooker05.pdf}{Hooker05} (0.87)& \cellcolor{red!20}\href{../works/ChuX05.pdf}{ChuX05} (0.88)& \cellcolor{red!20}\href{../works/BeniniBGM05.pdf}{BeniniBGM05} (0.88)& \cellcolor{red!20}\href{../works/CambazardJ05.pdf}{CambazardJ05} (0.89)\\
Euclid& \cellcolor{yellow!20}\href{../works/BeniniBGM06.pdf}{BeniniBGM06} (0.27)& \cellcolor{yellow!20}\href{../works/BeniniBGM05.pdf}{BeniniBGM05} (0.27)& \cellcolor{green!20}\href{../works/BeniniLMR08.pdf}{BeniniLMR08} (0.31)& \cellcolor{blue!20}\href{../works/LozanoCDS12.pdf}{LozanoCDS12} (0.33)& \cellcolor{blue!20}\href{../works/BeniniBGM05a.pdf}{BeniniBGM05a} (0.33)\\
Dot& \cellcolor{red!40}\href{../works/Lombardi10.pdf}{Lombardi10} (141.00)& \cellcolor{red!40}\href{../works/ZarandiASC20.pdf}{ZarandiASC20} (114.00)& \cellcolor{red!40}\href{../works/LombardiMRB10.pdf}{LombardiMRB10} (114.00)& \cellcolor{red!40}\href{../works/BeniniBGM05.pdf}{BeniniBGM05} (111.00)& \cellcolor{red!40}\href{../works/Astrand21.pdf}{Astrand21} (110.00)\\
Cosine& \cellcolor{red!40}\href{../works/BeniniBGM06.pdf}{BeniniBGM06} (0.81)& \cellcolor{red!40}\href{../works/BeniniBGM05.pdf}{BeniniBGM05} (0.81)& \cellcolor{red!40}\href{../works/BeniniLMR08.pdf}{BeniniLMR08} (0.74)& \cellcolor{red!40}\href{../works/LombardiMRB10.pdf}{LombardiMRB10} (0.74)& \cellcolor{red!40}\href{../works/BeniniLMR11.pdf}{BeniniLMR11} (0.70)\\
\index{RussellU06}\href{../works/RussellU06.pdf}{RussellU06} R\&C& \cellcolor{red!40}\href{../works/HenzMT04.pdf}{HenzMT04} (0.71)& \cellcolor{red!40}\href{../works/Trick03.pdf}{Trick03} (0.73)& \cellcolor{red!40}\href{../works/RasmussenT06.pdf}{RasmussenT06} (0.77)& \cellcolor{red!40}\href{../works/RasmussenT07.pdf}{RasmussenT07} (0.78)& \cellcolor{red!40}Henz01 (0.79)\\
Euclid& \cellcolor{red!20}\href{../works/HenzMT04.pdf}{HenzMT04} (0.26)& \cellcolor{yellow!20}\href{../works/Trick03.pdf}{Trick03} (0.26)& \cellcolor{yellow!20}\href{../works/RasmussenT06.pdf}{RasmussenT06} (0.26)& \cellcolor{yellow!20}\href{../works/EastonNT02.pdf}{EastonNT02} (0.26)& \cellcolor{yellow!20}\href{../works/SuCC13.pdf}{SuCC13} (0.28)\\
Dot& \cellcolor{red!40}\href{../works/KanetAG04.pdf}{KanetAG04} (87.00)& \cellcolor{red!40}\href{../works/Lemos21.pdf}{Lemos21} (87.00)& \cellcolor{red!40}\href{../works/KendallKRU10.pdf}{KendallKRU10} (87.00)& \cellcolor{red!40}\href{../works/ZarandiASC20.pdf}{ZarandiASC20} (86.00)& \cellcolor{red!40}\href{../works/Dejemeppe16.pdf}{Dejemeppe16} (79.00)\\
Cosine& \cellcolor{red!40}\href{../works/HenzMT04.pdf}{HenzMT04} (0.74)& \cellcolor{red!40}\href{../works/Trick03.pdf}{Trick03} (0.74)& \cellcolor{red!40}\href{../works/RasmussenT06.pdf}{RasmussenT06} (0.74)& \cellcolor{red!40}\href{../works/EastonNT02.pdf}{EastonNT02} (0.73)& \cellcolor{red!40}\href{../works/LarsonJC14.pdf}{LarsonJC14} (0.72)\\
\index{SacramentoSP20}\href{../works/SacramentoSP20.pdf}{SacramentoSP20} R\&C& \cellcolor{red!40}Pinarbasi21 (0.80)& \cellcolor{red!40}\href{../works/MengLZB21.pdf}{MengLZB21} (0.83)& \cellcolor{red!40}\href{../works/AbreuAPNM21.pdf}{AbreuAPNM21} (0.83)& \cellcolor{yellow!20}KizilayC20 (0.91)& \cellcolor{yellow!20}\href{../works/LunardiBLRV20.pdf}{LunardiBLRV20} (0.92)\\
Euclid& \href{../works/MalapertCGJLR13.pdf}{MalapertCGJLR13} (0.41)& \href{../works/UnsalO13.pdf}{UnsalO13} (0.43)& \href{../works/GedikKBR17.pdf}{GedikKBR17} (0.43)& \href{../works/HubnerGSV21.pdf}{HubnerGSV21} (0.44)& \href{../works/AfsarVPG23.pdf}{AfsarVPG23} (0.44)\\
Dot& \cellcolor{red!40}\href{../works/Groleaz21.pdf}{Groleaz21} (204.00)& \cellcolor{red!40}\href{../works/ZarandiASC20.pdf}{ZarandiASC20} (190.00)& \cellcolor{red!40}\href{../works/Lunardi20.pdf}{Lunardi20} (188.00)& \cellcolor{red!40}\href{../works/Astrand21.pdf}{Astrand21} (175.00)& \cellcolor{red!40}\href{../works/NaderiRR23.pdf}{NaderiRR23} (172.00)\\
Cosine& \cellcolor{red!40}\href{../works/UnsalO13.pdf}{UnsalO13} (0.69)& \cellcolor{red!40}\href{../works/MalapertCGJLR13.pdf}{MalapertCGJLR13} (0.69)& \cellcolor{red!40}\href{../works/AfsarVPG23.pdf}{AfsarVPG23} (0.68)& \cellcolor{red!40}\href{../works/AbreuPNF23.pdf}{AbreuPNF23} (0.68)& \cellcolor{red!40}\href{../works/HubnerGSV21.pdf}{HubnerGSV21} (0.67)\\
\index{SadehF96}\href{../works/SadehF96.pdf}{SadehF96} R\&C& \cellcolor{red!20}\href{../works/BeckF00.pdf}{BeckF00} (0.87)& \cellcolor{red!20}\href{../works/MintonJPL92.pdf}{MintonJPL92} (0.89)& \cellcolor{red!20}\href{../works/BartakSR10.pdf}{BartakSR10} (0.90)& \cellcolor{yellow!20}\href{../works/Wallace96.pdf}{Wallace96} (0.93)& \cellcolor{yellow!20}\href{../works/JussienL02.pdf}{JussienL02} (0.93)\\
Euclid& \cellcolor{red!20}\href{../works/OddiS97.pdf}{OddiS97} (0.26)& \cellcolor{red!20}\href{../works/CestaOF99.pdf}{CestaOF99} (0.26)& \cellcolor{yellow!20}\href{../works/SmithC93.pdf}{SmithC93} (0.27)& \cellcolor{yellow!20}\href{../works/CestaOS00.pdf}{CestaOS00} (0.27)& \cellcolor{yellow!20}\href{../works/BeckDF97.pdf}{BeckDF97} (0.28)\\
Dot& \cellcolor{red!40}\href{../works/ZarandiASC20.pdf}{ZarandiASC20} (150.00)& \cellcolor{red!40}\href{../works/Beck99.pdf}{Beck99} (145.00)& \cellcolor{red!40}\href{../works/Groleaz21.pdf}{Groleaz21} (144.00)& \cellcolor{red!40}\href{../works/Lombardi10.pdf}{Lombardi10} (142.00)& \cellcolor{red!40}\href{../works/Baptiste02.pdf}{Baptiste02} (141.00)\\
Cosine& \cellcolor{red!40}\href{../works/OddiS97.pdf}{OddiS97} (0.83)& \cellcolor{red!40}\href{../works/CestaOF99.pdf}{CestaOF99} (0.82)& \cellcolor{red!40}\href{../works/SmithC93.pdf}{SmithC93} (0.81)& \cellcolor{red!40}\href{../works/CestaOS00.pdf}{CestaOS00} (0.80)& \cellcolor{red!40}\href{../works/BeckF00.pdf}{BeckF00} (0.80)\\
\index{Sadykov04}\href{../works/Sadykov04.pdf}{Sadykov04} R\&C& \cellcolor{red!40}\href{../works/SadykovW06.pdf}{SadykovW06} (0.66)& \cellcolor{red!40}\href{../works/HamdiL13.pdf}{HamdiL13} (0.79)& \cellcolor{red!40}\href{../works/ChuX05.pdf}{ChuX05} (0.82)& \cellcolor{red!40}\href{../works/Beck10.pdf}{Beck10} (0.82)& \cellcolor{red!40}\href{../works/CireCH13.pdf}{CireCH13} (0.82)\\
Euclid& \cellcolor{red!40}\href{../works/SadykovW06.pdf}{SadykovW06} (0.19)& \cellcolor{red!40}\href{../works/Limtanyakul07.pdf}{Limtanyakul07} (0.22)& \cellcolor{red!20}\href{../works/BenediktSMVH18.pdf}{BenediktSMVH18} (0.25)& \cellcolor{red!20}\href{../works/Colombani96.pdf}{Colombani96} (0.25)& \cellcolor{red!20}\href{../works/MakMS10.pdf}{MakMS10} (0.25)\\
Dot& \cellcolor{red!40}\href{../works/Baptiste02.pdf}{Baptiste02} (96.00)& \cellcolor{red!40}\href{../works/Groleaz21.pdf}{Groleaz21} (93.00)& \cellcolor{red!40}\href{../works/ZarandiASC20.pdf}{ZarandiASC20} (87.00)& \cellcolor{red!40}\href{../works/BartakSR10.pdf}{BartakSR10} (83.00)& \cellcolor{red!40}\href{../works/Dejemeppe16.pdf}{Dejemeppe16} (82.00)\\
Cosine& \cellcolor{red!40}\href{../works/SadykovW06.pdf}{SadykovW06} (0.87)& \cellcolor{red!40}\href{../works/Limtanyakul07.pdf}{Limtanyakul07} (0.76)& \cellcolor{red!40}\href{../works/Colombani96.pdf}{Colombani96} (0.75)& \cellcolor{red!40}\href{../works/MakMS10.pdf}{MakMS10} (0.71)& \cellcolor{red!40}\href{../works/AkkerDH07.pdf}{AkkerDH07} (0.71)\\
\index{SadykovW06}\href{../works/SadykovW06.pdf}{SadykovW06} R\&C& \cellcolor{red!40}\href{../works/Sadykov04.pdf}{Sadykov04} (0.66)& \cellcolor{red!40}\href{../works/BockmayrP06.pdf}{BockmayrP06} (0.82)& \cellcolor{red!40}\href{../works/Hooker05.pdf}{Hooker05} (0.83)& \cellcolor{red!40}\href{../works/ChuX05.pdf}{ChuX05} (0.86)& \cellcolor{red!20}\href{../works/Beck10.pdf}{Beck10} (0.87)\\
Euclid& \cellcolor{red!40}\href{../works/Sadykov04.pdf}{Sadykov04} (0.19)& \cellcolor{yellow!20}\href{../works/BenediktSMVH18.pdf}{BenediktSMVH18} (0.26)& \cellcolor{green!20}\href{../works/CatusseCBL16.pdf}{CatusseCBL16} (0.29)& \cellcolor{green!20}\href{../works/Limtanyakul07.pdf}{Limtanyakul07} (0.29)& \cellcolor{green!20}\href{../works/ChuX05.pdf}{ChuX05} (0.31)\\
Dot& \cellcolor{red!40}\href{../works/Baptiste02.pdf}{Baptiste02} (103.00)& \cellcolor{red!40}\href{../works/Groleaz21.pdf}{Groleaz21} (98.00)& \cellcolor{red!40}\href{../works/ZarandiASC20.pdf}{ZarandiASC20} (90.00)& \cellcolor{red!40}\href{../works/BartakSR10.pdf}{BartakSR10} (84.00)& \cellcolor{red!40}\href{../works/BlazewiczDP96.pdf}{BlazewiczDP96} (84.00)\\
Cosine& \cellcolor{red!40}\href{../works/Sadykov04.pdf}{Sadykov04} (0.87)& \cellcolor{red!40}\href{../works/BenediktSMVH18.pdf}{BenediktSMVH18} (0.72)& \cellcolor{red!40}\href{../works/CatusseCBL16.pdf}{CatusseCBL16} (0.69)& \cellcolor{red!40}\href{../works/AkkerDH07.pdf}{AkkerDH07} (0.69)& \cellcolor{red!40}\href{../works/Limtanyakul07.pdf}{Limtanyakul07} (0.66)\\
\index{SakkoutRW98}SakkoutRW98 R\&C\\
Euclid\\
Dot\\
Cosine\\
\index{SakkoutW00}\href{../works/SakkoutW00.pdf}{SakkoutW00} R\&C& \cellcolor{green!20}\href{../works/KamarainenS02.pdf}{KamarainenS02} (0.95)& \cellcolor{green!20}\href{../works/EreminW01.pdf}{EreminW01} (0.96)& \cellcolor{blue!20}\href{../works/JussienL02.pdf}{JussienL02} (0.96)& \cellcolor{blue!20}\href{../works/ElkhyariGJ02a.pdf}{ElkhyariGJ02a} (0.96)& \cellcolor{blue!20}\href{../works/Hooker04.pdf}{Hooker04} (0.96)\\
Euclid& \cellcolor{green!20}\href{../works/KamarainenS02.pdf}{KamarainenS02} (0.31)& \cellcolor{blue!20}\href{../works/HeckmanB11.pdf}{HeckmanB11} (0.33)& \cellcolor{blue!20}\href{../works/Beck07.pdf}{Beck07} (0.33)& \cellcolor{blue!20}\href{../works/BeckF00a.pdf}{BeckF00a} (0.34)& \cellcolor{black!20}\href{../works/KovacsV04.pdf}{KovacsV04} (0.34)\\
Dot& \cellcolor{red!40}\href{../works/Lombardi10.pdf}{Lombardi10} (146.00)& \cellcolor{red!40}\href{../works/Baptiste02.pdf}{Baptiste02} (145.00)& \cellcolor{red!40}\href{../works/Fahimi16.pdf}{Fahimi16} (145.00)& \cellcolor{red!40}\href{../works/Beck99.pdf}{Beck99} (144.00)& \cellcolor{red!40}\href{../works/Malapert11.pdf}{Malapert11} (144.00)\\
Cosine& \cellcolor{red!40}\href{../works/KamarainenS02.pdf}{KamarainenS02} (0.75)& \cellcolor{red!40}\href{../works/HeckmanB11.pdf}{HeckmanB11} (0.74)& \cellcolor{red!40}\href{../works/Beck07.pdf}{Beck07} (0.73)& \cellcolor{red!40}\href{../works/BeckF00.pdf}{BeckF00} (0.73)& \cellcolor{red!40}\href{../works/BartakSR08.pdf}{BartakSR08} (0.73)\\
\index{Salido10}\href{../works/Salido10.pdf}{Salido10} R\&C& \cellcolor{red!40}\href{../works/BartakS11.pdf}{BartakS11} (0.75)& \cellcolor{red!40}\href{../works/Colombani96.pdf}{Colombani96} (0.83)& \cellcolor{red!40}\href{../works/NuijtenA96.pdf}{NuijtenA96} (0.85)& \cellcolor{red!40}\href{../works/Rodriguez07.pdf}{Rodriguez07} (0.86)& \cellcolor{red!20}\href{../works/Zhou96.pdf}{Zhou96} (0.88)\\
Euclid& \cellcolor{red!20}\href{../works/BartakS11.pdf}{BartakS11} (0.24)& \cellcolor{red!20}\href{../works/KengY89.pdf}{KengY89} (0.25)& \cellcolor{red!20}\href{../works/BocewiczBB09.pdf}{BocewiczBB09} (0.25)& \cellcolor{red!20}\href{../works/Prosser89.pdf}{Prosser89} (0.26)& \cellcolor{red!20}\href{../works/FukunagaHFAMN02.pdf}{FukunagaHFAMN02} (0.26)\\
Dot& \cellcolor{red!40}\href{../works/ZarandiASC20.pdf}{ZarandiASC20} (122.00)& \cellcolor{red!40}\href{../works/Dejemeppe16.pdf}{Dejemeppe16} (112.00)& \cellcolor{red!40}\href{../works/Beck99.pdf}{Beck99} (103.00)& \cellcolor{red!40}\href{../works/Lombardi10.pdf}{Lombardi10} (101.00)& \cellcolor{red!40}\href{../works/Astrand21.pdf}{Astrand21} (98.00)\\
Cosine& \cellcolor{red!40}\href{../works/BartakS11.pdf}{BartakS11} (0.76)& \cellcolor{red!40}\href{../works/BocewiczBB09.pdf}{BocewiczBB09} (0.75)& \cellcolor{red!40}\href{../works/KengY89.pdf}{KengY89} (0.73)& \cellcolor{red!40}\href{../works/Prosser89.pdf}{Prosser89} (0.73)& \cellcolor{red!40}\href{../works/ZarandiKS16.pdf}{ZarandiKS16} (0.72)\\
\index{Schaerf96}Schaerf96 R\&C\\
Euclid\\
Dot\\
Cosine\\
\index{Schaerf97}\href{../works/Schaerf97.pdf}{Schaerf97} R\&C\\
Euclid& \cellcolor{red!40}\href{../works/LimAHO02a.pdf}{LimAHO02a} (0.19)& \cellcolor{red!40}\href{../works/EastonNT02.pdf}{EastonNT02} (0.22)& \cellcolor{red!40}\href{../works/JelinekB16.pdf}{JelinekB16} (0.23)& \cellcolor{red!40}\href{../works/YoshikawaKNW94.pdf}{YoshikawaKNW94} (0.23)& \cellcolor{red!40}\href{../works/FrostD98.pdf}{FrostD98} (0.23)\\
Dot& \cellcolor{red!40}\href{../works/ZarandiASC20.pdf}{ZarandiASC20} (53.00)& \cellcolor{red!40}\href{../works/KendallKRU10.pdf}{KendallKRU10} (50.00)& \cellcolor{red!40}\href{../works/RussellU06.pdf}{RussellU06} (46.00)& \cellcolor{red!40}\href{../works/Ribeiro12.pdf}{Ribeiro12} (44.00)& \cellcolor{red!40}\href{../works/KanetAG04.pdf}{KanetAG04} (43.00)\\
Cosine& \cellcolor{red!40}\href{../works/LimAHO02a.pdf}{LimAHO02a} (0.70)& \cellcolor{red!40}\href{../works/EastonNT02.pdf}{EastonNT02} (0.69)& \cellcolor{red!40}\href{../works/RussellU06.pdf}{RussellU06} (0.65)& \cellcolor{red!40}\href{../works/HenzMT04.pdf}{HenzMT04} (0.63)& \cellcolor{red!40}\href{../works/YoshikawaKNW94.pdf}{YoshikawaKNW94} (0.62)\\
\index{SchausD08}\href{../works/SchausD08.pdf}{SchausD08} R\&C\\
Euclid& \cellcolor{yellow!20}\href{../works/BukchinR18.pdf}{BukchinR18} (0.26)& \cellcolor{yellow!20}\href{../works/TopalogluSS12.pdf}{TopalogluSS12} (0.28)& \cellcolor{yellow!20}\href{../works/AlakaPY19.pdf}{AlakaPY19} (0.28)& \cellcolor{green!20}\href{../works/GomesHS06.pdf}{GomesHS06} (0.29)& \cellcolor{green!20}\href{../works/BandaSC11.pdf}{BandaSC11} (0.29)\\
Dot& \cellcolor{red!40}\href{../works/Dejemeppe16.pdf}{Dejemeppe16} (75.00)& \cellcolor{red!40}\href{../works/Godet21a.pdf}{Godet21a} (73.00)& \cellcolor{red!40}\href{../works/Schutt11.pdf}{Schutt11} (70.00)& \cellcolor{red!40}\href{../works/Malapert11.pdf}{Malapert11} (69.00)& \cellcolor{red!40}\href{../works/Lombardi10.pdf}{Lombardi10} (68.00)\\
Cosine& \cellcolor{red!40}\href{../works/TopalogluSS12.pdf}{TopalogluSS12} (0.69)& \cellcolor{red!40}\href{../works/BukchinR18.pdf}{BukchinR18} (0.69)& \cellcolor{red!40}\href{../works/AlakaPY19.pdf}{AlakaPY19} (0.62)& \cellcolor{red!40}\href{../works/PinarbasiAY19.pdf}{PinarbasiAY19} (0.59)& \cellcolor{red!40}\href{../works/LombardiMB13.pdf}{LombardiMB13} (0.56)\\
\index{SchausHMCMD11}\href{../works/SchausHMCMD11.pdf}{SchausHMCMD11} R\&C& \cellcolor{red!40}\href{../works/HentenryckM08.pdf}{HentenryckM08} (0.64)& \cellcolor{red!40}\href{../works/GarganiR07.pdf}{GarganiR07} (0.67)& \cellcolor{red!40}\href{../works/HeinzSSW12.pdf}{HeinzSSW12} (0.69)& \cellcolor{red!40}\href{../works/PerronSF04.pdf}{PerronSF04} (0.82)& \cellcolor{red!40}\href{../works/GaySS14.pdf}{GaySS14} (0.83)\\
Euclid& \cellcolor{yellow!20}\href{../works/HentenryckM08.pdf}{HentenryckM08} (0.27)& \cellcolor{yellow!20}\href{../works/GarganiR07.pdf}{GarganiR07} (0.27)& \cellcolor{green!20}\href{../works/HeinzSSW12.pdf}{HeinzSSW12} (0.29)& \cellcolor{black!20}\href{../works/BandaSC11.pdf}{BandaSC11} (0.34)& \cellcolor{black!20}\href{../works/LiuLH19a.pdf}{LiuLH19a} (0.35)\\
Dot& \cellcolor{red!40}\href{../works/Dejemeppe16.pdf}{Dejemeppe16} (80.00)& \cellcolor{red!40}\href{../works/Lombardi10.pdf}{Lombardi10} (67.00)& \cellcolor{red!40}\href{../works/Siala15a.pdf}{Siala15a} (66.00)& \cellcolor{red!40}\href{../works/GarganiR07.pdf}{GarganiR07} (64.00)& \cellcolor{red!40}\href{../works/Godet21a.pdf}{Godet21a} (64.00)\\
Cosine& \cellcolor{red!40}\href{../works/HentenryckM08.pdf}{HentenryckM08} (0.73)& \cellcolor{red!40}\href{../works/GarganiR07.pdf}{GarganiR07} (0.72)& \cellcolor{red!40}\href{../works/HeinzSSW12.pdf}{HeinzSSW12} (0.67)& \cellcolor{red!40}\href{../works/LiuLH19a.pdf}{LiuLH19a} (0.56)& \cellcolor{red!40}\href{../works/BandaSC11.pdf}{BandaSC11} (0.51)\\
\index{SchildW00}\href{../works/SchildW00.pdf}{SchildW00} R\&C& \cellcolor{green!20}\href{../works/HladikCDJ08.pdf}{HladikCDJ08} (0.95)& \cellcolor{blue!20}\href{../works/BenoistGR02.pdf}{BenoistGR02} (0.97)& \cellcolor{black!20}\href{../works/CambazardHDJT04.pdf}{CambazardHDJT04} (0.98)& \cellcolor{black!20}\href{../works/Thorsteinsson01.pdf}{Thorsteinsson01} (0.99)& \cellcolor{black!20}\href{../works/JainG01.pdf}{JainG01} (1.00)\\
Euclid& \cellcolor{black!20}\href{../works/Zhou96.pdf}{Zhou96} (0.34)& \cellcolor{black!20}\href{../works/HoeveGSL07.pdf}{HoeveGSL07} (0.35)& \cellcolor{black!20}\href{../works/HeipckeCCS00.pdf}{HeipckeCCS00} (0.35)& \cellcolor{black!20}\href{../works/CarlssonKA99.pdf}{CarlssonKA99} (0.35)& \cellcolor{black!20}\href{../works/KengY89.pdf}{KengY89} (0.35)\\
Dot& \cellcolor{red!40}\href{../works/Lombardi10.pdf}{Lombardi10} (137.00)& \cellcolor{red!40}\href{../works/Groleaz21.pdf}{Groleaz21} (135.00)& \cellcolor{red!40}\href{../works/Malapert11.pdf}{Malapert11} (132.00)& \cellcolor{red!40}\href{../works/Astrand21.pdf}{Astrand21} (129.00)& \cellcolor{red!40}\href{../works/Fahimi16.pdf}{Fahimi16} (129.00)\\
Cosine& \cellcolor{red!40}\href{../works/BartakSR08.pdf}{BartakSR08} (0.69)& \cellcolor{red!40}\href{../works/Zhou96.pdf}{Zhou96} (0.68)& \cellcolor{red!40}\href{../works/CauwelaertDS20.pdf}{CauwelaertDS20} (0.68)& \cellcolor{red!40}\href{../works/HeipckeCCS00.pdf}{HeipckeCCS00} (0.67)& \cellcolor{red!40}\href{../works/FontaineMH16.pdf}{FontaineMH16} (0.67)\\
\index{SchnellH15}\href{../works/SchnellH15.pdf}{SchnellH15} R\&C& \cellcolor{red!40}\href{../works/SchnellH17.pdf}{SchnellH17} (0.61)& \cellcolor{red!40}\href{../works/SzerediS16.pdf}{SzerediS16} (0.79)& \cellcolor{red!40}\href{../works/SchuttFSW13.pdf}{SchuttFSW13} (0.79)& \cellcolor{red!40}\href{../works/KreterSSZ18.pdf}{KreterSSZ18} (0.80)& \cellcolor{red!40}\href{../works/KreterSS15.pdf}{KreterSS15} (0.82)\\
Euclid& \cellcolor{red!40}\href{../works/SchnellH17.pdf}{SchnellH17} (0.20)& \cellcolor{red!20}\href{../works/BofillCSV17.pdf}{BofillCSV17} (0.25)& \cellcolor{green!20}\href{../works/SchuttCSW12.pdf}{SchuttCSW12} (0.29)& \cellcolor{green!20}\href{../works/abs-1009-0347.pdf}{abs-1009-0347} (0.29)& \cellcolor{green!20}\href{../works/SzerediS16.pdf}{SzerediS16} (0.30)\\
Dot& \cellcolor{red!40}\href{../works/Schutt11.pdf}{Schutt11} (146.00)& \cellcolor{red!40}\href{../works/Godet21a.pdf}{Godet21a} (138.00)& \cellcolor{red!40}\href{../works/Caballero19.pdf}{Caballero19} (138.00)& \cellcolor{red!40}\href{../works/Lombardi10.pdf}{Lombardi10} (133.00)& \cellcolor{red!40}\href{../works/Groleaz21.pdf}{Groleaz21} (126.00)\\
Cosine& \cellcolor{red!40}\href{../works/SchnellH17.pdf}{SchnellH17} (0.90)& \cellcolor{red!40}\href{../works/BofillCSV17.pdf}{BofillCSV17} (0.84)& \cellcolor{red!40}\href{../works/abs-1009-0347.pdf}{abs-1009-0347} (0.79)& \cellcolor{red!40}\href{../works/SchuttCSW12.pdf}{SchuttCSW12} (0.78)& \cellcolor{red!40}\href{../works/SzerediS16.pdf}{SzerediS16} (0.78)\\
\index{SchnellH17}\href{../works/SchnellH17.pdf}{SchnellH17} R\&C& \cellcolor{red!40}\href{../works/SchnellH15.pdf}{SchnellH15} (0.61)& \cellcolor{red!40}\href{../works/KreterSSZ18.pdf}{KreterSSZ18} (0.81)& \cellcolor{red!40}\href{../works/KreterSS17.pdf}{KreterSS17} (0.85)& \cellcolor{red!40}\href{../works/SzerediS16.pdf}{SzerediS16} (0.85)& \cellcolor{red!20}\href{../works/SchuttS16.pdf}{SchuttS16} (0.88)\\
Euclid& \cellcolor{red!40}\href{../works/SchnellH15.pdf}{SchnellH15} (0.20)& \cellcolor{yellow!20}\href{../works/BertholdHLMS10.pdf}{BertholdHLMS10} (0.28)& \cellcolor{green!20}\href{../works/BofillCSV17.pdf}{BofillCSV17} (0.29)& \cellcolor{green!20}\href{../works/SchuttCSW12.pdf}{SchuttCSW12} (0.31)& \cellcolor{green!20}\href{../works/BofillCSV17a.pdf}{BofillCSV17a} (0.31)\\
Dot& \cellcolor{red!40}\href{../works/Schutt11.pdf}{Schutt11} (138.00)& \cellcolor{red!40}\href{../works/Caballero19.pdf}{Caballero19} (125.00)& \cellcolor{red!40}\href{../works/SchnellH15.pdf}{SchnellH15} (124.00)& \cellcolor{red!40}\href{../works/Godet21a.pdf}{Godet21a} (123.00)& \cellcolor{red!40}\href{../works/Groleaz21.pdf}{Groleaz21} (121.00)\\
Cosine& \cellcolor{red!40}\href{../works/SchnellH15.pdf}{SchnellH15} (0.90)& \cellcolor{red!40}\href{../works/BertholdHLMS10.pdf}{BertholdHLMS10} (0.77)& \cellcolor{red!40}\href{../works/BofillCSV17.pdf}{BofillCSV17} (0.76)& \cellcolor{red!40}\href{../works/abs-1009-0347.pdf}{abs-1009-0347} (0.74)& \cellcolor{red!40}\href{../works/SchuttFSW13.pdf}{SchuttFSW13} (0.74)\\
\index{Schutt11}\href{../works/Schutt11.pdf}{Schutt11} R\&C\\
Euclid& \href{../works/SchuttFSW11.pdf}{SchuttFSW11} (0.40)& \href{../works/SchuttFS13a.pdf}{SchuttFS13a} (0.45)& \href{../works/FahimiOQ18.pdf}{FahimiOQ18} (0.46)& \href{../works/Caballero19.pdf}{Caballero19} (0.47)& \href{../works/SchuttFSW13.pdf}{SchuttFSW13} (0.49)\\
Dot& \cellcolor{red!40}\href{../works/Dejemeppe16.pdf}{Dejemeppe16} (262.00)& \cellcolor{red!40}\href{../works/Lombardi10.pdf}{Lombardi10} (261.00)& \cellcolor{red!40}\href{../works/Malapert11.pdf}{Malapert11} (259.00)& \cellcolor{red!40}\href{../works/Baptiste02.pdf}{Baptiste02} (255.00)& \cellcolor{red!40}\href{../works/Fahimi16.pdf}{Fahimi16} (240.00)\\
Cosine& \cellcolor{red!40}\href{../works/SchuttFSW11.pdf}{SchuttFSW11} (0.83)& \cellcolor{red!40}\href{../works/SchuttFS13a.pdf}{SchuttFS13a} (0.77)& \cellcolor{red!40}\href{../works/FahimiOQ18.pdf}{FahimiOQ18} (0.76)& \cellcolor{red!40}\href{../works/Caballero19.pdf}{Caballero19} (0.74)& \cellcolor{red!40}\href{../works/SchuttFSW09.pdf}{SchuttFSW09} (0.73)\\
\index{SchuttCSW12}\href{../works/SchuttCSW12.pdf}{SchuttCSW12} R\&C& \cellcolor{red!40}\href{../works/GuSW12.pdf}{GuSW12} (0.59)& \cellcolor{red!40}\href{../works/GuSS13.pdf}{GuSS13} (0.66)& \cellcolor{red!40}\href{../works/KreterSS15.pdf}{KreterSS15} (0.70)& \cellcolor{red!40}GuSSWC14 (0.72)& \cellcolor{red!40}\href{../works/SchuttFSW11.pdf}{SchuttFSW11} (0.78)\\
Euclid& \cellcolor{red!40}\href{../works/GuSW12.pdf}{GuSW12} (0.23)& \cellcolor{red!20}\href{../works/SchuttS16.pdf}{SchuttS16} (0.24)& \cellcolor{red!20}\href{../works/abs-1009-0347.pdf}{abs-1009-0347} (0.25)& \cellcolor{red!20}\href{../works/GuSS13.pdf}{GuSS13} (0.25)& \cellcolor{red!20}\href{../works/BofillCSV17.pdf}{BofillCSV17} (0.26)\\
Dot& \cellcolor{red!40}\href{../works/Schutt11.pdf}{Schutt11} (109.00)& \cellcolor{red!40}\href{../works/SchuttFSW11.pdf}{SchuttFSW11} (107.00)& \cellcolor{red!40}\href{../works/Godet21a.pdf}{Godet21a} (107.00)& \cellcolor{red!40}\href{../works/Lombardi10.pdf}{Lombardi10} (105.00)& \cellcolor{red!40}\href{../works/Caballero19.pdf}{Caballero19} (104.00)\\
Cosine& \cellcolor{red!40}\href{../works/abs-1009-0347.pdf}{abs-1009-0347} (0.82)& \cellcolor{red!40}\href{../works/SchuttS16.pdf}{SchuttS16} (0.82)& \cellcolor{red!40}\href{../works/GuSW12.pdf}{GuSW12} (0.81)& \cellcolor{red!40}\href{../works/SchnellH15.pdf}{SchnellH15} (0.78)& \cellcolor{red!40}\href{../works/SchuttFSW09.pdf}{SchuttFSW09} (0.78)\\
\index{SchuttFS13}\href{../works/SchuttFS13.pdf}{SchuttFS13} R\&C& \cellcolor{red!40}\href{../works/KreterSS15.pdf}{KreterSS15} (0.68)& \cellcolor{red!40}\href{../works/SzerediS16.pdf}{SzerediS16} (0.72)& \cellcolor{red!40}\href{../works/SchuttFS13a.pdf}{SchuttFS13a} (0.76)& \cellcolor{red!40}\href{../works/SchuttCSW12.pdf}{SchuttCSW12} (0.82)& \cellcolor{red!40}\href{../works/SialaAH15.pdf}{SialaAH15} (0.82)\\
Euclid& \cellcolor{yellow!20}\href{../works/SialaAH15.pdf}{SialaAH15} (0.27)& \cellcolor{green!20}\href{../works/FontaineMH16.pdf}{FontaineMH16} (0.31)& \cellcolor{green!20}\href{../works/HentenryckM04.pdf}{HentenryckM04} (0.31)& \cellcolor{blue!20}\href{../works/KovacsV06.pdf}{KovacsV06} (0.33)& \cellcolor{blue!20}\href{../works/HeipckeCCS00.pdf}{HeipckeCCS00} (0.33)\\
Dot& \cellcolor{red!40}\href{../works/Lombardi10.pdf}{Lombardi10} (152.00)& \cellcolor{red!40}\href{../works/Schutt11.pdf}{Schutt11} (149.00)& \cellcolor{red!40}\href{../works/Godet21a.pdf}{Godet21a} (146.00)& \cellcolor{red!40}\href{../works/Siala15a.pdf}{Siala15a} (141.00)& \cellcolor{red!40}\href{../works/Baptiste02.pdf}{Baptiste02} (140.00)\\
Cosine& \cellcolor{red!40}\href{../works/SialaAH15.pdf}{SialaAH15} (0.82)& \cellcolor{red!40}\href{../works/HentenryckM04.pdf}{HentenryckM04} (0.76)& \cellcolor{red!40}\href{../works/FontaineMH16.pdf}{FontaineMH16} (0.75)& \cellcolor{red!40}\href{../works/VilimLS15.pdf}{VilimLS15} (0.75)& \cellcolor{red!40}\href{../works/Bit-Monnot23.pdf}{Bit-Monnot23} (0.74)\\
\index{SchuttFS13a}\href{../works/SchuttFS13a.pdf}{SchuttFS13a} R\&C& \cellcolor{red!40}SchuttFSW15 (0.68)& \cellcolor{red!40}\href{../works/SchuttFSW11.pdf}{SchuttFSW11} (0.68)& \cellcolor{red!40}\href{../works/OuelletQ13.pdf}{OuelletQ13} (0.70)& \cellcolor{red!40}\href{../works/Vilim11.pdf}{Vilim11} (0.70)& \cellcolor{red!40}\href{../works/SchuttW10.pdf}{SchuttW10} (0.70)\\
Euclid& \cellcolor{yellow!20}\href{../works/SchuttFSW11.pdf}{SchuttFSW11} (0.26)& \cellcolor{green!20}\href{../works/SchuttFSW09.pdf}{SchuttFSW09} (0.29)& \cellcolor{green!20}\href{../works/abs-1009-0347.pdf}{abs-1009-0347} (0.29)& \cellcolor{green!20}\href{../works/Vilim11.pdf}{Vilim11} (0.30)& \cellcolor{green!20}\href{../works/OuelletQ18.pdf}{OuelletQ18} (0.31)\\
Dot& \cellcolor{red!40}\href{../works/Schutt11.pdf}{Schutt11} (183.00)& \cellcolor{red!40}\href{../works/Lombardi10.pdf}{Lombardi10} (163.00)& \cellcolor{red!40}\href{../works/SchuttFSW11.pdf}{SchuttFSW11} (162.00)& \cellcolor{red!40}\href{../works/Caballero19.pdf}{Caballero19} (159.00)& \cellcolor{red!40}\href{../works/Baptiste02.pdf}{Baptiste02} (155.00)\\
Cosine& \cellcolor{red!40}\href{../works/SchuttFSW11.pdf}{SchuttFSW11} (0.88)& \cellcolor{red!40}\href{../works/abs-1009-0347.pdf}{abs-1009-0347} (0.81)& \cellcolor{red!40}\href{../works/SchuttFSW09.pdf}{SchuttFSW09} (0.81)& \cellcolor{red!40}\href{../works/Vilim11.pdf}{Vilim11} (0.79)& \cellcolor{red!40}\href{../works/Caballero19.pdf}{Caballero19} (0.79)\\
\index{SchuttFSW09}\href{../works/SchuttFSW09.pdf}{SchuttFSW09} R\&C& \cellcolor{red!40}\href{../works/SchuttFSW11.pdf}{SchuttFSW11} (0.63)& \cellcolor{red!20}\href{../works/HeinzS11.pdf}{HeinzS11} (0.86)& \cellcolor{red!20}\href{../works/SchuttFS13a.pdf}{SchuttFS13a} (0.88)& \cellcolor{red!20}\href{../works/SchuttCSW12.pdf}{SchuttCSW12} (0.88)& \cellcolor{red!20}\href{../works/Vilim09.pdf}{Vilim09} (0.88)\\
Euclid& \cellcolor{red!40}\href{../works/abs-1009-0347.pdf}{abs-1009-0347} (0.23)& \cellcolor{red!20}\href{../works/SchuttFSW13.pdf}{SchuttFSW13} (0.26)& \cellcolor{yellow!20}\href{../works/SchuttFSW11.pdf}{SchuttFSW11} (0.26)& \cellcolor{yellow!20}\href{../works/SchuttCSW12.pdf}{SchuttCSW12} (0.27)& \cellcolor{green!20}\href{../works/SchuttFS13a.pdf}{SchuttFS13a} (0.29)\\
Dot& \cellcolor{red!40}\href{../works/Schutt11.pdf}{Schutt11} (148.00)& \cellcolor{red!40}\href{../works/SchuttFSW11.pdf}{SchuttFSW11} (139.00)& \cellcolor{red!40}\href{../works/Godet21a.pdf}{Godet21a} (136.00)& \cellcolor{red!40}\href{../works/Lombardi10.pdf}{Lombardi10} (131.00)& \cellcolor{red!40}\href{../works/Caballero19.pdf}{Caballero19} (124.00)\\
Cosine& \cellcolor{red!40}\href{../works/SchuttFSW11.pdf}{SchuttFSW11} (0.88)& \cellcolor{red!40}\href{../works/abs-1009-0347.pdf}{abs-1009-0347} (0.85)& \cellcolor{red!40}\href{../works/SchuttFSW13.pdf}{SchuttFSW13} (0.83)& \cellcolor{red!40}\href{../works/SchuttFS13a.pdf}{SchuttFS13a} (0.81)& \cellcolor{red!40}\href{../works/SchuttCSW12.pdf}{SchuttCSW12} (0.78)\\
\index{SchuttFSW11}\href{../works/SchuttFSW11.pdf}{SchuttFSW11} R\&C& \cellcolor{red!40}\href{../works/SchuttFSW09.pdf}{SchuttFSW09} (0.63)& \cellcolor{red!40}\href{../works/SchuttFS13a.pdf}{SchuttFS13a} (0.68)& \cellcolor{red!40}\href{../works/SchuttFSW13.pdf}{SchuttFSW13} (0.71)& \cellcolor{red!40}\href{../works/SchuttW10.pdf}{SchuttW10} (0.75)& \cellcolor{red!40}\href{../works/SchuttCSW12.pdf}{SchuttCSW12} (0.78)\\
Euclid& \cellcolor{yellow!20}\href{../works/SchuttFSW09.pdf}{SchuttFSW09} (0.26)& \cellcolor{yellow!20}\href{../works/SchuttFS13a.pdf}{SchuttFS13a} (0.26)& \cellcolor{green!20}\href{../works/abs-1009-0347.pdf}{abs-1009-0347} (0.31)& \cellcolor{black!20}\href{../works/SchuttFSW13.pdf}{SchuttFSW13} (0.35)& \cellcolor{black!20}\href{../works/LiessM08.pdf}{LiessM08} (0.35)\\
Dot& \cellcolor{red!40}\href{../works/Schutt11.pdf}{Schutt11} (217.00)& \cellcolor{red!40}\href{../works/Lombardi10.pdf}{Lombardi10} (191.00)& \cellcolor{red!40}\href{../works/Baptiste02.pdf}{Baptiste02} (179.00)& \cellcolor{red!40}\href{../works/Dejemeppe16.pdf}{Dejemeppe16} (177.00)& \cellcolor{red!40}\href{../works/Godet21a.pdf}{Godet21a} (176.00)\\
Cosine& \cellcolor{red!40}\href{../works/SchuttFSW09.pdf}{SchuttFSW09} (0.88)& \cellcolor{red!40}\href{../works/SchuttFS13a.pdf}{SchuttFS13a} (0.88)& \cellcolor{red!40}\href{../works/Schutt11.pdf}{Schutt11} (0.83)& \cellcolor{red!40}\href{../works/abs-1009-0347.pdf}{abs-1009-0347} (0.82)& \cellcolor{red!40}\href{../works/SchuttFSW13.pdf}{SchuttFSW13} (0.77)\\
\index{SchuttFSW13}\href{../works/SchuttFSW13.pdf}{SchuttFSW13} R\&C& \cellcolor{red!40}SchuttFSW15 (0.64)& \cellcolor{red!40}\href{../works/SchuttFSW11.pdf}{SchuttFSW11} (0.71)& \cellcolor{red!40}\href{../works/SchuttFS13a.pdf}{SchuttFS13a} (0.77)& \cellcolor{red!40}\href{../works/KreterSS15.pdf}{KreterSS15} (0.78)& \cellcolor{red!40}\href{../works/SchuttCSW12.pdf}{SchuttCSW12} (0.78)\\
Euclid& \cellcolor{red!40}\href{../works/abs-1009-0347.pdf}{abs-1009-0347} (0.19)& \cellcolor{red!20}\href{../works/SchuttFSW09.pdf}{SchuttFSW09} (0.26)& \cellcolor{yellow!20}\href{../works/BofillCSV17a.pdf}{BofillCSV17a} (0.27)& \cellcolor{green!20}\href{../works/SchuttS16.pdf}{SchuttS16} (0.29)& \cellcolor{green!20}\href{../works/SchuttCSW12.pdf}{SchuttCSW12} (0.30)\\
Dot& \cellcolor{red!40}\href{../works/Schutt11.pdf}{Schutt11} (162.00)& \cellcolor{red!40}\href{../works/Caballero19.pdf}{Caballero19} (147.00)& \cellcolor{red!40}\href{../works/Godet21a.pdf}{Godet21a} (146.00)& \cellcolor{red!40}\href{../works/Lombardi10.pdf}{Lombardi10} (141.00)& \cellcolor{red!40}\href{../works/SchuttFSW11.pdf}{SchuttFSW11} (135.00)\\
Cosine& \cellcolor{red!40}\href{../works/abs-1009-0347.pdf}{abs-1009-0347} (0.91)& \cellcolor{red!40}\href{../works/SchuttFSW09.pdf}{SchuttFSW09} (0.83)& \cellcolor{red!40}\href{../works/BofillCSV17a.pdf}{BofillCSV17a} (0.81)& \cellcolor{red!40}\href{../works/KreterSSZ18.pdf}{KreterSSZ18} (0.80)& \cellcolor{red!40}\href{../works/SchuttS16.pdf}{SchuttS16} (0.79)\\
\index{SchuttFSW15}SchuttFSW15 R\&C& \cellcolor{red!40}\href{../works/SchuttFSW13.pdf}{SchuttFSW13} (0.64)& \cellcolor{red!40}\href{../works/SchuttFS13a.pdf}{SchuttFS13a} (0.68)& \cellcolor{red!40}\href{../works/KreterSS15.pdf}{KreterSS15} (0.76)& \cellcolor{red!40}GuSSWC14 (0.80)& \cellcolor{red!40}\href{../works/KreterSS17.pdf}{KreterSS17} (0.80)\\
Euclid\\
Dot\\
Cosine\\
\index{SchuttS16}\href{../works/SchuttS16.pdf}{SchuttS16} R\&C& \cellcolor{red!40}\href{../works/BeldiceanuP07.pdf}{BeldiceanuP07} (0.79)& \cellcolor{red!40}\href{../works/KovacsV06.pdf}{KovacsV06} (0.80)& \cellcolor{red!40}\href{../works/CauwelaertDS20.pdf}{CauwelaertDS20} (0.80)& \cellcolor{red!40}\href{../works/SzerediS16.pdf}{SzerediS16} (0.81)& \cellcolor{red!40}\href{../works/KreterSS17.pdf}{KreterSS17} (0.83)\\
Euclid& \cellcolor{red!20}\href{../works/SchuttCSW12.pdf}{SchuttCSW12} (0.24)& \cellcolor{yellow!20}\href{../works/YoungFS17.pdf}{YoungFS17} (0.27)& \cellcolor{yellow!20}\href{../works/BofillCSV17a.pdf}{BofillCSV17a} (0.27)& \cellcolor{yellow!20}\href{../works/BofillCSV17.pdf}{BofillCSV17} (0.27)& \cellcolor{yellow!20}\href{../works/abs-1009-0347.pdf}{abs-1009-0347} (0.28)\\
Dot& \cellcolor{red!40}\href{../works/Godet21a.pdf}{Godet21a} (127.00)& \cellcolor{red!40}\href{../works/Schutt11.pdf}{Schutt11} (124.00)& \cellcolor{red!40}\href{../works/Lombardi10.pdf}{Lombardi10} (121.00)& \cellcolor{red!40}\href{../works/Caballero19.pdf}{Caballero19} (119.00)& \cellcolor{red!40}\href{../works/Beck99.pdf}{Beck99} (115.00)\\
Cosine& \cellcolor{red!40}\href{../works/SchuttCSW12.pdf}{SchuttCSW12} (0.82)& \cellcolor{red!40}\href{../works/YoungFS17.pdf}{YoungFS17} (0.80)& \cellcolor{red!40}\href{../works/abs-1009-0347.pdf}{abs-1009-0347} (0.80)& \cellcolor{red!40}\href{../works/SchuttFSW13.pdf}{SchuttFSW13} (0.79)& \cellcolor{red!40}\href{../works/SzerediS16.pdf}{SzerediS16} (0.77)\\
\index{SchuttW10}\href{../works/SchuttW10.pdf}{SchuttW10} R\&C& \cellcolor{red!40}\href{../works/Vilim09.pdf}{Vilim09} (0.57)& \cellcolor{red!40}\href{../works/OuelletQ13.pdf}{OuelletQ13} (0.62)& \cellcolor{red!40}\href{../works/KameugneF13.pdf}{KameugneF13} (0.62)& \cellcolor{red!40}\href{../works/Vilim11.pdf}{Vilim11} (0.63)& \cellcolor{red!40}\href{../works/KameugneFSN11.pdf}{KameugneFSN11} (0.63)\\
Euclid& \cellcolor{red!20}\href{../works/SchuttWS05.pdf}{SchuttWS05} (0.25)& \cellcolor{red!20}\href{../works/KameugneFSN11.pdf}{KameugneFSN11} (0.26)& \cellcolor{yellow!20}\href{../works/KameugneFSN14.pdf}{KameugneFSN14} (0.28)& \cellcolor{green!20}\href{../works/OuelletQ18.pdf}{OuelletQ18} (0.29)& \cellcolor{green!20}\href{../works/KameugneFGOQ18.pdf}{KameugneFGOQ18} (0.29)\\
Dot& \cellcolor{red!40}\href{../works/Schutt11.pdf}{Schutt11} (124.00)& \cellcolor{red!40}\href{../works/Baptiste02.pdf}{Baptiste02} (120.00)& \cellcolor{red!40}\href{../works/Fahimi16.pdf}{Fahimi16} (118.00)& \cellcolor{red!40}\href{../works/Dejemeppe16.pdf}{Dejemeppe16} (115.00)& \cellcolor{red!40}\href{../works/Lombardi10.pdf}{Lombardi10} (114.00)\\
Cosine& \cellcolor{red!40}\href{../works/KameugneFSN11.pdf}{KameugneFSN11} (0.81)& \cellcolor{red!40}\href{../works/SchuttWS05.pdf}{SchuttWS05} (0.80)& \cellcolor{red!40}\href{../works/KameugneFSN14.pdf}{KameugneFSN14} (0.80)& \cellcolor{red!40}\href{../works/KameugneFGOQ18.pdf}{KameugneFGOQ18} (0.78)& \cellcolor{red!40}\href{../works/KameugneFND23.pdf}{KameugneFND23} (0.77)\\
\index{SchuttWS05}\href{../works/SchuttWS05.pdf}{SchuttWS05} R\&C& \cellcolor{red!40}\href{../works/WolfS05.pdf}{WolfS05} (0.45)& \cellcolor{red!40}\href{../works/SchuttW10.pdf}{SchuttW10} (0.67)& \cellcolor{red!40}\href{../works/MercierH08.pdf}{MercierH08} (0.76)& \cellcolor{red!40}\href{../works/KameugneFGOQ18.pdf}{KameugneFGOQ18} (0.79)& \cellcolor{red!40}\href{../works/Tesch16.pdf}{Tesch16} (0.80)\\
Euclid& \cellcolor{red!40}\href{../works/BeldiceanuP07.pdf}{BeldiceanuP07} (0.23)& \cellcolor{red!40}\href{../works/PoderB08.pdf}{PoderB08} (0.24)& \cellcolor{red!40}\href{../works/MercierH08.pdf}{MercierH08} (0.24)& \cellcolor{red!20}\href{../works/SchuttW10.pdf}{SchuttW10} (0.25)& \cellcolor{red!20}\href{../works/BockmayrP06.pdf}{BockmayrP06} (0.25)\\
Dot& \cellcolor{red!40}\href{../works/Baptiste02.pdf}{Baptiste02} (104.00)& \cellcolor{red!40}\href{../works/Dejemeppe16.pdf}{Dejemeppe16} (100.00)& \cellcolor{red!40}\href{../works/Beck99.pdf}{Beck99} (97.00)& \cellcolor{red!40}\href{../works/Lombardi10.pdf}{Lombardi10} (96.00)& \cellcolor{red!40}\href{../works/Groleaz21.pdf}{Groleaz21} (96.00)\\
Cosine& \cellcolor{red!40}\href{../works/SchuttW10.pdf}{SchuttW10} (0.80)& \cellcolor{red!40}\href{../works/MercierH08.pdf}{MercierH08} (0.78)& \cellcolor{red!40}\href{../works/BeldiceanuP07.pdf}{BeldiceanuP07} (0.78)& \cellcolor{red!40}\href{../works/PoderB08.pdf}{PoderB08} (0.76)& \cellcolor{red!40}\href{../works/Hooker05a.pdf}{Hooker05a} (0.75)\\
\index{SenderovichBB19}\href{../works/SenderovichBB19.pdf}{SenderovichBB19} R\&C\\
Euclid& \cellcolor{green!20}\href{../works/BeckPS03.pdf}{BeckPS03} (0.30)& \cellcolor{blue!20}\href{../works/PacinoH11.pdf}{PacinoH11} (0.32)& \cellcolor{blue!20}\href{../works/TanSD10.pdf}{TanSD10} (0.33)& \cellcolor{blue!20}\href{../works/VilimBC04.pdf}{VilimBC04} (0.33)& \cellcolor{blue!20}\href{../works/LombardiBM15.pdf}{LombardiBM15} (0.33)\\
Dot& \cellcolor{red!40}\href{../works/Dejemeppe16.pdf}{Dejemeppe16} (155.00)& \cellcolor{red!40}\href{../works/ZarandiASC20.pdf}{ZarandiASC20} (149.00)& \cellcolor{red!40}\href{../works/Lombardi10.pdf}{Lombardi10} (144.00)& \cellcolor{red!40}\href{../works/Lunardi20.pdf}{Lunardi20} (139.00)& \cellcolor{red!40}\href{../works/Astrand21.pdf}{Astrand21} (138.00)\\
Cosine& \cellcolor{red!40}\href{../works/BeckPS03.pdf}{BeckPS03} (0.79)& \cellcolor{red!40}\href{../works/DejemeppeCS15.pdf}{DejemeppeCS15} (0.74)& \cellcolor{red!40}\href{../works/PacinoH11.pdf}{PacinoH11} (0.74)& \cellcolor{red!40}\href{../works/TanSD10.pdf}{TanSD10} (0.74)& \cellcolor{red!40}\href{../works/Wolf05.pdf}{Wolf05} (0.73)\\
\index{SerraNM12}\href{../works/SerraNM12.pdf}{SerraNM12} R\&C& \cellcolor{red!20}\href{../works/DavenportKRSH07.pdf}{DavenportKRSH07} (0.90)& \cellcolor{yellow!20}\href{../works/WatsonB08.pdf}{WatsonB08} (0.92)& \cellcolor{yellow!20}\href{../works/SimoninAHL15.pdf}{SimoninAHL15} (0.92)& \cellcolor{green!20}\href{../works/BeldiceanuP07.pdf}{BeldiceanuP07} (0.93)& \cellcolor{green!20}\href{../works/BeckFW11.pdf}{BeckFW11} (0.94)\\
Euclid& \cellcolor{green!20}\href{../works/ZibranR11a.pdf}{ZibranR11a} (0.30)& \cellcolor{green!20}\href{../works/KhemmoudjPB06.pdf}{KhemmoudjPB06} (0.30)& \cellcolor{green!20}\href{../works/GilesH16.pdf}{GilesH16} (0.30)& \cellcolor{green!20}\href{../works/ZibranR11.pdf}{ZibranR11} (0.31)& \cellcolor{green!20}\href{../works/PoderB08.pdf}{PoderB08} (0.31)\\
Dot& \cellcolor{red!40}\href{../works/Beck99.pdf}{Beck99} (102.00)& \cellcolor{red!40}\href{../works/Lombardi10.pdf}{Lombardi10} (102.00)& \cellcolor{red!40}\href{../works/Dejemeppe16.pdf}{Dejemeppe16} (101.00)& \cellcolor{red!40}\href{../works/Groleaz21.pdf}{Groleaz21} (99.00)& \cellcolor{red!40}\href{../works/LaborieRSV18.pdf}{LaborieRSV18} (97.00)\\
Cosine& \cellcolor{red!40}\href{../works/AalianPG23.pdf}{AalianPG23} (0.69)& \cellcolor{red!40}\href{../works/GoelSHFS15.pdf}{GoelSHFS15} (0.69)& \cellcolor{red!40}\href{../works/KhemmoudjPB06.pdf}{KhemmoudjPB06} (0.68)& \cellcolor{red!40}\href{../works/GilesH16.pdf}{GilesH16} (0.68)& \cellcolor{red!40}\href{../works/ZibranR11a.pdf}{ZibranR11a} (0.67)\\
\index{ShaikhK23}\href{../works/ShaikhK23.pdf}{ShaikhK23} R\&C\\
Euclid& \cellcolor{red!20}\href{../works/PesantGPR99.pdf}{PesantGPR99} (0.26)& \cellcolor{red!20}\href{../works/Puget95.pdf}{Puget95} (0.26)& \cellcolor{red!20}\href{../works/Caseau97.pdf}{Caseau97} (0.26)& \cellcolor{yellow!20}\href{../works/CrawfordB94.pdf}{CrawfordB94} (0.26)& \cellcolor{yellow!20}\href{../works/LauLN08.pdf}{LauLN08} (0.27)\\
Dot& \cellcolor{red!40}\href{../works/ZarandiASC20.pdf}{ZarandiASC20} (97.00)& \cellcolor{red!40}\href{../works/Astrand21.pdf}{Astrand21} (91.00)& \cellcolor{red!40}\href{../works/Beck99.pdf}{Beck99} (88.00)& \cellcolor{red!40}\href{../works/Lemos21.pdf}{Lemos21} (86.00)& \cellcolor{red!40}\href{../works/Groleaz21.pdf}{Groleaz21} (82.00)\\
Cosine& \cellcolor{red!40}\href{../works/Salido10.pdf}{Salido10} (0.68)& \cellcolor{red!40}\href{../works/PesantGPR99.pdf}{PesantGPR99} (0.68)& \cellcolor{red!40}\href{../works/BorghesiBLMB18.pdf}{BorghesiBLMB18} (0.67)& \cellcolor{red!40}\href{../works/Puget95.pdf}{Puget95} (0.66)& \cellcolor{red!40}\href{../works/DemirovicS18.pdf}{DemirovicS18} (0.66)\\
\index{ShiYXQ22}ShiYXQ22 R\&C& \cellcolor{red!40}\href{../works/ZhuSZW23.pdf}{ZhuSZW23} (0.73)& \cellcolor{red!40}\href{../works/ZhangYW21.pdf}{ZhangYW21} (0.84)& \cellcolor{red!40}\href{../works/NaderiBZ22a.pdf}{NaderiBZ22a} (0.85)& \cellcolor{yellow!20}\href{../works/ColT19.pdf}{ColT19} (0.91)& \cellcolor{green!20}\href{../works/LunardiBLRV20.pdf}{LunardiBLRV20} (0.95)\\
Euclid\\
Dot\\
Cosine\\
\index{ShinBBHO18}\href{../works/ShinBBHO18.pdf}{ShinBBHO18} R\&C& \cellcolor{blue!20}\href{../works/BourdaisGP03.pdf}{BourdaisGP03} (0.97)& \cellcolor{blue!20}\href{../works/FrohnerTR19.pdf}{FrohnerTR19} (0.97)& \cellcolor{blue!20}\href{../works/HoYCLLCLC18.pdf}{HoYCLLCLC18} (0.98)& \cellcolor{blue!20}\href{../works/AntunesABD20.pdf}{AntunesABD20} (0.98)& \cellcolor{black!20}\href{../works/Tom19.pdf}{Tom19} (0.98)\\
Euclid& \cellcolor{red!40}\href{../works/HoYCLLCLC18.pdf}{HoYCLLCLC18} (0.24)& \cellcolor{green!20}\href{../works/LuoVLBM16.pdf}{LuoVLBM16} (0.29)& \cellcolor{green!20}\href{../works/WeilHFP95.pdf}{WeilHFP95} (0.30)& \cellcolor{green!20}\href{../works/BourdaisGP03.pdf}{BourdaisGP03} (0.31)& \cellcolor{green!20}\href{../works/LudwigKRBMS14.pdf}{LudwigKRBMS14} (0.31)\\
Dot& \cellcolor{red!40}\href{../works/Dejemeppe16.pdf}{Dejemeppe16} (87.00)& \cellcolor{red!40}\href{../works/ZarandiASC20.pdf}{ZarandiASC20} (83.00)& \cellcolor{red!40}\href{../works/SenderovichBB19.pdf}{SenderovichBB19} (75.00)& \cellcolor{red!40}\href{../works/Wolf11.pdf}{Wolf11} (72.00)& \cellcolor{red!40}\href{../works/Astrand21.pdf}{Astrand21} (70.00)\\
Cosine& \cellcolor{red!40}\href{../works/HoYCLLCLC18.pdf}{HoYCLLCLC18} (0.76)& \cellcolor{red!40}\href{../works/WeilHFP95.pdf}{WeilHFP95} (0.64)& \cellcolor{red!40}\href{../works/SenderovichBB19.pdf}{SenderovichBB19} (0.63)& \cellcolor{red!40}\href{../works/LuoVLBM16.pdf}{LuoVLBM16} (0.63)& \cellcolor{red!40}\href{../works/Wolf11.pdf}{Wolf11} (0.62)\\
\index{Siala15}\href{../works/Siala15.pdf}{Siala15} R\&C& \cellcolor{black!20}\href{../works/OhrimenkoSC09.pdf}{OhrimenkoSC09} (0.99)\\
Euclid& \cellcolor{green!20}\href{../works/Siala15a.pdf}{Siala15a} (0.29)& \href{../works/GrimesHM09.pdf}{GrimesHM09} (0.52)& \href{../works/Bit-Monnot23.pdf}{Bit-Monnot23} (0.52)& \href{../works/SialaAH15.pdf}{SialaAH15} (0.53)& \href{../works/FocacciLN00.pdf}{FocacciLN00} (0.53)\\
Dot& \cellcolor{red!40}\href{../works/Siala15a.pdf}{Siala15a} (302.00)& \cellcolor{red!40}\href{../works/Malapert11.pdf}{Malapert11} (196.00)& \cellcolor{red!40}\href{../works/Dejemeppe16.pdf}{Dejemeppe16} (193.00)& \cellcolor{red!40}\href{../works/Godet21a.pdf}{Godet21a} (178.00)& \cellcolor{red!40}\href{../works/Groleaz21.pdf}{Groleaz21} (177.00)\\
Cosine& \cellcolor{red!40}\href{../works/Siala15a.pdf}{Siala15a} (0.92)& \cellcolor{red!40}\href{../works/GrimesHM09.pdf}{GrimesHM09} (0.62)& \cellcolor{red!40}\href{../works/Bit-Monnot23.pdf}{Bit-Monnot23} (0.62)& \cellcolor{red!40}\href{../works/FocacciLN00.pdf}{FocacciLN00} (0.60)& \cellcolor{red!40}\href{../works/SialaAH15.pdf}{SialaAH15} (0.60)\\
\index{Siala15a}\href{../works/Siala15a.pdf}{Siala15a} R\&C\\
Euclid& \cellcolor{green!20}\href{../works/Siala15.pdf}{Siala15} (0.29)& \href{../works/Bit-Monnot23.pdf}{Bit-Monnot23} (0.53)& \href{../works/SialaAH15.pdf}{SialaAH15} (0.55)& \href{../works/GrimesHM09.pdf}{GrimesHM09} (0.58)& \href{../works/GrimesH10.pdf}{GrimesH10} (0.58)\\
Dot& \cellcolor{red!40}\href{../works/Siala15.pdf}{Siala15} (302.00)& \cellcolor{red!40}\href{../works/Dejemeppe16.pdf}{Dejemeppe16} (232.00)& \cellcolor{red!40}\href{../works/Malapert11.pdf}{Malapert11} (230.00)& \cellcolor{red!40}\href{../works/Godet21a.pdf}{Godet21a} (224.00)& \cellcolor{red!40}\href{../works/Groleaz21.pdf}{Groleaz21} (219.00)\\
Cosine& \cellcolor{red!40}\href{../works/Siala15.pdf}{Siala15} (0.92)& \cellcolor{red!40}\href{../works/Bit-Monnot23.pdf}{Bit-Monnot23} (0.68)& \cellcolor{red!40}\href{../works/SialaAH15.pdf}{SialaAH15} (0.67)& \cellcolor{red!40}\href{../works/Elkhyari03.pdf}{Elkhyari03} (0.62)& \cellcolor{red!40}\href{../works/GrimesHM09.pdf}{GrimesHM09} (0.62)\\
\index{SialaAH15}\href{../works/SialaAH15.pdf}{SialaAH15} R\&C& \cellcolor{red!40}\href{../works/KreterSS15.pdf}{KreterSS15} (0.79)& \cellcolor{red!40}\href{../works/SchuttFS13.pdf}{SchuttFS13} (0.82)& \cellcolor{red!40}\href{../works/SzerediS16.pdf}{SzerediS16} (0.83)& \cellcolor{red!40}\href{../works/SchuttCSW12.pdf}{SchuttCSW12} (0.85)& \cellcolor{red!20}\href{../works/GrimesHM09.pdf}{GrimesHM09} (0.88)\\
Euclid& \cellcolor{yellow!20}\href{../works/SchuttFS13.pdf}{SchuttFS13} (0.27)& \cellcolor{yellow!20}\href{../works/FontaineMH16.pdf}{FontaineMH16} (0.27)& \cellcolor{yellow!20}\href{../works/HeipckeCCS00.pdf}{HeipckeCCS00} (0.28)& \cellcolor{yellow!20}\href{../works/CarchraeB09.pdf}{CarchraeB09} (0.28)& \cellcolor{yellow!20}\href{../works/KovacsV06.pdf}{KovacsV06} (0.28)\\
Dot& \cellcolor{red!40}\href{../works/Siala15a.pdf}{Siala15a} (143.00)& \cellcolor{red!40}\href{../works/Godet21a.pdf}{Godet21a} (142.00)& \cellcolor{red!40}\href{../works/Baptiste02.pdf}{Baptiste02} (137.00)& \cellcolor{red!40}\href{../works/Schutt11.pdf}{Schutt11} (133.00)& \cellcolor{red!40}\href{../works/Groleaz21.pdf}{Groleaz21} (131.00)\\
Cosine& \cellcolor{red!40}\href{../works/SchuttFS13.pdf}{SchuttFS13} (0.82)& \cellcolor{red!40}\href{../works/Bit-Monnot23.pdf}{Bit-Monnot23} (0.81)& \cellcolor{red!40}\href{../works/VilimLS15.pdf}{VilimLS15} (0.79)& \cellcolor{red!40}\href{../works/FontaineMH16.pdf}{FontaineMH16} (0.78)& \cellcolor{red!40}\href{../works/CarchraeB09.pdf}{CarchraeB09} (0.78)\\
\index{SimoninAHL12}\href{../works/SimoninAHL12.pdf}{SimoninAHL12} R\&C& \cellcolor{red!40}\href{../works/SimoninAHL15.pdf}{SimoninAHL15} (0.62)& \cellcolor{red!40}\href{../works/BeldiceanuC02.pdf}{BeldiceanuC02} (0.82)& \cellcolor{red!40}\href{../works/Simonis07.pdf}{Simonis07} (0.84)& \cellcolor{red!40}\href{../works/WolfS05a.pdf}{WolfS05a} (0.86)& \cellcolor{red!20}\href{../works/LetortCB15.pdf}{LetortCB15} (0.86)\\
Euclid& \cellcolor{red!40}\href{../works/SimoninAHL15.pdf}{SimoninAHL15} (0.18)& \cellcolor{red!40}\href{../works/PoderB08.pdf}{PoderB08} (0.22)& \cellcolor{red!40}\href{../works/WolfS05.pdf}{WolfS05} (0.23)& \cellcolor{red!40}\href{../works/Bonfietti16.pdf}{Bonfietti16} (0.23)& \cellcolor{red!40}\href{../works/BeldiceanuP07.pdf}{BeldiceanuP07} (0.24)\\
Dot& \cellcolor{red!40}\href{../works/Lombardi10.pdf}{Lombardi10} (96.00)& \cellcolor{red!40}\href{../works/Schutt11.pdf}{Schutt11} (94.00)& \cellcolor{red!40}\href{../works/Fahimi16.pdf}{Fahimi16} (93.00)& \cellcolor{red!40}\href{../works/Godet21a.pdf}{Godet21a} (92.00)& \cellcolor{red!40}\href{../works/Malapert11.pdf}{Malapert11} (91.00)\\
Cosine& \cellcolor{red!40}\href{../works/SimoninAHL15.pdf}{SimoninAHL15} (0.88)& \cellcolor{red!40}\href{../works/PoderB08.pdf}{PoderB08} (0.79)& \cellcolor{red!40}\href{../works/WolfS05.pdf}{WolfS05} (0.77)& \cellcolor{red!40}\href{../works/Bonfietti16.pdf}{Bonfietti16} (0.76)& \cellcolor{red!40}\href{../works/BeldiceanuP07.pdf}{BeldiceanuP07} (0.76)\\
\index{SimoninAHL15}\href{../works/SimoninAHL15.pdf}{SimoninAHL15} R\&C& \cellcolor{red!40}\href{../works/SimoninAHL12.pdf}{SimoninAHL12} (0.62)& \cellcolor{red!40}\href{../works/DavenportKRSH07.pdf}{DavenportKRSH07} (0.86)& \cellcolor{red!20}\href{../works/Simonis95.pdf}{Simonis95} (0.88)& \cellcolor{red!20}\href{../works/Geske05.pdf}{Geske05} (0.88)& \cellcolor{red!20}AggounV04 (0.89)\\
Euclid& \cellcolor{red!40}\href{../works/SimoninAHL12.pdf}{SimoninAHL12} (0.18)& \cellcolor{red!40}\href{../works/BeldiceanuP07.pdf}{BeldiceanuP07} (0.23)& \cellcolor{red!40}\href{../works/PoderB08.pdf}{PoderB08} (0.23)& \cellcolor{red!40}\href{../works/WolfS05.pdf}{WolfS05} (0.24)& \cellcolor{red!20}\href{../works/LudwigKRBMS14.pdf}{LudwigKRBMS14} (0.25)\\
Dot& \cellcolor{red!40}\href{../works/Malapert11.pdf}{Malapert11} (99.00)& \cellcolor{red!40}\href{../works/Lombardi10.pdf}{Lombardi10} (97.00)& \cellcolor{red!40}\href{../works/Godet21a.pdf}{Godet21a} (96.00)& \cellcolor{red!40}\href{../works/Schutt11.pdf}{Schutt11} (95.00)& \cellcolor{red!40}\href{../works/Groleaz21.pdf}{Groleaz21} (93.00)\\
Cosine& \cellcolor{red!40}\href{../works/SimoninAHL12.pdf}{SimoninAHL12} (0.88)& \cellcolor{red!40}\href{../works/BeldiceanuP07.pdf}{BeldiceanuP07} (0.79)& \cellcolor{red!40}\href{../works/PoderB08.pdf}{PoderB08} (0.77)& \cellcolor{red!40}\href{../works/WolfS05.pdf}{WolfS05} (0.77)& \cellcolor{red!40}\href{../works/LozanoCDS12.pdf}{LozanoCDS12} (0.74)\\
\index{Simonis07}\href{../works/Simonis07.pdf}{Simonis07} R\&C& \cellcolor{red!40}\href{../works/SimoninAHL12.pdf}{SimoninAHL12} (0.84)& \cellcolor{red!40}\href{../works/Simonis99.pdf}{Simonis99} (0.86)& \cellcolor{red!20}\href{../works/BosiM2001.pdf}{BosiM2001} (0.86)& \cellcolor{red!20}\href{../works/SimonisCK00.pdf}{SimonisCK00} (0.86)& \cellcolor{red!20}\href{../works/SimonisC95.pdf}{SimonisC95} (0.88)\\
Euclid& \href{../works/Simonis99.pdf}{Simonis99} (0.40)& \href{../works/HarjunkoskiG02.pdf}{HarjunkoskiG02} (0.42)& \href{../works/BockmayrP06.pdf}{BockmayrP06} (0.42)& \href{../works/Simonis95a.pdf}{Simonis95a} (0.43)& \href{../works/JainG01.pdf}{JainG01} (0.43)\\
Dot& \cellcolor{red!40}\href{../works/Dejemeppe16.pdf}{Dejemeppe16} (168.00)& \cellcolor{red!40}\href{../works/ZarandiASC20.pdf}{ZarandiASC20} (165.00)& \cellcolor{red!40}\href{../works/Malapert11.pdf}{Malapert11} (159.00)& \cellcolor{red!40}\href{../works/Fahimi16.pdf}{Fahimi16} (148.00)& \cellcolor{red!40}\href{../works/Baptiste02.pdf}{Baptiste02} (145.00)\\
Cosine& \cellcolor{red!40}\href{../works/Simonis99.pdf}{Simonis99} (0.72)& \cellcolor{red!40}\href{../works/HarjunkoskiG02.pdf}{HarjunkoskiG02} (0.67)& \cellcolor{red!40}\href{../works/BockmayrP06.pdf}{BockmayrP06} (0.66)& \cellcolor{red!40}\href{../works/JainG01.pdf}{JainG01} (0.66)& \cellcolor{red!40}\href{../works/Simonis95a.pdf}{Simonis95a} (0.65)\\
\index{Simonis95}\href{../works/Simonis95.pdf}{Simonis95} R\&C& \cellcolor{red!40}\href{../works/Simonis95a.pdf}{Simonis95a} (0.68)& \cellcolor{red!40}AggounV04 (0.71)& \cellcolor{red!40}\href{../works/SimonisCK00.pdf}{SimonisCK00} (0.75)& \cellcolor{red!40}\href{../works/SimonisC95.pdf}{SimonisC95} (0.82)& \cellcolor{red!40}\href{../works/BeldiceanuCDP11.pdf}{BeldiceanuCDP11} (0.82)\\
Euclid& \cellcolor{red!40}\href{../works/DincbasS91.pdf}{DincbasS91} (0.24)& \cellcolor{red!40}\href{../works/FalaschiGMP97.pdf}{FalaschiGMP97} (0.24)& \cellcolor{red!20}\href{../works/Simonis95a.pdf}{Simonis95a} (0.25)& \cellcolor{red!20}\href{../works/Touraivane95.pdf}{Touraivane95} (0.25)& \cellcolor{red!20}\href{../works/CestaOS98.pdf}{CestaOS98} (0.25)\\
Dot& \cellcolor{red!40}\href{../works/Simonis99.pdf}{Simonis99} (74.00)& \cellcolor{red!40}\href{../works/Simonis95a.pdf}{Simonis95a} (67.00)& \cellcolor{red!40}\href{../works/Simonis07.pdf}{Simonis07} (65.00)& \cellcolor{red!40}\href{../works/BeldiceanuC94.pdf}{BeldiceanuC94} (65.00)& \cellcolor{red!40}\href{../works/Wallace96.pdf}{Wallace96} (64.00)\\
Cosine& \cellcolor{red!40}\href{../works/Simonis95a.pdf}{Simonis95a} (0.81)& \cellcolor{red!40}\href{../works/SimonisCK00.pdf}{SimonisCK00} (0.73)& \cellcolor{red!40}\href{../works/BeldiceanuC94.pdf}{BeldiceanuC94} (0.73)& \cellcolor{red!40}\href{../works/Simonis99.pdf}{Simonis99} (0.72)& \cellcolor{red!40}\href{../works/GruianK98.pdf}{GruianK98} (0.70)\\
\index{Simonis95a}\href{../works/Simonis95a.pdf}{Simonis95a} R\&C& \cellcolor{red!40}\href{../works/Simonis99.pdf}{Simonis99} (0.63)& \cellcolor{red!40}\href{../works/Simonis95.pdf}{Simonis95} (0.68)& \cellcolor{red!40}\href{../works/SimonisC95.pdf}{SimonisC95} (0.75)& \cellcolor{red!40}\href{../works/Goltz95.pdf}{Goltz95} (0.79)& \cellcolor{red!20}AggounV04 (0.88)\\
Euclid& \cellcolor{red!20}\href{../works/Simonis95.pdf}{Simonis95} (0.25)& \cellcolor{red!20}\href{../works/Simonis99.pdf}{Simonis99} (0.26)& \cellcolor{yellow!20}\href{../works/SimonisC95.pdf}{SimonisC95} (0.28)& \cellcolor{green!20}\href{../works/GruianK98.pdf}{GruianK98} (0.29)& \cellcolor{green!20}\href{../works/DincbasSH90.pdf}{DincbasSH90} (0.30)\\
Dot& \cellcolor{red!40}\href{../works/Simonis99.pdf}{Simonis99} (129.00)& \cellcolor{red!40}\href{../works/Wallace96.pdf}{Wallace96} (104.00)& \cellcolor{red!40}\href{../works/Simonis07.pdf}{Simonis07} (104.00)& \cellcolor{red!40}\href{../works/Beck99.pdf}{Beck99} (104.00)& \cellcolor{red!40}\href{../works/Lombardi10.pdf}{Lombardi10} (97.00)\\
Cosine& \cellcolor{red!40}\href{../works/Simonis99.pdf}{Simonis99} (0.87)& \cellcolor{red!40}\href{../works/Simonis95.pdf}{Simonis95} (0.81)& \cellcolor{red!40}\href{../works/SimonisC95.pdf}{SimonisC95} (0.78)& \cellcolor{red!40}\href{../works/Wallace96.pdf}{Wallace96} (0.74)& \cellcolor{red!40}\href{../works/GruianK98.pdf}{GruianK98} (0.73)\\
\index{Simonis99}\href{../works/Simonis99.pdf}{Simonis99} R\&C& \cellcolor{red!40}\href{../works/Simonis95a.pdf}{Simonis95a} (0.63)& \cellcolor{red!40}\href{../works/SimonisCK00.pdf}{SimonisCK00} (0.81)& \cellcolor{red!40}\href{../works/SimonisC95.pdf}{SimonisC95} (0.81)& \cellcolor{red!40}\href{../works/Goltz95.pdf}{Goltz95} (0.84)& \cellcolor{red!40}MilanoORT02 (0.85)\\
Euclid& \cellcolor{red!20}\href{../works/Simonis95a.pdf}{Simonis95a} (0.26)& \cellcolor{black!20}\href{../works/SimonisC95.pdf}{SimonisC95} (0.35)& \cellcolor{black!20}\href{../works/SimonisCK00.pdf}{SimonisCK00} (0.36)& \cellcolor{black!20}\href{../works/Simonis95.pdf}{Simonis95} (0.37)& \cellcolor{black!20}\href{../works/Wallace96.pdf}{Wallace96} (0.37)\\
Dot& \cellcolor{red!40}\href{../works/Malapert11.pdf}{Malapert11} (147.00)& \cellcolor{red!40}\href{../works/Simonis07.pdf}{Simonis07} (143.00)& \cellcolor{red!40}\href{../works/Beck99.pdf}{Beck99} (141.00)& \cellcolor{red!40}\href{../works/Schutt11.pdf}{Schutt11} (137.00)& \cellcolor{red!40}\href{../works/Lombardi10.pdf}{Lombardi10} (136.00)\\
Cosine& \cellcolor{red!40}\href{../works/Simonis95a.pdf}{Simonis95a} (0.87)& \cellcolor{red!40}\href{../works/SimonisC95.pdf}{SimonisC95} (0.74)& \cellcolor{red!40}\href{../works/Wallace96.pdf}{Wallace96} (0.73)& \cellcolor{red!40}\href{../works/Simonis95.pdf}{Simonis95} (0.72)& \cellcolor{red!40}\href{../works/SimonisCK00.pdf}{SimonisCK00} (0.72)\\
\index{SimonisC95}\href{../works/SimonisC95.pdf}{SimonisC95} R\&C& \cellcolor{red!40}\href{../works/Simonis95a.pdf}{Simonis95a} (0.75)& \cellcolor{red!40}\href{../works/SimonisCK00.pdf}{SimonisCK00} (0.77)& \cellcolor{red!40}\href{../works/Simonis99.pdf}{Simonis99} (0.81)& \cellcolor{red!40}\href{../works/Simonis95.pdf}{Simonis95} (0.82)& \cellcolor{red!40}AggounV04 (0.83)\\
Euclid& \cellcolor{yellow!20}\href{../works/Simonis95a.pdf}{Simonis95a} (0.28)& \cellcolor{green!20}\href{../works/Goltz95.pdf}{Goltz95} (0.29)& \cellcolor{green!20}\href{../works/Simonis95.pdf}{Simonis95} (0.30)& \cellcolor{green!20}\href{../works/PoderBS04.pdf}{PoderBS04} (0.31)& \cellcolor{blue!20}\href{../works/SimonisCK00.pdf}{SimonisCK00} (0.32)\\
Dot& \cellcolor{red!40}\href{../works/Simonis99.pdf}{Simonis99} (111.00)& \cellcolor{red!40}\href{../works/Simonis07.pdf}{Simonis07} (100.00)& \cellcolor{red!40}\href{../works/Beck99.pdf}{Beck99} (100.00)& \cellcolor{red!40}\href{../works/BosiM2001.pdf}{BosiM2001} (99.00)& \cellcolor{red!40}\href{../works/Malapert11.pdf}{Malapert11} (98.00)\\
Cosine& \cellcolor{red!40}\href{../works/Simonis95a.pdf}{Simonis95a} (0.78)& \cellcolor{red!40}\href{../works/Goltz95.pdf}{Goltz95} (0.75)& \cellcolor{red!40}\href{../works/Simonis99.pdf}{Simonis99} (0.74)& \cellcolor{red!40}\href{../works/Simonis95.pdf}{Simonis95} (0.70)& \cellcolor{red!40}\href{../works/SimonisCK00.pdf}{SimonisCK00} (0.69)\\
\index{SimonisCK00}\href{../works/SimonisCK00.pdf}{SimonisCK00} R\&C& \cellcolor{red!40}\href{../works/Simonis95.pdf}{Simonis95} (0.75)& \cellcolor{red!40}\href{../works/Geske05.pdf}{Geske05} (0.75)& \cellcolor{red!40}\href{../works/SimonisC95.pdf}{SimonisC95} (0.77)& \cellcolor{red!40}AggounV04 (0.78)& \cellcolor{red!40}\href{../works/Simonis99.pdf}{Simonis99} (0.81)\\
Euclid& \cellcolor{red!20}\href{../works/Simonis95.pdf}{Simonis95} (0.26)& \cellcolor{yellow!20}\href{../works/PoderB08.pdf}{PoderB08} (0.28)& \cellcolor{yellow!20}\href{../works/PoderBS04.pdf}{PoderBS04} (0.28)& \cellcolor{green!20}\href{../works/WolfS05.pdf}{WolfS05} (0.29)& \cellcolor{green!20}\href{../works/BockmayrP06.pdf}{BockmayrP06} (0.30)\\
Dot& \cellcolor{red!40}\href{../works/Simonis99.pdf}{Simonis99} (98.00)& \cellcolor{red!40}\href{../works/Simonis07.pdf}{Simonis07} (94.00)& \cellcolor{red!40}\href{../works/Malapert11.pdf}{Malapert11} (94.00)& \cellcolor{red!40}\href{../works/LaborieRSV18.pdf}{LaborieRSV18} (82.00)& \cellcolor{red!40}\href{../works/Godet21a.pdf}{Godet21a} (82.00)\\
Cosine& \cellcolor{red!40}\href{../works/Simonis95.pdf}{Simonis95} (0.73)& \cellcolor{red!40}\href{../works/Simonis99.pdf}{Simonis99} (0.72)& \cellcolor{red!40}\href{../works/PoderBS04.pdf}{PoderBS04} (0.72)& \cellcolor{red!40}\href{../works/BeldiceanuC94.pdf}{BeldiceanuC94} (0.69)& \cellcolor{red!40}\href{../works/PoderB08.pdf}{PoderB08} (0.69)\\
\index{SimonisH11}\href{../works/SimonisH11.pdf}{SimonisH11} R\&C& \cellcolor{red!40}\href{../works/OuelletQ13.pdf}{OuelletQ13} (0.74)& \cellcolor{red!40}\href{../works/HoundjiSWD14.pdf}{HoundjiSWD14} (0.75)& \cellcolor{red!40}\href{../works/Vilim09.pdf}{Vilim09} (0.77)& \cellcolor{red!40}\href{../works/LetortCB15.pdf}{LetortCB15} (0.83)& \cellcolor{red!40}\href{../works/KameugneFSN11.pdf}{KameugneFSN11} (0.83)\\
Euclid& \cellcolor{red!40}\href{../works/PoderB08.pdf}{PoderB08} (0.20)& \cellcolor{red!40}\href{../works/WolfS05.pdf}{WolfS05} (0.20)& \cellcolor{red!40}\href{../works/BeldiceanuP07.pdf}{BeldiceanuP07} (0.20)& \cellcolor{red!40}\href{../works/BeniniBGM05a.pdf}{BeniniBGM05a} (0.23)& \cellcolor{red!40}\href{../works/Caseau97.pdf}{Caseau97} (0.24)\\
Dot& \cellcolor{red!40}\href{../works/Malapert11.pdf}{Malapert11} (74.00)& \cellcolor{red!40}\href{../works/Dejemeppe16.pdf}{Dejemeppe16} (69.00)& \cellcolor{red!40}\href{../works/Fahimi16.pdf}{Fahimi16} (68.00)& \cellcolor{red!40}\href{../works/Beck99.pdf}{Beck99} (67.00)& \cellcolor{red!40}\href{../works/Baptiste02.pdf}{Baptiste02} (67.00)\\
Cosine& \cellcolor{red!40}\href{../works/WolfS05.pdf}{WolfS05} (0.81)& \cellcolor{red!40}\href{../works/PoderB08.pdf}{PoderB08} (0.80)& \cellcolor{red!40}\href{../works/BeldiceanuP07.pdf}{BeldiceanuP07} (0.80)& \cellcolor{red!40}\href{../works/PoderBS04.pdf}{PoderBS04} (0.74)& \cellcolor{red!40}\href{../works/BeldiceanuC02.pdf}{BeldiceanuC02} (0.73)\\
\index{SmithBHW96}\href{../works/SmithBHW96.pdf}{SmithBHW96} R\&C& \cellcolor{red!40}\href{../works/Darby-DowmanLMZ97.pdf}{Darby-DowmanLMZ97} (0.69)& \cellcolor{red!40}DarbyDowmanL98 (0.82)& \cellcolor{red!20}\href{../works/NuijtenA96.pdf}{NuijtenA96} (0.89)& \cellcolor{yellow!20}\href{../works/RodosekWH99.pdf}{RodosekWH99} (0.91)& \cellcolor{yellow!20}\href{../works/DincbasSH90.pdf}{DincbasSH90} (0.91)\\
Euclid& \cellcolor{red!40}\href{../works/FeldmanG89.pdf}{FeldmanG89} (0.20)& \cellcolor{red!40}\href{../works/AbrilSB05.pdf}{AbrilSB05} (0.21)& \cellcolor{red!40}\href{../works/FrostD98.pdf}{FrostD98} (0.22)& \cellcolor{red!40}\href{../works/GelainPRVW17.pdf}{GelainPRVW17} (0.22)& \cellcolor{red!40}\href{../works/ZhangLS12.pdf}{ZhangLS12} (0.22)\\
Dot& \cellcolor{red!40}\href{../works/Baptiste02.pdf}{Baptiste02} (49.00)& \cellcolor{red!40}\href{../works/ZarandiASC20.pdf}{ZarandiASC20} (45.00)& \cellcolor{red!40}\href{../works/Wallace96.pdf}{Wallace96} (44.00)& \cellcolor{red!40}\href{../works/BartakSR10.pdf}{BartakSR10} (44.00)& \cellcolor{red!40}\href{../works/Lemos21.pdf}{Lemos21} (44.00)\\
Cosine& \cellcolor{red!40}\href{../works/BartakS11.pdf}{BartakS11} (0.69)& \cellcolor{red!40}\href{../works/Bartak02.pdf}{Bartak02} (0.66)& \cellcolor{red!40}\href{../works/ZhangLS12.pdf}{ZhangLS12} (0.65)& \cellcolor{red!40}\href{../works/GelainPRVW17.pdf}{GelainPRVW17} (0.64)& \cellcolor{red!40}\href{../works/GarridoOS08.pdf}{GarridoOS08} (0.64)\\
\index{SmithC93}\href{../works/SmithC93.pdf}{SmithC93} R\&C\\
Euclid& \cellcolor{red!40}\href{../works/OddiS97.pdf}{OddiS97} (0.18)& \cellcolor{red!40}\href{../works/FoxAS82.pdf}{FoxAS82} (0.22)& \cellcolor{red!20}\href{../works/CrawfordB94.pdf}{CrawfordB94} (0.24)& \cellcolor{red!20}\href{../works/ValleMGT03.pdf}{ValleMGT03} (0.25)& \cellcolor{red!20}\href{../works/LauLN08.pdf}{LauLN08} (0.25)\\
Dot& \cellcolor{red!40}\href{../works/ZarandiASC20.pdf}{ZarandiASC20} (100.00)& \cellcolor{red!40}\href{../works/Groleaz21.pdf}{Groleaz21} (98.00)& \cellcolor{red!40}\href{../works/Dejemeppe16.pdf}{Dejemeppe16} (94.00)& \cellcolor{red!40}\href{../works/Astrand21.pdf}{Astrand21} (92.00)& \cellcolor{red!40}\href{../works/Beck99.pdf}{Beck99} (92.00)\\
Cosine& \cellcolor{red!40}\href{../works/OddiS97.pdf}{OddiS97} (0.86)& \cellcolor{red!40}\href{../works/SadehF96.pdf}{SadehF96} (0.81)& \cellcolor{red!40}\href{../works/FoxAS82.pdf}{FoxAS82} (0.78)& \cellcolor{red!40}\href{../works/CestaOF99.pdf}{CestaOF99} (0.76)& \cellcolor{red!40}\href{../works/BeckPS03.pdf}{BeckPS03} (0.76)\\
\index{SourdN00}\href{../works/SourdN00.pdf}{SourdN00} R\&C& \cellcolor{red!40}\href{../works/TorresL00.pdf}{TorresL00} (0.72)& \cellcolor{red!40}DorndorfHP99 (0.76)& \cellcolor{red!40}\href{../works/MonetteDD07.pdf}{MonetteDD07} (0.81)& \cellcolor{red!40}DorndorfPH99 (0.81)& \cellcolor{red!40}\href{../works/TanSD10.pdf}{TanSD10} (0.84)\\
Euclid& \cellcolor{yellow!20}\href{../works/MenciaSV13.pdf}{MenciaSV13} (0.28)& \cellcolor{green!20}\href{../works/MenciaSV12.pdf}{MenciaSV12} (0.29)& \cellcolor{green!20}\href{../works/NuijtenA96.pdf}{NuijtenA96} (0.30)& \cellcolor{green!20}\href{../works/NuijtenP98.pdf}{NuijtenP98} (0.31)& \cellcolor{green!20}\href{../works/ArtiguesBF04.pdf}{ArtiguesBF04} (0.31)\\
Dot& \cellcolor{red!40}\href{../works/Baptiste02.pdf}{Baptiste02} (178.00)& \cellcolor{red!40}\href{../works/Groleaz21.pdf}{Groleaz21} (156.00)& \cellcolor{red!40}\href{../works/Malapert11.pdf}{Malapert11} (151.00)& \cellcolor{red!40}\href{../works/ZarandiASC20.pdf}{ZarandiASC20} (149.00)& \cellcolor{red!40}\href{../works/Godet21a.pdf}{Godet21a} (147.00)\\
Cosine& \cellcolor{red!40}\href{../works/MenciaSV12.pdf}{MenciaSV12} (0.82)& \cellcolor{red!40}\href{../works/MenciaSV13.pdf}{MenciaSV13} (0.82)& \cellcolor{red!40}\href{../works/NuijtenP98.pdf}{NuijtenP98} (0.79)& \cellcolor{red!40}\href{../works/NuijtenA96.pdf}{NuijtenA96} (0.78)& \cellcolor{red!40}\href{../works/BartakSR08.pdf}{BartakSR08} (0.78)\\
\index{SquillaciPR23}\href{../works/SquillaciPR23.pdf}{SquillaciPR23} R\&C& \cellcolor{green!20}\href{../works/BartoliniBBLM14.pdf}{BartoliniBBLM14} (0.95)& \cellcolor{green!20}\href{../works/ParkUJR19.pdf}{ParkUJR19} (0.95)& \cellcolor{green!20}\href{../works/GilesH16.pdf}{GilesH16} (0.96)& \cellcolor{blue!20}\href{../works/ZhangYW21.pdf}{ZhangYW21} (0.96)& \cellcolor{blue!20}\href{../works/Laborie18a.pdf}{Laborie18a} (0.97)\\
Euclid& \cellcolor{yellow!20}\href{../works/KucukY19.pdf}{KucukY19} (0.26)& \cellcolor{green!20}\href{../works/VerfaillieL01.pdf}{VerfaillieL01} (0.31)& \cellcolor{blue!20}\href{../works/FrankDT16.pdf}{FrankDT16} (0.32)& \cellcolor{blue!20}\href{../works/ZibranR11.pdf}{ZibranR11} (0.32)& \cellcolor{blue!20}\href{../works/ZibranR11a.pdf}{ZibranR11a} (0.33)\\
Dot& \cellcolor{red!40}\href{../works/Groleaz21.pdf}{Groleaz21} (81.00)& \cellcolor{red!40}\href{../works/Lemos21.pdf}{Lemos21} (80.00)& \cellcolor{red!40}\href{../works/LaborieRSV18.pdf}{LaborieRSV18} (78.00)& \cellcolor{red!40}\href{../works/Astrand21.pdf}{Astrand21} (74.00)& \cellcolor{red!40}\href{../works/ZarandiASC20.pdf}{ZarandiASC20} (73.00)\\
Cosine& \cellcolor{red!40}\href{../works/KucukY19.pdf}{KucukY19} (0.73)& \cellcolor{red!40}\href{../works/VerfaillieL01.pdf}{VerfaillieL01} (0.64)& \cellcolor{red!40}\href{../works/PraletLJ15.pdf}{PraletLJ15} (0.59)& \cellcolor{red!40}\href{../works/AlesioBNG15.pdf}{AlesioBNG15} (0.59)& \cellcolor{red!40}\href{../works/FrankDT16.pdf}{FrankDT16} (0.57)\\
\index{StidsenKM96}StidsenKM96 R\&C\\
Euclid\\
Dot\\
Cosine\\
\index{SuCC13}\href{../works/SuCC13.pdf}{SuCC13} R\&C& \cellcolor{red!40}\href{../works/CarlssonJL17.pdf}{CarlssonJL17} (0.74)& \cellcolor{red!40}\href{../works/ZengM12.pdf}{ZengM12} (0.82)& \cellcolor{red!20}Trick11 (0.86)& \cellcolor{red!20}\href{../works/LarsonJC14.pdf}{LarsonJC14} (0.87)& \cellcolor{red!20}\href{../works/Ribeiro12.pdf}{Ribeiro12} (0.89)\\
Euclid& \cellcolor{red!40}\href{../works/RasmussenT06.pdf}{RasmussenT06} (0.15)& \cellcolor{red!40}\href{../works/EastonNT02.pdf}{EastonNT02} (0.16)& \cellcolor{red!40}\href{../works/Trick03.pdf}{Trick03} (0.18)& \cellcolor{red!40}\href{../works/Perron05.pdf}{Perron05} (0.19)& \cellcolor{red!40}\href{../works/ElfJR03.pdf}{ElfJR03} (0.19)\\
Dot& \cellcolor{red!40}\href{../works/KendallKRU10.pdf}{KendallKRU10} (65.00)& \cellcolor{red!40}\href{../works/Ribeiro12.pdf}{Ribeiro12} (61.00)& \cellcolor{red!40}\href{../works/RasmussenT09.pdf}{RasmussenT09} (60.00)& \cellcolor{red!40}\href{../works/RasmussenT07.pdf}{RasmussenT07} (58.00)& \cellcolor{red!40}\href{../works/RasmussenT06.pdf}{RasmussenT06} (55.00)\\
Cosine& \cellcolor{red!40}\href{../works/RasmussenT06.pdf}{RasmussenT06} (0.88)& \cellcolor{red!40}\href{../works/EastonNT02.pdf}{EastonNT02} (0.84)& \cellcolor{red!40}\href{../works/Trick03.pdf}{Trick03} (0.83)& \cellcolor{red!40}\href{../works/RasmussenT07.pdf}{RasmussenT07} (0.81)& \cellcolor{red!40}\href{../works/RasmussenT09.pdf}{RasmussenT09} (0.80)\\
\index{SubulanC22}\href{../works/SubulanC22.pdf}{SubulanC22} R\&C& \cellcolor{yellow!20}\href{../works/YuraszeckMPV22.pdf}{YuraszeckMPV22} (0.91)& \cellcolor{yellow!20}\href{../works/HauderBRPA20.pdf}{HauderBRPA20} (0.92)& \cellcolor{yellow!20}\href{../works/ArkhipovBL19.pdf}{ArkhipovBL19} (0.93)& \cellcolor{green!20}\href{../works/SchnellH17.pdf}{SchnellH17} (0.94)& \cellcolor{green!20}\href{../works/ZouZ20.pdf}{ZouZ20} (0.96)\\
Euclid& \cellcolor{black!20}\href{../works/HubnerGSV21.pdf}{HubnerGSV21} (0.36)& \href{../works/LombardiM09.pdf}{LombardiM09} (0.40)& \href{../works/CampeauG22.pdf}{CampeauG22} (0.40)& \href{../works/QuirogaZH05.pdf}{QuirogaZH05} (0.41)& \href{../works/ZouZ20.pdf}{ZouZ20} (0.42)\\
Dot& \cellcolor{red!40}\href{../works/ZarandiASC20.pdf}{ZarandiASC20} (189.00)& \cellcolor{red!40}\href{../works/Groleaz21.pdf}{Groleaz21} (173.00)& \cellcolor{red!40}\href{../works/Lombardi10.pdf}{Lombardi10} (168.00)& \cellcolor{red!40}\href{../works/Astrand21.pdf}{Astrand21} (166.00)& \cellcolor{red!40}\href{../works/Lunardi20.pdf}{Lunardi20} (165.00)\\
Cosine& \cellcolor{red!40}\href{../works/HubnerGSV21.pdf}{HubnerGSV21} (0.77)& \cellcolor{red!40}\href{../works/abs-1902-09244.pdf}{abs-1902-09244} (0.69)& \cellcolor{red!40}\href{../works/LombardiM09.pdf}{LombardiM09} (0.69)& \cellcolor{red!40}\href{../works/AfsarVPG23.pdf}{AfsarVPG23} (0.68)& \cellcolor{red!40}\href{../works/CampeauG22.pdf}{CampeauG22} (0.68)\\
\index{SultanikMR07}\href{../works/SultanikMR07.pdf}{SultanikMR07} R\&C\\
Euclid& \cellcolor{red!40}\href{../works/Hunsberger08.pdf}{Hunsberger08} (0.22)& \cellcolor{red!20}\href{../works/GomesHS06.pdf}{GomesHS06} (0.25)& \cellcolor{red!20}\href{../works/ElhouraniDM07.pdf}{ElhouraniDM07} (0.25)& \cellcolor{red!20}\href{../works/SunLYL10.pdf}{SunLYL10} (0.26)& \cellcolor{yellow!20}\href{../works/HoeveGSL07.pdf}{HoeveGSL07} (0.26)\\
Dot& \cellcolor{red!40}\href{../works/Lemos21.pdf}{Lemos21} (75.00)& \cellcolor{red!40}\href{../works/Dejemeppe16.pdf}{Dejemeppe16} (73.00)& \cellcolor{red!40}\href{../works/ZarandiASC20.pdf}{ZarandiASC20} (70.00)& \cellcolor{red!40}\href{../works/Godet21a.pdf}{Godet21a} (70.00)& \cellcolor{red!40}\href{../works/Beck99.pdf}{Beck99} (70.00)\\
Cosine& \cellcolor{red!40}\href{../works/Hunsberger08.pdf}{Hunsberger08} (0.76)& \cellcolor{red!40}\href{../works/HoeveGSL07.pdf}{HoeveGSL07} (0.70)& \cellcolor{red!40}\href{../works/GomesHS06.pdf}{GomesHS06} (0.67)& \cellcolor{red!40}\href{../works/LozanoCDS12.pdf}{LozanoCDS12} (0.67)& \cellcolor{red!40}\href{../works/ElhouraniDM07.pdf}{ElhouraniDM07} (0.66)\\
\index{SunLYL10}\href{../works/SunLYL10.pdf}{SunLYL10} R\&C& \cellcolor{yellow!20}\href{../works/CambazardHDJT04.pdf}{CambazardHDJT04} (0.90)& \cellcolor{green!20}\href{../works/HladikCDJ08.pdf}{HladikCDJ08} (0.94)& \cellcolor{blue!20}\href{../works/AlesioNBG14.pdf}{AlesioNBG14} (0.96)& \cellcolor{blue!20}\href{../works/LiuW11.pdf}{LiuW11} (0.98)& \cellcolor{black!20}\href{../works/AlesioBNG15.pdf}{AlesioBNG15} (0.98)\\
Euclid& \cellcolor{red!40}\href{../works/Hunsberger08.pdf}{Hunsberger08} (0.17)& \cellcolor{red!40}\href{../works/GomesHS06.pdf}{GomesHS06} (0.19)& \cellcolor{red!40}\href{../works/AngelsmarkJ00.pdf}{AngelsmarkJ00} (0.21)& \cellcolor{red!40}\href{../works/CarchraeBF05.pdf}{CarchraeBF05} (0.22)& \cellcolor{red!40}\href{../works/BarlattCG08.pdf}{BarlattCG08} (0.22)\\
Dot& \cellcolor{red!40}\href{../works/Groleaz21.pdf}{Groleaz21} (56.00)& \cellcolor{red!40}\href{../works/Lombardi10.pdf}{Lombardi10} (53.00)& \cellcolor{red!40}\href{../works/HarjunkoskiMBC14.pdf}{HarjunkoskiMBC14} (52.00)& \cellcolor{red!40}\href{../works/HladikCDJ08.pdf}{HladikCDJ08} (52.00)& \cellcolor{red!40}\href{../works/ZarandiASC20.pdf}{ZarandiASC20} (51.00)\\
Cosine& \cellcolor{red!40}\href{../works/Hunsberger08.pdf}{Hunsberger08} (0.79)& \cellcolor{red!40}\href{../works/GomesHS06.pdf}{GomesHS06} (0.72)& \cellcolor{red!40}\href{../works/LozanoCDS12.pdf}{LozanoCDS12} (0.68)& \cellcolor{red!40}\href{../works/RoweJCA96.pdf}{RoweJCA96} (0.67)& \cellcolor{red!40}\href{../works/AbidinK20.pdf}{AbidinK20} (0.67)\\
\index{SunTB19}\href{../works/SunTB19.pdf}{SunTB19} R\&C& \cellcolor{red!40}\href{../works/QinDCS20.pdf}{QinDCS20} (0.71)& \cellcolor{red!40}\href{../works/UnsalO13.pdf}{UnsalO13} (0.75)& \cellcolor{red!40}\href{../works/CireCH16.pdf}{CireCH16} (0.85)& \cellcolor{red!40}\href{../works/CobanH11.pdf}{CobanH11} (0.86)& \cellcolor{red!20}\href{../works/UnsalO19.pdf}{UnsalO19} (0.87)\\
Euclid& \cellcolor{blue!20}\href{../works/CorreaLR07.pdf}{CorreaLR07} (0.32)& \cellcolor{blue!20}\href{../works/CireCH13.pdf}{CireCH13} (0.33)& \cellcolor{blue!20}\href{../works/LipovetzkyBPS14.pdf}{LipovetzkyBPS14} (0.33)& \cellcolor{black!20}\href{../works/CireCH16.pdf}{CireCH16} (0.35)& \cellcolor{black!20}\href{../works/BeniniLMMR08.pdf}{BeniniLMMR08} (0.35)\\
Dot& \cellcolor{red!40}\href{../works/Lombardi10.pdf}{Lombardi10} (114.00)& \cellcolor{red!40}\href{../works/UnsalO13.pdf}{UnsalO13} (107.00)& \cellcolor{red!40}\href{../works/ZhuSZW23.pdf}{ZhuSZW23} (106.00)& \cellcolor{red!40}\href{../works/QinDCS20.pdf}{QinDCS20} (106.00)& \cellcolor{red!40}\href{../works/Groleaz21.pdf}{Groleaz21} (106.00)\\
Cosine& \cellcolor{red!40}\href{../works/CorreaLR07.pdf}{CorreaLR07} (0.70)& \cellcolor{red!40}\href{../works/QinDCS20.pdf}{QinDCS20} (0.70)& \cellcolor{red!40}\href{../works/UnsalO13.pdf}{UnsalO13} (0.69)& \cellcolor{red!40}\href{../works/EmeretlisTAV17.pdf}{EmeretlisTAV17} (0.67)& \cellcolor{red!40}\href{../works/LipovetzkyBPS14.pdf}{LipovetzkyBPS14} (0.67)\\
\index{SureshMOK06}\href{../works/SureshMOK06.pdf}{SureshMOK06} R\&C\\
Euclid& \cellcolor{yellow!20}\href{../works/Davis87.pdf}{Davis87} (0.27)& \cellcolor{yellow!20}\href{../works/Hunsberger08.pdf}{Hunsberger08} (0.27)& \cellcolor{yellow!20}\href{../works/AngelsmarkJ00.pdf}{AngelsmarkJ00} (0.27)& \cellcolor{yellow!20}\href{../works/RoweJCA96.pdf}{RoweJCA96} (0.27)& \cellcolor{yellow!20}\href{../works/SunLYL10.pdf}{SunLYL10} (0.28)\\
Dot& \cellcolor{red!40}\href{../works/ZarandiASC20.pdf}{ZarandiASC20} (74.00)& \cellcolor{red!40}\href{../works/Groleaz21.pdf}{Groleaz21} (72.00)& \cellcolor{red!40}\href{../works/Beck99.pdf}{Beck99} (68.00)& \cellcolor{red!40}\href{../works/Lombardi10.pdf}{Lombardi10} (66.00)& \cellcolor{red!40}\href{../works/Froger16.pdf}{Froger16} (66.00)\\
Cosine& \cellcolor{red!40}\href{../works/TranPZLDB18.pdf}{TranPZLDB18} (0.64)& \cellcolor{red!40}\href{../works/BarbulescuWH04.pdf}{BarbulescuWH04} (0.63)& \cellcolor{red!40}\href{../works/RoweJCA96.pdf}{RoweJCA96} (0.61)& \cellcolor{red!40}\href{../works/IfrimOS12.pdf}{IfrimOS12} (0.60)& \cellcolor{red!40}\href{../works/WuBB09.pdf}{WuBB09} (0.60)\\
\index{SvancaraB22}\href{../works/SvancaraB22.pdf}{SvancaraB22} R\&C\\
Euclid& \cellcolor{red!20}\href{../works/WallaceF00.pdf}{WallaceF00} (0.26)& \cellcolor{red!20}\href{../works/FortinZDF05.pdf}{FortinZDF05} (0.26)& \cellcolor{red!20}\href{../works/RodriguezDG02.pdf}{RodriguezDG02} (0.26)& \cellcolor{red!20}\href{../works/Hunsberger08.pdf}{Hunsberger08} (0.26)& \cellcolor{yellow!20}\href{../works/AngelsmarkJ00.pdf}{AngelsmarkJ00} (0.27)\\
Dot& \cellcolor{red!40}\href{../works/ZarandiASC20.pdf}{ZarandiASC20} (82.00)& \cellcolor{red!40}\href{../works/LaborieRSV18.pdf}{LaborieRSV18} (72.00)& \cellcolor{red!40}\href{../works/Lemos21.pdf}{Lemos21} (70.00)& \cellcolor{red!40}\href{../works/Beck99.pdf}{Beck99} (66.00)& \cellcolor{red!40}\href{../works/Fahimi16.pdf}{Fahimi16} (66.00)\\
Cosine& \cellcolor{red!40}\href{../works/TranVNB17.pdf}{TranVNB17} (0.67)& \cellcolor{red!40}\href{../works/BehrensLM19.pdf}{BehrensLM19} (0.66)& \cellcolor{red!40}\href{../works/FortinZDF05.pdf}{FortinZDF05} (0.66)& \cellcolor{red!40}\href{../works/RodriguezDG02.pdf}{RodriguezDG02} (0.64)& \cellcolor{red!40}\href{../works/ZeballosM09.pdf}{ZeballosM09} (0.63)\\
\index{SzerediS16}\href{../works/SzerediS16.pdf}{SzerediS16} R\&C& \cellcolor{red!40}\href{../works/YoungFS17.pdf}{YoungFS17} (0.57)& \cellcolor{red!40}\href{../works/KreterSS15.pdf}{KreterSS15} (0.68)& \cellcolor{red!40}\href{../works/SchuttFS13.pdf}{SchuttFS13} (0.72)& \cellcolor{red!40}\href{../works/SchnellH15.pdf}{SchnellH15} (0.79)& \cellcolor{red!40}\href{../works/GeibingerMM19.pdf}{GeibingerMM19} (0.80)\\
Euclid& \cellcolor{red!40}\href{../works/YoungFS17.pdf}{YoungFS17} (0.21)& \cellcolor{red!20}\href{../works/BofillCSV17.pdf}{BofillCSV17} (0.25)& \cellcolor{yellow!20}\href{../works/AmadiniGM16.pdf}{AmadiniGM16} (0.28)& \cellcolor{yellow!20}\href{../works/BofillCSV17a.pdf}{BofillCSV17a} (0.28)& \cellcolor{green!20}\href{../works/SchuttS16.pdf}{SchuttS16} (0.29)\\
Dot& \cellcolor{red!40}\href{../works/Godet21a.pdf}{Godet21a} (130.00)& \cellcolor{red!40}\href{../works/Caballero19.pdf}{Caballero19} (125.00)& \cellcolor{red!40}\href{../works/PovedaAA23.pdf}{PovedaAA23} (122.00)& \cellcolor{red!40}\href{../works/BoudreaultSLQ22.pdf}{BoudreaultSLQ22} (122.00)& \cellcolor{red!40}\href{../works/Schutt11.pdf}{Schutt11} (122.00)\\
Cosine& \cellcolor{red!40}\href{../works/YoungFS17.pdf}{YoungFS17} (0.88)& \cellcolor{red!40}\href{../works/BofillCSV17.pdf}{BofillCSV17} (0.81)& \cellcolor{red!40}\href{../works/KreterSSZ18.pdf}{KreterSSZ18} (0.79)& \cellcolor{red!40}\href{../works/SchnellH15.pdf}{SchnellH15} (0.78)& \cellcolor{red!40}\href{../works/KreterSS15.pdf}{KreterSS15} (0.77)\\
\index{TanSD10}\href{../works/TanSD10.pdf}{TanSD10} R\&C& \cellcolor{red!40}\href{../works/MonetteDD07.pdf}{MonetteDD07} (0.83)& \cellcolor{red!40}DorndorfHP99 (0.84)& \cellcolor{red!40}\href{../works/SourdN00.pdf}{SourdN00} (0.84)& \cellcolor{red!40}LiuGT10 (0.85)& \cellcolor{red!20}BaptisteLPN06 (0.86)\\
Euclid& \cellcolor{red!40}\href{../works/WatsonB08.pdf}{WatsonB08} (0.23)& \cellcolor{red!20}\href{../works/BeckFW11.pdf}{BeckFW11} (0.24)& \cellcolor{red!20}\href{../works/ArtiguesBF04.pdf}{ArtiguesBF04} (0.26)& \cellcolor{yellow!20}\href{../works/ArtiguesF07.pdf}{ArtiguesF07} (0.26)& \cellcolor{yellow!20}\href{../works/VilimBC04.pdf}{VilimBC04} (0.27)\\
Dot& \cellcolor{red!40}\href{../works/Baptiste02.pdf}{Baptiste02} (128.00)& \cellcolor{red!40}\href{../works/Groleaz21.pdf}{Groleaz21} (127.00)& \cellcolor{red!40}\href{../works/Schutt11.pdf}{Schutt11} (126.00)& \cellcolor{red!40}\href{../works/Malapert11.pdf}{Malapert11} (125.00)& \cellcolor{red!40}\href{../works/Fahimi16.pdf}{Fahimi16} (125.00)\\
Cosine& \cellcolor{red!40}\href{../works/WatsonB08.pdf}{WatsonB08} (0.84)& \cellcolor{red!40}\href{../works/BeckFW11.pdf}{BeckFW11} (0.83)& \cellcolor{red!40}\href{../works/ArtiguesF07.pdf}{ArtiguesF07} (0.82)& \cellcolor{red!40}\href{../works/ArtiguesBF04.pdf}{ArtiguesBF04} (0.82)& \cellcolor{red!40}\href{../works/MenciaSV13.pdf}{MenciaSV13} (0.80)\\
\index{TanT18}\href{../works/TanT18.pdf}{TanT18} R\&C& \cellcolor{red!40}\href{../works/EmeretlisTAV17.pdf}{EmeretlisTAV17} (0.77)& \cellcolor{red!20}\href{../works/CobanH11.pdf}{CobanH11} (0.88)& \cellcolor{red!20}MartnezAJ22 (0.88)& \cellcolor{red!20}\href{../works/CireCH16.pdf}{CireCH16} (0.89)& \cellcolor{red!20}NaderiRBAU21 (0.89)\\
Euclid& \cellcolor{blue!20}\href{../works/BillautHL12.pdf}{BillautHL12} (0.33)& \cellcolor{blue!20}\href{../works/Mehdizadeh-Somarin23.pdf}{Mehdizadeh-Somarin23} (0.33)& \cellcolor{blue!20}\href{../works/ArtiguesBF04.pdf}{ArtiguesBF04} (0.33)& \cellcolor{blue!20}\href{../works/GuyonLPR12.pdf}{GuyonLPR12} (0.33)& \cellcolor{blue!20}\href{../works/TranB12.pdf}{TranB12} (0.33)\\
Dot& \cellcolor{red!40}\href{../works/NaderiRR23.pdf}{NaderiRR23} (137.00)& \cellcolor{red!40}\href{../works/Groleaz21.pdf}{Groleaz21} (137.00)& \cellcolor{red!40}\href{../works/ZarandiASC20.pdf}{ZarandiASC20} (131.00)& \cellcolor{red!40}\href{../works/Baptiste02.pdf}{Baptiste02} (131.00)& \cellcolor{red!40}\href{../works/JuvinHL22.pdf}{JuvinHL22} (128.00)\\
Cosine& \cellcolor{red!40}\href{../works/JuvinHL22.pdf}{JuvinHL22} (0.77)& \cellcolor{red!40}\href{../works/NaderiBZ22a.pdf}{NaderiBZ22a} (0.74)& \cellcolor{red!40}\href{../works/TranB12.pdf}{TranB12} (0.74)& \cellcolor{red!40}\href{../works/ZhuSZW23.pdf}{ZhuSZW23} (0.74)& \cellcolor{red!40}\href{../works/GuyonLPR12.pdf}{GuyonLPR12} (0.74)\\
\index{TanZWGQ19}\href{../works/TanZWGQ19.pdf}{TanZWGQ19} R\&C& \cellcolor{green!20}\href{../works/CobanH10.pdf}{CobanH10} (0.94)& \cellcolor{green!20}\href{../works/HamdiL13.pdf}{HamdiL13} (0.94)& \cellcolor{green!20}\href{../works/ChuX05.pdf}{ChuX05} (0.94)& \cellcolor{green!20}GongLMW09 (0.94)& \cellcolor{green!20}\href{../works/ZeballosM09.pdf}{ZeballosM09} (0.94)\\
Euclid& \cellcolor{blue!20}\href{../works/CzerniachowskaWZ23.pdf}{CzerniachowskaWZ23} (0.33)& \cellcolor{blue!20}\href{../works/JuvinHL23.pdf}{JuvinHL23} (0.33)& \cellcolor{blue!20}\href{../works/Limtanyakul07.pdf}{Limtanyakul07} (0.33)& \cellcolor{blue!20}\href{../works/BarlattCG08.pdf}{BarlattCG08} (0.34)& \cellcolor{blue!20}\href{../works/LauLN08.pdf}{LauLN08} (0.34)\\
Dot& \cellcolor{red!40}\href{../works/Lunardi20.pdf}{Lunardi20} (135.00)& \cellcolor{red!40}\href{../works/ZarandiASC20.pdf}{ZarandiASC20} (124.00)& \cellcolor{red!40}\href{../works/Astrand21.pdf}{Astrand21} (111.00)& \cellcolor{red!40}\href{../works/MengZRZL20.pdf}{MengZRZL20} (111.00)& \cellcolor{red!40}\href{../works/Groleaz21.pdf}{Groleaz21} (111.00)\\
Cosine& \cellcolor{red!40}\href{../works/CzerniachowskaWZ23.pdf}{CzerniachowskaWZ23} (0.76)& \cellcolor{red!40}\href{../works/MurinR19.pdf}{MurinR19} (0.70)& \cellcolor{red!40}\href{../works/NaderiBZ23.pdf}{NaderiBZ23} (0.68)& \cellcolor{red!40}\href{../works/Ham18a.pdf}{Ham18a} (0.68)& \cellcolor{red!40}\href{../works/NaderiBZ22.pdf}{NaderiBZ22} (0.68)\\
\index{TangB20}\href{../works/TangB20.pdf}{TangB20} R\&C& \cellcolor{red!40}\href{../works/KoschB14.pdf}{KoschB14} (0.83)& \cellcolor{red!40}\href{../works/ArmstrongGOS21.pdf}{ArmstrongGOS21} (0.86)& \cellcolor{red!20}\href{../works/MalapertGR12.pdf}{MalapertGR12} (0.86)& \cellcolor{red!20}\href{../works/CappartS17.pdf}{CappartS17} (0.88)& \cellcolor{red!20}\href{../works/HamFC17.pdf}{HamFC17} (0.88)\\
Euclid& \cellcolor{black!20}\href{../works/Beck10.pdf}{Beck10} (0.35)& \cellcolor{black!20}\href{../works/HamdiL13.pdf}{HamdiL13} (0.35)& \cellcolor{black!20}\href{../works/BajestaniB11.pdf}{BajestaniB11} (0.36)& \cellcolor{black!20}\href{../works/Limtanyakul07.pdf}{Limtanyakul07} (0.36)& \cellcolor{black!20}\href{../works/CireCH13.pdf}{CireCH13} (0.37)\\
Dot& \cellcolor{red!40}\href{../works/ZarandiASC20.pdf}{ZarandiASC20} (112.00)& \cellcolor{red!40}\href{../works/Malapert11.pdf}{Malapert11} (111.00)& \cellcolor{red!40}\href{../works/Dejemeppe16.pdf}{Dejemeppe16} (109.00)& \cellcolor{red!40}\href{../works/Lunardi20.pdf}{Lunardi20} (106.00)& \cellcolor{red!40}\href{../works/LaborieRSV18.pdf}{LaborieRSV18} (105.00)\\
Cosine& \cellcolor{red!40}\href{../works/HamdiL13.pdf}{HamdiL13} (0.68)& \cellcolor{red!40}\href{../works/Beck10.pdf}{Beck10} (0.66)& \cellcolor{red!40}\href{../works/MalapertGR12.pdf}{MalapertGR12} (0.64)& \cellcolor{red!40}\href{../works/BajestaniB11.pdf}{BajestaniB11} (0.63)& \cellcolor{red!40}\href{../works/ParkUJR19.pdf}{ParkUJR19} (0.63)\\
\index{TangLWSK18}\href{../works/TangLWSK18.pdf}{TangLWSK18} R\&C& \cellcolor{red!40}\href{../works/ZouZ20.pdf}{ZouZ20} (0.80)& \cellcolor{yellow!20}\href{../works/LiuW11.pdf}{LiuW11} (0.93)& \cellcolor{green!20}\href{../works/QuirogaZH05.pdf}{QuirogaZH05} (0.95)& \cellcolor{green!20}\href{../works/KovacsV04.pdf}{KovacsV04} (0.95)& \cellcolor{green!20}\href{../works/GedikKBR17.pdf}{GedikKBR17} (0.95)\\
Euclid& \cellcolor{green!20}\href{../works/ZouZ20.pdf}{ZouZ20} (0.31)& \cellcolor{blue!20}\href{../works/ZibranR11a.pdf}{ZibranR11a} (0.32)& \cellcolor{blue!20}\href{../works/Salido10.pdf}{Salido10} (0.33)& \cellcolor{blue!20}\href{../works/KamarainenS02.pdf}{KamarainenS02} (0.34)& \cellcolor{blue!20}\href{../works/BhatnagarKL19.pdf}{BhatnagarKL19} (0.34)\\
Dot& \cellcolor{red!40}\href{../works/ZarandiASC20.pdf}{ZarandiASC20} (144.00)& \cellcolor{red!40}\href{../works/Lombardi10.pdf}{Lombardi10} (121.00)& \cellcolor{red!40}\href{../works/Dejemeppe16.pdf}{Dejemeppe16} (118.00)& \cellcolor{red!40}\href{../works/Lemos21.pdf}{Lemos21} (112.00)& \cellcolor{red!40}\href{../works/Groleaz21.pdf}{Groleaz21} (112.00)\\
Cosine& \cellcolor{red!40}\href{../works/ZouZ20.pdf}{ZouZ20} (0.73)& \cellcolor{red!40}\href{../works/ZibranR11a.pdf}{ZibranR11a} (0.70)& \cellcolor{red!40}\href{../works/Salido10.pdf}{Salido10} (0.68)& \cellcolor{red!40}\href{../works/LiuW11.pdf}{LiuW11} (0.66)& \cellcolor{red!40}\href{../works/ZeballosM09.pdf}{ZeballosM09} (0.66)\\
\index{TardivoDFMP23}\href{../works/TardivoDFMP23.pdf}{TardivoDFMP23} R\&C& \cellcolor{red!40}\href{../works/OuelletQ18.pdf}{OuelletQ18} (0.76)& \cellcolor{red!40}\href{../works/OuelletQ13.pdf}{OuelletQ13} (0.77)& \cellcolor{red!40}\href{../works/Tesch18.pdf}{Tesch18} (0.78)& \cellcolor{red!40}\href{../works/GayHS15a.pdf}{GayHS15a} (0.79)& \cellcolor{red!40}\href{../works/Tesch16.pdf}{Tesch16} (0.80)\\
Euclid& \cellcolor{green!20}\href{../works/GayHS15a.pdf}{GayHS15a} (0.31)& \cellcolor{blue!20}\href{../works/Vilim11.pdf}{Vilim11} (0.32)& \cellcolor{blue!20}\href{../works/OuelletQ18.pdf}{OuelletQ18} (0.32)& \cellcolor{black!20}\href{../works/KameugneFSN14.pdf}{KameugneFSN14} (0.36)& \cellcolor{black!20}\href{../works/KameugneFSN11.pdf}{KameugneFSN11} (0.36)\\
Dot& \cellcolor{red!40}\href{../works/Fahimi16.pdf}{Fahimi16} (154.00)& \cellcolor{red!40}\href{../works/Schutt11.pdf}{Schutt11} (146.00)& \cellcolor{red!40}\href{../works/Dejemeppe16.pdf}{Dejemeppe16} (137.00)& \cellcolor{red!40}\href{../works/Kameugne14.pdf}{Kameugne14} (132.00)& \cellcolor{red!40}\href{../works/Baptiste02.pdf}{Baptiste02} (131.00)\\
Cosine& \cellcolor{red!40}\href{../works/GayHS15a.pdf}{GayHS15a} (0.77)& \cellcolor{red!40}\href{../works/Vilim11.pdf}{Vilim11} (0.76)& \cellcolor{red!40}\href{../works/OuelletQ18.pdf}{OuelletQ18} (0.75)& \cellcolor{red!40}\href{../works/KameugneFSN14.pdf}{KameugneFSN14} (0.72)& \cellcolor{red!40}\href{../works/SchuttFS13a.pdf}{SchuttFS13a} (0.72)\\
\index{Tassel22}\href{../works/Tassel22.pdf}{Tassel22} R\&C\\
Euclid& \cellcolor{yellow!20}\href{../works/IklassovMR023.pdf}{IklassovMR023} (0.27)& \cellcolor{yellow!20}\href{../works/HebrardTW05.pdf}{HebrardTW05} (0.28)& \cellcolor{yellow!20}\href{../works/AngelsmarkJ00.pdf}{AngelsmarkJ00} (0.28)& \cellcolor{yellow!20}\href{../works/LauLN08.pdf}{LauLN08} (0.28)& \cellcolor{green!20}\href{../works/BarlattCG08.pdf}{BarlattCG08} (0.29)\\
Dot& \cellcolor{red!40}\href{../works/abs-2211-14492.pdf}{abs-2211-14492} (85.00)& \cellcolor{red!40}\href{../works/Groleaz21.pdf}{Groleaz21} (84.00)& \cellcolor{red!40}\href{../works/ZarandiASC20.pdf}{ZarandiASC20} (83.00)& \cellcolor{red!40}\href{../works/MullerMKP22.pdf}{MullerMKP22} (81.00)& \cellcolor{red!40}\href{../works/KovacsTKSG21.pdf}{KovacsTKSG21} (79.00)\\
Cosine& \cellcolor{red!40}\href{../works/IklassovMR023.pdf}{IklassovMR023} (0.73)& \cellcolor{red!40}\href{../works/KovacsTKSG21.pdf}{KovacsTKSG21} (0.70)& \cellcolor{red!40}\href{../works/abs-2211-14492.pdf}{abs-2211-14492} (0.68)& \cellcolor{red!40}\href{../works/CarchraeB09.pdf}{CarchraeB09} (0.67)& \cellcolor{red!40}\href{../works/BeckFW11.pdf}{BeckFW11} (0.67)\\
\index{TasselGS23}\href{../works/TasselGS23.pdf}{TasselGS23} R\&C\\
Euclid& \cellcolor{red!40}\href{../works/abs-2306-05747.pdf}{abs-2306-05747} (0.00)& \cellcolor{green!20}\href{../works/BeckFW11.pdf}{BeckFW11} (0.30)& \cellcolor{green!20}\href{../works/CarchraeB09.pdf}{CarchraeB09} (0.30)& \cellcolor{green!20}\href{../works/IklassovMR023.pdf}{IklassovMR023} (0.30)& \cellcolor{green!20}\href{../works/WatsonB08.pdf}{WatsonB08} (0.30)\\
Dot& \cellcolor{red!40}\href{../works/abs-2306-05747.pdf}{abs-2306-05747} (145.00)& \cellcolor{red!40}\href{../works/ZarandiASC20.pdf}{ZarandiASC20} (143.00)& \cellcolor{red!40}\href{../works/Groleaz21.pdf}{Groleaz21} (142.00)& \cellcolor{red!40}\href{../works/ColT22.pdf}{ColT22} (134.00)& \cellcolor{red!40}\href{../works/Lunardi20.pdf}{Lunardi20} (132.00)\\
Cosine& \cellcolor{red!40}\href{../works/abs-2306-05747.pdf}{abs-2306-05747} (1.00)& \cellcolor{red!40}\href{../works/BeckFW11.pdf}{BeckFW11} (0.78)& \cellcolor{red!40}\href{../works/CarchraeB09.pdf}{CarchraeB09} (0.77)& \cellcolor{red!40}\href{../works/IklassovMR023.pdf}{IklassovMR023} (0.77)& \cellcolor{red!40}\href{../works/WatsonB08.pdf}{WatsonB08} (0.76)\\
\index{Tay92}Tay92 R\&C\\
Euclid\\
Dot\\
Cosine\\
\index{Teppan22}\href{../works/Teppan22.pdf}{Teppan22} R\&C\\
Euclid& \cellcolor{yellow!20}\href{../works/abs-2102-08778.pdf}{abs-2102-08778} (0.27)& \cellcolor{green!20}\href{../works/HamP21.pdf}{HamP21} (0.29)& \cellcolor{green!20}\href{../works/WatsonB08.pdf}{WatsonB08} (0.30)& \cellcolor{green!20}\href{../works/TanSD10.pdf}{TanSD10} (0.31)& \cellcolor{green!20}\href{../works/JuvinHL23.pdf}{JuvinHL23} (0.31)\\
Dot& \cellcolor{red!40}\href{../works/ZarandiASC20.pdf}{ZarandiASC20} (127.00)& \cellcolor{red!40}\href{../works/ColT22.pdf}{ColT22} (126.00)& \cellcolor{red!40}\href{../works/Lunardi20.pdf}{Lunardi20} (122.00)& \cellcolor{red!40}\href{../works/Baptiste02.pdf}{Baptiste02} (122.00)& \cellcolor{red!40}\href{../works/Groleaz21.pdf}{Groleaz21} (119.00)\\
Cosine& \cellcolor{red!40}\href{../works/abs-2102-08778.pdf}{abs-2102-08778} (0.78)& \cellcolor{red!40}\href{../works/HamP21.pdf}{HamP21} (0.78)& \cellcolor{red!40}\href{../works/YuraszeckMCCR23.pdf}{YuraszeckMCCR23} (0.73)& \cellcolor{red!40}\href{../works/JuvinHL22.pdf}{JuvinHL22} (0.73)& \cellcolor{red!40}\href{../works/abs-2306-05747.pdf}{abs-2306-05747} (0.73)\\
\index{TerekhovDOB12}\href{../works/TerekhovDOB12.pdf}{TerekhovDOB12} R\&C& \cellcolor{red!20}\href{../works/CireCH13.pdf}{CireCH13} (0.88)& \cellcolor{red!20}\href{../works/HamdiL13.pdf}{HamdiL13} (0.88)& \cellcolor{red!20}\href{../works/TrojetHL11.pdf}{TrojetHL11} (0.90)& \cellcolor{yellow!20}\href{../works/AgussurjaKL18.pdf}{AgussurjaKL18} (0.92)& \cellcolor{yellow!20}\href{../works/CobanH10.pdf}{CobanH10} (0.92)\\
Euclid& \cellcolor{black!20}\href{../works/KovacsB11.pdf}{KovacsB11} (0.37)& \cellcolor{black!20}\href{../works/BeckR03.pdf}{BeckR03} (0.37)& \href{../works/BeckPS03.pdf}{BeckPS03} (0.38)& \href{../works/TouatBT22.pdf}{TouatBT22} (0.39)& \href{../works/MonetteDH09.pdf}{MonetteDH09} (0.39)\\
Dot& \cellcolor{red!40}\href{../works/ZarandiASC20.pdf}{ZarandiASC20} (204.00)& \cellcolor{red!40}\href{../works/Groleaz21.pdf}{Groleaz21} (202.00)& \cellcolor{red!40}\href{../works/Baptiste02.pdf}{Baptiste02} (199.00)& \cellcolor{red!40}\href{../works/Dejemeppe16.pdf}{Dejemeppe16} (177.00)& \cellcolor{red!40}\href{../works/PrataAN23.pdf}{PrataAN23} (175.00)\\
Cosine& \cellcolor{red!40}\href{../works/KovacsB11.pdf}{KovacsB11} (0.76)& \cellcolor{red!40}\href{../works/BeckR03.pdf}{BeckR03} (0.75)& \cellcolor{red!40}\href{../works/TouatBT22.pdf}{TouatBT22} (0.74)& \cellcolor{red!40}\href{../works/PrataAN23.pdf}{PrataAN23} (0.73)& \cellcolor{red!40}\href{../works/OrnekO16.pdf}{OrnekO16} (0.72)\\
\index{TerekhovTDB14}\href{../works/TerekhovTDB14.pdf}{TerekhovTDB14} R\&C& \cellcolor{blue!20}\href{../works/NovasH12.pdf}{NovasH12} (0.97)& \cellcolor{black!20}\href{../works/WuBB09.pdf}{WuBB09} (0.98)& \cellcolor{black!20}\href{../works/BidotVLB09.pdf}{BidotVLB09} (0.99)& \cellcolor{black!20}\href{../works/Hooker05.pdf}{Hooker05} (0.99)& \cellcolor{black!20}\href{../works/HarjunkoskiMBC14.pdf}{HarjunkoskiMBC14} (1.00)\\
Euclid& \cellcolor{blue!20}\href{../works/ParkUJR19.pdf}{ParkUJR19} (0.33)& \cellcolor{black!20}\href{../works/TranTDB13.pdf}{TranTDB13} (0.34)& \cellcolor{black!20}\href{../works/Mehdizadeh-Somarin23.pdf}{Mehdizadeh-Somarin23} (0.36)& \cellcolor{black!20}\href{../works/ZhouGL15.pdf}{ZhouGL15} (0.37)& \cellcolor{black!20}\href{../works/TranPZLDB18.pdf}{TranPZLDB18} (0.37)\\
Dot& \cellcolor{red!40}\href{../works/ZarandiASC20.pdf}{ZarandiASC20} (189.00)& \cellcolor{red!40}\href{../works/Groleaz21.pdf}{Groleaz21} (179.00)& \cellcolor{red!40}\href{../works/Baptiste02.pdf}{Baptiste02} (160.00)& \cellcolor{red!40}\href{../works/PrataAN23.pdf}{PrataAN23} (158.00)& \cellcolor{red!40}\href{../works/Astrand21.pdf}{Astrand21} (158.00)\\
Cosine& \cellcolor{red!40}\href{../works/ParkUJR19.pdf}{ParkUJR19} (0.77)& \cellcolor{red!40}\href{../works/TranTDB13.pdf}{TranTDB13} (0.75)& \cellcolor{red!40}\href{../works/Mehdizadeh-Somarin23.pdf}{Mehdizadeh-Somarin23} (0.73)& \cellcolor{red!40}\href{../works/ZhouGL15.pdf}{ZhouGL15} (0.73)& \cellcolor{red!40}\href{../works/TranPZLDB18.pdf}{TranPZLDB18} (0.71)\\
\index{Tesch16}\href{../works/Tesch16.pdf}{Tesch16} R\&C& \cellcolor{red!40}\href{../works/Tesch18.pdf}{Tesch18} (0.55)& \cellcolor{red!40}\href{../works/KameugneFGOQ18.pdf}{KameugneFGOQ18} (0.61)& \cellcolor{red!40}\href{../works/OuelletQ13.pdf}{OuelletQ13} (0.63)& \cellcolor{red!40}\href{../works/OuelletQ18.pdf}{OuelletQ18} (0.63)& \cellcolor{red!40}\href{../works/KameugneF13.pdf}{KameugneF13} (0.69)\\
Euclid& \cellcolor{red!20}\href{../works/DerrienP14.pdf}{DerrienP14} (0.26)& \cellcolor{yellow!20}\href{../works/HeinzS11.pdf}{HeinzS11} (0.27)& \cellcolor{yellow!20}\href{../works/Tesch18.pdf}{Tesch18} (0.27)& \cellcolor{yellow!20}\href{../works/BertholdHLMS10.pdf}{BertholdHLMS10} (0.27)& \cellcolor{yellow!20}\href{../works/WolfS05.pdf}{WolfS05} (0.28)\\
Dot& \cellcolor{red!40}\href{../works/Fahimi16.pdf}{Fahimi16} (102.00)& \cellcolor{red!40}\href{../works/Schutt11.pdf}{Schutt11} (101.00)& \cellcolor{red!40}\href{../works/Tesch18.pdf}{Tesch18} (100.00)& \cellcolor{red!40}\href{../works/Lombardi10.pdf}{Lombardi10} (94.00)& \cellcolor{red!40}\href{../works/Kameugne14.pdf}{Kameugne14} (94.00)\\
Cosine& \cellcolor{red!40}\href{../works/Tesch18.pdf}{Tesch18} (0.82)& \cellcolor{red!40}\href{../works/HeinzS11.pdf}{HeinzS11} (0.75)& \cellcolor{red!40}\href{../works/DerrienP14.pdf}{DerrienP14} (0.75)& \cellcolor{red!40}\href{../works/DerrienPZ14.pdf}{DerrienPZ14} (0.72)& \cellcolor{red!40}\href{../works/BertholdHLMS10.pdf}{BertholdHLMS10} (0.71)\\
\index{Tesch18}\href{../works/Tesch18.pdf}{Tesch18} R\&C& \cellcolor{red!40}\href{../works/OuelletQ18.pdf}{OuelletQ18} (0.38)& \cellcolor{red!40}\href{../works/Tesch16.pdf}{Tesch16} (0.55)& \cellcolor{red!40}\href{../works/OuelletQ13.pdf}{OuelletQ13} (0.62)& \cellcolor{red!40}\href{../works/KameugneF13.pdf}{KameugneF13} (0.65)& \cellcolor{red!40}\href{../works/KameugneFGOQ18.pdf}{KameugneFGOQ18} (0.68)\\
Euclid& \cellcolor{yellow!20}\href{../works/Tesch16.pdf}{Tesch16} (0.27)& \cellcolor{green!20}\href{../works/CarlierPSJ20.pdf}{CarlierPSJ20} (0.30)& \cellcolor{green!20}\href{../works/HanenKP21.pdf}{HanenKP21} (0.30)& \cellcolor{blue!20}\href{../works/Limtanyakul07.pdf}{Limtanyakul07} (0.32)& \cellcolor{blue!20}\href{../works/KameugneFSN11.pdf}{KameugneFSN11} (0.32)\\
Dot& \cellcolor{red!40}\href{../works/Baptiste02.pdf}{Baptiste02} (147.00)& \cellcolor{red!40}\href{../works/Fahimi16.pdf}{Fahimi16} (143.00)& \cellcolor{red!40}\href{../works/Lombardi10.pdf}{Lombardi10} (138.00)& \cellcolor{red!40}\href{../works/Schutt11.pdf}{Schutt11} (134.00)& \cellcolor{red!40}\href{../works/Dejemeppe16.pdf}{Dejemeppe16} (132.00)\\
Cosine& \cellcolor{red!40}\href{../works/Tesch16.pdf}{Tesch16} (0.82)& \cellcolor{red!40}\href{../works/HanenKP21.pdf}{HanenKP21} (0.78)& \cellcolor{red!40}\href{../works/CarlierPSJ20.pdf}{CarlierPSJ20} (0.77)& \cellcolor{red!40}\href{../works/KameugneFSN11.pdf}{KameugneFSN11} (0.74)& \cellcolor{red!40}\href{../works/Limtanyakul07.pdf}{Limtanyakul07} (0.74)\\
\index{ThiruvadyBME09}\href{../works/ThiruvadyBME09.pdf}{ThiruvadyBME09} R\&C& \cellcolor{red!40}\href{../works/MeyerE04.pdf}{MeyerE04} (0.80)& \cellcolor{yellow!20}\href{../works/ZhangLS12.pdf}{ZhangLS12} (0.93)& \cellcolor{green!20}\href{../works/QuirogaZH05.pdf}{QuirogaZH05} (0.93)& \cellcolor{green!20}\href{../works/Geske05.pdf}{Geske05} (0.93)& \cellcolor{green!20}\href{../works/EvenSH15.pdf}{EvenSH15} (0.94)\\
Euclid& \cellcolor{yellow!20}\href{../works/MeyerE04.pdf}{MeyerE04} (0.27)& \cellcolor{green!20}\href{../works/Limtanyakul07.pdf}{Limtanyakul07} (0.29)& \cellcolor{green!20}\href{../works/BenediktSMVH18.pdf}{BenediktSMVH18} (0.30)& \cellcolor{green!20}\href{../works/Sadykov04.pdf}{Sadykov04} (0.30)& \cellcolor{green!20}\href{../works/Hooker17.pdf}{Hooker17} (0.31)\\
Dot& \cellcolor{red!40}\href{../works/Groleaz21.pdf}{Groleaz21} (114.00)& \cellcolor{red!40}\href{../works/ZarandiASC20.pdf}{ZarandiASC20} (108.00)& \cellcolor{red!40}\href{../works/Lunardi20.pdf}{Lunardi20} (95.00)& \cellcolor{red!40}\href{../works/Astrand21.pdf}{Astrand21} (94.00)& \cellcolor{red!40}\href{../works/PrataAN23.pdf}{PrataAN23} (92.00)\\
Cosine& \cellcolor{red!40}\href{../works/MeyerE04.pdf}{MeyerE04} (0.78)& \cellcolor{red!40}\href{../works/BillautHL12.pdf}{BillautHL12} (0.65)& \cellcolor{red!40}\href{../works/Limtanyakul07.pdf}{Limtanyakul07} (0.65)& \cellcolor{red!40}\href{../works/YuraszeckMC23.pdf}{YuraszeckMC23} (0.65)& \cellcolor{red!40}\href{../works/TranAB16.pdf}{TranAB16} (0.65)\\
\index{ThiruvadyWGS14}\href{../works/ThiruvadyWGS14.pdf}{ThiruvadyWGS14} R\&C& \cellcolor{red!40}\href{../works/GuSS13.pdf}{GuSS13} (0.74)& \cellcolor{red!40}GuSSWC14 (0.85)& \cellcolor{red!20}\href{../works/GuSW12.pdf}{GuSW12} (0.88)& \cellcolor{yellow!20}\href{../works/SchuttCSW12.pdf}{SchuttCSW12} (0.92)& \cellcolor{yellow!20}\href{../works/SchnellH15.pdf}{SchnellH15} (0.93)\\
Euclid& \cellcolor{green!20}\href{../works/GuSS13.pdf}{GuSS13} (0.29)& \cellcolor{green!20}\href{../works/GuSW12.pdf}{GuSW12} (0.29)& \cellcolor{green!20}\href{../works/BeniniBGM05a.pdf}{BeniniBGM05a} (0.31)& \cellcolor{blue!20}\href{../works/KovacsV06.pdf}{KovacsV06} (0.32)& \cellcolor{blue!20}\href{../works/HeipckeCCS00.pdf}{HeipckeCCS00} (0.33)\\
Dot& \cellcolor{red!40}\href{../works/ZarandiASC20.pdf}{ZarandiASC20} (137.00)& \cellcolor{red!40}\href{../works/Groleaz21.pdf}{Groleaz21} (129.00)& \cellcolor{red!40}\href{../works/Dejemeppe16.pdf}{Dejemeppe16} (117.00)& \cellcolor{red!40}\href{../works/Lombardi10.pdf}{Lombardi10} (116.00)& \cellcolor{red!40}\href{../works/IsikYA23.pdf}{IsikYA23} (111.00)\\
Cosine& \cellcolor{red!40}\href{../works/GuSS13.pdf}{GuSS13} (0.73)& \cellcolor{red!40}\href{../works/GuSW12.pdf}{GuSW12} (0.72)& \cellcolor{red!40}\href{../works/abs-2402-00459.pdf}{abs-2402-00459} (0.72)& \cellcolor{red!40}\href{../works/HillTV21.pdf}{HillTV21} (0.68)& \cellcolor{red!40}\href{../works/KovacsV06.pdf}{KovacsV06} (0.67)\\
\index{ThomasKS20}\href{../works/ThomasKS20.pdf}{ThomasKS20} R\&C& \cellcolor{red!40}\href{../works/CappartTSR18.pdf}{CappartTSR18} (0.85)& \cellcolor{red!20}\href{../works/MurinR19.pdf}{MurinR19} (0.87)& \cellcolor{yellow!20}\href{../works/LaborieRSV18.pdf}{LaborieRSV18} (0.92)& \cellcolor{yellow!20}\href{../works/Laborie18a.pdf}{Laborie18a} (0.92)& \cellcolor{yellow!20}\href{../works/ColT2019a.pdf}{ColT2019a} (0.92)\\
Euclid& \cellcolor{red!40}\href{../works/CappartTSR18.pdf}{CappartTSR18} (0.22)& \cellcolor{green!20}\href{../works/ZibranR11.pdf}{ZibranR11} (0.29)& \cellcolor{green!20}\href{../works/ZibranR11a.pdf}{ZibranR11a} (0.30)& \cellcolor{green!20}\href{../works/ChapadosJR11.pdf}{ChapadosJR11} (0.30)& \cellcolor{green!20}\href{../works/TranVNB17a.pdf}{TranVNB17a} (0.31)\\
Dot& \cellcolor{red!40}\href{../works/Dejemeppe16.pdf}{Dejemeppe16} (95.00)& \cellcolor{red!40}\href{../works/CappartTSR18.pdf}{CappartTSR18} (94.00)& \cellcolor{red!40}\href{../works/LaborieRSV18.pdf}{LaborieRSV18} (89.00)& \cellcolor{red!40}\href{../works/Groleaz21.pdf}{Groleaz21} (81.00)& \cellcolor{red!40}\href{../works/SacramentoSP20.pdf}{SacramentoSP20} (78.00)\\
Cosine& \cellcolor{red!40}\href{../works/CappartTSR18.pdf}{CappartTSR18} (0.86)& \cellcolor{red!40}\href{../works/GoelSHFS15.pdf}{GoelSHFS15} (0.67)& \cellcolor{red!40}\href{../works/DejemeppeD14.pdf}{DejemeppeD14} (0.65)& \cellcolor{red!40}\href{../works/AalianPG23.pdf}{AalianPG23} (0.63)& \cellcolor{red!40}\href{../works/ZibranR11a.pdf}{ZibranR11a} (0.63)\\
\index{Thorsteinsson01}\href{../works/Thorsteinsson01.pdf}{Thorsteinsson01} R\&C& \cellcolor{red!40}\href{../works/Hooker04.pdf}{Hooker04} (0.72)& \cellcolor{red!40}\href{../works/JainG01.pdf}{JainG01} (0.74)& \cellcolor{red!40}\href{../works/Hooker05b.pdf}{Hooker05b} (0.76)& \cellcolor{red!40}\href{../works/Hooker05.pdf}{Hooker05} (0.77)& \cellcolor{red!40}\href{../works/CambazardHDJT04.pdf}{CambazardHDJT04} (0.78)\\
Euclid& \cellcolor{red!40}\href{../works/HookerO03.pdf}{HookerO03} (0.21)& \cellcolor{red!20}\href{../works/HookerY02.pdf}{HookerY02} (0.26)& \cellcolor{yellow!20}\href{../works/BockmayrP06.pdf}{BockmayrP06} (0.27)& \cellcolor{yellow!20}\href{../works/Hooker05a.pdf}{Hooker05a} (0.27)& \cellcolor{yellow!20}\href{../works/CireCH13.pdf}{CireCH13} (0.27)\\
Dot& \cellcolor{red!40}\href{../works/HookerH17.pdf}{HookerH17} (97.00)& \cellcolor{red!40}\href{../works/Baptiste02.pdf}{Baptiste02} (90.00)& \cellcolor{red!40}\href{../works/Lombardi10.pdf}{Lombardi10} (89.00)& \cellcolor{red!40}\href{../works/MilanoW06.pdf}{MilanoW06} (87.00)& \cellcolor{red!40}\href{../works/MilanoW09.pdf}{MilanoW09} (86.00)\\
Cosine& \cellcolor{red!40}\href{../works/HookerO03.pdf}{HookerO03} (0.83)& \cellcolor{red!40}\href{../works/Hooker06.pdf}{Hooker06} (0.73)& \cellcolor{red!40}\href{../works/JainG01.pdf}{JainG01} (0.73)& \cellcolor{red!40}\href{../works/HamdiL13.pdf}{HamdiL13} (0.72)& \cellcolor{red!40}\href{../works/Hooker05a.pdf}{Hooker05a} (0.72)\\
\index{Timpe02}\href{../works/Timpe02.pdf}{Timpe02} R\&C& \cellcolor{red!20}\href{../works/Hooker06.pdf}{Hooker06} (0.87)& \cellcolor{red!20}\href{../works/MaraveliasG04.pdf}{MaraveliasG04} (0.88)& \cellcolor{red!20}\href{../works/ChuX05.pdf}{ChuX05} (0.89)& \cellcolor{yellow!20}\href{../works/Hooker05a.pdf}{Hooker05a} (0.90)& \cellcolor{yellow!20}\href{../works/CambazardHDJT04.pdf}{CambazardHDJT04} (0.91)\\
Euclid& \cellcolor{yellow!20}\href{../works/BockmayrP06.pdf}{BockmayrP06} (0.28)& \cellcolor{green!20}\href{../works/GilesH16.pdf}{GilesH16} (0.29)& \cellcolor{green!20}\href{../works/AstrandJZ18.pdf}{AstrandJZ18} (0.30)& \cellcolor{green!20}\href{../works/PoderBS04.pdf}{PoderBS04} (0.31)& \cellcolor{green!20}\href{../works/abs-2312-13682.pdf}{abs-2312-13682} (0.31)\\
Dot& \cellcolor{red!40}\href{../works/LaborieRSV18.pdf}{LaborieRSV18} (94.00)& \cellcolor{red!40}\href{../works/Beck99.pdf}{Beck99} (93.00)& \cellcolor{red!40}\href{../works/Astrand21.pdf}{Astrand21} (92.00)& \cellcolor{red!40}\href{../works/ZarandiASC20.pdf}{ZarandiASC20} (91.00)& \cellcolor{red!40}\href{../works/Malapert11.pdf}{Malapert11} (91.00)\\
Cosine& \cellcolor{red!40}\href{../works/BockmayrP06.pdf}{BockmayrP06} (0.70)& \cellcolor{red!40}\href{../works/GilesH16.pdf}{GilesH16} (0.68)& \cellcolor{red!40}\href{../works/AstrandJZ18.pdf}{AstrandJZ18} (0.68)& \cellcolor{red!40}\href{../works/MaraveliasCG04.pdf}{MaraveliasCG04} (0.67)& \cellcolor{red!40}\href{../works/DavenportKRSH07.pdf}{DavenportKRSH07} (0.66)\\
\index{Tom19}\href{../works/Tom19.pdf}{Tom19} R\&C& \cellcolor{blue!20}\href{../works/HoYCLLCLC18.pdf}{HoYCLLCLC18} (0.97)& \cellcolor{blue!20}\href{../works/MusliuSS18.pdf}{MusliuSS18} (0.98)& \cellcolor{blue!20}\href{../works/AntunesABD18.pdf}{AntunesABD18} (0.98)& \cellcolor{black!20}\href{../works/ShinBBHO18.pdf}{ShinBBHO18} (0.98)& \cellcolor{black!20}\href{../works/SubulanC22.pdf}{SubulanC22} (0.98)\\
Euclid& \cellcolor{red!40}\href{../works/AngelsmarkJ00.pdf}{AngelsmarkJ00} (0.22)& \cellcolor{red!40}\href{../works/Caseau97.pdf}{Caseau97} (0.23)& \cellcolor{red!40}\href{../works/BarlattCG08.pdf}{BarlattCG08} (0.23)& \cellcolor{red!40}\href{../works/BridiLBBM16.pdf}{BridiLBBM16} (0.24)& \cellcolor{red!40}\href{../works/KovacsEKV05.pdf}{KovacsEKV05} (0.24)\\
Dot& \cellcolor{red!40}\href{../works/ZarandiASC20.pdf}{ZarandiASC20} (87.00)& \cellcolor{red!40}\href{../works/Dejemeppe16.pdf}{Dejemeppe16} (76.00)& \cellcolor{red!40}\href{../works/Lunardi20.pdf}{Lunardi20} (74.00)& \cellcolor{red!40}\href{../works/Astrand21.pdf}{Astrand21} (73.00)& \cellcolor{red!40}\href{../works/Beck99.pdf}{Beck99} (69.00)\\
Cosine& \cellcolor{red!40}\href{../works/NovasH14.pdf}{NovasH14} (0.70)& \cellcolor{red!40}\href{../works/BridiLBBM16.pdf}{BridiLBBM16} (0.68)& \cellcolor{red!40}\href{../works/Caseau97.pdf}{Caseau97} (0.67)& \cellcolor{red!40}\href{../works/Salido10.pdf}{Salido10} (0.67)& \cellcolor{red!40}\href{../works/KhayatLR06.pdf}{KhayatLR06} (0.66)\\
\index{TopalogluO11}\href{../works/TopalogluO11.pdf}{TopalogluO11} R\&C& \cellcolor{yellow!20}EdisO11a (0.93)& \cellcolor{green!20}\href{../works/Geske05.pdf}{Geske05} (0.94)& \cellcolor{green!20}\href{../works/BosiM2001.pdf}{BosiM2001} (0.94)& \cellcolor{green!20}\href{../works/Simonis07.pdf}{Simonis07} (0.95)& \cellcolor{green!20}\href{../works/Simonis99.pdf}{Simonis99} (0.95)\\
Euclid& \cellcolor{green!20}\href{../works/BourdaisGP03.pdf}{BourdaisGP03} (0.30)& \cellcolor{green!20}\href{../works/DoulabiRP16.pdf}{DoulabiRP16} (0.31)& \cellcolor{green!20}\href{../works/PesantGPR99.pdf}{PesantGPR99} (0.31)& \cellcolor{green!20}\href{../works/ZibranR11.pdf}{ZibranR11} (0.31)& \cellcolor{green!20}\href{../works/ZibranR11a.pdf}{ZibranR11a} (0.31)\\
Dot& \cellcolor{red!40}\href{../works/ZarandiASC20.pdf}{ZarandiASC20} (90.00)& \cellcolor{red!40}\href{../works/Dejemeppe16.pdf}{Dejemeppe16} (85.00)& \cellcolor{red!40}\href{../works/WangMD15.pdf}{WangMD15} (82.00)& \cellcolor{red!40}\href{../works/FrimodigECM23.pdf}{FrimodigECM23} (80.00)& \cellcolor{red!40}\href{../works/MeskensDL13.pdf}{MeskensDL13} (75.00)\\
Cosine& \cellcolor{red!40}\href{../works/FrimodigECM23.pdf}{FrimodigECM23} (0.71)& \cellcolor{red!40}\href{../works/WangMD15.pdf}{WangMD15} (0.70)& \cellcolor{red!40}\href{../works/DoulabiRP16.pdf}{DoulabiRP16} (0.67)& \cellcolor{red!40}\href{../works/MeskensDL13.pdf}{MeskensDL13} (0.65)& \cellcolor{red!40}\href{../works/GhandehariK22.pdf}{GhandehariK22} (0.64)\\
\index{TopalogluSS12}\href{../works/TopalogluSS12.pdf}{TopalogluSS12} R\&C& \cellcolor{red!40}\href{../works/BukchinR18.pdf}{BukchinR18} (0.80)& \cellcolor{red!40}\href{../works/PinarbasiAY19.pdf}{PinarbasiAY19} (0.82)& \cellcolor{red!40}KizilayC20 (0.84)& \cellcolor{red!40}\href{../works/OzturkTHO13.pdf}{OzturkTHO13} (0.85)& \cellcolor{red!20}\href{../works/AlakaPY19.pdf}{AlakaPY19} (0.87)\\
Euclid& \cellcolor{red!20}\href{../works/AbidinK20.pdf}{AbidinK20} (0.25)& \cellcolor{yellow!20}\href{../works/Alaka21.pdf}{Alaka21} (0.26)& \cellcolor{yellow!20}\href{../works/AlakaP23.pdf}{AlakaP23} (0.26)& \cellcolor{yellow!20}\href{../works/AlakaPY19.pdf}{AlakaPY19} (0.27)& \cellcolor{yellow!20}\href{../works/BukchinR18.pdf}{BukchinR18} (0.27)\\
Dot& \cellcolor{red!40}\href{../works/ZarandiASC20.pdf}{ZarandiASC20} (110.00)& \cellcolor{red!40}\href{../works/Groleaz21.pdf}{Groleaz21} (109.00)& \cellcolor{red!40}\href{../works/Lunardi20.pdf}{Lunardi20} (107.00)& \cellcolor{red!40}\href{../works/Dejemeppe16.pdf}{Dejemeppe16} (107.00)& \cellcolor{red!40}\href{../works/Edis21.pdf}{Edis21} (105.00)\\
Cosine& \cellcolor{red!40}\href{../works/AbidinK20.pdf}{AbidinK20} (0.81)& \cellcolor{red!40}\href{../works/Edis21.pdf}{Edis21} (0.79)& \cellcolor{red!40}\href{../works/CilKLO22.pdf}{CilKLO22} (0.77)& \cellcolor{red!40}\href{../works/AlakaP23.pdf}{AlakaP23} (0.76)& \cellcolor{red!40}\href{../works/Alaka21.pdf}{Alaka21} (0.74)\\
\index{TorresL00}\href{../works/TorresL00.pdf}{TorresL00} R\&C& \cellcolor{red!40}\href{../works/SourdN00.pdf}{SourdN00} (0.72)& \cellcolor{red!40}\href{../works/VilimBC05.pdf}{VilimBC05} (0.78)& \cellcolor{red!40}\href{../works/Vilim04.pdf}{Vilim04} (0.78)& \cellcolor{red!40}\href{../works/VilimBC04.pdf}{VilimBC04} (0.83)& \cellcolor{red!40}\href{../works/DemasseyAM05.pdf}{DemasseyAM05} (0.85)\\
Euclid& \cellcolor{yellow!20}\href{../works/KovacsV04.pdf}{KovacsV04} (0.28)& \cellcolor{green!20}\href{../works/NuijtenA94.pdf}{NuijtenA94} (0.29)& \cellcolor{green!20}\href{../works/NuijtenA96.pdf}{NuijtenA96} (0.29)& \cellcolor{green!20}\href{../works/BartakSR08.pdf}{BartakSR08} (0.29)& \cellcolor{green!20}\href{../works/Vilim05.pdf}{Vilim05} (0.29)\\
Dot& \cellcolor{red!40}\href{../works/Baptiste02.pdf}{Baptiste02} (159.00)& \cellcolor{red!40}\href{../works/Fahimi16.pdf}{Fahimi16} (144.00)& \cellcolor{red!40}\href{../works/Dejemeppe16.pdf}{Dejemeppe16} (143.00)& \cellcolor{red!40}\href{../works/Lombardi10.pdf}{Lombardi10} (142.00)& \cellcolor{red!40}\href{../works/Beck99.pdf}{Beck99} (141.00)\\
Cosine& \cellcolor{red!40}\href{../works/BartakSR08.pdf}{BartakSR08} (0.81)& \cellcolor{red!40}\href{../works/KovacsV04.pdf}{KovacsV04} (0.78)& \cellcolor{red!40}\href{../works/NuijtenA94.pdf}{NuijtenA94} (0.78)& \cellcolor{red!40}\href{../works/NuijtenA96.pdf}{NuijtenA96} (0.78)& \cellcolor{red!40}\href{../works/Vilim05.pdf}{Vilim05} (0.77)\\
\index{TouatBT22}\href{../works/TouatBT22.pdf}{TouatBT22} R\&C\\
Euclid& \cellcolor{black!20}\href{../works/QuirogaZH05.pdf}{QuirogaZH05} (0.35)& \cellcolor{black!20}\href{../works/LaborieR14.pdf}{LaborieR14} (0.35)& \cellcolor{black!20}\href{../works/NovasH14.pdf}{NovasH14} (0.35)& \cellcolor{black!20}\href{../works/MonetteDH09.pdf}{MonetteDH09} (0.36)& \cellcolor{black!20}\href{../works/BeckPS03.pdf}{BeckPS03} (0.36)\\
Dot& \cellcolor{red!40}\href{../works/ZarandiASC20.pdf}{ZarandiASC20} (185.00)& \cellcolor{red!40}\href{../works/Groleaz21.pdf}{Groleaz21} (180.00)& \cellcolor{red!40}\href{../works/Dejemeppe16.pdf}{Dejemeppe16} (167.00)& \cellcolor{red!40}\href{../works/LaborieRSV18.pdf}{LaborieRSV18} (162.00)& \cellcolor{red!40}\href{../works/Lunardi20.pdf}{Lunardi20} (160.00)\\
Cosine& \cellcolor{red!40}\href{../works/LaborieR14.pdf}{LaborieR14} (0.74)& \cellcolor{red!40}\href{../works/MonetteDH09.pdf}{MonetteDH09} (0.74)& \cellcolor{red!40}\href{../works/TerekhovDOB12.pdf}{TerekhovDOB12} (0.74)& \cellcolor{red!40}\href{../works/NovasH14.pdf}{NovasH14} (0.73)& \cellcolor{red!40}\href{../works/QuirogaZH05.pdf}{QuirogaZH05} (0.73)\\
\index{Touraivane95}\href{../works/Touraivane95.pdf}{Touraivane95} R\&C& \cellcolor{red!20}\href{../works/Puget95.pdf}{Puget95} (0.88)& \cellcolor{black!20}\href{../works/Pape94.pdf}{Pape94} (0.99)& \cellcolor{black!20}\href{../works/AggounB93.pdf}{AggounB93} (0.99)& \cellcolor{black!20}\href{../works/Davis87.pdf}{Davis87} (1.00)& \cellcolor{black!20}\href{../works/MintonJPL92.pdf}{MintonJPL92} (1.00)\\
Euclid& \cellcolor{red!40}\href{../works/FalaschiGMP97.pdf}{FalaschiGMP97} (0.13)& \cellcolor{red!40}\href{../works/CarchraeBF05.pdf}{CarchraeBF05} (0.22)& \cellcolor{red!40}\href{../works/Baptiste09.pdf}{Baptiste09} (0.22)& \cellcolor{red!40}\href{../works/AngelsmarkJ00.pdf}{AngelsmarkJ00} (0.23)& \cellcolor{red!40}\href{../works/ChapadosJR11.pdf}{ChapadosJR11} (0.23)\\
Dot& \cellcolor{red!40}\href{../works/Simonis99.pdf}{Simonis99} (45.00)& \cellcolor{red!40}\href{../works/Wallace96.pdf}{Wallace96} (44.00)& \cellcolor{red!40}\href{../works/BosiM2001.pdf}{BosiM2001} (42.00)& \cellcolor{red!40}\href{../works/MartinPY01.pdf}{MartinPY01} (41.00)& \cellcolor{red!40}\href{../works/TrentesauxPT01.pdf}{TrentesauxPT01} (40.00)\\
Cosine& \cellcolor{red!40}\href{../works/FalaschiGMP97.pdf}{FalaschiGMP97} (0.87)& \cellcolor{red!40}\href{../works/MartinPY01.pdf}{MartinPY01} (0.69)& \cellcolor{red!40}\href{../works/AbdennadherS99.pdf}{AbdennadherS99} (0.66)& \cellcolor{red!40}\href{../works/PesantGPR99.pdf}{PesantGPR99} (0.63)& \cellcolor{red!40}\href{../works/Simonis95a.pdf}{Simonis95a} (0.59)\\
\index{TranAB16}\href{../works/TranAB16.pdf}{TranAB16} R\&C& \cellcolor{red!40}HechingHK19 (0.75)& \cellcolor{red!40}ZarandiB12 (0.78)& \cellcolor{red!40}\href{../works/Hooker07.pdf}{Hooker07} (0.80)& \cellcolor{red!40}\href{../works/CireCH16.pdf}{CireCH16} (0.81)& \cellcolor{red!40}\href{../works/Beck10.pdf}{Beck10} (0.81)\\
Euclid& \cellcolor{red!40}\href{../works/TranB12.pdf}{TranB12} (0.17)& \cellcolor{green!20}\href{../works/GedikKEK18.pdf}{GedikKEK18} (0.31)& \cellcolor{green!20}\href{../works/GomesM17.pdf}{GomesM17} (0.31)& \cellcolor{black!20}\href{../works/TanT18.pdf}{TanT18} (0.37)& \cellcolor{black!20}\href{../works/EmeretlisTAV17.pdf}{EmeretlisTAV17} (0.37)\\
Dot& \cellcolor{red!40}\href{../works/Groleaz21.pdf}{Groleaz21} (183.00)& \cellcolor{red!40}\href{../works/ZarandiASC20.pdf}{ZarandiASC20} (168.00)& \cellcolor{red!40}\href{../works/TranB12.pdf}{TranB12} (164.00)& \cellcolor{red!40}\href{../works/NaderiRR23.pdf}{NaderiRR23} (163.00)& \cellcolor{red!40}\href{../works/Lunardi20.pdf}{Lunardi20} (153.00)\\
Cosine& \cellcolor{red!40}\href{../works/TranB12.pdf}{TranB12} (0.94)& \cellcolor{red!40}\href{../works/GedikKEK18.pdf}{GedikKEK18} (0.82)& \cellcolor{red!40}\href{../works/GomesM17.pdf}{GomesM17} (0.81)& \cellcolor{red!40}\href{../works/TanT18.pdf}{TanT18} (0.72)& \cellcolor{red!40}\href{../works/EmeretlisTAV17.pdf}{EmeretlisTAV17} (0.72)\\
\index{TranB12}\href{../works/TranB12.pdf}{TranB12} R\&C\\
Euclid& \cellcolor{red!40}\href{../works/TranAB16.pdf}{TranAB16} (0.17)& \cellcolor{green!20}\href{../works/GomesM17.pdf}{GomesM17} (0.29)& \cellcolor{blue!20}\href{../works/GedikKEK18.pdf}{GedikKEK18} (0.32)& \cellcolor{blue!20}\href{../works/TanT18.pdf}{TanT18} (0.33)& \cellcolor{black!20}\href{../works/ArbaouiY18.pdf}{ArbaouiY18} (0.34)\\
Dot& \cellcolor{red!40}\href{../works/TranAB16.pdf}{TranAB16} (164.00)& \cellcolor{red!40}\href{../works/Groleaz21.pdf}{Groleaz21} (157.00)& \cellcolor{red!40}\href{../works/NaderiRR23.pdf}{NaderiRR23} (149.00)& \cellcolor{red!40}\href{../works/ZarandiASC20.pdf}{ZarandiASC20} (144.00)& \cellcolor{red!40}\href{../works/YunusogluY22.pdf}{YunusogluY22} (136.00)\\
Cosine& \cellcolor{red!40}\href{../works/TranAB16.pdf}{TranAB16} (0.94)& \cellcolor{red!40}\href{../works/GomesM17.pdf}{GomesM17} (0.81)& \cellcolor{red!40}\href{../works/GedikKEK18.pdf}{GedikKEK18} (0.80)& \cellcolor{red!40}\href{../works/TanT18.pdf}{TanT18} (0.74)& \cellcolor{red!40}\href{../works/ArbaouiY18.pdf}{ArbaouiY18} (0.72)\\
\index{TranDRFWOVB16}\href{../works/TranDRFWOVB16.pdf}{TranDRFWOVB16} R\&C\\
Euclid& \cellcolor{red!20}\href{../works/Puget95.pdf}{Puget95} (0.25)& \cellcolor{red!20}\href{../works/CrawfordB94.pdf}{CrawfordB94} (0.25)& \cellcolor{red!20}\href{../works/JoLLH99.pdf}{JoLLH99} (0.25)& \cellcolor{red!20}\href{../works/TranWDRFOVB16.pdf}{TranWDRFOVB16} (0.26)& \cellcolor{red!20}\href{../works/AngelsmarkJ00.pdf}{AngelsmarkJ00} (0.26)\\
Dot& \cellcolor{red!40}\href{../works/ZarandiASC20.pdf}{ZarandiASC20} (96.00)& \cellcolor{red!40}\href{../works/HarjunkoskiMBC14.pdf}{HarjunkoskiMBC14} (82.00)& \cellcolor{red!40}\href{../works/Astrand21.pdf}{Astrand21} (81.00)& \cellcolor{red!40}\href{../works/Fahimi16.pdf}{Fahimi16} (81.00)& \cellcolor{red!40}\href{../works/Groleaz21.pdf}{Groleaz21} (80.00)\\
Cosine& \cellcolor{red!40}\href{../works/JoLLH99.pdf}{JoLLH99} (0.75)& \cellcolor{red!40}\href{../works/Puget95.pdf}{Puget95} (0.70)& \cellcolor{red!40}\href{../works/FrankK05.pdf}{FrankK05} (0.70)& \cellcolor{red!40}\href{../works/CrawfordB94.pdf}{CrawfordB94} (0.70)& \cellcolor{red!40}\href{../works/TranWDRFOVB16.pdf}{TranWDRFOVB16} (0.69)\\
\index{TranPZLDB18}\href{../works/TranPZLDB18.pdf}{TranPZLDB18} R\&C& \cellcolor{blue!20}\href{../works/IfrimOS12.pdf}{IfrimOS12} (0.98)& \cellcolor{black!20}\href{../works/ZarandiASC20.pdf}{ZarandiASC20} (1.00)\\
Euclid& \cellcolor{yellow!20}\href{../works/BridiLBBM16.pdf}{BridiLBBM16} (0.26)& \cellcolor{yellow!20}\href{../works/CrawfordB94.pdf}{CrawfordB94} (0.27)& \cellcolor{yellow!20}\href{../works/IfrimOS12.pdf}{IfrimOS12} (0.27)& \cellcolor{green!20}\href{../works/DoRZ08.pdf}{DoRZ08} (0.29)& \cellcolor{green!20}\href{../works/Prosser89.pdf}{Prosser89} (0.29)\\
Dot& \cellcolor{red!40}\href{../works/Groleaz21.pdf}{Groleaz21} (109.00)& \cellcolor{red!40}\href{../works/ZarandiASC20.pdf}{ZarandiASC20} (108.00)& \cellcolor{red!40}\href{../works/Lombardi10.pdf}{Lombardi10} (104.00)& \cellcolor{red!40}\href{../works/Astrand21.pdf}{Astrand21} (97.00)& \cellcolor{red!40}\href{../works/PrataAN23.pdf}{PrataAN23} (96.00)\\
Cosine& \cellcolor{red!40}\href{../works/BridiLBBM16.pdf}{BridiLBBM16} (0.72)& \cellcolor{red!40}\href{../works/IfrimOS12.pdf}{IfrimOS12} (0.71)& \cellcolor{red!40}\href{../works/TerekhovTDB14.pdf}{TerekhovTDB14} (0.71)& \cellcolor{red!40}\href{../works/CrawfordB94.pdf}{CrawfordB94} (0.71)& \cellcolor{red!40}\href{../works/WuBB09.pdf}{WuBB09} (0.70)\\
\index{TranTDB13}\href{../works/TranTDB13.pdf}{TranTDB13} R\&C& \cellcolor{blue!20}\href{../works/WuBB09.pdf}{WuBB09} (0.98)& \cellcolor{black!20}\href{../works/BidotVLB09.pdf}{BidotVLB09} (0.98)& \cellcolor{black!20}\href{../works/Hooker05.pdf}{Hooker05} (0.99)\\
Euclid& \cellcolor{blue!20}\href{../works/TranPZLDB18.pdf}{TranPZLDB18} (0.34)& \cellcolor{blue!20}\href{../works/DoRZ08.pdf}{DoRZ08} (0.34)& \cellcolor{black!20}\href{../works/TerekhovTDB14.pdf}{TerekhovTDB14} (0.34)& \cellcolor{black!20}\href{../works/CarlssonKA99.pdf}{CarlssonKA99} (0.34)& \cellcolor{black!20}\href{../works/HebrardHJMPV16.pdf}{HebrardHJMPV16} (0.35)\\
Dot& \cellcolor{red!40}\href{../works/ZarandiASC20.pdf}{ZarandiASC20} (126.00)& \cellcolor{red!40}\href{../works/Groleaz21.pdf}{Groleaz21} (122.00)& \cellcolor{red!40}\href{../works/TerekhovTDB14.pdf}{TerekhovTDB14} (116.00)& \cellcolor{red!40}\href{../works/Astrand21.pdf}{Astrand21} (109.00)& \cellcolor{red!40}\href{../works/Baptiste02.pdf}{Baptiste02} (105.00)\\
Cosine& \cellcolor{red!40}\href{../works/TerekhovTDB14.pdf}{TerekhovTDB14} (0.75)& \cellcolor{red!40}\href{../works/ParkUJR19.pdf}{ParkUJR19} (0.66)& \cellcolor{red!40}\href{../works/TranPZLDB18.pdf}{TranPZLDB18} (0.66)& \cellcolor{red!40}\href{../works/HebrardHJMPV16.pdf}{HebrardHJMPV16} (0.63)& \cellcolor{red!40}\href{../works/DoRZ08.pdf}{DoRZ08} (0.63)\\
\index{TranVNB17}\href{../works/TranVNB17.pdf}{TranVNB17} R\&C& \cellcolor{yellow!20}\href{../works/GilesH16.pdf}{GilesH16} (0.93)& \cellcolor{yellow!20}\href{../works/CappartS17.pdf}{CappartS17} (0.93)& \cellcolor{green!20}\href{../works/Hooker17.pdf}{Hooker17} (0.94)& \cellcolor{green!20}\href{../works/GayHS15.pdf}{GayHS15} (0.95)& \cellcolor{blue!20}\href{../works/CireCH16.pdf}{CireCH16} (0.96)\\
Euclid& \cellcolor{red!40}\href{../works/BoothNB16.pdf}{BoothNB16} (0.19)& \cellcolor{red!40}\href{../works/TranVNB17a.pdf}{TranVNB17a} (0.22)& \cellcolor{yellow!20}\href{../works/NishikawaSTT19.pdf}{NishikawaSTT19} (0.28)& \cellcolor{yellow!20}\href{../works/WolfS05.pdf}{WolfS05} (0.28)& \cellcolor{yellow!20}\href{../works/NishikawaSTT18a.pdf}{NishikawaSTT18a} (0.28)\\
Dot& \cellcolor{red!40}\href{../works/LaborieRSV18.pdf}{LaborieRSV18} (117.00)& \cellcolor{red!40}\href{../works/ZarandiASC20.pdf}{ZarandiASC20} (106.00)& \cellcolor{red!40}\href{../works/Dejemeppe16.pdf}{Dejemeppe16} (106.00)& \cellcolor{red!40}\href{../works/Lunardi20.pdf}{Lunardi20} (103.00)& \cellcolor{red!40}\href{../works/Astrand21.pdf}{Astrand21} (102.00)\\
Cosine& \cellcolor{red!40}\href{../works/BoothNB16.pdf}{BoothNB16} (0.88)& \cellcolor{red!40}\href{../works/TranVNB17a.pdf}{TranVNB17a} (0.84)& \cellcolor{red!40}\href{../works/NishikawaSTT19.pdf}{NishikawaSTT19} (0.75)& \cellcolor{red!40}\href{../works/NishikawaSTT18a.pdf}{NishikawaSTT18a} (0.73)& \cellcolor{red!40}\href{../works/BoothTNB16.pdf}{BoothTNB16} (0.72)\\
\index{TranVNB17a}\href{../works/TranVNB17a.pdf}{TranVNB17a} R\&C\\
Euclid& \cellcolor{red!40}\href{../works/BoothNB16.pdf}{BoothNB16} (0.19)& \cellcolor{red!40}\href{../works/ChapadosJR11.pdf}{ChapadosJR11} (0.22)& \cellcolor{red!40}\href{../works/TranVNB17.pdf}{TranVNB17} (0.22)& \cellcolor{red!40}\href{../works/BeniniBGM05a.pdf}{BeniniBGM05a} (0.23)& \cellcolor{red!40}\href{../works/ZibranR11.pdf}{ZibranR11} (0.23)\\
Dot& \cellcolor{red!40}\href{../works/LaborieRSV18.pdf}{LaborieRSV18} (79.00)& \cellcolor{red!40}\href{../works/Dejemeppe16.pdf}{Dejemeppe16} (75.00)& \cellcolor{red!40}\href{../works/ZarandiASC20.pdf}{ZarandiASC20} (72.00)& \cellcolor{red!40}\href{../works/Astrand21.pdf}{Astrand21} (72.00)& \cellcolor{red!40}\href{../works/Lunardi20.pdf}{Lunardi20} (71.00)\\
Cosine& \cellcolor{red!40}\href{../works/TranVNB17.pdf}{TranVNB17} (0.84)& \cellcolor{red!40}\href{../works/BoothNB16.pdf}{BoothNB16} (0.84)& \cellcolor{red!40}\href{../works/BoothTNB16.pdf}{BoothTNB16} (0.78)& \cellcolor{red!40}\href{../works/AstrandJZ18.pdf}{AstrandJZ18} (0.74)& \cellcolor{red!40}\href{../works/GilesH16.pdf}{GilesH16} (0.74)\\
\index{TranWDRFOVB16}\href{../works/TranWDRFOVB16.pdf}{TranWDRFOVB16} R\&C\\
Euclid& \cellcolor{red!40}\href{../works/LouieVNB14.pdf}{LouieVNB14} (0.23)& \cellcolor{red!20}\href{../works/Rit86.pdf}{Rit86} (0.24)& \cellcolor{red!20}\href{../works/AngelsmarkJ00.pdf}{AngelsmarkJ00} (0.25)& \cellcolor{red!20}\href{../works/Davis87.pdf}{Davis87} (0.25)& \cellcolor{red!20}\href{../works/LuoVLBM16.pdf}{LuoVLBM16} (0.25)\\
Dot& \cellcolor{red!40}\href{../works/ZarandiASC20.pdf}{ZarandiASC20} (84.00)& \cellcolor{red!40}\href{../works/Groleaz21.pdf}{Groleaz21} (80.00)& \cellcolor{red!40}\href{../works/LaborieRSV18.pdf}{LaborieRSV18} (79.00)& \cellcolor{red!40}\href{../works/Astrand21.pdf}{Astrand21} (79.00)& \cellcolor{red!40}\href{../works/Dejemeppe16.pdf}{Dejemeppe16} (79.00)\\
Cosine& \cellcolor{red!40}\href{../works/LouieVNB14.pdf}{LouieVNB14} (0.70)& \cellcolor{red!40}\href{../works/TranDRFWOVB16.pdf}{TranDRFWOVB16} (0.69)& \cellcolor{red!40}\href{../works/BeckPS03.pdf}{BeckPS03} (0.68)& \cellcolor{red!40}\href{../works/KotaryFH22.pdf}{KotaryFH22} (0.67)& \cellcolor{red!40}\href{../works/HeipckeCCS00.pdf}{HeipckeCCS00} (0.67)\\
\index{TrentesauxPT01}\href{../works/TrentesauxPT01.pdf}{TrentesauxPT01} R\&C& \cellcolor{green!20}\href{../works/Goltz95.pdf}{Goltz95} (0.94)& \cellcolor{green!20}\href{../works/RodosekW98.pdf}{RodosekW98} (0.95)& \cellcolor{green!20}\href{../works/AggounB93.pdf}{AggounB93} (0.95)& \cellcolor{green!20}\href{../works/Simonis95a.pdf}{Simonis95a} (0.95)& \cellcolor{green!20}\href{../works/BosiM2001.pdf}{BosiM2001} (0.95)\\
Euclid& \cellcolor{blue!20}\href{../works/Goltz95.pdf}{Goltz95} (0.33)& \cellcolor{blue!20}\href{../works/Pape94.pdf}{Pape94} (0.34)& \cellcolor{black!20}\href{../works/WikarekS19.pdf}{WikarekS19} (0.35)& \href{../works/DincbasSH90.pdf}{DincbasSH90} (0.38)& \href{../works/HentenryckM04.pdf}{HentenryckM04} (0.39)\\
Dot& \cellcolor{red!40}\href{../works/ZarandiASC20.pdf}{ZarandiASC20} (150.00)& \cellcolor{red!40}\href{../works/Beck99.pdf}{Beck99} (145.00)& \cellcolor{red!40}\href{../works/Malapert11.pdf}{Malapert11} (140.00)& \cellcolor{red!40}\href{../works/Baptiste02.pdf}{Baptiste02} (136.00)& \cellcolor{red!40}\href{../works/Astrand21.pdf}{Astrand21} (126.00)\\
Cosine& \cellcolor{red!40}\href{../works/Goltz95.pdf}{Goltz95} (0.77)& \cellcolor{red!40}\href{../works/Pape94.pdf}{Pape94} (0.76)& \cellcolor{red!40}\href{../works/WikarekS19.pdf}{WikarekS19} (0.73)& \cellcolor{red!40}\href{../works/BosiM2001.pdf}{BosiM2001} (0.70)& \cellcolor{red!40}\href{../works/AggounB93.pdf}{AggounB93} (0.68)\\
\index{Trick03}\href{../works/Trick03.pdf}{Trick03} R\&C& \cellcolor{red!40}\href{../works/HenzMT04.pdf}{HenzMT04} (0.69)& \cellcolor{red!40}\href{../works/RussellU06.pdf}{RussellU06} (0.73)& \cellcolor{red!40}\href{../works/RasmussenT06.pdf}{RasmussenT06} (0.73)& \cellcolor{red!40}\href{../works/RasmussenT07.pdf}{RasmussenT07} (0.73)& \cellcolor{red!40}Henz01 (0.73)\\
Euclid& \cellcolor{red!40}\href{../works/SuCC13.pdf}{SuCC13} (0.18)& \cellcolor{red!40}\href{../works/EastonNT02.pdf}{EastonNT02} (0.19)& \cellcolor{red!40}\href{../works/RasmussenT06.pdf}{RasmussenT06} (0.19)& \cellcolor{red!40}\href{../works/ElfJR03.pdf}{ElfJR03} (0.21)& \cellcolor{red!40}\href{../works/NaqviAIAAA22.pdf}{NaqviAIAAA22} (0.23)\\
Dot& \cellcolor{red!40}\href{../works/ZarandiASC20.pdf}{ZarandiASC20} (71.00)& \cellcolor{red!40}\href{../works/KendallKRU10.pdf}{KendallKRU10} (70.00)& \cellcolor{red!40}\href{../works/RussellU06.pdf}{RussellU06} (63.00)& \cellcolor{red!40}\href{../works/RasmussenT09.pdf}{RasmussenT09} (62.00)& \cellcolor{red!40}\href{../works/Ribeiro12.pdf}{Ribeiro12} (61.00)\\
Cosine& \cellcolor{red!40}\href{../works/SuCC13.pdf}{SuCC13} (0.83)& \cellcolor{red!40}\href{../works/EastonNT02.pdf}{EastonNT02} (0.82)& \cellcolor{red!40}\href{../works/RasmussenT06.pdf}{RasmussenT06} (0.81)& \cellcolor{red!40}\href{../works/ElfJR03.pdf}{ElfJR03} (0.76)& \cellcolor{red!40}\href{../works/NaqviAIAAA22.pdf}{NaqviAIAAA22} (0.75)\\
\index{Trick11}Trick11 R\&C& \cellcolor{red!40}\href{../works/ZengM12.pdf}{ZengM12} (0.75)& \cellcolor{red!40}\href{../works/RasmussenT09.pdf}{RasmussenT09} (0.78)& \cellcolor{red!40}\href{../works/RasmussenT06.pdf}{RasmussenT06} (0.79)& \cellcolor{red!40}\href{../works/Ribeiro12.pdf}{Ribeiro12} (0.79)& \cellcolor{red!40}\href{../works/RasmussenT07.pdf}{RasmussenT07} (0.83)\\
Euclid\\
Dot\\
Cosine\\
\index{TrojetHL11}\href{../works/TrojetHL11.pdf}{TrojetHL11} R\&C& \cellcolor{red!20}\href{../works/TerekhovDOB12.pdf}{TerekhovDOB12} (0.90)& \cellcolor{yellow!20}\href{../works/AmadiniGM16.pdf}{AmadiniGM16} (0.91)& \cellcolor{yellow!20}\href{../works/Rodriguez07.pdf}{Rodriguez07} (0.91)& \cellcolor{yellow!20}\href{../works/NuijtenA96.pdf}{NuijtenA96} (0.91)& \cellcolor{yellow!20}\href{../works/Zhou96.pdf}{Zhou96} (0.92)\\
Euclid& \cellcolor{yellow!20}\href{../works/ChuGNSW13.pdf}{ChuGNSW13} (0.27)& \cellcolor{green!20}\href{../works/KovacsV04.pdf}{KovacsV04} (0.29)& \cellcolor{green!20}\href{../works/LombardiM10.pdf}{LombardiM10} (0.29)& \cellcolor{green!20}\href{../works/PoderBS04.pdf}{PoderBS04} (0.29)& \cellcolor{green!20}\href{../works/GayHLS15.pdf}{GayHLS15} (0.30)\\
Dot& \cellcolor{red!40}\href{../works/Dejemeppe16.pdf}{Dejemeppe16} (139.00)& \cellcolor{red!40}\href{../works/Lombardi10.pdf}{Lombardi10} (138.00)& \cellcolor{red!40}\href{../works/Baptiste02.pdf}{Baptiste02} (138.00)& \cellcolor{red!40}\href{../works/Schutt11.pdf}{Schutt11} (136.00)& \cellcolor{red!40}\href{../works/Godet21a.pdf}{Godet21a} (135.00)\\
Cosine& \cellcolor{red!40}\href{../works/ChuGNSW13.pdf}{ChuGNSW13} (0.79)& \cellcolor{red!40}\href{../works/KovacsV04.pdf}{KovacsV04} (0.77)& \cellcolor{red!40}\href{../works/LombardiM10.pdf}{LombardiM10} (0.77)& \cellcolor{red!40}\href{../works/PoderBS04.pdf}{PoderBS04} (0.76)& \cellcolor{red!40}\href{../works/GayHLS15.pdf}{GayHLS15} (0.74)\\
\index{Tsang03}\href{../works/Tsang03.pdf}{Tsang03} R\&C& \cellcolor{blue!20}\href{../works/LombardiM12.pdf}{LombardiM12} (0.98)& \cellcolor{black!20}\href{../works/JainG01.pdf}{JainG01} (1.00)\\
Euclid& \cellcolor{red!40}\href{../works/KovacsEKV05.pdf}{KovacsEKV05} (0.15)& \cellcolor{red!40}\href{../works/ChapadosJR11.pdf}{ChapadosJR11} (0.17)& \cellcolor{red!40}\href{../works/Baptiste09.pdf}{Baptiste09} (0.18)& \cellcolor{red!40}\href{../works/CarchraeBF05.pdf}{CarchraeBF05} (0.18)& \cellcolor{red!40}\href{../works/LimAHO02a.pdf}{LimAHO02a} (0.18)\\
Dot& \cellcolor{red!40}\href{../works/Lemos21.pdf}{Lemos21} (32.00)& \cellcolor{red!40}\href{../works/Astrand21.pdf}{Astrand21} (31.00)& \cellcolor{red!40}\href{../works/Godet21a.pdf}{Godet21a} (31.00)& \cellcolor{red!40}\href{../works/ElkhyariGJ02a.pdf}{ElkhyariGJ02a} (31.00)& \cellcolor{red!40}\href{../works/BartakSR10.pdf}{BartakSR10} (31.00)\\
Cosine& \cellcolor{red!40}\href{../works/KovacsEKV05.pdf}{KovacsEKV05} (0.68)& \cellcolor{red!40}\href{../works/DilkinaH04.pdf}{DilkinaH04} (0.65)& \cellcolor{red!40}\href{../works/BartakS11.pdf}{BartakS11} (0.58)& \cellcolor{red!40}\href{../works/PesantGPR99.pdf}{PesantGPR99} (0.58)& \cellcolor{red!40}\href{../works/ZhangLS12.pdf}{ZhangLS12} (0.58)\\
\index{TsurutaS00}TsurutaS00 R\&C\\
Euclid\\
Dot\\
Cosine\\
\index{UnsalO13}\href{../works/UnsalO13.pdf}{UnsalO13} R\&C& \cellcolor{red!40}\href{../works/SunTB19.pdf}{SunTB19} (0.75)& \cellcolor{red!40}\href{../works/QinDCS20.pdf}{QinDCS20} (0.77)& \cellcolor{yellow!20}\href{../works/UnsalO19.pdf}{UnsalO19} (0.91)& \cellcolor{yellow!20}\href{../works/ZampelliVSDR13.pdf}{ZampelliVSDR13} (0.91)& \cellcolor{green!20}\href{../works/LimRX04.pdf}{LimRX04} (0.94)\\
Euclid& \cellcolor{black!20}\href{../works/NishikawaSTT19.pdf}{NishikawaSTT19} (0.37)& \href{../works/OzturkTHO12.pdf}{OzturkTHO12} (0.38)& \href{../works/SunTB19.pdf}{SunTB19} (0.38)& \href{../works/HeipckeCCS00.pdf}{HeipckeCCS00} (0.38)& \href{../works/ZampelliVSDR13.pdf}{ZampelliVSDR13} (0.39)\\
Dot& \cellcolor{red!40}\href{../works/Dejemeppe16.pdf}{Dejemeppe16} (168.00)& \cellcolor{red!40}\href{../works/Astrand21.pdf}{Astrand21} (165.00)& \cellcolor{red!40}\href{../works/Lunardi20.pdf}{Lunardi20} (156.00)& \cellcolor{red!40}\href{../works/Malapert11.pdf}{Malapert11} (156.00)& \cellcolor{red!40}\href{../works/ZarandiASC20.pdf}{ZarandiASC20} (155.00)\\
Cosine& \cellcolor{red!40}\href{../works/NishikawaSTT19.pdf}{NishikawaSTT19} (0.71)& \cellcolor{red!40}\href{../works/PovedaAA23.pdf}{PovedaAA23} (0.70)& \cellcolor{red!40}\href{../works/SacramentoSP20.pdf}{SacramentoSP20} (0.69)& \cellcolor{red!40}\href{../works/SunTB19.pdf}{SunTB19} (0.69)& \cellcolor{red!40}\href{../works/OzturkTHO12.pdf}{OzturkTHO12} (0.68)\\
\index{UnsalO19}\href{../works/UnsalO19.pdf}{UnsalO19} R\&C& \cellcolor{red!40}\href{../works/CireCH16.pdf}{CireCH16} (0.85)& \cellcolor{red!20}\href{../works/QinDS16.pdf}{QinDS16} (0.86)& \cellcolor{red!20}\href{../works/SunTB19.pdf}{SunTB19} (0.87)& \cellcolor{yellow!20}\href{../works/QinDCS20.pdf}{QinDCS20} (0.91)& \cellcolor{yellow!20}\href{../works/UnsalO13.pdf}{UnsalO13} (0.91)\\
Euclid& \cellcolor{green!20}\href{../works/CorreaLR07.pdf}{CorreaLR07} (0.30)& \cellcolor{green!20}\href{../works/BeniniLMMR08.pdf}{BeniniLMMR08} (0.30)& \cellcolor{green!20}\href{../works/QinDS16.pdf}{QinDS16} (0.31)& \cellcolor{green!20}\href{../works/BeniniLMR11.pdf}{BeniniLMR11} (0.31)& \cellcolor{blue!20}\href{../works/BoothNB16.pdf}{BoothNB16} (0.32)\\
Dot& \cellcolor{red!40}\href{../works/Lombardi10.pdf}{Lombardi10} (118.00)& \cellcolor{red!40}\href{../works/ZarandiASC20.pdf}{ZarandiASC20} (109.00)& \cellcolor{red!40}\href{../works/LombardiM12.pdf}{LombardiM12} (109.00)& \cellcolor{red!40}\href{../works/QinDS16.pdf}{QinDS16} (109.00)& \cellcolor{red!40}\href{../works/Froger16.pdf}{Froger16} (108.00)\\
Cosine& \cellcolor{red!40}\href{../works/QinDS16.pdf}{QinDS16} (0.77)& \cellcolor{red!40}\href{../works/CorreaLR07.pdf}{CorreaLR07} (0.74)& \cellcolor{red!40}\href{../works/BeniniLMR11.pdf}{BeniniLMR11} (0.73)& \cellcolor{red!40}\href{../works/BeniniLMMR08.pdf}{BeniniLMMR08} (0.72)& \cellcolor{red!40}\href{../works/TranVNB17.pdf}{TranVNB17} (0.68)\\
\index{Valdes87}\href{../works/Valdes87.pdf}{Valdes87} R\&C\\
Euclid& \cellcolor{red!40}\href{../works/Davis87.pdf}{Davis87} (0.12)& \cellcolor{red!40}\href{../works/KameugneF13.pdf}{KameugneF13} (0.17)& \cellcolor{red!40}\href{../works/LiuJ06.pdf}{LiuJ06} (0.18)& \cellcolor{red!40}\href{../works/FrostD98.pdf}{FrostD98} (0.19)& \cellcolor{red!40}\href{../works/Rit86.pdf}{Rit86} (0.19)\\
Dot& \cellcolor{red!40}\href{../works/Lombardi10.pdf}{Lombardi10} (32.00)& \cellcolor{red!40}\href{../works/Siala15a.pdf}{Siala15a} (31.00)& \cellcolor{red!40}\href{../works/Schutt11.pdf}{Schutt11} (31.00)& \cellcolor{red!40}\href{../works/Astrand21.pdf}{Astrand21} (30.00)& \cellcolor{red!40}\href{../works/Malapert11.pdf}{Malapert11} (30.00)\\
Cosine& \cellcolor{red!40}\href{../works/Davis87.pdf}{Davis87} (0.77)& \cellcolor{red!40}\href{../works/Rit86.pdf}{Rit86} (0.63)& \cellcolor{red!40}\href{../works/MoffittPP05.pdf}{MoffittPP05} (0.59)& \cellcolor{red!40}\href{../works/DincbasSH90.pdf}{DincbasSH90} (0.56)& \cellcolor{red!40}\href{../works/LiuJ06.pdf}{LiuJ06} (0.52)\\
\index{ValleMGT03}\href{../works/ValleMGT03.pdf}{ValleMGT03} R\&C& \cellcolor{red!40}\href{../works/PinarbasiAY19.pdf}{PinarbasiAY19} (0.83)& \cellcolor{red!40}PinarbasiA20 (0.86)& \cellcolor{red!20}\href{../works/AlakaPY19.pdf}{AlakaPY19} (0.86)& \cellcolor{red!20}\href{../works/TopalogluSS12.pdf}{TopalogluSS12} (0.87)& \cellcolor{red!20}\href{../works/ZhangLS12.pdf}{ZhangLS12} (0.88)\\
Euclid& \cellcolor{red!40}\href{../works/VanczaM01.pdf}{VanczaM01} (0.22)& \cellcolor{red!20}\href{../works/abs-1901-07914.pdf}{abs-1901-07914} (0.24)& \cellcolor{red!20}\href{../works/Bartak02a.pdf}{Bartak02a} (0.25)& \cellcolor{red!20}\href{../works/SmithC93.pdf}{SmithC93} (0.25)& \cellcolor{red!20}\href{../works/ChuGNSW13.pdf}{ChuGNSW13} (0.25)\\
Dot& \cellcolor{red!40}\href{../works/ZarandiASC20.pdf}{ZarandiASC20} (93.00)& \cellcolor{red!40}\href{../works/BartakSR10.pdf}{BartakSR10} (93.00)& \cellcolor{red!40}\href{../works/Astrand21.pdf}{Astrand21} (92.00)& \cellcolor{red!40}\href{../works/Beck99.pdf}{Beck99} (89.00)& \cellcolor{red!40}\href{../works/Dejemeppe16.pdf}{Dejemeppe16} (88.00)\\
Cosine& \cellcolor{red!40}\href{../works/VanczaM01.pdf}{VanczaM01} (0.80)& \cellcolor{red!40}\href{../works/PengLC14.pdf}{PengLC14} (0.78)& \cellcolor{red!40}\href{../works/Bartak02a.pdf}{Bartak02a} (0.77)& \cellcolor{red!40}\href{../works/abs-1901-07914.pdf}{abs-1901-07914} (0.77)& \cellcolor{red!40}\href{../works/PinarbasiAY19.pdf}{PinarbasiAY19} (0.76)\\
\index{VanczaM01}\href{../works/VanczaM01.pdf}{VanczaM01} R\&C\\
Euclid& \cellcolor{red!40}\href{../works/Alaka21.pdf}{Alaka21} (0.20)& \cellcolor{red!40}\href{../works/ValleMGT03.pdf}{ValleMGT03} (0.22)& \cellcolor{red!40}\href{../works/AlakaPY19.pdf}{AlakaPY19} (0.22)& \cellcolor{red!20}\href{../works/NishikawaSTT18a.pdf}{NishikawaSTT18a} (0.24)& \cellcolor{red!20}\href{../works/AlakaP23.pdf}{AlakaP23} (0.25)\\
Dot& \cellcolor{red!40}\href{../works/ZarandiASC20.pdf}{ZarandiASC20} (96.00)& \cellcolor{red!40}\href{../works/Astrand21.pdf}{Astrand21} (96.00)& \cellcolor{red!40}\href{../works/Dejemeppe16.pdf}{Dejemeppe16} (93.00)& \cellcolor{red!40}\href{../works/Lunardi20.pdf}{Lunardi20} (91.00)& \cellcolor{red!40}\href{../works/Beck99.pdf}{Beck99} (88.00)\\
Cosine& \cellcolor{red!40}\href{../works/Alaka21.pdf}{Alaka21} (0.84)& \cellcolor{red!40}\href{../works/ValleMGT03.pdf}{ValleMGT03} (0.80)& \cellcolor{red!40}\href{../works/AlakaPY19.pdf}{AlakaPY19} (0.79)& \cellcolor{red!40}\href{../works/AlakaP23.pdf}{AlakaP23} (0.77)& \cellcolor{red!40}\href{../works/NishikawaSTT18a.pdf}{NishikawaSTT18a} (0.76)\\
\index{VerfaillieL01}\href{../works/VerfaillieL01.pdf}{VerfaillieL01} R\&C& \cellcolor{red!40}\href{../works/OddiPCC03.pdf}{OddiPCC03} (0.81)& \cellcolor{red!20}\href{../works/DannaP03.pdf}{DannaP03} (0.89)& \cellcolor{yellow!20}\href{../works/SchausHMCMD11.pdf}{SchausHMCMD11} (0.91)& \cellcolor{yellow!20}\href{../works/GarganiR07.pdf}{GarganiR07} (0.91)& \cellcolor{yellow!20}\href{../works/PesantRR15.pdf}{PesantRR15} (0.92)\\
Euclid& \cellcolor{red!40}\href{../works/KucukY19.pdf}{KucukY19} (0.23)& \cellcolor{red!20}\href{../works/ZibranR11.pdf}{ZibranR11} (0.25)& \cellcolor{red!20}\href{../works/FrankDT16.pdf}{FrankDT16} (0.26)& \cellcolor{red!20}\href{../works/ZibranR11a.pdf}{ZibranR11a} (0.26)& \cellcolor{yellow!20}\href{../works/BensanaLV99.pdf}{BensanaLV99} (0.27)\\
Dot& \cellcolor{red!40}\href{../works/ZarandiASC20.pdf}{ZarandiASC20} (78.00)& \cellcolor{red!40}\href{../works/Groleaz21.pdf}{Groleaz21} (73.00)& \cellcolor{red!40}\href{../works/LaborieRSV18.pdf}{LaborieRSV18} (71.00)& \cellcolor{red!40}\href{../works/Astrand21.pdf}{Astrand21} (71.00)& \cellcolor{red!40}\href{../works/Dejemeppe16.pdf}{Dejemeppe16} (69.00)\\
Cosine& \cellcolor{red!40}\href{../works/KucukY19.pdf}{KucukY19} (0.75)& \cellcolor{red!40}\href{../works/ZibranR11.pdf}{ZibranR11} (0.66)& \cellcolor{red!40}\href{../works/FrankDT16.pdf}{FrankDT16} (0.66)& \cellcolor{red!40}\href{../works/ZibranR11a.pdf}{ZibranR11a} (0.66)& \cellcolor{red!40}\href{../works/PraletLJ15.pdf}{PraletLJ15} (0.65)\\
\index{Vilim02}\href{../works/Vilim02.pdf}{Vilim02} R\&C& \cellcolor{red!40}\href{../works/Vilim04.pdf}{Vilim04} (0.76)& \cellcolor{red!20}\href{../works/CauwelaertDS20.pdf}{CauwelaertDS20} (0.88)& \cellcolor{red!20}\href{../works/Wolf03.pdf}{Wolf03} (0.88)& \cellcolor{red!20}\href{../works/SchuttS16.pdf}{SchuttS16} (0.89)& \cellcolor{red!20}\href{../works/VilimBC04.pdf}{VilimBC04} (0.89)\\
Euclid& \cellcolor{red!40}\href{../works/KovacsEKV05.pdf}{KovacsEKV05} (0.15)& \cellcolor{red!40}\href{../works/CestaOS98.pdf}{CestaOS98} (0.17)& \cellcolor{red!40}\href{../works/Baptiste09.pdf}{Baptiste09} (0.17)& \cellcolor{red!40}\href{../works/Caballero23.pdf}{Caballero23} (0.18)& \cellcolor{red!40}\href{../works/WuBB05.pdf}{WuBB05} (0.19)\\
Dot& \cellcolor{red!40}\href{../works/Malapert11.pdf}{Malapert11} (43.00)& \cellcolor{red!40}\href{../works/Dejemeppe16.pdf}{Dejemeppe16} (42.00)& \cellcolor{red!40}\href{../works/Baptiste02.pdf}{Baptiste02} (40.00)& \cellcolor{red!40}\href{../works/ZarandiASC20.pdf}{ZarandiASC20} (39.00)& \cellcolor{red!40}\href{../works/LaborieRSV18.pdf}{LaborieRSV18} (39.00)\\
Cosine& \cellcolor{red!40}\href{../works/Vilim04.pdf}{Vilim04} (0.70)& \cellcolor{red!40}\href{../works/CauwelaertDMS16.pdf}{CauwelaertDMS16} (0.68)& \cellcolor{red!40}\href{../works/KovacsEKV05.pdf}{KovacsEKV05} (0.67)& \cellcolor{red!40}\href{../works/CauwelaertDS20.pdf}{CauwelaertDS20} (0.65)& \cellcolor{red!40}\href{../works/FocacciLN00.pdf}{FocacciLN00} (0.63)\\
\index{Vilim03}\href{../works/Vilim03.pdf}{Vilim03} R\&C& \cellcolor{red!40}\href{../works/KovacsV06.pdf}{KovacsV06} (0.67)& \cellcolor{red!20}\href{../works/Vilim05.pdf}{Vilim05} (0.89)& \cellcolor{green!20}\href{../works/JussienL02.pdf}{JussienL02} (0.94)& \cellcolor{green!20}\href{../works/ElkhyariGJ02a.pdf}{ElkhyariGJ02a} (0.95)& \cellcolor{green!20}\href{../works/MalapertCGJLR12.pdf}{MalapertCGJLR12} (0.95)\\
Euclid& \cellcolor{red!40}\href{../works/HebrardTW05.pdf}{HebrardTW05} (0.16)& \cellcolor{red!40}\href{../works/Baptiste09.pdf}{Baptiste09} (0.17)& \cellcolor{red!40}\href{../works/AngelsmarkJ00.pdf}{AngelsmarkJ00} (0.18)& \cellcolor{red!40}\href{../works/AbrilSB05.pdf}{AbrilSB05} (0.18)& \cellcolor{red!40}\href{../works/CarchraeBF05.pdf}{CarchraeBF05} (0.18)\\
Dot& \cellcolor{red!40}\href{../works/Beck99.pdf}{Beck99} (47.00)& \cellcolor{red!40}\href{../works/Schutt11.pdf}{Schutt11} (47.00)& \cellcolor{red!40}\href{../works/HookerH17.pdf}{HookerH17} (45.00)& \cellcolor{red!40}\href{../works/Lombardi10.pdf}{Lombardi10} (45.00)& \cellcolor{red!40}\href{../works/Siala15a.pdf}{Siala15a} (45.00)\\
Cosine& \cellcolor{red!40}\href{../works/BeckF00a.pdf}{BeckF00a} (0.68)& \cellcolor{red!40}\href{../works/BeckF99.pdf}{BeckF99} (0.68)& \cellcolor{red!40}\href{../works/Vilim05.pdf}{Vilim05} (0.68)& \cellcolor{red!40}\href{../works/BeckF00.pdf}{BeckF00} (0.66)& \cellcolor{red!40}\href{../works/SialaAH15.pdf}{SialaAH15} (0.65)\\
\index{Vilim04}\href{../works/Vilim04.pdf}{Vilim04} R\&C& \cellcolor{red!40}\href{../works/VilimBC04.pdf}{VilimBC04} (0.44)& \cellcolor{red!40}\href{../works/VilimBC05.pdf}{VilimBC05} (0.63)& \cellcolor{red!40}\href{../works/Vilim05.pdf}{Vilim05} (0.73)& \cellcolor{red!40}\href{../works/WolfS05a.pdf}{WolfS05a} (0.76)& \cellcolor{red!40}\href{../works/Vilim02.pdf}{Vilim02} (0.76)\\
Euclid& \cellcolor{red!40}\href{../works/VilimBC05.pdf}{VilimBC05} (0.20)& \cellcolor{red!40}\href{../works/VilimBC04.pdf}{VilimBC04} (0.20)& \cellcolor{red!40}\href{../works/CauwelaertDMS16.pdf}{CauwelaertDMS16} (0.21)& \cellcolor{red!40}\href{../works/Vilim09.pdf}{Vilim09} (0.22)& \cellcolor{red!40}\href{../works/Vilim09a.pdf}{Vilim09a} (0.24)\\
Dot& \cellcolor{red!40}\href{../works/Dejemeppe16.pdf}{Dejemeppe16} (105.00)& \cellcolor{red!40}\href{../works/Malapert11.pdf}{Malapert11} (104.00)& \cellcolor{red!40}\href{../works/Fahimi16.pdf}{Fahimi16} (103.00)& \cellcolor{red!40}\href{../works/Baptiste02.pdf}{Baptiste02} (102.00)& \cellcolor{red!40}\href{../works/Schutt11.pdf}{Schutt11} (101.00)\\
Cosine& \cellcolor{red!40}\href{../works/VilimBC05.pdf}{VilimBC05} (0.89)& \cellcolor{red!40}\href{../works/VilimBC04.pdf}{VilimBC04} (0.86)& \cellcolor{red!40}\href{../works/CauwelaertDS20.pdf}{CauwelaertDS20} (0.86)& \cellcolor{red!40}\href{../works/CauwelaertDMS16.pdf}{CauwelaertDMS16} (0.84)& \cellcolor{red!40}\href{../works/DejemeppeCS15.pdf}{DejemeppeCS15} (0.83)\\
\index{Vilim05}\href{../works/Vilim05.pdf}{Vilim05} R\&C& \cellcolor{red!40}\href{../works/Wolf05.pdf}{Wolf05} (0.65)& \cellcolor{red!40}\href{../works/ArtiouchineB05.pdf}{ArtiouchineB05} (0.66)& \cellcolor{red!40}\href{../works/VilimBC04.pdf}{VilimBC04} (0.69)& \cellcolor{red!40}\href{../works/Vilim04.pdf}{Vilim04} (0.73)& \cellcolor{red!40}\href{../works/VilimBC05.pdf}{VilimBC05} (0.77)\\
Euclid& \cellcolor{red!20}\href{../works/BeckF00a.pdf}{BeckF00a} (0.26)& \cellcolor{red!20}\href{../works/BeckF99.pdf}{BeckF99} (0.26)& \cellcolor{red!20}\href{../works/VilimBC04.pdf}{VilimBC04} (0.26)& \cellcolor{yellow!20}\href{../works/Vilim04.pdf}{Vilim04} (0.26)& \cellcolor{yellow!20}\href{../works/PacinoH11.pdf}{PacinoH11} (0.26)\\
Dot& \cellcolor{red!40}\href{../works/Fahimi16.pdf}{Fahimi16} (130.00)& \cellcolor{red!40}\href{../works/Schutt11.pdf}{Schutt11} (130.00)& \cellcolor{red!40}\href{../works/Baptiste02.pdf}{Baptiste02} (129.00)& \cellcolor{red!40}\href{../works/Dejemeppe16.pdf}{Dejemeppe16} (128.00)& \cellcolor{red!40}\href{../works/Beck99.pdf}{Beck99} (127.00)\\
Cosine& \cellcolor{red!40}\href{../works/BeckF00a.pdf}{BeckF00a} (0.81)& \cellcolor{red!40}\href{../works/BeckF00.pdf}{BeckF00} (0.81)& \cellcolor{red!40}\href{../works/BeckPS03.pdf}{BeckPS03} (0.78)& \cellcolor{red!40}\href{../works/VilimBC04.pdf}{VilimBC04} (0.78)& \cellcolor{red!40}\href{../works/VilimBC05.pdf}{VilimBC05} (0.78)\\
\index{Vilim09}\href{../works/Vilim09.pdf}{Vilim09} R\&C& \cellcolor{red!40}\href{../works/Vilim09a.pdf}{Vilim09a} (0.52)& \cellcolor{red!40}\href{../works/SchuttW10.pdf}{SchuttW10} (0.57)& \cellcolor{red!40}\href{../works/KameugneFSN11.pdf}{KameugneFSN11} (0.58)& \cellcolor{red!40}\href{../works/KameugneF13.pdf}{KameugneF13} (0.65)& \cellcolor{red!40}\href{../works/MercierH08.pdf}{MercierH08} (0.68)\\
Euclid& \cellcolor{red!40}\href{../works/Vilim09a.pdf}{Vilim09a} (0.17)& \cellcolor{red!40}\href{../works/Vilim04.pdf}{Vilim04} (0.22)& \cellcolor{red!40}\href{../works/VilimBC04.pdf}{VilimBC04} (0.23)& \cellcolor{red!40}\href{../works/BeckF99.pdf}{BeckF99} (0.23)& \cellcolor{red!40}\href{../works/Vilim11.pdf}{Vilim11} (0.24)\\
Dot& \cellcolor{red!40}\href{../works/Dejemeppe16.pdf}{Dejemeppe16} (95.00)& \cellcolor{red!40}\href{../works/Fahimi16.pdf}{Fahimi16} (94.00)& \cellcolor{red!40}\href{../works/Schutt11.pdf}{Schutt11} (94.00)& \cellcolor{red!40}\href{../works/Baptiste02.pdf}{Baptiste02} (93.00)& \cellcolor{red!40}\href{../works/Lombardi10.pdf}{Lombardi10} (89.00)\\
Cosine& \cellcolor{red!40}\href{../works/Vilim09a.pdf}{Vilim09a} (0.86)& \cellcolor{red!40}\href{../works/VilimBC04.pdf}{VilimBC04} (0.80)& \cellcolor{red!40}\href{../works/Vilim11.pdf}{Vilim11} (0.80)& \cellcolor{red!40}\href{../works/VilimBC05.pdf}{VilimBC05} (0.79)& \cellcolor{red!40}\href{../works/Vilim04.pdf}{Vilim04} (0.79)\\
\index{Vilim09a}\href{../works/Vilim09a.pdf}{Vilim09a} R\&C& \cellcolor{red!40}\href{../works/Vilim09.pdf}{Vilim09} (0.52)& \cellcolor{red!40}\href{../works/KameugneF13.pdf}{KameugneF13} (0.64)& \cellcolor{red!40}\href{../works/SchuttW10.pdf}{SchuttW10} (0.66)& \cellcolor{red!40}\href{../works/KameugneFSN11.pdf}{KameugneFSN11} (0.72)& \cellcolor{red!40}\href{../works/OuelletQ18.pdf}{OuelletQ18} (0.75)\\
Euclid& \cellcolor{red!40}\href{../works/Vilim09.pdf}{Vilim09} (0.17)& \cellcolor{red!40}\href{../works/WolfS05.pdf}{WolfS05} (0.18)& \cellcolor{red!40}\href{../works/PoderB08.pdf}{PoderB08} (0.19)& \cellcolor{red!40}\href{../works/BeniniBGM05a.pdf}{BeniniBGM05a} (0.21)& \cellcolor{red!40}\href{../works/Vilim11.pdf}{Vilim11} (0.22)\\
Dot& \cellcolor{red!40}\href{../works/Schutt11.pdf}{Schutt11} (87.00)& \cellcolor{red!40}\href{../works/Fahimi16.pdf}{Fahimi16} (86.00)& \cellcolor{red!40}\href{../works/Baptiste02.pdf}{Baptiste02} (85.00)& \cellcolor{red!40}\href{../works/Dejemeppe16.pdf}{Dejemeppe16} (84.00)& \cellcolor{red!40}\href{../works/Lombardi10.pdf}{Lombardi10} (84.00)\\
Cosine& \cellcolor{red!40}\href{../works/Vilim09.pdf}{Vilim09} (0.86)& \cellcolor{red!40}\href{../works/WolfS05.pdf}{WolfS05} (0.85)& \cellcolor{red!40}\href{../works/Vilim11.pdf}{Vilim11} (0.84)& \cellcolor{red!40}\href{../works/PoderB08.pdf}{PoderB08} (0.81)& \cellcolor{red!40}\href{../works/BeniniBGM05a.pdf}{BeniniBGM05a} (0.76)\\
\index{Vilim11}\href{../works/Vilim11.pdf}{Vilim11} R\&C& \cellcolor{red!40}\href{../works/SchuttW10.pdf}{SchuttW10} (0.63)& \cellcolor{red!40}\href{../works/OuelletQ13.pdf}{OuelletQ13} (0.65)& \cellcolor{red!40}\href{../works/KameugneF13.pdf}{KameugneF13} (0.65)& \cellcolor{red!40}\href{../works/LetortBC12.pdf}{LetortBC12} (0.67)& \cellcolor{red!40}\href{../works/Vilim09.pdf}{Vilim09} (0.69)\\
Euclid& \cellcolor{red!40}\href{../works/Vilim09a.pdf}{Vilim09a} (0.22)& \cellcolor{red!40}\href{../works/Vilim09.pdf}{Vilim09} (0.24)& \cellcolor{red!20}\href{../works/OuelletQ18.pdf}{OuelletQ18} (0.24)& \cellcolor{red!20}\href{../works/WolfS05.pdf}{WolfS05} (0.25)& \cellcolor{red!20}\href{../works/GayHS15a.pdf}{GayHS15a} (0.25)\\
Dot& \cellcolor{red!40}\href{../works/Schutt11.pdf}{Schutt11} (127.00)& \cellcolor{red!40}\href{../works/Fahimi16.pdf}{Fahimi16} (121.00)& \cellcolor{red!40}\href{../works/Lombardi10.pdf}{Lombardi10} (117.00)& \cellcolor{red!40}\href{../works/Dejemeppe16.pdf}{Dejemeppe16} (115.00)& \cellcolor{red!40}\href{../works/Baptiste02.pdf}{Baptiste02} (113.00)\\
Cosine& \cellcolor{red!40}\href{../works/Vilim09a.pdf}{Vilim09a} (0.84)& \cellcolor{red!40}\href{../works/GayHS15a.pdf}{GayHS15a} (0.82)& \cellcolor{red!40}\href{../works/OuelletQ18.pdf}{OuelletQ18} (0.81)& \cellcolor{red!40}\href{../works/Vilim09.pdf}{Vilim09} (0.80)& \cellcolor{red!40}\href{../works/SchuttFS13a.pdf}{SchuttFS13a} (0.79)\\
\index{VilimBC04}\href{../works/VilimBC04.pdf}{VilimBC04} R\&C& \cellcolor{red!40}\href{../works/Vilim04.pdf}{Vilim04} (0.44)& \cellcolor{red!40}\href{../works/VilimBC05.pdf}{VilimBC05} (0.67)& \cellcolor{red!40}\href{../works/Vilim05.pdf}{Vilim05} (0.69)& \cellcolor{red!40}\href{../works/MercierH07.pdf}{MercierH07} (0.78)& \cellcolor{red!40}\href{../works/ArtiouchineB05.pdf}{ArtiouchineB05} (0.80)\\
Euclid& \cellcolor{red!40}\href{../works/VilimBC05.pdf}{VilimBC05} (0.13)& \cellcolor{red!40}\href{../works/Vilim04.pdf}{Vilim04} (0.20)& \cellcolor{red!40}\href{../works/CauwelaertDMS16.pdf}{CauwelaertDMS16} (0.20)& \cellcolor{red!40}\href{../works/DejemeppeCS15.pdf}{DejemeppeCS15} (0.23)& \cellcolor{red!40}\href{../works/Vilim09.pdf}{Vilim09} (0.23)\\
Dot& \cellcolor{red!40}\href{../works/Baptiste02.pdf}{Baptiste02} (128.00)& \cellcolor{red!40}\href{../works/Fahimi16.pdf}{Fahimi16} (126.00)& \cellcolor{red!40}\href{../works/Dejemeppe16.pdf}{Dejemeppe16} (124.00)& \cellcolor{red!40}\href{../works/Schutt11.pdf}{Schutt11} (123.00)& \cellcolor{red!40}\href{../works/Malapert11.pdf}{Malapert11} (122.00)\\
Cosine& \cellcolor{red!40}\href{../works/VilimBC05.pdf}{VilimBC05} (0.95)& \cellcolor{red!40}\href{../works/CauwelaertDMS16.pdf}{CauwelaertDMS16} (0.86)& \cellcolor{red!40}\href{../works/DejemeppeCS15.pdf}{DejemeppeCS15} (0.86)& \cellcolor{red!40}\href{../works/Vilim04.pdf}{Vilim04} (0.86)& \cellcolor{red!40}\href{../works/CauwelaertDS20.pdf}{CauwelaertDS20} (0.84)\\
\index{VilimBC05}\href{../works/VilimBC05.pdf}{VilimBC05} R\&C& \cellcolor{red!40}\href{../works/Vilim04.pdf}{Vilim04} (0.63)& \cellcolor{red!40}\href{../works/VilimBC04.pdf}{VilimBC04} (0.67)& \cellcolor{red!40}\href{../works/Vilim05.pdf}{Vilim05} (0.77)& \cellcolor{red!40}\href{../works/WolfS05a.pdf}{WolfS05a} (0.78)& \cellcolor{red!40}\href{../works/TorresL00.pdf}{TorresL00} (0.78)\\
Euclid& \cellcolor{red!40}\href{../works/VilimBC04.pdf}{VilimBC04} (0.13)& \cellcolor{red!40}\href{../works/Vilim04.pdf}{Vilim04} (0.20)& \cellcolor{red!40}\href{../works/CauwelaertDMS16.pdf}{CauwelaertDMS16} (0.21)& \cellcolor{red!40}\href{../works/CauwelaertDS20.pdf}{CauwelaertDS20} (0.21)& \cellcolor{red!40}\href{../works/DejemeppeCS15.pdf}{DejemeppeCS15} (0.22)\\
Dot& \cellcolor{red!40}\href{../works/Malapert11.pdf}{Malapert11} (149.00)& \cellcolor{red!40}\href{../works/Baptiste02.pdf}{Baptiste02} (147.00)& \cellcolor{red!40}\href{../works/Dejemeppe16.pdf}{Dejemeppe16} (146.00)& \cellcolor{red!40}\href{../works/Fahimi16.pdf}{Fahimi16} (143.00)& \cellcolor{red!40}\href{../works/Schutt11.pdf}{Schutt11} (142.00)\\
Cosine& \cellcolor{red!40}\href{../works/VilimBC04.pdf}{VilimBC04} (0.95)& \cellcolor{red!40}\href{../works/Vilim04.pdf}{Vilim04} (0.89)& \cellcolor{red!40}\href{../works/CauwelaertDS20.pdf}{CauwelaertDS20} (0.89)& \cellcolor{red!40}\href{../works/DejemeppeCS15.pdf}{DejemeppeCS15} (0.88)& \cellcolor{red!40}\href{../works/CauwelaertDMS16.pdf}{CauwelaertDMS16} (0.87)\\
\index{VilimLS15}\href{../works/VilimLS15.pdf}{VilimLS15} R\&C& \cellcolor{red!40}\href{../works/GayHLS15.pdf}{GayHLS15} (0.77)& \cellcolor{red!40}\href{../works/ColT19.pdf}{ColT19} (0.82)& \cellcolor{red!40}\href{../works/OuelletQ13.pdf}{OuelletQ13} (0.86)& \cellcolor{red!20}\href{../works/ColT2019a.pdf}{ColT2019a} (0.86)& \cellcolor{red!20}\href{../works/LaborieRSV18.pdf}{LaborieRSV18} (0.87)\\
Euclid& \cellcolor{yellow!20}\href{../works/LiessM08.pdf}{LiessM08} (0.28)& \cellcolor{yellow!20}\href{../works/Laborie05.pdf}{Laborie05} (0.28)& \cellcolor{green!20}\href{../works/HeipckeCCS00.pdf}{HeipckeCCS00} (0.29)& \cellcolor{green!20}\href{../works/DemasseyAM05.pdf}{DemasseyAM05} (0.29)& \cellcolor{green!20}\href{../works/ArkhipovBL19.pdf}{ArkhipovBL19} (0.29)\\
Dot& \cellcolor{red!40}\href{../works/Groleaz21.pdf}{Groleaz21} (180.00)& \cellcolor{red!40}\href{../works/LaborieRSV18.pdf}{LaborieRSV18} (162.00)& \cellcolor{red!40}\href{../works/Godet21a.pdf}{Godet21a} (162.00)& \cellcolor{red!40}\href{../works/Lombardi10.pdf}{Lombardi10} (160.00)& \cellcolor{red!40}\href{../works/Schutt11.pdf}{Schutt11} (160.00)\\
Cosine& \cellcolor{red!40}\href{../works/LiessM08.pdf}{LiessM08} (0.83)& \cellcolor{red!40}\href{../works/Laborie05.pdf}{Laborie05} (0.82)& \cellcolor{red!40}\href{../works/ArkhipovBL19.pdf}{ArkhipovBL19} (0.81)& \cellcolor{red!40}\href{../works/HeipckeCCS00.pdf}{HeipckeCCS00} (0.81)& \cellcolor{red!40}\href{../works/DemasseyAM05.pdf}{DemasseyAM05} (0.81)\\
\index{VillaverdeP04}VillaverdeP04 R\&C\\
Euclid\\
Dot\\
Cosine\\
\index{VlkHT21}\href{../works/VlkHT21.pdf}{VlkHT21} R\&C& \cellcolor{green!20}\href{../works/ParkUJR19.pdf}{ParkUJR19} (0.96)& \cellcolor{blue!20}\href{../works/Laborie18a.pdf}{Laborie18a} (0.97)& \cellcolor{blue!20}\href{../works/ColT2019a.pdf}{ColT2019a} (0.97)& \cellcolor{blue!20}\href{../works/DejemeppeCS15.pdf}{DejemeppeCS15} (0.97)& \cellcolor{blue!20}\href{../works/ColT19.pdf}{ColT19} (0.97)\\
Euclid& \cellcolor{blue!20}\href{../works/CireCH13.pdf}{CireCH13} (0.33)& \cellcolor{black!20}\href{../works/CireCH16.pdf}{CireCH16} (0.34)& \cellcolor{black!20}\href{../works/LozanoCDS12.pdf}{LozanoCDS12} (0.34)& \cellcolor{black!20}\href{../works/BoothNB16.pdf}{BoothNB16} (0.34)& \cellcolor{black!20}\href{../works/GoldwaserS17.pdf}{GoldwaserS17} (0.35)\\
Dot& \cellcolor{red!40}\href{../works/NaderiRR23.pdf}{NaderiRR23} (112.00)& \cellcolor{red!40}\href{../works/Groleaz21.pdf}{Groleaz21} (96.00)& \cellcolor{red!40}\href{../works/Lombardi10.pdf}{Lombardi10} (91.00)& \cellcolor{red!40}\href{../works/ZarandiASC20.pdf}{ZarandiASC20} (86.00)& \cellcolor{red!40}\href{../works/LaborieRSV18.pdf}{LaborieRSV18} (85.00)\\
Cosine& \cellcolor{red!40}\href{../works/HladikCDJ08.pdf}{HladikCDJ08} (0.62)& \cellcolor{red!40}\href{../works/CambazardHDJT04.pdf}{CambazardHDJT04} (0.62)& \cellcolor{red!40}\href{../works/CireCH13.pdf}{CireCH13} (0.61)& \cellcolor{red!40}\href{../works/Hooker06.pdf}{Hooker06} (0.61)& \cellcolor{red!40}\href{../works/CireCH16.pdf}{CireCH16} (0.61)\\
\index{Wallace06}\href{../works/Wallace06.pdf}{Wallace06} R\&C& \cellcolor{red!40}AjiliW04 (0.85)& \cellcolor{red!20}\href{../works/KamarainenS02.pdf}{KamarainenS02} (0.88)& \cellcolor{red!20}\href{../works/MilanoW06.pdf}{MilanoW06} (0.88)& \cellcolor{red!20}\href{../works/MilanoW09.pdf}{MilanoW09} (0.90)& \cellcolor{yellow!20}Milano11 (0.90)\\
Euclid& \cellcolor{green!20}\href{../works/EreminW01.pdf}{EreminW01} (0.30)& \cellcolor{black!20}\href{../works/KamarainenS02.pdf}{KamarainenS02} (0.34)& \cellcolor{black!20}\href{../works/FallahiAC20.pdf}{FallahiAC20} (0.34)& \cellcolor{black!20}\href{../works/ErkingerM17.pdf}{ErkingerM17} (0.34)& \cellcolor{black!20}\href{../works/Puget95.pdf}{Puget95} (0.36)\\
Dot& \cellcolor{red!40}\href{../works/ZarandiASC20.pdf}{ZarandiASC20} (124.00)& \cellcolor{red!40}\href{../works/Froger16.pdf}{Froger16} (121.00)& \cellcolor{red!40}\href{../works/Groleaz21.pdf}{Groleaz21} (120.00)& \cellcolor{red!40}\href{../works/Dejemeppe16.pdf}{Dejemeppe16} (115.00)& \cellcolor{red!40}\href{../works/Lombardi10.pdf}{Lombardi10} (115.00)\\
Cosine& \cellcolor{red!40}\href{../works/EreminW01.pdf}{EreminW01} (0.74)& \cellcolor{red!40}\href{../works/Wallace96.pdf}{Wallace96} (0.67)& \cellcolor{red!40}\href{../works/FallahiAC20.pdf}{FallahiAC20} (0.66)& \cellcolor{red!40}\href{../works/BourreauGGLT22.pdf}{BourreauGGLT22} (0.65)& \cellcolor{red!40}\href{../works/MilanoW06.pdf}{MilanoW06} (0.65)\\
\index{Wallace94}Wallace94 R\&C\\
Euclid\\
Dot\\
Cosine\\
\index{Wallace96}\href{../works/Wallace96.pdf}{Wallace96} R\&C& \cellcolor{red!40}\href{../works/DincbasSH90.pdf}{DincbasSH90} (0.86)& \cellcolor{yellow!20}\href{../works/SadehF96.pdf}{SadehF96} (0.93)& \cellcolor{yellow!20}\href{../works/NuijtenA96.pdf}{NuijtenA96} (0.93)& \cellcolor{yellow!20}\href{../works/Dorndorf2000.pdf}{Dorndorf2000} (0.93)& \cellcolor{green!20}\href{../works/BeldiceanuC94.pdf}{BeldiceanuC94} (0.94)\\
Euclid& \cellcolor{blue!20}\href{../works/Simonis95a.pdf}{Simonis95a} (0.33)& \cellcolor{black!20}\href{../works/LammaMM97.pdf}{LammaMM97} (0.35)& \cellcolor{black!20}\href{../works/Bartak02.pdf}{Bartak02} (0.35)& \cellcolor{black!20}\href{../works/Simonis95.pdf}{Simonis95} (0.37)& \cellcolor{black!20}\href{../works/BocewiczBB09.pdf}{BocewiczBB09} (0.37)\\
Dot& \cellcolor{red!40}\href{../works/ZarandiASC20.pdf}{ZarandiASC20} (150.00)& \cellcolor{red!40}\href{../works/HookerH17.pdf}{HookerH17} (135.00)& \cellcolor{red!40}\href{../works/Simonis99.pdf}{Simonis99} (127.00)& \cellcolor{red!40}\href{../works/Beck99.pdf}{Beck99} (126.00)& \cellcolor{red!40}\href{../works/Lombardi10.pdf}{Lombardi10} (126.00)\\
Cosine& \cellcolor{red!40}\href{../works/Simonis95a.pdf}{Simonis95a} (0.74)& \cellcolor{red!40}\href{../works/Simonis99.pdf}{Simonis99} (0.73)& \cellcolor{red!40}\href{../works/LammaMM97.pdf}{LammaMM97} (0.72)& \cellcolor{red!40}\href{../works/Bartak02.pdf}{Bartak02} (0.70)& \cellcolor{red!40}\href{../works/Wallace06.pdf}{Wallace06} (0.67)\\
\index{WallaceF00}\href{../works/WallaceF00.pdf}{WallaceF00} R\&C\\
Euclid& \cellcolor{red!40}\href{../works/CestaOS98.pdf}{CestaOS98} (0.17)& \cellcolor{red!40}\href{../works/AngelsmarkJ00.pdf}{AngelsmarkJ00} (0.18)& \cellcolor{red!40}\href{../works/FukunagaHFAMN02.pdf}{FukunagaHFAMN02} (0.18)& \cellcolor{red!40}\href{../works/BonfiettiM12.pdf}{BonfiettiM12} (0.19)& \cellcolor{red!40}\href{../works/Caballero23.pdf}{Caballero23} (0.19)\\
Dot& \cellcolor{red!40}\href{../works/Lombardi10.pdf}{Lombardi10} (48.00)& \cellcolor{red!40}\href{../works/Siala15a.pdf}{Siala15a} (46.00)& \cellcolor{red!40}\href{../works/Godet21a.pdf}{Godet21a} (45.00)& \cellcolor{red!40}\href{../works/Beck99.pdf}{Beck99} (45.00)& \cellcolor{red!40}\href{../works/Baptiste02.pdf}{Baptiste02} (45.00)\\
Cosine& \cellcolor{red!40}\href{../works/ReddyFIBKAJ11.pdf}{ReddyFIBKAJ11} (0.75)& \cellcolor{red!40}\href{../works/PoderB08.pdf}{PoderB08} (0.72)& \cellcolor{red!40}\href{../works/ElkhyariGJ02.pdf}{ElkhyariGJ02} (0.72)& \cellcolor{red!40}\href{../works/LombardiM13.pdf}{LombardiM13} (0.72)& \cellcolor{red!40}\href{../works/Vilim09a.pdf}{Vilim09a} (0.71)\\
\index{WallaceY20}\href{../works/WallaceY20.pdf}{WallaceY20} R\&C& \cellcolor{red!40}\href{../works/WessenCS20.pdf}{WessenCS20} (0.86)& \cellcolor{yellow!20}\href{../works/EfthymiouY23.pdf}{EfthymiouY23} (0.90)& \cellcolor{green!20}\href{../works/KreterSS15.pdf}{KreterSS15} (0.94)& \cellcolor{green!20}\href{../works/BehrensLM19.pdf}{BehrensLM19} (0.94)& \cellcolor{green!20}\href{../works/KreterSS17.pdf}{KreterSS17} (0.95)\\
Euclid& \href{../works/EfthymiouY23.pdf}{EfthymiouY23} (0.41)& \href{../works/Caseau97.pdf}{Caseau97} (0.42)& \href{../works/BukchinR18.pdf}{BukchinR18} (0.42)& \href{../works/DemirovicS18.pdf}{DemirovicS18} (0.42)& \href{../works/LozanoCDS12.pdf}{LozanoCDS12} (0.42)\\
Dot& \cellcolor{red!40}\href{../works/Godet21a.pdf}{Godet21a} (130.00)& \cellcolor{red!40}\href{../works/Dejemeppe16.pdf}{Dejemeppe16} (129.00)& \cellcolor{red!40}\href{../works/KoehlerBFFHPSSS21.pdf}{KoehlerBFFHPSSS21} (128.00)& \cellcolor{red!40}\href{../works/Groleaz21.pdf}{Groleaz21} (127.00)& \cellcolor{red!40}\href{../works/Astrand21.pdf}{Astrand21} (123.00)\\
Cosine& \cellcolor{red!40}\href{../works/WessenCSFPM23.pdf}{WessenCSFPM23} (0.61)& \cellcolor{red!40}\href{../works/EfthymiouY23.pdf}{EfthymiouY23} (0.61)& \cellcolor{red!40}\href{../works/KoehlerBFFHPSSS21.pdf}{KoehlerBFFHPSSS21} (0.59)& \cellcolor{red!40}\href{../works/abs-1911-04766.pdf}{abs-1911-04766} (0.59)& \cellcolor{red!40}\href{../works/AbidinK20.pdf}{AbidinK20} (0.58)\\
\index{WangB20}\href{../works/WangB20.pdf}{WangB20} R\&C\\
Euclid& \cellcolor{red!40}\href{../works/WangB23.pdf}{WangB23} (0.23)& \cellcolor{blue!20}\href{../works/TranDRFWOVB16.pdf}{TranDRFWOVB16} (0.34)& \cellcolor{black!20}\href{../works/Puget95.pdf}{Puget95} (0.37)& \href{../works/Caseau97.pdf}{Caseau97} (0.38)& \href{../works/CrawfordB94.pdf}{CrawfordB94} (0.38)\\
Dot& \cellcolor{red!40}\href{../works/WangB23.pdf}{WangB23} (103.00)& \cellcolor{red!40}\href{../works/ZarandiASC20.pdf}{ZarandiASC20} (87.00)& \cellcolor{red!40}\href{../works/Lunardi20.pdf}{Lunardi20} (83.00)& \cellcolor{red!40}\href{../works/Lemos21.pdf}{Lemos21} (80.00)& \cellcolor{red!40}\href{../works/Fahimi16.pdf}{Fahimi16} (80.00)\\
Cosine& \cellcolor{red!40}\href{../works/WangB23.pdf}{WangB23} (0.85)& \cellcolor{red!40}\href{../works/TranDRFWOVB16.pdf}{TranDRFWOVB16} (0.66)& \cellcolor{red!40}\href{../works/ForbesHJST24.pdf}{ForbesHJST24} (0.58)& \cellcolor{red!40}\href{../works/GokGSTO20.pdf}{GokGSTO20} (0.58)& \cellcolor{red!40}\href{../works/Puget95.pdf}{Puget95} (0.55)\\
\index{WangB23}\href{../works/WangB23.pdf}{WangB23} R\&C\\
Euclid& \cellcolor{red!40}\href{../works/WangB20.pdf}{WangB20} (0.23)& \cellcolor{green!20}\href{../works/TranDRFWOVB16.pdf}{TranDRFWOVB16} (0.30)& \cellcolor{green!20}\href{../works/Puget95.pdf}{Puget95} (0.31)& \cellcolor{blue!20}\href{../works/Caseau97.pdf}{Caseau97} (0.32)& \cellcolor{blue!20}\href{../works/AngelsmarkJ00.pdf}{AngelsmarkJ00} (0.32)\\
Dot& \cellcolor{red!40}\href{../works/WangB20.pdf}{WangB20} (103.00)& \cellcolor{red!40}\href{../works/ZarandiASC20.pdf}{ZarandiASC20} (92.00)& \cellcolor{red!40}\href{../works/Lunardi20.pdf}{Lunardi20} (89.00)& \cellcolor{red!40}\href{../works/Lemos21.pdf}{Lemos21} (87.00)& \cellcolor{red!40}\href{../works/Dejemeppe16.pdf}{Dejemeppe16} (80.00)\\
Cosine& \cellcolor{red!40}\href{../works/WangB20.pdf}{WangB20} (0.85)& \cellcolor{red!40}\href{../works/TranDRFWOVB16.pdf}{TranDRFWOVB16} (0.67)& \cellcolor{red!40}\href{../works/Puget95.pdf}{Puget95} (0.63)& \cellcolor{red!40}\href{../works/Caseau97.pdf}{Caseau97} (0.61)& \cellcolor{red!40}\href{../works/ForbesHJST24.pdf}{ForbesHJST24} (0.60)\\
\index{WangMD15}\href{../works/WangMD15.pdf}{WangMD15} R\&C& \cellcolor{red!40}\href{../works/MeskensDL13.pdf}{MeskensDL13} (0.63)& \cellcolor{red!40}\href{../works/DoulabiRP16.pdf}{DoulabiRP16} (0.74)& \cellcolor{red!40}\href{../works/RoshanaeiLAU17.pdf}{RoshanaeiLAU17} (0.78)& \cellcolor{red!40}\href{../works/RoshanaeiBAUB20.pdf}{RoshanaeiBAUB20} (0.80)& \cellcolor{red!40}RoshanaeiLAU17a (0.80)\\
Euclid& \cellcolor{red!40}\href{../works/MeskensDHG11.pdf}{MeskensDHG11} (0.23)& \cellcolor{red!40}\href{../works/MeskensDL13.pdf}{MeskensDL13} (0.24)& \cellcolor{green!20}\href{../works/YounespourAKE19.pdf}{YounespourAKE19} (0.29)& \cellcolor{green!20}\href{../works/GurPAE23.pdf}{GurPAE23} (0.29)& \cellcolor{green!20}\href{../works/GurEA19.pdf}{GurEA19} (0.29)\\
Dot& \cellcolor{red!40}\href{../works/ZarandiASC20.pdf}{ZarandiASC20} (137.00)& \cellcolor{red!40}\href{../works/Dejemeppe16.pdf}{Dejemeppe16} (123.00)& \cellcolor{red!40}\href{../works/YounespourAKE19.pdf}{YounespourAKE19} (119.00)& \cellcolor{red!40}\href{../works/MeskensDL13.pdf}{MeskensDL13} (118.00)& \cellcolor{red!40}\href{../works/Baptiste02.pdf}{Baptiste02} (107.00)\\
Cosine& \cellcolor{red!40}\href{../works/MeskensDL13.pdf}{MeskensDL13} (0.86)& \cellcolor{red!40}\href{../works/MeskensDHG11.pdf}{MeskensDHG11} (0.85)& \cellcolor{red!40}\href{../works/YounespourAKE19.pdf}{YounespourAKE19} (0.81)& \cellcolor{red!40}\href{../works/GurPAE23.pdf}{GurPAE23} (0.78)& \cellcolor{red!40}\href{../works/GurEA19.pdf}{GurEA19} (0.77)\\
\index{WariZ19}WariZ19 R\&C& \cellcolor{yellow!20}\href{../works/PengLC14.pdf}{PengLC14} (0.92)& \cellcolor{yellow!20}\href{../works/ColT2019a.pdf}{ColT2019a} (0.93)& \cellcolor{green!20}\href{../works/KovacsB11.pdf}{KovacsB11} (0.93)& \cellcolor{green!20}\href{../works/Laborie18a.pdf}{Laborie18a} (0.95)& \cellcolor{green!20}\href{../works/EscobetPQPRA19.pdf}{EscobetPQPRA19} (0.95)\\
Euclid\\
Dot\\
Cosine\\
\index{WatsonB08}\href{../works/WatsonB08.pdf}{WatsonB08} R\&C& \cellcolor{red!40}\href{../works/BeckFW11.pdf}{BeckFW11} (0.61)& \cellcolor{red!40}\href{../works/GrimesHM09.pdf}{GrimesHM09} (0.62)& \cellcolor{red!40}\href{../works/HeckmanB11.pdf}{HeckmanB11} (0.82)& \cellcolor{red!40}\href{../works/Beck07.pdf}{Beck07} (0.83)& \cellcolor{red!40}\href{../works/BeckF00.pdf}{BeckF00} (0.84)\\
Euclid& \cellcolor{red!40}\href{../works/BeckFW11.pdf}{BeckFW11} (0.16)& \cellcolor{red!40}\href{../works/Beck06.pdf}{Beck06} (0.21)& \cellcolor{red!40}\href{../works/Beck07.pdf}{Beck07} (0.23)& \cellcolor{red!40}\href{../works/CarchraeB09.pdf}{CarchraeB09} (0.23)& \cellcolor{red!40}\href{../works/TanSD10.pdf}{TanSD10} (0.23)\\
Dot& \cellcolor{red!40}\href{../works/Beck99.pdf}{Beck99} (111.00)& \cellcolor{red!40}\href{../works/Schutt11.pdf}{Schutt11} (111.00)& \cellcolor{red!40}\href{../works/Groleaz21.pdf}{Groleaz21} (109.00)& \cellcolor{red!40}\href{../works/Astrand21.pdf}{Astrand21} (108.00)& \cellcolor{red!40}\href{../works/BeckFW11.pdf}{BeckFW11} (108.00)\\
Cosine& \cellcolor{red!40}\href{../works/BeckFW11.pdf}{BeckFW11} (0.93)& \cellcolor{red!40}\href{../works/Beck06.pdf}{Beck06} (0.86)& \cellcolor{red!40}\href{../works/Beck07.pdf}{Beck07} (0.85)& \cellcolor{red!40}\href{../works/CarchraeB09.pdf}{CarchraeB09} (0.85)& \cellcolor{red!40}\href{../works/TanSD10.pdf}{TanSD10} (0.84)\\
\index{WatsonBHW99}\href{../works/WatsonBHW99.pdf}{WatsonBHW99} R\&C\\
Euclid& \cellcolor{yellow!20}\href{../works/LiLZDZW24.pdf}{LiLZDZW24} (0.26)& \cellcolor{green!20}\href{../works/DilkinaDH05.pdf}{DilkinaDH05} (0.29)& \cellcolor{green!20}\href{../works/DoRZ08.pdf}{DoRZ08} (0.29)& \cellcolor{green!20}\href{../works/LauLN08.pdf}{LauLN08} (0.31)& \cellcolor{blue!20}\href{../works/JuvinHL23.pdf}{JuvinHL23} (0.32)\\
Dot& \cellcolor{red!40}\href{../works/Astrand21.pdf}{Astrand21} (103.00)& \cellcolor{red!40}\href{../works/Groleaz21.pdf}{Groleaz21} (93.00)& \cellcolor{red!40}\href{../works/Dejemeppe16.pdf}{Dejemeppe16} (91.00)& \cellcolor{red!40}\href{../works/Lombardi10.pdf}{Lombardi10} (91.00)& \cellcolor{red!40}\href{../works/ZarandiASC20.pdf}{ZarandiASC20} (90.00)\\
Cosine& \cellcolor{red!40}\href{../works/LiLZDZW24.pdf}{LiLZDZW24} (0.75)& \cellcolor{red!40}\href{../works/DilkinaDH05.pdf}{DilkinaDH05} (0.68)& \cellcolor{red!40}\href{../works/DoRZ08.pdf}{DoRZ08} (0.66)& \cellcolor{red!40}\href{../works/RiahiNS018.pdf}{RiahiNS018} (0.65)& \cellcolor{red!40}\href{../works/JuvinHL23.pdf}{JuvinHL23} (0.64)\\
\index{WeilHFP95}\href{../works/WeilHFP95.pdf}{WeilHFP95} R\&C& \cellcolor{red!20}\href{../works/BourdaisGP03.pdf}{BourdaisGP03} (0.88)& \cellcolor{blue!20}\href{../works/OzturkTHO12.pdf}{OzturkTHO12} (0.97)& \cellcolor{blue!20}\href{../works/TopalogluO11.pdf}{TopalogluO11} (0.97)& \cellcolor{blue!20}\href{../works/GoelSHFS15.pdf}{GoelSHFS15} (0.97)& \cellcolor{blue!20}\href{../works/NovasH12.pdf}{NovasH12} (0.97)\\
Euclid& \cellcolor{red!20}\href{../works/HoYCLLCLC18.pdf}{HoYCLLCLC18} (0.26)& \cellcolor{red!20}\href{../works/BourdaisGP03.pdf}{BourdaisGP03} (0.26)& \cellcolor{yellow!20}\href{../works/abs-1902-01193.pdf}{abs-1902-01193} (0.26)& \cellcolor{yellow!20}\href{../works/AngelsmarkJ00.pdf}{AngelsmarkJ00} (0.28)& \cellcolor{yellow!20}\href{../works/Puget95.pdf}{Puget95} (0.28)\\
Dot& \cellcolor{red!40}\href{../works/Dejemeppe16.pdf}{Dejemeppe16} (81.00)& \cellcolor{red!40}\href{../works/ZarandiASC20.pdf}{ZarandiASC20} (72.00)& \cellcolor{red!40}\href{../works/Simonis07.pdf}{Simonis07} (68.00)& \cellcolor{red!40}\href{../works/WangMD15.pdf}{WangMD15} (66.00)& \cellcolor{red!40}\href{../works/Wolf11.pdf}{Wolf11} (65.00)\\
Cosine& \cellcolor{red!40}\href{../works/BourdaisGP03.pdf}{BourdaisGP03} (0.67)& \cellcolor{red!40}\href{../works/abs-1902-01193.pdf}{abs-1902-01193} (0.67)& \cellcolor{red!40}\href{../works/HoYCLLCLC18.pdf}{HoYCLLCLC18} (0.65)& \cellcolor{red!40}\href{../works/ShinBBHO18.pdf}{ShinBBHO18} (0.64)& \cellcolor{red!40}\href{../works/WangMD15.pdf}{WangMD15} (0.64)\\
\index{WessenCS20}\href{../works/WessenCS20.pdf}{WessenCS20} R\&C& \cellcolor{red!40}\href{../works/BehrensLM19.pdf}{BehrensLM19} (0.79)& \cellcolor{red!40}\href{../works/WessenCSFPM23.pdf}{WessenCSFPM23} (0.80)& \cellcolor{red!40}\href{../works/WallaceY20.pdf}{WallaceY20} (0.86)& \cellcolor{green!20}\href{../works/GarridoAO09.pdf}{GarridoAO09} (0.96)& \cellcolor{blue!20}\href{../works/OzturkTHO15.pdf}{OzturkTHO15} (0.97)\\
Euclid& \cellcolor{red!40}\href{../works/BehrensLM19.pdf}{BehrensLM19} (0.24)& \cellcolor{red!40}\href{../works/FortinZDF05.pdf}{FortinZDF05} (0.24)& \cellcolor{red!20}\href{../works/abs-1901-07914.pdf}{abs-1901-07914} (0.25)& \cellcolor{red!20}\href{../works/NishikawaSTT18.pdf}{NishikawaSTT18} (0.26)& \cellcolor{red!20}\href{../works/NishikawaSTT18a.pdf}{NishikawaSTT18a} (0.26)\\
Dot& \cellcolor{red!40}\href{../works/Astrand21.pdf}{Astrand21} (86.00)& \cellcolor{red!40}\href{../works/ZarandiASC20.pdf}{ZarandiASC20} (82.00)& \cellcolor{red!40}\href{../works/ArmstrongGOS21.pdf}{ArmstrongGOS21} (81.00)& \cellcolor{red!40}\href{../works/Lunardi20.pdf}{Lunardi20} (81.00)& \cellcolor{red!40}\href{../works/Dejemeppe16.pdf}{Dejemeppe16} (81.00)\\
Cosine& \cellcolor{red!40}\href{../works/BehrensLM19.pdf}{BehrensLM19} (0.77)& \cellcolor{red!40}\href{../works/MokhtarzadehTNF20.pdf}{MokhtarzadehTNF20} (0.76)& \cellcolor{red!40}\href{../works/abs-1901-07914.pdf}{abs-1901-07914} (0.74)& \cellcolor{red!40}\href{../works/WessenCSFPM23.pdf}{WessenCSFPM23} (0.74)& \cellcolor{red!40}\href{../works/NishikawaSTT18.pdf}{NishikawaSTT18} (0.73)\\
\index{WessenCSFPM23}\href{../works/WessenCSFPM23.pdf}{WessenCSFPM23} R\&C& \cellcolor{red!40}\href{../works/WessenCS20.pdf}{WessenCS20} (0.80)& \cellcolor{yellow!20}\href{../works/BehrensLM19.pdf}{BehrensLM19} (0.93)& \cellcolor{green!20}\href{../works/BonfiettiLBM11.pdf}{BonfiettiLBM11} (0.96)& \cellcolor{blue!20}\href{../works/BonfiettiLBM14.pdf}{BonfiettiLBM14} (0.96)& \cellcolor{blue!20}\href{../works/KreterSS15.pdf}{KreterSS15} (0.96)\\
Euclid& \cellcolor{green!20}\href{../works/WessenCS20.pdf}{WessenCS20} (0.31)& \cellcolor{blue!20}\href{../works/BonfiettiLBM11.pdf}{BonfiettiLBM11} (0.32)& \cellcolor{blue!20}\href{../works/BonfiettiLBM12.pdf}{BonfiettiLBM12} (0.33)& \cellcolor{blue!20}\href{../works/LombardiBMB11.pdf}{LombardiBMB11} (0.33)& \cellcolor{blue!20}\href{../works/BonfiettiLBM14.pdf}{BonfiettiLBM14} (0.33)\\
Dot& \cellcolor{red!40}\href{../works/Dejemeppe16.pdf}{Dejemeppe16} (126.00)& \cellcolor{red!40}\href{../works/Astrand21.pdf}{Astrand21} (125.00)& \cellcolor{red!40}\href{../works/LaborieRSV18.pdf}{LaborieRSV18} (121.00)& \cellcolor{red!40}\href{../works/Malapert11.pdf}{Malapert11} (121.00)& \cellcolor{red!40}\href{../works/Godet21a.pdf}{Godet21a} (119.00)\\
Cosine& \cellcolor{red!40}\href{../works/BonfiettiLBM14.pdf}{BonfiettiLBM14} (0.75)& \cellcolor{red!40}\href{../works/WessenCS20.pdf}{WessenCS20} (0.74)& \cellcolor{red!40}\href{../works/BonfiettiLBM11.pdf}{BonfiettiLBM11} (0.73)& \cellcolor{red!40}\href{../works/LombardiBMB11.pdf}{LombardiBMB11} (0.71)& \cellcolor{red!40}\href{../works/DraperJCJ99.pdf}{DraperJCJ99} (0.71)\\
\index{WikarekS19}\href{../works/WikarekS19.pdf}{WikarekS19} R\&C& \cellcolor{green!20}\href{../works/SchnellH17.pdf}{SchnellH17} (0.94)& \cellcolor{green!20}\href{../works/BartakSR08.pdf}{BartakSR08} (0.94)& \cellcolor{green!20}\href{../works/OddiPCC03.pdf}{OddiPCC03} (0.94)& \cellcolor{green!20}\href{../works/ElkhyariGJ02.pdf}{ElkhyariGJ02} (0.94)& \cellcolor{green!20}\href{../works/HebrardHJMPV16.pdf}{HebrardHJMPV16} (0.95)\\
Euclid& \cellcolor{green!20}\href{../works/KovacsV06.pdf}{KovacsV06} (0.29)& \cellcolor{green!20}\href{../works/KhayatLR06.pdf}{KhayatLR06} (0.30)& \cellcolor{green!20}\href{../works/ValleMGT03.pdf}{ValleMGT03} (0.30)& \cellcolor{green!20}\href{../works/MalapertCGJLR13.pdf}{MalapertCGJLR13} (0.31)& \cellcolor{green!20}\href{../works/HeipckeCCS00.pdf}{HeipckeCCS00} (0.31)\\
Dot& \cellcolor{red!40}\href{../works/Baptiste02.pdf}{Baptiste02} (141.00)& \cellcolor{red!40}\href{../works/ZarandiASC20.pdf}{ZarandiASC20} (139.00)& \cellcolor{red!40}\href{../works/Malapert11.pdf}{Malapert11} (131.00)& \cellcolor{red!40}\href{../works/Groleaz21.pdf}{Groleaz21} (130.00)& \cellcolor{red!40}\href{../works/BartakSR10.pdf}{BartakSR10} (127.00)\\
Cosine& \cellcolor{red!40}\href{../works/GrimesH10.pdf}{GrimesH10} (0.76)& \cellcolor{red!40}\href{../works/KhayatLR06.pdf}{KhayatLR06} (0.75)& \cellcolor{red!40}\href{../works/BosiM2001.pdf}{BosiM2001} (0.75)& \cellcolor{red!40}\href{../works/KovacsV06.pdf}{KovacsV06} (0.74)& \cellcolor{red!40}\href{../works/TorresL00.pdf}{TorresL00} (0.73)\\
\index{WinterMMW22}\href{../works/WinterMMW22.pdf}{WinterMMW22} R\&C\\
Euclid& \href{../works/GedikKEK18.pdf}{GedikKEK18} (0.44)& \href{../works/GroleazNS20a.pdf}{GroleazNS20a} (0.44)& \href{../works/ArbaouiY18.pdf}{ArbaouiY18} (0.44)& \href{../works/NattafM20.pdf}{NattafM20} (0.45)& \href{../works/Adelgren2023.pdf}{Adelgren2023} (0.45)\\
Dot& \cellcolor{red!40}\href{../works/Groleaz21.pdf}{Groleaz21} (170.00)& \cellcolor{red!40}\href{../works/Lunardi20.pdf}{Lunardi20} (162.00)& \cellcolor{red!40}\href{../works/NaderiRR23.pdf}{NaderiRR23} (158.00)& \cellcolor{red!40}\href{../works/ZarandiASC20.pdf}{ZarandiASC20} (147.00)& \cellcolor{red!40}\href{../works/LacknerMMWW23.pdf}{LacknerMMWW23} (147.00)\\
Cosine& \cellcolor{red!40}\href{../works/GedikKEK18.pdf}{GedikKEK18} (0.65)& \cellcolor{red!40}\href{../works/YunusogluY22.pdf}{YunusogluY22} (0.64)& \cellcolor{red!40}\href{../works/OujanaAYB22.pdf}{OujanaAYB22} (0.64)& \cellcolor{red!40}\href{../works/GroleazNS20a.pdf}{GroleazNS20a} (0.63)& \cellcolor{red!40}\href{../works/NaderiBZ22.pdf}{NaderiBZ22} (0.62)\\
\index{Wolf03}\href{../works/Wolf03.pdf}{Wolf03} R\&C& \cellcolor{red!40}\href{../works/WolfS05a.pdf}{WolfS05a} (0.58)& \cellcolor{red!40}\href{../works/Wolf05.pdf}{Wolf05} (0.60)& \cellcolor{red!40}\href{../works/Colombani96.pdf}{Colombani96} (0.77)& \cellcolor{red!40}\href{../works/Vilim04.pdf}{Vilim04} (0.79)& \cellcolor{red!40}\href{../works/Goltz95.pdf}{Goltz95} (0.80)\\
Euclid& \cellcolor{red!40}\href{../works/Wolf05.pdf}{Wolf05} (0.19)& \cellcolor{red!40}\href{../works/MonetteDD07.pdf}{MonetteDD07} (0.24)& \cellcolor{yellow!20}\href{../works/TanSD10.pdf}{TanSD10} (0.28)& \cellcolor{green!20}\href{../works/CauwelaertDMS16.pdf}{CauwelaertDMS16} (0.30)& \cellcolor{green!20}\href{../works/VilimBC04.pdf}{VilimBC04} (0.31)\\
Dot& \cellcolor{red!40}\href{../works/Malapert11.pdf}{Malapert11} (152.00)& \cellcolor{red!40}\href{../works/Fahimi16.pdf}{Fahimi16} (150.00)& \cellcolor{red!40}\href{../works/Schutt11.pdf}{Schutt11} (149.00)& \cellcolor{red!40}\href{../works/Dejemeppe16.pdf}{Dejemeppe16} (147.00)& \cellcolor{red!40}\href{../works/Baptiste02.pdf}{Baptiste02} (147.00)\\
Cosine& \cellcolor{red!40}\href{../works/Wolf05.pdf}{Wolf05} (0.91)& \cellcolor{red!40}\href{../works/MonetteDD07.pdf}{MonetteDD07} (0.86)& \cellcolor{red!40}\href{../works/TanSD10.pdf}{TanSD10} (0.79)& \cellcolor{red!40}\href{../works/FahimiOQ18.pdf}{FahimiOQ18} (0.78)& \cellcolor{red!40}\href{../works/CauwelaertDMS16.pdf}{CauwelaertDMS16} (0.75)\\
\index{Wolf05}\href{../works/Wolf05.pdf}{Wolf05} R\&C& \cellcolor{red!40}\href{../works/Wolf03.pdf}{Wolf03} (0.60)& \cellcolor{red!40}\href{../works/WolfS05a.pdf}{WolfS05a} (0.64)& \cellcolor{red!40}\href{../works/Vilim05.pdf}{Vilim05} (0.65)& \cellcolor{red!40}\href{../works/ArtiouchineB05.pdf}{ArtiouchineB05} (0.70)& \cellcolor{red!40}\href{../works/WolfS05.pdf}{WolfS05} (0.79)\\
Euclid& \cellcolor{red!40}\href{../works/Wolf03.pdf}{Wolf03} (0.19)& \cellcolor{red!20}\href{../works/CauwelaertDMS16.pdf}{CauwelaertDMS16} (0.25)& \cellcolor{yellow!20}\href{../works/MonetteDD07.pdf}{MonetteDD07} (0.27)& \cellcolor{yellow!20}\href{../works/VilimBC05.pdf}{VilimBC05} (0.28)& \cellcolor{yellow!20}\href{../works/VilimBC04.pdf}{VilimBC04} (0.28)\\
Dot& \cellcolor{red!40}\href{../works/Malapert11.pdf}{Malapert11} (151.00)& \cellcolor{red!40}\href{../works/Fahimi16.pdf}{Fahimi16} (149.00)& \cellcolor{red!40}\href{../works/Schutt11.pdf}{Schutt11} (149.00)& \cellcolor{red!40}\href{../works/Dejemeppe16.pdf}{Dejemeppe16} (147.00)& \cellcolor{red!40}\href{../works/Baptiste02.pdf}{Baptiste02} (145.00)\\
Cosine& \cellcolor{red!40}\href{../works/Wolf03.pdf}{Wolf03} (0.91)& \cellcolor{red!40}\href{../works/CauwelaertDMS16.pdf}{CauwelaertDMS16} (0.83)& \cellcolor{red!40}\href{../works/MonetteDD07.pdf}{MonetteDD07} (0.81)& \cellcolor{red!40}\href{../works/DejemeppeCS15.pdf}{DejemeppeCS15} (0.80)& \cellcolor{red!40}\href{../works/VilimBC05.pdf}{VilimBC05} (0.80)\\
\index{Wolf09}\href{../works/Wolf09.pdf}{Wolf09} R\&C& \cellcolor{red!40}\href{../works/WolfS05a.pdf}{WolfS05a} (0.61)& \cellcolor{red!40}\href{../works/WolfS05.pdf}{WolfS05} (0.72)& \cellcolor{red!40}\href{../works/Wolf05.pdf}{Wolf05} (0.80)& \cellcolor{red!40}\href{../works/SchuttWS05.pdf}{SchuttWS05} (0.81)& \cellcolor{red!40}\href{../works/SchuttW10.pdf}{SchuttW10} (0.85)\\
Euclid& \cellcolor{green!20}\href{../works/WolfS05.pdf}{WolfS05} (0.30)& \cellcolor{green!20}\href{../works/Caseau97.pdf}{Caseau97} (0.30)& \cellcolor{blue!20}\href{../works/BeldiceanuP07.pdf}{BeldiceanuP07} (0.32)& \cellcolor{blue!20}\href{../works/BockmayrP06.pdf}{BockmayrP06} (0.32)& \cellcolor{blue!20}\href{../works/Puget95.pdf}{Puget95} (0.33)\\
Dot& \cellcolor{red!40}\href{../works/Dejemeppe16.pdf}{Dejemeppe16} (99.00)& \cellcolor{red!40}\href{../works/Malapert11.pdf}{Malapert11} (98.00)& \cellcolor{red!40}\href{../works/Baptiste02.pdf}{Baptiste02} (93.00)& \cellcolor{red!40}\href{../works/Schutt11.pdf}{Schutt11} (90.00)& \cellcolor{red!40}\href{../works/Beck99.pdf}{Beck99} (89.00)\\
Cosine& \cellcolor{red!40}\href{../works/WolfS05.pdf}{WolfS05} (0.67)& \cellcolor{red!40}\href{../works/AlesioNBG14.pdf}{AlesioNBG14} (0.66)& \cellcolor{red!40}\href{../works/Wolf05.pdf}{Wolf05} (0.65)& \cellcolor{red!40}\href{../works/Caseau97.pdf}{Caseau97} (0.64)& \cellcolor{red!40}\href{../works/BeldiceanuP07.pdf}{BeldiceanuP07} (0.61)\\
\index{Wolf11}\href{../works/Wolf11.pdf}{Wolf11} R\&C& \cellcolor{red!40}\href{../works/SimonisH11.pdf}{SimonisH11} (0.86)& \cellcolor{red!20}\href{../works/SimoninAHL12.pdf}{SimoninAHL12} (0.89)& \cellcolor{yellow!20}\href{../works/OuelletQ22.pdf}{OuelletQ22} (0.90)& \cellcolor{yellow!20}\href{../works/GayHS15a.pdf}{GayHS15a} (0.90)& \cellcolor{yellow!20}\href{../works/Wolf09.pdf}{Wolf09} (0.90)\\
Euclid& \cellcolor{black!20}\href{../works/DoulabiRP16.pdf}{DoulabiRP16} (0.35)& \cellcolor{black!20}\href{../works/DoulabiRP14.pdf}{DoulabiRP14} (0.36)& \cellcolor{black!20}\href{../works/WolfS05.pdf}{WolfS05} (0.36)& \cellcolor{black!20}\href{../works/GurPAE23.pdf}{GurPAE23} (0.36)& \cellcolor{black!20}\href{../works/WangMD15.pdf}{WangMD15} (0.36)\\
Dot& \cellcolor{red!40}\href{../works/Dejemeppe16.pdf}{Dejemeppe16} (127.00)& \cellcolor{red!40}\href{../works/ZarandiASC20.pdf}{ZarandiASC20} (124.00)& \cellcolor{red!40}\href{../works/Simonis07.pdf}{Simonis07} (105.00)& \cellcolor{red!40}\href{../works/Malapert11.pdf}{Malapert11} (101.00)& \cellcolor{red!40}\href{../works/RoshanaeiLAU17.pdf}{RoshanaeiLAU17} (100.00)\\
Cosine& \cellcolor{red!40}\href{../works/WangMD15.pdf}{WangMD15} (0.68)& \cellcolor{red!40}\href{../works/DoulabiRP16.pdf}{DoulabiRP16} (0.68)& \cellcolor{red!40}\href{../works/GurPAE23.pdf}{GurPAE23} (0.66)& \cellcolor{red!40}\href{../works/ZhaoL14.pdf}{ZhaoL14} (0.66)& \cellcolor{red!40}\href{../works/RoshanaeiLAU17.pdf}{RoshanaeiLAU17} (0.65)\\
\index{WolfS05}\href{../works/WolfS05.pdf}{WolfS05} R\&C& \cellcolor{red!40}\href{../works/SchuttWS05.pdf}{SchuttWS05} (0.45)& \cellcolor{red!40}\href{../works/Wolf09.pdf}{Wolf09} (0.72)& \cellcolor{red!40}\href{../works/SchuttW10.pdf}{SchuttW10} (0.74)& \cellcolor{red!40}\href{../works/WolfS05a.pdf}{WolfS05a} (0.75)& \cellcolor{red!40}\href{../works/MercierH08.pdf}{MercierH08} (0.78)\\
Euclid& \cellcolor{red!40}\href{../works/BeldiceanuP07.pdf}{BeldiceanuP07} (0.14)& \cellcolor{red!40}\href{../works/PoderB08.pdf}{PoderB08} (0.15)& \cellcolor{red!40}\href{../works/Vilim09a.pdf}{Vilim09a} (0.18)& \cellcolor{red!40}\href{../works/SimonisH11.pdf}{SimonisH11} (0.20)& \cellcolor{red!40}\href{../works/BockmayrP06.pdf}{BockmayrP06} (0.20)\\
Dot& \cellcolor{red!40}\href{../works/Schutt11.pdf}{Schutt11} (85.00)& \cellcolor{red!40}\href{../works/Lombardi10.pdf}{Lombardi10} (84.00)& \cellcolor{red!40}\href{../works/Malapert11.pdf}{Malapert11} (84.00)& \cellcolor{red!40}\href{../works/Fahimi16.pdf}{Fahimi16} (83.00)& \cellcolor{red!40}\href{../works/Baptiste02.pdf}{Baptiste02} (82.00)\\
Cosine& \cellcolor{red!40}\href{../works/BeldiceanuP07.pdf}{BeldiceanuP07} (0.90)& \cellcolor{red!40}\href{../works/PoderB08.pdf}{PoderB08} (0.89)& \cellcolor{red!40}\href{../works/Vilim09a.pdf}{Vilim09a} (0.85)& \cellcolor{red!40}\href{../works/SimonisH11.pdf}{SimonisH11} (0.81)& \cellcolor{red!40}\href{../works/BockmayrP06.pdf}{BockmayrP06} (0.81)\\
\index{WolfS05a}\href{../works/WolfS05a.pdf}{WolfS05a} R\&C& \cellcolor{red!40}\href{../works/Wolf03.pdf}{Wolf03} (0.58)& \cellcolor{red!40}\href{../works/Wolf09.pdf}{Wolf09} (0.61)& \cellcolor{red!40}\href{../works/Wolf05.pdf}{Wolf05} (0.64)& \cellcolor{red!40}\href{../works/WolfS05.pdf}{WolfS05} (0.75)& \cellcolor{red!40}\href{../works/Vilim04.pdf}{Vilim04} (0.76)\\
Euclid& \cellcolor{green!20}\href{../works/QuSN06.pdf}{QuSN06} (0.29)& \cellcolor{green!20}\href{../works/WolfS05.pdf}{WolfS05} (0.29)& \cellcolor{green!20}\href{../works/PoderB08.pdf}{PoderB08} (0.29)& \cellcolor{green!20}\href{../works/BeldiceanuC01.pdf}{BeldiceanuC01} (0.29)& \cellcolor{green!20}\href{../works/BeldiceanuP07.pdf}{BeldiceanuP07} (0.30)\\
Dot& \cellcolor{red!40}\href{../works/Malapert11.pdf}{Malapert11} (92.00)& \cellcolor{red!40}\href{../works/Schutt11.pdf}{Schutt11} (79.00)& \cellcolor{red!40}\href{../works/Fahimi16.pdf}{Fahimi16} (75.00)& \cellcolor{red!40}\href{../works/Letort13.pdf}{Letort13} (72.00)& \cellcolor{red!40}\href{../works/Dejemeppe16.pdf}{Dejemeppe16} (71.00)\\
Cosine& \cellcolor{red!40}\href{../works/LetortCB13.pdf}{LetortCB13} (0.66)& \cellcolor{red!40}\href{../works/LetortBC12.pdf}{LetortBC12} (0.65)& \cellcolor{red!40}\href{../works/WolfS05.pdf}{WolfS05} (0.64)& \cellcolor{red!40}\href{../works/BeldiceanuC02.pdf}{BeldiceanuC02} (0.64)& \cellcolor{red!40}\href{../works/Madi-WambaLOBM17.pdf}{Madi-WambaLOBM17} (0.62)\\
\index{WolinskiKG04}\href{../works/WolinskiKG04.pdf}{WolinskiKG04} R\&C& \cellcolor{yellow!20}\href{../works/QuSN06.pdf}{QuSN06} (0.93)& \cellcolor{blue!20}\href{../works/KuchcinskiW03.pdf}{KuchcinskiW03} (0.96)& \cellcolor{blue!20}\href{../works/NishikawaSTT19.pdf}{NishikawaSTT19} (0.97)& \cellcolor{blue!20}\href{../works/LombardiM10a.pdf}{LombardiM10a} (0.97)& \cellcolor{blue!20}\href{../works/LozanoCDS12.pdf}{LozanoCDS12} (0.97)\\
Euclid& \cellcolor{red!40}\href{../works/KuchcinskiW03.pdf}{KuchcinskiW03} (0.22)& \cellcolor{red!20}\href{../works/LozanoCDS12.pdf}{LozanoCDS12} (0.25)& \cellcolor{red!20}\href{../works/Kumar03.pdf}{Kumar03} (0.25)& \cellcolor{red!20}\href{../works/LombardiM13.pdf}{LombardiM13} (0.25)& \cellcolor{red!20}\href{../works/CestaOS98.pdf}{CestaOS98} (0.26)\\
Dot& \cellcolor{red!40}\href{../works/Lombardi10.pdf}{Lombardi10} (71.00)& \cellcolor{red!40}\href{../works/Kuchcinski03.pdf}{Kuchcinski03} (64.00)& \cellcolor{red!40}\href{../works/Malapert11.pdf}{Malapert11} (63.00)& \cellcolor{red!40}\href{../works/BonfiettiLBM14.pdf}{BonfiettiLBM14} (62.00)& \cellcolor{red!40}\href{../works/Groleaz21.pdf}{Groleaz21} (62.00)\\
Cosine& \cellcolor{red!40}\href{../works/KuchcinskiW03.pdf}{KuchcinskiW03} (0.77)& \cellcolor{red!40}\href{../works/LozanoCDS12.pdf}{LozanoCDS12} (0.70)& \cellcolor{red!40}\href{../works/BocewiczBB09.pdf}{BocewiczBB09} (0.69)& \cellcolor{red!40}\href{../works/Malik08.pdf}{Malik08} (0.66)& \cellcolor{red!40}\href{../works/LombardiM13.pdf}{LombardiM13} (0.64)\\
\index{WuBB05}\href{../works/WuBB05.pdf}{WuBB05} R\&C\\
Euclid& \cellcolor{red!40}\href{../works/CestaOS98.pdf}{CestaOS98} (0.13)& \cellcolor{red!40}\href{../works/Baptiste09.pdf}{Baptiste09} (0.14)& \cellcolor{red!40}\href{../works/KovacsEKV05.pdf}{KovacsEKV05} (0.15)& \cellcolor{red!40}\href{../works/Caballero23.pdf}{Caballero23} (0.15)& \cellcolor{red!40}\href{../works/CarchraeBF05.pdf}{CarchraeBF05} (0.15)\\
Dot& \cellcolor{red!40}\href{../works/Groleaz21.pdf}{Groleaz21} (33.00)& \cellcolor{red!40}\href{../works/Lombardi10.pdf}{Lombardi10} (32.00)& \cellcolor{red!40}\href{../works/Beck99.pdf}{Beck99} (31.00)& \cellcolor{red!40}\href{../works/SadehF96.pdf}{SadehF96} (31.00)& \cellcolor{red!40}\href{../works/LaborieRSV18.pdf}{LaborieRSV18} (30.00)\\
Cosine& \cellcolor{red!40}\href{../works/CestaOS98.pdf}{CestaOS98} (0.69)& \cellcolor{red!40}\href{../works/OddiS97.pdf}{OddiS97} (0.63)& \cellcolor{red!40}\href{../works/Laborie18a.pdf}{Laborie18a} (0.62)& \cellcolor{red!40}\href{../works/BeckW05.pdf}{BeckW05} (0.62)& \cellcolor{red!40}\href{../works/KovacsEKV05.pdf}{KovacsEKV05} (0.62)\\
\index{WuBB09}\href{../works/WuBB09.pdf}{WuBB09} R\&C& \cellcolor{yellow!20}FahimiQ23 (0.90)& \cellcolor{yellow!20}\href{../works/BonfiettiLM14.pdf}{BonfiettiLM14} (0.92)& \cellcolor{yellow!20}\href{../works/KletzanderM17.pdf}{KletzanderM17} (0.93)& \cellcolor{green!20}\href{../works/BidotVLB09.pdf}{BidotVLB09} (0.94)& \cellcolor{green!20}\href{../works/BeckW07.pdf}{BeckW07} (0.94)\\
Euclid& \cellcolor{green!20}\href{../works/BridiLBBM16.pdf}{BridiLBBM16} (0.30)& \cellcolor{green!20}\href{../works/TranPZLDB18.pdf}{TranPZLDB18} (0.31)& \cellcolor{blue!20}\href{../works/TranWDRFOVB16.pdf}{TranWDRFOVB16} (0.32)& \cellcolor{blue!20}\href{../works/MurphyMB15.pdf}{MurphyMB15} (0.32)& \cellcolor{blue!20}\href{../works/BeckPS03.pdf}{BeckPS03} (0.33)\\
Dot& \cellcolor{red!40}\href{../works/ZarandiASC20.pdf}{ZarandiASC20} (140.00)& \cellcolor{red!40}\href{../works/Baptiste02.pdf}{Baptiste02} (120.00)& \cellcolor{red!40}\href{../works/Lombardi10.pdf}{Lombardi10} (111.00)& \cellcolor{red!40}\href{../works/Beck99.pdf}{Beck99} (110.00)& \cellcolor{red!40}\href{../works/PrataAN23.pdf}{PrataAN23} (109.00)\\
Cosine& \cellcolor{red!40}\href{../works/BeckW07.pdf}{BeckW07} (0.72)& \cellcolor{red!40}\href{../works/BeckPS03.pdf}{BeckPS03} (0.70)& \cellcolor{red!40}\href{../works/TranPZLDB18.pdf}{TranPZLDB18} (0.70)& \cellcolor{red!40}\href{../works/BridiLBBM16.pdf}{BridiLBBM16} (0.68)& \cellcolor{red!40}\href{../works/Madi-WambaLOBM17.pdf}{Madi-WambaLOBM17} (0.68)\\
\index{YangSS19}\href{../works/YangSS19.pdf}{YangSS19} R\&C& \cellcolor{red!40}\href{../works/KameugneFSN14.pdf}{KameugneFSN14} (0.69)& \cellcolor{red!40}\href{../works/OuelletQ18.pdf}{OuelletQ18} (0.69)& \cellcolor{red!40}\href{../works/Tesch18.pdf}{Tesch18} (0.72)& \cellcolor{red!40}\href{../works/OuelletQ13.pdf}{OuelletQ13} (0.74)& \cellcolor{red!40}\href{../works/GayHS15.pdf}{GayHS15} (0.78)\\
Euclid& \cellcolor{yellow!20}\href{../works/Vilim09a.pdf}{Vilim09a} (0.28)& \cellcolor{green!20}\href{../works/WolfS05.pdf}{WolfS05} (0.29)& \cellcolor{green!20}\href{../works/BeldiceanuCP08.pdf}{BeldiceanuCP08} (0.30)& \cellcolor{green!20}\href{../works/Vilim09.pdf}{Vilim09} (0.31)& \cellcolor{green!20}\href{../works/Vilim11.pdf}{Vilim11} (0.31)\\
Dot& \cellcolor{red!40}\href{../works/Schutt11.pdf}{Schutt11} (121.00)& \cellcolor{red!40}\href{../works/Malapert11.pdf}{Malapert11} (115.00)& \cellcolor{red!40}\href{../works/Dejemeppe16.pdf}{Dejemeppe16} (110.00)& \cellcolor{red!40}\href{../works/Lombardi10.pdf}{Lombardi10} (110.00)& \cellcolor{red!40}\href{../works/Fahimi16.pdf}{Fahimi16} (110.00)\\
Cosine& \cellcolor{red!40}\href{../works/Vilim09a.pdf}{Vilim09a} (0.74)& \cellcolor{red!40}\href{../works/SchuttFS13a.pdf}{SchuttFS13a} (0.73)& \cellcolor{red!40}\href{../works/WolfS05.pdf}{WolfS05} (0.72)& \cellcolor{red!40}\href{../works/Vilim11.pdf}{Vilim11} (0.72)& \cellcolor{red!40}\href{../works/BeldiceanuCP08.pdf}{BeldiceanuCP08} (0.71)\\
\index{YeGMH94}YeGMH94 R\&C\\
Euclid\\
Dot\\
Cosine\\
\index{YoshikawaKNW94}\href{../works/YoshikawaKNW94.pdf}{YoshikawaKNW94} R\&C\\
Euclid& \cellcolor{red!40}\href{../works/LimAHO02a.pdf}{LimAHO02a} (0.19)& \cellcolor{red!40}\href{../works/FeldmanG89.pdf}{FeldmanG89} (0.20)& \cellcolor{red!40}\href{../works/CarchraeBF05.pdf}{CarchraeBF05} (0.23)& \cellcolor{red!40}\href{../works/FrostD98.pdf}{FrostD98} (0.23)& \cellcolor{red!40}\href{../works/AngelsmarkJ00.pdf}{AngelsmarkJ00} (0.23)\\
Dot& \cellcolor{red!40}\href{../works/Lemos21.pdf}{Lemos21} (52.00)& \cellcolor{red!40}\href{../works/ZarandiASC20.pdf}{ZarandiASC20} (46.00)& \cellcolor{red!40}\href{../works/Astrand21.pdf}{Astrand21} (45.00)& \cellcolor{red!40}\href{../works/KendallKRU10.pdf}{KendallKRU10} (45.00)& \cellcolor{red!40}\href{../works/Fahimi16.pdf}{Fahimi16} (43.00)\\
Cosine& \cellcolor{red!40}\href{../works/LimAHO02a.pdf}{LimAHO02a} (0.71)& \cellcolor{red!40}\href{../works/FeldmanG89.pdf}{FeldmanG89} (0.65)& \cellcolor{red!40}\href{../works/Schaerf97.pdf}{Schaerf97} (0.62)& \cellcolor{red!40}\href{../works/BofillGSV15.pdf}{BofillGSV15} (0.58)& \cellcolor{red!40}\href{../works/DemirovicS18.pdf}{DemirovicS18} (0.57)\\
\index{YounespourAKE19}\href{../works/YounespourAKE19.pdf}{YounespourAKE19} R\&C& \cellcolor{red!40}\href{../works/RoshanaeiBAUB20.pdf}{RoshanaeiBAUB20} (0.79)& \cellcolor{red!40}\href{../works/GhandehariK22.pdf}{GhandehariK22} (0.86)& \cellcolor{red!20}\href{../works/FarsiTM22.pdf}{FarsiTM22} (0.87)& \cellcolor{red!20}\href{../works/BenediktMH20.pdf}{BenediktMH20} (0.88)& \cellcolor{red!20}\href{../works/WangMD15.pdf}{WangMD15} (0.89)\\
Euclid& \cellcolor{green!20}\href{../works/WangMD15.pdf}{WangMD15} (0.29)& \cellcolor{blue!20}\href{../works/MeskensDHG11.pdf}{MeskensDHG11} (0.32)& \cellcolor{black!20}\href{../works/MeskensDL13.pdf}{MeskensDL13} (0.34)& \cellcolor{black!20}\href{../works/GhandehariK22.pdf}{GhandehariK22} (0.35)& \cellcolor{black!20}\href{../works/GurPAE23.pdf}{GurPAE23} (0.36)\\
Dot& \cellcolor{red!40}\href{../works/ZarandiASC20.pdf}{ZarandiASC20} (137.00)& \cellcolor{red!40}\href{../works/Dejemeppe16.pdf}{Dejemeppe16} (123.00)& \cellcolor{red!40}\href{../works/WangMD15.pdf}{WangMD15} (119.00)& \cellcolor{red!40}\href{../works/Lunardi20.pdf}{Lunardi20} (113.00)& \cellcolor{red!40}\href{../works/Baptiste02.pdf}{Baptiste02} (112.00)\\
Cosine& \cellcolor{red!40}\href{../works/WangMD15.pdf}{WangMD15} (0.81)& \cellcolor{red!40}\href{../works/MeskensDHG11.pdf}{MeskensDHG11} (0.74)& \cellcolor{red!40}\href{../works/MeskensDL13.pdf}{MeskensDL13} (0.72)& \cellcolor{red!40}\href{../works/GhandehariK22.pdf}{GhandehariK22} (0.69)& \cellcolor{red!40}\href{../works/SenderovichBB19.pdf}{SenderovichBB19} (0.68)\\
\index{YoungFS17}\href{../works/YoungFS17.pdf}{YoungFS17} R\&C& \cellcolor{red!40}\href{../works/SzerediS16.pdf}{SzerediS16} (0.57)& \cellcolor{red!40}\href{../works/GeibingerMM19.pdf}{GeibingerMM19} (0.75)& \cellcolor{red!40}\href{../works/KreterSS17.pdf}{KreterSS17} (0.81)& \cellcolor{red!40}\href{../works/SchuttS16.pdf}{SchuttS16} (0.84)& \cellcolor{red!40}\href{../works/SchuttFS13.pdf}{SchuttFS13} (0.85)\\
Euclid& \cellcolor{red!40}\href{../works/SzerediS16.pdf}{SzerediS16} (0.21)& \cellcolor{red!20}\href{../works/BofillCSV17a.pdf}{BofillCSV17a} (0.25)& \cellcolor{yellow!20}\href{../works/BofillCSV17.pdf}{BofillCSV17} (0.27)& \cellcolor{yellow!20}\href{../works/SchuttS16.pdf}{SchuttS16} (0.27)& \cellcolor{yellow!20}\href{../works/LiessM08.pdf}{LiessM08} (0.27)\\
Dot& \cellcolor{red!40}\href{../works/Godet21a.pdf}{Godet21a} (146.00)& \cellcolor{red!40}\href{../works/PovedaAA23.pdf}{PovedaAA23} (134.00)& \cellcolor{red!40}\href{../works/Schutt11.pdf}{Schutt11} (131.00)& \cellcolor{red!40}\href{../works/BoudreaultSLQ22.pdf}{BoudreaultSLQ22} (129.00)& \cellcolor{red!40}\href{../works/Lombardi10.pdf}{Lombardi10} (127.00)\\
Cosine& \cellcolor{red!40}\href{../works/SzerediS16.pdf}{SzerediS16} (0.88)& \cellcolor{red!40}\href{../works/PovedaAA23.pdf}{PovedaAA23} (0.84)& \cellcolor{red!40}\href{../works/BofillCSV17a.pdf}{BofillCSV17a} (0.82)& \cellcolor{red!40}\href{../works/BoudreaultSLQ22.pdf}{BoudreaultSLQ22} (0.81)& \cellcolor{red!40}\href{../works/SchuttS16.pdf}{SchuttS16} (0.80)\\
\index{YunusogluY22}\href{../works/YunusogluY22.pdf}{YunusogluY22} R\&C& \cellcolor{red!20}\href{../works/HeinzNVH22.pdf}{HeinzNVH22} (0.88)& \cellcolor{red!20}\href{../works/AbreuN22.pdf}{AbreuN22} (0.89)& \cellcolor{yellow!20}\href{../works/MengLZB21.pdf}{MengLZB21} (0.91)& \cellcolor{yellow!20}\href{../works/GedikKEK18.pdf}{GedikKEK18} (0.92)& \cellcolor{yellow!20}\href{../works/MengZRZL20.pdf}{MengZRZL20} (0.93)\\
Euclid& \href{../works/GedikKEK18.pdf}{GedikKEK18} (0.40)& \href{../works/IsikYA23.pdf}{IsikYA23} (0.41)& \href{../works/OrnekO16.pdf}{OrnekO16} (0.42)& \href{../works/Novas19.pdf}{Novas19} (0.42)& \href{../works/OujanaAYB22.pdf}{OujanaAYB22} (0.42)\\
Dot& \cellcolor{red!40}\href{../works/ZarandiASC20.pdf}{ZarandiASC20} (239.00)& \cellcolor{red!40}\href{../works/Lunardi20.pdf}{Lunardi20} (224.00)& \cellcolor{red!40}\href{../works/Groleaz21.pdf}{Groleaz21} (218.00)& \cellcolor{red!40}\href{../works/IsikYA23.pdf}{IsikYA23} (217.00)& \cellcolor{red!40}\href{../works/Dejemeppe16.pdf}{Dejemeppe16} (209.00)\\
Cosine& \cellcolor{red!40}\href{../works/IsikYA23.pdf}{IsikYA23} (0.80)& \cellcolor{red!40}\href{../works/GedikKEK18.pdf}{GedikKEK18} (0.75)& \cellcolor{red!40}\href{../works/Novas19.pdf}{Novas19} (0.74)& \cellcolor{red!40}\href{../works/Lunardi20.pdf}{Lunardi20} (0.73)& \cellcolor{red!40}\href{../works/MengZRZL20.pdf}{MengZRZL20} (0.73)\\
\index{YuraszeckMC23}\href{../works/YuraszeckMC23.pdf}{YuraszeckMC23} R\&C& \cellcolor{red!40}\href{../works/YuraszeckMPV22.pdf}{YuraszeckMPV22} (0.75)& \cellcolor{yellow!20}\href{../works/OddiPCC03.pdf}{OddiPCC03} (0.90)& \cellcolor{yellow!20}OddiPCC05 (0.93)& \cellcolor{green!20}\href{../works/WikarekS19.pdf}{WikarekS19} (0.96)& \cellcolor{green!20}\href{../works/HamdiL13.pdf}{HamdiL13} (0.96)\\
Euclid& \cellcolor{green!20}\href{../works/WatsonB08.pdf}{WatsonB08} (0.30)& \cellcolor{green!20}\href{../works/BillautHL12.pdf}{BillautHL12} (0.31)& \cellcolor{blue!20}\href{../works/Beck06.pdf}{Beck06} (0.32)& \cellcolor{blue!20}\href{../works/DilkinaDH05.pdf}{DilkinaDH05} (0.32)& \cellcolor{blue!20}\href{../works/ThiruvadyBME09.pdf}{ThiruvadyBME09} (0.32)\\
Dot& \cellcolor{red!40}\href{../works/ZarandiASC20.pdf}{ZarandiASC20} (118.00)& \cellcolor{red!40}\href{../works/Groleaz21.pdf}{Groleaz21} (114.00)& \cellcolor{red!40}\href{../works/Baptiste02.pdf}{Baptiste02} (113.00)& \cellcolor{red!40}\href{../works/YuraszeckMPV22.pdf}{YuraszeckMPV22} (107.00)& \cellcolor{red!40}\href{../works/Godet21a.pdf}{Godet21a} (106.00)\\
Cosine& \cellcolor{red!40}\href{../works/YuraszeckMCCR23.pdf}{YuraszeckMCCR23} (0.71)& \cellcolor{red!40}\href{../works/YuraszeckMPV22.pdf}{YuraszeckMPV22} (0.71)& \cellcolor{red!40}\href{../works/WatsonB08.pdf}{WatsonB08} (0.70)& \cellcolor{red!40}\href{../works/MenciaSV13.pdf}{MenciaSV13} (0.69)& \cellcolor{red!40}\href{../works/AbreuAPNM21.pdf}{AbreuAPNM21} (0.69)\\
\index{YuraszeckMCCR23}\href{../works/YuraszeckMCCR23.pdf}{YuraszeckMCCR23} R\&C\\
Euclid& \cellcolor{blue!20}\href{../works/OddiRCS11.pdf}{OddiRCS11} (0.32)& \cellcolor{blue!20}\href{../works/HeipckeCCS00.pdf}{HeipckeCCS00} (0.33)& \cellcolor{blue!20}\href{../works/LiessM08.pdf}{LiessM08} (0.34)& \cellcolor{black!20}\href{../works/VilimLS15.pdf}{VilimLS15} (0.34)& \cellcolor{black!20}\href{../works/KhayatLR06.pdf}{KhayatLR06} (0.35)\\
Dot& \cellcolor{red!40}\href{../works/ZarandiASC20.pdf}{ZarandiASC20} (181.00)& \cellcolor{red!40}\href{../works/Baptiste02.pdf}{Baptiste02} (163.00)& \cellcolor{red!40}\href{../works/Lunardi20.pdf}{Lunardi20} (161.00)& \cellcolor{red!40}\href{../works/Groleaz21.pdf}{Groleaz21} (159.00)& \cellcolor{red!40}\href{../works/Godet21a.pdf}{Godet21a} (158.00)\\
Cosine& \cellcolor{red!40}\href{../works/OddiRCS11.pdf}{OddiRCS11} (0.79)& \cellcolor{red!40}\href{../works/VilimLS15.pdf}{VilimLS15} (0.77)& \cellcolor{red!40}\href{../works/HeipckeCCS00.pdf}{HeipckeCCS00} (0.77)& \cellcolor{red!40}\href{../works/EtminaniesfahaniGNMS22.pdf}{EtminaniesfahaniGNMS22} (0.75)& \cellcolor{red!40}\href{../works/LiessM08.pdf}{LiessM08} (0.75)\\
\index{YuraszeckMPV22}\href{../works/YuraszeckMPV22.pdf}{YuraszeckMPV22} R\&C& \cellcolor{red!40}\href{../works/YuraszeckMC23.pdf}{YuraszeckMC23} (0.75)& \cellcolor{yellow!20}\href{../works/SubulanC22.pdf}{SubulanC22} (0.91)& \cellcolor{green!20}\href{../works/OujanaAYB22.pdf}{OujanaAYB22} (0.94)& \cellcolor{green!20}\href{../works/SchnellH17.pdf}{SchnellH17} (0.94)& \cellcolor{green!20}\href{../works/AbreuPNF23.pdf}{AbreuPNF23} (0.95)\\
Euclid& \href{../works/AbreuAPNM21.pdf}{AbreuAPNM21} (0.38)& \href{../works/YuraszeckMC23.pdf}{YuraszeckMC23} (0.39)& \href{../works/LiFJZLL22.pdf}{LiFJZLL22} (0.41)& \href{../works/MejiaY20.pdf}{MejiaY20} (0.41)& \href{../works/MalapertCGJLR13.pdf}{MalapertCGJLR13} (0.42)\\
Dot& \cellcolor{red!40}\href{../works/ZarandiASC20.pdf}{ZarandiASC20} (209.00)& \cellcolor{red!40}\href{../works/Groleaz21.pdf}{Groleaz21} (201.00)& \cellcolor{red!40}\href{../works/Baptiste02.pdf}{Baptiste02} (187.00)& \cellcolor{red!40}\href{../works/Lunardi20.pdf}{Lunardi20} (173.00)& \cellcolor{red!40}\href{../works/Astrand21.pdf}{Astrand21} (169.00)\\
Cosine& \cellcolor{red!40}\href{../works/AbreuAPNM21.pdf}{AbreuAPNM21} (0.75)& \cellcolor{red!40}\href{../works/AbreuN22.pdf}{AbreuN22} (0.72)& \cellcolor{red!40}\href{../works/MejiaY20.pdf}{MejiaY20} (0.72)& \cellcolor{red!40}\href{../works/AbreuPNF23.pdf}{AbreuPNF23} (0.72)& \cellcolor{red!40}\href{../works/YuraszeckMC23.pdf}{YuraszeckMC23} (0.71)\\
\index{Zahout21}\href{../works/Zahout21.pdf}{Zahout21} R\&C\\
Euclid& \href{../works/ZhangYW21.pdf}{ZhangYW21} (0.43)& \href{../works/HillTV21.pdf}{HillTV21} (0.44)& \href{../works/HeipckeCCS00.pdf}{HeipckeCCS00} (0.45)& \href{../works/TranPZLDB18.pdf}{TranPZLDB18} (0.45)& \href{../works/NishikawaSTT19.pdf}{NishikawaSTT19} (0.45)\\
Dot& \cellcolor{red!40}\href{../works/ZarandiASC20.pdf}{ZarandiASC20} (200.00)& \cellcolor{red!40}\href{../works/Groleaz21.pdf}{Groleaz21} (190.00)& \cellcolor{red!40}\href{../works/Dejemeppe16.pdf}{Dejemeppe16} (167.00)& \cellcolor{red!40}\href{../works/Lombardi10.pdf}{Lombardi10} (166.00)& \cellcolor{red!40}\href{../works/Baptiste02.pdf}{Baptiste02} (159.00)\\
Cosine& \cellcolor{red!40}\href{../works/ZhangYW21.pdf}{ZhangYW21} (0.65)& \cellcolor{red!40}\href{../works/JuvinHL23a.pdf}{JuvinHL23a} (0.64)& \cellcolor{red!40}\href{../works/HillTV21.pdf}{HillTV21} (0.63)& \cellcolor{red!40}\href{../works/NaderiBZ22a.pdf}{NaderiBZ22a} (0.63)& \cellcolor{red!40}\href{../works/Elkhyari03.pdf}{Elkhyari03} (0.61)\\
\index{ZampelliVSDR13}\href{../works/ZampelliVSDR13.pdf}{ZampelliVSDR13} R\&C& \cellcolor{red!20}\href{../works/GaySS14.pdf}{GaySS14} (0.89)& \cellcolor{yellow!20}\href{../works/CauwelaertLS15.pdf}{CauwelaertLS15} (0.91)& \cellcolor{yellow!20}\href{../works/GrimesH10.pdf}{GrimesH10} (0.91)& \cellcolor{yellow!20}\href{../works/UnsalO13.pdf}{UnsalO13} (0.91)& \cellcolor{yellow!20}\href{../works/QinDS16.pdf}{QinDS16} (0.91)\\
Euclid& \cellcolor{green!20}\href{../works/WolfS05.pdf}{WolfS05} (0.30)& \cellcolor{green!20}\href{../works/Vilim09a.pdf}{Vilim09a} (0.30)& \cellcolor{green!20}\href{../works/PoderB08.pdf}{PoderB08} (0.31)& \cellcolor{blue!20}\href{../works/TranVNB17a.pdf}{TranVNB17a} (0.32)& \cellcolor{blue!20}\href{../works/GilesH16.pdf}{GilesH16} (0.33)\\
Dot& \cellcolor{red!40}\href{../works/Dejemeppe16.pdf}{Dejemeppe16} (128.00)& \cellcolor{red!40}\href{../works/ZarandiASC20.pdf}{ZarandiASC20} (113.00)& \cellcolor{red!40}\href{../works/Groleaz21.pdf}{Groleaz21} (107.00)& \cellcolor{red!40}\href{../works/Malapert11.pdf}{Malapert11} (106.00)& \cellcolor{red!40}\href{../works/AwadMDMT22.pdf}{AwadMDMT22} (105.00)\\
Cosine& \cellcolor{red!40}\href{../works/WolfS05.pdf}{WolfS05} (0.71)& \cellcolor{red!40}\href{../works/Vilim09a.pdf}{Vilim09a} (0.70)& \cellcolor{red!40}\href{../works/EvenSH15.pdf}{EvenSH15} (0.68)& \cellcolor{red!40}\href{../works/UnsalO13.pdf}{UnsalO13} (0.68)& \cellcolor{red!40}\href{../works/UnsalO19.pdf}{UnsalO19} (0.68)\\
\index{ZarandiASC20}\href{../works/ZarandiASC20.pdf}{ZarandiASC20} R\&C& \cellcolor{blue!20}\href{../works/SacramentoSP20.pdf}{SacramentoSP20} (0.98)& \cellcolor{blue!20}\href{../works/MengLZB21.pdf}{MengLZB21} (0.98)& \cellcolor{blue!20}\href{../works/MengZRZL20.pdf}{MengZRZL20} (0.98)& \cellcolor{blue!20}\href{../works/GokgurHO18.pdf}{GokgurHO18} (0.98)& \cellcolor{blue!20}\href{../works/HamFC17.pdf}{HamFC17} (0.98)\\
Euclid& \href{../works/PrataAN23.pdf}{PrataAN23} (0.67)& \href{../works/Lunardi20.pdf}{Lunardi20} (0.70)& \href{../works/JainM99.pdf}{JainM99} (0.72)& \href{../works/IsikYA23.pdf}{IsikYA23} (0.72)& \href{../works/Astrand21.pdf}{Astrand21} (0.72)\\
Dot& \cellcolor{red!40}\href{../works/Groleaz21.pdf}{Groleaz21} (368.00)& \cellcolor{red!40}\href{../works/Dejemeppe16.pdf}{Dejemeppe16} (335.00)& \cellcolor{red!40}\href{../works/Baptiste02.pdf}{Baptiste02} (331.00)& \cellcolor{red!40}\href{../works/Lunardi20.pdf}{Lunardi20} (322.00)& \cellcolor{red!40}\href{../works/Astrand21.pdf}{Astrand21} (306.00)\\
Cosine& \cellcolor{red!40}\href{../works/PrataAN23.pdf}{PrataAN23} (0.71)& \cellcolor{red!40}\href{../works/Lunardi20.pdf}{Lunardi20} (0.67)& \cellcolor{red!40}\href{../works/Groleaz21.pdf}{Groleaz21} (0.67)& \cellcolor{red!40}\href{../works/Astrand21.pdf}{Astrand21} (0.65)& \cellcolor{red!40}\href{../works/JainM99.pdf}{JainM99} (0.65)\\
\index{ZarandiB12}ZarandiB12 R\&C& \cellcolor{red!40}HechingHK19 (0.74)& \cellcolor{red!40}\href{../works/RoshanaeiLAU17.pdf}{RoshanaeiLAU17} (0.75)& \cellcolor{red!40}\href{../works/Hooker07.pdf}{Hooker07} (0.75)& \cellcolor{red!40}\href{../works/CireCH16.pdf}{CireCH16} (0.77)& \cellcolor{red!40}\href{../works/TranAB16.pdf}{TranAB16} (0.78)\\
Euclid\\
Dot\\
Cosine\\
\index{ZarandiKS16}\href{../works/ZarandiKS16.pdf}{ZarandiKS16} R\&C& \cellcolor{yellow!20}\href{../works/NovasH14.pdf}{NovasH14} (0.91)& \cellcolor{yellow!20}\href{../works/QuirogaZH05.pdf}{QuirogaZH05} (0.92)& \cellcolor{yellow!20}\href{../works/AbreuN22.pdf}{AbreuN22} (0.92)& \cellcolor{yellow!20}\href{../works/KelbelH11.pdf}{KelbelH11} (0.93)& \cellcolor{yellow!20}\href{../works/PengLC14.pdf}{PengLC14} (0.93)\\
Euclid& \cellcolor{green!20}\href{../works/Salido10.pdf}{Salido10} (0.30)& \cellcolor{blue!20}\href{../works/KamarainenS02.pdf}{KamarainenS02} (0.32)& \cellcolor{blue!20}\href{../works/PengLC14.pdf}{PengLC14} (0.33)& \cellcolor{black!20}\href{../works/MakMS10.pdf}{MakMS10} (0.34)& \cellcolor{black!20}\href{../works/ZhangLS12.pdf}{ZhangLS12} (0.35)\\
Dot& \cellcolor{red!40}\href{../works/ZarandiASC20.pdf}{ZarandiASC20} (168.00)& \cellcolor{red!40}\href{../works/Groleaz21.pdf}{Groleaz21} (143.00)& \cellcolor{red!40}\href{../works/Dejemeppe16.pdf}{Dejemeppe16} (138.00)& \cellcolor{red!40}\href{../works/Baptiste02.pdf}{Baptiste02} (131.00)& \cellcolor{red!40}\href{../works/Lombardi10.pdf}{Lombardi10} (128.00)\\
Cosine& \cellcolor{red!40}\href{../works/Salido10.pdf}{Salido10} (0.72)& \cellcolor{red!40}\href{../works/PengLC14.pdf}{PengLC14} (0.71)& \cellcolor{red!40}\href{../works/BeckDDF98.pdf}{BeckDDF98} (0.70)& \cellcolor{red!40}\href{../works/BartakSR10.pdf}{BartakSR10} (0.68)& \cellcolor{red!40}\href{../works/PrataAN23.pdf}{PrataAN23} (0.67)\\
\index{Zeballos10}\href{../works/Zeballos10.pdf}{Zeballos10} R\&C& \cellcolor{red!40}\href{../works/ZeballosQH10.pdf}{ZeballosQH10} (0.54)& \cellcolor{red!40}\href{../works/NovasH14.pdf}{NovasH14} (0.72)& \cellcolor{red!40}\href{../works/QuirogaZH05.pdf}{QuirogaZH05} (0.76)& \cellcolor{red!40}\href{../works/ZeballosM09.pdf}{ZeballosM09} (0.83)& \cellcolor{red!20}\href{../works/ZeballosNH11.pdf}{ZeballosNH11} (0.87)\\
Euclid& \cellcolor{red!40}\href{../works/QuirogaZH05.pdf}{QuirogaZH05} (0.19)& \cellcolor{red!40}\href{../works/ZeballosQH10.pdf}{ZeballosQH10} (0.22)& \cellcolor{red!40}\href{../works/ZeballosH05.pdf}{ZeballosH05} (0.23)& \cellcolor{red!20}\href{../works/ZeballosM09.pdf}{ZeballosM09} (0.26)& \cellcolor{yellow!20}\href{../works/NovasH14.pdf}{NovasH14} (0.28)\\
Dot& \cellcolor{red!40}\href{../works/ZarandiASC20.pdf}{ZarandiASC20} (152.00)& \cellcolor{red!40}\href{../works/Dejemeppe16.pdf}{Dejemeppe16} (144.00)& \cellcolor{red!40}\href{../works/ZeballosQH10.pdf}{ZeballosQH10} (135.00)& \cellcolor{red!40}\href{../works/Malapert11.pdf}{Malapert11} (131.00)& \cellcolor{red!40}\href{../works/Baptiste02.pdf}{Baptiste02} (131.00)\\
Cosine& \cellcolor{red!40}\href{../works/QuirogaZH05.pdf}{QuirogaZH05} (0.92)& \cellcolor{red!40}\href{../works/ZeballosQH10.pdf}{ZeballosQH10} (0.89)& \cellcolor{red!40}\href{../works/ZeballosH05.pdf}{ZeballosH05} (0.88)& \cellcolor{red!40}\href{../works/ZeballosM09.pdf}{ZeballosM09} (0.83)& \cellcolor{red!40}\href{../works/NovasH14.pdf}{NovasH14} (0.80)\\
\index{ZeballosCM10}\href{../works/ZeballosCM10.pdf}{ZeballosCM10} R\&C& \cellcolor{red!40}\href{../works/NovasH12.pdf}{NovasH12} (0.78)& \cellcolor{red!40}\href{../works/ZeballosNH11.pdf}{ZeballosNH11} (0.79)& \cellcolor{red!40}\href{../works/ZeballosM09.pdf}{ZeballosM09} (0.83)& \cellcolor{red!40}\href{../works/NovaraNH16.pdf}{NovaraNH16} (0.84)& \cellcolor{red!20}\href{../works/ZeballosQH10.pdf}{ZeballosQH10} (0.90)\\
Euclid& \cellcolor{yellow!20}\href{../works/NovasH12.pdf}{NovasH12} (0.27)& \cellcolor{green!20}\href{../works/NovasH14.pdf}{NovasH14} (0.29)& \cellcolor{green!20}\href{../works/ZeballosM09.pdf}{ZeballosM09} (0.30)& \cellcolor{black!20}\href{../works/ZeballosH05.pdf}{ZeballosH05} (0.34)& \cellcolor{black!20}\href{../works/QuirogaZH05.pdf}{QuirogaZH05} (0.36)\\
Dot& \cellcolor{red!40}\href{../works/ZarandiASC20.pdf}{ZarandiASC20} (150.00)& \cellcolor{red!40}\href{../works/Astrand21.pdf}{Astrand21} (134.00)& \cellcolor{red!40}\href{../works/LaborieRSV18.pdf}{LaborieRSV18} (132.00)& \cellcolor{red!40}\href{../works/Malapert11.pdf}{Malapert11} (127.00)& \cellcolor{red!40}\href{../works/Lunardi20.pdf}{Lunardi20} (126.00)\\
Cosine& \cellcolor{red!40}\href{../works/NovasH12.pdf}{NovasH12} (0.84)& \cellcolor{red!40}\href{../works/NovasH14.pdf}{NovasH14} (0.81)& \cellcolor{red!40}\href{../works/ZeballosM09.pdf}{ZeballosM09} (0.77)& \cellcolor{red!40}\href{../works/ZeballosH05.pdf}{ZeballosH05} (0.72)& \cellcolor{red!40}\href{../works/Zeballos10.pdf}{Zeballos10} (0.69)\\
\index{ZeballosH05}\href{../works/ZeballosH05.pdf}{ZeballosH05} R\&C\\
Euclid& \cellcolor{red!40}\href{../works/NovasH14.pdf}{NovasH14} (0.22)& \cellcolor{red!40}\href{../works/Zeballos10.pdf}{Zeballos10} (0.23)& \cellcolor{red!40}\href{../works/QuirogaZH05.pdf}{QuirogaZH05} (0.23)& \cellcolor{red!40}\href{../works/ZeballosM09.pdf}{ZeballosM09} (0.24)& \cellcolor{red!20}\href{../works/BeckPS03.pdf}{BeckPS03} (0.26)\\
Dot& \cellcolor{red!40}\href{../works/ZarandiASC20.pdf}{ZarandiASC20} (143.00)& \cellcolor{red!40}\href{../works/Dejemeppe16.pdf}{Dejemeppe16} (137.00)& \cellcolor{red!40}\href{../works/Baptiste02.pdf}{Baptiste02} (134.00)& \cellcolor{red!40}\href{../works/Lunardi20.pdf}{Lunardi20} (133.00)& \cellcolor{red!40}\href{../works/Malapert11.pdf}{Malapert11} (133.00)\\
Cosine& \cellcolor{red!40}\href{../works/Zeballos10.pdf}{Zeballos10} (0.88)& \cellcolor{red!40}\href{../works/NovasH14.pdf}{NovasH14} (0.87)& \cellcolor{red!40}\href{../works/QuirogaZH05.pdf}{QuirogaZH05} (0.86)& \cellcolor{red!40}\href{../works/ZeballosM09.pdf}{ZeballosM09} (0.85)& \cellcolor{red!40}\href{../works/ZeballosQH10.pdf}{ZeballosQH10} (0.84)\\
\index{ZeballosM09}\href{../works/ZeballosM09.pdf}{ZeballosM09} R\&C& \cellcolor{red!40}\href{../works/ZeballosNH11.pdf}{ZeballosNH11} (0.78)& \cellcolor{red!40}\href{../works/ZeballosCM10.pdf}{ZeballosCM10} (0.83)& \cellcolor{red!40}\href{../works/Zeballos10.pdf}{Zeballos10} (0.83)& \cellcolor{red!40}\href{../works/RoePS05.pdf}{RoePS05} (0.84)& \cellcolor{red!40}\href{../works/ZeballosQH10.pdf}{ZeballosQH10} (0.86)\\
Euclid& \cellcolor{red!40}\href{../works/ZeballosH05.pdf}{ZeballosH05} (0.24)& \cellcolor{red!20}\href{../works/NovasH14.pdf}{NovasH14} (0.24)& \cellcolor{red!20}\href{../works/QuirogaZH05.pdf}{QuirogaZH05} (0.25)& \cellcolor{red!20}\href{../works/Zeballos10.pdf}{Zeballos10} (0.26)& \cellcolor{red!20}\href{../works/MaraveliasCG04.pdf}{MaraveliasCG04} (0.26)\\
Dot& \cellcolor{red!40}\href{../works/ZarandiASC20.pdf}{ZarandiASC20} (119.00)& \cellcolor{red!40}\href{../works/ZeballosQH10.pdf}{ZeballosQH10} (108.00)& \cellcolor{red!40}\href{../works/Dejemeppe16.pdf}{Dejemeppe16} (108.00)& \cellcolor{red!40}\href{../works/Malapert11.pdf}{Malapert11} (108.00)& \cellcolor{red!40}\href{../works/BeckDDF98.pdf}{BeckDDF98} (108.00)\\
Cosine& \cellcolor{red!40}\href{../works/ZeballosH05.pdf}{ZeballosH05} (0.85)& \cellcolor{red!40}\href{../works/NovasH14.pdf}{NovasH14} (0.83)& \cellcolor{red!40}\href{../works/Zeballos10.pdf}{Zeballos10} (0.83)& \cellcolor{red!40}\href{../works/ZeballosQH10.pdf}{ZeballosQH10} (0.83)& \cellcolor{red!40}\href{../works/QuirogaZH05.pdf}{QuirogaZH05} (0.81)\\
\index{ZeballosNH11}\href{../works/ZeballosNH11.pdf}{ZeballosNH11} R\&C& \cellcolor{red!40}\href{../works/NovaraNH16.pdf}{NovaraNH16} (0.63)& \cellcolor{red!40}\href{../works/ZeballosM09.pdf}{ZeballosM09} (0.78)& \cellcolor{red!40}\href{../works/ZeballosCM10.pdf}{ZeballosCM10} (0.79)& \cellcolor{red!40}\href{../works/MaraveliasCG04.pdf}{MaraveliasCG04} (0.85)& \cellcolor{red!40}\href{../works/ZeballosQH10.pdf}{ZeballosQH10} (0.86)\\
Euclid& \cellcolor{blue!20}\href{../works/NovaraNH16.pdf}{NovaraNH16} (0.32)& \cellcolor{blue!20}\href{../works/NovasH10.pdf}{NovasH10} (0.32)& \cellcolor{blue!20}\href{../works/ZeballosH05.pdf}{ZeballosH05} (0.33)& \cellcolor{blue!20}\href{../works/ZeballosM09.pdf}{ZeballosM09} (0.33)& \cellcolor{blue!20}\href{../works/QuirogaZH05.pdf}{QuirogaZH05} (0.33)\\
Dot& \cellcolor{red!40}\href{../works/ZarandiASC20.pdf}{ZarandiASC20} (166.00)& \cellcolor{red!40}\href{../works/Dejemeppe16.pdf}{Dejemeppe16} (165.00)& \cellcolor{red!40}\href{../works/Baptiste02.pdf}{Baptiste02} (157.00)& \cellcolor{red!40}\href{../works/Groleaz21.pdf}{Groleaz21} (151.00)& \cellcolor{red!40}\href{../works/Malapert11.pdf}{Malapert11} (150.00)\\
Cosine& \cellcolor{red!40}\href{../works/NovaraNH16.pdf}{NovaraNH16} (0.80)& \cellcolor{red!40}\href{../works/NovasH10.pdf}{NovasH10} (0.78)& \cellcolor{red!40}\href{../works/ZeballosH05.pdf}{ZeballosH05} (0.76)& \cellcolor{red!40}\href{../works/ZeballosQH10.pdf}{ZeballosQH10} (0.75)& \cellcolor{red!40}\href{../works/Zeballos10.pdf}{Zeballos10} (0.75)\\
\index{ZeballosQH10}\href{../works/ZeballosQH10.pdf}{ZeballosQH10} R\&C& \cellcolor{red!40}\href{../works/Zeballos10.pdf}{Zeballos10} (0.54)& \cellcolor{red!40}\href{../works/NovasH14.pdf}{NovasH14} (0.71)& \cellcolor{red!40}\href{../works/QuirogaZH05.pdf}{QuirogaZH05} (0.83)& \cellcolor{red!40}\href{../works/KhayatLR06.pdf}{KhayatLR06} (0.84)& \cellcolor{red!40}\href{../works/ZeballosM09.pdf}{ZeballosM09} (0.86)\\
Euclid& \cellcolor{red!40}\href{../works/Zeballos10.pdf}{Zeballos10} (0.22)& \cellcolor{red!40}\href{../works/QuirogaZH05.pdf}{QuirogaZH05} (0.23)& \cellcolor{yellow!20}\href{../works/ZeballosH05.pdf}{ZeballosH05} (0.27)& \cellcolor{yellow!20}\href{../works/ZeballosM09.pdf}{ZeballosM09} (0.27)& \cellcolor{green!20}\href{../works/NovasH14.pdf}{NovasH14} (0.29)\\
Dot& \cellcolor{red!40}\href{../works/ZarandiASC20.pdf}{ZarandiASC20} (161.00)& \cellcolor{red!40}\href{../works/Dejemeppe16.pdf}{Dejemeppe16} (156.00)& \cellcolor{red!40}\href{../works/Lunardi20.pdf}{Lunardi20} (150.00)& \cellcolor{red!40}\href{../works/Malapert11.pdf}{Malapert11} (150.00)& \cellcolor{red!40}\href{../works/Baptiste02.pdf}{Baptiste02} (149.00)\\
Cosine& \cellcolor{red!40}\href{../works/Zeballos10.pdf}{Zeballos10} (0.89)& \cellcolor{red!40}\href{../works/QuirogaZH05.pdf}{QuirogaZH05} (0.88)& \cellcolor{red!40}\href{../works/ZeballosH05.pdf}{ZeballosH05} (0.84)& \cellcolor{red!40}\href{../works/ZeballosM09.pdf}{ZeballosM09} (0.83)& \cellcolor{red!40}\href{../works/NovasH14.pdf}{NovasH14} (0.80)\\
\index{ZengM12}\href{../works/ZengM12.pdf}{ZengM12} R\&C& \cellcolor{red!40}Trick11 (0.75)& \cellcolor{red!40}\href{../works/RasmussenT09.pdf}{RasmussenT09} (0.76)& \cellcolor{red!40}\href{../works/RasmussenT07.pdf}{RasmussenT07} (0.79)& \cellcolor{red!40}\href{../works/CarlssonJL17.pdf}{CarlssonJL17} (0.80)& \cellcolor{red!40}\href{../works/RasmussenT06.pdf}{RasmussenT06} (0.81)\\
Euclid& \cellcolor{red!40}\href{../works/RasmussenT09.pdf}{RasmussenT09} (0.18)& \cellcolor{red!40}\href{../works/EastonNT02.pdf}{EastonNT02} (0.18)& \cellcolor{red!40}\href{../works/RasmussenT06.pdf}{RasmussenT06} (0.19)& \cellcolor{red!40}\href{../works/RasmussenT07.pdf}{RasmussenT07} (0.19)& \cellcolor{red!40}\href{../works/NaqviAIAAA22.pdf}{NaqviAIAAA22} (0.21)\\
Dot& \cellcolor{red!40}\href{../works/KendallKRU10.pdf}{KendallKRU10} (90.00)& \cellcolor{red!40}\href{../works/Ribeiro12.pdf}{Ribeiro12} (80.00)& \cellcolor{red!40}\href{../works/RasmussenT09.pdf}{RasmussenT09} (79.00)& \cellcolor{red!40}\href{../works/RasmussenT07.pdf}{RasmussenT07} (72.00)& \cellcolor{red!40}\href{../works/LarsonJC14.pdf}{LarsonJC14} (71.00)\\
Cosine& \cellcolor{red!40}\href{../works/RasmussenT09.pdf}{RasmussenT09} (0.88)& \cellcolor{red!40}\href{../works/EastonNT02.pdf}{EastonNT02} (0.85)& \cellcolor{red!40}\href{../works/RasmussenT07.pdf}{RasmussenT07} (0.85)& \cellcolor{red!40}\href{../works/RasmussenT06.pdf}{RasmussenT06} (0.83)& \cellcolor{red!40}\href{../works/NaqviAIAAA22.pdf}{NaqviAIAAA22} (0.80)\\
\index{ZhangBB22}\href{../works/ZhangBB22.pdf}{ZhangBB22} R\&C\\
Euclid& \cellcolor{green!20}\href{../works/TanSD10.pdf}{TanSD10} (0.29)& \cellcolor{green!20}\href{../works/FontaineMH16.pdf}{FontaineMH16} (0.31)& \cellcolor{green!20}\href{../works/ArtiguesF07.pdf}{ArtiguesF07} (0.31)& \cellcolor{green!20}\href{../works/ArtiguesBF04.pdf}{ArtiguesBF04} (0.31)& \cellcolor{blue!20}\href{../works/WatsonB08.pdf}{WatsonB08} (0.33)\\
Dot& \cellcolor{red!40}\href{../works/Groleaz21.pdf}{Groleaz21} (169.00)& \cellcolor{red!40}\href{../works/NaderiRR23.pdf}{NaderiRR23} (152.00)& \cellcolor{red!40}\href{../works/Dejemeppe16.pdf}{Dejemeppe16} (146.00)& \cellcolor{red!40}\href{../works/Baptiste02.pdf}{Baptiste02} (144.00)& \cellcolor{red!40}\href{../works/Malapert11.pdf}{Malapert11} (141.00)\\
Cosine& \cellcolor{red!40}\href{../works/TanSD10.pdf}{TanSD10} (0.79)& \cellcolor{red!40}\href{../works/ArtiguesF07.pdf}{ArtiguesF07} (0.79)& \cellcolor{red!40}\href{../works/FontaineMH16.pdf}{FontaineMH16} (0.77)& \cellcolor{red!40}\href{../works/ArtiguesBF04.pdf}{ArtiguesBF04} (0.77)& \cellcolor{red!40}\href{../works/BeckFW11.pdf}{BeckFW11} (0.73)\\
\index{ZhangJZL22}\href{../works/ZhangJZL22.pdf}{ZhangJZL22} R\&C& \cellcolor{black!20}\href{../works/AwadMDMT22.pdf}{AwadMDMT22} (0.98)\\
Euclid& \cellcolor{blue!20}\href{../works/Bedhief21.pdf}{Bedhief21} (0.33)& \cellcolor{blue!20}\href{../works/LiFJZLL22.pdf}{LiFJZLL22} (0.33)& \cellcolor{blue!20}\href{../works/ArbaouiY18.pdf}{ArbaouiY18} (0.33)& \cellcolor{blue!20}\href{../works/Limtanyakul07.pdf}{Limtanyakul07} (0.33)& \cellcolor{blue!20}\href{../works/BenediktSMVH18.pdf}{BenediktSMVH18} (0.34)\\
Dot& \cellcolor{red!40}\href{../works/IsikYA23.pdf}{IsikYA23} (142.00)& \cellcolor{red!40}\href{../works/Groleaz21.pdf}{Groleaz21} (139.00)& \cellcolor{red!40}\href{../works/ZarandiASC20.pdf}{ZarandiASC20} (137.00)& \cellcolor{red!40}\href{../works/Astrand21.pdf}{Astrand21} (129.00)& \cellcolor{red!40}\href{../works/Lunardi20.pdf}{Lunardi20} (128.00)\\
Cosine& \cellcolor{red!40}\href{../works/MengLZB21.pdf}{MengLZB21} (0.75)& \cellcolor{red!40}\href{../works/MengGRZSC22.pdf}{MengGRZSC22} (0.75)& \cellcolor{red!40}\href{../works/IsikYA23.pdf}{IsikYA23} (0.73)& \cellcolor{red!40}\href{../works/Bedhief21.pdf}{Bedhief21} (0.73)& \cellcolor{red!40}\href{../works/LiFJZLL22.pdf}{LiFJZLL22} (0.71)\\
\index{ZhangLS12}\href{../works/ZhangLS12.pdf}{ZhangLS12} R\&C& \cellcolor{red!40}\href{../works/QuirogaZH05.pdf}{QuirogaZH05} (0.67)& \cellcolor{red!40}\href{../works/Geske05.pdf}{Geske05} (0.67)& \cellcolor{red!40}\href{../works/EvenSH15.pdf}{EvenSH15} (0.75)& \cellcolor{red!40}\href{../works/KovacsV04.pdf}{KovacsV04} (0.75)& \cellcolor{red!40}\href{../works/LimtanyakulS12.pdf}{LimtanyakulS12} (0.80)\\
Euclid& \cellcolor{red!40}\href{../works/ZibranR11.pdf}{ZibranR11} (0.19)& \cellcolor{red!40}\href{../works/GelainPRVW17.pdf}{GelainPRVW17} (0.21)& \cellcolor{red!40}\href{../works/FeldmanG89.pdf}{FeldmanG89} (0.21)& \cellcolor{red!40}\href{../works/FrostD98.pdf}{FrostD98} (0.21)& \cellcolor{red!40}\href{../works/ZibranR11a.pdf}{ZibranR11a} (0.22)\\
Dot& \cellcolor{red!40}\href{../works/Siala15a.pdf}{Siala15a} (61.00)& \cellcolor{red!40}\href{../works/Godet21a.pdf}{Godet21a} (59.00)& \cellcolor{red!40}\href{../works/ZarandiASC20.pdf}{ZarandiASC20} (58.00)& \cellcolor{red!40}\href{../works/Baptiste02.pdf}{Baptiste02} (57.00)& \cellcolor{red!40}\href{../works/ZhuSZW23.pdf}{ZhuSZW23} (56.00)\\
Cosine& \cellcolor{red!40}\href{../works/ZibranR11.pdf}{ZibranR11} (0.73)& \cellcolor{red!40}\href{../works/GelainPRVW17.pdf}{GelainPRVW17} (0.73)& \cellcolor{red!40}\href{../works/LiuLH19.pdf}{LiuLH19} (0.72)& \cellcolor{red!40}\href{../works/ZibranR11a.pdf}{ZibranR11a} (0.71)& \cellcolor{red!40}\href{../works/LiuLH19a.pdf}{LiuLH19a} (0.69)\\
\index{ZhangW18}\href{../works/ZhangW18.pdf}{ZhangW18} R\&C& \cellcolor{yellow!20}\href{../works/Novas19.pdf}{Novas19} (0.91)& \cellcolor{yellow!20}\href{../works/HamC16.pdf}{HamC16} (0.92)& \cellcolor{yellow!20}\href{../works/LunardiBLRV20.pdf}{LunardiBLRV20} (0.93)& \cellcolor{green!20}\href{../works/TerekhovDOB12.pdf}{TerekhovDOB12} (0.93)& \cellcolor{green!20}\href{../works/MengZRZL20.pdf}{MengZRZL20} (0.96)\\
Euclid& \cellcolor{black!20}\href{../works/LunardiBLRV20.pdf}{LunardiBLRV20} (0.36)& \cellcolor{black!20}\href{../works/JuvinHL23.pdf}{JuvinHL23} (0.37)& \cellcolor{black!20}\href{../works/Mehdizadeh-Somarin23.pdf}{Mehdizadeh-Somarin23} (0.37)& \cellcolor{black!20}\href{../works/LiFJZLL22.pdf}{LiFJZLL22} (0.37)& \href{../works/FanXG21.pdf}{FanXG21} (0.37)\\
Dot& \cellcolor{red!40}\href{../works/ZarandiASC20.pdf}{ZarandiASC20} (204.00)& \cellcolor{red!40}\href{../works/Groleaz21.pdf}{Groleaz21} (179.00)& \cellcolor{red!40}\href{../works/Lunardi20.pdf}{Lunardi20} (177.00)& \cellcolor{red!40}\href{../works/Astrand21.pdf}{Astrand21} (159.00)& \cellcolor{red!40}\href{../works/Baptiste02.pdf}{Baptiste02} (153.00)\\
Cosine& \cellcolor{red!40}\href{../works/LunardiBLRV20.pdf}{LunardiBLRV20} (0.76)& \cellcolor{red!40}\href{../works/FanXG21.pdf}{FanXG21} (0.72)& \cellcolor{red!40}\href{../works/HamPK21.pdf}{HamPK21} (0.71)& \cellcolor{red!40}\href{../works/AlfieriGPS23.pdf}{AlfieriGPS23} (0.71)& \cellcolor{red!40}\href{../works/MengZRZL20.pdf}{MengZRZL20} (0.71)\\
\index{ZhangYW21}\href{../works/ZhangYW21.pdf}{ZhangYW21} R\&C& \cellcolor{red!40}\href{../works/NaderiBZ22a.pdf}{NaderiBZ22a} (0.79)& \cellcolor{red!40}ShiYXQ22 (0.84)& \cellcolor{red!20}\href{../works/ZhuSZW23.pdf}{ZhuSZW23} (0.89)& \cellcolor{red!20}\href{../works/MengLZB21.pdf}{MengLZB21} (0.90)& \cellcolor{red!20}\href{../works/CilKLO22.pdf}{CilKLO22} (0.90)\\
Euclid& \cellcolor{red!20}\href{../works/KhayatLR06.pdf}{KhayatLR06} (0.26)& \cellcolor{yellow!20}\href{../works/HeipckeCCS00.pdf}{HeipckeCCS00} (0.28)& \cellcolor{yellow!20}\href{../works/PacinoH11.pdf}{PacinoH11} (0.28)& \cellcolor{green!20}\href{../works/WatsonB08.pdf}{WatsonB08} (0.29)& \cellcolor{green!20}\href{../works/LahimerLH11.pdf}{LahimerLH11} (0.30)\\
Dot& \cellcolor{red!40}\href{../works/ZarandiASC20.pdf}{ZarandiASC20} (163.00)& \cellcolor{red!40}\href{../works/Groleaz21.pdf}{Groleaz21} (148.00)& \cellcolor{red!40}\href{../works/Lunardi20.pdf}{Lunardi20} (141.00)& \cellcolor{red!40}\href{../works/Dejemeppe16.pdf}{Dejemeppe16} (141.00)& \cellcolor{red!40}\href{../works/IsikYA23.pdf}{IsikYA23} (136.00)\\
Cosine& \cellcolor{red!40}\href{../works/KhayatLR06.pdf}{KhayatLR06} (0.82)& \cellcolor{red!40}\href{../works/NaderiBZ22a.pdf}{NaderiBZ22a} (0.82)& \cellcolor{red!40}\href{../works/HamPK21.pdf}{HamPK21} (0.79)& \cellcolor{red!40}\href{../works/HeipckeCCS00.pdf}{HeipckeCCS00} (0.78)& \cellcolor{red!40}\href{../works/PacinoH11.pdf}{PacinoH11} (0.77)\\
\index{ZhaoL14}\href{../works/ZhaoL14.pdf}{ZhaoL14} R\&C& \cellcolor{red!40}\href{../works/MeskensDL13.pdf}{MeskensDL13} (0.73)& \cellcolor{red!40}RoshanaeiLAU17a (0.81)& \cellcolor{red!40}\href{../works/WangMD15.pdf}{WangMD15} (0.83)& \cellcolor{red!40}\href{../works/RoshanaeiBAUB20.pdf}{RoshanaeiBAUB20} (0.85)& \cellcolor{red!40}\href{../works/DoulabiRP16.pdf}{DoulabiRP16} (0.85)\\
Euclid& \cellcolor{green!20}\href{../works/GhandehariK22.pdf}{GhandehariK22} (0.29)& \cellcolor{green!20}\href{../works/DoulabiRP16.pdf}{DoulabiRP16} (0.30)& \cellcolor{blue!20}\href{../works/GurEA19.pdf}{GurEA19} (0.33)& \cellcolor{blue!20}\href{../works/RiiseML16.pdf}{RiiseML16} (0.33)& \cellcolor{black!20}\href{../works/GurPAE23.pdf}{GurPAE23} (0.34)\\
Dot& \cellcolor{red!40}\href{../works/ZarandiASC20.pdf}{ZarandiASC20} (124.00)& \cellcolor{red!40}\href{../works/Dejemeppe16.pdf}{Dejemeppe16} (109.00)& \cellcolor{red!40}\href{../works/RoshanaeiLAU17.pdf}{RoshanaeiLAU17} (103.00)& \cellcolor{red!40}\href{../works/RoshanaeiBAUB20.pdf}{RoshanaeiBAUB20} (103.00)& \cellcolor{red!40}\href{../works/GhandehariK22.pdf}{GhandehariK22} (101.00)\\
Cosine& \cellcolor{red!40}\href{../works/GhandehariK22.pdf}{GhandehariK22} (0.78)& \cellcolor{red!40}\href{../works/DoulabiRP16.pdf}{DoulabiRP16} (0.75)& \cellcolor{red!40}\href{../works/RiiseML16.pdf}{RiiseML16} (0.71)& \cellcolor{red!40}\href{../works/WangMD15.pdf}{WangMD15} (0.71)& \cellcolor{red!40}\href{../works/GurEA19.pdf}{GurEA19} (0.70)\\
\index{Zhou96}\href{../works/Zhou96.pdf}{Zhou96} R\&C& \cellcolor{red!40}\href{../works/Colombani96.pdf}{Colombani96} (0.60)& \cellcolor{red!40}\href{../works/ArtiouchineB05.pdf}{ArtiouchineB05} (0.83)& \cellcolor{red!40}\href{../works/MonetteDD07.pdf}{MonetteDD07} (0.84)& \cellcolor{red!40}\href{../works/Rodriguez07.pdf}{Rodriguez07} (0.85)& \cellcolor{red!40}\href{../works/BeckF00.pdf}{BeckF00} (0.85)\\
Euclid& \cellcolor{red!40}\href{../works/Zhou97.pdf}{Zhou97} (0.19)& \cellcolor{red!40}\href{../works/Colombani96.pdf}{Colombani96} (0.22)& \cellcolor{yellow!20}\href{../works/FoxAS82.pdf}{FoxAS82} (0.27)& \cellcolor{green!20}\href{../works/Rit86.pdf}{Rit86} (0.29)& \cellcolor{green!20}\href{../works/Goltz95.pdf}{Goltz95} (0.30)\\
Dot& \cellcolor{red!40}\href{../works/Baptiste02.pdf}{Baptiste02} (115.00)& \cellcolor{red!40}\href{../works/Zhou97.pdf}{Zhou97} (113.00)& \cellcolor{red!40}\href{../works/BartakSR10.pdf}{BartakSR10} (106.00)& \cellcolor{red!40}\href{../works/Groleaz21.pdf}{Groleaz21} (101.00)& \cellcolor{red!40}\href{../works/Lombardi10.pdf}{Lombardi10} (100.00)\\
Cosine& \cellcolor{red!40}\href{../works/Zhou97.pdf}{Zhou97} (0.91)& \cellcolor{red!40}\href{../works/Colombani96.pdf}{Colombani96} (0.84)& \cellcolor{red!40}\href{../works/MercierH07.pdf}{MercierH07} (0.74)& \cellcolor{red!40}\href{../works/BelhadjiI98.pdf}{BelhadjiI98} (0.72)& \cellcolor{red!40}\href{../works/FoxAS82.pdf}{FoxAS82} (0.71)\\
\index{Zhou97}\href{../works/Zhou97.pdf}{Zhou97} R\&C& \cellcolor{green!20}\href{../works/SacramentoSP20.pdf}{SacramentoSP20} (0.94)& \cellcolor{green!20}\href{../works/Wolf05.pdf}{Wolf05} (0.94)& \cellcolor{green!20}\href{../works/AbreuAPNM21.pdf}{AbreuAPNM21} (0.95)& \cellcolor{green!20}\href{../works/MengLZB21.pdf}{MengLZB21} (0.95)& \cellcolor{green!20}\href{../works/Bartak02.pdf}{Bartak02} (0.95)\\
Euclid& \cellcolor{red!40}\href{../works/Zhou96.pdf}{Zhou96} (0.19)& \cellcolor{yellow!20}\href{../works/Colombani96.pdf}{Colombani96} (0.28)& \cellcolor{blue!20}\href{../works/HookerO99.pdf}{HookerO99} (0.32)& \cellcolor{blue!20}\href{../works/HeipckeCCS00.pdf}{HeipckeCCS00} (0.33)& \cellcolor{blue!20}\href{../works/Goltz95.pdf}{Goltz95} (0.33)\\
Dot& \cellcolor{red!40}\href{../works/Baptiste02.pdf}{Baptiste02} (158.00)& \cellcolor{red!40}\href{../works/Lombardi10.pdf}{Lombardi10} (141.00)& \cellcolor{red!40}\href{../works/Beck99.pdf}{Beck99} (140.00)& \cellcolor{red!40}\href{../works/Malapert11.pdf}{Malapert11} (139.00)& \cellcolor{red!40}\href{../works/Fahimi16.pdf}{Fahimi16} (139.00)\\
Cosine& \cellcolor{red!40}\href{../works/Zhou96.pdf}{Zhou96} (0.91)& \cellcolor{red!40}\href{../works/Colombani96.pdf}{Colombani96} (0.80)& \cellcolor{red!40}\href{../works/MercierH07.pdf}{MercierH07} (0.74)& \cellcolor{red!40}\href{../works/BartakSR08.pdf}{BartakSR08} (0.74)& \cellcolor{red!40}\href{../works/HookerO99.pdf}{HookerO99} (0.74)\\
\index{ZhouGL15}\href{../works/ZhouGL15.pdf}{ZhouGL15} R\&C& \cellcolor{yellow!20}\href{../works/SimoninAHL12.pdf}{SimoninAHL12} (0.92)& \cellcolor{yellow!20}\href{../works/LombardiM10.pdf}{LombardiM10} (0.93)& \cellcolor{yellow!20}\href{../works/GrimesHM09.pdf}{GrimesHM09} (0.93)& \cellcolor{yellow!20}OddiPCC05 (0.93)& \cellcolor{green!20}\href{../works/WatsonB08.pdf}{WatsonB08} (0.94)\\
Euclid& \cellcolor{green!20}\href{../works/JuvinHL23.pdf}{JuvinHL23} (0.31)& \cellcolor{blue!20}\href{../works/ParkUJR19.pdf}{ParkUJR19} (0.33)& \cellcolor{black!20}\href{../works/ArmstrongGOS22.pdf}{ArmstrongGOS22} (0.35)& \cellcolor{black!20}\href{../works/HebrardHJMPV16.pdf}{HebrardHJMPV16} (0.36)& \cellcolor{black!20}\href{../works/abs-2305-19888.pdf}{abs-2305-19888} (0.36)\\
Dot& \cellcolor{red!40}\href{../works/ZarandiASC20.pdf}{ZarandiASC20} (153.00)& \cellcolor{red!40}\href{../works/Groleaz21.pdf}{Groleaz21} (151.00)& \cellcolor{red!40}\href{../works/Astrand21.pdf}{Astrand21} (149.00)& \cellcolor{red!40}\href{../works/Lunardi20.pdf}{Lunardi20} (137.00)& \cellcolor{red!40}\href{../works/MengZRZL20.pdf}{MengZRZL20} (134.00)\\
Cosine& \cellcolor{red!40}\href{../works/JuvinHL23.pdf}{JuvinHL23} (0.75)& \cellcolor{red!40}\href{../works/OujanaAYB22.pdf}{OujanaAYB22} (0.73)& \cellcolor{red!40}\href{../works/TerekhovTDB14.pdf}{TerekhovTDB14} (0.73)& \cellcolor{red!40}\href{../works/ParkUJR19.pdf}{ParkUJR19} (0.72)& \cellcolor{red!40}\href{../works/ArmstrongGOS22.pdf}{ArmstrongGOS22} (0.71)\\
\index{ZhuS02}\href{../works/ZhuS02.pdf}{ZhuS02} R\&C\\
Euclid& \cellcolor{red!40}\href{../works/GelainPRVW17.pdf}{GelainPRVW17} (0.22)& \cellcolor{red!40}\href{../works/Caballero23.pdf}{Caballero23} (0.23)& \cellcolor{red!40}\href{../works/KovacsEKV05.pdf}{KovacsEKV05} (0.23)& \cellcolor{red!40}\href{../works/CestaOS98.pdf}{CestaOS98} (0.23)& \cellcolor{red!40}\href{../works/BofillCGGPSV23.pdf}{BofillCGGPSV23} (0.23)\\
Dot& \cellcolor{red!40}\href{../works/Simonis07.pdf}{Simonis07} (47.00)& \cellcolor{red!40}\href{../works/Wallace96.pdf}{Wallace96} (47.00)& \cellcolor{red!40}\href{../works/ZarandiASC20.pdf}{ZarandiASC20} (47.00)& \cellcolor{red!40}\href{../works/BadicaBI20.pdf}{BadicaBI20} (46.00)& \cellcolor{red!40}\href{../works/BartakSR10.pdf}{BartakSR10} (46.00)\\
Cosine& \cellcolor{red!40}\href{../works/GelainPRVW17.pdf}{GelainPRVW17} (0.69)& \cellcolor{red!40}\href{../works/QuSN06.pdf}{QuSN06} (0.64)& \cellcolor{red!40}\href{../works/BofillCGGPSV23.pdf}{BofillCGGPSV23} (0.63)& \cellcolor{red!40}\href{../works/BeniniBGM05a.pdf}{BeniniBGM05a} (0.61)& \cellcolor{red!40}\href{../works/Caballero23.pdf}{Caballero23} (0.60)\\
\index{ZhuSZW23}\href{../works/ZhuSZW23.pdf}{ZhuSZW23} R\&C& \cellcolor{red!40}ShiYXQ22 (0.73)& \cellcolor{red!20}\href{../works/NaderiBZ22a.pdf}{NaderiBZ22a} (0.88)& \cellcolor{red!20}\href{../works/ZhangYW21.pdf}{ZhangYW21} (0.89)& \cellcolor{green!20}\href{../works/TanSD10.pdf}{TanSD10} (0.94)& \cellcolor{green!20}\href{../works/LunardiBLRV20.pdf}{LunardiBLRV20} (0.94)\\
Euclid& \cellcolor{blue!20}\href{../works/NaderiBZ22a.pdf}{NaderiBZ22a} (0.34)& \cellcolor{black!20}\href{../works/ZhangYW21.pdf}{ZhangYW21} (0.36)& \cellcolor{black!20}\href{../works/TanT18.pdf}{TanT18} (0.37)& \href{../works/MurinR19.pdf}{MurinR19} (0.39)& \href{../works/HamPK21.pdf}{HamPK21} (0.40)\\
Dot& \cellcolor{red!40}\href{../works/Groleaz21.pdf}{Groleaz21} (186.00)& \cellcolor{red!40}\href{../works/Lunardi20.pdf}{Lunardi20} (180.00)& \cellcolor{red!40}\href{../works/ZarandiASC20.pdf}{ZarandiASC20} (176.00)& \cellcolor{red!40}\href{../works/NaderiRR23.pdf}{NaderiRR23} (171.00)& \cellcolor{red!40}\href{../works/Malapert11.pdf}{Malapert11} (162.00)\\
Cosine& \cellcolor{red!40}\href{../works/NaderiBZ22a.pdf}{NaderiBZ22a} (0.80)& \cellcolor{red!40}\href{../works/ZhangYW21.pdf}{ZhangYW21} (0.75)& \cellcolor{red!40}\href{../works/TanT18.pdf}{TanT18} (0.74)& \cellcolor{red!40}\href{../works/HamPK21.pdf}{HamPK21} (0.72)& \cellcolor{red!40}\href{../works/MurinR19.pdf}{MurinR19} (0.72)\\
\index{ZibranR11}\href{../works/ZibranR11.pdf}{ZibranR11} R\&C& \cellcolor{red!40}\href{../works/ZibranR11a.pdf}{ZibranR11a} (0.48)\\
Euclid& \cellcolor{red!40}\href{../works/ZibranR11a.pdf}{ZibranR11a} (0.11)& \cellcolor{red!40}\href{../works/ChapadosJR11.pdf}{ChapadosJR11} (0.16)& \cellcolor{red!40}\href{../works/Baptiste09.pdf}{Baptiste09} (0.17)& \cellcolor{red!40}\href{../works/CarchraeBF05.pdf}{CarchraeBF05} (0.18)& \cellcolor{red!40}\href{../works/HebrardALLCMR22.pdf}{HebrardALLCMR22} (0.18)\\
Dot& \cellcolor{red!40}\href{../works/ZarandiASC20.pdf}{ZarandiASC20} (50.00)& \cellcolor{red!40}\href{../works/Astrand21.pdf}{Astrand21} (48.00)& \cellcolor{red!40}\href{../works/Malapert11.pdf}{Malapert11} (48.00)& \cellcolor{red!40}\href{../works/Lunardi20.pdf}{Lunardi20} (47.00)& \cellcolor{red!40}\href{../works/Beck99.pdf}{Beck99} (47.00)\\
Cosine& \cellcolor{red!40}\href{../works/ZibranR11a.pdf}{ZibranR11a} (0.94)& \cellcolor{red!40}\href{../works/ZhangLS12.pdf}{ZhangLS12} (0.73)& \cellcolor{red!40}\href{../works/ChapadosJR11.pdf}{ChapadosJR11} (0.72)& \cellcolor{red!40}\href{../works/Baptiste09.pdf}{Baptiste09} (0.70)& \cellcolor{red!40}\href{../works/BarzegaranZP20.pdf}{BarzegaranZP20} (0.69)\\
\index{ZibranR11a}\href{../works/ZibranR11a.pdf}{ZibranR11a} R\&C& \cellcolor{red!40}\href{../works/ZibranR11.pdf}{ZibranR11} (0.48)& \cellcolor{blue!20}\href{../works/ZhangLS12.pdf}{ZhangLS12} (0.96)& \cellcolor{blue!20}\href{../works/QuirogaZH05.pdf}{QuirogaZH05} (0.96)& \cellcolor{blue!20}\href{../works/Geske05.pdf}{Geske05} (0.96)& \cellcolor{blue!20}\href{../works/EvenSH15.pdf}{EvenSH15} (0.97)\\
Euclid& \cellcolor{red!40}\href{../works/ZibranR11.pdf}{ZibranR11} (0.11)& \cellcolor{red!40}\href{../works/ChapadosJR11.pdf}{ChapadosJR11} (0.19)& \cellcolor{red!40}\href{../works/ZhangLS12.pdf}{ZhangLS12} (0.22)& \cellcolor{red!40}\href{../works/WallaceF00.pdf}{WallaceF00} (0.22)& \cellcolor{red!40}\href{../works/HebrardALLCMR22.pdf}{HebrardALLCMR22} (0.23)\\
Dot& \cellcolor{red!40}\href{../works/ZarandiASC20.pdf}{ZarandiASC20} (69.00)& \cellcolor{red!40}\href{../works/Astrand21.pdf}{Astrand21} (64.00)& \cellcolor{red!40}\href{../works/Beck99.pdf}{Beck99} (63.00)& \cellcolor{red!40}\href{../works/TangLWSK18.pdf}{TangLWSK18} (61.00)& \cellcolor{red!40}\href{../works/Dejemeppe16.pdf}{Dejemeppe16} (61.00)\\
Cosine& \cellcolor{red!40}\href{../works/ZibranR11.pdf}{ZibranR11} (0.94)& \cellcolor{red!40}\href{../works/ChapadosJR11.pdf}{ChapadosJR11} (0.78)& \cellcolor{red!40}\href{../works/ZouZ20.pdf}{ZouZ20} (0.74)& \cellcolor{red!40}\href{../works/ZhangLS12.pdf}{ZhangLS12} (0.71)& \cellcolor{red!40}\href{../works/ZeballosM09.pdf}{ZeballosM09} (0.71)\\
\index{ZouZ20}\href{../works/ZouZ20.pdf}{ZouZ20} R\&C& \cellcolor{red!40}\href{../works/TangLWSK18.pdf}{TangLWSK18} (0.80)& \cellcolor{green!20}\href{../works/SubulanC22.pdf}{SubulanC22} (0.96)& \cellcolor{green!20}\href{../works/YuraszeckMPV22.pdf}{YuraszeckMPV22} (0.96)& \cellcolor{blue!20}\href{../works/SchnellH17.pdf}{SchnellH17} (0.97)& \cellcolor{blue!20}\href{../works/HauderBRPA20.pdf}{HauderBRPA20} (0.97)\\
Euclid& \cellcolor{red!20}\href{../works/NishikawaSTT18.pdf}{NishikawaSTT18} (0.24)& \cellcolor{red!20}\href{../works/NishikawaSTT18a.pdf}{NishikawaSTT18a} (0.25)& \cellcolor{yellow!20}\href{../works/NishikawaSTT19.pdf}{NishikawaSTT19} (0.26)& \cellcolor{yellow!20}\href{../works/BeniniBGM05a.pdf}{BeniniBGM05a} (0.27)& \cellcolor{yellow!20}\href{../works/ZibranR11a.pdf}{ZibranR11a} (0.27)\\
Dot& \cellcolor{red!40}\href{../works/Lombardi10.pdf}{Lombardi10} (113.00)& \cellcolor{red!40}\href{../works/Groleaz21.pdf}{Groleaz21} (110.00)& \cellcolor{red!40}\href{../works/ZarandiASC20.pdf}{ZarandiASC20} (109.00)& \cellcolor{red!40}\href{../works/LaborieRSV18.pdf}{LaborieRSV18} (105.00)& \cellcolor{red!40}\href{../works/Lunardi20.pdf}{Lunardi20} (104.00)\\
Cosine& \cellcolor{red!40}\href{../works/NishikawaSTT18.pdf}{NishikawaSTT18} (0.80)& \cellcolor{red!40}\href{../works/NishikawaSTT18a.pdf}{NishikawaSTT18a} (0.79)& \cellcolor{red!40}\href{../works/NishikawaSTT19.pdf}{NishikawaSTT19} (0.77)& \cellcolor{red!40}\href{../works/BeniniBGM05a.pdf}{BeniniBGM05a} (0.75)& \cellcolor{red!40}\href{../works/ZibranR11a.pdf}{ZibranR11a} (0.74)\\
\index{abs-0907-0939}\href{../works/abs-0907-0939.pdf}{abs-0907-0939} R\&C\\
Euclid& \cellcolor{green!20}\href{../works/PoderB08.pdf}{PoderB08} (0.30)& \cellcolor{green!20}\href{../works/ClercqPBJ11.pdf}{ClercqPBJ11} (0.30)& \cellcolor{green!20}\href{../works/BeldiceanuP07.pdf}{BeldiceanuP07} (0.31)& \cellcolor{blue!20}\href{../works/WolfS05.pdf}{WolfS05} (0.33)& \cellcolor{black!20}\href{../works/Vilim09a.pdf}{Vilim09a} (0.35)\\
Dot& \cellcolor{red!40}\href{../works/Fahimi16.pdf}{Fahimi16} (103.00)& \cellcolor{red!40}\href{../works/Malapert11.pdf}{Malapert11} (102.00)& \cellcolor{red!40}\href{../works/Godet21a.pdf}{Godet21a} (100.00)& \cellcolor{red!40}\href{../works/Dejemeppe16.pdf}{Dejemeppe16} (99.00)& \cellcolor{red!40}\href{../works/Lombardi10.pdf}{Lombardi10} (99.00)\\
Cosine& \cellcolor{red!40}\href{../works/ClercqPBJ11.pdf}{ClercqPBJ11} (0.73)& \cellcolor{red!40}\href{../works/PoderB08.pdf}{PoderB08} (0.72)& \cellcolor{red!40}\href{../works/BeldiceanuP07.pdf}{BeldiceanuP07} (0.68)& \cellcolor{red!40}\href{../works/WolfS05.pdf}{WolfS05} (0.65)& \cellcolor{red!40}\href{../works/NattafAL15.pdf}{NattafAL15} (0.63)\\
\index{abs-1009-0347}\href{../works/abs-1009-0347.pdf}{abs-1009-0347} R\&C\\
Euclid& \cellcolor{red!40}\href{../works/SchuttFSW13.pdf}{SchuttFSW13} (0.19)& \cellcolor{red!40}\href{../works/SchuttFSW09.pdf}{SchuttFSW09} (0.23)& \cellcolor{red!20}\href{../works/SchuttCSW12.pdf}{SchuttCSW12} (0.25)& \cellcolor{yellow!20}\href{../works/BofillCSV17.pdf}{BofillCSV17} (0.27)& \cellcolor{yellow!20}\href{../works/BofillCSV17a.pdf}{BofillCSV17a} (0.27)\\
Dot& \cellcolor{red!40}\href{../works/Schutt11.pdf}{Schutt11} (153.00)& \cellcolor{red!40}\href{../works/Caballero19.pdf}{Caballero19} (143.00)& \cellcolor{red!40}\href{../works/Godet21a.pdf}{Godet21a} (142.00)& \cellcolor{red!40}\href{../works/Lombardi10.pdf}{Lombardi10} (138.00)& \cellcolor{red!40}\href{../works/SchuttFSW11.pdf}{SchuttFSW11} (137.00)\\
Cosine& \cellcolor{red!40}\href{../works/SchuttFSW13.pdf}{SchuttFSW13} (0.91)& \cellcolor{red!40}\href{../works/SchuttFSW09.pdf}{SchuttFSW09} (0.85)& \cellcolor{red!40}\href{../works/SchuttCSW12.pdf}{SchuttCSW12} (0.82)& \cellcolor{red!40}\href{../works/SchuttFSW11.pdf}{SchuttFSW11} (0.82)& \cellcolor{red!40}\href{../works/SchuttFS13a.pdf}{SchuttFS13a} (0.81)\\
\index{abs-1901-07914}\href{../works/abs-1901-07914.pdf}{abs-1901-07914} R\&C\\
Euclid& \cellcolor{red!40}\href{../works/BehrensLM19.pdf}{BehrensLM19} (0.09)& \cellcolor{red!20}\href{../works/ValleMGT03.pdf}{ValleMGT03} (0.24)& \cellcolor{red!20}\href{../works/JungblutK22.pdf}{JungblutK22} (0.25)& \cellcolor{red!20}\href{../works/WessenCS20.pdf}{WessenCS20} (0.25)& \cellcolor{yellow!20}\href{../works/BoothNB16.pdf}{BoothNB16} (0.27)\\
Dot& \cellcolor{red!40}\href{../works/ZarandiASC20.pdf}{ZarandiASC20} (96.00)& \cellcolor{red!40}\href{../works/Godet21a.pdf}{Godet21a} (94.00)& \cellcolor{red!40}\href{../works/Dejemeppe16.pdf}{Dejemeppe16} (94.00)& \cellcolor{red!40}\href{../works/Beck99.pdf}{Beck99} (94.00)& \cellcolor{red!40}\href{../works/BartakSR10.pdf}{BartakSR10} (92.00)\\
Cosine& \cellcolor{red!40}\href{../works/BehrensLM19.pdf}{BehrensLM19} (0.97)& \cellcolor{red!40}\href{../works/ValleMGT03.pdf}{ValleMGT03} (0.77)& \cellcolor{red!40}\href{../works/JungblutK22.pdf}{JungblutK22} (0.74)& \cellcolor{red!40}\href{../works/WessenCS20.pdf}{WessenCS20} (0.74)& \cellcolor{red!40}\href{../works/NovasH14.pdf}{NovasH14} (0.73)\\
\index{abs-1902-01193}\href{../works/abs-1902-01193.pdf}{abs-1902-01193} R\&C\\
Euclid& \cellcolor{red!40}\href{../works/ZibranR11.pdf}{ZibranR11} (0.23)& \cellcolor{red!40}\href{../works/ZibranR11a.pdf}{ZibranR11a} (0.23)& \cellcolor{red!40}\href{../works/BourdaisGP03.pdf}{BourdaisGP03} (0.23)& \cellcolor{red!20}\href{../works/PesantGPR99.pdf}{PesantGPR99} (0.25)& \cellcolor{red!20}\href{../works/ZhangLS12.pdf}{ZhangLS12} (0.25)\\
Dot& \cellcolor{red!40}\href{../works/ZarandiASC20.pdf}{ZarandiASC20} (87.00)& \cellcolor{red!40}\href{../works/Dejemeppe16.pdf}{Dejemeppe16} (70.00)& \cellcolor{red!40}\href{../works/Simonis07.pdf}{Simonis07} (69.00)& \cellcolor{red!40}\href{../works/Wallace96.pdf}{Wallace96} (69.00)& \cellcolor{red!40}\href{../works/Beck99.pdf}{Beck99} (68.00)\\
Cosine& \cellcolor{red!40}\href{../works/BourdaisGP03.pdf}{BourdaisGP03} (0.73)& \cellcolor{red!40}\href{../works/ZibranR11a.pdf}{ZibranR11a} (0.71)& \cellcolor{red!40}\href{../works/ZibranR11.pdf}{ZibranR11} (0.68)& \cellcolor{red!40}\href{../works/PesantGPR99.pdf}{PesantGPR99} (0.67)& \cellcolor{red!40}\href{../works/WeilHFP95.pdf}{WeilHFP95} (0.67)\\
\index{abs-1902-09244}\href{../works/abs-1902-09244.pdf}{abs-1902-09244} R\&C\\
Euclid& \cellcolor{red!40}\href{../works/HauderBRPA20.pdf}{HauderBRPA20} (0.11)& \cellcolor{black!20}\href{../works/BeckPS03.pdf}{BeckPS03} (0.36)& \href{../works/MonetteDH09.pdf}{MonetteDH09} (0.38)& \href{../works/HeipckeCCS00.pdf}{HeipckeCCS00} (0.38)& \href{../works/BeckR03.pdf}{BeckR03} (0.39)\\
Dot& \cellcolor{red!40}\href{../works/ZarandiASC20.pdf}{ZarandiASC20} (213.00)& \cellcolor{red!40}\href{../works/HauderBRPA20.pdf}{HauderBRPA20} (200.00)& \cellcolor{red!40}\href{../works/Groleaz21.pdf}{Groleaz21} (192.00)& \cellcolor{red!40}\href{../works/Dejemeppe16.pdf}{Dejemeppe16} (191.00)& \cellcolor{red!40}\href{../works/Lunardi20.pdf}{Lunardi20} (181.00)\\
Cosine& \cellcolor{red!40}\href{../works/HauderBRPA20.pdf}{HauderBRPA20} (0.98)& \cellcolor{red!40}\href{../works/BeckPS03.pdf}{BeckPS03} (0.76)& \cellcolor{red!40}\href{../works/YuraszeckMCCR23.pdf}{YuraszeckMCCR23} (0.73)& \cellcolor{red!40}\href{../works/MonetteDH09.pdf}{MonetteDH09} (0.73)& \cellcolor{red!40}\href{../works/HubnerGSV21.pdf}{HubnerGSV21} (0.72)\\
\index{abs-1911-04766}\href{../works/abs-1911-04766.pdf}{abs-1911-04766} R\&C\\
Euclid& \cellcolor{red!40}\href{../works/GeibingerMM19.pdf}{GeibingerMM19} (0.23)& \cellcolor{blue!20}\href{../works/GeibingerMM21.pdf}{GeibingerMM21} (0.32)& \href{../works/PovedaAA23.pdf}{PovedaAA23} (0.38)& \href{../works/CampeauG22.pdf}{CampeauG22} (0.40)& \href{../works/YoungFS17.pdf}{YoungFS17} (0.40)\\
Dot& \cellcolor{red!40}\href{../works/Groleaz21.pdf}{Groleaz21} (172.00)& \cellcolor{red!40}\href{../works/Dejemeppe16.pdf}{Dejemeppe16} (171.00)& \cellcolor{red!40}\href{../works/GeibingerMM19.pdf}{GeibingerMM19} (168.00)& \cellcolor{red!40}\href{../works/Lombardi10.pdf}{Lombardi10} (164.00)& \cellcolor{red!40}\href{../works/LaborieRSV18.pdf}{LaborieRSV18} (161.00)\\
Cosine& \cellcolor{red!40}\href{../works/GeibingerMM19.pdf}{GeibingerMM19} (0.91)& \cellcolor{red!40}\href{../works/GeibingerMM21.pdf}{GeibingerMM21} (0.81)& \cellcolor{red!40}\href{../works/PovedaAA23.pdf}{PovedaAA23} (0.75)& \cellcolor{red!40}\href{../works/YoungFS17.pdf}{YoungFS17} (0.69)& \cellcolor{red!40}\href{../works/CampeauG22.pdf}{CampeauG22} (0.69)\\
\index{abs-2102-08778}\href{../works/abs-2102-08778.pdf}{abs-2102-08778} R\&C\\
Euclid& \cellcolor{red!40}\href{../works/ColT19.pdf}{ColT19} (0.22)& \cellcolor{red!40}\href{../works/ColT2019a.pdf}{ColT2019a} (0.23)& \cellcolor{yellow!20}\href{../works/Teppan22.pdf}{Teppan22} (0.27)& \cellcolor{green!20}\href{../works/WatsonB08.pdf}{WatsonB08} (0.30)& \cellcolor{green!20}\href{../works/Beck06.pdf}{Beck06} (0.31)\\
Dot& \cellcolor{red!40}\href{../works/ColT22.pdf}{ColT22} (122.00)& \cellcolor{red!40}\href{../works/Groleaz21.pdf}{Groleaz21} (108.00)& \cellcolor{red!40}\href{../works/Godet21a.pdf}{Godet21a} (103.00)& \cellcolor{red!40}\href{../works/Dejemeppe16.pdf}{Dejemeppe16} (101.00)& \cellcolor{red!40}\href{../works/ColT19.pdf}{ColT19} (100.00)\\
Cosine& \cellcolor{red!40}\href{../works/ColT19.pdf}{ColT19} (0.86)& \cellcolor{red!40}\href{../works/ColT2019a.pdf}{ColT2019a} (0.83)& \cellcolor{red!40}\href{../works/Teppan22.pdf}{Teppan22} (0.78)& \cellcolor{red!40}\href{../works/abs-2306-05747.pdf}{abs-2306-05747} (0.73)& \cellcolor{red!40}\href{../works/TasselGS23.pdf}{TasselGS23} (0.73)\\
\index{abs-2211-14492}\href{../works/abs-2211-14492.pdf}{abs-2211-14492} R\&C\\
Euclid& \cellcolor{black!20}\href{../works/IklassovMR023.pdf}{IklassovMR023} (0.35)& \cellcolor{black!20}\href{../works/abs-2402-00459.pdf}{abs-2402-00459} (0.35)& \href{../works/BeckFW11.pdf}{BeckFW11} (0.38)& \href{../works/KovacsTKSG21.pdf}{KovacsTKSG21} (0.38)& \href{../works/CarchraeB09.pdf}{CarchraeB09} (0.38)\\
Dot& \cellcolor{red!40}\href{../works/Groleaz21.pdf}{Groleaz21} (188.00)& \cellcolor{red!40}\href{../works/ZarandiASC20.pdf}{ZarandiASC20} (183.00)& \cellcolor{red!40}\href{../works/Dejemeppe16.pdf}{Dejemeppe16} (167.00)& \cellcolor{red!40}\href{../works/Lunardi20.pdf}{Lunardi20} (166.00)& \cellcolor{red!40}\href{../works/IsikYA23.pdf}{IsikYA23} (160.00)\\
Cosine& \cellcolor{red!40}\href{../works/abs-2402-00459.pdf}{abs-2402-00459} (0.78)& \cellcolor{red!40}\href{../works/IklassovMR023.pdf}{IklassovMR023} (0.75)& \cellcolor{red!40}\href{../works/KovacsTKSG21.pdf}{KovacsTKSG21} (0.73)& \cellcolor{red!40}\href{../works/MullerMKP22.pdf}{MullerMKP22} (0.72)& \cellcolor{red!40}\href{../works/BeckFW11.pdf}{BeckFW11} (0.72)\\
\index{abs-2305-19888}\href{../works/abs-2305-19888.pdf}{abs-2305-19888} R\&C\\
Euclid& \cellcolor{red!40}\href{../works/HeinzNVH22.pdf}{HeinzNVH22} (0.13)& \cellcolor{green!20}\href{../works/ArbaouiY18.pdf}{ArbaouiY18} (0.30)& \cellcolor{blue!20}\href{../works/Ham18a.pdf}{Ham18a} (0.32)& \cellcolor{blue!20}\href{../works/LahimerLH11.pdf}{LahimerLH11} (0.33)& \cellcolor{blue!20}\href{../works/GedikKEK18.pdf}{GedikKEK18} (0.34)\\
Dot& \cellcolor{red!40}\href{../works/Lunardi20.pdf}{Lunardi20} (162.00)& \cellcolor{red!40}\href{../works/Groleaz21.pdf}{Groleaz21} (162.00)& \cellcolor{red!40}\href{../works/IsikYA23.pdf}{IsikYA23} (154.00)& \cellcolor{red!40}\href{../works/ZarandiASC20.pdf}{ZarandiASC20} (154.00)& \cellcolor{red!40}\href{../works/Astrand21.pdf}{Astrand21} (147.00)\\
Cosine& \cellcolor{red!40}\href{../works/HeinzNVH22.pdf}{HeinzNVH22} (0.96)& \cellcolor{red!40}\href{../works/ArbaouiY18.pdf}{ArbaouiY18} (0.78)& \cellcolor{red!40}\href{../works/Ham18a.pdf}{Ham18a} (0.77)& \cellcolor{red!40}\href{../works/GedikKEK18.pdf}{GedikKEK18} (0.77)& \cellcolor{red!40}\href{../works/LahimerLH11.pdf}{LahimerLH11} (0.72)\\
\index{abs-2306-05747}\href{../works/abs-2306-05747.pdf}{abs-2306-05747} R\&C\\
Euclid& \cellcolor{red!40}\href{../works/TasselGS23.pdf}{TasselGS23} (0.00)& \cellcolor{green!20}\href{../works/BeckFW11.pdf}{BeckFW11} (0.30)& \cellcolor{green!20}\href{../works/CarchraeB09.pdf}{CarchraeB09} (0.30)& \cellcolor{green!20}\href{../works/IklassovMR023.pdf}{IklassovMR023} (0.30)& \cellcolor{green!20}\href{../works/WatsonB08.pdf}{WatsonB08} (0.30)\\
Dot& \cellcolor{red!40}\href{../works/TasselGS23.pdf}{TasselGS23} (145.00)& \cellcolor{red!40}\href{../works/ZarandiASC20.pdf}{ZarandiASC20} (143.00)& \cellcolor{red!40}\href{../works/Groleaz21.pdf}{Groleaz21} (142.00)& \cellcolor{red!40}\href{../works/ColT22.pdf}{ColT22} (134.00)& \cellcolor{red!40}\href{../works/Lunardi20.pdf}{Lunardi20} (132.00)\\
Cosine& \cellcolor{red!40}\href{../works/TasselGS23.pdf}{TasselGS23} (1.00)& \cellcolor{red!40}\href{../works/BeckFW11.pdf}{BeckFW11} (0.78)& \cellcolor{red!40}\href{../works/CarchraeB09.pdf}{CarchraeB09} (0.77)& \cellcolor{red!40}\href{../works/IklassovMR023.pdf}{IklassovMR023} (0.77)& \cellcolor{red!40}\href{../works/WatsonB08.pdf}{WatsonB08} (0.76)\\
\index{abs-2312-13682}\href{../works/abs-2312-13682.pdf}{abs-2312-13682} R\&C\\
Euclid& \cellcolor{red!40}\href{../works/PerezGSL23.pdf}{PerezGSL23} (0.05)& \cellcolor{red!40}\href{../works/BockmayrP06.pdf}{BockmayrP06} (0.23)& \cellcolor{yellow!20}\href{../works/ZibranR11.pdf}{ZibranR11} (0.27)& \cellcolor{yellow!20}\href{../works/ZibranR11a.pdf}{ZibranR11a} (0.27)& \cellcolor{yellow!20}\href{../works/Limtanyakul07.pdf}{Limtanyakul07} (0.27)\\
Dot& \cellcolor{red!40}\href{../works/Dejemeppe16.pdf}{Dejemeppe16} (90.00)& \cellcolor{red!40}\href{../works/Astrand21.pdf}{Astrand21} (88.00)& \cellcolor{red!40}\href{../works/LaborieRSV18.pdf}{LaborieRSV18} (86.00)& \cellcolor{red!40}\href{../works/Lombardi10.pdf}{Lombardi10} (86.00)& \cellcolor{red!40}\href{../works/ZarandiASC20.pdf}{ZarandiASC20} (85.00)\\
Cosine& \cellcolor{red!40}\href{../works/PerezGSL23.pdf}{PerezGSL23} (0.99)& \cellcolor{red!40}\href{../works/BockmayrP06.pdf}{BockmayrP06} (0.75)& \cellcolor{red!40}\href{../works/AalianPG23.pdf}{AalianPG23} (0.74)& \cellcolor{red!40}\href{../works/CarchraeB09.pdf}{CarchraeB09} (0.71)& \cellcolor{red!40}\href{../works/AstrandJZ18.pdf}{AstrandJZ18} (0.70)\\
\index{abs-2402-00459}\href{../works/abs-2402-00459.pdf}{abs-2402-00459} R\&C\\
Euclid& \cellcolor{black!20}\href{../works/abs-2211-14492.pdf}{abs-2211-14492} (0.35)& \cellcolor{black!20}\href{../works/ThiruvadyWGS14.pdf}{ThiruvadyWGS14} (0.36)& \cellcolor{black!20}\href{../works/HeipckeCCS00.pdf}{HeipckeCCS00} (0.37)& \cellcolor{black!20}\href{../works/CarchraeB09.pdf}{CarchraeB09} (0.37)& \href{../works/KovacsTKSG21.pdf}{KovacsTKSG21} (0.38)\\
Dot& \cellcolor{red!40}\href{../works/ZarandiASC20.pdf}{ZarandiASC20} (179.00)& \cellcolor{red!40}\href{../works/Groleaz21.pdf}{Groleaz21} (179.00)& \cellcolor{red!40}\href{../works/Dejemeppe16.pdf}{Dejemeppe16} (162.00)& \cellcolor{red!40}\href{../works/IsikYA23.pdf}{IsikYA23} (155.00)& \cellcolor{red!40}\href{../works/Lunardi20.pdf}{Lunardi20} (155.00)\\
Cosine& \cellcolor{red!40}\href{../works/abs-2211-14492.pdf}{abs-2211-14492} (0.78)& \cellcolor{red!40}\href{../works/ThiruvadyWGS14.pdf}{ThiruvadyWGS14} (0.72)& \cellcolor{red!40}\href{../works/KovacsTKSG21.pdf}{KovacsTKSG21} (0.71)& \cellcolor{red!40}\href{../works/CarchraeB09.pdf}{CarchraeB09} (0.70)& \cellcolor{red!40}\href{../works/HeipckeCCS00.pdf}{HeipckeCCS00} (0.70)\\
\end{longtable}
}



\clearpage
\subsection{Most Similar Works Based on Euclidean Distance}

The following tables show pairs of work for which the similarity by concept value is smallest. This allows to check if the measure is really finding similar papers.

One of the challenge of this similarity measure is that works with few extracted features are considered to be quite similar. This should not be the case.

{\scriptsize
\begin{longtable}{>{\raggedright\arraybackslash}p{2.5cm}>{\raggedright\arraybackslash}p{4.5cm}>{\raggedright\arraybackslash}p{6.0cm}p{1.0cm}rr>{\raggedright\arraybackslash}p{2.0cm}r>{\raggedright\arraybackslash}p{1cm}p{1cm}p{1cm}p{1cm}}
\rowcolor{white}\caption{Works Close by Euclidean Distance (Total 40)}\\ \toprule
\rowcolor{white}\shortstack{Key\\Source} & Authors & Title (Colored by Open Access)& \shortstack{Details\\LC} & Cite & Year & \shortstack{Conference\\/Journal\\/School} & Pages & Relevance &\shortstack{Cites\\OC XR\\SC} & \shortstack{Refs\\OC\\XR} & \shortstack{Links\\Cites\\Refs}\\ \midrule\endhead
\bottomrule
\endfoot
TasselGS23 \href{https://doi.org/10.1609/icaps.v33i1.27243}{TasselGS23} & \hyperref[auth:a58]{P. Tassel}, \hyperref[auth:a61]{M. Gebser}, \hyperref[auth:a423]{K. Schekotihin} & \cellcolor{gold!20}An End-to-End Reinforcement Learning Approach for Job-Shop Scheduling Problems Based on Constraint Programming & \hyperref[detail:TasselGS23]{Details} \href{../works/TasselGS23.pdf}{Yes} & \cite{TasselGS23} & 2023 & ICAPS 2023 & 9 & \noindent{}\textbf{2.00} \textbf{2.00} \textbf{12.18} & 0 1 2 & 0 0 & 0 0 0\\
abs-2306-05747 \href{https://doi.org/10.48550/arXiv.2306.05747}{abs-2306-05747} & \hyperref[auth:a58]{P. Tassel}, \hyperref[auth:a61]{M. Gebser}, \hyperref[auth:a423]{K. Schekotihin} & An End-to-End Reinforcement Learning Approach for Job-Shop Scheduling Problems Based on Constraint Programming & \hyperref[detail:abs-2306-05747]{Details} \href{../works/abs-2306-05747.pdf}{Yes} & \cite{abs-2306-05747} & 2023 & CoRR & 9 & \noindent{}\textbf{2.00} \textbf{2.00} \textbf{12.09} & 0 0 0 & 0 0 & 0 0 0\\
PerezGSL23 \href{https://doi.org/10.1109/ICTAI59109.2023.00108}{PerezGSL23} & \hyperref[auth:a425]{G. Perez}, \hyperref[auth:a426]{G. Glorian}, \hyperref[auth:a427]{W. Suijlen}, \hyperref[auth:a428]{A. Lallouet} & A Constraint Programming Model for Scheduling the Unloading of Trains in Ports & \hyperref[detail:PerezGSL23]{Details} \href{../works/PerezGSL23.pdf}{Yes} & \cite{PerezGSL23} & 2023 & ICTAI 2023 & 7 & \noindent{}\textbf{1.00} \textbf{1.00} 0.94 & 0 0 0 & 0 19 & 0 0 0\\
abs-2312-13682 \href{https://doi.org/10.48550/arXiv.2312.13682}{abs-2312-13682} & \hyperref[auth:a425]{G. Perez}, \hyperref[auth:a426]{G. Glorian}, \hyperref[auth:a427]{W. Suijlen}, \hyperref[auth:a428]{A. Lallouet} & A Constraint Programming Model for Scheduling the Unloading of Trains in Ports: Extended & \hyperref[detail:abs-2312-13682]{Details} \href{../works/abs-2312-13682.pdf}{Yes} & \cite{abs-2312-13682} & 2023 & CoRR & 20 & \noindent{}\textbf{1.00} \textbf{1.00} 0.94 & 0 0 0 & 0 0 & 0 0 0\\
NuijtenA94 \href{}{NuijtenA94} & \hyperref[auth:a656]{W. Nuijten}, \hyperref[auth:a777]{E. H. L. Aarts} & Constraint Satisfaction for Multiple Capacitated Job Shop Scheduling & \hyperref[detail:NuijtenA94]{Details} \href{../works/NuijtenA94.pdf}{Yes} & \cite{NuijtenA94} & 1994 & ECAI 1994 & 5 & \noindent{}\textbf{2.00} \textbf{2.00} \textbf{2.80} & 0 0 0 & 0 0 & 0 0 0\\
NuijtenA96 \href{http://dx.doi.org/10.1016/0377-2217(95)00354-1}{NuijtenA96} & \hyperref[auth:a656]{W. Nuijten}, \hyperref[auth:a777]{E. H. L. Aarts} & A computational study of constraint satisfaction for multiple capacitated job shop scheduling & \hyperref[detail:NuijtenA96]{Details} \href{../works/NuijtenA96.pdf}{Yes} & \cite{NuijtenA96} & 1996 & European Journal of Operational Research & 16 & \noindent{}\textbf{2.00} \textbf{2.00} \textbf{2.88} & 65 65 90 & 6 21 & 27 23 4\\
NishikawaSTT18 \href{https://doi.org/10.1109/CANDAR.2018.00025}{NishikawaSTT18} & \hyperref[auth:a531]{H. Nishikawa}, \hyperref[auth:a532]{K. Shimada}, \hyperref[auth:a533]{I. Taniguchi}, \hyperref[auth:a534]{H. Tomiyama} & Scheduling of Malleable Fork-Join Tasks with Constraint Programming & \hyperref[detail:NishikawaSTT18]{Details} \href{../works/NishikawaSTT18.pdf}{Yes} & \cite{NishikawaSTT18} & 2018 & CANDAR 2018 & 6 & \noindent{}\textbf{2.00} \textbf{2.00} \textbf{31.05} & 2 2 2 & 14 21 & 4 0 4\\
NishikawaSTT18a \href{https://doi.org/10.1109/TENCON.2018.8650168}{NishikawaSTT18a} & \hyperref[auth:a531]{H. Nishikawa}, \hyperref[auth:a532]{K. Shimada}, \hyperref[auth:a533]{I. Taniguchi}, \hyperref[auth:a534]{H. Tomiyama} & Scheduling of Malleable Tasks Based on Constraint Programming & \hyperref[detail:NishikawaSTT18a]{Details} \href{../works/NishikawaSTT18a.pdf}{Yes} & \cite{NishikawaSTT18a} & 2018 & TENCON 2018 & 6 & \noindent{}\textbf{2.00} \textbf{2.00} \textbf{15.33} & 1 1 1 & 9 16 & 3 0 3\\
NaderiBZ22 \href{http://dx.doi.org/10.2139/ssrn.4140716}{NaderiBZ22} & \hyperref[auth:a726]{B. Naderi}, \hyperref[auth:a836]{M. A. Begen}, \hyperref[auth:a837]{G. Zhang} & Integrated Order Acceptance and Resource Decisions Under Uncertainty: Robust and Stochastic Approaches & \hyperref[detail:NaderiBZ22]{Details} \href{../works/NaderiBZ22.pdf}{Yes} & \cite{NaderiBZ22} & 2022 & SSRN Electronic Journal & 29 & \noindent{}\textcolor{black!50}{0.00} \textcolor{black!50}{0.00} \textbf{9.27} & 0 0 0 & 44 51 & 11 0 11\\
NaderiBZ23 \href{http://dx.doi.org/10.2139/ssrn.4494381}{NaderiBZ23} & \hyperref[auth:a726]{B. Naderi}, \hyperref[auth:a836]{M. A. Begen}, \hyperref[auth:a837]{G. Zhang} & Integrated Order Acceptance and Resource Decisions Under Uncertainty: Robust and Stochastic Approaches & \hyperref[detail:NaderiBZ23]{Details} \href{../works/NaderiBZ23.pdf}{Yes} & \cite{NaderiBZ23} & 2023 & SSRN Electronic Journal & 32 & \noindent{}\textcolor{black!50}{0.00} \textcolor{black!50}{0.00} \textbf{10.49} & 0 0 0 & 46 56 & 12 0 12\\
EvenSH15 \href{https://doi.org/10.1007/978-3-319-23219-5_40}{EvenSH15} & \hyperref[auth:a214]{C. Even}, \hyperref[auth:a124]{A. Schutt}, \hyperref[auth:a148]{P. V. Hentenryck} & \cellcolor{green!10}A Constraint Programming Approach for Non-preemptive Evacuation Scheduling & \hyperref[detail:EvenSH15]{Details} \href{../works/EvenSH15.pdf}{Yes} & \cite{EvenSH15} & 2015 & CP 2015 & 18 & \noindent{}\textbf{1.00} \textbf{1.00} \textbf{2.40} & 3 2 6 & 12 14 & 2 2 0\\
EvenSH15a \href{http://arxiv.org/abs/1505.02487}{EvenSH15a} & \hyperref[auth:a214]{C. Even}, \hyperref[auth:a124]{A. Schutt}, \hyperref[auth:a148]{P. V. Hentenryck} & A Constraint Programming Approach for Non-Preemptive Evacuation Scheduling & \hyperref[detail:EvenSH15a]{Details} \href{../works/EvenSH15a.pdf}{Yes} & \cite{EvenSH15a} & 2015 & CoRR & 16 & \noindent{}\textbf{1.00} \textbf{1.00} 0.42 & 0 0 0 & 0 0 & 0 0 0\\
Baptiste09 \href{https://doi.org/10.1007/978-3-642-04244-7_1}{Baptiste09} & \hyperref[auth:a162]{P. Baptiste} & Constraint-Based Schedulers, Do They Really Work? & \hyperref[detail:Baptiste09]{Details} \href{../works/Baptiste09.pdf}{Yes} & \cite{Baptiste09} & 2009 & CP 2009 & 1 & \noindent{}\textcolor{black!50}{0.00} \textcolor{black!50}{0.00} \textcolor{black!50}{0.03} & 0 0 0 & 0 0 & 0 0 0\\
CarchraeBF05 \href{https://doi.org/10.1007/11564751_80}{CarchraeBF05} & \hyperref[auth:a272]{T. Carchrae}, \hyperref[auth:a89]{J. C. Beck}, \hyperref[auth:a273]{E. C. Freuder} & \cellcolor{green!10}Methods to Learn Abstract Scheduling Models & \hyperref[detail:CarchraeBF05]{Details} \href{../works/CarchraeBF05.pdf}{Yes} & \cite{CarchraeBF05} & 2005 & CP 2005 & 1 & \noindent{}\textcolor{black!50}{0.00} \textcolor{black!50}{0.00} \textcolor{black!50}{0.01} & 0 0 0 & 0 0 & 0 0 0\\
BehrensLM19 \href{https://doi.org/10.1109/ICRA.2019.8794022}{BehrensLM19} & \hyperref[auth:a540]{J. K. Behrens}, \hyperref[auth:a541]{R. Lange}, \hyperref[auth:a542]{M. Mansouri} & \cellcolor{green!10}A Constraint Programming Approach to Simultaneous Task Allocation and Motion Scheduling for Industrial Dual-Arm Manipulation Tasks & \hyperref[detail:BehrensLM19]{Details} \href{../works/BehrensLM19.pdf}{Yes} & \cite{BehrensLM19} & 2019 & ICRA 2019 & 7 & \noindent{}\textbf{2.00} \textbf{2.00} \textbf{2.42} & 12 17 27 & 18 27 & 4 3 1\\
abs-1901-07914 \href{http://arxiv.org/abs/1901.07914}{abs-1901-07914} & \hyperref[auth:a540]{J. K. Behrens}, \hyperref[auth:a541]{R. Lange}, \hyperref[auth:a542]{M. Mansouri} & A Constraint Programming Approach to Simultaneous Task Allocation and Motion Scheduling for Industrial Dual-Arm Manipulation Tasks & \hyperref[detail:abs-1901-07914]{Details} \href{../works/abs-1901-07914.pdf}{Yes} & \cite{abs-1901-07914} & 2019 & CoRR & 8 & \noindent{}\textbf{2.00} \textbf{2.00} \textbf{4.13} & 0 0 0 & 0 0 & 0 0 0\\
AbrilSB05 \href{https://doi.org/10.1007/11564751_75}{AbrilSB05} & \hyperref[auth:a270]{M. Abril}, \hyperref[auth:a153]{M. A. Salido}, \hyperref[auth:a271]{F. Barber} & \cellcolor{green!10}Distributed Constraints for Large-Scale Scheduling Problems & \hyperref[detail:AbrilSB05]{Details} \href{../works/AbrilSB05.pdf}{Yes} & \cite{AbrilSB05} & 2005 & CP 2005 & 1 & \noindent{}\textcolor{black!50}{0.00} \textcolor{black!50}{0.00} \textcolor{black!50}{0.01} & 0 0 0 & 0 2 & 0 0 0\\
Baptiste09 \href{https://doi.org/10.1007/978-3-642-04244-7_1}{Baptiste09} & \hyperref[auth:a162]{P. Baptiste} & Constraint-Based Schedulers, Do They Really Work? & \hyperref[detail:Baptiste09]{Details} \href{../works/Baptiste09.pdf}{Yes} & \cite{Baptiste09} & 2009 & CP 2009 & 1 & \noindent{}\textcolor{black!50}{0.00} \textcolor{black!50}{0.00} \textcolor{black!50}{0.03} & 0 0 0 & 0 0 & 0 0 0\\
NattafAL15 \href{https://doi.org/10.1007/s10601-015-9192-z}{NattafAL15} & \hyperref[auth:a81]{M. Nattaf}, \hyperref[auth:a6]{C. Artigues}, \hyperref[auth:a3]{P. Lopez} & \cellcolor{green!10}A hybrid exact method for a scheduling problem with a continuous resource and energy constraints & \hyperref[detail:NattafAL15]{Details} \href{../works/NattafAL15.pdf}{Yes} & \cite{NattafAL15} & 2015 & Constraints An Int. J. & 21 & \noindent{}\textcolor{black!50}{0.00} \textcolor{black!50}{0.00} 0.96 & 14 15 15 & 13 18 & 7 3 4\\
NattafALR16 \href{https://doi.org/10.1007/s00291-015-0423-x}{NattafALR16} & \hyperref[auth:a81]{M. Nattaf}, \hyperref[auth:a6]{C. Artigues}, \hyperref[auth:a3]{P. Lopez}, \hyperref[auth:a979]{D. Rivreau} & \cellcolor{green!10}Energetic reasoning and mixed-integer linear programming for scheduling with a continuous resource and linear efficiency functions & \hyperref[detail:NattafALR16]{Details} \href{../works/NattafALR16.pdf}{Yes} & \cite{NattafALR16} & 2016 & {OR} Spectrum & 34 & \noindent{}\textcolor{black!50}{0.00} \textcolor{black!50}{0.00} \textbf{1.68} & 10 10 10 & 15 19 & 6 1 5\\
Alaka21 \href{http://dx.doi.org/10.1007/s00500-021-05602-x}{Alaka21} & \hyperref[auth:a764]{H. M. Alakaş} & General resource-constrained assembly line balancing problem: conjunction normal form based constraint programming models & \hyperref[detail:Alaka21]{Details} \href{../works/Alaka21.pdf}{Yes} & \cite{Alaka21} & 2021 & Soft Computing & 11 & \noindent{}0.50 0.50 \textbf{19.49} & 7 9 9 & 20 27 & 11 2 9\\
AlakaPY19 \href{http://dx.doi.org/10.1007/s00500-019-04294-8}{AlakaPY19} & \hyperref[auth:a764]{H. M. Alakaş}, \hyperref[auth:a1384]{M. Pınarbaşı}, \hyperref[auth:a1425]{M. Y\"{u}z\"{u}kırmızı} & Constraint programming model for resource-constrained assembly line balancing problem & \hyperref[detail:AlakaPY19]{Details} \href{../works/AlakaPY19.pdf}{Yes} & \cite{AlakaPY19} & 2019 & Soft Computing & 9 & \noindent{}0.50 0.50 \textbf{14.19} & 15 17 0 & 14 23 & 11 6 5\\
Hooker05a \href{https://doi.org/10.1007/11564751_25}{Hooker05a} & \hyperref[auth:a160]{J. N. Hooker} & \cellcolor{green!10}Planning and Scheduling to Minimize Tardiness & \hyperref[detail:Hooker05a]{Details} \href{../works/Hooker05a.pdf}{Yes} & \cite{Hooker05a} & 2005 & CP 2005 & 14 & \noindent{}\textcolor{black!50}{0.00} \textcolor{black!50}{0.00} \textbf{3.03} & 30 31 35 & 10 12 & 29 20 9\\
Hooker06 \href{https://doi.org/10.1007/s10601-006-8060-2}{Hooker06} & \hyperref[auth:a160]{J. N. Hooker} & \cellcolor{green!10}An Integrated Method for Planning and Scheduling to Minimize Tardiness & \hyperref[detail:Hooker06]{Details} \href{../works/Hooker06.pdf}{Yes} & \cite{Hooker06} & 2006 & Constraints An Int. J. & 19 & \noindent{}\textcolor{black!50}{0.00} \textcolor{black!50}{0.00} \textbf{8.82} & 19 20 27 & 13 20 & 24 16 8\\
AngelsmarkJ00 \href{https://doi.org/10.1007/3-540-45349-0_35}{AngelsmarkJ00} & \hyperref[auth:a295]{O. Angelsmark}, \hyperref[auth:a296]{P. Jonsson} & Some Observations on Durations, Scheduling and Allen's Algebra & \hyperref[detail:AngelsmarkJ00]{Details} \href{../works/AngelsmarkJ00.pdf}{Yes} & \cite{AngelsmarkJ00} & 2000 & CP 2000 & 5 & \noindent{}\textcolor{black!50}{0.00} \textcolor{black!50}{0.00} \textcolor{black!50}{0.05} & 1 1 6 & 9 17 & 2 0 2\\
CarchraeBF05 \href{https://doi.org/10.1007/11564751_80}{CarchraeBF05} & \hyperref[auth:a272]{T. Carchrae}, \hyperref[auth:a89]{J. C. Beck}, \hyperref[auth:a273]{E. C. Freuder} & \cellcolor{green!10}Methods to Learn Abstract Scheduling Models & \hyperref[detail:CarchraeBF05]{Details} \href{../works/CarchraeBF05.pdf}{Yes} & \cite{CarchraeBF05} & 2005 & CP 2005 & 1 & \noindent{}\textcolor{black!50}{0.00} \textcolor{black!50}{0.00} \textcolor{black!50}{0.01} & 0 0 0 & 0 0 & 0 0 0\\
HauderBRPA20 \href{http://dx.doi.org/10.1016/j.cie.2020.106857}{HauderBRPA20} & \hyperref[auth:a550]{V. A. Hauder}, \hyperref[auth:a551]{A. Beham}, \hyperref[auth:a552]{S. Raggl}, \hyperref[auth:a553]{S. N. Parragh}, \hyperref[auth:a554]{M. Affenzeller} & \cellcolor{green!10}Resource-constrained multi-project scheduling with activity and time flexibility & \hyperref[detail:HauderBRPA20]{Details} \href{../works/HauderBRPA20.pdf}{Yes} & \cite{HauderBRPA20} & 2020 & Computers \  Industrial Engineering & 14 & \noindent{}\textcolor{black!50}{0.00} \textcolor{black!50}{0.00} \textbf{25.41} & 14 19 27 & 46 56 & 17 3 14\\
abs-1902-09244 \href{http://arxiv.org/abs/1902.09244}{abs-1902-09244} & \hyperref[auth:a550]{V. A. Hauder}, \hyperref[auth:a551]{A. Beham}, \hyperref[auth:a552]{S. Raggl}, \hyperref[auth:a553]{S. N. Parragh}, \hyperref[auth:a554]{M. Affenzeller} & On constraint programming for a new flexible project scheduling problem with resource constraints & \hyperref[detail:abs-1902-09244]{Details} \href{../works/abs-1902-09244.pdf}{Yes} & \cite{abs-1902-09244} & 2019 & CoRR & 62 & \noindent{}\textbf{1.50} \textbf{1.50} \textbf{350.76} & 0 0 0 & 0 0 & 0 0 0\\
CestaOS98 \href{https://doi.org/10.1007/3-540-49481-2_36}{CestaOS98} & \hyperref[auth:a284]{A. Cesta}, \hyperref[auth:a282]{A. Oddi}, \hyperref[auth:a298]{S. F. Smith} & Scheduling Multi-capacitated Resources Under Complex Temporal Constraints & \hyperref[detail:CestaOS98]{Details} \href{../works/CestaOS98.pdf}{Yes} & \cite{CestaOS98} & 1998 & Constraint Programming 1998 & 1 & \noindent{}\textcolor{black!50}{0.00} \textcolor{black!50}{0.00} \textcolor{black!50}{0.03} & 5 5 4 & 0 3 & 4 4 0\\
KovacsEKV05 \href{https://doi.org/10.1007/11564751_118}{KovacsEKV05} & \hyperref[auth:a146]{A. Kov{\'{a}}cs}, \hyperref[auth:a277]{P. Egri}, \hyperref[auth:a155]{T. Kis}, \hyperref[auth:a278]{J. V{\'{a}}ncza} & Proterv-II: An Integrated Production Planning and Scheduling System & \hyperref[detail:KovacsEKV05]{Details} \href{../works/KovacsEKV05.pdf}{Yes} & \cite{KovacsEKV05} & 2005 & CP 2005 & 1 & \noindent{}\textcolor{black!50}{0.00} \textcolor{black!50}{0.00} \textcolor{black!50}{0.03} & 2 2 1 & 3 3 & 1 0 1\\
ZibranR11 \href{https://doi.org/10.1109/ICPC.2011.45}{ZibranR11} & \hyperref[auth:a619]{M. F. Zibran}, \hyperref[auth:a620]{C. K. Roy} & Conflict-Aware Optimal Scheduling of Code Clone Refactoring: {A} Constraint Programming Approach & \hyperref[detail:ZibranR11]{Details} \href{../works/ZibranR11.pdf}{Yes} & \cite{ZibranR11} & 2011 & CP 2011 & 4 & \noindent{}\textbf{1.00} \textbf{1.00} \textcolor{black!50}{0.17} & 17 16 19 & 18 24 & 1 1 0\\
ZibranR11a \href{https://doi.org/10.1109/SCAM.2011.21}{ZibranR11a} & \hyperref[auth:a619]{M. F. Zibran}, \hyperref[auth:a620]{C. K. Roy} & A Constraint Programming Approach to Conflict-Aware Optimal Scheduling of Prioritized Code Clone Refactoring & \hyperref[detail:ZibranR11a]{Details} \href{../works/ZibranR11a.pdf}{Yes} & \cite{ZibranR11a} & 2011 & SCAM 2011 & 10 & \noindent{}\textbf{1.00} \textbf{1.00} \textbf{2.45} & 26 26 33 & 27 35 & 3 2 1\\
Caballero23 \href{https://doi.org/10.1007/s10601-023-09357-0}{Caballero23} & \hyperref[auth:a102]{J. C. Caballero} & Scheduling through logic-based tools & \hyperref[detail:Caballero23]{Details} \href{../works/Caballero23.pdf}{Yes} & \cite{Caballero23} & 2023 & Constraints An Int. J. & 1 & \noindent{}\textcolor{black!50}{0.00} \textcolor{black!50}{0.00} \textcolor{black!50}{0.00} & 0 0 0 & 0 0 & 0 0 0\\
KovacsEKV05 \href{https://doi.org/10.1007/11564751_118}{KovacsEKV05} & \hyperref[auth:a146]{A. Kov{\'{a}}cs}, \hyperref[auth:a277]{P. Egri}, \hyperref[auth:a155]{T. Kis}, \hyperref[auth:a278]{J. V{\'{a}}ncza} & Proterv-II: An Integrated Production Planning and Scheduling System & \hyperref[detail:KovacsEKV05]{Details} \href{../works/KovacsEKV05.pdf}{Yes} & \cite{KovacsEKV05} & 2005 & CP 2005 & 1 & \noindent{}\textcolor{black!50}{0.00} \textcolor{black!50}{0.00} \textcolor{black!50}{0.03} & 2 2 1 & 3 3 & 1 0 1\\
Caballero23 \href{https://doi.org/10.1007/s10601-023-09357-0}{Caballero23} & \hyperref[auth:a102]{J. C. Caballero} & Scheduling through logic-based tools & \hyperref[detail:Caballero23]{Details} \href{../works/Caballero23.pdf}{Yes} & \cite{Caballero23} & 2023 & Constraints An Int. J. & 1 & \noindent{}\textcolor{black!50}{0.00} \textcolor{black!50}{0.00} \textcolor{black!50}{0.00} & 0 0 0 & 0 0 & 0 0 0\\
CestaOS98 \href{https://doi.org/10.1007/3-540-49481-2_36}{CestaOS98} & \hyperref[auth:a284]{A. Cesta}, \hyperref[auth:a282]{A. Oddi}, \hyperref[auth:a298]{S. F. Smith} & Scheduling Multi-capacitated Resources Under Complex Temporal Constraints & \hyperref[detail:CestaOS98]{Details} \href{../works/CestaOS98.pdf}{Yes} & \cite{CestaOS98} & 1998 & Constraint Programming 1998 & 1 & \noindent{}\textcolor{black!50}{0.00} \textcolor{black!50}{0.00} \textcolor{black!50}{0.03} & 5 5 4 & 0 3 & 4 4 0\\
Baptiste09 \href{https://doi.org/10.1007/978-3-642-04244-7_1}{Baptiste09} & \hyperref[auth:a162]{P. Baptiste} & Constraint-Based Schedulers, Do They Really Work? & \hyperref[detail:Baptiste09]{Details} \href{../works/Baptiste09.pdf}{Yes} & \cite{Baptiste09} & 2009 & CP 2009 & 1 & \noindent{}\textcolor{black!50}{0.00} \textcolor{black!50}{0.00} \textcolor{black!50}{0.03} & 0 0 0 & 0 0 & 0 0 0\\
FrostD98 \href{https://doi.org/10.1007/3-540-49481-2_40}{FrostD98} & \hyperref[auth:a299]{D. Frost}, \hyperref[auth:a300]{R. Dechter} & Optimizing with Constraints: {A} Case Study in Scheduling Maintenance of Electric Power Units & \hyperref[detail:FrostD98]{Details} \href{../works/FrostD98.pdf}{Yes} & \cite{FrostD98} & 1998 & Constraint Programming 1998 & 1 & \noindent{}\textcolor{black!50}{0.00} \textcolor{black!50}{0.00} \textcolor{black!50}{0.02} & 10 10 11 & 2 3 & 0 0 0\\
AbrilSB05 \href{https://doi.org/10.1007/11564751_75}{AbrilSB05} & \hyperref[auth:a270]{M. Abril}, \hyperref[auth:a153]{M. A. Salido}, \hyperref[auth:a271]{F. Barber} & \cellcolor{green!10}Distributed Constraints for Large-Scale Scheduling Problems & \hyperref[detail:AbrilSB05]{Details} \href{../works/AbrilSB05.pdf}{Yes} & \cite{AbrilSB05} & 2005 & CP 2005 & 1 & \noindent{}\textcolor{black!50}{0.00} \textcolor{black!50}{0.00} \textcolor{black!50}{0.01} & 0 0 0 & 0 2 & 0 0 0\\
CarchraeBF05 \href{https://doi.org/10.1007/11564751_80}{CarchraeBF05} & \hyperref[auth:a272]{T. Carchrae}, \hyperref[auth:a89]{J. C. Beck}, \hyperref[auth:a273]{E. C. Freuder} & \cellcolor{green!10}Methods to Learn Abstract Scheduling Models & \hyperref[detail:CarchraeBF05]{Details} \href{../works/CarchraeBF05.pdf}{Yes} & \cite{CarchraeBF05} & 2005 & CP 2005 & 1 & \noindent{}\textcolor{black!50}{0.00} \textcolor{black!50}{0.00} \textcolor{black!50}{0.01} & 0 0 0 & 0 0 & 0 0 0\\
\end{longtable}
}



{\scriptsize
\begin{longtable}{>{\raggedright\arraybackslash}p{3cm}r>{\raggedright\arraybackslash}p{4cm}p{1.5cm}p{2cm}p{1.5cm}p{1.5cm}p{1.5cm}p{1.5cm}p{2cm}p{1.5cm}rr}
\rowcolor{white}\caption{Automatically Extracted  Properties (Requires Local Copy)}\\ \toprule
\rowcolor{white}Work & Pages & Concepts & Classification & Constraints & \shortstack{Prog\\Languages} & \shortstack{CP\\Systems} & Areas & Industries & Benchmarks & Algorithm & a & c\\ \midrule\endhead
\bottomrule
\endfoot
\href{../works/TasselGS23.pdf}{TasselGS23}~\cite{TasselGS23} & 9 & job-shop, flow-shop, completion-time, CP, resource, flow-time, re-scheduling, job, constraint programming, precedence, order, tardiness, constraint optimization, scheduling, preempt, task, machine, make-span, periodic & JSSP & cumulative, disjunctive, noOverlap & Java & Choco Solver &  &  & industrial instance, real-world, github, benchmark, supplementary material & genetic algorithm, neural network, large neighborhood search, machine learning, simulated annealing, reinforcement learning, meta heuristic & \ref{a:TasselGS23} & \ref{c:TasselGS23}\\
\href{../works/abs-2306-05747.pdf}{abs-2306-05747}~\cite{abs-2306-05747} & 9 & re-scheduling, scheduling, order, make-span, preempt, constraint programming, CP, flow-time, completion-time, resource, job, periodic, job-shop, precedence, constraint optimization, task, tardiness, machine, flow-shop & JSSP & noOverlap, disjunctive, cumulative & Java & Choco Solver &  &  & real-world, github, industrial instance, supplementary material, benchmark & neural network, large neighborhood search, reinforcement learning, genetic algorithm, machine learning, meta heuristic, simulated annealing & \ref{a:abs-2306-05747} & \ref{c:abs-2306-05747}\\
\href{../works/PerezGSL23.pdf}{PerezGSL23}~\cite{PerezGSL23} & 7 & inventory, order, transportation, re-scheduling, scheduling, task, machine, make-span, resource, activity, constraint programming, completion-time, CP &  & table constraint, cumulative &  & OPL & container terminal, nurse, operating room, steel mill &  & real-world, generated instance & large neighborhood search, mat heuristic, meta heuristic & \ref{a:PerezGSL23} & \ref{c:PerezGSL23}\\
\href{../works/abs-2312-13682.pdf}{abs-2312-13682}~\cite{abs-2312-13682} & 20 & activity, constraint programming, machine, inventory, re-scheduling, scheduling, order, make-span, CP, resource, transportation, task &  & table constraint, cumulative &  & OPL & container terminal, train schedule, nurse, steel mill, operating room &  & real-world, generated instance & large neighborhood search, mat heuristic, meta heuristic & \ref{a:abs-2312-13682} & \ref{c:abs-2312-13682}\\
\href{../works/NishikawaSTT18.pdf}{NishikawaSTT18}~\cite{NishikawaSTT18} & 6 & precedence, scheduling, make-span, activity, distributed, constraint programming, order, CP, resource, task &  & alternative constraint, endBeforeStart &  & Cplex & robot, pipeline &  & real-world, benchmark & genetic algorithm & \ref{a:NishikawaSTT18} & n/a\\
\href{../works/NishikawaSTT18a.pdf}{NishikawaSTT18a}~\cite{NishikawaSTT18a} & 6 & CP, make-span, scheduling, resource, task, constraint programming, distributed, re-scheduling, order, precedence, activity &  & endBeforeStart, alternative constraint &  & Cplex & robot, nurse, pipeline &  & benchmark, real-life, real-world & genetic algorithm & \ref{a:NishikawaSTT18a} & n/a\\
\href{../works/NuijtenA94.pdf}{NuijtenA94}~\cite{NuijtenA94} & 5 & scheduling, preempt, machine, make-span, constraint satisfaction, preemptive, job-shop, completion-time, CP, resource, job, CSP, precedence, CLP, order & JSSP & disjunctive, Disjunctive constraint & C++ & Ilog Solver, CPO &  &  &  & time-tabling & \ref{a:NuijtenA94} & n/a\\
\href{../works/NuijtenA96.pdf}{NuijtenA96}~\cite{NuijtenA96} & 16 & scheduling, preempt, machine, make-span, constraint satisfaction, preemptive, job-shop, flow-shop, completion-time, CP, resource, job, constraint programming, CSP, precedence, CLP, order & JSSP & disjunctive, Disjunctive constraint &  & CPO &  &  &  & time-tabling & \ref{a:NuijtenA96} & n/a\\
\href{../works/BehrensLM19.pdf}{BehrensLM19}~\cite{BehrensLM19} & 7 & resource, setup-time, task, constraint satisfaction, constraint programming, make-span, order, machine, CP, scheduling, CSP, distributed, multi-agent, constraint optimization &  &  & Python & OR-Tools, MiniZinc & robot &  & github, real-world &  & \ref{a:BehrensLM19} & \ref{c:BehrensLM19}\\
\href{../works/abs-1901-07914.pdf}{abs-1901-07914}~\cite{abs-1901-07914} & 8 & constraint programming, CP, resource, CSP, constraint satisfaction, constraint optimization, task, distributed, machine, multi-agent, scheduling, order, make-span &  &  & Python & OR-Tools, MiniZinc & robot &  & real-world, github, benchmark &  & \ref{a:abs-1901-07914} & \ref{c:abs-1901-07914}\\
\href{../works/NaderiBZ22.pdf}{NaderiBZ22}~\cite{NaderiBZ22} & 29 & stochastic, setup-time, open-shop, order, scheduling, machine, make-span, distributed, Logic-Based Benders Decomposition, job-shop, due-date, tardiness, flow-shop, lateness, CP, resource, transportation, no-wait, job, constraint programming, completion-time, Benders Decomposition & parallel machine, single machine & disjunctive, noOverlap, Disjunctive constraint &  & Cplex, CPO & crew-scheduling, nurse, surgery, patient, operating room, automotive &  & benchmark, real-life & meta heuristic, memetic algorithm & \ref{a:NaderiBZ22} & n/a\\
\href{../works/NaderiBZ23.pdf}{NaderiBZ23}~\cite{NaderiBZ23} & 32 & stochastic, setup-time, open-shop, order, scheduling, machine, make-span, distributed, Logic-Based Benders Decomposition, job-shop, due-date, tardiness, flow-shop, lateness, CP, resource, transportation, no-wait, job, constraint programming, completion-time, Benders Decomposition & parallel machine, single machine & disjunctive, noOverlap, Disjunctive constraint & Python & Cplex, CPO & crew-scheduling, nurse, surgery, patient, operating room, automotive &  & benchmark, real-world & meta heuristic, memetic algorithm & \ref{a:NaderiBZ23} & n/a\\
\href{../works/EvenSH15.pdf}{EvenSH15}~\cite{EvenSH15} & 18 & transportation, CP, preempt, machine, distributed, resource, preemptive, order, scheduling, constraint programming, Benders Decomposition, completion-time, task &  & Disjunctive constraint, cumulative, disjunctive &  & OPL, Choco Solver & evacuation, emergency service &  & real-life, real-world & column generation, sweep, mat heuristic, ant colony & \ref{a:EvenSH15} & n/a\\
\href{../works/EvenSH15a.pdf}{EvenSH15a}~\cite{EvenSH15a} & 16 & distributed, constraint programming, resource, transportation, Benders Decomposition, order, preempt, scheduling, task, machine, preemptive, completion-time, CP &  & disjunctive, Disjunctive constraint, cumulative & Java & Choco Solver, OPL & emergency service, evacuation &  & real-world, real-life & ant colony, mat heuristic, meta heuristic, column generation, sweep & \ref{a:EvenSH15a} & n/a\\
\href{../works/Baptiste09.pdf}{Baptiste09}~\cite{Baptiste09} & 1 & scheduling, CP &  &  &  &  &  &  &  &  & \ref{a:Baptiste09} & n/a\\
\href{../works/CarchraeBF05.pdf}{CarchraeBF05}~\cite{CarchraeBF05} & 1 & task, scheduling, make-span, order, CP & Partial Order Schedule &  &  &  &  &  &  &  & \ref{a:CarchraeBF05} & n/a\\
\href{../works/NattafAL15.pdf}{NattafAL15}~\cite{NattafAL15} & 21 & release-date, scheduling, preempt, task, make-span, due-date, resource, preemptive, activity, constraint programming, CSP, CP, order & CECSP, RCPSP, Resource-constrained Project Scheduling Problem, CuSP & cumulative & C++ & Cplex &  &  & generated instance & sweep, energetic reasoning & \ref{a:NattafAL15} & \ref{c:NattafAL15}\\
\href{../works/NattafALR16.pdf}{NattafALR16}~\cite{NattafALR16} & 34 & preemptive, no preempt, task, constraint programming, precedence, make-span, order, preempt, CP, scheduling, due-date, CSP, activity, explanation, resource, release-date & CECSP, CuSP, Resource-constrained Project Scheduling Problem, RCPSP & cumulative & C++ & Cplex &  &  & generated instance & sweep, energetic reasoning & \ref{a:NattafALR16} & n/a\\
\href{../works/AngelsmarkJ00.pdf}{AngelsmarkJ00}~\cite{AngelsmarkJ00} & 5 & resource, job, order, constraint satisfaction, task, CP, scheduling, job-shop &  &  &  &  &  &  &  &  & \ref{a:AngelsmarkJ00} & n/a\\
\href{../works/CarchraeBF05.pdf}{CarchraeBF05}~\cite{CarchraeBF05} & 1 & task, scheduling, make-span, order, CP & Partial Order Schedule &  &  &  &  &  &  &  & \ref{a:CarchraeBF05} & n/a\\
\href{../works/ZibranR11.pdf}{ZibranR11}~\cite{ZibranR11} & 4 & CP, constraint programming, order, activity, constraint satisfaction, scheduling &  &  & Java & Cplex, OPL &  &  &  & simulated annealing, genetic algorithm, meta heuristic & \ref{a:ZibranR11} & n/a\\
\href{../works/ZibranR11a.pdf}{ZibranR11a}~\cite{ZibranR11a} & 10 & distributed, activity, CP, order, constraint programming, scheduling, constraint satisfaction, resource &  &  &  & Cplex, OPL &  &  &  & meta heuristic, time-tabling, genetic algorithm, simulated annealing & \ref{a:ZibranR11a} & n/a\\
\href{../works/Hooker05a.pdf}{Hooker05a}~\cite{Hooker05a} & 14 & CP, machine, constraint programming, resource, Benders Decomposition, order, release-date, scheduling, Logic-Based Benders Decomposition, make-span, task, constraint logic programming, job, due-date, precedence, tardiness &  & circuit, cumulative, disjunctive &  & Ilog Scheduler, OPL, Cplex &  &  &  & MINLP & \ref{a:Hooker05a} & n/a\\
\href{../works/Hooker06.pdf}{Hooker06}~\cite{Hooker06} & 19 & constraint satisfaction, machine, job, task, release-date, constraint programming, Logic-Based Benders Decomposition, CP, make-span, constraint logic programming, resource, precedence, due-date, order, tardiness, scheduling, Benders Decomposition &  & disjunctive, cumulative, circuit &  & OPL, Ilog Scheduler, Cplex &  &  & random instance & MINLP & \ref{a:Hooker06} & \ref{c:Hooker06}\\
\href{../works/HauderBRPA20.pdf}{HauderBRPA20}~\cite{HauderBRPA20} & 14 & setup-time, order, bi-objective, no-wait, job-shop, resource, stochastic, task, constraint programming, completion-time, precedence, earliness, machine, transportation, tardiness, make-span, activity, explanation, inventory, due-date, scheduling, flow-shop, job, CP, multi-objective, breakdown, manpower & RCPSP, RCMPSP, FJS, Resource-constrained Project Scheduling Problem & cumulative, cycle &  & OPL, Cplex & aircraft & automobile industry, food-processing industry, steel industry, processing industry & industry partner, benchmark, real-world, supplementary material & particle swarm, genetic algorithm, meta heuristic & \ref{a:HauderBRPA20} & \ref{c:HauderBRPA20}\\
\href{../works/abs-1902-09244.pdf}{abs-1902-09244}~\cite{abs-1902-09244} & 62 & setup-time, activity, constraint programming, machine, flow-shop, CP, job, order, due-date, earliness, bi-objective, stochastic, explanation, completion-time, breakdown, resource, task, job-shop, tardiness, inventory, multi-objective, no-wait, precedence, transportation, scheduling, make-span, release-date & Resource-constrained Project Scheduling Problem, FJS, RCMPSP, RCPSP & cycle, cumulative, endBeforeStart &  & OPL, Cplex & aircraft & automobile industry, steel industry, food-processing industry, glass industry, processing industry & real-world, benchmark, industry partner & genetic algorithm, particle swarm, simulated annealing, meta heuristic & \ref{a:abs-1902-09244} & n/a\\
\href{../works/Baptiste09.pdf}{Baptiste09}~\cite{Baptiste09} & 1 & scheduling, CP &  &  &  &  &  &  &  &  & \ref{a:Baptiste09} & n/a\\
\href{../works/Caballero23.pdf}{Caballero23}~\cite{Caballero23} & 1 & CP, scheduling, resource & Resource-constrained Project Scheduling Problem, RCPSP &  &  &  &  &  &  &  & \ref{a:Caballero23} & \ref{c:Caballero23}\\
\href{../works/AbrilSB05.pdf}{AbrilSB05}~\cite{AbrilSB05} & 1 & distributed, scheduling, multi-agent, CP, order, CSP &  &  &  &  & railway &  &  &  & \ref{a:AbrilSB05} & n/a\\
\href{../works/Baptiste09.pdf}{Baptiste09}~\cite{Baptiste09} & 1 & scheduling, CP &  &  &  &  &  &  &  &  & \ref{a:Baptiste09} & n/a\\
\href{../works/AngelsmarkJ00.pdf}{AngelsmarkJ00}~\cite{AngelsmarkJ00} & 5 & resource, job, order, constraint satisfaction, task, CP, scheduling, job-shop &  &  &  &  &  &  &  &  & \ref{a:AngelsmarkJ00} & n/a\\
\href{../works/HebrardTW05.pdf}{HebrardTW05}~\cite{HebrardTW05} & 1 & scheduling, job-shop, job, constraint programming, CP, constraint satisfaction, order, machine &  &  &  &  &  &  &  &  & \ref{a:HebrardTW05} & n/a\\
\href{../works/CestaOS98.pdf}{CestaOS98}~\cite{CestaOS98} & 1 & job, scheduling, CSP, CP, resource &  &  &  &  & robot &  &  &  & \ref{a:CestaOS98} & n/a\\
\href{../works/KovacsEKV05.pdf}{KovacsEKV05}~\cite{KovacsEKV05} & 1 & setup-time, CP, precedence, scheduling, resource, job-shop, job, constraint programming & Resource-constrained Project Scheduling Problem &  &  &  &  &  & real-life &  & \ref{a:KovacsEKV05} & n/a\\
\href{../works/Rodriguez07b.pdf}{Rodriguez07b}~\cite{Rodriguez07b} & 14 & re-scheduling, CP, blocking constraint, order, no-wait, job-shop, resource, CSP, task, constraint programming, release-date, precedence, scheduling, transportation, activity, job &  & circuit, disjunctive, Blocking constraint, Disjunctive constraint &  & Ilog Scheduler, Z3, Ilog Solver & train schedule, railway & railway industry &  & edge-finding & \ref{a:Rodriguez07b} & n/a\\
\href{../works/RodriguezS09.pdf}{RodriguezS09}~\cite{RodriguezS09} & 14 & blocking constraint, Benders Decomposition, order, no-wait, Logic-Based Benders Decomposition, job-shop, resource, CSP, task, constraint programming, completion-time, precedence, scheduling, transportation, activity, constraint satisfaction, job, CP &  & circuit, disjunctive, Blocking constraint, Disjunctive constraint &  & Ilog Scheduler, Ilog Solver & train schedule, railway &  &  & edge-finding & \ref{a:RodriguezS09} & n/a\\
\href{../works/Davis87.pdf}{Davis87}~\cite{Davis87} & 51 & machine, task, constraint satisfaction, job, order, scheduling, CP &  & disjunctive, circuit, cycle &  &  & robot &  &  &  & \ref{a:Davis87} & n/a\\
\href{../works/Valdes87.pdf}{Valdes87}~\cite{Valdes87} & 5 & precedence, order, constraint satisfaction, task &  & circuit, disjunctive, cycle &  &  &  &  &  &  & \ref{a:Valdes87} & n/a\\
\href{../works/Caballero23.pdf}{Caballero23}~\cite{Caballero23} & 1 & CP, scheduling, resource & Resource-constrained Project Scheduling Problem, RCPSP &  &  &  &  &  &  &  & \ref{a:Caballero23} & \ref{c:Caballero23}\\
\href{../works/KovacsEKV05.pdf}{KovacsEKV05}~\cite{KovacsEKV05} & 1 & setup-time, CP, precedence, scheduling, resource, job-shop, job, constraint programming & Resource-constrained Project Scheduling Problem &  &  &  &  &  & real-life &  & \ref{a:KovacsEKV05} & n/a\\
\end{longtable}
}



\clearpage
\subsection{Most Similar Works Based on Dot Product Similarity}

As before, but now based on the dot product similarity measure.

{\scriptsize
\begin{longtable}{>{\raggedright\arraybackslash}p{3cm}>{\raggedright\arraybackslash}p{6cm}>{\raggedright\arraybackslash}p{6.5cm}rrrp{2.5cm}rrrrr}
\rowcolor{white}\caption{Works from bibtex (Total 40)}\\ \toprule
\rowcolor{white}\shortstack{Key\\Source} & Authors & Title & LC & Cite & Year & \shortstack{Conference\\/Journal\\/School} & Pages & \shortstack{Nr\\Cites} & \shortstack{Nr\\Refs} & b & c \\ \midrule\endhead
\bottomrule
\endfoot
Groleaz21 \href{https://hal.science/tel-03266690}{Groleaz21} & \hyperref[auth:a83]{L. Groleaz} & {The Group Cumulative Scheduling Problem} & \href{../works/Groleaz21.pdf}{Yes} & \cite{Groleaz21} & 2021 & {Universit{\'e} de Lyon} & 153 & 0 & 0 & \ref{b:Groleaz21} & n/a\\
ZarandiASC20 \href{https://doi.org/10.1007/s10462-018-9667-6}{ZarandiASC20} & \hyperref[auth:a832]{Mohammad Hossein Fazel Zarandi}, \hyperref[auth:a833]{Ali Akbar Sadat Asl}, \hyperref[auth:a834]{S. Sotudian}, \hyperref[auth:a835]{O. Castillo} & A state of the art review of intelligent scheduling & \href{../works/ZarandiASC20.pdf}{Yes} & \cite{ZarandiASC20} & 2020 & Artif. Intell. Rev. & 93 & 55 & 445 & \ref{b:ZarandiASC20} & n/a\\
Dejemeppe16 \href{https://hdl.handle.net/2078.1/178078}{Dejemeppe16} & \hyperref[auth:a207]{C. Dejemeppe} & Constraint programming algorithms and models for scheduling applications & \href{../works/Dejemeppe16.pdf}{Yes} & \cite{Dejemeppe16} & 2016 & Catholic University of Louvain, Louvain-la-Neuve, Belgium & 274 & 0 & 0 & \ref{b:Dejemeppe16} & n/a\\
ZarandiASC20 \href{https://doi.org/10.1007/s10462-018-9667-6}{ZarandiASC20} & \hyperref[auth:a832]{Mohammad Hossein Fazel Zarandi}, \hyperref[auth:a833]{Ali Akbar Sadat Asl}, \hyperref[auth:a834]{S. Sotudian}, \hyperref[auth:a835]{O. Castillo} & A state of the art review of intelligent scheduling & \href{../works/ZarandiASC20.pdf}{Yes} & \cite{ZarandiASC20} & 2020 & Artif. Intell. Rev. & 93 & 55 & 445 & \ref{b:ZarandiASC20} & n/a\\
Baptiste02 \href{https://theses.hal.science/tel-00124998}{Baptiste02} & \hyperref[auth:a163]{P. Baptiste} & {R{\'e}sultats de complexit{\'e} et programmation par contraintes pour l'ordonnancement} & \href{../works/Baptiste02.pdf}{Yes} & \cite{Baptiste02} & 2002 & {Universit{\'e} de Technologie de Compi{\`e}gne} & 237 & 0 & 0 & \ref{b:Baptiste02} & n/a\\
ZarandiASC20 \href{https://doi.org/10.1007/s10462-018-9667-6}{ZarandiASC20} & \hyperref[auth:a832]{Mohammad Hossein Fazel Zarandi}, \hyperref[auth:a833]{Ali Akbar Sadat Asl}, \hyperref[auth:a834]{S. Sotudian}, \hyperref[auth:a835]{O. Castillo} & A state of the art review of intelligent scheduling & \href{../works/ZarandiASC20.pdf}{Yes} & \cite{ZarandiASC20} & 2020 & Artif. Intell. Rev. & 93 & 55 & 445 & \ref{b:ZarandiASC20} & n/a\\
Dejemeppe16 \href{https://hdl.handle.net/2078.1/178078}{Dejemeppe16} & \hyperref[auth:a207]{C. Dejemeppe} & Constraint programming algorithms and models for scheduling applications & \href{../works/Dejemeppe16.pdf}{Yes} & \cite{Dejemeppe16} & 2016 & Catholic University of Louvain, Louvain-la-Neuve, Belgium & 274 & 0 & 0 & \ref{b:Dejemeppe16} & n/a\\
Groleaz21 \href{https://hal.science/tel-03266690}{Groleaz21} & \hyperref[auth:a83]{L. Groleaz} & {The Group Cumulative Scheduling Problem} & \href{../works/Groleaz21.pdf}{Yes} & \cite{Groleaz21} & 2021 & {Universit{\'e} de Lyon} & 153 & 0 & 0 & \ref{b:Groleaz21} & n/a\\
Baptiste02 \href{https://theses.hal.science/tel-00124998}{Baptiste02} & \hyperref[auth:a163]{P. Baptiste} & {R{\'e}sultats de complexit{\'e} et programmation par contraintes pour l'ordonnancement} & \href{../works/Baptiste02.pdf}{Yes} & \cite{Baptiste02} & 2002 & {Universit{\'e} de Technologie de Compi{\`e}gne} & 237 & 0 & 0 & \ref{b:Baptiste02} & n/a\\
Groleaz21 \href{https://hal.science/tel-03266690}{Groleaz21} & \hyperref[auth:a83]{L. Groleaz} & {The Group Cumulative Scheduling Problem} & \href{../works/Groleaz21.pdf}{Yes} & \cite{Groleaz21} & 2021 & {Universit{\'e} de Lyon} & 153 & 0 & 0 & \ref{b:Groleaz21} & n/a\\
Lunardi20 \href{http://orbilu.uni.lu/handle/10993/43893}{Lunardi20} & \hyperref[auth:a499]{Willian Tessaro Lunardi} & A Real-World Flexible Job Shop Scheduling Problem With Sequencing Flexibility: Mathematical Programming, Constraint Programming, and Metaheuristics & \href{../works/Lunardi20.pdf}{Yes} & \cite{Lunardi20} & 2020 & University of Luxembourg, Luxembourg City, Luxembourg & 181 & 0 & 0 & \ref{b:Lunardi20} & n/a\\
ZarandiASC20 \href{https://doi.org/10.1007/s10462-018-9667-6}{ZarandiASC20} & \hyperref[auth:a832]{Mohammad Hossein Fazel Zarandi}, \hyperref[auth:a833]{Ali Akbar Sadat Asl}, \hyperref[auth:a834]{S. Sotudian}, \hyperref[auth:a835]{O. Castillo} & A state of the art review of intelligent scheduling & \href{../works/ZarandiASC20.pdf}{Yes} & \cite{ZarandiASC20} & 2020 & Artif. Intell. Rev. & 93 & 55 & 445 & \ref{b:ZarandiASC20} & n/a\\
Dejemeppe16 \href{https://hdl.handle.net/2078.1/178078}{Dejemeppe16} & \hyperref[auth:a207]{C. Dejemeppe} & Constraint programming algorithms and models for scheduling applications & \href{../works/Dejemeppe16.pdf}{Yes} & \cite{Dejemeppe16} & 2016 & Catholic University of Louvain, Louvain-la-Neuve, Belgium & 274 & 0 & 0 & \ref{b:Dejemeppe16} & n/a\\
Malapert11 \href{https://tel.archives-ouvertes.fr/tel-00630122}{Malapert11} & \hyperref[auth:a82]{A. Malapert} & Techniques d'ordonnancement d'atelier et de fourn{\'{e}}es bas{\'{e}}es sur la programmation par contraintes. (Shop and batch scheduling with constraints) & \href{../works/Malapert11.pdf}{Yes} & \cite{Malapert11} & 2011 & {\'{E}}cole des mines de Nantes, France & 194 & 0 & 0 & \ref{b:Malapert11} & n/a\\
Astrand21 \href{https://nbn-resolving.org/urn:nbn:se:kth:diva-294959}{Astrand21} & \hyperref[auth:a74]{M. {\AA}strand} & Short-term Underground Mine Scheduling: An Industrial Application of Constraint Programming & \href{../works/Astrand21.pdf}{Yes} & \cite{Astrand21} & 2021 & Royal Institute of Technology, Stockholm, Sweden & 142 & 0 & 0 & \ref{b:Astrand21} & n/a\\
ZarandiASC20 \href{https://doi.org/10.1007/s10462-018-9667-6}{ZarandiASC20} & \hyperref[auth:a832]{Mohammad Hossein Fazel Zarandi}, \hyperref[auth:a833]{Ali Akbar Sadat Asl}, \hyperref[auth:a834]{S. Sotudian}, \hyperref[auth:a835]{O. Castillo} & A state of the art review of intelligent scheduling & \href{../works/ZarandiASC20.pdf}{Yes} & \cite{ZarandiASC20} & 2020 & Artif. Intell. Rev. & 93 & 55 & 445 & \ref{b:ZarandiASC20} & n/a\\
Siala15 \href{https://doi.org/10.1007/s10601-015-9213-y}{Siala15} & \hyperref[auth:a130]{M. Siala} & Search, propagation, and learning in sequencing and scheduling problems & \href{../works/Siala15.pdf}{Yes} & \cite{Siala15} & 2015 & Constraints An Int. J. & 2 & 4 & 0 & \ref{b:Siala15} & \ref{c:Siala15}\\
Siala15a \href{https://tel.archives-ouvertes.fr/tel-01164291}{Siala15a} & \hyperref[auth:a130]{M. Siala} & Search, propagation, and learning in sequencing and scheduling problems. (Recherche, propagation et apprentissage dans les probl{\`{e}}mes de s{\'{e}}quencement et d'ordonnancement) & \href{../works/Siala15a.pdf}{Yes} & \cite{Siala15a} & 2015 & {INSA} Toulouse, France & 199 & 0 & 0 & \ref{b:Siala15a} & n/a\\
Baptiste02 \href{https://theses.hal.science/tel-00124998}{Baptiste02} & \hyperref[auth:a163]{P. Baptiste} & {R{\'e}sultats de complexit{\'e} et programmation par contraintes pour l'ordonnancement} & \href{../works/Baptiste02.pdf}{Yes} & \cite{Baptiste02} & 2002 & {Universit{\'e} de Technologie de Compi{\`e}gne} & 237 & 0 & 0 & \ref{b:Baptiste02} & n/a\\
Dejemeppe16 \href{https://hdl.handle.net/2078.1/178078}{Dejemeppe16} & \hyperref[auth:a207]{C. Dejemeppe} & Constraint programming algorithms and models for scheduling applications & \href{../works/Dejemeppe16.pdf}{Yes} & \cite{Dejemeppe16} & 2016 & Catholic University of Louvain, Louvain-la-Neuve, Belgium & 274 & 0 & 0 & \ref{b:Dejemeppe16} & n/a\\
Baptiste02 \href{https://theses.hal.science/tel-00124998}{Baptiste02} & \hyperref[auth:a163]{P. Baptiste} & {R{\'e}sultats de complexit{\'e} et programmation par contraintes pour l'ordonnancement} & \href{../works/Baptiste02.pdf}{Yes} & \cite{Baptiste02} & 2002 & {Universit{\'e} de Technologie de Compi{\`e}gne} & 237 & 0 & 0 & \ref{b:Baptiste02} & n/a\\
Malapert11 \href{https://tel.archives-ouvertes.fr/tel-00630122}{Malapert11} & \hyperref[auth:a82]{A. Malapert} & Techniques d'ordonnancement d'atelier et de fourn{\'{e}}es bas{\'{e}}es sur la programmation par contraintes. (Shop and batch scheduling with constraints) & \href{../works/Malapert11.pdf}{Yes} & \cite{Malapert11} & 2011 & {\'{E}}cole des mines de Nantes, France & 194 & 0 & 0 & \ref{b:Malapert11} & n/a\\
Astrand21 \href{https://nbn-resolving.org/urn:nbn:se:kth:diva-294959}{Astrand21} & \hyperref[auth:a74]{M. {\AA}strand} & Short-term Underground Mine Scheduling: An Industrial Application of Constraint Programming & \href{../works/Astrand21.pdf}{Yes} & \cite{Astrand21} & 2021 & Royal Institute of Technology, Stockholm, Sweden & 142 & 0 & 0 & \ref{b:Astrand21} & n/a\\
Groleaz21 \href{https://hal.science/tel-03266690}{Groleaz21} & \hyperref[auth:a83]{L. Groleaz} & {The Group Cumulative Scheduling Problem} & \href{../works/Groleaz21.pdf}{Yes} & \cite{Groleaz21} & 2021 & {Universit{\'e} de Lyon} & 153 & 0 & 0 & \ref{b:Groleaz21} & n/a\\
Groleaz21 \href{https://hal.science/tel-03266690}{Groleaz21} & \hyperref[auth:a83]{L. Groleaz} & {The Group Cumulative Scheduling Problem} & \href{../works/Groleaz21.pdf}{Yes} & \cite{Groleaz21} & 2021 & {Universit{\'e} de Lyon} & 153 & 0 & 0 & \ref{b:Groleaz21} & n/a\\
Lunardi20 \href{http://orbilu.uni.lu/handle/10993/43893}{Lunardi20} & \hyperref[auth:a499]{Willian Tessaro Lunardi} & A Real-World Flexible Job Shop Scheduling Problem With Sequencing Flexibility: Mathematical Programming, Constraint Programming, and Metaheuristics & \href{../works/Lunardi20.pdf}{Yes} & \cite{Lunardi20} & 2020 & University of Luxembourg, Luxembourg City, Luxembourg & 181 & 0 & 0 & \ref{b:Lunardi20} & n/a\\
PrataAN23 \href{https://www.sciencedirect.com/science/article/pii/S2666720723001522}{PrataAN23} & \hyperref[auth:a388]{Bruno A. Prata}, \hyperref[auth:a389]{Levi R. Abreu}, \hyperref[auth:a390]{Marcelo S. Nagano} & Applications of constraint programming in production scheduling problems: A descriptive bibliometric analysis & \href{../works/PrataAN23.pdf}{Yes} & \cite{PrataAN23} & 2024 & Results in Control and Optimization & 17 & 0 & 0 & \ref{b:PrataAN23} & \ref{c:PrataAN23}\\
ZarandiASC20 \href{https://doi.org/10.1007/s10462-018-9667-6}{ZarandiASC20} & \hyperref[auth:a832]{Mohammad Hossein Fazel Zarandi}, \hyperref[auth:a833]{Ali Akbar Sadat Asl}, \hyperref[auth:a834]{S. Sotudian}, \hyperref[auth:a835]{O. Castillo} & A state of the art review of intelligent scheduling & \href{../works/ZarandiASC20.pdf}{Yes} & \cite{ZarandiASC20} & 2020 & Artif. Intell. Rev. & 93 & 55 & 445 & \ref{b:ZarandiASC20} & n/a\\
Groleaz21 \href{https://hal.science/tel-03266690}{Groleaz21} & \hyperref[auth:a83]{L. Groleaz} & {The Group Cumulative Scheduling Problem} & \href{../works/Groleaz21.pdf}{Yes} & \cite{Groleaz21} & 2021 & {Universit{\'e} de Lyon} & 153 & 0 & 0 & \ref{b:Groleaz21} & n/a\\
Malapert11 \href{https://tel.archives-ouvertes.fr/tel-00630122}{Malapert11} & \hyperref[auth:a82]{A. Malapert} & Techniques d'ordonnancement d'atelier et de fourn{\'{e}}es bas{\'{e}}es sur la programmation par contraintes. (Shop and batch scheduling with constraints) & \href{../works/Malapert11.pdf}{Yes} & \cite{Malapert11} & 2011 & {\'{E}}cole des mines de Nantes, France & 194 & 0 & 0 & \ref{b:Malapert11} & n/a\\
Groleaz21 \href{https://hal.science/tel-03266690}{Groleaz21} & \hyperref[auth:a83]{L. Groleaz} & {The Group Cumulative Scheduling Problem} & \href{../works/Groleaz21.pdf}{Yes} & \cite{Groleaz21} & 2021 & {Universit{\'e} de Lyon} & 153 & 0 & 0 & \ref{b:Groleaz21} & n/a\\
Lombardi10 \href{http://amsdottorato.unibo.it/2961/}{Lombardi10} & \hyperref[auth:a143]{M. Lombardi} & Hybrid Methods for Resource Allocation and Scheduling Problems in Deterministic and Stochastic Environments & \href{../works/Lombardi10.pdf}{Yes} & \cite{Lombardi10} & 2010 & University of Bologna, Italy & 175 & 0 & 0 & \ref{b:Lombardi10} & n/a\\
Baptiste02 \href{https://theses.hal.science/tel-00124998}{Baptiste02} & \hyperref[auth:a163]{P. Baptiste} & {R{\'e}sultats de complexit{\'e} et programmation par contraintes pour l'ordonnancement} & \href{../works/Baptiste02.pdf}{Yes} & \cite{Baptiste02} & 2002 & {Universit{\'e} de Technologie de Compi{\`e}gne} & 237 & 0 & 0 & \ref{b:Baptiste02} & n/a\\
Lombardi10 \href{http://amsdottorato.unibo.it/2961/}{Lombardi10} & \hyperref[auth:a143]{M. Lombardi} & Hybrid Methods for Resource Allocation and Scheduling Problems in Deterministic and Stochastic Environments & \href{../works/Lombardi10.pdf}{Yes} & \cite{Lombardi10} & 2010 & University of Bologna, Italy & 175 & 0 & 0 & \ref{b:Lombardi10} & n/a\\
Fahimi16 \href{http://cp2014.a4cp.org/sites/default/files/hamed_fahimi_-_efficient_algorithms_to_solve_scheduling_problems_with_a_variety_of_optimization_criteria.pdf}{Fahimi16} & \hyperref[auth:a122]{H. Fahimi} & Efficient algorithms to solve scheduling problems with a variety of optimization criteria & \href{../works/Fahimi16.pdf}{Yes} & \cite{Fahimi16} & 2016 & Universit{\'{e}} Laval, Quebec, Canada & 120 & 0 & 0 & \ref{b:Fahimi16} & n/a\\
Malapert11 \href{https://tel.archives-ouvertes.fr/tel-00630122}{Malapert11} & \hyperref[auth:a82]{A. Malapert} & Techniques d'ordonnancement d'atelier et de fourn{\'{e}}es bas{\'{e}}es sur la programmation par contraintes. (Shop and batch scheduling with constraints) & \href{../works/Malapert11.pdf}{Yes} & \cite{Malapert11} & 2011 & {\'{E}}cole des mines de Nantes, France & 194 & 0 & 0 & \ref{b:Malapert11} & n/a\\
Dejemeppe16 \href{https://hdl.handle.net/2078.1/178078}{Dejemeppe16} & \hyperref[auth:a207]{C. Dejemeppe} & Constraint programming algorithms and models for scheduling applications & \href{../works/Dejemeppe16.pdf}{Yes} & \cite{Dejemeppe16} & 2016 & Catholic University of Louvain, Louvain-la-Neuve, Belgium & 274 & 0 & 0 & \ref{b:Dejemeppe16} & n/a\\
Lombardi10 \href{http://amsdottorato.unibo.it/2961/}{Lombardi10} & \hyperref[auth:a143]{M. Lombardi} & Hybrid Methods for Resource Allocation and Scheduling Problems in Deterministic and Stochastic Environments & \href{../works/Lombardi10.pdf}{Yes} & \cite{Lombardi10} & 2010 & University of Bologna, Italy & 175 & 0 & 0 & \ref{b:Lombardi10} & n/a\\
Lombardi10 \href{http://amsdottorato.unibo.it/2961/}{Lombardi10} & \hyperref[auth:a143]{M. Lombardi} & Hybrid Methods for Resource Allocation and Scheduling Problems in Deterministic and Stochastic Environments & \href{../works/Lombardi10.pdf}{Yes} & \cite{Lombardi10} & 2010 & University of Bologna, Italy & 175 & 0 & 0 & \ref{b:Lombardi10} & n/a\\
ZarandiASC20 \href{https://doi.org/10.1007/s10462-018-9667-6}{ZarandiASC20} & \hyperref[auth:a832]{Mohammad Hossein Fazel Zarandi}, \hyperref[auth:a833]{Ali Akbar Sadat Asl}, \hyperref[auth:a834]{S. Sotudian}, \hyperref[auth:a835]{O. Castillo} & A state of the art review of intelligent scheduling & \href{../works/ZarandiASC20.pdf}{Yes} & \cite{ZarandiASC20} & 2020 & Artif. Intell. Rev. & 93 & 55 & 445 & \ref{b:ZarandiASC20} & n/a\\
\end{longtable}
}



{\scriptsize
\begin{longtable}{>{\raggedright\arraybackslash}p{3cm}r>{\raggedright\arraybackslash}p{1.5cm}>{\raggedright\arraybackslash}p{1.5cm}>{\raggedright\arraybackslash}p{1.5cm}>{\raggedright\arraybackslash}p{1.5cm}>{\raggedright\arraybackslash}p{1.5cm}>{\raggedright\arraybackslash}p{1.5cm}>{\raggedright\arraybackslash}p{1.5cm}>{\raggedright\arraybackslash}p{1.5cm}>{\raggedright\arraybackslash}p{1.5cm}>{\raggedright\arraybackslash}p{1.5cm}rr}
\rowcolor{white}\caption{Features of Works Similar by Dot Product}\\ \toprule
\rowcolor{white}Work & Pages & Algorithms& ApplicationAreas& Benchmarks& CP& CPSystems& Classification& Concepts& Constraints& Industries& Scheduling & a & c\\ \midrule\endhead
\bottomrule
\endfoot
\href{../works/Groleaz21.pdf}{Groleaz21}~\cite{Groleaz21} & 153 & mat heuristic, evolutionary computing, memetic algorithm, meta heuristic, swarm intelligence, neural network, edge-finding, column generation, machine learning, simulated annealing, genetic algorithm, not-first, large neighborhood search, not-last, ant colony & dairy, robot, automotive, business process & benchmark, real-life & constraint satisfaction, COP, propagation, constraint programming, CSP, CP & Choco Solver, OR-Tools, SCIP, Cplex, Z3, OPL, Gurobi, CPO, Gecode & single machine, RCPSP, parallel machine, Resource-constrained Project Scheduling Problem, Open Shop Scheduling Problem, GCSP, OSP & preempt, setup-time, breakdown, release-date, periodic, single-machine scheduling, make-span, bi-objective, reactive scheduling, preemptive, sequence dependent setup, due-date, flow-shop, cmax, explanation, lateness, re-scheduling, stochastic, precedence, inventory, tardiness, earliness, completion-time, online scheduling, distributed, job-shop, transportation, open-shop & circuit, disjunctive, Disjunctive constraint, span constraint, cumulative, cycle, noOverlap & food industry, dairy industry, agrifood industry & activity, scheduling, task, machine, resource, job, order & \ref{a:Groleaz21} & n/a\\
\href{../works/ZarandiASC20.pdf}{ZarandiASC20}~\cite{ZarandiASC20} & 93 & memetic algorithm, column generation, max-flow, time-tabling, neural network, meta heuristic, ant colony, simulated annealing, genetic algorithm, reinforcement learning, particle swarm, machine learning, Lagrangian relaxation, swarm intelligence & satellite, robot, sports scheduling, surgery, medical, round-robin, railway, business process, container terminal, nurse, semiconductor, tournament, evacuation, drone, crew-scheduling, train schedule, maintenance scheduling, aircraft, operating room, airport & real-world, benchmark, real-life & propagation, constraint satisfaction, CP, CSP, constraint logic programming, constraint programming, CLP & OPL & HFS, parallel machine, OSSP, JSSP, Resource-constrained Project Scheduling Problem, Open Shop Scheduling Problem, PMSP, RCPSP, single machine, FJS, Resource-constrained Project Scheduling Problem with Discounted Cashflow & tardiness, batch process, multi-agent, completion-time, due-date, flow-shop, re-scheduling, open-shop, make-span, energy efficiency, multi-objective, breakdown, explanation, setup-time, preempt, single-machine scheduling, inventory, bi-objective, distributed, lateness, no-wait, two-stage scheduling, net present value, one-machine scheduling, cmax, stochastic, reactive scheduling, flow-time, preemptive, Pareto, release-date, precedence, earliness, sequence dependent setup, job-shop, transportation, periodic, Infeasible & disjunctive, cycle & textile industry, gas industry & activity, scheduling, machine, job, order, resource, task & \ref{a:ZarandiASC20} & n/a\\
\href{../works/Dejemeppe16.pdf}{Dejemeppe16}~\cite{Dejemeppe16} & 274 & Lagrangian relaxation, simulated annealing, ant colony, not-last, particle swarm, sweep, large neighborhood search, not-first, meta heuristic, edge-finding, genetic algorithm & super-computer, nurse, robot, container terminal, medical, patient, tournament, physician & benchmark, instance generator, industrial partner, real-world, bitbucket, generated instance, random instance & constraint programming, COP, CP, propagation, constraint satisfaction, CSP, constraint optimization & OPL, Gecode, OR-Tools, Ilog Solver, CHIP, CPO & single machine, PTC, psplib, Resource-constrained Project Scheduling Problem, RCPSP & Pareto, explanation, release-date, flow-shop, batch process, multi-objective, energy efficiency, preemptive, completion-time, setup-time, earliness, stochastic, lateness, bi-objective, precedence, sequence dependent setup, make-span, open-shop, continuous-process, preempt, tardiness, re-scheduling, due-date, no-wait, job-shop, Infeasible, Over-constrained, Unsatisfiable & disjunctive, Cumulatives constraint, GCC constraint, circuit, Cardinality constraint, Regular constraint, cumulative, Element constraint, Reified constraint, alldifferent, cycle, Disjunctive constraint & paper industry & job, activity, task, order, machine, scheduling, resource & \ref{a:Dejemeppe16} & n/a\\
\href{../works/ZarandiASC20.pdf}{ZarandiASC20}~\cite{ZarandiASC20} & 93 & memetic algorithm, column generation, max-flow, time-tabling, neural network, meta heuristic, ant colony, simulated annealing, genetic algorithm, reinforcement learning, particle swarm, machine learning, Lagrangian relaxation, swarm intelligence & satellite, robot, sports scheduling, surgery, medical, round-robin, railway, business process, container terminal, nurse, semiconductor, tournament, evacuation, drone, crew-scheduling, train schedule, maintenance scheduling, aircraft, operating room, airport & real-world, benchmark, real-life & propagation, constraint satisfaction, CP, CSP, constraint logic programming, constraint programming, CLP & OPL & HFS, parallel machine, OSSP, JSSP, Resource-constrained Project Scheduling Problem, Open Shop Scheduling Problem, PMSP, RCPSP, single machine, FJS, Resource-constrained Project Scheduling Problem with Discounted Cashflow & tardiness, batch process, multi-agent, completion-time, due-date, flow-shop, re-scheduling, open-shop, make-span, energy efficiency, multi-objective, breakdown, explanation, setup-time, preempt, single-machine scheduling, inventory, bi-objective, distributed, lateness, no-wait, two-stage scheduling, net present value, one-machine scheduling, cmax, stochastic, reactive scheduling, flow-time, preemptive, Pareto, release-date, precedence, earliness, sequence dependent setup, job-shop, transportation, periodic, Infeasible & disjunctive, cycle & textile industry, gas industry & activity, scheduling, machine, job, order, resource, task & \ref{a:ZarandiASC20} & n/a\\
\href{../works/Dejemeppe16.pdf}{Dejemeppe16}~\cite{Dejemeppe16} & 274 & Lagrangian relaxation, simulated annealing, ant colony, not-last, particle swarm, sweep, large neighborhood search, not-first, meta heuristic, edge-finding, genetic algorithm & super-computer, nurse, robot, container terminal, medical, patient, tournament, physician & benchmark, instance generator, industrial partner, real-world, bitbucket, generated instance, random instance & constraint programming, COP, CP, propagation, constraint satisfaction, CSP, constraint optimization & OPL, Gecode, OR-Tools, Ilog Solver, CHIP, CPO & single machine, PTC, psplib, Resource-constrained Project Scheduling Problem, RCPSP & Pareto, explanation, release-date, flow-shop, batch process, multi-objective, energy efficiency, preemptive, completion-time, setup-time, earliness, stochastic, lateness, bi-objective, precedence, sequence dependent setup, make-span, open-shop, continuous-process, preempt, tardiness, re-scheduling, due-date, no-wait, job-shop, Infeasible, Over-constrained, Unsatisfiable & disjunctive, Cumulatives constraint, GCC constraint, circuit, Cardinality constraint, Regular constraint, cumulative, Element constraint, Reified constraint, alldifferent, cycle, Disjunctive constraint & paper industry & job, activity, task, order, machine, scheduling, resource & \ref{a:Dejemeppe16} & n/a\\
\href{../works/Groleaz21.pdf}{Groleaz21}~\cite{Groleaz21} & 153 & mat heuristic, evolutionary computing, memetic algorithm, meta heuristic, swarm intelligence, neural network, edge-finding, column generation, machine learning, simulated annealing, genetic algorithm, not-first, large neighborhood search, not-last, ant colony & dairy, robot, automotive, business process & benchmark, real-life & constraint satisfaction, COP, propagation, constraint programming, CSP, CP & Choco Solver, OR-Tools, SCIP, Cplex, Z3, OPL, Gurobi, CPO, Gecode & single machine, RCPSP, parallel machine, Resource-constrained Project Scheduling Problem, Open Shop Scheduling Problem, GCSP, OSP & preempt, setup-time, breakdown, release-date, periodic, single-machine scheduling, make-span, bi-objective, reactive scheduling, preemptive, sequence dependent setup, due-date, flow-shop, cmax, explanation, lateness, re-scheduling, stochastic, precedence, inventory, tardiness, earliness, completion-time, online scheduling, distributed, job-shop, transportation, open-shop & circuit, disjunctive, Disjunctive constraint, span constraint, cumulative, cycle, noOverlap & food industry, dairy industry, agrifood industry & activity, scheduling, task, machine, resource, job, order & \ref{a:Groleaz21} & n/a\\
\href{../works/Baptiste02.pdf}{Baptiste02}~\cite{Baptiste02} & 237 & column generation, not-last, simulated annealing, edge-finding, genetic algorithm, not-first, Lagrangian relaxation, energetic reasoning & hoist & real-life, generated instance, benchmark & CSP, CLP, constraint satisfaction, propagation, constraint logic programming, constraint programming, CP & Choco Solver, Ilog Solver, OPL, ECLiPSe, Claire, CHIP, Ilog Scheduler, Z3 & single machine, OSSP, Open Shop Scheduling Problem, PJSSP, HFS, RCPSP, parallel machine, Resource-constrained Project Scheduling Problem, JSSP & re-scheduling, release-date, Pareto, preempt, make-span, distributed, no preempt, due-date, tardiness, lateness, earliness, sequence dependent setup, flow-time, preemptive, job-shop, reactive scheduling, flow-shop, completion-time, precedence, inventory, setup-time, single-machine scheduling, open-shop, one-machine scheduling, cmax & cumulative, circuit, disjunctive, Cardinality constraint, alternative constraint, Arithmetic constraint, Disjunctive constraint, table constraint &  & scheduling, task, machine, resource, activity, job, order & \ref{a:Baptiste02} & n/a\\
\href{../works/ZarandiASC20.pdf}{ZarandiASC20}~\cite{ZarandiASC20} & 93 & memetic algorithm, column generation, max-flow, time-tabling, neural network, meta heuristic, ant colony, simulated annealing, genetic algorithm, reinforcement learning, particle swarm, machine learning, Lagrangian relaxation, swarm intelligence & satellite, robot, sports scheduling, surgery, medical, round-robin, railway, business process, container terminal, nurse, semiconductor, tournament, evacuation, drone, crew-scheduling, train schedule, maintenance scheduling, aircraft, operating room, airport & real-world, benchmark, real-life & propagation, constraint satisfaction, CP, CSP, constraint logic programming, constraint programming, CLP & OPL & HFS, parallel machine, OSSP, JSSP, Resource-constrained Project Scheduling Problem, Open Shop Scheduling Problem, PMSP, RCPSP, single machine, FJS, Resource-constrained Project Scheduling Problem with Discounted Cashflow & tardiness, batch process, multi-agent, completion-time, due-date, flow-shop, re-scheduling, open-shop, make-span, energy efficiency, multi-objective, breakdown, explanation, setup-time, preempt, single-machine scheduling, inventory, bi-objective, distributed, lateness, no-wait, two-stage scheduling, net present value, one-machine scheduling, cmax, stochastic, reactive scheduling, flow-time, preemptive, Pareto, release-date, precedence, earliness, sequence dependent setup, job-shop, transportation, periodic, Infeasible & disjunctive, cycle & textile industry, gas industry & activity, scheduling, machine, job, order, resource, task & \ref{a:ZarandiASC20} & n/a\\
\href{../works/Baptiste02.pdf}{Baptiste02}~\cite{Baptiste02} & 237 & column generation, not-last, simulated annealing, edge-finding, genetic algorithm, not-first, Lagrangian relaxation, energetic reasoning & hoist & real-life, generated instance, benchmark & CSP, CLP, constraint satisfaction, propagation, constraint logic programming, constraint programming, CP & Choco Solver, Ilog Solver, OPL, ECLiPSe, Claire, CHIP, Ilog Scheduler, Z3 & single machine, OSSP, Open Shop Scheduling Problem, PJSSP, HFS, RCPSP, parallel machine, Resource-constrained Project Scheduling Problem, JSSP & re-scheduling, release-date, Pareto, preempt, make-span, distributed, no preempt, due-date, tardiness, lateness, earliness, sequence dependent setup, flow-time, preemptive, job-shop, reactive scheduling, flow-shop, completion-time, precedence, inventory, setup-time, single-machine scheduling, open-shop, one-machine scheduling, cmax & cumulative, circuit, disjunctive, Cardinality constraint, alternative constraint, Arithmetic constraint, Disjunctive constraint, table constraint &  & scheduling, task, machine, resource, activity, job, order & \ref{a:Baptiste02} & n/a\\
\href{../works/Groleaz21.pdf}{Groleaz21}~\cite{Groleaz21} & 153 & mat heuristic, evolutionary computing, memetic algorithm, meta heuristic, swarm intelligence, neural network, edge-finding, column generation, machine learning, simulated annealing, genetic algorithm, not-first, large neighborhood search, not-last, ant colony & dairy, robot, automotive, business process & benchmark, real-life & constraint satisfaction, COP, propagation, constraint programming, CSP, CP & Choco Solver, OR-Tools, SCIP, Cplex, Z3, OPL, Gurobi, CPO, Gecode & single machine, RCPSP, parallel machine, Resource-constrained Project Scheduling Problem, Open Shop Scheduling Problem, GCSP, OSP & preempt, setup-time, breakdown, release-date, periodic, single-machine scheduling, make-span, bi-objective, reactive scheduling, preemptive, sequence dependent setup, due-date, flow-shop, cmax, explanation, lateness, re-scheduling, stochastic, precedence, inventory, tardiness, earliness, completion-time, online scheduling, distributed, job-shop, transportation, open-shop & circuit, disjunctive, Disjunctive constraint, span constraint, cumulative, cycle, noOverlap & food industry, dairy industry, agrifood industry & activity, scheduling, task, machine, resource, job, order & \ref{a:Groleaz21} & n/a\\
\href{../works/Lunardi20.pdf}{Lunardi20}~\cite{Lunardi20} & 181 & particle swarm, ant colony, mat heuristic, memetic algorithm, meta heuristic, machine learning, simulated annealing, genetic algorithm, swarm intelligence, neural network, reinforcement learning & high performance computing, robot, airport, tournament & industrial partner, instance generator, benchmark, random instance, github, supplementary material, real-world, real-life, generated instance & CP, constraint programming, constraint satisfaction & CPO, OPL, Cplex & parallel machine, FJS, single machine & reactive scheduling, unavailability, cmax, lateness, re-scheduling, stochastic, no preempt, job-shop, transportation, open-shop, completion-time, multi-objective, setup-time, breakdown, Pareto, release-date, make-span, bi-objective, due-date, batch process, preempt, flow-shop, explanation, precedence, tardiness, Infeasible & cycle, endBeforeStart, noOverlap, alldifferent, disjunctive & printing industry, glass industry & machine, resource, order, activity, scheduling, task, job & \ref{a:Lunardi20} & n/a\\
\href{../works/ZarandiASC20.pdf}{ZarandiASC20}~\cite{ZarandiASC20} & 93 & memetic algorithm, column generation, max-flow, time-tabling, neural network, meta heuristic, ant colony, simulated annealing, genetic algorithm, reinforcement learning, particle swarm, machine learning, Lagrangian relaxation, swarm intelligence & satellite, robot, sports scheduling, surgery, medical, round-robin, railway, business process, container terminal, nurse, semiconductor, tournament, evacuation, drone, crew-scheduling, train schedule, maintenance scheduling, aircraft, operating room, airport & real-world, benchmark, real-life & propagation, constraint satisfaction, CP, CSP, constraint logic programming, constraint programming, CLP & OPL & HFS, parallel machine, OSSP, JSSP, Resource-constrained Project Scheduling Problem, Open Shop Scheduling Problem, PMSP, RCPSP, single machine, FJS, Resource-constrained Project Scheduling Problem with Discounted Cashflow & tardiness, batch process, multi-agent, completion-time, due-date, flow-shop, re-scheduling, open-shop, make-span, energy efficiency, multi-objective, breakdown, explanation, setup-time, preempt, single-machine scheduling, inventory, bi-objective, distributed, lateness, no-wait, two-stage scheduling, net present value, one-machine scheduling, cmax, stochastic, reactive scheduling, flow-time, preemptive, Pareto, release-date, precedence, earliness, sequence dependent setup, job-shop, transportation, periodic, Infeasible & disjunctive, cycle & textile industry, gas industry & activity, scheduling, machine, job, order, resource, task & \ref{a:ZarandiASC20} & n/a\\
\href{../works/Dejemeppe16.pdf}{Dejemeppe16}~\cite{Dejemeppe16} & 274 & Lagrangian relaxation, simulated annealing, ant colony, not-last, particle swarm, sweep, large neighborhood search, not-first, meta heuristic, edge-finding, genetic algorithm & super-computer, nurse, robot, container terminal, medical, patient, tournament, physician & benchmark, instance generator, industrial partner, real-world, bitbucket, generated instance, random instance & constraint programming, COP, CP, propagation, constraint satisfaction, CSP, constraint optimization & OPL, Gecode, OR-Tools, Ilog Solver, CHIP, CPO & single machine, PTC, psplib, Resource-constrained Project Scheduling Problem, RCPSP & Pareto, explanation, release-date, flow-shop, batch process, multi-objective, energy efficiency, preemptive, completion-time, setup-time, earliness, stochastic, lateness, bi-objective, precedence, sequence dependent setup, make-span, open-shop, continuous-process, preempt, tardiness, re-scheduling, due-date, no-wait, job-shop, Infeasible, Over-constrained, Unsatisfiable & disjunctive, Cumulatives constraint, GCC constraint, circuit, Cardinality constraint, Regular constraint, cumulative, Element constraint, Reified constraint, alldifferent, cycle, Disjunctive constraint & paper industry & job, activity, task, order, machine, scheduling, resource & \ref{a:Dejemeppe16} & n/a\\
\href{../works/Malapert11.pdf}{Malapert11}~\cite{Malapert11} & 194 & edge-finding, genetic algorithm, not-first, ant colony, energetic reasoning, time-tabling, particle swarm, column generation, not-last, meta heuristic, sweep & robot, semiconductor, rectangle-packing, maintenance scheduling, patient & real-world, industrial partner, generated instance, benchmark & COP, constraint satisfaction, constraint programming, propagation, CP, CSP, CLP & Mistral, Claire, ECLiPSe, SICStus, Cplex, CHIP, Ilog Scheduler, Choco Solver, Gecode, OPL & Open Shop Scheduling Problem, single machine & tardiness, lateness, preempt, batch process, flow-time, preemptive, job-shop, no-wait, flow-shop, completion-time, precedence, planned maintenance, inventory, setup-time, open-shop, cmax, multi-objective, transportation, make-span, due-date, Infeasible & diffn, cycle, alldifferent, Element constraint, bin-packing, Disjunctive constraint, cumulative, circuit, disjunctive, geost, Cumulatives constraint &  & resource, activity, job, order, scheduling, task, machine & \ref{a:Malapert11} & n/a\\
\href{../works/Astrand21.pdf}{Astrand21}~\cite{Astrand21} & 142 & time-tabling, not-first, large neighborhood search, not-last, meta heuristic, neural network, reinforcement learning, edge-finding, simulated annealing, genetic algorithm, NEH & satellite, agriculture, semiconductor, drone, robot & generated instance, benchmark, real-world, real-life & propagation, constraint programming, CSP, CP, constraint satisfaction & Cplex, OPL, Gecode & RCPSP, parallel machine, Resource-constrained Project Scheduling Problem, Partial Order Schedule, HFS, single machine & sequence dependent setup, due-date, flow-shop, net present value, re-scheduling, stochastic, precedence, inventory, two-stage scheduling, tardiness, completion-time, multi-objective, distributed, one-machine scheduling, job-shop, transportation, open-shop, preempt, setup-time, breakdown, release-date, planned maintenance, periodic, make-span, unavailability, Infeasible & cumulative, alldifferent, cycle, Reified constraint, circuit, disjunctive, Disjunctive constraint & potash industry, mineral industry, mining industry, maritime industry, shipping industry & machine, resource, job, order, activity, scheduling, task & \ref{a:Astrand21} & n/a\\
\href{../works/ZarandiASC20.pdf}{ZarandiASC20}~\cite{ZarandiASC20} & 93 & memetic algorithm, column generation, max-flow, time-tabling, neural network, meta heuristic, ant colony, simulated annealing, genetic algorithm, reinforcement learning, particle swarm, machine learning, Lagrangian relaxation, swarm intelligence & satellite, robot, sports scheduling, surgery, medical, round-robin, railway, business process, container terminal, nurse, semiconductor, tournament, evacuation, drone, crew-scheduling, train schedule, maintenance scheduling, aircraft, operating room, airport & real-world, benchmark, real-life & propagation, constraint satisfaction, CP, CSP, constraint logic programming, constraint programming, CLP & OPL & HFS, parallel machine, OSSP, JSSP, Resource-constrained Project Scheduling Problem, Open Shop Scheduling Problem, PMSP, RCPSP, single machine, FJS, Resource-constrained Project Scheduling Problem with Discounted Cashflow & tardiness, batch process, multi-agent, completion-time, due-date, flow-shop, re-scheduling, open-shop, make-span, energy efficiency, multi-objective, breakdown, explanation, setup-time, preempt, single-machine scheduling, inventory, bi-objective, distributed, lateness, no-wait, two-stage scheduling, net present value, one-machine scheduling, cmax, stochastic, reactive scheduling, flow-time, preemptive, Pareto, release-date, precedence, earliness, sequence dependent setup, job-shop, transportation, periodic, Infeasible & disjunctive, cycle & textile industry, gas industry & activity, scheduling, machine, job, order, resource, task & \ref{a:ZarandiASC20} & n/a\\
\href{../works/Siala15.pdf}{Siala15}~\cite{Siala15} & 2 & edge-finding, GRASP, time-tabling & rectangle-packing, automotive & github, Roadef, CSPlib, real-world, benchmark, random instance & propagation, CP, constraint programming & Ilog Solver, CHIP, Claire, OPL, Mistral & single machine, OSP, RCPSP, TMS & sequence dependent setup, setup-time, open-shop, precedence, cmax, job-shop, explanation, due-date, earliness, tardiness, make-span & AmongSeq constraint, circuit, alldifferent, Balance constraint, cumulative, table constraint, GCC constraint, AtMostSeqCard, Reified constraint, Regular constraint, Among constraint, Atmost constraint, Disjunctive constraint, Cardinality constraint, cycle, MultiAtMostSeqCard, disjunctive, CardPath, AtMostSeq &  & machine, activity, job, order, scheduling, task, resource & \ref{a:Siala15} & \ref{c:Siala15}\\
\href{../works/Siala15a.pdf}{Siala15a}~\cite{Siala15a} & 199 & evolutionary computing, edge-finding, swarm intelligence, conflict-driven clause learning, lazy clause generation, time-tabling, large neighborhood search, ant colony, GRASP & rectangle-packing, automotive & Roadef, real-world, github, benchmark, random instance, CSPlib & constraint programming, CP, constraint logic programming, COP, CSP, CLP, constraint satisfaction, propagation & Ilog Solver, Mistral, Claire, CHIP, OPL & OSP, RCPSP, single machine, TMS & job-shop, explanation, earliness, setup-time, make-span, precedence, cmax, sequence dependent setup, due-date, tardiness, open-shop, periodic, Infeasible, Unsatisfiable & Among constraint, AmongSeq constraint, Atmost constraint, Disjunctive constraint, GCC constraint, CardPath, AtMostSeqCard, AtMostSeq, table constraint, Balance constraint, cumulative, circuit, disjunctive, Regular constraint, Cardinality constraint, MultiAtMostSeqCard, Reified constraint, alldifferent, cycle &  & activity, machine, job, resource, order, scheduling, task & \ref{a:Siala15a} & n/a\\
\href{../works/Baptiste02.pdf}{Baptiste02}~\cite{Baptiste02} & 237 & column generation, not-last, simulated annealing, edge-finding, genetic algorithm, not-first, Lagrangian relaxation, energetic reasoning & hoist & real-life, generated instance, benchmark & CSP, CLP, constraint satisfaction, propagation, constraint logic programming, constraint programming, CP & Choco Solver, Ilog Solver, OPL, ECLiPSe, Claire, CHIP, Ilog Scheduler, Z3 & single machine, OSSP, Open Shop Scheduling Problem, PJSSP, HFS, RCPSP, parallel machine, Resource-constrained Project Scheduling Problem, JSSP & re-scheduling, release-date, Pareto, preempt, make-span, distributed, no preempt, due-date, tardiness, lateness, earliness, sequence dependent setup, flow-time, preemptive, job-shop, reactive scheduling, flow-shop, completion-time, precedence, inventory, setup-time, single-machine scheduling, open-shop, one-machine scheduling, cmax & cumulative, circuit, disjunctive, Cardinality constraint, alternative constraint, Arithmetic constraint, Disjunctive constraint, table constraint &  & scheduling, task, machine, resource, activity, job, order & \ref{a:Baptiste02} & n/a\\
\href{../works/Dejemeppe16.pdf}{Dejemeppe16}~\cite{Dejemeppe16} & 274 & Lagrangian relaxation, simulated annealing, ant colony, not-last, particle swarm, sweep, large neighborhood search, not-first, meta heuristic, edge-finding, genetic algorithm & super-computer, nurse, robot, container terminal, medical, patient, tournament, physician & benchmark, instance generator, industrial partner, real-world, bitbucket, generated instance, random instance & constraint programming, COP, CP, propagation, constraint satisfaction, CSP, constraint optimization & OPL, Gecode, OR-Tools, Ilog Solver, CHIP, CPO & single machine, PTC, psplib, Resource-constrained Project Scheduling Problem, RCPSP & Pareto, explanation, release-date, flow-shop, batch process, multi-objective, energy efficiency, preemptive, completion-time, setup-time, earliness, stochastic, lateness, bi-objective, precedence, sequence dependent setup, make-span, open-shop, continuous-process, preempt, tardiness, re-scheduling, due-date, no-wait, job-shop, Infeasible, Over-constrained, Unsatisfiable & disjunctive, Cumulatives constraint, GCC constraint, circuit, Cardinality constraint, Regular constraint, cumulative, Element constraint, Reified constraint, alldifferent, cycle, Disjunctive constraint & paper industry & job, activity, task, order, machine, scheduling, resource & \ref{a:Dejemeppe16} & n/a\\
\href{../works/Baptiste02.pdf}{Baptiste02}~\cite{Baptiste02} & 237 & column generation, not-last, simulated annealing, edge-finding, genetic algorithm, not-first, Lagrangian relaxation, energetic reasoning & hoist & real-life, generated instance, benchmark & CSP, CLP, constraint satisfaction, propagation, constraint logic programming, constraint programming, CP & Choco Solver, Ilog Solver, OPL, ECLiPSe, Claire, CHIP, Ilog Scheduler, Z3 & single machine, OSSP, Open Shop Scheduling Problem, PJSSP, HFS, RCPSP, parallel machine, Resource-constrained Project Scheduling Problem, JSSP & re-scheduling, release-date, Pareto, preempt, make-span, distributed, no preempt, due-date, tardiness, lateness, earliness, sequence dependent setup, flow-time, preemptive, job-shop, reactive scheduling, flow-shop, completion-time, precedence, inventory, setup-time, single-machine scheduling, open-shop, one-machine scheduling, cmax & cumulative, circuit, disjunctive, Cardinality constraint, alternative constraint, Arithmetic constraint, Disjunctive constraint, table constraint &  & scheduling, task, machine, resource, activity, job, order & \ref{a:Baptiste02} & n/a\\
\href{../works/Malapert11.pdf}{Malapert11}~\cite{Malapert11} & 194 & edge-finding, genetic algorithm, not-first, ant colony, energetic reasoning, time-tabling, particle swarm, column generation, not-last, meta heuristic, sweep & robot, semiconductor, rectangle-packing, maintenance scheduling, patient & real-world, industrial partner, generated instance, benchmark & COP, constraint satisfaction, constraint programming, propagation, CP, CSP, CLP & Mistral, Claire, ECLiPSe, SICStus, Cplex, CHIP, Ilog Scheduler, Choco Solver, Gecode, OPL & Open Shop Scheduling Problem, single machine & tardiness, lateness, preempt, batch process, flow-time, preemptive, job-shop, no-wait, flow-shop, completion-time, precedence, planned maintenance, inventory, setup-time, open-shop, cmax, multi-objective, transportation, make-span, due-date, Infeasible & diffn, cycle, alldifferent, Element constraint, bin-packing, Disjunctive constraint, cumulative, circuit, disjunctive, geost, Cumulatives constraint &  & resource, activity, job, order, scheduling, task, machine & \ref{a:Malapert11} & n/a\\
\href{../works/Astrand21.pdf}{Astrand21}~\cite{Astrand21} & 142 & time-tabling, not-first, large neighborhood search, not-last, meta heuristic, neural network, reinforcement learning, edge-finding, simulated annealing, genetic algorithm, NEH & satellite, agriculture, semiconductor, drone, robot & generated instance, benchmark, real-world, real-life & propagation, constraint programming, CSP, CP, constraint satisfaction & Cplex, OPL, Gecode & RCPSP, parallel machine, Resource-constrained Project Scheduling Problem, Partial Order Schedule, HFS, single machine & sequence dependent setup, due-date, flow-shop, net present value, re-scheduling, stochastic, precedence, inventory, two-stage scheduling, tardiness, completion-time, multi-objective, distributed, one-machine scheduling, job-shop, transportation, open-shop, preempt, setup-time, breakdown, release-date, planned maintenance, periodic, make-span, unavailability, Infeasible & cumulative, alldifferent, cycle, Reified constraint, circuit, disjunctive, Disjunctive constraint & potash industry, mineral industry, mining industry, maritime industry, shipping industry & machine, resource, job, order, activity, scheduling, task & \ref{a:Astrand21} & n/a\\
\href{../works/Groleaz21.pdf}{Groleaz21}~\cite{Groleaz21} & 153 & mat heuristic, evolutionary computing, memetic algorithm, meta heuristic, swarm intelligence, neural network, edge-finding, column generation, machine learning, simulated annealing, genetic algorithm, not-first, large neighborhood search, not-last, ant colony & dairy, robot, automotive, business process & benchmark, real-life & constraint satisfaction, COP, propagation, constraint programming, CSP, CP & Choco Solver, OR-Tools, SCIP, Cplex, Z3, OPL, Gurobi, CPO, Gecode & single machine, RCPSP, parallel machine, Resource-constrained Project Scheduling Problem, Open Shop Scheduling Problem, GCSP, OSP & preempt, setup-time, breakdown, release-date, periodic, single-machine scheduling, make-span, bi-objective, reactive scheduling, preemptive, sequence dependent setup, due-date, flow-shop, cmax, explanation, lateness, re-scheduling, stochastic, precedence, inventory, tardiness, earliness, completion-time, online scheduling, distributed, job-shop, transportation, open-shop & circuit, disjunctive, Disjunctive constraint, span constraint, cumulative, cycle, noOverlap & food industry, dairy industry, agrifood industry & activity, scheduling, task, machine, resource, job, order & \ref{a:Groleaz21} & n/a\\
\href{../works/Groleaz21.pdf}{Groleaz21}~\cite{Groleaz21} & 153 & mat heuristic, evolutionary computing, memetic algorithm, meta heuristic, swarm intelligence, neural network, edge-finding, column generation, machine learning, simulated annealing, genetic algorithm, not-first, large neighborhood search, not-last, ant colony & dairy, robot, automotive, business process & benchmark, real-life & constraint satisfaction, COP, propagation, constraint programming, CSP, CP & Choco Solver, OR-Tools, SCIP, Cplex, Z3, OPL, Gurobi, CPO, Gecode & single machine, RCPSP, parallel machine, Resource-constrained Project Scheduling Problem, Open Shop Scheduling Problem, GCSP, OSP & preempt, setup-time, breakdown, release-date, periodic, single-machine scheduling, make-span, bi-objective, reactive scheduling, preemptive, sequence dependent setup, due-date, flow-shop, cmax, explanation, lateness, re-scheduling, stochastic, precedence, inventory, tardiness, earliness, completion-time, online scheduling, distributed, job-shop, transportation, open-shop & circuit, disjunctive, Disjunctive constraint, span constraint, cumulative, cycle, noOverlap & food industry, dairy industry, agrifood industry & activity, scheduling, task, machine, resource, job, order & \ref{a:Groleaz21} & n/a\\
\href{../works/Lombardi10.pdf}{Lombardi10}~\cite{Lombardi10} & 175 & large neighborhood search, column generation, machine learning, not-last, simulated annealing, lazy clause generation, meta heuristic, sweep, edge-finder, edge-finding, energetic reasoning, genetic algorithm, time-tabling, not-first, Lagrangian relaxation & pipeline, aircraft, semiconductor, business process, medical, automotive & generated instance, benchmark, instance generator, real-life, real-world & constraint programming, CSP, constraint logic programming, constraint satisfaction, propagation, CP & Ilog Solver, OPL, Cplex & single machine, SCC, Resource-constrained Project Scheduling Problem, CTW, TCSP, RCPSP & re-scheduling, precedence, Benders Decomposition, release-date, stochastic, setup-time, net present value, multi-objective, tardiness, preempt, make-span, periodic, distributed, Logic-Based Benders Decomposition, preemptive, job-shop, due-date, explanation, completion-time, inventory, energy efficiency, Infeasible, Over-constrained & Disjunctive constraint, AllDiff constraint, cumulative, disjunctive, span constraint, cycle, Balance constraint, table constraint, bin-packing, circuit & semiconductor industry & job, order, scheduling, task, machine, activity, resource & \ref{a:Lombardi10} & n/a\\
\href{../works/Groleaz21.pdf}{Groleaz21}~\cite{Groleaz21} & 153 & mat heuristic, evolutionary computing, memetic algorithm, meta heuristic, swarm intelligence, neural network, edge-finding, column generation, machine learning, simulated annealing, genetic algorithm, not-first, large neighborhood search, not-last, ant colony & dairy, robot, automotive, business process & benchmark, real-life & constraint satisfaction, COP, propagation, constraint programming, CSP, CP & Choco Solver, OR-Tools, SCIP, Cplex, Z3, OPL, Gurobi, CPO, Gecode & single machine, RCPSP, parallel machine, Resource-constrained Project Scheduling Problem, Open Shop Scheduling Problem, GCSP, OSP & preempt, setup-time, breakdown, release-date, periodic, single-machine scheduling, make-span, bi-objective, reactive scheduling, preemptive, sequence dependent setup, due-date, flow-shop, cmax, explanation, lateness, re-scheduling, stochastic, precedence, inventory, tardiness, earliness, completion-time, online scheduling, distributed, job-shop, transportation, open-shop & circuit, disjunctive, Disjunctive constraint, span constraint, cumulative, cycle, noOverlap & food industry, dairy industry, agrifood industry & activity, scheduling, task, machine, resource, job, order & \ref{a:Groleaz21} & n/a\\
\href{../works/Malapert11.pdf}{Malapert11}~\cite{Malapert11} & 194 & edge-finding, genetic algorithm, not-first, ant colony, energetic reasoning, time-tabling, particle swarm, column generation, not-last, meta heuristic, sweep & robot, semiconductor, rectangle-packing, maintenance scheduling, patient & real-world, industrial partner, generated instance, benchmark & COP, constraint satisfaction, constraint programming, propagation, CP, CSP, CLP & Mistral, Claire, ECLiPSe, SICStus, Cplex, CHIP, Ilog Scheduler, Choco Solver, Gecode, OPL & Open Shop Scheduling Problem, single machine & tardiness, lateness, preempt, batch process, flow-time, preemptive, job-shop, no-wait, flow-shop, completion-time, precedence, planned maintenance, inventory, setup-time, open-shop, cmax, multi-objective, transportation, make-span, due-date, Infeasible & diffn, cycle, alldifferent, Element constraint, bin-packing, Disjunctive constraint, cumulative, circuit, disjunctive, geost, Cumulatives constraint &  & resource, activity, job, order, scheduling, task, machine & \ref{a:Malapert11} & n/a\\
\href{../works/Groleaz21.pdf}{Groleaz21}~\cite{Groleaz21} & 153 & mat heuristic, evolutionary computing, memetic algorithm, meta heuristic, swarm intelligence, neural network, edge-finding, column generation, machine learning, simulated annealing, genetic algorithm, not-first, large neighborhood search, not-last, ant colony & dairy, robot, automotive, business process & benchmark, real-life & constraint satisfaction, COP, propagation, constraint programming, CSP, CP & Choco Solver, OR-Tools, SCIP, Cplex, Z3, OPL, Gurobi, CPO, Gecode & single machine, RCPSP, parallel machine, Resource-constrained Project Scheduling Problem, Open Shop Scheduling Problem, GCSP, OSP & preempt, setup-time, breakdown, release-date, periodic, single-machine scheduling, make-span, bi-objective, reactive scheduling, preemptive, sequence dependent setup, due-date, flow-shop, cmax, explanation, lateness, re-scheduling, stochastic, precedence, inventory, tardiness, earliness, completion-time, online scheduling, distributed, job-shop, transportation, open-shop & circuit, disjunctive, Disjunctive constraint, span constraint, cumulative, cycle, noOverlap & food industry, dairy industry, agrifood industry & activity, scheduling, task, machine, resource, job, order & \ref{a:Groleaz21} & n/a\\
\href{../works/Lunardi20.pdf}{Lunardi20}~\cite{Lunardi20} & 181 & particle swarm, ant colony, mat heuristic, memetic algorithm, meta heuristic, machine learning, simulated annealing, genetic algorithm, swarm intelligence, neural network, reinforcement learning & high performance computing, robot, airport, tournament & industrial partner, instance generator, benchmark, random instance, github, supplementary material, real-world, real-life, generated instance & CP, constraint programming, constraint satisfaction & CPO, OPL, Cplex & parallel machine, FJS, single machine & reactive scheduling, unavailability, cmax, lateness, re-scheduling, stochastic, no preempt, job-shop, transportation, open-shop, completion-time, multi-objective, setup-time, breakdown, Pareto, release-date, make-span, bi-objective, due-date, batch process, preempt, flow-shop, explanation, precedence, tardiness, Infeasible & cycle, endBeforeStart, noOverlap, alldifferent, disjunctive & printing industry, glass industry & machine, resource, order, activity, scheduling, task, job & \ref{a:Lunardi20} & n/a\\
\href{../works/Baptiste02.pdf}{Baptiste02}~\cite{Baptiste02} & 237 & column generation, not-last, simulated annealing, edge-finding, genetic algorithm, not-first, Lagrangian relaxation, energetic reasoning & hoist & real-life, generated instance, benchmark & CSP, CLP, constraint satisfaction, propagation, constraint logic programming, constraint programming, CP & Choco Solver, Ilog Solver, OPL, ECLiPSe, Claire, CHIP, Ilog Scheduler, Z3 & single machine, OSSP, Open Shop Scheduling Problem, PJSSP, HFS, RCPSP, parallel machine, Resource-constrained Project Scheduling Problem, JSSP & re-scheduling, release-date, Pareto, preempt, make-span, distributed, no preempt, due-date, tardiness, lateness, earliness, sequence dependent setup, flow-time, preemptive, job-shop, reactive scheduling, flow-shop, completion-time, precedence, inventory, setup-time, single-machine scheduling, open-shop, one-machine scheduling, cmax & cumulative, circuit, disjunctive, Cardinality constraint, alternative constraint, Arithmetic constraint, Disjunctive constraint, table constraint &  & scheduling, task, machine, resource, activity, job, order & \ref{a:Baptiste02} & n/a\\
\href{../works/Lombardi10.pdf}{Lombardi10}~\cite{Lombardi10} & 175 & large neighborhood search, column generation, machine learning, not-last, simulated annealing, lazy clause generation, meta heuristic, sweep, edge-finder, edge-finding, energetic reasoning, genetic algorithm, time-tabling, not-first, Lagrangian relaxation & pipeline, aircraft, semiconductor, business process, medical, automotive & generated instance, benchmark, instance generator, real-life, real-world & constraint programming, CSP, constraint logic programming, constraint satisfaction, propagation, CP & Ilog Solver, OPL, Cplex & single machine, SCC, Resource-constrained Project Scheduling Problem, CTW, TCSP, RCPSP & re-scheduling, precedence, Benders Decomposition, release-date, stochastic, setup-time, net present value, multi-objective, tardiness, preempt, make-span, periodic, distributed, Logic-Based Benders Decomposition, preemptive, job-shop, due-date, explanation, completion-time, inventory, energy efficiency, Infeasible, Over-constrained & Disjunctive constraint, AllDiff constraint, cumulative, disjunctive, span constraint, cycle, Balance constraint, table constraint, bin-packing, circuit & semiconductor industry & job, order, scheduling, task, machine, activity, resource & \ref{a:Lombardi10} & n/a\\
\href{../works/PrataAN23.pdf}{PrataAN23}~\cite{PrataAN23} & 17 & mat heuristic, memetic algorithm, meta heuristic, machine learning, genetic algorithm, reinforcement learning, time-tabling, large neighborhood search, particle swarm & dairy, robot, energy-price, aircraft & real-life, benchmark, real-world & CP, constraint programming & CHIP & single machine, Open Shop Scheduling Problem, parallel machine & multi-objective, setup-time, release-date, no-wait, single-machine scheduling, Logic-Based Benders Decomposition, make-span, bi-objective, order scheduling, sequence dependent setup, due-date, batch process, preempt, flow-shop, precedence, tardiness, flow-time, earliness, preemptive, completion-time, energy efficiency, online scheduling, lateness, re-scheduling, stochastic, inventory, distributed, job-shop, open-shop, Benders Decomposition & circuit, cumulative & manufacturing industry & order, activity, scheduling, task, job, machine, resource & \ref{a:PrataAN23} & \ref{c:PrataAN23}\\
\href{../works/ZarandiASC20.pdf}{ZarandiASC20}~\cite{ZarandiASC20} & 93 & memetic algorithm, column generation, max-flow, time-tabling, neural network, meta heuristic, ant colony, simulated annealing, genetic algorithm, reinforcement learning, particle swarm, machine learning, Lagrangian relaxation, swarm intelligence & satellite, robot, sports scheduling, surgery, medical, round-robin, railway, business process, container terminal, nurse, semiconductor, tournament, evacuation, drone, crew-scheduling, train schedule, maintenance scheduling, aircraft, operating room, airport & real-world, benchmark, real-life & propagation, constraint satisfaction, CP, CSP, constraint logic programming, constraint programming, CLP & OPL & HFS, parallel machine, OSSP, JSSP, Resource-constrained Project Scheduling Problem, Open Shop Scheduling Problem, PMSP, RCPSP, single machine, FJS, Resource-constrained Project Scheduling Problem with Discounted Cashflow & tardiness, batch process, multi-agent, completion-time, due-date, flow-shop, re-scheduling, open-shop, make-span, energy efficiency, multi-objective, breakdown, explanation, setup-time, preempt, single-machine scheduling, inventory, bi-objective, distributed, lateness, no-wait, two-stage scheduling, net present value, one-machine scheduling, cmax, stochastic, reactive scheduling, flow-time, preemptive, Pareto, release-date, precedence, earliness, sequence dependent setup, job-shop, transportation, periodic, Infeasible & disjunctive, cycle & textile industry, gas industry & activity, scheduling, machine, job, order, resource, task & \ref{a:ZarandiASC20} & n/a\\
\href{../works/Dejemeppe16.pdf}{Dejemeppe16}~\cite{Dejemeppe16} & 274 & Lagrangian relaxation, simulated annealing, ant colony, not-last, particle swarm, sweep, large neighborhood search, not-first, meta heuristic, edge-finding, genetic algorithm & super-computer, nurse, robot, container terminal, medical, patient, tournament, physician & benchmark, instance generator, industrial partner, real-world, bitbucket, generated instance, random instance & constraint programming, COP, CP, propagation, constraint satisfaction, CSP, constraint optimization & OPL, Gecode, OR-Tools, Ilog Solver, CHIP, CPO & single machine, PTC, psplib, Resource-constrained Project Scheduling Problem, RCPSP & Pareto, explanation, release-date, flow-shop, batch process, multi-objective, energy efficiency, preemptive, completion-time, setup-time, earliness, stochastic, lateness, bi-objective, precedence, sequence dependent setup, make-span, open-shop, continuous-process, preempt, tardiness, re-scheduling, due-date, no-wait, job-shop, Infeasible, Over-constrained, Unsatisfiable & disjunctive, Cumulatives constraint, GCC constraint, circuit, Cardinality constraint, Regular constraint, cumulative, Element constraint, Reified constraint, alldifferent, cycle, Disjunctive constraint & paper industry & job, activity, task, order, machine, scheduling, resource & \ref{a:Dejemeppe16} & n/a\\
\href{../works/Lombardi10.pdf}{Lombardi10}~\cite{Lombardi10} & 175 & large neighborhood search, column generation, machine learning, not-last, simulated annealing, lazy clause generation, meta heuristic, sweep, edge-finder, edge-finding, energetic reasoning, genetic algorithm, time-tabling, not-first, Lagrangian relaxation & pipeline, aircraft, semiconductor, business process, medical, automotive & generated instance, benchmark, instance generator, real-life, real-world & constraint programming, CSP, constraint logic programming, constraint satisfaction, propagation, CP & Ilog Solver, OPL, Cplex & single machine, SCC, Resource-constrained Project Scheduling Problem, CTW, TCSP, RCPSP & re-scheduling, precedence, Benders Decomposition, release-date, stochastic, setup-time, net present value, multi-objective, tardiness, preempt, make-span, periodic, distributed, Logic-Based Benders Decomposition, preemptive, job-shop, due-date, explanation, completion-time, inventory, energy efficiency, Infeasible, Over-constrained & Disjunctive constraint, AllDiff constraint, cumulative, disjunctive, span constraint, cycle, Balance constraint, table constraint, bin-packing, circuit & semiconductor industry & job, order, scheduling, task, machine, activity, resource & \ref{a:Lombardi10} & n/a\\
\href{../works/Baptiste02.pdf}{Baptiste02}~\cite{Baptiste02} & 237 & column generation, not-last, simulated annealing, edge-finding, genetic algorithm, not-first, Lagrangian relaxation, energetic reasoning & hoist & real-life, generated instance, benchmark & CSP, CLP, constraint satisfaction, propagation, constraint logic programming, constraint programming, CP & Choco Solver, Ilog Solver, OPL, ECLiPSe, Claire, CHIP, Ilog Scheduler, Z3 & single machine, OSSP, Open Shop Scheduling Problem, PJSSP, HFS, RCPSP, parallel machine, Resource-constrained Project Scheduling Problem, JSSP & re-scheduling, release-date, Pareto, preempt, make-span, distributed, no preempt, due-date, tardiness, lateness, earliness, sequence dependent setup, flow-time, preemptive, job-shop, reactive scheduling, flow-shop, completion-time, precedence, inventory, setup-time, single-machine scheduling, open-shop, one-machine scheduling, cmax & cumulative, circuit, disjunctive, Cardinality constraint, alternative constraint, Arithmetic constraint, Disjunctive constraint, table constraint &  & scheduling, task, machine, resource, activity, job, order & \ref{a:Baptiste02} & n/a\\
\href{../works/Fahimi16.pdf}{Fahimi16}~\cite{Fahimi16} & 120 & time-tabling, energetic reasoning, sweep, not-first, not-last, edge-finding, max-flow & airport, aircraft & benchmark, random instance, Roadef, real-world & constraint satisfaction, propagation, CSP, CP, constraint logic programming, constraint programming & Choco Solver, Ilog Scheduler, Gecode, CHIP & single machine, CuSP, parallel machine, RCPSP & reactive scheduling, completion-time, flow-shop, open-shop, stochastic, periodic, job-shop, Logic-Based Benders Decomposition, preempt, precedence, batch process, setup-time, due-date, preemptive, make-span, lateness, transportation, sequence dependent setup, tardiness, Benders Decomposition, Unsatisfiable & Cardinality constraint, Cumulatives constraint, alldifferent, cycle, AllDiff constraint, Disjunctive constraint, cumulative, alternative constraint, disjunctive &  & task, activity, order, machine, job, resource, scheduling & \ref{a:Fahimi16} & n/a\\
\href{../works/Lombardi10.pdf}{Lombardi10}~\cite{Lombardi10} & 175 & large neighborhood search, column generation, machine learning, not-last, simulated annealing, lazy clause generation, meta heuristic, sweep, edge-finder, edge-finding, energetic reasoning, genetic algorithm, time-tabling, not-first, Lagrangian relaxation & pipeline, aircraft, semiconductor, business process, medical, automotive & generated instance, benchmark, instance generator, real-life, real-world & constraint programming, CSP, constraint logic programming, constraint satisfaction, propagation, CP & Ilog Solver, OPL, Cplex & single machine, SCC, Resource-constrained Project Scheduling Problem, CTW, TCSP, RCPSP & re-scheduling, precedence, Benders Decomposition, release-date, stochastic, setup-time, net present value, multi-objective, tardiness, preempt, make-span, periodic, distributed, Logic-Based Benders Decomposition, preemptive, job-shop, due-date, explanation, completion-time, inventory, energy efficiency, Infeasible, Over-constrained & Disjunctive constraint, AllDiff constraint, cumulative, disjunctive, span constraint, cycle, Balance constraint, table constraint, bin-packing, circuit & semiconductor industry & job, order, scheduling, task, machine, activity, resource & \ref{a:Lombardi10} & n/a\\
\href{../works/ZarandiASC20.pdf}{ZarandiASC20}~\cite{ZarandiASC20} & 93 & memetic algorithm, column generation, max-flow, time-tabling, neural network, meta heuristic, ant colony, simulated annealing, genetic algorithm, reinforcement learning, particle swarm, machine learning, Lagrangian relaxation, swarm intelligence & satellite, robot, sports scheduling, surgery, medical, round-robin, railway, business process, container terminal, nurse, semiconductor, tournament, evacuation, drone, crew-scheduling, train schedule, maintenance scheduling, aircraft, operating room, airport & real-world, benchmark, real-life & propagation, constraint satisfaction, CP, CSP, constraint logic programming, constraint programming, CLP & OPL & HFS, parallel machine, OSSP, JSSP, Resource-constrained Project Scheduling Problem, Open Shop Scheduling Problem, PMSP, RCPSP, single machine, FJS, Resource-constrained Project Scheduling Problem with Discounted Cashflow & tardiness, batch process, multi-agent, completion-time, due-date, flow-shop, re-scheduling, open-shop, make-span, energy efficiency, multi-objective, breakdown, explanation, setup-time, preempt, single-machine scheduling, inventory, bi-objective, distributed, lateness, no-wait, two-stage scheduling, net present value, one-machine scheduling, cmax, stochastic, reactive scheduling, flow-time, preemptive, Pareto, release-date, precedence, earliness, sequence dependent setup, job-shop, transportation, periodic, Infeasible & disjunctive, cycle & textile industry, gas industry & activity, scheduling, machine, job, order, resource, task & \ref{a:ZarandiASC20} & n/a\\
\end{longtable}
}



\clearpage
\subsection{Most Similar Works Based on Cosine Similarity}
As before, but now based on the cosine similarity measure. A high value means that works are similar.

{\scriptsize
\begin{longtable}{>{\raggedright\arraybackslash}p{3cm}>{\raggedright\arraybackslash}p{4.5cm}>{\raggedright\arraybackslash}p{6.0cm}rrrp{2.5cm}rp{1cm}p{1cm}rr}
\rowcolor{white}\caption{Works Similar by Cosine Similarity (Total 40)}\\ \toprule
\rowcolor{white}\shortstack{Key\\Source} & Authors & Title (Colored by Open Access)& LC & Cite & Year & \shortstack{Conference\\/Journal\\/School} & Pages & \shortstack{Cites\\OC XR\\SC} & \shortstack{Refs\\OC\\XR} & b & c \\ \midrule\endhead
\bottomrule
\endfoot
TasselGS23 \href{https://doi.org/10.1609/icaps.v33i1.27243}{TasselGS23} & \hyperref[auth:a58]{P. Tassel}, \hyperref[auth:a61]{M. Gebser}, \hyperref[auth:a423]{K. Schekotihin} & \cellcolor{gold!20}An End-to-End Reinforcement Learning Approach for Job-Shop Scheduling Problems Based on Constraint Programming & \href{../works/TasselGS23.pdf}{Yes} & \cite{TasselGS23} & 2023 & ICAPS 2023 & 9 & 0 1 2 & 0 0 & \ref{b:TasselGS23} & \ref{c:TasselGS23}\\
abs-2306-05747 \href{https://doi.org/10.48550/arXiv.2306.05747}{abs-2306-05747} & \hyperref[auth:a58]{P. Tassel}, \hyperref[auth:a61]{M. Gebser}, \hyperref[auth:a423]{K. Schekotihin} & An End-to-End Reinforcement Learning Approach for Job-Shop Scheduling Problems Based on Constraint Programming & \href{../works/abs-2306-05747.pdf}{Yes} & \cite{abs-2306-05747} & 2023 & CoRR & 9 & 0 0 0 & 0 0 & \ref{b:abs-2306-05747} & \ref{c:abs-2306-05747}\\
NaderiBZ22 \href{http://dx.doi.org/10.2139/ssrn.4140716}{NaderiBZ22} & \hyperref[auth:a726]{B. Naderi}, \hyperref[auth:a836]{M. A. Begen}, \hyperref[auth:a837]{G. Zhang} & Integrated Order Acceptance and Resource Decisions Under Uncertainty: Robust and Stochastic Approaches & \href{../works/NaderiBZ22.pdf}{Yes} & \cite{NaderiBZ22} & 2022 & SSRN Electronic Journal & 29 & 0 0 0 & 44 51 & \ref{b:NaderiBZ22} & n/a\\
NaderiBZ23 \href{http://dx.doi.org/10.2139/ssrn.4494381}{NaderiBZ23} & \hyperref[auth:a726]{B. Naderi}, \hyperref[auth:a836]{M. A. Begen}, \hyperref[auth:a837]{G. Zhang} & Integrated Order Acceptance and Resource Decisions Under Uncertainty: Robust and Stochastic Approaches & \href{../works/NaderiBZ23.pdf}{Yes} & \cite{NaderiBZ23} & 2023 & SSRN Electronic Journal & 32 & 0 0 0 & 46 56 & \ref{b:NaderiBZ23} & n/a\\
PerezGSL23 \href{https://doi.org/10.1109/ICTAI59109.2023.00108}{PerezGSL23} & \hyperref[auth:a425]{G. Perez}, \hyperref[auth:a426]{G. Glorian}, \hyperref[auth:a427]{W. Suijlen}, \hyperref[auth:a428]{A. Lallouet} & A Constraint Programming Model for Scheduling the Unloading of Trains in Ports & \href{../works/PerezGSL23.pdf}{Yes} & \cite{PerezGSL23} & 2023 & ICTAI 2023 & 7 & 0 0 0 & 0 19 & \ref{b:PerezGSL23} & \ref{c:PerezGSL23}\\
abs-2312-13682 \href{https://doi.org/10.48550/arXiv.2312.13682}{abs-2312-13682} & \hyperref[auth:a425]{G. Perez}, \hyperref[auth:a426]{G. Glorian}, \hyperref[auth:a427]{W. Suijlen}, \hyperref[auth:a428]{A. Lallouet} & A Constraint Programming Model for Scheduling the Unloading of Trains in Ports: Extended & \href{../works/abs-2312-13682.pdf}{Yes} & \cite{abs-2312-13682} & 2023 & CoRR & 20 & 0 0 0 & 0 0 & \ref{b:abs-2312-13682} & \ref{c:abs-2312-13682}\\
NuijtenA94 \href{}{NuijtenA94} & \hyperref[auth:a656]{W. Nuijten}, \hyperref[auth:a777]{E. H. L. Aarts} & Constraint Satisfaction for Multiple Capacitated Job Shop Scheduling & \href{../works/NuijtenA94.pdf}{Yes} & \cite{NuijtenA94} & 1994 & ECAI 1994 & 5 & 0 0 0 & 0 0 & \ref{b:NuijtenA94} & n/a\\
NuijtenA96 \href{http://dx.doi.org/10.1016/0377-2217(95)00354-1}{NuijtenA96} & \hyperref[auth:a656]{W. Nuijten}, \hyperref[auth:a777]{E. H. L. Aarts} & A computational study of constraint satisfaction for multiple capacitated job shop scheduling & \href{../works/NuijtenA96.pdf}{Yes} & \cite{NuijtenA96} & 1996 & European Journal of Operational Research & 16 & 65 65 90 & 6 21 & \ref{b:NuijtenA96} & n/a\\
EvenSH15 \href{https://doi.org/10.1007/978-3-319-23219-5_40}{EvenSH15} & \hyperref[auth:a214]{C. Even}, \hyperref[auth:a124]{A. Schutt}, \hyperref[auth:a148]{P. V. Hentenryck} & \cellcolor{green!10}A Constraint Programming Approach for Non-preemptive Evacuation Scheduling & \href{../works/EvenSH15.pdf}{Yes} & \cite{EvenSH15} & 2015 & CP 2015 & 18 & 3 2 6 & 12 14 & \ref{b:EvenSH15} & n/a\\
EvenSH15a \href{http://arxiv.org/abs/1505.02487}{EvenSH15a} & \hyperref[auth:a214]{C. Even}, \hyperref[auth:a124]{A. Schutt}, \hyperref[auth:a148]{P. V. Hentenryck} & A Constraint Programming Approach for Non-Preemptive Evacuation Scheduling & \href{../works/EvenSH15a.pdf}{Yes} & \cite{EvenSH15a} & 2015 & CoRR & 16 & 0 0 0 & 0 0 & \ref{b:EvenSH15a} & n/a\\
NishikawaSTT18 \href{https://doi.org/10.1109/CANDAR.2018.00025}{NishikawaSTT18} & \hyperref[auth:a531]{H. Nishikawa}, \hyperref[auth:a532]{K. Shimada}, \hyperref[auth:a533]{I. Taniguchi}, \hyperref[auth:a534]{H. Tomiyama} & Scheduling of Malleable Fork-Join Tasks with Constraint Programming & \href{../works/NishikawaSTT18.pdf}{Yes} & \cite{NishikawaSTT18} & 2018 & CANDAR 2018 & 6 & 2 2 2 & 14 21 & \ref{b:NishikawaSTT18} & n/a\\
NishikawaSTT18a \href{https://doi.org/10.1109/TENCON.2018.8650168}{NishikawaSTT18a} & \hyperref[auth:a531]{H. Nishikawa}, \hyperref[auth:a532]{K. Shimada}, \hyperref[auth:a533]{I. Taniguchi}, \hyperref[auth:a534]{H. Tomiyama} & Scheduling of Malleable Tasks Based on Constraint Programming & \href{../works/NishikawaSTT18a.pdf}{Yes} & \cite{NishikawaSTT18a} & 2018 & TENCON 2018 & 6 & 1 1 1 & 9 16 & \ref{b:NishikawaSTT18a} & n/a\\
HauderBRPA20 \href{http://dx.doi.org/10.1016/j.cie.2020.106857}{HauderBRPA20} & \hyperref[auth:a550]{V. A. Hauder}, \hyperref[auth:a551]{A. Beham}, \hyperref[auth:a552]{S. Raggl}, \hyperref[auth:a553]{S. N. Parragh}, \hyperref[auth:a554]{M. Affenzeller} & \cellcolor{green!10}Resource-constrained multi-project scheduling with activity and time flexibility & \href{../works/HauderBRPA20.pdf}{Yes} & \cite{HauderBRPA20} & 2020 & Computers \  Industrial Engineering & 14 & 14 19 27 & 46 56 & \ref{b:HauderBRPA20} & \ref{c:HauderBRPA20}\\
abs-1902-09244 \href{http://arxiv.org/abs/1902.09244}{abs-1902-09244} & \hyperref[auth:a550]{V. A. Hauder}, \hyperref[auth:a551]{A. Beham}, \hyperref[auth:a552]{S. Raggl}, \hyperref[auth:a553]{S. N. Parragh}, \hyperref[auth:a554]{M. Affenzeller} & On constraint programming for a new flexible project scheduling problem with resource constraints & \href{../works/abs-1902-09244.pdf}{Yes} & \cite{abs-1902-09244} & 2019 & CoRR & 62 & 0 0 0 & 0 0 & \ref{b:abs-1902-09244} & n/a\\
MilanoW06 \href{http://dx.doi.org/10.1007/s10288-006-0019-z}{MilanoW06} & \hyperref[auth:a143]{M. Milano}, \hyperref[auth:a117]{M. G. Wallace} & Integrating operations research in constraint programming & \href{../works/MilanoW06.pdf}{Yes} & \cite{MilanoW06} & 2006 & 4OR & 45 & 18 18 22 & 46 67 & \ref{b:MilanoW06} & n/a\\
MilanoW09 \href{http://dx.doi.org/10.1007/s10479-009-0654-9}{MilanoW09} & \hyperref[auth:a143]{M. Milano}, \hyperref[auth:a117]{M. G. Wallace} & Integrating Operations Research in Constraint Programming & \href{../works/MilanoW09.pdf}{Yes} & \cite{MilanoW09} & 2009 & Annals of Operations Research & 40 & 34 35 41 & 46 77 & \ref{b:MilanoW09} & n/a\\
NattafAL15 \href{https://doi.org/10.1007/s10601-015-9192-z}{NattafAL15} & \hyperref[auth:a81]{M. Nattaf}, \hyperref[auth:a6]{C. Artigues}, \hyperref[auth:a3]{P. Lopez} & \cellcolor{green!10}A hybrid exact method for a scheduling problem with a continuous resource and energy constraints & \href{../works/NattafAL15.pdf}{Yes} & \cite{NattafAL15} & 2015 & Constraints An Int. J. & 21 & 14 15 15 & 13 18 & \ref{b:NattafAL15} & \ref{c:NattafAL15}\\
NattafALR16 \href{https://doi.org/10.1007/s00291-015-0423-x}{NattafALR16} & \hyperref[auth:a81]{M. Nattaf}, \hyperref[auth:a6]{C. Artigues}, \hyperref[auth:a3]{P. Lopez}, \hyperref[auth:a980]{D. Rivreau} & \cellcolor{green!10}Energetic reasoning and mixed-integer linear programming for scheduling with a continuous resource and linear efficiency functions & \href{../works/NattafALR16.pdf}{Yes} & \cite{NattafALR16} & 2016 & {OR} Spectrum & 34 & 10 10 10 & 15 19 & \ref{b:NattafALR16} & n/a\\
Hooker05a \href{https://doi.org/10.1007/11564751_25}{Hooker05a} & \hyperref[auth:a160]{J. N. Hooker} & \cellcolor{green!10}Planning and Scheduling to Minimize Tardiness & \href{../works/Hooker05a.pdf}{Yes} & \cite{Hooker05a} & 2005 & CP 2005 & 14 & 30 31 35 & 10 12 & \ref{b:Hooker05a} & n/a\\
Hooker06 \href{https://doi.org/10.1007/s10601-006-8060-2}{Hooker06} & \hyperref[auth:a160]{J. N. Hooker} & \cellcolor{green!10}An Integrated Method for Planning and Scheduling to Minimize Tardiness & \href{../works/Hooker06.pdf}{Yes} & \cite{Hooker06} & 2006 & Constraints An Int. J. & 19 & 19 20 27 & 13 20 & \ref{b:Hooker06} & \ref{c:Hooker06}\\
BehrensLM19 \href{https://doi.org/10.1109/ICRA.2019.8794022}{BehrensLM19} & \hyperref[auth:a540]{J. K. Behrens}, \hyperref[auth:a541]{R. Lange}, \hyperref[auth:a542]{M. Mansouri} & \cellcolor{green!10}A Constraint Programming Approach to Simultaneous Task Allocation and Motion Scheduling for Industrial Dual-Arm Manipulation Tasks & \href{../works/BehrensLM19.pdf}{Yes} & \cite{BehrensLM19} & 2019 & ICRA 2019 & 7 & 12 17 27 & 18 27 & \ref{b:BehrensLM19} & \ref{c:BehrensLM19}\\
abs-1901-07914 \href{http://arxiv.org/abs/1901.07914}{abs-1901-07914} & \hyperref[auth:a540]{J. K. Behrens}, \hyperref[auth:a541]{R. Lange}, \hyperref[auth:a542]{M. Mansouri} & A Constraint Programming Approach to Simultaneous Task Allocation and Motion Scheduling for Industrial Dual-Arm Manipulation Tasks & \href{../works/abs-1901-07914.pdf}{Yes} & \cite{abs-1901-07914} & 2019 & CoRR & 8 & 0 0 0 & 0 0 & \ref{b:abs-1901-07914} & \ref{c:abs-1901-07914}\\
BaptisteP00 \href{https://doi.org/10.1023/A:1009822502231}{BaptisteP00} & \hyperref[auth:a162]{P. Baptiste}, \hyperref[auth:a163]{C. L. Pape} & Constraint Propagation and Decomposition Techniques for Highly Disjunctive and Highly Cumulative Project Scheduling Problems & \href{../works/BaptisteP00.pdf}{Yes} & \cite{BaptisteP00} & 2000 & Constraints An Int. J. & 21 & 46 0 62 & 0 0 & \ref{b:BaptisteP00} & \ref{c:BaptisteP00}\\
BaptisteP97 \href{https://doi.org/10.1007/BFb0017454}{BaptisteP97} & \hyperref[auth:a162]{P. Baptiste}, \hyperref[auth:a163]{C. L. Pape} & Constraint Propagation and Decomposition Techniques for Highly Disjunctive and Highly Cumulative Project Scheduling Problems & \href{../works/BaptisteP97.pdf}{Yes} & \cite{BaptisteP97} & 1997 & CP 1997 & 15 & 8 9 14 & 10 22 & \ref{b:BaptisteP97} & n/a\\
HeinzNVH22 \href{https://doi.org/10.1016/j.cie.2022.108586}{HeinzNVH22} & \hyperref[auth:a433]{V. Heinz}, \hyperref[auth:a434]{A. Nov{\'{a}}k}, \hyperref[auth:a311]{M. Vlk}, \hyperref[auth:a116]{Z. Hanz{\'{a}}lek} & \cellcolor{green!10}Constraint Programming and constructive heuristics for parallel machine scheduling with sequence-dependent setups and common servers & \href{../works/HeinzNVH22.pdf}{Yes} & \cite{HeinzNVH22} & 2022 & Computers \  Industrial Engineering & 16 & 5 7 8 & 25 31 & \ref{b:HeinzNVH22} & \ref{c:HeinzNVH22}\\
abs-2305-19888 \href{https://doi.org/10.48550/arXiv.2305.19888}{abs-2305-19888} & \hyperref[auth:a433]{V. Heinz}, \hyperref[auth:a434]{A. Nov{\'{a}}k}, \hyperref[auth:a311]{M. Vlk}, \hyperref[auth:a116]{Z. Hanz{\'{a}}lek} & Constraint Programming and Constructive Heuristics for Parallel Machine Scheduling with Sequence-Dependent Setups and Common Servers & \href{../works/abs-2305-19888.pdf}{Yes} & \cite{abs-2305-19888} & 2023 & CoRR & 42 & 0 0 0 & 0 0 & \ref{b:abs-2305-19888} & \ref{c:abs-2305-19888}\\
Rodriguez07b \href{}{Rodriguez07b} & \hyperref[auth:a781]{J. Rodriguez} & A study of the use of state resources in a constraint-based model for routing and scheduling trains & \href{../works/Rodriguez07b.pdf}{Yes} & \cite{Rodriguez07b} & 2007 & ICROMA 2007 & 14 & 0 0 0 & 0 0 & \ref{b:Rodriguez07b} & n/a\\
RodriguezS09 \href{}{RodriguezS09} & \hyperref[auth:a781]{J. Rodriguez}, \hyperref[auth:a1018]{S. Sobieraj} & A study of an incremental texture-based heuristic for the train routing and scheduling problem & \href{../works/RodriguezS09.pdf}{Yes} & \cite{RodriguezS09} & 2009 & ICROMA 2009 & 14 & 0 0 0 & 0 0 & \ref{b:RodriguezS09} & n/a\\
VilimBC04 \href{https://doi.org/10.1007/978-3-540-30201-8_8}{VilimBC04} & \hyperref[auth:a121]{P. Vil{\'{\i}}m}, \hyperref[auth:a152]{R. Bart{\'{a}}k}, \hyperref[auth:a161]{O. Cepek} & Unary Resource Constraint with Optional Activities & \href{../works/VilimBC04.pdf}{Yes} & \cite{VilimBC04} & 2004 & CP 2004 & 15 & 13 12 17 & 4 11 & \ref{b:VilimBC04} & n/a\\
VilimBC05 \href{https://doi.org/10.1007/s10601-005-2814-0}{VilimBC05} & \hyperref[auth:a121]{P. Vil{\'{\i}}m}, \hyperref[auth:a152]{R. Bart{\'{a}}k}, \hyperref[auth:a161]{O. Cepek} & Extension of \emph{O}(\emph{n} log \emph{n}) Filtering Algorithms for the Unary Resource Constraint to Optional Activities & \href{../works/VilimBC05.pdf}{Yes} & \cite{VilimBC05} & 2005 & Constraints An Int. J. & 23 & 21 21 32 & 5 16 & \ref{b:VilimBC05} & \ref{c:VilimBC05}\\
Alaka21 \href{http://dx.doi.org/10.1007/s00500-021-05602-x}{Alaka21} & \hyperref[auth:a764]{H. M. Alakaş} & General resource-constrained assembly line balancing problem: conjunction normal form based constraint programming models & \href{../works/Alaka21.pdf}{Yes} & \cite{Alaka21} & 2021 & Soft Computing & 11 & 7 9 9 & 20 27 & \ref{b:Alaka21} & n/a\\
AlakaPY19 \href{http://dx.doi.org/10.1007/s00500-019-04294-8}{AlakaPY19} & \hyperref[auth:a764]{H. M. Alakaş}, \hyperref[auth:a1385]{M. Pınarbaşı}, \hyperref[auth:a1426]{M. Y\"{u}z\"{u}kırmızı} & Constraint programming model for resource-constrained assembly line balancing problem & \href{../works/AlakaPY19.pdf}{Yes} & \cite{AlakaPY19} & 2019 & Soft Computing & 9 & 15 17 0 & 14 23 & \ref{b:AlakaPY19} & n/a\\
ZibranR11 \href{https://doi.org/10.1109/ICPC.2011.45}{ZibranR11} & \hyperref[auth:a619]{M. F. Zibran}, \hyperref[auth:a620]{C. K. Roy} & Conflict-Aware Optimal Scheduling of Code Clone Refactoring: {A} Constraint Programming Approach & \href{../works/ZibranR11.pdf}{Yes} & \cite{ZibranR11} & 2011 & ICPC 2011 & 4 & 17 16 19 & 18 24 & \ref{b:ZibranR11} & n/a\\
ZibranR11a \href{https://doi.org/10.1109/SCAM.2011.21}{ZibranR11a} & \hyperref[auth:a619]{M. F. Zibran}, \hyperref[auth:a620]{C. K. Roy} & A Constraint Programming Approach to Conflict-Aware Optimal Scheduling of Prioritized Code Clone Refactoring & \href{../works/ZibranR11a.pdf}{Yes} & \cite{ZibranR11a} & 2011 & SCAM 2011 & 10 & 26 26 33 & 27 35 & \ref{b:ZibranR11a} & n/a\\
LetortBC12 \href{https://doi.org/10.1007/978-3-642-33558-7_33}{LetortBC12} & \hyperref[auth:a127]{A. Letort}, \hyperref[auth:a128]{N. Beldiceanu}, \hyperref[auth:a91]{M. Carlsson} & A Scalable Sweep Algorithm for the cumulative Constraint & \href{../works/LetortBC12.pdf}{Yes} & \cite{LetortBC12} & 2012 & CP 2012 & 16 & 18 19 27 & 12 21 & \ref{b:LetortBC12} & n/a\\
LetortCB13 \href{https://doi.org/10.1007/978-3-642-38171-3_10}{LetortCB13} & \hyperref[auth:a127]{A. Letort}, \hyperref[auth:a91]{M. Carlsson}, \hyperref[auth:a128]{N. Beldiceanu} & A Synchronized Sweep Algorithm for the \emph{k-dimensional cumulative} Constraint & \href{../works/LetortCB13.pdf}{Yes} & \cite{LetortCB13} & 2013 & CPAIOR 2013 & 16 & 3 3 4 & 10 16 & \ref{b:LetortCB13} & \ref{c:LetortCB13}\\
LacknerMMWW21 \href{https://doi.org/10.4230/LIPIcs.CP.2021.37}{LacknerMMWW21} & \hyperref[auth:a62]{M.-L. Lackner}, \hyperref[auth:a63]{C. Mrkvicka}, \hyperref[auth:a45]{N. Musliu}, \hyperref[auth:a46]{D. Walkiewicz}, \hyperref[auth:a43]{F. Winter} & Minimizing Cumulative Batch Processing Time for an Industrial Oven Scheduling Problem & \href{../works/LacknerMMWW21.pdf}{Yes} & \cite{LacknerMMWW21} & 2021 & CP 2021 & 18 & 0 0 3 & 0 0 & \ref{b:LacknerMMWW21} & \ref{c:LacknerMMWW21}\\
LacknerMMWW23 \href{https://doi.org/10.1007/s10601-023-09347-2}{LacknerMMWW23} & \hyperref[auth:a62]{M.-L. Lackner}, \hyperref[auth:a63]{C. Mrkvicka}, \hyperref[auth:a45]{N. Musliu}, \hyperref[auth:a46]{D. Walkiewicz}, \hyperref[auth:a43]{F. Winter} & \cellcolor{gold!20}Exact methods for the Oven Scheduling Problem & \href{../works/LacknerMMWW23.pdf}{Yes} & \cite{LacknerMMWW23} & 2023 & Constraints An Int. J. & 42 & 0 1 0 & 32 38 & \ref{b:LacknerMMWW23} & \ref{c:LacknerMMWW23}\\
NishikawaSTT18 \href{https://doi.org/10.1109/CANDAR.2018.00025}{NishikawaSTT18} & \hyperref[auth:a531]{H. Nishikawa}, \hyperref[auth:a532]{K. Shimada}, \hyperref[auth:a533]{I. Taniguchi}, \hyperref[auth:a534]{H. Tomiyama} & Scheduling of Malleable Fork-Join Tasks with Constraint Programming & \href{../works/NishikawaSTT18.pdf}{Yes} & \cite{NishikawaSTT18} & 2018 & CANDAR 2018 & 6 & 2 2 2 & 14 21 & \ref{b:NishikawaSTT18} & n/a\\
NishikawaSTT19 \href{http://www.ijnc.org/index.php/ijnc/article/view/201}{NishikawaSTT19} & \hyperref[auth:a531]{H. Nishikawa}, \hyperref[auth:a532]{K. Shimada}, \hyperref[auth:a533]{I. Taniguchi}, \hyperref[auth:a534]{H. Tomiyama} & A Constraint Programming Approach to Scheduling of Malleable Tasks & \href{../works/NishikawaSTT19.pdf}{Yes} & \cite{NishikawaSTT19} & 2019 & Int. J. Netw. Comput. & 16 & 3 3 0 & 20 30 & \ref{b:NishikawaSTT19} & n/a\\
\end{longtable}
}



{\scriptsize
\begin{longtable}{>{\raggedright\arraybackslash}p{3cm}r>{\raggedright\arraybackslash}p{4cm}p{1.5cm}p{2cm}p{1.5cm}p{1.5cm}p{1.5cm}p{1.5cm}p{2cm}p{1.5cm}rr}
\rowcolor{white}\caption{Features of Works Similar by Cosine Similarity}\\ \toprule
\rowcolor{white}Work & Pages & Concepts & Classification & Constraints & \shortstack{Prog\\Languages} & \shortstack{CP\\Systems} & Areas & Industries & Benchmarks & Algorithm & a & c\\ \midrule\endhead
\bottomrule
\endfoot
\href{../works/TasselGS23.pdf}{TasselGS23}~\cite{TasselGS23} & 9 & job-shop, flow-shop, completion-time, CP, resource, flow-time, re-scheduling, job, constraint programming, precedence, order, tardiness, constraint optimization, scheduling, preempt, task, machine, make-span, periodic & JSSP & cumulative, disjunctive, noOverlap & Java & Choco Solver &  &  & industrial instance, real-world, github, benchmark, supplementary material & genetic algorithm, neural network, large neighborhood search, machine learning, simulated annealing, reinforcement learning, meta heuristic & \ref{a:TasselGS23} & \ref{c:TasselGS23}\\
\href{../works/abs-2306-05747.pdf}{abs-2306-05747}~\cite{abs-2306-05747} & 9 & re-scheduling, scheduling, order, make-span, preempt, constraint programming, CP, flow-time, completion-time, resource, job, periodic, job-shop, precedence, constraint optimization, task, tardiness, machine, flow-shop & JSSP & noOverlap, disjunctive, cumulative & Java & Choco Solver &  &  & real-world, github, industrial instance, supplementary material, benchmark & neural network, large neighborhood search, reinforcement learning, genetic algorithm, machine learning, meta heuristic, simulated annealing & \ref{a:abs-2306-05747} & \ref{c:abs-2306-05747}\\
\href{../works/PerezGSL23.pdf}{PerezGSL23}~\cite{PerezGSL23} & 7 & inventory, order, transportation, re-scheduling, scheduling, task, machine, make-span, resource, activity, constraint programming, completion-time, CP &  & table constraint, cumulative &  & OPL & container terminal, nurse, operating room, steel mill &  & real-world, generated instance & large neighborhood search, mat heuristic, meta heuristic & \ref{a:PerezGSL23} & \ref{c:PerezGSL23}\\
\href{../works/abs-2312-13682.pdf}{abs-2312-13682}~\cite{abs-2312-13682} & 20 & activity, constraint programming, machine, inventory, re-scheduling, scheduling, order, make-span, CP, resource, transportation, task &  & table constraint, cumulative &  & OPL & container terminal, train schedule, nurse, steel mill, operating room &  & real-world, generated instance & large neighborhood search, mat heuristic, meta heuristic & \ref{a:abs-2312-13682} & \ref{c:abs-2312-13682}\\
\href{../works/NaderiBZ22.pdf}{NaderiBZ22}~\cite{NaderiBZ22} & 29 & stochastic, setup-time, open-shop, order, scheduling, machine, make-span, distributed, Logic-Based Benders Decomposition, job-shop, due-date, tardiness, flow-shop, lateness, CP, resource, transportation, no-wait, job, constraint programming, completion-time, Benders Decomposition & parallel machine, single machine & disjunctive, noOverlap, Disjunctive constraint &  & Cplex, CPO & crew-scheduling, nurse, surgery, patient, operating room, automotive &  & benchmark, real-life & meta heuristic, memetic algorithm & \ref{a:NaderiBZ22} & n/a\\
\href{../works/NaderiBZ23.pdf}{NaderiBZ23}~\cite{NaderiBZ23} & 32 & stochastic, setup-time, open-shop, order, scheduling, machine, make-span, distributed, Logic-Based Benders Decomposition, job-shop, due-date, tardiness, flow-shop, lateness, CP, resource, transportation, no-wait, job, constraint programming, completion-time, Benders Decomposition & parallel machine, single machine & disjunctive, noOverlap, Disjunctive constraint & Python & Cplex, CPO & crew-scheduling, nurse, surgery, patient, operating room, automotive &  & benchmark, real-world & meta heuristic, memetic algorithm & \ref{a:NaderiBZ23} & n/a\\
\href{../works/NuijtenA94.pdf}{NuijtenA94}~\cite{NuijtenA94} & 5 & scheduling, preempt, machine, make-span, constraint satisfaction, preemptive, job-shop, completion-time, CP, resource, job, CSP, precedence, CLP, order & JSSP & disjunctive, Disjunctive constraint & C++ & Ilog Solver, CPO &  &  &  & time-tabling & \ref{a:NuijtenA94} & n/a\\
\href{../works/NuijtenA96.pdf}{NuijtenA96}~\cite{NuijtenA96} & 16 & scheduling, preempt, machine, make-span, constraint satisfaction, preemptive, job-shop, flow-shop, completion-time, CP, resource, job, constraint programming, CSP, precedence, CLP, order & JSSP & disjunctive, Disjunctive constraint &  & CPO &  &  &  & time-tabling & \ref{a:NuijtenA96} & n/a\\
\href{../works/HauderBRPA20.pdf}{HauderBRPA20}~\cite{HauderBRPA20} & 14 & setup-time, order, bi-objective, no-wait, job-shop, resource, stochastic, task, constraint programming, completion-time, precedence, earliness, machine, transportation, tardiness, make-span, activity, explanation, inventory, due-date, scheduling, flow-shop, job, CP, multi-objective, breakdown, manpower & RCPSP, RCMPSP, FJS, Resource-constrained Project Scheduling Problem & cumulative, cycle &  & OPL, Cplex & aircraft & automobile industry, food-processing industry, steel industry, processing industry & industry partner, benchmark, real-world, supplementary material & particle swarm, genetic algorithm, meta heuristic & \ref{a:HauderBRPA20} & \ref{c:HauderBRPA20}\\
\href{../works/abs-1902-09244.pdf}{abs-1902-09244}~\cite{abs-1902-09244} & 62 & setup-time, activity, constraint programming, machine, flow-shop, CP, job, order, due-date, earliness, bi-objective, stochastic, explanation, completion-time, breakdown, resource, task, job-shop, tardiness, inventory, multi-objective, no-wait, precedence, transportation, scheduling, make-span, release-date & Resource-constrained Project Scheduling Problem, FJS, RCMPSP, RCPSP & cycle, cumulative, endBeforeStart &  & OPL, Cplex & aircraft & automobile industry, steel industry, food-processing industry, glass industry, processing industry & real-world, benchmark, industry partner & genetic algorithm, particle swarm, simulated annealing, meta heuristic & \ref{a:abs-1902-09244} & n/a\\
\href{../works/NishikawaSTT18.pdf}{NishikawaSTT18}~\cite{NishikawaSTT18} & 6 & precedence, scheduling, make-span, activity, distributed, constraint programming, order, CP, resource, task &  & alternative constraint, endBeforeStart &  & Cplex & robot, pipeline &  & real-world, benchmark & genetic algorithm & \ref{a:NishikawaSTT18} & n/a\\
\href{../works/NishikawaSTT18a.pdf}{NishikawaSTT18a}~\cite{NishikawaSTT18a} & 6 & CP, make-span, scheduling, resource, task, constraint programming, distributed, re-scheduling, order, precedence, activity &  & endBeforeStart, alternative constraint &  & Cplex & robot, nurse, pipeline &  & benchmark, real-life, real-world & genetic algorithm & \ref{a:NishikawaSTT18a} & n/a\\
\href{../works/EvenSH15.pdf}{EvenSH15}~\cite{EvenSH15} & 18 & transportation, CP, preempt, machine, distributed, resource, preemptive, order, scheduling, constraint programming, Benders Decomposition, completion-time, task &  & Disjunctive constraint, cumulative, disjunctive &  & OPL, Choco Solver & evacuation, emergency service &  & real-life, real-world & column generation, sweep, mat heuristic, ant colony & \ref{a:EvenSH15} & n/a\\
\href{../works/EvenSH15a.pdf}{EvenSH15a}~\cite{EvenSH15a} & 16 & distributed, constraint programming, resource, transportation, Benders Decomposition, order, preempt, scheduling, task, machine, preemptive, completion-time, CP &  & disjunctive, Disjunctive constraint, cumulative & Java & Choco Solver, OPL & emergency service, evacuation &  & real-world, real-life & ant colony, mat heuristic, meta heuristic, column generation, sweep & \ref{a:EvenSH15a} & n/a\\
\href{../works/BehrensLM19.pdf}{BehrensLM19}~\cite{BehrensLM19} & 7 & resource, setup-time, task, constraint satisfaction, constraint programming, make-span, order, machine, CP, scheduling, CSP, distributed, multi-agent, constraint optimization &  &  & Python & OR-Tools, MiniZinc & robot &  & github, real-world &  & \ref{a:BehrensLM19} & \ref{c:BehrensLM19}\\
\href{../works/abs-1901-07914.pdf}{abs-1901-07914}~\cite{abs-1901-07914} & 8 & constraint programming, CP, resource, CSP, constraint satisfaction, constraint optimization, task, distributed, machine, multi-agent, scheduling, order, make-span &  &  & Python & OR-Tools, MiniZinc & robot &  & real-world, github, benchmark &  & \ref{a:abs-1901-07914} & \ref{c:abs-1901-07914}\\
\href{../works/NattafAL15.pdf}{NattafAL15}~\cite{NattafAL15} & 21 & release-date, scheduling, preempt, task, make-span, due-date, resource, preemptive, activity, constraint programming, CSP, CP, order & CECSP, RCPSP, Resource-constrained Project Scheduling Problem, CuSP & cumulative & C++ & Cplex &  &  & generated instance & sweep, energetic reasoning & \ref{a:NattafAL15} & \ref{c:NattafAL15}\\
\href{../works/NattafALR16.pdf}{NattafALR16}~\cite{NattafALR16} & 34 & preemptive, no preempt, task, constraint programming, precedence, make-span, order, preempt, CP, scheduling, due-date, CSP, activity, explanation, resource, release-date & CECSP, CuSP, Resource-constrained Project Scheduling Problem, RCPSP & cumulative & C++ & Cplex &  &  & generated instance & sweep, energetic reasoning & \ref{a:NattafALR16} & n/a\\
\href{../works/MilanoW06.pdf}{MilanoW06}~\cite{MilanoW06} & 45 & release-date, Logic-Based Benders Decomposition, distributed, one-machine scheduling, job-shop, resource, constraint logic programming, job, constraint satisfaction, preempt, setup-time, explanation, single-machine scheduling, scheduling, tardiness, task, preemptive, due-date, CP, machine, lateness, stochastic, constraint programming, transportation, Benders Decomposition, order, CSP, completion-time, activity & single machine, parallel machine & Cumulatives constraint, Reified constraint, Cardinality constraint, Channeling constraint, circuit, cumulative, alldifferent, GCC constraint &  & CHIP, ECLiPSe, Cplex, OPL & crew-scheduling &  & benchmark & column generation, edge-finder, meta heuristic, time-tabling, large neighborhood search, Lagrangian relaxation & \ref{a:MilanoW06} & n/a\\
\href{../works/MilanoW09.pdf}{MilanoW09}~\cite{MilanoW09} & 40 & release-date, Logic-Based Benders Decomposition, distributed, one-machine scheduling, job-shop, resource, constraint logic programming, job, constraint satisfaction, preempt, setup-time, explanation, single-machine scheduling, scheduling, tardiness, task, preemptive, due-date, CP, machine, lateness, stochastic, constraint programming, transportation, Benders Decomposition, order, CSP, completion-time, activity & single machine & Cumulatives constraint, Reified constraint, Cardinality constraint, Channeling constraint, circuit, cumulative, alldifferent, GCC constraint &  & CHIP, SCIP, ECLiPSe, Cplex, OPL & crew-scheduling &  & benchmark & column generation, edge-finder, meta heuristic, lazy clause generation, time-tabling, large neighborhood search, Lagrangian relaxation & \ref{a:MilanoW09} & n/a\\
\href{../works/Hooker05a.pdf}{Hooker05a}~\cite{Hooker05a} & 14 & CP, machine, constraint programming, resource, Benders Decomposition, order, release-date, scheduling, Logic-Based Benders Decomposition, make-span, task, constraint logic programming, job, due-date, precedence, tardiness &  & circuit, cumulative, disjunctive &  & Ilog Scheduler, OPL, Cplex &  &  &  & MINLP & \ref{a:Hooker05a} & n/a\\
\href{../works/Hooker06.pdf}{Hooker06}~\cite{Hooker06} & 19 & constraint satisfaction, machine, job, task, release-date, constraint programming, Logic-Based Benders Decomposition, CP, make-span, constraint logic programming, resource, precedence, due-date, order, tardiness, scheduling, Benders Decomposition &  & disjunctive, cumulative, circuit &  & OPL, Ilog Scheduler, Cplex &  &  & random instance & MINLP & \ref{a:Hooker06} & \ref{c:Hooker06}\\
\href{../works/BaptisteP00.pdf}{BaptisteP00}~\cite{BaptisteP00} & 21 & preempt, CP, cmax, scheduling, re-scheduling, due-date, CSP, job, activity, resource, preemptive, job-shop, task, constraint satisfaction, constraint programming, precedence, release-date, flow-shop, make-span, order & RCPSP & cumulative, Disjunctive constraint, disjunctive & C++ & Claire, CHIP, Ilog Scheduler &  &  & benchmark & energetic reasoning, edge-finding, edge-finder & \ref{a:BaptisteP00} & \ref{c:BaptisteP00}\\
\href{../works/BaptisteP97.pdf}{BaptisteP97}~\cite{BaptisteP97} & 15 & preempt, CP, scheduling, re-scheduling, due-date, CSP, job, activity, resource, preemptive, job-shop, task, constraint satisfaction, constraint programming, precedence, release-date, flow-shop, make-span, order & Resource-constrained Project Scheduling Problem, RCPSP & cumulative, Disjunctive constraint, disjunctive & C++ & Claire, CHIP &  &  & benchmark & edge-finding, edge-finder & \ref{a:BaptisteP97} & n/a\\
\href{../works/HeinzNVH22.pdf}{HeinzNVH22}~\cite{HeinzNVH22} & 16 & re-scheduling, bi-objective, scheduling, preempt, sequence dependent setup, task, unavailability, machine, make-span, distributed, flow-shop, completion-time, CP, resource, preemptive, activity, explanation, job, constraint programming, precedence, setup-time, order & parallel machine & cumulative, noOverlap, alternative constraint &  & Gurobi & high performance computing, robot, crew-scheduling &  & real-world, generated instance, benchmark, gitlab & meta heuristic, genetic algorithm, Lagrangian relaxation & \ref{a:HeinzNVH22} & \ref{c:HeinzNVH22}\\
\href{../works/abs-2305-19888.pdf}{abs-2305-19888}~\cite{abs-2305-19888} & 42 & sequence dependent setup, distributed, flow-shop, scheduling, order, make-span, preempt, setup-time, activity, constraint programming, machine, CP, job, re-scheduling, unavailability, preemptive, bi-objective, explanation, cmax, completion-time, resource, precedence, task & parallel machine & alternative constraint, noOverlap, cumulative &  & Gurobi & robot, high performance computing &  & gitlab, generated instance, real-world, benchmark & meta heuristic, Lagrangian relaxation, genetic algorithm & \ref{a:abs-2305-19888} & \ref{c:abs-2305-19888}\\
\href{../works/Alaka21.pdf}{Alaka21}~\cite{Alaka21} & 11 & CP, precedence, constraint programming, multi-objective, stochastic, completion-time, task, resource, cyclic scheduling, order, machine, scheduling &  & cycle &  &  &  &  & real-life & meta heuristic, genetic algorithm & \ref{a:Alaka21} & n/a\\
\href{../works/AlakaPY19.pdf}{AlakaPY19}~\cite{AlakaPY19} & 9 & scheduling, precedence, CP, machine, stochastic, constraint programming, order, completion-time, cyclic scheduling, multi-objective, task, resource &  & bin-packing, cycle &  & Cplex &  &  & real-life & genetic algorithm & \ref{a:AlakaPY19} & n/a\\
\href{../works/Rodriguez07b.pdf}{Rodriguez07b}~\cite{Rodriguez07b} & 14 & re-scheduling, CP, blocking constraint, order, no-wait, job-shop, resource, CSP, task, constraint programming, release-date, precedence, scheduling, transportation, activity, job &  & circuit, disjunctive, Blocking constraint, Disjunctive constraint &  & Ilog Scheduler, Z3, Ilog Solver & train schedule, railway & railway industry &  & edge-finding & \ref{a:Rodriguez07b} & n/a\\
\href{../works/RodriguezS09.pdf}{RodriguezS09}~\cite{RodriguezS09} & 14 & blocking constraint, Benders Decomposition, order, no-wait, Logic-Based Benders Decomposition, job-shop, resource, CSP, task, constraint programming, completion-time, precedence, scheduling, transportation, activity, constraint satisfaction, job, CP &  & circuit, disjunctive, Blocking constraint, Disjunctive constraint &  & Ilog Scheduler, Ilog Solver & train schedule, railway &  &  & edge-finding & \ref{a:RodriguezS09} & n/a\\
\href{../works/VilimBC04.pdf}{VilimBC04}~\cite{VilimBC04} & 15 & scheduling, make-span, job, machine, precedence, completion-time, CP, distributed, constraint programming, job-shop, resource, open-shop, order, activity &  & disjunctive, cumulative &  &  &  &  & benchmark, real-life & edge-finding, not-first, not-last & \ref{a:VilimBC04} & n/a\\
\href{../works/VilimBC05.pdf}{VilimBC05}~\cite{VilimBC05} & 23 & setup-time, scheduling, make-span, task, job, sequence dependent setup, batch process, machine, precedence, CLP, completion-time, CP, distributed, constraint programming, job-shop, resource, open-shop, order, activity &  & disjunctive, cumulative, cycle &  &  &  &  & benchmark, real-life & edge-finding, not-first, not-last, sweep & \ref{a:VilimBC05} & \ref{c:VilimBC05}\\
\href{../works/LetortBC12.pdf}{LetortBC12}~\cite{LetortBC12} & 16 & constraint programming, precedence, CP, order, scheduling, task, CLP, machine, make-span, resource & psplib & cumulative, geost, Cumulatives constraint, bin-packing & Java, Prolog & Choco Solver, CHIP, SICStus & datacenter &  & benchmark, random instance, Roadef & meta heuristic, sweep, edge-finding & \ref{a:LetortBC12} & n/a\\
\href{../works/LetortCB13.pdf}{LetortCB13}~\cite{LetortCB13} & 16 & constraint programming, precedence, CP, order, scheduling, task, machine, make-span, resource & psplib, RCPSP & Disjunctive constraint, cumulative, disjunctive, bin-packing & Java, Prolog & Choco Solver, SICStus &  &  & benchmark, random instance, Roadef & energetic reasoning, meta heuristic, sweep, edge-finding & \ref{a:LetortCB13} & \ref{c:LetortCB13}\\
\href{../works/TranAB16.pdf}{TranAB16}~\cite{TranAB16} & 13 & sequence dependent setup, due-date, order, tardiness, scheduling, completion-time, machine, job, release-date, cmax, constraint programming, Benders Decomposition, Logic-Based Benders Decomposition, stochastic, setup-time, CP, make-span, explanation, single-machine scheduling, constraint logic programming, resource, precedence & PMSP, single machine, parallel machine & cycle, circuit &  & SCIP, Gurobi, Cplex & aircraft &  & benchmark & simulated annealing, meta heuristic, ant colony, genetic algorithm, column generation & \ref{a:TranAB16} & n/a\\
\href{../works/TranB12.pdf}{TranB12}~\cite{TranB12} & 6 & setup-time, constraint programming, due-date, release-date, scheduling, CP, Logic-Based Benders Decomposition, tardiness, completion-time, Benders Decomposition, resource, make-span, single-machine scheduling, sequence dependent setup, job, order, machine, distributed, precedence, cmax & PMSP, single machine, parallel machine & circuit, cycle & C++ & Cplex &  &  & benchmark & column generation, meta heuristic, simulated annealing, ant colony & \ref{a:TranB12} & n/a\\
\href{../works/Hooker04.pdf}{Hooker04}~\cite{Hooker04} & 12 & constraint satisfaction, machine, task, release-date, constraint programming, Logic-Based Benders Decomposition, CP, make-span, constraint logic programming, distributed, resource, precedence, order, tardiness, scheduling, Benders Decomposition &  & disjunctive, cumulative, circuit &  & OPL, Ilog Scheduler, Cplex &  &  & random instance & MINLP & \ref{a:Hooker04} & n/a\\
\href{../works/Hooker07.pdf}{Hooker07}~\cite{Hooker07} & 15 & constraint satisfaction, machine, job, task, release-date, constraint programming, Logic-Based Benders Decomposition, inventory, activity, CP, make-span, constraint logic programming, distributed, resource, precedence, due-date, order, tardiness, scheduling, Benders Decomposition &  & disjunctive, cumulative, circuit &  & OPL, Ilog Scheduler, Cplex &  &  & random instance, generated instance & MINLP, edge-finding & \ref{a:Hooker07} & n/a\\
\href{../works/Hooker06.pdf}{Hooker06}~\cite{Hooker06} & 19 & constraint satisfaction, machine, job, task, release-date, constraint programming, Logic-Based Benders Decomposition, CP, make-span, constraint logic programming, resource, precedence, due-date, order, tardiness, scheduling, Benders Decomposition &  & disjunctive, cumulative, circuit &  & OPL, Ilog Scheduler, Cplex &  &  & random instance & MINLP & \ref{a:Hooker06} & \ref{c:Hooker06}\\
\href{../works/Hooker07.pdf}{Hooker07}~\cite{Hooker07} & 15 & constraint satisfaction, machine, job, task, release-date, constraint programming, Logic-Based Benders Decomposition, inventory, activity, CP, make-span, constraint logic programming, distributed, resource, precedence, due-date, order, tardiness, scheduling, Benders Decomposition &  & disjunctive, cumulative, circuit &  & OPL, Ilog Scheduler, Cplex &  &  & random instance, generated instance & MINLP, edge-finding & \ref{a:Hooker07} & n/a\\
\end{longtable}
}



\clearpage
\section{Missing Works}

The following table shows works that are currently not included in the survey, but which are links to works currently included. The "Nr Links" field shows how many connections exist according to OpenCitations.

{\scriptsize
\begin{longtable}{p{5cm}lp{11cm}rrrrrr}
\caption{Missing Work Considered Relevant (Total 22740 Works Checked, 0 Selected)}\\ \toprule
DOI & Type & Authors/Title & \shortstack{Nr\\Links} & \shortstack{Citing\\Survey} & \shortstack{Cited by\\Survey} & \shortstack{XRef\\Refs} & \shortstack{XRef\\Cite} & Relevance\\ \midrule\endhead
\bottomrule
\endfoot
\end{longtable}
}



\clearpage
\section{Highly Connected Missing Work}

The following table shows work currently not considered which is highly connected to the survey, but which does not have a high enough relevance to be included automatically. The top 50 entries are shown. Some of these work may be background material cited by many works in the survey, or we might be missing some features to show that they are relevant to the survey topic.

{\scriptsize
\begin{longtable}{p{5cm}lp{11cm}rrrrrr}
\caption{Highly Connected Missing Work Not Considered Relevant (Total 22742 Works Checked, 50 Selected)}\\ \toprule
DOI & Type & Authors/Title & \shortstack{Nr\\Links} & \shortstack{Citing\\Survey} & \shortstack{Cited by\\Survey} & \shortstack{XRef\\Refs} & \shortstack{XRef\\Cite} & Relevance\\ \midrule\endhead
\bottomrule
\endfoot
\href{http://dx.doi.org/10.1016/0004-3702(80)90051-x}{10.1016/0004-3702(80)90051-x} \href{https://www.doi2bib.org/bib/10.1016/0004-3702(80)90051-x}{(bib)} & journal-article & Robert M. Haralick, Gordon L. Elliott. Increasing tree search efficiency for constraint satisfaction problems. Artificial Intelligence, 1980. & 30 & 0 & 30 & 17 & 638 &  0.00\\
\href{http://dx.doi.org/10.1016/j.ejor.2010.03.037}{10.1016/j.ejor.2010.03.037} \href{https://www.doi2bib.org/bib/10.1016/j.ejor.2010.03.037}{(bib)} & journal-article & Jan Węglarz, Joanna Józefowska, Marek Mika, Grzegorz Waligóra. Project scheduling with finite or infinite number of activity processing modes – A survey. European Journal of Operational Research, 2011. & 20 & 7 & 13 & 217 & 195 &  0.00\\
\href{http://dx.doi.org/10.1007/3-540-61310-2_29}{10.1007/3-540-61310-2 29} \href{https://www.doi2bib.org/bib/10.1007/3-540-61310-2_29}{(bib)} & book-chapter & Paul Martin, David B. Shmoys. A new approach to computing optimal schedules for the job-shop scheduling problem. Integer Programming and Combinatorial Optimization, 1996. & 19 & 0 & 19 & 22 & 82 &  0.00\\
\href{http://dx.doi.org/10.1016/j.ejor.2015.11.020}{10.1016/j.ejor.2015.11.020} \href{https://www.doi2bib.org/bib/10.1016/j.ejor.2015.11.020}{(bib)} & journal-article & Ridvan Gedik, Chase Rainwater, Heather Nachtmann, Ed A. Pohl. Analysis of a parallel machine scheduling problem with sequence dependent setup times and job availability intervals. European Journal of Operational Research, 2016. & 18 & 9 & 9 & 38 & 41 &  0.00\\
\href{http://dx.doi.org/10.1016/s0377-2217(97)00442-6}{10.1016/s0377-2217(97)00442-6} \href{https://www.doi2bib.org/bib/10.1016/s0377-2217(97)00442-6}{(bib)} & journal-article & Robert Klein, Armin Scholl. Computing lower bounds by destructive improvement: An application to resource-constrained project scheduling. European Journal of Operational Research, 1999. & 17 & 3 & 14 & 23 & 86 &  0.00\\
\href{http://dx.doi.org/10.1016/s0377-2217(97)00335-4}{10.1016/s0377-2217(97)00335-4} \href{https://www.doi2bib.org/bib/10.1016/s0377-2217(97)00335-4}{(bib)} & journal-article & Peter Brucker, Sigrid Knust, Arno Schoo, Olaf Thiele. A branch and bound algorithm for the resource-constrained project scheduling problem. European Journal of Operational Research, 1998. & 17 & 1 & 16 & 25 & 193 &  0.00\\
\href{http://dx.doi.org/10.1287/mnsc.42.6.797}{10.1287/mnsc.42.6.797} \href{https://www.doi2bib.org/bib/10.1287/mnsc.42.6.797}{(bib)} & journal-article & Eugeniusz Nowicki, Czeslaw Smutnicki. A Fast Taboo Search Algorithm for the Job Shop Problem. Management Science, 1996. & 16 & 0 & 16 & 0 & 633 &  0.00\\
\href{http://dx.doi.org/10.1016/s0377-2217(97)00305-6}{10.1016/s0377-2217(97)00305-6} \href{https://www.doi2bib.org/bib/10.1016/s0377-2217(97)00305-6}{(bib)} & journal-article & Bert De Reyck, willy Herroelen. A branch-and-bound procedure for the resource-constrained project scheduling problem with generalized precedence relations. European Journal of Operational Research, 1998. & 16 & 4 & 12 & 41 & 98 &  0.00\\
\href{http://dx.doi.org/10.1287/mnsc.49.3.330.12737}{10.1287/mnsc.49.3.330.12737} \href{https://www.doi2bib.org/bib/10.1287/mnsc.49.3.330.12737}{(bib)} & journal-article & Rolf H. Möhring, Andreas S. Schulz, Frederik Stork, Marc Uetz. Solving Project Scheduling Problems by Minimum Cut Computations. Management Science, 2003. & 16 & 3 & 13 & 55 & 135 &  0.00\\
\href{http://dx.doi.org/10.1016/0377-2217(87)90240-2}{10.1016/0377-2217(87)90240-2} \href{https://www.doi2bib.org/bib/10.1016/0377-2217(87)90240-2}{(bib)} & journal-article & Nicos Christofides, R. Alvarez-Valdes, J. M. Tamarit. Project scheduling with resource constraints: A branch and bound approach. European Journal of Operational Research, 1987. & 16 & 1 & 15 & 15 & 256 &  0.00\\
\href{http://dx.doi.org/10.1287/mnsc.41.10.1693}{10.1287/mnsc.41.10.1693} \href{https://www.doi2bib.org/bib/10.1287/mnsc.41.10.1693}{(bib)} & journal-article & Rainer Kolisch, Arno Sprecher, Andreas Drexl. Characterization and Generation of a General Class of Resource-Constrained Project Scheduling Problems. Management Science, 1995. & 16 & 0 & 16 & 0 & 436 &  0.00\\
\href{http://dx.doi.org/10.1016/j.compchemeng.2006.02.008}{10.1016/j.compchemeng.2006.02.008} \href{https://www.doi2bib.org/bib/10.1016/j.compchemeng.2006.02.008}{(bib)} & journal-article & Carlos A. Méndez, Jaime Cerdá, Ignacio E. Grossmann, Iiro Harjunkoski, Marco Fahl. State-of-the-art review of optimization methods for short-term scheduling of batch processes. Computers \& Chemical Engineering, 2006. & 16 & 5 & 11 & 116 & 642 &  0.00\\
\href{http://dx.doi.org/10.1080/0740817x.2012.705452}{10.1080/0740817x.2012.705452} \href{https://www.doi2bib.org/bib/10.1080/0740817x.2012.705452}{(bib)} & journal-article & Mohammad M. Fazel-Zarandi, Oded Berman, J. Christopher Beck. Solving a stochastic facility location/fleet management problem with logic-based Benders' decomposition. IIE Transactions, 2013. & 16 & 6 & 10 & 34 & 20 &  0.00\\
\href{http://dx.doi.org/10.1016/j.compchemeng.2005.09.011}{10.1016/j.compchemeng.2005.09.011} \href{https://www.doi2bib.org/bib/10.1016/j.compchemeng.2005.09.011}{(bib)} & journal-article & Christos T. Maravelias. A decomposition framework for the scheduling of single- and multi-stage processes. Computers \& Chemical Engineering, 2006. & 16 & 8 & 8 & 30 & 82 &  0.00\\
\href{http://dx.doi.org/10.1287/opre.46.3.316}{10.1287/opre.46.3.316} \href{https://www.doi2bib.org/bib/10.1287/opre.46.3.316}{(bib)} & journal-article & Cynthia Barnhart, Ellis L. Johnson, George L. Nemhauser, Martin W. P. Savelsbergh, Pamela H. Vance. Branch-and-Price: Column Generation for Solving Huge Integer Programs. Operations Research, 1998. & 16 & 0 & 16 & 44 & 1347 &  0.00\\
\href{http://dx.doi.org/10.1007/bf01539706}{10.1007/bf01539706} \href{https://www.doi2bib.org/bib/10.1007/bf01539706}{(bib)} & journal-article & Peter Brucker, Olaf Thiele. A branch / and  bound method for the general-shop problem with sequence dependent setup-times. OR Spektrum, 1996. & 15 & 0 & 15 & 15 & 73 &  0.00\\
\href{http://dx.doi.org/10.1287/opre.8.2.219}{10.1287/opre.8.2.219} \href{https://www.doi2bib.org/bib/10.1287/opre.8.2.219}{(bib)} & journal-article & Alan S. Manne. On the Job-Shop Scheduling Problem. Operations Research, 1960. & 15 & 0 & 15 & 0 & 425 &  0.00\\
\href{http://dx.doi.org/10.1016/j.cie.2009.02.012}{10.1016/j.cie.2009.02.012} \href{https://www.doi2bib.org/bib/10.1016/j.cie.2009.02.012}{(bib)} & journal-article & H. Fei, N. Meskens, C. Chu. A planning and scheduling problem for an operating theatre using an open scheduling strategy. Computers \& Industrial Engineering, 2010. & 14 & 0 & 14 & 40 & 181 &  0.00\\
\href{http://dx.doi.org/10.1016/0020-0255(74)90008-5}{10.1016/0020-0255(74)90008-5} \href{https://www.doi2bib.org/bib/10.1016/0020-0255(74)90008-5}{(bib)} & journal-article & Ugo Montanari. Networks of constraints: Fundamental properties and applications to picture processing. Information Sciences, 1974. & 14 & 0 & 14 & 14 & 753 &  0.00\\
\href{http://dx.doi.org/10.1016/s0167-5060(08)70743-x}{10.1016/s0167-5060(08)70743-x} \href{https://www.doi2bib.org/bib/10.1016/s0167-5060(08)70743-x}{(bib)} & book-chapter & J. K. Lenstra, A. H. G. Rinnooy Kan, P. Brucker. Complexity of Machine Scheduling Problems. Studies in Integer Programming, 1977. & 14 & 0 & 14 & 51 & 1358 &  0.00\\
\href{http://dx.doi.org/10.1126/science.220.4598.671}{10.1126/science.220.4598.671} \href{https://www.doi2bib.org/bib/10.1126/science.220.4598.671}{(bib)} & journal-article & S. Kirkpatrick, C. D. Gelatt, M. P. Vecchi. Optimization by Simulated Annealing. Science, 1983. & 14 & 0 & 14 & 32 & 29914 &  0.00\\
\href{http://dx.doi.org/10.1016/j.ejor.2009.04.011}{10.1016/j.ejor.2009.04.011} \href{https://www.doi2bib.org/bib/10.1016/j.ejor.2009.04.011}{(bib)} & journal-article & Brecht Cardoen, Erik Demeulemeester, Jeroen Beliën. Operating room planning and scheduling: A literature review. European Journal of Operational Research, 2010. & 14 & 0 & 14 & 124 & 748 &  0.00\\
\href{http://dx.doi.org/10.1287/mnsc.44.5.714}{10.1287/mnsc.44.5.714} \href{https://www.doi2bib.org/bib/10.1287/mnsc.44.5.714}{(bib)} & journal-article & Aristide Mingozzi, Vittorio Maniezzo, Salvatore Ricciardelli, Lucio Bianco. An Exact Algorithm for the Resource-Constrained Project Scheduling Problem Based on a New Mathematical Formulation. Management Science, 1998. & 14 & 2 & 12 & 29 & 238 &  0.00\\
\href{http://dx.doi.org/10.1007/978-3-540-24664-0_9}{10.1007/978-3-540-24664-0 9} \href{https://www.doi2bib.org/bib/10.1007/978-3-540-24664-0_9}{(bib)} & book-chapter & Yingyi Chu, Quanshi Xia. Generating Benders Cuts for a General Class of Integer Programming Problems. Integration of AI and OR Techniques in Constraint Programming for Combinatorial Optimization Problems, 2004. & 14 & 3 & 11 & 12 & 33 &  0.00\\
\href{http://dx.doi.org/10.1016/j.ejor.2019.10.014}{10.1016/j.ejor.2019.10.014} \href{https://www.doi2bib.org/bib/10.1016/j.ejor.2019.10.014}{(bib)} & journal-article & Bahman Naderi, Vahid Roshanaei. Branch-Relax-and-Check: A tractable decomposition method for order acceptance and identical parallel machine scheduling. European Journal of Operational Research, 2020. & 14 & 8 & 6 & 38 & 34 &  0.00\\
\href{http://dx.doi.org/10.1021/ie060449m}{10.1021/ie060449m} \href{https://www.doi2bib.org/bib/10.1021/ie060449m}{(bib)} & journal-article & Pedro M. Castro, Ignacio E. Grossmann, Augusto Q. Novais. Two New Continuous-Time Models for the Scheduling of Multistage Batch Plants with Sequence Dependent Changeovers. Industrial \& Engineering Chemistry Research, 2006. & 14 & 6 & 8 & 27 & 74 &  0.00\\
\href{http://dx.doi.org/10.1016/j.apm.2009.09.002}{10.1016/j.apm.2009.09.002} \href{https://www.doi2bib.org/bib/10.1016/j.apm.2009.09.002}{(bib)} & journal-article & Cemal Özgüven, Lale Özbakır, Yasemin Yavuz. Mathematical models for job-shop scheduling problems with routing and process plan flexibility. Applied Mathematical Modelling, 2010. & 14 & 2 & 12 & 38 & 206 &  0.00\\
\href{http://dx.doi.org/10.1016/j.omega.2019.03.001}{10.1016/j.omega.2019.03.001} \href{https://www.doi2bib.org/bib/10.1016/j.omega.2019.03.001}{(bib)} & journal-article & Vahid Roshanaei, Curtiss Luong, Dionne M. Aleman, David R. Urbach. Reformulation, linearization, and decomposition techniques for balanced distributed operating room scheduling. Omega, 2020. & 14 & 9 & 5 & 53 & 24 &  0.00\\
\href{http://dx.doi.org/10.1007/978-3-540-30201-8_41}{10.1007/978-3-540-30201-8 41} \href{https://www.doi2bib.org/bib/10.1007/978-3-540-30201-8_41}{(bib)} & book-chapter & Philippe Refalo. Impact-Based Search Strategies for Constraint Programming. Principles and Practice of Constraint Programming – CP 2004, 2004. & 14 & 0 & 14 & 15 & 120 &  0.00\\
\href{http://dx.doi.org/10.1145/79204.79210}{10.1145/79204.79210} \href{https://www.doi2bib.org/bib/10.1145/79204.79210}{(bib)} & journal-article & Alain Colmerauer. An introduction to Prolog III. Communications of the ACM, 1990. & 14 & 0 & 14 & 25 & 292 &  0.00\\
\href{http://dx.doi.org/10.1016/s0377-2217(99)00485-3}{10.1016/s0377-2217(99)00485-3} \href{https://www.doi2bib.org/bib/10.1016/s0377-2217(99)00485-3}{(bib)} & journal-article & Sönke Hartmann, Rainer Kolisch. Experimental evaluation of state-of-the-art heuristics for the resource-constrained project scheduling problem. European Journal of Operational Research, 2000. & 14 & 1 & 13 & 39 & 334 &  0.00\\
\href{http://dx.doi.org/10.1016/s0305-0483(00)00046-3}{10.1016/s0305-0483(00)00046-3} \href{https://www.doi2bib.org/bib/10.1016/s0305-0483(00)00046-3}{(bib)} & journal-article & R. Kolisch, R. Padman. An integrated survey of deterministic project scheduling. Omega, 2001. & 14 & 6 & 8 & 256 & 313 &  0.00\\
\href{http://dx.doi.org/10.1016/j.ejor.2006.03.059}{10.1016/j.ejor.2006.03.059} \href{https://www.doi2bib.org/bib/10.1016/j.ejor.2006.03.059}{(bib)} & journal-article & Dinh-Nguyen Pham, Andreas Klinkert. Surgical case scheduling as a generalized job shop scheduling problem. European Journal of Operational Research, 2008. & 14 & 1 & 13 & 41 & 236 &  0.00\\
\href{http://dx.doi.org/10.1016/j.cor.2009.12.011}{10.1016/j.cor.2009.12.011} \href{https://www.doi2bib.org/bib/10.1016/j.cor.2009.12.011}{(bib)} & journal-article & Oumar Koné, Christian Artigues, Pierre Lopez, Marcel Mongeau. Event-based MILP models for resource-constrained project scheduling problems. Computers \& Operations Research, 2011. & 13 & 5 & 8 & 40 & 132 &  0.00\\
\href{http://dx.doi.org/10.1007/s10845-007-0026-8}{10.1007/s10845-007-0026-8} \href{https://www.doi2bib.org/bib/10.1007/s10845-007-0026-8}{(bib)} & journal-article & Parviz Fattahi, Mohammad Saidi Mehrabad, Fariborz Jolai. Mathematical modeling and heuristic approaches to flexible job shop scheduling problems. Journal of Intelligent Manufacturing, 2007. & 13 & 0 & 13 & 16 & 249 &  0.00\\
\href{http://dx.doi.org/10.1016/s1574-6526(07)03004-0}{10.1016/s1574-6526(07)03004-0} \href{https://www.doi2bib.org/bib/10.1016/s1574-6526(07)03004-0}{(bib)} & book-chapter & Francesca Rossi, Peter van Beek, Toby Walsh. Chapter 4 Constraint Programming. Handbook of Knowledge Representation, 2008. & 13 & 6 & 7 & 149 & 33 &  0.00\\
\href{http://dx.doi.org/10.1016/j.ijpe.2004.12.006}{10.1016/j.ijpe.2004.12.006} \href{https://www.doi2bib.org/bib/10.1016/j.ijpe.2004.12.006}{(bib)} & journal-article & Aïda Jebali, Atidel B. Hadj Alouane, Pierre Ladet. Operating rooms scheduling. International Journal of Production Economics, 2006. & 13 & 0 & 13 & 22 & 245 &  0.00\\
\href{http://dx.doi.org/10.1016/j.ejor.2004.04.002}{10.1016/j.ejor.2004.04.002} \href{https://www.doi2bib.org/bib/10.1016/j.ejor.2004.04.002}{(bib)} & journal-article & Willy Herroelen, Roel Leus. Project scheduling under uncertainty: Survey and research potentials. European Journal of Operational Research, 2005. & 13 & 2 & 11 & 92 & 631 &  0.00\\
\href{http://dx.doi.org/10.1007/bf02023076}{10.1007/bf02023076} \href{https://www.doi2bib.org/bib/10.1007/bf02023076}{(bib)} & journal-article & Mauro Dell'Amico, Marco Trubian. Applying tabu search to the job-shop scheduling problem. Annals of Operations Research, 1993. & 13 & 0 & 13 & 19 & 383 &  0.00\\
\href{http://dx.doi.org/10.1016/j.omega.2015.02.001}{10.1016/j.omega.2015.02.001} \href{https://www.doi2bib.org/bib/10.1016/j.omega.2015.02.001}{(bib)} & journal-article & David Wheatley, Fatma Gzara, Elizabeth Jewkes. Logic-based Benders decomposition for an inventory-location problem with service constraints. Omega, 2015. & 13 & 6 & 7 & 23 & 46 &  0.00\\
\href{http://dx.doi.org/10.1002/nav.3800010110}{10.1002/nav.3800010110} \href{https://www.doi2bib.org/bib/10.1002/nav.3800010110}{(bib)} & journal-article & S. M. Johnson. Optimal two‐ and three‐stage production schedules with setup times included. Naval Research Logistics Quarterly, 1954. & 13 & 0 & 13 & 2 & 2229 &  0.00\\
\href{http://dx.doi.org/10.1002/nav.3800060205}{10.1002/nav.3800060205} \href{https://www.doi2bib.org/bib/10.1002/nav.3800060205}{(bib)} & journal-article & Harvey M. Wagner. An integer linear‐programming model for machine scheduling. Naval Research Logistics Quarterly, 1959. & 13 & 0 & 13 & 14 & 257 &  0.00\\
\href{http://dx.doi.org/10.1007/978-3-540-30201-8_36}{10.1007/978-3-540-30201-8 36} \href{https://www.doi2bib.org/bib/10.1007/978-3-540-30201-8_36}{(bib)} & book-chapter & Gilles Pesant. A Regular Language Membership Constraint for Finite Sequences of Variables. Principles and Practice of Constraint Programming – CP 2004, 2004. & 13 & 0 & 13 & 15 & 132 &  0.00\\
\href{http://dx.doi.org/10.1287/opre.46.1.1}{10.1287/opre.46.1.1} \href{https://www.doi2bib.org/bib/10.1287/opre.46.1.1}{(bib)} & journal-article & George L. Nemhauser, Michael A. Trick. Scheduling A Major College Basketball Conference. Operations Research, 1998. & 13 & 0 & 13 & 13 & 156 &  0.00\\
\href{http://dx.doi.org/10.1016/j.ejor.2009.03.034}{10.1016/j.ejor.2009.03.034} \href{https://www.doi2bib.org/bib/10.1016/j.ejor.2009.03.034}{(bib)} & journal-article & Vincent Van Peteghem, Mario Vanhoucke. A genetic algorithm for the preemptive and non-preemptive multi-mode resource-constrained project scheduling problem. European Journal of Operational Research, 2010. & 13 & 0 & 13 & 59 & 220 &  0.00\\
\href{http://dx.doi.org/10.1007/s10479-010-0693-2}{10.1007/s10479-010-0693-2} \href{https://www.doi2bib.org/bib/10.1007/s10479-010-0693-2}{(bib)} & journal-article & Andrei Horbach. A Boolean satisfiability approach to the resource-constrained project scheduling problem. Annals of Operations Research, 2010. & 12 & 6 & 6 & 46 & 26 &  0.00\\
\href{http://dx.doi.org/10.1016/j.omega.2019.01.003}{10.1016/j.omega.2019.01.003} \href{https://www.doi2bib.org/bib/10.1016/j.omega.2019.01.003}{(bib)} & journal-article & Ramin Barzanji, Bahman Naderi, Mehmet A. Begen. Decomposition algorithms for the integrated process planning and scheduling problem. Omega, 2020. & 12 & 2 & 10 & 60 & 41 &  0.00\\
\href{http://dx.doi.org/10.1287/mnsc.28.10.1197}{10.1287/mnsc.28.10.1197} \href{https://www.doi2bib.org/bib/10.1287/mnsc.28.10.1197}{(bib)} & journal-article & F. Brian Talbot. Resource-Constrained Project Scheduling with Time-Resource Tradeoffs: The Nonpreemptive Case. Management Science, 1982. & 12 & 0 & 12 & 0 & 296 &  0.00\\
\href{http://dx.doi.org/10.1287/opre.40.1.113}{10.1287/opre.40.1.113} \href{https://www.doi2bib.org/bib/10.1287/opre.40.1.113}{(bib)} & journal-article & Peter J. M. van Laarhoven, Emile H. L. Aarts, Jan Karel Lenstra. Job Shop Scheduling by Simulated Annealing. Operations Research, 1992. & 12 & 0 & 12 & 0 & 742 &  0.00\\
\href{http://dx.doi.org/10.1016/s0377-2217(02)00763-4}{10.1016/s0377-2217(02)00763-4} \href{https://www.doi2bib.org/bib/10.1016/s0377-2217(02)00763-4}{(bib)} & journal-article & J. Carlier, E. Néron. On linear lower bounds for the resource constrained project scheduling problem. European Journal of Operational Research, 2003. & 12 & 4 & 8 & 27 & 34 &  0.00\\
\end{longtable}
}



\clearpage
\section{Excluded Work}

The following entries have not been included, as they are of the wrong type or are incomplete. Some of these might be highly relevant, but they will need some manual work to be included in the survey.

{\scriptsize
\begin{longtable}{p{5cm}lp{11cm}rrrrrr}
\caption{Excluded Work (Total 22740 Works Checked, 3 Selected)}\\ \toprule
DOI & Type & Authors/Title & \shortstack{Nr\\Links} & \shortstack{Citing\\Survey} & \shortstack{Cited by\\Survey} & \shortstack{XRef\\Refs} & \shortstack{XRef\\Cite} & Relevance\\ \midrule\endhead
\bottomrule
\endfoot
\href{http://dx.doi.org/10.1002/0470018860.s00026}{10.1002/0470018860.s00026} \href{https://www.doi2bib.org/bib/10.1002/0470018860.s00026}{(bib)} & other & Rina Dechter, Francesca Rossi. Constraint Satisfaction. Encyclopedia of Cognitive Science, 2006. \hyperref[mw:mw15640]{Abstract} & 2 & 2 & 0 & 34 & 0 &  3.75\\
\href{http://dx.doi.org/10.21236/ada293583}{10.21236/ada293583} \href{https://www.doi2bib.org/bib/10.21236/ada293583}{(bib)} & report & Cheng-Chung Cheng, Stephen F. Smith. Applying Constraint Satisfaction Techniques to Job Shop Scheduling.. , 1995. & 1 & 0 & 1 & 0 & 9 &  2.00\\
\href{http://dx.doi.org/10.1002/9780470400531.eorms0088}{10.1002/9780470400531.eorms0088} \href{https://www.doi2bib.org/bib/10.1002/9780470400531.eorms0088}{(bib)} & other & Laurent Michel, Pascal Van Hentenryck. BasicCPTheory: Search. Wiley Encyclopedia of Operations Research and Management Science, 2011. \hyperref[mw:mw19349]{Abstract} & 2 & 2 & 0 & 65 & 1 &  2.00\\
\end{longtable}
}



\clearpage
\section{Missing DOI}

The following works do not have a DOI entry in their bib entry. This might be due to an non-standard or defunct publisher, but for many works a DOI should exist. It would be helpful to add these to the bib entries, so that the OpenCitation and Crossref lookup procedures can report the appropriate numbers for citations and references.

{\scriptsize
\begin{longtable}{>{\raggedright\arraybackslash}p{2.5cm}>{\raggedright\arraybackslash}p{4.5cm}>{\raggedright\arraybackslash}p{6.0cm}p{1.0cm}rr>{\raggedright\arraybackslash}p{2.0cm}r>{\raggedright\arraybackslash}p{1cm}p{1cm}p{1cm}p{1cm}}
\rowcolor{white}\caption{Works with Missing DOI (Total 129)}\\ \toprule
\rowcolor{white}\shortstack{Key\\Source} & Authors & Title (Colored by Open Access)& \shortstack{Details\\LC} & Cite & Year & \shortstack{Conference\\/Journal\\/School} & Pages & Relevance &\shortstack{Cites\\OC XR\\SC} & \shortstack{Refs\\OC\\XR} & \shortstack{Links\\Cites\\Refs}\\ \midrule\endhead
\bottomrule
\endfoot
AbdennadherS99 \href{http://www.aaai.org/Library/IAAI/1999/iaai99-118.php}{AbdennadherS99} & \hyperref[auth:a1316]{S. Abdennadher}, \hyperref[auth:a710]{H. Schlenker} & Nurse Scheduling using Constraint Logic Programming & \hyperref[detail:AbdennadherS99]{Details} \href{../works/AbdennadherS99.pdf}{Yes} & \cite{AbdennadherS99} & 1999 & AAAI 1999 & 6 & \noindent{}\textbf{1.00} \textbf{1.00} 0.75 & 0 0 0 & 0 0 & 0 0 0\\
Arkhipov19 \href{http://www.theses.fr/2019TOU30107}{Arkhipov19} & \hyperref[auth:a1035]{D. Arkhipov} & Planification socio-responsable du travail dans les chaînes de montage d'aéronefs : comment satisfaire à la fois objectifs ergonomiques et économiques & \cellcolor{red!30}\hyperref[detail:Arkhipov19]{Details} No & \cite{Arkhipov19} & 2019 & Toulouse 3 & null & \noindent{}\textcolor{black!50}{0.00} \textcolor{black!50}{0.00} n/a & 0 0 0 & 0 0 & 0 0 0\\
Astrand21 \href{https://nbn-resolving.org/urn:nbn:se:kth:diva-294959}{Astrand21} & \hyperref[auth:a74]{M. {\AA}strand} & Short-term Underground Mine Scheduling: An Industrial Application of Constraint Programming & \hyperref[detail:Astrand21]{Details} \href{../works/Astrand21.pdf}{Yes} & \cite{Astrand21} & 2021 & Royal Institute of Technology, Stockholm, Sweden & 142 & \noindent{}\textbf{1.00} \textbf{1.00} \textbf{310.47} & 0 0 0 & 0 0 & 0 0 0\\
Baptiste02 \href{https://theses.hal.science/tel-00124998}{Baptiste02} & \hyperref[auth:a162]{P. Baptiste} & {R{\'e}sultats de complexit{\'e} et programmation par contraintes pour l'ordonnancement} & \hyperref[detail:Baptiste02]{Details} \href{../works/Baptiste02.pdf}{Yes} & \cite{Baptiste02} & 2002 & {Universit{\'e} de Technologie de Compi{\`e}gne} & 237 & \noindent{}\textcolor{black!50}{0.00} \textcolor{black!50}{0.00} \textbf{1096.83} & 0 0 0 & 0 0 & 0 0 0\\
BaptisteP95 \href{http://ijcai.org/Proceedings/95-1/Papers/079.pdf}{BaptisteP95} & \hyperref[auth:a162]{P. Baptiste}, \hyperref[auth:a163]{C. L. Pape} & A Theoretical and Experimental Comparison of Constraint Propagation Techniques for Disjunctive Scheduling & \hyperref[detail:BaptisteP95]{Details} \href{../works/BaptisteP95.pdf}{Yes} & \cite{BaptisteP95} & 1995 & IJCAI 1995 & 7 & \noindent{}\textbf{1.50} \textbf{1.50} \textbf{3.24} & 0 0 0 & 0 0 & 0 0 0\\
BarbulescuWH04 \href{http://www.aaai.org/Library/AAAI/2004/aaai04-023.php}{BarbulescuWH04} & \hyperref[auth:a1313]{L. Barbulescu}, \hyperref[auth:a1315]{L. D. Whitley}, \hyperref[auth:a1314]{A. E. Howe} & Leap Before You Look: An Effective Strategy in an Oversubscribed Scheduling Problem & \hyperref[detail:BarbulescuWH04]{Details} \href{../works/BarbulescuWH04.pdf}{Yes} & \cite{BarbulescuWH04} & 2004 & AAAI 2004 & 6 & \noindent{}\textcolor{black!50}{0.00} \textcolor{black!50}{0.00} 0.62 & 0 0 0 & 0 0 & 0 0 0\\
Bartak14 \href{}{Bartak14} & \hyperref[auth:a152]{R. Bart{\'{a}}k} & Planning and Scheduling & \cellcolor{red!30}\hyperref[detail:Bartak14]{Details} No & \cite{Bartak14} & 2014 & Computing Handbook, Third Edition: Computer Science and Software Engineering & null & \noindent{}\textcolor{black!50}{0.00} \textcolor{black!50}{0.00} n/a & 0 0 0 & 0 0 & 0 0 0\\
Beck06 \href{http://www.aaai.org/Library/ICAPS/2006/icaps06-028.php}{Beck06} & \hyperref[auth:a89]{J. C. Beck} & An Empirical Study of Multi-Point Constructive Search for Constraint-Based Scheduling & \hyperref[detail:Beck06]{Details} \href{../works/Beck06.pdf}{Yes} & \cite{Beck06} & 2006 & ICAPS 2006 & 10 & \noindent{}\textcolor{black!50}{0.00} \textcolor{black!50}{0.00} 0.71 & 0 0 0 & 0 0 & 0 0 0\\
Beck99 \href{https://librarysearch.library.utoronto.ca/permalink/01UTORONTO_INST/14bjeso/alma991106162342106196}{Beck99} & \hyperref[auth:a89]{J. C. Beck} & Texture measurements as a basis for heuristic commitment techniques in constraint-directed scheduling & \hyperref[detail:Beck99]{Details} \href{../works/Beck99.pdf}{Yes} & \cite{Beck99} & 1999 & University of Toronto, Canada & 418 & \noindent{}\textcolor{black!50}{0.00} \textcolor{black!50}{0.00} \textbf{270.43} & 0 0 0 & 0 0 & 0 0 0\\
BeckDSF97 \href{http://www.aaai.org/Library/AAAI/1997/aaai97-037.php}{BeckDSF97} & \hyperref[auth:a89]{J. C. Beck}, \hyperref[auth:a248]{A. J. Davenport}, \hyperref[auth:a1286]{E. M. Sitarski}, \hyperref[auth:a302]{M. S. Fox} & Beyond Contention: Extending Texture-Based Scheduling Heuristics & \hyperref[detail:BeckDSF97]{Details} \href{../works/BeckDSF97.pdf}{Yes} & \cite{BeckDSF97} & 1997 & AAAI 1997 & 8 & \noindent{}\textcolor{black!50}{0.00} \textcolor{black!50}{0.00} 0.79 & 0 0 0 & 0 0 & 0 0 0\\
BeckDSF97a \href{http://www.aaai.org/Library/AAAI/1997/aaai97-038.php}{BeckDSF97a} & \hyperref[auth:a89]{J. C. Beck}, \hyperref[auth:a248]{A. J. Davenport}, \hyperref[auth:a1286]{E. M. Sitarski}, \hyperref[auth:a302]{M. S. Fox} & Texture-Based Heuristics for Scheduling Revisited & \hyperref[detail:BeckDSF97a]{Details} \href{../works/BeckDSF97a.pdf}{Yes} & \cite{BeckDSF97a} & 1997 & AAAI 1997 & 8 & \noindent{}\textcolor{black!50}{0.00} \textcolor{black!50}{0.00} \textbf{1.58} & 0 0 0 & 0 0 & 0 0 0\\
BeckF99 \href{http://www.aaai.org/Library/AAAI/1999/aaai99-097.php}{BeckF99} & \hyperref[auth:a89]{J. C. Beck}, \hyperref[auth:a302]{M. S. Fox} & Scheduling Alternative Activities & \hyperref[detail:BeckF99]{Details} \href{../works/BeckF99.pdf}{Yes} & \cite{BeckF99} & 1999 & AAAI 1999 & 8 & \noindent{}\textcolor{black!50}{0.00} \textcolor{black!50}{0.00} \textbf{3.06} & 0 0 0 & 0 0 & 0 0 0\\
BeckPS03 \href{http://www.aaai.org/Library/ICAPS/2003/icaps03-027.php}{BeckPS03} & \hyperref[auth:a89]{J. C. Beck}, \hyperref[auth:a826]{P. Prosser}, \hyperref[auth:a827]{E. Selensky} & Vehicle Routing and Job Shop Scheduling: What's the Difference? & \hyperref[detail:BeckPS03]{Details} \href{../works/BeckPS03.pdf}{Yes} & \cite{BeckPS03} & 2003 & ICAPS 2003 & 10 & \noindent{}\textcolor{black!50}{0.00} \textcolor{black!50}{0.00} \textbf{2.92} & 0 0 0 & 0 0 & 0 0 0\\
BeckW04 \href{}{BeckW04} & \hyperref[auth:a89]{J. C. Beck}, \hyperref[auth:a825]{N. Wilson} & Job Shop Scheduling with Probabilistic Durations & \hyperref[detail:BeckW04]{Details} \href{../works/BeckW04.pdf}{Yes} & \cite{BeckW04} & 2004 & ECAI 2004 & 5 & \noindent{}\textcolor{black!50}{0.00} \textcolor{black!50}{0.00} 0.34 & 0 0 0 & 0 0 & 0 0 0\\
BeckW05 \href{http://ijcai.org/Proceedings/05/Papers/0748.pdf}{BeckW05} & \hyperref[auth:a89]{J. C. Beck}, \hyperref[auth:a825]{N. Wilson} & Proactive Algorithms for Scheduling with Probabilistic Durations & \hyperref[detail:BeckW05]{Details} \href{../works/BeckW05.pdf}{Yes} & \cite{BeckW05} & 2005 & IJCAI 2005 & 6 & \noindent{}\textcolor{black!50}{0.00} \textcolor{black!50}{0.00} 0.29 & 0 0 0 & 0 0 & 0 0 0\\
BeniniBGM05a \href{http://ijcai.org/Proceedings/05/Papers/post-0368.pdf}{BeniniBGM05a} & \hyperref[auth:a245]{L. Benini}, \hyperref[auth:a375]{D. Bertozzi}, \hyperref[auth:a376]{A. Guerri}, \hyperref[auth:a143]{M. Milano} & Allocation and Scheduling for MPSoCs via decomposition and no-good generation & \hyperref[detail:BeniniBGM05a]{Details} \href{../works/BeniniBGM05a.pdf}{Yes} & \cite{BeniniBGM05a} & 2005 & IJCAI 2005 & 2 & \noindent{}\textcolor{black!50}{0.00} \textcolor{black!50}{0.00} 0.80 & 0 0 0 & 0 0 & 0 0 0\\
BidotVLB07 \href{http://ijcai.org/Proceedings/07/Papers/007.pdf}{BidotVLB07} & \hyperref[auth:a823]{J. Bidot}, \hyperref[auth:a824]{T. Vidal}, \hyperref[auth:a118]{P. Laborie}, \hyperref[auth:a89]{J. C. Beck} & A General Framework for Scheduling in a Stochastic Environment & \hyperref[detail:BidotVLB07]{Details} \href{../works/BidotVLB07.pdf}{Yes} & \cite{BidotVLB07} & 2007 & IJCAI 2007 & 6 & \noindent{}\textcolor{black!50}{0.00} \textcolor{black!50}{0.00} 0.21 & 0 0 0 & 0 0 & 0 0 0\\
BonfiettiM12 \href{https://ceur-ws.org/Vol-926/paper2.pdf}{BonfiettiM12} & \hyperref[auth:a198]{A. Bonfietti}, \hyperref[auth:a143]{M. Milano} & A Constraint-based Approach to Cyclic Resource-Constrained Scheduling Problem & \hyperref[detail:BonfiettiM12]{Details} \href{../works/BonfiettiM12.pdf}{Yes} & \cite{BonfiettiM12} & 2012 & DC SIAAI 2012 & 3 & \noindent{}\textcolor{black!50}{0.00} \textcolor{black!50}{0.00} \textcolor{black!50}{0.06} & 0 0 0 & 0 0 & 0 0 0\\
BoucherBVBL97 \href{}{BoucherBVBL97} & \hyperref[auth:a689]{E. Boucher}, \hyperref[auth:a690]{A. Bachelu}, \hyperref[auth:a691]{C. Varnier}, \hyperref[auth:a692]{P. Baptiste}, \hyperref[auth:a693]{B. Legeard} & Multi-criteria Comparison Between Algorithmic, Constraint Logic and Specific Constraint Programming on a Real Schedulingt Problem & \cellcolor{red!30}\hyperref[detail:BoucherBVBL97]{Details} No & \cite{BoucherBVBL97} & 1997 & ACT 1997 & 18 & \noindent{}\textbf{1.00} \textbf{1.00} n/a & 0 0 0 & 0 0 & 0 0 0\\
BreitingerL95 \href{}{BreitingerL95} & \hyperref[auth:a694]{S. Breitinger}, \hyperref[auth:a695]{H. C. R. Lock} & Using Constraint Logic Programming for Industrial Scheduling Problems & \cellcolor{red!30}\hyperref[detail:BreitingerL95]{Details} No & \cite{BreitingerL95} & 1995 & Logic Programming: Formal Methods and Practical Applications, Studies in Computer Science and Artificial Intelligence & 27 & \noindent{}\textbf{1.00} \textbf{1.00} n/a & 0 0 0 & 0 0 & 0 0 0\\
Caballero19 \href{https://www.tesisenred.net/handle/10803/667963#page=1}{Caballero19} & \hyperref[auth:a102]{J. C. Caballero} & Scheduling Through Logic-Based Tools & \hyperref[detail:Caballero19]{Details} \href{../works/Caballero19.pdf}{Yes} & \cite{Caballero19} & 2019 & Universitat de Girona, Spain & 194 & \noindent{}\textcolor{black!50}{0.00} \textcolor{black!50}{0.00} \textbf{43.36} & 0 0 0 & 0 0 & 0 0 0\\
CatusseCBL16 \href{http://www.ijcai.org/Abstract/16/434}{CatusseCBL16} & \hyperref[auth:a996]{N. Catusse}, \hyperref[auth:a997]{H. Cambazard}, \hyperref[auth:a998]{N. Brauner}, \hyperref[auth:a977]{P. Lemaire}, \hyperref[auth:a999]{B. Penz}, \hyperref[auth:a1000]{A.-M. Lagrange}, \hyperref[auth:a1001]{P. Rubini} & A Branch-and-Price Algorithm for Scheduling Observations on a Telescope & \hyperref[detail:CatusseCBL16]{Details} \href{../works/CatusseCBL16.pdf}{Yes} & \cite{CatusseCBL16} & 2016 & IJCAI 2016 & 7 & \noindent{}\textcolor{black!50}{0.00} \textcolor{black!50}{0.00} \textcolor{black!50}{0.14} & 0 0 0 & 0 0 & 0 0 0\\
CestaOF99 \href{http://ijcai.org/Proceedings/99-2/Papers/051.pdf}{CestaOF99} & \hyperref[auth:a284]{A. Cesta}, \hyperref[auth:a282]{A. Oddi}, \hyperref[auth:a298]{S. F. Smith} & An Iterative Sampling Procedure for Resource Constrained Project Scheduling with Time Windows & \hyperref[detail:CestaOF99]{Details} \href{../works/CestaOF99.pdf}{Yes} & \cite{CestaOF99} & 1999 & IJCAI 1999 & 12 & \noindent{}\textcolor{black!50}{0.00} \textcolor{black!50}{0.00} \textbf{2.07} & 0 0 0 & 0 0 & 0 0 0\\
CestaOS00 \href{http://www.aaai.org/Library/AAAI/2000/aaai00-114.php}{CestaOS00} & \hyperref[auth:a284]{A. Cesta}, \hyperref[auth:a282]{A. Oddi}, \hyperref[auth:a298]{S. F. Smith} & Iterative Flattening: {A} Scalable Method for Solving Multi-Capacity Scheduling Problems & \hyperref[detail:CestaOS00]{Details} \href{../works/CestaOS00.pdf}{Yes} & \cite{CestaOS00} & 2000 & AAAI 2000 & 6 & \noindent{}\textcolor{black!50}{0.00} \textcolor{black!50}{0.00} \textbf{1.07} & 0 0 0 & 0 0 & 0 0 0\\
ChuGNSW13 \href{http://www.aaai.org/ocs/index.php/IJCAI/IJCAI13/paper/view/6878}{ChuGNSW13} & \hyperref[auth:a343]{G. Chu}, \hyperref[auth:a792]{S. Gaspers}, \hyperref[auth:a793]{N. Narodytska}, \hyperref[auth:a124]{A. Schutt}, \hyperref[auth:a276]{T. Walsh} & On the Complexity of Global Scheduling Constraints under Structural Restrictions & \hyperref[detail:ChuGNSW13]{Details} \href{../works/ChuGNSW13.pdf}{Yes} & \cite{ChuGNSW13} & 2013 & IJCAI 2013 & 7 & \noindent{}\textcolor{black!50}{0.00} \textcolor{black!50}{0.00} \textbf{2.43} & 0 0 0 & 0 0 & 0 0 0\\
ChunCTY99 \href{http://www.aaai.org/Library/IAAI/1999/iaai99-111.php}{ChunCTY99} & \hyperref[auth:a1321]{A. H. W. Chun}, \hyperref[auth:a1322]{S. H. C. Chan}, \hyperref[auth:a1323]{F. M. F. Tsang}, \hyperref[auth:a1324]{D. W. M. Yeung} & {HKIA} {SAS:} {A} Constraint-Based Airport Stand Allocation System Developed with Software Components & \hyperref[detail:ChunCTY99]{Details} \href{../works/ChunCTY99.pdf}{Yes} & \cite{ChunCTY99} & 1999 & AAAI 1999 & 8 & \noindent{}\textcolor{black!50}{0.00} \textcolor{black!50}{0.00} 0.66 & 0 0 0 & 0 0 & 0 0 0\\
Clercq12 \href{https://theses.hal.science/tel-00794323}{Clercq12} & \hyperref[auth:a246]{A. D. Clercq} & {Ordonnancement cumulatif avec d{\'e}passements de capacit{\'e} : Contrainte globale et d{\'e}compositions} & \hyperref[detail:Clercq12]{Details} \href{../works/Clercq12.pdf}{Yes} & \cite{Clercq12} & 2012 & {Ecole des Mines de Nantes} & 196 & \noindent{}\textcolor{black!50}{0.00} \textcolor{black!50}{0.00} \textbf{6.18} & 0 0 0 & 0 0 & 0 0 0\\
CrawfordB94 \href{http://www.aaai.org/Library/AAAI/1994/aaai94-168.php}{CrawfordB94} & \hyperref[auth:a1276]{J. M. Crawford}, \hyperref[auth:a1277]{A. B. Baker} & Experimental Results on the Application of Satisfiability Algorithms to Scheduling Problems & \hyperref[detail:CrawfordB94]{Details} \href{../works/CrawfordB94.pdf}{Yes} & \cite{CrawfordB94} & 1994 & AAAI 1994 & 6 & \noindent{}\textcolor{black!50}{0.00} \textcolor{black!50}{0.00} 0.37 & 0 0 0 & 0 0 & 0 0 0\\
Dejemeppe16 \href{https://hdl.handle.net/2078.1/178078}{Dejemeppe16} & \hyperref[auth:a202]{C. Dejemeppe} & Constraint programming algorithms and models for scheduling applications & \hyperref[detail:Dejemeppe16]{Details} \href{../works/Dejemeppe16.pdf}{Yes} & \cite{Dejemeppe16} & 2016 & Catholic University of Louvain, Louvain-la-Neuve, Belgium & 274 & \noindent{}\textbf{1.00} \textbf{1.00} \textbf{262.14} & 0 0 0 & 0 0 & 0 0 0\\
Demassey03 \href{https://tel.archives-ouvertes.fr/tel-00293564}{Demassey03} & \hyperref[auth:a243]{S. Demassey} & M{\'{e}}thodes hybrides de programmation par contraintes et programmation lin{\'{e}}aire pour le probl{\`{e}}me d'ordonnancement de projet {\`{a}} contraintes de ressources. (Hybrid Constraint Programming-Integer Linear Programming approaches for the Resource-Constrained Project Scheduling Problem) & \hyperref[detail:Demassey03]{Details} \href{../works/Demassey03.pdf}{Yes} & \cite{Demassey03} & 2003 & University of Avignon, France & 148 & \noindent{}\textbf{1.50} \textbf{1.50} \textbf{15.90} & 0 0 0 & 0 0 & 0 0 0\\
Derrien15 \href{https://tel.archives-ouvertes.fr/tel-01242789}{Derrien15} & \hyperref[auth:a220]{A. Derrien} & Ordonnancement cumulatif en programmation par contraintes : caract{\'{e}}risation {\'{e}}nerg{\'{e}}tique des raisonnements et solutions robustes. (Cumulative scheduling in constraint programming : energetic characterization of reasoning and robust solutions) & \hyperref[detail:Derrien15]{Details} \href{../works/Derrien15.pdf}{Yes} & \cite{Derrien15} & 2015 & {\'{E}}cole des mines de Nantes, France & 113 & \noindent{}\textbf{1.00} \textbf{1.00} \textbf{3.83} & 0 0 0 & 0 0 & 0 0 0\\
DilkinaH04 \href{}{DilkinaH04} & \hyperref[auth:a267]{B. N. Dilkina}, \hyperref[auth:a269]{W. S. Havens} & The {U.S.} National Football League Scheduling Problem & \hyperref[detail:DilkinaH04]{Details} \href{../works/DilkinaH04.pdf}{Yes} & \cite{DilkinaH04} & 2004 & AAAI 2004 & 6 & \noindent{}\textcolor{black!50}{0.00} \textcolor{black!50}{0.00} 0.89 & 0 0 0 & 0 0 & 0 0 0\\
DincbasHSAGB88 \href{}{DincbasHSAGB88} & \hyperref[auth:a716]{M. Dincbas}, \hyperref[auth:a148]{P. V. Hentenryck}, \hyperref[auth:a17]{H. Simonis}, \hyperref[auth:a724]{A. Aggoun}, T. Graf, F. Berthier & The Constraint Logic Programming Language {CHIP} & \hyperref[detail:DincbasHSAGB88]{Details} \href{../works/DincbasHSAGB88.pdf}{Yes} & \cite{DincbasHSAGB88} & 1988 & FGCS 1988 & 10 & \noindent{}\textcolor{black!50}{0.00} \textcolor{black!50}{0.00} \textcolor{black!50}{0.00} & 0 0 0 & 0 0 & 0 0 0\\
DincbasS91 \href{}{DincbasS91} & \hyperref[auth:a716]{M. Dincbas}, \hyperref[auth:a17]{H. Simonis} & Apache-a constraint based, automated stand allocation system & \hyperref[detail:DincbasS91]{Details} \href{../works/DincbasS91.pdf}{Yes} & \cite{DincbasS91} & 1991 & ASTAIR91 1991 & 13 & \noindent{}\textcolor{black!50}{0.00} \textcolor{black!50}{0.00} \textcolor{black!50}{0.12} & 0 0 0 & 0 0 & 0 0 0\\
DoRZ08 \href{http://www.aaai.org/Library/AAAI/2008/aaai08-253.php}{DoRZ08} & \hyperref[auth:a1344]{M. B. Do}, \hyperref[auth:a1345]{W. Ruml}, \hyperref[auth:a1346]{R. Zhou} & On-line Planning and Scheduling: An Application to Controlling Modular Printers & \hyperref[detail:DoRZ08]{Details} \href{../works/DoRZ08.pdf}{Yes} & \cite{DoRZ08} & 2008 & AAAI 2008 & 5 & \noindent{}\textcolor{black!50}{0.00} \textcolor{black!50}{0.00} \textcolor{black!50}{0.13} & 0 0 0 & 0 0 & 0 0 0\\
DraperJCJ99 \href{http://ijcai.org/Proceedings/99-2/Papers/050.pdf}{DraperJCJ99} & \hyperref[auth:a1438]{D. Draper}, \hyperref[auth:a1040]{A. K. J{\'{o}}nsson}, \hyperref[auth:a1439]{D. P. Clements}, \hyperref[auth:a1440]{D. Joslin} & Cyclic Scheduling & \hyperref[detail:DraperJCJ99]{Details} \href{../works/DraperJCJ99.pdf}{Yes} & \cite{DraperJCJ99} & 1999 & IJCAI 1999 & 6 & \noindent{}\textcolor{black!50}{0.00} \textcolor{black!50}{0.00} \textbf{2.43} & 0 0 0 & 0 0 & 0 0 0\\
ElhouraniDM07 \href{http://www.aaai.org/Library/AAAI/2007/aaai07-213.php}{ElhouraniDM07} & \hyperref[auth:a1341]{T. Elhourani}, \hyperref[auth:a1342]{N. Denny}, \hyperref[auth:a1343]{M. M. Marefat} & A Distributed Constraint Optimization Solution to the {P2P} Video Streaming Problem & \hyperref[detail:ElhouraniDM07]{Details} \href{../works/ElhouraniDM07.pdf}{Yes} & \cite{ElhouraniDM07} & 2007 & AAAI 2007 & 6 & \noindent{}\textcolor{black!50}{0.00} \textcolor{black!50}{0.00} \textcolor{black!50}{0.01} & 0 0 0 & 0 0 & 0 0 0\\
Elkhyari03 \href{https://theses.hal.science/tel-00008377}{Elkhyari03} & \hyperref[auth:a292]{A. Elkhyari} & {Outils d'aide {\`a} la d{\'e}cision pour des probl{\`e}mes d'ordonnancement dynamiques} & \hyperref[detail:Elkhyari03]{Details} \href{../works/Elkhyari03.pdf}{Yes} & \cite{Elkhyari03} & 2003 & {Universit{\'e} de Nantes} & 333 & \noindent{}\textcolor{black!50}{0.00} \textcolor{black!50}{0.00} \textbf{24.65} & 0 0 0 & 0 0 & 0 0 0\\
EskeyZ90 \href{http://www.aaai.org/Library/AAAI/1990/aaai90-136.php}{EskeyZ90} & \hyperref[auth:a1272]{M. Eskey}, \hyperref[auth:a1273]{M. Zweben} & Learning Search Control for Constraint-Based Scheduling & \hyperref[detail:EskeyZ90]{Details} \href{../works/EskeyZ90.pdf}{Yes} & \cite{EskeyZ90} & 1990 & AAAI 1990 & 8 & \noindent{}\textcolor{black!50}{0.00} \textcolor{black!50}{0.00} 0.74 & 0 0 0 & 0 0 & 0 0 0\\
EvenSH15a \href{http://arxiv.org/abs/1505.02487}{EvenSH15a} & \hyperref[auth:a214]{C. Even}, \hyperref[auth:a124]{A. Schutt}, \hyperref[auth:a148]{P. V. Hentenryck} & A Constraint Programming Approach for Non-Preemptive Evacuation Scheduling & \hyperref[detail:EvenSH15a]{Details} \href{../works/EvenSH15a.pdf}{Yes} & \cite{EvenSH15a} & 2015 & CoRR & 16 & \noindent{}\textbf{1.00} \textbf{1.00} 0.42 & 0 0 0 & 0 0 & 0 0 0\\
Fahimi16 \href{http://cp2014.a4cp.org/sites/default/files/hamed_fahimi_-_efficient_algorithms_to_solve_scheduling_problems_with_a_variety_of_optimization_criteria.pdf}{Fahimi16} & \hyperref[auth:a122]{H. Fahimi} & Efficient algorithms to solve scheduling problems with a variety of optimization criteria & \hyperref[detail:Fahimi16]{Details} \href{../works/Fahimi16.pdf}{Yes} & \cite{Fahimi16} & 2016 & Universit{\'{e}} Laval, Quebec, Canada & 120 & \noindent{}\textcolor{black!50}{0.00} \textcolor{black!50}{0.00} \textbf{142.81} & 0 0 0 & 0 0 & 0 0 0\\
FeldmanG89 \href{http://ijcai.org/Proceedings/89-2/Papers/026.pdf}{FeldmanG89} & \hyperref[auth:a1434]{R. Feldman}, \hyperref[auth:a1435]{M. C. Golumbic} & Constraint Satisfiability Algorithms for Interactive Student Scheduling & \hyperref[detail:FeldmanG89]{Details} \href{../works/FeldmanG89.pdf}{Yes} & \cite{FeldmanG89} & 1989 & IJCAI 1989 & 7 & \noindent{}\textcolor{black!50}{0.00} \textcolor{black!50}{0.00} 0.75 & 0 0 0 & 0 0 & 0 0 0\\
FocacciLN00 \href{http://www.aaai.org/Library/AIPS/2000/aips00-010.php}{FocacciLN00} & \hyperref[auth:a775]{F. Focacci}, \hyperref[auth:a118]{P. Laborie}, \hyperref[auth:a655]{W. Nuijten} & Solving Scheduling Problems with Setup Times and Alternative Resources & \hyperref[detail:FocacciLN00]{Details} \href{../works/FocacciLN00.pdf}{Yes} & \cite{FocacciLN00} & 2000 & AAAI 2000 & 10 & \noindent{}\textcolor{black!50}{0.00} \textcolor{black!50}{0.00} \textbf{2.65} & 0 0 0 & 0 0 & 0 0 0\\
FoxAS82 \href{http://www.aaai.org/Library/AAAI/1982/aaai82-037.php}{FoxAS82} & \hyperref[auth:a302]{M. S. Fox}, \hyperref[auth:a1004]{B. P. Allen}, \hyperref[auth:a1005]{G. Strohm} & Job-Shop Scheduling: An Investigation in Constraint-Directed Reasoning & \hyperref[detail:FoxAS82]{Details} \href{../works/FoxAS82.pdf}{Yes} & \cite{FoxAS82} & 1982 & AAAI 1982 & 4 & \noindent{}\textcolor{black!50}{0.00} \textcolor{black!50}{0.00} \textcolor{black!50}{0.00} & 0 0 0 & 0 0 & 0 0 0\\
FoxS90 \href{}{FoxS90} & \hyperref[auth:a302]{M. S. Fox}, \hyperref[auth:a1042]{N. M. Sadeh} & Why is Scheduling Difficult? {A} {CSP} Perspective & \hyperref[detail:FoxS90]{Details} \href{../works/FoxS90.pdf}{Yes} & \cite{FoxS90} & 1990 & ECAI 1990 & 14 & \noindent{}\textbf{1.00} \textbf{1.00} \textbf{13.47} & 0 0 0 & 0 0 & 0 0 0\\
FrankK03 \href{http://www.aaai.org/Library/ICAPS/2003/icaps03-023.php}{FrankK03} & \hyperref[auth:a379]{J. Frank}, \hyperref[auth:a380]{E. K{\"{u}}rkl{\"{u}}} & SOFIA's Choice: Scheduling Observations for an Airborne Observatory & \hyperref[detail:FrankK03]{Details} \href{../works/FrankK03.pdf}{Yes} & \cite{FrankK03} & 2003 & ICAPS 2003 & 10 & \noindent{}\textcolor{black!50}{0.00} \textcolor{black!50}{0.00} \textcolor{black!50}{0.00} & 0 0 0 & 0 0 & 0 0 0\\
Froger16 \href{https://theses.hal.science/tel-01440836}{Froger16} & \hyperref[auth:a887]{A. Froger} & {Maintenance scheduling in the electricity industry : a particular focus on a problem rising in the onshore wind industry} & \hyperref[detail:Froger16]{Details} \href{../works/Froger16.pdf}{Yes} & \cite{Froger16} & 2016 & {Universit{\'e} d'Angers} & 181 & \noindent{}\textcolor{black!50}{0.00} \textcolor{black!50}{0.00} \textbf{127.37} & 0 0 0 & 0 0 & 0 0 0\\
FukunagaHFAMN02 \href{http://www.aaai.org/Library/AAAI/2002/aaai02-123.php}{FukunagaHFAMN02} & \hyperref[auth:a1326]{A. S. Fukunaga}, \hyperref[auth:a1327]{E. Hamilton}, \hyperref[auth:a1328]{J. Fama}, \hyperref[auth:a1329]{D. Andre}, \hyperref[auth:a1330]{O. Matan}, \hyperref[auth:a1331]{I. R. Nourbakhsh} & Staff Scheduling for Inbound Call Centers and Customer Contact Centers & \hyperref[detail:FukunagaHFAMN02]{Details} \href{../works/FukunagaHFAMN02.pdf}{Yes} & \cite{FukunagaHFAMN02} & 2002 & AAAI 2002 & 8 & \noindent{}\textcolor{black!50}{0.00} \textcolor{black!50}{0.00} 0.56 & 0 0 0 & 0 0 & 0 0 0\\
German18 \href{https://theses.hal.science/tel-01896325}{German18} & \hyperref[auth:a889]{G. German} & {Constraint programming for lot-sizing problems} & \hyperref[detail:German18]{Details} \href{../works/German18.pdf}{Yes} & \cite{German18} & 2018 & {Universit{\'e} Grenoble Alpes} & 112 & \noindent{}\textcolor{black!50}{0.00} \textcolor{black!50}{0.00} \textbf{10.90} & 0 0 0 & 0 0 & 0 0 0\\
GetoorOFC97 \href{http://www.aaai.org/Library/AAAI/1997/aaai97-047.php}{GetoorOFC97} & \hyperref[auth:a1291]{L. Getoor}, \hyperref[auth:a851]{G. Ottosson}, \hyperref[auth:a1292]{M. P. J. Fromherz}, \hyperref[auth:a1293]{B. Carlson} & Effective Redundant Constraints for Online Scheduling & \hyperref[detail:GetoorOFC97]{Details} \href{../works/GetoorOFC97.pdf}{Yes} & \cite{GetoorOFC97} & 1997 & AAAI 1997 & 6 & \noindent{}\textcolor{black!50}{0.00} \textcolor{black!50}{0.00} \textbf{1.98} & 0 0 0 & 0 0 & 0 0 0\\
GingrasQ16 \href{http://www.ijcai.org/Abstract/16/440}{GingrasQ16} & \hyperref[auth:a313]{V. Gingras}, \hyperref[auth:a37]{C.-G. Quimper} & Generalizing the Edge-Finder Rule for the Cumulative Constraint & \hyperref[detail:GingrasQ16]{Details} \href{../works/GingrasQ16.pdf}{Yes} & \cite{GingrasQ16} & 2016 & IJCAI 2016 & 7 & \noindent{}\textcolor{black!50}{0.00} \textcolor{black!50}{0.00} \textbf{1.48} & 0 0 0 & 0 0 & 0 0 0\\
GlobusCLP04 \href{}{GlobusCLP04} & \hyperref[auth:a1335]{A. Globus}, \hyperref[auth:a1336]{J. Crawford}, \hyperref[auth:a1337]{J. D. Lohn}, \hyperref[auth:a1338]{A. Pryor} & A Comparison of Techniques for Scheduling Earth Observing Satellites & \hyperref[detail:GlobusCLP04]{Details} \href{../works/GlobusCLP04.pdf}{Yes} & \cite{GlobusCLP04} & 2004 & AAAI 2004 & 8 & \noindent{}\textcolor{black!50}{0.00} \textcolor{black!50}{0.00} \textcolor{black!50}{0.00} & 0 0 0 & 0 0 & 0 0 0\\
GodardLN05 \href{http://www.aaai.org/Library/ICAPS/2005/icaps05-009.php}{GodardLN05} & \hyperref[auth:a773]{D. Godard}, \hyperref[auth:a118]{P. Laborie}, \hyperref[auth:a655]{W. Nuijten} & Randomized Large Neighborhood Search for Cumulative Scheduling & \hyperref[detail:GodardLN05]{Details} \href{../works/GodardLN05.pdf}{Yes} & \cite{GodardLN05} & 2005 & ICAPS 2005 & 9 & \noindent{}\textcolor{black!50}{0.00} \textcolor{black!50}{0.00} \textbf{2.16} & 0 0 0 & 0 0 & 0 0 0\\
Godet21a \href{https://tel.archives-ouvertes.fr/tel-03681868}{Godet21a} & \hyperref[auth:a470]{A. Godet} & Sur le tri de t{\^{a}}ches pour r{\'{e}}soudre des probl{\`{e}}mes d'ordonnancement avec la programmation par contraintes. (On the use of tasks ordering to solve scheduling problems with constraint programming) & \hyperref[detail:Godet21a]{Details} \href{../works/Godet21a.pdf}{Yes} & \cite{Godet21a} & 2021 & {IMT} Atlantique Bretagne Pays de la Loire, Brest, France & 168 & \noindent{}\textbf{2.50} \textbf{2.50} \textbf{172.67} & 0 0 0 & 0 0 & 0 0 0\\
GomesHS06 \href{http://www.aaai.org/Library/Symposia/Spring/2006/ss06-04-024.php}{GomesHS06} & \hyperref[auth:a641]{C. P. Gomes}, \hyperref[auth:a206]{W.-J. van Hoeve}, \hyperref[auth:a642]{B. Selman} & Constraint Programming for Distributed Planning and Scheduling & \hyperref[detail:GomesHS06]{Details} \href{../works/GomesHS06.pdf}{Yes} & \cite{GomesHS06} & 2006 & AAAI 2006 & 2 & \noindent{}\textbf{1.00} \textbf{1.00} 0.63 & 0 0 0 & 0 0 & 0 0 0\\
Groleaz21 \href{https://hal.science/tel-03266690}{Groleaz21} & \hyperref[auth:a83]{L. Groleaz} & {The Group Cumulative Scheduling Problem} & \hyperref[detail:Groleaz21]{Details} \href{../works/Groleaz21.pdf}{Yes} & \cite{Groleaz21} & 2021 & {Universit{\'e} de Lyon} & 153 & \noindent{}\textcolor{black!50}{0.00} \textcolor{black!50}{0.00} \textbf{331.76} & 0 0 0 & 0 0 & 0 0 0\\
Hamscher91 \href{http://www.aaai.org/Library/AAAI/1991/aaai91-079.php}{Hamscher91} & \hyperref[auth:a1274]{W. Hamscher} & {ACP:} Reason Maintenance and Inference Control for Constraint Propagation Over Intervals & \hyperref[detail:Hamscher91]{Details} \href{../works/Hamscher91.pdf}{Yes} & \cite{Hamscher91} & 1991 & AAAI 1991 & 6 & \noindent{}\textcolor{black!50}{0.00} \textcolor{black!50}{0.00} \textcolor{black!50}{0.02} & 0 0 0 & 0 0 & 0 0 0\\
HoeveGSL07 \href{http://www.aaai.org/Library/AAAI/2007/aaai07-291.php}{HoeveGSL07} & \hyperref[auth:a206]{W.-J. van Hoeve}, \hyperref[auth:a641]{C. P. Gomes}, \hyperref[auth:a642]{B. Selman}, \hyperref[auth:a142]{M. Lombardi} & Optimal Multi-Agent Scheduling with Constraint Programming & \hyperref[detail:HoeveGSL07]{Details} \href{../works/HoeveGSL07.pdf}{Yes} & \cite{HoeveGSL07} & 2007 & AAAI 2007 & 6 & \noindent{}\textbf{1.00} \textbf{1.00} 0.78 & 0 0 0 & 0 0 & 0 0 0\\
JoLLH99 \href{http://www.aaai.org/Library/IAAI/1999/iaai99-114.php}{JoLLH99} & \hyperref[auth:a1317]{G. Jo}, \hyperref[auth:a1318]{K.-H. Lee}, \hyperref[auth:a1319]{H.-Y. Lee}, \hyperref[auth:a1320]{S.-H. Hyun} & Ramp Activity Expert System for Scheduling and Co-ordination at an Airport & \hyperref[detail:JoLLH99]{Details} \href{../works/JoLLH99.pdf}{Yes} & \cite{JoLLH99} & 1999 & AAAI 1999 & 6 & \noindent{}\textcolor{black!50}{0.00} \textcolor{black!50}{0.00} 0.41 & 0 0 0 & 0 0 & 0 0 0\\
Johnston05 \href{}{Johnston05} & \hyperref[auth:a1340]{B. J. Clement}, \hyperref[auth:a1210]{M. D. Johnston} & The Deep Space Network Scheduling Problem & \hyperref[detail:Johnston05]{Details} \href{../works/Johnston05.pdf}{Yes} & \cite{Johnston05} & 2005 & AAAI 2005 & 7 & \noindent{}\textcolor{black!50}{0.00} \textcolor{black!50}{0.00} 0.34 & 0 0 0 & 0 0 & 0 0 0\\
JourdanFRD94 \href{}{JourdanFRD94} & \hyperref[auth:a696]{J. Jourdan}, \hyperref[auth:a697]{F. Fages}, \hyperref[auth:a698]{D. Rozzonelli}, \hyperref[auth:a699]{A. Demeure} & Data Alignment and Task Scheduling On Parallel Machines Using Concurrent Constraint Model-based Programming & \cellcolor{red!30}\hyperref[detail:JourdanFRD94]{Details} No & \cite{JourdanFRD94} & 1994 & ILPS 1994 & 1 & \noindent{}\textcolor{black!50}{0.00} \textcolor{black!50}{0.00} n/a & 0 0 0 & 0 0 & 0 0 0\\
Junker00 \href{http://www.aaai.org/Library/AAAI/2000/aaai00-139.php}{Junker00} & \hyperref[auth:a1325]{U. Junker} & Preference-Based Search for Scheduling & \hyperref[detail:Junker00]{Details} \href{../works/Junker00.pdf}{Yes} & \cite{Junker00} & 2000 & AAAI 2000 & 6 & \noindent{}\textcolor{black!50}{0.00} \textcolor{black!50}{0.00} 0.82 & 0 0 0 & 0 0 & 0 0 0\\
Kameugne14 \href{http://cp2013.a4cp.org/sites/default/files/roger_kameugne_-_propagation_techniques_of_resource_constraint_for_cumulative_scheduling.pdf}{Kameugne14} & \hyperref[auth:a10]{R. Kameugne} & Techniques de Propagation de la Contrainte de Ressource en Ordonnancement Cumulatif & \hyperref[detail:Kameugne14]{Details} \href{../works/Kameugne14.pdf}{Yes} & \cite{Kameugne14} & 2014 & University of Yaounde I, Cameroon & 139 & \noindent{}\textcolor{black!50}{0.00} \textcolor{black!50}{0.00} \textbf{8.15} & 0 0 0 & 0 0 & 0 0 0\\
KengY89 \href{http://ijcai.org/Proceedings/89-2/Papers/024.pdf}{KengY89} & \hyperref[auth:a1436]{N. Keng}, \hyperref[auth:a1437]{D. Y. Y. Yun} & A Planning/Scheduling Methodology for the Constrained Resource Problem & \hyperref[detail:KengY89]{Details} \href{../works/KengY89.pdf}{Yes} & \cite{KengY89} & 1989 & IJCAI 1989 & 6 & \noindent{}\textcolor{black!50}{0.00} \textcolor{black!50}{0.00} \textbf{1.10} & 0 0 0 & 0 0 & 0 0 0\\
KusterJF07 \href{http://ijcai.org/Proceedings/07/Papers/316.pdf}{KusterJF07} & \hyperref[auth:a1444]{J. Kuster}, \hyperref[auth:a1445]{D. Jannach}, \hyperref[auth:a601]{G. Friedrich} & Handling Alternative Activities in Resource-Constrained Project Scheduling Problems & \hyperref[detail:KusterJF07]{Details} \href{../works/KusterJF07.pdf}{Yes} & \cite{KusterJF07} & 2007 & IJCAI 2007 & 6 & \noindent{}\textcolor{black!50}{0.00} \textcolor{black!50}{0.00} 0.34 & 0 0 0 & 0 0 & 0 0 0\\
Laborie05 \href{http://ijcai.org/Proceedings/05/Papers/0571.pdf}{Laborie05} & \hyperref[auth:a118]{P. Laborie} & Complete MCS-Based Search: Application to Resource Constrained Project Scheduling & \hyperref[detail:Laborie05]{Details} \href{../works/Laborie05.pdf}{Yes} & \cite{Laborie05} & 2005 & IJCAI 2005 & 6 & \noindent{}\textcolor{black!50}{0.00} \textcolor{black!50}{0.00} \textbf{1.14} & 0 0 0 & 0 0 & 0 0 0\\
Layfield02 \href{http://etheses.whiterose.ac.uk/1301/}{Layfield02} & \hyperref[auth:a669]{C. J. Layfield} & A constraint programming pre-processor for duty scheduling & \hyperref[detail:Layfield02]{Details} \href{../works/Layfield02.pdf}{Yes} & \cite{Layfield02} & 2002 & University of Leeds, {UK} & 230 & \noindent{}\textbf{1.00} \textbf{1.00} \textcolor{black!50}{0.00} & 0 0 0 & 0 0 & 0 0 0\\
LeeKLKKYHP97 \href{http://www.aaai.org/Library/IAAI/1997/iaai97-182.php}{LeeKLKKYHP97} & \hyperref[auth:a1301]{K. J. Lee}, \hyperref[auth:a1302]{H. W. Kim}, \hyperref[auth:a1303]{J. K. Lee}, \hyperref[auth:a1304]{T. H. Kim}, \hyperref[auth:a1305]{C. G. Kim}, \hyperref[auth:a1306]{M. K. Yoon}, \hyperref[auth:a1307]{E. J. Hwang}, \hyperref[auth:a1308]{H. J. Park} & Case and Constraint-Based Apartment Construction Project Planning System: FASTrak-APT & \hyperref[detail:LeeKLKKYHP97]{Details} \href{../works/LeeKLKKYHP97.pdf}{Yes} & \cite{LeeKLKKYHP97} & 1997 & AAAI 1997 & 6 & \noindent{}\textcolor{black!50}{0.00} \textcolor{black!50}{0.00} \textcolor{black!50}{0.00} & 0 0 0 & 0 0 & 0 0 0\\
Lemos21 \href{https://scholar.tecnico.ulisboa.pt/records/u5RPHM-pu_yoOLXJF7BHrgJx47D827b0xHb3}{Lemos21} & \hyperref[auth:a875]{Alexandre Duarte {de Almeida} Lemos} & Solving scheduling problems under disruptions & \hyperref[detail:Lemos21]{Details} \href{../works/Lemos21.pdf}{Yes} & \cite{Lemos21} & 2021 & UNIVERSIDADE DE LISBOA INSTITUTO SUPERIOR TÉCNICO & 188 & \noindent{}\textcolor{black!50}{0.00} \textcolor{black!50}{0.00} \textbf{17.57} & 0 0 0 & 0 0 & 0 0 0\\
Letort13 \href{https://theses.hal.science/tel-00932215}{Letort13} & \hyperref[auth:a127]{A. Letort} & {Passage {\`a} l'{\'e}chelle pour les contraintes d'ordonnancement multi-ressources} & \hyperref[detail:Letort13]{Details} \href{../works/Letort13.pdf}{Yes} & \cite{Letort13} & 2013 & {Ecole des Mines de Nantes} & 132 & \noindent{}\textcolor{black!50}{0.00} \textcolor{black!50}{0.00} \textbf{8.73} & 0 0 0 & 0 0 & 0 0 0\\
LimAHO02a \href{http://www.aaai.org/Library/AAAI/2002/aaai02-175.php}{LimAHO02a} & \hyperref[auth:a279]{A. Lim}, \hyperref[auth:a1332]{J. C. Ang}, \hyperref[auth:a1333]{W.-K. Ho}, \hyperref[auth:a1334]{W.-C. Oon} & UTTSExam: {A} University Examination Timetable Scheduler & \hyperref[detail:LimAHO02a]{Details} \href{../works/LimAHO02a.pdf}{Yes} & \cite{LimAHO02a} & 2002 & AAAI 2002 & 2 & \noindent{}\textcolor{black!50}{0.00} \textcolor{black!50}{0.00} \textcolor{black!50}{0.04} & 0 0 0 & 0 0 & 0 0 0\\
Lombardi10 \href{http://amsdottorato.unibo.it/2961/}{Lombardi10} & \hyperref[auth:a142]{M. Lombardi} & Hybrid Methods for Resource Allocation and Scheduling Problems in Deterministic and Stochastic Environments & \hyperref[detail:Lombardi10]{Details} \href{../works/Lombardi10.pdf}{Yes} & \cite{Lombardi10} & 2010 & University of Bologna, Italy & 175 & \noindent{}\textcolor{black!50}{0.00} \textcolor{black!50}{0.00} \textbf{251.65} & 0 0 0 & 0 0 & 0 0 0\\
Lunardi20 \href{http://orbilu.uni.lu/handle/10993/43893}{Lunardi20} & \hyperref[auth:a495]{W. T. Lunardi} & A Real-World Flexible Job Shop Scheduling Problem With Sequencing Flexibility: Mathematical Programming, Constraint Programming, and Metaheuristics & \hyperref[detail:Lunardi20]{Details} \href{../works/Lunardi20.pdf}{Yes} & \cite{Lunardi20} & 2020 & University of Luxembourg, Luxembourg City, Luxembourg & 181 & \noindent{}\textbf{2.00} \textbf{2.00} \textbf{239.22} & 0 0 0 & 0 0 & 0 0 0\\
LuoVLBM16 \href{http://www.aaai.org/ocs/index.php/KR/KR16/paper/view/12909}{LuoVLBM16} & \hyperref[auth:a812]{R. Luo}, \hyperref[auth:a813]{R. A. Valenzano}, \hyperref[auth:a814]{Y. Li}, \hyperref[auth:a89]{J. C. Beck}, \hyperref[auth:a815]{S. A. McIlraith} & Using Metric Temporal Logic to Specify Scheduling Problems & \hyperref[detail:LuoVLBM16]{Details} \href{../works/LuoVLBM16.pdf}{Yes} & \cite{LuoVLBM16} & 2016 & KR 2016 & 4 & \noindent{}\textcolor{black!50}{0.00} \textcolor{black!50}{0.00} \textcolor{black!50}{0.00} & 0 0 0 & 0 0 & 0 0 0\\
Maillard15 \href{http://ijcai.org/Abstract/15/637}{Maillard15} & \hyperref[auth:a786]{A. Maillard} & Flexible Scheduling for an Agile Earth-Observing Satelllite & \hyperref[detail:Maillard15]{Details} \href{../works/Maillard15.pdf}{Yes} & \cite{Maillard15} & 2015 & IJCAI 2015 & 2 & \noindent{}\textcolor{black!50}{0.00} \textcolor{black!50}{0.00} \textcolor{black!50}{0.00} & 0 0 0 & 0 0 & 0 0 0\\
Malapert11 \href{https://tel.archives-ouvertes.fr/tel-00630122}{Malapert11} & \hyperref[auth:a82]{A. Malapert} & Techniques d'ordonnancement d'atelier et de fourn{\'{e}}es bas{\'{e}}es sur la programmation par contraintes. (Shop and batch scheduling with constraints) & \hyperref[detail:Malapert11]{Details} \href{../works/Malapert11.pdf}{Yes} & \cite{Malapert11} & 2011 & {\'{E}}cole des mines de Nantes, France & 194 & \noindent{}\textcolor{black!50}{0.00} \textcolor{black!50}{0.00} \textbf{142.49} & 0 0 0 & 0 0 & 0 0 0\\
Malik08 \href{https://hdl.handle.net/10012/3612}{Malik08} & \hyperref[auth:a637]{A. M. Malik} & Constraint Programming Techniques for Optimal Instruction Scheduling & \hyperref[detail:Malik08]{Details} \href{../works/Malik08.pdf}{Yes} & \cite{Malik08} & 2008 & University of Waterloo, Ontario, Canada & 151 & \noindent{}\textbf{1.00} \textbf{1.00} \textbf{44.76} & 0 0 0 & 0 0 & 0 0 0\\
Menana11 \href{https://tel.archives-ouvertes.fr/tel-00785838}{Menana11} & \hyperref[auth:a613]{J. Menana} & Automates et programmation par contraintes pour la planification de personnel. (Automata and Constraint Programming for Personnel Scheduling Problems) & \hyperref[detail:Menana11]{Details} \href{../works/Menana11.pdf}{Yes} & \cite{Menana11} & 2011 & University of Nantes, France & 148 & \noindent{}\textbf{1.00} \textbf{1.00} \textbf{2.73} & 0 0 0 & 0 0 & 0 0 0\\
MeskensDHG11 \href{}{MeskensDHG11} & \hyperref[auth:a596]{N. Meskens}, \hyperref[auth:a597]{D. Duvivier}, \hyperref[auth:a1372]{A. Hanset}, \hyperref[auth:a1373]{D. Gossart} & Multi-objective Constraint programming for Scheduling Operating Theatres \hyperref[abs:MeskensDHG11]{Abstract} & \hyperref[detail:MeskensDHG11]{Details} \href{../works/MeskensDHG11.pdf}{Yes} & \cite{MeskensDHG11} & 2011 & ACT 2011 & 10 & \noindent{}\textbf{1.00} \textbf{1.50} 0.77 & 0 0 0 & 0 0 & 0 0 0\\
MintonJPL90 \href{http://www.aaai.org/Library/AAAI/1990/aaai90-003.php}{MintonJPL90} & \hyperref[auth:a1209]{S. Minton}, \hyperref[auth:a1210]{M. D. Johnston}, \hyperref[auth:a1211]{A. B. Philips}, \hyperref[auth:a1212]{P. Laird} & Solving Large-Scale Constraint-Satisfaction and Scheduling Problems Using a Heuristic Repair Method & \hyperref[detail:MintonJPL90]{Details} \href{../works/MintonJPL90.pdf}{Yes} & \cite{MintonJPL90} & 1990 & AAAI 1990 & 8 & \noindent{}\textcolor{black!50}{0.00} \textcolor{black!50}{0.00} 0.85 & 0 0 0 & 0 0 & 0 0 0\\
MoffittPP05 \href{http://www.aaai.org/Library/AAAI/2005/aaai05-188.php}{MoffittPP05} & \hyperref[auth:a770]{M. D. Moffitt}, \hyperref[auth:a771]{B. Peintner}, \hyperref[auth:a772]{M. E. Pollack} & Augmenting Disjunctive Temporal Problems with Finite-Domain Constraints & \hyperref[detail:MoffittPP05]{Details} \href{../works/MoffittPP05.pdf}{Yes} & \cite{MoffittPP05} & 2005 & AAAI 2005 & 6 & \noindent{}\textcolor{black!50}{0.00} \textcolor{black!50}{0.00} \textcolor{black!50}{0.08} & 0 0 0 & 0 0 & 0 0 0\\
MorgadoM97 \href{http://www.aaai.org/Library/IAAI/1997/iaai97-186.php}{MorgadoM97} & \hyperref[auth:a1294]{E. M. Morgado}, \hyperref[auth:a1295]{J. P. Martins} & CREWS{\ }NS: Scheduling Train Crew in The Netherlands & \hyperref[detail:MorgadoM97]{Details} \href{../works/MorgadoM97.pdf}{Yes} & \cite{MorgadoM97} & 1997 & AAAI 1997 & 10 & \noindent{}\textcolor{black!50}{0.00} \textcolor{black!50}{0.00} 0.92 & 0 0 0 & 0 0 & 0 0 0\\
MurphyRFSS97 \href{http://www.aaai.org/Library/IAAI/1997/iaai97-187.php}{MurphyRFSS97} & \hyperref[auth:a1296]{K. Murphy}, \hyperref[auth:a1297]{E. Ralston}, \hyperref[auth:a1298]{D. Friedlander}, \hyperref[auth:a1299]{R. Swab}, \hyperref[auth:a1300]{P. Steege} & The Scheduling of Rail at Union Pacific Railroad & \hyperref[detail:MurphyRFSS97]{Details} \href{../works/MurphyRFSS97.pdf}{Yes} & \cite{MurphyRFSS97} & 1997 & AAAI 1997 & 10 & \noindent{}\textcolor{black!50}{0.00} \textcolor{black!50}{0.00} 0.35 & 0 0 0 & 0 0 & 0 0 0\\
MurthyRAW97 \href{}{MurthyRAW97} & \hyperref[auth:a1309]{S. S. Murthy}, \hyperref[auth:a1310]{J. Rachlin}, \hyperref[auth:a1311]{R. Akkiraju}, \hyperref[auth:a1312]{F. Y. Wu} & Agent-Based Cooperative Scheduling & \cellcolor{red!30}\hyperref[detail:MurthyRAW97]{Details} No & \cite{MurthyRAW97} & 1997 & AAAI 1997 & 6 & \noindent{}\textcolor{black!50}{0.00} \textcolor{black!50}{0.00} n/a & 0 0 0 & 0 0 & 0 0 0\\
Muscettola94 \href{http://www.aaai.org/Library/AAAI/1994/aaai94-170.php}{Muscettola94} & \hyperref[auth:a289]{N. Muscettola} & On the Utility of Bottleneck Reasoning for Scheduling & \hyperref[detail:Muscettola94]{Details} \href{../works/Muscettola94.pdf}{Yes} & \cite{Muscettola94} & 1994 & AAAI 1994 & 6 & \noindent{}\textcolor{black!50}{0.00} \textcolor{black!50}{0.00} 0.38 & 0 0 0 & 0 0 & 0 0 0\\
Musliu05 \href{http://ijcai.org/Proceedings/05/Papers/post-0448.pdf}{Musliu05} & \hyperref[auth:a45]{N. Musliu} & Combination of Local Search Strategies for Rotating Workforce Scheduling Problem & \hyperref[detail:Musliu05]{Details} \href{../works/Musliu05.pdf}{Yes} & \cite{Musliu05} & 2005 & IJCAI 2005 & 2 & \noindent{}\textcolor{black!50}{0.00} \textcolor{black!50}{0.00} \textcolor{black!50}{0.01} & 0 0 0 & 0 0 & 0 0 0\\
Nattaf16 \href{https://laas.hal.science/tel-01417288}{Nattaf16} & \hyperref[auth:a81]{M. Nattaf} & {Ordonnancement sous contraintes d'{\'e}nergie} & \hyperref[detail:Nattaf16]{Details} \href{../works/Nattaf16.pdf}{Yes} & \cite{Nattaf16} & 2016 & {UPS Toulouse - Universit{\'e} Toulouse 3 Paul Sabatier} & 199 & \noindent{}\textcolor{black!50}{0.00} \textcolor{black!50}{0.00} \textbf{11.76} & 0 0 0 & 0 0 & 0 0 0\\
Nuijten94 \href{https://pure.tue.nl/ws/portalfiles/portal/2374269/431902.pdf}{Nuijten94} & \hyperref[auth:a655]{W. Nuijten} & Time and Resource Constrained Scheduling: a Constraint Satisfaction Approach & \hyperref[detail:Nuijten94]{Details} \href{../works/Nuijten94.pdf}{Yes} & \cite{Nuijten94} & 1994 & Eindhoven University of Technology & 172 & \noindent{}\textbf{1.50} \textbf{1.50} \textbf{68.38} & 0 0 0 & 0 0 & 0 0 0\\
NuijtenA94 \href{}{NuijtenA94} & \hyperref[auth:a655]{W. Nuijten}, \hyperref[auth:a776]{E. H. L. Aarts} & Constraint Satisfaction for Multiple Capacitated Job Shop Scheduling & \hyperref[detail:NuijtenA94]{Details} \href{../works/NuijtenA94.pdf}{Yes} & \cite{NuijtenA94} & 1994 & ECAI 1994 & 5 & \noindent{}\textbf{2.00} \textbf{2.00} \textbf{2.80} & 0 0 0 & 0 0 & 0 0 0\\
NuijtenA94a \href{}{NuijtenA94a} & \hyperref[auth:a1255]{W. P. M. Nuijten}, \hyperref[auth:a776]{E. H. L. Aarts} & Constraint Satisfaction for Multiple Capacitated Job Shop Scheduling & \cellcolor{red!30}\hyperref[detail:NuijtenA94a]{Details} No & \cite{NuijtenA94a} & 1994 & ECAI 1994 & 5 & \noindent{}\textbf{2.00} \textbf{2.00} n/a & 0 0 0 & 0 0 & 0 0 0\\
OddiS97 \href{http://www.aaai.org/Library/AAAI/1997/aaai97-048.php}{OddiS97} & \hyperref[auth:a282]{A. Oddi}, \hyperref[auth:a298]{S. F. Smith} & Stochastic Procedures for Generating Feasible Schedules & \hyperref[detail:OddiS97]{Details} \href{../works/OddiS97.pdf}{Yes} & \cite{OddiS97} & 1997 & AAAI 1997 & 7 & \noindent{}\textcolor{black!50}{0.00} \textcolor{black!50}{0.00} 0.22 & 0 0 0 & 0 0 & 0 0 0\\
PapeB96 \href{}{PapeB96} & \hyperref[auth:a163]{C. L. Pape}, \hyperref[auth:a162]{P. Baptiste} & Constraint Propagation Techniques for Disjunctive Scheduling: The Preemptive Case & \cellcolor{red!30}\hyperref[detail:PapeB96]{Details} No & \cite{PapeB96} & 1996 & ECAI 1996 & 5 & \noindent{}\textbf{1.50} \textbf{1.50} n/a & 0 0 0 & 0 0 & 0 0 0\\
PapeB97 \href{}{PapeB97} & \hyperref[auth:a163]{C. L. Pape}, \hyperref[auth:a162]{P. Baptiste} & A Constraint Programming Library for Preemptive and Non-Preemptive Scheduling & \hyperref[detail:PapeB97]{Details} \href{../works/PapeB97.pdf}{Yes} & \cite{PapeB97} & 1997 & ACT 1997 & 20 & \noindent{}\textbf{1.00} \textbf{1.00} \textbf{14.61} & 0 0 0 & 0 0 & 0 0 0\\
PoderB08 \href{http://www.aaai.org/Library/ICAPS/2008/icaps08-033.php}{PoderB08} & \hyperref[auth:a358]{E. Poder}, \hyperref[auth:a128]{N. Beldiceanu} & Filtering for a Continuous Multi-Resources cumulative Constraint with Resource Consumption and Production & \hyperref[detail:PoderB08]{Details} \href{../works/PoderB08.pdf}{Yes} & \cite{PoderB08} & 2008 & ICAPS 2008 & 8 & \noindent{}\textcolor{black!50}{0.00} \textcolor{black!50}{0.00} \textbf{1.07} & 0 0 0 & 0 0 & 0 0 0\\
PolicellaWSO05 \href{http://www.aaai.org/Library/AAAI/2005/aaai05-190.php}{PolicellaWSO05} & \hyperref[auth:a283]{N. Policella}, \hyperref[auth:a1339]{X. Wang}, \hyperref[auth:a298]{S. F. Smith}, \hyperref[auth:a282]{A. Oddi} & Exploiting Temporal Flexibility to Obtain High Quality Schedules & \hyperref[detail:PolicellaWSO05]{Details} \href{../works/PolicellaWSO05.pdf}{Yes} & \cite{PolicellaWSO05} & 2005 & AAAI 2005 & 6 & \noindent{}\textcolor{black!50}{0.00} \textcolor{black!50}{0.00} 0.36 & 0 0 0 & 0 0 & 0 0 0\\
Prosser89 \href{http://ijcai.org/Proceedings/89-2/Papers/025.pdf}{Prosser89} & \hyperref[auth:a826]{P. Prosser} & A Reactive Scheduling Agent & \hyperref[detail:Prosser89]{Details} \href{../works/Prosser89.pdf}{Yes} & \cite{Prosser89} & 1989 & IJCAI 1989 & 6 & \noindent{}\textcolor{black!50}{0.00} \textcolor{black!50}{0.00} 0.72 & 0 0 0 & 0 0 & 0 0 0\\
Rit86 \href{http://www.aaai.org/Library/AAAI/1986/aaai86-064.php}{Rit86} & \hyperref[auth:a1270]{J.-F. Rit} & Propagating Temporal Constraints for Scheduling & \hyperref[detail:Rit86]{Details} \href{../works/Rit86.pdf}{Yes} & \cite{Rit86} & 1986 & AAAI 1986 & 6 & \noindent{}\textcolor{black!50}{0.00} \textcolor{black!50}{0.00} 0.23 & 0 0 0 & 0 0 & 0 0 0\\
Rodosek94 \href{}{Rodosek94} & \hyperref[auth:a297]{R. Rodosek} & Combining Constraint Network and Causal Theory to Solve Scheduling Problems from a {CSP} Perspective & \cellcolor{red!30}\hyperref[detail:Rodosek94]{Details} No & \cite{Rodosek94} & 1994 & ECAI 1994 & 5 & \noindent{}\textbf{1.00} \textbf{1.00} n/a & 0 0 0 & 0 0 & 0 0 0\\
Rodriguez07b \href{}{Rodriguez07b} & \hyperref[auth:a780]{J. Rodriguez} & A study of the use of state resources in a constraint-based model for routing and scheduling trains & \hyperref[detail:Rodriguez07b]{Details} \href{../works/Rodriguez07b.pdf}{Yes} & \cite{Rodriguez07b} & 2007 & ICROMA 2007 & 14 & \noindent{}\textcolor{black!50}{0.00} \textcolor{black!50}{0.00} \textbf{1.78} & 0 0 0 & 0 0 & 0 0 0\\
RodriguezDG02 \href{}{RodriguezDG02} & \hyperref[auth:a780]{J. Rodriguez}, \hyperref[auth:a781]{X. Delorme}, \hyperref[auth:a782]{X. Gandibleux} & Railway infrastructure saturation using constraint programming approach & \hyperref[detail:RodriguezDG02]{Details} \href{../works/RodriguezDG02.pdf}{Yes} & \cite{RodriguezDG02} & 2002 & Computers in Railways VIII & 10 & \noindent{}\textcolor{black!50}{0.00} \textcolor{black!50}{0.00} 0.38 & 0 0 0 & 0 0 & 0 0 0\\
RodriguezS09 \href{}{RodriguezS09} & \hyperref[auth:a780]{J. Rodriguez}, \hyperref[auth:a1016]{S. Sobieraj} & A study of an incremental texture-based heuristic for the train routing and scheduling problem & \hyperref[detail:RodriguezS09]{Details} \href{../works/RodriguezS09.pdf}{Yes} & \cite{RodriguezS09} & 2009 & ICROMA 2009 & 14 & \noindent{}\textcolor{black!50}{0.00} \textcolor{black!50}{0.00} \textbf{1.60} & 0 0 0 & 0 0 & 0 0 0\\
RoweJCA96 \href{http://www.aaai.org/Library/IAAI/1996/iaai96-280.php}{RoweJCA96} & \hyperref[auth:a1282]{J. Rowe}, \hyperref[auth:a1283]{K. Jewers}, \hyperref[auth:a1284]{A. Codd}, \hyperref[auth:a1285]{A. Alcock} & Intelligent Retail Logistics Scheduling & \hyperref[detail:RoweJCA96]{Details} \href{../works/RoweJCA96.pdf}{Yes} & \cite{RoweJCA96} & 1996 & AAAI 1996 & 9 & \noindent{}\textcolor{black!50}{0.00} \textcolor{black!50}{0.00} \textcolor{black!50}{0.08} & 0 0 0 & 0 0 & 0 0 0\\
SakkoutRW98 \href{}{SakkoutRW98} & \hyperref[auth:a166]{H. E. Sakkout}, \hyperref[auth:a1264]{T. Richards}, \hyperref[auth:a117]{M. G. Wallace} & Minimal Perturbance in Dynamic Scheduling & \cellcolor{red!30}\hyperref[detail:SakkoutRW98]{Details} No & \cite{SakkoutRW98} & 1998 & ECAI 1998 & 5 & \noindent{}\textcolor{black!50}{0.00} \textcolor{black!50}{0.00} n/a & 0 0 0 & 0 0 & 0 0 0\\
Schaerf96 \href{}{Schaerf96} & \hyperref[auth:a1260]{A. Schaerf} & Scheduling Sport Tournaments using Constraint Logic Programming & \cellcolor{red!30}\hyperref[detail:Schaerf96]{Details} No & \cite{Schaerf96} & 1996 & ECAI 1996 & 5 & \noindent{}\textbf{1.00} \textbf{1.00} n/a & 0 0 0 & 0 0 & 0 0 0\\
Schaerf97 \href{http://ijcai.org/Proceedings/97-2/Papers/067.pdf}{Schaerf97} & \hyperref[auth:a1260]{A. Schaerf} & Combining Local Search and Look-Ahead for Scheduling and Constraint Satisfaction Problems & \hyperref[detail:Schaerf97]{Details} \href{../works/Schaerf97.pdf}{Yes} & \cite{Schaerf97} & 1997 & IJCAI 1997 & 6 & \noindent{}\textbf{1.00} \textbf{1.00} \textcolor{black!50}{0.20} & 0 0 0 & 0 0 & 0 0 0\\
SchausD08 \href{http://www.aaai.org/Library/AAAI/2008/aaai08-058.php}{SchausD08} & \hyperref[auth:a147]{P. Schaus}, \hyperref[auth:a151]{Y. Deville} & A Global Constraint for Bin-Packing with Precedences: Application to the Assembly Line Balancing Problem & \hyperref[detail:SchausD08]{Details} \href{../works/SchausD08.pdf}{Yes} & \cite{SchausD08} & 2008 & AAAI 2008 & 6 & \noindent{}\textcolor{black!50}{0.00} \textcolor{black!50}{0.00} 0.93 & 0 0 0 & 0 0 & 0 0 0\\
Schutt11 \href{https://www.a4cp.org/sites/default/files/andreas_schutt_-_improving_scheduling_by_learning.pdf}{Schutt11} & \hyperref[auth:a124]{A. Schutt} & Improving Scheduling by Learning & \hyperref[detail:Schutt11]{Details} \href{../works/Schutt11.pdf}{Yes} & \cite{Schutt11} & 2011 & University of Melbourne, Australia & 209 & \noindent{}\textcolor{black!50}{0.00} \textcolor{black!50}{0.00} \textbf{102.66} & 0 0 0 & 0 0 & 0 0 0\\
Siala15a \href{https://tel.archives-ouvertes.fr/tel-01164291}{Siala15a} & \hyperref[auth:a129]{M. Siala} & Search, propagation, and learning in sequencing and scheduling problems. (Recherche, propagation et apprentissage dans les probl{\`{e}}mes de s{\'{e}}quencement et d'ordonnancement) & \hyperref[detail:Siala15a]{Details} \href{../works/Siala15a.pdf}{Yes} & \cite{Siala15a} & 2015 & {INSA} Toulouse, France & 199 & \noindent{}0.50 0.50 \textbf{98.99} & 0 0 0 & 0 0 & 0 0 0\\
SmithC93 \href{http://www.aaai.org/Library/AAAI/1993/aaai93-022.php}{SmithC93} & \hyperref[auth:a298]{S. F. Smith}, \hyperref[auth:a1275]{C.-C. Cheng} & Slack-Based Heuristics for Constraint Satisfaction Scheduling & \hyperref[detail:SmithC93]{Details} \href{../works/SmithC93.pdf}{Yes} & \cite{SmithC93} & 1993 & AAAI 1993 & 6 & \noindent{}\textbf{1.00} \textbf{1.00} 0.54 & 0 0 0 & 0 0 & 0 0 0\\
StidsenKM96 \href{}{StidsenKM96} & \hyperref[auth:a1261]{T. R. Stidsen}, \hyperref[auth:a1262]{L. V. Kragelund}, \hyperref[auth:a1263]{O. Mateescu} & Jobshop Scheduling in a Shipyard & \cellcolor{red!30}\hyperref[detail:StidsenKM96]{Details} No & \cite{StidsenKM96} & 1996 & ECAI 1996 & 8 & \noindent{}\textcolor{black!50}{0.00} \textcolor{black!50}{0.00} n/a & 0 0 0 & 0 0 & 0 0 0\\
SultanikMR07 \href{http://ijcai.org/Proceedings/07/Papers/247.pdf}{SultanikMR07} & \hyperref[auth:a1441]{E. Sultanik}, \hyperref[auth:a1442]{P. J. Modi}, \hyperref[auth:a1443]{W. C. Regli} & On Modeling Multiagent Task Scheduling as a Distributed Constraint Optimization Problem & \hyperref[detail:SultanikMR07]{Details} \href{../works/SultanikMR07.pdf}{Yes} & \cite{SultanikMR07} & 2007 & IJCAI 2007 & 6 & \noindent{}\textbf{2.00} \textbf{2.00} \textbf{1.22} & 0 0 0 & 0 0 & 0 0 0\\
Tay92 \href{}{Tay92} & \hyperref[auth:a700]{D. B. H. Tay} & {COPS:} {A} Constraint Programming Approach to Resource-Limited Project Scheduling & \cellcolor{red!30}\hyperref[detail:Tay92]{Details} No & \cite{Tay92} & 1992 & Comput. J. & null & \noindent{}\textbf{1.50} \textbf{1.50} n/a & 0 0 0 & 0 0 & 0 0 0\\
TranWDRFOVB16 \href{http://www.aaai.org/ocs/index.php/WS/AAAIW16/paper/view/12664}{TranWDRFOVB16} & \hyperref[auth:a798]{T. T. Tran}, \hyperref[auth:a807]{Z. Wang}, \hyperref[auth:a808]{M. Do}, \hyperref[auth:a809]{E. G. Rieffel}, \hyperref[auth:a379]{J. Frank}, \hyperref[auth:a810]{B. O'Gorman}, \hyperref[auth:a811]{D. Venturelli}, \hyperref[auth:a89]{J. C. Beck} & Explorations of Quantum-Classical Approaches to Scheduling a Mars Lander Activity Problem & \hyperref[detail:TranWDRFOVB16]{Details} \href{../works/TranWDRFOVB16.pdf}{Yes} & \cite{TranWDRFOVB16} & 2016 & AAAI 2016 & 9 & \noindent{}\textcolor{black!50}{0.00} \textcolor{black!50}{0.00} 0.39 & 0 0 0 & 0 0 & 0 0 0\\
TsurutaS00 \href{}{TsurutaS00} & \hyperref[auth:a1265]{T. Tsuruta}, \hyperref[auth:a1266]{T. Shintani} & Scheduling Meetings Using Distributed Valued Constraint Satisfaction Algorithm & \cellcolor{red!30}\hyperref[detail:TsurutaS00]{Details} No & \cite{TsurutaS00} & 2000 & ECAI 2000 & 5 & \noindent{}\textbf{1.00} \textbf{1.00} n/a & 0 0 0 & 0 0 & 0 0 0\\
Valdes87 \href{http://www.aaai.org/Library/AAAI/1987/aaai87-046.php}{Valdes87} & \hyperref[auth:a1271]{R. E. Vald{\'{e}}s-P{\'{e}}rez} & The Satisfiability of Temporal Constraint Networks & \hyperref[detail:Valdes87]{Details} \href{../works/Valdes87.pdf}{Yes} & \cite{Valdes87} & 1987 & AAAI 1987 & 5 & \noindent{}\textcolor{black!50}{0.00} \textcolor{black!50}{0.00} \textcolor{black!50}{0.05} & 0 0 0 & 0 0 & 0 0 0\\
VillaverdeP04 \href{}{VillaverdeP04} & \hyperref[auth:a657]{K. Villaverde}, \hyperref[auth:a33]{E. Pontelli} & An Investigation of Scheduling in Distributed Constraint Logic Programming & \cellcolor{red!30}\hyperref[detail:VillaverdeP04]{Details} No & \cite{VillaverdeP04} & 2004 & ISCA 2004 & 6 & \noindent{}\textbf{1.00} \textbf{1.00} n/a & 0 0 0 & 0 0 & 0 0 0\\
Wallace94 \href{}{Wallace94} & \hyperref[auth:a117]{M. G. Wallace} & Applying Constraints for Scheduling \hyperref[abs:Wallace94]{Abstract} & \cellcolor{red!30}\hyperref[detail:Wallace94]{Details} No & \cite{Wallace94} & 1994 & Constraint Programming 1994 & 19 & \noindent{}\textcolor{black!50}{0.00} \textbf{3.00} n/a & 0 0 0 & 0 0 & 0 0 0\\
WallaceF00 \href{}{WallaceF00} & \hyperref[auth:a1267]{R. J. Wallace}, \hyperref[auth:a273]{E. C. Freuder} & Dispatchability Conditions for Schedules with Consumable Resources & \hyperref[detail:WallaceF00]{Details} \href{../works/WallaceF00.pdf}{Yes} & \cite{WallaceF00} & 2000 & ECAI 2000 & 7 & \noindent{}\textcolor{black!50}{0.00} \textcolor{black!50}{0.00} 0.30 & 0 0 0 & 0 0 & 0 0 0\\
WatsonBHW99 \href{http://www.aaai.org/Library/AAAI/1999/aaai99-098.php}{WatsonBHW99} & \hyperref[auth:a360]{J.-P. Watson}, \hyperref[auth:a1313]{L. Barbulescu}, \hyperref[auth:a1314]{A. E. Howe}, \hyperref[auth:a1315]{L. D. Whitley} & Algorithm Performance and Problem Structure for Flow-shop Scheduling & \hyperref[detail:WatsonBHW99]{Details} \href{../works/WatsonBHW99.pdf}{Yes} & \cite{WatsonBHW99} & 1999 & AAAI 1999 & 8 & \noindent{}\textcolor{black!50}{0.00} \textcolor{black!50}{0.00} \textcolor{black!50}{0.10} & 0 0 0 & 0 0 & 0 0 0\\
YeGMH94 \href{}{YeGMH94} & \hyperref[auth:a1256]{P. Ye}, \hyperref[auth:a1257]{D. Glass}, \hyperref[auth:a1258]{M. F. McTear}, \hyperref[auth:a1259]{J. G. Hughes} & Job Cost and Constraint Relaxation for Scheduling Problem Solving in the {CLP} Paradigm & \cellcolor{red!30}\hyperref[detail:YeGMH94]{Details} No & \cite{YeGMH94} & 1994 & ECAI 1994 & 5 & \noindent{}\textbf{2.00} \textbf{2.00} n/a & 0 0 0 & 0 0 & 0 0 0\\
YoshikawaKNW94 \href{http://www.aaai.org/Library/AAAI/1994/aaai94-171.php}{YoshikawaKNW94} & \hyperref[auth:a1278]{M. Yoshikawa}, \hyperref[auth:a1279]{K. Kaneko}, \hyperref[auth:a1280]{Y. Nomura}, \hyperref[auth:a1281]{M. Watanabe} & A Constraint-Based Approach to High-School Timetabling Problems: {A} Case Study & \hyperref[detail:YoshikawaKNW94]{Details} \href{../works/YoshikawaKNW94.pdf}{Yes} & \cite{YoshikawaKNW94} & 1994 & AAAI 1994 & 6 & \noindent{}\textcolor{black!50}{0.00} \textcolor{black!50}{0.00} \textcolor{black!50}{0.02} & 0 0 0 & 0 0 & 0 0 0\\
Zahout21 \href{https://hal.science/tel-03606639}{Zahout21} & \hyperref[auth:a888]{B. Zahout} & {Algorithmes exacts et approch{\'e}s pour l'ordonnancement des travaux multiressources {\`a} intervalles fixes dans des syst{\`e}mes distribu{\'e}s : approche monocrit{\`e}re et multiagent} & \hyperref[detail:Zahout21]{Details} \href{../works/Zahout21.pdf}{Yes} & \cite{Zahout21} & 2021 & {Universit{\'e} de Tours - LIFAT} & 185 & \noindent{}\textcolor{black!50}{0.00} \textcolor{black!50}{0.00} \textbf{17.91} & 0 0 0 & 0 0 & 0 0 0\\
abs-0907-0939 \href{http://arxiv.org/abs/0907.0939}{abs-0907-0939} & \hyperref[auth:a221]{T. Petit}, \hyperref[auth:a358]{E. Poder} & The Soft Cumulative Constraint & \hyperref[detail:abs-0907-0939]{Details} \href{../works/abs-0907-0939.pdf}{Yes} & \cite{abs-0907-0939} & 2009 & CoRR & 12 & \noindent{}\textcolor{black!50}{0.00} \textcolor{black!50}{0.00} 0.37 & 0 0 0 & 0 0 & 0 0 0\\
abs-1009-0347 \href{http://arxiv.org/abs/1009.0347}{abs-1009-0347} & \hyperref[auth:a124]{A. Schutt}, \hyperref[auth:a154]{T. Feydy}, \hyperref[auth:a125]{P. J. Stuckey}, \hyperref[auth:a117]{M. G. Wallace} & Solving the Resource Constrained Project Scheduling Problem with Generalized Precedences by Lazy Clause Generation & \hyperref[detail:abs-1009-0347]{Details} \href{../works/abs-1009-0347.pdf}{Yes} & \cite{abs-1009-0347} & 2010 & CoRR & 37 & \noindent{}\textcolor{black!50}{0.00} \textcolor{black!50}{0.00} \textbf{4.70} & 0 0 0 & 0 0 & 0 0 0\\
abs-1901-07914 \href{http://arxiv.org/abs/1901.07914}{abs-1901-07914} & \hyperref[auth:a539]{J. K. Behrens}, \hyperref[auth:a540]{R. Lange}, \hyperref[auth:a541]{M. Mansouri} & A Constraint Programming Approach to Simultaneous Task Allocation and Motion Scheduling for Industrial Dual-Arm Manipulation Tasks & \hyperref[detail:abs-1901-07914]{Details} \href{../works/abs-1901-07914.pdf}{Yes} & \cite{abs-1901-07914} & 2019 & CoRR & 8 & \noindent{}\textbf{2.00} \textbf{2.00} \textbf{4.13} & 0 0 0 & 0 0 & 0 0 0\\
abs-1902-01193 \href{http://arxiv.org/abs/1902.01193}{abs-1902-01193} & \hyperref[auth:a547]{O. M. Alade}, \hyperref[auth:a548]{A. O. Amusat} & Solving Nurse Scheduling Problem Using Constraint Programming Technique & \hyperref[detail:abs-1902-01193]{Details} \href{../works/abs-1902-01193.pdf}{Yes} & \cite{abs-1902-01193} & 2019 & CoRR & 9 & \noindent{}\textbf{1.00} \textbf{1.00} \textbf{1.99} & 0 0 0 & 0 0 & 0 0 0\\
abs-1902-09244 \href{http://arxiv.org/abs/1902.09244}{abs-1902-09244} & \hyperref[auth:a549]{V. A. Hauder}, \hyperref[auth:a550]{A. Beham}, \hyperref[auth:a551]{S. Raggl}, \hyperref[auth:a552]{S. N. Parragh}, \hyperref[auth:a553]{M. Affenzeller} & On constraint programming for a new flexible project scheduling problem with resource constraints & \hyperref[detail:abs-1902-09244]{Details} \href{../works/abs-1902-09244.pdf}{Yes} & \cite{abs-1902-09244} & 2019 & CoRR & 62 & \noindent{}\textbf{1.50} \textbf{1.50} \textbf{350.76} & 0 0 0 & 0 0 & 0 0 0\\
abs-1911-04766 \href{http://arxiv.org/abs/1911.04766}{abs-1911-04766} & \hyperref[auth:a77]{T. Geibinger}, \hyperref[auth:a80]{F. Mischek}, \hyperref[auth:a45]{N. Musliu} & Investigating Constraint Programming and Hybrid Methods for Real World Industrial Test Laboratory Scheduling & \hyperref[detail:abs-1911-04766]{Details} \href{../works/abs-1911-04766.pdf}{Yes} & \cite{abs-1911-04766} & 2019 & CoRR & 16 & \noindent{}\textbf{1.00} \textbf{1.00} \textbf{15.95} & 0 0 0 & 0 0 & 0 0 0\\
abs-2102-08778 \href{https://arxiv.org/abs/2102.08778}{abs-2102-08778} & \hyperref[auth:a93]{G. D. Col}, \hyperref[auth:a607]{E. Teppan} & Large-Scale Benchmarks for the Job Shop Scheduling Problem & \hyperref[detail:abs-2102-08778]{Details} \href{../works/abs-2102-08778.pdf}{Yes} & \cite{abs-2102-08778} & 2021 & CoRR & 10 & \noindent{}\textcolor{black!50}{0.00} \textcolor{black!50}{0.00} \textbf{1.00} & 0 0 0 & 0 0 & 0 0 0\\
\end{longtable}
}



\clearpage
\section{Details of Works}

\input{./exports/abstracts}

\clearpage
\section{Abstracts of Missing Works}

\subsection{Missing Works}

\subsection{Highly Connected Missing Works}

\subsubsection{mw7098}
\label{mw:mw7098}

Authors: Robert M. Haralick, Gordon L. Elliott

Title: Increasing tree search efficiency for constraint satisfaction problems

Relevance:  0.00

{\scriptsize
\begin{longtable}{p{2cm}p{20cm}}
\caption{Extracted Features from Title and Abstract}\\ \toprule
Type & Concepts Found\\ \midrule
\endhead
\bottomrule
\endfoot
Scheduling & \\ 
CP & constraint satisfaction\\ 
Concepts & \\ 
Classification & \\ 
Constraints & \\ 
ApplicationAreas & \\ 
Industries & \\ 
CPSystems & \\ 
Benchmarks & \\ 
Algorithms & \\ 
\end{longtable}
}



\subsubsection{mw5795}
\label{mw:mw5795}

Authors: Jan Węglarz, Joanna Józefowska, Marek Mika, Grzegorz Waligóra

Title: Project scheduling with finite or infinite number of activity processing modes – A survey

Relevance:  0.00

{\scriptsize
\begin{longtable}{p{2cm}p{20cm}}
\caption{Extracted Features from Title and Abstract}\\ \toprule
Type & Concepts Found\\ \midrule
\endhead
\bottomrule
\endfoot
Scheduling & scheduling, activity\\ 
CP & \\ 
Concepts & \\ 
Classification & \\ 
Constraints & \\ 
ApplicationAreas & \\ 
Industries & \\ 
CPSystems & \\ 
Benchmarks & \\ 
Algorithms & \\ 
\end{longtable}
}



\subsubsection{mw9632}
\label{mw:mw9632}

Authors: Paul Martin, David B. Shmoys

Title: A new approach to computing optimal schedules for the job-shop scheduling problem

Relevance:  0.00

{\scriptsize
\begin{longtable}{p{2cm}p{20cm}}
\caption{Extracted Features from Title and Abstract}\\ \toprule
Type & Concepts Found\\ \midrule
\endhead
\bottomrule
\endfoot
Scheduling & scheduling, job\\ 
CP & \\ 
Concepts & job-shop\\ 
Classification & \\ 
Constraints & \\ 
ApplicationAreas & \\ 
Industries & \\ 
CPSystems & \\ 
Benchmarks & \\ 
Algorithms & \\ 
\end{longtable}
}



\subsubsection{mw5646}
\label{mw:mw5646}

Authors: Ridvan Gedik, Chase Rainwater, Heather Nachtmann, Ed A. Pohl

Title: Analysis of a parallel machine scheduling problem with sequence dependent setup times and job availability intervals

Relevance:  0.00

{\scriptsize
\begin{longtable}{p{2cm}p{20cm}}
\caption{Extracted Features from Title and Abstract}\\ \toprule
Type & Concepts Found\\ \midrule
\endhead
\bottomrule
\endfoot
Scheduling & scheduling, job, machine\\ 
CP & \\ 
Concepts & sequence dependent setup, setup-time\\ 
Classification & parallel machine\\ 
Constraints & \\ 
ApplicationAreas & \\ 
Industries & \\ 
CPSystems & \\ 
Benchmarks & \\ 
Algorithms & \\ 
\end{longtable}
}



\subsubsection{mw5082}
\label{mw:mw5082}

Authors: Robert Klein, Armin Scholl

Title: Computing lower bounds by destructive improvement: An application to resource-constrained project scheduling

Relevance:  0.00

{\scriptsize
\begin{longtable}{p{2cm}p{20cm}}
\caption{Extracted Features from Title and Abstract}\\ \toprule
Type & Concepts Found\\ \midrule
\endhead
\bottomrule
\endfoot
Scheduling & scheduling, resource\\ 
CP & \\ 
Concepts & \\ 
Classification & \\ 
Constraints & \\ 
ApplicationAreas & \\ 
Industries & \\ 
CPSystems & \\ 
Benchmarks & \\ 
Algorithms & \\ 
\end{longtable}
}



\subsubsection{mw5228}
\label{mw:mw5228}

Authors: Peter Brucker, Sigrid Knust, Arno Schoo, Olaf Thiele

Title: A branch and bound algorithm for the resource-constrained project scheduling problem

Relevance:  0.00

{\scriptsize
\begin{longtable}{p{2cm}p{20cm}}
\caption{Extracted Features from Title and Abstract}\\ \toprule
Type & Concepts Found\\ \midrule
\endhead
\bottomrule
\endfoot
Scheduling & scheduling, resource\\ 
CP & \\ 
Concepts & \\ 
Classification & Resource-constrained Project Scheduling Problem\\ 
Constraints & \\ 
ApplicationAreas & \\ 
Industries & \\ 
CPSystems & \\ 
Benchmarks & \\ 
Algorithms & \\ 
\end{longtable}
}



\subsubsection{mw243}
\label{mw:mw243}

Authors: Eugeniusz Nowicki, Czeslaw Smutnicki

Title: A Fast Taboo Search Algorithm for the Job Shop Problem

Relevance:  0.00

{\scriptsize
\begin{longtable}{p{2cm}p{20cm}}
\caption{Extracted Features from Title and Abstract}\\ \toprule
Type & Concepts Found\\ \midrule
\endhead
\bottomrule
\endfoot
Scheduling & job\\ 
CP & \\ 
Concepts & make-span, job-shop\\ 
Classification & \\ 
Constraints & \\ 
ApplicationAreas & \\ 
Industries & \\ 
CPSystems & \\ 
Benchmarks & benchmark\\ 
Algorithms & \\ 
\end{longtable}
}

  A fast and easily implementable approximation algorithm for the problem of finding a minimum makespan in a job shop is presented. The algorithm is based on a taboo search technique with a specific neighborhood definition which employs a critical path and blocks of operations notions. Computational experiments (up to 2,000 operations) show that the algorithm not only finds shorter makespans than the best approximation approaches but also runs in shorter time. It solves the well-known 10 × 10 hard benchmark problem within 30 seconds on a personal computer.  

\subsubsection{mw407}
\label{mw:mw407}

Authors: Bert De Reyck, willy Herroelen

Title: A branch-and-bound procedure for the resource-constrained project scheduling problem with generalized precedence relations

Relevance:  0.00

{\scriptsize
\begin{longtable}{p{2cm}p{20cm}}
\caption{Extracted Features from Title and Abstract}\\ \toprule
Type & Concepts Found\\ \midrule
\endhead
\bottomrule
\endfoot
Scheduling & scheduling, resource\\ 
CP & \\ 
Concepts & precedence\\ 
Classification & Resource-constrained Project Scheduling Problem\\ 
Constraints & \\ 
ApplicationAreas & \\ 
Industries & \\ 
CPSystems & \\ 
Benchmarks & \\ 
Algorithms & \\ 
\end{longtable}
}



\subsubsection{mw2235}
\label{mw:mw2235}

Authors: Rolf H. Möhring, Andreas S. Schulz, Frederik Stork, Marc Uetz

Title: Solving Project Scheduling Problems by Minimum Cut Computations

Relevance:  0.00

{\scriptsize
\begin{longtable}{p{2cm}p{20cm}}
\caption{Extracted Features from Title and Abstract}\\ \toprule
Type & Concepts Found\\ \midrule
\endhead
\bottomrule
\endfoot
Scheduling & scheduling, job, resource, machine\\ 
CP & \\ 
Concepts & precedence\\ 
Classification & Resource-constrained Project Scheduling Problem\\ 
Constraints & \\ 
ApplicationAreas & \\ 
Industries & \\ 
CPSystems & \\ 
Benchmarks & \\ 
Algorithms & \\ 
\end{longtable}
}

  In project scheduling, a set of precedence-constrained jobs has to be scheduled so as to minimize a given objective. In resource-constrained project scheduling, the jobs additionally compete for scarce resources. Due to its universality, the latter problem has a variety of applications in manufacturing, production planning, project management, and elsewhere. It is one of the most intractable problems in operations research, and has therefore become a popular playground for the latest optimization techniques, including virtually all local search paradigms. We show that a somewhat more classical mathematical programming approach leads to both competitive feasible solutions and strong lower bounds, within reasonable computation times. The basic ingredients of our approach are the Lagrangian relaxation of a time-indexed integer programming formulation and relaxation-based list scheduling, enriched with a useful idea from recent approximation algorithms for machine scheduling problems. The efficiency of the algorithm results from the insight that the relaxed problem can be solved by computing a minimum cut in an appropriately defined directed graph. Our computational study covers different types of resource-constrained project scheduling problems, based on several notoriously hard test sets, including practical problem instances from chemical production planning.  

\subsubsection{mw4312}
\label{mw:mw4312}

Authors: Nicos Christofides, R. Alvarez-Valdes, J. M. Tamarit

Title: Project scheduling with resource constraints: A branch and bound approach

Relevance:  0.00

{\scriptsize
\begin{longtable}{p{2cm}p{20cm}}
\caption{Extracted Features from Title and Abstract}\\ \toprule
Type & Concepts Found\\ \midrule
\endhead
\bottomrule
\endfoot
Scheduling & scheduling, resource\\ 
CP & \\ 
Concepts & \\ 
Classification & \\ 
Constraints & \\ 
ApplicationAreas & \\ 
Industries & \\ 
CPSystems & \\ 
Benchmarks & \\ 
Algorithms & \\ 
\end{longtable}
}



\subsubsection{mw4685}
\label{mw:mw4685}

Authors: Rainer Kolisch, Arno Sprecher, Andreas Drexl

Title: Characterization and Generation of a General Class of Resource-Constrained Project Scheduling Problems

Relevance:  0.00

{\scriptsize
\begin{longtable}{p{2cm}p{20cm}}
\caption{Extracted Features from Title and Abstract}\\ \toprule
Type & Concepts Found\\ \midrule
\endhead
\bottomrule
\endfoot
Scheduling & scheduling, resource\\ 
CP & \\ 
Concepts & precedence\\ 
Classification & Resource-constrained Project Scheduling Problem\\ 
Constraints & \\ 
ApplicationAreas & \\ 
Industries & \\ 
CPSystems & \\ 
Benchmarks & benchmark\\ 
Algorithms & \\ 
\end{longtable}
}

  This paper addresses the issue of how to generate problem instances of controlled difficulty. It focuses on precedence- and resource-constrained (project) scheduling problems, but similar ideas may be applied to other network optimization problems. It describes a network construction procedure that takes into account a) constraints on the network topology, b) a resource factor that reflects the density of the coefficient matrix, and c) a resource strength, which measures the availability of resources. The strong impact of the chosen parametric characterization of the problems is shown via an in depth computational study. Instances for the single- and multi-mode resource-constrained project scheduling problem are benchmarked by using the state of the art (branch and bound) procedures. The results provided, demonstrate that the classical benchmark instances used by several researchers over decades belong to the subset of the very easy ones. In addition, it is shown that hard instances, being far more smaller in size than presumed in the literature, may not be solved to optimality even within a large amount of computation time.  

\subsubsection{mw6152}
\label{mw:mw6152}

Authors: Carlos A. Méndez, Jaime Cerdá, Ignacio E. Grossmann, Iiro Harjunkoski, Marco Fahl

Title: State-of-the-art review of optimization methods for short-term scheduling of batch processes

Relevance:  0.00

{\scriptsize
\begin{longtable}{p{2cm}p{20cm}}
\caption{Extracted Features from Title and Abstract}\\ \toprule
Type & Concepts Found\\ \midrule
\endhead
\bottomrule
\endfoot
Scheduling & scheduling\\ 
CP & \\ 
Concepts & batch process\\ 
Classification & \\ 
Constraints & \\ 
ApplicationAreas & \\ 
Industries & \\ 
CPSystems & \\ 
Benchmarks & \\ 
Algorithms & \\ 
\end{longtable}
}



\subsubsection{mw7790}
\label{mw:mw7790}

Authors: Mohammad M. Fazel-Zarandi, Oded Berman, J. Christopher Beck

Title: Solving a stochastic facility location/fleet management problem with logic-based Benders' decomposition

Relevance:  0.00

{\scriptsize
\begin{longtable}{p{2cm}p{20cm}}
\caption{Extracted Features from Title and Abstract}\\ \toprule
Type & Concepts Found\\ \midrule
\endhead
\bottomrule
\endfoot
Scheduling & \\ 
CP & \\ 
Concepts & stochastic\\ 
Classification & \\ 
Constraints & \\ 
ApplicationAreas & \\ 
Industries & \\ 
CPSystems & \\ 
Benchmarks & \\ 
Algorithms & \\ 
\end{longtable}
}



\subsubsection{mw8848}
\label{mw:mw8848}

Authors: Christos T. Maravelias

Title: A decomposition framework for the scheduling of single- and multi-stage processes

Relevance:  0.00

{\scriptsize
\begin{longtable}{p{2cm}p{20cm}}
\caption{Extracted Features from Title and Abstract}\\ \toprule
Type & Concepts Found\\ \midrule
\endhead
\bottomrule
\endfoot
Scheduling & scheduling\\ 
CP & \\ 
Concepts & \\ 
Classification & \\ 
Constraints & \\ 
ApplicationAreas & \\ 
Industries & \\ 
CPSystems & \\ 
Benchmarks & \\ 
Algorithms & \\ 
\end{longtable}
}



\subsubsection{mw9812}
\label{mw:mw9812}

Authors: Cynthia Barnhart, Ellis L. Johnson, George L. Nemhauser, Martin W. P. Savelsbergh, Pamela H. Vance

Title: Branch-and-Price: Column Generation for Solving Huge Integer Programs

Relevance:  0.00

{\scriptsize
\begin{longtable}{p{2cm}p{20cm}}
\caption{Extracted Features from Title and Abstract}\\ \toprule
Type & Concepts Found\\ \midrule
\endhead
\bottomrule
\endfoot
Scheduling & \\ 
CP & \\ 
Concepts & \\ 
Classification & \\ 
Constraints & \\ 
ApplicationAreas & \\ 
Industries & \\ 
CPSystems & \\ 
Benchmarks & \\ 
Algorithms & column generation\\ 
\end{longtable}
}

  We discuss formulations of integer programs with a huge number of variables and their solution by column generation methods, i.e., implicit pricing of nonbasic variables to generate new columns or to prove LP optimality at a node of the branch-and-bound tree. We present classes of models for which this approach decomposes the problem, provides tighter LP relaxations, and eliminates symmetry. We then discuss computational issues and implementation of column generation, branch-and-bound algorithms, including special branching rules and efficient ways to solve the LP relaxation. We also discuss the relationship with Lagrangian duality.  

\subsubsection{mw2992}
\label{mw:mw2992}

Authors: Peter Brucker, Olaf Thiele

Title: A branch / and  bound method for the general-shop problem with sequence dependent setup-times

Relevance:  0.00

{\scriptsize
\begin{longtable}{p{2cm}p{20cm}}
\caption{Extracted Features from Title and Abstract}\\ \toprule
Type & Concepts Found\\ \midrule
\endhead
\bottomrule
\endfoot
Scheduling & \\ 
CP & \\ 
Concepts & sequence dependent setup, setup-time\\ 
Classification & \\ 
Constraints & \\ 
ApplicationAreas & \\ 
Industries & \\ 
CPSystems & \\ 
Benchmarks & \\ 
Algorithms & \\ 
\end{longtable}
}



\subsubsection{mw5576}
\label{mw:mw5576}

Authors: Alan S. Manne

Title: On the Job-Shop Scheduling Problem

Relevance:  0.00

{\scriptsize
\begin{longtable}{p{2cm}p{20cm}}
\caption{Extracted Features from Title and Abstract}\\ \toprule
Type & Concepts Found\\ \midrule
\endhead
\bottomrule
\endfoot
Scheduling & scheduling, job, machine\\ 
CP & \\ 
Concepts & job-shop\\ 
Classification & \\ 
Constraints & \\ 
ApplicationAreas & \\ 
Industries & \\ 
CPSystems & \\ 
Benchmarks & \\ 
Algorithms & \\ 
\end{longtable}
}

  This is a proposal for the application of discrete linear programming to the typical job-shop scheduling problem—one that involves both sequencing restrictions and also noninterference constraints for individual pieces of equipment. Thus far, no attempt has been made to establish the computational feasibility of the approach in the case of large-scale realistic problems. This formulation seems, however, to involve considerably fewer variables than two other recent proposals [Bowman, E. H. 1959. The schedule-sequencing problem. Opns Res. 7 621–624; Wagner, H. 1959. An integer linear-programming model for machine scheduling. Naval Res. Log. Quart. (June).], and on these grounds may be worth some computer experimentation.  

\subsubsection{mw455}
\label{mw:mw455}

Authors: H. Fei, N. Meskens, C. Chu

Title: A planning and scheduling problem for an operating theatre using an open scheduling strategy

Relevance:  0.00

{\scriptsize
\begin{longtable}{p{2cm}p{20cm}}
\caption{Extracted Features from Title and Abstract}\\ \toprule
Type & Concepts Found\\ \midrule
\endhead
\bottomrule
\endfoot
Scheduling & scheduling\\ 
CP & \\ 
Concepts & \\ 
Classification & \\ 
Constraints & \\ 
ApplicationAreas & \\ 
Industries & \\ 
CPSystems & \\ 
Benchmarks & \\ 
Algorithms & \\ 
\end{longtable}
}



\subsubsection{mw1064}
\label{mw:mw1064}

Authors: Ugo Montanari

Title: Networks of constraints: Fundamental properties and applications to picture processing

Relevance:  0.00

{\scriptsize
\begin{longtable}{p{2cm}p{20cm}}
\caption{Extracted Features from Title and Abstract}\\ \toprule
Type & Concepts Found\\ \midrule
\endhead
\bottomrule
\endfoot
Scheduling & \\ 
CP & \\ 
Concepts & \\ 
Classification & \\ 
Constraints & \\ 
ApplicationAreas & \\ 
Industries & \\ 
CPSystems & \\ 
Benchmarks & \\ 
Algorithms & \\ 
\end{longtable}
}



\subsubsection{mw2739}
\label{mw:mw2739}

Authors: J. K. Lenstra, A. H. G. Rinnooy Kan, P. Brucker

Title: Complexity of Machine Scheduling Problems

Relevance:  0.00

{\scriptsize
\begin{longtable}{p{2cm}p{20cm}}
\caption{Extracted Features from Title and Abstract}\\ \toprule
Type & Concepts Found\\ \midrule
\endhead
\bottomrule
\endfoot
Scheduling & scheduling, machine\\ 
CP & \\ 
Concepts & \\ 
Classification & \\ 
Constraints & \\ 
ApplicationAreas & \\ 
Industries & \\ 
CPSystems & \\ 
Benchmarks & \\ 
Algorithms & \\ 
\end{longtable}
}



\subsubsection{mw3182}
\label{mw:mw3182}

Authors: S. Kirkpatrick, C. D. Gelatt, M. P. Vecchi

Title: Optimization by Simulated Annealing

Relevance:  0.00

{\scriptsize
\begin{longtable}{p{2cm}p{20cm}}
\caption{Extracted Features from Title and Abstract}\\ \toprule
Type & Concepts Found\\ \midrule
\endhead
\bottomrule
\endfoot
Scheduling & \\ 
CP & \\ 
Concepts & \\ 
Classification & \\ 
Constraints & \\ 
ApplicationAreas & \\ 
Industries & \\ 
CPSystems & \\ 
Benchmarks & \\ 
Algorithms & simulated annealing\\ 
\end{longtable}
}

 There is a deep and useful connection between statistical mechanics (the behavior of systems with many degrees of freedom in thermal equilibrium at a finite temperature) and multivariate or combinatorial optimization (finding the minimum of a given function depending on many parameters). A detailed analogy with annealing in solids provides a framework for optimization of the properties of very large and complex systems. This connection to statistical mechanics exposes new information and provides an unfamiliar perspective on traditional optimization problems and methods. 

\subsubsection{mw3289}
\label{mw:mw3289}

Authors: Brecht Cardoen, Erik Demeulemeester, Jeroen Beliën

Title: Operating room planning and scheduling: A literature review

Relevance:  0.00

{\scriptsize
\begin{longtable}{p{2cm}p{20cm}}
\caption{Extracted Features from Title and Abstract}\\ \toprule
Type & Concepts Found\\ \midrule
\endhead
\bottomrule
\endfoot
Scheduling & scheduling\\ 
CP & \\ 
Concepts & \\ 
Classification & \\ 
Constraints & \\ 
ApplicationAreas & operating room\\ 
Industries & \\ 
CPSystems & \\ 
Benchmarks & \\ 
Algorithms & \\ 
\end{longtable}
}



\subsubsection{mw3440}
\label{mw:mw3440}

Authors: Aristide Mingozzi, Vittorio Maniezzo, Salvatore Ricciardelli, Lucio Bianco

Title: An Exact Algorithm for the Resource-Constrained Project Scheduling Problem Based on a New Mathematical Formulation

Relevance:  0.00

{\scriptsize
\begin{longtable}{p{2cm}p{20cm}}
\caption{Extracted Features from Title and Abstract}\\ \toprule
Type & Concepts Found\\ \midrule
\endhead
\bottomrule
\endfoot
Scheduling & scheduling, resource\\ 
CP & \\ 
Concepts & precedence, make-span\\ 
Classification & Resource-constrained Project Scheduling Problem\\ 
Constraints & \\ 
ApplicationAreas & \\ 
Industries & \\ 
CPSystems & \\ 
Benchmarks & \\ 
Algorithms & \\ 
\end{longtable}
}

  In this paper we consider the Project Scheduling Problem with resource constraints, where the objective is to minimize the project makespan. We present a new 0-1 linear programming formulation of the problem that requires an exponential number of variables, corresponding to all feasible subsets of activities that can be simultaneously executed without violating resource or precedence constraints. Different relaxations of the above formulation are used to derive new lower bounds, which dominate the value of the longest path on the precedence graph and are tighter than the bound proposed by Stinson et al. (1978).    A tree search algorithm, based on the above formulation, that uses new lower bounds and dominance criteria is also presented. Computational results indicate that the exact algorithm can solve hard instances that cannot be solved by the best algorithms reported in the literature.  

\subsubsection{mw4347}
\label{mw:mw4347}

Authors: Yingyi Chu, Quanshi Xia

Title: Generating Benders Cuts for a General Class of Integer Programming Problems

Relevance:  0.00

{\scriptsize
\begin{longtable}{p{2cm}p{20cm}}
\caption{Extracted Features from Title and Abstract}\\ \toprule
Type & Concepts Found\\ \midrule
\endhead
\bottomrule
\endfoot
Scheduling & \\ 
CP & \\ 
Concepts & \\ 
Classification & \\ 
Constraints & \\ 
ApplicationAreas & \\ 
Industries & \\ 
CPSystems & \\ 
Benchmarks & \\ 
Algorithms & \\ 
\end{longtable}
}



\subsubsection{mw4975}
\label{mw:mw4975}

Authors: Bahman Naderi, Vahid Roshanaei

Title: Branch-Relax-and-Check: A tractable decomposition method for order acceptance and identical parallel machine scheduling

Relevance:  0.00

{\scriptsize
\begin{longtable}{p{2cm}p{20cm}}
\caption{Extracted Features from Title and Abstract}\\ \toprule
Type & Concepts Found\\ \midrule
\endhead
\bottomrule
\endfoot
Scheduling & scheduling, order, machine\\ 
CP & \\ 
Concepts & \\ 
Classification & parallel machine\\ 
Constraints & \\ 
ApplicationAreas & \\ 
Industries & \\ 
CPSystems & \\ 
Benchmarks & \\ 
Algorithms & \\ 
\end{longtable}
}



\subsubsection{mw5469}
\label{mw:mw5469}

Authors: Pedro M. Castro, Ignacio E. Grossmann, Augusto Q. Novais

Title: Two New Continuous-Time Models for the Scheduling of Multistage Batch Plants with Sequence Dependent Changeovers

Relevance:  0.00

{\scriptsize
\begin{longtable}{p{2cm}p{20cm}}
\caption{Extracted Features from Title and Abstract}\\ \toprule
Type & Concepts Found\\ \midrule
\endhead
\bottomrule
\endfoot
Scheduling & scheduling\\ 
CP & \\ 
Concepts & \\ 
Classification & \\ 
Constraints & \\ 
ApplicationAreas & \\ 
Industries & \\ 
CPSystems & \\ 
Benchmarks & \\ 
Algorithms & \\ 
\end{longtable}
}



\subsubsection{mw5774}
\label{mw:mw5774}

Authors: Cemal Özgüven, Lale Özbakır, Yasemin Yavuz

Title: Mathematical models for job-shop scheduling problems with routing and process plan flexibility

Relevance:  0.00

{\scriptsize
\begin{longtable}{p{2cm}p{20cm}}
\caption{Extracted Features from Title and Abstract}\\ \toprule
Type & Concepts Found\\ \midrule
\endhead
\bottomrule
\endfoot
Scheduling & scheduling, job\\ 
CP & \\ 
Concepts & job-shop\\ 
Classification & \\ 
Constraints & \\ 
ApplicationAreas & \\ 
Industries & \\ 
CPSystems & \\ 
Benchmarks & \\ 
Algorithms & \\ 
\end{longtable}
}



\subsubsection{mw5982}
\label{mw:mw5982}

Authors: Vahid Roshanaei, Curtiss Luong, Dionne M. Aleman, David R. Urbach

Title: Reformulation, linearization, and decomposition techniques for balanced distributed operating room scheduling

Relevance:  0.00

{\scriptsize
\begin{longtable}{p{2cm}p{20cm}}
\caption{Extracted Features from Title and Abstract}\\ \toprule
Type & Concepts Found\\ \midrule
\endhead
\bottomrule
\endfoot
Scheduling & scheduling\\ 
CP & \\ 
Concepts & distributed\\ 
Classification & \\ 
Constraints & \\ 
ApplicationAreas & operating room\\ 
Industries & \\ 
CPSystems & \\ 
Benchmarks & \\ 
Algorithms & \\ 
\end{longtable}
}



\subsubsection{mw6237}
\label{mw:mw6237}

Authors: Philippe Refalo

Title: Impact-Based Search Strategies for Constraint Programming

Relevance:  0.00

{\scriptsize
\begin{longtable}{p{2cm}p{20cm}}
\caption{Extracted Features from Title and Abstract}\\ \toprule
Type & Concepts Found\\ \midrule
\endhead
\bottomrule
\endfoot
Scheduling & \\ 
CP & constraint programming\\ 
Concepts & \\ 
Classification & \\ 
Constraints & \\ 
ApplicationAreas & \\ 
Industries & \\ 
CPSystems & \\ 
Benchmarks & \\ 
Algorithms & \\ 
\end{longtable}
}



\subsubsection{mw7559}
\label{mw:mw7559}

Authors: Alain Colmerauer

Title: An introduction to Prolog III

Relevance:  0.00

{\scriptsize
\begin{longtable}{p{2cm}p{20cm}}
\caption{Extracted Features from Title and Abstract}\\ \toprule
Type & Concepts Found\\ \midrule
\endhead
\bottomrule
\endfoot
Scheduling & \\ 
CP & \\ 
Concepts & \\ 
Classification & \\ 
Constraints & \\ 
ApplicationAreas & \\ 
Industries & \\ 
CPSystems & \\ 
Benchmarks & \\ 
Algorithms & \\ 
\end{longtable}
}

 The Prolog III programming language extends Prolog by redefining the fundamental process at its heart: unification. This article presents the specifications of this new language and illustrates its capabilities. 

\subsubsection{mw7675}
\label{mw:mw7675}

Authors: Sönke Hartmann, Rainer Kolisch

Title: Experimental evaluation of state-of-the-art heuristics for the resource-constrained project scheduling problem

Relevance:  0.00

{\scriptsize
\begin{longtable}{p{2cm}p{20cm}}
\caption{Extracted Features from Title and Abstract}\\ \toprule
Type & Concepts Found\\ \midrule
\endhead
\bottomrule
\endfoot
Scheduling & scheduling, resource\\ 
CP & \\ 
Concepts & \\ 
Classification & Resource-constrained Project Scheduling Problem\\ 
Constraints & \\ 
ApplicationAreas & \\ 
Industries & \\ 
CPSystems & \\ 
Benchmarks & \\ 
Algorithms & \\ 
\end{longtable}
}



\subsubsection{mw8692}
\label{mw:mw8692}

Authors: R. Kolisch, R. Padman

Title: An integrated survey of deterministic project scheduling

Relevance:  0.00

{\scriptsize
\begin{longtable}{p{2cm}p{20cm}}
\caption{Extracted Features from Title and Abstract}\\ \toprule
Type & Concepts Found\\ \midrule
\endhead
\bottomrule
\endfoot
Scheduling & scheduling\\ 
CP & \\ 
Concepts & \\ 
Classification & \\ 
Constraints & \\ 
ApplicationAreas & \\ 
Industries & \\ 
CPSystems & \\ 
Benchmarks & \\ 
Algorithms & \\ 
\end{longtable}
}



\subsubsection{mw9605}
\label{mw:mw9605}

Authors: Dinh-Nguyen Pham, Andreas Klinkert

Title: Surgical case scheduling as a generalized job shop scheduling problem

Relevance:  0.00

{\scriptsize
\begin{longtable}{p{2cm}p{20cm}}
\caption{Extracted Features from Title and Abstract}\\ \toprule
Type & Concepts Found\\ \midrule
\endhead
\bottomrule
\endfoot
Scheduling & scheduling, job\\ 
CP & \\ 
Concepts & job-shop\\ 
Classification & \\ 
Constraints & \\ 
ApplicationAreas & \\ 
Industries & \\ 
CPSystems & \\ 
Benchmarks & \\ 
Algorithms & \\ 
\end{longtable}
}



\subsubsection{mw223}
\label{mw:mw223}

Authors: Oumar Koné, Christian Artigues, Pierre Lopez, Marcel Mongeau

Title: Event-based MILP models for resource-constrained project scheduling problems

Relevance:  0.00

{\scriptsize
\begin{longtable}{p{2cm}p{20cm}}
\caption{Extracted Features from Title and Abstract}\\ \toprule
Type & Concepts Found\\ \midrule
\endhead
\bottomrule
\endfoot
Scheduling & scheduling, resource\\ 
CP & \\ 
Concepts & \\ 
Classification & Resource-constrained Project Scheduling Problem\\ 
Constraints & \\ 
ApplicationAreas & \\ 
Industries & \\ 
CPSystems & \\ 
Benchmarks & \\ 
Algorithms & \\ 
\end{longtable}
}



\subsubsection{mw381}
\label{mw:mw381}

Authors: Parviz Fattahi, Mohammad Saidi Mehrabad, Fariborz Jolai

Title: Mathematical modeling and heuristic approaches to flexible job shop scheduling problems

Relevance:  0.00

{\scriptsize
\begin{longtable}{p{2cm}p{20cm}}
\caption{Extracted Features from Title and Abstract}\\ \toprule
Type & Concepts Found\\ \midrule
\endhead
\bottomrule
\endfoot
Scheduling & scheduling, job\\ 
CP & \\ 
Concepts & job-shop\\ 
Classification & \\ 
Constraints & \\ 
ApplicationAreas & \\ 
Industries & \\ 
CPSystems & \\ 
Benchmarks & \\ 
Algorithms & \\ 
\end{longtable}
}



\subsubsection{mw1740}
\label{mw:mw1740}

Authors: Francesca Rossi, Peter van Beek, Toby Walsh

Title: Chapter 4 Constraint Programming

Relevance:  0.00

{\scriptsize
\begin{longtable}{p{2cm}p{20cm}}
\caption{Extracted Features from Title and Abstract}\\ \toprule
Type & Concepts Found\\ \midrule
\endhead
\bottomrule
\endfoot
Scheduling & \\ 
CP & constraint programming\\ 
Concepts & \\ 
Classification & \\ 
Constraints & \\ 
ApplicationAreas & \\ 
Industries & \\ 
CPSystems & \\ 
Benchmarks & \\ 
Algorithms & \\ 
\end{longtable}
}



\subsubsection{mw1906}
\label{mw:mw1906}

Authors: Aïda Jebali, Atidel B. Hadj Alouane, Pierre Ladet

Title: Operating rooms scheduling

Relevance:  0.00

{\scriptsize
\begin{longtable}{p{2cm}p{20cm}}
\caption{Extracted Features from Title and Abstract}\\ \toprule
Type & Concepts Found\\ \midrule
\endhead
\bottomrule
\endfoot
Scheduling & scheduling\\ 
CP & \\ 
Concepts & \\ 
Classification & \\ 
Constraints & \\ 
ApplicationAreas & operating room\\ 
Industries & \\ 
CPSystems & \\ 
Benchmarks & \\ 
Algorithms & \\ 
\end{longtable}
}



\subsubsection{mw2956}
\label{mw:mw2956}

Authors: Willy Herroelen, Roel Leus

Title: Project scheduling under uncertainty: Survey and research potentials

Relevance:  0.00

{\scriptsize
\begin{longtable}{p{2cm}p{20cm}}
\caption{Extracted Features from Title and Abstract}\\ \toprule
Type & Concepts Found\\ \midrule
\endhead
\bottomrule
\endfoot
Scheduling & scheduling\\ 
CP & \\ 
Concepts & \\ 
Classification & \\ 
Constraints & \\ 
ApplicationAreas & \\ 
Industries & \\ 
CPSystems & \\ 
Benchmarks & \\ 
Algorithms & \\ 
\end{longtable}
}



\subsubsection{mw4102}
\label{mw:mw4102}

Authors: Mauro Dell'Amico, Marco Trubian

Title: Applying tabu search to the job-shop scheduling problem

Relevance:  0.00

{\scriptsize
\begin{longtable}{p{2cm}p{20cm}}
\caption{Extracted Features from Title and Abstract}\\ \toprule
Type & Concepts Found\\ \midrule
\endhead
\bottomrule
\endfoot
Scheduling & scheduling, job\\ 
CP & \\ 
Concepts & job-shop\\ 
Classification & \\ 
Constraints & \\ 
ApplicationAreas & \\ 
Industries & \\ 
CPSystems & \\ 
Benchmarks & \\ 
Algorithms & \\ 
\end{longtable}
}



\subsubsection{mw4679}
\label{mw:mw4679}

Authors: David Wheatley, Fatma Gzara, Elizabeth Jewkes

Title: Logic-based Benders decomposition for an inventory-location problem with service constraints

Relevance:  0.00

{\scriptsize
\begin{longtable}{p{2cm}p{20cm}}
\caption{Extracted Features from Title and Abstract}\\ \toprule
Type & Concepts Found\\ \midrule
\endhead
\bottomrule
\endfoot
Scheduling & \\ 
CP & \\ 
Concepts & inventory, Benders Decomposition, Logic-Based Benders Decomposition\\ 
Classification & \\ 
Constraints & \\ 
ApplicationAreas & \\ 
Industries & \\ 
CPSystems & \\ 
Benchmarks & \\ 
Algorithms & \\ 
\end{longtable}
}



\subsubsection{mw4944}
\label{mw:mw4944}

Authors: S. M. Johnson

Title: Optimal two‐ and three‐stage production schedules with setup times included

Relevance:  0.00

{\scriptsize
\begin{longtable}{p{2cm}p{20cm}}
\caption{Extracted Features from Title and Abstract}\\ \toprule
Type & Concepts Found\\ \midrule
\endhead
\bottomrule
\endfoot
Scheduling & scheduling, machine\\ 
CP & \\ 
Concepts & setup-time\\ 
Classification & \\ 
Constraints & \\ 
ApplicationAreas & \\ 
Industries & \\ 
CPSystems & \\ 
Benchmarks & \\ 
Algorithms & \\ 
\end{longtable}
}

 Abstract  Each of a collection of items are to be produced on two machines (or stages). Each machine can handle only one item at a time and each item must be processed through machine one and then through machine two. The setup time plus work time for each item for each machine is known. A simple decision rule is obtained in this paper for the optimal scheduling of the production so that the total elapsed time is a minimum. A three‐machine problem is also discussed and solved for a restricted case. 

\subsubsection{mw6057}
\label{mw:mw6057}

Authors: Harvey M. Wagner

Title: An integer linear‐programming model for machine scheduling

Relevance:  0.00

{\scriptsize
\begin{longtable}{p{2cm}p{20cm}}
\caption{Extracted Features from Title and Abstract}\\ \toprule
Type & Concepts Found\\ \midrule
\endhead
\bottomrule
\endfoot
Scheduling & scheduling, machine\\ 
CP & \\ 
Concepts & \\ 
Classification & \\ 
Constraints & \\ 
ApplicationAreas & \\ 
Industries & \\ 
CPSystems & \\ 
Benchmarks & \\ 
Algorithms & \\ 
\end{longtable}
}



\subsubsection{mw6227}
\label{mw:mw6227}

Authors: Gilles Pesant

Title: A Regular Language Membership Constraint for Finite Sequences of Variables

Relevance:  0.00

{\scriptsize
\begin{longtable}{p{2cm}p{20cm}}
\caption{Extracted Features from Title and Abstract}\\ \toprule
Type & Concepts Found\\ \midrule
\endhead
\bottomrule
\endfoot
Scheduling & \\ 
CP & \\ 
Concepts & \\ 
Classification & \\ 
Constraints & \\ 
ApplicationAreas & \\ 
Industries & \\ 
CPSystems & \\ 
Benchmarks & \\ 
Algorithms & \\ 
\end{longtable}
}



\subsubsection{mw6257}
\label{mw:mw6257}

Authors: George L. Nemhauser, Michael A. Trick

Title: Scheduling A Major College Basketball Conference

Relevance:  0.00

{\scriptsize
\begin{longtable}{p{2cm}p{20cm}}
\caption{Extracted Features from Title and Abstract}\\ \toprule
Type & Concepts Found\\ \midrule
\endhead
\bottomrule
\endfoot
Scheduling & scheduling\\ 
CP & \\ 
Concepts & \\ 
Classification & \\ 
Constraints & \\ 
ApplicationAreas & \\ 
Industries & \\ 
CPSystems & \\ 
Benchmarks & \\ 
Algorithms & \\ 
\end{longtable}
}

  The nine universities in the Atlantic Coast Conference (ACC) have a basketball competition in which each school plays home and away games against each other over a nine-week period. The creation of a suitable schedule is a very difficult problem with a myriad of conflicting requirements and preferences. We develop an approach to scheduling problems that uses a combination of integer programming and enumerative techniques. Our approach yields reasonable schedules very quickly and gave a schedule that was accepted by the ACC for play in 1997-1998.  

\subsubsection{mw9536}
\label{mw:mw9536}

Authors: Vincent Van Peteghem, Mario Vanhoucke

Title: A genetic algorithm for the preemptive and non-preemptive multi-mode resource-constrained project scheduling problem

Relevance:  0.00

{\scriptsize
\begin{longtable}{p{2cm}p{20cm}}
\caption{Extracted Features from Title and Abstract}\\ \toprule
Type & Concepts Found\\ \midrule
\endhead
\bottomrule
\endfoot
Scheduling & scheduling, resource\\ 
CP & \\ 
Concepts & preempt, preemptive\\ 
Classification & Resource-constrained Project Scheduling Problem\\ 
Constraints & \\ 
ApplicationAreas & \\ 
Industries & \\ 
CPSystems & \\ 
Benchmarks & \\ 
Algorithms & genetic algorithm\\ 
\end{longtable}
}



\subsubsection{mw1582}
\label{mw:mw1582}

Authors: Andrei Horbach

Title: A Boolean satisfiability approach to the resource-constrained project scheduling problem

Relevance:  0.00

{\scriptsize
\begin{longtable}{p{2cm}p{20cm}}
\caption{Extracted Features from Title and Abstract}\\ \toprule
Type & Concepts Found\\ \midrule
\endhead
\bottomrule
\endfoot
Scheduling & scheduling, resource\\ 
CP & \\ 
Concepts & \\ 
Classification & Resource-constrained Project Scheduling Problem\\ 
Constraints & \\ 
ApplicationAreas & \\ 
Industries & \\ 
CPSystems & \\ 
Benchmarks & \\ 
Algorithms & \\ 
\end{longtable}
}



\subsubsection{mw2770}
\label{mw:mw2770}

Authors: Ramin Barzanji, Bahman Naderi, Mehmet A. Begen

Title: Decomposition algorithms for the integrated process planning and scheduling problem

Relevance:  0.00

{\scriptsize
\begin{longtable}{p{2cm}p{20cm}}
\caption{Extracted Features from Title and Abstract}\\ \toprule
Type & Concepts Found\\ \midrule
\endhead
\bottomrule
\endfoot
Scheduling & scheduling\\ 
CP & \\ 
Concepts & \\ 
Classification & \\ 
Constraints & \\ 
ApplicationAreas & \\ 
Industries & \\ 
CPSystems & \\ 
Benchmarks & \\ 
Algorithms & \\ 
\end{longtable}
}



\subsubsection{mw3234}
\label{mw:mw3234}

Authors: F. Brian Talbot

Title: Resource-Constrained Project Scheduling with Time-Resource Tradeoffs: The Nonpreemptive Case

Relevance:  0.00

{\scriptsize
\begin{longtable}{p{2cm}p{20cm}}
\caption{Extracted Features from Title and Abstract}\\ \toprule
Type & Concepts Found\\ \midrule
\endhead
\bottomrule
\endfoot
Scheduling & scheduling, order, job, resource\\ 
CP & \\ 
Concepts & preempt, completion-time, preemptive\\ 
Classification & Resource-constrained Project Scheduling Problem\\ 
Constraints & \\ 
ApplicationAreas & \\ 
Industries & \\ 
CPSystems & \\ 
Benchmarks & \\ 
Algorithms & \\ 
\end{longtable}
}

  This paper introduces methods for formulating and solving a general class of nonpreemptive resource-constrained project scheduling problems in which the duration of each job is a function of the resources committed to it. The approach is broad enough to permit the evaluation of numerous time or resource-based objective functions, while simultaneously taking into account a variety of constraint types. Typical of the objective functions permitted are minimize project duration, minimize project cost given performance payments and penalties, and minimize the consumption of a critical resource. Resources which may be considered include those which are limited on a period-to-period basis such as skilled labor, as well as those such as money, which are consumed and constrained over the life of the project. At the planning stage the user of this approach is permitted to identify several alternative ways, or modes, of accomplishing each job in the project. Each mode may have a different duration, reflecting the magnitude and mix of the resources allocated to it. At the scheduling phase, the procedure derives a solution which specifies how each job should be performed, that is, which mode should be selected, and when each mode should be scheduled. In order to make the presentation concrete, this paper focuses on two problems: given multiple resource restrictions, minimize project completion time, and minimize project cost. The latter problem is also known as the resource-constrained time-cost tradeoff problem.    Computational results indicate that the procedures provide cost-effective optimal solutions for small problems and good heuristic solutions for larger problems. The programmed solution algorithms are relatively simple and require only modest computing facilities, which permits them to be potentially useful scheduling tools for organizations having small computer systems.  

\subsubsection{mw3457}
\label{mw:mw3457}

Authors: Peter J. M. van Laarhoven, Emile H. L. Aarts, Jan Karel Lenstra

Title: Job Shop Scheduling by Simulated Annealing

Relevance:  0.00

{\scriptsize
\begin{longtable}{p{2cm}p{20cm}}
\caption{Extracted Features from Title and Abstract}\\ \toprule
Type & Concepts Found\\ \midrule
\endhead
\bottomrule
\endfoot
Scheduling & scheduling, job\\ 
CP & \\ 
Concepts & make-span, job-shop\\ 
Classification & \\ 
Constraints & \\ 
ApplicationAreas & \\ 
Industries & \\ 
CPSystems & \\ 
Benchmarks & \\ 
Algorithms & simulated annealing\\ 
\end{longtable}
}

  We describe an approximation algorithm for the problem of finding the minimum makespan in a job shop. The algorithm is based on simulated annealing, a generalization of the well known iterative improvement approach to combinatorial optimization problems. The generalization involves the acceptance of cost-increasing transitions with a nonzero probability to avoid getting stuck in local minima. We prove that our algorithm asymptotically converges in probability to a globally minimal solution, despite the fact that the Markov chains generated by the algorithm are generally not irreducible. Computational experiments show that our algorithm can find shorter makespans than two recent approximation approaches that are more tailored to the job shop scheduling problem. This is, however, at the cost of large running times.  

\subsubsection{mw3823}
\label{mw:mw3823}

Authors: J. Carlier, E. Néron

Title: On linear lower bounds for the resource constrained project scheduling problem

Relevance:  0.00

{\scriptsize
\begin{longtable}{p{2cm}p{20cm}}
\caption{Extracted Features from Title and Abstract}\\ \toprule
Type & Concepts Found\\ \midrule
\endhead
\bottomrule
\endfoot
Scheduling & scheduling, resource\\ 
CP & \\ 
Concepts & \\ 
Classification & Resource-constrained Project Scheduling Problem\\ 
Constraints & \\ 
ApplicationAreas & \\ 
Industries & \\ 
CPSystems & \\ 
Benchmarks & \\ 
Algorithms & \\ 
\end{longtable}
}



\subsection{Excluded Missing Works}

\subsubsection{mw15641}
\label{mw:mw15641}

Authors: Rina Dechter, Francesca Rossi

Title: Constraint Satisfaction

Relevance:  3.75

{\scriptsize
\begin{longtable}{p{2cm}p{20cm}}
\caption{Extracted Features from Title and Abstract}\\ \toprule
Type & Concepts Found\\ \midrule
\endhead
\bottomrule
\endfoot
Scheduling & scheduling, resource\\ 
CP & constraint programming, constraint satisfaction, propagation\\ 
Concepts & \\ 
Classification & \\ 
Constraints & \\ 
ApplicationAreas & \\ 
Industries & \\ 
CPSystems & \\ 
Benchmarks & real-life\\ 
Algorithms & \\ 
\end{longtable}
}

 Abstract 
           Constraints are a formalism for the representation of declarative knowledge that allows for a compact and expressive modeling of many real‐life problems. Constraint satisfaction and propagation tools, as well as constraint programming languages, are successfully used to model, solve, and reason about many classes of problems, such as design, diagnosis, scheduling, spatio‐temporal reasoning, resource allocation, configuration, network optimization, and graphical interfaces. 

\subsubsection{mw8231}
\label{mw:mw8231}

Authors: Cheng-Chung Cheng, Stephen F. Smith

Title: Applying Constraint Satisfaction Techniques to Job Shop Scheduling.

Relevance:  2.00

{\scriptsize
\begin{longtable}{p{2cm}p{20cm}}
\caption{Extracted Features from Title and Abstract}\\ \toprule
Type & Concepts Found\\ \midrule
\endhead
\bottomrule
\endfoot
Scheduling & scheduling, job\\ 
CP & constraint satisfaction\\ 
Concepts & job-shop\\ 
Classification & \\ 
Constraints & \\ 
ApplicationAreas & \\ 
Industries & \\ 
CPSystems & \\ 
Benchmarks & \\ 
Algorithms & \\ 
\end{longtable}
}



\subsubsection{mw19351}
\label{mw:mw19351}

Authors: Laurent Michel, Pascal Van Hentenryck

Title: BasicCPTheory: Search

Relevance:  2.00

{\scriptsize
\begin{longtable}{p{2cm}p{20cm}}
\caption{Extracted Features from Title and Abstract}\\ \toprule
Type & Concepts Found\\ \midrule
\endhead
\bottomrule
\endfoot
Scheduling & scheduling\\ 
CP & constraint programming, CP\\ 
Concepts & distributed\\ 
Classification & \\ 
Constraints & \\ 
ApplicationAreas & \\ 
Industries & \\ 
CPSystems & \\ 
Benchmarks & \\ 
Algorithms & \\ 
\end{longtable}
}

 Abstract  Constraint programming (CP) is based on the fundamental idea that the resolution of hard optimization problems is best approached by the combination of a declarative model specification and the writing of a search procedure. The search procedure explores the underlying search space while taking advantage of the pruning derived from the declarative model. The search is a fundamental ingredient to exploit the specificities of problems that are difficult or near impossible to capture through the declarative model. The flexibility it provides lead to marked successes for truly challenging problems in scheduling, bounded program verification, test generation, and distributed systems, to name just a few. This article offers a gentle introduction to search in CP, what it can do, and how it is used. 





\printindex

\end{document}

