\PassOptionsToPackage{table}{xcolor}
\documentclass[a4paper]{article}
\usepackage[a4paper,margin=2cm,landscape]{geometry}
\usepackage{tabularx}
\usepackage{tikz}
\usepackage{graphicx}
\usepackage{rotating}
\usepackage{float}
\usepackage{calc}
\usepackage{pdflscape}
\usepackage{booktabs}
\usepackage{colortbl}
\usepackage{longtable}
\usepackage{stackengine}
\usepackage{multicol}
%\usepackage{showkeys}
\newcounter{rowcounter}
\newcommand{\rowlabel}[1]{\refstepcounter{rowcounter}\label{#1}}

\usepackage{url}
\usepackage{hyperref}

\newcommand{\su}[1]{\Shortunderstack[l]{#1}}

\title{CP Papers on Scheduling}
\author{Helmut Simonis and Cemalettin Öztürk}
\begin{document}
\rowcolors{2}{gray!20}{white}

\maketitle
\section{Introduction}

This document shows the result of a survey on "Constraint Programming and Scheduling", which tries to find and classify all publications on the combination of these two concepts. It is based on a manually collected bibfile containing reference to relevant papers and articles, and on an automatic and manual analysis of local copies of the cited papers. For copyright reasons, we are obviously not able to distribute the collected copies, but we provide links to the original sources of the files. 

We identify the papers by a key which is the last name of the first author, the first character of the last names of all other authors, and a two digit year code for the date of publication. If multiple works would define the same key, we differentiate by adding a suffix "a", "b", etc, to the second and subsequent works found.

Most of the content of this document is generated by a Java program that parses the bib files, adds any manually extracted information, and which then extracts concept occurrences from the local copies of the works. It then produces tables and other LaTeX  artifacts that are included in a manually defined top-level document.

To add new works, first add bibtex entries for each work in the main \texttt{overview/bib.bib} file, then add local copies of the pdf of the work to the \texttt{overview/works/} directory, using the key of the bibtex entry as the file name (plus extension .pdf), and then run the main Java program \texttt{org.insightcentre.pthg24.JfxApp} to consolidate the information and extract the relevant concepts. Finally, run \texttt{pdflatex} on the \texttt{overview/scheduling.tex} file to produce this pdf document. Manually extracted information for the files can be added in the \texttt{imports/manual.csv} file. New concepts can be added in the file \texttt{imports/concepts.json}, new concept types need to be directly defined in the Java code.

We start the document by providing a table of all defined keys in the bib file in alphabetical order. This table can be helpful to see if a candidate paper is already in the survey, it suffices to see if the key is already present, and matches the authors, title and origin of the candidate paper. In the table link given by the key points to the local copy of the file, while the citation number links to the bibliography entry. That entry typically also contains a link to the original source of the paper.

This document heavily depends on the use of hyper links in the document, it has been tested with Acrobat Reader, other pdf reader may not use links in the same way. 

\clearpage
\begin{longtable}{*{6}{l}}
\rowcolor{white}\caption{Key Overview (Total: 1207)}\\ \toprule
\rowcolor{white}1 & 2 & 3 & 4 & 5 & 6\\ \midrule
\endhead
\bottomrule
\endfoot
\href{../}{2007}~\cite{2007} & \href{../works/AalianPG23.pdf}{AalianPG23}~\cite{AalianPG23} & \href{../works/AbdennadherS99.pdf}{AbdennadherS99}~\cite{AbdennadherS99} & \href{../works/Abdul-Niby2016.pdf}{Abdul-Niby2016}~\cite{Abdul-Niby2016} & \href{../works/AbidinK20.pdf}{AbidinK20}~\cite{AbidinK20} & \href{../works/AbohashimaEG21.pdf}{AbohashimaEG21}~\cite{AbohashimaEG21}\\ 
\href{../works/Abreu2023.pdf}{Abreu2023}~\cite{Abreu2023} & \href{../works/AbreuAPNM21.pdf}{AbreuAPNM21}~\cite{AbreuAPNM21} & \href{../works/AbreuN22.pdf}{AbreuN22}~\cite{AbreuN22} & \href{../works/AbreuNP23.pdf}{AbreuNP23}~\cite{AbreuNP23} & \href{../works/AbreuPNF23.pdf}{AbreuPNF23}~\cite{AbreuPNF23} & \href{../works/AbrilSB05.pdf}{AbrilSB05}~\cite{AbrilSB05}\\ 
\href{../}{Abuwarda2019}~\cite{Abuwarda2019} & \href{../works/AchterbergBKW08.pdf}{AchterbergBKW08}~\cite{AchterbergBKW08} & \href{../works/Acuna-Agost2011.pdf}{Acuna-Agost2011}~\cite{Acuna-Agost2011} & \href{../works/Acuna-AgostMFG09.pdf}{Acuna-AgostMFG09}~\cite{Acuna-AgostMFG09} & \href{../works/Adelgren2023.pdf}{Adelgren2023}~\cite{Adelgren2023} & \href{../works/AfsarVPG23.pdf}{AfsarVPG23}~\cite{AfsarVPG23}\\ 
\href{../works/AggounB93.pdf}{AggounB93}~\cite{AggounB93} & \href{../}{AggounMV08}~\cite{AggounMV08} & \href{../}{AggounV04}~\cite{AggounV04} & \href{../works/AgussurjaKL18.pdf}{AgussurjaKL18}~\cite{AgussurjaKL18} & \href{../}{Ahmadi-Javid2023}~\cite{Ahmadi-Javid2023} & \href{../works/Ahmed2006.pdf}{Ahmed2006}~\cite{Ahmed2006}\\ 
\href{../}{AjiliW04}~\cite{AjiliW04} & \href{../works/Akan2023.pdf}{Akan2023}~\cite{Akan2023} & \href{../works/AkkerDH07.pdf}{AkkerDH07}~\cite{AkkerDH07} & \href{../works/AkramNHRSA23.pdf}{AkramNHRSA23}~\cite{AkramNHRSA23} & \href{../works/Alaka21.pdf}{Alaka21}~\cite{Alaka21} & \href{../works/AlakaP23.pdf}{AlakaP23}~\cite{AlakaP23}\\ 
\href{../works/AlakaPY19.pdf}{AlakaPY19}~\cite{AlakaPY19} & \href{../works/Alesio2013.pdf}{Alesio2013}~\cite{Alesio2013} & \href{../works/AlesioBNG15.pdf}{AlesioBNG15}~\cite{AlesioBNG15} & \href{../works/AlesioNBG14.pdf}{AlesioNBG14}~\cite{AlesioNBG14} & \href{../works/AlfieriGPS23.pdf}{AlfieriGPS23}~\cite{AlfieriGPS23} & \href{../}{AlizdehS20}~\cite{AlizdehS20}\\ 
\href{../works/Amadini2014.pdf}{Amadini2014}~\cite{Amadini2014} & \href{../works/AmadiniGM16.pdf}{AmadiniGM16}~\cite{AmadiniGM16} & \href{../}{Ammar2013}~\cite{Ammar2013} & \href{../works/AngelsmarkJ00.pdf}{AngelsmarkJ00}~\cite{AngelsmarkJ00} & \href{../works/AntunesABD18.pdf}{AntunesABD18}~\cite{AntunesABD18} & \href{../works/AntunesABD20.pdf}{AntunesABD20}~\cite{AntunesABD20}\\ 
\href{../works/AntuoriHHEN20.pdf}{AntuoriHHEN20}~\cite{AntuoriHHEN20} & \href{../works/AntuoriHHEN21.pdf}{AntuoriHHEN21}~\cite{AntuoriHHEN21} & \href{../works/Apt2001.pdf}{Apt2001}~\cite{Apt2001} & \href{../works/ArbaouiY18.pdf}{ArbaouiY18}~\cite{ArbaouiY18} & \href{../}{Arkhipov19}~\cite{Arkhipov19} & \href{../works/ArkhipovBL19.pdf}{ArkhipovBL19}~\cite{ArkhipovBL19}\\ 
\href{../works/ArmstrongGOS21.pdf}{ArmstrongGOS21}~\cite{ArmstrongGOS21} & \href{../works/ArmstrongGOS22.pdf}{ArmstrongGOS22}~\cite{ArmstrongGOS22} & \href{../works/AronssonBK09.pdf}{AronssonBK09}~\cite{AronssonBK09} & \href{../works/Artigues2011.pdf}{Artigues2011}~\cite{Artigues2011} & \href{../works/ArtiguesBF04.pdf}{ArtiguesBF04}~\cite{ArtiguesBF04} & \href{../}{ArtiguesDN08}~\cite{ArtiguesDN08}\\ 
\href{../works/ArtiguesF07.pdf}{ArtiguesF07}~\cite{ArtiguesF07} & \href{../works/ArtiguesHQT21.pdf}{ArtiguesHQT21}~\cite{ArtiguesHQT21} & \href{../works/ArtiguesL14.pdf}{ArtiguesL14}~\cite{ArtiguesL14} & \href{../works/ArtiguesLH13.pdf}{ArtiguesLH13}~\cite{ArtiguesLH13} & \href{../works/ArtiguesR00.pdf}{ArtiguesR00}~\cite{ArtiguesR00} & \href{../works/ArtiouchineB05.pdf}{ArtiouchineB05}~\cite{ArtiouchineB05}\\ 
\href{../works/Astrand0F21.pdf}{Astrand0F21}~\cite{Astrand0F21} & \href{../works/Astrand2020.pdf}{Astrand2020}~\cite{Astrand2020} & \href{../works/Astrand21.pdf}{Astrand21}~\cite{Astrand21} & \href{../works/AstrandJZ18.pdf}{AstrandJZ18}~\cite{AstrandJZ18} & \href{../works/AstrandJZ20.pdf}{AstrandJZ20}~\cite{AstrandJZ20} & \href{../works/Austrin2013.pdf}{Austrin2013}~\cite{Austrin2013}\\ 
\href{../works/AwadMDMT22.pdf}{AwadMDMT22}~\cite{AwadMDMT22} & \href{../works/BadicaBI20.pdf}{BadicaBI20}~\cite{BadicaBI20} & \href{../works/BadicaBIL19.pdf}{BadicaBIL19}~\cite{BadicaBIL19} & \href{../works/BajestaniB11.pdf}{BajestaniB11}~\cite{BajestaniB11} & \href{../works/BajestaniB13.pdf}{BajestaniB13}~\cite{BajestaniB13} & \href{../works/BajestaniB15.pdf}{BajestaniB15}~\cite{BajestaniB15}\\ 
\href{../works/Balduccini11.pdf}{Balduccini11}~\cite{Balduccini11} & \href{../works/Balduccini2017.pdf}{Balduccini2017}~\cite{Balduccini2017} & \href{../}{BalochG20}~\cite{BalochG20} & \href{../works/Banaszak2008.pdf}{Banaszak2008}~\cite{Banaszak2008} & \href{../works/Banaszak2014.pdf}{Banaszak2014}~\cite{Banaszak2014} & \href{../works/BandaSC11.pdf}{BandaSC11}~\cite{BandaSC11}\\ 
\href{../works/Baptiste02.pdf}{Baptiste02}~\cite{Baptiste02} & \href{../works/Baptiste09.pdf}{Baptiste09}~\cite{Baptiste09} & \href{../works/Baptiste1998.pdf}{Baptiste1998}~\cite{Baptiste1998} & \href{../}{Baptiste2001}~\cite{Baptiste2001} & \href{../works/BaptisteB18.pdf}{BaptisteB18}~\cite{BaptisteB18} & \href{../}{BaptisteLPN06}~\cite{BaptisteLPN06}\\ 
\href{../works/BaptisteLV92.pdf}{BaptisteLV92}~\cite{BaptisteLV92} & \href{../works/BaptisteP00.pdf}{BaptisteP00}~\cite{BaptisteP00} & \href{../works/BaptisteP95.pdf}{BaptisteP95}~\cite{BaptisteP95} & \href{../works/BaptisteP97.pdf}{BaptisteP97}~\cite{BaptisteP97} & \href{../}{BaptistePN01}~\cite{BaptistePN01} & \href{../works/BaptistePN99.pdf}{BaptistePN99}~\cite{BaptistePN99}\\ 
\href{../works/Barber1993.pdf}{Barber1993}~\cite{Barber1993} & \href{../works/BarbulescuWH04.pdf}{BarbulescuWH04}~\cite{BarbulescuWH04} & \href{../works/BarlattCG08.pdf}{BarlattCG08}~\cite{BarlattCG08} & \href{../works/Bartak02.pdf}{Bartak02}~\cite{Bartak02} & \href{../works/Bartak02a.pdf}{Bartak02a}~\cite{Bartak02a} & \href{../}{Bartak14}~\cite{Bartak14}\\ 
\href{../}{Bartak2005}~\cite{Bartak2005} & \href{../works/BartakCS10.pdf}{BartakCS10}~\cite{BartakCS10} & \href{../works/BartakS11.pdf}{BartakS11}~\cite{BartakS11} & \href{../works/BartakSR08.pdf}{BartakSR08}~\cite{BartakSR08} & \href{../works/BartakSR10.pdf}{BartakSR10}~\cite{BartakSR10} & \href{../works/BartakV15.pdf}{BartakV15}~\cite{BartakV15}\\ 
\href{../works/Bartk2010.pdf}{Bartk2010}~\cite{Bartk2010} & \href{../works/BartoliniBBLM14.pdf}{BartoliniBBLM14}~\cite{BartoliniBBLM14} & \href{../works/BarzegaranZP20.pdf}{BarzegaranZP20}~\cite{BarzegaranZP20} & \href{../works/Baykan1997.pdf}{Baykan1997}~\cite{Baykan1997} & \href{../works/Beck06.pdf}{Beck06}~\cite{Beck06} & \href{../works/Beck07.pdf}{Beck07}~\cite{Beck07}\\ 
\href{../works/Beck10.pdf}{Beck10}~\cite{Beck10} & \href{../works/Beck99.pdf}{Beck99}~\cite{Beck99} & \href{../works/BeckDDF98.pdf}{BeckDDF98}~\cite{BeckDDF98} & \href{../works/BeckDF97.pdf}{BeckDF97}~\cite{BeckDF97} & \href{../works/BeckDSF97.pdf}{BeckDSF97}~\cite{BeckDSF97} & \href{../works/BeckDSF97a.pdf}{BeckDSF97a}~\cite{BeckDSF97a}\\ 
\href{../works/BeckF00.pdf}{BeckF00}~\cite{BeckF00} & \href{../works/BeckF00a.pdf}{BeckF00a}~\cite{BeckF00a} & \href{../works/BeckF98.pdf}{BeckF98}~\cite{BeckF98} & \href{../works/BeckF99.pdf}{BeckF99}~\cite{BeckF99} & \href{../works/BeckFW11.pdf}{BeckFW11}~\cite{BeckFW11} & \href{../works/BeckPS03.pdf}{BeckPS03}~\cite{BeckPS03}\\ 
\href{../works/BeckR03.pdf}{BeckR03}~\cite{BeckR03} & \href{../works/BeckW04.pdf}{BeckW04}~\cite{BeckW04} & \href{../works/BeckW05.pdf}{BeckW05}~\cite{BeckW05} & \href{../works/BeckW07.pdf}{BeckW07}~\cite{BeckW07} & \href{../works/Bedhief21.pdf}{Bedhief21}~\cite{Bedhief21} & \href{../works/BegB13.pdf}{BegB13}~\cite{BegB13}\\ 
\href{../works/BehrensLM19.pdf}{BehrensLM19}~\cite{BehrensLM19} & \href{../works/BeldiceanuC01.pdf}{BeldiceanuC01}~\cite{BeldiceanuC01} & \href{../works/BeldiceanuC02.pdf}{BeldiceanuC02}~\cite{BeldiceanuC02} & \href{../works/BeldiceanuC94.pdf}{BeldiceanuC94}~\cite{BeldiceanuC94} & \href{../works/BeldiceanuCDP11.pdf}{BeldiceanuCDP11}~\cite{BeldiceanuCDP11} & \href{../works/BeldiceanuCP08.pdf}{BeldiceanuCP08}~\cite{BeldiceanuCP08}\\ 
\href{../works/BeldiceanuP07.pdf}{BeldiceanuP07}~\cite{BeldiceanuP07} & \href{../works/BelhadjiI98.pdf}{BelhadjiI98}~\cite{BelhadjiI98} & \href{../works/Benda2019.pdf}{Benda2019}~\cite{Benda2019} & \href{../works/BenderWS21.pdf}{BenderWS21}~\cite{BenderWS21} & \href{../works/Benedetti2008.pdf}{Benedetti2008}~\cite{Benedetti2008} & \href{../works/BenediktMH20.pdf}{BenediktMH20}~\cite{BenediktMH20}\\ 
\href{../works/BenediktSMVH18.pdf}{BenediktSMVH18}~\cite{BenediktSMVH18} & \href{../works/BeniniBGM05.pdf}{BeniniBGM05}~\cite{BeniniBGM05} & \href{../works/BeniniBGM05a.pdf}{BeniniBGM05a}~\cite{BeniniBGM05a} & \href{../works/BeniniBGM06.pdf}{BeniniBGM06}~\cite{BeniniBGM06} & \href{../works/BeniniLMMR08.pdf}{BeniniLMMR08}~\cite{BeniniLMMR08} & \href{../works/BeniniLMR08.pdf}{BeniniLMR08}~\cite{BeniniLMR08}\\ 
\href{../works/BeniniLMR11.pdf}{BeniniLMR11}~\cite{BeniniLMR11} & \href{../works/BenoistGR02.pdf}{BenoistGR02}~\cite{BenoistGR02} & \href{../works/BensanaLV99.pdf}{BensanaLV99}~\cite{BensanaLV99} & \href{../}{Berbeglia2012}~\cite{Berbeglia2012} & \href{../works/Bergman2014.pdf}{Bergman2014}~\cite{Bergman2014} & \href{../works/BertholdHLMS10.pdf}{BertholdHLMS10}~\cite{BertholdHLMS10}\\ 
\href{../works/BessiereHMQW14.pdf}{BessiereHMQW14}~\cite{BessiereHMQW14} & \href{../}{Bgler2016}~\cite{Bgler2016} & \href{../}{Bgler2016a}~\cite{Bgler2016a} & \href{../works/BhatnagarKL19.pdf}{BhatnagarKL19}~\cite{BhatnagarKL19} & \href{../works/Bidot2006.pdf}{Bidot2006}~\cite{Bidot2006} & \href{../works/BidotVLB07.pdf}{BidotVLB07}~\cite{BidotVLB07}\\ 
\href{../works/BidotVLB09.pdf}{BidotVLB09}~\cite{BidotVLB09} & \href{../works/BillautHL12.pdf}{BillautHL12}~\cite{BillautHL12} & \href{../works/Biswas2010.pdf}{Biswas2010}~\cite{Biswas2010} & \href{../works/Bit-Monnot23.pdf}{Bit-Monnot23}~\cite{Bit-Monnot23} & \href{../works/Bittle2009.pdf}{Bittle2009}~\cite{Bittle2009} & \href{../works/Bixby2006.pdf}{Bixby2006}~\cite{Bixby2006}\\ 
\href{../works/BlazewiczDP96.pdf}{BlazewiczDP96}~\cite{BlazewiczDP96} & \href{../}{BlazewiczEP19}~\cite{BlazewiczEP19} & \href{../works/Bley2023.pdf}{Bley2023}~\cite{Bley2023} & \href{../works/BlomBPS14.pdf}{BlomBPS14}~\cite{BlomBPS14} & \href{../works/BlomPS16.pdf}{BlomPS16}~\cite{BlomPS16} & \href{../works/Bocewicz2009.pdf}{Bocewicz2009}~\cite{Bocewicz2009}\\ 
\href{../works/Bocewicz2013.pdf}{Bocewicz2013}~\cite{Bocewicz2013} & \href{../works/Bocewicz2021.pdf}{Bocewicz2021}~\cite{Bocewicz2021} & \href{../works/Bocewicz2023.pdf}{Bocewicz2023}~\cite{Bocewicz2023} & \href{../works/BocewiczBB09.pdf}{BocewiczBB09}~\cite{BocewiczBB09} & \href{../}{BockmayrK98}~\cite{BockmayrK98} & \href{../works/BockmayrP06.pdf}{BockmayrP06}~\cite{BockmayrP06}\\ 
\href{../works/Boek2016.pdf}{Boek2016}~\cite{Boek2016} & \href{../works/BofillCGGPSV23.pdf}{BofillCGGPSV23}~\cite{BofillCGGPSV23} & \href{../works/BofillCSV17.pdf}{BofillCSV17}~\cite{BofillCSV17} & \href{../works/BofillCSV17a.pdf}{BofillCSV17a}~\cite{BofillCSV17a} & \href{../works/BofillEGPSV14.pdf}{BofillEGPSV14}~\cite{BofillEGPSV14} & \href{../works/BofillGSV15.pdf}{BofillGSV15}~\cite{BofillGSV15}\\ 
\href{../works/BogaerdtW19.pdf}{BogaerdtW19}~\cite{BogaerdtW19} & \href{../works/Bonfietti16.pdf}{Bonfietti16}~\cite{Bonfietti16} & \href{../works/BonfiettiLBM11.pdf}{BonfiettiLBM11}~\cite{BonfiettiLBM11} & \href{../works/BonfiettiLBM12.pdf}{BonfiettiLBM12}~\cite{BonfiettiLBM12} & \href{../works/BonfiettiLBM14.pdf}{BonfiettiLBM14}~\cite{BonfiettiLBM14} & \href{../works/BonfiettiLM13.pdf}{BonfiettiLM13}~\cite{BonfiettiLM13}\\ 
\href{../works/BonfiettiLM14.pdf}{BonfiettiLM14}~\cite{BonfiettiLM14} & \href{../works/BonfiettiM12.pdf}{BonfiettiM12}~\cite{BonfiettiM12} & \href{../works/BonfiettiZLM16.pdf}{BonfiettiZLM16}~\cite{BonfiettiZLM16} & \href{../works/BonninMNE24.pdf}{BonninMNE24}~\cite{BonninMNE24} & \href{../works/BoothNB16.pdf}{BoothNB16}~\cite{BoothNB16} & \href{../works/BoothTNB16.pdf}{BoothTNB16}~\cite{BoothTNB16}\\ 
\href{../works/BorghesiBLMB18.pdf}{BorghesiBLMB18}~\cite{BorghesiBLMB18} & \href{../works/BosiM2001.pdf}{BosiM2001}~\cite{BosiM2001} & \href{../}{BoucherBVBL97}~\cite{BoucherBVBL97} & \href{../works/BoudreaultSLQ22.pdf}{BoudreaultSLQ22}~\cite{BoudreaultSLQ22} & \href{../works/BourdaisGP03.pdf}{BourdaisGP03}~\cite{BourdaisGP03} & \href{../}{Bourdeaudhuy2011}~\cite{Bourdeaudhuy2011}\\ 
\href{../works/BourreauGGLT22.pdf}{BourreauGGLT22}~\cite{BourreauGGLT22} & \href{../works/Braune2022.pdf}{Braune2022}~\cite{Braune2022} & \href{../}{Breitinger1994}~\cite{Breitinger1994} & \href{../}{BreitingerL95}~\cite{BreitingerL95} & \href{../}{BriandHHL08}~\cite{BriandHHL08} & \href{../works/BridiBLMB16.pdf}{BridiBLMB16}~\cite{BridiBLMB16}\\ 
\href{../works/BridiLBBM16.pdf}{BridiLBBM16}~\cite{BridiLBBM16} & \href{../works/Brucker2002.pdf}{Brucker2002}~\cite{Brucker2002} & \href{../works/BruckerK00.pdf}{BruckerK00}~\cite{BruckerK00} & \href{../works/BrusoniCLMMT96.pdf}{BrusoniCLMMT96}~\cite{BrusoniCLMMT96} & \href{../works/BukchinR18.pdf}{BukchinR18}~\cite{BukchinR18} & \href{../works/BulckG22.pdf}{BulckG22}~\cite{BulckG22}\\ 
\href{../works/BurtLPS15.pdf}{BurtLPS15}~\cite{BurtLPS15} & \href{../}{Bzdyra2015}~\cite{Bzdyra2015} & \href{../works/Caballero19.pdf}{Caballero19}~\cite{Caballero19} & \href{../works/Caballero23.pdf}{Caballero23}~\cite{Caballero23} & \href{../works/CambazardHDJT04.pdf}{CambazardHDJT04}~\cite{CambazardHDJT04} & \href{../works/CambazardJ05.pdf}{CambazardJ05}~\cite{CambazardJ05}\\ 
\href{../works/CampeauG22.pdf}{CampeauG22}~\cite{CampeauG22} & \href{../works/Capone2009.pdf}{Capone2009}~\cite{Capone2009} & \href{../works/CappartS17.pdf}{CappartS17}~\cite{CappartS17} & \href{../works/CappartTSR18.pdf}{CappartTSR18}~\cite{CappartTSR18} & \href{../works/CarchraeB09.pdf}{CarchraeB09}~\cite{CarchraeB09} & \href{../works/CarchraeBF05.pdf}{CarchraeBF05}~\cite{CarchraeBF05}\\ 
\href{../works/Caricato2020.pdf}{Caricato2020}~\cite{Caricato2020} & \href{../works/CarlierPSJ20.pdf}{CarlierPSJ20}~\cite{CarlierPSJ20} & \href{../}{CarlierSJP21}~\cite{CarlierSJP21} & \href{../works/CarlssonJL17.pdf}{CarlssonJL17}~\cite{CarlssonJL17} & \href{../works/CarlssonKA99.pdf}{CarlssonKA99}~\cite{CarlssonKA99} & \href{../}{Caseau1996}~\cite{Caseau1996}\\ 
\href{../works/Caseau2001.pdf}{Caseau2001}~\cite{Caseau2001} & \href{../works/Caseau97.pdf}{Caseau97}~\cite{Caseau97} & \href{../}{CastroGR10}~\cite{CastroGR10} & \href{../works/CatusseCBL16.pdf}{CatusseCBL16}~\cite{CatusseCBL16} & \href{../works/CauwelaertDMS16.pdf}{CauwelaertDMS16}~\cite{CauwelaertDMS16} & \href{../works/CauwelaertDS20.pdf}{CauwelaertDS20}~\cite{CauwelaertDS20}\\ 
\href{../works/CauwelaertLS15.pdf}{CauwelaertLS15}~\cite{CauwelaertLS15} & \href{../works/CauwelaertLS18.pdf}{CauwelaertLS18}~\cite{CauwelaertLS18} & \href{../works/CestaOF99.pdf}{CestaOF99}~\cite{CestaOF99} & \href{../}{CestaOPS14}~\cite{CestaOPS14} & \href{../works/CestaOS00.pdf}{CestaOS00}~\cite{CestaOS00} & \href{../works/CestaOS98.pdf}{CestaOS98}~\cite{CestaOS98}\\ 
\href{../works/Chakrabortty2019.pdf}{Chakrabortty2019}~\cite{Chakrabortty2019} & \href{../works/Chaleshtarti2014.pdf}{Chaleshtarti2014}~\cite{Chaleshtarti2014} & \href{../works/Chan2001.pdf}{Chan2001}~\cite{Chan2001} & \href{../}{Chan2002}~\cite{Chan2002} & \href{../works/ChapadosJR11.pdf}{ChapadosJR11}~\cite{ChapadosJR11} & \href{../works/Chen2021.pdf}{Chen2021}~\cite{Chen2021}\\ 
\href{../works/ChenGPSH10.pdf}{ChenGPSH10}~\cite{ChenGPSH10} & \href{../works/Choi2007.pdf}{Choi2007}~\cite{Choi2007} & \href{../works/Choosri2011.pdf}{Choosri2011}~\cite{Choosri2011} & \href{../works/ChuGNSW13.pdf}{ChuGNSW13}~\cite{ChuGNSW13} & \href{../works/ChuX05.pdf}{ChuX05}~\cite{ChuX05} & \href{../works/Chun2011.pdf}{Chun2011}~\cite{Chun2011}\\ 
\href{../works/ChunCTY99.pdf}{ChunCTY99}~\cite{ChunCTY99} & \href{../works/ChunS14.pdf}{ChunS14}~\cite{ChunS14} & \href{../works/CilKLO22.pdf}{CilKLO22}~\cite{CilKLO22} & \href{../works/CireCH13.pdf}{CireCH13}~\cite{CireCH13} & \href{../works/CireCH16.pdf}{CireCH16}~\cite{CireCH16} & \href{../}{Clautiaux2013}~\cite{Clautiaux2013}\\ 
\href{../works/ClautiauxJCM08.pdf}{ClautiauxJCM08}~\cite{ClautiauxJCM08} & \href{../}{Clearwater1991}~\cite{Clearwater1991} & \href{../works/Clercq12.pdf}{Clercq12}~\cite{Clercq12} & \href{../works/ClercqPBJ11.pdf}{ClercqPBJ11}~\cite{ClercqPBJ11} & \href{../works/CobanH10.pdf}{CobanH10}~\cite{CobanH10} & \href{../works/CobanH11.pdf}{CobanH11}~\cite{CobanH11}\\ 
\href{../works/Coelho2011.pdf}{Coelho2011}~\cite{Coelho2011} & \href{../works/CohenHB17.pdf}{CohenHB17}~\cite{CohenHB17} & \href{../works/ColT19.pdf}{ColT19}~\cite{ColT19} & \href{../works/ColT2019a.pdf}{ColT2019a}~\cite{ColT2019a} & \href{../works/ColT22.pdf}{ColT22}~\cite{ColT22} & \href{../works/Colombani96.pdf}{Colombani96}~\cite{Colombani96}\\ 
\href{../works/CorreaLR07.pdf}{CorreaLR07}~\cite{CorreaLR07} & \href{../works/Cox2019.pdf}{Cox2019}~\cite{Cox2019} & \href{../works/CrawfordB94.pdf}{CrawfordB94}~\cite{CrawfordB94} & \href{../works/Cucu-Grosjean2009.pdf}{Cucu-Grosjean2009}~\cite{Cucu-Grosjean2009} & \href{../works/CzerniachowskaWZ23.pdf}{CzerniachowskaWZ23}~\cite{CzerniachowskaWZ23} & \href{../works/Daneshamooz2021.pdf}{Daneshamooz2021}~\cite{Daneshamooz2021}\\ 
\href{../works/DannaP03.pdf}{DannaP03}~\cite{DannaP03} & \href{../}{DannaP04}~\cite{DannaP04} & \href{../works/Danzinger2020.pdf}{Danzinger2020}~\cite{Danzinger2020} & \href{../works/Danzinger2023.pdf}{Danzinger2023}~\cite{Danzinger2023} & \href{../works/Darby-DowmanLMZ97.pdf}{Darby-DowmanLMZ97}~\cite{Darby-DowmanLMZ97} & \href{../}{DarbyDowmanL98}~\cite{DarbyDowmanL98}\\ 
\href{../}{Dasygenis2018}~\cite{Dasygenis2018} & \href{../works/Davenport10.pdf}{Davenport10}~\cite{Davenport10} & \href{../works/DavenportKRSH07.pdf}{DavenportKRSH07}~\cite{DavenportKRSH07} & \href{../works/Davis87.pdf}{Davis87}~\cite{Davis87} & \href{../works/Deblaere2011.pdf}{Deblaere2011}~\cite{Deblaere2011} & \href{../works/Dejemeppe16.pdf}{Dejemeppe16}~\cite{Dejemeppe16}\\ 
\href{../works/DejemeppeCS15.pdf}{DejemeppeCS15}~\cite{DejemeppeCS15} & \href{../works/DejemeppeD14.pdf}{DejemeppeD14}~\cite{DejemeppeD14} & \href{../works/Demassey03.pdf}{Demassey03}~\cite{Demassey03} & \href{../works/DemasseyAM05.pdf}{DemasseyAM05}~\cite{DemasseyAM05} & \href{../}{Demeulemeester1992}~\cite{Demeulemeester1992} & \href{../}{Demeulemeester1997}~\cite{Demeulemeester1997}\\ 
\href{../works/DemirovicS18.pdf}{DemirovicS18}~\cite{DemirovicS18} & \href{../works/Derrien15.pdf}{Derrien15}~\cite{Derrien15} & \href{../works/DerrienP14.pdf}{DerrienP14}~\cite{DerrienP14} & \href{../works/DerrienPZ14.pdf}{DerrienPZ14}~\cite{DerrienPZ14} & \href{../works/DilkinaDH05.pdf}{DilkinaDH05}~\cite{DilkinaDH05} & \href{../works/DilkinaH04.pdf}{DilkinaH04}~\cite{DilkinaH04}\\ 
\href{../works/Dimny2023.pdf}{Dimny2023}~\cite{Dimny2023} & \href{../works/DincbasS91.pdf}{DincbasS91}~\cite{DincbasS91} & \href{../works/DincbasSH90.pdf}{DincbasSH90}~\cite{DincbasSH90} & \href{../works/DoRZ08.pdf}{DoRZ08}~\cite{DoRZ08} & \href{../works/Dolabi2014.pdf}{Dolabi2014}~\cite{Dolabi2014} & \href{../}{DomdorfPH03}~\cite{DomdorfPH03}\\ 
\href{../works/Dong2010.pdf}{Dong2010}~\cite{Dong2010} & \href{../works/Doolaard2022.pdf}{Doolaard2022}~\cite{Doolaard2022} & \href{../works/DoomsH08.pdf}{DoomsH08}~\cite{DoomsH08} & \href{../works/Dorndorf2000.pdf}{Dorndorf2000}~\cite{Dorndorf2000} & \href{../}{Dorndorf2000a}~\cite{Dorndorf2000a} & \href{../}{DorndorfHP99}~\cite{DorndorfHP99}\\ 
\href{../}{DorndorfPH99}~\cite{DorndorfPH99} & \href{../works/DoulabiRP14.pdf}{DoulabiRP14}~\cite{DoulabiRP14} & \href{../works/DoulabiRP16.pdf}{DoulabiRP16}~\cite{DoulabiRP16} & \href{../works/DraperJCJ99.pdf}{DraperJCJ99}~\cite{DraperJCJ99} & \href{../works/EastonNT02.pdf}{EastonNT02}~\cite{EastonNT02} & \href{../works/Edis21.pdf}{Edis21}~\cite{Edis21}\\ 
\href{../works/EdisO11.pdf}{EdisO11}~\cite{EdisO11} & \href{../}{EdisO11a}~\cite{EdisO11a} & \href{../}{EdwardsBSE19}~\cite{EdwardsBSE19} & \href{../works/EfthymiouY23.pdf}{EfthymiouY23}~\cite{EfthymiouY23} & \href{../}{Eirinakis2012}~\cite{Eirinakis2012} & \href{../works/Eiter2021.pdf}{Eiter2021}~\cite{Eiter2021}\\ 
\href{../works/Eiter2023.pdf}{Eiter2023}~\cite{Eiter2023} & \href{../works/El-Kholany2022.pdf}{El-Kholany2022}~\cite{El-Kholany2022} & \href{../works/ElciOH22.pdf}{ElciOH22}~\cite{ElciOH22} & \href{../works/ElfJR03.pdf}{ElfJR03}~\cite{ElfJR03} & \href{../works/ElhouraniDM07.pdf}{ElhouraniDM07}~\cite{ElhouraniDM07} & \href{../works/Elkhyari03.pdf}{Elkhyari03}~\cite{Elkhyari03}\\ 
\href{../works/Elkhyari2006.pdf}{Elkhyari2006}~\cite{Elkhyari2006} & \href{../works/Elkhyari2017.pdf}{Elkhyari2017}~\cite{Elkhyari2017} & \href{../works/ElkhyariGJ02.pdf}{ElkhyariGJ02}~\cite{ElkhyariGJ02} & \href{../works/ElkhyariGJ02a.pdf}{ElkhyariGJ02a}~\cite{ElkhyariGJ02a} & \href{../}{Elmaghraby1992}~\cite{Elmaghraby1992} & \href{../works/EmdeZD22.pdf}{EmdeZD22}~\cite{EmdeZD22}\\ 
\href{../works/Emeretlis2014.pdf}{Emeretlis2014}~\cite{Emeretlis2014} & \href{../works/EmeretlisTAV17.pdf}{EmeretlisTAV17}~\cite{EmeretlisTAV17} & \href{../works/EreminW01.pdf}{EreminW01}~\cite{EreminW01} & \href{../works/ErkingerM17.pdf}{ErkingerM17}~\cite{ErkingerM17} & \href{../works/ErtlK91.pdf}{ErtlK91}~\cite{ErtlK91} & \href{../works/EscobetPQPRA19.pdf}{EscobetPQPRA19}~\cite{EscobetPQPRA19}\\ 
\href{../works/EskeyZ90.pdf}{EskeyZ90}~\cite{EskeyZ90} & \href{../}{EsquirolLH2008}~\cite{EsquirolLH2008} & \href{../works/EtminaniesfahaniGNMS22.pdf}{EtminaniesfahaniGNMS22}~\cite{EtminaniesfahaniGNMS22} & \href{../works/EvenSH15.pdf}{EvenSH15}~\cite{EvenSH15} & \href{../works/EvenSH15a.pdf}{EvenSH15a}~\cite{EvenSH15a} & \href{../works/FachiniA20.pdf}{FachiniA20}~\cite{FachiniA20}\\ 
\href{../works/Fahimi16.pdf}{Fahimi16}~\cite{Fahimi16} & \href{../works/FahimiOQ18.pdf}{FahimiOQ18}~\cite{FahimiOQ18} & \href{../}{FahimiQ23}~\cite{FahimiQ23} & \href{../works/FalaschiGMP97.pdf}{FalaschiGMP97}~\cite{FalaschiGMP97} & \href{../works/FallahiAC20.pdf}{FallahiAC20}~\cite{FallahiAC20} & \href{../works/FalqueALM24.pdf}{FalqueALM24}~\cite{FalqueALM24}\\ 
\href{../works/FanXG21.pdf}{FanXG21}~\cite{FanXG21} & \href{../works/Farias2001.pdf}{Farias2001}~\cite{Farias2001} & \href{../works/FarsiTM22.pdf}{FarsiTM22}~\cite{FarsiTM22} & \href{../works/Fatemi-AnarakiTFV23.pdf}{Fatemi-AnarakiTFV23}~\cite{Fatemi-AnarakiTFV23} & \href{../works/FeldmanG89.pdf}{FeldmanG89}~\cite{FeldmanG89} & \href{../}{FelizariAL09}~\cite{FelizariAL09}\\ 
\href{../works/Feng2022.pdf}{Feng2022}~\cite{Feng2022} & \href{../works/FetgoD22.pdf}{FetgoD22}~\cite{FetgoD22} & \href{../works/Filho2012.pdf}{Filho2012}~\cite{Filho2012} & \href{../}{Fisher1985}~\cite{Fisher1985} & \href{../works/FocacciLN00.pdf}{FocacciLN00}~\cite{FocacciLN00} & \href{../works/FontaineMH16.pdf}{FontaineMH16}~\cite{FontaineMH16}\\ 
\href{../works/ForbesHJST24.pdf}{ForbesHJST24}~\cite{ForbesHJST24} & \href{../works/FortinZDF05.pdf}{FortinZDF05}~\cite{FortinZDF05} & \href{../works/FoxAS82.pdf}{FoxAS82}~\cite{FoxAS82} & \href{../works/FoxS90.pdf}{FoxS90}~\cite{FoxS90} & \href{../works/FrankDT16.pdf}{FrankDT16}~\cite{FrankDT16} & \href{../works/FrankK03.pdf}{FrankK03}~\cite{FrankK03}\\ 
\href{../works/FrankK05.pdf}{FrankK05}~\cite{FrankK05} & \href{../}{Freuder1994}~\cite{Freuder1994} & \href{../}{FriedrichFMRSST14}~\cite{FriedrichFMRSST14} & \href{../works/FrimodigECM23.pdf}{FrimodigECM23}~\cite{FrimodigECM23} & \href{../works/FrimodigS19.pdf}{FrimodigS19}~\cite{FrimodigS19} & \href{../works/Frisch2006.pdf}{Frisch2006}~\cite{Frisch2006}\\ 
\href{../works/Froger16.pdf}{Froger16}~\cite{Froger16} & \href{../works/FrohnerTR19.pdf}{FrohnerTR19}~\cite{FrohnerTR19} & \href{../works/FrostD98.pdf}{FrostD98}~\cite{FrostD98} & \href{../works/FukunagaHFAMN02.pdf}{FukunagaHFAMN02}~\cite{FukunagaHFAMN02} & \href{../works/Galipienso2001.pdf}{Galipienso2001}~\cite{Galipienso2001} & \href{../works/GalleguillosKSB19.pdf}{GalleguillosKSB19}~\cite{GalleguillosKSB19}\\ 
\href{../works/Gao2018.pdf}{Gao2018}~\cite{Gao2018} & \href{../works/Gao2022.pdf}{Gao2022}~\cite{Gao2022} & \href{../works/GarcaNieves2018.pdf}{GarcaNieves2018}~\cite{GarcaNieves2018} & \href{../works/GarganiR07.pdf}{GarganiR07}~\cite{GarganiR07} & \href{../works/GarridoAO09.pdf}{GarridoAO09}~\cite{GarridoAO09} & \href{../works/GarridoOS08.pdf}{GarridoOS08}~\cite{GarridoOS08}\\ 
\href{../works/Gaspero2014.pdf}{Gaspero2014}~\cite{Gaspero2014} & \href{../works/GayHLS15.pdf}{GayHLS15}~\cite{GayHLS15} & \href{../works/GayHS15.pdf}{GayHS15}~\cite{GayHS15} & \href{../works/GayHS15a.pdf}{GayHS15a}~\cite{GayHS15a} & \href{../works/GaySS14.pdf}{GaySS14}~\cite{GaySS14} & \href{../works/GedikKBR17.pdf}{GedikKBR17}~\cite{GedikKBR17}\\ 
\href{../works/GedikKEK18.pdf}{GedikKEK18}~\cite{GedikKEK18} & \href{../works/GeibingerKKMMW21.pdf}{GeibingerKKMMW21}~\cite{GeibingerKKMMW21} & \href{../works/GeibingerMM19.pdf}{GeibingerMM19}~\cite{GeibingerMM19} & \href{../works/GeibingerMM21.pdf}{GeibingerMM21}~\cite{GeibingerMM21} & \href{../works/Geiger2019.pdf}{Geiger2019}~\cite{Geiger2019} & \href{../works/GeitzGSSW22.pdf}{GeitzGSSW22}~\cite{GeitzGSSW22}\\ 
\href{../works/GelainPRVW17.pdf}{GelainPRVW17}~\cite{GelainPRVW17} & \href{../works/Gembarski2022.pdf}{Gembarski2022}~\cite{Gembarski2022} & \href{../works/Gent1996.pdf}{Gent1996}~\cite{Gent1996} & \href{../works/German18.pdf}{German18}~\cite{German18} & \href{../works/Geske05.pdf}{Geske05}~\cite{Geske05} & \href{../works/GetoorOFC97.pdf}{GetoorOFC97}~\cite{GetoorOFC97}\\ 
\href{../works/GhandehariK22.pdf}{GhandehariK22}~\cite{GhandehariK22} & \href{../}{GhasemiMH23}~\cite{GhasemiMH23} & \href{../works/GilesH16.pdf}{GilesH16}~\cite{GilesH16} & \href{../works/GingrasQ16.pdf}{GingrasQ16}~\cite{GingrasQ16} & \href{../works/GlobusCLP04.pdf}{GlobusCLP04}~\cite{GlobusCLP04} & \href{../works/GodardLN05.pdf}{GodardLN05}~\cite{GodardLN05}\\ 
\href{../works/Godet21a.pdf}{Godet21a}~\cite{Godet21a} & \href{../works/GodetLHS20.pdf}{GodetLHS20}~\cite{GodetLHS20} & \href{../works/GoelSHFS15.pdf}{GoelSHFS15}~\cite{GoelSHFS15} & \href{../works/GokGSTO20.pdf}{GokGSTO20}~\cite{GokGSTO20} & \href{../works/GokPTGO23.pdf}{GokPTGO23}~\cite{GokPTGO23} & \href{../works/Gokgur2022.pdf}{Gokgur2022}~\cite{Gokgur2022}\\ 
\href{../works/GokgurHO18.pdf}{GokgurHO18}~\cite{GokgurHO18} & \href{../works/GoldwaserS17.pdf}{GoldwaserS17}~\cite{GoldwaserS17} & \href{../works/GoldwaserS18.pdf}{GoldwaserS18}~\cite{GoldwaserS18} & \href{../works/Goltz95.pdf}{Goltz95}~\cite{Goltz95} & \href{../works/GombolayWS18.pdf}{GombolayWS18}~\cite{GombolayWS18} & \href{../works/GomesHS06.pdf}{GomesHS06}~\cite{GomesHS06}\\ 
\href{../works/GomesM17.pdf}{GomesM17}~\cite{GomesM17} & \href{../}{GongLMW09}~\cite{GongLMW09} & \href{../works/Gonzlez2017.pdf}{Gonzlez2017}~\cite{Gonzlez2017} & \href{../works/GrimesH10.pdf}{GrimesH10}~\cite{GrimesH10} & \href{../works/GrimesH11.pdf}{GrimesH11}~\cite{GrimesH11} & \href{../works/GrimesH15.pdf}{GrimesH15}~\cite{GrimesH15}\\ 
\href{../works/GrimesHM09.pdf}{GrimesHM09}~\cite{GrimesHM09} & \href{../works/GrimesIOS14.pdf}{GrimesIOS14}~\cite{GrimesIOS14} & \href{../works/Groleaz21.pdf}{Groleaz21}~\cite{Groleaz21} & \href{../works/GroleazNS20.pdf}{GroleazNS20}~\cite{GroleazNS20} & \href{../works/GroleazNS20a.pdf}{GroleazNS20a}~\cite{GroleazNS20a} & \href{../works/Gronkvist06.pdf}{Gronkvist06}~\cite{Gronkvist06}\\ 
\href{../works/GruianK98.pdf}{GruianK98}~\cite{GruianK98} & \href{../works/Grzegorz2021.pdf}{Grzegorz2021}~\cite{Grzegorz2021} & \href{../works/GuSS13.pdf}{GuSS13}~\cite{GuSS13} & \href{../}{GuSSWC14}~\cite{GuSSWC14} & \href{../works/GuSW12.pdf}{GuSW12}~\cite{GuSW12} & \href{../works/Guerinik1995.pdf}{Guerinik1995}~\cite{Guerinik1995}\\ 
\href{../}{Guimarans2013}~\cite{Guimarans2013} & \href{../}{GunerGSKD23}~\cite{GunerGSKD23} & \href{../}{GuoHLW20}~\cite{GuoHLW20} & \href{../works/GuoZ23.pdf}{GuoZ23}~\cite{GuoZ23} & \href{../works/GurEA19.pdf}{GurEA19}~\cite{GurEA19} & \href{../works/GurPAE23.pdf}{GurPAE23}~\cite{GurPAE23}\\ 
\href{../works/GuyonLPR12.pdf}{GuyonLPR12}~\cite{GuyonLPR12} & \href{../works/Hachemi2008.pdf}{Hachemi2008}~\cite{Hachemi2008} & \href{../works/Hachemi2009.pdf}{Hachemi2009}~\cite{Hachemi2009} & \href{../works/HachemiGR11.pdf}{HachemiGR11}~\cite{HachemiGR11} & \href{../works/Hajji2023.pdf}{Hajji2023}~\cite{Hajji2023} & \href{../works/Ham18.pdf}{Ham18}~\cite{Ham18}\\ 
\href{../works/Ham18a.pdf}{Ham18a}~\cite{Ham18a} & \href{../}{Ham20}~\cite{Ham20} & \href{../works/Ham20a.pdf}{Ham20a}~\cite{Ham20a} & \href{../works/HamC16.pdf}{HamC16}~\cite{HamC16} & \href{../works/HamFC17.pdf}{HamFC17}~\cite{HamFC17} & \href{../works/HamP21.pdf}{HamP21}~\cite{HamP21}\\ 
\href{../works/HamPK21.pdf}{HamPK21}~\cite{HamPK21} & \href{../works/HamdiL13.pdf}{HamdiL13}~\cite{HamdiL13} & \href{../works/Hamscher91.pdf}{Hamscher91}~\cite{Hamscher91} & \href{../works/Han2014.pdf}{Han2014}~\cite{Han2014} & \href{../works/HanenKP21.pdf}{HanenKP21}~\cite{HanenKP21} & \href{../works/Hannebauer2001.pdf}{Hannebauer2001}~\cite{Hannebauer2001}\\ 
\href{../}{Harjunkoski2001}~\cite{Harjunkoski2001} & \href{../works/HarjunkoskiG02.pdf}{HarjunkoskiG02}~\cite{HarjunkoskiG02} & \href{../works/HarjunkoskiJG00.pdf}{HarjunkoskiJG00}~\cite{HarjunkoskiJG00} & \href{../works/HarjunkoskiMBC14.pdf}{HarjunkoskiMBC14}~\cite{HarjunkoskiMBC14} & \href{../}{Hat2011}~\cite{Hat2011} & \href{../works/HauderBRPA20.pdf}{HauderBRPA20}~\cite{HauderBRPA20}\\ 
\href{../works/He0GLW18.pdf}{He0GLW18}~\cite{He0GLW18} & \href{../works/He2019.pdf}{He2019}~\cite{He2019} & \href{../works/HebrardALLCMR22.pdf}{HebrardALLCMR22}~\cite{HebrardALLCMR22} & \href{../works/HebrardHJMPV16.pdf}{HebrardHJMPV16}~\cite{HebrardHJMPV16} & \href{../works/HebrardTW05.pdf}{HebrardTW05}~\cite{HebrardTW05} & \href{../works/HechingH16.pdf}{HechingH16}~\cite{HechingH16}\\ 
\href{../}{HechingHK19}~\cite{HechingHK19} & \href{../works/HeckmanB11.pdf}{HeckmanB11}~\cite{HeckmanB11} & \href{../works/HeinzB12.pdf}{HeinzB12}~\cite{HeinzB12} & \href{../works/HeinzKB13.pdf}{HeinzKB13}~\cite{HeinzKB13} & \href{../works/HeinzNVH22.pdf}{HeinzNVH22}~\cite{HeinzNVH22} & \href{../works/HeinzS11.pdf}{HeinzS11}~\cite{HeinzS11}\\ 
\href{../works/HeinzSB13.pdf}{HeinzSB13}~\cite{HeinzSB13} & \href{../works/HeinzSSW12.pdf}{HeinzSSW12}~\cite{HeinzSSW12} & \href{../works/HeipckeCCS00.pdf}{HeipckeCCS00}~\cite{HeipckeCCS00} & \href{../works/Hentenryck2000.pdf}{Hentenryck2000}~\cite{Hentenryck2000} & \href{../works/HentenryckM04.pdf}{HentenryckM04}~\cite{HentenryckM04} & \href{../works/HentenryckM08.pdf}{HentenryckM08}~\cite{HentenryckM08}\\ 
\href{../}{Henz01}~\cite{Henz01} & \href{../works/HenzMT04.pdf}{HenzMT04}~\cite{HenzMT04} & \href{../works/HermenierDL11.pdf}{HermenierDL11}~\cite{HermenierDL11} & \href{../}{HillBCGN22}~\cite{HillBCGN22} & \href{../works/HillTV21.pdf}{HillTV21}~\cite{HillTV21} & \href{../works/Hindi2004.pdf}{Hindi2004}~\cite{Hindi2004}\\ 
\href{../works/HladikCDJ08.pdf}{HladikCDJ08}~\cite{HladikCDJ08} & \href{../works/HoYCLLCLC18.pdf}{HoYCLLCLC18}~\cite{HoYCLLCLC18} & \href{../works/Hoc2012.pdf}{Hoc2012}~\cite{Hoc2012} & \href{../works/HoeveGSL07.pdf}{HoeveGSL07}~\cite{HoeveGSL07} & \href{../}{Hofe2001}~\cite{Hofe2001} & \href{../}{Hooker00}~\cite{Hooker00}\\ 
\href{../}{Hooker02}~\cite{Hooker02} & \href{../works/Hooker04.pdf}{Hooker04}~\cite{Hooker04} & \href{../works/Hooker05.pdf}{Hooker05}~\cite{Hooker05} & \href{../works/Hooker05a.pdf}{Hooker05a}~\cite{Hooker05a} & \href{../works/Hooker05b.pdf}{Hooker05b}~\cite{Hooker05b} & \href{../works/Hooker06.pdf}{Hooker06}~\cite{Hooker06}\\ 
\href{../}{Hooker06a}~\cite{Hooker06a} & \href{../works/Hooker07.pdf}{Hooker07}~\cite{Hooker07} & \href{../}{Hooker10}~\cite{Hooker10} & \href{../works/Hooker17.pdf}{Hooker17}~\cite{Hooker17} & \href{../works/Hooker19.pdf}{Hooker19}~\cite{Hooker19} & \href{../works/HookerH17.pdf}{HookerH17}~\cite{HookerH17}\\ 
\href{../works/HookerO03.pdf}{HookerO03}~\cite{HookerO03} & \href{../works/HookerO99.pdf}{HookerO99}~\cite{HookerO99} & \href{../works/HookerOTK00.pdf}{HookerOTK00}~\cite{HookerOTK00} & \href{../works/HookerY02.pdf}{HookerY02}~\cite{HookerY02} & \href{../works/Hosseinian2019.pdf}{Hosseinian2019}~\cite{Hosseinian2019} & \href{../works/Hosseinian2021.pdf}{Hosseinian2021}~\cite{Hosseinian2021}\\ 
\href{../works/HoundjiSW19.pdf}{HoundjiSW19}~\cite{HoundjiSW19} & \href{../works/HoundjiSWD14.pdf}{HoundjiSWD14}~\cite{HoundjiSWD14} & \href{../works/Hu2009.pdf}{Hu2009}~\cite{Hu2009} & \href{../works/Huang2000.pdf}{Huang2000}~\cite{Huang2000} & \href{../works/HubnerGSV21.pdf}{HubnerGSV21}~\cite{HubnerGSV21} & \href{../works/Hunsberger08.pdf}{Hunsberger08}~\cite{Hunsberger08}\\ 
\href{../works/HurleyOS16.pdf}{HurleyOS16}~\cite{HurleyOS16} & \href{../works/Icmeli1993.pdf}{Icmeli1993}~\cite{Icmeli1993} & \href{../}{Icmeli1996}~\cite{Icmeli1996} & \href{../works/IfrimOS12.pdf}{IfrimOS12}~\cite{IfrimOS12} & \href{../works/IklassovMR023.pdf}{IklassovMR023}~\cite{IklassovMR023} & \href{../works/IsikYA23.pdf}{IsikYA23}~\cite{IsikYA23}\\ 
\href{../}{Jaffar1998}~\cite{Jaffar1998} & \href{../works/JainG01.pdf}{JainG01}~\cite{JainG01} & \href{../works/JainM99.pdf}{JainM99}~\cite{JainM99} & \href{../works/Janosikova2013.pdf}{Janosikova2013}~\cite{Janosikova2013} & \href{../works/Jans09.pdf}{Jans09}~\cite{Jans09} & \href{../works/JelinekB16.pdf}{JelinekB16}~\cite{JelinekB16}\\ 
\href{../works/JoLLH99.pdf}{JoLLH99}~\cite{JoLLH99} & \href{../works/Johnston05.pdf}{Johnston05}~\cite{Johnston05} & \href{../}{JourdanFRD94}~\cite{JourdanFRD94} & \href{../}{Juan2014}~\cite{Juan2014} & \href{../works/JungblutK22.pdf}{JungblutK22}~\cite{JungblutK22} & \href{../works/Junker00.pdf}{Junker00}~\cite{Junker00}\\ 
\href{../works/Junker2012.pdf}{Junker2012}~\cite{Junker2012} & \href{../works/JussienL02.pdf}{JussienL02}~\cite{JussienL02} & \href{../works/JuvinHHL23.pdf}{JuvinHHL23}~\cite{JuvinHHL23} & \href{../works/JuvinHL22.pdf}{JuvinHL22}~\cite{JuvinHL22} & \href{../works/JuvinHL23.pdf}{JuvinHL23}~\cite{JuvinHL23} & \href{../works/JuvinHL23a.pdf}{JuvinHL23a}~\cite{JuvinHL23a}\\ 
\href{../works/KamarainenS02.pdf}{KamarainenS02}~\cite{KamarainenS02} & \href{../works/Kambhampati2000.pdf}{Kambhampati2000}~\cite{Kambhampati2000} & \href{../works/Kameugne14.pdf}{Kameugne14}~\cite{Kameugne14} & \href{../works/Kameugne15.pdf}{Kameugne15}~\cite{Kameugne15} & \href{../works/KameugneF13.pdf}{KameugneF13}~\cite{KameugneF13} & \href{../works/KameugneFGOQ18.pdf}{KameugneFGOQ18}~\cite{KameugneFGOQ18}\\ 
\href{../works/KameugneFND23.pdf}{KameugneFND23}~\cite{KameugneFND23} & \href{../works/KameugneFSN11.pdf}{KameugneFSN11}~\cite{KameugneFSN11} & \href{../works/KameugneFSN14.pdf}{KameugneFSN14}~\cite{KameugneFSN14} & \href{../works/KanetAG04.pdf}{KanetAG04}~\cite{KanetAG04} & \href{../}{Kasapidis2021}~\cite{Kasapidis2021} & \href{../works/Kasapidis2023.pdf}{Kasapidis2023}~\cite{Kasapidis2023}\\ 
\href{../works/Kelareva2012.pdf}{Kelareva2012}~\cite{Kelareva2012} & \href{../}{Kelareva2014}~\cite{Kelareva2014} & \href{../works/KelarevaTK13.pdf}{KelarevaTK13}~\cite{KelarevaTK13} & \href{../works/KelbelH11.pdf}{KelbelH11}~\cite{KelbelH11} & \href{../works/KendallKRU10.pdf}{KendallKRU10}~\cite{KendallKRU10} & \href{../works/KengY89.pdf}{KengY89}~\cite{KengY89}\\ 
\href{../works/KeriK07.pdf}{KeriK07}~\cite{KeriK07} & \href{../works/KhayatLR06.pdf}{KhayatLR06}~\cite{KhayatLR06} & \href{../works/KhemmoudjPB06.pdf}{KhemmoudjPB06}~\cite{KhemmoudjPB06} & \href{../works/Kim2004.pdf}{Kim2004}~\cite{Kim2004} & \href{../works/KimCMLLP23.pdf}{KimCMLLP23}~\cite{KimCMLLP23} & \href{../works/KinsellaS0OS16.pdf}{KinsellaS0OS16}~\cite{KinsellaS0OS16}\\ 
\href{../works/Kizilay2019.pdf}{Kizilay2019}~\cite{Kizilay2019} & \href{../}{KizilayC20}~\cite{KizilayC20} & \href{../works/KlankeBYE21.pdf}{KlankeBYE21}~\cite{KlankeBYE21} & \href{../works/KletzanderM17.pdf}{KletzanderM17}~\cite{KletzanderM17} & \href{../works/KletzanderM20.pdf}{KletzanderM20}~\cite{KletzanderM20} & \href{../works/KletzanderMH21.pdf}{KletzanderMH21}~\cite{KletzanderMH21}\\ 
\href{../works/KoehlerBFFHPSSS21.pdf}{KoehlerBFFHPSSS21}~\cite{KoehlerBFFHPSSS21} & \href{../}{Kong2020}~\cite{Kong2020} & \href{../}{Kong2021}~\cite{Kong2021} & \href{../works/KonowalenkoMM19.pdf}{KonowalenkoMM19}~\cite{KonowalenkoMM19} & \href{../works/KorbaaYG00.pdf}{KorbaaYG00}~\cite{KorbaaYG00} & \href{../works/KorbaaYG99.pdf}{KorbaaYG99}~\cite{KorbaaYG99}\\ 
\href{../works/KoschB14.pdf}{KoschB14}~\cite{KoschB14} & \href{../works/KotaryFH22.pdf}{KotaryFH22}~\cite{KotaryFH22} & \href{../works/KovacsB07.pdf}{KovacsB07}~\cite{KovacsB07} & \href{../works/KovacsB08.pdf}{KovacsB08}~\cite{KovacsB08} & \href{../works/KovacsB11.pdf}{KovacsB11}~\cite{KovacsB11} & \href{../works/KovacsEKV05.pdf}{KovacsEKV05}~\cite{KovacsEKV05}\\ 
\href{../works/KovacsK11.pdf}{KovacsK11}~\cite{KovacsK11} & \href{../works/KovacsTKSG21.pdf}{KovacsTKSG21}~\cite{KovacsTKSG21} & \href{../works/KovacsV04.pdf}{KovacsV04}~\cite{KovacsV04} & \href{../works/KovacsV06.pdf}{KovacsV06}~\cite{KovacsV06} & \href{../works/Kovcs2003.pdf}{Kovcs2003}~\cite{Kovcs2003} & \href{../works/KreterSS15.pdf}{KreterSS15}~\cite{KreterSS15}\\ 
\href{../works/KreterSS17.pdf}{KreterSS17}~\cite{KreterSS17} & \href{../works/KreterSSZ18.pdf}{KreterSSZ18}~\cite{KreterSSZ18} & \href{../works/KrogtLPHJ07.pdf}{KrogtLPHJ07}~\cite{KrogtLPHJ07} & \href{../works/KuB16.pdf}{KuB16}~\cite{KuB16} & \href{../works/Kuchcinski03.pdf}{Kuchcinski03}~\cite{Kuchcinski03} & \href{../works/KuchcinskiW03.pdf}{KuchcinskiW03}~\cite{KuchcinskiW03}\\ 
\href{../works/KucukY19.pdf}{KucukY19}~\cite{KucukY19} & \href{../works/Kumar03.pdf}{Kumar03}~\cite{Kumar03} & \href{../works/Kuramata2022.pdf}{Kuramata2022}~\cite{Kuramata2022} & \href{../works/KusterJF07.pdf}{KusterJF07}~\cite{KusterJF07} & \href{../works/Laborie03.pdf}{Laborie03}~\cite{Laborie03} & \href{../works/Laborie05.pdf}{Laborie05}~\cite{Laborie05}\\ 
\href{../works/Laborie09.pdf}{Laborie09}~\cite{Laborie09} & \href{../works/Laborie18a.pdf}{Laborie18a}~\cite{Laborie18a} & \href{../}{Laborie2011}~\cite{Laborie2011} & \href{../works/Laborie2017.pdf}{Laborie2017}~\cite{Laborie2017} & \href{../works/LaborieR14.pdf}{LaborieR14}~\cite{LaborieR14} & \href{../works/LaborieRSV18.pdf}{LaborieRSV18}~\cite{LaborieRSV18}\\ 
\href{../works/LacknerMMWW21.pdf}{LacknerMMWW21}~\cite{LacknerMMWW21} & \href{../works/LacknerMMWW23.pdf}{LacknerMMWW23}~\cite{LacknerMMWW23} & \href{../works/Lacomme2011.pdf}{Lacomme2011}~\cite{Lacomme2011} & \href{../works/LahimerLH11.pdf}{LahimerLH11}~\cite{LahimerLH11} & \href{../}{Lallouet2007}~\cite{Lallouet2007} & \href{../}{Lambert2014}~\cite{Lambert2014}\\ 
\href{../works/LammaMM97.pdf}{LammaMM97}~\cite{LammaMM97} & \href{../}{Larrosa1998}~\cite{Larrosa1998} & \href{../works/Larrosa2002.pdf}{Larrosa2002}~\cite{Larrosa2002} & \href{../works/LarsonJC14.pdf}{LarsonJC14}~\cite{LarsonJC14} & \href{../works/LauLN08.pdf}{LauLN08}~\cite{LauLN08} & \href{../works/Layfield02.pdf}{Layfield02}~\cite{Layfield02}\\ 
\href{../works/LeeKLKKYHP97.pdf}{LeeKLKKYHP97}~\cite{LeeKLKKYHP97} & \href{../works/Lemos21.pdf}{Lemos21}~\cite{Lemos21} & \href{../works/Letort13.pdf}{Letort13}~\cite{Letort13} & \href{../works/LetortBC12.pdf}{LetortBC12}~\cite{LetortBC12} & \href{../works/LetortCB13.pdf}{LetortCB13}~\cite{LetortCB13} & \href{../works/LetortCB15.pdf}{LetortCB15}~\cite{LetortCB15}\\ 
\href{../works/Levine2014.pdf}{Levine2014}~\cite{Levine2014} & \href{../}{Li2014}~\cite{Li2014} & \href{../works/Li2014a.pdf}{Li2014a}~\cite{Li2014a} & \href{../works/Li2014b.pdf}{Li2014b}~\cite{Li2014b} & \href{../works/Li2015.pdf}{Li2015}~\cite{Li2015} & \href{../works/Li2016.pdf}{Li2016}~\cite{Li2016}\\ 
\href{../works/Li2018.pdf}{Li2018}~\cite{Li2018} & \href{../works/Li2020.pdf}{Li2020}~\cite{Li2020} & \href{../works/LiFJZLL22.pdf}{LiFJZLL22}~\cite{LiFJZLL22} & \href{../works/LiLZDZW24.pdf}{LiLZDZW24}~\cite{LiLZDZW24} & \href{../works/LiW08.pdf}{LiW08}~\cite{LiW08} & \href{../works/LiessM08.pdf}{LiessM08}~\cite{LiessM08}\\ 
\href{../works/Lim2004.pdf}{Lim2004}~\cite{Lim2004} & \href{../works/Lim2014.pdf}{Lim2014}~\cite{Lim2014} & \href{../works/Lim2015.pdf}{Lim2015}~\cite{Lim2015} & \href{../works/LimAHO02a.pdf}{LimAHO02a}~\cite{LimAHO02a} & \href{../works/LimBTBB15.pdf}{LimBTBB15}~\cite{LimBTBB15} & \href{../works/LimBTBB15a.pdf}{LimBTBB15a}~\cite{LimBTBB15a}\\ 
\href{../works/LimHTB16.pdf}{LimHTB16}~\cite{LimHTB16} & \href{../works/LimRX04.pdf}{LimRX04}~\cite{LimRX04} & \href{../works/Limtanyakul07.pdf}{Limtanyakul07}~\cite{Limtanyakul07} & \href{../works/LimtanyakulS12.pdf}{LimtanyakulS12}~\cite{LimtanyakulS12} & \href{../works/Lindauer2015.pdf}{Lindauer2015}~\cite{Lindauer2015} & \href{../works/LipovetzkyBPS14.pdf}{LipovetzkyBPS14}~\cite{LipovetzkyBPS14}\\ 
\href{../works/Liu2020.pdf}{Liu2020}~\cite{Liu2020} & \href{../works/Liu2021.pdf}{Liu2021}~\cite{Liu2021} & \href{../works/Liu2021a.pdf}{Liu2021a}~\cite{Liu2021a} & \href{../works/Liu2021b.pdf}{Liu2021b}~\cite{Liu2021b} & \href{../works/Liu2023.pdf}{Liu2023}~\cite{Liu2023} & \href{../works/LiuCGM17.pdf}{LiuCGM17}~\cite{LiuCGM17}\\ 
\href{../works/LiuGT10.pdf}{LiuGT10}~\cite{LiuGT10} & \href{../works/LiuJ06.pdf}{LiuJ06}~\cite{LiuJ06} & \href{../works/LiuLH18.pdf}{LiuLH18}~\cite{LiuLH18} & \href{../works/LiuLH19.pdf}{LiuLH19}~\cite{LiuLH19} & \href{../works/LiuLH19a.pdf}{LiuLH19a}~\cite{LiuLH19a} & \href{../works/LiuW11.pdf}{LiuW11}~\cite{LiuW11}\\ 
\href{../works/Lizarralde2011.pdf}{Lizarralde2011}~\cite{Lizarralde2011} & \href{../works/Lombardi10.pdf}{Lombardi10}~\cite{Lombardi10} & \href{../works/LombardiBM15.pdf}{LombardiBM15}~\cite{LombardiBM15} & \href{../works/LombardiBMB11.pdf}{LombardiBMB11}~\cite{LombardiBMB11} & \href{../works/LombardiM09.pdf}{LombardiM09}~\cite{LombardiM09} & \href{../works/LombardiM10.pdf}{LombardiM10}~\cite{LombardiM10}\\ 
\href{../works/LombardiM10a.pdf}{LombardiM10a}~\cite{LombardiM10a} & \href{../works/LombardiM12.pdf}{LombardiM12}~\cite{LombardiM12} & \href{../works/LombardiM12a.pdf}{LombardiM12a}~\cite{LombardiM12a} & \href{../works/LombardiM13.pdf}{LombardiM13}~\cite{LombardiM13} & \href{../works/LombardiMB13.pdf}{LombardiMB13}~\cite{LombardiMB13} & \href{../works/LombardiMRB10.pdf}{LombardiMRB10}~\cite{LombardiMRB10}\\ 
\href{../works/LopesCSM10.pdf}{LopesCSM10}~\cite{LopesCSM10} & \href{../works/LopezAKYG00.pdf}{LopezAKYG00}~\cite{LopezAKYG00} & \href{../works/Lorca2016.pdf}{Lorca2016}~\cite{Lorca2016} & \href{../works/LorigeonBB02.pdf}{LorigeonBB02}~\cite{LorigeonBB02} & \href{../}{Lorterapong2009}~\cite{Lorterapong2009} & \href{../}{Lorterapong2013}~\cite{Lorterapong2013}\\ 
\href{../works/Loth2013.pdf}{Loth2013}~\cite{Loth2013} & \href{../works/LouieVNB14.pdf}{LouieVNB14}~\cite{LouieVNB14} & \href{../works/Lozano2014.pdf}{Lozano2014}~\cite{Lozano2014} & \href{../works/Lozano2019.pdf}{Lozano2019}~\cite{Lozano2019} & \href{../works/Lozano2019a.pdf}{Lozano2019a}~\cite{Lozano2019a} & \href{../works/LozanoCDS12.pdf}{LozanoCDS12}~\cite{LozanoCDS12}\\ 
\href{../works/Lu2021.pdf}{Lu2021}~\cite{Lu2021} & \href{../works/LuZZYW24.pdf}{LuZZYW24}~\cite{LuZZYW24} & \href{../works/LudwigKRBMS14.pdf}{LudwigKRBMS14}~\cite{LudwigKRBMS14} & \href{../works/Lunardi20.pdf}{Lunardi20}~\cite{Lunardi20} & \href{../works/LunardiBLRV20.pdf}{LunardiBLRV20}~\cite{LunardiBLRV20} & \href{../works/LuoB22.pdf}{LuoB22}~\cite{LuoB22}\\ 
\href{../works/LuoVLBM16.pdf}{LuoVLBM16}~\cite{LuoVLBM16} & \href{../works/Lyons2023.pdf}{Lyons2023}~\cite{Lyons2023} & \href{../works/Madi-WambaB16.pdf}{Madi-WambaB16}~\cite{Madi-WambaB16} & \href{../works/Madi-WambaLOBM17.pdf}{Madi-WambaLOBM17}~\cite{Madi-WambaLOBM17} & \href{../}{MagataoAN05}~\cite{MagataoAN05} & \href{../works/Magato2008.pdf}{Magato2008}~\cite{Magato2008}\\ 
\href{../works/Magato2010.pdf}{Magato2010}~\cite{Magato2010} & \href{../works/Maillard15.pdf}{Maillard15}~\cite{Maillard15} & \href{../works/MakMS10.pdf}{MakMS10}~\cite{MakMS10} & \href{../works/Malapert11.pdf}{Malapert11}~\cite{Malapert11} & \href{../works/MalapertCGJLR12.pdf}{MalapertCGJLR12}~\cite{MalapertCGJLR12} & \href{../works/MalapertCGJLR13.pdf}{MalapertCGJLR13}~\cite{MalapertCGJLR13}\\ 
\href{../works/MalapertGR12.pdf}{MalapertGR12}~\cite{MalapertGR12} & \href{../works/MalapertN19.pdf}{MalapertN19}~\cite{MalapertN19} & \href{../works/Malik08.pdf}{Malik08}~\cite{Malik08} & \href{../works/Malik2008.pdf}{Malik2008}~\cite{Malik2008} & \href{../works/MalikMB08.pdf}{MalikMB08}~\cite{MalikMB08} & \href{../works/MaraveliasCG04.pdf}{MaraveliasCG04}~\cite{MaraveliasCG04}\\ 
\href{../works/MaraveliasG04.pdf}{MaraveliasG04}~\cite{MaraveliasG04} & \href{../}{Marcolini2022}~\cite{Marcolini2022} & \href{../works/MarliereSPR23.pdf}{MarliereSPR23}~\cite{MarliereSPR23} & \href{../works/Martin2012.pdf}{Martin2012}~\cite{Martin2012} & \href{../works/MartinPY01.pdf}{MartinPY01}~\cite{MartinPY01} & \href{../}{MartnezAJ22}~\cite{MartnezAJ22}\\ 
\href{../works/Mason01.pdf}{Mason01}~\cite{Mason01} & \href{../works/Mehdizadeh-Somarin23.pdf}{Mehdizadeh-Somarin23}~\cite{Mehdizadeh-Somarin23} & \href{../works/MejiaY20.pdf}{MejiaY20}~\cite{MejiaY20} & \href{../works/MelgarejoLS15.pdf}{MelgarejoLS15}~\cite{MelgarejoLS15} & \href{../works/Menana11.pdf}{Menana11}~\cite{Menana11} & \href{../works/MenciaSV12.pdf}{MenciaSV12}~\cite{MenciaSV12}\\ 
\href{../works/MenciaSV13.pdf}{MenciaSV13}~\cite{MenciaSV13} & \href{../works/MengGRZSC22.pdf}{MengGRZSC22}~\cite{MengGRZSC22} & \href{../works/MengLZB21.pdf}{MengLZB21}~\cite{MengLZB21} & \href{../works/MengZRZL20.pdf}{MengZRZL20}~\cite{MengZRZL20} & \href{../works/Menouer2016.pdf}{Menouer2016}~\cite{Menouer2016} & \href{../works/Mercier-AubinGQ20.pdf}{Mercier-AubinGQ20}~\cite{Mercier-AubinGQ20}\\ 
\href{../works/MercierH07.pdf}{MercierH07}~\cite{MercierH07} & \href{../works/MercierH08.pdf}{MercierH08}~\cite{MercierH08} & \href{../}{Mesghouni1997}~\cite{Mesghouni1997} & \href{../works/MeskensDHG11.pdf}{MeskensDHG11}~\cite{MeskensDHG11} & \href{../works/MeskensDL13.pdf}{MeskensDL13}~\cite{MeskensDL13} & \href{../works/MeyerE04.pdf}{MeyerE04}~\cite{MeyerE04}\\ 
\href{../works/Michel2004.pdf}{Michel2004}~\cite{Michel2004} & \href{../}{Michel2009}~\cite{Michel2009} & \href{../works/Michel2012.pdf}{Michel2012}~\cite{Michel2012} & \href{../works/Michels2022.pdf}{Michels2022}~\cite{Michels2022} & \href{../}{Milano11}~\cite{Milano11} & \href{../}{MilanoORT02}~\cite{MilanoORT02}\\ 
\href{../works/MilanoW06.pdf}{MilanoW06}~\cite{MilanoW06} & \href{../works/MilanoW09.pdf}{MilanoW09}~\cite{MilanoW09} & \href{../works/MintonJPL90.pdf}{MintonJPL90}~\cite{MintonJPL90} & \href{../works/MintonJPL92.pdf}{MintonJPL92}~\cite{MintonJPL92} & \href{../works/Mischek2021.pdf}{Mischek2021}~\cite{Mischek2021} & \href{../works/Mischek2021a.pdf}{Mischek2021a}~\cite{Mischek2021a}\\ 
\href{../works/Misra2022.pdf}{Misra2022}~\cite{Misra2022} & \href{../works/Mladenovic2007.pdf}{Mladenovic2007}~\cite{Mladenovic2007} & \href{../works/Mladenovic2015.pdf}{Mladenovic2015}~\cite{Mladenovic2015} & \href{../works/Mller2003.pdf}{Mller2003}~\cite{Mller2003} & \href{../}{Mnif2020}~\cite{Mnif2020} & \href{../works/Moccia2005.pdf}{Moccia2005}~\cite{Moccia2005}\\ 
\href{../works/MoffittPP05.pdf}{MoffittPP05}~\cite{MoffittPP05} & \href{../works/MokhtarzadehTNF20.pdf}{MokhtarzadehTNF20}~\cite{MokhtarzadehTNF20} & \href{../works/MonetteDD07.pdf}{MonetteDD07}~\cite{MonetteDD07} & \href{../works/MonetteDH09.pdf}{MonetteDH09}~\cite{MonetteDH09} & \href{../works/MontemanniD23.pdf}{MontemanniD23}~\cite{MontemanniD23} & \href{../works/MontemanniD23a.pdf}{MontemanniD23a}~\cite{MontemanniD23a}\\ 
\href{../works/Moreno-Scott2016.pdf}{Moreno-Scott2016}~\cite{Moreno-Scott2016} & \href{../works/MorgadoM97.pdf}{MorgadoM97}~\cite{MorgadoM97} & \href{../works/Morillo2017.pdf}{Morillo2017}~\cite{Morillo2017} & \href{../works/MossigeGSMC17.pdf}{MossigeGSMC17}~\cite{MossigeGSMC17} & \href{../}{Moukrim2014}~\cite{Moukrim2014} & \href{../works/MouraSCL08.pdf}{MouraSCL08}~\cite{MouraSCL08}\\ 
\href{../works/MouraSCL08a.pdf}{MouraSCL08a}~\cite{MouraSCL08a} & \href{../works/MullerMKP22.pdf}{MullerMKP22}~\cite{MullerMKP22} & \href{../works/MurinR19.pdf}{MurinR19}~\cite{MurinR19} & \href{../works/MurphyMB15.pdf}{MurphyMB15}~\cite{MurphyMB15} & \href{../works/MurphyRFSS97.pdf}{MurphyRFSS97}~\cite{MurphyRFSS97} & \href{../}{MurthyRAW97}~\cite{MurthyRAW97}\\ 
\href{../works/Muscettola02.pdf}{Muscettola02}~\cite{Muscettola02} & \href{../works/Muscettola94.pdf}{Muscettola94}~\cite{Muscettola94} & \href{../works/Musliu05.pdf}{Musliu05}~\cite{Musliu05} & \href{../works/MusliuSS18.pdf}{MusliuSS18}~\cite{MusliuSS18} & \href{../}{Mutha2017}~\cite{Mutha2017} & \href{../works/NaderiBZ22.pdf}{NaderiBZ22}~\cite{NaderiBZ22}\\ 
\href{../works/NaderiBZ22a.pdf}{NaderiBZ22a}~\cite{NaderiBZ22a} & \href{../works/NaderiBZ23.pdf}{NaderiBZ23}~\cite{NaderiBZ23} & \href{../works/NaderiBZR23.pdf}{NaderiBZR23}~\cite{NaderiBZR23} & \href{../}{NaderiR22}~\cite{NaderiR22} & \href{../}{NaderiRBAU21}~\cite{NaderiRBAU21} & \href{../works/NaderiRR23.pdf}{NaderiRR23}~\cite{NaderiRR23}\\ 
\href{../works/NaqviAIAAA22.pdf}{NaqviAIAAA22}~\cite{NaqviAIAAA22} & \href{../works/Nattaf16.pdf}{Nattaf16}~\cite{Nattaf16} & \href{../works/NattafAL15.pdf}{NattafAL15}~\cite{NattafAL15} & \href{../works/NattafAL17.pdf}{NattafAL17}~\cite{NattafAL17} & \href{../works/NattafALR16.pdf}{NattafALR16}~\cite{NattafALR16} & \href{../works/NattafDYW19.pdf}{NattafDYW19}~\cite{NattafDYW19}\\ 
\href{../works/NattafHKAL19.pdf}{NattafHKAL19}~\cite{NattafHKAL19} & \href{../works/NattafM20.pdf}{NattafM20}~\cite{NattafM20} & \href{../}{NeronABCDD06}~\cite{NeronABCDD06} & \href{../works/NishikawaSTT18.pdf}{NishikawaSTT18}~\cite{NishikawaSTT18} & \href{../works/NishikawaSTT18a.pdf}{NishikawaSTT18a}~\cite{NishikawaSTT18a} & \href{../works/NishikawaSTT19.pdf}{NishikawaSTT19}~\cite{NishikawaSTT19}\\ 
\href{../}{NouriMHD23}~\cite{NouriMHD23} & \href{../}{Novara2013}~\cite{Novara2013} & \href{../}{Novara2015}~\cite{Novara2015} & \href{../works/NovaraNH16.pdf}{NovaraNH16}~\cite{NovaraNH16} & \href{../works/Novas19.pdf}{Novas19}~\cite{Novas19} & \href{../works/NovasH10.pdf}{NovasH10}~\cite{NovasH10}\\ 
\href{../works/NovasH12.pdf}{NovasH12}~\cite{NovasH12} & \href{../works/NovasH14.pdf}{NovasH14}~\cite{NovasH14} & \href{../works/Nowatzki2013.pdf}{Nowatzki2013}~\cite{Nowatzki2013} & \href{../works/Nuijten94.pdf}{Nuijten94}~\cite{Nuijten94} & \href{../works/NuijtenA94.pdf}{NuijtenA94}~\cite{NuijtenA94} & \href{../}{NuijtenA94a}~\cite{NuijtenA94a}\\ 
\href{../works/NuijtenA96.pdf}{NuijtenA96}~\cite{NuijtenA96} & \href{../works/NuijtenP98.pdf}{NuijtenP98}~\cite{NuijtenP98} & \href{../works/OddiPCC03.pdf}{OddiPCC03}~\cite{OddiPCC03} & \href{../}{OddiPCC05}~\cite{OddiPCC05} & \href{../works/OddiRC10.pdf}{OddiRC10}~\cite{OddiRC10} & \href{../works/OddiRCS11.pdf}{OddiRCS11}~\cite{OddiRCS11}\\ 
\href{../works/OddiS97.pdf}{OddiS97}~\cite{OddiS97} & \href{../works/OhrimenkoSC09.pdf}{OhrimenkoSC09}~\cite{OhrimenkoSC09} & \href{../}{OkanoDTRYA04}~\cite{OkanoDTRYA04} & \href{../}{Oliveira2015}~\cite{Oliveira2015} & \href{../works/OrnekO16.pdf}{OrnekO16}~\cite{OrnekO16} & \href{../works/OrnekOS20.pdf}{OrnekOS20}~\cite{OrnekOS20}\\ 
\href{../works/Ortiz-Bayliss2013.pdf}{Ortiz-Bayliss2013}~\cite{Ortiz-Bayliss2013} & \href{../}{Ortiz-Bayliss2014}~\cite{Ortiz-Bayliss2014} & \href{../works/Ortiz-Bayliss2018.pdf}{Ortiz-Bayliss2018}~\cite{Ortiz-Bayliss2018} & \href{../works/Ortiz-Bayliss2021.pdf}{Ortiz-Bayliss2021}~\cite{Ortiz-Bayliss2021} & \href{../works/Ouaja2004.pdf}{Ouaja2004}~\cite{Ouaja2004} & \href{../works/Ouellet2022.pdf}{Ouellet2022}~\cite{Ouellet2022}\\ 
\href{../works/OuelletQ13.pdf}{OuelletQ13}~\cite{OuelletQ13} & \href{../works/OuelletQ18.pdf}{OuelletQ18}~\cite{OuelletQ18} & \href{../works/OuelletQ22.pdf}{OuelletQ22}~\cite{OuelletQ22} & \href{../works/Oujana2023.pdf}{Oujana2023}~\cite{Oujana2023} & \href{../works/OujanaAYB22.pdf}{OujanaAYB22}~\cite{OujanaAYB22} & \href{../works/Ozder2019.pdf}{Ozder2019}~\cite{Ozder2019}\\ 
\href{../works/OzturkTHO10.pdf}{OzturkTHO10}~\cite{OzturkTHO10} & \href{../works/OzturkTHO12.pdf}{OzturkTHO12}~\cite{OzturkTHO12} & \href{../works/OzturkTHO13.pdf}{OzturkTHO13}~\cite{OzturkTHO13} & \href{../works/OzturkTHO15.pdf}{OzturkTHO15}~\cite{OzturkTHO15} & \href{../works/PachecoPR19.pdf}{PachecoPR19}~\cite{PachecoPR19} & \href{../works/PacinoH11.pdf}{PacinoH11}~\cite{PacinoH11}\\ 
\href{../works/PandeyS21a.pdf}{PandeyS21a}~\cite{PandeyS21a} & \href{../works/PapaB98.pdf}{PapaB98}~\cite{PapaB98} & \href{../works/Pape94.pdf}{Pape94}~\cite{Pape94} & \href{../}{PapeB96}~\cite{PapeB96} & \href{../works/PapeB97.pdf}{PapeB97}~\cite{PapeB97} & \href{../}{Paredis1992}~\cite{Paredis1992}\\ 
\href{../}{Park2016}~\cite{Park2016} & \href{../works/ParkUJR19.pdf}{ParkUJR19}~\cite{ParkUJR19} & \href{../works/PembertonG98.pdf}{PembertonG98}~\cite{PembertonG98} & \href{../}{Peng2012}~\cite{Peng2012} & \href{../works/PengLC14.pdf}{PengLC14}~\cite{PengLC14} & \href{../works/PenzDN23.pdf}{PenzDN23}~\cite{PenzDN23}\\ 
\href{../works/PerezGSL23.pdf}{PerezGSL23}~\cite{PerezGSL23} & \href{../works/Perron05.pdf}{Perron05}~\cite{Perron05} & \href{../works/PerronSF04.pdf}{PerronSF04}~\cite{PerronSF04} & \href{../works/Pesant2012.pdf}{Pesant2012}~\cite{Pesant2012} & \href{../works/PesantGPR99.pdf}{PesantGPR99}~\cite{PesantGPR99} & \href{../works/PesantRR15.pdf}{PesantRR15}~\cite{PesantRR15}\\ 
\href{../}{PeschT96}~\cite{PeschT96} & \href{../works/Pessoa2013.pdf}{Pessoa2013}~\cite{Pessoa2013} & \href{../works/Petith2002.pdf}{Petith2002}~\cite{Petith2002} & \href{../}{Petrovic2008}~\cite{Petrovic2008} & \href{../}{Pinarbasi21}~\cite{Pinarbasi21} & \href{../}{PinarbasiA20}~\cite{PinarbasiA20}\\ 
\href{../works/PinarbasiAY19.pdf}{PinarbasiAY19}~\cite{PinarbasiAY19} & \href{../works/Pinto2012.pdf}{Pinto2012}~\cite{Pinto2012} & \href{../works/PintoG97.pdf}{PintoG97}~\cite{PintoG97} & \href{../works/PoderB08.pdf}{PoderB08}~\cite{PoderB08} & \href{../works/PoderBS04.pdf}{PoderBS04}~\cite{PoderBS04} & \href{../works/PohlAK22.pdf}{PohlAK22}~\cite{PohlAK22}\\ 
\href{../works/PolicellaWSO05.pdf}{PolicellaWSO05}~\cite{PolicellaWSO05} & \href{../works/Polo-MejiaALB20.pdf}{Polo-MejiaALB20}~\cite{Polo-MejiaALB20} & \href{../works/PopovicCGNC22.pdf}{PopovicCGNC22}~\cite{PopovicCGNC22} & \href{../works/PourDERB18.pdf}{PourDERB18}~\cite{PourDERB18} & \href{../works/PovedaAA23.pdf}{PovedaAA23}~\cite{PovedaAA23} & \href{../works/Pralet17.pdf}{Pralet17}~\cite{Pralet17}\\ 
\href{../works/PraletLJ15.pdf}{PraletLJ15}~\cite{PraletLJ15} & \href{../works/PrataAN23.pdf}{PrataAN23}~\cite{PrataAN23} & \href{../works/Priore2003.pdf}{Priore2003}~\cite{Priore2003} & \href{../works/Prosser89.pdf}{Prosser89}~\cite{Prosser89} & \href{../works/Psarras1997.pdf}{Psarras1997}~\cite{Psarras1997} & \href{../works/Puget95.pdf}{Puget95}~\cite{Puget95}\\ 
\href{../works/QinDCS20.pdf}{QinDCS20}~\cite{QinDCS20} & \href{../works/QinDS16.pdf}{QinDS16}~\cite{QinDS16} & \href{../works/QinWSLS21.pdf}{QinWSLS21}~\cite{QinWSLS21} & \href{../works/QuSN06.pdf}{QuSN06}~\cite{QuSN06} & \href{../works/QuirogaZH05.pdf}{QuirogaZH05}~\cite{QuirogaZH05} & \href{../}{RabbaniMM21}~\cite{RabbaniMM21}\\ 
\href{../works/Radzki2021.pdf}{Radzki2021}~\cite{Radzki2021} & \href{../}{Raffin2012}~\cite{Raffin2012} & \href{../works/Ramos2021.pdf}{Ramos2021}~\cite{Ramos2021} & \href{../works/Ramos2023.pdf}{Ramos2023}~\cite{Ramos2023} & \href{../works/RasmussenT06.pdf}{RasmussenT06}~\cite{RasmussenT06} & \href{../works/RasmussenT07.pdf}{RasmussenT07}~\cite{RasmussenT07}\\ 
\href{../works/RasmussenT09.pdf}{RasmussenT09}~\cite{RasmussenT09} & \href{../works/Reale2014.pdf}{Reale2014}~\cite{Reale2014} & \href{../works/ReddyFIBKAJ11.pdf}{ReddyFIBKAJ11}~\cite{ReddyFIBKAJ11} & \href{../works/Refalo00.pdf}{Refalo00}~\cite{Refalo00} & \href{../works/Refanidis2010.pdf}{Refanidis2010}~\cite{Refanidis2010} & \href{../works/Relich2020.pdf}{Relich2020}~\cite{Relich2020}\\ 
\href{../works/Relich2022.pdf}{Relich2022}~\cite{Relich2022} & \href{../works/Relich2023.pdf}{Relich2023}~\cite{Relich2023} & \href{../works/Ren2016.pdf}{Ren2016}~\cite{Ren2016} & \href{../works/RenT09.pdf}{RenT09}~\cite{RenT09} & \href{../works/RendlPHPR12.pdf}{RendlPHPR12}~\cite{RendlPHPR12} & \href{../}{Rgin2001}~\cite{Rgin2001}\\ 
\href{../works/RiahiNS018.pdf}{RiahiNS018}~\cite{RiahiNS018} & \href{../works/Ribeiro12.pdf}{Ribeiro12}~\cite{Ribeiro12} & \href{../works/Richard1998.pdf}{Richard1998}~\cite{Richard1998} & \href{../works/Richard2002.pdf}{Richard2002}~\cite{Richard2002} & \href{../works/Rieber2021.pdf}{Rieber2021}~\cite{Rieber2021} & \href{../works/RiiseML16.pdf}{RiiseML16}~\cite{RiiseML16}\\ 
\href{../works/Rit86.pdf}{Rit86}~\cite{Rit86} & \href{../}{Rodosek94}~\cite{Rodosek94} & \href{../works/RodosekW98.pdf}{RodosekW98}~\cite{RodosekW98} & \href{../works/RodosekWH99.pdf}{RodosekWH99}~\cite{RodosekWH99} & \href{../works/Rodriguez07.pdf}{Rodriguez07}~\cite{Rodriguez07} & \href{../works/Rodriguez07b.pdf}{Rodriguez07b}~\cite{Rodriguez07b}\\ 
\href{../works/RodriguezDG02.pdf}{RodriguezDG02}~\cite{RodriguezDG02} & \href{../works/RodriguezS09.pdf}{RodriguezS09}~\cite{RodriguezS09} & \href{../}{Roe2003}~\cite{Roe2003} & \href{../works/RoePS05.pdf}{RoePS05}~\cite{RoePS05} & \href{../works/RoshanaeiBAUB20.pdf}{RoshanaeiBAUB20}~\cite{RoshanaeiBAUB20} & \href{../works/RoshanaeiLAU17.pdf}{RoshanaeiLAU17}~\cite{RoshanaeiLAU17}\\ 
\href{../}{RoshanaeiLAU17a}~\cite{RoshanaeiLAU17a} & \href{../works/RoshanaeiN21.pdf}{RoshanaeiN21}~\cite{RoshanaeiN21} & \href{../works/RossiTHP07.pdf}{RossiTHP07}~\cite{RossiTHP07} & \href{../works/RoweJCA96.pdf}{RoweJCA96}~\cite{RoweJCA96} & \href{../works/RuggieroBBMA09.pdf}{RuggieroBBMA09}~\cite{RuggieroBBMA09} & \href{../works/Ruixin2018.pdf}{Ruixin2018}~\cite{Ruixin2018}\\ 
\href{../works/RussellU06.pdf}{RussellU06}~\cite{RussellU06} & \href{../works/SacramentoSP20.pdf}{SacramentoSP20}~\cite{SacramentoSP20} & \href{../works/Sadeh1995.pdf}{Sadeh1995}~\cite{Sadeh1995} & \href{../works/SadehF96.pdf}{SadehF96}~\cite{SadehF96} & \href{../works/Sadykov04.pdf}{Sadykov04}~\cite{Sadykov04} & \href{../works/Sadykov2003.pdf}{Sadykov2003}~\cite{Sadykov2003}\\ 
\href{../works/SadykovW06.pdf}{SadykovW06}~\cite{SadykovW06} & \href{../works/Sahli2021.pdf}{Sahli2021}~\cite{Sahli2021} & \href{../works/Sahraeian2015.pdf}{Sahraeian2015}~\cite{Sahraeian2015} & \href{../}{SakkoutRW98}~\cite{SakkoutRW98} & \href{../works/SakkoutW00.pdf}{SakkoutW00}~\cite{SakkoutW00} & \href{../works/Salido10.pdf}{Salido10}~\cite{Salido10}\\ 
\href{../works/Salido2008.pdf}{Salido2008}~\cite{Salido2008} & \href{../works/Salido2008a.pdf}{Salido2008a}~\cite{Salido2008a} & \href{../works/Salvagnin2012.pdf}{Salvagnin2012}~\cite{Salvagnin2012} & \href{../works/Satish2007.pdf}{Satish2007}~\cite{Satish2007} & \href{../}{Schaerf96}~\cite{Schaerf96} & \href{../works/Schaerf97.pdf}{Schaerf97}~\cite{Schaerf97}\\ 
\href{../works/SchausD08.pdf}{SchausD08}~\cite{SchausD08} & \href{../works/SchausHMCMD11.pdf}{SchausHMCMD11}~\cite{SchausHMCMD11} & \href{../}{Schiex1994}~\cite{Schiex1994} & \href{../works/SchildW00.pdf}{SchildW00}~\cite{SchildW00} & \href{../works/SchnellH15.pdf}{SchnellH15}~\cite{SchnellH15} & \href{../works/SchnellH17.pdf}{SchnellH17}~\cite{SchnellH17}\\ 
\href{../works/Schutt11.pdf}{Schutt11}~\cite{Schutt11} & \href{../works/SchuttCSW12.pdf}{SchuttCSW12}~\cite{SchuttCSW12} & \href{../works/SchuttFS13.pdf}{SchuttFS13}~\cite{SchuttFS13} & \href{../works/SchuttFS13a.pdf}{SchuttFS13a}~\cite{SchuttFS13a} & \href{../works/SchuttFSW09.pdf}{SchuttFSW09}~\cite{SchuttFSW09} & \href{../works/SchuttFSW11.pdf}{SchuttFSW11}~\cite{SchuttFSW11}\\ 
\href{../works/SchuttFSW13.pdf}{SchuttFSW13}~\cite{SchuttFSW13} & \href{../}{SchuttFSW15}~\cite{SchuttFSW15} & \href{../works/SchuttS16.pdf}{SchuttS16}~\cite{SchuttS16} & \href{../works/SchuttW10.pdf}{SchuttW10}~\cite{SchuttW10} & \href{../works/SchuttWS05.pdf}{SchuttWS05}~\cite{SchuttWS05} & \href{../works/Schwarz2019.pdf}{Schwarz2019}~\cite{Schwarz2019}\\ 
\href{../works/Schweitzer2023.pdf}{Schweitzer2023}~\cite{Schweitzer2023} & \href{../works/SenderovichBB19.pdf}{SenderovichBB19}~\cite{SenderovichBB19} & \href{../works/SerraNM12.pdf}{SerraNM12}~\cite{SerraNM12} & \href{../works/ShaikhK23.pdf}{ShaikhK23}~\cite{ShaikhK23} & \href{../}{ShangGuan2012}~\cite{ShangGuan2012} & \href{../}{ShiYXQ22}~\cite{ShiYXQ22}\\ 
\href{../works/ShinBBHO18.pdf}{ShinBBHO18}~\cite{ShinBBHO18} & \href{../works/Shobaki2013.pdf}{Shobaki2013}~\cite{Shobaki2013} & \href{../works/Siala15.pdf}{Siala15}~\cite{Siala15} & \href{../works/Siala15a.pdf}{Siala15a}~\cite{Siala15a} & \href{../works/SialaAH15.pdf}{SialaAH15}~\cite{SialaAH15} & \href{../works/Silva2014.pdf}{Silva2014}~\cite{Silva2014}\\ 
\href{../works/SimoninAHL12.pdf}{SimoninAHL12}~\cite{SimoninAHL12} & \href{../works/SimoninAHL15.pdf}{SimoninAHL15}~\cite{SimoninAHL15} & \href{../works/Simonis07.pdf}{Simonis07}~\cite{Simonis07} & \href{../works/Simonis95.pdf}{Simonis95}~\cite{Simonis95} & \href{../works/Simonis95a.pdf}{Simonis95a}~\cite{Simonis95a} & \href{../works/Simonis99.pdf}{Simonis99}~\cite{Simonis99}\\ 
\href{../works/SimonisC95.pdf}{SimonisC95}~\cite{SimonisC95} & \href{../works/SimonisCK00.pdf}{SimonisCK00}~\cite{SimonisCK00} & \href{../works/SimonisH11.pdf}{SimonisH11}~\cite{SimonisH11} & \href{../works/Sitek2016.pdf}{Sitek2016}~\cite{Sitek2016} & \href{../works/Sitek2017.pdf}{Sitek2017}~\cite{Sitek2017} & \href{../}{Sitek2017a}~\cite{Sitek2017a}\\ 
\href{../works/Sitek2018.pdf}{Sitek2018}~\cite{Sitek2018} & \href{../works/Smith-Miles2009.pdf}{Smith-Miles2009}~\cite{Smith-Miles2009} & \href{../works/SmithBHW96.pdf}{SmithBHW96}~\cite{SmithBHW96} & \href{../works/SmithC93.pdf}{SmithC93}~\cite{SmithC93} & \href{../works/Soh2015.pdf}{Soh2015}~\cite{Soh2015} & \href{../works/Song2022.pdf}{Song2022}~\cite{Song2022}\\ 
\href{../works/Soto2015.pdf}{Soto2015}~\cite{Soto2015} & \href{../works/SourdN00.pdf}{SourdN00}~\cite{SourdN00} & \href{../works/Spieker2021.pdf}{Spieker2021}~\cite{Spieker2021} & \href{../works/Squillaci2022.pdf}{Squillaci2022}~\cite{Squillaci2022} & \href{../works/SquillaciPR23.pdf}{SquillaciPR23}~\cite{SquillaciPR23} & \href{../}{Stebel2006}~\cite{Stebel2006}\\ 
\href{../}{StidsenKM96}~\cite{StidsenKM96} & \href{../works/Stobbe1999.pdf}{Stobbe1999}~\cite{Stobbe1999} & \href{../works/Strak2021.pdf}{Strak2021}~\cite{Strak2021} & \href{../works/SuCC13.pdf}{SuCC13}~\cite{SuCC13} & \href{../works/SubulanC22.pdf}{SubulanC22}~\cite{SubulanC22} & \href{../works/SultanikMR07.pdf}{SultanikMR07}~\cite{SultanikMR07}\\ 
\href{../works/Sun2006.pdf}{Sun2006}~\cite{Sun2006} & \href{../works/SunLYL10.pdf}{SunLYL10}~\cite{SunLYL10} & \href{../works/SunTB19.pdf}{SunTB19}~\cite{SunTB19} & \href{../works/SureshMOK06.pdf}{SureshMOK06}~\cite{SureshMOK06} & \href{../works/SvancaraB22.pdf}{SvancaraB22}~\cite{SvancaraB22} & \href{../works/SzerediS16.pdf}{SzerediS16}~\cite{SzerediS16}\\ 
\href{../works/Talbi2013.pdf}{Talbi2013}~\cite{Talbi2013} & \href{../}{Talbi2013a}~\cite{Talbi2013a} & \href{../works/Talbi2015.pdf}{Talbi2015}~\cite{Talbi2015} & \href{../}{Talbot1978}~\cite{Talbot1978} & \href{../works/TanSD10.pdf}{TanSD10}~\cite{TanSD10} & \href{../works/TanT18.pdf}{TanT18}~\cite{TanT18}\\ 
\href{../works/TanZWGQ19.pdf}{TanZWGQ19}~\cite{TanZWGQ19} & \href{../works/Tang2014.pdf}{Tang2014}~\cite{Tang2014} & \href{../works/Tang2018.pdf}{Tang2018}~\cite{Tang2018} & \href{../works/TangB20.pdf}{TangB20}~\cite{TangB20} & \href{../works/TangLWSK18.pdf}{TangLWSK18}~\cite{TangLWSK18} & \href{../}{Tapkan2022}~\cite{Tapkan2022}\\ 
\href{../works/TardivoDFMP23.pdf}{TardivoDFMP23}~\cite{TardivoDFMP23} & \href{../works/Tassel22.pdf}{Tassel22}~\cite{Tassel22} & \href{../works/TasselGS23.pdf}{TasselGS23}~\cite{TasselGS23} & \href{../}{Tay92}~\cite{Tay92} & \href{../works/Tayyab2023.pdf}{Tayyab2023}~\cite{Tayyab2023} & \href{../works/Teppan22.pdf}{Teppan22}~\cite{Teppan22}\\ 
\href{../works/Terashima-Marn2008.pdf}{Terashima-Marn2008}~\cite{Terashima-Marn2008} & \href{../works/Terashima-Marn2008a.pdf}{Terashima-Marn2008a}~\cite{Terashima-Marn2008a} & \href{../works/TerekhovDOB12.pdf}{TerekhovDOB12}~\cite{TerekhovDOB12} & \href{../works/TerekhovTDB14.pdf}{TerekhovTDB14}~\cite{TerekhovTDB14} & \href{../works/Tesch16.pdf}{Tesch16}~\cite{Tesch16} & \href{../works/Tesch18.pdf}{Tesch18}~\cite{Tesch18}\\ 
\href{../works/Tesch2020.pdf}{Tesch2020}~\cite{Tesch2020} & \href{../works/Teschemacher2016.pdf}{Teschemacher2016}~\cite{Teschemacher2016} & \href{../works/ThiruvadyBME09.pdf}{ThiruvadyBME09}~\cite{ThiruvadyBME09} & \href{../works/ThiruvadyWGS14.pdf}{ThiruvadyWGS14}~\cite{ThiruvadyWGS14} & \href{../works/ThomasKS20.pdf}{ThomasKS20}~\cite{ThomasKS20} & \href{../works/Thorsteinsson01.pdf}{Thorsteinsson01}~\cite{Thorsteinsson01}\\ 
\href{../works/Timpe02.pdf}{Timpe02}~\cite{Timpe02} & \href{../}{Timpe2003}~\cite{Timpe2003} & \href{../works/Tom19.pdf}{Tom19}~\cite{Tom19} & \href{../works/Tomczak2022.pdf}{Tomczak2022}~\cite{Tomczak2022} & \href{../works/TopalogluO11.pdf}{TopalogluO11}~\cite{TopalogluO11} & \href{../works/TopalogluSS12.pdf}{TopalogluSS12}~\cite{TopalogluSS12}\\ 
\href{../works/TorresL00.pdf}{TorresL00}~\cite{TorresL00} & \href{../works/TouatBT22.pdf}{TouatBT22}~\cite{TouatBT22} & \href{../works/Touraivane95.pdf}{Touraivane95}~\cite{Touraivane95} & \href{../works/TranAB16.pdf}{TranAB16}~\cite{TranAB16} & \href{../works/TranB12.pdf}{TranB12}~\cite{TranB12} & \href{../works/TranDRFWOVB16.pdf}{TranDRFWOVB16}~\cite{TranDRFWOVB16}\\ 
\href{../works/TranPZLDB18.pdf}{TranPZLDB18}~\cite{TranPZLDB18} & \href{../works/TranTDB13.pdf}{TranTDB13}~\cite{TranTDB13} & \href{../works/TranVNB17.pdf}{TranVNB17}~\cite{TranVNB17} & \href{../works/TranVNB17a.pdf}{TranVNB17a}~\cite{TranVNB17a} & \href{../works/TranWDRFOVB16.pdf}{TranWDRFOVB16}~\cite{TranWDRFOVB16} & \href{../works/TrentesauxPT01.pdf}{TrentesauxPT01}~\cite{TrentesauxPT01}\\ 
\href{../works/Trick03.pdf}{Trick03}~\cite{Trick03} & \href{../}{Trick11}~\cite{Trick11} & \href{../works/Trilling2006.pdf}{Trilling2006}~\cite{Trilling2006} & \href{../}{Triska2011}~\cite{Triska2011} & \href{../works/Trker2018.pdf}{Trker2018}~\cite{Trker2018} & \href{../works/TrojetHL11.pdf}{TrojetHL11}~\cite{TrojetHL11}\\ 
\href{../works/Tsang03.pdf}{Tsang03}~\cite{Tsang03} & \href{../works/Tseng2008.pdf}{Tseng2008}~\cite{Tseng2008} & \href{../}{TsurutaS00}~\cite{TsurutaS00} & \href{../works/UnsalO13.pdf}{UnsalO13}~\cite{UnsalO13} & \href{../works/UnsalO19.pdf}{UnsalO19}~\cite{UnsalO19} & \href{../works/Valdes87.pdf}{Valdes87}~\cite{Valdes87}\\ 
\href{../works/ValleMGT03.pdf}{ValleMGT03}~\cite{ValleMGT03} & \href{../works/Valouxis2022.pdf}{Valouxis2022}~\cite{Valouxis2022} & \href{../works/VanczaM01.pdf}{VanczaM01}~\cite{VanczaM01} & \href{../works/Varnier2002.pdf}{Varnier2002}~\cite{Varnier2002} & \href{../}{Vazacopoulos2005}~\cite{Vazacopoulos2005} & \href{../works/Velez2013.pdf}{Velez2013}~\cite{Velez2013}\\ 
\href{../works/Velez2014.pdf}{Velez2014}~\cite{Velez2014} & \href{../works/Verfaillie2010.pdf}{Verfaillie2010}~\cite{Verfaillie2010} & \href{../works/VerfaillieL01.pdf}{VerfaillieL01}~\cite{VerfaillieL01} & \href{../works/Vilim02.pdf}{Vilim02}~\cite{Vilim02} & \href{../works/Vilim03.pdf}{Vilim03}~\cite{Vilim03} & \href{../works/Vilim04.pdf}{Vilim04}~\cite{Vilim04}\\ 
\href{../works/Vilim05.pdf}{Vilim05}~\cite{Vilim05} & \href{../works/Vilim09.pdf}{Vilim09}~\cite{Vilim09} & \href{../works/Vilim09a.pdf}{Vilim09a}~\cite{Vilim09a} & \href{../works/Vilim11.pdf}{Vilim11}~\cite{Vilim11} & \href{../works/VilimBC04.pdf}{VilimBC04}~\cite{VilimBC04} & \href{../works/VilimBC05.pdf}{VilimBC05}~\cite{VilimBC05}\\ 
\href{../works/VilimLS15.pdf}{VilimLS15}~\cite{VilimLS15} & \href{../}{VillaverdeP04}~\cite{VillaverdeP04} & \href{../works/VlkHT21.pdf}{VlkHT21}~\cite{VlkHT21} & \href{../works/Wallace06.pdf}{Wallace06}~\cite{Wallace06} & \href{../}{Wallace2008}~\cite{Wallace2008} & \href{../}{Wallace94}~\cite{Wallace94}\\ 
\href{../works/Wallace96.pdf}{Wallace96}~\cite{Wallace96} & \href{../works/WallaceF00.pdf}{WallaceF00}~\cite{WallaceF00} & \href{../works/WallaceY20.pdf}{WallaceY20}~\cite{WallaceY20} & \href{../works/Wang2007.pdf}{Wang2007}~\cite{Wang2007} & \href{../}{Wang2013}~\cite{Wang2013} & \href{../works/Wang2014.pdf}{Wang2014}~\cite{Wang2014}\\ 
\href{../works/Wang2015.pdf}{Wang2015}~\cite{Wang2015} & \href{../works/Wang2021.pdf}{Wang2021}~\cite{Wang2021} & \href{../works/WangB20.pdf}{WangB20}~\cite{WangB20} & \href{../works/WangB23.pdf}{WangB23}~\cite{WangB23} & \href{../works/WangMD15.pdf}{WangMD15}~\cite{WangMD15} & \href{../}{WariZ19}~\cite{WariZ19}\\ 
\href{../works/Watermeyer2020.pdf}{Watermeyer2020}~\cite{Watermeyer2020} & \href{../works/WatsonB08.pdf}{WatsonB08}~\cite{WatsonB08} & \href{../works/WatsonBHW99.pdf}{WatsonBHW99}~\cite{WatsonBHW99} & \href{../works/Weil1992.pdf}{Weil1992}~\cite{Weil1992} & \href{../works/WeilHFP95.pdf}{WeilHFP95}~\cite{WeilHFP95} & \href{../works/WessenCS20.pdf}{WessenCS20}~\cite{WessenCS20}\\ 
\href{../works/WessenCSFPM23.pdf}{WessenCSFPM23}~\cite{WessenCSFPM23} & \href{../}{Wikarek2019}~\cite{Wikarek2019} & \href{../works/WikarekS19.pdf}{WikarekS19}~\cite{WikarekS19} & \href{../works/WinterMMW22.pdf}{WinterMMW22}~\cite{WinterMMW22} & \href{../works/Wolf03.pdf}{Wolf03}~\cite{Wolf03} & \href{../works/Wolf05.pdf}{Wolf05}~\cite{Wolf05}\\ 
\href{../works/Wolf09.pdf}{Wolf09}~\cite{Wolf09} & \href{../works/Wolf11.pdf}{Wolf11}~\cite{Wolf11} & \href{../works/WolfS05.pdf}{WolfS05}~\cite{WolfS05} & \href{../works/WolfS05a.pdf}{WolfS05a}~\cite{WolfS05a} & \href{../works/WolinskiKG04.pdf}{WolinskiKG04}~\cite{WolinskiKG04} & \href{../}{Wu2008}~\cite{Wu2008}\\ 
\href{../works/WuBB05.pdf}{WuBB05}~\cite{WuBB05} & \href{../works/WuBB09.pdf}{WuBB09}~\cite{WuBB09} & \href{../}{Xia2021}~\cite{Xia2021} & \href{../works/Xidias2019.pdf}{Xidias2019}~\cite{Xidias2019} & \href{../works/Xing2006.pdf}{Xing2006}~\cite{Xing2006} & \href{../works/Xu2023.pdf}{Xu2023}~\cite{Xu2023}\\ 
\href{../works/Xujun2009.pdf}{Xujun2009}~\cite{Xujun2009} & \href{../works/Yan2003.pdf}{Yan2003}~\cite{Yan2003} & \href{../works/Yang2000.pdf}{Yang2000}~\cite{Yang2000} & \href{../works/Yang2009.pdf}{Yang2009}~\cite{Yang2009} & \href{../works/YangSS19.pdf}{YangSS19}~\cite{YangSS19} & \href{../works/Yasmin2017.pdf}{Yasmin2017}~\cite{Yasmin2017}\\ 
\href{../}{YeGMH94}~\cite{YeGMH94} & \href{../works/Yin2007.pdf}{Yin2007}~\cite{Yin2007} & \href{../works/YoshikawaKNW94.pdf}{YoshikawaKNW94}~\cite{YoshikawaKNW94} & \href{../works/Younes2003.pdf}{Younes2003}~\cite{Younes2003} & \href{../works/YounespourAKE19.pdf}{YounespourAKE19}~\cite{YounespourAKE19} & \href{../works/YoungFS17.pdf}{YoungFS17}~\cite{YoungFS17}\\ 
\href{../works/YunG02.pdf}{YunG02}~\cite{YunG02} & \href{../}{Yunes2005}~\cite{Yunes2005} & \href{../works/YunusogluY22.pdf}{YunusogluY22}~\cite{YunusogluY22} & \href{../works/YuraszeckMC23.pdf}{YuraszeckMC23}~\cite{YuraszeckMC23} & \href{../works/YuraszeckMCCR23.pdf}{YuraszeckMCCR23}~\cite{YuraszeckMCCR23} & \href{../works/YuraszeckMPV22.pdf}{YuraszeckMPV22}~\cite{YuraszeckMPV22}\\ 
\href{../works/Yvars2018.pdf}{Yvars2018}~\cite{Yvars2018} & \href{../works/Zahout21.pdf}{Zahout21}~\cite{Zahout21} & \href{../works/ZampelliVSDR13.pdf}{ZampelliVSDR13}~\cite{ZampelliVSDR13} & \href{../works/ZarandiASC20.pdf}{ZarandiASC20}~\cite{ZarandiASC20} & \href{../}{ZarandiB12}~\cite{ZarandiB12} & \href{../works/ZarandiKS16.pdf}{ZarandiKS16}~\cite{ZarandiKS16}\\ 
\href{../works/Zeballos10.pdf}{Zeballos10}~\cite{Zeballos10} & \href{../}{Zeballos2006}~\cite{Zeballos2006} & \href{../works/ZeballosCM10.pdf}{ZeballosCM10}~\cite{ZeballosCM10} & \href{../works/ZeballosH05.pdf}{ZeballosH05}~\cite{ZeballosH05} & \href{../works/ZeballosM09.pdf}{ZeballosM09}~\cite{ZeballosM09} & \href{../works/ZeballosNH11.pdf}{ZeballosNH11}~\cite{ZeballosNH11}\\ 
\href{../works/ZeballosQH10.pdf}{ZeballosQH10}~\cite{ZeballosQH10} & \href{../works/ZengM12.pdf}{ZengM12}~\cite{ZengM12} & \href{../works/Zhang2005.pdf}{Zhang2005}~\cite{Zhang2005} & \href{../works/Zhang2013.pdf}{Zhang2013}~\cite{Zhang2013} & \href{../}{Zhang2019}~\cite{Zhang2019} & \href{../works/ZhangBB22.pdf}{ZhangBB22}~\cite{ZhangBB22}\\ 
\href{../works/ZhangJZL22.pdf}{ZhangJZL22}~\cite{ZhangJZL22} & \href{../works/ZhangLS12.pdf}{ZhangLS12}~\cite{ZhangLS12} & \href{../works/ZhangW18.pdf}{ZhangW18}~\cite{ZhangW18} & \href{../works/ZhangYW21.pdf}{ZhangYW21}~\cite{ZhangYW21} & \href{../works/ZhaoL14.pdf}{ZhaoL14}~\cite{ZhaoL14} & \href{../works/Zhou96.pdf}{Zhou96}~\cite{Zhou96}\\ 
\href{../works/Zhou97.pdf}{Zhou97}~\cite{Zhou97} & \href{../works/ZhouGL15.pdf}{ZhouGL15}~\cite{ZhouGL15} & \href{../}{Zhu2006}~\cite{Zhu2006} & \href{../works/ZhuS02.pdf}{ZhuS02}~\cite{ZhuS02} & \href{../works/ZhuSZW23.pdf}{ZhuSZW23}~\cite{ZhuSZW23} & \href{../works/ZibranR11.pdf}{ZibranR11}~\cite{ZibranR11}\\ 
\href{../works/ZibranR11a.pdf}{ZibranR11a}~\cite{ZibranR11a} & \href{../}{Zohali2022}~\cite{Zohali2022} & \href{../works/Zou2012.pdf}{Zou2012}~\cite{Zou2012} & \href{../works/Zou2021.pdf}{Zou2021}~\cite{Zou2021} & \href{../works/ZouZ20.pdf}{ZouZ20}~\cite{ZouZ20} & \href{../}{Zoulfaghari2013}~\cite{Zoulfaghari2013}\\ 
\href{../works/Zuenko2021.pdf}{Zuenko2021}~\cite{Zuenko2021} & \href{../works/abs-0907-0939.pdf}{abs-0907-0939}~\cite{abs-0907-0939} & \href{../works/abs-1009-0347.pdf}{abs-1009-0347}~\cite{abs-1009-0347} & \href{../works/abs-1901-07914.pdf}{abs-1901-07914}~\cite{abs-1901-07914} & \href{../works/abs-1902-01193.pdf}{abs-1902-01193}~\cite{abs-1902-01193} & \href{../works/abs-1902-09244.pdf}{abs-1902-09244}~\cite{abs-1902-09244}\\ 
\href{../works/abs-1911-04766.pdf}{abs-1911-04766}~\cite{abs-1911-04766} & \href{../works/abs-2102-08778.pdf}{abs-2102-08778}~\cite{abs-2102-08778} & \href{../works/abs-2211-14492.pdf}{abs-2211-14492}~\cite{abs-2211-14492} & \href{../works/abs-2305-19888.pdf}{abs-2305-19888}~\cite{abs-2305-19888} & \href{../works/abs-2306-05747.pdf}{abs-2306-05747}~\cite{abs-2306-05747} & \href{../works/abs-2312-13682.pdf}{abs-2312-13682}~\cite{abs-2312-13682}\\ 
\href{../works/abs-2402-00459.pdf}{abs-2402-00459}~\cite{abs-2402-00459} & \end{longtable}


\section{Conference Paper List}

This section presents the information for all conference papers included in the survey. For space reasons, not all information about the papers can be presented in a single table, we therefore split the data into three parts. The first part contains the main bibliographical information for the paper. The paper are sorted by year of publication (newest first), and then alphabetically by key. 

The key contains a hyperlink to the original source URL of the paper. You may have to navigate manually to download the actual paper content, and you may be unable to access the paper completely if it is behind a paywall for which you (or your organization) do not have access.

We then list the authors of the paper, in the other given in the bibtex file, abbreviating first names for space where we can identify them. Note that names with non-latin characters are not handled by latex. We use the form that is given in the bibtex file, but have excluded entries that cause latex to fail.  

We then give the title of the publication, using the original capitalization of the title entry in the bibtex entry, which may differ from the format shown in the bibliography. We then (column LC) provide a link to a local copy, if it is present, and a link to the bibliography entry of the paper.  We also show the year of publication, and the conference where the paper was published, using a short form abbreviation of the conference. This relies on a matching routine in the Java code to find the short title, new conference series may require an additional entry in \texttt{ImportBibtex.java} to work properly. Finally we list the number of pages of the paper, this information is using the bibtex entry where possible, otherwise uses \texttt{pdfinfo} to extract the actual number of pages from the local copy. The final columns b and c provide links to the corresponding tables of extracted concepts and manual information. Note that the links to typically show the correct page, not do not necessarily scroll to the correct line in the table.

\clearpage
\subsection{Papers from bibtex}
{\scriptsize
\begin{longtable}{>{\raggedright\arraybackslash}p{3cm}>{\raggedright\arraybackslash}p{6cm}>{\raggedright\arraybackslash}p{6.5cm}rrrp{2.5cm}rrrrr}
\rowcolor{white}\caption{Works from bibtex (Total 320)}\\ \toprule
\rowcolor{white}Key & Authors & Title & LC & Cite & Year & \shortstack{Conference\\/Journal} & Pages & \shortstack{Nr\\Cites} & \shortstack{Nr\\Refs} & b & c \\ \midrule\endhead
\bottomrule
\endfoot
\rowlabel{a:AalianPG23}AalianPG23 \href{https://doi.org/10.4230/LIPIcs.CP.2023.6}{AalianPG23} & \hyperref[auth:a7]{Y. Aalian}, \hyperref[auth:a8]{G. Pesant}, \hyperref[auth:a9]{M. Gamache} & Optimization of Short-Term Underground Mine Planning Using Constraint Programming & \href{works/AalianPG23.pdf}{Yes} & \cite{AalianPG23} & 2023 & CP 2023 & 16 & 0 & 0 & \ref{b:AalianPG23} & \ref{c:AalianPG23}\\
\rowlabel{a:Bit-Monnot23}Bit-Monnot23 \href{https://doi.org/10.3233/FAIA230278}{Bit-Monnot23} & \hyperref[auth:a398]{A. Bit{-}Monnot} & Enhancing Hybrid {CP-SAT} Search for Disjunctive Scheduling & \href{works/Bit-Monnot23.pdf}{Yes} & \cite{Bit-Monnot23} & 2023 & ECAI 2023 & 8 & 0 & 0 & \ref{b:Bit-Monnot23} & \ref{c:Bit-Monnot23}\\
\rowlabel{a:EfthymiouY23}EfthymiouY23 \href{https://doi.org/10.1007/978-3-031-33271-5\_16}{EfthymiouY23} & \hyperref[auth:a18]{N. Efthymiou}, \hyperref[auth:a19]{N. Yorke{-}Smith} & Predicting the Optimal Period for Cyclic Hoist Scheduling Problems & \href{works/EfthymiouY23.pdf}{Yes} & \cite{EfthymiouY23} & 2023 & CPAIOR 2023 & 16 & 0 & 23 & \ref{b:EfthymiouY23} & \ref{c:EfthymiouY23}\\
\rowlabel{a:JuvinHHL23}JuvinHHL23 \href{https://doi.org/10.4230/LIPIcs.CP.2023.19}{JuvinHHL23} & \hyperref[auth:a0]{C. Juvin}, \hyperref[auth:a1]{E. Hebrard}, \hyperref[auth:a2]{L. Houssin}, \hyperref[auth:a3]{P. Lopez} & An Efficient Constraint Programming Approach to Preemptive Job Shop Scheduling & \href{works/JuvinHHL23.pdf}{Yes} & \cite{JuvinHHL23} & 2023 & CP 2023 & 16 & 0 & 0 & \ref{b:JuvinHHL23} & \ref{c:JuvinHHL23}\\
\rowlabel{a:JuvinHL23}JuvinHL23 \href{https://doi.org/10.1007/978-3-031-33271-5\_23}{JuvinHL23} & \hyperref[auth:a0]{C. Juvin}, \hyperref[auth:a2]{L. Houssin}, \hyperref[auth:a3]{P. Lopez} & Constraint Programming for the Robust Two-Machine Flow-Shop Scheduling Problem with Budgeted Uncertainty & \href{works/JuvinHL23.pdf}{Yes} & \cite{JuvinHL23} & 2023 & CPAIOR 2023 & 16 & 0 & 11 & \ref{b:JuvinHL23} & \ref{c:JuvinHL23}\\
\rowlabel{a:KameugneFND23}KameugneFND23 \href{https://doi.org/10.4230/LIPIcs.CP.2023.20}{KameugneFND23} & \hyperref[auth:a10]{R. Kameugne}, \hyperref[auth:a11]{S{\'{e}}v{\'{e}}rine Betmbe Fetgo}, \hyperref[auth:a12]{T. Noulamo}, \hyperref[auth:a13]{Cl{\'{e}}mentin Tayou Djam{\'{e}}gni} & Horizontally Elastic Edge Finder Rule for Cumulative Constraint Based on Slack and Density & \href{works/KameugneFND23.pdf}{Yes} & \cite{KameugneFND23} & 2023 & CP 2023 & 17 & 0 & 0 & \ref{b:KameugneFND23} & \ref{c:KameugneFND23}\\
\rowlabel{a:KimCMLLP23}KimCMLLP23 \href{https://doi.org/10.1007/978-3-031-33271-5\_31}{KimCMLLP23} & \hyperref[auth:a23]{D. Kim}, \hyperref[auth:a24]{Y. Choi}, \hyperref[auth:a25]{K. Moon}, \hyperref[auth:a26]{M. Lee}, \hyperref[auth:a27]{K. Lee}, \hyperref[auth:a28]{Michael L. Pinedo} & Iterated Greedy Constraint Programming for Scheduling Steelmaking Continuous Casting & \href{works/KimCMLLP23.pdf}{Yes} & \cite{KimCMLLP23} & 2023 & CPAIOR 2023 & 16 & 0 & 13 & \ref{b:KimCMLLP23} & \ref{c:KimCMLLP23}\\
\rowlabel{a:Mehdizadeh-Somarin23}Mehdizadeh-Somarin23 \href{https://doi.org/10.1007/978-3-031-43670-3\_33}{Mehdizadeh-Somarin23} & \hyperref[auth:a435]{Z. Mehdizadeh{-}Somarin}, \hyperref[auth:a436]{R. Tavakkoli{-}Moghaddam}, \hyperref[auth:a437]{M. Rohaninejad}, \hyperref[auth:a116]{Z. Hanz{\'{a}}lek}, \hyperref[auth:a438]{Behdin Vahedi Nouri} & A Constraint Programming Model for a Reconfigurable Job Shop Scheduling Problem with Machine Availability & \href{works/Mehdizadeh-Somarin23.pdf}{Yes} & \cite{Mehdizadeh-Somarin23} & 2023 & APMS 2023 & 14 & 0 & 0 & \ref{b:Mehdizadeh-Somarin23} & \ref{c:Mehdizadeh-Somarin23}\\
\rowlabel{a:PerezGSL23}PerezGSL23 \href{https://doi.org/10.1109/ICTAI59109.2023.00108}{PerezGSL23} & \hyperref[auth:a431]{G. Perez}, \hyperref[auth:a432]{G. Glorian}, \hyperref[auth:a433]{W. Suijlen}, \hyperref[auth:a434]{A. Lallouet} & A Constraint Programming Model for Scheduling the Unloading of Trains in Ports & \href{works/PerezGSL23.pdf}{Yes} & \cite{PerezGSL23} & 2023 & ICTAI 2023 & 7 & 0 & 0 & \ref{b:PerezGSL23} & \ref{c:PerezGSL23}\\
\rowlabel{a:PovedaAA23}PovedaAA23 \href{https://doi.org/10.4230/LIPIcs.CP.2023.31}{PovedaAA23} & \hyperref[auth:a4]{G. Pov{\'{e}}da}, \hyperref[auth:a5]{N. {\'{A}}lvarez}, \hyperref[auth:a6]{C. Artigues} & Partially Preemptive Multi Skill/Mode Resource-Constrained Project Scheduling with Generalized Precedence Relations and Calendars & \href{works/PovedaAA23.pdf}{Yes} & \cite{PovedaAA23} & 2023 & CP 2023 & 21 & 0 & 0 & \ref{b:PovedaAA23} & \ref{c:PovedaAA23}\\
\rowlabel{a:SquillaciPR23}SquillaciPR23 \href{https://doi.org/10.1007/978-3-031-33271-5\_29}{SquillaciPR23} & \hyperref[auth:a20]{S. Squillaci}, \hyperref[auth:a21]{C. Pralet}, \hyperref[auth:a22]{S. Roussel} & Scheduling Complex Observation Requests for a Constellation of Satellites: Large Neighborhood Search Approaches & \href{works/SquillaciPR23.pdf}{Yes} & \cite{SquillaciPR23} & 2023 & CPAIOR 2023 & 17 & 0 & 19 & \ref{b:SquillaciPR23} & \ref{c:SquillaciPR23}\\
\rowlabel{a:TardivoDFMP23}TardivoDFMP23 \href{https://doi.org/10.1007/978-3-031-33271-5\_22}{TardivoDFMP23} & \hyperref[auth:a29]{F. Tardivo}, \hyperref[auth:a30]{A. Dovier}, \hyperref[auth:a31]{A. Formisano}, \hyperref[auth:a32]{L. Michel}, \hyperref[auth:a33]{E. Pontelli} & Constraint Propagation on {GPU:} {A} Case Study for the Cumulative Constraint & \href{works/TardivoDFMP23.pdf}{Yes} & \cite{TardivoDFMP23} & 2023 & CPAIOR 2023 & 18 & 0 & 30 & \ref{b:TardivoDFMP23} & \ref{c:TardivoDFMP23}\\
\rowlabel{a:TasselGS23}TasselGS23 \href{https://doi.org/10.1609/icaps.v33i1.27243}{TasselGS23} & \hyperref[auth:a58]{P. Tassel}, \hyperref[auth:a61]{M. Gebser}, \hyperref[auth:a429]{K. Schekotihin} & An End-to-End Reinforcement Learning Approach for Job-Shop Scheduling Problems Based on Constraint Programming & \href{works/TasselGS23.pdf}{Yes} & \cite{TasselGS23} & 2023 & ICAPS 2023 & 9 & 0 & 0 & \ref{b:TasselGS23} & \ref{c:TasselGS23}\\
\rowlabel{a:WangB23}WangB23 \href{https://doi.org/10.1109/ICTAI59109.2023.00062}{WangB23} & \hyperref[auth:a399]{R. Wang}, \hyperref[auth:a400]{N. Barnier} & Dynamic All-Different and Maximal Cliques Constraints for Fixed Job Scheduling & \href{works/WangB23.pdf}{Yes} & \cite{WangB23} & 2023 & ICTAI 2023 & 8 & 0 & 0 & \ref{b:WangB23} & \ref{c:WangB23}\\
\rowlabel{a:YuraszeckMC23}YuraszeckMC23 \href{https://doi.org/10.1016/j.procs.2023.03.130}{YuraszeckMC23} & \hyperref[auth:a411]{F. Yuraszeck}, \hyperref[auth:a430]{G. Mej{\'{\i}}a}, \hyperref[auth:a413]{D. Canut{-}de{-}Bon} & A competitive constraint programming approach for the group shop scheduling problem & \href{works/YuraszeckMC23.pdf}{Yes} & \cite{YuraszeckMC23} & 2023 & ANT 2023 & 6 & 1 & 15 & \ref{b:YuraszeckMC23} & \ref{c:YuraszeckMC23}\\
\rowlabel{a:ArmstrongGOS22}ArmstrongGOS22 \href{https://doi.org/10.1007/978-3-031-08011-1\_1}{ArmstrongGOS22} & \hyperref[auth:a14]{E. Armstrong}, \hyperref[auth:a15]{M. Garraffa}, \hyperref[auth:a16]{B. O'Sullivan}, \hyperref[auth:a17]{H. Simonis} & A Two-Phase Hybrid Approach for the Hybrid Flexible Flowshop with Transportation Times & \href{works/ArmstrongGOS22.pdf}{Yes} & \cite{ArmstrongGOS22} & 2022 & CPAIOR 2022 & 13 & 0 & 14 & \ref{b:ArmstrongGOS22} & \ref{c:ArmstrongGOS22}\\
\rowlabel{a:BoudreaultSLQ22}BoudreaultSLQ22 \href{https://doi.org/10.4230/LIPIcs.CP.2022.10}{BoudreaultSLQ22} & \hyperref[auth:a34]{R. Boudreault}, \hyperref[auth:a35]{V. Simard}, \hyperref[auth:a36]{D. Lafond}, \hyperref[auth:a37]{C. Quimper} & A Constraint Programming Approach to Ship Refit Project Scheduling & \href{works/BoudreaultSLQ22.pdf}{Yes} & \cite{BoudreaultSLQ22} & 2022 & CP 2022 & 16 & 0 & 0 & \ref{b:BoudreaultSLQ22} & \ref{c:BoudreaultSLQ22}\\
\rowlabel{a:GeitzGSSW22}GeitzGSSW22 \href{https://doi.org/10.1007/978-3-031-08011-1\_10}{GeitzGSSW22} & \hyperref[auth:a47]{M. Geitz}, \hyperref[auth:a48]{C. Grozea}, \hyperref[auth:a49]{W. Steigerwald}, \hyperref[auth:a50]{R. St{\"{o}}hr}, \hyperref[auth:a51]{A. Wolf} & Solving the Extended Job Shop Scheduling Problem with AGVs - Classical and Quantum Approaches & \href{works/GeitzGSSW22.pdf}{Yes} & \cite{GeitzGSSW22} & 2022 & CPAIOR 2022 & 18 & 0 & 24 & \ref{b:GeitzGSSW22} & \ref{c:GeitzGSSW22}\\
\rowlabel{a:HebrardALLCMR22}HebrardALLCMR22 \href{https://doi.org/10.24963/ijcai.2022/643}{HebrardALLCMR22} & \hyperref[auth:a1]{E. Hebrard}, \hyperref[auth:a6]{C. Artigues}, \hyperref[auth:a3]{P. Lopez}, \hyperref[auth:a797]{A. Lusson}, \hyperref[auth:a798]{Steve A. Chien}, \hyperref[auth:a799]{A. Maillard}, \hyperref[auth:a800]{Gregg R. Rabideau} & An Efficient Approach to Data Transfer Scheduling for Long Range Space Exploration & \href{works/HebrardALLCMR22.pdf}{Yes} & \cite{HebrardALLCMR22} & 2022 & IJCAI 2022 & 7 & 0 & 0 & \ref{b:HebrardALLCMR22} & \ref{c:HebrardALLCMR22}\\
\rowlabel{a:JungblutK22}JungblutK22 \href{https://doi.org/10.1109/IPDPSW55747.2022.00025}{JungblutK22} & \hyperref[auth:a750]{P. Jungblut}, \hyperref[auth:a751]{D. Kranzlm{\"{u}}ller} & Optimal Schedules for High-Level Programming Environments on FPGAs with Constraint Programming & \href{works/JungblutK22.pdf}{Yes} & \cite{JungblutK22} & 2022 & IPDPS 2022 & 4 & 0 & 0 & \ref{b:JungblutK22} & \ref{c:JungblutK22}\\
\rowlabel{a:LiFJZLL22}LiFJZLL22 \href{https://doi.org/10.1109/ICNSC55942.2022.10004158}{LiFJZLL22} & \hyperref[auth:a467]{X. Li}, \hyperref[auth:a468]{J. Fu}, \hyperref[auth:a469]{Z. Jia}, \hyperref[auth:a470]{Z. Zhao}, \hyperref[auth:a471]{S. Li}, \hyperref[auth:a472]{S. Liu} & Constraint Programming for a Novel Integrated Optimization of Blocking Job Shop Scheduling and Variable-Speed Transfer Robot Assignment & \href{works/LiFJZLL22.pdf}{Yes} & \cite{LiFJZLL22} & 2022 & ICNSC 2022 & 6 & 0 & 31 & \ref{b:LiFJZLL22} & \ref{c:LiFJZLL22}\\
\rowlabel{a:LuoB22}LuoB22 \href{https://doi.org/10.1007/978-3-031-08011-1\_17}{LuoB22} & \hyperref[auth:a755]{Yiqing L. Luo}, \hyperref[auth:a89]{J. Christopher Beck} & Packing by Scheduling: Using Constraint Programming to Solve a Complex 2D Cutting Stock Problem & \href{works/LuoB22.pdf}{Yes} & \cite{LuoB22} & 2022 & CPAIOR 2022 & 17 & 0 & 28 & \ref{b:LuoB22} & \ref{c:LuoB22}\\
\rowlabel{a:OuelletQ22}OuelletQ22 \href{https://doi.org/10.1007/978-3-031-08011-1\_21}{OuelletQ22} & \hyperref[auth:a52]{Y. Ouellet}, \hyperref[auth:a37]{C. Quimper} & A MinCumulative Resource Constraint & \href{works/OuelletQ22.pdf}{Yes} & \cite{OuelletQ22} & 2022 & CPAIOR 2022 & 17 & 1 & 22 & \ref{b:OuelletQ22} & \ref{c:OuelletQ22}\\
\rowlabel{a:OujanaAYB22}OujanaAYB22 \href{https://doi.org/10.1109/CoDIT55151.2022.9803972}{OujanaAYB22} & \hyperref[auth:a460]{S. Oujana}, \hyperref[auth:a461]{L. Amodeo}, \hyperref[auth:a462]{F. Yalaoui}, \hyperref[auth:a463]{D. Brodart} & Solving a realistic hybrid and flexible flow shop scheduling problem through constraint programming: industrial case in a packaging company & \href{works/OujanaAYB22.pdf}{Yes} & \cite{OujanaAYB22} & 2022 & CoDIT 2022 & 6 & 1 & 21 & \ref{b:OujanaAYB22} & \ref{c:OujanaAYB22}\\
\rowlabel{a:PopovicCGNC22}PopovicCGNC22 \href{https://doi.org/10.4230/LIPIcs.CP.2022.34}{PopovicCGNC22} & \hyperref[auth:a38]{L. Popovic}, \hyperref[auth:a39]{A. C{\^{o}}t{\'{e}}}, \hyperref[auth:a40]{M. Gaha}, \hyperref[auth:a41]{F. Nguewouo}, \hyperref[auth:a42]{Q. Cappart} & Scheduling the Equipment Maintenance of an Electric Power Transmission Network Using Constraint Programming & \href{works/PopovicCGNC22.pdf}{Yes} & \cite{PopovicCGNC22} & 2022 & CP 2022 & 15 & 0 & 0 & \ref{b:PopovicCGNC22} & \ref{c:PopovicCGNC22}\\
\rowlabel{a:SvancaraB22}SvancaraB22 \href{https://doi.org/10.5220/0010869700003116}{SvancaraB22} & \hyperref[auth:a788]{J. Svancara}, \hyperref[auth:a153]{R. Bart{\'{a}}k} & Tackling Train Routing via Multi-agent Pathfinding and Constraint-based Scheduling & \href{works/SvancaraB22.pdf}{Yes} & \cite{SvancaraB22} & 2022 & ICAART 2022 & 8 & 0 & 0 & \ref{b:SvancaraB22} & \ref{c:SvancaraB22}\\
\rowlabel{a:Teppan22}Teppan22 \href{https://doi.org/10.5220/0010849900003116}{Teppan22} & \hyperref[auth:a94]{Erich Christian Teppan} & Types of Flexible Job Shop Scheduling: {A} Constraint Programming Experiment & \href{works/Teppan22.pdf}{Yes} & \cite{Teppan22} & 2022 & ICAART 2022 & 8 & 0 & 0 & \ref{b:Teppan22} & \ref{c:Teppan22}\\
\rowlabel{a:TouatBT22}TouatBT22 \href{}{TouatBT22} & \hyperref[auth:a464]{M. Touat}, \hyperref[auth:a465]{B. Benhamou}, \hyperref[auth:a466]{Fatima Benbouzid{-}Si Tayeb} & A Constraint Programming Model for the Scheduling Problem with Flexible Maintenance under Human Resource Constraints & \href{works/TouatBT22.pdf}{Yes} & \cite{TouatBT22} & 2022 & ICAART 2022 & 8 & 0 & 0 & \ref{b:TouatBT22} & \ref{c:TouatBT22}\\
\rowlabel{a:WinterMMW22}WinterMMW22 \href{https://doi.org/10.4230/LIPIcs.CP.2022.41}{WinterMMW22} & \hyperref[auth:a43]{F. Winter}, \hyperref[auth:a44]{S. Meiswinkel}, \hyperref[auth:a45]{N. Musliu}, \hyperref[auth:a46]{D. Walkiewicz} & Modeling and Solving Parallel Machine Scheduling with Contamination Constraints in the Agricultural Industry & \href{works/WinterMMW22.pdf}{Yes} & \cite{WinterMMW22} & 2022 & CP 2022 & 18 & 0 & 0 & \ref{b:WinterMMW22} & \ref{c:WinterMMW22}\\
\rowlabel{a:ZhangBB22}ZhangBB22 \href{https://ojs.aaai.org/index.php/ICAPS/article/view/19826}{ZhangBB22} & \hyperref[auth:a809]{J. Zhang}, \hyperref[auth:a810]{Giovanni Lo Bianco}, \hyperref[auth:a89]{J. Christopher Beck} & Solving Job-Shop Scheduling Problems with QUBO-Based Specialized Hardware & \href{works/ZhangBB22.pdf}{Yes} & \cite{ZhangBB22} & 2022 & ICAPS 2022 & 9 & 0 & 0 & \ref{b:ZhangBB22} & \ref{c:ZhangBB22}\\
\rowlabel{a:ZhangJZL22}ZhangJZL22 \href{https://doi.org/10.1109/ICNSC55942.2022.10004154}{ZhangJZL22} & \hyperref[auth:a473]{H. Zhang}, \hyperref[auth:a474]{Y. Ji}, \hyperref[auth:a470]{Z. Zhao}, \hyperref[auth:a472]{S. Liu} & Constraint Programming for Modeling and Solving a Hybrid Flow Shop Scheduling Problem & \href{works/ZhangJZL22.pdf}{Yes} & \cite{ZhangJZL22} & 2022 & ICNSC 2022 & 6 & 0 & 21 & \ref{b:ZhangJZL22} & \ref{c:ZhangJZL22}\\
\rowlabel{a:AntuoriHHEN21}AntuoriHHEN21 \href{https://doi.org/10.4230/LIPIcs.CP.2021.14}{AntuoriHHEN21} & \hyperref[auth:a53]{V. Antuori}, \hyperref[auth:a1]{E. Hebrard}, \hyperref[auth:a54]{M. Huguet}, \hyperref[auth:a55]{S. Essodaigui}, \hyperref[auth:a56]{A. Nguyen} & Combining Monte Carlo Tree Search and Depth First Search Methods for a Car Manufacturing Workshop Scheduling Problem & \href{works/AntuoriHHEN21.pdf}{Yes} & \cite{AntuoriHHEN21} & 2021 & CP 2021 & 16 & 0 & 0 & \ref{b:AntuoriHHEN21} & \ref{c:AntuoriHHEN21}\\
\rowlabel{a:ArmstrongGOS21}ArmstrongGOS21 \href{https://doi.org/10.4230/LIPIcs.CP.2021.16}{ArmstrongGOS21} & \hyperref[auth:a14]{E. Armstrong}, \hyperref[auth:a15]{M. Garraffa}, \hyperref[auth:a16]{B. O'Sullivan}, \hyperref[auth:a17]{H. Simonis} & The Hybrid Flexible Flowshop with Transportation Times & \href{works/ArmstrongGOS21.pdf}{Yes} & \cite{ArmstrongGOS21} & 2021 & CP 2021 & 18 & 1 & 0 & \ref{b:ArmstrongGOS21} & \ref{c:ArmstrongGOS21}\\
\rowlabel{a:ArtiguesHQT21}ArtiguesHQT21 \href{https://doi.org/10.5220/0010190101290136}{ArtiguesHQT21} & \hyperref[auth:a6]{C. Artigues}, \hyperref[auth:a1]{E. Hebrard}, \hyperref[auth:a801]{A. Quilliot}, \hyperref[auth:a802]{H. Toussaint} & Multi-Mode {RCPSP} with Safety Margin Maximization: Models and Algorithms & No & \cite{ArtiguesHQT21} & 2021 & ICORES 2021 & 8 & 0 & 0 & No & \ref{c:ArtiguesHQT21}\\
\rowlabel{a:Astrand0F21}Astrand0F21 \href{https://doi.org/10.1007/978-3-030-78230-6\_23}{Astrand0F21} & \hyperref[auth:a74]{M. {\AA}strand}, \hyperref[auth:a75]{M. Johansson}, \hyperref[auth:a76]{Hamid Reza Feyzmahdavian} & Short-Term Scheduling of Production Fleets in Underground Mines Using CP-Based {LNS} & \href{works/Astrand0F21.pdf}{Yes} & \cite{Astrand0F21} & 2021 & CPAIOR 2021 & 18 & 2 & 25 & \ref{b:Astrand0F21} & \ref{c:Astrand0F21}\\
\rowlabel{a:BenderWS21}BenderWS21 \href{https://doi.org/10.1007/978-3-030-87672-2\_37}{BenderWS21} & \hyperref[auth:a500]{T. Bender}, \hyperref[auth:a501]{D. Wittwer}, \hyperref[auth:a502]{T. Schmidt} & Applying Constraint Programming to the Multi-mode Scheduling Problem in Harvest Logistics & \href{works/BenderWS21.pdf}{Yes} & \cite{BenderWS21} & 2021 & ICCL 2021 & 16 & 1 & 16 & \ref{b:BenderWS21} & \ref{c:BenderWS21}\\
\rowlabel{a:GeibingerKKMMW21}GeibingerKKMMW21 \href{https://doi.org/10.1007/978-3-030-78230-6\_29}{GeibingerKKMMW21} & \hyperref[auth:a77]{T. Geibinger}, \hyperref[auth:a78]{L. Kletzander}, \hyperref[auth:a79]{M. Krainz}, \hyperref[auth:a80]{F. Mischek}, \hyperref[auth:a45]{N. Musliu}, \hyperref[auth:a43]{F. Winter} & Physician Scheduling During a Pandemic & \href{works/GeibingerKKMMW21.pdf}{Yes} & \cite{GeibingerKKMMW21} & 2021 & CPAIOR 2021 & 10 & 0 & 6 & \ref{b:GeibingerKKMMW21} & \ref{c:GeibingerKKMMW21}\\
\rowlabel{a:GeibingerMM21}GeibingerMM21 \href{https://doi.org/10.1609/aaai.v35i7.16789}{GeibingerMM21} & \hyperref[auth:a77]{T. Geibinger}, \hyperref[auth:a80]{F. Mischek}, \hyperref[auth:a45]{N. Musliu} & Constraint Logic Programming for Real-World Test Laboratory Scheduling & \href{works/GeibingerMM21.pdf}{Yes} & \cite{GeibingerMM21} & 2021 & AAAI 2021 & 9 & 0 & 0 & \ref{b:GeibingerMM21} & \ref{c:GeibingerMM21}\\
\rowlabel{a:HanenKP21}HanenKP21 \href{https://doi.org/10.1007/978-3-030-78230-6\_14}{HanenKP21} & \hyperref[auth:a71]{C. Hanen}, \hyperref[auth:a72]{Alix Munier Kordon}, \hyperref[auth:a73]{T. Pedersen} & Two Deadline Reduction Algorithms for Scheduling Dependent Tasks on Parallel Processors & \href{works/HanenKP21.pdf}{Yes} & \cite{HanenKP21} & 2021 & CPAIOR 2021 & 17 & 1 & 24 & \ref{b:HanenKP21} & \ref{c:HanenKP21}\\
\rowlabel{a:HillTV21}HillTV21 \href{https://doi.org/10.1007/978-3-030-78230-6\_2}{HillTV21} & \hyperref[auth:a64]{A. Hill}, \hyperref[auth:a65]{J. Ticktin}, \hyperref[auth:a66]{Thomas W. M. Vossen} & A Computational Study of Constraint Programming Approaches for Resource-Constrained Project Scheduling with Autonomous Learning Effects & \href{works/HillTV21.pdf}{Yes} & \cite{HillTV21} & 2021 & CPAIOR 2021 & 19 & 0 & 38 & \ref{b:HillTV21} & \ref{c:HillTV21}\\
\rowlabel{a:KlankeBYE21}KlankeBYE21 \href{https://doi.org/10.1007/978-3-030-78230-6\_9}{KlankeBYE21} & \hyperref[auth:a67]{C. Klanke}, \hyperref[auth:a68]{Dominik R. Bleidorn}, \hyperref[auth:a69]{V. Yfantis}, \hyperref[auth:a70]{S. Engell} & Combining Constraint Programming and Temporal Decomposition Approaches - Scheduling of an Industrial Formulation Plant & \href{works/KlankeBYE21.pdf}{Yes} & \cite{KlankeBYE21} & 2021 & CPAIOR 2021 & 16 & 3 & 13 & \ref{b:KlankeBYE21} & \ref{c:KlankeBYE21}\\
\rowlabel{a:KovacsTKSG21}KovacsTKSG21 \href{https://doi.org/10.4230/LIPIcs.CP.2021.36}{KovacsTKSG21} & \hyperref[auth:a57]{B. Kov{\'{a}}cs}, \hyperref[auth:a58]{P. Tassel}, \hyperref[auth:a59]{W. Kohlenbrein}, \hyperref[auth:a60]{P. Schrott{-}Kostwein}, \hyperref[auth:a61]{M. Gebser} & Utilizing Constraint Optimization for Industrial Machine Workload Balancing & \href{works/KovacsTKSG21.pdf}{Yes} & \cite{KovacsTKSG21} & 2021 & CP 2021 & 17 & 0 & 0 & \ref{b:KovacsTKSG21} & \ref{c:KovacsTKSG21}\\
\rowlabel{a:LacknerMMWW21}LacknerMMWW21 \href{https://doi.org/10.4230/LIPIcs.CP.2021.37}{LacknerMMWW21} & \hyperref[auth:a62]{M. Lackner}, \hyperref[auth:a63]{C. Mrkvicka}, \hyperref[auth:a45]{N. Musliu}, \hyperref[auth:a46]{D. Walkiewicz}, \hyperref[auth:a43]{F. Winter} & Minimizing Cumulative Batch Processing Time for an Industrial Oven Scheduling Problem & \href{works/LacknerMMWW21.pdf}{Yes} & \cite{LacknerMMWW21} & 2021 & CP 2021 & 18 & 0 & 0 & \ref{b:LacknerMMWW21} & \ref{c:LacknerMMWW21}\\
\rowlabel{a:AntuoriHHEN20}AntuoriHHEN20 \href{https://doi.org/10.1007/978-3-030-58475-7\_38}{AntuoriHHEN20} & \hyperref[auth:a53]{V. Antuori}, \hyperref[auth:a1]{E. Hebrard}, \hyperref[auth:a54]{M. Huguet}, \hyperref[auth:a55]{S. Essodaigui}, \hyperref[auth:a56]{A. Nguyen} & Leveraging Reinforcement Learning, Constraint Programming and Local Search: {A} Case Study in Car Manufacturing & \href{works/AntuoriHHEN20.pdf}{Yes} & \cite{AntuoriHHEN20} & 2020 & CP 2020 & 16 & 3 & 8 & \ref{b:AntuoriHHEN20} & \ref{c:AntuoriHHEN20}\\
\rowlabel{a:BarzegaranZP20}BarzegaranZP20 \href{https://doi.org/10.4230/OASIcs.Fog-IoT.2020.3}{BarzegaranZP20} & \hyperref[auth:a528]{M. Barzegaran}, \hyperref[auth:a529]{B. Zarrin}, \hyperref[auth:a530]{P. Pop} & Quality-Of-Control-Aware Scheduling of Communication in TSN-Based Fog Computing Platforms Using Constraint Programming & \href{works/BarzegaranZP20.pdf}{Yes} & \cite{BarzegaranZP20} & 2020 & Fog-IoT 2020 & 9 & 0 & 0 & \ref{b:BarzegaranZP20} & \ref{c:BarzegaranZP20}\\
\rowlabel{a:GodetLHS20}GodetLHS20 \href{https://doi.org/10.1609/aaai.v34i02.5510}{GodetLHS20} & \hyperref[auth:a478]{A. Godet}, \hyperref[auth:a247]{X. Lorca}, \hyperref[auth:a1]{E. Hebrard}, \hyperref[auth:a127]{G. Simonin} & Using Approximation within Constraint Programming to Solve the Parallel Machine Scheduling Problem with Additional Unit Resources & \href{works/GodetLHS20.pdf}{Yes} & \cite{GodetLHS20} & 2020 & AAAI 2020 & 8 & 1 & 0 & \ref{b:GodetLHS20} & \ref{c:GodetLHS20}\\
\rowlabel{a:GroleazNS20}GroleazNS20 \href{https://doi.org/10.1007/978-3-030-58475-7\_36}{GroleazNS20} & \hyperref[auth:a83]{L. Groleaz}, \hyperref[auth:a84]{Samba Ndojh Ndiaye}, \hyperref[auth:a85]{C. Solnon} & Solving the Group Cumulative Scheduling Problem with {CPO} and {ACO} & \href{works/GroleazNS20.pdf}{Yes} & \cite{GroleazNS20} & 2020 & CP 2020 & 17 & 1 & 25 & \ref{b:GroleazNS20} & \ref{c:GroleazNS20}\\
\rowlabel{a:GroleazNS20a}GroleazNS20a \href{https://doi.org/10.1145/3377930.3389818}{GroleazNS20a} & \hyperref[auth:a83]{L. Groleaz}, \hyperref[auth:a84]{Samba Ndojh Ndiaye}, \hyperref[auth:a85]{C. Solnon} & {ACO} with automatic parameter selection for a scheduling problem with a group cumulative constraint & \href{works/GroleazNS20a.pdf}{Yes} & \cite{GroleazNS20a} & 2020 & GECCO 2020 & 9 & 3 & 28 & \ref{b:GroleazNS20a} & \ref{c:GroleazNS20a}\\
\rowlabel{a:Mercier-AubinGQ20}Mercier-AubinGQ20 \href{https://doi.org/10.1007/978-3-030-58942-4\_22}{Mercier-AubinGQ20} & \hyperref[auth:a86]{A. Mercier{-}Aubin}, \hyperref[auth:a87]{J. Gaudreault}, \hyperref[auth:a37]{C. Quimper} & Leveraging Constraint Scheduling: {A} Case Study to the Textile Industry & \href{works/Mercier-AubinGQ20.pdf}{Yes} & \cite{Mercier-AubinGQ20} & 2020 & CPAIOR 2020 & 13 & 2 & 13 & \ref{b:Mercier-AubinGQ20} & \ref{c:Mercier-AubinGQ20}\\
\rowlabel{a:NattafM20}NattafM20 \href{https://doi.org/10.1007/978-3-030-58475-7\_27}{NattafM20} & \hyperref[auth:a81]{M. Nattaf}, \hyperref[auth:a82]{A. Malapert} & Filtering Rules for Flow Time Minimization in a Parallel Machine Scheduling Problem & \href{works/NattafM20.pdf}{Yes} & \cite{NattafM20} & 2020 & CP 2020 & 16 & 0 & 6 & \ref{b:NattafM20} & \ref{c:NattafM20}\\
\rowlabel{a:TangB20}TangB20 \href{https://doi.org/10.1007/978-3-030-58942-4\_28}{TangB20} & \hyperref[auth:a88]{Tanya Y. Tang}, \hyperref[auth:a89]{J. Christopher Beck} & {CP} and Hybrid Models for Two-Stage Batching and Scheduling & \href{works/TangB20.pdf}{Yes} & \cite{TangB20} & 2020 & CPAIOR 2020 & 16 & 6 & 12 & \ref{b:TangB20} & \ref{c:TangB20}\\
\rowlabel{a:WangB20}WangB20 \href{https://doi.org/10.3233/FAIA200114}{WangB20} & \hyperref[auth:a399]{R. Wang}, \hyperref[auth:a400]{N. Barnier} & Global Propagation of Transition Cost for Fixed Job Scheduling & \href{works/WangB20.pdf}{Yes} & \cite{WangB20} & 2020 & ECAI 2020 & 8 & 0 & 0 & \ref{b:WangB20} & \ref{c:WangB20}\\
\rowlabel{a:WessenCS20}WessenCS20 \href{https://doi.org/10.1007/978-3-030-58942-4\_33}{WessenCS20} & \hyperref[auth:a90]{J. Wess{\'{e}}n}, \hyperref[auth:a91]{M. Carlsson}, \hyperref[auth:a92]{C. Schulte} & Scheduling of Dual-Arm Multi-tool Assembly Robots and Workspace Layout Optimization & \href{works/WessenCS20.pdf}{Yes} & \cite{WessenCS20} & 2020 & CPAIOR 2020 & 10 & 2 & 11 & \ref{b:WessenCS20} & \ref{c:WessenCS20}\\
\rowlabel{a:BadicaBIL19}BadicaBIL19 \href{https://doi.org/10.1007/978-3-030-32258-8\_17}{BadicaBIL19} & \hyperref[auth:a504]{A. Badica}, \hyperref[auth:a505]{C. Badica}, \hyperref[auth:a506]{M. Ivanovic}, \hyperref[auth:a550]{D. Logofatu} & Exploring the Space of Block Structured Scheduling Processes Using Constraint Logic Programming & \href{works/BadicaBIL19.pdf}{Yes} & \cite{BadicaBIL19} & 2019 & IDC 2019 & 11 & 2 & 6 & \ref{b:BadicaBIL19} & \ref{c:BadicaBIL19}\\
\rowlabel{a:BehrensLM19}BehrensLM19 \href{https://doi.org/10.1109/ICRA.2019.8794022}{BehrensLM19} & \hyperref[auth:a547]{Jan Kristof Behrens}, \hyperref[auth:a548]{R. Lange}, \hyperref[auth:a549]{M. Mansouri} & A Constraint Programming Approach to Simultaneous Task Allocation and Motion Scheduling for Industrial Dual-Arm Manipulation Tasks & \href{works/BehrensLM19.pdf}{Yes} & \cite{BehrensLM19} & 2019 & ICRA 2019 & 7 & 12 & 18 & \ref{b:BehrensLM19} & \ref{c:BehrensLM19}\\
\rowlabel{a:BogaerdtW19}BogaerdtW19 \href{https://doi.org/10.1007/978-3-030-19212-9\_38}{BogaerdtW19} & \hyperref[auth:a310]{Pim van den Bogaerdt}, \hyperref[auth:a311]{Mathijs de Weerdt} & Lower Bounds for Uniform Machine Scheduling Using Decision Diagrams & \href{works/BogaerdtW19.pdf}{Yes} & \cite{BogaerdtW19} & 2019 & CPAIOR 2019 & 16 & 1 & 16 & \ref{b:BogaerdtW19} & \ref{c:BogaerdtW19}\\
\rowlabel{a:ColT19}ColT19 \href{https://doi.org/10.1007/978-3-030-30048-7\_9}{ColT19} & \hyperref[auth:a93]{Giacomo Da Col}, \hyperref[auth:a94]{Erich Christian Teppan} & Industrial Size Job Shop Scheduling Tackled by Present Day {CP} Solvers & \href{works/ColT19.pdf}{Yes} & \cite{ColT19} & 2019 & CP 2019 & 17 & 11 & 12 & \ref{b:ColT19} & \ref{c:ColT19}\\
\rowlabel{a:FrimodigS19}FrimodigS19 \href{https://doi.org/10.1007/978-3-030-30048-7\_25}{FrimodigS19} & \hyperref[auth:a95]{S. Frimodig}, \hyperref[auth:a92]{C. Schulte} & Models for Radiation Therapy Patient Scheduling & \href{works/FrimodigS19.pdf}{Yes} & \cite{FrimodigS19} & 2019 & CP 2019 & 17 & 3 & 26 & \ref{b:FrimodigS19} & \ref{c:FrimodigS19}\\
\rowlabel{a:FrohnerTR19}FrohnerTR19 \href{https://doi.org/10.1007/978-3-030-45093-9\_34}{FrohnerTR19} & \hyperref[auth:a544]{N. Frohner}, \hyperref[auth:a545]{S. Teuschl}, \hyperref[auth:a348]{G{\"{u}}nther R. Raidl} & Casual Employee Scheduling with Constraint Programming and Metaheuristics & \href{works/FrohnerTR19.pdf}{Yes} & \cite{FrohnerTR19} & 2019 & EUROCAST 2019 & 9 & 0 & 6 & \ref{b:FrohnerTR19} & \ref{c:FrohnerTR19}\\
\rowlabel{a:GalleguillosKSB19}GalleguillosKSB19 \href{https://doi.org/10.1007/978-3-030-30048-7\_26}{GalleguillosKSB19} & \hyperref[auth:a96]{C. Galleguillos}, \hyperref[auth:a97]{Z. Kiziltan}, \hyperref[auth:a98]{A. S{\^{\i}}rbu}, \hyperref[auth:a99]{{\"{O}}zalp Babaoglu} & Constraint Programming-Based Job Dispatching for Modern {HPC} Applications & \href{works/GalleguillosKSB19.pdf}{Yes} & \cite{GalleguillosKSB19} & 2019 & CP 2019 & 18 & 1 & 27 & \ref{b:GalleguillosKSB19} & \ref{c:GalleguillosKSB19}\\
\rowlabel{a:GeibingerMM19}GeibingerMM19 \href{https://doi.org/10.1007/978-3-030-19212-9\_20}{GeibingerMM19} & \hyperref[auth:a77]{T. Geibinger}, \hyperref[auth:a80]{F. Mischek}, \hyperref[auth:a45]{N. Musliu} & Investigating Constraint Programming for Real World Industrial Test Laboratory Scheduling & \href{works/GeibingerMM19.pdf}{Yes} & \cite{GeibingerMM19} & 2019 & CPAIOR 2019 & 16 & 6 & 15 & \ref{b:GeibingerMM19} & \ref{c:GeibingerMM19}\\
\rowlabel{a:KucukY19}KucukY19 \href{https://api.semanticscholar.org/CorpusID:198146161}{KucukY19} & \hyperref[auth:a772]{M. K{\"u}ç{\"u}k}, \hyperref[auth:a427]{Seyda Topaloglu Yildiz} & A Constraint Programming Approach for Agile Earth Observation Satellite Scheduling Problem & \href{works/KucukY19.pdf}{Yes} & \cite{KucukY19} & 2019 & RAST 2019 & 5 & 0 & 0 & \ref{b:KucukY19} & \ref{c:KucukY19}\\
\rowlabel{a:LiuLH19}LiuLH19 \href{https://doi.org/10.1007/978-3-030-19823-7\_19}{LiuLH19} & \hyperref[auth:a551]{K. Liu}, \hyperref[auth:a552]{S. L{\"{o}}ffler}, \hyperref[auth:a553]{P. Hofstedt} & Solving the Talent Scheduling Problem by Parallel Constraint Programming & \href{works/LiuLH19.pdf}{Yes} & \cite{LiuLH19} & 2019 & AIAI 2019 & 9 & 1 & 5 & \ref{b:LiuLH19} & \ref{c:LiuLH19}\\
\rowlabel{a:MalapertN19}MalapertN19 \href{https://doi.org/10.1007/978-3-030-19212-9\_28}{MalapertN19} & \hyperref[auth:a82]{A. Malapert}, \hyperref[auth:a81]{M. Nattaf} & A New CP-Approach for a Parallel Machine Scheduling Problem with Time Constraints on Machine Qualifications & \href{works/MalapertN19.pdf}{Yes} & \cite{MalapertN19} & 2019 & CPAIOR 2019 & 17 & 1 & 7 & \ref{b:MalapertN19} & \ref{c:MalapertN19}\\
\rowlabel{a:MurinR19}MurinR19 \href{https://doi.org/10.1007/978-3-030-30048-7\_27}{MurinR19} & \hyperref[auth:a100]{S. Mur{\'{\i}}n}, \hyperref[auth:a101]{H. Rudov{\'{a}}} & Scheduling of Mobile Robots Using Constraint Programming & \href{works/MurinR19.pdf}{Yes} & \cite{MurinR19} & 2019 & CP 2019 & 16 & 2 & 22 & \ref{b:MurinR19} & \ref{c:MurinR19}\\
\rowlabel{a:ParkUJR19}ParkUJR19 \href{https://doi.org/10.1007/978-3-030-19648-6\_15}{ParkUJR19} & \hyperref[auth:a554]{H. Park}, \hyperref[auth:a555]{J. Um}, \hyperref[auth:a556]{J. Jung}, \hyperref[auth:a557]{M. Ruskowski} & Developing a Production Scheduling System for Modular Factory Using Constraint Programming & \href{works/ParkUJR19.pdf}{Yes} & \cite{ParkUJR19} & 2019 & RAAD 2019 & 8 & 1 & 3 & \ref{b:ParkUJR19} & \ref{c:ParkUJR19}\\
\rowlabel{a:Tom19}Tom19 \href{https://doi.org/10.1109/FUZZ-IEEE.2019.8859029}{Tom19} & \hyperref[auth:a546]{M. Tom} & Fuzzy Multi-Constraint Programming Model for Weekly Meals Scheduling & \href{works/Tom19.pdf}{Yes} & \cite{Tom19} & 2019 & FUZZ-IEEE 2019 & 6 & 0 & 21 & \ref{b:Tom19} & \ref{c:Tom19}\\
\rowlabel{a:YangSS19}YangSS19 \href{https://doi.org/10.1007/978-3-030-19212-9\_42}{YangSS19} & \hyperref[auth:a312]{M. Yang}, \hyperref[auth:a125]{A. Schutt}, \hyperref[auth:a126]{Peter J. Stuckey} & Time Table Edge Finding with Energy Variables & \href{works/YangSS19.pdf}{Yes} & \cite{YangSS19} & 2019 & CPAIOR 2019 & 10 & 1 & 14 & \ref{b:YangSS19} & \ref{c:YangSS19}\\
\rowlabel{a:AntunesABDEGGOL18}AntunesABDEGGOL18 \href{https://doi.org/10.1109/ICTAI.2018.00027}{AntunesABDEGGOL18} & \hyperref[auth:a891]{M. Antunes}, \hyperref[auth:a892]{V. Armant}, \hyperref[auth:a223]{Kenneth N. Brown}, \hyperref[auth:a893]{Daniel A. Desmond}, \hyperref[auth:a894]{G. Escamocher}, \hyperref[auth:a895]{A. George}, \hyperref[auth:a183]{D. Grimes}, \hyperref[auth:a896]{M. O'Keeffe}, \hyperref[auth:a897]{Y. Lin}, \hyperref[auth:a16]{B. O'Sullivan}, \hyperref[auth:a898]{C. Ozturk}, \hyperref[auth:a899]{L. Quesada}, \hyperref[auth:a130]{M. Siala}, \hyperref[auth:a17]{H. Simonis}, \hyperref[auth:a838]{N. Wilson} & Assigning and Scheduling Service Visits in a Mixed Urban/Rural Setting & \href{works/AntunesABDEGGOL18.pdf}{Yes} & \cite{AntunesABDEGGOL18} & 2018 & ICTAI 2018 & 8 & 1 & 24 & \ref{b:AntunesABDEGGOL18} & \ref{c:AntunesABDEGGOL18}\\
\rowlabel{a:ArbaouiY18}ArbaouiY18 \href{https://doi.org/10.1007/978-3-319-75420-8\_67}{ArbaouiY18} & \hyperref[auth:a588]{T. Arbaoui}, \hyperref[auth:a462]{F. Yalaoui} & Solving the Unrelated Parallel Machine Scheduling Problem with Additional Resources Using Constraint Programming & \href{works/ArbaouiY18.pdf}{Yes} & \cite{ArbaouiY18} & 2018 & ACIIDS 2018 & 10 & 2 & 14 & \ref{b:ArbaouiY18} & \ref{c:ArbaouiY18}\\
\rowlabel{a:AstrandJZ18}AstrandJZ18 \href{https://doi.org/10.1007/978-3-319-93031-2\_44}{AstrandJZ18} & \hyperref[auth:a74]{M. {\AA}strand}, \hyperref[auth:a75]{M. Johansson}, \hyperref[auth:a205]{A. Zanarini} & Fleet Scheduling in Underground Mines Using Constraint Programming & \href{works/AstrandJZ18.pdf}{Yes} & \cite{AstrandJZ18} & 2018 & CPAIOR 2018 & 9 & 9 & 10 & \ref{b:AstrandJZ18} & \ref{c:AstrandJZ18}\\
\rowlabel{a:BenediktSMVH18}BenediktSMVH18 \href{https://doi.org/10.1007/978-3-319-93031-2\_6}{BenediktSMVH18} & \hyperref[auth:a114]{O. Benedikt}, \hyperref[auth:a313]{P. Sucha}, \hyperref[auth:a115]{I. M{\'{o}}dos}, \hyperref[auth:a314]{M. Vlk}, \hyperref[auth:a116]{Z. Hanz{\'{a}}lek} & Energy-Aware Production Scheduling with Power-Saving Modes & \href{works/BenediktSMVH18.pdf}{Yes} & \cite{BenediktSMVH18} & 2018 & CPAIOR 2018 & 10 & 2 & 12 & \ref{b:BenediktSMVH18} & \ref{c:BenediktSMVH18}\\
\rowlabel{a:CappartTSR18}CappartTSR18 \href{https://doi.org/10.1007/978-3-319-98334-9\_32}{CappartTSR18} & \hyperref[auth:a42]{Q. Cappart}, \hyperref[auth:a847]{C. Thomas}, \hyperref[auth:a148]{P. Schaus}, \hyperref[auth:a332]{L. Rousseau} & A Constraint Programming Approach for Solving Patient Transportation Problems & \href{works/CappartTSR18.pdf}{Yes} & \cite{CappartTSR18} & 2018 & CP 2018 & 17 & 6 & 31 & \ref{b:CappartTSR18} & \ref{c:CappartTSR18}\\
\rowlabel{a:DemirovicS18}DemirovicS18 \href{https://doi.org/10.1007/978-3-319-93031-2\_10}{DemirovicS18} & \hyperref[auth:a315]{E. Demirovic}, \hyperref[auth:a126]{Peter J. Stuckey} & Constraint Programming for High School Timetabling: {A} Scheduling-Based Model with Hot Starts & \href{works/DemirovicS18.pdf}{Yes} & \cite{DemirovicS18} & 2018 & CPAIOR 2018 & 18 & 4 & 16 & \ref{b:DemirovicS18} & \ref{c:DemirovicS18}\\
\rowlabel{a:He0GLW18}He0GLW18 \href{https://doi.org/10.1007/978-3-319-98334-9\_42}{He0GLW18} & \hyperref[auth:a186]{S. He}, \hyperref[auth:a117]{M. Wallace}, \hyperref[auth:a187]{G. Gange}, \hyperref[auth:a188]{A. Liebman}, \hyperref[auth:a189]{C. Wilson} & A Fast and Scalable Algorithm for Scheduling Large Numbers of Devices Under Real-Time Pricing & \href{works/He0GLW18.pdf}{Yes} & \cite{He0GLW18} & 2018 & CP 2018 & 18 & 6 & 26 & \ref{b:He0GLW18} & \ref{c:He0GLW18}\\
\rowlabel{a:HoYCLLCLC18}HoYCLLCLC18 \href{https://doi.org/10.1145/3299819.3299825}{HoYCLLCLC18} & \hyperref[auth:a589]{T. Ho}, \hyperref[auth:a590]{J. Yao}, \hyperref[auth:a591]{Y. Chang}, \hyperref[auth:a592]{F. Lai}, \hyperref[auth:a593]{J. Lai}, \hyperref[auth:a594]{S. Chu}, \hyperref[auth:a595]{W. Liao}, \hyperref[auth:a596]{H. Chiu} & A Platform for Dynamic Optimal Nurse Scheduling Based on Integer Linear Programming along with Multiple Criteria Constraints & \href{works/HoYCLLCLC18.pdf}{Yes} & \cite{HoYCLLCLC18} & 2018 & AICCC 2018 & 6 & 2 & 14 & \ref{b:HoYCLLCLC18} & \ref{c:HoYCLLCLC18}\\
\rowlabel{a:KameugneFGOQ18}KameugneFGOQ18 \href{https://doi.org/10.1007/978-3-319-93031-2\_23}{KameugneFGOQ18} & \hyperref[auth:a10]{R. Kameugne}, \hyperref[auth:a11]{S{\'{e}}v{\'{e}}rine Betmbe Fetgo}, \hyperref[auth:a316]{V. Gingras}, \hyperref[auth:a52]{Y. Ouellet}, \hyperref[auth:a37]{C. Quimper} & Horizontally Elastic Not-First/Not-Last Filtering Algorithm for Cumulative Resource Constraint & \href{works/KameugneFGOQ18.pdf}{Yes} & \cite{KameugneFGOQ18} & 2018 & CPAIOR 2018 & 17 & 1 & 12 & \ref{b:KameugneFGOQ18} & \ref{c:KameugneFGOQ18}\\
\rowlabel{a:Laborie18a}Laborie18a \href{https://doi.org/10.1007/978-3-319-93031-2\_29}{Laborie18a} & \hyperref[auth:a118]{P. Laborie} & An Update on the Comparison of MIP, {CP} and Hybrid Approaches for Mixed Resource Allocation and Scheduling & \href{works/Laborie18a.pdf}{Yes} & \cite{Laborie18a} & 2018 & CPAIOR 2018 & 9 & 18 & 10 & \ref{b:Laborie18a} & \ref{c:Laborie18a}\\
\rowlabel{a:MusliuSS18}MusliuSS18 \href{https://doi.org/10.1007/978-3-319-93031-2\_31}{MusliuSS18} & \hyperref[auth:a45]{N. Musliu}, \hyperref[auth:a125]{A. Schutt}, \hyperref[auth:a126]{Peter J. Stuckey} & Solver Independent Rotating Workforce Scheduling & \href{works/MusliuSS18.pdf}{Yes} & \cite{MusliuSS18} & 2018 & CPAIOR 2018 & 17 & 7 & 23 & \ref{b:MusliuSS18} & \ref{c:MusliuSS18}\\
\rowlabel{a:NishikawaSTT18}NishikawaSTT18 \href{https://doi.org/10.1109/CANDAR.2018.00025}{NishikawaSTT18} & \hyperref[auth:a538]{H. Nishikawa}, \hyperref[auth:a539]{K. Shimada}, \hyperref[auth:a540]{I. Taniguchi}, \hyperref[auth:a541]{H. Tomiyama} & Scheduling of Malleable Fork-Join Tasks with Constraint Programming & \href{works/NishikawaSTT18.pdf}{Yes} & \cite{NishikawaSTT18} & 2018 & CANDAR 2018 & 6 & 2 & 14 & \ref{b:NishikawaSTT18} & \ref{c:NishikawaSTT18}\\
\rowlabel{a:NishikawaSTT18a}NishikawaSTT18a \href{https://doi.org/10.1109/TENCON.2018.8650168}{NishikawaSTT18a} & \hyperref[auth:a538]{H. Nishikawa}, \hyperref[auth:a539]{K. Shimada}, \hyperref[auth:a540]{I. Taniguchi}, \hyperref[auth:a541]{H. Tomiyama} & Scheduling of Malleable Tasks Based on Constraint Programming & \href{works/NishikawaSTT18a.pdf}{Yes} & \cite{NishikawaSTT18a} & 2018 & TENCON 2018 & 6 & 1 & 9 & \ref{b:NishikawaSTT18a} & \ref{c:NishikawaSTT18a}\\
\rowlabel{a:OuelletQ18}OuelletQ18 \href{https://doi.org/10.1007/978-3-319-93031-2\_34}{OuelletQ18} & \hyperref[auth:a52]{Y. Ouellet}, \hyperref[auth:a37]{C. Quimper} & A O(n {\textbackslash}log {\^{}}2 n) Checker and O(n{\^{}}2 {\textbackslash}log n) Filtering Algorithm for the Energetic Reasoning & \href{works/OuelletQ18.pdf}{Yes} & \cite{OuelletQ18} & 2018 & CPAIOR 2018 & 18 & 6 & 16 & \ref{b:OuelletQ18} & \ref{c:OuelletQ18}\\
\rowlabel{a:RiahiNS018}RiahiNS018 \href{https://aaai.org/ocs/index.php/ICAPS/ICAPS18/paper/view/17755}{RiahiNS018} & \hyperref[auth:a394]{V. Riahi}, \hyperref[auth:a395]{M. A. Hakim Newton}, \hyperref[auth:a396]{K. Su}, \hyperref[auth:a397]{A. Sattar} & Local Search for Flowshops with Setup Times and Blocking Constraints & \href{works/RiahiNS018.pdf}{Yes} & \cite{RiahiNS018} & 2018 & ICAPS 2018 & 9 & 0 & 0 & \ref{b:RiahiNS018} & \ref{c:RiahiNS018}\\
\rowlabel{a:TanT18}TanT18 \href{http://dx.doi.org/10.1007/978-3-319-89656-4_5}{TanT18} & \hyperref[auth:a928]{Y. Tan}, \hyperref[auth:a830]{D. Terekhov} & Logic-Based Benders Decomposition for Two-Stage Flexible Flow Shop Scheduling with Unrelated Parallel Machines & \href{works/TanT18.pdf}{Yes} & \cite{TanT18} & 2018 & Canadian AI 2018 & 12 & 1 & 23 & \ref{b:TanT18} & \ref{c:TanT18}\\
\rowlabel{a:Tesch18}Tesch18 \href{https://doi.org/10.1007/978-3-319-98334-9\_41}{Tesch18} & \hyperref[auth:a185]{A. Tesch} & Improving Energetic Propagations for Cumulative Scheduling & \href{works/Tesch18.pdf}{Yes} & \cite{Tesch18} & 2018 & CP 2018 & 17 & 5 & 21 & \ref{b:Tesch18} & \ref{c:Tesch18}\\
\rowlabel{a:BofillCSV17}BofillCSV17 \href{https://doi.org/10.1007/978-3-319-66158-2\_5}{BofillCSV17} & \hyperref[auth:a190]{M. Bofill}, \hyperref[auth:a191]{J. Coll}, \hyperref[auth:a192]{J. Suy}, \hyperref[auth:a193]{M. Villaret} & An Efficient {SMT} Approach to Solve MRCPSP/max Instances with Tight Constraints on Resources & \href{works/BofillCSV17.pdf}{Yes} & \cite{BofillCSV17} & 2017 & CP 2017 & 9 & 1 & 12 & \ref{b:BofillCSV17} & \ref{c:BofillCSV17}\\
\rowlabel{a:CappartS17}CappartS17 \href{https://doi.org/10.1007/978-3-319-59776-8\_26}{CappartS17} & \hyperref[auth:a42]{Q. Cappart}, \hyperref[auth:a148]{P. Schaus} & Rescheduling Railway Traffic on Real Time Situations Using Time-Interval Variables & \href{works/CappartS17.pdf}{Yes} & \cite{CappartS17} & 2017 & CPAIOR 2017 & 16 & 2 & 28 & \ref{b:CappartS17} & \ref{c:CappartS17}\\
\rowlabel{a:CohenHB17}CohenHB17 \href{https://doi.org/10.1007/978-3-319-66263-3\_10}{CohenHB17} & \hyperref[auth:a817]{E. Cohen}, \hyperref[auth:a818]{G. Huang}, \hyperref[auth:a89]{J. Christopher Beck} & {(I} Can Get) Satisfaction: Preference-Based Scheduling for Concert-Goers at Multi-venue Music Festivals & \href{works/CohenHB17.pdf}{Yes} & \cite{CohenHB17} & 2017 & SAT 2017 & 17 & 1 & 12 & \ref{b:CohenHB17} & \ref{c:CohenHB17}\\
\rowlabel{a:GelainPRVW17}GelainPRVW17 \href{https://doi.org/10.1007/978-3-319-59776-8\_32}{GelainPRVW17} & \hyperref[auth:a317]{M. Gelain}, \hyperref[auth:a318]{Maria Silvia Pini}, \hyperref[auth:a319]{F. Rossi}, \hyperref[auth:a320]{Kristen Brent Venable}, \hyperref[auth:a279]{T. Walsh} & A Local Search Approach for Incomplete Soft Constraint Problems: Experimental Results on Meeting Scheduling Problems & \href{works/GelainPRVW17.pdf}{Yes} & \cite{GelainPRVW17} & 2017 & CPAIOR 2017 & 16 & 1 & 5 & \ref{b:GelainPRVW17} & \ref{c:GelainPRVW17}\\
\rowlabel{a:GoldwaserS17}GoldwaserS17 \href{https://doi.org/10.1007/978-3-319-66158-2\_22}{GoldwaserS17} & \hyperref[auth:a195]{A. Goldwaser}, \hyperref[auth:a125]{A. Schutt} & Optimal Torpedo Scheduling & \href{works/GoldwaserS17.pdf}{Yes} & \cite{GoldwaserS17} & 2017 & CP 2017 & 16 & 0 & 10 & \ref{b:GoldwaserS17} & \ref{c:GoldwaserS17}\\
\rowlabel{a:Hooker17}Hooker17 \href{https://doi.org/10.1007/978-3-319-66158-2\_36}{Hooker17} & \hyperref[auth:a162]{John N. Hooker} & Job Sequencing Bounds from Decision Diagrams & \href{works/Hooker17.pdf}{Yes} & \cite{Hooker17} & 2017 & CP 2017 & 14 & 6 & 24 & \ref{b:Hooker17} & \ref{c:Hooker17}\\
\rowlabel{a:KletzanderM17}KletzanderM17 \href{https://doi.org/10.1007/978-3-319-59776-8\_28}{KletzanderM17} & \hyperref[auth:a78]{L. Kletzander}, \hyperref[auth:a45]{N. Musliu} & A Multi-stage Simulated Annealing Algorithm for the Torpedo Scheduling Problem & \href{works/KletzanderM17.pdf}{Yes} & \cite{KletzanderM17} & 2017 & CPAIOR 2017 & 15 & 1 & 9 & \ref{b:KletzanderM17} & \ref{c:KletzanderM17}\\
\rowlabel{a:LiuCGM17}LiuCGM17 \href{https://doi.org/10.1007/978-3-319-66158-2\_24}{LiuCGM17} & \hyperref[auth:a196]{T. Liu}, \hyperref[auth:a197]{Roberto Di Cosmo}, \hyperref[auth:a198]{M. Gabbrielli}, \hyperref[auth:a199]{J. Mauro} & NightSplitter: {A} Scheduling Tool to Optimize (Sub)group Activities & \href{works/LiuCGM17.pdf}{Yes} & \cite{LiuCGM17} & 2017 & CP 2017 & 17 & 0 & 15 & \ref{b:LiuCGM17} & \ref{c:LiuCGM17}\\
\rowlabel{a:Madi-WambaLOBM17}Madi-WambaLOBM17 \href{https://doi.org/10.1109/ICPADS.2017.00089}{Madi-WambaLOBM17} & \hyperref[auth:a324]{G. Madi{-}Wamba}, \hyperref[auth:a723]{Y. Li}, \hyperref[auth:a724]{A. Orgerie}, \hyperref[auth:a129]{N. Beldiceanu}, \hyperref[auth:a725]{J. Menaud} & Green Energy Aware Scheduling Problem in Virtualized Datacenters & \href{works/Madi-WambaLOBM17.pdf}{Yes} & \cite{Madi-WambaLOBM17} & 2017 & ICPADS 2017 & 8 & 1 & 8 & \ref{b:Madi-WambaLOBM17} & \ref{c:Madi-WambaLOBM17}\\
\rowlabel{a:MossigeGSMC17}MossigeGSMC17 \href{https://doi.org/10.1007/978-3-319-66158-2\_25}{MossigeGSMC17} & \hyperref[auth:a200]{M. Mossige}, \hyperref[auth:a201]{A. Gotlieb}, \hyperref[auth:a202]{H. Spieker}, \hyperref[auth:a203]{H. Meling}, \hyperref[auth:a91]{M. Carlsson} & Time-Aware Test Case Execution Scheduling for Cyber-Physical Systems & \href{works/MossigeGSMC17.pdf}{Yes} & \cite{MossigeGSMC17} & 2017 & CP 2017 & 18 & 6 & 33 & \ref{b:MossigeGSMC17} & \ref{c:MossigeGSMC17}\\
\rowlabel{a:Pralet17}Pralet17 \href{https://doi.org/10.1007/978-3-319-66158-2\_16}{Pralet17} & \hyperref[auth:a21]{C. Pralet} & An Incomplete Constraint-Based System for Scheduling with Renewable Resources & \href{works/Pralet17.pdf}{Yes} & \cite{Pralet17} & 2017 & CP 2017 & 19 & 1 & 30 & \ref{b:Pralet17} & \ref{c:Pralet17}\\
\rowlabel{a:TranVNB17a}TranVNB17a \href{https://doi.org/10.24963/ijcai.2017/726}{TranVNB17a} & \hyperref[auth:a811]{Tony T. Tran}, \hyperref[auth:a816]{Tiago Stegun Vaquero}, \hyperref[auth:a210]{G. Nejat}, \hyperref[auth:a89]{J. Christopher Beck} & Robots in Retirement Homes: Applying Off-the-Shelf Planning and Scheduling to a Team of Assistive Robots (Extended Abstract) & \href{works/TranVNB17a.pdf}{Yes} & \cite{TranVNB17a} & 2017 & IJCAI 2017 & 5 & 1 & 0 & \ref{b:TranVNB17a} & \ref{c:TranVNB17a}\\
\rowlabel{a:YoungFS17}YoungFS17 \href{https://doi.org/10.1007/978-3-319-66158-2\_20}{YoungFS17} & \hyperref[auth:a194]{Kenneth D. Young}, \hyperref[auth:a155]{T. Feydy}, \hyperref[auth:a125]{A. Schutt} & Constraint Programming Applied to the Multi-Skill Project Scheduling Problem & \href{works/YoungFS17.pdf}{Yes} & \cite{YoungFS17} & 2017 & CP 2017 & 10 & 6 & 21 & \ref{b:YoungFS17} & \ref{c:YoungFS17}\\
\rowlabel{a:AmadiniGM16}AmadiniGM16 \href{http://dx.doi.org/10.1007/978-3-319-50349-3_16}{AmadiniGM16} & \hyperref[auth:a929]{R. Amadini}, \hyperref[auth:a198]{M. Gabbrielli}, \hyperref[auth:a199]{J. Mauro} & Parallelizing Constraint Solvers for Hard RCPSP Instances & \href{works/AmadiniGM16.pdf}{Yes} & \cite{AmadiniGM16} & 2016 & LION 2016 & 7 & 2 & 16 & \ref{b:AmadiniGM16} & \ref{c:AmadiniGM16}\\
\rowlabel{a:BonfiettiZLM16}BonfiettiZLM16 \href{https://doi.org/10.1007/978-3-319-44953-1\_8}{BonfiettiZLM16} & \hyperref[auth:a204]{A. Bonfietti}, \hyperref[auth:a205]{A. Zanarini}, \hyperref[auth:a143]{M. Lombardi}, \hyperref[auth:a144]{M. Milano} & The Multirate Resource Constraint & \href{works/BonfiettiZLM16.pdf}{Yes} & \cite{BonfiettiZLM16} & 2016 & CP 2016 & 17 & 0 & 11 & \ref{b:BonfiettiZLM16} & \ref{c:BonfiettiZLM16}\\
\rowlabel{a:BoothNB16}BoothNB16 \href{https://doi.org/10.1007/978-3-319-44953-1\_34}{BoothNB16} & \hyperref[auth:a209]{Kyle E. C. Booth}, \hyperref[auth:a210]{G. Nejat}, \hyperref[auth:a89]{J. Christopher Beck} & A Constraint Programming Approach to Multi-Robot Task Allocation and Scheduling in Retirement Homes & \href{works/BoothNB16.pdf}{Yes} & \cite{BoothNB16} & 2016 & CP 2016 & 17 & 21 & 24 & \ref{b:BoothNB16} & \ref{c:BoothNB16}\\
\rowlabel{a:BridiLBBM16}BridiLBBM16 \href{https://doi.org/10.3233/978-1-61499-672-9-1598}{BridiLBBM16} & \hyperref[auth:a233]{T. Bridi}, \hyperref[auth:a143]{M. Lombardi}, \hyperref[auth:a231]{A. Bartolini}, \hyperref[auth:a248]{L. Benini}, \hyperref[auth:a144]{M. Milano} & {DARDIS:} Distributed And Randomized DIspatching and Scheduling & \href{works/BridiLBBM16.pdf}{Yes} & \cite{BridiLBBM16} & 2016 & ECAI 2016 & 2 & 0 & 0 & \ref{b:BridiLBBM16} & \ref{c:BridiLBBM16}\\
\rowlabel{a:CauwelaertDMS16}CauwelaertDMS16 \href{https://doi.org/10.1007/978-3-319-44953-1\_33}{CauwelaertDMS16} & \hyperref[auth:a207]{Sascha Van Cauwelaert}, \hyperref[auth:a208]{C. Dejemeppe}, \hyperref[auth:a150]{J. Monette}, \hyperref[auth:a148]{P. Schaus} & Efficient Filtering for the Unary Resource with Family-Based Transition Times & \href{works/CauwelaertDMS16.pdf}{Yes} & \cite{CauwelaertDMS16} & 2016 & CP 2016 & 16 & 1 & 12 & \ref{b:CauwelaertDMS16} & \ref{c:CauwelaertDMS16}\\
\rowlabel{a:FontaineMH16}FontaineMH16 \href{https://doi.org/10.1007/978-3-319-33954-2\_12}{FontaineMH16} & \hyperref[auth:a321]{D. Fontaine}, \hyperref[auth:a322]{Laurent D. Michel}, \hyperref[auth:a149]{Pascal Van Hentenryck} & Parallel Composition of Scheduling Solvers & \href{works/FontaineMH16.pdf}{Yes} & \cite{FontaineMH16} & 2016 & CPAIOR 2016 & 11 & 3 & 0 & \ref{b:FontaineMH16} & \ref{c:FontaineMH16}\\
\rowlabel{a:GilesH16}GilesH16 \href{https://doi.org/10.1007/978-3-319-44953-1\_38}{GilesH16} & \hyperref[auth:a211]{K. Giles}, \hyperref[auth:a212]{Willem{-}Jan van Hoeve} & Solving a Supply-Delivery Scheduling Problem with Constraint Programming & \href{works/GilesH16.pdf}{Yes} & \cite{GilesH16} & 2016 & CP 2016 & 16 & 2 & 6 & \ref{b:GilesH16} & \ref{c:GilesH16}\\
\rowlabel{a:GingrasQ16}GingrasQ16 \href{http://www.ijcai.org/Abstract/16/440}{GingrasQ16} & \hyperref[auth:a316]{V. Gingras}, \hyperref[auth:a37]{C. Quimper} & Generalizing the Edge-Finder Rule for the Cumulative Constraint & \href{works/GingrasQ16.pdf}{Yes} & \cite{GingrasQ16} & 2016 & IJCAI 2016 & 7 & 0 & 0 & \ref{b:GingrasQ16} & \ref{c:GingrasQ16}\\
\rowlabel{a:HechingH16}HechingH16 \href{https://doi.org/10.1007/978-3-319-33954-2\_14}{HechingH16} & \hyperref[auth:a323]{Aliza R. Heching}, \hyperref[auth:a162]{John N. Hooker} & Scheduling Home Hospice Care with Logic-Based Benders Decomposition & \href{works/HechingH16.pdf}{Yes} & \cite{HechingH16} & 2016 & CPAIOR 2016 & 11 & 10 & 0 & \ref{b:HechingH16} & \ref{c:HechingH16}\\
\rowlabel{a:JelinekB16}JelinekB16 \href{https://doi.org/10.1007/978-3-319-28228-2\_1}{JelinekB16} & \hyperref[auth:a789]{J. Jel{\'{\i}}nek}, \hyperref[auth:a153]{R. Bart{\'{a}}k} & Using Constraint Logic Programming to Schedule Solar Array Operations on the International Space Station & \href{works/JelinekB16.pdf}{Yes} & \cite{JelinekB16} & 2016 & PADL 2016 & 10 & 0 & 5 & \ref{b:JelinekB16} & \ref{c:JelinekB16}\\
\rowlabel{a:LimHTB16}LimHTB16 \href{https://doi.org/10.1007/978-3-319-44953-1\_43}{LimHTB16} & \hyperref[auth:a213]{B. Lim}, \hyperref[auth:a214]{Hassan L. Hijazi}, \hyperref[auth:a215]{S. Thi{\'{e}}baux}, \hyperref[auth:a216]{Menkes van den Briel} & Online HVAC-Aware Occupancy Scheduling with Adaptive Temperature Control & \href{works/LimHTB16.pdf}{Yes} & \cite{LimHTB16} & 2016 & CP 2016 & 18 & 2 & 23 & \ref{b:LimHTB16} & \ref{c:LimHTB16}\\
\rowlabel{a:LuoVLBM16}LuoVLBM16 \href{http://www.aaai.org/ocs/index.php/KR/KR16/paper/view/12909}{LuoVLBM16} & \hyperref[auth:a825]{R. Luo}, \hyperref[auth:a826]{Richard Anthony Valenzano}, \hyperref[auth:a827]{Y. Li}, \hyperref[auth:a89]{J. Christopher Beck}, \hyperref[auth:a828]{Sheila A. McIlraith} & Using Metric Temporal Logic to Specify Scheduling Problems & \href{works/LuoVLBM16.pdf}{Yes} & \cite{LuoVLBM16} & 2016 & KR 2016 & 4 & 0 & 0 & \ref{b:LuoVLBM16} & \ref{c:LuoVLBM16}\\
\rowlabel{a:Madi-WambaB16}Madi-WambaB16 \href{https://doi.org/10.1007/978-3-319-33954-2\_18}{Madi-WambaB16} & \hyperref[auth:a324]{G. Madi{-}Wamba}, \hyperref[auth:a129]{N. Beldiceanu} & The TaskIntersection Constraint & \href{works/Madi-WambaB16.pdf}{Yes} & \cite{Madi-WambaB16} & 2016 & CPAIOR 2016 & 16 & 0 & 0 & \ref{b:Madi-WambaB16} & \ref{c:Madi-WambaB16}\\
\rowlabel{a:SchuttS16}SchuttS16 \href{https://doi.org/10.1007/978-3-319-44953-1\_28}{SchuttS16} & \hyperref[auth:a125]{A. Schutt}, \hyperref[auth:a126]{Peter J. Stuckey} & Explaining Producer/Consumer Constraints & \href{works/SchuttS16.pdf}{Yes} & \cite{SchuttS16} & 2016 & CP 2016 & 17 & 3 & 23 & \ref{b:SchuttS16} & \ref{c:SchuttS16}\\
\rowlabel{a:SzerediS16}SzerediS16 \href{https://doi.org/10.1007/978-3-319-44953-1\_31}{SzerediS16} & \hyperref[auth:a206]{R. Szeredi}, \hyperref[auth:a125]{A. Schutt} & Modelling and Solving Multi-mode Resource-Constrained Project Scheduling & \href{works/SzerediS16.pdf}{Yes} & \cite{SzerediS16} & 2016 & CP 2016 & 10 & 9 & 14 & \ref{b:SzerediS16} & \ref{c:SzerediS16}\\
\rowlabel{a:Tesch16}Tesch16 \href{https://doi.org/10.1007/978-3-319-44953-1\_32}{Tesch16} & \hyperref[auth:a185]{A. Tesch} & A Nearly Exact Propagation Algorithm for Energetic Reasoning in {\textbackslash}mathcal O(n{\^{}}2 {\textbackslash}log n) & \href{works/Tesch16.pdf}{Yes} & \cite{Tesch16} & 2016 & CP 2016 & 27 & 4 & 14 & \ref{b:Tesch16} & \ref{c:Tesch16}\\
\rowlabel{a:TranDRFWOVB16}TranDRFWOVB16 \href{https://doi.org/10.1609/socs.v7i1.18390}{TranDRFWOVB16} & \hyperref[auth:a811]{Tony T. Tran}, \hyperref[auth:a821]{M. Do}, \hyperref[auth:a822]{Eleanor Gilbert Rieffel}, \hyperref[auth:a385]{J. Frank}, \hyperref[auth:a820]{Z. Wang}, \hyperref[auth:a823]{B. O'Gorman}, \hyperref[auth:a824]{D. Venturelli}, \hyperref[auth:a89]{J. Christopher Beck} & A Hybrid Quantum-Classical Approach to Solving Scheduling Problems & \href{works/TranDRFWOVB16.pdf}{Yes} & \cite{TranDRFWOVB16} & 2016 & SOCS 2016 & 9 & 3 & 0 & \ref{b:TranDRFWOVB16} & \ref{c:TranDRFWOVB16}\\
\rowlabel{a:TranWDRFOVB16}TranWDRFOVB16 \href{http://www.aaai.org/ocs/index.php/WS/AAAIW16/paper/view/12664}{TranWDRFOVB16} & \hyperref[auth:a811]{Tony T. Tran}, \hyperref[auth:a820]{Z. Wang}, \hyperref[auth:a821]{M. Do}, \hyperref[auth:a822]{Eleanor Gilbert Rieffel}, \hyperref[auth:a385]{J. Frank}, \hyperref[auth:a823]{B. O'Gorman}, \hyperref[auth:a824]{D. Venturelli}, \hyperref[auth:a89]{J. Christopher Beck} & Explorations of Quantum-Classical Approaches to Scheduling a Mars Lander Activity Problem & \href{works/TranWDRFOVB16.pdf}{Yes} & \cite{TranWDRFOVB16} & 2016 & AAAI 2016 & 9 & 0 & 0 & \ref{b:TranWDRFOVB16} & \ref{c:TranWDRFOVB16}\\
\rowlabel{a:BartakV15}BartakV15 \href{}{BartakV15} & \hyperref[auth:a153]{R. Bart{\'{a}}k}, \hyperref[auth:a314]{M. Vlk} & Reactive Recovery from Machine Breakdown in Production Scheduling with Temporal Distance and Resource Constraints & \href{works/BartakV15.pdf}{Yes} & \cite{BartakV15} & 2015 & ICAART 2015 & 12 & 0 & 0 & \ref{b:BartakV15} & \ref{c:BartakV15}\\
\rowlabel{a:BofillGSV15}BofillGSV15 \href{https://doi.org/10.1007/978-3-319-18008-3\_5}{BofillGSV15} & \hyperref[auth:a190]{M. Bofill}, \hyperref[auth:a235]{M. Garcia}, \hyperref[auth:a192]{J. Suy}, \hyperref[auth:a193]{M. Villaret} & MaxSAT-Based Scheduling of {B2B} Meetings & \href{works/BofillGSV15.pdf}{Yes} & \cite{BofillGSV15} & 2015 & CPAIOR 2015 & 9 & 7 & 8 & \ref{b:BofillGSV15} & \ref{c:BofillGSV15}\\
\rowlabel{a:BurtLPS15}BurtLPS15 \href{https://doi.org/10.1007/978-3-319-18008-3\_7}{BurtLPS15} & \hyperref[auth:a326]{Christina N. Burt}, \hyperref[auth:a327]{N. Lipovetzky}, \hyperref[auth:a328]{Adrian R. Pearce}, \hyperref[auth:a126]{Peter J. Stuckey} & Scheduling with Fixed Maintenance, Shared Resources and Nonlinear Feedrate Constraints: {A} Mine Planning Case Study & \href{works/BurtLPS15.pdf}{Yes} & \cite{BurtLPS15} & 2015 & CPAIOR 2015 & 17 & 0 & 8 & \ref{b:BurtLPS15} & \ref{c:BurtLPS15}\\
\rowlabel{a:DejemeppeCS15}DejemeppeCS15 \href{https://doi.org/10.1007/978-3-319-23219-5\_7}{DejemeppeCS15} & \hyperref[auth:a208]{C. Dejemeppe}, \hyperref[auth:a207]{Sascha Van Cauwelaert}, \hyperref[auth:a148]{P. Schaus} & The Unary Resource with Transition Times & \href{works/DejemeppeCS15.pdf}{Yes} & \cite{DejemeppeCS15} & 2015 & CP 2015 & 16 & 5 & 11 & \ref{b:DejemeppeCS15} & \ref{c:DejemeppeCS15}\\
\rowlabel{a:EvenSH15}EvenSH15 \href{https://doi.org/10.1007/978-3-319-23219-5\_40}{EvenSH15} & \hyperref[auth:a220]{C. Even}, \hyperref[auth:a125]{A. Schutt}, \hyperref[auth:a149]{Pascal Van Hentenryck} & A Constraint Programming Approach for Non-preemptive Evacuation Scheduling & \href{works/EvenSH15.pdf}{Yes} & \cite{EvenSH15} & 2015 & CP 2015 & 18 & 3 & 12 & \ref{b:EvenSH15} & \ref{c:EvenSH15}\\
\rowlabel{a:GayHLS15}GayHLS15 \href{https://doi.org/10.1007/978-3-319-23219-5\_10}{GayHLS15} & \hyperref[auth:a217]{S. Gay}, \hyperref[auth:a218]{R. Hartert}, \hyperref[auth:a219]{C. Lecoutre}, \hyperref[auth:a148]{P. Schaus} & Conflict Ordering Search for Scheduling Problems & \href{works/GayHLS15.pdf}{Yes} & \cite{GayHLS15} & 2015 & CP 2015 & 9 & 20 & 15 & \ref{b:GayHLS15} & \ref{c:GayHLS15}\\
\rowlabel{a:GayHS15}GayHS15 \href{https://doi.org/10.1007/978-3-319-23219-5\_11}{GayHS15} & \hyperref[auth:a217]{S. Gay}, \hyperref[auth:a218]{R. Hartert}, \hyperref[auth:a148]{P. Schaus} & Simple and Scalable Time-Table Filtering for the Cumulative Constraint & \href{works/GayHS15.pdf}{Yes} & \cite{GayHS15} & 2015 & CP 2015 & 9 & 10 & 9 & \ref{b:GayHS15} & \ref{c:GayHS15}\\
\rowlabel{a:GayHS15a}GayHS15a \href{https://doi.org/10.1007/978-3-319-18008-3\_11}{GayHS15a} & \hyperref[auth:a217]{S. Gay}, \hyperref[auth:a218]{R. Hartert}, \hyperref[auth:a148]{P. Schaus} & Time-Table Disjunctive Reasoning for the Cumulative Constraint & \href{works/GayHS15a.pdf}{Yes} & \cite{GayHS15a} & 2015 & CPAIOR 2015 & 16 & 5 & 12 & \ref{b:GayHS15a} & \ref{c:GayHS15a}\\
\rowlabel{a:KreterSS15}KreterSS15 \href{https://doi.org/10.1007/978-3-319-23219-5\_19}{KreterSS15} & \hyperref[auth:a124]{S. Kreter}, \hyperref[auth:a125]{A. Schutt}, \hyperref[auth:a126]{Peter J. Stuckey} & Modeling and Solving Project Scheduling with Calendars & \href{works/KreterSS15.pdf}{Yes} & \cite{KreterSS15} & 2015 & CP 2015 & 17 & 7 & 16 & \ref{b:KreterSS15} & \ref{c:KreterSS15}\\
\rowlabel{a:LimBTBB15}LimBTBB15 \href{https://doi.org/10.1007/978-3-319-18008-3\_17}{LimBTBB15} & \hyperref[auth:a213]{B. Lim}, \hyperref[auth:a216]{Menkes van den Briel}, \hyperref[auth:a215]{S. Thi{\'{e}}baux}, \hyperref[auth:a329]{R. Bent}, \hyperref[auth:a330]{S. Backhaus} & Large Neighborhood Search for Energy Aware Meeting Scheduling in Smart Buildings & \href{works/LimBTBB15.pdf}{Yes} & \cite{LimBTBB15} & 2015 & CPAIOR 2015 & 15 & 4 & 18 & \ref{b:LimBTBB15} & \ref{c:LimBTBB15}\\
\rowlabel{a:LombardiBM15}LombardiBM15 \href{https://doi.org/10.1007/978-3-319-23219-5\_20}{LombardiBM15} & \hyperref[auth:a143]{M. Lombardi}, \hyperref[auth:a204]{A. Bonfietti}, \hyperref[auth:a144]{M. Milano} & Deterministic Estimation of the Expected Makespan of a {POS} Under Duration Uncertainty & \href{works/LombardiBM15.pdf}{Yes} & \cite{LombardiBM15} & 2015 & CP 2015 & 16 & 0 & 8 & \ref{b:LombardiBM15} & \ref{c:LombardiBM15}\\
\rowlabel{a:MelgarejoLS15}MelgarejoLS15 \href{https://doi.org/10.1007/978-3-319-18008-3\_1}{MelgarejoLS15} & \hyperref[auth:a325]{P. Aguiar{-}Melgarejo}, \hyperref[auth:a118]{P. Laborie}, \hyperref[auth:a85]{C. Solnon} & A Time-Dependent No-Overlap Constraint: Application to Urban Delivery Problems & \href{works/MelgarejoLS15.pdf}{Yes} & \cite{MelgarejoLS15} & 2015 & CPAIOR 2015 & 17 & 14 & 17 & \ref{b:MelgarejoLS15} & \ref{c:MelgarejoLS15}\\
\rowlabel{a:MurphyMB15}MurphyMB15 \href{https://doi.org/10.1007/978-3-319-23219-5\_47}{MurphyMB15} & \hyperref[auth:a221]{Se{\'{a}}n {\'{O}}g Murphy}, \hyperref[auth:a222]{O. Manzano}, \hyperref[auth:a223]{Kenneth N. Brown} & Design and Evaluation of a Constraint-Based Energy Saving and Scheduling Recommender System & \href{works/MurphyMB15.pdf}{Yes} & \cite{MurphyMB15} & 2015 & CP 2015 & 17 & 1 & 20 & \ref{b:MurphyMB15} & \ref{c:MurphyMB15}\\
\rowlabel{a:PesantRR15}PesantRR15 \href{https://doi.org/10.1007/978-3-319-18008-3\_21}{PesantRR15} & \hyperref[auth:a8]{G. Pesant}, \hyperref[auth:a331]{G. Rix}, \hyperref[auth:a332]{L. Rousseau} & A Comparative Study of {MIP} and {CP} Formulations for the {B2B} Scheduling Optimization Problem & \href{works/PesantRR15.pdf}{Yes} & \cite{PesantRR15} & 2015 & CPAIOR 2015 & 16 & 1 & 7 & \ref{b:PesantRR15} & \ref{c:PesantRR15}\\
\rowlabel{a:PraletLJ15}PraletLJ15 \href{https://doi.org/10.1007/978-3-319-23219-5\_48}{PraletLJ15} & \hyperref[auth:a21]{C. Pralet}, \hyperref[auth:a224]{S. Lemai{-}Chenevier}, \hyperref[auth:a225]{J. Jaubert} & Scheduling Running Modes of Satellite Instruments Using Constraint-Based Local Search & \href{works/PraletLJ15.pdf}{Yes} & \cite{PraletLJ15} & 2015 & CP 2015 & 16 & 0 & 8 & \ref{b:PraletLJ15} & \ref{c:PraletLJ15}\\
\rowlabel{a:SialaAH15}SialaAH15 \href{https://doi.org/10.1007/978-3-319-23219-5\_28}{SialaAH15} & \hyperref[auth:a130]{M. Siala}, \hyperref[auth:a6]{C. Artigues}, \hyperref[auth:a1]{E. Hebrard} & Two Clause Learning Approaches for Disjunctive Scheduling & \href{works/SialaAH15.pdf}{Yes} & \cite{SialaAH15} & 2015 & CP 2015 & 10 & 4 & 17 & \ref{b:SialaAH15} & \ref{c:SialaAH15}\\
\rowlabel{a:VilimLS15}VilimLS15 \href{https://doi.org/10.1007/978-3-319-18008-3\_30}{VilimLS15} & \hyperref[auth:a121]{P. Vil{\'{\i}}m}, \hyperref[auth:a118]{P. Laborie}, \hyperref[auth:a120]{P. Shaw} & Failure-Directed Search for Constraint-Based Scheduling & \href{works/VilimLS15.pdf}{Yes} & \cite{VilimLS15} & 2015 & CPAIOR 2015 & 17 & 31 & 19 & \ref{b:VilimLS15} & \ref{c:VilimLS15}\\
\rowlabel{a:ZhouGL15}ZhouGL15 \href{https://doi.org/10.1109/FSKD.2015.7382064}{ZhouGL15} & \hyperref[auth:a609]{J. Zhou}, \hyperref[auth:a610]{Y. Guo}, \hyperref[auth:a611]{G. Li} & On complex hybrid flexible flowshop scheduling problems based on constraint programming & \href{works/ZhouGL15.pdf}{Yes} & \cite{ZhouGL15} & 2015 & FSKD 2015 & 5 & 0 & 16 & \ref{b:ZhouGL15} & \ref{c:ZhouGL15}\\
\rowlabel{a:AlesioNBG14}AlesioNBG14 \href{https://doi.org/10.1007/978-3-319-10428-7\_58}{AlesioNBG14} & \hyperref[auth:a237]{Stefano {Di Alesio}}, \hyperref[auth:a238]{S. Nejati}, \hyperref[auth:a239]{Lionel C. Briand}, \hyperref[auth:a201]{A. Gotlieb} & Worst-Case Scheduling of Software Tasks - {A} Constraint Optimization Model to Support Performance Testing & \href{works/AlesioNBG14.pdf}{Yes} & \cite{AlesioNBG14} & 2014 & CP 2014 & 18 & 3 & 19 & \ref{b:AlesioNBG14} & \ref{c:AlesioNBG14}\\
\rowlabel{a:BartoliniBBLM14}BartoliniBBLM14 \href{https://doi.org/10.1007/978-3-319-10428-7\_55}{BartoliniBBLM14} & \hyperref[auth:a231]{A. Bartolini}, \hyperref[auth:a232]{A. Borghesi}, \hyperref[auth:a233]{T. Bridi}, \hyperref[auth:a143]{M. Lombardi}, \hyperref[auth:a144]{M. Milano} & Proactive Workload Dispatching on the {EURORA} Supercomputer & \href{works/BartoliniBBLM14.pdf}{Yes} & \cite{BartoliniBBLM14} & 2014 & CP 2014 & 16 & 12 & 3 & \ref{b:BartoliniBBLM14} & \ref{c:BartoliniBBLM14}\\
\rowlabel{a:BessiereHMQW14}BessiereHMQW14 \href{https://doi.org/10.1007/978-3-319-07046-9\_23}{BessiereHMQW14} & \hyperref[auth:a334]{C. Bessiere}, \hyperref[auth:a1]{E. Hebrard}, \hyperref[auth:a335]{M. M{\'{e}}nard}, \hyperref[auth:a37]{C. Quimper}, \hyperref[auth:a279]{T. Walsh} & Buffered Resource Constraint: Algorithms and Complexity & \href{works/BessiereHMQW14.pdf}{Yes} & \cite{BessiereHMQW14} & 2014 & CPAIOR 2014 & 16 & 1 & 3 & \ref{b:BessiereHMQW14} & \ref{c:BessiereHMQW14}\\
\rowlabel{a:BofillEGPSV14}BofillEGPSV14 \href{https://doi.org/10.1007/978-3-319-10428-7\_56}{BofillEGPSV14} & \hyperref[auth:a190]{M. Bofill}, \hyperref[auth:a234]{J. Espasa}, \hyperref[auth:a235]{M. Garcia}, \hyperref[auth:a236]{M. Palah{\'{\i}}}, \hyperref[auth:a192]{J. Suy}, \hyperref[auth:a193]{M. Villaret} & Scheduling {B2B} Meetings & \href{works/BofillEGPSV14.pdf}{Yes} & \cite{BofillEGPSV14} & 2014 & CP 2014 & 16 & 3 & 10 & \ref{b:BofillEGPSV14} & \ref{c:BofillEGPSV14}\\
\rowlabel{a:BonfiettiLM14}BonfiettiLM14 \href{https://doi.org/10.1007/978-3-319-07046-9\_15}{BonfiettiLM14} & \hyperref[auth:a204]{A. Bonfietti}, \hyperref[auth:a143]{M. Lombardi}, \hyperref[auth:a144]{M. Milano} & Disregarding Duration Uncertainty in Partial Order Schedules? Yes, We Can! & \href{works/BonfiettiLM14.pdf}{Yes} & \cite{BonfiettiLM14} & 2014 & CPAIOR 2014 & 16 & 3 & 12 & \ref{b:BonfiettiLM14} & \ref{c:BonfiettiLM14}\\
\rowlabel{a:DejemeppeD14}DejemeppeD14 \href{https://doi.org/10.1007/978-3-319-07046-9\_20}{DejemeppeD14} & \hyperref[auth:a208]{C. Dejemeppe}, \hyperref[auth:a152]{Y. Deville} & Continuously Degrading Resource and Interval Dependent Activity Durations in Nuclear Medicine Patient Scheduling & \href{works/DejemeppeD14.pdf}{Yes} & \cite{DejemeppeD14} & 2014 & CPAIOR 2014 & 9 & 0 & 7 & \ref{b:DejemeppeD14} & \ref{c:DejemeppeD14}\\
\rowlabel{a:DerrienP14}DerrienP14 \href{https://doi.org/10.1007/978-3-319-10428-7\_22}{DerrienP14} & \hyperref[auth:a226]{A. Derrien}, \hyperref[auth:a227]{T. Petit} & A New Characterization of Relevant Intervals for Energetic Reasoning & \href{works/DerrienP14.pdf}{Yes} & \cite{DerrienP14} & 2014 & CP 2014 & 9 & 14 & 0 & \ref{b:DerrienP14} & \ref{c:DerrienP14}\\
\rowlabel{a:DerrienPZ14}DerrienPZ14 \href{https://doi.org/10.1007/978-3-319-10428-7\_23}{DerrienPZ14} & \hyperref[auth:a226]{A. Derrien}, \hyperref[auth:a227]{T. Petit}, \hyperref[auth:a228]{S. Zampelli} & A Declarative Paradigm for Robust Cumulative Scheduling & \href{works/DerrienPZ14.pdf}{Yes} & \cite{DerrienPZ14} & 2014 & CP 2014 & 9 & 3 & 10 & \ref{b:DerrienPZ14} & \ref{c:DerrienPZ14}\\
\rowlabel{a:DoulabiRP14}DoulabiRP14 \href{https://doi.org/10.1007/978-3-319-07046-9\_32}{DoulabiRP14} & \hyperref[auth:a336]{Seyed Hossein Hashemi Doulabi}, \hyperref[auth:a332]{L. Rousseau}, \hyperref[auth:a8]{G. Pesant} & A Constraint Programming-Based Column Generation Approach for Operating Room Planning and Scheduling & \href{works/DoulabiRP14.pdf}{Yes} & \cite{DoulabiRP14} & 2014 & CPAIOR 2014 & 9 & 3 & 10 & \ref{b:DoulabiRP14} & \ref{c:DoulabiRP14}\\
\rowlabel{a:FriedrichFMRSST14}FriedrichFMRSST14 \href{https://doi.org/10.1007/978-3-319-28697-6\_23}{FriedrichFMRSST14} & \hyperref[auth:a612]{G. Friedrich}, \hyperref[auth:a613]{M. Fr{\"{u}}hst{\"{u}}ck}, \hyperref[auth:a614]{V. Mersheeva}, \hyperref[auth:a615]{A. Ryabokon}, \hyperref[auth:a616]{M. Sander}, \hyperref[auth:a617]{A. Starzacher}, \hyperref[auth:a618]{E. Teppan} & Representing Production Scheduling with Constraint Answer Set Programming & No & \cite{FriedrichFMRSST14} & 2014 & GOR 2014 & 7 & 3 & 2 & No & \ref{c:FriedrichFMRSST14}\\
\rowlabel{a:GaySS14}GaySS14 \href{https://doi.org/10.1007/978-3-319-10428-7\_59}{GaySS14} & \hyperref[auth:a217]{S. Gay}, \hyperref[auth:a148]{P. Schaus}, \hyperref[auth:a240]{Vivian De Smedt} & Continuous Casting Scheduling with Constraint Programming & \href{works/GaySS14.pdf}{Yes} & \cite{GaySS14} & 2014 & CP 2014 & 15 & 7 & 11 & \ref{b:GaySS14} & \ref{c:GaySS14}\\
\rowlabel{a:HoundjiSWD14}HoundjiSWD14 \href{https://doi.org/10.1007/978-3-319-10428-7\_29}{HoundjiSWD14} & \hyperref[auth:a229]{Vinas{\'{e}}tan Ratheil Houndji}, \hyperref[auth:a148]{P. Schaus}, \hyperref[auth:a230]{Laurence A. Wolsey}, \hyperref[auth:a152]{Y. Deville} & The StockingCost Constraint & \href{works/HoundjiSWD14.pdf}{Yes} & \cite{HoundjiSWD14} & 2014 & CP 2014 & 16 & 5 & 7 & \ref{b:HoundjiSWD14} & \ref{c:HoundjiSWD14}\\
\rowlabel{a:KoschB14}KoschB14 \href{https://doi.org/10.1007/978-3-319-07046-9\_5}{KoschB14} & \hyperref[auth:a333]{S. Kosch}, \hyperref[auth:a89]{J. Christopher Beck} & A New {MIP} Model for Parallel-Batch Scheduling with Non-identical Job Sizes & \href{works/KoschB14.pdf}{Yes} & \cite{KoschB14} & 2014 & CPAIOR 2014 & 16 & 4 & 18 & \ref{b:KoschB14} & \ref{c:KoschB14}\\
\rowlabel{a:LipovetzkyBPS14}LipovetzkyBPS14 \href{http://www.aaai.org/ocs/index.php/ICAPS/ICAPS14/paper/view/7942}{LipovetzkyBPS14} & \hyperref[auth:a327]{N. Lipovetzky}, \hyperref[auth:a326]{Christina N. Burt}, \hyperref[auth:a328]{Adrian R. Pearce}, \hyperref[auth:a126]{Peter J. Stuckey} & Planning for Mining Operations with Time and Resource Constraints & \href{works/LipovetzkyBPS14.pdf}{Yes} & \cite{LipovetzkyBPS14} & 2014 & ICAPS 2014 & 9 & 0 & 0 & \ref{b:LipovetzkyBPS14} & \ref{c:LipovetzkyBPS14}\\
\rowlabel{a:LouieVNB14}LouieVNB14 \href{https://doi.org/10.1109/ICRA.2014.6907637}{LouieVNB14} & \hyperref[auth:a831]{Wing{-}Yue Geoffrey Louie}, \hyperref[auth:a816]{Tiago Stegun Vaquero}, \hyperref[auth:a210]{G. Nejat}, \hyperref[auth:a89]{J. Christopher Beck} & An autonomous assistive robot for planning, scheduling and facilitating multi-user activities & \href{works/LouieVNB14.pdf}{Yes} & \cite{LouieVNB14} & 2014 & ICRA 2014 & 7 & 16 & 9 & \ref{b:LouieVNB14} & \ref{c:LouieVNB14}\\
\rowlabel{a:BonfiettiLM13}BonfiettiLM13 \href{http://www.aaai.org/ocs/index.php/ICAPS/ICAPS13/paper/view/6050}{BonfiettiLM13} & \hyperref[auth:a204]{A. Bonfietti}, \hyperref[auth:a143]{M. Lombardi}, \hyperref[auth:a144]{M. Milano} & De-Cycling Cyclic Scheduling Problems & \href{works/BonfiettiLM13.pdf}{Yes} & \cite{BonfiettiLM13} & 2013 & ICAPS 2013 & 5 & 0 & 0 & \ref{b:BonfiettiLM13} & \ref{c:BonfiettiLM13}\\
\rowlabel{a:ChuGNSW13}ChuGNSW13 \href{http://www.aaai.org/ocs/index.php/IJCAI/IJCAI13/paper/view/6878}{ChuGNSW13} & \hyperref[auth:a349]{G. Chu}, \hyperref[auth:a805]{S. Gaspers}, \hyperref[auth:a806]{N. Narodytska}, \hyperref[auth:a125]{A. Schutt}, \hyperref[auth:a279]{T. Walsh} & On the Complexity of Global Scheduling Constraints under Structural Restrictions & \href{works/ChuGNSW13.pdf}{Yes} & \cite{ChuGNSW13} & 2013 & IJCAI 2013 & 7 & 0 & 0 & \ref{b:ChuGNSW13} & \ref{c:ChuGNSW13}\\
\rowlabel{a:CireCH13}CireCH13 \href{https://doi.org/10.1007/978-3-642-38171-3\_22}{CireCH13} & \hyperref[auth:a159]{Andr{\'{e}} A. Cir{\'{e}}}, \hyperref[auth:a341]{E. Coban}, \hyperref[auth:a162]{John N. Hooker} & Mixed Integer Programming vs. Logic-Based Benders Decomposition for Planning and Scheduling & \href{works/CireCH13.pdf}{Yes} & \cite{CireCH13} & 2013 & CPAIOR 2013 & 7 & 3 & 23 & \ref{b:CireCH13} & \ref{c:CireCH13}\\
\rowlabel{a:GuSS13}GuSS13 \href{https://doi.org/10.1007/978-3-642-38171-3\_24}{GuSS13} & \hyperref[auth:a342]{H. Gu}, \hyperref[auth:a125]{A. Schutt}, \hyperref[auth:a126]{Peter J. Stuckey} & A Lagrangian Relaxation Based Forward-Backward Improvement Heuristic for Maximising the Net Present Value of Resource-Constrained Projects & \href{works/GuSS13.pdf}{Yes} & \cite{GuSS13} & 2013 & CPAIOR 2013 & 7 & 10 & 24 & \ref{b:GuSS13} & \ref{c:GuSS13}\\
\rowlabel{a:HeinzKB13}HeinzKB13 \href{https://doi.org/10.1007/978-3-642-38171-3\_2}{HeinzKB13} & \hyperref[auth:a134]{S. Heinz}, \hyperref[auth:a337]{W. Ku}, \hyperref[auth:a89]{J. Christopher Beck} & Recent Improvements Using Constraint Integer Programming for Resource Allocation and Scheduling & \href{works/HeinzKB13.pdf}{Yes} & \cite{HeinzKB13} & 2013 & CPAIOR 2013 & 16 & 9 & 15 & \ref{b:HeinzKB13} & \ref{c:HeinzKB13}\\
\rowlabel{a:KelarevaTK13}KelarevaTK13 \href{https://doi.org/10.1007/978-3-642-38171-3\_8}{KelarevaTK13} & \hyperref[auth:a338]{E. Kelareva}, \hyperref[auth:a339]{K. Tierney}, \hyperref[auth:a340]{P. Kilby} & {CP} Methods for Scheduling and Routing with Time-Dependent Task Costs & \href{works/KelarevaTK13.pdf}{Yes} & \cite{KelarevaTK13} & 2013 & CPAIOR 2013 & 17 & 16 & 28 & \ref{b:KelarevaTK13} & \ref{c:KelarevaTK13}\\
\rowlabel{a:LetortCB13}LetortCB13 \href{https://doi.org/10.1007/978-3-642-38171-3\_10}{LetortCB13} & \hyperref[auth:a128]{A. Letort}, \hyperref[auth:a91]{M. Carlsson}, \hyperref[auth:a129]{N. Beldiceanu} & A Synchronized Sweep Algorithm for the \emph{k-dimensional cumulative} Constraint & \href{works/LetortCB13.pdf}{Yes} & \cite{LetortCB13} & 2013 & CPAIOR 2013 & 16 & 3 & 10 & \ref{b:LetortCB13} & \ref{c:LetortCB13}\\
\rowlabel{a:LombardiM13}LombardiM13 \href{http://www.aaai.org/ocs/index.php/ICAPS/ICAPS13/paper/view/6052}{LombardiM13} & \hyperref[auth:a143]{M. Lombardi}, \hyperref[auth:a144]{M. Milano} & A Min-Flow Algorithm for Minimal Critical Set Detection in Resource Constrained Project Scheduling & \href{works/LombardiM13.pdf}{Yes} & \cite{LombardiM13} & 2013 & ICAPS 2013 & 2 & 0 & 0 & \ref{b:LombardiM13} & \ref{c:LombardiM13}\\
\rowlabel{a:OuelletQ13}OuelletQ13 \href{https://doi.org/10.1007/978-3-642-40627-0\_42}{OuelletQ13} & \hyperref[auth:a241]{P. Ouellet}, \hyperref[auth:a37]{C. Quimper} & Time-Table Extended-Edge-Finding for the Cumulative Constraint & \href{works/OuelletQ13.pdf}{Yes} & \cite{OuelletQ13} & 2013 & CP 2013 & 16 & 12 & 14 & \ref{b:OuelletQ13} & \ref{c:OuelletQ13}\\
\rowlabel{a:SchuttFS13}SchuttFS13 \href{https://doi.org/10.1007/978-3-642-40627-0\_47}{SchuttFS13} & \hyperref[auth:a125]{A. Schutt}, \hyperref[auth:a155]{T. Feydy}, \hyperref[auth:a126]{Peter J. Stuckey} & Scheduling Optional Tasks with Explanation & \href{works/SchuttFS13.pdf}{Yes} & \cite{SchuttFS13} & 2013 & CP 2013 & 17 & 10 & 20 & \ref{b:SchuttFS13} & \ref{c:SchuttFS13}\\
\rowlabel{a:SchuttFS13a}SchuttFS13a \href{https://doi.org/10.1007/978-3-642-38171-3\_16}{SchuttFS13a} & \hyperref[auth:a125]{A. Schutt}, \hyperref[auth:a155]{T. Feydy}, \hyperref[auth:a126]{Peter J. Stuckey} & Explaining Time-Table-Edge-Finding Propagation for the Cumulative Resource Constraint & \href{works/SchuttFS13a.pdf}{Yes} & \cite{SchuttFS13a} & 2013 & CPAIOR 2013 & 17 & 20 & 27 & \ref{b:SchuttFS13a} & \ref{c:SchuttFS13a}\\
\rowlabel{a:TranTDB13}TranTDB13 \href{http://www.aaai.org/ocs/index.php/ICAPS/ICAPS13/paper/view/6005}{TranTDB13} & \hyperref[auth:a811]{Tony T. Tran}, \hyperref[auth:a830]{D. Terekhov}, \hyperref[auth:a815]{Douglas G. Down}, \hyperref[auth:a89]{J. Christopher Beck} & Hybrid Queueing Theory and Scheduling Models for Dynamic Environments with Sequence-Dependent Setup Times & \href{works/TranTDB13.pdf}{Yes} & \cite{TranTDB13} & 2013 & ICAPS 2013 & 9 & 0 & 0 & \ref{b:TranTDB13} & \ref{c:TranTDB13}\\
\rowlabel{a:BillautHL12}BillautHL12 \href{https://doi.org/10.1007/978-3-642-29828-8\_5}{BillautHL12} & \hyperref[auth:a343]{J. Billaut}, \hyperref[auth:a1]{E. Hebrard}, \hyperref[auth:a3]{P. Lopez} & Complete Characterization of Near-Optimal Sequences for the Two-Machine Flow Shop Scheduling Problem & \href{works/BillautHL12.pdf}{Yes} & \cite{BillautHL12} & 2012 & CPAIOR 2012 & 15 & 1 & 19 & \ref{b:BillautHL12} & \ref{c:BillautHL12}\\
\rowlabel{a:BonfiettiLBM12}BonfiettiLBM12 \href{https://doi.org/10.1007/978-3-642-29828-8\_6}{BonfiettiLBM12} & \hyperref[auth:a204]{A. Bonfietti}, \hyperref[auth:a143]{M. Lombardi}, \hyperref[auth:a248]{L. Benini}, \hyperref[auth:a144]{M. Milano} & Global Cyclic Cumulative Constraint & \href{works/BonfiettiLBM12.pdf}{Yes} & \cite{BonfiettiLBM12} & 2012 & CPAIOR 2012 & 16 & 2 & 11 & \ref{b:BonfiettiLBM12} & \ref{c:BonfiettiLBM12}\\
\rowlabel{a:BonfiettiM12}BonfiettiM12 \href{https://ceur-ws.org/Vol-926/paper2.pdf}{BonfiettiM12} & \hyperref[auth:a204]{A. Bonfietti}, \hyperref[auth:a144]{M. Milano} & A Constraint-based Approach to Cyclic Resource-Constrained Scheduling Problem & \href{works/BonfiettiM12.pdf}{Yes} & \cite{BonfiettiM12} & 2012 & DC SIAAI 2012 & 3 & 0 & 0 & \ref{b:BonfiettiM12} & \ref{c:BonfiettiM12}\\
\rowlabel{a:GuSW12}GuSW12 \href{https://doi.org/10.1007/978-3-642-33558-7\_55}{GuSW12} & \hyperref[auth:a342]{H. Gu}, \hyperref[auth:a126]{Peter J. Stuckey}, \hyperref[auth:a156]{Mark G. Wallace} & Maximising the Net Present Value of Large Resource-Constrained Projects & \href{works/GuSW12.pdf}{Yes} & \cite{GuSW12} & 2012 & CP 2012 & 15 & 5 & 20 & \ref{b:GuSW12} & \ref{c:GuSW12}\\
\rowlabel{a:HeinzB12}HeinzB12 \href{https://doi.org/10.1007/978-3-642-29828-8\_14}{HeinzB12} & \hyperref[auth:a134]{S. Heinz}, \hyperref[auth:a89]{J. Christopher Beck} & Reconsidering Mixed Integer Programming and MIP-Based Hybrids for Scheduling & \href{works/HeinzB12.pdf}{Yes} & \cite{HeinzB12} & 2012 & CPAIOR 2012 & 17 & 8 & 21 & \ref{b:HeinzB12} & \ref{c:HeinzB12}\\
\rowlabel{a:IfrimOS12}IfrimOS12 \href{https://doi.org/10.1007/978-3-642-33558-7\_68}{IfrimOS12} & \hyperref[auth:a184]{G. Ifrim}, \hyperref[auth:a16]{B. O'Sullivan}, \hyperref[auth:a17]{H. Simonis} & Properties of Energy-Price Forecasts for Scheduling & \href{works/IfrimOS12.pdf}{Yes} & \cite{IfrimOS12} & 2012 & CP 2012 & 16 & 6 & 20 & \ref{b:IfrimOS12} & \ref{c:IfrimOS12}\\
\rowlabel{a:LetortBC12}LetortBC12 \href{https://doi.org/10.1007/978-3-642-33558-7\_33}{LetortBC12} & \hyperref[auth:a128]{A. Letort}, \hyperref[auth:a129]{N. Beldiceanu}, \hyperref[auth:a91]{M. Carlsson} & A Scalable Sweep Algorithm for the cumulative Constraint & \href{works/LetortBC12.pdf}{Yes} & \cite{LetortBC12} & 2012 & CP 2012 & 16 & 18 & 12 & \ref{b:LetortBC12} & \ref{c:LetortBC12}\\
\rowlabel{a:RendlPHPR12}RendlPHPR12 \href{https://doi.org/10.1007/978-3-642-29828-8\_22}{RendlPHPR12} & \hyperref[auth:a344]{A. Rendl}, \hyperref[auth:a345]{M. Prandtstetter}, \hyperref[auth:a346]{G. Hiermann}, \hyperref[auth:a347]{J. Puchinger}, \hyperref[auth:a348]{G{\"{u}}nther R. Raidl} & Hybrid Heuristics for Multimodal Homecare Scheduling & \href{works/RendlPHPR12.pdf}{Yes} & \cite{RendlPHPR12} & 2012 & CPAIOR 2012 & 17 & 14 & 14 & \ref{b:RendlPHPR12} & \ref{c:RendlPHPR12}\\
\rowlabel{a:SchuttCSW12}SchuttCSW12 \href{https://doi.org/10.1007/978-3-642-29828-8\_24}{SchuttCSW12} & \hyperref[auth:a125]{A. Schutt}, \hyperref[auth:a349]{G. Chu}, \hyperref[auth:a126]{Peter J. Stuckey}, \hyperref[auth:a156]{Mark G. Wallace} & Maximising the Net Present Value for Resource-Constrained Project Scheduling & \href{works/SchuttCSW12.pdf}{Yes} & \cite{SchuttCSW12} & 2012 & CPAIOR 2012 & 17 & 18 & 21 & \ref{b:SchuttCSW12} & \ref{c:SchuttCSW12}\\
\rowlabel{a:SerraNM12}SerraNM12 \href{https://doi.org/10.1007/978-3-642-33558-7\_59}{SerraNM12} & \hyperref[auth:a242]{T. Serra}, \hyperref[auth:a243]{G. Nishioka}, \hyperref[auth:a244]{Fernando J. M. Marcellino} & The Offshore Resources Scheduling Problem: Detailing a Constraint Programming Approach & \href{works/SerraNM12.pdf}{Yes} & \cite{SerraNM12} & 2012 & CP 2012 & 17 & 0 & 8 & \ref{b:SerraNM12} & \ref{c:SerraNM12}\\
\rowlabel{a:SimoninAHL12}SimoninAHL12 \href{https://doi.org/10.1007/978-3-642-33558-7\_5}{SimoninAHL12} & \hyperref[auth:a127]{G. Simonin}, \hyperref[auth:a6]{C. Artigues}, \hyperref[auth:a1]{E. Hebrard}, \hyperref[auth:a3]{P. Lopez} & Scheduling Scientific Experiments on the Rosetta/Philae Mission & \href{works/SimoninAHL12.pdf}{Yes} & \cite{SimoninAHL12} & 2012 & CP 2012 & 15 & 3 & 8 & \ref{b:SimoninAHL12} & \ref{c:SimoninAHL12}\\
\rowlabel{a:TranB12}TranB12 \href{https://doi.org/10.3233/978-1-61499-098-7-774}{TranB12} & \hyperref[auth:a811]{Tony T. Tran}, \hyperref[auth:a89]{J. Christopher Beck} & Logic-based Benders Decomposition for Alternative Resource Scheduling with Sequence Dependent Setups & \href{works/TranB12.pdf}{Yes} & \cite{TranB12} & 2012 & ECAI 2012 & 6 & 0 & 0 & \ref{b:TranB12} & \ref{c:TranB12}\\
\rowlabel{a:ZhangLS12}ZhangLS12 \href{https://doi.org/10.1109/CIT.2012.96}{ZhangLS12} & \hyperref[auth:a621]{X. Zhang}, \hyperref[auth:a622]{Z. Lv}, \hyperref[auth:a623]{X. Song} & Model and Solution for Hot Strip Rolling Scheduling Problem Based on Constraint Programming Method & \href{works/ZhangLS12.pdf}{Yes} & \cite{ZhangLS12} & 2012 & CIT 2012 & 4 & 1 & 3 & \ref{b:ZhangLS12} & \ref{c:ZhangLS12}\\
\rowlabel{a:BajestaniB11}BajestaniB11 \href{http://aaai.org/ocs/index.php/ICAPS/ICAPS11/paper/view/2680}{BajestaniB11} & \hyperref[auth:a829]{Maliheh Aramon Bajestani}, \hyperref[auth:a89]{J. Christopher Beck} & Scheduling an Aircraft Repair Shop & \href{works/BajestaniB11.pdf}{Yes} & \cite{BajestaniB11} & 2011 & ICAPS 2011 & 8 & 0 & 0 & \ref{b:BajestaniB11} & \ref{c:BajestaniB11}\\
\rowlabel{a:BonfiettiLBM11}BonfiettiLBM11 \href{https://doi.org/10.1007/978-3-642-23786-7\_12}{BonfiettiLBM11} & \hyperref[auth:a204]{A. Bonfietti}, \hyperref[auth:a143]{M. Lombardi}, \hyperref[auth:a248]{L. Benini}, \hyperref[auth:a144]{M. Milano} & A Constraint Based Approach to Cyclic {RCPSP} & \href{works/BonfiettiLBM11.pdf}{Yes} & \cite{BonfiettiLBM11} & 2011 & CP 2011 & 15 & 3 & 14 & \ref{b:BonfiettiLBM11} & \ref{c:BonfiettiLBM11}\\
\rowlabel{a:ChapadosJR11}ChapadosJR11 \href{https://doi.org/10.1007/978-3-642-21311-3\_7}{ChapadosJR11} & \hyperref[auth:a350]{N. Chapados}, \hyperref[auth:a351]{M. Joliveau}, \hyperref[auth:a332]{L. Rousseau} & Retail Store Workforce Scheduling by Expected Operating Income Maximization & \href{works/ChapadosJR11.pdf}{Yes} & \cite{ChapadosJR11} & 2011 & CPAIOR 2011 & 6 & 5 & 12 & \ref{b:ChapadosJR11} & \ref{c:ChapadosJR11}\\
\rowlabel{a:ClercqPBJ11}ClercqPBJ11 \href{https://doi.org/10.1007/978-3-642-23786-7\_20}{ClercqPBJ11} & \hyperref[auth:a249]{Alexis De Clercq}, \hyperref[auth:a227]{T. Petit}, \hyperref[auth:a129]{N. Beldiceanu}, \hyperref[auth:a250]{N. Jussien} & Filtering Algorithms for Discrete Cumulative Problems with Overloads of Resource & \href{works/ClercqPBJ11.pdf}{Yes} & \cite{ClercqPBJ11} & 2011 & CP 2011 & 16 & 3 & 11 & \ref{b:ClercqPBJ11} & \ref{c:ClercqPBJ11}\\
\rowlabel{a:EdisO11}EdisO11 \href{https://doi.org/10.1007/978-3-642-21311-3\_10}{EdisO11} & \hyperref[auth:a352]{Emrah B. Edis}, \hyperref[auth:a353]{C. Oguz} & Parallel Machine Scheduling with Additional Resources: {A} Lagrangian-Based Constraint Programming Approach & \href{works/EdisO11.pdf}{Yes} & \cite{EdisO11} & 2011 & CPAIOR 2011 & 7 & 5 & 16 & \ref{b:EdisO11} & \ref{c:EdisO11}\\
\rowlabel{a:GrimesH11}GrimesH11 \href{https://doi.org/10.1007/978-3-642-23786-7\_28}{GrimesH11} & \hyperref[auth:a183]{D. Grimes}, \hyperref[auth:a1]{E. Hebrard} & Models and Strategies for Variants of the Job Shop Scheduling Problem & \href{works/GrimesH11.pdf}{Yes} & \cite{GrimesH11} & 2011 & CP 2011 & 17 & 5 & 18 & \ref{b:GrimesH11} & \ref{c:GrimesH11}\\
\rowlabel{a:HeinzS11}HeinzS11 \href{https://doi.org/10.1007/978-3-642-20662-7\_34}{HeinzS11} & \hyperref[auth:a134]{S. Heinz}, \hyperref[auth:a135]{J. Schulz} & Explanations for the Cumulative Constraint: An Experimental Study & \href{works/HeinzS11.pdf}{Yes} & \cite{HeinzS11} & 2011 & SEA 2011 & 10 & 5 & 12 & \ref{b:HeinzS11} & \ref{c:HeinzS11}\\
\rowlabel{a:HermenierDL11}HermenierDL11 \href{https://doi.org/10.1007/978-3-642-23786-7\_5}{HermenierDL11} & \hyperref[auth:a245]{F. Hermenier}, \hyperref[auth:a246]{S. Demassey}, \hyperref[auth:a247]{X. Lorca} & Bin Repacking Scheduling in Virtualized Datacenters & \href{works/HermenierDL11.pdf}{Yes} & \cite{HermenierDL11} & 2011 & CP 2011 & 15 & 28 & 5 & \ref{b:HermenierDL11} & \ref{c:HermenierDL11}\\
\rowlabel{a:KameugneFSN11}KameugneFSN11 \href{https://doi.org/10.1007/978-3-642-23786-7\_37}{KameugneFSN11} & \hyperref[auth:a10]{R. Kameugne}, \hyperref[auth:a131]{Laure Pauline Fotso}, \hyperref[auth:a132]{Joseph D. Scott}, \hyperref[auth:a133]{Y. Ngo{-}Kateu} & A Quadratic Edge-Finding Filtering Algorithm for Cumulative Resource Constraints & \href{works/KameugneFSN11.pdf}{Yes} & \cite{KameugneFSN11} & 2011 & CP 2011 & 15 & 7 & 9 & \ref{b:KameugneFSN11} & \ref{c:KameugneFSN11}\\
\rowlabel{a:LahimerLH11}LahimerLH11 \href{https://doi.org/10.1007/978-3-642-21311-3\_12}{LahimerLH11} & \hyperref[auth:a355]{A. Lahimer}, \hyperref[auth:a3]{P. Lopez}, \hyperref[auth:a356]{M. Haouari} & Climbing Depth-Bounded Adjacent Discrepancy Search for Solving Hybrid Flow Shop Scheduling Problems with Multiprocessor Tasks & \href{works/LahimerLH11.pdf}{Yes} & \cite{LahimerLH11} & 2011 & CPAIOR 2011 & 14 & 3 & 15 & \ref{b:LahimerLH11} & \ref{c:LahimerLH11}\\
\rowlabel{a:LombardiBMB11}LombardiBMB11 \href{https://doi.org/10.1007/978-3-642-21311-3\_14}{LombardiBMB11} & \hyperref[auth:a143]{M. Lombardi}, \hyperref[auth:a204]{A. Bonfietti}, \hyperref[auth:a144]{M. Milano}, \hyperref[auth:a248]{L. Benini} & Precedence Constraint Posting for Cyclic Scheduling Problems & \href{works/LombardiBMB11.pdf}{Yes} & \cite{LombardiBMB11} & 2011 & CPAIOR 2011 & 17 & 1 & 13 & \ref{b:LombardiBMB11} & \ref{c:LombardiBMB11}\\
\rowlabel{a:SimonisH11}SimonisH11 \href{http://dx.doi.org/10.1007/978-3-642-19486-3_5}{SimonisH11} & \hyperref[auth:a17]{H. Simonis}, \hyperref[auth:a924]{T. Hadzic} & A Resource Cost Aware Cumulative & \href{works/SimonisH11.pdf}{Yes} & \cite{SimonisH11} & 2011 & CSCLP 2011 & 14 & 3 & 9 & \ref{b:SimonisH11} & \ref{c:SimonisH11}\\
\rowlabel{a:Vilim11}Vilim11 \href{https://doi.org/10.1007/978-3-642-21311-3\_22}{Vilim11} & \hyperref[auth:a121]{P. Vil{\'{\i}}m} & Timetable Edge Finding Filtering Algorithm for Discrete Cumulative Resources & \href{works/Vilim11.pdf}{Yes} & \cite{Vilim11} & 2011 & CPAIOR 2011 & 16 & 28 & 6 & \ref{b:Vilim11} & \ref{c:Vilim11}\\
\rowlabel{a:Wolf11}Wolf11 \href{http://dx.doi.org/10.1007/978-3-642-19486-3_8}{Wolf11} & \hyperref[auth:a51]{A. Wolf} & Constraint-Based Modeling and Scheduling of Clinical Pathways & \href{works/Wolf11.pdf}{Yes} & \cite{Wolf11} & 2011 & CSCLP 2011 & 17 & 5 & 19 & \ref{b:Wolf11} & \ref{c:Wolf11}\\
\rowlabel{a:ZibranR11}ZibranR11 \href{https://doi.org/10.1109/ICPC.2011.45}{ZibranR11} & \hyperref[auth:a629]{Minhaz F. Zibran}, \hyperref[auth:a630]{Chanchal K. Roy} & Conflict-Aware Optimal Scheduling of Code Clone Refactoring: {A} Constraint Programming Approach & \href{works/ZibranR11.pdf}{Yes} & \cite{ZibranR11} & 2011 & ICPC 2011 & 4 & 17 & 18 & \ref{b:ZibranR11} & \ref{c:ZibranR11}\\
\rowlabel{a:ZibranR11a}ZibranR11a \href{https://doi.org/10.1109/SCAM.2011.21}{ZibranR11a} & \hyperref[auth:a629]{Minhaz F. Zibran}, \hyperref[auth:a630]{Chanchal K. Roy} & A Constraint Programming Approach to Conflict-Aware Optimal Scheduling of Prioritized Code Clone Refactoring & \href{works/ZibranR11a.pdf}{Yes} & \cite{ZibranR11a} & 2011 & SCAM 2011 & 10 & 26 & 27 & \ref{b:ZibranR11a} & \ref{c:ZibranR11a}\\
\rowlabel{a:BertholdHLMS10}BertholdHLMS10 \href{https://doi.org/10.1007/978-3-642-13520-0\_34}{BertholdHLMS10} & \hyperref[auth:a357]{T. Berthold}, \hyperref[auth:a134]{S. Heinz}, \hyperref[auth:a358]{Marco E. L{\"{u}}bbecke}, \hyperref[auth:a359]{Rolf H. M{\"{o}}hring}, \hyperref[auth:a135]{J. Schulz} & A Constraint Integer Programming Approach for Resource-Constrained Project Scheduling & \href{works/BertholdHLMS10.pdf}{Yes} & \cite{BertholdHLMS10} & 2010 & CPAIOR 2010 & 5 & 28 & 10 & \ref{b:BertholdHLMS10} & \ref{c:BertholdHLMS10}\\
\rowlabel{a:CobanH10}CobanH10 \href{https://doi.org/10.1007/978-3-642-13520-0\_11}{CobanH10} & \hyperref[auth:a341]{E. Coban}, \hyperref[auth:a162]{John N. Hooker} & Single-Facility Scheduling over Long Time Horizons by Logic-Based Benders Decomposition & \href{works/CobanH10.pdf}{Yes} & \cite{CobanH10} & 2010 & CPAIOR 2010 & 5 & 9 & 9 & \ref{b:CobanH10} & \ref{c:CobanH10}\\
\rowlabel{a:Davenport10}Davenport10 \href{https://doi.org/10.1007/978-3-642-13520-0\_12}{Davenport10} & \hyperref[auth:a251]{Andrew J. Davenport} & Integrated Maintenance Scheduling for Semiconductor Manufacturing & \href{works/Davenport10.pdf}{Yes} & \cite{Davenport10} & 2010 & CPAIOR 2010 & 5 & 9 & 2 & \ref{b:Davenport10} & \ref{c:Davenport10}\\
\rowlabel{a:GrimesH10}GrimesH10 \href{https://doi.org/10.1007/978-3-642-13520-0\_19}{GrimesH10} & \hyperref[auth:a183]{D. Grimes}, \hyperref[auth:a1]{E. Hebrard} & Job Shop Scheduling with Setup Times and Maximal Time-Lags: {A} Simple Constraint Programming Approach & \href{works/GrimesH10.pdf}{Yes} & \cite{GrimesH10} & 2010 & CPAIOR 2010 & 15 & 13 & 20 & \ref{b:GrimesH10} & \ref{c:GrimesH10}\\
\rowlabel{a:LombardiM10}LombardiM10 \href{https://doi.org/10.1007/978-3-642-15396-9\_32}{LombardiM10} & \hyperref[auth:a143]{M. Lombardi}, \hyperref[auth:a144]{M. Milano} & Constraint Based Scheduling to Deal with Uncertain Durations and Self-Timed Execution & \href{works/LombardiM10.pdf}{Yes} & \cite{LombardiM10} & 2010 & CP 2010 & 15 & 1 & 11 & \ref{b:LombardiM10} & \ref{c:LombardiM10}\\
\rowlabel{a:MakMS10}MakMS10 \href{https://doi.org/10.1109/ICNC.2010.5583494}{MakMS10} & \hyperref[auth:a637]{K. Mak}, \hyperref[auth:a638]{J. Ma}, \hyperref[auth:a639]{W. Su} & A constraint programming approach for production scheduling of multi-period virtual cellular manufacturing systems & \href{works/MakMS10.pdf}{Yes} & \cite{MakMS10} & 2010 & ICNC 2010 & 5 & 1 & 3 & \ref{b:MakMS10} & \ref{c:MakMS10}\\
\rowlabel{a:SchuttW10}SchuttW10 \href{https://doi.org/10.1007/978-3-642-15396-9\_36}{SchuttW10} & \hyperref[auth:a125]{A. Schutt}, \hyperref[auth:a51]{A. Wolf} & A New \emph{O}(\emph{n}\({}^{\mbox{2}}\)log\emph{n}) Not-First/Not-Last Pruning Algorithm for Cumulative Resource Constraints & \href{works/SchuttW10.pdf}{Yes} & \cite{SchuttW10} & 2010 & CP 2010 & 15 & 13 & 14 & \ref{b:SchuttW10} & \ref{c:SchuttW10}\\
\rowlabel{a:SunLYL10}SunLYL10 \href{https://doi.org/10.1109/GreenCom-CPSCom.2010.111}{SunLYL10} & \hyperref[auth:a633]{Z. Sun}, \hyperref[auth:a634]{H. Li}, \hyperref[auth:a635]{M. Yao}, \hyperref[auth:a636]{N. Li} & Scheduling Optimization Techniques for FlexRay Using Constraint-Programming & \href{works/SunLYL10.pdf}{Yes} & \cite{SunLYL10} & 2010 & GreenCom 2010 & 6 & 4 & 8 & \ref{b:SunLYL10} & \ref{c:SunLYL10}\\
\rowlabel{a:Acuna-AgostMFG09}Acuna-AgostMFG09 \href{https://doi.org/10.1007/978-3-642-01929-6\_24}{Acuna-AgostMFG09} & \hyperref[auth:a360]{R. Acuna{-}Agost}, \hyperref[auth:a361]{P. Michelon}, \hyperref[auth:a362]{D. Feillet}, \hyperref[auth:a363]{S. Gueye} & Constraint Programming and Mixed Integer Linear Programming for Rescheduling Trains under Disrupted Operations & \href{works/Acuna-AgostMFG09.pdf}{Yes} & \cite{Acuna-AgostMFG09} & 2009 & CPAIOR 2009 & 2 & 3 & 2 & \ref{b:Acuna-AgostMFG09} & \ref{c:Acuna-AgostMFG09}\\
\rowlabel{a:AronssonBK09}AronssonBK09 \href{http://drops.dagstuhl.de/opus/volltexte/2009/2141}{AronssonBK09} & \hyperref[auth:a717]{M. Aronsson}, \hyperref[auth:a718]{M. Bohlin}, \hyperref[auth:a719]{P. Kreuger} & {MILP} formulations of cumulative constraints for railway scheduling - {A} comparative study & \href{works/AronssonBK09.pdf}{Yes} & \cite{AronssonBK09} & 2009 & ATMOS 2009 & 13 & 0 & 0 & \ref{b:AronssonBK09} & \ref{c:AronssonBK09}\\
\rowlabel{a:Baptiste09}Baptiste09 \href{https://doi.org/10.1007/978-3-642-04244-7\_1}{Baptiste09} & \hyperref[auth:a164]{P. Baptiste} & Constraint-Based Schedulers, Do They Really Work? & \href{works/Baptiste09.pdf}{Yes} & \cite{Baptiste09} & 2009 & CP 2009 & 1 & 0 & 0 & \ref{b:Baptiste09} & \ref{c:Baptiste09}\\
\rowlabel{a:GrimesHM09}GrimesHM09 \href{https://doi.org/10.1007/978-3-642-04244-7\_33}{GrimesHM09} & \hyperref[auth:a183]{D. Grimes}, \hyperref[auth:a1]{E. Hebrard}, \hyperref[auth:a82]{A. Malapert} & Closing the Open Shop: Contradicting Conventional Wisdom & \href{works/GrimesHM09.pdf}{Yes} & \cite{GrimesHM09} & 2009 & CP 2009 & 9 & 15 & 12 & \ref{b:GrimesHM09} & \ref{c:GrimesHM09}\\
\rowlabel{a:Laborie09}Laborie09 \href{https://doi.org/10.1007/978-3-642-01929-6\_12}{Laborie09} & \hyperref[auth:a118]{P. Laborie} & {IBM} {ILOG} {CP} Optimizer for Detailed Scheduling Illustrated on Three Problems & \href{works/Laborie09.pdf}{Yes} & \cite{Laborie09} & 2009 & CPAIOR 2009 & 15 & 53 & 2 & \ref{b:Laborie09} & \ref{c:Laborie09}\\
\rowlabel{a:LombardiM09}LombardiM09 \href{https://doi.org/10.1007/978-3-642-04244-7\_45}{LombardiM09} & \hyperref[auth:a143]{M. Lombardi}, \hyperref[auth:a144]{M. Milano} & A Precedence Constraint Posting Approach for the {RCPSP} with Time Lags and Variable Durations & \href{works/LombardiM09.pdf}{Yes} & \cite{LombardiM09} & 2009 & CP 2009 & 15 & 7 & 12 & \ref{b:LombardiM09} & \ref{c:LombardiM09}\\
\rowlabel{a:MonetteDH09}MonetteDH09 \href{http://aaai.org/ocs/index.php/ICAPS/ICAPS09/paper/view/712}{MonetteDH09} & \hyperref[auth:a150]{J. Monette}, \hyperref[auth:a152]{Y. Deville}, \hyperref[auth:a149]{Pascal Van Hentenryck} & Just-In-Time Scheduling with Constraint Programming & \href{works/MonetteDH09.pdf}{Yes} & \cite{MonetteDH09} & 2009 & ICAPS 2009 & 8 & 0 & 0 & \ref{b:MonetteDH09} & \ref{c:MonetteDH09}\\
\rowlabel{a:SchuttFSW09}SchuttFSW09 \href{https://doi.org/10.1007/978-3-642-04244-7\_58}{SchuttFSW09} & \hyperref[auth:a125]{A. Schutt}, \hyperref[auth:a155]{T. Feydy}, \hyperref[auth:a126]{Peter J. Stuckey}, \hyperref[auth:a117]{M. Wallace} & Why Cumulative Decomposition Is Not as Bad as It Sounds & \href{works/SchuttFSW09.pdf}{Yes} & \cite{SchuttFSW09} & 2009 & CP 2009 & 16 & 34 & 11 & \ref{b:SchuttFSW09} & \ref{c:SchuttFSW09}\\
\rowlabel{a:ThiruvadyBME09}ThiruvadyBME09 \href{https://doi.org/10.1007/978-3-642-04918-7\_3}{ThiruvadyBME09} & \hyperref[auth:a402]{Dhananjay R. Thiruvady}, \hyperref[auth:a646]{C. Blum}, \hyperref[auth:a647]{B. Meyer}, \hyperref[auth:a476]{Andreas T. Ernst} & Hybridizing Beam-ACO with Constraint Programming for Single Machine Job Scheduling & \href{works/ThiruvadyBME09.pdf}{Yes} & \cite{ThiruvadyBME09} & 2009 & HM 2009 & 15 & 13 & 12 & \ref{b:ThiruvadyBME09} & \ref{c:ThiruvadyBME09}\\
\rowlabel{a:Vilim09}Vilim09 \href{https://doi.org/10.1007/978-3-642-04244-7\_62}{Vilim09} & \hyperref[auth:a121]{P. Vil{\'{\i}}m} & Edge Finding Filtering Algorithm for Discrete Cumulative Resources in \emph{O}(\emph{kn} log \emph{n})\{{\textbackslash}mathcal O\}(kn \{{\textbackslash}rm log\} n) & \href{works/Vilim09.pdf}{Yes} & \cite{Vilim09} & 2009 & CP 2009 & 15 & 25 & 4 & \ref{b:Vilim09} & \ref{c:Vilim09}\\
\rowlabel{a:Vilim09a}Vilim09a \href{https://doi.org/10.1007/978-3-642-01929-6\_22}{Vilim09a} & \hyperref[auth:a121]{P. Vil{\'{\i}}m} & Max Energy Filtering Algorithm for Discrete Cumulative Resources & \href{works/Vilim09a.pdf}{Yes} & \cite{Vilim09a} & 2009 & CPAIOR 2009 & 15 & 13 & 4 & \ref{b:Vilim09a} & \ref{c:Vilim09a}\\
\rowlabel{a:Wolf09}Wolf09 \href{http://dx.doi.org/10.1007/978-3-642-00675-3_2}{Wolf09} & \hyperref[auth:a51]{A. Wolf}, \hyperref[auth:a720]{G. Schrader} & Linear Weighted-Task-Sum – Scheduling Prioritized Tasks on a Single Resource & \href{works/Wolf09.pdf}{Yes} & \cite{Wolf09} & 2009 & INAP 2009 & 17 & 1 & 12 & \ref{b:Wolf09} & \ref{c:Wolf09}\\
\rowlabel{a:BarlattCG08}BarlattCG08 \href{https://doi.org/10.1007/978-3-540-68155-7\_24}{BarlattCG08} & \hyperref[auth:a367]{A. Barlatt}, \hyperref[auth:a368]{Amy Mainville Cohn}, \hyperref[auth:a369]{Oleg Yu. Gusikhin} & A Hybrid Approach for Solving Shift-Selection and Task-Sequencing Problems & \href{works/BarlattCG08.pdf}{Yes} & \cite{BarlattCG08} & 2008 & CPAIOR 2008 & 5 & 1 & 9 & \ref{b:BarlattCG08} & \ref{c:BarlattCG08}\\
\rowlabel{a:BeldiceanuCP08}BeldiceanuCP08 \href{https://doi.org/10.1007/978-3-540-68155-7\_5}{BeldiceanuCP08} & \hyperref[auth:a129]{N. Beldiceanu}, \hyperref[auth:a91]{M. Carlsson}, \hyperref[auth:a364]{E. Poder} & New Filtering for the cumulative Constraint in the Context of Non-Overlapping Rectangles & \href{works/BeldiceanuCP08.pdf}{Yes} & \cite{BeldiceanuCP08} & 2008 & CPAIOR 2008 & 15 & 8 & 9 & \ref{b:BeldiceanuCP08} & \ref{c:BeldiceanuCP08}\\
\rowlabel{a:BeniniLMR08}BeniniLMR08 \href{http://dx.doi.org/10.1007/978-3-540-85958-1_2}{BeniniLMR08} & \hyperref[auth:a248]{L. Benini}, \hyperref[auth:a143]{M. Lombardi}, \hyperref[auth:a144]{M. Milano}, \hyperref[auth:a727]{M. Ruggiero} & A Constraint Programming Approach for Allocation and Scheduling on the CELL Broadband Engine & \href{works/BeniniLMR08.pdf}{Yes} & \cite{BeniniLMR08} & 2008 & CP 2008 & 15 & 7 & 23 & \ref{b:BeniniLMR08} & \ref{c:BeniniLMR08}\\
\rowlabel{a:DoomsH08}DoomsH08 \href{https://doi.org/10.1007/978-3-540-68155-7\_8}{DoomsH08} & \hyperref[auth:a365]{G. Dooms}, \hyperref[auth:a149]{Pascal Van Hentenryck} & Gap Reduction Techniques for Online Stochastic Project Scheduling & \href{works/DoomsH08.pdf}{Yes} & \cite{DoomsH08} & 2008 & CPAIOR 2008 & 16 & 1 & 2 & \ref{b:DoomsH08} & \ref{c:DoomsH08}\\
\rowlabel{a:HentenryckM08}HentenryckM08 \href{https://doi.org/10.1007/978-3-540-68155-7\_41}{HentenryckM08} & \hyperref[auth:a149]{Pascal Van Hentenryck}, \hyperref[auth:a32]{L. Michel} & The Steel Mill Slab Design Problem Revisited & \href{works/HentenryckM08.pdf}{Yes} & \cite{HentenryckM08} & 2008 & CPAIOR 2008 & 5 & 13 & 3 & \ref{b:HentenryckM08} & \ref{c:HentenryckM08}\\
\rowlabel{a:LauLN08}LauLN08 \href{https://doi.org/10.1007/978-3-540-68155-7\_33}{LauLN08} & \hyperref[auth:a370]{Hoong Chuin Lau}, \hyperref[auth:a371]{Kong Wei Lye}, \hyperref[auth:a372]{Viet Bang Nguyen} & A Combinatorial Auction Framework for Solving Decentralized Scheduling Problems (Extended Abstract) & \href{works/LauLN08.pdf}{Yes} & \cite{LauLN08} & 2008 & CPAIOR 2008 & 5 & 0 & 4 & \ref{b:LauLN08} & \ref{c:LauLN08}\\
\rowlabel{a:MouraSCL08}MouraSCL08 \href{https://doi.org/10.1007/978-3-540-85958-1\_3}{MouraSCL08} & \hyperref[auth:a161]{Arnaldo Vieira Moura}, \hyperref[auth:a172]{Cid C. de Souza}, \hyperref[auth:a159]{Andr{\'{e}} A. Cir{\'{e}}}, \hyperref[auth:a158]{Tony Minoru Tamura Lopes} & Planning and Scheduling the Operation of a Very Large Oil Pipeline Network & \href{works/MouraSCL08.pdf}{Yes} & \cite{MouraSCL08} & 2008 & CP 2008 & 16 & 11 & 10 & \ref{b:MouraSCL08} & \ref{c:MouraSCL08}\\
\rowlabel{a:MouraSCL08a}MouraSCL08a \href{https://doi.org/10.1109/CSE.2008.24}{MouraSCL08a} & \hyperref[auth:a161]{Arnaldo Vieira Moura}, \hyperref[auth:a172]{Cid C. de Souza}, \hyperref[auth:a159]{Andr{\'{e}} A. Cir{\'{e}}}, \hyperref[auth:a158]{Tony Minoru Tamura Lopes} & Heuristics and Constraint Programming Hybridizations for a Real Pipeline Planning and Scheduling Problem & \href{works/MouraSCL08a.pdf}{Yes} & \cite{MouraSCL08a} & 2008 & CSE 2008 & 8 & 5 & 14 & \ref{b:MouraSCL08a} & \ref{c:MouraSCL08a}\\
\rowlabel{a:PoderB08}PoderB08 \href{http://www.aaai.org/Library/ICAPS/2008/icaps08-033.php}{PoderB08} & \hyperref[auth:a364]{E. Poder}, \hyperref[auth:a129]{N. Beldiceanu} & Filtering for a Continuous Multi-Resources cumulative Constraint with Resource Consumption and Production & \href{works/PoderB08.pdf}{Yes} & \cite{PoderB08} & 2008 & ICAPS 2008 & 8 & 0 & 0 & \ref{b:PoderB08} & \ref{c:PoderB08}\\
\rowlabel{a:WatsonB08}WatsonB08 \href{https://doi.org/10.1007/978-3-540-68155-7\_21}{WatsonB08} & \hyperref[auth:a366]{J. Watson}, \hyperref[auth:a89]{J. Christopher Beck} & A Hybrid Constraint Programming / Local Search Approach to the Job-Shop Scheduling Problem & \href{works/WatsonB08.pdf}{Yes} & \cite{WatsonB08} & 2008 & CPAIOR 2008 & 15 & 14 & 17 & \ref{b:WatsonB08} & \ref{c:WatsonB08}\\
\rowlabel{a:AkkerDH07}AkkerDH07 \href{https://doi.org/10.1007/978-3-540-72397-4\_27}{AkkerDH07} & \hyperref[auth:a378]{J. M. van den Akker}, \hyperref[auth:a379]{G. Diepen}, \hyperref[auth:a380]{J. A. Hoogeveen} & A Column Generation Based Destructive Lower Bound for Resource Constrained Project Scheduling Problems & \href{works/AkkerDH07.pdf}{Yes} & \cite{AkkerDH07} & 2007 & CPAIOR 2007 & 15 & 2 & 8 & \ref{b:AkkerDH07} & \ref{c:AkkerDH07}\\
\rowlabel{a:BeldiceanuP07}BeldiceanuP07 \href{https://doi.org/10.1007/978-3-540-72397-4\_16}{BeldiceanuP07} & \hyperref[auth:a129]{N. Beldiceanu}, \hyperref[auth:a364]{E. Poder} & A Continuous Multi-resources \emph{cumulative} Constraint with Positive-Negative Resource Consumption-Production & \href{works/BeldiceanuP07.pdf}{Yes} & \cite{BeldiceanuP07} & 2007 & CPAIOR 2007 & 15 & 4 & 7 & \ref{b:BeldiceanuP07} & \ref{c:BeldiceanuP07}\\
\rowlabel{a:DavenportKRSH07}DavenportKRSH07 \href{https://doi.org/10.1007/978-3-540-74970-7\_7}{DavenportKRSH07} & \hyperref[auth:a251]{Andrew J. Davenport}, \hyperref[auth:a252]{J. Kalagnanam}, \hyperref[auth:a253]{C. Reddy}, \hyperref[auth:a254]{S. Siegel}, \hyperref[auth:a255]{J. Hou} & An Application of Constraint Programming to Generating Detailed Operations Schedules for Steel Manufacturing & \href{works/DavenportKRSH07.pdf}{Yes} & \cite{DavenportKRSH07} & 2007 & CP 2007 & 13 & 1 & 2 & \ref{b:DavenportKRSH07} & \ref{c:DavenportKRSH07}\\
\rowlabel{a:GarganiR07}GarganiR07 \href{https://doi.org/10.1007/978-3-540-74970-7\_8}{GarganiR07} & \hyperref[auth:a256]{A. Gargani}, \hyperref[auth:a257]{P. Refalo} & An Efficient Model and Strategy for the Steel Mill Slab Design Problem & \href{works/GarganiR07.pdf}{Yes} & \cite{GarganiR07} & 2007 & CP 2007 & 13 & 17 & 5 & \ref{b:GarganiR07} & \ref{c:GarganiR07}\\
\rowlabel{a:HoeveGSL07}HoeveGSL07 \href{http://www.aaai.org/Library/AAAI/2007/aaai07-291.php}{HoeveGSL07} & \hyperref[auth:a212]{Willem{-}Jan van Hoeve}, \hyperref[auth:a652]{Carla P. Gomes}, \hyperref[auth:a653]{B. Selman}, \hyperref[auth:a143]{M. Lombardi} & Optimal Multi-Agent Scheduling with Constraint Programming & \href{works/HoeveGSL07.pdf}{Yes} & \cite{HoeveGSL07} & 2007 & AAAI 2007 & 6 & 0 & 0 & \ref{b:HoeveGSL07} & \ref{c:HoeveGSL07}\\
\rowlabel{a:KeriK07}KeriK07 \href{https://doi.org/10.1007/978-3-540-72397-4\_10}{KeriK07} & \hyperref[auth:a373]{A. K{\'{e}}ri}, \hyperref[auth:a157]{T. Kis} & Computing Tight Time Windows for {RCPSPWET} with the Primal-Dual Method & \href{works/KeriK07.pdf}{Yes} & \cite{KeriK07} & 2007 & CPAIOR 2007 & 14 & 1 & 13 & \ref{b:KeriK07} & \ref{c:KeriK07}\\
\rowlabel{a:KovacsB07}KovacsB07 \href{https://doi.org/10.1007/978-3-540-72397-4\_9}{KovacsB07} & \hyperref[auth:a147]{A. Kov{\'{a}}cs}, \hyperref[auth:a89]{J. Christopher Beck} & A Global Constraint for Total Weighted Completion Time & \href{works/KovacsB07.pdf}{Yes} & \cite{KovacsB07} & 2007 & CPAIOR 2007 & 15 & 2 & 12 & \ref{b:KovacsB07} & \ref{c:KovacsB07}\\
\rowlabel{a:KrogtLPHJ07}KrogtLPHJ07 \href{https://doi.org/10.1007/978-3-540-74970-7\_10}{KrogtLPHJ07} & \hyperref[auth:a258]{Roman van der Krogt}, \hyperref[auth:a180]{J. Little}, \hyperref[auth:a259]{K. Pulliam}, \hyperref[auth:a260]{S. Hanhilammi}, \hyperref[auth:a261]{Y. Jin} & Scheduling for Cellular Manufacturing & \href{works/KrogtLPHJ07.pdf}{Yes} & \cite{KrogtLPHJ07} & 2007 & CP 2007 & 13 & 2 & 3 & \ref{b:KrogtLPHJ07} & \ref{c:KrogtLPHJ07}\\
\rowlabel{a:Limtanyakul07}Limtanyakul07 \href{https://doi.org/10.1007/978-3-540-77903-2\_65}{Limtanyakul07} & \hyperref[auth:a145]{K. Limtanyakul} & Scheduling of Tests on Vehicle Prototypes Using Constraint and Integer Programming & \href{works/Limtanyakul07.pdf}{Yes} & \cite{Limtanyakul07} & 2007 & GOR 2007 & 6 & 2 & 3 & \ref{b:Limtanyakul07} & \ref{c:Limtanyakul07}\\
\rowlabel{a:MonetteDD07}MonetteDD07 \href{https://doi.org/10.1007/978-3-540-72397-4\_14}{MonetteDD07} & \hyperref[auth:a150]{J. Monette}, \hyperref[auth:a152]{Y. Deville}, \hyperref[auth:a374]{P. Dupont} & A Position-Based Propagator for the Open-Shop Problem & \href{works/MonetteDD07.pdf}{Yes} & \cite{MonetteDD07} & 2007 & CPAIOR 2007 & 14 & 0 & 12 & \ref{b:MonetteDD07} & \ref{c:MonetteDD07}\\
\rowlabel{a:RossiTHP07}RossiTHP07 \href{https://doi.org/10.1007/978-3-540-72397-4\_17}{RossiTHP07} & \hyperref[auth:a375]{R. Rossi}, \hyperref[auth:a376]{A. Tarim}, \hyperref[auth:a138]{B. Hnich}, \hyperref[auth:a377]{Steven D. Prestwich} & Replenishment Planning for Stochastic Inventory Systems with Shortage Cost & \href{works/RossiTHP07.pdf}{Yes} & \cite{RossiTHP07} & 2007 & CPAIOR 2007 & 15 & 6 & 10 & \ref{b:RossiTHP07} & \ref{c:RossiTHP07}\\
\rowlabel{a:Beck06}Beck06 \href{http://www.aaai.org/Library/ICAPS/2006/icaps06-028.php}{Beck06} & \hyperref[auth:a89]{J. Christopher Beck} & An Empirical Study of Multi-Point Constructive Search for Constraint-Based Scheduling & \href{works/Beck06.pdf}{Yes} & \cite{Beck06} & 2006 & ICAPS 2006 & 10 & 0 & 0 & \ref{b:Beck06} & \ref{c:Beck06}\\
\rowlabel{a:BeniniBGM06}BeniniBGM06 \href{https://doi.org/10.1007/11757375\_6}{BeniniBGM06} & \hyperref[auth:a248]{L. Benini}, \hyperref[auth:a381]{D. Bertozzi}, \hyperref[auth:a382]{A. Guerri}, \hyperref[auth:a144]{M. Milano} & Allocation, Scheduling and Voltage Scaling on Energy Aware MPSoCs & \href{works/BeniniBGM06.pdf}{Yes} & \cite{BeniniBGM06} & 2006 & CPAIOR 2006 & 15 & 18 & 10 & \ref{b:BeniniBGM06} & \ref{c:BeniniBGM06}\\
\rowlabel{a:GomesHS06}GomesHS06 \href{http://www.aaai.org/Library/Symposia/Spring/2006/ss06-04-024.php}{GomesHS06} & \hyperref[auth:a652]{Carla P. Gomes}, \hyperref[auth:a212]{Willem{-}Jan van Hoeve}, \hyperref[auth:a653]{B. Selman} & Constraint Programming for Distributed Planning and Scheduling & \href{works/GomesHS06.pdf}{Yes} & \cite{GomesHS06} & 2006 & AAAI 2006 & 2 & 0 & 0 & \ref{b:GomesHS06} & \ref{c:GomesHS06}\\
\rowlabel{a:KhemmoudjPB06}KhemmoudjPB06 \href{https://doi.org/10.1007/11889205\_21}{KhemmoudjPB06} & \hyperref[auth:a262]{Mohand Ou Idir Khemmoudj}, \hyperref[auth:a263]{M. Porcheron}, \hyperref[auth:a264]{H. Bennaceur} & When Constraint Programming and Local Search Solve the Scheduling Problem of Electricit{\'{e}} de France Nuclear Power Plant Outages & \href{works/KhemmoudjPB06.pdf}{Yes} & \cite{KhemmoudjPB06} & 2006 & CP 2006 & 13 & 8 & 8 & \ref{b:KhemmoudjPB06} & \ref{c:KhemmoudjPB06}\\
\rowlabel{a:KovacsV06}KovacsV06 \href{https://doi.org/10.1007/11757375\_13}{KovacsV06} & \hyperref[auth:a147]{A. Kov{\'{a}}cs}, \hyperref[auth:a281]{J. V{\'{a}}ncza} & Progressive Solutions: {A} Simple but Efficient Dominance Rule for Practical {RCPSP} & \href{works/KovacsV06.pdf}{Yes} & \cite{KovacsV06} & 2006 & CPAIOR 2006 & 13 & 2 & 7 & \ref{b:KovacsV06} & \ref{c:KovacsV06}\\
\rowlabel{a:LiuJ06}LiuJ06 \href{https://doi.org/10.1007/11801603\_92}{LiuJ06} & \hyperref[auth:a664]{Y. Liu}, \hyperref[auth:a665]{Y. Jiang} & {LP-TPOP:} Integrating Planning and Scheduling Through Constraint Programming & \href{works/LiuJ06.pdf}{Yes} & \cite{LiuJ06} & 2006 & PRICAI 2006 & 5 & 0 & 0 & \ref{b:LiuJ06} & \ref{c:LiuJ06}\\
\rowlabel{a:QuSN06}QuSN06 \href{https://doi.org/10.1109/ISSOC.2006.321973}{QuSN06} & \hyperref[auth:a661]{Y. Qu}, \hyperref[auth:a662]{J. Soininen}, \hyperref[auth:a663]{J. Nurmi} & Using Constraint Programming to Achieve Optimal Prefetch Scheduling for Dependent Tasks on Run-Time Reconfigurable Devices & \href{works/QuSN06.pdf}{Yes} & \cite{QuSN06} & 2006 & SoC 2006 & 4 & 2 & 5 & \ref{b:QuSN06} & \ref{c:QuSN06}\\
\rowlabel{a:Wallace06}Wallace06 \href{http://dx.doi.org/10.1007/978-3-540-73817-6_1}{Wallace06} & \hyperref[auth:a117]{M. Wallace} & Hybrid Algorithms in Constraint Programming & \href{works/Wallace06.pdf}{Yes} & \cite{Wallace06} & 2006 & CSCLP 2006 & 32 & 1 & 35 & \ref{b:Wallace06} & \ref{c:Wallace06}\\
\rowlabel{a:AbrilSB05}AbrilSB05 \href{https://doi.org/10.1007/11564751\_75}{AbrilSB05} & \hyperref[auth:a273]{M. Abril}, \hyperref[auth:a154]{Miguel A. Salido}, \hyperref[auth:a274]{F. Barber} & Distributed Constraints for Large-Scale Scheduling Problems & \href{works/AbrilSB05.pdf}{Yes} & \cite{AbrilSB05} & 2005 & CP 2005 & 1 & 0 & 0 & \ref{b:AbrilSB05} & \ref{c:AbrilSB05}\\
\rowlabel{a:ArtiouchineB05}ArtiouchineB05 \href{https://doi.org/10.1007/11564751\_8}{ArtiouchineB05} & \hyperref[auth:a265]{K. Artiouchine}, \hyperref[auth:a164]{P. Baptiste} & Inter-distance Constraint: An Extension of the All-Different Constraint for Scheduling Equal Length Jobs & \href{works/ArtiouchineB05.pdf}{Yes} & \cite{ArtiouchineB05} & 2005 & CP 2005 & 15 & 3 & 11 & \ref{b:ArtiouchineB05} & \ref{c:ArtiouchineB05}\\
\rowlabel{a:BeckW05}BeckW05 \href{http://ijcai.org/Proceedings/05/Papers/0748.pdf}{BeckW05} & \hyperref[auth:a89]{J. Christopher Beck}, \hyperref[auth:a838]{N. Wilson} & Proactive Algorithms for Scheduling with Probabilistic Durations & \href{works/BeckW05.pdf}{Yes} & \cite{BeckW05} & 2005 & IJCAI 2005 & 6 & 0 & 0 & \ref{b:BeckW05} & \ref{c:BeckW05}\\
\rowlabel{a:CarchraeBF05}CarchraeBF05 \href{https://doi.org/10.1007/11564751\_80}{CarchraeBF05} & \hyperref[auth:a275]{T. Carchrae}, \hyperref[auth:a89]{J. Christopher Beck}, \hyperref[auth:a276]{Eugene C. Freuder} & Methods to Learn Abstract Scheduling Models & \href{works/CarchraeBF05.pdf}{Yes} & \cite{CarchraeBF05} & 2005 & CP 2005 & 1 & 0 & 0 & \ref{b:CarchraeBF05} & \ref{c:CarchraeBF05}\\
\rowlabel{a:ChuX05}ChuX05 \href{https://doi.org/10.1007/11493853\_10}{ChuX05} & \hyperref[auth:a383]{Y. Chu}, \hyperref[auth:a384]{Q. Xia} & A Hybrid Algorithm for a Class of Resource Constrained Scheduling Problems & \href{works/ChuX05.pdf}{Yes} & \cite{ChuX05} & 2005 & CPAIOR 2005 & 15 & 13 & 13 & \ref{b:ChuX05} & \ref{c:ChuX05}\\
\rowlabel{a:DilkinaDH05}DilkinaDH05 \href{https://doi.org/10.1007/11564751\_60}{DilkinaDH05} & \hyperref[auth:a270]{B. Dilkina}, \hyperref[auth:a271]{L. Duan}, \hyperref[auth:a272]{William S. Havens} & Extending Systematic Local Search for Job Shop Scheduling Problems & \href{works/DilkinaDH05.pdf}{Yes} & \cite{DilkinaDH05} & 2005 & CP 2005 & 5 & 2 & 7 & \ref{b:DilkinaDH05} & \ref{c:DilkinaDH05}\\
\rowlabel{a:FortinZDF05}FortinZDF05 \href{https://doi.org/10.1007/11564751\_19}{FortinZDF05} & \hyperref[auth:a266]{J. Fortin}, \hyperref[auth:a267]{P. Zielinski}, \hyperref[auth:a268]{D. Dubois}, \hyperref[auth:a269]{H. Fargier} & Interval Analysis in Scheduling & \href{works/FortinZDF05.pdf}{Yes} & \cite{FortinZDF05} & 2005 & CP 2005 & 15 & 13 & 11 & \ref{b:FortinZDF05} & \ref{c:FortinZDF05}\\
\rowlabel{a:FrankK05}FrankK05 \href{https://doi.org/10.1007/11493853\_15}{FrankK05} & \hyperref[auth:a385]{J. Frank}, \hyperref[auth:a386]{E. K{\"{u}}rkl{\"{u}}} & Mixed Discrete and Continuous Algorithms for Scheduling Airborne Astronomy Observations & \href{works/FrankK05.pdf}{Yes} & \cite{FrankK05} & 2005 & CPAIOR 2005 & 18 & 4 & 4 & \ref{b:FrankK05} & \ref{c:FrankK05}\\
\rowlabel{a:Geske05}Geske05 \href{https://doi.org/10.1007/11963578\_10}{Geske05} & \hyperref[auth:a667]{U. Geske} & Railway Scheduling with Declarative Constraint Programming & \href{works/Geske05.pdf}{Yes} & \cite{Geske05} & 2005 & INAP 2005 & 18 & 2 & 3 & \ref{b:Geske05} & \ref{c:Geske05}\\
\rowlabel{a:GodardLN05}GodardLN05 \href{http://www.aaai.org/Library/ICAPS/2005/icaps05-009.php}{GodardLN05} & \hyperref[auth:a783]{D. Godard}, \hyperref[auth:a118]{P. Laborie}, \hyperref[auth:a666]{W. Nuijten} & Randomized Large Neighborhood Search for Cumulative Scheduling & \href{works/GodardLN05.pdf}{Yes} & \cite{GodardLN05} & 2005 & ICAPS 2005 & 9 & 0 & 0 & \ref{b:GodardLN05} & \ref{c:GodardLN05}\\
\rowlabel{a:HebrardTW05}HebrardTW05 \href{https://doi.org/10.1007/11564751\_117}{HebrardTW05} & \hyperref[auth:a1]{E. Hebrard}, \hyperref[auth:a278]{P. Tyler}, \hyperref[auth:a279]{T. Walsh} & Computing Super-Schedules & \href{works/HebrardTW05.pdf}{Yes} & \cite{HebrardTW05} & 2005 & CP 2005 & 1 & 0 & 3 & \ref{b:HebrardTW05} & \ref{c:HebrardTW05}\\
\rowlabel{a:Hooker05a}Hooker05a \href{https://doi.org/10.1007/11564751\_25}{Hooker05a} & \hyperref[auth:a162]{John N. Hooker} & Planning and Scheduling to Minimize Tardiness & \href{works/Hooker05a.pdf}{Yes} & \cite{Hooker05a} & 2005 & CP 2005 & 14 & 30 & 10 & \ref{b:Hooker05a} & \ref{c:Hooker05a}\\
\rowlabel{a:KovacsEKV05}KovacsEKV05 \href{https://doi.org/10.1007/11564751\_118}{KovacsEKV05} & \hyperref[auth:a147]{A. Kov{\'{a}}cs}, \hyperref[auth:a280]{P. Egri}, \hyperref[auth:a157]{T. Kis}, \hyperref[auth:a281]{J. V{\'{a}}ncza} & Proterv-II: An Integrated Production Planning and Scheduling System & \href{works/KovacsEKV05.pdf}{Yes} & \cite{KovacsEKV05} & 2005 & CP 2005 & 1 & 2 & 3 & \ref{b:KovacsEKV05} & \ref{c:KovacsEKV05}\\
\rowlabel{a:MoffittPP05}MoffittPP05 \href{http://www.aaai.org/Library/AAAI/2005/aaai05-188.php}{MoffittPP05} & \hyperref[auth:a780]{Michael D. Moffitt}, \hyperref[auth:a781]{B. Peintner}, \hyperref[auth:a782]{Martha E. Pollack} & Augmenting Disjunctive Temporal Problems with Finite-Domain Constraints & \href{works/MoffittPP05.pdf}{Yes} & \cite{MoffittPP05} & 2005 & AAAI 2005 & 6 & 0 & 0 & \ref{b:MoffittPP05} & \ref{c:MoffittPP05}\\
\rowlabel{a:QuirogaZH05}QuirogaZH05 \href{https://doi.org/10.1109/ROBOT.2005.1570686}{QuirogaZH05} & \hyperref[auth:a632]{O. Quiroga}, \hyperref[auth:a631]{L. Zeballos}, \hyperref[auth:a598]{Gabriela P. Henning} & A Constraint Programming Approach to Tool Allocation and Resource Scheduling in {FMS} & \href{works/QuirogaZH05.pdf}{Yes} & \cite{QuirogaZH05} & 2005 & ICRA 2005 & 6 & 2 & 7 & \ref{b:QuirogaZH05} & \ref{c:QuirogaZH05}\\
\rowlabel{a:SchuttWS05}SchuttWS05 \href{https://doi.org/10.1007/11963578\_6}{SchuttWS05} & \hyperref[auth:a125]{A. Schutt}, \hyperref[auth:a51]{A. Wolf}, \hyperref[auth:a720]{G. Schrader} & Not-First and Not-Last Detection for Cumulative Scheduling in \emph{O}(\emph{n}\({}^{\mbox{3}}\)log\emph{n}) & \href{works/SchuttWS05.pdf}{Yes} & \cite{SchuttWS05} & 2005 & INAP 2005 & 15 & 6 & 4 & \ref{b:SchuttWS05} & \ref{c:SchuttWS05}\\
\rowlabel{a:Vilim05}Vilim05 \href{https://doi.org/10.1007/11493853\_29}{Vilim05} & \hyperref[auth:a121]{P. Vil{\'{\i}}m} & Computing Explanations for the Unary Resource Constraint & \href{works/Vilim05.pdf}{Yes} & \cite{Vilim05} & 2005 & CPAIOR 2005 & 14 & 5 & 8 & \ref{b:Vilim05} & \ref{c:Vilim05}\\
\rowlabel{a:Wolf05}Wolf05 \href{http://dx.doi.org/10.1007/11402763_15}{Wolf05} & \hyperref[auth:a51]{A. Wolf} & Better Propagation for Non-preemptive Single-Resource Constraint Problems & \href{works/Wolf05.pdf}{Yes} & \cite{Wolf05} & 2005 & CSCLP 2005 & 15 & 4 & 8 & \ref{b:Wolf05} & \ref{c:Wolf05}\\
\rowlabel{a:WolfS05}WolfS05 \href{https://doi.org/10.1007/11963578\_8}{WolfS05} & \hyperref[auth:a51]{A. Wolf}, \hyperref[auth:a720]{G. Schrader} & \emph{O}(\emph{n} log\emph{n}) Overload Checking for the Cumulative Constraint and Its Application & \href{works/WolfS05.pdf}{Yes} & \cite{WolfS05} & 2005 & INAP 2005 & 14 & 6 & 6 & \ref{b:WolfS05} & \ref{c:WolfS05}\\
\rowlabel{a:WuBB05}WuBB05 \href{https://doi.org/10.1007/11564751\_110}{WuBB05} & \hyperref[auth:a277]{Christine Wei Wu}, \hyperref[auth:a223]{Kenneth N. Brown}, \hyperref[auth:a89]{J. Christopher Beck} & Scheduling with Uncertain Start Dates & \href{works/WuBB05.pdf}{Yes} & \cite{WuBB05} & 2005 & CP 2005 & 1 & 0 & 0 & \ref{b:WuBB05} & \ref{c:WuBB05}\\
\rowlabel{a:ArtiguesBF04}ArtiguesBF04 \href{https://doi.org/10.1007/978-3-540-24664-0\_3}{ArtiguesBF04} & \hyperref[auth:a6]{C. Artigues}, \hyperref[auth:a389]{S. Belmokhtar}, \hyperref[auth:a362]{D. Feillet} & A New Exact Solution Algorithm for the Job Shop Problem with Sequence-Dependent Setup Times & \href{works/ArtiguesBF04.pdf}{Yes} & \cite{ArtiguesBF04} & 2004 & CPAIOR 2004 & 13 & 16 & 9 & \ref{b:ArtiguesBF04} & \ref{c:ArtiguesBF04}\\
\rowlabel{a:BeckW04}BeckW04 \href{}{BeckW04} & \hyperref[auth:a89]{J. Christopher Beck}, \hyperref[auth:a838]{N. Wilson} & Job Shop Scheduling with Probabilistic Durations & \href{works/BeckW04.pdf}{Yes} & \cite{BeckW04} & 2004 & ECAI 2004 & 5 & 0 & 0 & \ref{b:BeckW04} & \ref{c:BeckW04}\\
\rowlabel{a:HentenryckM04}HentenryckM04 \href{https://doi.org/10.1007/978-3-540-24664-0\_22}{HentenryckM04} & \hyperref[auth:a149]{Pascal Van Hentenryck}, \hyperref[auth:a32]{L. Michel} & Scheduling Abstractions for Local Search & \href{works/HentenryckM04.pdf}{Yes} & \cite{HentenryckM04} & 2004 & CPAIOR 2004 & 16 & 12 & 14 & \ref{b:HentenryckM04} & \ref{c:HentenryckM04}\\
\rowlabel{a:Hooker04}Hooker04 \href{https://doi.org/10.1007/978-3-540-30201-8\_24}{Hooker04} & \hyperref[auth:a162]{John N. Hooker} & A Hybrid Method for Planning and Scheduling & \href{works/Hooker04.pdf}{Yes} & \cite{Hooker04} & 2004 & CP 2004 & 12 & 39 & 9 & \ref{b:Hooker04} & \ref{c:Hooker04}\\
\rowlabel{a:KovacsV04}KovacsV04 \href{https://doi.org/10.1007/978-3-540-30201-8\_26}{KovacsV04} & \hyperref[auth:a147]{A. Kov{\'{a}}cs}, \hyperref[auth:a281]{J. V{\'{a}}ncza} & Completable Partial Solutions in Constraint Programming and Constraint-Based Scheduling & \href{works/KovacsV04.pdf}{Yes} & \cite{KovacsV04} & 2004 & CP 2004 & 15 & 3 & 12 & \ref{b:KovacsV04} & \ref{c:KovacsV04}\\
\rowlabel{a:LimRX04}LimRX04 \href{https://doi.org/10.1007/978-3-540-30201-8\_59}{LimRX04} & \hyperref[auth:a282]{A. Lim}, \hyperref[auth:a283]{B. Rodrigues}, \hyperref[auth:a284]{Z. Xu} & Solving the Crane Scheduling Problem Using Intelligent Search Schemes & \href{works/LimRX04.pdf}{Yes} & \cite{LimRX04} & 2004 & CP 2004 & 5 & 5 & 6 & \ref{b:LimRX04} & \ref{c:LimRX04}\\
\rowlabel{a:MaraveliasG04}MaraveliasG04 \href{https://doi.org/10.1007/978-3-540-24664-0\_1}{MaraveliasG04} & \hyperref[auth:a387]{Christos T. Maravelias}, \hyperref[auth:a388]{Ignacio E. Grossmann} & Using {MILP} and {CP} for the Scheduling of Batch Chemical Processes & \href{works/MaraveliasG04.pdf}{Yes} & \cite{MaraveliasG04} & 2004 & CPAIOR 2004 & 20 & 15 & 15 & \ref{b:MaraveliasG04} & \ref{c:MaraveliasG04}\\
\rowlabel{a:Sadykov04}Sadykov04 \href{https://doi.org/10.1007/978-3-540-24664-0\_31}{Sadykov04} & \hyperref[auth:a390]{R. Sadykov} & A Hybrid Branch-And-Cut Algorithm for the One-Machine Scheduling Problem & \href{works/Sadykov04.pdf}{Yes} & \cite{Sadykov04} & 2004 & CPAIOR 2004 & 7 & 11 & 7 & \ref{b:Sadykov04} & \ref{c:Sadykov04}\\
\rowlabel{a:Vilim04}Vilim04 \href{https://doi.org/10.1007/978-3-540-24664-0\_23}{Vilim04} & \hyperref[auth:a121]{P. Vil{\'{\i}}m} & O(n log n) Filtering Algorithms for Unary Resource Constraint & \href{works/Vilim04.pdf}{Yes} & \cite{Vilim04} & 2004 & CPAIOR 2004 & 13 & 22 & 5 & \ref{b:Vilim04} & \ref{c:Vilim04}\\
\rowlabel{a:VilimBC04}VilimBC04 \href{https://doi.org/10.1007/978-3-540-30201-8\_8}{VilimBC04} & \hyperref[auth:a121]{P. Vil{\'{\i}}m}, \hyperref[auth:a153]{R. Bart{\'{a}}k}, \hyperref[auth:a163]{O. Cepek} & Unary Resource Constraint with Optional Activities & \href{works/VilimBC04.pdf}{Yes} & \cite{VilimBC04} & 2004 & CP 2004 & 15 & 13 & 4 & \ref{b:VilimBC04} & \ref{c:VilimBC04}\\
\rowlabel{a:VillaverdeP04}VillaverdeP04 \href{}{VillaverdeP04} & \hyperref[auth:a668]{K. Villaverde}, \hyperref[auth:a33]{E. Pontelli} & An Investigation of Scheduling in Distributed Constraint Logic Programming & No & \cite{VillaverdeP04} & 2004 & ISCA 2004 & 6 & 0 & 0 & No & \ref{c:VillaverdeP04}\\
\rowlabel{a:WolinskiKG04}WolinskiKG04 \href{https://doi.org/10.1109/DSD.2004.1333291}{WolinskiKG04} & \hyperref[auth:a669]{C. Wolinski}, \hyperref[auth:a670]{K. Kuchcinski}, \hyperref[auth:a671]{Maya B. Gokhale} & A Constraints Programming Approach to Communication Scheduling on SoPC Architectures & \href{works/WolinskiKG04.pdf}{Yes} & \cite{WolinskiKG04} & 2004 & DSD 2004 & 8 & 0 & 9 & \ref{b:WolinskiKG04} & \ref{c:WolinskiKG04}\\
\rowlabel{a:BeckPS03}BeckPS03 \href{http://www.aaai.org/Library/ICAPS/2003/icaps03-027.php}{BeckPS03} & \hyperref[auth:a89]{J. Christopher Beck}, \hyperref[auth:a839]{P. Prosser}, \hyperref[auth:a840]{E. Selensky} & Vehicle Routing and Job Shop Scheduling: What's the Difference? & \href{works/BeckPS03.pdf}{Yes} & \cite{BeckPS03} & 2003 & ICAPS 2003 & 10 & 0 & 0 & \ref{b:BeckPS03} & \ref{c:BeckPS03}\\
\rowlabel{a:DannaP03}DannaP03 \href{https://doi.org/10.1007/978-3-540-45193-8\_59}{DannaP03} & \hyperref[auth:a290]{E. Danna}, \hyperref[auth:a291]{L. Perron} & Structured vs. Unstructured Large Neighborhood Search: {A} Case Study on Job-Shop Scheduling Problems with Earliness and Tardiness Costs & \href{works/DannaP03.pdf}{Yes} & \cite{DannaP03} & 2003 & CP 2003 & 5 & 21 & 3 & \ref{b:DannaP03} & \ref{c:DannaP03}\\
\rowlabel{a:Kumar03}Kumar03 \href{https://doi.org/10.1007/978-3-540-45193-8\_45}{Kumar03} & \hyperref[auth:a289]{T. K. Satish Kumar} & Incremental Computation of Resource-Envelopes in Producer-Consumer Models & \href{works/Kumar03.pdf}{Yes} & \cite{Kumar03} & 2003 & CP 2003 & 15 & 4 & 2 & \ref{b:Kumar03} & \ref{c:Kumar03}\\
\rowlabel{a:OddiPCC03}OddiPCC03 \href{https://doi.org/10.1007/978-3-540-45193-8\_39}{OddiPCC03} & \hyperref[auth:a285]{A. Oddi}, \hyperref[auth:a286]{N. Policella}, \hyperref[auth:a287]{A. Cesta}, \hyperref[auth:a288]{G. Cortellessa} & Generating High Quality Schedules for a Spacecraft Memory Downlink Problem & \href{works/OddiPCC03.pdf}{Yes} & \cite{OddiPCC03} & 2003 & CP 2003 & 15 & 8 & 6 & \ref{b:OddiPCC03} & \ref{c:OddiPCC03}\\
\rowlabel{a:ValleMGT03}ValleMGT03 \href{https://doi.org/10.1007/978-3-540-45226-3\_180}{ValleMGT03} & \hyperref[auth:a676]{Carmelo Del Valle}, \hyperref[auth:a677]{Antonio A. M{\'{a}}rquez}, \hyperref[auth:a678]{Rafael M. Gasca}, \hyperref[auth:a679]{M. Toro} & On Selecting and Scheduling Assembly Plans Using Constraint Programming & \href{works/ValleMGT03.pdf}{Yes} & \cite{ValleMGT03} & 2003 & KES 2003 & 8 & 7 & 7 & \ref{b:ValleMGT03} & \ref{c:ValleMGT03}\\
\rowlabel{a:Vilim03}Vilim03 \href{https://doi.org/10.1007/978-3-540-45193-8\_124}{Vilim03} & \hyperref[auth:a121]{P. Vil{\'{\i}}m} & Computing Explanations for Global Scheduling Constraints & \href{works/Vilim03.pdf}{Yes} & \cite{Vilim03} & 2003 & CP 2003 & 1 & 1 & 1 & \ref{b:Vilim03} & \ref{c:Vilim03}\\
\rowlabel{a:Wolf03}Wolf03 \href{https://doi.org/10.1007/978-3-540-45193-8\_50}{Wolf03} & \hyperref[auth:a51]{A. Wolf} & Pruning while Sweeping over Task Intervals & \href{works/Wolf03.pdf}{Yes} & \cite{Wolf03} & 2003 & CP 2003 & 15 & 11 & 7 & \ref{b:Wolf03} & \ref{c:Wolf03}\\
\rowlabel{a:Bartak02}Bartak02 \href{https://doi.org/10.1007/3-540-46135-3\_39}{Bartak02} & \hyperref[auth:a153]{R. Bart{\'{a}}k} & Visopt ShopFloor: On the Edge of Planning and Scheduling & \href{works/Bartak02.pdf}{Yes} & \cite{Bartak02} & 2002 & CP 2002 & 16 & 6 & 4 & \ref{b:Bartak02} & \ref{c:Bartak02}\\
\rowlabel{a:Bartak02a}Bartak02a \href{https://doi.org/10.1007/3-540-36607-5\_14}{Bartak02a} & \hyperref[auth:a153]{R. Bart{\'{a}}k} & Visopt ShopFloor: Going Beyond Traditional Scheduling & \href{works/Bartak02a.pdf}{Yes} & \cite{Bartak02a} & 2002 & ERCIM/CologNet 2002 & 15 & 1 & 9 & \ref{b:Bartak02a} & \ref{c:Bartak02a}\\
\rowlabel{a:BeldiceanuC02}BeldiceanuC02 \href{https://doi.org/10.1007/3-540-46135-3\_5}{BeldiceanuC02} & \hyperref[auth:a129]{N. Beldiceanu}, \hyperref[auth:a91]{M. Carlsson} & A New Multi-resource cumulatives Constraint with Negative Heights & \href{works/BeldiceanuC02.pdf}{Yes} & \cite{BeldiceanuC02} & 2002 & CP 2002 & 17 & 33 & 9 & \ref{b:BeldiceanuC02} & \ref{c:BeldiceanuC02}\\
\rowlabel{a:ElkhyariGJ02}ElkhyariGJ02 \href{https://doi.org/10.1007/3-540-46135-3\_49}{ElkhyariGJ02} & \hyperref[auth:a295]{A. Elkhyari}, \hyperref[auth:a296]{C. Gu{\'{e}}ret}, \hyperref[auth:a250]{N. Jussien} & Conflict-Based Repair Techniques for Solving Dynamic Scheduling Problems & \href{works/ElkhyariGJ02.pdf}{Yes} & \cite{ElkhyariGJ02} & 2002 & CP 2002 & 6 & 1 & 6 & \ref{b:ElkhyariGJ02} & \ref{c:ElkhyariGJ02}\\
\rowlabel{a:ElkhyariGJ02a}ElkhyariGJ02a \href{https://doi.org/10.1007/978-3-540-45157-0\_3}{ElkhyariGJ02a} & \hyperref[auth:a295]{A. Elkhyari}, \hyperref[auth:a296]{C. Gu{\'{e}}ret}, \hyperref[auth:a250]{N. Jussien} & Solving Dynamic Resource Constraint Project Scheduling Problems Using New Constraint Programming Tools & \href{works/ElkhyariGJ02a.pdf}{Yes} & \cite{ElkhyariGJ02a} & 2002 & PATAT 2002 & 24 & 9 & 20 & \ref{b:ElkhyariGJ02a} & \ref{c:ElkhyariGJ02a}\\
\rowlabel{a:HookerY02}HookerY02 \href{https://doi.org/10.1007/3-540-46135-3\_46}{HookerY02} & \hyperref[auth:a162]{John N. Hooker}, \hyperref[auth:a294]{H. Yan} & A Relaxation of the Cumulative Constraint & \href{works/HookerY02.pdf}{Yes} & \cite{HookerY02} & 2002 & CP 2002 & 5 & 8 & 7 & \ref{b:HookerY02} & \ref{c:HookerY02}\\
\rowlabel{a:KamarainenS02}KamarainenS02 \href{https://doi.org/10.1007/3-540-46135-3\_11}{KamarainenS02} & \hyperref[auth:a293]{O. Kamarainen}, \hyperref[auth:a168]{Hani El Sakkout} & Local Probing Applied to Scheduling & \href{works/KamarainenS02.pdf}{Yes} & \cite{KamarainenS02} & 2002 & CP 2002 & 17 & 9 & 13 & \ref{b:KamarainenS02} & \ref{c:KamarainenS02}\\
\rowlabel{a:Muscettola02}Muscettola02 \href{https://doi.org/10.1007/3-540-46135-3\_10}{Muscettola02} & \hyperref[auth:a292]{N. Muscettola} & Computing the Envelope for Stepwise-Constant Resource Allocations & \href{works/Muscettola02.pdf}{Yes} & \cite{Muscettola02} & 2002 & CP 2002 & 16 & 14 & 4 & \ref{b:Muscettola02} & \ref{c:Muscettola02}\\
\rowlabel{a:Vilim02}Vilim02 \href{https://doi.org/10.1007/3-540-46135-3\_62}{Vilim02} & \hyperref[auth:a121]{P. Vil{\'{\i}}m} & Batch Processing with Sequence Dependent Setup Times & \href{works/Vilim02.pdf}{Yes} & \cite{Vilim02} & 2002 & CP 2002 & 1 & 6 & 1 & \ref{b:Vilim02} & \ref{c:Vilim02}\\
\rowlabel{a:ZhuS02}ZhuS02 \href{https://doi.org/10.1007/3-540-47961-9\_69}{ZhuS02} & \hyperref[auth:a684]{Kenny Qili Zhu}, \hyperref[auth:a685]{Andrew E. Santosa} & A Meeting Scheduling System Based on Open Constraint Programming & \href{works/ZhuS02.pdf}{Yes} & \cite{ZhuS02} & 2002 & CAiSE 2002 & 5 & 0 & 5 & \ref{b:ZhuS02} & \ref{c:ZhuS02}\\
\rowlabel{a:Thorsteinsson01}Thorsteinsson01 \href{https://doi.org/10.1007/3-540-45578-7\_2}{Thorsteinsson01} & \hyperref[auth:a887]{Erlendur S. Thorsteinsson} & Branch-and-Check: {A} Hybrid Framework Integrating Mixed Integer Programming and Constraint Logic Programming & \href{works/Thorsteinsson01.pdf}{Yes} & \cite{Thorsteinsson01} & 2001 & CP 2001 & 15 & 67 & 12 & \ref{b:Thorsteinsson01} & \ref{c:Thorsteinsson01}\\
\rowlabel{a:VanczaM01}VanczaM01 \href{https://doi.org/10.1007/3-540-45578-7\_60}{VanczaM01} & \hyperref[auth:a281]{J. V{\'{a}}ncza}, \hyperref[auth:a297]{A. M{\'{a}}rkus} & A Constraint Engine for Manufacturing Process Planning & \href{works/VanczaM01.pdf}{Yes} & \cite{VanczaM01} & 2001 & CP 2001 & 15 & 2 & 19 & \ref{b:VanczaM01} & \ref{c:VanczaM01}\\
\rowlabel{a:VerfaillieL01}VerfaillieL01 \href{https://doi.org/10.1007/3-540-45578-7\_55}{VerfaillieL01} & \hyperref[auth:a175]{G. Verfaillie}, \hyperref[auth:a174]{M. Lema{\^{\i}}tre} & Selecting and Scheduling Observations for Agile Satellites: Some Lessons from the Constraint Reasoning Community Point of View & \href{works/VerfaillieL01.pdf}{Yes} & \cite{VerfaillieL01} & 2001 & CP 2001 & 15 & 11 & 6 & \ref{b:VerfaillieL01} & \ref{c:VerfaillieL01}\\
\rowlabel{a:AngelsmarkJ00}AngelsmarkJ00 \href{https://doi.org/10.1007/3-540-45349-0\_35}{AngelsmarkJ00} & \hyperref[auth:a298]{O. Angelsmark}, \hyperref[auth:a299]{P. Jonsson} & Some Observations on Durations, Scheduling and Allen's Algebra & \href{works/AngelsmarkJ00.pdf}{Yes} & \cite{AngelsmarkJ00} & 2000 & CP 2000 & 5 & 1 & 9 & \ref{b:AngelsmarkJ00} & \ref{c:AngelsmarkJ00}\\
\rowlabel{a:FocacciLN00}FocacciLN00 \href{http://www.aaai.org/Library/AIPS/2000/aips00-010.php}{FocacciLN00} & \hyperref[auth:a785]{F. Focacci}, \hyperref[auth:a118]{P. Laborie}, \hyperref[auth:a666]{W. Nuijten} & Solving Scheduling Problems with Setup Times and Alternative Resources & \href{works/FocacciLN00.pdf}{Yes} & \cite{FocacciLN00} & 2000 & AIPS 2000 & 10 & 0 & 0 & \ref{b:FocacciLN00} & \ref{c:FocacciLN00}\\
\rowlabel{a:DorndorfPH99}DorndorfPH99 \href{http://dx.doi.org/10.1007/978-3-642-58409-1_35}{DorndorfPH99} & \hyperref[auth:a922]{U. Dorndorf}, \hyperref[auth:a445]{E. Pesch}, \hyperref[auth:a923]{Toàn Phan Huy} & Recent Developments in Scheduling & No & \cite{DorndorfPH99} & 1999 & Operations Research Proceedings 1999 & null & 0 & 34 & No & \ref{c:DorndorfPH99}\\
\rowlabel{a:KorbaaYG99}KorbaaYG99 \href{https://doi.org/10.23919/ECC.1999.7099947}{KorbaaYG99} & \hyperref[auth:a690]{O. Korbaa}, \hyperref[auth:a691]{P. Yim}, \hyperref[auth:a692]{J. Gentina} & Solving transient scheduling problem for cyclic production using timed Petri nets and constraint programming & \href{works/KorbaaYG99.pdf}{Yes} & \cite{KorbaaYG99} & 1999 & ECC 1999 & 8 & 1 & 0 & \ref{b:KorbaaYG99} & \ref{c:KorbaaYG99}\\
\rowlabel{a:Simonis99}Simonis99 \href{https://doi.org/10.1007/3-540-45406-3\_6}{Simonis99} & \hyperref[auth:a17]{H. Simonis} & Building Industrial Applications with Constraint Programming & \href{works/Simonis99.pdf}{Yes} & \cite{Simonis99} & 1999 & CCL'99 1999 & 39 & 5 & 18 & \ref{b:Simonis99} & \ref{c:Simonis99}\\
\rowlabel{a:CestaOS98}CestaOS98 \href{https://doi.org/10.1007/3-540-49481-2\_36}{CestaOS98} & \hyperref[auth:a287]{A. Cesta}, \hyperref[auth:a285]{A. Oddi}, \hyperref[auth:a301]{Stephen F. Smith} & Scheduling Multi-capacitated Resources Under Complex Temporal Constraints & \href{works/CestaOS98.pdf}{Yes} & \cite{CestaOS98} & 1998 & CP 1998 & 1 & 5 & 0 & \ref{b:CestaOS98} & \ref{c:CestaOS98}\\
\rowlabel{a:FrostD98}FrostD98 \href{https://doi.org/10.1007/3-540-49481-2\_40}{FrostD98} & \hyperref[auth:a302]{D. Frost}, \hyperref[auth:a303]{R. Dechter} & Optimizing with Constraints: {A} Case Study in Scheduling Maintenance of Electric Power Units & \href{works/FrostD98.pdf}{Yes} & \cite{FrostD98} & 1998 & CP 1998 & 1 & 10 & 2 & \ref{b:FrostD98} & \ref{c:FrostD98}\\
\rowlabel{a:GruianK98}GruianK98 \href{https://doi.org/10.1109/EURMIC.1998.711781}{GruianK98} & \hyperref[auth:a696]{F. Gruian}, \hyperref[auth:a670]{K. Kuchcinski} & Operation Binding and Scheduling for Low Power Using Constraint Logic Programming & \href{works/GruianK98.pdf}{Yes} & \cite{GruianK98} & 1998 & EUROMICRO 1998 & 8 & 5 & 10 & \ref{b:GruianK98} & \ref{c:GruianK98}\\
\rowlabel{a:PembertonG98}PembertonG98 \href{https://doi.org/10.1090/dimacs/057/06}{PembertonG98} & \hyperref[auth:a694]{Joseph C. Pemberton}, \hyperref[auth:a695]{Flavius Galiber III} & A constraint-based approach to satellite scheduling & \href{works/PembertonG98.pdf}{Yes} & \cite{PembertonG98} & 1998 & DIMACS 1998 & 14 & 26 & 0 & \ref{b:PembertonG98} & \ref{c:PembertonG98}\\
\rowlabel{a:RodosekW98}RodosekW98 \href{https://doi.org/10.1007/3-540-49481-2\_28}{RodosekW98} & \hyperref[auth:a300]{R. Rodosek}, \hyperref[auth:a117]{M. Wallace} & A Generic Model and Hybrid Algorithm for Hoist Scheduling Problems & \href{works/RodosekW98.pdf}{Yes} & \cite{RodosekW98} & 1998 & CP 1998 & 15 & 19 & 10 & \ref{b:RodosekW98} & \ref{c:RodosekW98}\\
\rowlabel{a:BaptisteP97}BaptisteP97 \href{https://doi.org/10.1007/BFb0017454}{BaptisteP97} & \hyperref[auth:a164]{P. Baptiste}, \hyperref[auth:a165]{Claude Le Pape} & Constraint Propagation and Decomposition Techniques for Highly Disjunctive and Highly Cumulative Project Scheduling Problems & \href{works/BaptisteP97.pdf}{Yes} & \cite{BaptisteP97} & 1997 & CP 1997 & 15 & 8 & 10 & \ref{b:BaptisteP97} & \ref{c:BaptisteP97}\\
\rowlabel{a:BeckDF97}BeckDF97 \href{https://doi.org/10.1007/BFb0017455}{BeckDF97} & \hyperref[auth:a89]{J. Christopher Beck}, \hyperref[auth:a251]{Andrew J. Davenport}, \hyperref[auth:a305]{Mark S. Fox} & Five Pitfalls of Empirical Scheduling Research & \href{works/BeckDF97.pdf}{Yes} & \cite{BeckDF97} & 1997 & CP 1997 & 15 & 3 & 12 & \ref{b:BeckDF97} & \ref{c:BeckDF97}\\
\rowlabel{a:BoucherBVBL97}BoucherBVBL97 \href{}{BoucherBVBL97} & \hyperref[auth:a700]{E. Boucher}, \hyperref[auth:a701]{A. Bachelu}, \hyperref[auth:a702]{C. Varnier}, \hyperref[auth:a703]{P. Baptiste}, \hyperref[auth:a704]{B. Legeard} & Multi-criteria Comparison Between Algorithmic, Constraint Logic and Specific Constraint Programming on a Real Schedulingt Problem & No & \cite{BoucherBVBL97} & 1997 & PACT 1997 & 18 & 0 & 0 & No & \ref{c:BoucherBVBL97}\\
\rowlabel{a:Caseau97}Caseau97 \href{https://doi.org/10.1007/BFb0017437}{Caseau97} & \hyperref[auth:a304]{Y. Caseau} & Using Constraint Propagation for Complex Scheduling Problems: Managing Size, Complex Resources and Travel & \href{works/Caseau97.pdf}{Yes} & \cite{Caseau97} & 1997 & CP 1997 & 4 & 0 & 0 & \ref{b:Caseau97} & \ref{c:Caseau97}\\
\rowlabel{a:PapeB97}PapeB97 \href{}{PapeB97} & \hyperref[auth:a165]{Claude Le Pape}, \hyperref[auth:a164]{P. Baptiste} & A Constraint Programming Library for Preemptive and Non-Preemptive Scheduling & No & \cite{PapeB97} & 1997 & PACT 1997 & 20 & 0 & 0 & No & \ref{c:PapeB97}\\
\rowlabel{a:BrusoniCLMMT96}BrusoniCLMMT96 \href{https://doi.org/10.1007/3-540-61286-6\_157}{BrusoniCLMMT96} & \hyperref[auth:a731]{V. Brusoni}, \hyperref[auth:a732]{L. Console}, \hyperref[auth:a729]{E. Lamma}, \hyperref[auth:a730]{P. Mello}, \hyperref[auth:a144]{M. Milano}, \hyperref[auth:a733]{P. Terenziani} & Resource-Based vs. Task-Based Approaches for Scheduling Problems & \href{works/BrusoniCLMMT96.pdf}{Yes} & \cite{BrusoniCLMMT96} & 1996 & ISMIS 1996 & 10 & 1 & 9 & \ref{b:BrusoniCLMMT96} & \ref{c:BrusoniCLMMT96}\\
\rowlabel{a:Colombani96}Colombani96 \href{https://doi.org/10.1007/3-540-61551-2\_72}{Colombani96} & \hyperref[auth:a170]{Y. Colombani} & Constraint Programming: an Efficient and Practical Approach to Solving the Job-Shop Problem & \href{works/Colombani96.pdf}{Yes} & \cite{Colombani96} & 1996 & CP 1996 & 15 & 4 & 5 & \ref{b:Colombani96} & \ref{c:Colombani96}\\
\rowlabel{a:Zhou96}Zhou96 \href{https://doi.org/10.1007/3-540-61551-2\_97}{Zhou96} & \hyperref[auth:a178]{J. Zhou} & A Constraint Program for Solving the Job-Shop Problem & \href{works/Zhou96.pdf}{Yes} & \cite{Zhou96} & 1996 & CP 1996 & 15 & 10 & 7 & \ref{b:Zhou96} & \ref{c:Zhou96}\\
\rowlabel{a:Goltz95}Goltz95 \href{https://doi.org/10.1007/3-540-60299-2\_33}{Goltz95} & \hyperref[auth:a307]{H. Goltz} & Reducing Domains for Search in {CLP(FD)} and Its Application to Job-Shop Scheduling & \href{works/Goltz95.pdf}{Yes} & \cite{Goltz95} & 1995 & CP 1995 & 14 & 7 & 7 & \ref{b:Goltz95} & \ref{c:Goltz95}\\
\rowlabel{a:Puget95}Puget95 \href{https://doi.org/10.1007/3-540-60299-2\_43}{Puget95} & \hyperref[auth:a308]{J. Puget} & Applications of Constraint Programming & \href{works/Puget95.pdf}{Yes} & \cite{Puget95} & 1995 & CP 1995 & 4 & 6 & 2 & \ref{b:Puget95} & \ref{c:Puget95}\\
\rowlabel{a:Simonis95}Simonis95 \href{https://doi.org/10.1007/3-540-60299-2\_42}{Simonis95} & \hyperref[auth:a17]{H. Simonis} & The {CHIP} System and Its Applications & \href{works/Simonis95.pdf}{Yes} & \cite{Simonis95} & 1995 & CP 1995 & 4 & 7 & 3 & \ref{b:Simonis95} & \ref{c:Simonis95}\\
\rowlabel{a:Simonis95a}Simonis95a \href{https://doi.org/10.1007/3-540-60794-3\_11}{Simonis95a} & \hyperref[auth:a17]{H. Simonis} & Application Development with the {CHIP} System & \href{works/Simonis95a.pdf}{Yes} & \cite{Simonis95a} & 1995 & CONTESSA 1995 & 21 & 1 & 12 & \ref{b:Simonis95a} & \ref{c:Simonis95a}\\
\rowlabel{a:SimonisC95}SimonisC95 \href{https://doi.org/10.1007/3-540-60299-2\_27}{SimonisC95} & \hyperref[auth:a17]{H. Simonis}, \hyperref[auth:a306]{T. Cornelissens} & Modelling Producer/Consumer Constraints & \href{works/SimonisC95.pdf}{Yes} & \cite{SimonisC95} & 1995 & CP 1995 & 14 & 17 & 8 & \ref{b:SimonisC95} & \ref{c:SimonisC95}\\
\rowlabel{a:Touraivane95}Touraivane95 \href{https://doi.org/10.1007/3-540-60299-2\_41}{Touraivane95} & \hyperref[auth:a309]{Toura{\"{\i}}vane} & Constraint Programming and Industrial Applications & \href{works/Touraivane95.pdf}{Yes} & \cite{Touraivane95} & 1995 & CP 1995 & 3 & 2 & 1 & \ref{b:Touraivane95} & \ref{c:Touraivane95}\\
\rowlabel{a:JourdanFRD94}JourdanFRD94 \href{}{JourdanFRD94} & \hyperref[auth:a707]{J. Jourdan}, \hyperref[auth:a708]{F. Fages}, \hyperref[auth:a709]{D. Rozzonelli}, \hyperref[auth:a710]{A. Demeure} & Data Alignment and Task Scheduling On Parallel Machines Using Concurrent Constraint Model-based Programming & No & \cite{JourdanFRD94} & 1994 & ILPS 1994 & 1 & 0 & 0 & No & \ref{c:JourdanFRD94}\\
\rowlabel{a:NuijtenA94}NuijtenA94 \href{}{NuijtenA94} & \hyperref[auth:a786]{W. P. M. Nuijten}, \hyperref[auth:a787]{Emile H. L. Aarts} & Constraint Satisfaction for Multiple Capacitated Job Shop Scheduling & \href{works/NuijtenA94.pdf}{Yes} & \cite{NuijtenA94} & 1994 & ECAI 1994 & 5 & 0 & 0 & \ref{b:NuijtenA94} & \ref{c:NuijtenA94}\\
\rowlabel{a:Wallace94}Wallace94 \href{}{Wallace94} & \hyperref[auth:a117]{M. Wallace} & Applying Constraints for Scheduling & No & \cite{Wallace94} & 1994 & Constraint Programming 1994 & 19 & 0 & 0 & No & \ref{c:Wallace94}\\
\rowlabel{a:BaptisteLV92}BaptisteLV92 \href{https://doi.org/10.1109/ROBOT.1992.220195}{BaptisteLV92} & \hyperref[auth:a703]{P. Baptiste}, \hyperref[auth:a704]{B. Legeard}, \hyperref[auth:a702]{C. Varnier} & Hoist scheduling problem: an approach based on constraint logic programming & \href{works/BaptisteLV92.pdf}{Yes} & \cite{BaptisteLV92} & 1992 & ICRA 1992 & 6 & 13 & 6 & \ref{b:BaptisteLV92} & \ref{c:BaptisteLV92}\\
\rowlabel{a:ErtlK91}ErtlK91 \href{https://doi.org/10.1007/3-540-54444-5\_89}{ErtlK91} & \hyperref[auth:a712]{M. Anton Ertl}, \hyperref[auth:a713]{A. Krall} & Optimal Instruction Scheduling using Constraint Logic Programming & \href{works/ErtlK91.pdf}{Yes} & \cite{ErtlK91} & 1991 & PLILP 1991 & 12 & 14 & 14 & \ref{b:ErtlK91} & \ref{c:ErtlK91}\\
\end{longtable}
}



\clearpage
\subsection{Extracted Concepts}
{\scriptsize
\begin{longtable}{>{\raggedright\arraybackslash}p{3cm}r>{\raggedright\arraybackslash}p{4cm}p{1.5cm}p{2cm}p{1.5cm}p{1.5cm}p{1.5cm}p{1.5cm}p{2cm}p{1.5cm}rr}
\rowcolor{white}\caption{Automatically Extracted PAPER Properties (Requires Local Copy)}\\ \toprule
\rowcolor{white}Work & Pages & Concepts & Classification & Constraints & \shortstack{Prog\\Languages} & \shortstack{CP\\Systems} & Areas & Industries & Benchmarks & Algorithm & a & c\\ \midrule\endhead
\bottomrule
\endfoot
\rowlabel{b:AalianPG23}\href{../works/AalianPG23.pdf}{AalianPG23}~\cite{AalianPG23} & 16 & scheduling, preempt, transportation, machine, make-span, activity, flow-shop, order, resource, preemptive &  & cycle, noOverlap, endBeforeStart, alwaysIn, cumulative &  & CPO, Cplex & steel cable & mining industry & real-world &  & \ref{a:AalianPG23} & \ref{c:AalianPG23}\\
\rowlabel{b:AbrilSB05}\href{../works/AbrilSB05.pdf}{AbrilSB05}~\cite{AbrilSB05} & 1 & distributed, multi-agent, scheduling, order &  &  &  &  & railway &  &  &  & \ref{a:AbrilSB05} & \ref{c:AbrilSB05}\\
\rowlabel{b:Acuna-AgostMFG09}\href{../works/Acuna-AgostMFG09.pdf}{Acuna-AgostMFG09}~\cite{Acuna-AgostMFG09} & 2 & re-scheduling, order, scheduling, transportation &  &  &  &  & railway &  & Roadef &  & \ref{a:Acuna-AgostMFG09} & \ref{c:Acuna-AgostMFG09}\\
\rowlabel{b:AkkerDH07}\href{../works/AkkerDH07.pdf}{AkkerDH07}~\cite{AkkerDH07} & 15 & due-date, cmax, machine, job, lateness, sequence dependent setup, preempt, resource, no-wait, scheduling, precedence, order, make-span, completion-time, release-date, preemptive & parallel machine, RCPSP, single machine & cumulative &  & Cplex &  &  &  &  & \ref{a:AkkerDH07} & \ref{c:AkkerDH07}\\
\rowlabel{b:AlesioNBG14}\href{../works/AlesioNBG14.pdf}{AlesioNBG14}~\cite{AlesioNBG14} & 18 & preempt, scheduling, completion-time, resource, task, job-shop, distributed, make-span, open-shop, order, job, activity, periodic, preemptive &  & alldifferent &  & OPL, Cplex & automotive &  & benchmark &  & \ref{a:AlesioNBG14} & \ref{c:AlesioNBG14}\\
\rowlabel{b:AmadiniGM16}\href{../works/AmadiniGM16.pdf}{AmadiniGM16}~\cite{AmadiniGM16} & 7 & make-span, lazy clause generation, scheduling, resource, task, distributed, precedence & RCPSP & cumulative &  & MiniZinc, Choco Solver, Gurobi, Gecode, OR-Tools &  &  & benchmark, real-life, github &  & \ref{a:AmadiniGM16} & \ref{c:AmadiniGM16}\\
\rowlabel{b:AngelsmarkJ00}\href{../works/AngelsmarkJ00.pdf}{AngelsmarkJ00}~\cite{AngelsmarkJ00} & 5 & resource, job, order, scheduling, task, job-shop &  &  &  &  &  &  &  &  & \ref{a:AngelsmarkJ00} & \ref{c:AngelsmarkJ00}\\
\rowlabel{b:AntunesABD18}\href{../works/AntunesABD18.pdf}{AntunesABD18}~\cite{AntunesABD18} & 8 & earliness, scheduling, machine, order, lateness, activity, due-date, re-scheduling, task, Benders Decomposition, Logic-Based Benders Decomposition, periodic, stochastic &  & bin-packing, BinPacking constraint &  & Cplex & workforce scheduling, maintenance scheduling & electricity industry & real-world, industry partner, industrial partner &  & \ref{a:AntunesABD18} & \ref{c:AntunesABD18}\\
\rowlabel{b:AntuoriHHEN20}\href{../works/AntuoriHHEN20.pdf}{AntuoriHHEN20}~\cite{AntuoriHHEN20} & 16 & due-date, task, job-shop, precedence, release-date, resource, job, order, completion-time, tardiness, scheduling, machine, periodic, stochastic &  & alldifferent, circuit, Element constraint, cycle, Channeling constraint &  & Choco Solver & torpedo &  & random instance, generated instance, gitlab, benchmark, industrial instance &  & \ref{a:AntuoriHHEN20} & \ref{c:AntuoriHHEN20}\\
\rowlabel{b:AntuoriHHEN21}\href{../works/AntuoriHHEN21.pdf}{AntuoriHHEN21}~\cite{AntuoriHHEN21} & 16 & transportation, due-date, task, job-shop, precedence, release-date, resource, job, order, tardiness, scheduling, machine, stochastic &  & cycle & C++, Java & Choco Solver, Gecode & automotive, car manufacturing, drone & automotive industry & gitlab, supplementary material & GRASP & \ref{a:AntuoriHHEN21} & \ref{c:AntuoriHHEN21}\\
\rowlabel{b:ArbaouiY18}\href{../works/ArbaouiY18.pdf}{ArbaouiY18}~\cite{ArbaouiY18} & 10 & order, sequence dependent setup, resource, job, scheduling, setup-time, machine, make-span, no-wait, completion-time, cmax & single machine, parallel machine & Pulse constraint, alternative constraint, noOverlap, cumulative & C++ & Cplex &  &  & benchmark &  & \ref{a:ArbaouiY18} & \ref{c:ArbaouiY18}\\
\rowlabel{b:ArmstrongGOS21}\href{../works/ArmstrongGOS21.pdf}{ArmstrongGOS21}~\cite{ArmstrongGOS21} & 18 & machine, flow-shop, job-shop, job, order, sequence dependent setup, cmax, transportation, scheduling, make-span, completion-time, preempt, resource, setup-time, precedence, task, preemptive & HFF, HFFTT, HFS & cycle, alternative constraint, table constraint, circuit, diffn, bin-packing, cumulative & Java, Prolog & Gecode, CHIP, MiniZinc, CPO, Chuffed, SICStus, Cplex & robot & packaging industry & instance generator, industry partner, zenodo, supplementary material, real-world, industrial partner, benchmark & energetic reasoning & \ref{a:ArmstrongGOS21} & \ref{c:ArmstrongGOS21}\\
\rowlabel{b:ArmstrongGOS22}\href{../works/ArmstrongGOS22.pdf}{ArmstrongGOS22}~\cite{ArmstrongGOS22} & 13 & machine, flow-shop, job, re-scheduling, order, cmax, no-wait, transportation, scheduling, make-span, completion-time, resource, task & HFF, parallel machine, HFFTT, HFS & noOverlap, cumulative & Prolog & OPL, SICStus &  &  & real-world, benchmark & IGT, GRASP, NEH & \ref{a:ArmstrongGOS22} & \ref{c:ArmstrongGOS22}\\
\rowlabel{b:AronssonBK09}\href{../works/AronssonBK09.pdf}{AronssonBK09}~\cite{AronssonBK09} & 13 & job-shop, transportation, order, job, task &  & cumulative & Prolog & CHIP, Cplex & railway &  & real-world, real-life & sweep & \ref{a:AronssonBK09} & \ref{c:AronssonBK09}\\
\rowlabel{b:ArtiguesBF04}\href{../works/ArtiguesBF04.pdf}{ArtiguesBF04}~\cite{ArtiguesBF04} & 13 & batch process, cmax, resource, completion-time, scheduling, machine, job, make-span, release-date, precedence, sequence dependent setup, job-shop, setup-time, preempt, order, one-machine scheduling, preemptive &  & Disjunctive constraint, disjunctive & C++ & Ilog Solver, Ilog Scheduler &  &  & benchmark & edge-finding & \ref{a:ArtiguesBF04} & \ref{c:ArtiguesBF04}\\
\rowlabel{b:ArtiguesHQT21}\href{../works/ArtiguesHQT21.pdf}{ArtiguesHQT21}~\cite{ArtiguesHQT21} & 8 & order, resource, preempt, scheduling, release-date, machine, job, preemptive & RCPSP & cumulative &  & Cplex &  &  &  &  & \ref{a:ArtiguesHQT21} & \ref{c:ArtiguesHQT21}\\
\rowlabel{b:ArtiouchineB05}\href{../works/ArtiouchineB05.pdf}{ArtiouchineB05}~\cite{ArtiouchineB05} & 15 & release-date, completion-time, job, resource, activity, open-shop, machine, job-shop, re-scheduling, scheduling, order, make-span, preempt, precedence, preemptive & parallel machine, single machine & Disjunctive constraint, cumulative, disjunctive &  & Ilog Scheduler & aircraft &  & generated instance, random instance & not-last, edge-finding, not-first & \ref{a:ArtiouchineB05} & \ref{c:ArtiouchineB05}\\
\rowlabel{b:Astrand0F21}\href{../works/Astrand0F21.pdf}{Astrand0F21}~\cite{Astrand0F21} & 18 & open-shop, task, precedence, make-span, order, job, activity, scheduling, resource, machine, job-shop &  & cycle, disjunctive, Disjunctive constraint &  & Gecode & farming, forestry, agriculture, drone, robot, satellite & potash industry, mining industry, mineral industry & benchmark, real-life, real-world, generated instance &  & \ref{a:Astrand0F21} & \ref{c:Astrand0F21}\\
\rowlabel{b:AstrandJZ18}\href{../works/AstrandJZ18.pdf}{AstrandJZ18}~\cite{AstrandJZ18} & 9 & task, make-span, order, activity, scheduling, resource, machine, periodic & single machine & disjunctive, cumulative, cycle &  & Gecode & hoist, robot & potash industry &  & time-tabling & \ref{a:AstrandJZ18} & \ref{c:AstrandJZ18}\\
\rowlabel{b:BadicaBIL19}\href{../works/BadicaBIL19.pdf}{BadicaBIL19}~\cite{BadicaBIL19} & 11 & completion-time, resource, distributed, order, activity, machine, multi-agent, make-span, scheduling &  & cycle, Arithmetic constraint &  & ECLiPSe, Gecode &  &  & github &  & \ref{a:BadicaBIL19} & \ref{c:BadicaBIL19}\\
\rowlabel{b:BajestaniB11}\href{../works/BajestaniB11.pdf}{BajestaniB11}~\cite{BajestaniB11} & 8 & re-scheduling, Benders Decomposition, scheduling, machine, transportation, order, tardiness, make-span, resource, inventory, due-date, job, Logic-Based Benders Decomposition, periodic, single-machine scheduling, stochastic & JSSP, single machine & cycle, Cardinality constraint, cumulative, circuit &  & Ilog Solver, Cplex & railway, maintenance scheduling, aircraft &  &  &  & \ref{a:BajestaniB11} & \ref{c:BajestaniB11}\\
\rowlabel{b:Baptiste09}\href{../works/Baptiste09.pdf}{Baptiste09}~\cite{Baptiste09} & 1 & scheduling &  &  &  &  &  &  &  &  & \ref{a:Baptiste09} & \ref{c:Baptiste09}\\
\rowlabel{b:BaptisteLV92}\href{../works/BaptisteLV92.pdf}{BaptisteLV92}~\cite{BaptisteLV92} & 6 &  &  &  &  &  &  &  &  &  & \ref{a:BaptisteLV92} & \ref{c:BaptisteLV92}\\
\rowlabel{b:BaptisteP97}\href{../works/BaptisteP97.pdf}{BaptisteP97}~\cite{BaptisteP97} & 15 & resource, preempt, job-shop, scheduling, re-scheduling, due-date, task, precedence, release-date, flow-shop, make-span, order, job, activity, preemptive & RCPSP & Disjunctive constraint, disjunctive, cumulative & C++ & Claire, CHIP &  &  & benchmark & edge-finding, edge-finder & \ref{a:BaptisteP97} & \ref{c:BaptisteP97}\\
\rowlabel{b:BarlattCG08}\href{../works/BarlattCG08.pdf}{BarlattCG08}~\cite{BarlattCG08} & 5 & scheduling, resource, setup-time, task, job-shop, transportation, job, machine, flow-shop &  &  &  &  & automotive, pipeline &  & real-world &  & \ref{a:BarlattCG08} & \ref{c:BarlattCG08}\\
\rowlabel{b:Bartak02}\href{../works/Bartak02.pdf}{Bartak02}~\cite{Bartak02} & 16 & make-span, machine, job, activity, resource, lateness, job-shop, precedence, earliness, scheduling, continuous-process, task, order &  & cumulative, disjunctive, Disjunctive constraint & Prolog & SICStus & dairies &  & real-life & edge-finding, time-tabling & \ref{a:Bartak02} & \ref{c:Bartak02}\\
\rowlabel{b:Bartak02a}\href{../works/Bartak02a.pdf}{Bartak02a}~\cite{Bartak02a} & 15 & activity, earliness, scheduling, make-span, task, machine, job, re-scheduling, job-shop, resource, precedence, order, tardiness &  & Disjunctive constraint, cumulative, disjunctive &  & Ilog Scheduler & dairies &  & benchmark, real-life & time-tabling, edge-finding & \ref{a:Bartak02a} & \ref{c:Bartak02a}\\
\rowlabel{b:BartakV15}\href{../works/BartakV15.pdf}{BartakV15}~\cite{BartakV15} & 12 & scheduling, make-span, machine, job, lateness, re-scheduling, job-shop, resource, precedence, order, activity, setup-time &  &  &  &  &  &  & real-world, real-life & sweep & \ref{a:BartakV15} & \ref{c:BartakV15}\\
\rowlabel{b:BartoliniBBLM14}\href{../works/BartoliniBBLM14.pdf}{BartoliniBBLM14}~\cite{BartoliniBBLM14} & 16 & tardiness, make-span, scheduling, resource, task, job, activity, machine &  & alternative constraint, cumulative &  &  & super-computer &  &  &  & \ref{a:BartoliniBBLM14} & \ref{c:BartoliniBBLM14}\\
\rowlabel{b:BarzegaranZP20}\href{../works/BarzegaranZP20.pdf}{BarzegaranZP20}~\cite{BarzegaranZP20} & 9 & resource, re-scheduling, distributed, machine, scheduling, order, task &  &  & Java & OR-Tools & automotive, robot &  &  &  & \ref{a:BarzegaranZP20} & \ref{c:BarzegaranZP20}\\
\rowlabel{b:Beck06}\href{../works/Beck06.pdf}{Beck06}~\cite{Beck06} & 10 & due-date, order, scheduling, machine, job-shop, tardiness, flow-shop, make-span, resource, job &  &  &  & Ilog Scheduler &  &  & benchmark &  & \ref{a:Beck06} & \ref{c:Beck06}\\
\rowlabel{b:BeckDF97}\href{../works/BeckDF97.pdf}{BeckDF97}~\cite{BeckDF97} & 15 & activity, release-date, make-span, resource, inventory, job-shop, precedence, due-date, re-scheduling, order, scheduling, machine, job, task & single machine & cycle, cumulative &  &  & robot &  & benchmark, real-world & edge-finding & \ref{a:BeckDF97} & \ref{c:BeckDF97}\\
\rowlabel{b:BeckPS03}\href{../works/BeckPS03.pdf}{BeckPS03}~\cite{BeckPS03} & 10 & job, task, activity, release-date, make-span, transportation, earliness, flow-time, resource, job-shop, precedence, due-date, re-scheduling, order, tardiness, scheduling, completion-time, machine, setup-time, stochastic & RCPSP &  &  & Ilog Scheduler & robot &  & benchmark, real-world &  & \ref{a:BeckPS03} & \ref{c:BeckPS03}\\
\rowlabel{b:BeckW04}\href{../works/BeckW04.pdf}{BeckW04}~\cite{BeckW04} & 5 & job-shop, machine, activity, distributed, flow-shop, resource, job, order, make-span, scheduling, one-machine scheduling, stochastic & single machine &  &  & Ilog Scheduler &  &  &  & edge-finding, time-tabling & \ref{a:BeckW04} & \ref{c:BeckW04}\\
\rowlabel{b:BeckW05}\href{../works/BeckW05.pdf}{BeckW05}~\cite{BeckW05} & 6 & job-shop, activity, flow-shop, resource, job, order, make-span, scheduling, stochastic &  & Balance constraint &  & Ilog Scheduler &  &  &  & edge-finder & \ref{a:BeckW05} & \ref{c:BeckW05}\\
\rowlabel{b:BehrensLM19}\href{../works/BehrensLM19.pdf}{BehrensLM19}~\cite{BehrensLM19} & 7 & order, resource, machine, scheduling, setup-time, task, distributed, multi-agent, make-span &  &  & Python & OR-Tools, MiniZinc & robot &  & github, real-world &  & \ref{a:BehrensLM19} & \ref{c:BehrensLM19}\\
\rowlabel{b:BeldiceanuC02}\href{../works/BeldiceanuC02.pdf}{BeldiceanuC02}~\cite{BeldiceanuC02} & 17 & task, resource, activity, order, producer/consumer, scheduling, machine & single machine & Cumulatives constraint, cumulative & Prolog & CHIP, SICStus & crew-scheduling &  & real-life, random instance, benchmark & sweep & \ref{a:BeldiceanuC02} & \ref{c:BeldiceanuC02}\\
\rowlabel{b:BeldiceanuCP08}\href{../works/BeldiceanuCP08.pdf}{BeldiceanuCP08}~\cite{BeldiceanuCP08} & 15 & scheduling, order, resource, task &  & disjunctive, geost, cumulative & Prolog & CHIP, SICStus, OPL & rectangle-packing, perfect-square &  & benchmark & edge-finding, sweep & \ref{a:BeldiceanuCP08} & \ref{c:BeldiceanuCP08}\\
\rowlabel{b:BeldiceanuP07}\href{../works/BeldiceanuP07.pdf}{BeldiceanuP07}~\cite{BeldiceanuP07} & 15 & preempt, task, resource, order, scheduling, release-date, due-date, preemptive &  & disjunctive, cumulative &  &  &  &  &  & sweep & \ref{a:BeldiceanuP07} & \ref{c:BeldiceanuP07}\\
\rowlabel{b:BenderWS21}\href{../works/BenderWS21.pdf}{BenderWS21}~\cite{BenderWS21} & 16 & activity, order, resource, scheduling, preempt, task, machine, make-span, job, distributed, setup-time, preemptive & RCPSP & noOverlap & Python &  & agriculture &  &  &  & \ref{a:BenderWS21} & \ref{c:BenderWS21}\\
\rowlabel{b:BenediktSMVH18}\href{../works/BenediktSMVH18.pdf}{BenediktSMVH18}~\cite{BenediktSMVH18} & 10 & job-shop, scheduling, order, preempt, resource, job, machine, single-machine scheduling & single machine, parallel machine & noOverlap &  & Gurobi & energy-price &  & github, random instance, generated instance &  & \ref{a:BenediktSMVH18} & \ref{c:BenediktSMVH18}\\
\rowlabel{b:BeniniBGM06}\href{../works/BeniniBGM06.pdf}{BeniniBGM06}~\cite{BeniniBGM06} & 15 & Benders Decomposition, task, distributed, precedence, make-span, order, activity, tardiness, scheduling, resource, setup-time, Logic-Based Benders Decomposition &  & cycle, cumulative &  & ECLiPSe, Cplex, Ilog Solver & automotive, pipeline &  & real-life &  & \ref{a:BeniniBGM06} & \ref{c:BeniniBGM06}\\
\rowlabel{b:BeniniLMR08}\href{../works/BeniniLMR08.pdf}{BeniniLMR08}~\cite{BeniniLMR08} & 15 & resource, Benders Decomposition, task, distributed, precedence, make-span, order, activity, machine, preempt, release-date, tardiness, scheduling, Logic-Based Benders Decomposition, periodic, preemptive & SCC & circuit &  & Ilog Scheduler, Cplex & medical, pipeline &  & benchmark &  & \ref{a:BeniniLMR08} & \ref{c:BeniniLMR08}\\
\rowlabel{b:BertholdHLMS10}\href{../works/BertholdHLMS10.pdf}{BertholdHLMS10}~\cite{BertholdHLMS10} & 5 & scheduling, order, preempt, precedence, completion-time, job, resource & psplib, RCPSP & disjunctive, cumulative &  & Cplex, SCIP, Z3 &  &  &  &  & \ref{a:BertholdHLMS10} & \ref{c:BertholdHLMS10}\\
\rowlabel{b:BessiereHMQW14}\href{../works/BessiereHMQW14.pdf}{BessiereHMQW14}~\cite{BessiereHMQW14} & 16 & scheduling, order, resource, setup-time, task, machine, job &  & BufferedResource, cycle, Cardinality constraint, alldifferent, Element constraint &  & Choco Solver & satellite & textile industry & benchmark, real-life &  & \ref{a:BessiereHMQW14} & \ref{c:BessiereHMQW14}\\
\rowlabel{b:BillautHL12}\href{../works/BillautHL12.pdf}{BillautHL12}~\cite{BillautHL12} & 15 & tardiness, job-shop, setup-time, due-date, open-shop, precedence, release-date, flow-shop, make-span, order, job, scheduling, completion-time, resource, machine, cmax, stochastic & single machine & cycle &  & Cplex, Mistral &  &  & random instance &  & \ref{a:BillautHL12} & \ref{c:BillautHL12}\\
\rowlabel{b:Bit-Monnot23}\href{../works/Bit-Monnot23.pdf}{Bit-Monnot23}~\cite{Bit-Monnot23} & 8 & distributed, job, open-shop, task, lazy clause generation, precedence, scheduling, machine, order, make-span, job-shop, resource, activity & OSP, Open Shop Scheduling Problem & Disjunctive constraint, cycle, cumulative, disjunctive &  & OR-Tools, CPO, MiniZinc, Mistral &  &  & benchmark, real-world, github &  & \ref{a:Bit-Monnot23} & \ref{c:Bit-Monnot23}\\
\rowlabel{b:BofillCSV17}\href{../works/BofillCSV17.pdf}{BofillCSV17}~\cite{BofillCSV17} & 9 & precedence, make-span, order, activity, machine, preempt, cmax, lazy clause generation, scheduling, resource, preemptive & RCPSP, psplib & cumulative &  & Z3, SCIP &  &  & benchmark & energetic reasoning & \ref{a:BofillCSV17} & \ref{c:BofillCSV17}\\
\rowlabel{b:BofillEGPSV14}\href{../works/BofillEGPSV14.pdf}{BofillEGPSV14}~\cite{BofillEGPSV14} & 16 & machine, order, scheduling, lazy clause generation, task &  & Channeling constraint &  & Cplex, Gecode, MiniZinc, SCIP &  &  & industrial instance & time-tabling & \ref{a:BofillEGPSV14} & \ref{c:BofillEGPSV14}\\
\rowlabel{b:BofillGSV15}\href{../works/BofillGSV15.pdf}{BofillGSV15}~\cite{BofillGSV15} & 9 & machine, scheduling, order &  & Channeling constraint, Cardinality constraint &  & Cplex &  &  & industrial instance & time-tabling & \ref{a:BofillGSV15} & \ref{c:BofillGSV15}\\
\rowlabel{b:BogaerdtW19}\href{../works/BogaerdtW19.pdf}{BogaerdtW19}~\cite{BogaerdtW19} & 16 & scheduling, completion-time, setup-time, job-shop, precedence, order, job, machine, tardiness, single-machine scheduling & single machine, parallel machine & noOverlap & C  & OPL, Cplex & railway &  & benchmark &  & \ref{a:BogaerdtW19} & \ref{c:BogaerdtW19}\\
\rowlabel{b:BonfiettiLBM11}\href{../works/BonfiettiLBM11.pdf}{BonfiettiLBM11}~\cite{BonfiettiLBM11} & 15 & scheduling, order, make-span, precedence, task, job, resource, activity, machine, job-shop, periodic & RCPSP & cumulative, cycle &  & Ilog Solver & hoist, robot &  & benchmark, generated instance, industrial instance &  & \ref{a:BonfiettiLBM11} & \ref{c:BonfiettiLBM11}\\
\rowlabel{b:BonfiettiLBM12}\href{../works/BonfiettiLBM12.pdf}{BonfiettiLBM12}~\cite{BonfiettiLBM12} & 16 & scheduling, order, make-span, precedence, job, resource, activity, distributed, machine, job-shop, periodic & RCPSP & cumulative, cycle &  & Ilog Solver & hoist, robot &  & benchmark & time-tabling & \ref{a:BonfiettiLBM12} & \ref{c:BonfiettiLBM12}\\
\rowlabel{b:BonfiettiLM13}\href{../works/BonfiettiLM13.pdf}{BonfiettiLM13}~\cite{BonfiettiLM13} & 5 & scheduling, make-span, job-shop, precedence, resource, activity, job, order, periodic & RCPSP & cycle, cumulative &  & Cplex &  &  &  &  & \ref{a:BonfiettiLM13} & \ref{c:BonfiettiLM13}\\
\rowlabel{b:BonfiettiLM14}\href{../works/BonfiettiLM14.pdf}{BonfiettiLM14}~\cite{BonfiettiLM14} & 16 & scheduling, machine, open-shop, distributed, make-span, task, job-shop, precedence, resource, activity, job, order, stochastic & RCPSP, psplib & cumulative &  &  &  &  & benchmark, real-world &  & \ref{a:BonfiettiLM14} & \ref{c:BonfiettiLM14}\\
\rowlabel{b:BonfiettiM12}\href{../works/BonfiettiM12.pdf}{BonfiettiM12}~\cite{BonfiettiM12} & 3 & job, task, scheduling, machine, precedence, job-shop, resource, activity, periodic & RCPSP & cumulative &  &  & hoist &  & industrial instance &  & \ref{a:BonfiettiM12} & \ref{c:BonfiettiM12}\\
\rowlabel{b:BonfiettiZLM16}\href{../works/BonfiettiZLM16.pdf}{BonfiettiZLM16}~\cite{BonfiettiZLM16} & 17 & resource, activity, scheduling, order, make-span, precedence, periodic & RCPSP & cumulative, cycle, disjunctive &  & OR-Tools & automotive & automotive industry, control system industry & generated instance, github, industrial instance, benchmark, real-world & sweep, edge-finder & \ref{a:BonfiettiZLM16} & \ref{c:BonfiettiZLM16}\\
\rowlabel{b:BonninMNE24}\href{../works/BonninMNE24.pdf}{BonninMNE24}~\cite{BonninMNE24} & 12 & open-shop, order, job, activity, flow-time, machine, preempt, precedence, release-date, flow-shop, make-span, scheduling, completion-time, resource, task, job-shop, preemptive, single-machine scheduling & single machine & noOverlap, Flowtime constraint, Completion constraint, disjunctive, cumulative, Disjunctive constraint & C++ & Cplex & patient, COVID, vaccine &  & benchmark, real-life & edge-finding, sweep, time-tabling & \ref{a:BonninMNE24} & \ref{c:BonninMNE24}\\
\rowlabel{b:BoothNB16}\href{../works/BoothNB16.pdf}{BoothNB16}~\cite{BoothNB16} & 17 & distributed, resource, machine, Benders Decomposition, precedence, order, activity, scheduling, task, re-scheduling, Logic-Based Benders Decomposition &  & cumulative, noOverlap, disjunctive & C++ & Cplex & robot, medical &  & real-world &  & \ref{a:BoothNB16} & \ref{c:BoothNB16}\\
\rowlabel{b:BoudreaultSLQ22}\href{../works/BoudreaultSLQ22.pdf}{BoudreaultSLQ22}~\cite{BoudreaultSLQ22} & 16 & activity, machine, transportation, distributed, lazy clause generation, order, make-span, scheduling, cmax, resource, preempt, precedence, task & RCPSP, psplib & disjunctive, Cumulatives constraint, Disjunctive constraint, cumulative &  & Chuffed, MiniZinc, OPL, OR-Tools & offshore & repair industry, ship repair industry & supplementary material, gitlab, benchmark, generated instance, real-life, industrial partner, github, real-world & edge-finding, not-first, not-last, energetic reasoning & \ref{a:BoudreaultSLQ22} & \ref{c:BoudreaultSLQ22}\\
\rowlabel{b:BridiLBBM16}\href{../works/BridiLBBM16.pdf}{BridiLBBM16}~\cite{BridiLBBM16} & 2 & task, distributed, make-span, order, job, activity, scheduling, resource, machine, periodic &  &  &  &  &  &  &  &  & \ref{a:BridiLBBM16} & \ref{c:BridiLBBM16}\\
\rowlabel{b:BrusoniCLMMT96}\href{../works/BrusoniCLMMT96.pdf}{BrusoniCLMMT96}~\cite{BrusoniCLMMT96} & 10 & no-wait, due-date, scheduling, order, resource, activity, precedence, task, distributed, job-shop, job &  & disjunctive, Disjunctive constraint & Prolog &  & railway, train schedule &  &  &  & \ref{a:BrusoniCLMMT96} & \ref{c:BrusoniCLMMT96}\\
\rowlabel{b:BurtLPS15}\href{../works/BurtLPS15.pdf}{BurtLPS15}~\cite{BurtLPS15} & 17 & task, job, job-shop, resource, machine, Benders Decomposition, precedence, order, tardiness, scheduling, make-span, completion-time, periodic, single-machine scheduling, stochastic & parallel machine, single machine & cumulative, cycle &  & Gurobi, Gecode, Cplex, MiniZinc &  &  & industry partner, real-world, benchmark &  & \ref{a:BurtLPS15} & \ref{c:BurtLPS15}\\
\rowlabel{b:CappartS17}\href{../works/CappartS17.pdf}{CappartS17}~\cite{CappartS17} & 16 & re-scheduling, resource, scheduling, task, machine, activity, job, precedence, job-shop, completion-time, order & TMS & cumulative, span constraint, noOverlap, alternative constraint &  & OPL & train schedule, railway &  & bitbucket, real-life, random instance &  & \ref{a:CappartS17} & \ref{c:CappartS17}\\
\rowlabel{b:CappartTSR18}\href{../works/CappartTSR18.pdf}{CappartTSR18}~\cite{CappartTSR18} & 17 & resource, setup-time, producer/consumer, activity, Benders Decomposition, scheduling, transportation, order, Logic-Based Benders Decomposition, periodic &  & cumulative, circuit, disjunctive, noOverlap &  & Cplex, CPO, MiniZinc, OPL & medical, patient &  & bitbucket, real-life, CSPlib &  & \ref{a:CappartTSR18} & \ref{c:CappartTSR18}\\
\rowlabel{b:CarchraeBF05}\href{../works/CarchraeBF05.pdf}{CarchraeBF05}~\cite{CarchraeBF05} & 1 & scheduling, task, make-span, order &  &  &  &  &  &  &  &  & \ref{a:CarchraeBF05} & \ref{c:CarchraeBF05}\\
\rowlabel{b:Caseau97}\href{../works/Caseau97.pdf}{Caseau97}~\cite{Caseau97} & 4 & preempt, order, scheduling, task, make-span, job, resource, job-shop, preemptive &  & cumulative &  &  & robot &  & benchmark & edge-finding & \ref{a:Caseau97} & \ref{c:Caseau97}\\
\rowlabel{b:CatusseCBL16}\href{../works/CatusseCBL16.pdf}{CatusseCBL16}~\cite{CatusseCBL16} & 7 & release-date, order, resource, due-date, scheduling, machine, job, task, stochastic & parallel machine, single machine & disjunctive & Julia & OPL &  &  &  &  & \ref{a:CatusseCBL16} & \ref{c:CatusseCBL16}\\
\rowlabel{b:CauwelaertDMS16}\href{../works/CauwelaertDMS16.pdf}{CauwelaertDMS16}~\cite{CauwelaertDMS16} & 16 & batch process, order, make-span, scheduling, completion-time, setup-time, resource, preempt, precedence, task, job, job-shop, activity, machine, sequence dependent setup, preemptive &  & cumulative, disjunctive & Java &  & container terminal &  & real-life, bitbucket, benchmark & not-last, edge-finding, not-first & \ref{a:CauwelaertDMS16} & \ref{c:CauwelaertDMS16}\\
\rowlabel{b:CestaOS98}\href{../works/CestaOS98.pdf}{CestaOS98}~\cite{CestaOS98} & 1 & job, resource, scheduling &  &  &  &  & robot &  &  &  & \ref{a:CestaOS98} & \ref{c:CestaOS98}\\
\rowlabel{b:ChapadosJR11}\href{../works/ChapadosJR11.pdf}{ChapadosJR11}~\cite{ChapadosJR11} & 6 & activity, task, scheduling, order &  & cycle, cumulative &  & OPL &  & retail industry &  & time-tabling & \ref{a:ChapadosJR11} & \ref{c:ChapadosJR11}\\
\rowlabel{b:ChuGNSW13}\href{../works/ChuGNSW13.pdf}{ChuGNSW13}~\cite{ChuGNSW13} & 7 & distributed, resource, machine, job, scheduling, precedence, order, task &  & cumulative, alldifferent, Cardinality constraint, disjunctive &  & CHIP &  &  &  & not-first, not-last, edge-finding & \ref{a:ChuGNSW13} & \ref{c:ChuGNSW13}\\
\rowlabel{b:ChuX05}\href{../works/ChuX05.pdf}{ChuX05}~\cite{ChuX05} & 15 & scheduling, machine, release-date, order, completion-time, resource, job, due-date, Benders Decomposition, Logic-Based Benders Decomposition, one-machine scheduling, single-machine scheduling & single machine & disjunctive, cumulative, Disjunctive constraint &  & ECLiPSe &  &  &  &  & \ref{a:ChuX05} & \ref{c:ChuX05}\\
\rowlabel{b:CireCH13}\href{../works/CireCH13.pdf}{CireCH13}~\cite{CireCH13} & 7 & tardiness, scheduling, Benders Decomposition, precedence, task, order, make-span, machine, job, resource, Logic-Based Benders Decomposition, stochastic &  & circuit, cumulative &  & SCIP, OPL, Cplex &  &  &  &  & \ref{a:CireCH13} & \ref{c:CireCH13}\\
\rowlabel{b:ClercqPBJ11}\href{../works/ClercqPBJ11.pdf}{ClercqPBJ11}~\cite{ClercqPBJ11} & 16 & order, activity, release-date, scheduling, completion-time, resource, due-date, distributed, precedence &  & cumulative, SoftCumulative, Cumulatives constraint, alldifferent, SoftCumulativeSum, Cardinality constraint & Java & Choco Solver, CHIP &  &  & benchmark & time-tabling, sweep, energetic reasoning, edge-finding & \ref{a:ClercqPBJ11} & \ref{c:ClercqPBJ11}\\
\rowlabel{b:CobanH10}\href{../works/CobanH10.pdf}{CobanH10}~\cite{CobanH10} & 5 & job, make-span, distributed, tardiness, Benders Decomposition, preempt, re-scheduling, order, scheduling, Logic-Based Benders Decomposition, preemptive &  & disjunctive, circuit &  & OPL, Cplex &  &  &  &  & \ref{a:CobanH10} & \ref{c:CobanH10}\\
\rowlabel{b:CohenHB17}\href{../works/CohenHB17.pdf}{CohenHB17}~\cite{CohenHB17} & 17 & machine, order, activity, scheduling, task &  & noOverlap, alternative constraint &  & Cplex, OPL &  &  &  & time-tabling & \ref{a:CohenHB17} & \ref{c:CohenHB17}\\
\rowlabel{b:ColT19}\href{../works/ColT19.pdf}{ColT19}~\cite{ColT19} & 17 & scheduling, machine, job-shop, earliness, order, precedence, make-span, resource, job & JSSP & noOverlap, disjunctive & Java & OR-Tools, MiniZinc, CPO &  &  & github, benchmark, real-world &  & \ref{a:ColT19} & \ref{c:ColT19}\\
\rowlabel{b:Colombani96}\href{../works/Colombani96.pdf}{Colombani96}~\cite{Colombani96} & 15 & job, scheduling, resource, preempt, due-date, job-shop, task, order, activity, machine, precedence, release-date, stochastic &  & disjunctive &  & CHIP &  &  &  &  & \ref{a:Colombani96} & \ref{c:Colombani96}\\
\rowlabel{b:DannaP03}\href{../works/DannaP03.pdf}{DannaP03}~\cite{DannaP03} & 5 & job-shop, order, tardiness, scheduling, machine, job, activity, earliness, resource &  & disjunctive &  & Cplex, Ilog Solver, Ilog Scheduler &  &  & benchmark &  & \ref{a:DannaP03} & \ref{c:DannaP03}\\
\rowlabel{b:Davenport10}\href{../works/Davenport10.pdf}{Davenport10}~\cite{Davenport10} & 5 & order, resource, release-date, tardiness, scheduling, completion-time, earliness, due-date, periodic &  &  &  & Cplex & semiconductor, maintenance scheduling &  &  &  & \ref{a:Davenport10} & \ref{c:Davenport10}\\
\rowlabel{b:DavenportKRSH07}\href{../works/DavenportKRSH07.pdf}{DavenportKRSH07}~\cite{DavenportKRSH07} & 13 & make to order, activity, machine, preempt, precedence, job-shop, sequence dependent setup, resource, inventory, order, scheduling, job, setup-time &  & disjunctive, bin-packing & C++ & Cplex, CHIP &  & steel industry &  &  & \ref{a:DavenportKRSH07} & \ref{c:DavenportKRSH07}\\
\rowlabel{b:DejemeppeCS15}\href{../works/DejemeppeCS15.pdf}{DejemeppeCS15}~\cite{DejemeppeCS15} & 16 & make-span, task, precedence, setup-time, resource, preempt, activity, completion-time, tardiness, job-shop, sequence dependent setup, scheduling, release-date, machine, job, order, preemptive & single machine & disjunctive, cumulative, cycle &  &  & container terminal &  & bitbucket, real-world, generated instance, benchmark & not-last, not-first, edge-finding & \ref{a:DejemeppeCS15} & \ref{c:DejemeppeCS15}\\
\rowlabel{b:DejemeppeD14}\href{../works/DejemeppeD14.pdf}{DejemeppeD14}~\cite{DejemeppeD14} & 9 & make-span, precedence, job-shop, resource, activity, setup-time, job, scheduling, order &  & cumulative &  &  & medical, patient &  & bitbucket &  & \ref{a:DejemeppeD14} & \ref{c:DejemeppeD14}\\
\rowlabel{b:DemirovicS18}\href{../works/DemirovicS18.pdf}{DemirovicS18}~\cite{DemirovicS18} & 18 & scheduling, task, precedence, order, resource, activity &  & Disjunctive constraint, cumulative, disjunctive &  & MiniZinc, Gurobi &  &  & benchmark, real-world & time-tabling & \ref{a:DemirovicS18} & \ref{c:DemirovicS18}\\
\rowlabel{b:DerrienP14}\href{../works/DerrienP14.pdf}{DerrienP14}~\cite{DerrienP14} & 9 & resource, scheduling, make-span, activity, order & psplib, CuSP & cumulative & Java & Choco Solver &  &  & random instance & sweep, edge-finding, energetic reasoning & \ref{a:DerrienP14} & \ref{c:DerrienP14}\\
\rowlabel{b:DerrienPZ14}\href{../works/DerrienPZ14.pdf}{DerrienPZ14}~\cite{DerrienPZ14} & 9 & re-scheduling, order, job, activity, machine, precedence, make-span, scheduling, resource & RCPSP, CuSP & cumulative, Balance constraint, Cumulatives constraint &  & Choco Solver, CHIP &  &  & real-world, benchmark, random instance & sweep & \ref{a:DerrienPZ14} & \ref{c:DerrienPZ14}\\
\rowlabel{b:DilkinaDH05}\href{../works/DilkinaDH05.pdf}{DilkinaDH05}~\cite{DilkinaDH05} & 5 & machine, precedence, make-span, job, scheduling, job-shop, order, stochastic &  &  &  & OPL &  &  &  &  & \ref{a:DilkinaDH05} & \ref{c:DilkinaDH05}\\
\rowlabel{b:DoomsH08}\href{../works/DoomsH08.pdf}{DoomsH08}~\cite{DoomsH08} & 16 & scheduling, completion-time, machine, job, activity, resource, job-shop, task, order, online scheduling, stochastic & RCPSP &  &  &  &  & service industry &  &  & \ref{a:DoomsH08} & \ref{c:DoomsH08}\\
\rowlabel{b:DoulabiRP14}\href{../works/DoulabiRP14.pdf}{DoulabiRP14}~\cite{DoulabiRP14} & 9 & due-date, task, order, activity, scheduling, resource &  & Cardinality constraint, bin-packing, Element constraint &  & Cplex & medical, patient, nurse, surgery, operating room &  &  &  & \ref{a:DoulabiRP14} & \ref{c:DoulabiRP14}\\
\rowlabel{b:EdisO11}\href{../works/EdisO11.pdf}{EdisO11}~\cite{EdisO11} & 7 & task, job, resource, make-span, scheduling, flow-time, tardiness, due-date, machine, completion-time, activity, lateness, earliness, Benders Decomposition, preempt, Logic-Based Benders Decomposition & parallel machine & bin-packing, noOverlap, cumulative &  & OPL, Cplex &  &  &  &  & \ref{a:EdisO11} & \ref{c:EdisO11}\\
\rowlabel{b:EfthymiouY23}\href{../works/EfthymiouY23.pdf}{EfthymiouY23}~\cite{EfthymiouY23} & 16 & setup-time, order, make-span, job-shop, job, re-scheduling, task, scheduling, machine & CHSP, JSSP & cumulative, disjunctive, cycle & Python & OPL, OR-Tools & pipeline, hoist, satellite, electroplating &  & generated instance, benchmark, random instance, real-life, industrial instance &  & \ref{a:EfthymiouY23} & \ref{c:EfthymiouY23}\\
\rowlabel{b:ElkhyariGJ02}\href{../works/ElkhyariGJ02.pdf}{ElkhyariGJ02}~\cite{ElkhyariGJ02} & 6 & precedence, scheduling, machine, preempt, make-span, resource, activity, due-date, re-scheduling, task & RCPSP & cumulative, disjunctive, table constraint &  &  &  &  &  &  & \ref{a:ElkhyariGJ02} & \ref{c:ElkhyariGJ02}\\
\rowlabel{b:ElkhyariGJ02a}\href{../works/ElkhyariGJ02a.pdf}{ElkhyariGJ02a}~\cite{ElkhyariGJ02a} & 24 & activity, re-scheduling, order, scheduling, open-shop, due-date, task, precedence, resource, online scheduling & RCPSP, psplib & cumulative, Disjunctive constraint, Arithmetic constraint, disjunctive &  & OPL &  &  & benchmark, real-life & time-tabling & \ref{a:ElkhyariGJ02a} & \ref{c:ElkhyariGJ02a}\\
\rowlabel{b:ErtlK91}\href{../works/ErtlK91.pdf}{ErtlK91}~\cite{ErtlK91} & 12 & setup-time, task, resource, scheduling, order, machine &  & cycle & Prolog &  & pipeline &  & real-world, benchmark &  & \ref{a:ErtlK91} & \ref{c:ErtlK91}\\
\rowlabel{b:EvenSH15}\href{../works/EvenSH15.pdf}{EvenSH15}~\cite{EvenSH15} & 18 & transportation, machine, distributed, resource, preempt, order, scheduling, Benders Decomposition, completion-time, task, preemptive &  & cumulative, disjunctive, Disjunctive constraint &  & OPL, Choco Solver & emergency service &  & real-life, real-world & sweep & \ref{a:EvenSH15} & \ref{c:EvenSH15}\\
\rowlabel{b:FocacciLN00}\href{../works/FocacciLN00.pdf}{FocacciLN00}~\cite{FocacciLN00} & 10 & machine, preempt, cmax, scheduling, resource, setup-time, due-date, task, job-shop, distributed, precedence, make-span, sequence dependent setup, open-shop, order, job, activity &  & Disjunctive constraint, disjunctive &  &  &  &  & real-world & edge-finding & \ref{a:FocacciLN00} & \ref{c:FocacciLN00}\\
\rowlabel{b:FontaineMH16}\href{../works/FontaineMH16.pdf}{FontaineMH16}~\cite{FontaineMH16} & 11 & order, job-shop, resource, scheduling, machine, job, task, completion-time, Benders Decomposition, make-span, precedence, Logic-Based Benders Decomposition & parallel machine & disjunctive &  & MiniZinc, Gurobi, CHIP &  &  & benchmark &  & \ref{a:FontaineMH16} & \ref{c:FontaineMH16}\\
\rowlabel{b:FortinZDF05}\href{../works/FortinZDF05.pdf}{FortinZDF05}~\cite{FortinZDF05} & 15 & resource, task, order, activity, precedence, temporal constraint reasoning, make-span, scheduling, stochastic & psplib &  &  &  &  &  &  &  & \ref{a:FortinZDF05} & \ref{c:FortinZDF05}\\
\rowlabel{b:FrankK05}\href{../works/FrankK05.pdf}{FrankK05}~\cite{FrankK05} & 18 & order, job, resource, precedence, scheduling, due-date, task, periodic, stochastic &  & cycle &  &  & satellite, aircraft &  & benchmark &  & \ref{a:FrankK05} & \ref{c:FrankK05}\\
\rowlabel{b:FrimodigS19}\href{../works/FrimodigS19.pdf}{FrimodigS19}~\cite{FrimodigS19} & 17 & order, machine, job, scheduling, resource, Benders Decomposition, task, job-shop, stochastic &  & cumulative, bin-packing, regular expression, Regular constraint & Python & Cplex, MiniZinc, Gecode & medical, patient, nurse, physician, radiation therapy, surgery &  & benchmark, real-world &  & \ref{a:FrimodigS19} & \ref{c:FrimodigS19}\\
\rowlabel{b:FrohnerTR19}\href{../works/FrohnerTR19.pdf}{FrohnerTR19}~\cite{FrohnerTR19} & 9 & order, scheduling, distributed &  &  & Java, Python & MiniZinc, Gecode, Gurobi & nurse &  & benchmark, real-world &  & \ref{a:FrohnerTR19} & \ref{c:FrohnerTR19}\\
\rowlabel{b:FrostD98}\href{../works/FrostD98.pdf}{FrostD98}~\cite{FrostD98} & 1 & scheduling, order &  &  &  &  & maintenance scheduling & power industry &  &  & \ref{a:FrostD98} & \ref{c:FrostD98}\\
\rowlabel{b:GalleguillosKSB19}\href{../works/GalleguillosKSB19.pdf}{GalleguillosKSB19}~\cite{GalleguillosKSB19} & 18 & resource, order, job, activity, make-span, re-scheduling, machine, distributed, scheduling, stochastic & JSSP & alternative constraint, cumulative & Python & OR-Tools & datacenter, super-computer &  &  &  & \ref{a:GalleguillosKSB19} & \ref{c:GalleguillosKSB19}\\
\rowlabel{b:GarganiR07}\href{../works/GarganiR07.pdf}{GarganiR07}~\cite{GarganiR07} & 13 & machine, inventory, order, resource &  & bin-packing, Channeling constraint, Element constraint & C++ & OPL & steel mill & steel industry & real-life, CSPlib &  & \ref{a:GarganiR07} & \ref{c:GarganiR07}\\
\rowlabel{b:GayHLS15}\href{../works/GayHLS15.pdf}{GayHLS15}~\cite{GayHLS15} & 9 & resource, scheduling, precedence, task, order, make-span, activity & RCPSP, OSP, psplib & cumulative, disjunctive &  &  &  &  & bitbucket, benchmark & time-tabling, edge-finding & \ref{a:GayHLS15} & \ref{c:GayHLS15}\\
\rowlabel{b:GayHS15}\href{../works/GayHS15.pdf}{GayHS15}~\cite{GayHS15} & 9 & resource, task, order, scheduling, precedence, preempt, preemptive &  & Cumulatives constraint, cumulative, table constraint, disjunctive &  & Choco Solver, OR-Tools, Gecode &  &  & bitbucket & time-tabling, sweep & \ref{a:GayHS15} & \ref{c:GayHS15}\\
\rowlabel{b:GayHS15a}\href{../works/GayHS15a.pdf}{GayHS15a}~\cite{GayHS15a} & 16 & task, order, machine, manpower, preempt, resource, scheduling, preemptive & psplib, RCPSP & Cumulatives constraint, cumulative, disjunctive & Java &  &  &  & benchmark, real-world, bitbucket & time-tabling, not-first, not-last, energetic reasoning, edge-finding, sweep & \ref{a:GayHS15a} & \ref{c:GayHS15a}\\
\rowlabel{b:GaySS14}\href{../works/GaySS14.pdf}{GaySS14}~\cite{GaySS14} & 15 & machine, completion-time, activity, setup-time, continuous-process, resource, job, order, make-span, scheduling, precedence, manpower, job-shop &  & cycle, cumulative, disjunctive &  &  & steel mill &  & real-life, CSPlib & sweep & \ref{a:GaySS14} & \ref{c:GaySS14}\\
\rowlabel{b:GeibingerKKMMW21}\href{../works/GeibingerKKMMW21.pdf}{GeibingerKKMMW21}~\cite{GeibingerKKMMW21} & 10 & scheduling, distributed &  & Cardinality constraint &  & MiniZinc, OR-Tools, Gurobi, Cplex, Gecode & nurse, physician, COVID, medical, patient & pharmaceutical industry & real-world &  & \ref{a:GeibingerKKMMW21} & \ref{c:GeibingerKKMMW21}\\
\rowlabel{b:GeibingerMM19}\href{../works/GeibingerMM19.pdf}{GeibingerMM19}~\cite{GeibingerMM19} & 16 & precedence, release-date, resource, activity, re-scheduling, job, order, completion-time, scheduling, due-date, make-span, task & RCPSP & alternative constraint, cumulative, endBeforeStart, Pulse constraint, noOverlap & Java & Cplex, Gecode, MiniZinc, CPO & automotive &  & real-world, benchmark, real-life, generated instance, industrial partner & time-tabling & \ref{a:GeibingerMM19} & \ref{c:GeibingerMM19}\\
\rowlabel{b:GeibingerMM21}\href{../works/GeibingerMM21.pdf}{GeibingerMM21}~\cite{GeibingerMM21} & 9 & precedence, release-date, resource, activity, job, order, completion-time, tardiness, scheduling, machine, lazy clause generation, due-date, task & RCPSP & disjunctive, cumulative &  & Chuffed, Cplex, CPO & nurse, train schedule, operating room &  & github, real-world, benchmark, real-life, generated instance & time-tabling & \ref{a:GeibingerMM21} & \ref{c:GeibingerMM21}\\
\rowlabel{b:GeitzGSSW22}\href{../works/GeitzGSSW22.pdf}{GeitzGSSW22}~\cite{GeitzGSSW22} & 18 & setup-time, sequence dependent setup, task, lateness, precedence, batch process, make-span, order, job, scheduling, completion-time, resource, machine, preempt, producer/consumer, lazy clause generation, job-shop, transportation, preemptive & single machine, RCPSP, JSSP & cumulative &  & OPL & robot &  & real-world, real-life, github & sweep, not-last & \ref{a:GeitzGSSW22} & \ref{c:GeitzGSSW22}\\
\rowlabel{b:GelainPRVW17}\href{../works/GelainPRVW17.pdf}{GelainPRVW17}~\cite{GelainPRVW17} & 16 & order, resource, scheduling &  &  &  &  &  &  & real-life, CSPlib, benchmark &  & \ref{a:GelainPRVW17} & \ref{c:GelainPRVW17}\\
\rowlabel{b:Geske05}\href{../works/Geske05.pdf}{Geske05}~\cite{Geske05} & 18 & machine, re-scheduling, activity, distributed, task, job, order, resource, scheduling, lateness, job-shop &  & cumulative & Prolog & SICStus, CHIP & train schedule, railway & railway industry & real-life &  & \ref{a:Geske05} & \ref{c:Geske05}\\
\rowlabel{b:GilesH16}\href{../works/GilesH16.pdf}{GilesH16}~\cite{GilesH16} & 16 & setup-time, activity, transportation, resource, inventory, task, order, scheduling &  & disjunctive, cumulative &  & Cplex & pipeline & chemical industry, processing industry, petro-chemical industry, chemical processing industry &  &  & \ref{a:GilesH16} & \ref{c:GilesH16}\\
\rowlabel{b:GingrasQ16}\href{../works/GingrasQ16.pdf}{GingrasQ16}~\cite{GingrasQ16} & 7 & resource, scheduling, task, make-span, completion-time, precedence, order & psplib, RCPSP, CuSP & disjunctive, cumulative &  & Choco Solver &  &  & benchmark & energetic reasoning, sweep, edge-finder, edge-finding & \ref{a:GingrasQ16} & \ref{c:GingrasQ16}\\
\rowlabel{b:GodardLN05}\href{../works/GodardLN05.pdf}{GodardLN05}~\cite{GodardLN05} & 9 & job-shop, activity, completion-time, order, earliness, tardiness, resource, scheduling, machine, make-span, job, precedence & JSSP & cumulative, disjunctive, table constraint &  & Ilog Solver, Ilog Scheduler &  &  & benchmark &  & \ref{a:GodardLN05} & \ref{c:GodardLN05}\\
\rowlabel{b:GodetLHS20}\href{../works/GodetLHS20.pdf}{GodetLHS20}~\cite{GodetLHS20} & 8 & lazy clause generation, release-date, scheduling, task, machine, make-span, completion-time, setup-time, order, cmax, resource, job & single machine, parallel machine, PMSP & alldifferent, bin-packing, Disjunctive constraint, cumulative, disjunctive &  & CHIP, Chuffed, Choco Solver & satellite &  & real-life, benchmark, generated instance, github & not-last, time-tabling & \ref{a:GodetLHS20} & \ref{c:GodetLHS20}\\
\rowlabel{b:GokGSTO20}\href{../works/GokGSTO20.pdf}{GokGSTO20}~\cite{GokGSTO20} & 17 & distributed, task, job-shop, resource, multi-agent, job, setup-time, scheduling, precedence, order, tardiness, activity, stochastic & RCPSP & cumulative, circuit, disjunctive & Python & Gecode, Z3, MiniZinc, Gurobi & aircraft &  & real-world, Roadef & GRASP & \ref{a:GokGSTO20} & \ref{c:GokGSTO20}\\
\rowlabel{b:GoldwaserS17}\href{../works/GoldwaserS17.pdf}{GoldwaserS17}~\cite{GoldwaserS17} & 16 & scheduling, machine, transportation, order, resource, due-date, lazy clause generation, Benders Decomposition, Logic-Based Benders Decomposition &  & cumulative, disjunctive & Python & Gurobi, Gecode & torpedo & steel industry & github, generated instance, instance generator &  & \ref{a:GoldwaserS17} & \ref{c:GoldwaserS17}\\
\rowlabel{b:Goltz95}\href{../works/Goltz95.pdf}{Goltz95}~\cite{Goltz95} & 14 & task, job, order, resource, scheduling, precedence, job-shop, due-date, machine, completion-time &  & cumulative, disjunctive & Prolog & CHIP &  &  & benchmark & edge-finding & \ref{a:Goltz95} & \ref{c:Goltz95}\\
\rowlabel{b:GomesHS06}\href{../works/GomesHS06.pdf}{GomesHS06}~\cite{GomesHS06} & 2 & order, scheduling, distributed, task, multi-agent &  &  &  & Ilog Solver &  &  & real-life &  & \ref{a:GomesHS06} & \ref{c:GomesHS06}\\
\rowlabel{b:GrimesH10}\href{../works/GrimesH10.pdf}{GrimesH10}~\cite{GrimesH10} & 15 & cmax, machine, job, job-shop, setup-time, flow-shop, no-wait, open-shop, scheduling, precedence, order, make-span, sequence dependent setup, task, batch process, resource & Open Shop Scheduling Problem & cycle, disjunctive, Disjunctive constraint, cumulative &  &  &  & steel industry & benchmark & time-tabling, edge-finding & \ref{a:GrimesH10} & \ref{c:GrimesH10}\\
\rowlabel{b:GrimesH11}\href{../works/GrimesH11.pdf}{GrimesH11}~\cite{GrimesH11} & 17 & cmax, machine, job, job-shop, flow-shop, no-wait, open-shop, scheduling, precedence, order, make-span, completion-time, tardiness, release-date, earliness, lazy clause generation, task, due-date, resource & RCPSP & disjunctive, Disjunctive constraint, cumulative &  & Cplex, Ilog Solver, OPL, Ilog Scheduler &  &  & benchmark & edge-finding & \ref{a:GrimesH11} & \ref{c:GrimesH11}\\
\rowlabel{b:GrimesHM09}\href{../works/GrimesHM09.pdf}{GrimesHM09}~\cite{GrimesHM09} & 9 & open-shop, order, make-span, resource, job, precedence, scheduling, task, job-shop, machine & OSP, Open Shop Scheduling Problem & Balance constraint, disjunctive, Disjunctive constraint & Java & Ilog Scheduler, Choco Solver, Mistral &  &  & benchmark & edge-finding, not-last & \ref{a:GrimesHM09} & \ref{c:GrimesHM09}\\
\rowlabel{b:GroleazNS20}\href{../works/GroleazNS20.pdf}{GroleazNS20}~\cite{GroleazNS20} & 17 & precedence, release-date, job, scheduling, resource, machine, preempt, due-date, tardiness, job-shop, setup-time, order, inventory, preemptive & GCSP & circuit, noOverlap, cycle, cumulative &  & OR-Tools, CPO &  & food industry & industrial instance, benchmark &  & \ref{a:GroleazNS20} & \ref{c:GroleazNS20}\\
\rowlabel{b:GroleazNS20a}\href{../works/GroleazNS20a.pdf}{GroleazNS20a}~\cite{GroleazNS20a} & 9 & scheduling, machine, transportation, order, tardiness, release-date, precedence, resource, setup-time, preempt, inventory, due-date, distributed, job, preemptive & parallel machine, RCPSP & noOverlap, cumulative, cycle &  & Cplex, CPO &  & food industry & industrial partner, benchmark & GRASP & \ref{a:GroleazNS20a} & \ref{c:GroleazNS20a}\\
\rowlabel{b:GruianK98}\href{../works/GruianK98.pdf}{GruianK98}~\cite{GruianK98} & 8 & task, resource, re-scheduling, scheduling, order, activity &  & cumulative, cycle, circuit, diffn &  & OPL, CHIP & pipeline, aircraft &  & benchmark &  & \ref{a:GruianK98} & \ref{c:GruianK98}\\
\rowlabel{b:GuSS13}\href{../works/GuSS13.pdf}{GuSS13}~\cite{GuSS13} & 7 & lazy clause generation, activity, order, precedence, make-span, resource, distributed, scheduling, machine, single-machine scheduling & single machine & cumulative &  &  &  &  & benchmark & edge-finding, edge-finder, time-tabling & \ref{a:GuSS13} & \ref{c:GuSS13}\\
\rowlabel{b:GuSW12}\href{../works/GuSW12.pdf}{GuSW12}~\cite{GuSW12} & 15 & lazy clause generation, activity, order, precedence, make-span, resource, job, preempt, scheduling, cmax, preemptive &  & cumulative & C++ &  &  &  & benchmark &  & \ref{a:GuSW12} & \ref{c:GuSW12}\\
\rowlabel{b:HanenKP21}\href{../works/HanenKP21.pdf}{HanenKP21}~\cite{HanenKP21} & 17 & job-shop, resource, machine, precedence, order, tardiness, preempt, release-date, scheduling, make-span, completion-time, task, cmax, job, lateness, due-date, preemptive & RCPSP, CuSP, parallel machine & cumulative & Python & Claire & pipeline &  & Roadef, generated instance, random instance & energetic reasoning & \ref{a:HanenKP21} & \ref{c:HanenKP21}\\
\rowlabel{b:He0GLW18}\href{../works/He0GLW18.pdf}{He0GLW18}~\cite{He0GLW18} & 18 & machine, transportation, multi-agent, distributed, precedence, re-scheduling, order, scheduling &  &  & Python & Gurobi & energy-price, real-time pricing &  & real-world, bitbucket &  & \ref{a:He0GLW18} & \ref{c:He0GLW18}\\
\rowlabel{b:HebrardALLCMR22}\href{../works/HebrardALLCMR22.pdf}{HebrardALLCMR22}~\cite{HebrardALLCMR22} & 7 & order, scheduling, activity &  & cumulative & Julia & Claire & deep space &  &  & sweep & \ref{a:HebrardALLCMR22} & \ref{c:HebrardALLCMR22}\\
\rowlabel{b:HebrardTW05}\href{../works/HebrardTW05.pdf}{HebrardTW05}~\cite{HebrardTW05} & 1 & job-shop, order, job, machine, scheduling &  &  &  &  &  &  &  &  & \ref{a:HebrardTW05} & \ref{c:HebrardTW05}\\
\rowlabel{b:HechingH16}\href{../works/HechingH16.pdf}{HechingH16}~\cite{HechingH16} & 11 & order, scheduling, manpower, re-scheduling, job, Benders Decomposition, task, Logic-Based Benders Decomposition, stochastic &  & circuit, noOverlap &  & OPL, Cplex & patient, medical &  & real-world &  & \ref{a:HechingH16} & \ref{c:HechingH16}\\
\rowlabel{b:HeinzB12}\href{../works/HeinzB12.pdf}{HeinzB12}~\cite{HeinzB12} & 17 & precedence, due-date, order, tardiness, scheduling, completion-time, machine, job, activity, release-date, earliness, resource, Benders Decomposition, Logic-Based Benders Decomposition, single-machine scheduling & single machine & cumulative, Channeling constraint, cycle, alternative constraint, IloAlternative &  & SCIP, Ilog Solver, OPL, Cplex, Ilog Scheduler &  &  &  & GRASP & \ref{a:HeinzB12} & \ref{c:HeinzB12}\\
\rowlabel{b:HeinzKB13}\href{../works/HeinzKB13.pdf}{HeinzKB13}~\cite{HeinzKB13} & 16 & release-date, job-shop, resource, machine, job, scheduling, Benders Decomposition, order, tardiness, Logic-Based Benders Decomposition & single machine & cumulative, Channeling constraint &  & SCIP, Cplex, OPL &  &  &  &  & \ref{a:HeinzKB13} & \ref{c:HeinzKB13}\\
\rowlabel{b:HeinzS11}\href{../works/HeinzS11.pdf}{HeinzS11}~\cite{HeinzS11} & 10 & preempt, order, scheduling, completion-time, machine, job, resource & psplib, RCPSP & disjunctive, cumulative &  & SCIP, Cplex &  &  & benchmark & time-tabling, energetic reasoning & \ref{a:HeinzS11} & \ref{c:HeinzS11}\\
\rowlabel{b:HentenryckM04}\href{../works/HentenryckM04.pdf}{HentenryckM04}~\cite{HentenryckM04} & 16 & resource, activity, job, completion-time, tardiness, scheduling, machine, open-shop, order, due-date, make-span, task, job-shop, precedence &  & disjunctive, cumulative, cycle &  &  &  &  & benchmark &  & \ref{a:HentenryckM04} & \ref{c:HentenryckM04}\\
\rowlabel{b:HentenryckM08}\href{../works/HentenryckM08.pdf}{HentenryckM08}~\cite{HentenryckM08} & 5 & order &  & bin-packing &  &  & steel mill &  & CSPlib &  & \ref{a:HentenryckM08} & \ref{c:HentenryckM08}\\
\rowlabel{b:HermenierDL11}\href{../works/HermenierDL11.pdf}{HermenierDL11}~\cite{HermenierDL11} & 15 & task, precedence, distributed, resource, completion-time, producer/consumer, machine, no-wait, order, scheduling, periodic &  & bin-packing, disjunctive, table constraint, alldifferent, cumulative, cycle &  & Choco Solver & datacenter &  &  &  & \ref{a:HermenierDL11} & \ref{c:HermenierDL11}\\
\rowlabel{b:HillTV21}\href{../works/HillTV21.pdf}{HillTV21}~\cite{HillTV21} & 19 & machine, job, activity, resource, release-date, precedence, preempt, lazy clause generation, scheduling, flow-shop, task, order, make-span, preemptive, single-machine scheduling & RCPSP, psplib, single machine & cycle, cumulative, alternative constraint &  &  &  &  & real-world &  & \ref{a:HillTV21} & \ref{c:HillTV21}\\
\rowlabel{b:HoYCLLCLC18}\href{../works/HoYCLLCLC18.pdf}{HoYCLLCLC18}~\cite{HoYCLLCLC18} & 6 & task, distributed, order, job, scheduling, resource, machine, re-scheduling, stochastic &  &  & C  &  & medical, patient, nurse &  & real-world &  & \ref{a:HoYCLLCLC18} & \ref{c:HoYCLLCLC18}\\
\rowlabel{b:HoeveGSL07}\href{../works/HoeveGSL07.pdf}{HoeveGSL07}~\cite{HoeveGSL07} & 6 & resource, multi-agent, scheduling, re-scheduling, job, precedence, distributed, task, job-shop, machine, order &  & disjunctive &  & Ilog Scheduler, Cplex &  &  & benchmark & edge-finding & \ref{a:HoeveGSL07} & \ref{c:HoeveGSL07}\\
\rowlabel{b:Hooker04}\href{../works/Hooker04.pdf}{Hooker04}~\cite{Hooker04} & 12 & machine, task, release-date, make-span, distributed, resource, precedence, order, tardiness, scheduling, Benders Decomposition, Logic-Based Benders Decomposition &  & disjunctive, cumulative, circuit &  & OPL, Ilog Scheduler, Cplex &  &  & random instance &  & \ref{a:Hooker04} & \ref{c:Hooker04}\\
\rowlabel{b:Hooker05a}\href{../works/Hooker05a.pdf}{Hooker05a}~\cite{Hooker05a} & 14 & release-date, scheduling, make-span, task, machine, job, due-date, resource, Benders Decomposition, precedence, order, tardiness, Logic-Based Benders Decomposition &  & circuit, cumulative, disjunctive &  & Ilog Scheduler, OPL, Cplex &  &  &  &  & \ref{a:Hooker05a} & \ref{c:Hooker05a}\\
\rowlabel{b:Hooker17}\href{../works/Hooker17.pdf}{Hooker17}~\cite{Hooker17} & 14 & job, resource, due-date, order, tardiness, scheduling &  & circuit &  &  &  &  & benchmark, random instance &  & \ref{a:Hooker17} & \ref{c:Hooker17}\\
\rowlabel{b:HookerY02}\href{../works/HookerY02.pdf}{HookerY02}~\cite{HookerY02} & 5 & scheduling, machine, job, resource, Benders Decomposition, order, Logic-Based Benders Decomposition & RCPSP & cumulative, disjunctive &  &  &  &  &  &  & \ref{a:HookerY02} & \ref{c:HookerY02}\\
\rowlabel{b:HoundjiSWD14}\href{../works/HoundjiSWD14.pdf}{HoundjiSWD14}~\cite{HoundjiSWD14} & 16 & scheduling, machine, transportation, order, precedence, resource, inventory, due-date & single machine & circuit, Cardinality constraint, Element constraint, GCC constraint &  &  &  &  & bitbucket, generated instance &  & \ref{a:HoundjiSWD14} & \ref{c:HoundjiSWD14}\\
\rowlabel{b:IfrimOS12}\href{../works/IfrimOS12.pdf}{IfrimOS12}~\cite{IfrimOS12} & 16 & order, scheduling, task, machine, job, re-scheduling, distributed, due-date, resource, periodic, stochastic &  & disjunctive &  &  & datacenter, energy-price &  & real-life &  & \ref{a:IfrimOS12} & \ref{c:IfrimOS12}\\
\rowlabel{b:JelinekB16}\href{../works/JelinekB16.pdf}{JelinekB16}~\cite{JelinekB16} & 10 & completion-time, order, scheduling, task &  & cumulative, table constraint & Prolog & SICStus, OPL &  &  & real-life &  & \ref{a:JelinekB16} & \ref{c:JelinekB16}\\
\rowlabel{b:JungblutK22}\href{../works/JungblutK22.pdf}{JungblutK22}~\cite{JungblutK22} & 4 & distributed, machine, make-span, scheduling, resource, preempt, task, order &  & circuit &  & MiniZinc &  &  & benchmark, github, real-world &  & \ref{a:JungblutK22} & \ref{c:JungblutK22}\\
\rowlabel{b:JuvinHHL23}\href{../works/JuvinHHL23.pdf}{JuvinHHL23}~\cite{JuvinHHL23} & 16 & resource, job, scheduling, task, job-shop, due-date, machine, make-span, flow-shop, completion-time, precedence, Benders Decomposition, cmax, setup-time, order, preempt, Logic-Based Benders Decomposition, preemptive & JSSP, parallel machine & disjunctive, Disjunctive constraint, PreemptiveNoOverlap, alldifferent, noOverlap, endBeforeStart, AllDiffPrec constraint, cumulative & C++ & CPO, Mistral &  &  & github, benchmark, supplementary material & not-last, edge-finding, not-first & \ref{a:JuvinHHL23} & \ref{c:JuvinHHL23}\\
\rowlabel{b:JuvinHL23}\href{../works/JuvinHL23.pdf}{JuvinHL23}~\cite{JuvinHL23} & 16 & precedence, order, tardiness, setup-time, scheduling, make-span, completion-time, task, cmax, machine, job, job-shop, flow-shop, stochastic &  & noOverlap, endBeforeStart &  & Cplex, CPO &  &  & real-world &  & \ref{a:JuvinHL23} & \ref{c:JuvinHL23}\\
\rowlabel{b:KamarainenS02}\href{../works/KamarainenS02.pdf}{KamarainenS02}~\cite{KamarainenS02} & 17 & job-shop, resource, earliness, activity, job, order, scheduling, machine, precedence, transportation, preempt, preemptive & KRFP &  &  & ECLiPSe &  &  & real-world, benchmark &  & \ref{a:KamarainenS02} & \ref{c:KamarainenS02}\\
\rowlabel{b:KameugneFGOQ18}\href{../works/KameugneFGOQ18.pdf}{KameugneFGOQ18}~\cite{KameugneFGOQ18} & 17 & cmax, precedence, make-span, completion-time, resource, task, scheduling, order & RCPSP, CuSP & Disjunctive constraint, cumulative, disjunctive & Java & CHIP, Choco Solver &  &  & real-world, benchmark & time-tabling, sweep, not-last, energetic reasoning, not-first & \ref{a:KameugneFGOQ18} & \ref{c:KameugneFGOQ18}\\
\rowlabel{b:KameugneFND23}\href{../works/KameugneFND23.pdf}{KameugneFND23}~\cite{KameugneFND23} & 17 & precedence, cmax, preempt, make-span, task, completion-time, machine, resource, order, scheduling, lazy clause generation & RCPSP, psplib, CuSP & Disjunctive constraint, disjunctive, Cumulatives constraint, cumulative & Java & Choco Solver, CHIP &  &  & benchmark & sweep, energetic reasoning, not-last, not-first, edge-finder, time-tabling, edge-finding & \ref{a:KameugneFND23} & \ref{c:KameugneFND23}\\
\rowlabel{b:KameugneFSN11}\href{../works/KameugneFSN11.pdf}{KameugneFSN11}~\cite{KameugneFSN11} & 15 & completion-time, job-shop, release-date, resource, job, order, scheduling, precedence, preempt, make-span, task, preemptive & RCPSP, psplib, CuSP & cumulative, disjunctive &  & Gecode &  &  & benchmark & edge-finding, not-last, not-first, time-tabling & \ref{a:KameugneFSN11} & \ref{c:KameugneFSN11}\\
\rowlabel{b:KelarevaTK13}\href{../works/KelarevaTK13.pdf}{KelarevaTK13}~\cite{KelarevaTK13} & 17 & re-scheduling, task, Benders Decomposition, precedence, scheduling, transportation, setup-time, order, tardiness, make-span, resource, activity, lazy clause generation, inventory & Liner Shipping Fleet Repositioning Problem, BPCTOP, LSFRP, Bulk Port Cargo Throughput Optimisation Problem & alldifferent &  & Cplex, SCIP, MiniZinc & earth observation, shipping line, satellite &  & real-world &  & \ref{a:KelarevaTK13} & \ref{c:KelarevaTK13}\\
\rowlabel{b:KeriK07}\href{../works/KeriK07.pdf}{KeriK07}~\cite{KeriK07} & 14 & due-date, activity, earliness, resource, tardiness, job, temporal constraint reasoning, order, make-span, scheduling, precedence, cmax, job-shop & RCPSP & cycle & C++ &  &  &  &  & edge-finding & \ref{a:KeriK07} & \ref{c:KeriK07}\\
\rowlabel{b:KhemmoudjPB06}\href{../works/KhemmoudjPB06.pdf}{KhemmoudjPB06}~\cite{KhemmoudjPB06} & 13 & distributed, resource, stock level, order, scheduling &  & cycle, cumulative & C++ & CHIP &  &  & real-world &  & \ref{a:KhemmoudjPB06} & \ref{c:KhemmoudjPB06}\\
\rowlabel{b:KimCMLLP23}\href{../works/KimCMLLP23.pdf}{KimCMLLP23}~\cite{KimCMLLP23} & 16 & open-shop, tardiness, earliness, scheduling, transportation, machine, make-span, job, precedence, distributed, setup-time, job-shop, due-date, order & parallel machine, SCC & noOverlap & Python & OR-Tools, Gurobi &  & steel industry & real-world, zenodo, benchmark &  & \ref{a:KimCMLLP23} & \ref{c:KimCMLLP23}\\
\rowlabel{b:KlankeBYE21}\href{../works/KlankeBYE21.pdf}{KlankeBYE21}~\cite{KlankeBYE21} & 16 & make-span, order, job, activity, scheduling, completion-time, resource, machine, producer/consumer, job-shop, re-scheduling, due-date, task, batch process &  & circuit, noOverlap, disjunctive, cumulative & Python & CHIP, OR-Tools, Gurobi, Cplex &  & processing industry, food-processing industry & random instance, benchmark, real-life &  & \ref{a:KlankeBYE21} & \ref{c:KlankeBYE21}\\
\rowlabel{b:KletzanderM17}\href{../works/KletzanderM17.pdf}{KletzanderM17}~\cite{KletzanderM17} & 15 & machine, resource, order, scheduling, transportation & parallel machine &  &  &  & torpedo & steel industry &  &  & \ref{a:KletzanderM17} & \ref{c:KletzanderM17}\\
\rowlabel{b:KorbaaYG99}\href{../works/KorbaaYG99.pdf}{KorbaaYG99}~\cite{KorbaaYG99} & 8 & resource, scheduling, transportation, make-span, job, task, job-shop, machine, flow-shop, order, periodic &  & circuit, cycle & Prolog & Ilog Solver, CHIP, OZ & robot, hoist &  &  &  & \ref{a:KorbaaYG99} & \ref{c:KorbaaYG99}\\
\rowlabel{b:KoschB14}\href{../works/KoschB14.pdf}{KoschB14}~\cite{KoschB14} & 16 & resource, lateness, job-shop, release-date, multi-agent, cmax, scheduling, Benders Decomposition, completion-time, batch process, due-date, order, make-span, machine, job, distributed, Logic-Based Benders Decomposition & RCPSP, single machine & cumulative, disjunctive, bin-packing & Java & Choco Solver, Cplex & semiconductor &  & benchmark &  & \ref{a:KoschB14} & \ref{c:KoschB14}\\
\rowlabel{b:KovacsB07}\href{../works/KovacsB07.pdf}{KovacsB07}~\cite{KovacsB07} & 15 & order, tardiness, activity, preempt, release-date, earliness, scheduling, make-span, completion-time, job, due-date, job-shop, flow-shop, resource, machine, preemptive, single-machine scheduling & parallel machine, single machine & Completion constraint, cumulative & C++ & Ilog Solver &  &  & benchmark &  & \ref{a:KovacsB07} & \ref{c:KovacsB07}\\
\rowlabel{b:KovacsEKV05}\href{../works/KovacsEKV05.pdf}{KovacsEKV05}~\cite{KovacsEKV05} & 1 & scheduling, resource, setup-time, job-shop, precedence, job &  &  &  &  &  &  & real-life &  & \ref{a:KovacsEKV05} & \ref{c:KovacsEKV05}\\
\rowlabel{b:KovacsTKSG21}\href{../works/KovacsTKSG21.pdf}{KovacsTKSG21}~\cite{KovacsTKSG21} & 17 & precedence, job-shop, preempt, order, tardiness, inventory, distributed, resource, due-date, scheduling, machine, flow-shop, job, re-scheduling, task, release-date, preemptive & RCPSP, single machine & cumulative &  & Gurobi, OR-Tools, Cplex &  &  & github, supplementary material, real-world, benchmark &  & \ref{a:KovacsTKSG21} & \ref{c:KovacsTKSG21}\\
\rowlabel{b:KovacsV04}\href{../works/KovacsV04.pdf}{KovacsV04}~\cite{KovacsV04} & 15 & scheduling, make-span, task, job, job-shop, resource, machine, precedence, order & single machine & disjunctive, cumulative &  & Ilog Scheduler &  &  & industrial partner, benchmark, real-life & edge-finding & \ref{a:KovacsV04} & \ref{c:KovacsV04}\\
\rowlabel{b:KovacsV06}\href{../works/KovacsV06.pdf}{KovacsV06}~\cite{KovacsV06} & 13 & tardiness, setup-time, earliness, scheduling, make-span, task, job, job-shop, resource, machine, precedence, order & single machine, RCPSP & cumulative &  & Ilog Scheduler & automotive & energy industry & industrial partner, benchmark, generated instance &  & \ref{a:KovacsV06} & \ref{c:KovacsV06}\\
\rowlabel{b:KreterSS15}\href{../works/KreterSS15.pdf}{KreterSS15}~\cite{KreterSS15} & 17 & order, preempt, resource, lazy clause generation, scheduling, task, machine, activity, make-span, completion-time, periodic, preemptive & RCPSP, parallel machine & cumulative, diffn, Element constraint, Calendar constraint &  & Cplex, MiniZinc, CHIP, Chuffed &  &  & benchmark &  & \ref{a:KreterSS15} & \ref{c:KreterSS15}\\
\rowlabel{b:KrogtLPHJ07}\href{../works/KrogtLPHJ07.pdf}{KrogtLPHJ07}~\cite{KrogtLPHJ07} & 13 & resource, due-date, job-shop, precedence, order, job, inventory, activity, machine, scheduling &  & circuit & Prolog & OPL & semiconductor, aircraft & semiconductor industry & real-world &  & \ref{a:KrogtLPHJ07} & \ref{c:KrogtLPHJ07}\\
\rowlabel{b:KucukY19}\href{../works/KucukY19.pdf}{KucukY19}~\cite{KucukY19} & 5 & distributed, resource, sequence dependent setup, task, order, scheduling, setup-time, stochastic &  & disjunctive, noOverlap, cycle &  & Cplex & earth observation, satellite &  & benchmark, generated instance & time-tabling & \ref{a:KucukY19} & \ref{c:KucukY19}\\
\rowlabel{b:Kumar03}\href{../works/Kumar03.pdf}{Kumar03}~\cite{Kumar03} & 15 & order, scheduling, producer/consumer, activity, resource &  & cycle &  &  &  &  &  & max-flow, bi-partite matching & \ref{a:Kumar03} & \ref{c:Kumar03}\\
\rowlabel{b:Laborie09}\href{../works/Laborie09.pdf}{Laborie09}~\cite{Laborie09} & 15 & task, machine, job, sequence dependent setup, inventory, due-date, job-shop, preempt, resource, precedence, order, tardiness, activity, setup-time, release-date, earliness, scheduling, preemptive &  & noOverlap, endBeforeStart, cumulative, disjunctive, alternative constraint & C  & CPO, OPL & satellite, aircraft &  & real-world, benchmark &  & \ref{a:Laborie09} & \ref{c:Laborie09}\\
\rowlabel{b:Laborie18a}\href{../works/Laborie18a.pdf}{Laborie18a}~\cite{Laborie18a} & 9 & resource, job, release-date, scheduling, task, due-date, machine, precedence, Benders Decomposition, Logic-Based Benders Decomposition &  & cumulative, alternative constraint &  & Ilog Scheduler, CPO, OPL &  &  & real-world, real-life, benchmark & energetic reasoning & \ref{a:Laborie18a} & \ref{c:Laborie18a}\\
\rowlabel{b:LacknerMMWW21}\href{../works/LacknerMMWW21.pdf}{LacknerMMWW21}~\cite{LacknerMMWW21} & 18 & release-date, flow-shop, job, order, tardiness, scheduling, machine, lateness, earliness, batch process, setup-time, due-date, make-span, task & OSP, single machine, parallel machine & cumulative, endBeforeStart, noOverlap, Element constraint &  & Chuffed, Cplex, OPL, CPO, MiniZinc, Gurobi, OR-Tools & semiconductor, oven scheduling & manufacturing industry, electronics industry, steel industry & benchmark, instance generator, real-life, random instance, industrial partner, supplementary material & GRASP & \ref{a:LacknerMMWW21} & \ref{c:LacknerMMWW21}\\
\rowlabel{b:LahimerLH11}\href{../works/LahimerLH11.pdf}{LahimerLH11}~\cite{LahimerLH11} & 14 & resource, machine, preempt, cmax, task, precedence, make-span, order, job, scheduling, completion-time, preemptive & parallel machine, RCPSP & Disjunctive constraint, disjunctive & C++ & Ilog Scheduler &  &  & benchmark & energetic reasoning & \ref{a:LahimerLH11} & \ref{c:LahimerLH11}\\
\rowlabel{b:LauLN08}\href{../works/LauLN08.pdf}{LauLN08}~\cite{LauLN08} & 5 & job, order, resource, scheduling, transportation, job-shop, machine, distributed, inventory, flow-shop &  &  &  &  &  &  & real-world, benchmark &  & \ref{a:LauLN08} & \ref{c:LauLN08}\\
\rowlabel{b:LetortBC12}\href{../works/LetortBC12.pdf}{LetortBC12}~\cite{LetortBC12} & 16 & task, machine, make-span, precedence, order, resource, scheduling & psplib & Cumulatives constraint, cumulative, geost, bin-packing & Java, Prolog & Choco Solver, CHIP, SICStus & datacenter &  & Roadef, benchmark, random instance & sweep, edge-finding & \ref{a:LetortBC12} & \ref{c:LetortBC12}\\
\rowlabel{b:LetortCB13}\href{../works/LetortCB13.pdf}{LetortCB13}~\cite{LetortCB13} & 16 & machine, make-span, precedence, order, resource, scheduling, task & psplib, RCPSP & Disjunctive constraint, cumulative, disjunctive, bin-packing & Java, Prolog & Choco Solver, SICStus &  &  & Roadef, benchmark, random instance & energetic reasoning, sweep, edge-finding & \ref{a:LetortCB13} & \ref{c:LetortCB13}\\
\rowlabel{b:LiFJZLL22}\href{../works/LiFJZLL22.pdf}{LiFJZLL22}~\cite{LiFJZLL22} & 6 & completion-time, task, tardiness, buffer-capacity, flow-time, blocking constraint, distributed, job-shop, batch process, flow-shop, transportation, machine, job, setup-time, no-wait, scheduling, order, make-span, stochastic & single machine & Blocking constraint &  & OPL & robot &  & benchmark &  & \ref{a:LiFJZLL22} & \ref{c:LiFJZLL22}\\
\rowlabel{b:LimBTBB15}\href{../works/LimBTBB15.pdf}{LimBTBB15}~\cite{LimBTBB15} & 15 & scheduling, order, tardiness, earliness, job-shop, multi-agent, machine, job, re-scheduling, stochastic &  &  &  & OPL & HVAC &  & benchmark & time-tabling & \ref{a:LimBTBB15} & \ref{c:LimBTBB15}\\
\rowlabel{b:LimHTB16}\href{../works/LimHTB16.pdf}{LimHTB16}~\cite{LimHTB16} & 18 & machine, activity, multi-agent, distributed, re-scheduling, order, scheduling, online scheduling, stochastic &  & cumulative &  & OPL & HVAC, energy-price, real-time pricing &  & real-world &  & \ref{a:LimHTB16} & \ref{c:LimHTB16}\\
\rowlabel{b:LimRX04}\href{../works/LimRX04.pdf}{LimRX04}~\cite{LimRX04} & 5 & scheduling, machine, preempt, completion-time, transportation, job, order, preemptive, stochastic &  &  &  &  & container terminal &  & generated instance &  & \ref{a:LimRX04} & \ref{c:LimRX04}\\
\rowlabel{b:Limtanyakul07}\href{../works/Limtanyakul07.pdf}{Limtanyakul07}~\cite{Limtanyakul07} & 6 & make-span, task, release-date, machine, resource, job, order, scheduling, due-date, precedence &  & cumulative &  & OPL & robot & automobile industry & real-life & energetic reasoning & \ref{a:Limtanyakul07} & \ref{c:Limtanyakul07}\\
\rowlabel{b:LipovetzkyBPS14}\href{../works/LipovetzkyBPS14.pdf}{LipovetzkyBPS14}~\cite{LipovetzkyBPS14} & 9 & make-span, scheduling, resource, precedence, Benders Decomposition, task, order, transportation &  & disjunctive &  & Cplex & crew-scheduling &  & real-life, real-world, industrial partner, industry partner, benchmark, generated instance &  & \ref{a:LipovetzkyBPS14} & \ref{c:LipovetzkyBPS14}\\
\rowlabel{b:LiuCGM17}\href{../works/LiuCGM17.pdf}{LiuCGM17}~\cite{LiuCGM17} & 17 & order, scheduling, machine, task, activity, transportation, cmax &  & Element constraint & Python & OR-Tools, OPL, MiniZinc &  & tourism industry & github &  & \ref{a:LiuCGM17} & \ref{c:LiuCGM17}\\
\rowlabel{b:LiuJ06}\href{../works/LiuJ06.pdf}{LiuJ06}~\cite{LiuJ06} & 5 & make-span, resource, task, order, scheduling &  & disjunctive, Disjunctive constraint, cycle &  &  &  &  &  &  & \ref{a:LiuJ06} & \ref{c:LiuJ06}\\
\rowlabel{b:LiuLH19}\href{../works/LiuLH19.pdf}{LiuLH19}~\cite{LiuLH19} & 9 & order, resource, scheduling &  & Channeling constraint &  & Choco Solver &  &  & benchmark, CSPlib & time-tabling & \ref{a:LiuLH19} & \ref{c:LiuLH19}\\
\rowlabel{b:LombardiBM15}\href{../works/LombardiBM15.pdf}{LombardiBM15}~\cite{LombardiBM15} & 16 & task, completion-time, precedence, scheduling, machine, order, make-span, job-shop, resource, activity, distributed, job, stochastic & JSSP, RCPSP, psplib &  &  &  &  &  & benchmark, real-world &  & \ref{a:LombardiBM15} & \ref{c:LombardiBM15}\\
\rowlabel{b:LombardiBMB11}\href{../works/LombardiBMB11.pdf}{LombardiBMB11}~\cite{LombardiBMB11} & 17 & order, make-span, task, precedence, resource, activity, completion-time, scheduling, machine, periodic, stochastic & RCPSP & cycle, cumulative & C++ &  & hoist &  & benchmark, industrial instance, real-life &  & \ref{a:LombardiBMB11} & \ref{c:LombardiBMB11}\\
\rowlabel{b:LombardiM09}\href{../works/LombardiM09.pdf}{LombardiM09}~\cite{LombardiM09} & 15 & precedence, make-span, order, activity, scheduling, resource, preempt, completion-time, task, preemptive, stochastic & RCPSP & Balance constraint &  & Ilog Solver &  &  & instance generator, real-world &  & \ref{a:LombardiM09} & \ref{c:LombardiM09}\\
\rowlabel{b:LombardiM10}\href{../works/LombardiM10.pdf}{LombardiM10}~\cite{LombardiM10} & 15 & precedence, make-span, order, activity, scheduling, resource, completion-time, task, stochastic & RCPSP & Disjunctive constraint, disjunctive, cumulative &  & Ilog Solver &  &  & real-world, benchmark &  & \ref{a:LombardiM10} & \ref{c:LombardiM10}\\
\rowlabel{b:LombardiM13}\href{../works/LombardiM13.pdf}{LombardiM13}~\cite{LombardiM13} & 2 & precedence, make-span, order, activity, scheduling, resource, task & RCPSP, psplib &  &  &  &  &  &  &  & \ref{a:LombardiM13} & \ref{c:LombardiM13}\\
\rowlabel{b:LouieVNB14}\href{../works/LouieVNB14.pdf}{LouieVNB14}~\cite{LouieVNB14} & 7 & order, resource, job, scheduling, task, machine, activity, periodic &  & cycle &  & OPL & patient, robot &  &  &  & \ref{a:LouieVNB14} & \ref{c:LouieVNB14}\\
\rowlabel{b:LuoB22}\href{../works/LuoB22.pdf}{LuoB22}~\cite{LuoB22} & 17 & order, scheduling, re-scheduling, job, Benders Decomposition, resource, machine, batch process, job-shop &  & AlwaysConstant, bin-packing, diffn, Element constraint, cumulative, alwaysIn & Python & CHIP, Cplex & super-computer, rectangle-packing, railway & metal industry, forging industry & real-life, industry partner, real-world, generated instance, github, industrial instance &  & \ref{a:LuoB22} & \ref{c:LuoB22}\\
\rowlabel{b:LuoVLBM16}\href{../works/LuoVLBM16.pdf}{LuoVLBM16}~\cite{LuoVLBM16} & 4 & task, job, job-shop, resource, machine, precedence, order, activity, scheduling &  &  &  &  & nurse &  &  & time-tabling & \ref{a:LuoVLBM16} & \ref{c:LuoVLBM16}\\
\rowlabel{b:Madi-WambaB16}\href{../works/Madi-WambaB16.pdf}{Madi-WambaB16}~\cite{Madi-WambaB16} & 16 & precedence, task, resource, job, order, scheduling &  & cumulative, TaskIntersection constraint & Java & Choco Solver, CHIP &  &  & real-world, benchmark, random instance, generated instance &  & \ref{a:Madi-WambaB16} & \ref{c:Madi-WambaB16}\\
\rowlabel{b:Madi-WambaLOBM17}\href{../works/Madi-WambaLOBM17.pdf}{Madi-WambaLOBM17}~\cite{Madi-WambaLOBM17} & 8 & job, distributed, scheduling, order, machine, task, re-scheduling, activity, precedence, resource &  & bin-packing, cumulative, Cumulatives constraint, Element constraint & Prolog & SICStus & datacenter &  & real-world & sweep & \ref{a:Madi-WambaLOBM17} & \ref{c:Madi-WambaLOBM17}\\
\rowlabel{b:MakMS10}\href{../works/MakMS10.pdf}{MakMS10}~\cite{MakMS10} & 5 & inventory, task, job, resource, scheduling, due-date, order, machine, activity, transportation, precedence &  & cycle &  &  &  &  &  &  & \ref{a:MakMS10} & \ref{c:MakMS10}\\
\rowlabel{b:MalapertCGJLR13}\href{../works/MalapertCGJLR13.pdf}{MalapertCGJLR13}~\cite{MalapertCGJLR13} & 2 & flow-shop, order, make-span, scheduling, cmax, open-shop, resource, preempt, precedence, task, job, job-shop, machine, preemptive & single machine, Open Shop Scheduling Problem & disjunctive, cumulative & Java & Choco Solver &  &  & benchmark, real-life &  & \ref{a:MalapertCGJLR13} & \ref{c:MalapertCGJLR13}\\
\rowlabel{b:MalapertN19}\href{../works/MalapertN19.pdf}{MalapertN19}~\cite{MalapertN19} & 17 & sequence dependent setup, order, job, flow-time, machine, cmax, make-span, scheduling, completion-time, resource, setup-time, task, single-machine scheduling & PMSP, PTC, parallel machine, single machine & noOverlap, cumulative, alternative constraint, alwaysIn &  & Cplex, CPO & semiconductor &  & benchmark, generated instance, industrial instance, Roadef &  & \ref{a:MalapertN19} & \ref{c:MalapertN19}\\
\rowlabel{b:MaraveliasG04}\href{../works/MaraveliasG04.pdf}{MaraveliasG04}~\cite{MaraveliasG04} & 20 &  &  &  &  & OZ &  &  &  &  & \ref{a:MaraveliasG04} & \ref{c:MaraveliasG04}\\
\rowlabel{b:Mehdizadeh-Somarin23}\href{../works/Mehdizadeh-Somarin23.pdf}{Mehdizadeh-Somarin23}~\cite{Mehdizadeh-Somarin23} & 14 & make-span, preempt, multi-agent, completion-time, tardiness, scheduling, cmax, job, setup-time, precedence, order, job-shop, re-scheduling, machine, flow-shop, task, online scheduling, periodic, preemptive, single-machine scheduling, stochastic & JSSP, parallel machine, single machine &  & Python & Cplex & COVID, robot &  & random instance &  & \ref{a:Mehdizadeh-Somarin23} & \ref{c:Mehdizadeh-Somarin23}\\
\rowlabel{b:MelgarejoLS15}\href{../works/MelgarejoLS15.pdf}{MelgarejoLS15}~\cite{MelgarejoLS15} & 17 & tardiness, scheduling, machine, order, task, precedence, transportation, setup-time, resource, job, one-machine scheduling & single machine & alldifferent, noOverlap, circuit, Disjunctive constraint, disjunctive, table constraint &  & Cplex &  &  & real-world, benchmark &  & \ref{a:MelgarejoLS15} & \ref{c:MelgarejoLS15}\\
\rowlabel{b:Mercier-AubinGQ20}\href{../works/Mercier-AubinGQ20.pdf}{Mercier-AubinGQ20}~\cite{Mercier-AubinGQ20} & 13 & order, Benders Decomposition, job, make-span, sequence dependent setup, tardiness, resource, precedence, completion-time, machine, activity, due-date, preempt, task, setup-time, earliness, lazy clause generation, job-shop, scheduling, Logic-Based Benders Decomposition, preemptive & RCPSP & circuit, cumulative, disjunctive, cycle & C++, Python & OPL, MiniZinc &  & textile industry, manufacturing industry & industrial instance, industrial partner &  & \ref{a:Mercier-AubinGQ20} & \ref{c:Mercier-AubinGQ20}\\
\rowlabel{b:MoffittPP05}\href{../works/MoffittPP05.pdf}{MoffittPP05}~\cite{MoffittPP05} & 6 & order, activity, machine, cmax, make-span, scheduling, resource & Temporal Constraint Satisfaction Problem & cycle, disjunctive &  &  &  &  &  &  & \ref{a:MoffittPP05} & \ref{c:MoffittPP05}\\
\rowlabel{b:MonetteDD07}\href{../works/MonetteDD07.pdf}{MonetteDD07}~\cite{MonetteDD07} & 14 & machine, precedence, make-span, job, scheduling, completion-time, resource, preempt, no preempt, task, job-shop, open-shop, order, preemptive & Open Shop Scheduling Problem, OSP & disjunctive &  & Gecode &  &  & benchmark & not-last, not-first, edge-finding & \ref{a:MonetteDD07} & \ref{c:MonetteDD07}\\
\rowlabel{b:MonetteDH09}\href{../works/MonetteDH09.pdf}{MonetteDH09}~\cite{MonetteDH09} & 8 & machine, precedence, release-date, tardiness, make-span, job, scheduling, completion-time, resource, preempt, earliness, due-date, task, job-shop, order, activity, distributed, preemptive &  & cycle, disjunctive, cumulative &  &  &  &  & benchmark & not-last & \ref{a:MonetteDH09} & \ref{c:MonetteDH09}\\
\rowlabel{b:MossigeGSMC17}\href{../works/MossigeGSMC17.pdf}{MossigeGSMC17}~\cite{MossigeGSMC17} & 18 & activity, job, order, completion-time, scheduling, machine, precedence, distributed, preempt, make-span, task, job-shop, resource, preemptive & single machine, FJS, RCPSP & Cumulatives constraint, cumulative, cycle, disjunctive & Prolog & CHIP, SICStus & robot, rectangle-packing &  & real-world, benchmark, random instance, CSPlib, generated instance, industrial partner &  & \ref{a:MossigeGSMC17} & \ref{c:MossigeGSMC17}\\
\rowlabel{b:MouraSCL08}\href{../works/MouraSCL08.pdf}{MouraSCL08}~\cite{MouraSCL08} & 16 & scheduling, preempt, transportation, precedence, distributed, activity, order, inventory, resource, preemptive &  & table constraint, Element constraint, Channeling constraint, cycle, disjunctive & C++ & Ilog Solver, Ilog Scheduler & pipeline &  &  & max-flow & \ref{a:MouraSCL08} & \ref{c:MouraSCL08}\\
\rowlabel{b:MouraSCL08a}\href{../works/MouraSCL08a.pdf}{MouraSCL08a}~\cite{MouraSCL08a} & 8 & order, scheduling, resource, transportation, re-scheduling, due-date, inventory, distributed &  & Channeling constraint, disjunctive, cumulative & C++ & Ilog Scheduler, Ilog Solver & pipeline &  & real-world, benchmark &  & \ref{a:MouraSCL08a} & \ref{c:MouraSCL08a}\\
\rowlabel{b:MurinR19}\href{../works/MurinR19.pdf}{MurinR19}~\cite{MurinR19} & 16 & job-shop, make-span, transportation, resource, scheduling, Benders Decomposition, completion-time, precedence, task, order, machine, setup-time, job, activity, Logic-Based Benders Decomposition & JSPT & alternative constraint, noOverlap, endBeforeStart &  & Cplex, OPL & robot, patient &  & github, benchmark, real-life &  & \ref{a:MurinR19} & \ref{c:MurinR19}\\
\rowlabel{b:MurphyMB15}\href{../works/MurphyMB15.pdf}{MurphyMB15}~\cite{MurphyMB15} & 17 & scheduling, task, machine, activity, order, re-scheduling, resource, periodic, stochastic &  & cycle, circuit, Disjunctive constraint, cumulative, disjunctive & Java & Choco Solver &  &  & real-world &  & \ref{a:MurphyMB15} & \ref{c:MurphyMB15}\\
\rowlabel{b:Muscettola02}\href{../works/Muscettola02.pdf}{Muscettola02}~\cite{Muscettola02} & 16 & job-shop, resource, activity, job, cmax, precedence, scheduling, order, stochastic &  & cycle, Balance constraint &  &  &  &  &  & edge-finding, max-flow & \ref{a:Muscettola02} & \ref{c:Muscettola02}\\
\rowlabel{b:MusliuSS18}\href{../works/MusliuSS18.pdf}{MusliuSS18}~\cite{MusliuSS18} & 17 & distributed, activity, order, scheduling, manpower, task, machine &  & Regular constraint, cycle, Cardinality constraint &  & Gecode, Gurobi, MiniZinc & workforce scheduling, operating room, nurse &  & generated instance, benchmark, real-life &  & \ref{a:MusliuSS18} & \ref{c:MusliuSS18}\\
\rowlabel{b:NattafM20}\href{../works/NattafM20.pdf}{NattafM20}~\cite{NattafM20} & 16 & setup-time, scheduling, order, make-span, completion-time, flow-time, resource, machine, job & single machine, PMSP, parallel machine, PTC & cumulative, noOverlap &  & CPO, Cplex & semiconductor &  & benchmark, industrial instance &  & \ref{a:NattafM20} & \ref{c:NattafM20}\\
\rowlabel{b:NishikawaSTT18}\href{../works/NishikawaSTT18.pdf}{NishikawaSTT18}~\cite{NishikawaSTT18} & 6 & order, precedence, scheduling, make-span, resource, activity, task, distributed &  & alternative constraint, endBeforeStart &  & Cplex & pipeline, robot &  & real-world, benchmark &  & \ref{a:NishikawaSTT18} & \ref{c:NishikawaSTT18}\\
\rowlabel{b:NishikawaSTT18a}\href{../works/NishikawaSTT18a.pdf}{NishikawaSTT18a}~\cite{NishikawaSTT18a} & 6 & order, make-span, scheduling, resource, precedence, task, activity, distributed, re-scheduling &  & endBeforeStart, alternative constraint &  & Cplex & nurse, pipeline, robot &  & benchmark, real-life, real-world &  & \ref{a:NishikawaSTT18a} & \ref{c:NishikawaSTT18a}\\
\rowlabel{b:NuijtenA94}\href{../works/NuijtenA94.pdf}{NuijtenA94}~\cite{NuijtenA94} & 5 & resource, scheduling, preempt, machine, make-span, job, precedence, job-shop, completion-time, order, preemptive & JSSP & disjunctive, Disjunctive constraint & C++ & Ilog Solver, CPO &  &  &  & time-tabling & \ref{a:NuijtenA94} & \ref{c:NuijtenA94}\\
\rowlabel{b:OddiPCC03}\href{../works/OddiPCC03.pdf}{OddiPCC03}~\cite{OddiPCC03} & 15 & distributed, resource, machine, preempt, scheduling, precedence, order, completion-time, task, activity, periodic, single-machine scheduling & single machine & cycle & Java &  & satellite, earth observation &  & benchmark &  & \ref{a:OddiPCC03} & \ref{c:OddiPCC03}\\
\rowlabel{b:OuelletQ13}\href{../works/OuelletQ13.pdf}{OuelletQ13}~\cite{OuelletQ13} & 16 & scheduling, task, make-span, completion-time, precedence, order, preempt, resource, preemptive & RCPSP, CuSP, psplib & Cumulatives constraint, cumulative, disjunctive &  & Choco Solver &  &  & benchmark & edge-finder, energetic reasoning, time-tabling, sweep, edge-finding, not-first, not-last & \ref{a:OuelletQ13} & \ref{c:OuelletQ13}\\
\rowlabel{b:OuelletQ18}\href{../works/OuelletQ18.pdf}{OuelletQ18}~\cite{OuelletQ18} & 18 & scheduling, task, make-span, completion-time, precedence, order, resource & RCPSP, psplib & Cumulatives constraint, cumulative, disjunctive & Java & Choco Solver &  &  & benchmark, Roadef & energetic reasoning, time-tabling, edge-finding, not-first, not-last & \ref{a:OuelletQ18} & \ref{c:OuelletQ18}\\
\rowlabel{b:OuelletQ22}\href{../works/OuelletQ22.pdf}{OuelletQ22}~\cite{OuelletQ22} & 17 & scheduling, task, activity, completion-time, order, preempt, resource, lazy clause generation &  & GCC constraint, Cumulatives constraint, cumulative, Cardinality constraint, disjunctive, SoftCumulative & Java & MiniZinc, Choco Solver & nurse &  & github, benchmark, random instance & energetic reasoning, time-tabling, sweep, edge-finding, not-first, not-last & \ref{a:OuelletQ22} & \ref{c:OuelletQ22}\\
\rowlabel{b:OujanaAYB22}\href{../works/OujanaAYB22.pdf}{OujanaAYB22}~\cite{OujanaAYB22} & 6 & due-date, tardiness, make to order, job-shop, buffer-capacity, setup-time, sequence dependent setup, open-shop, task, order, distributed, precedence, flow-shop, batch process, make-span, job, scheduling, completion-time, resource, machine, preempt & HFF, PMSP, parallel machine, FJS & span constraint, noOverlap, disjunctive &  & CPO, OPL & robot, COVID & steel industry, food industry & industrial instance, real-world, benchmark, real-life &  & \ref{a:OujanaAYB22} & \ref{c:OujanaAYB22}\\
\rowlabel{b:ParkUJR19}\href{../works/ParkUJR19.pdf}{ParkUJR19}~\cite{ParkUJR19} & 8 & machine, order, tardiness, preempt, scheduling, make-span, completion-time, task, flow-time, cmax, job, lateness, no preempt, distributed, due-date, job-shop, flow-shop, resource, open-shop, stochastic & parallel machine, single machine & endBeforeStart, cycle, noOverlap &  &  &  & trade industry & real-world &  & \ref{a:ParkUJR19} & \ref{c:ParkUJR19}\\
\rowlabel{b:PembertonG98}\href{../works/PembertonG98.pdf}{PembertonG98}~\cite{PembertonG98} & 14 & scheduling, machine, order, job-shop, resource, activity, preempt, job, task, periodic, preemptive, stochastic &  & geost, cycle &  & Ilog Solver, OPL & robot, satellite &  &  &  & \ref{a:PembertonG98} & \ref{c:PembertonG98}\\
\rowlabel{b:PerezGSL23}\href{../works/PerezGSL23.pdf}{PerezGSL23}~\cite{PerezGSL23} & 7 & inventory, order, transportation, re-scheduling, resource, scheduling, task, machine, activity, make-span, completion-time &  & table constraint, cumulative &  & OPL & container terminal, operating room, nurse, steel mill &  & real-world, generated instance &  & \ref{a:PerezGSL23} & \ref{c:PerezGSL23}\\
\rowlabel{b:PesantRR15}\href{../works/PesantRR15.pdf}{PesantRR15}~\cite{PesantRR15} & 16 & transportation, lazy clause generation, scheduling, activity, order &  & cumulative, Cardinality constraint, Regular constraint, table constraint &  & Ilog Solver, Gecode, Gurobi &  &  &  &  & \ref{a:PesantRR15} & \ref{c:PesantRR15}\\
\rowlabel{b:PoderB08}\href{../works/PoderB08.pdf}{PoderB08}~\cite{PoderB08} & 8 & resource, release-date, preempt, due-date, order, scheduling, producer/consumer, task, activity, preemptive &  & cumulative &  & CHIP &  &  &  & sweep & \ref{a:PoderB08} & \ref{c:PoderB08}\\
\rowlabel{b:PopovicCGNC22}\href{../works/PopovicCGNC22.pdf}{PopovicCGNC22}~\cite{PopovicCGNC22} & 15 & order, completion-time, scheduling, machine, transportation, make-span, task, resource, activity, periodic, stochastic & TMS & Balance constraint, cumulative, noOverlap, alwaysIn & C++, Prolog & SICStus, Cplex, CHIP & pipeline, maintenance scheduling & electricity industry &  &  & \ref{a:PopovicCGNC22} & \ref{c:PopovicCGNC22}\\
\rowlabel{b:PovedaAA23}\href{../works/PovedaAA23.pdf}{PovedaAA23}~\cite{PovedaAA23} & 21 & make-span, resource, job, precedence, Benders Decomposition, lazy clause generation, release-date, task, job-shop, activity, order, scheduling, preempt, preemptive & RCPSP & Calendar constraint, cumulative, disjunctive & Python & Cplex, MiniZinc, Chuffed, CPO & automotive, aircraft &  & github, benchmark, industrial instance, real-world, real-life & GRASP & \ref{a:PovedaAA23} & \ref{c:PovedaAA23}\\
\rowlabel{b:Pralet17}\href{../works/Pralet17.pdf}{Pralet17}~\cite{Pralet17} & 19 & setup-time, job, activity, job-shop, sequence dependent setup, resource, scheduling, precedence, due-date, order, make-span, machine & JSSP, RCPSP, psplib & cycle, cumulative, disjunctive &  & CPO, Cplex, CHIP & satellite &  & benchmark &  & \ref{a:Pralet17} & \ref{c:Pralet17}\\
\rowlabel{b:PraletLJ15}\href{../works/PraletLJ15.pdf}{PraletLJ15}~\cite{PraletLJ15} & 16 & task, job-shop, activity, make-span, precedence, due-date, tardiness, order, resource, job, scheduling & JSSP & alternative constraint, Regular constraint, noOverlap, cycle &  & CPO, Cplex & earth observation, satellite &  &  &  & \ref{a:PraletLJ15} & \ref{c:PraletLJ15}\\
\rowlabel{b:Puget95}\href{../works/Puget95.pdf}{Puget95}~\cite{Puget95} & 4 & resource, task, job, order, scheduling, transportation, manpower, job-shop, activity &  & disjunctive &  & OPL & maintenance scheduling &  & benchmark &  & \ref{a:Puget95} & \ref{c:Puget95}\\
\rowlabel{b:QuSN06}\href{../works/QuSN06.pdf}{QuSN06}~\cite{QuSN06} & 4 & task, scheduling, precedence, distributed, resource &  & circuit & Prolog & SICStus &  &  &  &  & \ref{a:QuSN06} & \ref{c:QuSN06}\\
\rowlabel{b:QuirogaZH05}\href{../works/QuirogaZH05.pdf}{QuirogaZH05}~\cite{QuirogaZH05} & 6 & machine, release-date, tardiness, scheduling, completion-time, resource, earliness, due-date, task, precedence, flow-shop, make-span, order, inventory, activity, flow-time &  &  &  & Ilog Solver, OPL, ECLiPSe, Ilog Scheduler & robot &  &  &  & \ref{a:QuirogaZH05} & \ref{c:QuirogaZH05}\\
\rowlabel{b:RendlPHPR12}\href{../works/RendlPHPR12.pdf}{RendlPHPR12}~\cite{RendlPHPR12} & 17 & job, scheduling, machine, transportation, re-scheduling, order, periodic &  &  & Java &  & medical, patient, nurse &  & real-world, CSPlib, benchmark &  & \ref{a:RendlPHPR12} & \ref{c:RendlPHPR12}\\
\rowlabel{b:RiahiNS018}\href{../works/RiahiNS018.pdf}{RiahiNS018}~\cite{RiahiNS018} & 9 & no-wait, flow-shop, completion-time, tardiness, order, buffer-capacity, sequence dependent setup, job, scheduling, blocking constraint, distributed, setup-time, machine, make-span &  & Blocking constraint &  &  &  & cutting industry, painting industry & real-world, real-life, benchmark & NEH, GRASP & \ref{a:RiahiNS018} & \ref{c:RiahiNS018}\\
\rowlabel{b:RodosekW98}\href{../works/RodosekW98.pdf}{RodosekW98}~\cite{RodosekW98} & 15 & order, resource, scheduling, task, transportation, machine, activity, make-span, job &  & disjunctive, cycle, circuit, Disjunctive constraint & Prolog & OPL, CHIP, ECLiPSe, Cplex & hoist, electroplating &  & benchmark &  & \ref{a:RodosekW98} & \ref{c:RodosekW98}\\
\rowlabel{b:Rodriguez07b}\href{../works/Rodriguez07b.pdf}{Rodriguez07b}~\cite{Rodriguez07b} & 14 & re-scheduling, task, blocking constraint, release-date, precedence, scheduling, transportation, order, no-wait, job-shop, resource, activity, job &  & Blocking constraint, Disjunctive constraint, circuit, disjunctive &  & Ilog Scheduler, Z3, Ilog Solver & railway, train schedule & railway industry &  & edge-finding & \ref{a:Rodriguez07b} & \ref{c:Rodriguez07b}\\
\rowlabel{b:RodriguezS09}\href{../works/RodriguezS09.pdf}{RodriguezS09}~\cite{RodriguezS09} & 14 & blocking constraint, completion-time, Benders Decomposition, precedence, scheduling, transportation, order, no-wait, job-shop, resource, activity, job, task, Logic-Based Benders Decomposition &  & Blocking constraint, Disjunctive constraint, circuit, disjunctive &  & Ilog Scheduler, Ilog Solver & railway, train schedule &  &  & edge-finding & \ref{a:RodriguezS09} & \ref{c:RodriguezS09}\\
\rowlabel{b:RossiTHP07}\href{../works/RossiTHP07.pdf}{RossiTHP07}~\cite{RossiTHP07} & 15 & inventory, order, resource, scheduling, distributed, stock level, periodic, stochastic &  & cumulative, cycle &  & OPL, Choco Solver &  &  &  &  & \ref{a:RossiTHP07} & \ref{c:RossiTHP07}\\
\rowlabel{b:Sadykov04}\href{../works/Sadykov04.pdf}{Sadykov04}~\cite{Sadykov04} & 7 & release-date, scheduling, completion-time, task, machine, job, lateness, due-date, preempt, precedence, one-machine scheduling & parallel machine, single machine & disjunctive &  &  &  &  &  & edge-finding & \ref{a:Sadykov04} & \ref{c:Sadykov04}\\
\rowlabel{b:SchausD08}\href{../works/SchausD08.pdf}{SchausD08}~\cite{SchausD08} & 6 & precedence, order, task, preempt, preemptive &  & IloPack, bin-packing, cycle, Reified constraint, Element constraint &  & Ilog Solver, OPL &  &  & real-life, benchmark &  & \ref{a:SchausD08} & \ref{c:SchausD08}\\
\rowlabel{b:SchuttCSW12}\href{../works/SchuttCSW12.pdf}{SchuttCSW12}~\cite{SchuttCSW12} & 17 & scheduling, resource, preempt, lazy clause generation, order, activity, precedence, make-span, preemptive &  & cumulative &  & CHIP &  &  & benchmark &  & \ref{a:SchuttCSW12} & \ref{c:SchuttCSW12}\\
\rowlabel{b:SchuttFS13}\href{../works/SchuttFS13.pdf}{SchuttFS13}~\cite{SchuttFS13} & 17 & resource, job, lazy clause generation, scheduling, task, job-shop, machine, activity, make-span, completion-time, precedence, order & RCPSP, FJS & disjunctive, Disjunctive constraint, span constraint, alternative constraint, cumulative &  & MiniZinc &  &  & benchmark & energetic reasoning, time-tabling & \ref{a:SchuttFS13} & \ref{c:SchuttFS13}\\
\rowlabel{b:SchuttFS13a}\href{../works/SchuttFS13a.pdf}{SchuttFS13a}~\cite{SchuttFS13a} & 17 & make-span, scheduling, completion-time, resource, machine, preempt, lazy clause generation, task, order, activity, precedence, preemptive & psplib, RCPSP & circuit, disjunctive, cumulative &  & SCIP, CHIP &  &  & benchmark & not-last, energetic reasoning, edge-finding & \ref{a:SchuttFS13a} & \ref{c:SchuttFS13a}\\
\rowlabel{b:SchuttFSW09}\href{../works/SchuttFSW09.pdf}{SchuttFSW09}~\cite{SchuttFSW09} & 16 & scheduling, resource, machine, preempt, lazy clause generation, open-shop, task, order, activity, precedence, make-span, job, periodic, preemptive & psplib & Disjunctive constraint, disjunctive, cumulative &  & ECLiPSe, CHIP, SICStus &  &  & real-world, benchmark & edge-finder & \ref{a:SchuttFSW09} & \ref{c:SchuttFSW09}\\
\rowlabel{b:SchuttS16}\href{../works/SchuttS16.pdf}{SchuttS16}~\cite{SchuttS16} & 17 & machine, precedence, order, inventory, activity, preempt, manpower, scheduling, make-span, producer/consumer, lazy clause generation, resource, preemptive & RCPSP & Balance constraint, Cumulatives constraint, cumulative &  & Chuffed, MiniZinc, OPL, Ilog Scheduler &  &  & benchmark &  & \ref{a:SchuttS16} & \ref{c:SchuttS16}\\
\rowlabel{b:SchuttW10}\href{../works/SchuttW10.pdf}{SchuttW10}~\cite{SchuttW10} & 15 & order, activity, preempt, release-date, scheduling, make-span, task, lazy clause generation, due-date, resource, preemptive & CuSP, psplib, RCPSP & disjunctive, Disjunctive constraint, cumulative & Java & CHIP & rectangle-packing &  & benchmark & not-last, edge-finding, not-first & \ref{a:SchuttW10} & \ref{c:SchuttW10}\\
\rowlabel{b:SchuttWS05}\href{../works/SchuttWS05.pdf}{SchuttWS05}~\cite{SchuttWS05} & 15 & task, due-date, machine, order, preempt, resource, release-date, scheduling, preemptive &  & cumulative, disjunctive &  & OPL, CHIP &  &  & benchmark & not-last & \ref{a:SchuttWS05} & \ref{c:SchuttWS05}\\
\rowlabel{b:SerraNM12}\href{../works/SerraNM12.pdf}{SerraNM12}~\cite{SerraNM12} & 17 & inventory, preempt, resource, precedence, order, activity, release-date, scheduling, machine, preemptive &  & cumulative, alwaysIn, cycle &  & OPL, Cplex &  &  & real-world, benchmark & GRASP & \ref{a:SerraNM12} & \ref{c:SerraNM12}\\
\rowlabel{b:SialaAH15}\href{../works/SialaAH15.pdf}{SialaAH15}~\cite{SialaAH15} & 10 & make-span, task, cmax, job, job-shop, resource, open-shop, machine, precedence, order, tardiness, setup-time, earliness, lazy clause generation, scheduling & RCPSP, JSSP & Disjunctive constraint, cumulative, disjunctive &  & Mistral &  &  & github, benchmark & edge-finding & \ref{a:SialaAH15} & \ref{c:SialaAH15}\\
\rowlabel{b:SimoninAHL12}\href{../works/SimoninAHL12.pdf}{SimoninAHL12}~\cite{SimoninAHL12} & 15 & resource, activity, scheduling, task, precedence, preempt, order, periodic, preemptive &  & disjunctive, span constraint, cycle, cumulative &  & CHIP & satellite &  &  & sweep & \ref{a:SimoninAHL12} & \ref{c:SimoninAHL12}\\
\rowlabel{b:Simonis95}\href{../works/Simonis95.pdf}{Simonis95}~\cite{Simonis95} & 4 & scheduling, task, producer/consumer, resource, transportation, machine, precedence, order &  & diffn, Among constraint, cumulative, cycle, circuit & Prolog & CHIP & aircraft & food industry &  &  & \ref{a:Simonis95} & \ref{c:Simonis95}\\
\rowlabel{b:Simonis95a}\href{../works/Simonis95a.pdf}{Simonis95a}~\cite{Simonis95a} & 21 & scheduling, manpower, task, machine, job, precedence, distributed, stock level, due-date, order, inventory, producer/consumer, resource &  & cycle, diffn, circuit, cumulative & Prolog, C++ & OPL, CHIP & aircraft, pipeline & chemical industry, drawing industry & real-life, benchmark &  & \ref{a:Simonis95a} & \ref{c:Simonis95a}\\
\rowlabel{b:Simonis99}\href{../works/Simonis99.pdf}{Simonis99}~\cite{Simonis99} & 39 & scheduling, task, producer/consumer, job, inventory, due-date, manpower, resource, transportation, stock level, machine, precedence, order, activity, stochastic &  & disjunctive, Disjunctive constraint, diffn, cumulative, alldifferent, cycle, circuit & C++, Prolog & OPL, CHIP, ECLiPSe, SICStus & aircraft, pipeline, maintenance scheduling, nurse & chemical industry, food industry, process industry & benchmark, real-world, real-life & bi-partite matching & \ref{a:Simonis99} & \ref{c:Simonis99}\\
\rowlabel{b:SimonisC95}\href{../works/SimonisC95.pdf}{SimonisC95}~\cite{SimonisC95} & 14 & scheduling, manpower, task, transportation, machine, job, stock level, continuous-process, job-shop, due-date, flow-shop, order, inventory, batch process, producer/consumer, resource &  & diffn, cumulative & Prolog & CHIP & aircraft, pipeline, maintenance scheduling & food industry & real-life &  & \ref{a:SimonisC95} & \ref{c:SimonisC95}\\
\rowlabel{b:SimonisH11}\href{../works/SimonisH11.pdf}{SimonisH11}~\cite{SimonisH11} & 14 & preempt, manpower, task, order, producer/consumer, resource, scheduling, preemptive &  & Element constraint, CumulativeCost, Cumulatives constraint, cumulative &  & Choco Solver, CHIP, Cplex &  &  & real-life, real-world & sweep, edge-finding & \ref{a:SimonisH11} & \ref{c:SimonisH11}\\
\rowlabel{b:SquillaciPR23}\href{../works/SquillaciPR23.pdf}{SquillaciPR23}~\cite{SquillaciPR23} & 17 & multi-agent, distributed, task, resource, activity, order, scheduling, periodic & EOSP, OSP, Earth Observation Scheduling Problem & noOverlap & Python & Cplex & earth orbit, earth observation, satellite &  & github, benchmark & GRASP & \ref{a:SquillaciPR23} & \ref{c:SquillaciPR23}\\
\rowlabel{b:SunLYL10}\href{../works/SunLYL10.pdf}{SunLYL10}~\cite{SunLYL10} & 6 & task, order, distributed, scheduling, periodic &  & cycle &  & OPL, Cplex & automotive &  &  &  & \ref{a:SunLYL10} & \ref{c:SunLYL10}\\
\rowlabel{b:SvancaraB22}\href{../works/SvancaraB22.pdf}{SvancaraB22}~\cite{SvancaraB22} & 8 & multi-agent, batch process, make-span, order, activity, scheduling, resource, task &  & alternative constraint, noOverlap &  &  & train schedule, railway &  & benchmark, real-world & time-tabling & \ref{a:SvancaraB22} & \ref{c:SvancaraB22}\\
\rowlabel{b:SzerediS16}\href{../works/SzerediS16.pdf}{SzerediS16}~\cite{SzerediS16} & 10 & task, machine, activity, order, preempt, make-span, resource, precedence, lazy clause generation, scheduling, preemptive & RCPSP, psplib & Element constraint, cumulative &  & Cplex, MiniZinc, SCIP, Chuffed, Gecode &  &  & benchmark &  & \ref{a:SzerediS16} & \ref{c:SzerediS16}\\
\rowlabel{b:TanT18}\href{../works/TanT18.pdf}{TanT18}~\cite{TanT18} & 12 & flow-shop, Benders Decomposition, machine, cmax, release-date, job-shop, task, scheduling, completion-time, precedence, make-span, re-scheduling, job, setup-time, Logic-Based Benders Decomposition, single-machine scheduling & single machine, parallel machine & Disjunctive constraint, disjunctive &  & Cplex & medical, operating room, patient, robot &  & benchmark &  & \ref{a:TanT18} & \ref{c:TanT18}\\
\rowlabel{b:TangB20}\href{../works/TangB20.pdf}{TangB20}~\cite{TangB20} & 16 & job, flow-shop, resource, make-span, scheduling, tardiness, due-date, order, batch process, machine, precedence, Benders Decomposition, Logic-Based Benders Decomposition, two-stage scheduling & HFS, 2BPHFSP, single machine & span constraint, bin-packing, alwaysIn, Cardinality constraint, Element constraint, cycle, endBeforeStart, GCC constraint & Java & CPO, Cplex & semiconductor & manufacturing industry & real-world &  & \ref{a:TangB20} & \ref{c:TangB20}\\
\rowlabel{b:TardivoDFMP23}\href{../works/TardivoDFMP23.pdf}{TardivoDFMP23}~\cite{TardivoDFMP23} & 18 & activity, order, scheduling, lazy clause generation, task, precedence, preempt, make-span, resource & RCPSP, psplib, CuSP & cumulative, disjunctive, Cumulatives constraint & C++ & CHIP, Gecode, MiniZinc &  &  & benchmark, bitbucket, github, real-world & sweep, energetic reasoning, not-last, not-first, edge-finding, time-tabling & \ref{a:TardivoDFMP23} & \ref{c:TardivoDFMP23}\\
\rowlabel{b:TasselGS23}\href{../works/TasselGS23.pdf}{TasselGS23}~\cite{TasselGS23} & 9 & flow-shop, completion-time, order, tardiness, resource, scheduling, preempt, flow-time, task, machine, re-scheduling, make-span, job, precedence, job-shop, periodic & JSSP & cumulative, disjunctive, noOverlap & Java & Choco Solver &  &  & industrial instance, real-world, supplementary material, github, benchmark &  & \ref{a:TasselGS23} & \ref{c:TasselGS23}\\
\rowlabel{b:Teppan22}\href{../works/Teppan22.pdf}{Teppan22}~\cite{Teppan22} & 8 & job-shop, make-span, cmax, preempt, distributed, resource, scheduling, flow-shop, task, order, completion-time, machine, setup-time, job & parallel machine, JSSP, PTC, FJS & noOverlap, endBeforeStart & Java & OR-Tools, OPL &  &  & benchmark, real-life &  & \ref{a:Teppan22} & \ref{c:Teppan22}\\
\rowlabel{b:Tesch16}\href{../works/Tesch16.pdf}{Tesch16}~\cite{Tesch16} & 27 & job, resource, make-span, scheduling, order, completion-time, precedence & CuSP, psplib, RCPSP & cumulative, disjunctive & C++ & OPL &  &  & Roadef & energetic reasoning, not-first, sweep, edge-finding, not-last, time-tabling & \ref{a:Tesch16} & \ref{c:Tesch16}\\
\rowlabel{b:Tesch18}\href{../works/Tesch18.pdf}{Tesch18}~\cite{Tesch18} & 17 & preempt, task, job, release-date, resource, make-span, scheduling, due-date, order, machine, completion-time, precedence, lateness, preemptive, single-machine scheduling & CuSP, psplib, RCPSP, single machine & cumulative &  &  &  &  & Roadef & energetic reasoning, sweep, edge-finding, not-last, time-tabling & \ref{a:Tesch18} & \ref{c:Tesch18}\\
\rowlabel{b:ThiruvadyBME09}\href{../works/ThiruvadyBME09.pdf}{ThiruvadyBME09}~\cite{ThiruvadyBME09} & 15 & due-date, make-span, resource, setup-time, tardiness, open-shop, machine, job, scheduling, order, single-machine scheduling, stochastic & single machine & cumulative & C++ & Gecode &  &  &  &  & \ref{a:ThiruvadyBME09} & \ref{c:ThiruvadyBME09}\\
\rowlabel{b:ThomasKS20}\href{../works/ThomasKS20.pdf}{ThomasKS20}~\cite{ThomasKS20} & 18 & order, transportation, resource, scheduling, activity &  & cumulative & C , Java & CPO, OR-Tools, OPL, Cplex & medical, patient &  & CSPlib, benchmark, generated instance, bitbucket &  & \ref{a:ThomasKS20} & \ref{c:ThomasKS20}\\
\rowlabel{b:Thorsteinsson01}\href{../works/Thorsteinsson01.pdf}{Thorsteinsson01}~\cite{Thorsteinsson01} & 15 & order, Benders Decomposition, scheduling, job, machine, precedence, task, due-date, Logic-Based Benders Decomposition & parallel machine & alldifferent, cumulative, circuit, Arithmetic constraint &  & OPL &  &  &  &  & \ref{a:Thorsteinsson01} & \ref{c:Thorsteinsson01}\\
\rowlabel{b:Tom19}\href{../works/Tom19.pdf}{Tom19}~\cite{Tom19} & 6 & task, tardiness, resource, job-shop, job, re-scheduling, activity, scheduling, make-span, machine, transportation, single-machine scheduling & single machine &  & Java & OPL &  &  & real-world &  & \ref{a:Tom19} & \ref{c:Tom19}\\
\rowlabel{b:TouatBT22}\href{../works/TouatBT22.pdf}{TouatBT22}~\cite{TouatBT22} & 8 & job, no preempt, distributed, due-date, job-shop, flow-shop, resource, machine, precedence, order, tardiness, activity, preempt, release-date, earliness, scheduling, make-span, completion-time, task, periodic, single-machine scheduling & RCPSP, single machine & noOverlap &  & Cplex, OPL & robot, satellite, container terminal &  & generated instance, benchmark & time-tabling & \ref{a:TouatBT22} & \ref{c:TouatBT22}\\
\rowlabel{b:Touraivane95}\href{../works/Touraivane95.pdf}{Touraivane95}~\cite{Touraivane95} & 3 & order, scheduling, task &  &  & Prolog &  & crew-scheduling &  & real-life &  & \ref{a:Touraivane95} & \ref{c:Touraivane95}\\
\rowlabel{b:TranB12}\href{../works/TranB12.pdf}{TranB12}~\cite{TranB12} & 6 & setup-time, due-date, Benders Decomposition, release-date, resource, make-span, scheduling, sequence dependent setup, tardiness, job, order, machine, completion-time, distributed, precedence, cmax, Logic-Based Benders Decomposition, single-machine scheduling & PMSP, single machine, parallel machine & cycle, circuit & C++ & Cplex &  &  & benchmark &  & \ref{a:TranB12} & \ref{c:TranB12}\\
\rowlabel{b:TranDRFWOVB16}\href{../works/TranDRFWOVB16.pdf}{TranDRFWOVB16}~\cite{TranDRFWOVB16} & 9 & resource, activity, re-scheduling, job, order, scheduling, machine, task, job-shop, precedence, stochastic &  & cycle & Python & OPL & aircraft &  &  &  & \ref{a:TranDRFWOVB16} & \ref{c:TranDRFWOVB16}\\
\rowlabel{b:TranTDB13}\href{../works/TranTDB13.pdf}{TranTDB13}~\cite{TranTDB13} & 9 & flow-shop, resource, cmax, machine, job, re-scheduling, setup-time, scheduling, order, make-span, task, flow-time, distributed, periodic, stochastic & parallel machine & cycle & C++ & Cplex &  &  & real-world &  & \ref{a:TranTDB13} & \ref{c:TranTDB13}\\
\rowlabel{b:TranVNB17a}\href{../works/TranVNB17a.pdf}{TranVNB17a}~\cite{TranVNB17a} & 5 & scheduling, task, transportation, machine, activity, setup-time, order, resource &  & alternative constraint, cumulative &  & Cplex & medical, robot &  & real-world &  & \ref{a:TranVNB17a} & \ref{c:TranVNB17a}\\
\rowlabel{b:TranWDRFOVB16}\href{../works/TranWDRFOVB16.pdf}{TranWDRFOVB16}~\cite{TranWDRFOVB16} & 9 & job, order, scheduling, task, precedence, activity, job-shop, machine, single-machine scheduling, stochastic & single machine & cumulative, cycle & Python & OPL, Ilog Scheduler & robot, satellite &  & benchmark &  & \ref{a:TranWDRFOVB16} & \ref{c:TranWDRFOVB16}\\
\rowlabel{b:ValleMGT03}\href{../works/ValleMGT03.pdf}{ValleMGT03}~\cite{ValleMGT03} & 8 & machine, order, scheduling, transportation, make-span, resource, job, precedence, task, job-shop &  &  &  & Ilog Solver & robot &  & real-life & edge-finder & \ref{a:ValleMGT03} & \ref{c:ValleMGT03}\\
\rowlabel{b:VanczaM01}\href{../works/VanczaM01.pdf}{VanczaM01}~\cite{VanczaM01} & 15 & resource, machine, order, scheduling, precedence, task &  & cycle, disjunctive, Disjunctive constraint &  &  & robot &  & real-world, real-life &  & \ref{a:VanczaM01} & \ref{c:VanczaM01}\\
\rowlabel{b:VerfaillieL01}\href{../works/VerfaillieL01.pdf}{VerfaillieL01}~\cite{VerfaillieL01} & 15 & task, job-shop, job, open-shop, order, scheduling, stochastic & Open Shop Scheduling Problem & cycle &  & Cplex, OPL & earth observation, satellite &  &  &  & \ref{a:VerfaillieL01} & \ref{c:VerfaillieL01}\\
\rowlabel{b:Vilim02}\href{../works/Vilim02.pdf}{Vilim02}~\cite{Vilim02} & 1 & resource, scheduling, precedence, sequence dependent setup, batch process, activity, setup-time &  & cumulative, disjunctive &  &  &  &  &  & edge-finding & \ref{a:Vilim02} & \ref{c:Vilim02}\\
\rowlabel{b:Vilim03}\href{../works/Vilim03.pdf}{Vilim03}~\cite{Vilim03} & 1 & job, open-shop, order, scheduling, job-shop &  & cumulative, disjunctive &  &  &  &  &  & edge-finding, not-last & \ref{a:Vilim03} & \ref{c:Vilim03}\\
\rowlabel{b:Vilim04}\href{../works/Vilim04.pdf}{Vilim04}~\cite{Vilim04} & 13 & task, job, order, resource, scheduling, precedence, sequence dependent setup, batch process, machine, completion-time, activity, setup-time, job-shop &  & cumulative, disjunctive &  &  &  &  & benchmark & edge-finding, sweep, not-last & \ref{a:Vilim04} & \ref{c:Vilim04}\\
\rowlabel{b:Vilim05}\href{../works/Vilim05.pdf}{Vilim05}~\cite{Vilim05} & 14 & preempt, task, job, open-shop, order, resource, make-span, scheduling, precedence, machine, completion-time, activity, job-shop, preemptive &  & cumulative, disjunctive & C++ &  &  &  & benchmark & not-last & \ref{a:Vilim05} & \ref{c:Vilim05}\\
\rowlabel{b:Vilim09}\href{../works/Vilim09.pdf}{Vilim09}~\cite{Vilim09} & 15 & preempt, job, order, resource, scheduling, precedence, completion-time, activity, job-shop, preemptive &  & cumulative, cycle &  & CPO &  &  &  & energetic reasoning, edge-finding, not-first, not-last & \ref{a:Vilim09} & \ref{c:Vilim09}\\
\rowlabel{b:Vilim09a}\href{../works/Vilim09a.pdf}{Vilim09a}~\cite{Vilim09a} & 15 & order, scheduling, completion-time, task, activity, resource, preempt, preemptive &  & cycle, cumulative &  & Ilog Scheduler &  &  &  & edge-finding, not-last, energetic reasoning & \ref{a:Vilim09a} & \ref{c:Vilim09a}\\
\rowlabel{b:Vilim11}\href{../works/Vilim11.pdf}{Vilim11}~\cite{Vilim11} & 16 & preempt, task, order, resource, scheduling, precedence, machine, completion-time, activity, manpower, preemptive & psplib, RCPSP & cumulative, disjunctive, cycle &  &  &  &  & benchmark & energetic reasoning, edge-finding, sweep, not-last, time-tabling & \ref{a:Vilim11} & \ref{c:Vilim11}\\
\rowlabel{b:VilimBC04}\href{../works/VilimBC04.pdf}{VilimBC04}~\cite{VilimBC04} & 15 & scheduling, make-span, completion-time, job, distributed, job-shop, resource, open-shop, machine, precedence, order, activity &  & disjunctive, cumulative &  &  &  &  & benchmark, real-life & edge-finding, not-first, not-last & \ref{a:VilimBC04} & \ref{c:VilimBC04}\\
\rowlabel{b:VilimLS15}\href{../works/VilimLS15.pdf}{VilimLS15}~\cite{VilimLS15} & 17 & machine, precedence, order, activity, earliness, scheduling, make-span, completion-time, task, cmax, job, job-shop, resource, periodic & psplib, RCPSP & disjunctive, noOverlap, cumulative &  & Cplex, CPO & rectangle-packing &  & benchmark & time-tabling & \ref{a:VilimLS15} & \ref{c:VilimLS15}\\
\rowlabel{b:Wallace06}\href{../works/Wallace06.pdf}{Wallace06}~\cite{Wallace06} & 32 & earliness, task, resource, machine, job, job-shop, transportation, scheduling, Benders Decomposition, order, tardiness, Logic-Based Benders Decomposition &  & cycle, Channeling constraint, circuit &  & Z3, CHIP, Cplex, ECLiPSe, OPL & workforce scheduling, hoist &  & benchmark, real-world, Roadef &  & \ref{a:Wallace06} & \ref{c:Wallace06}\\
\rowlabel{b:WangB20}\href{../works/WangB20.pdf}{WangB20}~\cite{WangB20} & 8 & task, resource, scheduling, job, order, machine, distributed & Fixed Job Scheduling, FJS & AllDiff constraint, alldifferent, MinWeightAllDiff, WeightAllDiff &  & Gurobi & aircraft &  & github &  & \ref{a:WangB20} & \ref{c:WangB20}\\
\rowlabel{b:WangB23}\href{../works/WangB23.pdf}{WangB23}~\cite{WangB23} & 8 & task, resource, scheduling, job, lazy clause generation, order, transportation & Fixed Job Scheduling, FJS & alldifferent, Channeling constraint, MinWeightAllDiff, WeightAllDiff &  & Gurobi & crew-scheduling, operating room, aircraft &  & random instance, real-world &  & \ref{a:WangB23} & \ref{c:WangB23}\\
\rowlabel{b:WatsonB08}\href{../works/WatsonB08.pdf}{WatsonB08}~\cite{WatsonB08} & 15 & job-shop, resource, machine, order, scheduling, make-span, completion-time, cmax, job, periodic &  & disjunctive & C++ & Ilog Scheduler &  &  & real-world, benchmark &  & \ref{a:WatsonB08} & \ref{c:WatsonB08}\\
\rowlabel{b:WessenCS20}\href{../works/WessenCS20.pdf}{WessenCS20}~\cite{WessenCS20} & 10 & make-span, completion-time, precedence, order, multi-agent, job, scheduling, task, job-shop &  & circuit &  & Gecode & robot &  & real-world &  & \ref{a:WessenCS20} & \ref{c:WessenCS20}\\
\rowlabel{b:WinterMMW22}\href{../works/WinterMMW22.pdf}{WinterMMW22}~\cite{WinterMMW22} & 18 & tardiness, setup-time, task, order, distributed, precedence, release-date, job, scheduling, completion-time, resource, machine, due-date & PMSP, parallel machine & noOverlap, alternative constraint &  & CPO, Gurobi, Cplex & farming & manufacturing industry, agricultural industry & supplementary material, zenodo, industrial partner, benchmark, real-life, industry partner &  & \ref{a:WinterMMW22} & \ref{c:WinterMMW22}\\
\rowlabel{b:Wolf03}\href{../works/Wolf03.pdf}{Wolf03}~\cite{Wolf03} & 15 & resource, job, machine, job-shop, task, order, preempt, scheduling, completion-time, make-span, activity, preemptive &  & cumulative, Disjunctive constraint, disjunctive & Java &  & pipeline &  & benchmark & not-last, edge-finding, not-first, sweep & \ref{a:Wolf03} & \ref{c:Wolf03}\\
\rowlabel{b:Wolf05}\href{../works/Wolf05.pdf}{Wolf05}~\cite{Wolf05} & 15 & resource, job, machine, job-shop, task, order, preempt, scheduling, completion-time, precedence, make-span, activity, preemptive &  & cumulative & Java & Ilog Scheduler &  &  & benchmark & not-last, edge-finding, not-first, sweep & \ref{a:Wolf05} & \ref{c:Wolf05}\\
\rowlabel{b:Wolf09}\href{../works/Wolf09.pdf}{Wolf09}~\cite{Wolf09} & 17 & resource, job, machine, job-shop, task, order, preempt, scheduling, preemptive &  & WeightedSum, WeightedTaskSum & Java & CHIP, SICStus, OPL & operating room, patient, surgery &  & real-life & not-last, edge-finding, not-first, sweep & \ref{a:Wolf09} & \ref{c:Wolf09}\\
\rowlabel{b:Wolf11}\href{../works/Wolf11.pdf}{Wolf11}~\cite{Wolf11} & 17 & distributed, resource, inventory, machine, producer/consumer, task, order, preempt, scheduling, sequence dependent setup, activity, transportation, setup-time, preemptive & single machine & cumulative, Element constraint, Cumulatives constraint, alternative constraint & Java & CHIP, OPL & medical, nurse, physician, operating room, patient, surgery &  &  &  & \ref{a:Wolf11} & \ref{c:Wolf11}\\
\rowlabel{b:WolfS05}\href{../works/WolfS05.pdf}{WolfS05}~\cite{WolfS05} & 14 & order, completion-time, scheduling, distributed, preempt, activity, task, resource, preemptive &  & cumulative &  & CHIP &  &  & real-world & energetic reasoning, sweep, not-last & \ref{a:WolfS05} & \ref{c:WolfS05}\\
\rowlabel{b:WolinskiKG04}\href{../works/WolinskiKG04.pdf}{WolinskiKG04}~\cite{WolinskiKG04} & 8 & resource, precedence, scheduling, machine, order, distributed & SCC & Diff2 constraint, cycle & Java &  & pipeline &  &  &  & \ref{a:WolinskiKG04} & \ref{c:WolinskiKG04}\\
\rowlabel{b:WuBB05}\href{../works/WuBB05.pdf}{WuBB05}~\cite{WuBB05} & 1 & resource, job, release-date, scheduling, make-span, stochastic &  &  &  & Ilog Scheduler &  &  & benchmark &  & \ref{a:WuBB05} & \ref{c:WuBB05}\\
\rowlabel{b:YangSS19}\href{../works/YangSS19.pdf}{YangSS19}~\cite{YangSS19} & 10 & resource, preempt, order, scheduling, completion-time, machine, task, activity, lazy clause generation, preemptive &  & cumulative, disjunctive & Prolog & Choco Solver, Gecode, CHIP, OR-Tools, SICStus, OPL & rectangle-packing &  & generated instance & energetic reasoning, edge-finding, not-last & \ref{a:YangSS19} & \ref{c:YangSS19}\\
\rowlabel{b:YoungFS17}\href{../works/YoungFS17.pdf}{YoungFS17}~\cite{YoungFS17} & 10 & lazy clause generation, scheduling, make-span, task, resource, machine, precedence, order, activity, preempt, preemptive & psplib, RCPSP & disjunctive, cumulative &  & Chuffed, MiniZinc &  &  & benchmark, github, instance generator & time-tabling & \ref{a:YoungFS17} & \ref{c:YoungFS17}\\
\rowlabel{b:YuraszeckMC23}\href{../works/YuraszeckMC23.pdf}{YuraszeckMC23}~\cite{YuraszeckMC23} & 6 & job, open-shop, order, scheduling, due-date, make-span, precedence, cmax, distributed, preempt, job-shop, flow-time, release-date, machine, preemptive, stochastic & OSSP, JSSP & noOverlap &  &  &  &  & benchmark, github &  & \ref{a:YuraszeckMC23} & \ref{c:YuraszeckMC23}\\
\rowlabel{b:ZhangBB22}\href{../works/ZhangBB22.pdf}{ZhangBB22}~\cite{ZhangBB22} & 9 & preempt, scheduling, precedence, order, make-span, completion-time, task, distributed, job-shop, resource, cmax, machine, job, lateness, one-machine scheduling & single machine & disjunctive, span constraint, Disjunctive constraint, cycle & Python & OPL, Gurobi, CPO &  &  & benchmark, generated instance &  & \ref{a:ZhangBB22} & \ref{c:ZhangBB22}\\
\rowlabel{b:ZhangJZL22}\href{../works/ZhangJZL22.pdf}{ZhangJZL22}~\cite{ZhangJZL22} & 6 & resource, scheduling, task, transportation, machine, make-span, job, precedence, setup-time, due-date, flow-shop, completion-time, order, tardiness, single-machine scheduling, stochastic & single machine, parallel machine, HFS & noOverlap, endBeforeStart, alternative constraint, cumulative &  &  & semiconductor &  & benchmark &  & \ref{a:ZhangJZL22} & \ref{c:ZhangJZL22}\\
\rowlabel{b:ZhangLS12}\href{../works/ZhangLS12.pdf}{ZhangLS12}~\cite{ZhangLS12} & 4 & scheduling, order, cmax &  &  &  &  &  &  &  & time-tabling & \ref{a:ZhangLS12} & \ref{c:ZhangLS12}\\
\rowlabel{b:Zhou96}\href{../works/Zhou96.pdf}{Zhou96}~\cite{Zhou96} & 15 & release-date, job-shop, due-date, task, order, scheduling, completion-time, precedence, job, machine &  & Disjunctive constraint, disjunctive & Prolog & Z3 &  &  &  & edge-finding & \ref{a:Zhou96} & \ref{c:Zhou96}\\
\rowlabel{b:ZhouGL15}\href{../works/ZhouGL15.pdf}{ZhouGL15}~\cite{ZhouGL15} & 5 & distributed, resource, tardiness, job-shop, flow-shop, re-scheduling, task, order, scheduling, completion-time, machine, setup-time, job, make-span, transportation, cmax, online scheduling, stochastic & HFF, FJS, HFS, parallel machine & cumulative &  & CHIP, Gecode, OR-Tools & railway &  & real-world & GRASP, NEH & \ref{a:ZhouGL15} & \ref{c:ZhouGL15}\\
\rowlabel{b:ZhuS02}\href{../works/ZhuS02.pdf}{ZhuS02}~\cite{ZhuS02} & 5 & activity, distributed, resource, scheduling &  &  &  &  &  &  &  &  & \ref{a:ZhuS02} & \ref{c:ZhuS02}\\
\rowlabel{b:ZibranR11}\href{../works/ZibranR11.pdf}{ZibranR11}~\cite{ZibranR11} & 4 & scheduling, order, activity &  &  & Java & Cplex, OPL &  &  &  &  & \ref{a:ZibranR11} & \ref{c:ZibranR11}\\
\rowlabel{b:ZibranR11a}\href{../works/ZibranR11a.pdf}{ZibranR11a}~\cite{ZibranR11a} & 10 & scheduling, distributed, activity, order, resource &  &  &  & Cplex, OPL &  &  &  & time-tabling & \ref{a:ZibranR11a} & \ref{c:ZibranR11a}\\
\end{longtable}
}



\clearpage
\subsection{Manually Defined Fields}
{\scriptsize
\begin{longtable}{>{\raggedright\arraybackslash}p{3cm}>{\raggedright\arraybackslash}p{6cm}lp{2cm}rrrrlp{2cm}p{2cm}rr}
\rowcolor{white}\caption{Manually Defined PAPER Properties}\\ \toprule
\rowcolor{white}Key & Title (Local Copy) & \shortstack{CP\\System} & Bench & Links & \shortstack{Data\\Avail} & \shortstack{Sol\\Avail} & \shortstack{Code\\Avail} & \shortstack{Based\\On} & Classification & Constraints & a & b\\ \midrule\endhead
\bottomrule
\endfoot
\rowlabel{c:AalianPG23}AalianPG23 \href{https://doi.org/10.4230/LIPIcs.CP.2023.6}{AalianPG23}~\cite{AalianPG23} & \href{works/AalianPG23.pdf}{Optimization of Short-Term Underground Mine Planning Using Constraint Programming} & CP Opt & real-world & 1 & n &  & n &  &  & ? & \ref{a:AalianPG23} & \ref{b:AalianPG23}\\
\rowlabel{c:Bit-Monnot23}Bit-Monnot23 \href{https://doi.org/10.3233/FAIA230278}{Bit-Monnot23}~\cite{Bit-Monnot23} & \href{works/Bit-Monnot23.pdf}{Enhancing Hybrid {CP-SAT} Search for Disjunctive Scheduling} & \su{ARIES {CP Opt} OR-Tools Mistral} & real-world, github, benchmark & 1 & \href{https://github.com/plaans/aries}{y} &  & \href{https://github.com/plaans/aries}{y} & - & \su{JSSP OSSP} & - & \ref{a:Bit-Monnot23} & \ref{b:Bit-Monnot23}\\
\rowlabel{c:EfthymiouY23}EfthymiouY23 \href{https://doi.org/10.1007/978-3-031-33271-5\_16}{EfthymiouY23}~\cite{EfthymiouY23} & \href{works/EfthymiouY23.pdf}{Predicting the Optimal Period for Cyclic Hoist Scheduling Problems} & OR-Tools & benchmark, random instance, generated instance, real-life, industrial instance & 3 & n &  & n & - & CHSP & - & \ref{a:EfthymiouY23} & \ref{b:EfthymiouY23}\\
\rowlabel{c:JuvinHHL23}JuvinHHL23 \href{https://doi.org/10.4230/LIPIcs.CP.2023.19}{JuvinHHL23}~\cite{JuvinHHL23} & \href{works/JuvinHHL23.pdf}{An Efficient Constraint Programming Approach to Preemptive Job Shop Scheduling} & \su{{CP Opt} Mistral} & supplementary material, github, benchmark & 6 & ref &  & y &  & PJSSP & \su{endBeforeStart span noOverlap} & \ref{a:JuvinHHL23} & \ref{b:JuvinHHL23}\\
\rowlabel{c:JuvinHL23}JuvinHL23 \href{https://doi.org/10.1007/978-3-031-33271-5\_23}{JuvinHL23}~\cite{JuvinHL23} & \href{works/JuvinHL23.pdf}{Constraint Programming for the Robust Two-Machine Flow-Shop Scheduling Problem with Budgeted Uncertainty} & \su{{CP Opt} Cplex} & real-world & 0 & ref &  & n & - & Perm FSSP & \su{endBeforeStart noOverlap sameSequence} & \ref{a:JuvinHL23} & \ref{b:JuvinHL23}\\
\rowlabel{c:KameugneFND23}KameugneFND23 \href{https://doi.org/10.4230/LIPIcs.CP.2023.20}{KameugneFND23}~\cite{KameugneFND23} & \href{works/KameugneFND23.pdf}{Horizontally Elastic Edge Finder Rule for Cumulative Constraint Based on Slack and Density} & ? & benchmark & 5 & \su{BL PSPlib} &  & n & - & RCPSPs & cumulative & \ref{a:KameugneFND23} & \ref{b:KameugneFND23}\\
\rowlabel{c:KimCMLLP23}KimCMLLP23 \href{https://doi.org/10.1007/978-3-031-33271-5\_31}{KimCMLLP23}~\cite{KimCMLLP23} & \href{works/KimCMLLP23.pdf}{Iterated Greedy Constraint Programming for Scheduling Steelmaking Continuous Casting} & \su{Gurobi OR-Tools} & real-world, benchmark, zenodo & 0 & \href{https://zenodo.org/records/5126007}{y} &  & n & - & SCC & \su{alternative noOverlap} & \ref{a:KimCMLLP23} & \ref{b:KimCMLLP23}\\
\rowlabel{c:Mehdizadeh-Somarin23}Mehdizadeh-Somarin23 \href{https://doi.org/10.1007/978-3-031-43670-3\_33}{Mehdizadeh-Somarin23}~\cite{Mehdizadeh-Somarin23} & \href{works/Mehdizadeh-Somarin23.pdf}{A Constraint Programming Model for a Reconfigurable Job Shop Scheduling Problem with Machine Availability} & CP Opt & random instance & 0 & n &  & n & - & \su{JSSP RMS} & \su{alternative endBeforeStart noOverlap} & \ref{a:Mehdizadeh-Somarin23} & \ref{b:Mehdizadeh-Somarin23}\\
\rowlabel{c:PerezGSL23}PerezGSL23 \href{https://doi.org/10.1109/ICTAI59109.2023.00108}{PerezGSL23}~\cite{PerezGSL23} & \href{works/PerezGSL23.pdf}{A Constraint Programming Model for Scheduling the Unloading of Trains in Ports} & custom & real-world, generated instance & 0 & n &  & n & - & SUTP & \su{table disjunctive} & \ref{a:PerezGSL23} & \ref{b:PerezGSL23}\\
\rowlabel{c:PovedaAA23}PovedaAA23 \href{https://doi.org/10.4230/LIPIcs.CP.2023.31}{PovedaAA23}~\cite{PovedaAA23} & \href{works/PovedaAA23.pdf}{Partially Preemptive Multi Skill/Mode Resource-Constrained Project Scheduling with Generalized Precedence Relations and Calendars} & \su{{CP Opt} MiniZinc Chuffed} & real-world, github, benchmark, industrial instance, real-life & 4 & y &  & \href{https://github.com/youngkd/MSPSP-InstLib/blob/master/models/mspsp.mzn}{y} &  & PP-MS-MMRCPSP/max-cal &  & \ref{a:PovedaAA23} & \ref{b:PovedaAA23}\\
\rowlabel{c:SquillaciPR23}SquillaciPR23 \href{https://doi.org/10.1007/978-3-031-33271-5\_29}{SquillaciPR23}~\cite{SquillaciPR23} & \href{works/SquillaciPR23.pdf}{Scheduling Complex Observation Requests for a Constellation of Satellites: Large Neighborhood Search Approaches} & Cplex Studio & github, benchmark & 2 & \href{https://github.com/ssquilla/Earth_Observing_Satellites_benchmarks}{y} &  & n & - & EOSP & ? & \ref{a:SquillaciPR23} & \ref{b:SquillaciPR23}\\
\rowlabel{c:TardivoDFMP23}TardivoDFMP23 \href{https://doi.org/10.1007/978-3-031-33271-5\_22}{TardivoDFMP23}~\cite{TardivoDFMP23} & \href{works/TardivoDFMP23.pdf}{Constraint Propagation on {GPU:} {A} Case Study for the Cumulative Constraint} & \su{MiniCPP MiniZinc} & bitbucket, github, benchmark, real-world & 9 & \href{https://bitbucket.org/constraint-programming/minicpp-benchmarks/src/main/rcpsp/}{\su{PSPLib BL Pack}} &  & y & - & RCPSP & cumulative & \ref{a:TardivoDFMP23} & \ref{b:TardivoDFMP23}\\
\rowlabel{c:TasselGS23}TasselGS23 \href{https://doi.org/10.1609/icaps.v33i1.27243}{TasselGS23}~\cite{TasselGS23} & \href{works/TasselGS23.pdf}{An End-to-End Reinforcement Learning Approach for Job-Shop Scheduling Problems Based on Constraint Programming} & \su{custom Choco} & industrial instance, real-world, supplementary material, github, benchmark & 0 & ref &  & \href{https://github.com/ingambe/End2End-Job-Shop-Scheduling-CP}{y} & - & JSSP & noOverlap & \ref{a:TasselGS23} & \ref{b:TasselGS23}\\
\rowlabel{c:WangB23}WangB23 \href{https://doi.org/10.1109/ICTAI59109.2023.00062}{WangB23}~\cite{WangB23} & \href{works/WangB23.pdf}{Dynamic All-Different and Maximal Cliques Constraints for Fixed Job Scheduling} & FaCiLe & real-world, random instance & 0 & (y) &  & n & \cite{WangB20} & FJS & - & \ref{a:WangB23} & \ref{b:WangB23}\\
\rowlabel{c:YuraszeckMC23}YuraszeckMC23 \href{https://doi.org/10.1016/j.procs.2023.03.130}{YuraszeckMC23}~\cite{YuraszeckMC23} & \href{works/YuraszeckMC23.pdf}{A competitive constraint programming approach for the group shop scheduling problem} & CP Opt & github, benchmark & 0 & ref &  & n & - & GSSP & \su{noOverlap endBeforeStart} & \ref{a:YuraszeckMC23} & \ref{b:YuraszeckMC23}\\
\rowlabel{c:ArmstrongGOS22}ArmstrongGOS22 \href{https://doi.org/10.1007/978-3-031-08011-1\_1}{ArmstrongGOS22}~\cite{ArmstrongGOS22} & \href{works/ArmstrongGOS22.pdf}{A Two-Phase Hybrid Approach for the Hybrid Flexible Flowshop with Transportation Times} & CP Opt & real-world, benchmark & 0 & (y) &  & - & \cite{ArmstrongGOS21} & $HFFm|tt|C_{\max}$ & \su{endBeforeStart alternative cumulative noOverlap} & \ref{a:ArmstrongGOS22} & \ref{b:ArmstrongGOS22}\\
\rowlabel{c:BoudreaultSLQ22}BoudreaultSLQ22 \href{https://doi.org/10.4230/LIPIcs.CP.2022.10}{BoudreaultSLQ22}~\cite{BoudreaultSLQ22} & \href{works/BoudreaultSLQ22.pdf}{A Constraint Programming Approach to Ship Refit Project Scheduling} & \su{MiniZinc Chuffed} & benchmark, generated instance, supplementary material, gitlab, real-life, industrial partner, github, real-world & 9 &  &  & \href{https://github.com/raphaelboudreault/chuffed/releases/tag/SBPS}{y} & - & RCPSP & cumulative & \ref{a:BoudreaultSLQ22} & \ref{b:BoudreaultSLQ22}\\
\rowlabel{c:GeitzGSSW22}GeitzGSSW22 \href{https://doi.org/10.1007/978-3-031-08011-1\_10}{GeitzGSSW22}~\cite{GeitzGSSW22} & \href{works/GeitzGSSW22.pdf}{Solving the Extended Job Shop Scheduling Problem with AGVs - Classical and Quantum Approaches} & \su{firstCS QUBO} & real-life, github, real-world & 8 & \href{https://github.com/cgrozea/Data4ExtJSSAGV}{y} &  & n & - & JSSP &  & \ref{a:GeitzGSSW22} & \ref{b:GeitzGSSW22}\\
\rowlabel{c:HebrardALLCMR22}HebrardALLCMR22 \href{https://doi.org/10.24963/ijcai.2022/643}{HebrardALLCMR22}~\cite{HebrardALLCMR22} & \href{works/HebrardALLCMR22.pdf}{An Efficient Approach to Data Transfer Scheduling for Long Range Space Exploration} &  &  & 0 &  &  &  &  &  &  & \ref{a:HebrardALLCMR22} & \ref{b:HebrardALLCMR22}\\
\rowlabel{c:JungblutK22}JungblutK22 \href{https://doi.org/10.1109/IPDPSW55747.2022.00025}{JungblutK22}~\cite{JungblutK22} & \href{works/JungblutK22.pdf}{Optimal Schedules for High-Level Programming Environments on FPGAs with Constraint Programming} & MiniZinc & benchmark, github, real-world & 0 & \href{https://github.com/pascalj/reconf-scheduling}{y} &  & y & - &  &  & \ref{a:JungblutK22} & \ref{b:JungblutK22}\\
\rowlabel{c:LiFJZLL22}LiFJZLL22 \href{https://doi.org/10.1109/ICNSC55942.2022.10004158}{LiFJZLL22}~\cite{LiFJZLL22} & \href{works/LiFJZLL22.pdf}{Constraint Programming for a Novel Integrated Optimization of Blocking Job Shop Scheduling and Variable-Speed Transfer Robot Assignment} & \su{OPL {CP Opt}} & benchmark & 0 & ref &  & n & - & BJSSP & \su{endBEforeStart alternative noOverlap} & \ref{a:LiFJZLL22} & \ref{b:LiFJZLL22}\\
\rowlabel{c:LuoB22}LuoB22 \href{https://doi.org/10.1007/978-3-031-08011-1\_17}{LuoB22}~\cite{LuoB22} & \href{works/LuoB22.pdf}{Packing by Scheduling: Using Constraint Programming to Solve a Complex 2D Cutting Stock Problem} & CPO & generated instance, github, real-life, real-world, industry partner, industrial instance & 2 & n &  & n & - & 2SCSP-FF & \su{pulse alwaysIn forbidExtent stateFunction} & \ref{a:LuoB22} & \ref{b:LuoB22}\\
\rowlabel{c:OuelletQ22}OuelletQ22 \href{https://doi.org/10.1007/978-3-031-08011-1\_21}{OuelletQ22}~\cite{OuelletQ22} & \href{works/OuelletQ22.pdf}{A MinCumulative Resource Constraint} & Choco & github, benchmark, random instance & 1 & \href{https://github.com/yanickouellet/min-cumulative-paper-public}{y} &  & \href{https://github.com/yanickouellet/min-cumulative-paper-public}{y} & - &  & \su{cumulative minCumulative} & \ref{a:OuelletQ22} & \ref{b:OuelletQ22}\\
\rowlabel{c:OujanaAYB22}OujanaAYB22 \href{https://doi.org/10.1109/CoDIT55151.2022.9803972}{OujanaAYB22}~\cite{OujanaAYB22} & \href{works/OujanaAYB22.pdf}{Solving a realistic hybrid and flexible flow shop scheduling problem through constraint programming: industrial case in a packaging company} & CP Opt & benchmark, industrial instance, real-world, real-life & 0 & n &  & n & - & HFFS & \su{alternative span noOverlap endBeforeStart} & \ref{a:OujanaAYB22} & \ref{b:OujanaAYB22}\\
\rowlabel{c:PopovicCGNC22}PopovicCGNC22 \href{https://doi.org/10.4230/LIPIcs.CP.2022.34}{PopovicCGNC22}~\cite{PopovicCGNC22} & \href{works/PopovicCGNC22.pdf}{Scheduling the Equipment Maintenance of an Electric Power Transmission Network Using Constraint Programming} & CP Opt &  & 0 & n &  & n & - & TMS & \su{alwaysIn noOverlap} & \ref{a:PopovicCGNC22} & \ref{b:PopovicCGNC22}\\
\rowlabel{c:SvancaraB22}SvancaraB22 \href{https://doi.org/10.5220/0010869700003116}{SvancaraB22}~\cite{SvancaraB22} & \href{works/SvancaraB22.pdf}{Tackling Train Routing via Multi-agent Pathfinding and Constraint-based Scheduling} &  & benchmark, real-world & 0 &  &  &  &  &  &  & \ref{a:SvancaraB22} & \ref{b:SvancaraB22}\\
\rowlabel{c:Teppan22}Teppan22 \href{https://doi.org/10.5220/0010849900003116}{Teppan22}~\cite{Teppan22} & \href{works/Teppan22.pdf}{Types of Flexible Job Shop Scheduling: {A} Constraint Programming Experiment} & OPL & real-life, benchmark & 0 & ref &  & n & - & FJSSP & \su{noOverlap alternative endBeforeStart} & \ref{a:Teppan22} & \ref{b:Teppan22}\\
\rowlabel{c:TouatBT22}TouatBT22 \href{}{TouatBT22}~\cite{TouatBT22} & \href{works/TouatBT22.pdf}{A Constraint Programming Model for the Scheduling Problem with Flexible Maintenance under Human Resource Constraints} & OPL & benchmark, generated instance & 0 & n &  & n & - & Single Machine Scheduling & \su{alternative noOverlap forbidExtent} & \ref{a:TouatBT22} & \ref{b:TouatBT22}\\
\rowlabel{c:WinterMMW22}WinterMMW22 \href{https://doi.org/10.4230/LIPIcs.CP.2022.41}{WinterMMW22}~\cite{WinterMMW22} & \href{works/WinterMMW22.pdf}{Modeling and Solving Parallel Machine Scheduling with Contamination Constraints in the Agricultural Industry} & \su{Cplex Gurobi {CP Opt} {Sim Anneal}} & supplementary material, real-life, industry partner, zenodo, industrial partner, benchmark & 0 & \href{https://zenodo.org/records/6797397}{y} &  & \href{https://zenodo.org/records/6797397}{y} & - & PMSP & \su{alternative noOverlap} & \ref{a:WinterMMW22} & \ref{b:WinterMMW22}\\
\rowlabel{c:ZhangBB22}ZhangBB22 \href{https://ojs.aaai.org/index.php/ICAPS/article/view/19826}{ZhangBB22}~\cite{ZhangBB22} & \href{works/ZhangBB22.pdf}{Solving Job-Shop Scheduling Problems with QUBO-Based Specialized Hardware} &  & benchmark, generated instance & 0 &  &  &  &  &  &  & \ref{a:ZhangBB22} & \ref{b:ZhangBB22}\\
\rowlabel{c:ZhangJZL22}ZhangJZL22 \href{https://doi.org/10.1109/ICNSC55942.2022.10004154}{ZhangJZL22}~\cite{ZhangJZL22} & \href{works/ZhangJZL22.pdf}{Constraint Programming for Modeling and Solving a Hybrid Flow Shop Scheduling Problem} & OP Opt & benchmark & 0 & ref &  & n & - & HFSP & \su{alternative endBeforeStart noOverlap cumulative} & \ref{a:ZhangJZL22} & \ref{b:ZhangJZL22}\\
\rowlabel{c:AntuoriHHEN21}AntuoriHHEN21 \href{https://doi.org/10.4230/LIPIcs.CP.2021.14}{AntuoriHHEN21}~\cite{AntuoriHHEN21} & \href{works/AntuoriHHEN21.pdf}{Combining Monte Carlo Tree Search and Depth First Search Methods for a Car Manufacturing Workshop Scheduling Problem} & MCTS & gitlab, supplementary material & 1 & \href{https://gitlab.laas.fr/vantuori/mcts-cp}{y} &  & \href{https://gitlab.laas.fr/vantuori/mcts-cp}{y} &  &  &  & \ref{a:AntuoriHHEN21} & \ref{b:AntuoriHHEN21}\\
\rowlabel{c:ArmstrongGOS21}ArmstrongGOS21 \href{https://doi.org/10.4230/LIPIcs.CP.2021.16}{ArmstrongGOS21}~\cite{ArmstrongGOS21} & \href{works/ArmstrongGOS21.pdf}{The Hybrid Flexible Flowshop with Transportation Times} & \su{MiniZinc Chuffed {CP Opt} SICStus} & instance generator, industry partner, zenodo, supplementary material, real-world, industrial partner, benchmark & 1 & \href{https://zenodo.org/record/5168966}{y} &  & y & - & $HFFm|tt|C_{\max}$ & \su{cumulative diffn table} & \ref{a:ArmstrongGOS21} & \ref{b:ArmstrongGOS21}\\
\rowlabel{c:ArtiguesHQT21}ArtiguesHQT21 \href{https://doi.org/10.5220/0010190101290136}{ArtiguesHQT21}~\cite{ArtiguesHQT21} & \href{}{Multi-Mode {RCPSP} with Safety Margin Maximization: Models and Algorithms} &  &  & 0 &  &  &  &  &  &  & \ref{a:ArtiguesHQT21} & No\\
\rowlabel{c:Astrand0F21}Astrand0F21 \href{https://doi.org/10.1007/978-3-030-78230-6\_23}{Astrand0F21}~\cite{Astrand0F21} & \href{works/Astrand0F21.pdf}{Short-Term Scheduling of Production Fleets in Underground Mines Using CP-Based {LNS}} & Gecode & benchmark, real-world, real-life, generated instance & 0 & \su{ref generated} &  & n & - &  & - & \ref{a:Astrand0F21} & \ref{b:Astrand0F21}\\
\rowlabel{c:BenderWS21}BenderWS21 \href{https://doi.org/10.1007/978-3-030-87672-2\_37}{BenderWS21}~\cite{BenderWS21} & \href{works/BenderWS21.pdf}{Applying Constraint Programming to the Multi-mode Scheduling Problem in Harvest Logistics} & CP Opt &  & 9 & \href{https://tud.link/47mz}{y} &  & n & - & MRCPSP & \su{noOverlap alternative} & \ref{a:BenderWS21} & \ref{b:BenderWS21}\\
\rowlabel{c:GeibingerKKMMW21}GeibingerKKMMW21 \href{https://doi.org/10.1007/978-3-030-78230-6\_29}{GeibingerKKMMW21}~\cite{GeibingerKKMMW21} & \href{works/GeibingerKKMMW21.pdf}{Physician Scheduling During a Pandemic} & MiniZinc & real-world & 3 & \href{https://cdlab-artis.dbai.tuwien.ac.at/papers/pandemic-scheduling/}{y} &  & n & - &  & nvalue & \ref{a:GeibingerKKMMW21} & \ref{b:GeibingerKKMMW21}\\
\rowlabel{c:GeibingerMM21}GeibingerMM21 \href{https://doi.org/10.1609/aaai.v35i7.16789}{GeibingerMM21}~\cite{GeibingerMM21} & \href{works/GeibingerMM21.pdf}{Constraint Logic Programming for Real-World Test Laboratory Scheduling} & clingcon & real-life, github, generated instance, real-world, benchmark & 0 & \href{dbai.tuwien.ac.at/staff/fmischek/TLSP}{y} &  &  &  & \su{TLSP RCPSP} & disjunctive & \ref{a:GeibingerMM21} & \ref{b:GeibingerMM21}\\
\rowlabel{c:HanenKP21}HanenKP21 \href{https://doi.org/10.1007/978-3-030-78230-6\_14}{HanenKP21}~\cite{HanenKP21} & \href{works/HanenKP21.pdf}{Two Deadline Reduction Algorithms for Scheduling Dependent Tasks on Parallel Processors} & Python & Roadef, generated instance, random instance & 1 & ref &  & n & - & $P|prec, r_i, d_i|*$ & - & \ref{a:HanenKP21} & \ref{b:HanenKP21}\\
\rowlabel{c:HillTV21}HillTV21 \href{https://doi.org/10.1007/978-3-030-78230-6\_2}{HillTV21}~\cite{HillTV21} & \href{works/HillTV21.pdf}{A Computational Study of Constraint Programming Approaches for Resource-Constrained Project Scheduling with Autonomous Learning Effects} & CP Opt & real-world & 0 & PSPlib &  & n & - & RCPSP & \su{cumulative alternative endBeforeStart} & \ref{a:HillTV21} & \ref{b:HillTV21}\\
\rowlabel{c:KlankeBYE21}KlankeBYE21 \href{https://doi.org/10.1007/978-3-030-78230-6\_9}{KlankeBYE21}~\cite{KlankeBYE21} & \href{works/KlankeBYE21.pdf}{Combining Constraint Programming and Temporal Decomposition Approaches - Scheduling of an Industrial Formulation Plant} & OR-Tools & benchmark, random instance, real-life & 0 & n &  & n & - &  & \su{cumulative circuit noOverlap} & \ref{a:KlankeBYE21} & \ref{b:KlankeBYE21}\\
\rowlabel{c:KovacsTKSG21}KovacsTKSG21 \href{https://doi.org/10.4230/LIPIcs.CP.2021.36}{KovacsTKSG21}~\cite{KovacsTKSG21} & \href{works/KovacsTKSG21.pdf}{Utilizing Constraint Optimization for Industrial Machine Workload Balancing} & \su{Gurobi OR-Tools Cplex {CP Opt}} & github, supplementary material, real-world, benchmark & 2 & \href{https://github.com/prosysscience/CPWorkloadBalancing}{y} &  & \href{https://github.com/prosysscience/CPWorkloadBalancing}{y} & - & extended RCPSP & cumulative & \ref{a:KovacsTKSG21} & \ref{b:KovacsTKSG21}\\
\rowlabel{c:LacknerMMWW21}LacknerMMWW21 \href{https://doi.org/10.4230/LIPIcs.CP.2021.37}{LacknerMMWW21}~\cite{LacknerMMWW21} & \href{works/LacknerMMWW21.pdf}{Minimizing Cumulative Batch Processing Time for an Industrial Oven Scheduling Problem} & \su{{CP Opt} Chuffed OR-Tools Gurobi OPL} & random instance, industrial partner, benchmark, instance generator, real-life, supplementary material & 3 & \href{https://cdlab-artis.dbai.tuwien.ac.at/papers/ovenscheduling/}{y} &  & \href{https://cdlab-artis.dbai.tuwien.ac.at/papers/ovenscheduling/}{y} &  & OSP &  & \ref{a:LacknerMMWW21} & \ref{b:LacknerMMWW21}\\
\rowlabel{c:AntuoriHHEN20}AntuoriHHEN20 \href{https://doi.org/10.1007/978-3-030-58475-7\_38}{AntuoriHHEN20}~\cite{AntuoriHHEN20} & \href{works/AntuoriHHEN20.pdf}{Leveraging Reinforcement Learning, Constraint Programming and Local Search: {A} Case Study in Car Manufacturing} &  & random instance, generated instance, gitlab, benchmark, industrial instance & 4 &  &  &  &  &  &  & \ref{a:AntuoriHHEN20} & \ref{b:AntuoriHHEN20}\\
\rowlabel{c:BarzegaranZP20}BarzegaranZP20 \href{https://doi.org/10.4230/OASIcs.Fog-IoT.2020.3}{BarzegaranZP20}~\cite{BarzegaranZP20} & \href{works/BarzegaranZP20.pdf}{Quality-Of-Control-Aware Scheduling of Communication in TSN-Based Fog Computing Platforms Using Constraint Programming} & OR-Tools &  & 5 & n &  & n & - & FCP &  & \ref{a:BarzegaranZP20} & \ref{b:BarzegaranZP20}\\
\rowlabel{c:GodetLHS20}GodetLHS20 \href{https://doi.org/10.1609/aaai.v34i02.5510}{GodetLHS20}~\cite{GodetLHS20} & \href{works/GodetLHS20.pdf}{Using Approximation within Constraint Programming to Solve the Parallel Machine Scheduling Problem with Additional Unit Resources} & \su{MiniZinc Choco Chuffed} & github, real-life, benchmark, generated instance & 0 & \href{https://github.com/ArthurGodet/PMSPAUR-public}{JSON} &  & \href{https://github.com/ArthurGodet/PMSPAUR-public}{y} & - & PMSPAUR & \su{disjunctive cumulative alldifferent enqueueCstr approxCstr} & \ref{a:GodetLHS20} & \ref{b:GodetLHS20}\\
\rowlabel{c:GroleazNS20}GroleazNS20 \href{https://doi.org/10.1007/978-3-030-58475-7\_36}{GroleazNS20}~\cite{GroleazNS20} & \href{works/GroleazNS20.pdf}{Solving the Group Cumulative Scheduling Problem with {CPO} and {ACO}} & \su{{CP Opt} ACO} & benchmark, industrial instance & 0 & - &  & - & \cite{GroleazNS20} & GCSP & groupCumulative & \ref{a:GroleazNS20} & \ref{b:GroleazNS20}\\
\rowlabel{c:GroleazNS20a}GroleazNS20a \href{https://doi.org/10.1145/3377930.3389818}{GroleazNS20a}~\cite{GroleazNS20a} & \href{works/GroleazNS20a.pdf}{{ACO} with automatic parameter selection for a scheduling problem with a group cumulative constraint} & \su{CPO ACO} & industrial partner, benchmark & 0 & \href{https://perso.citi-lab.fr/csolnon/gc-sched.html}{y} &  & n & - & GCSP & \su{groupCumulative} & \ref{a:GroleazNS20a} & \ref{b:GroleazNS20a}\\
\rowlabel{c:Mercier-AubinGQ20}Mercier-AubinGQ20 \href{https://doi.org/10.1007/978-3-030-58942-4\_22}{Mercier-AubinGQ20}~\cite{Mercier-AubinGQ20} & \href{works/Mercier-AubinGQ20.pdf}{Leveraging Constraint Scheduling: {A} Case Study to the Textile Industry} & \su{MiniZinc Chuffed} & industrial instance, industrial partner & 1 & a &  & a & - &  & \su{circuit cumulative} & \ref{a:Mercier-AubinGQ20} & \ref{b:Mercier-AubinGQ20}\\
\rowlabel{c:NattafM20}NattafM20 \href{https://doi.org/10.1007/978-3-030-58475-7\_27}{NattafM20}~\cite{NattafM20} & \href{works/NattafM20.pdf}{Filtering Rules for Flow Time Minimization in a Parallel Machine Scheduling Problem} & \su{Cplex {CP Opt}} & benchmark, industrial instance & 7 & - &  & - & \cite{MalapertN19} & PTC & \su{alternative noOverlap} & \ref{a:NattafM20} & \ref{b:NattafM20}\\
\rowlabel{c:TangB20}TangB20 \href{https://doi.org/10.1007/978-3-030-58942-4\_28}{TangB20}~\cite{TangB20} & \href{works/TangB20.pdf}{{CP} and Hybrid Models for Two-Stage Batching and Scheduling} & \su{Cplex {CP Opt}} & real-world & 0 & n &  & n & - & 2BPHFSP & \su{span alwaysIn} & \ref{a:TangB20} & \ref{b:TangB20}\\
\rowlabel{c:WangB20}WangB20 \href{https://doi.org/10.3233/FAIA200114}{WangB20}~\cite{WangB20} & \href{works/WangB20.pdf}{Global Propagation of Transition Cost for Fixed Job Scheduling} & FaCiLe & github & 0 & \href{http://recherche.enac.fr/~wangrx/ecai_gap/}{y} &  & n & - & FJS & - & \ref{a:WangB20} & \ref{b:WangB20}\\
\rowlabel{c:WessenCS20}WessenCS20 \href{https://doi.org/10.1007/978-3-030-58942-4\_33}{WessenCS20}~\cite{WessenCS20} & \href{works/WessenCS20.pdf}{Scheduling of Dual-Arm Multi-tool Assembly Robots and Workspace Layout Optimization} & Gecode & real-world & 10 & n &  & n & - &  & \su{circuit alldifferent} & \ref{a:WessenCS20} & \ref{b:WessenCS20}\\
\rowlabel{c:BadicaBIL19}BadicaBIL19 \href{https://doi.org/10.1007/978-3-030-32258-8\_17}{BadicaBIL19}~\cite{BadicaBIL19} & \href{works/BadicaBIL19.pdf}{Exploring the Space of Block Structured Scheduling Processes Using Constraint Logic Programming} & ECLiPSe & github & 0 & dead &  & dead & - &  &  & \ref{a:BadicaBIL19} & \ref{b:BadicaBIL19}\\
\rowlabel{c:BehrensLM19}BehrensLM19 \href{https://doi.org/10.1109/ICRA.2019.8794022}{BehrensLM19}~\cite{BehrensLM19} & \href{works/BehrensLM19.pdf}{A Constraint Programming Approach to Simultaneous Task Allocation and Motion Scheduling for Industrial Dual-Arm Manipulation Tasks} & OR-Tools & real-world, github & 0 & \href{https://github.com/boschresearch/STAAMS-SOLVER}{y} &  & \href{https://github.com/boschresearch/STAAMS-SOLVER}{y} & - & STAAMS &  & \ref{a:BehrensLM19} & \ref{b:BehrensLM19}\\
\rowlabel{c:BogaerdtW19}BogaerdtW19 \href{https://doi.org/10.1007/978-3-030-19212-9\_38}{BogaerdtW19}~\cite{BogaerdtW19} & \href{works/BogaerdtW19.pdf}{Lower Bounds for Uniform Machine Scheduling Using Decision Diagrams} & \su{custom Cplex CPO} & benchmark & 4 & n &  & n & - & Multi Machine Scheduling & \su{noOverlap} & \ref{a:BogaerdtW19} & \ref{b:BogaerdtW19}\\
\rowlabel{c:ColT19}ColT19 \href{https://doi.org/10.1007/978-3-030-30048-7\_9}{ColT19}~\cite{ColT19} & \href{works/ColT19.pdf}{Industrial Size Job Shop Scheduling Tackled by Present Day {CP} Solvers} & \su{{CP Opt} OR-Tools} & github, benchmark, real-world & 2 & \href{https://drive.google.com/drive/folders/1QuKEABR9aiNKPIFe0VMFXP7BNor8KW9b}{y} &  & \href{https://drive.google.com/drive/folders/1QuKEABR9aiNKPIFe0VMFXP7BNor8KW9b}{y} & - & JSSP & \su{noOverlap} & \ref{a:ColT19} & \ref{b:ColT19}\\
\rowlabel{c:FrimodigS19}FrimodigS19 \href{https://doi.org/10.1007/978-3-030-30048-7\_25}{FrimodigS19}~\cite{FrimodigS19} & \href{works/FrimodigS19.pdf}{Models for Radiation Therapy Patient Scheduling} & \su{Mini-Zinc Gecode Cplex} & benchmark, real-world & 1 & n &  & n & - &  & \su{cumulative regular bin-packing} & \ref{a:FrimodigS19} & \ref{b:FrimodigS19}\\
\rowlabel{c:FrohnerTR19}FrohnerTR19 \href{https://doi.org/10.1007/978-3-030-45093-9\_34}{FrohnerTR19}~\cite{FrohnerTR19} & \href{works/FrohnerTR19.pdf}{Casual Employee Scheduling with Constraint Programming and Metaheuristics} &  & benchmark, real-world & 0 &  &  &  &  &  &  & \ref{a:FrohnerTR19} & \ref{b:FrohnerTR19}\\
\rowlabel{c:GalleguillosKSB19}GalleguillosKSB19 \href{https://doi.org/10.1007/978-3-030-30048-7\_26}{GalleguillosKSB19}~\cite{GalleguillosKSB19} & \href{works/GalleguillosKSB19.pdf}{Constraint Programming-Based Job Dispatching for Modern {HPC} Applications} & \su{OR-Tools} &  & 5 &  &  & \href{https://github.com/cgalleguillosm/cp_dispatchers}{y} &  & on-line dispatch &  & \ref{a:GalleguillosKSB19} & \ref{b:GalleguillosKSB19}\\
\rowlabel{c:GeibingerMM19}GeibingerMM19 \href{https://doi.org/10.1007/978-3-030-19212-9\_20}{GeibingerMM19}~\cite{GeibingerMM19} & \href{works/GeibingerMM19.pdf}{Investigating Constraint Programming for Real World Industrial Test Laboratory Scheduling} &  & real-life, generated instance, industrial partner, real-world, benchmark & 3 &  &  &  &  &  &  & \ref{a:GeibingerMM19} & \ref{b:GeibingerMM19}\\
\rowlabel{c:KucukY19}KucukY19 \href{https://api.semanticscholar.org/CorpusID:198146161}{KucukY19}~\cite{KucukY19} & \href{works/KucukY19.pdf}{A Constraint Programming Approach for Agile Earth Observation Satellite Scheduling Problem} &  & benchmark, generated instance & 0 &  &  &  &  &  &  & \ref{a:KucukY19} & \ref{b:KucukY19}\\
\rowlabel{c:LiuLH19}LiuLH19 \href{https://doi.org/10.1007/978-3-030-19823-7\_19}{LiuLH19}~\cite{LiuLH19} & \href{works/LiuLH19.pdf}{Solving the Talent Scheduling Problem by Parallel Constraint Programming} &  & CSPlib, benchmark & 0 &  &  &  &  &  &  & \ref{a:LiuLH19} & \ref{b:LiuLH19}\\
\rowlabel{c:MalapertN19}MalapertN19 \href{https://doi.org/10.1007/978-3-030-19212-9\_28}{MalapertN19}~\cite{MalapertN19} & \href{works/MalapertN19.pdf}{A New CP-Approach for a Parallel Machine Scheduling Problem with Time Constraints on Machine Qualifications} &  & generated instance, benchmark, industrial instance, Roadef & 3 &  &  &  &  &  &  & \ref{a:MalapertN19} & \ref{b:MalapertN19}\\
\rowlabel{c:MurinR19}MurinR19 \href{https://doi.org/10.1007/978-3-030-30048-7\_27}{MurinR19}~\cite{MurinR19} & \href{works/MurinR19.pdf}{Scheduling of Mobile Robots Using Constraint Programming} & \su{{CP Opt} Cplex OPL} & real-life, benchmark, github & 3 & \href{https://github.com/StanislavMurin/Scheduling-of-Mobile-Robots-using-Constraint-Programming}{y} &  & \href{https://github.com/StanislavMurin/Scheduling-of-Mobile-Robots-using-Constraint-Programming}{y} &  & JSPT & \su{endBeforeStart alternative noOverlap} & \ref{a:MurinR19} & \ref{b:MurinR19}\\
\rowlabel{c:ParkUJR19}ParkUJR19 \href{https://doi.org/10.1007/978-3-030-19648-6\_15}{ParkUJR19}~\cite{ParkUJR19} & \href{works/ParkUJR19.pdf}{Developing a Production Scheduling System for Modular Factory Using Constraint Programming} &  & real-world & 0 &  &  &  &  &  &  & \ref{a:ParkUJR19} & \ref{b:ParkUJR19}\\
\rowlabel{c:Tom19}Tom19 \href{https://doi.org/10.1109/FUZZ-IEEE.2019.8859029}{Tom19}~\cite{Tom19} & \href{works/Tom19.pdf}{Fuzzy Multi-Constraint Programming Model for Weekly Meals Scheduling} &  & real-world & 0 &  &  &  &  &  &  & \ref{a:Tom19} & \ref{b:Tom19}\\
\rowlabel{c:YangSS19}YangSS19 \href{https://doi.org/10.1007/978-3-030-19212-9\_42}{YangSS19}~\cite{YangSS19} & \href{works/YangSS19.pdf}{Time Table Edge Finding with Energy Variables} &  & generated instance & 1 &  &  &  &  &  &  & \ref{a:YangSS19} & \ref{b:YangSS19}\\
\rowlabel{c:ArbaouiY18}ArbaouiY18 \href{https://doi.org/10.1007/978-3-319-75420-8\_67}{ArbaouiY18}~\cite{ArbaouiY18} & \href{works/ArbaouiY18.pdf}{Solving the Unrelated Parallel Machine Scheduling Problem with Additional Resources Using Constraint Programming} &  & benchmark & 0 &  &  &  &  &  &  & \ref{a:ArbaouiY18} & \ref{b:ArbaouiY18}\\
\rowlabel{c:AstrandJZ18}AstrandJZ18 \href{https://doi.org/10.1007/978-3-319-93031-2\_44}{AstrandJZ18}~\cite{AstrandJZ18} & \href{works/AstrandJZ18.pdf}{Fleet Scheduling in Underground Mines Using Constraint Programming} &  &  & 0 &  &  &  &  &  &  & \ref{a:AstrandJZ18} & \ref{b:AstrandJZ18}\\
\rowlabel{c:BenediktSMVH18}BenediktSMVH18 \href{https://doi.org/10.1007/978-3-319-93031-2\_6}{BenediktSMVH18}~\cite{BenediktSMVH18} & \href{works/BenediktSMVH18.pdf}{Energy-Aware Production Scheduling with Power-Saving Modes} & \su{CPO Gurobi} & github, random instance, generated instance & 1 & \href{https://github.com/CTU-IIG/PSPSM}{y} &  & \href{https://github.com/CTU-IIG/PSPSM}{y} & - & Energy Aware Production Scheduling &  & \ref{a:BenediktSMVH18} & \ref{b:BenediktSMVH18}\\
\rowlabel{c:CappartTSR18}CappartTSR18 \href{https://doi.org/10.1007/978-3-319-98334-9\_32}{CappartTSR18}~\cite{CappartTSR18} & \href{works/CappartTSR18.pdf}{A Constraint Programming Approach for Solving Patient Transportation Problems} &  & bitbucket, CSPlib, real-life & 1 &  &  &  &  &  &  & \ref{a:CappartTSR18} & \ref{b:CappartTSR18}\\
\rowlabel{c:DemirovicS18}DemirovicS18 \href{https://doi.org/10.1007/978-3-319-93031-2\_10}{DemirovicS18}~\cite{DemirovicS18} & \href{works/DemirovicS18.pdf}{Constraint Programming for High School Timetabling: {A} Scheduling-Based Model with Hot Starts} &  & real-world, benchmark & 5 &  &  &  &  &  &  & \ref{a:DemirovicS18} & \ref{b:DemirovicS18}\\
\rowlabel{c:He0GLW18}He0GLW18 \href{https://doi.org/10.1007/978-3-319-98334-9\_42}{He0GLW18}~\cite{He0GLW18} & \href{works/He0GLW18.pdf}{A Fast and Scalable Algorithm for Scheduling Large Numbers of Devices Under Real-Time Pricing} & \su{Gurobi Python} & real-world, bitbucket & 8 & \href{https://bitbucket.org/monash-dr/deterministic-rtp-ad/src/master/}{y} &  & \href{https://bitbucket.org/monash-dr/deterministic-rtp-ad/src/master/}{y} & - & \su{FSDN-DS DSP-MH-RTP} &  & \ref{a:He0GLW18} & \ref{b:He0GLW18}\\
\rowlabel{c:HoYCLLCLC18}HoYCLLCLC18 \href{https://doi.org/10.1145/3299819.3299825}{HoYCLLCLC18}~\cite{HoYCLLCLC18} & \href{works/HoYCLLCLC18.pdf}{A Platform for Dynamic Optimal Nurse Scheduling Based on Integer Linear Programming along with Multiple Criteria Constraints} &  & real-world & 0 &  &  &  &  &  &  & \ref{a:HoYCLLCLC18} & \ref{b:HoYCLLCLC18}\\
\rowlabel{c:KameugneFGOQ18}KameugneFGOQ18 \href{https://doi.org/10.1007/978-3-319-93031-2\_23}{KameugneFGOQ18}~\cite{KameugneFGOQ18} & \href{works/KameugneFGOQ18.pdf}{Horizontally Elastic Not-First/Not-Last Filtering Algorithm for Cumulative Resource Constraint} &  & benchmark, real-world & 0 &  &  &  &  &  &  & \ref{a:KameugneFGOQ18} & \ref{b:KameugneFGOQ18}\\
\rowlabel{c:Laborie18a}Laborie18a \href{https://doi.org/10.1007/978-3-319-93031-2\_29}{Laborie18a}~\cite{Laborie18a} & \href{works/Laborie18a.pdf}{An Update on the Comparison of MIP, {CP} and Hybrid Approaches for Mixed Resource Allocation and Scheduling} &  & real-life, benchmark, real-world & 0 &  &  &  &  &  &  & \ref{a:Laborie18a} & \ref{b:Laborie18a}\\
\rowlabel{c:MusliuSS18}MusliuSS18 \href{https://doi.org/10.1007/978-3-319-93031-2\_31}{MusliuSS18}~\cite{MusliuSS18} & \href{works/MusliuSS18.pdf}{Solver Independent Rotating Workforce Scheduling} &  & generated instance, benchmark, real-life & 2 &  &  &  &  &  &  & \ref{a:MusliuSS18} & \ref{b:MusliuSS18}\\
\rowlabel{c:NishikawaSTT18}NishikawaSTT18 \href{https://doi.org/10.1109/CANDAR.2018.00025}{NishikawaSTT18}~\cite{NishikawaSTT18} & \href{works/NishikawaSTT18.pdf}{Scheduling of Malleable Fork-Join Tasks with Constraint Programming} &  & real-world, benchmark & 0 &  &  &  &  &  &  & \ref{a:NishikawaSTT18} & \ref{b:NishikawaSTT18}\\
\rowlabel{c:NishikawaSTT18a}NishikawaSTT18a \href{https://doi.org/10.1109/TENCON.2018.8650168}{NishikawaSTT18a}~\cite{NishikawaSTT18a} & \href{works/NishikawaSTT18a.pdf}{Scheduling of Malleable Tasks Based on Constraint Programming} &  & real-world, benchmark, real-life & 0 &  &  &  &  &  &  & \ref{a:NishikawaSTT18a} & \ref{b:NishikawaSTT18a}\\
\rowlabel{c:OuelletQ18}OuelletQ18 \href{https://doi.org/10.1007/978-3-319-93031-2\_34}{OuelletQ18}~\cite{OuelletQ18} & \href{works/OuelletQ18.pdf}{A O(n {\textbackslash}log {\^{}}2 n) Checker and O(n{\^{}}2 {\textbackslash}log n) Filtering Algorithm for the Energetic Reasoning} &  & benchmark, Roadef & 0 &  &  &  &  &  &  & \ref{a:OuelletQ18} & \ref{b:OuelletQ18}\\
\rowlabel{c:RiahiNS018}RiahiNS018 \href{https://aaai.org/ocs/index.php/ICAPS/ICAPS18/paper/view/17755}{RiahiNS018}~\cite{RiahiNS018} & \href{works/RiahiNS018.pdf}{Local Search for Flowshops with Setup Times and Blocking Constraints} &  & real-world, real-life, benchmark & 0 &  &  &  &  &  &  & \ref{a:RiahiNS018} & \ref{b:RiahiNS018}\\
\rowlabel{c:Tesch18}Tesch18 \href{https://doi.org/10.1007/978-3-319-98334-9\_41}{Tesch18}~\cite{Tesch18} & \href{works/Tesch18.pdf}{Improving Energetic Propagations for Cumulative Scheduling} &  & Roadef & 0 &  &  &  &  &  &  & \ref{a:Tesch18} & \ref{b:Tesch18}\\
\rowlabel{c:BofillCSV17}BofillCSV17 \href{https://doi.org/10.1007/978-3-319-66158-2\_5}{BofillCSV17}~\cite{BofillCSV17} & \href{works/BofillCSV17.pdf}{An Efficient {SMT} Approach to Solve MRCPSP/max Instances with Tight Constraints on Resources} &  & benchmark & 2 &  &  &  &  &  &  & \ref{a:BofillCSV17} & \ref{b:BofillCSV17}\\
\rowlabel{c:CappartS17}CappartS17 \href{https://doi.org/10.1007/978-3-319-59776-8\_26}{CappartS17}~\cite{CappartS17} & \href{works/CappartS17.pdf}{Rescheduling Railway Traffic on Real Time Situations Using Time-Interval Variables} & CPO & bitbucket, random instance, real-life & 1 & \href{https://bitbucket.org/qcappart/qcappart_opendata/src/master/}{y} &  & n & - & Rescheduling Railway Traffic &  & \ref{a:CappartS17} & \ref{b:CappartS17}\\
\rowlabel{c:CohenHB17}CohenHB17 \href{https://doi.org/10.1007/978-3-319-66263-3\_10}{CohenHB17}~\cite{CohenHB17} & \href{works/CohenHB17.pdf}{{(I} Can Get) Satisfaction: Preference-Based Scheduling for Concert-Goers at Multi-venue Music Festivals} &  &  & 12 &  &  &  &  &  &  & \ref{a:CohenHB17} & \ref{b:CohenHB17}\\
\rowlabel{c:GelainPRVW17}GelainPRVW17 \href{https://doi.org/10.1007/978-3-319-59776-8\_32}{GelainPRVW17}~\cite{GelainPRVW17} & \href{works/GelainPRVW17.pdf}{A Local Search Approach for Incomplete Soft Constraint Problems: Experimental Results on Meeting Scheduling Problems} &  & CSPlib, real-life, benchmark & 2 &  &  &  &  &  &  & \ref{a:GelainPRVW17} & \ref{b:GelainPRVW17}\\
\rowlabel{c:GoldwaserS17}GoldwaserS17 \href{https://doi.org/10.1007/978-3-319-66158-2\_22}{GoldwaserS17}~\cite{GoldwaserS17} & \href{works/GoldwaserS17.pdf}{Optimal Torpedo Scheduling} & \su{Chuffed Gurobi} & instance generator, github, generated instance & 4 & \href{https://github.com/AdGold/TorpedoSchedulingInstances}{y} &  & n & - & Torpedo Scheduling &  & \ref{a:GoldwaserS17} & \ref{b:GoldwaserS17}\\
\rowlabel{c:Hooker17}Hooker17 \href{https://doi.org/10.1007/978-3-319-66158-2\_36}{Hooker17}~\cite{Hooker17} & \href{works/Hooker17.pdf}{Job Sequencing Bounds from Decision Diagrams} &  & benchmark, random instance & 0 &  &  &  &  &  &  & \ref{a:Hooker17} & \ref{b:Hooker17}\\
\rowlabel{c:KletzanderM17}KletzanderM17 \href{https://doi.org/10.1007/978-3-319-59776-8\_28}{KletzanderM17}~\cite{KletzanderM17} & \href{works/KletzanderM17.pdf}{A Multi-stage Simulated Annealing Algorithm for the Torpedo Scheduling Problem} &  &  & 2 &  &  &  &  &  &  & \ref{a:KletzanderM17} & \ref{b:KletzanderM17}\\
\rowlabel{c:LiuCGM17}LiuCGM17 \href{https://doi.org/10.1007/978-3-319-66158-2\_24}{LiuCGM17}~\cite{LiuCGM17} & \href{works/LiuCGM17.pdf}{NightSplitter: {A} Scheduling Tool to Optimize (Sub)group Activities} & \su{Chuffed OR-Tools HCSP SA} & github & 11 & n &  & \href{https://cs.unibo.it/t.liu/nightsplitter/mzn.html} & - & NightSplit &  & \ref{a:LiuCGM17} & \ref{b:LiuCGM17}\\
\rowlabel{c:Madi-WambaLOBM17}Madi-WambaLOBM17 \href{https://doi.org/10.1109/ICPADS.2017.00089}{Madi-WambaLOBM17}~\cite{Madi-WambaLOBM17} & \href{works/Madi-WambaLOBM17.pdf}{Green Energy Aware Scheduling Problem in Virtualized Datacenters} &  & real-world & 0 &  &  &  &  &  &  & \ref{a:Madi-WambaLOBM17} & \ref{b:Madi-WambaLOBM17}\\
\rowlabel{c:MossigeGSMC17}MossigeGSMC17 \href{https://doi.org/10.1007/978-3-319-66158-2\_25}{MossigeGSMC17}~\cite{MossigeGSMC17} & \href{works/MossigeGSMC17.pdf}{Time-Aware Test Case Execution Scheduling for Cyber-Physical Systems} &  & industrial partner, real-world, benchmark, random instance, CSPlib, generated instance & 4 &  &  &  &  &  &  & \ref{a:MossigeGSMC17} & \ref{b:MossigeGSMC17}\\
\rowlabel{c:Pralet17}Pralet17 \href{https://doi.org/10.1007/978-3-319-66158-2\_16}{Pralet17}~\cite{Pralet17} & \href{works/Pralet17.pdf}{An Incomplete Constraint-Based System for Scheduling with Renewable Resources} &  & benchmark & 1 &  &  &  &  &  &  & \ref{a:Pralet17} & \ref{b:Pralet17}\\
\rowlabel{c:TranVNB17a}TranVNB17a \href{https://doi.org/10.24963/ijcai.2017/726}{TranVNB17a}~\cite{TranVNB17a} & \href{works/TranVNB17a.pdf}{Robots in Retirement Homes: Applying Off-the-Shelf Planning and Scheduling to a Team of Assistive Robots (Extended Abstract)} &  & real-world & 0 &  &  &  &  &  &  & \ref{a:TranVNB17a} & \ref{b:TranVNB17a}\\
\rowlabel{c:YoungFS17}YoungFS17 \href{https://doi.org/10.1007/978-3-319-66158-2\_20}{YoungFS17}~\cite{YoungFS17} & \href{works/YoungFS17.pdf}{Constraint Programming Applied to the Multi-Skill Project Scheduling Problem} &  & benchmark, github, instance generator & 6 &  &  &  &  &  &  & \ref{a:YoungFS17} & \ref{b:YoungFS17}\\
\rowlabel{c:BonfiettiZLM16}BonfiettiZLM16 \href{https://doi.org/10.1007/978-3-319-44953-1\_8}{BonfiettiZLM16}~\cite{BonfiettiZLM16} & \href{works/BonfiettiZLM16.pdf}{The Multirate Resource Constraint} &  & generated instance, github, industrial instance, benchmark, real-world & 1 &  &  &  &  &  &  & \ref{a:BonfiettiZLM16} & \ref{b:BonfiettiZLM16}\\
\rowlabel{c:BoothNB16}BoothNB16 \href{https://doi.org/10.1007/978-3-319-44953-1\_34}{BoothNB16}~\cite{BoothNB16} & \href{works/BoothNB16.pdf}{A Constraint Programming Approach to Multi-Robot Task Allocation and Scheduling in Retirement Homes} &  & real-world & 0 &  &  &  &  &  &  & \ref{a:BoothNB16} & \ref{b:BoothNB16}\\
\rowlabel{c:BridiLBBM16}BridiLBBM16 \href{https://doi.org/10.3233/978-1-61499-672-9-1598}{BridiLBBM16}~\cite{BridiLBBM16} & \href{works/BridiLBBM16.pdf}{{DARDIS:} Distributed And Randomized DIspatching and Scheduling} &  &  & 0 &  &  &  &  &  &  & \ref{a:BridiLBBM16} & \ref{b:BridiLBBM16}\\
\rowlabel{c:CauwelaertDMS16}CauwelaertDMS16 \href{https://doi.org/10.1007/978-3-319-44953-1\_33}{CauwelaertDMS16}~\cite{CauwelaertDMS16} & \href{works/CauwelaertDMS16.pdf}{Efficient Filtering for the Unary Resource with Family-Based Transition Times} &  & real-life, bitbucket, benchmark & 2 &  &  &  &  &  &  & \ref{a:CauwelaertDMS16} & \ref{b:CauwelaertDMS16}\\
\rowlabel{c:FontaineMH16}FontaineMH16 \href{https://doi.org/10.1007/978-3-319-33954-2\_12}{FontaineMH16}~\cite{FontaineMH16} & \href{works/FontaineMH16.pdf}{Parallel Composition of Scheduling Solvers} &  & benchmark & 2 &  &  &  &  &  &  & \ref{a:FontaineMH16} & \ref{b:FontaineMH16}\\
\rowlabel{c:GilesH16}GilesH16 \href{https://doi.org/10.1007/978-3-319-44953-1\_38}{GilesH16}~\cite{GilesH16} & \href{works/GilesH16.pdf}{Solving a Supply-Delivery Scheduling Problem with Constraint Programming} &  &  & 0 &  &  &  &  &  &  & \ref{a:GilesH16} & \ref{b:GilesH16}\\
\rowlabel{c:GingrasQ16}GingrasQ16 \href{http://www.ijcai.org/Abstract/16/440}{GingrasQ16}~\cite{GingrasQ16} & \href{works/GingrasQ16.pdf}{Generalizing the Edge-Finder Rule for the Cumulative Constraint} &  & benchmark & 0 &  &  &  &  &  &  & \ref{a:GingrasQ16} & \ref{b:GingrasQ16}\\
\rowlabel{c:HechingH16}HechingH16 \href{https://doi.org/10.1007/978-3-319-33954-2\_14}{HechingH16}~\cite{HechingH16} & \href{works/HechingH16.pdf}{Scheduling Home Hospice Care with Logic-Based Benders Decomposition} &  & real-world & 0 &  &  &  &  &  &  & \ref{a:HechingH16} & \ref{b:HechingH16}\\
\rowlabel{c:JelinekB16}JelinekB16 \href{https://doi.org/10.1007/978-3-319-28228-2\_1}{JelinekB16}~\cite{JelinekB16} & \href{works/JelinekB16.pdf}{Using Constraint Logic Programming to Schedule Solar Array Operations on the International Space Station} &  & real-life & 2 &  &  &  &  &  &  & \ref{a:JelinekB16} & \ref{b:JelinekB16}\\
\rowlabel{c:LimHTB16}LimHTB16 \href{https://doi.org/10.1007/978-3-319-44953-1\_43}{LimHTB16}~\cite{LimHTB16} & \href{works/LimHTB16.pdf}{Online HVAC-Aware Occupancy Scheduling with Adaptive Temperature Control} &  & real-world & 4 &  &  &  &  &  &  & \ref{a:LimHTB16} & \ref{b:LimHTB16}\\
\rowlabel{c:LuoVLBM16}LuoVLBM16 \href{http://www.aaai.org/ocs/index.php/KR/KR16/paper/view/12909}{LuoVLBM16}~\cite{LuoVLBM16} & \href{works/LuoVLBM16.pdf}{Using Metric Temporal Logic to Specify Scheduling Problems} &  &  & 0 &  &  &  &  &  &  & \ref{a:LuoVLBM16} & \ref{b:LuoVLBM16}\\
\rowlabel{c:Madi-WambaB16}Madi-WambaB16 \href{https://doi.org/10.1007/978-3-319-33954-2\_18}{Madi-WambaB16}~\cite{Madi-WambaB16} & \href{works/Madi-WambaB16.pdf}{The TaskIntersection Constraint} &  & real-world, benchmark, random instance, generated instance & 3 &  &  &  &  &  &  & \ref{a:Madi-WambaB16} & \ref{b:Madi-WambaB16}\\
\rowlabel{c:SchuttS16}SchuttS16 \href{https://doi.org/10.1007/978-3-319-44953-1\_28}{SchuttS16}~\cite{SchuttS16} & \href{works/SchuttS16.pdf}{Explaining Producer/Consumer Constraints} &  & benchmark & 1 &  &  &  &  &  &  & \ref{a:SchuttS16} & \ref{b:SchuttS16}\\
\rowlabel{c:SzerediS16}SzerediS16 \href{https://doi.org/10.1007/978-3-319-44953-1\_31}{SzerediS16}~\cite{SzerediS16} & \href{works/SzerediS16.pdf}{Modelling and Solving Multi-mode Resource-Constrained Project Scheduling} &  & benchmark & 2 &  &  &  &  &  &  & \ref{a:SzerediS16} & \ref{b:SzerediS16}\\
\rowlabel{c:Tesch16}Tesch16 \href{https://doi.org/10.1007/978-3-319-44953-1\_32}{Tesch16}~\cite{Tesch16} & \href{works/Tesch16.pdf}{A Nearly Exact Propagation Algorithm for Energetic Reasoning in {\textbackslash}mathcal O(n{\^{}}2 {\textbackslash}log n)} &  & Roadef & 1 &  &  &  &  &  &  & \ref{a:Tesch16} & \ref{b:Tesch16}\\
\rowlabel{c:TranDRFWOVB16}TranDRFWOVB16 \href{https://doi.org/10.1609/socs.v7i1.18390}{TranDRFWOVB16}~\cite{TranDRFWOVB16} & \href{works/TranDRFWOVB16.pdf}{A Hybrid Quantum-Classical Approach to Solving Scheduling Problems} &  &  & 0 &  &  &  &  &  &  & \ref{a:TranDRFWOVB16} & \ref{b:TranDRFWOVB16}\\
\rowlabel{c:TranWDRFOVB16}TranWDRFOVB16 \href{http://www.aaai.org/ocs/index.php/WS/AAAIW16/paper/view/12664}{TranWDRFOVB16}~\cite{TranWDRFOVB16} & \href{works/TranWDRFOVB16.pdf}{Explorations of Quantum-Classical Approaches to Scheduling a Mars Lander Activity Problem} &  & benchmark & 0 &  &  &  &  &  &  & \ref{a:TranWDRFOVB16} & \ref{b:TranWDRFOVB16}\\
\rowlabel{c:BartakV15}BartakV15 \href{}{BartakV15}~\cite{BartakV15} & \href{works/BartakV15.pdf}{Reactive Recovery from Machine Breakdown in Production Scheduling with Temporal Distance and Resource Constraints} &  & real-world, real-life & 0 &  &  &  &  &  &  & \ref{a:BartakV15} & \ref{b:BartakV15}\\
\rowlabel{c:BofillGSV15}BofillGSV15 \href{https://doi.org/10.1007/978-3-319-18008-3\_5}{BofillGSV15}~\cite{BofillGSV15} & \href{works/BofillGSV15.pdf}{MaxSAT-Based Scheduling of {B2B} Meetings} &  & industrial instance & 3 &  &  &  &  &  &  & \ref{a:BofillGSV15} & \ref{b:BofillGSV15}\\
\rowlabel{c:BurtLPS15}BurtLPS15 \href{https://doi.org/10.1007/978-3-319-18008-3\_7}{BurtLPS15}~\cite{BurtLPS15} & \href{works/BurtLPS15.pdf}{Scheduling with Fixed Maintenance, Shared Resources and Nonlinear Feedrate Constraints: {A} Mine Planning Case Study} &  & real-world, benchmark, industry partner & 5 &  &  &  &  &  &  & \ref{a:BurtLPS15} & \ref{b:BurtLPS15}\\
\rowlabel{c:DejemeppeCS15}DejemeppeCS15 \href{https://doi.org/10.1007/978-3-319-23219-5\_7}{DejemeppeCS15}~\cite{DejemeppeCS15} & \href{works/DejemeppeCS15.pdf}{The Unary Resource with Transition Times} &  & real-world, bitbucket, generated instance, benchmark & 4 &  &  &  &  &  &  & \ref{a:DejemeppeCS15} & \ref{b:DejemeppeCS15}\\
\rowlabel{c:EvenSH15}EvenSH15 \href{https://doi.org/10.1007/978-3-319-23219-5\_40}{EvenSH15}~\cite{EvenSH15} & \href{works/EvenSH15.pdf}{A Constraint Programming Approach for Non-preemptive Evacuation Scheduling} &  & real-life, real-world & 0 &  &  &  &  &  &  & \ref{a:EvenSH15} & \ref{b:EvenSH15}\\
\rowlabel{c:GayHLS15}GayHLS15 \href{https://doi.org/10.1007/978-3-319-23219-5\_10}{GayHLS15}~\cite{GayHLS15} & \href{works/GayHLS15.pdf}{Conflict Ordering Search for Scheduling Problems} &  & benchmark, bitbucket & 0 &  &  &  &  &  &  & \ref{a:GayHLS15} & \ref{b:GayHLS15}\\
\rowlabel{c:GayHS15}GayHS15 \href{https://doi.org/10.1007/978-3-319-23219-5\_11}{GayHS15}~\cite{GayHS15} & \href{works/GayHS15.pdf}{Simple and Scalable Time-Table Filtering for the Cumulative Constraint} &  & bitbucket & 2 &  &  &  &  &  &  & \ref{a:GayHS15} & \ref{b:GayHS15}\\
\rowlabel{c:GayHS15a}GayHS15a \href{https://doi.org/10.1007/978-3-319-18008-3\_11}{GayHS15a}~\cite{GayHS15a} & \href{works/GayHS15a.pdf}{Time-Table Disjunctive Reasoning for the Cumulative Constraint} &  & benchmark, bitbucket, real-world & 0 &  &  &  &  &  &  & \ref{a:GayHS15a} & \ref{b:GayHS15a}\\
\rowlabel{c:KreterSS15}KreterSS15 \href{https://doi.org/10.1007/978-3-319-23219-5\_19}{KreterSS15}~\cite{KreterSS15} & \href{works/KreterSS15.pdf}{Modeling and Solving Project Scheduling with Calendars} &  & benchmark & 3 &  &  &  &  &  &  & \ref{a:KreterSS15} & \ref{b:KreterSS15}\\
\rowlabel{c:LimBTBB15}LimBTBB15 \href{https://doi.org/10.1007/978-3-319-18008-3\_17}{LimBTBB15}~\cite{LimBTBB15} & \href{works/LimBTBB15.pdf}{Large Neighborhood Search for Energy Aware Meeting Scheduling in Smart Buildings} &  & benchmark & 3 &  &  &  &  &  &  & \ref{a:LimBTBB15} & \ref{b:LimBTBB15}\\
\rowlabel{c:LombardiBM15}LombardiBM15 \href{https://doi.org/10.1007/978-3-319-23219-5\_20}{LombardiBM15}~\cite{LombardiBM15} & \href{works/LombardiBM15.pdf}{Deterministic Estimation of the Expected Makespan of a {POS} Under Duration Uncertainty} &  & benchmark, real-world & 0 &  &  &  &  &  &  & \ref{a:LombardiBM15} & \ref{b:LombardiBM15}\\
\rowlabel{c:MelgarejoLS15}MelgarejoLS15 \href{https://doi.org/10.1007/978-3-319-18008-3\_1}{MelgarejoLS15}~\cite{MelgarejoLS15} & \href{works/MelgarejoLS15.pdf}{A Time-Dependent No-Overlap Constraint: Application to Urban Delivery Problems} &  & real-world, benchmark & 1 &  &  &  &  &  &  & \ref{a:MelgarejoLS15} & \ref{b:MelgarejoLS15}\\
\rowlabel{c:MurphyMB15}MurphyMB15 \href{https://doi.org/10.1007/978-3-319-23219-5\_47}{MurphyMB15}~\cite{MurphyMB15} & \href{works/MurphyMB15.pdf}{Design and Evaluation of a Constraint-Based Energy Saving and Scheduling Recommender System} &  & real-world & 3 &  &  &  &  &  &  & \ref{a:MurphyMB15} & \ref{b:MurphyMB15}\\
\rowlabel{c:PesantRR15}PesantRR15 \href{https://doi.org/10.1007/978-3-319-18008-3\_21}{PesantRR15}~\cite{PesantRR15} & \href{works/PesantRR15.pdf}{A Comparative Study of {MIP} and {CP} Formulations for the {B2B} Scheduling Optimization Problem} &  &  & 1 &  &  &  &  &  &  & \ref{a:PesantRR15} & \ref{b:PesantRR15}\\
\rowlabel{c:PraletLJ15}PraletLJ15 \href{https://doi.org/10.1007/978-3-319-23219-5\_48}{PraletLJ15}~\cite{PraletLJ15} & \href{works/PraletLJ15.pdf}{Scheduling Running Modes of Satellite Instruments Using Constraint-Based Local Search} &  &  & 0 &  &  &  &  &  &  & \ref{a:PraletLJ15} & \ref{b:PraletLJ15}\\
\rowlabel{c:SialaAH15}SialaAH15 \href{https://doi.org/10.1007/978-3-319-23219-5\_28}{SialaAH15}~\cite{SialaAH15} & \href{works/SialaAH15.pdf}{Two Clause Learning Approaches for Disjunctive Scheduling} &  & github, benchmark & 5 &  &  &  &  &  &  & \ref{a:SialaAH15} & \ref{b:SialaAH15}\\
\rowlabel{c:VilimLS15}VilimLS15 \href{https://doi.org/10.1007/978-3-319-18008-3\_30}{VilimLS15}~\cite{VilimLS15} & \href{works/VilimLS15.pdf}{Failure-Directed Search for Constraint-Based Scheduling} &  & benchmark & 8 &  &  &  &  &  &  & \ref{a:VilimLS15} & \ref{b:VilimLS15}\\
\rowlabel{c:ZhouGL15}ZhouGL15 \href{https://doi.org/10.1109/FSKD.2015.7382064}{ZhouGL15}~\cite{ZhouGL15} & \href{works/ZhouGL15.pdf}{On complex hybrid flexible flowshop scheduling problems based on constraint programming} &  & real-world & 0 &  &  &  &  &  &  & \ref{a:ZhouGL15} & \ref{b:ZhouGL15}\\
\rowlabel{c:AlesioNBG14}AlesioNBG14 \href{https://doi.org/10.1007/978-3-319-10428-7\_58}{AlesioNBG14}~\cite{AlesioNBG14} & \href{works/AlesioNBG14.pdf}{Worst-Case Scheduling of Software Tasks - {A} Constraint Optimization Model to Support Performance Testing} &  & benchmark & 2 &  &  &  &  &  &  & \ref{a:AlesioNBG14} & \ref{b:AlesioNBG14}\\
\rowlabel{c:BartoliniBBLM14}BartoliniBBLM14 \href{https://doi.org/10.1007/978-3-319-10428-7\_55}{BartoliniBBLM14}~\cite{BartoliniBBLM14} & \href{works/BartoliniBBLM14.pdf}{Proactive Workload Dispatching on the {EURORA} Supercomputer} &  &  & 4 &  &  &  &  &  &  & \ref{a:BartoliniBBLM14} & \ref{b:BartoliniBBLM14}\\
\rowlabel{c:BessiereHMQW14}BessiereHMQW14 \href{https://doi.org/10.1007/978-3-319-07046-9\_23}{BessiereHMQW14}~\cite{BessiereHMQW14} & \href{works/BessiereHMQW14.pdf}{Buffered Resource Constraint: Algorithms and Complexity} &  & benchmark, real-life & 0 &  &  &  &  &  &  & \ref{a:BessiereHMQW14} & \ref{b:BessiereHMQW14}\\
\rowlabel{c:BofillEGPSV14}BofillEGPSV14 \href{https://doi.org/10.1007/978-3-319-10428-7\_56}{BofillEGPSV14}~\cite{BofillEGPSV14} & \href{works/BofillEGPSV14.pdf}{Scheduling {B2B} Meetings} &  & industrial instance & 6 &  &  &  &  &  &  & \ref{a:BofillEGPSV14} & \ref{b:BofillEGPSV14}\\
\rowlabel{c:BonfiettiLM14}BonfiettiLM14 \href{https://doi.org/10.1007/978-3-319-07046-9\_15}{BonfiettiLM14}~\cite{BonfiettiLM14} & \href{works/BonfiettiLM14.pdf}{Disregarding Duration Uncertainty in Partial Order Schedules? Yes, We Can!} &  & real-world, benchmark & 2 &  &  &  &  &  &  & \ref{a:BonfiettiLM14} & \ref{b:BonfiettiLM14}\\
\rowlabel{c:DejemeppeD14}DejemeppeD14 \href{https://doi.org/10.1007/978-3-319-07046-9\_20}{DejemeppeD14}~\cite{DejemeppeD14} & \href{works/DejemeppeD14.pdf}{Continuously Degrading Resource and Interval Dependent Activity Durations in Nuclear Medicine Patient Scheduling} &  & bitbucket & 0 &  &  &  &  &  &  & \ref{a:DejemeppeD14} & \ref{b:DejemeppeD14}\\
\rowlabel{c:DerrienP14}DerrienP14 \href{https://doi.org/10.1007/978-3-319-10428-7\_22}{DerrienP14}~\cite{DerrienP14} & \href{works/DerrienP14.pdf}{A New Characterization of Relevant Intervals for Energetic Reasoning} &  & random instance & 0 &  &  &  &  &  &  & \ref{a:DerrienP14} & \ref{b:DerrienP14}\\
\rowlabel{c:DerrienPZ14}DerrienPZ14 \href{https://doi.org/10.1007/978-3-319-10428-7\_23}{DerrienPZ14}~\cite{DerrienPZ14} & \href{works/DerrienPZ14.pdf}{A Declarative Paradigm for Robust Cumulative Scheduling} &  & benchmark, random instance, real-world & 0 &  &  &  &  &  &  & \ref{a:DerrienPZ14} & \ref{b:DerrienPZ14}\\
\rowlabel{c:DoulabiRP14}DoulabiRP14 \href{https://doi.org/10.1007/978-3-319-07046-9\_32}{DoulabiRP14}~\cite{DoulabiRP14} & \href{works/DoulabiRP14.pdf}{A Constraint Programming-Based Column Generation Approach for Operating Room Planning and Scheduling} &  &  & 0 &  &  &  &  &  &  & \ref{a:DoulabiRP14} & \ref{b:DoulabiRP14}\\
\rowlabel{c:FriedrichFMRSST14}FriedrichFMRSST14 \href{https://doi.org/10.1007/978-3-319-28697-6\_23}{FriedrichFMRSST14}~\cite{FriedrichFMRSST14} & \href{}{Representing Production Scheduling with Constraint Answer Set Programming} &  &  & 0 &  &  &  &  &  &  & \ref{a:FriedrichFMRSST14} & No\\
\rowlabel{c:GaySS14}GaySS14 \href{https://doi.org/10.1007/978-3-319-10428-7\_59}{GaySS14}~\cite{GaySS14} & \href{works/GaySS14.pdf}{Continuous Casting Scheduling with Constraint Programming} &  & real-life, CSPlib & 0 &  &  &  &  &  &  & \ref{a:GaySS14} & \ref{b:GaySS14}\\
\rowlabel{c:HoundjiSWD14}HoundjiSWD14 \href{https://doi.org/10.1007/978-3-319-10428-7\_29}{HoundjiSWD14}~\cite{HoundjiSWD14} & \href{works/HoundjiSWD14.pdf}{The StockingCost Constraint} &  & bitbucket, generated instance & 0 &  &  &  &  &  &  & \ref{a:HoundjiSWD14} & \ref{b:HoundjiSWD14}\\
\rowlabel{c:KoschB14}KoschB14 \href{https://doi.org/10.1007/978-3-319-07046-9\_5}{KoschB14}~\cite{KoschB14} & \href{works/KoschB14.pdf}{A New {MIP} Model for Parallel-Batch Scheduling with Non-identical Job Sizes} &  & benchmark & 0 &  &  &  &  &  &  & \ref{a:KoschB14} & \ref{b:KoschB14}\\
\rowlabel{c:LipovetzkyBPS14}LipovetzkyBPS14 \href{http://www.aaai.org/ocs/index.php/ICAPS/ICAPS14/paper/view/7942}{LipovetzkyBPS14}~\cite{LipovetzkyBPS14} & \href{works/LipovetzkyBPS14.pdf}{Planning for Mining Operations with Time and Resource Constraints} &  & industrial partner, real-life, industry partner, real-world, benchmark, generated instance & 0 &  &  &  &  &  &  & \ref{a:LipovetzkyBPS14} & \ref{b:LipovetzkyBPS14}\\
\rowlabel{c:LouieVNB14}LouieVNB14 \href{https://doi.org/10.1109/ICRA.2014.6907637}{LouieVNB14}~\cite{LouieVNB14} & \href{}{An autonomous assistive robot for planning, scheduling and facilitating multi-user activities} &  &  & 0 &  &  &  &  &  &  & \ref{a:LouieVNB14} & No\\
\rowlabel{c:BonfiettiLM13}BonfiettiLM13 \href{http://www.aaai.org/ocs/index.php/ICAPS/ICAPS13/paper/view/6050}{BonfiettiLM13}~\cite{BonfiettiLM13} & \href{works/BonfiettiLM13.pdf}{De-Cycling Cyclic Scheduling Problems} &  &  & 0 &  &  &  &  &  &  & \ref{a:BonfiettiLM13} & \ref{b:BonfiettiLM13}\\
\rowlabel{c:ChuGNSW13}ChuGNSW13 \href{http://www.aaai.org/ocs/index.php/IJCAI/IJCAI13/paper/view/6878}{ChuGNSW13}~\cite{ChuGNSW13} & \href{works/ChuGNSW13.pdf}{On the Complexity of Global Scheduling Constraints under Structural Restrictions} &  &  & 0 &  &  &  &  &  &  & \ref{a:ChuGNSW13} & \ref{b:ChuGNSW13}\\
\rowlabel{c:CireCH13}CireCH13 \href{https://doi.org/10.1007/978-3-642-38171-3\_22}{CireCH13}~\cite{CireCH13} & \href{works/CireCH13.pdf}{Mixed Integer Programming vs. Logic-Based Benders Decomposition for Planning and Scheduling} & \su{{CP Opt} Cplex} &  & 1 & dead &  & n & - &  &  & \ref{a:CireCH13} & \ref{b:CireCH13}\\
\rowlabel{c:GuSS13}GuSS13 \href{https://doi.org/10.1007/978-3-642-38171-3\_24}{GuSS13}~\cite{GuSS13} & \href{works/GuSS13.pdf}{A Lagrangian Relaxation Based Forward-Backward Improvement Heuristic for Maximising the Net Present Value of Resource-Constrained Projects} & Chuffed & benchmark & 1 & dead &  &  & - & RCPSPDC & \su{cumulative maxNVPProp} & \ref{a:GuSS13} & \ref{b:GuSS13}\\
\rowlabel{c:HeinzKB13}HeinzKB13 \href{https://doi.org/10.1007/978-3-642-38171-3\_2}{HeinzKB13}~\cite{HeinzKB13} & \href{works/HeinzKB13.pdf}{Recent Improvements Using Constraint Integer Programming for Resource Allocation and Scheduling} &  &  & 0 &  &  &  &  &  &  & \ref{a:HeinzKB13} & \ref{b:HeinzKB13}\\
\rowlabel{c:KelarevaTK13}KelarevaTK13 \href{https://doi.org/10.1007/978-3-642-38171-3\_8}{KelarevaTK13}~\cite{KelarevaTK13} & \href{works/KelarevaTK13.pdf}{{CP} Methods for Scheduling and Routing with Time-Dependent Task Costs} & \su{MiniZinc CPX G12FD} & real-world & 5 & ref &  & - & - & \su{LSFRP BPCTOP} & \su{alldifferent alldifferentExcept0} & \ref{a:KelarevaTK13} & \ref{b:KelarevaTK13}\\
\rowlabel{c:LetortCB13}LetortCB13 \href{https://doi.org/10.1007/978-3-642-38171-3\_10}{LetortCB13}~\cite{LetortCB13} & \href{works/LetortCB13.pdf}{A Synchronized Sweep Algorithm for the \emph{k-dimensional cumulative} Constraint} & \su{SICStus Choco} & Roadef, benchmark, random instance & 2 & PSPlib &  & - & - & RCPSP & \su{cumulative kDimensionalCumulative} & \ref{a:LetortCB13} & \ref{b:LetortCB13}\\
\rowlabel{c:LombardiM13}LombardiM13 \href{http://www.aaai.org/ocs/index.php/ICAPS/ICAPS13/paper/view/6052}{LombardiM13}~\cite{LombardiM13} & \href{works/LombardiM13.pdf}{A Min-Flow Algorithm for Minimal Critical Set Detection in Resource Constrained Project Scheduling} &  &  & 0 &  &  &  &  &  &  & \ref{a:LombardiM13} & \ref{b:LombardiM13}\\
\rowlabel{c:OuelletQ13}OuelletQ13 \href{https://doi.org/10.1007/978-3-642-40627-0\_42}{OuelletQ13}~\cite{OuelletQ13} & \href{works/OuelletQ13.pdf}{Time-Table Extended-Edge-Finding for the Cumulative Constraint} &  & benchmark & 1 &  &  &  &  &  &  & \ref{a:OuelletQ13} & \ref{b:OuelletQ13}\\
\rowlabel{c:SchuttFS13}SchuttFS13 \href{https://doi.org/10.1007/978-3-642-40627-0\_47}{SchuttFS13}~\cite{SchuttFS13} & \href{works/SchuttFS13.pdf}{Scheduling Optional Tasks with Explanation} &  & benchmark & 1 &  &  &  &  &  &  & \ref{a:SchuttFS13} & \ref{b:SchuttFS13}\\
\rowlabel{c:SchuttFS13a}SchuttFS13a \href{https://doi.org/10.1007/978-3-642-38171-3\_16}{SchuttFS13a}~\cite{SchuttFS13a} & \href{works/SchuttFS13a.pdf}{Explaining Time-Table-Edge-Finding Propagation for the Cumulative Resource Constraint} & \su{Mercury G12} & benchmark & 5 & \su{PSPlib AT BL Pack KSD15D PackD} &  & - & - & RCPSP & cumulative & \ref{a:SchuttFS13a} & \ref{b:SchuttFS13a}\\
\rowlabel{c:TranTDB13}TranTDB13 \href{http://www.aaai.org/ocs/index.php/ICAPS/ICAPS13/paper/view/6005}{TranTDB13}~\cite{TranTDB13} & \href{works/TranTDB13.pdf}{Hybrid Queueing Theory and Scheduling Models for Dynamic Environments with Sequence-Dependent Setup Times} &  & real-world & 0 &  &  &  &  &  &  & \ref{a:TranTDB13} & \ref{b:TranTDB13}\\
\rowlabel{c:BillautHL12}BillautHL12 \href{https://doi.org/10.1007/978-3-642-29828-8\_5}{BillautHL12}~\cite{BillautHL12} & \href{works/BillautHL12.pdf}{Complete Characterization of Near-Optimal Sequences for the Two-Machine Flow Shop Scheduling Problem} &  & random instance & 0 &  &  &  &  &  &  & \ref{a:BillautHL12} & \ref{b:BillautHL12}\\
\rowlabel{c:BonfiettiLBM12}BonfiettiLBM12 \href{https://doi.org/10.1007/978-3-642-29828-8\_6}{BonfiettiLBM12}~\cite{BonfiettiLBM12} & \href{works/BonfiettiLBM12.pdf}{Global Cyclic Cumulative Constraint} &  & benchmark & 3 &  &  &  &  &  &  & \ref{a:BonfiettiLBM12} & \ref{b:BonfiettiLBM12}\\
\rowlabel{c:BonfiettiM12}BonfiettiM12 \href{https://ceur-ws.org/Vol-926/paper2.pdf}{BonfiettiM12}~\cite{BonfiettiM12} & \href{works/BonfiettiM12.pdf}{A Constraint-based Approach to Cyclic Resource-Constrained Scheduling Problem} &  & industrial instance & 0 &  &  &  &  &  &  & \ref{a:BonfiettiM12} & \ref{b:BonfiettiM12}\\
\rowlabel{c:GuSW12}GuSW12 \href{https://doi.org/10.1007/978-3-642-33558-7\_55}{GuSW12}~\cite{GuSW12} & \href{works/GuSW12.pdf}{Maximising the Net Present Value of Large Resource-Constrained Projects} &  & benchmark & 2 &  &  &  &  &  &  & \ref{a:GuSW12} & \ref{b:GuSW12}\\
\rowlabel{c:HeinzB12}HeinzB12 \href{https://doi.org/10.1007/978-3-642-29828-8\_14}{HeinzB12}~\cite{HeinzB12} & \href{works/HeinzB12.pdf}{Reconsidering Mixed Integer Programming and MIP-Based Hybrids for Scheduling} &  &  & 0 &  &  &  &  &  &  & \ref{a:HeinzB12} & \ref{b:HeinzB12}\\
\rowlabel{c:IfrimOS12}IfrimOS12 \href{https://doi.org/10.1007/978-3-642-33558-7\_68}{IfrimOS12}~\cite{IfrimOS12} & \href{works/IfrimOS12.pdf}{Properties of Energy-Price Forecasts for Scheduling} &  & real-life & 1 &  &  &  &  &  &  & \ref{a:IfrimOS12} & \ref{b:IfrimOS12}\\
\rowlabel{c:LetortBC12}LetortBC12 \href{https://doi.org/10.1007/978-3-642-33558-7\_33}{LetortBC12}~\cite{LetortBC12} & \href{works/LetortBC12.pdf}{A Scalable Sweep Algorithm for the cumulative Constraint} &  & Roadef, benchmark, random instance & 2 &  &  &  &  &  &  & \ref{a:LetortBC12} & \ref{b:LetortBC12}\\
\rowlabel{c:RendlPHPR12}RendlPHPR12 \href{https://doi.org/10.1007/978-3-642-29828-8\_22}{RendlPHPR12}~\cite{RendlPHPR12} & \href{works/RendlPHPR12.pdf}{Hybrid Heuristics for Multimodal Homecare Scheduling} &  & real-world, CSPlib, benchmark & 2 &  &  &  &  &  &  & \ref{a:RendlPHPR12} & \ref{b:RendlPHPR12}\\
\rowlabel{c:SchuttCSW12}SchuttCSW12 \href{https://doi.org/10.1007/978-3-642-29828-8\_24}{SchuttCSW12}~\cite{SchuttCSW12} & \href{works/SchuttCSW12.pdf}{Maximising the Net Present Value for Resource-Constrained Project Scheduling} &  & benchmark & 1 &  &  &  &  &  &  & \ref{a:SchuttCSW12} & \ref{b:SchuttCSW12}\\
\rowlabel{c:SerraNM12}SerraNM12 \href{https://doi.org/10.1007/978-3-642-33558-7\_59}{SerraNM12}~\cite{SerraNM12} & \href{works/SerraNM12.pdf}{The Offshore Resources Scheduling Problem: Detailing a Constraint Programming Approach} &  & benchmark, real-world & 4 &  &  &  &  &  &  & \ref{a:SerraNM12} & \ref{b:SerraNM12}\\
\rowlabel{c:SimoninAHL12}SimoninAHL12 \href{https://doi.org/10.1007/978-3-642-33558-7\_5}{SimoninAHL12}~\cite{SimoninAHL12} & \href{works/SimoninAHL12.pdf}{Scheduling Scientific Experiments on the Rosetta/Philae Mission} & \su{MOST {Ilog Scheduler}} &  & 0 & n &  & n & - &  & \su{cumulative dataTransfer} & \ref{a:SimoninAHL12} & \ref{b:SimoninAHL12}\\
\rowlabel{c:TranB12}TranB12 \href{https://doi.org/10.3233/978-1-61499-098-7-774}{TranB12}~\cite{TranB12} & \href{works/TranB12.pdf}{Logic-based Benders Decomposition for Alternative Resource Scheduling with Sequence Dependent Setups} &  & benchmark & 0 &  &  &  &  &  &  & \ref{a:TranB12} & \ref{b:TranB12}\\
\rowlabel{c:ZhangLS12}ZhangLS12 \href{https://doi.org/10.1109/CIT.2012.96}{ZhangLS12}~\cite{ZhangLS12} & \href{works/ZhangLS12.pdf}{Model and Solution for Hot Strip Rolling Scheduling Problem Based on Constraint Programming Method} &  &  & 0 &  &  &  &  &  &  & \ref{a:ZhangLS12} & \ref{b:ZhangLS12}\\
\rowlabel{c:BajestaniB11}BajestaniB11 \href{http://aaai.org/ocs/index.php/ICAPS/ICAPS11/paper/view/2680}{BajestaniB11}~\cite{BajestaniB11} & \href{works/BajestaniB11.pdf}{Scheduling an Aircraft Repair Shop} &  &  & 0 &  &  &  &  &  &  & \ref{a:BajestaniB11} & \ref{b:BajestaniB11}\\
\rowlabel{c:BonfiettiLBM11}BonfiettiLBM11 \href{https://doi.org/10.1007/978-3-642-23786-7\_12}{BonfiettiLBM11}~\cite{BonfiettiLBM11} & \href{works/BonfiettiLBM11.pdf}{A Constraint Based Approach to Cyclic {RCPSP}} &  & generated instance, industrial instance, benchmark & 3 &  &  &  &  &  &  & \ref{a:BonfiettiLBM11} & \ref{b:BonfiettiLBM11}\\
\rowlabel{c:ChapadosJR11}ChapadosJR11 \href{https://doi.org/10.1007/978-3-642-21311-3\_7}{ChapadosJR11}~\cite{ChapadosJR11} & \href{works/ChapadosJR11.pdf}{Retail Store Workforce Scheduling by Expected Operating Income Maximization} &  &  & 0 &  &  &  &  &  &  & \ref{a:ChapadosJR11} & \ref{b:ChapadosJR11}\\
\rowlabel{c:ClercqPBJ11}ClercqPBJ11 \href{https://doi.org/10.1007/978-3-642-23786-7\_20}{ClercqPBJ11}~\cite{ClercqPBJ11} & \href{works/ClercqPBJ11.pdf}{Filtering Algorithms for Discrete Cumulative Problems with Overloads of Resource} &  & benchmark & 1 &  &  &  &  &  &  & \ref{a:ClercqPBJ11} & \ref{b:ClercqPBJ11}\\
\rowlabel{c:EdisO11}EdisO11 \href{https://doi.org/10.1007/978-3-642-21311-3\_10}{EdisO11}~\cite{EdisO11} & \href{works/EdisO11.pdf}{Parallel Machine Scheduling with Additional Resources: {A} Lagrangian-Based Constraint Programming Approach} &  &  & 0 &  &  &  &  &  &  & \ref{a:EdisO11} & \ref{b:EdisO11}\\
\rowlabel{c:GrimesH11}GrimesH11 \href{https://doi.org/10.1007/978-3-642-23786-7\_28}{GrimesH11}~\cite{GrimesH11} & \href{works/GrimesH11.pdf}{Models and Strategies for Variants of the Job Shop Scheduling Problem} &  & benchmark & 1 &  &  &  &  &  &  & \ref{a:GrimesH11} & \ref{b:GrimesH11}\\
\rowlabel{c:HeinzS11}HeinzS11 \href{https://doi.org/10.1007/978-3-642-20662-7\_34}{HeinzS11}~\cite{HeinzS11} & \href{works/HeinzS11.pdf}{Explanations for the Cumulative Constraint: An Experimental Study} &  & benchmark & 1 &  &  &  &  &  &  & \ref{a:HeinzS11} & \ref{b:HeinzS11}\\
\rowlabel{c:HermenierDL11}HermenierDL11 \href{https://doi.org/10.1007/978-3-642-23786-7\_5}{HermenierDL11}~\cite{HermenierDL11} & \href{works/HermenierDL11.pdf}{Bin Repacking Scheduling in Virtualized Datacenters} &  &  & 1 &  &  &  &  &  &  & \ref{a:HermenierDL11} & \ref{b:HermenierDL11}\\
\rowlabel{c:KameugneFSN11}KameugneFSN11 \href{https://doi.org/10.1007/978-3-642-23786-7\_37}{KameugneFSN11}~\cite{KameugneFSN11} & \href{works/KameugneFSN11.pdf}{A Quadratic Edge-Finding Filtering Algorithm for Cumulative Resource Constraints} &  & benchmark & 1 &  &  &  &  &  &  & \ref{a:KameugneFSN11} & \ref{b:KameugneFSN11}\\
\rowlabel{c:LahimerLH11}LahimerLH11 \href{https://doi.org/10.1007/978-3-642-21311-3\_12}{LahimerLH11}~\cite{LahimerLH11} & \href{works/LahimerLH11.pdf}{Climbing Depth-Bounded Adjacent Discrepancy Search for Solving Hybrid Flow Shop Scheduling Problems with Multiprocessor Tasks} &  & benchmark & 2 &  &  &  &  &  &  & \ref{a:LahimerLH11} & \ref{b:LahimerLH11}\\
\rowlabel{c:LombardiBMB11}LombardiBMB11 \href{https://doi.org/10.1007/978-3-642-21311-3\_14}{LombardiBMB11}~\cite{LombardiBMB11} & \href{works/LombardiBMB11.pdf}{Precedence Constraint Posting for Cyclic Scheduling Problems} &  & benchmark, industrial instance, real-life & 0 &  &  &  &  &  &  & \ref{a:LombardiBMB11} & \ref{b:LombardiBMB11}\\
\rowlabel{c:Vilim11}Vilim11 \href{https://doi.org/10.1007/978-3-642-21311-3\_22}{Vilim11}~\cite{Vilim11} & \href{works/Vilim11.pdf}{Timetable Edge Finding Filtering Algorithm for Discrete Cumulative Resources} &  & benchmark & 1 &  &  &  &  &  &  & \ref{a:Vilim11} & \ref{b:Vilim11}\\
\rowlabel{c:ZibranR11}ZibranR11 \href{https://doi.org/10.1109/ICPC.2011.45}{ZibranR11}~\cite{ZibranR11} & \href{works/ZibranR11.pdf}{Conflict-Aware Optimal Scheduling of Code Clone Refactoring: {A} Constraint Programming Approach} &  &  & 0 &  &  &  &  &  &  & \ref{a:ZibranR11} & \ref{b:ZibranR11}\\
\rowlabel{c:ZibranR11a}ZibranR11a \href{https://doi.org/10.1109/SCAM.2011.21}{ZibranR11a}~\cite{ZibranR11a} & \href{works/ZibranR11a.pdf}{A Constraint Programming Approach to Conflict-Aware Optimal Scheduling of Prioritized Code Clone Refactoring} &  &  & 0 &  &  &  &  &  &  & \ref{a:ZibranR11a} & \ref{b:ZibranR11a}\\
\rowlabel{c:BertholdHLMS10}BertholdHLMS10 \href{https://doi.org/10.1007/978-3-642-13520-0\_34}{BertholdHLMS10}~\cite{BertholdHLMS10} & \href{works/BertholdHLMS10.pdf}{A Constraint Integer Programming Approach for Resource-Constrained Project Scheduling} &  &  & 1 &  &  &  &  &  &  & \ref{a:BertholdHLMS10} & \ref{b:BertholdHLMS10}\\
\rowlabel{c:CobanH10}CobanH10 \href{https://doi.org/10.1007/978-3-642-13520-0\_11}{CobanH10}~\cite{CobanH10} & \href{works/CobanH10.pdf}{Single-Facility Scheduling over Long Time Horizons by Logic-Based Benders Decomposition} &  &  & 0 &  &  &  &  &  &  & \ref{a:CobanH10} & \ref{b:CobanH10}\\
\rowlabel{c:Davenport10}Davenport10 \href{https://doi.org/10.1007/978-3-642-13520-0\_12}{Davenport10}~\cite{Davenport10} & \href{works/Davenport10.pdf}{Integrated Maintenance Scheduling for Semiconductor Manufacturing} &  &  & 0 &  &  &  &  &  &  & \ref{a:Davenport10} & \ref{b:Davenport10}\\
\rowlabel{c:GrimesH10}GrimesH10 \href{https://doi.org/10.1007/978-3-642-13520-0\_19}{GrimesH10}~\cite{GrimesH10} & \href{works/GrimesH10.pdf}{Job Shop Scheduling with Setup Times and Maximal Time-Lags: {A} Simple Constraint Programming Approach} &  & benchmark & 1 &  &  &  &  &  &  & \ref{a:GrimesH10} & \ref{b:GrimesH10}\\
\rowlabel{c:LombardiM10}LombardiM10 \href{https://doi.org/10.1007/978-3-642-15396-9\_32}{LombardiM10}~\cite{LombardiM10} & \href{works/LombardiM10.pdf}{Constraint Based Scheduling to Deal with Uncertain Durations and Self-Timed Execution} &  & real-world, benchmark & 1 &  &  &  &  &  &  & \ref{a:LombardiM10} & \ref{b:LombardiM10}\\
\rowlabel{c:MakMS10}MakMS10 \href{https://doi.org/10.1109/ICNC.2010.5583494}{MakMS10}~\cite{MakMS10} & \href{works/MakMS10.pdf}{A constraint programming approach for production scheduling of multi-period virtual cellular manufacturing systems} &  &  & 0 &  &  &  &  &  &  & \ref{a:MakMS10} & \ref{b:MakMS10}\\
\rowlabel{c:SchuttW10}SchuttW10 \href{https://doi.org/10.1007/978-3-642-15396-9\_36}{SchuttW10}~\cite{SchuttW10} & \href{works/SchuttW10.pdf}{A New \emph{O}(\emph{n}\({}^{\mbox{2}}\)log\emph{n}) Not-First/Not-Last Pruning Algorithm for Cumulative Resource Constraints} &  & benchmark & 1 &  &  &  &  &  &  & \ref{a:SchuttW10} & \ref{b:SchuttW10}\\
\rowlabel{c:SunLYL10}SunLYL10 \href{https://doi.org/10.1109/GreenCom-CPSCom.2010.111}{SunLYL10}~\cite{SunLYL10} & \href{works/SunLYL10.pdf}{Scheduling Optimization Techniques for FlexRay Using Constraint-Programming} &  &  & 0 &  &  &  &  &  &  & \ref{a:SunLYL10} & \ref{b:SunLYL10}\\
\rowlabel{c:Acuna-AgostMFG09}Acuna-AgostMFG09 \href{https://doi.org/10.1007/978-3-642-01929-6\_24}{Acuna-AgostMFG09}~\cite{Acuna-AgostMFG09} & \href{works/Acuna-AgostMFG09.pdf}{Constraint Programming and Mixed Integer Linear Programming for Rescheduling Trains under Disrupted Operations} &  & Roadef & 1 &  &  &  &  &  &  & \ref{a:Acuna-AgostMFG09} & \ref{b:Acuna-AgostMFG09}\\
\rowlabel{c:AronssonBK09}AronssonBK09 \href{http://drops.dagstuhl.de/opus/volltexte/2009/2141}{AronssonBK09}~\cite{AronssonBK09} & \href{works/AronssonBK09.pdf}{{MILP} formulations of cumulative constraints for railway scheduling - {A} comparative study} &  & real-world, real-life & 0 &  &  &  &  &  &  & \ref{a:AronssonBK09} & \ref{b:AronssonBK09}\\
\rowlabel{c:Baptiste09}Baptiste09 \href{https://doi.org/10.1007/978-3-642-04244-7\_1}{Baptiste09}~\cite{Baptiste09} & \href{works/Baptiste09.pdf}{Constraint-Based Schedulers, Do They Really Work?} &  &  & 0 &  &  &  &  &  &  & \ref{a:Baptiste09} & \ref{b:Baptiste09}\\
\rowlabel{c:GrimesHM09}GrimesHM09 \href{https://doi.org/10.1007/978-3-642-04244-7\_33}{GrimesHM09}~\cite{GrimesHM09} & \href{works/GrimesHM09.pdf}{Closing the Open Shop: Contradicting Conventional Wisdom} &  & benchmark & 0 &  &  &  &  &  &  & \ref{a:GrimesHM09} & \ref{b:GrimesHM09}\\
\rowlabel{c:Laborie09}Laborie09 \href{https://doi.org/10.1007/978-3-642-01929-6\_12}{Laborie09}~\cite{Laborie09} & \href{works/Laborie09.pdf}{{IBM} {ILOG} {CP} Optimizer for Detailed Scheduling Illustrated on Three Problems} &  & real-world, benchmark & 2 &  &  &  &  &  &  & \ref{a:Laborie09} & \ref{b:Laborie09}\\
\rowlabel{c:LombardiM09}LombardiM09 \href{https://doi.org/10.1007/978-3-642-04244-7\_45}{LombardiM09}~\cite{LombardiM09} & \href{works/LombardiM09.pdf}{A Precedence Constraint Posting Approach for the {RCPSP} with Time Lags and Variable Durations} &  & real-world, instance generator & 1 &  &  &  &  &  &  & \ref{a:LombardiM09} & \ref{b:LombardiM09}\\
\rowlabel{c:MonetteDH09}MonetteDH09 \href{http://aaai.org/ocs/index.php/ICAPS/ICAPS09/paper/view/712}{MonetteDH09}~\cite{MonetteDH09} & \href{works/MonetteDH09.pdf}{Just-In-Time Scheduling with Constraint Programming} &  & benchmark & 0 &  &  &  &  &  &  & \ref{a:MonetteDH09} & \ref{b:MonetteDH09}\\
\rowlabel{c:SchuttFSW09}SchuttFSW09 \href{https://doi.org/10.1007/978-3-642-04244-7\_58}{SchuttFSW09}~\cite{SchuttFSW09} & \href{works/SchuttFSW09.pdf}{Why Cumulative Decomposition Is Not as Bad as It Sounds} &  & benchmark, real-world & 1 &  &  &  &  &  &  & \ref{a:SchuttFSW09} & \ref{b:SchuttFSW09}\\
\rowlabel{c:ThiruvadyBME09}ThiruvadyBME09 \href{https://doi.org/10.1007/978-3-642-04918-7\_3}{ThiruvadyBME09}~\cite{ThiruvadyBME09} & \href{works/ThiruvadyBME09.pdf}{Hybridizing Beam-ACO with Constraint Programming for Single Machine Job Scheduling} &  &  & 0 &  &  &  &  &  &  & \ref{a:ThiruvadyBME09} & \ref{b:ThiruvadyBME09}\\
\rowlabel{c:Vilim09}Vilim09 \href{https://doi.org/10.1007/978-3-642-04244-7\_62}{Vilim09}~\cite{Vilim09} & \href{works/Vilim09.pdf}{Edge Finding Filtering Algorithm for Discrete Cumulative Resources in \emph{O}(\emph{kn} log \emph{n})\{{\textbackslash}mathcal O\}(kn \{{\textbackslash}rm log\} n)} &  &  & 0 &  &  &  &  &  &  & \ref{a:Vilim09} & \ref{b:Vilim09}\\
\rowlabel{c:Vilim09a}Vilim09a \href{https://doi.org/10.1007/978-3-642-01929-6\_22}{Vilim09a}~\cite{Vilim09a} & \href{works/Vilim09a.pdf}{Max Energy Filtering Algorithm for Discrete Cumulative Resources} &  &  & 1 &  &  &  &  &  &  & \ref{a:Vilim09a} & \ref{b:Vilim09a}\\
\rowlabel{c:BarlattCG08}BarlattCG08 \href{https://doi.org/10.1007/978-3-540-68155-7\_24}{BarlattCG08}~\cite{BarlattCG08} & \href{works/BarlattCG08.pdf}{A Hybrid Approach for Solving Shift-Selection and Task-Sequencing Problems} &  & real-world & 1 &  &  &  &  &  &  & \ref{a:BarlattCG08} & \ref{b:BarlattCG08}\\
\rowlabel{c:BeldiceanuCP08}BeldiceanuCP08 \href{https://doi.org/10.1007/978-3-540-68155-7\_5}{BeldiceanuCP08}~\cite{BeldiceanuCP08} & \href{works/BeldiceanuCP08.pdf}{New Filtering for the cumulative Constraint in the Context of Non-Overlapping Rectangles} &  & benchmark & 0 &  &  &  &  &  &  & \ref{a:BeldiceanuCP08} & \ref{b:BeldiceanuCP08}\\
\rowlabel{c:DoomsH08}DoomsH08 \href{https://doi.org/10.1007/978-3-540-68155-7\_8}{DoomsH08}~\cite{DoomsH08} & \href{works/DoomsH08.pdf}{Gap Reduction Techniques for Online Stochastic Project Scheduling} &  &  & 0 &  &  &  &  &  &  & \ref{a:DoomsH08} & \ref{b:DoomsH08}\\
\rowlabel{c:HentenryckM08}HentenryckM08 \href{https://doi.org/10.1007/978-3-540-68155-7\_41}{HentenryckM08}~\cite{HentenryckM08} & \href{works/HentenryckM08.pdf}{The Steel Mill Slab Design Problem Revisited} &  & CSPlib & 0 &  &  &  &  &  &  & \ref{a:HentenryckM08} & \ref{b:HentenryckM08}\\
\rowlabel{c:LauLN08}LauLN08 \href{https://doi.org/10.1007/978-3-540-68155-7\_33}{LauLN08}~\cite{LauLN08} & \href{works/LauLN08.pdf}{A Combinatorial Auction Framework for Solving Decentralized Scheduling Problems (Extended Abstract)} &  & benchmark, real-world & 0 &  &  &  &  &  &  & \ref{a:LauLN08} & \ref{b:LauLN08}\\
\rowlabel{c:MouraSCL08}MouraSCL08 \href{https://doi.org/10.1007/978-3-540-85958-1\_3}{MouraSCL08}~\cite{MouraSCL08} & \href{works/MouraSCL08.pdf}{Planning and Scheduling the Operation of a Very Large Oil Pipeline Network} &  &  & 0 &  &  &  &  &  &  & \ref{a:MouraSCL08} & \ref{b:MouraSCL08}\\
\rowlabel{c:MouraSCL08a}MouraSCL08a \href{https://doi.org/10.1109/CSE.2008.24}{MouraSCL08a}~\cite{MouraSCL08a} & \href{works/MouraSCL08a.pdf}{Heuristics and Constraint Programming Hybridizations for a Real Pipeline Planning and Scheduling Problem} &  & real-world, benchmark & 0 &  &  &  &  &  &  & \ref{a:MouraSCL08a} & \ref{b:MouraSCL08a}\\
\rowlabel{c:PoderB08}PoderB08 \href{http://www.aaai.org/Library/ICAPS/2008/icaps08-033.php}{PoderB08}~\cite{PoderB08} & \href{works/PoderB08.pdf}{Filtering for a Continuous Multi-Resources cumulative Constraint with Resource Consumption and Production} &  &  & 0 &  &  &  &  &  &  & \ref{a:PoderB08} & \ref{b:PoderB08}\\
\rowlabel{c:WatsonB08}WatsonB08 \href{https://doi.org/10.1007/978-3-540-68155-7\_21}{WatsonB08}~\cite{WatsonB08} & \href{works/WatsonB08.pdf}{A Hybrid Constraint Programming / Local Search Approach to the Job-Shop Scheduling Problem} &  & benchmark, real-world & 1 &  &  &  &  &  &  & \ref{a:WatsonB08} & \ref{b:WatsonB08}\\
\rowlabel{c:AkkerDH07}AkkerDH07 \href{https://doi.org/10.1007/978-3-540-72397-4\_27}{AkkerDH07}~\cite{AkkerDH07} & \href{works/AkkerDH07.pdf}{A Column Generation Based Destructive Lower Bound for Resource Constrained Project Scheduling Problems} &  &  & 0 &  &  &  &  &  &  & \ref{a:AkkerDH07} & \ref{b:AkkerDH07}\\
\rowlabel{c:BeldiceanuP07}BeldiceanuP07 \href{https://doi.org/10.1007/978-3-540-72397-4\_16}{BeldiceanuP07}~\cite{BeldiceanuP07} & \href{works/BeldiceanuP07.pdf}{A Continuous Multi-resources \emph{cumulative} Constraint with Positive-Negative Resource Consumption-Production} &  &  & 0 &  &  &  &  &  &  & \ref{a:BeldiceanuP07} & \ref{b:BeldiceanuP07}\\
\rowlabel{c:DavenportKRSH07}DavenportKRSH07 \href{https://doi.org/10.1007/978-3-540-74970-7\_7}{DavenportKRSH07}~\cite{DavenportKRSH07} & \href{works/DavenportKRSH07.pdf}{An Application of Constraint Programming to Generating Detailed Operations Schedules for Steel Manufacturing} &  &  & 0 &  &  &  &  &  &  & \ref{a:DavenportKRSH07} & \ref{b:DavenportKRSH07}\\
\rowlabel{c:GarganiR07}GarganiR07 \href{https://doi.org/10.1007/978-3-540-74970-7\_8}{GarganiR07}~\cite{GarganiR07} & \href{works/GarganiR07.pdf}{An Efficient Model and Strategy for the Steel Mill Slab Design Problem} &  & real-life, CSPlib & 0 &  &  &  &  &  &  & \ref{a:GarganiR07} & \ref{b:GarganiR07}\\
\rowlabel{c:HoeveGSL07}HoeveGSL07 \href{http://www.aaai.org/Library/AAAI/2007/aaai07-291.php}{HoeveGSL07}~\cite{HoeveGSL07} & \href{works/HoeveGSL07.pdf}{Optimal Multi-Agent Scheduling with Constraint Programming} &  & benchmark & 0 &  &  &  &  &  &  & \ref{a:HoeveGSL07} & \ref{b:HoeveGSL07}\\
\rowlabel{c:KeriK07}KeriK07 \href{https://doi.org/10.1007/978-3-540-72397-4\_10}{KeriK07}~\cite{KeriK07} & \href{works/KeriK07.pdf}{Computing Tight Time Windows for {RCPSPWET} with the Primal-Dual Method} &  &  & 2 &  &  &  &  &  &  & \ref{a:KeriK07} & \ref{b:KeriK07}\\
\rowlabel{c:KovacsB07}KovacsB07 \href{https://doi.org/10.1007/978-3-540-72397-4\_9}{KovacsB07}~\cite{KovacsB07} & \href{works/KovacsB07.pdf}{A Global Constraint for Total Weighted Completion Time} &  & benchmark & 0 &  &  &  &  &  &  & \ref{a:KovacsB07} & \ref{b:KovacsB07}\\
\rowlabel{c:KrogtLPHJ07}KrogtLPHJ07 \href{https://doi.org/10.1007/978-3-540-74970-7\_10}{KrogtLPHJ07}~\cite{KrogtLPHJ07} & \href{works/KrogtLPHJ07.pdf}{Scheduling for Cellular Manufacturing} &  & real-world & 0 &  &  &  &  &  &  & \ref{a:KrogtLPHJ07} & \ref{b:KrogtLPHJ07}\\
\rowlabel{c:Limtanyakul07}Limtanyakul07 \href{https://doi.org/10.1007/978-3-540-77903-2\_65}{Limtanyakul07}~\cite{Limtanyakul07} & \href{works/Limtanyakul07.pdf}{Scheduling of Tests on Vehicle Prototypes Using Constraint and Integer Programming} &  & real-life & 0 &  &  &  &  &  &  & \ref{a:Limtanyakul07} & \ref{b:Limtanyakul07}\\
\rowlabel{c:MonetteDD07}MonetteDD07 \href{https://doi.org/10.1007/978-3-540-72397-4\_14}{MonetteDD07}~\cite{MonetteDD07} & \href{works/MonetteDD07.pdf}{A Position-Based Propagator for the Open-Shop Problem} &  & benchmark & 0 &  &  &  &  &  &  & \ref{a:MonetteDD07} & \ref{b:MonetteDD07}\\
\rowlabel{c:NethercoteSBBDT07}NethercoteSBBDT07 \href{https://doi.org/10.1007/978-3-540-74970-7\_38}{NethercoteSBBDT07}~\cite{NethercoteSBBDT07} & \href{works/NethercoteSBBDT07.pdf}{MiniZinc: Towards a Standard {CP} Modelling Language} &  & CSPlib, benchmark & 1 &  &  &  &  &  &  & \ref{a:NethercoteSBBDT07} & \ref{b:NethercoteSBBDT07}\\
\rowlabel{c:RossiTHP07}RossiTHP07 \href{https://doi.org/10.1007/978-3-540-72397-4\_17}{RossiTHP07}~\cite{RossiTHP07} & \href{works/RossiTHP07.pdf}{Replenishment Planning for Stochastic Inventory Systems with Shortage Cost} &  &  & 0 &  &  &  &  &  &  & \ref{a:RossiTHP07} & \ref{b:RossiTHP07}\\
\rowlabel{c:Beck06}Beck06 \href{http://www.aaai.org/Library/ICAPS/2006/icaps06-028.php}{Beck06}~\cite{Beck06} & \href{works/Beck06.pdf}{An Empirical Study of Multi-Point Constructive Search for Constraint-Based Scheduling} &  & benchmark & 0 &  &  &  &  &  &  & \ref{a:Beck06} & \ref{b:Beck06}\\
\rowlabel{c:BeniniBGM06}BeniniBGM06 \href{https://doi.org/10.1007/11757375\_6}{BeniniBGM06}~\cite{BeniniBGM06} & \href{works/BeniniBGM06.pdf}{Allocation, Scheduling and Voltage Scaling on Energy Aware MPSoCs} &  & real-life & 0 &  &  &  &  &  &  & \ref{a:BeniniBGM06} & \ref{b:BeniniBGM06}\\
\rowlabel{c:GomesHS06}GomesHS06 \href{http://www.aaai.org/Library/Symposia/Spring/2006/ss06-04-024.php}{GomesHS06}~\cite{GomesHS06} & \href{works/GomesHS06.pdf}{Constraint Programming for Distributed Planning and Scheduling} &  & real-life & 0 &  &  &  &  &  &  & \ref{a:GomesHS06} & \ref{b:GomesHS06}\\
\rowlabel{c:KhemmoudjPB06}KhemmoudjPB06 \href{https://doi.org/10.1007/11889205\_21}{KhemmoudjPB06}~\cite{KhemmoudjPB06} & \href{works/KhemmoudjPB06.pdf}{When Constraint Programming and Local Search Solve the Scheduling Problem of Electricit{\'{e}} de France Nuclear Power Plant Outages} &  & real-world & 0 &  &  &  &  &  &  & \ref{a:KhemmoudjPB06} & \ref{b:KhemmoudjPB06}\\
\rowlabel{c:KovacsV06}KovacsV06 \href{https://doi.org/10.1007/11757375\_13}{KovacsV06}~\cite{KovacsV06} & \href{works/KovacsV06.pdf}{Progressive Solutions: {A} Simple but Efficient Dominance Rule for Practical {RCPSP}} &  & industrial partner, benchmark, generated instance & 0 &  &  &  &  &  &  & \ref{a:KovacsV06} & \ref{b:KovacsV06}\\
\rowlabel{c:LiuJ06}LiuJ06 \href{https://doi.org/10.1007/11801603\_92}{LiuJ06}~\cite{LiuJ06} & \href{works/LiuJ06.pdf}{{LP-TPOP:} Integrating Planning and Scheduling Through Constraint Programming} &  &  & 0 &  &  &  &  &  &  & \ref{a:LiuJ06} & \ref{b:LiuJ06}\\
\rowlabel{c:QuSN06}QuSN06 \href{https://doi.org/10.1109/ISSOC.2006.321973}{QuSN06}~\cite{QuSN06} & \href{works/QuSN06.pdf}{Using Constraint Programming to Achieve Optimal Prefetch Scheduling for Dependent Tasks on Run-Time Reconfigurable Devices} &  &  & 0 &  &  &  &  &  &  & \ref{a:QuSN06} & \ref{b:QuSN06}\\
\rowlabel{c:AbrilSB05}AbrilSB05 \href{https://doi.org/10.1007/11564751\_75}{AbrilSB05}~\cite{AbrilSB05} & \href{works/AbrilSB05.pdf}{Distributed Constraints for Large-Scale Scheduling Problems} &  &  & 0 &  &  &  &  &  &  & \ref{a:AbrilSB05} & \ref{b:AbrilSB05}\\
\rowlabel{c:ArtiouchineB05}ArtiouchineB05 \href{https://doi.org/10.1007/11564751\_8}{ArtiouchineB05}~\cite{ArtiouchineB05} & \href{works/ArtiouchineB05.pdf}{Inter-distance Constraint: An Extension of the All-Different Constraint for Scheduling Equal Length Jobs} &  & generated instance, random instance & 0 &  &  &  &  &  &  & \ref{a:ArtiouchineB05} & \ref{b:ArtiouchineB05}\\
\rowlabel{c:BeckW05}BeckW05 \href{http://ijcai.org/Proceedings/05/Papers/0748.pdf}{BeckW05}~\cite{BeckW05} & \href{works/BeckW05.pdf}{Proactive Algorithms for Scheduling with Probabilistic Durations} &  &  & 0 &  &  &  &  &  &  & \ref{a:BeckW05} & \ref{b:BeckW05}\\
\rowlabel{c:CarchraeBF05}CarchraeBF05 \href{https://doi.org/10.1007/11564751\_80}{CarchraeBF05}~\cite{CarchraeBF05} & \href{works/CarchraeBF05.pdf}{Methods to Learn Abstract Scheduling Models} &  &  & 0 &  &  &  &  &  &  & \ref{a:CarchraeBF05} & \ref{b:CarchraeBF05}\\
\rowlabel{c:ChuX05}ChuX05 \href{https://doi.org/10.1007/11493853\_10}{ChuX05}~\cite{ChuX05} & \href{works/ChuX05.pdf}{A Hybrid Algorithm for a Class of Resource Constrained Scheduling Problems} &  &  & 0 &  &  &  &  &  &  & \ref{a:ChuX05} & \ref{b:ChuX05}\\
\rowlabel{c:DilkinaDH05}DilkinaDH05 \href{https://doi.org/10.1007/11564751\_60}{DilkinaDH05}~\cite{DilkinaDH05} & \href{works/DilkinaDH05.pdf}{Extending Systematic Local Search for Job Shop Scheduling Problems} &  &  & 0 &  &  &  &  &  &  & \ref{a:DilkinaDH05} & \ref{b:DilkinaDH05}\\
\rowlabel{c:FortinZDF05}FortinZDF05 \href{https://doi.org/10.1007/11564751\_19}{FortinZDF05}~\cite{FortinZDF05} & \href{works/FortinZDF05.pdf}{Interval Analysis in Scheduling} &  &  & 0 &  &  &  &  &  &  & \ref{a:FortinZDF05} & \ref{b:FortinZDF05}\\
\rowlabel{c:FrankK05}FrankK05 \href{https://doi.org/10.1007/11493853\_15}{FrankK05}~\cite{FrankK05} & \href{works/FrankK05.pdf}{Mixed Discrete and Continuous Algorithms for Scheduling Airborne Astronomy Observations} &  & benchmark & 0 &  &  &  &  &  &  & \ref{a:FrankK05} & \ref{b:FrankK05}\\
\rowlabel{c:Geske05}Geske05 \href{https://doi.org/10.1007/11963578\_10}{Geske05}~\cite{Geske05} & \href{works/Geske05.pdf}{Railway Scheduling with Declarative Constraint Programming} &  & real-life & 0 &  &  &  &  &  &  & \ref{a:Geske05} & \ref{b:Geske05}\\
\rowlabel{c:GodardLN05}GodardLN05 \href{http://www.aaai.org/Library/ICAPS/2005/icaps05-009.php}{GodardLN05}~\cite{GodardLN05} & \href{works/GodardLN05.pdf}{Randomized Large Neighborhood Search for Cumulative Scheduling} &  & benchmark & 0 &  &  &  &  &  &  & \ref{a:GodardLN05} & \ref{b:GodardLN05}\\
\rowlabel{c:HebrardTW05}HebrardTW05 \href{https://doi.org/10.1007/11564751\_117}{HebrardTW05}~\cite{HebrardTW05} & \href{works/HebrardTW05.pdf}{Computing Super-Schedules} &  &  & 0 &  &  &  &  &  &  & \ref{a:HebrardTW05} & \ref{b:HebrardTW05}\\
\rowlabel{c:Hooker05a}Hooker05a \href{https://doi.org/10.1007/11564751\_25}{Hooker05a}~\cite{Hooker05a} & \href{works/Hooker05a.pdf}{Planning and Scheduling to Minimize Tardiness} &  &  & 0 &  &  &  &  &  &  & \ref{a:Hooker05a} & \ref{b:Hooker05a}\\
\rowlabel{c:KovacsEKV05}KovacsEKV05 \href{https://doi.org/10.1007/11564751\_118}{KovacsEKV05}~\cite{KovacsEKV05} & \href{works/KovacsEKV05.pdf}{Proterv-II: An Integrated Production Planning and Scheduling System} &  & real-life & 0 &  &  &  &  &  &  & \ref{a:KovacsEKV05} & \ref{b:KovacsEKV05}\\
\rowlabel{c:MoffittPP05}MoffittPP05 \href{http://www.aaai.org/Library/AAAI/2005/aaai05-188.php}{MoffittPP05}~\cite{MoffittPP05} & \href{works/MoffittPP05.pdf}{Augmenting Disjunctive Temporal Problems with Finite-Domain Constraints} &  &  & 0 &  &  &  &  &  &  & \ref{a:MoffittPP05} & \ref{b:MoffittPP05}\\
\rowlabel{c:QuirogaZH05}QuirogaZH05 \href{https://doi.org/10.1109/ROBOT.2005.1570686}{QuirogaZH05}~\cite{QuirogaZH05} & \href{works/QuirogaZH05.pdf}{A Constraint Programming Approach to Tool Allocation and Resource Scheduling in {FMS}} &  &  & 0 &  &  &  &  &  &  & \ref{a:QuirogaZH05} & \ref{b:QuirogaZH05}\\
\rowlabel{c:SchuttWS05}SchuttWS05 \href{https://doi.org/10.1007/11963578\_6}{SchuttWS05}~\cite{SchuttWS05} & \href{works/SchuttWS05.pdf}{Not-First and Not-Last Detection for Cumulative Scheduling in \emph{O}(\emph{n}\({}^{\mbox{3}}\)log\emph{n})} &  & benchmark & 0 &  &  &  &  &  &  & \ref{a:SchuttWS05} & \ref{b:SchuttWS05}\\
\rowlabel{c:Vilim05}Vilim05 \href{https://doi.org/10.1007/11493853\_29}{Vilim05}~\cite{Vilim05} & \href{works/Vilim05.pdf}{Computing Explanations for the Unary Resource Constraint} &  & benchmark & 4 &  &  &  &  &  &  & \ref{a:Vilim05} & \ref{b:Vilim05}\\
\rowlabel{c:WolfS05}WolfS05 \href{https://doi.org/10.1007/11963578\_8}{WolfS05}~\cite{WolfS05} & \href{works/WolfS05.pdf}{\emph{O}(\emph{n} log\emph{n}) Overload Checking for the Cumulative Constraint and Its Application} &  & real-world & 0 &  &  &  &  &  &  & \ref{a:WolfS05} & \ref{b:WolfS05}\\
\rowlabel{c:WuBB05}WuBB05 \href{https://doi.org/10.1007/11564751\_110}{WuBB05}~\cite{WuBB05} & \href{works/WuBB05.pdf}{Scheduling with Uncertain Start Dates} &  & benchmark & 0 &  &  &  &  &  &  & \ref{a:WuBB05} & \ref{b:WuBB05}\\
\rowlabel{c:ArtiguesBF04}ArtiguesBF04 \href{https://doi.org/10.1007/978-3-540-24664-0\_3}{ArtiguesBF04}~\cite{ArtiguesBF04} & \href{works/ArtiguesBF04.pdf}{A New Exact Solution Algorithm for the Job Shop Problem with Sequence-Dependent Setup Times} &  & benchmark & 0 &  &  &  &  &  &  & \ref{a:ArtiguesBF04} & \ref{b:ArtiguesBF04}\\
\rowlabel{c:BeckW04}BeckW04 \href{}{BeckW04}~\cite{BeckW04} & \href{works/BeckW04.pdf}{Job Shop Scheduling with Probabilistic Durations} &  &  & 0 &  &  &  &  &  &  & \ref{a:BeckW04} & \ref{b:BeckW04}\\
\rowlabel{c:HentenryckM04}HentenryckM04 \href{https://doi.org/10.1007/978-3-540-24664-0\_22}{HentenryckM04}~\cite{HentenryckM04} & \href{works/HentenryckM04.pdf}{Scheduling Abstractions for Local Search} &  & benchmark & 0 &  &  &  &  &  &  & \ref{a:HentenryckM04} & \ref{b:HentenryckM04}\\
\rowlabel{c:Hooker04}Hooker04 \href{https://doi.org/10.1007/978-3-540-30201-8\_24}{Hooker04}~\cite{Hooker04} & \href{works/Hooker04.pdf}{A Hybrid Method for Planning and Scheduling} &  & random instance & 0 &  &  &  &  &  &  & \ref{a:Hooker04} & \ref{b:Hooker04}\\
\rowlabel{c:KovacsV04}KovacsV04 \href{https://doi.org/10.1007/978-3-540-30201-8\_26}{KovacsV04}~\cite{KovacsV04} & \href{works/KovacsV04.pdf}{Completable Partial Solutions in Constraint Programming and Constraint-Based Scheduling} &  & industrial partner, benchmark, real-life & 0 &  &  &  &  &  &  & \ref{a:KovacsV04} & \ref{b:KovacsV04}\\
\rowlabel{c:LimRX04}LimRX04 \href{https://doi.org/10.1007/978-3-540-30201-8\_59}{LimRX04}~\cite{LimRX04} & \href{works/LimRX04.pdf}{Solving the Crane Scheduling Problem Using Intelligent Search Schemes} &  & generated instance & 0 &  &  &  &  &  &  & \ref{a:LimRX04} & \ref{b:LimRX04}\\
\rowlabel{c:MaraveliasG04}MaraveliasG04 \href{https://doi.org/10.1007/978-3-540-24664-0\_1}{MaraveliasG04}~\cite{MaraveliasG04} & \href{works/MaraveliasG04.pdf}{Using {MILP} and {CP} for the Scheduling of Batch Chemical Processes} &  &  & 0 &  &  &  &  &  &  & \ref{a:MaraveliasG04} & \ref{b:MaraveliasG04}\\
\rowlabel{c:Sadykov04}Sadykov04 \href{https://doi.org/10.1007/978-3-540-24664-0\_31}{Sadykov04}~\cite{Sadykov04} & \href{works/Sadykov04.pdf}{A Hybrid Branch-And-Cut Algorithm for the One-Machine Scheduling Problem} &  &  & 0 &  &  &  &  &  &  & \ref{a:Sadykov04} & \ref{b:Sadykov04}\\
\rowlabel{c:Vilim04}Vilim04 \href{https://doi.org/10.1007/978-3-540-24664-0\_23}{Vilim04}~\cite{Vilim04} & \href{works/Vilim04.pdf}{O(n log n) Filtering Algorithms for Unary Resource Constraint} &  & benchmark & 1 &  &  &  &  &  &  & \ref{a:Vilim04} & \ref{b:Vilim04}\\
\rowlabel{c:VilimBC04}VilimBC04 \href{https://doi.org/10.1007/978-3-540-30201-8\_8}{VilimBC04}~\cite{VilimBC04} & \href{works/VilimBC04.pdf}{Unary Resource Constraint with Optional Activities} &  & benchmark, real-life & 0 &  &  &  &  &  &  & \ref{a:VilimBC04} & \ref{b:VilimBC04}\\
\rowlabel{c:VillaverdeP04}VillaverdeP04 \href{}{VillaverdeP04}~\cite{VillaverdeP04} & \href{}{An Investigation of Scheduling in Distributed Constraint Logic Programming} &  &  & 0 &  &  &  &  &  &  & \ref{a:VillaverdeP04} & No\\
\rowlabel{c:WolinskiKG04}WolinskiKG04 \href{https://doi.org/10.1109/DSD.2004.1333291}{WolinskiKG04}~\cite{WolinskiKG04} & \href{works/WolinskiKG04.pdf}{A Constraints Programming Approach to Communication Scheduling on SoPC Architectures} &  &  & 0 &  &  &  &  &  &  & \ref{a:WolinskiKG04} & \ref{b:WolinskiKG04}\\
\rowlabel{c:BeckPS03}BeckPS03 \href{http://www.aaai.org/Library/ICAPS/2003/icaps03-027.php}{BeckPS03}~\cite{BeckPS03} & \href{works/BeckPS03.pdf}{Vehicle Routing and Job Shop Scheduling: What's the Difference?} &  & benchmark, real-world & 0 &  &  &  &  &  &  & \ref{a:BeckPS03} & \ref{b:BeckPS03}\\
\rowlabel{c:DannaP03}DannaP03 \href{https://doi.org/10.1007/978-3-540-45193-8\_59}{DannaP03}~\cite{DannaP03} & \href{works/DannaP03.pdf}{Structured vs. Unstructured Large Neighborhood Search: {A} Case Study on Job-Shop Scheduling Problems with Earliness and Tardiness Costs} &  & benchmark & 0 &  &  &  &  &  &  & \ref{a:DannaP03} & \ref{b:DannaP03}\\
\rowlabel{c:Kumar03}Kumar03 \href{https://doi.org/10.1007/978-3-540-45193-8\_45}{Kumar03}~\cite{Kumar03} & \href{works/Kumar03.pdf}{Incremental Computation of Resource-Envelopes in Producer-Consumer Models} &  &  & 0 &  &  &  &  &  &  & \ref{a:Kumar03} & \ref{b:Kumar03}\\
\rowlabel{c:OddiPCC03}OddiPCC03 \href{https://doi.org/10.1007/978-3-540-45193-8\_39}{OddiPCC03}~\cite{OddiPCC03} & \href{works/OddiPCC03.pdf}{Generating High Quality Schedules for a Spacecraft Memory Downlink Problem} &  & benchmark & 0 &  &  &  &  &  &  & \ref{a:OddiPCC03} & \ref{b:OddiPCC03}\\
\rowlabel{c:ValleMGT03}ValleMGT03 \href{https://doi.org/10.1007/978-3-540-45226-3\_180}{ValleMGT03}~\cite{ValleMGT03} & \href{works/ValleMGT03.pdf}{On Selecting and Scheduling Assembly Plans Using Constraint Programming} &  & real-life & 0 &  &  &  &  &  &  & \ref{a:ValleMGT03} & \ref{b:ValleMGT03}\\
\rowlabel{c:Vilim03}Vilim03 \href{https://doi.org/10.1007/978-3-540-45193-8\_124}{Vilim03}~\cite{Vilim03} & \href{works/Vilim03.pdf}{Computing Explanations for Global Scheduling Constraints} &  &  & 0 &  &  &  &  &  &  & \ref{a:Vilim03} & \ref{b:Vilim03}\\
\rowlabel{c:Wolf03}Wolf03 \href{https://doi.org/10.1007/978-3-540-45193-8\_50}{Wolf03}~\cite{Wolf03} & \href{works/Wolf03.pdf}{Pruning while Sweeping over Task Intervals} &  & benchmark & 0 &  &  &  &  &  &  & \ref{a:Wolf03} & \ref{b:Wolf03}\\
\rowlabel{c:Bartak02}Bartak02 \href{https://doi.org/10.1007/3-540-46135-3\_39}{Bartak02}~\cite{Bartak02} & \href{works/Bartak02.pdf}{Visopt ShopFloor: On the Edge of Planning and Scheduling} &  & real-life & 0 &  &  &  &  &  &  & \ref{a:Bartak02} & \ref{b:Bartak02}\\
\rowlabel{c:Bartak02a}Bartak02a \href{https://doi.org/10.1007/3-540-36607-5\_14}{Bartak02a}~\cite{Bartak02a} & \href{works/Bartak02a.pdf}{Visopt ShopFloor: Going Beyond Traditional Scheduling} &  & benchmark, real-life & 0 &  &  &  &  &  &  & \ref{a:Bartak02a} & \ref{b:Bartak02a}\\
\rowlabel{c:BeldiceanuC02}BeldiceanuC02 \href{https://doi.org/10.1007/3-540-46135-3\_5}{BeldiceanuC02}~\cite{BeldiceanuC02} & \href{works/BeldiceanuC02.pdf}{A New Multi-resource cumulatives Constraint with Negative Heights} &  & real-life, random instance, benchmark & 0 &  &  &  &  &  &  & \ref{a:BeldiceanuC02} & \ref{b:BeldiceanuC02}\\
\rowlabel{c:ElkhyariGJ02}ElkhyariGJ02 \href{https://doi.org/10.1007/3-540-46135-3\_49}{ElkhyariGJ02}~\cite{ElkhyariGJ02} & \href{works/ElkhyariGJ02.pdf}{Conflict-Based Repair Techniques for Solving Dynamic Scheduling Problems} &  &  & 0 &  &  &  &  &  &  & \ref{a:ElkhyariGJ02} & \ref{b:ElkhyariGJ02}\\
\rowlabel{c:ElkhyariGJ02a}ElkhyariGJ02a \href{https://doi.org/10.1007/978-3-540-45157-0\_3}{ElkhyariGJ02a}~\cite{ElkhyariGJ02a} & \href{works/ElkhyariGJ02a.pdf}{Solving Dynamic Resource Constraint Project Scheduling Problems Using New Constraint Programming Tools} &  & benchmark, real-life & 0 &  &  &  &  &  &  & \ref{a:ElkhyariGJ02a} & \ref{b:ElkhyariGJ02a}\\
\rowlabel{c:HookerY02}HookerY02 \href{https://doi.org/10.1007/3-540-46135-3\_46}{HookerY02}~\cite{HookerY02} & \href{works/HookerY02.pdf}{A Relaxation of the Cumulative Constraint} &  &  & 0 &  &  &  &  &  &  & \ref{a:HookerY02} & \ref{b:HookerY02}\\
\rowlabel{c:KamarainenS02}KamarainenS02 \href{https://doi.org/10.1007/3-540-46135-3\_11}{KamarainenS02}~\cite{KamarainenS02} & \href{works/KamarainenS02.pdf}{Local Probing Applied to Scheduling} &  & real-world, benchmark & 2 &  &  &  &  &  &  & \ref{a:KamarainenS02} & \ref{b:KamarainenS02}\\
\rowlabel{c:Muscettola02}Muscettola02 \href{https://doi.org/10.1007/3-540-46135-3\_10}{Muscettola02}~\cite{Muscettola02} & \href{works/Muscettola02.pdf}{Computing the Envelope for Stepwise-Constant Resource Allocations} &  &  & 0 &  &  &  &  &  &  & \ref{a:Muscettola02} & \ref{b:Muscettola02}\\
\rowlabel{c:Vilim02}Vilim02 \href{https://doi.org/10.1007/3-540-46135-3\_62}{Vilim02}~\cite{Vilim02} & \href{works/Vilim02.pdf}{Batch Processing with Sequence Dependent Setup Times} &  &  & 0 &  &  &  &  &  &  & \ref{a:Vilim02} & \ref{b:Vilim02}\\
\rowlabel{c:ZhuS02}ZhuS02 \href{https://doi.org/10.1007/3-540-47961-9\_69}{ZhuS02}~\cite{ZhuS02} & \href{works/ZhuS02.pdf}{A Meeting Scheduling System Based on Open Constraint Programming} &  &  & 0 &  &  &  &  &  &  & \ref{a:ZhuS02} & \ref{b:ZhuS02}\\
\rowlabel{c:Thorsteinsson01}Thorsteinsson01 \href{https://doi.org/10.1007/3-540-45578-7\_2}{Thorsteinsson01}~\cite{Thorsteinsson01} & \href{works/Thorsteinsson01.pdf}{Branch-and-Check: {A} Hybrid Framework Integrating Mixed Integer Programming and Constraint Logic Programming} &  &  & 0 &  &  &  &  &  &  & \ref{a:Thorsteinsson01} & \ref{b:Thorsteinsson01}\\
\rowlabel{c:VanczaM01}VanczaM01 \href{https://doi.org/10.1007/3-540-45578-7\_60}{VanczaM01}~\cite{VanczaM01} & \href{works/VanczaM01.pdf}{A Constraint Engine for Manufacturing Process Planning} &  & real-life, real-world & 0 &  &  &  &  &  &  & \ref{a:VanczaM01} & \ref{b:VanczaM01}\\
\rowlabel{c:VerfaillieL01}VerfaillieL01 \href{https://doi.org/10.1007/3-540-45578-7\_55}{VerfaillieL01}~\cite{VerfaillieL01} & \href{works/VerfaillieL01.pdf}{Selecting and Scheduling Observations for Agile Satellites: Some Lessons from the Constraint Reasoning Community Point of View} &  &  & 0 &  &  &  &  &  &  & \ref{a:VerfaillieL01} & \ref{b:VerfaillieL01}\\
\rowlabel{c:AngelsmarkJ00}AngelsmarkJ00 \href{https://doi.org/10.1007/3-540-45349-0\_35}{AngelsmarkJ00}~\cite{AngelsmarkJ00} & \href{works/AngelsmarkJ00.pdf}{Some Observations on Durations, Scheduling and Allen's Algebra} &  &  & 0 &  &  &  &  &  &  & \ref{a:AngelsmarkJ00} & \ref{b:AngelsmarkJ00}\\
\rowlabel{c:FocacciLN00}FocacciLN00 \href{http://www.aaai.org/Library/AIPS/2000/aips00-010.php}{FocacciLN00}~\cite{FocacciLN00} & \href{works/FocacciLN00.pdf}{Solving Scheduling Problems with Setup Times and Alternative Resources} &  & real-world & 0 &  &  &  &  &  &  & \ref{a:FocacciLN00} & \ref{b:FocacciLN00}\\
\rowlabel{c:KorbaaYG99}KorbaaYG99 \href{https://doi.org/10.23919/ECC.1999.7099947}{KorbaaYG99}~\cite{KorbaaYG99} & \href{works/KorbaaYG99.pdf}{Solving transient scheduling problem for cyclic production using timed Petri nets and constraint programming} &  &  & 0 &  &  &  &  &  &  & \ref{a:KorbaaYG99} & \ref{b:KorbaaYG99}\\
\rowlabel{c:CestaOS98}CestaOS98 \href{https://doi.org/10.1007/3-540-49481-2\_36}{CestaOS98}~\cite{CestaOS98} & \href{works/CestaOS98.pdf}{Scheduling Multi-capacitated Resources Under Complex Temporal Constraints} &  &  & 0 &  &  &  &  &  &  & \ref{a:CestaOS98} & \ref{b:CestaOS98}\\
\rowlabel{c:FrostD98}FrostD98 \href{https://doi.org/10.1007/3-540-49481-2\_40}{FrostD98}~\cite{FrostD98} & \href{works/FrostD98.pdf}{Optimizing with Constraints: {A} Case Study in Scheduling Maintenance of Electric Power Units} &  &  & 0 &  &  &  &  &  &  & \ref{a:FrostD98} & \ref{b:FrostD98}\\
\rowlabel{c:GruianK98}GruianK98 \href{https://doi.org/10.1109/EURMIC.1998.711781}{GruianK98}~\cite{GruianK98} & \href{works/GruianK98.pdf}{Operation Binding and Scheduling for Low Power Using Constraint Logic Programming} &  & benchmark & 0 &  &  &  &  &  &  & \ref{a:GruianK98} & \ref{b:GruianK98}\\
\rowlabel{c:PembertonG98}PembertonG98 \href{https://doi.org/10.1090/dimacs/057/06}{PembertonG98}~\cite{PembertonG98} & \href{works/PembertonG98.pdf}{A constraint-based approach to satellite scheduling} &  &  & 0 &  &  &  &  &  &  & \ref{a:PembertonG98} & \ref{b:PembertonG98}\\
\rowlabel{c:RodosekW98}RodosekW98 \href{https://doi.org/10.1007/3-540-49481-2\_28}{RodosekW98}~\cite{RodosekW98} & \href{works/RodosekW98.pdf}{A Generic Model and Hybrid Algorithm for Hoist Scheduling Problems} &  & benchmark & 0 &  &  &  &  &  &  & \ref{a:RodosekW98} & \ref{b:RodosekW98}\\
\rowlabel{c:Shaw98}Shaw98 \href{https://doi.org/10.1007/3-540-49481-2\_30}{Shaw98}~\cite{Shaw98} & \href{works/Shaw98.pdf}{Using Constraint Programming and Local Search Methods to Solve Vehicle Routing Problems} &  & benchmark & 0 &  &  &  &  &  &  & \ref{a:Shaw98} & \ref{b:Shaw98}\\
\rowlabel{c:BaptisteP97}BaptisteP97 \href{https://doi.org/10.1007/BFb0017454}{BaptisteP97}~\cite{BaptisteP97} & \href{works/BaptisteP97.pdf}{Constraint Propagation and Decomposition Techniques for Highly Disjunctive and Highly Cumulative Project Scheduling Problems} &  & benchmark & 0 &  &  &  &  &  &  & \ref{a:BaptisteP97} & \ref{b:BaptisteP97}\\
\rowlabel{c:BeckDF97}BeckDF97 \href{https://doi.org/10.1007/BFb0017455}{BeckDF97}~\cite{BeckDF97} & \href{works/BeckDF97.pdf}{Five Pitfalls of Empirical Scheduling Research} &  & benchmark, real-world & 0 &  &  &  &  &  &  & \ref{a:BeckDF97} & \ref{b:BeckDF97}\\
\rowlabel{c:BoucherBVBL97}BoucherBVBL97 \href{}{BoucherBVBL97}~\cite{BoucherBVBL97} & \href{}{Multi-criteria Comparison Between Algorithmic, Constraint Logic and Specific Constraint Programming on a Real Schedulingt Problem} &  &  & 0 &  &  &  &  &  &  & \ref{a:BoucherBVBL97} & No\\
\rowlabel{c:Caseau97}Caseau97 \href{https://doi.org/10.1007/BFb0017437}{Caseau97}~\cite{Caseau97} & \href{works/Caseau97.pdf}{Using Constraint Propagation for Complex Scheduling Problems: Managing Size, Complex Resources and Travel} &  & benchmark & 0 &  &  &  &  &  &  & \ref{a:Caseau97} & \ref{b:Caseau97}\\
\rowlabel{c:PapeB97}PapeB97 \href{}{PapeB97}~\cite{PapeB97} & \href{}{A Constraint Programming Library for Preemptive and Non-Preemptive Scheduling} &  &  & 0 &  &  &  &  &  &  & \ref{a:PapeB97} & No\\
\rowlabel{c:BrusoniCLMMT96}BrusoniCLMMT96 \href{https://doi.org/10.1007/3-540-61286-6\_157}{BrusoniCLMMT96}~\cite{BrusoniCLMMT96} & \href{works/BrusoniCLMMT96.pdf}{Resource-Based vs. Task-Based Approaches for Scheduling Problems} &  &  & 0 &  &  &  &  &  &  & \ref{a:BrusoniCLMMT96} & \ref{b:BrusoniCLMMT96}\\
\rowlabel{c:Colombani96}Colombani96 \href{https://doi.org/10.1007/3-540-61551-2\_72}{Colombani96}~\cite{Colombani96} & \href{works/Colombani96.pdf}{Constraint Programming: an Efficient and Practical Approach to Solving the Job-Shop Problem} &  &  & 0 &  &  &  &  &  &  & \ref{a:Colombani96} & \ref{b:Colombani96}\\
\rowlabel{c:Zhou96}Zhou96 \href{https://doi.org/10.1007/3-540-61551-2\_97}{Zhou96}~\cite{Zhou96} & \href{works/Zhou96.pdf}{A Constraint Program for Solving the Job-Shop Problem} &  &  & 0 &  &  &  &  &  &  & \ref{a:Zhou96} & \ref{b:Zhou96}\\
\rowlabel{c:Goltz95}Goltz95 \href{https://doi.org/10.1007/3-540-60299-2\_33}{Goltz95}~\cite{Goltz95} & \href{works/Goltz95.pdf}{Reducing Domains for Search in {CLP(FD)} and Its Application to Job-Shop Scheduling} &  & benchmark & 0 &  &  &  &  &  &  & \ref{a:Goltz95} & \ref{b:Goltz95}\\
\rowlabel{c:Puget95}Puget95 \href{https://doi.org/10.1007/3-540-60299-2\_43}{Puget95}~\cite{Puget95} & \href{works/Puget95.pdf}{Applications of Constraint Programming} &  & benchmark & 0 &  &  &  &  &  &  & \ref{a:Puget95} & \ref{b:Puget95}\\
\rowlabel{c:Simonis95}Simonis95 \href{https://doi.org/10.1007/3-540-60299-2\_42}{Simonis95}~\cite{Simonis95} & \href{works/Simonis95.pdf}{The {CHIP} System and Its Applications} &  &  & 0 &  &  &  &  &  &  & \ref{a:Simonis95} & \ref{b:Simonis95}\\
\rowlabel{c:SimonisC95}SimonisC95 \href{https://doi.org/10.1007/3-540-60299-2\_27}{SimonisC95}~\cite{SimonisC95} & \href{works/SimonisC95.pdf}{Modelling Producer/Consumer Constraints} &  & real-life & 0 &  &  &  &  &  &  & \ref{a:SimonisC95} & \ref{b:SimonisC95}\\
\rowlabel{c:Touraivane95}Touraivane95 \href{https://doi.org/10.1007/3-540-60299-2\_41}{Touraivane95}~\cite{Touraivane95} & \href{works/Touraivane95.pdf}{Constraint Programming and Industrial Applications} &  & real-life & 0 &  &  &  &  &  &  & \ref{a:Touraivane95} & \ref{b:Touraivane95}\\
\rowlabel{c:JourdanFRD94}JourdanFRD94 \href{}{JourdanFRD94}~\cite{JourdanFRD94} & \href{}{Data Alignment and Task Scheduling On Parallel Machines Using Concurrent Constraint Model-based Programming} &  &  & 0 &  &  &  &  &  &  & \ref{a:JourdanFRD94} & No\\
\rowlabel{c:NuijtenA94}NuijtenA94 \href{}{NuijtenA94}~\cite{NuijtenA94} & \href{works/NuijtenA94.pdf}{Constraint Satisfaction for Multiple Capacitated Job Shop Scheduling} &  &  & 0 &  &  &  &  &  &  & \ref{a:NuijtenA94} & \ref{b:NuijtenA94}\\
\rowlabel{c:Wallace94}Wallace94 \href{}{Wallace94}~\cite{Wallace94} & \href{}{Applying Constraints for Scheduling} &  &  & 0 &  &  &  &  &  &  & \ref{a:Wallace94} & No\\
\rowlabel{c:BaptisteLV92}BaptisteLV92 \href{https://doi.org/10.1109/ROBOT.1992.220195}{BaptisteLV92}~\cite{BaptisteLV92} & \href{works/BaptisteLV92.pdf}{Hoist scheduling problem: an approach based on constraint logic programming} &  &  & 0 &  &  &  &  &  &  & \ref{a:BaptisteLV92} & \ref{b:BaptisteLV92}\\
\rowlabel{c:ErtlK91}ErtlK91 \href{https://doi.org/10.1007/3-540-54444-5\_89}{ErtlK91}~\cite{ErtlK91} & \href{works/ErtlK91.pdf}{Optimal Instruction Scheduling using Constraint Logic Programming} &  & real-world, benchmark & 0 &  &  &  &  &  &  & \ref{a:ErtlK91} & \ref{b:ErtlK91}\\
\end{longtable}
}



\clearpage
\section{Journal Articles}

\clearpage
\subsection{Articles from bibtex}
{\scriptsize
\begin{longtable}{>{\raggedright\arraybackslash}p{3cm}>{\raggedright\arraybackslash}p{6cm}>{\raggedright\arraybackslash}p{6.5cm}rrrp{2.5cm}rrrrr}
\rowcolor{white}\caption{Works from bibtex (Total 352)}\\ \toprule
\rowcolor{white}\shortstack{Key\\Source} & Authors & Title & LC & Cite & Year & \shortstack{Conference\\/Journal\\/School} & Pages & \shortstack{Nr\\Cites} & \shortstack{Nr\\Refs} & b & c \\ \midrule\endhead
\bottomrule
\endfoot
\rowlabel{a:ForbesHJST24}ForbesHJST24 \href{http://dx.doi.org/10.1016/j.ejor.2023.07.032}{ForbesHJST24} & \hyperref[auth:a998]{M. Forbes}, \hyperref[auth:a999]{M. Harris}, \hyperref[auth:a1000]{H. Jansen}, \hyperref[auth:a1001]{F.A. van der Schoot}, \hyperref[auth:a1002]{T. Taimre} & Combining optimisation and simulation using logic-based Benders decomposition & \href{../works/ForbesHJST24.pdf}{Yes} & \cite{ForbesHJST24} & 2024 & European Journal of Operational Research & 15 & 0 & 26 & \ref{b:ForbesHJST24} & \ref{c:ForbesHJST24}\\
\rowlabel{a:LuZZYW24}LuZZYW24 \href{https://www.mdpi.com/2077-1312/12/1/124}{LuZZYW24} & \hyperref[auth:a1279]{X. Lu}, \hyperref[auth:a1280]{Y. Zhang}, \hyperref[auth:a1281]{L. Zheng}, \hyperref[auth:a1282]{C. Yang}, \hyperref[auth:a1283]{J. Wang} & Integrated Inbound and Outbound Scheduling for Coal Port: Constraint Programming and Adaptive Local Search & \href{../works/LuZZYW24.pdf}{Yes} & \cite{LuZZYW24} & 2024 & Journal of Marine Science and Engineering & 36 & 0 & 0 & \ref{b:LuZZYW24} & \ref{c:LuZZYW24}\\
\rowlabel{a:PrataAN23}PrataAN23 \href{https://www.sciencedirect.com/science/article/pii/S2666720723001522}{PrataAN23} & \hyperref[auth:a390]{Bruno A. Prata}, \hyperref[auth:a391]{Levi R. Abreu}, \hyperref[auth:a392]{Marcelo S. Nagano} & Applications of constraint programming in production scheduling problems: A descriptive bibliometric analysis & \href{../works/PrataAN23.pdf}{Yes} & \cite{PrataAN23} & 2024 & Results in Control and Optimization & 17 & 0 & 0 & \ref{b:PrataAN23} & \ref{c:PrataAN23}\\
\rowlabel{a:abs-2402-00459}abs-2402-00459 \href{https://doi.org/10.48550/arXiv.2402.00459}{abs-2402-00459} & \hyperref[auth:a400]{S. Nguyen}, \hyperref[auth:a401]{Dhananjay R. Thiruvady}, \hyperref[auth:a402]{Y. Sun}, \hyperref[auth:a403]{M. Zhang} & Genetic-based Constraint Programming for Resource Constrained Job Scheduling & \href{../works/abs-2402-00459.pdf}{Yes} & \cite{abs-2402-00459} & 2024 & CoRR & 21 & 0 & 0 & \ref{b:abs-2402-00459} & \ref{c:abs-2402-00459}\\
\rowlabel{a:AbreuNP23}AbreuNP23 \href{https://doi.org/10.1080/00207543.2022.2154404}{AbreuNP23} & \hyperref[auth:a423]{Levi Ribeiro de Abreu}, \hyperref[auth:a424]{Marcelo Seido Nagano}, \hyperref[auth:a390]{Bruno A. Prata} & A new two-stage constraint programming approach for open shop scheduling problem with machine blocking & \href{../works/AbreuNP23.pdf}{Yes} & \cite{AbreuNP23} & 2023 & International Journal of Production Research & 20 & 1 & 47 & \ref{b:AbreuNP23} & \ref{c:AbreuNP23}\\
\rowlabel{a:AbreuPNF23}AbreuPNF23 \href{https://www.sciencedirect.com/science/article/pii/S0305054823002502}{AbreuPNF23} & \hyperref[auth:a391]{Levi R. Abreu}, \hyperref[auth:a390]{Bruno A. Prata}, \hyperref[auth:a392]{Marcelo S. Nagano}, \hyperref[auth:a842]{Jose M. Framinan} & A constraint programming-based iterated greedy algorithm for the open shop with sequence-dependent processing times and makespan minimization & \href{../works/AbreuPNF23.pdf}{Yes} & \cite{AbreuPNF23} & 2023 & Computers \  Operations Research & 12 & 0 & 46 & \ref{b:AbreuPNF23} & \ref{c:AbreuPNF23}\\
\rowlabel{a:Adelgren2023}Adelgren2023 \href{http://dx.doi.org/10.1016/j.cie.2023.109330}{Adelgren2023} & \hyperref[auth:a980]{N. Adelgren}, \hyperref[auth:a386]{Christos T. Maravelias} & On the utility of production scheduling formulations including record keeping variables & \href{../works/Adelgren2023.pdf}{Yes} & \cite{Adelgren2023} & 2023 & Computers \  Industrial Engineering & 12 & 0 & 43 & \ref{b:Adelgren2023} & \ref{c:Adelgren2023}\\
\rowlabel{a:AfsarVPG23}AfsarVPG23 \href{http://dx.doi.org/10.1016/j.cie.2023.109454}{AfsarVPG23} & \hyperref[auth:a974]{S. Afsar}, \hyperref[auth:a975]{Camino R. Vela}, \hyperref[auth:a976]{Juan José Palacios}, \hyperref[auth:a977]{I. González-Rodríguez} & Mathematical models and benchmarking for the fuzzy job shop scheduling problem & \href{../works/AfsarVPG23.pdf}{Yes} & \cite{AfsarVPG23} & 2023 & Computers \  Industrial Engineering & 14 & 0 & 50 & \ref{b:AfsarVPG23} & \ref{c:AfsarVPG23}\\
\rowlabel{a:AkramNHRSA23}AkramNHRSA23 \href{https://doi.org/10.1109/ACCESS.2023.3343409}{AkramNHRSA23} & \hyperref[auth:a404]{Bilal Omar Akram}, \hyperref[auth:a405]{Nor Kamariah Noordin}, \hyperref[auth:a406]{F. Hashim}, \hyperref[auth:a407]{Mohd Fadlee A. Rasid}, \hyperref[auth:a408]{Mustafa Ismael Salman}, \hyperref[auth:a409]{Abdulrahman M. Abdulghani} & Joint Scheduling and Routing Optimization for Deterministic Hybrid Traffic in Time-Sensitive Networks Using Constraint Programming & \href{../works/AkramNHRSA23.pdf}{Yes} & \cite{AkramNHRSA23} & 2023 & {IEEE} Access & 16 & 0 & 0 & \ref{b:AkramNHRSA23} & \ref{c:AkramNHRSA23}\\
\rowlabel{a:AlfieriGPS23}AlfieriGPS23 \href{https://www.sciencedirect.com/science/article/pii/S0360835223000074}{AlfieriGPS23} & \hyperref[auth:a737]{A. Alfieri}, \hyperref[auth:a15]{M. Garraffa}, \hyperref[auth:a738]{E. Pastore}, \hyperref[auth:a739]{F. Salassa} & Permutation flowshop problems minimizing core waiting time and core idle time & \href{../works/AlfieriGPS23.pdf}{Yes} & \cite{AlfieriGPS23} & 2023 & Computers \  Industrial Engineering & 13 & 0 & 37 & \ref{b:AlfieriGPS23} & \ref{c:AlfieriGPS23}\\
\rowlabel{a:Caballero23}Caballero23 \href{https://doi.org/10.1007/s10601-023-09357-0}{Caballero23} & \hyperref[auth:a102]{Jordi Coll Caballero} & Scheduling through logic-based tools & \href{../works/Caballero23.pdf}{Yes} & \cite{Caballero23} & 2023 & Constraints An Int. J. & 1 & 0 & 0 & \ref{b:Caballero23} & \ref{c:Caballero23}\\
\rowlabel{a:CzerniachowskaWZ23}CzerniachowskaWZ23 \href{https://doi.org/10.12913/22998624/166588}{CzerniachowskaWZ23} & \hyperref[auth:a740]{K. Czerniachowska}, \hyperref[auth:a741]{R. Wichniarek}, \hyperref[auth:a742]{K. Żywicki} & Constraint Programming for Flexible Flow Shop Scheduling Problem with Repeated Jobs and Repeated Operations & \href{../works/CzerniachowskaWZ23.pdf}{Yes} & \cite{CzerniachowskaWZ23} & 2023 & Advances in Science and Technology Research Journal & 14 & 0 & 0 & \ref{b:CzerniachowskaWZ23} & \ref{c:CzerniachowskaWZ23}\\
\rowlabel{a:FahimiQ23}FahimiQ23 \href{http://dx.doi.org/10.1287/ijoc.2021.0138}{FahimiQ23} & \hyperref[auth:a122]{H. Fahimi}, \hyperref[auth:a123]{C. Quimper} & Overload-Checking and Edge-Finding for Robust Cumulative Scheduling & No & \cite{FahimiQ23} & 2023 & INFORMS Journal on Computing & null & 0 & 16 & No & \ref{c:FahimiQ23}\\
\rowlabel{a:Fatemi-AnarakiTFV23}Fatemi-AnarakiTFV23 \href{http://dx.doi.org/10.1016/j.omega.2022.102770}{Fatemi-AnarakiTFV23} & \hyperref[auth:a743]{S. Fatemi-Anaraki}, \hyperref[auth:a435]{R. Tavakkoli{-}Moghaddam}, \hyperref[auth:a744]{M. Foumani}, \hyperref[auth:a745]{B. Vahedi-Nouri} & Scheduling of Multi-Robot Job Shop Systems in Dynamic Environments: Mixed-Integer Linear Programming and Constraint Programming Approaches & \href{../works/Fatemi-AnarakiTFV23.pdf}{Yes} & \cite{Fatemi-AnarakiTFV23} & 2023 & Omega & 15 & 7 & 60 & \ref{b:Fatemi-AnarakiTFV23} & \ref{c:Fatemi-AnarakiTFV23}\\
\rowlabel{a:GhasemiMH23}GhasemiMH23 \href{http://dx.doi.org/10.1080/23302674.2023.2224509}{GhasemiMH23} & \hyperref[auth:a996]{S. Ghasemi}, \hyperref[auth:a435]{R. Tavakkoli{-}Moghaddam}, \hyperref[auth:a997]{M. Hamid} & Operating room scheduling by emphasising human factors and dynamic decision-making styles: a constraint programming method & No & \cite{GhasemiMH23} & 2023 & International Journal of Systems Science: Operations \  Logistics & null & 0 & 104 & No & \ref{c:GhasemiMH23}\\
\rowlabel{a:GokPTGO23}GokPTGO23 \href{https://ideas.repec.org/a/spr/annopr/v320y2023i2d10.1007_s10479-022-04547-0.html}{GokPTGO23} & \hyperref[auth:a1024]{Yagmur S. G{\"{o}}k}, \hyperref[auth:a1025]{S. Padr{\'{o}}n}, \hyperref[auth:a1026]{M. Tomasella}, \hyperref[auth:a1027]{D. Guimarans}, \hyperref[auth:a136]{C. {\"{O}}zt{\"{u}}rk} & {Constraint-based robust planning and scheduling of airport apron operations through simheuristics} & \href{../works/GokPTGO23.pdf}{Yes} & \cite{GokPTGO23} & 2023 & Annals of Operations Research & 36 & 0 & 0 & \ref{b:GokPTGO23} & \ref{c:GokPTGO23}\\
\rowlabel{a:GuoZ23}GuoZ23 \href{http://dx.doi.org/10.1016/j.ejor.2023.06.006}{GuoZ23} & \hyperref[auth:a955]{P. Guo}, \hyperref[auth:a956]{J. Zhu} & Capacity reservation for humanitarian relief: A logic-based Benders decomposition method with subgradient cut & \href{../works/GuoZ23.pdf}{Yes} & \cite{GuoZ23} & 2023 & European Journal of Operational Research & 29 & 0 & 112 & \ref{b:GuoZ23} & \ref{c:GuoZ23}\\
\rowlabel{a:GurPAE23}GurPAE23 \href{https://doi.org/10.1007/s10100-022-00835-z}{GurPAE23} & \hyperref[auth:a417]{S. G{\"{u}}r}, \hyperref[auth:a418]{M. Pinarbasi}, \hyperref[auth:a419]{Haci Mehmet Alakas}, \hyperref[auth:a420]{T. Eren} & Operating room scheduling with surgical team: a new approach with constraint programming and goal programming & \href{../works/GurPAE23.pdf}{Yes} & \cite{GurPAE23} & 2023 & Central Eur. J. Oper. Res. & 25 & 1 & 40 & \ref{b:GurPAE23} & \ref{c:GurPAE23}\\
\rowlabel{a:IsikYA23}IsikYA23 \href{https://doi.org/10.1007/s00500-023-09086-9}{IsikYA23} & \hyperref[auth:a425]{Ey{\"{u}}p Ensar Isik}, \hyperref[auth:a426]{Seyda Topaloglu Yildiz}, \hyperref[auth:a427]{{\"{O}}zge Satir Akpunar} & Constraint programming models for the hybrid flow shop scheduling problem and its extensions & \href{../works/IsikYA23.pdf}{Yes} & \cite{IsikYA23} & 2023 & Soft Comput. & 28 & 0 & 127 & \ref{b:IsikYA23} & \ref{c:IsikYA23}\\
\rowlabel{a:JuvinHL23a}JuvinHL23a \href{http://dx.doi.org/10.1016/j.cor.2023.106156}{JuvinHL23a} & \hyperref[auth:a0]{C. Juvin}, \hyperref[auth:a2]{L. Houssin}, \hyperref[auth:a3]{P. Lopez} & Logic-based Benders decomposition for the preemptive flexible job-shop scheduling problem & \href{../works/JuvinHL23a.pdf}{Yes} & \cite{JuvinHL23a} & 2023 & Computers \  Operations Research & 17 & 0 & 40 & \ref{b:JuvinHL23a} & \ref{c:JuvinHL23a}\\
\rowlabel{a:LacknerMMWW23}LacknerMMWW23 \href{https://doi.org/10.1007/s10601-023-09347-2}{LacknerMMWW23} & \hyperref[auth:a62]{M. Lackner}, \hyperref[auth:a63]{C. Mrkvicka}, \hyperref[auth:a45]{N. Musliu}, \hyperref[auth:a46]{D. Walkiewicz}, \hyperref[auth:a43]{F. Winter} & Exact methods for the Oven Scheduling Problem & \href{../works/LacknerMMWW23.pdf}{Yes} & \cite{LacknerMMWW23} & 2023 & Constraints An Int. J. & 42 & 0 & 32 & \ref{b:LacknerMMWW23} & \ref{c:LacknerMMWW23}\\
\rowlabel{a:MarliereSPR23}MarliereSPR23 \href{https://www.sciencedirect.com/science/article/pii/S0967066122002611}{MarliereSPR23} & \hyperref[auth:a1033]{G. Marlière}, \hyperref[auth:a1034]{Sonia {Sobieraj Richard}}, \hyperref[auth:a1035]{P. Pellegrini}, \hyperref[auth:a789]{J. Rodriguez} & A conditional time-intervals formulation of the real-time Railway Traffic Management Problem & \href{../works/MarliereSPR23.pdf}{Yes} & \cite{MarliereSPR23} & 2023 & Control Engineering Practice & 22 & 1 & 75 & \ref{b:MarliereSPR23} & \ref{c:MarliereSPR23}\\
\rowlabel{a:MontemanniD23}MontemanniD23 \href{https://doi.org/10.3390/a16010040}{MontemanniD23} & \hyperref[auth:a415]{R. Montemanni}, \hyperref[auth:a416]{M. Dell'Amico} & Solving the Parallel Drone Scheduling Traveling Salesman Problem via Constraint Programming & \href{../works/MontemanniD23.pdf}{Yes} & \cite{MontemanniD23} & 2023 & Algorithms & 13 & 2 & 18 & \ref{b:MontemanniD23} & \ref{c:MontemanniD23}\\
\rowlabel{a:MontemanniD23a}MontemanniD23a \href{https://doi.org/10.1016/j.ejco.2023.100078}{MontemanniD23a} & \hyperref[auth:a415]{R. Montemanni}, \hyperref[auth:a416]{M. Dell'Amico} & Constraint programming models for the parallel drone scheduling vehicle routing problem & \href{../works/MontemanniD23a.pdf}{Yes} & \cite{MontemanniD23a} & 2023 & {EURO} J. Comput. Optim. & 20 & 0 & 14 & \ref{b:MontemanniD23a} & \ref{c:MontemanniD23a}\\
\rowlabel{a:NaderiBZ23}NaderiBZ23 \href{http://dx.doi.org/10.2139/ssrn.4494381}{NaderiBZ23} & \hyperref[auth:a734]{B. Naderi}, \hyperref[auth:a845]{Mehmet A. Begen}, \hyperref[auth:a846]{G. Zhang} & Integrated Order Acceptance and Resource Decisions Under Uncertainty: Robust and Stochastic Approaches & \href{../works/NaderiBZ23.pdf}{Yes} & \cite{NaderiBZ23} & 2023 & SSRN & 32 & 0 & 46 & \ref{b:NaderiBZ23} & \ref{c:NaderiBZ23}\\
\rowlabel{a:NaderiBZR23}NaderiBZR23 \href{http://dx.doi.org/10.1016/j.omega.2022.102805}{NaderiBZR23} & \hyperref[auth:a734]{B. Naderi}, \hyperref[auth:a845]{Mehmet A. Begen}, \hyperref[auth:a847]{Gregory S. Zaric}, \hyperref[auth:a736]{V. Roshanaei} & A novel and efficient exact technique for integrated staffing,  assignment,  routing,  and scheduling of home care services under uncertainty & No & \cite{NaderiBZR23} & 2023 & Omega & 1 & 4 & 64 & No & \ref{c:NaderiBZR23}\\
\rowlabel{a:NaderiRR23}NaderiRR23 \href{https://doi.org/10.1287/ijoc.2023.1287}{NaderiRR23} & \hyperref[auth:a734]{B. Naderi}, \hyperref[auth:a735]{R. Ruiz}, \hyperref[auth:a736]{V. Roshanaei} & Mixed-Integer Programming vs. Constraint Programming for Shop Scheduling Problems: New Results and Outlook & \href{../works/NaderiRR23.pdf}{Yes} & \cite{NaderiRR23} & 2023 & INFORMS Journal on Computing & 27 & 2 & 50 & \ref{b:NaderiRR23} & \ref{c:NaderiRR23}\\
\rowlabel{a:NouriMHD23}NouriMHD23 \href{http://dx.doi.org/10.1080/00207543.2023.2173503}{NouriMHD23} & \hyperref[auth:a745]{B. Vahedi-Nouri}, \hyperref[auth:a958]{R. Tavakkoli-Moghaddam}, \hyperref[auth:a959]{Z. Hanzálek}, \hyperref[auth:a960]{A. Dolgui} & Production scheduling in a reconfigurable manufacturing system benefiting from human-robot collaboration & No & \cite{NouriMHD23} & 2023 & International Journal of Production Research & null & 2 & 44 & No & \ref{c:NouriMHD23}\\
\rowlabel{a:PenzDN23}PenzDN23 \href{https://doi.org/10.1016/j.cor.2022.106092}{PenzDN23} & \hyperref[auth:a1007]{L. Penz}, \hyperref[auth:a1008]{S. Dauz{\`{e}}re{-}P{\'{e}}r{\`{e}}s}, \hyperref[auth:a81]{M. Nattaf} & Minimizing the sum of completion times on a single machine with health index and flexible maintenance operations & \href{../works/PenzDN23.pdf}{Yes} & \cite{PenzDN23} & 2023 & Computers \  Operations Research & 13 & 0 & 34 & \ref{b:PenzDN23} & \ref{c:PenzDN23}\\
\rowlabel{a:ShaikhK23}ShaikhK23 \href{https://doi.org/10.1504/IJESDF.2023.10045616}{ShaikhK23} & \hyperref[auth:a421]{Aftab Ahmed Shaikh}, \hyperref[auth:a422]{Abdullah Ayub Khan} & Management of electronic ledger: a constraint programming approach for solving curricula scheduling problems & \href{../works/ShaikhK23.pdf}{Yes} & \cite{ShaikhK23} & 2023 & Int. J. Electron. Secur. Digit. Forensics & 12 & 0 & 0 & \ref{b:ShaikhK23} & \ref{c:ShaikhK23}\\
\rowlabel{a:YuraszeckMCCR23}YuraszeckMCCR23 \href{https://doi.org/10.1109/ACCESS.2023.3345793}{YuraszeckMCCR23} & \hyperref[auth:a410]{F. Yuraszeck}, \hyperref[auth:a411]{E. Montero}, \hyperref[auth:a412]{D. Canut{-}de{-}Bon}, \hyperref[auth:a413]{N. Cuneo}, \hyperref[auth:a414]{M. Rojel} & A Constraint Programming Formulation of the Multi-Mode Resource-Constrained Project Scheduling Problem for the Flexible Job Shop Scheduling Problem & \href{../works/YuraszeckMCCR23.pdf}{Yes} & \cite{YuraszeckMCCR23} & 2023 & {IEEE} Access & 11 & 0 & 0 & \ref{b:YuraszeckMCCR23} & \ref{c:YuraszeckMCCR23}\\
\rowlabel{a:ZhuSZW23}ZhuSZW23 \href{http://dx.doi.org/10.1016/j.omega.2022.102823}{ZhuSZW23} & \hyperref[auth:a1003]{X. Zhu}, \hyperref[auth:a1004]{J. Son}, \hyperref[auth:a1005]{X. Zhang}, \hyperref[auth:a1006]{J. Wu} & Constraint programming and logic-based Benders decomposition for the integrated process planning and scheduling problem & \href{../works/ZhuSZW23.pdf}{Yes} & \cite{ZhuSZW23} & 2023 & Omega & 22 & 1 & 36 & \ref{b:ZhuSZW23} & \ref{c:ZhuSZW23}\\
\rowlabel{a:abs-2305-19888}abs-2305-19888 \href{https://doi.org/10.48550/arXiv.2305.19888}{abs-2305-19888} & \hyperref[auth:a438]{V. Heinz}, \hyperref[auth:a439]{A. Nov{\'{a}}k}, \hyperref[auth:a313]{M. Vlk}, \hyperref[auth:a116]{Z. Hanz{\'{a}}lek} & Constraint Programming and Constructive Heuristics for Parallel Machine Scheduling with Sequence-Dependent Setups and Common Servers & \href{../works/abs-2305-19888.pdf}{Yes} & \cite{abs-2305-19888} & 2023 & CoRR & 42 & 0 & 0 & \ref{b:abs-2305-19888} & \ref{c:abs-2305-19888}\\
\rowlabel{a:abs-2306-05747}abs-2306-05747 \href{https://doi.org/10.48550/arXiv.2306.05747}{abs-2306-05747} & \hyperref[auth:a58]{P. Tassel}, \hyperref[auth:a61]{M. Gebser}, \hyperref[auth:a428]{K. Schekotihin} & An End-to-End Reinforcement Learning Approach for Job-Shop Scheduling Problems Based on Constraint Programming & \href{../works/abs-2306-05747.pdf}{Yes} & \cite{abs-2306-05747} & 2023 & CoRR & 9 & 0 & 0 & \ref{b:abs-2306-05747} & \ref{c:abs-2306-05747}\\
\rowlabel{a:abs-2312-13682}abs-2312-13682 \href{https://doi.org/10.48550/arXiv.2312.13682}{abs-2312-13682} & \hyperref[auth:a430]{G. Perez}, \hyperref[auth:a431]{G. Glorian}, \hyperref[auth:a432]{W. Suijlen}, \hyperref[auth:a433]{A. Lallouet} & A Constraint Programming Model for Scheduling the Unloading of Trains in Ports: Extended & \href{../works/abs-2312-13682.pdf}{Yes} & \cite{abs-2312-13682} & 2023 & CoRR & 20 & 0 & 0 & \ref{b:abs-2312-13682} & \ref{c:abs-2312-13682}\\
\rowlabel{a:AbreuN22}AbreuN22 \href{https://doi.org/10.1016/j.cie.2022.108128}{AbreuN22} & \hyperref[auth:a423]{Levi Ribeiro de Abreu}, \hyperref[auth:a424]{Marcelo Seido Nagano} & A new hybridization of adaptive large neighborhood search with constraint programming for open shop scheduling with sequence-dependent setup times & \href{../works/AbreuN22.pdf}{Yes} & \cite{AbreuN22} & 2022 & Computers \  Industrial Engineering & 20 & 10 & 56 & \ref{b:AbreuN22} & \ref{c:AbreuN22}\\
\rowlabel{a:AwadMDMT22}AwadMDMT22 \href{http://dx.doi.org/10.1016/j.compchemeng.2021.107565}{AwadMDMT22} & \hyperref[auth:a1195]{M. Awad}, \hyperref[auth:a1196]{K. Mulrennan}, \hyperref[auth:a1197]{J. Donovan}, \hyperref[auth:a1198]{R. Macpherson}, \hyperref[auth:a1199]{D. Tormey} & A constraint programming model for makespan minimisation in batch manufacturing pharmaceutical facilities & No & \cite{AwadMDMT22} & 2022 & Computers \  Chemical Engineering & 1 & 3 & 41 & No & \ref{c:AwadMDMT22}\\
\rowlabel{a:BourreauGGLT22}BourreauGGLT22 \href{https://doi.org/10.1080/00207543.2020.1856436}{BourreauGGLT22} & \hyperref[auth:a446]{E. Bourreau}, \hyperref[auth:a447]{T. Garaix}, \hyperref[auth:a448]{M. Gondran}, \hyperref[auth:a449]{P. Lacomme}, \hyperref[auth:a450]{N. Tchernev} & A constraint-programming based decomposition method for the Generalised Workforce Scheduling and Routing Problem {(GWSRP)} & \href{../works/BourreauGGLT22.pdf}{Yes} & \cite{BourreauGGLT22} & 2022 & International Journal of Production Research & 19 & 4 & 44 & \ref{b:BourreauGGLT22} & \ref{c:BourreauGGLT22}\\
\rowlabel{a:CampeauG22}CampeauG22 \href{https://doi.org/10.1007/s10601-022-09337-w}{CampeauG22} & \hyperref[auth:a103]{L. Campeau}, \hyperref[auth:a9]{M. Gamache} & Short- and medium-term optimization of underground mine planning using constraint programming & \href{../works/CampeauG22.pdf}{Yes} & \cite{CampeauG22} & 2022 & Constraints An Int. J. & 18 & 0 & 22 & \ref{b:CampeauG22} & \ref{c:CampeauG22}\\
\rowlabel{a:ColT22}ColT22 \href{http://dx.doi.org/10.1016/j.orp.2022.100249}{ColT22} & \hyperref[auth:a93]{Giacomo Da Col}, \hyperref[auth:a746]{Erich C. Teppan} & Industrial-size job shop scheduling with constraint programming & \href{../works/ColT22.pdf}{Yes} & \cite{ColT22} & 2022 & Operations Research Perspectives & 19 & 3 & 55 & \ref{b:ColT22} & \ref{c:ColT22}\\
\rowlabel{a:ElciOH22}ElciOH22 \href{http://dx.doi.org/10.1287/ijoc.2022.1184}{ElciOH22} & \hyperref[auth:a942]{\"{O}zg\"{u}n El\c{c}i}, \hyperref[auth:a161]{John N. Hooker} & Stochastic Planning and Scheduling with Logic-Based Benders Decomposition & \href{../works/ElciOH22.pdf}{Yes} & \cite{ElciOH22} & 2022 & INFORMS Journal on Computing & 21 & 2 & 34 & \ref{b:ElciOH22} & \ref{c:ElciOH22}\\
\rowlabel{a:EmdeZD22}EmdeZD22 \href{http://dx.doi.org/10.1007/s10479-022-04891-1}{EmdeZD22} & \hyperref[auth:a969]{S. Emde}, \hyperref[auth:a970]{S. Zehtabian}, \hyperref[auth:a971]{Y. Disser} & Point-to-point and milk run delivery scheduling: models,  complexity results,  and algorithms based on Benders decomposition & \href{../works/EmdeZD22.pdf}{Yes} & \cite{EmdeZD22} & 2022 & Annals of Operations Research & 30 & 0 & 52 & \ref{b:EmdeZD22} & \ref{c:EmdeZD22}\\
\rowlabel{a:EtminaniesfahaniGNMS22}EtminaniesfahaniGNMS22 \href{http://dx.doi.org/10.1007/s42979-022-01487-1}{EtminaniesfahaniGNMS22} & \hyperref[auth:a910]{A. Etminaniesfahani}, \hyperref[auth:a341]{H. Gu}, \hyperref[auth:a911]{Leila Moslemi Naeni}, \hyperref[auth:a912]{A. Salehipour} & A Forward–Backward Relax-and-Solve Algorithm for the Resource-Constrained Project Scheduling Problem & \href{../works/EtminaniesfahaniGNMS22.pdf}{Yes} & \cite{EtminaniesfahaniGNMS22} & 2022 & SN Computer Science & 10 & 0 & 57 & \ref{b:EtminaniesfahaniGNMS22} & \ref{c:EtminaniesfahaniGNMS22}\\
\rowlabel{a:FarsiTM22}FarsiTM22 \href{https://api.semanticscholar.org/CorpusID:250301745}{FarsiTM22} & \hyperref[auth:a521]{A. Farsi}, \hyperref[auth:a747]{S. Ali Torabi}, \hyperref[auth:a520]{M. Mokhtarzadeh} & Integrated surgery scheduling by constraint programming and meta-heuristics & \href{../works/FarsiTM22.pdf}{Yes} & \cite{FarsiTM22} & 2022 & International Journal of Management Science and Engineering Management & 14 & 5 & 47 & \ref{b:FarsiTM22} & \ref{c:FarsiTM22}\\
\rowlabel{a:FetgoD22}FetgoD22 \href{https://doi.org/10.1007/s43069-022-00172-6}{FetgoD22} & \hyperref[auth:a11]{S{\'{e}}v{\'{e}}rine Betmbe Fetgo}, \hyperref[auth:a13]{Cl{\'{e}}mentin Tayou Djam{\'{e}}gni} & Horizontally Elastic Edge-Finder Algorithm for Cumulative Resource Constraint Revisited & \href{../works/FetgoD22.pdf}{Yes} & \cite{FetgoD22} & 2022 & Oper. Res. Forum & 32 & 0 & 20 & \ref{b:FetgoD22} & \ref{c:FetgoD22}\\
\rowlabel{a:HeinzNVH22}HeinzNVH22 \href{https://doi.org/10.1016/j.cie.2022.108586}{HeinzNVH22} & \hyperref[auth:a438]{V. Heinz}, \hyperref[auth:a439]{A. Nov{\'{a}}k}, \hyperref[auth:a313]{M. Vlk}, \hyperref[auth:a116]{Z. Hanz{\'{a}}lek} & Constraint Programming and constructive heuristics for parallel machine scheduling with sequence-dependent setups and common servers & \href{../works/HeinzNVH22.pdf}{Yes} & \cite{HeinzNVH22} & 2022 & Computers \  Industrial Engineering & 16 & 5 & 25 & \ref{b:HeinzNVH22} & \ref{c:HeinzNVH22}\\
\rowlabel{a:HillBCGN22}HillBCGN22 \href{http://dx.doi.org/10.1287/ijoc.2022.1222}{HillBCGN22} & \hyperref[auth:a64]{A. Hill}, \hyperref[auth:a984]{Andrea J. Brickey}, \hyperref[auth:a985]{I. Cipriano}, \hyperref[auth:a986]{M. Goycoolea}, \hyperref[auth:a987]{A. Newman} & Optimization Strategies for Resource-Constrained Project Scheduling Problems in Underground Mining & No & \cite{HillBCGN22} & 2022 & INFORMS Journal on Computing & null & 0 & 53 & No & \ref{c:HillBCGN22}\\
\rowlabel{a:JuvinHL22}JuvinHL22 \href{http://dx.doi.org/10.2139/ssrn.4068164}{JuvinHL22} & \hyperref[auth:a0]{C. Juvin}, \hyperref[auth:a2]{L. Houssin}, \hyperref[auth:a3]{P. Lopez} & Logic-Based Benders Decomposition for the Preemptive Flexible Job-Shop Scheduling Problem & \href{../works/JuvinHL22.pdf}{Yes} & \cite{JuvinHL22} & 2022 & SSRN Electronic Journal & 32 & 0 & 29 & \ref{b:JuvinHL22} & \ref{c:JuvinHL22}\\
\rowlabel{a:MartnezAJ22}MartnezAJ22 \href{http://dx.doi.org/10.1287/ijoc.2021.1079}{MartnezAJ22} & \hyperref[auth:a947]{Karim Pérez Martínez}, \hyperref[auth:a948]{Y. Adulyasak}, \hyperref[auth:a850]{R. Jans} & Logic-Based Benders Decomposition for Integrated Process Configuration and Production Planning Problems & No & \cite{MartnezAJ22} & 2022 & INFORMS Journal on Computing & null & 1 & 29 & No & \ref{c:MartnezAJ22}\\
\rowlabel{a:MengGRZSC22}MengGRZSC22 \href{http://dx.doi.org/10.1016/j.swevo.2022.101058}{MengGRZSC22} & \hyperref[auth:a505]{L. Meng}, \hyperref[auth:a1200]{K. Gao}, \hyperref[auth:a507]{Y. Ren}, \hyperref[auth:a508]{B. Zhang}, \hyperref[auth:a1179]{H. Sang}, \hyperref[auth:a1201]{Z. Chaoyong} & Novel MILP and CP models for distributed hybrid flowshop scheduling problem with sequence-dependent setup times & No & \cite{MengGRZSC22} & 2022 & Swarm and Evolutionary Computation & 1 & 38 & 37 & No & \ref{c:MengGRZSC22}\\
\rowlabel{a:MullerMKP22}MullerMKP22 \href{https://doi.org/10.1016/j.ejor.2022.01.034}{MullerMKP22} & \hyperref[auth:a440]{D. M{\"{u}}ller}, \hyperref[auth:a441]{Marcus Gerhard M{\"{u}}ller}, \hyperref[auth:a442]{D. Kress}, \hyperref[auth:a443]{E. Pesch} & An algorithm selection approach for the flexible job shop scheduling problem: Choosing constraint programming solvers through machine learning & \href{../works/MullerMKP22.pdf}{Yes} & \cite{MullerMKP22} & 2022 & European Journal of Operational Research & 18 & 17 & 59 & \ref{b:MullerMKP22} & \ref{c:MullerMKP22}\\
\rowlabel{a:NaderiBZ22}NaderiBZ22 \href{http://dx.doi.org/10.2139/ssrn.4140716}{NaderiBZ22} & \hyperref[auth:a734]{B. Naderi}, \hyperref[auth:a845]{Mehmet A. Begen}, \hyperref[auth:a846]{G. Zhang} & Integrated Order Acceptance and Resource Decisions Under Uncertainty: Robust and Stochastic Approaches & \href{../works/NaderiBZ22.pdf}{Yes} & \cite{NaderiBZ22} & 2022 & SSRN Electronic Journal & 29 & 0 & 44 & \ref{b:NaderiBZ22} & \ref{c:NaderiBZ22}\\
\rowlabel{a:NaderiBZ22a}NaderiBZ22a \href{http://dx.doi.org/10.1016/j.cor.2022.105728}{NaderiBZ22a} & \hyperref[auth:a734]{B. Naderi}, \hyperref[auth:a845]{Mehmet A. Begen}, \hyperref[auth:a847]{Gregory S. Zaric} & Type-2 integrated process-planning and scheduling problem: Reformulation and solution algorithms & \href{../works/NaderiBZ22a.pdf}{Yes} & \cite{NaderiBZ22a} & 2022 & Computers \  Operations Research & 19 & 3 & 44 & \ref{b:NaderiBZ22a} & \ref{c:NaderiBZ22a}\\
\rowlabel{a:NaderiR22}NaderiR22 \href{http://dx.doi.org/10.1287/ijoo.2021.0056}{NaderiR22} & \hyperref[auth:a734]{B. Naderi}, \hyperref[auth:a736]{V. Roshanaei} & Critical-Path-Search Logic-Based Benders Decomposition Approaches for Flexible Job Shop Scheduling & No & \cite{NaderiR22} & 2022 & INFORMS Journal on Optimization & null & 5 & 49 & No & \ref{c:NaderiR22}\\
\rowlabel{a:OrnekOS20}OrnekOS20 \href{https://ideas.repec.org/a/spr/operea/v22y2022i1d10.1007_s12351-020-00563-9.html}{OrnekOS20} & \hyperref[auth:a139]{A. {\"{O}}rnek}, \hyperref[auth:a136]{C. {\"{O}}zt{\"{u}}rk}, \hyperref[auth:a1028]{I. Sugut} & {Integer and constraint programming model formulations for flight-gate assignment problem} & \href{../works/OrnekOS20.pdf}{Yes} & \cite{OrnekOS20} & 2022 & Operational Research & 29 & 0 & 0 & \ref{b:OrnekOS20} & \ref{c:OrnekOS20}\\
\rowlabel{a:PohlAK22}PohlAK22 \href{https://doi.org/10.1016/j.ejor.2021.08.028}{PohlAK22} & \hyperref[auth:a444]{M. Pohl}, \hyperref[auth:a6]{C. Artigues}, \hyperref[auth:a445]{R. Kolisch} & Solving the time-discrete winter runway scheduling problem: {A} column generation and constraint programming approach & \href{../works/PohlAK22.pdf}{Yes} & \cite{PohlAK22} & 2022 & European Journal of Operational Research & 16 & 4 & 31 & \ref{b:PohlAK22} & \ref{c:PohlAK22}\\
\rowlabel{a:ShiYXQ22}ShiYXQ22 \href{https://doi.org/10.1080/00207543.2021.1963496}{ShiYXQ22} & \hyperref[auth:a451]{G. Shi}, \hyperref[auth:a452]{Z. Yang}, \hyperref[auth:a453]{Y. Xu}, \hyperref[auth:a454]{Y. Quan} & Solving the integrated process planning and scheduling problem using an enhanced constraint programming-based approach & No & \cite{ShiYXQ22} & 2022 & International Journal of Production Research & 18 & 2 & 45 & No & \ref{c:ShiYXQ22}\\
\rowlabel{a:SubulanC22}SubulanC22 \href{https://doi.org/10.1007/s00500-021-06399-5}{SubulanC22} & \hyperref[auth:a456]{K. Subulan}, \hyperref[auth:a457]{G. {\c{C}}akir} & Constraint programming-based transformation approach for a mixed fuzzy-stochastic resource investment project scheduling problem & \href{../works/SubulanC22.pdf}{Yes} & \cite{SubulanC22} & 2022 & Soft Comput. & 38 & 5 & 86 & \ref{b:SubulanC22} & \ref{c:SubulanC22}\\
\rowlabel{a:YunusogluY22}YunusogluY22 \href{https://doi.org/10.1080/00207543.2021.1885068}{YunusogluY22} & \hyperref[auth:a455]{P. Yunusoglu}, \hyperref[auth:a426]{Seyda Topaloglu Yildiz} & Constraint programming approach for multi-resource-constrained unrelated parallel machine scheduling problem with sequence-dependent setup times & \href{../works/YunusogluY22.pdf}{Yes} & \cite{YunusogluY22} & 2022 & International Journal of Production Research & 18 & 20 & 58 & \ref{b:YunusogluY22} & \ref{c:YunusogluY22}\\
\rowlabel{a:YuraszeckMPV22}YuraszeckMPV22 \href{http://dx.doi.org/10.3390/math10030329}{YuraszeckMPV22} & \hyperref[auth:a410]{F. Yuraszeck}, \hyperref[auth:a750]{G. Mejía}, \hyperref[auth:a751]{J. Pereira}, \hyperref[auth:a752]{M. Vilà} & A Novel Constraint Programming Decomposition Approach for the Total Flow Time Fixed Group Shop Scheduling Problem & \href{../works/YuraszeckMPV22.pdf}{Yes} & \cite{YuraszeckMPV22} & 2022 & Mathematics & 26 & 6 & 29 & \ref{b:YuraszeckMPV22} & \ref{c:YuraszeckMPV22}\\
\rowlabel{a:abs-2211-14492}abs-2211-14492 \href{https://doi.org/10.48550/arXiv.2211.14492}{abs-2211-14492} & \hyperref[auth:a402]{Y. Sun}, \hyperref[auth:a400]{S. Nguyen}, \hyperref[auth:a401]{Dhananjay R. Thiruvady}, \hyperref[auth:a473]{X. Li}, \hyperref[auth:a474]{Andreas T. Ernst}, \hyperref[auth:a475]{U. Aickelin} & Enhancing Constraint Programming via Supervised Learning for Job Shop Scheduling & \href{../works/abs-2211-14492.pdf}{Yes} & \cite{abs-2211-14492} & 2022 & CoRR & 17 & 0 & 0 & \ref{b:abs-2211-14492} & \ref{c:abs-2211-14492}\\
\rowlabel{a:AbohashimaEG21}AbohashimaEG21 \href{https://doi.org/10.1109/ACCESS.2021.3112600}{AbohashimaEG21} & \hyperref[auth:a477]{H. Abohashima}, \hyperref[auth:a478]{Amr B. Eltawil}, \hyperref[auth:a479]{Mohamed S. Gheith} & A Mathematical Programming Model and a Firefly-Based Heuristic for Real-Time Traffic Signal Scheduling With Physical Constraints & \href{../works/AbohashimaEG21.pdf}{Yes} & \cite{AbohashimaEG21} & 2021 & {IEEE} Access & 14 & 1 & 25 & \ref{b:AbohashimaEG21} & \ref{c:AbohashimaEG21}\\
\rowlabel{a:AbreuAPNM21}AbreuAPNM21 \href{http://dx.doi.org/10.1080/0305215x.2021.1957101}{AbreuAPNM21} & \hyperref[auth:a423]{Levi Ribeiro de Abreu}, \hyperref[auth:a755]{Kennedy Anderson Guimarães Araújo}, \hyperref[auth:a756]{Bruno de Athayde Prata}, \hyperref[auth:a424]{Marcelo Seido Nagano}, \hyperref[auth:a757]{João Vitor Moccellin} & A new variable neighbourhood search with a constraint programming search strategy for the open shop scheduling problem with operation repetitions & \href{../works/AbreuAPNM21.pdf}{Yes} & \cite{AbreuAPNM21} & 2021 & Engineering Optimization & 21 & 5 & 50 & \ref{b:AbreuAPNM21} & \ref{c:AbreuAPNM21}\\
\rowlabel{a:Bedhief21}Bedhief21 \href{https://api.semanticscholar.org/CorpusID:240611192}{Bedhief21} & \hyperref[auth:a754]{Asma Ouled Bedhief} & Comparing Mixed-Integer Programming and Constraint Programming Models for the Hybrid Flow Shop Scheduling Problem with Dedicated Machines & \href{../works/Bedhief21.pdf}{Yes} & \cite{Bedhief21} & 2021 & Journal Europ{\'e}en des Syst{\`e}mes Automatis{\'e}s & 7 & 0 & 0 & \ref{b:Bedhief21} & \ref{c:Bedhief21}\\
\rowlabel{a:CarlierSJP21}CarlierSJP21 \href{http://dx.doi.org/10.1080/00207543.2021.1923853}{CarlierSJP21} & \hyperref[auth:a854]{J. Carlier}, \hyperref[auth:a939]{A. Sahli}, \hyperref[auth:a940]{A. Jouglet}, \hyperref[auth:a855]{E. Pinson} & A faster checker of the energetic reasoning for the cumulative scheduling problem & No & \cite{CarlierSJP21} & 2021 & International Journal of Production Research & null & 3 & 26 & No & \ref{c:CarlierSJP21}\\
\rowlabel{a:Edis21}Edis21 \href{http://dx.doi.org/10.1016/j.cor.2020.105111}{Edis21} & \hyperref[auth:a351]{Emrah B. Edis} & Constraint programming approaches to disassembly line balancing problem with sequencing decisions & No & \cite{Edis21} & 2021 & Computers \  Operations Research & 1 & 13 & 48 & No & \ref{c:Edis21}\\
\rowlabel{a:FanXG21}FanXG21 \href{https://doi.org/10.1016/j.cor.2021.105401}{FanXG21} & \hyperref[auth:a481]{H. Fan}, \hyperref[auth:a482]{H. Xiong}, \hyperref[auth:a483]{M. Goh} & Genetic programming-based hyper-heuristic approach for solving dynamic job shop scheduling problem with extended technical precedence constraints & \href{../works/FanXG21.pdf}{Yes} & \cite{FanXG21} & 2021 & Computers \  Operations Research & 15 & 18 & 57 & \ref{b:FanXG21} & \ref{c:FanXG21}\\
\rowlabel{a:HamP21}HamP21 \href{http://dx.doi.org/10.1109/lra.2021.3056069}{HamP21} & \hyperref[auth:a758]{A. Ham}, \hyperref[auth:a759]{M. Park} & Human–Robot Task Allocation and Scheduling: Boeing 777 Case Study & No & \cite{HamP21} & 2021 & IEEE Robotics and Automation Letters & null & 13 & 26 & No & \ref{c:HamP21}\\
\rowlabel{a:HamPK21}HamPK21 \href{https://api.semanticscholar.org/CorpusID:237898414}{HamPK21} & \hyperref[auth:a758]{A. Ham}, \hyperref[auth:a759]{M. Park}, \hyperref[auth:a760]{Kyung Min Kim} & Energy-Aware Flexible Job Shop Scheduling Using Mixed Integer Programming and Constraint Programming & \href{../works/HamPK21.pdf}{Yes} & \cite{HamPK21} & 2021 & Mathematical Problems in Engineering & 12 & 6 & 46 & \ref{b:HamPK21} & \ref{c:HamPK21}\\
\rowlabel{a:HubnerGSV21}HubnerGSV21 \href{https://doi.org/10.1007/s10951-021-00682-x}{HubnerGSV21} & \hyperref[auth:a487]{F. H{\"{u}}bner}, \hyperref[auth:a488]{P. Gerhards}, \hyperref[auth:a489]{C. St{\"{u}}rck}, \hyperref[auth:a490]{R. Volk} & Solving the nuclear dismantling project scheduling problem by combining mixed-integer and constraint programming techniques and metaheuristics & \href{../works/HubnerGSV21.pdf}{Yes} & \cite{HubnerGSV21} & 2021 & Journal of Scheduling & 22 & 0 & 37 & \ref{b:HubnerGSV21} & \ref{c:HubnerGSV21}\\
\rowlabel{a:KoehlerBFFHPSSS21}KoehlerBFFHPSSS21 \href{https://doi.org/10.1007/s10601-021-09321-w}{KoehlerBFFHPSSS21} & \hyperref[auth:a104]{J. Koehler}, \hyperref[auth:a105]{J. B{\"{u}}rgler}, \hyperref[auth:a106]{U. Fontana}, \hyperref[auth:a107]{E. Fux}, \hyperref[auth:a108]{Florian A. Herzog}, \hyperref[auth:a109]{M. Pouly}, \hyperref[auth:a110]{S. Saller}, \hyperref[auth:a111]{A. Salyaeva}, \hyperref[auth:a112]{P. Scheiblechner}, \hyperref[auth:a113]{K. Waelti} & Cable tree wiring - benchmarking solvers on a real-world scheduling problem with a variety of precedence constraints & \href{../works/KoehlerBFFHPSSS21.pdf}{Yes} & \cite{KoehlerBFFHPSSS21} & 2021 & Constraints An Int. J. & 51 & 2 & 52 & \ref{b:KoehlerBFFHPSSS21} & \ref{c:KoehlerBFFHPSSS21}\\
\rowlabel{a:MengLZB21}MengLZB21 \href{http://dx.doi.org/10.1049/cim2.12005}{MengLZB21} & \hyperref[auth:a505]{L. Meng}, \hyperref[auth:a1178]{C. Lu}, \hyperref[auth:a508]{B. Zhang}, \hyperref[auth:a507]{Y. Ren}, \hyperref[auth:a509]{C. Lv}, \hyperref[auth:a1179]{H. Sang}, \hyperref[auth:a1180]{J. Li}, \hyperref[auth:a506]{C. Zhang} & Constraint programing for solving four complex flexible shop scheduling problems & No & \cite{MengLZB21} & 2021 & IET Collaborative Intelligent Manufacturing & null & 5 & 39 & No & \ref{c:MengLZB21}\\
\rowlabel{a:NaderiRBAU21}NaderiRBAU21 \href{http://dx.doi.org/10.1111/poms.13397}{NaderiRBAU21} & \hyperref[auth:a734]{B. Naderi}, \hyperref[auth:a736]{V. Roshanaei}, \hyperref[auth:a845]{Mehmet A. Begen}, \hyperref[auth:a904]{Dionne M. Aleman}, \hyperref[auth:a905]{David R. Urbach} & Increased Surgical Capacity without Additional Resources: Generalized Operating Room Planning and Scheduling & No & \cite{NaderiRBAU21} & 2021 & Production and Operations Management & null & 22 & 61 & No & \ref{c:NaderiRBAU21}\\
\rowlabel{a:PandeyS21a}PandeyS21a \href{https://doi.org/10.1007/s11227-020-03516-3}{PandeyS21a} & \hyperref[auth:a496]{V. Pandey}, \hyperref[auth:a497]{P. Saini} & Constraint programming versus heuristic approach to MapReduce scheduling problem in Hadoop {YARN} for energy minimization & \href{../works/PandeyS21a.pdf}{Yes} & \cite{PandeyS21a} & 2021 & J. Supercomput. & 29 & 3 & 32 & \ref{b:PandeyS21a} & \ref{c:PandeyS21a}\\
\rowlabel{a:QinWSLS21}QinWSLS21 \href{https://doi.org/10.1109/TASE.2019.2947398}{QinWSLS21} & \hyperref[auth:a491]{M. Qin}, \hyperref[auth:a492]{R. Wang}, \hyperref[auth:a493]{Z. Shi}, \hyperref[auth:a494]{L. Liu}, \hyperref[auth:a495]{L. Shi} & A Genetic Programming-Based Scheduling Approach for Hybrid Flow Shop With a Batch Processor and Waiting Time Constraint & \href{../works/QinWSLS21.pdf}{Yes} & \cite{QinWSLS21} & 2021 & {IEEE} Trans Autom. Sci. Eng. & 12 & 12 & 30 & \ref{b:QinWSLS21} & \ref{c:QinWSLS21}\\
\rowlabel{a:RabbaniMM21}RabbaniMM21 \href{http://dx.doi.org/10.1080/17509653.2021.1905096}{RabbaniMM21} & \hyperref[auth:a1274]{M. Rabbani}, \hyperref[auth:a520]{M. Mokhtarzadeh}, \hyperref[auth:a1275]{N. Manavizadeh} & A constraint programming approach and a hybrid of genetic and K-means algorithms to solve the p-hub location-allocation problems & No & \cite{RabbaniMM21} & 2021 & International Journal of Management Science and Engineering Management & null & 4 & 44 & No & \ref{c:RabbaniMM21}\\
\rowlabel{a:RoshanaeiN21}RoshanaeiN21 \href{http://dx.doi.org/10.1016/j.ejor.2020.12.004}{RoshanaeiN21} & \hyperref[auth:a736]{V. Roshanaei}, \hyperref[auth:a734]{B. Naderi} & Solving integrated operating room planning and scheduling: Logic-based Benders decomposition versus Branch-Price-and-Cut & No & \cite{RoshanaeiN21} & 2021 & European Journal of Operational Research & null & 15 & 44 & No & \ref{c:RoshanaeiN21}\\
\rowlabel{a:VlkHT21}VlkHT21 \href{https://doi.org/10.1016/j.cie.2021.107317}{VlkHT21} & \hyperref[auth:a313]{M. Vlk}, \hyperref[auth:a116]{Z. Hanz{\'{a}}lek}, \hyperref[auth:a480]{S. Tang} & Constraint programming approaches to joint routing and scheduling in time-sensitive networks & \href{../works/VlkHT21.pdf}{Yes} & \cite{VlkHT21} & 2021 & Computers \  Industrial Engineering & 14 & 7 & 22 & \ref{b:VlkHT21} & \ref{c:VlkHT21}\\
\rowlabel{a:ZhangYW21}ZhangYW21 \href{https://doi.org/10.1016/j.cor.2021.105282}{ZhangYW21} & \hyperref[auth:a484]{L. Zhang}, \hyperref[auth:a485]{C. Yu}, \hyperref[auth:a486]{T. N. Wong} & A graph-based constraint programming approach for the integrated process planning and scheduling problem & \href{../works/ZhangYW21.pdf}{Yes} & \cite{ZhangYW21} & 2021 & Computers \  Operations Research & 10 & 6 & 35 & \ref{b:ZhangYW21} & \ref{c:ZhangYW21}\\
\rowlabel{a:abs-2102-08778}abs-2102-08778 \href{https://arxiv.org/abs/2102.08778}{abs-2102-08778} & \hyperref[auth:a93]{Giacomo Da Col}, \hyperref[auth:a616]{E. Teppan} & Large-Scale Benchmarks for the Job Shop Scheduling Problem & \href{../works/abs-2102-08778.pdf}{Yes} & \cite{abs-2102-08778} & 2021 & CoRR & 10 & 0 & 0 & \ref{b:abs-2102-08778} & \ref{c:abs-2102-08778}\\
\rowlabel{a:AlizdehS20}AlizdehS20 \href{https://doi.org/10.1504/IJAIP.2020.106687}{AlizdehS20} & \hyperref[auth:a518]{S. Alizdeh}, \hyperref[auth:a519]{S. Saeidi} & Fuzzy project scheduling with critical path including risk and resource constraints using linear programming & No & \cite{AlizdehS20} & 2020 & Int. J. Adv. Intell. Paradigms & 14 & 1 & 0 & No & \ref{c:AlizdehS20}\\
\rowlabel{a:AntunesABD20}AntunesABD20 \href{https://doi.org/10.1142/S0218213020600076}{AntunesABD20} & \hyperref[auth:a886]{M. Antunes}, \hyperref[auth:a887]{V. Armant}, \hyperref[auth:a222]{Kenneth N. Brown}, \hyperref[auth:a888]{Daniel A. Desmond}, \hyperref[auth:a889]{G. Escamocher}, \hyperref[auth:a890]{A. George}, \hyperref[auth:a182]{D. Grimes}, \hyperref[auth:a891]{M. O'Keeffe}, \hyperref[auth:a892]{Y. Lin}, \hyperref[auth:a16]{B. O'Sullivan}, \hyperref[auth:a136]{C. {\"{O}}zt{\"{u}}rk}, \hyperref[auth:a893]{L. Quesada}, \hyperref[auth:a130]{M. Siala}, \hyperref[auth:a17]{H. Simonis}, \hyperref[auth:a834]{N. Wilson} & Assigning and Scheduling Service Visits in a Mixed Urban/Rural Setting & \href{../works/AntunesABD20.pdf}{Yes} & \cite{AntunesABD20} & 2020 & Int. J. Artif. Intell. Tools & 31 & 0 & 16 & \ref{b:AntunesABD20} & \ref{c:AntunesABD20}\\
\rowlabel{a:AstrandJZ20}AstrandJZ20 \href{https://doi.org/10.1016/j.cor.2020.105036}{AstrandJZ20} & \hyperref[auth:a74]{M. {\AA}strand}, \hyperref[auth:a75]{M. Johansson}, \hyperref[auth:a204]{A. Zanarini} & Underground mine scheduling of mobile machines using Constraint Programming and Large Neighborhood Search & \href{../works/AstrandJZ20.pdf}{Yes} & \cite{AstrandJZ20} & 2020 & Computers \  Operations Research & 13 & 16 & 24 & \ref{b:AstrandJZ20} & \ref{c:AstrandJZ20}\\
\rowlabel{a:BadicaBI20}BadicaBI20 \href{https://doi.org/10.3233/AIC-200650}{BadicaBI20} & \hyperref[auth:a502]{A. Badica}, \hyperref[auth:a503]{C. Badica}, \hyperref[auth:a504]{M. Ivanovic} & Block structured scheduling using constraint logic programming & \href{../works/BadicaBI20.pdf}{Yes} & \cite{BadicaBI20} & 2020 & {AI} Commun. & 17 & 2 & 28 & \ref{b:BadicaBI20} & \ref{c:BadicaBI20}\\
\rowlabel{a:BalochG20}BalochG20 \href{http://dx.doi.org/10.1287/trsc.2019.0928}{BalochG20} & \hyperref[auth:a1263]{G. Baloch}, \hyperref[auth:a1264]{F. Gzara} & Strategic Network Design for Parcel Delivery with Drones Under Competition & No & \cite{BalochG20} & 2020 & Transportation Science & null & 25 & 46 & No & \ref{c:BalochG20}\\
\rowlabel{a:BenediktMH20}BenediktMH20 \href{https://doi.org/10.1007/s10601-020-09317-y}{BenediktMH20} & \hyperref[auth:a114]{O. Benedikt}, \hyperref[auth:a115]{I. M{\'{o}}dos}, \hyperref[auth:a116]{Z. Hanz{\'{a}}lek} & Power of pre-processing: production scheduling with variable energy pricing and power-saving states & \href{../works/BenediktMH20.pdf}{Yes} & \cite{BenediktMH20} & 2020 & Constraints An Int. J. & 19 & 1 & 18 & \ref{b:BenediktMH20} & \ref{c:BenediktMH20}\\
\rowlabel{a:CarlierPSJ20}CarlierPSJ20 \href{http://dx.doi.org/10.1016/j.ejor.2020.03.079}{CarlierPSJ20} & \hyperref[auth:a1265]{J. Carlier}, \hyperref[auth:a1266]{E. Pinson}, \hyperref[auth:a1267]{A. Sahli}, \hyperref[auth:a1268]{A. Jouglet} & An O(n2) algorithm for time-bound adjustments for the cumulative scheduling problem & No & \cite{CarlierPSJ20} & 2020 & European Journal of Operational Research & null & 6 & 10 & No & \ref{c:CarlierPSJ20}\\
\rowlabel{a:CauwelaertDS20}CauwelaertDS20 \href{http://dx.doi.org/10.1007/s10951-019-00632-8}{CauwelaertDS20} & \hyperref[auth:a844]{Sasha Van Cauwelaert}, \hyperref[auth:a207]{C. Dejemeppe}, \hyperref[auth:a148]{P. Schaus} & An Efficient Filtering Algorithm for the Unary Resource Constraint with Transition Times and Optional Activities & \href{../works/CauwelaertDS20.pdf}{Yes} & \cite{CauwelaertDS20} & 2020 & Journal of Scheduling & 19 & 2 & 21 & \ref{b:CauwelaertDS20} & \ref{c:CauwelaertDS20}\\
\rowlabel{a:FachiniA20}FachiniA20 \href{http://dx.doi.org/10.1016/j.cie.2020.106641}{FachiniA20} & \hyperref[auth:a1039]{Ramon Faganello Fachini}, \hyperref[auth:a1040]{Vinícius Amaral Armentano} & Logic-based Benders decomposition for the heterogeneous fixed fleet vehicle routing problem with time windows & No & \cite{FachiniA20} & 2020 & Computers \  Industrial Engineering & 1 & 25 & 55 & No & \ref{c:FachiniA20}\\
\rowlabel{a:FallahiAC20}FallahiAC20 \href{https://api.semanticscholar.org/CorpusID:213449737}{FallahiAC20} & \hyperref[auth:a761]{Abdellah El Fallahi}, \hyperref[auth:a762]{El Yaakoubi Anass}, \hyperref[auth:a763]{M. Cherkaoui} & Tabu search and constraint programming-based approach for a real scheduling and routing problem & \href{../works/FallahiAC20.pdf}{Yes} & \cite{FallahiAC20} & 2020 & International Journal of Applied Management Science & 18 & 0 & 0 & \ref{b:FallahiAC20} & \ref{c:FallahiAC20}\\
\rowlabel{a:GuoHLW20}GuoHLW20 \href{http://dx.doi.org/10.1080/0305215x.2019.1699919}{GuoHLW20} & \hyperref[auth:a943]{P. Guo}, \hyperref[auth:a944]{X. He}, \hyperref[auth:a945]{Y. Luan}, \hyperref[auth:a946]{Y. Wang} & Logic-based Benders decomposition for gantry crane scheduling with transferring position constraints in a rail–road container terminal & No & \cite{GuoHLW20} & 2020 & Engineering Optimization & null & 8 & 31 & No & \ref{c:GuoHLW20}\\
\rowlabel{a:Ham20}Ham20 \href{http://dx.doi.org/10.1080/00207543.2019.1709671}{Ham20} & \hyperref[auth:a758]{A. Ham} & Transfer-robot task scheduling in job shop & No & \cite{Ham20} & 2020 & International Journal of Production Research & null & 15 & 27 & No & \ref{c:Ham20}\\
\rowlabel{a:Ham20a}Ham20a \href{http://dx.doi.org/10.1109/tase.2019.2952523}{Ham20a} & \hyperref[auth:a758]{A. Ham} & Drone-Based Material Transfer System in a Robotic Mobile Fulfillment Center & No & \cite{Ham20a} & 2020 & IEEE Transactions on Automation Science and Engineering & null & 15 & 27 & No & \ref{c:Ham20a}\\
\rowlabel{a:HauderBRPA20}HauderBRPA20 \href{http://dx.doi.org/10.1016/j.cie.2020.106857}{HauderBRPA20} & \hyperref[auth:a558]{Viktoria A. Hauder}, \hyperref[auth:a559]{A. Beham}, \hyperref[auth:a560]{S. Raggl}, \hyperref[auth:a561]{Sophie N. Parragh}, \hyperref[auth:a562]{M. Affenzeller} & Resource-constrained multi-project scheduling with activity and time flexibility & \href{../works/HauderBRPA20.pdf}{Yes} & \cite{HauderBRPA20} & 2020 & Computers \  Industrial Engineering & 14 & 14 & 46 & \ref{b:HauderBRPA20} & \ref{c:HauderBRPA20}\\
\rowlabel{a:LunardiBLRV20}LunardiBLRV20 \href{https://doi.org/10.1016/j.cor.2020.105020}{LunardiBLRV20} & \hyperref[auth:a510]{Willian T. Lunardi}, \hyperref[auth:a511]{Ernesto G. Birgin}, \hyperref[auth:a118]{P. Laborie}, \hyperref[auth:a512]{D{\'{e}}bora P. Ronconi}, \hyperref[auth:a513]{H. Voos} & Mixed Integer linear programming and constraint programming models for the online printing shop scheduling problem & \href{../works/LunardiBLRV20.pdf}{Yes} & \cite{LunardiBLRV20} & 2020 & Computers \  Operations Research & 20 & 30 & 18 & \ref{b:LunardiBLRV20} & \ref{c:LunardiBLRV20}\\
\rowlabel{a:MejiaY20}MejiaY20 \href{https://doi.org/10.1016/j.ejor.2020.02.010}{MejiaY20} & \hyperref[auth:a429]{G. Mej{\'{\i}}a}, \hyperref[auth:a410]{F. Yuraszeck} & A self-tuning variable neighborhood search algorithm and an effective decoding scheme for open shop scheduling problems with travel/setup times & \href{../works/MejiaY20.pdf}{Yes} & \cite{MejiaY20} & 2020 & European Journal of Operational Research & 13 & 24 & 45 & \ref{b:MejiaY20} & \ref{c:MejiaY20}\\
\rowlabel{a:MengZRZL20}MengZRZL20 \href{https://doi.org/10.1016/j.cie.2020.106347}{MengZRZL20} & \hyperref[auth:a505]{L. Meng}, \hyperref[auth:a506]{C. Zhang}, \hyperref[auth:a507]{Y. Ren}, \hyperref[auth:a508]{B. Zhang}, \hyperref[auth:a509]{C. Lv} & Mixed-integer linear programming and constraint programming formulations for solving distributed flexible job shop scheduling problem & \href{../works/MengZRZL20.pdf}{Yes} & \cite{MengZRZL20} & 2020 & Computers \  Industrial Engineering & 13 & 100 & 62 & \ref{b:MengZRZL20} & \ref{c:MengZRZL20}\\
\rowlabel{a:MokhtarzadehTNF20}MokhtarzadehTNF20 \href{https://doi.org/10.1080/0951192X.2020.1736713}{MokhtarzadehTNF20} & \hyperref[auth:a520]{M. Mokhtarzadeh}, \hyperref[auth:a435]{R. Tavakkoli{-}Moghaddam}, \hyperref[auth:a437]{Behdin Vahedi Nouri}, \hyperref[auth:a521]{A. Farsi} & Scheduling of human-robot collaboration in assembly of printed circuit boards: a constraint programming approach & \href{../works/MokhtarzadehTNF20.pdf}{Yes} & \cite{MokhtarzadehTNF20} & 2020 & Int. J. Comput. Integr. Manuf. & 14 & 25 & 32 & \ref{b:MokhtarzadehTNF20} & \ref{c:MokhtarzadehTNF20}\\
\rowlabel{a:Polo-MejiaALB20}Polo-MejiaALB20 \href{https://doi.org/10.1080/00207543.2019.1693654}{Polo-MejiaALB20} & \hyperref[auth:a522]{O. Polo{-}Mej{\'{\i}}a}, \hyperref[auth:a6]{C. Artigues}, \hyperref[auth:a3]{P. Lopez}, \hyperref[auth:a523]{V. Basini} & Mixed-integer/linear and constraint programming approaches for activity scheduling in a nuclear research facility & \href{../works/Polo-MejiaALB20.pdf}{Yes} & \cite{Polo-MejiaALB20} & 2020 & International Journal of Production Research & 18 & 8 & 23 & \ref{b:Polo-MejiaALB20} & \ref{c:Polo-MejiaALB20}\\
\rowlabel{a:QinDCS20}QinDCS20 \href{https://doi.org/10.1016/j.ejor.2020.02.021}{QinDCS20} & \hyperref[auth:a514]{T. Qin}, \hyperref[auth:a515]{Y. Du}, \hyperref[auth:a516]{Jiang Hang Chen}, \hyperref[auth:a517]{M. Sha} & Combining mixed integer programming and constraint programming to solve the integrated scheduling problem of container handling operations of a single vessel & \href{../works/QinDCS20.pdf}{Yes} & \cite{QinDCS20} & 2020 & European Journal of Operational Research & 18 & 27 & 30 & \ref{b:QinDCS20} & \ref{c:QinDCS20}\\
\rowlabel{a:RoshanaeiBAUB20}RoshanaeiBAUB20 \href{http://dx.doi.org/10.1016/j.ijpe.2019.07.006}{RoshanaeiBAUB20} & \hyperref[auth:a736]{V. Roshanaei}, \hyperref[auth:a994]{Kyle E.C. Booth}, \hyperref[auth:a904]{Dionne M. Aleman}, \hyperref[auth:a905]{David R. Urbach}, \hyperref[auth:a89]{J. Christopher Beck} & Branch-and-check methods for multi-level operating room planning and scheduling & \href{../works/RoshanaeiBAUB20.pdf}{Yes} & \cite{RoshanaeiBAUB20} & 2020 & International Journal of Production Economics & 19 & 24 & 43 & \ref{b:RoshanaeiBAUB20} & \ref{c:RoshanaeiBAUB20}\\
\rowlabel{a:SacramentoSP20}SacramentoSP20 \href{https://doi.org/10.1007/s43069-020-00036-x}{SacramentoSP20} & \hyperref[auth:a524]{D. Sacramento}, \hyperref[auth:a85]{C. Solnon}, \hyperref[auth:a525]{D. Pisinger} & Constraint Programming and Local Search Heuristic: a Matheuristic Approach for Routing and Scheduling Feeder Vessels in Multi-terminal Ports & \href{../works/SacramentoSP20.pdf}{Yes} & \cite{SacramentoSP20} & 2020 & Oper. Res. Forum & 33 & 2 & 38 & \ref{b:SacramentoSP20} & \ref{c:SacramentoSP20}\\
\rowlabel{a:WallaceY20}WallaceY20 \href{https://doi.org/10.1007/s10601-020-09316-z}{WallaceY20} & \hyperref[auth:a117]{Mark G. Wallace}, \hyperref[auth:a19]{N. Yorke{-}Smith} & A new constraint programming model and solving for the cyclic hoist scheduling problem & \href{../works/WallaceY20.pdf}{Yes} & \cite{WallaceY20} & 2020 & Constraints An Int. J. & 19 & 5 & 18 & \ref{b:WallaceY20} & \ref{c:WallaceY20}\\
\rowlabel{a:ZarandiASC20}ZarandiASC20 \href{https://doi.org/10.1007/s10462-018-9667-6}{ZarandiASC20} & \hyperref[auth:a837]{Mohammad Hossein Fazel Zarandi}, \hyperref[auth:a838]{Ali Akbar Sadat Asl}, \hyperref[auth:a839]{S. Sotudian}, \hyperref[auth:a840]{O. Castillo} & A state of the art review of intelligent scheduling & \href{../works/ZarandiASC20.pdf}{Yes} & \cite{ZarandiASC20} & 2020 & Artif. Intell. Rev. & 93 & 55 & 445 & \ref{b:ZarandiASC20} & \ref{c:ZarandiASC20}\\
\rowlabel{a:ZouZ20}ZouZ20 \href{https://api.semanticscholar.org/CorpusID:208840808}{ZouZ20} & \hyperref[auth:a764]{X. Zou}, \hyperref[auth:a765]{L. Zhang} & A constraint programming approach for scheduling repetitive projects with atypical activities considering soft logic & \href{../works/ZouZ20.pdf}{Yes} & \cite{ZouZ20} & 2020 & Automation in Construction & 10 & 18 & 48 & \ref{b:ZouZ20} & \ref{c:ZouZ20}\\
\rowlabel{a:ArkhipovBL19}ArkhipovBL19 \href{http://dx.doi.org/10.1016/j.ejor.2018.11.005}{ArkhipovBL19} & \hyperref[auth:a934]{D. Arkhipov}, \hyperref[auth:a935]{O. Battaïa}, \hyperref[auth:a936]{A. Lazarev} & An efficient pseudo-polynomial algorithm for finding a lower bound on the makespan for the Resource Constrained Project Scheduling Problem & \href{../works/ArkhipovBL19.pdf}{Yes} & \cite{ArkhipovBL19} & 2019 & European Journal of Operational Research & 10 & 12 & 24 & \ref{b:ArkhipovBL19} & \ref{c:ArkhipovBL19}\\
\rowlabel{a:ColT2019a}ColT2019a \href{http://dx.doi.org/10.4204/eptcs.306.30}{ColT2019a} & \hyperref[auth:a93]{Giacomo Da Col}, \hyperref[auth:a616]{E. Teppan} & Google vs IBM: A Constraint Solving Challenge on the Job-Shop Scheduling Problem & No & \cite{ColT2019a} & 2019 & Electronic Proceedings in Theoretical Computer Science & null & 10 & 10 & No & \ref{c:ColT2019a}\\
\rowlabel{a:EdwardsBSE19}EdwardsBSE19 \href{http://dx.doi.org/10.1080/01605682.2019.1595192}{EdwardsBSE19} & \hyperref[auth:a901]{Steven J. Edwards}, \hyperref[auth:a902]{D. Baatar}, \hyperref[auth:a903]{K. Smith-Miles}, \hyperref[auth:a474]{Andreas T. Ernst} & Symmetry breaking of identical projects in the high-multiplicity RCPSP/max & No & \cite{EdwardsBSE19} & 2019 & Journal of the Operational Research Society & null & 3 & 40 & No & \ref{c:EdwardsBSE19}\\
\rowlabel{a:EscobetPQPRA19}EscobetPQPRA19 \href{https://doi.org/10.1016/j.compchemeng.2018.08.040}{EscobetPQPRA19} & \hyperref[auth:a530]{T. Escobet}, \hyperref[auth:a531]{V. Puig}, \hyperref[auth:a532]{J. Quevedo}, \hyperref[auth:a533]{P. Pal{\`{a}}{-}Sch{\"{o}}nw{\"{a}}lder}, \hyperref[auth:a534]{J. Romera}, \hyperref[auth:a535]{W. Adelman} & Optimal batch scheduling of a multiproduct dairy process using a combined optimization/constraint programming approach & \href{../works/EscobetPQPRA19.pdf}{Yes} & \cite{EscobetPQPRA19} & 2019 & Computers \  Chemical Engineering & 10 & 17 & 18 & \ref{b:EscobetPQPRA19} & \ref{c:EscobetPQPRA19}\\
\rowlabel{a:GurEA19}GurEA19 \href{https://api.semanticscholar.org/CorpusID:88492001}{GurEA19} & \hyperref[auth:a771]{Şeyda G{\"u}r}, \hyperref[auth:a420]{T. Eren}, \hyperref[auth:a772]{Hacı Mehmet Alakaş} & Surgical Operation Scheduling with Goal Programming and Constraint Programming: A Case Study & \href{../works/GurEA19.pdf}{Yes} & \cite{GurEA19} & 2019 & Mathematics & 24 & 19 & 30 & \ref{b:GurEA19} & \ref{c:GurEA19}\\
\rowlabel{a:HechingHK19}HechingHK19 \href{http://dx.doi.org/10.1287/trsc.2018.0830}{HechingHK19} & \hyperref[auth:a1036]{A. Heching}, \hyperref[auth:a1037]{J. N. Hooker}, \hyperref[auth:a1038]{R. Kimura} & A Logic-Based Benders Approach to Home Healthcare Delivery & No & \cite{HechingHK19} & 2019 & Transportation Science & null & 35 & 29 & No & \ref{c:HechingHK19}\\
\rowlabel{a:HoundjiSW19}HoundjiSW19 \href{https://doi.org/10.1007/s10601-018-9300-y}{HoundjiSW19} & \hyperref[auth:a228]{Vinas{\'{e}}tan Ratheil Houndji}, \hyperref[auth:a148]{P. Schaus}, \hyperref[auth:a229]{Laurence A. Wolsey} & The item dependent stockingcost constraint & \href{../works/HoundjiSW19.pdf}{Yes} & \cite{HoundjiSW19} & 2019 & Constraints An Int. J. & 27 & 0 & 17 & \ref{b:HoundjiSW19} & \ref{c:HoundjiSW19}\\
\rowlabel{a:NattafDYW19}NattafDYW19 \href{https://doi.org/10.1016/j.cor.2019.03.004}{NattafDYW19} & \hyperref[auth:a81]{M. Nattaf}, \hyperref[auth:a1008]{S. Dauz{\`{e}}re{-}P{\'{e}}r{\`{e}}s}, \hyperref[auth:a1009]{C. Yugma}, \hyperref[auth:a1010]{C. Wu} & Parallel machine scheduling with time constraints on machine qualifications & \href{../works/NattafDYW19.pdf}{Yes} & \cite{NattafDYW19} & 2019 & Computers \  Operations Research & 16 & 14 & 21 & \ref{b:NattafDYW19} & \ref{c:NattafDYW19}\\
\rowlabel{a:NattafHKAL19}NattafHKAL19 \href{https://doi.org/10.1016/j.dam.2018.11.008}{NattafHKAL19} & \hyperref[auth:a81]{M. Nattaf}, \hyperref[auth:a1011]{M. Horv{\'{a}}th}, \hyperref[auth:a156]{T. Kis}, \hyperref[auth:a6]{C. Artigues}, \hyperref[auth:a3]{P. Lopez} & Polyhedral results and valid inequalities for the continuous energy-constrained scheduling problem & \href{../works/NattafHKAL19.pdf}{Yes} & \cite{NattafHKAL19} & 2019 & Discret. Appl. Math. & 16 & 5 & 12 & \ref{b:NattafHKAL19} & \ref{c:NattafHKAL19}\\
\rowlabel{a:NishikawaSTT19}NishikawaSTT19 \href{http://www.ijnc.org/index.php/ijnc/article/view/201}{NishikawaSTT19} & \hyperref[auth:a536]{H. Nishikawa}, \hyperref[auth:a537]{K. Shimada}, \hyperref[auth:a538]{I. Taniguchi}, \hyperref[auth:a539]{H. Tomiyama} & A Constraint Programming Approach to Scheduling of Malleable Tasks & \href{../works/NishikawaSTT19.pdf}{Yes} & \cite{NishikawaSTT19} & 2019 & Int. J. Netw. Comput. & 16 & 3 & 20 & \ref{b:NishikawaSTT19} & \ref{c:NishikawaSTT19}\\
\rowlabel{a:Novas19}Novas19 \href{https://doi.org/10.1016/j.cie.2019.07.011}{Novas19} & \hyperref[auth:a529]{Juan M. Novas} & Production scheduling and lot streaming at flexible job-shops environments using constraint programming & \href{../works/Novas19.pdf}{Yes} & \cite{Novas19} & 2019 & Computers \  Industrial Engineering & 13 & 30 & 29 & \ref{b:Novas19} & \ref{c:Novas19}\\
\rowlabel{a:SunTB19}SunTB19 \href{http://dx.doi.org/10.1016/j.ejor.2018.08.009}{SunTB19} & \hyperref[auth:a1221]{D. Sun}, \hyperref[auth:a1222]{L. Tang}, \hyperref[auth:a1223]{R. Baldacci} & A Benders decomposition-based framework for solving quay crane scheduling problems & No & \cite{SunTB19} & 2019 & European Journal of Operational Research & null & 31 & 29 & No & \ref{c:SunTB19}\\
\rowlabel{a:TanZWGQ19}TanZWGQ19 \href{http://dx.doi.org/10.1109/tase.2019.2894093}{TanZWGQ19} & \hyperref[auth:a1207]{Y. Tan}, \hyperref[auth:a1208]{M. Zhou}, \hyperref[auth:a1209]{Y. Wang}, \hyperref[auth:a1210]{X. Guo}, \hyperref[auth:a1211]{L. Qi} & A Hybrid MIP–CP Approach to Multistage Scheduling Problem in Continuous Casting and Hot-Rolling Processes & No & \cite{TanZWGQ19} & 2019 & IEEE Transactions on Automation Science and Engineering & null & 40 & 40 & No & \ref{c:TanZWGQ19}\\
\rowlabel{a:UnsalO19}UnsalO19 \href{http://dx.doi.org/10.1016/j.tre.2019.03.018}{UnsalO19} & \hyperref[auth:a1243]{O. Unsal}, \hyperref[auth:a352]{C. Oguz} & An exact algorithm for integrated planning of operations in dry bulk terminals & No & \cite{UnsalO19} & 2019 & Transportation Research Part E: Logistics and Transportation Review & null & 44 & 27 & No & \ref{c:UnsalO19}\\
\rowlabel{a:WariZ19}WariZ19 \href{http://dx.doi.org/10.1080/00207543.2019.1571250}{WariZ19} & \hyperref[auth:a848]{E. Wari}, \hyperref[auth:a849]{W. Zhu} & A Constraint Programming model for food processing industry: a case for an ice cream processing facility & No & \cite{WariZ19} & 2019 & International Journal of Production Research & null & 11 & 42 & No & \ref{c:WariZ19}\\
\rowlabel{a:WikarekS19}WikarekS19 \href{https://doi.org/10.1142/S2196888819500027}{WikarekS19} & \hyperref[auth:a540]{J. Wikarek}, \hyperref[auth:a541]{P. Sitek} & A Constraint-Based Declarative Programming Framework for Scheduling and Resource Allocation Problems & \href{../works/WikarekS19.pdf}{Yes} & \cite{WikarekS19} & 2019 & Vietnam. J. Comput. Sci. & 22 & 0 & 11 & \ref{b:WikarekS19} & \ref{c:WikarekS19}\\
\rowlabel{a:YounespourAKE19}YounespourAKE19 \href{https://api.semanticscholar.org/CorpusID:208103305}{YounespourAKE19} & \hyperref[auth:a766]{M. Younespour}, \hyperref[auth:a767]{A. Atighehchian}, \hyperref[auth:a768]{K. Kianfar}, \hyperref[auth:a769]{Ehsan Tarkesh Esfahani} & Using mixed integer programming and constraint programming for operating rooms scheduling with modified block strategy & \href{../works/YounespourAKE19.pdf}{Yes} & \cite{YounespourAKE19} & 2019 & Operations research for health care & 11 & 7 & 15 & \ref{b:YounespourAKE19} & \ref{c:YounespourAKE19}\\
\rowlabel{a:abs-1901-07914}abs-1901-07914 \href{http://arxiv.org/abs/1901.07914}{abs-1901-07914} & \hyperref[auth:a545]{Jan Kristof Behrens}, \hyperref[auth:a546]{R. Lange}, \hyperref[auth:a547]{M. Mansouri} & A Constraint Programming Approach to Simultaneous Task Allocation and Motion Scheduling for Industrial Dual-Arm Manipulation Tasks & \href{../works/abs-1901-07914.pdf}{Yes} & \cite{abs-1901-07914} & 2019 & CoRR & 8 & 0 & 0 & \ref{b:abs-1901-07914} & \ref{c:abs-1901-07914}\\
\rowlabel{a:abs-1902-01193}abs-1902-01193 \href{http://arxiv.org/abs/1902.01193}{abs-1902-01193} & \hyperref[auth:a556]{O. M. Alade}, \hyperref[auth:a557]{A. O. Amusat} & Solving Nurse Scheduling Problem Using Constraint Programming Technique & \href{../works/abs-1902-01193.pdf}{Yes} & \cite{abs-1902-01193} & 2019 & CoRR & 9 & 0 & 0 & \ref{b:abs-1902-01193} & \ref{c:abs-1902-01193}\\
\rowlabel{a:abs-1902-09244}abs-1902-09244 \href{http://arxiv.org/abs/1902.09244}{abs-1902-09244} & \hyperref[auth:a558]{Viktoria A. Hauder}, \hyperref[auth:a559]{A. Beham}, \hyperref[auth:a560]{S. Raggl}, \hyperref[auth:a561]{Sophie N. Parragh}, \hyperref[auth:a562]{M. Affenzeller} & On constraint programming for a new flexible project scheduling problem with resource constraints & \href{../works/abs-1902-09244.pdf}{Yes} & \cite{abs-1902-09244} & 2019 & CoRR & 62 & 0 & 0 & \ref{b:abs-1902-09244} & \ref{c:abs-1902-09244}\\
\rowlabel{a:abs-1911-04766}abs-1911-04766 \href{http://arxiv.org/abs/1911.04766}{abs-1911-04766} & \hyperref[auth:a77]{T. Geibinger}, \hyperref[auth:a80]{F. Mischek}, \hyperref[auth:a45]{N. Musliu} & Investigating Constraint Programming and Hybrid Methods for Real World Industrial Test Laboratory Scheduling & \href{../works/abs-1911-04766.pdf}{Yes} & \cite{abs-1911-04766} & 2019 & CoRR & 16 & 0 & 0 & \ref{b:abs-1911-04766} & \ref{c:abs-1911-04766}\\
\rowlabel{a:BaptisteB18}BaptisteB18 \href{https://doi.org/10.1016/j.dam.2017.05.001}{BaptisteB18} & \hyperref[auth:a163]{P. Baptiste}, \hyperref[auth:a712]{N. Bonifas} & Redundant cumulative constraints to compute preemptive bounds & \href{../works/BaptisteB18.pdf}{Yes} & \cite{BaptisteB18} & 2018 & Discret. Appl. Math. & 10 & 3 & 13 & \ref{b:BaptisteB18} & \ref{c:BaptisteB18}\\
\rowlabel{a:BorghesiBLMB18}BorghesiBLMB18 \href{https://doi.org/10.1016/j.suscom.2018.05.007}{BorghesiBLMB18} & \hyperref[auth:a231]{A. Borghesi}, \hyperref[auth:a230]{A. Bartolini}, \hyperref[auth:a143]{M. Lombardi}, \hyperref[auth:a144]{M. Milano}, \hyperref[auth:a247]{L. Benini} & Scheduling-based power capping in high performance computing systems & \href{../works/BorghesiBLMB18.pdf}{Yes} & \cite{BorghesiBLMB18} & 2018 & Sustain. Comput. Informatics Syst. & 13 & 11 & 22 & \ref{b:BorghesiBLMB18} & \ref{c:BorghesiBLMB18}\\
\rowlabel{a:BukchinR18}BukchinR18 \href{http://dx.doi.org/10.1016/j.omega.2017.06.008}{BukchinR18} & \hyperref[auth:a1205]{Y. Bukchin}, \hyperref[auth:a1206]{T. Raviv} & Constraint programming for solving various assembly line balancing problems & No & \cite{BukchinR18} & 2018 & Omega & null & 66 & 29 & No & \ref{c:BukchinR18}\\
\rowlabel{a:CauwelaertLS18}CauwelaertLS18 \href{https://doi.org/10.1007/s10601-017-9277-y}{CauwelaertLS18} & \hyperref[auth:a206]{Sascha Van Cauwelaert}, \hyperref[auth:a143]{M. Lombardi}, \hyperref[auth:a148]{P. Schaus} & How efficient is a global constraint in practice? - {A} fair experimental framework & \href{../works/CauwelaertLS18.pdf}{Yes} & \cite{CauwelaertLS18} & 2018 & Constraints An Int. J. & 36 & 2 & 39 & \ref{b:CauwelaertLS18} & \ref{c:CauwelaertLS18}\\
\rowlabel{a:FahimiOQ18}FahimiOQ18 \href{https://doi.org/10.1007/s10601-018-9282-9}{FahimiOQ18} & \hyperref[auth:a122]{H. Fahimi}, \hyperref[auth:a52]{Y. Ouellet}, \hyperref[auth:a37]{C. Quimper} & Linear-time filtering algorithms for the disjunctive constraint and a quadratic filtering algorithm for the cumulative not-first not-last & \href{../works/FahimiOQ18.pdf}{Yes} & \cite{FahimiOQ18} & 2018 & Constraints An Int. J. & 22 & 2 & 20 & \ref{b:FahimiOQ18} & \ref{c:FahimiOQ18}\\
\rowlabel{a:GedikKEK18}GedikKEK18 \href{https://doi.org/10.1016/j.cie.2018.05.014}{GedikKEK18} & \hyperref[auth:a568]{R. Gedik}, \hyperref[auth:a569]{D. Kalathia}, \hyperref[auth:a570]{G. Egilmez}, \hyperref[auth:a571]{E. Kirac} & A constraint programming approach for solving unrelated parallel machine scheduling problem & \href{../works/GedikKEK18.pdf}{Yes} & \cite{GedikKEK18} & 2018 & Computers \  Industrial Engineering & 11 & 43 & 22 & \ref{b:GedikKEK18} & \ref{c:GedikKEK18}\\
\rowlabel{a:GokgurHO18}GokgurHO18 \href{https://doi.org/10.1080/00207543.2017.1421781}{GokgurHO18} & \hyperref[auth:a577]{B. G{\"{o}}kg{\"{u}}r}, \hyperref[auth:a138]{B. Hnich}, \hyperref[auth:a578]{S. {\"{O}}zpeynirci} & Parallel machine scheduling with tool loading: a constraint programming approach & \href{../works/GokgurHO18.pdf}{Yes} & \cite{GokgurHO18} & 2018 & International Journal of Production Research & 17 & 31 & 43 & \ref{b:GokgurHO18} & \ref{c:GokgurHO18}\\
\rowlabel{a:GoldwaserS18}GoldwaserS18 \href{https://doi.org/10.1613/jair.1.11268}{GoldwaserS18} & \hyperref[auth:a194]{A. Goldwaser}, \hyperref[auth:a125]{A. Schutt} & Optimal Torpedo Scheduling & \href{../works/GoldwaserS18.pdf}{Yes} & \cite{GoldwaserS18} & 2018 & J. Artif. Intell. Res. & 32 & 8 & 0 & \ref{b:GoldwaserS18} & \ref{c:GoldwaserS18}\\
\rowlabel{a:GombolayWS18}GombolayWS18 \href{http://dx.doi.org/10.1109/tro.2018.2795034}{GombolayWS18} & \hyperref[auth:a931]{Matthew C. Gombolay}, \hyperref[auth:a932]{Ronald J. Wilcox}, \hyperref[auth:a933]{Julie A. Shah} & Fast Scheduling of Robot Teams Performing Tasks With Temporospatial Constraints & \href{../works/GombolayWS18.pdf}{Yes} & \cite{GombolayWS18} & 2018 & IEEE Transactions on Robotics & 20 & 71 & 75 & \ref{b:GombolayWS18} & \ref{c:GombolayWS18}\\
\rowlabel{a:Ham18}Ham18 \href{http://dx.doi.org/10.1016/j.trc.2018.03.025}{Ham18} & \hyperref[auth:a778]{Andy M. Ham} & Integrated scheduling of m-truck,  m-drone,  and m-depot constrained by time-window,  drop-pickup,  and m-visit using constraint programming & \href{../works/Ham18.pdf}{Yes} & \cite{Ham18} & 2018 & Transportation Research Part C: Emerging Technologies & 14 & 164 & 14 & \ref{b:Ham18} & \ref{c:Ham18}\\
\rowlabel{a:Ham18a}Ham18a \href{http://dx.doi.org/10.1109/tsm.2017.2768899}{Ham18a} & \hyperref[auth:a758]{A. Ham} & Scheduling of Dual Resource Constrained Lithography Production: Using CP and MIP/CP & \href{../works/Ham18a.pdf}{Yes} & \cite{Ham18a} & 2018 & IEEE Transactions on Semiconductor Manufacturing & 10 & 20 & 21 & \ref{b:Ham18a} & \ref{c:Ham18a}\\
\rowlabel{a:KreterSSZ18}KreterSSZ18 \href{https://doi.org/10.1016/j.ejor.2017.10.014}{KreterSSZ18} & \hyperref[auth:a124]{S. Kreter}, \hyperref[auth:a125]{A. Schutt}, \hyperref[auth:a126]{Peter J. Stuckey}, \hyperref[auth:a800]{J. Zimmermann} & Mixed-integer linear programming and constraint programming formulations for solving resource availability cost problems & \href{../works/KreterSSZ18.pdf}{Yes} & \cite{KreterSSZ18} & 2018 & European Journal of Operational Research & 15 & 25 & 31 & \ref{b:KreterSSZ18} & \ref{c:KreterSSZ18}\\
\rowlabel{a:LaborieRSV18}LaborieRSV18 \href{https://doi.org/10.1007/s10601-018-9281-x}{LaborieRSV18} & \hyperref[auth:a118]{P. Laborie}, \hyperref[auth:a119]{J. Rogerie}, \hyperref[auth:a120]{P. Shaw}, \hyperref[auth:a121]{P. Vil{\'{\i}}m} & {IBM} {ILOG} {CP} optimizer for scheduling - 20+ years of scheduling with constraints at {IBM/ILOG} & \href{../works/LaborieRSV18.pdf}{Yes} & \cite{LaborieRSV18} & 2018 & Constraints An Int. J. & 41 & 148 & 35 & \ref{b:LaborieRSV18} & \ref{c:LaborieRSV18}\\
\rowlabel{a:PourDERB18}PourDERB18 \href{https://doi.org/10.1016/j.ejor.2017.08.033}{PourDERB18} & \hyperref[auth:a572]{Shahrzad M. Pour}, \hyperref[auth:a573]{John H. Drake}, \hyperref[auth:a574]{Lena Secher Ejlertsen}, \hyperref[auth:a575]{Kourosh Marjani Rasmussen}, \hyperref[auth:a576]{Edmund K. Burke} & A hybrid Constraint Programming/Mixed Integer Programming framework for the preventive signaling maintenance crew scheduling problem & \href{../works/PourDERB18.pdf}{Yes} & \cite{PourDERB18} & 2018 & European Journal of Operational Research & 12 & 41 & 13 & \ref{b:PourDERB18} & \ref{c:PourDERB18}\\
\rowlabel{a:ShinBBHO18}ShinBBHO18 \href{https://doi.org/10.1109/TSMC.2017.2681623}{ShinBBHO18} & \hyperref[auth:a581]{Seung Yeob Shin}, \hyperref[auth:a582]{Y. Brun}, \hyperref[auth:a583]{H. Balasubramanian}, \hyperref[auth:a584]{Philip L. Henneman}, \hyperref[auth:a585]{Leon J. Osterweil} & Discrete-Event Simulation and Integer Linear Programming for Constraint-Aware Resource Scheduling & \href{../works/ShinBBHO18.pdf}{Yes} & \cite{ShinBBHO18} & 2018 & {IEEE} Trans. Syst. Man Cybern. Syst. & 16 & 9 & 31 & \ref{b:ShinBBHO18} & \ref{c:ShinBBHO18}\\
\rowlabel{a:TangLWSK18}TangLWSK18 \href{https://doi.org/10.1111/mice.12277}{TangLWSK18} & \hyperref[auth:a563]{Y. Tang}, \hyperref[auth:a564]{R. Liu}, \hyperref[auth:a565]{F. Wang}, \hyperref[auth:a566]{Q. Sun}, \hyperref[auth:a567]{Amr A. Kandil} & Scheduling Optimization of Linear Schedule with Constraint Programming & \href{../works/TangLWSK18.pdf}{Yes} & \cite{TangLWSK18} & 2018 & Comput. Aided Civ. Infrastructure Eng. & 28 & 24 & 76 & \ref{b:TangLWSK18} & \ref{c:TangLWSK18}\\
\rowlabel{a:TranPZLDB18}TranPZLDB18 \href{https://doi.org/10.1007/s10951-017-0537-x}{TranPZLDB18} & \hyperref[auth:a807]{Tony T. Tran}, \hyperref[auth:a808]{M. Padmanabhan}, \hyperref[auth:a809]{Peter Yun Zhang}, \hyperref[auth:a810]{H. Li}, \hyperref[auth:a811]{Douglas G. Down}, \hyperref[auth:a89]{J. Christopher Beck} & Multi-stage resource-aware scheduling for data centers with heterogeneous servers & \href{../works/TranPZLDB18.pdf}{Yes} & \cite{TranPZLDB18} & 2018 & Journal of Scheduling & 17 & 8 & 26 & \ref{b:TranPZLDB18} & \ref{c:TranPZLDB18}\\
\rowlabel{a:ZhangW18}ZhangW18 \href{https://doi.org/10.1109/TEM.2017.2785774}{ZhangW18} & \hyperref[auth:a579]{S. Zhang}, \hyperref[auth:a580]{S. Wang} & Flexible Assembly Job-Shop Scheduling With Sequence-Dependent Setup Times and Part Sharing in a Dynamic Environment: Constraint Programming Model, Mixed-Integer Programming Model, and Dispatching Rules & \href{../works/ZhangW18.pdf}{Yes} & \cite{ZhangW18} & 2018 & {IEEE} Trans. Engineering Management & 18 & 49 & 28 & \ref{b:ZhangW18} & \ref{c:ZhangW18}\\
\rowlabel{a:EmeretlisTAV17}EmeretlisTAV17 \href{http://dx.doi.org/10.1145/3133219}{EmeretlisTAV17} & \hyperref[auth:a1253]{A. Emeretlis}, \hyperref[auth:a1254]{G. Theodoridis}, \hyperref[auth:a1255]{P. Alefragis}, \hyperref[auth:a1256]{N. Voros} & Static Mapping of Applications on Heterogeneous Multi-Core Platforms Combining Logic-Based Benders Decomposition with Integer Linear Programming & No & \cite{EmeretlisTAV17} & 2017 & ACM Transactions on Design Automation of Electronic Systems & null & 4 & 42 & No & \ref{c:EmeretlisTAV17}\\
\rowlabel{a:GedikKBR17}GedikKBR17 \href{http://dx.doi.org/10.1016/j.cie.2017.03.017}{GedikKBR17} & \hyperref[auth:a568]{R. Gedik}, \hyperref[auth:a571]{E. Kirac}, \hyperref[auth:a1175]{Ashlea Bennet Milburn}, \hyperref[auth:a1176]{C. Rainwater} & A constraint programming approach for the team orienteering problem with time windows & No & \cite{GedikKBR17} & 2017 & Computers \  Industrial Engineering & null & 20 & 32 & No & \ref{c:GedikKBR17}\\
\rowlabel{a:GomesM17}GomesM17 \href{http://dx.doi.org/10.1155/2017/9452762}{GomesM17} & \hyperref[auth:a978]{Francisco Regis Abreu Gomes}, \hyperref[auth:a979]{Geraldo Robson Mateus} & Improved Combinatorial Benders Decomposition for a Scheduling Problem with Unrelated Parallel Machines & \href{../works/GomesM17.pdf}{Yes} & \cite{GomesM17} & 2017 & Journal of Applied Mathematics & 11 & 1 & 43 & \ref{b:GomesM17} & \ref{c:GomesM17}\\
\rowlabel{a:HamFC17}HamFC17 \href{http://dx.doi.org/10.1109/tsm.2017.2740340}{HamFC17} & \hyperref[auth:a758]{A. Ham}, \hyperref[auth:a1227]{John W. Fowler}, \hyperref[auth:a884]{E. Cakici} & Constraint Programming Approach for Scheduling Jobs With Release Times,  Non-Identical Sizes,  and Incompatible Families on Parallel Batching Machines & No & \cite{HamFC17} & 2017 & IEEE Transactions on Semiconductor Manufacturing & null & 21 & 28 & No & \ref{c:HamFC17}\\
\rowlabel{a:HookerH17}HookerH17 \href{http://dx.doi.org/10.1007/s10601-017-9280-3}{HookerH17} & \hyperref[auth:a161]{John N. Hooker}, \hyperref[auth:a841]{Willem-Jan van Hoeve} & Constraint programming and operations research & \href{../works/HookerH17.pdf}{Yes} & \cite{HookerH17} & 2017 & Constraints An Int. J. & 24 & 12 & 189 & \ref{b:HookerH17} & \ref{c:HookerH17}\\
\rowlabel{a:KreterSS17}KreterSS17 \href{https://doi.org/10.1007/s10601-016-9266-6}{KreterSS17} & \hyperref[auth:a124]{S. Kreter}, \hyperref[auth:a125]{A. Schutt}, \hyperref[auth:a126]{Peter J. Stuckey} & Using constraint programming for solving RCPSP/max-cal & \href{../works/KreterSS17.pdf}{Yes} & \cite{KreterSS17} & 2017 & Constraints An Int. J. & 31 & 15 & 20 & \ref{b:KreterSS17} & \ref{c:KreterSS17}\\
\rowlabel{a:NattafAL17}NattafAL17 \href{https://doi.org/10.1007/s10601-017-9271-4}{NattafAL17} & \hyperref[auth:a81]{M. Nattaf}, \hyperref[auth:a6]{C. Artigues}, \hyperref[auth:a3]{P. Lopez} & Cumulative scheduling with variable task profiles and concave piecewise linear processing rate functions & \href{../works/NattafAL17.pdf}{Yes} & \cite{NattafAL17} & 2017 & Constraints An Int. J. & 18 & 5 & 10 & \ref{b:NattafAL17} & \ref{c:NattafAL17}\\
\rowlabel{a:RoshanaeiLAU17}RoshanaeiLAU17 \href{http://dx.doi.org/10.1016/j.ejor.2016.08.024}{RoshanaeiLAU17} & \hyperref[auth:a736]{V. Roshanaei}, \hyperref[auth:a937]{C. Luong}, \hyperref[auth:a904]{Dionne M. Aleman}, \hyperref[auth:a938]{D. Urbach} & Propagating logic-based Benders' decomposition approaches for distributed operating room scheduling & \href{../works/RoshanaeiLAU17.pdf}{Yes} & \cite{RoshanaeiLAU17} & 2017 & European Journal of Operational Research & 17 & 61 & 46 & \ref{b:RoshanaeiLAU17} & \ref{c:RoshanaeiLAU17}\\
\rowlabel{a:RoshanaeiLAU17a}RoshanaeiLAU17a \href{http://dx.doi.org/10.1287/ijoc.2017.0745}{RoshanaeiLAU17a} & \hyperref[auth:a736]{V. Roshanaei}, \hyperref[auth:a937]{C. Luong}, \hyperref[auth:a904]{Dionne M. Aleman}, \hyperref[auth:a905]{David R. Urbach} & Collaborative Operating Room Planning and Scheduling & No & \cite{RoshanaeiLAU17a} & 2017 & INFORMS Journal on Computing & null & 54 & 42 & No & \ref{c:RoshanaeiLAU17a}\\
\rowlabel{a:SchnellH17}SchnellH17 \href{http://dx.doi.org/10.1016/j.orp.2017.01.002}{SchnellH17} & \hyperref[auth:a963]{A. Schnell}, \hyperref[auth:a964]{Richard F. Hartl} & On the generalization of constraint programming and boolean satisfiability solving techniques to schedule a resource-constrained project consisting of multi-mode jobs & No & \cite{SchnellH17} & 2017 & Operations Research Perspectives & null & 12 & 20 & No & \ref{c:SchnellH17}\\
\rowlabel{a:TranVNB17}TranVNB17 \href{https://doi.org/10.1613/jair.5306}{TranVNB17} & \hyperref[auth:a807]{Tony T. Tran}, \hyperref[auth:a812]{Tiago Stegun Vaquero}, \hyperref[auth:a209]{G. Nejat}, \hyperref[auth:a89]{J. Christopher Beck} & Robots in Retirement Homes: Applying Off-the-Shelf Planning and Scheduling to a Team of Assistive Robots & \href{../works/TranVNB17.pdf}{Yes} & \cite{TranVNB17} & 2017 & J. Artif. Intell. Res. & 68 & 12 & 0 & \ref{b:TranVNB17} & \ref{c:TranVNB17}\\
\rowlabel{a:BlomPS16}BlomPS16 \href{https://doi.org/10.1287/mnsc.2015.2284}{BlomPS16} & \hyperref[auth:a803]{Michelle L. Blom}, \hyperref[auth:a327]{Adrian R. Pearce}, \hyperref[auth:a126]{Peter J. Stuckey} & A Decomposition-Based Algorithm for the Scheduling of Open-Pit Networks Over Multiple Time Periods & \href{../works/BlomPS16.pdf}{Yes} & \cite{BlomPS16} & 2016 & Manag. Sci. & 26 & 20 & 36 & \ref{b:BlomPS16} & \ref{c:BlomPS16}\\
\rowlabel{a:Bonfietti16}Bonfietti16 \href{https://doi.org/10.3233/IA-160095}{Bonfietti16} & \hyperref[auth:a203]{A. Bonfietti} & A constraint programming scheduling solver for the MPOpt programming environment & \href{../works/Bonfietti16.pdf}{Yes} & \cite{Bonfietti16} & 2016 & Intelligenza Artificiale & 13 & 0 & 19 & \ref{b:Bonfietti16} & \ref{c:Bonfietti16}\\
\rowlabel{a:BoothTNB16}BoothTNB16 \href{http://dx.doi.org/10.1109/lra.2016.2522096}{BoothTNB16} & \hyperref[auth:a208]{Kyle E. C. Booth}, \hyperref[auth:a807]{Tony T. Tran}, \hyperref[auth:a209]{G. Nejat}, \hyperref[auth:a89]{J. Christopher Beck} & Mixed-Integer and Constraint Programming Techniques for Mobile Robot Task Planning & No & \cite{BoothTNB16} & 2016 & IEEE Robotics and Automation Letters & null & 27 & 21 & No & \ref{c:BoothTNB16}\\
\rowlabel{a:BridiBLMB16}BridiBLMB16 \href{https://doi.org/10.1109/TPDS.2016.2516997}{BridiBLMB16} & \hyperref[auth:a232]{T. Bridi}, \hyperref[auth:a230]{A. Bartolini}, \hyperref[auth:a143]{M. Lombardi}, \hyperref[auth:a144]{M. Milano}, \hyperref[auth:a247]{L. Benini} & A Constraint Programming Scheduler for Heterogeneous High-Performance Computing Machines & \href{../works/BridiBLMB16.pdf}{Yes} & \cite{BridiBLMB16} & 2016 & {IEEE} Trans. Parallel Distributed Syst. & 14 & 17 & 22 & \ref{b:BridiBLMB16} & \ref{c:BridiBLMB16}\\
\rowlabel{a:CireCH16}CireCH16 \href{http://dx.doi.org/10.1017/s0269888916000254}{CireCH16} & \hyperref[auth:a158]{Andr{\'{e}} A. Cir{\'{e}}}, \hyperref[auth:a340]{E. Coban}, \hyperref[auth:a161]{John N. Hooker} & Logic-based Benders decomposition for planning and scheduling: a computational analysis & \href{../works/CireCH16.pdf}{Yes} & \cite{CireCH16} & 2016 & The Knowledge Engineering Review & 12 & 15 & 21 & \ref{b:CireCH16} & \ref{c:CireCH16}\\
\rowlabel{a:DoulabiRP16}DoulabiRP16 \href{https://doi.org/10.1287/ijoc.2015.0686}{DoulabiRP16} & \hyperref[auth:a335]{Seyed Hossein Hashemi Doulabi}, \hyperref[auth:a331]{L. Rousseau}, \hyperref[auth:a8]{G. Pesant} & A Constraint-Programming-Based Branch-and-Price-and-Cut Approach for Operating Room Planning and Scheduling & \href{../works/DoulabiRP16.pdf}{Yes} & \cite{DoulabiRP16} & 2016 & INFORMS Journal on Computing & 17 & 56 & 28 & \ref{b:DoulabiRP16} & \ref{c:DoulabiRP16}\\
\rowlabel{a:HamC16}HamC16 \href{http://dx.doi.org/10.1016/j.cie.2016.11.001}{HamC16} & \hyperref[auth:a778]{Andy M. Ham}, \hyperref[auth:a884]{E. Cakici} & Flexible job shop scheduling problem with parallel batch processing machines: MIP and CP approaches & \href{../works/HamC16.pdf}{Yes} & \cite{HamC16} & 2016 & Computers \  Industrial Engineering & 6 & 50 & 26 & \ref{b:HamC16} & \ref{c:HamC16}\\
\rowlabel{a:HebrardHJMPV16}HebrardHJMPV16 \href{https://doi.org/10.1016/j.dam.2016.07.003}{HebrardHJMPV16} & \hyperref[auth:a1]{E. Hebrard}, \hyperref[auth:a54]{M. Huguet}, \hyperref[auth:a799]{N. Jozefowiez}, \hyperref[auth:a795]{A. Maillard}, \hyperref[auth:a21]{C. Pralet}, \hyperref[auth:a174]{G. Verfaillie} & Approximation of the parallel machine scheduling problem with additional unit resources & \href{../works/HebrardHJMPV16.pdf}{Yes} & \cite{HebrardHJMPV16} & 2016 & Discret. Appl. Math. & 10 & 9 & 8 & \ref{b:HebrardHJMPV16} & \ref{c:HebrardHJMPV16}\\
\rowlabel{a:KuB16}KuB16 \href{https://doi.org/10.1016/j.cor.2016.04.006}{KuB16} & \hyperref[auth:a336]{W. Ku}, \hyperref[auth:a89]{J. Christopher Beck} & Mixed Integer Programming models for job shop scheduling: {A} computational analysis & \href{../works/KuB16.pdf}{Yes} & \cite{KuB16} & 2016 & Computers \  Operations Research & 9 & 119 & 17 & \ref{b:KuB16} & \ref{c:KuB16}\\
\rowlabel{a:NattafALR16}NattafALR16 \href{https://doi.org/10.1007/s00291-015-0423-x}{NattafALR16} & \hyperref[auth:a81]{M. Nattaf}, \hyperref[auth:a6]{C. Artigues}, \hyperref[auth:a3]{P. Lopez}, \hyperref[auth:a993]{D. Rivreau} & Energetic reasoning and mixed-integer linear programming for scheduling with a continuous resource and linear efficiency functions & \href{../works/NattafALR16.pdf}{Yes} & \cite{NattafALR16} & 2016 & {OR} Spectr. & 34 & 10 & 15 & \ref{b:NattafALR16} & \ref{c:NattafALR16}\\
\rowlabel{a:NovaraNH16}NovaraNH16 \href{https://doi.org/10.1016/j.compchemeng.2016.04.030}{NovaraNH16} & \hyperref[auth:a595]{Franco M. Novara}, \hyperref[auth:a529]{Juan M. Novas}, \hyperref[auth:a596]{Gabriela P. Henning} & A novel constraint programming model for large-scale scheduling problems in multiproduct multistage batch plants: Limited resources and campaign-based operation & \href{../works/NovaraNH16.pdf}{Yes} & \cite{NovaraNH16} & 2016 & Computers \  Chemical Engineering & 17 & 18 & 31 & \ref{b:NovaraNH16} & \ref{c:NovaraNH16}\\
\rowlabel{a:OrnekO16}OrnekO16 \href{https://journals.sfu.ca/ijietap/index.php/ijie/article/view/1930}{OrnekO16} & \hyperref[auth:a139]{A. {\"{O}}rnek}, \hyperref[auth:a136]{C. {\"{O}}zt{\"{u}}rk} & Optimisation and Constraint Based Heuristic Methods for Advanced Planning and Scheduling Systems & \href{../works/OrnekO16.pdf}{Yes} & \cite{OrnekO16} & 2016 & International Journal of Industrial Engineering: Theory, Applications and Practice & 25 & 0 & 0 & \ref{b:OrnekO16} & \ref{c:OrnekO16}\\
\rowlabel{a:QinDS16}QinDS16 \href{http://dx.doi.org/10.1016/j.tre.2016.01.007}{QinDS16} & \hyperref[auth:a514]{T. Qin}, \hyperref[auth:a515]{Y. Du}, \hyperref[auth:a517]{M. Sha} & Evaluating the solution performance of IP and CP for berth allocation with time-varying water depth & No & \cite{QinDS16} & 2016 & Transportation Research Part E: Logistics and Transportation Review & null & 17 & 40 & No & \ref{c:QinDS16}\\
\rowlabel{a:RiiseML16}RiiseML16 \href{http://dx.doi.org/10.1016/j.ejor.2016.06.015}{RiiseML16} & \hyperref[auth:a1082]{A. Riise}, \hyperref[auth:a1083]{C. Mannino}, \hyperref[auth:a1084]{L. Lamorgese} & Recursive logic-based Benders' decomposition for multi-mode outpatient scheduling & No & \cite{RiiseML16} & 2016 & European Journal of Operational Research & null & 27 & 29 & No & \ref{c:RiiseML16}\\
\rowlabel{a:TranAB16}TranAB16 \href{https://doi.org/10.1287/ijoc.2015.0666}{TranAB16} & \hyperref[auth:a807]{Tony T. Tran}, \hyperref[auth:a815]{A. Araujo}, \hyperref[auth:a89]{J. Christopher Beck} & Decomposition Methods for the Parallel Machine Scheduling Problem with Setups & \href{../works/TranAB16.pdf}{Yes} & \cite{TranAB16} & 2016 & INFORMS Journal on Computing & 13 & 72 & 28 & \ref{b:TranAB16} & \ref{c:TranAB16}\\
\rowlabel{a:ZarandiKS16}ZarandiKS16 \href{https://doi.org/10.1007/s10845-013-0860-9}{ZarandiKS16} & \hyperref[auth:a597]{M. H. Fazel Zarandi}, \hyperref[auth:a598]{H. Khorshidian}, \hyperref[auth:a599]{Mohsen Akbarpour Shirazi} & A constraint programming model for the scheduling of {JIT} cross-docking systems with preemption & \href{../works/ZarandiKS16.pdf}{Yes} & \cite{ZarandiKS16} & 2016 & Journal of Intelligent Manufacturing & 17 & 28 & 14 & \ref{b:ZarandiKS16} & \ref{c:ZarandiKS16}\\
\rowlabel{a:AlesioBNG15}AlesioBNG15 \href{http://dx.doi.org/10.1145/2818640}{AlesioBNG15} & \hyperref[auth:a1249]{Stefano Di Alesio}, \hyperref[auth:a238]{Lionel C. Briand}, \hyperref[auth:a237]{S. Nejati}, \hyperref[auth:a200]{A. Gotlieb} & Combining Genetic Algorithms and Constraint Programming to Support Stress Testing of Task Deadlines & No & \cite{AlesioBNG15} & 2015 & ACM Transactions on Software Engineering and Methodology & null & 13 & 51 & No & \ref{c:AlesioBNG15}\\
\rowlabel{a:BajestaniB15}BajestaniB15 \href{https://doi.org/10.1007/s10951-015-0416-2}{BajestaniB15} & \hyperref[auth:a825]{Maliheh Aramon Bajestani}, \hyperref[auth:a89]{J. Christopher Beck} & A two-stage coupled algorithm for an integrated maintenance planning and flowshop scheduling problem with deteriorating machines & \href{../works/BajestaniB15.pdf}{Yes} & \cite{BajestaniB15} & 2015 & Journal of Scheduling & 16 & 17 & 59 & \ref{b:BajestaniB15} & \ref{c:BajestaniB15}\\
\rowlabel{a:EvenSH15a}EvenSH15a \href{http://arxiv.org/abs/1505.02487}{EvenSH15a} & \hyperref[auth:a219]{C. Even}, \hyperref[auth:a125]{A. Schutt}, \hyperref[auth:a149]{Pascal Van Hentenryck} & A Constraint Programming Approach for Non-Preemptive Evacuation Scheduling & \href{../works/EvenSH15a.pdf}{Yes} & \cite{EvenSH15a} & 2015 & CoRR & 16 & 0 & 0 & \ref{b:EvenSH15a} & \ref{c:EvenSH15a}\\
\rowlabel{a:GoelSHFS15}GoelSHFS15 \href{https://doi.org/10.1016/j.ejor.2014.09.048}{GoelSHFS15} & \hyperref[auth:a600]{V. Goel}, \hyperref[auth:a601]{M. Slusky}, \hyperref[auth:a211]{Willem{-}Jan van Hoeve}, \hyperref[auth:a602]{Kevin C. Furman}, \hyperref[auth:a603]{Y. Shao} & Constraint programming for {LNG} ship scheduling and inventory management & \href{../works/GoelSHFS15.pdf}{Yes} & \cite{GoelSHFS15} & 2015 & European Journal of Operational Research & 12 & 48 & 4 & \ref{b:GoelSHFS15} & \ref{c:GoelSHFS15}\\
\rowlabel{a:GrimesH15}GrimesH15 \href{https://doi.org/10.1287/ijoc.2014.0625}{GrimesH15} & \hyperref[auth:a182]{D. Grimes}, \hyperref[auth:a1]{E. Hebrard} & Solving Variants of the Job Shop Scheduling Problem Through Conflict-Directed Search & \href{../works/GrimesH15.pdf}{Yes} & \cite{GrimesH15} & 2015 & INFORMS Journal on Computing & 17 & 12 & 41 & \ref{b:GrimesH15} & \ref{c:GrimesH15}\\
\rowlabel{a:Kameugne15}Kameugne15 \href{https://doi.org/10.1007/s10601-015-9227-5}{Kameugne15} & \hyperref[auth:a10]{R. Kameugne} & Propagation techniques of resource constraint for cumulative scheduling & \href{../works/Kameugne15.pdf}{Yes} & \cite{Kameugne15} & 2015 & Constraints An Int. J. & 2 & 0 & 0 & \ref{b:Kameugne15} & \ref{c:Kameugne15}\\
\rowlabel{a:LetortCB15}LetortCB15 \href{https://doi.org/10.1007/s10601-014-9172-8}{LetortCB15} & \hyperref[auth:a128]{A. Letort}, \hyperref[auth:a91]{M. Carlsson}, \hyperref[auth:a129]{N. Beldiceanu} & Synchronized sweep algorithms for scalable scheduling constraints & \href{../works/LetortCB15.pdf}{Yes} & \cite{LetortCB15} & 2015 & Constraints An Int. J. & 52 & 2 & 14 & \ref{b:LetortCB15} & \ref{c:LetortCB15}\\
\rowlabel{a:NattafAL15}NattafAL15 \href{https://doi.org/10.1007/s10601-015-9192-z}{NattafAL15} & \hyperref[auth:a81]{M. Nattaf}, \hyperref[auth:a6]{C. Artigues}, \hyperref[auth:a3]{P. Lopez} & A hybrid exact method for a scheduling problem with a continuous resource and energy constraints & \href{../works/NattafAL15.pdf}{Yes} & \cite{NattafAL15} & 2015 & Constraints An Int. J. & 21 & 14 & 13 & \ref{b:NattafAL15} & \ref{c:NattafAL15}\\
\rowlabel{a:OzturkTHO15}OzturkTHO15 \href{https://www.sciencedirect.com/science/article/pii/S0278612515000527}{OzturkTHO15} & \hyperref[auth:a136]{C. {\"{O}}zt{\"{u}}rk}, \hyperref[auth:a1031]{S. Tunalı}, \hyperref[auth:a138]{B. Hnich}, \hyperref[auth:a139]{A. {\"{O}}rnek} & Cyclic scheduling of flexible mixed model assembly lines with parallel stations & \href{../works/OzturkTHO15.pdf}{Yes} & \cite{OzturkTHO15} & 2015 & Journal of Manufacturing Systems & 12 & 27 & 17 & \ref{b:OzturkTHO15} & \ref{c:OzturkTHO15}\\
\rowlabel{a:SchnellH15}SchnellH15 \href{http://dx.doi.org/10.1007/s00291-015-0419-6}{SchnellH15} & \hyperref[auth:a963]{A. Schnell}, \hyperref[auth:a964]{Richard F. Hartl} & On the efficient modeling and solution of the multi-mode resource-constrained project scheduling problem with generalized precedence relations & \href{../works/SchnellH15.pdf}{Yes} & \cite{SchnellH15} & 2015 & OR Spectrum & 21 & 24 & 20 & \ref{b:SchnellH15} & \ref{c:SchnellH15}\\
\rowlabel{a:Siala15}Siala15 \href{https://doi.org/10.1007/s10601-015-9213-y}{Siala15} & \hyperref[auth:a130]{M. Siala} & Search, propagation, and learning in sequencing and scheduling problems & \href{../works/Siala15.pdf}{Yes} & \cite{Siala15} & 2015 & Constraints An Int. J. & 2 & 4 & 0 & \ref{b:Siala15} & \ref{c:Siala15}\\
\rowlabel{a:SimoninAHL15}SimoninAHL15 \href{https://doi.org/10.1007/s10601-014-9169-3}{SimoninAHL15} & \hyperref[auth:a127]{G. Simonin}, \hyperref[auth:a6]{C. Artigues}, \hyperref[auth:a1]{E. Hebrard}, \hyperref[auth:a3]{P. Lopez} & Scheduling scientific experiments for comet exploration & \href{../works/SimoninAHL15.pdf}{Yes} & \cite{SimoninAHL15} & 2015 & Constraints An Int. J. & 23 & 4 & 5 & \ref{b:SimoninAHL15} & \ref{c:SimoninAHL15}\\
\rowlabel{a:WangMD15}WangMD15 \href{https://doi.org/10.1016/j.ejor.2015.06.008}{WangMD15} & \hyperref[auth:a604]{T. Wang}, \hyperref[auth:a605]{N. Meskens}, \hyperref[auth:a606]{D. Duvivier} & Scheduling operating theatres: Mixed integer programming vs. constraint programming & \href{../works/WangMD15.pdf}{Yes} & \cite{WangMD15} & 2015 & European Journal of Operational Research & 13 & 36 & 33 & \ref{b:WangMD15} & \ref{c:WangMD15}\\
\rowlabel{a:ArtiguesL14}ArtiguesL14 \href{http://dx.doi.org/10.1007/s10951-014-0404-y}{ArtiguesL14} & \hyperref[auth:a6]{C. Artigues}, \hyperref[auth:a3]{P. Lopez} & Energetic reasoning for energy-constrained scheduling with a continuous resource & No & \cite{ArtiguesL14} & 2014 & Journal of Scheduling & null & 11 & 19 & No & \ref{c:ArtiguesL14}\\
\rowlabel{a:BlomBPS14}BlomBPS14 \href{https://doi.org/10.1287/ijoc.2013.0590}{BlomBPS14} & \hyperref[auth:a803]{Michelle L. Blom}, \hyperref[auth:a325]{Christina N. Burt}, \hyperref[auth:a327]{Adrian R. Pearce}, \hyperref[auth:a126]{Peter J. Stuckey} & A Decomposition-Based Heuristic for Collaborative Scheduling in a Network of Open-Pit Mines & \href{../works/BlomBPS14.pdf}{Yes} & \cite{BlomBPS14} & 2014 & INFORMS Journal on Computing & 19 & 15 & 47 & \ref{b:BlomBPS14} & \ref{c:BlomBPS14}\\
\rowlabel{a:BonfiettiLBM14}BonfiettiLBM14 \href{https://doi.org/10.1016/j.artint.2013.09.006}{BonfiettiLBM14} & \hyperref[auth:a203]{A. Bonfietti}, \hyperref[auth:a143]{M. Lombardi}, \hyperref[auth:a247]{L. Benini}, \hyperref[auth:a144]{M. Milano} & {CROSS} cyclic resource-constrained scheduling solver & \href{../works/BonfiettiLBM14.pdf}{Yes} & \cite{BonfiettiLBM14} & 2014 & Artificial Intelligence & 28 & 8 & 15 & \ref{b:BonfiettiLBM14} & \ref{c:BonfiettiLBM14}\\
\rowlabel{a:GrimesIOS14}GrimesIOS14 \href{https://doi.org/10.1016/j.suscom.2014.08.009}{GrimesIOS14} & \hyperref[auth:a182]{D. Grimes}, \hyperref[auth:a183]{G. Ifrim}, \hyperref[auth:a16]{B. O'Sullivan}, \hyperref[auth:a17]{H. Simonis} & Analyzing the impact of electricity price forecasting on energy cost-aware scheduling & \href{../works/GrimesIOS14.pdf}{Yes} & \cite{GrimesIOS14} & 2014 & Sustain. Comput. Informatics Syst. & 16 & 6 & 7 & \ref{b:GrimesIOS14} & \ref{c:GrimesIOS14}\\
\rowlabel{a:HarjunkoskiMBC14}HarjunkoskiMBC14 \href{http://dx.doi.org/10.1016/j.compchemeng.2013.12.001}{HarjunkoskiMBC14} & \hyperref[auth:a880]{I. Harjunkoski}, \hyperref[auth:a386]{Christos T. Maravelias}, \hyperref[auth:a949]{P. Bongers}, \hyperref[auth:a900]{Pedro M. Castro}, \hyperref[auth:a70]{S. Engell}, \hyperref[auth:a387]{Ignacio E. Grossmann}, \hyperref[auth:a161]{John N. Hooker}, \hyperref[auth:a950]{C. Méndez}, \hyperref[auth:a951]{G. Sand}, \hyperref[auth:a952]{J. Wassick} & Scope for industrial applications of production scheduling models and solution methods & \href{../works/HarjunkoskiMBC14.pdf}{Yes} & \cite{HarjunkoskiMBC14} & 2014 & Computers \  Chemical Engineering & 33 & 381 & 176 & \ref{b:HarjunkoskiMBC14} & \ref{c:HarjunkoskiMBC14}\\
\rowlabel{a:KameugneFSN14}KameugneFSN14 \href{https://doi.org/10.1007/s10601-013-9157-z}{KameugneFSN14} & \hyperref[auth:a10]{R. Kameugne}, \hyperref[auth:a131]{Laure Pauline Fotso}, \hyperref[auth:a132]{Joseph D. Scott}, \hyperref[auth:a133]{Y. Ngo{-}Kateu} & A quadratic edge-finding filtering algorithm for cumulative resource constraints & \href{../works/KameugneFSN14.pdf}{Yes} & \cite{KameugneFSN14} & 2014 & Constraints An Int. J. & 27 & 6 & 10 & \ref{b:KameugneFSN14} & \ref{c:KameugneFSN14}\\
\rowlabel{a:LaborieR14}LaborieR14 \href{http://dx.doi.org/10.1007/s10951-014-0408-7}{LaborieR14} & \hyperref[auth:a118]{P. Laborie}, \hyperref[auth:a1087]{J. Rogerie} & Temporal linear relaxation in IBM ILOG CP Optimizer & \href{../works/LaborieR14.pdf}{Yes} & \cite{LaborieR14} & 2014 & Journal of Scheduling & 10 & 17 & 13 & \ref{b:LaborieR14} & \ref{c:LaborieR14}\\
\rowlabel{a:NovasH14}NovasH14 \href{https://doi.org/10.1016/j.eswa.2013.09.026}{NovasH14} & \hyperref[auth:a529]{Juan M. Novas}, \hyperref[auth:a596]{Gabriela P. Henning} & Integrated scheduling of resource-constrained flexible manufacturing systems using constraint programming & \href{../works/NovasH14.pdf}{Yes} & \cite{NovasH14} & 2014 & Expert Syst. Appl. & 14 & 35 & 26 & \ref{b:NovasH14} & \ref{c:NovasH14}\\
\rowlabel{a:TerekhovTDB14}TerekhovTDB14 \href{https://doi.org/10.1613/jair.4278}{TerekhovTDB14} & \hyperref[auth:a826]{D. Terekhov}, \hyperref[auth:a807]{Tony T. Tran}, \hyperref[auth:a811]{Douglas G. Down}, \hyperref[auth:a89]{J. Christopher Beck} & Integrating Queueing Theory and Scheduling for Dynamic Scheduling Problems & \href{../works/TerekhovTDB14.pdf}{Yes} & \cite{TerekhovTDB14} & 2014 & J. Artif. Intell. Res. & 38 & 12 & 0 & \ref{b:TerekhovTDB14} & \ref{c:TerekhovTDB14}\\
\rowlabel{a:ThiruvadyWGS14}ThiruvadyWGS14 \href{https://doi.org/10.1007/s10732-014-9260-3}{ThiruvadyWGS14} & \hyperref[auth:a401]{Dhananjay R. Thiruvady}, \hyperref[auth:a117]{Mark G. Wallace}, \hyperref[auth:a341]{H. Gu}, \hyperref[auth:a125]{A. Schutt} & A Lagrangian relaxation and {ACO} hybrid for resource constrained project scheduling with discounted cash flows & \href{../works/ThiruvadyWGS14.pdf}{Yes} & \cite{ThiruvadyWGS14} & 2014 & J. Heuristics & 34 & 19 & 18 & \ref{b:ThiruvadyWGS14} & \ref{c:ThiruvadyWGS14}\\
\rowlabel{a:ArtiguesLH13}ArtiguesLH13 \href{http://dx.doi.org/10.1016/j.ijpe.2010.09.030}{ArtiguesLH13} & \hyperref[auth:a6]{C. Artigues}, \hyperref[auth:a3]{P. Lopez}, \hyperref[auth:a1184]{A. Haït} & The energy scheduling problem: Industrial case-study and constraint propagation techniques & No & \cite{ArtiguesLH13} & 2013 & International Journal of Production Economics & null & 76 & 16 & No & \ref{c:ArtiguesLH13}\\
\rowlabel{a:BajestaniB13}BajestaniB13 \href{https://doi.org/10.1613/jair.3902}{BajestaniB13} & \hyperref[auth:a825]{Maliheh Aramon Bajestani}, \hyperref[auth:a89]{J. Christopher Beck} & Scheduling a Dynamic Aircraft Repair Shop with Limited Repair Resources & \href{../works/BajestaniB13.pdf}{Yes} & \cite{BajestaniB13} & 2013 & J. Artif. Intell. Res. & 36 & 14 & 0 & \ref{b:BajestaniB13} & \ref{c:BajestaniB13}\\
\rowlabel{a:BegB13}BegB13 \href{http://doi.acm.org/10.1145/2512470}{BegB13} & \hyperref[auth:a617]{Mirza Omer Beg}, \hyperref[auth:a618]{Peter van Beek} & A constraint programming approach for integrated spatial and temporal scheduling for clustered architectures & \href{../works/BegB13.pdf}{Yes} & \cite{BegB13} & 2013 & {ACM} Trans. Embed. Comput. Syst. & 23 & 1 & 28 & \ref{b:BegB13} & \ref{c:BegB13}\\
\rowlabel{a:HeinzSB13}HeinzSB13 \href{https://doi.org/10.1007/s10601-012-9136-9}{HeinzSB13} & \hyperref[auth:a134]{S. Heinz}, \hyperref[auth:a135]{J. Schulz}, \hyperref[auth:a89]{J. Christopher Beck} & Using dual presolving reductions to reformulate cumulative constraints & \href{../works/HeinzSB13.pdf}{Yes} & \cite{HeinzSB13} & 2013 & Constraints An Int. J. & 36 & 7 & 31 & \ref{b:HeinzSB13} & \ref{c:HeinzSB13}\\
\rowlabel{a:KameugneF13}KameugneF13 \href{http://dx.doi.org/10.1007/s13226-013-0005-z}{KameugneF13} & \hyperref[auth:a10]{R. Kameugne}, \hyperref[auth:a131]{Laure Pauline Fotso} & A cumulative not-first/not-last filtering algorithm in O(n 2log(n)) & No & \cite{KameugneF13} & 2013 & Indian Journal of Pure and Applied Mathematics & null & 6 & 4 & No & \ref{c:KameugneF13}\\
\rowlabel{a:LombardiMB13}LombardiMB13 \href{http://dx.doi.org/10.1109/tc.2011.203}{LombardiMB13} & \hyperref[auth:a143]{M. Lombardi}, \hyperref[auth:a144]{M. Milano}, \hyperref[auth:a247]{L. Benini} & Robust Scheduling of Task Graphs under Execution Time Uncertainty & \href{../works/LombardiMB13.pdf}{Yes} & \cite{LombardiMB13} & 2013 & IEEE Transactions on Computers & 14 & 28 & 29 & \ref{b:LombardiMB13} & \ref{c:LombardiMB13}\\
\rowlabel{a:MenciaSV13}MenciaSV13 \href{http://dx.doi.org/10.1007/s10845-012-0726-6}{MenciaSV13} & \hyperref[auth:a928]{C. Mencía}, \hyperref[auth:a929]{María R. Sierra}, \hyperref[auth:a930]{R. Varela} & Intensified iterative deepening A* with application to job shop scheduling & \href{../works/MenciaSV13.pdf}{Yes} & \cite{MenciaSV13} & 2013 & Journal of Intelligent Manufacturing & 11 & 9 & 43 & \ref{b:MenciaSV13} & \ref{c:MenciaSV13}\\
\rowlabel{a:OzturkTHO13}OzturkTHO13 \href{https://doi.org/10.1007/s10601-013-9142-6}{OzturkTHO13} & \hyperref[auth:a136]{C. {\"{O}}zt{\"{u}}rk}, \hyperref[auth:a137]{S. Tunali}, \hyperref[auth:a138]{B. Hnich}, \hyperref[auth:a139]{A. {\"{O}}rnek} & Balancing and scheduling of flexible mixed model assembly lines & \href{../works/OzturkTHO13.pdf}{Yes} & \cite{OzturkTHO13} & 2013 & Constraints An Int. J. & 36 & 31 & 44 & \ref{b:OzturkTHO13} & \ref{c:OzturkTHO13}\\
\rowlabel{a:SchuttFSW13}SchuttFSW13 \href{https://doi.org/10.1007/s10951-012-0285-x}{SchuttFSW13} & \hyperref[auth:a125]{A. Schutt}, \hyperref[auth:a155]{T. Feydy}, \hyperref[auth:a126]{Peter J. Stuckey}, \hyperref[auth:a117]{Mark G. Wallace} & Solving RCPSP/max by lazy clause generation & \href{../works/SchuttFSW13.pdf}{Yes} & \cite{SchuttFSW13} & 2013 & Journal of Scheduling & 17 & 43 & 23 & \ref{b:SchuttFSW13} & \ref{c:SchuttFSW13}\\
\rowlabel{a:UnsalO13}UnsalO13 \href{http://dx.doi.org/10.1016/j.tre.2013.08.006}{UnsalO13} & \hyperref[auth:a1243]{O. Unsal}, \hyperref[auth:a352]{C. Oguz} & Constraint programming approach to quay crane scheduling problem & No & \cite{UnsalO13} & 2013 & Transportation Research Part E: Logistics and Transportation Review & null & 44 & 25 & No & \ref{c:UnsalO13}\\
\rowlabel{a:GuyonLPR12}GuyonLPR12 \href{http://dx.doi.org/10.1007/s10479-012-1132-3}{GuyonLPR12} & \hyperref[auth:a990]{O. Guyon}, \hyperref[auth:a991]{P. Lemaire}, \hyperref[auth:a992]{Éric Pinson}, \hyperref[auth:a993]{D. Rivreau} & Solving an integrated job-shop problem with human resource constraints & \href{../works/GuyonLPR12.pdf}{Yes} & \cite{GuyonLPR12} & 2012 & Annals of Operations Research & 25 & 32 & 25 & \ref{b:GuyonLPR12} & \ref{c:GuyonLPR12}\\
\rowlabel{a:HeinzSSW12}HeinzSSW12 \href{https://doi.org/10.1007/s10601-011-9113-8}{HeinzSSW12} & \hyperref[auth:a134]{S. Heinz}, \hyperref[auth:a140]{T. Schlechte}, \hyperref[auth:a141]{R. Stephan}, \hyperref[auth:a142]{M. Winkler} & Solving steel mill slab design problems & \href{../works/HeinzSSW12.pdf}{Yes} & \cite{HeinzSSW12} & 2012 & Constraints An Int. J. & 12 & 10 & 9 & \ref{b:HeinzSSW12} & \ref{c:HeinzSSW12}\\
\rowlabel{a:LimtanyakulS12}LimtanyakulS12 \href{https://doi.org/10.1007/s10601-012-9118-y}{LimtanyakulS12} & \hyperref[auth:a145]{K. Limtanyakul}, \hyperref[auth:a146]{U. Schwiegelshohn} & Improvements of constraint programming and hybrid methods for scheduling of tests on vehicle prototypes & \href{../works/LimtanyakulS12.pdf}{Yes} & \cite{LimtanyakulS12} & 2012 & Constraints An Int. J. & 32 & 4 & 16 & \ref{b:LimtanyakulS12} & \ref{c:LimtanyakulS12}\\
\rowlabel{a:LombardiM12}LombardiM12 \href{https://doi.org/10.1007/s10601-011-9115-6}{LombardiM12} & \hyperref[auth:a143]{M. Lombardi}, \hyperref[auth:a144]{M. Milano} & Optimal methods for resource allocation and scheduling: a cross-disciplinary survey & \href{../works/LombardiM12.pdf}{Yes} & \cite{LombardiM12} & 2012 & Constraints An Int. J. & 35 & 39 & 68 & \ref{b:LombardiM12} & \ref{c:LombardiM12}\\
\rowlabel{a:LombardiM12a}LombardiM12a \href{https://doi.org/10.1016/j.artint.2011.12.001}{LombardiM12a} & \hyperref[auth:a143]{M. Lombardi}, \hyperref[auth:a144]{M. Milano} & A min-flow algorithm for Minimal Critical Set detection in Resource Constrained Project Scheduling & \href{../works/LombardiM12a.pdf}{Yes} & \cite{LombardiM12a} & 2012 & Artificial Intelligence & 10 & 3 & 13 & \ref{b:LombardiM12a} & \ref{c:LombardiM12a}\\
\rowlabel{a:MalapertCGJLR12}MalapertCGJLR12 \href{https://doi.org/10.1287/ijoc.1100.0446}{MalapertCGJLR12} & \hyperref[auth:a82]{A. Malapert}, \hyperref[auth:a1013]{H. Cambazard}, \hyperref[auth:a295]{C. Gu{\'{e}}ret}, \hyperref[auth:a249]{N. Jussien}, \hyperref[auth:a653]{A. Langevin}, \hyperref[auth:a331]{L. Rousseau} & An Optimal Constraint Programming Approach to the Open-Shop Problem & \href{../works/MalapertCGJLR12.pdf}{Yes} & \cite{MalapertCGJLR12} & 2012 & INFORMS Journal on Computing & 17 & 23 & 21 & \ref{b:MalapertCGJLR12} & \ref{c:MalapertCGJLR12}\\
\rowlabel{a:MenciaSV12}MenciaSV12 \href{http://dx.doi.org/10.1007/s10479-012-1296-x}{MenciaSV12} & \hyperref[auth:a928]{C. Mencía}, \hyperref[auth:a929]{María R. Sierra}, \hyperref[auth:a930]{R. Varela} & Depth-first heuristic search for the job shop scheduling problem & \href{../works/MenciaSV12.pdf}{Yes} & \cite{MenciaSV12} & 2012 & Annals of Operations Research & 32 & 16 & 40 & \ref{b:MenciaSV12} & \ref{c:MenciaSV12}\\
\rowlabel{a:NovasH12}NovasH12 \href{https://doi.org/10.1016/j.compchemeng.2012.01.005}{NovasH12} & \hyperref[auth:a529]{Juan M. Novas}, \hyperref[auth:a596]{Gabriela P. Henning} & A comprehensive constraint programming approach for the rolling horizon-based scheduling of automated wet-etch stations & \href{../works/NovasH12.pdf}{Yes} & \cite{NovasH12} & 2012 & Computers \  Chemical Engineering & 17 & 17 & 15 & \ref{b:NovasH12} & \ref{c:NovasH12}\\
\rowlabel{a:OzturkTHO12}OzturkTHO12 \href{https://www.sciencedirect.com/science/article/pii/S1474667016331858}{OzturkTHO12} & \hyperref[auth:a1030]{C. {\"{O}}zt{\"{u}}rk}, \hyperref[auth:a1031]{S. Tunalı}, \hyperref[auth:a138]{B. Hnich}, \hyperref[auth:a139]{A. {\"{O}}rnek} & A Constraint Programming Model for Balancing and Scheduling of Flexible Mixed Model Assembly Lines with Parallel Stations & \href{../works/OzturkTHO12.pdf}{Yes} & \cite{OzturkTHO12} & 2012 & IFAC Proceedings Volumes & 6 & 5 & 5 & \ref{b:OzturkTHO12} & \ref{c:OzturkTHO12}\\
\rowlabel{a:TerekhovDOB12}TerekhovDOB12 \href{https://doi.org/10.1016/j.cie.2012.02.006}{TerekhovDOB12} & \hyperref[auth:a826]{D. Terekhov}, \hyperref[auth:a828]{Mustafa K. Dogru}, \hyperref[auth:a829]{U. {\"{O}}zen}, \hyperref[auth:a89]{J. Christopher Beck} & Solving two-machine assembly scheduling problems with inventory constraints & \href{../works/TerekhovDOB12.pdf}{Yes} & \cite{TerekhovDOB12} & 2012 & Computers \  Industrial Engineering & 15 & 8 & 48 & \ref{b:TerekhovDOB12} & \ref{c:TerekhovDOB12}\\
\rowlabel{a:ZarandiB12}ZarandiB12 \href{http://dx.doi.org/10.1287/ijoc.1110.0458}{ZarandiB12} & \hyperref[auth:a957]{Mohammad M. Fazel-Zarandi}, \hyperref[auth:a89]{J. Christopher Beck} & Using Logic-Based Benders Decomposition to Solve the Capacity- and Distance-Constrained Plant Location Problem & No & \cite{ZarandiB12} & 2012 & INFORMS Journal on Computing & null & 38 & 57 & No & \ref{c:ZarandiB12}\\
\rowlabel{a:BandaSC11}BandaSC11 \href{https://doi.org/10.1287/ijoc.1090.0378}{BandaSC11} & \hyperref[auth:a804]{Maria Garcia de la Banda}, \hyperref[auth:a126]{Peter J. Stuckey}, \hyperref[auth:a348]{G. Chu} & Solving Talent Scheduling with Dynamic Programming & \href{../works/BandaSC11.pdf}{Yes} & \cite{BandaSC11} & 2011 & INFORMS Journal on Computing & 18 & 24 & 17 & \ref{b:BandaSC11} & \ref{c:BandaSC11}\\
\rowlabel{a:BartakS11}BartakS11 \href{https://doi.org/10.1007/s10601-011-9109-4}{BartakS11} & \hyperref[auth:a153]{R. Bart{\'{a}}k}, \hyperref[auth:a154]{Miguel A. Salido} & Constraint satisfaction for planning and scheduling problems & \href{../works/BartakS11.pdf}{Yes} & \cite{BartakS11} & 2011 & Constraints An Int. J. & 5 & 17 & 3 & \ref{b:BartakS11} & \ref{c:BartakS11}\\
\rowlabel{a:BeckFW11}BeckFW11 \href{https://doi.org/10.1287/ijoc.1100.0388}{BeckFW11} & \hyperref[auth:a89]{J. Christopher Beck}, \hyperref[auth:a830]{T. K. Feng}, \hyperref[auth:a365]{J. Watson} & Combining Constraint Programming and Local Search for Job-Shop Scheduling & \href{../works/BeckFW11.pdf}{Yes} & \cite{BeckFW11} & 2011 & INFORMS Journal on Computing & 14 & 43 & 23 & \ref{b:BeckFW11} & \ref{c:BeckFW11}\\
\rowlabel{a:BeldiceanuCDP11}BeldiceanuCDP11 \href{https://doi.org/10.1007/s10479-010-0731-0}{BeldiceanuCDP11} & \hyperref[auth:a129]{N. Beldiceanu}, \hyperref[auth:a91]{M. Carlsson}, \hyperref[auth:a245]{S. Demassey}, \hyperref[auth:a363]{E. Poder} & New filtering for the \emph{cumulative} constraint in the context of non-overlapping rectangles & \href{../works/BeldiceanuCDP11.pdf}{Yes} & \cite{BeldiceanuCDP11} & 2011 & Annals of Operations Research & 24 & 8 & 8 & \ref{b:BeldiceanuCDP11} & \ref{c:BeldiceanuCDP11}\\
\rowlabel{a:BeniniLMR11}BeniniLMR11 \href{https://doi.org/10.1007/s10479-010-0718-x}{BeniniLMR11} & \hyperref[auth:a247]{L. Benini}, \hyperref[auth:a143]{M. Lombardi}, \hyperref[auth:a144]{M. Milano}, \hyperref[auth:a726]{M. Ruggiero} & Optimal resource allocation and scheduling for the {CELL} {BE} platform & \href{../works/BeniniLMR11.pdf}{Yes} & \cite{BeniniLMR11} & 2011 & Annals of Operations Research & 27 & 18 & 16 & \ref{b:BeniniLMR11} & \ref{c:BeniniLMR11}\\
\rowlabel{a:CobanH11}CobanH11 \href{http://dx.doi.org/10.1007/s10479-011-1031-z}{CobanH11} & \hyperref[auth:a340]{E. Coban}, \hyperref[auth:a161]{John N. Hooker} & Single-facility scheduling by logic-based Benders decomposition & \href{../works/CobanH11.pdf}{Yes} & \cite{CobanH11} & 2011 & Annals of Operations Research & 28 & 14 & 37 & \ref{b:CobanH11} & \ref{c:CobanH11}\\
\rowlabel{a:EdisO11a}EdisO11a \href{http://dx.doi.org/10.1080/03052151003759117}{EdisO11a} & \hyperref[auth:a351]{Emrah B. Edis}, \hyperref[auth:a353]{I. Ozkarahan} & A combined integer/constraint programming approach to a resource-constrained parallel machine scheduling problem with machine eligibility restrictions & No & \cite{EdisO11a} & 2011 & Engineering Optimization & null & 43 & 37 & No & \ref{c:EdisO11a}\\
\rowlabel{a:HachemiGR11}HachemiGR11 \href{https://doi.org/10.1007/s10479-010-0698-x}{HachemiGR11} & \hyperref[auth:a623]{Nizar El Hachemi}, \hyperref[auth:a624]{M. Gendreau}, \hyperref[auth:a331]{L. Rousseau} & A hybrid constraint programming approach to the log-truck scheduling problem & \href{../works/HachemiGR11.pdf}{Yes} & \cite{HachemiGR11} & 2011 & Annals of Operations Research & 16 & 32 & 19 & \ref{b:HachemiGR11} & \ref{c:HachemiGR11}\\
\rowlabel{a:HeckmanB11}HeckmanB11 \href{https://doi.org/10.1007/s10951-009-0113-0}{HeckmanB11} & \hyperref[auth:a831]{I. Heckman}, \hyperref[auth:a89]{J. Christopher Beck} & Understanding the behavior of Solution-Guided Search for job-shop scheduling & \href{../works/HeckmanB11.pdf}{Yes} & \cite{HeckmanB11} & 2011 & Journal of Scheduling & 20 & 0 & 22 & \ref{b:HeckmanB11} & \ref{c:HeckmanB11}\\
\rowlabel{a:KelbelH11}KelbelH11 \href{https://doi.org/10.1007/s10845-009-0318-2}{KelbelH11} & \hyperref[auth:a626]{J. Kelbel}, \hyperref[auth:a116]{Z. Hanz{\'{a}}lek} & Solving production scheduling with earliness/tardiness penalties by constraint programming & \href{../works/KelbelH11.pdf}{Yes} & \cite{KelbelH11} & 2011 & Journal of Intelligent Manufacturing & 10 & 12 & 14 & \ref{b:KelbelH11} & \ref{c:KelbelH11}\\
\rowlabel{a:KovacsB11}KovacsB11 \href{https://doi.org/10.1007/s10601-009-9088-x}{KovacsB11} & \hyperref[auth:a147]{A. Kov{\'{a}}cs}, \hyperref[auth:a89]{J. Christopher Beck} & A global constraint for total weighted completion time for unary resources & \href{../works/KovacsB11.pdf}{Yes} & \cite{KovacsB11} & 2011 & Constraints An Int. J. & 24 & 4 & 26 & \ref{b:KovacsB11} & \ref{c:KovacsB11}\\
\rowlabel{a:KovacsK11}KovacsK11 \href{https://doi.org/10.1007/s10601-010-9102-3}{KovacsK11} & \hyperref[auth:a147]{A. Kov{\'{a}}cs}, \hyperref[auth:a156]{T. Kis} & Constraint programming approach to a bilevel scheduling problem & \href{../works/KovacsK11.pdf}{Yes} & \cite{KovacsK11} & 2011 & Constraints An Int. J. & 24 & 3 & 24 & \ref{b:KovacsK11} & \ref{c:KovacsK11}\\
\rowlabel{a:LiuW11}LiuW11 \href{http://dx.doi.org/10.1016/j.autcon.2011.04.012}{LiuW11} & \hyperref[auth:a1272]{S. Liu}, \hyperref[auth:a1273]{C. Wang} & Optimizing project selection and scheduling problems with time-dependent resource constraints & No & \cite{LiuW11} & 2011 & Automation in Construction & null & 57 & 35 & No & \ref{c:LiuW11}\\
\rowlabel{a:ReddyFIBKAJ11}ReddyFIBKAJ11 \href{https://doi.org/10.1145/1989734.1989745}{ReddyFIBKAJ11} & \hyperref[auth:a1055]{Sudhakar Y. Reddy}, \hyperref[auth:a384]{J. Frank}, \hyperref[auth:a1056]{M. Iatauro}, \hyperref[auth:a1057]{Matthew E. Boyce}, \hyperref[auth:a385]{E. K{\"{u}}rkl{\"{u}}}, \hyperref[auth:a1058]{M. Ai{-}Chang}, \hyperref[auth:a1059]{Ari K. J{\'{o}}nsson} & Planning solar array operations on the international space station & No & \cite{ReddyFIBKAJ11} & 2011 & {ACM} Trans. Intell. Syst. Technol. & 24 & 3 & 8 & No & \ref{c:ReddyFIBKAJ11}\\
\rowlabel{a:SchausHMCMD11}SchausHMCMD11 \href{https://doi.org/10.1007/s10601-010-9100-5}{SchausHMCMD11} & \hyperref[auth:a148]{P. Schaus}, \hyperref[auth:a149]{Pascal Van Hentenryck}, \hyperref[auth:a150]{J. Monette}, \hyperref[auth:a151]{C. Coffrin}, \hyperref[auth:a32]{L. Michel}, \hyperref[auth:a152]{Y. Deville} & Solving Steel Mill Slab Problems with constraint-based techniques: CP, LNS, and {CBLS} & \href{../works/SchausHMCMD11.pdf}{Yes} & \cite{SchausHMCMD11} & 2011 & Constraints An Int. J. & 23 & 14 & 5 & \ref{b:SchausHMCMD11} & \ref{c:SchausHMCMD11}\\
\rowlabel{a:SchuttFSW11}SchuttFSW11 \href{https://doi.org/10.1007/s10601-010-9103-2}{SchuttFSW11} & \hyperref[auth:a125]{A. Schutt}, \hyperref[auth:a155]{T. Feydy}, \hyperref[auth:a126]{Peter J. Stuckey}, \hyperref[auth:a117]{Mark G. Wallace} & Explaining the cumulative propagator & \href{../works/SchuttFSW11.pdf}{Yes} & \cite{SchuttFSW11} & 2011 & Constraints An Int. J. & 33 & 57 & 23 & \ref{b:SchuttFSW11} & \ref{c:SchuttFSW11}\\
\rowlabel{a:TopalogluO11}TopalogluO11 \href{https://doi.org/10.1016/j.cor.2010.04.018}{TopalogluO11} & \hyperref[auth:a625]{S. Topaloglu}, \hyperref[auth:a353]{I. Ozkarahan} & A constraint programming-based solution approach for medical resident scheduling problems & \href{../works/TopalogluO11.pdf}{Yes} & \cite{TopalogluO11} & 2011 & Computers \  Operations Research & 10 & 46 & 24 & \ref{b:TopalogluO11} & \ref{c:TopalogluO11}\\
\rowlabel{a:TrojetHL11}TrojetHL11 \href{https://doi.org/10.1016/j.cie.2010.08.014}{TrojetHL11} & \hyperref[auth:a713]{M. Trojet}, \hyperref[auth:a714]{F. H'Mida}, \hyperref[auth:a3]{P. Lopez} & Project scheduling under resource constraints: Application of the cumulative global constraint in a decision support framework & \href{../works/TrojetHL11.pdf}{Yes} & \cite{TrojetHL11} & 2011 & Computers \  Industrial Engineering & 7 & 11 & 17 & \ref{b:TrojetHL11} & \ref{c:TrojetHL11}\\
\rowlabel{a:ZeballosNH11}ZeballosNH11 \href{http://dx.doi.org/10.1016/j.compchemeng.2011.01.043}{ZeballosNH11} & \hyperref[auth:a1177]{Luis J. Zeballos}, \hyperref[auth:a529]{Juan M. Novas}, \hyperref[auth:a596]{Gabriela P. Henning} & A CP formulation for scheduling multiproduct multistage batch plants & No & \cite{ZeballosNH11} & 2011 & Computers \  Chemical Engineering & null & 26 & 29 & No & \ref{c:ZeballosNH11}\\
\rowlabel{a:BartakCS10}BartakCS10 \href{https://doi.org/10.1007/s10479-008-0492-1}{BartakCS10} & \hyperref[auth:a153]{R. Bart{\'{a}}k}, \hyperref[auth:a162]{O. Cepek}, \hyperref[auth:a788]{P. Surynek} & Discovering implied constraints in precedence graphs with alternatives & \href{../works/BartakCS10.pdf}{Yes} & \cite{BartakCS10} & 2010 & Annals of Operations Research & 31 & 2 & 9 & \ref{b:BartakCS10} & \ref{c:BartakCS10}\\
\rowlabel{a:BartakSR10}BartakSR10 \href{https://doi.org/10.1017/S0269888910000202}{BartakSR10} & \hyperref[auth:a153]{R. Bart{\'{a}}k}, \hyperref[auth:a154]{Miguel A. Salido}, \hyperref[auth:a318]{F. Rossi} & New trends in constraint satisfaction, planning, and scheduling: a survey & \href{../works/BartakSR10.pdf}{Yes} & \cite{BartakSR10} & 2010 & Knowl. Eng. Rev. & 31 & 28 & 47 & \ref{b:BartakSR10} & \ref{c:BartakSR10}\\
\rowlabel{a:ChenGPSH10}ChenGPSH10 \href{http://dx.doi.org/10.1007/s11465-010-0106-x}{ChenGPSH10} & \hyperref[auth:a923]{Y. Chen}, \hyperref[auth:a924]{Z. Guan}, \hyperref[auth:a925]{Y. Peng}, \hyperref[auth:a926]{X. Shao}, \hyperref[auth:a927]{M. Hasseb} & Technology and system of constraint programming for industry production scheduling — Part I: A brief survey and potential directions & \href{../works/ChenGPSH10.pdf}{Yes} & \cite{ChenGPSH10} & 2010 & Frontiers of Mechanical Engineering in China & 10 & 2 & 32 & \ref{b:ChenGPSH10} & \ref{c:ChenGPSH10}\\
\rowlabel{a:LiuGT10}LiuGT10 \href{http://dx.doi.org/10.3724/sp.j.1004.2010.00603}{LiuGT10} & \hyperref[auth:a1246]{S. Liu}, \hyperref[auth:a1247]{Z. Guo}, \hyperref[auth:a1248]{J. Tang} & Constraint Propagation for Cumulative Scheduling Problems with Precedences: Constraint Propagation for Cumulative Scheduling Problems with Precedences & No & \cite{LiuGT10} & 2010 & Acta Automatica Sinica & null & 2 & 15 & No & \ref{c:LiuGT10}\\
\rowlabel{a:LombardiM10a}LombardiM10a \href{https://doi.org/10.1016/j.artint.2010.02.004}{LombardiM10a} & \hyperref[auth:a143]{M. Lombardi}, \hyperref[auth:a144]{M. Milano} & Allocation and scheduling of Conditional Task Graphs & \href{../works/LombardiM10a.pdf}{Yes} & \cite{LombardiM10a} & 2010 & Artificial Intelligence & 30 & 8 & 24 & \ref{b:LombardiM10a} & \ref{c:LombardiM10a}\\
\rowlabel{a:LombardiMRB10}LombardiMRB10 \href{http://dx.doi.org/10.1007/s10951-010-0184-y}{LombardiMRB10} & \hyperref[auth:a143]{M. Lombardi}, \hyperref[auth:a144]{M. Milano}, \hyperref[auth:a726]{M. Ruggiero}, \hyperref[auth:a247]{L. Benini} & Stochastic allocation and scheduling for conditional task graphs in multi-processor systems-on-chip & \href{../works/LombardiMRB10.pdf}{Yes} & \cite{LombardiMRB10} & 2010 & Journal of Scheduling & 31 & 24 & 41 & \ref{b:LombardiMRB10} & \ref{c:LombardiMRB10}\\
\rowlabel{a:LopesCSM10}LopesCSM10 \href{https://doi.org/10.1007/s10601-009-9086-z}{LopesCSM10} & \hyperref[auth:a157]{Tony Minoru Tamura Lopes}, \hyperref[auth:a158]{Andr{\'{e}} A. Cir{\'{e}}}, \hyperref[auth:a159]{Cid Carvalho de Souza}, \hyperref[auth:a160]{Arnaldo Vieira Moura} & A hybrid model for a multiproduct pipeline planning and scheduling problem & \href{../works/LopesCSM10.pdf}{Yes} & \cite{LopesCSM10} & 2010 & Constraints An Int. J. & 39 & 31 & 18 & \ref{b:LopesCSM10} & \ref{c:LopesCSM10}\\
\rowlabel{a:NovasH10}NovasH10 \href{https://doi.org/10.1016/j.compchemeng.2010.07.011}{NovasH10} & \hyperref[auth:a529]{Juan M. Novas}, \hyperref[auth:a596]{Gabriela P. Henning} & Reactive scheduling framework based on domain knowledge and constraint programming & \href{../works/NovasH10.pdf}{Yes} & \cite{NovasH10} & 2010 & Computers \  Chemical Engineering & 20 & 48 & 19 & \ref{b:NovasH10} & \ref{c:NovasH10}\\
\rowlabel{a:OzturkTHO10}OzturkTHO10 \href{https://www.sciencedirect.com/science/article/pii/S1571065310000107}{OzturkTHO10} & \hyperref[auth:a136]{C. {\"{O}}zt{\"{u}}rk}, \hyperref[auth:a137]{S. Tunali}, \hyperref[auth:a138]{B. Hnich}, \hyperref[auth:a139]{A. {\"{O}}rnek} & Simultaneous Balancing and Scheduling of Flexible Mixed Model Assembly Lines with Sequence-Dependent Setup Times & \href{../works/OzturkTHO10.pdf}{Yes} & \cite{OzturkTHO10} & 2010 & Electronic Notes in Discrete Mathematics & 8 & 15 & 1 & \ref{b:OzturkTHO10} & \ref{c:OzturkTHO10}\\
\rowlabel{a:Zeballos10}Zeballos10 \href{http://dx.doi.org/10.1016/j.rcim.2010.04.005}{Zeballos10} & \hyperref[auth:a1185]{L. Zeballos} & A constraint programming approach to tool allocation and production scheduling in flexible manufacturing systems & No & \cite{Zeballos10} & 2010 & Robotics and Computer-Integrated Manufacturing & null & 41 & 16 & No & \ref{c:Zeballos10}\\
\rowlabel{a:ZeballosCM10}ZeballosCM10 \href{http://dx.doi.org/10.1021/ie1016199}{ZeballosCM10} & \hyperref[auth:a1177]{Luis J. Zeballos}, \hyperref[auth:a900]{Pedro M. Castro}, \hyperref[auth:a1216]{Carlos A. Méndez} & Integrated Constraint Programming Scheduling Approach for Automated Wet-Etch Stations in Semiconductor Manufacturing & No & \cite{ZeballosCM10} & 2010 & Industrial \  Engineering Chemistry Research & null & 22 & 30 & No & \ref{c:ZeballosCM10}\\
\rowlabel{a:ZeballosQH10}ZeballosQH10 \href{https://doi.org/10.1016/j.engappai.2009.07.002}{ZeballosQH10} & \hyperref[auth:a629]{L. Zeballos}, \hyperref[auth:a630]{O. Quiroga}, \hyperref[auth:a596]{Gabriela P. Henning} & A constraint programming model for the scheduling of flexible manufacturing systems with machine and tool limitations & \href{../works/ZeballosQH10.pdf}{Yes} & \cite{ZeballosQH10} & 2010 & Eng. Appl. Artif. Intell. & 20 & 33 & 28 & \ref{b:ZeballosQH10} & \ref{c:ZeballosQH10}\\
\rowlabel{a:abs-1009-0347}abs-1009-0347 \href{http://arxiv.org/abs/1009.0347}{abs-1009-0347} & \hyperref[auth:a125]{A. Schutt}, \hyperref[auth:a155]{T. Feydy}, \hyperref[auth:a126]{Peter J. Stuckey}, \hyperref[auth:a117]{Mark G. Wallace} & Solving the Resource Constrained Project Scheduling Problem with Generalized Precedences by Lazy Clause Generation & \href{../works/abs-1009-0347.pdf}{Yes} & \cite{abs-1009-0347} & 2010 & CoRR & 37 & 0 & 0 & \ref{b:abs-1009-0347} & \ref{c:abs-1009-0347}\\
\rowlabel{a:BidotVLB09}BidotVLB09 \href{https://doi.org/10.1007/s10951-008-0080-x}{BidotVLB09} & \hyperref[auth:a832]{J. Bidot}, \hyperref[auth:a833]{T. Vidal}, \hyperref[auth:a118]{P. Laborie}, \hyperref[auth:a89]{J. Christopher Beck} & A theoretic and practical framework for scheduling in a stochastic environment & \href{../works/BidotVLB09.pdf}{Yes} & \cite{BidotVLB09} & 2009 & Journal of Scheduling & 30 & 58 & 20 & \ref{b:BidotVLB09} & \ref{c:BidotVLB09}\\
\rowlabel{a:BocewiczBB09}BocewiczBB09 \href{https://doi.org/10.1504/IJIIDS.2009.023038}{BocewiczBB09} & \hyperref[auth:a638]{G. Bocewicz}, \hyperref[auth:a639]{I. Bach}, \hyperref[auth:a640]{Zbigniew Antoni Banaszak} & Logic-algebraic method based and constraints programming driven approach to AGVs scheduling & \href{../works/BocewiczBB09.pdf}{Yes} & \cite{BocewiczBB09} & 2009 & Int. J. Intell. Inf. Database Syst. & 19 & 0 & 0 & \ref{b:BocewiczBB09} & \ref{c:BocewiczBB09}\\
\rowlabel{a:CarchraeB09}CarchraeB09 \href{http://dx.doi.org/10.1007/s10852-008-9100-2}{CarchraeB09} & \hyperref[auth:a274]{T. Carchrae}, \hyperref[auth:a89]{J. Christopher Beck} & Principles for the Design of Large Neighborhood Search & \href{../works/CarchraeB09.pdf}{Yes} & \cite{CarchraeB09} & 2009 & Journal of Mathematical Modelling and Algorithms & 26 & 16 & 19 & \ref{b:CarchraeB09} & \ref{c:CarchraeB09}\\
\rowlabel{a:GarridoAO09}GarridoAO09 \href{https://doi.org/10.1007/s10951-008-0083-7}{GarridoAO09} & \hyperref[auth:a641]{A. Garrido}, \hyperref[auth:a642]{M. Arang{\'{u}}}, \hyperref[auth:a643]{E. Onaindia} & A constraint programming formulation for planning: from plan scheduling to plan generation & \href{../works/GarridoAO09.pdf}{Yes} & \cite{GarridoAO09} & 2009 & Journal of Scheduling & 30 & 5 & 14 & \ref{b:GarridoAO09} & \ref{c:GarridoAO09}\\
\rowlabel{a:Jans09}Jans09 \href{http://dx.doi.org/10.1287/ijoc.1080.0283}{Jans09} & \hyperref[auth:a850]{R. Jans} & Solving Lot-Sizing Problems on Parallel Identical Machines Using Symmetry-Breaking Constraints & \href{../works/Jans09.pdf}{Yes} & \cite{Jans09} & 2009 & INFORMS Journal on Computing & 24 & 59 & 73 & \ref{b:Jans09} & \ref{c:Jans09}\\
\rowlabel{a:MilanoW09}MilanoW09 \href{http://dx.doi.org/10.1007/s10479-009-0654-9}{MilanoW09} & \hyperref[auth:a144]{M. Milano}, \hyperref[auth:a117]{Mark G. Wallace} & Integrating Operations Research in Constraint Programming & \href{../works/MilanoW09.pdf}{Yes} & \cite{MilanoW09} & 2009 & Annals of Operations Research & 40 & 34 & 46 & \ref{b:MilanoW09} & \ref{c:MilanoW09}\\
\rowlabel{a:OhrimenkoSC09}OhrimenkoSC09 \href{http://dx.doi.org/10.1007/s10601-008-9064-x}{OhrimenkoSC09} & \hyperref[auth:a870]{O. Ohrimenko}, \hyperref[auth:a126]{Peter J. Stuckey}, \hyperref[auth:a871]{M. Codish} & Propagation via lazy clause generation & \href{../works/OhrimenkoSC09.pdf}{Yes} & \cite{OhrimenkoSC09} & 2009 & Constraints An Int. J. & 35 & 127 & 15 & \ref{b:OhrimenkoSC09} & \ref{c:OhrimenkoSC09}\\
\rowlabel{a:RuggieroBBMA09}RuggieroBBMA09 \href{https://doi.org/10.1109/TCAD.2009.2013536}{RuggieroBBMA09} & \hyperref[auth:a726]{M. Ruggiero}, \hyperref[auth:a380]{D. Bertozzi}, \hyperref[auth:a247]{L. Benini}, \hyperref[auth:a144]{M. Milano}, \hyperref[auth:a727]{A. Andrei} & Reducing the Abstraction and Optimality Gaps in the Allocation and Scheduling for Variable Voltage/Frequency MPSoC Platforms & \href{../works/RuggieroBBMA09.pdf}{Yes} & \cite{RuggieroBBMA09} & 2009 & {IEEE} Trans. Comput. Aided Des. Integr. Circuits Syst. & 14 & 9 & 27 & \ref{b:RuggieroBBMA09} & \ref{c:RuggieroBBMA09}\\
\rowlabel{a:WuBB09}WuBB09 \href{https://doi.org/10.1016/j.cor.2008.08.008}{WuBB09} & \hyperref[auth:a276]{Christine Wei Wu}, \hyperref[auth:a222]{Kenneth N. Brown}, \hyperref[auth:a89]{J. Christopher Beck} & Scheduling with uncertain durations: Modeling beta-robust scheduling with constraints & \href{../works/WuBB09.pdf}{Yes} & \cite{WuBB09} & 2009 & Computers \  Operations Research & 9 & 42 & 5 & \ref{b:WuBB09} & \ref{c:WuBB09}\\
\rowlabel{a:abs-0907-0939}abs-0907-0939 \href{http://arxiv.org/abs/0907.0939}{abs-0907-0939} & \hyperref[auth:a226]{T. Petit}, \hyperref[auth:a363]{E. Poder} & The Soft Cumulative Constraint & \href{../works/abs-0907-0939.pdf}{Yes} & \cite{abs-0907-0939} & 2009 & CoRR & 12 & 0 & 0 & \ref{b:abs-0907-0939} & \ref{c:abs-0907-0939}\\
\rowlabel{a:BartakSR08}BartakSR08 \href{http://dx.doi.org/10.1007/s10845-008-0203-4}{BartakSR08} & \hyperref[auth:a1081]{R. Barták}, \hyperref[auth:a154]{Miguel A. Salido}, \hyperref[auth:a318]{F. Rossi} & Constraint satisfaction techniques in planning and scheduling & No & \cite{BartakSR08} & 2008 & Journal of Intelligent Manufacturing & null & 54 & 21 & No & \ref{c:BartakSR08}\\
\rowlabel{a:ClautiauxJCM08}ClautiauxJCM08 \href{http://dx.doi.org/10.1016/j.cor.2006.05.012}{ClautiauxJCM08} & \hyperref[auth:a1193]{F. Clautiaux}, \hyperref[auth:a940]{A. Jouglet}, \hyperref[auth:a854]{J. Carlier}, \hyperref[auth:a1194]{A. Moukrim} & A new constraint programming approach for the orthogonal packing problem & No & \cite{ClautiauxJCM08} & 2008 & Computers  \  Operations Research & null & 64 & 14 & No & \ref{c:ClautiauxJCM08}\\
\rowlabel{a:GarridoOS08}GarridoOS08 \href{https://doi.org/10.1016/j.engappai.2008.03.009}{GarridoOS08} & \hyperref[auth:a641]{A. Garrido}, \hyperref[auth:a643]{E. Onaindia}, \hyperref[auth:a648]{{\'{O}}scar Sapena} & Planning and scheduling in an e-learning environment. {A} constraint-programming-based approach & \href{../works/GarridoOS08.pdf}{Yes} & \cite{GarridoOS08} & 2008 & Eng. Appl. Artif. Intell. & 11 & 22 & 7 & \ref{b:GarridoOS08} & \ref{c:GarridoOS08}\\
\rowlabel{a:HladikCDJ08}HladikCDJ08 \href{http://dx.doi.org/10.1016/j.jss.2007.02.032}{HladikCDJ08} & \hyperref[auth:a1182]{P. Hladik}, \hyperref[auth:a1013]{H. Cambazard}, \hyperref[auth:a1183]{A. Déplanche}, \hyperref[auth:a249]{N. Jussien} & Solving a real-time allocation problem with constraint programming & No & \cite{HladikCDJ08} & 2008 & Journal of Systems and Software & null & 36 & 27 & No & \ref{c:HladikCDJ08}\\
\rowlabel{a:KovacsB08}KovacsB08 \href{https://doi.org/10.1016/j.engappai.2008.03.004}{KovacsB08} & \hyperref[auth:a147]{A. Kov{\'{a}}cs}, \hyperref[auth:a89]{J. Christopher Beck} & A global constraint for total weighted completion time for cumulative resources & \href{../works/KovacsB08.pdf}{Yes} & \cite{KovacsB08} & 2008 & Eng. Appl. Artif. Intell. & 7 & 5 & 14 & \ref{b:KovacsB08} & \ref{c:KovacsB08}\\
\rowlabel{a:LiW08}LiW08 \href{http://dx.doi.org/10.1007/s10951-008-0079-3}{LiW08} & \hyperref[auth:a965]{H. Li}, \hyperref[auth:a966]{K. Womer} & Scheduling projects with multi-skilled personnel by a hybrid MILP/CP benders decomposition algorithm & \href{../works/LiW08.pdf}{Yes} & \cite{LiW08} & 2008 & Journal of Scheduling & 18 & 113 & 31 & \ref{b:LiW08} & \ref{c:LiW08}\\
\rowlabel{a:LiessM08}LiessM08 \href{https://doi.org/10.1007/s10479-007-0188-y}{LiessM08} & \hyperref[auth:a647]{O. Liess}, \hyperref[auth:a360]{P. Michelon} & A constraint programming approach for the resource-constrained project scheduling problem & \href{../works/LiessM08.pdf}{Yes} & \cite{LiessM08} & 2008 & Annals of Operations Research & 12 & 22 & 14 & \ref{b:LiessM08} & \ref{c:LiessM08}\\
\rowlabel{a:MalikMB08}MalikMB08 \href{https://doi.org/10.1142/S0218213008003765}{MalikMB08} & \hyperref[auth:a646]{Abid M. Malik}, \hyperref[auth:a649]{J. McInnes}, \hyperref[auth:a618]{Peter van Beek} & Optimal Basic Block Instruction Scheduling for Multiple-Issue Processors Using Constraint Programming & \href{../works/MalikMB08.pdf}{Yes} & \cite{MalikMB08} & 2008 & Int. J. Artif. Intell. Tools & 18 & 15 & 8 & \ref{b:MalikMB08} & \ref{c:MalikMB08}\\
\rowlabel{a:MercierH08}MercierH08 \href{http://dx.doi.org/10.1287/ijoc.1070.0226}{MercierH08} & \hyperref[auth:a860]{L. Mercier}, \hyperref[auth:a149]{Pascal Van Hentenryck} & Edge Finding for Cumulative Scheduling & \href{../works/MercierH08.pdf}{Yes} & \cite{MercierH08} & 2008 & INFORMS Journal on Computing & 21 & 32 & 5 & \ref{b:MercierH08} & \ref{c:MercierH08}\\
\rowlabel{a:ArtiguesF07}ArtiguesF07 \href{http://dx.doi.org/10.1007/s10479-007-0283-0}{ArtiguesF07} & \hyperref[auth:a6]{C. Artigues}, \hyperref[auth:a361]{D. Feillet} & A branch and bound method for the job-shop problem with sequence-dependent setup times & \href{../works/ArtiguesF07.pdf}{Yes} & \cite{ArtiguesF07} & 2007 & Annals of Operations Research & 25 & 49 & 32 & \ref{b:ArtiguesF07} & \ref{c:ArtiguesF07}\\
\rowlabel{a:Beck07}Beck07 \href{https://doi.org/10.1613/jair.2169}{Beck07} & \hyperref[auth:a89]{J. Christopher Beck} & Solution-Guided Multi-Point Constructive Search for Job Shop Scheduling & \href{../works/Beck07.pdf}{Yes} & \cite{Beck07} & 2007 & J. Artif. Intell. Res. & 29 & 34 & 0 & \ref{b:Beck07} & \ref{c:Beck07}\\
\rowlabel{a:BeckW07}BeckW07 \href{https://doi.org/10.1613/jair.2080}{BeckW07} & \hyperref[auth:a89]{J. Christopher Beck}, \hyperref[auth:a834]{N. Wilson} & Proactive Algorithms for Job Shop Scheduling with Probabilistic Durations & \href{../works/BeckW07.pdf}{Yes} & \cite{BeckW07} & 2007 & J. Artif. Intell. Res. & 50 & 27 & 0 & \ref{b:BeckW07} & \ref{c:BeckW07}\\
\rowlabel{a:CorreaLR07}CorreaLR07 \href{http://dx.doi.org/10.1016/j.cor.2005.07.004}{CorreaLR07} & \hyperref[auth:a961]{Ayoub Insa Corr{\'{e}}a}, \hyperref[auth:a653]{A. Langevin}, \hyperref[auth:a331]{L. Rousseau} & Scheduling and routing of automated guided vehicles: A hybrid approach & \href{../works/CorreaLR07.pdf}{Yes} & \cite{CorreaLR07} & 2007 & Computers \  Operations Research & 20 & 106 & 20 & \ref{b:CorreaLR07} & \ref{c:CorreaLR07}\\
\rowlabel{a:Hooker07}Hooker07 \href{http://dx.doi.org/10.1287/opre.1060.0371}{Hooker07} & \hyperref[auth:a161]{John N. Hooker} & Planning and Scheduling by Logic-Based Benders Decomposition & \href{../works/Hooker07.pdf}{Yes} & \cite{Hooker07} & 2007 & Operations Research & 29 & 181 & 19 & \ref{b:Hooker07} & \ref{c:Hooker07}\\
\rowlabel{a:MercierH07}MercierH07 \href{http://dx.doi.org/10.1016/j.disopt.2007.01.001}{MercierH07} & \hyperref[auth:a860]{L. Mercier}, \hyperref[auth:a149]{Pascal Van Hentenryck} & Strong polynomiality of resource constraint propagation & No & \cite{MercierH07} & 2007 & Discrete Optimization & null & 5 & 8 & No & \ref{c:MercierH07}\\
\rowlabel{a:Rodriguez07}Rodriguez07 \href{https://www.sciencedirect.com/science/article/pii/S0191261506000233}{Rodriguez07} & \hyperref[auth:a789]{J. Rodriguez} & A constraint programming model for real-time train scheduling at junctions & \href{../works/Rodriguez07.pdf}{Yes} & \cite{Rodriguez07} & 2007 & Transportation Research Part B: Methodological & 15 & 117 & 6 & \ref{b:Rodriguez07} & \ref{c:Rodriguez07}\\
\rowlabel{a:Simonis07}Simonis07 \href{https://doi.org/10.1007/s10601-006-9011-7}{Simonis07} & \hyperref[auth:a17]{H. Simonis} & Models for Global Constraint Applications & \href{../works/Simonis07.pdf}{Yes} & \cite{Simonis07} & 2007 & Constraints An Int. J. & 30 & 10 & 17 & \ref{b:Simonis07} & \ref{c:Simonis07}\\
\rowlabel{a:BockmayrP06}BockmayrP06 \href{http://dx.doi.org/10.1016/j.cor.2005.01.010}{BockmayrP06} & \hyperref[auth:a918]{A. Bockmayr}, \hyperref[auth:a1202]{N. Pisaruk} & Detecting infeasibility and generating cuts for mixed integer programming using constraint programming & No & \cite{BockmayrP06} & 2006 & Computers \  Operations Research & null & 12 & 7 & No & \ref{c:BockmayrP06}\\
\rowlabel{a:Gronkvist06}Gronkvist06 \href{http://dx.doi.org/10.1016/j.cor.2005.01.017}{Gronkvist06} & \hyperref[auth:a1240]{M. Gr\"{o}nkvist} & Accelerating column generation for aircraft scheduling using constraint propagation & No & \cite{Gronkvist06} & 2006 & Computers \  Operations Research & null & 28 & 15 & No & \ref{c:Gronkvist06}\\
\rowlabel{a:Hooker06}Hooker06 \href{https://doi.org/10.1007/s10601-006-8060-2}{Hooker06} & \hyperref[auth:a161]{John N. Hooker} & An Integrated Method for Planning and Scheduling to Minimize Tardiness & \href{../works/Hooker06.pdf}{Yes} & \cite{Hooker06} & 2006 & Constraints An Int. J. & 19 & 19 & 13 & \ref{b:Hooker06} & \ref{c:Hooker06}\\
\rowlabel{a:KhayatLR06}KhayatLR06 \href{https://doi.org/10.1016/j.ejor.2005.02.077}{KhayatLR06} & \hyperref[auth:a652]{Ghada El Khayat}, \hyperref[auth:a653]{A. Langevin}, \hyperref[auth:a654]{D. Riopel} & Integrated production and material handling scheduling using mathematical programming and constraint programming & \href{../works/KhayatLR06.pdf}{Yes} & \cite{KhayatLR06} & 2006 & European Journal of Operational Research & 15 & 84 & 14 & \ref{b:KhayatLR06} & \ref{c:KhayatLR06}\\
\rowlabel{a:MilanoW06}MilanoW06 \href{http://dx.doi.org/10.1007/s10288-006-0019-z}{MilanoW06} & \hyperref[auth:a144]{M. Milano}, \hyperref[auth:a117]{Mark G. Wallace} & Integrating operations research in constraint programming & \href{../works/MilanoW06.pdf}{Yes} & \cite{MilanoW06} & 2006 & 4OR & 45 & 18 & 46 & \ref{b:MilanoW06} & \ref{c:MilanoW06}\\
\rowlabel{a:SadykovW06}SadykovW06 \href{https://doi.org/10.1287/ijoc.1040.0110}{SadykovW06} & \hyperref[auth:a389]{R. Sadykov}, \hyperref[auth:a229]{Laurence A. Wolsey} & Integer Programming and Constraint Programming in Solving a Multimachine Assignment Scheduling Problem with Deadlines and Release Dates & \href{../works/SadykovW06.pdf}{Yes} & \cite{SadykovW06} & 2006 & INFORMS Journal on Computing & 9 & 45 & 6 & \ref{b:SadykovW06} & \ref{c:SadykovW06}\\
\rowlabel{a:SureshMOK06}SureshMOK06 \href{https://doi.org/10.1080/17445760600567842}{SureshMOK06} & \hyperref[auth:a655]{S. Sundaram}, \hyperref[auth:a656]{V. Mani}, \hyperref[auth:a657]{S. N. Omkar}, \hyperref[auth:a658]{H. J. Kim} & Divisible load scheduling in distributed system with buffer constraints: genetic algorithm and linear programming approach & \href{../works/SureshMOK06.pdf}{Yes} & \cite{SureshMOK06} & 2006 & Int. J. Parallel Emergent Distributed Syst. & 19 & 12 & 23 & \ref{b:SureshMOK06} & \ref{c:SureshMOK06}\\
\rowlabel{a:DemasseyAM05}DemasseyAM05 \href{http://dx.doi.org/10.1287/ijoc.1030.0043}{DemasseyAM05} & \hyperref[auth:a245]{S. Demassey}, \hyperref[auth:a6]{C. Artigues}, \hyperref[auth:a360]{P. Michelon} & Constraint-Propagation-Based Cutting Planes: An Application to the Resource-Constrained Project Scheduling Problem & \href{../works/DemasseyAM05.pdf}{Yes} & \cite{DemasseyAM05} & 2005 & INFORMS Journal on Computing & 18 & 43 & 25 & \ref{b:DemasseyAM05} & \ref{c:DemasseyAM05}\\
\rowlabel{a:Hooker05}Hooker05 \href{https://doi.org/10.1007/s10601-005-2812-2}{Hooker05} & \hyperref[auth:a161]{John N. Hooker} & A Hybrid Method for the Planning and Scheduling & \href{../works/Hooker05.pdf}{Yes} & \cite{Hooker05} & 2005 & Constraints An Int. J. & 17 & 68 & 11 & \ref{b:Hooker05} & \ref{c:Hooker05}\\
\rowlabel{a:RoePS05}RoePS05 \href{http://dx.doi.org/10.1016/j.compchemeng.2005.02.024}{RoePS05} & \hyperref[auth:a1269]{B. Roe}, \hyperref[auth:a1270]{Lazaros G. Papageorgiou}, \hyperref[auth:a1271]{N. Shah} & A hybrid MILP/CLP algorithm for multipurpose batch process scheduling & No & \cite{RoePS05} & 2005 & Computers \  Chemical Engineering & null & 48 & 15 & No & \ref{c:RoePS05}\\
\rowlabel{a:VilimBC05}VilimBC05 \href{https://doi.org/10.1007/s10601-005-2814-0}{VilimBC05} & \hyperref[auth:a121]{P. Vil{\'{\i}}m}, \hyperref[auth:a153]{R. Bart{\'{a}}k}, \hyperref[auth:a162]{O. Cepek} & Extension of \emph{O}(\emph{n} log \emph{n}) Filtering Algorithms for the Unary Resource Constraint to Optional Activities & \href{../works/VilimBC05.pdf}{Yes} & \cite{VilimBC05} & 2005 & Constraints An Int. J. & 23 & 21 & 5 & \ref{b:VilimBC05} & \ref{c:VilimBC05}\\
\rowlabel{a:ZeballosH05}ZeballosH05 \href{http://journal.iberamia.org/index.php/ia/article/view/452/article\%20\%281\%29.pdf}{ZeballosH05} & \hyperref[auth:a629]{L. Zeballos}, \hyperref[auth:a596]{Gabriela P. Henning} & A Constraint Programming Approach to {FMS} Scheduling. Consideration of Storage and Transportation Resources & \href{../works/ZeballosH05.pdf}{Yes} & \cite{ZeballosH05} & 2005 & Inteligencia Artif. & 10 & 0 & 0 & \ref{b:ZeballosH05} & \ref{c:ZeballosH05}\\
\rowlabel{a:MaraveliasCG04}MaraveliasCG04 \href{http://dx.doi.org/10.1016/j.compchemeng.2004.03.016}{MaraveliasCG04} & \hyperref[auth:a386]{Christos T. Maravelias}, \hyperref[auth:a387]{Ignacio E. Grossmann} & A hybrid MILP/CP decomposition approach for the continuous time scheduling of multipurpose batch plants & No & \cite{MaraveliasCG04} & 2004 & Computers \  Chemical Engineering & null & 116 & 24 & No & \ref{c:MaraveliasCG04}\\
\rowlabel{a:PoderBS04}PoderBS04 \href{https://doi.org/10.1016/S0377-2217(02)00756-7}{PoderBS04} & \hyperref[auth:a363]{E. Poder}, \hyperref[auth:a129]{N. Beldiceanu}, \hyperref[auth:a721]{E. Sanlaville} & Computing a lower approximation of the compulsory part of a task with varying duration and varying resource consumption & \href{../works/PoderBS04.pdf}{Yes} & \cite{PoderBS04} & 2004 & European Journal of Operational Research & 16 & 7 & 8 & \ref{b:PoderBS04} & \ref{c:PoderBS04}\\
\rowlabel{a:BeckR03}BeckR03 \href{https://doi.org/10.1023/A:1021849405707}{BeckR03} & \hyperref[auth:a89]{J. Christopher Beck}, \hyperref[auth:a256]{P. Refalo} & A Hybrid Approach to Scheduling with Earliness and Tardiness Costs & \href{../works/BeckR03.pdf}{Yes} & \cite{BeckR03} & 2003 & Annals of Operations Research & 23 & 29 & 0 & \ref{b:BeckR03} & \ref{c:BeckR03}\\
\rowlabel{a:HookerO03}HookerO03 \href{http://dx.doi.org/10.1007/s10107-003-0375-9}{HookerO03} & \hyperref[auth:a161]{John N. Hooker}, \hyperref[auth:a861]{G. Ottosson} & Logic-based Benders decomposition & \href{../works/HookerO03.pdf}{Yes} & \cite{HookerO03} & 2003 & Mathematical Programming & 28 & 317 & 0 & \ref{b:HookerO03} & \ref{c:HookerO03}\\
\rowlabel{a:Kuchcinski03}Kuchcinski03 \href{http://dx.doi.org/10.1145/785411.785416}{Kuchcinski03} & \hyperref[auth:a668]{K. Kuchcinski} & Constraints-driven scheduling and resource assignment & No & \cite{Kuchcinski03} & 2003 & ACM Transactions on Design Automation of Electronic Systems & null & 105 & 15 & No & \ref{c:Kuchcinski03}\\
\rowlabel{a:KuchcinskiW03}KuchcinskiW03 \href{https://doi.org/10.1016/S1383-7621(03)00075-4}{KuchcinskiW03} & \hyperref[auth:a668]{K. Kuchcinski}, \hyperref[auth:a667]{C. Wolinski} & Global approach to assignment and scheduling of complex behaviors based on {HCDG} and constraint programming & \href{../works/KuchcinskiW03.pdf}{Yes} & \cite{KuchcinskiW03} & 2003 & J. Syst. Archit. & 15 & 19 & 18 & \ref{b:KuchcinskiW03} & \ref{c:KuchcinskiW03}\\
\rowlabel{a:Laborie03}Laborie03 \href{http://dx.doi.org/10.1016/s0004-3702(02)00362-4}{Laborie03} & \hyperref[auth:a118]{P. Laborie} & Algorithms for propagating resource constraints in AI planning and scheduling: Existing approaches and new results & \href{../works/Laborie03.pdf}{Yes} & \cite{Laborie03} & 2003 & Artificial Intelligence & 38 & 128 & 10 & \ref{b:Laborie03} & \ref{c:Laborie03}\\
\rowlabel{a:Tsang03}Tsang03 \href{https://doi.org/10.1023/A:1024016929283}{Tsang03} & \hyperref[auth:a673]{Edward P. K. Tsang} & Constraint Based Scheduling: Applying Constraint Programming to Scheduling Problems & \href{../works/Tsang03.pdf}{Yes} & \cite{Tsang03} & 2003 & Journal of Scheduling & 2 & 1 & 0 & \ref{b:Tsang03} & \ref{c:Tsang03}\\
\rowlabel{a:HarjunkoskiG02}HarjunkoskiG02 \href{http://dx.doi.org/10.1016/s0098-1354(02)00100-x}{HarjunkoskiG02} & \hyperref[auth:a880]{I. Harjunkoski}, \hyperref[auth:a387]{Ignacio E. Grossmann} & Decomposition techniques for multistage scheduling problems using mixed-integer and constraint programming methods & \href{../works/HarjunkoskiG02.pdf}{Yes} & \cite{HarjunkoskiG02} & 2002 & Computers \  Chemical Engineering & 20 & 169 & 11 & \ref{b:HarjunkoskiG02} & \ref{c:HarjunkoskiG02}\\
\rowlabel{a:Hooker02}Hooker02 \href{http://dx.doi.org/10.1287/ijoc.14.4.295.2828}{Hooker02} & \hyperref[auth:a161]{John N. Hooker} & Logic,  Optimization,  and Constraint Programming & No & \cite{Hooker02} & 2002 & INFORMS Journal on Computing & null & 94 & 84 & No & \ref{c:Hooker02}\\
\rowlabel{a:JussienL02}JussienL02 \href{http://dx.doi.org/10.1016/s0004-3702(02)00221-7}{JussienL02} & \hyperref[auth:a249]{N. Jussien}, \hyperref[auth:a1091]{O. Lhomme} & Local search with constraint propagation and conflict-based heuristics & \href{../works/JussienL02.pdf}{Yes} & \cite{JussienL02} & 2002 & Artificial Intelligence & 25 & 88 & 16 & \ref{b:JussienL02} & \ref{c:JussienL02}\\
\rowlabel{a:LorigeonBB02}LorigeonBB02 \href{https://doi.org/10.1057/palgrave.jors.2601421}{LorigeonBB02} & \hyperref[auth:a679]{T. Lorigeon}, \hyperref[auth:a342]{J. Billaut}, \hyperref[auth:a680]{J. Bouquard} & A dynamic programming algorithm for scheduling jobs in a two-machine open shop with an availability constraint & \href{../works/LorigeonBB02.pdf}{Yes} & \cite{LorigeonBB02} & 2002 & Journal of the Operational Research Society & 8 & 22 & 0 & \ref{b:LorigeonBB02} & \ref{c:LorigeonBB02}\\
\rowlabel{a:MilanoORT02}MilanoORT02 \href{http://dx.doi.org/10.1287/ijoc.14.4.387.2830}{MilanoORT02} & \hyperref[auth:a144]{M. Milano}, \hyperref[auth:a861]{G. Ottosson}, \hyperref[auth:a256]{P. Refalo}, \hyperref[auth:a883]{Erlendur S. Thorsteinsson} & The Role of Integer Programming Techniques in Constraint Programming's Global Constraints & No & \cite{MilanoORT02} & 2002 & INFORMS Journal on Computing & null & 14 & 31 & No & \ref{c:MilanoORT02}\\
\rowlabel{a:RodriguezDG02}RodriguezDG02 \href{}{RodriguezDG02} & \hyperref[auth:a789]{J. Rodriguez}, \hyperref[auth:a790]{X. Delorme}, \hyperref[auth:a791]{X. Gandibleux} & Railway infrastructure saturation using constraint programming approach & \href{../works/RodriguezDG02.pdf}{Yes} & \cite{RodriguezDG02} & 2002 & Computers in Railways VIII & 10 & 0 & 0 & \ref{b:RodriguezDG02} & \ref{c:RodriguezDG02}\\
\rowlabel{a:Timpe02}Timpe02 \href{https://doi.org/10.1007/s00291-002-0107-1}{Timpe02} & \hyperref[auth:a681]{C. Timpe} & Solving planning and scheduling problems with combined integer and constraint programming & \href{../works/Timpe02.pdf}{Yes} & \cite{Timpe02} & 2002 & {OR} Spectr. & 18 & 42 & 0 & \ref{b:Timpe02} & \ref{c:Timpe02}\\
\rowlabel{a:BosiM2001}BosiM2001 \href{http://dx.doi.org/10.1002/1097-024x(200101)31:1<17::aid-spe355>3.0.co;2-l}{BosiM2001} & \hyperref[auth:a1250]{F. Bosi}, \hyperref[auth:a144]{M. Milano} & Enhancing CLP branch and bound techniques for scheduling problems & No & \cite{BosiM2001} & 2001 & Software: Practice and Experience & null & 3 & 12 & No & \ref{c:BosiM2001}\\
\rowlabel{a:JainG01}JainG01 \href{http://dx.doi.org/10.1287/ijoc.13.4.258.9733}{JainG01} & \hyperref[auth:a853]{V. Jain}, \hyperref[auth:a387]{Ignacio E. Grossmann} & Algorithms for Hybrid MILP/CP Models for a Class of Optimization Problems & \href{../works/JainG01.pdf}{Yes} & \cite{JainG01} & 2001 & INFORMS Journal on Computing & 19 & 279 & 23 & \ref{b:JainG01} & \ref{c:JainG01}\\
\rowlabel{a:MartinPY01}MartinPY01 \href{https://doi.org/10.1023/A:1016067230126}{MartinPY01} & \hyperref[auth:a684]{F. Martin}, \hyperref[auth:a685]{A. Pinkney}, \hyperref[auth:a686]{X. Yu} & Cane Railway Scheduling via Constraint Logic Programming: Labelling Order and Constraints in a Real-Life Application & \href{../works/MartinPY01.pdf}{Yes} & \cite{MartinPY01} & 2001 & Annals of Operations Research & 17 & 11 & 0 & \ref{b:MartinPY01} & \ref{c:MartinPY01}\\
\rowlabel{a:Mason01}Mason01 \href{https://doi.org/10.1023/A:1016023415105}{Mason01} & \hyperref[auth:a687]{Andrew J. Mason} & Elastic Constraint Branching, the Wedelin/Carmen Lagrangian Heuristic and Integer Programming for Personnel Scheduling & \href{../works/Mason01.pdf}{Yes} & \cite{Mason01} & 2001 & Annals of Operations Research & 38 & 5 & 0 & \ref{b:Mason01} & \ref{c:Mason01}\\
\rowlabel{a:ArtiguesR00}ArtiguesR00 \href{https://doi.org/10.1016/S0377-2217(99)00496-8}{ArtiguesR00} & \hyperref[auth:a6]{C. Artigues}, \hyperref[auth:a720]{F. Roubellat} & A polynomial activity insertion algorithm in a multi-resource schedule with cumulative constraints and multiple modes & \href{../works/ArtiguesR00.pdf}{Yes} & \cite{ArtiguesR00} & 2000 & European Journal of Operational Research & 20 & 84 & 3 & \ref{b:ArtiguesR00} & \ref{c:ArtiguesR00}\\
\rowlabel{a:BaptisteP00}BaptisteP00 \href{https://doi.org/10.1023/A:1009822502231}{BaptisteP00} & \hyperref[auth:a163]{P. Baptiste}, \hyperref[auth:a164]{Claude Le Pape} & Constraint Propagation and Decomposition Techniques for Highly Disjunctive and Highly Cumulative Project Scheduling Problems & \href{../works/BaptisteP00.pdf}{Yes} & \cite{BaptisteP00} & 2000 & Constraints An Int. J. & 21 & 46 & 0 & \ref{b:BaptisteP00} & \ref{c:BaptisteP00}\\
\rowlabel{a:BeckF00}BeckF00 \href{https://doi.org/10.1016/S0004-3702(99)00099-5}{BeckF00} & \hyperref[auth:a89]{J. Christopher Beck}, \hyperref[auth:a304]{Mark S. Fox} & Dynamic problem structure analysis as a basis for constraint-directed scheduling heuristics & \href{../works/BeckF00.pdf}{Yes} & \cite{BeckF00} & 2000 & Artificial Intelligence & 51 & 24 & 19 & \ref{b:BeckF00} & \ref{c:BeckF00}\\
\rowlabel{a:BeckF00a}BeckF00a \href{http://dx.doi.org/10.1016/s0004-3702(00)00035-7}{BeckF00a} & \hyperref[auth:a89]{J. Christopher Beck}, \hyperref[auth:a304]{Mark S. Fox} & Constraint-directed techniques for scheduling alternative activities & No & \cite{BeckF00a} & 2000 & Artificial Intelligence & null & 48 & 10 & No & \ref{c:BeckF00a}\\
\rowlabel{a:BruckerK00}BruckerK00 \href{http://dx.doi.org/10.1016/s0377-2217(99)00489-0}{BruckerK00} & \hyperref[auth:a856]{P. Brucker}, \hyperref[auth:a1190]{S. Knust} & A linear programming and constraint propagation-based lower bound for the RCPSP & No & \cite{BruckerK00} & 2000 & European Journal of Operational Research & null & 66 & 8 & No & \ref{c:BruckerK00}\\
\rowlabel{a:Dorndorf2000}Dorndorf2000 \href{http://dx.doi.org/10.1016/s0004-3702(00)00040-0}{Dorndorf2000} & \hyperref[auth:a913]{U. Dorndorf}, \hyperref[auth:a443]{E. Pesch}, \hyperref[auth:a1064]{T. Phan-Huy} & Constraint propagation techniques for the disjunctive scheduling problem & No & \cite{Dorndorf2000} & 2000 & Artificial Intelligence & null & 47 & 33 & No & \ref{c:Dorndorf2000}\\
\rowlabel{a:HarjunkoskiJG00}HarjunkoskiJG00 \href{http://dx.doi.org/10.1016/s0098-1354(00)00470-1}{HarjunkoskiJG00} & \hyperref[auth:a880]{I. Harjunkoski}, \hyperref[auth:a853]{V. Jain}, \hyperref[auth:a1181]{Ignacio E. Grossman} & Hybrid mixed-integer/constraint logic programming strategies for solving scheduling and combinatorial optimization problems & No & \cite{HarjunkoskiJG00} & 2000 & Computers \  Chemical Engineering & null & 44 & 3 & No & \ref{c:HarjunkoskiJG00}\\
\rowlabel{a:HeipckeCCS00}HeipckeCCS00 \href{https://doi.org/10.1023/A:1009860311452}{HeipckeCCS00} & \hyperref[auth:a168]{S. Heipcke}, \hyperref[auth:a169]{Y. Colombani}, \hyperref[auth:a170]{Cristina C. B. Cavalcante}, \hyperref[auth:a171]{Cid C. de Souza} & Scheduling under Labour Resource Constraints & \href{../works/HeipckeCCS00.pdf}{Yes} & \cite{HeipckeCCS00} & 2000 & Constraints An Int. J. & 8 & 5 & 0 & \ref{b:HeipckeCCS00} & \ref{c:HeipckeCCS00}\\
\rowlabel{a:HookerOTK00}HookerOTK00 \href{http://dx.doi.org/10.1017/s0269888900001077}{HookerOTK00} & \hyperref[auth:a1212]{J. HOOKER}, \hyperref[auth:a1213]{G. OTTOSSON}, \hyperref[auth:a1214]{ERLENDER S. THORSTEINSSON}, \hyperref[auth:a1215]{H. KIM} & A scheme for unifying optimization and constraint satisfaction methods & No & \cite{HookerOTK00} & 2000 & The Knowledge Engineering Review & null & 30 & 0 & No & \ref{c:HookerOTK00}\\
\rowlabel{a:KorbaaYG00}KorbaaYG00 \href{https://doi.org/10.1016/S0947-3580(00)71113-7}{KorbaaYG00} & \hyperref[auth:a688]{O. Korbaa}, \hyperref[auth:a689]{P. Yim}, \hyperref[auth:a690]{J. Gentina} & Solving Transient Scheduling Problems with Constraint Programming & \href{../works/KorbaaYG00.pdf}{Yes} & \cite{KorbaaYG00} & 2000 & Eur. J. Control & 10 & 7 & 4 & \ref{b:KorbaaYG00} & \ref{c:KorbaaYG00}\\
\rowlabel{a:LopezAKYG00}LopezAKYG00 \href{https://doi.org/10.1016/S0947-3580(00)71114-9}{LopezAKYG00} & \hyperref[auth:a3]{P. Lopez}, \hyperref[auth:a691]{H. Alla}, \hyperref[auth:a688]{O. Korbaa}, \hyperref[auth:a689]{P. Yim}, \hyperref[auth:a690]{J. Gentina} & Discussion on: 'Solving Transient Scheduling Problems with Constraint Programming' by O. Korbaa, P. Yim, and {J.-C.} Gentina & \href{../works/LopezAKYG00.pdf}{Yes} & \cite{LopezAKYG00} & 2000 & Eur. J. Control & 4 & 0 & 0 & \ref{b:LopezAKYG00} & \ref{c:LopezAKYG00}\\
\rowlabel{a:SakkoutW00}SakkoutW00 \href{https://doi.org/10.1023/A:1009856210543}{SakkoutW00} & \hyperref[auth:a167]{Hani El Sakkout}, \hyperref[auth:a117]{Mark G. Wallace} & Probe Backtrack Search for Minimal Perturbation in Dynamic Scheduling & \href{../works/SakkoutW00.pdf}{Yes} & \cite{SakkoutW00} & 2000 & Constraints An Int. J. & 30 & 73 & 0 & \ref{b:SakkoutW00} & \ref{c:SakkoutW00}\\
\rowlabel{a:SchildW00}SchildW00 \href{https://doi.org/10.1023/A:1009804226473}{SchildW00} & \hyperref[auth:a165]{K. Schild}, \hyperref[auth:a166]{J. W{\"{u}}rtz} & Scheduling of Time-Triggered Real-Time Systems & \href{../works/SchildW00.pdf}{Yes} & \cite{SchildW00} & 2000 & Constraints An Int. J. & 23 & 23 & 0 & \ref{b:SchildW00} & \ref{c:SchildW00}\\
\rowlabel{a:SimonisCK00}SimonisCK00 \href{https://doi.org/10.1109/5254.820326}{SimonisCK00} & \hyperref[auth:a17]{H. Simonis}, \hyperref[auth:a895]{P. Charlier}, \hyperref[auth:a896]{P. Kay} & Constraint Handling in an Integrated Transportation Problem & \href{../works/SimonisCK00.pdf}{Yes} & \cite{SimonisCK00} & 2000 & {IEEE} Intell. Syst. & 7 & 11 & 5 & \ref{b:SimonisCK00} & \ref{c:SimonisCK00}\\
\rowlabel{a:SourdN00}SourdN00 \href{https://doi.org/10.1287/ijoc.12.4.341.11881}{SourdN00} & \hyperref[auth:a783]{F. Sourd}, \hyperref[auth:a664]{W. Nuijten} & Multiple-Machine Lower Bounds for Shop-Scheduling Problems & \href{../works/SourdN00.pdf}{Yes} & \cite{SourdN00} & 2000 & INFORMS Journal on Computing & 12 & 7 & 14 & \ref{b:SourdN00} & \ref{c:SourdN00}\\
\rowlabel{a:TorresL00}TorresL00 \href{http://dx.doi.org/10.1016/s0377-2217(99)00497-x}{TorresL00} & \hyperref[auth:a882]{P. Torres}, \hyperref[auth:a3]{P. Lopez} & On Not-First/Not-Last conditions in disjunctive scheduling & \href{../works/TorresL00.pdf}{Yes} & \cite{TorresL00} & 2000 & European Journal of Operational Research & 12 & 26 & 13 & \ref{b:TorresL00} & \ref{c:TorresL00}\\
\rowlabel{a:BaptistePN99}BaptistePN99 \href{http://dx.doi.org/10.1023/a:1018995000688}{BaptistePN99} & \hyperref[auth:a163]{P. Baptiste}, \hyperref[auth:a164]{Claude Le Pape}, \hyperref[auth:a664]{W. Nuijten} & Satisfiability tests and time-bound adjustments for cumulative scheduling problems & \href{../works/BaptistePN99.pdf}{Yes} & \cite{BaptistePN99} & 1999 & Annals of Operations Research & 29 & 72 & 0 & \ref{b:BaptistePN99} & \ref{c:BaptistePN99}\\
\rowlabel{a:BensanaLV99}BensanaLV99 \href{https://doi.org/10.1023/A:1026488509554}{BensanaLV99} & \hyperref[auth:a172]{E. Bensana}, \hyperref[auth:a173]{M. Lema{\^{\i}}tre}, \hyperref[auth:a174]{G. Verfaillie} & Earth Observation Satellite Management & \href{../works/BensanaLV99.pdf}{Yes} & \cite{BensanaLV99} & 1999 & Constraints An Int. J. & 7 & 99 & 0 & \ref{b:BensanaLV99} & \ref{c:BensanaLV99}\\
\rowlabel{a:HookerO99}HookerO99 \href{http://dx.doi.org/10.1016/s0166-218x(99)00100-6}{HookerO99} & \hyperref[auth:a1172]{J. Hooker}, \hyperref[auth:a1173]{M. Osorio} & Mixed logical-linear programming & No & \cite{HookerO99} & 1999 & Discrete Applied Mathematics & null & 92 & 48 & No & \ref{c:HookerO99}\\
\rowlabel{a:JainM99}JainM99 \href{http://dx.doi.org/10.1016/s0377-2217(98)00113-1}{JainM99} & \hyperref[auth:a967]{A. Jain}, \hyperref[auth:a968]{S. Meeran} & Deterministic job-shop scheduling: Past, present and future & \href{../works/JainM99.pdf}{Yes} & \cite{JainM99} & 1999 & European Journal of Operational Research & 45 & 490 & 150 & \ref{b:JainM99} & \ref{c:JainM99}\\
\rowlabel{a:PesantGPR99}PesantGPR99 \href{http://dx.doi.org/10.1016/s0377-2217(98)00248-3}{PesantGPR99} & \hyperref[auth:a8]{G. Pesant}, \hyperref[auth:a624]{M. Gendreau}, \hyperref[auth:a1228]{J. Potvin}, \hyperref[auth:a1229]{J. Rousseau} & On the flexibility of constraint programming models: From single to multiple time windows for the traveling salesman problem & No & \cite{PesantGPR99} & 1999 & European Journal of Operational Research & null & 26 & 18 & No & \ref{c:PesantGPR99}\\
\rowlabel{a:RodosekWH99}RodosekWH99 \href{http://dx.doi.org/10.1023/a:1018904229454}{RodosekWH99} & \hyperref[auth:a299]{R. Rodosek}, \hyperref[auth:a117]{Mark G. Wallace}, \hyperref[auth:a1048]{M. Hajian} & A new approach to integrating mixed integer programming and constraint logic programming & No & \cite{RodosekWH99} & 1999 & Annals of Operations Research & null & 53 & 0 & No & \ref{c:RodosekWH99}\\
\rowlabel{a:BeckDDF98}BeckDDF98 \href{http://dx.doi.org/10.1002/(sici)1099-1425(199808)1:2<89::aid-jos9>3.0.co;2-h}{BeckDDF98} & \hyperref[auth:a89]{J. Christopher Beck}, \hyperref[auth:a250]{Andrew J. Davenport}, \hyperref[auth:a1244]{Eugene D. Davis}, \hyperref[auth:a304]{Mark S. Fox} & The ODO project: toward a unified basis for constraint-directed scheduling & No & \cite{BeckDDF98} & 1998 & Journal of Scheduling & null & 9 & 0 & No & \ref{c:BeckDDF98}\\
\rowlabel{a:BeckF98}BeckF98 \href{https://doi.org/10.1609/aimag.v19i4.1426}{BeckF98} & \hyperref[auth:a89]{J. Christopher Beck}, \hyperref[auth:a304]{Mark S. Fox} & A Generic Framework for Constraint-Directed Search and Scheduling & \href{../works/BeckF98.pdf}{Yes} & \cite{BeckF98} & 1998 & {AI} Mag. & 30 & 0 & 0 & \ref{b:BeckF98} & \ref{c:BeckF98}\\
\rowlabel{a:BelhadjiI98}BelhadjiI98 \href{https://doi.org/10.1023/A:1009777711218}{BelhadjiI98} & \hyperref[auth:a175]{S. Belhadji}, \hyperref[auth:a176]{A. Isli} & Temporal Constraint Satisfaction Techniques in Job Shop Scheduling Problem Solving & \href{../works/BelhadjiI98.pdf}{Yes} & \cite{BelhadjiI98} & 1998 & Constraints An Int. J. & 9 & 3 & 0 & \ref{b:BelhadjiI98} & \ref{c:BelhadjiI98}\\
\rowlabel{a:BockmayrK98}BockmayrK98 \href{http://dx.doi.org/10.1287/ijoc.10.3.287}{BockmayrK98} & \hyperref[auth:a918]{A. Bockmayr}, \hyperref[auth:a1063]{T. Kasper} & Branch and Infer: A Unifying Framework for Integer and Finite Domain Constraint Programming & No & \cite{BockmayrK98} & 1998 & INFORMS Journal on Computing & null & 79 & 27 & No & \ref{c:BockmayrK98}\\
\rowlabel{a:DarbyDowmanL98}DarbyDowmanL98 \href{http://dx.doi.org/10.1287/ijoc.10.3.276}{DarbyDowmanL98} & \hyperref[auth:a1089]{K. Darby-Dowman}, \hyperref[auth:a179]{J. Little} & Properties of Some Combinatorial Optimization Problems and Their Effect on the Performance of Integer Programming and Constraint Logic Programming & No & \cite{DarbyDowmanL98} & 1998 & INFORMS Journal on Computing & null & 28 & 6 & No & \ref{c:DarbyDowmanL98}\\
\rowlabel{a:NuijtenP98}NuijtenP98 \href{https://doi.org/10.1023/A:1009687210594}{NuijtenP98} & \hyperref[auth:a664]{W. Nuijten}, \hyperref[auth:a164]{Claude Le Pape} & Constraint-Based Job Shop Scheduling with {\textbackslash}sc Ilog Scheduler & \href{../works/NuijtenP98.pdf}{Yes} & \cite{NuijtenP98} & 1998 & J. Heuristics & 16 & 42 & 0 & \ref{b:NuijtenP98} & \ref{c:NuijtenP98}\\
\rowlabel{a:PapaB98}PapaB98 \href{https://doi.org/10.1023/A:1009723704757}{PapaB98} & \hyperref[auth:a164]{Claude Le Pape}, \hyperref[auth:a163]{P. Baptiste} & Resource Constraints for Preemptive Job-shop Scheduling & \href{../works/PapaB98.pdf}{Yes} & \cite{PapaB98} & 1998 & Constraints An Int. J. & 25 & 14 & 0 & \ref{b:PapaB98} & \ref{c:PapaB98}\\
\rowlabel{a:Darby-DowmanLMZ97}Darby-DowmanLMZ97 \href{https://doi.org/10.1007/BF00137871}{Darby-DowmanLMZ97} & \hyperref[auth:a178]{K. Darby{-}Dowman}, \hyperref[auth:a179]{J. Little}, \hyperref[auth:a180]{G. Mitra}, \hyperref[auth:a181]{M. Zaffalon} & Constraint Logic Programming and Integer Programming Approaches and Their Collaboration in Solving an Assignment Scheduling Problem & \href{../works/Darby-DowmanLMZ97.pdf}{Yes} & \cite{Darby-DowmanLMZ97} & 1997 & Constraints An Int. J. & 20 & 28 & 5 & \ref{b:Darby-DowmanLMZ97} & \ref{c:Darby-DowmanLMZ97}\\
\rowlabel{a:FalaschiGMP97}FalaschiGMP97 \href{https://doi.org/10.1006/inco.1997.2638}{FalaschiGMP97} & \hyperref[auth:a695]{M. Falaschi}, \hyperref[auth:a197]{M. Gabbrielli}, \hyperref[auth:a696]{K. Marriott}, \hyperref[auth:a697]{C. Palamidessi} & Constraint Logic Programming with Dynamic Scheduling: {A} Semantics Based on Closure Operators & \href{../works/FalaschiGMP97.pdf}{Yes} & \cite{FalaschiGMP97} & 1997 & Inf. Comput. & 27 & 10 & 9 & \ref{b:FalaschiGMP97} & \ref{c:FalaschiGMP97}\\
\rowlabel{a:LammaMM97}LammaMM97 \href{https://doi.org/10.1016/S0954-1810(96)00002-7}{LammaMM97} & \hyperref[auth:a728]{E. Lamma}, \hyperref[auth:a729]{P. Mello}, \hyperref[auth:a144]{M. Milano} & A distributed constraint-based scheduler & \href{../works/LammaMM97.pdf}{Yes} & \cite{LammaMM97} & 1997 & Artif. Intell. Eng. & 15 & 11 & 7 & \ref{b:LammaMM97} & \ref{c:LammaMM97}\\
\rowlabel{a:Zhou97}Zhou97 \href{https://doi.org/10.1023/A:1009757726572}{Zhou97} & \hyperref[auth:a177]{J. Zhou} & A Permutation-Based Approach for Solving the Job-Shop Problem & \href{../works/Zhou97.pdf}{Yes} & \cite{Zhou97} & 1997 & Constraints An Int. J. & 29 & 14 & 0 & \ref{b:Zhou97} & \ref{c:Zhou97}\\
\rowlabel{a:BlazewiczDP96}BlazewiczDP96 \href{http://dx.doi.org/10.1016/0377-2217(95)00362-2}{BlazewiczDP96} & \hyperref[auth:a988]{J. Błażewicz}, \hyperref[auth:a989]{W. Domschke}, \hyperref[auth:a443]{E. Pesch} & The job shop scheduling problem: Conventional and new solution techniques & \href{../works/BlazewiczDP96.pdf}{Yes} & \cite{BlazewiczDP96} & 1996 & European Journal of Operational Research & 33 & 344 & 127 & \ref{b:BlazewiczDP96} & \ref{c:BlazewiczDP96}\\
\rowlabel{a:NuijtenA96}NuijtenA96 \href{http://dx.doi.org/10.1016/0377-2217(95)00354-1}{NuijtenA96} & \hyperref[auth:a664]{W. Nuijten}, \hyperref[auth:a785]{E. Aarts} & A computational study of constraint satisfaction for multiple capacitated job shop scheduling & \href{../works/NuijtenA96.pdf}{Yes} & \cite{NuijtenA96} & 1996 & European Journal of Operational Research & 16 & 65 & 6 & \ref{b:NuijtenA96} & \ref{c:NuijtenA96}\\
\rowlabel{a:PeschT96}PeschT96 \href{http://dx.doi.org/10.1287/ijoc.8.2.144}{PeschT96} & \hyperref[auth:a443]{E. Pesch}, \hyperref[auth:a1242]{Ulrich A. W. Tetzlaff} & Constraint Propagation Based Scheduling of Job Shops & No & \cite{PeschT96} & 1996 & INFORMS Journal on Computing & null & 22 & 0 & No & \ref{c:PeschT96}\\
\rowlabel{a:SadehF96}SadehF96 \href{http://dx.doi.org/10.1016/0004-3702(95)00098-4}{SadehF96} & \hyperref[auth:a1189]{N. Sadeh}, \hyperref[auth:a304]{Mark S. Fox} & Variable and value ordering heuristics for the job shop scheduling constraint satisfaction problem & No & \cite{SadehF96} & 1996 & Artificial Intelligence & null & 95 & 17 & No & \ref{c:SadehF96}\\
\rowlabel{a:SmithBHW96}SmithBHW96 \href{http://dx.doi.org/10.1007/bf00143880}{SmithBHW96} & \hyperref[auth:a1071]{Barbara M. Smith}, \hyperref[auth:a1069]{Sally C. Brailsford}, \hyperref[auth:a1203]{Peter M. Hubbard}, \hyperref[auth:a1204]{H. Paul Williams} & The progressive party problem: Integer linear programming and constraint programming compared & No & \cite{SmithBHW96} & 1996 & Constraints An Int. J. & null & 56 & 4 & No & \ref{c:SmithBHW96}\\
\rowlabel{a:Wallace96}Wallace96 \href{https://doi.org/10.1007/BF00143881}{Wallace96} & \hyperref[auth:a117]{Mark G. Wallace} & Practical Applications of Constraint Programming & \href{../works/Wallace96.pdf}{Yes} & \cite{Wallace96} & 1996 & Constraints An Int. J. & 30 & 87 & 55 & \ref{b:Wallace96} & \ref{c:Wallace96}\\
\rowlabel{a:WeilHFP95}WeilHFP95 \href{http://dx.doi.org/10.1109/51.395324}{WeilHFP95} & \hyperref[auth:a1217]{G. Weil}, \hyperref[auth:a1218]{K. Heus}, \hyperref[auth:a1219]{P. Francois}, \hyperref[auth:a1220]{M. Poujade} & Constraint programming for nurse scheduling & No & \cite{WeilHFP95} & 1995 & IEEE Engineering in Medicine and Biology Magazine & null & 56 & 9 & No & \ref{c:WeilHFP95}\\
\rowlabel{a:BeldiceanuC94}BeldiceanuC94 \href{https://www.sciencedirect.com/science/article/pii/0895717794901279}{BeldiceanuC94} & \hyperref[auth:a129]{N. Beldiceanu}, \hyperref[auth:a792]{E. Contejean} & Introducing Global Constraints in {CHIP} & \href{../works/BeldiceanuC94.pdf}{Yes} & \cite{BeldiceanuC94} & 1994 & Mathematical and Computer Modelling & 27 & 167 & 8 & \ref{b:BeldiceanuC94} & \ref{c:BeldiceanuC94}\\
\rowlabel{a:Pape94}Pape94 \href{http://dx.doi.org/10.1049/ise.1994.0009}{Pape94} & \hyperref[auth:a164]{Claude Le Pape} & Implementation of resource constraints in ILOG SCHEDULE: a library for the development of constraint-based scheduling systems & \href{../works/Pape94.pdf}{Yes} & \cite{Pape94} & 1994 & Intelligent Systems Engineering & 34 & 98 & 0 & \ref{b:Pape94} & \ref{c:Pape94}\\
\rowlabel{a:AggounB93}AggounB93 \href{https://www.sciencedirect.com/science/article/pii/089571779390068A}{AggounB93} & \hyperref[auth:a733]{A. Aggoun}, \hyperref[auth:a129]{N. Beldiceanu} & Extending {CHIP} in order to solve complex scheduling and placement problems & \href{../works/AggounB93.pdf}{Yes} & \cite{AggounB93} & 1993 & Mathematical and Computer Modelling & 17 & 187 & 11 & \ref{b:AggounB93} & \ref{c:AggounB93}\\
\rowlabel{a:MintonJPL92}MintonJPL92 \href{http://dx.doi.org/10.1016/0004-3702(92)90007-k}{MintonJPL92} & \hyperref[auth:a1236]{S. Minton}, \hyperref[auth:a1237]{Mark D. Johnston}, \hyperref[auth:a1238]{Andrew B. Philips}, \hyperref[auth:a1239]{P. Laird} & Minimizing conflicts: a heuristic repair method for constraint satisfaction and scheduling problems & No & \cite{MintonJPL92} & 1992 & Artificial Intelligence & null & 437 & 13 & No & \ref{c:MintonJPL92}\\
\rowlabel{a:Tay92}Tay92 \href{}{Tay92} & \hyperref[auth:a709]{David B. H. Tay} & {COPS:} {A} Constraint Programming Approach to Resource-Limited Project Scheduling & No & \cite{Tay92} & 1992 & Comput. J. & null & 0 & 0 & No & \ref{c:Tay92}\\
\rowlabel{a:DincbasSH90}DincbasSH90 \href{https://doi.org/10.1016/0743-1066(90)90052-7}{DincbasSH90} & \hyperref[auth:a725]{M. Dincbas}, \hyperref[auth:a17]{H. Simonis}, \hyperref[auth:a149]{Pascal Van Hentenryck} & Solving Large Combinatorial Problems in Logic Programming & \href{../works/DincbasSH90.pdf}{Yes} & \cite{DincbasSH90} & 1990 & The Journal of Logic Programming & 19 & 86 & 9 & \ref{b:DincbasSH90} & \ref{c:DincbasSH90}\\
\rowlabel{a:Davis87}Davis87 \href{http://dx.doi.org/10.1016/0004-3702(87)90091-9}{Davis87} & \hyperref[auth:a1241]{E. Davis} & Constraint propagation with interval labels & No & \cite{Davis87} & 1987 & Artificial Intelligence & null & 308 & 21 & No & \ref{c:Davis87}\\
\end{longtable}
}




\clearpage
\subsection{Extracted Concepts}
{\scriptsize
\begin{longtable}{>{\raggedright\arraybackslash}p{3cm}r>{\raggedright\arraybackslash}p{4cm}p{1.5cm}p{2cm}p{1.5cm}p{1.5cm}p{1.5cm}p{1.5cm}p{2cm}p{1.5cm}rr}
\rowcolor{white}\caption{Automatically Extracted ARTICLE Features (Requires Local Copy)}\\ \toprule
\rowcolor{white}Work & Pages & Concepts & Classification & Constraints & \shortstack{Prog\\Languages} & \shortstack{CP\\Systems} & Areas & Industries & Benchmarks & Algorithm & a & c\\ \midrule\endhead
\bottomrule
\endfoot
\index{AbohashimaEG21}\rowlabel{b:AbohashimaEG21}\href{../works/AbohashimaEG21.pdf}{AbohashimaEG21}~\cite{AbohashimaEG21} & 14 & setup-time, machine, multi-objective, scheduling, order, stochastic, CP, resource, explanation, cmax, transportation & parallel machine & cycle & Python & Gurobi &  &  & real-world, generated instance, github & genetic algorithm, ant colony, memetic algorithm, machine learning, meta heuristic & \ref{a:AbohashimaEG21} & \ref{c:AbohashimaEG21}\\
\index{AbreuAPNM21}\rowlabel{b:AbreuAPNM21}\href{../works/AbreuAPNM21.pdf}{AbreuAPNM21}~\cite{AbreuAPNM21} & 20 & multi-objective, make-span, open-shop, order, machine, preempt, multi-agent, breakdown, cmax, tardiness, periodic, scheduling, no-wait, job-shop, distributed, job, resource, order scheduling, release-date, preemptive, completion-time, setup-time, CP, task, stochastic, constraint programming, precedence, flow-shop & parallel machine, OSSP, single machine, Open Shop Scheduling Problem & cycle, noOverlap & Python, C++ & Cplex & medical, patient, automotive & oil industry & generated instance, benchmark, real-world & mat heuristic, meta heuristic, genetic algorithm, simulated annealing, particle swarm, large neighborhood search & \ref{a:AbreuAPNM21} & n/a\\
\index{AbreuN22}\rowlabel{b:AbreuN22}\href{../works/AbreuN22.pdf}{AbreuN22}~\cite{AbreuN22} & 20 & CP, make-span, flow-time, distributed, resource, batch process, cmax, preemptive, order, tardiness, scheduling, multi-objective, completion-time, machine, job, task, no-wait, open-shop, transportation, stochastic, constraint programming, job-shop, flow-shop, bi-objective, preempt, inventory, setup-time & OSSP, single machine, Open Shop Scheduling Problem & cumulative, noOverlap, cycle & Python & Cplex & medical & chips industry & real-world, benchmark & meta heuristic, simulated annealing, mat heuristic, large neighborhood search, Lagrangian relaxation, ant colony, particle swarm, genetic algorithm & \ref{a:AbreuN22} & \ref{c:AbreuN22}\\
\index{AbreuNP23}\rowlabel{b:AbreuNP23}\href{../works/AbreuNP23.pdf}{AbreuNP23}~\cite{AbreuNP23} & 20 & order, completion-time, distributed, job-shop, resource, job, preempt, setup-time, no-wait, scheduling, make-span, tardiness, earliness, two-machine scheduling, energy efficiency, flow-shop, cmax, CP, machine, blocking constraint, stochastic, constraint programming, transportation, open-shop & OSSP, parallel machine, Open Shop Scheduling Problem & noOverlap, Blocking constraint & Python & Cplex, OPL & medical & oil industry & real-world, benchmark & genetic algorithm, mat heuristic, meta heuristic, simulated annealing, time-tabling, large neighborhood search & \ref{a:AbreuNP23} & \ref{c:AbreuNP23}\\
\index{AbreuPNF23}\rowlabel{b:AbreuPNF23}\href{../works/AbreuPNF23.pdf}{AbreuPNF23}~\cite{AbreuPNF23} & 12 & distributed, preemptive, job-shop, due-date, no-wait, flow-shop, constraint programming, completion-time, CP, stochastic, sustainability, setup-time, open-shop, order, order scheduling, multi-objective, transportation, bi-objective, job, scheduling, machine, make-span, periodic, tardiness, earliness, preempt, resource & RCPSP, parallel machine, OSSP, Open Shop Scheduling Problem & cumulative, disjunctive, noOverlap & Python & Cplex, OPL & medical, robot &  & real-life, benchmark, real-world & lazy clause generation, meta heuristic, ant colony, NEH, large neighborhood search, mat heuristic, simulated annealing, genetic algorithm & \ref{a:AbreuPNF23} & n/a\\
\index{Adelgren2023}\rowlabel{b:Adelgren2023}\href{../works/Adelgren2023.pdf}{Adelgren2023}~\cite{Adelgren2023} & 12 & setup-time, preempt, order, inventory, distributed, CP, resource, task, constraint programming, preemptive, release-date, sequence dependent setup, job-shop, transportation, periodic, batch process, completion-time, scheduling, machine, job, re-scheduling, make-span & parallel machine & disjunctive &  & Gurobi, Cplex & drone, crew-scheduling, aircraft, operating room, pipeline &  & benchmark, real-life, github, generated instance, supplementary material & MINLP, column generation & \ref{a:Adelgren2023} & \ref{c:Adelgren2023}\\
\index{AfsarVPG23}\rowlabel{b:AfsarVPG23}\href{../works/AfsarVPG23.pdf}{AfsarVPG23}~\cite{AfsarVPG23} & 14 & transportation, bi-objective, job, constraint programming, precedence, stochastic, task, setup-time, machine, multi-objective, scheduling, preempt, make-span, resource, job-shop, due-date, activity, flow-shop, completion-time, CP, open-shop, order &  & disjunctive &  & Cplex &  &  & benchmark, real-life, supplementary material, real-world & reinforcement learning, genetic algorithm, meta heuristic, memetic algorithm, neural network, particle swarm & \ref{a:AfsarVPG23} & \ref{c:AfsarVPG23}\\
\index{AggounB93}\rowlabel{b:AggounB93}\href{../works/AggounB93.pdf}{AggounB93}~\cite{AggounB93} & 17 & job-shop, resource, order, activity, constraint satisfaction, scheduling, task, constraint logic programming, job, due-date, flow-shop, machine, precedence, CLP &  & Cardinality constraint, circuit, Arithmetic constraint, disjunctive, Disjunctive constraint, bin-packing, Among constraint, cumulative & Prolog & CHIP, OPL & perfect-square, rectangle-packing &  & real-world & simulated annealing & \ref{a:AggounB93} & n/a\\
\index{AkramNHRSA23}\rowlabel{b:AkramNHRSA23}\href{../works/AkramNHRSA23.pdf}{AkramNHRSA23}~\cite{AkramNHRSA23} & 16 & CP, resource, task, constraint programming, periodic, completion-time, scheduling, machine, preempt, order, distributed &  & cycle, bin-packing & Python & OR-Tools & agriculture, medical &  & benchmark & genetic algorithm, reinforcement learning, machine learning, deep learning, GRASP, ant colony, simulated annealing & \ref{a:AkramNHRSA23} & \ref{c:AkramNHRSA23}\\
\index{Alaka21}\rowlabel{b:Alaka21}\href{../works/Alaka21.pdf}{Alaka21}~\cite{Alaka21} & 11 & CP, precedence, constraint programming, multi-objective, stochastic, completion-time, task, resource, cyclic scheduling, order, machine, scheduling &  & cycle &  &  &  &  & real-life & meta heuristic, genetic algorithm & \ref{a:Alaka21} & n/a\\
\index{AlakaP23}\rowlabel{b:AlakaP23}\href{../works/AlakaP23.pdf}{AlakaP23}~\cite{AlakaP23} & 14 & CP, make-span, resource, cyclic scheduling, precedence, order, scheduling, machine, task, stochastic, multi-objective, constraint programming &  & cycle &  & Cplex & robot & garment industry & real-life & meta heuristic, ant colony, time-tabling, genetic algorithm & \ref{a:AlakaP23} & n/a\\
\index{AlakaPY19}\rowlabel{b:AlakaPY19}\href{../works/AlakaPY19.pdf}{AlakaPY19}~\cite{AlakaPY19} & 9 & scheduling, precedence, CP, machine, stochastic, constraint programming, order, completion-time, cyclic scheduling, multi-objective, task, resource &  & bin-packing, cycle &  & Cplex &  &  & real-life & genetic algorithm & \ref{a:AlakaPY19} & n/a\\
\index{AlesioBNG15}\rowlabel{b:AlesioBNG15}\href{../works/AlesioBNG15.pdf}{AlesioBNG15}~\cite{AlesioBNG15} & 37 & multi-objective, open-shop, order, machine, preempt, periodic, scheduling, resource, job-shop, distributed, constraint optimization, job, activity, constraint programming, preemptive, completion-time, CP, task &  &  &  & Cplex, Ilog Solver, OPL & aircraft, satellite, telescope, automotive &  & benchmark, industrial partner & meta heuristic, genetic algorithm, Lagrangian relaxation, memetic algorithm & \ref{a:AlesioBNG15} & n/a\\
\index{AlfieriGPS23}\rowlabel{b:AlfieriGPS23}\href{../works/AlfieriGPS23.pdf}{AlfieriGPS23}~\cite{AlfieriGPS23} & 13 & stochastic, constraint programming, flow-time, completion-time, precedence, earliness, scheduling, machine, transportation, tardiness, make-span, inventory, flow-shop, job, CP, Benders Decomposition, multi-objective, setup-time, single-machine scheduling, order, distributed, no-wait, job-shop, resource & single machine, parallel machine &  & Java & Cplex & surgery, patient &  & benchmark & NEH, memetic algorithm, ant colony, meta heuristic, mat heuristic, particle swarm & \ref{a:AlfieriGPS23} & n/a\\
\index{AntunesABD20}\rowlabel{b:AntunesABD20}\href{../works/AntunesABD20.pdf}{AntunesABD20}~\cite{AntunesABD20} & 31 & precedence, earliness, scheduling, transportation, periodic, planned maintenance, activity, due-date, re-scheduling, CP, Benders Decomposition, COP, order, distributed, lateness, stochastic, task, constraint programming &  & bin-packing &  & Cplex & workforce scheduling, maintenance scheduling & electricity industry & real-world, industrial partner & column generation, genetic algorithm, meta heuristic & \ref{a:AntunesABD20} & n/a\\
\index{ArkhipovBL19}\rowlabel{b:ArkhipovBL19}\href{../works/ArkhipovBL19.pdf}{ArkhipovBL19}~\cite{ArkhipovBL19} & 10 & cmax, task, constraint programming, preemptive, completion-time, release-date, precedence, job-shop, constraint satisfaction, scheduling, machine, job, CP, make-span, preempt, order, lateness, resource & parallel machine, Resource-constrained Project Scheduling Problem, psplib, RCPSP & cycle, Cumulatives constraint, cumulative, disjunctive &  & Z3 &  &  & benchmark & sweep, time-tabling & \ref{a:ArkhipovBL19} & n/a\\
\index{ArtiguesF07}\rowlabel{b:ArtiguesF07}\href{../works/ArtiguesF07.pdf}{ArtiguesF07}~\cite{ArtiguesF07} & 25 & constraint satisfaction, constraint programming, precedence, batch process, sequence dependent setup, make-span, order, machine, preempt, cmax, tardiness, scheduling, resource, job-shop, job, preemptive, completion-time, setup-time, one-machine scheduling, CP & single machine, Resource-constrained Project Scheduling Problem & cycle, disjunctive, Disjunctive constraint & C++ & Ilog Scheduler & semiconductor &  & benchmark & large neighborhood search, meta heuristic, edge-finding, genetic algorithm & \ref{a:ArtiguesF07} & n/a\\
\index{ArtiguesL14}\rowlabel{b:ArtiguesL14}\href{../works/ArtiguesL14.pdf}{ArtiguesL14}~\cite{ArtiguesL14} & 17 & task, constraint programming, precedence, make-span, order, preempt, scheduling, re-scheduling, resource, due-date, CSP, transportation, activity, release-date, preemptive, CP & CECSP, CuSP, RCPSP & cumulative &  &  &  &  &  & energetic reasoning, lazy clause generation, column generation & \ref{a:ArtiguesL14} & n/a\\
\index{ArtiguesLH13}\rowlabel{b:ArtiguesLH13}\href{../works/ArtiguesLH13.pdf}{ArtiguesLH13}~\cite{ArtiguesLH13} & 11 & setup-time, preempt, order, CP, resource, task, constraint programming, preemptive, release-date, precedence, periodic, CSP, tardiness, activity, completion-time, constraint satisfaction, due-date, scheduling, machine, job, make-span, unavailability & CuSP, PMSP, parallel machine, single machine & circuit, cumulative & C++ & Cplex, OPL & electroplating, hoist &  & real-world & energetic reasoning, column generation & \ref{a:ArtiguesLH13} & n/a\\
\index{ArtiguesR00}\rowlabel{b:ArtiguesR00}\href{../works/ArtiguesR00.pdf}{ArtiguesR00}~\cite{ArtiguesR00} & 20 & no preempt, lateness, precedence, make-span, order, machine, preempt, cmax, scheduling, re-scheduling, resource, due-date, job-shop, transportation, job, activity, release-date, completion-time, setup-time, earliness, CP & RCMPSP, RCPSP & cumulative, cycle, disjunctive &  &  &  &  &  & simulated annealing & \ref{a:ArtiguesR00} & n/a\\
\index{AstrandJZ20}\rowlabel{b:AstrandJZ20}\href{../works/AstrandJZ20.pdf}{AstrandJZ20}~\cite{AstrandJZ20} & 13 & task, stochastic, constraint satisfaction, constraint programming, precedence, flow-shop, make-span, order, completion-time, machine, CP, breakdown, periodic, re-scheduling, unavailability, due-date, open-shop, net present value, job, activity, scheduling, resource, CSP, job-shop, setup-time & parallel machine, Resource-constrained Project Scheduling Problem & disjunctive, cycle, alldifferent, Disjunctive constraint & C++ & Gecode & robot & potash industry, mineral industry, mining industry & benchmark, real-world, real-life & large neighborhood search, meta heuristic, genetic algorithm & \ref{a:AstrandJZ20} & n/a\\
\index{AwadMDMT22}\rowlabel{b:AwadMDMT22}\href{../works/AwadMDMT22.pdf}{AwadMDMT22}~\cite{AwadMDMT22} & 22 & order, cmax, tardiness, earliness, scheduling, preempt, task, machine, make-span, preemptive, job-shop, due-date, activity, no-wait, flow-shop, completion-time, lateness, CP, inventory, batch process, resource, breakdown, job, constraint programming, precedence, release-date, stochastic, setup-time & parallel machine & cumulative, disjunctive, span constraint, noOverlap, endBeforeStart, cycle, alternative constraint, alwaysIn &  & OPL, Cplex & crew-scheduling & pharmaceutical industry & real-life, benchmark & simulated annealing, meta heuristic, genetic algorithm & \ref{a:AwadMDMT22} & n/a\\
\index{BadicaBI20}\rowlabel{b:BadicaBI20}\href{../works/BadicaBI20.pdf}{BadicaBI20}~\cite{BadicaBI20} & 17 & constraint programming, precedence, scheduling, task, CLP, machine, make-span, constraint logic programming, manpower, constraint satisfaction, resource, distributed, activity, completion-time, CP, stochastic, order & psplib, Resource-constrained Project Scheduling Problem & bin-packing, Reified constraint, Arithmetic constraint, cycle & Prolog & Gecode, ECLiPSe & business process &  & benchmark, real-world & meta heuristic & \ref{a:BadicaBI20} & n/a\\
\index{BajestaniB13}\rowlabel{b:BajestaniB13}\href{../works/BajestaniB13.pdf}{BajestaniB13}~\cite{BajestaniB13} & 36 & scheduling, machine, transportation, tardiness, make-span, setup-time, preempt, single-machine scheduling, inventory, due-date, job, re-scheduling, CP, Benders Decomposition, stochastic, breakdown, constraint programming, order, preemptive, precedence, earliness, Logic-Based Benders Decomposition, job-shop, resource, periodic, CSP, reactive scheduling & single machine, parallel machine & GCC constraint, alwaysIn, circuit, IloPulse, Cardinality constraint, cumulative, IloAlwaysIn &  & Cplex & railway, maintenance scheduling, aircraft &  &  & meta heuristic, reinforcement learning, machine learning & \ref{a:BajestaniB13} & n/a\\
\index{BajestaniB15}\rowlabel{b:BajestaniB15}\href{../works/BajestaniB15.pdf}{BajestaniB15}~\cite{BajestaniB15} & 16 & completion-time, scheduling, machine, tardiness, make-span, unavailability, planned maintenance, activity, setup-time, preempt, single-machine scheduling, due-date, distributed, flow-shop, job, CP, Benders Decomposition, stochastic, breakdown, constraint programming, flow-time, order, preemptive, precedence, sequence dependent setup, Logic-Based Benders Decomposition, job-shop, resource, periodic & single machine & circuit, disjunctive, cumulative, Disjunctive constraint &  & Cplex & railway, robot, semiconductor, maintenance scheduling & semiconductor industry & real-world & genetic algorithm, meta heuristic & \ref{a:BajestaniB15} & n/a\\
\index{BandaSC11}\rowlabel{b:BandaSC11}\href{../works/BandaSC11.pdf}{BandaSC11}~\cite{BandaSC11} & 18 & precedence, constraint optimization, CSP, CP, constraint programming, order, scheduling, task &  &  &  & Ilog Solver &  &  & benchmark, CSPlib, random instance &  & \ref{a:BandaSC11} & n/a\\
\index{BaptisteB18}\rowlabel{b:BaptisteB18}\href{../works/BaptisteB18.pdf}{BaptisteB18}~\cite{BaptisteB18} & 10 & machine, preempt, CP, scheduling, manpower, job, resource, preemptive, task, constraint satisfaction, constraint programming, precedence, make-span, order & psplib, Resource-constrained Project Scheduling Problem, RCPSP, parallel machine & cumulative, bin-packing &  & CHIP &  &  &  & lazy clause generation, time-tabling, edge-finding, edge-finder, Lagrangian relaxation & \ref{a:BaptisteB18} & n/a\\
\index{BaptisteP00}\rowlabel{b:BaptisteP00}\href{../works/BaptisteP00.pdf}{BaptisteP00}~\cite{BaptisteP00} & 21 & preempt, CP, cmax, scheduling, re-scheduling, due-date, CSP, job, activity, resource, preemptive, job-shop, task, constraint satisfaction, constraint programming, precedence, release-date, flow-shop, make-span, order & RCPSP & cumulative, Disjunctive constraint, disjunctive & C++ & Claire, CHIP, Ilog Scheduler &  &  & benchmark & energetic reasoning, edge-finding, edge-finder & \ref{a:BaptisteP00} & \ref{c:BaptisteP00}\\
\index{BaptistePN99}\rowlabel{b:BaptistePN99}\href{../works/BaptistePN99.pdf}{BaptistePN99}~\cite{BaptistePN99} & 29 & cmax, task, constraint programming, preemptive, release-date, precedence, CLP, activity, constraint satisfaction, due-date, scheduling, flow-shop, machine, job, re-scheduling, CP, make-span, setup-time, preempt, order, job-shop, resource & Resource-constrained Project Scheduling Problem, CuSP, RCPSP & cumulative, disjunctive & C++ & Claire &  &  & benchmark, real-life & energetic reasoning, edge-finding & \ref{a:BaptistePN99} & n/a\\
\index{BartakCS10}\rowlabel{b:BartakCS10}\href{../works/BartakCS10.pdf}{BartakCS10}~\cite{BartakCS10} & 31 & scheduling, constraint satisfaction, job-shop, activity, flow-shop, CP, resource, job, CSP, precedence, task, setup-time, machine, order & RCPSP & disjunctive & Prolog & SICStus &  &  & real-life, real-world, benchmark &  & \ref{a:BartakCS10} & n/a\\
\index{BartakS11}\rowlabel{b:BartakS11}\href{../works/BartakS11.pdf}{BartakS11}~\cite{BartakS11} & 5 & CSP, multi-agent, CP, distributed, constraint programming, resource, constraint optimization, order, constraint satisfaction, scheduling, task, COP, explanation & Resource-constrained Project Scheduling Problem & cumulative &  & OPL &  & software industry & random instance, real-world, real-life &  & \ref{a:BartakS11} & \ref{c:BartakS11}\\
\index{BartakSR08}\rowlabel{b:BartakSR08}\href{../works/BartakSR08.pdf}{BartakSR08}~\cite{BartakSR08} & 11 & job, constraint programming, CSP, precedence, release-date, stochastic, task, machine, order, cmax, tardiness, lateness, scheduling, preempt, make-span, constraint satisfaction, preemptive, job-shop, activity, flow-shop, completion-time, CP, open-shop, resource & single machine & Disjunctive constraint, cumulative, disjunctive &  &  & robot &  & real-life, real-world & not-last, meta heuristic, edge-finding, not-first & \ref{a:BartakSR08} & n/a\\
\index{BartakSR10}\rowlabel{b:BartakSR10}\href{../works/BartakSR10.pdf}{BartakSR10}~\cite{BartakSR10} & 31 & job, constraint programming, CSP, precedence, release-date, stochastic, task, CLP, machine, temporal constraint reasoning, order, cmax, tardiness, lateness, multi-agent, scheduling, preempt, make-span, constraint satisfaction, distributed, preemptive, job-shop, due-date, activity, flow-shop, explanation, completion-time, CP, open-shop, resource & single machine, TCSP, Temporal Constraint Satisfaction Problem & Disjunctive constraint, cumulative, disjunctive &  & CPO, Choco Solver, OPL & meeting scheduling, robot &  & real-life, real-world & sweep, machine learning, not-last, meta heuristic, edge-finding, not-first & \ref{a:BartakSR10} & n/a\\
\index{Beck07}\rowlabel{b:Beck07}\href{../works/Beck07.pdf}{Beck07}~\cite{Beck07} & 29 & order, explanation, CP, tardiness, activity, flow-shop, precedence, resource, job, stochastic, scheduling, machine, job-shop, constraint programming, periodic, constraint satisfaction, make-span &  & disjunctive, Disjunctive constraint &  & Ilog Scheduler &  &  & benchmark & systematic local search, machine learning, meta heuristic, genetic algorithm & \ref{a:Beck07} & n/a\\
\index{BeckDDF98}\rowlabel{b:BeckDDF98}\href{../works/BeckDDF98.pdf}{BeckDDF98}~\cite{BeckDDF98} & 37 & activity, constraint satisfaction, breakdown, release-date, scheduling, make-span, distributed, task, COP, job, due-date, preempt, precedence, tardiness, flow-time, reactive scheduling, earliness, preemptive, CSP, multi-agent, CP, machine, lateness, re-scheduling, stochastic, inventory, constraint programming, job-shop, resource, transportation, constraint optimization, CLP, order & single machine & cumulative, disjunctive & Prolog, C  & OPL & robot, telescope &  & real-world & edge-finding, column generation, simulated annealing, genetic algorithm & \ref{a:BeckDDF98} & n/a\\
\index{BeckF00}\rowlabel{b:BeckF00}\href{../works/BeckF00.pdf}{BeckF00}~\cite{BeckF00} & 51 & precedence, CLP, job-shop, constraint programming, explanation, due-date, preempt, release-date, task, stochastic, make-span, transportation, CSP, preemptive, reactive scheduling, machine, activity, inventory, resource, constraint satisfaction, job, order, scheduling, CP & single machine & disjunctive, cumulative, Disjunctive constraint, Cardinality constraint &  &  & robot &  & real-world, benchmark & not-last, not-first, edge-finding & \ref{a:BeckF00} & n/a\\
\index{BeckF00a}\rowlabel{b:BeckF00a}\href{../works/BeckF00a.pdf}{BeckF00a}~\cite{BeckF00a} & 40 & constraint programming, job-shop, constraint satisfaction, stochastic, CSP, activity, CP, make-span, explanation, distributed, resource, precedence, re-scheduling, order, CLP, scheduling, machine, job, task &  & disjunctive, Disjunctive constraint, cumulative &  & Ilog Solver, Ilog Scheduler & robot &  & real-world & not-last, not-first, edge-finding & \ref{a:BeckF00a} & n/a\\
\index{BeckF98}\rowlabel{b:BeckF98}\href{../works/BeckF98.pdf}{BeckF98}~\cite{BeckF98} & 30 & precedence, CLP, job-shop, explanation, due-date, preempt, re-scheduling, release-date, task, tardiness, make-span, CSP, preemptive, machine, multi-agent, activity, COP, distributed, inventory, resource, constraint satisfaction, job, order, scheduling, CP & single machine & disjunctive, circuit, cumulative & Prolog &  & robot, business process &  & real-world, benchmark & deep learning, machine learning, simulated annealing, genetic algorithm, column generation, edge-finding & \ref{a:BeckF98} & n/a\\
\index{BeckFW11}\rowlabel{b:BeckFW11}\href{../works/BeckFW11.pdf}{BeckFW11}~\cite{BeckFW11} & 14 & cmax, breakdown, constraint programming, job-shop, constraint satisfaction, preempt, periodic, CP, make-span, explanation, resource, precedence, order, scheduling, completion-time, machine, job &  & disjunctive, table constraint, cumulative & C++ & Ilog Scheduler &  &  & benchmark, real-world & machine learning, meta heuristic, simulated annealing, reinforcement learning & \ref{a:BeckFW11} & n/a\\
\index{BeckR03}\rowlabel{b:BeckR03}\href{../works/BeckR03.pdf}{BeckR03}~\cite{BeckR03} & 23 & job-shop, constraint programming, explanation, due-date, re-scheduling, completion-time, earliness, release-date, constraint logic programming, tardiness, make-span, flow-time, machine, activity, breakdown, inventory, flow-shop, resource, constraint satisfaction, job, order, scheduling, CP, precedence &  & disjunctive &  & Ilog Scheduler, Ilog Solver, Cplex & hoist &  & benchmark & genetic algorithm, column generation, edge-finder & \ref{a:BeckR03} & n/a\\
\index{BeckW07}\rowlabel{b:BeckW07}\href{../works/BeckW07.pdf}{BeckW07}~\cite{BeckW07} & 50 & job-shop, constraint programming, explanation, multi-objective, preempt, re-scheduling, no preempt, task, tardiness, stochastic, make-span, flow-time, reactive scheduling, machine, activity, distributed, flow-shop, resource, job, order, constraint optimization, scheduling, CP, precedence & RCPSP, single machine, Resource-constrained Project Scheduling Problem & Balance constraint &  & Ilog Scheduler & robot, telescope &  & benchmark & column generation, edge-finder, edge-finding & \ref{a:BeckW07} & n/a\\
\index{Bedhief21}\rowlabel{b:Bedhief21}\href{../works/Bedhief21.pdf}{Bedhief21}~\cite{Bedhief21} & 7 & preempt, setup-time, no-wait, scheduling, make-span, sequence dependent setup, due-date, flow-shop, machine, tardiness, constraint programming, completion-time, release-date, CP, no preempt, transportation, job, order & single machine, parallel machine, HFS & noOverlap &  & Cplex, OPL & robot, medical &  & real-life & meta heuristic, genetic algorithm & \ref{a:Bedhief21} & n/a\\
\index{BegB13}\rowlabel{b:BegB13}\href{../works/BegB13.pdf}{BegB13}~\cite{BegB13} & 23 & CP, re-scheduling, resource, scheduling, machine, breakdown, constraint programming, task, completion-time, order, distributed & TMS & cycle &  &  & pipeline &  & benchmark &  & \ref{a:BegB13} & n/a\\
\index{BeldiceanuC94}\rowlabel{b:BeldiceanuC94}\href{../works/BeldiceanuC94.pdf}{BeldiceanuC94}~\cite{BeldiceanuC94} & 27 & task, precedence, CP, scheduling, machine, resource, constraint logic programming, order, completion-time &  & circuit, Element constraint, Among constraint, Atmost constraint, diffn, Arithmetic constraint, cycle, bin-packing, cumulative, alldifferent & Prolog & CHIP, CPO, OZ, OPL & car manufacturing, pipeline &  & real-world, benchmark, real-life &  & \ref{a:BeldiceanuC94} & n/a\\
\index{BeldiceanuCDP11}\rowlabel{b:BeldiceanuCDP11}\href{../works/BeldiceanuCDP11.pdf}{BeldiceanuCDP11}~\cite{BeldiceanuCDP11} & 24 & cmax, constraint programming, preemptive, scheduling, preempt, task, CP, resource, order &  & disjunctive, diffn, bin-packing, geost, cumulative & Prolog & SICStus, CHIP & rectangle-packing, perfect-square &  & benchmark & sweep, energetic reasoning, edge-finding & \ref{a:BeldiceanuCDP11} & n/a\\
\index{BelhadjiI98}\rowlabel{b:BelhadjiI98}\href{../works/BelhadjiI98.pdf}{BelhadjiI98}~\cite{BelhadjiI98} & 9 & job, scheduling, explanation, resource, due-date, CSP, job-shop, task, precedence, release-date, preemptive, order, machine, preempt, constraint satisfaction & JSSP, Temporal Constraint Satisfaction Problem, TCSP & Disjunctive constraint, disjunctive &  &  &  &  & real-life &  & \ref{a:BelhadjiI98} & \ref{c:BelhadjiI98}\\
\index{BenediktMH20}\rowlabel{b:BenediktMH20}\href{../works/BenediktMH20.pdf}{BenediktMH20}~\cite{BenediktMH20} & 19 & task, preemptive, scheduling, machine, sustainability, energy efficiency, job, re-scheduling, CP, preempt, constraint programming, single-machine scheduling, order, job-shop & single machine & noOverlap, endBeforeStart &  & Gurobi & robot &  & benchmark, generated instance, random instance, github &  & \ref{a:BenediktMH20} & \ref{c:BenediktMH20}\\
\index{BeniniLMR11}\rowlabel{b:BeniniLMR11}\href{../works/BeniniLMR11.pdf}{BeniniLMR11}~\cite{BeniniLMR11} & 27 & one-machine scheduling, CP, Benders Decomposition, Logic-Based Benders Decomposition, task, precedence, make-span, order, machine, preempt, tardiness, periodic, scheduling, re-scheduling, resource, CSP, activity, explanation, constraint programming, release-date, energy efficiency, preemptive & SCC, single machine & table constraint, circuit, cumulative &  & Cplex, Ilog Scheduler & pipeline &  & real-world, benchmark, instance generator & column generation, machine learning & \ref{a:BeniniLMR11} & n/a\\
\index{BensanaLV99}\rowlabel{b:BensanaLV99}\href{../works/BensanaLV99.pdf}{BensanaLV99}~\cite{BensanaLV99} & 7 & constraint satisfaction, constraint programming, order, CP, CSP, constraint optimization, explanation &  & cycle &  & Ilog Solver, Cplex & satellite, earth observation &  & benchmark &  & \ref{a:BensanaLV99} & \ref{c:BensanaLV99}\\
\index{BidotVLB09}\rowlabel{b:BidotVLB09}\href{../works/BidotVLB09.pdf}{BidotVLB09}~\cite{BidotVLB09} & 30 & job-shop, due-date, activity, CP, inventory, order, breakdown, re-scheduling, reactive scheduling, job, constraint programming, precedence, release-date, stochastic, scheduling, task, machine, tardiness, make-span, resource, periodic, distributed & Resource-constrained Project Scheduling Problem, JSSP & cumulative, disjunctive & C++ & Ilog Scheduler, OPL & telescope, robot &  & real-world, real-life & edge-finder, edge-finding & \ref{a:BidotVLB09} & n/a\\
\index{BlazewiczDP96}\rowlabel{b:BlazewiczDP96}\href{../works/BlazewiczDP96.pdf}{BlazewiczDP96}~\cite{BlazewiczDP96} & 33 & CSP, stochastic, due-date, distributed, preempt, make-span, task, precedence, preemptive, flow-shop, CP, no-wait, activity, scheduling, machine, lateness, constraint satisfaction, inventory, job-shop, setup-time, constraint programming, release-date, resource, one-machine scheduling, job, order, completion-time, single-machine scheduling & single machine, parallel machine & disjunctive, cumulative, cycle, Disjunctive constraint &  & CHIP, OPL & robot &  & benchmark & neural network, edge-finding, simulated annealing, machine learning, energetic reasoning, Lagrangian relaxation, genetic algorithm & \ref{a:BlazewiczDP96} & n/a\\
\index{BlomBPS14}\rowlabel{b:BlomBPS14}\href{../works/BlomBPS14.pdf}{BlomBPS14}~\cite{BlomBPS14} & 19 & task, net present value, CP, stochastic, transportation, Benders Decomposition, order, distributed, resource, scheduling, precedence &  & disjunctive &  & Cplex & offshore & mineral industry & benchmark, industry partner & MINLP & \ref{a:BlomBPS14} & n/a\\
\index{BlomPS16}\rowlabel{b:BlomPS16}\href{../works/BlomPS16.pdf}{BlomPS16}~\cite{BlomPS16} & 26 & activity, CP, distributed, resource, precedence, producer/consumer, batch process, re-scheduling, order, scheduling, machine, task, transportation &  & disjunctive &  & Cplex & pipeline, offshore & process industry & industry partner, benchmark & genetic algorithm, Lagrangian relaxation, MINLP & \ref{a:BlomPS16} & n/a\\
\index{BocewiczBB09}\rowlabel{b:BocewiczBB09}\href{../works/BocewiczBB09.pdf}{BocewiczBB09}~\cite{BocewiczBB09} & 19 & precedence, scheduling, machine, transportation, periodic, CLP, tardiness, multi-agent, constraint satisfaction, job, CP, order, distributed, job-shop, resource, CSP, constraint logic programming, task, constraint programming, completion-time &  & cycle &  &  & robot &  &  & not-last & \ref{a:BocewiczBB09} & n/a\\
\index{BockmayrP06}\rowlabel{b:BockmayrP06}\href{../works/BockmayrP06.pdf}{BockmayrP06}~\cite{BockmayrP06} & 10 & scheduling, resource, due-date, task, constraint programming, release-date, order, CLP, completion-time, machine, CP &  & cycle, cumulative &  & CHIP &  &  & real-world &  & \ref{a:BockmayrP06} & n/a\\
\index{Bonfietti16}\rowlabel{b:Bonfietti16}\href{../works/Bonfietti16.pdf}{Bonfietti16}~\cite{Bonfietti16} & 13 & distributed, activity, scheduling, resource, task, constraint programming, precedence, order, CP, periodic &  & cumulative, disjunctive, circuit & C++ &  & pipeline &  & benchmark &  & \ref{a:Bonfietti16} & n/a\\
\index{BonfiettiLBM14}\rowlabel{b:BonfiettiLBM14}\href{../works/BonfiettiLBM14.pdf}{BonfiettiLBM14}~\cite{BonfiettiLBM14} & 28 & CSP, constraint satisfaction, precedence, task, buffer-capacity, activity, distributed, machine, scheduling, order, make-span, cyclic scheduling, constraint programming, CP, resource, job, periodic, job-shop & Partial Order Schedule, RCPSP & cycle, circuit, cumulative &  & Ilog Solver & hoist, medical, pipeline, robot &  & benchmark, real-world, generated instance, industrial instance & time-tabling, sweep & \ref{a:BonfiettiLBM14} & n/a\\
\index{BoothTNB16}\rowlabel{b:BoothTNB16}\href{../works/BoothTNB16.pdf}{BoothTNB16}~\cite{BoothTNB16} & 8 & constraint logic programming, scheduling, task, machine, job-shop, activity, completion-time, CP, resource, job, constraint programming, setup-time, order & parallel machine & noOverlap, Disjunctive constraint, cumulative, disjunctive & C++ & OPL, Cplex & robot &  & real-world, random instance &  & \ref{a:BoothTNB16} & n/a\\
\index{BorghesiBLMB18}\rowlabel{b:BorghesiBLMB18}\href{../works/BorghesiBLMB18.pdf}{BorghesiBLMB18}~\cite{BorghesiBLMB18} & 13 & distributed, scheduling, order, make-span, activity, constraint programming, machine, CP, job, re-scheduling, constraint satisfaction, resource, task &  & cycle, cumulative &  &  & high performance computing, super-computer &  & real-life, benchmark & machine learning & \ref{a:BorghesiBLMB18} & n/a\\
\index{BosiM2001}\rowlabel{b:BosiM2001}\href{../works/BosiM2001.pdf}{BosiM2001}~\cite{BosiM2001} & 26 & constraint satisfaction, explanation, single-machine scheduling, scheduling, precedence, make-span, due-date, cmax, CP, machine, constraint programming, CLP, order, release-date, one-machine scheduling, task, job-shop, resource, constraint logic programming, job & single machine & cumulative, Among constraint & Prolog & Ilog Scheduler, CHIP & crew-scheduling &  & real-life, real-world & meta heuristic, edge-finding & \ref{a:BosiM2001} & n/a\\
\index{BourreauGGLT22}\rowlabel{b:BourreauGGLT22}\href{../works/BourreauGGLT22.pdf}{BourreauGGLT22}~\cite{BourreauGGLT22} & 19 & re-scheduling, scheduling, order, CP, resource, job, manpower, no-wait, precedence, transportation, constraint programming &  & disjunctive, alldifferent, diffn, Disjunctive constraint, cycle & C++ & Choco Solver, CHIP, Cplex & workforce scheduling, crew-scheduling, maintenance scheduling, airport, nurse & printing industry & real-world, benchmark & large neighborhood search, meta heuristic, column generation, genetic algorithm & \ref{a:BourreauGGLT22} & n/a\\
\index{BridiBLMB16}\rowlabel{b:BridiBLMB16}\href{../works/BridiBLMB16.pdf}{BridiBLMB16}~\cite{BridiBLMB16} & 14 & energy efficiency, make-span, machine, CP, tardiness, re-scheduling, sustainability, order, distributed, job, scheduling, resource, activity, stochastic, Pareto, constraint programming &  & circuit, cycle, cumulative &  &  & medical, super-computer &  & real-world, real-life & large neighborhood search, meta heuristic, genetic algorithm, machine learning & \ref{a:BridiBLMB16} & n/a\\
\index{BruckerK00}\rowlabel{b:BruckerK00}\href{../works/BruckerK00.pdf}{BruckerK00}~\cite{BruckerK00} & 8 & make-span, cmax, preempt, resource, no preempt, preemptive, activity, completion-time, precedence, CP, order, release-date, scheduling & RCPSP, Resource-constrained Project Scheduling Problem, psplib &  &  & Cplex &  &  & benchmark & meta heuristic, column generation & \ref{a:BruckerK00} & n/a\\
\index{BukchinR18}\rowlabel{b:BukchinR18}\href{../works/BukchinR18.pdf}{BukchinR18}~\cite{BukchinR18} & 12 & scheduling, task, machine, job, constraint programming, precedence, CP, order &  & cycle, bin-packing &  & Cplex & robot &  & benchmark & machine learning, time-tabling & \ref{a:BukchinR18} & n/a\\
\index{BulckG22}\rowlabel{b:BulckG22}\href{../works/BulckG22.pdf}{BulckG22}~\cite{BulckG22} & 11 & CP, Logic-Based Benders Decomposition, distributed, order, unavailability, scheduling, constraint programming, Benders Decomposition &  &  &  & Gurobi, Cplex & round-robin, travelling tournament problem, tournament, sports scheduling &  & real-life, benchmark & memetic algorithm, meta heuristic, time-tabling, large neighborhood search & \ref{a:BulckG22} & n/a\\
\index{Caballero23}\rowlabel{b:Caballero23}\href{../works/Caballero23.pdf}{Caballero23}~\cite{Caballero23} & 1 & CP, scheduling, resource & Resource-constrained Project Scheduling Problem, RCPSP &  &  &  &  &  &  &  & \ref{a:Caballero23} & \ref{c:Caballero23}\\
\index{CampeauG22}\rowlabel{b:CampeauG22}\href{../works/CampeauG22.pdf}{CampeauG22}~\cite{CampeauG22} & 18 & activity, explanation, completion-time, precedence, CP, stochastic, order, job, constraint programming, scheduling, task, make-span, net present value, resource & RCPSP, Resource-constrained Project Scheduling Problem, RCPSPDC & noOverlap, endBeforeStart, cumulative, alwaysIn, cycle & Python & Cplex &  & mining industry & real-life, real-world & column generation, edge-finding & \ref{a:CampeauG22} & \ref{c:CampeauG22}\\
\index{CarchraeB09}\rowlabel{b:CarchraeB09}\href{../works/CarchraeB09.pdf}{CarchraeB09}~\cite{CarchraeB09} & 26 & make-span, order, machine, tardiness, scheduling, resource, job-shop, job, explanation, constraint programming, earliness, CP, task, constraint satisfaction, precedence &  & cumulative & C++ & Ilog Scheduler, OPL &  &  & benchmark, real-world & reinforcement learning, meta heuristic, machine learning, sweep, large neighborhood search & \ref{a:CarchraeB09} & n/a\\
\index{CarlierPSJ20}\rowlabel{b:CarlierPSJ20}\href{../works/CarlierPSJ20.pdf}{CarlierPSJ20}~\cite{CarlierPSJ20} & 9 & scheduling, machine, job, make-span, preempt, order, CP, resource, task, constraint programming, preemptive, release-date, job-shop, cmax & CuSP, RCPSP & cumulative &  &  &  &  & Roadef & energetic reasoning & \ref{a:CarlierPSJ20} & n/a\\
\index{CauwelaertDS20}\rowlabel{b:CauwelaertDS20}\href{../works/CauwelaertDS20.pdf}{CauwelaertDS20}~\cite{CauwelaertDS20} & 19 & constraint programming, CP, completion-time, resource, job, preemptive, constraint logic programming, job-shop, batch process, sequence dependent setup, precedence, transportation, task, CLP, activity, machine, scheduling, order, make-span, preempt, setup-time &  & cycle, Cardinality constraint, cumulative, disjunctive & Java &  & patient, container terminal &  & real-life, generated instance, benchmark, bitbucket & edge-finding, Lagrangian relaxation, not-last, not-first & \ref{a:CauwelaertDS20} & \ref{c:CauwelaertDS20}\\
\index{CauwelaertLS18}\rowlabel{b:CauwelaertLS18}\href{../works/CauwelaertLS18.pdf}{CauwelaertLS18}~\cite{CauwelaertLS18} & 36 & CSP, task, activity, machine, scheduling, order, constraint programming, CP, resource, job, constraint logic programming, job-shop & RCPSP, Resource-constrained Project Scheduling Problem, psplib & bin-packing, table constraint, circuit, alldifferent, cumulative, disjunctive, Reified constraint, GCC constraint & Java, Prolog & OPL, Gecode, CHIP &  &  & benchmark, bitbucket & meta heuristic, time-tabling, sweep, large neighborhood search, not-last, not-first, energetic reasoning, edge-finding & \ref{a:CauwelaertLS18} & \ref{c:CauwelaertLS18}\\
\index{ChenGPSH10}\rowlabel{b:ChenGPSH10}\href{../works/ChenGPSH10.pdf}{ChenGPSH10}~\cite{ChenGPSH10} & 10 & activity, job, constraint programming, CSP, precedence, Benders Decomposition, stochastic, open-shop, order, re-scheduling, constraint optimization, scheduling, preempt, manpower, task, machine, make-span, constraint satisfaction, job-shop, due-date, completion-time, lateness, COP, CP, producer/consumer, resource, preemptive, transportation & JSSP & Disjunctive constraint, cumulative, disjunctive, diffn, cycle & C++ & Ilog Scheduler, Ilog Solver &  & semiprocess industry, chemistry industry, process industry, chemical industry & real-life & particle swarm, not-last, energetic reasoning, genetic algorithm, time-tabling, neural network & \ref{a:ChenGPSH10} & n/a\\
\index{CireCH16}\rowlabel{b:CireCH16}\href{../works/CireCH16.pdf}{CireCH16}~\cite{CireCH16} & 12 & tardiness, scheduling, constraint programming, Benders Decomposition, constraint satisfaction, task, stochastic, CP, Logic-Based Benders Decomposition, order, make-span, resource, breakdown &  & cumulative &  & Cplex &  &  &  & mat heuristic & \ref{a:CireCH16} & n/a\\
\index{ClautiauxJCM08}\rowlabel{b:ClautiauxJCM08}\href{../works/ClautiauxJCM08.pdf}{ClautiauxJCM08}~\cite{ClautiauxJCM08} & 16 & scheduling, order, preempt, activity, constraint programming, CP, job, preemptive, resource, task &  & bin-packing, disjunctive, cumulative, Disjunctive constraint & Prolog & Z3, SICStus &  &  & benchmark & sweep, energetic reasoning & \ref{a:ClautiauxJCM08} & n/a\\
\index{CobanH11}\rowlabel{b:CobanH11}\href{../works/CobanH11.pdf}{CobanH11}~\cite{CobanH11} & 28 & stochastic, CP, make-span, constraint logic programming, distributed, resource, due-date, re-scheduling, preemptive, order, CLP, scheduling, completion-time, machine, job, task, release-date, tardiness, constraint programming, Benders Decomposition, Logic-Based Benders Decomposition, preempt & single machine & cumulative, circuit, noOverlap &  & OPL, Cplex & round-robin, tournament, sports scheduling &  & random instance & time-tabling & \ref{a:CobanH11} & n/a\\
\index{ColT2019a}\rowlabel{b:ColT2019a}\href{../works/ColT2019a.pdf}{ColT2019a}~\cite{ColT2019a} & 7 & constraint programming, open-shop, CLP, order, job-shop, resource, job, constraint satisfaction, scheduling, make-span, earliness, CP, machine & PTC & noOverlap, cumulative & Java & OPL, CPO, MiniZinc, OR-Tools &  &  & benchmark, github, real-world & machine learning, genetic algorithm, large neighborhood search & \ref{a:ColT2019a} & \ref{c:ColT2019a}\\
\index{ColT22}\rowlabel{b:ColT22}\href{../works/ColT22.pdf}{ColT22}~\cite{ColT22} & 19 & setup-time, scheduling, machine, batch process, breakdown, job-shop, constraint programming, task, order, one-machine scheduling, constraint satisfaction, completion-time, constraint logic programming, make-span, no preempt, due-date, COP, distributed, preempt, explanation, preemptive, CLP, multi-objective, open-shop, CP, lateness, CSP, tardiness, transportation, flow-shop, precedence, resource, job & Open Shop Scheduling Problem, FJS, single machine, PMSP, JSSP, OSSP, parallel machine & alldifferent, circuit, cumulative, noOverlap, Arithmetic constraint, disjunctive & C++, Java & OR-Tools, CPO, OPL, MiniZinc, Cplex & robot, semiconductor, oven scheduling &  & supplementary material, benchmark, generated instance, github, real-life, real-world & machine learning, genetic algorithm, particle swarm, memetic algorithm, simulated annealing, large neighborhood search & \ref{a:ColT22} & \ref{c:ColT22}\\
\index{CorreaLR07}\rowlabel{b:CorreaLR07}\href{../works/CorreaLR07.pdf}{CorreaLR07}~\cite{CorreaLR07} & 20 & task, machine, make-span, constraint logic programming, constraint satisfaction, Logic-Based Benders Decomposition, constraint programming, precedence, CP, Benders Decomposition, order, transportation, CSP, release-date, scheduling & parallel machine & disjunctive &  & OPL, Cplex, Choco Solver, Ilog Solver & workforce scheduling, container terminal & heavy industry & real-world & column generation & \ref{a:CorreaLR07} & n/a\\
\index{CzerniachowskaWZ23}\rowlabel{b:CzerniachowskaWZ23}\href{../works/CzerniachowskaWZ23.pdf}{CzerniachowskaWZ23}~\cite{CzerniachowskaWZ23} & 14 & periodic, make-span, constraint programming, transportation, flow-shop, activity, task, constraint satisfaction, resource, job-shop, sustainability, multi-objective, scheduling, Pareto, CP, setup-time, machine, order, completion-time, job & JSSP, PTC, parallel machine & endBeforeStart, noOverlap &  & CPO, OPL, Cplex & robot, automotive & pharmaceutical industry, manufacturing industry, automotive industry & real-world, benchmark, Roadef & meta heuristic, particle swarm & \ref{a:CzerniachowskaWZ23} & n/a\\
\index{Darby-DowmanLMZ97}\rowlabel{b:Darby-DowmanLMZ97}\href{../works/Darby-DowmanLMZ97.pdf}{Darby-DowmanLMZ97}~\cite{Darby-DowmanLMZ97} & 20 & constraint logic programming, scheduling, make-span, resource, CLP, CP, order, CSP, constraint programming, machine, task & single machine, MGAP & disjunctive, Element constraint, span constraint, Disjunctive constraint & Prolog & ECLiPSe, Cplex & pipeline, workforce scheduling, aircraft &  & benchmark, real-life, real-world &  & \ref{a:Darby-DowmanLMZ97} & \ref{c:Darby-DowmanLMZ97}\\
\index{Davis87}\rowlabel{b:Davis87}\href{../works/Davis87.pdf}{Davis87}~\cite{Davis87} & 51 & machine, task, constraint satisfaction, job, order, scheduling, CP &  & disjunctive, circuit, cycle &  &  & robot &  &  &  & \ref{a:Davis87} & n/a\\
\index{DemasseyAM05}\rowlabel{b:DemasseyAM05}\href{../works/DemasseyAM05.pdf}{DemasseyAM05}~\cite{DemasseyAM05} & 14 & preempt, order, CP, resource, CSP, constraint logic programming, task, constraint programming, preemptive, completion-time, release-date, precedence, job-shop, activity, constraint satisfaction, scheduling, machine, job, make-span & Resource-constrained Project Scheduling Problem, psplib, RCPSP, single machine & cumulative, disjunctive, cycle &  & Cplex &  &  & benchmark & edge-finding, energetic reasoning, Lagrangian relaxation, column generation & \ref{a:DemasseyAM05} & n/a\\
\index{DincbasSH90}\rowlabel{b:DincbasSH90}\href{../works/DincbasSH90.pdf}{DincbasSH90}~\cite{DincbasSH90} & 19 & job-shop, distributed, job, constraint programming, CLP, constraint logic programming, task, precedence, order, machine, scheduling, resource &  & Disjunctive constraint, disjunctive, circuit & Prolog & CHIP, OPL &  &  & real-life &  & \ref{a:DincbasSH90} & n/a\\
\index{Dorndorf2000}\rowlabel{b:Dorndorf2000}\href{../works/Dorndorf2000.pdf}{Dorndorf2000}~\cite{Dorndorf2000} & 52 & explanation, setup-time, preempt, single-machine scheduling, due-date, open-shop, constraint optimization, constraint logic programming, stochastic, breakdown, constraint programming, order, preemptive, precedence, sequence dependent setup, job-shop, resource, CSP, cmax, task, completion-time, constraint satisfaction, scheduling, machine, job, make-span, COP & parallel machine, OSP, Open Shop Scheduling Problem, single machine & disjunctive, cumulative, Disjunctive constraint &  & Ilog Scheduler &  &  &  & edge-finding, energetic reasoning & \ref{a:Dorndorf2000} & n/a\\
\index{DoulabiRP16}\rowlabel{b:DoulabiRP16}\href{../works/DoulabiRP16.pdf}{DoulabiRP16}~\cite{DoulabiRP16} & 17 & distributed, scheduling, resource, transportation, stochastic, constraint programming, order, machine, CP, single-machine scheduling & single machine & cycle, bin-packing, Element constraint &  & Cplex, OPL & medical, patient, nurse, operating room, rectangle-packing, surgery, steel mill, crew-scheduling, robot &  & real-world, generated instance & genetic algorithm, column generation & \ref{a:DoulabiRP16} & n/a\\
\index{Edis21}\rowlabel{b:Edis21}\href{../works/Edis21.pdf}{Edis21}~\cite{Edis21} & 20 & precedence, resource, job, constraint optimization, stochastic, scheduling, constraint programming, task, order, Pareto, COP, distributed, preempt, preemptive, multi-objective, CP, CSP &  & disjunctive, cycle &  & Cplex, Z3, OPL & crew-scheduling, medical, robot &  & benchmark & MINLP, simulated annealing, meta heuristic, genetic algorithm, particle swarm, ant colony, reinforcement learning & \ref{a:Edis21} & n/a\\
\index{ElciOH22}\rowlabel{b:ElciOH22}\href{../works/ElciOH22.pdf}{ElciOH22}~\cite{ElciOH22} & 15 & resource, due-date, order, tardiness, scheduling, Benders Decomposition, job, task, transportation, stochastic, constraint programming, CP, Logic-Based Benders Decomposition, constraint logic programming, make-span, single-machine scheduling, machine, distributed & single machine & cumulative, disjunctive & Julia & Cplex & patient, crew-scheduling, aircraft, operating room, surgery &  & benchmark, random instance, real-life &  & \ref{a:ElciOH22} & n/a\\
\index{EmdeZD22}\rowlabel{b:EmdeZD22}\href{../works/EmdeZD22.pdf}{EmdeZD22}~\cite{EmdeZD22} & 30 & constraint logic programming, flow-time, distributed, resource, inventory, precedence, batch process, order, machine, job, no-wait, release-date, transportation, tardiness, scheduling, constraint programming, Benders Decomposition, completion-time, constraint satisfaction, task, open-shop, stochastic, job-shop, CP, Logic-Based Benders Decomposition, make-span, single-machine scheduling, bi-objective & single machine, parallel machine & noOverlap, bin-packing & C  & Cplex & pipeline, drone, semiconductor, yard crane, automotive & automotive industry & github, random instance &  & \ref{a:EmdeZD22} & \ref{c:EmdeZD22}\\
\index{EmeretlisTAV17}\rowlabel{b:EmeretlisTAV17}\href{../works/EmeretlisTAV17.pdf}{EmeretlisTAV17}~\cite{EmeretlisTAV17} & 24 & stochastic, constraint programming, Logic-Based Benders Decomposition, CP, completion-time, resource, job, re-scheduling, periodic, explanation, sequence dependent setup, precedence, constraint optimization, task, release-date, tardiness, distributed, machine, multi-objective, scheduling, order, make-span, Benders Decomposition, preempt, setup-time &  & cycle, circuit, disjunctive &  &  & pipeline &  & real-life, generated instance & edge-finding, machine learning, meta heuristic, genetic algorithm & \ref{a:EmeretlisTAV17} & n/a\\
\index{EscobetPQPRA19}\rowlabel{b:EscobetPQPRA19}\href{../works/EscobetPQPRA19.pdf}{EscobetPQPRA19}~\cite{EscobetPQPRA19} & 10 & task, release-date, Pareto, activity, distributed, machine, multi-objective, scheduling, order, due-date, reactive scheduling, constraint programming, CP, resource, job, periodic, job-shop, batch process, constraint satisfaction &  & cycle, alternative constraint, noOverlap, circuit &  & OPL, Cplex & energy-price, dairy & food industry, manufacturing industry, dairy industry &  & MINLP, meta heuristic & \ref{a:EscobetPQPRA19} & n/a\\
\index{EtminaniesfahaniGNMS22}\rowlabel{b:EtminaniesfahaniGNMS22}\href{../works/EtminaniesfahaniGNMS22.pdf}{EtminaniesfahaniGNMS22}~\cite{EtminaniesfahaniGNMS22} & 10 & order, earliness, cmax, open-shop, resource, CP, preemptive, constraint programming, job, job-shop, preempt, machine, precedence, tardiness, net present value, stochastic, activity, make-span, task, scheduling & RCPSP, Resource-constrained Project Scheduling Problem, psplib, parallel machine &  & Python & Cplex, OR-Tools & crew-scheduling, aircraft &  & real-world & lazy clause generation, memetic algorithm, mat heuristic, ant colony, Lagrangian relaxation, meta heuristic, large neighborhood search, particle swarm, genetic algorithm & \ref{a:EtminaniesfahaniGNMS22} & n/a\\
\index{EvenSH15a}\rowlabel{b:EvenSH15a}\href{../works/EvenSH15a.pdf}{EvenSH15a}~\cite{EvenSH15a} & 16 & distributed, constraint programming, resource, transportation, Benders Decomposition, order, preempt, scheduling, task, machine, preemptive, completion-time, CP &  & disjunctive, Disjunctive constraint, cumulative & Java & Choco Solver, OPL & emergency service, evacuation &  & real-world, real-life & ant colony, mat heuristic, meta heuristic, column generation, sweep & \ref{a:EvenSH15a} & n/a\\
\index{FachiniA20}\rowlabel{b:FachiniA20}\href{../works/FachiniA20.pdf}{FachiniA20}~\cite{FachiniA20} & 18 & distributed, Logic-Based Benders Decomposition, CP, resource, transportation, constraint programming, Benders Decomposition, inventory, order, constraint logic programming, scheduling, task, machine & HFF & cycle, cumulative & C  & Gurobi, OR-Tools & robot, operating room, satellite &  & supplementary material, real-world, github, benchmark, real-life & meta heuristic, sweep, large neighborhood search, particle swarm, column generation, mat heuristic, simulated annealing & \ref{a:FachiniA20} & \ref{c:FachiniA20}\\
\index{FahimiOQ18}\rowlabel{b:FahimiOQ18}\href{../works/FahimiOQ18.pdf}{FahimiOQ18}~\cite{FahimiOQ18} & 22 & explanation, completion-time, CP, batch process, open-shop, order, job, constraint programming, precedence, scheduling, task, setup-time, machine, make-span, lateness, preempt, sequence dependent setup, constraint satisfaction, resource, distributed, preemptive, job-shop, due-date & Resource-constrained Project Scheduling Problem, psplib, RCPSP & Disjunctive constraint, AllDiff constraint, cumulative, disjunctive, alldifferent, Cumulatives constraint &  & Choco Solver &  &  & benchmark, random instance & time-tabling, not-first, sweep, edge-finding, not-last, lazy clause generation & \ref{a:FahimiOQ18} & \ref{c:FahimiOQ18}\\
\index{FalaschiGMP97}\rowlabel{b:FalaschiGMP97}\href{../works/FalaschiGMP97.pdf}{FalaschiGMP97}~\cite{FalaschiGMP97} & 27 & constraint programming, constraint logic programming, scheduling, CLP, order &  & Arithmetic constraint & Prolog &  &  &  &  &  & \ref{a:FalaschiGMP97} & n/a\\
\index{FallahiAC20}\rowlabel{b:FallahiAC20}\href{../works/FallahiAC20.pdf}{FallahiAC20}~\cite{FallahiAC20} & 18 & resource, COP, transportation, constraint logic programming, task, constraint satisfaction, constraint programming, order, CP, scheduling, CSP, constraint optimization &  & cycle &  & OR-Tools & nurse, container terminal, workforce scheduling, robot, medical &  & real-life, github & Lagrangian relaxation, sweep, simulated annealing, memetic algorithm, large neighborhood search, column generation, neural network, meta heuristic & \ref{a:FallahiAC20} & \ref{c:FallahiAC20}\\
\index{FanXG21}\rowlabel{b:FanXG21}\href{../works/FanXG21.pdf}{FanXG21}~\cite{FanXG21} & 15 & tardiness, multi-objective, stochastic, completion-time, no preempt, breakdown, task, flow-shop, resource, make-span, flow-time, job, order, reactive scheduling, batch process, machine, distributed, precedence, setup-time, job-shop, unavailability, due-date, preempt, earliness, one-machine scheduling, scheduling, CP & single machine, parallel machine & cycle & Python, Java & Cplex, ECLiPSe, Gurobi & semiconductor, tournament & manufacturing industry & benchmark & simulated annealing, neural network, ant colony, max-flow, machine learning, meta heuristic & \ref{a:FanXG21} & n/a\\
\index{FarsiTM22}\rowlabel{b:FarsiTM22}\href{../works/FarsiTM22.pdf}{FarsiTM22}~\cite{FarsiTM22} & 14 & tardiness, earliness, CP, re-scheduling, stochastic, constraint programming, Benders Decomposition, completion-time, multi-objective, periodic, Logic-Based Benders Decomposition, distributed, task, resource, bi-objective, continuous-process, Pareto, no-wait, scheduling, make-span &  & alldifferent, circuit &  & Cplex & physician, nurse, patient, operating room, surgery, robot, medical &  & supplementary material & time-tabling, ant colony, genetic algorithm, meta heuristic & \ref{a:FarsiTM22} & \ref{c:FarsiTM22}\\
\index{Fatemi-AnarakiTFV23}\rowlabel{b:Fatemi-AnarakiTFV23}\href{../works/Fatemi-AnarakiTFV23.pdf}{Fatemi-AnarakiTFV23}~\cite{Fatemi-AnarakiTFV23} & 15 & multi-agent, machine, cmax, resource, no-wait, order, completion-time, scheduling, CP, re-scheduling, distributed, constraint programming, job-shop, single-machine scheduling, breakdown, job, transportation, setup-time, task, cyclic scheduling, make-span & parallel machine, single machine & bin-packing, circuit, cycle, disjunctive & Python & Cplex, OPL & electroplating, COVID, robot, hoist, semiconductor & food industry & random instance, github, real-world & ant colony, mat heuristic, meta heuristic, time-tabling & \ref{a:Fatemi-AnarakiTFV23} & \ref{c:Fatemi-AnarakiTFV23}\\
\index{FetgoD22}\rowlabel{b:FetgoD22}\href{../works/FetgoD22.pdf}{FetgoD22}~\cite{FetgoD22} & 32 & CP, explanation, constraint optimization, preempt, COP, CSP, make-span, resource, precedence, cmax, order, scheduling, constraint programming, completion-time, constraint satisfaction, task & Resource-constrained Project Scheduling Problem, RCPSP, CuSP & cumulative & Java, Python & CHIP, Choco Solver &  &  & real-world, benchmark & not-first, not-last, edge-finding, lazy clause generation, edge-finder, time-tabling, energetic reasoning, sweep & \ref{a:FetgoD22} & n/a\\
\index{ForbesHJST24}\rowlabel{b:ForbesHJST24}\href{../works/ForbesHJST24.pdf}{ForbesHJST24}~\cite{ForbesHJST24} & 15 & Logic-Based Benders Decomposition, job-shop, constraint satisfaction, scheduling, machine, job, re-scheduling, make-span, order, distributed, CP, resource, Benders Decomposition, constraint logic programming, stochastic, task, constraint programming, release-date &  & cumulative & Python & Gurobi, OPL & emergency service, surgery, airport, patient, operating room &  & benchmark, real-life, github & genetic algorithm & \ref{a:ForbesHJST24} & \ref{c:ForbesHJST24}\\
\index{FrimodigECM23}\rowlabel{b:FrimodigECM23}\href{../works/FrimodigECM23.pdf}{FrimodigECM23}~\cite{FrimodigECM23} & 38 & CP, scheduling, unavailability, task, activity, machine, resource, order, constraint satisfaction, setup-time, constraint programming, Pareto, stochastic, distributed, multi-objective &  & bin-packing, cumulative & Python & Chuffed, Cplex, Gecode, MiniZinc & patient, nurse, surgery, radiation therapy, medical, operating room, physician &  & real-world, benchmark, instance generator & large neighborhood search, lazy clause generation, column generation & \ref{a:FrimodigECM23} & n/a\\
\index{GarridoAO09}\rowlabel{b:GarridoAO09}\href{../works/GarridoAO09.pdf}{GarridoAO09}~\cite{GarridoAO09} & 30 & multi-objective, CP, task, constraint satisfaction, re-scheduling, precedence, make-span, order, scheduling, resource, CSP, explanation, constraint programming &  & disjunctive & Java & OPL, Choco Solver, CPO & airport &  & benchmark &  & \ref{a:GarridoAO09} & n/a\\
\index{GarridoOS08}\rowlabel{b:GarridoOS08}\href{../works/GarridoOS08.pdf}{GarridoOS08}~\cite{GarridoOS08} & 11 & scheduling, resource, CSP, activity, explanation, constraint programming, CP, task, constraint satisfaction, make-span, order, machine &  &  & Java, C  & CPO, Choco Solver &  &  & real-world &  & \ref{a:GarridoOS08} & n/a\\
\index{GedikKBR17}\rowlabel{b:GedikKBR17}\href{../works/GedikKBR17.pdf}{GedikKBR17}~\cite{GedikKBR17} & 18 & sequence dependent setup, resource, explanation, constraint programming, CP, setup-time, order, transportation, job, scheduling, task, machine & parallel machine & alternative constraint, cumulative, noOverlap &  & OR-Tools, Cplex, Gecode & medical, tournament, nurse &  & real-world, benchmark & meta heuristic, ant colony, large neighborhood search, column generation, simulated annealing & \ref{a:GedikKBR17} & n/a\\
\index{GedikKEK18}\rowlabel{b:GedikKEK18}\href{../works/GedikKEK18.pdf}{GedikKEK18}~\cite{GedikKEK18} & 11 & resource, preemptive, constraint programming, completion-time, CP, stochastic, setup-time, order, breakdown, multi-objective, transportation, job, scheduling, task, machine, make-span, cmax, due-date, tardiness, preempt, sequence dependent setup & parallel machine, single machine, PMSP & cumulative, noOverlap &  & Cplex & medical, nurse, sports scheduling & manufacturing industry & benchmark & meta heuristic, ant colony, large neighborhood search, column generation, simulated annealing, genetic algorithm & \ref{a:GedikKEK18} & n/a\\
\index{GoelSHFS15}\rowlabel{b:GoelSHFS15}\href{../works/GoelSHFS15.pdf}{GoelSHFS15}~\cite{GoelSHFS15} & 12 & constraint programming, precedence, inventory, setup-time, order, scheduling, task, unavailability, machine, activity, CP, resource, transportation &  & cumulative, disjunctive, noOverlap, alwaysEqual constraint, alwaysIn &  & OPL, Cplex, CPO & pipeline & gas industry, transportation industry &  & large neighborhood search & \ref{a:GoelSHFS15} & n/a\\
\index{GokPTGO23}\rowlabel{b:GokPTGO23}\href{../works/GokPTGO23.pdf}{GokPTGO23}~\cite{GokPTGO23} & 36 & order, completion-time, multi-objective, activity, constraint satisfaction, distributed, task, resource, bi-objective, job, setup-time, Pareto, scheduling, precedence, make-span, tardiness, multi-agent, CP, machine, re-scheduling, stochastic, inventory, constraint programming, job-shop, transportation & RCPSP, Resource-constrained Project Scheduling Problem & circuit, alldifferent, disjunctive, cumulative, cycle &  & OPL & offshore, workforce scheduling, airport, aircraft & airline industry & real-world, github & machine learning, genetic algorithm, meta heuristic, reinforcement learning, large neighborhood search & \ref{a:GokPTGO23} & \ref{c:GokPTGO23}\\
\index{GokgurHO18}\rowlabel{b:GokgurHO18}\href{../works/GokgurHO18.pdf}{GokgurHO18}~\cite{GokgurHO18} & 17 & task, setup-time, CLP, machine, order, cmax, tardiness, earliness, scheduling, preempt, make-span, constraint satisfaction, preemptive, job-shop, due-date, activity, flow-shop, explanation, completion-time, CP, resource, transportation, job, constraint programming, CSP, precedence, release-date & parallel machine, single machine & cumulative, disjunctive, alternative constraint, Channeling constraint, Disjunctive constraint &  & OPL, CHIP & robot, semiconductor &  & real-life, real-world & mat heuristic, not-last, meta heuristic, edge-finding, energetic reasoning, genetic algorithm, not-first & \ref{a:GokgurHO18} & n/a\\
\index{GoldwaserS18}\rowlabel{b:GoldwaserS18}\href{../works/GoldwaserS18.pdf}{GoldwaserS18}~\cite{GoldwaserS18} & 32 & constraint programming, order, Logic-Based Benders Decomposition, resource, task, scheduling, machine, transportation, explanation, due-date, flow-shop, CP, Benders Decomposition, constraint optimization &  & cumulative & Python & Gurobi, CHIP, Gecode, Chuffed & torpedo & steel industry & github, instance generator, benchmark, generated instance & sweep, column generation, lazy clause generation, simulated annealing, time-tabling & \ref{a:GoldwaserS18} & \ref{c:GoldwaserS18}\\
\index{GombolayWS18}\rowlabel{b:GombolayWS18}\href{../works/GombolayWS18.pdf}{GombolayWS18}~\cite{GombolayWS18} & 20 & scheduling, machine, job, re-scheduling, open-shop, make-span, breakdown, setup-time, multi-agent, preempt, order, distributed, flow-shop, CP, resource, Benders Decomposition, task, constraint programming, preemptive, precedence, Logic-Based Benders Decomposition, job-shop, periodic, completion-time, constraint satisfaction & Resource-constrained Project Scheduling Problem, OSP & cumulative, disjunctive & Java & OPL, Gurobi & robot, patient, aircraft, crew-scheduling &  & real-world, instance generator, benchmark & simulated annealing, genetic algorithm, edge-finding, meta heuristic & \ref{a:GombolayWS18} & n/a\\
\index{GomesM17}\rowlabel{b:GomesM17}\href{../works/GomesM17.pdf}{GomesM17}~\cite{GomesM17} & 10 & Pareto, CP, Logic-Based Benders Decomposition, inventory, setup-time, make-span, single-machine scheduling, machine, distributed, resource, release-date, due-date, order, tardiness, scheduling, Benders Decomposition, completion-time, job, transportation, stochastic & parallel machine, PMSP, single machine & cycle & C++ & Cplex &  &  &  & genetic algorithm, meta heuristic, simulated annealing, Lagrangian relaxation, ant colony & \ref{a:GomesM17} & n/a\\
\index{GrimesH15}\rowlabel{b:GrimesH15}\href{../works/GrimesH15.pdf}{GrimesH15}~\cite{GrimesH15} & 17 & job, preempt, flow-shop, setup-time, no-wait, scheduling, precedence, make-span, tardiness, earliness, preemptive, sequence dependent setup, CSP, due-date, batch process, cmax, CP, machine, lateness, constraint programming, job-shop, constraint optimization, open-shop, order, completion-time, constraint satisfaction, release-date, distributed, task, COP, resource & OSP, Open Shop Scheduling Problem, JSSP & noOverlap, disjunctive, IloNoOverlap, cumulative, Balance constraint, endBeforeStart, Disjunctive constraint &  & Choco Solver, Mistral, Ilog Scheduler, CPO & semiconductor & semiconductor industry & real-world, benchmark & genetic algorithm, not-first, meta heuristic, not-last, simulated annealing, time-tabling, large neighborhood search, particle swarm, memetic algorithm, edge-finding & \ref{a:GrimesH15} & n/a\\
\index{GrimesIOS14}\rowlabel{b:GrimesIOS14}\href{../works/GrimesIOS14.pdf}{GrimesIOS14}~\cite{GrimesIOS14} & 16 & completion-time, machine, preempt, CP, periodic, re-scheduling, sustainability, due-date, distributed, activity, scheduling, resource, task, stochastic, constraint programming, preemptive, order &  & disjunctive &  & Cplex, CHIP & energy-price, real-time pricing, HVAC &  & real-world, real-life & machine learning & \ref{a:GrimesIOS14} & n/a\\
\index{Gronkvist06}\rowlabel{b:Gronkvist06}\href{../works/Gronkvist06.pdf}{Gronkvist06}~\cite{Gronkvist06} & 17 & CP, scheduling, CSP, activity, resource, transportation, constraint logic programming, constraint satisfaction, constraint programming, order &  & cycle, GCC constraint, Cardinality constraint &  & Ilog Solver, Cplex & aircraft, crew-scheduling, railway, airport &  & real-world & column generation, Lagrangian relaxation & \ref{a:Gronkvist06} & n/a\\
\index{GuoZ23}\rowlabel{b:GuoZ23}\href{../works/GuoZ23.pdf}{GuoZ23}~\cite{GuoZ23} & 29 & activity, sequence dependent setup, resource, job, setup-time, stochastic, Benders Decomposition, scheduling, inventory, machine, job-shop, constraint programming, task, order, unavailability, make-span, Logic-Based Benders Decomposition, transportation, distributed, constraint optimization, multi-objective, CP & parallel machine & bin-packing, cycle, Balance constraint & Python & Cplex, Gurobi, SCIP, OPL & railway, drone, medical, physician, operating room, patient, vaccine, COVID, automotive & automotive industry, garment industry & real-world, supplementary material, benchmark, github & machine learning, column generation, ant colony & \ref{a:GuoZ23} & \ref{c:GuoZ23}\\
\index{GurEA19}\rowlabel{b:GurEA19}\href{../works/GurEA19.pdf}{GurEA19}~\cite{GurEA19} & 24 & order, scheduling, CP, constraint programming, multi-objective, stochastic, re-scheduling, completion-time, Pareto, resource, distributed, job-shop, job &  &  &  & Cplex & medical, surgery, patient, operating room & service industry & real-life & meta heuristic, ant colony, Lagrangian relaxation & \ref{a:GurEA19} & n/a\\
\index{GurPAE23}\rowlabel{b:GurPAE23}\href{../works/GurPAE23.pdf}{GurPAE23}~\cite{GurPAE23} & 25 & re-scheduling, order, scheduling, machine, multi-objective, constraint programming, bi-objective, stochastic, CP, distributed, resource, inventory &  & cumulative &  & Cplex, OPL & patient, operating room, physician, surgery, nurse, COVID &  & real-life & neural network, meta heuristic, machine learning & \ref{a:GurPAE23} & \ref{c:GurPAE23}\\
\index{GuyonLPR12}\rowlabel{b:GuyonLPR12}\href{../works/GuyonLPR12.pdf}{GuyonLPR12}~\cite{GuyonLPR12} & 25 & precedence, CP, Benders Decomposition, order, cmax, release-date, scheduling, preempt, manpower, task, unavailability, machine, make-span, Logic-Based Benders Decomposition, COP, resource, preemptive, job-shop, activity, flow-shop, job, constraint programming, completion-time & parallel machine, single machine & disjunctive, cycle &  & Cplex & satellite &  & generated instance, benchmark, instance generator & time-tabling, column generation, Lagrangian relaxation, energetic reasoning & \ref{a:GuyonLPR12} & n/a\\
\index{HachemiGR11}\rowlabel{b:HachemiGR11}\href{../works/HachemiGR11.pdf}{HachemiGR11}~\cite{HachemiGR11} & 16 & constraint programming, precedence, make-span, CP, Benders Decomposition, Logic-Based Benders Decomposition, task, constraint satisfaction, order, scheduling, explanation, resource, job-shop, transportation, job, activity &  & GCC constraint, Cardinality constraint, alldifferent, cycle &  & OPL, Cplex, Ilog Scheduler & forestry, crew-scheduling & food industry, airline industry, forest industry &  & column generation, meta heuristic & \ref{a:HachemiGR11} & n/a\\
\index{Ham18}\rowlabel{b:Ham18}\href{../works/Ham18.pdf}{Ham18}~\cite{Ham18} & 14 & cmax, scheduling, inventory, constraint programming, distributed, constraint satisfaction, CP, CSP, batch process, resource, job, due-date, order, precedence, make-span, machine, transportation, task, completion-time, job-shop, sequence dependent setup & parallel machine & endBeforeStart, cumulative, disjunctive, cycle, noOverlap &  & Cplex, OPL & semiconductor, robot, drone, aircraft & taxi industry &  & genetic algorithm, meta heuristic & \ref{a:Ham18} & n/a\\
\index{Ham18a}\rowlabel{b:Ham18a}\href{../works/Ham18a.pdf}{Ham18a}~\cite{Ham18a} & 10 & CP, CSP, resource, job, setup-time, scheduling, inventory, machine, batch process, cmax, job-shop, constraint programming, task, order, constraint satisfaction, completion-time, make-span, tardiness, due-date & parallel machine & cycle, noOverlap, alternative constraint, disjunctive, circuit &  & Cplex, CPO, OPL & semiconductor, drone, robot &  & real-world & meta heuristic & \ref{a:Ham18a} & n/a\\
\index{Ham20a}\rowlabel{b:Ham20a}\href{../works/Ham20a.pdf}{Ham20a}~\cite{Ham20a} & 9 & CP, precedence, resource, job, multi-agent, scheduling, inventory, machine, batch process, job-shop, constraint programming, task, order, completion-time, make-span, transportation & parallel machine & cycle, noOverlap, endBeforeStart, cumulative &  & Cplex, Choco Solver, OPL & semiconductor, drone, robot, deep space &  & benchmark, generated instance & large neighborhood search, machine learning, meta heuristic & \ref{a:Ham20a} & n/a\\
\index{HamC16}\rowlabel{b:HamC16}\href{../works/HamC16.pdf}{HamC16}~\cite{HamC16} & 6 & multi-objective, CP, sequence dependent setup, precedence, resource, job, setup-time, scheduling, machine, batch process, cmax, bi-objective, job-shop, constraint programming, task, order, constraint satisfaction, completion-time, make-span, transportation & FJS & alwaysEqual constraint, cycle, endBeforeStart &  & Cplex, OPL & semiconductor & pharmaceutical industry & benchmark & particle swarm, meta heuristic, genetic algorithm & \ref{a:HamC16} & n/a\\
\index{HamFC17}\rowlabel{b:HamFC17}\href{../works/HamFC17.pdf}{HamFC17}~\cite{HamFC17} & 8 & order, batch process, machine, job-shop, preempt, job, scheduling, CP, constraint programming, tardiness, due-date, completion-time, preemptive, inventory, resource, make-span, single-machine scheduling & parallel machine, single machine & cycle, alwaysEqual constraint, bin-packing &  & Cplex, OPL & semiconductor, robot, drone & semiconductor industry & benchmark & meta heuristic, genetic algorithm & \ref{a:HamFC17} & n/a\\
\index{HamP21}\rowlabel{b:HamP21}\href{../works/HamP21.pdf}{HamP21}~\cite{HamP21} & 8 & distributed, CP, precedence, resource, job, scheduling, machine, cmax, job-shop, constraint programming, task, order, completion-time, make-span & FJS & noOverlap, endBeforeStart &  & Cplex, CPO, OPL & semiconductor, drone, robot, aircraft &  & real-world, benchmark, github & large neighborhood search, machine learning & \ref{a:HamP21} & \ref{c:HamP21}\\
\index{HamPK21}\rowlabel{b:HamPK21}\href{../works/HamPK21.pdf}{HamPK21}~\cite{HamPK21} & 12 & scheduling, CP, constraint programming, tardiness, multi-objective, completion-time, task, flow-shop, resource, make-span, single-machine scheduling, sequence dependent setup, order, machine, distributed, precedence, cmax, setup-time, job-shop, re-scheduling, bi-objective, job & single machine, FJS, parallel machine & noOverlap, cycle, endBeforeStart &  & OPL, Cplex & semiconductor, agriculture, robot &  & benchmark, github & genetic algorithm, simulated annealing, swarm intelligence, particle swarm, ant colony, Lagrangian relaxation, meta heuristic & \ref{a:HamPK21} & \ref{c:HamPK21}\\
\index{HarjunkoskiG02}\rowlabel{b:HarjunkoskiG02}\href{../works/HarjunkoskiG02.pdf}{HarjunkoskiG02}~\cite{HarjunkoskiG02} & 20 & CLP, scheduling, order, setup-time, activity, single-stage scheduling, constraint programming, machine, flow-shop, CP, job, constraint logic programming, due-date, constraint satisfaction, resource, task, release-date, job-shop &  & cumulative &  & ECLiPSe, Ilog Scheduler, CHIP, Ilog Solver, Cplex, OPL &  &  &  & simulated annealing, genetic algorithm & \ref{a:HarjunkoskiG02} & n/a\\
\index{HarjunkoskiJG00}\rowlabel{b:HarjunkoskiJG00}\href{../works/HarjunkoskiJG00.pdf}{HarjunkoskiJG00}~\cite{HarjunkoskiJG00} & 7 & due-date, constraint satisfaction, scheduling, machine, job-shop, CP, order, constraint logic programming, job, CLP & parallel machine & disjunctive &  & OPL, ECLiPSe, Ilog Solver, Cplex, CHIP &  &  &  & MINLP & \ref{a:HarjunkoskiJG00} & n/a\\
\index{HarjunkoskiMBC14}\rowlabel{b:HarjunkoskiMBC14}\href{../works/HarjunkoskiMBC14.pdf}{HarjunkoskiMBC14}~\cite{HarjunkoskiMBC14} & 33 & distributed, CLP, Benders Decomposition, precedence, lateness, unavailability, task, release-date, activity, setup-time, due-date, breakdown, Pareto, job, continuous-process, batch process, constraint programming, reactive scheduling, make-span, manpower, stochastic, constraint logic programming, make to stock, machine, re-scheduling, multi-objective, earliness, order, job-shop, Logic-Based Benders Decomposition, resource, inventory, CP, COP, periodic, scheduling, transportation, cyclic scheduling, tardiness & single machine & circuit, cycle, disjunctive &  & ECLiPSe, CHIP, Gurobi, Cplex, Gecode, SCIP, OPL & dairy, airport, semiconductor, automotive, pipeline & petro-chemical industry, oil industry, chemical industry, paper industry, pharmaceutical industry, dairy industry, process industry & benchmark, real-world, real-life & simulated annealing, particle swarm, column generation, MINLP, large neighborhood search, meta heuristic & \ref{a:HarjunkoskiMBC14} & n/a\\
\index{HauderBRPA20}\rowlabel{b:HauderBRPA20}\href{../works/HauderBRPA20.pdf}{HauderBRPA20}~\cite{HauderBRPA20} & 14 & setup-time, order, bi-objective, no-wait, job-shop, resource, stochastic, task, constraint programming, completion-time, precedence, earliness, machine, transportation, tardiness, make-span, activity, explanation, inventory, due-date, scheduling, flow-shop, job, CP, multi-objective, breakdown, manpower & RCPSP, RCMPSP, FJS, Resource-constrained Project Scheduling Problem & cumulative, cycle &  & OPL, Cplex & aircraft & automobile industry, food-processing industry, steel industry, processing industry & industry partner, benchmark, real-world, supplementary material & particle swarm, genetic algorithm, meta heuristic & \ref{a:HauderBRPA20} & \ref{c:HauderBRPA20}\\
\index{HebrardHJMPV16}\rowlabel{b:HebrardHJMPV16}\href{../works/HebrardHJMPV16.pdf}{HebrardHJMPV16}~\cite{HebrardHJMPV16} & 10 & online scheduling, cmax, scheduling, order, make-span, distributed, machine, job, completion-time, resource, task & parallel machine & cumulative &  &  & satellite, earth observation &  & industrial partner &  & \ref{a:HebrardHJMPV16} & n/a\\
\index{HeckmanB11}\rowlabel{b:HeckmanB11}\href{../works/HeckmanB11.pdf}{HeckmanB11}~\cite{HeckmanB11} & 20 & order, job, CSP, scheduling, Pareto, machine, make-span, tardiness, constraint satisfaction, resource, job-shop, activity, flow-shop, explanation, constraint programming, precedence, CP &  & disjunctive, Completion constraint &  & Ilog Scheduler &  &  & benchmark, real-world & simulated annealing, edge-finding, genetic algorithm, meta heuristic, edge-finder & \ref{a:HeckmanB11} & n/a\\
\index{HeinzNVH22}\rowlabel{b:HeinzNVH22}\href{../works/HeinzNVH22.pdf}{HeinzNVH22}~\cite{HeinzNVH22} & 16 & re-scheduling, bi-objective, scheduling, preempt, sequence dependent setup, task, unavailability, machine, make-span, distributed, flow-shop, completion-time, CP, resource, preemptive, activity, explanation, job, constraint programming, precedence, setup-time, order & parallel machine & cumulative, noOverlap, alternative constraint &  & Gurobi & high performance computing, robot, crew-scheduling &  & real-world, generated instance, benchmark, gitlab & meta heuristic, genetic algorithm, Lagrangian relaxation & \ref{a:HeinzNVH22} & \ref{c:HeinzNVH22}\\
\index{HeinzSB13}\rowlabel{b:HeinzSB13}\href{../works/HeinzSB13.pdf}{HeinzSB13}~\cite{HeinzSB13} & 36 & constraint programming, constraint optimization, order, CSP, completion-time, release-date, resource, job, constraint satisfaction, preempt, scheduling, precedence, COP, due-date, CP, machine, explanation & psplib, single machine, RCPSP, Resource-constrained Project Scheduling Problem & disjunctive, cumulative &  & MiniZinc, SCIP, Cplex & satellite &  & benchmark & edge-finding, time-tabling & \ref{a:HeinzSB13} & \ref{c:HeinzSB13}\\
\index{HeinzSSW12}\rowlabel{b:HeinzSSW12}\href{../works/HeinzSSW12.pdf}{HeinzSSW12}~\cite{HeinzSSW12} & 12 & explanation, constraint programming, inventory, order, task, constraint satisfaction, CP &  & bin-packing &  & Cplex & steel mill & steel industry, process industry & CSPlib, real-world & large neighborhood search, column generation & \ref{a:HeinzSSW12} & \ref{c:HeinzSSW12}\\
\index{HeipckeCCS00}\rowlabel{b:HeipckeCCS00}\href{../works/HeipckeCCS00.pdf}{HeipckeCCS00}~\cite{HeipckeCCS00} & 8 & CP, resource, task, constraint programming, preemptive, release-date, precedence, job-shop, activity, completion-time, due-date, scheduling, machine, job, make-span, preempt, order & RCPSP, single machine, Resource-constrained Project Scheduling Problem & disjunctive, cumulative, Disjunctive constraint &  &  &  &  & instance generator, benchmark &  & \ref{a:HeipckeCCS00} & \ref{c:HeipckeCCS00}\\
\index{HladikCDJ08}\rowlabel{b:HladikCDJ08}\href{../works/HladikCDJ08.pdf}{HladikCDJ08}~\cite{HladikCDJ08} & 18 & distributed, activity, Pareto, constraint programming, cyclic scheduling, precedence, multi-objective, preemptive, CLP, completion-time, preempt, CP, Benders Decomposition, constraint logic programming, Logic-Based Benders Decomposition, task, constraint satisfaction, order, machine, periodic, scheduling, explanation, resource &  & cycle &  & Choco Solver & automotive, robot, aircraft &  & benchmark & genetic algorithm, simulated annealing, neural network & \ref{a:HladikCDJ08} & n/a\\
\index{Hooker05}\rowlabel{b:Hooker05}\href{../works/Hooker05.pdf}{Hooker05}~\cite{Hooker05} & 17 & constraint satisfaction, machine, job, task, release-date, constraint programming, Logic-Based Benders Decomposition, CP, make-span, explanation, constraint logic programming, distributed, resource, precedence, due-date, order, tardiness, CLP, scheduling, Benders Decomposition &  & disjunctive, cumulative, circuit &  & OPL, Ilog Scheduler, Cplex &  &  & random instance & MINLP, edge-finding & \ref{a:Hooker05} & \ref{c:Hooker05}\\
\index{Hooker06}\rowlabel{b:Hooker06}\href{../works/Hooker06.pdf}{Hooker06}~\cite{Hooker06} & 19 & constraint satisfaction, machine, job, task, release-date, constraint programming, Logic-Based Benders Decomposition, CP, make-span, constraint logic programming, resource, precedence, due-date, order, tardiness, scheduling, Benders Decomposition &  & disjunctive, cumulative, circuit &  & OPL, Ilog Scheduler, Cplex &  &  & random instance & MINLP & \ref{a:Hooker06} & \ref{c:Hooker06}\\
\index{Hooker07}\rowlabel{b:Hooker07}\href{../works/Hooker07.pdf}{Hooker07}~\cite{Hooker07} & 15 & constraint satisfaction, machine, job, task, release-date, constraint programming, Logic-Based Benders Decomposition, inventory, activity, CP, make-span, constraint logic programming, distributed, resource, precedence, due-date, order, tardiness, scheduling, Benders Decomposition &  & disjunctive, cumulative, circuit &  & OPL, Ilog Scheduler, Cplex &  &  & random instance, generated instance & MINLP, edge-finding & \ref{a:Hooker07} & n/a\\
\index{HookerH17}\rowlabel{b:HookerH17}\href{../works/HookerH17.pdf}{HookerH17}~\cite{HookerH17} & 24 & preemptive, CLP, CSP, multi-agent, CP, machine, stochastic, constraint programming, one-machine scheduling, job-shop, resource, transportation, constraint optimization, open-shop, Benders Decomposition, order, multi-objective, activity, constraint satisfaction, setup-time, release-date, scheduling, Logic-Based Benders Decomposition, task, constraint logic programming, job, sequence dependent setup, preempt, flow-shop, net present value, explanation, tardiness & Open Shop Scheduling Problem, parallel machine, RCPSP & bin-packing, regular expression, Regular constraint, Cardinality constraint, Among constraint, circuit, cumulative, alldifferent, disjunctive &  & OPL, CHIP, SCIP, ECLiPSe, MiniZinc, Ilog Solver & aircraft, crew-scheduling, travelling tournament problem, sports scheduling, operating room, radiation therapy, tournament, nurse, physician &  & real-world, real-life & time-tabling, MINLP, bi-partite matching, energetic reasoning, edge-finding, column generation, not-first, not-last, neural network, Lagrangian relaxation & \ref{a:HookerH17} & n/a\\
\index{HookerO03}\rowlabel{b:HookerO03}\href{../works/HookerO03.pdf}{HookerO03}~\cite{HookerO03} & 28 & CLP, CP, machine, constraint programming, one-machine scheduling, resource, Benders Decomposition, order, constraint satisfaction, release-date, scheduling, Logic-Based Benders Decomposition, task, constraint logic programming, job, due-date &  & circuit, cumulative, disjunctive &  & Ilog Scheduler, OPL, Cplex &  &  & generated instance &  & \ref{a:HookerO03} & n/a\\
\index{HookerO99}\rowlabel{b:HookerO99}\href{../works/HookerO99.pdf}{HookerO99}~\cite{HookerO99} & 48 & CLP, CP, machine, stochastic, constraint programming, job-shop, resource, Benders Decomposition, order, make to order, constraint satisfaction, scheduling, Logic-Based Benders Decomposition, make-span, task, constraint logic programming, job, explanation &  & Disjunctive constraint, disjunctive & C++, Prolog & CHIP, Cplex, Ilog Solver &  & chemical industry &  & MINLP & \ref{a:HookerO99} & n/a\\
\index{HookerOTK00}\rowlabel{b:HookerOTK00}\href{../works/HookerOTK00.pdf}{HookerOTK00}~\cite{HookerOTK00} & 20 & order, machine, job, scheduling, explanation, resource, CSP, job-shop, transportation, constraint logic programming, stochastic, constraint programming, CLP, CP, Benders Decomposition, Logic-Based Benders Decomposition, task, constraint satisfaction &  & cumulative, Element constraint, disjunctive, Cardinality constraint &  & Cplex, Ilog Solver, CHIP, OPL & hoist &  &  & MINLP & \ref{a:HookerOTK00} & n/a\\
\index{HoundjiSW19}\rowlabel{b:HoundjiSW19}\href{../works/HoundjiSW19.pdf}{HoundjiSW19}~\cite{HoundjiSW19} & 27 & scheduling, resource, BOM, due-date, transportation, inventory, constraint programming, CP, constraint logic programming, task, order, machine & single machine & alldifferent, GCC constraint, circuit, Cardinality constraint, cumulative &  &  &  &  & random instance, benchmark, bitbucket & sweep, column generation, max-flow & \ref{a:HoundjiSW19} & \ref{c:HoundjiSW19}\\
\index{HubnerGSV21}\rowlabel{b:HubnerGSV21}\href{../works/HubnerGSV21.pdf}{HubnerGSV21}~\cite{HubnerGSV21} & 22 & completion-time, CP, task, stochastic, precedence, reactive scheduling, order, machine, preempt, cmax, tardiness, scheduling, resource, due-date, no-wait, transportation, inventory, job, activity, constraint programming, make-span, preemptive & RCPSPDC, Resource-constrained Project Scheduling Problem, RCPSP & cumulative, cycle, alternative constraint, endBeforeStart & C  & Cplex, Gurobi, OPL & automotive, tournament & dismantling industry & real-life, benchmark & large neighborhood search, meta heuristic, mat heuristic, genetic algorithm, simulated annealing & \ref{a:HubnerGSV21} & n/a\\
\index{IsikYA23}\rowlabel{b:IsikYA23}\href{../works/IsikYA23.pdf}{IsikYA23}~\cite{IsikYA23} & 28 & tardiness, scheduling, energy efficiency, multi-objective, constraint programming, completion-time, flow-shop, constraint satisfaction, task, no-wait, job-shop, blocking constraint, CP, explanation, earliness, bi-objective, constraint logic programming, preempt, batch process, setup-time, due-date, order, make-span, machine, job, distributed, resource, release-date, transportation, precedence, cmax, sequence dependent setup, breakdown & single machine, parallel machine, HFS & noOverlap, endBeforeStart, Calendar constraint, circuit, Blocking constraint, cumulative &  & OPL, Cplex & robot, medical & steel industry & benchmark, generated instance, real-life, real-world & Lagrangian relaxation, energetic reasoning, NEH, GRASP, genetic algorithm, memetic algorithm, machine learning, mat heuristic, meta heuristic, reinforcement learning, neural network, ant colony, particle swarm, simulated annealing & \ref{a:IsikYA23} & \ref{c:IsikYA23}\\
\index{JainG01}\rowlabel{b:JainG01}\href{../works/JainG01.pdf}{JainG01}~\cite{JainG01} & 19 & job-shop, constraint satisfaction, job, order, release-date, scheduling, constraint logic programming, CP, CLP, constraint programming, due-date, Benders Decomposition, task, resource, machine, activity & single machine, parallel machine & disjunctive, cumulative & Prolog & Ilog Solver, ECLiPSe, CHIP, OPL, Ilog Scheduler, Cplex & crew-scheduling &  &  & column generation, MINLP & \ref{a:JainG01} & n/a\\
\index{JainM99}\rowlabel{b:JainM99}\href{../works/JainM99.pdf}{JainM99}~\cite{JainM99} & 45 & flow-shop, preempt, constraint satisfaction, one-machine scheduling, job, open-shop, order, release-date, constraint optimization, scheduling, CP, precedence, CLP, cmax, tardiness, stochastic, due-date, re-scheduling, completion-time, CSP, preemptive, lateness, task, resource, make-span, single-machine scheduling, machine, distributed, inventory, job-shop & single machine & disjunctive, cycle &  & OPL & semiconductor, robot &  & real-world, benchmark, real-life & Lagrangian relaxation, edge-finder, memetic algorithm, simulated annealing, meta heuristic, GRASP, neural network, genetic algorithm, machine learning & \ref{a:JainM99} & n/a\\
\index{Jans09}\rowlabel{b:Jans09}\href{../works/Jans09.pdf}{Jans09}~\cite{Jans09} & 14 & distributed, CP, scheduling, sequence dependent setup, resource, job, setup-time, multi-agent, inventory, machine, order & single machine, parallel machine &  &  & Cplex & offshore, business process & foundry industry, tire industry, fashion industry, process industry & benchmark & column generation, meta heuristic & \ref{a:Jans09} & n/a\\
\index{JussienL02}\rowlabel{b:JussienL02}\href{../works/JussienL02.pdf}{JussienL02}~\cite{JussienL02} & 25 & job, constraint programming, CSP, precedence, stochastic, task, CLP, machine, order, constraint logic programming, scheduling, preempt, make-span, constraint satisfaction, preemptive, job-shop, explanation, CP, open-shop, resource & Open Shop Scheduling Problem, TMS &  &  &  & satellite &  & benchmark, real-life & genetic algorithm, time-tabling, neural network & \ref{a:JussienL02} & n/a\\
\index{JuvinHL22}\rowlabel{b:JuvinHL22}\href{../works/JuvinHL22.pdf}{JuvinHL22}~\cite{JuvinHL22} & 32 & precedence, preemptive, completion-time, cmax, CP, machine, re-scheduling, distributed, constraint programming, job-shop, resource, Benders Decomposition, order, activity, constraint satisfaction, setup-time, Pareto, release-date, scheduling, Logic-Based Benders Decomposition, make-span, task, job, preempt, flow-shop & parallel machine, single machine, JSSP, FJS & endBeforeStart, disjunctive, Disjunctive constraint, noOverlap, circuit, cumulative &  & Cplex, CPO &  &  & benchmark & meta heuristic, simulated annealing, genetic algorithm & \ref{a:JuvinHL22} & n/a\\
\index{JuvinHL23a}\rowlabel{b:JuvinHL23a}\href{../works/JuvinHL23a.pdf}{JuvinHL23a}~\cite{JuvinHL23a} & 17 & task, machine, make-span, preempt, constraint satisfaction, resource, distributed, Logic-Based Benders Decomposition, preemptive, job-shop, activity, flow-shop, constraint programming, precedence, CP, Benders Decomposition, stochastic, setup-time, order, re-scheduling, job, release-date, scheduling, Pareto & FJS, JSSP, parallel machine, single machine & noOverlap, endBeforeStart, bin-packing, cumulative, circuit, disjunctive, Disjunctive constraint &  & CPO, Cplex & vaccine, drone, operating room, COVID &  & benchmark & genetic algorithm, machine learning, simulated annealing, meta heuristic & \ref{a:JuvinHL23a} & n/a\\
\index{Kameugne15}\rowlabel{b:Kameugne15}\href{../works/Kameugne15.pdf}{Kameugne15}~\cite{Kameugne15} & 2 & resource, preemptive, completion-time, constraint programming, scheduling, task, preempt &  & cumulative &  &  &  &  &  & not-last, not-first, edge-finding & \ref{a:Kameugne15} & \ref{c:Kameugne15}\\
\index{KameugneF13}\rowlabel{b:KameugneF13}\href{../works/KameugneF13.pdf}{KameugneF13}~\cite{KameugneF13} & 21 & order, task, release-date &  & cumulative &  &  &  &  &  & not-first & \ref{a:KameugneF13} & n/a\\
\index{KameugneFSN14}\rowlabel{b:KameugneFSN14}\href{../works/KameugneFSN14.pdf}{KameugneFSN14}~\cite{KameugneFSN14} & 27 & completion-time, preemptive, CP, scheduling, CLP, precedence, job-shop, release-date, resource, job, order, constraint programming, preempt, make-span, task & RCPSP, psplib, CuSP, Resource-constrained Project Scheduling Problem & disjunctive, cumulative &  & CHIP, Gecode &  &  & random instance, benchmark & edge-finding, not-first, edge-finder, time-tabling, energetic reasoning, not-last & \ref{a:KameugneFSN14} & \ref{c:KameugneFSN14}\\
\index{KelbelH11}\rowlabel{b:KelbelH11}\href{../works/KelbelH11.pdf}{KelbelH11}~\cite{KelbelH11} & 10 & due-date, preempt, precedence, tardiness, earliness, CSP, CP, machine, inventory, constraint programming, job-shop, resource, order, completion-time, constraint satisfaction, release-date, scheduling, make-span, distributed, task, job & JSSP & cumulative, disjunctive &  & Cplex, Ilog Solver, OPL &  &  & generated instance, benchmark, random instance & edge-finder, large neighborhood search, edge-finding & \ref{a:KelbelH11} & n/a\\
\index{KhayatLR06}\rowlabel{b:KhayatLR06}\href{../works/KhayatLR06.pdf}{KhayatLR06}~\cite{KhayatLR06} & 15 & order, cmax, scheduling, preempt, task, machine, make-span, constraint satisfaction, job-shop, due-date, constraint logic programming, CP, resource, preemptive, activity, job, constraint programming, precedence, setup-time &  &  &  & OPL, Cplex &  &  & real-life, benchmark & genetic algorithm & \ref{a:KhayatLR06} & n/a\\
\index{KoehlerBFFHPSSS21}\rowlabel{b:KoehlerBFFHPSSS21}\href{../works/KoehlerBFFHPSSS21.pdf}{KoehlerBFFHPSSS21}~\cite{KoehlerBFFHPSSS21} & 51 & constraint satisfaction, flow-shop, CSP, job, constraint programming, tardiness, task, multi-objective, scheduling, single-machine scheduling, make-span, resource, precedence, job-shop, order, lateness, CP, machine, one-machine scheduling, flow-time, constraint optimization & CTW, single machine & Channeling constraint, alldifferent, Disjunctive constraint, circuit, cumulative, cycle, disjunctive & C , Python & MiniZinc, OPL, Cplex, Gurobi, OR-Tools, Chuffed, Z3 & cable tree, automotive, robot &  & real-world, benchmark, github & ant colony, genetic algorithm, particle swarm, simulated annealing & \ref{a:KoehlerBFFHPSSS21} & \ref{c:KoehlerBFFHPSSS21}\\
\index{KorbaaYG00}\rowlabel{b:KorbaaYG00}\href{../works/KorbaaYG00.pdf}{KorbaaYG00}~\cite{KorbaaYG00} & 10 &  &  &  &  &  &  &  &  &  & \ref{a:KorbaaYG00} & n/a\\
\index{KovacsB08}\rowlabel{b:KovacsB08}\href{../works/KovacsB08.pdf}{KovacsB08}~\cite{KovacsB08} & 7 & order, activity, preempt, release-date, single-machine scheduling, scheduling, job, preemptive, machine, tardiness, constraint programming, CSP, completion-time, CP, resource & single machine & disjunctive, Disjunctive constraint, bin-packing, cumulative, cycle, Regular constraint, Completion constraint, Cardinality constraint &  & Ilog Solver, Ilog Scheduler & aircraft &  & benchmark & genetic algorithm, sweep & \ref{a:KovacsB08} & n/a\\
\index{KovacsB11}\rowlabel{b:KovacsB11}\href{../works/KovacsB11.pdf}{KovacsB11}~\cite{KovacsB11} & 24 & order, activity, preempt, release-date, single-machine scheduling, scheduling, make-span, flow-time, job, preemptive, due-date, flow-shop, machine, precedence, tardiness, constraint programming, earliness, completion-time, CP, distributed, job-shop, resource & single machine, parallel machine, Resource-constrained Project Scheduling Problem & disjunctive, Disjunctive constraint, cumulative, cycle, Regular constraint, Completion constraint, Cardinality constraint, Channeling constraint & C++ & Ilog Solver, Ilog Scheduler &  &  & benchmark & edge-finding, column generation & \ref{a:KovacsB11} & \ref{c:KovacsB11}\\
\index{KovacsK11}\rowlabel{b:KovacsK11}\href{../works/KovacsK11.pdf}{KovacsK11}~\cite{KovacsK11} & 24 & Benders Decomposition, order, constraint satisfaction, breakdown, Pareto, release-date, scheduling, task, COP, job, sequence dependent setup, due-date, flow-shop, machine, stochastic, tardiness, constraint programming, constraint optimization, earliness, CSP, completion-time, CP, Logic-Based Benders Decomposition, job-shop, resource, transportation & single machine & cycle, Reified constraint & C++ & Cplex, Ilog Solver, Gecode &  &  &  &  & \ref{a:KovacsK11} & \ref{c:KovacsK11}\\
\index{KreterSS17}\rowlabel{b:KreterSS17}\href{../works/KreterSS17.pdf}{KreterSS17}~\cite{KreterSS17} & 31 & order, scheduling, task, unavailability, machine, make-span, periodic, preempt, resource, preemptive, activity, explanation, constraint programming, completion-time, precedence, CP & Resource-constrained Project Scheduling Problem, RCPSP, parallel machine & IloPulse, cumulative, IloForbidEnd, Pulse constraint, Reified constraint, Calendar constraint, alwaysIn, diffn, cycle, IloAlwaysIn, Element constraint &  & MiniZinc, Chuffed, CPO, Cplex, CHIP &  &  & benchmark & edge-finding, lazy clause generation & \ref{a:KreterSS17} & \ref{c:KreterSS17}\\
\index{KreterSSZ18}\rowlabel{b:KreterSSZ18}\href{../works/KreterSSZ18.pdf}{KreterSSZ18}~\cite{KreterSSZ18} & 15 & activity, constraint programming, precedence, release-date, preemptive, completion-time, preempt, CP, task, unavailability, order, machine, tardiness, periodic, scheduling, explanation, resource & Resource-constrained Project Scheduling Problem, RCPSP, psplib & cumulative, Element constraint, Calendar constraint &  & Chuffed, Cplex, MiniZinc &  &  & benchmark & genetic algorithm, particle swarm, GRASP, lazy clause generation, Lagrangian relaxation & \ref{a:KreterSSZ18} & n/a\\
\index{KuB16}\rowlabel{b:KuB16}\href{../works/KuB16.pdf}{KuB16}~\cite{KuB16} & 9 & earliness, job, order, precedence, make-span, machine, tardiness, completion-time, job-shop, scheduling, constraint programming, constraint satisfaction, CP &  & Disjunctive constraint, disjunctive &  & Ilog Scheduler, Cplex, SCIP, Gurobi &  &  & benchmark & meta heuristic, genetic algorithm & \ref{a:KuB16} & n/a\\
\index{Kuchcinski03}\rowlabel{b:Kuchcinski03}\href{../works/Kuchcinski03.pdf}{Kuchcinski03}~\cite{Kuchcinski03} & 29 & constraint logic programming, task, constraint programming, completion-time, precedence, periodic, CLP, scheduling, job, re-scheduling, CP, multi-objective, order, distributed, job-shop, resource, CSP, constraint optimization &  & Reified constraint, Disjunctive constraint, Arithmetic constraint, diffn, cumulative, circuit, Element constraint, disjunctive, Diff2 constraint, cycle & Prolog, Java & SICStus, CHIP & pipeline &  & benchmark & genetic algorithm, meta heuristic, edge-finding & \ref{a:Kuchcinski03} & n/a\\
\index{KuchcinskiW03}\rowlabel{b:KuchcinskiW03}\href{../works/KuchcinskiW03.pdf}{KuchcinskiW03}~\cite{KuchcinskiW03} & 15 & distributed, precedence, CP, CSP, scheduling, constraint programming, resource, constraint logic programming, order &  & cycle, circuit, Diff2 constraint & Java &  & pipeline &  & benchmark &  & \ref{a:KuchcinskiW03} & n/a\\
\index{Laborie03}\rowlabel{b:Laborie03}\href{../works/Laborie03.pdf}{Laborie03}~\cite{Laborie03} & 38 & task, job, preempt, precedence, preemptive, cmax, CP, machine, re-scheduling, inventory, constraint programming, job-shop, resource, order, activity, constraint satisfaction, setup-time, release-date, scheduling, make-span &  & cumulative, disjunctive, table constraint, cycle, Balance constraint, Disjunctive constraint & C++ & Ilog Scheduler &  &  & benchmark & edge-finding, not-first, not-last, time-tabling, energetic reasoning & \ref{a:Laborie03} & n/a\\
\index{LaborieR14}\rowlabel{b:LaborieR14}\href{../works/LaborieR14.pdf}{LaborieR14}~\cite{LaborieR14} & 10 & single-machine scheduling, order, breakdown, transportation, job, constraint programming, scheduling, task, machine, net present value, tardiness, earliness, preempt, resource, Logic-Based Benders Decomposition, job-shop, due-date, activity, flow-shop, precedence, CP, Benders Decomposition & single machine, Partial Order Schedule, RCPSP & disjunctive, span constraint, noOverlap, endBeforeStart, cumulative, alternative constraint &  & Cplex & satellite, aircraft &  & benchmark, real-world & column generation, large neighborhood search, machine learning & \ref{a:LaborieR14} & n/a\\
\index{LaborieRSV18}\rowlabel{b:LaborieRSV18}\href{../works/LaborieRSV18.pdf}{LaborieRSV18}~\cite{LaborieRSV18} & 41 & Benders Decomposition, multi-objective, breakdown, manpower, setup-time, order, distributed, Logic-Based Benders Decomposition, job-shop, resource, CSP, net present value, batch process, stochastic, task, constraint programming, release-date, precedence, earliness, sequence dependent setup, scheduling, machine, transportation, periodic, tardiness, make-span, activity, explanation, inventory, due-date, flow-shop, job, re-scheduling, CP & Resource-constrained Project Scheduling Problem, psplib, RCPSP, parallel machine & endBeforeStart, noOverlap, alternative constraint, cumulative, disjunctive, span constraint, cycle, Reified constraint, AlwaysConstant, Disjunctive constraint, alwaysEqual constraint, Arithmetic constraint, Calendar constraint, alwaysIn & Python, C++, C , Java & Ilog Scheduler, OPL, CHIP, Choco Solver, Gecode, Ilog Solver, Cplex, CPO & semiconductor, satellite, aircraft, robot, pipeline, shipping line, railway, container terminal & petro-chemical industry, chemical industry & CSPlib, benchmark, real-world & edge-finding, large neighborhood search & \ref{a:LaborieRSV18} & \ref{c:LaborieRSV18}\\
\index{LacknerMMWW23}\rowlabel{b:LacknerMMWW23}\href{../works/LacknerMMWW23.pdf}{LacknerMMWW23}~\cite{LacknerMMWW23} & 42 & explanation, CP, scheduling, machine, lateness, constraint optimization, batch process, job-shop, constraint programming, release-date, job, order, tardiness, earliness, setup-time, Pareto, due-date, multi-objective, make-span, task & single machine, parallel machine, OSP & disjunctive, bin-packing, alternative constraint, cumulative, endBeforeStart, noOverlap, Element constraint &  & Chuffed, Cplex, Gurobi, OPL, CPO, MiniZinc, OR-Tools & semiconductor, oven scheduling & manufacturing industry, electronics industry, steel industry & zenodo, random instance, benchmark, instance generator, real-life, industrial partner & ant colony, GRASP, simulated annealing, large neighborhood search, time-tabling, particle swarm, meta heuristic, genetic algorithm & \ref{a:LacknerMMWW23} & \ref{c:LacknerMMWW23}\\
\index{LammaMM97}\rowlabel{b:LammaMM97}\href{../works/LammaMM97.pdf}{LammaMM97}~\cite{LammaMM97} & 15 & CP, CLP, order, CSP, distributed, job-shop, resource, constraint logic programming, job, constraint satisfaction, no-wait, scheduling, precedence, task &  & Disjunctive constraint, circuit, disjunctive & C++, Prolog & ECLiPSe, OPL, CHIP & railway, train schedule &  & real-life &  & \ref{a:LammaMM97} & n/a\\
\index{LetortCB15}\rowlabel{b:LetortCB15}\href{../works/LetortCB15.pdf}{LetortCB15}~\cite{LetortCB15} & 52 & job, constraint programming, precedence, CP, order, scheduling, task, CLP, machine, make-span, resource & Resource-constrained Project Scheduling Problem, psplib & cumulative, Cumulatives constraint, cycle, bin-packing & Java, Prolog & Choco Solver, CHIP, SICStus &  &  & generated instance, benchmark, random instance, Roadef & energetic reasoning, large neighborhood search, meta heuristic, sweep, edge-finding & \ref{a:LetortCB15} & \ref{c:LetortCB15}\\
\index{LiW08}\rowlabel{b:LiW08}\href{../works/LiW08.pdf}{LiW08}~\cite{LiW08} & 18 & activity, setup-time, scheduling, constraint programming, no preempt, constraint satisfaction, CP, CSP, resource, explanation, job, due-date, order, precedence, make-span, machine, preempt, task, completion-time, job-shop, re-scheduling, open-shop, Benders Decomposition, constraint logic programming & RCPSP & disjunctive, bin-packing, cycle &  & Ilog Solver, Cplex, ECLiPSe, CHIP, OPL & tournament, travelling tournament problem, astronomy &  & real-world & Lagrangian relaxation & \ref{a:LiW08} & n/a\\
\index{LiessM08}\rowlabel{b:LiessM08}\href{../works/LiessM08.pdf}{LiessM08}~\cite{LiessM08} & 12 & activity, job-shop, CP, make-span, preempt, resource, precedence, preemptive, order, machine, job, cmax, scheduling, constraint programming, constraint satisfaction, task & psplib, Resource-constrained Project Scheduling Problem, RCPSP & cumulative, disjunctive & C++ &  &  &  & benchmark & edge-finding, meta heuristic, large neighborhood search, column generation & \ref{a:LiessM08} & n/a\\
\index{LimtanyakulS12}\rowlabel{b:LimtanyakulS12}\href{../works/LimtanyakulS12.pdf}{LimtanyakulS12}~\cite{LimtanyakulS12} & 32 & precedence, constraint optimization, release-date, activity, tardiness, constraint programming, machine, scheduling, order, Benders Decomposition, due-date, stochastic, CP, completion-time, resource, job, COP, CSP, constraint satisfaction & Resource-constrained Project Scheduling Problem & bin-packing, disjunctive, table constraint, Cardinality constraint, cumulative &  & Ilog Scheduler, Cplex & robot, automotive & automotive industry & real-life, generated instance, industrial partner, benchmark, random instance & not-last, not-first, genetic algorithm, energetic reasoning, edge-finding & \ref{a:LimtanyakulS12} & \ref{c:LimtanyakulS12}\\
\index{LiuW11}\rowlabel{b:LiuW11}\href{../works/LiuW11.pdf}{LiuW11}~\cite{LiuW11} & 10 & periodic, stochastic, inventory, machine, constraint programming, task, order, constraint satisfaction, completion-time, CP, transportation, due-date, distributed, resource, preempt, preemptive, multi-objective, CSP, activity, scheduling, precedence & Resource-constrained Project Scheduling Problem, RCPSP & cumulative &  &  & robot, train schedule &  &  & simulated annealing, time-tabling, genetic algorithm, neural network, ant colony & \ref{a:LiuW11} & n/a\\
\index{LombardiM10a}\rowlabel{b:LombardiM10a}\href{../works/LombardiM10a.pdf}{LombardiM10a}~\cite{LombardiM10a} & 30 & task, preemptive, completion-time, precedence, scheduling, machine, make-span, activity, preempt, constraint satisfaction, due-date, distributed, job, re-scheduling, CP, Benders Decomposition, stochastic, constraint programming, order, release-date, Logic-Based Benders Decomposition, resource, CSP & TCSP, Resource-constrained Project Scheduling Problem & Disjunctive constraint, cycle, table constraint, span constraint, cumulative, disjunctive & C  & Cplex & business process &  & benchmark, real-life, real-world & genetic algorithm, sweep & \ref{a:LombardiM10a} & n/a\\
\index{LombardiM12}\rowlabel{b:LombardiM12}\href{../works/LombardiM12.pdf}{LombardiM12}~\cite{LombardiM12} & 35 & precedence, flow-shop, reactive scheduling, sequence dependent setup, make-span, order, machine, preempt, CP, tardiness, re-scheduling, due-date, CSP, distributed, manpower, job, activity, scheduling, explanation, resource, energy efficiency, preemptive, job-shop, transportation, completion-time, setup-time, earliness, Benders Decomposition, Logic-Based Benders Decomposition, task, inventory, stochastic, constraint satisfaction, constraint programming & psplib, Partial Order Schedule, parallel machine, Resource-constrained Project Scheduling Problem, RCPSP & circuit, cycle, cumulative, Disjunctive constraint, disjunctive &  & OR-Tools & aircraft & chemical industry & real-world, benchmark & large neighborhood search, lazy clause generation, energetic reasoning, meta heuristic, edge-finding, genetic algorithm & \ref{a:LombardiM12} & \ref{c:LombardiM12}\\
\index{LombardiM12a}\rowlabel{b:LombardiM12a}\href{../works/LombardiM12a.pdf}{LombardiM12a}~\cite{LombardiM12a} & 10 & completion-time, precedence, scheduling, make-span, activity, producer/consumer, CP, stochastic, constraint programming, order, resource, CSP & psplib, Partial Order Schedule, Resource-constrained Project Scheduling Problem, RCPSP & disjunctive &  & Ilog Solver &  &  & benchmark &  & \ref{a:LombardiM12a} & n/a\\
\index{LombardiMB13}\rowlabel{b:LombardiMB13}\href{../works/LombardiMB13.pdf}{LombardiMB13}~\cite{LombardiMB13} & 14 & cmax, task, preemptive, completion-time, precedence, scheduling, periodic, make-span, energy efficiency, activity, explanation, preempt, constraint satisfaction, distributed, re-scheduling, CP, multi-objective, stochastic, constraint programming, order, resource, CSP & SCC, RCPSP & cycle, cumulative, circuit &  & Gecode, Ilog Solver, OR-Tools & pipeline, medical &  & benchmark, real-world &  & \ref{a:LombardiMB13} & n/a\\
\index{LombardiMRB10}\rowlabel{b:LombardiMRB10}\href{../works/LombardiMRB10.pdf}{LombardiMRB10}~\cite{LombardiMRB10} & 31 & constraint programming, stochastic, resource, constraint logic programming, periodic, Benders Decomposition, completion-time, tardiness, Logic-Based Benders Decomposition, distributed, no preempt, preempt, make-span, task, precedence, preemptive, CP, activity, re-scheduling, producer/consumer, scheduling, release-date, order, constraint satisfaction & SCC & table constraint, cumulative, Disjunctive constraint, circuit, disjunctive, cycle, bin-packing & C  & ECLiPSe, Cplex & pipeline, semiconductor & semiconductor industry & real-life, real-world, benchmark & genetic algorithm, simulated annealing & \ref{a:LombardiMRB10} & n/a\\
\index{LopesCSM10}\rowlabel{b:LopesCSM10}\href{../works/LopesCSM10.pdf}{LopesCSM10}~\cite{LopesCSM10} & 39 & distributed, stock level, job-shop, due-date, activity, explanation, CP, resource, transportation, reactive scheduling, job, constraint programming, precedence, inventory, CLP, order, multi-objective, re-scheduling, scheduling, task, make-span &  & table constraint, cycle, alldifferent, disjunctive & C++ & Ilog Scheduler, Ilog Solver, OPL & pipeline & oil industry & real-world, benchmark & meta heuristic, MINLP, genetic algorithm, max-flow & \ref{a:LopesCSM10} & \ref{c:LopesCSM10}\\
\index{LopezAKYG00}\rowlabel{b:LopezAKYG00}\href{../works/LopezAKYG00.pdf}{LopezAKYG00}~\cite{LopezAKYG00} & 4 &  &  &  &  &  &  &  &  &  & \ref{a:LopezAKYG00} & n/a\\
\index{LorigeonBB02}\rowlabel{b:LorigeonBB02}\href{../works/LorigeonBB02.pdf}{LorigeonBB02}~\cite{LorigeonBB02} & 8 & resource, cmax, completion-time, scheduling, machine, make-span, unavailability, activity, setup-time, preempt, flow-shop, job, CP, open-shop, order & parallel machine, Open Shop Scheduling Problem &  &  & Cplex, OPL &  &  &  &  & \ref{a:LorigeonBB02} & n/a\\
\index{LuZZYW24}\rowlabel{b:LuZZYW24}\href{../works/LuZZYW24.pdf}{LuZZYW24}~\cite{LuZZYW24} & 36 & job-shop, CP, explanation, single-machine scheduling, bi-objective, inventory, order, sustainability, machine, job, distributed, resource, transportation, precedence, scheduling, energy efficiency, multi-objective, constraint programming, Benders Decomposition, completion-time, flow-shop, constraint satisfaction, task, stochastic & Resource-constrained Project Scheduling Problem, single machine & cumulative, alwaysIn, disjunctive, noOverlap & Java & Cplex, OPL & energy-price, maintenance scheduling, train schedule, automotive, container terminal, railway & shipping industry & real-life, real-world & evolutionary computing, genetic algorithm, memetic algorithm, meta heuristic, ant colony, particle swarm, simulated annealing, large neighborhood search, column generation & \ref{a:LuZZYW24} & n/a\\
\index{LunardiBLRV20}\rowlabel{b:LunardiBLRV20}\href{../works/LunardiBLRV20.pdf}{LunardiBLRV20}~\cite{LunardiBLRV20} & 20 & job-shop, resource, job, order, tardiness, setup-time, constraint programming, preempt, make-span, unavailability, completion-time, flow-shop, CP, activity, re-scheduling, bi-objective, scheduling, due-date, machine, precedence & FJS & endBeforeStart, noOverlap & Python & Cplex & high performance computing & printing industry, glass industry & benchmark, github, random instance, generated instance & meta heuristic, genetic algorithm, large neighborhood search & \ref{a:LunardiBLRV20} & \ref{c:LunardiBLRV20}\\
\index{MalapertCGJLR12}\rowlabel{b:MalapertCGJLR12}\href{../works/MalapertCGJLR12.pdf}{MalapertCGJLR12}~\cite{MalapertCGJLR12} & 17 & constraint programming, transportation, flow-shop, CSP, order, cmax, open-shop, precedence, constraint satisfaction, completion-time, activity, machine, CP, preemptive, make-span, scheduling, resource, preempt, task, job, job-shop & OSP, Open Shop Scheduling Problem & disjunctive, cycle, Disjunctive constraint, cumulative & Java & Choco Solver &  &  & benchmark & meta heuristic, genetic algorithm, not-first, not-last, ant colony, edge-finding, particle swarm & \ref{a:MalapertCGJLR12} & n/a\\
\index{MalikMB08}\rowlabel{b:MalikMB08}\href{../works/MalikMB08.pdf}{MalikMB08}~\cite{MalikMB08} & 18 & machine, precedence, distributed, constraint programming, resource, constraint logic programming, order, scheduling &  & cycle, Cardinality constraint &  &  & pipeline &  & benchmark & edge-finding & \ref{a:MalikMB08} & n/a\\
\index{MaraveliasCG04}\rowlabel{b:MaraveliasCG04}\href{../works/MaraveliasCG04.pdf}{MaraveliasCG04}~\cite{MaraveliasCG04} & 29 & manpower, resource, task, inventory, constraint satisfaction, scheduling, order, make-span, activity, continuous-process, constraint programming, flow-shop, CP, job, re-scheduling, due-date &  & Balance constraint, cycle, disjunctive &  & Cplex, OPL &  & chemical industry &  & MINLP, edge-finding & \ref{a:MaraveliasCG04} & n/a\\
\index{MarliereSPR23}\rowlabel{b:MarliereSPR23}\href{../works/MarliereSPR23.pdf}{MarliereSPR23}~\cite{MarliereSPR23} & 22 & machine, precedence, transportation, job-shop, resource, job, order, constraint satisfaction, energy efficiency, constraint programming, distributed, multi-objective, task, explanation, CP, no-wait, activity, re-scheduling, scheduling & rtRTMP, RTMP & disjunctive, alternative constraint, cumulative, cycle, Disjunctive constraint, circuit, table constraint, noOverlap &  & Cplex & railway, robot, train schedule &  & benchmark, real-world & quadratic programming, time-tabling, Lagrangian relaxation, machine learning, meta heuristic & \ref{a:MarliereSPR23} & n/a\\
\index{MartinPY01}\rowlabel{b:MartinPY01}\href{../works/MartinPY01.pdf}{MartinPY01}~\cite{MartinPY01} & 17 & CP, order, breakdown, transportation, re-scheduling, constraint programming, scheduling, task, CLP, machine, constraint logic programming, constraint satisfaction, resource &  & circuit & Prolog & ECLiPSe, Ilog Solver & train schedule, railway, aircraft & sugar industry & real-life &  & \ref{a:MartinPY01} & n/a\\
\index{Mason01}\rowlabel{b:Mason01}\href{../works/Mason01.pdf}{Mason01}~\cite{Mason01} & 38 & cyclic scheduling, order, activity, scheduling, CP, transportation, task &  &  &  & Cplex, OPL & railway, airport, crew-scheduling, workforce scheduling, nurse & airline industry &  & Lagrangian relaxation, column generation & \ref{a:Mason01} & n/a\\
\index{MejiaY20}\rowlabel{b:MejiaY20}\href{../works/MejiaY20.pdf}{MejiaY20}~\cite{MejiaY20} & 13 & resource, cmax, sequence dependent setup, due-date, re-scheduling, preemptive, order, tardiness, scheduling, completion-time, machine, job, no-wait, open-shop, release-date, transportation, multi-agent, multi-objective, constraint programming, job-shop, bi-objective, preempt, setup-time, CP, make-span, distributed & OSSP, Open Shop Scheduling Problem, parallel machine & disjunctive, Disjunctive constraint & Java & Cplex, ECLiPSe & agriculture, robot &  & supplementary material, benchmark, generated instance & simulated annealing, ant colony, GRASP, particle swarm, genetic algorithm, meta heuristic & \ref{a:MejiaY20} & \ref{c:MejiaY20}\\
\index{MenciaSV12}\rowlabel{b:MenciaSV12}\href{../works/MenciaSV12.pdf}{MenciaSV12}~\cite{MenciaSV12} & 32 & lateness, preempt, sequence dependent setup, constraint satisfaction, resource, flow-time, preemptive, job-shop, job, constraint programming, completion-time, precedence, CP, setup-time, order, cmax, multi-objective, constraint optimization, scheduling, task, machine, make-span, distributed & JSSP, single machine & cycle, Disjunctive constraint, disjunctive &  &  & steel mill &  & benchmark, real-life & edge-finding, genetic algorithm, energetic reasoning, memetic algorithm, time-tabling, simulated annealing & \ref{a:MenciaSV12} & n/a\\
\index{MenciaSV13}\rowlabel{b:MenciaSV13}\href{../works/MenciaSV13.pdf}{MenciaSV13}~\cite{MenciaSV13} & 11 & lateness, preempt, sequence dependent setup, constraint satisfaction, resource, flow-time, preemptive, job-shop, flow-shop, job, constraint programming, completion-time, precedence, CP, setup-time, order, cmax, multi-objective, constraint optimization, scheduling, task, machine, make-span & JSSP, single machine & cycle, Disjunctive constraint, disjunctive &  &  & steel mill &  & benchmark, real-life, supplementary material & edge-finding, genetic algorithm, energetic reasoning, time-tabling, simulated annealing, meta heuristic & \ref{a:MenciaSV13} & \ref{c:MenciaSV13}\\
\index{MengGRZSC22}\rowlabel{b:MengGRZSC22}\href{../works/MengGRZSC22.pdf}{MengGRZSC22}~\cite{MengGRZSC22} & 13 & no-wait, distributed, job, job-shop, transportation, setup-time, Benders Decomposition, task, Pareto, constraint programming, bi-objective, precedence, flow-shop, multi-objective, make-span, order, machine, CP, tardiness, scheduling & parallel machine, FJS, HFS, PMSP &  &  & Cplex, Gurobi, OPL & semiconductor &  & benchmark & meta heuristic, genetic algorithm, memetic algorithm & \ref{a:MengGRZSC22} & n/a\\
\index{MengLZB21}\rowlabel{b:MengLZB21}\href{../works/MengLZB21.pdf}{MengLZB21}~\cite{MengLZB21} & 14 & flow-shop, machine, tardiness, earliness, completion-time, multi-agent, cmax, CP, distributed, job-shop, resource, transportation, order, multi-objective, setup-time, scheduling, make-span, task, job, sequence dependent setup, due-date & parallel machine, FJS, HFS & endBeforeStart, noOverlap, circuit, alternative constraint, cumulative &  & Cplex, OPL, Gurobi & semiconductor &  & benchmark & NEH, particle swarm, memetic algorithm, meta heuristic, simulated annealing, genetic algorithm & \ref{a:MengLZB21} & n/a\\
\index{MengZRZL20}\rowlabel{b:MengZRZL20}\href{../works/MengZRZL20.pdf}{MengZRZL20}~\cite{MengZRZL20} & 13 & job-shop, no-wait, flow-shop, completion-time, CP, batch process, open-shop, resource, energy efficiency, flow-time, transportation, job, constraint programming, precedence, Benders Decomposition, cyclic scheduling, task, setup-time, machine, order, cmax, multi-objective, tardiness, earliness, scheduling, preempt, sequence dependent setup, make-span, blocking constraint, distributed, no preempt & parallel machine, FJS, OSP, Open Shop Scheduling Problem, HFS & alternative constraint, Blocking constraint, noOverlap, endBeforeStart &  & OR-Tools, Gecode, OPL, Gurobi, Cplex & robot, semiconductor &  & benchmark, supplementary material & genetic algorithm, ant colony, particle swarm, simulated annealing, meta heuristic & \ref{a:MengZRZL20} & \ref{c:MengZRZL20}\\
\index{MercierH07}\rowlabel{b:MercierH07}\href{../works/MercierH07.pdf}{MercierH07}~\cite{MercierH07} & 27 & transportation, job, constraint programming, precedence, release-date, CLP, order, scheduling, task, machine, constraint satisfaction, job-shop, due-date, activity, explanation, completion-time, CP, resource & JSSP & cumulative, disjunctive &  & CHIP &  & steel industry & benchmark & not-last, edge-finder, edge-finding, energetic reasoning, not-first & \ref{a:MercierH07} & n/a\\
\index{MercierH08}\rowlabel{b:MercierH08}\href{../works/MercierH08.pdf}{MercierH08}~\cite{MercierH08} & 11 & preemptive, job, constraint programming, release-date, CLP, order, scheduling, preempt, task, job-shop, due-date, CP, resource &  & cumulative, disjunctive &  &  &  &  &  & edge-finder, edge-finding & \ref{a:MercierH08} & n/a\\
\index{MilanoW06}\rowlabel{b:MilanoW06}\href{../works/MilanoW06.pdf}{MilanoW06}~\cite{MilanoW06} & 45 & release-date, Logic-Based Benders Decomposition, distributed, one-machine scheduling, job-shop, resource, constraint logic programming, job, constraint satisfaction, preempt, setup-time, explanation, single-machine scheduling, scheduling, tardiness, task, preemptive, due-date, CP, machine, lateness, stochastic, constraint programming, transportation, Benders Decomposition, order, CSP, completion-time, activity & single machine, parallel machine & Cumulatives constraint, Reified constraint, Cardinality constraint, Channeling constraint, circuit, cumulative, alldifferent, GCC constraint &  & CHIP, ECLiPSe, Cplex, OPL & crew-scheduling &  & benchmark & column generation, edge-finder, meta heuristic, time-tabling, large neighborhood search, Lagrangian relaxation & \ref{a:MilanoW06} & n/a\\
\index{MilanoW09}\rowlabel{b:MilanoW09}\href{../works/MilanoW09.pdf}{MilanoW09}~\cite{MilanoW09} & 40 & release-date, Logic-Based Benders Decomposition, distributed, one-machine scheduling, job-shop, resource, constraint logic programming, job, constraint satisfaction, preempt, setup-time, explanation, single-machine scheduling, scheduling, tardiness, task, preemptive, due-date, CP, machine, lateness, stochastic, constraint programming, transportation, Benders Decomposition, order, CSP, completion-time, activity & single machine & Cumulatives constraint, Reified constraint, Cardinality constraint, Channeling constraint, circuit, cumulative, alldifferent, GCC constraint &  & CHIP, SCIP, ECLiPSe, Cplex, OPL & crew-scheduling &  & benchmark & column generation, edge-finder, meta heuristic, lazy clause generation, time-tabling, large neighborhood search, Lagrangian relaxation & \ref{a:MilanoW09} & n/a\\
\index{MintonJPL92}\rowlabel{b:MintonJPL92}\href{../works/MintonJPL92.pdf}{MintonJPL92}~\cite{MintonJPL92} & 45 & CSP, constraint satisfaction, distributed, explanation, job, job-shop, machine, order, periodic, re-scheduling, resource, scheduling, stochastic, task &  & cumulative, cycle & Lisp & OPL & astronomy, robot, telescope &  & benchmark, real-world & neural network & \ref{a:MintonJPL92} & n/a\\
\index{MokhtarzadehTNF20}\rowlabel{b:MokhtarzadehTNF20}\href{../works/MokhtarzadehTNF20.pdf}{MokhtarzadehTNF20}~\cite{MokhtarzadehTNF20} & 14 & CP, distributed, order, job, constraint programming, machine, task, multi-agent, setup-time, manpower, no-wait, scheduling, make-span, resource, precedence, completion-time & parallel machine & alldifferent, circuit, cycle &  & Cplex & robot, crew-scheduling & circuit boards industry & real-world, generated instance & simulated annealing, time-tabling, meta heuristic, particle swarm & \ref{a:MokhtarzadehTNF20} & n/a\\
\index{MontemanniD23}\rowlabel{b:MontemanniD23}\href{../works/MontemanniD23.pdf}{MontemanniD23}~\cite{MontemanniD23} & 13 & constraint programming, resource, order, distributed, sustainability, task, CP, scheduling, machine &  & circuit & Python & OPL, OR-Tools, Gurobi & robot, drone &  & benchmark, supplementary material & machine learning, swarm intelligence, meta heuristic, ant colony, mat heuristic & \ref{a:MontemanniD23} & \ref{c:MontemanniD23}\\
\index{MontemanniD23a}\rowlabel{b:MontemanniD23a}\href{../works/MontemanniD23a.pdf}{MontemanniD23a}~\cite{MontemanniD23a} & 20 & explanation, completion-time, task, sustainability, transportation, scheduling, order, constraint programming, CP &  & circuit & Python & OR-Tools & drone &  & benchmark & meta heuristic, mat heuristic, ant colony & \ref{a:MontemanniD23a} & \ref{c:MontemanniD23a}\\
\index{MullerMKP22}\rowlabel{b:MullerMKP22}\href{../works/MullerMKP22.pdf}{MullerMKP22}~\cite{MullerMKP22} & 18 & bi-objective, precedence, batch process, multi-objective, make-span, order, machine, preempt, breakdown, cmax, scheduling, sustainability, due-date, no-wait, job, activity, resource, preemptive, job-shop, completion-time, setup-time, CP, online scheduling, task, stochastic, constraint programming & Resource-constrained Project Scheduling Problem, FJS & circuit, disjunctive & Python, Java & Chuffed, Choco Solver, OPL, OR-Tools, Cplex, MiniZinc, Gecode & robot, semiconductor &  & random instance, benchmark, github, real-world & reinforcement learning, neural network, meta heuristic, genetic algorithm, deep learning, machine learning & \ref{a:MullerMKP22} & \ref{c:MullerMKP22}\\
\index{NaderiBZ22}\rowlabel{b:NaderiBZ22}\href{../works/NaderiBZ22.pdf}{NaderiBZ22}~\cite{NaderiBZ22} & 29 & stochastic, setup-time, open-shop, order, scheduling, machine, make-span, distributed, Logic-Based Benders Decomposition, job-shop, due-date, tardiness, flow-shop, lateness, CP, resource, transportation, no-wait, job, constraint programming, completion-time, Benders Decomposition & parallel machine, single machine & disjunctive, noOverlap, Disjunctive constraint &  & Cplex, CPO & crew-scheduling, nurse, surgery, patient, operating room, automotive &  & benchmark, real-life & meta heuristic, memetic algorithm & \ref{a:NaderiBZ22} & n/a\\
\index{NaderiBZ22a}\rowlabel{b:NaderiBZ22a}\href{../works/NaderiBZ22a.pdf}{NaderiBZ22a}~\cite{NaderiBZ22a} & 19 & job-shop, distributed, transportation, job, constraint programming, precedence, flow-shop, multi-objective, make-span, preemptive, completion-time, setup-time, CP, Benders Decomposition, Logic-Based Benders Decomposition, task, stochastic, re-scheduling, sequence dependent setup, order, machine, preempt, tardiness, scheduling, resource & parallel machine & Disjunctive constraint, noOverlap, disjunctive, endBeforeStart & C++ & CPO, Cplex & nurse, automotive, crew-scheduling, robot, operating room &  & benchmark & genetic algorithm, simulated annealing, ant colony, meta heuristic & \ref{a:NaderiBZ22a} & n/a\\
\index{NaderiBZ23}\rowlabel{b:NaderiBZ23}\href{../works/NaderiBZ23.pdf}{NaderiBZ23}~\cite{NaderiBZ23} & 32 & stochastic, setup-time, open-shop, order, scheduling, machine, make-span, distributed, Logic-Based Benders Decomposition, job-shop, due-date, tardiness, flow-shop, lateness, CP, resource, transportation, no-wait, job, constraint programming, completion-time, Benders Decomposition & parallel machine, single machine & disjunctive, noOverlap, Disjunctive constraint & Python & Cplex, CPO & crew-scheduling, nurse, surgery, patient, operating room, automotive &  & benchmark, real-world & meta heuristic, memetic algorithm & \ref{a:NaderiBZ23} & n/a\\
\index{NaderiBZR23}\rowlabel{b:NaderiBZR23}\href{../works/NaderiBZR23.pdf}{NaderiBZR23}~\cite{NaderiBZR23} & 15 & distributed, transportation, job, activity, constraint programming, precedence, completion-time, setup-time, CP, Benders Decomposition, constraint logic programming, Logic-Based Benders Decomposition, task, stochastic, re-scheduling, bi-objective, order, machine, breakdown, periodic, scheduling, resource & parallel machine & bin-packing & Python & Cplex & nurse, surgery, automotive, crew-scheduling, medical, patient, operating room &  & supplementary material, benchmark, real-world, real-life, generated instance & large neighborhood search, meta heuristic, mat heuristic & \ref{a:NaderiBZR23} & \ref{c:NaderiBZR23}\\
\index{NaderiRR23}\rowlabel{b:NaderiRR23}\href{../works/NaderiRR23.pdf}{NaderiRR23}~\cite{NaderiRR23} & 27 & tardiness, flow-shop, earliness, CP, resource, preemptive, transportation, no-wait, job, constraint programming, completion-time, precedence, Benders Decomposition, setup-time, open-shop, order, cmax, re-scheduling, bi-objective, scheduling, preempt, sequence dependent setup, task, machine, make-span, constraint satisfaction, distributed, Logic-Based Benders Decomposition, job-shop, due-date & Open Shop Scheduling Problem, PMSP, RCPSP, parallel machine, Resource-constrained Project Scheduling Problem, OSP, PTC, single machine, FJS & Disjunctive constraint, cumulative, disjunctive, noOverlap, endBeforeStart, alternative constraint & Python & CPO, Z3, Gurobi, SCIP, Cplex & operating room, automotive, crew-scheduling, airport &  & github, benchmark & genetic algorithm, large neighborhood search, meta heuristic & \ref{a:NaderiRR23} & \ref{c:NaderiRR23}\\
\index{NaqviAIAAA22}\rowlabel{b:NaqviAIAAA22}\href{../works/NaqviAIAAA22.pdf}{NaqviAIAAA22}~\cite{NaqviAIAAA22} & 18 & order, distributed, CP, multi-objective, task, constraint programming, scheduling, machine, job, unavailability & TMS & cumulative &  &  & tournament, round-robin, pipeline, sports scheduling, travelling tournament problem &  & real-life, real-world & simulated annealing, time-tabling & \ref{a:NaqviAIAAA22} & n/a\\
\index{NattafAL15}\rowlabel{b:NattafAL15}\href{../works/NattafAL15.pdf}{NattafAL15}~\cite{NattafAL15} & 21 & release-date, scheduling, preempt, task, make-span, due-date, resource, preemptive, activity, constraint programming, CSP, CP, order & CECSP, RCPSP, Resource-constrained Project Scheduling Problem, CuSP & cumulative & C++ & Cplex &  &  & generated instance & sweep, energetic reasoning & \ref{a:NattafAL15} & \ref{c:NattafAL15}\\
\index{NattafAL17}\rowlabel{b:NattafAL17}\href{../works/NattafAL17.pdf}{NattafAL17}~\cite{NattafAL17} & 18 & release-date, scheduling, task, make-span, resource, energy efficiency, activity, job, constraint programming, CSP, CP, order & CECSP & disjunctive, cumulative & C++ & Cplex &  &  & real-world & edge-finding, energetic reasoning & \ref{a:NattafAL17} & \ref{c:NattafAL17}\\
\index{NattafALR16}\rowlabel{b:NattafALR16}\href{../works/NattafALR16.pdf}{NattafALR16}~\cite{NattafALR16} & 34 & preemptive, no preempt, task, constraint programming, precedence, make-span, order, preempt, CP, scheduling, due-date, CSP, activity, explanation, resource, release-date & CECSP, CuSP, Resource-constrained Project Scheduling Problem, RCPSP & cumulative & C++ & Cplex &  &  & generated instance & sweep, energetic reasoning & \ref{a:NattafALR16} & n/a\\
\index{NattafDYW19}\rowlabel{b:NattafDYW19}\href{../works/NattafDYW19.pdf}{NattafDYW19}~\cite{NattafDYW19} & 16 & job-shop, completion-time, setup-time, stochastic, constraint satisfaction, constraint programming, make-span, order, machine, CP, cmax, periodic, single-machine scheduling, scheduling, job, resource & parallel machine, single machine, PTC & noOverlap, alternative constraint &  & OPL, Cplex & semiconductor & lumber industry, semiconductor industry & benchmark & simulated annealing, memetic algorithm, meta heuristic, genetic algorithm & \ref{a:NattafDYW19} & n/a\\
\index{NattafHKAL19}\rowlabel{b:NattafHKAL19}\href{../works/NattafHKAL19.pdf}{NattafHKAL19}~\cite{NattafHKAL19} & 16 & preempt, single-machine scheduling, order, CP, resource, CSP, task, preemptive, release-date, activity, scheduling, machine, make-span & RCPSP, single machine, CECSP, Resource-constrained Project Scheduling Problem & cumulative &  & Cplex &  &  & real-life, benchmark & energetic reasoning & \ref{a:NattafHKAL19} & n/a\\
\index{NishikawaSTT19}\rowlabel{b:NishikawaSTT19}\href{../works/NishikawaSTT19.pdf}{NishikawaSTT19}~\cite{NishikawaSTT19} & 16 & online scheduling, precedence, scheduling, make-span, preempt, activity, distributed, constraint programming, machine, re-scheduling, order, CP, resource, task, preemptive & parallel machine & alternative constraint, cumulative &  & Cplex & robot, pipeline &  & real-world, benchmark & genetic algorithm, large neighborhood search & \ref{a:NishikawaSTT19} & n/a\\
\index{NovaraNH16}\rowlabel{b:NovaraNH16}\href{../works/NovaraNH16.pdf}{NovaraNH16}~\cite{NovaraNH16} & 17 & job, constraint programming, CSP, precedence, setup-time, order, re-scheduling, tardiness, constraint logic programming, scheduling, sequence dependent setup, manpower, task, machine, make-span, constraint satisfaction, due-date, activity, completion-time, earliness, CP, batch process, resource &  & cumulative, disjunctive, noOverlap, endBeforeStart, alternative constraint &  & OPL, Cplex &  & pharmaceutical industry & benchmark, CSPlib &  & \ref{a:NovaraNH16} & n/a\\
\index{Novas19}\rowlabel{b:Novas19}\href{../works/Novas19.pdf}{Novas19}~\cite{Novas19} & 13 & scheduling, CP, precedence, cmax, job-shop, constraint programming, multi-objective, due-date, completion-time, lateness, release-date, task, tardiness, resource, make-span, flow-time, transportation, sequence dependent setup, machine, no-wait, activity, distributed, inventory, setup-time, flow-shop, constraint satisfaction, job, order & parallel machine, HFS, FJS & cycle, noOverlap, cumulative, endBeforeStart &  & OPL, Cplex & medical, semiconductor, robot, train schedule & solar cell industry & benchmark & meta heuristic, swarm intelligence, genetic algorithm, particle swarm & \ref{a:Novas19} & n/a\\
\index{NovasH10}\rowlabel{b:NovasH10}\href{../works/NovasH10.pdf}{NovasH10}~\cite{NovasH10} & 20 & unavailability, precedence, batch process, due-date, re-scheduling, order, tardiness, scheduling, completion-time, machine, job, task, no-wait, breakdown, periodic, multi-objective, constraint programming, reactive scheduling, CSP, setup-time, manpower, activity, CP, make-span, earliness, resource, lateness &  &  &  & OPL, Ilog Scheduler & pipeline &  &  & meta heuristic & \ref{a:NovasH10} & n/a\\
\index{NovasH12}\rowlabel{b:NovasH12}\href{../works/NovasH12.pdf}{NovasH12}~\cite{NovasH12} & 17 & precedence, order, scheduling, completion-time, machine, job, task, no-wait, transportation, breakdown, constraint programming, reactive scheduling, activity, CP, make-span, resource &  & cycle &  & OPL, Ilog Solver, Ilog Scheduler & hoist, electroplating, container terminal, semiconductor, robot & semiconductor industry, electroplating industry &  &  & \ref{a:NovasH12} & n/a\\
\index{NovasH14}\rowlabel{b:NovasH14}\href{../works/NovasH14.pdf}{NovasH14}~\cite{NovasH14} & 14 & unavailability, precedence, order, scheduling, completion-time, machine, job, task, transportation, multi-objective, constraint programming, job-shop, reactive scheduling, constraint satisfaction, activity, CP, make-span, buffer-capacity, resource & single machine, parallel machine &  &  & OPL, Ilog Solver, Ilog Scheduler & robot &  & benchmark & ant colony, genetic algorithm & \ref{a:NovasH14} & n/a\\
\index{NuijtenA96}\rowlabel{b:NuijtenA96}\href{../works/NuijtenA96.pdf}{NuijtenA96}~\cite{NuijtenA96} & 16 & scheduling, preempt, machine, make-span, constraint satisfaction, preemptive, job-shop, flow-shop, completion-time, CP, resource, job, constraint programming, CSP, precedence, CLP, order & JSSP & disjunctive, Disjunctive constraint &  & CPO &  &  &  & time-tabling & \ref{a:NuijtenA96} & n/a\\
\index{NuijtenP98}\rowlabel{b:NuijtenP98}\href{../works/NuijtenP98.pdf}{NuijtenP98}~\cite{NuijtenP98} & 16 & scheduling, preempt, manpower, task, machine, make-span, constraint satisfaction, preemptive, job-shop, flow-shop, completion-time, CP, resource, transportation, reactive scheduling, job, constraint programming, CSP, precedence, setup-time, CLP, single-machine scheduling, order & single machine, JSSP & disjunctive, Disjunctive constraint & C++ & Ilog Solver, OPL, Ilog Scheduler & satellite &  & real-life & simulated annealing, edge-finding & \ref{a:NuijtenP98} & n/a\\
\index{OhrimenkoSC09}\rowlabel{b:OhrimenkoSC09}\href{../works/OhrimenkoSC09.pdf}{OhrimenkoSC09}~\cite{OhrimenkoSC09} & 35 & scheduling, machine, order, constraint satisfaction, constraint programming, resource, job, completion-time, explanation, open-shop, CSP, make-span, CP & Open Shop Scheduling Problem & Reified constraint, Arithmetic constraint, alldifferent, Cardinality constraint, disjunctive &  & Gecode &  &  & benchmark & lazy clause generation & \ref{a:OhrimenkoSC09} & n/a\\
\index{OrnekO16}\rowlabel{b:OrnekO16}\href{../works/OrnekO16.pdf}{OrnekO16}~\cite{OrnekO16} & 25 & constraint satisfaction, preempt, inventory, setup-time, activity, CP, make-span, earliness, distributed, resource, precedence, cmax, due-date, preemptive, order, tardiness, scheduling, completion-time, machine, job, bill of material, release-date, BOM, multi-objective, constraint programming, job-shop & parallel machine & Disjunctive constraint, cumulative, Element constraint, disjunctive &  & Cplex, OPL &  &  & real-life, real-world & genetic algorithm, meta heuristic, neural network, edge-finding & \ref{a:OrnekO16} & n/a\\
\index{OrnekOS20}\rowlabel{b:OrnekOS20}\href{../works/OrnekOS20.pdf}{OrnekOS20}~\cite{OrnekOS20} & 29 & explanation, CP, machine, stochastic, distributed, constraint programming, resource, order, multi-objective, Pareto, periodic, scheduling & parallel machine & disjunctive, noOverlap &  & Cplex & aircraft, airport &  & generated instance, real-world & genetic algorithm, time-tabling, large neighborhood search, particle swarm, meta heuristic, simulated annealing & \ref{a:OrnekOS20} & n/a\\
\index{OzturkTHO10}\rowlabel{b:OzturkTHO10}\href{../works/OzturkTHO10.pdf}{OzturkTHO10}~\cite{OzturkTHO10} & 8 & job, activity, scheduling, resource, setup-time, task, constraint programming, precedence, make-span, order, completion-time, machine, CP, cmax & SBSFMMAL & disjunctive &  & Ilog Scheduler, OPL, Ilog Solver, Cplex & robot &  &  &  & \ref{a:OzturkTHO10} & n/a\\
\index{OzturkTHO12}\rowlabel{b:OzturkTHO12}\href{../works/OzturkTHO12.pdf}{OzturkTHO12}~\cite{OzturkTHO12} & 6 & job, activity, scheduling, resource, cyclic scheduling, job-shop, setup-time, task, constraint programming, precedence, make-span, preemptive, order, completion-time, machine, preempt, CP, distributed &  & disjunctive, Element constraint, cycle, cumulative &  & OPL, Cplex &  &  &  & edge-finding & \ref{a:OzturkTHO12} & n/a\\
\index{OzturkTHO13}\rowlabel{b:OzturkTHO13}\href{../works/OzturkTHO13.pdf}{OzturkTHO13}~\cite{OzturkTHO13} & 36 & job, activity, scheduling, resource, cyclic scheduling, setup-time, constraint logic programming, task, constraint satisfaction, constraint programming, precedence, flow-shop, make-span, preemptive, order, CLP, completion-time, machine, preempt, CP, breakdown, cmax, CSP & SBSFMMAL & Disjunctive constraint, disjunctive, Channeling constraint, cycle, cumulative &  & CHIP, OPL, Ilog Solver, Cplex &  &  & real-world, real-life & genetic algorithm, large neighborhood search, column generation, edge-finding & \ref{a:OzturkTHO13} & \ref{c:OzturkTHO13}\\
\index{OzturkTHO15}\rowlabel{b:OzturkTHO15}\href{../works/OzturkTHO15.pdf}{OzturkTHO15}~\cite{OzturkTHO15} & 12 & job, activity, scheduling, resource, cyclic scheduling, setup-time, task, inventory, constraint satisfaction, constraint programming, precedence, make-span, preemptive, order, completion-time, machine, preempt, CP, breakdown, distributed & SBSFMMAL & disjunctive, circuit, cycle, cumulative &  & OPL, Cplex &  &  & real-life & large neighborhood search & \ref{a:OzturkTHO15} & n/a\\
\index{PandeyS21a}\rowlabel{b:PandeyS21a}\href{../works/PandeyS21a.pdf}{PandeyS21a}~\cite{PandeyS21a} & 29 & constraint logic programming, scheduling, unavailability, make-span, constraint satisfaction, distributed, activity, flow-shop, completion-time, CP, resource, energy efficiency, re-scheduling, job, constraint programming, CSP, precedence, task, single-machine scheduling, machine, order & PMSP, parallel machine, single machine & cumulative, Pulse constraint, endBeforeStart, alternative constraint &  & OPL, Cplex & semiconductor &  & benchmark & quadratic programming, column generation, mat heuristic & \ref{a:PandeyS21a} & n/a\\
\index{PapaB98}\rowlabel{b:PapaB98}\href{../works/PapaB98.pdf}{PapaB98}~\cite{PapaB98} & 25 & reactive scheduling, machine, activity, task, flow-shop, resource, constraint satisfaction, job, order, scheduling, distributed, CP, CLP, cmax, setup-time, job-shop, constraint programming, due-date, preempt, re-scheduling, make-span, completion-time, CSP, preemptive & PJSSP, Resource-constrained Project Scheduling Problem, JSSP & cumulative, table constraint, Disjunctive constraint, Cardinality constraint, disjunctive & C++ & Ilog Solver, CHIP, Claire & hoist &  & benchmark & edge-finder, energetic reasoning, edge-finding & \ref{a:PapaB98} & \ref{c:PapaB98}\\
\index{Pape94}\rowlabel{b:Pape94}\href{../works/Pape94.pdf}{Pape94}~\cite{Pape94} & 34 & stochastic, multi-agent, inventory, machine, job-shop, constraint programming, task, order, constraint satisfaction, constraint logic programming, CP, activity, due-date, distributed, resource, CLP, release-date, scheduling, precedence, re-scheduling, job &  & cumulative, disjunctive & C++, Prolog, Lisp &  &  &  &  &  & \ref{a:Pape94} & n/a\\
\index{PengLC14}\rowlabel{b:PengLC14}\href{../works/PengLC14.pdf}{PengLC14}~\cite{PengLC14} & 7 & stochastic, job-shop, CP, earliness, constraint optimization, preempt, COP, CSP, setup-time, make-span, machine, job, distributed, resource, precedence, sequence dependent setup, due-date, preemptive, order, tardiness, scheduling, constraint programming, completion-time, constraint satisfaction, task & single machine &  &  &  &  &  & real-life, benchmark & genetic algorithm & \ref{a:PengLC14} & n/a\\
\index{PenzDN23}\rowlabel{b:PenzDN23}\href{../works/PenzDN23.pdf}{PenzDN23}~\cite{PenzDN23} & 13 & periodic, job-shop, CP, single-machine scheduling, earliness, preempt, COP, setup-time, activity, sustainability, make-span, machine, flow-time, job, resource, one-machine scheduling, release-date, unavailability, breakdown, preemptive, order, tardiness, scheduling, completion-time, stochastic & parallel machine, single machine &  &  & Cplex & maintenance scheduling, semiconductor & semiconductor industry &  & memetic algorithm, meta heuristic, ant colony, simulated annealing & \ref{a:PenzDN23} & n/a\\
\index{PesantGPR99}\rowlabel{b:PesantGPR99}\href{../works/PesantGPR99.pdf}{PesantGPR99}~\cite{PesantGPR99} & 11 & job-shop, transportation, constraint logic programming, task, constraint programming, make-span, order, CLP, CP, scheduling, re-scheduling, distributed, inventory, job, resource &  &  & Prolog, C++ & Ilog Solver, ECLiPSe &  &  & real-life, benchmark & time-tabling & \ref{a:PesantGPR99} & n/a\\
\index{PoderBS04}\rowlabel{b:PoderBS04}\href{../works/PoderBS04.pdf}{PoderBS04}~\cite{PoderBS04} & 16 & order, activity, release-date, resource, constraint satisfaction, preempt, scheduling, precedence, task, producer/consumer, preemptive, due-date, CP, machine, constraint programming & RCPSP & cumulative & Prolog & CHIP &  & chemical industry &  &  & \ref{a:PoderBS04} & n/a\\
\index{PohlAK22}\rowlabel{b:PohlAK22}\href{../works/PohlAK22.pdf}{PohlAK22}~\cite{PohlAK22} & 16 & job, resource, lateness, release-date, transportation, precedence, sequence dependent setup, re-scheduling, tardiness, scheduling, constraint programming, completion-time, stochastic, activity, CP, single-machine scheduling, earliness, inventory, setup-time, order, machine & single machine, SCC & cumulative, noOverlap & Python & Cplex, Gurobi & airport, aircraft &  & benchmark, real-world & simulated annealing, large neighborhood search, column generation & \ref{a:PohlAK22} & n/a\\
\index{Polo-MejiaALB20}\rowlabel{b:Polo-MejiaALB20}\href{../works/Polo-MejiaALB20.pdf}{Polo-MejiaALB20}~\cite{Polo-MejiaALB20} & 18 & setup-time, cmax, precedence, due-date, activity, machine, tardiness, CP, preemptive, make-span, scheduling, completion-time, periodic, multi-objective, resource, preempt, earliness, Benders Decomposition, task, job, constraint programming, order, release-date & RCPSP, Resource-constrained Project Scheduling Problem & alternative constraint, Calendar constraint, endBeforeStart, alwaysIn, Disjunctive constraint, cumulative, noOverlap, disjunctive & C++ & Cplex, CPO &  &  & github, Roadef & mat heuristic, meta heuristic, particle swarm & \ref{a:Polo-MejiaALB20} & \ref{c:Polo-MejiaALB20}\\
\index{PourDERB18}\rowlabel{b:PourDERB18}\href{../works/PourDERB18.pdf}{PourDERB18}~\cite{PourDERB18} & 12 & multi-objective, COP, constraint satisfaction, CP, stochastic, order, transportation, constraint optimization, job, constraint programming, CSP, scheduling, task, machine &  &  &  & OR-Tools, Cplex & maintenance scheduling, railway, crew-scheduling &  & real-world, generated instance, real-life, benchmark & genetic algorithm, meta heuristic, ant colony & \ref{a:PourDERB18} & n/a\\
\index{PrataAN23}\rowlabel{b:PrataAN23}\href{../works/PrataAN23.pdf}{PrataAN23}~\cite{PrataAN23} & 17 & order, multi-objective, activity, setup-time, release-date, no-wait, single-machine scheduling, scheduling, Logic-Based Benders Decomposition, make-span, task, bi-objective, order scheduling, job, sequence dependent setup, due-date, batch process, preempt, flow-shop, precedence, tardiness, flow-time, earliness, preemptive, completion-time, energy efficiency, online scheduling, CP, machine, lateness, re-scheduling, stochastic, inventory, distributed, constraint programming, job-shop, resource, open-shop, Benders Decomposition & single machine, Open Shop Scheduling Problem, parallel machine & circuit, cumulative &  & CHIP & dairy, robot, energy-price, aircraft & manufacturing industry & real-life, benchmark, real-world & mat heuristic, memetic algorithm, meta heuristic, machine learning, genetic algorithm, reinforcement learning, time-tabling, large neighborhood search, particle swarm & \ref{a:PrataAN23} & \ref{c:PrataAN23}\\
\index{QinDCS20}\rowlabel{b:QinDCS20}\href{../works/QinDCS20.pdf}{QinDCS20}~\cite{QinDCS20} & 18 & order, tardiness, scheduling, completion-time, machine, job, task, transportation, cmax, constraint programming, Benders Decomposition, Logic-Based Benders Decomposition, CSP, setup-time, activity, CP, make-span, explanation, resource, precedence & parallel machine & noOverlap, endBeforeStart, cycle &  & Cplex, OPL & shipping line, container terminal, yard crane & maritime industry, shipping industry & real-life, benchmark & meta heuristic, GRASP, particle swarm & \ref{a:QinDCS20} & n/a\\
\index{QinDS16}\rowlabel{b:QinDS16}\href{../works/QinDS16.pdf}{QinDS16}~\cite{QinDS16} & 19 & job, order, scheduling, CP, Logic-Based Benders Decomposition, constraint programming, explanation, stochastic, completion-time, transportation, Benders Decomposition, task, periodic, Pareto, resource, make-span, machine, activity, COP, stock level, constraint satisfaction, bi-objective & parallel machine & alwaysIn, noOverlap &  & OPL, SCIP, Cplex & sports scheduling, container terminal, crew-scheduling &  &  & simulated annealing, meta heuristic, column generation & \ref{a:QinDS16} & n/a\\
\index{QinWSLS21}\rowlabel{b:QinWSLS21}\href{../works/QinWSLS21.pdf}{QinWSLS21}~\cite{QinWSLS21} & 12 & job-shop, order, completion-time, multi-objective, single-machine scheduling, order scheduling, job, preempt, flow-shop, scheduling, make-span, two-stage scheduling, tardiness, preemptive, batch process, cmax, CP, machine, lateness & single machine &  & C++ & OPL, Cplex & tournament, agriculture, semiconductor & semiconductor industry &  & particle swarm, ant colony, memetic algorithm, meta heuristic, machine learning, genetic algorithm & \ref{a:QinWSLS21} & n/a\\
\index{RasmussenT09}\rowlabel{b:RasmussenT09}\href{../works/RasmussenT09.pdf}{RasmussenT09}~\cite{RasmussenT09} & 15 & scheduling, distributed, CP, Benders Decomposition, constraint logic programming, constraint programming, order &  &  &  & Ilog Solver, Cplex, OPL & sports scheduling, break minimization problem, travelling tournament problem, tournament, round-robin &  & benchmark & column generation, simulated annealing, time-tabling, meta heuristic & \ref{a:RasmussenT09} & n/a\\
\index{ReddyFIBKAJ11}\rowlabel{b:ReddyFIBKAJ11}\href{../works/ReddyFIBKAJ11.pdf}{ReddyFIBKAJ11}~\cite{ReddyFIBKAJ11} & 24 & distributed, task, CP, activity, CSP, scheduling, constraint optimization, precedence, machine, resource, order, tardiness, constraint satisfaction &  & table constraint, bin-packing, cycle &  &  & satellite, deep space, robot, telescope &  &  &  & \ref{a:ReddyFIBKAJ11} & n/a\\
\index{RiiseML16}\rowlabel{b:RiiseML16}\href{../works/RiiseML16.pdf}{RiiseML16}~\cite{RiiseML16} & 10 & stochastic, constraint programming, Benders Decomposition, CP, Logic-Based Benders Decomposition, job-shop, resource, job, order, activity, preempt, setup-time, scheduling, make-span, task, due-date & Resource-constrained Project Scheduling Problem, RCPSP & bin-packing &  & Cplex & nurse, sports scheduling, operating room, surgery, patient, medical &  & real-life, real-world & meta heuristic, column generation, genetic algorithm & \ref{a:RiiseML16} & n/a\\
\index{RodosekWH99}\rowlabel{b:RodosekWH99}\href{../works/RodosekWH99.pdf}{RodosekWH99}~\cite{RodosekWH99} & 25 & constraint programming, CLP, CP, constraint logic programming, task, constraint satisfaction, order, machine, scheduling, resource &  & disjunctive, cycle, Disjunctive constraint & Prolog & Cplex, ECLiPSe & hoist, pipeline, crew-scheduling &  & benchmark &  & \ref{a:RodosekWH99} & n/a\\
\index{Rodriguez07}\rowlabel{b:Rodriguez07}\href{../works/Rodriguez07.pdf}{Rodriguez07}~\cite{Rodriguez07} & 15 & blocking constraint, job, scheduling, explanation, resource, due-date, job-shop, transportation, task, activity, constraint programming, precedence, preemptive, preempt, CP, order &  & Disjunctive constraint, disjunctive, circuit, Blocking constraint &  & Ilog Solver, Z3, Ilog Scheduler, Cplex & railway, satellite, train schedule &  & real-life & GRASP, meta heuristic & \ref{a:Rodriguez07} & n/a\\
\index{RodriguezDG02}\rowlabel{b:RodriguezDG02}\href{../works/RodriguezDG02.pdf}{RodriguezDG02}~\cite{RodriguezDG02} & 10 & constraint programming, resource, order, completion-time, CP, activity, scheduling, transportation &  & circuit, disjunctive &  &  & railway, train schedule &  &  & edge-finding & \ref{a:RodriguezDG02} & n/a\\
\index{RoePS05}\rowlabel{b:RoePS05}\href{../works/RoePS05.pdf}{RoePS05}~\cite{RoePS05} & 15 & setup-time, job-shop, constraint programming, due-date, continuous-process, re-scheduling, constraint logic programming, tardiness, make-span, lateness, machine, inventory, task, flow-shop, periodic, resource, constraint satisfaction, job, order, batch process, scheduling, distributed, precedence, CLP &  & disjunctive, Balance constraint, cumulative, Disjunctive constraint & Prolog & ECLiPSe, CHIP & maintenance scheduling &  &  & time-tabling, MINLP & \ref{a:RoePS05} & n/a\\
\index{RoshanaeiBAUB20}\rowlabel{b:RoshanaeiBAUB20}\href{../works/RoshanaeiBAUB20.pdf}{RoshanaeiBAUB20}~\cite{RoshanaeiBAUB20} & 19 & activity, machine, stochastic, sequence dependent setup, CP, scheduling, Logic-Based Benders Decomposition, resource, order, Benders Decomposition, constraint logic programming, job, job-shop, setup-time, bi-objective, constraint programming, distributed, re-scheduling & parallel machine & bin-packing, noOverlap, disjunctive & C++ & Cplex & operating room, nurse, patient, surgery &  & real-world, benchmark, generated instance & genetic algorithm, column generation, meta heuristic & \ref{a:RoshanaeiBAUB20} & n/a\\
\index{RoshanaeiLAU17}\rowlabel{b:RoshanaeiLAU17}\href{../works/RoshanaeiLAU17.pdf}{RoshanaeiLAU17}~\cite{RoshanaeiLAU17} & 17 & Benders Decomposition, constraint logic programming, stochastic, Logic-Based Benders Decomposition, breakdown, resource, task, job-shop, tardiness, sequence dependent setup, transportation, scheduling, order, make-span, release-date, setup-time, distributed, constraint programming, machine, CP, job, re-scheduling & single machine, parallel machine & bin-packing &  & Cplex, Gurobi & operating room, patient, medical, surgery, nurse &  & real-world & meta heuristic, column generation & \ref{a:RoshanaeiLAU17} & n/a\\
\index{RoshanaeiN21}\rowlabel{b:RoshanaeiN21}\href{../works/RoshanaeiN21.pdf}{RoshanaeiN21}~\cite{RoshanaeiN21} & 14 & due-date, distributed, job, online scheduling, re-scheduling, CP, Benders Decomposition, constraint logic programming, stochastic, setup-time, constraint programming, order, Logic-Based Benders Decomposition, job-shop, resource, completion-time, scheduling, machine, explanation, inventory & parallel machine, OSSP & cumulative, noOverlap &  & Cplex, CPO & automotive, operating room, patient, surgery &  & benchmark & genetic algorithm, meta heuristic, column generation & \ref{a:RoshanaeiN21} & n/a\\
\index{RuggieroBBMA09}\rowlabel{b:RuggieroBBMA09}\href{../works/RuggieroBBMA09.pdf}{RuggieroBBMA09}~\cite{RuggieroBBMA09} & 14 & Logic-Based Benders Decomposition, CP, resource, energy efficiency, CSP, constraint satisfaction, precedence, task, Pareto, activity, distributed, machine, scheduling, order, Benders Decomposition, preempt, setup-time, constraint programming &  & circuit, cumulative, cycle &  & Ilog Solver, Cplex, Ilog Scheduler & pipeline, satellite &  & instance generator, real-life & genetic algorithm & \ref{a:RuggieroBBMA09} & n/a\\
\index{SacramentoSP20}\rowlabel{b:SacramentoSP20}\href{../works/SacramentoSP20.pdf}{SacramentoSP20}~\cite{SacramentoSP20} & 33 & CSP, precedence, task, open-shop, activity, distributed, machine, flow-shop, multi-objective, transportation, scheduling, order, make-span, preempt, stochastic, constraint programming, CP, completion-time, resource, job, preemptive, job-shop & parallel machine, Resource-constrained Project Scheduling Problem, Open Shop Scheduling Problem & alternative constraint, endBeforeStart, noOverlap, cumulative, disjunctive & Java & Cplex, CPO & container terminal & shipping industry, maritime industry & benchmark, real-life, zenodo, real-world & particle swarm, simulated annealing, meta heuristic, large neighborhood search, mat heuristic, genetic algorithm, machine learning & \ref{a:SacramentoSP20} & \ref{c:SacramentoSP20}\\
\index{SadehF96}\rowlabel{b:SadehF96}\href{../works/SadehF96.pdf}{SadehF96}~\cite{SadehF96} & 41 & inventory, CSP, activity, CP, make-span, resource, precedence, due-date, re-scheduling, order, tardiness, scheduling, constraint satisfaction, machine, job, task, release-date, multi-agent, stochastic, job-shop &  & Disjunctive constraint, cycle, circuit, disjunctive & Lisp, C++ &  & robot, aircraft &  & benchmark &  & \ref{a:SadehF96} & n/a\\
\index{SadykovW06}\rowlabel{b:SadykovW06}\href{../works/SadykovW06.pdf}{SadykovW06}~\cite{SadykovW06} & 9 & due-date, constraint programming, completion-time, CP, job, release-date, scheduling, CLP, machine, one-machine scheduling, lateness & parallel machine, single machine & disjunctive, Disjunctive constraint &  & CHIP & robot &  & generated instance & Lagrangian relaxation, column generation & \ref{a:SadykovW06} & n/a\\
\index{SakkoutW00}\rowlabel{b:SakkoutW00}\href{../works/SakkoutW00.pdf}{SakkoutW00}~\cite{SakkoutW00} & 30 & distributed, preemptive, job-shop, activity, precedence, CP, single-machine scheduling, order, transportation, re-scheduling, reactive scheduling, job, constraint programming, CSP, scheduling, task, machine, preempt, constraint satisfaction, resource & single machine, KRFP & bin-packing, disjunctive, Disjunctive constraint, Arithmetic constraint &  & Cplex, CHIP & emergency service, aircraft &  & benchmark, real-world & edge-finder, edge-finding, genetic algorithm, simulated annealing & \ref{a:SakkoutW00} & \ref{c:SakkoutW00}\\
\index{Salido10}\rowlabel{b:Salido10}\href{../works/Salido10.pdf}{Salido10}~\cite{Salido10} & 4 & COP, CP, CSP, activity, constraint programming, constraint satisfaction, distributed, flow-time, job, job-shop, machine, multi-agent, multi-objective, order, preempt, preemptive, resource, scheduling, tardiness, task, transportation &  & cycle &  & OPL & medical, patient, robot & software industry & benchmark, real-life, real-world & genetic algorithm, time-tabling & \ref{a:Salido10} & n/a\\
\index{SchausHMCMD11}\rowlabel{b:SchausHMCMD11}\href{../works/SchausHMCMD11.pdf}{SchausHMCMD11}~\cite{SchausHMCMD11} & 23 & stochastic, periodic, task, CP, CSP, constraint optimization, constraint programming, constraint logic programming, order & SCC & Element constraint, Cardinality constraint, bin-packing, GCC constraint &  &  & steel mill & steel industry & CSPlib, generated instance, benchmark & meta heuristic, large neighborhood search & \ref{a:SchausHMCMD11} & \ref{c:SchausHMCMD11}\\
\index{SchildW00}\rowlabel{b:SchildW00}\href{../works/SchildW00.pdf}{SchildW00}~\cite{SchildW00} & 23 & periodic, scheduling, task, constraint logic programming, job, flow-shop, machine, explanation, precedence, CLP, completion-time, distributed, constraint programming, job-shop, resource, order, constraint satisfaction & single machine & disjunctive, Disjunctive constraint, bin-packing, cycle, Reified constraint &  & Ilog Solver & automotive & automotive industry, aerospace industry &  & edge-finding, time-tabling & \ref{a:SchildW00} & \ref{c:SchildW00}\\
\index{SchnellH15}\rowlabel{b:SchnellH15}\href{../works/SchnellH15.pdf}{SchnellH15}~\cite{SchnellH15} & 21 & preempt, resource, preemptive, activity, explanation, precedence, CP, job, constraint programming, scheduling, machine, make-span, net present value, cmax & psplib, RCPSP, Resource-constrained Project Scheduling Problem & cycle, cumulative &  & SCIP & automotive & IT industry & real-life, benchmark, supplementary material & simulated annealing, lazy clause generation, meta heuristic, GRASP & \ref{a:SchnellH15} & \ref{c:SchnellH15}\\
\index{SchnellH17}\rowlabel{b:SchnellH17}\href{../works/SchnellH17.pdf}{SchnellH17}~\cite{SchnellH17} & 11 & preempt, resource, preemptive, explanation, precedence, CP, order, job, constraint programming, scheduling, machine, make-span, net present value & psplib, RCPSP, Resource-constrained Project Scheduling Problem & cumulative & Java & SCIP &  &  & benchmark, supplementary material & genetic algorithm, lazy clause generation, meta heuristic, GRASP & \ref{a:SchnellH17} & \ref{c:SchnellH17}\\
\index{SchuttFSW11}\rowlabel{b:SchuttFSW11}\href{../works/SchuttFSW11.pdf}{SchuttFSW11}~\cite{SchuttFSW11} & 33 & scheduling, explanation, resource, CSP, task, activity, constraint programming, precedence, make-span, preemptive, completion-time, machine, preempt, CP, periodic, constraint satisfaction, open-shop, order & psplib, Resource-constrained Project Scheduling Problem, RCPSP & Disjunctive constraint, span constraint, disjunctive, circuit, cumulative &  & ECLiPSe, CHIP, Ilog Scheduler, SICStus &  &  & benchmark, real-world & not-last, lazy clause generation, not-first, edge-finding, edge-finder & \ref{a:SchuttFSW11} & \ref{c:SchuttFSW11}\\
\index{SchuttFSW13}\rowlabel{b:SchuttFSW13}\href{../works/SchuttFSW13.pdf}{SchuttFSW13}~\cite{SchuttFSW13} & 17 & scheduling, explanation, resource, task, activity, constraint programming, precedence, release-date, preemptive, CLP, machine, setup-time, preempt, CP, cmax, constraint satisfaction, order & psplib, Resource-constrained Project Scheduling Problem, RCPSP, SCC & disjunctive, cycle, cumulative, Reified constraint & C++ & CHIP &  &  & supplementary material, benchmark & genetic algorithm, lazy clause generation, meta heuristic & \ref{a:SchuttFSW13} & \ref{c:SchuttFSW13}\\
\index{ShaikhK23}\rowlabel{b:ShaikhK23}\href{../works/ShaikhK23.pdf}{ShaikhK23}~\cite{ShaikhK23} & 12 & job, unavailability, machine, constraint programming, CLP, CP, re-scheduling, distributed, job-shop, resource, open-shop, order, activity, constraint satisfaction, scheduling, task &  &  &  &  & medical, drone &  & benchmark, real-world & genetic algorithm, time-tabling, meta heuristic, machine learning & \ref{a:ShaikhK23} & \ref{c:ShaikhK23}\\
\index{ShinBBHO18}\rowlabel{b:ShinBBHO18}\href{../works/ShinBBHO18.pdf}{ShinBBHO18}~\cite{ShinBBHO18} & 16 & order, transportation, job, scheduling, task, machine, preempt, resource, activity, stochastic, inventory &  &  &  &  & physician, nurse, patient, medical &  & github, real-world &  & \ref{a:ShinBBHO18} & \ref{c:ShinBBHO18}\\
\index{Siala15}\rowlabel{b:Siala15}\href{../works/Siala15.pdf}{Siala15}~\cite{Siala15} & 2 & sequence dependent setup, machine, activity, setup-time, job, open-shop, order, scheduling, CP, precedence, cmax, job-shop, constraint programming, explanation, due-date, earliness, task, tardiness, resource, make-span & single machine, OSP, RCPSP, TMS & AmongSeq constraint, circuit, alldifferent, Balance constraint, cumulative, table constraint, GCC constraint, AtMostSeqCard, Reified constraint, Regular constraint, Among constraint, Atmost constraint, Disjunctive constraint, Cardinality constraint, cycle, MultiAtMostSeqCard, disjunctive, CardPath, AtMostSeq &  & Ilog Solver, CHIP, Claire, OPL, Mistral & rectangle-packing, automotive &  & github, Roadef, CSPlib, real-world, benchmark, random instance & edge-finding, GRASP, time-tabling & \ref{a:Siala15} & \ref{c:Siala15}\\
\index{SimoninAHL15}\rowlabel{b:SimoninAHL15}\href{../works/SimoninAHL15.pdf}{SimoninAHL15}~\cite{SimoninAHL15} & 23 & CP, resource, task, constraint programming, precedence, periodic, activity, scheduling, transportation, make-span, preempt, order, inventory &  & disjunctive, span constraint, cycle, cumulative &  & CHIP & satellite, pipeline, earth observation, robot &  &  & sweep & \ref{a:SimoninAHL15} & \ref{c:SimoninAHL15}\\
\index{Simonis07}\rowlabel{b:Simonis07}\href{../works/Simonis07.pdf}{Simonis07}~\cite{Simonis07} & 30 & CLP, CSP, make to order, CP, re-scheduling, constraint programming, job-shop, resource, transportation, order, activity, constraint satisfaction, setup-time, release-date, periodic, scheduling, task, producer/consumer, bill of material, constraint logic programming, job, sequence dependent setup, due-date, batch process, machine &  & GCC constraint, Atmost constraint, diffn, Cardinality constraint, Cumulatives constraint, disjunctive, bin-packing, Among constraint, cumulative, alldifferent, cycle & Prolog & OPL, Ilog Scheduler, CHIP & aircraft, airport, patient, medical, round-robin, business process, nurse &  &  & sweep, bi-partite matching, meta heuristic, time-tabling & \ref{a:Simonis07} & \ref{c:Simonis07}\\
\index{SimonisCK00}\rowlabel{b:SimonisCK00}\href{../works/SimonisCK00.pdf}{SimonisCK00}~\cite{SimonisCK00} & 7 & activity, constraint programming, CP, task, stock level, order, machine, producer/consumer, scheduling, resource, transportation &  & disjunctive, bin-packing, circuit, cumulative, diffn, cycle & C++, Prolog & CHIP & business process, crew-scheduling, aircraft & food industry &  &  & \ref{a:SimonisCK00} & n/a\\
\index{SmithBHW96}\rowlabel{b:SmithBHW96}\href{../works/SmithBHW96.pdf}{SmithBHW96}~\cite{SmithBHW96} & 20 & constraint satisfaction, resource, order, constraint programming, CSP, task, CLP, constraint logic programming &  &  & C++ & OPL, Ilog Solver &  &  & real-life &  & \ref{a:SmithBHW96} & n/a\\
\index{SourdN00}\rowlabel{b:SourdN00}\href{../works/SourdN00.pdf}{SourdN00}~\cite{SourdN00} & 12 & CP, make-span, resource, precedence, cmax, preemptive, order, scheduling, completion-time, constraint satisfaction, machine, job, open-shop, release-date, job-shop, flow-shop, preempt, setup-time & single machine, JSSP & cumulative, disjunctive, Disjunctive constraint &  & Ilog Scheduler & robot &  & real-life, benchmark & not-first, edge-finding, genetic algorithm & \ref{a:SourdN00} & n/a\\
\index{SubulanC22}\rowlabel{b:SubulanC22}\href{../works/SubulanC22.pdf}{SubulanC22}~\cite{SubulanC22} & 38 & tardiness, preempt, resource, preemptive, due-date, activity, completion-time, precedence, CP, stochastic, inventory, order, BOM, breakdown, transportation, constraint programming, scheduling, task, machine, make-span, multi-objective & RCPSP, Resource-constrained Project Scheduling Problem & endBeforeStart, cumulative &  & Cplex, OPL & business process, offshore &  & real-world, real-life, benchmark & mat heuristic, genetic algorithm, meta heuristic, ant colony, particle swarm & \ref{a:SubulanC22} & n/a\\
\index{SunTB19}\rowlabel{b:SunTB19}\href{../works/SunTB19.pdf}{SunTB19}~\cite{SunTB19} & 12 & preempt, constraint satisfaction, job, order, scheduling, CP, precedence, Logic-Based Benders Decomposition, explanation, tardiness, completion-time, transportation, CSP, Benders Decomposition, task, resource, make-span &  &  &  & Cplex & container terminal, yard crane & maritime industry & generated instance, instance generator, benchmark, github, real-life & simulated annealing, meta heuristic, genetic algorithm & \ref{a:SunTB19} & \ref{c:SunTB19}\\
\index{SureshMOK06}\rowlabel{b:SureshMOK06}\href{../works/SureshMOK06.pdf}{SureshMOK06}~\cite{SureshMOK06} & 19 & task, stochastic, order, machine, CP, scheduling, buffer-capacity, distributed, job &  & cycle, cumulative &  & Z3 & tournament &  &  & genetic algorithm, machine learning & \ref{a:SureshMOK06} & n/a\\
\index{TanZWGQ19}\rowlabel{b:TanZWGQ19}\href{../works/TanZWGQ19.pdf}{TanZWGQ19}~\cite{TanZWGQ19} & 10 & resource, transportation, Benders Decomposition, order, completion-time, multi-objective, setup-time, periodic, task, job, flow-shop, Pareto, scheduling, precedence, make-span, two-stage scheduling, due-date, batch process, multi-agent, CP, machine, constraint programming & SCC, parallel machine & noOverlap & C++ & Cplex & robot, automotive &  & real-world, generated instance, supplementary material & meta heuristic, particle swarm & \ref{a:TanZWGQ19} & \ref{c:TanZWGQ19}\\
\index{TangLWSK18}\rowlabel{b:TangLWSK18}\href{../works/TangLWSK18.pdf}{TangLWSK18}~\cite{TangLWSK18} & 28 & preempt, constraint satisfaction, resource, preemptive, activity, job, constraint programming, CP, stochastic, order, multi-objective, transportation, re-scheduling, CSP, scheduling, task & RCPSP & cycle, circuit & C  & Cplex, OPL & pipeline, crew-scheduling, railway &  &  & meta heuristic, genetic algorithm, neural network, particle swarm & \ref{a:TangLWSK18} & n/a\\
\index{TerekhovDOB12}\rowlabel{b:TerekhovDOB12}\href{../works/TerekhovDOB12.pdf}{TerekhovDOB12}~\cite{TerekhovDOB12} & 15 & order scheduling, constraint programming, cmax, resource, periodic, job, Benders Decomposition, completion-time, tardiness, Logic-Based Benders Decomposition, flow-shop, earliness, open-shop, due-date, distributed, preempt, make-span, precedence, single-stage scheduling, inventory, CP, activity, job-shop, scheduling, release-date, machine, lateness, order, constraint satisfaction, single-machine scheduling & parallel machine, RCPSP, Resource-constrained Project Scheduling Problem, single machine & cumulative, Balance constraint, alldifferent, disjunctive & C++ & Ilog Scheduler, Ilog Solver, Cplex & robot &  & real-life & meta heuristic, genetic algorithm & \ref{a:TerekhovDOB12} & n/a\\
\index{TerekhovTDB14}\rowlabel{b:TerekhovTDB14}\href{../works/TerekhovTDB14.pdf}{TerekhovTDB14}~\cite{TerekhovTDB14} & 38 & flow-shop, distributed, no preempt, preempt, make-span, task, preemptive, inventory, CP, activity, re-scheduling, job-shop, scheduling, flow-time, release-date, machine, order, constraint programming, stochastic, cmax, resource, periodic, job, completion-time, tardiness, buffer-capacity, online scheduling & single machine, parallel machine &  &  & Cplex, Ilog Scheduler & round-robin, semiconductor, robot &  & real-world & meta heuristic, genetic algorithm & \ref{a:TerekhovTDB14} & n/a\\
\index{ThiruvadyWGS14}\rowlabel{b:ThiruvadyWGS14}\href{../works/ThiruvadyWGS14.pdf}{ThiruvadyWGS14}~\cite{ThiruvadyWGS14} & 34 & explanation, breakdown, precedence, task, make-span, activity, tardiness, distributed, constraint programming, machine, job, scheduling, order, net present value, stochastic, CP, completion-time, resource & single machine, Resource-constrained Project Scheduling Problem, psplib & cumulative &  &  &  & mining industry & benchmark & meta heuristic, Lagrangian relaxation, genetic algorithm, ant colony, simulated annealing, machine learning & \ref{a:ThiruvadyWGS14} & n/a\\
\index{Timpe02}\rowlabel{b:Timpe02}\href{../works/Timpe02.pdf}{Timpe02}~\cite{Timpe02} & 18 & breakdown, inventory, task, resource, make-span, order, machine, activity, stock level, setup-time, job, scheduling, constraint logic programming, CP, producer/consumer, constraint programming, due-date &  & Balance constraint, cumulative, cycle, diffn, disjunctive & C++ & CHIP, Cplex &  & chemical industry, process industry &  &  & \ref{a:Timpe02} & n/a\\
\index{TopalogluO11}\rowlabel{b:TopalogluO11}\href{../works/TopalogluO11.pdf}{TopalogluO11}~\cite{TopalogluO11} & 10 & scheduling, re-scheduling, CP, multi-objective, preempt, order, distributed, constraint logic programming, task, constraint programming, preemptive, transportation, constraint satisfaction &  &  &  & Cplex, OPL, Ilog Solver & nurse, physician, emergency service, patient, surgery, medical &  & real-life & column generation, time-tabling & \ref{a:TopalogluO11} & n/a\\
\index{TorresL00}\rowlabel{b:TorresL00}\href{../works/TorresL00.pdf}{TorresL00}~\cite{TorresL00} & 12 & precedence, constraint programming, CSP, CP, job-shop, resource, order, constraint satisfaction, preempt, release-date, scheduling, make-span, task, job, preemptive, machine & single machine, JSSP & disjunctive, cumulative, cycle & C++ &  & robot &  & benchmark & not-last, energetic reasoning, not-first & \ref{a:TorresL00} & n/a\\
\index{TranAB16}\rowlabel{b:TranAB16}\href{../works/TranAB16.pdf}{TranAB16}~\cite{TranAB16} & 13 & sequence dependent setup, due-date, order, tardiness, scheduling, completion-time, machine, job, release-date, cmax, constraint programming, Benders Decomposition, Logic-Based Benders Decomposition, stochastic, setup-time, CP, make-span, explanation, single-machine scheduling, constraint logic programming, resource, precedence & PMSP, single machine, parallel machine & cycle, circuit &  & SCIP, Gurobi, Cplex & aircraft &  & benchmark & simulated annealing, meta heuristic, ant colony, genetic algorithm, column generation & \ref{a:TranAB16} & n/a\\
\index{TranPZLDB18}\rowlabel{b:TranPZLDB18}\href{../works/TranPZLDB18.pdf}{TranPZLDB18}~\cite{TranPZLDB18} & 17 & machine, preempt, periodic, scheduling, resource, distributed, online scheduling, job, energy efficiency, make-span, preemptive, completion-time, CP, task, stochastic, re-scheduling, order & single machine & bin-packing & C++ & Cplex & high performance computing &  & generated instance, benchmark & machine learning & \ref{a:TranPZLDB18} & n/a\\
\index{TranVNB17}\rowlabel{b:TranVNB17}\href{../works/TranVNB17.pdf}{TranVNB17}~\cite{TranVNB17} & 68 & Benders Decomposition, order, multi-objective, activity, Logic-Based Benders Decomposition, resource, constraint logic programming, job, scheduling, precedence, task, unavailability, multi-agent, CP, machine, re-scheduling, constraint programming, transportation &  & alternative constraint, Cardinality constraint, noOverlap, cumulative &  & OPL, MiniZinc, Cplex & robot, medical, satellite &  & real-world &  & \ref{a:TranVNB17} & n/a\\
\index{TrojetHL11}\rowlabel{b:TrojetHL11}\href{../works/TrojetHL11.pdf}{TrojetHL11}~\cite{TrojetHL11} & 7 & job-shop, activity, job, constraint programming, completion-time, CSP, precedence, CP, order, scheduling, task, machine, make-span, constraint satisfaction, distributed, due-date, constraint logic programming, resource & RCPSP & cumulative, disjunctive, diffn, cycle, alldifferent & Prolog & CHIP, SICStus & robot &  & real-world &  & \ref{a:TrojetHL11} & n/a\\
\index{Tsang03}\rowlabel{b:Tsang03}\href{../works/Tsang03.pdf}{Tsang03}~\cite{Tsang03} & 2 & resource, constraint programming, explanation, constraint satisfaction, scheduling &  &  &  &  &  &  & real-life & time-tabling & \ref{a:Tsang03} & n/a\\
\index{UnsalO13}\rowlabel{b:UnsalO13}\href{../works/UnsalO13.pdf}{UnsalO13}~\cite{UnsalO13} & 15 & constraint programming, CP, preempt, CSP, activity, make-span, machine, resource, unavailability, precedence, due-date, preemptive, order, scheduling, completion-time, job, task, transportation & parallel machine & alldifferent, cumulative, disjunctive &  & Ilog Scheduler, Cplex & yard crane, hoist, container terminal &  & real-life, instance generator, real-world, benchmark & GRASP, genetic algorithm, meta heuristic, large neighborhood search & \ref{a:UnsalO13} & n/a\\
\index{UnsalO19}\rowlabel{b:UnsalO19}\href{../works/UnsalO19.pdf}{UnsalO19}~\cite{UnsalO19} & 19 & constraint programming, CP, Logic-Based Benders Decomposition, preempt, COP, activity, explanation, machine, resource, release-date, re-scheduling, preemptive, order, BOM, scheduling, Benders Decomposition, completion-time, stock level, task, transportation & parallel machine & endBeforeStart, disjunctive &  & OPL, Cplex & container terminal &  & real-world, random instance & genetic algorithm, column generation & \ref{a:UnsalO19} & n/a\\
\index{VilimBC05}\rowlabel{b:VilimBC05}\href{../works/VilimBC05.pdf}{VilimBC05}~\cite{VilimBC05} & 23 & setup-time, scheduling, make-span, task, job, sequence dependent setup, batch process, machine, precedence, CLP, completion-time, CP, distributed, constraint programming, job-shop, resource, open-shop, order, activity &  & disjunctive, cumulative, cycle &  &  &  &  & benchmark, real-life & edge-finding, not-first, not-last, sweep & \ref{a:VilimBC05} & \ref{c:VilimBC05}\\
\index{VlkHT21}\rowlabel{b:VlkHT21}\href{../works/VlkHT21.pdf}{VlkHT21}~\cite{VlkHT21} & 14 & scheduling, CP, Logic-Based Benders Decomposition, constraint programming, explanation, tardiness, stochastic, due-date, completion-time, Benders Decomposition, periodic, online scheduling, resource, no-wait, distributed, precedence, bi-objective, order & PMSP & noOverlap, alternative constraint &  & OPL, Cplex, Gurobi, Z3 & automotive, robot &  & benchmark, industrial partner, random instance, github & GRASP & \ref{a:VlkHT21} & \ref{c:VlkHT21}\\
\index{Wallace96}\rowlabel{b:Wallace96}\href{../works/Wallace96.pdf}{Wallace96}~\cite{Wallace96} & 30 & activity, constraint satisfaction, distributed, task, resource, constraint logic programming, job, explanation, scheduling, reactive scheduling, CSP, multi-agent, CP, machine, stochastic, constraint programming, job-shop, transportation, CLP, Benders Decomposition, order &  & circuit, disjunctive, cycle & Lisp, Prolog & CHIP, ECLiPSe, Ilog Solver, OPL & robot, train schedule, airport, railway, telescope, automotive, aircraft & automotive industry, process industry &  & column generation, genetic algorithm, neural network, Lagrangian relaxation, simulated annealing, time-tabling & \ref{a:Wallace96} & \ref{c:Wallace96}\\
\index{WallaceY20}\rowlabel{b:WallaceY20}\href{../works/WallaceY20.pdf}{WallaceY20}~\cite{WallaceY20} & 19 & job-shop, flow-shop, explanation, CP, resource, transportation, bi-objective, job, constraint programming, Benders Decomposition, cyclic scheduling, task, machine, order, scheduling, Logic-Based Benders Decomposition & CHSP & circuit, Disjunctive constraint, cumulative, disjunctive, cycle &  & Gecode, MiniZinc, Chuffed, OPL, Gurobi, Cplex & hoist, electroplating, container terminal, robot, yard crane &  & real-world, benchmark, random instance, real-life & edge-finding, genetic algorithm, time-tabling, lazy clause generation, meta heuristic & \ref{a:WallaceY20} & \ref{c:WallaceY20}\\
\index{WangMD15}\rowlabel{b:WangMD15}\href{../works/WangMD15.pdf}{WangMD15}~\cite{WangMD15} & 13 & stochastic, activity, job-shop, CP, CSP, order, make-span, job, resource, precedence, cmax, re-scheduling, scheduling, multi-objective, constraint programming, completion-time, constraint satisfaction, task, no-wait &  & cumulative, noOverlap &  & OPL, Cplex & nurse, operating room, physician, medical, patient, surgery &  & real-life, real-world & mat heuristic, particle swarm, time-tabling, column generation & \ref{a:WangMD15} & n/a\\
\index{WeilHFP95}\rowlabel{b:WeilHFP95}\href{../works/WeilHFP95.pdf}{WeilHFP95}~\cite{WeilHFP95} & 6 & task, resource, job, scheduling, CSP, constraint programming, job-shop, order, constraint satisfaction &  & cycle, Cardinality constraint & Lisp, Prolog, C++ & CHIP, OPL & patient, medical, nurse &  &  &  & \ref{a:WeilHFP95} & n/a\\
\index{WessenCSFPM23}\rowlabel{b:WessenCSFPM23}\href{../works/WessenCSFPM23.pdf}{WessenCSFPM23}~\cite{WessenCSFPM23} & 34 & precedence, blocking constraint, CP, no-wait, activity, scheduling, multi-agent, job-shop, constraint programming, resource, periodic, job, order, completion-time, flow-shop, distributed, cyclic scheduling, make-span, task &  & Blocking constraint, diffn, regular expression, alternative constraint, Regular constraint, cumulative, cycle, circuit &  & Gecode, MiniZinc & hoist, satellite, robot &  & real-world, benchmark, github &  & \ref{a:WessenCSFPM23} & \ref{c:WessenCSFPM23}\\
\index{WikarekS19}\rowlabel{b:WikarekS19}\href{../works/WikarekS19.pdf}{WikarekS19}~\cite{WikarekS19} & 22 & job, constraint programming, CSP, precedence, task, setup-time, CLP, machine, order, cmax, constraint logic programming, multi-agent, scheduling, preempt, manpower, make-span, constraint satisfaction, resource, distributed, preemptive, job-shop, flow-shop, CP, inventory & JSSP, RCPSP & cumulative, disjunctive &  & Z3, SCIP, ECLiPSe & robot &  &  & meta heuristic & \ref{a:WikarekS19} & n/a\\
\index{WuBB09}\rowlabel{b:WuBB09}\href{../works/WuBB09.pdf}{WuBB09}~\cite{WuBB09} & 9 & distributed, resource, job, constraint optimization, single-machine scheduling, CSP, scheduling, precedence, constraint satisfaction, stochastic, machine, job-shop, constraint programming, task, order, completion-time, CP, lateness, activity, flow-time, transportation & single machine & Channeling constraint, cumulative &  & Ilog Solver & railway, crew-scheduling &  & real-world &  & \ref{a:WuBB09} & n/a\\
\index{YounespourAKE19}\rowlabel{b:YounespourAKE19}\href{../works/YounespourAKE19.pdf}{YounespourAKE19}~\cite{YounespourAKE19} & 11 & re-scheduling, resource, inventory, order, cmax, precedence, constraint programming, Pareto, scheduling, completion-time, multi-objective, activity, machine, stochastic, CP, distributed, make-span &  & cumulative, noOverlap, alternative constraint, span constraint &  & OPL, Z3 & nurse, surgery, medical, operating room, patient &  & real-life, real-world & MINLP, ant colony & \ref{a:YounespourAKE19} & n/a\\
\index{YunusogluY22}\rowlabel{b:YunusogluY22}\href{../works/YunusogluY22.pdf}{YunusogluY22}~\cite{YunusogluY22} & 18 & constraint programming, order, release-date, bi-objective, lateness, precedence, sequence dependent setup, job-shop, resource, batch process, cmax, flow-time, completion-time, earliness, scheduling, machine, transportation, tardiness, make-span, unavailability, activity, setup-time, preempt, inventory, due-date, job, re-scheduling, CP, multi-objective, breakdown & PMSP, parallel machine & bin-packing, cumulative, noOverlap, endBeforeStart &  & OPL, Cplex & robot, medical & insulation industry & benchmark, real-life, real-world, generated instance, supplementary material & Lagrangian relaxation, mat heuristic, particle swarm, ant colony, simulated annealing, genetic algorithm, meta heuristic, GRASP & \ref{a:YunusogluY22} & \ref{c:YunusogluY22}\\
\index{YuraszeckMCCR23}\rowlabel{b:YuraszeckMCCR23}\href{../works/YuraszeckMCCR23.pdf}{YuraszeckMCCR23}~\cite{YuraszeckMCCR23} & 11 & flow-time, activity, machine, task, CP, make-span, multi-objective, resource, preempt, batch process, order, job, job-shop, setup-time, cmax, open-shop, precedence, constraint programming, flow-shop, scheduling & RCPSP, Open Shop Scheduling Problem, FJS, OSSP, Resource-constrained Project Scheduling Problem, JSSP & endBeforeStart, cumulative &  & OPL, Cplex &  & pharmaceutical industry & benchmark, github, real-world & GRASP, mat heuristic, meta heuristic & \ref{a:YuraszeckMCCR23} & \ref{c:YuraszeckMCCR23}\\
\index{YuraszeckMPV22}\rowlabel{b:YuraszeckMPV22}\href{../works/YuraszeckMPV22.pdf}{YuraszeckMPV22}~\cite{YuraszeckMPV22} & 26 & sequence dependent setup, no-wait, transportation, scheduling, order, make-span, release-date, setup-time, distributed, constraint programming, machine, flow-shop, CP, flow-time, job, re-scheduling, due-date, cyclic scheduling, stochastic, completion-time, resource, task, open-shop, job-shop & Open Shop Scheduling Problem, OSSP, single machine, JSSP & noOverlap, disjunctive, Disjunctive constraint & Java & Cplex & robot, semiconductor, automotive & manufacturing industry & real-life, generated instance, benchmark, github & meta heuristic, mat heuristic, genetic algorithm, ant colony, simulated annealing & \ref{a:YuraszeckMPV22} & \ref{c:YuraszeckMPV22}\\
\index{ZarandiASC20}\rowlabel{b:ZarandiASC20}\href{../works/ZarandiASC20.pdf}{ZarandiASC20}~\cite{ZarandiASC20} & 93 & tardiness, batch process, activity, multi-agent, completion-time, constraint satisfaction, due-date, scheduling, flow-shop, machine, job, re-scheduling, open-shop, make-span, energy efficiency, multi-objective, breakdown, explanation, setup-time, preempt, single-machine scheduling, order, inventory, bi-objective, distributed, lateness, no-wait, CP, resource, CSP, two-stage scheduling, net present value, one-machine scheduling, constraint logic programming, cmax, stochastic, reactive scheduling, task, constraint programming, flow-time, preemptive, Pareto, release-date, precedence, earliness, sequence dependent setup, job-shop, transportation, periodic, CLP & HFS, parallel machine, OSSP, JSSP, Resource-constrained Project Scheduling Problem, Open Shop Scheduling Problem, PMSP, RCPSP, single machine, FJS, Resource-constrained Project Scheduling Problem with Discounted Cashflow & disjunctive, cycle & Prolog & OPL & satellite, robot, sports scheduling, surgery, medical, round-robin, railway, business process, container terminal, nurse, semiconductor, tournament, evacuation, drone, crew-scheduling, train schedule, maintenance scheduling, aircraft, operating room, airport & textile industry, gas industry & real-world, benchmark, real-life & memetic algorithm, column generation, max-flow, time-tabling, neural network, meta heuristic, ant colony, simulated annealing, genetic algorithm, reinforcement learning, particle swarm, machine learning, Lagrangian relaxation, swarm intelligence & \ref{a:ZarandiASC20} & n/a\\
\index{ZarandiKS16}\rowlabel{b:ZarandiKS16}\href{../works/ZarandiKS16.pdf}{ZarandiKS16}~\cite{ZarandiKS16} & 17 & make-span, preemptive, completion-time, machine, preempt, earliness, CP, tardiness, single-machine scheduling, constraint satisfaction, order, distributed, breakdown, job, scheduling, resource, due-date, CSP, job-shop, transportation, task, constraint programming, flow-shop, multi-objective & single machine &  &  & Ilog Solver & robot &  & real-world & time-tabling, meta heuristic, genetic algorithm, machine learning, simulated annealing & \ref{a:ZarandiKS16} & n/a\\
\index{Zeballos10}\rowlabel{b:Zeballos10}\href{../works/Zeballos10.pdf}{Zeballos10}~\cite{Zeballos10} & 19 & order, multi-objective, tardiness, scheduling, task, machine, make-span, constraint satisfaction, due-date, activity, completion-time, CP, batch process, resource, breakdown, transportation, constraint programming, CSP, precedence &  &  &  & Ilog Solver, OPL, ECLiPSe, Ilog Scheduler & robot &  &  & ant colony & \ref{a:Zeballos10} & n/a\\
\index{ZeballosCM10}\rowlabel{b:ZeballosCM10}\href{../works/ZeballosCM10.pdf}{ZeballosCM10}~\cite{ZeballosCM10} & 11 & task, constraint programming, earliness, transportation, CLP, batch process, activity, scheduling, machine, job, re-scheduling, make-span, breakdown, order, inventory, flow-shop, CP, resource, reactive scheduling & single machine & circuit &  & Ilog Solver, Cplex, Z3, OPL & robot, semiconductor & process industry, semiconductor industry & real-life & simulated annealing, NEH & \ref{a:ZeballosCM10} & n/a\\
\index{ZeballosH05}\rowlabel{b:ZeballosH05}\href{../works/ZeballosH05.pdf}{ZeballosH05}~\cite{ZeballosH05} & 10 & make-span, order, machine, CP, tardiness, scheduling, buffer-capacity, due-date, CSP, job, activity, explanation, resource, transportation, completion-time, task, constraint satisfaction, constraint programming, precedence &  &  &  & OPL, Ilog Solver, Ilog Scheduler & robot &  &  & genetic algorithm & \ref{a:ZeballosH05} & n/a\\
\index{ZeballosNH11}\rowlabel{b:ZeballosNH11}\href{../works/ZeballosNH11.pdf}{ZeballosNH11}~\cite{ZeballosNH11} & 17 & due-date, scheduling, job, Benders Decomposition, make-span, manpower, setup-time, preempt, order, inventory, lateness, no-wait, CP, resource, CSP, task, constraint programming, preemptive, precedence, earliness, job-shop, CLP, tardiness, batch process, activity, completion-time, constraint satisfaction &  &  &  & Cplex, ECLiPSe, Ilog Scheduler, OPL, OZ, Ilog Solver &  & chemical industry & real-world & meta heuristic & \ref{a:ZeballosNH11} & n/a\\
\index{ZeballosQH10}\rowlabel{b:ZeballosQH10}\href{../works/ZeballosQH10.pdf}{ZeballosQH10}~\cite{ZeballosQH10} & 20 & make-span, breakdown, preempt, order, CP, resource, multi-objective, task, constraint programming, preemptive, precedence, earliness, job-shop, transportation, tardiness, cmax, activity, completion-time, constraint satisfaction, due-date, scheduling, machine, job &  &  &  & ECLiPSe, Ilog Scheduler, OPL, Ilog Solver, Cplex & robot &  & real-world, benchmark & ant colony, genetic algorithm & \ref{a:ZeballosQH10} & n/a\\
\index{ZhangW18}\rowlabel{b:ZhangW18}\href{../works/ZhangW18.pdf}{ZhangW18}~\cite{ZhangW18} & 18 & job, no-wait, lateness, transportation, unavailability, multi-agent, breakdown, tardiness, scheduling, multi-objective, constraint programming, completion-time, flow-shop, stochastic, setup-time, job-shop, CP, earliness, preempt, flow-time, distributed, resource, precedence, re-scheduling, order, make-span, machine & FJS & cumulative, noOverlap &  & Cplex, Z3, OPL & robot &  & benchmark & meta heuristic, ant colony, particle swarm, simulated annealing, genetic algorithm, memetic algorithm & \ref{a:ZhangW18} & n/a\\
\index{ZhangYW21}\rowlabel{b:ZhangYW21}\href{../works/ZhangYW21.pdf}{ZhangYW21}~\cite{ZhangYW21} & 10 & job, constraint satisfaction, preempt, setup-time, scheduling, precedence, make-span, task, preemptive, batch process, multi-agent, cmax, CP, machine, re-scheduling, constraint programming, order, multi-objective, activity, release-date, distributed, job-shop, resource & RCPSP, Resource-constrained Project Scheduling Problem & disjunctive, endBeforeStart &  & Cplex & robot &  & benchmark & memetic algorithm, meta heuristic, simulated annealing, genetic algorithm, particle swarm, ant colony & \ref{a:ZhangYW21} & n/a\\
\index{Zhou97}\rowlabel{b:Zhou97}\href{../works/Zhou97.pdf}{Zhou97}~\cite{Zhou97} & 29 & job-shop, due-date, constraint programming, task, order, preempt, preemptive, completion-time, constraint logic programming, CP, precedence, job, explanation, CLP, release-date, CSP, scheduling, constraint satisfaction, machine &  & Disjunctive constraint, disjunctive, cumulative & Prolog & CHIP, Z3, Ilog Scheduler &  &  & benchmark & edge-finder, edge-finding & \ref{a:Zhou97} & \ref{c:Zhou97}\\
\index{ZhuSZW23}\rowlabel{b:ZhuSZW23}\href{../works/ZhuSZW23.pdf}{ZhuSZW23}~\cite{ZhuSZW23} & 22 & order, scheduling, completion-time, machine, job, task, open-shop, transportation, multi-agent, cmax, constraint programming, job-shop, Benders Decomposition, Logic-Based Benders Decomposition, constraint satisfaction, preempt, CSP, setup-time, CP, make-span, distributed, resource, inventory, precedence, re-scheduling &  & alternative constraint, disjunctive, noOverlap, endBeforeStart &  & Cplex & robot & cable industry & real-world, benchmark & ant colony, particle swarm, genetic algorithm, column generation & \ref{a:ZhuSZW23} & n/a\\
\index{ZouZ20}\rowlabel{b:ZouZ20}\href{../works/ZouZ20.pdf}{ZouZ20}~\cite{ZouZ20} & 10 & resource, explanation, multi-objective, CSP, scheduling, constraint satisfaction, stochastic, constraint programming, task, order, completion-time, CP, activity, two-stage scheduling, precedence, distributed &  & noOverlap, span constraint, endBeforeStart, cumulative &  & Cplex, OPL & pipeline &  & benchmark & meta heuristic, genetic algorithm & \ref{a:ZouZ20} & n/a\\
\index{abs-0907-0939}\rowlabel{b:abs-0907-0939}\href{../works/abs-0907-0939.pdf}{abs-0907-0939}~\cite{abs-0907-0939} & 12 & constraint programming, resource, explanation, CSP, due-date, preempt, make-span, task, preemptive, CP, activity, scheduling, release-date, order &  & Cumulatives constraint, RelSoftCumulativeSum, cumulative, SoftCumulative, SoftCumulativeSum, Cardinality constraint, RelSoftCumulative & Java & Choco Solver, CHIP &  &  & real-world & sweep, energetic reasoning, edge-finding & \ref{a:abs-0907-0939} & n/a\\
\index{abs-1009-0347}\rowlabel{b:abs-1009-0347}\href{../works/abs-1009-0347.pdf}{abs-1009-0347}~\cite{abs-1009-0347} & 37 & constraint programming, cmax, resource, explanation, preempt, make-span, task, precedence, preemptive, CP, activity, scheduling, machine, order & RCPSP, SCC, Resource-constrained Project Scheduling Problem, psplib & cumulative, disjunctive, cycle & C++ & Ilog Scheduler, Ilog Solver, CHIP &  &  & benchmark, instance generator & genetic algorithm, lazy clause generation & \ref{a:abs-1009-0347} & n/a\\
\index{abs-1901-07914}\rowlabel{b:abs-1901-07914}\href{../works/abs-1901-07914.pdf}{abs-1901-07914}~\cite{abs-1901-07914} & 8 & constraint programming, CP, resource, CSP, constraint satisfaction, constraint optimization, task, distributed, machine, multi-agent, scheduling, order, make-span &  &  & Python & OR-Tools, MiniZinc & robot &  & real-world, github, benchmark &  & \ref{a:abs-1901-07914} & \ref{c:abs-1901-07914}\\
\index{abs-1902-01193}\rowlabel{b:abs-1902-01193}\href{../works/abs-1902-01193.pdf}{abs-1902-01193}~\cite{abs-1902-01193} & 9 & CSP, constraint satisfaction, constraint optimization, CLP, scheduling, activity, BOM, constraint programming, CP, order, constraint logic programming, stochastic, resource, task &  &  & Python, C++, Prolog & CHIP, Ilog Solver, OPL & medical, nurse &  &  & simulated annealing, meta heuristic, time-tabling, genetic algorithm, particle swarm & \ref{a:abs-1902-01193} & n/a\\
\index{abs-1902-09244}\rowlabel{b:abs-1902-09244}\href{../works/abs-1902-09244.pdf}{abs-1902-09244}~\cite{abs-1902-09244} & 62 & setup-time, activity, constraint programming, machine, flow-shop, CP, job, order, due-date, earliness, bi-objective, stochastic, explanation, completion-time, breakdown, resource, task, job-shop, tardiness, inventory, multi-objective, no-wait, precedence, transportation, scheduling, make-span, release-date & Resource-constrained Project Scheduling Problem, FJS, RCMPSP, RCPSP & cycle, cumulative, endBeforeStart &  & OPL, Cplex & aircraft & automobile industry, steel industry, food-processing industry, glass industry, processing industry & real-world, benchmark, industry partner & genetic algorithm, particle swarm, simulated annealing, meta heuristic & \ref{a:abs-1902-09244} & n/a\\
\index{abs-1911-04766}\rowlabel{b:abs-1911-04766}\href{../works/abs-1911-04766.pdf}{abs-1911-04766}~\cite{abs-1911-04766} & 16 & explanation, precedence, task, release-date, activity, multi-objective, scheduling, order, make-span, due-date, constraint programming, CP, completion-time, resource, job, re-scheduling & RCPSP, Resource-constrained Project Scheduling Problem & alternative constraint, endBeforeStart, noOverlap, Cardinality constraint, cumulative, disjunctive & Java & CPO, Cplex, MiniZinc, Chuffed, Gecode & automotive &  & real-world, benchmark, github, real-life, instance generator, generated instance, industrial partner & simulated annealing, meta heuristic, time-tabling, large neighborhood search & \ref{a:abs-1911-04766} & \ref{c:abs-1911-04766}\\
\index{abs-2102-08778}\rowlabel{b:abs-2102-08778}\href{../works/abs-2102-08778.pdf}{abs-2102-08778}~\cite{abs-2102-08778} & 10 & task, machine, flow-shop, scheduling, order, make-span, constraint programming, CP, resource, job, open-shop, job-shop, explanation & JSSP &  & Java & Cplex, OR-Tools, MiniZinc, OPL, CPO &  &  & benchmark, real-life, real-world, generated instance & genetic algorithm & \ref{a:abs-2102-08778} & n/a\\
\index{abs-2211-14492}\rowlabel{b:abs-2211-14492}\href{../works/abs-2211-14492.pdf}{abs-2211-14492}~\cite{abs-2211-14492} & 17 & distributed, flow-shop, multi-objective, transportation, scheduling, order, make-span, setup-time, activity, due-date, constraint programming, machine, CP, job, energy efficiency, cmax, completion-time, constraint satisfaction, resource, precedence, constraint optimization, task, job-shop, tardiness & single machine & bin-packing, disjunctive, cumulative, Disjunctive constraint & Python & OR-Tools, Cplex & semiconductor &  & generated instance, benchmark, random instance & quadratic programming, neural network, reinforcement learning, column generation, genetic algorithm, deep learning, ant colony, machine learning, meta heuristic & \ref{a:abs-2211-14492} & n/a\\
\index{abs-2305-19888}\rowlabel{b:abs-2305-19888}\href{../works/abs-2305-19888.pdf}{abs-2305-19888}~\cite{abs-2305-19888} & 42 & sequence dependent setup, distributed, flow-shop, scheduling, order, make-span, preempt, setup-time, activity, constraint programming, machine, CP, job, re-scheduling, unavailability, preemptive, bi-objective, explanation, cmax, completion-time, resource, precedence, task & parallel machine & alternative constraint, noOverlap, cumulative &  & Gurobi & robot, high performance computing &  & gitlab, generated instance, real-world, benchmark & meta heuristic, Lagrangian relaxation, genetic algorithm & \ref{a:abs-2305-19888} & \ref{c:abs-2305-19888}\\
\index{abs-2306-05747}\rowlabel{b:abs-2306-05747}\href{../works/abs-2306-05747.pdf}{abs-2306-05747}~\cite{abs-2306-05747} & 9 & re-scheduling, scheduling, order, make-span, preempt, constraint programming, CP, flow-time, completion-time, resource, job, periodic, job-shop, precedence, constraint optimization, task, tardiness, machine, flow-shop & JSSP & noOverlap, disjunctive, cumulative & Java & Choco Solver &  &  & real-world, github, industrial instance, supplementary material, benchmark & neural network, large neighborhood search, reinforcement learning, genetic algorithm, machine learning, meta heuristic, simulated annealing & \ref{a:abs-2306-05747} & \ref{c:abs-2306-05747}\\
\index{abs-2312-13682}\rowlabel{b:abs-2312-13682}\href{../works/abs-2312-13682.pdf}{abs-2312-13682}~\cite{abs-2312-13682} & 20 & activity, constraint programming, machine, inventory, re-scheduling, scheduling, order, make-span, CP, resource, transportation, task &  & table constraint, cumulative &  & OPL & container terminal, train schedule, nurse, steel mill, operating room &  & real-world, generated instance & large neighborhood search, mat heuristic, meta heuristic & \ref{a:abs-2312-13682} & \ref{c:abs-2312-13682}\\
\index{abs-2402-00459}\rowlabel{b:abs-2402-00459}\href{../works/abs-2402-00459.pdf}{abs-2402-00459}~\cite{abs-2402-00459} & 21 & job-shop, tardiness, multi-objective, scheduling, order, net present value, constraint programming, machine, CP, job, multi-agent, due-date, earliness, completion-time, resource, precedence, task & single machine, Resource-constrained Project Scheduling Problem & Disjunctive constraint, bin-packing, disjunctive, cumulative &  & OPL, OR-Tools & tournament & mining industry & instance generator, real-world, generated instance, benchmark, github & particle swarm, simulated annealing, meta heuristic, quadratic programming, Lagrangian relaxation, neural network, reinforcement learning, column generation, mat heuristic, genetic algorithm, ant colony, machine learning & \ref{a:abs-2402-00459} & \ref{c:abs-2402-00459}\\
\end{longtable}
}



\clearpage
\subsection{Manually Defined Fields}
{\scriptsize
\begin{longtable}{>{\raggedright\arraybackslash}p{3cm}>{\raggedright\arraybackslash}p{6cm}lp{2cm}rrrrlp{2cm}p{2cm}rr}
\rowcolor{white}\caption{Manually Defined ARTICLE Properties}\\ \toprule
\rowcolor{white}Key & Title (Local Copy) & \shortstack{CP\\System} & Bench & Links & \shortstack{Data\\Avail} & \shortstack{Sol\\Avail} & \shortstack{Code\\Avail} & \shortstack{Based\\On} & Classification & Constraints & a & b\\ \midrule\endhead
\bottomrule
\endfoot
\rowlabel{c:PrataAN23}PrataAN23 \href{https://www.sciencedirect.com/science/article/pii/S2666720723001522}{PrataAN23}~\cite{PrataAN23} & \href{works/PrataAN23.pdf}{Applications of constraint programming in production scheduling problems: A descriptive bibliometric analysis} & - & benchmark, real-world, real-life & 1 & - &  & - & - & survey & - & \ref{a:PrataAN23} & \ref{b:PrataAN23}\\
\rowlabel{c:abs-2402-00459}abs-2402-00459 \href{https://doi.org/10.48550/arXiv.2402.00459}{abs-2402-00459}~\cite{abs-2402-00459} & \href{works/abs-2402-00459.pdf}{Genetic-based Constraint Programming for Resource Constrained Job Scheduling} & OR-Tools & instance generator, real-world, generated instance, github, benchmark & 2 & \href{https://github.com/andreas-ernst/Mathprog-ORlib/blob/master/data/RCJS_new_instances.zip}{y} &  & n & - & RCJS & cumulatives & \ref{a:abs-2402-00459} & \ref{b:abs-2402-00459}\\
\rowlabel{c:AbreuNP23}AbreuNP23 \href{https://doi.org/10.1080/00207543.2022.2154404}{AbreuNP23}~\cite{AbreuNP23} & \href{works/AbreuNP23.pdf}{A new two-stage constraint programming approach for open shop scheduling problem with machine blocking} & ? & real-world, benchmark & 10 & ? &  & ? & ? & ? & ? & \ref{a:AbreuNP23} & \ref{b:AbreuNP23}\\
\rowlabel{c:AkramNHRSA23}AkramNHRSA23 \href{https://doi.org/10.1109/ACCESS.2023.3343409}{AkramNHRSA23}~\cite{AkramNHRSA23} & \href{works/AkramNHRSA23.pdf}{Joint Scheduling and Routing Optimization for Deterministic Hybrid Traffic in Time-Sensitive Networks Using Constraint Programming} & OR-Tools & benchmark & 0 & n &  & n & - & TSN & - & \ref{a:AkramNHRSA23} & \ref{b:AkramNHRSA23}\\
\rowlabel{c:AlfieriGPS23}AlfieriGPS23 \href{https://www.sciencedirect.com/science/article/pii/S0360835223000074}{AlfieriGPS23}~\cite{AlfieriGPS23} & \href{works/AlfieriGPS23.pdf}{Permutation flowshop problems minimizing core waiting time and core idle time} &  & benchmark & 0 &  &  &  &  &  &  & \ref{a:AlfieriGPS23} & \ref{b:AlfieriGPS23}\\
\rowlabel{c:Caballero23}Caballero23 \href{https://doi.org/10.1007/s10601-023-09357-0}{Caballero23}~\cite{Caballero23} & \href{works/Caballero23.pdf}{Scheduling through logic-based tools} & SAT &  & 1 & - &  & - & \href{http://hdl.handle.net/10803/667963}{PhD Thesis} & RCPSP & - & \ref{a:Caballero23} & \ref{b:Caballero23}\\
\rowlabel{c:CzerniachowskaWZ23}CzerniachowskaWZ23 \href{https://doi.org/10.12913/22998624/166588}{CzerniachowskaWZ23}~\cite{CzerniachowskaWZ23} & \href{works/CzerniachowskaWZ23.pdf}{Constraint Programming for Flexible Flow Shop Scheduling Problem with Repeated Jobs and Repeated Operations} &  & benchmark, Roadef, real-world & 0 &  &  &  &  &  &  & \ref{a:CzerniachowskaWZ23} & \ref{b:CzerniachowskaWZ23}\\
\rowlabel{c:GurPAE23}GurPAE23 \href{https://doi.org/10.1007/s10100-022-00835-z}{GurPAE23}~\cite{GurPAE23} & \href{works/GurPAE23.pdf}{Operating room scheduling with surgical team: a new approach with constraint programming and goal programming} & Cplex & real-life & 0 & n &  & n & - & - & - & \ref{a:GurPAE23} & \ref{b:GurPAE23}\\
\rowlabel{c:IsikYA23}IsikYA23 \href{https://doi.org/10.1007/s00500-023-09086-9}{IsikYA23}~\cite{IsikYA23} & \href{works/IsikYA23.pdf}{Constraint programming models for the hybrid flow shop scheduling problem and its extensions} & \su{OPL {CP Opt}} & real-world, benchmark, generated instance, real-life & 4 & \href{https://data.mendeley.com/datasets/n4g8cfjg87/1}{y} &  & \href{https://data.mendeley.com/datasets/n4g8cfjg87/1}{y} & - & HFSP & \su{alternative endBeforeStart noOverlap cumulative} & \ref{a:IsikYA23} & \ref{b:IsikYA23}\\
\rowlabel{c:LacknerMMWW23}LacknerMMWW23 \href{https://doi.org/10.1007/s10601-023-09347-2}{LacknerMMWW23}~\cite{LacknerMMWW23} & \href{works/LacknerMMWW23.pdf}{Exact methods for the Oven Scheduling Problem} & \su{MiniZinc OPL} & random instance, industrial partner, benchmark, instance generator, zenodo, real-life & 0 & \href{https://zenodo.org/records/7456938}{\su{DZN JSON}} &  & \href{https://zenodo.org/records/7456938}{y} & \cite{LacknerMMWW21} & OSP & \su{alternative noOverlap forbidExtent} & \ref{a:LacknerMMWW23} & \ref{b:LacknerMMWW23}\\
\rowlabel{c:MontemanniD23}MontemanniD23 \href{https://doi.org/10.3390/a16010040}{MontemanniD23}~\cite{MontemanniD23} & \href{works/MontemanniD23.pdf}{Solving the Parallel Drone Scheduling Traveling Salesman Problem via Constraint Programming} & OR-Tools & benchmark, supplementary material & 6 & ref & \href{https://www.mdpi.com/article/10.3390/a16010040/s1}{y} & n & - & PDSTSP & \su{circuit} & \ref{a:MontemanniD23} & \ref{b:MontemanniD23}\\
\rowlabel{c:MontemanniD23a}MontemanniD23a \href{https://doi.org/10.1016/j.ejco.2023.100078}{MontemanniD23a}~\cite{MontemanniD23a} & \href{works/MontemanniD23a.pdf}{Constraint programming models for the parallel drone scheduling vehicle routing problem} & OR-Tools & benchmark & 0 & ref &  & n & - & PDSTSP & \su{circuit multipleCircuit} & \ref{a:MontemanniD23a} & \ref{b:MontemanniD23a}\\
\rowlabel{c:NaderiRR23}NaderiRR23 \href{https://doi.org/10.1287/ijoc.2023.1287}{NaderiRR23}~\cite{NaderiRR23} & \href{works/NaderiRR23.pdf}{Mixed-Integer Programming vs. Constraint Programming for Shop Scheduling Problems: New Results and Outlook} &  & github, benchmark & 8 &  &  &  &  &  &  & \ref{a:NaderiRR23} & \ref{b:NaderiRR23}\\
\rowlabel{c:ShaikhK23}ShaikhK23 \href{https://doi.org/10.1504/IJESDF.2023.10045616}{ShaikhK23}~\cite{ShaikhK23} & \href{works/ShaikhK23.pdf}{Management of electronic ledger: a constraint programming approach for solving curricula scheduling problems} & ? & benchmark, real-world & 2 & ? &  & ? & ? & ? & ? & \ref{a:ShaikhK23} & \ref{b:ShaikhK23}\\
\rowlabel{c:YuraszeckMCCR23}YuraszeckMCCR23 \href{https://doi.org/10.1109/ACCESS.2023.3345793}{YuraszeckMCCR23}~\cite{YuraszeckMCCR23} & \href{works/YuraszeckMCCR23.pdf}{A Constraint Programming Formulation of the Multi-Mode Resource-Constrained Project Scheduling Problem for the Flexible Job Shop Scheduling Problem} & CP Opt & github, real-world, benchmark & 0 & ref &  & n & - & FJSSP & \su{alternative endBeforeStart cumulative} & \ref{a:YuraszeckMCCR23} & \ref{b:YuraszeckMCCR23}\\
\rowlabel{c:abs-2305-19888}abs-2305-19888 \href{https://doi.org/10.48550/arXiv.2305.19888}{abs-2305-19888}~\cite{abs-2305-19888} & \href{works/abs-2305-19888.pdf}{Constraint Programming and Constructive Heuristics for Parallel Machine Scheduling with Sequence-Dependent Setups and Common Servers} & \su{{CP Opt} Gurobi} & real-world, generated instance, gitlab, benchmark & 1 & \href{https://gitlab.com/vilem_heinz/cp_heur_paper_evalutation}{y} & \href{https://gitlab.com/vilem_heinz/cp_heur_paper_evalutation}{y} & n & - & $P|seq, ser|C_{max}$ & \su{alternative noOverlap cumulative} & \ref{a:abs-2305-19888} & \ref{b:abs-2305-19888}\\
\rowlabel{c:abs-2306-05747}abs-2306-05747 \href{https://doi.org/10.48550/arXiv.2306.05747}{abs-2306-05747}~\cite{abs-2306-05747} & \href{works/abs-2306-05747.pdf}{An End-to-End Reinforcement Learning Approach for Job-Shop Scheduling Problems Based on Constraint Programming} & \su{custom Choco} & real-world, supplementary material, github, industrial instance, benchmark & 0 & ref &  & n & - & JSSP & \su{noOverlap} & \ref{a:abs-2306-05747} & \ref{b:abs-2306-05747}\\
\rowlabel{c:abs-2312-13682}abs-2312-13682 \href{https://doi.org/10.48550/arXiv.2312.13682}{abs-2312-13682}~\cite{abs-2312-13682} & \href{works/abs-2312-13682.pdf}{A Constraint Programming Model for Scheduling the Unloading of Trains in Ports: Extended} & custom & real-world, generated instance & 0 & n &  & n & - & SUTP & \su{table disjunctive} & \ref{a:abs-2312-13682} & \ref{b:abs-2312-13682}\\
\rowlabel{c:AbreuN22}AbreuN22 \href{https://doi.org/10.1016/j.cie.2022.108128}{AbreuN22}~\cite{AbreuN22} & \href{works/AbreuN22.pdf}{A new hybridization of adaptive large neighborhood search with constraint programming for open shop scheduling with sequence-dependent setup times} & \su{Cplex {CP Opt}} & real-world, benchmark & 0 & \href{https://bit.ly/392wfZa}{y} &  & n & - & OSSPST & \su{noOverlap} & \ref{a:AbreuN22} & \ref{b:AbreuN22}\\
\rowlabel{c:BourreauGGLT22}BourreauGGLT22 \href{https://doi.org/10.1080/00207543.2020.1856436}{BourreauGGLT22}~\cite{BourreauGGLT22} & \href{works/BourreauGGLT22.pdf}{A constraint-programming based decomposition method for the Generalised Workforce Scheduling and Routing Problem {(GWSRP)}} &  & real-world, benchmark & 2 &  &  &  &  &  &  & \ref{a:BourreauGGLT22} & \ref{b:BourreauGGLT22}\\
\rowlabel{c:CampeauG22}CampeauG22 \href{https://doi.org/10.1007/s10601-022-09337-w}{CampeauG22}~\cite{CampeauG22} & \href{works/CampeauG22.pdf}{Short- and medium-term optimization of underground mine planning using constraint programming} & CP Opt & real-life, real-world & 0 & ref &  & n &  &  & \su{pulse alwaysIn endBeforeStart noOverlap} & \ref{a:CampeauG22} & \ref{b:CampeauG22}\\
\rowlabel{c:ColT22}ColT22 \href{https://api.semanticscholar.org/CorpusID:251551160}{ColT22}~\cite{ColT22} & \href{works/ColT22.pdf}{Industrial-size job shop scheduling with constraint programming} &  & generated instance, supplementary material, github, real-life, benchmark, real-world & 4 &  &  &  &  &  &  & \ref{a:ColT22} & \ref{b:ColT22}\\
\rowlabel{c:FarsiTM22}FarsiTM22 \href{https://api.semanticscholar.org/CorpusID:250301745}{FarsiTM22}~\cite{FarsiTM22} & \href{works/FarsiTM22.pdf}{Integrated surgery scheduling by constraint programming and meta-heuristics} &  & supplementary material & 10 &  &  &  &  &  &  & \ref{a:FarsiTM22} & \ref{b:FarsiTM22}\\
\rowlabel{c:Fatemi-AnarakiMFN22}Fatemi-AnarakiMFN22 \href{https://api.semanticscholar.org/CorpusID:252524295}{Fatemi-AnarakiMFN22}~\cite{Fatemi-AnarakiMFN22} & \href{}{Scheduling of Multi-Robot Job Shop Systems in Dynamic Environments: Mixed-Integer Linear Programming and Constraint Programming Approaches} &  &  & 0 &  &  &  &  &  &  & \ref{a:Fatemi-AnarakiMFN22} & No\\
\rowlabel{c:FetgoD22}FetgoD22 \href{https://doi.org/10.1007/s43069-022-00172-6}{FetgoD22}~\cite{FetgoD22} & \href{works/FetgoD22.pdf}{Horizontally Elastic Edge-Finder Algorithm for Cumulative Resource Constraint Revisited} &  & benchmark, real-world & 7 &  &  &  &  &  &  & \ref{a:FetgoD22} & \ref{b:FetgoD22}\\
\rowlabel{c:HeinzNVH22}HeinzNVH22 \href{https://doi.org/10.1016/j.cie.2022.108586}{HeinzNVH22}~\cite{HeinzNVH22} & \href{works/HeinzNVH22.pdf}{Constraint Programming and constructive heuristics for parallel machine scheduling with sequence-dependent setups and common servers} &  & real-world, generated instance, benchmark, gitlab & 3 &  &  &  &  &  &  & \ref{a:HeinzNVH22} & \ref{b:HeinzNVH22}\\
\rowlabel{c:MullerMKP22}MullerMKP22 \href{https://doi.org/10.1016/j.ejor.2022.01.034}{MullerMKP22}~\cite{MullerMKP22} & \href{works/MullerMKP22.pdf}{An algorithm selection approach for the flexible job shop scheduling problem: Choosing constraint programming solvers through machine learning} &  & benchmark, random instance, real-world, github & 3 &  &  &  &  &  &  & \ref{a:MullerMKP22} & \ref{b:MullerMKP22}\\
\rowlabel{c:PohlAK22}PohlAK22 \href{https://doi.org/10.1016/j.ejor.2021.08.028}{PohlAK22}~\cite{PohlAK22} & \href{works/PohlAK22.pdf}{Solving the time-discrete winter runway scheduling problem: {A} column generation and constraint programming approach} &  & benchmark, real-world & 2 &  &  &  &  &  &  & \ref{a:PohlAK22} & \ref{b:PohlAK22}\\
\rowlabel{c:ShiYXQ22}ShiYXQ22 \href{https://doi.org/10.1080/00207543.2021.1963496}{ShiYXQ22}~\cite{ShiYXQ22} & \href{}{Solving the integrated process planning and scheduling problem using an enhanced constraint programming-based approach} &  &  & 0 &  &  &  &  &  &  & \ref{a:ShiYXQ22} & No\\
\rowlabel{c:SubulanC22}SubulanC22 \href{https://doi.org/10.1007/s00500-021-06399-5}{SubulanC22}~\cite{SubulanC22} & \href{works/SubulanC22.pdf}{Constraint programming-based transformation approach for a mixed fuzzy-stochastic resource investment project scheduling problem} &  & real-life, benchmark, real-world & 2 &  &  &  &  &  &  & \ref{a:SubulanC22} & \ref{b:SubulanC22}\\
\rowlabel{c:YunusogluY22}YunusogluY22 \href{https://doi.org/10.1080/00207543.2021.1885068}{YunusogluY22}~\cite{YunusogluY22} & \href{works/YunusogluY22.pdf}{Constraint programming approach for multi-resource-constrained unrelated parallel machine scheduling problem with sequence-dependent setup times} &  & real-world, benchmark, generated instance, real-life, supplementary material & 10 &  &  &  &  &  &  & \ref{a:YunusogluY22} & \ref{b:YunusogluY22}\\
\rowlabel{c:YuraszeckMPV22}YuraszeckMPV22 \href{https://api.semanticscholar.org/CorpusID:246320449}{YuraszeckMPV22}~\cite{YuraszeckMPV22} & \href{works/YuraszeckMPV22.pdf}{A Novel Constraint Programming Decomposition Approach for the Total Flow Time Fixed Group Shop Scheduling Problem} &  & generated instance, github, benchmark, real-life & 5 &  &  &  &  &  &  & \ref{a:YuraszeckMPV22} & \ref{b:YuraszeckMPV22}\\
\rowlabel{c:abs-2211-14492}abs-2211-14492 \href{https://doi.org/10.48550/arXiv.2211.14492}{abs-2211-14492}~\cite{abs-2211-14492} & \href{works/abs-2211-14492.pdf}{Enhancing Constraint Programming via Supervised Learning for Job Shop Scheduling} &  & benchmark, random instance, generated instance & 1 &  &  &  &  &  &  & \ref{a:abs-2211-14492} & \ref{b:abs-2211-14492}\\
\rowlabel{c:AbohashimaEG21}AbohashimaEG21 \href{https://doi.org/10.1109/ACCESS.2021.3112600}{AbohashimaEG21}~\cite{AbohashimaEG21} & \href{works/AbohashimaEG21.pdf}{A Mathematical Programming Model and a Firefly-Based Heuristic for Real-Time Traffic Signal Scheduling With Physical Constraints} &  & real-world, generated instance, github & 0 &  &  &  &  &  &  & \ref{a:AbohashimaEG21} & \ref{b:AbohashimaEG21}\\
\rowlabel{c:AbreuAPNM21}AbreuAPNM21 \href{https://api.semanticscholar.org/CorpusID:238794651}{AbreuAPNM21}~\cite{AbreuAPNM21} & \href{works/AbreuAPNM21.pdf}{A new variable neighbourhood search with a constraint programming search strategy for the open shop scheduling problem with operation repetitions} &  & generated instance, benchmark, real-world & 8 &  &  &  &  &  &  & \ref{a:AbreuAPNM21} & \ref{b:AbreuAPNM21}\\
\rowlabel{c:Bedhief21}Bedhief21 \href{https://api.semanticscholar.org/CorpusID:240611192}{Bedhief21}~\cite{Bedhief21} & \href{works/Bedhief21.pdf}{Comparing Mixed-Integer Programming and Constraint Programming Models for the Hybrid Flow Shop Scheduling Problem with Dedicated Machines} &  & real-life & 0 &  &  &  &  &  &  & \ref{a:Bedhief21} & \ref{b:Bedhief21}\\
\rowlabel{c:FanXG21}FanXG21 \href{https://doi.org/10.1016/j.cor.2021.105401}{FanXG21}~\cite{FanXG21} & \href{works/FanXG21.pdf}{Genetic programming-based hyper-heuristic approach for solving dynamic job shop scheduling problem with extended technical precedence constraints} &  & benchmark & 0 &  &  &  &  &  &  & \ref{a:FanXG21} & \ref{b:FanXG21}\\
\rowlabel{c:HamPK21}HamPK21 \href{https://api.semanticscholar.org/CorpusID:237898414}{HamPK21}~\cite{HamPK21} & \href{works/HamPK21.pdf}{Energy-Aware Flexible Job Shop Scheduling Using Mixed Integer Programming and Constraint Programming} &  & benchmark, github & 4 &  &  &  &  &  &  & \ref{a:HamPK21} & \ref{b:HamPK21}\\
\rowlabel{c:HubnerGSV21}HubnerGSV21 \href{https://doi.org/10.1007/s10951-021-00682-x}{HubnerGSV21}~\cite{HubnerGSV21} & \href{works/HubnerGSV21.pdf}{Solving the nuclear dismantling project scheduling problem by combining mixed-integer and constraint programming techniques and metaheuristics} &  & benchmark, real-life & 4 &  &  &  &  &  &  & \ref{a:HubnerGSV21} & \ref{b:HubnerGSV21}\\
\rowlabel{c:KoehlerBFFHPSSS21}KoehlerBFFHPSSS21 \href{https://doi.org/10.1007/s10601-021-09321-w}{KoehlerBFFHPSSS21}~\cite{KoehlerBFFHPSSS21} & \href{works/KoehlerBFFHPSSS21.pdf}{Cable tree wiring - benchmarking solvers on a real-world scheduling problem with a variety of precedence constraints} & \su{{CP Opt} OR-Tools Chuffed Cplex Gurobi Z3 OptiMathSat} & real-world, benchmark, github & 9 & \href{https://github.com/kw90/ctw_toolchain}{DZN} &  & y & - & CTW & \su{alldifferent inverse} & \ref{a:KoehlerBFFHPSSS21} & \ref{b:KoehlerBFFHPSSS21}\\
\rowlabel{c:PandeyS21a}PandeyS21a \href{https://doi.org/10.1007/s11227-020-03516-3}{PandeyS21a}~\cite{PandeyS21a} & \href{works/PandeyS21a.pdf}{Constraint programming versus heuristic approach to MapReduce scheduling problem in Hadoop {YARN} for energy minimization} &  & benchmark & 1 &  &  &  &  &  &  & \ref{a:PandeyS21a} & \ref{b:PandeyS21a}\\
\rowlabel{c:QinWSLS21}QinWSLS21 \href{https://doi.org/10.1109/TASE.2019.2947398}{QinWSLS21}~\cite{QinWSLS21} & \href{works/QinWSLS21.pdf}{A Genetic Programming-Based Scheduling Approach for Hybrid Flow Shop With a Batch Processor and Waiting Time Constraint} &  &  & 0 &  &  &  &  &  &  & \ref{a:QinWSLS21} & \ref{b:QinWSLS21}\\
\rowlabel{c:VlkHT21}VlkHT21 \href{https://doi.org/10.1016/j.cie.2021.107317}{VlkHT21}~\cite{VlkHT21} & \href{works/VlkHT21.pdf}{Constraint programming approaches to joint routing and scheduling in time-sensitive networks} &  & industrial partner, random instance, github, benchmark & 0 &  &  &  &  &  &  & \ref{a:VlkHT21} & \ref{b:VlkHT21}\\
\rowlabel{c:ZhangYW21}ZhangYW21 \href{https://doi.org/10.1016/j.cor.2021.105282}{ZhangYW21}~\cite{ZhangYW21} & \href{works/ZhangYW21.pdf}{A graph-based constraint programming approach for the integrated process planning and scheduling problem} &  & benchmark & 0 &  &  &  &  &  &  & \ref{a:ZhangYW21} & \ref{b:ZhangYW21}\\
\rowlabel{c:abs-2102-08778}abs-2102-08778 \href{https://arxiv.org/abs/2102.08778}{abs-2102-08778}~\cite{abs-2102-08778} & \href{works/abs-2102-08778.pdf}{Large-Scale Benchmarks for the Job Shop Scheduling Problem} &  & generated instance, benchmark, real-life, real-world & 0 &  &  &  &  &  &  & \ref{a:abs-2102-08778} & \ref{b:abs-2102-08778}\\
\rowlabel{c:AlizdehS20}AlizdehS20 \href{https://doi.org/10.1504/IJAIP.2020.106687}{AlizdehS20}~\cite{AlizdehS20} & \href{}{Fuzzy project scheduling with critical path including risk and resource constraints using linear programming} &  &  & 0 &  &  &  &  &  &  & \ref{a:AlizdehS20} & No\\
\rowlabel{c:AstrandJZ20}AstrandJZ20 \href{https://doi.org/10.1016/j.cor.2020.105036}{AstrandJZ20}~\cite{AstrandJZ20} & \href{works/AstrandJZ20.pdf}{Underground mine scheduling of mobile machines using Constraint Programming and Large Neighborhood Search} &  & benchmark, real-world, real-life & 0 &  &  &  &  &  &  & \ref{a:AstrandJZ20} & \ref{b:AstrandJZ20}\\
\rowlabel{c:BadicaBI20}BadicaBI20 \href{https://doi.org/10.3233/AIC-200650}{BadicaBI20}~\cite{BadicaBI20} & \href{works/BadicaBI20.pdf}{Block structured scheduling using constraint logic programming} &  & real-world, benchmark & 5 &  &  &  &  &  &  & \ref{a:BadicaBI20} & \ref{b:BadicaBI20}\\
\rowlabel{c:BenediktMH20}BenediktMH20 \href{https://doi.org/10.1007/s10601-020-09317-y}{BenediktMH20}~\cite{BenediktMH20} & \href{works/BenediktMH20.pdf}{Power of pre-processing: production scheduling with variable energy pricing and power-saving states} & \su{{CP Opt} Gurobi} & github, benchmark, random instance, generated instance & 4 & \href{https://github.com/CTU-IIG/EnergyStatesAndCostsSchedulingData}{JSON} &  & \href{https://github.com/CTU-IIG/EnergyStatesAndCostsScheduling}{y} &  &  &  & \ref{a:BenediktMH20} & \ref{b:BenediktMH20}\\
\rowlabel{c:FallahiAC20}FallahiAC20 \href{https://api.semanticscholar.org/CorpusID:213449737}{FallahiAC20}~\cite{FallahiAC20} & \href{works/FallahiAC20.pdf}{Tabu search and constraint programming-based approach for a real scheduling and routing problem} &  & github, real-life & 0 &  &  &  &  &  &  & \ref{a:FallahiAC20} & \ref{b:FallahiAC20}\\
\rowlabel{c:LunardiBLRV20}LunardiBLRV20 \href{https://doi.org/10.1016/j.cor.2020.105020}{LunardiBLRV20}~\cite{LunardiBLRV20} & \href{works/LunardiBLRV20.pdf}{Mixed Integer linear programming and constraint programming models for the online printing shop scheduling problem} &  & benchmark, random instance, generated instance, github & 1 &  &  &  &  &  &  & \ref{a:LunardiBLRV20} & \ref{b:LunardiBLRV20}\\
\rowlabel{c:MejiaY20}MejiaY20 \href{https://doi.org/10.1016/j.ejor.2020.02.010}{MejiaY20}~\cite{MejiaY20} & \href{works/MejiaY20.pdf}{A self-tuning variable neighborhood search algorithm and an effective decoding scheme for open shop scheduling problems with travel/setup times} &  & supplementary material, benchmark, generated instance & 2 &  &  &  &  &  &  & \ref{a:MejiaY20} & \ref{b:MejiaY20}\\
\rowlabel{c:MengZRZL20}MengZRZL20 \href{https://doi.org/10.1016/j.cie.2020.106347}{MengZRZL20}~\cite{MengZRZL20} & \href{works/MengZRZL20.pdf}{Mixed-integer linear programming and constraint programming formulations for solving distributed flexible job shop scheduling problem} &  & supplementary material, benchmark & 0 &  &  &  &  &  &  & \ref{a:MengZRZL20} & \ref{b:MengZRZL20}\\
\rowlabel{c:MokhtarzadehTNF20}MokhtarzadehTNF20 \href{https://doi.org/10.1080/0951192X.2020.1736713}{MokhtarzadehTNF20}~\cite{MokhtarzadehTNF20} & \href{works/MokhtarzadehTNF20.pdf}{Scheduling of human-robot collaboration in assembly of printed circuit boards: a constraint programming approach} &  & generated instance, real-world & 12 &  &  &  &  &  &  & \ref{a:MokhtarzadehTNF20} & \ref{b:MokhtarzadehTNF20}\\
\rowlabel{c:Polo-MejiaALB20}Polo-MejiaALB20 \href{https://doi.org/10.1080/00207543.2019.1693654}{Polo-MejiaALB20}~\cite{Polo-MejiaALB20} & \href{works/Polo-MejiaALB20.pdf}{Mixed-integer/linear and constraint programming approaches for activity scheduling in a nuclear research facility} &  & Roadef, github & 2 &  &  &  &  &  &  & \ref{a:Polo-MejiaALB20} & \ref{b:Polo-MejiaALB20}\\
\rowlabel{c:QinDCS20}QinDCS20 \href{https://doi.org/10.1016/j.ejor.2020.02.021}{QinDCS20}~\cite{QinDCS20} & \href{works/QinDCS20.pdf}{Combining mixed integer programming and constraint programming to solve the integrated scheduling problem of container handling operations of a single vessel} &  & real-life, benchmark & 0 &  &  &  &  &  &  & \ref{a:QinDCS20} & \ref{b:QinDCS20}\\
\rowlabel{c:SacramentoSP20}SacramentoSP20 \href{https://doi.org/10.1007/s43069-020-00036-x}{SacramentoSP20}~\cite{SacramentoSP20} & \href{works/SacramentoSP20.pdf}{Constraint Programming and Local Search Heuristic: a Matheuristic Approach for Routing and Scheduling Feeder Vessels in Multi-terminal Ports} &  & benchmark, real-life, zenodo, real-world & 4 &  &  &  &  &  &  & \ref{a:SacramentoSP20} & \ref{b:SacramentoSP20}\\
\rowlabel{c:WallaceY20}WallaceY20 \href{https://doi.org/10.1007/s10601-020-09316-z}{WallaceY20}~\cite{WallaceY20} & \href{works/WallaceY20.pdf}{A new constraint programming model and solving for the cyclic hoist scheduling problem} & MiniZinc & random instance, real-life, real-world, benchmark & 2 & \href{https://data.4tu.nl/articles/_/12912413}{DZN} &  & \href{https://data.4tu.nl/articles/_/12912413}{y} &  & CHSP &  & \ref{a:WallaceY20} & \ref{b:WallaceY20}\\
\rowlabel{c:ZouZ20}ZouZ20 \href{https://api.semanticscholar.org/CorpusID:208840808}{ZouZ20}~\cite{ZouZ20} & \href{works/ZouZ20.pdf}{A constraint programming approach for scheduling repetitive projects with atypical activities considering soft logic} &  & benchmark & 3 &  &  &  &  &  &  & \ref{a:ZouZ20} & \ref{b:ZouZ20}\\
\rowlabel{c:EscobetPQPRA19}EscobetPQPRA19 \href{https://doi.org/10.1016/j.compchemeng.2018.08.040}{EscobetPQPRA19}~\cite{EscobetPQPRA19} & \href{works/EscobetPQPRA19.pdf}{Optimal batch scheduling of a multiproduct dairy process using a combined optimization/constraint programming approach} &  &  & 1 &  &  &  &  &  &  & \ref{a:EscobetPQPRA19} & \ref{b:EscobetPQPRA19}\\
\rowlabel{c:GurEA19}GurEA19 \href{https://api.semanticscholar.org/CorpusID:88492001}{GurEA19}~\cite{GurEA19} & \href{works/GurEA19.pdf}{Surgical Operation Scheduling with Goal Programming and Constraint Programming: A Case Study} &  & real-life & 11 &  &  &  &  &  &  & \ref{a:GurEA19} & \ref{b:GurEA19}\\
\rowlabel{c:NishikawaSTT19}NishikawaSTT19 \href{http://www.ijnc.org/index.php/ijnc/article/view/201}{NishikawaSTT19}~\cite{NishikawaSTT19} & \href{works/NishikawaSTT19.pdf}{A Constraint Programming Approach to Scheduling of Malleable Tasks} &  & real-world, benchmark & 0 &  &  &  &  &  &  & \ref{a:NishikawaSTT19} & \ref{b:NishikawaSTT19}\\
\rowlabel{c:Novas19}Novas19 \href{https://doi.org/10.1016/j.cie.2019.07.011}{Novas19}~\cite{Novas19} & \href{works/Novas19.pdf}{Production scheduling and lot streaming at flexible job-shops environments using constraint programming} &  & benchmark & 0 &  &  &  &  &  &  & \ref{a:Novas19} & \ref{b:Novas19}\\
\rowlabel{c:WikarekS19}WikarekS19 \href{https://doi.org/10.1142/S2196888819500027}{WikarekS19}~\cite{WikarekS19} & \href{works/WikarekS19.pdf}{A Constraint-Based Declarative Programming Framework for Scheduling and Resource Allocation Problems} &  &  & 0 &  &  &  &  &  &  & \ref{a:WikarekS19} & \ref{b:WikarekS19}\\
\rowlabel{c:YounespourAKE19}YounespourAKE19 \href{https://api.semanticscholar.org/CorpusID:208103305}{YounespourAKE19}~\cite{YounespourAKE19} & \href{works/YounespourAKE19.pdf}{Using mixed integer programming and constraint programming for operating rooms scheduling with modified block strategy} &  & real-life, real-world & 6 &  &  &  &  &  &  & \ref{a:YounespourAKE19} & \ref{b:YounespourAKE19}\\
\rowlabel{c:abs-1901-07914}abs-1901-07914 \href{http://arxiv.org/abs/1901.07914}{abs-1901-07914}~\cite{abs-1901-07914} & \href{works/abs-1901-07914.pdf}{A Constraint Programming Approach to Simultaneous Task Allocation and Motion Scheduling for Industrial Dual-Arm Manipulation Tasks} &  & benchmark, real-world, github & 0 &  &  &  &  &  &  & \ref{a:abs-1901-07914} & \ref{b:abs-1901-07914}\\
\rowlabel{c:abs-1902-01193}abs-1902-01193 \href{http://arxiv.org/abs/1902.01193}{abs-1902-01193}~\cite{abs-1902-01193} & \href{works/abs-1902-01193.pdf}{Solving Nurse Scheduling Problem Using Constraint Programming Technique} &  &  & 0 &  &  &  &  &  &  & \ref{a:abs-1902-01193} & \ref{b:abs-1902-01193}\\
\rowlabel{c:abs-1902-09244}abs-1902-09244 \href{http://arxiv.org/abs/1902.09244}{abs-1902-09244}~\cite{abs-1902-09244} & \href{works/abs-1902-09244.pdf}{On constraint programming for a new flexible project scheduling problem with resource constraints} &  & benchmark, industry partner, real-world & 0 &  &  &  &  &  &  & \ref{a:abs-1902-09244} & \ref{b:abs-1902-09244}\\
\rowlabel{c:abs-1911-04766}abs-1911-04766 \href{http://arxiv.org/abs/1911.04766}{abs-1911-04766}~\cite{abs-1911-04766} & \href{works/abs-1911-04766.pdf}{Investigating Constraint Programming and Hybrid Methods for Real World Industrial Test Laboratory Scheduling} &  & real-world, generated instance, industrial partner, github, benchmark, instance generator, real-life & 10 &  &  &  &  &  &  & \ref{a:abs-1911-04766} & \ref{b:abs-1911-04766}\\
\rowlabel{c:BaptisteB18}BaptisteB18 \href{https://doi.org/10.1016/j.dam.2017.05.001}{BaptisteB18}~\cite{BaptisteB18} & \href{works/BaptisteB18.pdf}{Redundant cumulative constraints to compute preemptive bounds} &  &  & 1 &  &  &  &  &  &  & \ref{a:BaptisteB18} & \ref{b:BaptisteB18}\\
\rowlabel{c:BorghesiBLMB18}BorghesiBLMB18 \href{https://doi.org/10.1016/j.suscom.2018.05.007}{BorghesiBLMB18}~\cite{BorghesiBLMB18} & \href{works/BorghesiBLMB18.pdf}{Scheduling-based power capping in high performance computing systems} &  & benchmark, real-life & 3 &  &  &  &  &  &  & \ref{a:BorghesiBLMB18} & \ref{b:BorghesiBLMB18}\\
\rowlabel{c:FahimiOQ18}FahimiOQ18 \href{https://doi.org/10.1007/s10601-018-9282-9}{FahimiOQ18}~\cite{FahimiOQ18} & \href{works/FahimiOQ18.pdf}{Linear-time filtering algorithms for the disjunctive constraint and a quadratic filtering algorithm for the cumulative not-first not-last} & Choco & benchmark, random instance & 0 & (y) &  & n &  & RCPSP & \su{disjunctive cumulative} & \ref{a:FahimiOQ18} & \ref{b:FahimiOQ18}\\
\rowlabel{c:GedikKEK18}GedikKEK18 \href{https://doi.org/10.1016/j.cie.2018.05.014}{GedikKEK18}~\cite{GedikKEK18} & \href{works/GedikKEK18.pdf}{A constraint programming approach for solving unrelated parallel machine scheduling problem} &  & benchmark & 9 &  &  &  &  &  &  & \ref{a:GedikKEK18} & \ref{b:GedikKEK18}\\
\rowlabel{c:GokgurHO18}GokgurHO18 \href{https://doi.org/10.1080/00207543.2017.1421781}{GokgurHO18}~\cite{GokgurHO18} & \href{works/GokgurHO18.pdf}{Parallel machine scheduling with tool loading: a constraint programming approach} &  & real-life, real-world & 9 &  &  &  &  &  &  & \ref{a:GokgurHO18} & \ref{b:GokgurHO18}\\
\rowlabel{c:Ham18}Ham18 \href{https://api.semanticscholar.org/CorpusID:116853255}{Ham18}~\cite{Ham18} & \href{works/Ham18.pdf}{Integrated scheduling of m-truck, m-drone, and m-depot constrained by time-window, drop-pickup, and m-visit using constraint programming} &  &  & 7 &  &  &  &  &  &  & \ref{a:Ham18} & \ref{b:Ham18}\\
\rowlabel{c:LaborieRSV18}LaborieRSV18 \href{https://doi.org/10.1007/s10601-018-9281-x}{LaborieRSV18}~\cite{LaborieRSV18} & \href{works/LaborieRSV18.pdf}{{IBM} {ILOG} {CP} optimizer for scheduling - 20+ years of scheduling with constraints at {IBM/ILOG}} & OP Opt & real-world, CSPlib, benchmark & 3 & - &  & - & - & - & - & \ref{a:LaborieRSV18} & \ref{b:LaborieRSV18}\\
\rowlabel{c:PourDERB18}PourDERB18 \href{https://doi.org/10.1016/j.ejor.2017.08.033}{PourDERB18}~\cite{PourDERB18} & \href{works/PourDERB18.pdf}{A hybrid Constraint Programming/Mixed Integer Programming framework for the preventive signaling maintenance crew scheduling problem} &  & real-life, benchmark, real-world, generated instance & 1 &  &  &  &  &  &  & \ref{a:PourDERB18} & \ref{b:PourDERB18}\\
\rowlabel{c:ShinBBHO18}ShinBBHO18 \href{https://doi.org/10.1109/TSMC.2017.2681623}{ShinBBHO18}~\cite{ShinBBHO18} & \href{works/ShinBBHO18.pdf}{Discrete-Event Simulation and Integer Linear Programming for Constraint-Aware Resource Scheduling} &  & github, real-world & 4 &  &  &  &  &  &  & \ref{a:ShinBBHO18} & \ref{b:ShinBBHO18}\\
\rowlabel{c:TangLWSK18}TangLWSK18 \href{https://doi.org/10.1111/mice.12277}{TangLWSK18}~\cite{TangLWSK18} & \href{works/TangLWSK18.pdf}{Scheduling Optimization of Linear Schedule with Constraint Programming} &  &  & 0 &  &  &  &  &  &  & \ref{a:TangLWSK18} & \ref{b:TangLWSK18}\\
\rowlabel{c:ZhangW18}ZhangW18 \href{https://doi.org/10.1109/TEM.2017.2785774}{ZhangW18}~\cite{ZhangW18} & \href{works/ZhangW18.pdf}{Flexible Assembly Job-Shop Scheduling With Sequence-Dependent Setup Times and Part Sharing in a Dynamic Environment: Constraint Programming Model, Mixed-Integer Programming Model, and Dispatching Rules} &  & benchmark & 0 &  &  &  &  &  &  & \ref{a:ZhangW18} & \ref{b:ZhangW18}\\
\rowlabel{c:KreterSS17}KreterSS17 \href{https://doi.org/10.1007/s10601-016-9266-6}{KreterSS17}~\cite{KreterSS17} & \href{works/KreterSS17.pdf}{Using constraint programming for solving RCPSP/max-cal} & \su{MiniZinc Chuffed Cplex} & benchmark & 5 & dead &  &  & \cite{KreterSS15} & RCPSP & \su{cumulative cumulativeCalendar} & \ref{a:KreterSS17} & \ref{b:KreterSS17}\\
\rowlabel{c:NattafAL17}NattafAL17 \href{https://doi.org/10.1007/s10601-017-9271-4}{NattafAL17}~\cite{NattafAL17} & \href{works/NattafAL17.pdf}{Cumulative scheduling with variable task profiles and concave piecewise linear processing rate functions} & Cplex & real-world & 2 & n &  & n & - & CECSP & - & \ref{a:NattafAL17} & \ref{b:NattafAL17}\\
\rowlabel{c:Bonfietti16}Bonfietti16 \href{https://doi.org/10.3233/IA-160095}{Bonfietti16}~\cite{Bonfietti16} & \href{works/Bonfietti16.pdf}{A constraint programming scheduling solver for the MPOpt programming environment} &  & benchmark & 10 &  &  &  &  &  &  & \ref{a:Bonfietti16} & \ref{b:Bonfietti16}\\
\rowlabel{c:BridiBLMB16}BridiBLMB16 \href{https://doi.org/10.1109/TPDS.2016.2516997}{BridiBLMB16}~\cite{BridiBLMB16} & \href{works/BridiBLMB16.pdf}{A Constraint Programming Scheduler for Heterogeneous High-Performance Computing Machines} &  & real-world, real-life & 0 &  &  &  &  &  &  & \ref{a:BridiBLMB16} & \ref{b:BridiBLMB16}\\
\rowlabel{c:DoulabiRP16}DoulabiRP16 \href{https://doi.org/10.1287/ijoc.2015.0686}{DoulabiRP16}~\cite{DoulabiRP16} & \href{works/DoulabiRP16.pdf}{A Constraint-Programming-Based Branch-and-Price-and-Cut Approach for Operating Room Planning and Scheduling} &  & real-world, generated instance & 3 &  &  &  &  &  &  & \ref{a:DoulabiRP16} & \ref{b:DoulabiRP16}\\
\rowlabel{c:NovaraNH16}NovaraNH16 \href{https://doi.org/10.1016/j.compchemeng.2016.04.030}{NovaraNH16}~\cite{NovaraNH16} & \href{works/NovaraNH16.pdf}{A novel constraint programming model for large-scale scheduling problems in multiproduct multistage batch plants: Limited resources and campaign-based operation} &  & CSPlib, benchmark & 5 &  &  &  &  &  &  & \ref{a:NovaraNH16} & \ref{b:NovaraNH16}\\
\rowlabel{c:ZarandiKS16}ZarandiKS16 \href{https://doi.org/10.1007/s10845-013-0860-9}{ZarandiKS16}~\cite{ZarandiKS16} & \href{works/ZarandiKS16.pdf}{A constraint programming model for the scheduling of {JIT} cross-docking systems with preemption} &  & real-world & 0 &  &  &  &  &  &  & \ref{a:ZarandiKS16} & \ref{b:ZarandiKS16}\\
\rowlabel{c:EvenSH15a}EvenSH15a \href{http://arxiv.org/abs/1505.02487}{EvenSH15a}~\cite{EvenSH15a} & \href{works/EvenSH15a.pdf}{A Constraint Programming Approach for Non-Preemptive Evacuation Scheduling} &  & real-world, real-life & 2 &  &  &  &  &  &  & \ref{a:EvenSH15a} & \ref{b:EvenSH15a}\\
\rowlabel{c:GoelSHFS15}GoelSHFS15 \href{https://doi.org/10.1016/j.ejor.2014.09.048}{GoelSHFS15}~\cite{GoelSHFS15} & \href{works/GoelSHFS15.pdf}{Constraint programming for {LNG} ship scheduling and inventory management} &  &  & 0 &  &  &  &  &  &  & \ref{a:GoelSHFS15} & \ref{b:GoelSHFS15}\\
\rowlabel{c:Kameugne15}Kameugne15 \href{https://doi.org/10.1007/s10601-015-9227-5}{Kameugne15}~\cite{Kameugne15} & \href{works/Kameugne15.pdf}{Propagation techniques of resource constraint for cumulative scheduling} & - &  & 2 & - &  & - & \href{https://www.a4cp.org/sites/default/files/roger_kameugne_-_propagation_techniques_of_resource_constraint_for_cumulative_scheduling.pdf}{PhDThesis} & RCPSP &  & \ref{a:Kameugne15} & \ref{b:Kameugne15}\\
\rowlabel{c:LetortCB15}LetortCB15 \href{https://doi.org/10.1007/s10601-014-9172-8}{LetortCB15}~\cite{LetortCB15} & \href{works/LetortCB15.pdf}{Synchronized sweep algorithms for scalable scheduling constraints} & \su{Choco SICStus} & generated instance, Roadef, benchmark, random instance & 4 & dead &  & - & \cite{LetortCB13} & - & \su{cumulative dimCumulative dimCumulativePrecedences} & \ref{a:LetortCB15} & \ref{b:LetortCB15}\\
\rowlabel{c:NattafAL15}NattafAL15 \href{https://doi.org/10.1007/s10601-015-9192-z}{NattafAL15}~\cite{NattafAL15} & \href{works/NattafAL15.pdf}{A hybrid exact method for a scheduling problem with a continuous resource and energy constraints} & Cplex & generated instance & 1 & n &  & n &  & CSCSP &  & \ref{a:NattafAL15} & \ref{b:NattafAL15}\\
\rowlabel{c:Siala15}Siala15 \href{https://doi.org/10.1007/s10601-015-9213-y}{Siala15}~\cite{Siala15} & \href{works/Siala15.pdf}{Search, propagation, and learning in sequencing and scheduling problems} & - & benchmark & 2 & - &  & - & \href{https://www.a4cp.org/sites/default/files/mohamed_siala_-_search_propagation_and_learning_in_sequencing_and_scheduling_problems.pdf}{PhD Thesis} &  &  & \ref{a:Siala15} & \ref{b:Siala15}\\
\rowlabel{c:SimoninAHL15}SimoninAHL15 \href{https://doi.org/10.1007/s10601-014-9169-3}{SimoninAHL15}~\cite{SimoninAHL15} & \href{works/SimoninAHL15.pdf}{Scheduling scientific experiments for comet exploration} & \su{MOST Ilog Scheduler} &  & 0 & n &  & n & \cite{SimoninAHL12} &  & \su{cumulative dataTransfer} & \ref{a:SimoninAHL15} & \ref{b:SimoninAHL15}\\
\rowlabel{c:WangMD15}WangMD15 \href{https://doi.org/10.1016/j.ejor.2015.06.008}{WangMD15}~\cite{WangMD15} & \href{works/WangMD15.pdf}{Scheduling operating theatres: Mixed integer programming vs. constraint programming} &  & real-life, real-world & 2 &  &  &  &  &  &  & \ref{a:WangMD15} & \ref{b:WangMD15}\\
\rowlabel{c:BonfiettiLBM14}BonfiettiLBM14 \href{https://doi.org/10.1016/j.artint.2013.09.006}{BonfiettiLBM14}~\cite{BonfiettiLBM14} & \href{works/BonfiettiLBM14.pdf}{{CROSS} cyclic resource-constrained scheduling solver} &  & real-world, generated instance, industrial instance, benchmark & 0 &  &  &  &  &  &  & \ref{a:BonfiettiLBM14} & \ref{b:BonfiettiLBM14}\\
\rowlabel{c:GrimesIOS14}GrimesIOS14 \href{https://doi.org/10.1016/j.suscom.2014.08.009}{GrimesIOS14}~\cite{GrimesIOS14} & \href{works/GrimesIOS14.pdf}{Analyzing the impact of electricity price forecasting on energy cost-aware scheduling} &  & real-world, real-life & 9 &  &  &  &  &  &  & \ref{a:GrimesIOS14} & \ref{b:GrimesIOS14}\\
\rowlabel{c:KameugneFSN14}KameugneFSN14 \href{https://doi.org/10.1007/s10601-013-9157-z}{KameugneFSN14}~\cite{KameugneFSN14} & \href{works/KameugneFSN14.pdf}{A quadratic edge-finding filtering algorithm for cumulative resource constraints} & Gecode & random instance, benchmark & 2 & \href{https://figshare.com/articles/dataset/Comparison_of_edge_finding_and_extended_edge_finding_filtering_algorithms/736454}{y} &  &  & \cite{KameugneFSN11} & CuSP & cumulative & \ref{a:KameugneFSN14} & \ref{b:KameugneFSN14}\\
\rowlabel{c:NovasH14}NovasH14 \href{https://doi.org/10.1016/j.eswa.2013.09.026}{NovasH14}~\cite{NovasH14} & \href{works/NovasH14.pdf}{Integrated scheduling of resource-constrained flexible manufacturing systems using constraint programming} &  & benchmark & 0 &  &  &  &  &  &  & \ref{a:NovasH14} & \ref{b:NovasH14}\\
\rowlabel{c:BegB13}BegB13 \href{http://doi.acm.org/10.1145/2512470}{BegB13}~\cite{BegB13} & \href{works/BegB13.pdf}{A constraint programming approach for integrated spatial and temporal scheduling for clustered architectures} &  & benchmark & 0 &  &  &  &  &  &  & \ref{a:BegB13} & \ref{b:BegB13}\\
\rowlabel{c:HeinzSB13}HeinzSB13 \href{https://doi.org/10.1007/s10601-012-9136-9}{HeinzSB13}~\cite{HeinzSB13} & \href{works/HeinzSB13.pdf}{Using dual presolving reductions to reformulate cumulative constraints} & \su{Cplex SCIP} & benchmark & 1 & ref &  & - & - & \su{RCPSP RCPSP/max} & cumulative & \ref{a:HeinzSB13} & \ref{b:HeinzSB13}\\
\rowlabel{c:OzturkTHO13}OzturkTHO13 \href{https://doi.org/10.1007/s10601-013-9142-6}{OzturkTHO13}~\cite{OzturkTHO13} & \href{works/OzturkTHO13.pdf}{Balancing and scheduling of flexible mixed model assembly lines} & \su{{Ilog Solver} {Ilog Scheduler} Cplex} & real-world, real-life & 2 & \href{https://github.com/ozturkcemal/SBSFMMAL}{y} &  & - & - & SBSFMMAL & \su{alddifferent disjunctive} & \ref{a:OzturkTHO13} & \ref{b:OzturkTHO13}\\
\rowlabel{c:HeinzSSW12}HeinzSSW12 \href{https://doi.org/10.1007/s10601-011-9113-8}{HeinzSSW12}~\cite{HeinzSSW12} & \href{works/HeinzSSW12.pdf}{Solving steel mill slab design problems} &  & real-world, CSPlib & 2 & Cplex &  & dead & - & SMSDP & - & \ref{a:HeinzSSW12} & \ref{b:HeinzSSW12}\\
\rowlabel{c:LimtanyakulS12}LimtanyakulS12 \href{https://doi.org/10.1007/s10601-012-9118-y}{LimtanyakulS12}~\cite{LimtanyakulS12} & \href{works/LimtanyakulS12.pdf}{Improvements of constraint programming and hybrid methods for scheduling of tests on vehicle prototypes} & \su{Cplex {Ilog Scheduler}} & random instance, real-life, generated instance, industrial partner, benchmark & 1 & dead &  & - & - &  &  & \ref{a:LimtanyakulS12} & \ref{b:LimtanyakulS12}\\
\rowlabel{c:LombardiM12}LombardiM12 \href{https://doi.org/10.1007/s10601-011-9115-6}{LombardiM12}~\cite{LombardiM12} & \href{works/LombardiM12.pdf}{Optimal methods for resource allocation and scheduling: a cross-disciplinary survey} & - & real-world, benchmark & 0 & - &  & - & - & survey & - & \ref{a:LombardiM12} & \ref{b:LombardiM12}\\
\rowlabel{c:LombardiM12a}LombardiM12a \href{https://doi.org/10.1016/j.artint.2011.12.001}{LombardiM12a}~\cite{LombardiM12a} & \href{works/LombardiM12a.pdf}{A min-flow algorithm for Minimal Critical Set detection in Resource Constrained Project Scheduling} &  & benchmark & 1 &  &  &  &  &  &  & \ref{a:LombardiM12a} & \ref{b:LombardiM12a}\\
\rowlabel{c:NovasH12}NovasH12 \href{https://doi.org/10.1016/j.compchemeng.2012.01.005}{NovasH12}~\cite{NovasH12} & \href{works/NovasH12.pdf}{A comprehensive constraint programming approach for the rolling horizon-based scheduling of automated wet-etch stations} &  &  & 0 &  &  &  &  &  &  & \ref{a:NovasH12} & \ref{b:NovasH12}\\
\rowlabel{c:BartakS11}BartakS11 \href{https://doi.org/10.1007/s10601-011-9109-4}{BartakS11}~\cite{BartakS11} & \href{works/BartakS11.pdf}{Constraint satisfaction for planning and scheduling problems} & - & random instance, real-world, real-life & 2 & - &  & - &  & survey &  & \ref{a:BartakS11} & \ref{b:BartakS11}\\
\rowlabel{c:BeckFW11}BeckFW11 \href{https://doi.org/10.1287/ijoc.1100.0388}{BeckFW11}~\cite{BeckFW11} & \href{works/BeckFW11.pdf}{Combining Constraint Programming and Local Search for Job-Shop Scheduling} &  & real-world, benchmark & 0 &  &  &  &  &  &  & \ref{a:BeckFW11} & \ref{b:BeckFW11}\\
\rowlabel{c:BeldiceanuCDP11}BeldiceanuCDP11 \href{https://doi.org/10.1007/s10479-010-0731-0}{BeldiceanuCDP11}~\cite{BeldiceanuCDP11} & \href{works/BeldiceanuCDP11.pdf}{New filtering for the \emph{cumulative} constraint in the context of non-overlapping rectangles} &  & benchmark & 1 &  &  &  &  &  &  & \ref{a:BeldiceanuCDP11} & \ref{b:BeldiceanuCDP11}\\
\rowlabel{c:BeniniLMR11}BeniniLMR11 \href{https://doi.org/10.1007/s10479-010-0718-x}{BeniniLMR11}~\cite{BeniniLMR11} & \href{works/BeniniLMR11.pdf}{Optimal resource allocation and scheduling for the {CELL} {BE} platform} &  & benchmark, real-world, instance generator & 0 &  &  &  &  &  &  & \ref{a:BeniniLMR11} & \ref{b:BeniniLMR11}\\
\rowlabel{c:HachemiGR11}HachemiGR11 \href{https://doi.org/10.1007/s10479-010-0698-x}{HachemiGR11}~\cite{HachemiGR11} & \href{works/HachemiGR11.pdf}{A hybrid constraint programming approach to the log-truck scheduling problem} &  &  & 1 &  &  &  &  &  &  & \ref{a:HachemiGR11} & \ref{b:HachemiGR11}\\
\rowlabel{c:KelbelH11}KelbelH11 \href{https://doi.org/10.1007/s10845-009-0318-2}{KelbelH11}~\cite{KelbelH11} & \href{works/KelbelH11.pdf}{Solving production scheduling with earliness/tardiness penalties by constraint programming} &  & benchmark, random instance, generated instance & 3 &  &  &  &  &  &  & \ref{a:KelbelH11} & \ref{b:KelbelH11}\\
\rowlabel{c:KovacsB11}KovacsB11 \href{https://doi.org/10.1007/s10601-009-9088-x}{KovacsB11}~\cite{KovacsB11} & \href{works/KovacsB11.pdf}{A global constraint for total weighted completion time for unary resources} & Ilog Scheduler & benchmark & 2 & n &  & n & - &  & Completion & \ref{a:KovacsB11} & \ref{b:KovacsB11}\\
\rowlabel{c:KovacsK11}KovacsK11 \href{https://doi.org/10.1007/s10601-010-9102-3}{KovacsK11}~\cite{KovacsK11} & \href{works/KovacsK11.pdf}{Constraint programming approach to a bilevel scheduling problem} & Ilog Solver &  & 2 & n &  & n & - & Bilevel Opt &  & \ref{a:KovacsK11} & \ref{b:KovacsK11}\\
\rowlabel{c:SchausHMCMD11}SchausHMCMD11 \href{https://doi.org/10.1007/s10601-010-9100-5}{SchausHMCMD11}~\cite{SchausHMCMD11} & \href{works/SchausHMCMD11.pdf}{Solving Steel Mill Slab Problems with constraint-based techniques: CP, LNS, and {CBLS}} & Comet & benchmark, CSPlib, generated instance & 3 & dead &  &  &  & SMSDP &  & \ref{a:SchausHMCMD11} & \ref{b:SchausHMCMD11}\\
\rowlabel{c:SchuttFSW11}SchuttFSW11 \href{https://doi.org/10.1007/s10601-010-9103-2}{SchuttFSW11}~\cite{SchuttFSW11} & \href{works/SchuttFSW11.pdf}{Explaining the cumulative propagator} & MiniZinc & benchmark, real-world & 7 & PSPLib &  & - & - & RCPSP & cumulative & \ref{a:SchuttFSW11} & \ref{b:SchuttFSW11}\\
\rowlabel{c:TopalogluO11}TopalogluO11 \href{https://doi.org/10.1016/j.cor.2010.04.018}{TopalogluO11}~\cite{TopalogluO11} & \href{works/TopalogluO11.pdf}{A constraint programming-based solution approach for medical resident scheduling problems} &  & real-life & 2 &  &  &  &  &  &  & \ref{a:TopalogluO11} & \ref{b:TopalogluO11}\\
\rowlabel{c:TrojetHL11}TrojetHL11 \href{https://doi.org/10.1016/j.cie.2010.08.014}{TrojetHL11}~\cite{TrojetHL11} & \href{works/TrojetHL11.pdf}{Project scheduling under resource constraints: Application of the cumulative global constraint in a decision support framework} &  & real-world & 2 &  &  &  &  &  &  & \ref{a:TrojetHL11} & \ref{b:TrojetHL11}\\
\rowlabel{c:BartakCS10}BartakCS10 \href{https://doi.org/10.1007/s10479-008-0492-1}{BartakCS10}~\cite{BartakCS10} & \href{works/BartakCS10.pdf}{Discovering implied constraints in precedence graphs with alternatives} &  & benchmark, real-life, real-world & 3 &  &  &  &  &  &  & \ref{a:BartakCS10} & \ref{b:BartakCS10}\\
\rowlabel{c:BartakSR10}BartakSR10 \href{https://doi.org/10.1017/S0269888910000202}{BartakSR10}~\cite{BartakSR10} & \href{works/BartakSR10.pdf}{New trends in constraint satisfaction, planning, and scheduling: a survey} &  & real-life, real-world & 0 &  &  &  &  &  &  & \ref{a:BartakSR10} & \ref{b:BartakSR10}\\
\rowlabel{c:LombardiM10a}LombardiM10a \href{https://doi.org/10.1016/j.artint.2010.02.004}{LombardiM10a}~\cite{LombardiM10a} & \href{works/LombardiM10a.pdf}{Allocation and scheduling of Conditional Task Graphs} &  & real-world, benchmark, real-life & 3 &  &  &  &  &  &  & \ref{a:LombardiM10a} & \ref{b:LombardiM10a}\\
\rowlabel{c:LopesCSM10}LopesCSM10 \href{https://doi.org/10.1007/s10601-009-9086-z}{LopesCSM10}~\cite{LopesCSM10} & \href{works/LopesCSM10.pdf}{A hybrid model for a multiproduct pipeline planning and scheduling problem} & Ilog Solver & benchmark, real-world & 2 & - &  & - & \cite{MouraSCL08,MouraSCL08a} &  &  & \ref{a:LopesCSM10} & \ref{b:LopesCSM10}\\
\rowlabel{c:NovasH10}NovasH10 \href{https://doi.org/10.1016/j.compchemeng.2010.07.011}{NovasH10}~\cite{NovasH10} & \href{works/NovasH10.pdf}{Reactive scheduling framework based on domain knowledge and constraint programming} &  &  & 0 &  &  &  &  &  &  & \ref{a:NovasH10} & \ref{b:NovasH10}\\
\rowlabel{c:ZeballosQH10}ZeballosQH10 \href{https://doi.org/10.1016/j.engappai.2009.07.002}{ZeballosQH10}~\cite{ZeballosQH10} & \href{works/ZeballosQH10.pdf}{A constraint programming model for the scheduling of flexible manufacturing systems with machine and tool limitations} &  & benchmark, real-world & 4 &  &  &  &  &  &  & \ref{a:ZeballosQH10} & \ref{b:ZeballosQH10}\\
\rowlabel{c:BocewiczBB09}BocewiczBB09 \href{https://doi.org/10.1504/IJIIDS.2009.023038}{BocewiczBB09}~\cite{BocewiczBB09} & \href{works/BocewiczBB09.pdf}{Logic-algebraic method based and constraints programming driven approach to AGVs scheduling} &  &  & 0 &  &  &  &  &  &  & \ref{a:BocewiczBB09} & \ref{b:BocewiczBB09}\\
\rowlabel{c:GarridoAO09}GarridoAO09 \href{https://doi.org/10.1007/s10951-008-0083-7}{GarridoAO09}~\cite{GarridoAO09} & \href{works/GarridoAO09.pdf}{A constraint programming formulation for planning: from plan scheduling to plan generation} &  & benchmark & 8 &  &  &  &  &  &  & \ref{a:GarridoAO09} & \ref{b:GarridoAO09}\\
\rowlabel{c:RuggieroBBMA09}RuggieroBBMA09 \href{https://doi.org/10.1109/TCAD.2009.2013536}{RuggieroBBMA09}~\cite{RuggieroBBMA09} & \href{works/RuggieroBBMA09.pdf}{Reducing the Abstraction and Optimality Gaps in the Allocation and Scheduling for Variable Voltage/Frequency MPSoC Platforms} &  & instance generator, real-life & 0 &  &  &  &  &  &  & \ref{a:RuggieroBBMA09} & \ref{b:RuggieroBBMA09}\\
\rowlabel{c:abs-0907-0939}abs-0907-0939 \href{http://arxiv.org/abs/0907.0939}{abs-0907-0939}~\cite{abs-0907-0939} & \href{works/abs-0907-0939.pdf}{The Soft Cumulative Constraint} &  & real-world & 0 &  &  &  &  &  &  & \ref{a:abs-0907-0939} & \ref{b:abs-0907-0939}\\
\rowlabel{c:GarridoOS08}GarridoOS08 \href{https://doi.org/10.1016/j.engappai.2008.03.009}{GarridoOS08}~\cite{GarridoOS08} & \href{works/GarridoOS08.pdf}{Planning and scheduling in an e-learning environment. {A} constraint-programming-based approach} &  & real-world & 0 &  &  &  &  &  &  & \ref{a:GarridoOS08} & \ref{b:GarridoOS08}\\
\rowlabel{c:KovacsB08}KovacsB08 \href{https://doi.org/10.1016/j.engappai.2008.03.004}{KovacsB08}~\cite{KovacsB08} & \href{works/KovacsB08.pdf}{A global constraint for total weighted completion time for cumulative resources} &  & benchmark & 0 &  &  &  &  &  &  & \ref{a:KovacsB08} & \ref{b:KovacsB08}\\
\rowlabel{c:LiessM08}LiessM08 \href{https://doi.org/10.1007/s10479-007-0188-y}{LiessM08}~\cite{LiessM08} & \href{works/LiessM08.pdf}{A constraint programming approach for the resource-constrained project scheduling problem} &  & benchmark & 0 &  &  &  &  &  &  & \ref{a:LiessM08} & \ref{b:LiessM08}\\
\rowlabel{c:MalikMB08}MalikMB08 \href{https://doi.org/10.1142/S0218213008003765}{MalikMB08}~\cite{MalikMB08} & \href{works/MalikMB08.pdf}{Optimal Basic Block Instruction Scheduling for Multiple-Issue Processors Using Constraint Programming} &  & benchmark & 0 &  &  &  &  &  &  & \ref{a:MalikMB08} & \ref{b:MalikMB08}\\
\rowlabel{c:Rodriguez07}Rodriguez07 \href{https://www.sciencedirect.com/science/article/pii/S0191261506000233}{Rodriguez07}~\cite{Rodriguez07} & \href{works/Rodriguez07.pdf}{A constraint programming model for real-time train scheduling at junctions} &  & real-life & 2 &  &  &  &  &  &  & \ref{a:Rodriguez07} & \ref{b:Rodriguez07}\\
\rowlabel{c:Simonis07}Simonis07 \href{https://doi.org/10.1007/s10601-006-9011-7}{Simonis07}~\cite{Simonis07} & \href{works/Simonis07.pdf}{Models for Global Constraint Applications} & CHIP &  & 0 & n &  & n &  &  & \su{cumulative diffn cycle inverse} & \ref{a:Simonis07} & \ref{b:Simonis07}\\
\rowlabel{c:Hooker06}Hooker06 \href{https://doi.org/10.1007/s10601-006-8060-2}{Hooker06}~\cite{Hooker06} & \href{works/Hooker06.pdf}{An Integrated Method for Planning and Scheduling to Minimize Tardiness} & \su{OPL Cplex {Ilog Scheduler}} & random instance & 2 & n &  & n & \cite{Hooker05a} & CuSP & \su{cumulative} & \ref{a:Hooker06} & \ref{b:Hooker06}\\
\rowlabel{c:KhayatLR06}KhayatLR06 \href{https://doi.org/10.1016/j.ejor.2005.02.077}{KhayatLR06}~\cite{KhayatLR06} & \href{works/KhayatLR06.pdf}{Integrated production and material handling scheduling using mathematical programming and constraint programming} &  & real-life, benchmark & 1 &  &  &  &  &  &  & \ref{a:KhayatLR06} & \ref{b:KhayatLR06}\\
\rowlabel{c:SadykovW06}SadykovW06 \href{https://doi.org/10.1287/ijoc.1040.0110}{SadykovW06}~\cite{SadykovW06} & \href{works/SadykovW06.pdf}{Integer Programming and Constraint Programming in Solving a Multimachine Assignment Scheduling Problem with Deadlines and Release Dates} &  & generated instance & 1 &  &  &  &  &  &  & \ref{a:SadykovW06} & \ref{b:SadykovW06}\\
\rowlabel{c:SureshMOK06}SureshMOK06 \href{https://doi.org/10.1080/17445760600567842}{SureshMOK06}~\cite{SureshMOK06} & \href{works/SureshMOK06.pdf}{Divisible load scheduling in distributed system with buffer constraints: genetic algorithm and linear programming approach} &  &  & 0 &  &  &  &  &  &  & \ref{a:SureshMOK06} & \ref{b:SureshMOK06}\\
\rowlabel{c:Hooker05}Hooker05 \href{https://doi.org/10.1007/s10601-005-2812-2}{Hooker05}~\cite{Hooker05} & \href{works/Hooker05.pdf}{A Hybrid Method for the Planning and Scheduling} & \su{OPL Cplex {Ilog Scheduler}} & random instance & 0 & n &  & n & \cite{Hooker04} & CuSP & \su{cumulative} & \ref{a:Hooker05} & \ref{b:Hooker05}\\
\rowlabel{c:VilimBC05}VilimBC05 \href{https://doi.org/10.1007/s10601-005-2814-0}{VilimBC05}~\cite{VilimBC05} & \href{works/VilimBC05.pdf}{Extension of \emph{O}(\emph{n} log \emph{n}) Filtering Algorithms for the Unary Resource Constraint to Optional Activities} &  & benchmark, real-life & 0 & n &  & n & \cite{VilimBC04} & JSSP & disjunctive & \ref{a:VilimBC05} & \ref{b:VilimBC05}\\
\rowlabel{c:ZeballosH05}ZeballosH05 \href{http://journal.iberamia.org/index.php/ia/article/view/452/article\%20\%281\%29.pdf}{ZeballosH05}~\cite{ZeballosH05} & \href{works/ZeballosH05.pdf}{A Constraint Programming Approach to {FMS} Scheduling. Consideration of Storage and Transportation Resources} &  &  & 0 &  &  &  &  &  &  & \ref{a:ZeballosH05} & \ref{b:ZeballosH05}\\
\rowlabel{c:PoderBS04}PoderBS04 \href{https://doi.org/10.1016/S0377-2217(02)00756-7}{PoderBS04}~\cite{PoderBS04} & \href{works/PoderBS04.pdf}{Computing a lower approximation of the compulsory part of a task with varying duration and varying resource consumption} &  &  & 0 &  &  &  &  &  &  & \ref{a:PoderBS04} & \ref{b:PoderBS04}\\
\rowlabel{c:KuchcinskiW03}KuchcinskiW03 \href{https://doi.org/10.1016/S1383-7621(03)00075-4}{KuchcinskiW03}~\cite{KuchcinskiW03} & \href{works/KuchcinskiW03.pdf}{Global approach to assignment and scheduling of complex behaviors based on {HCDG} and constraint programming} &  & benchmark & 0 &  &  &  &  &  &  & \ref{a:KuchcinskiW03} & \ref{b:KuchcinskiW03}\\
\rowlabel{c:Tsang03}Tsang03 \href{https://doi.org/10.1023/A:1024016929283}{Tsang03}~\cite{Tsang03} & \href{works/Tsang03.pdf}{Constraint Based Scheduling: Applying Constraint Programming to Scheduling Problems} &  & real-life & 0 &  &  &  &  &  &  & \ref{a:Tsang03} & \ref{b:Tsang03}\\
\rowlabel{c:LorigeonBB02}LorigeonBB02 \href{https://doi.org/10.1057/palgrave.jors.2601421}{LorigeonBB02}~\cite{LorigeonBB02} & \href{works/LorigeonBB02.pdf}{A dynamic programming algorithm for scheduling jobs in a two-machine open shop with an availability constraint} &  &  & 0 &  &  &  &  &  &  & \ref{a:LorigeonBB02} & \ref{b:LorigeonBB02}\\
\rowlabel{c:RodriguezDG02}RodriguezDG02 \href{}{RodriguezDG02}~\cite{RodriguezDG02} & \href{works/RodriguezDG02.pdf}{Railway infrastructure saturation using constraint programming approach} &  &  & 0 &  &  &  &  &  &  & \ref{a:RodriguezDG02} & \ref{b:RodriguezDG02}\\
\rowlabel{c:Timpe02}Timpe02 \href{https://doi.org/10.1007/s00291-002-0107-1}{Timpe02}~\cite{Timpe02} & \href{works/Timpe02.pdf}{Solving planning and scheduling problems with combined integer and constraint programming} &  &  & 0 &  &  &  &  &  &  & \ref{a:Timpe02} & \ref{b:Timpe02}\\
\rowlabel{c:MartinPY01}MartinPY01 \href{https://doi.org/10.1023/A:1016067230126}{MartinPY01}~\cite{MartinPY01} & \href{works/MartinPY01.pdf}{Cane Railway Scheduling via Constraint Logic Programming: Labelling Order and Constraints in a Real-Life Application} &  & real-life & 0 &  &  &  &  &  &  & \ref{a:MartinPY01} & \ref{b:MartinPY01}\\
\rowlabel{c:Mason01}Mason01 \href{https://doi.org/10.1023/A:1016023415105}{Mason01}~\cite{Mason01} & \href{works/Mason01.pdf}{Elastic Constraint Branching, the Wedelin/Carmen Lagrangian Heuristic and Integer Programming for Personnel Scheduling} &  &  & 0 &  &  &  &  &  &  & \ref{a:Mason01} & \ref{b:Mason01}\\
\rowlabel{c:ArtiguesR00}ArtiguesR00 \href{https://doi.org/10.1016/S0377-2217(99)00496-8}{ArtiguesR00}~\cite{ArtiguesR00} & \href{works/ArtiguesR00.pdf}{A polynomial activity insertion algorithm in a multi-resource schedule with cumulative constraints and multiple modes} &  &  & 0 &  &  &  &  &  &  & \ref{a:ArtiguesR00} & \ref{b:ArtiguesR00}\\
\rowlabel{c:BaptisteP00}BaptisteP00 \href{https://doi.org/10.1023/A:1009822502231}{BaptisteP00}~\cite{BaptisteP00} & \href{works/BaptisteP00.pdf}{Constraint Propagation and Decomposition Techniques for Highly Disjunctive and Highly Cumulative Project Scheduling Problems} & CLAIRE & benchmark & 0 & n &  & n &  & RCCSP & cumulative & \ref{a:BaptisteP00} & \ref{b:BaptisteP00}\\
\rowlabel{c:HeipckeCCS00}HeipckeCCS00 \href{https://doi.org/10.1023/A:1009860311452}{HeipckeCCS00}~\cite{HeipckeCCS00} & \href{works/HeipckeCCS00.pdf}{Scheduling under Labour Resource Constraints} & \su{COME SchedEns} & benchmark, instance generator & 0 & dead &  & n & - &  &  & \ref{a:HeipckeCCS00} & \ref{b:HeipckeCCS00}\\
\rowlabel{c:KorbaaYG00}KorbaaYG00 \href{https://doi.org/10.1016/S0947-3580(00)71113-7}{KorbaaYG00}~\cite{KorbaaYG00} & \href{works/KorbaaYG00.pdf}{Solving Transient Scheduling Problems with Constraint Programming} &  &  & 0 &  &  &  &  &  &  & \ref{a:KorbaaYG00} & \ref{b:KorbaaYG00}\\
\rowlabel{c:LopezAKYG00}LopezAKYG00 \href{https://doi.org/10.1016/S0947-3580(00)71114-9}{LopezAKYG00}~\cite{LopezAKYG00} & \href{works/LopezAKYG00.pdf}{Discussion on: 'Solving Transient Scheduling Problems with Constraint Programming' by O. Korbaa, P. Yim, and {J.-C.} Gentina} &  &  & 0 &  &  &  &  &  &  & \ref{a:LopezAKYG00} & \ref{b:LopezAKYG00}\\
\rowlabel{c:SakkoutW00}SakkoutW00 \href{https://doi.org/10.1023/A:1009856210543}{SakkoutW00}~\cite{SakkoutW00} & \href{works/SakkoutW00.pdf}{Probe Backtrack Search for Minimal Perturbation in Dynamic Scheduling} & \su{Cplex ECLiPSe} & benchmark, real-world & 0 & n &  & n & - & KRFP &  & \ref{a:SakkoutW00} & \ref{b:SakkoutW00}\\
\rowlabel{c:SchildW00}SchildW00 \href{https://doi.org/10.1023/A:1009804226473}{SchildW00}~\cite{SchildW00} & \href{works/SchildW00.pdf}{Scheduling of Time-Triggered Real-Time Systems} & OZ &  & 0 & n &  & n & - &  & disjunctive & \ref{a:SchildW00} & \ref{b:SchildW00}\\
\rowlabel{c:SourdN00}SourdN00 \href{https://doi.org/10.1287/ijoc.12.4.341.11881}{SourdN00}~\cite{SourdN00} & \href{works/SourdN00.pdf}{Multiple-Machine Lower Bounds for Shop-Scheduling Problems} &  & real-life, benchmark & 1 &  &  &  &  &  &  & \ref{a:SourdN00} & \ref{b:SourdN00}\\
\rowlabel{c:BensanaLV99}BensanaLV99 \href{https://doi.org/10.1023/A:1026488509554}{BensanaLV99}~\cite{BensanaLV99} & \href{works/BensanaLV99.pdf}{Earth Observation Satellite Management} & Ilog Solver & benchmark & 0 & ? &  & - & - &  &  & \ref{a:BensanaLV99} & \ref{b:BensanaLV99}\\
\rowlabel{c:BelhadjiI98}BelhadjiI98 \href{https://doi.org/10.1023/A:1009777711218}{BelhadjiI98}~\cite{BelhadjiI98} & \href{works/BelhadjiI98.pdf}{Temporal Constraint Satisfaction Techniques in Job Shop Scheduling Problem Solving} & - & real-life & 0 & n &  & n & - & \su{TCSP JSSP} &  & \ref{a:BelhadjiI98} & \ref{b:BelhadjiI98}\\
\rowlabel{c:NuijtenP98}NuijtenP98 \href{https://doi.org/10.1023/A:1009687210594}{NuijtenP98}~\cite{NuijtenP98} & \href{works/NuijtenP98.pdf}{Constraint-Based Job Shop Scheduling with {\textbackslash}sc Ilog Scheduler} &  & real-life & 0 &  &  &  &  &  &  & \ref{a:NuijtenP98} & \ref{b:NuijtenP98}\\
\rowlabel{c:PapaB98}PapaB98 \href{https://doi.org/10.1023/A:1009723704757}{PapaB98}~\cite{PapaB98} & \href{works/PapaB98.pdf}{Resource Constraints for Preemptive Job-shop Scheduling} & \su{{Ilog Solver} Claire} & benchmark & 0 & dead &  & - & - & PJSSP & \su{disjunctive flow} & \ref{a:PapaB98} & \ref{b:PapaB98}\\
\rowlabel{c:Darby-DowmanLMZ97}Darby-DowmanLMZ97 \href{https://doi.org/10.1007/BF00137871}{Darby-DowmanLMZ97}~\cite{Darby-DowmanLMZ97} & \href{works/Darby-DowmanLMZ97.pdf}{Constraint Logic Programming and Integer Programming Approaches and Their Collaboration in Solving an Assignment Scheduling Problem} & \su{Cplex ECLiPSe} & real-life, real-world, benchmark & 0 & n &  & n & - & MGAP &  & \ref{a:Darby-DowmanLMZ97} & \ref{b:Darby-DowmanLMZ97}\\
\rowlabel{c:FalaschiGMP97}FalaschiGMP97 \href{https://doi.org/10.1006/inco.1997.2638}{FalaschiGMP97}~\cite{FalaschiGMP97} & \href{works/FalaschiGMP97.pdf}{Constraint Logic Programming with Dynamic Scheduling: {A} Semantics Based on Closure Operators} &  &  & 0 &  &  &  &  &  &  & \ref{a:FalaschiGMP97} & \ref{b:FalaschiGMP97}\\
\rowlabel{c:LammaMM97}LammaMM97 \href{https://doi.org/10.1016/S0954-1810(96)00002-7}{LammaMM97}~\cite{LammaMM97} & \href{works/LammaMM97.pdf}{A distributed constraint-based scheduler} &  & real-life & 0 &  &  &  &  &  &  & \ref{a:LammaMM97} & \ref{b:LammaMM97}\\
\rowlabel{c:Zhou97}Zhou97 \href{https://doi.org/10.1023/A:1009757726572}{Zhou97}~\cite{Zhou97} & \href{works/Zhou97.pdf}{A Permutation-Based Approach for Solving the Job-Shop Problem} & - & benchmark & 0 & n &  & n & \cite{Zhou96} & JSSP & \su{sort alldifferent permutation} & \ref{a:Zhou97} & \ref{b:Zhou97}\\
\rowlabel{c:Wallace96}Wallace96 \href{https://doi.org/10.1007/BF00143881}{Wallace96}~\cite{Wallace96} & \href{works/Wallace96.pdf}{Practical Applications of Constraint Programming} & - &  & 0 & - &  & - & - & Survey & - & \ref{a:Wallace96} & \ref{b:Wallace96}\\
\rowlabel{c:BeldiceanuC94}BeldiceanuC94 \href{https://www.sciencedirect.com/science/article/pii/0895717794901279}{BeldiceanuC94}~\cite{BeldiceanuC94} & \href{works/BeldiceanuC94.pdf}{Introducing Global Constraints in {CHIP}} &  & real-world, real-life, benchmark & 0 &  &  &  &  &  &  & \ref{a:BeldiceanuC94} & \ref{b:BeldiceanuC94}\\
\rowlabel{c:AggounB93}AggounB93 \href{https://www.sciencedirect.com/science/article/pii/089571779390068A}{AggounB93}~\cite{AggounB93} & \href{works/AggounB93.pdf}{Extending {CHIP} in order to solve complex scheduling and placement problems} &  & real-world & 0 &  &  &  &  &  &  & \ref{a:AggounB93} & \ref{b:AggounB93}\\
\rowlabel{c:Tay92}Tay92 \href{}{Tay92}~\cite{Tay92} & \href{}{{COPS:} {A} Constraint Programming Approach to Resource-Limited Project Scheduling} &  &  & 0 &  &  &  &  &  &  & \ref{a:Tay92} & No\\
\rowlabel{c:DincbasSH90}DincbasSH90 \href{https://doi.org/10.1016/0743-1066(90)90052-7}{DincbasSH90}~\cite{DincbasSH90} & \href{works/DincbasSH90.pdf}{Solving Large Combinatorial Problems in Logic Programming} &  & real-life & 0 &  &  &  &  &  &  & \ref{a:DincbasSH90} & \ref{b:DincbasSH90}\\
\end{longtable}
}



\clearpage
\section{Authors}

{\scriptsize
\begin{longtable}{p{4cm}rrp{18cm}}
\rowcolor{white}\caption{Co-Authors of Articles/Papers}\\ \toprule
\rowcolor{white}Author & \shortstack{Nr\\Works} & \shortstack{Nr\\Cites} & Entries \\ \midrule\endhead
\bottomrule
\endfoot
\rowlabel{auth:a89}J. Christopher Beck & 49 &701 &\href{works/LuoB22.pdf}{LuoB22}~\cite{LuoB22}, \href{works/ZhangBB22.pdf}{ZhangBB22}~\cite{ZhangBB22}, \href{works/TangB20.pdf}{TangB20}~\cite{TangB20}, \href{}{RoshanaeiBAUB20}~\cite{RoshanaeiBAUB20}, \href{works/TranPZLDB18.pdf}{TranPZLDB18}~\cite{TranPZLDB18}, \href{works/TranVNB17.pdf}{TranVNB17}~\cite{TranVNB17}, \href{works/TranVNB17a.pdf}{TranVNB17a}~\cite{TranVNB17a}, \href{works/CohenHB17.pdf}{CohenHB17}~\cite{CohenHB17}, \href{works/BoothNB16.pdf}{BoothNB16}~\cite{BoothNB16}, \href{works/KuB16.pdf}{KuB16}~\cite{KuB16}, \href{works/TranAB16.pdf}{TranAB16}~\cite{TranAB16}, \href{works/TranWDRFOVB16.pdf}{TranWDRFOVB16}~\cite{TranWDRFOVB16}, \href{works/LuoVLBM16.pdf}{LuoVLBM16}~\cite{LuoVLBM16}, \href{works/TranDRFWOVB16.pdf}{TranDRFWOVB16}~\cite{TranDRFWOVB16}, \href{works/BajestaniB15.pdf}{BajestaniB15}~\cite{BajestaniB15}, \href{works/KoschB14.pdf}{KoschB14}~\cite{KoschB14}, \href{works/TerekhovTDB14.pdf}{TerekhovTDB14}~\cite{TerekhovTDB14}, \href{works/LouieVNB14.pdf}{LouieVNB14}~\cite{LouieVNB14}, \href{works/HeinzSB13.pdf}{HeinzSB13}~\cite{HeinzSB13}, \href{works/HeinzKB13.pdf}{HeinzKB13}~\cite{HeinzKB13}, \href{works/BajestaniB13.pdf}{BajestaniB13}~\cite{BajestaniB13}, \href{works/TranTDB13.pdf}{TranTDB13}~\cite{TranTDB13}, \href{works/HeinzB12.pdf}{HeinzB12}~\cite{HeinzB12}, \href{works/TerekhovDOB12.pdf}{TerekhovDOB12}~\cite{TerekhovDOB12}, \href{works/TranB12.pdf}{TranB12}~\cite{TranB12}, \href{}{ZarandiB12}~\cite{ZarandiB12}, \href{works/KovacsB11.pdf}{KovacsB11}~\cite{KovacsB11}, \href{works/BeckFW11.pdf}{BeckFW11}~\cite{BeckFW11}, \href{works/HeckmanB11.pdf}{HeckmanB11}~\cite{HeckmanB11}, \href{works/BajestaniB11.pdf}{BajestaniB11}~\cite{BajestaniB11}, \href{works/WuBB09.pdf}{WuBB09}~\cite{WuBB09}, \href{works/BidotVLB09.pdf}{BidotVLB09}~\cite{BidotVLB09}, \href{works/CarchraeB09.pdf}{CarchraeB09}~\cite{CarchraeB09}, \href{works/WatsonB08.pdf}{WatsonB08}~\cite{WatsonB08}, \href{works/KovacsB08.pdf}{KovacsB08}~\cite{KovacsB08}, \href{works/BeckW07.pdf}{BeckW07}~\cite{BeckW07}, \href{works/Beck07.pdf}{Beck07}~\cite{Beck07}, \href{works/KovacsB07.pdf}{KovacsB07}~\cite{KovacsB07}, \href{works/Beck06.pdf}{Beck06}~\cite{Beck06}, \href{works/CarchraeBF05.pdf}{CarchraeBF05}~\cite{CarchraeBF05}, \href{works/WuBB05.pdf}{WuBB05}~\cite{WuBB05}, \href{works/BeckW05.pdf}{BeckW05}~\cite{BeckW05}, \href{works/BeckW04.pdf}{BeckW04}~\cite{BeckW04}, \href{works/BeckR03.pdf}{BeckR03}~\cite{BeckR03}, \href{works/BeckPS03.pdf}{BeckPS03}~\cite{BeckPS03}, \href{works/BeckF00.pdf}{BeckF00}~\cite{BeckF00}, \href{works/Beck99.pdf}{Beck99}~\cite{Beck99}, \href{works/BeckF98.pdf}{BeckF98}~\cite{BeckF98}, \href{works/BeckDF97.pdf}{BeckDF97}~\cite{BeckDF97}\\
\rowlabel{auth:a144}Michela Milano & 31 &297 &\href{works/BorghesiBLMB18.pdf}{BorghesiBLMB18}~\cite{BorghesiBLMB18}, \href{works/BonfiettiZLM16.pdf}{BonfiettiZLM16}~\cite{BonfiettiZLM16}, \href{works/BridiBLMB16.pdf}{BridiBLMB16}~\cite{BridiBLMB16}, \href{works/BridiLBBM16.pdf}{BridiLBBM16}~\cite{BridiLBBM16}, \href{works/LombardiBM15.pdf}{LombardiBM15}~\cite{LombardiBM15}, \href{works/BartoliniBBLM14.pdf}{BartoliniBBLM14}~\cite{BartoliniBBLM14}, \href{works/BonfiettiLM14.pdf}{BonfiettiLM14}~\cite{BonfiettiLM14}, \href{works/BonfiettiLBM14.pdf}{BonfiettiLBM14}~\cite{BonfiettiLBM14}, \href{works/BonfiettiLM13.pdf}{BonfiettiLM13}~\cite{BonfiettiLM13}, \href{works/LombardiM13.pdf}{LombardiM13}~\cite{LombardiM13}, \href{}{LombardiMB13}~\cite{LombardiMB13}, \href{works/LombardiM12.pdf}{LombardiM12}~\cite{LombardiM12}, \href{works/BonfiettiLBM12.pdf}{BonfiettiLBM12}~\cite{BonfiettiLBM12}, \href{works/LombardiM12a.pdf}{LombardiM12a}~\cite{LombardiM12a}, \href{works/BonfiettiM12.pdf}{BonfiettiM12}~\cite{BonfiettiM12}, \href{works/BonfiettiLBM11.pdf}{BonfiettiLBM11}~\cite{BonfiettiLBM11}, \href{works/LombardiBMB11.pdf}{LombardiBMB11}~\cite{LombardiBMB11}, \href{works/BeniniLMR11.pdf}{BeniniLMR11}~\cite{BeniniLMR11}, \href{}{Milano11}~\cite{Milano11}, \href{works/LombardiM10.pdf}{LombardiM10}~\cite{LombardiM10}, \href{works/LombardiM10a.pdf}{LombardiM10a}~\cite{LombardiM10a}, \href{works/LombardiMRB10.pdf}{LombardiMRB10}~\cite{LombardiMRB10}, \href{works/LombardiM09.pdf}{LombardiM09}~\cite{LombardiM09}, \href{works/RuggieroBBMA09.pdf}{RuggieroBBMA09}~\cite{RuggieroBBMA09}, \href{works/MilanoW09.pdf}{MilanoW09}~\cite{MilanoW09}, \href{works/BeniniLMR08.pdf}{BeniniLMR08}~\cite{BeniniLMR08}, \href{works/BeniniBGM06.pdf}{BeniniBGM06}~\cite{BeniniBGM06}, \href{works/MilanoW06.pdf}{MilanoW06}~\cite{MilanoW06}, \href{}{MilanoORT02}~\cite{MilanoORT02}, \href{works/LammaMM97.pdf}{LammaMM97}~\cite{LammaMM97}, \href{works/BrusoniCLMMT96.pdf}{BrusoniCLMMT96}~\cite{BrusoniCLMMT96}\\
\rowlabel{auth:a125}Andreas Schutt & 27 &322 &\href{works/YangSS19.pdf}{YangSS19}~\cite{YangSS19}, \href{works/KreterSSZ18.pdf}{KreterSSZ18}~\cite{KreterSSZ18}, \href{works/GoldwaserS18.pdf}{GoldwaserS18}~\cite{GoldwaserS18}, \href{works/MusliuSS18.pdf}{MusliuSS18}~\cite{MusliuSS18}, \href{works/KreterSS17.pdf}{KreterSS17}~\cite{KreterSS17}, \href{works/YoungFS17.pdf}{YoungFS17}~\cite{YoungFS17}, \href{works/GoldwaserS17.pdf}{GoldwaserS17}~\cite{GoldwaserS17}, \href{works/SchuttS16.pdf}{SchuttS16}~\cite{SchuttS16}, \href{works/SzerediS16.pdf}{SzerediS16}~\cite{SzerediS16}, \href{works/KreterSS15.pdf}{KreterSS15}~\cite{KreterSS15}, \href{works/EvenSH15.pdf}{EvenSH15}~\cite{EvenSH15}, \href{works/EvenSH15a.pdf}{EvenSH15a}~\cite{EvenSH15a}, \href{}{SchuttFSW15}~\cite{SchuttFSW15}, \href{works/ThiruvadyWGS14.pdf}{ThiruvadyWGS14}~\cite{ThiruvadyWGS14}, \href{}{GuSSWC14}~\cite{GuSSWC14}, \href{works/SchuttFS13.pdf}{SchuttFS13}~\cite{SchuttFS13}, \href{works/SchuttFS13a.pdf}{SchuttFS13a}~\cite{SchuttFS13a}, \href{works/GuSS13.pdf}{GuSS13}~\cite{GuSS13}, \href{works/SchuttFSW13.pdf}{SchuttFSW13}~\cite{SchuttFSW13}, \href{works/ChuGNSW13.pdf}{ChuGNSW13}~\cite{ChuGNSW13}, \href{works/SchuttCSW12.pdf}{SchuttCSW12}~\cite{SchuttCSW12}, \href{works/SchuttFSW11.pdf}{SchuttFSW11}~\cite{SchuttFSW11}, \href{works/Schutt11.pdf}{Schutt11}~\cite{Schutt11}, \href{works/SchuttW10.pdf}{SchuttW10}~\cite{SchuttW10}, \href{works/abs-1009-0347.pdf}{abs-1009-0347}~\cite{abs-1009-0347}, \href{works/SchuttFSW09.pdf}{SchuttFSW09}~\cite{SchuttFSW09}, \href{works/SchuttWS05.pdf}{SchuttWS05}~\cite{SchuttWS05}\\
\rowlabel{auth:a143}Michele Lombardi & 25 &194 &\href{works/BorghesiBLMB18.pdf}{BorghesiBLMB18}~\cite{BorghesiBLMB18}, \href{works/CauwelaertLS18.pdf}{CauwelaertLS18}~\cite{CauwelaertLS18}, \href{works/BonfiettiZLM16.pdf}{BonfiettiZLM16}~\cite{BonfiettiZLM16}, \href{works/BridiBLMB16.pdf}{BridiBLMB16}~\cite{BridiBLMB16}, \href{works/BridiLBBM16.pdf}{BridiLBBM16}~\cite{BridiLBBM16}, \href{works/LombardiBM15.pdf}{LombardiBM15}~\cite{LombardiBM15}, \href{works/BartoliniBBLM14.pdf}{BartoliniBBLM14}~\cite{BartoliniBBLM14}, \href{works/BonfiettiLM14.pdf}{BonfiettiLM14}~\cite{BonfiettiLM14}, \href{works/BonfiettiLBM14.pdf}{BonfiettiLBM14}~\cite{BonfiettiLBM14}, \href{works/BonfiettiLM13.pdf}{BonfiettiLM13}~\cite{BonfiettiLM13}, \href{works/LombardiM13.pdf}{LombardiM13}~\cite{LombardiM13}, \href{}{LombardiMB13}~\cite{LombardiMB13}, \href{works/LombardiM12.pdf}{LombardiM12}~\cite{LombardiM12}, \href{works/BonfiettiLBM12.pdf}{BonfiettiLBM12}~\cite{BonfiettiLBM12}, \href{works/LombardiM12a.pdf}{LombardiM12a}~\cite{LombardiM12a}, \href{works/BonfiettiLBM11.pdf}{BonfiettiLBM11}~\cite{BonfiettiLBM11}, \href{works/LombardiBMB11.pdf}{LombardiBMB11}~\cite{LombardiBMB11}, \href{works/BeniniLMR11.pdf}{BeniniLMR11}~\cite{BeniniLMR11}, \href{works/LombardiM10.pdf}{LombardiM10}~\cite{LombardiM10}, \href{works/LombardiM10a.pdf}{LombardiM10a}~\cite{LombardiM10a}, \href{works/Lombardi10.pdf}{Lombardi10}~\cite{Lombardi10}, \href{works/LombardiMRB10.pdf}{LombardiMRB10}~\cite{LombardiMRB10}, \href{works/LombardiM09.pdf}{LombardiM09}~\cite{LombardiM09}, \href{works/BeniniLMR08.pdf}{BeniniLMR08}~\cite{BeniniLMR08}, \href{works/HoeveGSL07.pdf}{HoeveGSL07}~\cite{HoeveGSL07}\\
\rowlabel{auth:a126}Peter J. Stuckey & 24 &453 &\href{works/YangSS19.pdf}{YangSS19}~\cite{YangSS19}, \href{works/DemirovicS18.pdf}{DemirovicS18}~\cite{DemirovicS18}, \href{works/KreterSSZ18.pdf}{KreterSSZ18}~\cite{KreterSSZ18}, \href{works/MusliuSS18.pdf}{MusliuSS18}~\cite{MusliuSS18}, \href{works/KreterSS17.pdf}{KreterSS17}~\cite{KreterSS17}, \href{works/SchuttS16.pdf}{SchuttS16}~\cite{SchuttS16}, \href{works/BlomPS16.pdf}{BlomPS16}~\cite{BlomPS16}, \href{works/KreterSS15.pdf}{KreterSS15}~\cite{KreterSS15}, \href{works/BurtLPS15.pdf}{BurtLPS15}~\cite{BurtLPS15}, \href{}{SchuttFSW15}~\cite{SchuttFSW15}, \href{works/BlomBPS14.pdf}{BlomBPS14}~\cite{BlomBPS14}, \href{works/LipovetzkyBPS14.pdf}{LipovetzkyBPS14}~\cite{LipovetzkyBPS14}, \href{}{GuSSWC14}~\cite{GuSSWC14}, \href{works/SchuttFS13.pdf}{SchuttFS13}~\cite{SchuttFS13}, \href{works/SchuttFS13a.pdf}{SchuttFS13a}~\cite{SchuttFS13a}, \href{works/GuSS13.pdf}{GuSS13}~\cite{GuSS13}, \href{works/SchuttFSW13.pdf}{SchuttFSW13}~\cite{SchuttFSW13}, \href{works/SchuttCSW12.pdf}{SchuttCSW12}~\cite{SchuttCSW12}, \href{works/GuSW12.pdf}{GuSW12}~\cite{GuSW12}, \href{works/SchuttFSW11.pdf}{SchuttFSW11}~\cite{SchuttFSW11}, \href{works/BandaSC11.pdf}{BandaSC11}~\cite{BandaSC11}, \href{works/abs-1009-0347.pdf}{abs-1009-0347}~\cite{abs-1009-0347}, \href{works/SchuttFSW09.pdf}{SchuttFSW09}~\cite{SchuttFSW09}, \href{works/OhrimenkoSC09.pdf}{OhrimenkoSC09}~\cite{OhrimenkoSC09}\\
\rowlabel{auth:a162}John N. Hooker & 19 &1316 &\href{}{ElciOH22}~\cite{ElciOH22}, \href{works/Hooker19.pdf}{Hooker19}~\cite{Hooker19}, \href{works/Hooker17.pdf}{Hooker17}~\cite{Hooker17}, \href{works/HookerH17.pdf}{HookerH17}~\cite{HookerH17}, \href{works/HechingH16.pdf}{HechingH16}~\cite{HechingH16}, \href{}{CireCH16}~\cite{CireCH16}, \href{}{HarjunkoskiMBC14}~\cite{HarjunkoskiMBC14}, \href{works/CireCH13.pdf}{CireCH13}~\cite{CireCH13}, \href{works/CobanH11.pdf}{CobanH11}~\cite{CobanH11}, \href{works/CobanH10.pdf}{CobanH10}~\cite{CobanH10}, \href{}{Hooker10}~\cite{Hooker10}, \href{works/Hooker07.pdf}{Hooker07}~\cite{Hooker07}, \href{works/Hooker06.pdf}{Hooker06}~\cite{Hooker06}, \href{works/Hooker05.pdf}{Hooker05}~\cite{Hooker05}, \href{works/Hooker05a.pdf}{Hooker05a}~\cite{Hooker05a}, \href{works/Hooker04.pdf}{Hooker04}~\cite{Hooker04}, \href{works/HookerO03.pdf}{HookerO03}~\cite{HookerO03}, \href{works/HookerY02.pdf}{HookerY02}~\cite{HookerY02}, \href{}{Hooker00}~\cite{Hooker00}\\
\rowlabel{auth:a1}Emmanuel Hebrard & 17 &71 &\href{works/JuvinHHL23.pdf}{JuvinHHL23}~\cite{JuvinHHL23}, \href{works/HebrardALLCMR22.pdf}{HebrardALLCMR22}~\cite{HebrardALLCMR22}, \href{works/AntuoriHHEN21.pdf}{AntuoriHHEN21}~\cite{AntuoriHHEN21}, \href{}{ArtiguesHQT21}~\cite{ArtiguesHQT21}, \href{works/GodetLHS20.pdf}{GodetLHS20}~\cite{GodetLHS20}, \href{works/AntuoriHHEN20.pdf}{AntuoriHHEN20}~\cite{AntuoriHHEN20}, \href{works/HebrardHJMPV16.pdf}{HebrardHJMPV16}~\cite{HebrardHJMPV16}, \href{works/SimoninAHL15.pdf}{SimoninAHL15}~\cite{SimoninAHL15}, \href{works/SialaAH15.pdf}{SialaAH15}~\cite{SialaAH15}, \href{works/GrimesH15.pdf}{GrimesH15}~\cite{GrimesH15}, \href{works/BessiereHMQW14.pdf}{BessiereHMQW14}~\cite{BessiereHMQW14}, \href{works/SimoninAHL12.pdf}{SimoninAHL12}~\cite{SimoninAHL12}, \href{works/BillautHL12.pdf}{BillautHL12}~\cite{BillautHL12}, \href{works/GrimesH11.pdf}{GrimesH11}~\cite{GrimesH11}, \href{works/GrimesH10.pdf}{GrimesH10}~\cite{GrimesH10}, \href{works/GrimesHM09.pdf}{GrimesHM09}~\cite{GrimesHM09}, \href{works/HebrardTW05.pdf}{HebrardTW05}~\cite{HebrardTW05}\\
\rowlabel{auth:a3}Pierre Lopez & 17 &90 &\href{works/JuvinHHL23.pdf}{JuvinHHL23}~\cite{JuvinHHL23}, \href{}{JuvinHL23a}~\cite{JuvinHL23a}, \href{works/JuvinHL23.pdf}{JuvinHL23}~\cite{JuvinHL23}, \href{works/HebrardALLCMR22.pdf}{HebrardALLCMR22}~\cite{HebrardALLCMR22}, \href{works/JuvinHL22.pdf}{JuvinHL22}~\cite{JuvinHL22}, \href{works/Polo-MejiaALB20.pdf}{Polo-MejiaALB20}~\cite{Polo-MejiaALB20}, \href{works/NattafHKAL19.pdf}{NattafHKAL19}~\cite{NattafHKAL19}, \href{works/NattafAL17.pdf}{NattafAL17}~\cite{NattafAL17}, \href{works/NattafALR16.pdf}{NattafALR16}~\cite{NattafALR16}, \href{works/SimoninAHL15.pdf}{SimoninAHL15}~\cite{SimoninAHL15}, \href{works/NattafAL15.pdf}{NattafAL15}~\cite{NattafAL15}, \href{works/SimoninAHL12.pdf}{SimoninAHL12}~\cite{SimoninAHL12}, \href{works/BillautHL12.pdf}{BillautHL12}~\cite{BillautHL12}, \href{works/LahimerLH11.pdf}{LahimerLH11}~\cite{LahimerLH11}, \href{works/TrojetHL11.pdf}{TrojetHL11}~\cite{TrojetHL11}, \href{works/LopezAKYG00.pdf}{LopezAKYG00}~\cite{LopezAKYG00}, \href{works/TorresL00.pdf}{TorresL00}~\cite{TorresL00}\\
\rowlabel{auth:a6}Christian Artigues & 16 &203 &\href{works/PovedaAA23.pdf}{PovedaAA23}~\cite{PovedaAA23}, \href{works/PohlAK22.pdf}{PohlAK22}~\cite{PohlAK22}, \href{works/HebrardALLCMR22.pdf}{HebrardALLCMR22}~\cite{HebrardALLCMR22}, \href{}{ArtiguesHQT21}~\cite{ArtiguesHQT21}, \href{works/Polo-MejiaALB20.pdf}{Polo-MejiaALB20}~\cite{Polo-MejiaALB20}, \href{works/NattafHKAL19.pdf}{NattafHKAL19}~\cite{NattafHKAL19}, \href{works/NattafAL17.pdf}{NattafAL17}~\cite{NattafAL17}, \href{works/NattafALR16.pdf}{NattafALR16}~\cite{NattafALR16}, \href{works/SimoninAHL15.pdf}{SimoninAHL15}~\cite{SimoninAHL15}, \href{works/NattafAL15.pdf}{NattafAL15}~\cite{NattafAL15}, \href{works/SialaAH15.pdf}{SialaAH15}~\cite{SialaAH15}, \href{works/SimoninAHL12.pdf}{SimoninAHL12}~\cite{SimoninAHL12}, \href{}{NeronABCDD06}~\cite{NeronABCDD06}, \href{}{DemasseyAM05}~\cite{DemasseyAM05}, \href{works/ArtiguesBF04.pdf}{ArtiguesBF04}~\cite{ArtiguesBF04}, \href{works/ArtiguesR00.pdf}{ArtiguesR00}~\cite{ArtiguesR00}\\
\rowlabel{auth:a148}Pierre Schaus & 15 &79 &\href{works/CauwelaertDS20.pdf}{CauwelaertDS20}~\cite{CauwelaertDS20}, \href{works/ThomasKS20.pdf}{ThomasKS20}~\cite{ThomasKS20}, \href{works/HoundjiSW19.pdf}{HoundjiSW19}~\cite{HoundjiSW19}, \href{works/CappartTSR18.pdf}{CappartTSR18}~\cite{CappartTSR18}, \href{works/CauwelaertLS18.pdf}{CauwelaertLS18}~\cite{CauwelaertLS18}, \href{works/CappartS17.pdf}{CappartS17}~\cite{CappartS17}, \href{works/CauwelaertDMS16.pdf}{CauwelaertDMS16}~\cite{CauwelaertDMS16}, \href{works/DejemeppeCS15.pdf}{DejemeppeCS15}~\cite{DejemeppeCS15}, \href{works/GayHLS15.pdf}{GayHLS15}~\cite{GayHLS15}, \href{works/GayHS15.pdf}{GayHS15}~\cite{GayHS15}, \href{works/GayHS15a.pdf}{GayHS15a}~\cite{GayHS15a}, \href{works/HoundjiSWD14.pdf}{HoundjiSWD14}~\cite{HoundjiSWD14}, \href{works/GaySS14.pdf}{GaySS14}~\cite{GaySS14}, \href{works/SchausHMCMD11.pdf}{SchausHMCMD11}~\cite{SchausHMCMD11}, \href{works/SchausD08.pdf}{SchausD08}~\cite{SchausD08}\\
\rowlabel{auth:a17}Helmut Simonis & 15 &154 &\href{works/ArmstrongGOS22.pdf}{ArmstrongGOS22}~\cite{ArmstrongGOS22}, \href{works/ArmstrongGOS21.pdf}{ArmstrongGOS21}~\cite{ArmstrongGOS21}, \href{works/AntunesABD20.pdf}{AntunesABD20}~\cite{AntunesABD20}, \href{works/AntunesABD18.pdf}{AntunesABD18}~\cite{AntunesABD18}, \href{works/HurleyOS16.pdf}{HurleyOS16}~\cite{HurleyOS16}, \href{works/GrimesIOS14.pdf}{GrimesIOS14}~\cite{GrimesIOS14}, \href{works/IfrimOS12.pdf}{IfrimOS12}~\cite{IfrimOS12}, \href{works/SimonisH11.pdf}{SimonisH11}~\cite{SimonisH11}, \href{works/Simonis07.pdf}{Simonis07}~\cite{Simonis07}, \href{works/SimonisCK00.pdf}{SimonisCK00}~\cite{SimonisCK00}, \href{works/Simonis99.pdf}{Simonis99}~\cite{Simonis99}, \href{works/SimonisC95.pdf}{SimonisC95}~\cite{SimonisC95}, \href{works/Simonis95.pdf}{Simonis95}~\cite{Simonis95}, \href{works/Simonis95a.pdf}{Simonis95a}~\cite{Simonis95a}, \href{works/DincbasSH90.pdf}{DincbasSH90}~\cite{DincbasSH90}\\
\rowlabel{auth:a129}Nicolas Beldiceanu & 13 &274 &\href{works/Madi-WambaLOBM17.pdf}{Madi-WambaLOBM17}~\cite{Madi-WambaLOBM17}, \href{works/Madi-WambaB16.pdf}{Madi-WambaB16}~\cite{Madi-WambaB16}, \href{works/LetortCB15.pdf}{LetortCB15}~\cite{LetortCB15}, \href{works/LetortCB13.pdf}{LetortCB13}~\cite{LetortCB13}, \href{works/LetortBC12.pdf}{LetortBC12}~\cite{LetortBC12}, \href{works/ClercqPBJ11.pdf}{ClercqPBJ11}~\cite{ClercqPBJ11}, \href{works/BeldiceanuCDP11.pdf}{BeldiceanuCDP11}~\cite{BeldiceanuCDP11}, \href{works/BeldiceanuCP08.pdf}{BeldiceanuCP08}~\cite{BeldiceanuCP08}, \href{works/PoderB08.pdf}{PoderB08}~\cite{PoderB08}, \href{works/BeldiceanuP07.pdf}{BeldiceanuP07}~\cite{BeldiceanuP07}, \href{works/PoderBS04.pdf}{PoderBS04}~\cite{PoderBS04}, \href{works/BeldiceanuC02.pdf}{BeldiceanuC02}~\cite{BeldiceanuC02}, \href{works/AggounB93.pdf}{AggounB93}~\cite{AggounB93}\\
\rowlabel{auth:a248}Luca Benini & 13 &146 &\href{works/BorghesiBLMB18.pdf}{BorghesiBLMB18}~\cite{BorghesiBLMB18}, \href{works/BridiBLMB16.pdf}{BridiBLMB16}~\cite{BridiBLMB16}, \href{works/BridiLBBM16.pdf}{BridiLBBM16}~\cite{BridiLBBM16}, \href{works/BonfiettiLBM14.pdf}{BonfiettiLBM14}~\cite{BonfiettiLBM14}, \href{}{LombardiMB13}~\cite{LombardiMB13}, \href{works/BonfiettiLBM12.pdf}{BonfiettiLBM12}~\cite{BonfiettiLBM12}, \href{works/BonfiettiLBM11.pdf}{BonfiettiLBM11}~\cite{BonfiettiLBM11}, \href{works/LombardiBMB11.pdf}{LombardiBMB11}~\cite{LombardiBMB11}, \href{works/BeniniLMR11.pdf}{BeniniLMR11}~\cite{BeniniLMR11}, \href{works/LombardiMRB10.pdf}{LombardiMRB10}~\cite{LombardiMRB10}, \href{works/RuggieroBBMA09.pdf}{RuggieroBBMA09}~\cite{RuggieroBBMA09}, \href{works/BeniniLMR08.pdf}{BeniniLMR08}~\cite{BeniniLMR08}, \href{works/BeniniBGM06.pdf}{BeniniBGM06}~\cite{BeniniBGM06}\\
\rowlabel{auth:a118}Philippe Laborie & 12 &513 &\href{works/LunardiBLRV20.pdf}{LunardiBLRV20}~\cite{LunardiBLRV20}, \href{works/LaborieRSV18.pdf}{LaborieRSV18}~\cite{LaborieRSV18}, \href{works/Laborie18a.pdf}{Laborie18a}~\cite{Laborie18a}, \href{works/MelgarejoLS15.pdf}{MelgarejoLS15}~\cite{MelgarejoLS15}, \href{works/VilimLS15.pdf}{VilimLS15}~\cite{VilimLS15}, \href{works/Laborie09.pdf}{Laborie09}~\cite{Laborie09}, \href{works/BidotVLB09.pdf}{BidotVLB09}~\cite{BidotVLB09}, \href{}{BaptisteLPN06}~\cite{BaptisteLPN06}, \href{}{NeronABCDD06}~\cite{NeronABCDD06}, \href{works/GodardLN05.pdf}{GodardLN05}~\cite{GodardLN05}, \href{works/Laborie03.pdf}{Laborie03}~\cite{Laborie03}, \href{works/FocacciLN00.pdf}{FocacciLN00}~\cite{FocacciLN00}\\
\rowlabel{auth:a164}Philippe Baptiste & 11 &403 &\href{works/BaptisteB18.pdf}{BaptisteB18}~\cite{BaptisteB18}, \href{works/Baptiste09.pdf}{Baptiste09}~\cite{Baptiste09}, \href{}{BaptisteLPN06}~\cite{BaptisteLPN06}, \href{}{NeronABCDD06}~\cite{NeronABCDD06}, \href{works/ArtiouchineB05.pdf}{ArtiouchineB05}~\cite{ArtiouchineB05}, \href{works/Baptiste02.pdf}{Baptiste02}~\cite{Baptiste02}, \href{}{BaptistePN01}~\cite{BaptistePN01}, \href{works/BaptisteP00.pdf}{BaptisteP00}~\cite{BaptisteP00}, \href{works/PapaB98.pdf}{PapaB98}~\cite{PapaB98}, \href{works/BaptisteP97.pdf}{BaptisteP97}~\cite{BaptisteP97}, \href{}{PapeB97}~\cite{PapeB97}\\
\rowlabel{auth:a153}Roman Bart{\'{a}}k & 11 &88 &\href{works/SvancaraB22.pdf}{SvancaraB22}~\cite{SvancaraB22}, \href{works/JelinekB16.pdf}{JelinekB16}~\cite{JelinekB16}, \href{works/BartakV15.pdf}{BartakV15}~\cite{BartakV15}, \href{}{Bartak14}~\cite{Bartak14}, \href{works/BartakS11.pdf}{BartakS11}~\cite{BartakS11}, \href{works/BartakCS10.pdf}{BartakCS10}~\cite{BartakCS10}, \href{works/BartakSR10.pdf}{BartakSR10}~\cite{BartakSR10}, \href{works/VilimBC05.pdf}{VilimBC05}~\cite{VilimBC05}, \href{works/VilimBC04.pdf}{VilimBC04}~\cite{VilimBC04}, \href{works/Bartak02.pdf}{Bartak02}~\cite{Bartak02}, \href{works/Bartak02a.pdf}{Bartak02a}~\cite{Bartak02a}\\
\rowlabel{auth:a121}Petr Vil{\'{\i}}m & 11 &313 &\href{works/LaborieRSV18.pdf}{LaborieRSV18}~\cite{LaborieRSV18}, \href{works/VilimLS15.pdf}{VilimLS15}~\cite{VilimLS15}, \href{works/Vilim11.pdf}{Vilim11}~\cite{Vilim11}, \href{works/Vilim09.pdf}{Vilim09}~\cite{Vilim09}, \href{works/Vilim09a.pdf}{Vilim09a}~\cite{Vilim09a}, \href{works/VilimBC05.pdf}{VilimBC05}~\cite{VilimBC05}, \href{works/Vilim05.pdf}{Vilim05}~\cite{Vilim05}, \href{works/VilimBC04.pdf}{VilimBC04}~\cite{VilimBC04}, \href{works/Vilim04.pdf}{Vilim04}~\cite{Vilim04}, \href{works/Vilim03.pdf}{Vilim03}~\cite{Vilim03}, \href{works/Vilim02.pdf}{Vilim02}~\cite{Vilim02}\\
\rowlabel{auth:a117}Mark Wallace & 11 &296 &\href{works/WallaceY20.pdf}{WallaceY20}~\cite{WallaceY20}, \href{works/He0GLW18.pdf}{He0GLW18}~\cite{He0GLW18}, \href{works/ThiruvadyWGS14.pdf}{ThiruvadyWGS14}~\cite{ThiruvadyWGS14}, \href{works/SchuttFSW09.pdf}{SchuttFSW09}~\cite{SchuttFSW09}, \href{works/MilanoW09.pdf}{MilanoW09}~\cite{MilanoW09}, \href{works/MilanoW06.pdf}{MilanoW06}~\cite{MilanoW06}, \href{works/Wallace06.pdf}{Wallace06}~\cite{Wallace06}, \href{works/SakkoutW00.pdf}{SakkoutW00}~\cite{SakkoutW00}, \href{works/RodosekW98.pdf}{RodosekW98}~\cite{RodosekW98}, \href{works/Wallace96.pdf}{Wallace96}~\cite{Wallace96}, \href{}{Wallace94}~\cite{Wallace94}\\
\rowlabel{auth:a204}Alessio Bonfietti & 10 &17 &\href{works/BonfiettiZLM16.pdf}{BonfiettiZLM16}~\cite{BonfiettiZLM16}, \href{works/Bonfietti16.pdf}{Bonfietti16}~\cite{Bonfietti16}, \href{works/LombardiBM15.pdf}{LombardiBM15}~\cite{LombardiBM15}, \href{works/BonfiettiLM14.pdf}{BonfiettiLM14}~\cite{BonfiettiLM14}, \href{works/BonfiettiLBM14.pdf}{BonfiettiLBM14}~\cite{BonfiettiLBM14}, \href{works/BonfiettiLM13.pdf}{BonfiettiLM13}~\cite{BonfiettiLM13}, \href{works/BonfiettiLBM12.pdf}{BonfiettiLBM12}~\cite{BonfiettiLBM12}, \href{works/BonfiettiM12.pdf}{BonfiettiM12}~\cite{BonfiettiM12}, \href{works/BonfiettiLBM11.pdf}{BonfiettiLBM11}~\cite{BonfiettiLBM11}, \href{works/LombardiBMB11.pdf}{LombardiBMB11}~\cite{LombardiBMB11}\\
\rowlabel{auth:a149}Pascal Van Hentenryck & 10 &164 &\href{works/FontaineMH16.pdf}{FontaineMH16}~\cite{FontaineMH16}, \href{works/EvenSH15.pdf}{EvenSH15}~\cite{EvenSH15}, \href{works/EvenSH15a.pdf}{EvenSH15a}~\cite{EvenSH15a}, \href{works/SchausHMCMD11.pdf}{SchausHMCMD11}~\cite{SchausHMCMD11}, \href{works/MonetteDH09.pdf}{MonetteDH09}~\cite{MonetteDH09}, \href{works/DoomsH08.pdf}{DoomsH08}~\cite{DoomsH08}, \href{works/HentenryckM08.pdf}{HentenryckM08}~\cite{HentenryckM08}, \href{works/MercierH08.pdf}{MercierH08}~\cite{MercierH08}, \href{works/HentenryckM04.pdf}{HentenryckM04}~\cite{HentenryckM04}, \href{works/DincbasSH90.pdf}{DincbasSH90}~\cite{DincbasSH90}\\
\rowlabel{auth:a165}Claude Le Pape & 9 &536 &\href{}{BaptisteLPN06}~\cite{BaptisteLPN06}, \href{}{DannaP04}~\cite{DannaP04}, \href{}{BaptistePN01}~\cite{BaptistePN01}, \href{works/BaptisteP00.pdf}{BaptisteP00}~\cite{BaptisteP00}, \href{works/PapaB98.pdf}{PapaB98}~\cite{PapaB98}, \href{works/NuijtenP98.pdf}{NuijtenP98}~\cite{NuijtenP98}, \href{works/BaptisteP97.pdf}{BaptisteP97}~\cite{BaptisteP97}, \href{}{PapeB97}~\cite{PapeB97}, \href{}{Pape94}~\cite{Pape94}\\
\rowlabel{auth:a45}Nysret Musliu & 9 &14 &\href{works/LacknerMMWW23.pdf}{LacknerMMWW23}~\cite{LacknerMMWW23}, \href{works/WinterMMW22.pdf}{WinterMMW22}~\cite{WinterMMW22}, \href{works/LacknerMMWW21.pdf}{LacknerMMWW21}~\cite{LacknerMMWW21}, \href{works/GeibingerKKMMW21.pdf}{GeibingerKKMMW21}~\cite{GeibingerKKMMW21}, \href{works/GeibingerMM21.pdf}{GeibingerMM21}~\cite{GeibingerMM21}, \href{works/GeibingerMM19.pdf}{GeibingerMM19}~\cite{GeibingerMM19}, \href{works/abs-1911-04766.pdf}{abs-1911-04766}~\cite{abs-1911-04766}, \href{works/MusliuSS18.pdf}{MusliuSS18}~\cite{MusliuSS18}, \href{works/KletzanderM17.pdf}{KletzanderM17}~\cite{KletzanderM17}\\
\rowlabel{auth:a81}Margaux Nattaf & 9 &49 &\href{}{PenzDN23}~\cite{PenzDN23}, \href{works/NattafM20.pdf}{NattafM20}~\cite{NattafM20}, \href{works/MalapertN19.pdf}{MalapertN19}~\cite{MalapertN19}, \href{}{NattafDYW19}~\cite{NattafDYW19}, \href{works/NattafHKAL19.pdf}{NattafHKAL19}~\cite{NattafHKAL19}, \href{works/NattafAL17.pdf}{NattafAL17}~\cite{NattafAL17}, \href{works/Nattaf16.pdf}{Nattaf16}~\cite{Nattaf16}, \href{works/NattafALR16.pdf}{NattafALR16}~\cite{NattafALR16}, \href{works/NattafAL15.pdf}{NattafAL15}~\cite{NattafAL15}\\
\rowlabel{auth:a37}Claude{-}Guy Quimper & 9 &25 &\href{works/BoudreaultSLQ22.pdf}{BoudreaultSLQ22}~\cite{BoudreaultSLQ22}, \href{works/OuelletQ22.pdf}{OuelletQ22}~\cite{OuelletQ22}, \href{works/Mercier-AubinGQ20.pdf}{Mercier-AubinGQ20}~\cite{Mercier-AubinGQ20}, \href{works/FahimiOQ18.pdf}{FahimiOQ18}~\cite{FahimiOQ18}, \href{works/KameugneFGOQ18.pdf}{KameugneFGOQ18}~\cite{KameugneFGOQ18}, \href{works/OuelletQ18.pdf}{OuelletQ18}~\cite{OuelletQ18}, \href{works/GingrasQ16.pdf}{GingrasQ16}~\cite{GingrasQ16}, \href{works/BessiereHMQW14.pdf}{BessiereHMQW14}~\cite{BessiereHMQW14}, \href{works/OuelletQ13.pdf}{OuelletQ13}~\cite{OuelletQ13}\\
\rowlabel{auth:a811}Tony T. Tran & 9 &108 &\href{works/TranPZLDB18.pdf}{TranPZLDB18}~\cite{TranPZLDB18}, \href{works/TranVNB17.pdf}{TranVNB17}~\cite{TranVNB17}, \href{works/TranVNB17a.pdf}{TranVNB17a}~\cite{TranVNB17a}, \href{works/TranAB16.pdf}{TranAB16}~\cite{TranAB16}, \href{works/TranWDRFOVB16.pdf}{TranWDRFOVB16}~\cite{TranWDRFOVB16}, \href{works/TranDRFWOVB16.pdf}{TranDRFWOVB16}~\cite{TranDRFWOVB16}, \href{works/TerekhovTDB14.pdf}{TerekhovTDB14}~\cite{TerekhovTDB14}, \href{works/TranTDB13.pdf}{TranTDB13}~\cite{TranTDB13}, \href{works/TranB12.pdf}{TranB12}~\cite{TranB12}\\
\rowlabel{auth:a91}Mats Carlsson & 8 &80 &\href{works/WessenCS20.pdf}{WessenCS20}~\cite{WessenCS20}, \href{works/MossigeGSMC17.pdf}{MossigeGSMC17}~\cite{MossigeGSMC17}, \href{works/LetortCB15.pdf}{LetortCB15}~\cite{LetortCB15}, \href{works/LetortCB13.pdf}{LetortCB13}~\cite{LetortCB13}, \href{works/LetortBC12.pdf}{LetortBC12}~\cite{LetortBC12}, \href{works/BeldiceanuCDP11.pdf}{BeldiceanuCDP11}~\cite{BeldiceanuCDP11}, \href{works/BeldiceanuCP08.pdf}{BeldiceanuCP08}~\cite{BeldiceanuCP08}, \href{works/BeldiceanuC02.pdf}{BeldiceanuC02}~\cite{BeldiceanuC02}\\
\rowlabel{auth:a155}Thibaut Feydy & 8 &173 &\href{works/YoungFS17.pdf}{YoungFS17}~\cite{YoungFS17}, \href{}{SchuttFSW15}~\cite{SchuttFSW15}, \href{works/SchuttFS13.pdf}{SchuttFS13}~\cite{SchuttFS13}, \href{works/SchuttFS13a.pdf}{SchuttFS13a}~\cite{SchuttFS13a}, \href{works/SchuttFSW13.pdf}{SchuttFSW13}~\cite{SchuttFSW13}, \href{works/SchuttFSW11.pdf}{SchuttFSW11}~\cite{SchuttFSW11}, \href{works/abs-1009-0347.pdf}{abs-1009-0347}~\cite{abs-1009-0347}, \href{works/SchuttFSW09.pdf}{SchuttFSW09}~\cite{SchuttFSW09}\\
\rowlabel{auth:a156}Mark G. Wallace & 8 &135 &\href{}{SchuttFSW15}~\cite{SchuttFSW15}, \href{}{GuSSWC14}~\cite{GuSSWC14}, \href{works/SchuttFSW13.pdf}{SchuttFSW13}~\cite{SchuttFSW13}, \href{works/SchuttCSW12.pdf}{SchuttCSW12}~\cite{SchuttCSW12}, \href{works/GuSW12.pdf}{GuSW12}~\cite{GuSW12}, \href{works/SchuttFSW11.pdf}{SchuttFSW11}~\cite{SchuttFSW11}, \href{works/abs-1009-0347.pdf}{abs-1009-0347}~\cite{abs-1009-0347}, \href{}{AjiliW04}~\cite{AjiliW04}\\
\rowlabel{auth:a332}Louis{-}Martin Rousseau & 8 &126 &\href{works/CappartTSR18.pdf}{CappartTSR18}~\cite{CappartTSR18}, \href{works/DoulabiRP16.pdf}{DoulabiRP16}~\cite{DoulabiRP16}, \href{works/PesantRR15.pdf}{PesantRR15}~\cite{PesantRR15}, \href{works/DoulabiRP14.pdf}{DoulabiRP14}~\cite{DoulabiRP14}, \href{works/MalapertCGJLR13.pdf}{MalapertCGJLR13}~\cite{MalapertCGJLR13}, \href{}{MalapertCGJLR12}~\cite{MalapertCGJLR12}, \href{works/ChapadosJR11.pdf}{ChapadosJR11}~\cite{ChapadosJR11}, \href{works/HachemiGR11.pdf}{HachemiGR11}~\cite{HachemiGR11}\\
\rowlabel{auth:a51}Armin Wolf & 8 &46 &\href{works/GeitzGSSW22.pdf}{GeitzGSSW22}~\cite{GeitzGSSW22}, \href{works/Wolf11.pdf}{Wolf11}~\cite{Wolf11}, \href{works/SchuttW10.pdf}{SchuttW10}~\cite{SchuttW10}, \href{works/Wolf09.pdf}{Wolf09}~\cite{Wolf09}, \href{works/WolfS05.pdf}{WolfS05}~\cite{WolfS05}, \href{works/SchuttWS05.pdf}{SchuttWS05}~\cite{SchuttWS05}, \href{works/Wolf05.pdf}{Wolf05}~\cite{Wolf05}, \href{works/Wolf03.pdf}{Wolf03}~\cite{Wolf03}\\
\rowlabel{auth:a183}Diarmuid Grimes & 7 &52 &\href{works/AntunesABD20.pdf}{AntunesABD20}~\cite{AntunesABD20}, \href{works/AntunesABD18.pdf}{AntunesABD18}~\cite{AntunesABD18}, \href{works/GrimesH15.pdf}{GrimesH15}~\cite{GrimesH15}, \href{works/GrimesIOS14.pdf}{GrimesIOS14}~\cite{GrimesIOS14}, \href{works/GrimesH11.pdf}{GrimesH11}~\cite{GrimesH11}, \href{works/GrimesH10.pdf}{GrimesH10}~\cite{GrimesH10}, \href{works/GrimesHM09.pdf}{GrimesHM09}~\cite{GrimesHM09}\\
\rowlabel{auth:a116}Zdenek Hanz{\'{a}}lek & 7 &27 &\href{works/Mehdizadeh-Somarin23.pdf}{Mehdizadeh-Somarin23}~\cite{Mehdizadeh-Somarin23}, \href{works/abs-2305-19888.pdf}{abs-2305-19888}~\cite{abs-2305-19888}, \href{works/HeinzNVH22.pdf}{HeinzNVH22}~\cite{HeinzNVH22}, \href{works/VlkHT21.pdf}{VlkHT21}~\cite{VlkHT21}, \href{works/BenediktMH20.pdf}{BenediktMH20}~\cite{BenediktMH20}, \href{works/BenediktSMVH18.pdf}{BenediktSMVH18}~\cite{BenediktSMVH18}, \href{works/KelbelH11.pdf}{KelbelH11}~\cite{KelbelH11}\\
\rowlabel{auth:a10}Roger Kameugne & 7 &14 &\href{works/KameugneFND23.pdf}{KameugneFND23}~\cite{KameugneFND23}, \href{works/ThomasKS20.pdf}{ThomasKS20}~\cite{ThomasKS20}, \href{works/KameugneFGOQ18.pdf}{KameugneFGOQ18}~\cite{KameugneFGOQ18}, \href{works/Kameugne15.pdf}{Kameugne15}~\cite{Kameugne15}, \href{works/KameugneFSN14.pdf}{KameugneFSN14}~\cite{KameugneFSN14}, \href{works/Kameugne14.pdf}{Kameugne14}~\cite{Kameugne14}, \href{works/KameugneFSN11.pdf}{KameugneFSN11}~\cite{KameugneFSN11}\\
\rowlabel{auth:a147}Andr{\'{a}}s Kov{\'{a}}cs & 7 &21 &\href{works/KovacsB11.pdf}{KovacsB11}~\cite{KovacsB11}, \href{works/KovacsK11.pdf}{KovacsK11}~\cite{KovacsK11}, \href{works/KovacsB08.pdf}{KovacsB08}~\cite{KovacsB08}, \href{works/KovacsB07.pdf}{KovacsB07}~\cite{KovacsB07}, \href{works/KovacsV06.pdf}{KovacsV06}~\cite{KovacsV06}, \href{works/KovacsEKV05.pdf}{KovacsEKV05}~\cite{KovacsEKV05}, \href{works/KovacsV04.pdf}{KovacsV04}~\cite{KovacsV04}\\
\rowlabel{auth:a16}Barry O'Sullivan & 7 &14 &\href{works/ArmstrongGOS22.pdf}{ArmstrongGOS22}~\cite{ArmstrongGOS22}, \href{works/ArmstrongGOS21.pdf}{ArmstrongGOS21}~\cite{ArmstrongGOS21}, \href{works/AntunesABD20.pdf}{AntunesABD20}~\cite{AntunesABD20}, \href{works/AntunesABD18.pdf}{AntunesABD18}~\cite{AntunesABD18}, \href{works/HurleyOS16.pdf}{HurleyOS16}~\cite{HurleyOS16}, \href{works/GrimesIOS14.pdf}{GrimesIOS14}~\cite{GrimesIOS14}, \href{works/IfrimOS12.pdf}{IfrimOS12}~\cite{IfrimOS12}\\
\rowlabel{auth:a598}Gabriela P. Henning & 7 &153 &\href{works/NovaraNH16.pdf}{NovaraNH16}~\cite{NovaraNH16}, \href{works/NovasH14.pdf}{NovasH14}~\cite{NovasH14}, \href{works/NovasH12.pdf}{NovasH12}~\cite{NovasH12}, \href{works/NovasH10.pdf}{NovasH10}~\cite{NovasH10}, \href{works/ZeballosQH10.pdf}{ZeballosQH10}~\cite{ZeballosQH10}, \href{works/ZeballosH05.pdf}{ZeballosH05}~\cite{ZeballosH05}, \href{works/QuirogaZH05.pdf}{QuirogaZH05}~\cite{QuirogaZH05}\\
\rowlabel{auth:a152}Yves Deville & 6 &19 &\href{works/HoundjiSWD14.pdf}{HoundjiSWD14}~\cite{HoundjiSWD14}, \href{works/DejemeppeD14.pdf}{DejemeppeD14}~\cite{DejemeppeD14}, \href{works/SchausHMCMD11.pdf}{SchausHMCMD11}~\cite{SchausHMCMD11}, \href{works/MonetteDH09.pdf}{MonetteDH09}~\cite{MonetteDH09}, \href{works/SchausD08.pdf}{SchausD08}~\cite{SchausD08}, \href{works/MonetteDD07.pdf}{MonetteDD07}~\cite{MonetteDD07}\\
\rowlabel{auth:a134}Stefan Heinz & 6 &67 &\href{works/HeinzSB13.pdf}{HeinzSB13}~\cite{HeinzSB13}, \href{works/HeinzKB13.pdf}{HeinzKB13}~\cite{HeinzKB13}, \href{works/HeinzSSW12.pdf}{HeinzSSW12}~\cite{HeinzSSW12}, \href{works/HeinzB12.pdf}{HeinzB12}~\cite{HeinzB12}, \href{works/HeinzS11.pdf}{HeinzS11}~\cite{HeinzS11}, \href{works/BertholdHLMS10.pdf}{BertholdHLMS10}~\cite{BertholdHLMS10}\\
\rowlabel{auth:a82}Arnaud Malapert & 6 &39 &\href{works/NattafM20.pdf}{NattafM20}~\cite{NattafM20}, \href{works/MalapertN19.pdf}{MalapertN19}~\cite{MalapertN19}, \href{works/MalapertCGJLR13.pdf}{MalapertCGJLR13}~\cite{MalapertCGJLR13}, \href{}{MalapertCGJLR12}~\cite{MalapertCGJLR12}, \href{works/Malapert11.pdf}{Malapert11}~\cite{Malapert11}, \href{works/GrimesHM09.pdf}{GrimesHM09}~\cite{GrimesHM09}\\
\rowlabel{auth:a666}Wim Nuijten & 6 &375 &\href{}{BaptisteLPN06}~\cite{BaptisteLPN06}, \href{works/GodardLN05.pdf}{GodardLN05}~\cite{GodardLN05}, \href{}{BaptistePN01}~\cite{BaptistePN01}, \href{works/SourdN00.pdf}{SourdN00}~\cite{SourdN00}, \href{works/FocacciLN00.pdf}{FocacciLN00}~\cite{FocacciLN00}, \href{works/NuijtenP98.pdf}{NuijtenP98}~\cite{NuijtenP98}\\
\rowlabel{auth:a445}Erwin Pesch & 6 &417 &\href{works/MullerMKP22.pdf}{MullerMKP22}~\cite{MullerMKP22}, \href{}{BlazewiczEP19}~\cite{BlazewiczEP19}, \href{}{DomdorfPH03}~\cite{DomdorfPH03}, \href{}{DorndorfPH99}~\cite{DorndorfPH99}, \href{}{DorndorfHP99}~\cite{DorndorfHP99}, \href{}{BlazewiczDP96}~\cite{BlazewiczDP96}\\
\rowlabel{auth:a364}Emmanuel Poder & 6 &27 &\href{works/BeldiceanuCDP11.pdf}{BeldiceanuCDP11}~\cite{BeldiceanuCDP11}, \href{works/abs-0907-0939.pdf}{abs-0907-0939}~\cite{abs-0907-0939}, \href{works/BeldiceanuCP08.pdf}{BeldiceanuCP08}~\cite{BeldiceanuCP08}, \href{works/PoderB08.pdf}{PoderB08}~\cite{PoderB08}, \href{works/BeldiceanuP07.pdf}{BeldiceanuP07}~\cite{BeldiceanuP07}, \href{works/PoderBS04.pdf}{PoderBS04}~\cite{PoderBS04}\\
\rowlabel{auth:a737}Vahid Roshanaei & 6 &168 &\href{works/NaderiRR23.pdf}{NaderiRR23}~\cite{NaderiRR23}, \href{}{NaderiR22}~\cite{NaderiR22}, \href{}{NaderiRBAU21}~\cite{NaderiRBAU21}, \href{}{RoshanaeiBAUB20}~\cite{RoshanaeiBAUB20}, \href{}{RoshanaeiLAU17}~\cite{RoshanaeiLAU17}, \href{}{RoshanaeiLAU17a}~\cite{RoshanaeiLAU17a}\\
\rowlabel{auth:a208}Cyrille Dejemeppe & 5 &8 &\href{works/CauwelaertDS20.pdf}{CauwelaertDS20}~\cite{CauwelaertDS20}, \href{works/CauwelaertDMS16.pdf}{CauwelaertDMS16}~\cite{CauwelaertDMS16}, \href{works/Dejemeppe16.pdf}{Dejemeppe16}~\cite{Dejemeppe16}, \href{works/DejemeppeCS15.pdf}{DejemeppeCS15}~\cite{DejemeppeCS15}, \href{works/DejemeppeD14.pdf}{DejemeppeD14}~\cite{DejemeppeD14}\\
\rowlabel{auth:a246}Sophie Demassey & 5 &82 &\href{works/HermenierDL11.pdf}{HermenierDL11}~\cite{HermenierDL11}, \href{works/BeldiceanuCDP11.pdf}{BeldiceanuCDP11}~\cite{BeldiceanuCDP11}, \href{}{NeronABCDD06}~\cite{NeronABCDD06}, \href{}{DemasseyAM05}~\cite{DemasseyAM05}, \href{works/Demassey03.pdf}{Demassey03}~\cite{Demassey03}\\
\rowlabel{auth:a388}Ignacio E. Grossmann & 5 &844 &\href{}{HarjunkoskiMBC14}~\cite{HarjunkoskiMBC14}, \href{}{CastroGR10}~\cite{CastroGR10}, \href{works/MaraveliasG04.pdf}{MaraveliasG04}~\cite{MaraveliasG04}, \href{works/HarjunkoskiG02.pdf}{HarjunkoskiG02}~\cite{HarjunkoskiG02}, \href{works/JainG01.pdf}{JainG01}~\cite{JainG01}\\
\rowlabel{auth:a342}Hanyu Gu & 5 &39 &\href{works/EtminaniesfahaniGNMS22.pdf}{EtminaniesfahaniGNMS22}~\cite{EtminaniesfahaniGNMS22}, \href{works/ThiruvadyWGS14.pdf}{ThiruvadyWGS14}~\cite{ThiruvadyWGS14}, \href{}{GuSSWC14}~\cite{GuSSWC14}, \href{works/GuSS13.pdf}{GuSS13}~\cite{GuSS13}, \href{works/GuSW12.pdf}{GuSW12}~\cite{GuSW12}\\
\rowlabel{auth:a250}Narendra Jussien & 5 &36 &\href{works/MalapertCGJLR13.pdf}{MalapertCGJLR13}~\cite{MalapertCGJLR13}, \href{}{MalapertCGJLR12}~\cite{MalapertCGJLR12}, \href{works/ClercqPBJ11.pdf}{ClercqPBJ11}~\cite{ClercqPBJ11}, \href{works/ElkhyariGJ02.pdf}{ElkhyariGJ02}~\cite{ElkhyariGJ02}, \href{works/ElkhyariGJ02a.pdf}{ElkhyariGJ02a}~\cite{ElkhyariGJ02a}\\
\rowlabel{auth:a531}Juan M. Novas & 5 &148 &\href{works/Novas19.pdf}{Novas19}~\cite{Novas19}, \href{works/NovaraNH16.pdf}{NovaraNH16}~\cite{NovaraNH16}, \href{works/NovasH14.pdf}{NovasH14}~\cite{NovasH14}, \href{works/NovasH12.pdf}{NovasH12}~\cite{NovasH12}, \href{works/NovasH10.pdf}{NovasH10}~\cite{NovasH10}\\
\rowlabel{auth:a223}Kenneth N. Brown & 5 &44 &\href{works/AntunesABD20.pdf}{AntunesABD20}~\cite{AntunesABD20}, \href{works/AntunesABD18.pdf}{AntunesABD18}~\cite{AntunesABD18}, \href{works/MurphyMB15.pdf}{MurphyMB15}~\cite{MurphyMB15}, \href{works/WuBB09.pdf}{WuBB09}~\cite{WuBB09}, \href{works/WuBB05.pdf}{WuBB05}~\cite{WuBB05}\\
\rowlabel{auth:a735}Bahman Naderi & 5 &32 &\href{works/NaderiRR23.pdf}{NaderiRR23}~\cite{NaderiRR23}, \href{works/NaderiBZ22.pdf}{NaderiBZ22}~\cite{NaderiBZ22}, \href{}{NaderiBZ22a}~\cite{NaderiBZ22a}, \href{}{NaderiR22}~\cite{NaderiR22}, \href{}{NaderiRBAU21}~\cite{NaderiRBAU21}\\
\rowlabel{auth:a130}Mohamed Siala & 5 &9 &\href{works/AntunesABD20.pdf}{AntunesABD20}~\cite{AntunesABD20}, \href{works/AntunesABD18.pdf}{AntunesABD18}~\cite{AntunesABD18}, \href{works/Siala15.pdf}{Siala15}~\cite{Siala15}, \href{works/SialaAH15.pdf}{SialaAH15}~\cite{SialaAH15}, \href{works/Siala15a.pdf}{Siala15a}~\cite{Siala15a}\\
\rowlabel{auth:a314}Marek Vlk & 5 &14 &\href{works/abs-2305-19888.pdf}{abs-2305-19888}~\cite{abs-2305-19888}, \href{works/HeinzNVH22.pdf}{HeinzNVH22}~\cite{HeinzNVH22}, \href{works/VlkHT21.pdf}{VlkHT21}~\cite{VlkHT21}, \href{works/BenediktSMVH18.pdf}{BenediktSMVH18}~\cite{BenediktSMVH18}, \href{works/BartakV15.pdf}{BartakV15}~\cite{BartakV15}\\
\rowlabel{auth:a838}Nic Wilson & 5 &28 &\href{works/AntunesABD20.pdf}{AntunesABD20}~\cite{AntunesABD20}, \href{works/AntunesABD18.pdf}{AntunesABD18}~\cite{AntunesABD18}, \href{works/BeckW07.pdf}{BeckW07}~\cite{BeckW07}, \href{works/BeckW05.pdf}{BeckW05}~\cite{BeckW05}, \href{works/BeckW04.pdf}{BeckW04}~\cite{BeckW04}\\
\rowlabel{auth:a159}Andr{\'{e}} A. Cir{\'{e}} & 4 &50 &\href{works/CireCH13.pdf}{CireCH13}~\cite{CireCH13}, \href{works/LopesCSM10.pdf}{LopesCSM10}~\cite{LopesCSM10}, \href{works/MouraSCL08.pdf}{MouraSCL08}~\cite{MouraSCL08}, \href{works/MouraSCL08a.pdf}{MouraSCL08a}~\cite{MouraSCL08a}\\
\rowlabel{auth:a231}Andrea Bartolini & 4 &40 &\href{works/BorghesiBLMB18.pdf}{BorghesiBLMB18}~\cite{BorghesiBLMB18}, \href{works/BridiBLMB16.pdf}{BridiBLMB16}~\cite{BridiBLMB16}, \href{works/BridiLBBM16.pdf}{BridiLBBM16}~\cite{BridiLBBM16}, \href{works/BartoliniBBLM14.pdf}{BartoliniBBLM14}~\cite{BartoliniBBLM14}\\
\rowlabel{auth:a349}Geoffrey Chu & 4 &47 &\href{}{GuSSWC14}~\cite{GuSSWC14}, \href{works/ChuGNSW13.pdf}{ChuGNSW13}~\cite{ChuGNSW13}, \href{works/SchuttCSW12.pdf}{SchuttCSW12}~\cite{SchuttCSW12}, \href{works/BandaSC11.pdf}{BandaSC11}~\cite{BandaSC11}\\
\rowlabel{auth:a341}Elvin Coban & 4 &41 &\href{}{CireCH16}~\cite{CireCH16}, \href{works/CireCH13.pdf}{CireCH13}~\cite{CireCH13}, \href{works/CobanH11.pdf}{CobanH11}~\cite{CobanH11}, \href{works/CobanH10.pdf}{CobanH10}~\cite{CobanH10}\\
\rowlabel{auth:a217}Steven Gay & 4 &42 &\href{works/GayHLS15.pdf}{GayHLS15}~\cite{GayHLS15}, \href{works/GayHS15.pdf}{GayHS15}~\cite{GayHS15}, \href{works/GayHS15a.pdf}{GayHS15a}~\cite{GayHS15a}, \href{works/GaySS14.pdf}{GaySS14}~\cite{GaySS14}\\
\rowlabel{auth:a77}Tobias Geibinger & 4 &6 &\href{works/GeibingerKKMMW21.pdf}{GeibingerKKMMW21}~\cite{GeibingerKKMMW21}, \href{works/GeibingerMM21.pdf}{GeibingerMM21}~\cite{GeibingerMM21}, \href{works/GeibingerMM19.pdf}{GeibingerMM19}~\cite{GeibingerMM19}, \href{works/abs-1911-04766.pdf}{abs-1911-04766}~\cite{abs-1911-04766}\\
\rowlabel{auth:a296}Christelle Gu{\'{e}}ret & 4 &33 &\href{works/MalapertCGJLR13.pdf}{MalapertCGJLR13}~\cite{MalapertCGJLR13}, \href{}{MalapertCGJLR12}~\cite{MalapertCGJLR12}, \href{works/ElkhyariGJ02.pdf}{ElkhyariGJ02}~\cite{ElkhyariGJ02}, \href{works/ElkhyariGJ02a.pdf}{ElkhyariGJ02a}~\cite{ElkhyariGJ02a}\\
\rowlabel{auth:a2}Laurent Houssin & 4 &0 &\href{works/JuvinHHL23.pdf}{JuvinHHL23}~\cite{JuvinHHL23}, \href{}{JuvinHL23a}~\cite{JuvinHL23a}, \href{works/JuvinHL23.pdf}{JuvinHL23}~\cite{JuvinHL23}, \href{works/JuvinHL22.pdf}{JuvinHL22}~\cite{JuvinHL22}\\
\rowlabel{auth:a0}Carla Juvin & 4 &0 &\href{works/JuvinHHL23.pdf}{JuvinHHL23}~\cite{JuvinHHL23}, \href{}{JuvinHL23a}~\cite{JuvinHL23a}, \href{works/JuvinHL23.pdf}{JuvinHL23}~\cite{JuvinHL23}, \href{works/JuvinHL22.pdf}{JuvinHL22}~\cite{JuvinHL22}\\
\rowlabel{auth:a157}Tam{\'{a}}s Kis & 4 &11 &\href{works/NattafHKAL19.pdf}{NattafHKAL19}~\cite{NattafHKAL19}, \href{works/KovacsK11.pdf}{KovacsK11}~\cite{KovacsK11}, \href{works/KeriK07.pdf}{KeriK07}~\cite{KeriK07}, \href{works/KovacsEKV05.pdf}{KovacsEKV05}~\cite{KovacsEKV05}\\
\rowlabel{auth:a128}Arnaud Letort & 4 &23 &\href{works/LetortCB15.pdf}{LetortCB15}~\cite{LetortCB15}, \href{works/LetortCB13.pdf}{LetortCB13}~\cite{LetortCB13}, \href{works/Letort13.pdf}{Letort13}~\cite{Letort13}, \href{works/LetortBC12.pdf}{LetortBC12}~\cite{LetortBC12}\\
\rowlabel{auth:a913}Dionne M. Aleman & 4 &161 &\href{}{NaderiRBAU21}~\cite{NaderiRBAU21}, \href{}{RoshanaeiBAUB20}~\cite{RoshanaeiBAUB20}, \href{}{RoshanaeiLAU17}~\cite{RoshanaeiLAU17}, \href{}{RoshanaeiLAU17a}~\cite{RoshanaeiLAU17a}\\
\rowlabel{auth:a32}Laurent Michel & 4 &39 &\href{works/TardivoDFMP23.pdf}{TardivoDFMP23}~\cite{TardivoDFMP23}, \href{works/SchausHMCMD11.pdf}{SchausHMCMD11}~\cite{SchausHMCMD11}, \href{works/HentenryckM08.pdf}{HentenryckM08}~\cite{HentenryckM08}, \href{works/HentenryckM04.pdf}{HentenryckM04}~\cite{HentenryckM04}\\
\rowlabel{auth:a80}Florian Mischek & 4 &6 &\href{works/GeibingerKKMMW21.pdf}{GeibingerKKMMW21}~\cite{GeibingerKKMMW21}, \href{works/GeibingerMM21.pdf}{GeibingerMM21}~\cite{GeibingerMM21}, \href{works/GeibingerMM19.pdf}{GeibingerMM19}~\cite{GeibingerMM19}, \href{works/abs-1911-04766.pdf}{abs-1911-04766}~\cite{abs-1911-04766}\\
\rowlabel{auth:a150}Jean{-}No{\"{e}}l Monette & 4 &15 &\href{works/CauwelaertDMS16.pdf}{CauwelaertDMS16}~\cite{CauwelaertDMS16}, \href{works/SchausHMCMD11.pdf}{SchausHMCMD11}~\cite{SchausHMCMD11}, \href{works/MonetteDH09.pdf}{MonetteDH09}~\cite{MonetteDH09}, \href{works/MonetteDD07.pdf}{MonetteDD07}~\cite{MonetteDD07}\\
\rowlabel{auth:a210}Goldie Nejat & 4 &50 &\href{works/TranVNB17.pdf}{TranVNB17}~\cite{TranVNB17}, \href{works/TranVNB17a.pdf}{TranVNB17a}~\cite{TranVNB17a}, \href{works/BoothNB16.pdf}{BoothNB16}~\cite{BoothNB16}, \href{works/LouieVNB14.pdf}{LouieVNB14}~\cite{LouieVNB14}\\
\rowlabel{auth:a52}Yanick Ouellet & 4 &10 &\href{works/OuelletQ22.pdf}{OuelletQ22}~\cite{OuelletQ22}, \href{works/FahimiOQ18.pdf}{FahimiOQ18}~\cite{FahimiOQ18}, \href{works/KameugneFGOQ18.pdf}{KameugneFGOQ18}~\cite{KameugneFGOQ18}, \href{works/OuelletQ18.pdf}{OuelletQ18}~\cite{OuelletQ18}\\
\rowlabel{auth:a8}Gilles Pesant & 4 &60 &\href{works/AalianPG23.pdf}{AalianPG23}~\cite{AalianPG23}, \href{works/DoulabiRP16.pdf}{DoulabiRP16}~\cite{DoulabiRP16}, \href{works/PesantRR15.pdf}{PesantRR15}~\cite{PesantRR15}, \href{works/DoulabiRP14.pdf}{DoulabiRP14}~\cite{DoulabiRP14}\\
\rowlabel{auth:a227}Thierry Petit & 4 &20 &\href{works/DerrienP14.pdf}{DerrienP14}~\cite{DerrienP14}, \href{works/DerrienPZ14.pdf}{DerrienPZ14}~\cite{DerrienPZ14}, \href{works/ClercqPBJ11.pdf}{ClercqPBJ11}~\cite{ClercqPBJ11}, \href{works/abs-0907-0939.pdf}{abs-0907-0939}~\cite{abs-0907-0939}\\
\rowlabel{auth:a21}C{\'{e}}dric Pralet & 4 &10 &\href{works/SquillaciPR23.pdf}{SquillaciPR23}~\cite{SquillaciPR23}, \href{works/Pralet17.pdf}{Pralet17}~\cite{Pralet17}, \href{works/HebrardHJMPV16.pdf}{HebrardHJMPV16}~\cite{HebrardHJMPV16}, \href{works/PraletLJ15.pdf}{PraletLJ15}~\cite{PraletLJ15}\\
\rowlabel{auth:a328}Adrian R. Pearce & 4 &35 &\href{works/BlomPS16.pdf}{BlomPS16}~\cite{BlomPS16}, \href{works/BurtLPS15.pdf}{BurtLPS15}~\cite{BurtLPS15}, \href{works/BlomBPS14.pdf}{BlomBPS14}~\cite{BlomBPS14}, \href{works/LipovetzkyBPS14.pdf}{LipovetzkyBPS14}~\cite{LipovetzkyBPS14}\\
\rowlabel{auth:a402}Dhananjay R. Thiruvady & 4 &32 &\href{works/abs-2402-00459.pdf}{abs-2402-00459}~\cite{abs-2402-00459}, \href{works/abs-2211-14492.pdf}{abs-2211-14492}~\cite{abs-2211-14492}, \href{works/ThiruvadyWGS14.pdf}{ThiruvadyWGS14}~\cite{ThiruvadyWGS14}, \href{works/ThiruvadyBME09.pdf}{ThiruvadyBME09}~\cite{ThiruvadyBME09}\\
\rowlabel{auth:a727}Martino Ruggiero & 4 &58 &\href{works/BeniniLMR11.pdf}{BeniniLMR11}~\cite{BeniniLMR11}, \href{works/LombardiMRB10.pdf}{LombardiMRB10}~\cite{LombardiMRB10}, \href{works/RuggieroBBMA09.pdf}{RuggieroBBMA09}~\cite{RuggieroBBMA09}, \href{works/BeniniLMR08.pdf}{BeniniLMR08}~\cite{BeniniLMR08}\\
\rowlabel{auth:a85}Christine Solnon & 4 &20 &\href{works/GroleazNS20.pdf}{GroleazNS20}~\cite{GroleazNS20}, \href{works/GroleazNS20a.pdf}{GroleazNS20a}~\cite{GroleazNS20a}, \href{works/SacramentoSP20.pdf}{SacramentoSP20}~\cite{SacramentoSP20}, \href{works/MelgarejoLS15.pdf}{MelgarejoLS15}~\cite{MelgarejoLS15}\\
\rowlabel{auth:a830}Daria Terekhov & 4 &21 &\href{works/TanT18.pdf}{TanT18}~\cite{TanT18}, \href{works/TerekhovTDB14.pdf}{TerekhovTDB14}~\cite{TerekhovTDB14}, \href{works/TranTDB13.pdf}{TranTDB13}~\cite{TranTDB13}, \href{works/TerekhovDOB12.pdf}{TerekhovDOB12}~\cite{TerekhovDOB12}\\
\rowlabel{auth:a281}J{\'{o}}zsef V{\'{a}}ncza & 4 &9 &\href{works/KovacsV06.pdf}{KovacsV06}~\cite{KovacsV06}, \href{works/KovacsEKV05.pdf}{KovacsEKV05}~\cite{KovacsEKV05}, \href{works/KovacsV04.pdf}{KovacsV04}~\cite{KovacsV04}, \href{works/VanczaM01.pdf}{VanczaM01}~\cite{VanczaM01}\\
\rowlabel{auth:a279}Toby Walsh & 4 &2 &\href{works/GelainPRVW17.pdf}{GelainPRVW17}~\cite{GelainPRVW17}, \href{works/BessiereHMQW14.pdf}{BessiereHMQW14}~\cite{BessiereHMQW14}, \href{works/ChuGNSW13.pdf}{ChuGNSW13}~\cite{ChuGNSW13}, \href{works/HebrardTW05.pdf}{HebrardTW05}~\cite{HebrardTW05}\\
\rowlabel{auth:a43}Felix Winter & 4 &0 &\href{works/LacknerMMWW23.pdf}{LacknerMMWW23}~\cite{LacknerMMWW23}, \href{works/WinterMMW22.pdf}{WinterMMW22}~\cite{WinterMMW22}, \href{works/LacknerMMWW21.pdf}{LacknerMMWW21}~\cite{LacknerMMWW21}, \href{works/GeibingerKKMMW21.pdf}{GeibingerKKMMW21}~\cite{GeibingerKKMMW21}\\
\rowlabel{auth:a411}Francisco Yuraszeck & 4 &31 &\href{works/YuraszeckMCCR23.pdf}{YuraszeckMCCR23}~\cite{YuraszeckMCCR23}, \href{works/YuraszeckMC23.pdf}{YuraszeckMC23}~\cite{YuraszeckMC23}, \href{works/YuraszeckMPV22.pdf}{YuraszeckMPV22}~\cite{YuraszeckMPV22}, \href{works/MejiaY20.pdf}{MejiaY20}~\cite{MejiaY20}\\
\rowlabel{auth:a212}Willem{-}Jan van Hoeve & 4 &50 &\href{works/GilesH16.pdf}{GilesH16}~\cite{GilesH16}, \href{works/GoelSHFS15.pdf}{GoelSHFS15}~\cite{GoelSHFS15}, \href{works/HoeveGSL07.pdf}{HoeveGSL07}~\cite{HoeveGSL07}, \href{works/GomesHS06.pdf}{GomesHS06}~\cite{GomesHS06}\\
\rowlabel{auth:a74}Max {\AA}strand & 4 &27 &\href{works/Astrand0F21.pdf}{Astrand0F21}~\cite{Astrand0F21}, \href{works/Astrand21.pdf}{Astrand21}~\cite{Astrand21}, \href{works/AstrandJZ20.pdf}{AstrandJZ20}~\cite{AstrandJZ20}, \href{works/AstrandJZ18.pdf}{AstrandJZ18}~\cite{AstrandJZ18}\\
\rowlabel{auth:a154}Miguel A. Salido & 3 &45 &\href{works/BartakS11.pdf}{BartakS11}~\cite{BartakS11}, \href{works/BartakSR10.pdf}{BartakSR10}~\cite{BartakSR10}, \href{works/AbrilSB05.pdf}{AbrilSB05}~\cite{AbrilSB05}\\
\rowlabel{auth:a230}Laurence A. Wolsey & 3 &50 &\href{works/HoundjiSW19.pdf}{HoundjiSW19}~\cite{HoundjiSW19}, \href{works/HoundjiSWD14.pdf}{HoundjiSWD14}~\cite{HoundjiSWD14}, \href{works/SadykovW06.pdf}{SadykovW06}~\cite{SadykovW06}\\
\rowlabel{auth:a391}Bruno A. Prata & 3 &1 &\href{works/PrataAN23.pdf}{PrataAN23}~\cite{PrataAN23}, \href{works/AbreuNP23.pdf}{AbreuNP23}~\cite{AbreuNP23}, \href{}{AbreuPNF23}~\cite{AbreuPNF23}\\
\rowlabel{auth:a849}Mehmet A. Begen & 3 &25 &\href{works/NaderiBZ22.pdf}{NaderiBZ22}~\cite{NaderiBZ22}, \href{}{NaderiBZ22a}~\cite{NaderiBZ22a}, \href{}{NaderiRBAU21}~\cite{NaderiRBAU21}\\
\rowlabel{auth:a829}Maliheh Aramon Bajestani & 3 &31 &\href{works/BajestaniB15.pdf}{BajestaniB15}~\cite{BajestaniB15}, \href{works/BajestaniB13.pdf}{BajestaniB13}~\cite{BajestaniB13}, \href{works/BajestaniB11.pdf}{BajestaniB11}~\cite{BajestaniB11}\\
\rowlabel{auth:a11}S{\'{e}}v{\'{e}}rine Betmbe Fetgo & 3 &1 &\href{works/KameugneFND23.pdf}{KameugneFND23}~\cite{KameugneFND23}, \href{works/FetgoD22.pdf}{FetgoD22}~\cite{FetgoD22}, \href{works/KameugneFGOQ18.pdf}{KameugneFGOQ18}~\cite{KameugneFGOQ18}\\
\rowlabel{auth:a190}Miquel Bofill & 3 &11 &\href{works/BofillCSV17.pdf}{BofillCSV17}~\cite{BofillCSV17}, \href{works/BofillGSV15.pdf}{BofillGSV15}~\cite{BofillGSV15}, \href{works/BofillEGPSV14.pdf}{BofillEGPSV14}~\cite{BofillEGPSV14}\\
\rowlabel{auth:a233}Thomas Bridi & 3 &29 &\href{works/BridiBLMB16.pdf}{BridiBLMB16}~\cite{BridiBLMB16}, \href{works/BridiLBBM16.pdf}{BridiLBBM16}~\cite{BridiLBBM16}, \href{works/BartoliniBBLM14.pdf}{BartoliniBBLM14}~\cite{BartoliniBBLM14}\\
\rowlabel{auth:a172}Cid C. de Souza & 3 &21 &\href{works/MouraSCL08.pdf}{MouraSCL08}~\cite{MouraSCL08}, \href{works/MouraSCL08a.pdf}{MouraSCL08a}~\cite{MouraSCL08a}, \href{works/HeipckeCCS00.pdf}{HeipckeCCS00}~\cite{HeipckeCCS00}\\
\rowlabel{auth:a1025}Hadrien Cambazard & 3 &23 &\href{works/CatusseCBL16.pdf}{CatusseCBL16}~\cite{CatusseCBL16}, \href{works/MalapertCGJLR13.pdf}{MalapertCGJLR13}~\cite{MalapertCGJLR13}, \href{}{MalapertCGJLR12}~\cite{MalapertCGJLR12}\\
\rowlabel{auth:a42}Quentin Cappart & 3 &8 &\href{works/PopovicCGNC22.pdf}{PopovicCGNC22}~\cite{PopovicCGNC22}, \href{works/CappartTSR18.pdf}{CappartTSR18}~\cite{CappartTSR18}, \href{works/CappartS17.pdf}{CappartS17}~\cite{CappartS17}\\
\rowlabel{auth:a163}Ondrej Cepek & 3 &36 &\href{works/BartakCS10.pdf}{BartakCS10}~\cite{BartakCS10}, \href{works/VilimBC05.pdf}{VilimBC05}~\cite{VilimBC05}, \href{works/VilimBC04.pdf}{VilimBC04}~\cite{VilimBC04}\\
\rowlabel{auth:a287}Amedeo Cesta & 3 &15 &\href{}{CestaOPS14}~\cite{CestaOPS14}, \href{works/OddiPCC03.pdf}{OddiPCC03}~\cite{OddiPCC03}, \href{works/CestaOS98.pdf}{CestaOS98}~\cite{CestaOS98}\\
\rowlabel{auth:a93}Giacomo Da Col & 3 &14 &\href{works/ColT22.pdf}{ColT22}~\cite{ColT22}, \href{works/abs-2102-08778.pdf}{abs-2102-08778}~\cite{abs-2102-08778}, \href{works/ColT19.pdf}{ColT19}~\cite{ColT19}\\
\rowlabel{auth:a226}Alban Derrien & 3 &17 &\href{works/Derrien15.pdf}{Derrien15}~\cite{Derrien15}, \href{works/DerrienP14.pdf}{DerrienP14}~\cite{DerrienP14}, \href{works/DerrienPZ14.pdf}{DerrienPZ14}~\cite{DerrienPZ14}\\
\rowlabel{auth:a295}Abdallah Elkhyari & 3 &10 &\href{works/Elkhyari03.pdf}{Elkhyari03}~\cite{Elkhyari03}, \href{works/ElkhyariGJ02.pdf}{ElkhyariGJ02}~\cite{ElkhyariGJ02}, \href{works/ElkhyariGJ02a.pdf}{ElkhyariGJ02a}~\cite{ElkhyariGJ02a}\\
\rowlabel{auth:a122}Hamed Fahimi & 3 &2 &\href{}{FahimiQ23}~\cite{FahimiQ23}, \href{works/FahimiOQ18.pdf}{FahimiOQ18}~\cite{FahimiOQ18}, \href{works/Fahimi16.pdf}{Fahimi16}~\cite{Fahimi16}\\
\rowlabel{auth:a385}Jeremy Frank & 3 &7 &\href{works/TranWDRFOVB16.pdf}{TranWDRFOVB16}~\cite{TranWDRFOVB16}, \href{works/TranDRFWOVB16.pdf}{TranDRFWOVB16}~\cite{TranDRFWOVB16}, \href{works/FrankK05.pdf}{FrankK05}~\cite{FrankK05}\\
\rowlabel{auth:a815}Douglas G. Down & 3 &20 &\href{works/TranPZLDB18.pdf}{TranPZLDB18}~\cite{TranPZLDB18}, \href{works/TerekhovTDB14.pdf}{TerekhovTDB14}~\cite{TerekhovTDB14}, \href{works/TranTDB13.pdf}{TranTDB13}~\cite{TranTDB13}\\
\rowlabel{auth:a198}Maurizio Gabbrielli & 3 &12 &\href{works/LiuCGM17.pdf}{LiuCGM17}~\cite{LiuCGM17}, \href{works/AmadiniGM16.pdf}{AmadiniGM16}~\cite{AmadiniGM16}, \href{works/FalaschiGMP97.pdf}{FalaschiGMP97}~\cite{FalaschiGMP97}\\
\rowlabel{auth:a15}Michele Garraffa & 3 &1 &\href{works/AlfieriGPS23.pdf}{AlfieriGPS23}~\cite{AlfieriGPS23}, \href{works/ArmstrongGOS22.pdf}{ArmstrongGOS22}~\cite{ArmstrongGOS22}, \href{works/ArmstrongGOS21.pdf}{ArmstrongGOS21}~\cite{ArmstrongGOS21}\\
\rowlabel{auth:a61}Martin Gebser & 3 &0 &\href{works/TasselGS23.pdf}{TasselGS23}~\cite{TasselGS23}, \href{works/abs-2306-05747.pdf}{abs-2306-05747}~\cite{abs-2306-05747}, \href{works/KovacsTKSG21.pdf}{KovacsTKSG21}~\cite{KovacsTKSG21}\\
\rowlabel{auth:a692}Jean{-}Claude Gentina & 3 &8 &\href{works/KorbaaYG00.pdf}{KorbaaYG00}~\cite{KorbaaYG00}, \href{works/LopezAKYG00.pdf}{LopezAKYG00}~\cite{LopezAKYG00}, \href{works/KorbaaYG99.pdf}{KorbaaYG99}~\cite{KorbaaYG99}\\
\rowlabel{auth:a83}Lucas Groleaz & 3 &4 &\href{works/Groleaz21.pdf}{Groleaz21}~\cite{Groleaz21}, \href{works/GroleazNS20.pdf}{GroleazNS20}~\cite{GroleazNS20}, \href{works/GroleazNS20a.pdf}{GroleazNS20a}~\cite{GroleazNS20a}\\
\rowlabel{auth:a760}Andy Ham & 3 &20 &\href{works/HamPK21.pdf}{HamPK21}~\cite{HamPK21}, \href{works/Ham18.pdf}{Ham18}~\cite{Ham18}, \href{}{Ham18a}~\cite{Ham18a}\\
\rowlabel{auth:a218}Renaud Hartert & 3 &35 &\href{works/GayHLS15.pdf}{GayHLS15}~\cite{GayHLS15}, \href{works/GayHS15.pdf}{GayHS15}~\cite{GayHS15}, \href{works/GayHS15a.pdf}{GayHS15a}~\cite{GayHS15a}\\
\rowlabel{auth:a138}Brahim Hnich & 3 &68 &\href{works/GokgurHO18.pdf}{GokgurHO18}~\cite{GokgurHO18}, \href{works/OzturkTHO13.pdf}{OzturkTHO13}~\cite{OzturkTHO13}, \href{works/RossiTHP07.pdf}{RossiTHP07}~\cite{RossiTHP07}\\
\rowlabel{auth:a54}Marie{-}Jos{\'{e}} Huguet & 3 &12 &\href{works/AntuoriHHEN21.pdf}{AntuoriHHEN21}~\cite{AntuoriHHEN21}, \href{works/AntuoriHHEN20.pdf}{AntuoriHHEN20}~\cite{AntuoriHHEN20}, \href{works/HebrardHJMPV16.pdf}{HebrardHJMPV16}~\cite{HebrardHJMPV16}\\
\rowlabel{auth:a251}Andrew J. Davenport & 3 &13 &\href{works/Davenport10.pdf}{Davenport10}~\cite{Davenport10}, \href{works/DavenportKRSH07.pdf}{DavenportKRSH07}~\cite{DavenportKRSH07}, \href{works/BeckDF97.pdf}{BeckDF97}~\cite{BeckDF97}\\
\rowlabel{auth:a75}Mikael Johansson & 3 &27 &\href{works/Astrand0F21.pdf}{Astrand0F21}~\cite{Astrand0F21}, \href{works/AstrandJZ20.pdf}{AstrandJZ20}~\cite{AstrandJZ20}, \href{works/AstrandJZ18.pdf}{AstrandJZ18}~\cite{AstrandJZ18}\\
\rowlabel{auth:a690}Ouajdi Korbaa & 3 &8 &\href{works/KorbaaYG00.pdf}{KorbaaYG00}~\cite{KorbaaYG00}, \href{works/LopezAKYG00.pdf}{LopezAKYG00}~\cite{LopezAKYG00}, \href{works/KorbaaYG99.pdf}{KorbaaYG99}~\cite{KorbaaYG99}\\
\rowlabel{auth:a124}Stefan Kreter & 3 &47 &\href{works/KreterSSZ18.pdf}{KreterSSZ18}~\cite{KreterSSZ18}, \href{works/KreterSS17.pdf}{KreterSS17}~\cite{KreterSS17}, \href{works/KreterSS15.pdf}{KreterSS15}~\cite{KreterSS15}\\
\rowlabel{auth:a670}Krzysztof Kuchcinski & 3 &24 &\href{works/WolinskiKG04.pdf}{WolinskiKG04}~\cite{WolinskiKG04}, \href{works/KuchcinskiW03.pdf}{KuchcinskiW03}~\cite{KuchcinskiW03}, \href{works/GruianK98.pdf}{GruianK98}~\cite{GruianK98}\\
\rowlabel{auth:a655}Andr{\'{e}} Langevin & 3 &107 &\href{works/MalapertCGJLR13.pdf}{MalapertCGJLR13}~\cite{MalapertCGJLR13}, \href{}{MalapertCGJLR12}~\cite{MalapertCGJLR12}, \href{works/KhayatLR06.pdf}{KhayatLR06}~\cite{KhayatLR06}\\
\rowlabel{auth:a361}Philippe Michelon & 3 &68 &\href{works/Acuna-AgostMFG09.pdf}{Acuna-AgostMFG09}~\cite{Acuna-AgostMFG09}, \href{works/LiessM08.pdf}{LiessM08}~\cite{LiessM08}, \href{}{DemasseyAM05}~\cite{DemasseyAM05}\\
\rowlabel{auth:a158}Tony Minoru Tamura Lopes & 3 &47 &\href{works/LopesCSM10.pdf}{LopesCSM10}~\cite{LopesCSM10}, \href{works/MouraSCL08.pdf}{MouraSCL08}~\cite{MouraSCL08}, \href{works/MouraSCL08a.pdf}{MouraSCL08a}~\cite{MouraSCL08a}\\
\rowlabel{auth:a326}Christina N. Burt & 3 &15 &\href{works/BurtLPS15.pdf}{BurtLPS15}~\cite{BurtLPS15}, \href{works/BlomBPS14.pdf}{BlomBPS14}~\cite{BlomBPS14}, \href{works/LipovetzkyBPS14.pdf}{LipovetzkyBPS14}~\cite{LipovetzkyBPS14}\\
\rowlabel{auth:a538}Hiroki Nishikawa & 3 &3 &\href{works/NishikawaSTT19.pdf}{NishikawaSTT19}~\cite{NishikawaSTT19}, \href{works/NishikawaSTT18.pdf}{NishikawaSTT18}~\cite{NishikawaSTT18}, \href{works/NishikawaSTT18a.pdf}{NishikawaSTT18a}~\cite{NishikawaSTT18a}\\
\rowlabel{auth:a285}Angelo Oddi & 3 &15 &\href{}{CestaOPS14}~\cite{CestaOPS14}, \href{works/OddiPCC03.pdf}{OddiPCC03}~\cite{OddiPCC03}, \href{works/CestaOS98.pdf}{CestaOS98}~\cite{CestaOS98}\\
\rowlabel{auth:a914}David R. Urbach & 3 &100 &\href{}{NaderiRBAU21}~\cite{NaderiRBAU21}, \href{}{RoshanaeiBAUB20}~\cite{RoshanaeiBAUB20}, \href{}{RoshanaeiLAU17a}~\cite{RoshanaeiLAU17a}\\
\rowlabel{auth:a257}Philippe Refalo & 3 &60 &\href{works/GarganiR07.pdf}{GarganiR07}~\cite{GarganiR07}, \href{works/BeckR03.pdf}{BeckR03}~\cite{BeckR03}, \href{}{MilanoORT02}~\cite{MilanoORT02}\\
\rowlabel{auth:a424}Levi Ribeiro de Abreu & 3 &11 &\href{works/AbreuNP23.pdf}{AbreuNP23}~\cite{AbreuNP23}, \href{works/AbreuN22.pdf}{AbreuN22}~\cite{AbreuN22}, \href{works/AbreuAPNM21.pdf}{AbreuAPNM21}~\cite{AbreuAPNM21}\\
\rowlabel{auth:a305}Mark S. Fox & 3 &27 &\href{works/BeckF00.pdf}{BeckF00}~\cite{BeckF00}, \href{works/BeckF98.pdf}{BeckF98}~\cite{BeckF98}, \href{works/BeckDF97.pdf}{BeckDF97}~\cite{BeckDF97}\\
\rowlabel{auth:a720}Gunnar Schrader & 3 &13 &\href{works/Wolf09.pdf}{Wolf09}~\cite{Wolf09}, \href{works/WolfS05.pdf}{WolfS05}~\cite{WolfS05}, \href{works/SchuttWS05.pdf}{SchuttWS05}~\cite{SchuttWS05}\\
\rowlabel{auth:a135}Jens Schulz & 3 &40 &\href{works/HeinzSB13.pdf}{HeinzSB13}~\cite{HeinzSB13}, \href{works/HeinzS11.pdf}{HeinzS11}~\cite{HeinzS11}, \href{works/BertholdHLMS10.pdf}{BertholdHLMS10}~\cite{BertholdHLMS10}\\
\rowlabel{auth:a425}Marcelo Seido Nagano & 3 &11 &\href{works/AbreuNP23.pdf}{AbreuNP23}~\cite{AbreuNP23}, \href{works/AbreuN22.pdf}{AbreuN22}~\cite{AbreuN22}, \href{works/AbreuAPNM21.pdf}{AbreuAPNM21}~\cite{AbreuAPNM21}\\
\rowlabel{auth:a539}Kana Shimada & 3 &3 &\href{works/NishikawaSTT19.pdf}{NishikawaSTT19}~\cite{NishikawaSTT19}, \href{works/NishikawaSTT18.pdf}{NishikawaSTT18}~\cite{NishikawaSTT18}, \href{works/NishikawaSTT18a.pdf}{NishikawaSTT18a}~\cite{NishikawaSTT18a}\\
\rowlabel{auth:a127}Gilles Simonin & 3 &8 &\href{works/GodetLHS20.pdf}{GodetLHS20}~\cite{GodetLHS20}, \href{works/SimoninAHL15.pdf}{SimoninAHL15}~\cite{SimoninAHL15}, \href{works/SimoninAHL12.pdf}{SimoninAHL12}~\cite{SimoninAHL12}\\
\rowlabel{auth:a816}Tiago Stegun Vaquero & 3 &29 &\href{works/TranVNB17.pdf}{TranVNB17}~\cite{TranVNB17}, \href{works/TranVNB17a.pdf}{TranVNB17a}~\cite{TranVNB17a}, \href{works/LouieVNB14.pdf}{LouieVNB14}~\cite{LouieVNB14}\\
\rowlabel{auth:a192}Josep Suy & 3 &11 &\href{works/BofillCSV17.pdf}{BofillCSV17}~\cite{BofillCSV17}, \href{works/BofillGSV15.pdf}{BofillGSV15}~\cite{BofillGSV15}, \href{works/BofillEGPSV14.pdf}{BofillEGPSV14}~\cite{BofillEGPSV14}\\
\rowlabel{auth:a387}Christos T. Maravelias & 3 &396 &\href{}{Adelgren2023}~\cite{Adelgren2023}, \href{}{HarjunkoskiMBC14}~\cite{HarjunkoskiMBC14}, \href{works/MaraveliasG04.pdf}{MaraveliasG04}~\cite{MaraveliasG04}\\
\rowlabel{auth:a476}Andreas T. Ernst & 3 &16 &\href{works/abs-2211-14492.pdf}{abs-2211-14492}~\cite{abs-2211-14492}, \href{}{EdwardsBSE19}~\cite{EdwardsBSE19}, \href{works/ThiruvadyBME09.pdf}{ThiruvadyBME09}~\cite{ThiruvadyBME09}\\
\rowlabel{auth:a540}Ittetsu Taniguchi & 3 &3 &\href{works/NishikawaSTT19.pdf}{NishikawaSTT19}~\cite{NishikawaSTT19}, \href{works/NishikawaSTT18.pdf}{NishikawaSTT18}~\cite{NishikawaSTT18}, \href{works/NishikawaSTT18a.pdf}{NishikawaSTT18a}~\cite{NishikawaSTT18a}\\
\rowlabel{auth:a58}Pierre Tassel & 3 &0 &\href{works/TasselGS23.pdf}{TasselGS23}~\cite{TasselGS23}, \href{works/abs-2306-05747.pdf}{abs-2306-05747}~\cite{abs-2306-05747}, \href{works/KovacsTKSG21.pdf}{KovacsTKSG21}~\cite{KovacsTKSG21}\\
\rowlabel{auth:a745}Reza Tavakkoli-Moghaddam & 3 &9 &\href{}{Fatemi-AnarakiTFV23}~\cite{Fatemi-AnarakiTFV23}, \href{}{NouriMHD23}~\cite{NouriMHD23}, \href{}{GhasemiMH23}~\cite{GhasemiMH23}\\
\rowlabel{auth:a541}Hiroyuki Tomiyama & 3 &3 &\href{works/NishikawaSTT19.pdf}{NishikawaSTT19}~\cite{NishikawaSTT19}, \href{works/NishikawaSTT18.pdf}{NishikawaSTT18}~\cite{NishikawaSTT18}, \href{works/NishikawaSTT18a.pdf}{NishikawaSTT18a}~\cite{NishikawaSTT18a}\\
\rowlabel{auth:a427}Seyda Topaloglu Yildiz & 3 &20 &\href{works/IsikYA23.pdf}{IsikYA23}~\cite{IsikYA23}, \href{works/YunusogluY22.pdf}{YunusogluY22}~\cite{YunusogluY22}, \href{works/KucukY19.pdf}{KucukY19}~\cite{KucukY19}\\
\rowlabel{auth:a207}Sascha Van Cauwelaert & 3 &8 &\href{works/CauwelaertLS18.pdf}{CauwelaertLS18}~\cite{CauwelaertLS18}, \href{works/CauwelaertDMS16.pdf}{CauwelaertDMS16}~\cite{CauwelaertDMS16}, \href{works/DejemeppeCS15.pdf}{DejemeppeCS15}~\cite{DejemeppeCS15}\\
\rowlabel{auth:a175}G{\'{e}}rard Verfaillie & 3 &119 &\href{works/HebrardHJMPV16.pdf}{HebrardHJMPV16}~\cite{HebrardHJMPV16}, \href{works/VerfaillieL01.pdf}{VerfaillieL01}~\cite{VerfaillieL01}, \href{works/BensanaLV99.pdf}{BensanaLV99}~\cite{BensanaLV99}\\
\rowlabel{auth:a161}Arnaldo Vieira Moura & 3 &47 &\href{works/LopesCSM10.pdf}{LopesCSM10}~\cite{LopesCSM10}, \href{works/MouraSCL08.pdf}{MouraSCL08}~\cite{MouraSCL08}, \href{works/MouraSCL08a.pdf}{MouraSCL08a}~\cite{MouraSCL08a}\\
\rowlabel{auth:a193}Mateu Villaret & 3 &11 &\href{works/BofillCSV17.pdf}{BofillCSV17}~\cite{BofillCSV17}, \href{works/BofillGSV15.pdf}{BofillGSV15}~\cite{BofillGSV15}, \href{works/BofillEGPSV14.pdf}{BofillEGPSV14}~\cite{BofillEGPSV14}\\
\rowlabel{auth:a46}Daniel Walkiewicz & 3 &0 &\href{works/LacknerMMWW23.pdf}{LacknerMMWW23}~\cite{LacknerMMWW23}, \href{works/WinterMMW22.pdf}{WinterMMW22}~\cite{WinterMMW22}, \href{works/LacknerMMWW21.pdf}{LacknerMMWW21}~\cite{LacknerMMWW21}\\
\rowlabel{auth:a691}Pascal Yim & 3 &8 &\href{works/KorbaaYG00.pdf}{KorbaaYG00}~\cite{KorbaaYG00}, \href{works/LopezAKYG00.pdf}{LopezAKYG00}~\cite{LopezAKYG00}, \href{works/KorbaaYG99.pdf}{KorbaaYG99}~\cite{KorbaaYG99}\\
\rowlabel{auth:a205}Alessandro Zanarini & 3 &25 &\href{works/AstrandJZ20.pdf}{AstrandJZ20}~\cite{AstrandJZ20}, \href{works/AstrandJZ18.pdf}{AstrandJZ18}~\cite{AstrandJZ18}, \href{works/BonfiettiZLM16.pdf}{BonfiettiZLM16}~\cite{BonfiettiZLM16}\\
\rowlabel{auth:a631}Luis Zeballos & 3 &35 &\href{works/ZeballosQH10.pdf}{ZeballosQH10}~\cite{ZeballosQH10}, \href{works/ZeballosH05.pdf}{ZeballosH05}~\cite{ZeballosH05}, \href{works/QuirogaZH05.pdf}{QuirogaZH05}~\cite{QuirogaZH05}\\
\rowlabel{auth:a560}Viktoria A. Hauder & 2 &14 &\href{}{HauderBRPA20}~\cite{HauderBRPA20}, \href{works/abs-1902-09244.pdf}{abs-1902-09244}~\cite{abs-1902-09244}\\
\rowlabel{auth:a893}Daniel A. Desmond & 2 &1 &\href{works/AntunesABD20.pdf}{AntunesABD20}~\cite{AntunesABD20}, \href{works/AntunesABD18.pdf}{AntunesABD18}~\cite{AntunesABD18}\\
\rowlabel{auth:a564}Michael Affenzeller & 2 &14 &\href{}{HauderBRPA20}~\cite{HauderBRPA20}, \href{works/abs-1902-09244.pdf}{abs-1902-09244}~\cite{abs-1902-09244}\\
\rowlabel{auth:a734}Abderrahmane Aggoun & 2 &187 &\href{}{AggounMV08}~\cite{AggounMV08}, \href{works/AggounB93.pdf}{AggounB93}~\cite{AggounB93}\\
\rowlabel{auth:a891}Mark Antunes & 2 &1 &\href{works/AntunesABD20.pdf}{AntunesABD20}~\cite{AntunesABD20}, \href{works/AntunesABD18.pdf}{AntunesABD18}~\cite{AntunesABD18}\\
\rowlabel{auth:a53}Valentin Antuori & 2 &3 &\href{works/AntuoriHHEN21.pdf}{AntuoriHHEN21}~\cite{AntuoriHHEN21}, \href{works/AntuoriHHEN20.pdf}{AntuoriHHEN20}~\cite{AntuoriHHEN20}\\
\rowlabel{auth:a892}Vincent Armant & 2 &1 &\href{works/AntunesABD20.pdf}{AntunesABD20}~\cite{AntunesABD20}, \href{works/AntunesABD18.pdf}{AntunesABD18}~\cite{AntunesABD18}\\
\rowlabel{auth:a14}Eddie Armstrong & 2 &1 &\href{works/ArmstrongGOS22.pdf}{ArmstrongGOS22}~\cite{ArmstrongGOS22}, \href{works/ArmstrongGOS21.pdf}{ArmstrongGOS21}~\cite{ArmstrongGOS21}\\
\rowlabel{auth:a352}Emrah B. Edis & 2 &48 &\href{works/EdisO11.pdf}{EdisO11}~\cite{EdisO11}, \href{}{EdisO11a}~\cite{EdisO11a}\\
\rowlabel{auth:a504}Amelia Badica & 2 &4 &\href{works/BadicaBI20.pdf}{BadicaBI20}~\cite{BadicaBI20}, \href{works/BadicaBIL19.pdf}{BadicaBIL19}~\cite{BadicaBIL19}\\
\rowlabel{auth:a505}Costin Badica & 2 &4 &\href{works/BadicaBI20.pdf}{BadicaBI20}~\cite{BadicaBI20}, \href{works/BadicaBIL19.pdf}{BadicaBIL19}~\cite{BadicaBIL19}\\
\rowlabel{auth:a703}Pierre Baptiste & 2 &13 &\href{}{BoucherBVBL97}~\cite{BoucherBVBL97}, \href{works/BaptisteLV92.pdf}{BaptisteLV92}~\cite{BaptisteLV92}\\
\rowlabel{auth:a400}Nicolas Barnier & 2 &0 &\href{works/WangB23.pdf}{WangB23}~\cite{WangB23}, \href{works/WangB20.pdf}{WangB20}~\cite{WangB20}\\
\rowlabel{auth:a561}Andreas Beham & 2 &14 &\href{}{HauderBRPA20}~\cite{HauderBRPA20}, \href{works/abs-1902-09244.pdf}{abs-1902-09244}~\cite{abs-1902-09244}\\
\rowlabel{auth:a114}Ondrej Benedikt & 2 &3 &\href{works/BenediktMH20.pdf}{BenediktMH20}~\cite{BenediktMH20}, \href{works/BenediktSMVH18.pdf}{BenediktSMVH18}~\cite{BenediktSMVH18}\\
\rowlabel{auth:a381}Davide Bertozzi & 2 &27 &\href{works/RuggieroBBMA09.pdf}{RuggieroBBMA09}~\cite{RuggieroBBMA09}, \href{works/BeniniBGM06.pdf}{BeniniBGM06}~\cite{BeniniBGM06}\\
\rowlabel{auth:a343}Jean{-}Charles Billaut & 2 &23 &\href{works/BillautHL12.pdf}{BillautHL12}~\cite{BillautHL12}, \href{works/LorigeonBB02.pdf}{LorigeonBB02}~\cite{LorigeonBB02}\\
\rowlabel{auth:a232}Andrea Borghesi & 2 &23 &\href{works/BorghesiBLMB18.pdf}{BorghesiBLMB18}~\cite{BorghesiBLMB18}, \href{works/BartoliniBBLM14.pdf}{BartoliniBBLM14}~\cite{BartoliniBBLM14}\\
\rowlabel{auth:a413}Dario Canut{-}de{-}Bon & 2 &1 &\href{works/YuraszeckMCCR23.pdf}{YuraszeckMCCR23}~\cite{YuraszeckMCCR23}, \href{works/YuraszeckMC23.pdf}{YuraszeckMC23}~\cite{YuraszeckMC23}\\
\rowlabel{auth:a275}Tom Carchrae & 2 &16 &\href{works/CarchraeB09.pdf}{CarchraeB09}~\cite{CarchraeB09}, \href{works/CarchraeBF05.pdf}{CarchraeBF05}~\cite{CarchraeBF05}\\
\rowlabel{auth:a858}Jacques Carlier & 2 &6 &\href{}{CarlierSJP21}~\cite{CarlierSJP21}, \href{}{NeronABCDD06}~\cite{NeronABCDD06}\\
\rowlabel{auth:a94}Erich Christian Teppan & 2 &11 &\href{works/Teppan22.pdf}{Teppan22}~\cite{Teppan22}, \href{works/ColT19.pdf}{ColT19}~\cite{ColT19}\\
\rowlabel{auth:a102}Jordi Coll Caballero & 2 &0 &\href{works/Caballero23.pdf}{Caballero23}~\cite{Caballero23}, \href{works/Caballero19.pdf}{Caballero19}~\cite{Caballero19}\\
\rowlabel{auth:a170}Yves Colombani & 2 &9 &\href{works/HeipckeCCS00.pdf}{HeipckeCCS00}~\cite{HeipckeCCS00}, \href{works/Colombani96.pdf}{Colombani96}~\cite{Colombani96}\\
\rowlabel{auth:a132}Joseph D. Scott & 2 &13 &\href{works/KameugneFSN14.pdf}{KameugneFSN14}~\cite{KameugneFSN14}, \href{works/KameugneFSN11.pdf}{KameugneFSN11}~\cite{KameugneFSN11}\\
\rowlabel{auth:a290}Emilie Danna & 2 &23 &\href{}{DannaP04}~\cite{DannaP04}, \href{works/DannaP03.pdf}{DannaP03}~\cite{DannaP03}\\
\rowlabel{auth:a1020}St{\'{e}}phane Dauz{\`{e}}re{-}P{\'{e}}r{\`{e}}s & 2 &14 &\href{}{PenzDN23}~\cite{PenzDN23}, \href{}{NattafDYW19}~\cite{NattafDYW19}\\
\rowlabel{auth:a417}Mauro Dell'Amico & 2 &2 &\href{works/MontemanniD23.pdf}{MontemanniD23}~\cite{MontemanniD23}, \href{works/MontemanniD23a.pdf}{MontemanniD23a}~\cite{MontemanniD23a}\\
\rowlabel{auth:a821}Minh Do & 2 &3 &\href{works/TranWDRFOVB16.pdf}{TranWDRFOVB16}~\cite{TranWDRFOVB16}, \href{works/TranDRFWOVB16.pdf}{TranDRFWOVB16}~\cite{TranDRFWOVB16}\\
\rowlabel{auth:a922}Ulrich Dorndorf & 2 &18 &\href{}{DorndorfPH99}~\cite{DorndorfPH99}, \href{}{DorndorfHP99}~\cite{DorndorfHP99}\\
\rowlabel{auth:a168}Hani El Sakkout & 2 &82 &\href{works/KamarainenS02.pdf}{KamarainenS02}~\cite{KamarainenS02}, \href{works/SakkoutW00.pdf}{SakkoutW00}~\cite{SakkoutW00}\\
\rowlabel{auth:a70}Sebastian Engell & 2 &384 &\href{works/KlankeBYE21.pdf}{KlankeBYE21}~\cite{KlankeBYE21}, \href{}{HarjunkoskiMBC14}~\cite{HarjunkoskiMBC14}\\
\rowlabel{auth:a421}Tamer Eren & 2 &1 &\href{works/GurPAE23.pdf}{GurPAE23}~\cite{GurPAE23}, \href{works/GurEA19.pdf}{GurEA19}~\cite{GurEA19}\\
\rowlabel{auth:a894}Guillaume Escamocher & 2 &1 &\href{works/AntunesABD20.pdf}{AntunesABD20}~\cite{AntunesABD20}, \href{works/AntunesABD18.pdf}{AntunesABD18}~\cite{AntunesABD18}\\
\rowlabel{auth:a55}Siham Essodaigui & 2 &3 &\href{works/AntuoriHHEN21.pdf}{AntuoriHHEN21}~\cite{AntuoriHHEN21}, \href{works/AntuoriHHEN20.pdf}{AntuoriHHEN20}~\cite{AntuoriHHEN20}\\
\rowlabel{auth:a220}Caroline Even & 2 &3 &\href{works/EvenSH15.pdf}{EvenSH15}~\cite{EvenSH15}, \href{works/EvenSH15a.pdf}{EvenSH15a}~\cite{EvenSH15a}\\
\rowlabel{auth:a301}Stephen F. Smith & 2 &7 &\href{}{CestaOPS14}~\cite{CestaOPS14}, \href{works/CestaOS98.pdf}{CestaOS98}~\cite{CestaOS98}\\
\rowlabel{auth:a629}Minhaz F. Zibran & 2 &43 &\href{works/ZibranR11.pdf}{ZibranR11}~\cite{ZibranR11}, \href{works/ZibranR11a.pdf}{ZibranR11a}~\cite{ZibranR11a}\\
\rowlabel{auth:a523}Azadeh Farsi & 2 &25 &\href{works/FarsiTM22.pdf}{FarsiTM22}~\cite{FarsiTM22}, \href{works/MokhtarzadehTNF20.pdf}{MokhtarzadehTNF20}~\cite{MokhtarzadehTNF20}\\
\rowlabel{auth:a362}Dominique Feillet & 2 &19 &\href{works/Acuna-AgostMFG09.pdf}{Acuna-AgostMFG09}~\cite{Acuna-AgostMFG09}, \href{works/ArtiguesBF04.pdf}{ArtiguesBF04}~\cite{ArtiguesBF04}\\
\rowlabel{auth:a9}Michel Gamache & 2 &0 &\href{works/AalianPG23.pdf}{AalianPG23}~\cite{AalianPG23}, \href{works/CampeauG22.pdf}{CampeauG22}~\cite{CampeauG22}\\
\rowlabel{auth:a235}Marc Garcia & 2 &10 &\href{works/BofillGSV15.pdf}{BofillGSV15}~\cite{BofillGSV15}, \href{works/BofillEGPSV14.pdf}{BofillEGPSV14}~\cite{BofillEGPSV14}\\
\rowlabel{auth:a643}Antonio Garrido & 2 &27 &\href{works/GarridoAO09.pdf}{GarridoAO09}~\cite{GarridoAO09}, \href{works/GarridoOS08.pdf}{GarridoOS08}~\cite{GarridoOS08}\\
\rowlabel{auth:a895}Anne{-}Marie George & 2 &1 &\href{works/AntunesABD20.pdf}{AntunesABD20}~\cite{AntunesABD20}, \href{works/AntunesABD18.pdf}{AntunesABD18}~\cite{AntunesABD18}\\
\rowlabel{auth:a822}Eleanor Gilbert Rieffel & 2 &3 &\href{works/TranWDRFOVB16.pdf}{TranWDRFOVB16}~\cite{TranWDRFOVB16}, \href{works/TranDRFWOVB16.pdf}{TranDRFWOVB16}~\cite{TranDRFWOVB16}\\
\rowlabel{auth:a316}Vincent Gingras & 2 &1 &\href{works/KameugneFGOQ18.pdf}{KameugneFGOQ18}~\cite{KameugneFGOQ18}, \href{works/GingrasQ16.pdf}{GingrasQ16}~\cite{GingrasQ16}\\
\rowlabel{auth:a478}Arthur Godet & 2 &1 &\href{works/Godet21a.pdf}{Godet21a}~\cite{Godet21a}, \href{works/GodetLHS20.pdf}{GodetLHS20}~\cite{GodetLHS20}\\
\rowlabel{auth:a195}Adrian Goldwaser & 2 &8 &\href{works/GoldwaserS18.pdf}{GoldwaserS18}~\cite{GoldwaserS18}, \href{works/GoldwaserS17.pdf}{GoldwaserS17}~\cite{GoldwaserS17}\\
\rowlabel{auth:a201}Arnaud Gotlieb & 2 &9 &\href{works/MossigeGSMC17.pdf}{MossigeGSMC17}~\cite{MossigeGSMC17}, \href{works/AlesioNBG14.pdf}{AlesioNBG14}~\cite{AlesioNBG14}\\
\rowlabel{auth:a884}Iiro Harjunkoski & 2 &550 &\href{}{HarjunkoskiMBC14}~\cite{HarjunkoskiMBC14}, \href{works/HarjunkoskiG02.pdf}{HarjunkoskiG02}~\cite{HarjunkoskiG02}\\
\rowlabel{auth:a439}Vil{\'{e}}m Heinz & 2 &5 &\href{works/abs-2305-19888.pdf}{abs-2305-19888}~\cite{abs-2305-19888}, \href{works/HeinzNVH22.pdf}{HeinzNVH22}~\cite{HeinzNVH22}\\
\rowlabel{auth:a64}Alessandro Hill & 2 &0 &\href{}{HillBCGN22}~\cite{HillBCGN22}, \href{works/HillTV21.pdf}{HillTV21}~\cite{HillTV21}\\
\rowlabel{auth:a336}Seyed Hossein Hashemi Doulabi & 2 &59 &\href{works/DoulabiRP16.pdf}{DoulabiRP16}~\cite{DoulabiRP16}, \href{works/DoulabiRP14.pdf}{DoulabiRP14}~\cite{DoulabiRP14}\\
\rowlabel{auth:a184}Georgiana Ifrim & 2 &12 &\href{works/GrimesIOS14.pdf}{GrimesIOS14}~\cite{GrimesIOS14}, \href{works/IfrimOS12.pdf}{IfrimOS12}~\cite{IfrimOS12}\\
\rowlabel{auth:a506}Mirjana Ivanovic & 2 &4 &\href{works/BadicaBI20.pdf}{BadicaBI20}~\cite{BadicaBI20}, \href{works/BadicaBIL19.pdf}{BadicaBIL19}~\cite{BadicaBIL19}\\
\rowlabel{auth:a854}Raf Jans & 2 &60 &\href{}{MartnezAJ22}~\cite{MartnezAJ22}, \href{works/Jans09.pdf}{Jans09}~\cite{Jans09}\\
\rowlabel{auth:a630}Chanchal K. Roy & 2 &43 &\href{works/ZibranR11.pdf}{ZibranR11}~\cite{ZibranR11}, \href{works/ZibranR11a.pdf}{ZibranR11a}~\cite{ZibranR11a}\\
\rowlabel{auth:a78}Lucas Kletzander & 2 &1 &\href{works/GeibingerKKMMW21.pdf}{GeibingerKKMMW21}~\cite{GeibingerKKMMW21}, \href{works/KletzanderM17.pdf}{KletzanderM17}~\cite{KletzanderM17}\\
\rowlabel{auth:a547}Jan Kristof Behrens & 2 &12 &\href{works/BehrensLM19.pdf}{BehrensLM19}~\cite{BehrensLM19}, \href{works/abs-1901-07914.pdf}{abs-1901-07914}~\cite{abs-1901-07914}\\
\rowlabel{auth:a337}Wen{-}Yang Ku & 2 &128 &\href{works/KuB16.pdf}{KuB16}~\cite{KuB16}, \href{works/HeinzKB13.pdf}{HeinzKB13}~\cite{HeinzKB13}\\
\rowlabel{auth:a807}Michelle L. Blom & 2 &35 &\href{works/BlomPS16.pdf}{BlomPS16}~\cite{BlomPS16}, \href{works/BlomBPS14.pdf}{BlomBPS14}~\cite{BlomBPS14}\\
\rowlabel{auth:a62}Marie{-}Louise Lackner & 2 &0 &\href{works/LacknerMMWW23.pdf}{LacknerMMWW23}~\cite{LacknerMMWW23}, \href{works/LacknerMMWW21.pdf}{LacknerMMWW21}~\cite{LacknerMMWW21}\\
\rowlabel{auth:a434}Arnaud Lallouet & 2 &0 &\href{works/PerezGSL23.pdf}{PerezGSL23}~\cite{PerezGSL23}, \href{works/abs-2312-13682.pdf}{abs-2312-13682}~\cite{abs-2312-13682}\\
\rowlabel{auth:a729}Evelina Lamma & 2 &12 &\href{works/LammaMM97.pdf}{LammaMM97}~\cite{LammaMM97}, \href{works/BrusoniCLMMT96.pdf}{BrusoniCLMMT96}~\cite{BrusoniCLMMT96}\\
\rowlabel{auth:a548}Ralph Lange & 2 &12 &\href{works/BehrensLM19.pdf}{BehrensLM19}~\cite{BehrensLM19}, \href{works/abs-1901-07914.pdf}{abs-1901-07914}~\cite{abs-1901-07914}\\
\rowlabel{auth:a704}Bruno Legeard & 2 &13 &\href{}{BoucherBVBL97}~\cite{BoucherBVBL97}, \href{works/BaptisteLV92.pdf}{BaptisteLV92}~\cite{BaptisteLV92}\\
\rowlabel{auth:a1001}Pierre Lemaire & 2 &32 &\href{works/CatusseCBL16.pdf}{CatusseCBL16}~\cite{CatusseCBL16}, \href{works/GuyonLPR12.pdf}{GuyonLPR12}~\cite{GuyonLPR12}\\
\rowlabel{auth:a174}Michel Lema{\^{\i}}tre & 2 &110 &\href{works/VerfaillieL01.pdf}{VerfaillieL01}~\cite{VerfaillieL01}, \href{works/BensanaLV99.pdf}{BensanaLV99}~\cite{BensanaLV99}\\
\rowlabel{auth:a213}BoonPing Lim & 2 &6 &\href{works/LimHTB16.pdf}{LimHTB16}~\cite{LimHTB16}, \href{works/LimBTBB15.pdf}{LimBTBB15}~\cite{LimBTBB15}\\
\rowlabel{auth:a145}Kamol Limtanyakul & 2 &6 &\href{works/LimtanyakulS12.pdf}{LimtanyakulS12}~\cite{LimtanyakulS12}, \href{works/Limtanyakul07.pdf}{Limtanyakul07}~\cite{Limtanyakul07}\\
\rowlabel{auth:a897}Yiqing Lin & 2 &1 &\href{works/AntunesABD20.pdf}{AntunesABD20}~\cite{AntunesABD20}, \href{works/AntunesABD18.pdf}{AntunesABD18}~\cite{AntunesABD18}\\
\rowlabel{auth:a327}Nir Lipovetzky & 2 &0 &\href{works/BurtLPS15.pdf}{BurtLPS15}~\cite{BurtLPS15}, \href{works/LipovetzkyBPS14.pdf}{LipovetzkyBPS14}~\cite{LipovetzkyBPS14}\\
\rowlabel{auth:a180}James Little & 2 &30 &\href{works/KrogtLPHJ07.pdf}{KrogtLPHJ07}~\cite{KrogtLPHJ07}, \href{works/Darby-DowmanLMZ97.pdf}{Darby-DowmanLMZ97}~\cite{Darby-DowmanLMZ97}\\
\rowlabel{auth:a472}Shixin Liu & 2 &0 &\href{works/LiFJZLL22.pdf}{LiFJZLL22}~\cite{LiFJZLL22}, \href{works/ZhangJZL22.pdf}{ZhangJZL22}~\cite{ZhangJZL22}\\
\rowlabel{auth:a247}Xavier Lorca & 2 &29 &\href{works/GodetLHS20.pdf}{GodetLHS20}~\cite{GodetLHS20}, \href{works/HermenierDL11.pdf}{HermenierDL11}~\cite{HermenierDL11}\\
\rowlabel{auth:a946}Curtiss Luong & 2 &115 &\href{}{RoshanaeiLAU17}~\cite{RoshanaeiLAU17}, \href{}{RoshanaeiLAU17a}~\cite{RoshanaeiLAU17a}\\
\rowlabel{auth:a648}Abid M. Malik & 2 &15 &\href{works/Malik08.pdf}{Malik08}~\cite{Malik08}, \href{works/MalikMB08.pdf}{MalikMB08}~\cite{MalikMB08}\\
\rowlabel{auth:a907}Pedro M. Castro & 2 &381 &\href{}{HarjunkoskiMBC14}~\cite{HarjunkoskiMBC14}, \href{}{CastroGR10}~\cite{CastroGR10}\\
\rowlabel{auth:a324}Gilles Madi{-}Wamba & 2 &1 &\href{works/Madi-WambaLOBM17.pdf}{Madi-WambaLOBM17}~\cite{Madi-WambaLOBM17}, \href{works/Madi-WambaB16.pdf}{Madi-WambaB16}~\cite{Madi-WambaB16}\\
\rowlabel{auth:a799}Adrien Maillard & 2 &9 &\href{works/HebrardALLCMR22.pdf}{HebrardALLCMR22}~\cite{HebrardALLCMR22}, \href{works/HebrardHJMPV16.pdf}{HebrardHJMPV16}~\cite{HebrardHJMPV16}\\
\rowlabel{auth:a549}Masoumeh Mansouri & 2 &12 &\href{works/BehrensLM19.pdf}{BehrensLM19}~\cite{BehrensLM19}, \href{works/abs-1901-07914.pdf}{abs-1901-07914}~\cite{abs-1901-07914}\\
\rowlabel{auth:a199}Jacopo Mauro & 2 &2 &\href{works/LiuCGM17.pdf}{LiuCGM17}~\cite{LiuCGM17}, \href{works/AmadiniGM16.pdf}{AmadiniGM16}~\cite{AmadiniGM16}\\
\rowlabel{auth:a430}Gonzalo Mej{\'{\i}}a & 2 &25 &\href{works/YuraszeckMC23.pdf}{YuraszeckMC23}~\cite{YuraszeckMC23}, \href{works/MejiaY20.pdf}{MejiaY20}~\cite{MejiaY20}\\
\rowlabel{auth:a730}Paola Mello & 2 &12 &\href{works/LammaMM97.pdf}{LammaMM97}~\cite{LammaMM97}, \href{works/BrusoniCLMMT96.pdf}{BrusoniCLMMT96}~\cite{BrusoniCLMMT96}\\
\rowlabel{auth:a937}Carlos Mencía & 2 &25 &\href{works/MenciaSV13.pdf}{MenciaSV13}~\cite{MenciaSV13}, \href{works/MenciaSV12.pdf}{MenciaSV12}~\cite{MenciaSV12}\\
\rowlabel{auth:a522}Mahdi Mokhtarzadeh & 2 &25 &\href{works/FarsiTM22.pdf}{FarsiTM22}~\cite{FarsiTM22}, \href{works/MokhtarzadehTNF20.pdf}{MokhtarzadehTNF20}~\cite{MokhtarzadehTNF20}\\
\rowlabel{auth:a416}Roberto Montemanni & 2 &2 &\href{works/MontemanniD23.pdf}{MontemanniD23}~\cite{MontemanniD23}, \href{works/MontemanniD23a.pdf}{MontemanniD23a}~\cite{MontemanniD23a}\\
\rowlabel{auth:a63}Christoph Mrkvicka & 2 &0 &\href{works/LacknerMMWW23.pdf}{LacknerMMWW23}~\cite{LacknerMMWW23}, \href{works/LacknerMMWW21.pdf}{LacknerMMWW21}~\cite{LacknerMMWW21}\\
\rowlabel{auth:a115}Istv{\'{a}}n M{\'{o}}dos & 2 &3 &\href{works/BenediktMH20.pdf}{BenediktMH20}~\cite{BenediktMH20}, \href{works/BenediktSMVH18.pdf}{BenediktSMVH18}~\cite{BenediktSMVH18}\\
\rowlabel{auth:a563}Sophie N. Parragh & 2 &14 &\href{}{HauderBRPA20}~\cite{HauderBRPA20}, \href{works/abs-1902-09244.pdf}{abs-1902-09244}~\cite{abs-1902-09244}\\
\rowlabel{auth:a84}Samba Ndojh Ndiaye & 2 &4 &\href{works/GroleazNS20.pdf}{GroleazNS20}~\cite{GroleazNS20}, \href{works/GroleazNS20a.pdf}{GroleazNS20a}~\cite{GroleazNS20a}\\
\rowlabel{auth:a133}Youcheu Ngo{-}Kateu & 2 &13 &\href{works/KameugneFSN14.pdf}{KameugneFSN14}~\cite{KameugneFSN14}, \href{works/KameugneFSN11.pdf}{KameugneFSN11}~\cite{KameugneFSN11}\\
\rowlabel{auth:a56}Alain Nguyen & 2 &3 &\href{works/AntuoriHHEN21.pdf}{AntuoriHHEN21}~\cite{AntuoriHHEN21}, \href{works/AntuoriHHEN20.pdf}{AntuoriHHEN20}~\cite{AntuoriHHEN20}\\
\rowlabel{auth:a401}Su Nguyen & 2 &0 &\href{works/abs-2402-00459.pdf}{abs-2402-00459}~\cite{abs-2402-00459}, \href{works/abs-2211-14492.pdf}{abs-2211-14492}~\cite{abs-2211-14492}\\
\rowlabel{auth:a440}Anton{\'{\i}}n Nov{\'{a}}k & 2 &5 &\href{works/abs-2305-19888.pdf}{abs-2305-19888}~\cite{abs-2305-19888}, \href{works/HeinzNVH22.pdf}{HeinzNVH22}~\cite{HeinzNVH22}\\
\rowlabel{auth:a823}Bryan O'Gorman & 2 &3 &\href{works/TranWDRFOVB16.pdf}{TranWDRFOVB16}~\cite{TranWDRFOVB16}, \href{works/TranDRFWOVB16.pdf}{TranDRFWOVB16}~\cite{TranDRFWOVB16}\\
\rowlabel{auth:a896}Mike O'Keeffe & 2 &1 &\href{works/AntunesABD20.pdf}{AntunesABD20}~\cite{AntunesABD20}, \href{works/AntunesABD18.pdf}{AntunesABD18}~\cite{AntunesABD18}\\
\rowlabel{auth:a645}Eva Onaindia & 2 &27 &\href{works/GarridoAO09.pdf}{GarridoAO09}~\cite{GarridoAO09}, \href{works/GarridoOS08.pdf}{GarridoOS08}~\cite{GarridoOS08}\\
\rowlabel{auth:a354}Irem Ozkarahan & 2 &89 &\href{}{EdisO11a}~\cite{EdisO11a}, \href{works/TopalogluO11.pdf}{TopalogluO11}~\cite{TopalogluO11}\\
\rowlabel{auth:a898}Cemalettin Ozturk & 2 &1 &\href{works/AntunesABD20.pdf}{AntunesABD20}~\cite{AntunesABD20}, \href{works/AntunesABD18.pdf}{AntunesABD18}~\cite{AntunesABD18}\\
\rowlabel{auth:a652}Carla P. Gomes & 2 &0 &\href{works/HoeveGSL07.pdf}{HoeveGSL07}~\cite{HoeveGSL07}, \href{works/GomesHS06.pdf}{GomesHS06}~\cite{GomesHS06}\\
\rowlabel{auth:a131}Laure Pauline Fotso & 2 &13 &\href{works/KameugneFSN14.pdf}{KameugneFSN14}~\cite{KameugneFSN14}, \href{works/KameugneFSN11.pdf}{KameugneFSN11}~\cite{KameugneFSN11}\\
\rowlabel{auth:a431}Guillaume Perez & 2 &0 &\href{works/PerezGSL23.pdf}{PerezGSL23}~\cite{PerezGSL23}, \href{works/abs-2312-13682.pdf}{abs-2312-13682}~\cite{abs-2312-13682}\\
\rowlabel{auth:a923}Toàn Phan Huy & 2 &18 &\href{}{DorndorfPH99}~\cite{DorndorfPH99}, \href{}{DorndorfHP99}~\cite{DorndorfHP99}\\
\rowlabel{auth:a286}Nicola Policella & 2 &10 &\href{}{CestaOPS14}~\cite{CestaOPS14}, \href{works/OddiPCC03.pdf}{OddiPCC03}~\cite{OddiPCC03}\\
\rowlabel{auth:a33}Enrico Pontelli & 2 &0 &\href{works/TardivoDFMP23.pdf}{TardivoDFMP23}~\cite{TardivoDFMP23}, \href{}{VillaverdeP04}~\cite{VillaverdeP04}\\
\rowlabel{auth:a899}Luis Quesada & 2 &1 &\href{works/AntunesABD20.pdf}{AntunesABD20}~\cite{AntunesABD20}, \href{works/AntunesABD18.pdf}{AntunesABD18}~\cite{AntunesABD18}\\
\rowlabel{auth:a632}Oscar Quiroga & 2 &35 &\href{works/ZeballosQH10.pdf}{ZeballosQH10}~\cite{ZeballosQH10}, \href{works/QuirogaZH05.pdf}{QuirogaZH05}~\cite{QuirogaZH05}\\
\rowlabel{auth:a348}G{\"{u}}nther R. Raidl & 2 &14 &\href{works/FrohnerTR19.pdf}{FrohnerTR19}~\cite{FrohnerTR19}, \href{works/RendlPHPR12.pdf}{RendlPHPR12}~\cite{RendlPHPR12}\\
\rowlabel{auth:a392}Levi R. Abreu & 2 &0 &\href{works/PrataAN23.pdf}{PrataAN23}~\cite{PrataAN23}, \href{}{AbreuPNF23}~\cite{AbreuPNF23}\\
\rowlabel{auth:a938}María R. Sierra & 2 &25 &\href{works/MenciaSV13.pdf}{MenciaSV13}~\cite{MenciaSV13}, \href{works/MenciaSV12.pdf}{MenciaSV12}~\cite{MenciaSV12}\\
\rowlabel{auth:a562}Sebastian Raggl & 2 &14 &\href{}{HauderBRPA20}~\cite{HauderBRPA20}, \href{works/abs-1902-09244.pdf}{abs-1902-09244}~\cite{abs-1902-09244}\\
\rowlabel{auth:a229}Vinas{\'{e}}tan Ratheil Houndji & 2 &5 &\href{works/HoundjiSW19.pdf}{HoundjiSW19}~\cite{HoundjiSW19}, \href{works/HoundjiSWD14.pdf}{HoundjiSWD14}~\cite{HoundjiSWD14}\\
\rowlabel{auth:a1003}David Rivreau & 2 &42 &\href{works/NattafALR16.pdf}{NattafALR16}~\cite{NattafALR16}, \href{works/GuyonLPR12.pdf}{GuyonLPR12}~\cite{GuyonLPR12}\\
\rowlabel{auth:a319}Francesca Rossi & 2 &29 &\href{works/GelainPRVW17.pdf}{GelainPRVW17}~\cite{GelainPRVW17}, \href{works/BartakSR10.pdf}{BartakSR10}~\cite{BartakSR10}\\
\rowlabel{auth:a908}Louis-Martin Rousseau & 2 &106 &\href{}{CastroGR10}~\cite{CastroGR10}, \href{}{CorreaLR07}~\cite{CorreaLR07}\\
\rowlabel{auth:a393}Marcelo S. Nagano & 2 &0 &\href{works/PrataAN23.pdf}{PrataAN23}~\cite{PrataAN23}, \href{}{AbreuPNF23}~\cite{AbreuPNF23}\\
\rowlabel{auth:a887}Erlendur S. Thorsteinsson & 2 &81 &\href{}{MilanoORT02}~\cite{MilanoORT02}, \href{works/Thorsteinsson01.pdf}{Thorsteinsson01}~\cite{Thorsteinsson01}\\
\rowlabel{auth:a390}Ruslan Sadykov & 2 &56 &\href{works/SadykovW06.pdf}{SadykovW06}~\cite{SadykovW06}, \href{works/Sadykov04.pdf}{Sadykov04}~\cite{Sadykov04}\\
\rowlabel{auth:a429}Konstantin Schekotihin & 2 &0 &\href{works/TasselGS23.pdf}{TasselGS23}~\cite{TasselGS23}, \href{works/abs-2306-05747.pdf}{abs-2306-05747}~\cite{abs-2306-05747}\\
\rowlabel{auth:a92}Christian Schulte & 2 &5 &\href{works/WessenCS20.pdf}{WessenCS20}~\cite{WessenCS20}, \href{works/FrimodigS19.pdf}{FrimodigS19}~\cite{FrimodigS19}\\
\rowlabel{auth:a653}Bart Selman & 2 &0 &\href{works/HoeveGSL07.pdf}{HoeveGSL07}~\cite{HoeveGSL07}, \href{works/GomesHS06.pdf}{GomesHS06}~\cite{GomesHS06}\\
\rowlabel{auth:a120}Paul Shaw & 2 &179 &\href{works/LaborieRSV18.pdf}{LaborieRSV18}~\cite{LaborieRSV18}, \href{works/VilimLS15.pdf}{VilimLS15}~\cite{VilimLS15}\\
\rowlabel{auth:a433}Wijnand Suijlen & 2 &0 &\href{works/PerezGSL23.pdf}{PerezGSL23}~\cite{PerezGSL23}, \href{works/abs-2312-13682.pdf}{abs-2312-13682}~\cite{abs-2312-13682}\\
\rowlabel{auth:a403}Yuan Sun & 2 &0 &\href{works/abs-2402-00459.pdf}{abs-2402-00459}~\cite{abs-2402-00459}, \href{works/abs-2211-14492.pdf}{abs-2211-14492}~\cite{abs-2211-14492}\\
\rowlabel{auth:a436}Reza Tavakkoli{-}Moghaddam & 2 &25 &\href{works/Mehdizadeh-Somarin23.pdf}{Mehdizadeh-Somarin23}~\cite{Mehdizadeh-Somarin23}, \href{works/MokhtarzadehTNF20.pdf}{MokhtarzadehTNF20}~\cite{MokhtarzadehTNF20}\\
\rowlabel{auth:a13}Cl{\'{e}}mentin Tayou Djam{\'{e}}gni & 2 &0 &\href{works/KameugneFND23.pdf}{KameugneFND23}~\cite{KameugneFND23}, \href{works/FetgoD22.pdf}{FetgoD22}~\cite{FetgoD22}\\
\rowlabel{auth:a618}Erich Teppan & 2 &3 &\href{works/abs-2102-08778.pdf}{abs-2102-08778}~\cite{abs-2102-08778}, \href{}{FriedrichFMRSST14}~\cite{FriedrichFMRSST14}\\
\rowlabel{auth:a185}Alexander Tesch & 2 &9 &\href{works/Tesch18.pdf}{Tesch18}~\cite{Tesch18}, \href{works/Tesch16.pdf}{Tesch16}~\cite{Tesch16}\\
\rowlabel{auth:a215}Sylvie Thi{\'{e}}baux & 2 &6 &\href{works/LimHTB16.pdf}{LimHTB16}~\cite{LimHTB16}, \href{works/LimBTBB15.pdf}{LimBTBB15}~\cite{LimBTBB15}\\
\rowlabel{auth:a847}Charles Thomas & 2 &6 &\href{works/ThomasKS20.pdf}{ThomasKS20}~\cite{ThomasKS20}, \href{works/CappartTSR18.pdf}{CappartTSR18}~\cite{CappartTSR18}\\
\rowlabel{auth:a438}Behdin Vahedi Nouri & 2 &25 &\href{works/Mehdizadeh-Somarin23.pdf}{Mehdizadeh-Somarin23}~\cite{Mehdizadeh-Somarin23}, \href{works/MokhtarzadehTNF20.pdf}{MokhtarzadehTNF20}~\cite{MokhtarzadehTNF20}\\
\rowlabel{auth:a747}Behdin Vahedi-Nouri & 2 &9 &\href{}{Fatemi-AnarakiTFV23}~\cite{Fatemi-AnarakiTFV23}, \href{}{NouriMHD23}~\cite{NouriMHD23}\\
\rowlabel{auth:a939}Ramiro Varela & 2 &25 &\href{works/MenciaSV13.pdf}{MenciaSV13}~\cite{MenciaSV13}, \href{works/MenciaSV12.pdf}{MenciaSV12}~\cite{MenciaSV12}\\
\rowlabel{auth:a702}Christophe Varnier & 2 &13 &\href{}{BoucherBVBL97}~\cite{BoucherBVBL97}, \href{works/BaptisteLV92.pdf}{BaptisteLV92}~\cite{BaptisteLV92}\\
\rowlabel{auth:a824}Davide Venturelli & 2 &3 &\href{works/TranWDRFOVB16.pdf}{TranWDRFOVB16}~\cite{TranWDRFOVB16}, \href{works/TranDRFWOVB16.pdf}{TranDRFWOVB16}~\cite{TranDRFWOVB16}\\
\rowlabel{auth:a399}Ruixin Wang & 2 &0 &\href{works/WangB23.pdf}{WangB23}~\cite{WangB23}, \href{works/WangB20.pdf}{WangB20}~\cite{WangB20}\\
\rowlabel{auth:a820}Zhihui Wang & 2 &3 &\href{works/TranWDRFOVB16.pdf}{TranWDRFOVB16}~\cite{TranWDRFOVB16}, \href{works/TranDRFWOVB16.pdf}{TranDRFWOVB16}~\cite{TranDRFWOVB16}\\
\rowlabel{auth:a366}Jean{-}Paul Watson & 2 &57 &\href{works/BeckFW11.pdf}{BeckFW11}~\cite{BeckFW11}, \href{works/WatsonB08.pdf}{WatsonB08}~\cite{WatsonB08}\\
\rowlabel{auth:a277}Christine Wei Wu & 2 &42 &\href{works/WuBB09.pdf}{WuBB09}~\cite{WuBB09}, \href{works/WuBB05.pdf}{WuBB05}~\cite{WuBB05}\\
\rowlabel{auth:a669}Christophe Wolinski & 2 &19 &\href{works/WolinskiKG04.pdf}{WolinskiKG04}~\cite{WolinskiKG04}, \href{works/KuchcinskiW03.pdf}{KuchcinskiW03}~\cite{KuchcinskiW03}\\
\rowlabel{auth:a462}Farouk Yalaoui & 2 &3 &\href{works/OujanaAYB22.pdf}{OujanaAYB22}~\cite{OujanaAYB22}, \href{works/ArbaouiY18.pdf}{ArbaouiY18}~\cite{ArbaouiY18}\\
\rowlabel{auth:a19}Neil Yorke{-}Smith & 2 &5 &\href{works/EfthymiouY23.pdf}{EfthymiouY23}~\cite{EfthymiouY23}, \href{works/WallaceY20.pdf}{WallaceY20}~\cite{WallaceY20}\\
\rowlabel{auth:a470}Ziyan Zhao & 2 &0 &\href{works/LiFJZLL22.pdf}{LiFJZLL22}~\cite{LiFJZLL22}, \href{works/ZhangJZL22.pdf}{ZhangJZL22}~\cite{ZhangJZL22}\\
\rowlabel{auth:a178}Jianyang Zhou & 2 &24 &\href{works/Zhou97.pdf}{Zhou97}~\cite{Zhou97}, \href{works/Zhou96.pdf}{Zhou96}~\cite{Zhou96}\\
\rowlabel{auth:a216}Menkes van den Briel & 2 &6 &\href{works/LimHTB16.pdf}{LimHTB16}~\cite{LimHTB16}, \href{works/LimBTBB15.pdf}{LimBTBB15}~\cite{LimBTBB15}\\
\rowlabel{auth:a620}Peter van Beek & 2 &16 &\href{works/BegB13.pdf}{BegB13}~\cite{BegB13}, \href{works/MalikMB08.pdf}{MalikMB08}~\cite{MalikMB08}\\
\rowlabel{auth:a951} & 1 &63 &\href{}{ArtiguesDN08}~\cite{ArtiguesDN08}\\
\rowlabel{auth:a108}Florian A. Herzog & 1 &2 &\href{works/KoehlerBFFHPSSS21.pdf}{KoehlerBFFHPSSS21}~\cite{KoehlerBFFHPSSS21}\\
\rowlabel{auth:a380}J. A. Hoogeveen & 1 &2 &\href{works/AkkerDH07.pdf}{AkkerDH07}~\cite{AkkerDH07}\\
\rowlabel{auth:a395}M. A. Hakim Newton & 1 &0 &\href{works/RiahiNS018.pdf}{RiahiNS018}~\cite{RiahiNS018}\\
\rowlabel{auth:a569}Amr A. Kandil & 1 &24 &\href{works/TangLWSK18.pdf}{TangLWSK18}~\cite{TangLWSK18}\\
\rowlabel{auth:a677}Antonio A. M{\'{a}}rquez & 1 &7 &\href{works/ValleMGT03.pdf}{ValleMGT03}~\cite{ValleMGT03}\\
\rowlabel{auth:a757}Kennedy A. G. Ara{\'u}jo & 1 &0 &\href{works/AbreuAPNM21.pdf}{AbreuAPNM21}~\cite{AbreuAPNM21}\\
\rowlabel{auth:a798}Steve A. Chien & 1 &0 &\href{works/HebrardALLCMR22.pdf}{HebrardALLCMR22}~\cite{HebrardALLCMR22}\\
\rowlabel{auth:a828}Sheila A. McIlraith & 1 &0 &\href{works/LuoVLBM16.pdf}{LuoVLBM16}~\cite{LuoVLBM16}\\
\rowlabel{auth:a912}Andre A. Ciré & 1 &15 &\href{}{CireCH16}~\cite{CireCH16}\\
\rowlabel{auth:a942}Julie A. Shah & 1 &71 &\href{}{GombolayWS18}~\cite{GombolayWS18}\\
\rowlabel{auth:a7}Younes Aalian & 1 &0 &\href{works/AalianPG23.pdf}{AalianPG23}~\cite{AalianPG23}\\
\rowlabel{auth:a1006}E.H.L. Aarts & 1 &65 &\href{}{NuijtenA96}~\cite{NuijtenA96}\\
\rowlabel{auth:a479}Hanaa Abohashima & 1 &1 &\href{works/AbohashimaEG21.pdf}{AbohashimaEG21}~\cite{AbohashimaEG21}\\
\rowlabel{auth:a273}Montserrat Abril & 1 &0 &\href{works/AbrilSB05.pdf}{AbrilSB05}~\cite{AbrilSB05}\\
\rowlabel{auth:a360}Rodrigo Acuna{-}Agost & 1 &3 &\href{works/Acuna-AgostMFG09.pdf}{Acuna-AgostMFG09}~\cite{Acuna-AgostMFG09}\\
\rowlabel{auth:a990}Nathan Adelgren & 1 &0 &\href{}{Adelgren2023}~\cite{Adelgren2023}\\
\rowlabel{auth:a537}W. Adelman & 1 &17 &\href{works/EscobetPQPRA19.pdf}{EscobetPQPRA19}~\cite{EscobetPQPRA19}\\
\rowlabel{auth:a958}Yossiri Adulyasak & 1 &1 &\href{}{MartnezAJ22}~\cite{MartnezAJ22}\\
\rowlabel{auth:a984}Sezin Afsar & 1 &0 &\href{}{AfsarVPG23}~\cite{AfsarVPG23}\\
\rowlabel{auth:a325}Pen{\'{e}}lope Aguiar{-}Melgarejo & 1 &14 &\href{works/MelgarejoLS15.pdf}{MelgarejoLS15}~\cite{MelgarejoLS15}\\
\rowlabel{auth:a673}Sanjay Ahire & 1 &0 &\href{works/KanetAG04.pdf}{KanetAG04}~\cite{KanetAG04}\\
\rowlabel{auth:a422}Aftab Ahmed Shaikh & 1 &0 &\href{works/ShaikhK23.pdf}{ShaikhK23}~\cite{ShaikhK23}\\
\rowlabel{auth:a477}Uwe Aickelin & 1 &0 &\href{works/abs-2211-14492.pdf}{abs-2211-14492}~\cite{abs-2211-14492}\\
\rowlabel{auth:a972}Farid Ajili & 1 &4 &\href{}{AjiliW04}~\cite{AjiliW04}\\
\rowlabel{auth:a842}Ali Akbar Sadat Asl & 1 &55 &\href{works/ZarandiASC20.pdf}{ZarandiASC20}~\cite{ZarandiASC20}\\
\rowlabel{auth:a601}Mohsen Akbarpour Shirazi & 1 &28 &\href{works/ZarandiKS16.pdf}{ZarandiKS16}~\cite{ZarandiKS16}\\
\rowlabel{auth:a738}Arianna Alfieri & 1 &0 &\href{works/AlfieriGPS23.pdf}{AlfieriGPS23}~\cite{AlfieriGPS23}\\
\rowlabel{auth:a749}S. Ali Torabi & 1 &0 &\href{works/FarsiTM22.pdf}{FarsiTM22}~\cite{FarsiTM22}\\
\rowlabel{auth:a520}Samira Alizdeh & 1 &1 &\href{}{AlizdehS20}~\cite{AlizdehS20}\\
\rowlabel{auth:a693}Hassane Alla & 1 &0 &\href{works/LopezAKYG00.pdf}{LopezAKYG00}~\cite{LopezAKYG00}\\
\rowlabel{auth:a929}Roberto Amadini & 1 &2 &\href{works/AmadiniGM16.pdf}{AmadiniGM16}~\cite{AmadiniGM16}\\
\rowlabel{auth:a461}Lionel Amodeo & 1 &1 &\href{works/OujanaAYB22.pdf}{OujanaAYB22}~\cite{OujanaAYB22}\\
\rowlabel{auth:a728}Alexandru Andrei & 1 &9 &\href{works/RuggieroBBMA09.pdf}{RuggieroBBMA09}~\cite{RuggieroBBMA09}\\
\rowlabel{auth:a298}Ola Angelsmark & 1 &1 &\href{works/AngelsmarkJ00.pdf}{AngelsmarkJ00}~\cite{AngelsmarkJ00}\\
\rowlabel{auth:a826}Richard Anthony Valenzano & 1 &0 &\href{works/LuoVLBM16.pdf}{LuoVLBM16}~\cite{LuoVLBM16}\\
\rowlabel{auth:a712}M. Anton Ertl & 1 &14 &\href{works/ErtlK91.pdf}{ErtlK91}~\cite{ErtlK91}\\
\rowlabel{auth:a642}Zbigniew Antoni Banaszak & 1 &0 &\href{works/BocewiczBB09.pdf}{BocewiczBB09}~\cite{BocewiczBB09}\\
\rowlabel{auth:a644}Marlene Arang{\'{u}} & 1 &5 &\href{works/GarridoAO09.pdf}{GarridoAO09}~\cite{GarridoAO09}\\
\rowlabel{auth:a819}Arthur Araujo & 1 &72 &\href{works/TranAB16.pdf}{TranAB16}~\cite{TranAB16}\\
\rowlabel{auth:a588}Taha Arbaoui & 1 &2 &\href{works/ArbaouiY18.pdf}{ArbaouiY18}~\cite{ArbaouiY18}\\
\rowlabel{auth:a943}Dmitry Arkhipov & 1 &12 &\href{}{ArkhipovBL19}~\cite{ArkhipovBL19}\\
\rowlabel{auth:a717}Martin Aronsson & 1 &0 &\href{works/AronssonBK09.pdf}{AronssonBK09}~\cite{AronssonBK09}\\
\rowlabel{auth:a139}M. Arslan Ornek & 1 &31 &\href{works/OzturkTHO13.pdf}{OzturkTHO13}~\cite{OzturkTHO13}\\
\rowlabel{auth:a265}Konstantin Artiouchine & 1 &3 &\href{works/ArtiouchineB05.pdf}{ArtiouchineB05}~\cite{ArtiouchineB05}\\
\rowlabel{auth:a769}Arezoo Atighehchian & 1 &0 &\href{works/YounespourAKE19.pdf}{YounespourAKE19}~\cite{YounespourAKE19}\\
\rowlabel{auth:a423}Abdullah Ayub Khan & 1 &0 &\href{works/ShaikhK23.pdf}{ShaikhK23}~\cite{ShaikhK23}\\
\rowlabel{auth:a480}Amr B. Eltawil & 1 &1 &\href{works/AbohashimaEG21.pdf}{AbohashimaEG21}~\cite{AbohashimaEG21}\\
\rowlabel{auth:a671}Maya B. Gokhale & 1 &0 &\href{works/WolinskiKG04.pdf}{WolinskiKG04}~\cite{WolinskiKG04}\\
\rowlabel{auth:a711}David B. H. Tay & 1 &0 &\href{}{Tay92}~\cite{Tay92}\\
\rowlabel{auth:a910}Davaatseren Baatar & 1 &3 &\href{}{EdwardsBSE19}~\cite{EdwardsBSE19}\\
\rowlabel{auth:a99}{\"{O}}zalp Babaoglu & 1 &1 &\href{works/GalleguillosKSB19.pdf}{GalleguillosKSB19}~\cite{GalleguillosKSB19}\\
\rowlabel{auth:a641}Irena Bach & 1 &0 &\href{works/BocewiczBB09.pdf}{BocewiczBB09}~\cite{BocewiczBB09}\\
\rowlabel{auth:a701}Astrid Bachelu & 1 &0 &\href{}{BoucherBVBL97}~\cite{BoucherBVBL97}\\
\rowlabel{auth:a330}Scott Backhaus & 1 &4 &\href{works/LimBTBB15.pdf}{LimBTBB15}~\cite{LimBTBB15}\\
\rowlabel{auth:a585}Hari Balasubramanian & 1 &9 &\href{works/ShinBBHO18.pdf}{ShinBBHO18}~\cite{ShinBBHO18}\\
\rowlabel{auth:a372}Viet Bang Nguyen & 1 &0 &\href{works/LauLN08.pdf}{LauLN08}~\cite{LauLN08}\\
\rowlabel{auth:a274}Federico Barber & 1 &0 &\href{works/AbrilSB05.pdf}{AbrilSB05}~\cite{AbrilSB05}\\
\rowlabel{auth:a367}Ada Barlatt & 1 &1 &\href{works/BarlattCG08.pdf}{BarlattCG08}~\cite{BarlattCG08}\\
\rowlabel{auth:a528}Mohammadreza Barzegaran & 1 &0 &\href{works/BarzegaranZP20.pdf}{BarzegaranZP20}~\cite{BarzegaranZP20}\\
\rowlabel{auth:a525}Virginie Basini & 1 &8 &\href{works/Polo-MejiaALB20.pdf}{Polo-MejiaALB20}~\cite{Polo-MejiaALB20}\\
\rowlabel{auth:a944}Olga Battaïa & 1 &12 &\href{}{ArkhipovBL19}~\cite{ArkhipovBL19}\\
\rowlabel{auth:a795}N Beldiceanu & 1 &167 &\href{works/BeldiceanuC94.pdf}{BeldiceanuC94}~\cite{BeldiceanuC94}\\
\rowlabel{auth:a176}Said Belhadji & 1 &3 &\href{works/BelhadjiI98.pdf}{BelhadjiI98}~\cite{BelhadjiI98}\\
\rowlabel{auth:a389}Sana Belmokhtar & 1 &16 &\href{works/ArtiguesBF04.pdf}{ArtiguesBF04}~\cite{ArtiguesBF04}\\
\rowlabel{auth:a466}Fatima Benbouzid{-}Si Tayeb & 1 &0 &\href{works/TouatBT22.pdf}{TouatBT22}~\cite{TouatBT22}\\
\rowlabel{auth:a500}Till Bender & 1 &1 &\href{works/BenderWS21.pdf}{BenderWS21}~\cite{BenderWS21}\\
\rowlabel{auth:a465}Belaid Benhamou & 1 &0 &\href{works/TouatBT22.pdf}{TouatBT22}~\cite{TouatBT22}\\
\rowlabel{auth:a264}Hachemi Bennaceur & 1 &8 &\href{works/KhemmoudjPB06.pdf}{KhemmoudjPB06}~\cite{KhemmoudjPB06}\\
\rowlabel{auth:a173}E. Bensana & 1 &99 &\href{works/BensanaLV99.pdf}{BensanaLV99}~\cite{BensanaLV99}\\
\rowlabel{auth:a329}Russell Bent & 1 &4 &\href{works/LimBTBB15.pdf}{LimBTBB15}~\cite{LimBTBB15}\\
\rowlabel{auth:a357}Timo Berthold & 1 &28 &\href{works/BertholdHLMS10.pdf}{BertholdHLMS10}~\cite{BertholdHLMS10}\\
\rowlabel{auth:a334}Christian Bessiere & 1 &1 &\href{works/BessiereHMQW14.pdf}{BessiereHMQW14}~\cite{BessiereHMQW14}\\
\rowlabel{auth:a836}Julien Bidot & 1 &58 &\href{works/BidotVLB09.pdf}{BidotVLB09}~\cite{BidotVLB09}\\
\rowlabel{auth:a398}Arthur Bit{-}Monnot & 1 &0 &\href{works/Bit-Monnot23.pdf}{Bit-Monnot23}~\cite{Bit-Monnot23}\\
\rowlabel{auth:a775}Jacek Blazewicz & 1 &38 &\href{}{BlazewiczEP19}~\cite{BlazewiczEP19}\\
\rowlabel{auth:a646}Christian Blum & 1 &13 &\href{works/ThiruvadyBME09.pdf}{ThiruvadyBME09}~\cite{ThiruvadyBME09}\\
\rowlabel{auth:a640}Grzegorz Bocewicz & 1 &0 &\href{works/BocewiczBB09.pdf}{BocewiczBB09}~\cite{BocewiczBB09}\\
\rowlabel{auth:a718}Markus Bohlin & 1 &0 &\href{works/AronssonBK09.pdf}{AronssonBK09}~\cite{AronssonBK09}\\
\rowlabel{auth:a959}Peter Bongers & 1 &381 &\href{}{HarjunkoskiMBC14}~\cite{HarjunkoskiMBC14}\\
\rowlabel{auth:a714}Nicolas Bonifas & 1 &3 &\href{works/BaptisteB18.pdf}{BaptisteB18}~\cite{BaptisteB18}\\
\rowlabel{auth:a700}Eric Boucher & 1 &0 &\href{}{BoucherBVBL97}~\cite{BoucherBVBL97}\\
\rowlabel{auth:a34}Rapha{\"{e}}l Boudreault & 1 &0 &\href{works/BoudreaultSLQ22.pdf}{BoudreaultSLQ22}~\cite{BoudreaultSLQ22}\\
\rowlabel{auth:a682}Jean{-}Louis Bouquard & 1 &22 &\href{works/LorigeonBB02.pdf}{LorigeonBB02}~\cite{LorigeonBB02}\\
\rowlabel{auth:a448}Eric Bourreau & 1 &4 &\href{works/BourreauGGLT22.pdf}{BourreauGGLT22}~\cite{BourreauGGLT22}\\
\rowlabel{auth:a1026}Nadia Brauner & 1 &0 &\href{works/CatusseCBL16.pdf}{CatusseCBL16}~\cite{CatusseCBL16}\\
\rowlabel{auth:a705}Silvia Breitinger & 1 &0 &\href{}{BreitingerL95}~\cite{BreitingerL95}\\
\rowlabel{auth:a320}Kristen Brent Venable & 1 &1 &\href{works/GelainPRVW17.pdf}{GelainPRVW17}~\cite{GelainPRVW17}\\
\rowlabel{auth:a463}D. Brodart & 1 &1 &\href{works/OujanaAYB22.pdf}{OujanaAYB22}~\cite{OujanaAYB22}\\
\rowlabel{auth:a584}Yuriy Brun & 1 &9 &\href{works/ShinBBHO18.pdf}{ShinBBHO18}~\cite{ShinBBHO18}\\
\rowlabel{auth:a731}Vittorio Brusoni & 1 &1 &\href{works/BrusoniCLMMT96.pdf}{BrusoniCLMMT96}~\cite{BrusoniCLMMT96}\\
\rowlabel{auth:a105}Josef B{\"{u}}rgler & 1 &2 &\href{works/KoehlerBFFHPSSS21.pdf}{KoehlerBFFHPSSS21}~\cite{KoehlerBFFHPSSS21}\\
\rowlabel{auth:a998}Jacek Błażewicz & 1 &344 &\href{}{BlazewiczDP96}~\cite{BlazewiczDP96}\\
\rowlabel{auth:a171}Cristina C. B. Cavalcante & 1 &5 &\href{works/HeipckeCCS00.pdf}{HeipckeCCS00}~\cite{HeipckeCCS00}\\
\rowlabel{auth:a239}Lionel C. Briand & 1 &3 &\href{works/AlesioNBG14.pdf}{AlesioNBG14}~\cite{AlesioNBG14}\\
\rowlabel{auth:a276}Eugene C. Freuder & 1 &0 &\href{works/CarchraeBF05.pdf}{CarchraeBF05}~\cite{CarchraeBF05}\\
\rowlabel{auth:a604}Kevin C. Furman & 1 &48 &\href{works/GoelSHFS15.pdf}{GoelSHFS15}~\cite{GoelSHFS15}\\
\rowlabel{auth:a694}Joseph C. Pemberton & 1 &26 &\href{works/PembertonG98.pdf}{PembertonG98}~\cite{PembertonG98}\\
\rowlabel{auth:a706}Hendrik C. R. Lock & 1 &0 &\href{}{BreitingerL95}~\cite{BreitingerL95}\\
\rowlabel{auth:a748}Erich C. Teppan & 1 &3 &\href{works/ColT22.pdf}{ColT22}~\cite{ColT22}\\
\rowlabel{auth:a940}Matthew C. Gombolay & 1 &71 &\href{}{GombolayWS18}~\cite{GombolayWS18}\\
\rowlabel{auth:a889}Eray Cakici & 1 &50 &\href{works/HamC16.pdf}{HamC16}~\cite{HamC16}\\
\rowlabel{auth:a103}Louis{-}Pierre Campeau & 1 &0 &\href{works/CampeauG22.pdf}{CampeauG22}~\cite{CampeauG22}\\
\rowlabel{auth:a160}Cid Carvalho de Souza & 1 &31 &\href{works/LopesCSM10.pdf}{LopesCSM10}~\cite{LopesCSM10}\\
\rowlabel{auth:a304}Yves Caseau & 1 &0 &\href{works/Caseau97.pdf}{Caseau97}~\cite{Caseau97}\\
\rowlabel{auth:a844}Oscar Castillo & 1 &55 &\href{works/ZarandiASC20.pdf}{ZarandiASC20}~\cite{ZarandiASC20}\\
\rowlabel{auth:a1024}Nicolas Catusse & 1 &0 &\href{works/CatusseCBL16.pdf}{CatusseCBL16}~\cite{CatusseCBL16}\\
\rowlabel{auth:a591}Yao{-}Ting Chang & 1 &2 &\href{works/HoYCLLCLC18.pdf}{HoYCLLCLC18}~\cite{HoYCLLCLC18}\\
\rowlabel{auth:a350}Nicolas Chapados & 1 &5 &\href{works/ChapadosJR11.pdf}{ChapadosJR11}~\cite{ChapadosJR11}\\
\rowlabel{auth:a901}Philippe Charlier & 1 &11 &\href{works/SimonisCK00.pdf}{SimonisCK00}~\cite{SimonisCK00}\\
\rowlabel{auth:a932}Yarong Chen & 1 &2 &\href{works/ChenGPSH10.pdf}{ChenGPSH10}~\cite{ChenGPSH10}\\
\rowlabel{auth:a765}Mohammad Cherkaoui & 1 &0 &\href{works/FallahiAC20.pdf}{FallahiAC20}~\cite{FallahiAC20}\\
\rowlabel{auth:a596}Han{-}Mo Chiu & 1 &2 &\href{works/HoYCLLCLC18.pdf}{HoYCLLCLC18}~\cite{HoYCLLCLC18}\\
\rowlabel{auth:a24}Yeonjun Choi & 1 &0 &\href{works/KimCMLLP23.pdf}{KimCMLLP23}~\cite{KimCMLLP23}\\
\rowlabel{auth:a383}Yingyi Chu & 1 &13 &\href{works/ChuX05.pdf}{ChuX05}~\cite{ChuX05}\\
\rowlabel{auth:a594}Sue{-}Min Chu & 1 &2 &\href{works/HoYCLLCLC18.pdf}{HoYCLLCLC18}~\cite{HoYCLLCLC18}\\
\rowlabel{auth:a370}Hoong Chuin Lau & 1 &0 &\href{works/LauLN08.pdf}{LauLN08}~\cite{LauLN08}\\
\rowlabel{auth:a995}Italo Cipriano & 1 &0 &\href{}{HillBCGN22}~\cite{HillBCGN22}\\
\rowlabel{auth:a875}Michael Codish & 1 &127 &\href{works/OhrimenkoSC09.pdf}{OhrimenkoSC09}~\cite{OhrimenkoSC09}\\
\rowlabel{auth:a151}Carleton Coffrin & 1 &14 &\href{works/SchausHMCMD11.pdf}{SchausHMCMD11}~\cite{SchausHMCMD11}\\
\rowlabel{auth:a817}Eldan Cohen & 1 &1 &\href{works/CohenHB17.pdf}{CohenHB17}~\cite{CohenHB17}\\
\rowlabel{auth:a191}Jordi Coll & 1 &1 &\href{works/BofillCSV17.pdf}{BofillCSV17}~\cite{BofillCSV17}\\
\rowlabel{auth:a732}Luca Console & 1 &1 &\href{works/BrusoniCLMMT96.pdf}{BrusoniCLMMT96}~\cite{BrusoniCLMMT96}\\
\rowlabel{auth:a796}E Contejean & 1 &167 &\href{works/BeldiceanuC94.pdf}{BeldiceanuC94}~\cite{BeldiceanuC94}\\
\rowlabel{auth:a306}Trijntje Cornelissens & 1 &17 &\href{works/SimonisC95.pdf}{SimonisC95}~\cite{SimonisC95}\\
\rowlabel{auth:a288}Gabriella Cortellessa & 1 &8 &\href{works/OddiPCC03.pdf}{OddiPCC03}~\cite{OddiPCC03}\\
\rowlabel{auth:a414}Nicol{\'{a}}s Cuneo & 1 &0 &\href{works/YuraszeckMCCR23.pdf}{YuraszeckMCCR23}~\cite{YuraszeckMCCR23}\\
\rowlabel{auth:a741}Kateryna Czerniachowska & 1 &0 &\href{works/CzerniachowskaWZ23.pdf}{CzerniachowskaWZ23}~\cite{CzerniachowskaWZ23}\\
\rowlabel{auth:a39}Alain C{\^{o}}t{\'{e}} & 1 &0 &\href{works/PopovicCGNC22.pdf}{PopovicCGNC22}~\cite{PopovicCGNC22}\\
\rowlabel{auth:a194}Kenneth D. Young & 1 &6 &\href{works/YoungFS17.pdf}{YoungFS17}~\cite{YoungFS17}\\
\rowlabel{auth:a322}Laurent D. Michel & 1 &3 &\href{works/FontaineMH16.pdf}{FontaineMH16}~\cite{FontaineMH16}\\
\rowlabel{auth:a377}Steven D. Prestwich & 1 &6 &\href{works/RossiTHP07.pdf}{RossiTHP07}~\cite{RossiTHP07}\\
\rowlabel{auth:a780}Michael D. Moffitt & 1 &0 &\href{works/MoffittPP05.pdf}{MoffittPP05}~\cite{MoffittPP05}\\
\rowlabel{auth:a918}Jean Damay & 1 &3 &\href{}{NeronABCDD06}~\cite{NeronABCDD06}\\
\rowlabel{auth:a179}Ken Darby{-}Dowman & 1 &28 &\href{works/Darby-DowmanLMZ97.pdf}{Darby-DowmanLMZ97}~\cite{Darby-DowmanLMZ97}\\
\rowlabel{auth:a240}Vivian De Smedt & 1 &7 &\href{works/GaySS14.pdf}{GaySS14}~\cite{GaySS14}\\
\rowlabel{auth:a249}Alexis De Clercq & 1 &3 &\href{works/ClercqPBJ11.pdf}{ClercqPBJ11}~\cite{ClercqPBJ11}\\
\rowlabel{auth:a303}Rina Dechter & 1 &10 &\href{works/FrostD98.pdf}{FrostD98}~\cite{FrostD98}\\
\rowlabel{auth:a676}Carmelo Del Valle & 1 &7 &\href{works/ValleMGT03.pdf}{ValleMGT03}~\cite{ValleMGT03}\\
\rowlabel{auth:a793}Xavier Delorme & 1 &0 &\href{works/RodriguezDG02.pdf}{RodriguezDG02}~\cite{RodriguezDG02}\\
\rowlabel{auth:a710}Alain Demeure & 1 &0 &\href{}{JourdanFRD94}~\cite{JourdanFRD94}\\
\rowlabel{auth:a315}Emir Demirovic & 1 &4 &\href{works/DemirovicS18.pdf}{DemirovicS18}~\cite{DemirovicS18}\\
\rowlabel{auth:a197}Roberto Di Cosmo & 1 &0 &\href{works/LiuCGM17.pdf}{LiuCGM17}~\cite{LiuCGM17}\\
\rowlabel{auth:a379}Guido Diepen & 1 &2 &\href{works/AkkerDH07.pdf}{AkkerDH07}~\cite{AkkerDH07}\\
\rowlabel{auth:a270}Bistra Dilkina & 1 &2 &\href{works/DilkinaDH05.pdf}{DilkinaDH05}~\cite{DilkinaDH05}\\
\rowlabel{auth:a726}Mehmet Dincbas & 1 &86 &\href{works/DincbasSH90.pdf}{DincbasSH90}~\cite{DincbasSH90}\\
\rowlabel{auth:a981}Yann Disser & 1 &0 &\href{works/EmdeZD22.pdf}{EmdeZD22}~\cite{EmdeZD22}\\
\rowlabel{auth:a969}Alexandre Dolgui & 1 &2 &\href{}{NouriMHD23}~\cite{NouriMHD23}\\
\rowlabel{auth:a982}Ulrich Domdorf & 1 &0 &\href{}{DomdorfPH03}~\cite{DomdorfPH03}\\
\rowlabel{auth:a999}Wolfgang Domschke & 1 &344 &\href{}{BlazewiczDP96}~\cite{BlazewiczDP96}\\
\rowlabel{auth:a365}Gr{\'{e}}goire Dooms & 1 &1 &\href{works/DoomsH08.pdf}{DoomsH08}~\cite{DoomsH08}\\
\rowlabel{auth:a30}Agostino Dovier & 1 &0 &\href{works/TardivoDFMP23.pdf}{TardivoDFMP23}~\cite{TardivoDFMP23}\\
\rowlabel{auth:a517}Yuquan Du & 1 &27 &\href{works/QinDCS20.pdf}{QinDCS20}~\cite{QinDCS20}\\
\rowlabel{auth:a271}Lei Duan & 1 &2 &\href{works/DilkinaDH05.pdf}{DilkinaDH05}~\cite{DilkinaDH05}\\
\rowlabel{auth:a890}Alexandre Duarte {de Almeida} Lemos & 1 &0 &\href{works/Lemos21.pdf}{Lemos21}~\cite{Lemos21}\\
\rowlabel{auth:a268}Didier Dubois & 1 &13 &\href{works/FortinZDF05.pdf}{FortinZDF05}~\cite{FortinZDF05}\\
\rowlabel{auth:a374}Pierre Dupont & 1 &0 &\href{works/MonetteDD07.pdf}{MonetteDD07}~\cite{MonetteDD07}\\
\rowlabel{auth:a608}David Duvivier & 1 &36 &\href{works/WangMD15.pdf}{WangMD15}~\cite{WangMD15}\\
\rowlabel{auth:a209}Kyle E. C. Booth & 1 &21 &\href{works/BoothNB16.pdf}{BoothNB16}~\cite{BoothNB16}\\
\rowlabel{auth:a358}Marco E. L{\"{u}}bbecke & 1 &28 &\href{works/BertholdHLMS10.pdf}{BertholdHLMS10}~\cite{BertholdHLMS10}\\
\rowlabel{auth:a685}Andrew E. Santosa & 1 &0 &\href{works/ZhuS02.pdf}{ZhuS02}~\cite{ZhuS02}\\
\rowlabel{auth:a782}Martha E. Pollack & 1 &0 &\href{works/MoffittPP05.pdf}{MoffittPP05}~\cite{MoffittPP05}\\
\rowlabel{auth:a1004}Kyle E.C. Booth & 1 &24 &\href{}{RoshanaeiBAUB20}~\cite{RoshanaeiBAUB20}\\
\rowlabel{auth:a18}Nikolaos Efthymiou & 1 &0 &\href{works/EfthymiouY23.pdf}{EfthymiouY23}~\cite{EfthymiouY23}\\
\rowlabel{auth:a572}Gokhan Egilmez & 1 &43 &\href{works/GedikKEK18.pdf}{GedikKEK18}~\cite{GedikKEK18}\\
\rowlabel{auth:a280}P{\'{e}}ter Egri & 1 &2 &\href{works/KovacsEKV05.pdf}{KovacsEKV05}~\cite{KovacsEKV05}\\
\rowlabel{auth:a625}Nizar El Hachemi & 1 &32 &\href{works/HachemiGR11.pdf}{HachemiGR11}~\cite{HachemiGR11}\\
\rowlabel{auth:a654}Ghada El Khayat & 1 &84 &\href{works/KhayatLR06.pdf}{KhayatLR06}~\cite{KhayatLR06}\\
\rowlabel{auth:a763}Abdellah El Fallahi & 1 &0 &\href{works/FallahiAC20.pdf}{FallahiAC20}~\cite{FallahiAC20}\\
\rowlabel{auth:a952}\"{O}zg\"{u}n El\c{c}i & 1 &2 &\href{}{ElciOH22}~\cite{ElciOH22}\\
\rowlabel{auth:a979}Simon Emde & 1 &0 &\href{works/EmdeZD22.pdf}{EmdeZD22}~\cite{EmdeZD22}\\
\rowlabel{auth:a426}Ey{\"{u}}p Ensar Isik & 1 &0 &\href{works/IsikYA23.pdf}{IsikYA23}~\cite{IsikYA23}\\
\rowlabel{auth:a532}Teresa Escobet & 1 &17 &\href{works/EscobetPQPRA19.pdf}{EscobetPQPRA19}~\cite{EscobetPQPRA19}\\
\rowlabel{auth:a234}Joan Espasa & 1 &3 &\href{works/BofillEGPSV14.pdf}{BofillEGPSV14}~\cite{BofillEGPSV14}\\
\rowlabel{auth:a919}Alireza Etminaniesfahani & 1 &0 &\href{works/EtminaniesfahaniGNMS22.pdf}{EtminaniesfahaniGNMS22}~\cite{EtminaniesfahaniGNMS22}\\
\rowlabel{auth:a674}Michael F. Gorman & 1 &0 &\href{works/KanetAG04.pdf}{KanetAG04}~\cite{KanetAG04}\\
\rowlabel{auth:a974}Richard F. Hartl & 1 &24 &\href{works/SchnellH15.pdf}{SchnellH15}~\cite{SchnellH15}\\
\rowlabel{auth:a408}Mohd Fadlee A. Rasid & 1 &0 &\href{works/AkramNHRSA23.pdf}{AkramNHRSA23}~\cite{AkramNHRSA23}\\
\rowlabel{auth:a708}Fran{\c{c}}ois Fages & 1 &0 &\href{}{JourdanFRD94}~\cite{JourdanFRD94}\\
\rowlabel{auth:a697}Moreno Falaschi & 1 &10 &\href{works/FalaschiGMP97.pdf}{FalaschiGMP97}~\cite{FalaschiGMP97}\\
\rowlabel{auth:a483}Huali Fan & 1 &18 &\href{works/FanXG21.pdf}{FanXG21}~\cite{FanXG21}\\
\rowlabel{auth:a269}H{\'{e}}l{\`{e}}ne Fargier & 1 &13 &\href{works/FortinZDF05.pdf}{FortinZDF05}~\cite{FortinZDF05}\\
\rowlabel{auth:a744}Soroush Fatemi-Anaraki & 1 &7 &\href{}{Fatemi-AnarakiTFV23}~\cite{Fatemi-AnarakiTFV23}\\
\rowlabel{auth:a785}Filippo Focacci & 1 &0 &\href{works/FocacciLN00.pdf}{FocacciLN00}~\cite{FocacciLN00}\\
\rowlabel{auth:a321}Daniel Fontaine & 1 &3 &\href{works/FontaineMH16.pdf}{FontaineMH16}~\cite{FontaineMH16}\\
\rowlabel{auth:a106}Urs Fontana & 1 &2 &\href{works/KoehlerBFFHPSSS21.pdf}{KoehlerBFFHPSSS21}~\cite{KoehlerBFFHPSSS21}\\
\rowlabel{auth:a1010}M.A. Forbes & 1 &0 &\href{works/ForbesHJST24.pdf}{ForbesHJST24}~\cite{ForbesHJST24}\\
\rowlabel{auth:a31}Andrea Formisano & 1 &0 &\href{works/TardivoDFMP23.pdf}{TardivoDFMP23}~\cite{TardivoDFMP23}\\
\rowlabel{auth:a266}J{\'{e}}r{\^{o}}me Fortin & 1 &13 &\href{works/FortinZDF05.pdf}{FortinZDF05}~\cite{FortinZDF05}\\
\rowlabel{auth:a746}Mehdi Foumani & 1 &7 &\href{}{Fatemi-AnarakiTFV23}~\cite{Fatemi-AnarakiTFV23}\\
\rowlabel{auth:a612}Gerhard Friedrich & 1 &3 &\href{}{FriedrichFMRSST14}~\cite{FriedrichFMRSST14}\\
\rowlabel{auth:a95}Sara Frimodig & 1 &3 &\href{works/FrimodigS19.pdf}{FrimodigS19}~\cite{FrimodigS19}\\
\rowlabel{auth:a904}Aur{\'e}lien Froger & 1 &0 &\href{works/Froger16.pdf}{Froger16}~\cite{Froger16}\\
\rowlabel{auth:a544}Nikolaus Frohner & 1 &0 &\href{works/FrohnerTR19.pdf}{FrohnerTR19}~\cite{FrohnerTR19}\\
\rowlabel{auth:a302}Daniel Frost & 1 &10 &\href{works/FrostD98.pdf}{FrostD98}~\cite{FrostD98}\\
\rowlabel{auth:a613}Melanie Fr{\"{u}}hst{\"{u}}ck & 1 &3 &\href{}{FriedrichFMRSST14}~\cite{FriedrichFMRSST14}\\
\rowlabel{auth:a468}Jun Fu & 1 &0 &\href{works/LiFJZLL22.pdf}{LiFJZLL22}~\cite{LiFJZLL22}\\
\rowlabel{auth:a107}Etienne Fux & 1 &2 &\href{works/KoehlerBFFHPSSS21.pdf}{KoehlerBFFHPSSS21}~\cite{KoehlerBFFHPSSS21}\\
\rowlabel{auth:a513}Ernesto G. Birgin & 1 &30 &\href{works/LunardiBLRV20.pdf}{LunardiBLRV20}~\cite{LunardiBLRV20}\\
\rowlabel{auth:a40}Mohamed Gaha & 1 &0 &\href{works/PopovicCGNC22.pdf}{PopovicCGNC22}~\cite{PopovicCGNC22}\\
\rowlabel{auth:a695}Flavius Galiber III & 1 &26 &\href{works/PembertonG98.pdf}{PembertonG98}~\cite{PembertonG98}\\
\rowlabel{auth:a96}Cristian Galleguillos & 1 &1 &\href{works/GalleguillosKSB19.pdf}{GalleguillosKSB19}~\cite{GalleguillosKSB19}\\
\rowlabel{auth:a794}Xavier Gandibleux & 1 &0 &\href{works/RodriguezDG02.pdf}{RodriguezDG02}~\cite{RodriguezDG02}\\
\rowlabel{auth:a187}Graeme Gange & 1 &6 &\href{works/He0GLW18.pdf}{He0GLW18}~\cite{He0GLW18}\\
\rowlabel{auth:a449}Thierry Garaix & 1 &4 &\href{works/BourreauGGLT22.pdf}{BourreauGGLT22}~\cite{BourreauGGLT22}\\
\rowlabel{auth:a808}Maria Garcia de la Banda & 1 &24 &\href{works/BandaSC11.pdf}{BandaSC11}~\cite{BandaSC11}\\
\rowlabel{auth:a256}Antoine Gargani & 1 &17 &\href{works/GarganiR07.pdf}{GarganiR07}~\cite{GarganiR07}\\
\rowlabel{auth:a805}Serge Gaspers & 1 &0 &\href{works/ChuGNSW13.pdf}{ChuGNSW13}~\cite{ChuGNSW13}\\
\rowlabel{auth:a87}Jonathan Gaudreault & 1 &2 &\href{works/Mercier-AubinGQ20.pdf}{Mercier-AubinGQ20}~\cite{Mercier-AubinGQ20}\\
\rowlabel{auth:a570}Ridvan Gedik & 1 &43 &\href{works/GedikKEK18.pdf}{GedikKEK18}~\cite{GedikKEK18}\\
\rowlabel{auth:a47}Marc Geitz & 1 &0 &\href{works/GeitzGSSW22.pdf}{GeitzGSSW22}~\cite{GeitzGSSW22}\\
\rowlabel{auth:a317}Mirco Gelain & 1 &1 &\href{works/GelainPRVW17.pdf}{GelainPRVW17}~\cite{GelainPRVW17}\\
\rowlabel{auth:a626}Michel Gendreau & 1 &32 &\href{works/HachemiGR11.pdf}{HachemiGR11}~\cite{HachemiGR11}\\
\rowlabel{auth:a831}Wing{-}Yue Geoffrey Louie & 1 &16 &\href{works/LouieVNB14.pdf}{LouieVNB14}~\cite{LouieVNB14}\\
\rowlabel{auth:a443}Marcus Gerhard M{\"{u}}ller & 1 &17 &\href{works/MullerMKP22.pdf}{MullerMKP22}~\cite{MullerMKP22}\\
\rowlabel{auth:a490}Patrick Gerhards & 1 &0 &\href{works/HubnerGSV21.pdf}{HubnerGSV21}~\cite{HubnerGSV21}\\
\rowlabel{auth:a906}Grigori German & 1 &0 &\href{works/German18.pdf}{German18}~\cite{German18}\\
\rowlabel{auth:a667}Ulrich Geske & 1 &2 &\href{works/Geske05.pdf}{Geske05}~\cite{Geske05}\\
\rowlabel{auth:a1008}Shirin Ghasemi & 1 &0 &\href{}{GhasemiMH23}~\cite{GhasemiMH23}\\
\rowlabel{auth:a211}Katherine Giles & 1 &2 &\href{works/GilesH16.pdf}{GilesH16}~\cite{GilesH16}\\
\rowlabel{auth:a432}Ga{\"{e}}l Glorian & 1 &0 &\href{works/PerezGSL23.pdf}{PerezGSL23}~\cite{PerezGSL23}\\
\rowlabel{auth:a441}Gael Glorian & 1 &0 &\href{works/abs-2312-13682.pdf}{abs-2312-13682}~\cite{abs-2312-13682}\\
\rowlabel{auth:a783}Daniel Godard & 1 &0 &\href{works/GodardLN05.pdf}{GodardLN05}~\cite{GodardLN05}\\
\rowlabel{auth:a602}Vikas Goel & 1 &48 &\href{works/GoelSHFS15.pdf}{GoelSHFS15}~\cite{GoelSHFS15}\\
\rowlabel{auth:a485}Mark Goh & 1 &18 &\href{works/FanXG21.pdf}{FanXG21}~\cite{FanXG21}\\
\rowlabel{auth:a307}Hans{-}Joachim Goltz & 1 &7 &\href{works/Goltz95.pdf}{Goltz95}~\cite{Goltz95}\\
\rowlabel{auth:a450}Matthieu Gondran & 1 &4 &\href{works/BourreauGGLT22.pdf}{BourreauGGLT22}~\cite{BourreauGGLT22}\\
\rowlabel{auth:a987}Inés González-Rodríguez & 1 &0 &\href{}{AfsarVPG23}~\cite{AfsarVPG23}\\
\rowlabel{auth:a996}Marcos Goycoolea & 1 &0 &\href{}{HillBCGN22}~\cite{HillBCGN22}\\
\rowlabel{auth:a48}Cristian Grozea & 1 &0 &\href{works/GeitzGSSW22.pdf}{GeitzGSSW22}~\cite{GeitzGSSW22}\\
\rowlabel{auth:a696}Flavius Gruian & 1 &5 &\href{works/GruianK98.pdf}{GruianK98}~\cite{GruianK98}\\
\rowlabel{auth:a933}Zailin Guan & 1 &2 &\href{works/ChenGPSH10.pdf}{ChenGPSH10}~\cite{ChenGPSH10}\\
\rowlabel{auth:a382}Alessio Guerri & 1 &18 &\href{works/BeniniBGM06.pdf}{BeniniBGM06}~\cite{BeniniBGM06}\\
\rowlabel{auth:a363}Serigne Gueye & 1 &3 &\href{works/Acuna-AgostMFG09.pdf}{Acuna-AgostMFG09}~\cite{Acuna-AgostMFG09}\\
\rowlabel{auth:a610}Ying Guo & 1 &0 &\href{works/ZhouGL15.pdf}{ZhouGL15}~\cite{ZhouGL15}\\
\rowlabel{auth:a953}Peng Guo & 1 &8 &\href{}{GuoHLW20}~\cite{GuoHLW20}\\
\rowlabel{auth:a965}Penghui Guo & 1 &0 &\href{}{GuoZ23}~\cite{GuoZ23}\\
\rowlabel{auth:a1000}Olivier Guyon & 1 &32 &\href{works/GuyonLPR12.pdf}{GuyonLPR12}~\cite{GuyonLPR12}\\
\rowlabel{auth:a773}Şeyda G{\"u}r & 1 &0 &\href{works/GurEA19.pdf}{GurEA19}~\cite{GurEA19}\\
\rowlabel{auth:a579}Burak G{\"{o}}kg{\"{u}}r & 1 &31 &\href{works/GokgurHO18.pdf}{GokgurHO18}~\cite{GokgurHO18}\\
\rowlabel{auth:a418}Seyda G{\"{u}}r & 1 &1 &\href{works/GurPAE23.pdf}{GurPAE23}~\cite{GurPAE23}\\
\rowlabel{auth:a716}Fehmi H'Mida & 1 &11 &\href{works/TrojetHL11.pdf}{TrojetHL11}~\cite{TrojetHL11}\\
\rowlabel{auth:a359}Rolf H. M{\"{o}}hring & 1 &28 &\href{works/BertholdHLMS10.pdf}{BertholdHLMS10}~\cite{BertholdHLMS10}\\
\rowlabel{auth:a575}John H. Drake & 1 &41 &\href{works/PourDERB18.pdf}{PourDERB18}~\cite{PourDERB18}\\
\rowlabel{auth:a599}M. H. Fazel Zarandi & 1 &28 &\href{works/ZarandiKS16.pdf}{ZarandiKS16}~\cite{ZarandiKS16}\\
\rowlabel{auth:a776}Klaus H. Ecker & 1 &38 &\href{}{BlazewiczEP19}~\cite{BlazewiczEP19}\\
\rowlabel{auth:a787}Emile H. L. Aarts & 1 &0 &\href{works/NuijtenA94.pdf}{NuijtenA94}~\cite{NuijtenA94}\\
\rowlabel{auth:a924}Tarik Hadzic & 1 &3 &\href{works/SimonisH11.pdf}{SimonisH11}~\cite{SimonisH11}\\
\rowlabel{auth:a1009}Mahdi Hamid & 1 &0 &\href{}{GhasemiMH23}~\cite{GhasemiMH23}\\
\rowlabel{auth:a71}Claire Hanen & 1 &1 &\href{works/HanenKP21.pdf}{HanenKP21}~\cite{HanenKP21}\\
\rowlabel{auth:a518}Jiang Hang Chen & 1 &27 &\href{works/QinDCS20.pdf}{QinDCS20}~\cite{QinDCS20}\\
\rowlabel{auth:a260}Sue Hanhilammi & 1 &2 &\href{works/KrogtLPHJ07.pdf}{KrogtLPHJ07}~\cite{KrogtLPHJ07}\\
\rowlabel{auth:a968}Zdeněk Hanzálek & 1 &2 &\href{}{NouriMHD23}~\cite{NouriMHD23}\\
\rowlabel{auth:a356}Mohamed Haouari & 1 &3 &\href{works/LahimerLH11.pdf}{LahimerLH11}~\cite{LahimerLH11}\\
\rowlabel{auth:a1011}M.G. Harris & 1 &0 &\href{works/ForbesHJST24.pdf}{ForbesHJST24}~\cite{ForbesHJST24}\\
\rowlabel{auth:a407}Fazirulhisyam Hashim & 1 &0 &\href{works/AkramNHRSA23.pdf}{AkramNHRSA23}~\cite{AkramNHRSA23}\\
\rowlabel{auth:a936}Muhammad Hasseb & 1 &2 &\href{works/ChenGPSH10.pdf}{ChenGPSH10}~\cite{ChenGPSH10}\\
\rowlabel{auth:a186}Shan He & 1 &6 &\href{works/He0GLW18.pdf}{He0GLW18}~\cite{He0GLW18}\\
\rowlabel{auth:a954}Xun He & 1 &8 &\href{}{GuoHLW20}~\cite{GuoHLW20}\\
\rowlabel{auth:a835}Ivan Heckman & 1 &0 &\href{works/HeckmanB11.pdf}{HeckmanB11}~\cite{HeckmanB11}\\
\rowlabel{auth:a169}Susanne Heipcke & 1 &5 &\href{works/HeipckeCCS00.pdf}{HeipckeCCS00}~\cite{HeipckeCCS00}\\
\rowlabel{auth:a245}Fabien Hermenier & 1 &28 &\href{works/HermenierDL11.pdf}{HermenierDL11}~\cite{HermenierDL11}\\
\rowlabel{auth:a346}Gerhard Hiermann & 1 &14 &\href{works/RendlPHPR12.pdf}{RendlPHPR12}~\cite{RendlPHPR12}\\
\rowlabel{auth:a589}Te{-}Wei Ho & 1 &2 &\href{works/HoYCLLCLC18.pdf}{HoYCLLCLC18}~\cite{HoYCLLCLC18}\\
\rowlabel{auth:a553}Petra Hofstedt & 1 &1 &\href{works/LiuLH19.pdf}{LiuLH19}~\cite{LiuLH19}\\
\rowlabel{auth:a1023}Mark{\'{o}} Horv{\'{a}}th & 1 &5 &\href{works/NattafHKAL19.pdf}{NattafHKAL19}~\cite{NattafHKAL19}\\
\rowlabel{auth:a841}Mohammad Hossein Fazel Zarandi & 1 &55 &\href{works/ZarandiASC20.pdf}{ZarandiASC20}~\cite{ZarandiASC20}\\
\rowlabel{auth:a255}John Hou & 1 &1 &\href{works/DavenportKRSH07.pdf}{DavenportKRSH07}~\cite{DavenportKRSH07}\\
\rowlabel{auth:a818}Guoyu Huang & 1 &1 &\href{works/CohenHB17.pdf}{CohenHB17}~\cite{CohenHB17}\\
\rowlabel{auth:a900}Barry Hurley & 1 &0 &\href{works/HurleyOS16.pdf}{HurleyOS16}~\cite{HurleyOS16}\\
\rowlabel{auth:a489}Felix H{\"{u}}bner & 1 &0 &\href{works/HubnerGSV21.pdf}{HubnerGSV21}~\cite{HubnerGSV21}\\
\rowlabel{auth:a970}Ayoub Insa Corréa & 1 &106 &\href{}{CorreaLR07}~\cite{CorreaLR07}\\
\rowlabel{auth:a177}Amar Isli & 1 &3 &\href{works/BelhadjiI98.pdf}{BelhadjiI98}~\cite{BelhadjiI98}\\
\rowlabel{auth:a409}Mustafa Ismael Salman & 1 &0 &\href{works/AkramNHRSA23.pdf}{AkramNHRSA23}~\cite{AkramNHRSA23}\\
\rowlabel{auth:a244}Fernando J. M. Marcellino & 1 &0 &\href{works/SerraNM12.pdf}{SerraNM12}~\cite{SerraNM12}\\
\rowlabel{auth:a587}Leon J. Osterweil & 1 &9 &\href{works/ShinBBHO18.pdf}{ShinBBHO18}~\cite{ShinBBHO18}\\
\rowlabel{auth:a660}H. J. Kim & 1 &12 &\href{works/SureshMOK06.pdf}{SureshMOK06}~\cite{SureshMOK06}\\
\rowlabel{auth:a672}John J. Kanet & 1 &0 &\href{works/KanetAG04.pdf}{KanetAG04}~\cite{KanetAG04}\\
\rowlabel{auth:a680}Colin J. Layfield & 1 &0 &\href{works/Layfield02.pdf}{Layfield02}~\cite{Layfield02}\\
\rowlabel{auth:a689}Andrew J. Mason & 1 &5 &\href{works/Mason01.pdf}{Mason01}~\cite{Mason01}\\
\rowlabel{auth:a909}Steven J. Edwards & 1 &3 &\href{}{EdwardsBSE19}~\cite{EdwardsBSE19}\\
\rowlabel{auth:a941}Ronald J. Wilcox & 1 &71 &\href{}{GombolayWS18}~\cite{GombolayWS18}\\
\rowlabel{auth:a994}Andrea J. Brickey & 1 &0 &\href{}{HillBCGN22}~\cite{HillBCGN22}\\
\rowlabel{auth:a857}Vipul Jain & 1 &279 &\href{works/JainG01.pdf}{JainG01}~\cite{JainG01}\\
\rowlabel{auth:a977}A.S. Jain & 1 &490 &\href{}{JainM99}~\cite{JainM99}\\
\rowlabel{auth:a1012}H.M. Jansen & 1 &0 &\href{works/ForbesHJST24.pdf}{ForbesHJST24}~\cite{ForbesHJST24}\\
\rowlabel{auth:a225}Jean Jaubert & 1 &0 &\href{works/PraletLJ15.pdf}{PraletLJ15}~\cite{PraletLJ15}\\
\rowlabel{auth:a789}Jan Jel{\'{\i}}nek & 1 &0 &\href{works/JelinekB16.pdf}{JelinekB16}~\cite{JelinekB16}\\
\rowlabel{auth:a474}Yingjun Ji & 1 &0 &\href{works/ZhangJZL22.pdf}{ZhangJZL22}~\cite{ZhangJZL22}\\
\rowlabel{auth:a469}Zixi Jia & 1 &0 &\href{works/LiFJZLL22.pdf}{LiFJZLL22}~\cite{LiFJZLL22}\\
\rowlabel{auth:a665}Yunfei Jiang & 1 &0 &\href{works/LiuJ06.pdf}{LiuJ06}~\cite{LiuJ06}\\
\rowlabel{auth:a261}Yue Jin & 1 &2 &\href{works/KrogtLPHJ07.pdf}{KrogtLPHJ07}~\cite{KrogtLPHJ07}\\
\rowlabel{auth:a351}Marc Joliveau & 1 &5 &\href{works/ChapadosJR11.pdf}{ChapadosJR11}~\cite{ChapadosJR11}\\
\rowlabel{auth:a299}Peter Jonsson & 1 &1 &\href{works/AngelsmarkJ00.pdf}{AngelsmarkJ00}~\cite{AngelsmarkJ00}\\
\rowlabel{auth:a986}Juan José Palacios & 1 &0 &\href{}{AfsarVPG23}~\cite{AfsarVPG23}\\
\rowlabel{auth:a949}Antoine Jouglet & 1 &3 &\href{}{CarlierSJP21}~\cite{CarlierSJP21}\\
\rowlabel{auth:a707}Jean Jourdan & 1 &0 &\href{}{JourdanFRD94}~\cite{JourdanFRD94}\\
\rowlabel{auth:a803}Nicolas Jozefowiez & 1 &9 &\href{works/HebrardHJMPV16.pdf}{HebrardHJMPV16}~\cite{HebrardHJMPV16}\\
\rowlabel{auth:a556}Jae{-}Yoon Jung & 1 &1 &\href{works/ParkUJR19.pdf}{ParkUJR19}~\cite{ParkUJR19}\\
\rowlabel{auth:a750}Pascal Jungblut & 1 &0 &\href{works/JungblutK22.pdf}{JungblutK22}~\cite{JungblutK22}\\
\rowlabel{auth:a289}T. K. Satish Kumar & 1 &4 &\href{works/Kumar03.pdf}{Kumar03}~\cite{Kumar03}\\
\rowlabel{auth:a578}Edmund K. Burke & 1 &41 &\href{works/PourDERB18.pdf}{PourDERB18}~\cite{PourDERB18}\\
\rowlabel{auth:a832}Mustafa K. Dogru & 1 &8 &\href{works/TerekhovDOB12.pdf}{TerekhovDOB12}~\cite{TerekhovDOB12}\\
\rowlabel{auth:a834}T. K. Feng & 1 &43 &\href{works/BeckFW11.pdf}{BeckFW11}~\cite{BeckFW11}\\
\rowlabel{auth:a252}Jayant Kalagnanam & 1 &1 &\href{works/DavenportKRSH07.pdf}{DavenportKRSH07}~\cite{DavenportKRSH07}\\
\rowlabel{auth:a571}Darshan Kalathia & 1 &43 &\href{works/GedikKEK18.pdf}{GedikKEK18}~\cite{GedikKEK18}\\
\rowlabel{auth:a293}Olli Kamarainen & 1 &9 &\href{works/KamarainenS02.pdf}{KamarainenS02}~\cite{KamarainenS02}\\
\rowlabel{auth:a406}Nor Kamariah Noordin & 1 &0 &\href{works/AkramNHRSA23.pdf}{AkramNHRSA23}~\cite{AkramNHRSA23}\\
\rowlabel{auth:a902}Philip Kay & 1 &11 &\href{works/SimonisCK00.pdf}{SimonisCK00}~\cite{SimonisCK00}\\
\rowlabel{auth:a338}Elena Kelareva & 1 &16 &\href{works/KelarevaTK13.pdf}{KelarevaTK13}~\cite{KelarevaTK13}\\
\rowlabel{auth:a628}Jan Kelbel & 1 &12 &\href{works/KelbelH11.pdf}{KelbelH11}~\cite{KelbelH11}\\
\rowlabel{auth:a600}H. Khorshidian & 1 &28 &\href{works/ZarandiKS16.pdf}{ZarandiKS16}~\cite{ZarandiKS16}\\
\rowlabel{auth:a770}Kamran Kianfar & 1 &0 &\href{works/YounespourAKE19.pdf}{YounespourAKE19}~\cite{YounespourAKE19}\\
\rowlabel{auth:a340}Philip Kilby & 1 &16 &\href{works/KelarevaTK13.pdf}{KelarevaTK13}~\cite{KelarevaTK13}\\
\rowlabel{auth:a23}Dongyun Kim & 1 &0 &\href{works/KimCMLLP23.pdf}{KimCMLLP23}~\cite{KimCMLLP23}\\
\rowlabel{auth:a573}Emre Kirac & 1 &43 &\href{works/GedikKEK18.pdf}{GedikKEK18}~\cite{GedikKEK18}\\
\rowlabel{auth:a97}Zeynep Kiziltan & 1 &1 &\href{works/GalleguillosKSB19.pdf}{GalleguillosKSB19}~\cite{GalleguillosKSB19}\\
\rowlabel{auth:a67}Christian Klanke & 1 &3 &\href{works/KlankeBYE21.pdf}{KlankeBYE21}~\cite{KlankeBYE21}\\
\rowlabel{auth:a104}Jana Koehler & 1 &2 &\href{works/KoehlerBFFHPSSS21.pdf}{KoehlerBFFHPSSS21}~\cite{KoehlerBFFHPSSS21}\\
\rowlabel{auth:a59}Wolfgang Kohlenbrein & 1 &0 &\href{works/KovacsTKSG21.pdf}{KovacsTKSG21}~\cite{KovacsTKSG21}\\
\rowlabel{auth:a447}Rainer Kolisch & 1 &4 &\href{works/PohlAK22.pdf}{PohlAK22}~\cite{PohlAK22}\\
\rowlabel{auth:a333}Sebastian Kosch & 1 &4 &\href{works/KoschB14.pdf}{KoschB14}~\cite{KoschB14}\\
\rowlabel{auth:a57}Benjamin Kov{\'{a}}cs & 1 &0 &\href{works/KovacsTKSG21.pdf}{KovacsTKSG21}~\cite{KovacsTKSG21}\\
\rowlabel{auth:a79}Matthias Krainz & 1 &0 &\href{works/GeibingerKKMMW21.pdf}{GeibingerKKMMW21}~\cite{GeibingerKKMMW21}\\
\rowlabel{auth:a713}Andreas Krall & 1 &14 &\href{works/ErtlK91.pdf}{ErtlK91}~\cite{ErtlK91}\\
\rowlabel{auth:a751}Dieter Kranzlm{\"{u}}ller & 1 &0 &\href{works/JungblutK22.pdf}{JungblutK22}~\cite{JungblutK22}\\
\rowlabel{auth:a444}Dominik Kress & 1 &17 &\href{works/MullerMKP22.pdf}{MullerMKP22}~\cite{MullerMKP22}\\
\rowlabel{auth:a719}Per Kreuger & 1 &0 &\href{works/AronssonBK09.pdf}{AronssonBK09}~\cite{AronssonBK09}\\
\rowlabel{auth:a772}Mustafa K{\"u}ç{\"u}k & 1 &0 &\href{works/KucukY19.pdf}{KucukY19}~\cite{KucukY19}\\
\rowlabel{auth:a386}Elif K{\"{u}}rkl{\"{u}} & 1 &4 &\href{works/FrankK05.pdf}{FrankK05}~\cite{FrankK05}\\
\rowlabel{auth:a373}Andr{\'{a}}s K{\'{e}}ri & 1 &1 &\href{works/KeriK07.pdf}{KeriK07}~\cite{KeriK07}\\
\rowlabel{auth:a28}Michael L. Pinedo & 1 &0 &\href{works/KimCMLLP23.pdf}{KimCMLLP23}~\cite{KimCMLLP23}\\
\rowlabel{auth:a214}Hassan L. Hijazi & 1 &2 &\href{works/LimHTB16.pdf}{LimHTB16}~\cite{LimHTB16}\\
\rowlabel{auth:a586}Philip L. Henneman & 1 &9 &\href{works/ShinBBHO18.pdf}{ShinBBHO18}~\cite{ShinBBHO18}\\
\rowlabel{auth:a755}Yiqing L. Luo & 1 &0 &\href{works/LuoB22.pdf}{LuoB22}~\cite{LuoB22}\\
\rowlabel{auth:a451}Philippe Lacomme & 1 &4 &\href{works/BourreauGGLT22.pdf}{BourreauGGLT22}~\cite{BourreauGGLT22}\\
\rowlabel{auth:a36}Daniel Lafond & 1 &0 &\href{works/BoudreaultSLQ22.pdf}{BoudreaultSLQ22}~\cite{BoudreaultSLQ22}\\
\rowlabel{auth:a1028}Anne{-}Marie Lagrange & 1 &0 &\href{works/CatusseCBL16.pdf}{CatusseCBL16}~\cite{CatusseCBL16}\\
\rowlabel{auth:a355}Asma Lahimer & 1 &3 &\href{works/LahimerLH11.pdf}{LahimerLH11}~\cite{LahimerLH11}\\
\rowlabel{auth:a592}Feipei Lai & 1 &2 &\href{works/HoYCLLCLC18.pdf}{HoYCLLCLC18}~\cite{HoYCLLCLC18}\\
\rowlabel{auth:a593}Jui{-}Fen Lai & 1 &2 &\href{works/HoYCLLCLC18.pdf}{HoYCLLCLC18}~\cite{HoYCLLCLC18}\\
\rowlabel{auth:a971}André Langevin & 1 &106 &\href{}{CorreaLR07}~\cite{CorreaLR07}\\
\rowlabel{auth:a945}Alexander Lazarev & 1 &12 &\href{}{ArkhipovBL19}~\cite{ArkhipovBL19}\\
\rowlabel{auth:a219}Christophe Lecoutre & 1 &20 &\href{works/GayHLS15.pdf}{GayHLS15}~\cite{GayHLS15}\\
\rowlabel{auth:a26}Myungho Lee & 1 &0 &\href{works/KimCMLLP23.pdf}{KimCMLLP23}~\cite{KimCMLLP23}\\
\rowlabel{auth:a27}Kangbok Lee & 1 &0 &\href{works/KimCMLLP23.pdf}{KimCMLLP23}~\cite{KimCMLLP23}\\
\rowlabel{auth:a224}Solange Lemai{-}Chenevier & 1 &0 &\href{works/PraletLJ15.pdf}{PraletLJ15}~\cite{PraletLJ15}\\
\rowlabel{auth:a467}Xingyang Li & 1 &0 &\href{works/LiFJZLL22.pdf}{LiFJZLL22}~\cite{LiFJZLL22}\\
\rowlabel{auth:a471}Siyi Li & 1 &0 &\href{works/LiFJZLL22.pdf}{LiFJZLL22}~\cite{LiFJZLL22}\\
\rowlabel{auth:a475}Xiaodong Li & 1 &0 &\href{works/abs-2211-14492.pdf}{abs-2211-14492}~\cite{abs-2211-14492}\\
\rowlabel{auth:a611}Guipeng Li & 1 &0 &\href{works/ZhouGL15.pdf}{ZhouGL15}~\cite{ZhouGL15}\\
\rowlabel{auth:a634}Hong Li & 1 &4 &\href{works/SunLYL10.pdf}{SunLYL10}~\cite{SunLYL10}\\
\rowlabel{auth:a636}Nan Li & 1 &4 &\href{works/SunLYL10.pdf}{SunLYL10}~\cite{SunLYL10}\\
\rowlabel{auth:a723}Yunbo Li & 1 &1 &\href{works/Madi-WambaLOBM17.pdf}{Madi-WambaLOBM17}~\cite{Madi-WambaLOBM17}\\
\rowlabel{auth:a814}Heyse Li & 1 &8 &\href{works/TranPZLDB18.pdf}{TranPZLDB18}~\cite{TranPZLDB18}\\
\rowlabel{auth:a827}Yi Li & 1 &0 &\href{works/LuoVLBM16.pdf}{LuoVLBM16}~\cite{LuoVLBM16}\\
\rowlabel{auth:a975}Haitao Li & 1 &113 &\href{works/LiW08.pdf}{LiW08}~\cite{LiW08}\\
\rowlabel{auth:a595}Wan{-}Chung Liao & 1 &2 &\href{works/HoYCLLCLC18.pdf}{HoYCLLCLC18}~\cite{HoYCLLCLC18}\\
\rowlabel{auth:a188}Ariel Liebman & 1 &6 &\href{works/He0GLW18.pdf}{He0GLW18}~\cite{He0GLW18}\\
\rowlabel{auth:a649}Olivier Liess & 1 &22 &\href{works/LiessM08.pdf}{LiessM08}~\cite{LiessM08}\\
\rowlabel{auth:a282}Andrew Lim & 1 &5 &\href{works/LimRX04.pdf}{LimRX04}~\cite{LimRX04}\\
\rowlabel{auth:a196}Tong Liu & 1 &0 &\href{works/LiuCGM17.pdf}{LiuCGM17}~\cite{LiuCGM17}\\
\rowlabel{auth:a496}Lingxuan Liu & 1 &12 &\href{works/QinWSLS21.pdf}{QinWSLS21}~\cite{QinWSLS21}\\
\rowlabel{auth:a551}Ke Liu & 1 &1 &\href{works/LiuLH19.pdf}{LiuLH19}~\cite{LiuLH19}\\
\rowlabel{auth:a566}Rengkui Liu & 1 &24 &\href{works/TangLWSK18.pdf}{TangLWSK18}~\cite{TangLWSK18}\\
\rowlabel{auth:a664}Yuechang Liu & 1 &0 &\href{works/LiuJ06.pdf}{LiuJ06}~\cite{LiuJ06}\\
\rowlabel{auth:a810}Giovanni Lo Bianco & 1 &0 &\href{works/ZhangBB22.pdf}{ZhangBB22}~\cite{ZhangBB22}\\
\rowlabel{auth:a550}Doina Logofatu & 1 &2 &\href{works/BadicaBIL19.pdf}{BadicaBIL19}~\cite{BadicaBIL19}\\
\rowlabel{auth:a681}Thomas Lorigeon & 1 &22 &\href{works/LorigeonBB02.pdf}{LorigeonBB02}~\cite{LorigeonBB02}\\
\rowlabel{auth:a955}Yulin Luan & 1 &8 &\href{}{GuoHLW20}~\cite{GuoHLW20}\\
\rowlabel{auth:a825}Roy Luo & 1 &0 &\href{works/LuoVLBM16.pdf}{LuoVLBM16}~\cite{LuoVLBM16}\\
\rowlabel{auth:a797}Arnaud Lusson & 1 &0 &\href{works/HebrardALLCMR22.pdf}{HebrardALLCMR22}~\cite{HebrardALLCMR22}\\
\rowlabel{auth:a511}Chang Lv & 1 &100 &\href{works/MengZRZL20.pdf}{MengZRZL20}~\cite{MengZRZL20}\\
\rowlabel{auth:a622}Zhimin Lv & 1 &1 &\href{works/ZhangLS12.pdf}{ZhangLS12}~\cite{ZhangLS12}\\
\rowlabel{auth:a552}Sven L{\"{o}}ffler & 1 &1 &\href{works/LiuLH19.pdf}{LiuLH19}~\cite{LiuLH19}\\
\rowlabel{auth:a378}J. M. van den Akker & 1 &2 &\href{works/AkkerDH07.pdf}{AkkerDH07}~\cite{AkkerDH07}\\
\rowlabel{auth:a410}Abdulrahman M. Abdulghani & 1 &0 &\href{works/AkramNHRSA23.pdf}{AkramNHRSA23}~\cite{AkramNHRSA23}\\
\rowlabel{auth:a558}O. M. Alade & 1 &0 &\href{works/abs-1902-01193.pdf}{abs-1902-01193}~\cite{abs-1902-01193}\\
\rowlabel{auth:a574}Shahrzad M. Pour & 1 &41 &\href{works/PourDERB18.pdf}{PourDERB18}~\cite{PourDERB18}\\
\rowlabel{auth:a597}Franco M. Novara & 1 &18 &\href{works/NovaraNH16.pdf}{NovaraNH16}~\cite{NovaraNH16}\\
\rowlabel{auth:a678}Rafael M. Gasca & 1 &7 &\href{works/ValleMGT03.pdf}{ValleMGT03}~\cite{ValleMGT03}\\
\rowlabel{auth:a846}Jose M. Framinan & 1 &0 &\href{}{AbreuPNF23}~\cite{AbreuPNF23}\\
\rowlabel{auth:a888}Andy M. Ham & 1 &50 &\href{works/HamC16.pdf}{HamC16}~\cite{HamC16}\\
\rowlabel{auth:a967}Mohammad M. Fazel-Zarandi & 1 &38 &\href{}{ZarandiB12}~\cite{ZarandiB12}\\
\rowlabel{auth:a638}Jun Ma & 1 &1 &\href{works/MakMS10.pdf}{MakMS10}~\cite{MakMS10}\\
\rowlabel{auth:a368}Amy Mainville Cohn & 1 &1 &\href{works/BarlattCG08.pdf}{BarlattCG08}~\cite{BarlattCG08}\\
\rowlabel{auth:a637}Kai{-}Ling Mak & 1 &1 &\href{works/MakMS10.pdf}{MakMS10}~\cite{MakMS10}\\
\rowlabel{auth:a658}V. Mani & 1 &12 &\href{works/SureshMOK06.pdf}{SureshMOK06}~\cite{SureshMOK06}\\
\rowlabel{auth:a222}Oscar Manzano & 1 &1 &\href{works/MurphyMB15.pdf}{MurphyMB15}~\cite{MurphyMB15}\\
\rowlabel{auth:a925}Christos Maravelias & 1 &0 &\href{}{AggounMV08}~\cite{AggounMV08}\\
\rowlabel{auth:a577}Kourosh Marjani Rasmussen & 1 &41 &\href{works/PourDERB18.pdf}{PourDERB18}~\cite{PourDERB18}\\
\rowlabel{auth:a698}Kim Marriott & 1 &10 &\href{works/FalaschiGMP97.pdf}{FalaschiGMP97}~\cite{FalaschiGMP97}\\
\rowlabel{auth:a686}Fae Martin & 1 &11 &\href{works/MartinPY01.pdf}{MartinPY01}~\cite{MartinPY01}\\
\rowlabel{auth:a651}Jim McInnes & 1 &15 &\href{works/MalikMB08.pdf}{MalikMB08}~\cite{MalikMB08}\\
\rowlabel{auth:a978}S. Meeran & 1 &490 &\href{}{JainM99}~\cite{JainM99}\\
\rowlabel{auth:a435}Zahra Mehdizadeh{-}Somarin & 1 &0 &\href{works/Mehdizadeh-Somarin23.pdf}{Mehdizadeh-Somarin23}~\cite{Mehdizadeh-Somarin23}\\
\rowlabel{auth:a420}Haci Mehmet Alakas & 1 &1 &\href{works/GurPAE23.pdf}{GurPAE23}~\cite{GurPAE23}\\
\rowlabel{auth:a774}Hacı Mehmet Alakaş & 1 &0 &\href{works/GurEA19.pdf}{GurEA19}~\cite{GurEA19}\\
\rowlabel{auth:a44}Sebastian Meiswinkel & 1 &0 &\href{works/WinterMMW22.pdf}{WinterMMW22}~\cite{WinterMMW22}\\
\rowlabel{auth:a752}Gonzalo Mejía & 1 &6 &\href{works/YuraszeckMPV22.pdf}{YuraszeckMPV22}~\cite{YuraszeckMPV22}\\
\rowlabel{auth:a203}Hein Meling & 1 &6 &\href{works/MossigeGSMC17.pdf}{MossigeGSMC17}~\cite{MossigeGSMC17}\\
\rowlabel{auth:a624}Julien Menana & 1 &0 &\href{works/Menana11.pdf}{Menana11}~\cite{Menana11}\\
\rowlabel{auth:a725}Jean{-}Marc Menaud & 1 &1 &\href{works/Madi-WambaLOBM17.pdf}{Madi-WambaLOBM17}~\cite{Madi-WambaLOBM17}\\
\rowlabel{auth:a507}Leilei Meng & 1 &100 &\href{works/MengZRZL20.pdf}{MengZRZL20}~\cite{MengZRZL20}\\
\rowlabel{auth:a864}Luc Mercier & 1 &32 &\href{works/MercierH08.pdf}{MercierH08}~\cite{MercierH08}\\
\rowlabel{auth:a86}Alexandre Mercier{-}Aubin & 1 &2 &\href{works/Mercier-AubinGQ20.pdf}{Mercier-AubinGQ20}~\cite{Mercier-AubinGQ20}\\
\rowlabel{auth:a614}Vera Mersheeva & 1 &3 &\href{}{FriedrichFMRSST14}~\cite{FriedrichFMRSST14}\\
\rowlabel{auth:a607}Nadine Meskens & 1 &36 &\href{works/WangMD15.pdf}{WangMD15}~\cite{WangMD15}\\
\rowlabel{auth:a647}Bernd Meyer & 1 &13 &\href{works/ThiruvadyBME09.pdf}{ThiruvadyBME09}~\cite{ThiruvadyBME09}\\
\rowlabel{auth:a762}Kyung Min Kim & 1 &0 &\href{works/HamPK21.pdf}{HamPK21}~\cite{HamPK21}\\
\rowlabel{auth:a181}Gautam Mitra & 1 &28 &\href{works/Darby-DowmanLMZ97.pdf}{Darby-DowmanLMZ97}~\cite{Darby-DowmanLMZ97}\\
\rowlabel{auth:a412}Elizabeth Montero & 1 &0 &\href{works/YuraszeckMCCR23.pdf}{YuraszeckMCCR23}~\cite{YuraszeckMCCR23}\\
\rowlabel{auth:a25}Kyungduk Moon & 1 &0 &\href{works/KimCMLLP23.pdf}{KimCMLLP23}~\cite{KimCMLLP23}\\
\rowlabel{auth:a920}Leila Moslemi Naeni & 1 &0 &\href{works/EtminaniesfahaniGNMS22.pdf}{EtminaniesfahaniGNMS22}~\cite{EtminaniesfahaniGNMS22}\\
\rowlabel{auth:a200}Morten Mossige & 1 &6 &\href{works/MossigeGSMC17.pdf}{MossigeGSMC17}~\cite{MossigeGSMC17}\\
\rowlabel{auth:a72}Alix Munier Kordon & 1 &1 &\href{works/HanenKP21.pdf}{HanenKP21}~\cite{HanenKP21}\\
\rowlabel{auth:a100}Stanislav Mur{\'{\i}}n & 1 &2 &\href{works/MurinR19.pdf}{MurinR19}~\cite{MurinR19}\\
\rowlabel{auth:a292}Nicola Muscettola & 1 &14 &\href{works/Muscettola02.pdf}{Muscettola02}~\cite{Muscettola02}\\
\rowlabel{auth:a442}David M{\"{u}}ller & 1 &17 &\href{works/MullerMKP22.pdf}{MullerMKP22}~\cite{MullerMKP22}\\
\rowlabel{auth:a297}Andr{\'{a}}s M{\'{a}}rkus & 1 &2 &\href{works/VanczaM01.pdf}{VanczaM01}~\cite{VanczaM01}\\
\rowlabel{auth:a335}Marc{-}Andr{\'{e}} M{\'{e}}nard & 1 &1 &\href{works/BessiereHMQW14.pdf}{BessiereHMQW14}~\cite{BessiereHMQW14}\\
\rowlabel{auth:a960}Carlos Méndez & 1 &381 &\href{}{HarjunkoskiMBC14}~\cite{HarjunkoskiMBC14}\\
\rowlabel{auth:a488}T. N. Wong & 1 &6 &\href{works/ZhangYW21.pdf}{ZhangYW21}~\cite{ZhangYW21}\\
\rowlabel{auth:a659}S. N. Omkar & 1 &12 &\href{works/SureshMOK06.pdf}{SureshMOK06}~\cite{SureshMOK06}\\
\rowlabel{auth:a806}Nina Narodytska & 1 &0 &\href{works/ChuGNSW13.pdf}{ChuGNSW13}~\cite{ChuGNSW13}\\
\rowlabel{auth:a238}Shiva Nejati & 1 &3 &\href{works/AlesioNBG14.pdf}{AlesioNBG14}~\cite{AlesioNBG14}\\
\rowlabel{auth:a997}Alexandra Newman & 1 &0 &\href{}{HillBCGN22}~\cite{HillBCGN22}\\
\rowlabel{auth:a41}Franklin Nguewouo & 1 &0 &\href{works/PopovicCGNC22.pdf}{PopovicCGNC22}~\cite{PopovicCGNC22}\\
\rowlabel{auth:a243}Gilberto Nishioka & 1 &0 &\href{works/SerraNM12.pdf}{SerraNM12}~\cite{SerraNM12}\\
\rowlabel{auth:a12}Thierry Noulamo & 1 &0 &\href{works/KameugneFND23.pdf}{KameugneFND23}~\cite{KameugneFND23}\\
\rowlabel{auth:a1005}W.P.M. Nuijten & 1 &65 &\href{}{NuijtenA96}~\cite{NuijtenA96}\\
\rowlabel{auth:a663}Jari Nurmi & 1 &2 &\href{works/QuSN06.pdf}{QuSN06}~\cite{QuSN06}\\
\rowlabel{auth:a917}Emmanuel Néron & 1 &3 &\href{}{NeronABCDD06}~\cite{NeronABCDD06}\\
\rowlabel{auth:a559}A. O. Amusat & 1 &0 &\href{works/abs-1902-01193.pdf}{abs-1902-01193}~\cite{abs-1902-01193}\\
\rowlabel{auth:a353}Ceyda Oguz & 1 &5 &\href{works/EdisO11.pdf}{EdisO11}~\cite{EdisO11}\\
\rowlabel{auth:a874}Olga Ohrimenko & 1 &127 &\href{works/OhrimenkoSC09.pdf}{OhrimenkoSC09}~\cite{OhrimenkoSC09}\\
\rowlabel{auth:a405}Bilal Omar Akram & 1 &0 &\href{works/AkramNHRSA23.pdf}{AkramNHRSA23}~\cite{AkramNHRSA23}\\
\rowlabel{auth:a619}Mirza Omer Beg & 1 &1 &\href{works/BegB13.pdf}{BegB13}~\cite{BegB13}\\
\rowlabel{auth:a724}Anne{-}C{\'{e}}cile Orgerie & 1 &1 &\href{works/Madi-WambaLOBM17.pdf}{Madi-WambaLOBM17}~\cite{Madi-WambaLOBM17}\\
\rowlabel{auth:a865}Gregor Ottosson & 1 &317 &\href{works/HookerO03.pdf}{HookerO03}~\cite{HookerO03}\\
\rowlabel{auth:a950}Greger Ottosson & 1 &14 &\href{}{MilanoORT02}~\cite{MilanoORT02}\\
\rowlabel{auth:a262}Mohand Ou Idir Khemmoudj & 1 &8 &\href{works/KhemmoudjPB06.pdf}{KhemmoudjPB06}~\cite{KhemmoudjPB06}\\
\rowlabel{auth:a241}Pierre Ouellet & 1 &12 &\href{works/OuelletQ13.pdf}{OuelletQ13}~\cite{OuelletQ13}\\
\rowlabel{auth:a460}Soukaina Oujana & 1 &1 &\href{works/OujanaAYB22.pdf}{OujanaAYB22}~\cite{OujanaAYB22}\\
\rowlabel{auth:a756}Asma Ouled Bedhief & 1 &0 &\href{works/Bedhief21.pdf}{Bedhief21}~\cite{Bedhief21}\\
\rowlabel{auth:a514}D{\'{e}}bora P. Ronconi & 1 &30 &\href{works/LunardiBLRV20.pdf}{LunardiBLRV20}~\cite{LunardiBLRV20}\\
\rowlabel{auth:a675}Edward P. K. Tsang & 1 &1 &\href{works/Tsang03.pdf}{Tsang03}~\cite{Tsang03}\\
\rowlabel{auth:a786}W. P. M. Nuijten & 1 &0 &\href{works/NuijtenA94.pdf}{NuijtenA94}~\cite{NuijtenA94}\\
\rowlabel{auth:a812}Meghana Padmanabhan & 1 &8 &\href{works/TranPZLDB18.pdf}{TranPZLDB18}~\cite{TranPZLDB18}\\
\rowlabel{auth:a236}Miquel Palah{\'{\i}} & 1 &3 &\href{works/BofillEGPSV14.pdf}{BofillEGPSV14}~\cite{BofillEGPSV14}\\
\rowlabel{auth:a699}Catuscia Palamidessi & 1 &10 &\href{works/FalaschiGMP97.pdf}{FalaschiGMP97}~\cite{FalaschiGMP97}\\
\rowlabel{auth:a535}Pere Pal{\`{a}}{-}Sch{\"{o}}nw{\"{a}}lder & 1 &17 &\href{works/EscobetPQPRA19.pdf}{EscobetPQPRA19}~\cite{EscobetPQPRA19}\\
\rowlabel{auth:a498}Vaibhav Pandey & 1 &3 &\href{works/PandeyS21a.pdf}{PandeyS21a}~\cite{PandeyS21a}\\
\rowlabel{auth:a554}Hoonseok Park & 1 &1 &\href{works/ParkUJR19.pdf}{ParkUJR19}~\cite{ParkUJR19}\\
\rowlabel{auth:a761}Myoung-Ju Park & 1 &0 &\href{works/HamPK21.pdf}{HamPK21}~\cite{HamPK21}\\
\rowlabel{auth:a739}Erica Pastore & 1 &0 &\href{works/AlfieriGPS23.pdf}{AlfieriGPS23}~\cite{AlfieriGPS23}\\
\rowlabel{auth:a73}Theo Pedersen & 1 &1 &\href{works/HanenKP21.pdf}{HanenKP21}~\cite{HanenKP21}\\
\rowlabel{auth:a781}Bart Peintner & 1 &0 &\href{works/MoffittPP05.pdf}{MoffittPP05}~\cite{MoffittPP05}\\
\rowlabel{auth:a934}Yunfang Peng & 1 &2 &\href{works/ChenGPSH10.pdf}{ChenGPSH10}~\cite{ChenGPSH10}\\
\rowlabel{auth:a1019}Louise Penz & 1 &0 &\href{}{PenzDN23}~\cite{PenzDN23}\\
\rowlabel{auth:a1027}Bernard Penz & 1 &0 &\href{works/CatusseCBL16.pdf}{CatusseCBL16}~\cite{CatusseCBL16}\\
\rowlabel{auth:a753}Jordi Pereira & 1 &6 &\href{works/YuraszeckMPV22.pdf}{YuraszeckMPV22}~\cite{YuraszeckMPV22}\\
\rowlabel{auth:a291}Laurent Perron & 1 &21 &\href{works/DannaP03.pdf}{DannaP03}~\cite{DannaP03}\\
\rowlabel{auth:a983}To\"{a}n Phan Huy & 1 &0 &\href{}{DomdorfPH03}~\cite{DomdorfPH03}\\
\rowlabel{auth:a419}Mehmet Pinarbasi & 1 &1 &\href{works/GurPAE23.pdf}{GurPAE23}~\cite{GurPAE23}\\
\rowlabel{auth:a687}Arthur Pinkney & 1 &11 &\href{works/MartinPY01.pdf}{MartinPY01}~\cite{MartinPY01}\\
\rowlabel{auth:a859}Eric Pinson & 1 &3 &\href{}{CarlierSJP21}~\cite{CarlierSJP21}\\
\rowlabel{auth:a1002}Éric Pinson & 1 &32 &\href{works/GuyonLPR12.pdf}{GuyonLPR12}~\cite{GuyonLPR12}\\
\rowlabel{auth:a527}David Pisinger & 1 &2 &\href{works/SacramentoSP20.pdf}{SacramentoSP20}~\cite{SacramentoSP20}\\
\rowlabel{auth:a446}Maximilian Pohl & 1 &4 &\href{works/PohlAK22.pdf}{PohlAK22}~\cite{PohlAK22}\\
\rowlabel{auth:a524}Oliver Polo{-}Mej{\'{\i}}a & 1 &8 &\href{works/Polo-MejiaALB20.pdf}{Polo-MejiaALB20}~\cite{Polo-MejiaALB20}\\
\rowlabel{auth:a530}Paul Pop & 1 &0 &\href{works/BarzegaranZP20.pdf}{BarzegaranZP20}~\cite{BarzegaranZP20}\\
\rowlabel{auth:a38}Louis Popovic & 1 &0 &\href{works/PopovicCGNC22.pdf}{PopovicCGNC22}~\cite{PopovicCGNC22}\\
\rowlabel{auth:a263}Marc Porcheron & 1 &8 &\href{works/KhemmoudjPB06.pdf}{KhemmoudjPB06}~\cite{KhemmoudjPB06}\\
\rowlabel{auth:a109}Marc Pouly & 1 &2 &\href{works/KoehlerBFFHPSSS21.pdf}{KoehlerBFFHPSSS21}~\cite{KoehlerBFFHPSSS21}\\
\rowlabel{auth:a4}Guillaume Pov{\'{e}}da & 1 &0 &\href{works/PovedaAA23.pdf}{PovedaAA23}~\cite{PovedaAA23}\\
\rowlabel{auth:a345}Matthias Prandtstetter & 1 &14 &\href{works/RendlPHPR12.pdf}{RendlPHPR12}~\cite{RendlPHPR12}\\
\rowlabel{auth:a839}Patrick Prosser & 1 &0 &\href{works/BeckPS03.pdf}{BeckPS03}~\cite{BeckPS03}\\
\rowlabel{auth:a347}Jakob Puchinger & 1 &14 &\href{works/RendlPHPR12.pdf}{RendlPHPR12}~\cite{RendlPHPR12}\\
\rowlabel{auth:a308}Jean{-}Francois Puget & 1 &6 &\href{works/Puget95.pdf}{Puget95}~\cite{Puget95}\\
\rowlabel{auth:a533}Vicen{\c{c}} Puig & 1 &17 &\href{works/EscobetPQPRA19.pdf}{EscobetPQPRA19}~\cite{EscobetPQPRA19}\\
\rowlabel{auth:a259}Kenneth Pulliam & 1 &2 &\href{works/KrogtLPHJ07.pdf}{KrogtLPHJ07}~\cite{KrogtLPHJ07}\\
\rowlabel{auth:a957}Karim Pérez Martínez & 1 &1 &\href{}{MartnezAJ22}~\cite{MartnezAJ22}\\
\rowlabel{auth:a684}Kenny Qili Zhu & 1 &0 &\href{works/ZhuS02.pdf}{ZhuS02}~\cite{ZhuS02}\\
\rowlabel{auth:a493}Ming Qin & 1 &12 &\href{works/QinWSLS21.pdf}{QinWSLS21}~\cite{QinWSLS21}\\
\rowlabel{auth:a516}Tianbao Qin & 1 &27 &\href{works/QinDCS20.pdf}{QinDCS20}~\cite{QinDCS20}\\
\rowlabel{auth:a661}Yang Qu & 1 &2 &\href{works/QuSN06.pdf}{QuSN06}~\cite{QuSN06}\\
\rowlabel{auth:a456}Yuchen Quan & 1 &2 &\href{}{ShiYXQ22}~\cite{ShiYXQ22}\\
\rowlabel{auth:a534}Joseba Quevedo & 1 &17 &\href{works/EscobetPQPRA19.pdf}{EscobetPQPRA19}~\cite{EscobetPQPRA19}\\
\rowlabel{auth:a801}Alain Quilliot & 1 &0 &\href{}{ArtiguesHQT21}~\cite{ArtiguesHQT21}\\
\rowlabel{auth:a123}Claude-Guy Quimper & 1 &0 &\href{}{FahimiQ23}~\cite{FahimiQ23}\\
\rowlabel{auth:a68}Dominik R. Bleidorn & 1 &3 &\href{works/KlankeBYE21.pdf}{KlankeBYE21}~\cite{KlankeBYE21}\\
\rowlabel{auth:a323}Aliza R. Heching & 1 &10 &\href{works/HechingH16.pdf}{HechingH16}~\cite{HechingH16}\\
\rowlabel{auth:a800}Gregg R. Rabideau & 1 &0 &\href{works/HebrardALLCMR22.pdf}{HebrardALLCMR22}~\cite{HebrardALLCMR22}\\
\rowlabel{auth:a985}Camino R. Vela & 1 &0 &\href{}{AfsarVPG23}~\cite{AfsarVPG23}\\
\rowlabel{auth:a253}Chandra Reddy & 1 &1 &\href{works/DavenportKRSH07.pdf}{DavenportKRSH07}~\cite{DavenportKRSH07}\\
\rowlabel{auth:a988}Francisco Regis Abreu Gomes & 1 &1 &\href{works/GomesM17.pdf}{GomesM17}~\cite{GomesM17}\\
\rowlabel{auth:a509}Yaping Ren & 1 &100 &\href{works/MengZRZL20.pdf}{MengZRZL20}~\cite{MengZRZL20}\\
\rowlabel{auth:a344}Andrea Rendl & 1 &14 &\href{works/RendlPHPR12.pdf}{RendlPHPR12}~\cite{RendlPHPR12}\\
\rowlabel{auth:a76}Hamid Reza Feyzmahdavian & 1 &2 &\href{works/Astrand0F21.pdf}{Astrand0F21}~\cite{Astrand0F21}\\
\rowlabel{auth:a394}Vahid Riahi & 1 &0 &\href{works/RiahiNS018.pdf}{RiahiNS018}~\cite{RiahiNS018}\\
\rowlabel{auth:a656}Diane Riopel & 1 &84 &\href{works/KhayatLR06.pdf}{KhayatLR06}~\cite{KhayatLR06}\\
\rowlabel{auth:a331}Gregory Rix & 1 &1 &\href{works/PesantRR15.pdf}{PesantRR15}~\cite{PesantRR15}\\
\rowlabel{auth:a989}Geraldo Robson Mateus & 1 &1 &\href{works/GomesM17.pdf}{GomesM17}~\cite{GomesM17}\\
\rowlabel{auth:a300}Robert Rodosek & 1 &19 &\href{works/RodosekW98.pdf}{RodosekW98}~\cite{RodosekW98}\\
\rowlabel{auth:a283}Brian Rodrigues & 1 &5 &\href{works/LimRX04.pdf}{LimRX04}~\cite{LimRX04}\\
\rowlabel{auth:a791}Joaquín Rodriguez & 1 &117 &\href{works/Rodriguez07.pdf}{Rodriguez07}~\cite{Rodriguez07}\\
\rowlabel{auth:a792}Joaquin Rodriguez & 1 &0 &\href{works/RodriguezDG02.pdf}{RodriguezDG02}~\cite{RodriguezDG02}\\
\rowlabel{auth:a119}Jerome Rogerie & 1 &148 &\href{works/LaborieRSV18.pdf}{LaborieRSV18}~\cite{LaborieRSV18}\\
\rowlabel{auth:a437}Mohammad Rohaninejad & 1 &0 &\href{works/Mehdizadeh-Somarin23.pdf}{Mehdizadeh-Somarin23}~\cite{Mehdizadeh-Somarin23}\\
\rowlabel{auth:a415}Maximiliano Rojel & 1 &0 &\href{works/YuraszeckMCCR23.pdf}{YuraszeckMCCR23}~\cite{YuraszeckMCCR23}\\
\rowlabel{auth:a536}Juli Romera & 1 &17 &\href{works/EscobetPQPRA19.pdf}{EscobetPQPRA19}~\cite{EscobetPQPRA19}\\
\rowlabel{auth:a375}Roberto Rossi & 1 &6 &\href{works/RossiTHP07.pdf}{RossiTHP07}~\cite{RossiTHP07}\\
\rowlabel{auth:a721}Fran{\c{c}}ois Roubellat & 1 &84 &\href{works/ArtiguesR00.pdf}{ArtiguesR00}~\cite{ArtiguesR00}\\
\rowlabel{auth:a22}St{\'{e}}phanie Roussel & 1 &0 &\href{works/SquillaciPR23.pdf}{SquillaciPR23}~\cite{SquillaciPR23}\\
\rowlabel{auth:a709}Didier Rozzonelli & 1 &0 &\href{}{JourdanFRD94}~\cite{JourdanFRD94}\\
\rowlabel{auth:a1029}Pascal Rubini & 1 &0 &\href{works/CatusseCBL16.pdf}{CatusseCBL16}~\cite{CatusseCBL16}\\
\rowlabel{auth:a101}Hana Rudov{\'{a}} & 1 &2 &\href{works/MurinR19.pdf}{MurinR19}~\cite{MurinR19}\\
\rowlabel{auth:a736}Rub\'{e}n Ruiz & 1 &2 &\href{works/NaderiRR23.pdf}{NaderiRR23}~\cite{NaderiRR23}\\
\rowlabel{auth:a557}Martin Ruskowski & 1 &1 &\href{works/ParkUJR19.pdf}{ParkUJR19}~\cite{ParkUJR19}\\
\rowlabel{auth:a615}Anna Ryabokon & 1 &3 &\href{}{FriedrichFMRSST14}~\cite{FriedrichFMRSST14}\\
\rowlabel{auth:a272}William S. Havens & 1 &2 &\href{works/DilkinaDH05.pdf}{DilkinaDH05}~\cite{DilkinaDH05}\\
\rowlabel{auth:a481}Mohamed S. Gheith & 1 &1 &\href{works/AbohashimaEG21.pdf}{AbohashimaEG21}~\cite{AbohashimaEG21}\\
\rowlabel{auth:a851}Gregory S. Zaric & 1 &3 &\href{}{NaderiBZ22a}~\cite{NaderiBZ22a}\\
\rowlabel{auth:a526}David Sacramento & 1 &2 &\href{works/SacramentoSP20.pdf}{SacramentoSP20}~\cite{SacramentoSP20}\\
\rowlabel{auth:a521}Shahram Saeidi & 1 &1 &\href{}{AlizdehS20}~\cite{AlizdehS20}\\
\rowlabel{auth:a948}Abderrahim Sahli & 1 &3 &\href{}{CarlierSJP21}~\cite{CarlierSJP21}\\
\rowlabel{auth:a499}Poonam Saini & 1 &3 &\href{works/PandeyS21a.pdf}{PandeyS21a}~\cite{PandeyS21a}\\
\rowlabel{auth:a740}Fabio Salassa & 1 &0 &\href{works/AlfieriGPS23.pdf}{AlfieriGPS23}~\cite{AlfieriGPS23}\\
\rowlabel{auth:a921}Amir Salehipour & 1 &0 &\href{works/EtminaniesfahaniGNMS22.pdf}{EtminaniesfahaniGNMS22}~\cite{EtminaniesfahaniGNMS22}\\
\rowlabel{auth:a110}Sophia Saller & 1 &2 &\href{works/KoehlerBFFHPSSS21.pdf}{KoehlerBFFHPSSS21}~\cite{KoehlerBFFHPSSS21}\\
\rowlabel{auth:a111}Anastasia Salyaeva & 1 &2 &\href{works/KoehlerBFFHPSSS21.pdf}{KoehlerBFFHPSSS21}~\cite{KoehlerBFFHPSSS21}\\
\rowlabel{auth:a961}Guido Sand & 1 &381 &\href{}{HarjunkoskiMBC14}~\cite{HarjunkoskiMBC14}\\
\rowlabel{auth:a616}Maria Sander & 1 &3 &\href{}{FriedrichFMRSST14}~\cite{FriedrichFMRSST14}\\
\rowlabel{auth:a722}Eric Sanlaville & 1 &7 &\href{works/PoderBS04.pdf}{PoderBS04}~\cite{PoderBS04}\\
\rowlabel{auth:a650}{\'{O}}scar Sapena & 1 &22 &\href{works/GarridoOS08.pdf}{GarridoOS08}~\cite{GarridoOS08}\\
\rowlabel{auth:a428}{\"{O}}zge Satir Akpunar & 1 &0 &\href{works/IsikYA23.pdf}{IsikYA23}~\cite{IsikYA23}\\
\rowlabel{auth:a397}Abdul Sattar & 1 &0 &\href{works/RiahiNS018.pdf}{RiahiNS018}~\cite{RiahiNS018}\\
\rowlabel{auth:a112}Peter Scheiblechner & 1 &2 &\href{works/KoehlerBFFHPSSS21.pdf}{KoehlerBFFHPSSS21}~\cite{KoehlerBFFHPSSS21}\\
\rowlabel{auth:a166}Klaus Schild & 1 &23 &\href{works/SchildW00.pdf}{SchildW00}~\cite{SchildW00}\\
\rowlabel{auth:a140}Thomas Schlechte & 1 &10 &\href{works/HeinzSSW12.pdf}{HeinzSSW12}~\cite{HeinzSSW12}\\
\rowlabel{auth:a502}Thorsten Schmidt & 1 &1 &\href{works/BenderWS21.pdf}{BenderWS21}~\cite{BenderWS21}\\
\rowlabel{auth:a777}Günter Schmidt & 1 &38 &\href{}{BlazewiczEP19}~\cite{BlazewiczEP19}\\
\rowlabel{auth:a973}Alexander Schnell & 1 &24 &\href{works/SchnellH15.pdf}{SchnellH15}~\cite{SchnellH15}\\
\rowlabel{auth:a60}Philipp Schrott{-}Kostwein & 1 &0 &\href{works/KovacsTKSG21.pdf}{KovacsTKSG21}~\cite{KovacsTKSG21}\\
\rowlabel{auth:a146}Uwe Schwiegelshohn & 1 &4 &\href{works/LimtanyakulS12.pdf}{LimtanyakulS12}~\cite{LimtanyakulS12}\\
\rowlabel{auth:a576}Lena Secher Ejlertsen & 1 &41 &\href{works/PourDERB18.pdf}{PourDERB18}~\cite{PourDERB18}\\
\rowlabel{auth:a840}Evgeny Selensky & 1 &0 &\href{works/BeckPS03.pdf}{BeckPS03}~\cite{BeckPS03}\\
\rowlabel{auth:a242}Thiago Serra & 1 &0 &\href{works/SerraNM12.pdf}{SerraNM12}~\cite{SerraNM12}\\
\rowlabel{auth:a519}Mei Sha & 1 &27 &\href{works/QinDCS20.pdf}{QinDCS20}~\cite{QinDCS20}\\
\rowlabel{auth:a605}Yufen Shao & 1 &48 &\href{works/GoelSHFS15.pdf}{GoelSHFS15}~\cite{GoelSHFS15}\\
\rowlabel{auth:a935}Xinyu Shao & 1 &2 &\href{works/ChenGPSH10.pdf}{ChenGPSH10}~\cite{ChenGPSH10}\\
\rowlabel{auth:a453}Ganquan Shi & 1 &2 &\href{}{ShiYXQ22}~\cite{ShiYXQ22}\\
\rowlabel{auth:a495}Zhongshun Shi & 1 &12 &\href{works/QinWSLS21.pdf}{QinWSLS21}~\cite{QinWSLS21}\\
\rowlabel{auth:a497}Leyuan Shi & 1 &12 &\href{works/QinWSLS21.pdf}{QinWSLS21}~\cite{QinWSLS21}\\
\rowlabel{auth:a254}Stuart Siegel & 1 &1 &\href{works/DavenportKRSH07.pdf}{DavenportKRSH07}~\cite{DavenportKRSH07}\\
\rowlabel{auth:a318}Maria Silvia Pini & 1 &1 &\href{works/GelainPRVW17.pdf}{GelainPRVW17}~\cite{GelainPRVW17}\\
\rowlabel{auth:a35}Vanessa Simard & 1 &0 &\href{works/BoudreaultSLQ22.pdf}{BoudreaultSLQ22}~\cite{BoudreaultSLQ22}\\
\rowlabel{auth:a543}Pawel Sitek & 1 &0 &\href{works/WikarekS19.pdf}{WikarekS19}~\cite{WikarekS19}\\
\rowlabel{auth:a603}M. Slusky & 1 &48 &\href{works/GoelSHFS15.pdf}{GoelSHFS15}~\cite{GoelSHFS15}\\
\rowlabel{auth:a911}Kate Smith-Miles & 1 &3 &\href{}{EdwardsBSE19}~\cite{EdwardsBSE19}\\
\rowlabel{auth:a662}Juha{-}Pekka Soininen & 1 &2 &\href{works/QuSN06.pdf}{QuSN06}~\cite{QuSN06}\\
\rowlabel{auth:a1016}Junbo Son & 1 &1 &\href{}{ZhuSZW23}~\cite{ZhuSZW23}\\
\rowlabel{auth:a623}Xiaoqing Song & 1 &1 &\href{works/ZhangLS12.pdf}{ZhangLS12}~\cite{ZhangLS12}\\
\rowlabel{auth:a843}Shahabeddin Sotudian & 1 &55 &\href{works/ZarandiASC20.pdf}{ZarandiASC20}~\cite{ZarandiASC20}\\
\rowlabel{auth:a784}Francis Sourd & 1 &7 &\href{works/SourdN00.pdf}{SourdN00}~\cite{SourdN00}\\
\rowlabel{auth:a202}Helge Spieker & 1 &6 &\href{works/MossigeGSMC17.pdf}{MossigeGSMC17}~\cite{MossigeGSMC17}\\
\rowlabel{auth:a20}Samuel Squillaci & 1 &0 &\href{works/SquillaciPR23.pdf}{SquillaciPR23}~\cite{SquillaciPR23}\\
\rowlabel{auth:a617}Andreas Starzacher & 1 &3 &\href{}{FriedrichFMRSST14}~\cite{FriedrichFMRSST14}\\
\rowlabel{auth:a49}Wolfgang Steigerwald & 1 &0 &\href{works/GeitzGSSW22.pdf}{GeitzGSSW22}~\cite{GeitzGSSW22}\\
\rowlabel{auth:a141}R{\"{u}}diger Stephan & 1 &10 &\href{works/HeinzSSW12.pdf}{HeinzSSW12}~\cite{HeinzSSW12}\\
\rowlabel{auth:a778}Malgorzata Sterna & 1 &38 &\href{}{BlazewiczEP19}~\cite{BlazewiczEP19}\\
\rowlabel{auth:a50}Robin St{\"{o}}hr & 1 &0 &\href{works/GeitzGSSW22.pdf}{GeitzGSSW22}~\cite{GeitzGSSW22}\\
\rowlabel{auth:a491}Christian St{\"{u}}rck & 1 &0 &\href{works/HubnerGSV21.pdf}{HubnerGSV21}~\cite{HubnerGSV21}\\
\rowlabel{auth:a396}Kaile Su & 1 &0 &\href{works/RiahiNS018.pdf}{RiahiNS018}~\cite{RiahiNS018}\\
\rowlabel{auth:a639}Wei Su & 1 &1 &\href{works/MakMS10.pdf}{MakMS10}~\cite{MakMS10}\\
\rowlabel{auth:a458}Kemal Subulan & 1 &5 &\href{works/SubulanC22.pdf}{SubulanC22}~\cite{SubulanC22}\\
\rowlabel{auth:a313}Premysl Sucha & 1 &2 &\href{works/BenediktSMVH18.pdf}{BenediktSMVH18}~\cite{BenediktSMVH18}\\
\rowlabel{auth:a568}Quanxin Sun & 1 &24 &\href{works/TangLWSK18.pdf}{TangLWSK18}~\cite{TangLWSK18}\\
\rowlabel{auth:a633}Zheng Sun & 1 &4 &\href{works/SunLYL10.pdf}{SunLYL10}~\cite{SunLYL10}\\
\rowlabel{auth:a657}Suresh Sundaram & 1 &12 &\href{works/SureshMOK06.pdf}{SureshMOK06}~\cite{SureshMOK06}\\
\rowlabel{auth:a790}Pavel Surynek & 1 &2 &\href{works/BartakCS10.pdf}{BartakCS10}~\cite{BartakCS10}\\
\rowlabel{auth:a788}Jir{\'{\i}} Svancara & 1 &0 &\href{works/SvancaraB22.pdf}{SvancaraB22}~\cite{SvancaraB22}\\
\rowlabel{auth:a206}Ria Szeredi & 1 &9 &\href{works/SzerediS16.pdf}{SzerediS16}~\cite{SzerediS16}\\
\rowlabel{auth:a98}Alina S{\^{\i}}rbu & 1 &1 &\href{works/GalleguillosKSB19.pdf}{GalleguillosKSB19}~\cite{GalleguillosKSB19}\\
\rowlabel{auth:a512}Willian T. Lunardi & 1 &30 &\href{works/LunardiBLRV20.pdf}{LunardiBLRV20}~\cite{LunardiBLRV20}\\
\rowlabel{auth:a1014}T. Taimre & 1 &0 &\href{works/ForbesHJST24.pdf}{ForbesHJST24}~\cite{ForbesHJST24}\\
\rowlabel{auth:a928}Yingcong Tan & 1 &1 &\href{works/TanT18.pdf}{TanT18}~\cite{TanT18}\\
\rowlabel{auth:a482}Siyu Tang & 1 &7 &\href{works/VlkHT21.pdf}{VlkHT21}~\cite{VlkHT21}\\
\rowlabel{auth:a565}Yuanjie Tang & 1 &24 &\href{works/TangLWSK18.pdf}{TangLWSK18}~\cite{TangLWSK18}\\
\rowlabel{auth:a29}Fabio Tardivo & 1 &0 &\href{works/TardivoDFMP23.pdf}{TardivoDFMP23}~\cite{TardivoDFMP23}\\
\rowlabel{auth:a376}Armagan Tarim & 1 &6 &\href{works/RossiTHP07.pdf}{RossiTHP07}~\cite{RossiTHP07}\\
\rowlabel{auth:a771}Ehsan Tarkesh Esfahani & 1 &0 &\href{works/YounespourAKE19.pdf}{YounespourAKE19}~\cite{YounespourAKE19}\\
\rowlabel{auth:a452}Nikolay Tchernev & 1 &4 &\href{works/BourreauGGLT22.pdf}{BourreauGGLT22}~\cite{BourreauGGLT22}\\
\rowlabel{auth:a733}Paolo Terenziani & 1 &1 &\href{works/BrusoniCLMMT96.pdf}{BrusoniCLMMT96}~\cite{BrusoniCLMMT96}\\
\rowlabel{auth:a503}Willian Tessaro Lunardi & 1 &0 &\href{works/Lunardi20.pdf}{Lunardi20}~\cite{Lunardi20}\\
\rowlabel{auth:a545}Stephan Teuschl & 1 &0 &\href{works/FrohnerTR19.pdf}{FrohnerTR19}~\cite{FrohnerTR19}\\
\rowlabel{auth:a65}Jordan Ticktin & 1 &0 &\href{works/HillTV21.pdf}{HillTV21}~\cite{HillTV21}\\
\rowlabel{auth:a339}Kevin Tierney & 1 &16 &\href{works/KelarevaTK13.pdf}{KelarevaTK13}~\cite{KelarevaTK13}\\
\rowlabel{auth:a683}Christian Timpe & 1 &42 &\href{works/Timpe02.pdf}{Timpe02}~\cite{Timpe02}\\
\rowlabel{auth:a546}Mary Tom & 1 &0 &\href{works/Tom19.pdf}{Tom19}~\cite{Tom19}\\
\rowlabel{auth:a627}Seyda Topaloglu & 1 &46 &\href{works/TopalogluO11.pdf}{TopalogluO11}~\cite{TopalogluO11}\\
\rowlabel{auth:a679}Miguel Toro & 1 &7 &\href{works/ValleMGT03.pdf}{ValleMGT03}~\cite{ValleMGT03}\\
\rowlabel{auth:a886}Philippe Torres & 1 &26 &\href{works/TorresL00.pdf}{TorresL00}~\cite{TorresL00}\\
\rowlabel{auth:a464}Meriem Touat & 1 &0 &\href{works/TouatBT22.pdf}{TouatBT22}~\cite{TouatBT22}\\
\rowlabel{auth:a309}Toura{\"{\i}}vane & 1 &2 &\href{works/Touraivane95.pdf}{Touraivane95}~\cite{Touraivane95}\\
\rowlabel{auth:a802}H{\'{e}}l{\`{e}}ne Toussaint & 1 &0 &\href{}{ArtiguesHQT21}~\cite{ArtiguesHQT21}\\
\rowlabel{auth:a715}Mariem Trojet & 1 &11 &\href{works/TrojetHL11.pdf}{TrojetHL11}~\cite{TrojetHL11}\\
\rowlabel{auth:a137}Semra Tunali & 1 &31 &\href{works/OzturkTHO13.pdf}{OzturkTHO13}~\cite{OzturkTHO13}\\
\rowlabel{auth:a278}Paul Tyler & 1 &0 &\href{works/HebrardTW05.pdf}{HebrardTW05}~\cite{HebrardTW05}\\
\rowlabel{auth:a555}Jumyung Um & 1 &1 &\href{works/ParkUJR19.pdf}{ParkUJR19}~\cite{ParkUJR19}\\
\rowlabel{auth:a947}David Urbach & 1 &61 &\href{}{RoshanaeiLAU17}~\cite{RoshanaeiLAU17}\\
\rowlabel{auth:a759}J. V. Moccellin & 1 &0 &\href{works/AbreuAPNM21.pdf}{AbreuAPNM21}~\cite{AbreuAPNM21}\\
\rowlabel{auth:a848}Sasha Van Cauwelaert & 1 &2 &\href{works/CauwelaertDS20.pdf}{CauwelaertDS20}~\cite{CauwelaertDS20}\\
\rowlabel{auth:a926}Alkis Vazacopoulos & 1 &0 &\href{}{AggounMV08}~\cite{AggounMV08}\\
\rowlabel{auth:a837}Thierry Vidal & 1 &58 &\href{works/BidotVLB09.pdf}{BidotVLB09}~\cite{BidotVLB09}\\
\rowlabel{auth:a668}Karen Villaverde & 1 &0 &\href{}{VillaverdeP04}~\cite{VillaverdeP04}\\
\rowlabel{auth:a754}Mariona Vilà & 1 &6 &\href{works/YuraszeckMPV22.pdf}{YuraszeckMPV22}~\cite{YuraszeckMPV22}\\
\rowlabel{auth:a492}Rebekka Volk & 1 &0 &\href{works/HubnerGSV21.pdf}{HubnerGSV21}~\cite{HubnerGSV21}\\
\rowlabel{auth:a515}Holger Voos & 1 &30 &\href{works/LunardiBLRV20.pdf}{LunardiBLRV20}~\cite{LunardiBLRV20}\\
\rowlabel{auth:a66}Thomas W. M. Vossen & 1 &0 &\href{works/HillTV21.pdf}{HillTV21}~\cite{HillTV21}\\
\rowlabel{auth:a113}Kai Waelti & 1 &2 &\href{works/KoehlerBFFHPSSS21.pdf}{KoehlerBFFHPSSS21}~\cite{KoehlerBFFHPSSS21}\\
\rowlabel{auth:a494}Runsen Wang & 1 &12 &\href{works/QinWSLS21.pdf}{QinWSLS21}~\cite{QinWSLS21}\\
\rowlabel{auth:a567}Futian Wang & 1 &24 &\href{works/TangLWSK18.pdf}{TangLWSK18}~\cite{TangLWSK18}\\
\rowlabel{auth:a582}Shouyang Wang & 1 &49 &\href{works/ZhangW18.pdf}{ZhangW18}~\cite{ZhangW18}\\
\rowlabel{auth:a606}Tao Wang & 1 &36 &\href{works/WangMD15.pdf}{WangMD15}~\cite{WangMD15}\\
\rowlabel{auth:a956}Yi Wang & 1 &8 &\href{}{GuoHLW20}~\cite{GuoHLW20}\\
\rowlabel{auth:a852}Ezra Wari & 1 &11 &\href{}{WariZ19}~\cite{WariZ19}\\
\rowlabel{auth:a962}John Wassick & 1 &381 &\href{}{HarjunkoskiMBC14}~\cite{HarjunkoskiMBC14}\\
\rowlabel{auth:a779}Jan Weglarz & 1 &38 &\href{}{BlazewiczEP19}~\cite{BlazewiczEP19}\\
\rowlabel{auth:a371}Kong Wei Lye & 1 &0 &\href{works/LauLN08.pdf}{LauLN08}~\cite{LauLN08}\\
\rowlabel{auth:a90}Johan Wess{\'{e}}n & 1 &2 &\href{works/WessenCS20.pdf}{WessenCS20}~\cite{WessenCS20}\\
\rowlabel{auth:a742}Radosław Wichniarek & 1 &0 &\href{works/CzerniachowskaWZ23.pdf}{CzerniachowskaWZ23}~\cite{CzerniachowskaWZ23}\\
\rowlabel{auth:a542}Jaroslaw Wikarek & 1 &0 &\href{works/WikarekS19.pdf}{WikarekS19}~\cite{WikarekS19}\\
\rowlabel{auth:a189}Campbell Wilson & 1 &6 &\href{works/He0GLW18.pdf}{He0GLW18}~\cite{He0GLW18}\\
\rowlabel{auth:a142}Michael Winkler & 1 &10 &\href{works/HeinzSSW12.pdf}{HeinzSSW12}~\cite{HeinzSSW12}\\
\rowlabel{auth:a501}David Wittwer & 1 &1 &\href{works/BenderWS21.pdf}{BenderWS21}~\cite{BenderWS21}\\
\rowlabel{auth:a976}Keith Womer & 1 &113 &\href{works/LiW08.pdf}{LiW08}~\cite{LiW08}\\
\rowlabel{auth:a1018}Jianguo Wu & 1 &1 &\href{}{ZhuSZW23}~\cite{ZhuSZW23}\\
\rowlabel{auth:a1022}Cheng{-}Hung Wu & 1 &14 &\href{}{NattafDYW19}~\cite{NattafDYW19}\\
\rowlabel{auth:a167}J{\"{o}}rg W{\"{u}}rtz & 1 &23 &\href{works/SchildW00.pdf}{SchildW00}~\cite{SchildW00}\\
\rowlabel{auth:a384}Quanshi Xia & 1 &13 &\href{works/ChuX05.pdf}{ChuX05}~\cite{ChuX05}\\
\rowlabel{auth:a484}Hegen Xiong & 1 &18 &\href{works/FanXG21.pdf}{FanXG21}~\cite{FanXG21}\\
\rowlabel{auth:a284}Zhou Xu & 1 &5 &\href{works/LimRX04.pdf}{LimRX04}~\cite{LimRX04}\\
\rowlabel{auth:a455}Yang Xu & 1 &2 &\href{}{ShiYXQ22}~\cite{ShiYXQ22}\\
\rowlabel{auth:a88}Tanya Y. Tang & 1 &6 &\href{works/TangB20.pdf}{TangB20}~\cite{TangB20}\\
\rowlabel{auth:a764}El Yaakoubi Anass & 1 &0 &\href{works/FallahiAC20.pdf}{FallahiAC20}~\cite{FallahiAC20}\\
\rowlabel{auth:a294}Hong Yan & 1 &8 &\href{works/HookerY02.pdf}{HookerY02}~\cite{HookerY02}\\
\rowlabel{auth:a312}Moli Yang & 1 &1 &\href{works/YangSS19.pdf}{YangSS19}~\cite{YangSS19}\\
\rowlabel{auth:a454}Zhouwang Yang & 1 &2 &\href{}{ShiYXQ22}~\cite{ShiYXQ22}\\
\rowlabel{auth:a590}Jia{-}Sheng Yao & 1 &2 &\href{works/HoYCLLCLC18.pdf}{HoYCLLCLC18}~\cite{HoYCLLCLC18}\\
\rowlabel{auth:a635}Min Yao & 1 &4 &\href{works/SunLYL10.pdf}{SunLYL10}~\cite{SunLYL10}\\
\rowlabel{auth:a583}Seung Yeob Shin & 1 &9 &\href{works/ShinBBHO18.pdf}{ShinBBHO18}~\cite{ShinBBHO18}\\
\rowlabel{auth:a69}Vassilios Yfantis & 1 &3 &\href{works/KlankeBYE21.pdf}{KlankeBYE21}~\cite{KlankeBYE21}\\
\rowlabel{auth:a768}Maryam Younespour & 1 &0 &\href{works/YounespourAKE19.pdf}{YounespourAKE19}~\cite{YounespourAKE19}\\
\rowlabel{auth:a487}Chunxia Yu & 1 &6 &\href{works/ZhangYW21.pdf}{ZhangYW21}~\cite{ZhangYW21}\\
\rowlabel{auth:a688}Xinghuo Yu & 1 &11 &\href{works/MartinPY01.pdf}{MartinPY01}~\cite{MartinPY01}\\
\rowlabel{auth:a369}Oleg Yu. Gusikhin & 1 &1 &\href{works/BarlattCG08.pdf}{BarlattCG08}~\cite{BarlattCG08}\\
\rowlabel{auth:a1021}Claude Yugma & 1 &14 &\href{}{NattafDYW19}~\cite{NattafDYW19}\\
\rowlabel{auth:a813}Peter Yun Zhang & 1 &8 &\href{works/TranPZLDB18.pdf}{TranPZLDB18}~\cite{TranPZLDB18}\\
\rowlabel{auth:a457}Pinar Yunusoglu & 1 &20 &\href{works/YunusogluY22.pdf}{YunusogluY22}~\cite{YunusogluY22}\\
\rowlabel{auth:a182}Marco Zaffalon & 1 &28 &\href{works/Darby-DowmanLMZ97.pdf}{Darby-DowmanLMZ97}~\cite{Darby-DowmanLMZ97}\\
\rowlabel{auth:a905}Boukhalfa Zahout & 1 &0 &\href{works/Zahout21.pdf}{Zahout21}~\cite{Zahout21}\\
\rowlabel{auth:a228}St{\'{e}}phane Zampelli & 1 &3 &\href{works/DerrienPZ14.pdf}{DerrienPZ14}~\cite{DerrienPZ14}\\
\rowlabel{auth:a529}Bahram Zarrin & 1 &0 &\href{works/BarzegaranZP20.pdf}{BarzegaranZP20}~\cite{BarzegaranZP20}\\
\rowlabel{auth:a980}Shohre Zehtabian & 1 &0 &\href{works/EmdeZD22.pdf}{EmdeZD22}~\cite{EmdeZD22}\\
\rowlabel{auth:a404}Mengjie Zhang & 1 &0 &\href{works/abs-2402-00459.pdf}{abs-2402-00459}~\cite{abs-2402-00459}\\
\rowlabel{auth:a473}Haotian Zhang & 1 &0 &\href{works/ZhangJZL22.pdf}{ZhangJZL22}~\cite{ZhangJZL22}\\
\rowlabel{auth:a486}Luping Zhang & 1 &6 &\href{works/ZhangYW21.pdf}{ZhangYW21}~\cite{ZhangYW21}\\
\rowlabel{auth:a508}Chaoyong Zhang & 1 &100 &\href{works/MengZRZL20.pdf}{MengZRZL20}~\cite{MengZRZL20}\\
\rowlabel{auth:a510}Biao Zhang & 1 &100 &\href{works/MengZRZL20.pdf}{MengZRZL20}~\cite{MengZRZL20}\\
\rowlabel{auth:a581}Sicheng Zhang & 1 &49 &\href{works/ZhangW18.pdf}{ZhangW18}~\cite{ZhangW18}\\
\rowlabel{auth:a621}Xujun Zhang & 1 &1 &\href{works/ZhangLS12.pdf}{ZhangLS12}~\cite{ZhangLS12}\\
\rowlabel{auth:a767}Lihui Zhang & 1 &0 &\href{works/ZouZ20.pdf}{ZouZ20}~\cite{ZouZ20}\\
\rowlabel{auth:a809}Jiachen Zhang & 1 &0 &\href{works/ZhangBB22.pdf}{ZhangBB22}~\cite{ZhangBB22}\\
\rowlabel{auth:a850}Guoqing Zhang & 1 &0 &\href{works/NaderiBZ22.pdf}{NaderiBZ22}~\cite{NaderiBZ22}\\
\rowlabel{auth:a1017}Xi Zhang & 1 &1 &\href{}{ZhuSZW23}~\cite{ZhuSZW23}\\
\rowlabel{auth:a609}Jinlian Zhou & 1 &0 &\href{works/ZhouGL15.pdf}{ZhouGL15}~\cite{ZhouGL15}\\
\rowlabel{auth:a853}Weihang Zhu & 1 &11 &\href{}{WariZ19}~\cite{WariZ19}\\
\rowlabel{auth:a966}Jianjun Zhu & 1 &0 &\href{}{GuoZ23}~\cite{GuoZ23}\\
\rowlabel{auth:a1015}Xuedong Zhu & 1 &1 &\href{}{ZhuSZW23}~\cite{ZhuSZW23}\\
\rowlabel{auth:a267}Pawel Zielinski & 1 &13 &\href{works/FortinZDF05.pdf}{FortinZDF05}~\cite{FortinZDF05}\\
\rowlabel{auth:a804}J{\"{u}}rgen Zimmermann & 1 &25 &\href{works/KreterSSZ18.pdf}{KreterSSZ18}~\cite{KreterSSZ18}\\
\rowlabel{auth:a766}Xin Zou & 1 &0 &\href{works/ZouZ20.pdf}{ZouZ20}~\cite{ZouZ20}\\
\rowlabel{auth:a311}Mathijs de Weerdt & 1 &1 &\href{works/BogaerdtW19.pdf}{BogaerdtW19}~\cite{BogaerdtW19}\\
\rowlabel{auth:a758}Bruno de Athayde Prata & 1 &0 &\href{works/AbreuAPNM21.pdf}{AbreuAPNM21}~\cite{AbreuAPNM21}\\
\rowlabel{auth:a903}Alexis de Clercq & 1 &0 &\href{works/Clercq12.pdf}{Clercq12}~\cite{Clercq12}\\
\rowlabel{auth:a258}Roman van der Krogt & 1 &2 &\href{works/KrogtLPHJ07.pdf}{KrogtLPHJ07}~\cite{KrogtLPHJ07}\\
\rowlabel{auth:a310}Pim van den Bogaerdt & 1 &1 &\href{works/BogaerdtW19.pdf}{BogaerdtW19}~\cite{BogaerdtW19}\\
\rowlabel{auth:a845}Willem-Jan van Hoeve & 1 &12 &\href{works/HookerH17.pdf}{HookerH17}~\cite{HookerH17}\\
\rowlabel{auth:a1013}F.A. van der Schoot & 1 &0 &\href{works/ForbesHJST24.pdf}{ForbesHJST24}~\cite{ForbesHJST24}\\
\rowlabel{auth:a237}Stefano {Di Alesio} & 1 &3 &\href{works/AlesioNBG14.pdf}{AlesioNBG14}~\cite{AlesioNBG14}\\
\rowlabel{auth:a833}Ulas {\"{O}}zen & 1 &8 &\href{works/TerekhovDOB12.pdf}{TerekhovDOB12}~\cite{TerekhovDOB12}\\
\rowlabel{auth:a580}Selin {\"{O}}zpeynirci & 1 &31 &\href{works/GokgurHO18.pdf}{GokgurHO18}~\cite{GokgurHO18}\\
\rowlabel{auth:a136}Cemalettin {\"{O}}zt{\"{u}}rk & 1 &31 &\href{works/OzturkTHO13.pdf}{OzturkTHO13}~\cite{OzturkTHO13}\\
\rowlabel{auth:a5}Nahum {\'{A}}lvarez & 1 &0 &\href{works/PovedaAA23.pdf}{PovedaAA23}~\cite{PovedaAA23}\\
\rowlabel{auth:a221}Se{\'{a}}n {\'{O}}g Murphy & 1 &1 &\href{works/MurphyMB15.pdf}{MurphyMB15}~\cite{MurphyMB15}\\
\rowlabel{auth:a459}Gizem {\c{C}}akir & 1 &5 &\href{works/SubulanC22.pdf}{SubulanC22}~\cite{SubulanC22}\\
\rowlabel{auth:a743}Krzysztof Żywicki & 1 &0 &\href{works/CzerniachowskaWZ23.pdf}{CzerniachowskaWZ23}~\cite{CzerniachowskaWZ23}\\
\end{longtable}
}




\clearpage
\section{Problem Classification}


\begin{table}[htbp]
\caption{\label{tab:classification}Problem Classification Types}
\centering
{\scriptsize
\begin{tabular}{lp{8cm}}\toprule
Code & Name \\ \midrule
JSSP & Job-Shop Scheduling Problem \\
JSPT & Job-Shop Scheduling Problem with Transportation \\
PP-MS-MMRCPSP/max-cal & partially preemptive- multi-skill/mode resource-constrained project scheduling problem with generalized precedence relations and resource calendars\\
RCPSP & Resource Constrained Project Scheduling Problem \\
TMS & Transmission Network Maintenance Planning \\
PMSP & Parallel Machine Scheduling Problem\\
HFF & Hybrid Flexible Flow-shop \\
$HFFm|tt|C_{\max}$ & Hybrid Flexible Flowshop with Transportation Times\\
OSP & Oven Scheduling Problem \\
PTC & Scheduling Problem with Time Constraints\\
GCSP & Group Cumulative Scheduling Problem \\
2BPHFSP & Two-Stage Bin Packing and Hybrid Flow Shop Scheduling Problem\\
CTW & Cable Tree Wiring Problem\\
CHSP & Cyclic Hoist Scheduling Problem \\
CECSP & Continuous Energy-Constrained Scheduling Problem \\
CuSP & Cumulative Scheduling Problem \\
SBSFMMAL & Simultaneous Balancing and Scheduling of Flexible Mixed Model Assembly Lines\\
SMSDP & steel mill slab design problem \\
KRFP & kernel resource feasibility problem\\
TCSP & Temporal Constraint Satisfaction Problem\\
PJSSP & Pre-emptive Job-Shop scheduling Problem\\
MGAP & Modified Generalized Assignment Problem\\
EOSP & Earth Observation Scheduling Problem \\
SCC & Steel-making and continuous casting \\
OSSP & Open Shop Scheduling Problem\\
FJS & Fixed Job Scheduling\\
RCPSPDC & Resource-constrained Project Scheduling Problem with Discounted Cashflow \\
LSFRP & Liner Shipping Fleet Repositioning Problem\\
BPCTOP & Bulk Port Cargo Throughput Optimisation Problem\\

\bottomrule
\end{tabular}
}
\end{table}



\clearpage
\section{Concept Matching}

In order to automatically find out properties of the articles, we try to find certain concepts in the pdf versions of the articles. We manually defined an ontology of important concepts to look for, and defined regular expressions that would recognize these concepts in the text. We use the \emph{pdfgrep} command to search for the number of occurrences of certain regular expressions in the files. This often clearly identifies the constraints used in the model. We group the results by number of occurrences of the concept in the text of the work. Note that this is only approximate, as we do include the full pdf file in the search. A concept might only be mentioned in some of the title of citations used in the paper, we do count them in our results, as we were not able to remove the bibliography from the main body of the work.

Overall, if a work is not mentioned as using the concept, the the text does not contain a match to the corresponding regular expression. A fundamental limitation of this approach is that it only really works for text written in the language the regular expressions are designed for (in our case English), and not those written in another language. We could overcome this limitation by defining all concepts in other languages as well, and then using a language flag to identify the language the text is written in. 

Note that we only show the first 30 matching entries in each concept category, and list the total number of matches if there are more than 30 matches.


\clearpage
\subsection{Concept Type Concepts}
\label{sec:Concepts}
{\scriptsize
\begin{longtable}{lp{3cm}>{\raggedright\arraybackslash}p{6cm}>{\raggedright\arraybackslash}p{6cm}>{\raggedright\arraybackslash}p{8cm}}
\rowcolor{white}\caption{Works for Concepts of Type Concepts}\\ \toprule
\rowcolor{white}Type & Keyword & High & Medium & Low\\ \midrule\endhead
\bottomrule
\endfoot
Concepts & Allen's algebra &  &  & \\
Concepts & BOM & \href{works/SubulanC22.pdf}{SubulanC22}~\cite{SubulanC22} &  & \href{works/abs-1902-01193.pdf}{abs-1902-01193}~\cite{abs-1902-01193}\\
Concepts & activity & \href{works/TardivoDFMP23.pdf}{TardivoDFMP23}~\cite{TardivoDFMP23}, \href{works/AalianPG23.pdf}{AalianPG23}~\cite{AalianPG23}, \href{works/PovedaAA23.pdf}{PovedaAA23}~\cite{PovedaAA23}, \href{works/TouatBT22.pdf}{TouatBT22}~\cite{TouatBT22}, \href{works/CampeauG22.pdf}{CampeauG22}~\cite{CampeauG22}, \href{works/SubulanC22.pdf}{SubulanC22}~\cite{SubulanC22}, \href{works/SvancaraB22.pdf}{SvancaraB22}~\cite{SvancaraB22}, \href{works/BenderWS21.pdf}{BenderWS21}~\cite{BenderWS21}, \href{works/KlankeBYE21.pdf}{KlankeBYE21}~\cite{KlankeBYE21}, \href{works/HubnerGSV21.pdf}{HubnerGSV21}~\cite{HubnerGSV21}, \href{works/BadicaBI20.pdf}{BadicaBI20}~\cite{BadicaBI20}, \href{works/ZouZ20.pdf}{ZouZ20}~\cite{ZouZ20}, \href{works/Polo-MejiaALB20.pdf}{Polo-MejiaALB20}~\cite{Polo-MejiaALB20}, \href{works/AstrandJZ20.pdf}{AstrandJZ20}~\cite{AstrandJZ20}, \href{works/BadicaBIL19.pdf}{BadicaBIL19}~\cite{BadicaBIL19}, \href{works/abs-1902-09244.pdf}{abs-1902-09244}~\cite{abs-1902-09244}, \href{works/abs-1911-04766.pdf}{abs-1911-04766}~\cite{abs-1911-04766}, \href{works/GeibingerMM19.pdf}{GeibingerMM19}~\cite{GeibingerMM19}, \href{works/MurinR19.pdf}{MurinR19}~\cite{MurinR19}, \href{works/YounespourAKE19.pdf}{YounespourAKE19}~\cite{YounespourAKE19}, \href{works/LaborieRSV18.pdf}{LaborieRSV18}~\cite{LaborieRSV18}, \href{works/GokgurHO18.pdf}{GokgurHO18}~\cite{GokgurHO18}, \href{works/BorghesiBLMB18.pdf}{BorghesiBLMB18}~\cite{BorghesiBLMB18}, \href{works/TangLWSK18.pdf}{TangLWSK18}~\cite{TangLWSK18}, \href{works/MusliuSS18.pdf}{MusliuSS18}~\cite{MusliuSS18}, \href{works/AstrandJZ18.pdf}{AstrandJZ18}~\cite{AstrandJZ18}, \href{works/CappartS17.pdf}{CappartS17}~\cite{CappartS17}, \href{works/Pralet17.pdf}{Pralet17}~\cite{Pralet17}, \href{works/KreterSS17.pdf}{KreterSS17}~\cite{KreterSS17}... (Total: 130) & \href{works/YuraszeckMCCR23.pdf}{YuraszeckMCCR23}~\cite{YuraszeckMCCR23}, \href{works/Bit-Monnot23.pdf}{Bit-Monnot23}~\cite{Bit-Monnot23}, \href{works/BoudreaultSLQ22.pdf}{BoudreaultSLQ22}~\cite{BoudreaultSLQ22}, \href{works/PopovicCGNC22.pdf}{PopovicCGNC22}~\cite{PopovicCGNC22}, \href{works/LunardiBLRV20.pdf}{LunardiBLRV20}~\cite{LunardiBLRV20}, \href{works/YangSS19.pdf}{YangSS19}~\cite{YangSS19}, \href{works/EscobetPQPRA19.pdf}{EscobetPQPRA19}~\cite{EscobetPQPRA19}, \href{works/Novas19.pdf}{Novas19}~\cite{Novas19}, \href{works/ShinBBHO18.pdf}{ShinBBHO18}~\cite{ShinBBHO18}, \href{works/SchuttS16.pdf}{SchuttS16}~\cite{SchuttS16}, \href{works/BoothNB16.pdf}{BoothNB16}~\cite{BoothNB16}, \href{works/TranWDRFOVB16.pdf}{TranWDRFOVB16}~\cite{TranWDRFOVB16}, \href{works/VilimLS15.pdf}{VilimLS15}~\cite{VilimLS15}, \href{works/GoelSHFS15.pdf}{GoelSHFS15}~\cite{GoelSHFS15}, \href{works/DoulabiRP14.pdf}{DoulabiRP14}~\cite{DoulabiRP14}, \href{works/LombardiM13.pdf}{LombardiM13}~\cite{LombardiM13}, \href{works/BonfiettiM12.pdf}{BonfiettiM12}~\cite{BonfiettiM12}, \href{works/ChapadosJR11.pdf}{ChapadosJR11}~\cite{ChapadosJR11}, \href{works/ZibranR11.pdf}{ZibranR11}~\cite{ZibranR11}, \href{works/SchuttFSW09.pdf}{SchuttFSW09}~\cite{SchuttFSW09}, \href{works/PoderB08.pdf}{PoderB08}~\cite{PoderB08}, \href{works/GarridoOS08.pdf}{GarridoOS08}~\cite{GarridoOS08}, \href{works/KrogtLPHJ07.pdf}{KrogtLPHJ07}~\cite{KrogtLPHJ07}, \href{works/Simonis07.pdf}{Simonis07}~\cite{Simonis07}, \href{works/KhayatLR06.pdf}{KhayatLR06}~\cite{KhayatLR06}, \href{works/Geske05.pdf}{Geske05}~\cite{Geske05}, \href{works/MoffittPP05.pdf}{MoffittPP05}~\cite{MoffittPP05}, \href{works/DannaP03.pdf}{DannaP03}~\cite{DannaP03}, \href{works/Bartak02.pdf}{Bartak02}~\cite{Bartak02}... (Total: 34) & \href{works/PrataAN23.pdf}{PrataAN23}~\cite{PrataAN23}, \href{works/CzerniachowskaWZ23.pdf}{CzerniachowskaWZ23}~\cite{CzerniachowskaWZ23}, \href{works/ShaikhK23.pdf}{ShaikhK23}~\cite{ShaikhK23}, \href{works/abs-2312-13682.pdf}{abs-2312-13682}~\cite{abs-2312-13682}, \href{works/SquillaciPR23.pdf}{SquillaciPR23}~\cite{SquillaciPR23}, \href{works/abs-2305-19888.pdf}{abs-2305-19888}~\cite{abs-2305-19888}, \href{works/PerezGSL23.pdf}{PerezGSL23}~\cite{PerezGSL23}, \href{works/HeinzNVH22.pdf}{HeinzNVH22}~\cite{HeinzNVH22}, \href{works/PohlAK22.pdf}{PohlAK22}~\cite{PohlAK22}, \href{works/abs-2211-14492.pdf}{abs-2211-14492}~\cite{abs-2211-14492}, \href{works/HebrardALLCMR22.pdf}{HebrardALLCMR22}~\cite{HebrardALLCMR22}, \href{works/OuelletQ22.pdf}{OuelletQ22}~\cite{OuelletQ22}, \href{works/MullerMKP22.pdf}{MullerMKP22}~\cite{MullerMKP22}, \href{works/YunusogluY22.pdf}{YunusogluY22}~\cite{YunusogluY22}, \href{works/ZhangYW21.pdf}{ZhangYW21}~\cite{ZhangYW21}, \href{works/HillTV21.pdf}{HillTV21}~\cite{HillTV21}, \href{works/GeibingerMM21.pdf}{GeibingerMM21}~\cite{GeibingerMM21}, \href{works/PandeyS21a.pdf}{PandeyS21a}~\cite{PandeyS21a}, \href{works/Astrand0F21.pdf}{Astrand0F21}~\cite{Astrand0F21}, \href{works/QinDCS20.pdf}{QinDCS20}~\cite{QinDCS20}, \href{works/Mercier-AubinGQ20.pdf}{Mercier-AubinGQ20}~\cite{Mercier-AubinGQ20}, \href{works/SacramentoSP20.pdf}{SacramentoSP20}~\cite{SacramentoSP20}, \href{works/NishikawaSTT19.pdf}{NishikawaSTT19}~\cite{NishikawaSTT19}, \href{works/abs-1902-01193.pdf}{abs-1902-01193}~\cite{abs-1902-01193}, \href{works/Tom19.pdf}{Tom19}~\cite{Tom19}, \href{works/GalleguillosKSB19.pdf}{GalleguillosKSB19}~\cite{GalleguillosKSB19}, \href{works/NishikawaSTT18.pdf}{NishikawaSTT18}~\cite{NishikawaSTT18}, \href{works/NishikawaSTT18a.pdf}{NishikawaSTT18a}~\cite{NishikawaSTT18a}, \href{works/DemirovicS18.pdf}{DemirovicS18}~\cite{DemirovicS18}... (Total: 72)\\
Concepts & batch process & \href{works/LacknerMMWW23.pdf}{LacknerMMWW23}~\cite{LacknerMMWW23}, \href{works/LacknerMMWW21.pdf}{LacknerMMWW21}~\cite{LacknerMMWW21}, \href{works/QinWSLS21.pdf}{QinWSLS21}~\cite{QinWSLS21}, \href{works/NovaraNH16.pdf}{NovaraNH16}~\cite{NovaraNH16}, \href{works/KoschB14.pdf}{KoschB14}~\cite{KoschB14} & \href{works/TangB20.pdf}{TangB20}~\cite{TangB20}, \href{works/NovasH10.pdf}{NovasH10}~\cite{NovasH10}, \href{works/Vilim02.pdf}{Vilim02}~\cite{Vilim02}, \href{works/SimonisC95.pdf}{SimonisC95}~\cite{SimonisC95} & \href{works/PrataAN23.pdf}{PrataAN23}~\cite{PrataAN23}, \href{works/IsikYA23.pdf}{IsikYA23}~\cite{IsikYA23}, \href{works/YuraszeckMCCR23.pdf}{YuraszeckMCCR23}~\cite{YuraszeckMCCR23}, \href{works/YunusogluY22.pdf}{YunusogluY22}~\cite{YunusogluY22}, \href{works/MullerMKP22.pdf}{MullerMKP22}~\cite{MullerMKP22}, \href{works/SvancaraB22.pdf}{SvancaraB22}~\cite{SvancaraB22}, \href{works/OujanaAYB22.pdf}{OujanaAYB22}~\cite{OujanaAYB22}, \href{works/LuoB22.pdf}{LuoB22}~\cite{LuoB22}, \href{works/LiFJZLL22.pdf}{LiFJZLL22}~\cite{LiFJZLL22}, \href{works/ColT22.pdf}{ColT22}~\cite{ColT22}, \href{works/AbreuN22.pdf}{AbreuN22}~\cite{AbreuN22}, \href{works/GeitzGSSW22.pdf}{GeitzGSSW22}~\cite{GeitzGSSW22}, \href{works/FanXG21.pdf}{FanXG21}~\cite{FanXG21}, \href{works/ZhangYW21.pdf}{ZhangYW21}~\cite{ZhangYW21}, \href{works/KlankeBYE21.pdf}{KlankeBYE21}~\cite{KlankeBYE21}, \href{works/MengZRZL20.pdf}{MengZRZL20}~\cite{MengZRZL20}, \href{works/EscobetPQPRA19.pdf}{EscobetPQPRA19}~\cite{EscobetPQPRA19}, \href{works/Ham18.pdf}{Ham18}~\cite{Ham18}, \href{works/FahimiOQ18.pdf}{FahimiOQ18}~\cite{FahimiOQ18}, \href{works/LaborieRSV18.pdf}{LaborieRSV18}~\cite{LaborieRSV18}, \href{works/CauwelaertDMS16.pdf}{CauwelaertDMS16}~\cite{CauwelaertDMS16}, \href{works/GrimesH10.pdf}{GrimesH10}~\cite{GrimesH10}, \href{works/Simonis07.pdf}{Simonis07}~\cite{Simonis07}, \href{works/VilimBC05.pdf}{VilimBC05}~\cite{VilimBC05}, \href{works/ArtiguesBF04.pdf}{ArtiguesBF04}~\cite{ArtiguesBF04}, \href{works/Vilim04.pdf}{Vilim04}~\cite{Vilim04}\\
Concepts & bill of material &  &  & \href{works/Simonis07.pdf}{Simonis07}~\cite{Simonis07}\\
Concepts & buffer-capacity &  & \href{works/SureshMOK06.pdf}{SureshMOK06}~\cite{SureshMOK06} & \href{works/LiFJZLL22.pdf}{LiFJZLL22}~\cite{LiFJZLL22}, \href{works/OujanaAYB22.pdf}{OujanaAYB22}~\cite{OujanaAYB22}, \href{works/RiahiNS018.pdf}{RiahiNS018}~\cite{RiahiNS018}, \href{works/BonfiettiLBM14.pdf}{BonfiettiLBM14}~\cite{BonfiettiLBM14}, \href{works/NovasH14.pdf}{NovasH14}~\cite{NovasH14}, \href{works/TerekhovTDB14.pdf}{TerekhovTDB14}~\cite{TerekhovTDB14}, \href{works/ZeballosH05.pdf}{ZeballosH05}~\cite{ZeballosH05}\\
Concepts & cmax & \href{works/JuvinHHL23.pdf}{JuvinHHL23}~\cite{JuvinHHL23}, \href{works/YuraszeckMCCR23.pdf}{YuraszeckMCCR23}~\cite{YuraszeckMCCR23}, \href{works/AbreuNP23.pdf}{AbreuNP23}~\cite{AbreuNP23}, \href{works/YuraszeckMC23.pdf}{YuraszeckMC23}~\cite{YuraszeckMC23}, \href{works/KameugneFND23.pdf}{KameugneFND23}~\cite{KameugneFND23}, \href{works/NaderiRR23.pdf}{NaderiRR23}~\cite{NaderiRR23}, \href{works/abs-2305-19888.pdf}{abs-2305-19888}~\cite{abs-2305-19888}, \href{works/IsikYA23.pdf}{IsikYA23}~\cite{IsikYA23}, \href{works/YunusogluY22.pdf}{YunusogluY22}~\cite{YunusogluY22}, \href{works/FetgoD22.pdf}{FetgoD22}~\cite{FetgoD22}, \href{works/AbreuN22.pdf}{AbreuN22}~\cite{AbreuN22}, \href{works/abs-2211-14492.pdf}{abs-2211-14492}~\cite{abs-2211-14492}, \href{works/ZhangBB22.pdf}{ZhangBB22}~\cite{ZhangBB22}, \href{works/QinWSLS21.pdf}{QinWSLS21}~\cite{QinWSLS21}, \href{works/AbohashimaEG21.pdf}{AbohashimaEG21}~\cite{AbohashimaEG21}, \href{works/ArmstrongGOS21.pdf}{ArmstrongGOS21}~\cite{ArmstrongGOS21}, \href{works/Polo-MejiaALB20.pdf}{Polo-MejiaALB20}~\cite{Polo-MejiaALB20}, \href{works/QinDCS20.pdf}{QinDCS20}~\cite{QinDCS20}, \href{works/MejiaY20.pdf}{MejiaY20}~\cite{MejiaY20}, \href{works/MengZRZL20.pdf}{MengZRZL20}~\cite{MengZRZL20}, \href{works/GodetLHS20.pdf}{GodetLHS20}~\cite{GodetLHS20}, \href{works/WikarekS19.pdf}{WikarekS19}~\cite{WikarekS19}, \href{works/YounespourAKE19.pdf}{YounespourAKE19}~\cite{YounespourAKE19}, \href{works/MalapertN19.pdf}{MalapertN19}~\cite{MalapertN19}, \href{works/Ham18.pdf}{Ham18}~\cite{Ham18}, \href{works/GedikKEK18.pdf}{GedikKEK18}~\cite{GedikKEK18}, \href{works/KameugneFGOQ18.pdf}{KameugneFGOQ18}~\cite{KameugneFGOQ18}, \href{works/HebrardHJMPV16.pdf}{HebrardHJMPV16}~\cite{HebrardHJMPV16}, \href{works/VilimLS15.pdf}{VilimLS15}~\cite{VilimLS15}... (Total: 42) & \href{works/Mehdizadeh-Somarin23.pdf}{Mehdizadeh-Somarin23}~\cite{Mehdizadeh-Somarin23}, \href{works/BoudreaultSLQ22.pdf}{BoudreaultSLQ22}~\cite{BoudreaultSLQ22}, \href{works/MullerMKP22.pdf}{MullerMKP22}~\cite{MullerMKP22}, \href{works/ArmstrongGOS22.pdf}{ArmstrongGOS22}~\cite{ArmstrongGOS22}, \href{works/HamPK21.pdf}{HamPK21}~\cite{HamPK21}, \href{works/AbreuAPNM21.pdf}{AbreuAPNM21}~\cite{AbreuAPNM21}, \href{works/ParkUJR19.pdf}{ParkUJR19}~\cite{ParkUJR19}, \href{works/Novas19.pdf}{Novas19}~\cite{Novas19}, \href{works/ArbaouiY18.pdf}{ArbaouiY18}~\cite{ArbaouiY18}, \href{works/WangMD15.pdf}{WangMD15}~\cite{WangMD15}, \href{works/ZhouGL15.pdf}{ZhouGL15}~\cite{ZhouGL15}, \href{works/ZhangLS12.pdf}{ZhangLS12}~\cite{ZhangLS12}, \href{works/BeckFW11.pdf}{BeckFW11}~\cite{BeckFW11}, \href{works/BartakSR10.pdf}{BartakSR10}~\cite{BartakSR10}, \href{works/MoffittPP05.pdf}{MoffittPP05}~\cite{MoffittPP05}, \href{works/Muscettola02.pdf}{Muscettola02}~\cite{Muscettola02}, \href{works/ArtiguesR00.pdf}{ArtiguesR00}~\cite{ArtiguesR00}, \href{works/SourdN00.pdf}{SourdN00}~\cite{SourdN00} & \href{works/JuvinHL23.pdf}{JuvinHL23}~\cite{JuvinHL23}, \href{works/Teppan22.pdf}{Teppan22}~\cite{Teppan22}, \href{works/ZhangYW21.pdf}{ZhangYW21}~\cite{ZhangYW21}, \href{works/HanenKP21.pdf}{HanenKP21}~\cite{HanenKP21}, \href{works/HubnerGSV21.pdf}{HubnerGSV21}~\cite{HubnerGSV21}, \href{works/GokgurHO18.pdf}{GokgurHO18}~\cite{GokgurHO18}, \href{works/LiuCGM17.pdf}{LiuCGM17}~\cite{LiuCGM17}, \href{works/BofillCSV17.pdf}{BofillCSV17}~\cite{BofillCSV17}, \href{works/SialaAH15.pdf}{SialaAH15}~\cite{SialaAH15}, \href{works/KoschB14.pdf}{KoschB14}~\cite{KoschB14}, \href{works/SchuttFSW13.pdf}{SchuttFSW13}~\cite{SchuttFSW13}, \href{works/GuSW12.pdf}{GuSW12}~\cite{GuSW12}, \href{works/abs-1009-0347.pdf}{abs-1009-0347}~\cite{abs-1009-0347}, \href{works/WatsonB08.pdf}{WatsonB08}~\cite{WatsonB08}, \href{works/LiessM08.pdf}{LiessM08}~\cite{LiessM08}, \href{works/AkkerDH07.pdf}{AkkerDH07}~\cite{AkkerDH07}, \href{works/KeriK07.pdf}{KeriK07}~\cite{KeriK07}, \href{works/KhayatLR06.pdf}{KhayatLR06}~\cite{KhayatLR06}, \href{works/BaptisteP00.pdf}{BaptisteP00}~\cite{BaptisteP00}, \href{works/FocacciLN00.pdf}{FocacciLN00}~\cite{FocacciLN00}\\
Concepts & completion-time & \href{works/PrataAN23.pdf}{PrataAN23}~\cite{PrataAN23}, \href{works/JuvinHL23.pdf}{JuvinHL23}~\cite{JuvinHL23}, \href{works/AbreuNP23.pdf}{AbreuNP23}~\cite{AbreuNP23}, \href{works/Mehdizadeh-Somarin23.pdf}{Mehdizadeh-Somarin23}~\cite{Mehdizadeh-Somarin23}, \href{works/AlfieriGPS23.pdf}{AlfieriGPS23}~\cite{AlfieriGPS23}, \href{works/NaderiRR23.pdf}{NaderiRR23}~\cite{NaderiRR23}, \href{works/KameugneFND23.pdf}{KameugneFND23}~\cite{KameugneFND23}, \href{works/YuraszeckMPV22.pdf}{YuraszeckMPV22}~\cite{YuraszeckMPV22}, \href{works/AbreuN22.pdf}{AbreuN22}~\cite{AbreuN22}, \href{works/YunusogluY22.pdf}{YunusogluY22}~\cite{YunusogluY22}, \href{works/SubulanC22.pdf}{SubulanC22}~\cite{SubulanC22}, \href{works/OuelletQ22.pdf}{OuelletQ22}~\cite{OuelletQ22}, \href{works/FetgoD22.pdf}{FetgoD22}~\cite{FetgoD22}, \href{works/KlankeBYE21.pdf}{KlankeBYE21}~\cite{KlankeBYE21}, \href{works/Bedhief21.pdf}{Bedhief21}~\cite{Bedhief21}, \href{works/ArmstrongGOS21.pdf}{ArmstrongGOS21}~\cite{ArmstrongGOS21}, \href{works/MejiaY20.pdf}{MejiaY20}~\cite{MejiaY20}, \href{works/LunardiBLRV20.pdf}{LunardiBLRV20}~\cite{LunardiBLRV20}, \href{works/QinDCS20.pdf}{QinDCS20}~\cite{QinDCS20}, \href{works/YounespourAKE19.pdf}{YounespourAKE19}~\cite{YounespourAKE19}, \href{works/FahimiOQ18.pdf}{FahimiOQ18}~\cite{FahimiOQ18}, \href{works/RiahiNS018.pdf}{RiahiNS018}~\cite{RiahiNS018}, \href{works/ZhangW18.pdf}{ZhangW18}~\cite{ZhangW18}, \href{works/ArbaouiY18.pdf}{ArbaouiY18}~\cite{ArbaouiY18}, \href{works/GedikKEK18.pdf}{GedikKEK18}~\cite{GedikKEK18}, \href{works/KameugneFGOQ18.pdf}{KameugneFGOQ18}~\cite{KameugneFGOQ18}, \href{works/HebrardHJMPV16.pdf}{HebrardHJMPV16}~\cite{HebrardHJMPV16}, \href{works/GingrasQ16.pdf}{GingrasQ16}~\cite{GingrasQ16}, \href{works/DejemeppeCS15.pdf}{DejemeppeCS15}~\cite{DejemeppeCS15}... (Total: 55) & \href{works/CzerniachowskaWZ23.pdf}{CzerniachowskaWZ23}~\cite{CzerniachowskaWZ23}, \href{works/abs-2305-19888.pdf}{abs-2305-19888}~\cite{abs-2305-19888}, \href{works/MullerMKP22.pdf}{MullerMKP22}~\cite{MullerMKP22}, \href{works/ColT22.pdf}{ColT22}~\cite{ColT22}, \href{works/Teppan22.pdf}{Teppan22}~\cite{Teppan22}, \href{works/TouatBT22.pdf}{TouatBT22}~\cite{TouatBT22}, \href{works/OujanaAYB22.pdf}{OujanaAYB22}~\cite{OujanaAYB22}, \href{works/HeinzNVH22.pdf}{HeinzNVH22}~\cite{HeinzNVH22}, \href{works/abs-2211-14492.pdf}{abs-2211-14492}~\cite{abs-2211-14492}, \href{works/LiFJZLL22.pdf}{LiFJZLL22}~\cite{LiFJZLL22}, \href{works/ZhangBB22.pdf}{ZhangBB22}~\cite{ZhangBB22}, \href{works/AbreuAPNM21.pdf}{AbreuAPNM21}~\cite{AbreuAPNM21}, \href{works/HanenKP21.pdf}{HanenKP21}~\cite{HanenKP21}, \href{works/FanXG21.pdf}{FanXG21}~\cite{FanXG21}, \href{works/GeibingerMM21.pdf}{GeibingerMM21}~\cite{GeibingerMM21}, \href{works/QinWSLS21.pdf}{QinWSLS21}~\cite{QinWSLS21}, \href{works/NattafM20.pdf}{NattafM20}~\cite{NattafM20}, \href{works/Mercier-AubinGQ20.pdf}{Mercier-AubinGQ20}~\cite{Mercier-AubinGQ20}, \href{works/Polo-MejiaALB20.pdf}{Polo-MejiaALB20}~\cite{Polo-MejiaALB20}, \href{works/YangSS19.pdf}{YangSS19}~\cite{YangSS19}, \href{works/abs-1902-09244.pdf}{abs-1902-09244}~\cite{abs-1902-09244}, \href{works/BogaerdtW19.pdf}{BogaerdtW19}~\cite{BogaerdtW19}, \href{works/abs-1911-04766.pdf}{abs-1911-04766}~\cite{abs-1911-04766}, \href{works/MalapertN19.pdf}{MalapertN19}~\cite{MalapertN19}, \href{works/GeibingerMM19.pdf}{GeibingerMM19}~\cite{GeibingerMM19}, \href{works/ParkUJR19.pdf}{ParkUJR19}~\cite{ParkUJR19}, \href{works/Ham18.pdf}{Ham18}~\cite{Ham18}, \href{works/OuelletQ18.pdf}{OuelletQ18}~\cite{OuelletQ18}, \href{works/KreterSS17.pdf}{KreterSS17}~\cite{KreterSS17}... (Total: 55) & \href{works/abs-2402-00459.pdf}{abs-2402-00459}~\cite{abs-2402-00459}, \href{works/TasselGS23.pdf}{TasselGS23}~\cite{TasselGS23}, \href{works/MontemanniD23a.pdf}{MontemanniD23a}~\cite{MontemanniD23a}, \href{works/AkramNHRSA23.pdf}{AkramNHRSA23}~\cite{AkramNHRSA23}, \href{works/IsikYA23.pdf}{IsikYA23}~\cite{IsikYA23}, \href{works/abs-2306-05747.pdf}{abs-2306-05747}~\cite{abs-2306-05747}, \href{works/PerezGSL23.pdf}{PerezGSL23}~\cite{PerezGSL23}, \href{works/JuvinHHL23.pdf}{JuvinHHL23}~\cite{JuvinHHL23}, \href{works/FarsiTM22.pdf}{FarsiTM22}~\cite{FarsiTM22}, \href{works/PopovicCGNC22.pdf}{PopovicCGNC22}~\cite{PopovicCGNC22}, \href{works/PohlAK22.pdf}{PohlAK22}~\cite{PohlAK22}, \href{works/GeitzGSSW22.pdf}{GeitzGSSW22}~\cite{GeitzGSSW22}, \href{works/CampeauG22.pdf}{CampeauG22}~\cite{CampeauG22}, \href{works/ZhangJZL22.pdf}{ZhangJZL22}~\cite{ZhangJZL22}, \href{works/WinterMMW22.pdf}{WinterMMW22}~\cite{WinterMMW22}, \href{works/ArmstrongGOS22.pdf}{ArmstrongGOS22}~\cite{ArmstrongGOS22}, \href{works/HubnerGSV21.pdf}{HubnerGSV21}~\cite{HubnerGSV21}, \href{works/VlkHT21.pdf}{VlkHT21}~\cite{VlkHT21}, \href{works/PandeyS21a.pdf}{PandeyS21a}~\cite{PandeyS21a}, \href{works/HamPK21.pdf}{HamPK21}~\cite{HamPK21}, \href{works/WessenCS20.pdf}{WessenCS20}~\cite{WessenCS20}, \href{works/BadicaBI20.pdf}{BadicaBI20}~\cite{BadicaBI20}, \href{works/MengZRZL20.pdf}{MengZRZL20}~\cite{MengZRZL20}, \href{works/MokhtarzadehTNF20.pdf}{MokhtarzadehTNF20}~\cite{MokhtarzadehTNF20}, \href{works/AntuoriHHEN20.pdf}{AntuoriHHEN20}~\cite{AntuoriHHEN20}, \href{works/GodetLHS20.pdf}{GodetLHS20}~\cite{GodetLHS20}, \href{works/SacramentoSP20.pdf}{SacramentoSP20}~\cite{SacramentoSP20}, \href{works/ZouZ20.pdf}{ZouZ20}~\cite{ZouZ20}, \href{works/AstrandJZ20.pdf}{AstrandJZ20}~\cite{AstrandJZ20}... (Total: 87)\\
Concepts & continuous-process &  &  & \href{works/FarsiTM22.pdf}{FarsiTM22}~\cite{FarsiTM22}, \href{works/GaySS14.pdf}{GaySS14}~\cite{GaySS14}, \href{works/Bartak02.pdf}{Bartak02}~\cite{Bartak02}, \href{works/SimonisC95.pdf}{SimonisC95}~\cite{SimonisC95}\\
Concepts & distributed & \href{works/PrataAN23.pdf}{PrataAN23}~\cite{PrataAN23}, \href{works/NaderiRR23.pdf}{NaderiRR23}~\cite{NaderiRR23}, \href{works/MengZRZL20.pdf}{MengZRZL20}~\cite{MengZRZL20}, \href{works/He0GLW18.pdf}{He0GLW18}~\cite{He0GLW18}, \href{works/TranPZLDB18.pdf}{TranPZLDB18}~\cite{TranPZLDB18}, \href{works/BridiLBBM16.pdf}{BridiLBBM16}~\cite{BridiLBBM16}, \href{works/BridiBLMB16.pdf}{BridiBLMB16}~\cite{BridiBLMB16}, \href{works/ZhouGL15.pdf}{ZhouGL15}~\cite{ZhouGL15}, \href{works/BonfiettiLM14.pdf}{BonfiettiLM14}~\cite{BonfiettiLM14}, \href{works/TerekhovTDB14.pdf}{TerekhovTDB14}~\cite{TerekhovTDB14}, \href{works/BartakS11.pdf}{BartakS11}~\cite{BartakS11}, \href{works/BartakSR10.pdf}{BartakSR10}~\cite{BartakSR10}, \href{works/RuggieroBBMA09.pdf}{RuggieroBBMA09}~\cite{RuggieroBBMA09}, \href{works/HoeveGSL07.pdf}{HoeveGSL07}~\cite{HoeveGSL07}, \href{works/RossiTHP07.pdf}{RossiTHP07}~\cite{RossiTHP07}, \href{works/BeckW07.pdf}{BeckW07}~\cite{BeckW07}, \href{works/SureshMOK06.pdf}{SureshMOK06}~\cite{SureshMOK06}, \href{works/GomesHS06.pdf}{GomesHS06}~\cite{GomesHS06}, \href{works/Geske05.pdf}{Geske05}~\cite{Geske05}, \href{works/LammaMM97.pdf}{LammaMM97}~\cite{LammaMM97} & \href{works/IsikYA23.pdf}{IsikYA23}~\cite{IsikYA23}, \href{works/ShaikhK23.pdf}{ShaikhK23}~\cite{ShaikhK23}, \href{works/AbreuNP23.pdf}{AbreuNP23}~\cite{AbreuNP23}, \href{works/OujanaAYB22.pdf}{OujanaAYB22}~\cite{OujanaAYB22}, \href{works/JungblutK22.pdf}{JungblutK22}~\cite{JungblutK22}, \href{works/AbreuN22.pdf}{AbreuN22}~\cite{AbreuN22}, \href{works/YuraszeckMPV22.pdf}{YuraszeckMPV22}~\cite{YuraszeckMPV22}, \href{works/AbreuAPNM21.pdf}{AbreuAPNM21}~\cite{AbreuAPNM21}, \href{works/MokhtarzadehTNF20.pdf}{MokhtarzadehTNF20}~\cite{MokhtarzadehTNF20}, \href{works/ZouZ20.pdf}{ZouZ20}~\cite{ZouZ20}, \href{works/NishikawaSTT19.pdf}{NishikawaSTT19}~\cite{NishikawaSTT19}, \href{works/BorghesiBLMB18.pdf}{BorghesiBLMB18}~\cite{BorghesiBLMB18}, \href{works/ZhangW18.pdf}{ZhangW18}~\cite{ZhangW18}, \href{works/ZarandiKS16.pdf}{ZarandiKS16}~\cite{ZarandiKS16}, \href{works/AlesioNBG14.pdf}{AlesioNBG14}~\cite{AlesioNBG14}, \href{works/BegB13.pdf}{BegB13}~\cite{BegB13}, \href{works/TranTDB13.pdf}{TranTDB13}~\cite{TranTDB13}, \href{works/HermenierDL11.pdf}{HermenierDL11}~\cite{HermenierDL11}, \href{works/LopesCSM10.pdf}{LopesCSM10}~\cite{LopesCSM10}, \href{works/SunLYL10.pdf}{SunLYL10}~\cite{SunLYL10}, \href{works/BeniniBGM06.pdf}{BeniniBGM06}~\cite{BeniniBGM06}, \href{works/ZhuS02.pdf}{ZhuS02}~\cite{ZhuS02}, \href{works/SchildW00.pdf}{SchildW00}~\cite{SchildW00}, \href{works/Wallace96.pdf}{Wallace96}~\cite{Wallace96} & \href{works/YuraszeckMC23.pdf}{YuraszeckMC23}~\cite{YuraszeckMC23}, \href{works/KimCMLLP23.pdf}{KimCMLLP23}~\cite{KimCMLLP23}, \href{works/Bit-Monnot23.pdf}{Bit-Monnot23}~\cite{Bit-Monnot23}, \href{works/AlfieriGPS23.pdf}{AlfieriGPS23}~\cite{AlfieriGPS23}, \href{works/MontemanniD23.pdf}{MontemanniD23}~\cite{MontemanniD23}, \href{works/abs-2305-19888.pdf}{abs-2305-19888}~\cite{abs-2305-19888}, \href{works/SquillaciPR23.pdf}{SquillaciPR23}~\cite{SquillaciPR23}, \href{works/GurPAE23.pdf}{GurPAE23}~\cite{GurPAE23}, \href{works/AkramNHRSA23.pdf}{AkramNHRSA23}~\cite{AkramNHRSA23}, \href{works/abs-2211-14492.pdf}{abs-2211-14492}~\cite{abs-2211-14492}, \href{works/HeinzNVH22.pdf}{HeinzNVH22}~\cite{HeinzNVH22}, \href{works/TouatBT22.pdf}{TouatBT22}~\cite{TouatBT22}, \href{works/BoudreaultSLQ22.pdf}{BoudreaultSLQ22}~\cite{BoudreaultSLQ22}, \href{works/Teppan22.pdf}{Teppan22}~\cite{Teppan22}, \href{works/ColT22.pdf}{ColT22}~\cite{ColT22}, \href{works/LiFJZLL22.pdf}{LiFJZLL22}~\cite{LiFJZLL22}, \href{works/FarsiTM22.pdf}{FarsiTM22}~\cite{FarsiTM22}, \href{works/WinterMMW22.pdf}{WinterMMW22}~\cite{WinterMMW22}, \href{works/ZhangBB22.pdf}{ZhangBB22}~\cite{ZhangBB22}, \href{works/HamPK21.pdf}{HamPK21}~\cite{HamPK21}, \href{works/GeibingerKKMMW21.pdf}{GeibingerKKMMW21}~\cite{GeibingerKKMMW21}, \href{works/PandeyS21a.pdf}{PandeyS21a}~\cite{PandeyS21a}, \href{works/FanXG21.pdf}{FanXG21}~\cite{FanXG21}, \href{works/BenderWS21.pdf}{BenderWS21}~\cite{BenderWS21}, \href{works/KovacsTKSG21.pdf}{KovacsTKSG21}~\cite{KovacsTKSG21}, \href{works/ZhangYW21.pdf}{ZhangYW21}~\cite{ZhangYW21}, \href{works/VlkHT21.pdf}{VlkHT21}~\cite{VlkHT21}, \href{works/SacramentoSP20.pdf}{SacramentoSP20}~\cite{SacramentoSP20}, \href{works/GroleazNS20a.pdf}{GroleazNS20a}~\cite{GroleazNS20a}... (Total: 106)\\
Concepts & due-date & \href{works/OujanaAYB22.pdf}{OujanaAYB22}~\cite{OujanaAYB22}, \href{works/ColT22.pdf}{ColT22}~\cite{ColT22}, \href{works/FanXG21.pdf}{FanXG21}~\cite{FanXG21}, \href{works/AntuoriHHEN21.pdf}{AntuoriHHEN21}~\cite{AntuoriHHEN21}, \href{works/AntuoriHHEN20.pdf}{AntuoriHHEN20}~\cite{AntuoriHHEN20}, \href{works/TangB20.pdf}{TangB20}~\cite{TangB20}, \href{works/Mercier-AubinGQ20.pdf}{Mercier-AubinGQ20}~\cite{Mercier-AubinGQ20}, \href{works/abs-1902-09244.pdf}{abs-1902-09244}~\cite{abs-1902-09244}, \href{works/Novas19.pdf}{Novas19}~\cite{Novas19}, \href{works/abs-1911-04766.pdf}{abs-1911-04766}~\cite{abs-1911-04766}, \href{works/GoldwaserS18.pdf}{GoldwaserS18}~\cite{GoldwaserS18}, \href{works/Tesch18.pdf}{Tesch18}~\cite{Tesch18}, \href{works/GoldwaserS17.pdf}{GoldwaserS17}~\cite{GoldwaserS17}, \href{works/NovaraNH16.pdf}{NovaraNH16}~\cite{NovaraNH16}, \href{works/BajestaniB15.pdf}{BajestaniB15}~\cite{BajestaniB15}, \href{works/DoulabiRP14.pdf}{DoulabiRP14}~\cite{DoulabiRP14}, \href{works/KoschB14.pdf}{KoschB14}~\cite{KoschB14}, \href{works/HoundjiSWD14.pdf}{HoundjiSWD14}~\cite{HoundjiSWD14}, \href{works/BajestaniB13.pdf}{BajestaniB13}~\cite{BajestaniB13}, \href{works/LimtanyakulS12.pdf}{LimtanyakulS12}~\cite{LimtanyakulS12}, \href{works/KelbelH11.pdf}{KelbelH11}~\cite{KelbelH11}, \href{works/BajestaniB11.pdf}{BajestaniB11}~\cite{BajestaniB11}, \href{works/NovasH10.pdf}{NovasH10}~\cite{NovasH10}, \href{works/ZeballosQH10.pdf}{ZeballosQH10}~\cite{ZeballosQH10}, \href{works/BartakSR10.pdf}{BartakSR10}~\cite{BartakSR10}, \href{works/MonetteDH09.pdf}{MonetteDH09}~\cite{MonetteDH09}, \href{works/BidotVLB09.pdf}{BidotVLB09}~\cite{BidotVLB09}, \href{works/Simonis07.pdf}{Simonis07}~\cite{Simonis07}, \href{works/KrogtLPHJ07.pdf}{KrogtLPHJ07}~\cite{KrogtLPHJ07}... (Total: 37) & \href{works/PrataAN23.pdf}{PrataAN23}~\cite{PrataAN23}, \href{works/LacknerMMWW23.pdf}{LacknerMMWW23}~\cite{LacknerMMWW23}, \href{works/IsikYA23.pdf}{IsikYA23}~\cite{IsikYA23}, \href{works/NaderiRR23.pdf}{NaderiRR23}~\cite{NaderiRR23}, \href{works/YunusogluY22.pdf}{YunusogluY22}~\cite{YunusogluY22}, \href{works/abs-2211-14492.pdf}{abs-2211-14492}~\cite{abs-2211-14492}, \href{works/WinterMMW22.pdf}{WinterMMW22}~\cite{WinterMMW22}, \href{works/LacknerMMWW21.pdf}{LacknerMMWW21}~\cite{LacknerMMWW21}, \href{works/GeibingerMM21.pdf}{GeibingerMM21}~\cite{GeibingerMM21}, \href{works/GroleazNS20a.pdf}{GroleazNS20a}~\cite{GroleazNS20a}, \href{works/GeibingerMM19.pdf}{GeibingerMM19}~\cite{GeibingerMM19}, \href{works/FahimiOQ18.pdf}{FahimiOQ18}~\cite{FahimiOQ18}, \href{works/ZarandiKS16.pdf}{ZarandiKS16}~\cite{ZarandiKS16}, \href{works/GrimesIOS14.pdf}{GrimesIOS14}~\cite{GrimesIOS14}, \href{works/HeinzSB13.pdf}{HeinzSB13}~\cite{HeinzSB13}, \href{works/GrimesH11.pdf}{GrimesH11}~\cite{GrimesH11}, \href{works/LombardiM10a.pdf}{LombardiM10a}~\cite{LombardiM10a}, \href{works/MakMS10.pdf}{MakMS10}~\cite{MakMS10}, \href{works/SchuttW10.pdf}{SchuttW10}~\cite{SchuttW10}, \href{works/Davenport10.pdf}{Davenport10}~\cite{Davenport10}, \href{works/ThiruvadyBME09.pdf}{ThiruvadyBME09}~\cite{ThiruvadyBME09}, \href{works/abs-0907-0939.pdf}{abs-0907-0939}~\cite{abs-0907-0939}, \href{works/MouraSCL08a.pdf}{MouraSCL08a}~\cite{MouraSCL08a}, \href{works/Limtanyakul07.pdf}{Limtanyakul07}~\cite{Limtanyakul07}, \href{works/SadykovW06.pdf}{SadykovW06}~\cite{SadykovW06}, \href{works/Hooker05a.pdf}{Hooker05a}~\cite{Hooker05a}, \href{works/ZeballosH05.pdf}{ZeballosH05}~\cite{ZeballosH05}, \href{works/ChuX05.pdf}{ChuX05}~\cite{ChuX05}, \href{works/QuirogaZH05.pdf}{QuirogaZH05}~\cite{QuirogaZH05}... (Total: 35) & \href{works/abs-2402-00459.pdf}{abs-2402-00459}~\cite{abs-2402-00459}, \href{works/YuraszeckMC23.pdf}{YuraszeckMC23}~\cite{YuraszeckMC23}, \href{works/KimCMLLP23.pdf}{KimCMLLP23}~\cite{KimCMLLP23}, \href{works/JuvinHHL23.pdf}{JuvinHHL23}~\cite{JuvinHHL23}, \href{works/ZhangJZL22.pdf}{ZhangJZL22}~\cite{ZhangJZL22}, \href{works/SubulanC22.pdf}{SubulanC22}~\cite{SubulanC22}, \href{works/TouatBT22.pdf}{TouatBT22}~\cite{TouatBT22}, \href{works/YuraszeckMPV22.pdf}{YuraszeckMPV22}~\cite{YuraszeckMPV22}, \href{works/MullerMKP22.pdf}{MullerMKP22}~\cite{MullerMKP22}, \href{works/KlankeBYE21.pdf}{KlankeBYE21}~\cite{KlankeBYE21}, \href{works/HubnerGSV21.pdf}{HubnerGSV21}~\cite{HubnerGSV21}, \href{works/Bedhief21.pdf}{Bedhief21}~\cite{Bedhief21}, \href{works/KovacsTKSG21.pdf}{KovacsTKSG21}~\cite{KovacsTKSG21}, \href{works/VlkHT21.pdf}{VlkHT21}~\cite{VlkHT21}, \href{works/HanenKP21.pdf}{HanenKP21}~\cite{HanenKP21}, \href{works/LunardiBLRV20.pdf}{LunardiBLRV20}~\cite{LunardiBLRV20}, \href{works/MejiaY20.pdf}{MejiaY20}~\cite{MejiaY20}, \href{works/Polo-MejiaALB20.pdf}{Polo-MejiaALB20}~\cite{Polo-MejiaALB20}, \href{works/GroleazNS20.pdf}{GroleazNS20}~\cite{GroleazNS20}, \href{works/AstrandJZ20.pdf}{AstrandJZ20}~\cite{AstrandJZ20}, \href{works/ParkUJR19.pdf}{ParkUJR19}~\cite{ParkUJR19}, \href{works/EscobetPQPRA19.pdf}{EscobetPQPRA19}~\cite{EscobetPQPRA19}, \href{works/GokgurHO18.pdf}{GokgurHO18}~\cite{GokgurHO18}, \href{works/GedikKEK18.pdf}{GedikKEK18}~\cite{GedikKEK18}, \href{works/LaborieRSV18.pdf}{LaborieRSV18}~\cite{LaborieRSV18}, \href{works/Laborie18a.pdf}{Laborie18a}~\cite{Laborie18a}, \href{works/Ham18.pdf}{Ham18}~\cite{Ham18}, \href{works/Pralet17.pdf}{Pralet17}~\cite{Pralet17}, \href{works/Hooker17.pdf}{Hooker17}~\cite{Hooker17}... (Total: 64)\\
Concepts & earliness & \href{works/PrataAN23.pdf}{PrataAN23}~\cite{PrataAN23}, \href{works/KimCMLLP23.pdf}{KimCMLLP23}~\cite{KimCMLLP23}, \href{works/TouatBT22.pdf}{TouatBT22}~\cite{TouatBT22}, \href{works/PohlAK22.pdf}{PohlAK22}~\cite{PohlAK22}, \href{works/abs-1902-09244.pdf}{abs-1902-09244}~\cite{abs-1902-09244}, \href{works/LaborieRSV18.pdf}{LaborieRSV18}~\cite{LaborieRSV18}, \href{works/ZarandiKS16.pdf}{ZarandiKS16}~\cite{ZarandiKS16}, \href{works/LombardiM12.pdf}{LombardiM12}~\cite{LombardiM12}, \href{works/KelbelH11.pdf}{KelbelH11}~\cite{KelbelH11}, \href{works/GrimesH11.pdf}{GrimesH11}~\cite{GrimesH11}, \href{works/Laborie09.pdf}{Laborie09}~\cite{Laborie09}, \href{works/MonetteDH09.pdf}{MonetteDH09}~\cite{MonetteDH09}, \href{works/KeriK07.pdf}{KeriK07}~\cite{KeriK07}, \href{works/DannaP03.pdf}{DannaP03}~\cite{DannaP03}, \href{works/BeckR03.pdf}{BeckR03}~\cite{BeckR03} & \href{works/FarsiTM22.pdf}{FarsiTM22}~\cite{FarsiTM22}, \href{works/MengZRZL20.pdf}{MengZRZL20}~\cite{MengZRZL20}, \href{works/KovacsB11.pdf}{KovacsB11}~\cite{KovacsB11}, \href{works/Davenport10.pdf}{Davenport10}~\cite{Davenport10} & \href{works/abs-2402-00459.pdf}{abs-2402-00459}~\cite{abs-2402-00459}, \href{works/NaderiRR23.pdf}{NaderiRR23}~\cite{NaderiRR23}, \href{works/AbreuNP23.pdf}{AbreuNP23}~\cite{AbreuNP23}, \href{works/IsikYA23.pdf}{IsikYA23}~\cite{IsikYA23}, \href{works/AlfieriGPS23.pdf}{AlfieriGPS23}~\cite{AlfieriGPS23}, \href{works/LacknerMMWW23.pdf}{LacknerMMWW23}~\cite{LacknerMMWW23}, \href{works/YunusogluY22.pdf}{YunusogluY22}~\cite{YunusogluY22}, \href{works/FanXG21.pdf}{FanXG21}~\cite{FanXG21}, \href{works/LacknerMMWW21.pdf}{LacknerMMWW21}~\cite{LacknerMMWW21}, \href{works/Polo-MejiaALB20.pdf}{Polo-MejiaALB20}~\cite{Polo-MejiaALB20}, \href{works/Mercier-AubinGQ20.pdf}{Mercier-AubinGQ20}~\cite{Mercier-AubinGQ20}, \href{works/ColT19.pdf}{ColT19}~\cite{ColT19}, \href{works/GokgurHO18.pdf}{GokgurHO18}~\cite{GokgurHO18}, \href{works/ZhangW18.pdf}{ZhangW18}~\cite{ZhangW18}, \href{works/NovaraNH16.pdf}{NovaraNH16}~\cite{NovaraNH16}, \href{works/VilimLS15.pdf}{VilimLS15}~\cite{VilimLS15}, \href{works/LimBTBB15.pdf}{LimBTBB15}~\cite{LimBTBB15}, \href{works/SialaAH15.pdf}{SialaAH15}~\cite{SialaAH15}, \href{works/BajestaniB13.pdf}{BajestaniB13}~\cite{BajestaniB13}, \href{works/HeinzB12.pdf}{HeinzB12}~\cite{HeinzB12}, \href{works/EdisO11.pdf}{EdisO11}~\cite{EdisO11}, \href{works/KovacsK11.pdf}{KovacsK11}~\cite{KovacsK11}, \href{works/ZeballosQH10.pdf}{ZeballosQH10}~\cite{ZeballosQH10}, \href{works/NovasH10.pdf}{NovasH10}~\cite{NovasH10}, \href{works/KovacsB07.pdf}{KovacsB07}~\cite{KovacsB07}, \href{works/KovacsV06.pdf}{KovacsV06}~\cite{KovacsV06}, \href{works/GodardLN05.pdf}{GodardLN05}~\cite{GodardLN05}, \href{works/QuirogaZH05.pdf}{QuirogaZH05}~\cite{QuirogaZH05}, \href{works/Bartak02a.pdf}{Bartak02a}~\cite{Bartak02a}... (Total: 32)\\
Concepts & flow-shop & \href{works/PrataAN23.pdf}{PrataAN23}~\cite{PrataAN23}, \href{works/CzerniachowskaWZ23.pdf}{CzerniachowskaWZ23}~\cite{CzerniachowskaWZ23}, \href{works/NaderiRR23.pdf}{NaderiRR23}~\cite{NaderiRR23}, \href{works/AlfieriGPS23.pdf}{AlfieriGPS23}~\cite{AlfieriGPS23}, \href{works/IsikYA23.pdf}{IsikYA23}~\cite{IsikYA23}, \href{works/JuvinHL23.pdf}{JuvinHL23}~\cite{JuvinHL23}, \href{works/AbreuNP23.pdf}{AbreuNP23}~\cite{AbreuNP23}, \href{works/ArmstrongGOS22.pdf}{ArmstrongGOS22}~\cite{ArmstrongGOS22}, \href{works/OujanaAYB22.pdf}{OujanaAYB22}~\cite{OujanaAYB22}, \href{works/ColT22.pdf}{ColT22}~\cite{ColT22}, \href{works/ZhangJZL22.pdf}{ZhangJZL22}~\cite{ZhangJZL22}, \href{works/AbreuN22.pdf}{AbreuN22}~\cite{AbreuN22}, \href{works/LiFJZLL22.pdf}{LiFJZLL22}~\cite{LiFJZLL22}, \href{works/QinWSLS21.pdf}{QinWSLS21}~\cite{QinWSLS21}, \href{works/ArmstrongGOS21.pdf}{ArmstrongGOS21}~\cite{ArmstrongGOS21}, \href{works/Bedhief21.pdf}{Bedhief21}~\cite{Bedhief21}, \href{works/AbreuAPNM21.pdf}{AbreuAPNM21}~\cite{AbreuAPNM21}, \href{works/MengZRZL20.pdf}{MengZRZL20}~\cite{MengZRZL20}, \href{works/AstrandJZ20.pdf}{AstrandJZ20}~\cite{AstrandJZ20}, \href{works/Novas19.pdf}{Novas19}~\cite{Novas19}, \href{works/ParkUJR19.pdf}{ParkUJR19}~\cite{ParkUJR19}, \href{works/ZhangW18.pdf}{ZhangW18}~\cite{ZhangW18}, \href{works/ZhouGL15.pdf}{ZhouGL15}~\cite{ZhouGL15}, \href{works/BajestaniB15.pdf}{BajestaniB15}~\cite{BajestaniB15}, \href{works/TerekhovTDB14.pdf}{TerekhovTDB14}~\cite{TerekhovTDB14}, \href{works/LorigeonBB02.pdf}{LorigeonBB02}~\cite{LorigeonBB02}, \href{works/SourdN00.pdf}{SourdN00}~\cite{SourdN00} & \href{works/Mehdizadeh-Somarin23.pdf}{Mehdizadeh-Somarin23}~\cite{Mehdizadeh-Somarin23}, \href{works/YuraszeckMPV22.pdf}{YuraszeckMPV22}~\cite{YuraszeckMPV22}, \href{works/KoehlerBFFHPSSS21.pdf}{KoehlerBFFHPSSS21}~\cite{KoehlerBFFHPSSS21}, \href{works/FanXG21.pdf}{FanXG21}~\cite{FanXG21}, \href{works/TangB20.pdf}{TangB20}~\cite{TangB20}, \href{works/abs-1902-09244.pdf}{abs-1902-09244}~\cite{abs-1902-09244}, \href{works/LaborieRSV18.pdf}{LaborieRSV18}~\cite{LaborieRSV18}, \href{works/GrimesH11.pdf}{GrimesH11}~\cite{GrimesH11}, \href{works/KovacsB11.pdf}{KovacsB11}~\cite{KovacsB11}, \href{works/BartakSR10.pdf}{BartakSR10}~\cite{BartakSR10}, \href{works/AggounB93.pdf}{AggounB93}~\cite{AggounB93} & \href{works/TasselGS23.pdf}{TasselGS23}~\cite{TasselGS23}, \href{works/AalianPG23.pdf}{AalianPG23}~\cite{AalianPG23}, \href{works/YuraszeckMCCR23.pdf}{YuraszeckMCCR23}~\cite{YuraszeckMCCR23}, \href{works/abs-2305-19888.pdf}{abs-2305-19888}~\cite{abs-2305-19888}, \href{works/JuvinHHL23.pdf}{JuvinHHL23}~\cite{JuvinHHL23}, \href{works/abs-2306-05747.pdf}{abs-2306-05747}~\cite{abs-2306-05747}, \href{works/abs-2211-14492.pdf}{abs-2211-14492}~\cite{abs-2211-14492}, \href{works/TouatBT22.pdf}{TouatBT22}~\cite{TouatBT22}, \href{works/HeinzNVH22.pdf}{HeinzNVH22}~\cite{HeinzNVH22}, \href{works/Teppan22.pdf}{Teppan22}~\cite{Teppan22}, \href{works/LacknerMMWW21.pdf}{LacknerMMWW21}~\cite{LacknerMMWW21}, \href{works/HillTV21.pdf}{HillTV21}~\cite{HillTV21}, \href{works/abs-2102-08778.pdf}{abs-2102-08778}~\cite{abs-2102-08778}, \href{works/KovacsTKSG21.pdf}{KovacsTKSG21}~\cite{KovacsTKSG21}, \href{works/PandeyS21a.pdf}{PandeyS21a}~\cite{PandeyS21a}, \href{works/HamPK21.pdf}{HamPK21}~\cite{HamPK21}, \href{works/WallaceY20.pdf}{WallaceY20}~\cite{WallaceY20}, \href{works/SacramentoSP20.pdf}{SacramentoSP20}~\cite{SacramentoSP20}, \href{works/LunardiBLRV20.pdf}{LunardiBLRV20}~\cite{LunardiBLRV20}, \href{works/WikarekS19.pdf}{WikarekS19}~\cite{WikarekS19}, \href{works/RiahiNS018.pdf}{RiahiNS018}~\cite{RiahiNS018}, \href{works/GokgurHO18.pdf}{GokgurHO18}~\cite{GokgurHO18}, \href{works/GoldwaserS18.pdf}{GoldwaserS18}~\cite{GoldwaserS18}, \href{works/ZarandiKS16.pdf}{ZarandiKS16}~\cite{ZarandiKS16}, \href{works/OzturkTHO13.pdf}{OzturkTHO13}~\cite{OzturkTHO13}, \href{works/TranTDB13.pdf}{TranTDB13}~\cite{TranTDB13}, \href{works/LombardiM12.pdf}{LombardiM12}~\cite{LombardiM12}, \href{works/BillautHL12.pdf}{BillautHL12}~\cite{BillautHL12}, \href{works/KovacsK11.pdf}{KovacsK11}~\cite{KovacsK11}... (Total: 47)\\
Concepts & flow-time & \href{works/YuraszeckMPV22.pdf}{YuraszeckMPV22}~\cite{YuraszeckMPV22}, \href{works/FanXG21.pdf}{FanXG21}~\cite{FanXG21}, \href{works/NattafM20.pdf}{NattafM20}~\cite{NattafM20}, \href{works/MalapertN19.pdf}{MalapertN19}~\cite{MalapertN19}, \href{works/ZhangW18.pdf}{ZhangW18}~\cite{ZhangW18}, \href{works/TerekhovTDB14.pdf}{TerekhovTDB14}~\cite{TerekhovTDB14}, \href{works/TranTDB13.pdf}{TranTDB13}~\cite{TranTDB13} & \href{works/PrataAN23.pdf}{PrataAN23}~\cite{PrataAN23}, \href{works/AlfieriGPS23.pdf}{AlfieriGPS23}~\cite{AlfieriGPS23}, \href{works/YunusogluY22.pdf}{YunusogluY22}~\cite{YunusogluY22}, \href{works/BeckW07.pdf}{BeckW07}~\cite{BeckW07} & \href{works/TasselGS23.pdf}{TasselGS23}~\cite{TasselGS23}, \href{works/abs-2306-05747.pdf}{abs-2306-05747}~\cite{abs-2306-05747}, \href{works/YuraszeckMC23.pdf}{YuraszeckMC23}~\cite{YuraszeckMC23}, \href{works/YuraszeckMCCR23.pdf}{YuraszeckMCCR23}~\cite{YuraszeckMCCR23}, \href{works/LiFJZLL22.pdf}{LiFJZLL22}~\cite{LiFJZLL22}, \href{works/AbreuN22.pdf}{AbreuN22}~\cite{AbreuN22}, \href{works/KoehlerBFFHPSSS21.pdf}{KoehlerBFFHPSSS21}~\cite{KoehlerBFFHPSSS21}, \href{works/MengZRZL20.pdf}{MengZRZL20}~\cite{MengZRZL20}, \href{works/ParkUJR19.pdf}{ParkUJR19}~\cite{ParkUJR19}, \href{works/Novas19.pdf}{Novas19}~\cite{Novas19}, \href{works/BajestaniB15.pdf}{BajestaniB15}~\cite{BajestaniB15}, \href{works/KovacsB11.pdf}{KovacsB11}~\cite{KovacsB11}, \href{works/EdisO11.pdf}{EdisO11}~\cite{EdisO11}, \href{works/QuirogaZH05.pdf}{QuirogaZH05}~\cite{QuirogaZH05}, \href{works/BeckR03.pdf}{BeckR03}~\cite{BeckR03}\\
Concepts & inventory & \href{works/SubulanC22.pdf}{SubulanC22}~\cite{SubulanC22}, \href{works/GilesH16.pdf}{GilesH16}~\cite{GilesH16}, \href{works/GoelSHFS15.pdf}{GoelSHFS15}~\cite{GoelSHFS15}, \href{works/SerraNM12.pdf}{SerraNM12}~\cite{SerraNM12}, \href{works/LopesCSM10.pdf}{LopesCSM10}~\cite{LopesCSM10}, \href{works/RossiTHP07.pdf}{RossiTHP07}~\cite{RossiTHP07}, \href{works/Timpe02.pdf}{Timpe02}~\cite{Timpe02}, \href{works/BeckDF97.pdf}{BeckDF97}~\cite{BeckDF97} & \href{works/Novas19.pdf}{Novas19}~\cite{Novas19}, \href{works/BajestaniB13.pdf}{BajestaniB13}~\cite{BajestaniB13}, \href{works/MakMS10.pdf}{MakMS10}~\cite{MakMS10}, \href{works/LauLN08.pdf}{LauLN08}~\cite{LauLN08}, \href{works/MouraSCL08a.pdf}{MouraSCL08a}~\cite{MouraSCL08a}, \href{works/DavenportKRSH07.pdf}{DavenportKRSH07}~\cite{DavenportKRSH07}, \href{works/GarganiR07.pdf}{GarganiR07}~\cite{GarganiR07}, \href{works/BeckF00.pdf}{BeckF00}~\cite{BeckF00} & \href{works/PrataAN23.pdf}{PrataAN23}~\cite{PrataAN23}, \href{works/PerezGSL23.pdf}{PerezGSL23}~\cite{PerezGSL23}, \href{works/abs-2312-13682.pdf}{abs-2312-13682}~\cite{abs-2312-13682}, \href{works/AlfieriGPS23.pdf}{AlfieriGPS23}~\cite{AlfieriGPS23}, \href{works/GurPAE23.pdf}{GurPAE23}~\cite{GurPAE23}, \href{works/AbreuN22.pdf}{AbreuN22}~\cite{AbreuN22}, \href{works/PohlAK22.pdf}{PohlAK22}~\cite{PohlAK22}, \href{works/YunusogluY22.pdf}{YunusogluY22}~\cite{YunusogluY22}, \href{works/HubnerGSV21.pdf}{HubnerGSV21}~\cite{HubnerGSV21}, \href{works/KovacsTKSG21.pdf}{KovacsTKSG21}~\cite{KovacsTKSG21}, \href{works/GroleazNS20a.pdf}{GroleazNS20a}~\cite{GroleazNS20a}, \href{works/GroleazNS20.pdf}{GroleazNS20}~\cite{GroleazNS20}, \href{works/abs-1902-09244.pdf}{abs-1902-09244}~\cite{abs-1902-09244}, \href{works/YounespourAKE19.pdf}{YounespourAKE19}~\cite{YounespourAKE19}, \href{works/WikarekS19.pdf}{WikarekS19}~\cite{WikarekS19}, \href{works/Ham18.pdf}{Ham18}~\cite{Ham18}, \href{works/LaborieRSV18.pdf}{LaborieRSV18}~\cite{LaborieRSV18}, \href{works/ShinBBHO18.pdf}{ShinBBHO18}~\cite{ShinBBHO18}, \href{works/SchuttS16.pdf}{SchuttS16}~\cite{SchuttS16}, \href{works/SimoninAHL15.pdf}{SimoninAHL15}~\cite{SimoninAHL15}, \href{works/HoundjiSWD14.pdf}{HoundjiSWD14}~\cite{HoundjiSWD14}, \href{works/TerekhovTDB14.pdf}{TerekhovTDB14}~\cite{TerekhovTDB14}, \href{works/KelarevaTK13.pdf}{KelarevaTK13}~\cite{KelarevaTK13}, \href{works/HeinzSSW12.pdf}{HeinzSSW12}~\cite{HeinzSSW12}, \href{works/LombardiM12.pdf}{LombardiM12}~\cite{LombardiM12}, \href{works/KelbelH11.pdf}{KelbelH11}~\cite{KelbelH11}, \href{works/BajestaniB11.pdf}{BajestaniB11}~\cite{BajestaniB11}, \href{works/Laborie09.pdf}{Laborie09}~\cite{Laborie09}, \href{works/BidotVLB09.pdf}{BidotVLB09}~\cite{BidotVLB09}... (Total: 35)\\
Concepts & job & \href{works/PrataAN23.pdf}{PrataAN23}~\cite{PrataAN23}, \href{works/abs-2402-00459.pdf}{abs-2402-00459}~\cite{abs-2402-00459}, \href{works/KimCMLLP23.pdf}{KimCMLLP23}~\cite{KimCMLLP23}, \href{works/JuvinHHL23.pdf}{JuvinHHL23}~\cite{JuvinHHL23}, \href{works/AlfieriGPS23.pdf}{AlfieriGPS23}~\cite{AlfieriGPS23}, \href{works/YuraszeckMC23.pdf}{YuraszeckMC23}~\cite{YuraszeckMC23}, \href{works/AbreuNP23.pdf}{AbreuNP23}~\cite{AbreuNP23}, \href{works/IsikYA23.pdf}{IsikYA23}~\cite{IsikYA23}, \href{works/WangB23.pdf}{WangB23}~\cite{WangB23}, \href{works/LacknerMMWW23.pdf}{LacknerMMWW23}~\cite{LacknerMMWW23}, \href{works/Bit-Monnot23.pdf}{Bit-Monnot23}~\cite{Bit-Monnot23}, \href{works/CzerniachowskaWZ23.pdf}{CzerniachowskaWZ23}~\cite{CzerniachowskaWZ23}, \href{works/abs-2306-05747.pdf}{abs-2306-05747}~\cite{abs-2306-05747}, \href{works/NaderiRR23.pdf}{NaderiRR23}~\cite{NaderiRR23}, \href{works/JuvinHL23.pdf}{JuvinHL23}~\cite{JuvinHL23}, \href{works/TasselGS23.pdf}{TasselGS23}~\cite{TasselGS23}, \href{works/Mehdizadeh-Somarin23.pdf}{Mehdizadeh-Somarin23}~\cite{Mehdizadeh-Somarin23}, \href{works/YuraszeckMCCR23.pdf}{YuraszeckMCCR23}~\cite{YuraszeckMCCR23}, \href{works/LiFJZLL22.pdf}{LiFJZLL22}~\cite{LiFJZLL22}, \href{works/TouatBT22.pdf}{TouatBT22}~\cite{TouatBT22}, \href{works/YunusogluY22.pdf}{YunusogluY22}~\cite{YunusogluY22}, \href{works/GeitzGSSW22.pdf}{GeitzGSSW22}~\cite{GeitzGSSW22}, \href{works/MullerMKP22.pdf}{MullerMKP22}~\cite{MullerMKP22}, \href{works/WinterMMW22.pdf}{WinterMMW22}~\cite{WinterMMW22}, \href{works/ArmstrongGOS22.pdf}{ArmstrongGOS22}~\cite{ArmstrongGOS22}, \href{works/OujanaAYB22.pdf}{OujanaAYB22}~\cite{OujanaAYB22}, \href{works/AbreuN22.pdf}{AbreuN22}~\cite{AbreuN22}, \href{works/ZhangJZL22.pdf}{ZhangJZL22}~\cite{ZhangJZL22}, \href{works/abs-2211-14492.pdf}{abs-2211-14492}~\cite{abs-2211-14492}... (Total: 192) & \href{works/EfthymiouY23.pdf}{EfthymiouY23}~\cite{EfthymiouY23}, \href{works/ShaikhK23.pdf}{ShaikhK23}~\cite{ShaikhK23}, \href{works/abs-2305-19888.pdf}{abs-2305-19888}~\cite{abs-2305-19888}, \href{works/HeinzNVH22.pdf}{HeinzNVH22}~\cite{HeinzNVH22}, \href{works/BourreauGGLT22.pdf}{BourreauGGLT22}~\cite{BourreauGGLT22}, \href{works/LuoB22.pdf}{LuoB22}~\cite{LuoB22}, \href{works/HanenKP21.pdf}{HanenKP21}~\cite{HanenKP21}, \href{works/Mercier-AubinGQ20.pdf}{Mercier-AubinGQ20}~\cite{Mercier-AubinGQ20}, \href{works/MokhtarzadehTNF20.pdf}{MokhtarzadehTNF20}~\cite{MokhtarzadehTNF20}, \href{works/Tom19.pdf}{Tom19}~\cite{Tom19}, \href{works/EscobetPQPRA19.pdf}{EscobetPQPRA19}~\cite{EscobetPQPRA19}, \href{works/GurEA19.pdf}{GurEA19}~\cite{GurEA19}, \href{works/PourDERB18.pdf}{PourDERB18}~\cite{PourDERB18}, \href{works/CappartS17.pdf}{CappartS17}~\cite{CappartS17}, \href{works/NattafAL17.pdf}{NattafAL17}~\cite{NattafAL17}, \href{works/ZarandiKS16.pdf}{ZarandiKS16}~\cite{ZarandiKS16}, \href{works/Madi-WambaB16.pdf}{Madi-WambaB16}~\cite{Madi-WambaB16}, \href{works/TranWDRFOVB16.pdf}{TranWDRFOVB16}~\cite{TranWDRFOVB16}, \href{works/LetortCB15.pdf}{LetortCB15}~\cite{LetortCB15}, \href{works/ZhouGL15.pdf}{ZhouGL15}~\cite{ZhouGL15}, \href{works/PraletLJ15.pdf}{PraletLJ15}~\cite{PraletLJ15}, \href{works/BonfiettiLBM14.pdf}{BonfiettiLBM14}~\cite{BonfiettiLBM14}, \href{works/BonfiettiLM14.pdf}{BonfiettiLM14}~\cite{BonfiettiLM14}, \href{works/ThiruvadyWGS14.pdf}{ThiruvadyWGS14}~\cite{ThiruvadyWGS14}, \href{works/LombardiM12.pdf}{LombardiM12}~\cite{LombardiM12}, \href{works/KovacsK11.pdf}{KovacsK11}~\cite{KovacsK11}, \href{works/Rodriguez07.pdf}{Rodriguez07}~\cite{Rodriguez07}, \href{works/Simonis07.pdf}{Simonis07}~\cite{Simonis07}, \href{works/KovacsV06.pdf}{KovacsV06}~\cite{KovacsV06}... (Total: 41) & \href{works/PovedaAA23.pdf}{PovedaAA23}~\cite{PovedaAA23}, \href{works/CampeauG22.pdf}{CampeauG22}~\cite{CampeauG22}, \href{works/PohlAK22.pdf}{PohlAK22}~\cite{PohlAK22}, \href{works/KlankeBYE21.pdf}{KlankeBYE21}~\cite{KlankeBYE21}, \href{works/HubnerGSV21.pdf}{HubnerGSV21}~\cite{HubnerGSV21}, \href{works/AntuoriHHEN21.pdf}{AntuoriHHEN21}~\cite{AntuoriHHEN21}, \href{works/BenderWS21.pdf}{BenderWS21}~\cite{BenderWS21}, \href{works/WessenCS20.pdf}{WessenCS20}~\cite{WessenCS20}, \href{works/AntuoriHHEN20.pdf}{AntuoriHHEN20}~\cite{AntuoriHHEN20}, \href{works/QinDCS20.pdf}{QinDCS20}~\cite{QinDCS20}, \href{works/Polo-MejiaALB20.pdf}{Polo-MejiaALB20}~\cite{Polo-MejiaALB20}, \href{works/FrimodigS19.pdf}{FrimodigS19}~\cite{FrimodigS19}, \href{works/TangLWSK18.pdf}{TangLWSK18}~\cite{TangLWSK18}, \href{works/HoYCLLCLC18.pdf}{HoYCLLCLC18}~\cite{HoYCLLCLC18}, \href{works/BaptisteB18.pdf}{BaptisteB18}~\cite{BaptisteB18}, \href{works/ShinBBHO18.pdf}{ShinBBHO18}~\cite{ShinBBHO18}, \href{works/TranVNB17.pdf}{TranVNB17}~\cite{TranVNB17}, \href{works/HechingH16.pdf}{HechingH16}~\cite{HechingH16}, \href{works/NovaraNH16.pdf}{NovaraNH16}~\cite{NovaraNH16}, \href{works/BurtLPS15.pdf}{BurtLPS15}~\cite{BurtLPS15}, \href{works/WangMD15.pdf}{WangMD15}~\cite{WangMD15}, \href{works/LimBTBB15.pdf}{LimBTBB15}~\cite{LimBTBB15}, \href{works/BartakV15.pdf}{BartakV15}~\cite{BartakV15}, \href{works/LombardiBM15.pdf}{LombardiBM15}~\cite{LombardiBM15}, \href{works/MelgarejoLS15.pdf}{MelgarejoLS15}~\cite{MelgarejoLS15}, \href{works/BessiereHMQW14.pdf}{BessiereHMQW14}~\cite{BessiereHMQW14}, \href{works/DerrienPZ14.pdf}{DerrienPZ14}~\cite{DerrienPZ14}, \href{works/KameugneFSN14.pdf}{KameugneFSN14}~\cite{KameugneFSN14}, \href{works/AlesioNBG14.pdf}{AlesioNBG14}~\cite{AlesioNBG14}... (Total: 70)\\
Concepts & job-shop & \href{works/abs-2402-00459.pdf}{abs-2402-00459}~\cite{abs-2402-00459}, \href{works/PrataAN23.pdf}{PrataAN23}~\cite{PrataAN23}, \href{works/abs-2306-05747.pdf}{abs-2306-05747}~\cite{abs-2306-05747}, \href{works/Mehdizadeh-Somarin23.pdf}{Mehdizadeh-Somarin23}~\cite{Mehdizadeh-Somarin23}, \href{works/KimCMLLP23.pdf}{KimCMLLP23}~\cite{KimCMLLP23}, \href{works/CzerniachowskaWZ23.pdf}{CzerniachowskaWZ23}~\cite{CzerniachowskaWZ23}, \href{works/JuvinHHL23.pdf}{JuvinHHL23}~\cite{JuvinHHL23}, \href{works/Bit-Monnot23.pdf}{Bit-Monnot23}~\cite{Bit-Monnot23}, \href{works/NaderiRR23.pdf}{NaderiRR23}~\cite{NaderiRR23}, \href{works/AbreuNP23.pdf}{AbreuNP23}~\cite{AbreuNP23}, \href{works/YuraszeckMCCR23.pdf}{YuraszeckMCCR23}~\cite{YuraszeckMCCR23}, \href{works/TasselGS23.pdf}{TasselGS23}~\cite{TasselGS23}, \href{works/MullerMKP22.pdf}{MullerMKP22}~\cite{MullerMKP22}, \href{works/Teppan22.pdf}{Teppan22}~\cite{Teppan22}, \href{works/OujanaAYB22.pdf}{OujanaAYB22}~\cite{OujanaAYB22}, \href{works/abs-2211-14492.pdf}{abs-2211-14492}~\cite{abs-2211-14492}, \href{works/YuraszeckMPV22.pdf}{YuraszeckMPV22}~\cite{YuraszeckMPV22}, \href{works/LiFJZLL22.pdf}{LiFJZLL22}~\cite{LiFJZLL22}, \href{works/GeitzGSSW22.pdf}{GeitzGSSW22}~\cite{GeitzGSSW22}, \href{works/ColT22.pdf}{ColT22}~\cite{ColT22}, \href{works/ZhangBB22.pdf}{ZhangBB22}~\cite{ZhangBB22}, \href{works/HamPK21.pdf}{HamPK21}~\cite{HamPK21}, \href{works/KovacsTKSG21.pdf}{KovacsTKSG21}~\cite{KovacsTKSG21}, \href{works/abs-2102-08778.pdf}{abs-2102-08778}~\cite{abs-2102-08778}, \href{works/AbreuAPNM21.pdf}{AbreuAPNM21}~\cite{AbreuAPNM21}, \href{works/FanXG21.pdf}{FanXG21}~\cite{FanXG21}, \href{works/ZhangYW21.pdf}{ZhangYW21}~\cite{ZhangYW21}, \href{works/MengZRZL20.pdf}{MengZRZL20}~\cite{MengZRZL20}, \href{works/LunardiBLRV20.pdf}{LunardiBLRV20}~\cite{LunardiBLRV20}... (Total: 87) & \href{works/IsikYA23.pdf}{IsikYA23}~\cite{IsikYA23}, \href{works/EfthymiouY23.pdf}{EfthymiouY23}~\cite{EfthymiouY23}, \href{works/AlfieriGPS23.pdf}{AlfieriGPS23}~\cite{AlfieriGPS23}, \href{works/TouatBT22.pdf}{TouatBT22}~\cite{TouatBT22}, \href{works/YunusogluY22.pdf}{YunusogluY22}~\cite{YunusogluY22}, \href{works/AbreuN22.pdf}{AbreuN22}~\cite{AbreuN22}, \href{works/LuoB22.pdf}{LuoB22}~\cite{LuoB22}, \href{works/QinWSLS21.pdf}{QinWSLS21}~\cite{QinWSLS21}, \href{works/ArmstrongGOS21.pdf}{ArmstrongGOS21}~\cite{ArmstrongGOS21}, \href{works/Astrand0F21.pdf}{Astrand0F21}~\cite{Astrand0F21}, \href{works/KoehlerBFFHPSSS21.pdf}{KoehlerBFFHPSSS21}~\cite{KoehlerBFFHPSSS21}, \href{works/GroleazNS20.pdf}{GroleazNS20}~\cite{GroleazNS20}, \href{works/MejiaY20.pdf}{MejiaY20}~\cite{MejiaY20}, \href{works/SacramentoSP20.pdf}{SacramentoSP20}~\cite{SacramentoSP20}, \href{works/EscobetPQPRA19.pdf}{EscobetPQPRA19}~\cite{EscobetPQPRA19}, \href{works/WikarekS19.pdf}{WikarekS19}~\cite{WikarekS19}, \href{works/GokgurHO18.pdf}{GokgurHO18}~\cite{GokgurHO18}, \href{works/MossigeGSMC17.pdf}{MossigeGSMC17}~\cite{MossigeGSMC17}, \href{works/CappartS17.pdf}{CappartS17}~\cite{CappartS17}, \href{works/BonfiettiLM14.pdf}{BonfiettiLM14}~\cite{BonfiettiLM14}, \href{works/GaySS14.pdf}{GaySS14}~\cite{GaySS14}, \href{works/BonfiettiLBM14.pdf}{BonfiettiLBM14}~\cite{BonfiettiLBM14}, \href{works/BajestaniB13.pdf}{BajestaniB13}~\cite{BajestaniB13}, \href{works/LombardiM12.pdf}{LombardiM12}~\cite{LombardiM12}, \href{works/AronssonBK09.pdf}{AronssonBK09}~\cite{AronssonBK09}, \href{works/LauLN08.pdf}{LauLN08}~\cite{LauLN08}, \href{works/KovacsV06.pdf}{KovacsV06}~\cite{KovacsV06}, \href{works/VilimBC05.pdf}{VilimBC05}~\cite{VilimBC05}, \href{works/HebrardTW05.pdf}{HebrardTW05}~\cite{HebrardTW05}... (Total: 38) & \href{works/ShaikhK23.pdf}{ShaikhK23}~\cite{ShaikhK23}, \href{works/YuraszeckMC23.pdf}{YuraszeckMC23}~\cite{YuraszeckMC23}, \href{works/PovedaAA23.pdf}{PovedaAA23}~\cite{PovedaAA23}, \href{works/LacknerMMWW23.pdf}{LacknerMMWW23}~\cite{LacknerMMWW23}, \href{works/JuvinHL23.pdf}{JuvinHL23}~\cite{JuvinHL23}, \href{works/HanenKP21.pdf}{HanenKP21}~\cite{HanenKP21}, \href{works/KlankeBYE21.pdf}{KlankeBYE21}~\cite{KlankeBYE21}, \href{works/AntuoriHHEN21.pdf}{AntuoriHHEN21}~\cite{AntuoriHHEN21}, \href{works/BenediktMH20.pdf}{BenediktMH20}~\cite{BenediktMH20}, \href{works/WessenCS20.pdf}{WessenCS20}~\cite{WessenCS20}, \href{works/AntuoriHHEN20.pdf}{AntuoriHHEN20}~\cite{AntuoriHHEN20}, \href{works/Mercier-AubinGQ20.pdf}{Mercier-AubinGQ20}~\cite{Mercier-AubinGQ20}, \href{works/WallaceY20.pdf}{WallaceY20}~\cite{WallaceY20}, \href{works/Tom19.pdf}{Tom19}~\cite{Tom19}, \href{works/GurEA19.pdf}{GurEA19}~\cite{GurEA19}, \href{works/FrimodigS19.pdf}{FrimodigS19}~\cite{FrimodigS19}, \href{works/BogaerdtW19.pdf}{BogaerdtW19}~\cite{BogaerdtW19}, \href{works/abs-1902-09244.pdf}{abs-1902-09244}~\cite{abs-1902-09244}, \href{works/ParkUJR19.pdf}{ParkUJR19}~\cite{ParkUJR19}, \href{works/BenediktSMVH18.pdf}{BenediktSMVH18}~\cite{BenediktSMVH18}, \href{works/Ham18.pdf}{Ham18}~\cite{Ham18}, \href{works/ZarandiKS16.pdf}{ZarandiKS16}~\cite{ZarandiKS16}, \href{works/LuoVLBM16.pdf}{LuoVLBM16}~\cite{LuoVLBM16}, \href{works/TranDRFWOVB16.pdf}{TranDRFWOVB16}~\cite{TranDRFWOVB16}, \href{works/TranWDRFOVB16.pdf}{TranWDRFOVB16}~\cite{TranWDRFOVB16}, \href{works/PraletLJ15.pdf}{PraletLJ15}~\cite{PraletLJ15}, \href{works/LimBTBB15.pdf}{LimBTBB15}~\cite{LimBTBB15}, \href{works/BartakV15.pdf}{BartakV15}~\cite{BartakV15}, \href{works/LombardiBM15.pdf}{LombardiBM15}~\cite{LombardiBM15}... (Total: 79)\\
Concepts & lateness & \href{works/FahimiOQ18.pdf}{FahimiOQ18}~\cite{FahimiOQ18}, \href{works/KoschB14.pdf}{KoschB14}~\cite{KoschB14}, \href{works/BartakSR10.pdf}{BartakSR10}~\cite{BartakSR10}, \href{works/Geske05.pdf}{Geske05}~\cite{Geske05}, \href{works/ArtiguesR00.pdf}{ArtiguesR00}~\cite{ArtiguesR00} & \href{works/PrataAN23.pdf}{PrataAN23}~\cite{PrataAN23}, \href{works/PohlAK22.pdf}{PohlAK22}~\cite{PohlAK22}, \href{works/ZhangW18.pdf}{ZhangW18}~\cite{ZhangW18}, \href{works/AkkerDH07.pdf}{AkkerDH07}~\cite{AkkerDH07}, \href{works/Sadykov04.pdf}{Sadykov04}~\cite{Sadykov04} & \href{works/LacknerMMWW23.pdf}{LacknerMMWW23}~\cite{LacknerMMWW23}, \href{works/YunusogluY22.pdf}{YunusogluY22}~\cite{YunusogluY22}, \href{works/GeitzGSSW22.pdf}{GeitzGSSW22}~\cite{GeitzGSSW22}, \href{works/ColT22.pdf}{ColT22}~\cite{ColT22}, \href{works/ZhangBB22.pdf}{ZhangBB22}~\cite{ZhangBB22}, \href{works/KoehlerBFFHPSSS21.pdf}{KoehlerBFFHPSSS21}~\cite{KoehlerBFFHPSSS21}, \href{works/HanenKP21.pdf}{HanenKP21}~\cite{HanenKP21}, \href{works/QinWSLS21.pdf}{QinWSLS21}~\cite{QinWSLS21}, \href{works/LacknerMMWW21.pdf}{LacknerMMWW21}~\cite{LacknerMMWW21}, \href{works/Novas19.pdf}{Novas19}~\cite{Novas19}, \href{works/ParkUJR19.pdf}{ParkUJR19}~\cite{ParkUJR19}, \href{works/Tesch18.pdf}{Tesch18}~\cite{Tesch18}, \href{works/BartakV15.pdf}{BartakV15}~\cite{BartakV15}, \href{works/EdisO11.pdf}{EdisO11}~\cite{EdisO11}, \href{works/NovasH10.pdf}{NovasH10}~\cite{NovasH10}, \href{works/SadykovW06.pdf}{SadykovW06}~\cite{SadykovW06}, \href{works/Bartak02.pdf}{Bartak02}~\cite{Bartak02}\\
Concepts & lazy clause generation & \href{works/KreterSS17.pdf}{KreterSS17}~\cite{KreterSS17}, \href{works/KreterSS15.pdf}{KreterSS15}~\cite{KreterSS15}, \href{works/SchuttFS13.pdf}{SchuttFS13}~\cite{SchuttFS13}, \href{works/SchuttFSW13.pdf}{SchuttFSW13}~\cite{SchuttFSW13}, \href{works/KelarevaTK13.pdf}{KelarevaTK13}~\cite{KelarevaTK13}, \href{works/SchuttFS13a.pdf}{SchuttFS13a}~\cite{SchuttFS13a}, \href{works/SchuttFSW11.pdf}{SchuttFSW11}~\cite{SchuttFSW11}, \href{works/abs-1009-0347.pdf}{abs-1009-0347}~\cite{abs-1009-0347}, \href{works/SchuttFSW09.pdf}{SchuttFSW09}~\cite{SchuttFSW09} & \href{works/PovedaAA23.pdf}{PovedaAA23}~\cite{PovedaAA23}, \href{works/Bit-Monnot23.pdf}{Bit-Monnot23}~\cite{Bit-Monnot23}, \href{works/BoudreaultSLQ22.pdf}{BoudreaultSLQ22}~\cite{BoudreaultSLQ22}, \href{works/GeitzGSSW22.pdf}{GeitzGSSW22}~\cite{GeitzGSSW22}, \href{works/OuelletQ22.pdf}{OuelletQ22}~\cite{OuelletQ22}, \href{works/FahimiOQ18.pdf}{FahimiOQ18}~\cite{FahimiOQ18}, \href{works/SchuttS16.pdf}{SchuttS16}~\cite{SchuttS16}, \href{works/SzerediS16.pdf}{SzerediS16}~\cite{SzerediS16}, \href{works/SialaAH15.pdf}{SialaAH15}~\cite{SialaAH15}, \href{works/BofillEGPSV14.pdf}{BofillEGPSV14}~\cite{BofillEGPSV14}, \href{works/GuSS13.pdf}{GuSS13}~\cite{GuSS13}, \href{works/SchuttCSW12.pdf}{SchuttCSW12}~\cite{SchuttCSW12} & \href{works/WangB23.pdf}{WangB23}~\cite{WangB23}, \href{works/TardivoDFMP23.pdf}{TardivoDFMP23}~\cite{TardivoDFMP23}, \href{works/KameugneFND23.pdf}{KameugneFND23}~\cite{KameugneFND23}, \href{works/FetgoD22.pdf}{FetgoD22}~\cite{FetgoD22}, \href{works/GeibingerMM21.pdf}{GeibingerMM21}~\cite{GeibingerMM21}, \href{works/HillTV21.pdf}{HillTV21}~\cite{HillTV21}, \href{works/GodetLHS20.pdf}{GodetLHS20}~\cite{GodetLHS20}, \href{works/WallaceY20.pdf}{WallaceY20}~\cite{WallaceY20}, \href{works/Mercier-AubinGQ20.pdf}{Mercier-AubinGQ20}~\cite{Mercier-AubinGQ20}, \href{works/YangSS19.pdf}{YangSS19}~\cite{YangSS19}, \href{works/BaptisteB18.pdf}{BaptisteB18}~\cite{BaptisteB18}, \href{works/GoldwaserS18.pdf}{GoldwaserS18}~\cite{GoldwaserS18}, \href{works/YoungFS17.pdf}{YoungFS17}~\cite{YoungFS17}, \href{works/BofillCSV17.pdf}{BofillCSV17}~\cite{BofillCSV17}, \href{works/GoldwaserS17.pdf}{GoldwaserS17}~\cite{GoldwaserS17}, \href{works/PesantRR15.pdf}{PesantRR15}~\cite{PesantRR15}, \href{works/GuSW12.pdf}{GuSW12}~\cite{GuSW12}, \href{works/LombardiM12.pdf}{LombardiM12}~\cite{LombardiM12}, \href{works/GrimesH11.pdf}{GrimesH11}~\cite{GrimesH11}, \href{works/SchuttW10.pdf}{SchuttW10}~\cite{SchuttW10}\\
Concepts & machine & \href{works/abs-2402-00459.pdf}{abs-2402-00459}~\cite{abs-2402-00459}, \href{works/PrataAN23.pdf}{PrataAN23}~\cite{PrataAN23}, \href{works/IsikYA23.pdf}{IsikYA23}~\cite{IsikYA23}, \href{works/CzerniachowskaWZ23.pdf}{CzerniachowskaWZ23}~\cite{CzerniachowskaWZ23}, \href{works/YuraszeckMCCR23.pdf}{YuraszeckMCCR23}~\cite{YuraszeckMCCR23}, \href{works/AbreuNP23.pdf}{AbreuNP23}~\cite{AbreuNP23}, \href{works/NaderiRR23.pdf}{NaderiRR23}~\cite{NaderiRR23}, \href{works/TasselGS23.pdf}{TasselGS23}~\cite{TasselGS23}, \href{works/Mehdizadeh-Somarin23.pdf}{Mehdizadeh-Somarin23}~\cite{Mehdizadeh-Somarin23}, \href{works/AalianPG23.pdf}{AalianPG23}~\cite{AalianPG23}, \href{works/JuvinHL23.pdf}{JuvinHL23}~\cite{JuvinHL23}, \href{works/PerezGSL23.pdf}{PerezGSL23}~\cite{PerezGSL23}, \href{works/JuvinHHL23.pdf}{JuvinHHL23}~\cite{JuvinHHL23}, \href{works/abs-2312-13682.pdf}{abs-2312-13682}~\cite{abs-2312-13682}, \href{works/LacknerMMWW23.pdf}{LacknerMMWW23}~\cite{LacknerMMWW23}, \href{works/EfthymiouY23.pdf}{EfthymiouY23}~\cite{EfthymiouY23}, \href{works/abs-2306-05747.pdf}{abs-2306-05747}~\cite{abs-2306-05747}, \href{works/AlfieriGPS23.pdf}{AlfieriGPS23}~\cite{AlfieriGPS23}, \href{works/YuraszeckMC23.pdf}{YuraszeckMC23}~\cite{YuraszeckMC23}, \href{works/abs-2305-19888.pdf}{abs-2305-19888}~\cite{abs-2305-19888}, \href{works/KimCMLLP23.pdf}{KimCMLLP23}~\cite{KimCMLLP23}, \href{works/LiFJZLL22.pdf}{LiFJZLL22}~\cite{LiFJZLL22}, \href{works/ArmstrongGOS22.pdf}{ArmstrongGOS22}~\cite{ArmstrongGOS22}, \href{works/JungblutK22.pdf}{JungblutK22}~\cite{JungblutK22}, \href{works/abs-2211-14492.pdf}{abs-2211-14492}~\cite{abs-2211-14492}, \href{works/GeitzGSSW22.pdf}{GeitzGSSW22}~\cite{GeitzGSSW22}, \href{works/YuraszeckMPV22.pdf}{YuraszeckMPV22}~\cite{YuraszeckMPV22}, \href{works/ZhangJZL22.pdf}{ZhangJZL22}~\cite{ZhangJZL22}, \href{works/AbreuN22.pdf}{AbreuN22}~\cite{AbreuN22}... (Total: 179) & \href{works/Bit-Monnot23.pdf}{Bit-Monnot23}~\cite{Bit-Monnot23}, \href{works/AkramNHRSA23.pdf}{AkramNHRSA23}~\cite{AkramNHRSA23}, \href{works/GurPAE23.pdf}{GurPAE23}~\cite{GurPAE23}, \href{works/LuoB22.pdf}{LuoB22}~\cite{LuoB22}, \href{works/HillTV21.pdf}{HillTV21}~\cite{HillTV21}, \href{works/KlankeBYE21.pdf}{KlankeBYE21}~\cite{KlankeBYE21}, \href{works/AbohashimaEG21.pdf}{AbohashimaEG21}~\cite{AbohashimaEG21}, \href{works/AntuoriHHEN20.pdf}{AntuoriHHEN20}~\cite{AntuoriHHEN20}, \href{works/Polo-MejiaALB20.pdf}{Polo-MejiaALB20}~\cite{Polo-MejiaALB20}, \href{works/BehrensLM19.pdf}{BehrensLM19}~\cite{BehrensLM19}, \href{works/GoldwaserS18.pdf}{GoldwaserS18}~\cite{GoldwaserS18}, \href{works/BaptisteB18.pdf}{BaptisteB18}~\cite{BaptisteB18}, \href{works/He0GLW18.pdf}{He0GLW18}~\cite{He0GLW18}, \href{works/Ham18.pdf}{Ham18}~\cite{Ham18}, \href{works/ShinBBHO18.pdf}{ShinBBHO18}~\cite{ShinBBHO18}, \href{works/MusliuSS18.pdf}{MusliuSS18}~\cite{MusliuSS18}, \href{works/FahimiOQ18.pdf}{FahimiOQ18}~\cite{FahimiOQ18}, \href{works/GoldwaserS17.pdf}{GoldwaserS17}~\cite{GoldwaserS17}, \href{works/KreterSS17.pdf}{KreterSS17}~\cite{KreterSS17}, \href{works/Pralet17.pdf}{Pralet17}~\cite{Pralet17}, \href{works/CohenHB17.pdf}{CohenHB17}~\cite{CohenHB17}, \href{works/BridiLBBM16.pdf}{BridiLBBM16}~\cite{BridiLBBM16}, \href{works/SchuttS16.pdf}{SchuttS16}~\cite{SchuttS16}, \href{works/CauwelaertDMS16.pdf}{CauwelaertDMS16}~\cite{CauwelaertDMS16}, \href{works/ZarandiKS16.pdf}{ZarandiKS16}~\cite{ZarandiKS16}, \href{works/TranWDRFOVB16.pdf}{TranWDRFOVB16}~\cite{TranWDRFOVB16}, \href{works/SialaAH15.pdf}{SialaAH15}~\cite{SialaAH15}, \href{works/DejemeppeCS15.pdf}{DejemeppeCS15}~\cite{DejemeppeCS15}, \href{works/MurphyMB15.pdf}{MurphyMB15}~\cite{MurphyMB15}... (Total: 54) & \href{works/KameugneFND23.pdf}{KameugneFND23}~\cite{KameugneFND23}, \href{works/MontemanniD23.pdf}{MontemanniD23}~\cite{MontemanniD23}, \href{works/ShaikhK23.pdf}{ShaikhK23}~\cite{ShaikhK23}, \href{works/BoudreaultSLQ22.pdf}{BoudreaultSLQ22}~\cite{BoudreaultSLQ22}, \href{works/PopovicCGNC22.pdf}{PopovicCGNC22}~\cite{PopovicCGNC22}, \href{works/SubulanC22.pdf}{SubulanC22}~\cite{SubulanC22}, \href{works/PohlAK22.pdf}{PohlAK22}~\cite{PohlAK22}, \href{works/GeibingerMM21.pdf}{GeibingerMM21}~\cite{GeibingerMM21}, \href{works/WallaceY20.pdf}{WallaceY20}~\cite{WallaceY20}, \href{works/WangB20.pdf}{WangB20}~\cite{WangB20}, \href{works/BarzegaranZP20.pdf}{BarzegaranZP20}~\cite{BarzegaranZP20}, \href{works/Mercier-AubinGQ20.pdf}{Mercier-AubinGQ20}~\cite{Mercier-AubinGQ20}, \href{works/YangSS19.pdf}{YangSS19}~\cite{YangSS19}, \href{works/BadicaBIL19.pdf}{BadicaBIL19}~\cite{BadicaBIL19}, \href{works/NishikawaSTT19.pdf}{NishikawaSTT19}~\cite{NishikawaSTT19}, \href{works/Tom19.pdf}{Tom19}~\cite{Tom19}, \href{works/YounespourAKE19.pdf}{YounespourAKE19}~\cite{YounespourAKE19}, \href{works/HoYCLLCLC18.pdf}{HoYCLLCLC18}~\cite{HoYCLLCLC18}, \href{works/PourDERB18.pdf}{PourDERB18}~\cite{PourDERB18}, \href{works/Laborie18a.pdf}{Laborie18a}~\cite{Laborie18a}, \href{works/BofillCSV17.pdf}{BofillCSV17}~\cite{BofillCSV17}, \href{works/CappartS17.pdf}{CappartS17}~\cite{CappartS17}, \href{works/KletzanderM17.pdf}{KletzanderM17}~\cite{KletzanderM17}, \href{works/YoungFS17.pdf}{YoungFS17}~\cite{YoungFS17}, \href{works/LiuCGM17.pdf}{LiuCGM17}~\cite{LiuCGM17}, \href{works/TranVNB17.pdf}{TranVNB17}~\cite{TranVNB17}, \href{works/TranVNB17a.pdf}{TranVNB17a}~\cite{TranVNB17a}, \href{works/LimHTB16.pdf}{LimHTB16}~\cite{LimHTB16}, \href{works/NovaraNH16.pdf}{NovaraNH16}~\cite{NovaraNH16}... (Total: 104)\\
Concepts & make to order &  &  & \href{works/OujanaAYB22.pdf}{OujanaAYB22}~\cite{OujanaAYB22}, \href{works/DavenportKRSH07.pdf}{DavenportKRSH07}~\cite{DavenportKRSH07}, \href{works/Simonis07.pdf}{Simonis07}~\cite{Simonis07}\\
Concepts & make to stock &  &  & \\
Concepts & make-span & \href{works/PrataAN23.pdf}{PrataAN23}~\cite{PrataAN23}, \href{works/JuvinHL23.pdf}{JuvinHL23}~\cite{JuvinHL23}, \href{works/AbreuNP23.pdf}{AbreuNP23}~\cite{AbreuNP23}, \href{works/EfthymiouY23.pdf}{EfthymiouY23}~\cite{EfthymiouY23}, \href{works/PovedaAA23.pdf}{PovedaAA23}~\cite{PovedaAA23}, \href{works/AlfieriGPS23.pdf}{AlfieriGPS23}~\cite{AlfieriGPS23}, \href{works/abs-2305-19888.pdf}{abs-2305-19888}~\cite{abs-2305-19888}, \href{works/NaderiRR23.pdf}{NaderiRR23}~\cite{NaderiRR23}, \href{works/TasselGS23.pdf}{TasselGS23}~\cite{TasselGS23}, \href{works/Bit-Monnot23.pdf}{Bit-Monnot23}~\cite{Bit-Monnot23}, \href{works/abs-2306-05747.pdf}{abs-2306-05747}~\cite{abs-2306-05747}, \href{works/AalianPG23.pdf}{AalianPG23}~\cite{AalianPG23}, \href{works/CzerniachowskaWZ23.pdf}{CzerniachowskaWZ23}~\cite{CzerniachowskaWZ23}, \href{works/LacknerMMWW23.pdf}{LacknerMMWW23}~\cite{LacknerMMWW23}, \href{works/JuvinHHL23.pdf}{JuvinHHL23}~\cite{JuvinHHL23}, \href{works/YuraszeckMC23.pdf}{YuraszeckMC23}~\cite{YuraszeckMC23}, \href{works/IsikYA23.pdf}{IsikYA23}~\cite{IsikYA23}, \href{works/Mehdizadeh-Somarin23.pdf}{Mehdizadeh-Somarin23}~\cite{Mehdizadeh-Somarin23}, \href{works/HeinzNVH22.pdf}{HeinzNVH22}~\cite{HeinzNVH22}, \href{works/AbreuN22.pdf}{AbreuN22}~\cite{AbreuN22}, \href{works/GeitzGSSW22.pdf}{GeitzGSSW22}~\cite{GeitzGSSW22}, \href{works/BoudreaultSLQ22.pdf}{BoudreaultSLQ22}~\cite{BoudreaultSLQ22}, \href{works/YunusogluY22.pdf}{YunusogluY22}~\cite{YunusogluY22}, \href{works/SubulanC22.pdf}{SubulanC22}~\cite{SubulanC22}, \href{works/ArmstrongGOS22.pdf}{ArmstrongGOS22}~\cite{ArmstrongGOS22}, \href{works/TouatBT22.pdf}{TouatBT22}~\cite{TouatBT22}, \href{works/ColT22.pdf}{ColT22}~\cite{ColT22}, \href{works/FarsiTM22.pdf}{FarsiTM22}~\cite{FarsiTM22}, \href{works/ZhangBB22.pdf}{ZhangBB22}~\cite{ZhangBB22}... (Total: 140) & \href{works/YuraszeckMCCR23.pdf}{YuraszeckMCCR23}~\cite{YuraszeckMCCR23}, \href{works/abs-2312-13682.pdf}{abs-2312-13682}~\cite{abs-2312-13682}, \href{works/PerezGSL23.pdf}{PerezGSL23}~\cite{PerezGSL23}, \href{works/KameugneFND23.pdf}{KameugneFND23}~\cite{KameugneFND23}, \href{works/MullerMKP22.pdf}{MullerMKP22}~\cite{MullerMKP22}, \href{works/SvancaraB22.pdf}{SvancaraB22}~\cite{SvancaraB22}, \href{works/OujanaAYB22.pdf}{OujanaAYB22}~\cite{OujanaAYB22}, \href{works/ZhangJZL22.pdf}{ZhangJZL22}~\cite{ZhangJZL22}, \href{works/abs-2211-14492.pdf}{abs-2211-14492}~\cite{abs-2211-14492}, \href{works/YuraszeckMPV22.pdf}{YuraszeckMPV22}~\cite{YuraszeckMPV22}, \href{works/LiFJZLL22.pdf}{LiFJZLL22}~\cite{LiFJZLL22}, \href{works/PandeyS21a.pdf}{PandeyS21a}~\cite{PandeyS21a}, \href{works/FanXG21.pdf}{FanXG21}~\cite{FanXG21}, \href{works/QinDCS20.pdf}{QinDCS20}~\cite{QinDCS20}, \href{works/AstrandJZ18.pdf}{AstrandJZ18}~\cite{AstrandJZ18}, \href{works/KreterSS17.pdf}{KreterSS17}~\cite{KreterSS17}, \href{works/YoungFS17.pdf}{YoungFS17}~\cite{YoungFS17}, \href{works/BonfiettiZLM16.pdf}{BonfiettiZLM16}~\cite{BonfiettiZLM16}, \href{works/GingrasQ16.pdf}{GingrasQ16}~\cite{GingrasQ16}, \href{works/SialaAH15.pdf}{SialaAH15}~\cite{SialaAH15}, \href{works/DejemeppeCS15.pdf}{DejemeppeCS15}~\cite{DejemeppeCS15}, \href{works/GayHLS15.pdf}{GayHLS15}~\cite{GayHLS15}, \href{works/BajestaniB15.pdf}{BajestaniB15}~\cite{BajestaniB15}, \href{works/BonfiettiLBM14.pdf}{BonfiettiLBM14}~\cite{BonfiettiLBM14}, \href{works/ThiruvadyWGS14.pdf}{ThiruvadyWGS14}~\cite{ThiruvadyWGS14}, \href{works/KameugneFSN14.pdf}{KameugneFSN14}~\cite{KameugneFSN14}, \href{works/GuSS13.pdf}{GuSS13}~\cite{GuSS13}, \href{works/LombardiM12.pdf}{LombardiM12}~\cite{LombardiM12}, \href{works/BillautHL12.pdf}{BillautHL12}~\cite{BillautHL12}... (Total: 45) & \href{works/KimCMLLP23.pdf}{KimCMLLP23}~\cite{KimCMLLP23}, \href{works/TardivoDFMP23.pdf}{TardivoDFMP23}~\cite{TardivoDFMP23}, \href{works/Teppan22.pdf}{Teppan22}~\cite{Teppan22}, \href{works/PopovicCGNC22.pdf}{PopovicCGNC22}~\cite{PopovicCGNC22}, \href{works/CampeauG22.pdf}{CampeauG22}~\cite{CampeauG22}, \href{works/JungblutK22.pdf}{JungblutK22}~\cite{JungblutK22}, \href{works/FetgoD22.pdf}{FetgoD22}~\cite{FetgoD22}, \href{works/HanenKP21.pdf}{HanenKP21}~\cite{HanenKP21}, \href{works/KoehlerBFFHPSSS21.pdf}{KoehlerBFFHPSSS21}~\cite{KoehlerBFFHPSSS21}, \href{works/HubnerGSV21.pdf}{HubnerGSV21}~\cite{HubnerGSV21}, \href{works/Mercier-AubinGQ20.pdf}{Mercier-AubinGQ20}~\cite{Mercier-AubinGQ20}, \href{works/TangB20.pdf}{TangB20}~\cite{TangB20}, \href{works/NattafM20.pdf}{NattafM20}~\cite{NattafM20}, \href{works/SacramentoSP20.pdf}{SacramentoSP20}~\cite{SacramentoSP20}, \href{works/NishikawaSTT19.pdf}{NishikawaSTT19}~\cite{NishikawaSTT19}, \href{works/MurinR19.pdf}{MurinR19}~\cite{MurinR19}, \href{works/abs-1911-04766.pdf}{abs-1911-04766}~\cite{abs-1911-04766}, \href{works/BadicaBIL19.pdf}{BadicaBIL19}~\cite{BadicaBIL19}, \href{works/Tom19.pdf}{Tom19}~\cite{Tom19}, \href{works/GeibingerMM19.pdf}{GeibingerMM19}~\cite{GeibingerMM19}, \href{works/NishikawaSTT18.pdf}{NishikawaSTT18}~\cite{NishikawaSTT18}, \href{works/BorghesiBLMB18.pdf}{BorghesiBLMB18}~\cite{BorghesiBLMB18}, \href{works/ArbaouiY18.pdf}{ArbaouiY18}~\cite{ArbaouiY18}, \href{works/Ham18.pdf}{Ham18}~\cite{Ham18}, \href{works/NishikawaSTT18a.pdf}{NishikawaSTT18a}~\cite{NishikawaSTT18a}, \href{works/OuelletQ18.pdf}{OuelletQ18}~\cite{OuelletQ18}, \href{works/KameugneFGOQ18.pdf}{KameugneFGOQ18}~\cite{KameugneFGOQ18}, \href{works/Tesch18.pdf}{Tesch18}~\cite{Tesch18}, \href{works/TranPZLDB18.pdf}{TranPZLDB18}~\cite{TranPZLDB18}... (Total: 85)\\
Concepts & manpower & \href{works/NovaraNH16.pdf}{NovaraNH16}~\cite{NovaraNH16} & \href{works/LaborieRSV18.pdf}{LaborieRSV18}~\cite{LaborieRSV18} & \href{works/BourreauGGLT22.pdf}{BourreauGGLT22}~\cite{BourreauGGLT22}, \href{works/BadicaBI20.pdf}{BadicaBI20}~\cite{BadicaBI20}, \href{works/MokhtarzadehTNF20.pdf}{MokhtarzadehTNF20}~\cite{MokhtarzadehTNF20}, \href{works/WikarekS19.pdf}{WikarekS19}~\cite{WikarekS19}, \href{works/BaptisteB18.pdf}{BaptisteB18}~\cite{BaptisteB18}, \href{works/MusliuSS18.pdf}{MusliuSS18}~\cite{MusliuSS18}, \href{works/SchuttS16.pdf}{SchuttS16}~\cite{SchuttS16}, \href{works/HechingH16.pdf}{HechingH16}~\cite{HechingH16}, \href{works/GayHS15a.pdf}{GayHS15a}~\cite{GayHS15a}, \href{works/GaySS14.pdf}{GaySS14}~\cite{GaySS14}, \href{works/LombardiM12.pdf}{LombardiM12}~\cite{LombardiM12}, \href{works/Vilim11.pdf}{Vilim11}~\cite{Vilim11}, \href{works/NovasH10.pdf}{NovasH10}~\cite{NovasH10}, \href{works/NuijtenP98.pdf}{NuijtenP98}~\cite{NuijtenP98}, \href{works/SimonisC95.pdf}{SimonisC95}~\cite{SimonisC95}, \href{works/Puget95.pdf}{Puget95}~\cite{Puget95}\\
Concepts & multi-agent & \href{works/SvancaraB22.pdf}{SvancaraB22}~\cite{SvancaraB22}, \href{works/BehrensLM19.pdf}{BehrensLM19}~\cite{BehrensLM19}, \href{works/He0GLW18.pdf}{He0GLW18}~\cite{He0GLW18}, \href{works/HoeveGSL07.pdf}{HoeveGSL07}~\cite{HoeveGSL07} & \href{works/MokhtarzadehTNF20.pdf}{MokhtarzadehTNF20}~\cite{MokhtarzadehTNF20}, \href{works/abs-1901-07914.pdf}{abs-1901-07914}~\cite{abs-1901-07914}, \href{works/TranVNB17.pdf}{TranVNB17}~\cite{TranVNB17}, \href{works/LimHTB16.pdf}{LimHTB16}~\cite{LimHTB16}, \href{works/BartakSR10.pdf}{BartakSR10}~\cite{BartakSR10}, \href{works/BocewiczBB09.pdf}{BocewiczBB09}~\cite{BocewiczBB09} & \href{works/abs-2402-00459.pdf}{abs-2402-00459}~\cite{abs-2402-00459}, \href{works/Mehdizadeh-Somarin23.pdf}{Mehdizadeh-Somarin23}~\cite{Mehdizadeh-Somarin23}, \href{works/SquillaciPR23.pdf}{SquillaciPR23}~\cite{SquillaciPR23}, \href{works/AbreuAPNM21.pdf}{AbreuAPNM21}~\cite{AbreuAPNM21}, \href{works/ZhangYW21.pdf}{ZhangYW21}~\cite{ZhangYW21}, \href{works/MejiaY20.pdf}{MejiaY20}~\cite{MejiaY20}, \href{works/WessenCS20.pdf}{WessenCS20}~\cite{WessenCS20}, \href{works/WikarekS19.pdf}{WikarekS19}~\cite{WikarekS19}, \href{works/BadicaBIL19.pdf}{BadicaBIL19}~\cite{BadicaBIL19}, \href{works/ZhangW18.pdf}{ZhangW18}~\cite{ZhangW18}, \href{works/LimBTBB15.pdf}{LimBTBB15}~\cite{LimBTBB15}, \href{works/KoschB14.pdf}{KoschB14}~\cite{KoschB14}, \href{works/BartakS11.pdf}{BartakS11}~\cite{BartakS11}, \href{works/GomesHS06.pdf}{GomesHS06}~\cite{GomesHS06}, \href{works/AbrilSB05.pdf}{AbrilSB05}~\cite{AbrilSB05}, \href{works/BeckF98.pdf}{BeckF98}~\cite{BeckF98}, \href{works/Wallace96.pdf}{Wallace96}~\cite{Wallace96}\\
Concepts & no preempt &  &  & \href{works/ColT22.pdf}{ColT22}~\cite{ColT22}, \href{works/TouatBT22.pdf}{TouatBT22}~\cite{TouatBT22}, \href{works/FanXG21.pdf}{FanXG21}~\cite{FanXG21}, \href{works/Bedhief21.pdf}{Bedhief21}~\cite{Bedhief21}, \href{works/MengZRZL20.pdf}{MengZRZL20}~\cite{MengZRZL20}, \href{works/ParkUJR19.pdf}{ParkUJR19}~\cite{ParkUJR19}, \href{works/TerekhovTDB14.pdf}{TerekhovTDB14}~\cite{TerekhovTDB14}, \href{works/MonetteDD07.pdf}{MonetteDD07}~\cite{MonetteDD07}, \href{works/BeckW07.pdf}{BeckW07}~\cite{BeckW07}, \href{works/ArtiguesR00.pdf}{ArtiguesR00}~\cite{ArtiguesR00}\\
Concepts & open-shop & \href{works/PrataAN23.pdf}{PrataAN23}~\cite{PrataAN23}, \href{works/Bit-Monnot23.pdf}{Bit-Monnot23}~\cite{Bit-Monnot23}, \href{works/AbreuNP23.pdf}{AbreuNP23}~\cite{AbreuNP23}, \href{works/NaderiRR23.pdf}{NaderiRR23}~\cite{NaderiRR23}, \href{works/YuraszeckMPV22.pdf}{YuraszeckMPV22}~\cite{YuraszeckMPV22}, \href{works/AbreuN22.pdf}{AbreuN22}~\cite{AbreuN22}, \href{works/AbreuAPNM21.pdf}{AbreuAPNM21}~\cite{AbreuAPNM21}, \href{works/MejiaY20.pdf}{MejiaY20}~\cite{MejiaY20}, \href{works/FahimiOQ18.pdf}{FahimiOQ18}~\cite{FahimiOQ18}, \href{works/GrimesHM09.pdf}{GrimesHM09}~\cite{GrimesHM09}, \href{works/MonetteDD07.pdf}{MonetteDD07}~\cite{MonetteDD07}, \href{works/LorigeonBB02.pdf}{LorigeonBB02}~\cite{LorigeonBB02}, \href{works/FocacciLN00.pdf}{FocacciLN00}~\cite{FocacciLN00} & \href{works/SacramentoSP20.pdf}{SacramentoSP20}~\cite{SacramentoSP20}, \href{works/MengZRZL20.pdf}{MengZRZL20}~\cite{MengZRZL20}, \href{works/GrimesH10.pdf}{GrimesH10}~\cite{GrimesH10}, \href{works/Vilim05.pdf}{Vilim05}~\cite{Vilim05} & \href{works/YuraszeckMCCR23.pdf}{YuraszeckMCCR23}~\cite{YuraszeckMCCR23}, \href{works/YuraszeckMC23.pdf}{YuraszeckMC23}~\cite{YuraszeckMC23}, \href{works/KimCMLLP23.pdf}{KimCMLLP23}~\cite{KimCMLLP23}, \href{works/ShaikhK23.pdf}{ShaikhK23}~\cite{ShaikhK23}, \href{works/OujanaAYB22.pdf}{OujanaAYB22}~\cite{OujanaAYB22}, \href{works/ColT22.pdf}{ColT22}~\cite{ColT22}, \href{works/Astrand0F21.pdf}{Astrand0F21}~\cite{Astrand0F21}, \href{works/abs-2102-08778.pdf}{abs-2102-08778}~\cite{abs-2102-08778}, \href{works/AstrandJZ20.pdf}{AstrandJZ20}~\cite{AstrandJZ20}, \href{works/ParkUJR19.pdf}{ParkUJR19}~\cite{ParkUJR19}, \href{works/SialaAH15.pdf}{SialaAH15}~\cite{SialaAH15}, \href{works/BonfiettiLM14.pdf}{BonfiettiLM14}~\cite{BonfiettiLM14}, \href{works/AlesioNBG14.pdf}{AlesioNBG14}~\cite{AlesioNBG14}, \href{works/BillautHL12.pdf}{BillautHL12}~\cite{BillautHL12}, \href{works/SchuttFSW11.pdf}{SchuttFSW11}~\cite{SchuttFSW11}, \href{works/GrimesH11.pdf}{GrimesH11}~\cite{GrimesH11}, \href{works/BartakSR10.pdf}{BartakSR10}~\cite{BartakSR10}, \href{works/SchuttFSW09.pdf}{SchuttFSW09}~\cite{SchuttFSW09}, \href{works/ThiruvadyBME09.pdf}{ThiruvadyBME09}~\cite{ThiruvadyBME09}, \href{works/VilimBC05.pdf}{VilimBC05}~\cite{VilimBC05}, \href{works/ArtiouchineB05.pdf}{ArtiouchineB05}~\cite{ArtiouchineB05}, \href{works/HentenryckM04.pdf}{HentenryckM04}~\cite{HentenryckM04}, \href{works/VilimBC04.pdf}{VilimBC04}~\cite{VilimBC04}, \href{works/Vilim03.pdf}{Vilim03}~\cite{Vilim03}, \href{works/ElkhyariGJ02a.pdf}{ElkhyariGJ02a}~\cite{ElkhyariGJ02a}, \href{works/VerfaillieL01.pdf}{VerfaillieL01}~\cite{VerfaillieL01}, \href{works/SourdN00.pdf}{SourdN00}~\cite{SourdN00}\\
Concepts & order & \href{works/abs-2402-00459.pdf}{abs-2402-00459}~\cite{abs-2402-00459}, \href{works/PrataAN23.pdf}{PrataAN23}~\cite{PrataAN23}, \href{works/EfthymiouY23.pdf}{EfthymiouY23}~\cite{EfthymiouY23}, \href{works/AbreuNP23.pdf}{AbreuNP23}~\cite{AbreuNP23}, \href{works/AlfieriGPS23.pdf}{AlfieriGPS23}~\cite{AlfieriGPS23}, \href{works/abs-2312-13682.pdf}{abs-2312-13682}~\cite{abs-2312-13682}, \href{works/CzerniachowskaWZ23.pdf}{CzerniachowskaWZ23}~\cite{CzerniachowskaWZ23}, \href{works/TasselGS23.pdf}{TasselGS23}~\cite{TasselGS23}, \href{works/AalianPG23.pdf}{AalianPG23}~\cite{AalianPG23}, \href{works/abs-2306-05747.pdf}{abs-2306-05747}~\cite{abs-2306-05747}, \href{works/Bit-Monnot23.pdf}{Bit-Monnot23}~\cite{Bit-Monnot23}, \href{works/JuvinHL23.pdf}{JuvinHL23}~\cite{JuvinHL23}, \href{works/WangB23.pdf}{WangB23}~\cite{WangB23}, \href{works/KameugneFND23.pdf}{KameugneFND23}~\cite{KameugneFND23}, \href{works/LacknerMMWW23.pdf}{LacknerMMWW23}~\cite{LacknerMMWW23}, \href{works/PerezGSL23.pdf}{PerezGSL23}~\cite{PerezGSL23}, \href{works/JuvinHHL23.pdf}{JuvinHHL23}~\cite{JuvinHHL23}, \href{works/SquillaciPR23.pdf}{SquillaciPR23}~\cite{SquillaciPR23}, \href{works/IsikYA23.pdf}{IsikYA23}~\cite{IsikYA23}, \href{works/YuraszeckMCCR23.pdf}{YuraszeckMCCR23}~\cite{YuraszeckMCCR23}, \href{works/KimCMLLP23.pdf}{KimCMLLP23}~\cite{KimCMLLP23}, \href{works/PovedaAA23.pdf}{PovedaAA23}~\cite{PovedaAA23}, \href{works/PopovicCGNC22.pdf}{PopovicCGNC22}~\cite{PopovicCGNC22}, \href{works/BoudreaultSLQ22.pdf}{BoudreaultSLQ22}~\cite{BoudreaultSLQ22}, \href{works/LuoB22.pdf}{LuoB22}~\cite{LuoB22}, \href{works/CampeauG22.pdf}{CampeauG22}~\cite{CampeauG22}, \href{works/YunusogluY22.pdf}{YunusogluY22}~\cite{YunusogluY22}, \href{works/AbreuN22.pdf}{AbreuN22}~\cite{AbreuN22}, \href{works/BourreauGGLT22.pdf}{BourreauGGLT22}~\cite{BourreauGGLT22}... (Total: 310) & \href{works/MontemanniD23a.pdf}{MontemanniD23a}~\cite{MontemanniD23a}, \href{works/ShaikhK23.pdf}{ShaikhK23}~\cite{ShaikhK23}, \href{works/abs-2305-19888.pdf}{abs-2305-19888}~\cite{abs-2305-19888}, \href{works/NaderiRR23.pdf}{NaderiRR23}~\cite{NaderiRR23}, \href{works/TardivoDFMP23.pdf}{TardivoDFMP23}~\cite{TardivoDFMP23}, \href{works/YuraszeckMC23.pdf}{YuraszeckMC23}~\cite{YuraszeckMC23}, \href{works/GurPAE23.pdf}{GurPAE23}~\cite{GurPAE23}, \href{works/OuelletQ22.pdf}{OuelletQ22}~\cite{OuelletQ22}, \href{works/SvancaraB22.pdf}{SvancaraB22}~\cite{SvancaraB22}, \href{works/ArmstrongGOS22.pdf}{ArmstrongGOS22}~\cite{ArmstrongGOS22}, \href{works/WinterMMW22.pdf}{WinterMMW22}~\cite{WinterMMW22}, \href{works/HeinzNVH22.pdf}{HeinzNVH22}~\cite{HeinzNVH22}, \href{works/JungblutK22.pdf}{JungblutK22}~\cite{JungblutK22}, \href{works/TouatBT22.pdf}{TouatBT22}~\cite{TouatBT22}, \href{works/ZhangBB22.pdf}{ZhangBB22}~\cite{ZhangBB22}, \href{works/BenderWS21.pdf}{BenderWS21}~\cite{BenderWS21}, \href{works/GeibingerMM21.pdf}{GeibingerMM21}~\cite{GeibingerMM21}, \href{works/HillTV21.pdf}{HillTV21}~\cite{HillTV21}, \href{works/abs-2102-08778.pdf}{abs-2102-08778}~\cite{abs-2102-08778}, \href{works/QinDCS20.pdf}{QinDCS20}~\cite{QinDCS20}, \href{works/WallaceY20.pdf}{WallaceY20}~\cite{WallaceY20}, \href{works/ZouZ20.pdf}{ZouZ20}~\cite{ZouZ20}, \href{works/TangB20.pdf}{TangB20}~\cite{TangB20}, \href{works/ColT19.pdf}{ColT19}~\cite{ColT19}, \href{works/BogaerdtW19.pdf}{BogaerdtW19}~\cite{BogaerdtW19}, \href{works/FrohnerTR19.pdf}{FrohnerTR19}~\cite{FrohnerTR19}, \href{works/YounespourAKE19.pdf}{YounespourAKE19}~\cite{YounespourAKE19}, \href{works/DemirovicS18.pdf}{DemirovicS18}~\cite{DemirovicS18}, \href{works/ShinBBHO18.pdf}{ShinBBHO18}~\cite{ShinBBHO18}... (Total: 89) & \href{works/MontemanniD23.pdf}{MontemanniD23}~\cite{MontemanniD23}, \href{works/AkramNHRSA23.pdf}{AkramNHRSA23}~\cite{AkramNHRSA23}, \href{works/Mehdizadeh-Somarin23.pdf}{Mehdizadeh-Somarin23}~\cite{Mehdizadeh-Somarin23}, \href{works/ZhangJZL22.pdf}{ZhangJZL22}~\cite{ZhangJZL22}, \href{works/AbohashimaEG21.pdf}{AbohashimaEG21}~\cite{AbohashimaEG21}, \href{works/ZhangYW21.pdf}{ZhangYW21}~\cite{ZhangYW21}, \href{works/MokhtarzadehTNF20.pdf}{MokhtarzadehTNF20}~\cite{MokhtarzadehTNF20}, \href{works/KucukY19.pdf}{KucukY19}~\cite{KucukY19}, \href{works/abs-1902-01193.pdf}{abs-1902-01193}~\cite{abs-1902-01193}, \href{works/GalleguillosKSB19.pdf}{GalleguillosKSB19}~\cite{GalleguillosKSB19}, \href{works/ArbaouiY18.pdf}{ArbaouiY18}~\cite{ArbaouiY18}, \href{works/BenediktSMVH18.pdf}{BenediktSMVH18}~\cite{BenediktSMVH18}, \href{works/He0GLW18.pdf}{He0GLW18}~\cite{He0GLW18}, \href{works/Hooker17.pdf}{Hooker17}~\cite{Hooker17}, \href{works/TranVNB17a.pdf}{TranVNB17a}~\cite{TranVNB17a}, \href{works/Bonfietti16.pdf}{Bonfietti16}~\cite{Bonfietti16}, \href{works/SzerediS16.pdf}{SzerediS16}~\cite{SzerediS16}, \href{works/HechingH16.pdf}{HechingH16}~\cite{HechingH16}, \href{works/BridiLBBM16.pdf}{BridiLBBM16}~\cite{BridiLBBM16}, \href{works/GayHS15a.pdf}{GayHS15a}~\cite{GayHS15a}, \href{works/ThiruvadyWGS14.pdf}{ThiruvadyWGS14}~\cite{ThiruvadyWGS14}, \href{works/DoulabiRP14.pdf}{DoulabiRP14}~\cite{DoulabiRP14}, \href{works/GuSS13.pdf}{GuSS13}~\cite{GuSS13}, \href{works/LombardiM13.pdf}{LombardiM13}~\cite{LombardiM13}, \href{works/SchuttFS13.pdf}{SchuttFS13}~\cite{SchuttFS13}, \href{works/BonfiettiLM13.pdf}{BonfiettiLM13}~\cite{BonfiettiLM13}, \href{works/HeinzKB13.pdf}{HeinzKB13}~\cite{HeinzKB13}, \href{works/HeinzB12.pdf}{HeinzB12}~\cite{HeinzB12}, \href{works/BonfiettiLBM11.pdf}{BonfiettiLBM11}~\cite{BonfiettiLBM11}... (Total: 54)\\
Concepts & precedence & \href{works/abs-2402-00459.pdf}{abs-2402-00459}~\cite{abs-2402-00459}, \href{works/PovedaAA23.pdf}{PovedaAA23}~\cite{PovedaAA23}, \href{works/YuraszeckMCCR23.pdf}{YuraszeckMCCR23}~\cite{YuraszeckMCCR23}, \href{works/NaderiRR23.pdf}{NaderiRR23}~\cite{NaderiRR23}, \href{works/IsikYA23.pdf}{IsikYA23}~\cite{IsikYA23}, \href{works/AlfieriGPS23.pdf}{AlfieriGPS23}~\cite{AlfieriGPS23}, \href{works/JuvinHHL23.pdf}{JuvinHHL23}~\cite{JuvinHHL23}, \href{works/FetgoD22.pdf}{FetgoD22}~\cite{FetgoD22}, \href{works/PohlAK22.pdf}{PohlAK22}~\cite{PohlAK22}, \href{works/CampeauG22.pdf}{CampeauG22}~\cite{CampeauG22}, \href{works/YunusogluY22.pdf}{YunusogluY22}~\cite{YunusogluY22}, \href{works/BoudreaultSLQ22.pdf}{BoudreaultSLQ22}~\cite{BoudreaultSLQ22}, \href{works/ZhangBB22.pdf}{ZhangBB22}~\cite{ZhangBB22}, \href{works/GeibingerMM21.pdf}{GeibingerMM21}~\cite{GeibingerMM21}, \href{works/HamPK21.pdf}{HamPK21}~\cite{HamPK21}, \href{works/HanenKP21.pdf}{HanenKP21}~\cite{HanenKP21}, \href{works/Astrand0F21.pdf}{Astrand0F21}~\cite{Astrand0F21}, \href{works/HillTV21.pdf}{HillTV21}~\cite{HillTV21}, \href{works/KoehlerBFFHPSSS21.pdf}{KoehlerBFFHPSSS21}~\cite{KoehlerBFFHPSSS21}, \href{works/FanXG21.pdf}{FanXG21}~\cite{FanXG21}, \href{works/HubnerGSV21.pdf}{HubnerGSV21}~\cite{HubnerGSV21}, \href{works/ArmstrongGOS21.pdf}{ArmstrongGOS21}~\cite{ArmstrongGOS21}, \href{works/ZhangYW21.pdf}{ZhangYW21}~\cite{ZhangYW21}, \href{works/GroleazNS20.pdf}{GroleazNS20}~\cite{GroleazNS20}, \href{works/SacramentoSP20.pdf}{SacramentoSP20}~\cite{SacramentoSP20}, \href{works/Polo-MejiaALB20.pdf}{Polo-MejiaALB20}~\cite{Polo-MejiaALB20}, \href{works/AstrandJZ20.pdf}{AstrandJZ20}~\cite{AstrandJZ20}, \href{works/Mercier-AubinGQ20.pdf}{Mercier-AubinGQ20}~\cite{Mercier-AubinGQ20}, \href{works/LunardiBLRV20.pdf}{LunardiBLRV20}~\cite{LunardiBLRV20}... (Total: 130) & \href{works/Bit-Monnot23.pdf}{Bit-Monnot23}~\cite{Bit-Monnot23}, \href{works/KameugneFND23.pdf}{KameugneFND23}~\cite{KameugneFND23}, \href{works/TardivoDFMP23.pdf}{TardivoDFMP23}~\cite{TardivoDFMP23}, \href{works/OujanaAYB22.pdf}{OujanaAYB22}~\cite{OujanaAYB22}, \href{works/SubulanC22.pdf}{SubulanC22}~\cite{SubulanC22}, \href{works/ColT22.pdf}{ColT22}~\cite{ColT22}, \href{works/VlkHT21.pdf}{VlkHT21}~\cite{VlkHT21}, \href{works/AntuoriHHEN21.pdf}{AntuoriHHEN21}~\cite{AntuoriHHEN21}, \href{works/WessenCS20.pdf}{WessenCS20}~\cite{WessenCS20}, \href{works/MokhtarzadehTNF20.pdf}{MokhtarzadehTNF20}~\cite{MokhtarzadehTNF20}, \href{works/QinDCS20.pdf}{QinDCS20}~\cite{QinDCS20}, \href{works/GeibingerMM19.pdf}{GeibingerMM19}~\cite{GeibingerMM19}, \href{works/Novas19.pdf}{Novas19}~\cite{Novas19}, \href{works/abs-1911-04766.pdf}{abs-1911-04766}~\cite{abs-1911-04766}, \href{works/ColT19.pdf}{ColT19}~\cite{ColT19}, \href{works/BogaerdtW19.pdf}{BogaerdtW19}~\cite{BogaerdtW19}, \href{works/MurinR19.pdf}{MurinR19}~\cite{MurinR19}, \href{works/Ham18.pdf}{Ham18}~\cite{Ham18}, \href{works/KameugneFGOQ18.pdf}{KameugneFGOQ18}~\cite{KameugneFGOQ18}, \href{works/Madi-WambaLOBM17.pdf}{Madi-WambaLOBM17}~\cite{Madi-WambaLOBM17}, \href{works/MossigeGSMC17.pdf}{MossigeGSMC17}~\cite{MossigeGSMC17}, \href{works/Madi-WambaB16.pdf}{Madi-WambaB16}~\cite{Madi-WambaB16}, \href{works/GayHLS15.pdf}{GayHLS15}~\cite{GayHLS15}, \href{works/VilimLS15.pdf}{VilimLS15}~\cite{VilimLS15}, \href{works/BurtLPS15.pdf}{BurtLPS15}~\cite{BurtLPS15}, \href{works/LombardiBM15.pdf}{LombardiBM15}~\cite{LombardiBM15}, \href{works/BartakV15.pdf}{BartakV15}~\cite{BartakV15}, \href{works/WangMD15.pdf}{WangMD15}~\cite{WangMD15}, \href{works/BonfiettiLM14.pdf}{BonfiettiLM14}~\cite{BonfiettiLM14}... (Total: 61) & \href{works/PrataAN23.pdf}{PrataAN23}~\cite{PrataAN23}, \href{works/KimCMLLP23.pdf}{KimCMLLP23}~\cite{KimCMLLP23}, \href{works/JuvinHL23.pdf}{JuvinHL23}~\cite{JuvinHL23}, \href{works/TasselGS23.pdf}{TasselGS23}~\cite{TasselGS23}, \href{works/abs-2305-19888.pdf}{abs-2305-19888}~\cite{abs-2305-19888}, \href{works/Mehdizadeh-Somarin23.pdf}{Mehdizadeh-Somarin23}~\cite{Mehdizadeh-Somarin23}, \href{works/abs-2306-05747.pdf}{abs-2306-05747}~\cite{abs-2306-05747}, \href{works/YuraszeckMC23.pdf}{YuraszeckMC23}~\cite{YuraszeckMC23}, \href{works/MullerMKP22.pdf}{MullerMKP22}~\cite{MullerMKP22}, \href{works/WinterMMW22.pdf}{WinterMMW22}~\cite{WinterMMW22}, \href{works/abs-2211-14492.pdf}{abs-2211-14492}~\cite{abs-2211-14492}, \href{works/HeinzNVH22.pdf}{HeinzNVH22}~\cite{HeinzNVH22}, \href{works/BourreauGGLT22.pdf}{BourreauGGLT22}~\cite{BourreauGGLT22}, \href{works/ZhangJZL22.pdf}{ZhangJZL22}~\cite{ZhangJZL22}, \href{works/GeitzGSSW22.pdf}{GeitzGSSW22}~\cite{GeitzGSSW22}, \href{works/TouatBT22.pdf}{TouatBT22}~\cite{TouatBT22}, \href{works/KovacsTKSG21.pdf}{KovacsTKSG21}~\cite{KovacsTKSG21}, \href{works/PandeyS21a.pdf}{PandeyS21a}~\cite{PandeyS21a}, \href{works/AbreuAPNM21.pdf}{AbreuAPNM21}~\cite{AbreuAPNM21}, \href{works/TangB20.pdf}{TangB20}~\cite{TangB20}, \href{works/GroleazNS20a.pdf}{GroleazNS20a}~\cite{GroleazNS20a}, \href{works/BaptisteB18.pdf}{BaptisteB18}~\cite{BaptisteB18}, \href{works/He0GLW18.pdf}{He0GLW18}~\cite{He0GLW18}, \href{works/OuelletQ18.pdf}{OuelletQ18}~\cite{OuelletQ18}, \href{works/GokgurHO18.pdf}{GokgurHO18}~\cite{GokgurHO18}, \href{works/DemirovicS18.pdf}{DemirovicS18}~\cite{DemirovicS18}, \href{works/CappartS17.pdf}{CappartS17}~\cite{CappartS17}, \href{works/KreterSS17.pdf}{KreterSS17}~\cite{KreterSS17}, \href{works/TranVNB17.pdf}{TranVNB17}~\cite{TranVNB17}... (Total: 85)\\
Concepts & preempt & \href{works/JuvinHHL23.pdf}{JuvinHHL23}~\cite{JuvinHHL23}, \href{works/PovedaAA23.pdf}{PovedaAA23}~\cite{PovedaAA23}, \href{works/SubulanC22.pdf}{SubulanC22}~\cite{SubulanC22}, \href{works/HanenKP21.pdf}{HanenKP21}~\cite{HanenKP21}, \href{works/Polo-MejiaALB20.pdf}{Polo-MejiaALB20}~\cite{Polo-MejiaALB20}, \href{works/BaptisteB18.pdf}{BaptisteB18}~\cite{BaptisteB18}, \href{works/GokgurHO18.pdf}{GokgurHO18}~\cite{GokgurHO18}, \href{works/FahimiOQ18.pdf}{FahimiOQ18}~\cite{FahimiOQ18}, \href{works/ZarandiKS16.pdf}{ZarandiKS16}~\cite{ZarandiKS16}, \href{works/EvenSH15.pdf}{EvenSH15}~\cite{EvenSH15}, \href{works/EvenSH15a.pdf}{EvenSH15a}~\cite{EvenSH15a}, \href{works/AlesioNBG14.pdf}{AlesioNBG14}~\cite{AlesioNBG14}, \href{works/LombardiM12.pdf}{LombardiM12}~\cite{LombardiM12}, \href{works/BeldiceanuCDP11.pdf}{BeldiceanuCDP11}~\cite{BeldiceanuCDP11}, \href{works/KovacsB11.pdf}{KovacsB11}~\cite{KovacsB11}, \href{works/BartakSR10.pdf}{BartakSR10}~\cite{BartakSR10}, \href{works/MonetteDD07.pdf}{MonetteDD07}~\cite{MonetteDD07}, \href{works/KovacsB07.pdf}{KovacsB07}~\cite{KovacsB07}, \href{works/Wolf03.pdf}{Wolf03}~\cite{Wolf03}, \href{works/BaptisteP00.pdf}{BaptisteP00}~\cite{BaptisteP00}, \href{works/PapaB98.pdf}{PapaB98}~\cite{PapaB98}, \href{works/PembertonG98.pdf}{PembertonG98}~\cite{PembertonG98}, \href{works/BaptisteP97.pdf}{BaptisteP97}~\cite{BaptisteP97} & \href{works/PrataAN23.pdf}{PrataAN23}~\cite{PrataAN23}, \href{works/abs-2305-19888.pdf}{abs-2305-19888}~\cite{abs-2305-19888}, \href{works/OuelletQ22.pdf}{OuelletQ22}~\cite{OuelletQ22}, \href{works/FetgoD22.pdf}{FetgoD22}~\cite{FetgoD22}, \href{works/HeinzNVH22.pdf}{HeinzNVH22}~\cite{HeinzNVH22}, \href{works/SacramentoSP20.pdf}{SacramentoSP20}~\cite{SacramentoSP20}, \href{works/Mercier-AubinGQ20.pdf}{Mercier-AubinGQ20}~\cite{Mercier-AubinGQ20}, \href{works/LunardiBLRV20.pdf}{LunardiBLRV20}~\cite{LunardiBLRV20}, \href{works/YoungFS17.pdf}{YoungFS17}~\cite{YoungFS17}, \href{works/NattafAL15.pdf}{NattafAL15}~\cite{NattafAL15}, \href{works/SimoninAHL15.pdf}{SimoninAHL15}~\cite{SimoninAHL15}, \href{works/TerekhovTDB14.pdf}{TerekhovTDB14}~\cite{TerekhovTDB14}, \href{works/OzturkTHO13.pdf}{OzturkTHO13}~\cite{OzturkTHO13}, \href{works/BajestaniB13.pdf}{BajestaniB13}~\cite{BajestaniB13}, \href{works/SimoninAHL12.pdf}{SimoninAHL12}~\cite{SimoninAHL12}, \href{works/SchuttFSW11.pdf}{SchuttFSW11}~\cite{SchuttFSW11}, \href{works/SchuttFSW09.pdf}{SchuttFSW09}~\cite{SchuttFSW09}, \href{works/Laborie09.pdf}{Laborie09}~\cite{Laborie09}, \href{works/KovacsB08.pdf}{KovacsB08}~\cite{KovacsB08}, \href{works/ArtiouchineB05.pdf}{ArtiouchineB05}~\cite{ArtiouchineB05}, \href{works/SourdN00.pdf}{SourdN00}~\cite{SourdN00}, \href{works/NuijtenP98.pdf}{NuijtenP98}~\cite{NuijtenP98} & \href{works/NaderiRR23.pdf}{NaderiRR23}~\cite{NaderiRR23}, \href{works/TasselGS23.pdf}{TasselGS23}~\cite{TasselGS23}, \href{works/AalianPG23.pdf}{AalianPG23}~\cite{AalianPG23}, \href{works/TardivoDFMP23.pdf}{TardivoDFMP23}~\cite{TardivoDFMP23}, \href{works/YuraszeckMC23.pdf}{YuraszeckMC23}~\cite{YuraszeckMC23}, \href{works/YuraszeckMCCR23.pdf}{YuraszeckMCCR23}~\cite{YuraszeckMCCR23}, \href{works/KameugneFND23.pdf}{KameugneFND23}~\cite{KameugneFND23}, \href{works/AkramNHRSA23.pdf}{AkramNHRSA23}~\cite{AkramNHRSA23}, \href{works/AbreuNP23.pdf}{AbreuNP23}~\cite{AbreuNP23}, \href{works/abs-2306-05747.pdf}{abs-2306-05747}~\cite{abs-2306-05747}, \href{works/IsikYA23.pdf}{IsikYA23}~\cite{IsikYA23}, \href{works/Mehdizadeh-Somarin23.pdf}{Mehdizadeh-Somarin23}~\cite{Mehdizadeh-Somarin23}, \href{works/AbreuN22.pdf}{AbreuN22}~\cite{AbreuN22}, \href{works/TouatBT22.pdf}{TouatBT22}~\cite{TouatBT22}, \href{works/Teppan22.pdf}{Teppan22}~\cite{Teppan22}, \href{works/GeitzGSSW22.pdf}{GeitzGSSW22}~\cite{GeitzGSSW22}, \href{works/BoudreaultSLQ22.pdf}{BoudreaultSLQ22}~\cite{BoudreaultSLQ22}, \href{works/ColT22.pdf}{ColT22}~\cite{ColT22}, \href{works/MullerMKP22.pdf}{MullerMKP22}~\cite{MullerMKP22}, \href{works/YunusogluY22.pdf}{YunusogluY22}~\cite{YunusogluY22}, \href{works/OujanaAYB22.pdf}{OujanaAYB22}~\cite{OujanaAYB22}, \href{works/JungblutK22.pdf}{JungblutK22}~\cite{JungblutK22}, \href{works/ZhangBB22.pdf}{ZhangBB22}~\cite{ZhangBB22}, \href{works/Bedhief21.pdf}{Bedhief21}~\cite{Bedhief21}, \href{works/BenderWS21.pdf}{BenderWS21}~\cite{BenderWS21}, \href{works/FanXG21.pdf}{FanXG21}~\cite{FanXG21}, \href{works/QinWSLS21.pdf}{QinWSLS21}~\cite{QinWSLS21}, \href{works/KovacsTKSG21.pdf}{KovacsTKSG21}~\cite{KovacsTKSG21}, \href{works/HubnerGSV21.pdf}{HubnerGSV21}~\cite{HubnerGSV21}... (Total: 122)\\
Concepts & producer/consumer & \href{works/SchuttS16.pdf}{SchuttS16}~\cite{SchuttS16}, \href{works/PoderBS04.pdf}{PoderBS04}~\cite{PoderBS04}, \href{works/Kumar03.pdf}{Kumar03}~\cite{Kumar03}, \href{works/SimonisC95.pdf}{SimonisC95}~\cite{SimonisC95} & \href{works/HermenierDL11.pdf}{HermenierDL11}~\cite{HermenierDL11}, \href{works/BeldiceanuC02.pdf}{BeldiceanuC02}~\cite{BeldiceanuC02} & \href{works/GeitzGSSW22.pdf}{GeitzGSSW22}~\cite{GeitzGSSW22}, \href{works/KlankeBYE21.pdf}{KlankeBYE21}~\cite{KlankeBYE21}, \href{works/LombardiM12a.pdf}{LombardiM12a}~\cite{LombardiM12a}, \href{works/PoderB08.pdf}{PoderB08}~\cite{PoderB08}, \href{works/Simonis07.pdf}{Simonis07}~\cite{Simonis07}, \href{works/Timpe02.pdf}{Timpe02}~\cite{Timpe02}, \href{works/Simonis95.pdf}{Simonis95}~\cite{Simonis95}\\
Concepts & re-scheduling & \href{works/HamPK21.pdf}{HamPK21}~\cite{HamPK21}, \href{works/BarzegaranZP20.pdf}{BarzegaranZP20}~\cite{BarzegaranZP20}, \href{works/ZhangW18.pdf}{ZhangW18}~\cite{ZhangW18}, \href{works/Madi-WambaLOBM17.pdf}{Madi-WambaLOBM17}~\cite{Madi-WambaLOBM17}, \href{works/CappartS17.pdf}{CappartS17}~\cite{CappartS17}, \href{works/BartakV15.pdf}{BartakV15}~\cite{BartakV15}, \href{works/GrimesIOS14.pdf}{GrimesIOS14}~\cite{GrimesIOS14}, \href{works/BajestaniB13.pdf}{BajestaniB13}~\cite{BajestaniB13}, \href{works/TranTDB13.pdf}{TranTDB13}~\cite{TranTDB13}, \href{works/RendlPHPR12.pdf}{RendlPHPR12}~\cite{RendlPHPR12}, \href{works/LombardiM12.pdf}{LombardiM12}~\cite{LombardiM12}, \href{works/IfrimOS12.pdf}{IfrimOS12}~\cite{IfrimOS12}, \href{works/NovasH10.pdf}{NovasH10}~\cite{NovasH10}, \href{works/BidotVLB09.pdf}{BidotVLB09}~\cite{BidotVLB09}, \href{works/MartinPY01.pdf}{MartinPY01}~\cite{MartinPY01}, \href{works/ArtiguesR00.pdf}{ArtiguesR00}~\cite{ArtiguesR00} & \href{works/Mehdizadeh-Somarin23.pdf}{Mehdizadeh-Somarin23}~\cite{Mehdizadeh-Somarin23}, \href{works/KovacsTKSG21.pdf}{KovacsTKSG21}~\cite{KovacsTKSG21}, \href{works/AstrandJZ20.pdf}{AstrandJZ20}~\cite{AstrandJZ20}, \href{works/HoYCLLCLC18.pdf}{HoYCLLCLC18}~\cite{HoYCLLCLC18}, \href{works/TranPZLDB18.pdf}{TranPZLDB18}~\cite{TranPZLDB18}, \href{works/LimHTB16.pdf}{LimHTB16}~\cite{LimHTB16}, \href{works/LimBTBB15.pdf}{LimBTBB15}~\cite{LimBTBB15}, \href{works/CobanH10.pdf}{CobanH10}~\cite{CobanH10}, \href{works/Acuna-AgostMFG09.pdf}{Acuna-AgostMFG09}~\cite{Acuna-AgostMFG09} & \href{works/PrataAN23.pdf}{PrataAN23}~\cite{PrataAN23}, \href{works/abs-2312-13682.pdf}{abs-2312-13682}~\cite{abs-2312-13682}, \href{works/abs-2306-05747.pdf}{abs-2306-05747}~\cite{abs-2306-05747}, \href{works/EfthymiouY23.pdf}{EfthymiouY23}~\cite{EfthymiouY23}, \href{works/ShaikhK23.pdf}{ShaikhK23}~\cite{ShaikhK23}, \href{works/abs-2305-19888.pdf}{abs-2305-19888}~\cite{abs-2305-19888}, \href{works/TasselGS23.pdf}{TasselGS23}~\cite{TasselGS23}, \href{works/GurPAE23.pdf}{GurPAE23}~\cite{GurPAE23}, \href{works/NaderiRR23.pdf}{NaderiRR23}~\cite{NaderiRR23}, \href{works/PerezGSL23.pdf}{PerezGSL23}~\cite{PerezGSL23}, \href{works/BourreauGGLT22.pdf}{BourreauGGLT22}~\cite{BourreauGGLT22}, \href{works/FarsiTM22.pdf}{FarsiTM22}~\cite{FarsiTM22}, \href{works/YunusogluY22.pdf}{YunusogluY22}~\cite{YunusogluY22}, \href{works/HeinzNVH22.pdf}{HeinzNVH22}~\cite{HeinzNVH22}, \href{works/ArmstrongGOS22.pdf}{ArmstrongGOS22}~\cite{ArmstrongGOS22}, \href{works/LuoB22.pdf}{LuoB22}~\cite{LuoB22}, \href{works/PohlAK22.pdf}{PohlAK22}~\cite{PohlAK22}, \href{works/YuraszeckMPV22.pdf}{YuraszeckMPV22}~\cite{YuraszeckMPV22}, \href{works/KlankeBYE21.pdf}{KlankeBYE21}~\cite{KlankeBYE21}, \href{works/PandeyS21a.pdf}{PandeyS21a}~\cite{PandeyS21a}, \href{works/ZhangYW21.pdf}{ZhangYW21}~\cite{ZhangYW21}, \href{works/BenediktMH20.pdf}{BenediktMH20}~\cite{BenediktMH20}, \href{works/MejiaY20.pdf}{MejiaY20}~\cite{MejiaY20}, \href{works/LunardiBLRV20.pdf}{LunardiBLRV20}~\cite{LunardiBLRV20}, \href{works/NishikawaSTT19.pdf}{NishikawaSTT19}~\cite{NishikawaSTT19}, \href{works/YounespourAKE19.pdf}{YounespourAKE19}~\cite{YounespourAKE19}, \href{works/GalleguillosKSB19.pdf}{GalleguillosKSB19}~\cite{GalleguillosKSB19}, \href{works/Tom19.pdf}{Tom19}~\cite{Tom19}, \href{works/abs-1911-04766.pdf}{abs-1911-04766}~\cite{abs-1911-04766}... (Total: 72)\\
Concepts & release-date & \href{works/WinterMMW22.pdf}{WinterMMW22}~\cite{WinterMMW22}, \href{works/YunusogluY22.pdf}{YunusogluY22}~\cite{YunusogluY22}, \href{works/YuraszeckMPV22.pdf}{YuraszeckMPV22}~\cite{YuraszeckMPV22}, \href{works/HanenKP21.pdf}{HanenKP21}~\cite{HanenKP21}, \href{works/Bedhief21.pdf}{Bedhief21}~\cite{Bedhief21}, \href{works/Polo-MejiaALB20.pdf}{Polo-MejiaALB20}~\cite{Polo-MejiaALB20}, \href{works/EscobetPQPRA19.pdf}{EscobetPQPRA19}~\cite{EscobetPQPRA19}, \href{works/Tesch18.pdf}{Tesch18}~\cite{Tesch18}, \href{works/KameugneFSN14.pdf}{KameugneFSN14}~\cite{KameugneFSN14}, \href{works/LimtanyakulS12.pdf}{LimtanyakulS12}~\cite{LimtanyakulS12}, \href{works/SerraNM12.pdf}{SerraNM12}~\cite{SerraNM12}, \href{works/KameugneFSN11.pdf}{KameugneFSN11}~\cite{KameugneFSN11}, \href{works/KovacsB11.pdf}{KovacsB11}~\cite{KovacsB11}, \href{works/LombardiM10a.pdf}{LombardiM10a}~\cite{LombardiM10a}, \href{works/BartakSR10.pdf}{BartakSR10}~\cite{BartakSR10}, \href{works/abs-0907-0939.pdf}{abs-0907-0939}~\cite{abs-0907-0939}, \href{works/AkkerDH07.pdf}{AkkerDH07}~\cite{AkkerDH07}, \href{works/KovacsB07.pdf}{KovacsB07}~\cite{KovacsB07}, \href{works/SadykovW06.pdf}{SadykovW06}~\cite{SadykovW06}, \href{works/ArtiouchineB05.pdf}{ArtiouchineB05}~\cite{ArtiouchineB05}, \href{works/Hooker05.pdf}{Hooker05}~\cite{Hooker05}, \href{works/SchuttWS05.pdf}{SchuttWS05}~\cite{SchuttWS05}, \href{works/Hooker04.pdf}{Hooker04}~\cite{Hooker04}, \href{works/Zhou97.pdf}{Zhou97}~\cite{Zhou97}, \href{works/Zhou96.pdf}{Zhou96}~\cite{Zhou96}, \href{works/Colombani96.pdf}{Colombani96}~\cite{Colombani96} & \href{works/PrataAN23.pdf}{PrataAN23}~\cite{PrataAN23}, \href{works/LacknerMMWW23.pdf}{LacknerMMWW23}~\cite{LacknerMMWW23}, \href{works/LacknerMMWW21.pdf}{LacknerMMWW21}~\cite{LacknerMMWW21}, \href{works/AntuoriHHEN20.pdf}{AntuoriHHEN20}~\cite{AntuoriHHEN20}, \href{works/GroleazNS20.pdf}{GroleazNS20}~\cite{GroleazNS20}, \href{works/GroleazNS20a.pdf}{GroleazNS20a}~\cite{GroleazNS20a}, \href{works/abs-1911-04766.pdf}{abs-1911-04766}~\cite{abs-1911-04766}, \href{works/GeibingerMM19.pdf}{GeibingerMM19}~\cite{GeibingerMM19}, \href{works/HeinzSB13.pdf}{HeinzSB13}~\cite{HeinzSB13}, \href{works/KelbelH11.pdf}{KelbelH11}~\cite{KelbelH11}, \href{works/Laborie09.pdf}{Laborie09}~\cite{Laborie09}, \href{works/Limtanyakul07.pdf}{Limtanyakul07}~\cite{Limtanyakul07}, \href{works/Simonis07.pdf}{Simonis07}~\cite{Simonis07}, \href{works/Hooker06.pdf}{Hooker06}~\cite{Hooker06}, \href{works/Hooker05a.pdf}{Hooker05a}~\cite{Hooker05a}, \href{works/WuBB05.pdf}{WuBB05}~\cite{WuBB05}, \href{works/Sadykov04.pdf}{Sadykov04}~\cite{Sadykov04}, \href{works/SourdN00.pdf}{SourdN00}~\cite{SourdN00}, \href{works/BeckF98.pdf}{BeckF98}~\cite{BeckF98} & \href{works/PovedaAA23.pdf}{PovedaAA23}~\cite{PovedaAA23}, \href{works/IsikYA23.pdf}{IsikYA23}~\cite{IsikYA23}, \href{works/YuraszeckMC23.pdf}{YuraszeckMC23}~\cite{YuraszeckMC23}, \href{works/TouatBT22.pdf}{TouatBT22}~\cite{TouatBT22}, \href{works/PohlAK22.pdf}{PohlAK22}~\cite{PohlAK22}, \href{works/AntuoriHHEN21.pdf}{AntuoriHHEN21}~\cite{AntuoriHHEN21}, \href{works/GeibingerMM21.pdf}{GeibingerMM21}~\cite{GeibingerMM21}, \href{works/ZhangYW21.pdf}{ZhangYW21}~\cite{ZhangYW21}, \href{works/HillTV21.pdf}{HillTV21}~\cite{HillTV21}, \href{works/AbreuAPNM21.pdf}{AbreuAPNM21}~\cite{AbreuAPNM21}, \href{works/KovacsTKSG21.pdf}{KovacsTKSG21}~\cite{KovacsTKSG21}, \href{works/GodetLHS20.pdf}{GodetLHS20}~\cite{GodetLHS20}, \href{works/MejiaY20.pdf}{MejiaY20}~\cite{MejiaY20}, \href{works/Novas19.pdf}{Novas19}~\cite{Novas19}, \href{works/abs-1902-09244.pdf}{abs-1902-09244}~\cite{abs-1902-09244}, \href{works/LaborieRSV18.pdf}{LaborieRSV18}~\cite{LaborieRSV18}, \href{works/Laborie18a.pdf}{Laborie18a}~\cite{Laborie18a}, \href{works/GokgurHO18.pdf}{GokgurHO18}~\cite{GokgurHO18}, \href{works/NattafAL17.pdf}{NattafAL17}~\cite{NattafAL17}, \href{works/NattafAL15.pdf}{NattafAL15}~\cite{NattafAL15}, \href{works/DejemeppeCS15.pdf}{DejemeppeCS15}~\cite{DejemeppeCS15}, \href{works/KoschB14.pdf}{KoschB14}~\cite{KoschB14}, \href{works/TerekhovTDB14.pdf}{TerekhovTDB14}~\cite{TerekhovTDB14}, \href{works/HeinzKB13.pdf}{HeinzKB13}~\cite{HeinzKB13}, \href{works/SchuttFSW13.pdf}{SchuttFSW13}~\cite{SchuttFSW13}, \href{works/BillautHL12.pdf}{BillautHL12}~\cite{BillautHL12}, \href{works/HeinzB12.pdf}{HeinzB12}~\cite{HeinzB12}, \href{works/TranB12.pdf}{TranB12}~\cite{TranB12}, \href{works/GrimesH11.pdf}{GrimesH11}~\cite{GrimesH11}... (Total: 51)\\
Concepts & resource & \href{works/PrataAN23.pdf}{PrataAN23}~\cite{PrataAN23}, \href{works/abs-2402-00459.pdf}{abs-2402-00459}~\cite{abs-2402-00459}, \href{works/JuvinHHL23.pdf}{JuvinHHL23}~\cite{JuvinHHL23}, \href{works/KameugneFND23.pdf}{KameugneFND23}~\cite{KameugneFND23}, \href{works/PovedaAA23.pdf}{PovedaAA23}~\cite{PovedaAA23}, \href{works/YuraszeckMCCR23.pdf}{YuraszeckMCCR23}~\cite{YuraszeckMCCR23}, \href{works/abs-2305-19888.pdf}{abs-2305-19888}~\cite{abs-2305-19888}, \href{works/CzerniachowskaWZ23.pdf}{CzerniachowskaWZ23}~\cite{CzerniachowskaWZ23}, \href{works/ShaikhK23.pdf}{ShaikhK23}~\cite{ShaikhK23}, \href{works/AlfieriGPS23.pdf}{AlfieriGPS23}~\cite{AlfieriGPS23}, \href{works/NaderiRR23.pdf}{NaderiRR23}~\cite{NaderiRR23}, \href{works/AalianPG23.pdf}{AalianPG23}~\cite{AalianPG23}, \href{works/WangB23.pdf}{WangB23}~\cite{WangB23}, \href{works/TardivoDFMP23.pdf}{TardivoDFMP23}~\cite{TardivoDFMP23}, \href{works/GurPAE23.pdf}{GurPAE23}~\cite{GurPAE23}, \href{works/BourreauGGLT22.pdf}{BourreauGGLT22}~\cite{BourreauGGLT22}, \href{works/HeinzNVH22.pdf}{HeinzNVH22}~\cite{HeinzNVH22}, \href{works/GeitzGSSW22.pdf}{GeitzGSSW22}~\cite{GeitzGSSW22}, \href{works/LuoB22.pdf}{LuoB22}~\cite{LuoB22}, \href{works/AbreuN22.pdf}{AbreuN22}~\cite{AbreuN22}, \href{works/BoudreaultSLQ22.pdf}{BoudreaultSLQ22}~\cite{BoudreaultSLQ22}, \href{works/TouatBT22.pdf}{TouatBT22}~\cite{TouatBT22}, \href{works/YunusogluY22.pdf}{YunusogluY22}~\cite{YunusogluY22}, \href{works/CampeauG22.pdf}{CampeauG22}~\cite{CampeauG22}, \href{works/SubulanC22.pdf}{SubulanC22}~\cite{SubulanC22}, \href{works/OuelletQ22.pdf}{OuelletQ22}~\cite{OuelletQ22}, \href{works/FarsiTM22.pdf}{FarsiTM22}~\cite{FarsiTM22}, \href{works/ColT22.pdf}{ColT22}~\cite{ColT22}, \href{works/OujanaAYB22.pdf}{OujanaAYB22}~\cite{OujanaAYB22}... (Total: 302) & \href{works/Caballero23.pdf}{Caballero23}~\cite{Caballero23}, \href{works/PerezGSL23.pdf}{PerezGSL23}~\cite{PerezGSL23}, \href{works/abs-2312-13682.pdf}{abs-2312-13682}~\cite{abs-2312-13682}, \href{works/IsikYA23.pdf}{IsikYA23}~\cite{IsikYA23}, \href{works/abs-2306-05747.pdf}{abs-2306-05747}~\cite{abs-2306-05747}, \href{works/TasselGS23.pdf}{TasselGS23}~\cite{TasselGS23}, \href{works/Bit-Monnot23.pdf}{Bit-Monnot23}~\cite{Bit-Monnot23}, \href{works/AbreuNP23.pdf}{AbreuNP23}~\cite{AbreuNP23}, \href{works/abs-2211-14492.pdf}{abs-2211-14492}~\cite{abs-2211-14492}, \href{works/PohlAK22.pdf}{PohlAK22}~\cite{PohlAK22}, \href{works/YuraszeckMPV22.pdf}{YuraszeckMPV22}~\cite{YuraszeckMPV22}, \href{works/MullerMKP22.pdf}{MullerMKP22}~\cite{MullerMKP22}, \href{works/WinterMMW22.pdf}{WinterMMW22}~\cite{WinterMMW22}, \href{works/SvancaraB22.pdf}{SvancaraB22}~\cite{SvancaraB22}, \href{works/Astrand0F21.pdf}{Astrand0F21}~\cite{Astrand0F21}, \href{works/KlankeBYE21.pdf}{KlankeBYE21}~\cite{KlankeBYE21}, \href{works/MokhtarzadehTNF20.pdf}{MokhtarzadehTNF20}~\cite{MokhtarzadehTNF20}, \href{works/TangB20.pdf}{TangB20}~\cite{TangB20}, \href{works/LunardiBLRV20.pdf}{LunardiBLRV20}~\cite{LunardiBLRV20}, \href{works/WallaceY20.pdf}{WallaceY20}~\cite{WallaceY20}, \href{works/FrimodigS19.pdf}{FrimodigS19}~\cite{FrimodigS19}, \href{works/abs-1902-01193.pdf}{abs-1902-01193}~\cite{abs-1902-01193}, \href{works/ParkUJR19.pdf}{ParkUJR19}~\cite{ParkUJR19}, \href{works/HoYCLLCLC18.pdf}{HoYCLLCLC18}~\cite{HoYCLLCLC18}, \href{works/GedikKEK18.pdf}{GedikKEK18}~\cite{GedikKEK18}, \href{works/Ham18.pdf}{Ham18}~\cite{Ham18}, \href{works/BenediktSMVH18.pdf}{BenediktSMVH18}~\cite{BenediktSMVH18}, \href{works/GelainPRVW17.pdf}{GelainPRVW17}~\cite{GelainPRVW17}, \href{works/GoldwaserS17.pdf}{GoldwaserS17}~\cite{GoldwaserS17}... (Total: 56) & \href{works/MontemanniD23.pdf}{MontemanniD23}~\cite{MontemanniD23}, \href{works/AkramNHRSA23.pdf}{AkramNHRSA23}~\cite{AkramNHRSA23}, \href{works/SquillaciPR23.pdf}{SquillaciPR23}~\cite{SquillaciPR23}, \href{works/Teppan22.pdf}{Teppan22}~\cite{Teppan22}, \href{works/PopovicCGNC22.pdf}{PopovicCGNC22}~\cite{PopovicCGNC22}, \href{works/ArmstrongGOS22.pdf}{ArmstrongGOS22}~\cite{ArmstrongGOS22}, \href{works/JungblutK22.pdf}{JungblutK22}~\cite{JungblutK22}, \href{works/ZhangJZL22.pdf}{ZhangJZL22}~\cite{ZhangJZL22}, \href{works/AntuoriHHEN21.pdf}{AntuoriHHEN21}~\cite{AntuoriHHEN21}, \href{works/HamPK21.pdf}{HamPK21}~\cite{HamPK21}, \href{works/AbreuAPNM21.pdf}{AbreuAPNM21}~\cite{AbreuAPNM21}, \href{works/AbohashimaEG21.pdf}{AbohashimaEG21}~\cite{AbohashimaEG21}, \href{works/KoehlerBFFHPSSS21.pdf}{KoehlerBFFHPSSS21}~\cite{KoehlerBFFHPSSS21}, \href{works/ArmstrongGOS21.pdf}{ArmstrongGOS21}~\cite{ArmstrongGOS21}, \href{works/FanXG21.pdf}{FanXG21}~\cite{FanXG21}, \href{works/abs-2102-08778.pdf}{abs-2102-08778}~\cite{abs-2102-08778}, \href{works/MejiaY20.pdf}{MejiaY20}~\cite{MejiaY20}, \href{works/BarzegaranZP20.pdf}{BarzegaranZP20}~\cite{BarzegaranZP20}, \href{works/NattafM20.pdf}{NattafM20}~\cite{NattafM20}, \href{works/BadicaBIL19.pdf}{BadicaBIL19}~\cite{BadicaBIL19}, \href{works/KucukY19.pdf}{KucukY19}~\cite{KucukY19}, \href{works/ColT19.pdf}{ColT19}~\cite{ColT19}, \href{works/AstrandJZ18.pdf}{AstrandJZ18}~\cite{AstrandJZ18}, \href{works/ZhangW18.pdf}{ZhangW18}~\cite{ZhangW18}, \href{works/KletzanderM17.pdf}{KletzanderM17}~\cite{KletzanderM17}, \href{works/Hooker17.pdf}{Hooker17}~\cite{Hooker17}, \href{works/TranVNB17a.pdf}{TranVNB17a}~\cite{TranVNB17a}, \href{works/ZarandiKS16.pdf}{ZarandiKS16}~\cite{ZarandiKS16}, \href{works/GayHLS15.pdf}{GayHLS15}~\cite{GayHLS15}... (Total: 55)\\
Concepts & scheduling & \href{works/abs-2402-00459.pdf}{abs-2402-00459}~\cite{abs-2402-00459}, \href{works/PrataAN23.pdf}{PrataAN23}~\cite{PrataAN23}, \href{works/AbreuNP23.pdf}{AbreuNP23}~\cite{AbreuNP23}, \href{works/TasselGS23.pdf}{TasselGS23}~\cite{TasselGS23}, \href{works/Bit-Monnot23.pdf}{Bit-Monnot23}~\cite{Bit-Monnot23}, \href{works/IsikYA23.pdf}{IsikYA23}~\cite{IsikYA23}, \href{works/AalianPG23.pdf}{AalianPG23}~\cite{AalianPG23}, \href{works/abs-2305-19888.pdf}{abs-2305-19888}~\cite{abs-2305-19888}, \href{works/abs-2312-13682.pdf}{abs-2312-13682}~\cite{abs-2312-13682}, \href{works/PerezGSL23.pdf}{PerezGSL23}~\cite{PerezGSL23}, \href{works/abs-2306-05747.pdf}{abs-2306-05747}~\cite{abs-2306-05747}, \href{works/JuvinHHL23.pdf}{JuvinHHL23}~\cite{JuvinHHL23}, \href{works/TardivoDFMP23.pdf}{TardivoDFMP23}~\cite{TardivoDFMP23}, \href{works/YuraszeckMC23.pdf}{YuraszeckMC23}~\cite{YuraszeckMC23}, \href{works/Mehdizadeh-Somarin23.pdf}{Mehdizadeh-Somarin23}~\cite{Mehdizadeh-Somarin23}, \href{works/MontemanniD23.pdf}{MontemanniD23}~\cite{MontemanniD23}, \href{works/KimCMLLP23.pdf}{KimCMLLP23}~\cite{KimCMLLP23}, \href{works/AkramNHRSA23.pdf}{AkramNHRSA23}~\cite{AkramNHRSA23}, \href{works/ShaikhK23.pdf}{ShaikhK23}~\cite{ShaikhK23}, \href{works/KameugneFND23.pdf}{KameugneFND23}~\cite{KameugneFND23}, \href{works/LacknerMMWW23.pdf}{LacknerMMWW23}~\cite{LacknerMMWW23}, \href{works/GurPAE23.pdf}{GurPAE23}~\cite{GurPAE23}, \href{works/PovedaAA23.pdf}{PovedaAA23}~\cite{PovedaAA23}, \href{works/EfthymiouY23.pdf}{EfthymiouY23}~\cite{EfthymiouY23}, \href{works/AlfieriGPS23.pdf}{AlfieriGPS23}~\cite{AlfieriGPS23}, \href{works/SquillaciPR23.pdf}{SquillaciPR23}~\cite{SquillaciPR23}, \href{works/Caballero23.pdf}{Caballero23}~\cite{Caballero23}, \href{works/CzerniachowskaWZ23.pdf}{CzerniachowskaWZ23}~\cite{CzerniachowskaWZ23}, \href{works/YuraszeckMCCR23.pdf}{YuraszeckMCCR23}~\cite{YuraszeckMCCR23}... (Total: 440) & \href{works/HebrardALLCMR22.pdf}{HebrardALLCMR22}~\cite{HebrardALLCMR22}, \href{works/GayHS15.pdf}{GayHS15}~\cite{GayHS15}, \href{works/Kameugne15.pdf}{Kameugne15}~\cite{Kameugne15}, \href{works/BessiereHMQW14.pdf}{BessiereHMQW14}~\cite{BessiereHMQW14}, \href{works/HoundjiSWD14.pdf}{HoundjiSWD14}~\cite{HoundjiSWD14}, \href{works/LetortCB13.pdf}{LetortCB13}~\cite{LetortCB13}, \href{works/LetortBC12.pdf}{LetortBC12}~\cite{LetortBC12}, \href{works/ChapadosJR11.pdf}{ChapadosJR11}~\cite{ChapadosJR11}, \href{works/ClercqPBJ11.pdf}{ClercqPBJ11}~\cite{ClercqPBJ11}, \href{works/Baptiste09.pdf}{Baptiste09}~\cite{Baptiste09}, \href{works/Acuna-AgostMFG09.pdf}{Acuna-AgostMFG09}~\cite{Acuna-AgostMFG09}, \href{works/abs-0907-0939.pdf}{abs-0907-0939}~\cite{abs-0907-0939}, \href{works/GomesHS06.pdf}{GomesHS06}~\cite{GomesHS06}, \href{works/MoffittPP05.pdf}{MoffittPP05}~\cite{MoffittPP05}, \href{works/WuBB05.pdf}{WuBB05}~\cite{WuBB05}, \href{works/DilkinaDH05.pdf}{DilkinaDH05}~\cite{DilkinaDH05}, \href{works/HebrardTW05.pdf}{HebrardTW05}~\cite{HebrardTW05}, \href{works/Vilim03.pdf}{Vilim03}~\cite{Vilim03}, \href{works/ValleMGT03.pdf}{ValleMGT03}~\cite{ValleMGT03}, \href{works/Vilim02.pdf}{Vilim02}~\cite{Vilim02}, \href{works/HookerY02.pdf}{HookerY02}~\cite{HookerY02}, \href{works/RodriguezDG02.pdf}{RodriguezDG02}~\cite{RodriguezDG02}, \href{works/CestaOS98.pdf}{CestaOS98}~\cite{CestaOS98}, \href{works/FrostD98.pdf}{FrostD98}~\cite{FrostD98}, \href{works/Touraivane95.pdf}{Touraivane95}~\cite{Touraivane95} & \href{works/Hooker17.pdf}{Hooker17}~\cite{Hooker17}, \href{works/RossiTHP07.pdf}{RossiTHP07}~\cite{RossiTHP07}, \href{works/AbrilSB05.pdf}{AbrilSB05}~\cite{AbrilSB05}, \href{works/VanczaM01.pdf}{VanczaM01}~\cite{VanczaM01}\\
Concepts & sequence dependent setup & \href{works/GedikKEK18.pdf}{GedikKEK18}~\cite{GedikKEK18}, \href{works/TranB12.pdf}{TranB12}~\cite{TranB12}, \href{works/FocacciLN00.pdf}{FocacciLN00}~\cite{FocacciLN00} & \href{works/IsikYA23.pdf}{IsikYA23}~\cite{IsikYA23}, \href{works/YuraszeckMPV22.pdf}{YuraszeckMPV22}~\cite{YuraszeckMPV22}, \href{works/GeitzGSSW22.pdf}{GeitzGSSW22}~\cite{GeitzGSSW22}, \href{works/MengZRZL20.pdf}{MengZRZL20}~\cite{MengZRZL20}, \href{works/RiahiNS018.pdf}{RiahiNS018}~\cite{RiahiNS018}, \href{works/LombardiM12.pdf}{LombardiM12}~\cite{LombardiM12}, \href{works/Simonis07.pdf}{Simonis07}~\cite{Simonis07}, \href{works/ArtiguesBF04.pdf}{ArtiguesBF04}~\cite{ArtiguesBF04} & \href{works/PrataAN23.pdf}{PrataAN23}~\cite{PrataAN23}, \href{works/NaderiRR23.pdf}{NaderiRR23}~\cite{NaderiRR23}, \href{works/abs-2305-19888.pdf}{abs-2305-19888}~\cite{abs-2305-19888}, \href{works/YunusogluY22.pdf}{YunusogluY22}~\cite{YunusogluY22}, \href{works/PohlAK22.pdf}{PohlAK22}~\cite{PohlAK22}, \href{works/HeinzNVH22.pdf}{HeinzNVH22}~\cite{HeinzNVH22}, \href{works/OujanaAYB22.pdf}{OujanaAYB22}~\cite{OujanaAYB22}, \href{works/Bedhief21.pdf}{Bedhief21}~\cite{Bedhief21}, \href{works/HamPK21.pdf}{HamPK21}~\cite{HamPK21}, \href{works/ArmstrongGOS21.pdf}{ArmstrongGOS21}~\cite{ArmstrongGOS21}, \href{works/Mercier-AubinGQ20.pdf}{Mercier-AubinGQ20}~\cite{Mercier-AubinGQ20}, \href{works/MejiaY20.pdf}{MejiaY20}~\cite{MejiaY20}, \href{works/MalapertN19.pdf}{MalapertN19}~\cite{MalapertN19}, \href{works/Novas19.pdf}{Novas19}~\cite{Novas19}, \href{works/KucukY19.pdf}{KucukY19}~\cite{KucukY19}, \href{works/ArbaouiY18.pdf}{ArbaouiY18}~\cite{ArbaouiY18}, \href{works/LaborieRSV18.pdf}{LaborieRSV18}~\cite{LaborieRSV18}, \href{works/Ham18.pdf}{Ham18}~\cite{Ham18}, \href{works/FahimiOQ18.pdf}{FahimiOQ18}~\cite{FahimiOQ18}, \href{works/Pralet17.pdf}{Pralet17}~\cite{Pralet17}, \href{works/CauwelaertDMS16.pdf}{CauwelaertDMS16}~\cite{CauwelaertDMS16}, \href{works/NovaraNH16.pdf}{NovaraNH16}~\cite{NovaraNH16}, \href{works/DejemeppeCS15.pdf}{DejemeppeCS15}~\cite{DejemeppeCS15}, \href{works/BajestaniB15.pdf}{BajestaniB15}~\cite{BajestaniB15}, \href{works/KovacsK11.pdf}{KovacsK11}~\cite{KovacsK11}, \href{works/GrimesH10.pdf}{GrimesH10}~\cite{GrimesH10}, \href{works/Laborie09.pdf}{Laborie09}~\cite{Laborie09}, \href{works/DavenportKRSH07.pdf}{DavenportKRSH07}~\cite{DavenportKRSH07}, \href{works/AkkerDH07.pdf}{AkkerDH07}~\cite{AkkerDH07}... (Total: 32)\\
Concepts & setup-time & \href{works/PrataAN23.pdf}{PrataAN23}~\cite{PrataAN23}, \href{works/LacknerMMWW23.pdf}{LacknerMMWW23}~\cite{LacknerMMWW23}, \href{works/IsikYA23.pdf}{IsikYA23}~\cite{IsikYA23}, \href{works/abs-2305-19888.pdf}{abs-2305-19888}~\cite{abs-2305-19888}, \href{works/AbreuNP23.pdf}{AbreuNP23}~\cite{AbreuNP23}, \href{works/NaderiRR23.pdf}{NaderiRR23}~\cite{NaderiRR23}, \href{works/YuraszeckMPV22.pdf}{YuraszeckMPV22}~\cite{YuraszeckMPV22}, \href{works/PohlAK22.pdf}{PohlAK22}~\cite{PohlAK22}, \href{works/GeitzGSSW22.pdf}{GeitzGSSW22}~\cite{GeitzGSSW22}, \href{works/WinterMMW22.pdf}{WinterMMW22}~\cite{WinterMMW22}, \href{works/HeinzNVH22.pdf}{HeinzNVH22}~\cite{HeinzNVH22}, \href{works/AbreuN22.pdf}{AbreuN22}~\cite{AbreuN22}, \href{works/OujanaAYB22.pdf}{OujanaAYB22}~\cite{OujanaAYB22}, \href{works/YunusogluY22.pdf}{YunusogluY22}~\cite{YunusogluY22}, \href{works/ColT22.pdf}{ColT22}~\cite{ColT22}, \href{works/LacknerMMWW21.pdf}{LacknerMMWW21}~\cite{LacknerMMWW21}, \href{works/NattafM20.pdf}{NattafM20}~\cite{NattafM20}, \href{works/MejiaY20.pdf}{MejiaY20}~\cite{MejiaY20}, \href{works/GroleazNS20.pdf}{GroleazNS20}~\cite{GroleazNS20}, \href{works/Mercier-AubinGQ20.pdf}{Mercier-AubinGQ20}~\cite{Mercier-AubinGQ20}, \href{works/QinDCS20.pdf}{QinDCS20}~\cite{QinDCS20}, \href{works/LunardiBLRV20.pdf}{LunardiBLRV20}~\cite{LunardiBLRV20}, \href{works/GroleazNS20a.pdf}{GroleazNS20a}~\cite{GroleazNS20a}, \href{works/MengZRZL20.pdf}{MengZRZL20}~\cite{MengZRZL20}, \href{works/Novas19.pdf}{Novas19}~\cite{Novas19}, \href{works/BogaerdtW19.pdf}{BogaerdtW19}~\cite{BogaerdtW19}, \href{works/MalapertN19.pdf}{MalapertN19}~\cite{MalapertN19}, \href{works/MurinR19.pdf}{MurinR19}~\cite{MurinR19}, \href{works/ArbaouiY18.pdf}{ArbaouiY18}~\cite{ArbaouiY18}... (Total: 43) & \href{works/AlfieriGPS23.pdf}{AlfieriGPS23}~\cite{AlfieriGPS23}, \href{works/CzerniachowskaWZ23.pdf}{CzerniachowskaWZ23}~\cite{CzerniachowskaWZ23}, \href{works/KimCMLLP23.pdf}{KimCMLLP23}~\cite{KimCMLLP23}, \href{works/LiFJZLL22.pdf}{LiFJZLL22}~\cite{LiFJZLL22}, \href{works/Bedhief21.pdf}{Bedhief21}~\cite{Bedhief21}, \href{works/AbreuAPNM21.pdf}{AbreuAPNM21}~\cite{AbreuAPNM21}, \href{works/ArmstrongGOS21.pdf}{ArmstrongGOS21}~\cite{ArmstrongGOS21}, \href{works/FanXG21.pdf}{FanXG21}~\cite{FanXG21}, \href{works/AstrandJZ20.pdf}{AstrandJZ20}~\cite{AstrandJZ20}, \href{works/LaborieRSV18.pdf}{LaborieRSV18}~\cite{LaborieRSV18}, \href{works/NovaraNH16.pdf}{NovaraNH16}~\cite{NovaraNH16}, \href{works/GaySS14.pdf}{GaySS14}~\cite{GaySS14}, \href{works/OzturkTHO13.pdf}{OzturkTHO13}~\cite{OzturkTHO13}, \href{works/KelarevaTK13.pdf}{KelarevaTK13}~\cite{KelarevaTK13}, \href{works/ThiruvadyBME09.pdf}{ThiruvadyBME09}~\cite{ThiruvadyBME09}, \href{works/BeniniBGM06.pdf}{BeniniBGM06}~\cite{BeniniBGM06}, \href{works/Timpe02.pdf}{Timpe02}~\cite{Timpe02}, \href{works/Vilim02.pdf}{Vilim02}~\cite{Vilim02} & \href{works/YuraszeckMCCR23.pdf}{YuraszeckMCCR23}~\cite{YuraszeckMCCR23}, \href{works/JuvinHHL23.pdf}{JuvinHHL23}~\cite{JuvinHHL23}, \href{works/JuvinHL23.pdf}{JuvinHL23}~\cite{JuvinHL23}, \href{works/Mehdizadeh-Somarin23.pdf}{Mehdizadeh-Somarin23}~\cite{Mehdizadeh-Somarin23}, \href{works/EfthymiouY23.pdf}{EfthymiouY23}~\cite{EfthymiouY23}, \href{works/abs-2211-14492.pdf}{abs-2211-14492}~\cite{abs-2211-14492}, \href{works/ZhangJZL22.pdf}{ZhangJZL22}~\cite{ZhangJZL22}, \href{works/MullerMKP22.pdf}{MullerMKP22}~\cite{MullerMKP22}, \href{works/Teppan22.pdf}{Teppan22}~\cite{Teppan22}, \href{works/HamPK21.pdf}{HamPK21}~\cite{HamPK21}, \href{works/ZhangYW21.pdf}{ZhangYW21}~\cite{ZhangYW21}, \href{works/AbohashimaEG21.pdf}{AbohashimaEG21}~\cite{AbohashimaEG21}, \href{works/BenderWS21.pdf}{BenderWS21}~\cite{BenderWS21}, \href{works/GodetLHS20.pdf}{GodetLHS20}~\cite{GodetLHS20}, \href{works/MokhtarzadehTNF20.pdf}{MokhtarzadehTNF20}~\cite{MokhtarzadehTNF20}, \href{works/Polo-MejiaALB20.pdf}{Polo-MejiaALB20}~\cite{Polo-MejiaALB20}, \href{works/BehrensLM19.pdf}{BehrensLM19}~\cite{BehrensLM19}, \href{works/abs-1902-09244.pdf}{abs-1902-09244}~\cite{abs-1902-09244}, \href{works/KucukY19.pdf}{KucukY19}~\cite{KucukY19}, \href{works/WikarekS19.pdf}{WikarekS19}~\cite{WikarekS19}, \href{works/GokgurHO18.pdf}{GokgurHO18}~\cite{GokgurHO18}, \href{works/FahimiOQ18.pdf}{FahimiOQ18}~\cite{FahimiOQ18}, \href{works/TranVNB17a.pdf}{TranVNB17a}~\cite{TranVNB17a}, \href{works/GilesH16.pdf}{GilesH16}~\cite{GilesH16}, \href{works/ZhouGL15.pdf}{ZhouGL15}~\cite{ZhouGL15}, \href{works/MelgarejoLS15.pdf}{MelgarejoLS15}~\cite{MelgarejoLS15}, \href{works/GoelSHFS15.pdf}{GoelSHFS15}~\cite{GoelSHFS15}, \href{works/SialaAH15.pdf}{SialaAH15}~\cite{SialaAH15}, \href{works/BartakV15.pdf}{BartakV15}~\cite{BartakV15}... (Total: 51)\\
Concepts & stock level & \href{works/LopesCSM10.pdf}{LopesCSM10}~\cite{LopesCSM10}, \href{works/SimonisC95.pdf}{SimonisC95}~\cite{SimonisC95} & \href{works/RossiTHP07.pdf}{RossiTHP07}~\cite{RossiTHP07}, \href{works/Timpe02.pdf}{Timpe02}~\cite{Timpe02} & \href{works/KhemmoudjPB06.pdf}{KhemmoudjPB06}~\cite{KhemmoudjPB06}\\
Concepts & tardiness & \href{works/PrataAN23.pdf}{PrataAN23}~\cite{PrataAN23}, \href{works/IsikYA23.pdf}{IsikYA23}~\cite{IsikYA23}, \href{works/AlfieriGPS23.pdf}{AlfieriGPS23}~\cite{AlfieriGPS23}, \href{works/KimCMLLP23.pdf}{KimCMLLP23}~\cite{KimCMLLP23}, \href{works/LacknerMMWW23.pdf}{LacknerMMWW23}~\cite{LacknerMMWW23}, \href{works/NaderiRR23.pdf}{NaderiRR23}~\cite{NaderiRR23}, \href{works/WinterMMW22.pdf}{WinterMMW22}~\cite{WinterMMW22}, \href{works/TouatBT22.pdf}{TouatBT22}~\cite{TouatBT22}, \href{works/YunusogluY22.pdf}{YunusogluY22}~\cite{YunusogluY22}, \href{works/AbreuN22.pdf}{AbreuN22}~\cite{AbreuN22}, \href{works/OujanaAYB22.pdf}{OujanaAYB22}~\cite{OujanaAYB22}, \href{works/PohlAK22.pdf}{PohlAK22}~\cite{PohlAK22}, \href{works/abs-2211-14492.pdf}{abs-2211-14492}~\cite{abs-2211-14492}, \href{works/FanXG21.pdf}{FanXG21}~\cite{FanXG21}, \href{works/AntuoriHHEN21.pdf}{AntuoriHHEN21}~\cite{AntuoriHHEN21}, \href{works/LacknerMMWW21.pdf}{LacknerMMWW21}~\cite{LacknerMMWW21}, \href{works/GroleazNS20a.pdf}{GroleazNS20a}~\cite{GroleazNS20a}, \href{works/Mercier-AubinGQ20.pdf}{Mercier-AubinGQ20}~\cite{Mercier-AubinGQ20}, \href{works/AntuoriHHEN20.pdf}{AntuoriHHEN20}~\cite{AntuoriHHEN20}, \href{works/MengZRZL20.pdf}{MengZRZL20}~\cite{MengZRZL20}, \href{works/TangB20.pdf}{TangB20}~\cite{TangB20}, \href{works/abs-1902-09244.pdf}{abs-1902-09244}~\cite{abs-1902-09244}, \href{works/ParkUJR19.pdf}{ParkUJR19}~\cite{ParkUJR19}, \href{works/BogaerdtW19.pdf}{BogaerdtW19}~\cite{BogaerdtW19}, \href{works/LaborieRSV18.pdf}{LaborieRSV18}~\cite{LaborieRSV18}, \href{works/NovaraNH16.pdf}{NovaraNH16}~\cite{NovaraNH16}, \href{works/ZarandiKS16.pdf}{ZarandiKS16}~\cite{ZarandiKS16}, \href{works/BridiBLMB16.pdf}{BridiBLMB16}~\cite{BridiBLMB16}, \href{works/BartoliniBBLM14.pdf}{BartoliniBBLM14}~\cite{BartoliniBBLM14}... (Total: 46) & \href{works/abs-2402-00459.pdf}{abs-2402-00459}~\cite{abs-2402-00459}, \href{works/AbreuNP23.pdf}{AbreuNP23}~\cite{AbreuNP23}, \href{works/SubulanC22.pdf}{SubulanC22}~\cite{SubulanC22}, \href{works/FarsiTM22.pdf}{FarsiTM22}~\cite{FarsiTM22}, \href{works/ColT22.pdf}{ColT22}~\cite{ColT22}, \href{works/KovacsTKSG21.pdf}{KovacsTKSG21}~\cite{KovacsTKSG21}, \href{works/AbreuAPNM21.pdf}{AbreuAPNM21}~\cite{AbreuAPNM21}, \href{works/GroleazNS20.pdf}{GroleazNS20}~\cite{GroleazNS20}, \href{works/GedikKEK18.pdf}{GedikKEK18}~\cite{GedikKEK18}, \href{works/GokgurHO18.pdf}{GokgurHO18}~\cite{GokgurHO18}, \href{works/Hooker17.pdf}{Hooker17}~\cite{Hooker17}, \href{works/ThiruvadyWGS14.pdf}{ThiruvadyWGS14}~\cite{ThiruvadyWGS14}, \href{works/TerekhovTDB14.pdf}{TerekhovTDB14}~\cite{TerekhovTDB14}, \href{works/BajestaniB13.pdf}{BajestaniB13}~\cite{BajestaniB13}, \href{works/NovasH10.pdf}{NovasH10}~\cite{NovasH10}, \href{works/BartakSR10.pdf}{BartakSR10}~\cite{BartakSR10}, \href{works/QuirogaZH05.pdf}{QuirogaZH05}~\cite{QuirogaZH05}, \href{works/Hooker05.pdf}{Hooker05}~\cite{Hooker05}, \href{works/GodardLN05.pdf}{GodardLN05}~\cite{GodardLN05} & \href{works/Mehdizadeh-Somarin23.pdf}{Mehdizadeh-Somarin23}~\cite{Mehdizadeh-Somarin23}, \href{works/JuvinHL23.pdf}{JuvinHL23}~\cite{JuvinHL23}, \href{works/abs-2306-05747.pdf}{abs-2306-05747}~\cite{abs-2306-05747}, \href{works/TasselGS23.pdf}{TasselGS23}~\cite{TasselGS23}, \href{works/LiFJZLL22.pdf}{LiFJZLL22}~\cite{LiFJZLL22}, \href{works/ZhangJZL22.pdf}{ZhangJZL22}~\cite{ZhangJZL22}, \href{works/VlkHT21.pdf}{VlkHT21}~\cite{VlkHT21}, \href{works/HanenKP21.pdf}{HanenKP21}~\cite{HanenKP21}, \href{works/KoehlerBFFHPSSS21.pdf}{KoehlerBFFHPSSS21}~\cite{KoehlerBFFHPSSS21}, \href{works/HamPK21.pdf}{HamPK21}~\cite{HamPK21}, \href{works/GeibingerMM21.pdf}{GeibingerMM21}~\cite{GeibingerMM21}, \href{works/HubnerGSV21.pdf}{HubnerGSV21}~\cite{HubnerGSV21}, \href{works/QinWSLS21.pdf}{QinWSLS21}~\cite{QinWSLS21}, \href{works/Bedhief21.pdf}{Bedhief21}~\cite{Bedhief21}, \href{works/QinDCS20.pdf}{QinDCS20}~\cite{QinDCS20}, \href{works/Polo-MejiaALB20.pdf}{Polo-MejiaALB20}~\cite{Polo-MejiaALB20}, \href{works/MejiaY20.pdf}{MejiaY20}~\cite{MejiaY20}, \href{works/LunardiBLRV20.pdf}{LunardiBLRV20}~\cite{LunardiBLRV20}, \href{works/Tom19.pdf}{Tom19}~\cite{Tom19}, \href{works/Novas19.pdf}{Novas19}~\cite{Novas19}, \href{works/RiahiNS018.pdf}{RiahiNS018}~\cite{RiahiNS018}, \href{works/ZhangW18.pdf}{ZhangW18}~\cite{ZhangW18}, \href{works/DejemeppeCS15.pdf}{DejemeppeCS15}~\cite{DejemeppeCS15}, \href{works/MelgarejoLS15.pdf}{MelgarejoLS15}~\cite{MelgarejoLS15}, \href{works/ZhouGL15.pdf}{ZhouGL15}~\cite{ZhouGL15}, \href{works/BurtLPS15.pdf}{BurtLPS15}~\cite{BurtLPS15}, \href{works/LimBTBB15.pdf}{LimBTBB15}~\cite{LimBTBB15}, \href{works/SialaAH15.pdf}{SialaAH15}~\cite{SialaAH15}, \href{works/PraletLJ15.pdf}{PraletLJ15}~\cite{PraletLJ15}... (Total: 53)\\
Concepts & task & \href{works/PrataAN23.pdf}{PrataAN23}~\cite{PrataAN23}, \href{works/abs-2402-00459.pdf}{abs-2402-00459}~\cite{abs-2402-00459}, \href{works/JuvinHL23.pdf}{JuvinHL23}~\cite{JuvinHL23}, \href{works/CzerniachowskaWZ23.pdf}{CzerniachowskaWZ23}~\cite{CzerniachowskaWZ23}, \href{works/JuvinHHL23.pdf}{JuvinHHL23}~\cite{JuvinHHL23}, \href{works/WangB23.pdf}{WangB23}~\cite{WangB23}, \href{works/YuraszeckMCCR23.pdf}{YuraszeckMCCR23}~\cite{YuraszeckMCCR23}, \href{works/PovedaAA23.pdf}{PovedaAA23}~\cite{PovedaAA23}, \href{works/abs-2305-19888.pdf}{abs-2305-19888}~\cite{abs-2305-19888}, \href{works/KameugneFND23.pdf}{KameugneFND23}~\cite{KameugneFND23}, \href{works/AkramNHRSA23.pdf}{AkramNHRSA23}~\cite{AkramNHRSA23}, \href{works/LiFJZLL22.pdf}{LiFJZLL22}~\cite{LiFJZLL22}, \href{works/CampeauG22.pdf}{CampeauG22}~\cite{CampeauG22}, \href{works/ColT22.pdf}{ColT22}~\cite{ColT22}, \href{works/SubulanC22.pdf}{SubulanC22}~\cite{SubulanC22}, \href{works/OuelletQ22.pdf}{OuelletQ22}~\cite{OuelletQ22}, \href{works/FetgoD22.pdf}{FetgoD22}~\cite{FetgoD22}, \href{works/abs-2211-14492.pdf}{abs-2211-14492}~\cite{abs-2211-14492}, \href{works/GeitzGSSW22.pdf}{GeitzGSSW22}~\cite{GeitzGSSW22}, \href{works/TouatBT22.pdf}{TouatBT22}~\cite{TouatBT22}, \href{works/HeinzNVH22.pdf}{HeinzNVH22}~\cite{HeinzNVH22}, \href{works/JungblutK22.pdf}{JungblutK22}~\cite{JungblutK22}, \href{works/BoudreaultSLQ22.pdf}{BoudreaultSLQ22}~\cite{BoudreaultSLQ22}, \href{works/Astrand0F21.pdf}{Astrand0F21}~\cite{Astrand0F21}, \href{works/HanenKP21.pdf}{HanenKP21}~\cite{HanenKP21}, \href{works/KoehlerBFFHPSSS21.pdf}{KoehlerBFFHPSSS21}~\cite{KoehlerBFFHPSSS21}, \href{works/KlankeBYE21.pdf}{KlankeBYE21}~\cite{KlankeBYE21}, \href{works/HillTV21.pdf}{HillTV21}~\cite{HillTV21}, \href{works/PandeyS21a.pdf}{PandeyS21a}~\cite{PandeyS21a}... (Total: 199) & \href{works/MontemanniD23a.pdf}{MontemanniD23a}~\cite{MontemanniD23a}, \href{works/Bit-Monnot23.pdf}{Bit-Monnot23}~\cite{Bit-Monnot23}, \href{works/IsikYA23.pdf}{IsikYA23}~\cite{IsikYA23}, \href{works/MontemanniD23.pdf}{MontemanniD23}~\cite{MontemanniD23}, \href{works/LacknerMMWW23.pdf}{LacknerMMWW23}~\cite{LacknerMMWW23}, \href{works/ShaikhK23.pdf}{ShaikhK23}~\cite{ShaikhK23}, \href{works/SquillaciPR23.pdf}{SquillaciPR23}~\cite{SquillaciPR23}, \href{works/YuraszeckMPV22.pdf}{YuraszeckMPV22}~\cite{YuraszeckMPV22}, \href{works/PopovicCGNC22.pdf}{PopovicCGNC22}~\cite{PopovicCGNC22}, \href{works/MullerMKP22.pdf}{MullerMKP22}~\cite{MullerMKP22}, \href{works/WinterMMW22.pdf}{WinterMMW22}~\cite{WinterMMW22}, \href{works/AbreuN22.pdf}{AbreuN22}~\cite{AbreuN22}, \href{works/FarsiTM22.pdf}{FarsiTM22}~\cite{FarsiTM22}, \href{works/SvancaraB22.pdf}{SvancaraB22}~\cite{SvancaraB22}, \href{works/OujanaAYB22.pdf}{OujanaAYB22}~\cite{OujanaAYB22}, \href{works/BenderWS21.pdf}{BenderWS21}~\cite{BenderWS21}, \href{works/HubnerGSV21.pdf}{HubnerGSV21}~\cite{HubnerGSV21}, \href{works/GeibingerMM21.pdf}{GeibingerMM21}~\cite{GeibingerMM21}, \href{works/ZouZ20.pdf}{ZouZ20}~\cite{ZouZ20}, \href{works/BarzegaranZP20.pdf}{BarzegaranZP20}~\cite{BarzegaranZP20}, \href{works/Polo-MejiaALB20.pdf}{Polo-MejiaALB20}~\cite{Polo-MejiaALB20}, \href{works/AntuoriHHEN20.pdf}{AntuoriHHEN20}~\cite{AntuoriHHEN20}, \href{works/BadicaBI20.pdf}{BadicaBI20}~\cite{BadicaBI20}, \href{works/WallaceY20.pdf}{WallaceY20}~\cite{WallaceY20}, \href{works/WikarekS19.pdf}{WikarekS19}~\cite{WikarekS19}, \href{works/DemirovicS18.pdf}{DemirovicS18}~\cite{DemirovicS18}, \href{works/GoldwaserS18.pdf}{GoldwaserS18}~\cite{GoldwaserS18}, \href{works/MusliuSS18.pdf}{MusliuSS18}~\cite{MusliuSS18}, \href{works/YoungFS17.pdf}{YoungFS17}~\cite{YoungFS17}... (Total: 50) & \href{works/NaderiRR23.pdf}{NaderiRR23}~\cite{NaderiRR23}, \href{works/TasselGS23.pdf}{TasselGS23}~\cite{TasselGS23}, \href{works/EfthymiouY23.pdf}{EfthymiouY23}~\cite{EfthymiouY23}, \href{works/PerezGSL23.pdf}{PerezGSL23}~\cite{PerezGSL23}, \href{works/abs-2312-13682.pdf}{abs-2312-13682}~\cite{abs-2312-13682}, \href{works/Mehdizadeh-Somarin23.pdf}{Mehdizadeh-Somarin23}~\cite{Mehdizadeh-Somarin23}, \href{works/TardivoDFMP23.pdf}{TardivoDFMP23}~\cite{TardivoDFMP23}, \href{works/abs-2306-05747.pdf}{abs-2306-05747}~\cite{abs-2306-05747}, \href{works/Teppan22.pdf}{Teppan22}~\cite{Teppan22}, \href{works/ZhangJZL22.pdf}{ZhangJZL22}~\cite{ZhangJZL22}, \href{works/ArmstrongGOS22.pdf}{ArmstrongGOS22}~\cite{ArmstrongGOS22}, \href{works/ZhangBB22.pdf}{ZhangBB22}~\cite{ZhangBB22}, \href{works/ZhangYW21.pdf}{ZhangYW21}~\cite{ZhangYW21}, \href{works/abs-2102-08778.pdf}{abs-2102-08778}~\cite{abs-2102-08778}, \href{works/FanXG21.pdf}{FanXG21}~\cite{FanXG21}, \href{works/AbreuAPNM21.pdf}{AbreuAPNM21}~\cite{AbreuAPNM21}, \href{works/AntuoriHHEN21.pdf}{AntuoriHHEN21}~\cite{AntuoriHHEN21}, \href{works/LacknerMMWW21.pdf}{LacknerMMWW21}~\cite{LacknerMMWW21}, \href{works/HamPK21.pdf}{HamPK21}~\cite{HamPK21}, \href{works/AstrandJZ20.pdf}{AstrandJZ20}~\cite{AstrandJZ20}, \href{works/SacramentoSP20.pdf}{SacramentoSP20}~\cite{SacramentoSP20}, \href{works/FallahiAC20.pdf}{FallahiAC20}~\cite{FallahiAC20}, \href{works/BenediktMH20.pdf}{BenediktMH20}~\cite{BenediktMH20}, \href{works/MengZRZL20.pdf}{MengZRZL20}~\cite{MengZRZL20}, \href{works/ParkUJR19.pdf}{ParkUJR19}~\cite{ParkUJR19}, \href{works/MurinR19.pdf}{MurinR19}~\cite{MurinR19}, \href{works/abs-1902-09244.pdf}{abs-1902-09244}~\cite{abs-1902-09244}, \href{works/FrimodigS19.pdf}{FrimodigS19}~\cite{FrimodigS19}, \href{works/abs-1902-01193.pdf}{abs-1902-01193}~\cite{abs-1902-01193}... (Total: 84)\\
Concepts & temporal constraint reasoning &  &  & \href{works/BartakSR10.pdf}{BartakSR10}~\cite{BartakSR10}, \href{works/KeriK07.pdf}{KeriK07}~\cite{KeriK07}, \href{works/FortinZDF05.pdf}{FortinZDF05}~\cite{FortinZDF05}\\
Concepts & transportation & \href{works/CzerniachowskaWZ23.pdf}{CzerniachowskaWZ23}~\cite{CzerniachowskaWZ23}, \href{works/ArmstrongGOS22.pdf}{ArmstrongGOS22}~\cite{ArmstrongGOS22}, \href{works/PohlAK22.pdf}{PohlAK22}~\cite{PohlAK22}, \href{works/BourreauGGLT22.pdf}{BourreauGGLT22}~\cite{BourreauGGLT22}, \href{works/GeitzGSSW22.pdf}{GeitzGSSW22}~\cite{GeitzGSSW22}, \href{works/ArmstrongGOS21.pdf}{ArmstrongGOS21}~\cite{ArmstrongGOS21}, \href{works/QinDCS20.pdf}{QinDCS20}~\cite{QinDCS20}, \href{works/SacramentoSP20.pdf}{SacramentoSP20}~\cite{SacramentoSP20}, \href{works/MurinR19.pdf}{MurinR19}~\cite{MurinR19}, \href{works/Ham18.pdf}{Ham18}~\cite{Ham18}, \href{works/PourDERB18.pdf}{PourDERB18}~\cite{PourDERB18}, \href{works/TangLWSK18.pdf}{TangLWSK18}~\cite{TangLWSK18}, \href{works/GoelSHFS15.pdf}{GoelSHFS15}~\cite{GoelSHFS15}, \href{works/NovasH14.pdf}{NovasH14}~\cite{NovasH14}, \href{works/KelarevaTK13.pdf}{KelarevaTK13}~\cite{KelarevaTK13}, \href{works/NovasH12.pdf}{NovasH12}~\cite{NovasH12}, \href{works/HachemiGR11.pdf}{HachemiGR11}~\cite{HachemiGR11}, \href{works/LopesCSM10.pdf}{LopesCSM10}~\cite{LopesCSM10}, \href{works/BocewiczBB09.pdf}{BocewiczBB09}~\cite{BocewiczBB09}, \href{works/Rodriguez07.pdf}{Rodriguez07}~\cite{Rodriguez07}, \href{works/ZeballosH05.pdf}{ZeballosH05}~\cite{ZeballosH05} & \href{works/NaderiRR23.pdf}{NaderiRR23}~\cite{NaderiRR23}, \href{works/KimCMLLP23.pdf}{KimCMLLP23}~\cite{KimCMLLP23}, \href{works/AbreuN22.pdf}{AbreuN22}~\cite{AbreuN22}, \href{works/SubulanC22.pdf}{SubulanC22}~\cite{SubulanC22}, \href{works/PopovicCGNC22.pdf}{PopovicCGNC22}~\cite{PopovicCGNC22}, \href{works/AbohashimaEG21.pdf}{AbohashimaEG21}~\cite{AbohashimaEG21}, \href{works/MengZRZL20.pdf}{MengZRZL20}~\cite{MengZRZL20}, \href{works/MejiaY20.pdf}{MejiaY20}~\cite{MejiaY20}, \href{works/FallahiAC20.pdf}{FallahiAC20}~\cite{FallahiAC20}, \href{works/LaborieRSV18.pdf}{LaborieRSV18}~\cite{LaborieRSV18}, \href{works/EvenSH15.pdf}{EvenSH15}~\cite{EvenSH15}, \href{works/MelgarejoLS15.pdf}{MelgarejoLS15}~\cite{MelgarejoLS15}, \href{works/RendlPHPR12.pdf}{RendlPHPR12}~\cite{RendlPHPR12}, \href{works/MakMS10.pdf}{MakMS10}~\cite{MakMS10}, \href{works/MouraSCL08a.pdf}{MouraSCL08a}~\cite{MouraSCL08a}, \href{works/MouraSCL08.pdf}{MouraSCL08}~\cite{MouraSCL08}, \href{works/LimRX04.pdf}{LimRX04}~\cite{LimRX04}, \href{works/Mason01.pdf}{Mason01}~\cite{Mason01}, \href{works/ArtiguesR00.pdf}{ArtiguesR00}~\cite{ArtiguesR00}, \href{works/Wallace96.pdf}{Wallace96}~\cite{Wallace96} & \href{works/AalianPG23.pdf}{AalianPG23}~\cite{AalianPG23}, \href{works/IsikYA23.pdf}{IsikYA23}~\cite{IsikYA23}, \href{works/AbreuNP23.pdf}{AbreuNP23}~\cite{AbreuNP23}, \href{works/abs-2312-13682.pdf}{abs-2312-13682}~\cite{abs-2312-13682}, \href{works/WangB23.pdf}{WangB23}~\cite{WangB23}, \href{works/MontemanniD23a.pdf}{MontemanniD23a}~\cite{MontemanniD23a}, \href{works/PerezGSL23.pdf}{PerezGSL23}~\cite{PerezGSL23}, \href{works/AlfieriGPS23.pdf}{AlfieriGPS23}~\cite{AlfieriGPS23}, \href{works/ColT22.pdf}{ColT22}~\cite{ColT22}, \href{works/BoudreaultSLQ22.pdf}{BoudreaultSLQ22}~\cite{BoudreaultSLQ22}, \href{works/abs-2211-14492.pdf}{abs-2211-14492}~\cite{abs-2211-14492}, \href{works/ZhangJZL22.pdf}{ZhangJZL22}~\cite{ZhangJZL22}, \href{works/YuraszeckMPV22.pdf}{YuraszeckMPV22}~\cite{YuraszeckMPV22}, \href{works/LiFJZLL22.pdf}{LiFJZLL22}~\cite{LiFJZLL22}, \href{works/YunusogluY22.pdf}{YunusogluY22}~\cite{YunusogluY22}, \href{works/AntuoriHHEN21.pdf}{AntuoriHHEN21}~\cite{AntuoriHHEN21}, \href{works/Bedhief21.pdf}{Bedhief21}~\cite{Bedhief21}, \href{works/HubnerGSV21.pdf}{HubnerGSV21}~\cite{HubnerGSV21}, \href{works/GroleazNS20a.pdf}{GroleazNS20a}~\cite{GroleazNS20a}, \href{works/WallaceY20.pdf}{WallaceY20}~\cite{WallaceY20}, \href{works/Novas19.pdf}{Novas19}~\cite{Novas19}, \href{works/abs-1902-09244.pdf}{abs-1902-09244}~\cite{abs-1902-09244}, \href{works/Tom19.pdf}{Tom19}~\cite{Tom19}, \href{works/GoldwaserS18.pdf}{GoldwaserS18}~\cite{GoldwaserS18}, \href{works/GokgurHO18.pdf}{GokgurHO18}~\cite{GokgurHO18}, \href{works/ZhangW18.pdf}{ZhangW18}~\cite{ZhangW18}, \href{works/ShinBBHO18.pdf}{ShinBBHO18}~\cite{ShinBBHO18}, \href{works/He0GLW18.pdf}{He0GLW18}~\cite{He0GLW18}, \href{works/GedikKEK18.pdf}{GedikKEK18}~\cite{GedikKEK18}... (Total: 66)\\
\end{longtable}
}


\clearpage
\subsection{Concept Type Classification}
\label{sec:Classification}
{\scriptsize
\begin{longtable}{lp{3cm}>{\raggedright\arraybackslash}p{6cm}>{\raggedright\arraybackslash}p{6cm}>{\raggedright\arraybackslash}p{8cm}}
\rowcolor{white}\caption{Works for Concepts of Type Classification}\\ \toprule
\rowcolor{white}Type & Keyword & High & Medium & Low\\ \midrule\endhead
\bottomrule
\endfoot
Classification & 2BPHFSP & \href{../works/TangB20.pdf}{TangB20}~\cite{TangB20} &  & \\
Classification & BPCTOP & \href{../works/KelarevaTK13.pdf}{KelarevaTK13}~\cite{KelarevaTK13} &  & \\
Classification & Bulk Port Cargo Throughput Optimisation Problem &  &  & \href{../works/KelarevaTK13.pdf}{KelarevaTK13}~\cite{KelarevaTK13}\\
Classification & CECSP & \href{../works/NattafHKAL19.pdf}{NattafHKAL19}~\cite{NattafHKAL19}, \href{../works/NattafAL17.pdf}{NattafAL17}~\cite{NattafAL17}, \href{../works/Nattaf16.pdf}{Nattaf16}~\cite{Nattaf16}, \href{../works/NattafALR16.pdf}{NattafALR16}~\cite{NattafALR16}, \href{../works/NattafAL15.pdf}{NattafAL15}~\cite{NattafAL15} &  & \\
Classification & CHSP & \href{../works/EfthymiouY23.pdf}{EfthymiouY23}~\cite{EfthymiouY23}, \href{../works/WallaceY20.pdf}{WallaceY20}~\cite{WallaceY20} &  & \\
Classification & CTW & \href{../works/KoehlerBFFHPSSS21.pdf}{KoehlerBFFHPSSS21}~\cite{KoehlerBFFHPSSS21} & \href{../works/Lombardi10.pdf}{Lombardi10}~\cite{Lombardi10} & \\
Classification & CuSP & \href{../works/KameugneFND23.pdf}{KameugneFND23}~\cite{KameugneFND23}, \href{../works/FetgoD22.pdf}{FetgoD22}~\cite{FetgoD22}, \href{../works/Tesch18.pdf}{Tesch18}~\cite{Tesch18}, \href{../works/KameugneFGOQ18.pdf}{KameugneFGOQ18}~\cite{KameugneFGOQ18}, \href{../works/Tesch16.pdf}{Tesch16}~\cite{Tesch16}, \href{../works/NattafALR16.pdf}{NattafALR16}~\cite{NattafALR16}, \href{../works/Nattaf16.pdf}{Nattaf16}~\cite{Nattaf16}, \href{../works/Froger16.pdf}{Froger16}~\cite{Froger16}, \href{../works/NattafAL15.pdf}{NattafAL15}~\cite{NattafAL15}, \href{../works/Derrien15.pdf}{Derrien15}~\cite{Derrien15}, \href{../works/Kameugne14.pdf}{Kameugne14}~\cite{Kameugne14}, \href{../works/KameugneFSN14.pdf}{KameugneFSN14}~\cite{KameugneFSN14}, \href{../works/DerrienPZ14.pdf}{DerrienPZ14}~\cite{DerrienPZ14}, \href{../works/KameugneFSN11.pdf}{KameugneFSN11}~\cite{KameugneFSN11}, \href{../works/SchuttW10.pdf}{SchuttW10}~\cite{SchuttW10}, \href{../works/Demassey03.pdf}{Demassey03}~\cite{Demassey03} & \href{../works/Fahimi16.pdf}{Fahimi16}~\cite{Fahimi16}, \href{../works/GingrasQ16.pdf}{GingrasQ16}~\cite{GingrasQ16}, \href{../works/OuelletQ13.pdf}{OuelletQ13}~\cite{OuelletQ13}, \href{../works/Elkhyari03.pdf}{Elkhyari03}~\cite{Elkhyari03} & \href{../works/TardivoDFMP23.pdf}{TardivoDFMP23}~\cite{TardivoDFMP23}, \href{../works/HanenKP21.pdf}{HanenKP21}~\cite{HanenKP21}, \href{../works/Zahout21.pdf}{Zahout21}~\cite{Zahout21}, \href{../works/DerrienP14.pdf}{DerrienP14}~\cite{DerrienP14}\\
Classification & EOSP &  & \href{../works/SquillaciPR23.pdf}{SquillaciPR23}~\cite{SquillaciPR23} & \\
Classification & Earth Observation Scheduling Problem &  & \href{../works/SquillaciPR23.pdf}{SquillaciPR23}~\cite{SquillaciPR23} & \\
Classification & FJS & \href{../works/JuvinHL23a.pdf}{JuvinHL23a}~\cite{JuvinHL23a}, \href{../works/WangB23.pdf}{WangB23}~\cite{WangB23}, \href{../works/YuraszeckMCCR23.pdf}{YuraszeckMCCR23}~\cite{YuraszeckMCCR23}, \href{../works/JuvinHL22.pdf}{JuvinHL22}~\cite{JuvinHL22}, \href{../works/MullerMKP22.pdf}{MullerMKP22}~\cite{MullerMKP22}, \href{../works/Teppan22.pdf}{Teppan22}~\cite{Teppan22}, \href{../works/HamPK21.pdf}{HamPK21}~\cite{HamPK21}, \href{../works/WangB20.pdf}{WangB20}~\cite{WangB20}, \href{../works/Lunardi20.pdf}{Lunardi20}~\cite{Lunardi20}, \href{../works/LunardiBLRV20.pdf}{LunardiBLRV20}~\cite{LunardiBLRV20}, \href{../works/ZarandiASC20.pdf}{ZarandiASC20}~\cite{ZarandiASC20}, \href{../works/MengZRZL20.pdf}{MengZRZL20}~\cite{MengZRZL20}, \href{../works/Novas19.pdf}{Novas19}~\cite{Novas19}, \href{../works/MossigeGSMC17.pdf}{MossigeGSMC17}~\cite{MossigeGSMC17}, \href{../works/HamC16.pdf}{HamC16}~\cite{HamC16} & \href{../works/OujanaAYB22.pdf}{OujanaAYB22}~\cite{OujanaAYB22}, \href{../works/HauderBRPA20.pdf}{HauderBRPA20}~\cite{HauderBRPA20}, \href{../works/abs-1902-09244.pdf}{abs-1902-09244}~\cite{abs-1902-09244}, \href{../works/ZhangW18.pdf}{ZhangW18}~\cite{ZhangW18}, \href{../works/SchuttFS13.pdf}{SchuttFS13}~\cite{SchuttFS13} & \href{../works/NaderiRR23.pdf}{NaderiRR23}~\cite{NaderiRR23}, \href{../works/ColT22.pdf}{ColT22}~\cite{ColT22}, \href{../works/ZhouGL15.pdf}{ZhouGL15}~\cite{ZhouGL15}\\
Classification & Fixed Job Scheduling & \href{../works/WangB20.pdf}{WangB20}~\cite{WangB20} & \href{../works/WangB23.pdf}{WangB23}~\cite{WangB23} & \\
Classification & GCSP & \href{../works/Groleaz21.pdf}{Groleaz21}~\cite{Groleaz21}, \href{../works/GroleazNS20.pdf}{GroleazNS20}~\cite{GroleazNS20} &  & \\
Classification & HFF & \href{../works/ArmstrongGOS22.pdf}{ArmstrongGOS22}~\cite{ArmstrongGOS22}, \href{../works/OujanaAYB22.pdf}{OujanaAYB22}~\cite{OujanaAYB22}, \href{../works/ArmstrongGOS21.pdf}{ArmstrongGOS21}~\cite{ArmstrongGOS21}, \href{../works/ZhouGL15.pdf}{ZhouGL15}~\cite{ZhouGL15} &  & \\
Classification & HFFTT & \href{../works/ArmstrongGOS22.pdf}{ArmstrongGOS22}~\cite{ArmstrongGOS22}, \href{../works/ArmstrongGOS21.pdf}{ArmstrongGOS21}~\cite{ArmstrongGOS21} &  & \\
Classification & HFS & \href{../works/IsikYA23.pdf}{IsikYA23}~\cite{IsikYA23}, \href{../works/ZhangJZL22.pdf}{ZhangJZL22}~\cite{ZhangJZL22}, \href{../works/Astrand21.pdf}{Astrand21}~\cite{Astrand21}, \href{../works/ArmstrongGOS21.pdf}{ArmstrongGOS21}~\cite{ArmstrongGOS21}, \href{../works/Bedhief21.pdf}{Bedhief21}~\cite{Bedhief21}, \href{../works/TangB20.pdf}{TangB20}~\cite{TangB20}, \href{../works/MengZRZL20.pdf}{MengZRZL20}~\cite{MengZRZL20}, \href{../works/Baptiste02.pdf}{Baptiste02}~\cite{Baptiste02} &  & \href{../works/ArmstrongGOS22.pdf}{ArmstrongGOS22}~\cite{ArmstrongGOS22}, \href{../works/ZarandiASC20.pdf}{ZarandiASC20}~\cite{ZarandiASC20}, \href{../works/Novas19.pdf}{Novas19}~\cite{Novas19}, \href{../works/ZhouGL15.pdf}{ZhouGL15}~\cite{ZhouGL15}\\
Classification & JSPT &  & \href{../works/MurinR19.pdf}{MurinR19}~\cite{MurinR19} & \\
Classification & JSSP & \href{../works/TasselGS23.pdf}{TasselGS23}~\cite{TasselGS23}, \href{../works/JuvinHL23a.pdf}{JuvinHL23a}~\cite{JuvinHL23a}, \href{../works/JuvinHHL23.pdf}{JuvinHHL23}~\cite{JuvinHHL23}, \href{../works/YuraszeckMC23.pdf}{YuraszeckMC23}~\cite{YuraszeckMC23}, \href{../works/YuraszeckMCCR23.pdf}{YuraszeckMCCR23}~\cite{YuraszeckMCCR23}, \href{../works/abs-2306-05747.pdf}{abs-2306-05747}~\cite{abs-2306-05747}, \href{../works/JuvinHL22.pdf}{JuvinHL22}~\cite{JuvinHL22}, \href{../works/Teppan22.pdf}{Teppan22}~\cite{Teppan22}, \href{../works/ColT22.pdf}{ColT22}~\cite{ColT22}, \href{../works/YuraszeckMPV22.pdf}{YuraszeckMPV22}~\cite{YuraszeckMPV22}, \href{../works/GeitzGSSW22.pdf}{GeitzGSSW22}~\cite{GeitzGSSW22}, \href{../works/Godet21a.pdf}{Godet21a}~\cite{Godet21a}, \href{../works/abs-2102-08778.pdf}{abs-2102-08778}~\cite{abs-2102-08778}, \href{../works/ZarandiASC20.pdf}{ZarandiASC20}~\cite{ZarandiASC20}, \href{../works/ColT19.pdf}{ColT19}~\cite{ColT19}, \href{../works/Pralet17.pdf}{Pralet17}~\cite{Pralet17}, \href{../works/MenciaSV13.pdf}{MenciaSV13}~\cite{MenciaSV13}, \href{../works/MenciaSV12.pdf}{MenciaSV12}~\cite{MenciaSV12}, \href{../works/KelbelH11.pdf}{KelbelH11}~\cite{KelbelH11}, \href{../works/BidotVLB09.pdf}{BidotVLB09}~\cite{BidotVLB09}, \href{../works/GodardLN05.pdf}{GodardLN05}~\cite{GodardLN05}, \href{../works/Baptiste02.pdf}{Baptiste02}~\cite{Baptiste02}, \href{../works/SourdN00.pdf}{SourdN00}~\cite{SourdN00}, \href{../works/TorresL00.pdf}{TorresL00}~\cite{TorresL00}, \href{../works/PapaB98.pdf}{PapaB98}~\cite{PapaB98}, \href{../works/NuijtenP98.pdf}{NuijtenP98}~\cite{NuijtenP98}, \href{../works/NuijtenA96.pdf}{NuijtenA96}~\cite{NuijtenA96}, \href{../works/NuijtenA94.pdf}{NuijtenA94}~\cite{NuijtenA94} & \href{../works/GalleguillosKSB19.pdf}{GalleguillosKSB19}~\cite{GalleguillosKSB19}, \href{../works/LombardiBM15.pdf}{LombardiBM15}~\cite{LombardiBM15}, \href{../works/SialaAH15.pdf}{SialaAH15}~\cite{SialaAH15}, \href{../works/BelhadjiI98.pdf}{BelhadjiI98}~\cite{BelhadjiI98} & \href{../works/Mehdizadeh-Somarin23.pdf}{Mehdizadeh-Somarin23}~\cite{Mehdizadeh-Somarin23}, \href{../works/CzerniachowskaWZ23.pdf}{CzerniachowskaWZ23}~\cite{CzerniachowskaWZ23}, \href{../works/EfthymiouY23.pdf}{EfthymiouY23}~\cite{EfthymiouY23}, \href{../works/WikarekS19.pdf}{WikarekS19}~\cite{WikarekS19}, \href{../works/PraletLJ15.pdf}{PraletLJ15}~\cite{PraletLJ15}, \href{../works/GrimesH15.pdf}{GrimesH15}~\cite{GrimesH15}, \href{../works/BajestaniB11.pdf}{BajestaniB11}~\cite{BajestaniB11}, \href{../works/ChenGPSH10.pdf}{ChenGPSH10}~\cite{ChenGPSH10}\\
Classification & KRFP & \href{../works/KamarainenS02.pdf}{KamarainenS02}~\cite{KamarainenS02}, \href{../works/SakkoutW00.pdf}{SakkoutW00}~\cite{SakkoutW00} &  & \\
Classification & LSFRP & \href{../works/KelarevaTK13.pdf}{KelarevaTK13}~\cite{KelarevaTK13} &  & \\
Classification & Liner Shipping Fleet Repositioning Problem &  & \href{../works/KelarevaTK13.pdf}{KelarevaTK13}~\cite{KelarevaTK13} & \\
Classification & MGAP & \href{../works/Darby-DowmanLMZ97.pdf}{Darby-DowmanLMZ97}~\cite{Darby-DowmanLMZ97} &  & \\
Classification & Modified Generalized Assignment Problem &  &  & \\
Classification & OSP & \href{../works/NaderiRR23.pdf}{NaderiRR23}~\cite{NaderiRR23}, \href{../works/LacknerMMWW23.pdf}{LacknerMMWW23}~\cite{LacknerMMWW23}, \href{../works/Bit-Monnot23.pdf}{Bit-Monnot23}~\cite{Bit-Monnot23}, \href{../works/LacknerMMWW21.pdf}{LacknerMMWW21}~\cite{LacknerMMWW21}, \href{../works/Groleaz21.pdf}{Groleaz21}~\cite{Groleaz21}, \href{../works/GombolayWS18.pdf}{GombolayWS18}~\cite{GombolayWS18}, \href{../works/GrimesH15.pdf}{GrimesH15}~\cite{GrimesH15}, \href{../works/Siala15.pdf}{Siala15}~\cite{Siala15}, \href{../works/GayHLS15.pdf}{GayHLS15}~\cite{GayHLS15}, \href{../works/Siala15a.pdf}{Siala15a}~\cite{Siala15a}, \href{../works/MalapertCGJLR12.pdf}{MalapertCGJLR12}~\cite{MalapertCGJLR12} & \href{../works/SquillaciPR23.pdf}{SquillaciPR23}~\cite{SquillaciPR23}, \href{../works/GrimesHM09.pdf}{GrimesHM09}~\cite{GrimesHM09}, \href{../works/MonetteDD07.pdf}{MonetteDD07}~\cite{MonetteDD07} & \href{../works/MengZRZL20.pdf}{MengZRZL20}~\cite{MengZRZL20}\\
Classification & OSSP & \href{../works/YuraszeckMC23.pdf}{YuraszeckMC23}~\cite{YuraszeckMC23}, \href{../works/AbreuPNF23.pdf}{AbreuPNF23}~\cite{AbreuPNF23}, \href{../works/AbreuNP23.pdf}{AbreuNP23}~\cite{AbreuNP23}, \href{../works/YuraszeckMPV22.pdf}{YuraszeckMPV22}~\cite{YuraszeckMPV22}, \href{../works/ColT22.pdf}{ColT22}~\cite{ColT22}, \href{../works/AbreuN22.pdf}{AbreuN22}~\cite{AbreuN22}, \href{../works/AbreuAPNM21.pdf}{AbreuAPNM21}~\cite{AbreuAPNM21}, \href{../works/MejiaY20.pdf}{MejiaY20}~\cite{MejiaY20}, \href{../works/Baptiste02.pdf}{Baptiste02}~\cite{Baptiste02} &  & \href{../works/YuraszeckMCCR23.pdf}{YuraszeckMCCR23}~\cite{YuraszeckMCCR23}, \href{../works/ZarandiASC20.pdf}{ZarandiASC20}~\cite{ZarandiASC20}\\
Classification & Open Shop Scheduling Problem & \href{../works/AbreuPNF23.pdf}{AbreuPNF23}~\cite{AbreuPNF23}, \href{../works/AbreuNP23.pdf}{AbreuNP23}~\cite{AbreuNP23}, \href{../works/AbreuN22.pdf}{AbreuN22}~\cite{AbreuN22}, \href{../works/AbreuAPNM21.pdf}{AbreuAPNM21}~\cite{AbreuAPNM21}, \href{../works/MejiaY20.pdf}{MejiaY20}~\cite{MejiaY20}, \href{../works/ZarandiASC20.pdf}{ZarandiASC20}~\cite{ZarandiASC20} & \href{../works/Malapert11.pdf}{Malapert11}~\cite{Malapert11}, \href{../works/LorigeonBB02.pdf}{LorigeonBB02}~\cite{LorigeonBB02} & \href{../works/PrataAN23.pdf}{PrataAN23}~\cite{PrataAN23}, \href{../works/NaderiRR23.pdf}{NaderiRR23}~\cite{NaderiRR23}, \href{../works/Bit-Monnot23.pdf}{Bit-Monnot23}~\cite{Bit-Monnot23}, \href{../works/YuraszeckMCCR23.pdf}{YuraszeckMCCR23}~\cite{YuraszeckMCCR23}, \href{../works/YuraszeckMPV22.pdf}{YuraszeckMPV22}~\cite{YuraszeckMPV22}, \href{../works/ColT22.pdf}{ColT22}~\cite{ColT22}, \href{../works/Groleaz21.pdf}{Groleaz21}~\cite{Groleaz21}, \href{../works/MengZRZL20.pdf}{MengZRZL20}~\cite{MengZRZL20}, \href{../works/SacramentoSP20.pdf}{SacramentoSP20}~\cite{SacramentoSP20}, \href{../works/HookerH17.pdf}{HookerH17}~\cite{HookerH17}, \href{../works/GrimesH15.pdf}{GrimesH15}~\cite{GrimesH15}, \href{../works/MalapertCGJLR13.pdf}{MalapertCGJLR13}~\cite{MalapertCGJLR13}, \href{../works/MalapertCGJLR12.pdf}{MalapertCGJLR12}~\cite{MalapertCGJLR12}, \href{../works/Schutt11.pdf}{Schutt11}~\cite{Schutt11}, \href{../works/GrimesH10.pdf}{GrimesH10}~\cite{GrimesH10}, \href{../works/OhrimenkoSC09.pdf}{OhrimenkoSC09}~\cite{OhrimenkoSC09}, \href{../works/GrimesHM09.pdf}{GrimesHM09}~\cite{GrimesHM09}, \href{../works/MonetteDD07.pdf}{MonetteDD07}~\cite{MonetteDD07}, \href{../works/Baptiste02.pdf}{Baptiste02}~\cite{Baptiste02}, \href{../works/VerfaillieL01.pdf}{VerfaillieL01}~\cite{VerfaillieL01}\\
Classification & PJSSP & \href{../works/Baptiste02.pdf}{Baptiste02}~\cite{Baptiste02} & \href{../works/PapaB98.pdf}{PapaB98}~\cite{PapaB98} & \\
Classification & PMSP & \href{../works/NaderiRR23.pdf}{NaderiRR23}~\cite{NaderiRR23}, \href{../works/YunusogluY22.pdf}{YunusogluY22}~\cite{YunusogluY22}, \href{../works/WinterMMW22.pdf}{WinterMMW22}~\cite{WinterMMW22}, \href{../works/PandeyS21a.pdf}{PandeyS21a}~\cite{PandeyS21a}, \href{../works/Godet21a.pdf}{Godet21a}~\cite{Godet21a}, \href{../works/GodetLHS20.pdf}{GodetLHS20}~\cite{GodetLHS20}, \href{../works/MalapertN19.pdf}{MalapertN19}~\cite{MalapertN19}, \href{../works/GedikKEK18.pdf}{GedikKEK18}~\cite{GedikKEK18}, \href{../works/GomesM17.pdf}{GomesM17}~\cite{GomesM17}, \href{../works/TranAB16.pdf}{TranAB16}~\cite{TranAB16}, \href{../works/TranB12.pdf}{TranB12}~\cite{TranB12} & \href{../works/VlkHT21.pdf}{VlkHT21}~\cite{VlkHT21}, \href{../works/NattafM20.pdf}{NattafM20}~\cite{NattafM20} & \href{../works/ColT22.pdf}{ColT22}~\cite{ColT22}, \href{../works/OujanaAYB22.pdf}{OujanaAYB22}~\cite{OujanaAYB22}, \href{../works/ZarandiASC20.pdf}{ZarandiASC20}~\cite{ZarandiASC20}\\
Classification & PP-MS-MMRCPSP &  &  & \\
Classification & PTC & \href{../works/NattafM20.pdf}{NattafM20}~\cite{NattafM20}, \href{../works/MalapertN19.pdf}{MalapertN19}~\cite{MalapertN19}, \href{../works/NattafDYW19.pdf}{NattafDYW19}~\cite{NattafDYW19} & \href{../works/NaderiRR23.pdf}{NaderiRR23}~\cite{NaderiRR23} & \href{../works/CzerniachowskaWZ23.pdf}{CzerniachowskaWZ23}~\cite{CzerniachowskaWZ23}, \href{../works/Teppan22.pdf}{Teppan22}~\cite{Teppan22}, \href{../works/Dejemeppe16.pdf}{Dejemeppe16}~\cite{Dejemeppe16}\\
Classification & Pre-emptive Job-Shop scheduling Problem &  &  & \\
Classification & RCMPSP & \href{../works/HauderBRPA20.pdf}{HauderBRPA20}~\cite{HauderBRPA20}, \href{../works/abs-1902-09244.pdf}{abs-1902-09244}~\cite{abs-1902-09244} &  & \href{../works/ArtiguesR00.pdf}{ArtiguesR00}~\cite{ArtiguesR00}\\
Classification & RCPSP & \href{../works/YuraszeckMCCR23.pdf}{YuraszeckMCCR23}~\cite{YuraszeckMCCR23}, \href{../works/GokPTGO23.pdf}{GokPTGO23}~\cite{GokPTGO23}, \href{../works/PovedaAA23.pdf}{PovedaAA23}~\cite{PovedaAA23}, \href{../works/CampeauG22.pdf}{CampeauG22}~\cite{CampeauG22}, \href{../works/BoudreaultSLQ22.pdf}{BoudreaultSLQ22}~\cite{BoudreaultSLQ22}, \href{../works/EtminaniesfahaniGNMS22.pdf}{EtminaniesfahaniGNMS22}~\cite{EtminaniesfahaniGNMS22}, \href{../works/FetgoD22.pdf}{FetgoD22}~\cite{FetgoD22}, \href{../works/SubulanC22.pdf}{SubulanC22}~\cite{SubulanC22}, \href{../works/GeibingerMM21.pdf}{GeibingerMM21}~\cite{GeibingerMM21}, \href{../works/HubnerGSV21.pdf}{HubnerGSV21}~\cite{HubnerGSV21}, \href{../works/Godet21a.pdf}{Godet21a}~\cite{Godet21a}, \href{../works/BenderWS21.pdf}{BenderWS21}~\cite{BenderWS21}, \href{../works/HillTV21.pdf}{HillTV21}~\cite{HillTV21}, \href{../works/Zahout21.pdf}{Zahout21}~\cite{Zahout21}, \href{../works/ArtiguesHQT21.pdf}{ArtiguesHQT21}~\cite{ArtiguesHQT21}, \href{../works/Groleaz21.pdf}{Groleaz21}~\cite{Groleaz21}, \href{../works/ZarandiASC20.pdf}{ZarandiASC20}~\cite{ZarandiASC20}, \href{../works/HauderBRPA20.pdf}{HauderBRPA20}~\cite{HauderBRPA20}, \href{../works/Polo-MejiaALB20.pdf}{Polo-MejiaALB20}~\cite{Polo-MejiaALB20}, \href{../works/GokGSTO20.pdf}{GokGSTO20}~\cite{GokGSTO20}, \href{../works/GeibingerMM19.pdf}{GeibingerMM19}~\cite{GeibingerMM19}, \href{../works/abs-1911-04766.pdf}{abs-1911-04766}~\cite{abs-1911-04766}, \href{../works/Caballero19.pdf}{Caballero19}~\cite{Caballero19}, \href{../works/abs-1902-09244.pdf}{abs-1902-09244}~\cite{abs-1902-09244}, \href{../works/ArkhipovBL19.pdf}{ArkhipovBL19}~\cite{ArkhipovBL19}, \href{../works/KreterSSZ18.pdf}{KreterSSZ18}~\cite{KreterSSZ18}, \href{../works/KameugneFGOQ18.pdf}{KameugneFGOQ18}~\cite{KameugneFGOQ18}, \href{../works/LaborieRSV18.pdf}{LaborieRSV18}~\cite{LaborieRSV18}, \href{../works/TangLWSK18.pdf}{TangLWSK18}~\cite{TangLWSK18}... (Total: 66) & \href{../works/Caballero23.pdf}{Caballero23}~\cite{Caballero23}, \href{../works/KameugneFND23.pdf}{KameugneFND23}~\cite{KameugneFND23}, \href{../works/TardivoDFMP23.pdf}{TardivoDFMP23}~\cite{TardivoDFMP23}, \href{../works/KovacsTKSG21.pdf}{KovacsTKSG21}~\cite{KovacsTKSG21}, \href{../works/GroleazNS20a.pdf}{GroleazNS20a}~\cite{GroleazNS20a}, \href{../works/Tesch18.pdf}{Tesch18}~\cite{Tesch18}, \href{../works/CauwelaertLS18.pdf}{CauwelaertLS18}~\cite{CauwelaertLS18}, \href{../works/BaptisteB18.pdf}{BaptisteB18}~\cite{BaptisteB18}, \href{../works/Dejemeppe16.pdf}{Dejemeppe16}~\cite{Dejemeppe16}, \href{../works/NattafAL15.pdf}{NattafAL15}~\cite{NattafAL15}, \href{../works/GayHLS15.pdf}{GayHLS15}~\cite{GayHLS15}, \href{../works/LombardiBM15.pdf}{LombardiBM15}~\cite{LombardiBM15}, \href{../works/KameugneFSN14.pdf}{KameugneFSN14}~\cite{KameugneFSN14}, \href{../works/LombardiM13.pdf}{LombardiM13}~\cite{LombardiM13}, \href{../works/LombardiMB13.pdf}{LombardiMB13}~\cite{LombardiMB13}, \href{../works/KameugneFSN11.pdf}{KameugneFSN11}~\cite{KameugneFSN11}, \href{../works/HeinzS11.pdf}{HeinzS11}~\cite{HeinzS11}, \href{../works/abs-1009-0347.pdf}{abs-1009-0347}~\cite{abs-1009-0347}, \href{../works/KeriK07.pdf}{KeriK07}~\cite{KeriK07}, \href{../works/KovacsV06.pdf}{KovacsV06}~\cite{KovacsV06}, \href{../works/HeipckeCCS00.pdf}{HeipckeCCS00}~\cite{HeipckeCCS00}, \href{../works/ArtiguesR00.pdf}{ArtiguesR00}~\cite{ArtiguesR00} & \href{../works/AbreuPNF23.pdf}{AbreuPNF23}~\cite{AbreuPNF23}, \href{../works/NaderiRR23.pdf}{NaderiRR23}~\cite{NaderiRR23}, \href{../works/GeitzGSSW22.pdf}{GeitzGSSW22}~\cite{GeitzGSSW22}, \href{../works/TouatBT22.pdf}{TouatBT22}~\cite{TouatBT22}, \href{../works/HanenKP21.pdf}{HanenKP21}~\cite{HanenKP21}, \href{../works/Astrand21.pdf}{Astrand21}~\cite{Astrand21}, \href{../works/Lemos21.pdf}{Lemos21}~\cite{Lemos21}, \href{../works/ZhangYW21.pdf}{ZhangYW21}~\cite{ZhangYW21}, \href{../works/Mercier-AubinGQ20.pdf}{Mercier-AubinGQ20}~\cite{Mercier-AubinGQ20}, \href{../works/NattafHKAL19.pdf}{NattafHKAL19}~\cite{NattafHKAL19}, \href{../works/WikarekS19.pdf}{WikarekS19}~\cite{WikarekS19}, \href{../works/OuelletQ18.pdf}{OuelletQ18}~\cite{OuelletQ18}, \href{../works/FahimiOQ18.pdf}{FahimiOQ18}~\cite{FahimiOQ18}, \href{../works/HookerH17.pdf}{HookerH17}~\cite{HookerH17}, \href{../works/GingrasQ16.pdf}{GingrasQ16}~\cite{GingrasQ16}, \href{../works/Tesch16.pdf}{Tesch16}~\cite{Tesch16}, \href{../works/NattafALR16.pdf}{NattafALR16}~\cite{NattafALR16}, \href{../works/BonfiettiZLM16.pdf}{BonfiettiZLM16}~\cite{BonfiettiZLM16}, \href{../works/Fahimi16.pdf}{Fahimi16}~\cite{Fahimi16}, \href{../works/Siala15.pdf}{Siala15}~\cite{Siala15}, \href{../works/Siala15a.pdf}{Siala15a}~\cite{Siala15a}, \href{../works/SialaAH15.pdf}{SialaAH15}~\cite{SialaAH15}, \href{../works/GayHS15a.pdf}{GayHS15a}~\cite{GayHS15a}, \href{../works/DerrienPZ14.pdf}{DerrienPZ14}~\cite{DerrienPZ14}, \href{../works/BonfiettiLBM14.pdf}{BonfiettiLBM14}~\cite{BonfiettiLBM14}, \href{../works/KoschB14.pdf}{KoschB14}~\cite{KoschB14}, \href{../works/BonfiettiLM14.pdf}{BonfiettiLM14}~\cite{BonfiettiLM14}, \href{../works/OuelletQ13.pdf}{OuelletQ13}~\cite{OuelletQ13}, \href{../works/SchuttFS13.pdf}{SchuttFS13}~\cite{SchuttFS13}... (Total: 45)\\
Classification & RCPSPDC &  &  & \href{../works/CampeauG22.pdf}{CampeauG22}~\cite{CampeauG22}, \href{../works/HubnerGSV21.pdf}{HubnerGSV21}~\cite{HubnerGSV21}\\
Classification & RTMP & \href{../works/MarliereSPR23.pdf}{MarliereSPR23}~\cite{MarliereSPR23} &  & \\
Classification & Resource-constrained Project Scheduling Problem with Discounted Cashflow &  &  & \\
Classification & SBSFMMAL & \href{../works/OzturkTHO13.pdf}{OzturkTHO13}~\cite{OzturkTHO13}, \href{../works/OzturkTHO10.pdf}{OzturkTHO10}~\cite{OzturkTHO10} & \href{../works/OzturkTHO15.pdf}{OzturkTHO15}~\cite{OzturkTHO15} & \\
Classification & SCC & \href{../works/KimCMLLP23.pdf}{KimCMLLP23}~\cite{KimCMLLP23}, \href{../works/WolinskiKG04.pdf}{WolinskiKG04}~\cite{WolinskiKG04} & \href{../works/SchuttFSW13.pdf}{SchuttFSW13}~\cite{SchuttFSW13}, \href{../works/Lombardi10.pdf}{Lombardi10}~\cite{Lombardi10}, \href{../works/abs-1009-0347.pdf}{abs-1009-0347}~\cite{abs-1009-0347} & \href{../works/PohlAK22.pdf}{PohlAK22}~\cite{PohlAK22}, \href{../works/Zahout21.pdf}{Zahout21}~\cite{Zahout21}, \href{../works/LombardiMB13.pdf}{LombardiMB13}~\cite{LombardiMB13}, \href{../works/BeniniLMR11.pdf}{BeniniLMR11}~\cite{BeniniLMR11}, \href{../works/SchausHMCMD11.pdf}{SchausHMCMD11}~\cite{SchausHMCMD11}, \href{../works/LombardiMRB10.pdf}{LombardiMRB10}~\cite{LombardiMRB10}, \href{../works/BeniniLMR08.pdf}{BeniniLMR08}~\cite{BeniniLMR08}\\
Classification & SMSDP &  &  & \\
Classification & Steel-making and continuous casting &  &  & \\
Classification & TCSP & \href{../works/BelhadjiI98.pdf}{BelhadjiI98}~\cite{BelhadjiI98} &  & \href{../works/Zahout21.pdf}{Zahout21}~\cite{Zahout21}, \href{../works/BartakSR10.pdf}{BartakSR10}~\cite{BartakSR10}, \href{../works/LombardiM10a.pdf}{LombardiM10a}~\cite{LombardiM10a}, \href{../works/Lombardi10.pdf}{Lombardi10}~\cite{Lombardi10}, \href{../works/Demassey03.pdf}{Demassey03}~\cite{Demassey03}\\
Classification & TMS & \href{../works/PopovicCGNC22.pdf}{PopovicCGNC22}~\cite{PopovicCGNC22}, \href{../works/Froger16.pdf}{Froger16}~\cite{Froger16} & \href{../works/BegB13.pdf}{BegB13}~\cite{BegB13} & \href{../works/CappartS17.pdf}{CappartS17}~\cite{CappartS17}, \href{../works/Siala15a.pdf}{Siala15a}~\cite{Siala15a}, \href{../works/Siala15.pdf}{Siala15}~\cite{Siala15}\\
Classification & Temporal Constraint Satisfaction Problem &  & \href{../works/BelhadjiI98.pdf}{BelhadjiI98}~\cite{BelhadjiI98} & \href{../works/BartakSR10.pdf}{BartakSR10}~\cite{BartakSR10}, \href{../works/MoffittPP05.pdf}{MoffittPP05}~\cite{MoffittPP05}, \href{../works/Elkhyari03.pdf}{Elkhyari03}~\cite{Elkhyari03}\\
Classification & parallel machine & \href{../works/PrataAN23.pdf}{PrataAN23}~\cite{PrataAN23}, \href{../works/abs-2305-19888.pdf}{abs-2305-19888}~\cite{abs-2305-19888}, \href{../works/Adelgren2023.pdf}{Adelgren2023}~\cite{Adelgren2023}, \href{../works/IsikYA23.pdf}{IsikYA23}~\cite{IsikYA23}, \href{../works/CzerniachowskaWZ23.pdf}{CzerniachowskaWZ23}~\cite{CzerniachowskaWZ23}, \href{../works/NaderiRR23.pdf}{NaderiRR23}~\cite{NaderiRR23}, \href{../works/YunusogluY22.pdf}{YunusogluY22}~\cite{YunusogluY22}, \href{../works/ZhangJZL22.pdf}{ZhangJZL22}~\cite{ZhangJZL22}, \href{../works/WinterMMW22.pdf}{WinterMMW22}~\cite{WinterMMW22}, \href{../works/HeinzNVH22.pdf}{HeinzNVH22}~\cite{HeinzNVH22}, \href{../works/OujanaAYB22.pdf}{OujanaAYB22}~\cite{OujanaAYB22}, \href{../works/PandeyS21a.pdf}{PandeyS21a}~\cite{PandeyS21a}, \href{../works/Astrand21.pdf}{Astrand21}~\cite{Astrand21}, \href{../works/Godet21a.pdf}{Godet21a}~\cite{Godet21a}, \href{../works/Groleaz21.pdf}{Groleaz21}~\cite{Groleaz21}, \href{../works/ZarandiASC20.pdf}{ZarandiASC20}~\cite{ZarandiASC20}, \href{../works/MengZRZL20.pdf}{MengZRZL20}~\cite{MengZRZL20}, \href{../works/Lunardi20.pdf}{Lunardi20}~\cite{Lunardi20}, \href{../works/GodetLHS20.pdf}{GodetLHS20}~\cite{GodetLHS20}, \href{../works/NattafM20.pdf}{NattafM20}~\cite{NattafM20}, \href{../works/NattafDYW19.pdf}{NattafDYW19}~\cite{NattafDYW19}, \href{../works/MalapertN19.pdf}{MalapertN19}~\cite{MalapertN19}, \href{../works/GokgurHO18.pdf}{GokgurHO18}~\cite{GokgurHO18}, \href{../works/GedikKEK18.pdf}{GedikKEK18}~\cite{GedikKEK18}, \href{../works/ArbaouiY18.pdf}{ArbaouiY18}~\cite{ArbaouiY18}, \href{../works/TanT18.pdf}{TanT18}~\cite{TanT18}, \href{../works/GomesM17.pdf}{GomesM17}~\cite{GomesM17}, \href{../works/HebrardHJMPV16.pdf}{HebrardHJMPV16}~\cite{HebrardHJMPV16}, \href{../works/TranAB16.pdf}{TranAB16}~\cite{TranAB16}... (Total: 35) & \href{../works/PenzDN23.pdf}{PenzDN23}~\cite{PenzDN23}, \href{../works/JuvinHL23a.pdf}{JuvinHL23a}~\cite{JuvinHL23a}, \href{../works/Fatemi-AnarakiTFV23.pdf}{Fatemi-AnarakiTFV23}~\cite{Fatemi-AnarakiTFV23}, \href{../works/AbreuPNF23.pdf}{AbreuPNF23}~\cite{AbreuPNF23}, \href{../works/AbreuNP23.pdf}{AbreuNP23}~\cite{AbreuNP23}, \href{../works/Teppan22.pdf}{Teppan22}~\cite{Teppan22}, \href{../works/NaderiBZ22.pdf}{NaderiBZ22}~\cite{NaderiBZ22}, \href{../works/EmdeZD22.pdf}{EmdeZD22}~\cite{EmdeZD22}, \href{../works/ColT22.pdf}{ColT22}~\cite{ColT22}, \href{../works/Zahout21.pdf}{Zahout21}~\cite{Zahout21}, \href{../works/Bedhief21.pdf}{Bedhief21}~\cite{Bedhief21}, \href{../works/MokhtarzadehTNF20.pdf}{MokhtarzadehTNF20}~\cite{MokhtarzadehTNF20}, \href{../works/SacramentoSP20.pdf}{SacramentoSP20}~\cite{SacramentoSP20}, \href{../works/MejiaY20.pdf}{MejiaY20}~\cite{MejiaY20}, \href{../works/ParkUJR19.pdf}{ParkUJR19}~\cite{ParkUJR19}, \href{../works/Novas19.pdf}{Novas19}~\cite{Novas19}, \href{../works/BogaerdtW19.pdf}{BogaerdtW19}~\cite{BogaerdtW19}, \href{../works/Ham18a.pdf}{Ham18a}~\cite{Ham18a}, \href{../works/BenediktSMVH18.pdf}{BenediktSMVH18}~\cite{BenediktSMVH18}, \href{../works/RoshanaeiLAU17.pdf}{RoshanaeiLAU17}~\cite{RoshanaeiLAU17}, \href{../works/CatusseCBL16.pdf}{CatusseCBL16}~\cite{CatusseCBL16}, \href{../works/ZhouGL15.pdf}{ZhouGL15}~\cite{ZhouGL15}, \href{../works/TerekhovTDB14.pdf}{TerekhovTDB14}~\cite{TerekhovTDB14}, \href{../works/TranTDB13.pdf}{TranTDB13}~\cite{TranTDB13}, \href{../works/BajestaniB13.pdf}{BajestaniB13}~\cite{BajestaniB13}, \href{../works/GuyonLPR12.pdf}{GuyonLPR12}~\cite{GuyonLPR12}, \href{../works/KovacsB11.pdf}{KovacsB11}~\cite{KovacsB11}, \href{../works/AkkerDH07.pdf}{AkkerDH07}~\cite{AkkerDH07}, \href{../works/SadykovW06.pdf}{SadykovW06}~\cite{SadykovW06}, \href{../works/Thorsteinsson01.pdf}{Thorsteinsson01}~\cite{Thorsteinsson01} & \href{../works/KimCMLLP23.pdf}{KimCMLLP23}~\cite{KimCMLLP23}, \href{../works/GuoZ23.pdf}{GuoZ23}~\cite{GuoZ23}, \href{../works/JuvinHHL23.pdf}{JuvinHHL23}~\cite{JuvinHHL23}, \href{../works/LacknerMMWW23.pdf}{LacknerMMWW23}~\cite{LacknerMMWW23}, \href{../works/Mehdizadeh-Somarin23.pdf}{Mehdizadeh-Somarin23}~\cite{Mehdizadeh-Somarin23}, \href{../works/AlfieriGPS23.pdf}{AlfieriGPS23}~\cite{AlfieriGPS23}, \href{../works/JuvinHL22.pdf}{JuvinHL22}~\cite{JuvinHL22}, \href{../works/ArmstrongGOS22.pdf}{ArmstrongGOS22}~\cite{ArmstrongGOS22}, \href{../works/OrnekOS20.pdf}{OrnekOS20}~\cite{OrnekOS20}, \href{../works/EtminaniesfahaniGNMS22.pdf}{EtminaniesfahaniGNMS22}~\cite{EtminaniesfahaniGNMS22}, \href{../works/NaderiBZ22a.pdf}{NaderiBZ22a}~\cite{NaderiBZ22a}, \href{../works/HanenKP21.pdf}{HanenKP21}~\cite{HanenKP21}, \href{../works/FanXG21.pdf}{FanXG21}~\cite{FanXG21}, \href{../works/AbohashimaEG21.pdf}{AbohashimaEG21}~\cite{AbohashimaEG21}, \href{../works/AbreuAPNM21.pdf}{AbreuAPNM21}~\cite{AbreuAPNM21}, \href{../works/HamPK21.pdf}{HamPK21}~\cite{HamPK21}, \href{../works/LacknerMMWW21.pdf}{LacknerMMWW21}~\cite{LacknerMMWW21}, \href{../works/RoshanaeiBAUB20.pdf}{RoshanaeiBAUB20}~\cite{RoshanaeiBAUB20}, \href{../works/GroleazNS20a.pdf}{GroleazNS20a}~\cite{GroleazNS20a}, \href{../works/QinDCS20.pdf}{QinDCS20}~\cite{QinDCS20}, \href{../works/AstrandJZ20.pdf}{AstrandJZ20}~\cite{AstrandJZ20}, \href{../works/NishikawaSTT19.pdf}{NishikawaSTT19}~\cite{NishikawaSTT19}, \href{../works/Hooker19.pdf}{Hooker19}~\cite{Hooker19}, \href{../works/ArkhipovBL19.pdf}{ArkhipovBL19}~\cite{ArkhipovBL19}, \href{../works/Ham18.pdf}{Ham18}~\cite{Ham18}, \href{../works/BaptisteB18.pdf}{BaptisteB18}~\cite{BaptisteB18}, \href{../works/LaborieRSV18.pdf}{LaborieRSV18}~\cite{LaborieRSV18}, \href{../works/HookerH17.pdf}{HookerH17}~\cite{HookerH17}, \href{../works/KletzanderM17.pdf}{KletzanderM17}~\cite{KletzanderM17}... (Total: 49)\\
Classification & psplib & \href{../works/TardivoDFMP23.pdf}{TardivoDFMP23}~\cite{TardivoDFMP23}, \href{../works/Caballero19.pdf}{Caballero19}~\cite{Caballero19}, \href{../works/ArkhipovBL19.pdf}{ArkhipovBL19}~\cite{ArkhipovBL19}, \href{../works/KreterSSZ18.pdf}{KreterSSZ18}~\cite{KreterSSZ18}, \href{../works/OuelletQ18.pdf}{OuelletQ18}~\cite{OuelletQ18}, \href{../works/GayHS15a.pdf}{GayHS15a}~\cite{GayHS15a}, \href{../works/Derrien15.pdf}{Derrien15}~\cite{Derrien15}, \href{../works/LetortCB15.pdf}{LetortCB15}~\cite{LetortCB15}, \href{../works/KameugneFSN14.pdf}{KameugneFSN14}~\cite{KameugneFSN14}, \href{../works/DerrienP14.pdf}{DerrienP14}~\cite{DerrienP14}, \href{../works/Kameugne14.pdf}{Kameugne14}~\cite{Kameugne14}, \href{../works/SchuttFSW13.pdf}{SchuttFSW13}~\cite{SchuttFSW13}, \href{../works/SchuttFS13a.pdf}{SchuttFS13a}~\cite{SchuttFS13a}, \href{../works/HeinzSB13.pdf}{HeinzSB13}~\cite{HeinzSB13}, \href{../works/Letort13.pdf}{Letort13}~\cite{Letort13}, \href{../works/Clercq12.pdf}{Clercq12}~\cite{Clercq12}, \href{../works/SchuttFSW11.pdf}{SchuttFSW11}~\cite{SchuttFSW11}, \href{../works/Schutt11.pdf}{Schutt11}~\cite{Schutt11}, \href{../works/BertholdHLMS10.pdf}{BertholdHLMS10}~\cite{BertholdHLMS10}, \href{../works/SchuttFSW09.pdf}{SchuttFSW09}~\cite{SchuttFSW09}, \href{../works/Demassey03.pdf}{Demassey03}~\cite{Demassey03} & \href{../works/KameugneFND23.pdf}{KameugneFND23}~\cite{KameugneFND23}, \href{../works/BoudreaultSLQ22.pdf}{BoudreaultSLQ22}~\cite{BoudreaultSLQ22}, \href{../works/EtminaniesfahaniGNMS22.pdf}{EtminaniesfahaniGNMS22}~\cite{EtminaniesfahaniGNMS22}, \href{../works/HillTV21.pdf}{HillTV21}~\cite{HillTV21}, \href{../works/BadicaBI20.pdf}{BadicaBI20}~\cite{BadicaBI20}, \href{../works/Tesch18.pdf}{Tesch18}~\cite{Tesch18}, \href{../works/FahimiOQ18.pdf}{FahimiOQ18}~\cite{FahimiOQ18}, \href{../works/BaptisteB18.pdf}{BaptisteB18}~\cite{BaptisteB18}, \href{../works/Tesch16.pdf}{Tesch16}~\cite{Tesch16}, \href{../works/GingrasQ16.pdf}{GingrasQ16}~\cite{GingrasQ16}, \href{../works/Nattaf16.pdf}{Nattaf16}~\cite{Nattaf16}, \href{../works/SzerediS16.pdf}{SzerediS16}~\cite{SzerediS16}, \href{../works/VilimLS15.pdf}{VilimLS15}~\cite{VilimLS15}, \href{../works/GayHLS15.pdf}{GayHLS15}~\cite{GayHLS15}, \href{../works/LombardiBM15.pdf}{LombardiBM15}~\cite{LombardiBM15}, \href{../works/BonfiettiLM14.pdf}{BonfiettiLM14}~\cite{BonfiettiLM14}, \href{../works/LetortCB13.pdf}{LetortCB13}~\cite{LetortCB13}, \href{../works/LombardiM12a.pdf}{LombardiM12a}~\cite{LombardiM12a}, \href{../works/LetortBC12.pdf}{LetortBC12}~\cite{LetortBC12}, \href{../works/HeinzS11.pdf}{HeinzS11}~\cite{HeinzS11}, \href{../works/Vilim11.pdf}{Vilim11}~\cite{Vilim11}, \href{../works/abs-1009-0347.pdf}{abs-1009-0347}~\cite{abs-1009-0347}, \href{../works/SchuttW10.pdf}{SchuttW10}~\cite{SchuttW10} & \href{../works/Godet21a.pdf}{Godet21a}~\cite{Godet21a}, \href{../works/CauwelaertLS18.pdf}{CauwelaertLS18}~\cite{CauwelaertLS18}, \href{../works/LaborieRSV18.pdf}{LaborieRSV18}~\cite{LaborieRSV18}, \href{../works/YoungFS17.pdf}{YoungFS17}~\cite{YoungFS17}, \href{../works/Pralet17.pdf}{Pralet17}~\cite{Pralet17}, \href{../works/BofillCSV17.pdf}{BofillCSV17}~\cite{BofillCSV17}, \href{../works/Dejemeppe16.pdf}{Dejemeppe16}~\cite{Dejemeppe16}, \href{../works/SchnellH15.pdf}{SchnellH15}~\cite{SchnellH15}, \href{../works/ThiruvadyWGS14.pdf}{ThiruvadyWGS14}~\cite{ThiruvadyWGS14}, \href{../works/LombardiM13.pdf}{LombardiM13}~\cite{LombardiM13}, \href{../works/OuelletQ13.pdf}{OuelletQ13}~\cite{OuelletQ13}, \href{../works/LombardiM12.pdf}{LombardiM12}~\cite{LombardiM12}, \href{../works/KameugneFSN11.pdf}{KameugneFSN11}~\cite{KameugneFSN11}, \href{../works/LiessM08.pdf}{LiessM08}~\cite{LiessM08}, \href{../works/FortinZDF05.pdf}{FortinZDF05}~\cite{FortinZDF05}, \href{../works/DemasseyAM05.pdf}{DemasseyAM05}~\cite{DemasseyAM05}, \href{../works/ElkhyariGJ02a.pdf}{ElkhyariGJ02a}~\cite{ElkhyariGJ02a}\\
Classification & rtRTMP & \href{../works/MarliereSPR23.pdf}{MarliereSPR23}~\cite{MarliereSPR23} &  & \\
Classification & single machine & \href{../works/BonninMNE24.pdf}{BonninMNE24}~\cite{BonninMNE24}, \href{../works/PrataAN23.pdf}{PrataAN23}~\cite{PrataAN23}, \href{../works/AlfieriGPS23.pdf}{AlfieriGPS23}~\cite{AlfieriGPS23}, \href{../works/LacknerMMWW23.pdf}{LacknerMMWW23}~\cite{LacknerMMWW23}, \href{../works/PenzDN23.pdf}{PenzDN23}~\cite{PenzDN23}, \href{../works/TouatBT22.pdf}{TouatBT22}~\cite{TouatBT22}, \href{../works/HamPK21.pdf}{HamPK21}~\cite{HamPK21}, \href{../works/Groleaz21.pdf}{Groleaz21}~\cite{Groleaz21}, \href{../works/BenediktMH20.pdf}{BenediktMH20}~\cite{BenediktMH20}, \href{../works/ZarandiASC20.pdf}{ZarandiASC20}~\cite{ZarandiASC20}, \href{../works/BogaerdtW19.pdf}{BogaerdtW19}~\cite{BogaerdtW19}, \href{../works/BajestaniB15.pdf}{BajestaniB15}~\cite{BajestaniB15}, \href{../works/BajestaniB13.pdf}{BajestaniB13}~\cite{BajestaniB13}, \href{../works/TerekhovDOB12.pdf}{TerekhovDOB12}~\cite{TerekhovDOB12}, \href{../works/KovacsB11.pdf}{KovacsB11}~\cite{KovacsB11}, \href{../works/ThiruvadyBME09.pdf}{ThiruvadyBME09}~\cite{ThiruvadyBME09}, \href{../works/WuBB09.pdf}{WuBB09}~\cite{WuBB09}, \href{../works/KovacsB07.pdf}{KovacsB07}~\cite{KovacsB07}, \href{../works/SadykovW06.pdf}{SadykovW06}~\cite{SadykovW06}, \href{../works/KanetAG04.pdf}{KanetAG04}~\cite{KanetAG04}, \href{../works/Elkhyari03.pdf}{Elkhyari03}~\cite{Elkhyari03}, \href{../works/Baptiste02.pdf}{Baptiste02}~\cite{Baptiste02}, \href{../works/SourdN00.pdf}{SourdN00}~\cite{SourdN00}, \href{../works/BlazewiczDP96.pdf}{BlazewiczDP96}~\cite{BlazewiczDP96} & \href{../works/ZhangBB22.pdf}{ZhangBB22}~\cite{ZhangBB22}, \href{../works/EmdeZD22.pdf}{EmdeZD22}~\cite{EmdeZD22}, \href{../works/NaderiBZ22.pdf}{NaderiBZ22}~\cite{NaderiBZ22}, \href{../works/ElciOH22.pdf}{ElciOH22}~\cite{ElciOH22}, \href{../works/YuraszeckMPV22.pdf}{YuraszeckMPV22}~\cite{YuraszeckMPV22}, \href{../works/Bedhief21.pdf}{Bedhief21}~\cite{Bedhief21}, \href{../works/KoehlerBFFHPSSS21.pdf}{KoehlerBFFHPSSS21}~\cite{KoehlerBFFHPSSS21}, \href{../works/LacknerMMWW21.pdf}{LacknerMMWW21}~\cite{LacknerMMWW21}, \href{../works/PandeyS21a.pdf}{PandeyS21a}~\cite{PandeyS21a}, \href{../works/Astrand21.pdf}{Astrand21}~\cite{Astrand21}, \href{../works/HillTV21.pdf}{HillTV21}~\cite{HillTV21}, \href{../works/Zahout21.pdf}{Zahout21}~\cite{Zahout21}, \href{../works/AbreuAPNM21.pdf}{AbreuAPNM21}~\cite{AbreuAPNM21}, \href{../works/NattafM20.pdf}{NattafM20}~\cite{NattafM20}, \href{../works/Lunardi20.pdf}{Lunardi20}~\cite{Lunardi20}, \href{../works/BenediktSMVH18.pdf}{BenediktSMVH18}~\cite{BenediktSMVH18}, \href{../works/Tesch18.pdf}{Tesch18}~\cite{Tesch18}, \href{../works/TranPZLDB18.pdf}{TranPZLDB18}~\cite{TranPZLDB18}, \href{../works/TanT18.pdf}{TanT18}~\cite{TanT18}, \href{../works/GomesM17.pdf}{GomesM17}~\cite{GomesM17}, \href{../works/TranAB16.pdf}{TranAB16}~\cite{TranAB16}, \href{../works/KoschB14.pdf}{KoschB14}~\cite{KoschB14}, \href{../works/BillautHL12.pdf}{BillautHL12}~\cite{BillautHL12}, \href{../works/TranB12.pdf}{TranB12}~\cite{TranB12}, \href{../works/KovacsK11.pdf}{KovacsK11}~\cite{KovacsK11}, \href{../works/Malapert11.pdf}{Malapert11}~\cite{Malapert11}, \href{../works/MilanoW09.pdf}{MilanoW09}~\cite{MilanoW09}, \href{../works/Jans09.pdf}{Jans09}~\cite{Jans09}, \href{../works/AkkerDH07.pdf}{AkkerDH07}~\cite{AkkerDH07}... (Total: 35) & \href{../works/abs-2402-00459.pdf}{abs-2402-00459}~\cite{abs-2402-00459}, \href{../works/IsikYA23.pdf}{IsikYA23}~\cite{IsikYA23}, \href{../works/NaderiRR23.pdf}{NaderiRR23}~\cite{NaderiRR23}, \href{../works/Fatemi-AnarakiTFV23.pdf}{Fatemi-AnarakiTFV23}~\cite{Fatemi-AnarakiTFV23}, \href{../works/JuvinHL23a.pdf}{JuvinHL23a}~\cite{JuvinHL23a}, \href{../works/Mehdizadeh-Somarin23.pdf}{Mehdizadeh-Somarin23}~\cite{Mehdizadeh-Somarin23}, \href{../works/GeitzGSSW22.pdf}{GeitzGSSW22}~\cite{GeitzGSSW22}, \href{../works/JuvinHL22.pdf}{JuvinHL22}~\cite{JuvinHL22}, \href{../works/ZhangJZL22.pdf}{ZhangJZL22}~\cite{ZhangJZL22}, \href{../works/AbreuN22.pdf}{AbreuN22}~\cite{AbreuN22}, \href{../works/ColT22.pdf}{ColT22}~\cite{ColT22}, \href{../works/abs-2211-14492.pdf}{abs-2211-14492}~\cite{abs-2211-14492}, \href{../works/PohlAK22.pdf}{PohlAK22}~\cite{PohlAK22}, \href{../works/LiFJZLL22.pdf}{LiFJZLL22}~\cite{LiFJZLL22}, \href{../works/Godet21a.pdf}{Godet21a}~\cite{Godet21a}, \href{../works/FanXG21.pdf}{FanXG21}~\cite{FanXG21}, \href{../works/QinWSLS21.pdf}{QinWSLS21}~\cite{QinWSLS21}, \href{../works/KovacsTKSG21.pdf}{KovacsTKSG21}~\cite{KovacsTKSG21}, \href{../works/GodetLHS20.pdf}{GodetLHS20}~\cite{GodetLHS20}, \href{../works/TangB20.pdf}{TangB20}~\cite{TangB20}, \href{../works/ParkUJR19.pdf}{ParkUJR19}~\cite{ParkUJR19}, \href{../works/Tom19.pdf}{Tom19}~\cite{Tom19}, \href{../works/HoundjiSW19.pdf}{HoundjiSW19}~\cite{HoundjiSW19}, \href{../works/NattafDYW19.pdf}{NattafDYW19}~\cite{NattafDYW19}, \href{../works/NattafHKAL19.pdf}{NattafHKAL19}~\cite{NattafHKAL19}, \href{../works/Hooker19.pdf}{Hooker19}~\cite{Hooker19}, \href{../works/MalapertN19.pdf}{MalapertN19}~\cite{MalapertN19}, \href{../works/GedikKEK18.pdf}{GedikKEK18}~\cite{GedikKEK18}, \href{../works/ArbaouiY18.pdf}{ArbaouiY18}~\cite{ArbaouiY18}... (Total: 84)\\
\end{longtable}
}


\clearpage
\subsection{Concept Type Constraints}
\label{sec:Constraints}
{\scriptsize
\begin{longtable}{lp{3cm}>{\raggedright\arraybackslash}p{6cm}>{\raggedright\arraybackslash}p{6cm}>{\raggedright\arraybackslash}p{8cm}}
\rowcolor{white}\caption{Works for Concepts of Type Constraints}\\ \toprule
\rowcolor{white}Type & Keyword & High & Medium & Low\\ \midrule\endhead
\bottomrule
\endfoot
Constraints & AllDiff constraint & \href{../works/WangB20.pdf}{WangB20}~\cite{WangB20} &  & \href{../works/Godet21a.pdf}{Godet21a}~\cite{Godet21a}, \href{../works/FahimiOQ18.pdf}{FahimiOQ18}~\cite{FahimiOQ18}, \href{../works/Fahimi16.pdf}{Fahimi16}~\cite{Fahimi16}, \href{../works/Lombardi10.pdf}{Lombardi10}~\cite{Lombardi10}\\
Constraints & AllDiffPrec constraint & \href{../works/Godet21a.pdf}{Godet21a}~\cite{Godet21a} &  & \href{../works/JuvinHHL23.pdf}{JuvinHHL23}~\cite{JuvinHHL23}\\
Constraints & AlwaysConstant &  & \href{../works/LuoB22.pdf}{LuoB22}~\cite{LuoB22}, \href{../works/LaborieRSV18.pdf}{LaborieRSV18}~\cite{LaborieRSV18} & \\
Constraints & Among constraint & \href{../works/Siala15a.pdf}{Siala15a}~\cite{Siala15a}, \href{../works/Siala15.pdf}{Siala15}~\cite{Siala15}, \href{../works/BeldiceanuC94.pdf}{BeldiceanuC94}~\cite{BeldiceanuC94} & \href{../works/Simonis07.pdf}{Simonis07}~\cite{Simonis07}, \href{../works/BosiM2001.pdf}{BosiM2001}~\cite{BosiM2001} & \href{../works/German18.pdf}{German18}~\cite{German18}, \href{../works/HookerH17.pdf}{HookerH17}~\cite{HookerH17}, \href{../works/Refalo00.pdf}{Refalo00}~\cite{Refalo00}, \href{../works/Simonis95.pdf}{Simonis95}~\cite{Simonis95}, \href{../works/AggounB93.pdf}{AggounB93}~\cite{AggounB93}\\
Constraints & AmongSeq constraint &  & \href{../works/Siala15.pdf}{Siala15}~\cite{Siala15}, \href{../works/Siala15a.pdf}{Siala15a}~\cite{Siala15a} & \\
Constraints & Arithmetic constraint &  & \href{../works/ColT22.pdf}{ColT22}~\cite{ColT22} & \href{../works/BadicaBI20.pdf}{BadicaBI20}~\cite{BadicaBI20}, \href{../works/Caballero19.pdf}{Caballero19}~\cite{Caballero19}, \href{../works/BadicaBIL19.pdf}{BadicaBIL19}~\cite{BadicaBIL19}, \href{../works/LaborieRSV18.pdf}{LaborieRSV18}~\cite{LaborieRSV18}, \href{../works/Schutt11.pdf}{Schutt11}~\cite{Schutt11}, \href{../works/OhrimenkoSC09.pdf}{OhrimenkoSC09}~\cite{OhrimenkoSC09}, \href{../works/Kuchcinski03.pdf}{Kuchcinski03}~\cite{Kuchcinski03}, \href{../works/Baptiste02.pdf}{Baptiste02}~\cite{Baptiste02}, \href{../works/ElkhyariGJ02a.pdf}{ElkhyariGJ02a}~\cite{ElkhyariGJ02a}, \href{../works/Thorsteinsson01.pdf}{Thorsteinsson01}~\cite{Thorsteinsson01}, \href{../works/Refalo00.pdf}{Refalo00}~\cite{Refalo00}, \href{../works/SakkoutW00.pdf}{SakkoutW00}~\cite{SakkoutW00}, \href{../works/AbdennadherS99.pdf}{AbdennadherS99}~\cite{AbdennadherS99}, \href{../works/FalaschiGMP97.pdf}{FalaschiGMP97}~\cite{FalaschiGMP97}, \href{../works/BeldiceanuC94.pdf}{BeldiceanuC94}~\cite{BeldiceanuC94}, \href{../works/AggounB93.pdf}{AggounB93}~\cite{AggounB93}\\
Constraints & AtMostSeq & \href{../works/Siala15a.pdf}{Siala15a}~\cite{Siala15a}, \href{../works/Siala15.pdf}{Siala15}~\cite{Siala15} &  & \\
Constraints & AtMostSeqCard & \href{../works/Siala15.pdf}{Siala15}~\cite{Siala15}, \href{../works/Siala15a.pdf}{Siala15a}~\cite{Siala15a} &  & \\
Constraints & Atmost constraint & \href{../works/Siala15a.pdf}{Siala15a}~\cite{Siala15a}, \href{../works/Siala15.pdf}{Siala15}~\cite{Siala15} &  & \href{../works/Simonis07.pdf}{Simonis07}~\cite{Simonis07}, \href{../works/BeldiceanuC94.pdf}{BeldiceanuC94}~\cite{BeldiceanuC94}\\
Constraints & Balance constraint & \href{../works/RoePS05.pdf}{RoePS05}~\cite{RoePS05}, \href{../works/Laborie03.pdf}{Laborie03}~\cite{Laborie03} & \href{../works/Timpe02.pdf}{Timpe02}~\cite{Timpe02}, \href{../works/Muscettola02.pdf}{Muscettola02}~\cite{Muscettola02} & \href{../works/GuoZ23.pdf}{GuoZ23}~\cite{GuoZ23}, \href{../works/PopovicCGNC22.pdf}{PopovicCGNC22}~\cite{PopovicCGNC22}, \href{../works/German18.pdf}{German18}~\cite{German18}, \href{../works/SchuttS16.pdf}{SchuttS16}~\cite{SchuttS16}, \href{../works/Siala15.pdf}{Siala15}~\cite{Siala15}, \href{../works/Siala15a.pdf}{Siala15a}~\cite{Siala15a}, \href{../works/GrimesH15.pdf}{GrimesH15}~\cite{GrimesH15}, \href{../works/Kameugne14.pdf}{Kameugne14}~\cite{Kameugne14}, \href{../works/DerrienPZ14.pdf}{DerrienPZ14}~\cite{DerrienPZ14}, \href{../works/TerekhovDOB12.pdf}{TerekhovDOB12}~\cite{TerekhovDOB12}, \href{../works/Lombardi10.pdf}{Lombardi10}~\cite{Lombardi10}, \href{../works/GrimesHM09.pdf}{GrimesHM09}~\cite{GrimesHM09}, \href{../works/LombardiM09.pdf}{LombardiM09}~\cite{LombardiM09}, \href{../works/BeckW07.pdf}{BeckW07}~\cite{BeckW07}, \href{../works/BeckW05.pdf}{BeckW05}~\cite{BeckW05}, \href{../works/MaraveliasCG04.pdf}{MaraveliasCG04}~\cite{MaraveliasCG04}\\
Constraints & BinPacking constraint &  &  & \href{../works/Godet21a.pdf}{Godet21a}~\cite{Godet21a}, \href{../works/AntunesABD18.pdf}{AntunesABD18}~\cite{AntunesABD18}\\
Constraints & Blocking constraint & \href{../works/AbreuNP23.pdf}{AbreuNP23}~\cite{AbreuNP23}, \href{../works/RiahiNS018.pdf}{RiahiNS018}~\cite{RiahiNS018} &  & \href{../works/IsikYA23.pdf}{IsikYA23}~\cite{IsikYA23}, \href{../works/LiFJZLL22.pdf}{LiFJZLL22}~\cite{LiFJZLL22}, \href{../works/MengZRZL20.pdf}{MengZRZL20}~\cite{MengZRZL20}, \href{../works/RodriguezS09.pdf}{RodriguezS09}~\cite{RodriguezS09}, \href{../works/Rodriguez07b.pdf}{Rodriguez07b}~\cite{Rodriguez07b}, \href{../works/Rodriguez07.pdf}{Rodriguez07}~\cite{Rodriguez07}\\
Constraints & BufferedResource & \href{../works/BessiereHMQW14.pdf}{BessiereHMQW14}~\cite{BessiereHMQW14} &  & \\
Constraints & Calendar constraint & \href{../works/KreterSSZ18.pdf}{KreterSSZ18}~\cite{KreterSSZ18}, \href{../works/KreterSS17.pdf}{KreterSS17}~\cite{KreterSS17} & \href{../works/KreterSS15.pdf}{KreterSS15}~\cite{KreterSS15} & \href{../works/PovedaAA23.pdf}{PovedaAA23}~\cite{PovedaAA23}, \href{../works/IsikYA23.pdf}{IsikYA23}~\cite{IsikYA23}, \href{../works/Polo-MejiaALB20.pdf}{Polo-MejiaALB20}~\cite{Polo-MejiaALB20}, \href{../works/LaborieRSV18.pdf}{LaborieRSV18}~\cite{LaborieRSV18}\\
Constraints & CardPath &  &  & \href{../works/Siala15a.pdf}{Siala15a}~\cite{Siala15a}, \href{../works/Siala15.pdf}{Siala15}~\cite{Siala15}\\
Constraints & Cardinality constraint & \href{../works/Caballero19.pdf}{Caballero19}~\cite{Caballero19}, \href{../works/Dejemeppe16.pdf}{Dejemeppe16}~\cite{Dejemeppe16}, \href{../works/Siala15.pdf}{Siala15}~\cite{Siala15}, \href{../works/Siala15a.pdf}{Siala15a}~\cite{Siala15a}, \href{../works/SchausHMCMD11.pdf}{SchausHMCMD11}~\cite{SchausHMCMD11}, \href{../works/Malik08.pdf}{Malik08}~\cite{Malik08} & \href{../works/OuelletQ22.pdf}{OuelletQ22}~\cite{OuelletQ22}, \href{../works/HoundjiSW19.pdf}{HoundjiSW19}~\cite{HoundjiSW19}, \href{../works/German18.pdf}{German18}~\cite{German18}, \href{../works/MusliuSS18.pdf}{MusliuSS18}~\cite{MusliuSS18}, \href{../works/HookerH17.pdf}{HookerH17}~\cite{HookerH17}, \href{../works/Fahimi16.pdf}{Fahimi16}~\cite{Fahimi16}, \href{../works/BofillGSV15.pdf}{BofillGSV15}~\cite{BofillGSV15}, \href{../works/HoundjiSWD14.pdf}{HoundjiSWD14}~\cite{HoundjiSWD14}, \href{../works/ChuGNSW13.pdf}{ChuGNSW13}~\cite{ChuGNSW13}, \href{../works/HachemiGR11.pdf}{HachemiGR11}~\cite{HachemiGR11}, \href{../works/MilanoW09.pdf}{MilanoW09}~\cite{MilanoW09}, \href{../works/MalikMB08.pdf}{MalikMB08}~\cite{MalikMB08}, \href{../works/Simonis07.pdf}{Simonis07}~\cite{Simonis07}, \href{../works/MilanoW06.pdf}{MilanoW06}~\cite{MilanoW06}, \href{../works/Gronkvist06.pdf}{Gronkvist06}~\cite{Gronkvist06} & \href{../works/Godet21a.pdf}{Godet21a}~\cite{Godet21a}, \href{../works/Lemos21.pdf}{Lemos21}~\cite{Lemos21}, \href{../works/GeibingerKKMMW21.pdf}{GeibingerKKMMW21}~\cite{GeibingerKKMMW21}, \href{../works/CauwelaertDS20.pdf}{CauwelaertDS20}~\cite{CauwelaertDS20}, \href{../works/TangB20.pdf}{TangB20}~\cite{TangB20}, \href{../works/abs-1911-04766.pdf}{abs-1911-04766}~\cite{abs-1911-04766}, \href{../works/TranVNB17.pdf}{TranVNB17}~\cite{TranVNB17}, \href{../works/PesantRR15.pdf}{PesantRR15}~\cite{PesantRR15}, \href{../works/DoulabiRP14.pdf}{DoulabiRP14}~\cite{DoulabiRP14}, \href{../works/BessiereHMQW14.pdf}{BessiereHMQW14}~\cite{BessiereHMQW14}, \href{../works/BajestaniB13.pdf}{BajestaniB13}~\cite{BajestaniB13}, \href{../works/LimtanyakulS12.pdf}{LimtanyakulS12}~\cite{LimtanyakulS12}, \href{../works/Menana11.pdf}{Menana11}~\cite{Menana11}, \href{../works/BajestaniB11.pdf}{BajestaniB11}~\cite{BajestaniB11}, \href{../works/ClercqPBJ11.pdf}{ClercqPBJ11}~\cite{ClercqPBJ11}, \href{../works/KovacsB11.pdf}{KovacsB11}~\cite{KovacsB11}, \href{../works/abs-0907-0939.pdf}{abs-0907-0939}~\cite{abs-0907-0939}, \href{../works/OhrimenkoSC09.pdf}{OhrimenkoSC09}~\cite{OhrimenkoSC09}, \href{../works/KovacsB08.pdf}{KovacsB08}~\cite{KovacsB08}, \href{../works/CambazardHDJT04.pdf}{CambazardHDJT04}~\cite{CambazardHDJT04}, \href{../works/BourdaisGP03.pdf}{BourdaisGP03}~\cite{BourdaisGP03}, \href{../works/Baptiste02.pdf}{Baptiste02}~\cite{Baptiste02}, \href{../works/Refalo00.pdf}{Refalo00}~\cite{Refalo00}, \href{../works/HookerOTK00.pdf}{HookerOTK00}~\cite{HookerOTK00}, \href{../works/BeckF00.pdf}{BeckF00}~\cite{BeckF00}, \href{../works/AbdennadherS99.pdf}{AbdennadherS99}~\cite{AbdennadherS99}, \href{../works/PapaB98.pdf}{PapaB98}~\cite{PapaB98}, \href{../works/WeilHFP95.pdf}{WeilHFP95}~\cite{WeilHFP95}, \href{../works/AggounB93.pdf}{AggounB93}~\cite{AggounB93}\\
Constraints & Channeling constraint & \href{../works/OzturkTHO13.pdf}{OzturkTHO13}~\cite{OzturkTHO13}, \href{../works/Wallace06.pdf}{Wallace06}~\cite{Wallace06} & \href{../works/KoehlerBFFHPSSS21.pdf}{KoehlerBFFHPSSS21}~\cite{KoehlerBFFHPSSS21}, \href{../works/BofillEGPSV14.pdf}{BofillEGPSV14}~\cite{BofillEGPSV14}, \href{../works/HeinzB12.pdf}{HeinzB12}~\cite{HeinzB12} & \href{../works/WangB23.pdf}{WangB23}~\cite{WangB23}, \href{../works/AntuoriHHEN20.pdf}{AntuoriHHEN20}~\cite{AntuoriHHEN20}, \href{../works/LiuLH19.pdf}{LiuLH19}~\cite{LiuLH19}, \href{../works/GokgurHO18.pdf}{GokgurHO18}~\cite{GokgurHO18}, \href{../works/BofillGSV15.pdf}{BofillGSV15}~\cite{BofillGSV15}, \href{../works/HeinzKB13.pdf}{HeinzKB13}~\cite{HeinzKB13}, \href{../works/KovacsB11.pdf}{KovacsB11}~\cite{KovacsB11}, \href{../works/WuBB09.pdf}{WuBB09}~\cite{WuBB09}, \href{../works/MilanoW09.pdf}{MilanoW09}~\cite{MilanoW09}, \href{../works/MouraSCL08.pdf}{MouraSCL08}~\cite{MouraSCL08}, \href{../works/MouraSCL08a.pdf}{MouraSCL08a}~\cite{MouraSCL08a}, \href{../works/GarganiR07.pdf}{GarganiR07}~\cite{GarganiR07}, \href{../works/MilanoW06.pdf}{MilanoW06}~\cite{MilanoW06}, \href{../works/CambazardHDJT04.pdf}{CambazardHDJT04}~\cite{CambazardHDJT04}\\
Constraints & Completion constraint & \href{../works/KovacsB11.pdf}{KovacsB11}~\cite{KovacsB11}, \href{../works/KovacsB08.pdf}{KovacsB08}~\cite{KovacsB08}, \href{../works/KovacsB07.pdf}{KovacsB07}~\cite{KovacsB07} & \href{../works/BonninMNE24.pdf}{BonninMNE24}~\cite{BonninMNE24} & \href{../works/HeckmanB11.pdf}{HeckmanB11}~\cite{HeckmanB11}\\
Constraints & CumulativeCost & \href{../works/SimonisH11.pdf}{SimonisH11}~\cite{SimonisH11} &  & \\
Constraints & Cumulatives constraint & \href{../works/BeldiceanuC02.pdf}{BeldiceanuC02}~\cite{BeldiceanuC02} & \href{../works/MossigeGSMC17.pdf}{MossigeGSMC17}~\cite{MossigeGSMC17}, \href{../works/Madi-WambaLOBM17.pdf}{Madi-WambaLOBM17}~\cite{Madi-WambaLOBM17} & \href{../works/KameugneFND23.pdf}{KameugneFND23}~\cite{KameugneFND23}, \href{../works/TardivoDFMP23.pdf}{TardivoDFMP23}~\cite{TardivoDFMP23}, \href{../works/OuelletQ22.pdf}{OuelletQ22}~\cite{OuelletQ22}, \href{../works/BoudreaultSLQ22.pdf}{BoudreaultSLQ22}~\cite{BoudreaultSLQ22}, \href{../works/ArkhipovBL19.pdf}{ArkhipovBL19}~\cite{ArkhipovBL19}, \href{../works/OuelletQ18.pdf}{OuelletQ18}~\cite{OuelletQ18}, \href{../works/FahimiOQ18.pdf}{FahimiOQ18}~\cite{FahimiOQ18}, \href{../works/Fahimi16.pdf}{Fahimi16}~\cite{Fahimi16}, \href{../works/SchuttS16.pdf}{SchuttS16}~\cite{SchuttS16}, \href{../works/Dejemeppe16.pdf}{Dejemeppe16}~\cite{Dejemeppe16}, \href{../works/GayHS15a.pdf}{GayHS15a}~\cite{GayHS15a}, \href{../works/LetortCB15.pdf}{LetortCB15}~\cite{LetortCB15}, \href{../works/GayHS15.pdf}{GayHS15}~\cite{GayHS15}, \href{../works/CauwelaertLS15.pdf}{CauwelaertLS15}~\cite{CauwelaertLS15}, \href{../works/Kameugne14.pdf}{Kameugne14}~\cite{Kameugne14}, \href{../works/DerrienPZ14.pdf}{DerrienPZ14}~\cite{DerrienPZ14}, \href{../works/Letort13.pdf}{Letort13}~\cite{Letort13}, \href{../works/OuelletQ13.pdf}{OuelletQ13}~\cite{OuelletQ13}, \href{../works/Clercq12.pdf}{Clercq12}~\cite{Clercq12}, \href{../works/LetortBC12.pdf}{LetortBC12}~\cite{LetortBC12}, \href{../works/SimonisH11.pdf}{SimonisH11}~\cite{SimonisH11}, \href{../works/ClercqPBJ11.pdf}{ClercqPBJ11}~\cite{ClercqPBJ11}, \href{../works/Malapert11.pdf}{Malapert11}~\cite{Malapert11}, \href{../works/Wolf11.pdf}{Wolf11}~\cite{Wolf11}, \href{../works/MilanoW09.pdf}{MilanoW09}~\cite{MilanoW09}, \href{../works/abs-0907-0939.pdf}{abs-0907-0939}~\cite{abs-0907-0939}, \href{../works/Simonis07.pdf}{Simonis07}~\cite{Simonis07}, \href{../works/MilanoW06.pdf}{MilanoW06}~\cite{MilanoW06}\\
Constraints & Diff2 constraint & \href{../works/Kuchcinski03.pdf}{Kuchcinski03}~\cite{Kuchcinski03} &  & \href{../works/WolinskiKG04.pdf}{WolinskiKG04}~\cite{WolinskiKG04}, \href{../works/KuchcinskiW03.pdf}{KuchcinskiW03}~\cite{KuchcinskiW03}\\
Constraints & Disjunctive constraint & \href{../works/KoehlerBFFHPSSS21.pdf}{KoehlerBFFHPSSS21}~\cite{KoehlerBFFHPSSS21}, \href{../works/Godet21a.pdf}{Godet21a}~\cite{Godet21a}, \href{../works/GrimesH15.pdf}{GrimesH15}~\cite{GrimesH15}, \href{../works/Malapert11.pdf}{Malapert11}~\cite{Malapert11}, \href{../works/RoePS05.pdf}{RoePS05}~\cite{RoePS05}, \href{../works/Baptiste02.pdf}{Baptiste02}~\cite{Baptiste02}, \href{../works/Dorndorf2000.pdf}{Dorndorf2000}~\cite{Dorndorf2000}, \href{../works/SourdN00.pdf}{SourdN00}~\cite{SourdN00}, \href{../works/HookerO99.pdf}{HookerO99}~\cite{HookerO99}, \href{../works/RodosekWH99.pdf}{RodosekWH99}~\cite{RodosekWH99}, \href{../works/PapaB98.pdf}{PapaB98}~\cite{PapaB98}, \href{../works/RodosekW98.pdf}{RodosekW98}~\cite{RodosekW98}, \href{../works/Zhou97.pdf}{Zhou97}~\cite{Zhou97}, \href{../works/DincbasSH90.pdf}{DincbasSH90}~\cite{DincbasSH90} & \href{../works/BonninMNE24.pdf}{BonninMNE24}~\cite{BonninMNE24}, \href{../works/JuvinHHL23.pdf}{JuvinHHL23}~\cite{JuvinHHL23}, \href{../works/NaderiRR23.pdf}{NaderiRR23}~\cite{NaderiRR23}, \href{../works/BourreauGGLT22.pdf}{BourreauGGLT22}~\cite{BourreauGGLT22}, \href{../works/GodetLHS20.pdf}{GodetLHS20}~\cite{GodetLHS20}, \href{../works/GokgurHO18.pdf}{GokgurHO18}~\cite{GokgurHO18}, \href{../works/KuB16.pdf}{KuB16}~\cite{KuB16}, \href{../works/Fahimi16.pdf}{Fahimi16}~\cite{Fahimi16}, \href{../works/Siala15a.pdf}{Siala15a}~\cite{Siala15a}, \href{../works/MelgarejoLS15.pdf}{MelgarejoLS15}~\cite{MelgarejoLS15}, \href{../works/Siala15.pdf}{Siala15}~\cite{Siala15}, \href{../works/SialaAH15.pdf}{SialaAH15}~\cite{SialaAH15}, \href{../works/SchuttFS13.pdf}{SchuttFS13}~\cite{SchuttFS13}, \href{../works/OzturkTHO13.pdf}{OzturkTHO13}~\cite{OzturkTHO13}, \href{../works/GrimesH11.pdf}{GrimesH11}~\cite{GrimesH11}, \href{../works/LombardiM10a.pdf}{LombardiM10a}~\cite{LombardiM10a}, \href{../works/Lombardi10.pdf}{Lombardi10}~\cite{Lombardi10}, \href{../works/BartakSR10.pdf}{BartakSR10}~\cite{BartakSR10}, \href{../works/GrimesH10.pdf}{GrimesH10}~\cite{GrimesH10}, \href{../works/GrimesHM09.pdf}{GrimesHM09}~\cite{GrimesHM09}, \href{../works/BartakSR08.pdf}{BartakSR08}~\cite{BartakSR08}, \href{../works/ArtiguesBF04.pdf}{ArtiguesBF04}~\cite{ArtiguesBF04}, \href{../works/KanetAG04.pdf}{KanetAG04}~\cite{KanetAG04}, \href{../works/Kuchcinski03.pdf}{Kuchcinski03}~\cite{Kuchcinski03}, \href{../works/Laborie03.pdf}{Laborie03}~\cite{Laborie03}, \href{../works/ElkhyariGJ02a.pdf}{ElkhyariGJ02a}~\cite{ElkhyariGJ02a}, \href{../works/SchildW00.pdf}{SchildW00}~\cite{SchildW00}, \href{../works/FocacciLN00.pdf}{FocacciLN00}~\cite{FocacciLN00}, \href{../works/SakkoutW00.pdf}{SakkoutW00}~\cite{SakkoutW00}... (Total: 34) & \href{../works/abs-2402-00459.pdf}{abs-2402-00459}~\cite{abs-2402-00459}, \href{../works/KameugneFND23.pdf}{KameugneFND23}~\cite{KameugneFND23}, \href{../works/Bit-Monnot23.pdf}{Bit-Monnot23}~\cite{Bit-Monnot23}, \href{../works/MarliereSPR23.pdf}{MarliereSPR23}~\cite{MarliereSPR23}, \href{../works/JuvinHL23a.pdf}{JuvinHL23a}~\cite{JuvinHL23a}, \href{../works/NaderiBZ23.pdf}{NaderiBZ23}~\cite{NaderiBZ23}, \href{../works/NaderiBZ22a.pdf}{NaderiBZ22a}~\cite{NaderiBZ22a}, \href{../works/KotaryFH22.pdf}{KotaryFH22}~\cite{KotaryFH22}, \href{../works/JuvinHL22.pdf}{JuvinHL22}~\cite{JuvinHL22}, \href{../works/ZhangBB22.pdf}{ZhangBB22}~\cite{ZhangBB22}, \href{../works/abs-2211-14492.pdf}{abs-2211-14492}~\cite{abs-2211-14492}, \href{../works/BoudreaultSLQ22.pdf}{BoudreaultSLQ22}~\cite{BoudreaultSLQ22}, \href{../works/YuraszeckMPV22.pdf}{YuraszeckMPV22}~\cite{YuraszeckMPV22}, \href{../works/NaderiBZ22.pdf}{NaderiBZ22}~\cite{NaderiBZ22}, \href{../works/Groleaz21.pdf}{Groleaz21}~\cite{Groleaz21}, \href{../works/Astrand21.pdf}{Astrand21}~\cite{Astrand21}, \href{../works/Astrand0F21.pdf}{Astrand0F21}~\cite{Astrand0F21}, \href{../works/WallaceY20.pdf}{WallaceY20}~\cite{WallaceY20}, \href{../works/Polo-MejiaALB20.pdf}{Polo-MejiaALB20}~\cite{Polo-MejiaALB20}, \href{../works/MejiaY20.pdf}{MejiaY20}~\cite{MejiaY20}, \href{../works/AstrandJZ20.pdf}{AstrandJZ20}~\cite{AstrandJZ20}, \href{../works/German18.pdf}{German18}~\cite{German18}, \href{../works/TanT18.pdf}{TanT18}~\cite{TanT18}, \href{../works/FahimiOQ18.pdf}{FahimiOQ18}~\cite{FahimiOQ18}, \href{../works/LaborieRSV18.pdf}{LaborieRSV18}~\cite{LaborieRSV18}, \href{../works/KameugneFGOQ18.pdf}{KameugneFGOQ18}~\cite{KameugneFGOQ18}, \href{../works/DemirovicS18.pdf}{DemirovicS18}~\cite{DemirovicS18}, \href{../works/OrnekO16.pdf}{OrnekO16}~\cite{OrnekO16}, \href{../works/BoothTNB16.pdf}{BoothTNB16}~\cite{BoothTNB16}... (Total: 81)\\
Constraints & Element constraint & \href{../works/Dejemeppe16.pdf}{Dejemeppe16}~\cite{Dejemeppe16}, \href{../works/HookerOTK00.pdf}{HookerOTK00}~\cite{HookerOTK00} & \href{../works/KreterSS17.pdf}{KreterSS17}~\cite{KreterSS17}, \href{../works/Wolf11.pdf}{Wolf11}~\cite{Wolf11}, \href{../works/Kuchcinski03.pdf}{Kuchcinski03}~\cite{Kuchcinski03}, \href{../works/Darby-DowmanLMZ97.pdf}{Darby-DowmanLMZ97}~\cite{Darby-DowmanLMZ97} & \href{../works/LacknerMMWW23.pdf}{LacknerMMWW23}~\cite{LacknerMMWW23}, \href{../works/LuoB22.pdf}{LuoB22}~\cite{LuoB22}, \href{../works/Godet21a.pdf}{Godet21a}~\cite{Godet21a}, \href{../works/LacknerMMWW21.pdf}{LacknerMMWW21}~\cite{LacknerMMWW21}, \href{../works/TangB20.pdf}{TangB20}~\cite{TangB20}, \href{../works/AntuoriHHEN20.pdf}{AntuoriHHEN20}~\cite{AntuoriHHEN20}, \href{../works/KreterSSZ18.pdf}{KreterSSZ18}~\cite{KreterSSZ18}, \href{../works/LiuCGM17.pdf}{LiuCGM17}~\cite{LiuCGM17}, \href{../works/Madi-WambaLOBM17.pdf}{Madi-WambaLOBM17}~\cite{Madi-WambaLOBM17}, \href{../works/SzerediS16.pdf}{SzerediS16}~\cite{SzerediS16}, \href{../works/OrnekO16.pdf}{OrnekO16}~\cite{OrnekO16}, \href{../works/DoulabiRP16.pdf}{DoulabiRP16}~\cite{DoulabiRP16}, \href{../works/KreterSS15.pdf}{KreterSS15}~\cite{KreterSS15}, \href{../works/HoundjiSWD14.pdf}{HoundjiSWD14}~\cite{HoundjiSWD14}, \href{../works/BessiereHMQW14.pdf}{BessiereHMQW14}~\cite{BessiereHMQW14}, \href{../works/DoulabiRP14.pdf}{DoulabiRP14}~\cite{DoulabiRP14}, \href{../works/OzturkTHO12.pdf}{OzturkTHO12}~\cite{OzturkTHO12}, \href{../works/SimonisH11.pdf}{SimonisH11}~\cite{SimonisH11}, \href{../works/SchausHMCMD11.pdf}{SchausHMCMD11}~\cite{SchausHMCMD11}, \href{../works/Malapert11.pdf}{Malapert11}~\cite{Malapert11}, \href{../works/Schutt11.pdf}{Schutt11}~\cite{Schutt11}, \href{../works/MouraSCL08.pdf}{MouraSCL08}~\cite{MouraSCL08}, \href{../works/SchausD08.pdf}{SchausD08}~\cite{SchausD08}, \href{../works/GarganiR07.pdf}{GarganiR07}~\cite{GarganiR07}, \href{../works/CambazardHDJT04.pdf}{CambazardHDJT04}~\cite{CambazardHDJT04}, \href{../works/Refalo00.pdf}{Refalo00}~\cite{Refalo00}, \href{../works/BeldiceanuC94.pdf}{BeldiceanuC94}~\cite{BeldiceanuC94}\\
Constraints & Flowtime constraint & \href{../works/BonninMNE24.pdf}{BonninMNE24}~\cite{BonninMNE24} &  & \\
Constraints & GCC constraint & \href{../works/HoundjiSW19.pdf}{HoundjiSW19}~\cite{HoundjiSW19}, \href{../works/Dejemeppe16.pdf}{Dejemeppe16}~\cite{Dejemeppe16}, \href{../works/HoundjiSWD14.pdf}{HoundjiSWD14}~\cite{HoundjiSWD14} & \href{../works/SchausHMCMD11.pdf}{SchausHMCMD11}~\cite{SchausHMCMD11} & \href{../works/OuelletQ22.pdf}{OuelletQ22}~\cite{OuelletQ22}, \href{../works/TangB20.pdf}{TangB20}~\cite{TangB20}, \href{../works/CauwelaertLS18.pdf}{CauwelaertLS18}~\cite{CauwelaertLS18}, \href{../works/Siala15.pdf}{Siala15}~\cite{Siala15}, \href{../works/Siala15a.pdf}{Siala15a}~\cite{Siala15a}, \href{../works/CauwelaertLS15.pdf}{CauwelaertLS15}~\cite{CauwelaertLS15}, \href{../works/BajestaniB13.pdf}{BajestaniB13}~\cite{BajestaniB13}, \href{../works/HachemiGR11.pdf}{HachemiGR11}~\cite{HachemiGR11}, \href{../works/MilanoW09.pdf}{MilanoW09}~\cite{MilanoW09}, \href{../works/Simonis07.pdf}{Simonis07}~\cite{Simonis07}, \href{../works/Gronkvist06.pdf}{Gronkvist06}~\cite{Gronkvist06}, \href{../works/MilanoW06.pdf}{MilanoW06}~\cite{MilanoW06}\\
Constraints & GeneralizedAllDiffPrec & \href{../works/Godet21a.pdf}{Godet21a}~\cite{Godet21a} &  & \\
Constraints & IloAlternative &  &  & \href{../works/HeinzB12.pdf}{HeinzB12}~\cite{HeinzB12}\\
Constraints & IloAlwaysIn &  &  & \href{../works/KreterSS17.pdf}{KreterSS17}~\cite{KreterSS17}, \href{../works/BajestaniB13.pdf}{BajestaniB13}~\cite{BajestaniB13}\\
Constraints & IloForbidEnd &  &  & \href{../works/KreterSS17.pdf}{KreterSS17}~\cite{KreterSS17}\\
Constraints & IloNoOverlap &  &  & \href{../works/GrimesH15.pdf}{GrimesH15}~\cite{GrimesH15}\\
Constraints & IloPack &  & \href{../works/SchausD08.pdf}{SchausD08}~\cite{SchausD08} & \\
Constraints & IloPulse &  &  & \href{../works/KreterSS17.pdf}{KreterSS17}~\cite{KreterSS17}, \href{../works/BajestaniB13.pdf}{BajestaniB13}~\cite{BajestaniB13}\\
Constraints & MinWeightAllDiff & \href{../works/WangB20.pdf}{WangB20}~\cite{WangB20} &  & \href{../works/WangB23.pdf}{WangB23}~\cite{WangB23}\\
Constraints & MultiAtMostSeqCard & \href{../works/Siala15.pdf}{Siala15}~\cite{Siala15}, \href{../works/Siala15a.pdf}{Siala15a}~\cite{Siala15a} &  & \\
Constraints & PreemptiveNoOverlap & \href{../works/JuvinHHL23.pdf}{JuvinHHL23}~\cite{JuvinHHL23} &  & \\
Constraints & Pulse constraint &  &  & \href{../works/PandeyS21a.pdf}{PandeyS21a}~\cite{PandeyS21a}, \href{../works/GeibingerMM19.pdf}{GeibingerMM19}~\cite{GeibingerMM19}, \href{../works/ArbaouiY18.pdf}{ArbaouiY18}~\cite{ArbaouiY18}, \href{../works/KreterSS17.pdf}{KreterSS17}~\cite{KreterSS17}\\
Constraints & Regular constraint & \href{../works/MusliuSS18.pdf}{MusliuSS18}~\cite{MusliuSS18}, \href{../works/Siala15.pdf}{Siala15}~\cite{Siala15}, \href{../works/Siala15a.pdf}{Siala15a}~\cite{Siala15a}, \href{../works/PesantRR15.pdf}{PesantRR15}~\cite{PesantRR15} & \href{../works/HookerH17.pdf}{HookerH17}~\cite{HookerH17}, \href{../works/Dejemeppe16.pdf}{Dejemeppe16}~\cite{Dejemeppe16} & \href{../works/FrimodigS19.pdf}{FrimodigS19}~\cite{FrimodigS19}, \href{../works/PraletLJ15.pdf}{PraletLJ15}~\cite{PraletLJ15}, \href{../works/KovacsB11.pdf}{KovacsB11}~\cite{KovacsB11}, \href{../works/Menana11.pdf}{Menana11}~\cite{Menana11}, \href{../works/KovacsB08.pdf}{KovacsB08}~\cite{KovacsB08}\\
Constraints & Reified constraint & \href{../works/Schutt11.pdf}{Schutt11}~\cite{Schutt11}, \href{../works/MilanoW09.pdf}{MilanoW09}~\cite{MilanoW09}, \href{../works/Kuchcinski03.pdf}{Kuchcinski03}~\cite{Kuchcinski03} & \href{../works/KovacsK11.pdf}{KovacsK11}~\cite{KovacsK11}, \href{../works/MilanoW06.pdf}{MilanoW06}~\cite{MilanoW06} & \href{../works/Astrand21.pdf}{Astrand21}~\cite{Astrand21}, \href{../works/BadicaBI20.pdf}{BadicaBI20}~\cite{BadicaBI20}, \href{../works/LaborieRSV18.pdf}{LaborieRSV18}~\cite{LaborieRSV18}, \href{../works/CauwelaertLS18.pdf}{CauwelaertLS18}~\cite{CauwelaertLS18}, \href{../works/KreterSS17.pdf}{KreterSS17}~\cite{KreterSS17}, \href{../works/Dejemeppe16.pdf}{Dejemeppe16}~\cite{Dejemeppe16}, \href{../works/Siala15.pdf}{Siala15}~\cite{Siala15}, \href{../works/Siala15a.pdf}{Siala15a}~\cite{Siala15a}, \href{../works/SchuttFSW13.pdf}{SchuttFSW13}~\cite{SchuttFSW13}, \href{../works/OhrimenkoSC09.pdf}{OhrimenkoSC09}~\cite{OhrimenkoSC09}, \href{../works/SchausD08.pdf}{SchausD08}~\cite{SchausD08}, \href{../works/SchildW00.pdf}{SchildW00}~\cite{SchildW00}\\
Constraints & RelSoftCumulative & \href{../works/abs-0907-0939.pdf}{abs-0907-0939}~\cite{abs-0907-0939} &  & \\
Constraints & RelSoftCumulativeSum &  &  & \href{../works/abs-0907-0939.pdf}{abs-0907-0939}~\cite{abs-0907-0939}\\
Constraints & SoftCumulative & \href{../works/Clercq12.pdf}{Clercq12}~\cite{Clercq12}, \href{../works/ClercqPBJ11.pdf}{ClercqPBJ11}~\cite{ClercqPBJ11}, \href{../works/abs-0907-0939.pdf}{abs-0907-0939}~\cite{abs-0907-0939} & \href{../works/OuelletQ22.pdf}{OuelletQ22}~\cite{OuelletQ22} & \\
Constraints & SoftCumulativeSum & \href{../works/Clercq12.pdf}{Clercq12}~\cite{Clercq12}, \href{../works/abs-0907-0939.pdf}{abs-0907-0939}~\cite{abs-0907-0939} &  & \href{../works/ClercqPBJ11.pdf}{ClercqPBJ11}~\cite{ClercqPBJ11}\\
Constraints & TaskIntersection constraint & \href{../works/Madi-WambaB16.pdf}{Madi-WambaB16}~\cite{Madi-WambaB16} &  & \\
Constraints & UTVPI constraint & \href{../works/Schutt11.pdf}{Schutt11}~\cite{Schutt11} &  & \\
Constraints & WeightAllDiff & \href{../works/WangB20.pdf}{WangB20}~\cite{WangB20} &  & \href{../works/WangB23.pdf}{WangB23}~\cite{WangB23}\\
Constraints & WeightedSum & \href{../works/Wolf09.pdf}{Wolf09}~\cite{Wolf09} &  & \\
Constraints & WeightedTaskSum & \href{../works/Wolf09.pdf}{Wolf09}~\cite{Wolf09} &  & \\
Constraints & alldifferent & \href{../works/FalqueALM24.pdf}{FalqueALM24}~\cite{FalqueALM24}, \href{../works/JuvinHHL23.pdf}{JuvinHHL23}~\cite{JuvinHHL23}, \href{../works/Lemos21.pdf}{Lemos21}~\cite{Lemos21}, \href{../works/KoehlerBFFHPSSS21.pdf}{KoehlerBFFHPSSS21}~\cite{KoehlerBFFHPSSS21}, \href{../works/Godet21a.pdf}{Godet21a}~\cite{Godet21a}, \href{../works/HoundjiSW19.pdf}{HoundjiSW19}~\cite{HoundjiSW19}, \href{../works/CauwelaertLS18.pdf}{CauwelaertLS18}~\cite{CauwelaertLS18}, \href{../works/Dejemeppe16.pdf}{Dejemeppe16}~\cite{Dejemeppe16}, \href{../works/Siala15.pdf}{Siala15}~\cite{Siala15}, \href{../works/Derrien15.pdf}{Derrien15}~\cite{Derrien15}, \href{../works/Siala15a.pdf}{Siala15a}~\cite{Siala15a}, \href{../works/Clercq12.pdf}{Clercq12}~\cite{Clercq12}, \href{../works/Menana11.pdf}{Menana11}~\cite{Menana11}, \href{../works/Malapert11.pdf}{Malapert11}~\cite{Malapert11}, \href{../works/MilanoW09.pdf}{MilanoW09}~\cite{MilanoW09}, \href{../works/OhrimenkoSC09.pdf}{OhrimenkoSC09}~\cite{OhrimenkoSC09}, \href{../works/Simonis07.pdf}{Simonis07}~\cite{Simonis07}, \href{../works/MilanoW06.pdf}{MilanoW06}~\cite{MilanoW06}, \href{../works/KanetAG04.pdf}{KanetAG04}~\cite{KanetAG04} & \href{../works/GodetLHS20.pdf}{GodetLHS20}~\cite{GodetLHS20}, \href{../works/HookerH17.pdf}{HookerH17}~\cite{HookerH17}, \href{../works/Fahimi16.pdf}{Fahimi16}~\cite{Fahimi16}, \href{../works/BessiereHMQW14.pdf}{BessiereHMQW14}~\cite{BessiereHMQW14}, \href{../works/UnsalO13.pdf}{UnsalO13}~\cite{UnsalO13}, \href{../works/KelarevaTK13.pdf}{KelarevaTK13}~\cite{KelarevaTK13}, \href{../works/TerekhovDOB12.pdf}{TerekhovDOB12}~\cite{TerekhovDOB12}, \href{../works/Schutt11.pdf}{Schutt11}~\cite{Schutt11} & \href{../works/GokPTGO23.pdf}{GokPTGO23}~\cite{GokPTGO23}, \href{../works/WangB23.pdf}{WangB23}~\cite{WangB23}, \href{../works/ColT22.pdf}{ColT22}~\cite{ColT22}, \href{../works/FarsiTM22.pdf}{FarsiTM22}~\cite{FarsiTM22}, \href{../works/BourreauGGLT22.pdf}{BourreauGGLT22}~\cite{BourreauGGLT22}, \href{../works/Astrand21.pdf}{Astrand21}~\cite{Astrand21}, \href{../works/MokhtarzadehTNF20.pdf}{MokhtarzadehTNF20}~\cite{MokhtarzadehTNF20}, \href{../works/AntuoriHHEN20.pdf}{AntuoriHHEN20}~\cite{AntuoriHHEN20}, \href{../works/AstrandJZ20.pdf}{AstrandJZ20}~\cite{AstrandJZ20}, \href{../works/WangB20.pdf}{WangB20}~\cite{WangB20}, \href{../works/Lunardi20.pdf}{Lunardi20}~\cite{Lunardi20}, \href{../works/Caballero19.pdf}{Caballero19}~\cite{Caballero19}, \href{../works/FahimiOQ18.pdf}{FahimiOQ18}~\cite{FahimiOQ18}, \href{../works/Nattaf16.pdf}{Nattaf16}~\cite{Nattaf16}, \href{../works/MelgarejoLS15.pdf}{MelgarejoLS15}~\cite{MelgarejoLS15}, \href{../works/AlesioNBG14.pdf}{AlesioNBG14}~\cite{AlesioNBG14}, \href{../works/Letort13.pdf}{Letort13}~\cite{Letort13}, \href{../works/ChuGNSW13.pdf}{ChuGNSW13}~\cite{ChuGNSW13}, \href{../works/ClercqPBJ11.pdf}{ClercqPBJ11}~\cite{ClercqPBJ11}, \href{../works/HachemiGR11.pdf}{HachemiGR11}~\cite{HachemiGR11}, \href{../works/HermenierDL11.pdf}{HermenierDL11}~\cite{HermenierDL11}, \href{../works/TrojetHL11.pdf}{TrojetHL11}~\cite{TrojetHL11}, \href{../works/LopesCSM10.pdf}{LopesCSM10}~\cite{LopesCSM10}, \href{../works/Malik08.pdf}{Malik08}~\cite{Malik08}, \href{../works/Thorsteinsson01.pdf}{Thorsteinsson01}~\cite{Thorsteinsson01}, \href{../works/BeldiceanuC01.pdf}{BeldiceanuC01}~\cite{BeldiceanuC01}, \href{../works/Simonis99.pdf}{Simonis99}~\cite{Simonis99}, \href{../works/BeldiceanuC94.pdf}{BeldiceanuC94}~\cite{BeldiceanuC94}\\
Constraints & alternative constraint & \href{../works/LaborieRSV18.pdf}{LaborieRSV18}~\cite{LaborieRSV18} & \href{../works/abs-2305-19888.pdf}{abs-2305-19888}~\cite{abs-2305-19888}, \href{../works/MurinR19.pdf}{MurinR19}~\cite{MurinR19}, \href{../works/GokgurHO18.pdf}{GokgurHO18}~\cite{GokgurHO18}, \href{../works/GedikKBR17.pdf}{GedikKBR17}~\cite{GedikKBR17}, \href{../works/LaborieR14.pdf}{LaborieR14}~\cite{LaborieR14} & \href{../works/ZhuSZW23.pdf}{ZhuSZW23}~\cite{ZhuSZW23}, \href{../works/MarliereSPR23.pdf}{MarliereSPR23}~\cite{MarliereSPR23}, \href{../works/LacknerMMWW23.pdf}{LacknerMMWW23}~\cite{LacknerMMWW23}, \href{../works/NaderiRR23.pdf}{NaderiRR23}~\cite{NaderiRR23}, \href{../works/SvancaraB22.pdf}{SvancaraB22}~\cite{SvancaraB22}, \href{../works/WinterMMW22.pdf}{WinterMMW22}~\cite{WinterMMW22}, \href{../works/HeinzNVH22.pdf}{HeinzNVH22}~\cite{HeinzNVH22}, \href{../works/AwadMDMT22.pdf}{AwadMDMT22}~\cite{AwadMDMT22}, \href{../works/ZhangJZL22.pdf}{ZhangJZL22}~\cite{ZhangJZL22}, \href{../works/ArmstrongGOS21.pdf}{ArmstrongGOS21}~\cite{ArmstrongGOS21}, \href{../works/PandeyS21a.pdf}{PandeyS21a}~\cite{PandeyS21a}, \href{../works/VlkHT21.pdf}{VlkHT21}~\cite{VlkHT21}, \href{../works/HillTV21.pdf}{HillTV21}~\cite{HillTV21}, \href{../works/MengLZB21.pdf}{MengLZB21}~\cite{MengLZB21}, \href{../works/HubnerGSV21.pdf}{HubnerGSV21}~\cite{HubnerGSV21}, \href{../works/MengZRZL20.pdf}{MengZRZL20}~\cite{MengZRZL20}, \href{../works/Polo-MejiaALB20.pdf}{Polo-MejiaALB20}~\cite{Polo-MejiaALB20}, \href{../works/SacramentoSP20.pdf}{SacramentoSP20}~\cite{SacramentoSP20}, \href{../works/NishikawaSTT19.pdf}{NishikawaSTT19}~\cite{NishikawaSTT19}, \href{../works/GalleguillosKSB19.pdf}{GalleguillosKSB19}~\cite{GalleguillosKSB19}, \href{../works/EscobetPQPRA19.pdf}{EscobetPQPRA19}~\cite{EscobetPQPRA19}, \href{../works/abs-1911-04766.pdf}{abs-1911-04766}~\cite{abs-1911-04766}, \href{../works/YounespourAKE19.pdf}{YounespourAKE19}~\cite{YounespourAKE19}, \href{../works/GeibingerMM19.pdf}{GeibingerMM19}~\cite{GeibingerMM19}, \href{../works/MalapertN19.pdf}{MalapertN19}~\cite{MalapertN19}, \href{../works/NattafDYW19.pdf}{NattafDYW19}~\cite{NattafDYW19}, \href{../works/NishikawaSTT18.pdf}{NishikawaSTT18}~\cite{NishikawaSTT18}, \href{../works/Ham18a.pdf}{Ham18a}~\cite{Ham18a}, \href{../works/NishikawaSTT18a.pdf}{NishikawaSTT18a}~\cite{NishikawaSTT18a}... (Total: 45)\\
Constraints & alwaysEqual constraint &  & \href{../works/LaborieRSV18.pdf}{LaborieRSV18}~\cite{LaborieRSV18}, \href{../works/GoelSHFS15.pdf}{GoelSHFS15}~\cite{GoelSHFS15} & \href{../works/HamFC17.pdf}{HamFC17}~\cite{HamFC17}, \href{../works/HamC16.pdf}{HamC16}~\cite{HamC16}\\
Constraints & alwaysIn & \href{../works/PopovicCGNC22.pdf}{PopovicCGNC22}~\cite{PopovicCGNC22}, \href{../works/SerraNM12.pdf}{SerraNM12}~\cite{SerraNM12} & \href{../works/LuZZYW24.pdf}{LuZZYW24}~\cite{LuZZYW24}, \href{../works/AalianPG23.pdf}{AalianPG23}~\cite{AalianPG23}, \href{../works/LuoB22.pdf}{LuoB22}~\cite{LuoB22}, \href{../works/TangB20.pdf}{TangB20}~\cite{TangB20}, \href{../works/Polo-MejiaALB20.pdf}{Polo-MejiaALB20}~\cite{Polo-MejiaALB20}, \href{../works/MalapertN19.pdf}{MalapertN19}~\cite{MalapertN19}, \href{../works/LaborieRSV18.pdf}{LaborieRSV18}~\cite{LaborieRSV18}, \href{../works/GoelSHFS15.pdf}{GoelSHFS15}~\cite{GoelSHFS15} & \href{../works/AwadMDMT22.pdf}{AwadMDMT22}~\cite{AwadMDMT22}, \href{../works/CampeauG22.pdf}{CampeauG22}~\cite{CampeauG22}, \href{../works/KreterSS17.pdf}{KreterSS17}~\cite{KreterSS17}, \href{../works/QinDS16.pdf}{QinDS16}~\cite{QinDS16}, \href{../works/BajestaniB13.pdf}{BajestaniB13}~\cite{BajestaniB13}\\
Constraints & bin-packing & \href{../works/NaderiBZR23.pdf}{NaderiBZR23}~\cite{NaderiBZR23}, \href{../works/Zahout21.pdf}{Zahout21}~\cite{Zahout21}, \href{../works/Godet21a.pdf}{Godet21a}~\cite{Godet21a}, \href{../works/TangB20.pdf}{TangB20}~\cite{TangB20}, \href{../works/CauwelaertLS18.pdf}{CauwelaertLS18}~\cite{CauwelaertLS18}, \href{../works/RoshanaeiLAU17.pdf}{RoshanaeiLAU17}~\cite{RoshanaeiLAU17}, \href{../works/CauwelaertLS15.pdf}{CauwelaertLS15}~\cite{CauwelaertLS15}, \href{../works/LetortCB15.pdf}{LetortCB15}~\cite{LetortCB15}, \href{../works/Letort13.pdf}{Letort13}~\cite{Letort13}, \href{../works/LetortCB13.pdf}{LetortCB13}~\cite{LetortCB13}, \href{../works/HeinzSSW12.pdf}{HeinzSSW12}~\cite{HeinzSSW12}, \href{../works/LetortBC12.pdf}{LetortBC12}~\cite{LetortBC12}, \href{../works/Malapert11.pdf}{Malapert11}~\cite{Malapert11}, \href{../works/SchausHMCMD11.pdf}{SchausHMCMD11}~\cite{SchausHMCMD11}, \href{../works/ClautiauxJCM08.pdf}{ClautiauxJCM08}~\cite{ClautiauxJCM08}, \href{../works/SchausD08.pdf}{SchausD08}~\cite{SchausD08} & \href{../works/JuvinHL23a.pdf}{JuvinHL23a}~\cite{JuvinHL23a}, \href{../works/LuoB22.pdf}{LuoB22}~\cite{LuoB22}, \href{../works/EmdeZD22.pdf}{EmdeZD22}~\cite{EmdeZD22}, \href{../works/BadicaBI20.pdf}{BadicaBI20}~\cite{BadicaBI20}, \href{../works/AntunesABD20.pdf}{AntunesABD20}~\cite{AntunesABD20}, \href{../works/FrimodigS19.pdf}{FrimodigS19}~\cite{FrimodigS19}, \href{../works/AntunesABD18.pdf}{AntunesABD18}~\cite{AntunesABD18}, \href{../works/BaptisteB18.pdf}{BaptisteB18}~\cite{BaptisteB18}, \href{../works/Beck10.pdf}{Beck10}~\cite{Beck10}, \href{../works/LiW08.pdf}{LiW08}~\cite{LiW08}, \href{../works/GarganiR07.pdf}{GarganiR07}~\cite{GarganiR07}, \href{../works/SchildW00.pdf}{SchildW00}~\cite{SchildW00}, \href{../works/SakkoutW00.pdf}{SakkoutW00}~\cite{SakkoutW00} & \href{../works/abs-2402-00459.pdf}{abs-2402-00459}~\cite{abs-2402-00459}, \href{../works/Fatemi-AnarakiTFV23.pdf}{Fatemi-AnarakiTFV23}~\cite{Fatemi-AnarakiTFV23}, \href{../works/LacknerMMWW23.pdf}{LacknerMMWW23}~\cite{LacknerMMWW23}, \href{../works/GuoZ23.pdf}{GuoZ23}~\cite{GuoZ23}, \href{../works/AkramNHRSA23.pdf}{AkramNHRSA23}~\cite{AkramNHRSA23}, \href{../works/YunusogluY22.pdf}{YunusogluY22}~\cite{YunusogluY22}, \href{../works/abs-2211-14492.pdf}{abs-2211-14492}~\cite{abs-2211-14492}, \href{../works/ArmstrongGOS21.pdf}{ArmstrongGOS21}~\cite{ArmstrongGOS21}, \href{../works/RoshanaeiBAUB20.pdf}{RoshanaeiBAUB20}~\cite{RoshanaeiBAUB20}, \href{../works/GodetLHS20.pdf}{GodetLHS20}~\cite{GodetLHS20}, \href{../works/TranPZLDB18.pdf}{TranPZLDB18}~\cite{TranPZLDB18}, \href{../works/BukchinR18.pdf}{BukchinR18}~\cite{BukchinR18}, \href{../works/German18.pdf}{German18}~\cite{German18}, \href{../works/HookerH17.pdf}{HookerH17}~\cite{HookerH17}, \href{../works/HamFC17.pdf}{HamFC17}~\cite{HamFC17}, \href{../works/Madi-WambaLOBM17.pdf}{Madi-WambaLOBM17}~\cite{Madi-WambaLOBM17}, \href{../works/DoulabiRP16.pdf}{DoulabiRP16}~\cite{DoulabiRP16}, \href{../works/RiiseML16.pdf}{RiiseML16}~\cite{RiiseML16}, \href{../works/DoulabiRP14.pdf}{DoulabiRP14}~\cite{DoulabiRP14}, \href{../works/KoschB14.pdf}{KoschB14}~\cite{KoschB14}, \href{../works/LimtanyakulS12.pdf}{LimtanyakulS12}~\cite{LimtanyakulS12}, \href{../works/EdisO11.pdf}{EdisO11}~\cite{EdisO11}, \href{../works/HermenierDL11.pdf}{HermenierDL11}~\cite{HermenierDL11}, \href{../works/BeldiceanuCDP11.pdf}{BeldiceanuCDP11}~\cite{BeldiceanuCDP11}, \href{../works/ReddyFIBKAJ11.pdf}{ReddyFIBKAJ11}~\cite{ReddyFIBKAJ11}, \href{../works/Schutt11.pdf}{Schutt11}~\cite{Schutt11}, \href{../works/Lombardi10.pdf}{Lombardi10}~\cite{Lombardi10}, \href{../works/LombardiMRB10.pdf}{LombardiMRB10}~\cite{LombardiMRB10}, \href{../works/KovacsB08.pdf}{KovacsB08}~\cite{KovacsB08}... (Total: 36)\\
Constraints & circuit & \href{../works/MontemanniD23a.pdf}{MontemanniD23a}~\cite{MontemanniD23a}, \href{../works/KlankeBYE21.pdf}{KlankeBYE21}~\cite{KlankeBYE21}, \href{../works/MokhtarzadehTNF20.pdf}{MokhtarzadehTNF20}~\cite{MokhtarzadehTNF20}, \href{../works/Mercier-AubinGQ20.pdf}{Mercier-AubinGQ20}~\cite{Mercier-AubinGQ20}, \href{../works/Caballero19.pdf}{Caballero19}~\cite{Caballero19}, \href{../works/HookerH17.pdf}{HookerH17}~\cite{HookerH17}, \href{../works/Lombardi10.pdf}{Lombardi10}~\cite{Lombardi10}, \href{../works/RuggieroBBMA09.pdf}{RuggieroBBMA09}~\cite{RuggieroBBMA09}, \href{../works/RodriguezS09.pdf}{RodriguezS09}~\cite{RodriguezS09}, \href{../works/AchterbergBKW08.pdf}{AchterbergBKW08}~\cite{AchterbergBKW08}, \href{../works/Rodriguez07b.pdf}{Rodriguez07b}~\cite{Rodriguez07b}, \href{../works/Rodriguez07.pdf}{Rodriguez07}~\cite{Rodriguez07}, \href{../works/BeniniBGM05.pdf}{BeniniBGM05}~\cite{BeniniBGM05}, \href{../works/RodriguezDG02.pdf}{RodriguezDG02}~\cite{RodriguezDG02}, \href{../works/GruianK98.pdf}{GruianK98}~\cite{GruianK98}, \href{../works/Wallace96.pdf}{Wallace96}~\cite{Wallace96}, \href{../works/BeldiceanuC94.pdf}{BeldiceanuC94}~\cite{BeldiceanuC94} & \href{../works/Groleaz21.pdf}{Groleaz21}~\cite{Groleaz21}, \href{../works/WessenCS20.pdf}{WessenCS20}~\cite{WessenCS20}, \href{../works/AntuoriHHEN20.pdf}{AntuoriHHEN20}~\cite{AntuoriHHEN20}, \href{../works/Siala15.pdf}{Siala15}~\cite{Siala15}, \href{../works/Siala15a.pdf}{Siala15a}~\cite{Siala15a}, \href{../works/LombardiMB13.pdf}{LombardiMB13}~\cite{LombardiMB13}, \href{../works/TranB12.pdf}{TranB12}~\cite{TranB12}, \href{../works/Malapert11.pdf}{Malapert11}~\cite{Malapert11}, \href{../works/ZeballosCM10.pdf}{ZeballosCM10}~\cite{ZeballosCM10}, \href{../works/KrogtLPHJ07.pdf}{KrogtLPHJ07}~\cite{KrogtLPHJ07}, \href{../works/KuchcinskiW03.pdf}{KuchcinskiW03}~\cite{KuchcinskiW03}, \href{../works/HookerO03.pdf}{HookerO03}~\cite{HookerO03}, \href{../works/Thorsteinsson01.pdf}{Thorsteinsson01}~\cite{Thorsteinsson01}, \href{../works/WatsonBHW99.pdf}{WatsonBHW99}~\cite{WatsonBHW99}, \href{../works/Simonis99.pdf}{Simonis99}~\cite{Simonis99}, \href{../works/Simonis95a.pdf}{Simonis95a}~\cite{Simonis95a}, \href{../works/DincbasSH90.pdf}{DincbasSH90}~\cite{DincbasSH90} & \href{../works/PrataAN23.pdf}{PrataAN23}~\cite{PrataAN23}, \href{../works/Fatemi-AnarakiTFV23.pdf}{Fatemi-AnarakiTFV23}~\cite{Fatemi-AnarakiTFV23}, \href{../works/GokPTGO23.pdf}{GokPTGO23}~\cite{GokPTGO23}, \href{../works/IsikYA23.pdf}{IsikYA23}~\cite{IsikYA23}, \href{../works/MontemanniD23.pdf}{MontemanniD23}~\cite{MontemanniD23}, \href{../works/MarliereSPR23.pdf}{MarliereSPR23}~\cite{MarliereSPR23}, \href{../works/JuvinHL23a.pdf}{JuvinHL23a}~\cite{JuvinHL23a}, \href{../works/ColT22.pdf}{ColT22}~\cite{ColT22}, \href{../works/MullerMKP22.pdf}{MullerMKP22}~\cite{MullerMKP22}, \href{../works/JungblutK22.pdf}{JungblutK22}~\cite{JungblutK22}, \href{../works/FarsiTM22.pdf}{FarsiTM22}~\cite{FarsiTM22}, \href{../works/JuvinHL22.pdf}{JuvinHL22}~\cite{JuvinHL22}, \href{../works/KoehlerBFFHPSSS21.pdf}{KoehlerBFFHPSSS21}~\cite{KoehlerBFFHPSSS21}, \href{../works/MengLZB21.pdf}{MengLZB21}~\cite{MengLZB21}, \href{../works/Astrand21.pdf}{Astrand21}~\cite{Astrand21}, \href{../works/Zahout21.pdf}{Zahout21}~\cite{Zahout21}, \href{../works/ArmstrongGOS21.pdf}{ArmstrongGOS21}~\cite{ArmstrongGOS21}, \href{../works/WallaceY20.pdf}{WallaceY20}~\cite{WallaceY20}, \href{../works/GokGSTO20.pdf}{GokGSTO20}~\cite{GokGSTO20}, \href{../works/GroleazNS20.pdf}{GroleazNS20}~\cite{GroleazNS20}, \href{../works/HoundjiSW19.pdf}{HoundjiSW19}~\cite{HoundjiSW19}, \href{../works/EscobetPQPRA19.pdf}{EscobetPQPRA19}~\cite{EscobetPQPRA19}, \href{../works/Hooker19.pdf}{Hooker19}~\cite{Hooker19}, \href{../works/Ham18a.pdf}{Ham18a}~\cite{Ham18a}, \href{../works/TangLWSK18.pdf}{TangLWSK18}~\cite{TangLWSK18}, \href{../works/CappartTSR18.pdf}{CappartTSR18}~\cite{CappartTSR18}, \href{../works/CauwelaertLS18.pdf}{CauwelaertLS18}~\cite{CauwelaertLS18}, \href{../works/EmeretlisTAV17.pdf}{EmeretlisTAV17}~\cite{EmeretlisTAV17}, \href{../works/Hooker17.pdf}{Hooker17}~\cite{Hooker17}... (Total: 86)\\
Constraints & cumulative & \href{../works/AalianPG23.pdf}{AalianPG23}~\cite{AalianPG23}, \href{../works/TardivoDFMP23.pdf}{TardivoDFMP23}~\cite{TardivoDFMP23}, \href{../works/NaderiRR23.pdf}{NaderiRR23}~\cite{NaderiRR23}, \href{../works/LacknerMMWW23.pdf}{LacknerMMWW23}~\cite{LacknerMMWW23}, \href{../works/PovedaAA23.pdf}{PovedaAA23}~\cite{PovedaAA23}, \href{../works/KameugneFND23.pdf}{KameugneFND23}~\cite{KameugneFND23}, \href{../works/IsikYA23.pdf}{IsikYA23}~\cite{IsikYA23}, \href{../works/PohlAK22.pdf}{PohlAK22}~\cite{PohlAK22}, \href{../works/AwadMDMT22.pdf}{AwadMDMT22}~\cite{AwadMDMT22}, \href{../works/ZhangJZL22.pdf}{ZhangJZL22}~\cite{ZhangJZL22}, \href{../works/LuoB22.pdf}{LuoB22}~\cite{LuoB22}, \href{../works/FetgoD22.pdf}{FetgoD22}~\cite{FetgoD22}, \href{../works/OuelletQ22.pdf}{OuelletQ22}~\cite{OuelletQ22}, \href{../works/BoudreaultSLQ22.pdf}{BoudreaultSLQ22}~\cite{BoudreaultSLQ22}, \href{../works/Lemos21.pdf}{Lemos21}~\cite{Lemos21}, \href{../works/Godet21a.pdf}{Godet21a}~\cite{Godet21a}, \href{../works/Groleaz21.pdf}{Groleaz21}~\cite{Groleaz21}, \href{../works/LacknerMMWW21.pdf}{LacknerMMWW21}~\cite{LacknerMMWW21}, \href{../works/KovacsTKSG21.pdf}{KovacsTKSG21}~\cite{KovacsTKSG21}, \href{../works/Zahout21.pdf}{Zahout21}~\cite{Zahout21}, \href{../works/HanenKP21.pdf}{HanenKP21}~\cite{HanenKP21}, \href{../works/GroleazNS20a.pdf}{GroleazNS20a}~\cite{GroleazNS20a}, \href{../works/CarlierPSJ20.pdf}{CarlierPSJ20}~\cite{CarlierPSJ20}, \href{../works/ThomasKS20.pdf}{ThomasKS20}~\cite{ThomasKS20}, \href{../works/Polo-MejiaALB20.pdf}{Polo-MejiaALB20}~\cite{Polo-MejiaALB20}, \href{../works/Mercier-AubinGQ20.pdf}{Mercier-AubinGQ20}~\cite{Mercier-AubinGQ20}, \href{../works/WallaceY20.pdf}{WallaceY20}~\cite{WallaceY20}, \href{../works/SacramentoSP20.pdf}{SacramentoSP20}~\cite{SacramentoSP20}, \href{../works/GodetLHS20.pdf}{GodetLHS20}~\cite{GodetLHS20}... (Total: 177) & \href{../works/ForbesHJST24.pdf}{ForbesHJST24}~\cite{ForbesHJST24}, \href{../works/PrataAN23.pdf}{PrataAN23}~\cite{PrataAN23}, \href{../works/LuZZYW24.pdf}{LuZZYW24}~\cite{LuZZYW24}, \href{../works/BonninMNE24.pdf}{BonninMNE24}~\cite{BonninMNE24}, \href{../works/abs-2402-00459.pdf}{abs-2402-00459}~\cite{abs-2402-00459}, \href{../works/PerezGSL23.pdf}{PerezGSL23}~\cite{PerezGSL23}, \href{../works/EfthymiouY23.pdf}{EfthymiouY23}~\cite{EfthymiouY23}, \href{../works/abs-2312-13682.pdf}{abs-2312-13682}~\cite{abs-2312-13682}, \href{../works/GokPTGO23.pdf}{GokPTGO23}~\cite{GokPTGO23}, \href{../works/ElciOH22.pdf}{ElciOH22}~\cite{ElciOH22}, \href{../works/YunusogluY22.pdf}{YunusogluY22}~\cite{YunusogluY22}, \href{../works/GeitzGSSW22.pdf}{GeitzGSSW22}~\cite{GeitzGSSW22}, \href{../works/AbreuN22.pdf}{AbreuN22}~\cite{AbreuN22}, \href{../works/ColT22.pdf}{ColT22}~\cite{ColT22}, \href{../works/CampeauG22.pdf}{CampeauG22}~\cite{CampeauG22}, \href{../works/HubnerGSV21.pdf}{HubnerGSV21}~\cite{HubnerGSV21}, \href{../works/KlankeBYE21.pdf}{KlankeBYE21}~\cite{KlankeBYE21}, \href{../works/HillTV21.pdf}{HillTV21}~\cite{HillTV21}, \href{../works/MengLZB21.pdf}{MengLZB21}~\cite{MengLZB21}, \href{../works/NattafM20.pdf}{NattafM20}~\cite{NattafM20}, \href{../works/NattafHKAL19.pdf}{NattafHKAL19}~\cite{NattafHKAL19}, \href{../works/GalleguillosKSB19.pdf}{GalleguillosKSB19}~\cite{GalleguillosKSB19}, \href{../works/NishikawaSTT19.pdf}{NishikawaSTT19}~\cite{NishikawaSTT19}, \href{../works/GedikKEK18.pdf}{GedikKEK18}~\cite{GedikKEK18}, \href{../works/BorghesiBLMB18.pdf}{BorghesiBLMB18}~\cite{BorghesiBLMB18}, \href{../works/TranVNB17a.pdf}{TranVNB17a}~\cite{TranVNB17a}, \href{../works/BonfiettiZLM16.pdf}{BonfiettiZLM16}~\cite{BonfiettiZLM16}, \href{../works/Bonfietti16.pdf}{Bonfietti16}~\cite{Bonfietti16}, \href{../works/HurleyOS16.pdf}{HurleyOS16}~\cite{HurleyOS16}... (Total: 66) & \href{../works/GurPAE23.pdf}{GurPAE23}~\cite{GurPAE23}, \href{../works/abs-2306-05747.pdf}{abs-2306-05747}~\cite{abs-2306-05747}, \href{../works/AbreuPNF23.pdf}{AbreuPNF23}~\cite{AbreuPNF23}, \href{../works/YuraszeckMCCR23.pdf}{YuraszeckMCCR23}~\cite{YuraszeckMCCR23}, \href{../works/MarliereSPR23.pdf}{MarliereSPR23}~\cite{MarliereSPR23}, \href{../works/TasselGS23.pdf}{TasselGS23}~\cite{TasselGS23}, \href{../works/JuvinHL23a.pdf}{JuvinHL23a}~\cite{JuvinHL23a}, \href{../works/abs-2305-19888.pdf}{abs-2305-19888}~\cite{abs-2305-19888}, \href{../works/Bit-Monnot23.pdf}{Bit-Monnot23}~\cite{Bit-Monnot23}, \href{../works/JuvinHHL23.pdf}{JuvinHHL23}~\cite{JuvinHHL23}, \href{../works/HeinzNVH22.pdf}{HeinzNVH22}~\cite{HeinzNVH22}, \href{../works/PopovicCGNC22.pdf}{PopovicCGNC22}~\cite{PopovicCGNC22}, \href{../works/HebrardALLCMR22.pdf}{HebrardALLCMR22}~\cite{HebrardALLCMR22}, \href{../works/abs-2211-14492.pdf}{abs-2211-14492}~\cite{abs-2211-14492}, \href{../works/SubulanC22.pdf}{SubulanC22}~\cite{SubulanC22}, \href{../works/JuvinHL22.pdf}{JuvinHL22}~\cite{JuvinHL22}, \href{../works/ArmstrongGOS22.pdf}{ArmstrongGOS22}~\cite{ArmstrongGOS22}, \href{../works/Astrand21.pdf}{Astrand21}~\cite{Astrand21}, \href{../works/PandeyS21a.pdf}{PandeyS21a}~\cite{PandeyS21a}, \href{../works/KoehlerBFFHPSSS21.pdf}{KoehlerBFFHPSSS21}~\cite{KoehlerBFFHPSSS21}, \href{../works/RoshanaeiN21.pdf}{RoshanaeiN21}~\cite{RoshanaeiN21}, \href{../works/ArtiguesHQT21.pdf}{ArtiguesHQT21}~\cite{ArtiguesHQT21}, \href{../works/GeibingerMM21.pdf}{GeibingerMM21}~\cite{GeibingerMM21}, \href{../works/ArmstrongGOS21.pdf}{ArmstrongGOS21}~\cite{ArmstrongGOS21}, \href{../works/GokGSTO20.pdf}{GokGSTO20}~\cite{GokGSTO20}, \href{../works/HauderBRPA20.pdf}{HauderBRPA20}~\cite{HauderBRPA20}, \href{../works/CauwelaertDS20.pdf}{CauwelaertDS20}~\cite{CauwelaertDS20}, \href{../works/ZouZ20.pdf}{ZouZ20}~\cite{ZouZ20}, \href{../works/Ham20a.pdf}{Ham20a}~\cite{Ham20a}... (Total: 142)\\
Constraints & cycle & \href{../works/AalianPG23.pdf}{AalianPG23}~\cite{AalianPG23}, \href{../works/AwadMDMT22.pdf}{AwadMDMT22}~\cite{AwadMDMT22}, \href{../works/Astrand21.pdf}{Astrand21}~\cite{Astrand21}, \href{../works/AbohashimaEG21.pdf}{AbohashimaEG21}~\cite{AbohashimaEG21}, \href{../works/AntuoriHHEN21.pdf}{AntuoriHHEN21}~\cite{AntuoriHHEN21}, \href{../works/Groleaz21.pdf}{Groleaz21}~\cite{Groleaz21}, \href{../works/Astrand0F21.pdf}{Astrand0F21}~\cite{Astrand0F21}, \href{../works/Edis21.pdf}{Edis21}~\cite{Edis21}, \href{../works/GroleazNS20a.pdf}{GroleazNS20a}~\cite{GroleazNS20a}, \href{../works/AntuoriHHEN20.pdf}{AntuoriHHEN20}~\cite{AntuoriHHEN20}, \href{../works/AstrandJZ20.pdf}{AstrandJZ20}~\cite{AstrandJZ20}, \href{../works/WallaceY20.pdf}{WallaceY20}~\cite{WallaceY20}, \href{../works/Caballero19.pdf}{Caballero19}~\cite{Caballero19}, \href{../works/ParkUJR19.pdf}{ParkUJR19}~\cite{ParkUJR19}, \href{../works/BorghesiBLMB18.pdf}{BorghesiBLMB18}~\cite{BorghesiBLMB18}, \href{../works/AstrandJZ18.pdf}{AstrandJZ18}~\cite{AstrandJZ18}, \href{../works/BukchinR18.pdf}{BukchinR18}~\cite{BukchinR18}, \href{../works/GomesM17.pdf}{GomesM17}~\cite{GomesM17}, \href{../works/BridiBLMB16.pdf}{BridiBLMB16}~\cite{BridiBLMB16}, \href{../works/Dejemeppe16.pdf}{Dejemeppe16}~\cite{Dejemeppe16}, \href{../works/OzturkTHO15.pdf}{OzturkTHO15}~\cite{OzturkTHO15}, \href{../works/BonfiettiLBM14.pdf}{BonfiettiLBM14}~\cite{BonfiettiLBM14}, \href{../works/BessiereHMQW14.pdf}{BessiereHMQW14}~\cite{BessiereHMQW14}, \href{../works/BegB13.pdf}{BegB13}~\cite{BegB13}, \href{../works/MalapertCGJLR12.pdf}{MalapertCGJLR12}~\cite{MalapertCGJLR12}, \href{../works/MenciaSV12.pdf}{MenciaSV12}~\cite{MenciaSV12}, \href{../works/LozanoCDS12.pdf}{LozanoCDS12}~\cite{LozanoCDS12}, \href{../works/LombardiBMB11.pdf}{LombardiBMB11}~\cite{LombardiBMB11}, \href{../works/Malapert11.pdf}{Malapert11}~\cite{Malapert11}... (Total: 54) & \href{../works/EfthymiouY23.pdf}{EfthymiouY23}~\cite{EfthymiouY23}, \href{../works/CampeauG22.pdf}{CampeauG22}~\cite{CampeauG22}, \href{../works/Godet21a.pdf}{Godet21a}~\cite{Godet21a}, \href{../works/HillTV21.pdf}{HillTV21}~\cite{HillTV21}, \href{../works/Lemos21.pdf}{Lemos21}~\cite{Lemos21}, \href{../works/KoehlerBFFHPSSS21.pdf}{KoehlerBFFHPSSS21}~\cite{KoehlerBFFHPSSS21}, \href{../works/HubnerGSV21.pdf}{HubnerGSV21}~\cite{HubnerGSV21}, \href{../works/CauwelaertDS20.pdf}{CauwelaertDS20}~\cite{CauwelaertDS20}, \href{../works/Lunardi20.pdf}{Lunardi20}~\cite{Lunardi20}, \href{../works/ZarandiASC20.pdf}{ZarandiASC20}~\cite{ZarandiASC20}, \href{../works/GroleazNS20.pdf}{GroleazNS20}~\cite{GroleazNS20}, \href{../works/ArkhipovBL19.pdf}{ArkhipovBL19}~\cite{ArkhipovBL19}, \href{../works/MossigeGSMC17.pdf}{MossigeGSMC17}~\cite{MossigeGSMC17}, \href{../works/HamFC17.pdf}{HamFC17}~\cite{HamFC17}, \href{../works/TranAB16.pdf}{TranAB16}~\cite{TranAB16}, \href{../works/Froger16.pdf}{Froger16}~\cite{Froger16}, \href{../works/SimoninAHL15.pdf}{SimoninAHL15}~\cite{SimoninAHL15}, \href{../works/BurtLPS15.pdf}{BurtLPS15}~\cite{BurtLPS15}, \href{../works/Siala15.pdf}{Siala15}~\cite{Siala15}, \href{../works/PraletLJ15.pdf}{PraletLJ15}~\cite{PraletLJ15}, \href{../works/Siala15a.pdf}{Siala15a}~\cite{Siala15a}, \href{../works/HarjunkoskiMBC14.pdf}{HarjunkoskiMBC14}~\cite{HarjunkoskiMBC14}, \href{../works/TranTDB13.pdf}{TranTDB13}~\cite{TranTDB13}, \href{../works/SchuttFSW13.pdf}{SchuttFSW13}~\cite{SchuttFSW13}, \href{../works/SimoninAHL12.pdf}{SimoninAHL12}~\cite{SimoninAHL12}, \href{../works/BonfiettiLBM12.pdf}{BonfiettiLBM12}~\cite{BonfiettiLBM12}, \href{../works/OzturkTHO12.pdf}{OzturkTHO12}~\cite{OzturkTHO12}, \href{../works/KovacsB11.pdf}{KovacsB11}~\cite{KovacsB11}, \href{../works/BonfiettiLBM11.pdf}{BonfiettiLBM11}~\cite{BonfiettiLBM11}... (Total: 53) & \href{../works/Bit-Monnot23.pdf}{Bit-Monnot23}~\cite{Bit-Monnot23}, \href{../works/AkramNHRSA23.pdf}{AkramNHRSA23}~\cite{AkramNHRSA23}, \href{../works/Fatemi-AnarakiTFV23.pdf}{Fatemi-AnarakiTFV23}~\cite{Fatemi-AnarakiTFV23}, \href{../works/MarliereSPR23.pdf}{MarliereSPR23}~\cite{MarliereSPR23}, \href{../works/GokPTGO23.pdf}{GokPTGO23}~\cite{GokPTGO23}, \href{../works/GuoZ23.pdf}{GuoZ23}~\cite{GuoZ23}, \href{../works/ZhangBB22.pdf}{ZhangBB22}~\cite{ZhangBB22}, \href{../works/BourreauGGLT22.pdf}{BourreauGGLT22}~\cite{BourreauGGLT22}, \href{../works/AbreuN22.pdf}{AbreuN22}~\cite{AbreuN22}, \href{../works/KotaryFH22.pdf}{KotaryFH22}~\cite{KotaryFH22}, \href{../works/AbreuAPNM21.pdf}{AbreuAPNM21}~\cite{AbreuAPNM21}, \href{../works/ArmstrongGOS21.pdf}{ArmstrongGOS21}~\cite{ArmstrongGOS21}, \href{../works/Zahout21.pdf}{Zahout21}~\cite{Zahout21}, \href{../works/FanXG21.pdf}{FanXG21}~\cite{FanXG21}, \href{../works/HamPK21.pdf}{HamPK21}~\cite{HamPK21}, \href{../works/Ham20a.pdf}{Ham20a}~\cite{Ham20a}, \href{../works/HauderBRPA20.pdf}{HauderBRPA20}~\cite{HauderBRPA20}, \href{../works/FachiniA20.pdf}{FachiniA20}~\cite{FachiniA20}, \href{../works/QinDCS20.pdf}{QinDCS20}~\cite{QinDCS20}, \href{../works/BadicaBI20.pdf}{BadicaBI20}~\cite{BadicaBI20}, \href{../works/MokhtarzadehTNF20.pdf}{MokhtarzadehTNF20}~\cite{MokhtarzadehTNF20}, \href{../works/TangB20.pdf}{TangB20}~\cite{TangB20}, \href{../works/FallahiAC20.pdf}{FallahiAC20}~\cite{FallahiAC20}, \href{../works/Mercier-AubinGQ20.pdf}{Mercier-AubinGQ20}~\cite{Mercier-AubinGQ20}, \href{../works/Novas19.pdf}{Novas19}~\cite{Novas19}, \href{../works/abs-1902-09244.pdf}{abs-1902-09244}~\cite{abs-1902-09244}, \href{../works/EscobetPQPRA19.pdf}{EscobetPQPRA19}~\cite{EscobetPQPRA19}, \href{../works/KucukY19.pdf}{KucukY19}~\cite{KucukY19}, \href{../works/Hooker19.pdf}{Hooker19}~\cite{Hooker19}... (Total: 114)\\
Constraints & diffn & \href{../works/ArmstrongGOS21.pdf}{ArmstrongGOS21}~\cite{ArmstrongGOS21}, \href{../works/Simonis07.pdf}{Simonis07}~\cite{Simonis07}, \href{../works/SimonisCK00.pdf}{SimonisCK00}~\cite{SimonisCK00}, \href{../works/BeldiceanuC94.pdf}{BeldiceanuC94}~\cite{BeldiceanuC94} & \href{../works/BeldiceanuCDP11.pdf}{BeldiceanuCDP11}~\cite{BeldiceanuCDP11} & \href{../works/LuoB22.pdf}{LuoB22}~\cite{LuoB22}, \href{../works/BourreauGGLT22.pdf}{BourreauGGLT22}~\cite{BourreauGGLT22}, \href{../works/KreterSS17.pdf}{KreterSS17}~\cite{KreterSS17}, \href{../works/KreterSS15.pdf}{KreterSS15}~\cite{KreterSS15}, \href{../works/Malapert11.pdf}{Malapert11}~\cite{Malapert11}, \href{../works/TrojetHL11.pdf}{TrojetHL11}~\cite{TrojetHL11}, \href{../works/ChenGPSH10.pdf}{ChenGPSH10}~\cite{ChenGPSH10}, \href{../works/Kuchcinski03.pdf}{Kuchcinski03}~\cite{Kuchcinski03}, \href{../works/Timpe02.pdf}{Timpe02}~\cite{Timpe02}, \href{../works/Simonis99.pdf}{Simonis99}~\cite{Simonis99}, \href{../works/GruianK98.pdf}{GruianK98}~\cite{GruianK98}, \href{../works/Simonis95.pdf}{Simonis95}~\cite{Simonis95}, \href{../works/SimonisC95.pdf}{SimonisC95}~\cite{SimonisC95}, \href{../works/Simonis95a.pdf}{Simonis95a}~\cite{Simonis95a}\\
Constraints & disjunctive & \href{../works/BonninMNE24.pdf}{BonninMNE24}~\cite{BonninMNE24}, \href{../works/JuvinHHL23.pdf}{JuvinHHL23}~\cite{JuvinHHL23}, \href{../works/AfsarVPG23.pdf}{AfsarVPG23}~\cite{AfsarVPG23}, \href{../works/NaderiRR23.pdf}{NaderiRR23}~\cite{NaderiRR23}, \href{../works/Bit-Monnot23.pdf}{Bit-Monnot23}~\cite{Bit-Monnot23}, \href{../works/BourreauGGLT22.pdf}{BourreauGGLT22}~\cite{BourreauGGLT22}, \href{../works/ZhangBB22.pdf}{ZhangBB22}~\cite{ZhangBB22}, \href{../works/YuraszeckMPV22.pdf}{YuraszeckMPV22}~\cite{YuraszeckMPV22}, \href{../works/JuvinHL22.pdf}{JuvinHL22}~\cite{JuvinHL22}, \href{../works/Groleaz21.pdf}{Groleaz21}~\cite{Groleaz21}, \href{../works/Godet21a.pdf}{Godet21a}~\cite{Godet21a}, \href{../works/KoehlerBFFHPSSS21.pdf}{KoehlerBFFHPSSS21}~\cite{KoehlerBFFHPSSS21}, \href{../works/Astrand21.pdf}{Astrand21}~\cite{Astrand21}, \href{../works/GodetLHS20.pdf}{GodetLHS20}~\cite{GodetLHS20}, \href{../works/GokgurHO18.pdf}{GokgurHO18}~\cite{GokgurHO18}, \href{../works/LaborieRSV18.pdf}{LaborieRSV18}~\cite{LaborieRSV18}, \href{../works/German18.pdf}{German18}~\cite{German18}, \href{../works/FahimiOQ18.pdf}{FahimiOQ18}~\cite{FahimiOQ18}, \href{../works/NattafAL17.pdf}{NattafAL17}~\cite{NattafAL17}, \href{../works/MossigeGSMC17.pdf}{MossigeGSMC17}~\cite{MossigeGSMC17}, \href{../works/Pralet17.pdf}{Pralet17}~\cite{Pralet17}, \href{../works/HookerH17.pdf}{HookerH17}~\cite{HookerH17}, \href{../works/KuB16.pdf}{KuB16}~\cite{KuB16}, \href{../works/FontaineMH16.pdf}{FontaineMH16}~\cite{FontaineMH16}, \href{../works/Fahimi16.pdf}{Fahimi16}~\cite{Fahimi16}, \href{../works/OrnekO16.pdf}{OrnekO16}~\cite{OrnekO16}, \href{../works/BoothTNB16.pdf}{BoothTNB16}~\cite{BoothTNB16}, \href{../works/GrimesH15.pdf}{GrimesH15}~\cite{GrimesH15}, \href{../works/GoelSHFS15.pdf}{GoelSHFS15}~\cite{GoelSHFS15}... (Total: 97) & \href{../works/MarliereSPR23.pdf}{MarliereSPR23}~\cite{MarliereSPR23}, \href{../works/Adelgren2023.pdf}{Adelgren2023}~\cite{Adelgren2023}, \href{../works/JuvinHL23a.pdf}{JuvinHL23a}~\cite{JuvinHL23a}, \href{../works/BoudreaultSLQ22.pdf}{BoudreaultSLQ22}~\cite{BoudreaultSLQ22}, \href{../works/OrnekOS20.pdf}{OrnekOS20}~\cite{OrnekOS20}, \href{../works/KotaryFH22.pdf}{KotaryFH22}~\cite{KotaryFH22}, \href{../works/Astrand0F21.pdf}{Astrand0F21}~\cite{Astrand0F21}, \href{../works/GeibingerMM21.pdf}{GeibingerMM21}~\cite{GeibingerMM21}, \href{../works/AstrandJZ20.pdf}{AstrandJZ20}~\cite{AstrandJZ20}, \href{../works/MejiaY20.pdf}{MejiaY20}~\cite{MejiaY20}, \href{../works/Polo-MejiaALB20.pdf}{Polo-MejiaALB20}~\cite{Polo-MejiaALB20}, \href{../works/SacramentoSP20.pdf}{SacramentoSP20}~\cite{SacramentoSP20}, \href{../works/RoshanaeiBAUB20.pdf}{RoshanaeiBAUB20}~\cite{RoshanaeiBAUB20}, \href{../works/UnsalO19.pdf}{UnsalO19}~\cite{UnsalO19}, \href{../works/YangSS19.pdf}{YangSS19}~\cite{YangSS19}, \href{../works/DemirovicS18.pdf}{DemirovicS18}~\cite{DemirovicS18}, \href{../works/TanT18.pdf}{TanT18}~\cite{TanT18}, \href{../works/CauwelaertLS18.pdf}{CauwelaertLS18}~\cite{CauwelaertLS18}, \href{../works/KameugneFGOQ18.pdf}{KameugneFGOQ18}~\cite{KameugneFGOQ18}, \href{../works/Dejemeppe16.pdf}{Dejemeppe16}~\cite{Dejemeppe16}, \href{../works/Nattaf16.pdf}{Nattaf16}~\cite{Nattaf16}, \href{../works/SimoninAHL15.pdf}{SimoninAHL15}~\cite{SimoninAHL15}, \href{../works/EvenSH15a.pdf}{EvenSH15a}~\cite{EvenSH15a}, \href{../works/VilimLS15.pdf}{VilimLS15}~\cite{VilimLS15}, \href{../works/EvenSH15.pdf}{EvenSH15}~\cite{EvenSH15}, \href{../works/GayHS15.pdf}{GayHS15}~\cite{GayHS15}, \href{../works/LaborieR14.pdf}{LaborieR14}~\cite{LaborieR14}, \href{../works/KameugneFSN14.pdf}{KameugneFSN14}~\cite{KameugneFSN14}, \href{../works/GaySS14.pdf}{GaySS14}~\cite{GaySS14}... (Total: 55) & \href{../works/LuZZYW24.pdf}{LuZZYW24}~\cite{LuZZYW24}, \href{../works/abs-2402-00459.pdf}{abs-2402-00459}~\cite{abs-2402-00459}, \href{../works/LacknerMMWW23.pdf}{LacknerMMWW23}~\cite{LacknerMMWW23}, \href{../works/abs-2306-05747.pdf}{abs-2306-05747}~\cite{abs-2306-05747}, \href{../works/NaderiBZ23.pdf}{NaderiBZ23}~\cite{NaderiBZ23}, \href{../works/TardivoDFMP23.pdf}{TardivoDFMP23}~\cite{TardivoDFMP23}, \href{../works/ZhuSZW23.pdf}{ZhuSZW23}~\cite{ZhuSZW23}, \href{../works/GokPTGO23.pdf}{GokPTGO23}~\cite{GokPTGO23}, \href{../works/AbreuPNF23.pdf}{AbreuPNF23}~\cite{AbreuPNF23}, \href{../works/KameugneFND23.pdf}{KameugneFND23}~\cite{KameugneFND23}, \href{../works/EfthymiouY23.pdf}{EfthymiouY23}~\cite{EfthymiouY23}, \href{../works/TasselGS23.pdf}{TasselGS23}~\cite{TasselGS23}, \href{../works/Fatemi-AnarakiTFV23.pdf}{Fatemi-AnarakiTFV23}~\cite{Fatemi-AnarakiTFV23}, \href{../works/PovedaAA23.pdf}{PovedaAA23}~\cite{PovedaAA23}, \href{../works/ElciOH22.pdf}{ElciOH22}~\cite{ElciOH22}, \href{../works/AwadMDMT22.pdf}{AwadMDMT22}~\cite{AwadMDMT22}, \href{../works/NaderiBZ22a.pdf}{NaderiBZ22a}~\cite{NaderiBZ22a}, \href{../works/NaderiBZ22.pdf}{NaderiBZ22}~\cite{NaderiBZ22}, \href{../works/abs-2211-14492.pdf}{abs-2211-14492}~\cite{abs-2211-14492}, \href{../works/MullerMKP22.pdf}{MullerMKP22}~\cite{MullerMKP22}, \href{../works/OujanaAYB22.pdf}{OujanaAYB22}~\cite{OujanaAYB22}, \href{../works/OuelletQ22.pdf}{OuelletQ22}~\cite{OuelletQ22}, \href{../works/ColT22.pdf}{ColT22}~\cite{ColT22}, \href{../works/Edis21.pdf}{Edis21}~\cite{Edis21}, \href{../works/ZhangYW21.pdf}{ZhangYW21}~\cite{ZhangYW21}, \href{../works/KlankeBYE21.pdf}{KlankeBYE21}~\cite{KlankeBYE21}, \href{../works/CauwelaertDS20.pdf}{CauwelaertDS20}~\cite{CauwelaertDS20}, \href{../works/ZarandiASC20.pdf}{ZarandiASC20}~\cite{ZarandiASC20}, \href{../works/Mercier-AubinGQ20.pdf}{Mercier-AubinGQ20}~\cite{Mercier-AubinGQ20}... (Total: 163)\\
Constraints & endBeforeStart & \href{../works/SubulanC22.pdf}{SubulanC22}~\cite{SubulanC22}, \href{../works/QinDCS20.pdf}{QinDCS20}~\cite{QinDCS20} & \href{../works/IsikYA23.pdf}{IsikYA23}~\cite{IsikYA23}, \href{../works/ZhuSZW23.pdf}{ZhuSZW23}~\cite{ZhuSZW23}, \href{../works/NaderiRR23.pdf}{NaderiRR23}~\cite{NaderiRR23}, \href{../works/NaderiBZ22a.pdf}{NaderiBZ22a}~\cite{NaderiBZ22a}, \href{../works/PandeyS21a.pdf}{PandeyS21a}~\cite{PandeyS21a}, \href{../works/MengLZB21.pdf}{MengLZB21}~\cite{MengLZB21}, \href{../works/LunardiBLRV20.pdf}{LunardiBLRV20}~\cite{LunardiBLRV20}, \href{../works/Lunardi20.pdf}{Lunardi20}~\cite{Lunardi20}, \href{../works/MengZRZL20.pdf}{MengZRZL20}~\cite{MengZRZL20}, \href{../works/LaborieRSV18.pdf}{LaborieRSV18}~\cite{LaborieRSV18}, \href{../works/NovaraNH16.pdf}{NovaraNH16}~\cite{NovaraNH16}, \href{../works/Laborie09.pdf}{Laborie09}~\cite{Laborie09} & \href{../works/JuvinHL23a.pdf}{JuvinHL23a}~\cite{JuvinHL23a}, \href{../works/JuvinHHL23.pdf}{JuvinHHL23}~\cite{JuvinHHL23}, \href{../works/YuraszeckMCCR23.pdf}{YuraszeckMCCR23}~\cite{YuraszeckMCCR23}, \href{../works/JuvinHL23.pdf}{JuvinHL23}~\cite{JuvinHL23}, \href{../works/LacknerMMWW23.pdf}{LacknerMMWW23}~\cite{LacknerMMWW23}, \href{../works/AalianPG23.pdf}{AalianPG23}~\cite{AalianPG23}, \href{../works/CzerniachowskaWZ23.pdf}{CzerniachowskaWZ23}~\cite{CzerniachowskaWZ23}, \href{../works/Teppan22.pdf}{Teppan22}~\cite{Teppan22}, \href{../works/AwadMDMT22.pdf}{AwadMDMT22}~\cite{AwadMDMT22}, \href{../works/JuvinHL22.pdf}{JuvinHL22}~\cite{JuvinHL22}, \href{../works/CampeauG22.pdf}{CampeauG22}~\cite{CampeauG22}, \href{../works/ZhangJZL22.pdf}{ZhangJZL22}~\cite{ZhangJZL22}, \href{../works/YunusogluY22.pdf}{YunusogluY22}~\cite{YunusogluY22}, \href{../works/HamP21.pdf}{HamP21}~\cite{HamP21}, \href{../works/LacknerMMWW21.pdf}{LacknerMMWW21}~\cite{LacknerMMWW21}, \href{../works/HamPK21.pdf}{HamPK21}~\cite{HamPK21}, \href{../works/HubnerGSV21.pdf}{HubnerGSV21}~\cite{HubnerGSV21}, \href{../works/ZhangYW21.pdf}{ZhangYW21}~\cite{ZhangYW21}, \href{../works/ZouZ20.pdf}{ZouZ20}~\cite{ZouZ20}, \href{../works/SacramentoSP20.pdf}{SacramentoSP20}~\cite{SacramentoSP20}, \href{../works/Ham20a.pdf}{Ham20a}~\cite{Ham20a}, \href{../works/Polo-MejiaALB20.pdf}{Polo-MejiaALB20}~\cite{Polo-MejiaALB20}, \href{../works/BenediktMH20.pdf}{BenediktMH20}~\cite{BenediktMH20}, \href{../works/TangB20.pdf}{TangB20}~\cite{TangB20}, \href{../works/SenderovichBB19.pdf}{SenderovichBB19}~\cite{SenderovichBB19}, \href{../works/UnsalO19.pdf}{UnsalO19}~\cite{UnsalO19}, \href{../works/MurinR19.pdf}{MurinR19}~\cite{MurinR19}, \href{../works/ParkUJR19.pdf}{ParkUJR19}~\cite{ParkUJR19}, \href{../works/abs-1911-04766.pdf}{abs-1911-04766}~\cite{abs-1911-04766}... (Total: 38)\\
Constraints & geost & \href{../works/BeldiceanuCDP11.pdf}{BeldiceanuCDP11}~\cite{BeldiceanuCDP11} & \href{../works/LetortBC12.pdf}{LetortBC12}~\cite{LetortBC12}, \href{../works/PembertonG98.pdf}{PembertonG98}~\cite{PembertonG98} & \href{../works/FrankDT16.pdf}{FrankDT16}~\cite{FrankDT16}, \href{../works/Letort13.pdf}{Letort13}~\cite{Letort13}, \href{../works/Schutt11.pdf}{Schutt11}~\cite{Schutt11}, \href{../works/Malapert11.pdf}{Malapert11}~\cite{Malapert11}, \href{../works/BeldiceanuCP08.pdf}{BeldiceanuCP08}~\cite{BeldiceanuCP08}\\
Constraints & noOverlap & \href{../works/IsikYA23.pdf}{IsikYA23}~\cite{IsikYA23}, \href{../works/JuvinHHL23.pdf}{JuvinHHL23}~\cite{JuvinHHL23}, \href{../works/ZhuSZW23.pdf}{ZhuSZW23}~\cite{ZhuSZW23}, \href{../works/abs-2305-19888.pdf}{abs-2305-19888}~\cite{abs-2305-19888}, \href{../works/NaderiRR23.pdf}{NaderiRR23}~\cite{NaderiRR23}, \href{../works/PopovicCGNC22.pdf}{PopovicCGNC22}~\cite{PopovicCGNC22}, \href{../works/HeinzNVH22.pdf}{HeinzNVH22}~\cite{HeinzNVH22}, \href{../works/ColT22.pdf}{ColT22}~\cite{ColT22}, \href{../works/VlkHT21.pdf}{VlkHT21}~\cite{VlkHT21}, \href{../works/MengLZB21.pdf}{MengLZB21}~\cite{MengLZB21}, \href{../works/RoshanaeiN21.pdf}{RoshanaeiN21}~\cite{RoshanaeiN21}, \href{../works/Groleaz21.pdf}{Groleaz21}~\cite{Groleaz21}, \href{../works/QinDCS20.pdf}{QinDCS20}~\cite{QinDCS20}, \href{../works/Lunardi20.pdf}{Lunardi20}~\cite{Lunardi20}, \href{../works/LunardiBLRV20.pdf}{LunardiBLRV20}~\cite{LunardiBLRV20}, \href{../works/GedikKEK18.pdf}{GedikKEK18}~\cite{GedikKEK18}, \href{../works/MelgarejoLS15.pdf}{MelgarejoLS15}~\cite{MelgarejoLS15} & \href{../works/abs-2306-05747.pdf}{abs-2306-05747}~\cite{abs-2306-05747}, \href{../works/KimCMLLP23.pdf}{KimCMLLP23}~\cite{KimCMLLP23}, \href{../works/LacknerMMWW23.pdf}{LacknerMMWW23}~\cite{LacknerMMWW23}, \href{../works/TasselGS23.pdf}{TasselGS23}~\cite{TasselGS23}, \href{../works/NaderiBZ22a.pdf}{NaderiBZ22a}~\cite{NaderiBZ22a}, \href{../works/PohlAK22.pdf}{PohlAK22}~\cite{PohlAK22}, \href{../works/AwadMDMT22.pdf}{AwadMDMT22}~\cite{AwadMDMT22}, \href{../works/YuraszeckMPV22.pdf}{YuraszeckMPV22}~\cite{YuraszeckMPV22}, \href{../works/AbreuN22.pdf}{AbreuN22}~\cite{AbreuN22}, \href{../works/SvancaraB22.pdf}{SvancaraB22}~\cite{SvancaraB22}, \href{../works/KlankeBYE21.pdf}{KlankeBYE21}~\cite{KlankeBYE21}, \href{../works/Bedhief21.pdf}{Bedhief21}~\cite{Bedhief21}, \href{../works/BenderWS21.pdf}{BenderWS21}~\cite{BenderWS21}, \href{../works/ZouZ20.pdf}{ZouZ20}~\cite{ZouZ20}, \href{../works/Ham20a.pdf}{Ham20a}~\cite{Ham20a}, \href{../works/BenediktMH20.pdf}{BenediktMH20}~\cite{BenediktMH20}, \href{../works/SacramentoSP20.pdf}{SacramentoSP20}~\cite{SacramentoSP20}, \href{../works/RoshanaeiBAUB20.pdf}{RoshanaeiBAUB20}~\cite{RoshanaeiBAUB20}, \href{../works/MengZRZL20.pdf}{MengZRZL20}~\cite{MengZRZL20}, \href{../works/YounespourAKE19.pdf}{YounespourAKE19}~\cite{YounespourAKE19}, \href{../works/MurinR19.pdf}{MurinR19}~\cite{MurinR19}, \href{../works/EscobetPQPRA19.pdf}{EscobetPQPRA19}~\cite{EscobetPQPRA19}, \href{../works/Novas19.pdf}{Novas19}~\cite{Novas19}, \href{../works/MalapertN19.pdf}{MalapertN19}~\cite{MalapertN19}, \href{../works/abs-1911-04766.pdf}{abs-1911-04766}~\cite{abs-1911-04766}, \href{../works/LaborieRSV18.pdf}{LaborieRSV18}~\cite{LaborieRSV18}, \href{../works/Ham18a.pdf}{Ham18a}~\cite{Ham18a}, \href{../works/ZhangW18.pdf}{ZhangW18}~\cite{ZhangW18}, \href{../works/ArbaouiY18.pdf}{ArbaouiY18}~\cite{ArbaouiY18}... (Total: 40) & \href{../works/BonninMNE24.pdf}{BonninMNE24}~\cite{BonninMNE24}, \href{../works/LuZZYW24.pdf}{LuZZYW24}~\cite{LuZZYW24}, \href{../works/JuvinHL23a.pdf}{JuvinHL23a}~\cite{JuvinHL23a}, \href{../works/AbreuNP23.pdf}{AbreuNP23}~\cite{AbreuNP23}, \href{../works/SquillaciPR23.pdf}{SquillaciPR23}~\cite{SquillaciPR23}, \href{../works/NaderiBZ23.pdf}{NaderiBZ23}~\cite{NaderiBZ23}, \href{../works/YuraszeckMC23.pdf}{YuraszeckMC23}~\cite{YuraszeckMC23}, \href{../works/AalianPG23.pdf}{AalianPG23}~\cite{AalianPG23}, \href{../works/AbreuPNF23.pdf}{AbreuPNF23}~\cite{AbreuPNF23}, \href{../works/JuvinHL23.pdf}{JuvinHL23}~\cite{JuvinHL23}, \href{../works/CzerniachowskaWZ23.pdf}{CzerniachowskaWZ23}~\cite{CzerniachowskaWZ23}, \href{../works/MarliereSPR23.pdf}{MarliereSPR23}~\cite{MarliereSPR23}, \href{../works/WinterMMW22.pdf}{WinterMMW22}~\cite{WinterMMW22}, \href{../works/Teppan22.pdf}{Teppan22}~\cite{Teppan22}, \href{../works/NaderiBZ22.pdf}{NaderiBZ22}~\cite{NaderiBZ22}, \href{../works/YunusogluY22.pdf}{YunusogluY22}~\cite{YunusogluY22}, \href{../works/CampeauG22.pdf}{CampeauG22}~\cite{CampeauG22}, \href{../works/OujanaAYB22.pdf}{OujanaAYB22}~\cite{OujanaAYB22}, \href{../works/ArmstrongGOS22.pdf}{ArmstrongGOS22}~\cite{ArmstrongGOS22}, \href{../works/TouatBT22.pdf}{TouatBT22}~\cite{TouatBT22}, \href{../works/EmdeZD22.pdf}{EmdeZD22}~\cite{EmdeZD22}, \href{../works/ZhangJZL22.pdf}{ZhangJZL22}~\cite{ZhangJZL22}, \href{../works/JuvinHL22.pdf}{JuvinHL22}~\cite{JuvinHL22}, \href{../works/OrnekOS20.pdf}{OrnekOS20}~\cite{OrnekOS20}, \href{../works/HamP21.pdf}{HamP21}~\cite{HamP21}, \href{../works/HamPK21.pdf}{HamPK21}~\cite{HamPK21}, \href{../works/AbreuAPNM21.pdf}{AbreuAPNM21}~\cite{AbreuAPNM21}, \href{../works/LacknerMMWW21.pdf}{LacknerMMWW21}~\cite{LacknerMMWW21}, \href{../works/GroleazNS20.pdf}{GroleazNS20}~\cite{GroleazNS20}... (Total: 52)\\
Constraints & regular expression &  & \href{../works/FrimodigS19.pdf}{FrimodigS19}~\cite{FrimodigS19} & \href{../works/HookerH17.pdf}{HookerH17}~\cite{HookerH17}\\
Constraints & span constraint &  & \href{../works/Groleaz21.pdf}{Groleaz21}~\cite{Groleaz21}, \href{../works/CappartS17.pdf}{CappartS17}~\cite{CappartS17}, \href{../works/LaborieR14.pdf}{LaborieR14}~\cite{LaborieR14}, \href{../works/SchuttFS13.pdf}{SchuttFS13}~\cite{SchuttFS13}, \href{../works/Lombardi10.pdf}{Lombardi10}~\cite{Lombardi10}, \href{../works/LombardiM10a.pdf}{LombardiM10a}~\cite{LombardiM10a}, \href{../works/Darby-DowmanLMZ97.pdf}{Darby-DowmanLMZ97}~\cite{Darby-DowmanLMZ97} & \href{../works/ZhangBB22.pdf}{ZhangBB22}~\cite{ZhangBB22}, \href{../works/AwadMDMT22.pdf}{AwadMDMT22}~\cite{AwadMDMT22}, \href{../works/OujanaAYB22.pdf}{OujanaAYB22}~\cite{OujanaAYB22}, \href{../works/ZouZ20.pdf}{ZouZ20}~\cite{ZouZ20}, \href{../works/TangB20.pdf}{TangB20}~\cite{TangB20}, \href{../works/YounespourAKE19.pdf}{YounespourAKE19}~\cite{YounespourAKE19}, \href{../works/LaborieRSV18.pdf}{LaborieRSV18}~\cite{LaborieRSV18}, \href{../works/SimoninAHL15.pdf}{SimoninAHL15}~\cite{SimoninAHL15}, \href{../works/SimoninAHL12.pdf}{SimoninAHL12}~\cite{SimoninAHL12}, \href{../works/SchuttFSW11.pdf}{SchuttFSW11}~\cite{SchuttFSW11}, \href{../works/GetoorOFC97.pdf}{GetoorOFC97}~\cite{GetoorOFC97}\\
Constraints & table constraint & \href{../works/FalqueALM24.pdf}{FalqueALM24}~\cite{FalqueALM24}, \href{../works/ReddyFIBKAJ11.pdf}{ReddyFIBKAJ11}~\cite{ReddyFIBKAJ11}, \href{../works/LombardiM10a.pdf}{LombardiM10a}~\cite{LombardiM10a}, \href{../works/Lombardi10.pdf}{Lombardi10}~\cite{Lombardi10}, \href{../works/Baptiste02.pdf}{Baptiste02}~\cite{Baptiste02}, \href{../works/PapaB98.pdf}{PapaB98}~\cite{PapaB98} & \href{../works/MarliereSPR23.pdf}{MarliereSPR23}~\cite{MarliereSPR23}, \href{../works/JelinekB16.pdf}{JelinekB16}~\cite{JelinekB16}, \href{../works/LombardiMRB10.pdf}{LombardiMRB10}~\cite{LombardiMRB10} & \href{../works/PerezGSL23.pdf}{PerezGSL23}~\cite{PerezGSL23}, \href{../works/abs-2312-13682.pdf}{abs-2312-13682}~\cite{abs-2312-13682}, \href{../works/ArmstrongGOS21.pdf}{ArmstrongGOS21}~\cite{ArmstrongGOS21}, \href{../works/CauwelaertLS18.pdf}{CauwelaertLS18}~\cite{CauwelaertLS18}, \href{../works/Siala15.pdf}{Siala15}~\cite{Siala15}, \href{../works/GayHS15.pdf}{GayHS15}~\cite{GayHS15}, \href{../works/PesantRR15.pdf}{PesantRR15}~\cite{PesantRR15}, \href{../works/MelgarejoLS15.pdf}{MelgarejoLS15}~\cite{MelgarejoLS15}, \href{../works/Siala15a.pdf}{Siala15a}~\cite{Siala15a}, \href{../works/CauwelaertLS15.pdf}{CauwelaertLS15}~\cite{CauwelaertLS15}, \href{../works/LimtanyakulS12.pdf}{LimtanyakulS12}~\cite{LimtanyakulS12}, \href{../works/BeniniLMR11.pdf}{BeniniLMR11}~\cite{BeniniLMR11}, \href{../works/BeckFW11.pdf}{BeckFW11}~\cite{BeckFW11}, \href{../works/HermenierDL11.pdf}{HermenierDL11}~\cite{HermenierDL11}, \href{../works/LopesCSM10.pdf}{LopesCSM10}~\cite{LopesCSM10}, \href{../works/MouraSCL08.pdf}{MouraSCL08}~\cite{MouraSCL08}, \href{../works/GodardLN05.pdf}{GodardLN05}~\cite{GodardLN05}, \href{../works/Laborie03.pdf}{Laborie03}~\cite{Laborie03}, \href{../works/ElkhyariGJ02.pdf}{ElkhyariGJ02}~\cite{ElkhyariGJ02}\\
\end{longtable}
}


\clearpage
\subsection{Concept Type ProgLanguages}
\label{sec:ProgLanguages}
{\scriptsize
\begin{longtable}{lp{3cm}>{\raggedright\arraybackslash}p{6cm}>{\raggedright\arraybackslash}p{6cm}>{\raggedright\arraybackslash}p{8cm}}
\rowcolor{white}\caption{Works for Concepts of Type ProgLanguages}\\ \toprule
\rowcolor{white}Type & Keyword & High & Medium & Low\\ \midrule\endhead
\bottomrule
\endfoot
ProgLanguages & C  & \href{../works/KoehlerBFFHPSSS21.pdf}{KoehlerBFFHPSSS21}~\cite{KoehlerBFFHPSSS21} &  & \href{../works/EmdeZD22.pdf}{EmdeZD22}~\cite{EmdeZD22}, \href{../works/HubnerGSV21.pdf}{HubnerGSV21}~\cite{HubnerGSV21}, \href{../works/ThomasKS20.pdf}{ThomasKS20}~\cite{ThomasKS20}, \href{../works/BogaerdtW19.pdf}{BogaerdtW19}~\cite{BogaerdtW19}, \href{../works/HoYCLLCLC18.pdf}{HoYCLLCLC18}~\cite{HoYCLLCLC18}, \href{../works/TangLWSK18.pdf}{TangLWSK18}~\cite{TangLWSK18}, \href{../works/LaborieRSV18.pdf}{LaborieRSV18}~\cite{LaborieRSV18}, \href{../works/LombardiMRB10.pdf}{LombardiMRB10}~\cite{LombardiMRB10}, \href{../works/Lombardi10.pdf}{Lombardi10}~\cite{Lombardi10}, \href{../works/LombardiM10a.pdf}{LombardiM10a}~\cite{LombardiM10a}, \href{../works/Laborie09.pdf}{Laborie09}~\cite{Laborie09}, \href{../works/GarridoOS08.pdf}{GarridoOS08}~\cite{GarridoOS08}, \href{../works/Layfield02.pdf}{Layfield02}~\cite{Layfield02}\\
ProgLanguages & C++ & \href{../works/Pape94.pdf}{Pape94}~\cite{Pape94} & \href{../works/BourreauGGLT22.pdf}{BourreauGGLT22}~\cite{BourreauGGLT22}, \href{../works/Demassey03.pdf}{Demassey03}~\cite{Demassey03} & \href{../works/BonninMNE24.pdf}{BonninMNE24}~\cite{BonninMNE24}, \href{../works/TardivoDFMP23.pdf}{TardivoDFMP23}~\cite{TardivoDFMP23}, \href{../works/JuvinHHL23.pdf}{JuvinHHL23}~\cite{JuvinHHL23}, \href{../works/ColT22.pdf}{ColT22}~\cite{ColT22}, \href{../works/NaderiBZ22a.pdf}{NaderiBZ22a}~\cite{NaderiBZ22a}, \href{../works/PopovicCGNC22.pdf}{PopovicCGNC22}~\cite{PopovicCGNC22}, \href{../works/QinWSLS21.pdf}{QinWSLS21}~\cite{QinWSLS21}, \href{../works/AbreuAPNM21.pdf}{AbreuAPNM21}~\cite{AbreuAPNM21}, \href{../works/Lemos21.pdf}{Lemos21}~\cite{Lemos21}, \href{../works/Astrand21.pdf}{Astrand21}~\cite{Astrand21}, \href{../works/AntuoriHHEN21.pdf}{AntuoriHHEN21}~\cite{AntuoriHHEN21}, \href{../works/Mercier-AubinGQ20.pdf}{Mercier-AubinGQ20}~\cite{Mercier-AubinGQ20}, \href{../works/Polo-MejiaALB20.pdf}{Polo-MejiaALB20}~\cite{Polo-MejiaALB20}, \href{../works/AstrandJZ20.pdf}{AstrandJZ20}~\cite{AstrandJZ20}, \href{../works/RoshanaeiBAUB20.pdf}{RoshanaeiBAUB20}~\cite{RoshanaeiBAUB20}, \href{../works/Caballero19.pdf}{Caballero19}~\cite{Caballero19}, \href{../works/abs-1902-01193.pdf}{abs-1902-01193}~\cite{abs-1902-01193}, \href{../works/LaborieRSV18.pdf}{LaborieRSV18}~\cite{LaborieRSV18}, \href{../works/TranPZLDB18.pdf}{TranPZLDB18}~\cite{TranPZLDB18}, \href{../works/ArbaouiY18.pdf}{ArbaouiY18}~\cite{ArbaouiY18}, \href{../works/NattafAL17.pdf}{NattafAL17}~\cite{NattafAL17}, \href{../works/GomesM17.pdf}{GomesM17}~\cite{GomesM17}, \href{../works/Nattaf16.pdf}{Nattaf16}~\cite{Nattaf16}, \href{../works/Tesch16.pdf}{Tesch16}~\cite{Tesch16}, \href{../works/BoothNB16.pdf}{BoothNB16}~\cite{BoothNB16}, \href{../works/Bonfietti16.pdf}{Bonfietti16}~\cite{Bonfietti16}, \href{../works/NattafALR16.pdf}{NattafALR16}~\cite{NattafALR16}, \href{../works/Fahimi16.pdf}{Fahimi16}~\cite{Fahimi16}, \href{../works/NattafAL15.pdf}{NattafAL15}~\cite{NattafAL15}... (Total: 73)\\
ProgLanguages & Java & \href{../works/abs-2102-08778.pdf}{abs-2102-08778}~\cite{abs-2102-08778}, \href{../works/Malapert11.pdf}{Malapert11}~\cite{Malapert11} & \href{../works/Froger16.pdf}{Froger16}~\cite{Froger16}, \href{../works/Wolf11.pdf}{Wolf11}~\cite{Wolf11}, \href{../works/KuchcinskiW03.pdf}{KuchcinskiW03}~\cite{KuchcinskiW03} & \href{../works/AlfieriGPS23.pdf}{AlfieriGPS23}~\cite{AlfieriGPS23}, \href{../works/KameugneFND23.pdf}{KameugneFND23}~\cite{KameugneFND23}, \href{../works/abs-2306-05747.pdf}{abs-2306-05747}~\cite{abs-2306-05747}, \href{../works/TasselGS23.pdf}{TasselGS23}~\cite{TasselGS23}, \href{../works/MullerMKP22.pdf}{MullerMKP22}~\cite{MullerMKP22}, \href{../works/FetgoD22.pdf}{FetgoD22}~\cite{FetgoD22}, \href{../works/ColT22.pdf}{ColT22}~\cite{ColT22}, \href{../works/Teppan22.pdf}{Teppan22}~\cite{Teppan22}, \href{../works/YuraszeckMPV22.pdf}{YuraszeckMPV22}~\cite{YuraszeckMPV22}, \href{../works/OuelletQ22.pdf}{OuelletQ22}~\cite{OuelletQ22}, \href{../works/Lemos21.pdf}{Lemos21}~\cite{Lemos21}, \href{../works/Groleaz21.pdf}{Groleaz21}~\cite{Groleaz21}, \href{../works/FanXG21.pdf}{FanXG21}~\cite{FanXG21}, \href{../works/AntuoriHHEN21.pdf}{AntuoriHHEN21}~\cite{AntuoriHHEN21}, \href{../works/ArmstrongGOS21.pdf}{ArmstrongGOS21}~\cite{ArmstrongGOS21}, \href{../works/CauwelaertDS20.pdf}{CauwelaertDS20}~\cite{CauwelaertDS20}, \href{../works/MejiaY20.pdf}{MejiaY20}~\cite{MejiaY20}, \href{../works/SacramentoSP20.pdf}{SacramentoSP20}~\cite{SacramentoSP20}, \href{../works/ThomasKS20.pdf}{ThomasKS20}~\cite{ThomasKS20}, \href{../works/TangB20.pdf}{TangB20}~\cite{TangB20}, \href{../works/BarzegaranZP20.pdf}{BarzegaranZP20}~\cite{BarzegaranZP20}, \href{../works/FrohnerTR19.pdf}{FrohnerTR19}~\cite{FrohnerTR19}, \href{../works/Tom19.pdf}{Tom19}~\cite{Tom19}, \href{../works/ColT19.pdf}{ColT19}~\cite{ColT19}, \href{../works/GeibingerMM19.pdf}{GeibingerMM19}~\cite{GeibingerMM19}, \href{../works/abs-1911-04766.pdf}{abs-1911-04766}~\cite{abs-1911-04766}, \href{../works/GombolayWS18.pdf}{GombolayWS18}~\cite{GombolayWS18}, \href{../works/KameugneFGOQ18.pdf}{KameugneFGOQ18}~\cite{KameugneFGOQ18}, \href{../works/CauwelaertLS18.pdf}{CauwelaertLS18}~\cite{CauwelaertLS18}... (Total: 59)\\
ProgLanguages & Julia &  &  & \href{../works/HebrardALLCMR22.pdf}{HebrardALLCMR22}~\cite{HebrardALLCMR22}, \href{../works/ElciOH22.pdf}{ElciOH22}~\cite{ElciOH22}, \href{../works/Groleaz21.pdf}{Groleaz21}~\cite{Groleaz21}, \href{../works/Astrand21.pdf}{Astrand21}~\cite{Astrand21}, \href{../works/CatusseCBL16.pdf}{CatusseCBL16}~\cite{CatusseCBL16}\\
ProgLanguages & Lisp & \href{../works/Pape94.pdf}{Pape94}~\cite{Pape94} &  & \href{../works/Wallace96.pdf}{Wallace96}~\cite{Wallace96}\\
ProgLanguages & Prolog & \href{../works/ArmstrongGOS21.pdf}{ArmstrongGOS21}~\cite{ArmstrongGOS21}, \href{../works/Simonis99.pdf}{Simonis99}~\cite{Simonis99}, \href{../works/LammaMM97.pdf}{LammaMM97}~\cite{LammaMM97}, \href{../works/FalaschiGMP97.pdf}{FalaschiGMP97}~\cite{FalaschiGMP97}, \href{../works/Zhou97.pdf}{Zhou97}~\cite{Zhou97}, \href{../works/Wallace96.pdf}{Wallace96}~\cite{Wallace96}, \href{../works/Touraivane95.pdf}{Touraivane95}~\cite{Touraivane95}, \href{../works/Simonis95a.pdf}{Simonis95a}~\cite{Simonis95a}, \href{../works/Simonis95.pdf}{Simonis95}~\cite{Simonis95}, \href{../works/DincbasSH90.pdf}{DincbasSH90}~\cite{DincbasSH90} & \href{../works/BadicaBI20.pdf}{BadicaBI20}~\cite{BadicaBI20}, \href{../works/MossigeGSMC17.pdf}{MossigeGSMC17}~\cite{MossigeGSMC17}, \href{../works/Madi-WambaLOBM17.pdf}{Madi-WambaLOBM17}~\cite{Madi-WambaLOBM17}, \href{../works/Malapert11.pdf}{Malapert11}~\cite{Malapert11}, \href{../works/MartinPY01.pdf}{MartinPY01}~\cite{MartinPY01}, \href{../works/SimonisCK00.pdf}{SimonisCK00}~\cite{SimonisCK00}, \href{../works/RodosekW98.pdf}{RodosekW98}~\cite{RodosekW98}, \href{../works/Zhou96.pdf}{Zhou96}~\cite{Zhou96}, \href{../works/SimonisC95.pdf}{SimonisC95}~\cite{SimonisC95}, \href{../works/BeldiceanuC94.pdf}{BeldiceanuC94}~\cite{BeldiceanuC94}, \href{../works/AggounB93.pdf}{AggounB93}~\cite{AggounB93} & \href{../works/PopovicCGNC22.pdf}{PopovicCGNC22}~\cite{PopovicCGNC22}, \href{../works/ArmstrongGOS22.pdf}{ArmstrongGOS22}~\cite{ArmstrongGOS22}, \href{../works/ZarandiASC20.pdf}{ZarandiASC20}~\cite{ZarandiASC20}, \href{../works/YangSS19.pdf}{YangSS19}~\cite{YangSS19}, \href{../works/abs-1902-01193.pdf}{abs-1902-01193}~\cite{abs-1902-01193}, \href{../works/CauwelaertLS18.pdf}{CauwelaertLS18}~\cite{CauwelaertLS18}, \href{../works/German18.pdf}{German18}~\cite{German18}, \href{../works/JelinekB16.pdf}{JelinekB16}~\cite{JelinekB16}, \href{../works/LetortCB15.pdf}{LetortCB15}~\cite{LetortCB15}, \href{../works/Kameugne14.pdf}{Kameugne14}~\cite{Kameugne14}, \href{../works/LetortCB13.pdf}{LetortCB13}~\cite{LetortCB13}, \href{../works/Letort13.pdf}{Letort13}~\cite{Letort13}, \href{../works/Clercq12.pdf}{Clercq12}~\cite{Clercq12}, \href{../works/LetortBC12.pdf}{LetortBC12}~\cite{LetortBC12}, \href{../works/Schutt11.pdf}{Schutt11}~\cite{Schutt11}, \href{../works/TrojetHL11.pdf}{TrojetHL11}~\cite{TrojetHL11}, \href{../works/BeldiceanuCDP11.pdf}{BeldiceanuCDP11}~\cite{BeldiceanuCDP11}, \href{../works/Menana11.pdf}{Menana11}~\cite{Menana11}, \href{../works/BartakCS10.pdf}{BartakCS10}~\cite{BartakCS10}, \href{../works/AronssonBK09.pdf}{AronssonBK09}~\cite{AronssonBK09}, \href{../works/BeldiceanuCP08.pdf}{BeldiceanuCP08}~\cite{BeldiceanuCP08}, \href{../works/KrogtLPHJ07.pdf}{KrogtLPHJ07}~\cite{KrogtLPHJ07}, \href{../works/Simonis07.pdf}{Simonis07}~\cite{Simonis07}, \href{../works/QuSN06.pdf}{QuSN06}~\cite{QuSN06}, \href{../works/Geske05.pdf}{Geske05}~\cite{Geske05}, \href{../works/PoderBS04.pdf}{PoderBS04}~\cite{PoderBS04}, \href{../works/Baptiste02.pdf}{Baptiste02}~\cite{Baptiste02}, \href{../works/Bartak02.pdf}{Bartak02}~\cite{Bartak02}, \href{../works/BeldiceanuC02.pdf}{BeldiceanuC02}~\cite{BeldiceanuC02}... (Total: 38)\\
ProgLanguages & Python & \href{../works/KoehlerBFFHPSSS21.pdf}{KoehlerBFFHPSSS21}~\cite{KoehlerBFFHPSSS21} & \href{../works/ForbesHJST24.pdf}{ForbesHJST24}~\cite{ForbesHJST24}, \href{../works/Fatemi-AnarakiTFV23.pdf}{Fatemi-AnarakiTFV23}~\cite{Fatemi-AnarakiTFV23}, \href{../works/GuoZ23.pdf}{GuoZ23}~\cite{GuoZ23}, \href{../works/abs-2211-14492.pdf}{abs-2211-14492}~\cite{abs-2211-14492}, \href{../works/AbreuN22.pdf}{AbreuN22}~\cite{AbreuN22}, \href{../works/AbreuAPNM21.pdf}{AbreuAPNM21}~\cite{AbreuAPNM21}, \href{../works/LaborieRSV18.pdf}{LaborieRSV18}~\cite{LaborieRSV18} & \href{../works/AbreuPNF23.pdf}{AbreuPNF23}~\cite{AbreuPNF23}, \href{../works/EfthymiouY23.pdf}{EfthymiouY23}~\cite{EfthymiouY23}, \href{../works/AbreuNP23.pdf}{AbreuNP23}~\cite{AbreuNP23}, \href{../works/KimCMLLP23.pdf}{KimCMLLP23}~\cite{KimCMLLP23}, \href{../works/NaderiRR23.pdf}{NaderiRR23}~\cite{NaderiRR23}, \href{../works/SquillaciPR23.pdf}{SquillaciPR23}~\cite{SquillaciPR23}, \href{../works/Mehdizadeh-Somarin23.pdf}{Mehdizadeh-Somarin23}~\cite{Mehdizadeh-Somarin23}, \href{../works/MontemanniD23.pdf}{MontemanniD23}~\cite{MontemanniD23}, \href{../works/PovedaAA23.pdf}{PovedaAA23}~\cite{PovedaAA23}, \href{../works/MontemanniD23a.pdf}{MontemanniD23a}~\cite{MontemanniD23a}, \href{../works/AkramNHRSA23.pdf}{AkramNHRSA23}~\cite{AkramNHRSA23}, \href{../works/MullerMKP22.pdf}{MullerMKP22}~\cite{MullerMKP22}, \href{../works/ZhangBB22.pdf}{ZhangBB22}~\cite{ZhangBB22}, \href{../works/FetgoD22.pdf}{FetgoD22}~\cite{FetgoD22}, \href{../works/PohlAK22.pdf}{PohlAK22}~\cite{PohlAK22}, \href{../works/EtminaniesfahaniGNMS22.pdf}{EtminaniesfahaniGNMS22}~\cite{EtminaniesfahaniGNMS22}, \href{../works/LuoB22.pdf}{LuoB22}~\cite{LuoB22}, \href{../works/CampeauG22.pdf}{CampeauG22}~\cite{CampeauG22}, \href{../works/FanXG21.pdf}{FanXG21}~\cite{FanXG21}, \href{../works/HanenKP21.pdf}{HanenKP21}~\cite{HanenKP21}, \href{../works/BenderWS21.pdf}{BenderWS21}~\cite{BenderWS21}, \href{../works/KlankeBYE21.pdf}{KlankeBYE21}~\cite{KlankeBYE21}, \href{../works/Lemos21.pdf}{Lemos21}~\cite{Lemos21}, \href{../works/AbohashimaEG21.pdf}{AbohashimaEG21}~\cite{AbohashimaEG21}, \href{../works/Lunardi20.pdf}{Lunardi20}~\cite{Lunardi20}, \href{../works/LunardiBLRV20.pdf}{LunardiBLRV20}~\cite{LunardiBLRV20}, \href{../works/Mercier-AubinGQ20.pdf}{Mercier-AubinGQ20}~\cite{Mercier-AubinGQ20}, \href{../works/FrimodigS19.pdf}{FrimodigS19}~\cite{FrimodigS19}, \href{../works/FrohnerTR19.pdf}{FrohnerTR19}~\cite{FrohnerTR19}... (Total: 39)\\
\end{longtable}
}


\clearpage
\subsection{Concept Type CPSystems}
\label{sec:CPSystems}
{\scriptsize
\begin{longtable}{lp{3cm}>{\raggedright\arraybackslash}p{6cm}>{\raggedright\arraybackslash}p{6cm}>{\raggedright\arraybackslash}p{8cm}}
\rowcolor{white}\caption{Works for Concepts of Type CPSystems}\\ \toprule
\rowcolor{white}Type & Keyword & High & Medium & Low\\ \midrule\endhead
\bottomrule
\endfoot
CPSystems & CHIP & \href{works/TrojetHL11.pdf}{TrojetHL11}~\cite{TrojetHL11}, \href{works/Simonis07.pdf}{Simonis07}~\cite{Simonis07}, \href{works/GruianK98.pdf}{GruianK98}~\cite{GruianK98}, \href{works/Wallace96.pdf}{Wallace96}~\cite{Wallace96}, \href{works/Simonis95.pdf}{Simonis95}~\cite{Simonis95}, \href{works/Goltz95.pdf}{Goltz95}~\cite{Goltz95}, \href{works/SimonisC95.pdf}{SimonisC95}~\cite{SimonisC95}, \href{works/BeldiceanuC94.pdf}{BeldiceanuC94}~\cite{BeldiceanuC94}, \href{works/AggounB93.pdf}{AggounB93}~\cite{AggounB93}, \href{works/DincbasSH90.pdf}{DincbasSH90}~\cite{DincbasSH90} & \href{works/ArmstrongGOS21.pdf}{ArmstrongGOS21}~\cite{ArmstrongGOS21}, \href{works/YangSS19.pdf}{YangSS19}~\cite{YangSS19}, \href{works/LaborieRSV18.pdf}{LaborieRSV18}~\cite{LaborieRSV18}, \href{works/Geske05.pdf}{Geske05}~\cite{Geske05}, \href{works/PoderBS04.pdf}{PoderBS04}~\cite{PoderBS04}, \href{works/Timpe02.pdf}{Timpe02}~\cite{Timpe02}, \href{works/RodosekW98.pdf}{RodosekW98}~\cite{RodosekW98}, \href{works/Zhou97.pdf}{Zhou97}~\cite{Zhou97}, \href{works/LammaMM97.pdf}{LammaMM97}~\cite{LammaMM97} & \href{works/PrataAN23.pdf}{PrataAN23}~\cite{PrataAN23}, \href{works/TardivoDFMP23.pdf}{TardivoDFMP23}~\cite{TardivoDFMP23}, \href{works/KameugneFND23.pdf}{KameugneFND23}~\cite{KameugneFND23}, \href{works/LuoB22.pdf}{LuoB22}~\cite{LuoB22}, \href{works/FetgoD22.pdf}{FetgoD22}~\cite{FetgoD22}, \href{works/BourreauGGLT22.pdf}{BourreauGGLT22}~\cite{BourreauGGLT22}, \href{works/PopovicCGNC22.pdf}{PopovicCGNC22}~\cite{PopovicCGNC22}, \href{works/KlankeBYE21.pdf}{KlankeBYE21}~\cite{KlankeBYE21}, \href{works/GodetLHS20.pdf}{GodetLHS20}~\cite{GodetLHS20}, \href{works/abs-1902-01193.pdf}{abs-1902-01193}~\cite{abs-1902-01193}, \href{works/BaptisteB18.pdf}{BaptisteB18}~\cite{BaptisteB18}, \href{works/KameugneFGOQ18.pdf}{KameugneFGOQ18}~\cite{KameugneFGOQ18}, \href{works/GoldwaserS18.pdf}{GoldwaserS18}~\cite{GoldwaserS18}, \href{works/GokgurHO18.pdf}{GokgurHO18}~\cite{GokgurHO18}, \href{works/MossigeGSMC17.pdf}{MossigeGSMC17}~\cite{MossigeGSMC17}, \href{works/Pralet17.pdf}{Pralet17}~\cite{Pralet17}, \href{works/KreterSS17.pdf}{KreterSS17}~\cite{KreterSS17}, \href{works/Madi-WambaB16.pdf}{Madi-WambaB16}~\cite{Madi-WambaB16}, \href{works/FontaineMH16.pdf}{FontaineMH16}~\cite{FontaineMH16}, \href{works/ZhouGL15.pdf}{ZhouGL15}~\cite{ZhouGL15}, \href{works/SimoninAHL15.pdf}{SimoninAHL15}~\cite{SimoninAHL15}, \href{works/LetortCB15.pdf}{LetortCB15}~\cite{LetortCB15}, \href{works/KreterSS15.pdf}{KreterSS15}~\cite{KreterSS15}, \href{works/GrimesIOS14.pdf}{GrimesIOS14}~\cite{GrimesIOS14}, \href{works/KameugneFSN14.pdf}{KameugneFSN14}~\cite{KameugneFSN14}, \href{works/DerrienPZ14.pdf}{DerrienPZ14}~\cite{DerrienPZ14}, \href{works/ChuGNSW13.pdf}{ChuGNSW13}~\cite{ChuGNSW13}, \href{works/SchuttFSW13.pdf}{SchuttFSW13}~\cite{SchuttFSW13}, \href{works/OzturkTHO13.pdf}{OzturkTHO13}~\cite{OzturkTHO13}... (Total: 55)\\
CPSystems & CPO & \href{works/NaderiRR23.pdf}{NaderiRR23}~\cite{NaderiRR23}, \href{works/LacknerMMWW23.pdf}{LacknerMMWW23}~\cite{LacknerMMWW23}, \href{works/JuvinHHL23.pdf}{JuvinHHL23}~\cite{JuvinHHL23}, \href{works/Bit-Monnot23.pdf}{Bit-Monnot23}~\cite{Bit-Monnot23}, \href{works/CzerniachowskaWZ23.pdf}{CzerniachowskaWZ23}~\cite{CzerniachowskaWZ23}, \href{works/WinterMMW22.pdf}{WinterMMW22}~\cite{WinterMMW22}, \href{works/ColT22.pdf}{ColT22}~\cite{ColT22}, \href{works/ZhangBB22.pdf}{ZhangBB22}~\cite{ZhangBB22}, \href{works/LacknerMMWW21.pdf}{LacknerMMWW21}~\cite{LacknerMMWW21}, \href{works/ArmstrongGOS21.pdf}{ArmstrongGOS21}~\cite{ArmstrongGOS21}, \href{works/NattafM20.pdf}{NattafM20}~\cite{NattafM20}, \href{works/GroleazNS20.pdf}{GroleazNS20}~\cite{GroleazNS20}, \href{works/Polo-MejiaALB20.pdf}{Polo-MejiaALB20}~\cite{Polo-MejiaALB20}, \href{works/GroleazNS20a.pdf}{GroleazNS20a}~\cite{GroleazNS20a}, \href{works/SacramentoSP20.pdf}{SacramentoSP20}~\cite{SacramentoSP20}, \href{works/GeibingerMM19.pdf}{GeibingerMM19}~\cite{GeibingerMM19}, \href{works/ColT19.pdf}{ColT19}~\cite{ColT19}, \href{works/MalapertN19.pdf}{MalapertN19}~\cite{MalapertN19}, \href{works/LaborieRSV18.pdf}{LaborieRSV18}~\cite{LaborieRSV18}, \href{works/KreterSS17.pdf}{KreterSS17}~\cite{KreterSS17}, \href{works/GoelSHFS15.pdf}{GoelSHFS15}~\cite{GoelSHFS15}, \href{works/PraletLJ15.pdf}{PraletLJ15}~\cite{PraletLJ15}, \href{works/Laborie09.pdf}{Laborie09}~\cite{Laborie09} & \href{works/AalianPG23.pdf}{AalianPG23}~\cite{AalianPG23}, \href{works/abs-1911-04766.pdf}{abs-1911-04766}~\cite{abs-1911-04766}, \href{works/NuijtenA94.pdf}{NuijtenA94}~\cite{NuijtenA94} & \href{works/JuvinHL23.pdf}{JuvinHL23}~\cite{JuvinHL23}, \href{works/PovedaAA23.pdf}{PovedaAA23}~\cite{PovedaAA23}, \href{works/OujanaAYB22.pdf}{OujanaAYB22}~\cite{OujanaAYB22}, \href{works/GeibingerMM21.pdf}{GeibingerMM21}~\cite{GeibingerMM21}, \href{works/abs-2102-08778.pdf}{abs-2102-08778}~\cite{abs-2102-08778}, \href{works/TangB20.pdf}{TangB20}~\cite{TangB20}, \href{works/Laborie18a.pdf}{Laborie18a}~\cite{Laborie18a}, \href{works/Pralet17.pdf}{Pralet17}~\cite{Pralet17}, \href{works/VilimLS15.pdf}{VilimLS15}~\cite{VilimLS15}, \href{works/BartakSR10.pdf}{BartakSR10}~\cite{BartakSR10}, \href{works/GarridoAO09.pdf}{GarridoAO09}~\cite{GarridoAO09}, \href{works/Vilim09.pdf}{Vilim09}~\cite{Vilim09}, \href{works/GarridoOS08.pdf}{GarridoOS08}~\cite{GarridoOS08}, \href{works/BeldiceanuC94.pdf}{BeldiceanuC94}~\cite{BeldiceanuC94}\\
CPSystems & Choco Solver & \href{works/TasselGS23.pdf}{TasselGS23}~\cite{TasselGS23}, \href{works/abs-2306-05747.pdf}{abs-2306-05747}~\cite{abs-2306-05747}, \href{works/LetortCB15.pdf}{LetortCB15}~\cite{LetortCB15}, \href{works/LetortCB13.pdf}{LetortCB13}~\cite{LetortCB13}, \href{works/OuelletQ13.pdf}{OuelletQ13}~\cite{OuelletQ13}, \href{works/LetortBC12.pdf}{LetortBC12}~\cite{LetortBC12}, \href{works/GrimesHM09.pdf}{GrimesHM09}~\cite{GrimesHM09}, \href{works/abs-0907-0939.pdf}{abs-0907-0939}~\cite{abs-0907-0939}, \href{works/GarridoAO09.pdf}{GarridoAO09}~\cite{GarridoAO09}, \href{works/GarridoOS08.pdf}{GarridoOS08}~\cite{GarridoOS08} & \href{works/KameugneFND23.pdf}{KameugneFND23}~\cite{KameugneFND23}, \href{works/MullerMKP22.pdf}{MullerMKP22}~\cite{MullerMKP22}, \href{works/FetgoD22.pdf}{FetgoD22}~\cite{FetgoD22}, \href{works/AntuoriHHEN21.pdf}{AntuoriHHEN21}~\cite{AntuoriHHEN21}, \href{works/AntuoriHHEN20.pdf}{AntuoriHHEN20}~\cite{AntuoriHHEN20}, \href{works/LiuLH19.pdf}{LiuLH19}~\cite{LiuLH19}, \href{works/FahimiOQ18.pdf}{FahimiOQ18}~\cite{FahimiOQ18}, \href{works/KameugneFGOQ18.pdf}{KameugneFGOQ18}~\cite{KameugneFGOQ18}, \href{works/LaborieRSV18.pdf}{LaborieRSV18}~\cite{LaborieRSV18}, \href{works/GayHS15.pdf}{GayHS15}~\cite{GayHS15}, \href{works/KoschB14.pdf}{KoschB14}~\cite{KoschB14}, \href{works/DerrienPZ14.pdf}{DerrienPZ14}~\cite{DerrienPZ14}, \href{works/DerrienP14.pdf}{DerrienP14}~\cite{DerrienP14}, \href{works/HermenierDL11.pdf}{HermenierDL11}~\cite{HermenierDL11}, \href{works/ClercqPBJ11.pdf}{ClercqPBJ11}~\cite{ClercqPBJ11} & \href{works/BourreauGGLT22.pdf}{BourreauGGLT22}~\cite{BourreauGGLT22}, \href{works/OuelletQ22.pdf}{OuelletQ22}~\cite{OuelletQ22}, \href{works/GodetLHS20.pdf}{GodetLHS20}~\cite{GodetLHS20}, \href{works/YangSS19.pdf}{YangSS19}~\cite{YangSS19}, \href{works/OuelletQ18.pdf}{OuelletQ18}~\cite{OuelletQ18}, \href{works/GingrasQ16.pdf}{GingrasQ16}~\cite{GingrasQ16}, \href{works/Madi-WambaB16.pdf}{Madi-WambaB16}~\cite{Madi-WambaB16}, \href{works/EvenSH15a.pdf}{EvenSH15a}~\cite{EvenSH15a}, \href{works/MurphyMB15.pdf}{MurphyMB15}~\cite{MurphyMB15}, \href{works/EvenSH15.pdf}{EvenSH15}~\cite{EvenSH15}, \href{works/BessiereHMQW14.pdf}{BessiereHMQW14}~\cite{BessiereHMQW14}, \href{works/BartakSR10.pdf}{BartakSR10}~\cite{BartakSR10}, \href{works/RossiTHP07.pdf}{RossiTHP07}~\cite{RossiTHP07}\\
CPSystems & Chuffed & \href{works/LacknerMMWW23.pdf}{LacknerMMWW23}~\cite{LacknerMMWW23}, \href{works/PovedaAA23.pdf}{PovedaAA23}~\cite{PovedaAA23}, \href{works/BoudreaultSLQ22.pdf}{BoudreaultSLQ22}~\cite{BoudreaultSLQ22}, \href{works/MullerMKP22.pdf}{MullerMKP22}~\cite{MullerMKP22}, \href{works/LacknerMMWW21.pdf}{LacknerMMWW21}~\cite{LacknerMMWW21}, \href{works/GeibingerMM21.pdf}{GeibingerMM21}~\cite{GeibingerMM21}, \href{works/ArmstrongGOS21.pdf}{ArmstrongGOS21}~\cite{ArmstrongGOS21}, \href{works/KoehlerBFFHPSSS21.pdf}{KoehlerBFFHPSSS21}~\cite{KoehlerBFFHPSSS21}, \href{works/WallaceY20.pdf}{WallaceY20}~\cite{WallaceY20}, \href{works/GodetLHS20.pdf}{GodetLHS20}~\cite{GodetLHS20}, \href{works/abs-1911-04766.pdf}{abs-1911-04766}~\cite{abs-1911-04766}, \href{works/YoungFS17.pdf}{YoungFS17}~\cite{YoungFS17}, \href{works/KreterSS17.pdf}{KreterSS17}~\cite{KreterSS17}, \href{works/SzerediS16.pdf}{SzerediS16}~\cite{SzerediS16}, \href{works/KreterSS15.pdf}{KreterSS15}~\cite{KreterSS15} & \href{works/GoldwaserS18.pdf}{GoldwaserS18}~\cite{GoldwaserS18} & \href{works/SchuttS16.pdf}{SchuttS16}~\cite{SchuttS16}\\
CPSystems & Claire & \href{works/BaptisteP00.pdf}{BaptisteP00}~\cite{BaptisteP00} & \href{works/BaptisteP97.pdf}{BaptisteP97}~\cite{BaptisteP97} & \href{works/HebrardALLCMR22.pdf}{HebrardALLCMR22}~\cite{HebrardALLCMR22}, \href{works/HanenKP21.pdf}{HanenKP21}~\cite{HanenKP21}, \href{works/PapaB98.pdf}{PapaB98}~\cite{PapaB98}\\
CPSystems & Cplex & \href{works/CzerniachowskaWZ23.pdf}{CzerniachowskaWZ23}~\cite{CzerniachowskaWZ23}, \href{works/NaderiRR23.pdf}{NaderiRR23}~\cite{NaderiRR23}, \href{works/SubulanC22.pdf}{SubulanC22}~\cite{SubulanC22}, \href{works/BourreauGGLT22.pdf}{BourreauGGLT22}~\cite{BourreauGGLT22}, \href{works/MullerMKP22.pdf}{MullerMKP22}~\cite{MullerMKP22}, \href{works/WinterMMW22.pdf}{WinterMMW22}~\cite{WinterMMW22}, \href{works/HubnerGSV21.pdf}{HubnerGSV21}~\cite{HubnerGSV21}, \href{works/GeibingerKKMMW21.pdf}{GeibingerKKMMW21}~\cite{GeibingerKKMMW21}, \href{works/KoehlerBFFHPSSS21.pdf}{KoehlerBFFHPSSS21}~\cite{KoehlerBFFHPSSS21}, \href{works/PandeyS21a.pdf}{PandeyS21a}~\cite{PandeyS21a}, \href{works/Bedhief21.pdf}{Bedhief21}~\cite{Bedhief21}, \href{works/HamPK21.pdf}{HamPK21}~\cite{HamPK21}, \href{works/QinDCS20.pdf}{QinDCS20}~\cite{QinDCS20}, \href{works/ZouZ20.pdf}{ZouZ20}~\cite{ZouZ20}, \href{works/SacramentoSP20.pdf}{SacramentoSP20}~\cite{SacramentoSP20}, \href{works/MejiaY20.pdf}{MejiaY20}~\cite{MejiaY20}, \href{works/LunardiBLRV20.pdf}{LunardiBLRV20}~\cite{LunardiBLRV20}, \href{works/MengZRZL20.pdf}{MengZRZL20}~\cite{MengZRZL20}, \href{works/MurinR19.pdf}{MurinR19}~\cite{MurinR19}, \href{works/GeibingerMM19.pdf}{GeibingerMM19}~\cite{GeibingerMM19}, \href{works/abs-1911-04766.pdf}{abs-1911-04766}~\cite{abs-1911-04766}, \href{works/NishikawaSTT19.pdf}{NishikawaSTT19}~\cite{NishikawaSTT19}, \href{works/GurEA19.pdf}{GurEA19}~\cite{GurEA19}, \href{works/LaborieRSV18.pdf}{LaborieRSV18}~\cite{LaborieRSV18}, \href{works/NishikawaSTT18.pdf}{NishikawaSTT18}~\cite{NishikawaSTT18}, \href{works/NishikawaSTT18a.pdf}{NishikawaSTT18a}~\cite{NishikawaSTT18a}, \href{works/KreterSS17.pdf}{KreterSS17}~\cite{KreterSS17}, \href{works/NovaraNH16.pdf}{NovaraNH16}~\cite{NovaraNH16}, \href{works/KoschB14.pdf}{KoschB14}~\cite{KoschB14}... (Total: 36) & \href{works/LacknerMMWW23.pdf}{LacknerMMWW23}~\cite{LacknerMMWW23}, \href{works/Mehdizadeh-Somarin23.pdf}{Mehdizadeh-Somarin23}~\cite{Mehdizadeh-Somarin23}, \href{works/AbreuNP23.pdf}{AbreuNP23}~\cite{AbreuNP23}, \href{works/IsikYA23.pdf}{IsikYA23}~\cite{IsikYA23}, \href{works/CampeauG22.pdf}{CampeauG22}~\cite{CampeauG22}, \href{works/YunusogluY22.pdf}{YunusogluY22}~\cite{YunusogluY22}, \href{works/LuoB22.pdf}{LuoB22}~\cite{LuoB22}, \href{works/ColT22.pdf}{ColT22}~\cite{ColT22}, \href{works/TouatBT22.pdf}{TouatBT22}~\cite{TouatBT22}, \href{works/LacknerMMWW21.pdf}{LacknerMMWW21}~\cite{LacknerMMWW21}, \href{works/KovacsTKSG21.pdf}{KovacsTKSG21}~\cite{KovacsTKSG21}, \href{works/QinWSLS21.pdf}{QinWSLS21}~\cite{QinWSLS21}, \href{works/ArmstrongGOS21.pdf}{ArmstrongGOS21}~\cite{ArmstrongGOS21}, \href{works/MokhtarzadehTNF20.pdf}{MokhtarzadehTNF20}~\cite{MokhtarzadehTNF20}, \href{works/NattafM20.pdf}{NattafM20}~\cite{NattafM20}, \href{works/WallaceY20.pdf}{WallaceY20}~\cite{WallaceY20}, \href{works/abs-1902-09244.pdf}{abs-1902-09244}~\cite{abs-1902-09244}, \href{works/MalapertN19.pdf}{MalapertN19}~\cite{MalapertN19}, \href{works/Novas19.pdf}{Novas19}~\cite{Novas19}, \href{works/DoulabiRP16.pdf}{DoulabiRP16}~\cite{DoulabiRP16}, \href{works/HechingH16.pdf}{HechingH16}~\cite{HechingH16}, \href{works/VilimLS15.pdf}{VilimLS15}~\cite{VilimLS15}, \href{works/BofillGSV15.pdf}{BofillGSV15}~\cite{BofillGSV15}, \href{works/NattafAL15.pdf}{NattafAL15}~\cite{NattafAL15}, \href{works/PraletLJ15.pdf}{PraletLJ15}~\cite{PraletLJ15}, \href{works/BofillEGPSV14.pdf}{BofillEGPSV14}~\cite{BofillEGPSV14}, \href{works/GrimesIOS14.pdf}{GrimesIOS14}~\cite{GrimesIOS14}, \href{works/HeinzKB13.pdf}{HeinzKB13}~\cite{HeinzKB13}, \href{works/HeinzB12.pdf}{HeinzB12}~\cite{HeinzB12}... (Total: 42) & \href{works/AlfieriGPS23.pdf}{AlfieriGPS23}~\cite{AlfieriGPS23}, \href{works/JuvinHL23.pdf}{JuvinHL23}~\cite{JuvinHL23}, \href{works/SquillaciPR23.pdf}{SquillaciPR23}~\cite{SquillaciPR23}, \href{works/GurPAE23.pdf}{GurPAE23}~\cite{GurPAE23}, \href{works/PovedaAA23.pdf}{PovedaAA23}~\cite{PovedaAA23}, \href{works/YuraszeckMCCR23.pdf}{YuraszeckMCCR23}~\cite{YuraszeckMCCR23}, \href{works/AalianPG23.pdf}{AalianPG23}~\cite{AalianPG23}, \href{works/FarsiTM22.pdf}{FarsiTM22}~\cite{FarsiTM22}, \href{works/abs-2211-14492.pdf}{abs-2211-14492}~\cite{abs-2211-14492}, \href{works/YuraszeckMPV22.pdf}{YuraszeckMPV22}~\cite{YuraszeckMPV22}, \href{works/PohlAK22.pdf}{PohlAK22}~\cite{PohlAK22}, \href{works/PopovicCGNC22.pdf}{PopovicCGNC22}~\cite{PopovicCGNC22}, \href{works/AbreuN22.pdf}{AbreuN22}~\cite{AbreuN22}, \href{works/ZhangYW21.pdf}{ZhangYW21}~\cite{ZhangYW21}, \href{works/abs-2102-08778.pdf}{abs-2102-08778}~\cite{abs-2102-08778}, \href{works/GeibingerMM21.pdf}{GeibingerMM21}~\cite{GeibingerMM21}, \href{works/FanXG21.pdf}{FanXG21}~\cite{FanXG21}, \href{works/VlkHT21.pdf}{VlkHT21}~\cite{VlkHT21}, \href{works/KlankeBYE21.pdf}{KlankeBYE21}~\cite{KlankeBYE21}, \href{works/AbreuAPNM21.pdf}{AbreuAPNM21}~\cite{AbreuAPNM21}, \href{works/TangB20.pdf}{TangB20}~\cite{TangB20}, \href{works/Polo-MejiaALB20.pdf}{Polo-MejiaALB20}~\cite{Polo-MejiaALB20}, \href{works/GroleazNS20a.pdf}{GroleazNS20a}~\cite{GroleazNS20a}, \href{works/FrimodigS19.pdf}{FrimodigS19}~\cite{FrimodigS19}, \href{works/BogaerdtW19.pdf}{BogaerdtW19}~\cite{BogaerdtW19}, \href{works/EscobetPQPRA19.pdf}{EscobetPQPRA19}~\cite{EscobetPQPRA19}, \href{works/KucukY19.pdf}{KucukY19}~\cite{KucukY19}, \href{works/Ham18.pdf}{Ham18}~\cite{Ham18}, \href{works/PourDERB18.pdf}{PourDERB18}~\cite{PourDERB18}... (Total: 83)\\
CPSystems & ECLiPSe & \href{works/BadicaBI20.pdf}{BadicaBI20}~\cite{BadicaBI20}, \href{works/BadicaBIL19.pdf}{BadicaBIL19}~\cite{BadicaBIL19}, \href{works/RodosekW98.pdf}{RodosekW98}~\cite{RodosekW98} & \href{works/SchuttFSW11.pdf}{SchuttFSW11}~\cite{SchuttFSW11}, \href{works/KamarainenS02.pdf}{KamarainenS02}~\cite{KamarainenS02}, \href{works/Darby-DowmanLMZ97.pdf}{Darby-DowmanLMZ97}~\cite{Darby-DowmanLMZ97}, \href{works/Wallace96.pdf}{Wallace96}~\cite{Wallace96} & \href{works/FanXG21.pdf}{FanXG21}~\cite{FanXG21}, \href{works/MejiaY20.pdf}{MejiaY20}~\cite{MejiaY20}, \href{works/WikarekS19.pdf}{WikarekS19}~\cite{WikarekS19}, \href{works/ZeballosQH10.pdf}{ZeballosQH10}~\cite{ZeballosQH10}, \href{works/SchuttFSW09.pdf}{SchuttFSW09}~\cite{SchuttFSW09}, \href{works/BeniniBGM06.pdf}{BeniniBGM06}~\cite{BeniniBGM06}, \href{works/ChuX05.pdf}{ChuX05}~\cite{ChuX05}, \href{works/QuirogaZH05.pdf}{QuirogaZH05}~\cite{QuirogaZH05}, \href{works/MartinPY01.pdf}{MartinPY01}~\cite{MartinPY01}, \href{works/LammaMM97.pdf}{LammaMM97}~\cite{LammaMM97}\\
CPSystems & Gecode & \href{works/TardivoDFMP23.pdf}{TardivoDFMP23}~\cite{TardivoDFMP23}, \href{works/BadicaBI20.pdf}{BadicaBI20}~\cite{BadicaBI20}, \href{works/AstrandJZ20.pdf}{AstrandJZ20}~\cite{AstrandJZ20}, \href{works/BadicaBIL19.pdf}{BadicaBIL19}~\cite{BadicaBIL19}, \href{works/SzerediS16.pdf}{SzerediS16}~\cite{SzerediS16}, \href{works/ZhouGL15.pdf}{ZhouGL15}~\cite{ZhouGL15}, \href{works/GayHS15.pdf}{GayHS15}~\cite{GayHS15}, \href{works/KameugneFSN14.pdf}{KameugneFSN14}~\cite{KameugneFSN14} & \href{works/MullerMKP22.pdf}{MullerMKP22}~\cite{MullerMKP22}, \href{works/AntuoriHHEN21.pdf}{AntuoriHHEN21}~\cite{AntuoriHHEN21}, \href{works/GeibingerKKMMW21.pdf}{GeibingerKKMMW21}~\cite{GeibingerKKMMW21}, \href{works/Astrand0F21.pdf}{Astrand0F21}~\cite{Astrand0F21}, \href{works/FrohnerTR19.pdf}{FrohnerTR19}~\cite{FrohnerTR19}, \href{works/abs-1911-04766.pdf}{abs-1911-04766}~\cite{abs-1911-04766}, \href{works/GeibingerMM19.pdf}{GeibingerMM19}~\cite{GeibingerMM19}, \href{works/LaborieRSV18.pdf}{LaborieRSV18}~\cite{LaborieRSV18}, \href{works/BurtLPS15.pdf}{BurtLPS15}~\cite{BurtLPS15}, \href{works/BofillEGPSV14.pdf}{BofillEGPSV14}~\cite{BofillEGPSV14}, \href{works/KovacsK11.pdf}{KovacsK11}~\cite{KovacsK11}, \href{works/KameugneFSN11.pdf}{KameugneFSN11}~\cite{KameugneFSN11}, \href{works/ThiruvadyBME09.pdf}{ThiruvadyBME09}~\cite{ThiruvadyBME09} & \href{works/ArmstrongGOS21.pdf}{ArmstrongGOS21}~\cite{ArmstrongGOS21}, \href{works/WessenCS20.pdf}{WessenCS20}~\cite{WessenCS20}, \href{works/WallaceY20.pdf}{WallaceY20}~\cite{WallaceY20}, \href{works/MengZRZL20.pdf}{MengZRZL20}~\cite{MengZRZL20}, \href{works/FrimodigS19.pdf}{FrimodigS19}~\cite{FrimodigS19}, \href{works/YangSS19.pdf}{YangSS19}~\cite{YangSS19}, \href{works/MusliuSS18.pdf}{MusliuSS18}~\cite{MusliuSS18}, \href{works/AstrandJZ18.pdf}{AstrandJZ18}~\cite{AstrandJZ18}, \href{works/GoldwaserS18.pdf}{GoldwaserS18}~\cite{GoldwaserS18}, \href{works/GoldwaserS17.pdf}{GoldwaserS17}~\cite{GoldwaserS17}, \href{works/PesantRR15.pdf}{PesantRR15}~\cite{PesantRR15}, \href{works/MonetteDD07.pdf}{MonetteDD07}~\cite{MonetteDD07}\\
CPSystems & Gurobi & \href{works/WangB23.pdf}{WangB23}~\cite{WangB23}, \href{works/NaderiRR23.pdf}{NaderiRR23}~\cite{NaderiRR23}, \href{works/LacknerMMWW23.pdf}{LacknerMMWW23}~\cite{LacknerMMWW23}, \href{works/WinterMMW22.pdf}{WinterMMW22}~\cite{WinterMMW22}, \href{works/ZhangBB22.pdf}{ZhangBB22}~\cite{ZhangBB22}, \href{works/KovacsTKSG21.pdf}{KovacsTKSG21}~\cite{KovacsTKSG21}, \href{works/GeibingerKKMMW21.pdf}{GeibingerKKMMW21}~\cite{GeibingerKKMMW21}, \href{works/KoehlerBFFHPSSS21.pdf}{KoehlerBFFHPSSS21}~\cite{KoehlerBFFHPSSS21}, \href{works/LacknerMMWW21.pdf}{LacknerMMWW21}~\cite{LacknerMMWW21}, \href{works/WangB20.pdf}{WangB20}~\cite{WangB20}, \href{works/WallaceY20.pdf}{WallaceY20}~\cite{WallaceY20}, \href{works/FrohnerTR19.pdf}{FrohnerTR19}~\cite{FrohnerTR19}, \href{works/MusliuSS18.pdf}{MusliuSS18}~\cite{MusliuSS18} & \href{works/VlkHT21.pdf}{VlkHT21}~\cite{VlkHT21}, \href{works/GoldwaserS18.pdf}{GoldwaserS18}~\cite{GoldwaserS18}, \href{works/GoldwaserS17.pdf}{GoldwaserS17}~\cite{GoldwaserS17}, \href{works/FontaineMH16.pdf}{FontaineMH16}~\cite{FontaineMH16} & \href{works/KimCMLLP23.pdf}{KimCMLLP23}~\cite{KimCMLLP23}, \href{works/abs-2305-19888.pdf}{abs-2305-19888}~\cite{abs-2305-19888}, \href{works/MontemanniD23.pdf}{MontemanniD23}~\cite{MontemanniD23}, \href{works/HeinzNVH22.pdf}{HeinzNVH22}~\cite{HeinzNVH22}, \href{works/PohlAK22.pdf}{PohlAK22}~\cite{PohlAK22}, \href{works/HubnerGSV21.pdf}{HubnerGSV21}~\cite{HubnerGSV21}, \href{works/FanXG21.pdf}{FanXG21}~\cite{FanXG21}, \href{works/KlankeBYE21.pdf}{KlankeBYE21}~\cite{KlankeBYE21}, \href{works/AbohashimaEG21.pdf}{AbohashimaEG21}~\cite{AbohashimaEG21}, \href{works/BenediktMH20.pdf}{BenediktMH20}~\cite{BenediktMH20}, \href{works/MengZRZL20.pdf}{MengZRZL20}~\cite{MengZRZL20}, \href{works/He0GLW18.pdf}{He0GLW18}~\cite{He0GLW18}, \href{works/DemirovicS18.pdf}{DemirovicS18}~\cite{DemirovicS18}, \href{works/BenediktSMVH18.pdf}{BenediktSMVH18}~\cite{BenediktSMVH18}, \href{works/BurtLPS15.pdf}{BurtLPS15}~\cite{BurtLPS15}, \href{works/PesantRR15.pdf}{PesantRR15}~\cite{PesantRR15}\\
CPSystems & Ilog Scheduler & \href{works/GrimesH11.pdf}{GrimesH11}~\cite{GrimesH11}, \href{works/ZeballosQH10.pdf}{ZeballosQH10}~\cite{ZeballosQH10} & \href{works/LaborieRSV18.pdf}{LaborieRSV18}~\cite{LaborieRSV18}, \href{works/NovasH12.pdf}{NovasH12}~\cite{NovasH12}, \href{works/HeinzB12.pdf}{HeinzB12}~\cite{HeinzB12}, \href{works/LimtanyakulS12.pdf}{LimtanyakulS12}~\cite{LimtanyakulS12}, \href{works/BeckFW11.pdf}{BeckFW11}~\cite{BeckFW11}, \href{works/HeckmanB11.pdf}{HeckmanB11}~\cite{HeckmanB11}, \href{works/GrimesHM09.pdf}{GrimesHM09}~\cite{GrimesHM09}, \href{works/WatsonB08.pdf}{WatsonB08}~\cite{WatsonB08}, \href{works/ZeballosH05.pdf}{ZeballosH05}~\cite{ZeballosH05}, \href{works/BeckR03.pdf}{BeckR03}~\cite{BeckR03}, \href{works/NuijtenP98.pdf}{NuijtenP98}~\cite{NuijtenP98} & \href{works/Laborie18a.pdf}{Laborie18a}~\cite{Laborie18a}, \href{works/SchuttS16.pdf}{SchuttS16}~\cite{SchuttS16}, \href{works/TranWDRFOVB16.pdf}{TranWDRFOVB16}~\cite{TranWDRFOVB16}, \href{works/NovasH14.pdf}{NovasH14}~\cite{NovasH14}, \href{works/TerekhovTDB14.pdf}{TerekhovTDB14}~\cite{TerekhovTDB14}, \href{works/BeniniLMR11.pdf}{BeniniLMR11}~\cite{BeniniLMR11}, \href{works/KovacsB11.pdf}{KovacsB11}~\cite{KovacsB11}, \href{works/SchuttFSW11.pdf}{SchuttFSW11}~\cite{SchuttFSW11}, \href{works/LahimerLH11.pdf}{LahimerLH11}~\cite{LahimerLH11}, \href{works/HachemiGR11.pdf}{HachemiGR11}~\cite{HachemiGR11}, \href{works/LopesCSM10.pdf}{LopesCSM10}~\cite{LopesCSM10}, \href{works/abs-1009-0347.pdf}{abs-1009-0347}~\cite{abs-1009-0347}, \href{works/NovasH10.pdf}{NovasH10}~\cite{NovasH10}, \href{works/Vilim09a.pdf}{Vilim09a}~\cite{Vilim09a}, \href{works/RuggieroBBMA09.pdf}{RuggieroBBMA09}~\cite{RuggieroBBMA09}, \href{works/BidotVLB09.pdf}{BidotVLB09}~\cite{BidotVLB09}, \href{works/KovacsB08.pdf}{KovacsB08}~\cite{KovacsB08}, \href{works/MouraSCL08a.pdf}{MouraSCL08a}~\cite{MouraSCL08a}, \href{works/MouraSCL08.pdf}{MouraSCL08}~\cite{MouraSCL08}, \href{works/HoeveGSL07.pdf}{HoeveGSL07}~\cite{HoeveGSL07}, \href{works/Rodriguez07.pdf}{Rodriguez07}~\cite{Rodriguez07}, \href{works/Simonis07.pdf}{Simonis07}~\cite{Simonis07}, \href{works/Beck07.pdf}{Beck07}~\cite{Beck07}, \href{works/BeckW07.pdf}{BeckW07}~\cite{BeckW07}, \href{works/KovacsV06.pdf}{KovacsV06}~\cite{KovacsV06}, \href{works/Hooker06.pdf}{Hooker06}~\cite{Hooker06}, \href{works/WuBB05.pdf}{WuBB05}~\cite{WuBB05}, \href{works/ArtiouchineB05.pdf}{ArtiouchineB05}~\cite{ArtiouchineB05}, \href{works/QuirogaZH05.pdf}{QuirogaZH05}~\cite{QuirogaZH05}... (Total: 41)\\
CPSystems & Ilog Solver &  & \href{works/GrimesH11.pdf}{GrimesH11}~\cite{GrimesH11}, \href{works/ZeballosQH10.pdf}{ZeballosQH10}~\cite{ZeballosQH10} & \href{works/abs-1902-01193.pdf}{abs-1902-01193}~\cite{abs-1902-01193}, \href{works/LaborieRSV18.pdf}{LaborieRSV18}~\cite{LaborieRSV18}, \href{works/ZarandiKS16.pdf}{ZarandiKS16}~\cite{ZarandiKS16}, \href{works/PesantRR15.pdf}{PesantRR15}~\cite{PesantRR15}, \href{works/BonfiettiLBM14.pdf}{BonfiettiLBM14}~\cite{BonfiettiLBM14}, \href{works/NovasH14.pdf}{NovasH14}~\cite{NovasH14}, \href{works/OzturkTHO13.pdf}{OzturkTHO13}~\cite{OzturkTHO13}, \href{works/BonfiettiLBM12.pdf}{BonfiettiLBM12}~\cite{BonfiettiLBM12}, \href{works/NovasH12.pdf}{NovasH12}~\cite{NovasH12}, \href{works/HeinzB12.pdf}{HeinzB12}~\cite{HeinzB12}, \href{works/LombardiM12a.pdf}{LombardiM12a}~\cite{LombardiM12a}, \href{works/KelbelH11.pdf}{KelbelH11}~\cite{KelbelH11}, \href{works/BonfiettiLBM11.pdf}{BonfiettiLBM11}~\cite{BonfiettiLBM11}, \href{works/KovacsK11.pdf}{KovacsK11}~\cite{KovacsK11}, \href{works/KovacsB11.pdf}{KovacsB11}~\cite{KovacsB11}, \href{works/TopalogluO11.pdf}{TopalogluO11}~\cite{TopalogluO11}, \href{works/BajestaniB11.pdf}{BajestaniB11}~\cite{BajestaniB11}, \href{works/LombardiM10.pdf}{LombardiM10}~\cite{LombardiM10}, \href{works/abs-1009-0347.pdf}{abs-1009-0347}~\cite{abs-1009-0347}, \href{works/LopesCSM10.pdf}{LopesCSM10}~\cite{LopesCSM10}, \href{works/LombardiM09.pdf}{LombardiM09}~\cite{LombardiM09}, \href{works/RuggieroBBMA09.pdf}{RuggieroBBMA09}~\cite{RuggieroBBMA09}, \href{works/MouraSCL08a.pdf}{MouraSCL08a}~\cite{MouraSCL08a}, \href{works/MouraSCL08.pdf}{MouraSCL08}~\cite{MouraSCL08}, \href{works/KovacsB08.pdf}{KovacsB08}~\cite{KovacsB08}, \href{works/Rodriguez07.pdf}{Rodriguez07}~\cite{Rodriguez07}, \href{works/KovacsB07.pdf}{KovacsB07}~\cite{KovacsB07}, \href{works/GomesHS06.pdf}{GomesHS06}~\cite{GomesHS06}, \href{works/BeniniBGM06.pdf}{BeniniBGM06}~\cite{BeniniBGM06}... (Total: 45)\\
CPSystems & MiniZinc & \href{works/LacknerMMWW23.pdf}{LacknerMMWW23}~\cite{LacknerMMWW23}, \href{works/TardivoDFMP23.pdf}{TardivoDFMP23}~\cite{TardivoDFMP23}, \href{works/ColT22.pdf}{ColT22}~\cite{ColT22}, \href{works/BoudreaultSLQ22.pdf}{BoudreaultSLQ22}~\cite{BoudreaultSLQ22}, \href{works/MullerMKP22.pdf}{MullerMKP22}~\cite{MullerMKP22}, \href{works/JungblutK22.pdf}{JungblutK22}~\cite{JungblutK22}, \href{works/ArmstrongGOS21.pdf}{ArmstrongGOS21}~\cite{ArmstrongGOS21}, \href{works/KoehlerBFFHPSSS21.pdf}{KoehlerBFFHPSSS21}~\cite{KoehlerBFFHPSSS21}, \href{works/LacknerMMWW21.pdf}{LacknerMMWW21}~\cite{LacknerMMWW21}, \href{works/Mercier-AubinGQ20.pdf}{Mercier-AubinGQ20}~\cite{Mercier-AubinGQ20}, \href{works/WallaceY20.pdf}{WallaceY20}~\cite{WallaceY20}, \href{works/abs-1911-04766.pdf}{abs-1911-04766}~\cite{abs-1911-04766}, \href{works/ColT19.pdf}{ColT19}~\cite{ColT19}, \href{works/FrohnerTR19.pdf}{FrohnerTR19}~\cite{FrohnerTR19}, \href{works/GeibingerMM19.pdf}{GeibingerMM19}~\cite{GeibingerMM19}, \href{works/YoungFS17.pdf}{YoungFS17}~\cite{YoungFS17}, \href{works/LiuCGM17.pdf}{LiuCGM17}~\cite{LiuCGM17}, \href{works/SzerediS16.pdf}{SzerediS16}~\cite{SzerediS16}, \href{works/BofillEGPSV14.pdf}{BofillEGPSV14}~\cite{BofillEGPSV14}, \href{works/KelarevaTK13.pdf}{KelarevaTK13}~\cite{KelarevaTK13} & \href{works/PovedaAA23.pdf}{PovedaAA23}~\cite{PovedaAA23}, \href{works/MusliuSS18.pdf}{MusliuSS18}~\cite{MusliuSS18}, \href{works/KreterSS17.pdf}{KreterSS17}~\cite{KreterSS17}, \href{works/KreterSS15.pdf}{KreterSS15}~\cite{KreterSS15} & \href{works/Bit-Monnot23.pdf}{Bit-Monnot23}~\cite{Bit-Monnot23}, \href{works/OuelletQ22.pdf}{OuelletQ22}~\cite{OuelletQ22}, \href{works/GeibingerKKMMW21.pdf}{GeibingerKKMMW21}~\cite{GeibingerKKMMW21}, \href{works/abs-2102-08778.pdf}{abs-2102-08778}~\cite{abs-2102-08778}, \href{works/abs-1901-07914.pdf}{abs-1901-07914}~\cite{abs-1901-07914}, \href{works/FrimodigS19.pdf}{FrimodigS19}~\cite{FrimodigS19}, \href{works/BehrensLM19.pdf}{BehrensLM19}~\cite{BehrensLM19}, \href{works/DemirovicS18.pdf}{DemirovicS18}~\cite{DemirovicS18}, \href{works/TranVNB17.pdf}{TranVNB17}~\cite{TranVNB17}, \href{works/FontaineMH16.pdf}{FontaineMH16}~\cite{FontaineMH16}, \href{works/SchuttS16.pdf}{SchuttS16}~\cite{SchuttS16}, \href{works/BurtLPS15.pdf}{BurtLPS15}~\cite{BurtLPS15}, \href{works/HeinzSB13.pdf}{HeinzSB13}~\cite{HeinzSB13}, \href{works/SchuttFS13.pdf}{SchuttFS13}~\cite{SchuttFS13}\\
CPSystems & Mistral & \href{works/JuvinHHL23.pdf}{JuvinHHL23}~\cite{JuvinHHL23}, \href{works/GrimesHM09.pdf}{GrimesHM09}~\cite{GrimesHM09} & \href{works/Bit-Monnot23.pdf}{Bit-Monnot23}~\cite{Bit-Monnot23}, \href{works/BillautHL12.pdf}{BillautHL12}~\cite{BillautHL12} & \href{works/SialaAH15.pdf}{SialaAH15}~\cite{SialaAH15}\\
CPSystems & OPL & \href{works/LacknerMMWW23.pdf}{LacknerMMWW23}~\cite{LacknerMMWW23}, \href{works/YunusogluY22.pdf}{YunusogluY22}~\cite{YunusogluY22}, \href{works/MullerMKP22.pdf}{MullerMKP22}~\cite{MullerMKP22}, \href{works/TouatBT22.pdf}{TouatBT22}~\cite{TouatBT22}, \href{works/ColT22.pdf}{ColT22}~\cite{ColT22}, \href{works/LacknerMMWW21.pdf}{LacknerMMWW21}~\cite{LacknerMMWW21}, \href{works/PandeyS21a.pdf}{PandeyS21a}~\cite{PandeyS21a}, \href{works/KoehlerBFFHPSSS21.pdf}{KoehlerBFFHPSSS21}~\cite{KoehlerBFFHPSSS21}, \href{works/QinDCS20.pdf}{QinDCS20}~\cite{QinDCS20}, \href{works/Novas19.pdf}{Novas19}~\cite{Novas19}, \href{works/EscobetPQPRA19.pdf}{EscobetPQPRA19}~\cite{EscobetPQPRA19}, \href{works/TangLWSK18.pdf}{TangLWSK18}~\cite{TangLWSK18}, \href{works/LaborieRSV18.pdf}{LaborieRSV18}~\cite{LaborieRSV18}, \href{works/NovaraNH16.pdf}{NovaraNH16}~\cite{NovaraNH16}, \href{works/AlesioNBG14.pdf}{AlesioNBG14}~\cite{AlesioNBG14}, \href{works/NovasH12.pdf}{NovasH12}~\cite{NovasH12}, \href{works/HachemiGR11.pdf}{HachemiGR11}~\cite{HachemiGR11}, \href{works/ZeballosQH10.pdf}{ZeballosQH10}~\cite{ZeballosQH10}, \href{works/Laborie09.pdf}{Laborie09}~\cite{Laborie09}, \href{works/KhayatLR06.pdf}{KhayatLR06}~\cite{KhayatLR06}, \href{works/AggounB93.pdf}{AggounB93}~\cite{AggounB93} & \href{works/SubulanC22.pdf}{SubulanC22}~\cite{SubulanC22}, \href{works/Teppan22.pdf}{Teppan22}~\cite{Teppan22}, \href{works/Mercier-AubinGQ20.pdf}{Mercier-AubinGQ20}~\cite{Mercier-AubinGQ20}, \href{works/ZouZ20.pdf}{ZouZ20}~\cite{ZouZ20}, \href{works/MurinR19.pdf}{MurinR19}~\cite{MurinR19}, \href{works/Laborie18a.pdf}{Laborie18a}~\cite{Laborie18a}, \href{works/LimBTBB15.pdf}{LimBTBB15}~\cite{LimBTBB15}, \href{works/WangMD15.pdf}{WangMD15}~\cite{WangMD15}, \href{works/EvenSH15a.pdf}{EvenSH15a}~\cite{EvenSH15a}, \href{works/NovasH14.pdf}{NovasH14}~\cite{NovasH14}, \href{works/OzturkTHO13.pdf}{OzturkTHO13}~\cite{OzturkTHO13}, \href{works/SerraNM12.pdf}{SerraNM12}~\cite{SerraNM12}, \href{works/HeinzB12.pdf}{HeinzB12}~\cite{HeinzB12}, \href{works/TopalogluO11.pdf}{TopalogluO11}~\cite{TopalogluO11}, \href{works/EdisO11.pdf}{EdisO11}~\cite{EdisO11}, \href{works/KelbelH11.pdf}{KelbelH11}~\cite{KelbelH11}, \href{works/ZibranR11a.pdf}{ZibranR11a}~\cite{ZibranR11a}, \href{works/NovasH10.pdf}{NovasH10}~\cite{NovasH10}, \href{works/Simonis07.pdf}{Simonis07}~\cite{Simonis07}, \href{works/GarganiR07.pdf}{GarganiR07}~\cite{GarganiR07}, \href{works/KrogtLPHJ07.pdf}{KrogtLPHJ07}~\cite{KrogtLPHJ07}, \href{works/Hooker06.pdf}{Hooker06}~\cite{Hooker06}, \href{works/ZeballosH05.pdf}{ZeballosH05}~\cite{ZeballosH05}, \href{works/QuirogaZH05.pdf}{QuirogaZH05}~\cite{QuirogaZH05}, \href{works/Hooker05a.pdf}{Hooker05a}~\cite{Hooker05a}, \href{works/LorigeonBB02.pdf}{LorigeonBB02}~\cite{LorigeonBB02}, \href{works/VerfaillieL01.pdf}{VerfaillieL01}~\cite{VerfaillieL01}, \href{works/RodosekW98.pdf}{RodosekW98}~\cite{RodosekW98} & \href{works/abs-2402-00459.pdf}{abs-2402-00459}~\cite{abs-2402-00459}, \href{works/GurPAE23.pdf}{GurPAE23}~\cite{GurPAE23}, \href{works/CzerniachowskaWZ23.pdf}{CzerniachowskaWZ23}~\cite{CzerniachowskaWZ23}, \href{works/MontemanniD23.pdf}{MontemanniD23}~\cite{MontemanniD23}, \href{works/IsikYA23.pdf}{IsikYA23}~\cite{IsikYA23}, \href{works/EfthymiouY23.pdf}{EfthymiouY23}~\cite{EfthymiouY23}, \href{works/YuraszeckMCCR23.pdf}{YuraszeckMCCR23}~\cite{YuraszeckMCCR23}, \href{works/PerezGSL23.pdf}{PerezGSL23}~\cite{PerezGSL23}, \href{works/AbreuNP23.pdf}{AbreuNP23}~\cite{AbreuNP23}, \href{works/abs-2312-13682.pdf}{abs-2312-13682}~\cite{abs-2312-13682}, \href{works/GeitzGSSW22.pdf}{GeitzGSSW22}~\cite{GeitzGSSW22}, \href{works/ArmstrongGOS22.pdf}{ArmstrongGOS22}~\cite{ArmstrongGOS22}, \href{works/BoudreaultSLQ22.pdf}{BoudreaultSLQ22}~\cite{BoudreaultSLQ22}, \href{works/OujanaAYB22.pdf}{OujanaAYB22}~\cite{OujanaAYB22}, \href{works/LiFJZLL22.pdf}{LiFJZLL22}~\cite{LiFJZLL22}, \href{works/ZhangBB22.pdf}{ZhangBB22}~\cite{ZhangBB22}, \href{works/VlkHT21.pdf}{VlkHT21}~\cite{VlkHT21}, \href{works/Bedhief21.pdf}{Bedhief21}~\cite{Bedhief21}, \href{works/HamPK21.pdf}{HamPK21}~\cite{HamPK21}, \href{works/QinWSLS21.pdf}{QinWSLS21}~\cite{QinWSLS21}, \href{works/abs-2102-08778.pdf}{abs-2102-08778}~\cite{abs-2102-08778}, \href{works/HubnerGSV21.pdf}{HubnerGSV21}~\cite{HubnerGSV21}, \href{works/WallaceY20.pdf}{WallaceY20}~\cite{WallaceY20}, \href{works/MengZRZL20.pdf}{MengZRZL20}~\cite{MengZRZL20}, \href{works/BogaerdtW19.pdf}{BogaerdtW19}~\cite{BogaerdtW19}, \href{works/YounespourAKE19.pdf}{YounespourAKE19}~\cite{YounespourAKE19}, \href{works/abs-1902-09244.pdf}{abs-1902-09244}~\cite{abs-1902-09244}, \href{works/Tom19.pdf}{Tom19}~\cite{Tom19}, \href{works/YangSS19.pdf}{YangSS19}~\cite{YangSS19}... (Total: 76)\\
CPSystems & OR-Tools & \href{works/abs-2402-00459.pdf}{abs-2402-00459}~\cite{abs-2402-00459}, \href{works/LacknerMMWW23.pdf}{LacknerMMWW23}~\cite{LacknerMMWW23}, \href{works/abs-2211-14492.pdf}{abs-2211-14492}~\cite{abs-2211-14492}, \href{works/ColT22.pdf}{ColT22}~\cite{ColT22}, \href{works/MullerMKP22.pdf}{MullerMKP22}~\cite{MullerMKP22}, \href{works/abs-2102-08778.pdf}{abs-2102-08778}~\cite{abs-2102-08778}, \href{works/KovacsTKSG21.pdf}{KovacsTKSG21}~\cite{KovacsTKSG21}, \href{works/LacknerMMWW21.pdf}{LacknerMMWW21}~\cite{LacknerMMWW21}, \href{works/KoehlerBFFHPSSS21.pdf}{KoehlerBFFHPSSS21}~\cite{KoehlerBFFHPSSS21}, \href{works/FallahiAC20.pdf}{FallahiAC20}~\cite{FallahiAC20}, \href{works/ColT19.pdf}{ColT19}~\cite{ColT19}, \href{works/GayHS15.pdf}{GayHS15}~\cite{GayHS15} & \href{works/EfthymiouY23.pdf}{EfthymiouY23}~\cite{EfthymiouY23}, \href{works/BoudreaultSLQ22.pdf}{BoudreaultSLQ22}~\cite{BoudreaultSLQ22}, \href{works/GeibingerKKMMW21.pdf}{GeibingerKKMMW21}~\cite{GeibingerKKMMW21}, \href{works/BarzegaranZP20.pdf}{BarzegaranZP20}~\cite{BarzegaranZP20}, \href{works/LiuCGM17.pdf}{LiuCGM17}~\cite{LiuCGM17} & \href{works/Bit-Monnot23.pdf}{Bit-Monnot23}~\cite{Bit-Monnot23}, \href{works/KimCMLLP23.pdf}{KimCMLLP23}~\cite{KimCMLLP23}, \href{works/MontemanniD23.pdf}{MontemanniD23}~\cite{MontemanniD23}, \href{works/AkramNHRSA23.pdf}{AkramNHRSA23}~\cite{AkramNHRSA23}, \href{works/MontemanniD23a.pdf}{MontemanniD23a}~\cite{MontemanniD23a}, \href{works/Teppan22.pdf}{Teppan22}~\cite{Teppan22}, \href{works/KlankeBYE21.pdf}{KlankeBYE21}~\cite{KlankeBYE21}, \href{works/MengZRZL20.pdf}{MengZRZL20}~\cite{MengZRZL20}, \href{works/GroleazNS20.pdf}{GroleazNS20}~\cite{GroleazNS20}, \href{works/GalleguillosKSB19.pdf}{GalleguillosKSB19}~\cite{GalleguillosKSB19}, \href{works/BehrensLM19.pdf}{BehrensLM19}~\cite{BehrensLM19}, \href{works/abs-1901-07914.pdf}{abs-1901-07914}~\cite{abs-1901-07914}, \href{works/YangSS19.pdf}{YangSS19}~\cite{YangSS19}, \href{works/PourDERB18.pdf}{PourDERB18}~\cite{PourDERB18}, \href{works/BonfiettiZLM16.pdf}{BonfiettiZLM16}~\cite{BonfiettiZLM16}, \href{works/ZhouGL15.pdf}{ZhouGL15}~\cite{ZhouGL15}, \href{works/LombardiM12.pdf}{LombardiM12}~\cite{LombardiM12}\\
CPSystems & OZ & \href{works/PrataAN23.pdf}{PrataAN23}~\cite{PrataAN23}, \href{works/NaderiRR23.pdf}{NaderiRR23}~\cite{NaderiRR23}, \href{works/CzerniachowskaWZ23.pdf}{CzerniachowskaWZ23}~\cite{CzerniachowskaWZ23}, \href{works/IsikYA23.pdf}{IsikYA23}~\cite{IsikYA23}, \href{works/YunusogluY22.pdf}{YunusogluY22}~\cite{YunusogluY22}, \href{works/WikarekS19.pdf}{WikarekS19}~\cite{WikarekS19}, \href{works/GokgurHO18.pdf}{GokgurHO18}~\cite{GokgurHO18}, \href{works/CohenHB17.pdf}{CohenHB17}~\cite{CohenHB17}, \href{works/TopalogluO11.pdf}{TopalogluO11}~\cite{TopalogluO11}, \href{works/NovasH10.pdf}{NovasH10}~\cite{NovasH10}, \href{works/RuggieroBBMA09.pdf}{RuggieroBBMA09}~\cite{RuggieroBBMA09}, \href{works/VanczaM01.pdf}{VanczaM01}~\cite{VanczaM01}, \href{works/SchildW00.pdf}{SchildW00}~\cite{SchildW00}, \href{works/BeldiceanuC94.pdf}{BeldiceanuC94}~\cite{BeldiceanuC94} & \href{works/GeitzGSSW22.pdf}{GeitzGSSW22}~\cite{GeitzGSSW22}, \href{works/BourreauGGLT22.pdf}{BourreauGGLT22}~\cite{BourreauGGLT22}, \href{works/AbreuN22.pdf}{AbreuN22}~\cite{AbreuN22}, \href{works/SubulanC22.pdf}{SubulanC22}~\cite{SubulanC22}, \href{works/PohlAK22.pdf}{PohlAK22}~\cite{PohlAK22}, \href{works/FanXG21.pdf}{FanXG21}~\cite{FanXG21}, \href{works/GodetLHS20.pdf}{GodetLHS20}~\cite{GodetLHS20}, \href{works/AstrandJZ20.pdf}{AstrandJZ20}~\cite{AstrandJZ20}, \href{works/WessenCS20.pdf}{WessenCS20}~\cite{WessenCS20}, \href{works/abs-1901-07914.pdf}{abs-1901-07914}~\cite{abs-1901-07914}, \href{works/LiuLH19.pdf}{LiuLH19}~\cite{LiuLH19}, \href{works/Novas19.pdf}{Novas19}~\cite{Novas19}, \href{works/BehrensLM19.pdf}{BehrensLM19}~\cite{BehrensLM19}, \href{works/Hooker17.pdf}{Hooker17}~\cite{Hooker17}, \href{works/BridiBLMB16.pdf}{BridiBLMB16}~\cite{BridiBLMB16}, \href{works/HebrardHJMPV16.pdf}{HebrardHJMPV16}~\cite{HebrardHJMPV16}, \href{works/BajestaniB13.pdf}{BajestaniB13}~\cite{BajestaniB13}, \href{works/EdisO11.pdf}{EdisO11}~\cite{EdisO11}, \href{works/GrimesH11.pdf}{GrimesH11}~\cite{GrimesH11}, \href{works/ZeballosQH10.pdf}{ZeballosQH10}~\cite{ZeballosQH10}, \href{works/BocewiczBB09.pdf}{BocewiczBB09}~\cite{BocewiczBB09}, \href{works/LiessM08.pdf}{LiessM08}~\cite{LiessM08}, \href{works/SureshMOK06.pdf}{SureshMOK06}~\cite{SureshMOK06}, \href{works/BeniniBGM06.pdf}{BeniniBGM06}~\cite{BeniniBGM06}, \href{works/GodardLN05.pdf}{GodardLN05}~\cite{GodardLN05}, \href{works/MaraveliasG04.pdf}{MaraveliasG04}~\cite{MaraveliasG04} & \href{works/Mehdizadeh-Somarin23.pdf}{Mehdizadeh-Somarin23}~\cite{Mehdizadeh-Somarin23}, \href{works/GurPAE23.pdf}{GurPAE23}~\cite{GurPAE23}, \href{works/MullerMKP22.pdf}{MullerMKP22}~\cite{MullerMKP22}, \href{works/CampeauG22.pdf}{CampeauG22}~\cite{CampeauG22}, \href{works/HebrardALLCMR22.pdf}{HebrardALLCMR22}~\cite{HebrardALLCMR22}, \href{works/ZhangJZL22.pdf}{ZhangJZL22}~\cite{ZhangJZL22}, \href{works/ArmstrongGOS22.pdf}{ArmstrongGOS22}~\cite{ArmstrongGOS22}, \href{works/FetgoD22.pdf}{FetgoD22}~\cite{FetgoD22}, \href{works/TouatBT22.pdf}{TouatBT22}~\cite{TouatBT22}, \href{works/abs-2211-14492.pdf}{abs-2211-14492}~\cite{abs-2211-14492}, \href{works/LiFJZLL22.pdf}{LiFJZLL22}~\cite{LiFJZLL22}, \href{works/PopovicCGNC22.pdf}{PopovicCGNC22}~\cite{PopovicCGNC22}, \href{works/AbreuAPNM21.pdf}{AbreuAPNM21}~\cite{AbreuAPNM21}, \href{works/ArmstrongGOS21.pdf}{ArmstrongGOS21}~\cite{ArmstrongGOS21}, \href{works/Bedhief21.pdf}{Bedhief21}~\cite{Bedhief21}, \href{works/LacknerMMWW21.pdf}{LacknerMMWW21}~\cite{LacknerMMWW21}, \href{works/QinWSLS21.pdf}{QinWSLS21}~\cite{QinWSLS21}, \href{works/PandeyS21a.pdf}{PandeyS21a}~\cite{PandeyS21a}, \href{works/WangB20.pdf}{WangB20}~\cite{WangB20}, \href{works/SacramentoSP20.pdf}{SacramentoSP20}~\cite{SacramentoSP20}, \href{works/FallahiAC20.pdf}{FallahiAC20}~\cite{FallahiAC20}, \href{works/abs-1911-04766.pdf}{abs-1911-04766}~\cite{abs-1911-04766}, \href{works/GurEA19.pdf}{GurEA19}~\cite{GurEA19}, \href{works/Tom19.pdf}{Tom19}~\cite{Tom19}, \href{works/abs-1902-09244.pdf}{abs-1902-09244}~\cite{abs-1902-09244}, \href{works/FrimodigS19.pdf}{FrimodigS19}~\cite{FrimodigS19}, \href{works/NishikawaSTT19.pdf}{NishikawaSTT19}~\cite{NishikawaSTT19}, \href{works/GalleguillosKSB19.pdf}{GalleguillosKSB19}~\cite{GalleguillosKSB19}, \href{works/ArbaouiY18.pdf}{ArbaouiY18}~\cite{ArbaouiY18}... (Total: 80)\\
CPSystems & SICStus & \href{works/ArmstrongGOS21.pdf}{ArmstrongGOS21}~\cite{ArmstrongGOS21}, \href{works/LetortCB15.pdf}{LetortCB15}~\cite{LetortCB15}, \href{works/LetortCB13.pdf}{LetortCB13}~\cite{LetortCB13}, \href{works/LetortBC12.pdf}{LetortBC12}~\cite{LetortBC12} & \href{works/MossigeGSMC17.pdf}{MossigeGSMC17}~\cite{MossigeGSMC17}, \href{works/SchuttFSW11.pdf}{SchuttFSW11}~\cite{SchuttFSW11}, \href{works/QuSN06.pdf}{QuSN06}~\cite{QuSN06} & \href{works/ArmstrongGOS22.pdf}{ArmstrongGOS22}~\cite{ArmstrongGOS22}, \href{works/PopovicCGNC22.pdf}{PopovicCGNC22}~\cite{PopovicCGNC22}, \href{works/YangSS19.pdf}{YangSS19}~\cite{YangSS19}, \href{works/Madi-WambaLOBM17.pdf}{Madi-WambaLOBM17}~\cite{Madi-WambaLOBM17}, \href{works/JelinekB16.pdf}{JelinekB16}~\cite{JelinekB16}, \href{works/BeldiceanuCDP11.pdf}{BeldiceanuCDP11}~\cite{BeldiceanuCDP11}, \href{works/TrojetHL11.pdf}{TrojetHL11}~\cite{TrojetHL11}, \href{works/BartakCS10.pdf}{BartakCS10}~\cite{BartakCS10}, \href{works/SchuttFSW09.pdf}{SchuttFSW09}~\cite{SchuttFSW09}, \href{works/BeldiceanuCP08.pdf}{BeldiceanuCP08}~\cite{BeldiceanuCP08}, \href{works/Geske05.pdf}{Geske05}~\cite{Geske05}, \href{works/Bartak02.pdf}{Bartak02}~\cite{Bartak02}, \href{works/BeldiceanuC02.pdf}{BeldiceanuC02}~\cite{BeldiceanuC02}\\
CPSystems & Z3 & \href{works/KoehlerBFFHPSSS21.pdf}{KoehlerBFFHPSSS21}~\cite{KoehlerBFFHPSSS21}, \href{works/YounespourAKE19.pdf}{YounespourAKE19}~\cite{YounespourAKE19}, \href{works/SureshMOK06.pdf}{SureshMOK06}~\cite{SureshMOK06} & \href{works/NaderiRR23.pdf}{NaderiRR23}~\cite{NaderiRR23}, \href{works/VlkHT21.pdf}{VlkHT21}~\cite{VlkHT21}, \href{works/WikarekS19.pdf}{WikarekS19}~\cite{WikarekS19}, \href{works/Zhou97.pdf}{Zhou97}~\cite{Zhou97} & \href{works/ZhangW18.pdf}{ZhangW18}~\cite{ZhangW18}, \href{works/BofillCSV17.pdf}{BofillCSV17}~\cite{BofillCSV17}, \href{works/BertholdHLMS10.pdf}{BertholdHLMS10}~\cite{BertholdHLMS10}, \href{works/Rodriguez07.pdf}{Rodriguez07}~\cite{Rodriguez07}, \href{works/Zhou96.pdf}{Zhou96}~\cite{Zhou96}\\
\end{longtable}
}


\clearpage
\subsection{Concept Type ApplicationAreas}
\label{sec:ApplicationAreas}
{\scriptsize
\begin{longtable}{lp{3cm}>{\raggedright\arraybackslash}p{6cm}>{\raggedright\arraybackslash}p{6cm}>{\raggedright\arraybackslash}p{8cm}}
\rowcolor{white}\caption{Works for Concepts of Type ApplicationAreas}\\ \toprule
\rowcolor{white}Type & Keyword & High & Medium & Low\\ \midrule\endhead
\bottomrule
\endfoot
ApplicationAreas & COVID & \href{../works/GuoZ23.pdf}{GuoZ23}~\cite{GuoZ23} & \href{../works/GeibingerKKMMW21.pdf}{GeibingerKKMMW21}~\cite{GeibingerKKMMW21} & \href{../works/BonninMNE24.pdf}{BonninMNE24}~\cite{BonninMNE24}, \href{../works/Mehdizadeh-Somarin23.pdf}{Mehdizadeh-Somarin23}~\cite{Mehdizadeh-Somarin23}, \href{../works/JuvinHL23a.pdf}{JuvinHL23a}~\cite{JuvinHL23a}, \href{../works/Fatemi-AnarakiTFV23.pdf}{Fatemi-AnarakiTFV23}~\cite{Fatemi-AnarakiTFV23}, \href{../works/GurPAE23.pdf}{GurPAE23}~\cite{GurPAE23}, \href{../works/OujanaAYB22.pdf}{OujanaAYB22}~\cite{OujanaAYB22}, \href{../works/Lemos21.pdf}{Lemos21}~\cite{Lemos21}\\
ApplicationAreas & HVAC & \href{../works/LimHTB16.pdf}{LimHTB16}~\cite{LimHTB16}, \href{../works/LimBTBB15.pdf}{LimBTBB15}~\cite{LimBTBB15}, \href{../works/GrimesIOS14.pdf}{GrimesIOS14}~\cite{GrimesIOS14} &  & \\
ApplicationAreas & agriculture &  &  & \href{../works/AkramNHRSA23.pdf}{AkramNHRSA23}~\cite{AkramNHRSA23}, \href{../works/BenderWS21.pdf}{BenderWS21}~\cite{BenderWS21}, \href{../works/Astrand0F21.pdf}{Astrand0F21}~\cite{Astrand0F21}, \href{../works/HamPK21.pdf}{HamPK21}~\cite{HamPK21}, \href{../works/Astrand21.pdf}{Astrand21}~\cite{Astrand21}, \href{../works/QinWSLS21.pdf}{QinWSLS21}~\cite{QinWSLS21}, \href{../works/MejiaY20.pdf}{MejiaY20}~\cite{MejiaY20}\\
ApplicationAreas & aircraft & \href{../works/GokPTGO23.pdf}{GokPTGO23}~\cite{GokPTGO23}, \href{../works/PohlAK22.pdf}{PohlAK22}~\cite{PohlAK22}, \href{../works/OrnekOS20.pdf}{OrnekOS20}~\cite{OrnekOS20}, \href{../works/WangB20.pdf}{WangB20}~\cite{WangB20}, \href{../works/GokGSTO20.pdf}{GokGSTO20}~\cite{GokGSTO20}, \href{../works/TranDRFWOVB16.pdf}{TranDRFWOVB16}~\cite{TranDRFWOVB16}, \href{../works/Fahimi16.pdf}{Fahimi16}~\cite{Fahimi16}, \href{../works/BajestaniB13.pdf}{BajestaniB13}~\cite{BajestaniB13}, \href{../works/LombardiM12.pdf}{LombardiM12}~\cite{LombardiM12}, \href{../works/BajestaniB11.pdf}{BajestaniB11}~\cite{BajestaniB11}, \href{../works/ArtiouchineB05.pdf}{ArtiouchineB05}~\cite{ArtiouchineB05}, \href{../works/FrankK05.pdf}{FrankK05}~\cite{FrankK05}, \href{../works/Simonis99.pdf}{Simonis99}~\cite{Simonis99} & \href{../works/WangB23.pdf}{WangB23}~\cite{WangB23}, \href{../works/GombolayWS18.pdf}{GombolayWS18}~\cite{GombolayWS18}, \href{../works/Ham18.pdf}{Ham18}~\cite{Ham18}, \href{../works/Simonis07.pdf}{Simonis07}~\cite{Simonis07}, \href{../works/SakkoutW00.pdf}{SakkoutW00}~\cite{SakkoutW00}, \href{../works/Simonis95a.pdf}{Simonis95a}~\cite{Simonis95a} & \href{../works/PrataAN23.pdf}{PrataAN23}~\cite{PrataAN23}, \href{../works/PovedaAA23.pdf}{PovedaAA23}~\cite{PovedaAA23}, \href{../works/Adelgren2023.pdf}{Adelgren2023}~\cite{Adelgren2023}, \href{../works/ElciOH22.pdf}{ElciOH22}~\cite{ElciOH22}, \href{../works/EtminaniesfahaniGNMS22.pdf}{EtminaniesfahaniGNMS22}~\cite{EtminaniesfahaniGNMS22}, \href{../works/ZarandiASC20.pdf}{ZarandiASC20}~\cite{ZarandiASC20}, \href{../works/HauderBRPA20.pdf}{HauderBRPA20}~\cite{HauderBRPA20}, \href{../works/abs-1902-09244.pdf}{abs-1902-09244}~\cite{abs-1902-09244}, \href{../works/Hooker19.pdf}{Hooker19}~\cite{Hooker19}, \href{../works/LaborieRSV18.pdf}{LaborieRSV18}~\cite{LaborieRSV18}, \href{../works/HookerH17.pdf}{HookerH17}~\cite{HookerH17}, \href{../works/TranAB16.pdf}{TranAB16}~\cite{TranAB16}, \href{../works/Lombardi10.pdf}{Lombardi10}~\cite{Lombardi10}, \href{../works/Laborie09.pdf}{Laborie09}~\cite{Laborie09}, \href{../works/KovacsB08.pdf}{KovacsB08}~\cite{KovacsB08}, \href{../works/KrogtLPHJ07.pdf}{KrogtLPHJ07}~\cite{KrogtLPHJ07}, \href{../works/MartinPY01.pdf}{MartinPY01}~\cite{MartinPY01}, \href{../works/SimonisCK00.pdf}{SimonisCK00}~\cite{SimonisCK00}, \href{../works/GruianK98.pdf}{GruianK98}~\cite{GruianK98}, \href{../works/Darby-DowmanLMZ97.pdf}{Darby-DowmanLMZ97}~\cite{Darby-DowmanLMZ97}, \href{../works/Wallace96.pdf}{Wallace96}~\cite{Wallace96}, \href{../works/Simonis95.pdf}{Simonis95}~\cite{Simonis95}, \href{../works/SimonisC95.pdf}{SimonisC95}~\cite{SimonisC95}\\
ApplicationAreas & astronomy &  & \href{../works/FrankK05.pdf}{FrankK05}~\cite{FrankK05} & \href{../works/CatusseCBL16.pdf}{CatusseCBL16}~\cite{CatusseCBL16}, \href{../works/LiW08.pdf}{LiW08}~\cite{LiW08}\\
ApplicationAreas & automotive &  & \href{../works/GuoZ23.pdf}{GuoZ23}~\cite{GuoZ23}, \href{../works/YuraszeckMPV22.pdf}{YuraszeckMPV22}~\cite{YuraszeckMPV22}, \href{../works/EmdeZD22.pdf}{EmdeZD22}~\cite{EmdeZD22}, \href{../works/Groleaz21.pdf}{Groleaz21}~\cite{Groleaz21}, \href{../works/LimtanyakulS12.pdf}{LimtanyakulS12}~\cite{LimtanyakulS12}, \href{../works/SunLYL10.pdf}{SunLYL10}~\cite{SunLYL10}, \href{../works/Lombardi10.pdf}{Lombardi10}~\cite{Lombardi10}, \href{../works/BarlattCG08.pdf}{BarlattCG08}~\cite{BarlattCG08}, \href{../works/SchildW00.pdf}{SchildW00}~\cite{SchildW00} & \href{../works/PovedaAA23.pdf}{PovedaAA23}~\cite{PovedaAA23}, \href{../works/CzerniachowskaWZ23.pdf}{CzerniachowskaWZ23}~\cite{CzerniachowskaWZ23}, \href{../works/NaderiRR23.pdf}{NaderiRR23}~\cite{NaderiRR23}, \href{../works/NaderiBZ22.pdf}{NaderiBZ22}~\cite{NaderiBZ22}, \href{../works/NaderiBZ22a.pdf}{NaderiBZ22a}~\cite{NaderiBZ22a}, \href{../works/AntuoriHHEN21.pdf}{AntuoriHHEN21}~\cite{AntuoriHHEN21}, \href{../works/HubnerGSV21.pdf}{HubnerGSV21}~\cite{HubnerGSV21}, \href{../works/VlkHT21.pdf}{VlkHT21}~\cite{VlkHT21}, \href{../works/AbreuAPNM21.pdf}{AbreuAPNM21}~\cite{AbreuAPNM21}, \href{../works/KoehlerBFFHPSSS21.pdf}{KoehlerBFFHPSSS21}~\cite{KoehlerBFFHPSSS21}, \href{../works/BarzegaranZP20.pdf}{BarzegaranZP20}~\cite{BarzegaranZP20}, \href{../works/abs-1911-04766.pdf}{abs-1911-04766}~\cite{abs-1911-04766}, \href{../works/GeibingerMM19.pdf}{GeibingerMM19}~\cite{GeibingerMM19}, \href{../works/BonfiettiZLM16.pdf}{BonfiettiZLM16}~\cite{BonfiettiZLM16}, \href{../works/Siala15.pdf}{Siala15}~\cite{Siala15}, \href{../works/Siala15a.pdf}{Siala15a}~\cite{Siala15a}, \href{../works/SchnellH15.pdf}{SchnellH15}~\cite{SchnellH15}, \href{../works/AlesioNBG14.pdf}{AlesioNBG14}~\cite{AlesioNBG14}, \href{../works/HarjunkoskiMBC14.pdf}{HarjunkoskiMBC14}~\cite{HarjunkoskiMBC14}, \href{../works/BeniniBGM06.pdf}{BeniniBGM06}~\cite{BeniniBGM06}, \href{../works/KovacsV06.pdf}{KovacsV06}~\cite{KovacsV06}, \href{../works/Wallace96.pdf}{Wallace96}~\cite{Wallace96}\\
ApplicationAreas & business process & \href{../works/BadicaBI20.pdf}{BadicaBI20}~\cite{BadicaBI20}, \href{../works/Lombardi10.pdf}{Lombardi10}~\cite{Lombardi10}, \href{../works/LombardiM10a.pdf}{LombardiM10a}~\cite{LombardiM10a} &  & \href{../works/SubulanC22.pdf}{SubulanC22}~\cite{SubulanC22}, \href{../works/Zahout21.pdf}{Zahout21}~\cite{Zahout21}, \href{../works/Groleaz21.pdf}{Groleaz21}~\cite{Groleaz21}, \href{../works/ZarandiASC20.pdf}{ZarandiASC20}~\cite{ZarandiASC20}, \href{../works/BadicaBIL19.pdf}{BadicaBIL19}~\cite{BadicaBIL19}, \href{../works/Jans09.pdf}{Jans09}~\cite{Jans09}, \href{../works/Simonis07.pdf}{Simonis07}~\cite{Simonis07}, \href{../works/SimonisCK00.pdf}{SimonisCK00}~\cite{SimonisCK00}, \href{../works/Simonis99.pdf}{Simonis99}~\cite{Simonis99}, \href{../works/BeckF98.pdf}{BeckF98}~\cite{BeckF98}, \href{../works/Simonis95a.pdf}{Simonis95a}~\cite{Simonis95a}\\
ApplicationAreas & cable tree & \href{../works/KoehlerBFFHPSSS21.pdf}{KoehlerBFFHPSSS21}~\cite{KoehlerBFFHPSSS21} &  & \\
ApplicationAreas & car manufacturing &  & \href{../works/AntuoriHHEN21.pdf}{AntuoriHHEN21}~\cite{AntuoriHHEN21} & \href{../works/BeldiceanuC94.pdf}{BeldiceanuC94}~\cite{BeldiceanuC94}\\
ApplicationAreas & container terminal & \href{../works/QinDCS20.pdf}{QinDCS20}~\cite{QinDCS20}, \href{../works/SacramentoSP20.pdf}{SacramentoSP20}~\cite{SacramentoSP20} & \href{../works/LaborieRSV18.pdf}{LaborieRSV18}~\cite{LaborieRSV18} & \href{../works/abs-2312-13682.pdf}{abs-2312-13682}~\cite{abs-2312-13682}, \href{../works/PerezGSL23.pdf}{PerezGSL23}~\cite{PerezGSL23}, \href{../works/TouatBT22.pdf}{TouatBT22}~\cite{TouatBT22}, \href{../works/CauwelaertDS20.pdf}{CauwelaertDS20}~\cite{CauwelaertDS20}, \href{../works/WallaceY20.pdf}{WallaceY20}~\cite{WallaceY20}, \href{../works/ZarandiASC20.pdf}{ZarandiASC20}~\cite{ZarandiASC20}, \href{../works/FallahiAC20.pdf}{FallahiAC20}~\cite{FallahiAC20}, \href{../works/Hooker19.pdf}{Hooker19}~\cite{Hooker19}, \href{../works/CauwelaertDMS16.pdf}{CauwelaertDMS16}~\cite{CauwelaertDMS16}, \href{../works/Dejemeppe16.pdf}{Dejemeppe16}~\cite{Dejemeppe16}, \href{../works/DejemeppeCS15.pdf}{DejemeppeCS15}~\cite{DejemeppeCS15}, \href{../works/NovasH12.pdf}{NovasH12}~\cite{NovasH12}, \href{../works/CorreaLR07.pdf}{CorreaLR07}~\cite{CorreaLR07}, \href{../works/LimRX04.pdf}{LimRX04}~\cite{LimRX04}\\
ApplicationAreas & crew-scheduling & \href{../works/ZarandiASC20.pdf}{ZarandiASC20}~\cite{ZarandiASC20}, \href{../works/PourDERB18.pdf}{PourDERB18}~\cite{PourDERB18} & \href{../works/BourreauGGLT22.pdf}{BourreauGGLT22}~\cite{BourreauGGLT22}, \href{../works/Zahout21.pdf}{Zahout21}~\cite{Zahout21}, \href{../works/GombolayWS18.pdf}{GombolayWS18}~\cite{GombolayWS18}, \href{../works/Mason01.pdf}{Mason01}~\cite{Mason01}, \href{../works/Touraivane95.pdf}{Touraivane95}~\cite{Touraivane95} & \href{../works/NaderiRR23.pdf}{NaderiRR23}~\cite{NaderiRR23}, \href{../works/WangB23.pdf}{WangB23}~\cite{WangB23}, \href{../works/Adelgren2023.pdf}{Adelgren2023}~\cite{Adelgren2023}, \href{../works/NaderiBZ22a.pdf}{NaderiBZ22a}~\cite{NaderiBZ22a}, \href{../works/NaderiBZ22.pdf}{NaderiBZ22}~\cite{NaderiBZ22}, \href{../works/ElciOH22.pdf}{ElciOH22}~\cite{ElciOH22}, \href{../works/EtminaniesfahaniGNMS22.pdf}{EtminaniesfahaniGNMS22}~\cite{EtminaniesfahaniGNMS22}, \href{../works/HeinzNVH22.pdf}{HeinzNVH22}~\cite{HeinzNVH22}, \href{../works/Lemos21.pdf}{Lemos21}~\cite{Lemos21}, \href{../works/MokhtarzadehTNF20.pdf}{MokhtarzadehTNF20}~\cite{MokhtarzadehTNF20}, \href{../works/TangLWSK18.pdf}{TangLWSK18}~\cite{TangLWSK18}, \href{../works/HookerH17.pdf}{HookerH17}~\cite{HookerH17}, \href{../works/DoulabiRP16.pdf}{DoulabiRP16}~\cite{DoulabiRP16}, \href{../works/LipovetzkyBPS14.pdf}{LipovetzkyBPS14}~\cite{LipovetzkyBPS14}, \href{../works/HachemiGR11.pdf}{HachemiGR11}~\cite{HachemiGR11}, \href{../works/MilanoW09.pdf}{MilanoW09}~\cite{MilanoW09}, \href{../works/WuBB09.pdf}{WuBB09}~\cite{WuBB09}, \href{../works/MilanoW06.pdf}{MilanoW06}~\cite{MilanoW06}, \href{../works/BeldiceanuC02.pdf}{BeldiceanuC02}~\cite{BeldiceanuC02}, \href{../works/JainG01.pdf}{JainG01}~\cite{JainG01}, \href{../works/SimonisCK00.pdf}{SimonisCK00}~\cite{SimonisCK00}\\
ApplicationAreas & dairies &  &  & \href{../works/Bartak02.pdf}{Bartak02}~\cite{Bartak02}, \href{../works/Bartak02a.pdf}{Bartak02a}~\cite{Bartak02a}\\
ApplicationAreas & dairy & \href{../works/EscobetPQPRA19.pdf}{EscobetPQPRA19}~\cite{EscobetPQPRA19} & \href{../works/PrataAN23.pdf}{PrataAN23}~\cite{PrataAN23}, \href{../works/HarjunkoskiMBC14.pdf}{HarjunkoskiMBC14}~\cite{HarjunkoskiMBC14} & \href{../works/Groleaz21.pdf}{Groleaz21}~\cite{Groleaz21}\\
ApplicationAreas & datacenter & \href{../works/HermenierDL11.pdf}{HermenierDL11}~\cite{HermenierDL11} &  & \href{../works/Zahout21.pdf}{Zahout21}~\cite{Zahout21}, \href{../works/GalleguillosKSB19.pdf}{GalleguillosKSB19}~\cite{GalleguillosKSB19}, \href{../works/Madi-WambaLOBM17.pdf}{Madi-WambaLOBM17}~\cite{Madi-WambaLOBM17}, \href{../works/Letort13.pdf}{Letort13}~\cite{Letort13}, \href{../works/IfrimOS12.pdf}{IfrimOS12}~\cite{IfrimOS12}, \href{../works/LetortBC12.pdf}{LetortBC12}~\cite{LetortBC12}\\
ApplicationAreas & datacentre &  & \href{../works/HurleyOS16.pdf}{HurleyOS16}~\cite{HurleyOS16} & \\
ApplicationAreas & day-ahead market &  &  & \\
ApplicationAreas & deep space &  &  & \href{../works/HebrardALLCMR22.pdf}{HebrardALLCMR22}~\cite{HebrardALLCMR22}\\
ApplicationAreas & drone & \href{../works/MontemanniD23a.pdf}{MontemanniD23a}~\cite{MontemanniD23a}, \href{../works/MontemanniD23.pdf}{MontemanniD23}~\cite{MontemanniD23}, \href{../works/Ham18.pdf}{Ham18}~\cite{Ham18} &  & \href{../works/Adelgren2023.pdf}{Adelgren2023}~\cite{Adelgren2023}, \href{../works/ShaikhK23.pdf}{ShaikhK23}~\cite{ShaikhK23}, \href{../works/GuoZ23.pdf}{GuoZ23}~\cite{GuoZ23}, \href{../works/JuvinHL23a.pdf}{JuvinHL23a}~\cite{JuvinHL23a}, \href{../works/EmdeZD22.pdf}{EmdeZD22}~\cite{EmdeZD22}, \href{../works/Astrand21.pdf}{Astrand21}~\cite{Astrand21}, \href{../works/Astrand0F21.pdf}{Astrand0F21}~\cite{Astrand0F21}, \href{../works/AntuoriHHEN21.pdf}{AntuoriHHEN21}~\cite{AntuoriHHEN21}, \href{../works/ZarandiASC20.pdf}{ZarandiASC20}~\cite{ZarandiASC20}, \href{../works/Ham18a.pdf}{Ham18a}~\cite{Ham18a}\\
ApplicationAreas & earth observation & \href{../works/SquillaciPR23.pdf}{SquillaciPR23}~\cite{SquillaciPR23}, \href{../works/KucukY19.pdf}{KucukY19}~\cite{KucukY19}, \href{../works/VerfaillieL01.pdf}{VerfaillieL01}~\cite{VerfaillieL01} & \href{../works/BensanaLV99.pdf}{BensanaLV99}~\cite{BensanaLV99} & \href{../works/HebrardHJMPV16.pdf}{HebrardHJMPV16}~\cite{HebrardHJMPV16}, \href{../works/PraletLJ15.pdf}{PraletLJ15}~\cite{PraletLJ15}, \href{../works/SimoninAHL15.pdf}{SimoninAHL15}~\cite{SimoninAHL15}, \href{../works/KelarevaTK13.pdf}{KelarevaTK13}~\cite{KelarevaTK13}, \href{../works/OddiPCC03.pdf}{OddiPCC03}~\cite{OddiPCC03}\\
ApplicationAreas & earth orbit &  &  & \href{../works/SquillaciPR23.pdf}{SquillaciPR23}~\cite{SquillaciPR23}\\
ApplicationAreas & electroplating &  & \href{../works/RodosekW98.pdf}{RodosekW98}~\cite{RodosekW98} & \href{../works/Fatemi-AnarakiTFV23.pdf}{Fatemi-AnarakiTFV23}~\cite{Fatemi-AnarakiTFV23}, \href{../works/EfthymiouY23.pdf}{EfthymiouY23}~\cite{EfthymiouY23}, \href{../works/WallaceY20.pdf}{WallaceY20}~\cite{WallaceY20}, \href{../works/NovasH12.pdf}{NovasH12}~\cite{NovasH12}\\
ApplicationAreas & emergency service &  & \href{../works/EvenSH15a.pdf}{EvenSH15a}~\cite{EvenSH15a}, \href{../works/TopalogluO11.pdf}{TopalogluO11}~\cite{TopalogluO11} & \href{../works/ForbesHJST24.pdf}{ForbesHJST24}~\cite{ForbesHJST24}, \href{../works/EvenSH15.pdf}{EvenSH15}~\cite{EvenSH15}, \href{../works/SakkoutW00.pdf}{SakkoutW00}~\cite{SakkoutW00}\\
ApplicationAreas & energy-price & \href{../works/GrimesIOS14.pdf}{GrimesIOS14}~\cite{GrimesIOS14}, \href{../works/IfrimOS12.pdf}{IfrimOS12}~\cite{IfrimOS12} & \href{../works/HurleyOS16.pdf}{HurleyOS16}~\cite{HurleyOS16}, \href{../works/Froger16.pdf}{Froger16}~\cite{Froger16} & \href{../works/PrataAN23.pdf}{PrataAN23}~\cite{PrataAN23}, \href{../works/EscobetPQPRA19.pdf}{EscobetPQPRA19}~\cite{EscobetPQPRA19}, \href{../works/He0GLW18.pdf}{He0GLW18}~\cite{He0GLW18}, \href{../works/BenediktSMVH18.pdf}{BenediktSMVH18}~\cite{BenediktSMVH18}, \href{../works/LimHTB16.pdf}{LimHTB16}~\cite{LimHTB16}\\
ApplicationAreas & evacuation & \href{../works/ArtiguesHQT21.pdf}{ArtiguesHQT21}~\cite{ArtiguesHQT21}, \href{../works/ZarandiASC20.pdf}{ZarandiASC20}~\cite{ZarandiASC20}, \href{../works/YangSS19.pdf}{YangSS19}~\cite{YangSS19}, \href{../works/EvenSH15.pdf}{EvenSH15}~\cite{EvenSH15}, \href{../works/EvenSH15a.pdf}{EvenSH15a}~\cite{EvenSH15a} &  & \href{../works/Beck99.pdf}{Beck99}~\cite{Beck99}\\
ApplicationAreas & farming &  &  & \href{../works/WinterMMW22.pdf}{WinterMMW22}~\cite{WinterMMW22}, \href{../works/Astrand0F21.pdf}{Astrand0F21}~\cite{Astrand0F21}\\
ApplicationAreas & forestry & \href{../works/HachemiGR11.pdf}{HachemiGR11}~\cite{HachemiGR11} &  & \href{../works/Astrand0F21.pdf}{Astrand0F21}~\cite{Astrand0F21}\\
ApplicationAreas & high performance computing & \href{../works/BorghesiBLMB18.pdf}{BorghesiBLMB18}~\cite{BorghesiBLMB18} & \href{../works/GalleguillosKSB19.pdf}{GalleguillosKSB19}~\cite{GalleguillosKSB19} & \href{../works/abs-2305-19888.pdf}{abs-2305-19888}~\cite{abs-2305-19888}, \href{../works/HeinzNVH22.pdf}{HeinzNVH22}~\cite{HeinzNVH22}, \href{../works/Zahout21.pdf}{Zahout21}~\cite{Zahout21}, \href{../works/LunardiBLRV20.pdf}{LunardiBLRV20}~\cite{LunardiBLRV20}, \href{../works/Lunardi20.pdf}{Lunardi20}~\cite{Lunardi20}, \href{../works/TranPZLDB18.pdf}{TranPZLDB18}~\cite{TranPZLDB18}, \href{../works/RiahiNS018.pdf}{RiahiNS018}~\cite{RiahiNS018}, \href{../works/HurleyOS16.pdf}{HurleyOS16}~\cite{HurleyOS16}, \href{../works/BartoliniBBLM14.pdf}{BartoliniBBLM14}~\cite{BartoliniBBLM14}\\
ApplicationAreas & high school timetabling & \href{../works/DemirovicS18.pdf}{DemirovicS18}~\cite{DemirovicS18} &  & \href{../works/Lemos21.pdf}{Lemos21}~\cite{Lemos21}, \href{../works/BofillGSV15.pdf}{BofillGSV15}~\cite{BofillGSV15}, \href{../works/KanetAG04.pdf}{KanetAG04}~\cite{KanetAG04}, \href{../works/ElkhyariGJ02a.pdf}{ElkhyariGJ02a}~\cite{ElkhyariGJ02a}\\
ApplicationAreas & hoist & \href{../works/EfthymiouY23.pdf}{EfthymiouY23}~\cite{EfthymiouY23}, \href{../works/WallaceY20.pdf}{WallaceY20}~\cite{WallaceY20}, \href{../works/RodosekW98.pdf}{RodosekW98}~\cite{RodosekW98} & \href{../works/Fatemi-AnarakiTFV23.pdf}{Fatemi-AnarakiTFV23}~\cite{Fatemi-AnarakiTFV23}, \href{../works/NovasH12.pdf}{NovasH12}~\cite{NovasH12}, \href{../works/BonfiettiLBM11.pdf}{BonfiettiLBM11}~\cite{BonfiettiLBM11} & \href{../works/AstrandJZ18.pdf}{AstrandJZ18}~\cite{AstrandJZ18}, \href{../works/BonfiettiLBM14.pdf}{BonfiettiLBM14}~\cite{BonfiettiLBM14}, \href{../works/BonfiettiM12.pdf}{BonfiettiM12}~\cite{BonfiettiM12}, \href{../works/BonfiettiLBM12.pdf}{BonfiettiLBM12}~\cite{BonfiettiLBM12}, \href{../works/LombardiBMB11.pdf}{LombardiBMB11}~\cite{LombardiBMB11}, \href{../works/Wallace06.pdf}{Wallace06}~\cite{Wallace06}, \href{../works/BeckR03.pdf}{BeckR03}~\cite{BeckR03}, \href{../works/Baptiste02.pdf}{Baptiste02}~\cite{Baptiste02}, \href{../works/KorbaaYG99.pdf}{KorbaaYG99}~\cite{KorbaaYG99}, \href{../works/PapaB98.pdf}{PapaB98}~\cite{PapaB98}\\
ApplicationAreas & maintenance scheduling & \href{../works/PopovicCGNC22.pdf}{PopovicCGNC22}~\cite{PopovicCGNC22}, \href{../works/Froger16.pdf}{Froger16}~\cite{Froger16}, \href{../works/BajestaniB13.pdf}{BajestaniB13}~\cite{BajestaniB13}, \href{../works/Malapert11.pdf}{Malapert11}~\cite{Malapert11} & \href{../works/PenzDN23.pdf}{PenzDN23}~\cite{PenzDN23}, \href{../works/AntunesABD20.pdf}{AntunesABD20}~\cite{AntunesABD20}, \href{../works/BajestaniB11.pdf}{BajestaniB11}~\cite{BajestaniB11}, \href{../works/Davenport10.pdf}{Davenport10}~\cite{Davenport10}, \href{../works/FrostD98.pdf}{FrostD98}~\cite{FrostD98} & \href{../works/BourreauGGLT22.pdf}{BourreauGGLT22}~\cite{BourreauGGLT22}, \href{../works/Godet21a.pdf}{Godet21a}~\cite{Godet21a}, \href{../works/ZarandiASC20.pdf}{ZarandiASC20}~\cite{ZarandiASC20}, \href{../works/Hooker19.pdf}{Hooker19}~\cite{Hooker19}, \href{../works/PourDERB18.pdf}{PourDERB18}~\cite{PourDERB18}, \href{../works/AntunesABD18.pdf}{AntunesABD18}~\cite{AntunesABD18}, \href{../works/Nattaf16.pdf}{Nattaf16}~\cite{Nattaf16}, \href{../works/BajestaniB15.pdf}{BajestaniB15}~\cite{BajestaniB15}, \href{../works/Simonis99.pdf}{Simonis99}~\cite{Simonis99}, \href{../works/Puget95.pdf}{Puget95}~\cite{Puget95}, \href{../works/SimonisC95.pdf}{SimonisC95}~\cite{SimonisC95}\\
ApplicationAreas & medical & \href{../works/ShinBBHO18.pdf}{ShinBBHO18}~\cite{ShinBBHO18}, \href{../works/Dejemeppe16.pdf}{Dejemeppe16}~\cite{Dejemeppe16}, \href{../works/WangMD15.pdf}{WangMD15}~\cite{WangMD15}, \href{../works/Wolf11.pdf}{Wolf11}~\cite{Wolf11}, \href{../works/TopalogluO11.pdf}{TopalogluO11}~\cite{TopalogluO11} & \href{../works/GuoZ23.pdf}{GuoZ23}~\cite{GuoZ23}, \href{../works/ZarandiASC20.pdf}{ZarandiASC20}~\cite{ZarandiASC20}, \href{../works/HechingH16.pdf}{HechingH16}~\cite{HechingH16}, \href{../works/DejemeppeD14.pdf}{DejemeppeD14}~\cite{DejemeppeD14}, \href{../works/RendlPHPR12.pdf}{RendlPHPR12}~\cite{RendlPHPR12} & \href{../works/ShaikhK23.pdf}{ShaikhK23}~\cite{ShaikhK23}, \href{../works/AbreuPNF23.pdf}{AbreuPNF23}~\cite{AbreuPNF23}, \href{../works/IsikYA23.pdf}{IsikYA23}~\cite{IsikYA23}, \href{../works/AbreuNP23.pdf}{AbreuNP23}~\cite{AbreuNP23}, \href{../works/AkramNHRSA23.pdf}{AkramNHRSA23}~\cite{AkramNHRSA23}, \href{../works/YunusogluY22.pdf}{YunusogluY22}~\cite{YunusogluY22}, \href{../works/FarsiTM22.pdf}{FarsiTM22}~\cite{FarsiTM22}, \href{../works/AbreuN22.pdf}{AbreuN22}~\cite{AbreuN22}, \href{../works/GeibingerKKMMW21.pdf}{GeibingerKKMMW21}~\cite{GeibingerKKMMW21}, \href{../works/Bedhief21.pdf}{Bedhief21}~\cite{Bedhief21}, \href{../works/Lemos21.pdf}{Lemos21}~\cite{Lemos21}, \href{../works/AbreuAPNM21.pdf}{AbreuAPNM21}~\cite{AbreuAPNM21}, \href{../works/ThomasKS20.pdf}{ThomasKS20}~\cite{ThomasKS20}, \href{../works/FallahiAC20.pdf}{FallahiAC20}~\cite{FallahiAC20}, \href{../works/FrimodigS19.pdf}{FrimodigS19}~\cite{FrimodigS19}, \href{../works/abs-1902-01193.pdf}{abs-1902-01193}~\cite{abs-1902-01193}, \href{../works/Novas19.pdf}{Novas19}~\cite{Novas19}, \href{../works/GurEA19.pdf}{GurEA19}~\cite{GurEA19}, \href{../works/YounespourAKE19.pdf}{YounespourAKE19}~\cite{YounespourAKE19}, \href{../works/CappartTSR18.pdf}{CappartTSR18}~\cite{CappartTSR18}, \href{../works/HoYCLLCLC18.pdf}{HoYCLLCLC18}~\cite{HoYCLLCLC18}, \href{../works/TanT18.pdf}{TanT18}~\cite{TanT18}, \href{../works/GedikKEK18.pdf}{GedikKEK18}~\cite{GedikKEK18}, \href{../works/TranVNB17a.pdf}{TranVNB17a}~\cite{TranVNB17a}, \href{../works/RoshanaeiLAU17.pdf}{RoshanaeiLAU17}~\cite{RoshanaeiLAU17}, \href{../works/TranVNB17.pdf}{TranVNB17}~\cite{TranVNB17}, \href{../works/DoulabiRP16.pdf}{DoulabiRP16}~\cite{DoulabiRP16}, \href{../works/BridiBLMB16.pdf}{BridiBLMB16}~\cite{BridiBLMB16}, \href{../works/BoothNB16.pdf}{BoothNB16}~\cite{BoothNB16}... (Total: 36)\\
ApplicationAreas & meeting scheduling & \href{../works/GelainPRVW17.pdf}{GelainPRVW17}~\cite{GelainPRVW17}, \href{../works/LimHTB16.pdf}{LimHTB16}~\cite{LimHTB16}, \href{../works/LimBTBB15.pdf}{LimBTBB15}~\cite{LimBTBB15}, \href{../works/PesantRR15.pdf}{PesantRR15}~\cite{PesantRR15}, \href{../works/ZhuS02.pdf}{ZhuS02}~\cite{ZhuS02} & \href{../works/BofillEGPSV14.pdf}{BofillEGPSV14}~\cite{BofillEGPSV14} & \href{../works/Lemos21.pdf}{Lemos21}~\cite{Lemos21}, \href{../works/BofillGSV15.pdf}{BofillGSV15}~\cite{BofillGSV15}, \href{../works/MurphyMB15.pdf}{MurphyMB15}~\cite{MurphyMB15}, \href{../works/BartakSR10.pdf}{BartakSR10}~\cite{BartakSR10}, \href{../works/MoffittPP05.pdf}{MoffittPP05}~\cite{MoffittPP05}\\
ApplicationAreas & music festival & \href{../works/CohenHB17.pdf}{CohenHB17}~\cite{CohenHB17} &  & \\
ApplicationAreas & nurse & \href{../works/GurPAE23.pdf}{GurPAE23}~\cite{GurPAE23}, \href{../works/FarsiTM22.pdf}{FarsiTM22}~\cite{FarsiTM22}, \href{../works/ZarandiASC20.pdf}{ZarandiASC20}~\cite{ZarandiASC20}, \href{../works/abs-1902-01193.pdf}{abs-1902-01193}~\cite{abs-1902-01193}, \href{../works/ShinBBHO18.pdf}{ShinBBHO18}~\cite{ShinBBHO18}, \href{../works/HoYCLLCLC18.pdf}{HoYCLLCLC18}~\cite{HoYCLLCLC18}, \href{../works/LuoVLBM16.pdf}{LuoVLBM16}~\cite{LuoVLBM16}, \href{../works/WangMD15.pdf}{WangMD15}~\cite{WangMD15}, \href{../works/RendlPHPR12.pdf}{RendlPHPR12}~\cite{RendlPHPR12}, \href{../works/Menana11.pdf}{Menana11}~\cite{Menana11}, \href{../works/Wolf11.pdf}{Wolf11}~\cite{Wolf11}, \href{../works/Simonis07.pdf}{Simonis07}~\cite{Simonis07}, \href{../works/Mason01.pdf}{Mason01}~\cite{Mason01} & \href{../works/OuelletQ22.pdf}{OuelletQ22}~\cite{OuelletQ22}, \href{../works/GeibingerKKMMW21.pdf}{GeibingerKKMMW21}~\cite{GeibingerKKMMW21}, \href{../works/GeibingerMM21.pdf}{GeibingerMM21}~\cite{GeibingerMM21}, \href{../works/YounespourAKE19.pdf}{YounespourAKE19}~\cite{YounespourAKE19}, \href{../works/FrohnerTR19.pdf}{FrohnerTR19}~\cite{FrohnerTR19}, \href{../works/RoshanaeiLAU17.pdf}{RoshanaeiLAU17}~\cite{RoshanaeiLAU17} & \href{../works/abs-2312-13682.pdf}{abs-2312-13682}~\cite{abs-2312-13682}, \href{../works/PerezGSL23.pdf}{PerezGSL23}~\cite{PerezGSL23}, \href{../works/NaderiBZ22a.pdf}{NaderiBZ22a}~\cite{NaderiBZ22a}, \href{../works/NaderiBZ22.pdf}{NaderiBZ22}~\cite{NaderiBZ22}, \href{../works/BourreauGGLT22.pdf}{BourreauGGLT22}~\cite{BourreauGGLT22}, \href{../works/FallahiAC20.pdf}{FallahiAC20}~\cite{FallahiAC20}, \href{../works/RoshanaeiBAUB20.pdf}{RoshanaeiBAUB20}~\cite{RoshanaeiBAUB20}, \href{../works/FrimodigS19.pdf}{FrimodigS19}~\cite{FrimodigS19}, \href{../works/German18.pdf}{German18}~\cite{German18}, \href{../works/GedikKEK18.pdf}{GedikKEK18}~\cite{GedikKEK18}, \href{../works/NishikawaSTT18a.pdf}{NishikawaSTT18a}~\cite{NishikawaSTT18a}, \href{../works/MusliuSS18.pdf}{MusliuSS18}~\cite{MusliuSS18}, \href{../works/HookerH17.pdf}{HookerH17}~\cite{HookerH17}, \href{../works/Dejemeppe16.pdf}{Dejemeppe16}~\cite{Dejemeppe16}, \href{../works/DoulabiRP16.pdf}{DoulabiRP16}~\cite{DoulabiRP16}, \href{../works/DoulabiRP14.pdf}{DoulabiRP14}~\cite{DoulabiRP14}, \href{../works/TopalogluO11.pdf}{TopalogluO11}~\cite{TopalogluO11}, \href{../works/Simonis99.pdf}{Simonis99}~\cite{Simonis99}\\
ApplicationAreas & offshore &  & \href{../works/SubulanC22.pdf}{SubulanC22}~\cite{SubulanC22}, \href{../works/Froger16.pdf}{Froger16}~\cite{Froger16} & \href{../works/GokPTGO23.pdf}{GokPTGO23}~\cite{GokPTGO23}, \href{../works/BoudreaultSLQ22.pdf}{BoudreaultSLQ22}~\cite{BoudreaultSLQ22}, \href{../works/BlomPS16.pdf}{BlomPS16}~\cite{BlomPS16}, \href{../works/BlomBPS14.pdf}{BlomBPS14}~\cite{BlomBPS14}, \href{../works/Jans09.pdf}{Jans09}~\cite{Jans09}\\
ApplicationAreas & operating room & \href{../works/NaderiRR23.pdf}{NaderiRR23}~\cite{NaderiRR23}, \href{../works/GurPAE23.pdf}{GurPAE23}~\cite{GurPAE23}, \href{../works/FarsiTM22.pdf}{FarsiTM22}~\cite{FarsiTM22}, \href{../works/NaderiBZ22.pdf}{NaderiBZ22}~\cite{NaderiBZ22}, \href{../works/RoshanaeiBAUB20.pdf}{RoshanaeiBAUB20}~\cite{RoshanaeiBAUB20}, \href{../works/YounespourAKE19.pdf}{YounespourAKE19}~\cite{YounespourAKE19}, \href{../works/GurEA19.pdf}{GurEA19}~\cite{GurEA19}, \href{../works/RoshanaeiLAU17.pdf}{RoshanaeiLAU17}~\cite{RoshanaeiLAU17}, \href{../works/DoulabiRP16.pdf}{DoulabiRP16}~\cite{DoulabiRP16}, \href{../works/WangMD15.pdf}{WangMD15}~\cite{WangMD15}, \href{../works/DoulabiRP14.pdf}{DoulabiRP14}~\cite{DoulabiRP14}, \href{../works/Wolf11.pdf}{Wolf11}~\cite{Wolf11} & \href{../works/GuoZ23.pdf}{GuoZ23}~\cite{GuoZ23}, \href{../works/NaderiBZ22a.pdf}{NaderiBZ22a}~\cite{NaderiBZ22a}, \href{../works/ElciOH22.pdf}{ElciOH22}~\cite{ElciOH22}, \href{../works/ZarandiASC20.pdf}{ZarandiASC20}~\cite{ZarandiASC20}, \href{../works/Hooker19.pdf}{Hooker19}~\cite{Hooker19}, \href{../works/HookerH17.pdf}{HookerH17}~\cite{HookerH17} & \href{../works/ForbesHJST24.pdf}{ForbesHJST24}~\cite{ForbesHJST24}, \href{../works/WangB23.pdf}{WangB23}~\cite{WangB23}, \href{../works/PerezGSL23.pdf}{PerezGSL23}~\cite{PerezGSL23}, \href{../works/abs-2312-13682.pdf}{abs-2312-13682}~\cite{abs-2312-13682}, \href{../works/JuvinHL23a.pdf}{JuvinHL23a}~\cite{JuvinHL23a}, \href{../works/Adelgren2023.pdf}{Adelgren2023}~\cite{Adelgren2023}, \href{../works/GeibingerMM21.pdf}{GeibingerMM21}~\cite{GeibingerMM21}, \href{../works/TanT18.pdf}{TanT18}~\cite{TanT18}, \href{../works/MusliuSS18.pdf}{MusliuSS18}~\cite{MusliuSS18}, \href{../works/Wolf09.pdf}{Wolf09}~\cite{Wolf09}\\
ApplicationAreas & oven scheduling & \href{../works/LacknerMMWW23.pdf}{LacknerMMWW23}~\cite{LacknerMMWW23}, \href{../works/LacknerMMWW21.pdf}{LacknerMMWW21}~\cite{LacknerMMWW21} &  & \href{../works/ColT22.pdf}{ColT22}~\cite{ColT22}\\
ApplicationAreas & patient & \href{../works/GurPAE23.pdf}{GurPAE23}~\cite{GurPAE23}, \href{../works/FarsiTM22.pdf}{FarsiTM22}~\cite{FarsiTM22}, \href{../works/RoshanaeiBAUB20.pdf}{RoshanaeiBAUB20}~\cite{RoshanaeiBAUB20}, \href{../works/ThomasKS20.pdf}{ThomasKS20}~\cite{ThomasKS20}, \href{../works/FrimodigS19.pdf}{FrimodigS19}~\cite{FrimodigS19}, \href{../works/GurEA19.pdf}{GurEA19}~\cite{GurEA19}, \href{../works/YounespourAKE19.pdf}{YounespourAKE19}~\cite{YounespourAKE19}, \href{../works/ShinBBHO18.pdf}{ShinBBHO18}~\cite{ShinBBHO18}, \href{../works/CappartTSR18.pdf}{CappartTSR18}~\cite{CappartTSR18}, \href{../works/RoshanaeiLAU17.pdf}{RoshanaeiLAU17}~\cite{RoshanaeiLAU17}, \href{../works/HechingH16.pdf}{HechingH16}~\cite{HechingH16}, \href{../works/Dejemeppe16.pdf}{Dejemeppe16}~\cite{Dejemeppe16}, \href{../works/DoulabiRP16.pdf}{DoulabiRP16}~\cite{DoulabiRP16}, \href{../works/WangMD15.pdf}{WangMD15}~\cite{WangMD15}, \href{../works/DejemeppeD14.pdf}{DejemeppeD14}~\cite{DejemeppeD14}, \href{../works/RendlPHPR12.pdf}{RendlPHPR12}~\cite{RendlPHPR12}, \href{../works/Wolf11.pdf}{Wolf11}~\cite{Wolf11}, \href{../works/TopalogluO11.pdf}{TopalogluO11}~\cite{TopalogluO11} & \href{../works/GeibingerKKMMW21.pdf}{GeibingerKKMMW21}~\cite{GeibingerKKMMW21} & \href{../works/BonninMNE24.pdf}{BonninMNE24}~\cite{BonninMNE24}, \href{../works/ForbesHJST24.pdf}{ForbesHJST24}~\cite{ForbesHJST24}, \href{../works/GuoZ23.pdf}{GuoZ23}~\cite{GuoZ23}, \href{../works/AlfieriGPS23.pdf}{AlfieriGPS23}~\cite{AlfieriGPS23}, \href{../works/NaderiBZ22.pdf}{NaderiBZ22}~\cite{NaderiBZ22}, \href{../works/ElciOH22.pdf}{ElciOH22}~\cite{ElciOH22}, \href{../works/AbreuAPNM21.pdf}{AbreuAPNM21}~\cite{AbreuAPNM21}, \href{../works/CauwelaertDS20.pdf}{CauwelaertDS20}~\cite{CauwelaertDS20}, \href{../works/MurinR19.pdf}{MurinR19}~\cite{MurinR19}, \href{../works/Hooker19.pdf}{Hooker19}~\cite{Hooker19}, \href{../works/HoYCLLCLC18.pdf}{HoYCLLCLC18}~\cite{HoYCLLCLC18}, \href{../works/TanT18.pdf}{TanT18}~\cite{TanT18}, \href{../works/GombolayWS18.pdf}{GombolayWS18}~\cite{GombolayWS18}, \href{../works/LouieVNB14.pdf}{LouieVNB14}~\cite{LouieVNB14}, \href{../works/DoulabiRP14.pdf}{DoulabiRP14}~\cite{DoulabiRP14}, \href{../works/Clercq12.pdf}{Clercq12}~\cite{Clercq12}, \href{../works/Malapert11.pdf}{Malapert11}~\cite{Malapert11}, \href{../works/Wolf09.pdf}{Wolf09}~\cite{Wolf09}, \href{../works/Simonis07.pdf}{Simonis07}~\cite{Simonis07}, \href{../works/KanetAG04.pdf}{KanetAG04}~\cite{KanetAG04}\\
ApplicationAreas & perfect-square & \href{../works/BeldiceanuCDP11.pdf}{BeldiceanuCDP11}~\cite{BeldiceanuCDP11}, \href{../works/BeldiceanuCP08.pdf}{BeldiceanuCP08}~\cite{BeldiceanuCP08}, \href{../works/AggounB93.pdf}{AggounB93}~\cite{AggounB93} &  & \\
ApplicationAreas & physician & \href{../works/GeibingerKKMMW21.pdf}{GeibingerKKMMW21}~\cite{GeibingerKKMMW21}, \href{../works/ShinBBHO18.pdf}{ShinBBHO18}~\cite{ShinBBHO18} & \href{../works/Dejemeppe16.pdf}{Dejemeppe16}~\cite{Dejemeppe16} & \href{../works/GurPAE23.pdf}{GurPAE23}~\cite{GurPAE23}, \href{../works/GuoZ23.pdf}{GuoZ23}~\cite{GuoZ23}, \href{../works/FarsiTM22.pdf}{FarsiTM22}~\cite{FarsiTM22}, \href{../works/FrimodigS19.pdf}{FrimodigS19}~\cite{FrimodigS19}, \href{../works/HookerH17.pdf}{HookerH17}~\cite{HookerH17}, \href{../works/WangMD15.pdf}{WangMD15}~\cite{WangMD15}, \href{../works/Wolf11.pdf}{Wolf11}~\cite{Wolf11}, \href{../works/TopalogluO11.pdf}{TopalogluO11}~\cite{TopalogluO11}\\
ApplicationAreas & pipeline & \href{../works/HarjunkoskiMBC14.pdf}{HarjunkoskiMBC14}~\cite{HarjunkoskiMBC14}, \href{../works/BegB13.pdf}{BegB13}~\cite{BegB13}, \href{../works/LopesCSM10.pdf}{LopesCSM10}~\cite{LopesCSM10}, \href{../works/Lombardi10.pdf}{Lombardi10}~\cite{Lombardi10}, \href{../works/RuggieroBBMA09.pdf}{RuggieroBBMA09}~\cite{RuggieroBBMA09}, \href{../works/MouraSCL08a.pdf}{MouraSCL08a}~\cite{MouraSCL08a}, \href{../works/Malik08.pdf}{Malik08}~\cite{Malik08}, \href{../works/MouraSCL08.pdf}{MouraSCL08}~\cite{MouraSCL08}, \href{../works/BeniniLMR08.pdf}{BeniniLMR08}~\cite{BeniniLMR08}, \href{../works/ErtlK91.pdf}{ErtlK91}~\cite{ErtlK91} & \href{../works/ZouZ20.pdf}{ZouZ20}~\cite{ZouZ20}, \href{../works/TangLWSK18.pdf}{TangLWSK18}~\cite{TangLWSK18}, \href{../works/LombardiMRB10.pdf}{LombardiMRB10}~\cite{LombardiMRB10}, \href{../works/MalikMB08.pdf}{MalikMB08}~\cite{MalikMB08}, \href{../works/BeniniBGM06.pdf}{BeniniBGM06}~\cite{BeniniBGM06}, \href{../works/WolinskiKG04.pdf}{WolinskiKG04}~\cite{WolinskiKG04}, \href{../works/BeldiceanuC94.pdf}{BeldiceanuC94}~\cite{BeldiceanuC94} & \href{../works/EfthymiouY23.pdf}{EfthymiouY23}~\cite{EfthymiouY23}, \href{../works/Adelgren2023.pdf}{Adelgren2023}~\cite{Adelgren2023}, \href{../works/PopovicCGNC22.pdf}{PopovicCGNC22}~\cite{PopovicCGNC22}, \href{../works/EmdeZD22.pdf}{EmdeZD22}~\cite{EmdeZD22}, \href{../works/HanenKP21.pdf}{HanenKP21}~\cite{HanenKP21}, \href{../works/NishikawaSTT19.pdf}{NishikawaSTT19}~\cite{NishikawaSTT19}, \href{../works/NishikawaSTT18a.pdf}{NishikawaSTT18a}~\cite{NishikawaSTT18a}, \href{../works/LaborieRSV18.pdf}{LaborieRSV18}~\cite{LaborieRSV18}, \href{../works/NishikawaSTT18.pdf}{NishikawaSTT18}~\cite{NishikawaSTT18}, \href{../works/BlomPS16.pdf}{BlomPS16}~\cite{BlomPS16}, \href{../works/Bonfietti16.pdf}{Bonfietti16}~\cite{Bonfietti16}, \href{../works/GilesH16.pdf}{GilesH16}~\cite{GilesH16}, \href{../works/GoelSHFS15.pdf}{GoelSHFS15}~\cite{GoelSHFS15}, \href{../works/SimoninAHL15.pdf}{SimoninAHL15}~\cite{SimoninAHL15}, \href{../works/BonfiettiLBM14.pdf}{BonfiettiLBM14}~\cite{BonfiettiLBM14}, \href{../works/LombardiMB13.pdf}{LombardiMB13}~\cite{LombardiMB13}, \href{../works/BeniniLMR11.pdf}{BeniniLMR11}~\cite{BeniniLMR11}, \href{../works/NovasH10.pdf}{NovasH10}~\cite{NovasH10}, \href{../works/BarlattCG08.pdf}{BarlattCG08}~\cite{BarlattCG08}, \href{../works/KuchcinskiW03.pdf}{KuchcinskiW03}~\cite{KuchcinskiW03}, \href{../works/Wolf03.pdf}{Wolf03}~\cite{Wolf03}, \href{../works/Simonis99.pdf}{Simonis99}~\cite{Simonis99}, \href{../works/GruianK98.pdf}{GruianK98}~\cite{GruianK98}, \href{../works/Darby-DowmanLMZ97.pdf}{Darby-DowmanLMZ97}~\cite{Darby-DowmanLMZ97}, \href{../works/SimonisC95.pdf}{SimonisC95}~\cite{SimonisC95}, \href{../works/Simonis95a.pdf}{Simonis95a}~\cite{Simonis95a}\\
ApplicationAreas & radiation therapy & \href{../works/FrimodigS19.pdf}{FrimodigS19}~\cite{FrimodigS19} &  & \href{../works/HookerH17.pdf}{HookerH17}~\cite{HookerH17}\\
ApplicationAreas & railway & \href{../works/MarliereSPR23.pdf}{MarliereSPR23}~\cite{MarliereSPR23}, \href{../works/SvancaraB22.pdf}{SvancaraB22}~\cite{SvancaraB22}, \href{../works/Lemos21.pdf}{Lemos21}~\cite{Lemos21}, \href{../works/PourDERB18.pdf}{PourDERB18}~\cite{PourDERB18}, \href{../works/CappartS17.pdf}{CappartS17}~\cite{CappartS17}, \href{../works/Acuna-AgostMFG09.pdf}{Acuna-AgostMFG09}~\cite{Acuna-AgostMFG09}, \href{../works/AronssonBK09.pdf}{AronssonBK09}~\cite{AronssonBK09}, \href{../works/RodriguezS09.pdf}{RodriguezS09}~\cite{RodriguezS09}, \href{../works/Rodriguez07.pdf}{Rodriguez07}~\cite{Rodriguez07}, \href{../works/Rodriguez07b.pdf}{Rodriguez07b}~\cite{Rodriguez07b}, \href{../works/Geske05.pdf}{Geske05}~\cite{Geske05}, \href{../works/RodriguezDG02.pdf}{RodriguezDG02}~\cite{RodriguezDG02}, \href{../works/MartinPY01.pdf}{MartinPY01}~\cite{MartinPY01}, \href{../works/LammaMM97.pdf}{LammaMM97}~\cite{LammaMM97} & \href{../works/ZarandiASC20.pdf}{ZarandiASC20}~\cite{ZarandiASC20}, \href{../works/LaborieRSV18.pdf}{LaborieRSV18}~\cite{LaborieRSV18}, \href{../works/TangLWSK18.pdf}{TangLWSK18}~\cite{TangLWSK18}, \href{../works/Mason01.pdf}{Mason01}~\cite{Mason01}, \href{../works/BrusoniCLMMT96.pdf}{BrusoniCLMMT96}~\cite{BrusoniCLMMT96} & \href{../works/GuoZ23.pdf}{GuoZ23}~\cite{GuoZ23}, \href{../works/LuoB22.pdf}{LuoB22}~\cite{LuoB22}, \href{../works/Godet21a.pdf}{Godet21a}~\cite{Godet21a}, \href{../works/BogaerdtW19.pdf}{BogaerdtW19}~\cite{BogaerdtW19}, \href{../works/Hooker19.pdf}{Hooker19}~\cite{Hooker19}, \href{../works/BajestaniB15.pdf}{BajestaniB15}~\cite{BajestaniB15}, \href{../works/ZhouGL15.pdf}{ZhouGL15}~\cite{ZhouGL15}, \href{../works/BajestaniB13.pdf}{BajestaniB13}~\cite{BajestaniB13}, \href{../works/BajestaniB11.pdf}{BajestaniB11}~\cite{BajestaniB11}, \href{../works/WuBB09.pdf}{WuBB09}~\cite{WuBB09}, \href{../works/AbrilSB05.pdf}{AbrilSB05}~\cite{AbrilSB05}, \href{../works/Wallace96.pdf}{Wallace96}~\cite{Wallace96}\\
ApplicationAreas & real-time pricing &  & \href{../works/He0GLW18.pdf}{He0GLW18}~\cite{He0GLW18}, \href{../works/GrimesIOS14.pdf}{GrimesIOS14}~\cite{GrimesIOS14} & \href{../works/LimHTB16.pdf}{LimHTB16}~\cite{LimHTB16}\\
ApplicationAreas & rectangle-packing & \href{../works/YangSS19.pdf}{YangSS19}~\cite{YangSS19}, \href{../works/AggounB93.pdf}{AggounB93}~\cite{AggounB93} & \href{../works/LuoB22.pdf}{LuoB22}~\cite{LuoB22}, \href{../works/Malapert11.pdf}{Malapert11}~\cite{Malapert11} & \href{../works/MossigeGSMC17.pdf}{MossigeGSMC17}~\cite{MossigeGSMC17}, \href{../works/DoulabiRP16.pdf}{DoulabiRP16}~\cite{DoulabiRP16}, \href{../works/Siala15.pdf}{Siala15}~\cite{Siala15}, \href{../works/VilimLS15.pdf}{VilimLS15}~\cite{VilimLS15}, \href{../works/Siala15a.pdf}{Siala15a}~\cite{Siala15a}, \href{../works/BeldiceanuCDP11.pdf}{BeldiceanuCDP11}~\cite{BeldiceanuCDP11}, \href{../works/Schutt11.pdf}{Schutt11}~\cite{Schutt11}, \href{../works/SchuttW10.pdf}{SchuttW10}~\cite{SchuttW10}, \href{../works/BeldiceanuCP08.pdf}{BeldiceanuCP08}~\cite{BeldiceanuCP08}\\
ApplicationAreas & robot & \href{../works/Fatemi-AnarakiTFV23.pdf}{Fatemi-AnarakiTFV23}~\cite{Fatemi-AnarakiTFV23}, \href{../works/IsikYA23.pdf}{IsikYA23}~\cite{IsikYA23}, \href{../works/LiFJZLL22.pdf}{LiFJZLL22}~\cite{LiFJZLL22}, \href{../works/ArmstrongGOS21.pdf}{ArmstrongGOS21}~\cite{ArmstrongGOS21}, \href{../works/Astrand21.pdf}{Astrand21}~\cite{Astrand21}, \href{../works/KoehlerBFFHPSSS21.pdf}{KoehlerBFFHPSSS21}~\cite{KoehlerBFFHPSSS21}, \href{../works/ZarandiASC20.pdf}{ZarandiASC20}~\cite{ZarandiASC20}, \href{../works/MokhtarzadehTNF20.pdf}{MokhtarzadehTNF20}~\cite{MokhtarzadehTNF20}, \href{../works/Lunardi20.pdf}{Lunardi20}~\cite{Lunardi20}, \href{../works/WessenCS20.pdf}{WessenCS20}~\cite{WessenCS20}, \href{../works/MurinR19.pdf}{MurinR19}~\cite{MurinR19}, \href{../works/abs-1901-07914.pdf}{abs-1901-07914}~\cite{abs-1901-07914}, \href{../works/BehrensLM19.pdf}{BehrensLM19}~\cite{BehrensLM19}, \href{../works/GombolayWS18.pdf}{GombolayWS18}~\cite{GombolayWS18}, \href{../works/LaborieRSV18.pdf}{LaborieRSV18}~\cite{LaborieRSV18}, \href{../works/MossigeGSMC17.pdf}{MossigeGSMC17}~\cite{MossigeGSMC17}, \href{../works/TranVNB17.pdf}{TranVNB17}~\cite{TranVNB17}, \href{../works/TranVNB17a.pdf}{TranVNB17a}~\cite{TranVNB17a}, \href{../works/BoothNB16.pdf}{BoothNB16}~\cite{BoothNB16}, \href{../works/LouieVNB14.pdf}{LouieVNB14}~\cite{LouieVNB14}, \href{../works/NovasH14.pdf}{NovasH14}~\cite{NovasH14}, \href{../works/NovasH12.pdf}{NovasH12}~\cite{NovasH12}, \href{../works/BartakSR10.pdf}{BartakSR10}~\cite{BartakSR10}, \href{../works/BidotVLB09.pdf}{BidotVLB09}~\cite{BidotVLB09}, \href{../works/ValleMGT03.pdf}{ValleMGT03}~\cite{ValleMGT03}, \href{../works/BeckF98.pdf}{BeckF98}~\cite{BeckF98} & \href{../works/PrataAN23.pdf}{PrataAN23}~\cite{PrataAN23}, \href{../works/CzerniachowskaWZ23.pdf}{CzerniachowskaWZ23}~\cite{CzerniachowskaWZ23}, \href{../works/ZhuSZW23.pdf}{ZhuSZW23}~\cite{ZhuSZW23}, \href{../works/Mehdizadeh-Somarin23.pdf}{Mehdizadeh-Somarin23}~\cite{Mehdizadeh-Somarin23}, \href{../works/TouatBT22.pdf}{TouatBT22}~\cite{TouatBT22}, \href{../works/YunusogluY22.pdf}{YunusogluY22}~\cite{YunusogluY22}, \href{../works/NaderiBZ22a.pdf}{NaderiBZ22a}~\cite{NaderiBZ22a}, \href{../works/OujanaAYB22.pdf}{OujanaAYB22}~\cite{OujanaAYB22}, \href{../works/Astrand0F21.pdf}{Astrand0F21}~\cite{Astrand0F21}, \href{../works/WallaceY20.pdf}{WallaceY20}~\cite{WallaceY20}, \href{../works/WikarekS19.pdf}{WikarekS19}~\cite{WikarekS19}, \href{../works/NishikawaSTT19.pdf}{NishikawaSTT19}~\cite{NishikawaSTT19}, \href{../works/NishikawaSTT18a.pdf}{NishikawaSTT18a}~\cite{NishikawaSTT18a}, \href{../works/NishikawaSTT18.pdf}{NishikawaSTT18}~\cite{NishikawaSTT18}, \href{../works/Dejemeppe16.pdf}{Dejemeppe16}~\cite{Dejemeppe16}, \href{../works/VanczaM01.pdf}{VanczaM01}~\cite{VanczaM01}, \href{../works/BeckF00.pdf}{BeckF00}~\cite{BeckF00}, \href{../works/Beck99.pdf}{Beck99}~\cite{Beck99} & \href{../works/abs-2305-19888.pdf}{abs-2305-19888}~\cite{abs-2305-19888}, \href{../works/AbreuPNF23.pdf}{AbreuPNF23}~\cite{AbreuPNF23}, \href{../works/MontemanniD23.pdf}{MontemanniD23}~\cite{MontemanniD23}, \href{../works/MarliereSPR23.pdf}{MarliereSPR23}~\cite{MarliereSPR23}, \href{../works/HeinzNVH22.pdf}{HeinzNVH22}~\cite{HeinzNVH22}, \href{../works/GeitzGSSW22.pdf}{GeitzGSSW22}~\cite{GeitzGSSW22}, \href{../works/FarsiTM22.pdf}{FarsiTM22}~\cite{FarsiTM22}, \href{../works/MullerMKP22.pdf}{MullerMKP22}~\cite{MullerMKP22}, \href{../works/ColT22.pdf}{ColT22}~\cite{ColT22}, \href{../works/YuraszeckMPV22.pdf}{YuraszeckMPV22}~\cite{YuraszeckMPV22}, \href{../works/HamPK21.pdf}{HamPK21}~\cite{HamPK21}, \href{../works/ZhangYW21.pdf}{ZhangYW21}~\cite{ZhangYW21}, \href{../works/Godet21a.pdf}{Godet21a}~\cite{Godet21a}, \href{../works/Bedhief21.pdf}{Bedhief21}~\cite{Bedhief21}, \href{../works/Groleaz21.pdf}{Groleaz21}~\cite{Groleaz21}, \href{../works/VlkHT21.pdf}{VlkHT21}~\cite{VlkHT21}, \href{../works/FallahiAC20.pdf}{FallahiAC20}~\cite{FallahiAC20}, \href{../works/MengZRZL20.pdf}{MengZRZL20}~\cite{MengZRZL20}, \href{../works/BenediktMH20.pdf}{BenediktMH20}~\cite{BenediktMH20}, \href{../works/MejiaY20.pdf}{MejiaY20}~\cite{MejiaY20}, \href{../works/AstrandJZ20.pdf}{AstrandJZ20}~\cite{AstrandJZ20}, \href{../works/BarzegaranZP20.pdf}{BarzegaranZP20}~\cite{BarzegaranZP20}, \href{../works/Novas19.pdf}{Novas19}~\cite{Novas19}, \href{../works/ZhangW18.pdf}{ZhangW18}~\cite{ZhangW18}, \href{../works/GokgurHO18.pdf}{GokgurHO18}~\cite{GokgurHO18}, \href{../works/Ham18a.pdf}{Ham18a}~\cite{Ham18a}, \href{../works/Ham18.pdf}{Ham18}~\cite{Ham18}, \href{../works/TanT18.pdf}{TanT18}~\cite{TanT18}, \href{../works/AstrandJZ18.pdf}{AstrandJZ18}~\cite{AstrandJZ18}... (Total: 63)\\
ApplicationAreas & satellite & \href{../works/SquillaciPR23.pdf}{SquillaciPR23}~\cite{SquillaciPR23}, \href{../works/Godet21a.pdf}{Godet21a}~\cite{Godet21a}, \href{../works/GodetLHS20.pdf}{GodetLHS20}~\cite{GodetLHS20}, \href{../works/KucukY19.pdf}{KucukY19}~\cite{KucukY19}, \href{../works/LaborieRSV18.pdf}{LaborieRSV18}~\cite{LaborieRSV18}, \href{../works/HebrardHJMPV16.pdf}{HebrardHJMPV16}~\cite{HebrardHJMPV16}, \href{../works/PraletLJ15.pdf}{PraletLJ15}~\cite{PraletLJ15}, \href{../works/KelarevaTK13.pdf}{KelarevaTK13}~\cite{KelarevaTK13}, \href{../works/VerfaillieL01.pdf}{VerfaillieL01}~\cite{VerfaillieL01}, \href{../works/BensanaLV99.pdf}{BensanaLV99}~\cite{BensanaLV99}, \href{../works/PembertonG98.pdf}{PembertonG98}~\cite{PembertonG98} & \href{../works/Laborie09.pdf}{Laborie09}~\cite{Laborie09}, \href{../works/FrankK05.pdf}{FrankK05}~\cite{FrankK05} & \href{../works/EfthymiouY23.pdf}{EfthymiouY23}~\cite{EfthymiouY23}, \href{../works/TouatBT22.pdf}{TouatBT22}~\cite{TouatBT22}, \href{../works/Astrand21.pdf}{Astrand21}~\cite{Astrand21}, \href{../works/Astrand0F21.pdf}{Astrand0F21}~\cite{Astrand0F21}, \href{../works/Zahout21.pdf}{Zahout21}~\cite{Zahout21}, \href{../works/ZarandiASC20.pdf}{ZarandiASC20}~\cite{ZarandiASC20}, \href{../works/Hooker19.pdf}{Hooker19}~\cite{Hooker19}, \href{../works/TranVNB17.pdf}{TranVNB17}~\cite{TranVNB17}, \href{../works/Pralet17.pdf}{Pralet17}~\cite{Pralet17}, \href{../works/TranWDRFOVB16.pdf}{TranWDRFOVB16}~\cite{TranWDRFOVB16}, \href{../works/Froger16.pdf}{Froger16}~\cite{Froger16}, \href{../works/SimoninAHL15.pdf}{SimoninAHL15}~\cite{SimoninAHL15}, \href{../works/BessiereHMQW14.pdf}{BessiereHMQW14}~\cite{BessiereHMQW14}, \href{../works/HeinzSB13.pdf}{HeinzSB13}~\cite{HeinzSB13}, \href{../works/GuyonLPR12.pdf}{GuyonLPR12}~\cite{GuyonLPR12}, \href{../works/SimoninAHL12.pdf}{SimoninAHL12}~\cite{SimoninAHL12}, \href{../works/RuggieroBBMA09.pdf}{RuggieroBBMA09}~\cite{RuggieroBBMA09}, \href{../works/Rodriguez07.pdf}{Rodriguez07}~\cite{Rodriguez07}, \href{../works/OddiPCC03.pdf}{OddiPCC03}~\cite{OddiPCC03}, \href{../works/NuijtenP98.pdf}{NuijtenP98}~\cite{NuijtenP98}\\
ApplicationAreas & semiconductor & \href{../works/ZarandiASC20.pdf}{ZarandiASC20}~\cite{ZarandiASC20}, \href{../works/MalapertN19.pdf}{MalapertN19}~\cite{MalapertN19}, \href{../works/NattafDYW19.pdf}{NattafDYW19}~\cite{NattafDYW19}, \href{../works/Ham18a.pdf}{Ham18a}~\cite{Ham18a}, \href{../works/BajestaniB15.pdf}{BajestaniB15}~\cite{BajestaniB15}, \href{../works/NovasH12.pdf}{NovasH12}~\cite{NovasH12} & \href{../works/PenzDN23.pdf}{PenzDN23}~\cite{PenzDN23}, \href{../works/QinWSLS21.pdf}{QinWSLS21}~\cite{QinWSLS21}, \href{../works/GokgurHO18.pdf}{GokgurHO18}~\cite{GokgurHO18}, \href{../works/HamC16.pdf}{HamC16}~\cite{HamC16}, \href{../works/LombardiMRB10.pdf}{LombardiMRB10}~\cite{LombardiMRB10}, \href{../works/Davenport10.pdf}{Davenport10}~\cite{Davenport10}, \href{../works/KrogtLPHJ07.pdf}{KrogtLPHJ07}~\cite{KrogtLPHJ07}, \href{../works/JainM99.pdf}{JainM99}~\cite{JainM99} & \href{../works/LacknerMMWW23.pdf}{LacknerMMWW23}~\cite{LacknerMMWW23}, \href{../works/Fatemi-AnarakiTFV23.pdf}{Fatemi-AnarakiTFV23}~\cite{Fatemi-AnarakiTFV23}, \href{../works/YuraszeckMPV22.pdf}{YuraszeckMPV22}~\cite{YuraszeckMPV22}, \href{../works/abs-2211-14492.pdf}{abs-2211-14492}~\cite{abs-2211-14492}, \href{../works/MullerMKP22.pdf}{MullerMKP22}~\cite{MullerMKP22}, \href{../works/ColT22.pdf}{ColT22}~\cite{ColT22}, \href{../works/EmdeZD22.pdf}{EmdeZD22}~\cite{EmdeZD22}, \href{../works/ZhangJZL22.pdf}{ZhangJZL22}~\cite{ZhangJZL22}, \href{../works/FanXG21.pdf}{FanXG21}~\cite{FanXG21}, \href{../works/LacknerMMWW21.pdf}{LacknerMMWW21}~\cite{LacknerMMWW21}, \href{../works/HamPK21.pdf}{HamPK21}~\cite{HamPK21}, \href{../works/PandeyS21a.pdf}{PandeyS21a}~\cite{PandeyS21a}, \href{../works/Astrand21.pdf}{Astrand21}~\cite{Astrand21}, \href{../works/TangB20.pdf}{TangB20}~\cite{TangB20}, \href{../works/MengZRZL20.pdf}{MengZRZL20}~\cite{MengZRZL20}, \href{../works/NattafM20.pdf}{NattafM20}~\cite{NattafM20}, \href{../works/Novas19.pdf}{Novas19}~\cite{Novas19}, \href{../works/LaborieRSV18.pdf}{LaborieRSV18}~\cite{LaborieRSV18}, \href{../works/Ham18.pdf}{Ham18}~\cite{Ham18}, \href{../works/GrimesH15.pdf}{GrimesH15}~\cite{GrimesH15}, \href{../works/KoschB14.pdf}{KoschB14}~\cite{KoschB14}, \href{../works/HarjunkoskiMBC14.pdf}{HarjunkoskiMBC14}~\cite{HarjunkoskiMBC14}, \href{../works/TerekhovTDB14.pdf}{TerekhovTDB14}~\cite{TerekhovTDB14}, \href{../works/Malapert11.pdf}{Malapert11}~\cite{Malapert11}, \href{../works/Lombardi10.pdf}{Lombardi10}~\cite{Lombardi10}\\
ApplicationAreas & ship building &  &  & \\
ApplicationAreas & shipping line &  &  & \href{../works/QinDCS20.pdf}{QinDCS20}~\cite{QinDCS20}, \href{../works/LaborieRSV18.pdf}{LaborieRSV18}~\cite{LaborieRSV18}, \href{../works/KelarevaTK13.pdf}{KelarevaTK13}~\cite{KelarevaTK13}\\
ApplicationAreas & steel cable &  &  & \href{../works/AalianPG23.pdf}{AalianPG23}~\cite{AalianPG23}\\
ApplicationAreas & steel mill & \href{../works/GaySS14.pdf}{GaySS14}~\cite{GaySS14}, \href{../works/Letort13.pdf}{Letort13}~\cite{Letort13}, \href{../works/HeinzSSW12.pdf}{HeinzSSW12}~\cite{HeinzSSW12}, \href{../works/SchausHMCMD11.pdf}{SchausHMCMD11}~\cite{SchausHMCMD11}, \href{../works/HentenryckM08.pdf}{HentenryckM08}~\cite{HentenryckM08}, \href{../works/GarganiR07.pdf}{GarganiR07}~\cite{GarganiR07} &  & \href{../works/abs-2312-13682.pdf}{abs-2312-13682}~\cite{abs-2312-13682}, \href{../works/PerezGSL23.pdf}{PerezGSL23}~\cite{PerezGSL23}, \href{../works/DoulabiRP16.pdf}{DoulabiRP16}~\cite{DoulabiRP16}, \href{../works/MenciaSV13.pdf}{MenciaSV13}~\cite{MenciaSV13}, \href{../works/MenciaSV12.pdf}{MenciaSV12}~\cite{MenciaSV12}\\
ApplicationAreas & super-computer & \href{../works/BorghesiBLMB18.pdf}{BorghesiBLMB18}~\cite{BorghesiBLMB18}, \href{../works/BridiBLMB16.pdf}{BridiBLMB16}~\cite{BridiBLMB16}, \href{../works/BartoliniBBLM14.pdf}{BartoliniBBLM14}~\cite{BartoliniBBLM14} &  & \href{../works/LuoB22.pdf}{LuoB22}~\cite{LuoB22}, \href{../works/GalleguillosKSB19.pdf}{GalleguillosKSB19}~\cite{GalleguillosKSB19}, \href{../works/Dejemeppe16.pdf}{Dejemeppe16}~\cite{Dejemeppe16}, \href{../works/HurleyOS16.pdf}{HurleyOS16}~\cite{HurleyOS16}\\
ApplicationAreas & surgery & \href{../works/GurPAE23.pdf}{GurPAE23}~\cite{GurPAE23}, \href{../works/FarsiTM22.pdf}{FarsiTM22}~\cite{FarsiTM22}, \href{../works/RoshanaeiBAUB20.pdf}{RoshanaeiBAUB20}~\cite{RoshanaeiBAUB20}, \href{../works/GurEA19.pdf}{GurEA19}~\cite{GurEA19}, \href{../works/YounespourAKE19.pdf}{YounespourAKE19}~\cite{YounespourAKE19}, \href{../works/RoshanaeiLAU17.pdf}{RoshanaeiLAU17}~\cite{RoshanaeiLAU17}, \href{../works/DoulabiRP16.pdf}{DoulabiRP16}~\cite{DoulabiRP16}, \href{../works/WangMD15.pdf}{WangMD15}~\cite{WangMD15}, \href{../works/DoulabiRP14.pdf}{DoulabiRP14}~\cite{DoulabiRP14}, \href{../works/Wolf11.pdf}{Wolf11}~\cite{Wolf11}, \href{../works/Wolf09.pdf}{Wolf09}~\cite{Wolf09} & \href{../works/ZarandiASC20.pdf}{ZarandiASC20}~\cite{ZarandiASC20}, \href{../works/TopalogluO11.pdf}{TopalogluO11}~\cite{TopalogluO11} & \href{../works/ForbesHJST24.pdf}{ForbesHJST24}~\cite{ForbesHJST24}, \href{../works/AlfieriGPS23.pdf}{AlfieriGPS23}~\cite{AlfieriGPS23}, \href{../works/NaderiBZ22.pdf}{NaderiBZ22}~\cite{NaderiBZ22}, \href{../works/ElciOH22.pdf}{ElciOH22}~\cite{ElciOH22}, \href{../works/Lemos21.pdf}{Lemos21}~\cite{Lemos21}, \href{../works/FrimodigS19.pdf}{FrimodigS19}~\cite{FrimodigS19}\\
ApplicationAreas & telescope & \href{../works/FrankK05.pdf}{FrankK05}~\cite{FrankK05} & \href{../works/CatusseCBL16.pdf}{CatusseCBL16}~\cite{CatusseCBL16} & \href{../works/BidotVLB09.pdf}{BidotVLB09}~\cite{BidotVLB09}, \href{../works/BeckW07.pdf}{BeckW07}~\cite{BeckW07}, \href{../works/Beck99.pdf}{Beck99}~\cite{Beck99}, \href{../works/PembertonG98.pdf}{PembertonG98}~\cite{PembertonG98}, \href{../works/Wallace96.pdf}{Wallace96}~\cite{Wallace96}\\
ApplicationAreas & torpedo & \href{../works/GoldwaserS18.pdf}{GoldwaserS18}~\cite{GoldwaserS18}, \href{../works/GoldwaserS17.pdf}{GoldwaserS17}~\cite{GoldwaserS17}, \href{../works/KletzanderM17.pdf}{KletzanderM17}~\cite{KletzanderM17} & \href{../works/AntuoriHHEN20.pdf}{AntuoriHHEN20}~\cite{AntuoriHHEN20} & \href{../works/Hooker19.pdf}{Hooker19}~\cite{Hooker19}\\
ApplicationAreas & train schedule & \href{../works/MarliereSPR23.pdf}{MarliereSPR23}~\cite{MarliereSPR23}, \href{../works/Lemos21.pdf}{Lemos21}~\cite{Lemos21}, \href{../works/CappartS17.pdf}{CappartS17}~\cite{CappartS17}, \href{../works/RodriguezS09.pdf}{RodriguezS09}~\cite{RodriguezS09}, \href{../works/Rodriguez07b.pdf}{Rodriguez07b}~\cite{Rodriguez07b}, \href{../works/Geske05.pdf}{Geske05}~\cite{Geske05} & \href{../works/ZarandiASC20.pdf}{ZarandiASC20}~\cite{ZarandiASC20}, \href{../works/LammaMM97.pdf}{LammaMM97}~\cite{LammaMM97}, \href{../works/BrusoniCLMMT96.pdf}{BrusoniCLMMT96}~\cite{BrusoniCLMMT96} & \href{../works/abs-2312-13682.pdf}{abs-2312-13682}~\cite{abs-2312-13682}, \href{../works/SvancaraB22.pdf}{SvancaraB22}~\cite{SvancaraB22}, \href{../works/GeibingerMM21.pdf}{GeibingerMM21}~\cite{GeibingerMM21}, \href{../works/Novas19.pdf}{Novas19}~\cite{Novas19}, \href{../works/Froger16.pdf}{Froger16}~\cite{Froger16}, \href{../works/Rodriguez07.pdf}{Rodriguez07}~\cite{Rodriguez07}, \href{../works/RodriguezDG02.pdf}{RodriguezDG02}~\cite{RodriguezDG02}, \href{../works/MartinPY01.pdf}{MartinPY01}~\cite{MartinPY01}, \href{../works/Wallace96.pdf}{Wallace96}~\cite{Wallace96}\\
ApplicationAreas & vaccine &  & \href{../works/GuoZ23.pdf}{GuoZ23}~\cite{GuoZ23} & \href{../works/BonninMNE24.pdf}{BonninMNE24}~\cite{BonninMNE24}, \href{../works/JuvinHL23a.pdf}{JuvinHL23a}~\cite{JuvinHL23a}\\
ApplicationAreas & wildfire &  & \href{../works/ArtiguesHQT21.pdf}{ArtiguesHQT21}~\cite{ArtiguesHQT21} & \\
ApplicationAreas & workforce scheduling & \href{../works/BourreauGGLT22.pdf}{BourreauGGLT22}~\cite{BourreauGGLT22}, \href{../works/MusliuSS18.pdf}{MusliuSS18}~\cite{MusliuSS18}, \href{../works/Wallace06.pdf}{Wallace06}~\cite{Wallace06} & \href{../works/AntunesABD20.pdf}{AntunesABD20}~\cite{AntunesABD20}, \href{../works/AntunesABD18.pdf}{AntunesABD18}~\cite{AntunesABD18} & \href{../works/GokPTGO23.pdf}{GokPTGO23}~\cite{GokPTGO23}, \href{../works/FallahiAC20.pdf}{FallahiAC20}~\cite{FallahiAC20}, \href{../works/CorreaLR07.pdf}{CorreaLR07}~\cite{CorreaLR07}, \href{../works/Mason01.pdf}{Mason01}~\cite{Mason01}, \href{../works/Darby-DowmanLMZ97.pdf}{Darby-DowmanLMZ97}~\cite{Darby-DowmanLMZ97}\\
ApplicationAreas & yard crane &  & \href{../works/QinDCS20.pdf}{QinDCS20}~\cite{QinDCS20}, \href{../works/Hooker19.pdf}{Hooker19}~\cite{Hooker19} & \href{../works/EmdeZD22.pdf}{EmdeZD22}~\cite{EmdeZD22}, \href{../works/WallaceY20.pdf}{WallaceY20}~\cite{WallaceY20}\\
\end{longtable}
}


\clearpage
\subsection{Concept Type Industries}
\label{sec:Industries}
{\scriptsize
\begin{longtable}{lp{3cm}>{\raggedright\arraybackslash}p{6cm}>{\raggedright\arraybackslash}p{6cm}>{\raggedright\arraybackslash}p{8cm}}
\rowcolor{white}\caption{Works for Concepts of Type Industries}\\ \toprule
\rowcolor{white}Type & Keyword & High & Medium & Low\\ \midrule\endhead
\bottomrule
\endfoot
Industries & IT industry &  &  & \href{../works/SchnellH15.pdf}{SchnellH15}~\cite{SchnellH15}\\
Industries & PCB industry &  &  & \\
Industries & aerospace industry &  &  & \href{../works/SchildW00.pdf}{SchildW00}~\cite{SchildW00}\\
Industries & agricultural industry & \href{../works/WinterMMW22.pdf}{WinterMMW22}~\cite{WinterMMW22} &  & \\
Industries & agrifood industry &  &  & \href{../works/Groleaz21.pdf}{Groleaz21}~\cite{Groleaz21}\\
Industries & airline industry &  &  & \href{../works/GokPTGO23.pdf}{GokPTGO23}~\cite{GokPTGO23}, \href{../works/HachemiGR11.pdf}{HachemiGR11}~\cite{HachemiGR11}, \href{../works/Mason01.pdf}{Mason01}~\cite{Mason01}\\
Industries & automobile industry &  &  & \href{../works/HauderBRPA20.pdf}{HauderBRPA20}~\cite{HauderBRPA20}, \href{../works/abs-1902-09244.pdf}{abs-1902-09244}~\cite{abs-1902-09244}, \href{../works/Limtanyakul07.pdf}{Limtanyakul07}~\cite{Limtanyakul07}\\
Industries & automotive industry &  & \href{../works/GuoZ23.pdf}{GuoZ23}~\cite{GuoZ23}, \href{../works/LimtanyakulS12.pdf}{LimtanyakulS12}~\cite{LimtanyakulS12} & \href{../works/CzerniachowskaWZ23.pdf}{CzerniachowskaWZ23}~\cite{CzerniachowskaWZ23}, \href{../works/EmdeZD22.pdf}{EmdeZD22}~\cite{EmdeZD22}, \href{../works/AntuoriHHEN21.pdf}{AntuoriHHEN21}~\cite{AntuoriHHEN21}, \href{../works/BonfiettiZLM16.pdf}{BonfiettiZLM16}~\cite{BonfiettiZLM16}, \href{../works/SchildW00.pdf}{SchildW00}~\cite{SchildW00}, \href{../works/Wallace96.pdf}{Wallace96}~\cite{Wallace96}\\
Industries & aviation industry &  &  & \\
Industries & cable industry &  &  & \href{../works/ZhuSZW23.pdf}{ZhuSZW23}~\cite{ZhuSZW23}\\
Industries & carpet industry &  &  & \href{../works/Schutt11.pdf}{Schutt11}~\cite{Schutt11}\\
Industries & chemical industry &  & \href{../works/Timpe02.pdf}{Timpe02}~\cite{Timpe02} & \href{../works/LaborieRSV18.pdf}{LaborieRSV18}~\cite{LaborieRSV18}, \href{../works/GilesH16.pdf}{GilesH16}~\cite{GilesH16}, \href{../works/HarjunkoskiMBC14.pdf}{HarjunkoskiMBC14}~\cite{HarjunkoskiMBC14}, \href{../works/LombardiM12.pdf}{LombardiM12}~\cite{LombardiM12}, \href{../works/ChenGPSH10.pdf}{ChenGPSH10}~\cite{ChenGPSH10}, \href{../works/PoderBS04.pdf}{PoderBS04}~\cite{PoderBS04}, \href{../works/Simonis99.pdf}{Simonis99}~\cite{Simonis99}, \href{../works/Simonis95a.pdf}{Simonis95a}~\cite{Simonis95a}\\
Industries & chemical processing industry &  &  & \href{../works/GilesH16.pdf}{GilesH16}~\cite{GilesH16}\\
Industries & chemistry industry &  &  & \href{../works/ChenGPSH10.pdf}{ChenGPSH10}~\cite{ChenGPSH10}\\
Industries & chips industry &  &  & \href{../works/AbreuN22.pdf}{AbreuN22}~\cite{AbreuN22}\\
Industries & circuit boards industry &  &  & \href{../works/MokhtarzadehTNF20.pdf}{MokhtarzadehTNF20}~\cite{MokhtarzadehTNF20}\\
Industries & control system industry &  &  & \href{../works/BonfiettiZLM16.pdf}{BonfiettiZLM16}~\cite{BonfiettiZLM16}\\
Industries & cutting industry &  &  & \href{../works/RiahiNS018.pdf}{RiahiNS018}~\cite{RiahiNS018}\\
Industries & dairy industry &  & \href{../works/EscobetPQPRA19.pdf}{EscobetPQPRA19}~\cite{EscobetPQPRA19}, \href{../works/HarjunkoskiMBC14.pdf}{HarjunkoskiMBC14}~\cite{HarjunkoskiMBC14} & \href{../works/Groleaz21.pdf}{Groleaz21}~\cite{Groleaz21}\\
Industries & dismantling industry &  &  & \href{../works/HubnerGSV21.pdf}{HubnerGSV21}~\cite{HubnerGSV21}\\
Industries & drawing industry &  &  & \href{../works/Simonis95a.pdf}{Simonis95a}~\cite{Simonis95a}\\
Industries & electricity industry & \href{../works/Froger16.pdf}{Froger16}~\cite{Froger16} &  & \href{../works/PopovicCGNC22.pdf}{PopovicCGNC22}~\cite{PopovicCGNC22}, \href{../works/Godet21a.pdf}{Godet21a}~\cite{Godet21a}, \href{../works/AntunesABD20.pdf}{AntunesABD20}~\cite{AntunesABD20}, \href{../works/AntunesABD18.pdf}{AntunesABD18}~\cite{AntunesABD18}\\
Industries & electricity industry &  &  & \\
Industries & electronics industry &  &  & \href{../works/LacknerMMWW23.pdf}{LacknerMMWW23}~\cite{LacknerMMWW23}, \href{../works/LacknerMMWW21.pdf}{LacknerMMWW21}~\cite{LacknerMMWW21}\\
Industries & electroplating industry &  &  & \href{../works/NovasH12.pdf}{NovasH12}~\cite{NovasH12}\\
Industries & energy industry &  & \href{../works/Froger16.pdf}{Froger16}~\cite{Froger16} & \href{../works/KovacsV06.pdf}{KovacsV06}~\cite{KovacsV06}\\
Industries & fashion industry &  &  & \href{../works/Jans09.pdf}{Jans09}~\cite{Jans09}\\
Industries & food industry &  & \href{../works/Groleaz21.pdf}{Groleaz21}~\cite{Groleaz21} & \href{../works/Fatemi-AnarakiTFV23.pdf}{Fatemi-AnarakiTFV23}~\cite{Fatemi-AnarakiTFV23}, \href{../works/OujanaAYB22.pdf}{OujanaAYB22}~\cite{OujanaAYB22}, \href{../works/GroleazNS20.pdf}{GroleazNS20}~\cite{GroleazNS20}, \href{../works/GroleazNS20a.pdf}{GroleazNS20a}~\cite{GroleazNS20a}, \href{../works/EscobetPQPRA19.pdf}{EscobetPQPRA19}~\cite{EscobetPQPRA19}, \href{../works/HachemiGR11.pdf}{HachemiGR11}~\cite{HachemiGR11}, \href{../works/SimonisCK00.pdf}{SimonisCK00}~\cite{SimonisCK00}, \href{../works/Simonis99.pdf}{Simonis99}~\cite{Simonis99}, \href{../works/SimonisC95.pdf}{SimonisC95}~\cite{SimonisC95}, \href{../works/Simonis95.pdf}{Simonis95}~\cite{Simonis95}\\
Industries & food-processing industry &  &  & \href{../works/KlankeBYE21.pdf}{KlankeBYE21}~\cite{KlankeBYE21}, \href{../works/HauderBRPA20.pdf}{HauderBRPA20}~\cite{HauderBRPA20}, \href{../works/abs-1902-09244.pdf}{abs-1902-09244}~\cite{abs-1902-09244}\\
Industries & forest industry &  &  & \href{../works/HachemiGR11.pdf}{HachemiGR11}~\cite{HachemiGR11}\\
Industries & forging industry &  &  & \href{../works/LuoB22.pdf}{LuoB22}~\cite{LuoB22}\\
Industries & foundry industry &  &  & \href{../works/Jans09.pdf}{Jans09}~\cite{Jans09}\\
Industries & garment industry &  &  & \href{../works/GuoZ23.pdf}{GuoZ23}~\cite{GuoZ23}\\
Industries & gas industry &  &  & \href{../works/ZarandiASC20.pdf}{ZarandiASC20}~\cite{ZarandiASC20}, \href{../works/GoelSHFS15.pdf}{GoelSHFS15}~\cite{GoelSHFS15}\\
Industries & glass industry &  &  & \href{../works/Lunardi20.pdf}{Lunardi20}~\cite{Lunardi20}, \href{../works/LunardiBLRV20.pdf}{LunardiBLRV20}~\cite{LunardiBLRV20}, \href{../works/abs-1902-09244.pdf}{abs-1902-09244}~\cite{abs-1902-09244}\\
Industries & heavy industry &  &  & \href{../works/CorreaLR07.pdf}{CorreaLR07}~\cite{CorreaLR07}\\
Industries & insulation industry &  &  & \href{../works/YunusogluY22.pdf}{YunusogluY22}~\cite{YunusogluY22}\\
Industries & leisure industry &  &  & \\
Industries & lumber industry &  &  & \href{../works/NattafDYW19.pdf}{NattafDYW19}~\cite{NattafDYW19}\\
Industries & manufacturing industry &  &  & \href{../works/PrataAN23.pdf}{PrataAN23}~\cite{PrataAN23}, \href{../works/CzerniachowskaWZ23.pdf}{CzerniachowskaWZ23}~\cite{CzerniachowskaWZ23}, \href{../works/LacknerMMWW23.pdf}{LacknerMMWW23}~\cite{LacknerMMWW23}, \href{../works/WinterMMW22.pdf}{WinterMMW22}~\cite{WinterMMW22}, \href{../works/YuraszeckMPV22.pdf}{YuraszeckMPV22}~\cite{YuraszeckMPV22}, \href{../works/LacknerMMWW21.pdf}{LacknerMMWW21}~\cite{LacknerMMWW21}, \href{../works/FanXG21.pdf}{FanXG21}~\cite{FanXG21}, \href{../works/TangB20.pdf}{TangB20}~\cite{TangB20}, \href{../works/Mercier-AubinGQ20.pdf}{Mercier-AubinGQ20}~\cite{Mercier-AubinGQ20}, \href{../works/EscobetPQPRA19.pdf}{EscobetPQPRA19}~\cite{EscobetPQPRA19}, \href{../works/GedikKEK18.pdf}{GedikKEK18}~\cite{GedikKEK18}\\
Industries & maritime industry &  &  & \href{../works/Astrand21.pdf}{Astrand21}~\cite{Astrand21}, \href{../works/QinDCS20.pdf}{QinDCS20}~\cite{QinDCS20}, \href{../works/SacramentoSP20.pdf}{SacramentoSP20}~\cite{SacramentoSP20}\\
Industries & metal industry &  &  & \href{../works/LuoB22.pdf}{LuoB22}~\cite{LuoB22}\\
Industries & metalworking industry &  &  & \\
Industries & mineral industry &  &  & \href{../works/Astrand21.pdf}{Astrand21}~\cite{Astrand21}, \href{../works/Astrand0F21.pdf}{Astrand0F21}~\cite{Astrand0F21}, \href{../works/AstrandJZ20.pdf}{AstrandJZ20}~\cite{AstrandJZ20}, \href{../works/BlomBPS14.pdf}{BlomBPS14}~\cite{BlomBPS14}\\
Industries & mining industry &  & \href{../works/AalianPG23.pdf}{AalianPG23}~\cite{AalianPG23} & \href{../works/abs-2402-00459.pdf}{abs-2402-00459}~\cite{abs-2402-00459}, \href{../works/CampeauG22.pdf}{CampeauG22}~\cite{CampeauG22}, \href{../works/Astrand21.pdf}{Astrand21}~\cite{Astrand21}, \href{../works/Astrand0F21.pdf}{Astrand0F21}~\cite{Astrand0F21}, \href{../works/AstrandJZ20.pdf}{AstrandJZ20}~\cite{AstrandJZ20}, \href{../works/ThiruvadyWGS14.pdf}{ThiruvadyWGS14}~\cite{ThiruvadyWGS14}\\
Industries & nuclear industry &  &  & \\
Industries & oil industry &  &  & \href{../works/AbreuNP23.pdf}{AbreuNP23}~\cite{AbreuNP23}, \href{../works/AbreuAPNM21.pdf}{AbreuAPNM21}~\cite{AbreuAPNM21}, \href{../works/HarjunkoskiMBC14.pdf}{HarjunkoskiMBC14}~\cite{HarjunkoskiMBC14}, \href{../works/LopesCSM10.pdf}{LopesCSM10}~\cite{LopesCSM10}\\
Industries & packaging industry &  &  & \href{../works/ArmstrongGOS21.pdf}{ArmstrongGOS21}~\cite{ArmstrongGOS21}\\
Industries & painting industry &  &  & \href{../works/RiahiNS018.pdf}{RiahiNS018}~\cite{RiahiNS018}\\
Industries & paper industry &  &  & \href{../works/Dejemeppe16.pdf}{Dejemeppe16}~\cite{Dejemeppe16}, \href{../works/HarjunkoskiMBC14.pdf}{HarjunkoskiMBC14}~\cite{HarjunkoskiMBC14}\\
Industries & petro-chemical industry &  &  & \href{../works/LaborieRSV18.pdf}{LaborieRSV18}~\cite{LaborieRSV18}, \href{../works/GilesH16.pdf}{GilesH16}~\cite{GilesH16}, \href{../works/HarjunkoskiMBC14.pdf}{HarjunkoskiMBC14}~\cite{HarjunkoskiMBC14}\\
Industries & pharmaceutical industry &  &  & \href{../works/YuraszeckMCCR23.pdf}{YuraszeckMCCR23}~\cite{YuraszeckMCCR23}, \href{../works/CzerniachowskaWZ23.pdf}{CzerniachowskaWZ23}~\cite{CzerniachowskaWZ23}, \href{../works/GeibingerKKMMW21.pdf}{GeibingerKKMMW21}~\cite{GeibingerKKMMW21}, \href{../works/HamC16.pdf}{HamC16}~\cite{HamC16}, \href{../works/NovaraNH16.pdf}{NovaraNH16}~\cite{NovaraNH16}, \href{../works/HarjunkoskiMBC14.pdf}{HarjunkoskiMBC14}~\cite{HarjunkoskiMBC14}\\
Industries & potash industry &  &  & \href{../works/Astrand21.pdf}{Astrand21}~\cite{Astrand21}, \href{../works/Astrand0F21.pdf}{Astrand0F21}~\cite{Astrand0F21}, \href{../works/AstrandJZ20.pdf}{AstrandJZ20}~\cite{AstrandJZ20}, \href{../works/AstrandJZ18.pdf}{AstrandJZ18}~\cite{AstrandJZ18}\\
Industries & power industry & \href{../works/Froger16.pdf}{Froger16}~\cite{Froger16} &  & \href{../works/FrostD98.pdf}{FrostD98}~\cite{FrostD98}\\
Industries & printing industry & \href{../works/Lunardi20.pdf}{Lunardi20}~\cite{Lunardi20} & \href{../works/LunardiBLRV20.pdf}{LunardiBLRV20}~\cite{LunardiBLRV20} & \href{../works/BourreauGGLT22.pdf}{BourreauGGLT22}~\cite{BourreauGGLT22}\\
Industries & process industry &  & \href{../works/Timpe02.pdf}{Timpe02}~\cite{Timpe02} & \href{../works/Nattaf16.pdf}{Nattaf16}~\cite{Nattaf16}, \href{../works/BlomPS16.pdf}{BlomPS16}~\cite{BlomPS16}, \href{../works/HarjunkoskiMBC14.pdf}{HarjunkoskiMBC14}~\cite{HarjunkoskiMBC14}, \href{../works/HeinzSSW12.pdf}{HeinzSSW12}~\cite{HeinzSSW12}, \href{../works/ChenGPSH10.pdf}{ChenGPSH10}~\cite{ChenGPSH10}, \href{../works/Jans09.pdf}{Jans09}~\cite{Jans09}, \href{../works/Simonis99.pdf}{Simonis99}~\cite{Simonis99}, \href{../works/Wallace96.pdf}{Wallace96}~\cite{Wallace96}\\
Industries & processing industry &  & \href{../works/HauderBRPA20.pdf}{HauderBRPA20}~\cite{HauderBRPA20} & \href{../works/KlankeBYE21.pdf}{KlankeBYE21}~\cite{KlankeBYE21}, \href{../works/abs-1902-09244.pdf}{abs-1902-09244}~\cite{abs-1902-09244}, \href{../works/GilesH16.pdf}{GilesH16}~\cite{GilesH16}\\
Industries & railway industry &  &  & \href{../works/Lemos21.pdf}{Lemos21}~\cite{Lemos21}, \href{../works/Rodriguez07b.pdf}{Rodriguez07b}~\cite{Rodriguez07b}, \href{../works/Geske05.pdf}{Geske05}~\cite{Geske05}\\
Industries & repair industry &  &  & \href{../works/BoudreaultSLQ22.pdf}{BoudreaultSLQ22}~\cite{BoudreaultSLQ22}\\
Industries & retail industry &  &  & \href{../works/ChapadosJR11.pdf}{ChapadosJR11}~\cite{ChapadosJR11}\\
Industries & semiconductor industry &  &  & \href{../works/PenzDN23.pdf}{PenzDN23}~\cite{PenzDN23}, \href{../works/QinWSLS21.pdf}{QinWSLS21}~\cite{QinWSLS21}, \href{../works/NattafDYW19.pdf}{NattafDYW19}~\cite{NattafDYW19}, \href{../works/BajestaniB15.pdf}{BajestaniB15}~\cite{BajestaniB15}, \href{../works/GrimesH15.pdf}{GrimesH15}~\cite{GrimesH15}, \href{../works/NovasH12.pdf}{NovasH12}~\cite{NovasH12}, \href{../works/Lombardi10.pdf}{Lombardi10}~\cite{Lombardi10}, \href{../works/LombardiMRB10.pdf}{LombardiMRB10}~\cite{LombardiMRB10}, \href{../works/KrogtLPHJ07.pdf}{KrogtLPHJ07}~\cite{KrogtLPHJ07}\\
Industries & semiprocess industry &  &  & \href{../works/ChenGPSH10.pdf}{ChenGPSH10}~\cite{ChenGPSH10}\\
Industries & service industry &  &  & \href{../works/GurEA19.pdf}{GurEA19}~\cite{GurEA19}, \href{../works/DoomsH08.pdf}{DoomsH08}~\cite{DoomsH08}\\
Industries & ship repair industry &  &  & \href{../works/BoudreaultSLQ22.pdf}{BoudreaultSLQ22}~\cite{BoudreaultSLQ22}\\
Industries & shipping industry &  &  & \href{../works/Astrand21.pdf}{Astrand21}~\cite{Astrand21}, \href{../works/SacramentoSP20.pdf}{SacramentoSP20}~\cite{SacramentoSP20}, \href{../works/QinDCS20.pdf}{QinDCS20}~\cite{QinDCS20}\\
Industries & software industry &  &  & \href{../works/BartakS11.pdf}{BartakS11}~\cite{BartakS11}\\
Industries & solar cell industry &  &  & \href{../works/Novas19.pdf}{Novas19}~\cite{Novas19}\\
Industries & steel industry &  & \href{../works/DavenportKRSH07.pdf}{DavenportKRSH07}~\cite{DavenportKRSH07} & \href{../works/LacknerMMWW23.pdf}{LacknerMMWW23}~\cite{LacknerMMWW23}, \href{../works/KimCMLLP23.pdf}{KimCMLLP23}~\cite{KimCMLLP23}, \href{../works/IsikYA23.pdf}{IsikYA23}~\cite{IsikYA23}, \href{../works/OujanaAYB22.pdf}{OujanaAYB22}~\cite{OujanaAYB22}, \href{../works/LacknerMMWW21.pdf}{LacknerMMWW21}~\cite{LacknerMMWW21}, \href{../works/HauderBRPA20.pdf}{HauderBRPA20}~\cite{HauderBRPA20}, \href{../works/abs-1902-09244.pdf}{abs-1902-09244}~\cite{abs-1902-09244}, \href{../works/GoldwaserS18.pdf}{GoldwaserS18}~\cite{GoldwaserS18}, \href{../works/GoldwaserS17.pdf}{GoldwaserS17}~\cite{GoldwaserS17}, \href{../works/KletzanderM17.pdf}{KletzanderM17}~\cite{KletzanderM17}, \href{../works/HeinzSSW12.pdf}{HeinzSSW12}~\cite{HeinzSSW12}, \href{../works/SchausHMCMD11.pdf}{SchausHMCMD11}~\cite{SchausHMCMD11}, \href{../works/GrimesH10.pdf}{GrimesH10}~\cite{GrimesH10}, \href{../works/GarganiR07.pdf}{GarganiR07}~\cite{GarganiR07}\\
Industries & steel making industry &  &  & \\
Industries & sugar industry &  &  & \href{../works/MartinPY01.pdf}{MartinPY01}~\cite{MartinPY01}\\
Industries & taxi industry &  &  & \href{../works/Ham18.pdf}{Ham18}~\cite{Ham18}\\
Industries & telecommunication industry &  &  & \\
Industries & textile industry & \href{../works/Mercier-AubinGQ20.pdf}{Mercier-AubinGQ20}~\cite{Mercier-AubinGQ20} &  & \href{../works/ZarandiASC20.pdf}{ZarandiASC20}~\cite{ZarandiASC20}, \href{../works/BessiereHMQW14.pdf}{BessiereHMQW14}~\cite{BessiereHMQW14}\\
Industries & tire industry &  &  & \href{../works/Jans09.pdf}{Jans09}~\cite{Jans09}\\
Industries & tourism industry &  &  & \href{../works/LiuCGM17.pdf}{LiuCGM17}~\cite{LiuCGM17}\\
Industries & trade industry &  &  & \href{../works/ParkUJR19.pdf}{ParkUJR19}~\cite{ParkUJR19}\\
Industries & transportation industry &  &  & \href{../works/GoelSHFS15.pdf}{GoelSHFS15}~\cite{GoelSHFS15}\\
Industries & wind industry & \href{../works/Froger16.pdf}{Froger16}~\cite{Froger16} &  & \\
\end{longtable}
}


\clearpage
\subsection{Concept Type Benchmarks}
\label{sec:Benchmarks}
{\scriptsize
\begin{longtable}{lp{3cm}>{\raggedright\arraybackslash}p{6cm}>{\raggedright\arraybackslash}p{6cm}>{\raggedright\arraybackslash}p{8cm}}
\rowcolor{white}\caption{Works for Concepts of Type Benchmarks}\\ \toprule
\rowcolor{white}Type & Keyword & High & Medium & Low\\ \midrule\endhead
\bottomrule
\endfoot
Benchmarks & CSPlib & \href{../works/Siala15.pdf}{Siala15}~\cite{Siala15}, \href{../works/Siala15a.pdf}{Siala15a}~\cite{Siala15a}, \href{../works/SchausHMCMD11.pdf}{SchausHMCMD11}~\cite{SchausHMCMD11}, \href{../works/GarganiR07.pdf}{GarganiR07}~\cite{GarganiR07} & \href{../works/LaborieRSV18.pdf}{LaborieRSV18}~\cite{LaborieRSV18}, \href{../works/German18.pdf}{German18}~\cite{German18}, \href{../works/CappartTSR18.pdf}{CappartTSR18}~\cite{CappartTSR18}, \href{../works/MossigeGSMC17.pdf}{MossigeGSMC17}~\cite{MossigeGSMC17}, \href{../works/NovaraNH16.pdf}{NovaraNH16}~\cite{NovaraNH16}, \href{../works/Letort13.pdf}{Letort13}~\cite{Letort13}, \href{../works/HeinzSSW12.pdf}{HeinzSSW12}~\cite{HeinzSSW12}, \href{../works/BandaSC11.pdf}{BandaSC11}~\cite{BandaSC11} & \href{../works/ThomasKS20.pdf}{ThomasKS20}~\cite{ThomasKS20}, \href{../works/LiuLH19.pdf}{LiuLH19}~\cite{LiuLH19}, \href{../works/GelainPRVW17.pdf}{GelainPRVW17}~\cite{GelainPRVW17}, \href{../works/GaySS14.pdf}{GaySS14}~\cite{GaySS14}, \href{../works/RendlPHPR12.pdf}{RendlPHPR12}~\cite{RendlPHPR12}, \href{../works/HentenryckM08.pdf}{HentenryckM08}~\cite{HentenryckM08}\\
Benchmarks & Roadef & \href{../works/Froger16.pdf}{Froger16}~\cite{Froger16}, \href{../works/Siala15.pdf}{Siala15}~\cite{Siala15}, \href{../works/Siala15a.pdf}{Siala15a}~\cite{Siala15a} & \href{../works/Nattaf16.pdf}{Nattaf16}~\cite{Nattaf16}, \href{../works/LetortCB15.pdf}{LetortCB15}~\cite{LetortCB15}, \href{../works/Kameugne14.pdf}{Kameugne14}~\cite{Kameugne14}, \href{../works/Letort13.pdf}{Letort13}~\cite{Letort13}, \href{../works/LetortCB13.pdf}{LetortCB13}~\cite{LetortCB13}, \href{../works/LetortBC12.pdf}{LetortBC12}~\cite{LetortBC12} & \href{../works/CzerniachowskaWZ23.pdf}{CzerniachowskaWZ23}~\cite{CzerniachowskaWZ23}, \href{../works/HanenKP21.pdf}{HanenKP21}~\cite{HanenKP21}, \href{../works/Lemos21.pdf}{Lemos21}~\cite{Lemos21}, \href{../works/GokGSTO20.pdf}{GokGSTO20}~\cite{GokGSTO20}, \href{../works/CarlierPSJ20.pdf}{CarlierPSJ20}~\cite{CarlierPSJ20}, \href{../works/Polo-MejiaALB20.pdf}{Polo-MejiaALB20}~\cite{Polo-MejiaALB20}, \href{../works/MalapertN19.pdf}{MalapertN19}~\cite{MalapertN19}, \href{../works/OuelletQ18.pdf}{OuelletQ18}~\cite{OuelletQ18}, \href{../works/Tesch18.pdf}{Tesch18}~\cite{Tesch18}, \href{../works/Fahimi16.pdf}{Fahimi16}~\cite{Fahimi16}, \href{../works/Tesch16.pdf}{Tesch16}~\cite{Tesch16}, \href{../works/Menana11.pdf}{Menana11}~\cite{Menana11}, \href{../works/Acuna-AgostMFG09.pdf}{Acuna-AgostMFG09}~\cite{Acuna-AgostMFG09}, \href{../works/Wallace06.pdf}{Wallace06}~\cite{Wallace06}, \href{../works/Elkhyari03.pdf}{Elkhyari03}~\cite{Elkhyari03}\\
Benchmarks & benchmark & \href{../works/LiLZDZW24.pdf}{LiLZDZW24}~\cite{LiLZDZW24}, \href{../works/JuvinHL23a.pdf}{JuvinHL23a}~\cite{JuvinHL23a}, \href{../works/IsikYA23.pdf}{IsikYA23}~\cite{IsikYA23}, \href{../works/AlfieriGPS23.pdf}{AlfieriGPS23}~\cite{AlfieriGPS23}, \href{../works/JuvinHHL23.pdf}{JuvinHHL23}~\cite{JuvinHHL23}, \href{../works/Bit-Monnot23.pdf}{Bit-Monnot23}~\cite{Bit-Monnot23}, \href{../works/NaderiBZR23.pdf}{NaderiBZR23}~\cite{NaderiBZR23}, \href{../works/AfsarVPG23.pdf}{AfsarVPG23}~\cite{AfsarVPG23}, \href{../works/YuraszeckMCCR23.pdf}{YuraszeckMCCR23}~\cite{YuraszeckMCCR23}, \href{../works/ShaikhK23.pdf}{ShaikhK23}~\cite{ShaikhK23}, \href{../works/ZhuSZW23.pdf}{ZhuSZW23}~\cite{ZhuSZW23}, \href{../works/NaderiRR23.pdf}{NaderiRR23}~\cite{NaderiRR23}, \href{../works/TasselGS23.pdf}{TasselGS23}~\cite{TasselGS23}, \href{../works/AbreuPNF23.pdf}{AbreuPNF23}~\cite{AbreuPNF23}, \href{../works/TardivoDFMP23.pdf}{TardivoDFMP23}~\cite{TardivoDFMP23}, \href{../works/LacknerMMWW23.pdf}{LacknerMMWW23}~\cite{LacknerMMWW23}, \href{../works/PovedaAA23.pdf}{PovedaAA23}~\cite{PovedaAA23}, \href{../works/abs-2306-05747.pdf}{abs-2306-05747}~\cite{abs-2306-05747}, \href{../works/AbreuNP23.pdf}{AbreuNP23}~\cite{AbreuNP23}, \href{../works/OuelletQ22.pdf}{OuelletQ22}~\cite{OuelletQ22}, \href{../works/ColT22.pdf}{ColT22}~\cite{ColT22}, \href{../works/WinterMMW22.pdf}{WinterMMW22}~\cite{WinterMMW22}, \href{../works/JuvinHL22.pdf}{JuvinHL22}~\cite{JuvinHL22}, \href{../works/NaderiBZ22a.pdf}{NaderiBZ22a}~\cite{NaderiBZ22a}, \href{../works/Teppan22.pdf}{Teppan22}~\cite{Teppan22}, \href{../works/MengGRZSC22.pdf}{MengGRZSC22}~\cite{MengGRZSC22}, \href{../works/ZhangJZL22.pdf}{ZhangJZL22}~\cite{ZhangJZL22}, \href{../works/TouatBT22.pdf}{TouatBT22}~\cite{TouatBT22}, \href{../works/AbreuN22.pdf}{AbreuN22}~\cite{AbreuN22}... (Total: 126) & \href{../works/ForbesHJST24.pdf}{ForbesHJST24}~\cite{ForbesHJST24}, \href{../works/abs-2402-00459.pdf}{abs-2402-00459}~\cite{abs-2402-00459}, \href{../works/NaderiBZ23.pdf}{NaderiBZ23}~\cite{NaderiBZ23}, \href{../works/YuraszeckMC23.pdf}{YuraszeckMC23}~\cite{YuraszeckMC23}, \href{../works/MontemanniD23a.pdf}{MontemanniD23a}~\cite{MontemanniD23a}, \href{../works/MarliereSPR23.pdf}{MarliereSPR23}~\cite{MarliereSPR23}, \href{../works/AkramNHRSA23.pdf}{AkramNHRSA23}~\cite{AkramNHRSA23}, \href{../works/KameugneFND23.pdf}{KameugneFND23}~\cite{KameugneFND23}, \href{../works/abs-2305-19888.pdf}{abs-2305-19888}~\cite{abs-2305-19888}, \href{../works/NaderiBZ22.pdf}{NaderiBZ22}~\cite{NaderiBZ22}, \href{../works/BourreauGGLT22.pdf}{BourreauGGLT22}~\cite{BourreauGGLT22}, \href{../works/KotaryFH22.pdf}{KotaryFH22}~\cite{KotaryFH22}, \href{../works/ZhangBB22.pdf}{ZhangBB22}~\cite{ZhangBB22}, \href{../works/FetgoD22.pdf}{FetgoD22}~\cite{FetgoD22}, \href{../works/OujanaAYB22.pdf}{OujanaAYB22}~\cite{OujanaAYB22}, \href{../works/HeinzNVH22.pdf}{HeinzNVH22}~\cite{HeinzNVH22}, \href{../works/MengLZB21.pdf}{MengLZB21}~\cite{MengLZB21}, \href{../works/Astrand21.pdf}{Astrand21}~\cite{Astrand21}, \href{../works/AbreuAPNM21.pdf}{AbreuAPNM21}~\cite{AbreuAPNM21}, \href{../works/KovacsTKSG21.pdf}{KovacsTKSG21}~\cite{KovacsTKSG21}, \href{../works/KletzanderMH21.pdf}{KletzanderMH21}~\cite{KletzanderMH21}, \href{../works/Lunardi20.pdf}{Lunardi20}~\cite{Lunardi20}, \href{../works/SacramentoSP20.pdf}{SacramentoSP20}~\cite{SacramentoSP20}, \href{../works/BenediktMH20.pdf}{BenediktMH20}~\cite{BenediktMH20}, \href{../works/BadicaBI20.pdf}{BadicaBI20}~\cite{BadicaBI20}, \href{../works/AntuoriHHEN20.pdf}{AntuoriHHEN20}~\cite{AntuoriHHEN20}, \href{../works/GroleazNS20.pdf}{GroleazNS20}~\cite{GroleazNS20}, \href{../works/MengZRZL20.pdf}{MengZRZL20}~\cite{MengZRZL20}, \href{../works/Ham20a.pdf}{Ham20a}~\cite{Ham20a}... (Total: 102) & \href{../works/BonninMNE24.pdf}{BonninMNE24}~\cite{BonninMNE24}, \href{../works/PrataAN23.pdf}{PrataAN23}~\cite{PrataAN23}, \href{../works/MontemanniD23.pdf}{MontemanniD23}~\cite{MontemanniD23}, \href{../works/GuoZ23.pdf}{GuoZ23}~\cite{GuoZ23}, \href{../works/Adelgren2023.pdf}{Adelgren2023}~\cite{Adelgren2023}, \href{../works/CzerniachowskaWZ23.pdf}{CzerniachowskaWZ23}~\cite{CzerniachowskaWZ23}, \href{../works/EfthymiouY23.pdf}{EfthymiouY23}~\cite{EfthymiouY23}, \href{../works/KimCMLLP23.pdf}{KimCMLLP23}~\cite{KimCMLLP23}, \href{../works/SquillaciPR23.pdf}{SquillaciPR23}~\cite{SquillaciPR23}, \href{../works/SvancaraB22.pdf}{SvancaraB22}~\cite{SvancaraB22}, \href{../works/JungblutK22.pdf}{JungblutK22}~\cite{JungblutK22}, \href{../works/ElciOH22.pdf}{ElciOH22}~\cite{ElciOH22}, \href{../works/PohlAK22.pdf}{PohlAK22}~\cite{PohlAK22}, \href{../works/YunusogluY22.pdf}{YunusogluY22}~\cite{YunusogluY22}, \href{../works/SubulanC22.pdf}{SubulanC22}~\cite{SubulanC22}, \href{../works/YuraszeckMPV22.pdf}{YuraszeckMPV22}~\cite{YuraszeckMPV22}, \href{../works/AwadMDMT22.pdf}{AwadMDMT22}~\cite{AwadMDMT22}, \href{../works/ArmstrongGOS22.pdf}{ArmstrongGOS22}~\cite{ArmstrongGOS22}, \href{../works/Astrand0F21.pdf}{Astrand0F21}~\cite{Astrand0F21}, \href{../works/VlkHT21.pdf}{VlkHT21}~\cite{VlkHT21}, \href{../works/Zahout21.pdf}{Zahout21}~\cite{Zahout21}, \href{../works/RoshanaeiN21.pdf}{RoshanaeiN21}~\cite{RoshanaeiN21}, \href{../works/HubnerGSV21.pdf}{HubnerGSV21}~\cite{HubnerGSV21}, \href{../works/KlankeBYE21.pdf}{KlankeBYE21}~\cite{KlankeBYE21}, \href{../works/ArmstrongGOS21.pdf}{ArmstrongGOS21}~\cite{ArmstrongGOS21}, \href{../works/AstrandJZ20.pdf}{AstrandJZ20}~\cite{AstrandJZ20}, \href{../works/LunardiBLRV20.pdf}{LunardiBLRV20}~\cite{LunardiBLRV20}, \href{../works/ThomasKS20.pdf}{ThomasKS20}~\cite{ThomasKS20}, \href{../works/QinDCS20.pdf}{QinDCS20}~\cite{QinDCS20}... (Total: 164)\\
Benchmarks & bitbucket &  & \href{../works/TardivoDFMP23.pdf}{TardivoDFMP23}~\cite{TardivoDFMP23}, \href{../works/Dejemeppe16.pdf}{Dejemeppe16}~\cite{Dejemeppe16} & \href{../works/ThomasKS20.pdf}{ThomasKS20}~\cite{ThomasKS20}, \href{../works/CauwelaertDS20.pdf}{CauwelaertDS20}~\cite{CauwelaertDS20}, \href{../works/HoundjiSW19.pdf}{HoundjiSW19}~\cite{HoundjiSW19}, \href{../works/CappartTSR18.pdf}{CappartTSR18}~\cite{CappartTSR18}, \href{../works/CauwelaertLS18.pdf}{CauwelaertLS18}~\cite{CauwelaertLS18}, \href{../works/He0GLW18.pdf}{He0GLW18}~\cite{He0GLW18}, \href{../works/CappartS17.pdf}{CappartS17}~\cite{CappartS17}, \href{../works/CauwelaertDMS16.pdf}{CauwelaertDMS16}~\cite{CauwelaertDMS16}, \href{../works/GayHLS15.pdf}{GayHLS15}~\cite{GayHLS15}, \href{../works/CauwelaertLS15.pdf}{CauwelaertLS15}~\cite{CauwelaertLS15}, \href{../works/DejemeppeCS15.pdf}{DejemeppeCS15}~\cite{DejemeppeCS15}, \href{../works/GayHS15a.pdf}{GayHS15a}~\cite{GayHS15a}, \href{../works/GayHS15.pdf}{GayHS15}~\cite{GayHS15}, \href{../works/DejemeppeD14.pdf}{DejemeppeD14}~\cite{DejemeppeD14}, \href{../works/HoundjiSWD14.pdf}{HoundjiSWD14}~\cite{HoundjiSWD14}\\
Benchmarks & generated instance & \href{../works/IsikYA23.pdf}{IsikYA23}~\cite{IsikYA23}, \href{../works/LuoB22.pdf}{LuoB22}~\cite{LuoB22}, \href{../works/abs-1911-04766.pdf}{abs-1911-04766}~\cite{abs-1911-04766} & \href{../works/abs-2312-13682.pdf}{abs-2312-13682}~\cite{abs-2312-13682}, \href{../works/PerezGSL23.pdf}{PerezGSL23}~\cite{PerezGSL23}, \href{../works/OrnekOS20.pdf}{OrnekOS20}~\cite{OrnekOS20}, \href{../works/Godet21a.pdf}{Godet21a}~\cite{Godet21a}, \href{../works/GodetLHS20.pdf}{GodetLHS20}~\cite{GodetLHS20}, \href{../works/KletzanderM20.pdf}{KletzanderM20}~\cite{KletzanderM20}, \href{../works/MejiaY20.pdf}{MejiaY20}~\cite{MejiaY20}, \href{../works/SunTB19.pdf}{SunTB19}~\cite{SunTB19}, \href{../works/Madi-WambaB16.pdf}{Madi-WambaB16}~\cite{Madi-WambaB16}, \href{../works/NattafALR16.pdf}{NattafALR16}~\cite{NattafALR16}, \href{../works/Dejemeppe16.pdf}{Dejemeppe16}~\cite{Dejemeppe16}, \href{../works/KelbelH11.pdf}{KelbelH11}~\cite{KelbelH11}, \href{../works/SchausHMCMD11.pdf}{SchausHMCMD11}~\cite{SchausHMCMD11} & \href{../works/abs-2402-00459.pdf}{abs-2402-00459}~\cite{abs-2402-00459}, \href{../works/EfthymiouY23.pdf}{EfthymiouY23}~\cite{EfthymiouY23}, \href{../works/abs-2305-19888.pdf}{abs-2305-19888}~\cite{abs-2305-19888}, \href{../works/Adelgren2023.pdf}{Adelgren2023}~\cite{Adelgren2023}, \href{../works/NaderiBZR23.pdf}{NaderiBZR23}~\cite{NaderiBZR23}, \href{../works/TouatBT22.pdf}{TouatBT22}~\cite{TouatBT22}, \href{../works/ZhangBB22.pdf}{ZhangBB22}~\cite{ZhangBB22}, \href{../works/abs-2211-14492.pdf}{abs-2211-14492}~\cite{abs-2211-14492}, \href{../works/ColT22.pdf}{ColT22}~\cite{ColT22}, \href{../works/YunusogluY22.pdf}{YunusogluY22}~\cite{YunusogluY22}, \href{../works/BoudreaultSLQ22.pdf}{BoudreaultSLQ22}~\cite{BoudreaultSLQ22}, \href{../works/YuraszeckMPV22.pdf}{YuraszeckMPV22}~\cite{YuraszeckMPV22}, \href{../works/HeinzNVH22.pdf}{HeinzNVH22}~\cite{HeinzNVH22}, \href{../works/HanenKP21.pdf}{HanenKP21}~\cite{HanenKP21}, \href{../works/Astrand21.pdf}{Astrand21}~\cite{Astrand21}, \href{../works/AbreuAPNM21.pdf}{AbreuAPNM21}~\cite{AbreuAPNM21}, \href{../works/GeibingerMM21.pdf}{GeibingerMM21}~\cite{GeibingerMM21}, \href{../works/Astrand0F21.pdf}{Astrand0F21}~\cite{Astrand0F21}, \href{../works/AbohashimaEG21.pdf}{AbohashimaEG21}~\cite{AbohashimaEG21}, \href{../works/abs-2102-08778.pdf}{abs-2102-08778}~\cite{abs-2102-08778}, \href{../works/AntuoriHHEN20.pdf}{AntuoriHHEN20}~\cite{AntuoriHHEN20}, \href{../works/CauwelaertDS20.pdf}{CauwelaertDS20}~\cite{CauwelaertDS20}, \href{../works/BenediktMH20.pdf}{BenediktMH20}~\cite{BenediktMH20}, \href{../works/MokhtarzadehTNF20.pdf}{MokhtarzadehTNF20}~\cite{MokhtarzadehTNF20}, \href{../works/RoshanaeiBAUB20.pdf}{RoshanaeiBAUB20}~\cite{RoshanaeiBAUB20}, \href{../works/LunardiBLRV20.pdf}{LunardiBLRV20}~\cite{LunardiBLRV20}, \href{../works/ThomasKS20.pdf}{ThomasKS20}~\cite{ThomasKS20}, \href{../works/Lunardi20.pdf}{Lunardi20}~\cite{Lunardi20}, \href{../works/Ham20a.pdf}{Ham20a}~\cite{Ham20a}... (Total: 65)\\
Benchmarks & github & \href{../works/Lemos21.pdf}{Lemos21}~\cite{Lemos21}, \href{../works/Godet21a.pdf}{Godet21a}~\cite{Godet21a}, \href{../works/KoehlerBFFHPSSS21.pdf}{KoehlerBFFHPSSS21}~\cite{KoehlerBFFHPSSS21} & \href{../works/FalqueALM24.pdf}{FalqueALM24}~\cite{FalqueALM24}, \href{../works/PovedaAA23.pdf}{PovedaAA23}~\cite{PovedaAA23}, \href{../works/TardivoDFMP23.pdf}{TardivoDFMP23}~\cite{TardivoDFMP23}, \href{../works/BoudreaultSLQ22.pdf}{BoudreaultSLQ22}~\cite{BoudreaultSLQ22}, \href{../works/JungblutK22.pdf}{JungblutK22}~\cite{JungblutK22}, \href{../works/HamPK21.pdf}{HamPK21}~\cite{HamPK21}, \href{../works/LunardiBLRV20.pdf}{LunardiBLRV20}~\cite{LunardiBLRV20}, \href{../works/GodetLHS20.pdf}{GodetLHS20}~\cite{GodetLHS20}, \href{../works/BenediktMH20.pdf}{BenediktMH20}~\cite{BenediktMH20}, \href{../works/Siala15.pdf}{Siala15}~\cite{Siala15}, \href{../works/Siala15a.pdf}{Siala15a}~\cite{Siala15a} & \href{../works/LiLZDZW24.pdf}{LiLZDZW24}~\cite{LiLZDZW24}, \href{../works/ForbesHJST24.pdf}{ForbesHJST24}~\cite{ForbesHJST24}, \href{../works/abs-2402-00459.pdf}{abs-2402-00459}~\cite{abs-2402-00459}, \href{../works/SquillaciPR23.pdf}{SquillaciPR23}~\cite{SquillaciPR23}, \href{../works/JuvinHHL23.pdf}{JuvinHHL23}~\cite{JuvinHHL23}, \href{../works/YuraszeckMC23.pdf}{YuraszeckMC23}~\cite{YuraszeckMC23}, \href{../works/abs-2306-05747.pdf}{abs-2306-05747}~\cite{abs-2306-05747}, \href{../works/NaderiRR23.pdf}{NaderiRR23}~\cite{NaderiRR23}, \href{../works/Adelgren2023.pdf}{Adelgren2023}~\cite{Adelgren2023}, \href{../works/TasselGS23.pdf}{TasselGS23}~\cite{TasselGS23}, \href{../works/YuraszeckMCCR23.pdf}{YuraszeckMCCR23}~\cite{YuraszeckMCCR23}, \href{../works/Fatemi-AnarakiTFV23.pdf}{Fatemi-AnarakiTFV23}~\cite{Fatemi-AnarakiTFV23}, \href{../works/GuoZ23.pdf}{GuoZ23}~\cite{GuoZ23}, \href{../works/GokPTGO23.pdf}{GokPTGO23}~\cite{GokPTGO23}, \href{../works/Bit-Monnot23.pdf}{Bit-Monnot23}~\cite{Bit-Monnot23}, \href{../works/OuelletQ22.pdf}{OuelletQ22}~\cite{OuelletQ22}, \href{../works/EmdeZD22.pdf}{EmdeZD22}~\cite{EmdeZD22}, \href{../works/GeitzGSSW22.pdf}{GeitzGSSW22}~\cite{GeitzGSSW22}, \href{../works/KotaryFH22.pdf}{KotaryFH22}~\cite{KotaryFH22}, \href{../works/ColT22.pdf}{ColT22}~\cite{ColT22}, \href{../works/MullerMKP22.pdf}{MullerMKP22}~\cite{MullerMKP22}, \href{../works/LuoB22.pdf}{LuoB22}~\cite{LuoB22}, \href{../works/YuraszeckMPV22.pdf}{YuraszeckMPV22}~\cite{YuraszeckMPV22}, \href{../works/KovacsTKSG21.pdf}{KovacsTKSG21}~\cite{KovacsTKSG21}, \href{../works/GeibingerMM21.pdf}{GeibingerMM21}~\cite{GeibingerMM21}, \href{../works/VlkHT21.pdf}{VlkHT21}~\cite{VlkHT21}, \href{../works/AbohashimaEG21.pdf}{AbohashimaEG21}~\cite{AbohashimaEG21}, \href{../works/HamP21.pdf}{HamP21}~\cite{HamP21}, \href{../works/Polo-MejiaALB20.pdf}{Polo-MejiaALB20}~\cite{Polo-MejiaALB20}... (Total: 52)\\
Benchmarks & gitlab &  & \href{../works/HeinzNVH22.pdf}{HeinzNVH22}~\cite{HeinzNVH22} & \href{../works/FalqueALM24.pdf}{FalqueALM24}~\cite{FalqueALM24}, \href{../works/abs-2305-19888.pdf}{abs-2305-19888}~\cite{abs-2305-19888}, \href{../works/BoudreaultSLQ22.pdf}{BoudreaultSLQ22}~\cite{BoudreaultSLQ22}, \href{../works/AntuoriHHEN21.pdf}{AntuoriHHEN21}~\cite{AntuoriHHEN21}, \href{../works/AntuoriHHEN20.pdf}{AntuoriHHEN20}~\cite{AntuoriHHEN20}\\
Benchmarks & industrial instance & \href{../works/LuoB22.pdf}{LuoB22}~\cite{LuoB22}, \href{../works/AntuoriHHEN20.pdf}{AntuoriHHEN20}~\cite{AntuoriHHEN20} & \href{../works/BonfiettiZLM16.pdf}{BonfiettiZLM16}~\cite{BonfiettiZLM16}, \href{../works/BonfiettiLBM14.pdf}{BonfiettiLBM14}~\cite{BonfiettiLBM14}, \href{../works/Schutt11.pdf}{Schutt11}~\cite{Schutt11} & \href{../works/TasselGS23.pdf}{TasselGS23}~\cite{TasselGS23}, \href{../works/PovedaAA23.pdf}{PovedaAA23}~\cite{PovedaAA23}, \href{../works/EfthymiouY23.pdf}{EfthymiouY23}~\cite{EfthymiouY23}, \href{../works/abs-2306-05747.pdf}{abs-2306-05747}~\cite{abs-2306-05747}, \href{../works/OujanaAYB22.pdf}{OujanaAYB22}~\cite{OujanaAYB22}, \href{../works/GroleazNS20.pdf}{GroleazNS20}~\cite{GroleazNS20}, \href{../works/NattafM20.pdf}{NattafM20}~\cite{NattafM20}, \href{../works/Mercier-AubinGQ20.pdf}{Mercier-AubinGQ20}~\cite{Mercier-AubinGQ20}, \href{../works/MalapertN19.pdf}{MalapertN19}~\cite{MalapertN19}, \href{../works/Hooker19.pdf}{Hooker19}~\cite{Hooker19}, \href{../works/BofillGSV15.pdf}{BofillGSV15}~\cite{BofillGSV15}, \href{../works/BofillEGPSV14.pdf}{BofillEGPSV14}~\cite{BofillEGPSV14}, \href{../works/ZampelliVSDR13.pdf}{ZampelliVSDR13}~\cite{ZampelliVSDR13}, \href{../works/BonfiettiM12.pdf}{BonfiettiM12}~\cite{BonfiettiM12}, \href{../works/LombardiBMB11.pdf}{LombardiBMB11}~\cite{LombardiBMB11}, \href{../works/BonfiettiLBM11.pdf}{BonfiettiLBM11}~\cite{BonfiettiLBM11}\\
Benchmarks & industrial partner & \href{../works/BoudreaultSLQ22.pdf}{BoudreaultSLQ22}~\cite{BoudreaultSLQ22}, \href{../works/Lunardi20.pdf}{Lunardi20}~\cite{Lunardi20}, \href{../works/Dejemeppe16.pdf}{Dejemeppe16}~\cite{Dejemeppe16} & \href{../works/LacknerMMWW23.pdf}{LacknerMMWW23}~\cite{LacknerMMWW23}, \href{../works/ArmstrongGOS21.pdf}{ArmstrongGOS21}~\cite{ArmstrongGOS21}, \href{../works/ZampelliVSDR13.pdf}{ZampelliVSDR13}~\cite{ZampelliVSDR13} & \href{../works/WinterMMW22.pdf}{WinterMMW22}~\cite{WinterMMW22}, \href{../works/VlkHT21.pdf}{VlkHT21}~\cite{VlkHT21}, \href{../works/LacknerMMWW21.pdf}{LacknerMMWW21}~\cite{LacknerMMWW21}, \href{../works/GroleazNS20a.pdf}{GroleazNS20a}~\cite{GroleazNS20a}, \href{../works/Mercier-AubinGQ20.pdf}{Mercier-AubinGQ20}~\cite{Mercier-AubinGQ20}, \href{../works/AntunesABD20.pdf}{AntunesABD20}~\cite{AntunesABD20}, \href{../works/abs-1911-04766.pdf}{abs-1911-04766}~\cite{abs-1911-04766}, \href{../works/GeibingerMM19.pdf}{GeibingerMM19}~\cite{GeibingerMM19}, \href{../works/AntunesABD18.pdf}{AntunesABD18}~\cite{AntunesABD18}, \href{../works/MossigeGSMC17.pdf}{MossigeGSMC17}~\cite{MossigeGSMC17}, \href{../works/Froger16.pdf}{Froger16}~\cite{Froger16}, \href{../works/HebrardHJMPV16.pdf}{HebrardHJMPV16}~\cite{HebrardHJMPV16}, \href{../works/AlesioBNG15.pdf}{AlesioBNG15}~\cite{AlesioBNG15}, \href{../works/LipovetzkyBPS14.pdf}{LipovetzkyBPS14}~\cite{LipovetzkyBPS14}, \href{../works/LimtanyakulS12.pdf}{LimtanyakulS12}~\cite{LimtanyakulS12}, \href{../works/Malapert11.pdf}{Malapert11}~\cite{Malapert11}, \href{../works/DoRZ08.pdf}{DoRZ08}~\cite{DoRZ08}, \href{../works/KovacsV06.pdf}{KovacsV06}~\cite{KovacsV06}, \href{../works/KovacsV04.pdf}{KovacsV04}~\cite{KovacsV04}, \href{../works/EreminW01.pdf}{EreminW01}~\cite{EreminW01}\\
Benchmarks & industry partner & \href{../works/BurtLPS15.pdf}{BurtLPS15}~\cite{BurtLPS15}, \href{../works/LipovetzkyBPS14.pdf}{LipovetzkyBPS14}~\cite{LipovetzkyBPS14} & \href{../works/BlomBPS14.pdf}{BlomBPS14}~\cite{BlomBPS14} & \href{../works/LuoB22.pdf}{LuoB22}~\cite{LuoB22}, \href{../works/WinterMMW22.pdf}{WinterMMW22}~\cite{WinterMMW22}, \href{../works/ArmstrongGOS21.pdf}{ArmstrongGOS21}~\cite{ArmstrongGOS21}, \href{../works/HauderBRPA20.pdf}{HauderBRPA20}~\cite{HauderBRPA20}, \href{../works/abs-1902-09244.pdf}{abs-1902-09244}~\cite{abs-1902-09244}, \href{../works/AntunesABD18.pdf}{AntunesABD18}~\cite{AntunesABD18}, \href{../works/BlomPS16.pdf}{BlomPS16}~\cite{BlomPS16}\\
Benchmarks & instance generator & \href{../works/LacknerMMWW23.pdf}{LacknerMMWW23}~\cite{LacknerMMWW23}, \href{../works/LacknerMMWW21.pdf}{LacknerMMWW21}~\cite{LacknerMMWW21} & \href{../works/GoldwaserS18.pdf}{GoldwaserS18}~\cite{GoldwaserS18}, \href{../works/Froger16.pdf}{Froger16}~\cite{Froger16} & \href{../works/abs-2402-00459.pdf}{abs-2402-00459}~\cite{abs-2402-00459}, \href{../works/ArmstrongGOS21.pdf}{ArmstrongGOS21}~\cite{ArmstrongGOS21}, \href{../works/Lunardi20.pdf}{Lunardi20}~\cite{Lunardi20}, \href{../works/KletzanderM20.pdf}{KletzanderM20}~\cite{KletzanderM20}, \href{../works/SunTB19.pdf}{SunTB19}~\cite{SunTB19}, \href{../works/abs-1911-04766.pdf}{abs-1911-04766}~\cite{abs-1911-04766}, \href{../works/Caballero19.pdf}{Caballero19}~\cite{Caballero19}, \href{../works/GombolayWS18.pdf}{GombolayWS18}~\cite{GombolayWS18}, \href{../works/GoldwaserS17.pdf}{GoldwaserS17}~\cite{GoldwaserS17}, \href{../works/YoungFS17.pdf}{YoungFS17}~\cite{YoungFS17}, \href{../works/Dejemeppe16.pdf}{Dejemeppe16}~\cite{Dejemeppe16}, \href{../works/UnsalO13.pdf}{UnsalO13}~\cite{UnsalO13}, \href{../works/GuyonLPR12.pdf}{GuyonLPR12}~\cite{GuyonLPR12}, \href{../works/Schutt11.pdf}{Schutt11}~\cite{Schutt11}, \href{../works/BeniniLMR11.pdf}{BeniniLMR11}~\cite{BeniniLMR11}, \href{../works/Lombardi10.pdf}{Lombardi10}~\cite{Lombardi10}, \href{../works/abs-1009-0347.pdf}{abs-1009-0347}~\cite{abs-1009-0347}, \href{../works/RuggieroBBMA09.pdf}{RuggieroBBMA09}~\cite{RuggieroBBMA09}, \href{../works/LombardiM09.pdf}{LombardiM09}~\cite{LombardiM09}, \href{../works/HeipckeCCS00.pdf}{HeipckeCCS00}~\cite{HeipckeCCS00}\\
Benchmarks & random instance & \href{../works/LacknerMMWW21.pdf}{LacknerMMWW21}~\cite{LacknerMMWW21}, \href{../works/WallaceY20.pdf}{WallaceY20}~\cite{WallaceY20}, \href{../works/Dejemeppe16.pdf}{Dejemeppe16}~\cite{Dejemeppe16} & \href{../works/WangB23.pdf}{WangB23}~\cite{WangB23}, \href{../works/LacknerMMWW23.pdf}{LacknerMMWW23}~\cite{LacknerMMWW23}, \href{../works/EfthymiouY23.pdf}{EfthymiouY23}~\cite{EfthymiouY23}, \href{../works/LetortCB15.pdf}{LetortCB15}~\cite{LetortCB15}, \href{../works/KelbelH11.pdf}{KelbelH11}~\cite{KelbelH11} & \href{../works/Mehdizadeh-Somarin23.pdf}{Mehdizadeh-Somarin23}~\cite{Mehdizadeh-Somarin23}, \href{../works/Fatemi-AnarakiTFV23.pdf}{Fatemi-AnarakiTFV23}~\cite{Fatemi-AnarakiTFV23}, \href{../works/OuelletQ22.pdf}{OuelletQ22}~\cite{OuelletQ22}, \href{../works/ElciOH22.pdf}{ElciOH22}~\cite{ElciOH22}, \href{../works/MullerMKP22.pdf}{MullerMKP22}~\cite{MullerMKP22}, \href{../works/EmdeZD22.pdf}{EmdeZD22}~\cite{EmdeZD22}, \href{../works/abs-2211-14492.pdf}{abs-2211-14492}~\cite{abs-2211-14492}, \href{../works/VlkHT21.pdf}{VlkHT21}~\cite{VlkHT21}, \href{../works/Godet21a.pdf}{Godet21a}~\cite{Godet21a}, \href{../works/KlankeBYE21.pdf}{KlankeBYE21}~\cite{KlankeBYE21}, \href{../works/HanenKP21.pdf}{HanenKP21}~\cite{HanenKP21}, \href{../works/AntuoriHHEN20.pdf}{AntuoriHHEN20}~\cite{AntuoriHHEN20}, \href{../works/Lunardi20.pdf}{Lunardi20}~\cite{Lunardi20}, \href{../works/BenediktMH20.pdf}{BenediktMH20}~\cite{BenediktMH20}, \href{../works/LunardiBLRV20.pdf}{LunardiBLRV20}~\cite{LunardiBLRV20}, \href{../works/HoundjiSW19.pdf}{HoundjiSW19}~\cite{HoundjiSW19}, \href{../works/UnsalO19.pdf}{UnsalO19}~\cite{UnsalO19}, \href{../works/FahimiOQ18.pdf}{FahimiOQ18}~\cite{FahimiOQ18}, \href{../works/BenediktSMVH18.pdf}{BenediktSMVH18}~\cite{BenediktSMVH18}, \href{../works/Hooker17.pdf}{Hooker17}~\cite{Hooker17}, \href{../works/MossigeGSMC17.pdf}{MossigeGSMC17}~\cite{MossigeGSMC17}, \href{../works/CappartS17.pdf}{CappartS17}~\cite{CappartS17}, \href{../works/Fahimi16.pdf}{Fahimi16}~\cite{Fahimi16}, \href{../works/Madi-WambaB16.pdf}{Madi-WambaB16}~\cite{Madi-WambaB16}, \href{../works/BoothTNB16.pdf}{BoothTNB16}~\cite{BoothTNB16}, \href{../works/Siala15.pdf}{Siala15}~\cite{Siala15}, \href{../works/Siala15a.pdf}{Siala15a}~\cite{Siala15a}, \href{../works/KameugneFSN14.pdf}{KameugneFSN14}~\cite{KameugneFSN14}, \href{../works/DerrienP14.pdf}{DerrienP14}~\cite{DerrienP14}... (Total: 44)\\
Benchmarks & real-life & \href{../works/GurPAE23.pdf}{GurPAE23}~\cite{GurPAE23}, \href{../works/SubulanC22.pdf}{SubulanC22}~\cite{SubulanC22}, \href{../works/WinterMMW22.pdf}{WinterMMW22}~\cite{WinterMMW22}, \href{../works/HubnerGSV21.pdf}{HubnerGSV21}~\cite{HubnerGSV21}, \href{../works/Astrand21.pdf}{Astrand21}~\cite{Astrand21}, \href{../works/QinDCS20.pdf}{QinDCS20}~\cite{QinDCS20}, \href{../works/KletzanderM20.pdf}{KletzanderM20}~\cite{KletzanderM20}, \href{../works/GurEA19.pdf}{GurEA19}~\cite{GurEA19}, \href{../works/RiiseML16.pdf}{RiiseML16}~\cite{RiiseML16}, \href{../works/WangMD15.pdf}{WangMD15}~\cite{WangMD15}, \href{../works/BartakCS10.pdf}{BartakCS10}~\cite{BartakCS10}, \href{../works/ChenGPSH10.pdf}{ChenGPSH10}~\cite{ChenGPSH10}, \href{../works/BartakSR10.pdf}{BartakSR10}~\cite{BartakSR10}, \href{../works/Baptiste02.pdf}{Baptiste02}~\cite{Baptiste02}, \href{../works/Bartak02a.pdf}{Bartak02a}~\cite{Bartak02a}, \href{../works/MartinPY01.pdf}{MartinPY01}~\cite{MartinPY01} & \href{../works/LuZZYW24.pdf}{LuZZYW24}~\cite{LuZZYW24}, \href{../works/AfsarVPG23.pdf}{AfsarVPG23}~\cite{AfsarVPG23}, \href{../works/LacknerMMWW23.pdf}{LacknerMMWW23}~\cite{LacknerMMWW23}, \href{../works/OujanaAYB22.pdf}{OujanaAYB22}~\cite{OujanaAYB22}, \href{../works/Lemos21.pdf}{Lemos21}~\cite{Lemos21}, \href{../works/KlankeBYE21.pdf}{KlankeBYE21}~\cite{KlankeBYE21}, \href{../works/Astrand0F21.pdf}{Astrand0F21}~\cite{Astrand0F21}, \href{../works/LacknerMMWW21.pdf}{LacknerMMWW21}~\cite{LacknerMMWW21}, \href{../works/KletzanderMH21.pdf}{KletzanderMH21}~\cite{KletzanderMH21}, \href{../works/FallahiAC20.pdf}{FallahiAC20}~\cite{FallahiAC20}, \href{../works/Lunardi20.pdf}{Lunardi20}~\cite{Lunardi20}, \href{../works/abs-1911-04766.pdf}{abs-1911-04766}~\cite{abs-1911-04766}, \href{../works/PourDERB18.pdf}{PourDERB18}~\cite{PourDERB18}, \href{../works/MusliuSS18.pdf}{MusliuSS18}~\cite{MusliuSS18}, \href{../works/AmadiniGM16.pdf}{AmadiniGM16}~\cite{AmadiniGM16}, \href{../works/Froger16.pdf}{Froger16}~\cite{Froger16}, \href{../works/BartakV15.pdf}{BartakV15}~\cite{BartakV15}, \href{../works/GaySS14.pdf}{GaySS14}~\cite{GaySS14}, \href{../works/LimtanyakulS12.pdf}{LimtanyakulS12}~\cite{LimtanyakulS12}, \href{../works/MenciaSV12.pdf}{MenciaSV12}~\cite{MenciaSV12}, \href{../works/LombardiMRB10.pdf}{LombardiMRB10}~\cite{LombardiMRB10}, \href{../works/RuggieroBBMA09.pdf}{RuggieroBBMA09}~\cite{RuggieroBBMA09}, \href{../works/BartakSR08.pdf}{BartakSR08}~\cite{BartakSR08}, \href{../works/Tsang03.pdf}{Tsang03}~\cite{Tsang03}, \href{../works/BosiM2001.pdf}{BosiM2001}~\cite{BosiM2001}, \href{../works/JainM99.pdf}{JainM99}~\cite{JainM99}, \href{../works/NuijtenP98.pdf}{NuijtenP98}~\cite{NuijtenP98}, \href{../works/SimonisC95.pdf}{SimonisC95}~\cite{SimonisC95}, \href{../works/DincbasSH90.pdf}{DincbasSH90}~\cite{DincbasSH90} & \href{../works/PrataAN23.pdf}{PrataAN23}~\cite{PrataAN23}, \href{../works/LiLZDZW24.pdf}{LiLZDZW24}~\cite{LiLZDZW24}, \href{../works/BonninMNE24.pdf}{BonninMNE24}~\cite{BonninMNE24}, \href{../works/ForbesHJST24.pdf}{ForbesHJST24}~\cite{ForbesHJST24}, \href{../works/Adelgren2023.pdf}{Adelgren2023}~\cite{Adelgren2023}, \href{../works/AbreuPNF23.pdf}{AbreuPNF23}~\cite{AbreuPNF23}, \href{../works/IsikYA23.pdf}{IsikYA23}~\cite{IsikYA23}, \href{../works/NaderiBZR23.pdf}{NaderiBZR23}~\cite{NaderiBZR23}, \href{../works/EfthymiouY23.pdf}{EfthymiouY23}~\cite{EfthymiouY23}, \href{../works/PovedaAA23.pdf}{PovedaAA23}~\cite{PovedaAA23}, \href{../works/LuoB22.pdf}{LuoB22}~\cite{LuoB22}, \href{../works/GeitzGSSW22.pdf}{GeitzGSSW22}~\cite{GeitzGSSW22}, \href{../works/AwadMDMT22.pdf}{AwadMDMT22}~\cite{AwadMDMT22}, \href{../works/NaderiBZ22.pdf}{NaderiBZ22}~\cite{NaderiBZ22}, \href{../works/YunusogluY22.pdf}{YunusogluY22}~\cite{YunusogluY22}, \href{../works/CampeauG22.pdf}{CampeauG22}~\cite{CampeauG22}, \href{../works/YuraszeckMPV22.pdf}{YuraszeckMPV22}~\cite{YuraszeckMPV22}, \href{../works/ColT22.pdf}{ColT22}~\cite{ColT22}, \href{../works/Teppan22.pdf}{Teppan22}~\cite{Teppan22}, \href{../works/BoudreaultSLQ22.pdf}{BoudreaultSLQ22}~\cite{BoudreaultSLQ22}, \href{../works/ElciOH22.pdf}{ElciOH22}~\cite{ElciOH22}, \href{../works/Godet21a.pdf}{Godet21a}~\cite{Godet21a}, \href{../works/Bedhief21.pdf}{Bedhief21}~\cite{Bedhief21}, \href{../works/abs-2102-08778.pdf}{abs-2102-08778}~\cite{abs-2102-08778}, \href{../works/GeibingerMM21.pdf}{GeibingerMM21}~\cite{GeibingerMM21}, \href{../works/Groleaz21.pdf}{Groleaz21}~\cite{Groleaz21}, \href{../works/CauwelaertDS20.pdf}{CauwelaertDS20}~\cite{CauwelaertDS20}, \href{../works/GodetLHS20.pdf}{GodetLHS20}~\cite{GodetLHS20}, \href{../works/SacramentoSP20.pdf}{SacramentoSP20}~\cite{SacramentoSP20}... (Total: 115)\\
Benchmarks & real-world & \href{../works/LuZZYW24.pdf}{LuZZYW24}~\cite{LuZZYW24}, \href{../works/GokPTGO23.pdf}{GokPTGO23}~\cite{GokPTGO23}, \href{../works/abs-2305-19888.pdf}{abs-2305-19888}~\cite{abs-2305-19888}, \href{../works/HeinzNVH22.pdf}{HeinzNVH22}~\cite{HeinzNVH22}, \href{../works/YunusogluY22.pdf}{YunusogluY22}~\cite{YunusogluY22}, \href{../works/ColT22.pdf}{ColT22}~\cite{ColT22}, \href{../works/GeibingerMM21.pdf}{GeibingerMM21}~\cite{GeibingerMM21}, \href{../works/KoehlerBFFHPSSS21.pdf}{KoehlerBFFHPSSS21}~\cite{KoehlerBFFHPSSS21}, \href{../works/Lemos21.pdf}{Lemos21}~\cite{Lemos21}, \href{../works/Astrand21.pdf}{Astrand21}~\cite{Astrand21}, \href{../works/Lunardi20.pdf}{Lunardi20}~\cite{Lunardi20}, \href{../works/MokhtarzadehTNF20.pdf}{MokhtarzadehTNF20}~\cite{MokhtarzadehTNF20}, \href{../works/HauderBRPA20.pdf}{HauderBRPA20}~\cite{HauderBRPA20}, \href{../works/abs-1911-04766.pdf}{abs-1911-04766}~\cite{abs-1911-04766}, \href{../works/GeibingerMM19.pdf}{GeibingerMM19}~\cite{GeibingerMM19}, \href{../works/abs-1902-09244.pdf}{abs-1902-09244}~\cite{abs-1902-09244}, \href{../works/FrohnerTR19.pdf}{FrohnerTR19}~\cite{FrohnerTR19}, \href{../works/SenderovichBB19.pdf}{SenderovichBB19}~\cite{SenderovichBB19}, \href{../works/GombolayWS18.pdf}{GombolayWS18}~\cite{GombolayWS18}, \href{../works/AgussurjaKL18.pdf}{AgussurjaKL18}~\cite{AgussurjaKL18}, \href{../works/Dejemeppe16.pdf}{Dejemeppe16}~\cite{Dejemeppe16}, \href{../works/MelgarejoLS15.pdf}{MelgarejoLS15}~\cite{MelgarejoLS15}, \href{../works/EvenSH15a.pdf}{EvenSH15a}~\cite{EvenSH15a}, \href{../works/EvenSH15.pdf}{EvenSH15}~\cite{EvenSH15}, \href{../works/UnsalO13.pdf}{UnsalO13}~\cite{UnsalO13}, \href{../works/RendlPHPR12.pdf}{RendlPHPR12}~\cite{RendlPHPR12}, \href{../works/Lombardi10.pdf}{Lombardi10}~\cite{Lombardi10}, \href{../works/MouraSCL08a.pdf}{MouraSCL08a}~\cite{MouraSCL08a}, \href{../works/WatsonBHW99.pdf}{WatsonBHW99}~\cite{WatsonBHW99}... (Total: 32) & \href{../works/PrataAN23.pdf}{PrataAN23}~\cite{PrataAN23}, \href{../works/TasselGS23.pdf}{TasselGS23}~\cite{TasselGS23}, \href{../works/abs-2306-05747.pdf}{abs-2306-05747}~\cite{abs-2306-05747}, \href{../works/AbreuNP23.pdf}{AbreuNP23}~\cite{AbreuNP23}, \href{../works/IsikYA23.pdf}{IsikYA23}~\cite{IsikYA23}, \href{../works/Fatemi-AnarakiTFV23.pdf}{Fatemi-AnarakiTFV23}~\cite{Fatemi-AnarakiTFV23}, \href{../works/AalianPG23.pdf}{AalianPG23}~\cite{AalianPG23}, \href{../works/AbreuPNF23.pdf}{AbreuPNF23}~\cite{AbreuPNF23}, \href{../works/WangB23.pdf}{WangB23}~\cite{WangB23}, \href{../works/YuraszeckMCCR23.pdf}{YuraszeckMCCR23}~\cite{YuraszeckMCCR23}, \href{../works/OujanaAYB22.pdf}{OujanaAYB22}~\cite{OujanaAYB22}, \href{../works/LuoB22.pdf}{LuoB22}~\cite{LuoB22}, \href{../works/SvancaraB22.pdf}{SvancaraB22}~\cite{SvancaraB22}, \href{../works/MullerMKP22.pdf}{MullerMKP22}~\cite{MullerMKP22}, \href{../works/ArmstrongGOS21.pdf}{ArmstrongGOS21}~\cite{ArmstrongGOS21}, \href{../works/ZarandiASC20.pdf}{ZarandiASC20}~\cite{ZarandiASC20}, \href{../works/WallaceY20.pdf}{WallaceY20}~\cite{WallaceY20}, \href{../works/AntunesABD20.pdf}{AntunesABD20}~\cite{AntunesABD20}, \href{../works/RoshanaeiBAUB20.pdf}{RoshanaeiBAUB20}~\cite{RoshanaeiBAUB20}, \href{../works/AstrandJZ20.pdf}{AstrandJZ20}~\cite{AstrandJZ20}, \href{../works/WessenCS20.pdf}{WessenCS20}~\cite{WessenCS20}, \href{../works/TangB20.pdf}{TangB20}~\cite{TangB20}, \href{../works/ParkUJR19.pdf}{ParkUJR19}~\cite{ParkUJR19}, \href{../works/YounespourAKE19.pdf}{YounespourAKE19}~\cite{YounespourAKE19}, \href{../works/FrimodigS19.pdf}{FrimodigS19}~\cite{FrimodigS19}, \href{../works/PourDERB18.pdf}{PourDERB18}~\cite{PourDERB18}, \href{../works/HoYCLLCLC18.pdf}{HoYCLLCLC18}~\cite{HoYCLLCLC18}, \href{../works/LaborieRSV18.pdf}{LaborieRSV18}~\cite{LaborieRSV18}, \href{../works/ShinBBHO18.pdf}{ShinBBHO18}~\cite{ShinBBHO18}... (Total: 54) & \href{../works/FalqueALM24.pdf}{FalqueALM24}~\cite{FalqueALM24}, \href{../works/abs-2402-00459.pdf}{abs-2402-00459}~\cite{abs-2402-00459}, \href{../works/ZhuSZW23.pdf}{ZhuSZW23}~\cite{ZhuSZW23}, \href{../works/GuoZ23.pdf}{GuoZ23}~\cite{GuoZ23}, \href{../works/PovedaAA23.pdf}{PovedaAA23}~\cite{PovedaAA23}, \href{../works/Bit-Monnot23.pdf}{Bit-Monnot23}~\cite{Bit-Monnot23}, \href{../works/TardivoDFMP23.pdf}{TardivoDFMP23}~\cite{TardivoDFMP23}, \href{../works/CzerniachowskaWZ23.pdf}{CzerniachowskaWZ23}~\cite{CzerniachowskaWZ23}, \href{../works/abs-2312-13682.pdf}{abs-2312-13682}~\cite{abs-2312-13682}, \href{../works/KimCMLLP23.pdf}{KimCMLLP23}~\cite{KimCMLLP23}, \href{../works/NaderiBZR23.pdf}{NaderiBZR23}~\cite{NaderiBZR23}, \href{../works/JuvinHL23.pdf}{JuvinHL23}~\cite{JuvinHL23}, \href{../works/PerezGSL23.pdf}{PerezGSL23}~\cite{PerezGSL23}, \href{../works/NaderiBZ23.pdf}{NaderiBZ23}~\cite{NaderiBZ23}, \href{../works/ShaikhK23.pdf}{ShaikhK23}~\cite{ShaikhK23}, \href{../works/AfsarVPG23.pdf}{AfsarVPG23}~\cite{AfsarVPG23}, \href{../works/MarliereSPR23.pdf}{MarliereSPR23}~\cite{MarliereSPR23}, \href{../works/GeitzGSSW22.pdf}{GeitzGSSW22}~\cite{GeitzGSSW22}, \href{../works/SubulanC22.pdf}{SubulanC22}~\cite{SubulanC22}, \href{../works/BourreauGGLT22.pdf}{BourreauGGLT22}~\cite{BourreauGGLT22}, \href{../works/JungblutK22.pdf}{JungblutK22}~\cite{JungblutK22}, \href{../works/AbreuN22.pdf}{AbreuN22}~\cite{AbreuN22}, \href{../works/FetgoD22.pdf}{FetgoD22}~\cite{FetgoD22}, \href{../works/BoudreaultSLQ22.pdf}{BoudreaultSLQ22}~\cite{BoudreaultSLQ22}, \href{../works/OrnekOS20.pdf}{OrnekOS20}~\cite{OrnekOS20}, \href{../works/CampeauG22.pdf}{CampeauG22}~\cite{CampeauG22}, \href{../works/EtminaniesfahaniGNMS22.pdf}{EtminaniesfahaniGNMS22}~\cite{EtminaniesfahaniGNMS22}, \href{../works/ArmstrongGOS22.pdf}{ArmstrongGOS22}~\cite{ArmstrongGOS22}, \href{../works/PohlAK22.pdf}{PohlAK22}~\cite{PohlAK22}... (Total: 161)\\
Benchmarks & supplementary material & \href{../works/GuoZ23.pdf}{GuoZ23}~\cite{GuoZ23}, \href{../works/FarsiTM22.pdf}{FarsiTM22}~\cite{FarsiTM22}, \href{../works/MejiaY20.pdf}{MejiaY20}~\cite{MejiaY20}, \href{../works/Lunardi20.pdf}{Lunardi20}~\cite{Lunardi20}, \href{../works/TanZWGQ19.pdf}{TanZWGQ19}~\cite{TanZWGQ19} & \href{../works/NaderiBZR23.pdf}{NaderiBZR23}~\cite{NaderiBZR23}, \href{../works/MontemanniD23.pdf}{MontemanniD23}~\cite{MontemanniD23}, \href{../works/AfsarVPG23.pdf}{AfsarVPG23}~\cite{AfsarVPG23}, \href{../works/FachiniA20.pdf}{FachiniA20}~\cite{FachiniA20}, \href{../works/SchuttFSW13.pdf}{SchuttFSW13}~\cite{SchuttFSW13} & \href{../works/FalqueALM24.pdf}{FalqueALM24}~\cite{FalqueALM24}, \href{../works/JuvinHHL23.pdf}{JuvinHHL23}~\cite{JuvinHHL23}, \href{../works/abs-2306-05747.pdf}{abs-2306-05747}~\cite{abs-2306-05747}, \href{../works/TasselGS23.pdf}{TasselGS23}~\cite{TasselGS23}, \href{../works/Adelgren2023.pdf}{Adelgren2023}~\cite{Adelgren2023}, \href{../works/WinterMMW22.pdf}{WinterMMW22}~\cite{WinterMMW22}, \href{../works/ColT22.pdf}{ColT22}~\cite{ColT22}, \href{../works/BoudreaultSLQ22.pdf}{BoudreaultSLQ22}~\cite{BoudreaultSLQ22}, \href{../works/YunusogluY22.pdf}{YunusogluY22}~\cite{YunusogluY22}, \href{../works/AntuoriHHEN21.pdf}{AntuoriHHEN21}~\cite{AntuoriHHEN21}, \href{../works/LacknerMMWW21.pdf}{LacknerMMWW21}~\cite{LacknerMMWW21}, \href{../works/KovacsTKSG21.pdf}{KovacsTKSG21}~\cite{KovacsTKSG21}, \href{../works/ArmstrongGOS21.pdf}{ArmstrongGOS21}~\cite{ArmstrongGOS21}, \href{../works/MengZRZL20.pdf}{MengZRZL20}~\cite{MengZRZL20}, \href{../works/HauderBRPA20.pdf}{HauderBRPA20}~\cite{HauderBRPA20}, \href{../works/SchnellH17.pdf}{SchnellH17}~\cite{SchnellH17}, \href{../works/SchnellH15.pdf}{SchnellH15}~\cite{SchnellH15}, \href{../works/MenciaSV13.pdf}{MenciaSV13}~\cite{MenciaSV13}\\
Benchmarks & zenodo & \href{../works/LacknerMMWW23.pdf}{LacknerMMWW23}~\cite{LacknerMMWW23}, \href{../works/SacramentoSP20.pdf}{SacramentoSP20}~\cite{SacramentoSP20} &  & \href{../works/KimCMLLP23.pdf}{KimCMLLP23}~\cite{KimCMLLP23}, \href{../works/WinterMMW22.pdf}{WinterMMW22}~\cite{WinterMMW22}, \href{../works/ArmstrongGOS21.pdf}{ArmstrongGOS21}~\cite{ArmstrongGOS21}\\
\end{longtable}
}


\clearpage
\subsection{Concept Type Algorithms}
\label{sec:Algorithms}
{\scriptsize
\begin{longtable}{lp{3cm}>{\raggedright\arraybackslash}p{6cm}>{\raggedright\arraybackslash}p{6cm}>{\raggedright\arraybackslash}p{8cm}}
\rowcolor{white}\caption{Works for Concepts of Type Algorithms}\\ \toprule
\rowcolor{white}Type & Keyword & High & Medium & Low\\ \midrule\endhead
\bottomrule
\endfoot
Algorithms & GRASP & \href{../works/Lemos21.pdf}{Lemos21}~\cite{Lemos21} & \href{../works/YuraszeckMCCR23.pdf}{YuraszeckMCCR23}~\cite{YuraszeckMCCR23}, \href{../works/PovedaAA23.pdf}{PovedaAA23}~\cite{PovedaAA23}, \href{../works/YunusogluY22.pdf}{YunusogluY22}~\cite{YunusogluY22}, \href{../works/RiahiNS018.pdf}{RiahiNS018}~\cite{RiahiNS018} & \href{../works/LacknerMMWW23.pdf}{LacknerMMWW23}~\cite{LacknerMMWW23}, \href{../works/AkramNHRSA23.pdf}{AkramNHRSA23}~\cite{AkramNHRSA23}, \href{../works/IsikYA23.pdf}{IsikYA23}~\cite{IsikYA23}, \href{../works/SquillaciPR23.pdf}{SquillaciPR23}~\cite{SquillaciPR23}, \href{../works/ArmstrongGOS22.pdf}{ArmstrongGOS22}~\cite{ArmstrongGOS22}, \href{../works/LacknerMMWW21.pdf}{LacknerMMWW21}~\cite{LacknerMMWW21}, \href{../works/Zahout21.pdf}{Zahout21}~\cite{Zahout21}, \href{../works/VlkHT21.pdf}{VlkHT21}~\cite{VlkHT21}, \href{../works/AntuoriHHEN21.pdf}{AntuoriHHEN21}~\cite{AntuoriHHEN21}, \href{../works/GokGSTO20.pdf}{GokGSTO20}~\cite{GokGSTO20}, \href{../works/QinDCS20.pdf}{QinDCS20}~\cite{QinDCS20}, \href{../works/MejiaY20.pdf}{MejiaY20}~\cite{MejiaY20}, \href{../works/GroleazNS20a.pdf}{GroleazNS20a}~\cite{GroleazNS20a}, \href{../works/Caballero19.pdf}{Caballero19}~\cite{Caballero19}, \href{../works/KreterSSZ18.pdf}{KreterSSZ18}~\cite{KreterSSZ18}, \href{../works/ZhouGL15.pdf}{ZhouGL15}~\cite{ZhouGL15}, \href{../works/Siala15.pdf}{Siala15}~\cite{Siala15}, \href{../works/Siala15a.pdf}{Siala15a}~\cite{Siala15a}, \href{../works/SchnellH15.pdf}{SchnellH15}~\cite{SchnellH15}, \href{../works/SerraNM12.pdf}{SerraNM12}~\cite{SerraNM12}, \href{../works/HeinzB12.pdf}{HeinzB12}~\cite{HeinzB12}, \href{../works/Rodriguez07.pdf}{Rodriguez07}~\cite{Rodriguez07}, \href{../works/JainM99.pdf}{JainM99}~\cite{JainM99}\\
Algorithms & IGT & \href{../works/ArmstrongGOS22.pdf}{ArmstrongGOS22}~\cite{ArmstrongGOS22} &  & \\
Algorithms & NEH & \href{../works/AlfieriGPS23.pdf}{AlfieriGPS23}~\cite{AlfieriGPS23}, \href{../works/ArmstrongGOS22.pdf}{ArmstrongGOS22}~\cite{ArmstrongGOS22}, \href{../works/Astrand21.pdf}{Astrand21}~\cite{Astrand21}, \href{../works/RiahiNS018.pdf}{RiahiNS018}~\cite{RiahiNS018} &  & \href{../works/AbreuPNF23.pdf}{AbreuPNF23}~\cite{AbreuPNF23}, \href{../works/IsikYA23.pdf}{IsikYA23}~\cite{IsikYA23}, \href{../works/ZhouGL15.pdf}{ZhouGL15}~\cite{ZhouGL15}\\
Algorithms & bi-partite matching &  &  & \href{../works/Caballero19.pdf}{Caballero19}~\cite{Caballero19}, \href{../works/HookerH17.pdf}{HookerH17}~\cite{HookerH17}, \href{../works/Simonis07.pdf}{Simonis07}~\cite{Simonis07}, \href{../works/Kumar03.pdf}{Kumar03}~\cite{Kumar03}, \href{../works/Simonis99.pdf}{Simonis99}~\cite{Simonis99}\\
Algorithms & edge-finder & \href{../works/KameugneFND23.pdf}{KameugneFND23}~\cite{KameugneFND23}, \href{../works/FetgoD22.pdf}{FetgoD22}~\cite{FetgoD22}, \href{../works/GingrasQ16.pdf}{GingrasQ16}~\cite{GingrasQ16}, \href{../works/KameugneFSN14.pdf}{KameugneFSN14}~\cite{KameugneFSN14}, \href{../works/Lombardi10.pdf}{Lombardi10}~\cite{Lombardi10}, \href{../works/MercierH08.pdf}{MercierH08}~\cite{MercierH08}, \href{../works/BaptisteP00.pdf}{BaptisteP00}~\cite{BaptisteP00} & \href{../works/OuelletQ13.pdf}{OuelletQ13}~\cite{OuelletQ13}, \href{../works/KelbelH11.pdf}{KelbelH11}~\cite{KelbelH11}, \href{../works/PapaB98.pdf}{PapaB98}~\cite{PapaB98} & \href{../works/BaptisteB18.pdf}{BaptisteB18}~\cite{BaptisteB18}, \href{../works/BonfiettiZLM16.pdf}{BonfiettiZLM16}~\cite{BonfiettiZLM16}, \href{../works/Kameugne14.pdf}{Kameugne14}~\cite{Kameugne14}, \href{../works/GuSS13.pdf}{GuSS13}~\cite{GuSS13}, \href{../works/Schutt11.pdf}{Schutt11}~\cite{Schutt11}, \href{../works/SchuttFSW11.pdf}{SchuttFSW11}~\cite{SchuttFSW11}, \href{../works/HeckmanB11.pdf}{HeckmanB11}~\cite{HeckmanB11}, \href{../works/BidotVLB09.pdf}{BidotVLB09}~\cite{BidotVLB09}, \href{../works/MilanoW09.pdf}{MilanoW09}~\cite{MilanoW09}, \href{../works/SchuttFSW09.pdf}{SchuttFSW09}~\cite{SchuttFSW09}, \href{../works/BeckW07.pdf}{BeckW07}~\cite{BeckW07}, \href{../works/MilanoW06.pdf}{MilanoW06}~\cite{MilanoW06}, \href{../works/BeckW05.pdf}{BeckW05}~\cite{BeckW05}, \href{../works/BeckR03.pdf}{BeckR03}~\cite{BeckR03}, \href{../works/ValleMGT03.pdf}{ValleMGT03}~\cite{ValleMGT03}, \href{../works/SakkoutW00.pdf}{SakkoutW00}~\cite{SakkoutW00}, \href{../works/JainM99.pdf}{JainM99}~\cite{JainM99}, \href{../works/Zhou97.pdf}{Zhou97}~\cite{Zhou97}, \href{../works/BaptisteP97.pdf}{BaptisteP97}~\cite{BaptisteP97}\\
Algorithms & edge-finding & \href{../works/KameugneFND23.pdf}{KameugneFND23}~\cite{KameugneFND23}, \href{../works/JuvinHHL23.pdf}{JuvinHHL23}~\cite{JuvinHHL23}, \href{../works/TardivoDFMP23.pdf}{TardivoDFMP23}~\cite{TardivoDFMP23}, \href{../works/OuelletQ22.pdf}{OuelletQ22}~\cite{OuelletQ22}, \href{../works/FetgoD22.pdf}{FetgoD22}~\cite{FetgoD22}, \href{../works/CauwelaertDS20.pdf}{CauwelaertDS20}~\cite{CauwelaertDS20}, \href{../works/YangSS19.pdf}{YangSS19}~\cite{YangSS19}, \href{../works/Caballero19.pdf}{Caballero19}~\cite{Caballero19}, \href{../works/GokgurHO18.pdf}{GokgurHO18}~\cite{GokgurHO18}, \href{../works/FahimiOQ18.pdf}{FahimiOQ18}~\cite{FahimiOQ18}, \href{../works/BaptisteB18.pdf}{BaptisteB18}~\cite{BaptisteB18}, \href{../works/KreterSS17.pdf}{KreterSS17}~\cite{KreterSS17}, \href{../works/HookerH17.pdf}{HookerH17}~\cite{HookerH17}, \href{../works/Fahimi16.pdf}{Fahimi16}~\cite{Fahimi16}, \href{../works/Nattaf16.pdf}{Nattaf16}~\cite{Nattaf16}, \href{../works/Dejemeppe16.pdf}{Dejemeppe16}~\cite{Dejemeppe16}, \href{../works/Derrien15.pdf}{Derrien15}~\cite{Derrien15}, \href{../works/GayHS15a.pdf}{GayHS15a}~\cite{GayHS15a}, \href{../works/Kameugne15.pdf}{Kameugne15}~\cite{Kameugne15}, \href{../works/GrimesH15.pdf}{GrimesH15}~\cite{GrimesH15}, \href{../works/KameugneFSN14.pdf}{KameugneFSN14}~\cite{KameugneFSN14}, \href{../works/Kameugne14.pdf}{Kameugne14}~\cite{Kameugne14}, \href{../works/Letort13.pdf}{Letort13}~\cite{Letort13}, \href{../works/OuelletQ13.pdf}{OuelletQ13}~\cite{OuelletQ13}, \href{../works/SchuttFS13a.pdf}{SchuttFS13a}~\cite{SchuttFS13a}, \href{../works/Clercq12.pdf}{Clercq12}~\cite{Clercq12}, \href{../works/Malapert11.pdf}{Malapert11}~\cite{Malapert11}, \href{../works/KameugneFSN11.pdf}{KameugneFSN11}~\cite{KameugneFSN11}, \href{../works/Vilim11.pdf}{Vilim11}~\cite{Vilim11}... (Total: 50) & \href{../works/BoudreaultSLQ22.pdf}{BoudreaultSLQ22}~\cite{BoudreaultSLQ22}, \href{../works/LaborieRSV18.pdf}{LaborieRSV18}~\cite{LaborieRSV18}, \href{../works/Tesch18.pdf}{Tesch18}~\cite{Tesch18}, \href{../works/GingrasQ16.pdf}{GingrasQ16}~\cite{GingrasQ16}, \href{../works/CauwelaertDMS16.pdf}{CauwelaertDMS16}~\cite{CauwelaertDMS16}, \href{../works/LetortCB15.pdf}{LetortCB15}~\cite{LetortCB15}, \href{../works/DejemeppeCS15.pdf}{DejemeppeCS15}~\cite{DejemeppeCS15}, \href{../works/Siala15a.pdf}{Siala15a}~\cite{Siala15a}, \href{../works/Siala15.pdf}{Siala15}~\cite{Siala15}, \href{../works/MenciaSV13.pdf}{MenciaSV13}~\cite{MenciaSV13}, \href{../works/LetortCB13.pdf}{LetortCB13}~\cite{LetortCB13}, \href{../works/LetortBC12.pdf}{LetortBC12}~\cite{LetortBC12}, \href{../works/LombardiM12.pdf}{LombardiM12}~\cite{LombardiM12}, \href{../works/Lombardi10.pdf}{Lombardi10}~\cite{Lombardi10}, \href{../works/BartakSR10.pdf}{BartakSR10}~\cite{BartakSR10}, \href{../works/LiessM08.pdf}{LiessM08}~\cite{LiessM08}, \href{../works/HoeveGSL07.pdf}{HoeveGSL07}~\cite{HoeveGSL07}, \href{../works/MonetteDD07.pdf}{MonetteDD07}~\cite{MonetteDD07}, \href{../works/Vilim04.pdf}{Vilim04}~\cite{Vilim04}, \href{../works/Bartak02.pdf}{Bartak02}~\cite{Bartak02}, \href{../works/SchildW00.pdf}{SchildW00}~\cite{SchildW00}, \href{../works/Zhou97.pdf}{Zhou97}~\cite{Zhou97} & \href{../works/BonninMNE24.pdf}{BonninMNE24}~\cite{BonninMNE24}, \href{../works/CampeauG22.pdf}{CampeauG22}~\cite{CampeauG22}, \href{../works/Groleaz21.pdf}{Groleaz21}~\cite{Groleaz21}, \href{../works/Astrand21.pdf}{Astrand21}~\cite{Astrand21}, \href{../works/Godet21a.pdf}{Godet21a}~\cite{Godet21a}, \href{../works/WallaceY20.pdf}{WallaceY20}~\cite{WallaceY20}, \href{../works/OuelletQ18.pdf}{OuelletQ18}~\cite{OuelletQ18}, \href{../works/GombolayWS18.pdf}{GombolayWS18}~\cite{GombolayWS18}, \href{../works/CauwelaertLS18.pdf}{CauwelaertLS18}~\cite{CauwelaertLS18}, \href{../works/NattafAL17.pdf}{NattafAL17}~\cite{NattafAL17}, \href{../works/OrnekO16.pdf}{OrnekO16}~\cite{OrnekO16}, \href{../works/Tesch16.pdf}{Tesch16}~\cite{Tesch16}, \href{../works/SialaAH15.pdf}{SialaAH15}~\cite{SialaAH15}, \href{../works/GayHLS15.pdf}{GayHLS15}~\cite{GayHLS15}, \href{../works/DerrienP14.pdf}{DerrienP14}~\cite{DerrienP14}, \href{../works/GuSS13.pdf}{GuSS13}~\cite{GuSS13}, \href{../works/HeinzSB13.pdf}{HeinzSB13}~\cite{HeinzSB13}, \href{../works/OzturkTHO13.pdf}{OzturkTHO13}~\cite{OzturkTHO13}, \href{../works/ChuGNSW13.pdf}{ChuGNSW13}~\cite{ChuGNSW13}, \href{../works/MenciaSV12.pdf}{MenciaSV12}~\cite{MenciaSV12}, \href{../works/LimtanyakulS12.pdf}{LimtanyakulS12}~\cite{LimtanyakulS12}, \href{../works/MalapertCGJLR12.pdf}{MalapertCGJLR12}~\cite{MalapertCGJLR12}, \href{../works/OzturkTHO12.pdf}{OzturkTHO12}~\cite{OzturkTHO12}, \href{../works/HeckmanB11.pdf}{HeckmanB11}~\cite{HeckmanB11}, \href{../works/KovacsB11.pdf}{KovacsB11}~\cite{KovacsB11}, \href{../works/SimonisH11.pdf}{SimonisH11}~\cite{SimonisH11}, \href{../works/BeldiceanuCDP11.pdf}{BeldiceanuCDP11}~\cite{BeldiceanuCDP11}, \href{../works/KelbelH11.pdf}{KelbelH11}~\cite{KelbelH11}, \href{../works/GrimesH11.pdf}{GrimesH11}~\cite{GrimesH11}... (Total: 62)\\
Algorithms & energetic reasoning & \href{../works/TardivoDFMP23.pdf}{TardivoDFMP23}~\cite{TardivoDFMP23}, \href{../works/OuelletQ22.pdf}{OuelletQ22}~\cite{OuelletQ22}, \href{../works/FetgoD22.pdf}{FetgoD22}~\cite{FetgoD22}, \href{../works/HanenKP21.pdf}{HanenKP21}~\cite{HanenKP21}, \href{../works/OuelletQ18.pdf}{OuelletQ18}~\cite{OuelletQ18}, \href{../works/Tesch18.pdf}{Tesch18}~\cite{Tesch18}, \href{../works/CauwelaertLS18.pdf}{CauwelaertLS18}~\cite{CauwelaertLS18}, \href{../works/NattafAL17.pdf}{NattafAL17}~\cite{NattafAL17}, \href{../works/NattafALR16.pdf}{NattafALR16}~\cite{NattafALR16}, \href{../works/Fahimi16.pdf}{Fahimi16}~\cite{Fahimi16}, \href{../works/Tesch16.pdf}{Tesch16}~\cite{Tesch16}, \href{../works/GayHS15a.pdf}{GayHS15a}~\cite{GayHS15a}, \href{../works/NattafAL15.pdf}{NattafAL15}~\cite{NattafAL15}, \href{../works/DerrienP14.pdf}{DerrienP14}~\cite{DerrienP14}, \href{../works/SchuttFS13a.pdf}{SchuttFS13a}~\cite{SchuttFS13a}, \href{../works/LimtanyakulS12.pdf}{LimtanyakulS12}~\cite{LimtanyakulS12}, \href{../works/HeinzS11.pdf}{HeinzS11}~\cite{HeinzS11}, \href{../works/Vilim11.pdf}{Vilim11}~\cite{Vilim11}, \href{../works/Lombardi10.pdf}{Lombardi10}~\cite{Lombardi10}, \href{../works/Laborie03.pdf}{Laborie03}~\cite{Laborie03}, \href{../works/Baptiste02.pdf}{Baptiste02}~\cite{Baptiste02} & \href{../works/KameugneFND23.pdf}{KameugneFND23}~\cite{KameugneFND23}, \href{../works/NattafHKAL19.pdf}{NattafHKAL19}~\cite{NattafHKAL19}, \href{../works/KameugneFGOQ18.pdf}{KameugneFGOQ18}~\cite{KameugneFGOQ18}, \href{../works/Nattaf16.pdf}{Nattaf16}~\cite{Nattaf16}, \href{../works/Kameugne14.pdf}{Kameugne14}~\cite{Kameugne14}, \href{../works/Letort13.pdf}{Letort13}~\cite{Letort13}, \href{../works/SchuttFS13.pdf}{SchuttFS13}~\cite{SchuttFS13}, \href{../works/Schutt11.pdf}{Schutt11}~\cite{Schutt11} & \href{../works/IsikYA23.pdf}{IsikYA23}~\cite{IsikYA23}, \href{../works/BoudreaultSLQ22.pdf}{BoudreaultSLQ22}~\cite{BoudreaultSLQ22}, \href{../works/ArmstrongGOS21.pdf}{ArmstrongGOS21}~\cite{ArmstrongGOS21}, \href{../works/Caballero19.pdf}{Caballero19}~\cite{Caballero19}, \href{../works/YangSS19.pdf}{YangSS19}~\cite{YangSS19}, \href{../works/GokgurHO18.pdf}{GokgurHO18}~\cite{GokgurHO18}, \href{../works/Laborie18a.pdf}{Laborie18a}~\cite{Laborie18a}, \href{../works/BofillCSV17.pdf}{BofillCSV17}~\cite{BofillCSV17}, \href{../works/HookerH17.pdf}{HookerH17}~\cite{HookerH17}, \href{../works/GingrasQ16.pdf}{GingrasQ16}~\cite{GingrasQ16}, \href{../works/LetortCB15.pdf}{LetortCB15}~\cite{LetortCB15}, \href{../works/Derrien15.pdf}{Derrien15}~\cite{Derrien15}, \href{../works/KameugneFSN14.pdf}{KameugneFSN14}~\cite{KameugneFSN14}, \href{../works/LetortCB13.pdf}{LetortCB13}~\cite{LetortCB13}, \href{../works/OuelletQ13.pdf}{OuelletQ13}~\cite{OuelletQ13}, \href{../works/MenciaSV13.pdf}{MenciaSV13}~\cite{MenciaSV13}, \href{../works/Clercq12.pdf}{Clercq12}~\cite{Clercq12}, \href{../works/LombardiM12.pdf}{LombardiM12}~\cite{LombardiM12}, \href{../works/MenciaSV12.pdf}{MenciaSV12}~\cite{MenciaSV12}, \href{../works/GuyonLPR12.pdf}{GuyonLPR12}~\cite{GuyonLPR12}, \href{../works/LahimerLH11.pdf}{LahimerLH11}~\cite{LahimerLH11}, \href{../works/Malapert11.pdf}{Malapert11}~\cite{Malapert11}, \href{../works/ClercqPBJ11.pdf}{ClercqPBJ11}~\cite{ClercqPBJ11}, \href{../works/BeldiceanuCDP11.pdf}{BeldiceanuCDP11}~\cite{BeldiceanuCDP11}, \href{../works/ChenGPSH10.pdf}{ChenGPSH10}~\cite{ChenGPSH10}, \href{../works/abs-0907-0939.pdf}{abs-0907-0939}~\cite{abs-0907-0939}, \href{../works/Vilim09.pdf}{Vilim09}~\cite{Vilim09}, \href{../works/Vilim09a.pdf}{Vilim09a}~\cite{Vilim09a}, \href{../works/Limtanyakul07.pdf}{Limtanyakul07}~\cite{Limtanyakul07}... (Total: 35)\\
Algorithms & max-flow &  & \href{../works/LopesCSM10.pdf}{LopesCSM10}~\cite{LopesCSM10}, \href{../works/MouraSCL08.pdf}{MouraSCL08}~\cite{MouraSCL08}, \href{../works/Muscettola02.pdf}{Muscettola02}~\cite{Muscettola02} & \href{../works/FanXG21.pdf}{FanXG21}~\cite{FanXG21}, \href{../works/ZarandiASC20.pdf}{ZarandiASC20}~\cite{ZarandiASC20}, \href{../works/HoundjiSW19.pdf}{HoundjiSW19}~\cite{HoundjiSW19}, \href{../works/Fahimi16.pdf}{Fahimi16}~\cite{Fahimi16}, \href{../works/Froger16.pdf}{Froger16}~\cite{Froger16}, \href{../works/Kumar03.pdf}{Kumar03}~\cite{Kumar03}\\
Algorithms & not-first & \href{../works/KameugneFND23.pdf}{KameugneFND23}~\cite{KameugneFND23}, \href{../works/FahimiOQ18.pdf}{FahimiOQ18}~\cite{FahimiOQ18}, \href{../works/KameugneFGOQ18.pdf}{KameugneFGOQ18}~\cite{KameugneFGOQ18}, \href{../works/Fahimi16.pdf}{Fahimi16}~\cite{Fahimi16}, \href{../works/Dejemeppe16.pdf}{Dejemeppe16}~\cite{Dejemeppe16}, \href{../works/GayHS15a.pdf}{GayHS15a}~\cite{GayHS15a}, \href{../works/Kameugne14.pdf}{Kameugne14}~\cite{Kameugne14}, \href{../works/Clercq12.pdf}{Clercq12}~\cite{Clercq12}, \href{../works/Schutt11.pdf}{Schutt11}~\cite{Schutt11}, \href{../works/Malapert11.pdf}{Malapert11}~\cite{Malapert11}, \href{../works/SchuttFSW11.pdf}{SchuttFSW11}~\cite{SchuttFSW11}, \href{../works/VilimBC05.pdf}{VilimBC05}~\cite{VilimBC05}, \href{../works/ArtiouchineB05.pdf}{ArtiouchineB05}~\cite{ArtiouchineB05}, \href{../works/Demassey03.pdf}{Demassey03}~\cite{Demassey03}, \href{../works/Baptiste02.pdf}{Baptiste02}~\cite{Baptiste02}, \href{../works/Beck99.pdf}{Beck99}~\cite{Beck99} & \href{../works/TardivoDFMP23.pdf}{TardivoDFMP23}~\cite{TardivoDFMP23}, \href{../works/FetgoD22.pdf}{FetgoD22}~\cite{FetgoD22}, \href{../works/GokgurHO18.pdf}{GokgurHO18}~\cite{GokgurHO18}, \href{../works/OuelletQ18.pdf}{OuelletQ18}~\cite{OuelletQ18}, \href{../works/HookerH17.pdf}{HookerH17}~\cite{HookerH17}, \href{../works/DejemeppeCS15.pdf}{DejemeppeCS15}~\cite{DejemeppeCS15}, \href{../works/Kameugne15.pdf}{Kameugne15}~\cite{Kameugne15}, \href{../works/KameugneFSN14.pdf}{KameugneFSN14}~\cite{KameugneFSN14}, \href{../works/Letort13.pdf}{Letort13}~\cite{Letort13}, \href{../works/OuelletQ13.pdf}{OuelletQ13}~\cite{OuelletQ13}, \href{../works/Lombardi10.pdf}{Lombardi10}~\cite{Lombardi10}, \href{../works/SchuttW10.pdf}{SchuttW10}~\cite{SchuttW10}, \href{../works/BartakSR10.pdf}{BartakSR10}~\cite{BartakSR10}, \href{../works/MonetteDD07.pdf}{MonetteDD07}~\cite{MonetteDD07}, \href{../works/VilimBC04.pdf}{VilimBC04}~\cite{VilimBC04}, \href{../works/Wolf03.pdf}{Wolf03}~\cite{Wolf03}, \href{../works/BeckF00.pdf}{BeckF00}~\cite{BeckF00}, \href{../works/TorresL00.pdf}{TorresL00}~\cite{TorresL00} & \href{../works/JuvinHHL23.pdf}{JuvinHHL23}~\cite{JuvinHHL23}, \href{../works/BoudreaultSLQ22.pdf}{BoudreaultSLQ22}~\cite{BoudreaultSLQ22}, \href{../works/OuelletQ22.pdf}{OuelletQ22}~\cite{OuelletQ22}, \href{../works/Astrand21.pdf}{Astrand21}~\cite{Astrand21}, \href{../works/Groleaz21.pdf}{Groleaz21}~\cite{Groleaz21}, \href{../works/CauwelaertDS20.pdf}{CauwelaertDS20}~\cite{CauwelaertDS20}, \href{../works/CauwelaertLS18.pdf}{CauwelaertLS18}~\cite{CauwelaertLS18}, \href{../works/Tesch16.pdf}{Tesch16}~\cite{Tesch16}, \href{../works/CauwelaertDMS16.pdf}{CauwelaertDMS16}~\cite{CauwelaertDMS16}, \href{../works/GrimesH15.pdf}{GrimesH15}~\cite{GrimesH15}, \href{../works/ChuGNSW13.pdf}{ChuGNSW13}~\cite{ChuGNSW13}, \href{../works/MalapertCGJLR12.pdf}{MalapertCGJLR12}~\cite{MalapertCGJLR12}, \href{../works/LimtanyakulS12.pdf}{LimtanyakulS12}~\cite{LimtanyakulS12}, \href{../works/KameugneFSN11.pdf}{KameugneFSN11}~\cite{KameugneFSN11}, \href{../works/Vilim09.pdf}{Vilim09}~\cite{Vilim09}, \href{../works/Wolf09.pdf}{Wolf09}~\cite{Wolf09}, \href{../works/Wolf05.pdf}{Wolf05}~\cite{Wolf05}, \href{../works/Laborie03.pdf}{Laborie03}~\cite{Laborie03}, \href{../works/SourdN00.pdf}{SourdN00}~\cite{SourdN00}\\
Algorithms & not-last & \href{../works/KameugneFND23.pdf}{KameugneFND23}~\cite{KameugneFND23}, \href{../works/TardivoDFMP23.pdf}{TardivoDFMP23}~\cite{TardivoDFMP23}, \href{../works/KameugneFGOQ18.pdf}{KameugneFGOQ18}~\cite{KameugneFGOQ18}, \href{../works/FahimiOQ18.pdf}{FahimiOQ18}~\cite{FahimiOQ18}, \href{../works/OuelletQ18.pdf}{OuelletQ18}~\cite{OuelletQ18}, \href{../works/Fahimi16.pdf}{Fahimi16}~\cite{Fahimi16}, \href{../works/Dejemeppe16.pdf}{Dejemeppe16}~\cite{Dejemeppe16}, \href{../works/GayHS15a.pdf}{GayHS15a}~\cite{GayHS15a}, \href{../works/Kameugne14.pdf}{Kameugne14}~\cite{Kameugne14}, \href{../works/Clercq12.pdf}{Clercq12}~\cite{Clercq12}, \href{../works/Malapert11.pdf}{Malapert11}~\cite{Malapert11}, \href{../works/Schutt11.pdf}{Schutt11}~\cite{Schutt11}, \href{../works/SchuttW10.pdf}{SchuttW10}~\cite{SchuttW10}, \href{../works/ArtiouchineB05.pdf}{ArtiouchineB05}~\cite{ArtiouchineB05}, \href{../works/SchuttWS05.pdf}{SchuttWS05}~\cite{SchuttWS05}, \href{../works/Vilim05.pdf}{Vilim05}~\cite{Vilim05}, \href{../works/VilimBC05.pdf}{VilimBC05}~\cite{VilimBC05}, \href{../works/Vilim04.pdf}{Vilim04}~\cite{Vilim04}, \href{../works/Wolf03.pdf}{Wolf03}~\cite{Wolf03}, \href{../works/Demassey03.pdf}{Demassey03}~\cite{Demassey03}, \href{../works/Baptiste02.pdf}{Baptiste02}~\cite{Baptiste02}, \href{../works/Beck99.pdf}{Beck99}~\cite{Beck99} & \href{../works/FetgoD22.pdf}{FetgoD22}~\cite{FetgoD22}, \href{../works/CauwelaertDS20.pdf}{CauwelaertDS20}~\cite{CauwelaertDS20}, \href{../works/GokgurHO18.pdf}{GokgurHO18}~\cite{GokgurHO18}, \href{../works/Tesch18.pdf}{Tesch18}~\cite{Tesch18}, \href{../works/Kameugne15.pdf}{Kameugne15}~\cite{Kameugne15}, \href{../works/DejemeppeCS15.pdf}{DejemeppeCS15}~\cite{DejemeppeCS15}, \href{../works/KameugneFSN14.pdf}{KameugneFSN14}~\cite{KameugneFSN14}, \href{../works/SchuttFS13a.pdf}{SchuttFS13a}~\cite{SchuttFS13a}, \href{../works/OuelletQ13.pdf}{OuelletQ13}~\cite{OuelletQ13}, \href{../works/Letort13.pdf}{Letort13}~\cite{Letort13}, \href{../works/SchuttFSW11.pdf}{SchuttFSW11}~\cite{SchuttFSW11}, \href{../works/Vilim11.pdf}{Vilim11}~\cite{Vilim11}, \href{../works/KameugneFSN11.pdf}{KameugneFSN11}~\cite{KameugneFSN11}, \href{../works/Lombardi10.pdf}{Lombardi10}~\cite{Lombardi10}, \href{../works/BartakSR10.pdf}{BartakSR10}~\cite{BartakSR10}, \href{../works/MonetteDD07.pdf}{MonetteDD07}~\cite{MonetteDD07}, \href{../works/Wolf05.pdf}{Wolf05}~\cite{Wolf05}, \href{../works/VilimBC04.pdf}{VilimBC04}~\cite{VilimBC04}, \href{../works/TorresL00.pdf}{TorresL00}~\cite{TorresL00}, \href{../works/BeckF00.pdf}{BeckF00}~\cite{BeckF00} & \href{../works/JuvinHHL23.pdf}{JuvinHHL23}~\cite{JuvinHHL23}, \href{../works/BoudreaultSLQ22.pdf}{BoudreaultSLQ22}~\cite{BoudreaultSLQ22}, \href{../works/GeitzGSSW22.pdf}{GeitzGSSW22}~\cite{GeitzGSSW22}, \href{../works/OuelletQ22.pdf}{OuelletQ22}~\cite{OuelletQ22}, \href{../works/Astrand21.pdf}{Astrand21}~\cite{Astrand21}, \href{../works/Groleaz21.pdf}{Groleaz21}~\cite{Groleaz21}, \href{../works/GodetLHS20.pdf}{GodetLHS20}~\cite{GodetLHS20}, \href{../works/YangSS19.pdf}{YangSS19}~\cite{YangSS19}, \href{../works/CauwelaertLS18.pdf}{CauwelaertLS18}~\cite{CauwelaertLS18}, \href{../works/HookerH17.pdf}{HookerH17}~\cite{HookerH17}, \href{../works/CauwelaertDMS16.pdf}{CauwelaertDMS16}~\cite{CauwelaertDMS16}, \href{../works/Tesch16.pdf}{Tesch16}~\cite{Tesch16}, \href{../works/GrimesH15.pdf}{GrimesH15}~\cite{GrimesH15}, \href{../works/ChuGNSW13.pdf}{ChuGNSW13}~\cite{ChuGNSW13}, \href{../works/LimtanyakulS12.pdf}{LimtanyakulS12}~\cite{LimtanyakulS12}, \href{../works/MalapertCGJLR12.pdf}{MalapertCGJLR12}~\cite{MalapertCGJLR12}, \href{../works/ChenGPSH10.pdf}{ChenGPSH10}~\cite{ChenGPSH10}, \href{../works/Wolf09.pdf}{Wolf09}~\cite{Wolf09}, \href{../works/MonetteDH09.pdf}{MonetteDH09}~\cite{MonetteDH09}, \href{../works/Vilim09a.pdf}{Vilim09a}~\cite{Vilim09a}, \href{../works/GrimesHM09.pdf}{GrimesHM09}~\cite{GrimesHM09}, \href{../works/Vilim09.pdf}{Vilim09}~\cite{Vilim09}, \href{../works/BocewiczBB09.pdf}{BocewiczBB09}~\cite{BocewiczBB09}, \href{../works/WolfS05.pdf}{WolfS05}~\cite{WolfS05}, \href{../works/Laborie03.pdf}{Laborie03}~\cite{Laborie03}, \href{../works/Vilim03.pdf}{Vilim03}~\cite{Vilim03}\\
Algorithms & sweep & \href{../works/Tesch18.pdf}{Tesch18}~\cite{Tesch18}, \href{../works/BonfiettiZLM16.pdf}{BonfiettiZLM16}~\cite{BonfiettiZLM16}, \href{../works/NattafALR16.pdf}{NattafALR16}~\cite{NattafALR16}, \href{../works/Tesch16.pdf}{Tesch16}~\cite{Tesch16}, \href{../works/LetortCB15.pdf}{LetortCB15}~\cite{LetortCB15}, \href{../works/Derrien15.pdf}{Derrien15}~\cite{Derrien15}, \href{../works/SimoninAHL15.pdf}{SimoninAHL15}~\cite{SimoninAHL15}, \href{../works/NattafAL15.pdf}{NattafAL15}~\cite{NattafAL15}, \href{../works/GayHS15.pdf}{GayHS15}~\cite{GayHS15}, \href{../works/DerrienPZ14.pdf}{DerrienPZ14}~\cite{DerrienPZ14}, \href{../works/Letort13.pdf}{Letort13}~\cite{Letort13}, \href{../works/LetortCB13.pdf}{LetortCB13}~\cite{LetortCB13}, \href{../works/Clercq12.pdf}{Clercq12}~\cite{Clercq12}, \href{../works/LetortBC12.pdf}{LetortBC12}~\cite{LetortBC12}, \href{../works/SimoninAHL12.pdf}{SimoninAHL12}~\cite{SimoninAHL12}, \href{../works/ClercqPBJ11.pdf}{ClercqPBJ11}~\cite{ClercqPBJ11}, \href{../works/Malapert11.pdf}{Malapert11}~\cite{Malapert11}, \href{../works/abs-0907-0939.pdf}{abs-0907-0939}~\cite{abs-0907-0939}, \href{../works/BeldiceanuP07.pdf}{BeldiceanuP07}~\cite{BeldiceanuP07}, \href{../works/Wolf05.pdf}{Wolf05}~\cite{Wolf05}, \href{../works/Wolf03.pdf}{Wolf03}~\cite{Wolf03}, \href{../works/BeldiceanuC02.pdf}{BeldiceanuC02}~\cite{BeldiceanuC02} & \href{../works/ArkhipovBL19.pdf}{ArkhipovBL19}~\cite{ArkhipovBL19}, \href{../works/FahimiOQ18.pdf}{FahimiOQ18}~\cite{FahimiOQ18}, \href{../works/GoldwaserS18.pdf}{GoldwaserS18}~\cite{GoldwaserS18}, \href{../works/GayHS15a.pdf}{GayHS15a}~\cite{GayHS15a}, \href{../works/Schutt11.pdf}{Schutt11}~\cite{Schutt11}, \href{../works/AronssonBK09.pdf}{AronssonBK09}~\cite{AronssonBK09}, \href{../works/PoderB08.pdf}{PoderB08}~\cite{PoderB08}, \href{../works/WolfS05.pdf}{WolfS05}~\cite{WolfS05} & \href{../works/BonninMNE24.pdf}{BonninMNE24}~\cite{BonninMNE24}, \href{../works/KameugneFND23.pdf}{KameugneFND23}~\cite{KameugneFND23}, \href{../works/TardivoDFMP23.pdf}{TardivoDFMP23}~\cite{TardivoDFMP23}, \href{../works/HebrardALLCMR22.pdf}{HebrardALLCMR22}~\cite{HebrardALLCMR22}, \href{../works/GeitzGSSW22.pdf}{GeitzGSSW22}~\cite{GeitzGSSW22}, \href{../works/OuelletQ22.pdf}{OuelletQ22}~\cite{OuelletQ22}, \href{../works/FetgoD22.pdf}{FetgoD22}~\cite{FetgoD22}, \href{../works/Godet21a.pdf}{Godet21a}~\cite{Godet21a}, \href{../works/FallahiAC20.pdf}{FallahiAC20}~\cite{FallahiAC20}, \href{../works/HoundjiSW19.pdf}{HoundjiSW19}~\cite{HoundjiSW19}, \href{../works/KameugneFGOQ18.pdf}{KameugneFGOQ18}~\cite{KameugneFGOQ18}, \href{../works/CauwelaertLS18.pdf}{CauwelaertLS18}~\cite{CauwelaertLS18}, \href{../works/Madi-WambaLOBM17.pdf}{Madi-WambaLOBM17}~\cite{Madi-WambaLOBM17}, \href{../works/Fahimi16.pdf}{Fahimi16}~\cite{Fahimi16}, \href{../works/Nattaf16.pdf}{Nattaf16}~\cite{Nattaf16}, \href{../works/GingrasQ16.pdf}{GingrasQ16}~\cite{GingrasQ16}, \href{../works/Dejemeppe16.pdf}{Dejemeppe16}~\cite{Dejemeppe16}, \href{../works/BartakV15.pdf}{BartakV15}~\cite{BartakV15}, \href{../works/EvenSH15.pdf}{EvenSH15}~\cite{EvenSH15}, \href{../works/EvenSH15a.pdf}{EvenSH15a}~\cite{EvenSH15a}, \href{../works/DerrienP14.pdf}{DerrienP14}~\cite{DerrienP14}, \href{../works/BonfiettiLBM14.pdf}{BonfiettiLBM14}~\cite{BonfiettiLBM14}, \href{../works/GaySS14.pdf}{GaySS14}~\cite{GaySS14}, \href{../works/OuelletQ13.pdf}{OuelletQ13}~\cite{OuelletQ13}, \href{../works/SimonisH11.pdf}{SimonisH11}~\cite{SimonisH11}, \href{../works/BeldiceanuCDP11.pdf}{BeldiceanuCDP11}~\cite{BeldiceanuCDP11}, \href{../works/Vilim11.pdf}{Vilim11}~\cite{Vilim11}, \href{../works/Lombardi10.pdf}{Lombardi10}~\cite{Lombardi10}, \href{../works/LombardiM10a.pdf}{LombardiM10a}~\cite{LombardiM10a}... (Total: 37)\\
Algorithms & time-tabling & \href{../works/ShaikhK23.pdf}{ShaikhK23}~\cite{ShaikhK23}, \href{../works/TardivoDFMP23.pdf}{TardivoDFMP23}~\cite{TardivoDFMP23}, \href{../works/OuelletQ22.pdf}{OuelletQ22}~\cite{OuelletQ22}, \href{../works/OrnekOS20.pdf}{OrnekOS20}~\cite{OrnekOS20}, \href{../works/Lemos21.pdf}{Lemos21}~\cite{Lemos21}, \href{../works/DemirovicS18.pdf}{DemirovicS18}~\cite{DemirovicS18}, \href{../works/FahimiOQ18.pdf}{FahimiOQ18}~\cite{FahimiOQ18}, \href{../works/Fahimi16.pdf}{Fahimi16}~\cite{Fahimi16}, \href{../works/GayHS15a.pdf}{GayHS15a}~\cite{GayHS15a}, \href{../works/Kameugne14.pdf}{Kameugne14}~\cite{Kameugne14}, \href{../works/OuelletQ13.pdf}{OuelletQ13}~\cite{OuelletQ13}, \href{../works/Letort13.pdf}{Letort13}~\cite{Letort13}, \href{../works/GuyonLPR12.pdf}{GuyonLPR12}~\cite{GuyonLPR12}, \href{../works/HeinzS11.pdf}{HeinzS11}~\cite{HeinzS11}, \href{../works/Menana11.pdf}{Menana11}~\cite{Menana11}, \href{../works/KanetAG04.pdf}{KanetAG04}~\cite{KanetAG04}, \href{../works/Laborie03.pdf}{Laborie03}~\cite{Laborie03}, \href{../works/ElkhyariGJ02a.pdf}{ElkhyariGJ02a}~\cite{ElkhyariGJ02a}, \href{../works/Wallace96.pdf}{Wallace96}~\cite{Wallace96} & \href{../works/Astrand21.pdf}{Astrand21}~\cite{Astrand21}, \href{../works/Godet21a.pdf}{Godet21a}~\cite{Godet21a}, \href{../works/WallaceY20.pdf}{WallaceY20}~\cite{WallaceY20}, \href{../works/ZarandiASC20.pdf}{ZarandiASC20}~\cite{ZarandiASC20}, \href{../works/abs-1902-01193.pdf}{abs-1902-01193}~\cite{abs-1902-01193}, \href{../works/OuelletQ18.pdf}{OuelletQ18}~\cite{OuelletQ18}, \href{../works/CauwelaertLS18.pdf}{CauwelaertLS18}~\cite{CauwelaertLS18}, \href{../works/Tesch18.pdf}{Tesch18}~\cite{Tesch18}, \href{../works/HookerH17.pdf}{HookerH17}~\cite{HookerH17}, \href{../works/Siala15a.pdf}{Siala15a}~\cite{Siala15a}, \href{../works/Derrien15.pdf}{Derrien15}~\cite{Derrien15}, \href{../works/GayHS15.pdf}{GayHS15}~\cite{GayHS15}, \href{../works/Siala15.pdf}{Siala15}~\cite{Siala15}, \href{../works/BofillGSV15.pdf}{BofillGSV15}~\cite{BofillGSV15}, \href{../works/Vilim11.pdf}{Vilim11}~\cite{Vilim11}, \href{../works/Elkhyari03.pdf}{Elkhyari03}~\cite{Elkhyari03}, \href{../works/Demassey03.pdf}{Demassey03}~\cite{Demassey03}, \href{../works/Bartak02.pdf}{Bartak02}~\cite{Bartak02} & \href{../works/BonninMNE24.pdf}{BonninMNE24}~\cite{BonninMNE24}, \href{../works/PrataAN23.pdf}{PrataAN23}~\cite{PrataAN23}, \href{../works/KameugneFND23.pdf}{KameugneFND23}~\cite{KameugneFND23}, \href{../works/AbreuNP23.pdf}{AbreuNP23}~\cite{AbreuNP23}, \href{../works/MarliereSPR23.pdf}{MarliereSPR23}~\cite{MarliereSPR23}, \href{../works/Fatemi-AnarakiTFV23.pdf}{Fatemi-AnarakiTFV23}~\cite{Fatemi-AnarakiTFV23}, \href{../works/LacknerMMWW23.pdf}{LacknerMMWW23}~\cite{LacknerMMWW23}, \href{../works/TouatBT22.pdf}{TouatBT22}~\cite{TouatBT22}, \href{../works/FarsiTM22.pdf}{FarsiTM22}~\cite{FarsiTM22}, \href{../works/FetgoD22.pdf}{FetgoD22}~\cite{FetgoD22}, \href{../works/SvancaraB22.pdf}{SvancaraB22}~\cite{SvancaraB22}, \href{../works/GeibingerMM21.pdf}{GeibingerMM21}~\cite{GeibingerMM21}, \href{../works/MokhtarzadehTNF20.pdf}{MokhtarzadehTNF20}~\cite{MokhtarzadehTNF20}, \href{../works/GodetLHS20.pdf}{GodetLHS20}~\cite{GodetLHS20}, \href{../works/LiuLH19.pdf}{LiuLH19}~\cite{LiuLH19}, \href{../works/KucukY19.pdf}{KucukY19}~\cite{KucukY19}, \href{../works/Caballero19.pdf}{Caballero19}~\cite{Caballero19}, \href{../works/Hooker19.pdf}{Hooker19}~\cite{Hooker19}, \href{../works/abs-1911-04766.pdf}{abs-1911-04766}~\cite{abs-1911-04766}, \href{../works/GeibingerMM19.pdf}{GeibingerMM19}~\cite{GeibingerMM19}, \href{../works/ArkhipovBL19.pdf}{ArkhipovBL19}~\cite{ArkhipovBL19}, \href{../works/KameugneFGOQ18.pdf}{KameugneFGOQ18}~\cite{KameugneFGOQ18}, \href{../works/AstrandJZ18.pdf}{AstrandJZ18}~\cite{AstrandJZ18}, \href{../works/BaptisteB18.pdf}{BaptisteB18}~\cite{BaptisteB18}, \href{../works/GoldwaserS18.pdf}{GoldwaserS18}~\cite{GoldwaserS18}, \href{../works/CohenHB17.pdf}{CohenHB17}~\cite{CohenHB17}, \href{../works/YoungFS17.pdf}{YoungFS17}~\cite{YoungFS17}, \href{../works/LuoVLBM16.pdf}{LuoVLBM16}~\cite{LuoVLBM16}, \href{../works/ZarandiKS16.pdf}{ZarandiKS16}~\cite{ZarandiKS16}... (Total: 66)\\
\end{longtable}
}





\clearpage
\phantomsection
\addcontentsline{toc}{section}{Bibliography}
\bibliographystyle{plainurl}
\bibliography{bib}



\appendix
\clearpage
\section{Papers and Articles Missing a Local Copy}

This section lists all papers and articles for which we were not able to locate an electronic copy that we could download to our system. This might be because the work is behind a paywall for which we do not have access, or since the paper only exists in hardcopy, for works from the start of the period covered. As in either case we are not able to extract useful information from the work, either automatically, or manually, without the actual text itself, these gaps should be closed where possible.

{\scriptsize
\begin{longtable}{llp{5cm}p{10cm}rp{3cm}l}
\caption{Paper without Local Copy}\\ \toprule
Key & URL & Authors & Title & Year & \shortstack{Conference\\/Journal} & Cite\\ \midrule
\endhead
\bottomrule
\endfoot
FriedrichFMRSST14 & \href{https://doi.org/10.1007/978-3-319-28697-6\_23}{FriedrichFMRSST14} & Gerhard Friedrich and Melanie Fr{\"{u}}hst{\"{u}}ck and Vera Mersheeva and Anna Ryabokon and Maria Sander and Andreas Starzacher and Erich Teppan & Representing Production Scheduling with Constraint Answer Set Programming & 2014 & GOR 2014 & \cite{FriedrichFMRSST14}\\VillaverdeP04 & \href{}{VillaverdeP04} & Karen Villaverde and Enrico Pontelli & An Investigation of Scheduling in Distributed Constraint Logic Programming & 2004 & ISCA 2004 & \cite{VillaverdeP04}\\WolinskiKG04a & \href{https://doi.org/10.1145/968280.968336}{WolinskiKG04a} & Christophe Wolinski and Krzysztof Kuchcinski and Maya B. Gokhale & A constraints programming approach to communication scheduling on SoPC architectures & 2004 & FPGA 2004 & \cite{WolinskiKG04a}\\BoucherBVBL97 & \href{}{BoucherBVBL97} & Eric Boucher and Astrid Bachelu and Christophe Varnier and Pierre Baptiste and Bruno Legeard & Multi-criteria Comparison Between Algorithmic, Constraint Logic and Specific Constraint Programming on a Real Schedulingt Problem & 1997 & PACT 1997 & \cite{BoucherBVBL97}\\PapeB97 & \href{}{PapeB97} & Claude Le Pape and Philippe Baptiste & A Constraint Programming Library for Preemptive and Non-Preemptive Scheduling & 1997 & PACT 1997 & \cite{PapeB97}\\JourdanFRD94 & \href{}{JourdanFRD94} & Jean Jourdan and Fran{\c{c}}ois Fages and Didier Rozzonelli and Alain Demeure & Data Alignment and Task Scheduling On Parallel Machines Using Concurrent Constraint Model-based Programming & 1994 & ILPS 1994 & \cite{JourdanFRD94}\\AggounB92 & \href{}{AggounB92} & Abderrahmane Aggoun and Nicolas Beldiceanu & Extending {CHIP} in order to solve complex scheduling and placement problems & 1992 & JFPL 1992 & \cite{AggounB92}\\\end{longtable}
}



{\scriptsize
\begin{longtable}{p{2cm}p{2cm}p{5cm}p{10cm}rp{3cm}l}
\rowcolor{white}\caption{ARTICLE without Local Copy}\\ \toprule
\rowcolor{white}Key & URL & Authors & Title & Year & \shortstack{Conference\\/Journal} & Cite\\ \midrule
\endhead
\bottomrule
\endfoot
AlakaP23 & \href{http://dx.doi.org/10.1007/s00500-023-09105-9}{AlakaP23} & \hyperref[auth:a770]{Hacı Mehmet Alakaş}, \hyperref[auth:a1410]{M. Pınarbaşı} & Balancing of cost-oriented U-type general resource-constrained assembly line: new constraint programming models & 2023 & Soft Computing & \cite{AlakaP23}\\FahimiQ23 & \href{http://dx.doi.org/10.1287/ijoc.2021.0138}{FahimiQ23} & \hyperref[auth:a122]{H. Fahimi}, \hyperref[auth:a123]{C. Quimper} & Overload-Checking and Edge-Finding for Robust Cumulative Scheduling & 2023 & INFORMS Journal on Computing & \cite{FahimiQ23}\\GhasemiMH23 & \href{http://dx.doi.org/10.1080/23302674.2023.2224509}{GhasemiMH23} & \hyperref[auth:a994]{S. Ghasemi}, \hyperref[auth:a433]{R. Tavakkoli{-}Moghaddam}, \hyperref[auth:a995]{M. Hamid} & Operating room scheduling by emphasising human factors and dynamic decision-making styles: a constraint programming method & 2023 & International Journal of Systems Science: Operations \  Logistics & \cite{GhasemiMH23}\\NouriMHD23 & \href{http://dx.doi.org/10.1080/00207543.2023.2173503}{NouriMHD23} & \hyperref[auth:a743]{B. Vahedi-Nouri}, \hyperref[auth:a956]{R. Tavakkoli-Moghaddam}, \hyperref[auth:a957]{Z. Hanzálek}, \hyperref[auth:a958]{A. Dolgui} & Production scheduling in a reconfigurable manufacturing system benefiting from human-robot collaboration & 2023 & International Journal of Production Research & \cite{NouriMHD23}\\CilKLO22 & \href{http://dx.doi.org/10.1016/j.eswa.2022.117529}{CilKLO22} & \hyperref[auth:a1407]{Zeynel Abidin Cil}, \hyperref[auth:a1406]{D. Kizilay}, \hyperref[auth:a1408]{Z. Li}, \hyperref[auth:a1409]{H. \"{O}ztop} & Two-sided disassembly line balancing problem with sequence-dependent setup time: A constraint programming model and artificial bee colony algorithm & 2022 & Expert Systems with Applications & \cite{CilKLO22}\\HillBCGN22 & \href{http://dx.doi.org/10.1287/ijoc.2022.1222}{HillBCGN22} & \hyperref[auth:a64]{A. Hill}, \hyperref[auth:a982]{Andrea J. Brickey}, \hyperref[auth:a983]{I. Cipriano}, \hyperref[auth:a984]{M. Goycoolea}, \hyperref[auth:a985]{A. Newman} & Optimization Strategies for Resource-Constrained Project Scheduling Problems in Underground Mining & 2022 & INFORMS Journal on Computing & \cite{HillBCGN22}\\MartnezAJ22 & \href{http://dx.doi.org/10.1287/ijoc.2021.1079}{MartnezAJ22} & \hyperref[auth:a945]{Karim Pérez Martínez}, \hyperref[auth:a946]{Y. Adulyasak}, \hyperref[auth:a848]{R. Jans} & Logic-Based Benders Decomposition for Integrated Process Configuration and Production Planning Problems & 2022 & INFORMS Journal on Computing & \cite{MartnezAJ22}\\NaderiR22 & \href{http://dx.doi.org/10.1287/ijoo.2021.0056}{NaderiR22} & \hyperref[auth:a732]{B. Naderi}, \hyperref[auth:a734]{V. Roshanaei} & Critical-Path-Search Logic-Based Benders Decomposition Approaches for Flexible Job Shop Scheduling & 2022 & INFORMS Journal on Optimization & \cite{NaderiR22}\\ShiYXQ22 & \href{https://doi.org/10.1080/00207543.2021.1963496}{ShiYXQ22} & \hyperref[auth:a449]{G. Shi}, \hyperref[auth:a450]{Z. Yang}, \hyperref[auth:a451]{Y. Xu}, \hyperref[auth:a452]{Y. Quan} & Solving the integrated process planning and scheduling problem using an enhanced constraint programming-based approach & 2022 & International Journal of Production Research & \cite{ShiYXQ22}\\Alaka21 & \href{http://dx.doi.org/10.1007/s00500-021-05602-x}{Alaka21} & \hyperref[auth:a770]{Hacı Mehmet Alakaş} & General resource-constrained assembly line balancing problem: conjunction normal form based constraint programming models & 2021 & Soft Computing & \cite{Alaka21}\\CarlierSJP21 & \href{http://dx.doi.org/10.1080/00207543.2021.1923853}{CarlierSJP21} & \hyperref[auth:a852]{J. Carlier}, \hyperref[auth:a937]{A. Sahli}, \hyperref[auth:a938]{A. Jouglet}, \hyperref[auth:a853]{E. Pinson} & A faster checker of the energetic reasoning for the cumulative scheduling problem & 2021 & International Journal of Production Research & \cite{CarlierSJP21}\\NaderiRBAU21 & \href{http://dx.doi.org/10.1111/poms.13397}{NaderiRBAU21} & \hyperref[auth:a732]{B. Naderi}, \hyperref[auth:a734]{V. Roshanaei}, \hyperref[auth:a843]{Mehmet A. Begen}, \hyperref[auth:a902]{Dionne M. Aleman}, \hyperref[auth:a903]{David R. Urbach} & Increased Surgical Capacity without Additional Resources: Generalized Operating Room Planning and Scheduling & 2021 & Production and Operations Management & \cite{NaderiRBAU21}\\RabbaniMM21 & \href{http://dx.doi.org/10.1080/17509653.2021.1905096}{RabbaniMM21} & \hyperref[auth:a1268]{M. Rabbani}, \hyperref[auth:a518]{M. Mokhtarzadeh}, \hyperref[auth:a1269]{N. Manavizadeh} & A constraint programming approach and a hybrid of genetic and K-means algorithms to solve the p-hub location-allocation problems & 2021 & International Journal of Management Science and Engineering Management & \cite{RabbaniMM21}\\AlizdehS20 & \href{https://doi.org/10.1504/IJAIP.2020.106687}{AlizdehS20} & \hyperref[auth:a516]{S. Alizdeh}, \hyperref[auth:a517]{S. Saeidi} & Fuzzy project scheduling with critical path including risk and resource constraints using linear programming & 2020 & Int. J. Adv. Intell. Paradigms & \cite{AlizdehS20}\\BalochG20 & \href{http://dx.doi.org/10.1287/trsc.2019.0928}{BalochG20} & \hyperref[auth:a1257]{G. Baloch}, \hyperref[auth:a1258]{F. Gzara} & Strategic Network Design for Parcel Delivery with Drones Under Competition & 2020 & Transportation Science & \cite{BalochG20}\\GuoHLW20 & \href{http://dx.doi.org/10.1080/0305215x.2019.1699919}{GuoHLW20} & \hyperref[auth:a941]{P. Guo}, \hyperref[auth:a942]{X. He}, \hyperref[auth:a943]{Y. Luan}, \hyperref[auth:a944]{Y. Wang} & Logic-based Benders decomposition for gantry crane scheduling with transferring position constraints in a rail–road container terminal & 2020 & Engineering Optimization & \cite{GuoHLW20}\\Ham20 & \href{http://dx.doi.org/10.1080/00207543.2019.1709671}{Ham20} & \hyperref[auth:a756]{A. Ham} & Transfer-robot task scheduling in job shop & 2020 & International Journal of Production Research & \cite{Ham20}\\KizilayC20 & \href{http://dx.doi.org/10.1080/0305215x.2020.1786081}{KizilayC20} & \hyperref[auth:a1406]{D. Kizilay}, \hyperref[auth:a1407]{Zeynel Abidin Cil} & Constraint programming approach for multi-objective two-sided assembly line balancing problem with multi-operator stations & 2020 & Engineering Optimization & \cite{KizilayC20}\\PinarbasiA20 & \href{http://dx.doi.org/10.1080/0305215x.2020.1716746}{PinarbasiA20} & \hyperref[auth:a1410]{M. Pınarbaşı}, \hyperref[auth:a770]{Hacı Mehmet Alakaş} & Balancing stochastic type-II assembly lines: chance-constrained mixed integer and constraint programming models & 2020 & Engineering Optimization & \cite{PinarbasiA20}\\EdwardsBSE19 & \href{http://dx.doi.org/10.1080/01605682.2019.1595192}{EdwardsBSE19} & \hyperref[auth:a899]{Steven J. Edwards}, \hyperref[auth:a900]{D. Baatar}, \hyperref[auth:a901]{K. Smith-Miles}, \hyperref[auth:a472]{Andreas T. Ernst} & Symmetry breaking of identical projects in the high-multiplicity RCPSP/max & 2019 & Journal of the Operational Research Society & \cite{EdwardsBSE19}\\HechingHK19 & \href{http://dx.doi.org/10.1287/trsc.2018.0830}{HechingHK19} & \hyperref[auth:a1034]{A. Heching}, \hyperref[auth:a161]{John N. Hooker}, \hyperref[auth:a1035]{R. Kimura} & A Logic-Based Benders Approach to Home Healthcare Delivery & 2019 & Transportation Science & \cite{HechingHK19}\\WariZ19 & \href{http://dx.doi.org/10.1080/00207543.2019.1571250}{WariZ19} & \hyperref[auth:a846]{E. Wari}, \hyperref[auth:a847]{W. Zhu} & A Constraint Programming model for food processing industry: a case for an ice cream processing facility & 2019 & International Journal of Production Research & \cite{WariZ19}\\RoshanaeiLAU17a & \href{http://dx.doi.org/10.1287/ijoc.2017.0745}{RoshanaeiLAU17a} & \hyperref[auth:a734]{V. Roshanaei}, \hyperref[auth:a935]{C. Luong}, \hyperref[auth:a902]{Dionne M. Aleman}, \hyperref[auth:a903]{David R. Urbach} & Collaborative Operating Room Planning and Scheduling & 2017 & INFORMS Journal on Computing & \cite{RoshanaeiLAU17a}\\PengLC14 & \href{http://dx.doi.org/10.1155/2014/917685}{PengLC14} & \hyperref[auth:a923]{Y. Peng}, \hyperref[auth:a1411]{D. Lu}, \hyperref[auth:a921]{Y. Chen} & A Constraint Programming Method for Advanced Planning and Scheduling System with Multilevel Structured Products & 2014 & Discrete Dynamics in Nature and Society & \cite{PengLC14}\\ZhaoL14 & \href{http://dx.doi.org/10.1016/j.orhc.2014.05.003}{ZhaoL14} & \hyperref[auth:a1402]{Z. Zhao}, \hyperref[auth:a1403]{X. Li} & Scheduling elective surgeries with sequence-dependent setup times to multiple operating rooms using constraint programming & 2014 & Operations Research for Health Care & \cite{ZhaoL14}\\MalapertGR12 & \href{http://dx.doi.org/10.1016/j.ejor.2012.04.008}{MalapertGR12} & \hyperref[auth:a82]{A. Malapert}, \hyperref[auth:a1400]{C. Guéret}, \hyperref[auth:a1401]{L. Rousseau} & A constraint programming approach for a batch processing problem with non-identical job sizes & 2012 & European Journal of Operational Research & \cite{MalapertGR12}\\TopalogluSS12 & \href{http://dx.doi.org/10.1016/j.eswa.2011.09.038}{TopalogluSS12} & \hyperref[auth:a623]{S. Topaloglu}, \hyperref[auth:a1404]{L. Salum}, \hyperref[auth:a1405]{Aliye Ayca Supciller} & Rule-based modeling and constraint programming based solution of the assembly line balancing problem & 2012 & Expert Systems with Applications & \cite{TopalogluSS12}\\ZarandiB12 & \href{http://dx.doi.org/10.1287/ijoc.1110.0458}{ZarandiB12} & \hyperref[auth:a955]{Mohammad M. Fazel-Zarandi}, \hyperref[auth:a89]{J. Christopher Beck} & Using Logic-Based Benders Decomposition to Solve the Capacity- and Distance-Constrained Plant Location Problem & 2012 & INFORMS Journal on Computing & \cite{ZarandiB12}\\EdisO11a & \href{http://dx.doi.org/10.1080/03052151003759117}{EdisO11a} & \hyperref[auth:a349]{Emrah B. Edis}, \hyperref[auth:a351]{I. Ozkarahan} & A combined integer/constraint programming approach to a resource-constrained parallel machine scheduling problem with machine eligibility restrictions & 2011 & Engineering Optimization & \cite{EdisO11a}\\LiuGT10 & \href{http://dx.doi.org/10.3724/sp.j.1004.2010.00603}{LiuGT10} & \hyperref[auth:a1240]{S. Liu}, \hyperref[auth:a1241]{Z. Guo}, \hyperref[auth:a1242]{J. Tang} & Constraint Propagation for Cumulative Scheduling Problems with Precedences: Constraint Propagation for Cumulative Scheduling Problems with Precedences & 2010 & Acta Automatica Sinica & \cite{LiuGT10}\\ZeballosM09 & \href{http://dx.doi.org/10.1021/ie901176n}{ZeballosM09} & \hyperref[auth:a1173]{Luis J. Zeballos}, \hyperref[auth:a1210]{Carlos A. Méndez} & An Integrated CP-Based Approach for Scheduling of Processing and Transport Units in Pipeless Plants & 2009 & Industrial \  Engineering Chemistry Research & \cite{ZeballosM09}\\OkanoDTRYA04 & \href{https://doi.org/10.1147/rd.485.0811}{OkanoDTRYA04} & \hyperref[auth:a1312]{H. Okano}, \hyperref[auth:a250]{Andrew J. Davenport}, \hyperref[auth:a1313]{M. Trumbo}, \hyperref[auth:a252]{C. Reddy}, \hyperref[auth:a1314]{K. Yoda}, \hyperref[auth:a1315]{M. Amano} & Finishing Line Scheduling in the steel industry & 2004 & {IBM} J. Res. Dev. & \cite{OkanoDTRYA04}\\Hentenryck02 & \href{http://dx.doi.org/10.1287/ijoc.14.4.345.2826}{Hentenryck02} & \hyperref[auth:a149]{Pascal Van Hentenryck} & Constraint and Integer Programming in OPL & 2002 & INFORMS Journal on Computing & \cite{Hentenryck02}\\Hooker02 & \href{http://dx.doi.org/10.1287/ijoc.14.4.295.2828}{Hooker02} & \hyperref[auth:a161]{John N. Hooker} & Logic, Optimization, and Constraint Programming & 2002 & INFORMS Journal on Computing & \cite{Hooker02}\\MilanoORT02 & \href{http://dx.doi.org/10.1287/ijoc.14.4.387.2830}{MilanoORT02} & \hyperref[auth:a144]{M. Milano}, \hyperref[auth:a859]{G. Ottosson}, \hyperref[auth:a256]{P. Refalo}, \hyperref[auth:a881]{Erlendur S. Thorsteinsson} & The Role of Integer Programming Techniques in Constraint Programming's Global Constraints & 2002 & INFORMS Journal on Computing & \cite{MilanoORT02}\\LustigP01 & \href{http://dx.doi.org/10.1287/inte.31.6.29.9647}{LustigP01} & Irvin J. Lustig, J. Puget & Program Does Not Equal Program: Constraint Programming and Its Relationship to Mathematical Programming & 2001 & Interfaces & \cite{LustigP01}\\BockmayrK98 & \href{http://dx.doi.org/10.1287/ijoc.10.3.287}{BockmayrK98} & \hyperref[auth:a916]{A. Bockmayr}, \hyperref[auth:a1060]{T. Kasper} & Branch and Infer: A Unifying Framework for Integer and Finite Domain Constraint Programming & 1998 & INFORMS Journal on Computing & \cite{BockmayrK98}\\DarbyDowmanL98 & \href{http://dx.doi.org/10.1287/ijoc.10.3.276}{DarbyDowmanL98} & \hyperref[auth:a1086]{K. Darby-Dowman}, \hyperref[auth:a179]{J. Little} & Properties of Some Combinatorial Optimization Problems and Their Effect on the Performance of Integer Programming and Constraint Logic Programming & 1998 & INFORMS Journal on Computing & \cite{DarbyDowmanL98}\\PintoG97 & \href{https://www.sciencedirect.com/science/article/pii/S0098135496003183}{PintoG97} & \hyperref[auth:a1277]{Jose M. Pinto}, \hyperref[auth:a385]{Ignacio E. Grossmann} & A logic-based approach to scheduling problems with resource constraints & 1997 & Computers \  Chemical Engineering & \cite{PintoG97}\\PeschT96 & \href{http://dx.doi.org/10.1287/ijoc.8.2.144}{PeschT96} & \hyperref[auth:a441]{E. Pesch}, \hyperref[auth:a1236]{Ulrich A. W. Tetzlaff} & Constraint Propagation Based Scheduling of Job Shops & 1996 & INFORMS Journal on Computing & \cite{PeschT96}\\LubySZ93 & \href{http://dx.doi.org/10.1016/0020-0190(93)90029-9}{LubySZ93} & M. Luby, A. Sinclair, D. Zuckerman & Optimal speedup of Las Vegas algorithms & 1993 & Information Processing Letters & \cite{LubySZ93}\\MintonJPL92 & \href{http://dx.doi.org/10.1016/0004-3702(92)90007-k}{MintonJPL92} & \hyperref[auth:a1230]{S. Minton}, \hyperref[auth:a1231]{Mark D. Johnston}, \hyperref[auth:a1232]{Andrew B. Philips}, \hyperref[auth:a1233]{P. Laird} & Minimizing conflicts: a heuristic repair method for constraint satisfaction and scheduling problems & 1992 & Artificial Intelligence & \cite{MintonJPL92}\\Tay92 & \href{}{Tay92} & \hyperref[auth:a707]{David B. H. Tay} & {COPS:} {A} Constraint Programming Approach to Resource-Limited Project Scheduling & 1992 & Comput. J. & \cite{Tay92}\\Carlier82 & \href{http://dx.doi.org/10.1016/s0377-2217(82)80007-6}{Carlier82} & \hyperref[auth:a852]{J. Carlier} & The one-machine sequencing problem & 1982 & European Journal of Operational Research & \cite{Carlier82}\\Lauriere78 & \href{http://dx.doi.org/10.1016/0004-3702(78)90029-2}{Lauriere78} & J. Lauriere & A language and a program for stating and solving combinatorial problems & 1978 & Artificial Intelligence & \cite{Lauriere78}\\Mackworth77 & \href{http://dx.doi.org/10.1016/0004-3702(77)90007-8}{Mackworth77} & Alan K. Mackworth & Consistency in networks of relations & 1977 & Artificial Intelligence & \cite{Mackworth77}\\GareyJS76 & \href{http://dx.doi.org/10.1287/moor.1.2.117}{GareyJS76} & M. R. Garey, D. S. Johnson, R. Sethi & The Complexity of Flowshop and Jobshop Scheduling & 1976 & Mathematics of Operations Research & \cite{GareyJS76}\\PritskerWW69 & \href{http://dx.doi.org/10.1287/mnsc.16.1.93}{PritskerWW69} & A. Alan B. Pritsker, Lawrence J. Waiters, Philip M. Wolfe & Multiproject Scheduling with Limited Resources: A Zero-One Programming Approach & 1969 & Management Science & \cite{PritskerWW69}\\\end{longtable}
}



\clearpage
\section{Papers and Articles Without Recognized Concepts}

This section lists papers and articles for which we have a pdf local copy, but where we were not able to extract any of the defined concepts. This can basically have two reasons. We either have included a paper which is not at all related to scheduling, so that none of the defined concepts occur in the paper. A  more likely cause is that the pdf file is a scanned document for which optical character recognition was not run or not successful, so that the pdf consists of a series of bitmap images. In that case, pdfgrep is unable to find any text in the document, and no matches for concepts are found. It may be useful to check the pdf files to see if that is the case.

{\scriptsize
\begin{longtable}{llp{5cm}p{10cm}rp{3cm}lr}
\rowcolor{white}\caption{PAPER without Concepts}\\ \toprule
\rowcolor{white}Key & \shortstack{Local\\Copy} & Authors & Title & Year & \shortstack{Conference\\/Journal} & Cite & Pages\\ \midrule
\endhead
\bottomrule
\endfoot
BaptisteLV92 & \href{../works/BaptisteLV92.pdf}{Yes} & \hyperref[auth:a703]{P. Baptiste}, \hyperref[auth:a704]{B. Legeard}, \hyperref[auth:a702]{C. Varnier} & Hoist scheduling problem: an approach based on constraint logic programming & 1992 & ICRA 1992 & \cite{BaptisteLV92} & 6\\DincbasHSAGB88 & \href{../works/DincbasHSAGB88.pdf}{Yes} & \hyperref[auth:a726]{M. Dincbas}, \hyperref[auth:a149]{Pascal Van Hentenryck}, \hyperref[auth:a17]{H. Simonis}, \hyperref[auth:a734]{A. Aggoun}, T. Graf, F. Berthier & The Constraint Logic Programming Language {CHIP} & 1988 & FGCS 1988 & \cite{DincbasHSAGB88} & 10\\\end{longtable}
}



{\scriptsize
\begin{longtable}{llp{5cm}p{10cm}rp{3cm}lr}
\rowcolor{white}\caption{ARTICLE without Concepts}\\ \toprule
\rowcolor{white}Key & \shortstack{Local\\Copy} & Authors & Title & Year & \shortstack{Conference\\/Journal} & Cite & Pages\\ \midrule
\endhead
\bottomrule
\endfoot
KorbaaYG00 & \href{../works/KorbaaYG00.pdf}{Yes} & \hyperref[auth:a690]{O. Korbaa}, \hyperref[auth:a691]{P. Yim}, \hyperref[auth:a692]{J. Gentina} & Solving Transient Scheduling Problems with Constraint Programming & 2000 & Eur. J. Control & \cite{KorbaaYG00} & 10\\LopezAKYG00 & \href{../works/LopezAKYG00.pdf}{Yes} & \hyperref[auth:a3]{P. Lopez}, \hyperref[auth:a693]{H. Alla}, \hyperref[auth:a690]{O. Korbaa}, \hyperref[auth:a691]{P. Yim}, \hyperref[auth:a692]{J. Gentina} & Discussion on: 'Solving Transient Scheduling Problems with Constraint Programming' by O. Korbaa, P. Yim, and {J.-C.} Gentina & 2000 & Eur. J. Control & \cite{LopezAKYG00} & 4\\CarlierP94 & \href{../works/CarlierP94.pdf}{Yes} & \hyperref[auth:a857]{J. Carlier}, \hyperref[auth:a858]{E. Pinson} & Adjustment of heads and tails for the job-shop problem & 1994 & European Journal of Operational Research & \cite{CarlierP94} & 16\\ApplegateC91 & \href{../works/ApplegateC91.pdf}{Yes} & D. Applegate, W. Cook & A Computational Study of the Job-Shop Scheduling Problem & 1991 & ORSA Journal on Computing & \cite{ApplegateC91} & 8\\\end{longtable}
}



\clearpage
\section{Unmatched Concepts}

This section lists those concepts for which no matches were found. The most likely cause is a mistake in the regular expression used to find the concept, but it is also possible that some concept simply is not mentioned in any of the documents. 

{\scriptsize
\begin{longtable}{lp{10cm}rr}
\caption{Unmatched Concepts}\\ \toprule
Type & Name & CaseSensitive & Revision\\ \midrule
\endhead
\bottomrule
\endfoot
ProgLanguages & Julia &  & 0\\Industries & steel making industry &  & 0\\ApplicationAreas & datacentre &  & 0\\ApplicationAreas & day-ahead market &  & 0\\ApplicationAreas & deep space &  & 0\\ApplicationAreas & ship building &  & 0\\ApplicationAreas & vaccine &  & 0\\Classification & Modified Generalized Assignment Problem &  & 0\\Classification & PP-MS-MMRCPSP & Y & 1\\Classification & Pre-emptive Job-Shop scheduling Problem &  & 0\\Classification & Resource-constrained Project Scheduling Problem with Discounted Cashflow &  & 0\\Classification & SMSDP & Y & 1\\Classification & Steel-making and continuous casting &  & 0\\Concepts & Allen's algebra &  & 0\\Concepts & make to stock &  & 0\\\end{longtable}
}



\clearpage
\section{Works by Author}

\subsection{Works by J. Christopher Beck}
\label{sec:a89}
{\scriptsize
\begin{longtable}{>{\raggedright\arraybackslash}p{3cm}>{\raggedright\arraybackslash}p{6cm}>{\raggedright\arraybackslash}p{6.5cm}rrrp{2.5cm}rrrrr}
\rowcolor{white}\caption{Works from bibtex (Total 46)}\\ \toprule
\rowcolor{white}Key & Authors & Title & LC & Cite & Year & \shortstack{Conference\\/Journal} & Pages & \shortstack{Nr\\Cites} & \shortstack{Nr\\Refs} & b & c \\ \midrule\endhead
\bottomrule
\endfoot
LuoB22 \href{https://doi.org/10.1007/978-3-031-08011-1\_17}{LuoB22} & \hyperref[auth:a754]{Yiqing L. Luo}, \hyperref[auth:a89]{J. Christopher Beck} & Packing by Scheduling: Using Constraint Programming to Solve a Complex 2D Cutting Stock Problem & \href{works/LuoB22.pdf}{Yes} & \cite{LuoB22} & 2022 & CPAIOR 2022 & 17 & 0 & 28 & \ref{b:LuoB22} & \ref{c:LuoB22}\\
ZhangBB22 \href{https://ojs.aaai.org/index.php/ICAPS/article/view/19826}{ZhangBB22} & \hyperref[auth:a808]{J. Zhang}, \hyperref[auth:a809]{Giovanni Lo Bianco}, \hyperref[auth:a89]{J. Christopher Beck} & Solving Job-Shop Scheduling Problems with QUBO-Based Specialized Hardware & \href{works/ZhangBB22.pdf}{Yes} & \cite{ZhangBB22} & 2022 & ICAPS 2022 & 9 & 0 & 0 & \ref{b:ZhangBB22} & \ref{c:ZhangBB22}\\
TangB20 \href{https://doi.org/10.1007/978-3-030-58942-4\_28}{TangB20} & \hyperref[auth:a88]{Tanya Y. Tang}, \hyperref[auth:a89]{J. Christopher Beck} & {CP} and Hybrid Models for Two-Stage Batching and Scheduling & \href{works/TangB20.pdf}{Yes} & \cite{TangB20} & 2020 & CPAIOR 2020 & 16 & 6 & 12 & \ref{b:TangB20} & \ref{c:TangB20}\\
TranPZLDB18 \href{https://doi.org/10.1007/s10951-017-0537-x}{TranPZLDB18} & \hyperref[auth:a810]{Tony T. Tran}, \hyperref[auth:a811]{M. Padmanabhan}, \hyperref[auth:a812]{Peter Yun Zhang}, \hyperref[auth:a813]{H. Li}, \hyperref[auth:a814]{Douglas G. Down}, \hyperref[auth:a89]{J. Christopher Beck} & Multi-stage resource-aware scheduling for data centers with heterogeneous servers & \href{works/TranPZLDB18.pdf}{Yes} & \cite{TranPZLDB18} & 2018 & J. Sched. & 17 & 8 & 26 & \ref{b:TranPZLDB18} & \ref{c:TranPZLDB18}\\
CohenHB17 \href{https://doi.org/10.1007/978-3-319-66263-3\_10}{CohenHB17} & \hyperref[auth:a816]{E. Cohen}, \hyperref[auth:a817]{G. Huang}, \hyperref[auth:a89]{J. Christopher Beck} & {(I} Can Get) Satisfaction: Preference-Based Scheduling for Concert-Goers at Multi-venue Music Festivals & \href{works/CohenHB17.pdf}{Yes} & \cite{CohenHB17} & 2017 & SAT 2017 & 17 & 1 & 12 & \ref{b:CohenHB17} & \ref{c:CohenHB17}\\
TranVNB17 \href{https://doi.org/10.1613/jair.5306}{TranVNB17} & \hyperref[auth:a810]{Tony T. Tran}, \hyperref[auth:a815]{Tiago Stegun Vaquero}, \hyperref[auth:a209]{G. Nejat}, \hyperref[auth:a89]{J. Christopher Beck} & Robots in Retirement Homes: Applying Off-the-Shelf Planning and Scheduling to a Team of Assistive Robots & \href{works/TranVNB17.pdf}{Yes} & \cite{TranVNB17} & 2017 & J. Artif. Intell. Res. & 68 & 12 & 0 & \ref{b:TranVNB17} & \ref{c:TranVNB17}\\
TranVNB17a \href{https://doi.org/10.24963/ijcai.2017/726}{TranVNB17a} & \hyperref[auth:a810]{Tony T. Tran}, \hyperref[auth:a815]{Tiago Stegun Vaquero}, \hyperref[auth:a209]{G. Nejat}, \hyperref[auth:a89]{J. Christopher Beck} & Robots in Retirement Homes: Applying Off-the-Shelf Planning and Scheduling to a Team of Assistive Robots (Extended Abstract) & \href{works/TranVNB17a.pdf}{Yes} & \cite{TranVNB17a} & 2017 & IJCAI 2017 & 5 & 1 & 0 & \ref{b:TranVNB17a} & \ref{c:TranVNB17a}\\
BoothNB16 \href{https://doi.org/10.1007/978-3-319-44953-1\_34}{BoothNB16} & \hyperref[auth:a208]{Kyle E. C. Booth}, \hyperref[auth:a209]{G. Nejat}, \hyperref[auth:a89]{J. Christopher Beck} & A Constraint Programming Approach to Multi-Robot Task Allocation and Scheduling in Retirement Homes & \href{works/BoothNB16.pdf}{Yes} & \cite{BoothNB16} & 2016 & CP 2016 & 17 & 21 & 24 & \ref{b:BoothNB16} & \ref{c:BoothNB16}\\
KuB16 \href{https://doi.org/10.1016/j.cor.2016.04.006}{KuB16} & \hyperref[auth:a336]{W. Ku}, \hyperref[auth:a89]{J. Christopher Beck} & Mixed Integer Programming models for job shop scheduling: {A} computational analysis & No & \cite{KuB16} & 2016 & Comput. Oper. Res. & 9 & 119 & 17 & No & \ref{c:KuB16}\\
LuoVLBM16 \href{http://www.aaai.org/ocs/index.php/KR/KR16/paper/view/12909}{LuoVLBM16} & \hyperref[auth:a824]{R. Luo}, \hyperref[auth:a825]{Richard Anthony Valenzano}, \hyperref[auth:a826]{Y. Li}, \hyperref[auth:a89]{J. Christopher Beck}, \hyperref[auth:a827]{Sheila A. McIlraith} & Using Metric Temporal Logic to Specify Scheduling Problems & \href{works/LuoVLBM16.pdf}{Yes} & \cite{LuoVLBM16} & 2016 & KR 2016 & 4 & 0 & 0 & \ref{b:LuoVLBM16} & \ref{c:LuoVLBM16}\\
TranAB16 \href{https://doi.org/10.1287/ijoc.2015.0666}{TranAB16} & \hyperref[auth:a810]{Tony T. Tran}, \hyperref[auth:a818]{A. Araujo}, \hyperref[auth:a89]{J. Christopher Beck} & Decomposition Methods for the Parallel Machine Scheduling Problem with Setups & No & \cite{TranAB16} & 2016 & {INFORMS} J. Comput. & 13 & 72 & 28 & No & \ref{c:TranAB16}\\
TranDRFWOVB16 \href{https://doi.org/10.1609/socs.v7i1.18390}{TranDRFWOVB16} & \hyperref[auth:a810]{Tony T. Tran}, \hyperref[auth:a820]{M. Do}, \hyperref[auth:a821]{Eleanor Gilbert Rieffel}, \hyperref[auth:a383]{J. Frank}, \hyperref[auth:a819]{Z. Wang}, \hyperref[auth:a822]{B. O'Gorman}, \hyperref[auth:a823]{D. Venturelli}, \hyperref[auth:a89]{J. Christopher Beck} & A Hybrid Quantum-Classical Approach to Solving Scheduling Problems & \href{works/TranDRFWOVB16.pdf}{Yes} & \cite{TranDRFWOVB16} & 2016 & SOCS 2016 & 9 & 3 & 0 & \ref{b:TranDRFWOVB16} & \ref{c:TranDRFWOVB16}\\
TranWDRFOVB16 \href{http://www.aaai.org/ocs/index.php/WS/AAAIW16/paper/view/12664}{TranWDRFOVB16} & \hyperref[auth:a810]{Tony T. Tran}, \hyperref[auth:a819]{Z. Wang}, \hyperref[auth:a820]{M. Do}, \hyperref[auth:a821]{Eleanor Gilbert Rieffel}, \hyperref[auth:a383]{J. Frank}, \hyperref[auth:a822]{B. O'Gorman}, \hyperref[auth:a823]{D. Venturelli}, \hyperref[auth:a89]{J. Christopher Beck} & Explorations of Quantum-Classical Approaches to Scheduling a Mars Lander Activity Problem & \href{works/TranWDRFOVB16.pdf}{Yes} & \cite{TranWDRFOVB16} & 2016 & AAAI 2016 & 9 & 0 & 0 & \ref{b:TranWDRFOVB16} & \ref{c:TranWDRFOVB16}\\
BajestaniB15 \href{https://doi.org/10.1007/s10951-015-0416-2}{BajestaniB15} & \hyperref[auth:a828]{Maliheh Aramon Bajestani}, \hyperref[auth:a89]{J. Christopher Beck} & A two-stage coupled algorithm for an integrated maintenance planning and flowshop scheduling problem with deteriorating machines & \href{works/BajestaniB15.pdf}{Yes} & \cite{BajestaniB15} & 2015 & J. Sched. & 16 & 17 & 59 & \ref{b:BajestaniB15} & \ref{c:BajestaniB15}\\
KoschB14 \href{https://doi.org/10.1007/978-3-319-07046-9\_5}{KoschB14} & \hyperref[auth:a332]{S. Kosch}, \hyperref[auth:a89]{J. Christopher Beck} & A New {MIP} Model for Parallel-Batch Scheduling with Non-identical Job Sizes & \href{works/KoschB14.pdf}{Yes} & \cite{KoschB14} & 2014 & CPAIOR 2014 & 16 & 4 & 18 & \ref{b:KoschB14} & \ref{c:KoschB14}\\
LouieVNB14 \href{https://doi.org/10.1109/ICRA.2014.6907637}{LouieVNB14} & \hyperref[auth:a830]{Wing{-}Yue Geoffrey Louie}, \hyperref[auth:a815]{Tiago Stegun Vaquero}, \hyperref[auth:a209]{G. Nejat}, \hyperref[auth:a89]{J. Christopher Beck} & An autonomous assistive robot for planning, scheduling and facilitating multi-user activities & No & \cite{LouieVNB14} & 2014 & ICRA 2014 & 7 & 16 & 9 & No & \ref{c:LouieVNB14}\\
TerekhovTDB14 \href{https://doi.org/10.1613/jair.4278}{TerekhovTDB14} & \hyperref[auth:a829]{D. Terekhov}, \hyperref[auth:a810]{Tony T. Tran}, \hyperref[auth:a814]{Douglas G. Down}, \hyperref[auth:a89]{J. Christopher Beck} & Integrating Queueing Theory and Scheduling for Dynamic Scheduling Problems & \href{works/TerekhovTDB14.pdf}{Yes} & \cite{TerekhovTDB14} & 2014 & J. Artif. Intell. Res. & 38 & 12 & 0 & \ref{b:TerekhovTDB14} & \ref{c:TerekhovTDB14}\\
BajestaniB13 \href{https://doi.org/10.1613/jair.3902}{BajestaniB13} & \hyperref[auth:a828]{Maliheh Aramon Bajestani}, \hyperref[auth:a89]{J. Christopher Beck} & Scheduling a Dynamic Aircraft Repair Shop with Limited Repair Resources & \href{works/BajestaniB13.pdf}{Yes} & \cite{BajestaniB13} & 2013 & J. Artif. Intell. Res. & 36 & 14 & 0 & \ref{b:BajestaniB13} & \ref{c:BajestaniB13}\\
HeinzKB13 \href{https://doi.org/10.1007/978-3-642-38171-3\_2}{HeinzKB13} & \hyperref[auth:a133]{S. Heinz}, \hyperref[auth:a336]{W. Ku}, \hyperref[auth:a89]{J. Christopher Beck} & Recent Improvements Using Constraint Integer Programming for Resource Allocation and Scheduling & \href{works/HeinzKB13.pdf}{Yes} & \cite{HeinzKB13} & 2013 & CPAIOR 2013 & 16 & 9 & 15 & \ref{b:HeinzKB13} & \ref{c:HeinzKB13}\\
HeinzSB13 \href{https://doi.org/10.1007/s10601-012-9136-9}{HeinzSB13} & \hyperref[auth:a133]{S. Heinz}, \hyperref[auth:a134]{J. Schulz}, \hyperref[auth:a89]{J. Christopher Beck} & Using dual presolving reductions to reformulate cumulative constraints & \href{works/HeinzSB13.pdf}{Yes} & \cite{HeinzSB13} & 2013 & Constraints An Int. J. & 36 & 7 & 31 & \ref{b:HeinzSB13} & \ref{c:HeinzSB13}\\
TranTDB13 \href{http://www.aaai.org/ocs/index.php/ICAPS/ICAPS13/paper/view/6005}{TranTDB13} & \hyperref[auth:a810]{Tony T. Tran}, \hyperref[auth:a829]{D. Terekhov}, \hyperref[auth:a814]{Douglas G. Down}, \hyperref[auth:a89]{J. Christopher Beck} & Hybrid Queueing Theory and Scheduling Models for Dynamic Environments with Sequence-Dependent Setup Times & \href{works/TranTDB13.pdf}{Yes} & \cite{TranTDB13} & 2013 & ICAPS 2013 & 9 & 0 & 0 & \ref{b:TranTDB13} & \ref{c:TranTDB13}\\
HeinzB12 \href{https://doi.org/10.1007/978-3-642-29828-8\_14}{HeinzB12} & \hyperref[auth:a133]{S. Heinz}, \hyperref[auth:a89]{J. Christopher Beck} & Reconsidering Mixed Integer Programming and MIP-Based Hybrids for Scheduling & \href{works/HeinzB12.pdf}{Yes} & \cite{HeinzB12} & 2012 & CPAIOR 2012 & 17 & 8 & 21 & \ref{b:HeinzB12} & \ref{c:HeinzB12}\\
TerekhovDOB12 \href{https://doi.org/10.1016/j.cie.2012.02.006}{TerekhovDOB12} & \hyperref[auth:a829]{D. Terekhov}, \hyperref[auth:a831]{Mustafa K. Dogru}, \hyperref[auth:a832]{U. {\"{O}}zen}, \hyperref[auth:a89]{J. Christopher Beck} & Solving two-machine assembly scheduling problems with inventory constraints & No & \cite{TerekhovDOB12} & 2012 & Comput. Ind. Eng. & 15 & 8 & 48 & No & \ref{c:TerekhovDOB12}\\
TranB12 \href{https://doi.org/10.3233/978-1-61499-098-7-774}{TranB12} & \hyperref[auth:a810]{Tony T. Tran}, \hyperref[auth:a89]{J. Christopher Beck} & Logic-based Benders Decomposition for Alternative Resource Scheduling with Sequence Dependent Setups & \href{works/TranB12.pdf}{Yes} & \cite{TranB12} & 2012 & ECAI 2012 & 6 & 0 & 0 & \ref{b:TranB12} & \ref{c:TranB12}\\
BajestaniB11 \href{http://aaai.org/ocs/index.php/ICAPS/ICAPS11/paper/view/2680}{BajestaniB11} & \hyperref[auth:a828]{Maliheh Aramon Bajestani}, \hyperref[auth:a89]{J. Christopher Beck} & Scheduling an Aircraft Repair Shop & \href{works/BajestaniB11.pdf}{Yes} & \cite{BajestaniB11} & 2011 & ICAPS 2011 & 8 & 0 & 0 & \ref{b:BajestaniB11} & \ref{c:BajestaniB11}\\
BeckFW11 \href{https://doi.org/10.1287/ijoc.1100.0388}{BeckFW11} & \hyperref[auth:a89]{J. Christopher Beck}, \hyperref[auth:a833]{T. K. Feng}, \hyperref[auth:a364]{J. Watson} & Combining Constraint Programming and Local Search for Job-Shop Scheduling & \href{works/BeckFW11.pdf}{Yes} & \cite{BeckFW11} & 2011 & {INFORMS} J. Comput. & 14 & 43 & 23 & \ref{b:BeckFW11} & \ref{c:BeckFW11}\\
HeckmanB11 \href{https://doi.org/10.1007/s10951-009-0113-0}{HeckmanB11} & \hyperref[auth:a834]{I. Heckman}, \hyperref[auth:a89]{J. Christopher Beck} & Understanding the behavior of Solution-Guided Search for job-shop scheduling & \href{works/HeckmanB11.pdf}{Yes} & \cite{HeckmanB11} & 2011 & J. Sched. & 20 & 0 & 22 & \ref{b:HeckmanB11} & \ref{c:HeckmanB11}\\
KovacsB11 \href{https://doi.org/10.1007/s10601-009-9088-x}{KovacsB11} & \hyperref[auth:a146]{A. Kov{\'{a}}cs}, \hyperref[auth:a89]{J. Christopher Beck} & A global constraint for total weighted completion time for unary resources & \href{works/KovacsB11.pdf}{Yes} & \cite{KovacsB11} & 2011 & Constraints An Int. J. & 24 & 4 & 26 & \ref{b:KovacsB11} & \ref{c:KovacsB11}\\
BidotVLB09 \href{https://doi.org/10.1007/s10951-008-0080-x}{BidotVLB09} & \hyperref[auth:a835]{J. Bidot}, \hyperref[auth:a836]{T. Vidal}, \hyperref[auth:a118]{P. Laborie}, \hyperref[auth:a89]{J. Christopher Beck} & A theoretic and practical framework for scheduling in a stochastic environment & \href{works/BidotVLB09.pdf}{Yes} & \cite{BidotVLB09} & 2009 & J. Sched. & 30 & 58 & 20 & \ref{b:BidotVLB09} & \ref{c:BidotVLB09}\\
WuBB09 \href{https://doi.org/10.1016/j.cor.2008.08.008}{WuBB09} & \hyperref[auth:a276]{Christine Wei Wu}, \hyperref[auth:a222]{Kenneth N. Brown}, \hyperref[auth:a89]{J. Christopher Beck} & Scheduling with uncertain durations: Modeling beta-robust scheduling with constraints & No & \cite{WuBB09} & 2009 & Comput. Oper. Res. & 9 & 42 & 5 & No & \ref{c:WuBB09}\\
KovacsB08 \href{https://doi.org/10.1016/j.engappai.2008.03.004}{KovacsB08} & \hyperref[auth:a146]{A. Kov{\'{a}}cs}, \hyperref[auth:a89]{J. Christopher Beck} & A global constraint for total weighted completion time for cumulative resources & \href{works/KovacsB08.pdf}{Yes} & \cite{KovacsB08} & 2008 & Eng. Appl. Artif. Intell. & 7 & 5 & 14 & \ref{b:KovacsB08} & \ref{c:KovacsB08}\\
WatsonB08 \href{https://doi.org/10.1007/978-3-540-68155-7\_21}{WatsonB08} & \hyperref[auth:a364]{J. Watson}, \hyperref[auth:a89]{J. Christopher Beck} & A Hybrid Constraint Programming / Local Search Approach to the Job-Shop Scheduling Problem & \href{works/WatsonB08.pdf}{Yes} & \cite{WatsonB08} & 2008 & CPAIOR 2008 & 15 & 14 & 17 & \ref{b:WatsonB08} & \ref{c:WatsonB08}\\
Beck07 \href{https://doi.org/10.1613/jair.2169}{Beck07} & \hyperref[auth:a89]{J. Christopher Beck} & Solution-Guided Multi-Point Constructive Search for Job Shop Scheduling & \href{works/Beck07.pdf}{Yes} & \cite{Beck07} & 2007 & J. Artif. Intell. Res. & 29 & 34 & 0 & \ref{b:Beck07} & \ref{c:Beck07}\\
BeckW07 \href{https://doi.org/10.1613/jair.2080}{BeckW07} & \hyperref[auth:a89]{J. Christopher Beck}, \hyperref[auth:a837]{N. Wilson} & Proactive Algorithms for Job Shop Scheduling with Probabilistic Durations & \href{works/BeckW07.pdf}{Yes} & \cite{BeckW07} & 2007 & J. Artif. Intell. Res. & 50 & 27 & 0 & \ref{b:BeckW07} & \ref{c:BeckW07}\\
KovacsB07 \href{https://doi.org/10.1007/978-3-540-72397-4\_9}{KovacsB07} & \hyperref[auth:a146]{A. Kov{\'{a}}cs}, \hyperref[auth:a89]{J. Christopher Beck} & A Global Constraint for Total Weighted Completion Time & \href{works/KovacsB07.pdf}{Yes} & \cite{KovacsB07} & 2007 & CPAIOR 2007 & 15 & 2 & 12 & \ref{b:KovacsB07} & \ref{c:KovacsB07}\\
Beck06 \href{http://www.aaai.org/Library/ICAPS/2006/icaps06-028.php}{Beck06} & \hyperref[auth:a89]{J. Christopher Beck} & An Empirical Study of Multi-Point Constructive Search for Constraint-Based Scheduling & \href{works/Beck06.pdf}{Yes} & \cite{Beck06} & 2006 & ICAPS 2006 & 10 & 0 & 0 & \ref{b:Beck06} & \ref{c:Beck06}\\
BeckW05 \href{http://ijcai.org/Proceedings/05/Papers/0748.pdf}{BeckW05} & \hyperref[auth:a89]{J. Christopher Beck}, \hyperref[auth:a837]{N. Wilson} & Proactive Algorithms for Scheduling with Probabilistic Durations & \href{works/BeckW05.pdf}{Yes} & \cite{BeckW05} & 2005 & IJCAI 2005 & 6 & 0 & 0 & \ref{b:BeckW05} & \ref{c:BeckW05}\\
CarchraeBF05 \href{https://doi.org/10.1007/11564751\_80}{CarchraeBF05} & \hyperref[auth:a274]{T. Carchrae}, \hyperref[auth:a89]{J. Christopher Beck}, \hyperref[auth:a275]{Eugene C. Freuder} & Methods to Learn Abstract Scheduling Models & \href{works/CarchraeBF05.pdf}{Yes} & \cite{CarchraeBF05} & 2005 & CP 2005 & 1 & 0 & 0 & \ref{b:CarchraeBF05} & \ref{c:CarchraeBF05}\\
WuBB05 \href{https://doi.org/10.1007/11564751\_110}{WuBB05} & \hyperref[auth:a276]{Christine Wei Wu}, \hyperref[auth:a222]{Kenneth N. Brown}, \hyperref[auth:a89]{J. Christopher Beck} & Scheduling with Uncertain Start Dates & \href{works/WuBB05.pdf}{Yes} & \cite{WuBB05} & 2005 & CP 2005 & 1 & 0 & 0 & \ref{b:WuBB05} & \ref{c:WuBB05}\\
BeckW04 \href{}{BeckW04} & \hyperref[auth:a89]{J. Christopher Beck}, \hyperref[auth:a837]{N. Wilson} & Job Shop Scheduling with Probabilistic Durations & \href{works/BeckW04.pdf}{Yes} & \cite{BeckW04} & 2004 & ECAI 2004 & 5 & 0 & 0 & \ref{b:BeckW04} & \ref{c:BeckW04}\\
BeckPS03 \href{http://www.aaai.org/Library/ICAPS/2003/icaps03-027.php}{BeckPS03} & \hyperref[auth:a89]{J. Christopher Beck}, \hyperref[auth:a838]{P. Prosser}, \hyperref[auth:a839]{E. Selensky} & Vehicle Routing and Job Shop Scheduling: What's the Difference? & \href{works/BeckPS03.pdf}{Yes} & \cite{BeckPS03} & 2003 & ICAPS 2003 & 10 & 0 & 0 & \ref{b:BeckPS03} & \ref{c:BeckPS03}\\
BeckR03 \href{https://doi.org/10.1023/A:1021849405707}{BeckR03} & \hyperref[auth:a89]{J. Christopher Beck}, \hyperref[auth:a256]{P. Refalo} & A Hybrid Approach to Scheduling with Earliness and Tardiness Costs & \href{works/BeckR03.pdf}{Yes} & \cite{BeckR03} & 2003 & Ann. Oper. Res. & 23 & 29 & 0 & \ref{b:BeckR03} & \ref{c:BeckR03}\\
BeckF00 \href{https://doi.org/10.1016/S0004-3702(99)00099-5}{BeckF00} & \hyperref[auth:a89]{J. Christopher Beck}, \hyperref[auth:a304]{Mark S. Fox} & Dynamic problem structure analysis as a basis for constraint-directed scheduling heuristics & \href{works/BeckF00.pdf}{Yes} & \cite{BeckF00} & 2000 & Artif. Intell. & 51 & 24 & 19 & \ref{b:BeckF00} & \ref{c:BeckF00}\\
Beck99 \href{https://librarysearch.library.utoronto.ca/permalink/01UTORONTO\_INST/14bjeso/alma991106162342106196}{Beck99} & \hyperref[auth:a89]{J. Christopher Beck} & Texture measurements as a basis for heuristic commitment techniques in constraint-directed scheduling & \href{works/Beck99.pdf}{Yes} & \cite{Beck99} & 1999 & University of Toronto, Canada & 418 & 0 & 0 & \ref{b:Beck99} & \ref{c:Beck99}\\
BeckF98 \href{https://doi.org/10.1609/aimag.v19i4.1426}{BeckF98} & \hyperref[auth:a89]{J. Christopher Beck}, \hyperref[auth:a304]{Mark S. Fox} & A Generic Framework for Constraint-Directed Search and Scheduling & \href{works/BeckF98.pdf}{Yes} & \cite{BeckF98} & 1998 & {AI} Mag. & 30 & 0 & 0 & \ref{b:BeckF98} & \ref{c:BeckF98}\\
BeckDF97 \href{https://doi.org/10.1007/BFb0017455}{BeckDF97} & \hyperref[auth:a89]{J. Christopher Beck}, \hyperref[auth:a250]{Andrew J. Davenport}, \hyperref[auth:a304]{Mark S. Fox} & Five Pitfalls of Empirical Scheduling Research & \href{works/BeckDF97.pdf}{Yes} & \cite{BeckDF97} & 1997 & CP 1997 & 15 & 3 & 12 & \ref{b:BeckDF97} & \ref{c:BeckDF97}\\
\end{longtable}
}

\subsection{Works by Michela Milano}
\label{sec:a143}
{\scriptsize
\begin{longtable}{>{\raggedright\arraybackslash}p{3cm}>{\raggedright\arraybackslash}p{6cm}>{\raggedright\arraybackslash}p{6.5cm}rrrp{2.5cm}rrrrr}
\rowcolor{white}\caption{Works from bibtex (Total 24)}\\ \toprule
\rowcolor{white}Key & Authors & Title & LC & Cite & Year & \shortstack{Conference\\/Journal} & Pages & \shortstack{Nr\\Cites} & \shortstack{Nr\\Refs} & b & c \\ \midrule\endhead
\bottomrule
\endfoot
BorghesiBLMB18 \href{https://doi.org/10.1016/j.suscom.2018.05.007}{BorghesiBLMB18} & \hyperref[auth:a231]{A. Borghesi}, \hyperref[auth:a230]{A. Bartolini}, \hyperref[auth:a142]{M. Lombardi}, \hyperref[auth:a143]{M. Milano}, \hyperref[auth:a247]{L. Benini} & Scheduling-based power capping in high performance computing systems & \href{works/BorghesiBLMB18.pdf}{Yes} & \cite{BorghesiBLMB18} & 2018 & Sustain. Comput. Informatics Syst. & 13 & 11 & 22 & \ref{b:BorghesiBLMB18} & \ref{c:BorghesiBLMB18}\\
BonfiettiZLM16 \href{https://doi.org/10.1007/978-3-319-44953-1\_8}{BonfiettiZLM16} & \hyperref[auth:a203]{A. Bonfietti}, \hyperref[auth:a204]{A. Zanarini}, \hyperref[auth:a142]{M. Lombardi}, \hyperref[auth:a143]{M. Milano} & The Multirate Resource Constraint & \href{works/BonfiettiZLM16.pdf}{Yes} & \cite{BonfiettiZLM16} & 2016 & CP 2016 & 17 & 0 & 11 & \ref{b:BonfiettiZLM16} & \ref{c:BonfiettiZLM16}\\
BridiBLMB16 \href{https://doi.org/10.1109/TPDS.2016.2516997}{BridiBLMB16} & \hyperref[auth:a232]{T. Bridi}, \hyperref[auth:a230]{A. Bartolini}, \hyperref[auth:a142]{M. Lombardi}, \hyperref[auth:a143]{M. Milano}, \hyperref[auth:a247]{L. Benini} & A Constraint Programming Scheduler for Heterogeneous High-Performance Computing Machines & \href{works/BridiBLMB16.pdf}{Yes} & \cite{BridiBLMB16} & 2016 & {IEEE} Trans. Parallel Distributed Syst. & 14 & 17 & 22 & \ref{b:BridiBLMB16} & \ref{c:BridiBLMB16}\\
BridiLBBM16 \href{https://doi.org/10.3233/978-1-61499-672-9-1598}{BridiLBBM16} & \hyperref[auth:a232]{T. Bridi}, \hyperref[auth:a142]{M. Lombardi}, \hyperref[auth:a230]{A. Bartolini}, \hyperref[auth:a247]{L. Benini}, \hyperref[auth:a143]{M. Milano} & {DARDIS:} Distributed And Randomized DIspatching and Scheduling & \href{works/BridiLBBM16.pdf}{Yes} & \cite{BridiLBBM16} & 2016 & ECAI 2016 & 2 & 0 & 0 & \ref{b:BridiLBBM16} & \ref{c:BridiLBBM16}\\
LombardiBM15 \href{https://doi.org/10.1007/978-3-319-23219-5\_20}{LombardiBM15} & \hyperref[auth:a142]{M. Lombardi}, \hyperref[auth:a203]{A. Bonfietti}, \hyperref[auth:a143]{M. Milano} & Deterministic Estimation of the Expected Makespan of a {POS} Under Duration Uncertainty & \href{works/LombardiBM15.pdf}{Yes} & \cite{LombardiBM15} & 2015 & CP 2015 & 16 & 0 & 8 & \ref{b:LombardiBM15} & \ref{c:LombardiBM15}\\
BartoliniBBLM14 \href{https://doi.org/10.1007/978-3-319-10428-7\_55}{BartoliniBBLM14} & \hyperref[auth:a230]{A. Bartolini}, \hyperref[auth:a231]{A. Borghesi}, \hyperref[auth:a232]{T. Bridi}, \hyperref[auth:a142]{M. Lombardi}, \hyperref[auth:a143]{M. Milano} & Proactive Workload Dispatching on the {EURORA} Supercomputer & \href{works/BartoliniBBLM14.pdf}{Yes} & \cite{BartoliniBBLM14} & 2014 & CP 2014 & 16 & 12 & 3 & \ref{b:BartoliniBBLM14} & \ref{c:BartoliniBBLM14}\\
BonfiettiLBM14 \href{https://doi.org/10.1016/j.artint.2013.09.006}{BonfiettiLBM14} & \hyperref[auth:a203]{A. Bonfietti}, \hyperref[auth:a142]{M. Lombardi}, \hyperref[auth:a247]{L. Benini}, \hyperref[auth:a143]{M. Milano} & {CROSS} cyclic resource-constrained scheduling solver & \href{works/BonfiettiLBM14.pdf}{Yes} & \cite{BonfiettiLBM14} & 2014 & Artif. Intell. & 28 & 8 & 15 & \ref{b:BonfiettiLBM14} & \ref{c:BonfiettiLBM14}\\
BonfiettiLM14 \href{https://doi.org/10.1007/978-3-319-07046-9\_15}{BonfiettiLM14} & \hyperref[auth:a203]{A. Bonfietti}, \hyperref[auth:a142]{M. Lombardi}, \hyperref[auth:a143]{M. Milano} & Disregarding Duration Uncertainty in Partial Order Schedules? Yes, We Can! & \href{works/BonfiettiLM14.pdf}{Yes} & \cite{BonfiettiLM14} & 2014 & CPAIOR 2014 & 16 & 3 & 12 & \ref{b:BonfiettiLM14} & \ref{c:BonfiettiLM14}\\
BonfiettiLM13 \href{http://www.aaai.org/ocs/index.php/ICAPS/ICAPS13/paper/view/6050}{BonfiettiLM13} & \hyperref[auth:a203]{A. Bonfietti}, \hyperref[auth:a142]{M. Lombardi}, \hyperref[auth:a143]{M. Milano} & De-Cycling Cyclic Scheduling Problems & \href{works/BonfiettiLM13.pdf}{Yes} & \cite{BonfiettiLM13} & 2013 & ICAPS 2013 & 5 & 0 & 0 & \ref{b:BonfiettiLM13} & \ref{c:BonfiettiLM13}\\
LombardiM13 \href{http://www.aaai.org/ocs/index.php/ICAPS/ICAPS13/paper/view/6052}{LombardiM13} & \hyperref[auth:a142]{M. Lombardi}, \hyperref[auth:a143]{M. Milano} & A Min-Flow Algorithm for Minimal Critical Set Detection in Resource Constrained Project Scheduling & \href{works/LombardiM13.pdf}{Yes} & \cite{LombardiM13} & 2013 & ICAPS 2013 & 2 & 0 & 0 & \ref{b:LombardiM13} & \ref{c:LombardiM13}\\
BonfiettiLBM12 \href{https://doi.org/10.1007/978-3-642-29828-8\_6}{BonfiettiLBM12} & \hyperref[auth:a203]{A. Bonfietti}, \hyperref[auth:a142]{M. Lombardi}, \hyperref[auth:a247]{L. Benini}, \hyperref[auth:a143]{M. Milano} & Global Cyclic Cumulative Constraint & \href{works/BonfiettiLBM12.pdf}{Yes} & \cite{BonfiettiLBM12} & 2012 & CPAIOR 2012 & 16 & 2 & 11 & \ref{b:BonfiettiLBM12} & \ref{c:BonfiettiLBM12}\\
BonfiettiM12 \href{https://ceur-ws.org/Vol-926/paper2.pdf}{BonfiettiM12} & \hyperref[auth:a203]{A. Bonfietti}, \hyperref[auth:a143]{M. Milano} & A Constraint-based Approach to Cyclic Resource-Constrained Scheduling Problem & \href{works/BonfiettiM12.pdf}{Yes} & \cite{BonfiettiM12} & 2012 & DC SIAAI 2012 & 3 & 0 & 0 & \ref{b:BonfiettiM12} & \ref{c:BonfiettiM12}\\
LombardiM12 \href{https://doi.org/10.1007/s10601-011-9115-6}{LombardiM12} & \hyperref[auth:a142]{M. Lombardi}, \hyperref[auth:a143]{M. Milano} & Optimal methods for resource allocation and scheduling: a cross-disciplinary survey & \href{works/LombardiM12.pdf}{Yes} & \cite{LombardiM12} & 2012 & Constraints An Int. J. & 35 & 39 & 68 & \ref{b:LombardiM12} & \ref{c:LombardiM12}\\
LombardiM12a \href{https://doi.org/10.1016/j.artint.2011.12.001}{LombardiM12a} & \hyperref[auth:a142]{M. Lombardi}, \hyperref[auth:a143]{M. Milano} & A min-flow algorithm for Minimal Critical Set detection in Resource Constrained Project Scheduling & \href{works/LombardiM12a.pdf}{Yes} & \cite{LombardiM12a} & 2012 & Artif. Intell. & 10 & 3 & 13 & \ref{b:LombardiM12a} & \ref{c:LombardiM12a}\\
BeniniLMR11 \href{https://doi.org/10.1007/s10479-010-0718-x}{BeniniLMR11} & \hyperref[auth:a247]{L. Benini}, \hyperref[auth:a142]{M. Lombardi}, \hyperref[auth:a143]{M. Milano}, \hyperref[auth:a727]{M. Ruggiero} & Optimal resource allocation and scheduling for the {CELL} {BE} platform & \href{works/BeniniLMR11.pdf}{Yes} & \cite{BeniniLMR11} & 2011 & Ann. Oper. Res. & 27 & 18 & 16 & \ref{b:BeniniLMR11} & \ref{c:BeniniLMR11}\\
BonfiettiLBM11 \href{https://doi.org/10.1007/978-3-642-23786-7\_12}{BonfiettiLBM11} & \hyperref[auth:a203]{A. Bonfietti}, \hyperref[auth:a142]{M. Lombardi}, \hyperref[auth:a247]{L. Benini}, \hyperref[auth:a143]{M. Milano} & A Constraint Based Approach to Cyclic {RCPSP} & \href{works/BonfiettiLBM11.pdf}{Yes} & \cite{BonfiettiLBM11} & 2011 & CP 2011 & 15 & 3 & 14 & \ref{b:BonfiettiLBM11} & \ref{c:BonfiettiLBM11}\\
LombardiBMB11 \href{https://doi.org/10.1007/978-3-642-21311-3\_14}{LombardiBMB11} & \hyperref[auth:a142]{M. Lombardi}, \hyperref[auth:a203]{A. Bonfietti}, \hyperref[auth:a143]{M. Milano}, \hyperref[auth:a247]{L. Benini} & Precedence Constraint Posting for Cyclic Scheduling Problems & \href{works/LombardiBMB11.pdf}{Yes} & \cite{LombardiBMB11} & 2011 & CPAIOR 2011 & 17 & 1 & 13 & \ref{b:LombardiBMB11} & \ref{c:LombardiBMB11}\\
LombardiM10 \href{https://doi.org/10.1007/978-3-642-15396-9\_32}{LombardiM10} & \hyperref[auth:a142]{M. Lombardi}, \hyperref[auth:a143]{M. Milano} & Constraint Based Scheduling to Deal with Uncertain Durations and Self-Timed Execution & \href{works/LombardiM10.pdf}{Yes} & \cite{LombardiM10} & 2010 & CP 2010 & 15 & 1 & 11 & \ref{b:LombardiM10} & \ref{c:LombardiM10}\\
LombardiM10a \href{https://doi.org/10.1016/j.artint.2010.02.004}{LombardiM10a} & \hyperref[auth:a142]{M. Lombardi}, \hyperref[auth:a143]{M. Milano} & Allocation and scheduling of Conditional Task Graphs & \href{works/LombardiM10a.pdf}{Yes} & \cite{LombardiM10a} & 2010 & Artif. Intell. & 30 & 8 & 24 & \ref{b:LombardiM10a} & \ref{c:LombardiM10a}\\
LombardiM09 \href{https://doi.org/10.1007/978-3-642-04244-7\_45}{LombardiM09} & \hyperref[auth:a142]{M. Lombardi}, \hyperref[auth:a143]{M. Milano} & A Precedence Constraint Posting Approach for the {RCPSP} with Time Lags and Variable Durations & \href{works/LombardiM09.pdf}{Yes} & \cite{LombardiM09} & 2009 & CP 2009 & 15 & 7 & 12 & \ref{b:LombardiM09} & \ref{c:LombardiM09}\\
RuggieroBBMA09 \href{https://doi.org/10.1109/TCAD.2009.2013536}{RuggieroBBMA09} & \hyperref[auth:a727]{M. Ruggiero}, \hyperref[auth:a379]{D. Bertozzi}, \hyperref[auth:a247]{L. Benini}, \hyperref[auth:a143]{M. Milano}, \hyperref[auth:a728]{A. Andrei} & Reducing the Abstraction and Optimality Gaps in the Allocation and Scheduling for Variable Voltage/Frequency MPSoC Platforms & \href{works/RuggieroBBMA09.pdf}{Yes} & \cite{RuggieroBBMA09} & 2009 & {IEEE} Trans. Comput. Aided Des. Integr. Circuits Syst. & 14 & 9 & 27 & \ref{b:RuggieroBBMA09} & \ref{c:RuggieroBBMA09}\\
BeniniBGM06 \href{https://doi.org/10.1007/11757375\_6}{BeniniBGM06} & \hyperref[auth:a247]{L. Benini}, \hyperref[auth:a379]{D. Bertozzi}, \hyperref[auth:a380]{A. Guerri}, \hyperref[auth:a143]{M. Milano} & Allocation, Scheduling and Voltage Scaling on Energy Aware MPSoCs & \href{works/BeniniBGM06.pdf}{Yes} & \cite{BeniniBGM06} & 2006 & CPAIOR 2006 & 15 & 18 & 10 & \ref{b:BeniniBGM06} & \ref{c:BeniniBGM06}\\
LammaMM97 \href{https://doi.org/10.1016/S0954-1810(96)00002-7}{LammaMM97} & \hyperref[auth:a729]{E. Lamma}, \hyperref[auth:a730]{P. Mello}, \hyperref[auth:a143]{M. Milano} & A distributed constraint-based scheduler & \href{works/LammaMM97.pdf}{Yes} & \cite{LammaMM97} & 1997 & Artif. Intell. Eng. & 15 & 11 & 7 & \ref{b:LammaMM97} & \ref{c:LammaMM97}\\
BrusoniCLMMT96 \href{https://doi.org/10.1007/3-540-61286-6\_157}{BrusoniCLMMT96} & \hyperref[auth:a731]{V. Brusoni}, \hyperref[auth:a732]{L. Console}, \hyperref[auth:a729]{E. Lamma}, \hyperref[auth:a730]{P. Mello}, \hyperref[auth:a143]{M. Milano}, \hyperref[auth:a733]{P. Terenziani} & Resource-Based vs. Task-Based Approaches for Scheduling Problems & \href{works/BrusoniCLMMT96.pdf}{Yes} & \cite{BrusoniCLMMT96} & 1996 & ISMIS 1996 & 10 & 1 & 9 & \ref{b:BrusoniCLMMT96} & \ref{c:BrusoniCLMMT96}\\
\end{longtable}
}

\subsection{Works by Andreas Schutt}
\label{sec:a124}
{\scriptsize
\begin{longtable}{>{\raggedright\arraybackslash}p{3cm}>{\raggedright\arraybackslash}p{6cm}>{\raggedright\arraybackslash}p{6.5cm}rrrp{2.5cm}rrrrr}
\rowcolor{white}\caption{Works from bibtex (Total 24)}\\ \toprule
\rowcolor{white}Key & Authors & Title & LC & Cite & Year & \shortstack{Conference\\/Journal} & Pages & \shortstack{Nr\\Cites} & \shortstack{Nr\\Refs} & b & c \\ \midrule\endhead
\bottomrule
\endfoot
YangSS19 \href{https://doi.org/10.1007/978-3-030-19212-9\_42}{YangSS19} & \hyperref[auth:a311]{M. Yang}, \hyperref[auth:a124]{A. Schutt}, \hyperref[auth:a125]{Peter J. Stuckey} & Time Table Edge Finding with Energy Variables & \href{works/YangSS19.pdf}{Yes} & \cite{YangSS19} & 2019 & CPAIOR 2019 & 10 & 1 & 14 & \ref{b:YangSS19} & \ref{c:YangSS19}\\
GoldwaserS18 \href{https://doi.org/10.1613/jair.1.11268}{GoldwaserS18} & \hyperref[auth:a194]{A. Goldwaser}, \hyperref[auth:a124]{A. Schutt} & Optimal Torpedo Scheduling & \href{works/GoldwaserS18.pdf}{Yes} & \cite{GoldwaserS18} & 2018 & J. Artif. Intell. Res. & 32 & 8 & 0 & \ref{b:GoldwaserS18} & \ref{c:GoldwaserS18}\\
KreterSSZ18 \href{https://doi.org/10.1016/j.ejor.2017.10.014}{KreterSSZ18} & \hyperref[auth:a123]{S. Kreter}, \hyperref[auth:a124]{A. Schutt}, \hyperref[auth:a125]{Peter J. Stuckey}, \hyperref[auth:a803]{J. Zimmermann} & Mixed-integer linear programming and constraint programming formulations for solving resource availability cost problems & No & \cite{KreterSSZ18} & 2018 & Eur. J. Oper. Res. & 15 & 25 & 31 & No & \ref{c:KreterSSZ18}\\
MusliuSS18 \href{https://doi.org/10.1007/978-3-319-93031-2\_31}{MusliuSS18} & \hyperref[auth:a45]{N. Musliu}, \hyperref[auth:a124]{A. Schutt}, \hyperref[auth:a125]{Peter J. Stuckey} & Solver Independent Rotating Workforce Scheduling & \href{works/MusliuSS18.pdf}{Yes} & \cite{MusliuSS18} & 2018 & CPAIOR 2018 & 17 & 7 & 23 & \ref{b:MusliuSS18} & \ref{c:MusliuSS18}\\
GoldwaserS17 \href{https://doi.org/10.1007/978-3-319-66158-2\_22}{GoldwaserS17} & \hyperref[auth:a194]{A. Goldwaser}, \hyperref[auth:a124]{A. Schutt} & Optimal Torpedo Scheduling & \href{works/GoldwaserS17.pdf}{Yes} & \cite{GoldwaserS17} & 2017 & CP 2017 & 16 & 0 & 10 & \ref{b:GoldwaserS17} & \ref{c:GoldwaserS17}\\
KreterSS17 \href{https://doi.org/10.1007/s10601-016-9266-6}{KreterSS17} & \hyperref[auth:a123]{S. Kreter}, \hyperref[auth:a124]{A. Schutt}, \hyperref[auth:a125]{Peter J. Stuckey} & Using constraint programming for solving RCPSP/max-cal & \href{works/KreterSS17.pdf}{Yes} & \cite{KreterSS17} & 2017 & Constraints An Int. J. & 31 & 15 & 20 & \ref{b:KreterSS17} & \ref{c:KreterSS17}\\
YoungFS17 \href{https://doi.org/10.1007/978-3-319-66158-2\_20}{YoungFS17} & \hyperref[auth:a193]{Kenneth D. Young}, \hyperref[auth:a154]{T. Feydy}, \hyperref[auth:a124]{A. Schutt} & Constraint Programming Applied to the Multi-Skill Project Scheduling Problem & \href{works/YoungFS17.pdf}{Yes} & \cite{YoungFS17} & 2017 & CP 2017 & 10 & 6 & 21 & \ref{b:YoungFS17} & \ref{c:YoungFS17}\\
SchuttS16 \href{https://doi.org/10.1007/978-3-319-44953-1\_28}{SchuttS16} & \hyperref[auth:a124]{A. Schutt}, \hyperref[auth:a125]{Peter J. Stuckey} & Explaining Producer/Consumer Constraints & \href{works/SchuttS16.pdf}{Yes} & \cite{SchuttS16} & 2016 & CP 2016 & 17 & 3 & 23 & \ref{b:SchuttS16} & \ref{c:SchuttS16}\\
SzerediS16 \href{https://doi.org/10.1007/978-3-319-44953-1\_31}{SzerediS16} & \hyperref[auth:a205]{R. Szeredi}, \hyperref[auth:a124]{A. Schutt} & Modelling and Solving Multi-mode Resource-Constrained Project Scheduling & \href{works/SzerediS16.pdf}{Yes} & \cite{SzerediS16} & 2016 & CP 2016 & 10 & 9 & 14 & \ref{b:SzerediS16} & \ref{c:SzerediS16}\\
EvenSH15 \href{https://doi.org/10.1007/978-3-319-23219-5\_40}{EvenSH15} & \hyperref[auth:a219]{C. Even}, \hyperref[auth:a124]{A. Schutt}, \hyperref[auth:a148]{Pascal Van Hentenryck} & A Constraint Programming Approach for Non-preemptive Evacuation Scheduling & \href{works/EvenSH15.pdf}{Yes} & \cite{EvenSH15} & 2015 & CP 2015 & 18 & 3 & 12 & \ref{b:EvenSH15} & \ref{c:EvenSH15}\\
EvenSH15a \href{http://arxiv.org/abs/1505.02487}{EvenSH15a} & \hyperref[auth:a219]{C. Even}, \hyperref[auth:a124]{A. Schutt}, \hyperref[auth:a148]{Pascal Van Hentenryck} & A Constraint Programming Approach for Non-Preemptive Evacuation Scheduling & \href{works/EvenSH15a.pdf}{Yes} & \cite{EvenSH15a} & 2015 & CoRR & 16 & 0 & 0 & \ref{b:EvenSH15a} & \ref{c:EvenSH15a}\\
KreterSS15 \href{https://doi.org/10.1007/978-3-319-23219-5\_19}{KreterSS15} & \hyperref[auth:a123]{S. Kreter}, \hyperref[auth:a124]{A. Schutt}, \hyperref[auth:a125]{Peter J. Stuckey} & Modeling and Solving Project Scheduling with Calendars & \href{works/KreterSS15.pdf}{Yes} & \cite{KreterSS15} & 2015 & CP 2015 & 17 & 7 & 16 & \ref{b:KreterSS15} & \ref{c:KreterSS15}\\
ThiruvadyWGS14 \href{https://doi.org/10.1007/s10732-014-9260-3}{ThiruvadyWGS14} & \hyperref[auth:a400]{Dhananjay R. Thiruvady}, \hyperref[auth:a117]{M. Wallace}, \hyperref[auth:a341]{H. Gu}, \hyperref[auth:a124]{A. Schutt} & A Lagrangian relaxation and {ACO} hybrid for resource constrained project scheduling with discounted cash flows & \href{works/ThiruvadyWGS14.pdf}{Yes} & \cite{ThiruvadyWGS14} & 2014 & J. Heuristics & 34 & 19 & 18 & \ref{b:ThiruvadyWGS14} & \ref{c:ThiruvadyWGS14}\\
ChuGNSW13 \href{http://www.aaai.org/ocs/index.php/IJCAI/IJCAI13/paper/view/6878}{ChuGNSW13} & \hyperref[auth:a348]{G. Chu}, \hyperref[auth:a804]{S. Gaspers}, \hyperref[auth:a805]{N. Narodytska}, \hyperref[auth:a124]{A. Schutt}, \hyperref[auth:a278]{T. Walsh} & On the Complexity of Global Scheduling Constraints under Structural Restrictions & \href{works/ChuGNSW13.pdf}{Yes} & \cite{ChuGNSW13} & 2013 & IJCAI 2013 & 7 & 0 & 0 & \ref{b:ChuGNSW13} & \ref{c:ChuGNSW13}\\
GuSS13 \href{https://doi.org/10.1007/978-3-642-38171-3\_24}{GuSS13} & \hyperref[auth:a341]{H. Gu}, \hyperref[auth:a124]{A. Schutt}, \hyperref[auth:a125]{Peter J. Stuckey} & A Lagrangian Relaxation Based Forward-Backward Improvement Heuristic for Maximising the Net Present Value of Resource-Constrained Projects & \href{works/GuSS13.pdf}{Yes} & \cite{GuSS13} & 2013 & CPAIOR 2013 & 7 & 10 & 24 & \ref{b:GuSS13} & \ref{c:GuSS13}\\
SchuttFS13 \href{https://doi.org/10.1007/978-3-642-40627-0\_47}{SchuttFS13} & \hyperref[auth:a124]{A. Schutt}, \hyperref[auth:a154]{T. Feydy}, \hyperref[auth:a125]{Peter J. Stuckey} & Scheduling Optional Tasks with Explanation & \href{works/SchuttFS13.pdf}{Yes} & \cite{SchuttFS13} & 2013 & CP 2013 & 17 & 10 & 20 & \ref{b:SchuttFS13} & \ref{c:SchuttFS13}\\
SchuttFS13a \href{https://doi.org/10.1007/978-3-642-38171-3\_16}{SchuttFS13a} & \hyperref[auth:a124]{A. Schutt}, \hyperref[auth:a154]{T. Feydy}, \hyperref[auth:a125]{Peter J. Stuckey} & Explaining Time-Table-Edge-Finding Propagation for the Cumulative Resource Constraint & \href{works/SchuttFS13a.pdf}{Yes} & \cite{SchuttFS13a} & 2013 & CPAIOR 2013 & 17 & 20 & 27 & \ref{b:SchuttFS13a} & \ref{c:SchuttFS13a}\\
SchuttFSW13 \href{https://doi.org/10.1007/s10951-012-0285-x}{SchuttFSW13} & \hyperref[auth:a124]{A. Schutt}, \hyperref[auth:a154]{T. Feydy}, \hyperref[auth:a125]{Peter J. Stuckey}, \hyperref[auth:a155]{Mark G. Wallace} & Solving RCPSP/max by lazy clause generation & \href{works/SchuttFSW13.pdf}{Yes} & \cite{SchuttFSW13} & 2013 & J. Sched. & 17 & 43 & 23 & \ref{b:SchuttFSW13} & \ref{c:SchuttFSW13}\\
SchuttCSW12 \href{https://doi.org/10.1007/978-3-642-29828-8\_24}{SchuttCSW12} & \hyperref[auth:a124]{A. Schutt}, \hyperref[auth:a348]{G. Chu}, \hyperref[auth:a125]{Peter J. Stuckey}, \hyperref[auth:a155]{Mark G. Wallace} & Maximising the Net Present Value for Resource-Constrained Project Scheduling & \href{works/SchuttCSW12.pdf}{Yes} & \cite{SchuttCSW12} & 2012 & CPAIOR 2012 & 17 & 18 & 21 & \ref{b:SchuttCSW12} & \ref{c:SchuttCSW12}\\
SchuttFSW11 \href{https://doi.org/10.1007/s10601-010-9103-2}{SchuttFSW11} & \hyperref[auth:a124]{A. Schutt}, \hyperref[auth:a154]{T. Feydy}, \hyperref[auth:a125]{Peter J. Stuckey}, \hyperref[auth:a155]{Mark G. Wallace} & Explaining the cumulative propagator & \href{works/SchuttFSW11.pdf}{Yes} & \cite{SchuttFSW11} & 2011 & Constraints An Int. J. & 33 & 57 & 23 & \ref{b:SchuttFSW11} & \ref{c:SchuttFSW11}\\
SchuttW10 \href{https://doi.org/10.1007/978-3-642-15396-9\_36}{SchuttW10} & \hyperref[auth:a124]{A. Schutt}, \hyperref[auth:a51]{A. Wolf} & A New \emph{O}(\emph{n}\({}^{\mbox{2}}\)log\emph{n}) Not-First/Not-Last Pruning Algorithm for Cumulative Resource Constraints & \href{works/SchuttW10.pdf}{Yes} & \cite{SchuttW10} & 2010 & CP 2010 & 15 & 13 & 14 & \ref{b:SchuttW10} & \ref{c:SchuttW10}\\
abs-1009-0347 \href{http://arxiv.org/abs/1009.0347}{abs-1009-0347} & \hyperref[auth:a124]{A. Schutt}, \hyperref[auth:a154]{T. Feydy}, \hyperref[auth:a125]{Peter J. Stuckey}, \hyperref[auth:a155]{Mark G. Wallace} & Solving the Resource Constrained Project Scheduling Problem with Generalized Precedences by Lazy Clause Generation & \href{works/abs-1009-0347.pdf}{Yes} & \cite{abs-1009-0347} & 2010 & CoRR & 37 & 0 & 0 & \ref{b:abs-1009-0347} & \ref{c:abs-1009-0347}\\
SchuttFSW09 \href{https://doi.org/10.1007/978-3-642-04244-7\_58}{SchuttFSW09} & \hyperref[auth:a124]{A. Schutt}, \hyperref[auth:a154]{T. Feydy}, \hyperref[auth:a125]{Peter J. Stuckey}, \hyperref[auth:a117]{M. Wallace} & Why Cumulative Decomposition Is Not as Bad as It Sounds & \href{works/SchuttFSW09.pdf}{Yes} & \cite{SchuttFSW09} & 2009 & CP 2009 & 16 & 34 & 11 & \ref{b:SchuttFSW09} & \ref{c:SchuttFSW09}\\
SchuttWS05 \href{https://doi.org/10.1007/11963578\_6}{SchuttWS05} & \hyperref[auth:a124]{A. Schutt}, \hyperref[auth:a51]{A. Wolf}, \hyperref[auth:a720]{G. Schrader} & Not-First and Not-Last Detection for Cumulative Scheduling in \emph{O}(\emph{n}\({}^{\mbox{3}}\)log\emph{n}) & \href{works/SchuttWS05.pdf}{Yes} & \cite{SchuttWS05} & 2005 & INAP 2005 & 15 & 6 & 4 & \ref{b:SchuttWS05} & \ref{c:SchuttWS05}\\
\end{longtable}
}

\subsection{Works by Peter J. Stuckey}
\label{sec:a125}
{\scriptsize
\begin{longtable}{>{\raggedright\arraybackslash}p{3cm}>{\raggedright\arraybackslash}p{6cm}>{\raggedright\arraybackslash}p{6.5cm}rrrp{2.5cm}rrrrr}
\rowcolor{white}\caption{Works from bibtex (Total 23)}\\ \toprule
\rowcolor{white}Key & Authors & Title & LC & Cite & Year & \shortstack{Conference\\/Journal} & Pages & \shortstack{Nr\\Cites} & \shortstack{Nr\\Refs} & b & c \\ \midrule\endhead
\bottomrule
\endfoot
YangSS19 \href{https://doi.org/10.1007/978-3-030-19212-9\_42}{YangSS19} & \hyperref[auth:a311]{M. Yang}, \hyperref[auth:a124]{A. Schutt}, \hyperref[auth:a125]{Peter J. Stuckey} & Time Table Edge Finding with Energy Variables & \href{works/YangSS19.pdf}{Yes} & \cite{YangSS19} & 2019 & CPAIOR 2019 & 10 & 1 & 14 & \ref{b:YangSS19} & \ref{c:YangSS19}\\
DemirovicS18 \href{https://doi.org/10.1007/978-3-319-93031-2\_10}{DemirovicS18} & \hyperref[auth:a314]{E. Demirovic}, \hyperref[auth:a125]{Peter J. Stuckey} & Constraint Programming for High School Timetabling: {A} Scheduling-Based Model with Hot Starts & \href{works/DemirovicS18.pdf}{Yes} & \cite{DemirovicS18} & 2018 & CPAIOR 2018 & 18 & 4 & 16 & \ref{b:DemirovicS18} & \ref{c:DemirovicS18}\\
KreterSSZ18 \href{https://doi.org/10.1016/j.ejor.2017.10.014}{KreterSSZ18} & \hyperref[auth:a123]{S. Kreter}, \hyperref[auth:a124]{A. Schutt}, \hyperref[auth:a125]{Peter J. Stuckey}, \hyperref[auth:a803]{J. Zimmermann} & Mixed-integer linear programming and constraint programming formulations for solving resource availability cost problems & No & \cite{KreterSSZ18} & 2018 & Eur. J. Oper. Res. & 15 & 25 & 31 & No & \ref{c:KreterSSZ18}\\
MusliuSS18 \href{https://doi.org/10.1007/978-3-319-93031-2\_31}{MusliuSS18} & \hyperref[auth:a45]{N. Musliu}, \hyperref[auth:a124]{A. Schutt}, \hyperref[auth:a125]{Peter J. Stuckey} & Solver Independent Rotating Workforce Scheduling & \href{works/MusliuSS18.pdf}{Yes} & \cite{MusliuSS18} & 2018 & CPAIOR 2018 & 17 & 7 & 23 & \ref{b:MusliuSS18} & \ref{c:MusliuSS18}\\
KreterSS17 \href{https://doi.org/10.1007/s10601-016-9266-6}{KreterSS17} & \hyperref[auth:a123]{S. Kreter}, \hyperref[auth:a124]{A. Schutt}, \hyperref[auth:a125]{Peter J. Stuckey} & Using constraint programming for solving RCPSP/max-cal & \href{works/KreterSS17.pdf}{Yes} & \cite{KreterSS17} & 2017 & Constraints An Int. J. & 31 & 15 & 20 & \ref{b:KreterSS17} & \ref{c:KreterSS17}\\
BlomPS16 \href{https://doi.org/10.1287/mnsc.2015.2284}{BlomPS16} & \hyperref[auth:a806]{Michelle L. Blom}, \hyperref[auth:a327]{Adrian R. Pearce}, \hyperref[auth:a125]{Peter J. Stuckey} & A Decomposition-Based Algorithm for the Scheduling of Open-Pit Networks Over Multiple Time Periods & No & \cite{BlomPS16} & 2016 & Manag. Sci. & 26 & 20 & 36 & No & \ref{c:BlomPS16}\\
SchuttS16 \href{https://doi.org/10.1007/978-3-319-44953-1\_28}{SchuttS16} & \hyperref[auth:a124]{A. Schutt}, \hyperref[auth:a125]{Peter J. Stuckey} & Explaining Producer/Consumer Constraints & \href{works/SchuttS16.pdf}{Yes} & \cite{SchuttS16} & 2016 & CP 2016 & 17 & 3 & 23 & \ref{b:SchuttS16} & \ref{c:SchuttS16}\\
BurtLPS15 \href{https://doi.org/10.1007/978-3-319-18008-3\_7}{BurtLPS15} & \hyperref[auth:a325]{Christina N. Burt}, \hyperref[auth:a326]{N. Lipovetzky}, \hyperref[auth:a327]{Adrian R. Pearce}, \hyperref[auth:a125]{Peter J. Stuckey} & Scheduling with Fixed Maintenance, Shared Resources and Nonlinear Feedrate Constraints: {A} Mine Planning Case Study & \href{works/BurtLPS15.pdf}{Yes} & \cite{BurtLPS15} & 2015 & CPAIOR 2015 & 17 & 0 & 8 & \ref{b:BurtLPS15} & \ref{c:BurtLPS15}\\
KreterSS15 \href{https://doi.org/10.1007/978-3-319-23219-5\_19}{KreterSS15} & \hyperref[auth:a123]{S. Kreter}, \hyperref[auth:a124]{A. Schutt}, \hyperref[auth:a125]{Peter J. Stuckey} & Modeling and Solving Project Scheduling with Calendars & \href{works/KreterSS15.pdf}{Yes} & \cite{KreterSS15} & 2015 & CP 2015 & 17 & 7 & 16 & \ref{b:KreterSS15} & \ref{c:KreterSS15}\\
BlomBPS14 \href{https://doi.org/10.1287/ijoc.2013.0590}{BlomBPS14} & \hyperref[auth:a806]{Michelle L. Blom}, \hyperref[auth:a325]{Christina N. Burt}, \hyperref[auth:a327]{Adrian R. Pearce}, \hyperref[auth:a125]{Peter J. Stuckey} & A Decomposition-Based Heuristic for Collaborative Scheduling in a Network of Open-Pit Mines & No & \cite{BlomBPS14} & 2014 & {INFORMS} J. Comput. & 19 & 15 & 47 & No & \ref{c:BlomBPS14}\\
LipovetzkyBPS14 \href{http://www.aaai.org/ocs/index.php/ICAPS/ICAPS14/paper/view/7942}{LipovetzkyBPS14} & \hyperref[auth:a326]{N. Lipovetzky}, \hyperref[auth:a325]{Christina N. Burt}, \hyperref[auth:a327]{Adrian R. Pearce}, \hyperref[auth:a125]{Peter J. Stuckey} & Planning for Mining Operations with Time and Resource Constraints & \href{works/LipovetzkyBPS14.pdf}{Yes} & \cite{LipovetzkyBPS14} & 2014 & ICAPS 2014 & 9 & 0 & 0 & \ref{b:LipovetzkyBPS14} & \ref{c:LipovetzkyBPS14}\\
GuSS13 \href{https://doi.org/10.1007/978-3-642-38171-3\_24}{GuSS13} & \hyperref[auth:a341]{H. Gu}, \hyperref[auth:a124]{A. Schutt}, \hyperref[auth:a125]{Peter J. Stuckey} & A Lagrangian Relaxation Based Forward-Backward Improvement Heuristic for Maximising the Net Present Value of Resource-Constrained Projects & \href{works/GuSS13.pdf}{Yes} & \cite{GuSS13} & 2013 & CPAIOR 2013 & 7 & 10 & 24 & \ref{b:GuSS13} & \ref{c:GuSS13}\\
SchuttFS13 \href{https://doi.org/10.1007/978-3-642-40627-0\_47}{SchuttFS13} & \hyperref[auth:a124]{A. Schutt}, \hyperref[auth:a154]{T. Feydy}, \hyperref[auth:a125]{Peter J. Stuckey} & Scheduling Optional Tasks with Explanation & \href{works/SchuttFS13.pdf}{Yes} & \cite{SchuttFS13} & 2013 & CP 2013 & 17 & 10 & 20 & \ref{b:SchuttFS13} & \ref{c:SchuttFS13}\\
SchuttFS13a \href{https://doi.org/10.1007/978-3-642-38171-3\_16}{SchuttFS13a} & \hyperref[auth:a124]{A. Schutt}, \hyperref[auth:a154]{T. Feydy}, \hyperref[auth:a125]{Peter J. Stuckey} & Explaining Time-Table-Edge-Finding Propagation for the Cumulative Resource Constraint & \href{works/SchuttFS13a.pdf}{Yes} & \cite{SchuttFS13a} & 2013 & CPAIOR 2013 & 17 & 20 & 27 & \ref{b:SchuttFS13a} & \ref{c:SchuttFS13a}\\
SchuttFSW13 \href{https://doi.org/10.1007/s10951-012-0285-x}{SchuttFSW13} & \hyperref[auth:a124]{A. Schutt}, \hyperref[auth:a154]{T. Feydy}, \hyperref[auth:a125]{Peter J. Stuckey}, \hyperref[auth:a155]{Mark G. Wallace} & Solving RCPSP/max by lazy clause generation & \href{works/SchuttFSW13.pdf}{Yes} & \cite{SchuttFSW13} & 2013 & J. Sched. & 17 & 43 & 23 & \ref{b:SchuttFSW13} & \ref{c:SchuttFSW13}\\
GuSW12 \href{https://doi.org/10.1007/978-3-642-33558-7\_55}{GuSW12} & \hyperref[auth:a341]{H. Gu}, \hyperref[auth:a125]{Peter J. Stuckey}, \hyperref[auth:a155]{Mark G. Wallace} & Maximising the Net Present Value of Large Resource-Constrained Projects & \href{works/GuSW12.pdf}{Yes} & \cite{GuSW12} & 2012 & CP 2012 & 15 & 5 & 20 & \ref{b:GuSW12} & \ref{c:GuSW12}\\
SchuttCSW12 \href{https://doi.org/10.1007/978-3-642-29828-8\_24}{SchuttCSW12} & \hyperref[auth:a124]{A. Schutt}, \hyperref[auth:a348]{G. Chu}, \hyperref[auth:a125]{Peter J. Stuckey}, \hyperref[auth:a155]{Mark G. Wallace} & Maximising the Net Present Value for Resource-Constrained Project Scheduling & \href{works/SchuttCSW12.pdf}{Yes} & \cite{SchuttCSW12} & 2012 & CPAIOR 2012 & 17 & 18 & 21 & \ref{b:SchuttCSW12} & \ref{c:SchuttCSW12}\\
BandaSC11 \href{https://doi.org/10.1287/ijoc.1090.0378}{BandaSC11} & \hyperref[auth:a807]{Maria Garcia de la Banda}, \hyperref[auth:a125]{Peter J. Stuckey}, \hyperref[auth:a348]{G. Chu} & Solving Talent Scheduling with Dynamic Programming & No & \cite{BandaSC11} & 2011 & {INFORMS} J. Comput. & 18 & 24 & 17 & No & \ref{c:BandaSC11}\\
SchuttFSW11 \href{https://doi.org/10.1007/s10601-010-9103-2}{SchuttFSW11} & \hyperref[auth:a124]{A. Schutt}, \hyperref[auth:a154]{T. Feydy}, \hyperref[auth:a125]{Peter J. Stuckey}, \hyperref[auth:a155]{Mark G. Wallace} & Explaining the cumulative propagator & \href{works/SchuttFSW11.pdf}{Yes} & \cite{SchuttFSW11} & 2011 & Constraints An Int. J. & 33 & 57 & 23 & \ref{b:SchuttFSW11} & \ref{c:SchuttFSW11}\\
abs-1009-0347 \href{http://arxiv.org/abs/1009.0347}{abs-1009-0347} & \hyperref[auth:a124]{A. Schutt}, \hyperref[auth:a154]{T. Feydy}, \hyperref[auth:a125]{Peter J. Stuckey}, \hyperref[auth:a155]{Mark G. Wallace} & Solving the Resource Constrained Project Scheduling Problem with Generalized Precedences by Lazy Clause Generation & \href{works/abs-1009-0347.pdf}{Yes} & \cite{abs-1009-0347} & 2010 & CoRR & 37 & 0 & 0 & \ref{b:abs-1009-0347} & \ref{c:abs-1009-0347}\\
OhrimenkoSC09 \href{http://dx.doi.org/10.1007/s10601-008-9064-x}{OhrimenkoSC09} & \hyperref[auth:a875]{O. Ohrimenko}, \hyperref[auth:a125]{Peter J. Stuckey}, \hyperref[auth:a876]{M. Codish} & Propagation via lazy clause generation & \href{works/OhrimenkoSC09.pdf}{Yes} & \cite{OhrimenkoSC09} & 2009 & Constraints & 35 & 127 & 15 & \ref{b:OhrimenkoSC09} & \ref{c:OhrimenkoSC09}\\
SchuttFSW09 \href{https://doi.org/10.1007/978-3-642-04244-7\_58}{SchuttFSW09} & \hyperref[auth:a124]{A. Schutt}, \hyperref[auth:a154]{T. Feydy}, \hyperref[auth:a125]{Peter J. Stuckey}, \hyperref[auth:a117]{M. Wallace} & Why Cumulative Decomposition Is Not as Bad as It Sounds & \href{works/SchuttFSW09.pdf}{Yes} & \cite{SchuttFSW09} & 2009 & CP 2009 & 16 & 34 & 11 & \ref{b:SchuttFSW09} & \ref{c:SchuttFSW09}\\
NethercoteSBBDT07 \href{https://doi.org/10.1007/978-3-540-74970-7\_38}{NethercoteSBBDT07} & \hyperref[auth:a867]{N. Nethercote}, \hyperref[auth:a125]{Peter J. Stuckey}, \hyperref[auth:a868]{R. Becket}, \hyperref[auth:a869]{S. Brand}, \hyperref[auth:a870]{Gregory J. Duck}, \hyperref[auth:a871]{G. Tack} & MiniZinc: Towards a Standard {CP} Modelling Language & \href{works/NethercoteSBBDT07.pdf}{Yes} & \cite{NethercoteSBBDT07} & 2007 & CP 2007 & 15 & 344 & 5 & \ref{b:NethercoteSBBDT07} & \ref{c:NethercoteSBBDT07}\\
\end{longtable}
}

\subsection{Works by Michele Lombardi}
\label{sec:a142}
{\scriptsize
\begin{longtable}{>{\raggedright\arraybackslash}p{3cm}>{\raggedright\arraybackslash}p{6cm}>{\raggedright\arraybackslash}p{6.5cm}rrrp{2.5cm}rrrrr}
\rowcolor{white}\caption{Works from bibtex (Total 22)}\\ \toprule
\rowcolor{white}Key & Authors & Title & LC & Cite & Year & \shortstack{Conference\\/Journal} & Pages & \shortstack{Nr\\Cites} & \shortstack{Nr\\Refs} & b & c \\ \midrule\endhead
\bottomrule
\endfoot
BorghesiBLMB18 \href{https://doi.org/10.1016/j.suscom.2018.05.007}{BorghesiBLMB18} & \hyperref[auth:a231]{A. Borghesi}, \hyperref[auth:a230]{A. Bartolini}, \hyperref[auth:a142]{M. Lombardi}, \hyperref[auth:a143]{M. Milano}, \hyperref[auth:a247]{L. Benini} & Scheduling-based power capping in high performance computing systems & \href{works/BorghesiBLMB18.pdf}{Yes} & \cite{BorghesiBLMB18} & 2018 & Sustain. Comput. Informatics Syst. & 13 & 11 & 22 & \ref{b:BorghesiBLMB18} & \ref{c:BorghesiBLMB18}\\
CauwelaertLS18 \href{https://doi.org/10.1007/s10601-017-9277-y}{CauwelaertLS18} & \hyperref[auth:a206]{Sascha Van Cauwelaert}, \hyperref[auth:a142]{M. Lombardi}, \hyperref[auth:a147]{P. Schaus} & How efficient is a global constraint in practice? - {A} fair experimental framework & \href{works/CauwelaertLS18.pdf}{Yes} & \cite{CauwelaertLS18} & 2018 & Constraints An Int. J. & 36 & 2 & 39 & \ref{b:CauwelaertLS18} & \ref{c:CauwelaertLS18}\\
BonfiettiZLM16 \href{https://doi.org/10.1007/978-3-319-44953-1\_8}{BonfiettiZLM16} & \hyperref[auth:a203]{A. Bonfietti}, \hyperref[auth:a204]{A. Zanarini}, \hyperref[auth:a142]{M. Lombardi}, \hyperref[auth:a143]{M. Milano} & The Multirate Resource Constraint & \href{works/BonfiettiZLM16.pdf}{Yes} & \cite{BonfiettiZLM16} & 2016 & CP 2016 & 17 & 0 & 11 & \ref{b:BonfiettiZLM16} & \ref{c:BonfiettiZLM16}\\
BridiBLMB16 \href{https://doi.org/10.1109/TPDS.2016.2516997}{BridiBLMB16} & \hyperref[auth:a232]{T. Bridi}, \hyperref[auth:a230]{A. Bartolini}, \hyperref[auth:a142]{M. Lombardi}, \hyperref[auth:a143]{M. Milano}, \hyperref[auth:a247]{L. Benini} & A Constraint Programming Scheduler for Heterogeneous High-Performance Computing Machines & \href{works/BridiBLMB16.pdf}{Yes} & \cite{BridiBLMB16} & 2016 & {IEEE} Trans. Parallel Distributed Syst. & 14 & 17 & 22 & \ref{b:BridiBLMB16} & \ref{c:BridiBLMB16}\\
BridiLBBM16 \href{https://doi.org/10.3233/978-1-61499-672-9-1598}{BridiLBBM16} & \hyperref[auth:a232]{T. Bridi}, \hyperref[auth:a142]{M. Lombardi}, \hyperref[auth:a230]{A. Bartolini}, \hyperref[auth:a247]{L. Benini}, \hyperref[auth:a143]{M. Milano} & {DARDIS:} Distributed And Randomized DIspatching and Scheduling & \href{works/BridiLBBM16.pdf}{Yes} & \cite{BridiLBBM16} & 2016 & ECAI 2016 & 2 & 0 & 0 & \ref{b:BridiLBBM16} & \ref{c:BridiLBBM16}\\
LombardiBM15 \href{https://doi.org/10.1007/978-3-319-23219-5\_20}{LombardiBM15} & \hyperref[auth:a142]{M. Lombardi}, \hyperref[auth:a203]{A. Bonfietti}, \hyperref[auth:a143]{M. Milano} & Deterministic Estimation of the Expected Makespan of a {POS} Under Duration Uncertainty & \href{works/LombardiBM15.pdf}{Yes} & \cite{LombardiBM15} & 2015 & CP 2015 & 16 & 0 & 8 & \ref{b:LombardiBM15} & \ref{c:LombardiBM15}\\
BartoliniBBLM14 \href{https://doi.org/10.1007/978-3-319-10428-7\_55}{BartoliniBBLM14} & \hyperref[auth:a230]{A. Bartolini}, \hyperref[auth:a231]{A. Borghesi}, \hyperref[auth:a232]{T. Bridi}, \hyperref[auth:a142]{M. Lombardi}, \hyperref[auth:a143]{M. Milano} & Proactive Workload Dispatching on the {EURORA} Supercomputer & \href{works/BartoliniBBLM14.pdf}{Yes} & \cite{BartoliniBBLM14} & 2014 & CP 2014 & 16 & 12 & 3 & \ref{b:BartoliniBBLM14} & \ref{c:BartoliniBBLM14}\\
BonfiettiLBM14 \href{https://doi.org/10.1016/j.artint.2013.09.006}{BonfiettiLBM14} & \hyperref[auth:a203]{A. Bonfietti}, \hyperref[auth:a142]{M. Lombardi}, \hyperref[auth:a247]{L. Benini}, \hyperref[auth:a143]{M. Milano} & {CROSS} cyclic resource-constrained scheduling solver & \href{works/BonfiettiLBM14.pdf}{Yes} & \cite{BonfiettiLBM14} & 2014 & Artif. Intell. & 28 & 8 & 15 & \ref{b:BonfiettiLBM14} & \ref{c:BonfiettiLBM14}\\
BonfiettiLM14 \href{https://doi.org/10.1007/978-3-319-07046-9\_15}{BonfiettiLM14} & \hyperref[auth:a203]{A. Bonfietti}, \hyperref[auth:a142]{M. Lombardi}, \hyperref[auth:a143]{M. Milano} & Disregarding Duration Uncertainty in Partial Order Schedules? Yes, We Can! & \href{works/BonfiettiLM14.pdf}{Yes} & \cite{BonfiettiLM14} & 2014 & CPAIOR 2014 & 16 & 3 & 12 & \ref{b:BonfiettiLM14} & \ref{c:BonfiettiLM14}\\
BonfiettiLM13 \href{http://www.aaai.org/ocs/index.php/ICAPS/ICAPS13/paper/view/6050}{BonfiettiLM13} & \hyperref[auth:a203]{A. Bonfietti}, \hyperref[auth:a142]{M. Lombardi}, \hyperref[auth:a143]{M. Milano} & De-Cycling Cyclic Scheduling Problems & \href{works/BonfiettiLM13.pdf}{Yes} & \cite{BonfiettiLM13} & 2013 & ICAPS 2013 & 5 & 0 & 0 & \ref{b:BonfiettiLM13} & \ref{c:BonfiettiLM13}\\
LombardiM13 \href{http://www.aaai.org/ocs/index.php/ICAPS/ICAPS13/paper/view/6052}{LombardiM13} & \hyperref[auth:a142]{M. Lombardi}, \hyperref[auth:a143]{M. Milano} & A Min-Flow Algorithm for Minimal Critical Set Detection in Resource Constrained Project Scheduling & \href{works/LombardiM13.pdf}{Yes} & \cite{LombardiM13} & 2013 & ICAPS 2013 & 2 & 0 & 0 & \ref{b:LombardiM13} & \ref{c:LombardiM13}\\
BonfiettiLBM12 \href{https://doi.org/10.1007/978-3-642-29828-8\_6}{BonfiettiLBM12} & \hyperref[auth:a203]{A. Bonfietti}, \hyperref[auth:a142]{M. Lombardi}, \hyperref[auth:a247]{L. Benini}, \hyperref[auth:a143]{M. Milano} & Global Cyclic Cumulative Constraint & \href{works/BonfiettiLBM12.pdf}{Yes} & \cite{BonfiettiLBM12} & 2012 & CPAIOR 2012 & 16 & 2 & 11 & \ref{b:BonfiettiLBM12} & \ref{c:BonfiettiLBM12}\\
LombardiM12 \href{https://doi.org/10.1007/s10601-011-9115-6}{LombardiM12} & \hyperref[auth:a142]{M. Lombardi}, \hyperref[auth:a143]{M. Milano} & Optimal methods for resource allocation and scheduling: a cross-disciplinary survey & \href{works/LombardiM12.pdf}{Yes} & \cite{LombardiM12} & 2012 & Constraints An Int. J. & 35 & 39 & 68 & \ref{b:LombardiM12} & \ref{c:LombardiM12}\\
LombardiM12a \href{https://doi.org/10.1016/j.artint.2011.12.001}{LombardiM12a} & \hyperref[auth:a142]{M. Lombardi}, \hyperref[auth:a143]{M. Milano} & A min-flow algorithm for Minimal Critical Set detection in Resource Constrained Project Scheduling & \href{works/LombardiM12a.pdf}{Yes} & \cite{LombardiM12a} & 2012 & Artif. Intell. & 10 & 3 & 13 & \ref{b:LombardiM12a} & \ref{c:LombardiM12a}\\
BeniniLMR11 \href{https://doi.org/10.1007/s10479-010-0718-x}{BeniniLMR11} & \hyperref[auth:a247]{L. Benini}, \hyperref[auth:a142]{M. Lombardi}, \hyperref[auth:a143]{M. Milano}, \hyperref[auth:a727]{M. Ruggiero} & Optimal resource allocation and scheduling for the {CELL} {BE} platform & \href{works/BeniniLMR11.pdf}{Yes} & \cite{BeniniLMR11} & 2011 & Ann. Oper. Res. & 27 & 18 & 16 & \ref{b:BeniniLMR11} & \ref{c:BeniniLMR11}\\
BonfiettiLBM11 \href{https://doi.org/10.1007/978-3-642-23786-7\_12}{BonfiettiLBM11} & \hyperref[auth:a203]{A. Bonfietti}, \hyperref[auth:a142]{M. Lombardi}, \hyperref[auth:a247]{L. Benini}, \hyperref[auth:a143]{M. Milano} & A Constraint Based Approach to Cyclic {RCPSP} & \href{works/BonfiettiLBM11.pdf}{Yes} & \cite{BonfiettiLBM11} & 2011 & CP 2011 & 15 & 3 & 14 & \ref{b:BonfiettiLBM11} & \ref{c:BonfiettiLBM11}\\
LombardiBMB11 \href{https://doi.org/10.1007/978-3-642-21311-3\_14}{LombardiBMB11} & \hyperref[auth:a142]{M. Lombardi}, \hyperref[auth:a203]{A. Bonfietti}, \hyperref[auth:a143]{M. Milano}, \hyperref[auth:a247]{L. Benini} & Precedence Constraint Posting for Cyclic Scheduling Problems & \href{works/LombardiBMB11.pdf}{Yes} & \cite{LombardiBMB11} & 2011 & CPAIOR 2011 & 17 & 1 & 13 & \ref{b:LombardiBMB11} & \ref{c:LombardiBMB11}\\
Lombardi10 \href{http://amsdottorato.unibo.it/2961/}{Lombardi10} & \hyperref[auth:a142]{M. Lombardi} & Hybrid Methods for Resource Allocation and Scheduling Problems in Deterministic and Stochastic Environments & \href{works/Lombardi10.pdf}{Yes} & \cite{Lombardi10} & 2010 & University of Bologna, Italy & 175 & 0 & 0 & \ref{b:Lombardi10} & \ref{c:Lombardi10}\\
LombardiM10 \href{https://doi.org/10.1007/978-3-642-15396-9\_32}{LombardiM10} & \hyperref[auth:a142]{M. Lombardi}, \hyperref[auth:a143]{M. Milano} & Constraint Based Scheduling to Deal with Uncertain Durations and Self-Timed Execution & \href{works/LombardiM10.pdf}{Yes} & \cite{LombardiM10} & 2010 & CP 2010 & 15 & 1 & 11 & \ref{b:LombardiM10} & \ref{c:LombardiM10}\\
LombardiM10a \href{https://doi.org/10.1016/j.artint.2010.02.004}{LombardiM10a} & \hyperref[auth:a142]{M. Lombardi}, \hyperref[auth:a143]{M. Milano} & Allocation and scheduling of Conditional Task Graphs & \href{works/LombardiM10a.pdf}{Yes} & \cite{LombardiM10a} & 2010 & Artif. Intell. & 30 & 8 & 24 & \ref{b:LombardiM10a} & \ref{c:LombardiM10a}\\
LombardiM09 \href{https://doi.org/10.1007/978-3-642-04244-7\_45}{LombardiM09} & \hyperref[auth:a142]{M. Lombardi}, \hyperref[auth:a143]{M. Milano} & A Precedence Constraint Posting Approach for the {RCPSP} with Time Lags and Variable Durations & \href{works/LombardiM09.pdf}{Yes} & \cite{LombardiM09} & 2009 & CP 2009 & 15 & 7 & 12 & \ref{b:LombardiM09} & \ref{c:LombardiM09}\\
HoeveGSL07 \href{http://www.aaai.org/Library/AAAI/2007/aaai07-291.php}{HoeveGSL07} & \hyperref[auth:a651]{Willem Jan van Hoeve}, \hyperref[auth:a652]{Carla P. Gomes}, \hyperref[auth:a653]{B. Selman}, \hyperref[auth:a142]{M. Lombardi} & Optimal Multi-Agent Scheduling with Constraint Programming & \href{works/HoeveGSL07.pdf}{Yes} & \cite{HoeveGSL07} & 2007 & AAAI 2007 & 6 & 0 & 0 & \ref{b:HoeveGSL07} & \ref{c:HoeveGSL07}\\
\end{longtable}
}

\subsection{Works by Emmanuel Hebrard}
\label{sec:a1}
{\scriptsize
\begin{longtable}{>{\raggedright\arraybackslash}p{3cm}>{\raggedright\arraybackslash}p{6cm}>{\raggedright\arraybackslash}p{6.5cm}rrrp{2.5cm}rrrrr}
\rowcolor{white}\caption{Works from bibtex (Total 17)}\\ \toprule
\rowcolor{white}Key & Authors & Title & LC & Cite & Year & \shortstack{Conference\\/Journal} & Pages & \shortstack{Nr\\Cites} & \shortstack{Nr\\Refs} & b & c \\ \midrule\endhead
\bottomrule
\endfoot
JuvinHHL23 \href{https://doi.org/10.4230/LIPIcs.CP.2023.19}{JuvinHHL23} & \hyperref[auth:a0]{C. Juvin}, \hyperref[auth:a1]{E. Hebrard}, \hyperref[auth:a2]{L. Houssin}, \hyperref[auth:a3]{P. Lopez} & An Efficient Constraint Programming Approach to Preemptive Job Shop Scheduling & \href{works/JuvinHHL23.pdf}{Yes} & \cite{JuvinHHL23} & 2023 & CP 2023 & 16 & 0 & 0 & \ref{b:JuvinHHL23} & \ref{c:JuvinHHL23}\\
HebrardALLCMR22 \href{https://doi.org/10.24963/ijcai.2022/643}{HebrardALLCMR22} & \hyperref[auth:a1]{E. Hebrard}, \hyperref[auth:a6]{C. Artigues}, \hyperref[auth:a3]{P. Lopez}, \hyperref[auth:a796]{A. Lusson}, \hyperref[auth:a797]{Steve A. Chien}, \hyperref[auth:a798]{A. Maillard}, \hyperref[auth:a799]{Gregg R. Rabideau} & An Efficient Approach to Data Transfer Scheduling for Long Range Space Exploration & \href{works/HebrardALLCMR22.pdf}{Yes} & \cite{HebrardALLCMR22} & 2022 & IJCAI 2022 & 7 & 0 & 0 & \ref{b:HebrardALLCMR22} & \ref{c:HebrardALLCMR22}\\
AntuoriHHEN21 \href{https://doi.org/10.4230/LIPIcs.CP.2021.14}{AntuoriHHEN21} & \hyperref[auth:a53]{V. Antuori}, \hyperref[auth:a1]{E. Hebrard}, \hyperref[auth:a54]{M. Huguet}, \hyperref[auth:a55]{S. Essodaigui}, \hyperref[auth:a56]{A. Nguyen} & Combining Monte Carlo Tree Search and Depth First Search Methods for a Car Manufacturing Workshop Scheduling Problem & \href{works/AntuoriHHEN21.pdf}{Yes} & \cite{AntuoriHHEN21} & 2021 & CP 2021 & 16 & 0 & 0 & \ref{b:AntuoriHHEN21} & \ref{c:AntuoriHHEN21}\\
ArtiguesHQT21 \href{https://doi.org/10.5220/0010190101290136}{ArtiguesHQT21} & \hyperref[auth:a6]{C. Artigues}, \hyperref[auth:a1]{E. Hebrard}, \hyperref[auth:a800]{A. Quilliot}, \hyperref[auth:a801]{H. Toussaint} & Multi-Mode {RCPSP} with Safety Margin Maximization: Models and Algorithms & No & \cite{ArtiguesHQT21} & 2021 & ICORES 2021 & 8 & 0 & 0 & No & \ref{c:ArtiguesHQT21}\\
AntuoriHHEN20 \href{https://doi.org/10.1007/978-3-030-58475-7\_38}{AntuoriHHEN20} & \hyperref[auth:a53]{V. Antuori}, \hyperref[auth:a1]{E. Hebrard}, \hyperref[auth:a54]{M. Huguet}, \hyperref[auth:a55]{S. Essodaigui}, \hyperref[auth:a56]{A. Nguyen} & Leveraging Reinforcement Learning, Constraint Programming and Local Search: {A} Case Study in Car Manufacturing & \href{works/AntuoriHHEN20.pdf}{Yes} & \cite{AntuoriHHEN20} & 2020 & CP 2020 & 16 & 3 & 8 & \ref{b:AntuoriHHEN20} & \ref{c:AntuoriHHEN20}\\
GodetLHS20 \href{https://doi.org/10.1609/aaai.v34i02.5510}{GodetLHS20} & \hyperref[auth:a476]{A. Godet}, \hyperref[auth:a246]{X. Lorca}, \hyperref[auth:a1]{E. Hebrard}, \hyperref[auth:a126]{G. Simonin} & Using Approximation within Constraint Programming to Solve the Parallel Machine Scheduling Problem with Additional Unit Resources & \href{works/GodetLHS20.pdf}{Yes} & \cite{GodetLHS20} & 2020 & AAAI 2020 & 8 & 1 & 0 & \ref{b:GodetLHS20} & \ref{c:GodetLHS20}\\
HebrardHJMPV16 \href{https://doi.org/10.1016/j.dam.2016.07.003}{HebrardHJMPV16} & \hyperref[auth:a1]{E. Hebrard}, \hyperref[auth:a54]{M. Huguet}, \hyperref[auth:a802]{N. Jozefowiez}, \hyperref[auth:a798]{A. Maillard}, \hyperref[auth:a21]{C. Pralet}, \hyperref[auth:a174]{G. Verfaillie} & Approximation of the parallel machine scheduling problem with additional unit resources & \href{works/HebrardHJMPV16.pdf}{Yes} & \cite{HebrardHJMPV16} & 2016 & Discret. Appl. Math. & 10 & 9 & 8 & \ref{b:HebrardHJMPV16} & \ref{c:HebrardHJMPV16}\\
GrimesH15 \href{https://doi.org/10.1287/ijoc.2014.0625}{GrimesH15} & \hyperref[auth:a182]{D. Grimes}, \hyperref[auth:a1]{E. Hebrard} & Solving Variants of the Job Shop Scheduling Problem Through Conflict-Directed Search & No & \cite{GrimesH15} & 2015 & {INFORMS} J. Comput. & 17 & 12 & 41 & No & \ref{c:GrimesH15}\\
SialaAH15 \href{https://doi.org/10.1007/978-3-319-23219-5\_28}{SialaAH15} & \hyperref[auth:a129]{M. Siala}, \hyperref[auth:a6]{C. Artigues}, \hyperref[auth:a1]{E. Hebrard} & Two Clause Learning Approaches for Disjunctive Scheduling & \href{works/SialaAH15.pdf}{Yes} & \cite{SialaAH15} & 2015 & CP 2015 & 10 & 4 & 17 & \ref{b:SialaAH15} & \ref{c:SialaAH15}\\
SimoninAHL15 \href{https://doi.org/10.1007/s10601-014-9169-3}{SimoninAHL15} & \hyperref[auth:a126]{G. Simonin}, \hyperref[auth:a6]{C. Artigues}, \hyperref[auth:a1]{E. Hebrard}, \hyperref[auth:a3]{P. Lopez} & Scheduling scientific experiments for comet exploration & \href{works/SimoninAHL15.pdf}{Yes} & \cite{SimoninAHL15} & 2015 & Constraints An Int. J. & 23 & 4 & 5 & \ref{b:SimoninAHL15} & \ref{c:SimoninAHL15}\\
BessiereHMQW14 \href{https://doi.org/10.1007/978-3-319-07046-9\_23}{BessiereHMQW14} & \hyperref[auth:a333]{C. Bessiere}, \hyperref[auth:a1]{E. Hebrard}, \hyperref[auth:a334]{M. M{\'{e}}nard}, \hyperref[auth:a37]{C. Quimper}, \hyperref[auth:a278]{T. Walsh} & Buffered Resource Constraint: Algorithms and Complexity & \href{works/BessiereHMQW14.pdf}{Yes} & \cite{BessiereHMQW14} & 2014 & CPAIOR 2014 & 16 & 1 & 3 & \ref{b:BessiereHMQW14} & \ref{c:BessiereHMQW14}\\
BillautHL12 \href{https://doi.org/10.1007/978-3-642-29828-8\_5}{BillautHL12} & \hyperref[auth:a342]{J. Billaut}, \hyperref[auth:a1]{E. Hebrard}, \hyperref[auth:a3]{P. Lopez} & Complete Characterization of Near-Optimal Sequences for the Two-Machine Flow Shop Scheduling Problem & \href{works/BillautHL12.pdf}{Yes} & \cite{BillautHL12} & 2012 & CPAIOR 2012 & 15 & 1 & 19 & \ref{b:BillautHL12} & \ref{c:BillautHL12}\\
SimoninAHL12 \href{https://doi.org/10.1007/978-3-642-33558-7\_5}{SimoninAHL12} & \hyperref[auth:a126]{G. Simonin}, \hyperref[auth:a6]{C. Artigues}, \hyperref[auth:a1]{E. Hebrard}, \hyperref[auth:a3]{P. Lopez} & Scheduling Scientific Experiments on the Rosetta/Philae Mission & \href{works/SimoninAHL12.pdf}{Yes} & \cite{SimoninAHL12} & 2012 & CP 2012 & 15 & 3 & 8 & \ref{b:SimoninAHL12} & \ref{c:SimoninAHL12}\\
GrimesH11 \href{https://doi.org/10.1007/978-3-642-23786-7\_28}{GrimesH11} & \hyperref[auth:a182]{D. Grimes}, \hyperref[auth:a1]{E. Hebrard} & Models and Strategies for Variants of the Job Shop Scheduling Problem & \href{works/GrimesH11.pdf}{Yes} & \cite{GrimesH11} & 2011 & CP 2011 & 17 & 5 & 18 & \ref{b:GrimesH11} & \ref{c:GrimesH11}\\
GrimesH10 \href{https://doi.org/10.1007/978-3-642-13520-0\_19}{GrimesH10} & \hyperref[auth:a182]{D. Grimes}, \hyperref[auth:a1]{E. Hebrard} & Job Shop Scheduling with Setup Times and Maximal Time-Lags: {A} Simple Constraint Programming Approach & \href{works/GrimesH10.pdf}{Yes} & \cite{GrimesH10} & 2010 & CPAIOR 2010 & 15 & 13 & 20 & \ref{b:GrimesH10} & \ref{c:GrimesH10}\\
GrimesHM09 \href{https://doi.org/10.1007/978-3-642-04244-7\_33}{GrimesHM09} & \hyperref[auth:a182]{D. Grimes}, \hyperref[auth:a1]{E. Hebrard}, \hyperref[auth:a82]{A. Malapert} & Closing the Open Shop: Contradicting Conventional Wisdom & \href{works/GrimesHM09.pdf}{Yes} & \cite{GrimesHM09} & 2009 & CP 2009 & 9 & 15 & 12 & \ref{b:GrimesHM09} & \ref{c:GrimesHM09}\\
HebrardTW05 \href{https://doi.org/10.1007/11564751\_117}{HebrardTW05} & \hyperref[auth:a1]{E. Hebrard}, \hyperref[auth:a277]{P. Tyler}, \hyperref[auth:a278]{T. Walsh} & Computing Super-Schedules & \href{works/HebrardTW05.pdf}{Yes} & \cite{HebrardTW05} & 2005 & CP 2005 & 1 & 0 & 3 & \ref{b:HebrardTW05} & \ref{c:HebrardTW05}\\
\end{longtable}
}

\subsection{Works by John N. Hooker}
\label{sec:a161}
{\scriptsize
\begin{longtable}{>{\raggedright\arraybackslash}p{3cm}>{\raggedright\arraybackslash}p{6cm}>{\raggedright\arraybackslash}p{6.5cm}rrrp{2.5cm}rrrrr}
\rowcolor{white}\caption{Works from bibtex (Total 14)}\\ \toprule
\rowcolor{white}Key & Authors & Title & LC & Cite & Year & \shortstack{Conference\\/Journal} & Pages & \shortstack{Nr\\Cites} & \shortstack{Nr\\Refs} & b & c \\ \midrule\endhead
\bottomrule
\endfoot
Hooker19 \href{https://ideas.repec.org/h/spr/spochp/978-3-030-22788-3_1.html}{Hooker19} & \hyperref[auth:a161]{John N. Hooker} & {Logic-Based Benders Decomposition for Large-Scale Optimization} & No & \cite{Hooker19} & 2019 & {Large Scale Optimization in Supply Chains and Smart Manufacturing} & 26 & 8 & 0 & No & \ref{c:Hooker19}\\
HookerH18 \href{https://doi.org/10.1007/s10601-017-9280-3}{HookerH18} & \hyperref[auth:a161]{John N. Hooker}, \hyperref[auth:a651]{Willem Jan van Hoeve} & Constraint programming and operations research & \href{works/HookerH18.pdf}{Yes} & \cite{HookerH18} & 2018 & Constraints An Int. J. & 24 & 12 & 189 & \ref{b:HookerH18} & \ref{c:HookerH18}\\
Hooker17 \href{https://doi.org/10.1007/978-3-319-66158-2\_36}{Hooker17} & \hyperref[auth:a161]{John N. Hooker} & Job Sequencing Bounds from Decision Diagrams & \href{works/Hooker17.pdf}{Yes} & \cite{Hooker17} & 2017 & CP 2017 & 14 & 6 & 24 & \ref{b:Hooker17} & \ref{c:Hooker17}\\
HechingH16 \href{https://doi.org/10.1007/978-3-319-33954-2\_14}{HechingH16} & \hyperref[auth:a322]{Aliza R. Heching}, \hyperref[auth:a161]{John N. Hooker} & Scheduling Home Hospice Care with Logic-Based Benders Decomposition & \href{works/HechingH16.pdf}{Yes} & \cite{HechingH16} & 2016 & CPAIOR 2016 & 11 & 10 & 0 & \ref{b:HechingH16} & \ref{c:HechingH16}\\
CireCH13 \href{https://doi.org/10.1007/978-3-642-38171-3\_22}{CireCH13} & \hyperref[auth:a158]{Andr{\'{e}} A. Cir{\'{e}}}, \hyperref[auth:a340]{E. Coban}, \hyperref[auth:a161]{John N. Hooker} & Mixed Integer Programming vs. Logic-Based Benders Decomposition for Planning and Scheduling & \href{works/CireCH13.pdf}{Yes} & \cite{CireCH13} & 2013 & CPAIOR 2013 & 7 & 3 & 23 & \ref{b:CireCH13} & \ref{c:CireCH13}\\
CobanH10 \href{https://doi.org/10.1007/978-3-642-13520-0\_11}{CobanH10} & \hyperref[auth:a340]{E. Coban}, \hyperref[auth:a161]{John N. Hooker} & Single-Facility Scheduling over Long Time Horizons by Logic-Based Benders Decomposition & \href{works/CobanH10.pdf}{Yes} & \cite{CobanH10} & 2010 & CPAIOR 2010 & 5 & 9 & 9 & \ref{b:CobanH10} & \ref{c:CobanH10}\\
Hooker07 \href{http://dx.doi.org/10.1287/opre.1060.0371}{Hooker07} & \hyperref[auth:a161]{John N. Hooker} & Planning and Scheduling by Logic-Based Benders Decomposition & No & \cite{Hooker07} & 2007 & Operations Research & null & 181 & 19 & No & \ref{c:Hooker07}\\
Hooker06 \href{https://doi.org/10.1007/s10601-006-8060-2}{Hooker06} & \hyperref[auth:a161]{John N. Hooker} & An Integrated Method for Planning and Scheduling to Minimize Tardiness & \href{works/Hooker06.pdf}{Yes} & \cite{Hooker06} & 2006 & Constraints An Int. J. & 19 & 19 & 13 & \ref{b:Hooker06} & \ref{c:Hooker06}\\
Hooker05 \href{https://doi.org/10.1007/s10601-005-2812-2}{Hooker05} & \hyperref[auth:a161]{John N. Hooker} & A Hybrid Method for the Planning and Scheduling & \href{works/Hooker05.pdf}{Yes} & \cite{Hooker05} & 2005 & Constraints An Int. J. & 17 & 68 & 11 & \ref{b:Hooker05} & \ref{c:Hooker05}\\
Hooker05a \href{https://doi.org/10.1007/11564751\_25}{Hooker05a} & \hyperref[auth:a161]{John N. Hooker} & Planning and Scheduling to Minimize Tardiness & \href{works/Hooker05a.pdf}{Yes} & \cite{Hooker05a} & 2005 & CP 2005 & 14 & 30 & 10 & \ref{b:Hooker05a} & \ref{c:Hooker05a}\\
Hooker04 \href{https://doi.org/10.1007/978-3-540-30201-8\_24}{Hooker04} & \hyperref[auth:a161]{John N. Hooker} & A Hybrid Method for Planning and Scheduling & \href{works/Hooker04.pdf}{Yes} & \cite{Hooker04} & 2004 & CP 2004 & 12 & 39 & 9 & \ref{b:Hooker04} & \ref{c:Hooker04}\\
HookerO03 \href{http://dx.doi.org/10.1007/s10107-003-0375-9}{HookerO03} & \hyperref[auth:a161]{John N. Hooker}, \hyperref[auth:a866]{G. Ottosson} & Logic-based Benders decomposition & \href{works/HookerO03.pdf}{Yes} & \cite{HookerO03} & 2003 & Mathematical Programming & 28 & 317 & 0 & \ref{b:HookerO03} & \ref{c:HookerO03}\\
HookerY02 \href{https://doi.org/10.1007/3-540-46135-3\_46}{HookerY02} & \hyperref[auth:a161]{John N. Hooker}, \hyperref[auth:a293]{H. Yan} & A Relaxation of the Cumulative Constraint & \href{works/HookerY02.pdf}{Yes} & \cite{HookerY02} & 2002 & CP 2002 & 5 & 8 & 7 & \ref{b:HookerY02} & \ref{c:HookerY02}\\
Hooker00 \href{http://dx.doi.org/10.1002/9781118033036}{Hooker00} & \hyperref[auth:a161]{John N. Hooker} & Logic Based Methods for Optimization: Combining Optimization and Constraint Satisfaction & No & \cite{Hooker00} & 2000 & Book & null & 185 & 0 & No & \ref{c:Hooker00}\\
\end{longtable}
}

\subsection{Works by Helmut Simonis}
\label{sec:a17}
{\scriptsize
\begin{longtable}{>{\raggedright\arraybackslash}p{3cm}>{\raggedright\arraybackslash}p{6cm}>{\raggedright\arraybackslash}p{6.5cm}rrrp{2.5cm}rrrrr}
\rowcolor{white}\caption{Works from bibtex (Total 14)}\\ \toprule
\rowcolor{white}Key & Authors & Title & LC & Cite & Year & \shortstack{Conference\\/Journal} & Pages & \shortstack{Nr\\Cites} & \shortstack{Nr\\Refs} & b & c \\ \midrule\endhead
\bottomrule
\endfoot
ArmstrongGOS22 \href{https://doi.org/10.1007/978-3-031-08011-1\_1}{ArmstrongGOS22} & \hyperref[auth:a14]{E. Armstrong}, \hyperref[auth:a15]{M. Garraffa}, \hyperref[auth:a16]{B. O'Sullivan}, \hyperref[auth:a17]{H. Simonis} & A Two-Phase Hybrid Approach for the Hybrid Flexible Flowshop with Transportation Times & \href{works/ArmstrongGOS22.pdf}{Yes} & \cite{ArmstrongGOS22} & 2022 & CPAIOR 2022 & 13 & 0 & 14 & \ref{b:ArmstrongGOS22} & \ref{c:ArmstrongGOS22}\\
ArmstrongGOS21 \href{https://doi.org/10.4230/LIPIcs.CP.2021.16}{ArmstrongGOS21} & \hyperref[auth:a14]{E. Armstrong}, \hyperref[auth:a15]{M. Garraffa}, \hyperref[auth:a16]{B. O'Sullivan}, \hyperref[auth:a17]{H. Simonis} & The Hybrid Flexible Flowshop with Transportation Times & \href{works/ArmstrongGOS21.pdf}{Yes} & \cite{ArmstrongGOS21} & 2021 & CP 2021 & 18 & 1 & 0 & \ref{b:ArmstrongGOS21} & \ref{c:ArmstrongGOS21}\\
AntunesABDEGGOL20 \href{https://doi.org/10.1142/S0218213020600076}{AntunesABDEGGOL20} & \hyperref[auth:a893]{M. Antunes}, \hyperref[auth:a894]{V. Armant}, \hyperref[auth:a222]{Kenneth N. Brown}, \hyperref[auth:a895]{Daniel A. Desmond}, \hyperref[auth:a896]{G. Escamocher}, \hyperref[auth:a897]{A. George}, \hyperref[auth:a182]{D. Grimes}, \hyperref[auth:a898]{M. O'Keeffe}, \hyperref[auth:a899]{Y. Lin}, \hyperref[auth:a16]{B. O'Sullivan}, \hyperref[auth:a900]{C. Ozturk}, \hyperref[auth:a901]{L. Quesada}, \hyperref[auth:a129]{M. Siala}, \hyperref[auth:a17]{H. Simonis}, \hyperref[auth:a837]{N. Wilson} & Assigning and Scheduling Service Visits in a Mixed Urban/Rural Setting & No & \cite{AntunesABDEGGOL20} & 2020 & Int. J. Artif. Intell. Tools & 31 & 0 & 16 & No & \ref{c:AntunesABDEGGOL20}\\
AntunesABDEGGOL18 \href{https://doi.org/10.1109/ICTAI.2018.00027}{AntunesABDEGGOL18} & \hyperref[auth:a893]{M. Antunes}, \hyperref[auth:a894]{V. Armant}, \hyperref[auth:a222]{Kenneth N. Brown}, \hyperref[auth:a895]{Daniel A. Desmond}, \hyperref[auth:a896]{G. Escamocher}, \hyperref[auth:a897]{A. George}, \hyperref[auth:a182]{D. Grimes}, \hyperref[auth:a898]{M. O'Keeffe}, \hyperref[auth:a899]{Y. Lin}, \hyperref[auth:a16]{B. O'Sullivan}, \hyperref[auth:a900]{C. Ozturk}, \hyperref[auth:a901]{L. Quesada}, \hyperref[auth:a129]{M. Siala}, \hyperref[auth:a17]{H. Simonis}, \hyperref[auth:a837]{N. Wilson} & Assigning and Scheduling Service Visits in a Mixed Urban/Rural Setting & No & \cite{AntunesABDEGGOL18} & 2018 & ICTAI 2018 & 8 & 1 & 24 & No & \ref{c:AntunesABDEGGOL18}\\
HurleyOS16 \href{https://doi.org/10.1007/978-3-319-50137-6\_15}{HurleyOS16} & \hyperref[auth:a902]{B. Hurley}, \hyperref[auth:a16]{B. O'Sullivan}, \hyperref[auth:a17]{H. Simonis} & {ICON} Loop Energy Show Case & \href{works/HurleyOS16.pdf}{Yes} & \cite{HurleyOS16} & 2016 & Data Mining and Constraint Programming - Foundations of a Cross-Disciplinary Approach & 14 & 0 & 16 & \ref{b:HurleyOS16} & \ref{c:HurleyOS16}\\
GrimesIOS14 \href{https://doi.org/10.1016/j.suscom.2014.08.009}{GrimesIOS14} & \hyperref[auth:a182]{D. Grimes}, \hyperref[auth:a183]{G. Ifrim}, \hyperref[auth:a16]{B. O'Sullivan}, \hyperref[auth:a17]{H. Simonis} & Analyzing the impact of electricity price forecasting on energy cost-aware scheduling & \href{works/GrimesIOS14.pdf}{Yes} & \cite{GrimesIOS14} & 2014 & Sustain. Comput. Informatics Syst. & 16 & 6 & 7 & \ref{b:GrimesIOS14} & \ref{c:GrimesIOS14}\\
IfrimOS12 \href{https://doi.org/10.1007/978-3-642-33558-7\_68}{IfrimOS12} & \hyperref[auth:a183]{G. Ifrim}, \hyperref[auth:a16]{B. O'Sullivan}, \hyperref[auth:a17]{H. Simonis} & Properties of Energy-Price Forecasts for Scheduling & \href{works/IfrimOS12.pdf}{Yes} & \cite{IfrimOS12} & 2012 & CP 2012 & 16 & 6 & 20 & \ref{b:IfrimOS12} & \ref{c:IfrimOS12}\\
Simonis07 \href{https://doi.org/10.1007/s10601-006-9011-7}{Simonis07} & \hyperref[auth:a17]{H. Simonis} & Models for Global Constraint Applications & \href{works/Simonis07.pdf}{Yes} & \cite{Simonis07} & 2007 & Constraints An Int. J. & 30 & 10 & 17 & \ref{b:Simonis07} & \ref{c:Simonis07}\\
SimonisCK00 \href{https://doi.org/10.1109/5254.820326}{SimonisCK00} & \hyperref[auth:a17]{H. Simonis}, \hyperref[auth:a903]{P. Charlier}, \hyperref[auth:a904]{P. Kay} & Constraint Handling in an Integrated Transportation Problem & No & \cite{SimonisCK00} & 2000 & {IEEE} Intell. Syst. & 7 & 11 & 5 & No & \ref{c:SimonisCK00}\\
Simonis99 \href{https://doi.org/10.1007/3-540-45406-3\_6}{Simonis99} & \hyperref[auth:a17]{H. Simonis} & Building Industrial Applications with Constraint Programming & \href{works/Simonis99.pdf}{Yes} & \cite{Simonis99} & 1999 & CCL'99 1999 & 39 & 5 & 18 & \ref{b:Simonis99} & \ref{c:Simonis99}\\
Simonis95 \href{https://doi.org/10.1007/3-540-60299-2\_42}{Simonis95} & \hyperref[auth:a17]{H. Simonis} & The {CHIP} System and Its Applications & \href{works/Simonis95.pdf}{Yes} & \cite{Simonis95} & 1995 & CP 1995 & 4 & 7 & 3 & \ref{b:Simonis95} & \ref{c:Simonis95}\\
Simonis95a \href{https://doi.org/10.1007/3-540-60794-3\_11}{Simonis95a} & \hyperref[auth:a17]{H. Simonis} & Application Development with the {CHIP} System & \href{works/Simonis95a.pdf}{Yes} & \cite{Simonis95a} & 1995 & CONTESSA 1995 & 21 & 1 & 12 & \ref{b:Simonis95a} & \ref{c:Simonis95a}\\
SimonisC95 \href{https://doi.org/10.1007/3-540-60299-2\_27}{SimonisC95} & \hyperref[auth:a17]{H. Simonis}, \hyperref[auth:a305]{T. Cornelissens} & Modelling Producer/Consumer Constraints & \href{works/SimonisC95.pdf}{Yes} & \cite{SimonisC95} & 1995 & CP 1995 & 14 & 17 & 8 & \ref{b:SimonisC95} & \ref{c:SimonisC95}\\
DincbasSH90 \href{https://doi.org/10.1016/0743-1066(90)90052-7}{DincbasSH90} & \hyperref[auth:a726]{M. Dincbas}, \hyperref[auth:a17]{H. Simonis}, \hyperref[auth:a148]{Pascal Van Hentenryck} & Solving Large Combinatorial Problems in Logic Programming & \href{works/DincbasSH90.pdf}{Yes} & \cite{DincbasSH90} & 1990 & J. Log. Program. & 19 & 86 & 9 & \ref{b:DincbasSH90} & \ref{c:DincbasSH90}\\
\end{longtable}
}

\subsection{Works by Nicolas Beldiceanu}
\label{sec:a128}
{\scriptsize
\begin{longtable}{>{\raggedright\arraybackslash}p{3cm}>{\raggedright\arraybackslash}p{6cm}>{\raggedright\arraybackslash}p{6.5cm}rrrp{2.5cm}rrrrr}
\rowcolor{white}\caption{Works from bibtex (Total 13)}\\ \toprule
\rowcolor{white}Key & Authors & Title & LC & Cite & Year & \shortstack{Conference\\/Journal} & Pages & \shortstack{Nr\\Cites} & \shortstack{Nr\\Refs} & b & c \\ \midrule\endhead
\bottomrule
\endfoot
Madi-WambaLOBM17 \href{https://doi.org/10.1109/ICPADS.2017.00089}{Madi-WambaLOBM17} & \hyperref[auth:a323]{G. Madi{-}Wamba}, \hyperref[auth:a723]{Y. Li}, \hyperref[auth:a724]{A. Orgerie}, \hyperref[auth:a128]{N. Beldiceanu}, \hyperref[auth:a725]{J. Menaud} & Green Energy Aware Scheduling Problem in Virtualized Datacenters & \href{works/Madi-WambaLOBM17.pdf}{Yes} & \cite{Madi-WambaLOBM17} & 2017 & ICPADS 2017 & 8 & 1 & 8 & \ref{b:Madi-WambaLOBM17} & \ref{c:Madi-WambaLOBM17}\\
Madi-WambaB16 \href{https://doi.org/10.1007/978-3-319-33954-2\_18}{Madi-WambaB16} & \hyperref[auth:a323]{G. Madi{-}Wamba}, \hyperref[auth:a128]{N. Beldiceanu} & The TaskIntersection Constraint & \href{works/Madi-WambaB16.pdf}{Yes} & \cite{Madi-WambaB16} & 2016 & CPAIOR 2016 & 16 & 0 & 0 & \ref{b:Madi-WambaB16} & \ref{c:Madi-WambaB16}\\
LetortCB15 \href{https://doi.org/10.1007/s10601-014-9172-8}{LetortCB15} & \hyperref[auth:a127]{A. Letort}, \hyperref[auth:a91]{M. Carlsson}, \hyperref[auth:a128]{N. Beldiceanu} & Synchronized sweep algorithms for scalable scheduling constraints & \href{works/LetortCB15.pdf}{Yes} & \cite{LetortCB15} & 2015 & Constraints An Int. J. & 52 & 2 & 14 & \ref{b:LetortCB15} & \ref{c:LetortCB15}\\
LetortCB13 \href{https://doi.org/10.1007/978-3-642-38171-3\_10}{LetortCB13} & \hyperref[auth:a127]{A. Letort}, \hyperref[auth:a91]{M. Carlsson}, \hyperref[auth:a128]{N. Beldiceanu} & A Synchronized Sweep Algorithm for the \emph{k-dimensional cumulative} Constraint & \href{works/LetortCB13.pdf}{Yes} & \cite{LetortCB13} & 2013 & CPAIOR 2013 & 16 & 3 & 10 & \ref{b:LetortCB13} & \ref{c:LetortCB13}\\
LetortBC12 \href{https://doi.org/10.1007/978-3-642-33558-7\_33}{LetortBC12} & \hyperref[auth:a127]{A. Letort}, \hyperref[auth:a128]{N. Beldiceanu}, \hyperref[auth:a91]{M. Carlsson} & A Scalable Sweep Algorithm for the cumulative Constraint & \href{works/LetortBC12.pdf}{Yes} & \cite{LetortBC12} & 2012 & CP 2012 & 16 & 18 & 12 & \ref{b:LetortBC12} & \ref{c:LetortBC12}\\
BeldiceanuCDP11 \href{https://doi.org/10.1007/s10479-010-0731-0}{BeldiceanuCDP11} & \hyperref[auth:a128]{N. Beldiceanu}, \hyperref[auth:a91]{M. Carlsson}, \hyperref[auth:a245]{S. Demassey}, \hyperref[auth:a362]{E. Poder} & New filtering for the \emph{cumulative} constraint in the context of non-overlapping rectangles & \href{works/BeldiceanuCDP11.pdf}{Yes} & \cite{BeldiceanuCDP11} & 2011 & Ann. Oper. Res. & 24 & 8 & 8 & \ref{b:BeldiceanuCDP11} & \ref{c:BeldiceanuCDP11}\\
ClercqPBJ11 \href{https://doi.org/10.1007/978-3-642-23786-7\_20}{ClercqPBJ11} & \hyperref[auth:a248]{Alexis De Clercq}, \hyperref[auth:a226]{T. Petit}, \hyperref[auth:a128]{N. Beldiceanu}, \hyperref[auth:a249]{N. Jussien} & Filtering Algorithms for Discrete Cumulative Problems with Overloads of Resource & \href{works/ClercqPBJ11.pdf}{Yes} & \cite{ClercqPBJ11} & 2011 & CP 2011 & 16 & 3 & 11 & \ref{b:ClercqPBJ11} & \ref{c:ClercqPBJ11}\\
BeldiceanuCP08 \href{https://doi.org/10.1007/978-3-540-68155-7\_5}{BeldiceanuCP08} & \hyperref[auth:a128]{N. Beldiceanu}, \hyperref[auth:a91]{M. Carlsson}, \hyperref[auth:a362]{E. Poder} & New Filtering for the cumulative Constraint in the Context of Non-Overlapping Rectangles & \href{works/BeldiceanuCP08.pdf}{Yes} & \cite{BeldiceanuCP08} & 2008 & CPAIOR 2008 & 15 & 8 & 9 & \ref{b:BeldiceanuCP08} & \ref{c:BeldiceanuCP08}\\
PoderB08 \href{http://www.aaai.org/Library/ICAPS/2008/icaps08-033.php}{PoderB08} & \hyperref[auth:a362]{E. Poder}, \hyperref[auth:a128]{N. Beldiceanu} & Filtering for a Continuous Multi-Resources cumulative Constraint with Resource Consumption and Production & \href{works/PoderB08.pdf}{Yes} & \cite{PoderB08} & 2008 & ICAPS 2008 & 8 & 0 & 0 & \ref{b:PoderB08} & \ref{c:PoderB08}\\
BeldiceanuP07 \href{https://doi.org/10.1007/978-3-540-72397-4\_16}{BeldiceanuP07} & \hyperref[auth:a128]{N. Beldiceanu}, \hyperref[auth:a362]{E. Poder} & A Continuous Multi-resources \emph{cumulative} Constraint with Positive-Negative Resource Consumption-Production & \href{works/BeldiceanuP07.pdf}{Yes} & \cite{BeldiceanuP07} & 2007 & CPAIOR 2007 & 15 & 4 & 7 & \ref{b:BeldiceanuP07} & \ref{c:BeldiceanuP07}\\
PoderBS04 \href{https://doi.org/10.1016/S0377-2217(02)00756-7}{PoderBS04} & \hyperref[auth:a362]{E. Poder}, \hyperref[auth:a128]{N. Beldiceanu}, \hyperref[auth:a722]{E. Sanlaville} & Computing a lower approximation of the compulsory part of a task with varying duration and varying resource consumption & \href{works/PoderBS04.pdf}{Yes} & \cite{PoderBS04} & 2004 & Eur. J. Oper. Res. & 16 & 7 & 8 & \ref{b:PoderBS04} & \ref{c:PoderBS04}\\
BeldiceanuC02 \href{https://doi.org/10.1007/3-540-46135-3\_5}{BeldiceanuC02} & \hyperref[auth:a128]{N. Beldiceanu}, \hyperref[auth:a91]{M. Carlsson} & A New Multi-resource cumulatives Constraint with Negative Heights & \href{works/BeldiceanuC02.pdf}{Yes} & \cite{BeldiceanuC02} & 2002 & CP 2002 & 17 & 33 & 9 & \ref{b:BeldiceanuC02} & \ref{c:BeldiceanuC02}\\
AggounB93 \href{https://www.sciencedirect.com/science/article/pii/089571779390068A}{AggounB93} & \hyperref[auth:a734]{A. Aggoun}, \hyperref[auth:a128]{N. Beldiceanu} & Extending {CHIP} in order to solve complex scheduling and placement problems & \href{works/AggounB93.pdf}{Yes} & \cite{AggounB93} & 1993 & Mathematical and Computer Modelling & 17 & 187 & 11 & \ref{b:AggounB93} & \ref{c:AggounB93}\\
\end{longtable}
}

\subsection{Works by Pierre Lopez}
\label{sec:a3}
{\scriptsize
\begin{longtable}{>{\raggedright\arraybackslash}p{3cm}>{\raggedright\arraybackslash}p{6cm}>{\raggedright\arraybackslash}p{6.5cm}rrrp{2.5cm}rrrrr}
\rowcolor{white}\caption{Works from bibtex (Total 13)}\\ \toprule
\rowcolor{white}Key & Authors & Title & LC & Cite & Year & \shortstack{Conference\\/Journal} & Pages & \shortstack{Nr\\Cites} & \shortstack{Nr\\Refs} & b & c \\ \midrule\endhead
\bottomrule
\endfoot
JuvinHHL23 \href{https://doi.org/10.4230/LIPIcs.CP.2023.19}{JuvinHHL23} & \hyperref[auth:a0]{C. Juvin}, \hyperref[auth:a1]{E. Hebrard}, \hyperref[auth:a2]{L. Houssin}, \hyperref[auth:a3]{P. Lopez} & An Efficient Constraint Programming Approach to Preemptive Job Shop Scheduling & \href{works/JuvinHHL23.pdf}{Yes} & \cite{JuvinHHL23} & 2023 & CP 2023 & 16 & 0 & 0 & \ref{b:JuvinHHL23} & \ref{c:JuvinHHL23}\\
JuvinHL23 \href{https://doi.org/10.1007/978-3-031-33271-5\_23}{JuvinHL23} & \hyperref[auth:a0]{C. Juvin}, \hyperref[auth:a2]{L. Houssin}, \hyperref[auth:a3]{P. Lopez} & Constraint Programming for the Robust Two-Machine Flow-Shop Scheduling Problem with Budgeted Uncertainty & \href{works/JuvinHL23.pdf}{Yes} & \cite{JuvinHL23} & 2023 & CPAIOR 2023 & 16 & 0 & 11 & \ref{b:JuvinHL23} & \ref{c:JuvinHL23}\\
HebrardALLCMR22 \href{https://doi.org/10.24963/ijcai.2022/643}{HebrardALLCMR22} & \hyperref[auth:a1]{E. Hebrard}, \hyperref[auth:a6]{C. Artigues}, \hyperref[auth:a3]{P. Lopez}, \hyperref[auth:a796]{A. Lusson}, \hyperref[auth:a797]{Steve A. Chien}, \hyperref[auth:a798]{A. Maillard}, \hyperref[auth:a799]{Gregg R. Rabideau} & An Efficient Approach to Data Transfer Scheduling for Long Range Space Exploration & \href{works/HebrardALLCMR22.pdf}{Yes} & \cite{HebrardALLCMR22} & 2022 & IJCAI 2022 & 7 & 0 & 0 & \ref{b:HebrardALLCMR22} & \ref{c:HebrardALLCMR22}\\
Polo-MejiaALB20 \href{https://doi.org/10.1080/00207543.2019.1693654}{Polo-MejiaALB20} & \hyperref[auth:a522]{O. Polo{-}Mej{\'{\i}}a}, \hyperref[auth:a6]{C. Artigues}, \hyperref[auth:a3]{P. Lopez}, \hyperref[auth:a523]{V. Basini} & Mixed-integer/linear and constraint programming approaches for activity scheduling in a nuclear research facility & \href{works/Polo-MejiaALB20.pdf}{Yes} & \cite{Polo-MejiaALB20} & 2020 & Int. J. Prod. Res. & 18 & 8 & 23 & \ref{b:Polo-MejiaALB20} & \ref{c:Polo-MejiaALB20}\\
NattafAL17 \href{https://doi.org/10.1007/s10601-017-9271-4}{NattafAL17} & \hyperref[auth:a81]{M. Nattaf}, \hyperref[auth:a6]{C. Artigues}, \hyperref[auth:a3]{P. Lopez} & Cumulative scheduling with variable task profiles and concave piecewise linear processing rate functions & \href{works/NattafAL17.pdf}{Yes} & \cite{NattafAL17} & 2017 & Constraints An Int. J. & 18 & 5 & 10 & \ref{b:NattafAL17} & \ref{c:NattafAL17}\\
NattafAL15 \href{https://doi.org/10.1007/s10601-015-9192-z}{NattafAL15} & \hyperref[auth:a81]{M. Nattaf}, \hyperref[auth:a6]{C. Artigues}, \hyperref[auth:a3]{P. Lopez} & A hybrid exact method for a scheduling problem with a continuous resource and energy constraints & \href{works/NattafAL15.pdf}{Yes} & \cite{NattafAL15} & 2015 & Constraints An Int. J. & 21 & 14 & 13 & \ref{b:NattafAL15} & \ref{c:NattafAL15}\\
SimoninAHL15 \href{https://doi.org/10.1007/s10601-014-9169-3}{SimoninAHL15} & \hyperref[auth:a126]{G. Simonin}, \hyperref[auth:a6]{C. Artigues}, \hyperref[auth:a1]{E. Hebrard}, \hyperref[auth:a3]{P. Lopez} & Scheduling scientific experiments for comet exploration & \href{works/SimoninAHL15.pdf}{Yes} & \cite{SimoninAHL15} & 2015 & Constraints An Int. J. & 23 & 4 & 5 & \ref{b:SimoninAHL15} & \ref{c:SimoninAHL15}\\
BillautHL12 \href{https://doi.org/10.1007/978-3-642-29828-8\_5}{BillautHL12} & \hyperref[auth:a342]{J. Billaut}, \hyperref[auth:a1]{E. Hebrard}, \hyperref[auth:a3]{P. Lopez} & Complete Characterization of Near-Optimal Sequences for the Two-Machine Flow Shop Scheduling Problem & \href{works/BillautHL12.pdf}{Yes} & \cite{BillautHL12} & 2012 & CPAIOR 2012 & 15 & 1 & 19 & \ref{b:BillautHL12} & \ref{c:BillautHL12}\\
SimoninAHL12 \href{https://doi.org/10.1007/978-3-642-33558-7\_5}{SimoninAHL12} & \hyperref[auth:a126]{G. Simonin}, \hyperref[auth:a6]{C. Artigues}, \hyperref[auth:a1]{E. Hebrard}, \hyperref[auth:a3]{P. Lopez} & Scheduling Scientific Experiments on the Rosetta/Philae Mission & \href{works/SimoninAHL12.pdf}{Yes} & \cite{SimoninAHL12} & 2012 & CP 2012 & 15 & 3 & 8 & \ref{b:SimoninAHL12} & \ref{c:SimoninAHL12}\\
LahimerLH11 \href{https://doi.org/10.1007/978-3-642-21311-3\_12}{LahimerLH11} & \hyperref[auth:a353]{A. Lahimer}, \hyperref[auth:a3]{P. Lopez}, \hyperref[auth:a354]{M. Haouari} & Climbing Depth-Bounded Adjacent Discrepancy Search for Solving Hybrid Flow Shop Scheduling Problems with Multiprocessor Tasks & \href{works/LahimerLH11.pdf}{Yes} & \cite{LahimerLH11} & 2011 & CPAIOR 2011 & 14 & 3 & 15 & \ref{b:LahimerLH11} & \ref{c:LahimerLH11}\\
TrojetHL11 \href{https://doi.org/10.1016/j.cie.2010.08.014}{TrojetHL11} & \hyperref[auth:a715]{M. Trojet}, \hyperref[auth:a716]{F. H'Mida}, \hyperref[auth:a3]{P. Lopez} & Project scheduling under resource constraints: Application of the cumulative global constraint in a decision support framework & \href{works/TrojetHL11.pdf}{Yes} & \cite{TrojetHL11} & 2011 & Comput. Ind. Eng. & 7 & 11 & 17 & \ref{b:TrojetHL11} & \ref{c:TrojetHL11}\\
LopezAKYG00 \href{https://doi.org/10.1016/S0947-3580(00)71114-9}{LopezAKYG00} & \hyperref[auth:a3]{P. Lopez}, \hyperref[auth:a693]{H. Alla}, \hyperref[auth:a690]{O. Korbaa}, \hyperref[auth:a691]{P. Yim}, \hyperref[auth:a692]{J. Gentina} & Discussion on: 'Solving Transient Scheduling Problems with Constraint Programming' by O. Korbaa, P. Yim, and {J.-C.} Gentina & \href{works/LopezAKYG00.pdf}{Yes} & \cite{LopezAKYG00} & 2000 & Eur. J. Control & 4 & 0 & 0 & \ref{b:LopezAKYG00} & \ref{c:LopezAKYG00}\\
TorresL00 \href{http://dx.doi.org/10.1016/s0377-2217(99)00497-x}{TorresL00} & \hyperref[auth:a888]{P. Torres}, \hyperref[auth:a3]{P. Lopez} & On Not-First/Not-Last conditions in disjunctive scheduling & No & \cite{TorresL00} & 2000 & European Journal of Operational Research & null & 26 & 13 & No & \ref{c:TorresL00}\\
\end{longtable}
}

\subsection{Works by Christian Artigues}
\label{sec:a6}
{\scriptsize
\begin{longtable}{>{\raggedright\arraybackslash}p{3cm}>{\raggedright\arraybackslash}p{6cm}>{\raggedright\arraybackslash}p{6.5cm}rrrp{2.5cm}rrrrr}
\rowcolor{white}\caption{Works from bibtex (Total 12)}\\ \toprule
\rowcolor{white}Key & Authors & Title & LC & Cite & Year & \shortstack{Conference\\/Journal} & Pages & \shortstack{Nr\\Cites} & \shortstack{Nr\\Refs} & b & c \\ \midrule\endhead
\bottomrule
\endfoot
PovedaAA23 \href{https://doi.org/10.4230/LIPIcs.CP.2023.31}{PovedaAA23} & \hyperref[auth:a4]{G. Pov{\'{e}}da}, \hyperref[auth:a5]{N. {\'{A}}lvarez}, \hyperref[auth:a6]{C. Artigues} & Partially Preemptive Multi Skill/Mode Resource-Constrained Project Scheduling with Generalized Precedence Relations and Calendars & \href{works/PovedaAA23.pdf}{Yes} & \cite{PovedaAA23} & 2023 & CP 2023 & 21 & 0 & 0 & \ref{b:PovedaAA23} & \ref{c:PovedaAA23}\\
HebrardALLCMR22 \href{https://doi.org/10.24963/ijcai.2022/643}{HebrardALLCMR22} & \hyperref[auth:a1]{E. Hebrard}, \hyperref[auth:a6]{C. Artigues}, \hyperref[auth:a3]{P. Lopez}, \hyperref[auth:a796]{A. Lusson}, \hyperref[auth:a797]{Steve A. Chien}, \hyperref[auth:a798]{A. Maillard}, \hyperref[auth:a799]{Gregg R. Rabideau} & An Efficient Approach to Data Transfer Scheduling for Long Range Space Exploration & \href{works/HebrardALLCMR22.pdf}{Yes} & \cite{HebrardALLCMR22} & 2022 & IJCAI 2022 & 7 & 0 & 0 & \ref{b:HebrardALLCMR22} & \ref{c:HebrardALLCMR22}\\
PohlAK22 \href{https://doi.org/10.1016/j.ejor.2021.08.028}{PohlAK22} & \hyperref[auth:a444]{M. Pohl}, \hyperref[auth:a6]{C. Artigues}, \hyperref[auth:a445]{R. Kolisch} & Solving the time-discrete winter runway scheduling problem: {A} column generation and constraint programming approach & \href{works/PohlAK22.pdf}{Yes} & \cite{PohlAK22} & 2022 & Eur. J. Oper. Res. & 16 & 4 & 31 & \ref{b:PohlAK22} & \ref{c:PohlAK22}\\
ArtiguesHQT21 \href{https://doi.org/10.5220/0010190101290136}{ArtiguesHQT21} & \hyperref[auth:a6]{C. Artigues}, \hyperref[auth:a1]{E. Hebrard}, \hyperref[auth:a800]{A. Quilliot}, \hyperref[auth:a801]{H. Toussaint} & Multi-Mode {RCPSP} with Safety Margin Maximization: Models and Algorithms & No & \cite{ArtiguesHQT21} & 2021 & ICORES 2021 & 8 & 0 & 0 & No & \ref{c:ArtiguesHQT21}\\
Polo-MejiaALB20 \href{https://doi.org/10.1080/00207543.2019.1693654}{Polo-MejiaALB20} & \hyperref[auth:a522]{O. Polo{-}Mej{\'{\i}}a}, \hyperref[auth:a6]{C. Artigues}, \hyperref[auth:a3]{P. Lopez}, \hyperref[auth:a523]{V. Basini} & Mixed-integer/linear and constraint programming approaches for activity scheduling in a nuclear research facility & \href{works/Polo-MejiaALB20.pdf}{Yes} & \cite{Polo-MejiaALB20} & 2020 & Int. J. Prod. Res. & 18 & 8 & 23 & \ref{b:Polo-MejiaALB20} & \ref{c:Polo-MejiaALB20}\\
NattafAL17 \href{https://doi.org/10.1007/s10601-017-9271-4}{NattafAL17} & \hyperref[auth:a81]{M. Nattaf}, \hyperref[auth:a6]{C. Artigues}, \hyperref[auth:a3]{P. Lopez} & Cumulative scheduling with variable task profiles and concave piecewise linear processing rate functions & \href{works/NattafAL17.pdf}{Yes} & \cite{NattafAL17} & 2017 & Constraints An Int. J. & 18 & 5 & 10 & \ref{b:NattafAL17} & \ref{c:NattafAL17}\\
NattafAL15 \href{https://doi.org/10.1007/s10601-015-9192-z}{NattafAL15} & \hyperref[auth:a81]{M. Nattaf}, \hyperref[auth:a6]{C. Artigues}, \hyperref[auth:a3]{P. Lopez} & A hybrid exact method for a scheduling problem with a continuous resource and energy constraints & \href{works/NattafAL15.pdf}{Yes} & \cite{NattafAL15} & 2015 & Constraints An Int. J. & 21 & 14 & 13 & \ref{b:NattafAL15} & \ref{c:NattafAL15}\\
SialaAH15 \href{https://doi.org/10.1007/978-3-319-23219-5\_28}{SialaAH15} & \hyperref[auth:a129]{M. Siala}, \hyperref[auth:a6]{C. Artigues}, \hyperref[auth:a1]{E. Hebrard} & Two Clause Learning Approaches for Disjunctive Scheduling & \href{works/SialaAH15.pdf}{Yes} & \cite{SialaAH15} & 2015 & CP 2015 & 10 & 4 & 17 & \ref{b:SialaAH15} & \ref{c:SialaAH15}\\
SimoninAHL15 \href{https://doi.org/10.1007/s10601-014-9169-3}{SimoninAHL15} & \hyperref[auth:a126]{G. Simonin}, \hyperref[auth:a6]{C. Artigues}, \hyperref[auth:a1]{E. Hebrard}, \hyperref[auth:a3]{P. Lopez} & Scheduling scientific experiments for comet exploration & \href{works/SimoninAHL15.pdf}{Yes} & \cite{SimoninAHL15} & 2015 & Constraints An Int. J. & 23 & 4 & 5 & \ref{b:SimoninAHL15} & \ref{c:SimoninAHL15}\\
SimoninAHL12 \href{https://doi.org/10.1007/978-3-642-33558-7\_5}{SimoninAHL12} & \hyperref[auth:a126]{G. Simonin}, \hyperref[auth:a6]{C. Artigues}, \hyperref[auth:a1]{E. Hebrard}, \hyperref[auth:a3]{P. Lopez} & Scheduling Scientific Experiments on the Rosetta/Philae Mission & \href{works/SimoninAHL12.pdf}{Yes} & \cite{SimoninAHL12} & 2012 & CP 2012 & 15 & 3 & 8 & \ref{b:SimoninAHL12} & \ref{c:SimoninAHL12}\\
ArtiguesBF04 \href{https://doi.org/10.1007/978-3-540-24664-0\_3}{ArtiguesBF04} & \hyperref[auth:a6]{C. Artigues}, \hyperref[auth:a387]{S. Belmokhtar}, \hyperref[auth:a360]{D. Feillet} & A New Exact Solution Algorithm for the Job Shop Problem with Sequence-Dependent Setup Times & \href{works/ArtiguesBF04.pdf}{Yes} & \cite{ArtiguesBF04} & 2004 & CPAIOR 2004 & 13 & 16 & 9 & \ref{b:ArtiguesBF04} & \ref{c:ArtiguesBF04}\\
ArtiguesR00 \href{https://doi.org/10.1016/S0377-2217(99)00496-8}{ArtiguesR00} & \hyperref[auth:a6]{C. Artigues}, \hyperref[auth:a721]{F. Roubellat} & A polynomial activity insertion algorithm in a multi-resource schedule with cumulative constraints and multiple modes & \href{works/ArtiguesR00.pdf}{Yes} & \cite{ArtiguesR00} & 2000 & Eur. J. Oper. Res. & 20 & 84 & 3 & \ref{b:ArtiguesR00} & \ref{c:ArtiguesR00}\\
\end{longtable}
}

\subsection{Works by Pierre Schaus}
\label{sec:a147}
{\scriptsize
\begin{longtable}{>{\raggedright\arraybackslash}p{3cm}>{\raggedright\arraybackslash}p{6cm}>{\raggedright\arraybackslash}p{6.5cm}rrrp{2.5cm}rrrrr}
\rowcolor{white}\caption{Works from bibtex (Total 12)}\\ \toprule
\rowcolor{white}Key & Authors & Title & LC & Cite & Year & \shortstack{Conference\\/Journal} & Pages & \shortstack{Nr\\Cites} & \shortstack{Nr\\Refs} & b & c \\ \midrule\endhead
\bottomrule
\endfoot
CauwelaertDS20 \href{http://dx.doi.org/10.1007/s10951-019-00632-8}{CauwelaertDS20} & \hyperref[auth:a850]{Sasha Van Cauwelaert}, \hyperref[auth:a207]{C. Dejemeppe}, \hyperref[auth:a147]{P. Schaus} & An Efficient Filtering Algorithm for the Unary Resource Constraint with Transition Times and Optional Activities & \href{works/CauwelaertDS20.pdf}{Yes} & \cite{CauwelaertDS20} & 2020 & Journal of Scheduling & 19 & 2 & 21 & \ref{b:CauwelaertDS20} & \ref{c:CauwelaertDS20}\\
CappartTSR18 \href{https://doi.org/10.1007/978-3-319-98334-9\_32}{CappartTSR18} & \hyperref[auth:a42]{Q. Cappart}, \hyperref[auth:a849]{C. Thomas}, \hyperref[auth:a147]{P. Schaus}, \hyperref[auth:a331]{L. Rousseau} & A Constraint Programming Approach for Solving Patient Transportation Problems & \href{works/CappartTSR18.pdf}{Yes} & \cite{CappartTSR18} & 2018 & CP 2018 & 17 & 6 & 31 & \ref{b:CappartTSR18} & \ref{c:CappartTSR18}\\
CauwelaertLS18 \href{https://doi.org/10.1007/s10601-017-9277-y}{CauwelaertLS18} & \hyperref[auth:a206]{Sascha Van Cauwelaert}, \hyperref[auth:a142]{M. Lombardi}, \hyperref[auth:a147]{P. Schaus} & How efficient is a global constraint in practice? - {A} fair experimental framework & \href{works/CauwelaertLS18.pdf}{Yes} & \cite{CauwelaertLS18} & 2018 & Constraints An Int. J. & 36 & 2 & 39 & \ref{b:CauwelaertLS18} & \ref{c:CauwelaertLS18}\\
CappartS17 \href{https://doi.org/10.1007/978-3-319-59776-8\_26}{CappartS17} & \hyperref[auth:a42]{Q. Cappart}, \hyperref[auth:a147]{P. Schaus} & Rescheduling Railway Traffic on Real Time Situations Using Time-Interval Variables & \href{works/CappartS17.pdf}{Yes} & \cite{CappartS17} & 2017 & CPAIOR 2017 & 16 & 2 & 28 & \ref{b:CappartS17} & \ref{c:CappartS17}\\
CauwelaertDMS16 \href{https://doi.org/10.1007/978-3-319-44953-1\_33}{CauwelaertDMS16} & \hyperref[auth:a206]{Sascha Van Cauwelaert}, \hyperref[auth:a207]{C. Dejemeppe}, \hyperref[auth:a149]{J. Monette}, \hyperref[auth:a147]{P. Schaus} & Efficient Filtering for the Unary Resource with Family-Based Transition Times & \href{works/CauwelaertDMS16.pdf}{Yes} & \cite{CauwelaertDMS16} & 2016 & CP 2016 & 16 & 1 & 12 & \ref{b:CauwelaertDMS16} & \ref{c:CauwelaertDMS16}\\
DejemeppeCS15 \href{https://doi.org/10.1007/978-3-319-23219-5\_7}{DejemeppeCS15} & \hyperref[auth:a207]{C. Dejemeppe}, \hyperref[auth:a206]{Sascha Van Cauwelaert}, \hyperref[auth:a147]{P. Schaus} & The Unary Resource with Transition Times & \href{works/DejemeppeCS15.pdf}{Yes} & \cite{DejemeppeCS15} & 2015 & CP 2015 & 16 & 5 & 11 & \ref{b:DejemeppeCS15} & \ref{c:DejemeppeCS15}\\
GayHLS15 \href{https://doi.org/10.1007/978-3-319-23219-5\_10}{GayHLS15} & \hyperref[auth:a216]{S. Gay}, \hyperref[auth:a217]{R. Hartert}, \hyperref[auth:a218]{C. Lecoutre}, \hyperref[auth:a147]{P. Schaus} & Conflict Ordering Search for Scheduling Problems & \href{works/GayHLS15.pdf}{Yes} & \cite{GayHLS15} & 2015 & CP 2015 & 9 & 20 & 15 & \ref{b:GayHLS15} & \ref{c:GayHLS15}\\
GayHS15 \href{https://doi.org/10.1007/978-3-319-23219-5\_11}{GayHS15} & \hyperref[auth:a216]{S. Gay}, \hyperref[auth:a217]{R. Hartert}, \hyperref[auth:a147]{P. Schaus} & Simple and Scalable Time-Table Filtering for the Cumulative Constraint & \href{works/GayHS15.pdf}{Yes} & \cite{GayHS15} & 2015 & CP 2015 & 9 & 10 & 9 & \ref{b:GayHS15} & \ref{c:GayHS15}\\
GayHS15a \href{https://doi.org/10.1007/978-3-319-18008-3\_11}{GayHS15a} & \hyperref[auth:a216]{S. Gay}, \hyperref[auth:a217]{R. Hartert}, \hyperref[auth:a147]{P. Schaus} & Time-Table Disjunctive Reasoning for the Cumulative Constraint & \href{works/GayHS15a.pdf}{Yes} & \cite{GayHS15a} & 2015 & CPAIOR 2015 & 16 & 5 & 12 & \ref{b:GayHS15a} & \ref{c:GayHS15a}\\
GaySS14 \href{https://doi.org/10.1007/978-3-319-10428-7\_59}{GaySS14} & \hyperref[auth:a216]{S. Gay}, \hyperref[auth:a147]{P. Schaus}, \hyperref[auth:a239]{Vivian De Smedt} & Continuous Casting Scheduling with Constraint Programming & \href{works/GaySS14.pdf}{Yes} & \cite{GaySS14} & 2014 & CP 2014 & 15 & 7 & 11 & \ref{b:GaySS14} & \ref{c:GaySS14}\\
HoundjiSWD14 \href{https://doi.org/10.1007/978-3-319-10428-7\_29}{HoundjiSWD14} & \hyperref[auth:a228]{Vinas{\'{e}}tan Ratheil Houndji}, \hyperref[auth:a147]{P. Schaus}, \hyperref[auth:a229]{Laurence A. Wolsey}, \hyperref[auth:a151]{Y. Deville} & The StockingCost Constraint & \href{works/HoundjiSWD14.pdf}{Yes} & \cite{HoundjiSWD14} & 2014 & CP 2014 & 16 & 5 & 7 & \ref{b:HoundjiSWD14} & \ref{c:HoundjiSWD14}\\
SchausHMCMD11 \href{https://doi.org/10.1007/s10601-010-9100-5}{SchausHMCMD11} & \hyperref[auth:a147]{P. Schaus}, \hyperref[auth:a148]{Pascal Van Hentenryck}, \hyperref[auth:a149]{J. Monette}, \hyperref[auth:a150]{C. Coffrin}, \hyperref[auth:a32]{L. Michel}, \hyperref[auth:a151]{Y. Deville} & Solving Steel Mill Slab Problems with constraint-based techniques: CP, LNS, and {CBLS} & \href{works/SchausHMCMD11.pdf}{Yes} & \cite{SchausHMCMD11} & 2011 & Constraints An Int. J. & 23 & 14 & 5 & \ref{b:SchausHMCMD11} & \ref{c:SchausHMCMD11}\\
\end{longtable}
}

\subsection{Works by Roman Bart{\'{a}}k}
\label{sec:a152}
{\scriptsize
\begin{longtable}{>{\raggedright\arraybackslash}p{3cm}>{\raggedright\arraybackslash}p{6cm}>{\raggedright\arraybackslash}p{6.5cm}rrrp{2.5cm}rrrrr}
\rowcolor{white}\caption{Works from bibtex (Total 11)}\\ \toprule
\rowcolor{white}Key & Authors & Title & LC & Cite & Year & \shortstack{Conference\\/Journal} & Pages & \shortstack{Nr\\Cites} & \shortstack{Nr\\Refs} & b & c \\ \midrule\endhead
\bottomrule
\endfoot
SvancaraB22 \href{https://doi.org/10.5220/0010869700003116}{SvancaraB22} & \hyperref[auth:a787]{J. Svancara}, \hyperref[auth:a152]{R. Bart{\'{a}}k} & Tackling Train Routing via Multi-agent Pathfinding and Constraint-based Scheduling & \href{works/SvancaraB22.pdf}{Yes} & \cite{SvancaraB22} & 2022 & ICAART 2022 & 8 & 0 & 0 & \ref{b:SvancaraB22} & \ref{c:SvancaraB22}\\
JelinekB16 \href{https://doi.org/10.1007/978-3-319-28228-2\_1}{JelinekB16} & \hyperref[auth:a788]{J. Jel{\'{\i}}nek}, \hyperref[auth:a152]{R. Bart{\'{a}}k} & Using Constraint Logic Programming to Schedule Solar Array Operations on the International Space Station & \href{works/JelinekB16.pdf}{Yes} & \cite{JelinekB16} & 2016 & PADL 2016 & 10 & 0 & 5 & \ref{b:JelinekB16} & \ref{c:JelinekB16}\\
BartakV15 \href{}{BartakV15} & \hyperref[auth:a152]{R. Bart{\'{a}}k}, \hyperref[auth:a313]{M. Vlk} & Reactive Recovery from Machine Breakdown in Production Scheduling with Temporal Distance and Resource Constraints & \href{works/BartakV15.pdf}{Yes} & \cite{BartakV15} & 2015 & ICAART 2015 & 12 & 0 & 0 & \ref{b:BartakV15} & \ref{c:BartakV15}\\
Bartak14 \href{}{Bartak14} & \hyperref[auth:a152]{R. Bart{\'{a}}k} & Planning and Scheduling & No & \cite{Bartak14} & 2014 & Computing Handbook, Third Edition: Computer Science and Software Engineering & null & 0 & 0 & No & \ref{c:Bartak14}\\
BartakS11 \href{https://doi.org/10.1007/s10601-011-9109-4}{BartakS11} & \hyperref[auth:a152]{R. Bart{\'{a}}k}, \hyperref[auth:a153]{Miguel A. Salido} & Constraint satisfaction for planning and scheduling problems & \href{works/BartakS11.pdf}{Yes} & \cite{BartakS11} & 2011 & Constraints An Int. J. & 5 & 17 & 3 & \ref{b:BartakS11} & \ref{c:BartakS11}\\
BartakCS10 \href{https://doi.org/10.1007/s10479-008-0492-1}{BartakCS10} & \hyperref[auth:a152]{R. Bart{\'{a}}k}, \hyperref[auth:a162]{O. Cepek}, \hyperref[auth:a789]{P. Surynek} & Discovering implied constraints in precedence graphs with alternatives & \href{works/BartakCS10.pdf}{Yes} & \cite{BartakCS10} & 2010 & Ann. Oper. Res. & 31 & 2 & 9 & \ref{b:BartakCS10} & \ref{c:BartakCS10}\\
BartakSR10 \href{https://doi.org/10.1017/S0269888910000202}{BartakSR10} & \hyperref[auth:a152]{R. Bart{\'{a}}k}, \hyperref[auth:a153]{Miguel A. Salido}, \hyperref[auth:a318]{F. Rossi} & New trends in constraint satisfaction, planning, and scheduling: a survey & \href{works/BartakSR10.pdf}{Yes} & \cite{BartakSR10} & 2010 & Knowl. Eng. Rev. & 31 & 28 & 47 & \ref{b:BartakSR10} & \ref{c:BartakSR10}\\
VilimBC05 \href{https://doi.org/10.1007/s10601-005-2814-0}{VilimBC05} & \hyperref[auth:a121]{P. Vil{\'{\i}}m}, \hyperref[auth:a152]{R. Bart{\'{a}}k}, \hyperref[auth:a162]{O. Cepek} & Extension of \emph{O}(\emph{n} log \emph{n}) Filtering Algorithms for the Unary Resource Constraint to Optional Activities & \href{works/VilimBC05.pdf}{Yes} & \cite{VilimBC05} & 2005 & Constraints An Int. J. & 23 & 21 & 5 & \ref{b:VilimBC05} & \ref{c:VilimBC05}\\
VilimBC04 \href{https://doi.org/10.1007/978-3-540-30201-8\_8}{VilimBC04} & \hyperref[auth:a121]{P. Vil{\'{\i}}m}, \hyperref[auth:a152]{R. Bart{\'{a}}k}, \hyperref[auth:a162]{O. Cepek} & Unary Resource Constraint with Optional Activities & \href{works/VilimBC04.pdf}{Yes} & \cite{VilimBC04} & 2004 & CP 2004 & 15 & 13 & 4 & \ref{b:VilimBC04} & \ref{c:VilimBC04}\\
Bartak02 \href{https://doi.org/10.1007/3-540-46135-3\_39}{Bartak02} & \hyperref[auth:a152]{R. Bart{\'{a}}k} & Visopt ShopFloor: On the Edge of Planning and Scheduling & \href{works/Bartak02.pdf}{Yes} & \cite{Bartak02} & 2002 & CP 2002 & 16 & 6 & 4 & \ref{b:Bartak02} & \ref{c:Bartak02}\\
Bartak02a \href{https://doi.org/10.1007/3-540-36607-5\_14}{Bartak02a} & \hyperref[auth:a152]{R. Bart{\'{a}}k} & Visopt ShopFloor: Going Beyond Traditional Scheduling & \href{works/Bartak02a.pdf}{Yes} & \cite{Bartak02a} & 2002 & ERCIM/CologNet 2002 & 15 & 1 & 9 & \ref{b:Bartak02a} & \ref{c:Bartak02a}\\
\end{longtable}
}

\subsection{Works by Philippe Laborie}
\label{sec:a118}
{\scriptsize
\begin{longtable}{>{\raggedright\arraybackslash}p{3cm}>{\raggedright\arraybackslash}p{6cm}>{\raggedright\arraybackslash}p{6.5cm}rrrp{2.5cm}rrrrr}
\rowcolor{white}\caption{Works from bibtex (Total 11)}\\ \toprule
\rowcolor{white}Key & Authors & Title & LC & Cite & Year & \shortstack{Conference\\/Journal} & Pages & \shortstack{Nr\\Cites} & \shortstack{Nr\\Refs} & b & c \\ \midrule\endhead
\bottomrule
\endfoot
LunardiBLRV20 \href{https://doi.org/10.1016/j.cor.2020.105020}{LunardiBLRV20} & \hyperref[auth:a510]{Willian T. Lunardi}, \hyperref[auth:a511]{Ernesto G. Birgin}, \hyperref[auth:a118]{P. Laborie}, \hyperref[auth:a512]{D{\'{e}}bora P. Ronconi}, \hyperref[auth:a513]{H. Voos} & Mixed Integer linear programming and constraint programming models for the online printing shop scheduling problem & \href{works/LunardiBLRV20.pdf}{Yes} & \cite{LunardiBLRV20} & 2020 & Comput. Oper. Res. & 20 & 30 & 18 & \ref{b:LunardiBLRV20} & \ref{c:LunardiBLRV20}\\
Laborie18a \href{https://doi.org/10.1007/978-3-319-93031-2\_29}{Laborie18a} & \hyperref[auth:a118]{P. Laborie} & An Update on the Comparison of MIP, {CP} and Hybrid Approaches for Mixed Resource Allocation and Scheduling & \href{works/Laborie18a.pdf}{Yes} & \cite{Laborie18a} & 2018 & CPAIOR 2018 & 9 & 18 & 10 & \ref{b:Laborie18a} & \ref{c:Laborie18a}\\
LaborieRSV18 \href{https://doi.org/10.1007/s10601-018-9281-x}{LaborieRSV18} & \hyperref[auth:a118]{P. Laborie}, \hyperref[auth:a119]{J. Rogerie}, \hyperref[auth:a120]{P. Shaw}, \hyperref[auth:a121]{P. Vil{\'{\i}}m} & {IBM} {ILOG} {CP} optimizer for scheduling - 20+ years of scheduling with constraints at {IBM/ILOG} & \href{works/LaborieRSV18.pdf}{Yes} & \cite{LaborieRSV18} & 2018 & Constraints An Int. J. & 41 & 148 & 35 & \ref{b:LaborieRSV18} & \ref{c:LaborieRSV18}\\
MelgarejoLS15 \href{https://doi.org/10.1007/978-3-319-18008-3\_1}{MelgarejoLS15} & \hyperref[auth:a324]{P. Aguiar{-}Melgarejo}, \hyperref[auth:a118]{P. Laborie}, \hyperref[auth:a85]{C. Solnon} & A Time-Dependent No-Overlap Constraint: Application to Urban Delivery Problems & \href{works/MelgarejoLS15.pdf}{Yes} & \cite{MelgarejoLS15} & 2015 & CPAIOR 2015 & 17 & 14 & 17 & \ref{b:MelgarejoLS15} & \ref{c:MelgarejoLS15}\\
VilimLS15 \href{https://doi.org/10.1007/978-3-319-18008-3\_30}{VilimLS15} & \hyperref[auth:a121]{P. Vil{\'{\i}}m}, \hyperref[auth:a118]{P. Laborie}, \hyperref[auth:a120]{P. Shaw} & Failure-Directed Search for Constraint-Based Scheduling & \href{works/VilimLS15.pdf}{Yes} & \cite{VilimLS15} & 2015 & CPAIOR 2015 & 17 & 31 & 19 & \ref{b:VilimLS15} & \ref{c:VilimLS15}\\
BidotVLB09 \href{https://doi.org/10.1007/s10951-008-0080-x}{BidotVLB09} & \hyperref[auth:a835]{J. Bidot}, \hyperref[auth:a836]{T. Vidal}, \hyperref[auth:a118]{P. Laborie}, \hyperref[auth:a89]{J. Christopher Beck} & A theoretic and practical framework for scheduling in a stochastic environment & \href{works/BidotVLB09.pdf}{Yes} & \cite{BidotVLB09} & 2009 & J. Sched. & 30 & 58 & 20 & \ref{b:BidotVLB09} & \ref{c:BidotVLB09}\\
Laborie09 \href{https://doi.org/10.1007/978-3-642-01929-6\_12}{Laborie09} & \hyperref[auth:a118]{P. Laborie} & {IBM} {ILOG} {CP} Optimizer for Detailed Scheduling Illustrated on Three Problems & \href{works/Laborie09.pdf}{Yes} & \cite{Laborie09} & 2009 & CPAIOR 2009 & 15 & 53 & 2 & \ref{b:Laborie09} & \ref{c:Laborie09}\\
BaptisteLPN06 \href{https://doi.org/10.1016/S1574-6526(06)80026-X}{BaptisteLPN06} & \hyperref[auth:a163]{P. Baptiste}, \hyperref[auth:a118]{P. Laborie}, \hyperref[auth:a164]{Claude Le Pape}, \hyperref[auth:a666]{W. Nuijten} & Constraint-Based Scheduling and Planning & No & \cite{BaptisteLPN06} & 2006 & Handbook of Constraint Programming & 39 & 30 & 25 & No & \ref{c:BaptisteLPN06}\\
GodardLN05 \href{http://www.aaai.org/Library/ICAPS/2005/icaps05-009.php}{GodardLN05} & \hyperref[auth:a782]{D. Godard}, \hyperref[auth:a118]{P. Laborie}, \hyperref[auth:a666]{W. Nuijten} & Randomized Large Neighborhood Search for Cumulative Scheduling & \href{works/GodardLN05.pdf}{Yes} & \cite{GodardLN05} & 2005 & ICAPS 2005 & 9 & 0 & 0 & \ref{b:GodardLN05} & \ref{c:GodardLN05}\\
Laborie03 \href{http://dx.doi.org/10.1016/s0004-3702(02)00362-4}{Laborie03} & \hyperref[auth:a118]{P. Laborie} & Algorithms for propagating resource constraints in AI planning and scheduling: Existing approaches and new results & \href{works/Laborie03.pdf}{Yes} & \cite{Laborie03} & 2003 & Artificial Intelligence & 38 & 128 & 10 & \ref{b:Laborie03} & \ref{c:Laborie03}\\
FocacciLN00 \href{http://www.aaai.org/Library/AIPS/2000/aips00-010.php}{FocacciLN00} & \hyperref[auth:a784]{F. Focacci}, \hyperref[auth:a118]{P. Laborie}, \hyperref[auth:a666]{W. Nuijten} & Solving Scheduling Problems with Setup Times and Alternative Resources & \href{works/FocacciLN00.pdf}{Yes} & \cite{FocacciLN00} & 2000 & AIPS 2000 & 10 & 0 & 0 & \ref{b:FocacciLN00} & \ref{c:FocacciLN00}\\
\end{longtable}
}

\subsection{Works by Petr Vil{\'{\i}}m}
\label{sec:a121}
{\scriptsize
\begin{longtable}{>{\raggedright\arraybackslash}p{3cm}>{\raggedright\arraybackslash}p{6cm}>{\raggedright\arraybackslash}p{6.5cm}rrrp{2.5cm}rrrrr}
\rowcolor{white}\caption{Works from bibtex (Total 11)}\\ \toprule
\rowcolor{white}Key & Authors & Title & LC & Cite & Year & \shortstack{Conference\\/Journal} & Pages & \shortstack{Nr\\Cites} & \shortstack{Nr\\Refs} & b & c \\ \midrule\endhead
\bottomrule
\endfoot
LaborieRSV18 \href{https://doi.org/10.1007/s10601-018-9281-x}{LaborieRSV18} & \hyperref[auth:a118]{P. Laborie}, \hyperref[auth:a119]{J. Rogerie}, \hyperref[auth:a120]{P. Shaw}, \hyperref[auth:a121]{P. Vil{\'{\i}}m} & {IBM} {ILOG} {CP} optimizer for scheduling - 20+ years of scheduling with constraints at {IBM/ILOG} & \href{works/LaborieRSV18.pdf}{Yes} & \cite{LaborieRSV18} & 2018 & Constraints An Int. J. & 41 & 148 & 35 & \ref{b:LaborieRSV18} & \ref{c:LaborieRSV18}\\
VilimLS15 \href{https://doi.org/10.1007/978-3-319-18008-3\_30}{VilimLS15} & \hyperref[auth:a121]{P. Vil{\'{\i}}m}, \hyperref[auth:a118]{P. Laborie}, \hyperref[auth:a120]{P. Shaw} & Failure-Directed Search for Constraint-Based Scheduling & \href{works/VilimLS15.pdf}{Yes} & \cite{VilimLS15} & 2015 & CPAIOR 2015 & 17 & 31 & 19 & \ref{b:VilimLS15} & \ref{c:VilimLS15}\\
Vilim11 \href{https://doi.org/10.1007/978-3-642-21311-3\_22}{Vilim11} & \hyperref[auth:a121]{P. Vil{\'{\i}}m} & Timetable Edge Finding Filtering Algorithm for Discrete Cumulative Resources & \href{works/Vilim11.pdf}{Yes} & \cite{Vilim11} & 2011 & CPAIOR 2011 & 16 & 28 & 6 & \ref{b:Vilim11} & \ref{c:Vilim11}\\
Vilim09 \href{https://doi.org/10.1007/978-3-642-04244-7\_62}{Vilim09} & \hyperref[auth:a121]{P. Vil{\'{\i}}m} & Edge Finding Filtering Algorithm for Discrete Cumulative Resources in \emph{O}(\emph{kn} log \emph{n})\{{\textbackslash}mathcal O\}(kn \{{\textbackslash}rm log\} n) & \href{works/Vilim09.pdf}{Yes} & \cite{Vilim09} & 2009 & CP 2009 & 15 & 25 & 4 & \ref{b:Vilim09} & \ref{c:Vilim09}\\
Vilim09a \href{https://doi.org/10.1007/978-3-642-01929-6\_22}{Vilim09a} & \hyperref[auth:a121]{P. Vil{\'{\i}}m} & Max Energy Filtering Algorithm for Discrete Cumulative Resources & \href{works/Vilim09a.pdf}{Yes} & \cite{Vilim09a} & 2009 & CPAIOR 2009 & 15 & 13 & 4 & \ref{b:Vilim09a} & \ref{c:Vilim09a}\\
Vilim05 \href{https://doi.org/10.1007/11493853\_29}{Vilim05} & \hyperref[auth:a121]{P. Vil{\'{\i}}m} & Computing Explanations for the Unary Resource Constraint & \href{works/Vilim05.pdf}{Yes} & \cite{Vilim05} & 2005 & CPAIOR 2005 & 14 & 5 & 8 & \ref{b:Vilim05} & \ref{c:Vilim05}\\
VilimBC05 \href{https://doi.org/10.1007/s10601-005-2814-0}{VilimBC05} & \hyperref[auth:a121]{P. Vil{\'{\i}}m}, \hyperref[auth:a152]{R. Bart{\'{a}}k}, \hyperref[auth:a162]{O. Cepek} & Extension of \emph{O}(\emph{n} log \emph{n}) Filtering Algorithms for the Unary Resource Constraint to Optional Activities & \href{works/VilimBC05.pdf}{Yes} & \cite{VilimBC05} & 2005 & Constraints An Int. J. & 23 & 21 & 5 & \ref{b:VilimBC05} & \ref{c:VilimBC05}\\
Vilim04 \href{https://doi.org/10.1007/978-3-540-24664-0\_23}{Vilim04} & \hyperref[auth:a121]{P. Vil{\'{\i}}m} & O(n log n) Filtering Algorithms for Unary Resource Constraint & \href{works/Vilim04.pdf}{Yes} & \cite{Vilim04} & 2004 & CPAIOR 2004 & 13 & 22 & 5 & \ref{b:Vilim04} & \ref{c:Vilim04}\\
VilimBC04 \href{https://doi.org/10.1007/978-3-540-30201-8\_8}{VilimBC04} & \hyperref[auth:a121]{P. Vil{\'{\i}}m}, \hyperref[auth:a152]{R. Bart{\'{a}}k}, \hyperref[auth:a162]{O. Cepek} & Unary Resource Constraint with Optional Activities & \href{works/VilimBC04.pdf}{Yes} & \cite{VilimBC04} & 2004 & CP 2004 & 15 & 13 & 4 & \ref{b:VilimBC04} & \ref{c:VilimBC04}\\
Vilim03 \href{https://doi.org/10.1007/978-3-540-45193-8\_124}{Vilim03} & \hyperref[auth:a121]{P. Vil{\'{\i}}m} & Computing Explanations for Global Scheduling Constraints & \href{works/Vilim03.pdf}{Yes} & \cite{Vilim03} & 2003 & CP 2003 & 1 & 1 & 1 & \ref{b:Vilim03} & \ref{c:Vilim03}\\
Vilim02 \href{https://doi.org/10.1007/3-540-46135-3\_62}{Vilim02} & \hyperref[auth:a121]{P. Vil{\'{\i}}m} & Batch Processing with Sequence Dependent Setup Times & \href{works/Vilim02.pdf}{Yes} & \cite{Vilim02} & 2002 & CP 2002 & 1 & 6 & 1 & \ref{b:Vilim02} & \ref{c:Vilim02}\\
\end{longtable}
}

\subsection{Works by Luca Benini}
\label{sec:a247}
{\scriptsize
\begin{longtable}{>{\raggedright\arraybackslash}p{3cm}>{\raggedright\arraybackslash}p{6cm}>{\raggedright\arraybackslash}p{6.5cm}rrrp{2.5cm}rrrrr}
\rowcolor{white}\caption{Works from bibtex (Total 10)}\\ \toprule
\rowcolor{white}Key & Authors & Title & LC & Cite & Year & \shortstack{Conference\\/Journal} & Pages & \shortstack{Nr\\Cites} & \shortstack{Nr\\Refs} & b & c \\ \midrule\endhead
\bottomrule
\endfoot
BorghesiBLMB18 \href{https://doi.org/10.1016/j.suscom.2018.05.007}{BorghesiBLMB18} & \hyperref[auth:a231]{A. Borghesi}, \hyperref[auth:a230]{A. Bartolini}, \hyperref[auth:a142]{M. Lombardi}, \hyperref[auth:a143]{M. Milano}, \hyperref[auth:a247]{L. Benini} & Scheduling-based power capping in high performance computing systems & \href{works/BorghesiBLMB18.pdf}{Yes} & \cite{BorghesiBLMB18} & 2018 & Sustain. Comput. Informatics Syst. & 13 & 11 & 22 & \ref{b:BorghesiBLMB18} & \ref{c:BorghesiBLMB18}\\
BridiBLMB16 \href{https://doi.org/10.1109/TPDS.2016.2516997}{BridiBLMB16} & \hyperref[auth:a232]{T. Bridi}, \hyperref[auth:a230]{A. Bartolini}, \hyperref[auth:a142]{M. Lombardi}, \hyperref[auth:a143]{M. Milano}, \hyperref[auth:a247]{L. Benini} & A Constraint Programming Scheduler for Heterogeneous High-Performance Computing Machines & \href{works/BridiBLMB16.pdf}{Yes} & \cite{BridiBLMB16} & 2016 & {IEEE} Trans. Parallel Distributed Syst. & 14 & 17 & 22 & \ref{b:BridiBLMB16} & \ref{c:BridiBLMB16}\\
BridiLBBM16 \href{https://doi.org/10.3233/978-1-61499-672-9-1598}{BridiLBBM16} & \hyperref[auth:a232]{T. Bridi}, \hyperref[auth:a142]{M. Lombardi}, \hyperref[auth:a230]{A. Bartolini}, \hyperref[auth:a247]{L. Benini}, \hyperref[auth:a143]{M. Milano} & {DARDIS:} Distributed And Randomized DIspatching and Scheduling & \href{works/BridiLBBM16.pdf}{Yes} & \cite{BridiLBBM16} & 2016 & ECAI 2016 & 2 & 0 & 0 & \ref{b:BridiLBBM16} & \ref{c:BridiLBBM16}\\
BonfiettiLBM14 \href{https://doi.org/10.1016/j.artint.2013.09.006}{BonfiettiLBM14} & \hyperref[auth:a203]{A. Bonfietti}, \hyperref[auth:a142]{M. Lombardi}, \hyperref[auth:a247]{L. Benini}, \hyperref[auth:a143]{M. Milano} & {CROSS} cyclic resource-constrained scheduling solver & \href{works/BonfiettiLBM14.pdf}{Yes} & \cite{BonfiettiLBM14} & 2014 & Artif. Intell. & 28 & 8 & 15 & \ref{b:BonfiettiLBM14} & \ref{c:BonfiettiLBM14}\\
BonfiettiLBM12 \href{https://doi.org/10.1007/978-3-642-29828-8\_6}{BonfiettiLBM12} & \hyperref[auth:a203]{A. Bonfietti}, \hyperref[auth:a142]{M. Lombardi}, \hyperref[auth:a247]{L. Benini}, \hyperref[auth:a143]{M. Milano} & Global Cyclic Cumulative Constraint & \href{works/BonfiettiLBM12.pdf}{Yes} & \cite{BonfiettiLBM12} & 2012 & CPAIOR 2012 & 16 & 2 & 11 & \ref{b:BonfiettiLBM12} & \ref{c:BonfiettiLBM12}\\
BeniniLMR11 \href{https://doi.org/10.1007/s10479-010-0718-x}{BeniniLMR11} & \hyperref[auth:a247]{L. Benini}, \hyperref[auth:a142]{M. Lombardi}, \hyperref[auth:a143]{M. Milano}, \hyperref[auth:a727]{M. Ruggiero} & Optimal resource allocation and scheduling for the {CELL} {BE} platform & \href{works/BeniniLMR11.pdf}{Yes} & \cite{BeniniLMR11} & 2011 & Ann. Oper. Res. & 27 & 18 & 16 & \ref{b:BeniniLMR11} & \ref{c:BeniniLMR11}\\
BonfiettiLBM11 \href{https://doi.org/10.1007/978-3-642-23786-7\_12}{BonfiettiLBM11} & \hyperref[auth:a203]{A. Bonfietti}, \hyperref[auth:a142]{M. Lombardi}, \hyperref[auth:a247]{L. Benini}, \hyperref[auth:a143]{M. Milano} & A Constraint Based Approach to Cyclic {RCPSP} & \href{works/BonfiettiLBM11.pdf}{Yes} & \cite{BonfiettiLBM11} & 2011 & CP 2011 & 15 & 3 & 14 & \ref{b:BonfiettiLBM11} & \ref{c:BonfiettiLBM11}\\
LombardiBMB11 \href{https://doi.org/10.1007/978-3-642-21311-3\_14}{LombardiBMB11} & \hyperref[auth:a142]{M. Lombardi}, \hyperref[auth:a203]{A. Bonfietti}, \hyperref[auth:a143]{M. Milano}, \hyperref[auth:a247]{L. Benini} & Precedence Constraint Posting for Cyclic Scheduling Problems & \href{works/LombardiBMB11.pdf}{Yes} & \cite{LombardiBMB11} & 2011 & CPAIOR 2011 & 17 & 1 & 13 & \ref{b:LombardiBMB11} & \ref{c:LombardiBMB11}\\
RuggieroBBMA09 \href{https://doi.org/10.1109/TCAD.2009.2013536}{RuggieroBBMA09} & \hyperref[auth:a727]{M. Ruggiero}, \hyperref[auth:a379]{D. Bertozzi}, \hyperref[auth:a247]{L. Benini}, \hyperref[auth:a143]{M. Milano}, \hyperref[auth:a728]{A. Andrei} & Reducing the Abstraction and Optimality Gaps in the Allocation and Scheduling for Variable Voltage/Frequency MPSoC Platforms & \href{works/RuggieroBBMA09.pdf}{Yes} & \cite{RuggieroBBMA09} & 2009 & {IEEE} Trans. Comput. Aided Des. Integr. Circuits Syst. & 14 & 9 & 27 & \ref{b:RuggieroBBMA09} & \ref{c:RuggieroBBMA09}\\
BeniniBGM06 \href{https://doi.org/10.1007/11757375\_6}{BeniniBGM06} & \hyperref[auth:a247]{L. Benini}, \hyperref[auth:a379]{D. Bertozzi}, \hyperref[auth:a380]{A. Guerri}, \hyperref[auth:a143]{M. Milano} & Allocation, Scheduling and Voltage Scaling on Energy Aware MPSoCs & \href{works/BeniniBGM06.pdf}{Yes} & \cite{BeniniBGM06} & 2006 & CPAIOR 2006 & 15 & 18 & 10 & \ref{b:BeniniBGM06} & \ref{c:BeniniBGM06}\\
\end{longtable}
}

\subsection{Works by Alessio Bonfietti}
\label{sec:a203}
{\scriptsize
\begin{longtable}{>{\raggedright\arraybackslash}p{3cm}>{\raggedright\arraybackslash}p{6cm}>{\raggedright\arraybackslash}p{6.5cm}rrrp{2.5cm}rrrrr}
\rowcolor{white}\caption{Works from bibtex (Total 10)}\\ \toprule
\rowcolor{white}Key & Authors & Title & LC & Cite & Year & \shortstack{Conference\\/Journal} & Pages & \shortstack{Nr\\Cites} & \shortstack{Nr\\Refs} & b & c \\ \midrule\endhead
\bottomrule
\endfoot
Bonfietti16 \href{https://doi.org/10.3233/IA-160095}{Bonfietti16} & \hyperref[auth:a203]{A. Bonfietti} & A constraint programming scheduling solver for the MPOpt programming environment & \href{works/Bonfietti16.pdf}{Yes} & \cite{Bonfietti16} & 2016 & Intelligenza Artificiale & 13 & 0 & 19 & \ref{b:Bonfietti16} & \ref{c:Bonfietti16}\\
BonfiettiZLM16 \href{https://doi.org/10.1007/978-3-319-44953-1\_8}{BonfiettiZLM16} & \hyperref[auth:a203]{A. Bonfietti}, \hyperref[auth:a204]{A. Zanarini}, \hyperref[auth:a142]{M. Lombardi}, \hyperref[auth:a143]{M. Milano} & The Multirate Resource Constraint & \href{works/BonfiettiZLM16.pdf}{Yes} & \cite{BonfiettiZLM16} & 2016 & CP 2016 & 17 & 0 & 11 & \ref{b:BonfiettiZLM16} & \ref{c:BonfiettiZLM16}\\
LombardiBM15 \href{https://doi.org/10.1007/978-3-319-23219-5\_20}{LombardiBM15} & \hyperref[auth:a142]{M. Lombardi}, \hyperref[auth:a203]{A. Bonfietti}, \hyperref[auth:a143]{M. Milano} & Deterministic Estimation of the Expected Makespan of a {POS} Under Duration Uncertainty & \href{works/LombardiBM15.pdf}{Yes} & \cite{LombardiBM15} & 2015 & CP 2015 & 16 & 0 & 8 & \ref{b:LombardiBM15} & \ref{c:LombardiBM15}\\
BonfiettiLBM14 \href{https://doi.org/10.1016/j.artint.2013.09.006}{BonfiettiLBM14} & \hyperref[auth:a203]{A. Bonfietti}, \hyperref[auth:a142]{M. Lombardi}, \hyperref[auth:a247]{L. Benini}, \hyperref[auth:a143]{M. Milano} & {CROSS} cyclic resource-constrained scheduling solver & \href{works/BonfiettiLBM14.pdf}{Yes} & \cite{BonfiettiLBM14} & 2014 & Artif. Intell. & 28 & 8 & 15 & \ref{b:BonfiettiLBM14} & \ref{c:BonfiettiLBM14}\\
BonfiettiLM14 \href{https://doi.org/10.1007/978-3-319-07046-9\_15}{BonfiettiLM14} & \hyperref[auth:a203]{A. Bonfietti}, \hyperref[auth:a142]{M. Lombardi}, \hyperref[auth:a143]{M. Milano} & Disregarding Duration Uncertainty in Partial Order Schedules? Yes, We Can! & \href{works/BonfiettiLM14.pdf}{Yes} & \cite{BonfiettiLM14} & 2014 & CPAIOR 2014 & 16 & 3 & 12 & \ref{b:BonfiettiLM14} & \ref{c:BonfiettiLM14}\\
BonfiettiLM13 \href{http://www.aaai.org/ocs/index.php/ICAPS/ICAPS13/paper/view/6050}{BonfiettiLM13} & \hyperref[auth:a203]{A. Bonfietti}, \hyperref[auth:a142]{M. Lombardi}, \hyperref[auth:a143]{M. Milano} & De-Cycling Cyclic Scheduling Problems & \href{works/BonfiettiLM13.pdf}{Yes} & \cite{BonfiettiLM13} & 2013 & ICAPS 2013 & 5 & 0 & 0 & \ref{b:BonfiettiLM13} & \ref{c:BonfiettiLM13}\\
BonfiettiLBM12 \href{https://doi.org/10.1007/978-3-642-29828-8\_6}{BonfiettiLBM12} & \hyperref[auth:a203]{A. Bonfietti}, \hyperref[auth:a142]{M. Lombardi}, \hyperref[auth:a247]{L. Benini}, \hyperref[auth:a143]{M. Milano} & Global Cyclic Cumulative Constraint & \href{works/BonfiettiLBM12.pdf}{Yes} & \cite{BonfiettiLBM12} & 2012 & CPAIOR 2012 & 16 & 2 & 11 & \ref{b:BonfiettiLBM12} & \ref{c:BonfiettiLBM12}\\
BonfiettiM12 \href{https://ceur-ws.org/Vol-926/paper2.pdf}{BonfiettiM12} & \hyperref[auth:a203]{A. Bonfietti}, \hyperref[auth:a143]{M. Milano} & A Constraint-based Approach to Cyclic Resource-Constrained Scheduling Problem & \href{works/BonfiettiM12.pdf}{Yes} & \cite{BonfiettiM12} & 2012 & DC SIAAI 2012 & 3 & 0 & 0 & \ref{b:BonfiettiM12} & \ref{c:BonfiettiM12}\\
BonfiettiLBM11 \href{https://doi.org/10.1007/978-3-642-23786-7\_12}{BonfiettiLBM11} & \hyperref[auth:a203]{A. Bonfietti}, \hyperref[auth:a142]{M. Lombardi}, \hyperref[auth:a247]{L. Benini}, \hyperref[auth:a143]{M. Milano} & A Constraint Based Approach to Cyclic {RCPSP} & \href{works/BonfiettiLBM11.pdf}{Yes} & \cite{BonfiettiLBM11} & 2011 & CP 2011 & 15 & 3 & 14 & \ref{b:BonfiettiLBM11} & \ref{c:BonfiettiLBM11}\\
LombardiBMB11 \href{https://doi.org/10.1007/978-3-642-21311-3\_14}{LombardiBMB11} & \hyperref[auth:a142]{M. Lombardi}, \hyperref[auth:a203]{A. Bonfietti}, \hyperref[auth:a143]{M. Milano}, \hyperref[auth:a247]{L. Benini} & Precedence Constraint Posting for Cyclic Scheduling Problems & \href{works/LombardiBMB11.pdf}{Yes} & \cite{LombardiBMB11} & 2011 & CPAIOR 2011 & 17 & 1 & 13 & \ref{b:LombardiBMB11} & \ref{c:LombardiBMB11}\\
\end{longtable}
}

\subsection{Works by Pascal Van Hentenryck}
\label{sec:a148}
{\scriptsize
\begin{longtable}{>{\raggedright\arraybackslash}p{3cm}>{\raggedright\arraybackslash}p{6cm}>{\raggedright\arraybackslash}p{6.5cm}rrrp{2.5cm}rrrrr}
\rowcolor{white}\caption{Works from bibtex (Total 10)}\\ \toprule
\rowcolor{white}Key & Authors & Title & LC & Cite & Year & \shortstack{Conference\\/Journal} & Pages & \shortstack{Nr\\Cites} & \shortstack{Nr\\Refs} & b & c \\ \midrule\endhead
\bottomrule
\endfoot
FontaineMH16 \href{https://doi.org/10.1007/978-3-319-33954-2\_12}{FontaineMH16} & \hyperref[auth:a320]{D. Fontaine}, \hyperref[auth:a321]{Laurent D. Michel}, \hyperref[auth:a148]{Pascal Van Hentenryck} & Parallel Composition of Scheduling Solvers & \href{works/FontaineMH16.pdf}{Yes} & \cite{FontaineMH16} & 2016 & CPAIOR 2016 & 11 & 3 & 0 & \ref{b:FontaineMH16} & \ref{c:FontaineMH16}\\
EvenSH15 \href{https://doi.org/10.1007/978-3-319-23219-5\_40}{EvenSH15} & \hyperref[auth:a219]{C. Even}, \hyperref[auth:a124]{A. Schutt}, \hyperref[auth:a148]{Pascal Van Hentenryck} & A Constraint Programming Approach for Non-preemptive Evacuation Scheduling & \href{works/EvenSH15.pdf}{Yes} & \cite{EvenSH15} & 2015 & CP 2015 & 18 & 3 & 12 & \ref{b:EvenSH15} & \ref{c:EvenSH15}\\
EvenSH15a \href{http://arxiv.org/abs/1505.02487}{EvenSH15a} & \hyperref[auth:a219]{C. Even}, \hyperref[auth:a124]{A. Schutt}, \hyperref[auth:a148]{Pascal Van Hentenryck} & A Constraint Programming Approach for Non-Preemptive Evacuation Scheduling & \href{works/EvenSH15a.pdf}{Yes} & \cite{EvenSH15a} & 2015 & CoRR & 16 & 0 & 0 & \ref{b:EvenSH15a} & \ref{c:EvenSH15a}\\
SchausHMCMD11 \href{https://doi.org/10.1007/s10601-010-9100-5}{SchausHMCMD11} & \hyperref[auth:a147]{P. Schaus}, \hyperref[auth:a148]{Pascal Van Hentenryck}, \hyperref[auth:a149]{J. Monette}, \hyperref[auth:a150]{C. Coffrin}, \hyperref[auth:a32]{L. Michel}, \hyperref[auth:a151]{Y. Deville} & Solving Steel Mill Slab Problems with constraint-based techniques: CP, LNS, and {CBLS} & \href{works/SchausHMCMD11.pdf}{Yes} & \cite{SchausHMCMD11} & 2011 & Constraints An Int. J. & 23 & 14 & 5 & \ref{b:SchausHMCMD11} & \ref{c:SchausHMCMD11}\\
MonetteDH09 \href{http://aaai.org/ocs/index.php/ICAPS/ICAPS09/paper/view/712}{MonetteDH09} & \hyperref[auth:a149]{J. Monette}, \hyperref[auth:a151]{Y. Deville}, \hyperref[auth:a148]{Pascal Van Hentenryck} & Just-In-Time Scheduling with Constraint Programming & \href{works/MonetteDH09.pdf}{Yes} & \cite{MonetteDH09} & 2009 & ICAPS 2009 & 8 & 0 & 0 & \ref{b:MonetteDH09} & \ref{c:MonetteDH09}\\
DoomsH08 \href{https://doi.org/10.1007/978-3-540-68155-7\_8}{DoomsH08} & \hyperref[auth:a363]{G. Dooms}, \hyperref[auth:a148]{Pascal Van Hentenryck} & Gap Reduction Techniques for Online Stochastic Project Scheduling & \href{works/DoomsH08.pdf}{Yes} & \cite{DoomsH08} & 2008 & CPAIOR 2008 & 16 & 1 & 2 & \ref{b:DoomsH08} & \ref{c:DoomsH08}\\
HentenryckM08 \href{https://doi.org/10.1007/978-3-540-68155-7\_41}{HentenryckM08} & \hyperref[auth:a148]{Pascal Van Hentenryck}, \hyperref[auth:a32]{L. Michel} & The Steel Mill Slab Design Problem Revisited & \href{works/HentenryckM08.pdf}{Yes} & \cite{HentenryckM08} & 2008 & CPAIOR 2008 & 5 & 13 & 3 & \ref{b:HentenryckM08} & \ref{c:HentenryckM08}\\
MercierH08 \href{http://dx.doi.org/10.1287/ijoc.1070.0226}{MercierH08} & \hyperref[auth:a865]{L. Mercier}, \hyperref[auth:a148]{Pascal Van Hentenryck} & Edge Finding for Cumulative Scheduling & No & \cite{MercierH08} & 2008 & INFORMS Journal on Computing & null & 32 & 5 & No & \ref{c:MercierH08}\\
HentenryckM04 \href{https://doi.org/10.1007/978-3-540-24664-0\_22}{HentenryckM04} & \hyperref[auth:a148]{Pascal Van Hentenryck}, \hyperref[auth:a32]{L. Michel} & Scheduling Abstractions for Local Search & \href{works/HentenryckM04.pdf}{Yes} & \cite{HentenryckM04} & 2004 & CPAIOR 2004 & 16 & 12 & 14 & \ref{b:HentenryckM04} & \ref{c:HentenryckM04}\\
DincbasSH90 \href{https://doi.org/10.1016/0743-1066(90)90052-7}{DincbasSH90} & \hyperref[auth:a726]{M. Dincbas}, \hyperref[auth:a17]{H. Simonis}, \hyperref[auth:a148]{Pascal Van Hentenryck} & Solving Large Combinatorial Problems in Logic Programming & \href{works/DincbasSH90.pdf}{Yes} & \cite{DincbasSH90} & 1990 & J. Log. Program. & 19 & 86 & 9 & \ref{b:DincbasSH90} & \ref{c:DincbasSH90}\\
\end{longtable}
}

\subsection{Works by Philippe Baptiste}
\label{sec:a163}
{\scriptsize
\begin{longtable}{>{\raggedright\arraybackslash}p{3cm}>{\raggedright\arraybackslash}p{6cm}>{\raggedright\arraybackslash}p{6.5cm}rrrp{2.5cm}rrrrr}
\rowcolor{white}\caption{Works from bibtex (Total 9)}\\ \toprule
\rowcolor{white}Key & Authors & Title & LC & Cite & Year & \shortstack{Conference\\/Journal} & Pages & \shortstack{Nr\\Cites} & \shortstack{Nr\\Refs} & b & c \\ \midrule\endhead
\bottomrule
\endfoot
BaptisteB18 \href{https://doi.org/10.1016/j.dam.2017.05.001}{BaptisteB18} & \hyperref[auth:a163]{P. Baptiste}, \hyperref[auth:a714]{N. Bonifas} & Redundant cumulative constraints to compute preemptive bounds & \href{works/BaptisteB18.pdf}{Yes} & \cite{BaptisteB18} & 2018 & Discret. Appl. Math. & 10 & 3 & 13 & \ref{b:BaptisteB18} & \ref{c:BaptisteB18}\\
Baptiste09 \href{https://doi.org/10.1007/978-3-642-04244-7\_1}{Baptiste09} & \hyperref[auth:a163]{P. Baptiste} & Constraint-Based Schedulers, Do They Really Work? & \href{works/Baptiste09.pdf}{Yes} & \cite{Baptiste09} & 2009 & CP 2009 & 1 & 0 & 0 & \ref{b:Baptiste09} & \ref{c:Baptiste09}\\
BaptisteLPN06 \href{https://doi.org/10.1016/S1574-6526(06)80026-X}{BaptisteLPN06} & \hyperref[auth:a163]{P. Baptiste}, \hyperref[auth:a118]{P. Laborie}, \hyperref[auth:a164]{Claude Le Pape}, \hyperref[auth:a666]{W. Nuijten} & Constraint-Based Scheduling and Planning & No & \cite{BaptisteLPN06} & 2006 & Handbook of Constraint Programming & 39 & 30 & 25 & No & \ref{c:BaptisteLPN06}\\
ArtiouchineB05 \href{https://doi.org/10.1007/11564751\_8}{ArtiouchineB05} & \hyperref[auth:a264]{K. Artiouchine}, \hyperref[auth:a163]{P. Baptiste} & Inter-distance Constraint: An Extension of the All-Different Constraint for Scheduling Equal Length Jobs & \href{works/ArtiouchineB05.pdf}{Yes} & \cite{ArtiouchineB05} & 2005 & CP 2005 & 15 & 3 & 11 & \ref{b:ArtiouchineB05} & \ref{c:ArtiouchineB05}\\
BaptistePN01 \href{http://dx.doi.org/10.1007/978-1-4615-1479-4}{BaptistePN01} & \hyperref[auth:a163]{P. Baptiste}, \hyperref[auth:a164]{Claude Le Pape}, \hyperref[auth:a666]{W. Nuijten} & Constraint-Based Scheduling & No & \cite{BaptistePN01} & 2001 & Book & null & 296 & 0 & No & \ref{c:BaptistePN01}\\
BaptisteP00 \href{https://doi.org/10.1023/A:1009822502231}{BaptisteP00} & \hyperref[auth:a163]{P. Baptiste}, \hyperref[auth:a164]{Claude Le Pape} & Constraint Propagation and Decomposition Techniques for Highly Disjunctive and Highly Cumulative Project Scheduling Problems & \href{works/BaptisteP00.pdf}{Yes} & \cite{BaptisteP00} & 2000 & Constraints An Int. J. & 21 & 46 & 0 & \ref{b:BaptisteP00} & \ref{c:BaptisteP00}\\
PapaB98 \href{https://doi.org/10.1023/A:1009723704757}{PapaB98} & \hyperref[auth:a164]{Claude Le Pape}, \hyperref[auth:a163]{P. Baptiste} & Resource Constraints for Preemptive Job-shop Scheduling & \href{works/PapaB98.pdf}{Yes} & \cite{PapaB98} & 1998 & Constraints An Int. J. & 25 & 14 & 0 & \ref{b:PapaB98} & \ref{c:PapaB98}\\
BaptisteP97 \href{https://doi.org/10.1007/BFb0017454}{BaptisteP97} & \hyperref[auth:a163]{P. Baptiste}, \hyperref[auth:a164]{Claude Le Pape} & Constraint Propagation and Decomposition Techniques for Highly Disjunctive and Highly Cumulative Project Scheduling Problems & \href{works/BaptisteP97.pdf}{Yes} & \cite{BaptisteP97} & 1997 & CP 1997 & 15 & 8 & 10 & \ref{b:BaptisteP97} & \ref{c:BaptisteP97}\\
PapeB97 \href{}{PapeB97} & \hyperref[auth:a164]{Claude Le Pape}, \hyperref[auth:a163]{P. Baptiste} & A Constraint Programming Library for Preemptive and Non-Preemptive Scheduling & No & \cite{PapeB97} & 1997 & PACT 1997 & 20 & 0 & 0 & No & \ref{c:PapeB97}\\
\end{longtable}
}

\subsection{Works by Nysret Musliu}
\label{sec:a45}
{\scriptsize
\begin{longtable}{>{\raggedright\arraybackslash}p{3cm}>{\raggedright\arraybackslash}p{6cm}>{\raggedright\arraybackslash}p{6.5cm}rrrp{2.5cm}rrrrr}
\rowcolor{white}\caption{Works from bibtex (Total 9)}\\ \toprule
\rowcolor{white}Key & Authors & Title & LC & Cite & Year & \shortstack{Conference\\/Journal} & Pages & \shortstack{Nr\\Cites} & \shortstack{Nr\\Refs} & b & c \\ \midrule\endhead
\bottomrule
\endfoot
LacknerMMWW23 \href{https://doi.org/10.1007/s10601-023-09347-2}{LacknerMMWW23} & \hyperref[auth:a62]{M. Lackner}, \hyperref[auth:a63]{C. Mrkvicka}, \hyperref[auth:a45]{N. Musliu}, \hyperref[auth:a46]{D. Walkiewicz}, \hyperref[auth:a43]{F. Winter} & Exact methods for the Oven Scheduling Problem & \href{works/LacknerMMWW23.pdf}{Yes} & \cite{LacknerMMWW23} & 2023 & Constraints An Int. J. & 42 & 0 & 32 & \ref{b:LacknerMMWW23} & \ref{c:LacknerMMWW23}\\
WinterMMW22 \href{https://doi.org/10.4230/LIPIcs.CP.2022.41}{WinterMMW22} & \hyperref[auth:a43]{F. Winter}, \hyperref[auth:a44]{S. Meiswinkel}, \hyperref[auth:a45]{N. Musliu}, \hyperref[auth:a46]{D. Walkiewicz} & Modeling and Solving Parallel Machine Scheduling with Contamination Constraints in the Agricultural Industry & \href{works/WinterMMW22.pdf}{Yes} & \cite{WinterMMW22} & 2022 & CP 2022 & 18 & 0 & 0 & \ref{b:WinterMMW22} & \ref{c:WinterMMW22}\\
GeibingerKKMMW21 \href{https://doi.org/10.1007/978-3-030-78230-6\_29}{GeibingerKKMMW21} & \hyperref[auth:a77]{T. Geibinger}, \hyperref[auth:a78]{L. Kletzander}, \hyperref[auth:a79]{M. Krainz}, \hyperref[auth:a80]{F. Mischek}, \hyperref[auth:a45]{N. Musliu}, \hyperref[auth:a43]{F. Winter} & Physician Scheduling During a Pandemic & \href{works/GeibingerKKMMW21.pdf}{Yes} & \cite{GeibingerKKMMW21} & 2021 & CPAIOR 2021 & 10 & 0 & 6 & \ref{b:GeibingerKKMMW21} & \ref{c:GeibingerKKMMW21}\\
GeibingerMM21 \href{https://doi.org/10.1609/aaai.v35i7.16789}{GeibingerMM21} & \hyperref[auth:a77]{T. Geibinger}, \hyperref[auth:a80]{F. Mischek}, \hyperref[auth:a45]{N. Musliu} & Constraint Logic Programming for Real-World Test Laboratory Scheduling & \href{works/GeibingerMM21.pdf}{Yes} & \cite{GeibingerMM21} & 2021 & AAAI 2021 & 9 & 0 & 0 & \ref{b:GeibingerMM21} & \ref{c:GeibingerMM21}\\
LacknerMMWW21 \href{https://doi.org/10.4230/LIPIcs.CP.2021.37}{LacknerMMWW21} & \hyperref[auth:a62]{M. Lackner}, \hyperref[auth:a63]{C. Mrkvicka}, \hyperref[auth:a45]{N. Musliu}, \hyperref[auth:a46]{D. Walkiewicz}, \hyperref[auth:a43]{F. Winter} & Minimizing Cumulative Batch Processing Time for an Industrial Oven Scheduling Problem & \href{works/LacknerMMWW21.pdf}{Yes} & \cite{LacknerMMWW21} & 2021 & CP 2021 & 18 & 0 & 0 & \ref{b:LacknerMMWW21} & \ref{c:LacknerMMWW21}\\
GeibingerMM19 \href{https://doi.org/10.1007/978-3-030-19212-9\_20}{GeibingerMM19} & \hyperref[auth:a77]{T. Geibinger}, \hyperref[auth:a80]{F. Mischek}, \hyperref[auth:a45]{N. Musliu} & Investigating Constraint Programming for Real World Industrial Test Laboratory Scheduling & \href{works/GeibingerMM19.pdf}{Yes} & \cite{GeibingerMM19} & 2019 & CPAIOR 2019 & 16 & 6 & 15 & \ref{b:GeibingerMM19} & \ref{c:GeibingerMM19}\\
abs-1911-04766 \href{http://arxiv.org/abs/1911.04766}{abs-1911-04766} & \hyperref[auth:a77]{T. Geibinger}, \hyperref[auth:a80]{F. Mischek}, \hyperref[auth:a45]{N. Musliu} & Investigating Constraint Programming and Hybrid Methods for Real World Industrial Test Laboratory Scheduling & \href{works/abs-1911-04766.pdf}{Yes} & \cite{abs-1911-04766} & 2019 & CoRR & 16 & 0 & 0 & \ref{b:abs-1911-04766} & \ref{c:abs-1911-04766}\\
MusliuSS18 \href{https://doi.org/10.1007/978-3-319-93031-2\_31}{MusliuSS18} & \hyperref[auth:a45]{N. Musliu}, \hyperref[auth:a124]{A. Schutt}, \hyperref[auth:a125]{Peter J. Stuckey} & Solver Independent Rotating Workforce Scheduling & \href{works/MusliuSS18.pdf}{Yes} & \cite{MusliuSS18} & 2018 & CPAIOR 2018 & 17 & 7 & 23 & \ref{b:MusliuSS18} & \ref{c:MusliuSS18}\\
KletzanderM17 \href{https://doi.org/10.1007/978-3-319-59776-8\_28}{KletzanderM17} & \hyperref[auth:a78]{L. Kletzander}, \hyperref[auth:a45]{N. Musliu} & A Multi-stage Simulated Annealing Algorithm for the Torpedo Scheduling Problem & \href{works/KletzanderM17.pdf}{Yes} & \cite{KletzanderM17} & 2017 & CPAIOR 2017 & 15 & 1 & 9 & \ref{b:KletzanderM17} & \ref{c:KletzanderM17}\\
\end{longtable}
}

\subsection{Works by Claude{-}Guy Quimper}
\label{sec:a37}
{\scriptsize
\begin{longtable}{>{\raggedright\arraybackslash}p{3cm}>{\raggedright\arraybackslash}p{6cm}>{\raggedright\arraybackslash}p{6.5cm}rrrp{2.5cm}rrrrr}
\rowcolor{white}\caption{Works from bibtex (Total 9)}\\ \toprule
\rowcolor{white}Key & Authors & Title & LC & Cite & Year & \shortstack{Conference\\/Journal} & Pages & \shortstack{Nr\\Cites} & \shortstack{Nr\\Refs} & b & c \\ \midrule\endhead
\bottomrule
\endfoot
BoudreaultSLQ22 \href{https://doi.org/10.4230/LIPIcs.CP.2022.10}{BoudreaultSLQ22} & \hyperref[auth:a34]{R. Boudreault}, \hyperref[auth:a35]{V. Simard}, \hyperref[auth:a36]{D. Lafond}, \hyperref[auth:a37]{C. Quimper} & A Constraint Programming Approach to Ship Refit Project Scheduling & \href{works/BoudreaultSLQ22.pdf}{Yes} & \cite{BoudreaultSLQ22} & 2022 & CP 2022 & 16 & 0 & 0 & \ref{b:BoudreaultSLQ22} & \ref{c:BoudreaultSLQ22}\\
OuelletQ22 \href{https://doi.org/10.1007/978-3-031-08011-1\_21}{OuelletQ22} & \hyperref[auth:a52]{Y. Ouellet}, \hyperref[auth:a37]{C. Quimper} & A MinCumulative Resource Constraint & \href{works/OuelletQ22.pdf}{Yes} & \cite{OuelletQ22} & 2022 & CPAIOR 2022 & 17 & 1 & 22 & \ref{b:OuelletQ22} & \ref{c:OuelletQ22}\\
Mercier-AubinGQ20 \href{https://doi.org/10.1007/978-3-030-58942-4\_22}{Mercier-AubinGQ20} & \hyperref[auth:a86]{A. Mercier{-}Aubin}, \hyperref[auth:a87]{J. Gaudreault}, \hyperref[auth:a37]{C. Quimper} & Leveraging Constraint Scheduling: {A} Case Study to the Textile Industry & \href{works/Mercier-AubinGQ20.pdf}{Yes} & \cite{Mercier-AubinGQ20} & 2020 & CPAIOR 2020 & 13 & 2 & 13 & \ref{b:Mercier-AubinGQ20} & \ref{c:Mercier-AubinGQ20}\\
FahimiOQ18 \href{https://doi.org/10.1007/s10601-018-9282-9}{FahimiOQ18} & \hyperref[auth:a122]{H. Fahimi}, \hyperref[auth:a52]{Y. Ouellet}, \hyperref[auth:a37]{C. Quimper} & Linear-time filtering algorithms for the disjunctive constraint and a quadratic filtering algorithm for the cumulative not-first not-last & \href{works/FahimiOQ18.pdf}{Yes} & \cite{FahimiOQ18} & 2018 & Constraints An Int. J. & 22 & 2 & 20 & \ref{b:FahimiOQ18} & \ref{c:FahimiOQ18}\\
KameugneFGOQ18 \href{https://doi.org/10.1007/978-3-319-93031-2\_23}{KameugneFGOQ18} & \hyperref[auth:a10]{R. Kameugne}, \hyperref[auth:a11]{S{\'{e}}v{\'{e}}rine Betmbe Fetgo}, \hyperref[auth:a315]{V. Gingras}, \hyperref[auth:a52]{Y. Ouellet}, \hyperref[auth:a37]{C. Quimper} & Horizontally Elastic Not-First/Not-Last Filtering Algorithm for Cumulative Resource Constraint & \href{works/KameugneFGOQ18.pdf}{Yes} & \cite{KameugneFGOQ18} & 2018 & CPAIOR 2018 & 17 & 1 & 12 & \ref{b:KameugneFGOQ18} & \ref{c:KameugneFGOQ18}\\
OuelletQ18 \href{https://doi.org/10.1007/978-3-319-93031-2\_34}{OuelletQ18} & \hyperref[auth:a52]{Y. Ouellet}, \hyperref[auth:a37]{C. Quimper} & A O(n {\textbackslash}log {\^{}}2 n) Checker and O(n{\^{}}2 {\textbackslash}log n) Filtering Algorithm for the Energetic Reasoning & \href{works/OuelletQ18.pdf}{Yes} & \cite{OuelletQ18} & 2018 & CPAIOR 2018 & 18 & 6 & 16 & \ref{b:OuelletQ18} & \ref{c:OuelletQ18}\\
GingrasQ16 \href{http://www.ijcai.org/Abstract/16/440}{GingrasQ16} & \hyperref[auth:a315]{V. Gingras}, \hyperref[auth:a37]{C. Quimper} & Generalizing the Edge-Finder Rule for the Cumulative Constraint & \href{works/GingrasQ16.pdf}{Yes} & \cite{GingrasQ16} & 2016 & IJCAI 2016 & 7 & 0 & 0 & \ref{b:GingrasQ16} & \ref{c:GingrasQ16}\\
BessiereHMQW14 \href{https://doi.org/10.1007/978-3-319-07046-9\_23}{BessiereHMQW14} & \hyperref[auth:a333]{C. Bessiere}, \hyperref[auth:a1]{E. Hebrard}, \hyperref[auth:a334]{M. M{\'{e}}nard}, \hyperref[auth:a37]{C. Quimper}, \hyperref[auth:a278]{T. Walsh} & Buffered Resource Constraint: Algorithms and Complexity & \href{works/BessiereHMQW14.pdf}{Yes} & \cite{BessiereHMQW14} & 2014 & CPAIOR 2014 & 16 & 1 & 3 & \ref{b:BessiereHMQW14} & \ref{c:BessiereHMQW14}\\
OuelletQ13 \href{https://doi.org/10.1007/978-3-642-40627-0\_42}{OuelletQ13} & \hyperref[auth:a240]{P. Ouellet}, \hyperref[auth:a37]{C. Quimper} & Time-Table Extended-Edge-Finding for the Cumulative Constraint & \href{works/OuelletQ13.pdf}{Yes} & \cite{OuelletQ13} & 2013 & CP 2013 & 16 & 12 & 14 & \ref{b:OuelletQ13} & \ref{c:OuelletQ13}\\
\end{longtable}
}

\subsection{Works by Tony T. Tran}
\label{sec:a810}
{\scriptsize
\begin{longtable}{>{\raggedright\arraybackslash}p{3cm}>{\raggedright\arraybackslash}p{6cm}>{\raggedright\arraybackslash}p{6.5cm}rrrp{2.5cm}rrrrr}
\rowcolor{white}\caption{Works from bibtex (Total 9)}\\ \toprule
\rowcolor{white}Key & Authors & Title & LC & Cite & Year & \shortstack{Conference\\/Journal} & Pages & \shortstack{Nr\\Cites} & \shortstack{Nr\\Refs} & b & c \\ \midrule\endhead
\bottomrule
\endfoot
TranPZLDB18 \href{https://doi.org/10.1007/s10951-017-0537-x}{TranPZLDB18} & \hyperref[auth:a810]{Tony T. Tran}, \hyperref[auth:a811]{M. Padmanabhan}, \hyperref[auth:a812]{Peter Yun Zhang}, \hyperref[auth:a813]{H. Li}, \hyperref[auth:a814]{Douglas G. Down}, \hyperref[auth:a89]{J. Christopher Beck} & Multi-stage resource-aware scheduling for data centers with heterogeneous servers & \href{works/TranPZLDB18.pdf}{Yes} & \cite{TranPZLDB18} & 2018 & J. Sched. & 17 & 8 & 26 & \ref{b:TranPZLDB18} & \ref{c:TranPZLDB18}\\
TranVNB17 \href{https://doi.org/10.1613/jair.5306}{TranVNB17} & \hyperref[auth:a810]{Tony T. Tran}, \hyperref[auth:a815]{Tiago Stegun Vaquero}, \hyperref[auth:a209]{G. Nejat}, \hyperref[auth:a89]{J. Christopher Beck} & Robots in Retirement Homes: Applying Off-the-Shelf Planning and Scheduling to a Team of Assistive Robots & \href{works/TranVNB17.pdf}{Yes} & \cite{TranVNB17} & 2017 & J. Artif. Intell. Res. & 68 & 12 & 0 & \ref{b:TranVNB17} & \ref{c:TranVNB17}\\
TranVNB17a \href{https://doi.org/10.24963/ijcai.2017/726}{TranVNB17a} & \hyperref[auth:a810]{Tony T. Tran}, \hyperref[auth:a815]{Tiago Stegun Vaquero}, \hyperref[auth:a209]{G. Nejat}, \hyperref[auth:a89]{J. Christopher Beck} & Robots in Retirement Homes: Applying Off-the-Shelf Planning and Scheduling to a Team of Assistive Robots (Extended Abstract) & \href{works/TranVNB17a.pdf}{Yes} & \cite{TranVNB17a} & 2017 & IJCAI 2017 & 5 & 1 & 0 & \ref{b:TranVNB17a} & \ref{c:TranVNB17a}\\
TranAB16 \href{https://doi.org/10.1287/ijoc.2015.0666}{TranAB16} & \hyperref[auth:a810]{Tony T. Tran}, \hyperref[auth:a818]{A. Araujo}, \hyperref[auth:a89]{J. Christopher Beck} & Decomposition Methods for the Parallel Machine Scheduling Problem with Setups & No & \cite{TranAB16} & 2016 & {INFORMS} J. Comput. & 13 & 72 & 28 & No & \ref{c:TranAB16}\\
TranDRFWOVB16 \href{https://doi.org/10.1609/socs.v7i1.18390}{TranDRFWOVB16} & \hyperref[auth:a810]{Tony T. Tran}, \hyperref[auth:a820]{M. Do}, \hyperref[auth:a821]{Eleanor Gilbert Rieffel}, \hyperref[auth:a383]{J. Frank}, \hyperref[auth:a819]{Z. Wang}, \hyperref[auth:a822]{B. O'Gorman}, \hyperref[auth:a823]{D. Venturelli}, \hyperref[auth:a89]{J. Christopher Beck} & A Hybrid Quantum-Classical Approach to Solving Scheduling Problems & \href{works/TranDRFWOVB16.pdf}{Yes} & \cite{TranDRFWOVB16} & 2016 & SOCS 2016 & 9 & 3 & 0 & \ref{b:TranDRFWOVB16} & \ref{c:TranDRFWOVB16}\\
TranWDRFOVB16 \href{http://www.aaai.org/ocs/index.php/WS/AAAIW16/paper/view/12664}{TranWDRFOVB16} & \hyperref[auth:a810]{Tony T. Tran}, \hyperref[auth:a819]{Z. Wang}, \hyperref[auth:a820]{M. Do}, \hyperref[auth:a821]{Eleanor Gilbert Rieffel}, \hyperref[auth:a383]{J. Frank}, \hyperref[auth:a822]{B. O'Gorman}, \hyperref[auth:a823]{D. Venturelli}, \hyperref[auth:a89]{J. Christopher Beck} & Explorations of Quantum-Classical Approaches to Scheduling a Mars Lander Activity Problem & \href{works/TranWDRFOVB16.pdf}{Yes} & \cite{TranWDRFOVB16} & 2016 & AAAI 2016 & 9 & 0 & 0 & \ref{b:TranWDRFOVB16} & \ref{c:TranWDRFOVB16}\\
TerekhovTDB14 \href{https://doi.org/10.1613/jair.4278}{TerekhovTDB14} & \hyperref[auth:a829]{D. Terekhov}, \hyperref[auth:a810]{Tony T. Tran}, \hyperref[auth:a814]{Douglas G. Down}, \hyperref[auth:a89]{J. Christopher Beck} & Integrating Queueing Theory and Scheduling for Dynamic Scheduling Problems & \href{works/TerekhovTDB14.pdf}{Yes} & \cite{TerekhovTDB14} & 2014 & J. Artif. Intell. Res. & 38 & 12 & 0 & \ref{b:TerekhovTDB14} & \ref{c:TerekhovTDB14}\\
TranTDB13 \href{http://www.aaai.org/ocs/index.php/ICAPS/ICAPS13/paper/view/6005}{TranTDB13} & \hyperref[auth:a810]{Tony T. Tran}, \hyperref[auth:a829]{D. Terekhov}, \hyperref[auth:a814]{Douglas G. Down}, \hyperref[auth:a89]{J. Christopher Beck} & Hybrid Queueing Theory and Scheduling Models for Dynamic Environments with Sequence-Dependent Setup Times & \href{works/TranTDB13.pdf}{Yes} & \cite{TranTDB13} & 2013 & ICAPS 2013 & 9 & 0 & 0 & \ref{b:TranTDB13} & \ref{c:TranTDB13}\\
TranB12 \href{https://doi.org/10.3233/978-1-61499-098-7-774}{TranB12} & \hyperref[auth:a810]{Tony T. Tran}, \hyperref[auth:a89]{J. Christopher Beck} & Logic-based Benders Decomposition for Alternative Resource Scheduling with Sequence Dependent Setups & \href{works/TranB12.pdf}{Yes} & \cite{TranB12} & 2012 & ECAI 2012 & 6 & 0 & 0 & \ref{b:TranB12} & \ref{c:TranB12}\\
\end{longtable}
}

\subsection{Works by Mats Carlsson}
\label{sec:a91}
{\scriptsize
\begin{longtable}{>{\raggedright\arraybackslash}p{3cm}>{\raggedright\arraybackslash}p{6cm}>{\raggedright\arraybackslash}p{6.5cm}rrrp{2.5cm}rrrrr}
\rowcolor{white}\caption{Works from bibtex (Total 8)}\\ \toprule
\rowcolor{white}Key & Authors & Title & LC & Cite & Year & \shortstack{Conference\\/Journal} & Pages & \shortstack{Nr\\Cites} & \shortstack{Nr\\Refs} & b & c \\ \midrule\endhead
\bottomrule
\endfoot
WessenCS20 \href{https://doi.org/10.1007/978-3-030-58942-4\_33}{WessenCS20} & \hyperref[auth:a90]{J. Wess{\'{e}}n}, \hyperref[auth:a91]{M. Carlsson}, \hyperref[auth:a92]{C. Schulte} & Scheduling of Dual-Arm Multi-tool Assembly Robots and Workspace Layout Optimization & \href{works/WessenCS20.pdf}{Yes} & \cite{WessenCS20} & 2020 & CPAIOR 2020 & 10 & 2 & 11 & \ref{b:WessenCS20} & \ref{c:WessenCS20}\\
MossigeGSMC17 \href{https://doi.org/10.1007/978-3-319-66158-2\_25}{MossigeGSMC17} & \hyperref[auth:a199]{M. Mossige}, \hyperref[auth:a200]{A. Gotlieb}, \hyperref[auth:a201]{H. Spieker}, \hyperref[auth:a202]{H. Meling}, \hyperref[auth:a91]{M. Carlsson} & Time-Aware Test Case Execution Scheduling for Cyber-Physical Systems & \href{works/MossigeGSMC17.pdf}{Yes} & \cite{MossigeGSMC17} & 2017 & CP 2017 & 18 & 6 & 33 & \ref{b:MossigeGSMC17} & \ref{c:MossigeGSMC17}\\
LetortCB15 \href{https://doi.org/10.1007/s10601-014-9172-8}{LetortCB15} & \hyperref[auth:a127]{A. Letort}, \hyperref[auth:a91]{M. Carlsson}, \hyperref[auth:a128]{N. Beldiceanu} & Synchronized sweep algorithms for scalable scheduling constraints & \href{works/LetortCB15.pdf}{Yes} & \cite{LetortCB15} & 2015 & Constraints An Int. J. & 52 & 2 & 14 & \ref{b:LetortCB15} & \ref{c:LetortCB15}\\
LetortCB13 \href{https://doi.org/10.1007/978-3-642-38171-3\_10}{LetortCB13} & \hyperref[auth:a127]{A. Letort}, \hyperref[auth:a91]{M. Carlsson}, \hyperref[auth:a128]{N. Beldiceanu} & A Synchronized Sweep Algorithm for the \emph{k-dimensional cumulative} Constraint & \href{works/LetortCB13.pdf}{Yes} & \cite{LetortCB13} & 2013 & CPAIOR 2013 & 16 & 3 & 10 & \ref{b:LetortCB13} & \ref{c:LetortCB13}\\
LetortBC12 \href{https://doi.org/10.1007/978-3-642-33558-7\_33}{LetortBC12} & \hyperref[auth:a127]{A. Letort}, \hyperref[auth:a128]{N. Beldiceanu}, \hyperref[auth:a91]{M. Carlsson} & A Scalable Sweep Algorithm for the cumulative Constraint & \href{works/LetortBC12.pdf}{Yes} & \cite{LetortBC12} & 2012 & CP 2012 & 16 & 18 & 12 & \ref{b:LetortBC12} & \ref{c:LetortBC12}\\
BeldiceanuCDP11 \href{https://doi.org/10.1007/s10479-010-0731-0}{BeldiceanuCDP11} & \hyperref[auth:a128]{N. Beldiceanu}, \hyperref[auth:a91]{M. Carlsson}, \hyperref[auth:a245]{S. Demassey}, \hyperref[auth:a362]{E. Poder} & New filtering for the \emph{cumulative} constraint in the context of non-overlapping rectangles & \href{works/BeldiceanuCDP11.pdf}{Yes} & \cite{BeldiceanuCDP11} & 2011 & Ann. Oper. Res. & 24 & 8 & 8 & \ref{b:BeldiceanuCDP11} & \ref{c:BeldiceanuCDP11}\\
BeldiceanuCP08 \href{https://doi.org/10.1007/978-3-540-68155-7\_5}{BeldiceanuCP08} & \hyperref[auth:a128]{N. Beldiceanu}, \hyperref[auth:a91]{M. Carlsson}, \hyperref[auth:a362]{E. Poder} & New Filtering for the cumulative Constraint in the Context of Non-Overlapping Rectangles & \href{works/BeldiceanuCP08.pdf}{Yes} & \cite{BeldiceanuCP08} & 2008 & CPAIOR 2008 & 15 & 8 & 9 & \ref{b:BeldiceanuCP08} & \ref{c:BeldiceanuCP08}\\
BeldiceanuC02 \href{https://doi.org/10.1007/3-540-46135-3\_5}{BeldiceanuC02} & \hyperref[auth:a128]{N. Beldiceanu}, \hyperref[auth:a91]{M. Carlsson} & A New Multi-resource cumulatives Constraint with Negative Heights & \href{works/BeldiceanuC02.pdf}{Yes} & \cite{BeldiceanuC02} & 2002 & CP 2002 & 17 & 33 & 9 & \ref{b:BeldiceanuC02} & \ref{c:BeldiceanuC02}\\
\end{longtable}
}

\subsection{Works by Claude Le Pape}
\label{sec:a164}
{\scriptsize
\begin{longtable}{>{\raggedright\arraybackslash}p{3cm}>{\raggedright\arraybackslash}p{6cm}>{\raggedright\arraybackslash}p{6.5cm}rrrp{2.5cm}rrrrr}
\rowcolor{white}\caption{Works from bibtex (Total 8)}\\ \toprule
\rowcolor{white}Key & Authors & Title & LC & Cite & Year & \shortstack{Conference\\/Journal} & Pages & \shortstack{Nr\\Cites} & \shortstack{Nr\\Refs} & b & c \\ \midrule\endhead
\bottomrule
\endfoot
BaptisteLPN06 \href{https://doi.org/10.1016/S1574-6526(06)80026-X}{BaptisteLPN06} & \hyperref[auth:a163]{P. Baptiste}, \hyperref[auth:a118]{P. Laborie}, \hyperref[auth:a164]{Claude Le Pape}, \hyperref[auth:a666]{W. Nuijten} & Constraint-Based Scheduling and Planning & No & \cite{BaptisteLPN06} & 2006 & Handbook of Constraint Programming & 39 & 30 & 25 & No & \ref{c:BaptisteLPN06}\\
BaptistePN01 \href{http://dx.doi.org/10.1007/978-1-4615-1479-4}{BaptistePN01} & \hyperref[auth:a163]{P. Baptiste}, \hyperref[auth:a164]{Claude Le Pape}, \hyperref[auth:a666]{W. Nuijten} & Constraint-Based Scheduling & No & \cite{BaptistePN01} & 2001 & Book & null & 296 & 0 & No & \ref{c:BaptistePN01}\\
BaptisteP00 \href{https://doi.org/10.1023/A:1009822502231}{BaptisteP00} & \hyperref[auth:a163]{P. Baptiste}, \hyperref[auth:a164]{Claude Le Pape} & Constraint Propagation and Decomposition Techniques for Highly Disjunctive and Highly Cumulative Project Scheduling Problems & \href{works/BaptisteP00.pdf}{Yes} & \cite{BaptisteP00} & 2000 & Constraints An Int. J. & 21 & 46 & 0 & \ref{b:BaptisteP00} & \ref{c:BaptisteP00}\\
NuijtenP98 \href{https://doi.org/10.1023/A:1009687210594}{NuijtenP98} & \hyperref[auth:a666]{W. Nuijten}, \hyperref[auth:a164]{Claude Le Pape} & Constraint-Based Job Shop Scheduling with {\textbackslash}sc Ilog Scheduler & \href{works/NuijtenP98.pdf}{Yes} & \cite{NuijtenP98} & 1998 & J. Heuristics & 16 & 42 & 0 & \ref{b:NuijtenP98} & \ref{c:NuijtenP98}\\
PapaB98 \href{https://doi.org/10.1023/A:1009723704757}{PapaB98} & \hyperref[auth:a164]{Claude Le Pape}, \hyperref[auth:a163]{P. Baptiste} & Resource Constraints for Preemptive Job-shop Scheduling & \href{works/PapaB98.pdf}{Yes} & \cite{PapaB98} & 1998 & Constraints An Int. J. & 25 & 14 & 0 & \ref{b:PapaB98} & \ref{c:PapaB98}\\
BaptisteP97 \href{https://doi.org/10.1007/BFb0017454}{BaptisteP97} & \hyperref[auth:a163]{P. Baptiste}, \hyperref[auth:a164]{Claude Le Pape} & Constraint Propagation and Decomposition Techniques for Highly Disjunctive and Highly Cumulative Project Scheduling Problems & \href{works/BaptisteP97.pdf}{Yes} & \cite{BaptisteP97} & 1997 & CP 1997 & 15 & 8 & 10 & \ref{b:BaptisteP97} & \ref{c:BaptisteP97}\\
PapeB97 \href{}{PapeB97} & \hyperref[auth:a164]{Claude Le Pape}, \hyperref[auth:a163]{P. Baptiste} & A Constraint Programming Library for Preemptive and Non-Preemptive Scheduling & No & \cite{PapeB97} & 1997 & PACT 1997 & 20 & 0 & 0 & No & \ref{c:PapeB97}\\
Pape94 \href{http://dx.doi.org/10.1049/ise.1994.0009}{Pape94} & \hyperref[auth:a164]{Claude Le Pape} & Implementation of resource constraints in ILOG SCHEDULE: a library for the development of constraint-based scheduling systems & No & \cite{Pape94} & 1994 & Intelligent Systems Engineering & 1 & 98 & 0 & No & \ref{c:Pape94}\\
\end{longtable}
}

\subsection{Works by Mark Wallace}
\label{sec:a117}
{\scriptsize
\begin{longtable}{>{\raggedright\arraybackslash}p{3cm}>{\raggedright\arraybackslash}p{6cm}>{\raggedright\arraybackslash}p{6.5cm}rrrp{2.5cm}rrrrr}
\rowcolor{white}\caption{Works from bibtex (Total 8)}\\ \toprule
\rowcolor{white}Key & Authors & Title & LC & Cite & Year & \shortstack{Conference\\/Journal} & Pages & \shortstack{Nr\\Cites} & \shortstack{Nr\\Refs} & b & c \\ \midrule\endhead
\bottomrule
\endfoot
WallaceY20 \href{https://doi.org/10.1007/s10601-020-09316-z}{WallaceY20} & \hyperref[auth:a117]{M. Wallace}, \hyperref[auth:a19]{N. Yorke{-}Smith} & A new constraint programming model and solving for the cyclic hoist scheduling problem & \href{works/WallaceY20.pdf}{Yes} & \cite{WallaceY20} & 2020 & Constraints An Int. J. & 19 & 5 & 18 & \ref{b:WallaceY20} & \ref{c:WallaceY20}\\
He0GLW18 \href{https://doi.org/10.1007/978-3-319-98334-9\_42}{He0GLW18} & \hyperref[auth:a185]{S. He}, \hyperref[auth:a117]{M. Wallace}, \hyperref[auth:a186]{G. Gange}, \hyperref[auth:a187]{A. Liebman}, \hyperref[auth:a188]{C. Wilson} & A Fast and Scalable Algorithm for Scheduling Large Numbers of Devices Under Real-Time Pricing & \href{works/He0GLW18.pdf}{Yes} & \cite{He0GLW18} & 2018 & CP 2018 & 18 & 6 & 26 & \ref{b:He0GLW18} & \ref{c:He0GLW18}\\
ThiruvadyWGS14 \href{https://doi.org/10.1007/s10732-014-9260-3}{ThiruvadyWGS14} & \hyperref[auth:a400]{Dhananjay R. Thiruvady}, \hyperref[auth:a117]{M. Wallace}, \hyperref[auth:a341]{H. Gu}, \hyperref[auth:a124]{A. Schutt} & A Lagrangian relaxation and {ACO} hybrid for resource constrained project scheduling with discounted cash flows & \href{works/ThiruvadyWGS14.pdf}{Yes} & \cite{ThiruvadyWGS14} & 2014 & J. Heuristics & 34 & 19 & 18 & \ref{b:ThiruvadyWGS14} & \ref{c:ThiruvadyWGS14}\\
SchuttFSW09 \href{https://doi.org/10.1007/978-3-642-04244-7\_58}{SchuttFSW09} & \hyperref[auth:a124]{A. Schutt}, \hyperref[auth:a154]{T. Feydy}, \hyperref[auth:a125]{Peter J. Stuckey}, \hyperref[auth:a117]{M. Wallace} & Why Cumulative Decomposition Is Not as Bad as It Sounds & \href{works/SchuttFSW09.pdf}{Yes} & \cite{SchuttFSW09} & 2009 & CP 2009 & 16 & 34 & 11 & \ref{b:SchuttFSW09} & \ref{c:SchuttFSW09}\\
SakkoutW00 \href{https://doi.org/10.1023/A:1009856210543}{SakkoutW00} & \hyperref[auth:a167]{Hani El Sakkout}, \hyperref[auth:a117]{M. Wallace} & Probe Backtrack Search for Minimal Perturbation in Dynamic Scheduling & \href{works/SakkoutW00.pdf}{Yes} & \cite{SakkoutW00} & 2000 & Constraints An Int. J. & 30 & 73 & 0 & \ref{b:SakkoutW00} & \ref{c:SakkoutW00}\\
RodosekW98 \href{https://doi.org/10.1007/3-540-49481-2\_28}{RodosekW98} & \hyperref[auth:a299]{R. Rodosek}, \hyperref[auth:a117]{M. Wallace} & A Generic Model and Hybrid Algorithm for Hoist Scheduling Problems & \href{works/RodosekW98.pdf}{Yes} & \cite{RodosekW98} & 1998 & CP 1998 & 15 & 19 & 10 & \ref{b:RodosekW98} & \ref{c:RodosekW98}\\
Wallace96 \href{https://doi.org/10.1007/BF00143881}{Wallace96} & \hyperref[auth:a117]{M. Wallace} & Practical Applications of Constraint Programming & \href{works/Wallace96.pdf}{Yes} & \cite{Wallace96} & 1996 & Constraints An Int. J. & 30 & 87 & 55 & \ref{b:Wallace96} & \ref{c:Wallace96}\\
Wallace94 \href{}{Wallace94} & \hyperref[auth:a117]{M. Wallace} & Applying Constraints for Scheduling & No & \cite{Wallace94} & 1994 & Constraint Programming 1994 & 19 & 0 & 0 & No & \ref{c:Wallace94}\\
\end{longtable}
}

\subsection{Works by Thibaut Feydy}
\label{sec:a154}
{\scriptsize
\begin{longtable}{>{\raggedright\arraybackslash}p{3cm}>{\raggedright\arraybackslash}p{6cm}>{\raggedright\arraybackslash}p{6.5cm}rrrp{2.5cm}rrrrr}
\rowcolor{white}\caption{Works from bibtex (Total 7)}\\ \toprule
\rowcolor{white}Key & Authors & Title & LC & Cite & Year & \shortstack{Conference\\/Journal} & Pages & \shortstack{Nr\\Cites} & \shortstack{Nr\\Refs} & b & c \\ \midrule\endhead
\bottomrule
\endfoot
YoungFS17 \href{https://doi.org/10.1007/978-3-319-66158-2\_20}{YoungFS17} & \hyperref[auth:a193]{Kenneth D. Young}, \hyperref[auth:a154]{T. Feydy}, \hyperref[auth:a124]{A. Schutt} & Constraint Programming Applied to the Multi-Skill Project Scheduling Problem & \href{works/YoungFS17.pdf}{Yes} & \cite{YoungFS17} & 2017 & CP 2017 & 10 & 6 & 21 & \ref{b:YoungFS17} & \ref{c:YoungFS17}\\
SchuttFS13 \href{https://doi.org/10.1007/978-3-642-40627-0\_47}{SchuttFS13} & \hyperref[auth:a124]{A. Schutt}, \hyperref[auth:a154]{T. Feydy}, \hyperref[auth:a125]{Peter J. Stuckey} & Scheduling Optional Tasks with Explanation & \href{works/SchuttFS13.pdf}{Yes} & \cite{SchuttFS13} & 2013 & CP 2013 & 17 & 10 & 20 & \ref{b:SchuttFS13} & \ref{c:SchuttFS13}\\
SchuttFS13a \href{https://doi.org/10.1007/978-3-642-38171-3\_16}{SchuttFS13a} & \hyperref[auth:a124]{A. Schutt}, \hyperref[auth:a154]{T. Feydy}, \hyperref[auth:a125]{Peter J. Stuckey} & Explaining Time-Table-Edge-Finding Propagation for the Cumulative Resource Constraint & \href{works/SchuttFS13a.pdf}{Yes} & \cite{SchuttFS13a} & 2013 & CPAIOR 2013 & 17 & 20 & 27 & \ref{b:SchuttFS13a} & \ref{c:SchuttFS13a}\\
SchuttFSW13 \href{https://doi.org/10.1007/s10951-012-0285-x}{SchuttFSW13} & \hyperref[auth:a124]{A. Schutt}, \hyperref[auth:a154]{T. Feydy}, \hyperref[auth:a125]{Peter J. Stuckey}, \hyperref[auth:a155]{Mark G. Wallace} & Solving RCPSP/max by lazy clause generation & \href{works/SchuttFSW13.pdf}{Yes} & \cite{SchuttFSW13} & 2013 & J. Sched. & 17 & 43 & 23 & \ref{b:SchuttFSW13} & \ref{c:SchuttFSW13}\\
SchuttFSW11 \href{https://doi.org/10.1007/s10601-010-9103-2}{SchuttFSW11} & \hyperref[auth:a124]{A. Schutt}, \hyperref[auth:a154]{T. Feydy}, \hyperref[auth:a125]{Peter J. Stuckey}, \hyperref[auth:a155]{Mark G. Wallace} & Explaining the cumulative propagator & \href{works/SchuttFSW11.pdf}{Yes} & \cite{SchuttFSW11} & 2011 & Constraints An Int. J. & 33 & 57 & 23 & \ref{b:SchuttFSW11} & \ref{c:SchuttFSW11}\\
abs-1009-0347 \href{http://arxiv.org/abs/1009.0347}{abs-1009-0347} & \hyperref[auth:a124]{A. Schutt}, \hyperref[auth:a154]{T. Feydy}, \hyperref[auth:a125]{Peter J. Stuckey}, \hyperref[auth:a155]{Mark G. Wallace} & Solving the Resource Constrained Project Scheduling Problem with Generalized Precedences by Lazy Clause Generation & \href{works/abs-1009-0347.pdf}{Yes} & \cite{abs-1009-0347} & 2010 & CoRR & 37 & 0 & 0 & \ref{b:abs-1009-0347} & \ref{c:abs-1009-0347}\\
SchuttFSW09 \href{https://doi.org/10.1007/978-3-642-04244-7\_58}{SchuttFSW09} & \hyperref[auth:a124]{A. Schutt}, \hyperref[auth:a154]{T. Feydy}, \hyperref[auth:a125]{Peter J. Stuckey}, \hyperref[auth:a117]{M. Wallace} & Why Cumulative Decomposition Is Not as Bad as It Sounds & \href{works/SchuttFSW09.pdf}{Yes} & \cite{SchuttFSW09} & 2009 & CP 2009 & 16 & 34 & 11 & \ref{b:SchuttFSW09} & \ref{c:SchuttFSW09}\\
\end{longtable}
}

\subsection{Works by Diarmuid Grimes}
\label{sec:a182}
{\scriptsize
\begin{longtable}{>{\raggedright\arraybackslash}p{3cm}>{\raggedright\arraybackslash}p{6cm}>{\raggedright\arraybackslash}p{6.5cm}rrrp{2.5cm}rrrrr}
\rowcolor{white}\caption{Works from bibtex (Total 7)}\\ \toprule
\rowcolor{white}Key & Authors & Title & LC & Cite & Year & \shortstack{Conference\\/Journal} & Pages & \shortstack{Nr\\Cites} & \shortstack{Nr\\Refs} & b & c \\ \midrule\endhead
\bottomrule
\endfoot
AntunesABDEGGOL20 \href{https://doi.org/10.1142/S0218213020600076}{AntunesABDEGGOL20} & \hyperref[auth:a893]{M. Antunes}, \hyperref[auth:a894]{V. Armant}, \hyperref[auth:a222]{Kenneth N. Brown}, \hyperref[auth:a895]{Daniel A. Desmond}, \hyperref[auth:a896]{G. Escamocher}, \hyperref[auth:a897]{A. George}, \hyperref[auth:a182]{D. Grimes}, \hyperref[auth:a898]{M. O'Keeffe}, \hyperref[auth:a899]{Y. Lin}, \hyperref[auth:a16]{B. O'Sullivan}, \hyperref[auth:a900]{C. Ozturk}, \hyperref[auth:a901]{L. Quesada}, \hyperref[auth:a129]{M. Siala}, \hyperref[auth:a17]{H. Simonis}, \hyperref[auth:a837]{N. Wilson} & Assigning and Scheduling Service Visits in a Mixed Urban/Rural Setting & No & \cite{AntunesABDEGGOL20} & 2020 & Int. J. Artif. Intell. Tools & 31 & 0 & 16 & No & \ref{c:AntunesABDEGGOL20}\\
AntunesABDEGGOL18 \href{https://doi.org/10.1109/ICTAI.2018.00027}{AntunesABDEGGOL18} & \hyperref[auth:a893]{M. Antunes}, \hyperref[auth:a894]{V. Armant}, \hyperref[auth:a222]{Kenneth N. Brown}, \hyperref[auth:a895]{Daniel A. Desmond}, \hyperref[auth:a896]{G. Escamocher}, \hyperref[auth:a897]{A. George}, \hyperref[auth:a182]{D. Grimes}, \hyperref[auth:a898]{M. O'Keeffe}, \hyperref[auth:a899]{Y. Lin}, \hyperref[auth:a16]{B. O'Sullivan}, \hyperref[auth:a900]{C. Ozturk}, \hyperref[auth:a901]{L. Quesada}, \hyperref[auth:a129]{M. Siala}, \hyperref[auth:a17]{H. Simonis}, \hyperref[auth:a837]{N. Wilson} & Assigning and Scheduling Service Visits in a Mixed Urban/Rural Setting & No & \cite{AntunesABDEGGOL18} & 2018 & ICTAI 2018 & 8 & 1 & 24 & No & \ref{c:AntunesABDEGGOL18}\\
GrimesH15 \href{https://doi.org/10.1287/ijoc.2014.0625}{GrimesH15} & \hyperref[auth:a182]{D. Grimes}, \hyperref[auth:a1]{E. Hebrard} & Solving Variants of the Job Shop Scheduling Problem Through Conflict-Directed Search & No & \cite{GrimesH15} & 2015 & {INFORMS} J. Comput. & 17 & 12 & 41 & No & \ref{c:GrimesH15}\\
GrimesIOS14 \href{https://doi.org/10.1016/j.suscom.2014.08.009}{GrimesIOS14} & \hyperref[auth:a182]{D. Grimes}, \hyperref[auth:a183]{G. Ifrim}, \hyperref[auth:a16]{B. O'Sullivan}, \hyperref[auth:a17]{H. Simonis} & Analyzing the impact of electricity price forecasting on energy cost-aware scheduling & \href{works/GrimesIOS14.pdf}{Yes} & \cite{GrimesIOS14} & 2014 & Sustain. Comput. Informatics Syst. & 16 & 6 & 7 & \ref{b:GrimesIOS14} & \ref{c:GrimesIOS14}\\
GrimesH11 \href{https://doi.org/10.1007/978-3-642-23786-7\_28}{GrimesH11} & \hyperref[auth:a182]{D. Grimes}, \hyperref[auth:a1]{E. Hebrard} & Models and Strategies for Variants of the Job Shop Scheduling Problem & \href{works/GrimesH11.pdf}{Yes} & \cite{GrimesH11} & 2011 & CP 2011 & 17 & 5 & 18 & \ref{b:GrimesH11} & \ref{c:GrimesH11}\\
GrimesH10 \href{https://doi.org/10.1007/978-3-642-13520-0\_19}{GrimesH10} & \hyperref[auth:a182]{D. Grimes}, \hyperref[auth:a1]{E. Hebrard} & Job Shop Scheduling with Setup Times and Maximal Time-Lags: {A} Simple Constraint Programming Approach & \href{works/GrimesH10.pdf}{Yes} & \cite{GrimesH10} & 2010 & CPAIOR 2010 & 15 & 13 & 20 & \ref{b:GrimesH10} & \ref{c:GrimesH10}\\
GrimesHM09 \href{https://doi.org/10.1007/978-3-642-04244-7\_33}{GrimesHM09} & \hyperref[auth:a182]{D. Grimes}, \hyperref[auth:a1]{E. Hebrard}, \hyperref[auth:a82]{A. Malapert} & Closing the Open Shop: Contradicting Conventional Wisdom & \href{works/GrimesHM09.pdf}{Yes} & \cite{GrimesHM09} & 2009 & CP 2009 & 9 & 15 & 12 & \ref{b:GrimesHM09} & \ref{c:GrimesHM09}\\
\end{longtable}
}

\subsection{Works by Zdenek Hanz{\'{a}}lek}
\label{sec:a116}
{\scriptsize
\begin{longtable}{>{\raggedright\arraybackslash}p{3cm}>{\raggedright\arraybackslash}p{6cm}>{\raggedright\arraybackslash}p{6.5cm}rrrp{2.5cm}rrrrr}
\rowcolor{white}\caption{Works from bibtex (Total 7)}\\ \toprule
\rowcolor{white}Key & Authors & Title & LC & Cite & Year & \shortstack{Conference\\/Journal} & Pages & \shortstack{Nr\\Cites} & \shortstack{Nr\\Refs} & b & c \\ \midrule\endhead
\bottomrule
\endfoot
Mehdizadeh-Somarin23 \href{https://doi.org/10.1007/978-3-031-43670-3\_33}{Mehdizadeh-Somarin23} & \hyperref[auth:a433]{Z. Mehdizadeh{-}Somarin}, \hyperref[auth:a434]{R. Tavakkoli{-}Moghaddam}, \hyperref[auth:a435]{M. Rohaninejad}, \hyperref[auth:a116]{Z. Hanz{\'{a}}lek}, \hyperref[auth:a436]{Behdin Vahedi Nouri} & A Constraint Programming Model for a Reconfigurable Job Shop Scheduling Problem with Machine Availability & \href{works/Mehdizadeh-Somarin23.pdf}{Yes} & \cite{Mehdizadeh-Somarin23} & 2023 & APMS 2023 & 14 & 0 & 0 & \ref{b:Mehdizadeh-Somarin23} & \ref{c:Mehdizadeh-Somarin23}\\
abs-2305-19888 \href{https://doi.org/10.48550/arXiv.2305.19888}{abs-2305-19888} & \hyperref[auth:a437]{V. Heinz}, \hyperref[auth:a438]{A. Nov{\'{a}}k}, \hyperref[auth:a313]{M. Vlk}, \hyperref[auth:a116]{Z. Hanz{\'{a}}lek} & Constraint Programming and Constructive Heuristics for Parallel Machine Scheduling with Sequence-Dependent Setups and Common Servers & \href{works/abs-2305-19888.pdf}{Yes} & \cite{abs-2305-19888} & 2023 & CoRR & 42 & 0 & 0 & \ref{b:abs-2305-19888} & \ref{c:abs-2305-19888}\\
HeinzNVH22 \href{https://doi.org/10.1016/j.cie.2022.108586}{HeinzNVH22} & \hyperref[auth:a437]{V. Heinz}, \hyperref[auth:a438]{A. Nov{\'{a}}k}, \hyperref[auth:a313]{M. Vlk}, \hyperref[auth:a116]{Z. Hanz{\'{a}}lek} & Constraint Programming and constructive heuristics for parallel machine scheduling with sequence-dependent setups and common servers & \href{works/HeinzNVH22.pdf}{Yes} & \cite{HeinzNVH22} & 2022 & Comput. Ind. Eng. & 16 & 5 & 25 & \ref{b:HeinzNVH22} & \ref{c:HeinzNVH22}\\
VlkHT21 \href{https://doi.org/10.1016/j.cie.2021.107317}{VlkHT21} & \hyperref[auth:a313]{M. Vlk}, \hyperref[auth:a116]{Z. Hanz{\'{a}}lek}, \hyperref[auth:a480]{S. Tang} & Constraint programming approaches to joint routing and scheduling in time-sensitive networks & \href{works/VlkHT21.pdf}{Yes} & \cite{VlkHT21} & 2021 & Comput. Ind. Eng. & 14 & 7 & 22 & \ref{b:VlkHT21} & \ref{c:VlkHT21}\\
BenediktMH20 \href{https://doi.org/10.1007/s10601-020-09317-y}{BenediktMH20} & \hyperref[auth:a114]{O. Benedikt}, \hyperref[auth:a115]{I. M{\'{o}}dos}, \hyperref[auth:a116]{Z. Hanz{\'{a}}lek} & Power of pre-processing: production scheduling with variable energy pricing and power-saving states & \href{works/BenediktMH20.pdf}{Yes} & \cite{BenediktMH20} & 2020 & Constraints An Int. J. & 19 & 1 & 18 & \ref{b:BenediktMH20} & \ref{c:BenediktMH20}\\
BenediktSMVH18 \href{https://doi.org/10.1007/978-3-319-93031-2\_6}{BenediktSMVH18} & \hyperref[auth:a114]{O. Benedikt}, \hyperref[auth:a312]{P. Sucha}, \hyperref[auth:a115]{I. M{\'{o}}dos}, \hyperref[auth:a313]{M. Vlk}, \hyperref[auth:a116]{Z. Hanz{\'{a}}lek} & Energy-Aware Production Scheduling with Power-Saving Modes & \href{works/BenediktSMVH18.pdf}{Yes} & \cite{BenediktSMVH18} & 2018 & CPAIOR 2018 & 10 & 2 & 12 & \ref{b:BenediktSMVH18} & \ref{c:BenediktSMVH18}\\
KelbelH11 \href{https://doi.org/10.1007/s10845-009-0318-2}{KelbelH11} & \hyperref[auth:a627]{J. Kelbel}, \hyperref[auth:a116]{Z. Hanz{\'{a}}lek} & Solving production scheduling with earliness/tardiness penalties by constraint programming & \href{works/KelbelH11.pdf}{Yes} & \cite{KelbelH11} & 2011 & J. Intell. Manuf. & 10 & 12 & 14 & \ref{b:KelbelH11} & \ref{c:KelbelH11}\\
\end{longtable}
}

\subsection{Works by Andr{\'{a}}s Kov{\'{a}}cs}
\label{sec:a146}
{\scriptsize
\begin{longtable}{>{\raggedright\arraybackslash}p{3cm}>{\raggedright\arraybackslash}p{6cm}>{\raggedright\arraybackslash}p{6.5cm}rrrp{2.5cm}rrrrr}
\rowcolor{white}\caption{Works from bibtex (Total 7)}\\ \toprule
\rowcolor{white}Key & Authors & Title & LC & Cite & Year & \shortstack{Conference\\/Journal} & Pages & \shortstack{Nr\\Cites} & \shortstack{Nr\\Refs} & b & c \\ \midrule\endhead
\bottomrule
\endfoot
KovacsB11 \href{https://doi.org/10.1007/s10601-009-9088-x}{KovacsB11} & \hyperref[auth:a146]{A. Kov{\'{a}}cs}, \hyperref[auth:a89]{J. Christopher Beck} & A global constraint for total weighted completion time for unary resources & \href{works/KovacsB11.pdf}{Yes} & \cite{KovacsB11} & 2011 & Constraints An Int. J. & 24 & 4 & 26 & \ref{b:KovacsB11} & \ref{c:KovacsB11}\\
KovacsK11 \href{https://doi.org/10.1007/s10601-010-9102-3}{KovacsK11} & \hyperref[auth:a146]{A. Kov{\'{a}}cs}, \hyperref[auth:a156]{T. Kis} & Constraint programming approach to a bilevel scheduling problem & \href{works/KovacsK11.pdf}{Yes} & \cite{KovacsK11} & 2011 & Constraints An Int. J. & 24 & 3 & 24 & \ref{b:KovacsK11} & \ref{c:KovacsK11}\\
KovacsB08 \href{https://doi.org/10.1016/j.engappai.2008.03.004}{KovacsB08} & \hyperref[auth:a146]{A. Kov{\'{a}}cs}, \hyperref[auth:a89]{J. Christopher Beck} & A global constraint for total weighted completion time for cumulative resources & \href{works/KovacsB08.pdf}{Yes} & \cite{KovacsB08} & 2008 & Eng. Appl. Artif. Intell. & 7 & 5 & 14 & \ref{b:KovacsB08} & \ref{c:KovacsB08}\\
KovacsB07 \href{https://doi.org/10.1007/978-3-540-72397-4\_9}{KovacsB07} & \hyperref[auth:a146]{A. Kov{\'{a}}cs}, \hyperref[auth:a89]{J. Christopher Beck} & A Global Constraint for Total Weighted Completion Time & \href{works/KovacsB07.pdf}{Yes} & \cite{KovacsB07} & 2007 & CPAIOR 2007 & 15 & 2 & 12 & \ref{b:KovacsB07} & \ref{c:KovacsB07}\\
KovacsV06 \href{https://doi.org/10.1007/11757375\_13}{KovacsV06} & \hyperref[auth:a146]{A. Kov{\'{a}}cs}, \hyperref[auth:a280]{J. V{\'{a}}ncza} & Progressive Solutions: {A} Simple but Efficient Dominance Rule for Practical {RCPSP} & \href{works/KovacsV06.pdf}{Yes} & \cite{KovacsV06} & 2006 & CPAIOR 2006 & 13 & 2 & 7 & \ref{b:KovacsV06} & \ref{c:KovacsV06}\\
KovacsEKV05 \href{https://doi.org/10.1007/11564751\_118}{KovacsEKV05} & \hyperref[auth:a146]{A. Kov{\'{a}}cs}, \hyperref[auth:a279]{P. Egri}, \hyperref[auth:a156]{T. Kis}, \hyperref[auth:a280]{J. V{\'{a}}ncza} & Proterv-II: An Integrated Production Planning and Scheduling System & \href{works/KovacsEKV05.pdf}{Yes} & \cite{KovacsEKV05} & 2005 & CP 2005 & 1 & 2 & 3 & \ref{b:KovacsEKV05} & \ref{c:KovacsEKV05}\\
KovacsV04 \href{https://doi.org/10.1007/978-3-540-30201-8\_26}{KovacsV04} & \hyperref[auth:a146]{A. Kov{\'{a}}cs}, \hyperref[auth:a280]{J. V{\'{a}}ncza} & Completable Partial Solutions in Constraint Programming and Constraint-Based Scheduling & \href{works/KovacsV04.pdf}{Yes} & \cite{KovacsV04} & 2004 & CP 2004 & 15 & 3 & 12 & \ref{b:KovacsV04} & \ref{c:KovacsV04}\\
\end{longtable}
}

\subsection{Works by Barry O'Sullivan}
\label{sec:a16}
{\scriptsize
\begin{longtable}{>{\raggedright\arraybackslash}p{3cm}>{\raggedright\arraybackslash}p{6cm}>{\raggedright\arraybackslash}p{6.5cm}rrrp{2.5cm}rrrrr}
\rowcolor{white}\caption{Works from bibtex (Total 7)}\\ \toprule
\rowcolor{white}Key & Authors & Title & LC & Cite & Year & \shortstack{Conference\\/Journal} & Pages & \shortstack{Nr\\Cites} & \shortstack{Nr\\Refs} & b & c \\ \midrule\endhead
\bottomrule
\endfoot
ArmstrongGOS22 \href{https://doi.org/10.1007/978-3-031-08011-1\_1}{ArmstrongGOS22} & \hyperref[auth:a14]{E. Armstrong}, \hyperref[auth:a15]{M. Garraffa}, \hyperref[auth:a16]{B. O'Sullivan}, \hyperref[auth:a17]{H. Simonis} & A Two-Phase Hybrid Approach for the Hybrid Flexible Flowshop with Transportation Times & \href{works/ArmstrongGOS22.pdf}{Yes} & \cite{ArmstrongGOS22} & 2022 & CPAIOR 2022 & 13 & 0 & 14 & \ref{b:ArmstrongGOS22} & \ref{c:ArmstrongGOS22}\\
ArmstrongGOS21 \href{https://doi.org/10.4230/LIPIcs.CP.2021.16}{ArmstrongGOS21} & \hyperref[auth:a14]{E. Armstrong}, \hyperref[auth:a15]{M. Garraffa}, \hyperref[auth:a16]{B. O'Sullivan}, \hyperref[auth:a17]{H. Simonis} & The Hybrid Flexible Flowshop with Transportation Times & \href{works/ArmstrongGOS21.pdf}{Yes} & \cite{ArmstrongGOS21} & 2021 & CP 2021 & 18 & 1 & 0 & \ref{b:ArmstrongGOS21} & \ref{c:ArmstrongGOS21}\\
AntunesABDEGGOL20 \href{https://doi.org/10.1142/S0218213020600076}{AntunesABDEGGOL20} & \hyperref[auth:a893]{M. Antunes}, \hyperref[auth:a894]{V. Armant}, \hyperref[auth:a222]{Kenneth N. Brown}, \hyperref[auth:a895]{Daniel A. Desmond}, \hyperref[auth:a896]{G. Escamocher}, \hyperref[auth:a897]{A. George}, \hyperref[auth:a182]{D. Grimes}, \hyperref[auth:a898]{M. O'Keeffe}, \hyperref[auth:a899]{Y. Lin}, \hyperref[auth:a16]{B. O'Sullivan}, \hyperref[auth:a900]{C. Ozturk}, \hyperref[auth:a901]{L. Quesada}, \hyperref[auth:a129]{M. Siala}, \hyperref[auth:a17]{H. Simonis}, \hyperref[auth:a837]{N. Wilson} & Assigning and Scheduling Service Visits in a Mixed Urban/Rural Setting & No & \cite{AntunesABDEGGOL20} & 2020 & Int. J. Artif. Intell. Tools & 31 & 0 & 16 & No & \ref{c:AntunesABDEGGOL20}\\
AntunesABDEGGOL18 \href{https://doi.org/10.1109/ICTAI.2018.00027}{AntunesABDEGGOL18} & \hyperref[auth:a893]{M. Antunes}, \hyperref[auth:a894]{V. Armant}, \hyperref[auth:a222]{Kenneth N. Brown}, \hyperref[auth:a895]{Daniel A. Desmond}, \hyperref[auth:a896]{G. Escamocher}, \hyperref[auth:a897]{A. George}, \hyperref[auth:a182]{D. Grimes}, \hyperref[auth:a898]{M. O'Keeffe}, \hyperref[auth:a899]{Y. Lin}, \hyperref[auth:a16]{B. O'Sullivan}, \hyperref[auth:a900]{C. Ozturk}, \hyperref[auth:a901]{L. Quesada}, \hyperref[auth:a129]{M. Siala}, \hyperref[auth:a17]{H. Simonis}, \hyperref[auth:a837]{N. Wilson} & Assigning and Scheduling Service Visits in a Mixed Urban/Rural Setting & No & \cite{AntunesABDEGGOL18} & 2018 & ICTAI 2018 & 8 & 1 & 24 & No & \ref{c:AntunesABDEGGOL18}\\
HurleyOS16 \href{https://doi.org/10.1007/978-3-319-50137-6\_15}{HurleyOS16} & \hyperref[auth:a902]{B. Hurley}, \hyperref[auth:a16]{B. O'Sullivan}, \hyperref[auth:a17]{H. Simonis} & {ICON} Loop Energy Show Case & \href{works/HurleyOS16.pdf}{Yes} & \cite{HurleyOS16} & 2016 & Data Mining and Constraint Programming - Foundations of a Cross-Disciplinary Approach & 14 & 0 & 16 & \ref{b:HurleyOS16} & \ref{c:HurleyOS16}\\
GrimesIOS14 \href{https://doi.org/10.1016/j.suscom.2014.08.009}{GrimesIOS14} & \hyperref[auth:a182]{D. Grimes}, \hyperref[auth:a183]{G. Ifrim}, \hyperref[auth:a16]{B. O'Sullivan}, \hyperref[auth:a17]{H. Simonis} & Analyzing the impact of electricity price forecasting on energy cost-aware scheduling & \href{works/GrimesIOS14.pdf}{Yes} & \cite{GrimesIOS14} & 2014 & Sustain. Comput. Informatics Syst. & 16 & 6 & 7 & \ref{b:GrimesIOS14} & \ref{c:GrimesIOS14}\\
IfrimOS12 \href{https://doi.org/10.1007/978-3-642-33558-7\_68}{IfrimOS12} & \hyperref[auth:a183]{G. Ifrim}, \hyperref[auth:a16]{B. O'Sullivan}, \hyperref[auth:a17]{H. Simonis} & Properties of Energy-Price Forecasts for Scheduling & \href{works/IfrimOS12.pdf}{Yes} & \cite{IfrimOS12} & 2012 & CP 2012 & 16 & 6 & 20 & \ref{b:IfrimOS12} & \ref{c:IfrimOS12}\\
\end{longtable}
}

\subsection{Works by Gabriela P. Henning}
\label{sec:a596}
{\scriptsize
\begin{longtable}{>{\raggedright\arraybackslash}p{3cm}>{\raggedright\arraybackslash}p{6cm}>{\raggedright\arraybackslash}p{6.5cm}rrrp{2.5cm}rrrrr}
\rowcolor{white}\caption{Works from bibtex (Total 7)}\\ \toprule
\rowcolor{white}Key & Authors & Title & LC & Cite & Year & \shortstack{Conference\\/Journal} & Pages & \shortstack{Nr\\Cites} & \shortstack{Nr\\Refs} & b & c \\ \midrule\endhead
\bottomrule
\endfoot
NovaraNH16 \href{https://doi.org/10.1016/j.compchemeng.2016.04.030}{NovaraNH16} & \hyperref[auth:a595]{Franco M. Novara}, \hyperref[auth:a529]{Juan M. Novas}, \hyperref[auth:a596]{Gabriela P. Henning} & A novel constraint programming model for large-scale scheduling problems in multiproduct multistage batch plants: Limited resources and campaign-based operation & \href{works/NovaraNH16.pdf}{Yes} & \cite{NovaraNH16} & 2016 & Comput. Chem. Eng. & 17 & 18 & 31 & \ref{b:NovaraNH16} & \ref{c:NovaraNH16}\\
NovasH14 \href{https://doi.org/10.1016/j.eswa.2013.09.026}{NovasH14} & \hyperref[auth:a529]{Juan M. Novas}, \hyperref[auth:a596]{Gabriela P. Henning} & Integrated scheduling of resource-constrained flexible manufacturing systems using constraint programming & \href{works/NovasH14.pdf}{Yes} & \cite{NovasH14} & 2014 & Expert Syst. Appl. & 14 & 35 & 26 & \ref{b:NovasH14} & \ref{c:NovasH14}\\
NovasH12 \href{https://doi.org/10.1016/j.compchemeng.2012.01.005}{NovasH12} & \hyperref[auth:a529]{Juan M. Novas}, \hyperref[auth:a596]{Gabriela P. Henning} & A comprehensive constraint programming approach for the rolling horizon-based scheduling of automated wet-etch stations & \href{works/NovasH12.pdf}{Yes} & \cite{NovasH12} & 2012 & Comput. Chem. Eng. & 17 & 17 & 15 & \ref{b:NovasH12} & \ref{c:NovasH12}\\
NovasH10 \href{https://doi.org/10.1016/j.compchemeng.2010.07.011}{NovasH10} & \hyperref[auth:a529]{Juan M. Novas}, \hyperref[auth:a596]{Gabriela P. Henning} & Reactive scheduling framework based on domain knowledge and constraint programming & \href{works/NovasH10.pdf}{Yes} & \cite{NovasH10} & 2010 & Comput. Chem. Eng. & 20 & 48 & 19 & \ref{b:NovasH10} & \ref{c:NovasH10}\\
ZeballosQH10 \href{https://doi.org/10.1016/j.engappai.2009.07.002}{ZeballosQH10} & \hyperref[auth:a630]{L. Zeballos}, \hyperref[auth:a631]{O. Quiroga}, \hyperref[auth:a596]{Gabriela P. Henning} & A constraint programming model for the scheduling of flexible manufacturing systems with machine and tool limitations & \href{works/ZeballosQH10.pdf}{Yes} & \cite{ZeballosQH10} & 2010 & Eng. Appl. Artif. Intell. & 20 & 33 & 28 & \ref{b:ZeballosQH10} & \ref{c:ZeballosQH10}\\
QuirogaZH05 \href{https://doi.org/10.1109/ROBOT.2005.1570686}{QuirogaZH05} & \hyperref[auth:a631]{O. Quiroga}, \hyperref[auth:a630]{L. Zeballos}, \hyperref[auth:a596]{Gabriela P. Henning} & A Constraint Programming Approach to Tool Allocation and Resource Scheduling in {FMS} & \href{works/QuirogaZH05.pdf}{Yes} & \cite{QuirogaZH05} & 2005 & ICRA 2005 & 6 & 2 & 7 & \ref{b:QuirogaZH05} & \ref{c:QuirogaZH05}\\
ZeballosH05 \href{http://journal.iberamia.org/index.php/ia/article/view/452/article\%20\%281\%29.pdf}{ZeballosH05} & \hyperref[auth:a630]{L. Zeballos}, \hyperref[auth:a596]{Gabriela P. Henning} & A Constraint Programming Approach to {FMS} Scheduling. Consideration of Storage and Transportation Resources & \href{works/ZeballosH05.pdf}{Yes} & \cite{ZeballosH05} & 2005 & Inteligencia Artif. & 10 & 0 & 0 & \ref{b:ZeballosH05} & \ref{c:ZeballosH05}\\
\end{longtable}
}

\subsection{Works by Stefan Heinz}
\label{sec:a133}
{\scriptsize
\begin{longtable}{>{\raggedright\arraybackslash}p{3cm}>{\raggedright\arraybackslash}p{6cm}>{\raggedright\arraybackslash}p{6.5cm}rrrp{2.5cm}rrrrr}
\rowcolor{white}\caption{Works from bibtex (Total 6)}\\ \toprule
\rowcolor{white}Key & Authors & Title & LC & Cite & Year & \shortstack{Conference\\/Journal} & Pages & \shortstack{Nr\\Cites} & \shortstack{Nr\\Refs} & b & c \\ \midrule\endhead
\bottomrule
\endfoot
HeinzKB13 \href{https://doi.org/10.1007/978-3-642-38171-3\_2}{HeinzKB13} & \hyperref[auth:a133]{S. Heinz}, \hyperref[auth:a336]{W. Ku}, \hyperref[auth:a89]{J. Christopher Beck} & Recent Improvements Using Constraint Integer Programming for Resource Allocation and Scheduling & \href{works/HeinzKB13.pdf}{Yes} & \cite{HeinzKB13} & 2013 & CPAIOR 2013 & 16 & 9 & 15 & \ref{b:HeinzKB13} & \ref{c:HeinzKB13}\\
HeinzSB13 \href{https://doi.org/10.1007/s10601-012-9136-9}{HeinzSB13} & \hyperref[auth:a133]{S. Heinz}, \hyperref[auth:a134]{J. Schulz}, \hyperref[auth:a89]{J. Christopher Beck} & Using dual presolving reductions to reformulate cumulative constraints & \href{works/HeinzSB13.pdf}{Yes} & \cite{HeinzSB13} & 2013 & Constraints An Int. J. & 36 & 7 & 31 & \ref{b:HeinzSB13} & \ref{c:HeinzSB13}\\
HeinzB12 \href{https://doi.org/10.1007/978-3-642-29828-8\_14}{HeinzB12} & \hyperref[auth:a133]{S. Heinz}, \hyperref[auth:a89]{J. Christopher Beck} & Reconsidering Mixed Integer Programming and MIP-Based Hybrids for Scheduling & \href{works/HeinzB12.pdf}{Yes} & \cite{HeinzB12} & 2012 & CPAIOR 2012 & 17 & 8 & 21 & \ref{b:HeinzB12} & \ref{c:HeinzB12}\\
HeinzSSW12 \href{https://doi.org/10.1007/s10601-011-9113-8}{HeinzSSW12} & \hyperref[auth:a133]{S. Heinz}, \hyperref[auth:a139]{T. Schlechte}, \hyperref[auth:a140]{R. Stephan}, \hyperref[auth:a141]{M. Winkler} & Solving steel mill slab design problems & \href{works/HeinzSSW12.pdf}{Yes} & \cite{HeinzSSW12} & 2012 & Constraints An Int. J. & 12 & 10 & 9 & \ref{b:HeinzSSW12} & \ref{c:HeinzSSW12}\\
HeinzS11 \href{https://doi.org/10.1007/978-3-642-20662-7\_34}{HeinzS11} & \hyperref[auth:a133]{S. Heinz}, \hyperref[auth:a134]{J. Schulz} & Explanations for the Cumulative Constraint: An Experimental Study & \href{works/HeinzS11.pdf}{Yes} & \cite{HeinzS11} & 2011 & SEA 2011 & 10 & 5 & 12 & \ref{b:HeinzS11} & \ref{c:HeinzS11}\\
BertholdHLMS10 \href{https://doi.org/10.1007/978-3-642-13520-0\_34}{BertholdHLMS10} & \hyperref[auth:a355]{T. Berthold}, \hyperref[auth:a133]{S. Heinz}, \hyperref[auth:a356]{Marco E. L{\"{u}}bbecke}, \hyperref[auth:a357]{Rolf H. M{\"{o}}hring}, \hyperref[auth:a134]{J. Schulz} & A Constraint Integer Programming Approach for Resource-Constrained Project Scheduling & \href{works/BertholdHLMS10.pdf}{Yes} & \cite{BertholdHLMS10} & 2010 & CPAIOR 2010 & 5 & 28 & 10 & \ref{b:BertholdHLMS10} & \ref{c:BertholdHLMS10}\\
\end{longtable}
}

\subsection{Works by Wim Nuijten}
\label{sec:a666}
{\scriptsize
\begin{longtable}{>{\raggedright\arraybackslash}p{3cm}>{\raggedright\arraybackslash}p{6cm}>{\raggedright\arraybackslash}p{6.5cm}rrrp{2.5cm}rrrrr}
\rowcolor{white}\caption{Works from bibtex (Total 6)}\\ \toprule
\rowcolor{white}Key & Authors & Title & LC & Cite & Year & \shortstack{Conference\\/Journal} & Pages & \shortstack{Nr\\Cites} & \shortstack{Nr\\Refs} & b & c \\ \midrule\endhead
\bottomrule
\endfoot
BaptisteLPN06 \href{https://doi.org/10.1016/S1574-6526(06)80026-X}{BaptisteLPN06} & \hyperref[auth:a163]{P. Baptiste}, \hyperref[auth:a118]{P. Laborie}, \hyperref[auth:a164]{Claude Le Pape}, \hyperref[auth:a666]{W. Nuijten} & Constraint-Based Scheduling and Planning & No & \cite{BaptisteLPN06} & 2006 & Handbook of Constraint Programming & 39 & 30 & 25 & No & \ref{c:BaptisteLPN06}\\
GodardLN05 \href{http://www.aaai.org/Library/ICAPS/2005/icaps05-009.php}{GodardLN05} & \hyperref[auth:a782]{D. Godard}, \hyperref[auth:a118]{P. Laborie}, \hyperref[auth:a666]{W. Nuijten} & Randomized Large Neighborhood Search for Cumulative Scheduling & \href{works/GodardLN05.pdf}{Yes} & \cite{GodardLN05} & 2005 & ICAPS 2005 & 9 & 0 & 0 & \ref{b:GodardLN05} & \ref{c:GodardLN05}\\
BaptistePN01 \href{http://dx.doi.org/10.1007/978-1-4615-1479-4}{BaptistePN01} & \hyperref[auth:a163]{P. Baptiste}, \hyperref[auth:a164]{Claude Le Pape}, \hyperref[auth:a666]{W. Nuijten} & Constraint-Based Scheduling & No & \cite{BaptistePN01} & 2001 & Book & null & 296 & 0 & No & \ref{c:BaptistePN01}\\
FocacciLN00 \href{http://www.aaai.org/Library/AIPS/2000/aips00-010.php}{FocacciLN00} & \hyperref[auth:a784]{F. Focacci}, \hyperref[auth:a118]{P. Laborie}, \hyperref[auth:a666]{W. Nuijten} & Solving Scheduling Problems with Setup Times and Alternative Resources & \href{works/FocacciLN00.pdf}{Yes} & \cite{FocacciLN00} & 2000 & AIPS 2000 & 10 & 0 & 0 & \ref{b:FocacciLN00} & \ref{c:FocacciLN00}\\
SourdN00 \href{https://doi.org/10.1287/ijoc.12.4.341.11881}{SourdN00} & \hyperref[auth:a783]{F. Sourd}, \hyperref[auth:a666]{W. Nuijten} & Multiple-Machine Lower Bounds for Shop-Scheduling Problems & \href{works/SourdN00.pdf}{Yes} & \cite{SourdN00} & 2000 & {INFORMS} J. Comput. & 12 & 7 & 14 & \ref{b:SourdN00} & \ref{c:SourdN00}\\
NuijtenP98 \href{https://doi.org/10.1023/A:1009687210594}{NuijtenP98} & \hyperref[auth:a666]{W. Nuijten}, \hyperref[auth:a164]{Claude Le Pape} & Constraint-Based Job Shop Scheduling with {\textbackslash}sc Ilog Scheduler & \href{works/NuijtenP98.pdf}{Yes} & \cite{NuijtenP98} & 1998 & J. Heuristics & 16 & 42 & 0 & \ref{b:NuijtenP98} & \ref{c:NuijtenP98}\\
\end{longtable}
}

\subsection{Works by Emmanuel Poder}
\label{sec:a362}
{\scriptsize
\begin{longtable}{>{\raggedright\arraybackslash}p{3cm}>{\raggedright\arraybackslash}p{6cm}>{\raggedright\arraybackslash}p{6.5cm}rrrp{2.5cm}rrrrr}
\rowcolor{white}\caption{Works from bibtex (Total 6)}\\ \toprule
\rowcolor{white}Key & Authors & Title & LC & Cite & Year & \shortstack{Conference\\/Journal} & Pages & \shortstack{Nr\\Cites} & \shortstack{Nr\\Refs} & b & c \\ \midrule\endhead
\bottomrule
\endfoot
BeldiceanuCDP11 \href{https://doi.org/10.1007/s10479-010-0731-0}{BeldiceanuCDP11} & \hyperref[auth:a128]{N. Beldiceanu}, \hyperref[auth:a91]{M. Carlsson}, \hyperref[auth:a245]{S. Demassey}, \hyperref[auth:a362]{E. Poder} & New filtering for the \emph{cumulative} constraint in the context of non-overlapping rectangles & \href{works/BeldiceanuCDP11.pdf}{Yes} & \cite{BeldiceanuCDP11} & 2011 & Ann. Oper. Res. & 24 & 8 & 8 & \ref{b:BeldiceanuCDP11} & \ref{c:BeldiceanuCDP11}\\
abs-0907-0939 \href{http://arxiv.org/abs/0907.0939}{abs-0907-0939} & \hyperref[auth:a226]{T. Petit}, \hyperref[auth:a362]{E. Poder} & The Soft Cumulative Constraint & \href{works/abs-0907-0939.pdf}{Yes} & \cite{abs-0907-0939} & 2009 & CoRR & 12 & 0 & 0 & \ref{b:abs-0907-0939} & \ref{c:abs-0907-0939}\\
BeldiceanuCP08 \href{https://doi.org/10.1007/978-3-540-68155-7\_5}{BeldiceanuCP08} & \hyperref[auth:a128]{N. Beldiceanu}, \hyperref[auth:a91]{M. Carlsson}, \hyperref[auth:a362]{E. Poder} & New Filtering for the cumulative Constraint in the Context of Non-Overlapping Rectangles & \href{works/BeldiceanuCP08.pdf}{Yes} & \cite{BeldiceanuCP08} & 2008 & CPAIOR 2008 & 15 & 8 & 9 & \ref{b:BeldiceanuCP08} & \ref{c:BeldiceanuCP08}\\
PoderB08 \href{http://www.aaai.org/Library/ICAPS/2008/icaps08-033.php}{PoderB08} & \hyperref[auth:a362]{E. Poder}, \hyperref[auth:a128]{N. Beldiceanu} & Filtering for a Continuous Multi-Resources cumulative Constraint with Resource Consumption and Production & \href{works/PoderB08.pdf}{Yes} & \cite{PoderB08} & 2008 & ICAPS 2008 & 8 & 0 & 0 & \ref{b:PoderB08} & \ref{c:PoderB08}\\
BeldiceanuP07 \href{https://doi.org/10.1007/978-3-540-72397-4\_16}{BeldiceanuP07} & \hyperref[auth:a128]{N. Beldiceanu}, \hyperref[auth:a362]{E. Poder} & A Continuous Multi-resources \emph{cumulative} Constraint with Positive-Negative Resource Consumption-Production & \href{works/BeldiceanuP07.pdf}{Yes} & \cite{BeldiceanuP07} & 2007 & CPAIOR 2007 & 15 & 4 & 7 & \ref{b:BeldiceanuP07} & \ref{c:BeldiceanuP07}\\
PoderBS04 \href{https://doi.org/10.1016/S0377-2217(02)00756-7}{PoderBS04} & \hyperref[auth:a362]{E. Poder}, \hyperref[auth:a128]{N. Beldiceanu}, \hyperref[auth:a722]{E. Sanlaville} & Computing a lower approximation of the compulsory part of a task with varying duration and varying resource consumption & \href{works/PoderBS04.pdf}{Yes} & \cite{PoderBS04} & 2004 & Eur. J. Oper. Res. & 16 & 7 & 8 & \ref{b:PoderBS04} & \ref{c:PoderBS04}\\
\end{longtable}
}

\subsection{Works by Louis{-}Martin Rousseau}
\label{sec:a331}
{\scriptsize
\begin{longtable}{>{\raggedright\arraybackslash}p{3cm}>{\raggedright\arraybackslash}p{6cm}>{\raggedright\arraybackslash}p{6.5cm}rrrp{2.5cm}rrrrr}
\rowcolor{white}\caption{Works from bibtex (Total 6)}\\ \toprule
\rowcolor{white}Key & Authors & Title & LC & Cite & Year & \shortstack{Conference\\/Journal} & Pages & \shortstack{Nr\\Cites} & \shortstack{Nr\\Refs} & b & c \\ \midrule\endhead
\bottomrule
\endfoot
CappartTSR18 \href{https://doi.org/10.1007/978-3-319-98334-9\_32}{CappartTSR18} & \hyperref[auth:a42]{Q. Cappart}, \hyperref[auth:a849]{C. Thomas}, \hyperref[auth:a147]{P. Schaus}, \hyperref[auth:a331]{L. Rousseau} & A Constraint Programming Approach for Solving Patient Transportation Problems & \href{works/CappartTSR18.pdf}{Yes} & \cite{CappartTSR18} & 2018 & CP 2018 & 17 & 6 & 31 & \ref{b:CappartTSR18} & \ref{c:CappartTSR18}\\
DoulabiRP16 \href{https://doi.org/10.1287/ijoc.2015.0686}{DoulabiRP16} & \hyperref[auth:a335]{Seyed Hossein Hashemi Doulabi}, \hyperref[auth:a331]{L. Rousseau}, \hyperref[auth:a8]{G. Pesant} & A Constraint-Programming-Based Branch-and-Price-and-Cut Approach for Operating Room Planning and Scheduling & \href{works/DoulabiRP16.pdf}{Yes} & \cite{DoulabiRP16} & 2016 & {INFORMS} J. Comput. & 17 & 56 & 28 & \ref{b:DoulabiRP16} & \ref{c:DoulabiRP16}\\
PesantRR15 \href{https://doi.org/10.1007/978-3-319-18008-3\_21}{PesantRR15} & \hyperref[auth:a8]{G. Pesant}, \hyperref[auth:a330]{G. Rix}, \hyperref[auth:a331]{L. Rousseau} & A Comparative Study of {MIP} and {CP} Formulations for the {B2B} Scheduling Optimization Problem & \href{works/PesantRR15.pdf}{Yes} & \cite{PesantRR15} & 2015 & CPAIOR 2015 & 16 & 1 & 7 & \ref{b:PesantRR15} & \ref{c:PesantRR15}\\
DoulabiRP14 \href{https://doi.org/10.1007/978-3-319-07046-9\_32}{DoulabiRP14} & \hyperref[auth:a335]{Seyed Hossein Hashemi Doulabi}, \hyperref[auth:a331]{L. Rousseau}, \hyperref[auth:a8]{G. Pesant} & A Constraint Programming-Based Column Generation Approach for Operating Room Planning and Scheduling & \href{works/DoulabiRP14.pdf}{Yes} & \cite{DoulabiRP14} & 2014 & CPAIOR 2014 & 9 & 3 & 10 & \ref{b:DoulabiRP14} & \ref{c:DoulabiRP14}\\
ChapadosJR11 \href{https://doi.org/10.1007/978-3-642-21311-3\_7}{ChapadosJR11} & \hyperref[auth:a349]{N. Chapados}, \hyperref[auth:a350]{M. Joliveau}, \hyperref[auth:a331]{L. Rousseau} & Retail Store Workforce Scheduling by Expected Operating Income Maximization & \href{works/ChapadosJR11.pdf}{Yes} & \cite{ChapadosJR11} & 2011 & CPAIOR 2011 & 6 & 5 & 12 & \ref{b:ChapadosJR11} & \ref{c:ChapadosJR11}\\
HachemiGR11 \href{https://doi.org/10.1007/s10479-010-0698-x}{HachemiGR11} & \hyperref[auth:a623]{Nizar El Hachemi}, \hyperref[auth:a624]{M. Gendreau}, \hyperref[auth:a331]{L. Rousseau} & A hybrid constraint programming approach to the log-truck scheduling problem & \href{works/HachemiGR11.pdf}{Yes} & \cite{HachemiGR11} & 2011 & Ann. Oper. Res. & 16 & 32 & 19 & \ref{b:HachemiGR11} & \ref{c:HachemiGR11}\\
\end{longtable}
}

\subsection{Works by Cyrille Dejemeppe}
\label{sec:a207}
{\scriptsize
\begin{longtable}{>{\raggedright\arraybackslash}p{3cm}>{\raggedright\arraybackslash}p{6cm}>{\raggedright\arraybackslash}p{6.5cm}rrrp{2.5cm}rrrrr}
\rowcolor{white}\caption{Works from bibtex (Total 5)}\\ \toprule
\rowcolor{white}Key & Authors & Title & LC & Cite & Year & \shortstack{Conference\\/Journal} & Pages & \shortstack{Nr\\Cites} & \shortstack{Nr\\Refs} & b & c \\ \midrule\endhead
\bottomrule
\endfoot
CauwelaertDS20 \href{http://dx.doi.org/10.1007/s10951-019-00632-8}{CauwelaertDS20} & \hyperref[auth:a850]{Sasha Van Cauwelaert}, \hyperref[auth:a207]{C. Dejemeppe}, \hyperref[auth:a147]{P. Schaus} & An Efficient Filtering Algorithm for the Unary Resource Constraint with Transition Times and Optional Activities & \href{works/CauwelaertDS20.pdf}{Yes} & \cite{CauwelaertDS20} & 2020 & Journal of Scheduling & 19 & 2 & 21 & \ref{b:CauwelaertDS20} & \ref{c:CauwelaertDS20}\\
CauwelaertDMS16 \href{https://doi.org/10.1007/978-3-319-44953-1\_33}{CauwelaertDMS16} & \hyperref[auth:a206]{Sascha Van Cauwelaert}, \hyperref[auth:a207]{C. Dejemeppe}, \hyperref[auth:a149]{J. Monette}, \hyperref[auth:a147]{P. Schaus} & Efficient Filtering for the Unary Resource with Family-Based Transition Times & \href{works/CauwelaertDMS16.pdf}{Yes} & \cite{CauwelaertDMS16} & 2016 & CP 2016 & 16 & 1 & 12 & \ref{b:CauwelaertDMS16} & \ref{c:CauwelaertDMS16}\\
Dejemeppe16 \href{https://hdl.handle.net/2078.1/178078}{Dejemeppe16} & \hyperref[auth:a207]{C. Dejemeppe} & Constraint programming algorithms and models for scheduling applications & \href{works/Dejemeppe16.pdf}{Yes} & \cite{Dejemeppe16} & 2016 & Catholic University of Louvain, Louvain-la-Neuve, Belgium & 274 & 0 & 0 & \ref{b:Dejemeppe16} & \ref{c:Dejemeppe16}\\
DejemeppeCS15 \href{https://doi.org/10.1007/978-3-319-23219-5\_7}{DejemeppeCS15} & \hyperref[auth:a207]{C. Dejemeppe}, \hyperref[auth:a206]{Sascha Van Cauwelaert}, \hyperref[auth:a147]{P. Schaus} & The Unary Resource with Transition Times & \href{works/DejemeppeCS15.pdf}{Yes} & \cite{DejemeppeCS15} & 2015 & CP 2015 & 16 & 5 & 11 & \ref{b:DejemeppeCS15} & \ref{c:DejemeppeCS15}\\
DejemeppeD14 \href{https://doi.org/10.1007/978-3-319-07046-9\_20}{DejemeppeD14} & \hyperref[auth:a207]{C. Dejemeppe}, \hyperref[auth:a151]{Y. Deville} & Continuously Degrading Resource and Interval Dependent Activity Durations in Nuclear Medicine Patient Scheduling & \href{works/DejemeppeD14.pdf}{Yes} & \cite{DejemeppeD14} & 2014 & CPAIOR 2014 & 9 & 0 & 7 & \ref{b:DejemeppeD14} & \ref{c:DejemeppeD14}\\
\end{longtable}
}

\subsection{Works by Yves Deville}
\label{sec:a151}
{\scriptsize
\begin{longtable}{>{\raggedright\arraybackslash}p{3cm}>{\raggedright\arraybackslash}p{6cm}>{\raggedright\arraybackslash}p{6.5cm}rrrp{2.5cm}rrrrr}
\rowcolor{white}\caption{Works from bibtex (Total 5)}\\ \toprule
\rowcolor{white}Key & Authors & Title & LC & Cite & Year & \shortstack{Conference\\/Journal} & Pages & \shortstack{Nr\\Cites} & \shortstack{Nr\\Refs} & b & c \\ \midrule\endhead
\bottomrule
\endfoot
DejemeppeD14 \href{https://doi.org/10.1007/978-3-319-07046-9\_20}{DejemeppeD14} & \hyperref[auth:a207]{C. Dejemeppe}, \hyperref[auth:a151]{Y. Deville} & Continuously Degrading Resource and Interval Dependent Activity Durations in Nuclear Medicine Patient Scheduling & \href{works/DejemeppeD14.pdf}{Yes} & \cite{DejemeppeD14} & 2014 & CPAIOR 2014 & 9 & 0 & 7 & \ref{b:DejemeppeD14} & \ref{c:DejemeppeD14}\\
HoundjiSWD14 \href{https://doi.org/10.1007/978-3-319-10428-7\_29}{HoundjiSWD14} & \hyperref[auth:a228]{Vinas{\'{e}}tan Ratheil Houndji}, \hyperref[auth:a147]{P. Schaus}, \hyperref[auth:a229]{Laurence A. Wolsey}, \hyperref[auth:a151]{Y. Deville} & The StockingCost Constraint & \href{works/HoundjiSWD14.pdf}{Yes} & \cite{HoundjiSWD14} & 2014 & CP 2014 & 16 & 5 & 7 & \ref{b:HoundjiSWD14} & \ref{c:HoundjiSWD14}\\
SchausHMCMD11 \href{https://doi.org/10.1007/s10601-010-9100-5}{SchausHMCMD11} & \hyperref[auth:a147]{P. Schaus}, \hyperref[auth:a148]{Pascal Van Hentenryck}, \hyperref[auth:a149]{J. Monette}, \hyperref[auth:a150]{C. Coffrin}, \hyperref[auth:a32]{L. Michel}, \hyperref[auth:a151]{Y. Deville} & Solving Steel Mill Slab Problems with constraint-based techniques: CP, LNS, and {CBLS} & \href{works/SchausHMCMD11.pdf}{Yes} & \cite{SchausHMCMD11} & 2011 & Constraints An Int. J. & 23 & 14 & 5 & \ref{b:SchausHMCMD11} & \ref{c:SchausHMCMD11}\\
MonetteDH09 \href{http://aaai.org/ocs/index.php/ICAPS/ICAPS09/paper/view/712}{MonetteDH09} & \hyperref[auth:a149]{J. Monette}, \hyperref[auth:a151]{Y. Deville}, \hyperref[auth:a148]{Pascal Van Hentenryck} & Just-In-Time Scheduling with Constraint Programming & \href{works/MonetteDH09.pdf}{Yes} & \cite{MonetteDH09} & 2009 & ICAPS 2009 & 8 & 0 & 0 & \ref{b:MonetteDH09} & \ref{c:MonetteDH09}\\
MonetteDD07 \href{https://doi.org/10.1007/978-3-540-72397-4\_14}{MonetteDD07} & \hyperref[auth:a149]{J. Monette}, \hyperref[auth:a151]{Y. Deville}, \hyperref[auth:a372]{P. Dupont} & A Position-Based Propagator for the Open-Shop Problem & \href{works/MonetteDD07.pdf}{Yes} & \cite{MonetteDD07} & 2007 & CPAIOR 2007 & 14 & 0 & 12 & \ref{b:MonetteDD07} & \ref{c:MonetteDD07}\\
\end{longtable}
}

\subsection{Works by Mark G. Wallace}
\label{sec:a155}
{\scriptsize
\begin{longtable}{>{\raggedright\arraybackslash}p{3cm}>{\raggedright\arraybackslash}p{6cm}>{\raggedright\arraybackslash}p{6.5cm}rrrp{2.5cm}rrrrr}
\rowcolor{white}\caption{Works from bibtex (Total 5)}\\ \toprule
\rowcolor{white}Key & Authors & Title & LC & Cite & Year & \shortstack{Conference\\/Journal} & Pages & \shortstack{Nr\\Cites} & \shortstack{Nr\\Refs} & b & c \\ \midrule\endhead
\bottomrule
\endfoot
SchuttFSW13 \href{https://doi.org/10.1007/s10951-012-0285-x}{SchuttFSW13} & \hyperref[auth:a124]{A. Schutt}, \hyperref[auth:a154]{T. Feydy}, \hyperref[auth:a125]{Peter J. Stuckey}, \hyperref[auth:a155]{Mark G. Wallace} & Solving RCPSP/max by lazy clause generation & \href{works/SchuttFSW13.pdf}{Yes} & \cite{SchuttFSW13} & 2013 & J. Sched. & 17 & 43 & 23 & \ref{b:SchuttFSW13} & \ref{c:SchuttFSW13}\\
GuSW12 \href{https://doi.org/10.1007/978-3-642-33558-7\_55}{GuSW12} & \hyperref[auth:a341]{H. Gu}, \hyperref[auth:a125]{Peter J. Stuckey}, \hyperref[auth:a155]{Mark G. Wallace} & Maximising the Net Present Value of Large Resource-Constrained Projects & \href{works/GuSW12.pdf}{Yes} & \cite{GuSW12} & 2012 & CP 2012 & 15 & 5 & 20 & \ref{b:GuSW12} & \ref{c:GuSW12}\\
SchuttCSW12 \href{https://doi.org/10.1007/978-3-642-29828-8\_24}{SchuttCSW12} & \hyperref[auth:a124]{A. Schutt}, \hyperref[auth:a348]{G. Chu}, \hyperref[auth:a125]{Peter J. Stuckey}, \hyperref[auth:a155]{Mark G. Wallace} & Maximising the Net Present Value for Resource-Constrained Project Scheduling & \href{works/SchuttCSW12.pdf}{Yes} & \cite{SchuttCSW12} & 2012 & CPAIOR 2012 & 17 & 18 & 21 & \ref{b:SchuttCSW12} & \ref{c:SchuttCSW12}\\
SchuttFSW11 \href{https://doi.org/10.1007/s10601-010-9103-2}{SchuttFSW11} & \hyperref[auth:a124]{A. Schutt}, \hyperref[auth:a154]{T. Feydy}, \hyperref[auth:a125]{Peter J. Stuckey}, \hyperref[auth:a155]{Mark G. Wallace} & Explaining the cumulative propagator & \href{works/SchuttFSW11.pdf}{Yes} & \cite{SchuttFSW11} & 2011 & Constraints An Int. J. & 33 & 57 & 23 & \ref{b:SchuttFSW11} & \ref{c:SchuttFSW11}\\
abs-1009-0347 \href{http://arxiv.org/abs/1009.0347}{abs-1009-0347} & \hyperref[auth:a124]{A. Schutt}, \hyperref[auth:a154]{T. Feydy}, \hyperref[auth:a125]{Peter J. Stuckey}, \hyperref[auth:a155]{Mark G. Wallace} & Solving the Resource Constrained Project Scheduling Problem with Generalized Precedences by Lazy Clause Generation & \href{works/abs-1009-0347.pdf}{Yes} & \cite{abs-1009-0347} & 2010 & CoRR & 37 & 0 & 0 & \ref{b:abs-1009-0347} & \ref{c:abs-1009-0347}\\
\end{longtable}
}

\subsection{Works by Roger Kameugne}
\label{sec:a10}
{\scriptsize
\begin{longtable}{>{\raggedright\arraybackslash}p{3cm}>{\raggedright\arraybackslash}p{6cm}>{\raggedright\arraybackslash}p{6.5cm}rrrp{2.5cm}rrrrr}
\rowcolor{white}\caption{Works from bibtex (Total 5)}\\ \toprule
\rowcolor{white}Key & Authors & Title & LC & Cite & Year & \shortstack{Conference\\/Journal} & Pages & \shortstack{Nr\\Cites} & \shortstack{Nr\\Refs} & b & c \\ \midrule\endhead
\bottomrule
\endfoot
KameugneFND23 \href{https://doi.org/10.4230/LIPIcs.CP.2023.20}{KameugneFND23} & \hyperref[auth:a10]{R. Kameugne}, \hyperref[auth:a11]{S{\'{e}}v{\'{e}}rine Betmbe Fetgo}, \hyperref[auth:a12]{T. Noulamo}, \hyperref[auth:a13]{Cl{\'{e}}mentin Tayou Djam{\'{e}}gni} & Horizontally Elastic Edge Finder Rule for Cumulative Constraint Based on Slack and Density & \href{works/KameugneFND23.pdf}{Yes} & \cite{KameugneFND23} & 2023 & CP 2023 & 17 & 0 & 0 & \ref{b:KameugneFND23} & \ref{c:KameugneFND23}\\
KameugneFGOQ18 \href{https://doi.org/10.1007/978-3-319-93031-2\_23}{KameugneFGOQ18} & \hyperref[auth:a10]{R. Kameugne}, \hyperref[auth:a11]{S{\'{e}}v{\'{e}}rine Betmbe Fetgo}, \hyperref[auth:a315]{V. Gingras}, \hyperref[auth:a52]{Y. Ouellet}, \hyperref[auth:a37]{C. Quimper} & Horizontally Elastic Not-First/Not-Last Filtering Algorithm for Cumulative Resource Constraint & \href{works/KameugneFGOQ18.pdf}{Yes} & \cite{KameugneFGOQ18} & 2018 & CPAIOR 2018 & 17 & 1 & 12 & \ref{b:KameugneFGOQ18} & \ref{c:KameugneFGOQ18}\\
Kameugne15 \href{https://doi.org/10.1007/s10601-015-9227-5}{Kameugne15} & \hyperref[auth:a10]{R. Kameugne} & Propagation techniques of resource constraint for cumulative scheduling & \href{works/Kameugne15.pdf}{Yes} & \cite{Kameugne15} & 2015 & Constraints An Int. J. & 2 & 0 & 0 & \ref{b:Kameugne15} & \ref{c:Kameugne15}\\
KameugneFSN14 \href{https://doi.org/10.1007/s10601-013-9157-z}{KameugneFSN14} & \hyperref[auth:a10]{R. Kameugne}, \hyperref[auth:a130]{Laure Pauline Fotso}, \hyperref[auth:a131]{Joseph D. Scott}, \hyperref[auth:a132]{Y. Ngo{-}Kateu} & A quadratic edge-finding filtering algorithm for cumulative resource constraints & \href{works/KameugneFSN14.pdf}{Yes} & \cite{KameugneFSN14} & 2014 & Constraints An Int. J. & 27 & 6 & 10 & \ref{b:KameugneFSN14} & \ref{c:KameugneFSN14}\\
KameugneFSN11 \href{https://doi.org/10.1007/978-3-642-23786-7\_37}{KameugneFSN11} & \hyperref[auth:a10]{R. Kameugne}, \hyperref[auth:a130]{Laure Pauline Fotso}, \hyperref[auth:a131]{Joseph D. Scott}, \hyperref[auth:a132]{Y. Ngo{-}Kateu} & A Quadratic Edge-Finding Filtering Algorithm for Cumulative Resource Constraints & \href{works/KameugneFSN11.pdf}{Yes} & \cite{KameugneFSN11} & 2011 & CP 2011 & 15 & 7 & 9 & \ref{b:KameugneFSN11} & \ref{c:KameugneFSN11}\\
\end{longtable}
}

\subsection{Works by Juan M. Novas}
\label{sec:a529}
{\scriptsize
\begin{longtable}{>{\raggedright\arraybackslash}p{3cm}>{\raggedright\arraybackslash}p{6cm}>{\raggedright\arraybackslash}p{6.5cm}rrrp{2.5cm}rrrrr}
\rowcolor{white}\caption{Works from bibtex (Total 5)}\\ \toprule
\rowcolor{white}Key & Authors & Title & LC & Cite & Year & \shortstack{Conference\\/Journal} & Pages & \shortstack{Nr\\Cites} & \shortstack{Nr\\Refs} & b & c \\ \midrule\endhead
\bottomrule
\endfoot
Novas19 \href{https://doi.org/10.1016/j.cie.2019.07.011}{Novas19} & \hyperref[auth:a529]{Juan M. Novas} & Production scheduling and lot streaming at flexible job-shops environments using constraint programming & \href{works/Novas19.pdf}{Yes} & \cite{Novas19} & 2019 & Comput. Ind. Eng. & 13 & 30 & 29 & \ref{b:Novas19} & \ref{c:Novas19}\\
NovaraNH16 \href{https://doi.org/10.1016/j.compchemeng.2016.04.030}{NovaraNH16} & \hyperref[auth:a595]{Franco M. Novara}, \hyperref[auth:a529]{Juan M. Novas}, \hyperref[auth:a596]{Gabriela P. Henning} & A novel constraint programming model for large-scale scheduling problems in multiproduct multistage batch plants: Limited resources and campaign-based operation & \href{works/NovaraNH16.pdf}{Yes} & \cite{NovaraNH16} & 2016 & Comput. Chem. Eng. & 17 & 18 & 31 & \ref{b:NovaraNH16} & \ref{c:NovaraNH16}\\
NovasH14 \href{https://doi.org/10.1016/j.eswa.2013.09.026}{NovasH14} & \hyperref[auth:a529]{Juan M. Novas}, \hyperref[auth:a596]{Gabriela P. Henning} & Integrated scheduling of resource-constrained flexible manufacturing systems using constraint programming & \href{works/NovasH14.pdf}{Yes} & \cite{NovasH14} & 2014 & Expert Syst. Appl. & 14 & 35 & 26 & \ref{b:NovasH14} & \ref{c:NovasH14}\\
NovasH12 \href{https://doi.org/10.1016/j.compchemeng.2012.01.005}{NovasH12} & \hyperref[auth:a529]{Juan M. Novas}, \hyperref[auth:a596]{Gabriela P. Henning} & A comprehensive constraint programming approach for the rolling horizon-based scheduling of automated wet-etch stations & \href{works/NovasH12.pdf}{Yes} & \cite{NovasH12} & 2012 & Comput. Chem. Eng. & 17 & 17 & 15 & \ref{b:NovasH12} & \ref{c:NovasH12}\\
NovasH10 \href{https://doi.org/10.1016/j.compchemeng.2010.07.011}{NovasH10} & \hyperref[auth:a529]{Juan M. Novas}, \hyperref[auth:a596]{Gabriela P. Henning} & Reactive scheduling framework based on domain knowledge and constraint programming & \href{works/NovasH10.pdf}{Yes} & \cite{NovasH10} & 2010 & Comput. Chem. Eng. & 20 & 48 & 19 & \ref{b:NovasH10} & \ref{c:NovasH10}\\
\end{longtable}
}

\subsection{Works by Kenneth N. Brown}
\label{sec:a222}
{\scriptsize
\begin{longtable}{>{\raggedright\arraybackslash}p{3cm}>{\raggedright\arraybackslash}p{6cm}>{\raggedright\arraybackslash}p{6.5cm}rrrp{2.5cm}rrrrr}
\rowcolor{white}\caption{Works from bibtex (Total 5)}\\ \toprule
\rowcolor{white}Key & Authors & Title & LC & Cite & Year & \shortstack{Conference\\/Journal} & Pages & \shortstack{Nr\\Cites} & \shortstack{Nr\\Refs} & b & c \\ \midrule\endhead
\bottomrule
\endfoot
AntunesABDEGGOL20 \href{https://doi.org/10.1142/S0218213020600076}{AntunesABDEGGOL20} & \hyperref[auth:a893]{M. Antunes}, \hyperref[auth:a894]{V. Armant}, \hyperref[auth:a222]{Kenneth N. Brown}, \hyperref[auth:a895]{Daniel A. Desmond}, \hyperref[auth:a896]{G. Escamocher}, \hyperref[auth:a897]{A. George}, \hyperref[auth:a182]{D. Grimes}, \hyperref[auth:a898]{M. O'Keeffe}, \hyperref[auth:a899]{Y. Lin}, \hyperref[auth:a16]{B. O'Sullivan}, \hyperref[auth:a900]{C. Ozturk}, \hyperref[auth:a901]{L. Quesada}, \hyperref[auth:a129]{M. Siala}, \hyperref[auth:a17]{H. Simonis}, \hyperref[auth:a837]{N. Wilson} & Assigning and Scheduling Service Visits in a Mixed Urban/Rural Setting & No & \cite{AntunesABDEGGOL20} & 2020 & Int. J. Artif. Intell. Tools & 31 & 0 & 16 & No & \ref{c:AntunesABDEGGOL20}\\
AntunesABDEGGOL18 \href{https://doi.org/10.1109/ICTAI.2018.00027}{AntunesABDEGGOL18} & \hyperref[auth:a893]{M. Antunes}, \hyperref[auth:a894]{V. Armant}, \hyperref[auth:a222]{Kenneth N. Brown}, \hyperref[auth:a895]{Daniel A. Desmond}, \hyperref[auth:a896]{G. Escamocher}, \hyperref[auth:a897]{A. George}, \hyperref[auth:a182]{D. Grimes}, \hyperref[auth:a898]{M. O'Keeffe}, \hyperref[auth:a899]{Y. Lin}, \hyperref[auth:a16]{B. O'Sullivan}, \hyperref[auth:a900]{C. Ozturk}, \hyperref[auth:a901]{L. Quesada}, \hyperref[auth:a129]{M. Siala}, \hyperref[auth:a17]{H. Simonis}, \hyperref[auth:a837]{N. Wilson} & Assigning and Scheduling Service Visits in a Mixed Urban/Rural Setting & No & \cite{AntunesABDEGGOL18} & 2018 & ICTAI 2018 & 8 & 1 & 24 & No & \ref{c:AntunesABDEGGOL18}\\
MurphyMB15 \href{https://doi.org/10.1007/978-3-319-23219-5\_47}{MurphyMB15} & \hyperref[auth:a220]{Se{\'{a}}n {\'{O}}g Murphy}, \hyperref[auth:a221]{O. Manzano}, \hyperref[auth:a222]{Kenneth N. Brown} & Design and Evaluation of a Constraint-Based Energy Saving and Scheduling Recommender System & \href{works/MurphyMB15.pdf}{Yes} & \cite{MurphyMB15} & 2015 & CP 2015 & 17 & 1 & 20 & \ref{b:MurphyMB15} & \ref{c:MurphyMB15}\\
WuBB09 \href{https://doi.org/10.1016/j.cor.2008.08.008}{WuBB09} & \hyperref[auth:a276]{Christine Wei Wu}, \hyperref[auth:a222]{Kenneth N. Brown}, \hyperref[auth:a89]{J. Christopher Beck} & Scheduling with uncertain durations: Modeling beta-robust scheduling with constraints & No & \cite{WuBB09} & 2009 & Comput. Oper. Res. & 9 & 42 & 5 & No & \ref{c:WuBB09}\\
WuBB05 \href{https://doi.org/10.1007/11564751\_110}{WuBB05} & \hyperref[auth:a276]{Christine Wei Wu}, \hyperref[auth:a222]{Kenneth N. Brown}, \hyperref[auth:a89]{J. Christopher Beck} & Scheduling with Uncertain Start Dates & \href{works/WuBB05.pdf}{Yes} & \cite{WuBB05} & 2005 & CP 2005 & 1 & 0 & 0 & \ref{b:WuBB05} & \ref{c:WuBB05}\\
\end{longtable}
}

\subsection{Works by Mohamed Siala}
\label{sec:a129}
{\scriptsize
\begin{longtable}{>{\raggedright\arraybackslash}p{3cm}>{\raggedright\arraybackslash}p{6cm}>{\raggedright\arraybackslash}p{6.5cm}rrrp{2.5cm}rrrrr}
\rowcolor{white}\caption{Works from bibtex (Total 5)}\\ \toprule
\rowcolor{white}Key & Authors & Title & LC & Cite & Year & \shortstack{Conference\\/Journal} & Pages & \shortstack{Nr\\Cites} & \shortstack{Nr\\Refs} & b & c \\ \midrule\endhead
\bottomrule
\endfoot
AntunesABDEGGOL20 \href{https://doi.org/10.1142/S0218213020600076}{AntunesABDEGGOL20} & \hyperref[auth:a893]{M. Antunes}, \hyperref[auth:a894]{V. Armant}, \hyperref[auth:a222]{Kenneth N. Brown}, \hyperref[auth:a895]{Daniel A. Desmond}, \hyperref[auth:a896]{G. Escamocher}, \hyperref[auth:a897]{A. George}, \hyperref[auth:a182]{D. Grimes}, \hyperref[auth:a898]{M. O'Keeffe}, \hyperref[auth:a899]{Y. Lin}, \hyperref[auth:a16]{B. O'Sullivan}, \hyperref[auth:a900]{C. Ozturk}, \hyperref[auth:a901]{L. Quesada}, \hyperref[auth:a129]{M. Siala}, \hyperref[auth:a17]{H. Simonis}, \hyperref[auth:a837]{N. Wilson} & Assigning and Scheduling Service Visits in a Mixed Urban/Rural Setting & No & \cite{AntunesABDEGGOL20} & 2020 & Int. J. Artif. Intell. Tools & 31 & 0 & 16 & No & \ref{c:AntunesABDEGGOL20}\\
AntunesABDEGGOL18 \href{https://doi.org/10.1109/ICTAI.2018.00027}{AntunesABDEGGOL18} & \hyperref[auth:a893]{M. Antunes}, \hyperref[auth:a894]{V. Armant}, \hyperref[auth:a222]{Kenneth N. Brown}, \hyperref[auth:a895]{Daniel A. Desmond}, \hyperref[auth:a896]{G. Escamocher}, \hyperref[auth:a897]{A. George}, \hyperref[auth:a182]{D. Grimes}, \hyperref[auth:a898]{M. O'Keeffe}, \hyperref[auth:a899]{Y. Lin}, \hyperref[auth:a16]{B. O'Sullivan}, \hyperref[auth:a900]{C. Ozturk}, \hyperref[auth:a901]{L. Quesada}, \hyperref[auth:a129]{M. Siala}, \hyperref[auth:a17]{H. Simonis}, \hyperref[auth:a837]{N. Wilson} & Assigning and Scheduling Service Visits in a Mixed Urban/Rural Setting & No & \cite{AntunesABDEGGOL18} & 2018 & ICTAI 2018 & 8 & 1 & 24 & No & \ref{c:AntunesABDEGGOL18}\\
Siala15 \href{https://doi.org/10.1007/s10601-015-9213-y}{Siala15} & \hyperref[auth:a129]{M. Siala} & Search, propagation, and learning in sequencing and scheduling problems & \href{works/Siala15.pdf}{Yes} & \cite{Siala15} & 2015 & Constraints An Int. J. & 2 & 4 & 0 & \ref{b:Siala15} & \ref{c:Siala15}\\
Siala15a \href{https://tel.archives-ouvertes.fr/tel-01164291}{Siala15a} & \hyperref[auth:a129]{M. Siala} & Search, propagation, and learning in sequencing and scheduling problems. (Recherche, propagation et apprentissage dans les probl{\`{e}}mes de s{\'{e}}quencement et d'ordonnancement) & \href{works/Siala15a.pdf}{Yes} & \cite{Siala15a} & 2015 & {INSA} Toulouse, France & 199 & 0 & 0 & \ref{b:Siala15a} & \ref{c:Siala15a}\\
SialaAH15 \href{https://doi.org/10.1007/978-3-319-23219-5\_28}{SialaAH15} & \hyperref[auth:a129]{M. Siala}, \hyperref[auth:a6]{C. Artigues}, \hyperref[auth:a1]{E. Hebrard} & Two Clause Learning Approaches for Disjunctive Scheduling & \href{works/SialaAH15.pdf}{Yes} & \cite{SialaAH15} & 2015 & CP 2015 & 10 & 4 & 17 & \ref{b:SialaAH15} & \ref{c:SialaAH15}\\
\end{longtable}
}

\subsection{Works by Marek Vlk}
\label{sec:a313}
{\scriptsize
\begin{longtable}{>{\raggedright\arraybackslash}p{3cm}>{\raggedright\arraybackslash}p{6cm}>{\raggedright\arraybackslash}p{6.5cm}rrrp{2.5cm}rrrrr}
\rowcolor{white}\caption{Works from bibtex (Total 5)}\\ \toprule
\rowcolor{white}Key & Authors & Title & LC & Cite & Year & \shortstack{Conference\\/Journal} & Pages & \shortstack{Nr\\Cites} & \shortstack{Nr\\Refs} & b & c \\ \midrule\endhead
\bottomrule
\endfoot
abs-2305-19888 \href{https://doi.org/10.48550/arXiv.2305.19888}{abs-2305-19888} & \hyperref[auth:a437]{V. Heinz}, \hyperref[auth:a438]{A. Nov{\'{a}}k}, \hyperref[auth:a313]{M. Vlk}, \hyperref[auth:a116]{Z. Hanz{\'{a}}lek} & Constraint Programming and Constructive Heuristics for Parallel Machine Scheduling with Sequence-Dependent Setups and Common Servers & \href{works/abs-2305-19888.pdf}{Yes} & \cite{abs-2305-19888} & 2023 & CoRR & 42 & 0 & 0 & \ref{b:abs-2305-19888} & \ref{c:abs-2305-19888}\\
HeinzNVH22 \href{https://doi.org/10.1016/j.cie.2022.108586}{HeinzNVH22} & \hyperref[auth:a437]{V. Heinz}, \hyperref[auth:a438]{A. Nov{\'{a}}k}, \hyperref[auth:a313]{M. Vlk}, \hyperref[auth:a116]{Z. Hanz{\'{a}}lek} & Constraint Programming and constructive heuristics for parallel machine scheduling with sequence-dependent setups and common servers & \href{works/HeinzNVH22.pdf}{Yes} & \cite{HeinzNVH22} & 2022 & Comput. Ind. Eng. & 16 & 5 & 25 & \ref{b:HeinzNVH22} & \ref{c:HeinzNVH22}\\
VlkHT21 \href{https://doi.org/10.1016/j.cie.2021.107317}{VlkHT21} & \hyperref[auth:a313]{M. Vlk}, \hyperref[auth:a116]{Z. Hanz{\'{a}}lek}, \hyperref[auth:a480]{S. Tang} & Constraint programming approaches to joint routing and scheduling in time-sensitive networks & \href{works/VlkHT21.pdf}{Yes} & \cite{VlkHT21} & 2021 & Comput. Ind. Eng. & 14 & 7 & 22 & \ref{b:VlkHT21} & \ref{c:VlkHT21}\\
BenediktSMVH18 \href{https://doi.org/10.1007/978-3-319-93031-2\_6}{BenediktSMVH18} & \hyperref[auth:a114]{O. Benedikt}, \hyperref[auth:a312]{P. Sucha}, \hyperref[auth:a115]{I. M{\'{o}}dos}, \hyperref[auth:a313]{M. Vlk}, \hyperref[auth:a116]{Z. Hanz{\'{a}}lek} & Energy-Aware Production Scheduling with Power-Saving Modes & \href{works/BenediktSMVH18.pdf}{Yes} & \cite{BenediktSMVH18} & 2018 & CPAIOR 2018 & 10 & 2 & 12 & \ref{b:BenediktSMVH18} & \ref{c:BenediktSMVH18}\\
BartakV15 \href{}{BartakV15} & \hyperref[auth:a152]{R. Bart{\'{a}}k}, \hyperref[auth:a313]{M. Vlk} & Reactive Recovery from Machine Breakdown in Production Scheduling with Temporal Distance and Resource Constraints & \href{works/BartakV15.pdf}{Yes} & \cite{BartakV15} & 2015 & ICAART 2015 & 12 & 0 & 0 & \ref{b:BartakV15} & \ref{c:BartakV15}\\
\end{longtable}
}

\subsection{Works by Nic Wilson}
\label{sec:a837}
{\scriptsize
\begin{longtable}{>{\raggedright\arraybackslash}p{3cm}>{\raggedright\arraybackslash}p{6cm}>{\raggedright\arraybackslash}p{6.5cm}rrrp{2.5cm}rrrrr}
\rowcolor{white}\caption{Works from bibtex (Total 5)}\\ \toprule
\rowcolor{white}Key & Authors & Title & LC & Cite & Year & \shortstack{Conference\\/Journal} & Pages & \shortstack{Nr\\Cites} & \shortstack{Nr\\Refs} & b & c \\ \midrule\endhead
\bottomrule
\endfoot
AntunesABDEGGOL20 \href{https://doi.org/10.1142/S0218213020600076}{AntunesABDEGGOL20} & \hyperref[auth:a893]{M. Antunes}, \hyperref[auth:a894]{V. Armant}, \hyperref[auth:a222]{Kenneth N. Brown}, \hyperref[auth:a895]{Daniel A. Desmond}, \hyperref[auth:a896]{G. Escamocher}, \hyperref[auth:a897]{A. George}, \hyperref[auth:a182]{D. Grimes}, \hyperref[auth:a898]{M. O'Keeffe}, \hyperref[auth:a899]{Y. Lin}, \hyperref[auth:a16]{B. O'Sullivan}, \hyperref[auth:a900]{C. Ozturk}, \hyperref[auth:a901]{L. Quesada}, \hyperref[auth:a129]{M. Siala}, \hyperref[auth:a17]{H. Simonis}, \hyperref[auth:a837]{N. Wilson} & Assigning and Scheduling Service Visits in a Mixed Urban/Rural Setting & No & \cite{AntunesABDEGGOL20} & 2020 & Int. J. Artif. Intell. Tools & 31 & 0 & 16 & No & \ref{c:AntunesABDEGGOL20}\\
AntunesABDEGGOL18 \href{https://doi.org/10.1109/ICTAI.2018.00027}{AntunesABDEGGOL18} & \hyperref[auth:a893]{M. Antunes}, \hyperref[auth:a894]{V. Armant}, \hyperref[auth:a222]{Kenneth N. Brown}, \hyperref[auth:a895]{Daniel A. Desmond}, \hyperref[auth:a896]{G. Escamocher}, \hyperref[auth:a897]{A. George}, \hyperref[auth:a182]{D. Grimes}, \hyperref[auth:a898]{M. O'Keeffe}, \hyperref[auth:a899]{Y. Lin}, \hyperref[auth:a16]{B. O'Sullivan}, \hyperref[auth:a900]{C. Ozturk}, \hyperref[auth:a901]{L. Quesada}, \hyperref[auth:a129]{M. Siala}, \hyperref[auth:a17]{H. Simonis}, \hyperref[auth:a837]{N. Wilson} & Assigning and Scheduling Service Visits in a Mixed Urban/Rural Setting & No & \cite{AntunesABDEGGOL18} & 2018 & ICTAI 2018 & 8 & 1 & 24 & No & \ref{c:AntunesABDEGGOL18}\\
BeckW07 \href{https://doi.org/10.1613/jair.2080}{BeckW07} & \hyperref[auth:a89]{J. Christopher Beck}, \hyperref[auth:a837]{N. Wilson} & Proactive Algorithms for Job Shop Scheduling with Probabilistic Durations & \href{works/BeckW07.pdf}{Yes} & \cite{BeckW07} & 2007 & J. Artif. Intell. Res. & 50 & 27 & 0 & \ref{b:BeckW07} & \ref{c:BeckW07}\\
BeckW05 \href{http://ijcai.org/Proceedings/05/Papers/0748.pdf}{BeckW05} & \hyperref[auth:a89]{J. Christopher Beck}, \hyperref[auth:a837]{N. Wilson} & Proactive Algorithms for Scheduling with Probabilistic Durations & \href{works/BeckW05.pdf}{Yes} & \cite{BeckW05} & 2005 & IJCAI 2005 & 6 & 0 & 0 & \ref{b:BeckW05} & \ref{c:BeckW05}\\
BeckW04 \href{}{BeckW04} & \hyperref[auth:a89]{J. Christopher Beck}, \hyperref[auth:a837]{N. Wilson} & Job Shop Scheduling with Probabilistic Durations & \href{works/BeckW04.pdf}{Yes} & \cite{BeckW04} & 2004 & ECAI 2004 & 5 & 0 & 0 & \ref{b:BeckW04} & \ref{c:BeckW04}\\
\end{longtable}
}

\subsection{Works by Armin Wolf}
\label{sec:a51}
{\scriptsize
\begin{longtable}{>{\raggedright\arraybackslash}p{3cm}>{\raggedright\arraybackslash}p{6cm}>{\raggedright\arraybackslash}p{6.5cm}rrrp{2.5cm}rrrrr}
\rowcolor{white}\caption{Works from bibtex (Total 5)}\\ \toprule
\rowcolor{white}Key & Authors & Title & LC & Cite & Year & \shortstack{Conference\\/Journal} & Pages & \shortstack{Nr\\Cites} & \shortstack{Nr\\Refs} & b & c \\ \midrule\endhead
\bottomrule
\endfoot
GeitzGSSW22 \href{https://doi.org/10.1007/978-3-031-08011-1\_10}{GeitzGSSW22} & \hyperref[auth:a47]{M. Geitz}, \hyperref[auth:a48]{C. Grozea}, \hyperref[auth:a49]{W. Steigerwald}, \hyperref[auth:a50]{R. St{\"{o}}hr}, \hyperref[auth:a51]{A. Wolf} & Solving the Extended Job Shop Scheduling Problem with AGVs - Classical and Quantum Approaches & \href{works/GeitzGSSW22.pdf}{Yes} & \cite{GeitzGSSW22} & 2022 & CPAIOR 2022 & 18 & 0 & 24 & \ref{b:GeitzGSSW22} & \ref{c:GeitzGSSW22}\\
SchuttW10 \href{https://doi.org/10.1007/978-3-642-15396-9\_36}{SchuttW10} & \hyperref[auth:a124]{A. Schutt}, \hyperref[auth:a51]{A. Wolf} & A New \emph{O}(\emph{n}\({}^{\mbox{2}}\)log\emph{n}) Not-First/Not-Last Pruning Algorithm for Cumulative Resource Constraints & \href{works/SchuttW10.pdf}{Yes} & \cite{SchuttW10} & 2010 & CP 2010 & 15 & 13 & 14 & \ref{b:SchuttW10} & \ref{c:SchuttW10}\\
SchuttWS05 \href{https://doi.org/10.1007/11963578\_6}{SchuttWS05} & \hyperref[auth:a124]{A. Schutt}, \hyperref[auth:a51]{A. Wolf}, \hyperref[auth:a720]{G. Schrader} & Not-First and Not-Last Detection for Cumulative Scheduling in \emph{O}(\emph{n}\({}^{\mbox{3}}\)log\emph{n}) & \href{works/SchuttWS05.pdf}{Yes} & \cite{SchuttWS05} & 2005 & INAP 2005 & 15 & 6 & 4 & \ref{b:SchuttWS05} & \ref{c:SchuttWS05}\\
WolfS05 \href{https://doi.org/10.1007/11963578\_8}{WolfS05} & \hyperref[auth:a51]{A. Wolf}, \hyperref[auth:a720]{G. Schrader} & \emph{O}(\emph{n} log\emph{n}) Overload Checking for the Cumulative Constraint and Its Application & \href{works/WolfS05.pdf}{Yes} & \cite{WolfS05} & 2005 & INAP 2005 & 14 & 6 & 6 & \ref{b:WolfS05} & \ref{c:WolfS05}\\
Wolf03 \href{https://doi.org/10.1007/978-3-540-45193-8\_50}{Wolf03} & \hyperref[auth:a51]{A. Wolf} & Pruning while Sweeping over Task Intervals & \href{works/Wolf03.pdf}{Yes} & \cite{Wolf03} & 2003 & CP 2003 & 15 & 11 & 7 & \ref{b:Wolf03} & \ref{c:Wolf03}\\
\end{longtable}
}



\clearpage
\section{Other Works}

\clearpage
\subsection{Books from bibtex}
{\scriptsize
\begin{longtable}{>{\raggedright\arraybackslash}p{3cm}>{\raggedright\arraybackslash}p{4.5cm}>{\raggedright\arraybackslash}p{6.0cm}rrrp{2.5cm}rp{1cm}p{1cm}rr}
\rowcolor{white}\caption{BOOK (Total 3)}\\ \toprule
\rowcolor{white}\shortstack{Key\\Source} & Authors & Title (Colored by Open Access)& LC & Cite & Year & \shortstack{Conference\\/Journal\\/School} & Pages & \shortstack{Cites\\OC XR\\SC} & \shortstack{Refs\\OC\\XR} & b & c \\ \midrule\endhead
\bottomrule
\endfoot
\index{ArtiguesDN08}\rowlabel{a:ArtiguesDN08}ArtiguesDN08 \href{http://dx.doi.org/10.1002/9780470611227}{ArtiguesDN08} & \hyperref[auth:a930]{} & Resource Constrained Project Scheduling & No & \cite{ArtiguesDN08} & 2008 & Book & null & 63 0 0 & 0 0 & No & n/a\\
\index{BaptistePN01}\rowlabel{a:BaptistePN01}BaptistePN01 \href{http://dx.doi.org/10.1007/978-1-4615-1479-4}{BaptistePN01} & \hyperref[auth:a162]{P. Baptiste}, \hyperref[auth:a163]{C. L. Pape}, \hyperref[auth:a656]{W. Nuijten} & Constraint-Based Scheduling & No & \cite{BaptistePN01} & 2001 & Book & null & 296 302 0 & 0 0 & No & n/a\\
\index{Hooker00}\rowlabel{a:Hooker00}Hooker00 \href{http://dx.doi.org/10.1002/9781118033036}{Hooker00} & \hyperref[auth:a160]{J. N. Hooker} & Logic Based Methods for Optimization: Combining Optimization and Constraint Satisfaction & No & \cite{Hooker00} & 2000 & Book & null & 185 186 0 & 0 0 & No & n/a\\
\end{longtable}
}



\clearpage
\subsection{PhDThesis from bibtex}
{\scriptsize
\begin{longtable}{>{\raggedright\arraybackslash}p{3cm}>{\raggedright\arraybackslash}p{6cm}>{\raggedright\arraybackslash}p{6.5cm}rrrp{2.5cm}rrrrr}
\rowcolor{white}\caption{Works from bibtex (Total 27)}\\ \toprule
\rowcolor{white}Key & Authors & Title & LC & Cite & Year & \shortstack{Conference\\/Journal} & Pages & \shortstack{Nr\\Cites} & \shortstack{Nr\\Refs} & b & c \\ \midrule\endhead
\bottomrule
\endfoot
\rowlabel{a:Astrand21}Astrand21 \href{https://nbn-resolving.org/urn:nbn:se:kth:diva-294959}{Astrand21} & \hyperref[auth:a74]{M. {\AA}strand} & Short-term Underground Mine Scheduling: An Industrial Application of Constraint Programming & \href{works/Astrand21.pdf}{Yes} & \cite{Astrand21} & 2021 & Royal Institute of Technology, Stockholm, Sweden & 142 & 0 & 0 & \ref{b:Astrand21} & \ref{c:Astrand21}\\
\rowlabel{a:Godet21a}Godet21a \href{https://tel.archives-ouvertes.fr/tel-03681868}{Godet21a} & \hyperref[auth:a476]{A. Godet} & Sur le tri de t{\^{a}}ches pour r{\'{e}}soudre des probl{\`{e}}mes d'ordonnancement avec la programmation par contraintes. (On the use of tasks ordering to solve scheduling problems with constraint programming) & \href{works/Godet21a.pdf}{Yes} & \cite{Godet21a} & 2021 & {IMT} Atlantique Bretagne Pays de la Loire, Brest, France & 168 & 0 & 0 & \ref{b:Godet21a} & \ref{c:Godet21a}\\
\rowlabel{a:Groleaz21}Groleaz21 \href{https://hal.science/tel-03266690}{Groleaz21} & \hyperref[auth:a83]{L. Groleaz} & {The Group Cumulative Scheduling Problem} & \href{works/Groleaz21.pdf}{Yes} & \cite{Groleaz21} & 2021 & {Universit{\'e} de Lyon} & 153 & 0 & 0 & \ref{b:Groleaz21} & \ref{c:Groleaz21}\\
\rowlabel{a:Lemos21}Lemos21 \href{https://scholar.tecnico.ulisboa.pt/records/u5RPHM-pu_yoOLXJF7BHrgJx47D827b0xHb3}{Lemos21} & \hyperref[auth:a887]{Alexandre Duarte {de Almeida} Lemos} & Solving scheduling problems under disruptions & \href{works/Lemos21.pdf}{Yes} & \cite{Lemos21} & 2021 & UNIVERSIDADE DE LISBOA INSTITUTO SUPERIOR TÉCNICO & 188 & 0 & 0 & \ref{b:Lemos21} & \ref{c:Lemos21}\\
\rowlabel{a:Zahout21}Zahout21 \href{https://hal.science/tel-03606639}{Zahout21} & \hyperref[auth:a902]{B. Zahout} & {Algorithmes exacts et approch{\'e}s pour l'ordonnancement des travaux multiressources {\`a} intervalles fixes dans des syst{\`e}mes distribu{\'e}s : approche monocrit{\`e}re et multiagent} & \href{works/Zahout21.pdf}{Yes} & \cite{Zahout21} & 2021 & {Universit{\'e} de Tours - LIFAT} & 185 & 0 & 0 & \ref{b:Zahout21} & \ref{c:Zahout21}\\
\rowlabel{a:Lunardi20}Lunardi20 \href{http://orbilu.uni.lu/handle/10993/43893}{Lunardi20} & \hyperref[auth:a501]{Willian Tessaro Lunardi} & A Real-World Flexible Job Shop Scheduling Problem With Sequencing Flexibility: Mathematical Programming, Constraint Programming, and Metaheuristics & \href{works/Lunardi20.pdf}{Yes} & \cite{Lunardi20} & 2020 & University of Luxembourg, Luxembourg City, Luxembourg & 181 & 0 & 0 & \ref{b:Lunardi20} & \ref{c:Lunardi20}\\
\rowlabel{a:Caballero19}Caballero19 \href{https://www.tesisenred.net/handle/10803/667963#page=1}{Caballero19} & \hyperref[auth:a102]{Jordi Coll Caballero} & Scheduling Through Logic-Based Tools & \href{works/Caballero19.pdf}{Yes} & \cite{Caballero19} & 2019 & Universitat de Girona, Spain & 194 & 0 & 0 & \ref{b:Caballero19} & \ref{c:Caballero19}\\
\rowlabel{a:German18}German18 \href{https://theses.hal.science/tel-01896325}{German18} & \hyperref[auth:a903]{G. German} & {Constraint programming for lot-sizing problems} & \href{works/German18.pdf}{Yes} & \cite{German18} & 2018 & {Universit{\'e} Grenoble Alpes} & 112 & 0 & 0 & \ref{b:German18} & \ref{c:German18}\\
\rowlabel{a:Dejemeppe16}Dejemeppe16 \href{https://hdl.handle.net/2078.1/178078}{Dejemeppe16} & \hyperref[auth:a207]{C. Dejemeppe} & Constraint programming algorithms and models for scheduling applications & \href{works/Dejemeppe16.pdf}{Yes} & \cite{Dejemeppe16} & 2016 & Catholic University of Louvain, Louvain-la-Neuve, Belgium & 274 & 0 & 0 & \ref{b:Dejemeppe16} & \ref{c:Dejemeppe16}\\
\rowlabel{a:Fahimi16}Fahimi16 \href{http://cp2014.a4cp.org/sites/default/files/hamed_fahimi_-_efficient_algorithms_to_solve_scheduling_problems_with_a_variety_of_optimization_criteria.pdf}{Fahimi16} & \hyperref[auth:a122]{H. Fahimi} & Efficient algorithms to solve scheduling problems with a variety of optimization criteria & \href{works/Fahimi16.pdf}{Yes} & \cite{Fahimi16} & 2016 & Universit{\'{e}} Laval, Quebec, Canada & 120 & 0 & 0 & \ref{b:Fahimi16} & \ref{c:Fahimi16}\\
\rowlabel{a:Froger16}Froger16 \href{https://theses.hal.science/tel-01440836}{Froger16} & \hyperref[auth:a901]{A. Froger} & {Maintenance scheduling in the electricity industry : a particular focus on a problem rising in the onshore wind industry} & \href{works/Froger16.pdf}{Yes} & \cite{Froger16} & 2016 & {Universit{\'e} d'Angers} & 181 & 0 & 0 & \ref{b:Froger16} & \ref{c:Froger16}\\
\rowlabel{a:Nattaf16}Nattaf16 \href{https://laas.hal.science/tel-01417288}{Nattaf16} & \hyperref[auth:a81]{M. Nattaf} & {Ordonnancement sous contraintes d'{\'e}nergie} & \href{works/Nattaf16.pdf}{Yes} & \cite{Nattaf16} & 2016 & {UPS Toulouse - Universit{\'e} Toulouse 3 Paul Sabatier} & 199 & 0 & 0 & \ref{b:Nattaf16} & \ref{c:Nattaf16}\\
\rowlabel{a:Derrien15}Derrien15 \href{https://tel.archives-ouvertes.fr/tel-01242789}{Derrien15} & \hyperref[auth:a225]{A. Derrien} & Ordonnancement cumulatif en programmation par contraintes : caract{\'{e}}risation {\'{e}}nerg{\'{e}}tique des raisonnements et solutions robustes. (Cumulative scheduling in constraint programming : energetic characterization of reasoning and robust solutions) & \href{works/Derrien15.pdf}{Yes} & \cite{Derrien15} & 2015 & {\'{E}}cole des mines de Nantes, France & 113 & 0 & 0 & \ref{b:Derrien15} & \ref{c:Derrien15}\\
\rowlabel{a:Siala15a}Siala15a \href{https://tel.archives-ouvertes.fr/tel-01164291}{Siala15a} & \hyperref[auth:a129]{M. Siala} & Search, propagation, and learning in sequencing and scheduling problems. (Recherche, propagation et apprentissage dans les probl{\`{e}}mes de s{\'{e}}quencement et d'ordonnancement) & \href{works/Siala15a.pdf}{Yes} & \cite{Siala15a} & 2015 & {INSA} Toulouse, France & 199 & 0 & 0 & \ref{b:Siala15a} & \ref{c:Siala15a}\\
\rowlabel{a:Kameugne14}Kameugne14 \href{http://cp2013.a4cp.org/sites/default/files/roger_kameugne_-_propagation_techniques_of_resource_constraint_for_cumulative_scheduling.pdf}{Kameugne14} & \hyperref[auth:a10]{R. Kameugne} & Techniques de Propagation de la Contrainte de Ressource en Ordonnancement Cumulatif & \href{works/Kameugne14.pdf}{Yes} & \cite{Kameugne14} & 2014 & University of Yaounde I, Cameroon & 139 & 0 & 0 & \ref{b:Kameugne14} & \ref{c:Kameugne14}\\
\rowlabel{a:Letort13}Letort13 \href{https://theses.hal.science/tel-00932215}{Letort13} & \hyperref[auth:a127]{A. Letort} & {Passage {\`a} l'{\'e}chelle pour les contraintes d'ordonnancement multi-ressources} & \href{works/Letort13.pdf}{Yes} & \cite{Letort13} & 2013 & {Ecole des Mines de Nantes} & 132 & 0 & 0 & \ref{b:Letort13} & \ref{c:Letort13}\\
\rowlabel{a:Clercq12}Clercq12 \href{https://theses.hal.science/tel-00794323}{Clercq12} & \hyperref[auth:a900]{Alexis de Clercq} & {Ordonnancement cumulatif avec d{\'e}passements de capacit{\'e} : Contrainte globale et d{\'e}compositions} & \href{works/Clercq12.pdf}{Yes} & \cite{Clercq12} & 2012 & {Ecole des Mines de Nantes} & 196 & 0 & 0 & \ref{b:Clercq12} & \ref{c:Clercq12}\\
\rowlabel{a:Malapert11}Malapert11 \href{https://tel.archives-ouvertes.fr/tel-00630122}{Malapert11} & \hyperref[auth:a82]{A. Malapert} & Techniques d'ordonnancement d'atelier et de fourn{\'{e}}es bas{\'{e}}es sur la programmation par contraintes. (Shop and batch scheduling with constraints) & \href{works/Malapert11.pdf}{Yes} & \cite{Malapert11} & 2011 & {\'{E}}cole des mines de Nantes, France & 194 & 0 & 0 & \ref{b:Malapert11} & \ref{c:Malapert11}\\
\rowlabel{a:Menana11}Menana11 \href{https://tel.archives-ouvertes.fr/tel-00785838}{Menana11} & \hyperref[auth:a622]{J. Menana} & Automates et programmation par contraintes pour la planification de personnel. (Automata and Constraint Programming for Personnel Scheduling Problems) & \href{works/Menana11.pdf}{Yes} & \cite{Menana11} & 2011 & University of Nantes, France & 148 & 0 & 0 & \ref{b:Menana11} & \ref{c:Menana11}\\
\rowlabel{a:Schutt11}Schutt11 \href{https://www.a4cp.org/sites/default/files/andreas_schutt_-_improving_scheduling_by_learning.pdf}{Schutt11} & \hyperref[auth:a124]{A. Schutt} & Improving Scheduling by Learning & \href{works/Schutt11.pdf}{Yes} & \cite{Schutt11} & 2011 & University of Melbourne, Australia & 209 & 0 & 0 & \ref{b:Schutt11} & \ref{c:Schutt11}\\
\rowlabel{a:Lombardi10}Lombardi10 \href{http://amsdottorato.unibo.it/2961/}{Lombardi10} & \hyperref[auth:a142]{M. Lombardi} & Hybrid Methods for Resource Allocation and Scheduling Problems in Deterministic and Stochastic Environments & \href{works/Lombardi10.pdf}{Yes} & \cite{Lombardi10} & 2010 & University of Bologna, Italy & 175 & 0 & 0 & \ref{b:Lombardi10} & \ref{c:Lombardi10}\\
\rowlabel{a:Malik08}Malik08 \href{https://hdl.handle.net/10012/3612}{Malik08} & \hyperref[auth:a647]{Abid M. Malik} & Constraint Programming Techniques for Optimal Instruction Scheduling & \href{works/Malik08.pdf}{Yes} & \cite{Malik08} & 2008 & University of Waterloo, Ontario, Canada & 151 & 0 & 0 & \ref{b:Malik08} & \ref{c:Malik08}\\
\rowlabel{a:Demassey03}Demassey03 \href{https://tel.archives-ouvertes.fr/tel-00293564}{Demassey03} & \hyperref[auth:a245]{S. Demassey} & M{\'{e}}thodes hybrides de programmation par contraintes et programmation lin{\'{e}}aire pour le probl{\`{e}}me d'ordonnancement de projet {\`{a}} contraintes de ressources. (Hybrid Constraint Programming-Integer Linear Programming approaches for the Resource-Constrained Project Scheduling Problem) & \href{works/Demassey03.pdf}{Yes} & \cite{Demassey03} & 2003 & University of Avignon, France & 148 & 0 & 0 & \ref{b:Demassey03} & \ref{c:Demassey03}\\
\rowlabel{a:Elkhyari03}Elkhyari03 \href{https://theses.hal.science/tel-00008377}{Elkhyari03} & \hyperref[auth:a294]{A. Elkhyari} & {Outils d'aide {\`a} la d{\'e}cision pour des probl{\`e}mes d'ordonnancement dynamiques} & \href{works/Elkhyari03.pdf}{Yes} & \cite{Elkhyari03} & 2003 & {Universit{\'e} de Nantes} & 333 & 0 & 0 & \ref{b:Elkhyari03} & \ref{c:Elkhyari03}\\
\rowlabel{a:Baptiste02}Baptiste02 \href{https://theses.hal.science/tel-00124998}{Baptiste02} & \hyperref[auth:a163]{P. Baptiste} & {R{\'e}sultats de complexit{\'e} et programmation par contraintes pour l'ordonnancement} & \href{works/Baptiste02.pdf}{Yes} & \cite{Baptiste02} & 2002 & {Universit{\'e} de Technologie de Compi{\`e}gne} & 237 & 0 & 0 & \ref{b:Baptiste02} & \ref{c:Baptiste02}\\
\rowlabel{a:Layfield02}Layfield02 \href{http://etheses.whiterose.ac.uk/1301/}{Layfield02} & \hyperref[auth:a680]{Colin J. Layfield} & A constraint programming pre-processor for duty scheduling & \href{works/Layfield02.pdf}{Yes} & \cite{Layfield02} & 2002 & University of Leeds, {UK} & 230 & 0 & 0 & \ref{b:Layfield02} & \ref{c:Layfield02}\\
\rowlabel{a:Beck99}Beck99 \href{https://librarysearch.library.utoronto.ca/permalink/01UTORONTO\_INST/14bjeso/alma991106162342106196}{Beck99} & \hyperref[auth:a89]{J. Christopher Beck} & Texture measurements as a basis for heuristic commitment techniques in constraint-directed scheduling & \href{works/Beck99.pdf}{Yes} & \cite{Beck99} & 1999 & University of Toronto, Canada & 418 & 0 & 0 & \ref{b:Beck99} & \ref{c:Beck99}\\
\end{longtable}
}



\clearpage
{\scriptsize
\begin{longtable}{>{\raggedright\arraybackslash}p{3cm}r>{\raggedright\arraybackslash}p{4cm}p{1.5cm}p{2cm}p{1.5cm}p{1.5cm}p{1.5cm}p{1.5cm}p{2cm}p{1.5cm}rr}
\rowcolor{white}\caption{Automatically Extracted THESIS Properties (Requires Local Copy)}\\ \toprule
\rowcolor{white}Work & Pages & Concepts & Classification & Constraints & \shortstack{Prog\\Languages} & \shortstack{CP\\Systems} & Areas & Industries & Benchmarks & Algorithm & a & c\\ \midrule\endhead
\bottomrule
\endfoot
\rowlabel{b:Astrand21}\href{../works/Astrand21.pdf}{Astrand21}~\cite{Astrand21} & 142 & distributed, one-machine scheduling, due-date, job-shop, flow-shop, resource, transportation, net present value, open-shop, machine, job, re-scheduling, stochastic, precedence, order, inventory, two-stage scheduling, tardiness, activity, setup-time, preempt, breakdown, release-date, planned maintenance, periodic, scheduling, make-span, completion-time, multi-objective, task, unavailability, sequence dependent setup & RCPSP, parallel machine, Resource-constrained Project Scheduling Problem, HFS, single machine, Partial Order Schedule & cumulative, alldifferent, cycle, circuit, disjunctive, Disjunctive constraint, Reified constraint & C++, Julia & Cplex, OPL, Gecode & satellite, drone, agriculture, semiconductor, robot & mineral industry, mining industry, maritime industry, potash industry, shipping industry & real-world, generated instance, real-life, benchmark & time-tabling, not-first, large neighborhood search, not-last, meta heuristic, neural network, reinforcement learning, edge-finding, simulated annealing, genetic algorithm, NEH & \ref{a:Astrand21} & n/a\\
\rowlabel{b:Baptiste02}\href{../works/Baptiste02.pdf}{Baptiste02}~\cite{Baptiste02} & 237 & re-scheduling, resource, release-date, scheduling, Pareto, preempt, flow-time, task, job-shop, preemptive, machine, activity, make-span, reactive scheduling, flow-shop, job, completion-time, precedence, distributed, inventory, no preempt, setup-time, due-date, single-machine scheduling, open-shop, tardiness, order, lateness, earliness, one-machine scheduling, cmax, sequence dependent setup & Open Shop Scheduling Problem, PJSSP, HFS, single machine, RCPSP, OSSP, parallel machine, Resource-constrained Project Scheduling Problem, JSSP & cumulative, circuit, disjunctive, Cardinality constraint, Disjunctive constraint, alternative constraint, table constraint, Arithmetic constraint & Prolog, C++ & Choco Solver, Claire, Ilog Solver, OPL, CHIP, ECLiPSe, Ilog Scheduler, Z3 & hoist &  & real-life, generated instance, benchmark & genetic algorithm, column generation, not-first, Lagrangian relaxation, energetic reasoning, not-last, simulated annealing, edge-finding & \ref{a:Baptiste02} & n/a\\
\rowlabel{b:Beck99}\href{../works/Beck99.pdf}{Beck99}~\cite{Beck99} & 418 & stochastic, due-date, multi-agent, order, distributed, preempt, scheduling, inventory, preemptive, machine, release-date, job-shop, task, tardiness, activity, transportation, stock level, precedence, make-span, re-scheduling, resource, job, producer/consumer & single machine & cumulative, Disjunctive constraint, circuit, disjunctive & Prolog, C++ & Ilog Solver, CHIP, Ilog Scheduler, OPL & telescope, robot, evacuation, medical &  & benchmark, real-world & column generation, not-last, machine learning, edge-finding, meta heuristic, not-first, simulated annealing, genetic algorithm & \ref{a:Beck99} & n/a\\
\rowlabel{b:Caballero19}\href{../works/Caballero19.pdf}{Caballero19}~\cite{Caballero19} & 194 & resource, machine, setup-time, preempt, periodic, task, order, activity, distributed, precedence, release-date, cmax, make-span, preemptive, scheduling, completion-time & psplib, Resource-constrained Project Scheduling Problem, RCPSP & alldifferent, circuit, Cardinality constraint, cycle, Arithmetic constraint, cumulative & C++ & SCIP, CHIP, Z3, CPO, Chuffed, MiniZinc, OPL &  &  & benchmark, real-life, instance generator & lazy clause generation, energetic reasoning, GRASP, time-tabling, meta heuristic, edge-finding, bi-partite matching, conflict-driven clause learning & \ref{a:Caballero19} & n/a\\
\rowlabel{b:Clercq12}\href{../works/Clercq12.pdf}{Clercq12}~\cite{Clercq12} & 196 & task, order, machine, job, manpower, activity, job-shop, make-span, resource, scheduling, due-date & psplib & Cumulatives constraint, alldifferent, cumulative, disjunctive, SoftCumulativeSum, circuit, SoftCumulative & Prolog & ECLiPSe, SICStus, Choco Solver, CHIP, Gecode & patient &  & benchmark & not-last, energetic reasoning, edge-finding, sweep, time-tabling, not-first & \ref{a:Clercq12} & n/a\\
\rowlabel{b:Dejemeppe16}\href{../works/Dejemeppe16.pdf}{Dejemeppe16}~\cite{Dejemeppe16} & 274 & make-span, sequence dependent setup, open-shop, order, job, activity, Pareto, continuous-process, machine, preempt, release-date, flow-shop, batch process, multi-objective, energy efficiency, tardiness, preemptive, scheduling, completion-time, re-scheduling, resource, setup-time, earliness, due-date, no-wait, task, stochastic, job-shop, lateness, precedence, bi-objective & PTC, psplib, single machine, Resource-constrained Project Scheduling Problem, RCPSP & disjunctive, cumulative, Element constraint, Reified constraint, Cumulatives constraint, alldifferent, GCC constraint, cycle, circuit, Disjunctive constraint, Cardinality constraint, Regular constraint &  & Ilog Solver, OPL, Gecode, CHIP, OR-Tools, CPO & medical, patient, super-computer, nurse, physician, robot, container terminal & paper industry & benchmark, instance generator, generated instance, industrial partner, random instance, real-world, bitbucket & Lagrangian relaxation, simulated annealing, not-first, meta heuristic, ant colony, not-last, particle swarm, sweep, edge-finding, genetic algorithm, large neighborhood search & \ref{a:Dejemeppe16} & n/a\\
\rowlabel{b:Demassey03}\href{../works/Demassey03.pdf}{Demassey03}~\cite{Demassey03} & 148 & machine, job, precedence, Benders Decomposition, release-date, stochastic, Logic-Based Benders Decomposition, job-shop, preemptive, single-machine scheduling, open-shop, activity, flow-shop, order, resource, scheduling, preempt, task & single machine, Resource-constrained Project Scheduling Problem, CuSP, psplib, RCPSP, TCSP & circuit, cumulative, disjunctive, cycle & C++ & Cplex, Claire, Ilog Solver &  &  & benchmark & not-last, meta heuristic, edge-finding, time-tabling, column generation, not-first, Lagrangian relaxation & \ref{a:Demassey03} & n/a\\
\rowlabel{b:Derrien15}\href{../works/Derrien15.pdf}{Derrien15}~\cite{Derrien15} & 113 & scheduling, precedence, order, make-span, task, activity, preemptive, job-shop, resource, machine, job, stochastic, preempt, open-shop & psplib, CuSP & Disjunctive constraint, cumulative, alldifferent, circuit, disjunctive &  & Claire, Choco Solver & robot &  & benchmark & edge-finding, sweep, time-tabling, energetic reasoning & \ref{a:Derrien15} & n/a\\
\rowlabel{b:Elkhyari03}\href{../works/Elkhyari03.pdf}{Elkhyari03}~\cite{Elkhyari03} & 333 & scheduling, task, job-shop, preemptive, machine, activity, make-span, flow-shop, cmax, open-shop, tardiness, order, preempt, breakdown, re-scheduling, reactive scheduling, resource, job, precedence, release-date, periodic & RCPSP, CuSP, parallel machine, Resource-constrained Project Scheduling Problem, Temporal Constraint Satisfaction Problem, single machine & cycle, cumulative, disjunctive &  & CPO, Choco Solver, Claire &  &  & benchmark, Roadef & meta heuristic, time-tabling, mat heuristic, genetic algorithm & \ref{a:Elkhyari03} & n/a\\
\rowlabel{b:Fahimi16}\href{../works/Fahimi16.pdf}{Fahimi16}~\cite{Fahimi16} & 120 & reactive scheduling, completion-time, flow-shop, precedence, batch process, setup-time, due-date, task, open-shop, preemptive, order, make-span, stochastic, machine, job, periodic, activity, resource, lateness, job-shop, Logic-Based Benders Decomposition, transportation, sequence dependent setup, preempt, tardiness, scheduling, Benders Decomposition & single machine, CuSP, parallel machine, RCPSP & Disjunctive constraint, Cardinality constraint, Cumulatives constraint, alldifferent, cycle, AllDiff constraint, cumulative, alternative constraint, disjunctive & Java, C++ & Choco Solver, CHIP, Ilog Scheduler, Gecode & aircraft &  & benchmark, random instance, real-world, Roadef & time-tabling, not-first, not-last, energetic reasoning, edge-finding, max-flow, sweep & \ref{a:Fahimi16} & n/a\\
\rowlabel{b:Froger16}\href{../works/Froger16.pdf}{Froger16}~\cite{Froger16} & 181 & breakdown, preempt, distributed, resource, inventory, scheduling, multi-objective, Benders Decomposition, reactive scheduling, batch process, re-scheduling, task, preemptive, order, sustainability, stochastic, completion-time, machine, job, manpower, Pareto, release-date, Logic-Based Benders Decomposition, unavailability, transportation & single machine, CuSP, TMS & disjunctive, cycle, cumulative & Java & Gurobi, OZ, Choco Solver & train schedule, maintenance scheduling, satellite, energy-price, offshore & power industry, electricity industry, energy industry, wind industry & benchmark, real-life, real-world, industrial partner, instance generator, Roadef, generated instance & ant colony, particle swarm, genetic algorithm, neural network, large neighborhood search, Lagrangian relaxation, simulated annealing, column generation, max-flow, mat heuristic, meta heuristic & \ref{a:Froger16} & n/a\\
\rowlabel{b:German18}\href{../works/German18.pdf}{German18}~\cite{German18} & 112 & stock level, setup-time, job, task, activity, stochastic, earliness, machine, resource, job-shop, cmax, order, inventory, scheduling &  & Disjunctive constraint, Cardinality constraint, bin-packing, Balance constraint, cumulative, Among constraint, disjunctive & Prolog & Z3, SICStus, OPL, Choco Solver, Cplex & nurse &  & real-world, benchmark, real-life, CSPlib, generated instance &  & \ref{a:German18} & n/a\\
\rowlabel{b:Godet21a}\href{../works/Godet21a.pdf}{Godet21a}~\cite{Godet21a} & 168 & open-shop, release-date, make-span, transportation, machine, distributed, periodic, resource, lateness, job-shop, flow-shop, precedence, cmax, preempt, due-date, preemptive, order, scheduling, Benders Decomposition, completion-time, job, task, activity & single machine, RCPSP, parallel machine, JSSP, PMSP, Resource-constrained Project Scheduling Problem, psplib & AllDiff constraint, bin-packing, GeneralizedAllDiffPrec, disjunctive, BinPacking constraint, cumulative, AllDiffPrec constraint, Disjunctive constraint, Element constraint, alldifferent, Cardinality constraint, cycle &  & OR-Tools, OPL, Claire, Choco Solver, Chuffed, MiniZinc, CHIP & satellite, robot, railway, maintenance scheduling & electricity industry & real-life, github, generated instance, benchmark, random instance & sweep, lazy clause generation, meta heuristic, time-tabling, edge-finding & \ref{a:Godet21a} & n/a\\
\rowlabel{b:Groleaz21}\href{../works/Groleaz21.pdf}{Groleaz21}~\cite{Groleaz21} & 153 & inventory, tardiness, activity, setup-time, preempt, breakdown, release-date, earliness, periodic, single-machine scheduling, scheduling, make-span, completion-time, task, online scheduling, bi-objective, reactive scheduling, preemptive, sequence dependent setup, distributed, due-date, job-shop, flow-shop, resource, transportation, cmax, open-shop, machine, job, lateness, re-scheduling, stochastic, precedence, order & Resource-constrained Project Scheduling Problem, Open Shop Scheduling Problem, single machine, GCSP, RCPSP, OSP, parallel machine & circuit, disjunctive, Disjunctive constraint, span constraint, cumulative, cycle, noOverlap & Julia, Java & Choco Solver, Z3, OPL, OR-Tools, Gurobi, CPO, Gecode, SCIP, Cplex & dairy, robot, automotive, business process & food industry, agrifood industry, dairy industry & benchmark, real-life & mat heuristic, evolutionary computing, memetic algorithm, meta heuristic, swarm intelligence, neural network, column generation, edge-finding, machine learning, simulated annealing, genetic algorithm, not-first, ant colony, large neighborhood search, not-last & \ref{a:Groleaz21} & n/a\\
\rowlabel{b:Kameugne14}\href{../works/Kameugne14.pdf}{Kameugne14}~\cite{Kameugne14} & 139 & resource, job, scheduling, task, job-shop, preemptive, machine, make-span, flow-shop, completion-time, order, preempt & RCPSP, CuSP, parallel machine, Resource-constrained Project Scheduling Problem, psplib & circuit, Disjunctive constraint, Cumulatives constraint, Balance constraint, cumulative, disjunctive & Java, Prolog, C++ & Choco Solver, Claire, Gecode, CHIP, ECLiPSe, SICStus, Cplex, Mistral &  &  & Roadef & not-last, edge-finder, energetic reasoning, time-tabling, edge-finding, not-first & \ref{a:Kameugne14} & n/a\\
\rowlabel{b:Layfield02}\href{../works/Layfield02.pdf}{Layfield02}~\cite{Layfield02} & 230 &  &  &  & C  & OPL, OZ, Z3 &  &  &  &  & \ref{a:Layfield02} & n/a\\
\rowlabel{b:Lemos21}\href{../works/Lemos21.pdf}{Lemos21}~\cite{Lemos21} & 188 & transportation, precedence, job-shop, multi-objective, machine, re-scheduling, distributed, unavailability, multi-agent, bi-objective, task, job, stochastic, order, periodic, energy efficiency, Pareto, resource, scheduling & RCPSP & cycle, alldifferent, cumulative, Cardinality constraint & Java, C++, Python & OPL, Gurobi, Cplex & surgery, meeting scheduling, COVID, train schedule, high school timetabling, medical, crew-scheduling, railway & railway industry & real-world, github, real-life, benchmark, Roadef & machine learning, simulated annealing, large neighborhood search, meta heuristic, GRASP, reinforcement learning, genetic algorithm, conflict-driven clause learning, evolutionary computing, time-tabling & \ref{a:Lemos21} & n/a\\
\rowlabel{b:Letort13}\href{../works/Letort13.pdf}{Letort13}~\cite{Letort13} & 132 & machine, resource, job-shop, precedence, cmax, order, scheduling, job, task & psplib & bin-packing, alldifferent, cumulative, geost, Cumulatives constraint, disjunctive & Java, Prolog & SICStus, Claire, Choco Solver, CHIP & steel mill, datacenter &  & Roadef, CSPlib, benchmark & energetic reasoning, edge-finding, sweep, meta heuristic, not-first, time-tabling, large neighborhood search, not-last & \ref{a:Letort13} & n/a\\
\rowlabel{b:Lombardi10}\href{../works/Lombardi10.pdf}{Lombardi10}~\cite{Lombardi10} & 175 & re-scheduling, make-span, job, precedence, Benders Decomposition, release-date, periodic, stochastic, distributed, Logic-Based Benders Decomposition, setup-time, job-shop, preemptive, due-date, activity, completion-time, order, net present value, inventory, multi-objective, tardiness, resource, energy efficiency, scheduling, preempt, task, machine & single machine, SCC, Resource-constrained Project Scheduling Problem, CTW, RCPSP, TCSP & Disjunctive constraint, cycle, Balance constraint, AllDiff constraint, cumulative, disjunctive, table constraint, span constraint, bin-packing, circuit & C  & OPL, Cplex, Ilog Solver & aircraft, pipeline, semiconductor, business process, medical, automotive & semiconductor industry & generated instance, benchmark, real-world, instance generator, real-life & not-last, simulated annealing, lazy clause generation, meta heuristic, sweep, large neighborhood search, edge-finder, edge-finding, energetic reasoning, genetic algorithm, time-tabling, column generation, not-first, Lagrangian relaxation, machine learning & \ref{a:Lombardi10} & n/a\\
\rowlabel{b:Lunardi20}\href{../works/Lunardi20.pdf}{Lunardi20}~\cite{Lunardi20} & 181 & activity, setup-time, breakdown, Pareto, release-date, reactive scheduling, unavailability, scheduling, make-span, task, cmax, bi-objective, machine, job, lateness, re-scheduling, stochastic, no preempt, due-date, job-shop, batch process, preempt, flow-shop, resource, transportation, open-shop, precedence, order, completion-time, multi-objective, tardiness & FJS, parallel machine, single machine & cycle, noOverlap, endBeforeStart, alldifferent, disjunctive & Python & CPO, OPL, Cplex & robot, high performance computing & printing industry, glass industry & industrial partner, instance generator, benchmark, random instance, github, supplementary material, real-world, generated instance, real-life & mat heuristic, memetic algorithm, meta heuristic, machine learning, simulated annealing, genetic algorithm, particle swarm, ant colony, swarm intelligence, neural network, reinforcement learning & \ref{a:Lunardi20} & n/a\\
\rowlabel{b:Malapert11}\href{../works/Malapert11.pdf}{Malapert11}~\cite{Malapert11} & 194 & tardiness, order, lateness, preempt, cmax, multi-objective, batch process, transportation, resource, scheduling, flow-time, task, job-shop, preemptive, machine, activity, make-span, no-wait, flow-shop, job, completion-time, precedence, planned maintenance, inventory, setup-time, due-date, open-shop & Open Shop Scheduling Problem, single machine & cumulative, diffn, circuit, disjunctive, geost, cycle, alldifferent, Element constraint, bin-packing, Disjunctive constraint, Cumulatives constraint & Prolog, C++, Java & Mistral, Choco Solver, Claire, Gecode, ECLiPSe, SICStus, Cplex, OPL, CHIP, Ilog Scheduler & rectangle-packing, robot, semiconductor, maintenance scheduling, patient &  & real-world, industrial partner, generated instance, benchmark & edge-finding, particle swarm, genetic algorithm, column generation, not-first, ant colony, energetic reasoning, not-last, time-tabling, meta heuristic, sweep & \ref{a:Malapert11} & n/a\\
\rowlabel{b:Malik08}\href{../works/Malik08.pdf}{Malik08}~\cite{Malik08} & 151 & order, machine, completion-time, activity, distributed, precedence, breakdown, task, job, resource, make-span, cyclic scheduling, scheduling &  & alldifferent, Cardinality constraint, cycle &  &  & pipeline &  & real-life, benchmark & edge-finding, machine learning & \ref{a:Malik08} & n/a\\
\rowlabel{b:Menana11}\href{../works/Menana11.pdf}{Menana11}~\cite{Menana11} & 148 & machine, task, manpower, activity, distributed, resource, multi-objective, cyclic scheduling, precedence, scheduling &  & Regular constraint, alldifferent, Cardinality constraint & Prolog & Z3, CHIP, OPL, Claire, Choco Solver & nurse &  & Roadef, github, benchmark & large neighborhood search, Lagrangian relaxation, meta heuristic, time-tabling, genetic algorithm, column generation & \ref{a:Menana11} & n/a\\
\rowlabel{b:Nattaf16}\href{../works/Nattaf16.pdf}{Nattaf16}~\cite{Nattaf16} & 199 & preemptive, order, tardiness, inventory, scheduling, flow-shop, setup-time, job, task, make-span, machine, resource, job-shop, bi-objective, cmax, preempt, due-date & RCPSP, CECSP, Resource-constrained Project Scheduling Problem, psplib, single machine, CuSP, parallel machine & alldifferent, cumulative, disjunctive & C++ & Claire, Cplex & maintenance scheduling, robot & process industry & Roadef & genetic algorithm, column generation, energetic reasoning, edge-finding, sweep, mat heuristic & \ref{a:Nattaf16} & n/a\\
\rowlabel{b:Schutt11}\href{../works/Schutt11.pdf}{Schutt11}~\cite{Schutt11} & 209 & resource, job-shop, precedence, cmax, preempt, preemptive, order, tardiness, scheduling, completion-time, machine, setup-time, job, periodic, task, activity, open-shop, one-machine scheduling, release-date, make-span & RCPSP, Resource-constrained Project Scheduling Problem, Open Shop Scheduling Problem, psplib & disjunctive, Arithmetic constraint, UTVPI constraint, cumulative, circuit, bin-packing, Reified constraint, Disjunctive constraint, Element constraint, alldifferent, cycle, geost & Prolog, C++ & CHIP, SICStus, Ilog Scheduler, SCIP, ECLiPSe, Ilog Solver & rectangle-packing & carpet industry & benchmark, real-world, industrial instance, instance generator & sweep, ant colony, lazy clause generation, meta heuristic, edge-finder, time-tabling, not-first, simulated annealing, energetic reasoning, edge-finding, not-last & \ref{a:Schutt11} & n/a\\
\rowlabel{b:Siala15a}\href{../works/Siala15a.pdf}{Siala15a}~\cite{Siala15a} & 199 & job-shop, precedence, earliness, cmax, sequence dependent setup, due-date, order, tardiness, scheduling, setup-time, task, activity, open-shop, make-span, machine, job, periodic, resource & RCPSP, OSP, single machine, TMS & AtMostSeq, table constraint, Balance constraint, cumulative, circuit, Among constraint, AmongSeq constraint, disjunctive, Atmost constraint, Regular constraint, Disjunctive constraint, GCC constraint, Cardinality constraint, CardPath, MultiAtMostSeqCard, AtMostSeqCard, Reified constraint, alldifferent, cycle &  & CHIP, Ilog Solver, Mistral, OPL, Claire & automotive, rectangle-packing &  & github, benchmark, random instance, Roadef, real-world, CSPlib & conflict-driven clause learning, evolutionary computing, lazy clause generation, time-tabling, large neighborhood search, edge-finding, ant colony, GRASP, swarm intelligence & \ref{a:Siala15a} & n/a\\
\rowlabel{b:Zahout21}\href{../works/Zahout21.pdf}{Zahout21}~\cite{Zahout21} & 185 & completion-time, machine, job, activity, Pareto, online scheduling, release-date, make-span, multi-agent, distributed, resource, energy efficiency, multi-objective, job-shop, flow-shop, precedence, bi-objective, preempt, due-date, re-scheduling, task, preemptive, scheduling & CuSP, parallel machine, RCPSP, SCC, TCSP, single machine & cycle, cumulative, circuit, bin-packing &  & CPO, Cplex, Claire & datacenter, business process, high performance computing, crew-scheduling, satellite &  & benchmark & meta heuristic, reinforcement learning, GRASP, genetic algorithm, column generation & \ref{a:Zahout21} & n/a\\
\end{longtable}
}




\clearpage
\subsection{InBook from bibtex}
{\scriptsize
\begin{longtable}{>{\raggedright\arraybackslash}p{2.5cm}>{\raggedright\arraybackslash}p{4.5cm}>{\raggedright\arraybackslash}p{6.0cm}p{1.0cm}rr>{\raggedright\arraybackslash}p{2.0cm}r>{\raggedright\arraybackslash}p{1cm}p{1cm}p{1cm}p{1cm}}
\rowcolor{white}\caption{INBOOK (Total 46)}\\ \toprule
\rowcolor{white}\shortstack{Key\\Source} & Authors & Title (Colored by Open Access)& \shortstack{Details\\LC} & Cite & Year & \shortstack{Conference\\/Journal\\/School} & Pages & Relevance &\shortstack{Cites\\OC XR\\SC} & \shortstack{Refs\\OC\\XR} & \shortstack{Links\\Cites\\Refs}\\ \midrule\endhead
\bottomrule
\endfoot
\index{Marcolini2022}\rowlabel{a:Marcolini2022}Marcolini2022 \href{http://dx.doi.org/10.1016/b978-0-323-85159-6.50083-x}{Marcolini2022} & \hyperref[auth:a2042]{L. D. Marcolini}, \hyperref[auth:a586]{F. M. Novara}, \hyperref[auth:a587]{G. P. Henning} & Production scheduling in multiproduct multistage semicontinuous processes. A constraint programming approach & \cellcolor{red!30}\hyperref[detail:Marcolini2022]{Details} No & \cite{Marcolini2022} & 2022 & Computer Aided Chemical Engineering & null & \noindent{}\textbf{1.00} \textbf{1.00} n/a & 0 0 0 & 4 5 & 1 0 1\\
\index{Sitek2018}\rowlabel{a:Sitek2018}Sitek2018 \href{http://dx.doi.org/10.1007/978-3-319-90287-6_8}{Sitek2018} & \hyperref[auth:a1474]{P. Sitek}, \hyperref[auth:a534]{J. Wikarek} & An MP/CP-Based Hybrid Approach to Optimization of the Resource-Constrained Scheduling Problems & \hyperref[detail:Sitek2018]{Details} \href{../works/Sitek2018.pdf}{Yes} & \cite{Sitek2018} & 2018 & Transactions on Computational Collective Intelligence XXIX & 19 & \noindent{}\textbf{1.50} \textbf{1.50} \textbf{10.08} & 0 0 0 & 20 24 & 5 0 5\\
\index{Bgler2016}\rowlabel{a:Bgler2016}Bgler2016 \href{http://dx.doi.org/10.4018/978-1-4666-9619-8.ch006}{Bgler2016} & \hyperref[auth:a1542]{M. Bügler}, \hyperref[auth:a1543]{A. Borrmann} & \cellcolor{gold!20}Simulation Based Construction Project Schedule Optimization \hyperref[abs:Bgler2016]{Abstract} & \cellcolor{red!30}\hyperref[detail:Bgler2016]{Details} No & \cite{Bgler2016} & 2016 & Civil and Environmental Engineering & null & \noindent{}\textcolor{black!50}{0.00} \textcolor{black!50}{0.00} n/a & 0 0 0 & 56 71 & 7 0 7\\
\index{Bgler2016a}\rowlabel{a:Bgler2016a}Bgler2016a \href{http://dx.doi.org/10.4018/978-1-4666-8823-0.ch016}{Bgler2016a} & \hyperref[auth:a1542]{M. Bügler}, \hyperref[auth:a1543]{A. Borrmann} & Simulation Based Construction Project Schedule Optimization \hyperref[abs:Bgler2016a]{Abstract} & \cellcolor{red!30}\hyperref[detail:Bgler2016a]{Details} No & \cite{Bgler2016a} & 2016 & Advances in Systems Analysis, Software Engineering, and High Performance Computing & null & \noindent{}\textcolor{black!50}{0.00} \textcolor{black!50}{0.00} n/a & 0 0 1 & 58 71 & 7 0 7\\
\index{Novara2015}\rowlabel{a:Novara2015}Novara2015 \href{http://dx.doi.org/10.1016/b978-0-444-63576-1.50032-7}{Novara2015} & \hyperref[auth:a586]{F. M. Novara}, \hyperref[auth:a587]{G. P. Henning} & A Hybrid CP/MILP Approach for Big Size Scheduling Problems of Multiproduct, Multistage Batch Plants & \cellcolor{red!30}\hyperref[detail:Novara2015]{Details} No & \cite{Novara2015} & 2015 & 12th International Symposium on Process Systems Engineering and 25th European Symposium on Computer Aided Process Engineering & null & \noindent{}\textbf{1.00} \textbf{1.00} n/a & 3 3 4 & 3 4 & 2 0 2\\
\index{SchuttFSW15}\rowlabel{a:SchuttFSW15}SchuttFSW15 \href{https://doi.org/10.1007/978-3-319-05443-8_7}{SchuttFSW15} & \hyperref[auth:a124]{A. Schutt}, \hyperref[auth:a154]{T. Feydy}, \hyperref[auth:a125]{P. J. Stuckey}, \hyperref[auth:a117]{M. G. Wallace} & A Satisfiability Solving Approach \hyperref[abs:SchuttFSW15]{Abstract} & \cellcolor{red!30}\hyperref[detail:SchuttFSW15]{Details} No & \cite{SchuttFSW15} & 2015 & Handbook on Project Management and Scheduling Vol.1 & 26 & \noindent{}\textcolor{black!50}{0.00} \textbf{1.50} n/a & 3 4 6 & 28 41 & 24 3 21\\
\index{CestaOPS14}\rowlabel{a:CestaOPS14}CestaOPS14 \href{http://dx.doi.org/10.1007/978-3-319-05443-8_6}{CestaOPS14} & \hyperref[auth:a284]{A. Cesta}, \hyperref[auth:a282]{A. Oddi}, \hyperref[auth:a283]{N. Policella}, \hyperref[auth:a298]{S. F. Smith} & A Precedence Constraint Posting Approach & \cellcolor{red!30}\hyperref[detail:CestaOPS14]{Details} No & \cite{CestaOPS14} & 2014 & Handbook on Project Management and Scheduling Vol.1 & 21 & \noindent{}\textcolor{black!50}{0.00} \textcolor{black!50}{0.00} n/a & 2 2 3 & 17 40 & 11 0 11\\
\index{Gaspero2014}\rowlabel{a:Gaspero2014}Gaspero2014 \href{http://dx.doi.org/10.1007/978-3-319-07644-7_1}{Gaspero2014} & \hyperref[auth:a2040]{L. D. Gaspero}, \hyperref[auth:a2041]{T. Urli} & A CP/LNS Approach for Multi-day Homecare Scheduling Problems & \hyperref[detail:Gaspero2014]{Details} \href{../works/Gaspero2014.pdf}{Yes} & \cite{Gaspero2014} & 2014 & Hybrid Metaheuristics & 15 & \noindent{}\textbf{1.00} \textbf{1.00} \textbf{1.92} & 5 5 13 & 12 18 & 2 0 2\\
\index{GuSSWC14}\rowlabel{a:GuSSWC14}GuSSWC14 \href{http://dx.doi.org/10.1007/978-3-319-05443-8_14}{GuSSWC14} & \hyperref[auth:a336]{H. Gu}, \hyperref[auth:a124]{A. Schutt}, \hyperref[auth:a125]{P. J. Stuckey}, \hyperref[auth:a117]{M. G. Wallace}, \hyperref[auth:a343]{G. Chu} & Exact and Heuristic Methods for the Resource-Constrained Net Present Value Problem & \cellcolor{red!30}\hyperref[detail:GuSSWC14]{Details} No & \cite{GuSSWC14} & 2014 & Handbook on Project Management and Scheduling Vol.1 & 20 & \noindent{}\textcolor{black!50}{0.00} \textcolor{black!50}{0.00} n/a & 5 6 7 & 35 39 & 13 1 12\\
\index{Moukrim2014}\rowlabel{a:Moukrim2014}Moukrim2014 \href{http://dx.doi.org/10.1007/978-3-319-12631-9_6}{Moukrim2014} & \hyperref[auth:a1169]{A. Moukrim}, \hyperref[auth:a788]{A. Quilliot}, \hyperref[auth:a1698]{H. Toussaint} & Branch and Price for Preemptive and Non Preemptive RCPSP Based on Interval Orders on Precedence Graphs & \cellcolor{red!30}\hyperref[detail:Moukrim2014]{Details} No & \cite{Moukrim2014} & 2014 & Recent Advances in Computational Optimization & null & \noindent{}\textcolor{black!50}{0.00} \textcolor{black!50}{0.00} n/a & 1 1 1 & 15 25 & 5 0 5\\
\index{Ortiz-Bayliss2014}\rowlabel{a:Ortiz-Bayliss2014}Ortiz-Bayliss2014 \href{http://dx.doi.org/10.1007/978-3-319-01692-4_25}{Ortiz-Bayliss2014} & \hyperref[auth:a1778]{J. C. Ortiz-Bayliss}, \hyperref[auth:a1606]{H. Terashima-Marín}, \hyperref[auth:a1779]{S. E. Conant-Pablos} & \cellcolor{green!10}Branching Schemes and Variable Ordering Heuristics for Constraint Satisfaction Problems: Is There Something to Learn? & \cellcolor{red!30}\hyperref[detail:Ortiz-Bayliss2014]{Details} No & \cite{Ortiz-Bayliss2014} & 2014 & Nature Inspired Cooperative Strategies for Optimization (NICSO 2013) & null & \noindent{}0.50 0.50 n/a & 0 0 0 & 24 40 & 5 0 5\\
\index{Austrin2013}\rowlabel{a:Austrin2013}Austrin2013 \href{http://dx.doi.org/10.1007/978-3-642-40328-6_3}{Austrin2013} & \hyperref[auth:a1926]{P. Austrin}, \hyperref[auth:a1927]{R. Manokaran}, \hyperref[auth:a1928]{C. Wenner} & \cellcolor{green!10}On the NP-Hardness of Approximating Ordering Constraint Satisfaction Problems & \hyperref[detail:Austrin2013]{Details} \href{../works/Austrin2013.pdf}{Yes} & \cite{Austrin2013} & 2013 & Approximation, Randomization, and Combinatorial Optimization. Algorithms and Techniques & 16 & \noindent{}0.50 0.50 0.35 & 1 1 5 & 21 23 & 1 1 0\\
\index{Guimarans2013}\rowlabel{a:Guimarans2013}Guimarans2013 \href{http://dx.doi.org/10.4018/978-1-4666-2461-0.ch007}{Guimarans2013} & \hyperref[auth:a1837]{D. Guimarans}, \hyperref[auth:a1838]{R. Herrero}, \hyperref[auth:a1839]{J. J. Ramos}, \hyperref[auth:a1840]{S. Padrón} & Solving Vehicle Routing Problems Using Constraint Programming and Lagrangean Relaxation in a Metaheuristics Framework \hyperref[abs:Guimarans2013]{Abstract} & \cellcolor{red!30}\hyperref[detail:Guimarans2013]{Details} No & \cite{Guimarans2013} & 2013 & Management Innovations for Intelligent Supply Chains & null & \noindent{}\textcolor{black!50}{0.00} 0.50 n/a & 1 1 0 & 21 32 & 1 1 0\\
\index{Novara2013}\rowlabel{a:Novara2013}Novara2013 \href{http://dx.doi.org/10.1016/b978-0-444-63234-0.50099-3}{Novara2013} & \hyperref[auth:a586]{F. M. Novara}, \hyperref[auth:a523]{J. M. Novas}, \hyperref[auth:a587]{G. P. Henning} & A comprehensive CP approach for the scheduling of resource-constrained multiproduct multistage batch plants & \cellcolor{red!30}\hyperref[detail:Novara2013]{Details} No & \cite{Novara2013} & 2013 & Computer Aided Chemical Engineering & null & \noindent{}\textbf{1.50} \textbf{1.50} n/a & 3 3 6 & 4 5 & 2 1 1\\
\index{Talbi2013a}\rowlabel{a:Talbi2013a}Talbi2013a \href{http://dx.doi.org/10.1007/978-3-642-30671-6_1}{Talbi2013a} & \hyperref[auth:a1657]{E.-G. Talbi} & A Unified Taxonomy of Hybrid Metaheuristics with Mathematical Programming, Constraint Programming and Machine Learning & \cellcolor{red!30}\hyperref[detail:Talbi2013a]{Details} No & \cite{Talbi2013a} & 2013 & Hybrid Metaheuristics & null & \noindent{}0.50 0.50 n/a & 15 15 32 & 107 183 & 7 0 7\\
\index{Laborie2011}\rowlabel{a:Laborie2011}Laborie2011 \href{http://dx.doi.org/10.1007/978-3-642-23592-4_6}{Laborie2011} & \hyperref[auth:a118]{P. Laborie}, \hyperref[auth:a1673]{J. Rogerie}, \hyperref[auth:a120]{P. Shaw}, \hyperref[auth:a1674]{P. Vilím}, \hyperref[auth:a1675]{F. Katai} & Interval-Based Language for Modeling Scheduling Problems: An Extension to Constraint Programming & \cellcolor{red!30}\hyperref[detail:Laborie2011]{Details} No & \cite{Laborie2011} & 2011 & Algebraic Modeling Systems & null & \noindent{}\textbf{1.00} \textbf{1.00} n/a & 2 2 0 & 4 20 & 3 1 2\\
\index{Milano11}\rowlabel{a:Milano11}Milano11 \href{http://dx.doi.org/10.1002/9780470400531.eorms0473}{Milano11} & \hyperref[auth:a143]{M. Milano} & Constraint Programming Links with Math Programming & \cellcolor{red!30}\hyperref[detail:Milano11]{Details} No & \cite{Milano11} & 2011 & Wiley Encyclopedia of Operations Research and Management Science & null & \noindent{}\textcolor{black!50}{0.00} \textcolor{black!50}{0.00} n/a & 0 0 0 & 28 65 & 15 0 15\\
\index{Triska2011}\rowlabel{a:Triska2011}Triska2011 \href{http://dx.doi.org/10.1007/978-3-642-21332-8_12}{Triska2011} & \hyperref[auth:a1843]{M. Triska}, \hyperref[auth:a45]{N. Musliu} & A Constraint Programming Application for Rotating Workforce Scheduling & \cellcolor{red!30}\hyperref[detail:Triska2011]{Details} No & \cite{Triska2011} & 2011 & Developing Concepts in Applied Intelligence & null & \noindent{}\textbf{1.00} \textbf{1.00} n/a & 3 3 5 & 8 11 & 2 1 1\\
\index{CastroGR10}\rowlabel{a:CastroGR10}CastroGR10 \href{http://dx.doi.org/10.1007/978-1-4419-1644-0_4}{CastroGR10} & \hyperref[auth:a890]{P. M. Castro}, \hyperref[auth:a382]{I. E. Grossmann}, \hyperref[auth:a326]{L.-M. Rousseau} & Decomposition Techniques for Hybrid MILP/CP Models applied to Scheduling and Routing Problems & \cellcolor{red!30}\hyperref[detail:CastroGR10]{Details} No & \cite{CastroGR10} & 2010 & Hybrid Optimization & 33 & \noindent{}\textbf{1.00} \textbf{1.00} n/a & 0 0 4 & 67 88 & 30 0 30\\
\index{Hooker10}\rowlabel{a:Hooker10}Hooker10 \href{http://dx.doi.org/10.1007/978-1-4419-1644-0_2}{Hooker10} & \hyperref[auth:a160]{J. N. Hooker} & Hybrid Modeling & \cellcolor{red!30}\hyperref[detail:Hooker10]{Details} No & \cite{Hooker10} & 2010 & Hybrid Optimization & 52 & \noindent{}\textcolor{black!50}{0.00} \textcolor{black!50}{0.00} n/a & 9 9 12 & 39 65 & 12 0 12\\
\index{GongLMW09}\rowlabel{a:GongLMW09}GongLMW09 \href{http://dx.doi.org/10.1007/978-0-387-88617-6_11}{GongLMW09} & \hyperref[auth:a1232]{J. Gong}, \hyperref[auth:a1233]{E. E. Lee}, \hyperref[auth:a1234]{J. E. Mitchell}, \hyperref[auth:a1235]{W. A. Wallace} & Logic-based MultiObjective Optimization for Restoration Planning & \cellcolor{red!30}\hyperref[detail:GongLMW09]{Details} No & \cite{GongLMW09} & 2009 & Optimization and Logistics Challenges in the Enterprise & 20 & \noindent{}\textcolor{black!50}{0.00} \textcolor{black!50}{0.00} n/a & 14 14 30 & 13 20 & 10 1 9\\
\index{AggounMV08}\rowlabel{a:AggounMV08}AggounMV08 \href{http://dx.doi.org/10.1007/978-0-387-74759-0_396}{AggounMV08} & \hyperref[auth:a724]{A. Aggoun}, \hyperref[auth:a381]{C. T. Maravelias}, \hyperref[auth:a906]{A. Vazacopoulos} & Mixed Integer Programming/Constraint Programming Hybrid Methods & \cellcolor{red!30}\hyperref[detail:AggounMV08]{Details} No & \cite{AggounMV08} & 2008 & Encyclopedia of Optimization & 7 & \noindent{}\textcolor{black!50}{0.00} \textcolor{black!50}{0.00} n/a & 0 0 0 & 34 53 & 19 0 19\\
\index{2007}\rowlabel{a:2007}2007 \href{http://dx.doi.org/10.1007/978-3-540-32220-7_13}{2007} &  & Constraint Programming and Disjunctive Scheduling & \cellcolor{red!30}\hyperref[detail:2007]{Details} No & \cite{2007} & 2007 & International Handbook on Information Systems & null & \noindent{}\textbf{1.00} \textbf{1.00} n/a & 0 0 0 & 37 63 & 11 0 11\\
\index{Hooker06a}\rowlabel{a:Hooker06a}Hooker06a \href{http://dx.doi.org/10.1016/s1574-6526(06)80019-2}{Hooker06a} & \hyperref[auth:a160]{J. N. Hooker} & Operations Research Methods in Constraint Programming & \cellcolor{red!30}\hyperref[detail:Hooker06a]{Details} No & \cite{Hooker06a} & 2006 & Foundations of Artificial Intelligence & 44 & \noindent{}\textcolor{black!50}{0.00} \textcolor{black!50}{0.00} n/a & 11 11 13 & 44 133 & 16 5 11\\
\index{NeronABCDD06}\rowlabel{a:NeronABCDD06}NeronABCDD06 \href{http://dx.doi.org/10.1007/978-0-387-33768-5_7}{NeronABCDD06} & \hyperref[auth:a898]{E. Néron}, \hyperref[auth:a6]{C. Artigues}, \hyperref[auth:a162]{P. Baptiste}, \hyperref[auth:a844]{J. Carlier}, \hyperref[auth:a899]{J. Damay}, \hyperref[auth:a243]{S. Demassey}, \hyperref[auth:a118]{P. Laborie} & Lower Bounds for Resource Constrained Project Scheduling Problem & \cellcolor{red!30}\hyperref[detail:NeronABCDD06]{Details} No & \cite{NeronABCDD06} & 2006 & Perspectives in Modern Project Scheduling & 38 & \noindent{}\textcolor{black!50}{0.00} \textcolor{black!50}{0.00} n/a & 3 3 0 & 34 49 & 15 1 14\\
\index{Xing2006}\rowlabel{a:Xing2006}Xing2006 \href{http://dx.doi.org/10.1007/11760191_135}{Xing2006} & \hyperref[auth:a1983]{L.-N. Xing}, \hyperref[auth:a1984]{Y.-W. Chen}, \hyperref[auth:a1985]{X.-S. Shen} & A Constraint Satisfaction Adaptive Neural Network with Dynamic Model for Job-Shop Scheduling Problem & \hyperref[detail:Xing2006]{Details} \href{../works/Xing2006.pdf}{Yes} & \cite{Xing2006} & 2006 & Advances in Neural Networks - ISNN 2006 & 6 & \noindent{}\textbf{2.00} \textbf{2.00} \textcolor{black!50}{0.14} & 3 3 11 & 3 5 & 2 0 2\\
\index{Zeballos2006}\rowlabel{a:Zeballos2006}Zeballos2006 \href{http://dx.doi.org/10.1016/s1570-7946(06)80335-4}{Zeballos2006} & \hyperref[auth:a620]{L. J. Zeballos}, \hyperref[auth:a587]{G. P. Henning} & A CP method for the scheduling of multiproduct continuous plants with resource constraints & \cellcolor{red!30}\hyperref[detail:Zeballos2006]{Details} No & \cite{Zeballos2006} & 2006 & 16th European Symposium on Computer Aided Process Engineering and 9th International Symposium on Process Systems Engineering & null & \noindent{}\textbf{1.50} \textbf{1.50} n/a & 1 1 1 & 4 7 & 3 1 2\\
\index{Bartak2005}\rowlabel{a:Bartak2005}Bartak2005 \href{http://dx.doi.org/10.4018/978-1-59140-450-7.ch010}{Bartak2005} & \hyperref[auth:a1480]{R. Bartak} & Constraint Satisfaction for Planning and Scheduling \hyperref[abs:Bartak2005]{Abstract} & \cellcolor{red!30}\hyperref[detail:Bartak2005]{Details} No & \cite{Bartak2005} & 2005 & Intelligent Techniques for Planning & null & \noindent{}\textbf{1.00} \textbf{3.75} n/a & 3 3 0 & 0 0 & 2 2 0\\
\index{Vazacopoulos2005}\rowlabel{a:Vazacopoulos2005}Vazacopoulos2005 \href{http://dx.doi.org/10.1007/0-387-26281-4_12}{Vazacopoulos2005} & \hyperref[auth:a906]{A. Vazacopoulos}, \hyperref[auth:a1560]{N. Verma} & Hybrid MIP-CP Techniques to Solve a Multi-Machine Assignment and Scheduling Problem in Xpress-CP & \cellcolor{red!30}\hyperref[detail:Vazacopoulos2005]{Details} No & \cite{Vazacopoulos2005} & 2005 & Applied Optimization & null & \noindent{}\textbf{1.50} \textbf{1.50} n/a & 3 3 0 & 8 16 & 6 1 5\\
\index{WolfS05a}\rowlabel{a:WolfS05a}WolfS05a \href{http://dx.doi.org/10.1007/11415763_12}{WolfS05a} & \hyperref[auth:a51]{A. Wolf}, \hyperref[auth:a710]{H. Schlenker} & Realising the Alternative Resources Constraint & \hyperref[detail:WolfS05a]{Details} \href{../works/WolfS05a.pdf}{Yes} & \cite{WolfS05a} & 2005 & Applications of Declarative Programming and Knowledge Management & 15 & \noindent{}\textcolor{black!50}{0.00} \textcolor{black!50}{0.00} \textbf{1.44} & 5 5 5 & 6 8 & 8 3 5\\
\index{AggounV04}\rowlabel{a:AggounV04}AggounV04 \href{http://dx.doi.org/10.1007/978-3-540-24734-0_15}{AggounV04} & \hyperref[auth:a724]{A. Aggoun}, \hyperref[auth:a906]{A. Vazacopoulos} & Solving Sports Scheduling and Timetabling Problems with Constraint Programming & \cellcolor{red!30}\hyperref[detail:AggounV04]{Details} No & \cite{AggounV04} & 2004 & Economics, Management and Optimization in Sports & 22 & \noindent{}\textbf{1.00} \textbf{1.00} n/a & 7 7 0 & 4 18 & 7 4 3\\
\index{AjiliW04}\rowlabel{a:AjiliW04}AjiliW04 \href{http://dx.doi.org/10.1007/978-1-4419-8917-8_6}{AjiliW04} & \hyperref[auth:a948]{F. Ajili}, \hyperref[auth:a117]{M. G. Wallace} & Hybrid Problem Solving in ECLiPSe & \cellcolor{red!30}\hyperref[detail:AjiliW04]{Details} No & \cite{AjiliW04} & 2004 & Constraint and Integer Programming & 38 & \noindent{}\textcolor{black!50}{0.00} \textcolor{black!50}{0.00} n/a & 4 4 0 & 24 42 & 13 1 12\\
\index{DannaP04}\rowlabel{a:DannaP04}DannaP04 \href{http://dx.doi.org/10.1007/978-1-4419-8917-8_2}{DannaP04} & \hyperref[auth:a287]{E. Danna}, \hyperref[auth:a163]{C. L. Pape} & Two Generic Schemes for Efficient and Robust Cooperative Algorithms & \cellcolor{red!30}\hyperref[detail:DannaP04]{Details} No & \cite{DannaP04} & 2004 & Constraints and Integer Programming & 25 & \noindent{}\textcolor{black!50}{0.00} \textcolor{black!50}{0.00} n/a & 2 2 0 & 34 63 & 18 0 18\\
\index{DomdorfPH03}\rowlabel{a:DomdorfPH03}DomdorfPH03 \href{http://dx.doi.org/10.1007/978-3-642-18965-4_31}{DomdorfPH03} & \hyperref[auth:a958]{U. Domdorf}, \hyperref[auth:a437]{E. Pesch}, \hyperref[auth:a959]{T. P. Huy} & Machine Learning by Schedule Decomposition — Prospects for an Integration of AI and OR Techniques for Job Shop Scheduling & \cellcolor{red!30}\hyperref[detail:DomdorfPH03]{Details} No & \cite{DomdorfPH03} & 2003 & Advances in Evolutionary Computing & 26 & \noindent{}\textcolor{black!50}{0.00} \textcolor{black!50}{0.00} n/a & 0 0 0 & 57 96 & 14 0 14\\
\index{Roe2003}\rowlabel{a:Roe2003}Roe2003 \href{http://dx.doi.org/10.1016/s1570-7946(03)80608-9}{Roe2003} & \hyperref[auth:a1240]{B. Roe}, \hyperref[auth:a1242]{N. Shah}, \hyperref[auth:a1241]{L. G. Papageorgiou} & A hybrid CLP and MILP approach to batch process scheduling & \cellcolor{red!30}\hyperref[detail:Roe2003]{Details} No & \cite{Roe2003} & 2003 & Computer Aided Chemical Engineering & null & \noindent{}\textbf{1.00} \textbf{1.00} n/a & 0 1 2 & 2 5 & 1 0 1\\
\index{Timpe2003}\rowlabel{a:Timpe2003}Timpe2003 \href{http://dx.doi.org/10.1007/978-3-662-05607-3_5}{Timpe2003} & \hyperref[auth:a672]{C. Timpe} & Solving planning and scheduling problems with combined integer and constraint programming & \cellcolor{red!30}\hyperref[detail:Timpe2003]{Details} No & \cite{Timpe2003} & 2003 & Advanced Planning and Scheduling Solutions in Process Industry & null & \noindent{}\textbf{1.00} \textbf{1.00} n/a & 2 2 0 & 9 15 & 4 0 4\\
\index{Baptiste2001}\rowlabel{a:Baptiste2001}Baptiste2001 \href{http://dx.doi.org/10.1007/978-1-4615-1479-4_2}{Baptiste2001} & \hyperref[auth:a162]{P. Baptiste}, \hyperref[auth:a163]{C. L. Pape}, \hyperref[auth:a655]{W. Nuijten} & Propagation of the One-Machine Resource Constraint & \cellcolor{red!30}\hyperref[detail:Baptiste2001]{Details} No & \cite{Baptiste2001} & 2001 & International Series in Operations Research \  Management Science & null & \noindent{}0.50 0.50 n/a & 1 1 0 & 0 0 & 1 1 0\\
\index{Galipienso2001}\rowlabel{a:Galipienso2001}Galipienso2001 \href{http://dx.doi.org/10.1007/3-540-45517-5_63}{Galipienso2001} & \hyperref[auth:a1875]{M. I. A. Galipienso}, \hyperref[auth:a1876]{F. B. Sanchís} & A Mixed Closure-CSP Method to Solve Scheduling Problems & \hyperref[detail:Galipienso2001]{Details} \href{../works/Galipienso2001.pdf}{Yes} & \cite{Galipienso2001} & 2001 & Engineering of Intelligent Systems & 12 & \noindent{}\textbf{1.00} \textbf{1.00} \textbf{1.86} & 1 1 3 & 7 16 & 3 0 3\\
\index{Harjunkoski2001}\rowlabel{a:Harjunkoski2001}Harjunkoski2001 \href{http://dx.doi.org/10.1016/s1570-7946(01)80140-1}{Harjunkoski2001} & \hyperref[auth:a870]{I. Harjunkoski}, \hyperref[auth:a382]{I. E. Grossmann} & Combined MILP-constraint programming approach for the optimal scheduling of multistage batch processes & \cellcolor{red!30}\hyperref[detail:Harjunkoski2001]{Details} No & \cite{Harjunkoski2001} & 2001 & Computer Aided Chemical Engineering & null & \noindent{}\textbf{1.00} \textbf{1.00} n/a & 2 2 2 & 3 9 & 2 2 0\\
\index{Rgin2001}\rowlabel{a:Rgin2001}Rgin2001 \href{http://dx.doi.org/10.1090/dimacs/057/07}{Rgin2001} & \hyperref[auth:a1419]{J.-C. Régin} & Minimization of the number of breaks in sports scheduling problems using constraint programming & \cellcolor{red!30}\hyperref[detail:Rgin2001]{Details} No & \cite{Rgin2001} & 2001 & DIMACS Series in Discrete Mathematics and Theoretical Computer Science & 16 & \noindent{}\textbf{1.00} \textbf{1.00} n/a & 28 29 0 & 0 0 & 11 11 0\\
\index{DorndorfHP99}\rowlabel{a:DorndorfHP99}DorndorfHP99 \href{http://dx.doi.org/10.1007/978-1-4615-5533-9_10}{DorndorfHP99} & \hyperref[auth:a903]{U. Dorndorf}, \hyperref[auth:a904]{T. P. Huy}, \hyperref[auth:a437]{E. Pesch} & A Survey of Interval Capacity Consistency Tests for Time- and Resource-Constrained Scheduling & \cellcolor{red!30}\hyperref[detail:DorndorfHP99]{Details} No & \cite{DorndorfHP99} & 1999 & Project Scheduling & 26 & \noindent{}\textcolor{black!50}{0.00} \textcolor{black!50}{0.00} n/a & 18 18 0 & 20 40 & 18 5 13\\
\index{Jaffar1998}\rowlabel{a:Jaffar1998}Jaffar1998 \href{http://dx.doi.org/10.1093/oso/9780198537922.003.0012}{Jaffar1998} & \hyperref[auth:a1066]{J. Jaffar}, \hyperref[auth:a1067]{M. J. Maher} & Constraint Logic Programming: A Survey \hyperref[abs:Jaffar1998]{Abstract} & \cellcolor{red!30}\hyperref[detail:Jaffar1998]{Details} No & \cite{Jaffar1998} & 1998 & Handbook of Logic in Artificial Intelligence and Logic Programming: Volume 5: Logic Programming & null & \noindent{}\textcolor{black!50}{0.00} \textbf{1.00} n/a & 3 3 0 & 0 0 & 1 1 0\\
\index{Larrosa1998}\rowlabel{a:Larrosa1998}Larrosa1998 \href{http://dx.doi.org/10.1007/3-540-64574-8_390}{Larrosa1998} & \hyperref[auth:a1791]{J. Larrosa}, \hyperref[auth:a1792]{P. Meseguer} & \cellcolor{green!10}Generic CSP techniques for the job-shop problem & \cellcolor{red!30}\hyperref[detail:Larrosa1998]{Details} No & \cite{Larrosa1998} & 1998 & Tasks and Methods in Applied Artificial Intelligence & null & \noindent{}\textbf{1.00} \textbf{1.00} n/a & 2 2 4 & 3 13 & 2 0 2\\
\index{Mesghouni1997}\rowlabel{a:Mesghouni1997}Mesghouni1997 \href{http://dx.doi.org/10.1007/978-0-387-35086-8_12}{Mesghouni1997} & \hyperref[auth:a1906]{K. Mesghouni}, \hyperref[auth:a1457]{P. Pesin}, \hyperref[auth:a1907]{S. Hammadi}, \hyperref[auth:a1458]{C. Tahon}, \hyperref[auth:a1908]{P. Borne} & Genetic Algorithms — Constraint Logic Programming. Hybrid Method for Job Shop Scheduling & \cellcolor{red!30}\hyperref[detail:Mesghouni1997]{Details} No & \cite{Mesghouni1997} & 1997 & Re-engineering for Sustainable Industrial Production & null & \noindent{}\textbf{2.00} \textbf{2.00} n/a & 2 2 0 & 1 11 & 1 1 0\\
\index{Caseau1996}\rowlabel{a:Caseau1996}Caseau1996 \href{http://dx.doi.org/10.1007/3-540-61576-8_79}{Caseau1996} & \hyperref[auth:a301]{Y. Caseau}, \hyperref[auth:a1511]{F. Laburthe} & Improving branch and bound for Jobshop scheduling with constraint propagation & \cellcolor{red!30}\hyperref[detail:Caseau1996]{Details} No & \cite{Caseau1996} & 1996 & Combinatorics and Computer Science & null & \noindent{}\textbf{3.00} \textbf{3.00} n/a & 11 11 10 & 12 23 & 11 6 5\\
\index{Breitinger1994}\rowlabel{a:Breitinger1994}Breitinger1994 \href{http://dx.doi.org/10.1007/3-540-58402-1_20}{Breitinger1994} & \hyperref[auth:a694]{S. Breitinger}, \hyperref[auth:a695]{H. C. R. Lock} & Improving search for job-shop scheduling with CLP(FD) & \cellcolor{red!30}\hyperref[detail:Breitinger1994]{Details} No & \cite{Breitinger1994} & 1994 & Programming Language Implementation and Logic Programming & null & \noindent{}\textbf{2.00} \textbf{2.00} n/a & 3 3 3 & 6 17 & 2 1 1\\
\end{longtable}
}



\clearpage
\subsection{InCollection from bibtex}
{\scriptsize
\begin{longtable}{>{\raggedright\arraybackslash}p{3cm}>{\raggedright\arraybackslash}p{6cm}>{\raggedright\arraybackslash}p{6.5cm}rrrp{2.5cm}rrrrr}
\rowcolor{white}\caption{Works from bibtex (Total 7)}\\ \toprule
\rowcolor{white}Key & Authors & Title & LC & Cite & Year & \shortstack{Conference\\/Journal} & Pages & \shortstack{Nr\\Cites} & \shortstack{Nr\\Refs} & b & c \\ \midrule\endhead
\bottomrule
\endfoot
\rowlabel{a:BlazewiczEP19}BlazewiczEP19 \href{https://ideas.repec.org/h/spr/ihichp/978-3-319-99849-7_16.html}{BlazewiczEP19} & \hyperref[auth:a774]{J. Blazewicz}, \hyperref[auth:a775]{Klaus H. Ecker}, \hyperref[auth:a443]{E. Pesch}, \hyperref[auth:a776]{G. Schmidt}, \hyperref[auth:a777]{M. Sterna}, \hyperref[auth:a778]{J. Weglarz} & {Constraint Programming and Disjunctive Scheduling} & No & \cite{BlazewiczEP19} & 2019 & {Handbook on Scheduling} & 62 & 38 & 0 & No & \ref{c:BlazewiczEP19}\\
\rowlabel{a:Hooker19}Hooker19 \href{https://ideas.repec.org/h/spr/spochp/978-3-030-22788-3_1.html}{Hooker19} & \hyperref[auth:a161]{John N. Hooker} & {Logic-Based Benders Decomposition for Large-Scale Optimization} & No & \cite{Hooker19} & 2019 & {Large Scale Optimization in Supply Chains and Smart Manufacturing} & 26 & 8 & 0 & No & \ref{c:Hooker19}\\
\rowlabel{a:HurleyOS16}HurleyOS16 \href{https://doi.org/10.1007/978-3-319-50137-6\_15}{HurleyOS16} & \hyperref[auth:a902]{B. Hurley}, \hyperref[auth:a16]{B. O'Sullivan}, \hyperref[auth:a17]{H. Simonis} & {ICON} Loop Energy Show Case & \href{works/HurleyOS16.pdf}{Yes} & \cite{HurleyOS16} & 2016 & Data Mining and Constraint Programming - Foundations of a Cross-Disciplinary Approach & 14 & 0 & 16 & \ref{b:HurleyOS16} & \ref{c:HurleyOS16}\\
\rowlabel{a:Bartak14}Bartak14 \href{}{Bartak14} & \hyperref[auth:a152]{R. Bart{\'{a}}k} & Planning and Scheduling & No & \cite{Bartak14} & 2014 & Computing Handbook, Third Edition: Computer Science and Software Engineering & null & 0 & 0 & No & \ref{c:Bartak14}\\
\rowlabel{a:BaptisteLPN06}BaptisteLPN06 \href{https://doi.org/10.1016/S1574-6526(06)80026-X}{BaptisteLPN06} & \hyperref[auth:a163]{P. Baptiste}, \hyperref[auth:a118]{P. Laborie}, \hyperref[auth:a164]{Claude Le Pape}, \hyperref[auth:a666]{W. Nuijten} & Constraint-Based Scheduling and Planning & No & \cite{BaptisteLPN06} & 2006 & Handbook of Constraint Programming & 39 & 30 & 25 & No & \ref{c:BaptisteLPN06}\\
\rowlabel{a:KanetAG04}KanetAG04 \href{http://www.crcnetbase.com/doi/abs/10.1201/9780203489802.ch47}{KanetAG04} & \hyperref[auth:a672]{John J. Kanet}, \hyperref[auth:a673]{S. Ahire}, \hyperref[auth:a674]{Michael F. Gorman} & Constraint Programming for Scheduling & No & \cite{KanetAG04} & 2004 & Handbook of Scheduling - Algorithms, Models, and Performance Analysis & null & 0 & 0 & No & \ref{c:KanetAG04}\\
\rowlabel{a:BreitingerL95}BreitingerL95 \href{}{BreitingerL95} & \hyperref[auth:a705]{S. Breitinger}, \hyperref[auth:a706]{Hendrik C. R. Lock} & Using Constraint Logic Programming for Industrial Scheduling Problems & No & \cite{BreitingerL95} & 1995 & Logic Programming: Formal Methods and Practical Applications, Studies in Computer Science and Artificial Intelligence & 27 & 0 & 0 & No & \ref{c:BreitingerL95}\\
\end{longtable}
}




% \subsection{CSPLib}

% {\scriptsize
% \begin{longtable}{rlp{8cm}lcccll}
% \caption{\label{tab:csplib}CSPLib scheduling problems}\\ \toprule
% Nr & Name & Description & CP System & Data & Code & Solutions & Classification & Constraints\\ \midrule
% \endfirsthead
% \caption{CSPLib scheduling problems}\\ \toprule
% Nr & Name & Description & CP System & Data & Code & Solutions & Classification & Constraints\\ \midrule
% \endhead
% \bottomrule
% \endfoot
% 59 & \href{https://www.csplib.org/Problems/prob059/}{Energy Cost Aware Scheduling} & & - & 50 TXT & - & - & & \\
% 61 & \href{https://www.csplib.org/Problems/prob061/}{RCPSP} & Resource-Constrained Scheduling Problem & PyCSP3 & PSPLIB & y & PSPLIB & RCPSP & \\
% 73 & \href{https://www.csplib.org/Problems/prob073/}{Test Scheduling Problem} & & \su{ECLiPSe OPL} & 840 Prolog & y & & \\
% 77 & \href{https://www.csplib.org/Problems/prob077/}{Stochastic Assignment and Scheduling Problem} & & MiniZinc & 9 DZN & y & & \\
% \end{longtable}
% }





\end{document}


& \href{papers/.pdf}{} & \cite{} & 2019 & CP & & \su{} & & & & & \su{} \\

& \href{articles/.pdf}{} & \cite{} & & & & & & & & & \\
& \href{articles/.pdf}{} & \cite{} & & & & & & & & & \\
& \href{articles/.pdf}{} & \cite{} & & & & & & & & & \\
& \href{articles/.pdf}{} & \cite{} & & & & & & & & & \\
& \href{articles/.pdf}{} & \cite{} & & & & & & & & & \\
& \href{articles/.pdf}{} & \cite{} & & & & & & & & & \\
& \href{articles/.pdf}{} & \cite{} & & & & & & & & & \\
& \href{articles/.pdf}{} & \cite{} & & & & & & & & & \\

& \href{papers/.pdf}{} & \cite{} & & CP & & & & & & & \\

& \href{papers/.pdf}{} & \cite{} &  & CPAIOR & & & & & & & \\
& \href{papers/.pdf}{} & \cite{} &  & CPAIOR & & & & & & & \\
& \href{papers/.pdf}{} & \cite{} &  & CPAIOR & & & & & & & \\
& \href{papers/.pdf}{} & \cite{} &  & CPAIOR & & & & & & & \\
& \href{papers/.pdf}{} & \cite{} &  & CPAIOR & & & & & & & \\
& \href{papers/.pdf}{} & \cite{} &  & CPAIOR & & & & & & & \\
& \href{papers/.pdf}{} & \cite{} &  & CPAIOR & & & & & & & \\
& \href{papers/.pdf}{} & \cite{} &  & CPAIOR & & & & & & & \\
& \href{papers/.pdf}{} & \cite{} &  & CPAIOR & & & & & & & \\
& \href{papers/.pdf}{} & \cite{} &  & CPAIOR & & & & & & & \\
& \href{papers/.pdf}{} & \cite{} &  & CPAIOR & & & & & & & \\
& \href{papers/.pdf}{} & \cite{} &  & CPAIOR & & & & & & & \\
& \href{papers/.pdf}{} & \cite{} &  & CPAIOR & & & & & & & \\
