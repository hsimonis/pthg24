\PassOptionsToPackage{table}{xcolor}
\documentclass[a4paper]{article}
\usepackage[a4paper,margin=2cm,landscape]{geometry}
\usepackage{tabularx}
\usepackage{tikz}
\usepackage{graphicx}
\usepackage{rotating}
\usepackage{float}
\usepackage{calc}
\usepackage{pdflscape}
\usepackage{booktabs}
\usepackage{colortbl}
\usepackage{longtable}
\usepackage{stackengine}
\usepackage{multicol}
%\usepackage{showkeys}
\newcounter{rowcounter}
\newcommand{\rowlabel}[1]{\refstepcounter{rowcounter}\label{#1}}

\usepackage{url}
\usepackage{hyperref}

\newcommand{\su}[1]{\Shortunderstack[l]{#1}}

\title{CP Papers on Scheduling}
\author{Helmut Simonis and Cemalettin Öztürk}
\begin{document}
\rowcolors{2}{gray!20}{white}

\maketitle
\section{Introduction}

This document shows the result of a survey on "Constraint Programming and Scheduling", which tries to find and classify all publications on the combination of these two concepts. It is based on a manually collected bibfile containing reference to relevant papers and articles, and on an automatic and manual analysis of local copies of the cited papers. For copyright reasons, we are obviously not able to distribute the collected copies, but we provide links to the original sources of the files. 

We identify the papers by a key which is the last name of the first author, the first character of the last names of all other authors, and a two digit year code for the date of publication. If multiple works would define the same key, we differentiate by adding a suffix "a", "b", etc, to the second and subsequent works found.

Most of the content of this document is generated by a Java program that parses the bib files, adds any manually extracted information, and which then extracts concept occurrences from the local copies of the works. It then produces tables and other LaTeX  artifacts that are included in a manually defined top-level document.

To add new works, first add bibtex entries for each work in the main \texttt{overview/bib.bib} file, then add local copies of the pdf of the work to the \texttt{overview/works/} directory, using the key of the bibtex entry as the file name (plus extension .pdf), and then run the main Java program \texttt{org.insightcentre.pthg24.JfxApp} to consolidate the information and extract the relevant concepts. Finally, run \texttt{pdflatex} on the \texttt{overview/scheduling.tex} file to produce this pdf document. Manually extracted information for the files can be added in the \texttt{imports/manual.csv} file. New concepts can be added in the file \texttt{imports/concepts.json}, new concept types need to be directly defined in the Java code.

We start the document by providing a table of all defined keys in the bib file in alphabetical order. This table can be helpful to see if a candidate paper is already in the survey, it suffices to see if the key is already present, and matches the authors, title and origin of the candidate paper. In the table link given by the key points to the local copy of the file, while the citation number links to the bibliography entry. That entry typically also contains a link to the original source of the paper.

This document heavily depends on the use of hyper links in the document, it has been tested with Acrobat Reader, other pdf reader may not use links in the same way. 

\clearpage
\begin{longtable}{*{6}{l}}
\rowcolor{white}\caption{Key Overview (Total: 892)}\\ \toprule
\rowcolor{white}1 & 2 & 3 & 4 & 5 & 6\\ \midrule
\endhead
\bottomrule
\endfoot
\href{../works/AalianPG23.pdf}{AalianPG23}~\cite{AalianPG23} & \href{../works/AbdennadherS99.pdf}{AbdennadherS99}~\cite{AbdennadherS99} & \href{../works/AbidinK20.pdf}{AbidinK20}~\cite{AbidinK20} & \href{../works/AbohashimaEG21.pdf}{AbohashimaEG21}~\cite{AbohashimaEG21} & \href{../works/AbreuAPNM21.pdf}{AbreuAPNM21}~\cite{AbreuAPNM21} & \href{../works/AbreuN22.pdf}{AbreuN22}~\cite{AbreuN22}\\ 
\href{../works/AbreuNP23.pdf}{AbreuNP23}~\cite{AbreuNP23} & \href{../works/AbreuPNF23.pdf}{AbreuPNF23}~\cite{AbreuPNF23} & \href{../works/AbrilSB05.pdf}{AbrilSB05}~\cite{AbrilSB05} & \href{../works/AchterbergBKW08.pdf}{AchterbergBKW08}~\cite{AchterbergBKW08} & \href{../works/Acuna-AgostMFG09.pdf}{Acuna-AgostMFG09}~\cite{Acuna-AgostMFG09} & \href{../works/Adelgren2023.pdf}{Adelgren2023}~\cite{Adelgren2023}\\ 
\href{../works/AfsarVPG23.pdf}{AfsarVPG23}~\cite{AfsarVPG23} & \href{../works/AggounB93.pdf}{AggounB93}~\cite{AggounB93} & \href{../}{AggounMV08}~\cite{AggounMV08} & \href{../}{AggounV04}~\cite{AggounV04} & \href{../works/AgussurjaKL18.pdf}{AgussurjaKL18}~\cite{AgussurjaKL18} & \href{../}{AjiliW04}~\cite{AjiliW04}\\ 
\href{../works/AkkerDH07.pdf}{AkkerDH07}~\cite{AkkerDH07} & \href{../works/AkramNHRSA23.pdf}{AkramNHRSA23}~\cite{AkramNHRSA23} & \href{../works/Alaka21.pdf}{Alaka21}~\cite{Alaka21} & \href{../works/AlakaP23.pdf}{AlakaP23}~\cite{AlakaP23} & \href{../works/AlakaPY19.pdf}{AlakaPY19}~\cite{AlakaPY19} & \href{../works/AlesioBNG15.pdf}{AlesioBNG15}~\cite{AlesioBNG15}\\ 
\href{../works/AlesioNBG14.pdf}{AlesioNBG14}~\cite{AlesioNBG14} & \href{../works/AlfieriGPS23.pdf}{AlfieriGPS23}~\cite{AlfieriGPS23} & \href{../}{AlizdehS20}~\cite{AlizdehS20} & \href{../works/AmadiniGM16.pdf}{AmadiniGM16}~\cite{AmadiniGM16} & \href{../works/AngelsmarkJ00.pdf}{AngelsmarkJ00}~\cite{AngelsmarkJ00} & \href{../works/AntunesABD18.pdf}{AntunesABD18}~\cite{AntunesABD18}\\ 
\href{../works/AntunesABD20.pdf}{AntunesABD20}~\cite{AntunesABD20} & \href{../works/AntuoriHHEN20.pdf}{AntuoriHHEN20}~\cite{AntuoriHHEN20} & \href{../works/AntuoriHHEN21.pdf}{AntuoriHHEN21}~\cite{AntuoriHHEN21} & \href{../works/ArbaouiY18.pdf}{ArbaouiY18}~\cite{ArbaouiY18} & \href{../}{Arkhipov19}~\cite{Arkhipov19} & \href{../works/ArkhipovBL19.pdf}{ArkhipovBL19}~\cite{ArkhipovBL19}\\ 
\href{../works/ArmstrongGOS21.pdf}{ArmstrongGOS21}~\cite{ArmstrongGOS21} & \href{../works/ArmstrongGOS22.pdf}{ArmstrongGOS22}~\cite{ArmstrongGOS22} & \href{../works/AronssonBK09.pdf}{AronssonBK09}~\cite{AronssonBK09} & \href{../works/ArtiguesBF04.pdf}{ArtiguesBF04}~\cite{ArtiguesBF04} & \href{../}{ArtiguesDN08}~\cite{ArtiguesDN08} & \href{../works/ArtiguesF07.pdf}{ArtiguesF07}~\cite{ArtiguesF07}\\ 
\href{../works/ArtiguesHQT21.pdf}{ArtiguesHQT21}~\cite{ArtiguesHQT21} & \href{../works/ArtiguesL14.pdf}{ArtiguesL14}~\cite{ArtiguesL14} & \href{../works/ArtiguesLH13.pdf}{ArtiguesLH13}~\cite{ArtiguesLH13} & \href{../works/ArtiguesR00.pdf}{ArtiguesR00}~\cite{ArtiguesR00} & \href{../works/ArtiouchineB05.pdf}{ArtiouchineB05}~\cite{ArtiouchineB05} & \href{../works/Astrand0F21.pdf}{Astrand0F21}~\cite{Astrand0F21}\\ 
\href{../works/Astrand21.pdf}{Astrand21}~\cite{Astrand21} & \href{../works/AstrandJZ18.pdf}{AstrandJZ18}~\cite{AstrandJZ18} & \href{../works/AstrandJZ20.pdf}{AstrandJZ20}~\cite{AstrandJZ20} & \href{../works/AwadMDMT22.pdf}{AwadMDMT22}~\cite{AwadMDMT22} & \href{../works/BadicaBI20.pdf}{BadicaBI20}~\cite{BadicaBI20} & \href{../works/BadicaBIL19.pdf}{BadicaBIL19}~\cite{BadicaBIL19}\\ 
\href{../works/BajestaniB11.pdf}{BajestaniB11}~\cite{BajestaniB11} & \href{../works/BajestaniB13.pdf}{BajestaniB13}~\cite{BajestaniB13} & \href{../works/BajestaniB15.pdf}{BajestaniB15}~\cite{BajestaniB15} & \href{../works/Balduccini11.pdf}{Balduccini11}~\cite{Balduccini11} & \href{../}{BalochG20}~\cite{BalochG20} & \href{../works/BandaSC11.pdf}{BandaSC11}~\cite{BandaSC11}\\ 
\href{../works/Baptiste02.pdf}{Baptiste02}~\cite{Baptiste02} & \href{../works/Baptiste09.pdf}{Baptiste09}~\cite{Baptiste09} & \href{../works/BaptisteB18.pdf}{BaptisteB18}~\cite{BaptisteB18} & \href{../}{BaptisteLPN06}~\cite{BaptisteLPN06} & \href{../works/BaptisteLV92.pdf}{BaptisteLV92}~\cite{BaptisteLV92} & \href{../works/BaptisteP00.pdf}{BaptisteP00}~\cite{BaptisteP00}\\ 
\href{../works/BaptisteP95.pdf}{BaptisteP95}~\cite{BaptisteP95} & \href{../works/BaptisteP97.pdf}{BaptisteP97}~\cite{BaptisteP97} & \href{../}{BaptistePN01}~\cite{BaptistePN01} & \href{../works/BaptistePN99.pdf}{BaptistePN99}~\cite{BaptistePN99} & \href{../works/BarbulescuWH04.pdf}{BarbulescuWH04}~\cite{BarbulescuWH04} & \href{../works/BarlattCG08.pdf}{BarlattCG08}~\cite{BarlattCG08}\\ 
\href{../works/Bartak02.pdf}{Bartak02}~\cite{Bartak02} & \href{../works/Bartak02a.pdf}{Bartak02a}~\cite{Bartak02a} & \href{../}{Bartak14}~\cite{Bartak14} & \href{../works/BartakCS10.pdf}{BartakCS10}~\cite{BartakCS10} & \href{../works/BartakS11.pdf}{BartakS11}~\cite{BartakS11} & \href{../works/BartakSR08.pdf}{BartakSR08}~\cite{BartakSR08}\\ 
\href{../works/BartakSR10.pdf}{BartakSR10}~\cite{BartakSR10} & \href{../works/BartakV15.pdf}{BartakV15}~\cite{BartakV15} & \href{../works/BartoliniBBLM14.pdf}{BartoliniBBLM14}~\cite{BartoliniBBLM14} & \href{../works/BarzegaranZP20.pdf}{BarzegaranZP20}~\cite{BarzegaranZP20} & \href{../works/Beck06.pdf}{Beck06}~\cite{Beck06} & \href{../works/Beck07.pdf}{Beck07}~\cite{Beck07}\\ 
\href{../works/Beck10.pdf}{Beck10}~\cite{Beck10} & \href{../works/Beck99.pdf}{Beck99}~\cite{Beck99} & \href{../works/BeckDDF98.pdf}{BeckDDF98}~\cite{BeckDDF98} & \href{../works/BeckDF97.pdf}{BeckDF97}~\cite{BeckDF97} & \href{../works/BeckDSF97.pdf}{BeckDSF97}~\cite{BeckDSF97} & \href{../works/BeckDSF97a.pdf}{BeckDSF97a}~\cite{BeckDSF97a}\\ 
\href{../works/BeckF00.pdf}{BeckF00}~\cite{BeckF00} & \href{../works/BeckF00a.pdf}{BeckF00a}~\cite{BeckF00a} & \href{../works/BeckF98.pdf}{BeckF98}~\cite{BeckF98} & \href{../works/BeckF99.pdf}{BeckF99}~\cite{BeckF99} & \href{../works/BeckFW11.pdf}{BeckFW11}~\cite{BeckFW11} & \href{../works/BeckPS03.pdf}{BeckPS03}~\cite{BeckPS03}\\ 
\href{../works/BeckR03.pdf}{BeckR03}~\cite{BeckR03} & \href{../works/BeckW04.pdf}{BeckW04}~\cite{BeckW04} & \href{../works/BeckW05.pdf}{BeckW05}~\cite{BeckW05} & \href{../works/BeckW07.pdf}{BeckW07}~\cite{BeckW07} & \href{../works/Bedhief21.pdf}{Bedhief21}~\cite{Bedhief21} & \href{../works/BegB13.pdf}{BegB13}~\cite{BegB13}\\ 
\href{../works/BehrensLM19.pdf}{BehrensLM19}~\cite{BehrensLM19} & \href{../works/BeldiceanuC01.pdf}{BeldiceanuC01}~\cite{BeldiceanuC01} & \href{../works/BeldiceanuC02.pdf}{BeldiceanuC02}~\cite{BeldiceanuC02} & \href{../works/BeldiceanuC94.pdf}{BeldiceanuC94}~\cite{BeldiceanuC94} & \href{../works/BeldiceanuCDP11.pdf}{BeldiceanuCDP11}~\cite{BeldiceanuCDP11} & \href{../works/BeldiceanuCP08.pdf}{BeldiceanuCP08}~\cite{BeldiceanuCP08}\\ 
\href{../works/BeldiceanuP07.pdf}{BeldiceanuP07}~\cite{BeldiceanuP07} & \href{../works/BelhadjiI98.pdf}{BelhadjiI98}~\cite{BelhadjiI98} & \href{../works/BenderWS21.pdf}{BenderWS21}~\cite{BenderWS21} & \href{../works/BenediktMH20.pdf}{BenediktMH20}~\cite{BenediktMH20} & \href{../works/BenediktSMVH18.pdf}{BenediktSMVH18}~\cite{BenediktSMVH18} & \href{../works/BeniniBGM05.pdf}{BeniniBGM05}~\cite{BeniniBGM05}\\ 
\href{../works/BeniniBGM05a.pdf}{BeniniBGM05a}~\cite{BeniniBGM05a} & \href{../works/BeniniBGM06.pdf}{BeniniBGM06}~\cite{BeniniBGM06} & \href{../works/BeniniLMMR08.pdf}{BeniniLMMR08}~\cite{BeniniLMMR08} & \href{../works/BeniniLMR08.pdf}{BeniniLMR08}~\cite{BeniniLMR08} & \href{../works/BeniniLMR11.pdf}{BeniniLMR11}~\cite{BeniniLMR11} & \href{../works/BenoistGR02.pdf}{BenoistGR02}~\cite{BenoistGR02}\\ 
\href{../works/BensanaLV99.pdf}{BensanaLV99}~\cite{BensanaLV99} & \href{../works/BertholdHLMS10.pdf}{BertholdHLMS10}~\cite{BertholdHLMS10} & \href{../works/BessiereHMQW14.pdf}{BessiereHMQW14}~\cite{BessiereHMQW14} & \href{../works/BhatnagarKL19.pdf}{BhatnagarKL19}~\cite{BhatnagarKL19} & \href{../works/BidotVLB07.pdf}{BidotVLB07}~\cite{BidotVLB07} & \href{../works/BidotVLB09.pdf}{BidotVLB09}~\cite{BidotVLB09}\\ 
\href{../works/BillautHL12.pdf}{BillautHL12}~\cite{BillautHL12} & \href{../works/Bit-Monnot23.pdf}{Bit-Monnot23}~\cite{Bit-Monnot23} & \href{../works/BlazewiczDP96.pdf}{BlazewiczDP96}~\cite{BlazewiczDP96} & \href{../}{BlazewiczEP19}~\cite{BlazewiczEP19} & \href{../works/BlomBPS14.pdf}{BlomBPS14}~\cite{BlomBPS14} & \href{../works/BlomPS16.pdf}{BlomPS16}~\cite{BlomPS16}\\ 
\href{../works/BocewiczBB09.pdf}{BocewiczBB09}~\cite{BocewiczBB09} & \href{../}{BockmayrK98}~\cite{BockmayrK98} & \href{../works/BockmayrP06.pdf}{BockmayrP06}~\cite{BockmayrP06} & \href{../works/BofillCGGPSV23.pdf}{BofillCGGPSV23}~\cite{BofillCGGPSV23} & \href{../works/BofillCSV17.pdf}{BofillCSV17}~\cite{BofillCSV17} & \href{../works/BofillCSV17a.pdf}{BofillCSV17a}~\cite{BofillCSV17a}\\ 
\href{../works/BofillEGPSV14.pdf}{BofillEGPSV14}~\cite{BofillEGPSV14} & \href{../works/BofillGSV15.pdf}{BofillGSV15}~\cite{BofillGSV15} & \href{../works/BogaerdtW19.pdf}{BogaerdtW19}~\cite{BogaerdtW19} & \href{../works/Bonfietti16.pdf}{Bonfietti16}~\cite{Bonfietti16} & \href{../works/BonfiettiLBM11.pdf}{BonfiettiLBM11}~\cite{BonfiettiLBM11} & \href{../works/BonfiettiLBM12.pdf}{BonfiettiLBM12}~\cite{BonfiettiLBM12}\\ 
\href{../works/BonfiettiLBM14.pdf}{BonfiettiLBM14}~\cite{BonfiettiLBM14} & \href{../works/BonfiettiLM13.pdf}{BonfiettiLM13}~\cite{BonfiettiLM13} & \href{../works/BonfiettiLM14.pdf}{BonfiettiLM14}~\cite{BonfiettiLM14} & \href{../works/BonfiettiM12.pdf}{BonfiettiM12}~\cite{BonfiettiM12} & \href{../works/BonfiettiZLM16.pdf}{BonfiettiZLM16}~\cite{BonfiettiZLM16} & \href{../works/BonninMNE24.pdf}{BonninMNE24}~\cite{BonninMNE24}\\ 
\href{../works/BoothNB16.pdf}{BoothNB16}~\cite{BoothNB16} & \href{../works/BoothTNB16.pdf}{BoothTNB16}~\cite{BoothTNB16} & \href{../works/BorghesiBLMB18.pdf}{BorghesiBLMB18}~\cite{BorghesiBLMB18} & \href{../works/BosiM2001.pdf}{BosiM2001}~\cite{BosiM2001} & \href{../}{BoucherBVBL97}~\cite{BoucherBVBL97} & \href{../works/BoudreaultSLQ22.pdf}{BoudreaultSLQ22}~\cite{BoudreaultSLQ22}\\ 
\href{../works/BourdaisGP03.pdf}{BourdaisGP03}~\cite{BourdaisGP03} & \href{../works/BourreauGGLT22.pdf}{BourreauGGLT22}~\cite{BourreauGGLT22} & \href{../}{BreitingerL95}~\cite{BreitingerL95} & \href{../}{BriandHHL08}~\cite{BriandHHL08} & \href{../works/BridiBLMB16.pdf}{BridiBLMB16}~\cite{BridiBLMB16} & \href{../works/BridiLBBM16.pdf}{BridiLBBM16}~\cite{BridiLBBM16}\\ 
\href{../works/BruckerK00.pdf}{BruckerK00}~\cite{BruckerK00} & \href{../works/BrusoniCLMMT96.pdf}{BrusoniCLMMT96}~\cite{BrusoniCLMMT96} & \href{../works/BukchinR18.pdf}{BukchinR18}~\cite{BukchinR18} & \href{../works/BulckG22.pdf}{BulckG22}~\cite{BulckG22} & \href{../works/BurtLPS15.pdf}{BurtLPS15}~\cite{BurtLPS15} & \href{../works/Caballero19.pdf}{Caballero19}~\cite{Caballero19}\\ 
\href{../works/Caballero23.pdf}{Caballero23}~\cite{Caballero23} & \href{../works/CambazardHDJT04.pdf}{CambazardHDJT04}~\cite{CambazardHDJT04} & \href{../works/CambazardJ05.pdf}{CambazardJ05}~\cite{CambazardJ05} & \href{../works/CampeauG22.pdf}{CampeauG22}~\cite{CampeauG22} & \href{../works/CappartS17.pdf}{CappartS17}~\cite{CappartS17} & \href{../works/CappartTSR18.pdf}{CappartTSR18}~\cite{CappartTSR18}\\ 
\href{../works/CarchraeB09.pdf}{CarchraeB09}~\cite{CarchraeB09} & \href{../works/CarchraeBF05.pdf}{CarchraeBF05}~\cite{CarchraeBF05} & \href{../works/CarlierPSJ20.pdf}{CarlierPSJ20}~\cite{CarlierPSJ20} & \href{../}{CarlierSJP21}~\cite{CarlierSJP21} & \href{../works/CarlssonJL17.pdf}{CarlssonJL17}~\cite{CarlssonJL17} & \href{../works/CarlssonKA99.pdf}{CarlssonKA99}~\cite{CarlssonKA99}\\ 
\href{../works/Caseau97.pdf}{Caseau97}~\cite{Caseau97} & \href{../}{CastroGR10}~\cite{CastroGR10} & \href{../works/CatusseCBL16.pdf}{CatusseCBL16}~\cite{CatusseCBL16} & \href{../works/CauwelaertDMS16.pdf}{CauwelaertDMS16}~\cite{CauwelaertDMS16} & \href{../works/CauwelaertDS20.pdf}{CauwelaertDS20}~\cite{CauwelaertDS20} & \href{../works/CauwelaertLS15.pdf}{CauwelaertLS15}~\cite{CauwelaertLS15}\\ 
\href{../works/CauwelaertLS18.pdf}{CauwelaertLS18}~\cite{CauwelaertLS18} & \href{../works/CestaOF99.pdf}{CestaOF99}~\cite{CestaOF99} & \href{../}{CestaOPS14}~\cite{CestaOPS14} & \href{../works/CestaOS00.pdf}{CestaOS00}~\cite{CestaOS00} & \href{../works/CestaOS98.pdf}{CestaOS98}~\cite{CestaOS98} & \href{../works/ChapadosJR11.pdf}{ChapadosJR11}~\cite{ChapadosJR11}\\ 
\href{../works/ChenGPSH10.pdf}{ChenGPSH10}~\cite{ChenGPSH10} & \href{../works/ChuGNSW13.pdf}{ChuGNSW13}~\cite{ChuGNSW13} & \href{../works/ChuX05.pdf}{ChuX05}~\cite{ChuX05} & \href{../works/ChunCTY99.pdf}{ChunCTY99}~\cite{ChunCTY99} & \href{../works/ChunS14.pdf}{ChunS14}~\cite{ChunS14} & \href{../works/CilKLO22.pdf}{CilKLO22}~\cite{CilKLO22}\\ 
\href{../works/CireCH13.pdf}{CireCH13}~\cite{CireCH13} & \href{../works/CireCH16.pdf}{CireCH16}~\cite{CireCH16} & \href{../works/ClautiauxJCM08.pdf}{ClautiauxJCM08}~\cite{ClautiauxJCM08} & \href{../works/Clercq12.pdf}{Clercq12}~\cite{Clercq12} & \href{../works/ClercqPBJ11.pdf}{ClercqPBJ11}~\cite{ClercqPBJ11} & \href{../works/CobanH10.pdf}{CobanH10}~\cite{CobanH10}\\ 
\href{../works/CobanH11.pdf}{CobanH11}~\cite{CobanH11} & \href{../works/CohenHB17.pdf}{CohenHB17}~\cite{CohenHB17} & \href{../works/ColT19.pdf}{ColT19}~\cite{ColT19} & \href{../works/ColT2019a.pdf}{ColT2019a}~\cite{ColT2019a} & \href{../works/ColT22.pdf}{ColT22}~\cite{ColT22} & \href{../works/Colombani96.pdf}{Colombani96}~\cite{Colombani96}\\ 
\href{../works/CorreaLR07.pdf}{CorreaLR07}~\cite{CorreaLR07} & \href{../works/CrawfordB94.pdf}{CrawfordB94}~\cite{CrawfordB94} & \href{../works/CzerniachowskaWZ23.pdf}{CzerniachowskaWZ23}~\cite{CzerniachowskaWZ23} & \href{../works/DannaP03.pdf}{DannaP03}~\cite{DannaP03} & \href{../}{DannaP04}~\cite{DannaP04} & \href{../works/Darby-DowmanLMZ97.pdf}{Darby-DowmanLMZ97}~\cite{Darby-DowmanLMZ97}\\ 
\href{../}{DarbyDowmanL98}~\cite{DarbyDowmanL98} & \href{../works/Davenport10.pdf}{Davenport10}~\cite{Davenport10} & \href{../works/DavenportKRSH07.pdf}{DavenportKRSH07}~\cite{DavenportKRSH07} & \href{../works/Davis87.pdf}{Davis87}~\cite{Davis87} & \href{../works/Dejemeppe16.pdf}{Dejemeppe16}~\cite{Dejemeppe16} & \href{../works/DejemeppeCS15.pdf}{DejemeppeCS15}~\cite{DejemeppeCS15}\\ 
\href{../works/DejemeppeD14.pdf}{DejemeppeD14}~\cite{DejemeppeD14} & \href{../works/Demassey03.pdf}{Demassey03}~\cite{Demassey03} & \href{../works/DemasseyAM05.pdf}{DemasseyAM05}~\cite{DemasseyAM05} & \href{../works/DemirovicS18.pdf}{DemirovicS18}~\cite{DemirovicS18} & \href{../works/Derrien15.pdf}{Derrien15}~\cite{Derrien15} & \href{../works/DerrienP14.pdf}{DerrienP14}~\cite{DerrienP14}\\ 
\href{../works/DerrienPZ14.pdf}{DerrienPZ14}~\cite{DerrienPZ14} & \href{../works/DilkinaDH05.pdf}{DilkinaDH05}~\cite{DilkinaDH05} & \href{../works/DilkinaH04.pdf}{DilkinaH04}~\cite{DilkinaH04} & \href{../works/DincbasS91.pdf}{DincbasS91}~\cite{DincbasS91} & \href{../works/DincbasSH90.pdf}{DincbasSH90}~\cite{DincbasSH90} & \href{../works/DoRZ08.pdf}{DoRZ08}~\cite{DoRZ08}\\ 
\href{../}{DomdorfPH03}~\cite{DomdorfPH03} & \href{../works/DoomsH08.pdf}{DoomsH08}~\cite{DoomsH08} & \href{../works/Dorndorf2000.pdf}{Dorndorf2000}~\cite{Dorndorf2000} & \href{../}{DorndorfHP99}~\cite{DorndorfHP99} & \href{../}{DorndorfPH99}~\cite{DorndorfPH99} & \href{../works/DoulabiRP14.pdf}{DoulabiRP14}~\cite{DoulabiRP14}\\ 
\href{../works/DoulabiRP16.pdf}{DoulabiRP16}~\cite{DoulabiRP16} & \href{../works/DraperJCJ99.pdf}{DraperJCJ99}~\cite{DraperJCJ99} & \href{../works/EastonNT02.pdf}{EastonNT02}~\cite{EastonNT02} & \href{../works/Edis21.pdf}{Edis21}~\cite{Edis21} & \href{../works/EdisO11.pdf}{EdisO11}~\cite{EdisO11} & \href{../}{EdisO11a}~\cite{EdisO11a}\\ 
\href{../}{EdwardsBSE19}~\cite{EdwardsBSE19} & \href{../works/EfthymiouY23.pdf}{EfthymiouY23}~\cite{EfthymiouY23} & \href{../works/ElciOH22.pdf}{ElciOH22}~\cite{ElciOH22} & \href{../works/ElfJR03.pdf}{ElfJR03}~\cite{ElfJR03} & \href{../works/ElhouraniDM07.pdf}{ElhouraniDM07}~\cite{ElhouraniDM07} & \href{../works/Elkhyari03.pdf}{Elkhyari03}~\cite{Elkhyari03}\\ 
\href{../works/ElkhyariGJ02.pdf}{ElkhyariGJ02}~\cite{ElkhyariGJ02} & \href{../works/ElkhyariGJ02a.pdf}{ElkhyariGJ02a}~\cite{ElkhyariGJ02a} & \href{../works/EmdeZD22.pdf}{EmdeZD22}~\cite{EmdeZD22} & \href{../works/EmeretlisTAV17.pdf}{EmeretlisTAV17}~\cite{EmeretlisTAV17} & \href{../works/EreminW01.pdf}{EreminW01}~\cite{EreminW01} & \href{../works/ErkingerM17.pdf}{ErkingerM17}~\cite{ErkingerM17}\\ 
\href{../works/ErtlK91.pdf}{ErtlK91}~\cite{ErtlK91} & \href{../works/EscobetPQPRA19.pdf}{EscobetPQPRA19}~\cite{EscobetPQPRA19} & \href{../works/EskeyZ90.pdf}{EskeyZ90}~\cite{EskeyZ90} & \href{../}{EsquirolLH2008}~\cite{EsquirolLH2008} & \href{../works/EtminaniesfahaniGNMS22.pdf}{EtminaniesfahaniGNMS22}~\cite{EtminaniesfahaniGNMS22} & \href{../works/EvenSH15.pdf}{EvenSH15}~\cite{EvenSH15}\\ 
\href{../works/EvenSH15a.pdf}{EvenSH15a}~\cite{EvenSH15a} & \href{../works/FachiniA20.pdf}{FachiniA20}~\cite{FachiniA20} & \href{../works/Fahimi16.pdf}{Fahimi16}~\cite{Fahimi16} & \href{../works/FahimiOQ18.pdf}{FahimiOQ18}~\cite{FahimiOQ18} & \href{../}{FahimiQ23}~\cite{FahimiQ23} & \href{../works/FalaschiGMP97.pdf}{FalaschiGMP97}~\cite{FalaschiGMP97}\\ 
\href{../works/FallahiAC20.pdf}{FallahiAC20}~\cite{FallahiAC20} & \href{../works/FalqueALM24.pdf}{FalqueALM24}~\cite{FalqueALM24} & \href{../works/FanXG21.pdf}{FanXG21}~\cite{FanXG21} & \href{../works/FarsiTM22.pdf}{FarsiTM22}~\cite{FarsiTM22} & \href{../works/Fatemi-AnarakiTFV23.pdf}{Fatemi-AnarakiTFV23}~\cite{Fatemi-AnarakiTFV23} & \href{../works/FeldmanG89.pdf}{FeldmanG89}~\cite{FeldmanG89}\\ 
\href{../}{FelizariAL09}~\cite{FelizariAL09} & \href{../works/FetgoD22.pdf}{FetgoD22}~\cite{FetgoD22} & \href{../works/FocacciLN00.pdf}{FocacciLN00}~\cite{FocacciLN00} & \href{../works/FontaineMH16.pdf}{FontaineMH16}~\cite{FontaineMH16} & \href{../works/ForbesHJST24.pdf}{ForbesHJST24}~\cite{ForbesHJST24} & \href{../works/FortinZDF05.pdf}{FortinZDF05}~\cite{FortinZDF05}\\ 
\href{../works/FoxAS82.pdf}{FoxAS82}~\cite{FoxAS82} & \href{../works/FoxS90.pdf}{FoxS90}~\cite{FoxS90} & \href{../works/FrankDT16.pdf}{FrankDT16}~\cite{FrankDT16} & \href{../works/FrankK03.pdf}{FrankK03}~\cite{FrankK03} & \href{../works/FrankK05.pdf}{FrankK05}~\cite{FrankK05} & \href{../}{FriedrichFMRSST14}~\cite{FriedrichFMRSST14}\\ 
\href{../works/FrimodigECM23.pdf}{FrimodigECM23}~\cite{FrimodigECM23} & \href{../works/FrimodigS19.pdf}{FrimodigS19}~\cite{FrimodigS19} & \href{../works/Froger16.pdf}{Froger16}~\cite{Froger16} & \href{../works/FrohnerTR19.pdf}{FrohnerTR19}~\cite{FrohnerTR19} & \href{../works/FrostD98.pdf}{FrostD98}~\cite{FrostD98} & \href{../works/FukunagaHFAMN02.pdf}{FukunagaHFAMN02}~\cite{FukunagaHFAMN02}\\ 
\href{../works/GalleguillosKSB19.pdf}{GalleguillosKSB19}~\cite{GalleguillosKSB19} & \href{../works/GarganiR07.pdf}{GarganiR07}~\cite{GarganiR07} & \href{../works/GarridoAO09.pdf}{GarridoAO09}~\cite{GarridoAO09} & \href{../works/GarridoOS08.pdf}{GarridoOS08}~\cite{GarridoOS08} & \href{../works/GayHLS15.pdf}{GayHLS15}~\cite{GayHLS15} & \href{../works/GayHS15.pdf}{GayHS15}~\cite{GayHS15}\\ 
\href{../works/GayHS15a.pdf}{GayHS15a}~\cite{GayHS15a} & \href{../works/GaySS14.pdf}{GaySS14}~\cite{GaySS14} & \href{../works/GedikKBR17.pdf}{GedikKBR17}~\cite{GedikKBR17} & \href{../works/GedikKEK18.pdf}{GedikKEK18}~\cite{GedikKEK18} & \href{../works/GeibingerKKMMW21.pdf}{GeibingerKKMMW21}~\cite{GeibingerKKMMW21} & \href{../works/GeibingerMM19.pdf}{GeibingerMM19}~\cite{GeibingerMM19}\\ 
\href{../works/GeibingerMM21.pdf}{GeibingerMM21}~\cite{GeibingerMM21} & \href{../works/GeitzGSSW22.pdf}{GeitzGSSW22}~\cite{GeitzGSSW22} & \href{../works/GelainPRVW17.pdf}{GelainPRVW17}~\cite{GelainPRVW17} & \href{../works/German18.pdf}{German18}~\cite{German18} & \href{../works/Geske05.pdf}{Geske05}~\cite{Geske05} & \href{../works/GetoorOFC97.pdf}{GetoorOFC97}~\cite{GetoorOFC97}\\ 
\href{../works/GhandehariK22.pdf}{GhandehariK22}~\cite{GhandehariK22} & \href{../}{GhasemiMH23}~\cite{GhasemiMH23} & \href{../works/GilesH16.pdf}{GilesH16}~\cite{GilesH16} & \href{../works/GingrasQ16.pdf}{GingrasQ16}~\cite{GingrasQ16} & \href{../works/GlobusCLP04.pdf}{GlobusCLP04}~\cite{GlobusCLP04} & \href{../works/GodardLN05.pdf}{GodardLN05}~\cite{GodardLN05}\\ 
\href{../works/Godet21a.pdf}{Godet21a}~\cite{Godet21a} & \href{../works/GodetLHS20.pdf}{GodetLHS20}~\cite{GodetLHS20} & \href{../works/GoelSHFS15.pdf}{GoelSHFS15}~\cite{GoelSHFS15} & \href{../works/GokGSTO20.pdf}{GokGSTO20}~\cite{GokGSTO20} & \href{../works/GokPTGO23.pdf}{GokPTGO23}~\cite{GokPTGO23} & \href{../works/GokgurHO18.pdf}{GokgurHO18}~\cite{GokgurHO18}\\ 
\href{../works/GoldwaserS17.pdf}{GoldwaserS17}~\cite{GoldwaserS17} & \href{../works/GoldwaserS18.pdf}{GoldwaserS18}~\cite{GoldwaserS18} & \href{../works/Goltz95.pdf}{Goltz95}~\cite{Goltz95} & \href{../works/GombolayWS18.pdf}{GombolayWS18}~\cite{GombolayWS18} & \href{../works/GomesHS06.pdf}{GomesHS06}~\cite{GomesHS06} & \href{../works/GomesM17.pdf}{GomesM17}~\cite{GomesM17}\\ 
\href{../}{GongLMW09}~\cite{GongLMW09} & \href{../works/GrimesH10.pdf}{GrimesH10}~\cite{GrimesH10} & \href{../works/GrimesH11.pdf}{GrimesH11}~\cite{GrimesH11} & \href{../works/GrimesH15.pdf}{GrimesH15}~\cite{GrimesH15} & \href{../works/GrimesHM09.pdf}{GrimesHM09}~\cite{GrimesHM09} & \href{../works/GrimesIOS14.pdf}{GrimesIOS14}~\cite{GrimesIOS14}\\ 
\href{../works/Groleaz21.pdf}{Groleaz21}~\cite{Groleaz21} & \href{../works/GroleazNS20.pdf}{GroleazNS20}~\cite{GroleazNS20} & \href{../works/GroleazNS20a.pdf}{GroleazNS20a}~\cite{GroleazNS20a} & \href{../works/Gronkvist06.pdf}{Gronkvist06}~\cite{Gronkvist06} & \href{../works/GruianK98.pdf}{GruianK98}~\cite{GruianK98} & \href{../works/GuSS13.pdf}{GuSS13}~\cite{GuSS13}\\ 
\href{../}{GuSSWC14}~\cite{GuSSWC14} & \href{../works/GuSW12.pdf}{GuSW12}~\cite{GuSW12} & \href{../}{GunerGSKD23}~\cite{GunerGSKD23} & \href{../}{GuoHLW20}~\cite{GuoHLW20} & \href{../works/GuoZ23.pdf}{GuoZ23}~\cite{GuoZ23} & \href{../works/GurEA19.pdf}{GurEA19}~\cite{GurEA19}\\ 
\href{../works/GurPAE23.pdf}{GurPAE23}~\cite{GurPAE23} & \href{../works/GuyonLPR12.pdf}{GuyonLPR12}~\cite{GuyonLPR12} & \href{../works/HachemiGR11.pdf}{HachemiGR11}~\cite{HachemiGR11} & \href{../works/Ham18.pdf}{Ham18}~\cite{Ham18} & \href{../works/Ham18a.pdf}{Ham18a}~\cite{Ham18a} & \href{../}{Ham20}~\cite{Ham20}\\ 
\href{../works/Ham20a.pdf}{Ham20a}~\cite{Ham20a} & \href{../works/HamC16.pdf}{HamC16}~\cite{HamC16} & \href{../works/HamFC17.pdf}{HamFC17}~\cite{HamFC17} & \href{../works/HamP21.pdf}{HamP21}~\cite{HamP21} & \href{../works/HamPK21.pdf}{HamPK21}~\cite{HamPK21} & \href{../works/HamdiL13.pdf}{HamdiL13}~\cite{HamdiL13}\\ 
\href{../works/Hamscher91.pdf}{Hamscher91}~\cite{Hamscher91} & \href{../works/HanenKP21.pdf}{HanenKP21}~\cite{HanenKP21} & \href{../works/HarjunkoskiG02.pdf}{HarjunkoskiG02}~\cite{HarjunkoskiG02} & \href{../works/HarjunkoskiJG00.pdf}{HarjunkoskiJG00}~\cite{HarjunkoskiJG00} & \href{../works/HarjunkoskiMBC14.pdf}{HarjunkoskiMBC14}~\cite{HarjunkoskiMBC14} & \href{../works/HauderBRPA20.pdf}{HauderBRPA20}~\cite{HauderBRPA20}\\ 
\href{../works/He0GLW18.pdf}{He0GLW18}~\cite{He0GLW18} & \href{../works/HebrardALLCMR22.pdf}{HebrardALLCMR22}~\cite{HebrardALLCMR22} & \href{../works/HebrardHJMPV16.pdf}{HebrardHJMPV16}~\cite{HebrardHJMPV16} & \href{../works/HebrardTW05.pdf}{HebrardTW05}~\cite{HebrardTW05} & \href{../works/HechingH16.pdf}{HechingH16}~\cite{HechingH16} & \href{../}{HechingHK19}~\cite{HechingHK19}\\ 
\href{../works/HeckmanB11.pdf}{HeckmanB11}~\cite{HeckmanB11} & \href{../works/HeinzB12.pdf}{HeinzB12}~\cite{HeinzB12} & \href{../works/HeinzKB13.pdf}{HeinzKB13}~\cite{HeinzKB13} & \href{../works/HeinzNVH22.pdf}{HeinzNVH22}~\cite{HeinzNVH22} & \href{../works/HeinzS11.pdf}{HeinzS11}~\cite{HeinzS11} & \href{../works/HeinzSB13.pdf}{HeinzSB13}~\cite{HeinzSB13}\\ 
\href{../works/HeinzSSW12.pdf}{HeinzSSW12}~\cite{HeinzSSW12} & \href{../works/HeipckeCCS00.pdf}{HeipckeCCS00}~\cite{HeipckeCCS00} & \href{../works/HentenryckM04.pdf}{HentenryckM04}~\cite{HentenryckM04} & \href{../works/HentenryckM08.pdf}{HentenryckM08}~\cite{HentenryckM08} & \href{../}{Henz01}~\cite{Henz01} & \href{../works/HenzMT04.pdf}{HenzMT04}~\cite{HenzMT04}\\ 
\href{../works/HermenierDL11.pdf}{HermenierDL11}~\cite{HermenierDL11} & \href{../}{HillBCGN22}~\cite{HillBCGN22} & \href{../works/HillTV21.pdf}{HillTV21}~\cite{HillTV21} & \href{../works/HladikCDJ08.pdf}{HladikCDJ08}~\cite{HladikCDJ08} & \href{../works/HoYCLLCLC18.pdf}{HoYCLLCLC18}~\cite{HoYCLLCLC18} & \href{../works/HoeveGSL07.pdf}{HoeveGSL07}~\cite{HoeveGSL07}\\ 
\href{../}{Hooker00}~\cite{Hooker00} & \href{../}{Hooker02}~\cite{Hooker02} & \href{../works/Hooker04.pdf}{Hooker04}~\cite{Hooker04} & \href{../works/Hooker05.pdf}{Hooker05}~\cite{Hooker05} & \href{../works/Hooker05a.pdf}{Hooker05a}~\cite{Hooker05a} & \href{../works/Hooker05b.pdf}{Hooker05b}~\cite{Hooker05b}\\ 
\href{../works/Hooker06.pdf}{Hooker06}~\cite{Hooker06} & \href{../}{Hooker06a}~\cite{Hooker06a} & \href{../works/Hooker07.pdf}{Hooker07}~\cite{Hooker07} & \href{../}{Hooker10}~\cite{Hooker10} & \href{../works/Hooker17.pdf}{Hooker17}~\cite{Hooker17} & \href{../works/Hooker19.pdf}{Hooker19}~\cite{Hooker19}\\ 
\href{../works/HookerH17.pdf}{HookerH17}~\cite{HookerH17} & \href{../works/HookerO03.pdf}{HookerO03}~\cite{HookerO03} & \href{../works/HookerO99.pdf}{HookerO99}~\cite{HookerO99} & \href{../works/HookerOTK00.pdf}{HookerOTK00}~\cite{HookerOTK00} & \href{../works/HookerY02.pdf}{HookerY02}~\cite{HookerY02} & \href{../works/HoundjiSW19.pdf}{HoundjiSW19}~\cite{HoundjiSW19}\\ 
\href{../works/HoundjiSWD14.pdf}{HoundjiSWD14}~\cite{HoundjiSWD14} & \href{../works/HubnerGSV21.pdf}{HubnerGSV21}~\cite{HubnerGSV21} & \href{../works/Hunsberger08.pdf}{Hunsberger08}~\cite{Hunsberger08} & \href{../works/HurleyOS16.pdf}{HurleyOS16}~\cite{HurleyOS16} & \href{../works/IfrimOS12.pdf}{IfrimOS12}~\cite{IfrimOS12} & \href{../works/IklassovMR023.pdf}{IklassovMR023}~\cite{IklassovMR023}\\ 
\href{../works/IsikYA23.pdf}{IsikYA23}~\cite{IsikYA23} & \href{../works/JainG01.pdf}{JainG01}~\cite{JainG01} & \href{../works/JainM99.pdf}{JainM99}~\cite{JainM99} & \href{../works/Jans09.pdf}{Jans09}~\cite{Jans09} & \href{../works/JelinekB16.pdf}{JelinekB16}~\cite{JelinekB16} & \href{../works/JoLLH99.pdf}{JoLLH99}~\cite{JoLLH99}\\ 
\href{../works/Johnston05.pdf}{Johnston05}~\cite{Johnston05} & \href{../}{JourdanFRD94}~\cite{JourdanFRD94} & \href{../works/JungblutK22.pdf}{JungblutK22}~\cite{JungblutK22} & \href{../works/Junker00.pdf}{Junker00}~\cite{Junker00} & \href{../works/JussienL02.pdf}{JussienL02}~\cite{JussienL02} & \href{../works/JuvinHHL23.pdf}{JuvinHHL23}~\cite{JuvinHHL23}\\ 
\href{../works/JuvinHL22.pdf}{JuvinHL22}~\cite{JuvinHL22} & \href{../works/JuvinHL23.pdf}{JuvinHL23}~\cite{JuvinHL23} & \href{../works/JuvinHL23a.pdf}{JuvinHL23a}~\cite{JuvinHL23a} & \href{../works/KamarainenS02.pdf}{KamarainenS02}~\cite{KamarainenS02} & \href{../works/Kameugne14.pdf}{Kameugne14}~\cite{Kameugne14} & \href{../works/Kameugne15.pdf}{Kameugne15}~\cite{Kameugne15}\\ 
\href{../works/KameugneF13.pdf}{KameugneF13}~\cite{KameugneF13} & \href{../works/KameugneFGOQ18.pdf}{KameugneFGOQ18}~\cite{KameugneFGOQ18} & \href{../works/KameugneFND23.pdf}{KameugneFND23}~\cite{KameugneFND23} & \href{../works/KameugneFSN11.pdf}{KameugneFSN11}~\cite{KameugneFSN11} & \href{../works/KameugneFSN14.pdf}{KameugneFSN14}~\cite{KameugneFSN14} & \href{../works/KanetAG04.pdf}{KanetAG04}~\cite{KanetAG04}\\ 
\href{../works/KelarevaTK13.pdf}{KelarevaTK13}~\cite{KelarevaTK13} & \href{../works/KelbelH11.pdf}{KelbelH11}~\cite{KelbelH11} & \href{../works/KendallKRU10.pdf}{KendallKRU10}~\cite{KendallKRU10} & \href{../works/KengY89.pdf}{KengY89}~\cite{KengY89} & \href{../works/KeriK07.pdf}{KeriK07}~\cite{KeriK07} & \href{../works/KhayatLR06.pdf}{KhayatLR06}~\cite{KhayatLR06}\\ 
\href{../works/KhemmoudjPB06.pdf}{KhemmoudjPB06}~\cite{KhemmoudjPB06} & \href{../works/KimCMLLP23.pdf}{KimCMLLP23}~\cite{KimCMLLP23} & \href{../works/KinsellaS0OS16.pdf}{KinsellaS0OS16}~\cite{KinsellaS0OS16} & \href{../}{KizilayC20}~\cite{KizilayC20} & \href{../works/KlankeBYE21.pdf}{KlankeBYE21}~\cite{KlankeBYE21} & \href{../works/KletzanderM17.pdf}{KletzanderM17}~\cite{KletzanderM17}\\ 
\href{../works/KletzanderM20.pdf}{KletzanderM20}~\cite{KletzanderM20} & \href{../works/KletzanderMH21.pdf}{KletzanderMH21}~\cite{KletzanderMH21} & \href{../works/KoehlerBFFHPSSS21.pdf}{KoehlerBFFHPSSS21}~\cite{KoehlerBFFHPSSS21} & \href{../works/KonowalenkoMM19.pdf}{KonowalenkoMM19}~\cite{KonowalenkoMM19} & \href{../works/KorbaaYG00.pdf}{KorbaaYG00}~\cite{KorbaaYG00} & \href{../works/KorbaaYG99.pdf}{KorbaaYG99}~\cite{KorbaaYG99}\\ 
\href{../works/KoschB14.pdf}{KoschB14}~\cite{KoschB14} & \href{../works/KotaryFH22.pdf}{KotaryFH22}~\cite{KotaryFH22} & \href{../works/KovacsB07.pdf}{KovacsB07}~\cite{KovacsB07} & \href{../works/KovacsB08.pdf}{KovacsB08}~\cite{KovacsB08} & \href{../works/KovacsB11.pdf}{KovacsB11}~\cite{KovacsB11} & \href{../works/KovacsEKV05.pdf}{KovacsEKV05}~\cite{KovacsEKV05}\\ 
\href{../works/KovacsK11.pdf}{KovacsK11}~\cite{KovacsK11} & \href{../works/KovacsTKSG21.pdf}{KovacsTKSG21}~\cite{KovacsTKSG21} & \href{../works/KovacsV04.pdf}{KovacsV04}~\cite{KovacsV04} & \href{../works/KovacsV06.pdf}{KovacsV06}~\cite{KovacsV06} & \href{../works/KreterSS15.pdf}{KreterSS15}~\cite{KreterSS15} & \href{../works/KreterSS17.pdf}{KreterSS17}~\cite{KreterSS17}\\ 
\href{../works/KreterSSZ18.pdf}{KreterSSZ18}~\cite{KreterSSZ18} & \href{../works/KrogtLPHJ07.pdf}{KrogtLPHJ07}~\cite{KrogtLPHJ07} & \href{../works/KuB16.pdf}{KuB16}~\cite{KuB16} & \href{../works/Kuchcinski03.pdf}{Kuchcinski03}~\cite{Kuchcinski03} & \href{../works/KuchcinskiW03.pdf}{KuchcinskiW03}~\cite{KuchcinskiW03} & \href{../works/KucukY19.pdf}{KucukY19}~\cite{KucukY19}\\ 
\href{../works/Kumar03.pdf}{Kumar03}~\cite{Kumar03} & \href{../works/KusterJF07.pdf}{KusterJF07}~\cite{KusterJF07} & \href{../works/Laborie03.pdf}{Laborie03}~\cite{Laborie03} & \href{../works/Laborie05.pdf}{Laborie05}~\cite{Laborie05} & \href{../works/Laborie09.pdf}{Laborie09}~\cite{Laborie09} & \href{../works/Laborie18a.pdf}{Laborie18a}~\cite{Laborie18a}\\ 
\href{../works/LaborieR14.pdf}{LaborieR14}~\cite{LaborieR14} & \href{../works/LaborieRSV18.pdf}{LaborieRSV18}~\cite{LaborieRSV18} & \href{../works/LacknerMMWW21.pdf}{LacknerMMWW21}~\cite{LacknerMMWW21} & \href{../works/LacknerMMWW23.pdf}{LacknerMMWW23}~\cite{LacknerMMWW23} & \href{../works/LahimerLH11.pdf}{LahimerLH11}~\cite{LahimerLH11} & \href{../works/LammaMM97.pdf}{LammaMM97}~\cite{LammaMM97}\\ 
\href{../works/LarsonJC14.pdf}{LarsonJC14}~\cite{LarsonJC14} & \href{../works/LauLN08.pdf}{LauLN08}~\cite{LauLN08} & \href{../works/Layfield02.pdf}{Layfield02}~\cite{Layfield02} & \href{../works/LeeKLKKYHP97.pdf}{LeeKLKKYHP97}~\cite{LeeKLKKYHP97} & \href{../works/Lemos21.pdf}{Lemos21}~\cite{Lemos21} & \href{../works/Letort13.pdf}{Letort13}~\cite{Letort13}\\ 
\href{../works/LetortBC12.pdf}{LetortBC12}~\cite{LetortBC12} & \href{../works/LetortCB13.pdf}{LetortCB13}~\cite{LetortCB13} & \href{../works/LetortCB15.pdf}{LetortCB15}~\cite{LetortCB15} & \href{../works/LiFJZLL22.pdf}{LiFJZLL22}~\cite{LiFJZLL22} & \href{../works/LiLZDZW24.pdf}{LiLZDZW24}~\cite{LiLZDZW24} & \href{../works/LiW08.pdf}{LiW08}~\cite{LiW08}\\ 
\href{../works/LiessM08.pdf}{LiessM08}~\cite{LiessM08} & \href{../works/LimAHO02a.pdf}{LimAHO02a}~\cite{LimAHO02a} & \href{../works/LimBTBB15.pdf}{LimBTBB15}~\cite{LimBTBB15} & \href{../works/LimBTBB15a.pdf}{LimBTBB15a}~\cite{LimBTBB15a} & \href{../works/LimHTB16.pdf}{LimHTB16}~\cite{LimHTB16} & \href{../works/LimRX04.pdf}{LimRX04}~\cite{LimRX04}\\ 
\href{../works/Limtanyakul07.pdf}{Limtanyakul07}~\cite{Limtanyakul07} & \href{../works/LimtanyakulS12.pdf}{LimtanyakulS12}~\cite{LimtanyakulS12} & \href{../works/LipovetzkyBPS14.pdf}{LipovetzkyBPS14}~\cite{LipovetzkyBPS14} & \href{../works/LiuCGM17.pdf}{LiuCGM17}~\cite{LiuCGM17} & \href{../}{LiuGT10}~\cite{LiuGT10} & \href{../works/LiuJ06.pdf}{LiuJ06}~\cite{LiuJ06}\\ 
\href{../works/LiuLH18.pdf}{LiuLH18}~\cite{LiuLH18} & \href{../works/LiuLH19.pdf}{LiuLH19}~\cite{LiuLH19} & \href{../works/LiuLH19a.pdf}{LiuLH19a}~\cite{LiuLH19a} & \href{../works/LiuW11.pdf}{LiuW11}~\cite{LiuW11} & \href{../works/Lombardi10.pdf}{Lombardi10}~\cite{Lombardi10} & \href{../works/LombardiBM15.pdf}{LombardiBM15}~\cite{LombardiBM15}\\ 
\href{../works/LombardiBMB11.pdf}{LombardiBMB11}~\cite{LombardiBMB11} & \href{../works/LombardiM09.pdf}{LombardiM09}~\cite{LombardiM09} & \href{../works/LombardiM10.pdf}{LombardiM10}~\cite{LombardiM10} & \href{../works/LombardiM10a.pdf}{LombardiM10a}~\cite{LombardiM10a} & \href{../works/LombardiM12.pdf}{LombardiM12}~\cite{LombardiM12} & \href{../works/LombardiM12a.pdf}{LombardiM12a}~\cite{LombardiM12a}\\ 
\href{../works/LombardiM13.pdf}{LombardiM13}~\cite{LombardiM13} & \href{../works/LombardiMB13.pdf}{LombardiMB13}~\cite{LombardiMB13} & \href{../works/LombardiMRB10.pdf}{LombardiMRB10}~\cite{LombardiMRB10} & \href{../works/LopesCSM10.pdf}{LopesCSM10}~\cite{LopesCSM10} & \href{../works/LopezAKYG00.pdf}{LopezAKYG00}~\cite{LopezAKYG00} & \href{../works/LorigeonBB02.pdf}{LorigeonBB02}~\cite{LorigeonBB02}\\ 
\href{../works/LouieVNB14.pdf}{LouieVNB14}~\cite{LouieVNB14} & \href{../works/LozanoCDS12.pdf}{LozanoCDS12}~\cite{LozanoCDS12} & \href{../works/LuZZYW24.pdf}{LuZZYW24}~\cite{LuZZYW24} & \href{../works/LudwigKRBMS14.pdf}{LudwigKRBMS14}~\cite{LudwigKRBMS14} & \href{../works/Lunardi20.pdf}{Lunardi20}~\cite{Lunardi20} & \href{../works/LunardiBLRV20.pdf}{LunardiBLRV20}~\cite{LunardiBLRV20}\\ 
\href{../works/LuoB22.pdf}{LuoB22}~\cite{LuoB22} & \href{../works/LuoVLBM16.pdf}{LuoVLBM16}~\cite{LuoVLBM16} & \href{../works/Madi-WambaB16.pdf}{Madi-WambaB16}~\cite{Madi-WambaB16} & \href{../works/Madi-WambaLOBM17.pdf}{Madi-WambaLOBM17}~\cite{Madi-WambaLOBM17} & \href{../}{MagataoAN05}~\cite{MagataoAN05} & \href{../works/Maillard15.pdf}{Maillard15}~\cite{Maillard15}\\ 
\href{../works/MakMS10.pdf}{MakMS10}~\cite{MakMS10} & \href{../works/Malapert11.pdf}{Malapert11}~\cite{Malapert11} & \href{../works/MalapertCGJLR12.pdf}{MalapertCGJLR12}~\cite{MalapertCGJLR12} & \href{../works/MalapertCGJLR13.pdf}{MalapertCGJLR13}~\cite{MalapertCGJLR13} & \href{../works/MalapertGR12.pdf}{MalapertGR12}~\cite{MalapertGR12} & \href{../works/MalapertN19.pdf}{MalapertN19}~\cite{MalapertN19}\\ 
\href{../works/Malik08.pdf}{Malik08}~\cite{Malik08} & \href{../works/MalikMB08.pdf}{MalikMB08}~\cite{MalikMB08} & \href{../works/MaraveliasCG04.pdf}{MaraveliasCG04}~\cite{MaraveliasCG04} & \href{../works/MaraveliasG04.pdf}{MaraveliasG04}~\cite{MaraveliasG04} & \href{../works/MarliereSPR23.pdf}{MarliereSPR23}~\cite{MarliereSPR23} & \href{../works/MartinPY01.pdf}{MartinPY01}~\cite{MartinPY01}\\ 
\href{../}{MartnezAJ22}~\cite{MartnezAJ22} & \href{../works/Mason01.pdf}{Mason01}~\cite{Mason01} & \href{../works/Mehdizadeh-Somarin23.pdf}{Mehdizadeh-Somarin23}~\cite{Mehdizadeh-Somarin23} & \href{../works/MejiaY20.pdf}{MejiaY20}~\cite{MejiaY20} & \href{../works/MelgarejoLS15.pdf}{MelgarejoLS15}~\cite{MelgarejoLS15} & \href{../works/Menana11.pdf}{Menana11}~\cite{Menana11}\\ 
\href{../works/MenciaSV12.pdf}{MenciaSV12}~\cite{MenciaSV12} & \href{../works/MenciaSV13.pdf}{MenciaSV13}~\cite{MenciaSV13} & \href{../works/MengGRZSC22.pdf}{MengGRZSC22}~\cite{MengGRZSC22} & \href{../works/MengLZB21.pdf}{MengLZB21}~\cite{MengLZB21} & \href{../works/MengZRZL20.pdf}{MengZRZL20}~\cite{MengZRZL20} & \href{../works/Mercier-AubinGQ20.pdf}{Mercier-AubinGQ20}~\cite{Mercier-AubinGQ20}\\ 
\href{../works/MercierH07.pdf}{MercierH07}~\cite{MercierH07} & \href{../works/MercierH08.pdf}{MercierH08}~\cite{MercierH08} & \href{../works/MeskensDHG11.pdf}{MeskensDHG11}~\cite{MeskensDHG11} & \href{../works/MeskensDL13.pdf}{MeskensDL13}~\cite{MeskensDL13} & \href{../works/MeyerE04.pdf}{MeyerE04}~\cite{MeyerE04} & \href{../}{Milano11}~\cite{Milano11}\\ 
\href{../}{MilanoORT02}~\cite{MilanoORT02} & \href{../works/MilanoW06.pdf}{MilanoW06}~\cite{MilanoW06} & \href{../works/MilanoW09.pdf}{MilanoW09}~\cite{MilanoW09} & \href{../works/MintonJPL90.pdf}{MintonJPL90}~\cite{MintonJPL90} & \href{../works/MintonJPL92.pdf}{MintonJPL92}~\cite{MintonJPL92} & \href{../works/MoffittPP05.pdf}{MoffittPP05}~\cite{MoffittPP05}\\ 
\href{../works/MokhtarzadehTNF20.pdf}{MokhtarzadehTNF20}~\cite{MokhtarzadehTNF20} & \href{../works/MonetteDD07.pdf}{MonetteDD07}~\cite{MonetteDD07} & \href{../works/MonetteDH09.pdf}{MonetteDH09}~\cite{MonetteDH09} & \href{../works/MontemanniD23.pdf}{MontemanniD23}~\cite{MontemanniD23} & \href{../works/MontemanniD23a.pdf}{MontemanniD23a}~\cite{MontemanniD23a} & \href{../works/MorgadoM97.pdf}{MorgadoM97}~\cite{MorgadoM97}\\ 
\href{../works/MossigeGSMC17.pdf}{MossigeGSMC17}~\cite{MossigeGSMC17} & \href{../works/MouraSCL08.pdf}{MouraSCL08}~\cite{MouraSCL08} & \href{../works/MouraSCL08a.pdf}{MouraSCL08a}~\cite{MouraSCL08a} & \href{../works/MullerMKP22.pdf}{MullerMKP22}~\cite{MullerMKP22} & \href{../works/MurinR19.pdf}{MurinR19}~\cite{MurinR19} & \href{../works/MurphyMB15.pdf}{MurphyMB15}~\cite{MurphyMB15}\\ 
\href{../works/MurphyRFSS97.pdf}{MurphyRFSS97}~\cite{MurphyRFSS97} & \href{../}{MurthyRAW97}~\cite{MurthyRAW97} & \href{../works/Muscettola02.pdf}{Muscettola02}~\cite{Muscettola02} & \href{../works/Muscettola94.pdf}{Muscettola94}~\cite{Muscettola94} & \href{../works/Musliu05.pdf}{Musliu05}~\cite{Musliu05} & \href{../works/MusliuSS18.pdf}{MusliuSS18}~\cite{MusliuSS18}\\ 
\href{../works/NaderiBZ22.pdf}{NaderiBZ22}~\cite{NaderiBZ22} & \href{../works/NaderiBZ22a.pdf}{NaderiBZ22a}~\cite{NaderiBZ22a} & \href{../works/NaderiBZ23.pdf}{NaderiBZ23}~\cite{NaderiBZ23} & \href{../works/NaderiBZR23.pdf}{NaderiBZR23}~\cite{NaderiBZR23} & \href{../}{NaderiR22}~\cite{NaderiR22} & \href{../}{NaderiRBAU21}~\cite{NaderiRBAU21}\\ 
\href{../works/NaderiRR23.pdf}{NaderiRR23}~\cite{NaderiRR23} & \href{../works/NaqviAIAAA22.pdf}{NaqviAIAAA22}~\cite{NaqviAIAAA22} & \href{../works/Nattaf16.pdf}{Nattaf16}~\cite{Nattaf16} & \href{../works/NattafAL15.pdf}{NattafAL15}~\cite{NattafAL15} & \href{../works/NattafAL17.pdf}{NattafAL17}~\cite{NattafAL17} & \href{../works/NattafALR16.pdf}{NattafALR16}~\cite{NattafALR16}\\ 
\href{../works/NattafDYW19.pdf}{NattafDYW19}~\cite{NattafDYW19} & \href{../works/NattafHKAL19.pdf}{NattafHKAL19}~\cite{NattafHKAL19} & \href{../works/NattafM20.pdf}{NattafM20}~\cite{NattafM20} & \href{../}{NeronABCDD06}~\cite{NeronABCDD06} & \href{../works/NishikawaSTT18.pdf}{NishikawaSTT18}~\cite{NishikawaSTT18} & \href{../works/NishikawaSTT18a.pdf}{NishikawaSTT18a}~\cite{NishikawaSTT18a}\\ 
\href{../works/NishikawaSTT19.pdf}{NishikawaSTT19}~\cite{NishikawaSTT19} & \href{../}{NouriMHD23}~\cite{NouriMHD23} & \href{../works/NovaraNH16.pdf}{NovaraNH16}~\cite{NovaraNH16} & \href{../works/Novas19.pdf}{Novas19}~\cite{Novas19} & \href{../works/NovasH10.pdf}{NovasH10}~\cite{NovasH10} & \href{../works/NovasH12.pdf}{NovasH12}~\cite{NovasH12}\\ 
\href{../works/NovasH14.pdf}{NovasH14}~\cite{NovasH14} & \href{../works/NuijtenA94.pdf}{NuijtenA94}~\cite{NuijtenA94} & \href{../}{NuijtenA94a}~\cite{NuijtenA94a} & \href{../works/NuijtenA96.pdf}{NuijtenA96}~\cite{NuijtenA96} & \href{../works/NuijtenP98.pdf}{NuijtenP98}~\cite{NuijtenP98} & \href{../works/OddiPCC03.pdf}{OddiPCC03}~\cite{OddiPCC03}\\ 
\href{../}{OddiPCC05}~\cite{OddiPCC05} & \href{../works/OddiRC10.pdf}{OddiRC10}~\cite{OddiRC10} & \href{../works/OddiRCS11.pdf}{OddiRCS11}~\cite{OddiRCS11} & \href{../works/OddiS97.pdf}{OddiS97}~\cite{OddiS97} & \href{../works/OhrimenkoSC09.pdf}{OhrimenkoSC09}~\cite{OhrimenkoSC09} & \href{../}{OkanoDTRYA04}~\cite{OkanoDTRYA04}\\ 
\href{../works/OrnekO16.pdf}{OrnekO16}~\cite{OrnekO16} & \href{../works/OrnekOS20.pdf}{OrnekOS20}~\cite{OrnekOS20} & \href{../works/OuelletQ13.pdf}{OuelletQ13}~\cite{OuelletQ13} & \href{../works/OuelletQ18.pdf}{OuelletQ18}~\cite{OuelletQ18} & \href{../works/OuelletQ22.pdf}{OuelletQ22}~\cite{OuelletQ22} & \href{../works/OujanaAYB22.pdf}{OujanaAYB22}~\cite{OujanaAYB22}\\ 
\href{../works/OzturkTHO10.pdf}{OzturkTHO10}~\cite{OzturkTHO10} & \href{../works/OzturkTHO12.pdf}{OzturkTHO12}~\cite{OzturkTHO12} & \href{../works/OzturkTHO13.pdf}{OzturkTHO13}~\cite{OzturkTHO13} & \href{../works/OzturkTHO15.pdf}{OzturkTHO15}~\cite{OzturkTHO15} & \href{../works/PachecoPR19.pdf}{PachecoPR19}~\cite{PachecoPR19} & \href{../works/PacinoH11.pdf}{PacinoH11}~\cite{PacinoH11}\\ 
\href{../works/PandeyS21a.pdf}{PandeyS21a}~\cite{PandeyS21a} & \href{../works/PapaB98.pdf}{PapaB98}~\cite{PapaB98} & \href{../works/Pape94.pdf}{Pape94}~\cite{Pape94} & \href{../}{PapeB96}~\cite{PapeB96} & \href{../works/PapeB97.pdf}{PapeB97}~\cite{PapeB97} & \href{../works/ParkUJR19.pdf}{ParkUJR19}~\cite{ParkUJR19}\\ 
\href{../works/PembertonG98.pdf}{PembertonG98}~\cite{PembertonG98} & \href{../works/PengLC14.pdf}{PengLC14}~\cite{PengLC14} & \href{../works/PenzDN23.pdf}{PenzDN23}~\cite{PenzDN23} & \href{../works/PerezGSL23.pdf}{PerezGSL23}~\cite{PerezGSL23} & \href{../works/Perron05.pdf}{Perron05}~\cite{Perron05} & \href{../works/PerronSF04.pdf}{PerronSF04}~\cite{PerronSF04}\\ 
\href{../works/PesantGPR99.pdf}{PesantGPR99}~\cite{PesantGPR99} & \href{../works/PesantRR15.pdf}{PesantRR15}~\cite{PesantRR15} & \href{../}{PeschT96}~\cite{PeschT96} & \href{../}{Pinarbasi21}~\cite{Pinarbasi21} & \href{../}{PinarbasiA20}~\cite{PinarbasiA20} & \href{../works/PinarbasiAY19.pdf}{PinarbasiAY19}~\cite{PinarbasiAY19}\\ 
\href{../works/PintoG97.pdf}{PintoG97}~\cite{PintoG97} & \href{../works/PoderB08.pdf}{PoderB08}~\cite{PoderB08} & \href{../works/PoderBS04.pdf}{PoderBS04}~\cite{PoderBS04} & \href{../works/PohlAK22.pdf}{PohlAK22}~\cite{PohlAK22} & \href{../works/PolicellaWSO05.pdf}{PolicellaWSO05}~\cite{PolicellaWSO05} & \href{../works/Polo-MejiaALB20.pdf}{Polo-MejiaALB20}~\cite{Polo-MejiaALB20}\\ 
\href{../works/PopovicCGNC22.pdf}{PopovicCGNC22}~\cite{PopovicCGNC22} & \href{../works/PourDERB18.pdf}{PourDERB18}~\cite{PourDERB18} & \href{../works/PovedaAA23.pdf}{PovedaAA23}~\cite{PovedaAA23} & \href{../works/Pralet17.pdf}{Pralet17}~\cite{Pralet17} & \href{../works/PraletLJ15.pdf}{PraletLJ15}~\cite{PraletLJ15} & \href{../works/PrataAN23.pdf}{PrataAN23}~\cite{PrataAN23}\\ 
\href{../works/Prosser89.pdf}{Prosser89}~\cite{Prosser89} & \href{../works/Puget95.pdf}{Puget95}~\cite{Puget95} & \href{../works/QinDCS20.pdf}{QinDCS20}~\cite{QinDCS20} & \href{../works/QinDS16.pdf}{QinDS16}~\cite{QinDS16} & \href{../works/QinWSLS21.pdf}{QinWSLS21}~\cite{QinWSLS21} & \href{../works/QuSN06.pdf}{QuSN06}~\cite{QuSN06}\\ 
\href{../works/QuirogaZH05.pdf}{QuirogaZH05}~\cite{QuirogaZH05} & \href{../}{RabbaniMM21}~\cite{RabbaniMM21} & \href{../works/RasmussenT06.pdf}{RasmussenT06}~\cite{RasmussenT06} & \href{../works/RasmussenT07.pdf}{RasmussenT07}~\cite{RasmussenT07} & \href{../works/RasmussenT09.pdf}{RasmussenT09}~\cite{RasmussenT09} & \href{../works/ReddyFIBKAJ11.pdf}{ReddyFIBKAJ11}~\cite{ReddyFIBKAJ11}\\ 
\href{../works/Refalo00.pdf}{Refalo00}~\cite{Refalo00} & \href{../works/RenT09.pdf}{RenT09}~\cite{RenT09} & \href{../works/RendlPHPR12.pdf}{RendlPHPR12}~\cite{RendlPHPR12} & \href{../}{Rgin2001}~\cite{Rgin2001} & \href{../works/RiahiNS018.pdf}{RiahiNS018}~\cite{RiahiNS018} & \href{../works/Ribeiro12.pdf}{Ribeiro12}~\cite{Ribeiro12}\\ 
\href{../works/RiiseML16.pdf}{RiiseML16}~\cite{RiiseML16} & \href{../works/Rit86.pdf}{Rit86}~\cite{Rit86} & \href{../}{Rodosek94}~\cite{Rodosek94} & \href{../works/RodosekW98.pdf}{RodosekW98}~\cite{RodosekW98} & \href{../works/RodosekWH99.pdf}{RodosekWH99}~\cite{RodosekWH99} & \href{../works/Rodriguez07.pdf}{Rodriguez07}~\cite{Rodriguez07}\\ 
\href{../works/Rodriguez07b.pdf}{Rodriguez07b}~\cite{Rodriguez07b} & \href{../works/RodriguezDG02.pdf}{RodriguezDG02}~\cite{RodriguezDG02} & \href{../works/RodriguezS09.pdf}{RodriguezS09}~\cite{RodriguezS09} & \href{../works/RoePS05.pdf}{RoePS05}~\cite{RoePS05} & \href{../works/RoshanaeiBAUB20.pdf}{RoshanaeiBAUB20}~\cite{RoshanaeiBAUB20} & \href{../works/RoshanaeiLAU17.pdf}{RoshanaeiLAU17}~\cite{RoshanaeiLAU17}\\ 
\href{../}{RoshanaeiLAU17a}~\cite{RoshanaeiLAU17a} & \href{../works/RoshanaeiN21.pdf}{RoshanaeiN21}~\cite{RoshanaeiN21} & \href{../works/RossiTHP07.pdf}{RossiTHP07}~\cite{RossiTHP07} & \href{../works/RoweJCA96.pdf}{RoweJCA96}~\cite{RoweJCA96} & \href{../works/RuggieroBBMA09.pdf}{RuggieroBBMA09}~\cite{RuggieroBBMA09} & \href{../works/RussellU06.pdf}{RussellU06}~\cite{RussellU06}\\ 
\href{../works/SacramentoSP20.pdf}{SacramentoSP20}~\cite{SacramentoSP20} & \href{../works/SadehF96.pdf}{SadehF96}~\cite{SadehF96} & \href{../works/Sadykov04.pdf}{Sadykov04}~\cite{Sadykov04} & \href{../works/SadykovW06.pdf}{SadykovW06}~\cite{SadykovW06} & \href{../}{SakkoutRW98}~\cite{SakkoutRW98} & \href{../works/SakkoutW00.pdf}{SakkoutW00}~\cite{SakkoutW00}\\ 
\href{../works/Salido10.pdf}{Salido10}~\cite{Salido10} & \href{../}{Schaerf96}~\cite{Schaerf96} & \href{../works/Schaerf97.pdf}{Schaerf97}~\cite{Schaerf97} & \href{../works/SchausD08.pdf}{SchausD08}~\cite{SchausD08} & \href{../works/SchausHMCMD11.pdf}{SchausHMCMD11}~\cite{SchausHMCMD11} & \href{../works/SchildW00.pdf}{SchildW00}~\cite{SchildW00}\\ 
\href{../works/SchnellH15.pdf}{SchnellH15}~\cite{SchnellH15} & \href{../works/SchnellH17.pdf}{SchnellH17}~\cite{SchnellH17} & \href{../works/Schutt11.pdf}{Schutt11}~\cite{Schutt11} & \href{../works/SchuttCSW12.pdf}{SchuttCSW12}~\cite{SchuttCSW12} & \href{../works/SchuttFS13.pdf}{SchuttFS13}~\cite{SchuttFS13} & \href{../works/SchuttFS13a.pdf}{SchuttFS13a}~\cite{SchuttFS13a}\\ 
\href{../works/SchuttFSW09.pdf}{SchuttFSW09}~\cite{SchuttFSW09} & \href{../works/SchuttFSW11.pdf}{SchuttFSW11}~\cite{SchuttFSW11} & \href{../works/SchuttFSW13.pdf}{SchuttFSW13}~\cite{SchuttFSW13} & \href{../}{SchuttFSW15}~\cite{SchuttFSW15} & \href{../works/SchuttS16.pdf}{SchuttS16}~\cite{SchuttS16} & \href{../works/SchuttW10.pdf}{SchuttW10}~\cite{SchuttW10}\\ 
\href{../works/SchuttWS05.pdf}{SchuttWS05}~\cite{SchuttWS05} & \href{../works/SenderovichBB19.pdf}{SenderovichBB19}~\cite{SenderovichBB19} & \href{../works/SerraNM12.pdf}{SerraNM12}~\cite{SerraNM12} & \href{../works/ShaikhK23.pdf}{ShaikhK23}~\cite{ShaikhK23} & \href{../}{ShiYXQ22}~\cite{ShiYXQ22} & \href{../works/ShinBBHO18.pdf}{ShinBBHO18}~\cite{ShinBBHO18}\\ 
\href{../works/Siala15.pdf}{Siala15}~\cite{Siala15} & \href{../works/Siala15a.pdf}{Siala15a}~\cite{Siala15a} & \href{../works/SialaAH15.pdf}{SialaAH15}~\cite{SialaAH15} & \href{../works/SimoninAHL12.pdf}{SimoninAHL12}~\cite{SimoninAHL12} & \href{../works/SimoninAHL15.pdf}{SimoninAHL15}~\cite{SimoninAHL15} & \href{../works/Simonis07.pdf}{Simonis07}~\cite{Simonis07}\\ 
\href{../works/Simonis95.pdf}{Simonis95}~\cite{Simonis95} & \href{../works/Simonis95a.pdf}{Simonis95a}~\cite{Simonis95a} & \href{../works/Simonis99.pdf}{Simonis99}~\cite{Simonis99} & \href{../works/SimonisC95.pdf}{SimonisC95}~\cite{SimonisC95} & \href{../works/SimonisCK00.pdf}{SimonisCK00}~\cite{SimonisCK00} & \href{../works/SimonisH11.pdf}{SimonisH11}~\cite{SimonisH11}\\ 
\href{../works/SmithBHW96.pdf}{SmithBHW96}~\cite{SmithBHW96} & \href{../works/SmithC93.pdf}{SmithC93}~\cite{SmithC93} & \href{../works/SourdN00.pdf}{SourdN00}~\cite{SourdN00} & \href{../works/SquillaciPR23.pdf}{SquillaciPR23}~\cite{SquillaciPR23} & \href{../}{StidsenKM96}~\cite{StidsenKM96} & \href{../works/SuCC13.pdf}{SuCC13}~\cite{SuCC13}\\ 
\href{../works/SubulanC22.pdf}{SubulanC22}~\cite{SubulanC22} & \href{../works/SultanikMR07.pdf}{SultanikMR07}~\cite{SultanikMR07} & \href{../works/SunLYL10.pdf}{SunLYL10}~\cite{SunLYL10} & \href{../works/SunTB19.pdf}{SunTB19}~\cite{SunTB19} & \href{../works/SureshMOK06.pdf}{SureshMOK06}~\cite{SureshMOK06} & \href{../works/SvancaraB22.pdf}{SvancaraB22}~\cite{SvancaraB22}\\ 
\href{../works/SzerediS16.pdf}{SzerediS16}~\cite{SzerediS16} & \href{../works/TanSD10.pdf}{TanSD10}~\cite{TanSD10} & \href{../works/TanT18.pdf}{TanT18}~\cite{TanT18} & \href{../works/TanZWGQ19.pdf}{TanZWGQ19}~\cite{TanZWGQ19} & \href{../works/TangB20.pdf}{TangB20}~\cite{TangB20} & \href{../works/TangLWSK18.pdf}{TangLWSK18}~\cite{TangLWSK18}\\ 
\href{../works/TardivoDFMP23.pdf}{TardivoDFMP23}~\cite{TardivoDFMP23} & \href{../works/Tassel22.pdf}{Tassel22}~\cite{Tassel22} & \href{../works/TasselGS23.pdf}{TasselGS23}~\cite{TasselGS23} & \href{../}{Tay92}~\cite{Tay92} & \href{../works/Teppan22.pdf}{Teppan22}~\cite{Teppan22} & \href{../works/TerekhovDOB12.pdf}{TerekhovDOB12}~\cite{TerekhovDOB12}\\ 
\href{../works/TerekhovTDB14.pdf}{TerekhovTDB14}~\cite{TerekhovTDB14} & \href{../works/Tesch16.pdf}{Tesch16}~\cite{Tesch16} & \href{../works/Tesch18.pdf}{Tesch18}~\cite{Tesch18} & \href{../works/ThiruvadyBME09.pdf}{ThiruvadyBME09}~\cite{ThiruvadyBME09} & \href{../works/ThiruvadyWGS14.pdf}{ThiruvadyWGS14}~\cite{ThiruvadyWGS14} & \href{../works/ThomasKS20.pdf}{ThomasKS20}~\cite{ThomasKS20}\\ 
\href{../works/Thorsteinsson01.pdf}{Thorsteinsson01}~\cite{Thorsteinsson01} & \href{../works/Timpe02.pdf}{Timpe02}~\cite{Timpe02} & \href{../works/Tom19.pdf}{Tom19}~\cite{Tom19} & \href{../works/TopalogluO11.pdf}{TopalogluO11}~\cite{TopalogluO11} & \href{../works/TopalogluSS12.pdf}{TopalogluSS12}~\cite{TopalogluSS12} & \href{../works/TorresL00.pdf}{TorresL00}~\cite{TorresL00}\\ 
\href{../works/TouatBT22.pdf}{TouatBT22}~\cite{TouatBT22} & \href{../works/Touraivane95.pdf}{Touraivane95}~\cite{Touraivane95} & \href{../works/TranAB16.pdf}{TranAB16}~\cite{TranAB16} & \href{../works/TranB12.pdf}{TranB12}~\cite{TranB12} & \href{../works/TranDRFWOVB16.pdf}{TranDRFWOVB16}~\cite{TranDRFWOVB16} & \href{../works/TranPZLDB18.pdf}{TranPZLDB18}~\cite{TranPZLDB18}\\ 
\href{../works/TranTDB13.pdf}{TranTDB13}~\cite{TranTDB13} & \href{../works/TranVNB17.pdf}{TranVNB17}~\cite{TranVNB17} & \href{../works/TranVNB17a.pdf}{TranVNB17a}~\cite{TranVNB17a} & \href{../works/TranWDRFOVB16.pdf}{TranWDRFOVB16}~\cite{TranWDRFOVB16} & \href{../works/TrentesauxPT01.pdf}{TrentesauxPT01}~\cite{TrentesauxPT01} & \href{../works/Trick03.pdf}{Trick03}~\cite{Trick03}\\ 
\href{../}{Trick11}~\cite{Trick11} & \href{../works/TrojetHL11.pdf}{TrojetHL11}~\cite{TrojetHL11} & \href{../works/Tsang03.pdf}{Tsang03}~\cite{Tsang03} & \href{../}{TsurutaS00}~\cite{TsurutaS00} & \href{../works/UnsalO13.pdf}{UnsalO13}~\cite{UnsalO13} & \href{../works/UnsalO19.pdf}{UnsalO19}~\cite{UnsalO19}\\ 
\href{../works/Valdes87.pdf}{Valdes87}~\cite{Valdes87} & \href{../works/ValleMGT03.pdf}{ValleMGT03}~\cite{ValleMGT03} & \href{../works/VanczaM01.pdf}{VanczaM01}~\cite{VanczaM01} & \href{../works/VerfaillieL01.pdf}{VerfaillieL01}~\cite{VerfaillieL01} & \href{../works/Vilim02.pdf}{Vilim02}~\cite{Vilim02} & \href{../works/Vilim03.pdf}{Vilim03}~\cite{Vilim03}\\ 
\href{../works/Vilim04.pdf}{Vilim04}~\cite{Vilim04} & \href{../works/Vilim05.pdf}{Vilim05}~\cite{Vilim05} & \href{../works/Vilim09.pdf}{Vilim09}~\cite{Vilim09} & \href{../works/Vilim09a.pdf}{Vilim09a}~\cite{Vilim09a} & \href{../works/Vilim11.pdf}{Vilim11}~\cite{Vilim11} & \href{../works/VilimBC04.pdf}{VilimBC04}~\cite{VilimBC04}\\ 
\href{../works/VilimBC05.pdf}{VilimBC05}~\cite{VilimBC05} & \href{../works/VilimLS15.pdf}{VilimLS15}~\cite{VilimLS15} & \href{../}{VillaverdeP04}~\cite{VillaverdeP04} & \href{../works/VlkHT21.pdf}{VlkHT21}~\cite{VlkHT21} & \href{../works/Wallace06.pdf}{Wallace06}~\cite{Wallace06} & \href{../}{Wallace94}~\cite{Wallace94}\\ 
\href{../works/Wallace96.pdf}{Wallace96}~\cite{Wallace96} & \href{../works/WallaceF00.pdf}{WallaceF00}~\cite{WallaceF00} & \href{../works/WallaceY20.pdf}{WallaceY20}~\cite{WallaceY20} & \href{../works/WangB20.pdf}{WangB20}~\cite{WangB20} & \href{../works/WangB23.pdf}{WangB23}~\cite{WangB23} & \href{../works/WangMD15.pdf}{WangMD15}~\cite{WangMD15}\\ 
\href{../}{WariZ19}~\cite{WariZ19} & \href{../works/WatsonB08.pdf}{WatsonB08}~\cite{WatsonB08} & \href{../works/WatsonBHW99.pdf}{WatsonBHW99}~\cite{WatsonBHW99} & \href{../works/WeilHFP95.pdf}{WeilHFP95}~\cite{WeilHFP95} & \href{../works/WessenCS20.pdf}{WessenCS20}~\cite{WessenCS20} & \href{../works/WessenCSFPM23.pdf}{WessenCSFPM23}~\cite{WessenCSFPM23}\\ 
\href{../works/WikarekS19.pdf}{WikarekS19}~\cite{WikarekS19} & \href{../works/WinterMMW22.pdf}{WinterMMW22}~\cite{WinterMMW22} & \href{../works/Wolf03.pdf}{Wolf03}~\cite{Wolf03} & \href{../works/Wolf05.pdf}{Wolf05}~\cite{Wolf05} & \href{../works/Wolf09.pdf}{Wolf09}~\cite{Wolf09} & \href{../works/Wolf11.pdf}{Wolf11}~\cite{Wolf11}\\ 
\href{../works/WolfS05.pdf}{WolfS05}~\cite{WolfS05} & \href{../works/WolfS05a.pdf}{WolfS05a}~\cite{WolfS05a} & \href{../works/WolinskiKG04.pdf}{WolinskiKG04}~\cite{WolinskiKG04} & \href{../works/WuBB05.pdf}{WuBB05}~\cite{WuBB05} & \href{../works/WuBB09.pdf}{WuBB09}~\cite{WuBB09} & \href{../works/YangSS19.pdf}{YangSS19}~\cite{YangSS19}\\ 
\href{../}{YeGMH94}~\cite{YeGMH94} & \href{../works/YoshikawaKNW94.pdf}{YoshikawaKNW94}~\cite{YoshikawaKNW94} & \href{../works/YounespourAKE19.pdf}{YounespourAKE19}~\cite{YounespourAKE19} & \href{../works/YoungFS17.pdf}{YoungFS17}~\cite{YoungFS17} & \href{../works/YunusogluY22.pdf}{YunusogluY22}~\cite{YunusogluY22} & \href{../works/YuraszeckMC23.pdf}{YuraszeckMC23}~\cite{YuraszeckMC23}\\ 
\href{../works/YuraszeckMCCR23.pdf}{YuraszeckMCCR23}~\cite{YuraszeckMCCR23} & \href{../works/YuraszeckMPV22.pdf}{YuraszeckMPV22}~\cite{YuraszeckMPV22} & \href{../works/Zahout21.pdf}{Zahout21}~\cite{Zahout21} & \href{../works/ZampelliVSDR13.pdf}{ZampelliVSDR13}~\cite{ZampelliVSDR13} & \href{../works/ZarandiASC20.pdf}{ZarandiASC20}~\cite{ZarandiASC20} & \href{../}{ZarandiB12}~\cite{ZarandiB12}\\ 
\href{../works/ZarandiKS16.pdf}{ZarandiKS16}~\cite{ZarandiKS16} & \href{../works/Zeballos10.pdf}{Zeballos10}~\cite{Zeballos10} & \href{../works/ZeballosCM10.pdf}{ZeballosCM10}~\cite{ZeballosCM10} & \href{../works/ZeballosH05.pdf}{ZeballosH05}~\cite{ZeballosH05} & \href{../works/ZeballosM09.pdf}{ZeballosM09}~\cite{ZeballosM09} & \href{../works/ZeballosNH11.pdf}{ZeballosNH11}~\cite{ZeballosNH11}\\ 
\href{../works/ZeballosQH10.pdf}{ZeballosQH10}~\cite{ZeballosQH10} & \href{../works/ZengM12.pdf}{ZengM12}~\cite{ZengM12} & \href{../works/ZhangBB22.pdf}{ZhangBB22}~\cite{ZhangBB22} & \href{../works/ZhangJZL22.pdf}{ZhangJZL22}~\cite{ZhangJZL22} & \href{../works/ZhangLS12.pdf}{ZhangLS12}~\cite{ZhangLS12} & \href{../works/ZhangW18.pdf}{ZhangW18}~\cite{ZhangW18}\\ 
\href{../works/ZhangYW21.pdf}{ZhangYW21}~\cite{ZhangYW21} & \href{../works/ZhaoL14.pdf}{ZhaoL14}~\cite{ZhaoL14} & \href{../works/Zhou96.pdf}{Zhou96}~\cite{Zhou96} & \href{../works/Zhou97.pdf}{Zhou97}~\cite{Zhou97} & \href{../works/ZhouGL15.pdf}{ZhouGL15}~\cite{ZhouGL15} & \href{../works/ZhuS02.pdf}{ZhuS02}~\cite{ZhuS02}\\ 
\href{../works/ZhuSZW23.pdf}{ZhuSZW23}~\cite{ZhuSZW23} & \href{../works/ZibranR11.pdf}{ZibranR11}~\cite{ZibranR11} & \href{../works/ZibranR11a.pdf}{ZibranR11a}~\cite{ZibranR11a} & \href{../works/ZouZ20.pdf}{ZouZ20}~\cite{ZouZ20} & \href{../works/abs-0907-0939.pdf}{abs-0907-0939}~\cite{abs-0907-0939} & \href{../works/abs-1009-0347.pdf}{abs-1009-0347}~\cite{abs-1009-0347}\\ 
\href{../works/abs-1901-07914.pdf}{abs-1901-07914}~\cite{abs-1901-07914} & \href{../works/abs-1902-01193.pdf}{abs-1902-01193}~\cite{abs-1902-01193} & \href{../works/abs-1902-09244.pdf}{abs-1902-09244}~\cite{abs-1902-09244} & \href{../works/abs-1911-04766.pdf}{abs-1911-04766}~\cite{abs-1911-04766} & \href{../works/abs-2102-08778.pdf}{abs-2102-08778}~\cite{abs-2102-08778} & \href{../works/abs-2211-14492.pdf}{abs-2211-14492}~\cite{abs-2211-14492}\\ 
\href{../works/abs-2305-19888.pdf}{abs-2305-19888}~\cite{abs-2305-19888} & \href{../works/abs-2306-05747.pdf}{abs-2306-05747}~\cite{abs-2306-05747} & \href{../works/abs-2312-13682.pdf}{abs-2312-13682}~\cite{abs-2312-13682} & \href{../works/abs-2402-00459.pdf}{abs-2402-00459}~\cite{abs-2402-00459} & \end{longtable}


\section{Conference Paper List}

This section presents the information for all conference papers included in the survey. For space reasons, not all information about the papers can be presented in a single table, we therefore split the data into three parts. The first part contains the main bibliographical information for the paper. The paper are sorted by year of publication (newest first), and then alphabetically by key. 

The key contains a hyperlink to the original source URL of the paper. You may have to navigate manually to download the actual paper content, and you may be unable to access the paper completely if it is behind a paywall for which you (or your organization) do not have access.

We then list the authors of the paper, in the other given in the bibtex file, abbreviating first names for space where we can identify them. Note that names with non-latin characters are not handled by latex. We use the form that is given in the bibtex file, but have excluded entries that cause latex to fail.  

We then give the title of the publication, using the original capitalization of the title entry in the bibtex entry, which may differ from the format shown in the bibliography. We then (column LC) provide a link to a local copy, if it is present, and a link to the bibliography entry of the paper.  We also show the year of publication, and the conference where the paper was published, using a short form abbreviation of the conference. This relies on a matching routine in the Java code to find the short title, new conference series may require an additional entry in \texttt{ImportBibtex.java} to work properly. Finally we list the number of pages of the paper, this information is using the bibtex entry where possible, otherwise uses \texttt{pdfinfo} to extract the actual number of pages from the local copy. The final columns b and c provide links to the corresponding tables of extracted concepts and manual information. Note that the links to typically show the correct page, not do not necessarily scroll to the correct line in the table.

\clearpage
\subsection{Papers from bibtex}
{\scriptsize
\begin{longtable}{>{\raggedright\arraybackslash}p{3cm}>{\raggedright\arraybackslash}p{6cm}>{\raggedright\arraybackslash}p{6.5cm}rrrp{2.5cm}rrrrr}
\rowcolor{white}\caption{Works from bibtex (Total 411)}\\ \toprule
\rowcolor{white}\shortstack{Key\\Source} & Authors & Title & LC & Cite & Year & \shortstack{Conference\\/Journal\\/School} & Pages & \shortstack{Nr\\Cites} & \shortstack{Nr\\Refs} & b & c \\ \midrule\endhead
\bottomrule
\endfoot
\rowlabel{a:BonninMNE24}BonninMNE24 \href{https://doi.org/10.5220/0012310200003639}{BonninMNE24} & \hyperref[auth:a1020]{C. Bonnin}, \hyperref[auth:a82]{A. Malapert}, \hyperref[auth:a81]{M. Nattaf}, \hyperref[auth:a1021]{M. Espinouse} & Toward a Global Constraint for Minimizing the Flowtime & \href{../works/BonninMNE24.pdf}{Yes} & \cite{BonninMNE24} & 2024 & ICORES 2024 & 12 & 0 & 0 & \ref{b:BonninMNE24} & n/a\\
\rowlabel{a:FalqueALM24}FalqueALM24 \href{https://doi.org/10.1609/aaai.v38i21.30308}{FalqueALM24} & \hyperref[auth:a1393]{T. Falque}, \hyperref[auth:a1394]{G. Audemard}, \hyperref[auth:a218]{C. Lecoutre}, \hyperref[auth:a1395]{B. Mazure} & Check-In Desk Scheduling Optimisation at {CDG} International Airport & \href{../works/FalqueALM24.pdf}{Yes} & \cite{FalqueALM24} & 2024 & AAAI 2024 & 9 & 0 & 0 & \ref{b:FalqueALM24} & \ref{c:FalqueALM24}\\
\rowlabel{a:LiLZDZW24}LiLZDZW24 \href{https://doi.org/10.1609/aaai.v38i18.29998}{LiLZDZW24} & \hyperref[auth:a1387]{L. Li}, \hyperref[auth:a1388]{S. Liang}, \hyperref[auth:a1389]{Z. Zhu}, \hyperref[auth:a1390]{C. Ding}, \hyperref[auth:a1391]{H. Zha}, \hyperref[auth:a1392]{B. Wu} & Learning to Optimize Permutation Flow Shop Scheduling via Graph-Based Imitation Learning & \href{../works/LiLZDZW24.pdf}{Yes} & \cite{LiLZDZW24} & 2024 & AAAI 2024 & 9 & 0 & 0 & \ref{b:LiLZDZW24} & \ref{c:LiLZDZW24}\\
\rowlabel{a:AalianPG23}AalianPG23 \href{https://doi.org/10.4230/LIPIcs.CP.2023.6}{AalianPG23} & \hyperref[auth:a7]{Y. Aalian}, \hyperref[auth:a8]{G. Pesant}, \hyperref[auth:a9]{M. Gamache} & Optimization of Short-Term Underground Mine Planning Using Constraint Programming & \href{../works/AalianPG23.pdf}{Yes} & \cite{AalianPG23} & 2023 & CP 2023 & 16 & 0 & 0 & \ref{b:AalianPG23} & \ref{c:AalianPG23}\\
\rowlabel{a:Bit-Monnot23}Bit-Monnot23 \href{https://doi.org/10.3233/FAIA230278}{Bit-Monnot23} & \hyperref[auth:a395]{A. Bit{-}Monnot} & Enhancing Hybrid {CP-SAT} Search for Disjunctive Scheduling & \href{../works/Bit-Monnot23.pdf}{Yes} & \cite{Bit-Monnot23} & 2023 & ECAI 2023 & 8 & 0 & 0 & \ref{b:Bit-Monnot23} & \ref{c:Bit-Monnot23}\\
\rowlabel{a:EfthymiouY23}EfthymiouY23 \href{https://doi.org/10.1007/978-3-031-33271-5_16}{EfthymiouY23} & \hyperref[auth:a18]{N. Efthymiou}, \hyperref[auth:a19]{N. Yorke{-}Smith} & Predicting the Optimal Period for Cyclic Hoist Scheduling Problems & \href{../works/EfthymiouY23.pdf}{Yes} & \cite{EfthymiouY23} & 2023 & CPAIOR 2023 & 16 & 0 & 23 & \ref{b:EfthymiouY23} & \ref{c:EfthymiouY23}\\
\rowlabel{a:JuvinHHL23}JuvinHHL23 \href{https://doi.org/10.4230/LIPIcs.CP.2023.19}{JuvinHHL23} & \hyperref[auth:a0]{C. Juvin}, \hyperref[auth:a1]{E. Hebrard}, \hyperref[auth:a2]{L. Houssin}, \hyperref[auth:a3]{P. Lopez} & An Efficient Constraint Programming Approach to Preemptive Job Shop Scheduling & \href{../works/JuvinHHL23.pdf}{Yes} & \cite{JuvinHHL23} & 2023 & CP 2023 & 16 & 0 & 0 & \ref{b:JuvinHHL23} & \ref{c:JuvinHHL23}\\
\rowlabel{a:JuvinHL23}JuvinHL23 \href{https://doi.org/10.1007/978-3-031-33271-5_23}{JuvinHL23} & \hyperref[auth:a0]{C. Juvin}, \hyperref[auth:a2]{L. Houssin}, \hyperref[auth:a3]{P. Lopez} & Constraint Programming for the Robust Two-Machine Flow-Shop Scheduling Problem with Budgeted Uncertainty & \href{../works/JuvinHL23.pdf}{Yes} & \cite{JuvinHL23} & 2023 & CPAIOR 2023 & 16 & 0 & 11 & \ref{b:JuvinHL23} & \ref{c:JuvinHL23}\\
\rowlabel{a:KameugneFND23}KameugneFND23 \href{https://doi.org/10.4230/LIPIcs.CP.2023.20}{KameugneFND23} & \hyperref[auth:a10]{R. Kameugne}, \hyperref[auth:a11]{S{\'{e}}v{\'{e}}rine Betmbe Fetgo}, \hyperref[auth:a12]{T. Noulamo}, \hyperref[auth:a13]{Cl{\'{e}}mentin Tayou Djam{\'{e}}gni} & Horizontally Elastic Edge Finder Rule for Cumulative Constraint Based on Slack and Density & \href{../works/KameugneFND23.pdf}{Yes} & \cite{KameugneFND23} & 2023 & CP 2023 & 17 & 0 & 0 & \ref{b:KameugneFND23} & \ref{c:KameugneFND23}\\
\rowlabel{a:KimCMLLP23}KimCMLLP23 \href{https://doi.org/10.1007/978-3-031-33271-5_31}{KimCMLLP23} & \hyperref[auth:a23]{D. Kim}, \hyperref[auth:a24]{Y. Choi}, \hyperref[auth:a25]{K. Moon}, \hyperref[auth:a26]{M. Lee}, \hyperref[auth:a27]{K. Lee}, \hyperref[auth:a28]{Michael L. Pinedo} & Iterated Greedy Constraint Programming for Scheduling Steelmaking Continuous Casting & \href{../works/KimCMLLP23.pdf}{Yes} & \cite{KimCMLLP23} & 2023 & CPAIOR 2023 & 16 & 0 & 13 & \ref{b:KimCMLLP23} & \ref{c:KimCMLLP23}\\
\rowlabel{a:Mehdizadeh-Somarin23}Mehdizadeh-Somarin23 \href{https://doi.org/10.1007/978-3-031-43670-3_33}{Mehdizadeh-Somarin23} & \hyperref[auth:a432]{Z. Mehdizadeh{-}Somarin}, \hyperref[auth:a433]{R. Tavakkoli{-}Moghaddam}, \hyperref[auth:a434]{M. Rohaninejad}, \hyperref[auth:a116]{Z. Hanz{\'{a}}lek}, \hyperref[auth:a435]{Behdin Vahedi Nouri} & A Constraint Programming Model for a Reconfigurable Job Shop Scheduling Problem with Machine Availability & \href{../works/Mehdizadeh-Somarin23.pdf}{Yes} & \cite{Mehdizadeh-Somarin23} & 2023 & APMS 2023 & 14 & 0 & 0 & \ref{b:Mehdizadeh-Somarin23} & \ref{c:Mehdizadeh-Somarin23}\\
\rowlabel{a:PerezGSL23}PerezGSL23 \href{https://doi.org/10.1109/ICTAI59109.2023.00108}{PerezGSL23} & \hyperref[auth:a428]{G. Perez}, \hyperref[auth:a429]{G. Glorian}, \hyperref[auth:a430]{W. Suijlen}, \hyperref[auth:a431]{A. Lallouet} & A Constraint Programming Model for Scheduling the Unloading of Trains in Ports & \href{../works/PerezGSL23.pdf}{Yes} & \cite{PerezGSL23} & 2023 & ICTAI 2023 & 7 & 0 & 0 & \ref{b:PerezGSL23} & \ref{c:PerezGSL23}\\
\rowlabel{a:PovedaAA23}PovedaAA23 \href{https://doi.org/10.4230/LIPIcs.CP.2023.31}{PovedaAA23} & \hyperref[auth:a4]{G. Pov{\'{e}}da}, \hyperref[auth:a5]{N. {\'{A}}lvarez}, \hyperref[auth:a6]{C. Artigues} & Partially Preemptive Multi Skill/Mode Resource-Constrained Project Scheduling with Generalized Precedence Relations and Calendars & \href{../works/PovedaAA23.pdf}{Yes} & \cite{PovedaAA23} & 2023 & CP 2023 & 21 & 0 & 0 & \ref{b:PovedaAA23} & \ref{c:PovedaAA23}\\
\rowlabel{a:SquillaciPR23}SquillaciPR23 \href{https://doi.org/10.1007/978-3-031-33271-5_29}{SquillaciPR23} & \hyperref[auth:a20]{S. Squillaci}, \hyperref[auth:a21]{C. Pralet}, \hyperref[auth:a22]{S. Roussel} & Scheduling Complex Observation Requests for a Constellation of Satellites: Large Neighborhood Search Approaches & \href{../works/SquillaciPR23.pdf}{Yes} & \cite{SquillaciPR23} & 2023 & CPAIOR 2023 & 17 & 0 & 19 & \ref{b:SquillaciPR23} & \ref{c:SquillaciPR23}\\
\rowlabel{a:TardivoDFMP23}TardivoDFMP23 \href{https://doi.org/10.1007/978-3-031-33271-5_22}{TardivoDFMP23} & \hyperref[auth:a29]{F. Tardivo}, \hyperref[auth:a30]{A. Dovier}, \hyperref[auth:a31]{A. Formisano}, \hyperref[auth:a32]{L. Michel}, \hyperref[auth:a33]{E. Pontelli} & Constraint Propagation on {GPU:} {A} Case Study for the Cumulative Constraint & \href{../works/TardivoDFMP23.pdf}{Yes} & \cite{TardivoDFMP23} & 2023 & CPAIOR 2023 & 18 & 0 & 30 & \ref{b:TardivoDFMP23} & \ref{c:TardivoDFMP23}\\
\rowlabel{a:TasselGS23}TasselGS23 \href{https://doi.org/10.1609/icaps.v33i1.27243}{TasselGS23} & \hyperref[auth:a58]{P. Tassel}, \hyperref[auth:a61]{M. Gebser}, \hyperref[auth:a426]{K. Schekotihin} & An End-to-End Reinforcement Learning Approach for Job-Shop Scheduling Problems Based on Constraint Programming & \href{../works/TasselGS23.pdf}{Yes} & \cite{TasselGS23} & 2023 & ICAPS 2023 & 9 & 0 & 0 & \ref{b:TasselGS23} & \ref{c:TasselGS23}\\
\rowlabel{a:WangB23}WangB23 \href{https://doi.org/10.1109/ICTAI59109.2023.00062}{WangB23} & \hyperref[auth:a396]{R. Wang}, \hyperref[auth:a397]{N. Barnier} & Dynamic All-Different and Maximal Cliques Constraints for Fixed Job Scheduling & \href{../works/WangB23.pdf}{Yes} & \cite{WangB23} & 2023 & ICTAI 2023 & 8 & 0 & 0 & \ref{b:WangB23} & \ref{c:WangB23}\\
\rowlabel{a:YuraszeckMC23}YuraszeckMC23 \href{https://doi.org/10.1016/j.procs.2023.03.130}{YuraszeckMC23} & \hyperref[auth:a408]{F. Yuraszeck}, \hyperref[auth:a427]{G. Mej{\'{\i}}a}, \hyperref[auth:a410]{D. Canut{-}de{-}Bon} & A competitive constraint programming approach for the group shop scheduling problem & \href{../works/YuraszeckMC23.pdf}{Yes} & \cite{YuraszeckMC23} & 2023 & ANT 2023 & 6 & 1 & 15 & \ref{b:YuraszeckMC23} & \ref{c:YuraszeckMC23}\\
\rowlabel{a:ArmstrongGOS22}ArmstrongGOS22 \href{https://doi.org/10.1007/978-3-031-08011-1_1}{ArmstrongGOS22} & \hyperref[auth:a14]{E. Armstrong}, \hyperref[auth:a15]{M. Garraffa}, \hyperref[auth:a16]{B. O'Sullivan}, \hyperref[auth:a17]{H. Simonis} & A Two-Phase Hybrid Approach for the Hybrid Flexible Flowshop with Transportation Times & \href{../works/ArmstrongGOS22.pdf}{Yes} & \cite{ArmstrongGOS22} & 2022 & CPAIOR 2022 & 13 & 0 & 14 & \ref{b:ArmstrongGOS22} & \ref{c:ArmstrongGOS22}\\
\rowlabel{a:BoudreaultSLQ22}BoudreaultSLQ22 \href{https://doi.org/10.4230/LIPIcs.CP.2022.10}{BoudreaultSLQ22} & \hyperref[auth:a34]{R. Boudreault}, \hyperref[auth:a35]{V. Simard}, \hyperref[auth:a36]{D. Lafond}, \hyperref[auth:a37]{C. Quimper} & A Constraint Programming Approach to Ship Refit Project Scheduling & \href{../works/BoudreaultSLQ22.pdf}{Yes} & \cite{BoudreaultSLQ22} & 2022 & CP 2022 & 16 & 0 & 0 & \ref{b:BoudreaultSLQ22} & \ref{c:BoudreaultSLQ22}\\
\rowlabel{a:GeitzGSSW22}GeitzGSSW22 \href{https://doi.org/10.1007/978-3-031-08011-1_10}{GeitzGSSW22} & \hyperref[auth:a47]{M. Geitz}, \hyperref[auth:a48]{C. Grozea}, \hyperref[auth:a49]{W. Steigerwald}, \hyperref[auth:a50]{R. St{\"{o}}hr}, \hyperref[auth:a51]{A. Wolf} & Solving the Extended Job Shop Scheduling Problem with AGVs - Classical and Quantum Approaches & \href{../works/GeitzGSSW22.pdf}{Yes} & \cite{GeitzGSSW22} & 2022 & CPAIOR 2022 & 18 & 0 & 24 & \ref{b:GeitzGSSW22} & \ref{c:GeitzGSSW22}\\
\rowlabel{a:HebrardALLCMR22}HebrardALLCMR22 \href{https://doi.org/10.24963/ijcai.2022/643}{HebrardALLCMR22} & \hyperref[auth:a1]{E. Hebrard}, \hyperref[auth:a6]{C. Artigues}, \hyperref[auth:a3]{P. Lopez}, \hyperref[auth:a791]{A. Lusson}, \hyperref[auth:a792]{Steve A. Chien}, \hyperref[auth:a793]{A. Maillard}, \hyperref[auth:a794]{Gregg R. Rabideau} & An Efficient Approach to Data Transfer Scheduling for Long Range Space Exploration & \href{../works/HebrardALLCMR22.pdf}{Yes} & \cite{HebrardALLCMR22} & 2022 & IJCAI 2022 & 7 & 0 & 0 & \ref{b:HebrardALLCMR22} & n/a\\
\rowlabel{a:JungblutK22}JungblutK22 \href{https://doi.org/10.1109/IPDPSW55747.2022.00025}{JungblutK22} & \hyperref[auth:a746]{P. Jungblut}, \hyperref[auth:a747]{D. Kranzlm{\"{u}}ller} & Optimal Schedules for High-Level Programming Environments on FPGAs with Constraint Programming & \href{../works/JungblutK22.pdf}{Yes} & \cite{JungblutK22} & 2022 & IPDPS 2022 & 4 & 0 & 0 & \ref{b:JungblutK22} & \ref{c:JungblutK22}\\
\rowlabel{a:KotaryFH22}KotaryFH22 \href{https://doi.org/10.1609/aaai.v36i7.20685}{KotaryFH22} & \hyperref[auth:a1385]{J. Kotary}, \hyperref[auth:a1386]{F. Fioretto}, \hyperref[auth:a149]{Pascal Van Hentenryck} & Fast Approximations for Job Shop Scheduling: {A} Lagrangian Dual Deep Learning Method & \href{../works/KotaryFH22.pdf}{Yes} & \cite{KotaryFH22} & 2022 & AAAI 2022 & 8 & 0 & 0 & \ref{b:KotaryFH22} & \ref{c:KotaryFH22}\\
\rowlabel{a:LiFJZLL22}LiFJZLL22 \href{https://doi.org/10.1109/ICNSC55942.2022.10004158}{LiFJZLL22} & \hyperref[auth:a463]{X. Li}, \hyperref[auth:a464]{J. Fu}, \hyperref[auth:a465]{Z. Jia}, \hyperref[auth:a466]{Z. Zhao}, \hyperref[auth:a467]{S. Li}, \hyperref[auth:a468]{S. Liu} & Constraint Programming for a Novel Integrated Optimization of Blocking Job Shop Scheduling and Variable-Speed Transfer Robot Assignment & \href{../works/LiFJZLL22.pdf}{Yes} & \cite{LiFJZLL22} & 2022 & ICNSC 2022 & 6 & 0 & 31 & \ref{b:LiFJZLL22} & \ref{c:LiFJZLL22}\\
\rowlabel{a:LuoB22}LuoB22 \href{https://doi.org/10.1007/978-3-031-08011-1_17}{LuoB22} & \hyperref[auth:a751]{Yiqing L. Luo}, \hyperref[auth:a89]{J. Christopher Beck} & Packing by Scheduling: Using Constraint Programming to Solve a Complex 2D Cutting Stock Problem & \href{../works/LuoB22.pdf}{Yes} & \cite{LuoB22} & 2022 & CPAIOR 2022 & 17 & 0 & 28 & \ref{b:LuoB22} & \ref{c:LuoB22}\\
\rowlabel{a:OuelletQ22}OuelletQ22 \href{https://doi.org/10.1007/978-3-031-08011-1_21}{OuelletQ22} & \hyperref[auth:a52]{Y. Ouellet}, \hyperref[auth:a37]{C. Quimper} & A MinCumulative Resource Constraint & \href{../works/OuelletQ22.pdf}{Yes} & \cite{OuelletQ22} & 2022 & CPAIOR 2022 & 17 & 1 & 22 & \ref{b:OuelletQ22} & \ref{c:OuelletQ22}\\
\rowlabel{a:OujanaAYB22}OujanaAYB22 \href{https://doi.org/10.1109/CoDIT55151.2022.9803972}{OujanaAYB22} & \hyperref[auth:a456]{S. Oujana}, \hyperref[auth:a457]{L. Amodeo}, \hyperref[auth:a458]{F. Yalaoui}, \hyperref[auth:a459]{D. Brodart} & Solving a realistic hybrid and flexible flow shop scheduling problem through constraint programming: industrial case in a packaging company & \href{../works/OujanaAYB22.pdf}{Yes} & \cite{OujanaAYB22} & 2022 & CoDIT 2022 & 6 & 1 & 21 & \ref{b:OujanaAYB22} & \ref{c:OujanaAYB22}\\
\rowlabel{a:PopovicCGNC22}PopovicCGNC22 \href{https://doi.org/10.4230/LIPIcs.CP.2022.34}{PopovicCGNC22} & \hyperref[auth:a38]{L. Popovic}, \hyperref[auth:a39]{A. C{\^{o}}t{\'{e}}}, \hyperref[auth:a40]{M. Gaha}, \hyperref[auth:a41]{F. Nguewouo}, \hyperref[auth:a42]{Q. Cappart} & Scheduling the Equipment Maintenance of an Electric Power Transmission Network Using Constraint Programming & \href{../works/PopovicCGNC22.pdf}{Yes} & \cite{PopovicCGNC22} & 2022 & CP 2022 & 15 & 0 & 0 & \ref{b:PopovicCGNC22} & \ref{c:PopovicCGNC22}\\
\rowlabel{a:SvancaraB22}SvancaraB22 \href{https://doi.org/10.5220/0010869700003116}{SvancaraB22} & \hyperref[auth:a784]{J. Svancara}, \hyperref[auth:a153]{R. Bart{\'{a}}k} & Tackling Train Routing via Multi-agent Pathfinding and Constraint-based Scheduling & \href{../works/SvancaraB22.pdf}{Yes} & \cite{SvancaraB22} & 2022 & ICAART 2022 & 8 & 0 & 0 & \ref{b:SvancaraB22} & n/a\\
\rowlabel{a:Teppan22}Teppan22 \href{https://doi.org/10.5220/0010849900003116}{Teppan22} & \hyperref[auth:a94]{Erich Christian Teppan} & Types of Flexible Job Shop Scheduling: {A} Constraint Programming Experiment & \href{../works/Teppan22.pdf}{Yes} & \cite{Teppan22} & 2022 & ICAART 2022 & 8 & 0 & 0 & \ref{b:Teppan22} & \ref{c:Teppan22}\\
\rowlabel{a:TouatBT22}TouatBT22 \href{}{TouatBT22} & \hyperref[auth:a460]{M. Touat}, \hyperref[auth:a461]{B. Benhamou}, \hyperref[auth:a462]{Fatima Benbouzid{-}Si Tayeb} & A Constraint Programming Model for the Scheduling Problem with Flexible Maintenance under Human Resource Constraints & \href{../works/TouatBT22.pdf}{Yes} & \cite{TouatBT22} & 2022 & ICAART 2022 & 8 & 0 & 0 & \ref{b:TouatBT22} & \ref{c:TouatBT22}\\
\rowlabel{a:WinterMMW22}WinterMMW22 \href{https://doi.org/10.4230/LIPIcs.CP.2022.41}{WinterMMW22} & \hyperref[auth:a43]{F. Winter}, \hyperref[auth:a44]{S. Meiswinkel}, \hyperref[auth:a45]{N. Musliu}, \hyperref[auth:a46]{D. Walkiewicz} & Modeling and Solving Parallel Machine Scheduling with Contamination Constraints in the Agricultural Industry & \href{../works/WinterMMW22.pdf}{Yes} & \cite{WinterMMW22} & 2022 & CP 2022 & 18 & 0 & 0 & \ref{b:WinterMMW22} & \ref{c:WinterMMW22}\\
\rowlabel{a:ZhangBB22}ZhangBB22 \href{https://ojs.aaai.org/index.php/ICAPS/article/view/19826}{ZhangBB22} & \hyperref[auth:a803]{J. Zhang}, \hyperref[auth:a804]{Giovanni Lo Bianco}, \hyperref[auth:a89]{J. Christopher Beck} & Solving Job-Shop Scheduling Problems with QUBO-Based Specialized Hardware & \href{../works/ZhangBB22.pdf}{Yes} & \cite{ZhangBB22} & 2022 & ICAPS 2022 & 9 & 1 & 0 & \ref{b:ZhangBB22} & n/a\\
\rowlabel{a:ZhangJZL22}ZhangJZL22 \href{https://doi.org/10.1109/ICNSC55942.2022.10004154}{ZhangJZL22} & \hyperref[auth:a469]{H. Zhang}, \hyperref[auth:a470]{Y. Ji}, \hyperref[auth:a466]{Z. Zhao}, \hyperref[auth:a468]{S. Liu} & Constraint Programming for Modeling and Solving a Hybrid Flow Shop Scheduling Problem & \href{../works/ZhangJZL22.pdf}{Yes} & \cite{ZhangJZL22} & 2022 & ICNSC 2022 & 6 & 0 & 21 & \ref{b:ZhangJZL22} & \ref{c:ZhangJZL22}\\
\rowlabel{a:AntuoriHHEN21}AntuoriHHEN21 \href{https://doi.org/10.4230/LIPIcs.CP.2021.14}{AntuoriHHEN21} & \hyperref[auth:a53]{V. Antuori}, \hyperref[auth:a1]{E. Hebrard}, \hyperref[auth:a54]{M. Huguet}, \hyperref[auth:a55]{S. Essodaigui}, \hyperref[auth:a56]{A. Nguyen} & Combining Monte Carlo Tree Search and Depth First Search Methods for a Car Manufacturing Workshop Scheduling Problem & \href{../works/AntuoriHHEN21.pdf}{Yes} & \cite{AntuoriHHEN21} & 2021 & CP 2021 & 16 & 0 & 0 & \ref{b:AntuoriHHEN21} & \ref{c:AntuoriHHEN21}\\
\rowlabel{a:ArmstrongGOS21}ArmstrongGOS21 \href{https://doi.org/10.4230/LIPIcs.CP.2021.16}{ArmstrongGOS21} & \hyperref[auth:a14]{E. Armstrong}, \hyperref[auth:a15]{M. Garraffa}, \hyperref[auth:a16]{B. O'Sullivan}, \hyperref[auth:a17]{H. Simonis} & The Hybrid Flexible Flowshop with Transportation Times & \href{../works/ArmstrongGOS21.pdf}{Yes} & \cite{ArmstrongGOS21} & 2021 & CP 2021 & 18 & 1 & 0 & \ref{b:ArmstrongGOS21} & \ref{c:ArmstrongGOS21}\\
\rowlabel{a:ArtiguesHQT21}ArtiguesHQT21 \href{https://doi.org/10.5220/0010190101290136}{ArtiguesHQT21} & \hyperref[auth:a6]{C. Artigues}, \hyperref[auth:a1]{E. Hebrard}, \hyperref[auth:a795]{A. Quilliot}, \hyperref[auth:a796]{H. Toussaint} & Multi-Mode {RCPSP} with Safety Margin Maximization: Models and Algorithms & \href{../works/ArtiguesHQT21.pdf}{Yes} & \cite{ArtiguesHQT21} & 2021 & ICORES 2021 & 8 & 0 & 0 & \ref{b:ArtiguesHQT21} & n/a\\
\rowlabel{a:Astrand0F21}Astrand0F21 \href{https://doi.org/10.1007/978-3-030-78230-6_23}{Astrand0F21} & \hyperref[auth:a74]{M. {\AA}strand}, \hyperref[auth:a75]{M. Johansson}, \hyperref[auth:a76]{Hamid Reza Feyzmahdavian} & Short-Term Scheduling of Production Fleets in Underground Mines Using CP-Based {LNS} & \href{../works/Astrand0F21.pdf}{Yes} & \cite{Astrand0F21} & 2021 & CPAIOR 2021 & 18 & 2 & 25 & \ref{b:Astrand0F21} & \ref{c:Astrand0F21}\\
\rowlabel{a:BenderWS21}BenderWS21 \href{https://doi.org/10.1007/978-3-030-87672-2_37}{BenderWS21} & \hyperref[auth:a496]{T. Bender}, \hyperref[auth:a497]{D. Wittwer}, \hyperref[auth:a498]{T. Schmidt} & Applying Constraint Programming to the Multi-mode Scheduling Problem in Harvest Logistics & \href{../works/BenderWS21.pdf}{Yes} & \cite{BenderWS21} & 2021 & ICCL 2021 & 16 & 1 & 16 & \ref{b:BenderWS21} & \ref{c:BenderWS21}\\
\rowlabel{a:GeibingerKKMMW21}GeibingerKKMMW21 \href{https://doi.org/10.1007/978-3-030-78230-6_29}{GeibingerKKMMW21} & \hyperref[auth:a77]{T. Geibinger}, \hyperref[auth:a78]{L. Kletzander}, \hyperref[auth:a79]{M. Krainz}, \hyperref[auth:a80]{F. Mischek}, \hyperref[auth:a45]{N. Musliu}, \hyperref[auth:a43]{F. Winter} & Physician Scheduling During a Pandemic & \href{../works/GeibingerKKMMW21.pdf}{Yes} & \cite{GeibingerKKMMW21} & 2021 & CPAIOR 2021 & 10 & 0 & 6 & \ref{b:GeibingerKKMMW21} & \ref{c:GeibingerKKMMW21}\\
\rowlabel{a:GeibingerMM21}GeibingerMM21 \href{https://doi.org/10.1609/aaai.v35i7.16789}{GeibingerMM21} & \hyperref[auth:a77]{T. Geibinger}, \hyperref[auth:a80]{F. Mischek}, \hyperref[auth:a45]{N. Musliu} & Constraint Logic Programming for Real-World Test Laboratory Scheduling & \href{../works/GeibingerMM21.pdf}{Yes} & \cite{GeibingerMM21} & 2021 & AAAI 2021 & 9 & 0 & 0 & \ref{b:GeibingerMM21} & \ref{c:GeibingerMM21}\\
\rowlabel{a:HanenKP21}HanenKP21 \href{https://doi.org/10.1007/978-3-030-78230-6_14}{HanenKP21} & \hyperref[auth:a71]{C. Hanen}, \hyperref[auth:a72]{Alix Munier Kordon}, \hyperref[auth:a73]{T. Pedersen} & Two Deadline Reduction Algorithms for Scheduling Dependent Tasks on Parallel Processors & \href{../works/HanenKP21.pdf}{Yes} & \cite{HanenKP21} & 2021 & CPAIOR 2021 & 17 & 1 & 24 & \ref{b:HanenKP21} & \ref{c:HanenKP21}\\
\rowlabel{a:HillTV21}HillTV21 \href{https://doi.org/10.1007/978-3-030-78230-6_2}{HillTV21} & \hyperref[auth:a64]{A. Hill}, \hyperref[auth:a65]{J. Ticktin}, \hyperref[auth:a66]{Thomas W. M. Vossen} & A Computational Study of Constraint Programming Approaches for Resource-Constrained Project Scheduling with Autonomous Learning Effects & \href{../works/HillTV21.pdf}{Yes} & \cite{HillTV21} & 2021 & CPAIOR 2021 & 19 & 0 & 38 & \ref{b:HillTV21} & \ref{c:HillTV21}\\
\rowlabel{a:KlankeBYE21}KlankeBYE21 \href{https://doi.org/10.1007/978-3-030-78230-6_9}{KlankeBYE21} & \hyperref[auth:a67]{C. Klanke}, \hyperref[auth:a68]{Dominik R. Bleidorn}, \hyperref[auth:a69]{V. Yfantis}, \hyperref[auth:a70]{S. Engell} & Combining Constraint Programming and Temporal Decomposition Approaches - Scheduling of an Industrial Formulation Plant & \href{../works/KlankeBYE21.pdf}{Yes} & \cite{KlankeBYE21} & 2021 & CPAIOR 2021 & 16 & 3 & 13 & \ref{b:KlankeBYE21} & \ref{c:KlankeBYE21}\\
\rowlabel{a:KletzanderMH21}KletzanderMH21 \href{https://doi.org/10.1609/aaai.v35i13.17408}{KletzanderMH21} & \hyperref[auth:a78]{L. Kletzander}, \hyperref[auth:a45]{N. Musliu}, \hyperref[auth:a149]{Pascal Van Hentenryck} & Branch and Price for Bus Driver Scheduling with Complex Break Constraints & \href{../works/KletzanderMH21.pdf}{Yes} & \cite{KletzanderMH21} & 2021 & AAAI 2021 & 9 & 2 & 0 & \ref{b:KletzanderMH21} & n/a\\
\rowlabel{a:KovacsTKSG21}KovacsTKSG21 \href{https://doi.org/10.4230/LIPIcs.CP.2021.36}{KovacsTKSG21} & \hyperref[auth:a57]{B. Kov{\'{a}}cs}, \hyperref[auth:a58]{P. Tassel}, \hyperref[auth:a59]{W. Kohlenbrein}, \hyperref[auth:a60]{P. Schrott{-}Kostwein}, \hyperref[auth:a61]{M. Gebser} & Utilizing Constraint Optimization for Industrial Machine Workload Balancing & \href{../works/KovacsTKSG21.pdf}{Yes} & \cite{KovacsTKSG21} & 2021 & CP 2021 & 17 & 0 & 0 & \ref{b:KovacsTKSG21} & \ref{c:KovacsTKSG21}\\
\rowlabel{a:LacknerMMWW21}LacknerMMWW21 \href{https://doi.org/10.4230/LIPIcs.CP.2021.37}{LacknerMMWW21} & \hyperref[auth:a62]{M. Lackner}, \hyperref[auth:a63]{C. Mrkvicka}, \hyperref[auth:a45]{N. Musliu}, \hyperref[auth:a46]{D. Walkiewicz}, \hyperref[auth:a43]{F. Winter} & Minimizing Cumulative Batch Processing Time for an Industrial Oven Scheduling Problem & \href{../works/LacknerMMWW21.pdf}{Yes} & \cite{LacknerMMWW21} & 2021 & CP 2021 & 18 & 0 & 0 & \ref{b:LacknerMMWW21} & \ref{c:LacknerMMWW21}\\
\rowlabel{a:AntuoriHHEN20}AntuoriHHEN20 \href{https://doi.org/10.1007/978-3-030-58475-7_38}{AntuoriHHEN20} & \hyperref[auth:a53]{V. Antuori}, \hyperref[auth:a1]{E. Hebrard}, \hyperref[auth:a54]{M. Huguet}, \hyperref[auth:a55]{S. Essodaigui}, \hyperref[auth:a56]{A. Nguyen} & Leveraging Reinforcement Learning, Constraint Programming and Local Search: {A} Case Study in Car Manufacturing & \href{../works/AntuoriHHEN20.pdf}{Yes} & \cite{AntuoriHHEN20} & 2020 & CP 2020 & 16 & 3 & 8 & \ref{b:AntuoriHHEN20} & \ref{c:AntuoriHHEN20}\\
\rowlabel{a:BarzegaranZP20}BarzegaranZP20 \href{https://doi.org/10.4230/OASIcs.Fog-IoT.2020.3}{BarzegaranZP20} & \hyperref[auth:a524]{M. Barzegaran}, \hyperref[auth:a525]{B. Zarrin}, \hyperref[auth:a526]{P. Pop} & Quality-Of-Control-Aware Scheduling of Communication in TSN-Based Fog Computing Platforms Using Constraint Programming & \href{../works/BarzegaranZP20.pdf}{Yes} & \cite{BarzegaranZP20} & 2020 & Fog-IoT 2020 & 9 & 0 & 0 & \ref{b:BarzegaranZP20} & \ref{c:BarzegaranZP20}\\
\rowlabel{a:GodetLHS20}GodetLHS20 \href{https://doi.org/10.1609/aaai.v34i02.5510}{GodetLHS20} & \hyperref[auth:a474]{A. Godet}, \hyperref[auth:a246]{X. Lorca}, \hyperref[auth:a1]{E. Hebrard}, \hyperref[auth:a127]{G. Simonin} & Using Approximation within Constraint Programming to Solve the Parallel Machine Scheduling Problem with Additional Unit Resources & \href{../works/GodetLHS20.pdf}{Yes} & \cite{GodetLHS20} & 2020 & AAAI 2020 & 8 & 1 & 0 & \ref{b:GodetLHS20} & \ref{c:GodetLHS20}\\
\rowlabel{a:GokGSTO20}GokGSTO20 \href{https://doi.org/10.1007/978-3-030-58942-4_15}{GokGSTO20} & \hyperref[auth:a1027]{Yagmur S. G\"{o}k}, \hyperref[auth:a1025]{D. Guimarans}, \hyperref[auth:a126]{Peter J. Stuckey}, \hyperref[auth:a1024]{M. Tomasella}, \hyperref[auth:a136]{C. {\"{O}}zt{\"{u}}rk} & Robust Resource Planning for Aircraft Ground Operations & \href{../works/GokGSTO20.pdf}{Yes} & \cite{GokGSTO20} & 2020 & CPAIOR 2020 & 17 & 2 & 14 & \ref{b:GokGSTO20} & n/a\\
\rowlabel{a:GroleazNS20}GroleazNS20 \href{https://doi.org/10.1007/978-3-030-58475-7_36}{GroleazNS20} & \hyperref[auth:a83]{L. Groleaz}, \hyperref[auth:a84]{Samba Ndojh Ndiaye}, \hyperref[auth:a85]{C. Solnon} & Solving the Group Cumulative Scheduling Problem with {CPO} and {ACO} & \href{../works/GroleazNS20.pdf}{Yes} & \cite{GroleazNS20} & 2020 & CP 2020 & 17 & 1 & 25 & \ref{b:GroleazNS20} & \ref{c:GroleazNS20}\\
\rowlabel{a:GroleazNS20a}GroleazNS20a \href{https://doi.org/10.1145/3377930.3389818}{GroleazNS20a} & \hyperref[auth:a83]{L. Groleaz}, \hyperref[auth:a84]{Samba Ndojh Ndiaye}, \hyperref[auth:a85]{C. Solnon} & {ACO} with automatic parameter selection for a scheduling problem with a group cumulative constraint & \href{../works/GroleazNS20a.pdf}{Yes} & \cite{GroleazNS20a} & 2020 & GECCO 2020 & 9 & 3 & 28 & \ref{b:GroleazNS20a} & \ref{c:GroleazNS20a}\\
\rowlabel{a:KletzanderM20}KletzanderM20 \href{https://ojs.aaai.org/index.php/ICAPS/article/view/6688}{KletzanderM20} & \hyperref[auth:a78]{L. Kletzander}, \hyperref[auth:a45]{N. Musliu} & Solving Large Real-Life Bus Driver Scheduling Problems with Complex Break Constraints & \href{../works/KletzanderM20.pdf}{Yes} & \cite{KletzanderM20} & 2020 & ICAPS 2020 & 10 & 0 & 0 & \ref{b:KletzanderM20} & n/a\\
\rowlabel{a:Mercier-AubinGQ20}Mercier-AubinGQ20 \href{https://doi.org/10.1007/978-3-030-58942-4_22}{Mercier-AubinGQ20} & \hyperref[auth:a86]{A. Mercier{-}Aubin}, \hyperref[auth:a87]{J. Gaudreault}, \hyperref[auth:a37]{C. Quimper} & Leveraging Constraint Scheduling: {A} Case Study to the Textile Industry & \href{../works/Mercier-AubinGQ20.pdf}{Yes} & \cite{Mercier-AubinGQ20} & 2020 & CPAIOR 2020 & 13 & 2 & 13 & \ref{b:Mercier-AubinGQ20} & \ref{c:Mercier-AubinGQ20}\\
\rowlabel{a:NattafM20}NattafM20 \href{https://doi.org/10.1007/978-3-030-58475-7_27}{NattafM20} & \hyperref[auth:a81]{M. Nattaf}, \hyperref[auth:a82]{A. Malapert} & Filtering Rules for Flow Time Minimization in a Parallel Machine Scheduling Problem & \href{../works/NattafM20.pdf}{Yes} & \cite{NattafM20} & 2020 & CP 2020 & 16 & 0 & 6 & \ref{b:NattafM20} & \ref{c:NattafM20}\\
\rowlabel{a:TangB20}TangB20 \href{https://doi.org/10.1007/978-3-030-58942-4_28}{TangB20} & \hyperref[auth:a88]{Tanya Y. Tang}, \hyperref[auth:a89]{J. Christopher Beck} & {CP} and Hybrid Models for Two-Stage Batching and Scheduling & \href{../works/TangB20.pdf}{Yes} & \cite{TangB20} & 2020 & CPAIOR 2020 & 16 & 6 & 12 & \ref{b:TangB20} & \ref{c:TangB20}\\
\rowlabel{a:ThomasKS20}ThomasKS20 \href{https://doi.org/10.1007/978-3-030-58942-4_30}{ThomasKS20} & \hyperref[auth:a841]{C. Thomas}, \hyperref[auth:a10]{R. Kameugne}, \hyperref[auth:a148]{P. Schaus} & Insertion Sequence Variables for Hybrid Routing and Scheduling Problems & \href{../works/ThomasKS20.pdf}{Yes} & \cite{ThomasKS20} & 2020 & CPAIOR 2020 & 18 & 0 & 16 & \ref{b:ThomasKS20} & \ref{c:ThomasKS20}\\
\rowlabel{a:WangB20}WangB20 \href{https://doi.org/10.3233/FAIA200114}{WangB20} & \hyperref[auth:a396]{R. Wang}, \hyperref[auth:a397]{N. Barnier} & Global Propagation of Transition Cost for Fixed Job Scheduling & \href{../works/WangB20.pdf}{Yes} & \cite{WangB20} & 2020 & ECAI 2020 & 8 & 0 & 0 & \ref{b:WangB20} & \ref{c:WangB20}\\
\rowlabel{a:WessenCS20}WessenCS20 \href{https://doi.org/10.1007/978-3-030-58942-4_33}{WessenCS20} & \hyperref[auth:a90]{J. Wess{\'{e}}n}, \hyperref[auth:a91]{M. Carlsson}, \hyperref[auth:a92]{C. Schulte} & Scheduling of Dual-Arm Multi-tool Assembly Robots and Workspace Layout Optimization & \href{../works/WessenCS20.pdf}{Yes} & \cite{WessenCS20} & 2020 & CPAIOR 2020 & 10 & 2 & 11 & \ref{b:WessenCS20} & \ref{c:WessenCS20}\\
\rowlabel{a:BadicaBIL19}BadicaBIL19 \href{https://doi.org/10.1007/978-3-030-32258-8_17}{BadicaBIL19} & \hyperref[auth:a500]{A. Badica}, \hyperref[auth:a501]{C. Badica}, \hyperref[auth:a502]{M. Ivanovic}, \hyperref[auth:a546]{D. Logofatu} & Exploring the Space of Block Structured Scheduling Processes Using Constraint Logic Programming & \href{../works/BadicaBIL19.pdf}{Yes} & \cite{BadicaBIL19} & 2019 & IDC 2019 & 11 & 2 & 6 & \ref{b:BadicaBIL19} & \ref{c:BadicaBIL19}\\
\rowlabel{a:BehrensLM19}BehrensLM19 \href{https://doi.org/10.1109/ICRA.2019.8794022}{BehrensLM19} & \hyperref[auth:a543]{Jan Kristof Behrens}, \hyperref[auth:a544]{R. Lange}, \hyperref[auth:a545]{M. Mansouri} & A Constraint Programming Approach to Simultaneous Task Allocation and Motion Scheduling for Industrial Dual-Arm Manipulation Tasks & \href{../works/BehrensLM19.pdf}{Yes} & \cite{BehrensLM19} & 2019 & ICRA 2019 & 7 & 12 & 18 & \ref{b:BehrensLM19} & \ref{c:BehrensLM19}\\
\rowlabel{a:BogaerdtW19}BogaerdtW19 \href{https://doi.org/10.1007/978-3-030-19212-9_38}{BogaerdtW19} & \hyperref[auth:a309]{Pim van den Bogaerdt}, \hyperref[auth:a310]{Mathijs de Weerdt} & Lower Bounds for Uniform Machine Scheduling Using Decision Diagrams & \href{../works/BogaerdtW19.pdf}{Yes} & \cite{BogaerdtW19} & 2019 & CPAIOR 2019 & 16 & 1 & 16 & \ref{b:BogaerdtW19} & \ref{c:BogaerdtW19}\\
\rowlabel{a:ColT19}ColT19 \href{https://doi.org/10.1007/978-3-030-30048-7_9}{ColT19} & \hyperref[auth:a93]{Giacomo Da Col}, \hyperref[auth:a94]{Erich Christian Teppan} & Industrial Size Job Shop Scheduling Tackled by Present Day {CP} Solvers & \href{../works/ColT19.pdf}{Yes} & \cite{ColT19} & 2019 & CP 2019 & 17 & 11 & 12 & \ref{b:ColT19} & \ref{c:ColT19}\\
\rowlabel{a:FrimodigS19}FrimodigS19 \href{https://doi.org/10.1007/978-3-030-30048-7_25}{FrimodigS19} & \hyperref[auth:a95]{S. Frimodig}, \hyperref[auth:a92]{C. Schulte} & Models for Radiation Therapy Patient Scheduling & \href{../works/FrimodigS19.pdf}{Yes} & \cite{FrimodigS19} & 2019 & CP 2019 & 17 & 3 & 26 & \ref{b:FrimodigS19} & \ref{c:FrimodigS19}\\
\rowlabel{a:FrohnerTR19}FrohnerTR19 \href{https://doi.org/10.1007/978-3-030-45093-9_34}{FrohnerTR19} & \hyperref[auth:a540]{N. Frohner}, \hyperref[auth:a541]{S. Teuschl}, \hyperref[auth:a345]{G{\"{u}}nther R. Raidl} & Casual Employee Scheduling with Constraint Programming and Metaheuristics & \href{../works/FrohnerTR19.pdf}{Yes} & \cite{FrohnerTR19} & 2019 & EUROCAST 2019 & 9 & 0 & 6 & \ref{b:FrohnerTR19} & n/a\\
\rowlabel{a:GalleguillosKSB19}GalleguillosKSB19 \href{https://doi.org/10.1007/978-3-030-30048-7_26}{GalleguillosKSB19} & \hyperref[auth:a96]{C. Galleguillos}, \hyperref[auth:a97]{Z. Kiziltan}, \hyperref[auth:a98]{A. S{\^{\i}}rbu}, \hyperref[auth:a99]{{\"{O}}zalp Babaoglu} & Constraint Programming-Based Job Dispatching for Modern {HPC} Applications & \href{../works/GalleguillosKSB19.pdf}{Yes} & \cite{GalleguillosKSB19} & 2019 & CP 2019 & 18 & 1 & 27 & \ref{b:GalleguillosKSB19} & \ref{c:GalleguillosKSB19}\\
\rowlabel{a:GeibingerMM19}GeibingerMM19 \href{https://doi.org/10.1007/978-3-030-19212-9_20}{GeibingerMM19} & \hyperref[auth:a77]{T. Geibinger}, \hyperref[auth:a80]{F. Mischek}, \hyperref[auth:a45]{N. Musliu} & Investigating Constraint Programming for Real World Industrial Test Laboratory Scheduling & \href{../works/GeibingerMM19.pdf}{Yes} & \cite{GeibingerMM19} & 2019 & CPAIOR 2019 & 16 & 6 & 15 & \ref{b:GeibingerMM19} & n/a\\
\rowlabel{a:KucukY19}KucukY19 \href{https://api.semanticscholar.org/CorpusID:198146161}{KucukY19} & \hyperref[auth:a768]{M. K{\"u}ç{\"u}k}, \hyperref[auth:a424]{Seyda Topaloglu Yildiz} & A Constraint Programming Approach for Agile Earth Observation Satellite Scheduling Problem & \href{../works/KucukY19.pdf}{Yes} & \cite{KucukY19} & 2019 & RAST 2019 & 5 & 2 & 17 & \ref{b:KucukY19} & n/a\\
\rowlabel{a:LiuLH19}LiuLH19 \href{https://doi.org/10.1007/978-3-030-19823-7_19}{LiuLH19} & \hyperref[auth:a547]{K. Liu}, \hyperref[auth:a548]{S. L{\"{o}}ffler}, \hyperref[auth:a549]{P. Hofstedt} & Solving the Talent Scheduling Problem by Parallel Constraint Programming & \href{../works/LiuLH19.pdf}{Yes} & \cite{LiuLH19} & 2019 & AIAI 2019 & 9 & 1 & 5 & \ref{b:LiuLH19} & n/a\\
\rowlabel{a:MalapertN19}MalapertN19 \href{https://doi.org/10.1007/978-3-030-19212-9_28}{MalapertN19} & \hyperref[auth:a82]{A. Malapert}, \hyperref[auth:a81]{M. Nattaf} & A New CP-Approach for a Parallel Machine Scheduling Problem with Time Constraints on Machine Qualifications & \href{../works/MalapertN19.pdf}{Yes} & \cite{MalapertN19} & 2019 & CPAIOR 2019 & 17 & 1 & 7 & \ref{b:MalapertN19} & n/a\\
\rowlabel{a:MurinR19}MurinR19 \href{https://doi.org/10.1007/978-3-030-30048-7_27}{MurinR19} & \hyperref[auth:a100]{S. Mur{\'{\i}}n}, \hyperref[auth:a101]{H. Rudov{\'{a}}} & Scheduling of Mobile Robots Using Constraint Programming & \href{../works/MurinR19.pdf}{Yes} & \cite{MurinR19} & 2019 & CP 2019 & 16 & 2 & 22 & \ref{b:MurinR19} & \ref{c:MurinR19}\\
\rowlabel{a:ParkUJR19}ParkUJR19 \href{https://doi.org/10.1007/978-3-030-19648-6_15}{ParkUJR19} & \hyperref[auth:a550]{H. Park}, \hyperref[auth:a551]{J. Um}, \hyperref[auth:a552]{J. Jung}, \hyperref[auth:a553]{M. Ruskowski} & Developing a Production Scheduling System for Modular Factory Using Constraint Programming & \href{../works/ParkUJR19.pdf}{Yes} & \cite{ParkUJR19} & 2019 & RAAD 2019 & 8 & 1 & 3 & \ref{b:ParkUJR19} & n/a\\
\rowlabel{a:SenderovichBB19}SenderovichBB19 \href{https://ojs.aaai.org/index.php/ICAPS/article/view/3504}{SenderovichBB19} & \hyperref[auth:a1396]{A. Senderovich}, \hyperref[auth:a208]{Kyle E. C. Booth}, \hyperref[auth:a89]{J. Christopher Beck} & Learning Scheduling Models from Event Data & \href{../works/SenderovichBB19.pdf}{Yes} & \cite{SenderovichBB19} & 2019 & ICAPS 2019 & 9 & 0 & 0 & \ref{b:SenderovichBB19} & \ref{c:SenderovichBB19}\\
\rowlabel{a:Tom19}Tom19 \href{https://doi.org/10.1109/FUZZ-IEEE.2019.8859029}{Tom19} & \hyperref[auth:a542]{M. Tom} & Fuzzy Multi-Constraint Programming Model for Weekly Meals Scheduling & \href{../works/Tom19.pdf}{Yes} & \cite{Tom19} & 2019 & FUZZ-IEEE 2019 & 6 & 0 & 21 & \ref{b:Tom19} & n/a\\
\rowlabel{a:YangSS19}YangSS19 \href{https://doi.org/10.1007/978-3-030-19212-9_42}{YangSS19} & \hyperref[auth:a311]{M. Yang}, \hyperref[auth:a125]{A. Schutt}, \hyperref[auth:a126]{Peter J. Stuckey} & Time Table Edge Finding with Energy Variables & \href{../works/YangSS19.pdf}{Yes} & \cite{YangSS19} & 2019 & CPAIOR 2019 & 10 & 1 & 14 & \ref{b:YangSS19} & n/a\\
\rowlabel{a:AgussurjaKL18}AgussurjaKL18 \href{https://doi.org/10.1609/aaai.v32i1.12086}{AgussurjaKL18} & \hyperref[auth:a1383]{L. Agussurja}, \hyperref[auth:a1384]{A. Kumar}, \hyperref[auth:a367]{Hoong Chuin Lau} & Resource-Constrained Scheduling for Maritime Traffic Management & \href{../works/AgussurjaKL18.pdf}{Yes} & \cite{AgussurjaKL18} & 2018 & AAAI 2018 & 8 & 4 & 0 & \ref{b:AgussurjaKL18} & n/a\\
\rowlabel{a:AntunesABD18}AntunesABD18 \href{https://doi.org/10.1109/ICTAI.2018.00027}{AntunesABD18} & \hyperref[auth:a884]{M. Antunes}, \hyperref[auth:a885]{V. Armant}, \hyperref[auth:a222]{Kenneth N. Brown}, \hyperref[auth:a886]{Daniel A. Desmond}, \hyperref[auth:a887]{G. Escamocher}, \hyperref[auth:a888]{A. George}, \hyperref[auth:a182]{D. Grimes}, \hyperref[auth:a889]{M. O'Keeffe}, \hyperref[auth:a890]{Y. Lin}, \hyperref[auth:a16]{B. O'Sullivan}, \hyperref[auth:a136]{C. {\"{O}}zt{\"{u}}rk}, \hyperref[auth:a891]{L. Quesada}, \hyperref[auth:a130]{M. Siala}, \hyperref[auth:a17]{H. Simonis}, \hyperref[auth:a832]{N. Wilson} & Assigning and Scheduling Service Visits in a Mixed Urban/Rural Setting & \href{../works/AntunesABD18.pdf}{Yes} & \cite{AntunesABD18} & 2018 & ICTAI 2018 & 8 & 1 & 24 & \ref{b:AntunesABD18} & n/a\\
\rowlabel{a:ArbaouiY18}ArbaouiY18 \href{https://doi.org/10.1007/978-3-319-75420-8_67}{ArbaouiY18} & \hyperref[auth:a584]{T. Arbaoui}, \hyperref[auth:a458]{F. Yalaoui} & Solving the Unrelated Parallel Machine Scheduling Problem with Additional Resources Using Constraint Programming & \href{../works/ArbaouiY18.pdf}{Yes} & \cite{ArbaouiY18} & 2018 & ACIIDS 2018 & 10 & 2 & 14 & \ref{b:ArbaouiY18} & n/a\\
\rowlabel{a:AstrandJZ18}AstrandJZ18 \href{https://doi.org/10.1007/978-3-319-93031-2_44}{AstrandJZ18} & \hyperref[auth:a74]{M. {\AA}strand}, \hyperref[auth:a75]{M. Johansson}, \hyperref[auth:a204]{A. Zanarini} & Fleet Scheduling in Underground Mines Using Constraint Programming & \href{../works/AstrandJZ18.pdf}{Yes} & \cite{AstrandJZ18} & 2018 & CPAIOR 2018 & 9 & 9 & 10 & \ref{b:AstrandJZ18} & n/a\\
\rowlabel{a:BenediktSMVH18}BenediktSMVH18 \href{https://doi.org/10.1007/978-3-319-93031-2_6}{BenediktSMVH18} & \hyperref[auth:a114]{O. Benedikt}, \hyperref[auth:a312]{P. Sucha}, \hyperref[auth:a115]{I. M{\'{o}}dos}, \hyperref[auth:a313]{M. Vlk}, \hyperref[auth:a116]{Z. Hanz{\'{a}}lek} & Energy-Aware Production Scheduling with Power-Saving Modes & \href{../works/BenediktSMVH18.pdf}{Yes} & \cite{BenediktSMVH18} & 2018 & CPAIOR 2018 & 10 & 2 & 12 & \ref{b:BenediktSMVH18} & \ref{c:BenediktSMVH18}\\
\rowlabel{a:CappartTSR18}CappartTSR18 \href{https://doi.org/10.1007/978-3-319-98334-9_32}{CappartTSR18} & \hyperref[auth:a42]{Q. Cappart}, \hyperref[auth:a841]{C. Thomas}, \hyperref[auth:a148]{P. Schaus}, \hyperref[auth:a329]{L. Rousseau} & A Constraint Programming Approach for Solving Patient Transportation Problems & \href{../works/CappartTSR18.pdf}{Yes} & \cite{CappartTSR18} & 2018 & CP 2018 & 17 & 6 & 31 & \ref{b:CappartTSR18} & \ref{c:CappartTSR18}\\
\rowlabel{a:DemirovicS18}DemirovicS18 \href{https://doi.org/10.1007/978-3-319-93031-2_10}{DemirovicS18} & \hyperref[auth:a314]{E. Demirovic}, \hyperref[auth:a126]{Peter J. Stuckey} & Constraint Programming for High School Timetabling: {A} Scheduling-Based Model with Hot Starts & \href{../works/DemirovicS18.pdf}{Yes} & \cite{DemirovicS18} & 2018 & CPAIOR 2018 & 18 & 4 & 16 & \ref{b:DemirovicS18} & n/a\\
\rowlabel{a:He0GLW18}He0GLW18 \href{https://doi.org/10.1007/978-3-319-98334-9_42}{He0GLW18} & \hyperref[auth:a185]{S. He}, \hyperref[auth:a117]{Mark G. Wallace}, \hyperref[auth:a186]{G. Gange}, \hyperref[auth:a187]{A. Liebman}, \hyperref[auth:a188]{C. Wilson} & A Fast and Scalable Algorithm for Scheduling Large Numbers of Devices Under Real-Time Pricing & \href{../works/He0GLW18.pdf}{Yes} & \cite{He0GLW18} & 2018 & CP 2018 & 18 & 6 & 26 & \ref{b:He0GLW18} & \ref{c:He0GLW18}\\
\rowlabel{a:HoYCLLCLC18}HoYCLLCLC18 \href{https://doi.org/10.1145/3299819.3299825}{HoYCLLCLC18} & \hyperref[auth:a585]{T. Ho}, \hyperref[auth:a586]{J. Yao}, \hyperref[auth:a587]{Y. Chang}, \hyperref[auth:a588]{F. Lai}, \hyperref[auth:a589]{J. Lai}, \hyperref[auth:a590]{S. Chu}, \hyperref[auth:a591]{W. Liao}, \hyperref[auth:a592]{H. Chiu} & A Platform for Dynamic Optimal Nurse Scheduling Based on Integer Linear Programming along with Multiple Criteria Constraints & \href{../works/HoYCLLCLC18.pdf}{Yes} & \cite{HoYCLLCLC18} & 2018 & AICCC 2018 & 6 & 2 & 14 & \ref{b:HoYCLLCLC18} & n/a\\
\rowlabel{a:KameugneFGOQ18}KameugneFGOQ18 \href{https://doi.org/10.1007/978-3-319-93031-2_23}{KameugneFGOQ18} & \hyperref[auth:a10]{R. Kameugne}, \hyperref[auth:a11]{S{\'{e}}v{\'{e}}rine Betmbe Fetgo}, \hyperref[auth:a315]{V. Gingras}, \hyperref[auth:a52]{Y. Ouellet}, \hyperref[auth:a37]{C. Quimper} & Horizontally Elastic Not-First/Not-Last Filtering Algorithm for Cumulative Resource Constraint & \href{../works/KameugneFGOQ18.pdf}{Yes} & \cite{KameugneFGOQ18} & 2018 & CPAIOR 2018 & 17 & 1 & 12 & \ref{b:KameugneFGOQ18} & n/a\\
\rowlabel{a:Laborie18a}Laborie18a \href{https://doi.org/10.1007/978-3-319-93031-2_29}{Laborie18a} & \hyperref[auth:a118]{P. Laborie} & An Update on the Comparison of MIP, {CP} and Hybrid Approaches for Mixed Resource Allocation and Scheduling & \href{../works/Laborie18a.pdf}{Yes} & \cite{Laborie18a} & 2018 & CPAIOR 2018 & 9 & 18 & 10 & \ref{b:Laborie18a} & n/a\\
\rowlabel{a:MusliuSS18}MusliuSS18 \href{https://doi.org/10.1007/978-3-319-93031-2_31}{MusliuSS18} & \hyperref[auth:a45]{N. Musliu}, \hyperref[auth:a125]{A. Schutt}, \hyperref[auth:a126]{Peter J. Stuckey} & Solver Independent Rotating Workforce Scheduling & \href{../works/MusliuSS18.pdf}{Yes} & \cite{MusliuSS18} & 2018 & CPAIOR 2018 & 17 & 7 & 23 & \ref{b:MusliuSS18} & n/a\\
\rowlabel{a:NishikawaSTT18}NishikawaSTT18 \href{https://doi.org/10.1109/CANDAR.2018.00025}{NishikawaSTT18} & \hyperref[auth:a534]{H. Nishikawa}, \hyperref[auth:a535]{K. Shimada}, \hyperref[auth:a536]{I. Taniguchi}, \hyperref[auth:a537]{H. Tomiyama} & Scheduling of Malleable Fork-Join Tasks with Constraint Programming & \href{../works/NishikawaSTT18.pdf}{Yes} & \cite{NishikawaSTT18} & 2018 & CANDAR 2018 & 6 & 2 & 14 & \ref{b:NishikawaSTT18} & n/a\\
\rowlabel{a:NishikawaSTT18a}NishikawaSTT18a \href{https://doi.org/10.1109/TENCON.2018.8650168}{NishikawaSTT18a} & \hyperref[auth:a534]{H. Nishikawa}, \hyperref[auth:a535]{K. Shimada}, \hyperref[auth:a536]{I. Taniguchi}, \hyperref[auth:a537]{H. Tomiyama} & Scheduling of Malleable Tasks Based on Constraint Programming & \href{../works/NishikawaSTT18a.pdf}{Yes} & \cite{NishikawaSTT18a} & 2018 & TENCON 2018 & 6 & 1 & 9 & \ref{b:NishikawaSTT18a} & n/a\\
\rowlabel{a:OuelletQ18}OuelletQ18 \href{https://doi.org/10.1007/978-3-319-93031-2_34}{OuelletQ18} & \hyperref[auth:a52]{Y. Ouellet}, \hyperref[auth:a37]{C. Quimper} & A O(n {\textbackslash}log {\^{}}2 n) Checker and O(n{\^{}}2 {\textbackslash}log n) Filtering Algorithm for the Energetic Reasoning & \href{../works/OuelletQ18.pdf}{Yes} & \cite{OuelletQ18} & 2018 & CPAIOR 2018 & 18 & 6 & 16 & \ref{b:OuelletQ18} & n/a\\
\rowlabel{a:RiahiNS018}RiahiNS018 \href{https://aaai.org/ocs/index.php/ICAPS/ICAPS18/paper/view/17755}{RiahiNS018} & \hyperref[auth:a391]{V. Riahi}, \hyperref[auth:a392]{M. A. Hakim Newton}, \hyperref[auth:a393]{K. Su}, \hyperref[auth:a394]{A. Sattar} & Local Search for Flowshops with Setup Times and Blocking Constraints & \href{../works/RiahiNS018.pdf}{Yes} & \cite{RiahiNS018} & 2018 & ICAPS 2018 & 9 & 4 & 0 & \ref{b:RiahiNS018} & n/a\\
\rowlabel{a:TanT18}TanT18 \href{http://dx.doi.org/10.1007/978-3-319-89656-4_5}{TanT18} & \hyperref[auth:a917]{Y. Tan}, \hyperref[auth:a824]{D. Terekhov} & Logic-Based Benders Decomposition for Two-Stage Flexible Flow Shop Scheduling with Unrelated Parallel Machines & \href{../works/TanT18.pdf}{Yes} & \cite{TanT18} & 2018 & Canadian AI 2018 & 12 & 1 & 23 & \ref{b:TanT18} & n/a\\
\rowlabel{a:Tesch18}Tesch18 \href{https://doi.org/10.1007/978-3-319-98334-9_41}{Tesch18} & \hyperref[auth:a184]{A. Tesch} & Improving Energetic Propagations for Cumulative Scheduling & \href{../works/Tesch18.pdf}{Yes} & \cite{Tesch18} & 2018 & CP 2018 & 17 & 5 & 21 & \ref{b:Tesch18} & n/a\\
\rowlabel{a:BofillCSV17}BofillCSV17 \href{https://doi.org/10.1007/978-3-319-66158-2_5}{BofillCSV17} & \hyperref[auth:a189]{M. Bofill}, \hyperref[auth:a190]{J. Coll}, \hyperref[auth:a191]{J. Suy}, \hyperref[auth:a192]{M. Villaret} & An Efficient {SMT} Approach to Solve MRCPSP/max Instances with Tight Constraints on Resources & \href{../works/BofillCSV17.pdf}{Yes} & \cite{BofillCSV17} & 2017 & CP 2017 & 9 & 1 & 12 & \ref{b:BofillCSV17} & n/a\\
\rowlabel{a:CappartS17}CappartS17 \href{https://doi.org/10.1007/978-3-319-59776-8_26}{CappartS17} & \hyperref[auth:a42]{Q. Cappart}, \hyperref[auth:a148]{P. Schaus} & Rescheduling Railway Traffic on Real Time Situations Using Time-Interval Variables & \href{../works/CappartS17.pdf}{Yes} & \cite{CappartS17} & 2017 & CPAIOR 2017 & 16 & 2 & 28 & \ref{b:CappartS17} & \ref{c:CappartS17}\\
\rowlabel{a:CohenHB17}CohenHB17 \href{https://doi.org/10.1007/978-3-319-66263-3_10}{CohenHB17} & \hyperref[auth:a811]{E. Cohen}, \hyperref[auth:a812]{G. Huang}, \hyperref[auth:a89]{J. Christopher Beck} & {(I} Can Get) Satisfaction: Preference-Based Scheduling for Concert-Goers at Multi-venue Music Festivals & \href{../works/CohenHB17.pdf}{Yes} & \cite{CohenHB17} & 2017 & SAT 2017 & 17 & 1 & 12 & \ref{b:CohenHB17} & n/a\\
\rowlabel{a:GelainPRVW17}GelainPRVW17 \href{https://doi.org/10.1007/978-3-319-59776-8_32}{GelainPRVW17} & \hyperref[auth:a316]{M. Gelain}, \hyperref[auth:a317]{Maria Silvia Pini}, \hyperref[auth:a318]{F. Rossi}, \hyperref[auth:a319]{Kristen Brent Venable}, \hyperref[auth:a278]{T. Walsh} & A Local Search Approach for Incomplete Soft Constraint Problems: Experimental Results on Meeting Scheduling Problems & \href{../works/GelainPRVW17.pdf}{Yes} & \cite{GelainPRVW17} & 2017 & CPAIOR 2017 & 16 & 1 & 5 & \ref{b:GelainPRVW17} & n/a\\
\rowlabel{a:GoldwaserS17}GoldwaserS17 \href{https://doi.org/10.1007/978-3-319-66158-2_22}{GoldwaserS17} & \hyperref[auth:a194]{A. Goldwaser}, \hyperref[auth:a125]{A. Schutt} & Optimal Torpedo Scheduling & \href{../works/GoldwaserS17.pdf}{Yes} & \cite{GoldwaserS17} & 2017 & CP 2017 & 16 & 0 & 10 & \ref{b:GoldwaserS17} & \ref{c:GoldwaserS17}\\
\rowlabel{a:Hooker17}Hooker17 \href{https://doi.org/10.1007/978-3-319-66158-2_36}{Hooker17} & \hyperref[auth:a161]{John N. Hooker} & Job Sequencing Bounds from Decision Diagrams & \href{../works/Hooker17.pdf}{Yes} & \cite{Hooker17} & 2017 & CP 2017 & 14 & 6 & 24 & \ref{b:Hooker17} & n/a\\
\rowlabel{a:KletzanderM17}KletzanderM17 \href{https://doi.org/10.1007/978-3-319-59776-8_28}{KletzanderM17} & \hyperref[auth:a78]{L. Kletzander}, \hyperref[auth:a45]{N. Musliu} & A Multi-stage Simulated Annealing Algorithm for the Torpedo Scheduling Problem & \href{../works/KletzanderM17.pdf}{Yes} & \cite{KletzanderM17} & 2017 & CPAIOR 2017 & 15 & 1 & 9 & \ref{b:KletzanderM17} & n/a\\
\rowlabel{a:LiuCGM17}LiuCGM17 \href{https://doi.org/10.1007/978-3-319-66158-2_24}{LiuCGM17} & \hyperref[auth:a195]{T. Liu}, \hyperref[auth:a196]{Roberto Di Cosmo}, \hyperref[auth:a197]{M. Gabbrielli}, \hyperref[auth:a198]{J. Mauro} & NightSplitter: {A} Scheduling Tool to Optimize (Sub)group Activities & \href{../works/LiuCGM17.pdf}{Yes} & \cite{LiuCGM17} & 2017 & CP 2017 & 17 & 0 & 15 & \ref{b:LiuCGM17} & \ref{c:LiuCGM17}\\
\rowlabel{a:Madi-WambaLOBM17}Madi-WambaLOBM17 \href{https://doi.org/10.1109/ICPADS.2017.00089}{Madi-WambaLOBM17} & \hyperref[auth:a323]{G. Madi{-}Wamba}, \hyperref[auth:a720]{Y. Li}, \hyperref[auth:a721]{A. Orgerie}, \hyperref[auth:a129]{N. Beldiceanu}, \hyperref[auth:a722]{J. Menaud} & Green Energy Aware Scheduling Problem in Virtualized Datacenters & \href{../works/Madi-WambaLOBM17.pdf}{Yes} & \cite{Madi-WambaLOBM17} & 2017 & ICPADS 2017 & 8 & 1 & 8 & \ref{b:Madi-WambaLOBM17} & n/a\\
\rowlabel{a:MossigeGSMC17}MossigeGSMC17 \href{https://doi.org/10.1007/978-3-319-66158-2_25}{MossigeGSMC17} & \hyperref[auth:a199]{M. Mossige}, \hyperref[auth:a200]{A. Gotlieb}, \hyperref[auth:a201]{H. Spieker}, \hyperref[auth:a202]{H. Meling}, \hyperref[auth:a91]{M. Carlsson} & Time-Aware Test Case Execution Scheduling for Cyber-Physical Systems & \href{../works/MossigeGSMC17.pdf}{Yes} & \cite{MossigeGSMC17} & 2017 & CP 2017 & 18 & 6 & 33 & \ref{b:MossigeGSMC17} & n/a\\
\rowlabel{a:Pralet17}Pralet17 \href{https://doi.org/10.1007/978-3-319-66158-2_16}{Pralet17} & \hyperref[auth:a21]{C. Pralet} & An Incomplete Constraint-Based System for Scheduling with Renewable Resources & \href{../works/Pralet17.pdf}{Yes} & \cite{Pralet17} & 2017 & CP 2017 & 19 & 1 & 30 & \ref{b:Pralet17} & n/a\\
\rowlabel{a:TranVNB17a}TranVNB17a \href{https://doi.org/10.24963/ijcai.2017/726}{TranVNB17a} & \hyperref[auth:a805]{Tony T. Tran}, \hyperref[auth:a810]{Tiago Stegun Vaquero}, \hyperref[auth:a209]{G. Nejat}, \hyperref[auth:a89]{J. Christopher Beck} & Robots in Retirement Homes: Applying Off-the-Shelf Planning and Scheduling to a Team of Assistive Robots (Extended Abstract) & \href{../works/TranVNB17a.pdf}{Yes} & \cite{TranVNB17a} & 2017 & IJCAI 2017 & 5 & 1 & 0 & \ref{b:TranVNB17a} & n/a\\
\rowlabel{a:YoungFS17}YoungFS17 \href{https://doi.org/10.1007/978-3-319-66158-2_20}{YoungFS17} & \hyperref[auth:a193]{Kenneth D. Young}, \hyperref[auth:a155]{T. Feydy}, \hyperref[auth:a125]{A. Schutt} & Constraint Programming Applied to the Multi-Skill Project Scheduling Problem & \href{../works/YoungFS17.pdf}{Yes} & \cite{YoungFS17} & 2017 & CP 2017 & 10 & 6 & 21 & \ref{b:YoungFS17} & \ref{c:YoungFS17}\\
\rowlabel{a:AmadiniGM16}AmadiniGM16 \href{http://dx.doi.org/10.1007/978-3-319-50349-3_16}{AmadiniGM16} & \hyperref[auth:a918]{R. Amadini}, \hyperref[auth:a197]{M. Gabbrielli}, \hyperref[auth:a198]{J. Mauro} & Parallelizing Constraint Solvers for Hard RCPSP Instances & \href{../works/AmadiniGM16.pdf}{Yes} & \cite{AmadiniGM16} & 2016 & LION 2016 & 7 & 2 & 16 & \ref{b:AmadiniGM16} & \ref{c:AmadiniGM16}\\
\rowlabel{a:BonfiettiZLM16}BonfiettiZLM16 \href{https://doi.org/10.1007/978-3-319-44953-1_8}{BonfiettiZLM16} & \hyperref[auth:a203]{A. Bonfietti}, \hyperref[auth:a204]{A. Zanarini}, \hyperref[auth:a143]{M. Lombardi}, \hyperref[auth:a144]{M. Milano} & The Multirate Resource Constraint & \href{../works/BonfiettiZLM16.pdf}{Yes} & \cite{BonfiettiZLM16} & 2016 & CP 2016 & 17 & 0 & 11 & \ref{b:BonfiettiZLM16} & \ref{c:BonfiettiZLM16}\\
\rowlabel{a:BoothNB16}BoothNB16 \href{https://doi.org/10.1007/978-3-319-44953-1_34}{BoothNB16} & \hyperref[auth:a208]{Kyle E. C. Booth}, \hyperref[auth:a209]{G. Nejat}, \hyperref[auth:a89]{J. Christopher Beck} & A Constraint Programming Approach to Multi-Robot Task Allocation and Scheduling in Retirement Homes & \href{../works/BoothNB16.pdf}{Yes} & \cite{BoothNB16} & 2016 & CP 2016 & 17 & 21 & 24 & \ref{b:BoothNB16} & n/a\\
\rowlabel{a:BridiLBBM16}BridiLBBM16 \href{https://doi.org/10.3233/978-1-61499-672-9-1598}{BridiLBBM16} & \hyperref[auth:a232]{T. Bridi}, \hyperref[auth:a143]{M. Lombardi}, \hyperref[auth:a230]{A. Bartolini}, \hyperref[auth:a247]{L. Benini}, \hyperref[auth:a144]{M. Milano} & {DARDIS:} Distributed And Randomized DIspatching and Scheduling & \href{../works/BridiLBBM16.pdf}{Yes} & \cite{BridiLBBM16} & 2016 & ECAI 2016 & 2 & 0 & 0 & \ref{b:BridiLBBM16} & n/a\\
\rowlabel{a:CatusseCBL16}CatusseCBL16 \href{http://www.ijcai.org/Abstract/16/434}{CatusseCBL16} & \hyperref[auth:a1010]{N. Catusse}, \hyperref[auth:a1011]{H. Cambazard}, \hyperref[auth:a1012]{N. Brauner}, \hyperref[auth:a989]{P. Lemaire}, \hyperref[auth:a1013]{B. Penz}, \hyperref[auth:a1014]{A. Lagrange}, \hyperref[auth:a1015]{P. Rubini} & A Branch-and-Price Algorithm for Scheduling Observations on a Telescope & \href{../works/CatusseCBL16.pdf}{Yes} & \cite{CatusseCBL16} & 2016 & IJCAI 2016 & 7 & 0 & 0 & \ref{b:CatusseCBL16} & n/a\\
\rowlabel{a:CauwelaertDMS16}CauwelaertDMS16 \href{https://doi.org/10.1007/978-3-319-44953-1_33}{CauwelaertDMS16} & \hyperref[auth:a206]{Sascha Van Cauwelaert}, \hyperref[auth:a207]{C. Dejemeppe}, \hyperref[auth:a150]{J. Monette}, \hyperref[auth:a148]{P. Schaus} & Efficient Filtering for the Unary Resource with Family-Based Transition Times & \href{../works/CauwelaertDMS16.pdf}{Yes} & \cite{CauwelaertDMS16} & 2016 & CP 2016 & 16 & 1 & 12 & \ref{b:CauwelaertDMS16} & \ref{c:CauwelaertDMS16}\\
\rowlabel{a:FontaineMH16}FontaineMH16 \href{https://doi.org/10.1007/978-3-319-33954-2_12}{FontaineMH16} & \hyperref[auth:a320]{D. Fontaine}, \hyperref[auth:a321]{Laurent D. Michel}, \hyperref[auth:a149]{Pascal Van Hentenryck} & Parallel Composition of Scheduling Solvers & \href{../works/FontaineMH16.pdf}{Yes} & \cite{FontaineMH16} & 2016 & CPAIOR 2016 & 11 & 3 & 0 & \ref{b:FontaineMH16} & n/a\\
\rowlabel{a:FrankDT16}FrankDT16 \href{http://www.aaai.org/ocs/index.php/ICAPS/ICAPS16/paper/view/13072}{FrankDT16} & \hyperref[auth:a382]{J. Frank}, \hyperref[auth:a815]{M. Do}, \hyperref[auth:a805]{Tony T. Tran} & Scheduling Ocean Color Observations for a GEO-Stationary Satellite & \href{../works/FrankDT16.pdf}{Yes} & \cite{FrankDT16} & 2016 & ICAPS 2016 & 9 & 4 & 0 & \ref{b:FrankDT16} & n/a\\
\rowlabel{a:GilesH16}GilesH16 \href{https://doi.org/10.1007/978-3-319-44953-1_38}{GilesH16} & \hyperref[auth:a210]{K. Giles}, \hyperref[auth:a211]{Willem{-}Jan van Hoeve} & Solving a Supply-Delivery Scheduling Problem with Constraint Programming & \href{../works/GilesH16.pdf}{Yes} & \cite{GilesH16} & 2016 & CP 2016 & 16 & 2 & 6 & \ref{b:GilesH16} & n/a\\
\rowlabel{a:GingrasQ16}GingrasQ16 \href{http://www.ijcai.org/Abstract/16/440}{GingrasQ16} & \hyperref[auth:a315]{V. Gingras}, \hyperref[auth:a37]{C. Quimper} & Generalizing the Edge-Finder Rule for the Cumulative Constraint & \href{../works/GingrasQ16.pdf}{Yes} & \cite{GingrasQ16} & 2016 & IJCAI 2016 & 7 & 0 & 0 & \ref{b:GingrasQ16} & n/a\\
\rowlabel{a:HechingH16}HechingH16 \href{https://doi.org/10.1007/978-3-319-33954-2_14}{HechingH16} & \hyperref[auth:a322]{Aliza R. Heching}, \hyperref[auth:a161]{John N. Hooker} & Scheduling Home Hospice Care with Logic-Based Benders Decomposition & \href{../works/HechingH16.pdf}{Yes} & \cite{HechingH16} & 2016 & CPAIOR 2016 & 11 & 10 & 0 & \ref{b:HechingH16} & n/a\\
\rowlabel{a:JelinekB16}JelinekB16 \href{https://doi.org/10.1007/978-3-319-28228-2_1}{JelinekB16} & \hyperref[auth:a785]{J. Jel{\'{\i}}nek}, \hyperref[auth:a153]{R. Bart{\'{a}}k} & Using Constraint Logic Programming to Schedule Solar Array Operations on the International Space Station & \href{../works/JelinekB16.pdf}{Yes} & \cite{JelinekB16} & 2016 & PADL 2016 & 10 & 0 & 5 & \ref{b:JelinekB16} & n/a\\
\rowlabel{a:KinsellaS0OS16}KinsellaS0OS16 \href{https://doi.org/10.1609/aaai.v30i2.19079}{KinsellaS0OS16} & \hyperref[auth:a1381]{A. Kinsella}, \hyperref[auth:a1382]{Alan F. Smeaton}, \hyperref[auth:a892]{B. Hurley}, \hyperref[auth:a16]{B. O'Sullivan}, \hyperref[auth:a17]{H. Simonis} & Optimizing Energy Costs in a Zinc and Lead Mine & \href{../works/KinsellaS0OS16.pdf}{Yes} & \cite{KinsellaS0OS16} & 2016 & AAAI 2016 & 6 & 1 & 0 & \ref{b:KinsellaS0OS16} & n/a\\
\rowlabel{a:LimHTB16}LimHTB16 \href{https://doi.org/10.1007/978-3-319-44953-1_43}{LimHTB16} & \hyperref[auth:a212]{B. Lim}, \hyperref[auth:a213]{Hassan L. Hijazi}, \hyperref[auth:a214]{S. Thi{\'{e}}baux}, \hyperref[auth:a215]{Menkes van den Briel} & Online HVAC-Aware Occupancy Scheduling with Adaptive Temperature Control & \href{../works/LimHTB16.pdf}{Yes} & \cite{LimHTB16} & 2016 & CP 2016 & 18 & 2 & 23 & \ref{b:LimHTB16} & n/a\\
\rowlabel{a:LuoVLBM16}LuoVLBM16 \href{http://www.aaai.org/ocs/index.php/KR/KR16/paper/view/12909}{LuoVLBM16} & \hyperref[auth:a819]{R. Luo}, \hyperref[auth:a820]{Richard Anthony Valenzano}, \hyperref[auth:a821]{Y. Li}, \hyperref[auth:a89]{J. Christopher Beck}, \hyperref[auth:a822]{Sheila A. McIlraith} & Using Metric Temporal Logic to Specify Scheduling Problems & \href{../works/LuoVLBM16.pdf}{Yes} & \cite{LuoVLBM16} & 2016 & KR 2016 & 4 & 0 & 0 & \ref{b:LuoVLBM16} & n/a\\
\rowlabel{a:Madi-WambaB16}Madi-WambaB16 \href{https://doi.org/10.1007/978-3-319-33954-2_18}{Madi-WambaB16} & \hyperref[auth:a323]{G. Madi{-}Wamba}, \hyperref[auth:a129]{N. Beldiceanu} & The TaskIntersection Constraint & \href{../works/Madi-WambaB16.pdf}{Yes} & \cite{Madi-WambaB16} & 2016 & CPAIOR 2016 & 16 & 0 & 0 & \ref{b:Madi-WambaB16} & n/a\\
\rowlabel{a:SchuttS16}SchuttS16 \href{https://doi.org/10.1007/978-3-319-44953-1_28}{SchuttS16} & \hyperref[auth:a125]{A. Schutt}, \hyperref[auth:a126]{Peter J. Stuckey} & Explaining Producer/Consumer Constraints & \href{../works/SchuttS16.pdf}{Yes} & \cite{SchuttS16} & 2016 & CP 2016 & 17 & 3 & 23 & \ref{b:SchuttS16} & n/a\\
\rowlabel{a:SzerediS16}SzerediS16 \href{https://doi.org/10.1007/978-3-319-44953-1_31}{SzerediS16} & \hyperref[auth:a205]{R. Szeredi}, \hyperref[auth:a125]{A. Schutt} & Modelling and Solving Multi-mode Resource-Constrained Project Scheduling & \href{../works/SzerediS16.pdf}{Yes} & \cite{SzerediS16} & 2016 & CP 2016 & 10 & 9 & 14 & \ref{b:SzerediS16} & n/a\\
\rowlabel{a:Tesch16}Tesch16 \href{https://doi.org/10.1007/978-3-319-44953-1_32}{Tesch16} & \hyperref[auth:a184]{A. Tesch} & A Nearly Exact Propagation Algorithm for Energetic Reasoning in {\textbackslash}mathcal O(n{\^{}}2 {\textbackslash}log n) & \href{../works/Tesch16.pdf}{Yes} & \cite{Tesch16} & 2016 & CP 2016 & 27 & 4 & 14 & \ref{b:Tesch16} & n/a\\
\rowlabel{a:TranDRFWOVB16}TranDRFWOVB16 \href{https://doi.org/10.1609/socs.v7i1.18390}{TranDRFWOVB16} & \hyperref[auth:a805]{Tony T. Tran}, \hyperref[auth:a815]{M. Do}, \hyperref[auth:a816]{Eleanor Gilbert Rieffel}, \hyperref[auth:a382]{J. Frank}, \hyperref[auth:a814]{Z. Wang}, \hyperref[auth:a817]{B. O'Gorman}, \hyperref[auth:a818]{D. Venturelli}, \hyperref[auth:a89]{J. Christopher Beck} & A Hybrid Quantum-Classical Approach to Solving Scheduling Problems & \href{../works/TranDRFWOVB16.pdf}{Yes} & \cite{TranDRFWOVB16} & 2016 & SOCS 2016 & 9 & 3 & 0 & \ref{b:TranDRFWOVB16} & n/a\\
\rowlabel{a:TranWDRFOVB16}TranWDRFOVB16 \href{http://www.aaai.org/ocs/index.php/WS/AAAIW16/paper/view/12664}{TranWDRFOVB16} & \hyperref[auth:a805]{Tony T. Tran}, \hyperref[auth:a814]{Z. Wang}, \hyperref[auth:a815]{M. Do}, \hyperref[auth:a816]{Eleanor Gilbert Rieffel}, \hyperref[auth:a382]{J. Frank}, \hyperref[auth:a817]{B. O'Gorman}, \hyperref[auth:a818]{D. Venturelli}, \hyperref[auth:a89]{J. Christopher Beck} & Explorations of Quantum-Classical Approaches to Scheduling a Mars Lander Activity Problem & \href{../works/TranWDRFOVB16.pdf}{Yes} & \cite{TranWDRFOVB16} & 2016 & AAAI 2016 & 9 & 0 & 0 & \ref{b:TranWDRFOVB16} & n/a\\
\rowlabel{a:BartakV15}BartakV15 \href{}{BartakV15} & \hyperref[auth:a153]{R. Bart{\'{a}}k}, \hyperref[auth:a313]{M. Vlk} & Reactive Recovery from Machine Breakdown in Production Scheduling with Temporal Distance and Resource Constraints & \href{../works/BartakV15.pdf}{Yes} & \cite{BartakV15} & 2015 & ICAART 2015 & 12 & 0 & 0 & \ref{b:BartakV15} & n/a\\
\rowlabel{a:BofillGSV15}BofillGSV15 \href{https://doi.org/10.1007/978-3-319-18008-3_5}{BofillGSV15} & \hyperref[auth:a189]{M. Bofill}, \hyperref[auth:a234]{M. Garcia}, \hyperref[auth:a191]{J. Suy}, \hyperref[auth:a192]{M. Villaret} & MaxSAT-Based Scheduling of {B2B} Meetings & \href{../works/BofillGSV15.pdf}{Yes} & \cite{BofillGSV15} & 2015 & CPAIOR 2015 & 9 & 7 & 8 & \ref{b:BofillGSV15} & n/a\\
\rowlabel{a:BurtLPS15}BurtLPS15 \href{https://doi.org/10.1007/978-3-319-18008-3_7}{BurtLPS15} & \hyperref[auth:a325]{Christina N. Burt}, \hyperref[auth:a326]{N. Lipovetzky}, \hyperref[auth:a327]{Adrian R. Pearce}, \hyperref[auth:a126]{Peter J. Stuckey} & Scheduling with Fixed Maintenance, Shared Resources and Nonlinear Feedrate Constraints: {A} Mine Planning Case Study & \href{../works/BurtLPS15.pdf}{Yes} & \cite{BurtLPS15} & 2015 & CPAIOR 2015 & 17 & 0 & 8 & \ref{b:BurtLPS15} & n/a\\
\rowlabel{a:CauwelaertLS15}CauwelaertLS15 \href{https://doi.org/10.1007/978-3-319-18008-3_29}{CauwelaertLS15} & \hyperref[auth:a206]{Sascha Van Cauwelaert}, \hyperref[auth:a143]{M. Lombardi}, \hyperref[auth:a148]{P. Schaus} & Understanding the Potential of Propagators & \href{../works/CauwelaertLS15.pdf}{Yes} & \cite{CauwelaertLS15} & 2015 & CPAIOR 2015 & 10 & 12 & 0 & \ref{b:CauwelaertLS15} & \ref{c:CauwelaertLS15}\\
\rowlabel{a:DejemeppeCS15}DejemeppeCS15 \href{https://doi.org/10.1007/978-3-319-23219-5_7}{DejemeppeCS15} & \hyperref[auth:a207]{C. Dejemeppe}, \hyperref[auth:a206]{Sascha Van Cauwelaert}, \hyperref[auth:a148]{P. Schaus} & The Unary Resource with Transition Times & \href{../works/DejemeppeCS15.pdf}{Yes} & \cite{DejemeppeCS15} & 2015 & CP 2015 & 16 & 5 & 11 & \ref{b:DejemeppeCS15} & \ref{c:DejemeppeCS15}\\
\rowlabel{a:EvenSH15}EvenSH15 \href{https://doi.org/10.1007/978-3-319-23219-5_40}{EvenSH15} & \hyperref[auth:a219]{C. Even}, \hyperref[auth:a125]{A. Schutt}, \hyperref[auth:a149]{Pascal Van Hentenryck} & A Constraint Programming Approach for Non-preemptive Evacuation Scheduling & \href{../works/EvenSH15.pdf}{Yes} & \cite{EvenSH15} & 2015 & CP 2015 & 18 & 3 & 12 & \ref{b:EvenSH15} & n/a\\
\rowlabel{a:GayHLS15}GayHLS15 \href{https://doi.org/10.1007/978-3-319-23219-5_10}{GayHLS15} & \hyperref[auth:a216]{S. Gay}, \hyperref[auth:a217]{R. Hartert}, \hyperref[auth:a218]{C. Lecoutre}, \hyperref[auth:a148]{P. Schaus} & Conflict Ordering Search for Scheduling Problems & \href{../works/GayHLS15.pdf}{Yes} & \cite{GayHLS15} & 2015 & CP 2015 & 9 & 20 & 15 & \ref{b:GayHLS15} & \ref{c:GayHLS15}\\
\rowlabel{a:GayHS15}GayHS15 \href{https://doi.org/10.1007/978-3-319-23219-5_11}{GayHS15} & \hyperref[auth:a216]{S. Gay}, \hyperref[auth:a217]{R. Hartert}, \hyperref[auth:a148]{P. Schaus} & Simple and Scalable Time-Table Filtering for the Cumulative Constraint & \href{../works/GayHS15.pdf}{Yes} & \cite{GayHS15} & 2015 & CP 2015 & 9 & 10 & 9 & \ref{b:GayHS15} & \ref{c:GayHS15}\\
\rowlabel{a:GayHS15a}GayHS15a \href{https://doi.org/10.1007/978-3-319-18008-3_11}{GayHS15a} & \hyperref[auth:a216]{S. Gay}, \hyperref[auth:a217]{R. Hartert}, \hyperref[auth:a148]{P. Schaus} & Time-Table Disjunctive Reasoning for the Cumulative Constraint & \href{../works/GayHS15a.pdf}{Yes} & \cite{GayHS15a} & 2015 & CPAIOR 2015 & 16 & 5 & 12 & \ref{b:GayHS15a} & \ref{c:GayHS15a}\\
\rowlabel{a:KreterSS15}KreterSS15 \href{https://doi.org/10.1007/978-3-319-23219-5_19}{KreterSS15} & \hyperref[auth:a124]{S. Kreter}, \hyperref[auth:a125]{A. Schutt}, \hyperref[auth:a126]{Peter J. Stuckey} & Modeling and Solving Project Scheduling with Calendars & \href{../works/KreterSS15.pdf}{Yes} & \cite{KreterSS15} & 2015 & CP 2015 & 17 & 7 & 16 & \ref{b:KreterSS15} & n/a\\
\rowlabel{a:LimBTBB15}LimBTBB15 \href{https://doi.org/10.1007/978-3-319-18008-3_17}{LimBTBB15} & \hyperref[auth:a212]{B. Lim}, \hyperref[auth:a215]{Menkes van den Briel}, \hyperref[auth:a214]{S. Thi{\'{e}}baux}, \hyperref[auth:a1379]{R. Bent}, \hyperref[auth:a1380]{S. Backhaus} & Large Neighborhood Search for Energy Aware Meeting Scheduling in Smart Buildings & \href{../works/LimBTBB15.pdf}{Yes} & \cite{LimBTBB15} & 2015 & CPAIOR 2015 & 15 & 4 & 18 & \ref{b:LimBTBB15} & n/a\\
\rowlabel{a:LimBTBB15a}LimBTBB15a \href{https://doi.org/10.1609/aaai.v29i1.9236}{LimBTBB15a} & \hyperref[auth:a212]{B. Lim}, \hyperref[auth:a215]{Menkes van den Briel}, \hyperref[auth:a214]{S. Thi{\'{e}}baux}, \hyperref[auth:a1380]{S. Backhaus}, \hyperref[auth:a1379]{R. Bent} & HVAC-Aware Occupancy Scheduling & No & \cite{LimBTBB15a} & 2015 & AAAI 2015 & 8 & 3 & 0 & No & n/a\\
\rowlabel{a:LombardiBM15}LombardiBM15 \href{https://doi.org/10.1007/978-3-319-23219-5_20}{LombardiBM15} & \hyperref[auth:a143]{M. Lombardi}, \hyperref[auth:a203]{A. Bonfietti}, \hyperref[auth:a144]{M. Milano} & Deterministic Estimation of the Expected Makespan of a {POS} Under Duration Uncertainty & \href{../works/LombardiBM15.pdf}{Yes} & \cite{LombardiBM15} & 2015 & CP 2015 & 16 & 0 & 8 & \ref{b:LombardiBM15} & n/a\\
\rowlabel{a:MelgarejoLS15}MelgarejoLS15 \href{https://doi.org/10.1007/978-3-319-18008-3_1}{MelgarejoLS15} & \hyperref[auth:a324]{P. Aguiar{-}Melgarejo}, \hyperref[auth:a118]{P. Laborie}, \hyperref[auth:a85]{C. Solnon} & A Time-Dependent No-Overlap Constraint: Application to Urban Delivery Problems & \href{../works/MelgarejoLS15.pdf}{Yes} & \cite{MelgarejoLS15} & 2015 & CPAIOR 2015 & 17 & 14 & 17 & \ref{b:MelgarejoLS15} & n/a\\
\rowlabel{a:MurphyMB15}MurphyMB15 \href{https://doi.org/10.1007/978-3-319-23219-5_47}{MurphyMB15} & \hyperref[auth:a220]{Se{\'{a}}n {\'{O}}g Murphy}, \hyperref[auth:a221]{O. Manzano}, \hyperref[auth:a222]{Kenneth N. Brown} & Design and Evaluation of a Constraint-Based Energy Saving and Scheduling Recommender System & \href{../works/MurphyMB15.pdf}{Yes} & \cite{MurphyMB15} & 2015 & CP 2015 & 17 & 1 & 20 & \ref{b:MurphyMB15} & n/a\\
\rowlabel{a:PesantRR15}PesantRR15 \href{https://doi.org/10.1007/978-3-319-18008-3_21}{PesantRR15} & \hyperref[auth:a8]{G. Pesant}, \hyperref[auth:a328]{G. Rix}, \hyperref[auth:a329]{L. Rousseau} & A Comparative Study of {MIP} and {CP} Formulations for the {B2B} Scheduling Optimization Problem & \href{../works/PesantRR15.pdf}{Yes} & \cite{PesantRR15} & 2015 & CPAIOR 2015 & 16 & 1 & 7 & \ref{b:PesantRR15} & n/a\\
\rowlabel{a:PraletLJ15}PraletLJ15 \href{https://doi.org/10.1007/978-3-319-23219-5_48}{PraletLJ15} & \hyperref[auth:a21]{C. Pralet}, \hyperref[auth:a223]{S. Lemai{-}Chenevier}, \hyperref[auth:a224]{J. Jaubert} & Scheduling Running Modes of Satellite Instruments Using Constraint-Based Local Search & \href{../works/PraletLJ15.pdf}{Yes} & \cite{PraletLJ15} & 2015 & CP 2015 & 16 & 0 & 8 & \ref{b:PraletLJ15} & n/a\\
\rowlabel{a:SialaAH15}SialaAH15 \href{https://doi.org/10.1007/978-3-319-23219-5_28}{SialaAH15} & \hyperref[auth:a130]{M. Siala}, \hyperref[auth:a6]{C. Artigues}, \hyperref[auth:a1]{E. Hebrard} & Two Clause Learning Approaches for Disjunctive Scheduling & \href{../works/SialaAH15.pdf}{Yes} & \cite{SialaAH15} & 2015 & CP 2015 & 10 & 4 & 17 & \ref{b:SialaAH15} & \ref{c:SialaAH15}\\
\rowlabel{a:VilimLS15}VilimLS15 \href{https://doi.org/10.1007/978-3-319-18008-3_30}{VilimLS15} & \hyperref[auth:a121]{P. Vil{\'{\i}}m}, \hyperref[auth:a118]{P. Laborie}, \hyperref[auth:a120]{P. Shaw} & Failure-Directed Search for Constraint-Based Scheduling & \href{../works/VilimLS15.pdf}{Yes} & \cite{VilimLS15} & 2015 & CPAIOR 2015 & 17 & 31 & 19 & \ref{b:VilimLS15} & n/a\\
\rowlabel{a:ZhouGL15}ZhouGL15 \href{https://doi.org/10.1109/FSKD.2015.7382064}{ZhouGL15} & \hyperref[auth:a605]{J. Zhou}, \hyperref[auth:a606]{Y. Guo}, \hyperref[auth:a607]{G. Li} & On complex hybrid flexible flowshop scheduling problems based on constraint programming & \href{../works/ZhouGL15.pdf}{Yes} & \cite{ZhouGL15} & 2015 & FSKD 2015 & 5 & 0 & 16 & \ref{b:ZhouGL15} & n/a\\
\rowlabel{a:AlesioNBG14}AlesioNBG14 \href{https://doi.org/10.1007/978-3-319-10428-7_58}{AlesioNBG14} & \hyperref[auth:a236]{Stefano {Di Alesio}}, \hyperref[auth:a237]{S. Nejati}, \hyperref[auth:a238]{Lionel C. Briand}, \hyperref[auth:a200]{A. Gotlieb} & Worst-Case Scheduling of Software Tasks - {A} Constraint Optimization Model to Support Performance Testing & \href{../works/AlesioNBG14.pdf}{Yes} & \cite{AlesioNBG14} & 2014 & CP 2014 & 18 & 3 & 19 & \ref{b:AlesioNBG14} & n/a\\
\rowlabel{a:BartoliniBBLM14}BartoliniBBLM14 \href{https://doi.org/10.1007/978-3-319-10428-7_55}{BartoliniBBLM14} & \hyperref[auth:a230]{A. Bartolini}, \hyperref[auth:a231]{A. Borghesi}, \hyperref[auth:a232]{T. Bridi}, \hyperref[auth:a143]{M. Lombardi}, \hyperref[auth:a144]{M. Milano} & Proactive Workload Dispatching on the {EURORA} Supercomputer & \href{../works/BartoliniBBLM14.pdf}{Yes} & \cite{BartoliniBBLM14} & 2014 & CP 2014 & 16 & 12 & 3 & \ref{b:BartoliniBBLM14} & n/a\\
\rowlabel{a:BessiereHMQW14}BessiereHMQW14 \href{https://doi.org/10.1007/978-3-319-07046-9_23}{BessiereHMQW14} & \hyperref[auth:a331]{C. Bessiere}, \hyperref[auth:a1]{E. Hebrard}, \hyperref[auth:a332]{M. M{\'{e}}nard}, \hyperref[auth:a37]{C. Quimper}, \hyperref[auth:a278]{T. Walsh} & Buffered Resource Constraint: Algorithms and Complexity & \href{../works/BessiereHMQW14.pdf}{Yes} & \cite{BessiereHMQW14} & 2014 & CPAIOR 2014 & 16 & 1 & 3 & \ref{b:BessiereHMQW14} & n/a\\
\rowlabel{a:BofillEGPSV14}BofillEGPSV14 \href{https://doi.org/10.1007/978-3-319-10428-7_56}{BofillEGPSV14} & \hyperref[auth:a189]{M. Bofill}, \hyperref[auth:a233]{J. Espasa}, \hyperref[auth:a234]{M. Garcia}, \hyperref[auth:a235]{M. Palah{\'{\i}}}, \hyperref[auth:a191]{J. Suy}, \hyperref[auth:a192]{M. Villaret} & Scheduling {B2B} Meetings & \href{../works/BofillEGPSV14.pdf}{Yes} & \cite{BofillEGPSV14} & 2014 & CP 2014 & 16 & 3 & 10 & \ref{b:BofillEGPSV14} & n/a\\
\rowlabel{a:BonfiettiLM14}BonfiettiLM14 \href{https://doi.org/10.1007/978-3-319-07046-9_15}{BonfiettiLM14} & \hyperref[auth:a203]{A. Bonfietti}, \hyperref[auth:a143]{M. Lombardi}, \hyperref[auth:a144]{M. Milano} & Disregarding Duration Uncertainty in Partial Order Schedules? Yes, We Can! & \href{../works/BonfiettiLM14.pdf}{Yes} & \cite{BonfiettiLM14} & 2014 & CPAIOR 2014 & 16 & 3 & 12 & \ref{b:BonfiettiLM14} & n/a\\
\rowlabel{a:ChunS14}ChunS14 \href{https://doi.org/10.1609/aaai.v28i2.19013}{ChunS14} & \hyperref[auth:a1346]{Andy Hon Wai Chun}, \hyperref[auth:a1397]{Ted Yiu Tat Suen} & Engineering Works Scheduling for Hong Kong's Rail Network & \href{../works/ChunS14.pdf}{Yes} & \cite{ChunS14} & 2014 & AAAI 2014 & 8 & 3 & 0 & \ref{b:ChunS14} & n/a\\
\rowlabel{a:DejemeppeD14}DejemeppeD14 \href{https://doi.org/10.1007/978-3-319-07046-9_20}{DejemeppeD14} & \hyperref[auth:a207]{C. Dejemeppe}, \hyperref[auth:a152]{Y. Deville} & Continuously Degrading Resource and Interval Dependent Activity Durations in Nuclear Medicine Patient Scheduling & \href{../works/DejemeppeD14.pdf}{Yes} & \cite{DejemeppeD14} & 2014 & CPAIOR 2014 & 9 & 0 & 7 & \ref{b:DejemeppeD14} & \ref{c:DejemeppeD14}\\
\rowlabel{a:DerrienP14}DerrienP14 \href{https://doi.org/10.1007/978-3-319-10428-7_22}{DerrienP14} & \hyperref[auth:a225]{A. Derrien}, \hyperref[auth:a226]{T. Petit} & A New Characterization of Relevant Intervals for Energetic Reasoning & \href{../works/DerrienP14.pdf}{Yes} & \cite{DerrienP14} & 2014 & CP 2014 & 9 & 14 & 0 & \ref{b:DerrienP14} & n/a\\
\rowlabel{a:DerrienPZ14}DerrienPZ14 \href{https://doi.org/10.1007/978-3-319-10428-7_23}{DerrienPZ14} & \hyperref[auth:a225]{A. Derrien}, \hyperref[auth:a226]{T. Petit}, \hyperref[auth:a227]{S. Zampelli} & A Declarative Paradigm for Robust Cumulative Scheduling & \href{../works/DerrienPZ14.pdf}{Yes} & \cite{DerrienPZ14} & 2014 & CP 2014 & 9 & 3 & 10 & \ref{b:DerrienPZ14} & n/a\\
\rowlabel{a:DoulabiRP14}DoulabiRP14 \href{https://doi.org/10.1007/978-3-319-07046-9_32}{DoulabiRP14} & \hyperref[auth:a333]{Seyed Hossein Hashemi Doulabi}, \hyperref[auth:a329]{L. Rousseau}, \hyperref[auth:a8]{G. Pesant} & A Constraint Programming-Based Column Generation Approach for Operating Room Planning and Scheduling & \href{../works/DoulabiRP14.pdf}{Yes} & \cite{DoulabiRP14} & 2014 & CPAIOR 2014 & 9 & 3 & 10 & \ref{b:DoulabiRP14} & n/a\\
\rowlabel{a:FriedrichFMRSST14}FriedrichFMRSST14 \href{https://doi.org/10.1007/978-3-319-28697-6_23}{FriedrichFMRSST14} & \hyperref[auth:a608]{G. Friedrich}, \hyperref[auth:a609]{M. Fr{\"{u}}hst{\"{u}}ck}, \hyperref[auth:a610]{V. Mersheeva}, \hyperref[auth:a611]{A. Ryabokon}, \hyperref[auth:a612]{M. Sander}, \hyperref[auth:a613]{A. Starzacher}, \hyperref[auth:a614]{E. Teppan} & Representing Production Scheduling with Constraint Answer Set Programming & No & \cite{FriedrichFMRSST14} & 2014 & GOR 2014 & 7 & 3 & 2 & No & n/a\\
\rowlabel{a:GaySS14}GaySS14 \href{https://doi.org/10.1007/978-3-319-10428-7_59}{GaySS14} & \hyperref[auth:a216]{S. Gay}, \hyperref[auth:a148]{P. Schaus}, \hyperref[auth:a239]{Vivian De Smedt} & Continuous Casting Scheduling with Constraint Programming & \href{../works/GaySS14.pdf}{Yes} & \cite{GaySS14} & 2014 & CP 2014 & 15 & 7 & 11 & \ref{b:GaySS14} & n/a\\
\rowlabel{a:HoundjiSWD14}HoundjiSWD14 \href{https://doi.org/10.1007/978-3-319-10428-7_29}{HoundjiSWD14} & \hyperref[auth:a228]{Vinas{\'{e}}tan Ratheil Houndji}, \hyperref[auth:a148]{P. Schaus}, \hyperref[auth:a229]{Laurence A. Wolsey}, \hyperref[auth:a152]{Y. Deville} & The StockingCost Constraint & \href{../works/HoundjiSWD14.pdf}{Yes} & \cite{HoundjiSWD14} & 2014 & CP 2014 & 16 & 5 & 7 & \ref{b:HoundjiSWD14} & \ref{c:HoundjiSWD14}\\
\rowlabel{a:KoschB14}KoschB14 \href{https://doi.org/10.1007/978-3-319-07046-9_5}{KoschB14} & \hyperref[auth:a330]{S. Kosch}, \hyperref[auth:a89]{J. Christopher Beck} & A New {MIP} Model for Parallel-Batch Scheduling with Non-identical Job Sizes & \href{../works/KoschB14.pdf}{Yes} & \cite{KoschB14} & 2014 & CPAIOR 2014 & 16 & 4 & 18 & \ref{b:KoschB14} & n/a\\
\rowlabel{a:LipovetzkyBPS14}LipovetzkyBPS14 \href{http://www.aaai.org/ocs/index.php/ICAPS/ICAPS14/paper/view/7942}{LipovetzkyBPS14} & \hyperref[auth:a326]{N. Lipovetzky}, \hyperref[auth:a325]{Christina N. Burt}, \hyperref[auth:a327]{Adrian R. Pearce}, \hyperref[auth:a126]{Peter J. Stuckey} & Planning for Mining Operations with Time and Resource Constraints & \href{../works/LipovetzkyBPS14.pdf}{Yes} & \cite{LipovetzkyBPS14} & 2014 & ICAPS 2014 & 9 & 5 & 0 & \ref{b:LipovetzkyBPS14} & n/a\\
\rowlabel{a:LouieVNB14}LouieVNB14 \href{https://doi.org/10.1109/ICRA.2014.6907637}{LouieVNB14} & \hyperref[auth:a825]{Wing{-}Yue Geoffrey Louie}, \hyperref[auth:a810]{Tiago Stegun Vaquero}, \hyperref[auth:a209]{G. Nejat}, \hyperref[auth:a89]{J. Christopher Beck} & An autonomous assistive robot for planning, scheduling and facilitating multi-user activities & \href{../works/LouieVNB14.pdf}{Yes} & \cite{LouieVNB14} & 2014 & ICRA 2014 & 7 & 16 & 9 & \ref{b:LouieVNB14} & n/a\\
\rowlabel{a:LudwigKRBMS14}LudwigKRBMS14 \href{https://doi.org/10.1609/aaai.v28i2.19030}{LudwigKRBMS14} & \hyperref[auth:a1373]{J. Ludwig}, \hyperref[auth:a1374]{A. Kalton}, \hyperref[auth:a1375]{R. Richards}, \hyperref[auth:a1376]{B. Bautsch}, \hyperref[auth:a1377]{C. Markusic}, \hyperref[auth:a1378]{J. Schumacher} & A Schedule Optimization Tool for Destructive and Non-Destructive Vehicle Tests & \href{../works/LudwigKRBMS14.pdf}{Yes} & \cite{LudwigKRBMS14} & 2014 & AAAI 2014 & 6 & 1 & 0 & \ref{b:LudwigKRBMS14} & n/a\\
\rowlabel{a:BonfiettiLM13}BonfiettiLM13 \href{http://www.aaai.org/ocs/index.php/ICAPS/ICAPS13/paper/view/6050}{BonfiettiLM13} & \hyperref[auth:a203]{A. Bonfietti}, \hyperref[auth:a143]{M. Lombardi}, \hyperref[auth:a144]{M. Milano} & De-Cycling Cyclic Scheduling Problems & \href{../works/BonfiettiLM13.pdf}{Yes} & \cite{BonfiettiLM13} & 2013 & ICAPS 2013 & 5 & 1 & 0 & \ref{b:BonfiettiLM13} & n/a\\
\rowlabel{a:ChuGNSW13}ChuGNSW13 \href{http://www.aaai.org/ocs/index.php/IJCAI/IJCAI13/paper/view/6878}{ChuGNSW13} & \hyperref[auth:a346]{G. Chu}, \hyperref[auth:a799]{S. Gaspers}, \hyperref[auth:a800]{N. Narodytska}, \hyperref[auth:a125]{A. Schutt}, \hyperref[auth:a278]{T. Walsh} & On the Complexity of Global Scheduling Constraints under Structural Restrictions & \href{../works/ChuGNSW13.pdf}{Yes} & \cite{ChuGNSW13} & 2013 & IJCAI 2013 & 7 & 0 & 0 & \ref{b:ChuGNSW13} & n/a\\
\rowlabel{a:CireCH13}CireCH13 \href{https://doi.org/10.1007/978-3-642-38171-3_22}{CireCH13} & \hyperref[auth:a158]{Andr{\'{e}} A. Cir{\'{e}}}, \hyperref[auth:a338]{E. Coban}, \hyperref[auth:a161]{John N. Hooker} & Mixed Integer Programming vs. Logic-Based Benders Decomposition for Planning and Scheduling & \href{../works/CireCH13.pdf}{Yes} & \cite{CireCH13} & 2013 & CPAIOR 2013 & 7 & 3 & 23 & \ref{b:CireCH13} & \ref{c:CireCH13}\\
\rowlabel{a:GuSS13}GuSS13 \href{https://doi.org/10.1007/978-3-642-38171-3_24}{GuSS13} & \hyperref[auth:a339]{H. Gu}, \hyperref[auth:a125]{A. Schutt}, \hyperref[auth:a126]{Peter J. Stuckey} & A Lagrangian Relaxation Based Forward-Backward Improvement Heuristic for Maximising the Net Present Value of Resource-Constrained Projects & \href{../works/GuSS13.pdf}{Yes} & \cite{GuSS13} & 2013 & CPAIOR 2013 & 7 & 10 & 24 & \ref{b:GuSS13} & \ref{c:GuSS13}\\
\rowlabel{a:HamdiL13}HamdiL13 \href{http://dx.doi.org/10.1109/icmsao.2013.6552689}{HamdiL13} & \hyperref[auth:a1251]{I. Hamdi}, \hyperref[auth:a1252]{T. Loukil} & Logic-based Benders decomposition to solve the permutation flowshop scheduling problem with time lags & \href{../works/HamdiL13.pdf}{Yes} & \cite{HamdiL13} & 2013 & ICMSAO 2013 & 7 & 2 & 11 & \ref{b:HamdiL13} & n/a\\
\rowlabel{a:HeinzKB13}HeinzKB13 \href{https://doi.org/10.1007/978-3-642-38171-3_2}{HeinzKB13} & \hyperref[auth:a134]{S. Heinz}, \hyperref[auth:a334]{W. Ku}, \hyperref[auth:a89]{J. Christopher Beck} & Recent Improvements Using Constraint Integer Programming for Resource Allocation and Scheduling & \href{../works/HeinzKB13.pdf}{Yes} & \cite{HeinzKB13} & 2013 & CPAIOR 2013 & 16 & 9 & 15 & \ref{b:HeinzKB13} & n/a\\
\rowlabel{a:KelarevaTK13}KelarevaTK13 \href{https://doi.org/10.1007/978-3-642-38171-3_8}{KelarevaTK13} & \hyperref[auth:a335]{E. Kelareva}, \hyperref[auth:a336]{K. Tierney}, \hyperref[auth:a337]{P. Kilby} & {CP} Methods for Scheduling and Routing with Time-Dependent Task Costs & \href{../works/KelarevaTK13.pdf}{Yes} & \cite{KelarevaTK13} & 2013 & CPAIOR 2013 & 17 & 16 & 28 & \ref{b:KelarevaTK13} & \ref{c:KelarevaTK13}\\
\rowlabel{a:LetortCB13}LetortCB13 \href{https://doi.org/10.1007/978-3-642-38171-3_10}{LetortCB13} & \hyperref[auth:a128]{A. Letort}, \hyperref[auth:a91]{M. Carlsson}, \hyperref[auth:a129]{N. Beldiceanu} & A Synchronized Sweep Algorithm for the \emph{k-dimensional cumulative} Constraint & \href{../works/LetortCB13.pdf}{Yes} & \cite{LetortCB13} & 2013 & CPAIOR 2013 & 16 & 3 & 10 & \ref{b:LetortCB13} & \ref{c:LetortCB13}\\
\rowlabel{a:LombardiM13}LombardiM13 \href{http://www.aaai.org/ocs/index.php/ICAPS/ICAPS13/paper/view/6052}{LombardiM13} & \hyperref[auth:a143]{M. Lombardi}, \hyperref[auth:a144]{M. Milano} & A Min-Flow Algorithm for Minimal Critical Set Detection in Resource Constrained Project Scheduling & \href{../works/LombardiM13.pdf}{Yes} & \cite{LombardiM13} & 2013 & ICAPS 2013 & 2 & 3 & 13 & \ref{b:LombardiM13} & n/a\\
\rowlabel{a:MalapertCGJLR13}MalapertCGJLR13 \href{http://www.aaai.org/ocs/index.php/ICAPS/ICAPS13/paper/view/6016}{MalapertCGJLR13} & \hyperref[auth:a82]{A. Malapert}, \hyperref[auth:a1011]{H. Cambazard}, \hyperref[auth:a295]{C. Gu{\'{e}}ret}, \hyperref[auth:a249]{N. Jussien}, \hyperref[auth:a651]{A. Langevin}, \hyperref[auth:a329]{L. Rousseau} & An Optimal Constraint Programming Approach to the Open-Shop Problem & \href{../works/MalapertCGJLR13.pdf}{Yes} & \cite{MalapertCGJLR13} & 2013 & ICAPS 2013 & 2 & 0 & 0 & \ref{b:MalapertCGJLR13} & n/a\\
\rowlabel{a:OuelletQ13}OuelletQ13 \href{https://doi.org/10.1007/978-3-642-40627-0_42}{OuelletQ13} & \hyperref[auth:a240]{P. Ouellet}, \hyperref[auth:a37]{C. Quimper} & Time-Table Extended-Edge-Finding for the Cumulative Constraint & \href{../works/OuelletQ13.pdf}{Yes} & \cite{OuelletQ13} & 2013 & CP 2013 & 16 & 12 & 14 & \ref{b:OuelletQ13} & n/a\\
\rowlabel{a:SchuttFS13}SchuttFS13 \href{https://doi.org/10.1007/978-3-642-40627-0_47}{SchuttFS13} & \hyperref[auth:a125]{A. Schutt}, \hyperref[auth:a155]{T. Feydy}, \hyperref[auth:a126]{Peter J. Stuckey} & Scheduling Optional Tasks with Explanation & \href{../works/SchuttFS13.pdf}{Yes} & \cite{SchuttFS13} & 2013 & CP 2013 & 17 & 10 & 20 & \ref{b:SchuttFS13} & n/a\\
\rowlabel{a:SchuttFS13a}SchuttFS13a \href{https://doi.org/10.1007/978-3-642-38171-3_16}{SchuttFS13a} & \hyperref[auth:a125]{A. Schutt}, \hyperref[auth:a155]{T. Feydy}, \hyperref[auth:a126]{Peter J. Stuckey} & Explaining Time-Table-Edge-Finding Propagation for the Cumulative Resource Constraint & \href{../works/SchuttFS13a.pdf}{Yes} & \cite{SchuttFS13a} & 2013 & CPAIOR 2013 & 17 & 20 & 27 & \ref{b:SchuttFS13a} & \ref{c:SchuttFS13a}\\
\rowlabel{a:TranTDB13}TranTDB13 \href{http://www.aaai.org/ocs/index.php/ICAPS/ICAPS13/paper/view/6005}{TranTDB13} & \hyperref[auth:a805]{Tony T. Tran}, \hyperref[auth:a824]{D. Terekhov}, \hyperref[auth:a809]{Douglas G. Down}, \hyperref[auth:a89]{J. Christopher Beck} & Hybrid Queueing Theory and Scheduling Models for Dynamic Environments with Sequence-Dependent Setup Times & \href{../works/TranTDB13.pdf}{Yes} & \cite{TranTDB13} & 2013 & ICAPS 2013 & 9 & 2 & 0 & \ref{b:TranTDB13} & n/a\\
\rowlabel{a:ZampelliVSDR13}ZampelliVSDR13 \href{https://doi.org/10.1007/978-3-642-40627-0_64}{ZampelliVSDR13} & \hyperref[auth:a227]{S. Zampelli}, \hyperref[auth:a1226]{Y. Vergados}, \hyperref[auth:a1227]{Rowan Van Schaeren}, \hyperref[auth:a1228]{W. Dullaert}, \hyperref[auth:a1229]{B. Raa} & The Berth Allocation and Quay Crane Assignment Problem Using a {CP} Approach & \href{../works/ZampelliVSDR13.pdf}{Yes} & \cite{ZampelliVSDR13} & 2013 & CP 2013 & 17 & 20 & 19 & \ref{b:ZampelliVSDR13} & n/a\\
\rowlabel{a:BillautHL12}BillautHL12 \href{https://doi.org/10.1007/978-3-642-29828-8_5}{BillautHL12} & \hyperref[auth:a340]{J. Billaut}, \hyperref[auth:a1]{E. Hebrard}, \hyperref[auth:a3]{P. Lopez} & Complete Characterization of Near-Optimal Sequences for the Two-Machine Flow Shop Scheduling Problem & \href{../works/BillautHL12.pdf}{Yes} & \cite{BillautHL12} & 2012 & CPAIOR 2012 & 15 & 1 & 19 & \ref{b:BillautHL12} & n/a\\
\rowlabel{a:BonfiettiLBM12}BonfiettiLBM12 \href{https://doi.org/10.1007/978-3-642-29828-8_6}{BonfiettiLBM12} & \hyperref[auth:a203]{A. Bonfietti}, \hyperref[auth:a143]{M. Lombardi}, \hyperref[auth:a247]{L. Benini}, \hyperref[auth:a144]{M. Milano} & Global Cyclic Cumulative Constraint & \href{../works/BonfiettiLBM12.pdf}{Yes} & \cite{BonfiettiLBM12} & 2012 & CPAIOR 2012 & 16 & 2 & 11 & \ref{b:BonfiettiLBM12} & n/a\\
\rowlabel{a:BonfiettiM12}BonfiettiM12 \href{https://ceur-ws.org/Vol-926/paper2.pdf}{BonfiettiM12} & \hyperref[auth:a203]{A. Bonfietti}, \hyperref[auth:a144]{M. Milano} & A Constraint-based Approach to Cyclic Resource-Constrained Scheduling Problem & \href{../works/BonfiettiM12.pdf}{Yes} & \cite{BonfiettiM12} & 2012 & DC SIAAI 2012 & 3 & 0 & 0 & \ref{b:BonfiettiM12} & n/a\\
\rowlabel{a:GuSW12}GuSW12 \href{https://doi.org/10.1007/978-3-642-33558-7_55}{GuSW12} & \hyperref[auth:a339]{H. Gu}, \hyperref[auth:a126]{Peter J. Stuckey}, \hyperref[auth:a117]{Mark G. Wallace} & Maximising the Net Present Value of Large Resource-Constrained Projects & \href{../works/GuSW12.pdf}{Yes} & \cite{GuSW12} & 2012 & CP 2012 & 15 & 5 & 20 & \ref{b:GuSW12} & n/a\\
\rowlabel{a:HeinzB12}HeinzB12 \href{https://doi.org/10.1007/978-3-642-29828-8_14}{HeinzB12} & \hyperref[auth:a134]{S. Heinz}, \hyperref[auth:a89]{J. Christopher Beck} & Reconsidering Mixed Integer Programming and MIP-Based Hybrids for Scheduling & \href{../works/HeinzB12.pdf}{Yes} & \cite{HeinzB12} & 2012 & CPAIOR 2012 & 17 & 8 & 21 & \ref{b:HeinzB12} & n/a\\
\rowlabel{a:IfrimOS12}IfrimOS12 \href{https://doi.org/10.1007/978-3-642-33558-7_68}{IfrimOS12} & \hyperref[auth:a183]{G. Ifrim}, \hyperref[auth:a16]{B. O'Sullivan}, \hyperref[auth:a17]{H. Simonis} & Properties of Energy-Price Forecasts for Scheduling & \href{../works/IfrimOS12.pdf}{Yes} & \cite{IfrimOS12} & 2012 & CP 2012 & 16 & 6 & 20 & \ref{b:IfrimOS12} & n/a\\
\rowlabel{a:LetortBC12}LetortBC12 \href{https://doi.org/10.1007/978-3-642-33558-7_33}{LetortBC12} & \hyperref[auth:a128]{A. Letort}, \hyperref[auth:a129]{N. Beldiceanu}, \hyperref[auth:a91]{M. Carlsson} & A Scalable Sweep Algorithm for the cumulative Constraint & \href{../works/LetortBC12.pdf}{Yes} & \cite{LetortBC12} & 2012 & CP 2012 & 16 & 18 & 12 & \ref{b:LetortBC12} & n/a\\
\rowlabel{a:LozanoCDS12}LozanoCDS12 \href{https://doi.org/10.1007/978-3-642-33558-7_54}{LozanoCDS12} & \hyperref[auth:a1245]{Roberto Casta{\~{n}}eda Lozano}, \hyperref[auth:a91]{M. Carlsson}, \hyperref[auth:a1246]{F. Drejhammar}, \hyperref[auth:a92]{C. Schulte} & Constraint-Based Register Allocation and Instruction Scheduling & \href{../works/LozanoCDS12.pdf}{Yes} & \cite{LozanoCDS12} & 2012 & CP 2012 & 17 & 21 & 30 & \ref{b:LozanoCDS12} & n/a\\
\rowlabel{a:RendlPHPR12}RendlPHPR12 \href{https://doi.org/10.1007/978-3-642-29828-8_22}{RendlPHPR12} & \hyperref[auth:a341]{A. Rendl}, \hyperref[auth:a342]{M. Prandtstetter}, \hyperref[auth:a343]{G. Hiermann}, \hyperref[auth:a344]{J. Puchinger}, \hyperref[auth:a345]{G{\"{u}}nther R. Raidl} & Hybrid Heuristics for Multimodal Homecare Scheduling & \href{../works/RendlPHPR12.pdf}{Yes} & \cite{RendlPHPR12} & 2012 & CPAIOR 2012 & 17 & 14 & 14 & \ref{b:RendlPHPR12} & n/a\\
\rowlabel{a:SchuttCSW12}SchuttCSW12 \href{https://doi.org/10.1007/978-3-642-29828-8_24}{SchuttCSW12} & \hyperref[auth:a125]{A. Schutt}, \hyperref[auth:a346]{G. Chu}, \hyperref[auth:a126]{Peter J. Stuckey}, \hyperref[auth:a117]{Mark G. Wallace} & Maximising the Net Present Value for Resource-Constrained Project Scheduling & \href{../works/SchuttCSW12.pdf}{Yes} & \cite{SchuttCSW12} & 2012 & CPAIOR 2012 & 17 & 18 & 21 & \ref{b:SchuttCSW12} & n/a\\
\rowlabel{a:SerraNM12}SerraNM12 \href{https://doi.org/10.1007/978-3-642-33558-7_59}{SerraNM12} & \hyperref[auth:a241]{T. Serra}, \hyperref[auth:a242]{G. Nishioka}, \hyperref[auth:a243]{Fernando J. M. Marcellino} & The Offshore Resources Scheduling Problem: Detailing a Constraint Programming Approach & \href{../works/SerraNM12.pdf}{Yes} & \cite{SerraNM12} & 2012 & CP 2012 & 17 & 0 & 8 & \ref{b:SerraNM12} & n/a\\
\rowlabel{a:SimoninAHL12}SimoninAHL12 \href{https://doi.org/10.1007/978-3-642-33558-7_5}{SimoninAHL12} & \hyperref[auth:a127]{G. Simonin}, \hyperref[auth:a6]{C. Artigues}, \hyperref[auth:a1]{E. Hebrard}, \hyperref[auth:a3]{P. Lopez} & Scheduling Scientific Experiments on the Rosetta/Philae Mission & \href{../works/SimoninAHL12.pdf}{Yes} & \cite{SimoninAHL12} & 2012 & CP 2012 & 15 & 3 & 8 & \ref{b:SimoninAHL12} & \ref{c:SimoninAHL12}\\
\rowlabel{a:TranB12}TranB12 \href{https://doi.org/10.3233/978-1-61499-098-7-774}{TranB12} & \hyperref[auth:a805]{Tony T. Tran}, \hyperref[auth:a89]{J. Christopher Beck} & Logic-based Benders Decomposition for Alternative Resource Scheduling with Sequence Dependent Setups & \href{../works/TranB12.pdf}{Yes} & \cite{TranB12} & 2012 & ECAI 2012 & 6 & 0 & 0 & \ref{b:TranB12} & n/a\\
\rowlabel{a:ZhangLS12}ZhangLS12 \href{https://doi.org/10.1109/CIT.2012.96}{ZhangLS12} & \hyperref[auth:a617]{X. Zhang}, \hyperref[auth:a618]{Z. Lv}, \hyperref[auth:a619]{X. Song} & Model and Solution for Hot Strip Rolling Scheduling Problem Based on Constraint Programming Method & \href{../works/ZhangLS12.pdf}{Yes} & \cite{ZhangLS12} & 2012 & CIT 2012 & 4 & 1 & 3 & \ref{b:ZhangLS12} & n/a\\
\rowlabel{a:BajestaniB11}BajestaniB11 \href{http://aaai.org/ocs/index.php/ICAPS/ICAPS11/paper/view/2680}{BajestaniB11} & \hyperref[auth:a823]{Maliheh Aramon Bajestani}, \hyperref[auth:a89]{J. Christopher Beck} & Scheduling an Aircraft Repair Shop & \href{../works/BajestaniB11.pdf}{Yes} & \cite{BajestaniB11} & 2011 & ICAPS 2011 & 8 & 2 & 0 & \ref{b:BajestaniB11} & n/a\\
\rowlabel{a:Balduccini11}Balduccini11 \href{https://doi.org/10.1007/978-3-642-20895-9_33}{Balduccini11} & \hyperref[auth:a1057]{M. Balduccini} & Industrial-Size Scheduling with {ASP+CP} & \href{../works/Balduccini11.pdf}{Yes} & \cite{Balduccini11} & 2011 & LPNMR 2011 & 13 & 20 & 9 & \ref{b:Balduccini11} & n/a\\
\rowlabel{a:BonfiettiLBM11}BonfiettiLBM11 \href{https://doi.org/10.1007/978-3-642-23786-7_12}{BonfiettiLBM11} & \hyperref[auth:a203]{A. Bonfietti}, \hyperref[auth:a143]{M. Lombardi}, \hyperref[auth:a247]{L. Benini}, \hyperref[auth:a144]{M. Milano} & A Constraint Based Approach to Cyclic {RCPSP} & \href{../works/BonfiettiLBM11.pdf}{Yes} & \cite{BonfiettiLBM11} & 2011 & CP 2011 & 15 & 3 & 14 & \ref{b:BonfiettiLBM11} & n/a\\
\rowlabel{a:ChapadosJR11}ChapadosJR11 \href{https://doi.org/10.1007/978-3-642-21311-3_7}{ChapadosJR11} & \hyperref[auth:a347]{N. Chapados}, \hyperref[auth:a348]{M. Joliveau}, \hyperref[auth:a329]{L. Rousseau} & Retail Store Workforce Scheduling by Expected Operating Income Maximization & \href{../works/ChapadosJR11.pdf}{Yes} & \cite{ChapadosJR11} & 2011 & CPAIOR 2011 & 6 & 5 & 12 & \ref{b:ChapadosJR11} & n/a\\
\rowlabel{a:ClercqPBJ11}ClercqPBJ11 \href{https://doi.org/10.1007/978-3-642-23786-7_20}{ClercqPBJ11} & \hyperref[auth:a248]{Alexis De Clercq}, \hyperref[auth:a226]{T. Petit}, \hyperref[auth:a129]{N. Beldiceanu}, \hyperref[auth:a249]{N. Jussien} & Filtering Algorithms for Discrete Cumulative Problems with Overloads of Resource & \href{../works/ClercqPBJ11.pdf}{Yes} & \cite{ClercqPBJ11} & 2011 & CP 2011 & 16 & 3 & 11 & \ref{b:ClercqPBJ11} & n/a\\
\rowlabel{a:EdisO11}EdisO11 \href{https://doi.org/10.1007/978-3-642-21311-3_10}{EdisO11} & \hyperref[auth:a349]{Emrah B. Edis}, \hyperref[auth:a350]{C. Oguz} & Parallel Machine Scheduling with Additional Resources: {A} Lagrangian-Based Constraint Programming Approach & \href{../works/EdisO11.pdf}{Yes} & \cite{EdisO11} & 2011 & CPAIOR 2011 & 7 & 5 & 16 & \ref{b:EdisO11} & n/a\\
\rowlabel{a:GrimesH11}GrimesH11 \href{https://doi.org/10.1007/978-3-642-23786-7_28}{GrimesH11} & \hyperref[auth:a182]{D. Grimes}, \hyperref[auth:a1]{E. Hebrard} & Models and Strategies for Variants of the Job Shop Scheduling Problem & \href{../works/GrimesH11.pdf}{Yes} & \cite{GrimesH11} & 2011 & CP 2011 & 17 & 5 & 18 & \ref{b:GrimesH11} & n/a\\
\rowlabel{a:HeinzS11}HeinzS11 \href{https://doi.org/10.1007/978-3-642-20662-7_34}{HeinzS11} & \hyperref[auth:a134]{S. Heinz}, \hyperref[auth:a135]{J. Schulz} & Explanations for the Cumulative Constraint: An Experimental Study & \href{../works/HeinzS11.pdf}{Yes} & \cite{HeinzS11} & 2011 & SEA 2011 & 10 & 5 & 12 & \ref{b:HeinzS11} & n/a\\
\rowlabel{a:HermenierDL11}HermenierDL11 \href{https://doi.org/10.1007/978-3-642-23786-7_5}{HermenierDL11} & \hyperref[auth:a244]{F. Hermenier}, \hyperref[auth:a245]{S. Demassey}, \hyperref[auth:a246]{X. Lorca} & Bin Repacking Scheduling in Virtualized Datacenters & \href{../works/HermenierDL11.pdf}{Yes} & \cite{HermenierDL11} & 2011 & CP 2011 & 15 & 28 & 5 & \ref{b:HermenierDL11} & n/a\\
\rowlabel{a:KameugneFSN11}KameugneFSN11 \href{https://doi.org/10.1007/978-3-642-23786-7_37}{KameugneFSN11} & \hyperref[auth:a10]{R. Kameugne}, \hyperref[auth:a131]{Laure Pauline Fotso}, \hyperref[auth:a132]{Joseph D. Scott}, \hyperref[auth:a133]{Y. Ngo{-}Kateu} & A Quadratic Edge-Finding Filtering Algorithm for Cumulative Resource Constraints & \href{../works/KameugneFSN11.pdf}{Yes} & \cite{KameugneFSN11} & 2011 & CP 2011 & 15 & 7 & 9 & \ref{b:KameugneFSN11} & n/a\\
\rowlabel{a:LahimerLH11}LahimerLH11 \href{https://doi.org/10.1007/978-3-642-21311-3_12}{LahimerLH11} & \hyperref[auth:a352]{A. Lahimer}, \hyperref[auth:a3]{P. Lopez}, \hyperref[auth:a353]{M. Haouari} & Climbing Depth-Bounded Adjacent Discrepancy Search for Solving Hybrid Flow Shop Scheduling Problems with Multiprocessor Tasks & \href{../works/LahimerLH11.pdf}{Yes} & \cite{LahimerLH11} & 2011 & CPAIOR 2011 & 14 & 3 & 15 & \ref{b:LahimerLH11} & n/a\\
\rowlabel{a:LombardiBMB11}LombardiBMB11 \href{https://doi.org/10.1007/978-3-642-21311-3_14}{LombardiBMB11} & \hyperref[auth:a143]{M. Lombardi}, \hyperref[auth:a203]{A. Bonfietti}, \hyperref[auth:a144]{M. Milano}, \hyperref[auth:a247]{L. Benini} & Precedence Constraint Posting for Cyclic Scheduling Problems & \href{../works/LombardiBMB11.pdf}{Yes} & \cite{LombardiBMB11} & 2011 & CPAIOR 2011 & 17 & 1 & 13 & \ref{b:LombardiBMB11} & n/a\\
\rowlabel{a:MeskensDHG11}MeskensDHG11 \href{}{MeskensDHG11} & \hyperref[auth:a603]{N. Meskens}, \hyperref[auth:a604]{D. Duvivier}, \hyperref[auth:a1398]{A. Hanset}, \hyperref[auth:a1399]{D. Gossart} & Multi-objective Constraint programming for Scheduling Operating Theatres & No & \cite{MeskensDHG11} & 2011 & unknown 2011 & 10 & 0 & 0 & No & n/a\\
\rowlabel{a:SimonisH11}SimonisH11 \href{http://dx.doi.org/10.1007/978-3-642-19486-3_5}{SimonisH11} & \hyperref[auth:a17]{H. Simonis}, \hyperref[auth:a913]{T. Hadzic} & A Resource Cost Aware Cumulative & \href{../works/SimonisH11.pdf}{Yes} & \cite{SimonisH11} & 2011 & CSCLP 2011 & 14 & 3 & 9 & \ref{b:SimonisH11} & n/a\\
\rowlabel{a:Vilim11}Vilim11 \href{https://doi.org/10.1007/978-3-642-21311-3_22}{Vilim11} & \hyperref[auth:a121]{P. Vil{\'{\i}}m} & Timetable Edge Finding Filtering Algorithm for Discrete Cumulative Resources & \href{../works/Vilim11.pdf}{Yes} & \cite{Vilim11} & 2011 & CPAIOR 2011 & 16 & 28 & 6 & \ref{b:Vilim11} & n/a\\
\rowlabel{a:Wolf11}Wolf11 \href{http://dx.doi.org/10.1007/978-3-642-19486-3_8}{Wolf11} & \hyperref[auth:a51]{A. Wolf} & Constraint-Based Modeling and Scheduling of Clinical Pathways & \href{../works/Wolf11.pdf}{Yes} & \cite{Wolf11} & 2011 & CSCLP 2011 & 17 & 5 & 19 & \ref{b:Wolf11} & n/a\\
\rowlabel{a:ZibranR11}ZibranR11 \href{https://doi.org/10.1109/ICPC.2011.45}{ZibranR11} & \hyperref[auth:a625]{Minhaz F. Zibran}, \hyperref[auth:a626]{Chanchal K. Roy} & Conflict-Aware Optimal Scheduling of Code Clone Refactoring: {A} Constraint Programming Approach & \href{../works/ZibranR11.pdf}{Yes} & \cite{ZibranR11} & 2011 & ICPC 2011 & 4 & 17 & 18 & \ref{b:ZibranR11} & n/a\\
\rowlabel{a:ZibranR11a}ZibranR11a \href{https://doi.org/10.1109/SCAM.2011.21}{ZibranR11a} & \hyperref[auth:a625]{Minhaz F. Zibran}, \hyperref[auth:a626]{Chanchal K. Roy} & A Constraint Programming Approach to Conflict-Aware Optimal Scheduling of Prioritized Code Clone Refactoring & \href{../works/ZibranR11a.pdf}{Yes} & \cite{ZibranR11a} & 2011 & SCAM 2011 & 10 & 26 & 27 & \ref{b:ZibranR11a} & n/a\\
\rowlabel{a:Beck10}Beck10 \href{https://doi.org/10.1007/978-3-642-15396-9_10}{Beck10} & \hyperref[auth:a89]{J. Christopher Beck} & Checking-Up on Branch-and-Check & \href{../works/Beck10.pdf}{Yes} & \cite{Beck10} & 2010 & CP 2010 & 15 & 19 & 11 & \ref{b:Beck10} & n/a\\
\rowlabel{a:BertholdHLMS10}BertholdHLMS10 \href{https://doi.org/10.1007/978-3-642-13520-0_34}{BertholdHLMS10} & \hyperref[auth:a354]{T. Berthold}, \hyperref[auth:a134]{S. Heinz}, \hyperref[auth:a355]{Marco E. L{\"{u}}bbecke}, \hyperref[auth:a356]{Rolf H. M{\"{o}}hring}, \hyperref[auth:a135]{J. Schulz} & A Constraint Integer Programming Approach for Resource-Constrained Project Scheduling & \href{../works/BertholdHLMS10.pdf}{Yes} & \cite{BertholdHLMS10} & 2010 & CPAIOR 2010 & 5 & 28 & 10 & \ref{b:BertholdHLMS10} & n/a\\
\rowlabel{a:CobanH10}CobanH10 \href{https://doi.org/10.1007/978-3-642-13520-0_11}{CobanH10} & \hyperref[auth:a338]{E. Coban}, \hyperref[auth:a161]{John N. Hooker} & Single-Facility Scheduling over Long Time Horizons by Logic-Based Benders Decomposition & \href{../works/CobanH10.pdf}{Yes} & \cite{CobanH10} & 2010 & CPAIOR 2010 & 5 & 9 & 9 & \ref{b:CobanH10} & n/a\\
\rowlabel{a:Davenport10}Davenport10 \href{https://doi.org/10.1007/978-3-642-13520-0_12}{Davenport10} & \hyperref[auth:a250]{Andrew J. Davenport} & Integrated Maintenance Scheduling for Semiconductor Manufacturing & \href{../works/Davenport10.pdf}{Yes} & \cite{Davenport10} & 2010 & CPAIOR 2010 & 5 & 9 & 2 & \ref{b:Davenport10} & n/a\\
\rowlabel{a:GrimesH10}GrimesH10 \href{https://doi.org/10.1007/978-3-642-13520-0_19}{GrimesH10} & \hyperref[auth:a182]{D. Grimes}, \hyperref[auth:a1]{E. Hebrard} & Job Shop Scheduling with Setup Times and Maximal Time-Lags: {A} Simple Constraint Programming Approach & \href{../works/GrimesH10.pdf}{Yes} & \cite{GrimesH10} & 2010 & CPAIOR 2010 & 15 & 13 & 20 & \ref{b:GrimesH10} & n/a\\
\rowlabel{a:LombardiM10}LombardiM10 \href{https://doi.org/10.1007/978-3-642-15396-9_32}{LombardiM10} & \hyperref[auth:a143]{M. Lombardi}, \hyperref[auth:a144]{M. Milano} & Constraint Based Scheduling to Deal with Uncertain Durations and Self-Timed Execution & \href{../works/LombardiM10.pdf}{Yes} & \cite{LombardiM10} & 2010 & CP 2010 & 15 & 1 & 11 & \ref{b:LombardiM10} & n/a\\
\rowlabel{a:MakMS10}MakMS10 \href{https://doi.org/10.1109/ICNC.2010.5583494}{MakMS10} & \hyperref[auth:a633]{K. Mak}, \hyperref[auth:a634]{J. Ma}, \hyperref[auth:a635]{W. Su} & A constraint programming approach for production scheduling of multi-period virtual cellular manufacturing systems & \href{../works/MakMS10.pdf}{Yes} & \cite{MakMS10} & 2010 & ICNC 2010 & 5 & 1 & 3 & \ref{b:MakMS10} & n/a\\
\rowlabel{a:OddiRC10}OddiRC10 \href{https://doi.org/10.3233/978-1-60750-606-5-967}{OddiRC10} & \hyperref[auth:a284]{A. Oddi}, \hyperref[auth:a1294]{R. Rasconi}, \hyperref[auth:a286]{A. Cesta} & Project Scheduling as a Disjunctive Temporal Problem & \href{../works/OddiRC10.pdf}{Yes} & \cite{OddiRC10} & 2010 & ECAI 2010 & 2 & 0 & 0 & \ref{b:OddiRC10} & n/a\\
\rowlabel{a:SchuttW10}SchuttW10 \href{https://doi.org/10.1007/978-3-642-15396-9_36}{SchuttW10} & \hyperref[auth:a125]{A. Schutt}, \hyperref[auth:a51]{A. Wolf} & A New \emph{O}(\emph{n}\({}^{\mbox{2}}\)log\emph{n}) Not-First/Not-Last Pruning Algorithm for Cumulative Resource Constraints & \href{../works/SchuttW10.pdf}{Yes} & \cite{SchuttW10} & 2010 & CP 2010 & 15 & 13 & 14 & \ref{b:SchuttW10} & n/a\\
\rowlabel{a:SunLYL10}SunLYL10 \href{https://doi.org/10.1109/GreenCom-CPSCom.2010.111}{SunLYL10} & \hyperref[auth:a629]{Z. Sun}, \hyperref[auth:a630]{H. Li}, \hyperref[auth:a631]{M. Yao}, \hyperref[auth:a632]{N. Li} & Scheduling Optimization Techniques for FlexRay Using Constraint-Programming & \href{../works/SunLYL10.pdf}{Yes} & \cite{SunLYL10} & 2010 & GreenCom 2010 & 6 & 4 & 8 & \ref{b:SunLYL10} & n/a\\
\rowlabel{a:TanSD10}TanSD10 \href{http://dx.doi.org/10.1109/ccdc.2010.5499100}{TanSD10} & \hyperref[auth:a1203]{Y. Tan}, \hyperref[auth:a468]{S. Liu}, \hyperref[auth:a1239]{D. Wang} & A constraint programming-based branch and bound algorithm for job shop problems & \href{../works/TanSD10.pdf}{Yes} & \cite{TanSD10} & 2010 & Chinese Control and Decision Conference 2010 & 6 & 1 & 11 & \ref{b:TanSD10} & n/a\\
\rowlabel{a:Acuna-AgostMFG09}Acuna-AgostMFG09 \href{https://doi.org/10.1007/978-3-642-01929-6_24}{Acuna-AgostMFG09} & \hyperref[auth:a357]{R. Acuna{-}Agost}, \hyperref[auth:a358]{P. Michelon}, \hyperref[auth:a359]{D. Feillet}, \hyperref[auth:a360]{S. Gueye} & Constraint Programming and Mixed Integer Linear Programming for Rescheduling Trains under Disrupted Operations & \href{../works/Acuna-AgostMFG09.pdf}{Yes} & \cite{Acuna-AgostMFG09} & 2009 & CPAIOR 2009 & 2 & 3 & 2 & \ref{b:Acuna-AgostMFG09} & n/a\\
\rowlabel{a:AronssonBK09}AronssonBK09 \href{http://drops.dagstuhl.de/opus/volltexte/2009/2141}{AronssonBK09} & \hyperref[auth:a713]{M. Aronsson}, \hyperref[auth:a714]{M. Bohlin}, \hyperref[auth:a715]{P. Kreuger} & {MILP} formulations of cumulative constraints for railway scheduling - {A} comparative study & \href{../works/AronssonBK09.pdf}{Yes} & \cite{AronssonBK09} & 2009 & ATMOS 2009 & 13 & 0 & 0 & \ref{b:AronssonBK09} & n/a\\
\rowlabel{a:Baptiste09}Baptiste09 \href{https://doi.org/10.1007/978-3-642-04244-7_1}{Baptiste09} & \hyperref[auth:a163]{P. Baptiste} & Constraint-Based Schedulers, Do They Really Work? & \href{../works/Baptiste09.pdf}{Yes} & \cite{Baptiste09} & 2009 & CP 2009 & 1 & 0 & 0 & \ref{b:Baptiste09} & n/a\\
\rowlabel{a:GrimesHM09}GrimesHM09 \href{https://doi.org/10.1007/978-3-642-04244-7_33}{GrimesHM09} & \hyperref[auth:a182]{D. Grimes}, \hyperref[auth:a1]{E. Hebrard}, \hyperref[auth:a82]{A. Malapert} & Closing the Open Shop: Contradicting Conventional Wisdom & \href{../works/GrimesHM09.pdf}{Yes} & \cite{GrimesHM09} & 2009 & CP 2009 & 9 & 15 & 12 & \ref{b:GrimesHM09} & n/a\\
\rowlabel{a:Laborie09}Laborie09 \href{https://doi.org/10.1007/978-3-642-01929-6_12}{Laborie09} & \hyperref[auth:a118]{P. Laborie} & {IBM} {ILOG} {CP} Optimizer for Detailed Scheduling Illustrated on Three Problems & \href{../works/Laborie09.pdf}{Yes} & \cite{Laborie09} & 2009 & CPAIOR 2009 & 15 & 53 & 2 & \ref{b:Laborie09} & n/a\\
\rowlabel{a:LombardiM09}LombardiM09 \href{https://doi.org/10.1007/978-3-642-04244-7_45}{LombardiM09} & \hyperref[auth:a143]{M. Lombardi}, \hyperref[auth:a144]{M. Milano} & A Precedence Constraint Posting Approach for the {RCPSP} with Time Lags and Variable Durations & \href{../works/LombardiM09.pdf}{Yes} & \cite{LombardiM09} & 2009 & CP 2009 & 15 & 7 & 12 & \ref{b:LombardiM09} & n/a\\
\rowlabel{a:MonetteDH09}MonetteDH09 \href{http://aaai.org/ocs/index.php/ICAPS/ICAPS09/paper/view/712}{MonetteDH09} & \hyperref[auth:a150]{J. Monette}, \hyperref[auth:a152]{Y. Deville}, \hyperref[auth:a149]{Pascal Van Hentenryck} & Just-In-Time Scheduling with Constraint Programming & \href{../works/MonetteDH09.pdf}{Yes} & \cite{MonetteDH09} & 2009 & ICAPS 2009 & 8 & 9 & 0 & \ref{b:MonetteDH09} & n/a\\
\rowlabel{a:RenT09}RenT09 \href{http://dx.doi.org/10.1109/ical.2009.5262795}{RenT09} & \hyperref[auth:a1271]{H. Ren}, \hyperref[auth:a1216]{L. Tang} & An improved hybrid MILP/CP algorithm framework for the job-shop scheduling & \href{../works/RenT09.pdf}{Yes} & \cite{RenT09} & 2009 & IEEE International Conference on Automation and Logistics 2009 & 5 & 2 & 12 & \ref{b:RenT09} & n/a\\
\rowlabel{a:RodriguezS09}RodriguezS09 \href{}{RodriguezS09} & \hyperref[auth:a787]{J. Rodriguez}, \hyperref[auth:a1030]{S. Sobieraj} & A study of an incremental texture-based heuristic for the train routing and scheduling problem & \href{../works/RodriguezS09.pdf}{Yes} & \cite{RodriguezS09} & 2009 & ICROMA 2009 & 14 & 0 & 0 & \ref{b:RodriguezS09} & n/a\\
\rowlabel{a:SchuttFSW09}SchuttFSW09 \href{https://doi.org/10.1007/978-3-642-04244-7_58}{SchuttFSW09} & \hyperref[auth:a125]{A. Schutt}, \hyperref[auth:a155]{T. Feydy}, \hyperref[auth:a126]{Peter J. Stuckey}, \hyperref[auth:a117]{Mark G. Wallace} & Why Cumulative Decomposition Is Not as Bad as It Sounds & \href{../works/SchuttFSW09.pdf}{Yes} & \cite{SchuttFSW09} & 2009 & CP 2009 & 16 & 34 & 11 & \ref{b:SchuttFSW09} & n/a\\
\rowlabel{a:ThiruvadyBME09}ThiruvadyBME09 \href{https://doi.org/10.1007/978-3-642-04918-7_3}{ThiruvadyBME09} & \hyperref[auth:a399]{Dhananjay R. Thiruvady}, \hyperref[auth:a642]{C. Blum}, \hyperref[auth:a643]{B. Meyer}, \hyperref[auth:a472]{Andreas T. Ernst} & Hybridizing Beam-ACO with Constraint Programming for Single Machine Job Scheduling & \href{../works/ThiruvadyBME09.pdf}{Yes} & \cite{ThiruvadyBME09} & 2009 & HM 2009 & 15 & 13 & 12 & \ref{b:ThiruvadyBME09} & n/a\\
\rowlabel{a:Vilim09}Vilim09 \href{https://doi.org/10.1007/978-3-642-04244-7_62}{Vilim09} & \hyperref[auth:a121]{P. Vil{\'{\i}}m} & Edge Finding Filtering Algorithm for Discrete Cumulative Resources in \emph{O}(\emph{kn} log \emph{n})\{{\textbackslash}mathcal O\}(kn \{{\textbackslash}rm log\} n) & \href{../works/Vilim09.pdf}{Yes} & \cite{Vilim09} & 2009 & CP 2009 & 15 & 25 & 4 & \ref{b:Vilim09} & n/a\\
\rowlabel{a:Vilim09a}Vilim09a \href{https://doi.org/10.1007/978-3-642-01929-6_22}{Vilim09a} & \hyperref[auth:a121]{P. Vil{\'{\i}}m} & Max Energy Filtering Algorithm for Discrete Cumulative Resources & \href{../works/Vilim09a.pdf}{Yes} & \cite{Vilim09a} & 2009 & CPAIOR 2009 & 15 & 13 & 4 & \ref{b:Vilim09a} & n/a\\
\rowlabel{a:Wolf09}Wolf09 \href{http://dx.doi.org/10.1007/978-3-642-00675-3_2}{Wolf09} & \hyperref[auth:a51]{A. Wolf}, \hyperref[auth:a716]{G. Schrader} & Linear Weighted-Task-Sum – Scheduling Prioritized Tasks on a Single Resource & \href{../works/Wolf09.pdf}{Yes} & \cite{Wolf09} & 2009 & INAP 2009 & 17 & 1 & 12 & \ref{b:Wolf09} & n/a\\
\rowlabel{a:AchterbergBKW08}AchterbergBKW08 \href{https://doi.org/10.1007/978-3-540-68155-7_4}{AchterbergBKW08} & \hyperref[auth:a1059]{T. Achterberg}, \hyperref[auth:a354]{T. Berthold}, \hyperref[auth:a1187]{T. Koch}, \hyperref[auth:a1188]{K. Wolter} & Constraint Integer Programming: {A} New Approach to Integrate {CP} and {MIP} & \href{../works/AchterbergBKW08.pdf}{Yes} & \cite{AchterbergBKW08} & 2008 & CPAIOR 2008 & 15 & 80 & 25 & \ref{b:AchterbergBKW08} & n/a\\
\rowlabel{a:BarlattCG08}BarlattCG08 \href{https://doi.org/10.1007/978-3-540-68155-7_24}{BarlattCG08} & \hyperref[auth:a364]{A. Barlatt}, \hyperref[auth:a365]{Amy Mainville Cohn}, \hyperref[auth:a366]{Oleg Yu. Gusikhin} & A Hybrid Approach for Solving Shift-Selection and Task-Sequencing Problems & \href{../works/BarlattCG08.pdf}{Yes} & \cite{BarlattCG08} & 2008 & CPAIOR 2008 & 5 & 1 & 9 & \ref{b:BarlattCG08} & n/a\\
\rowlabel{a:BeldiceanuCP08}BeldiceanuCP08 \href{https://doi.org/10.1007/978-3-540-68155-7_5}{BeldiceanuCP08} & \hyperref[auth:a129]{N. Beldiceanu}, \hyperref[auth:a91]{M. Carlsson}, \hyperref[auth:a361]{E. Poder} & New Filtering for the cumulative Constraint in the Context of Non-Overlapping Rectangles & \href{../works/BeldiceanuCP08.pdf}{Yes} & \cite{BeldiceanuCP08} & 2008 & CPAIOR 2008 & 15 & 8 & 9 & \ref{b:BeldiceanuCP08} & n/a\\
\rowlabel{a:BeniniLMMR08}BeniniLMMR08 \href{https://doi.org/10.1007/978-3-540-68155-7_6}{BeniniLMMR08} & \hyperref[auth:a247]{L. Benini}, \hyperref[auth:a143]{M. Lombardi}, \hyperref[auth:a1168]{M. Mantovani}, \hyperref[auth:a144]{M. Milano}, \hyperref[auth:a724]{M. Ruggiero} & Multi-stage Benders Decomposition for Optimizing Multicore Architectures & \href{../works/BeniniLMMR08.pdf}{Yes} & \cite{BeniniLMMR08} & 2008 & CPAIOR 2008 & 15 & 12 & 13 & \ref{b:BeniniLMMR08} & n/a\\
\rowlabel{a:BeniniLMR08}BeniniLMR08 \href{http://dx.doi.org/10.1007/978-3-540-85958-1_2}{BeniniLMR08} & \hyperref[auth:a247]{L. Benini}, \hyperref[auth:a143]{M. Lombardi}, \hyperref[auth:a144]{M. Milano}, \hyperref[auth:a724]{M. Ruggiero} & A Constraint Programming Approach for Allocation and Scheduling on the CELL Broadband Engine & \href{../works/BeniniLMR08.pdf}{Yes} & \cite{BeniniLMR08} & 2008 & CP 2008 & 15 & 7 & 23 & \ref{b:BeniniLMR08} & n/a\\
\rowlabel{a:DoRZ08}DoRZ08 \href{http://www.aaai.org/Library/AAAI/2008/aaai08-253.php}{DoRZ08} & \hyperref[auth:a1370]{Minh Binh Do}, \hyperref[auth:a1371]{W. Ruml}, \hyperref[auth:a1372]{R. Zhou} & On-line Planning and Scheduling: An Application to Controlling Modular Printers & \href{../works/DoRZ08.pdf}{Yes} & \cite{DoRZ08} & 2008 & AAAI 2008 & 5 & 0 & 0 & \ref{b:DoRZ08} & n/a\\
\rowlabel{a:DoomsH08}DoomsH08 \href{https://doi.org/10.1007/978-3-540-68155-7_8}{DoomsH08} & \hyperref[auth:a362]{G. Dooms}, \hyperref[auth:a149]{Pascal Van Hentenryck} & Gap Reduction Techniques for Online Stochastic Project Scheduling & \href{../works/DoomsH08.pdf}{Yes} & \cite{DoomsH08} & 2008 & CPAIOR 2008 & 16 & 1 & 2 & \ref{b:DoomsH08} & n/a\\
\rowlabel{a:HentenryckM08}HentenryckM08 \href{https://doi.org/10.1007/978-3-540-68155-7_41}{HentenryckM08} & \hyperref[auth:a149]{Pascal Van Hentenryck}, \hyperref[auth:a32]{L. Michel} & The Steel Mill Slab Design Problem Revisited & \href{../works/HentenryckM08.pdf}{Yes} & \cite{HentenryckM08} & 2008 & CPAIOR 2008 & 5 & 13 & 3 & \ref{b:HentenryckM08} & n/a\\
\rowlabel{a:Hunsberger08}Hunsberger08 \href{https://doi.org/10.3233/978-1-58603-891-5-553}{Hunsberger08} & \hyperref[auth:a1293]{L. Hunsberger} & A Practical Temporal Constraint Management System for Real-Time Applications & \href{../works/Hunsberger08.pdf}{Yes} & \cite{Hunsberger08} & 2008 & ECAI 2008 & 5 & 0 & 0 & \ref{b:Hunsberger08} & n/a\\
\rowlabel{a:LauLN08}LauLN08 \href{https://doi.org/10.1007/978-3-540-68155-7_33}{LauLN08} & \hyperref[auth:a367]{Hoong Chuin Lau}, \hyperref[auth:a368]{Kong Wei Lye}, \hyperref[auth:a369]{Viet Bang Nguyen} & A Combinatorial Auction Framework for Solving Decentralized Scheduling Problems (Extended Abstract) & \href{../works/LauLN08.pdf}{Yes} & \cite{LauLN08} & 2008 & CPAIOR 2008 & 5 & 0 & 4 & \ref{b:LauLN08} & n/a\\
\rowlabel{a:MouraSCL08}MouraSCL08 \href{https://doi.org/10.1007/978-3-540-85958-1_3}{MouraSCL08} & \hyperref[auth:a160]{Arnaldo Vieira Moura}, \hyperref[auth:a171]{Cid C. de Souza}, \hyperref[auth:a158]{Andr{\'{e}} A. Cir{\'{e}}}, \hyperref[auth:a157]{Tony Minoru Tamura Lopes} & Planning and Scheduling the Operation of a Very Large Oil Pipeline Network & \href{../works/MouraSCL08.pdf}{Yes} & \cite{MouraSCL08} & 2008 & CP 2008 & 16 & 11 & 10 & \ref{b:MouraSCL08} & n/a\\
\rowlabel{a:MouraSCL08a}MouraSCL08a \href{https://doi.org/10.1109/CSE.2008.24}{MouraSCL08a} & \hyperref[auth:a160]{Arnaldo Vieira Moura}, \hyperref[auth:a171]{Cid C. de Souza}, \hyperref[auth:a158]{Andr{\'{e}} A. Cir{\'{e}}}, \hyperref[auth:a157]{Tony Minoru Tamura Lopes} & Heuristics and Constraint Programming Hybridizations for a Real Pipeline Planning and Scheduling Problem & \href{../works/MouraSCL08a.pdf}{Yes} & \cite{MouraSCL08a} & 2008 & CSE 2008 & 8 & 5 & 14 & \ref{b:MouraSCL08a} & n/a\\
\rowlabel{a:PoderB08}PoderB08 \href{http://www.aaai.org/Library/ICAPS/2008/icaps08-033.php}{PoderB08} & \hyperref[auth:a361]{E. Poder}, \hyperref[auth:a129]{N. Beldiceanu} & Filtering for a Continuous Multi-Resources cumulative Constraint with Resource Consumption and Production & \href{../works/PoderB08.pdf}{Yes} & \cite{PoderB08} & 2008 & ICAPS 2008 & 8 & 0 & 0 & \ref{b:PoderB08} & n/a\\
\rowlabel{a:SchausD08}SchausD08 \href{http://www.aaai.org/Library/AAAI/2008/aaai08-058.php}{SchausD08} & \hyperref[auth:a148]{P. Schaus}, \hyperref[auth:a152]{Y. Deville} & A Global Constraint for Bin-Packing with Precedences: Application to the Assembly Line Balancing Problem & \href{../works/SchausD08.pdf}{Yes} & \cite{SchausD08} & 2008 & AAAI 2008 & 6 & 0 & 0 & \ref{b:SchausD08} & n/a\\
\rowlabel{a:WatsonB08}WatsonB08 \href{https://doi.org/10.1007/978-3-540-68155-7_21}{WatsonB08} & \hyperref[auth:a363]{J. Watson}, \hyperref[auth:a89]{J. Christopher Beck} & A Hybrid Constraint Programming / Local Search Approach to the Job-Shop Scheduling Problem & \href{../works/WatsonB08.pdf}{Yes} & \cite{WatsonB08} & 2008 & CPAIOR 2008 & 15 & 14 & 17 & \ref{b:WatsonB08} & n/a\\
\rowlabel{a:AkkerDH07}AkkerDH07 \href{https://doi.org/10.1007/978-3-540-72397-4_27}{AkkerDH07} & \hyperref[auth:a375]{J. M. van den Akker}, \hyperref[auth:a376]{G. Diepen}, \hyperref[auth:a377]{J. A. Hoogeveen} & A Column Generation Based Destructive Lower Bound for Resource Constrained Project Scheduling Problems & \href{../works/AkkerDH07.pdf}{Yes} & \cite{AkkerDH07} & 2007 & CPAIOR 2007 & 15 & 2 & 8 & \ref{b:AkkerDH07} & n/a\\
\rowlabel{a:BeldiceanuP07}BeldiceanuP07 \href{https://doi.org/10.1007/978-3-540-72397-4_16}{BeldiceanuP07} & \hyperref[auth:a129]{N. Beldiceanu}, \hyperref[auth:a361]{E. Poder} & A Continuous Multi-resources \emph{cumulative} Constraint with Positive-Negative Resource Consumption-Production & \href{../works/BeldiceanuP07.pdf}{Yes} & \cite{BeldiceanuP07} & 2007 & CPAIOR 2007 & 15 & 4 & 7 & \ref{b:BeldiceanuP07} & n/a\\
\rowlabel{a:DavenportKRSH07}DavenportKRSH07 \href{https://doi.org/10.1007/978-3-540-74970-7_7}{DavenportKRSH07} & \hyperref[auth:a250]{Andrew J. Davenport}, \hyperref[auth:a251]{J. Kalagnanam}, \hyperref[auth:a252]{C. Reddy}, \hyperref[auth:a253]{S. Siegel}, \hyperref[auth:a254]{J. Hou} & An Application of Constraint Programming to Generating Detailed Operations Schedules for Steel Manufacturing & \href{../works/DavenportKRSH07.pdf}{Yes} & \cite{DavenportKRSH07} & 2007 & CP 2007 & 13 & 1 & 2 & \ref{b:DavenportKRSH07} & n/a\\
\rowlabel{a:ElhouraniDM07}ElhouraniDM07 \href{http://www.aaai.org/Library/AAAI/2007/aaai07-213.php}{ElhouraniDM07} & \hyperref[auth:a1367]{T. Elhourani}, \hyperref[auth:a1368]{N. Denny}, \hyperref[auth:a1369]{Michael M. Marefat} & A Distributed Constraint Optimization Solution to the {P2P} Video Streaming Problem & \href{../works/ElhouraniDM07.pdf}{Yes} & \cite{ElhouraniDM07} & 2007 & AAAI 2007 & 6 & 0 & 0 & \ref{b:ElhouraniDM07} & n/a\\
\rowlabel{a:GarganiR07}GarganiR07 \href{https://doi.org/10.1007/978-3-540-74970-7_8}{GarganiR07} & \hyperref[auth:a255]{A. Gargani}, \hyperref[auth:a256]{P. Refalo} & An Efficient Model and Strategy for the Steel Mill Slab Design Problem & \href{../works/GarganiR07.pdf}{Yes} & \cite{GarganiR07} & 2007 & CP 2007 & 13 & 17 & 5 & \ref{b:GarganiR07} & n/a\\
\rowlabel{a:HoeveGSL07}HoeveGSL07 \href{http://www.aaai.org/Library/AAAI/2007/aaai07-291.php}{HoeveGSL07} & \hyperref[auth:a211]{Willem{-}Jan van Hoeve}, \hyperref[auth:a648]{Carla P. Gomes}, \hyperref[auth:a649]{B. Selman}, \hyperref[auth:a143]{M. Lombardi} & Optimal Multi-Agent Scheduling with Constraint Programming & \href{../works/HoeveGSL07.pdf}{Yes} & \cite{HoeveGSL07} & 2007 & AAAI 2007 & 6 & 0 & 0 & \ref{b:HoeveGSL07} & n/a\\
\rowlabel{a:KeriK07}KeriK07 \href{https://doi.org/10.1007/978-3-540-72397-4_10}{KeriK07} & \hyperref[auth:a370]{A. K{\'{e}}ri}, \hyperref[auth:a156]{T. Kis} & Computing Tight Time Windows for {RCPSPWET} with the Primal-Dual Method & \href{../works/KeriK07.pdf}{Yes} & \cite{KeriK07} & 2007 & CPAIOR 2007 & 14 & 1 & 13 & \ref{b:KeriK07} & n/a\\
\rowlabel{a:KovacsB07}KovacsB07 \href{https://doi.org/10.1007/978-3-540-72397-4_9}{KovacsB07} & \hyperref[auth:a147]{A. Kov{\'{a}}cs}, \hyperref[auth:a89]{J. Christopher Beck} & A Global Constraint for Total Weighted Completion Time & \href{../works/KovacsB07.pdf}{Yes} & \cite{KovacsB07} & 2007 & CPAIOR 2007 & 15 & 2 & 12 & \ref{b:KovacsB07} & n/a\\
\rowlabel{a:KrogtLPHJ07}KrogtLPHJ07 \href{https://doi.org/10.1007/978-3-540-74970-7_10}{KrogtLPHJ07} & \hyperref[auth:a257]{Roman van der Krogt}, \hyperref[auth:a179]{J. Little}, \hyperref[auth:a258]{K. Pulliam}, \hyperref[auth:a259]{S. Hanhilammi}, \hyperref[auth:a260]{Y. Jin} & Scheduling for Cellular Manufacturing & \href{../works/KrogtLPHJ07.pdf}{Yes} & \cite{KrogtLPHJ07} & 2007 & CP 2007 & 13 & 2 & 3 & \ref{b:KrogtLPHJ07} & n/a\\
\rowlabel{a:Limtanyakul07}Limtanyakul07 \href{https://doi.org/10.1007/978-3-540-77903-2_65}{Limtanyakul07} & \hyperref[auth:a145]{K. Limtanyakul} & Scheduling of Tests on Vehicle Prototypes Using Constraint and Integer Programming & \href{../works/Limtanyakul07.pdf}{Yes} & \cite{Limtanyakul07} & 2007 & GOR 2007 & 6 & 2 & 3 & \ref{b:Limtanyakul07} & n/a\\
\rowlabel{a:MonetteDD07}MonetteDD07 \href{https://doi.org/10.1007/978-3-540-72397-4_14}{MonetteDD07} & \hyperref[auth:a150]{J. Monette}, \hyperref[auth:a152]{Y. Deville}, \hyperref[auth:a371]{P. Dupont} & A Position-Based Propagator for the Open-Shop Problem & \href{../works/MonetteDD07.pdf}{Yes} & \cite{MonetteDD07} & 2007 & CPAIOR 2007 & 14 & 0 & 12 & \ref{b:MonetteDD07} & n/a\\
\rowlabel{a:Rodriguez07b}Rodriguez07b \href{}{Rodriguez07b} & \hyperref[auth:a787]{J. Rodriguez} & A study of the use of state resources in a constraint-based model for routing and scheduling trains & \href{../works/Rodriguez07b.pdf}{Yes} & \cite{Rodriguez07b} & 2007 & ICROMA 2007 & 14 & 0 & 0 & \ref{b:Rodriguez07b} & n/a\\
\rowlabel{a:RossiTHP07}RossiTHP07 \href{https://doi.org/10.1007/978-3-540-72397-4_17}{RossiTHP07} & \hyperref[auth:a372]{R. Rossi}, \hyperref[auth:a373]{A. Tarim}, \hyperref[auth:a138]{B. Hnich}, \hyperref[auth:a374]{Steven D. Prestwich} & Replenishment Planning for Stochastic Inventory Systems with Shortage Cost & \href{../works/RossiTHP07.pdf}{Yes} & \cite{RossiTHP07} & 2007 & CPAIOR 2007 & 15 & 6 & 10 & \ref{b:RossiTHP07} & n/a\\
\rowlabel{a:Beck06}Beck06 \href{http://www.aaai.org/Library/ICAPS/2006/icaps06-028.php}{Beck06} & \hyperref[auth:a89]{J. Christopher Beck} & An Empirical Study of Multi-Point Constructive Search for Constraint-Based Scheduling & \href{../works/Beck06.pdf}{Yes} & \cite{Beck06} & 2006 & ICAPS 2006 & 10 & 0 & 0 & \ref{b:Beck06} & n/a\\
\rowlabel{a:BeniniBGM06}BeniniBGM06 \href{https://doi.org/10.1007/11757375_6}{BeniniBGM06} & \hyperref[auth:a247]{L. Benini}, \hyperref[auth:a378]{D. Bertozzi}, \hyperref[auth:a379]{A. Guerri}, \hyperref[auth:a144]{M. Milano} & Allocation, Scheduling and Voltage Scaling on Energy Aware MPSoCs & \href{../works/BeniniBGM06.pdf}{Yes} & \cite{BeniniBGM06} & 2006 & CPAIOR 2006 & 15 & 18 & 10 & \ref{b:BeniniBGM06} & n/a\\
\rowlabel{a:GomesHS06}GomesHS06 \href{http://www.aaai.org/Library/Symposia/Spring/2006/ss06-04-024.php}{GomesHS06} & \hyperref[auth:a648]{Carla P. Gomes}, \hyperref[auth:a211]{Willem{-}Jan van Hoeve}, \hyperref[auth:a649]{B. Selman} & Constraint Programming for Distributed Planning and Scheduling & \href{../works/GomesHS06.pdf}{Yes} & \cite{GomesHS06} & 2006 & AAAI 2006 & 2 & 0 & 0 & \ref{b:GomesHS06} & n/a\\
\rowlabel{a:KhemmoudjPB06}KhemmoudjPB06 \href{https://doi.org/10.1007/11889205_21}{KhemmoudjPB06} & \hyperref[auth:a261]{Mohand Ou Idir Khemmoudj}, \hyperref[auth:a262]{M. Porcheron}, \hyperref[auth:a263]{H. Bennaceur} & When Constraint Programming and Local Search Solve the Scheduling Problem of Electricit{\'{e}} de France Nuclear Power Plant Outages & \href{../works/KhemmoudjPB06.pdf}{Yes} & \cite{KhemmoudjPB06} & 2006 & CP 2006 & 13 & 8 & 8 & \ref{b:KhemmoudjPB06} & n/a\\
\rowlabel{a:KovacsV06}KovacsV06 \href{https://doi.org/10.1007/11757375_13}{KovacsV06} & \hyperref[auth:a147]{A. Kov{\'{a}}cs}, \hyperref[auth:a280]{J. V{\'{a}}ncza} & Progressive Solutions: {A} Simple but Efficient Dominance Rule for Practical {RCPSP} & \href{../works/KovacsV06.pdf}{Yes} & \cite{KovacsV06} & 2006 & CPAIOR 2006 & 13 & 2 & 7 & \ref{b:KovacsV06} & n/a\\
\rowlabel{a:LiuJ06}LiuJ06 \href{https://doi.org/10.1007/11801603_92}{LiuJ06} & \hyperref[auth:a660]{Y. Liu}, \hyperref[auth:a661]{Y. Jiang} & {LP-TPOP:} Integrating Planning and Scheduling Through Constraint Programming & \href{../works/LiuJ06.pdf}{Yes} & \cite{LiuJ06} & 2006 & PRICAI 2006 & 5 & 0 & 0 & \ref{b:LiuJ06} & n/a\\
\rowlabel{a:QuSN06}QuSN06 \href{https://doi.org/10.1109/ISSOC.2006.321973}{QuSN06} & \hyperref[auth:a657]{Y. Qu}, \hyperref[auth:a658]{J. Soininen}, \hyperref[auth:a659]{J. Nurmi} & Using Constraint Programming to Achieve Optimal Prefetch Scheduling for Dependent Tasks on Run-Time Reconfigurable Devices & \href{../works/QuSN06.pdf}{Yes} & \cite{QuSN06} & 2006 & SoC 2006 & 4 & 2 & 5 & \ref{b:QuSN06} & n/a\\
\rowlabel{a:Wallace06}Wallace06 \href{http://dx.doi.org/10.1007/978-3-540-73817-6_1}{Wallace06} & \hyperref[auth:a117]{Mark G. Wallace} & Hybrid Algorithms in Constraint Programming & \href{../works/Wallace06.pdf}{Yes} & \cite{Wallace06} & 2006 & CSCLP 2006 & 32 & 1 & 35 & \ref{b:Wallace06} & n/a\\
\rowlabel{a:AbrilSB05}AbrilSB05 \href{https://doi.org/10.1007/11564751_75}{AbrilSB05} & \hyperref[auth:a272]{M. Abril}, \hyperref[auth:a154]{Miguel A. Salido}, \hyperref[auth:a273]{F. Barber} & Distributed Constraints for Large-Scale Scheduling Problems & \href{../works/AbrilSB05.pdf}{Yes} & \cite{AbrilSB05} & 2005 & CP 2005 & 1 & 0 & 0 & \ref{b:AbrilSB05} & n/a\\
\rowlabel{a:ArtiouchineB05}ArtiouchineB05 \href{https://doi.org/10.1007/11564751_8}{ArtiouchineB05} & \hyperref[auth:a264]{K. Artiouchine}, \hyperref[auth:a163]{P. Baptiste} & Inter-distance Constraint: An Extension of the All-Different Constraint for Scheduling Equal Length Jobs & \href{../works/ArtiouchineB05.pdf}{Yes} & \cite{ArtiouchineB05} & 2005 & CP 2005 & 15 & 3 & 11 & \ref{b:ArtiouchineB05} & n/a\\
\rowlabel{a:BeckW05}BeckW05 \href{http://ijcai.org/Proceedings/05/Papers/0748.pdf}{BeckW05} & \hyperref[auth:a89]{J. Christopher Beck}, \hyperref[auth:a832]{N. Wilson} & Proactive Algorithms for Scheduling with Probabilistic Durations & \href{../works/BeckW05.pdf}{Yes} & \cite{BeckW05} & 2005 & IJCAI 2005 & 6 & 0 & 0 & \ref{b:BeckW05} & n/a\\
\rowlabel{a:BeniniBGM05}BeniniBGM05 \href{https://doi.org/10.1007/11564751_11}{BeniniBGM05} & \hyperref[auth:a247]{L. Benini}, \hyperref[auth:a378]{D. Bertozzi}, \hyperref[auth:a379]{A. Guerri}, \hyperref[auth:a144]{M. Milano} & Allocation and Scheduling for MPSoCs via Decomposition and No-Good Generation & \href{../works/BeniniBGM05.pdf}{Yes} & \cite{BeniniBGM05} & 2005 & CP 2005 & 15 & 25 & 21 & \ref{b:BeniniBGM05} & n/a\\
\rowlabel{a:CambazardJ05}CambazardJ05 \href{https://doi.org/10.1007/11564751_58}{CambazardJ05} & \hyperref[auth:a1011]{H. Cambazard}, \hyperref[auth:a249]{N. Jussien} & Integrating Benders Decomposition Within Constraint Programming & \href{../works/CambazardJ05.pdf}{Yes} & \cite{CambazardJ05} & 2005 & CP 2005 & 5 & 6 & 8 & \ref{b:CambazardJ05} & n/a\\
\rowlabel{a:CarchraeBF05}CarchraeBF05 \href{https://doi.org/10.1007/11564751_80}{CarchraeBF05} & \hyperref[auth:a274]{T. Carchrae}, \hyperref[auth:a89]{J. Christopher Beck}, \hyperref[auth:a275]{Eugene C. Freuder} & Methods to Learn Abstract Scheduling Models & \href{../works/CarchraeBF05.pdf}{Yes} & \cite{CarchraeBF05} & 2005 & CP 2005 & 1 & 0 & 0 & \ref{b:CarchraeBF05} & n/a\\
\rowlabel{a:ChuX05}ChuX05 \href{https://doi.org/10.1007/11493853_10}{ChuX05} & \hyperref[auth:a380]{Y. Chu}, \hyperref[auth:a381]{Q. Xia} & A Hybrid Algorithm for a Class of Resource Constrained Scheduling Problems & \href{../works/ChuX05.pdf}{Yes} & \cite{ChuX05} & 2005 & CPAIOR 2005 & 15 & 13 & 13 & \ref{b:ChuX05} & n/a\\
\rowlabel{a:DilkinaDH05}DilkinaDH05 \href{https://doi.org/10.1007/11564751_60}{DilkinaDH05} & \hyperref[auth:a269]{B. Dilkina}, \hyperref[auth:a270]{L. Duan}, \hyperref[auth:a271]{William S. Havens} & Extending Systematic Local Search for Job Shop Scheduling Problems & \href{../works/DilkinaDH05.pdf}{Yes} & \cite{DilkinaDH05} & 2005 & CP 2005 & 5 & 2 & 7 & \ref{b:DilkinaDH05} & n/a\\
\rowlabel{a:FortinZDF05}FortinZDF05 \href{https://doi.org/10.1007/11564751_19}{FortinZDF05} & \hyperref[auth:a265]{J. Fortin}, \hyperref[auth:a266]{P. Zielinski}, \hyperref[auth:a267]{D. Dubois}, \hyperref[auth:a268]{H. Fargier} & Interval Analysis in Scheduling & \href{../works/FortinZDF05.pdf}{Yes} & \cite{FortinZDF05} & 2005 & CP 2005 & 15 & 13 & 11 & \ref{b:FortinZDF05} & n/a\\
\rowlabel{a:FrankK05}FrankK05 \href{https://doi.org/10.1007/11493853_15}{FrankK05} & \hyperref[auth:a382]{J. Frank}, \hyperref[auth:a383]{E. K{\"{u}}rkl{\"{u}}} & Mixed Discrete and Continuous Algorithms for Scheduling Airborne Astronomy Observations & \href{../works/FrankK05.pdf}{Yes} & \cite{FrankK05} & 2005 & CPAIOR 2005 & 18 & 4 & 4 & \ref{b:FrankK05} & n/a\\
\rowlabel{a:Geske05}Geske05 \href{https://doi.org/10.1007/11963578_10}{Geske05} & \hyperref[auth:a663]{U. Geske} & Railway Scheduling with Declarative Constraint Programming & \href{../works/Geske05.pdf}{Yes} & \cite{Geske05} & 2005 & INAP 2005 & 18 & 2 & 3 & \ref{b:Geske05} & n/a\\
\rowlabel{a:GodardLN05}GodardLN05 \href{http://www.aaai.org/Library/ICAPS/2005/icaps05-009.php}{GodardLN05} & \hyperref[auth:a780]{D. Godard}, \hyperref[auth:a118]{P. Laborie}, \hyperref[auth:a662]{W. Nuijten} & Randomized Large Neighborhood Search for Cumulative Scheduling & \href{../works/GodardLN05.pdf}{Yes} & \cite{GodardLN05} & 2005 & ICAPS 2005 & 9 & 0 & 0 & \ref{b:GodardLN05} & n/a\\
\rowlabel{a:HebrardTW05}HebrardTW05 \href{https://doi.org/10.1007/11564751_117}{HebrardTW05} & \hyperref[auth:a1]{E. Hebrard}, \hyperref[auth:a277]{P. Tyler}, \hyperref[auth:a278]{T. Walsh} & Computing Super-Schedules & \href{../works/HebrardTW05.pdf}{Yes} & \cite{HebrardTW05} & 2005 & CP 2005 & 1 & 0 & 3 & \ref{b:HebrardTW05} & n/a\\
\rowlabel{a:Hooker05a}Hooker05a \href{https://doi.org/10.1007/11564751_25}{Hooker05a} & \hyperref[auth:a161]{John N. Hooker} & Planning and Scheduling to Minimize Tardiness & \href{../works/Hooker05a.pdf}{Yes} & \cite{Hooker05a} & 2005 & CP 2005 & 14 & 30 & 10 & \ref{b:Hooker05a} & n/a\\
\rowlabel{a:Hooker05b}Hooker05b \href{https://doi.org/10.1007/11493853_19}{Hooker05b} & \hyperref[auth:a161]{John N. Hooker} & A Search-Infer-and-Relax Framework for Integrating Solution Methods & \href{../works/Hooker05b.pdf}{Yes} & \cite{Hooker05b} & 2005 & CPAIOR 2005 & 15 & 7 & 19 & \ref{b:Hooker05b} & n/a\\
\rowlabel{a:Johnston05}Johnston05 \href{}{Johnston05} & \hyperref[auth:a1366]{Bradley J. Clement}, \hyperref[auth:a1231]{Mark D. Johnston} & The Deep Space Network Scheduling Problem & \href{../works/Johnston05.pdf}{Yes} & \cite{Johnston05} & 2005 & AAAI 2005 & 7 & 0 & 0 & \ref{b:Johnston05} & n/a\\
\rowlabel{a:KovacsEKV05}KovacsEKV05 \href{https://doi.org/10.1007/11564751_118}{KovacsEKV05} & \hyperref[auth:a147]{A. Kov{\'{a}}cs}, \hyperref[auth:a279]{P. Egri}, \hyperref[auth:a156]{T. Kis}, \hyperref[auth:a280]{J. V{\'{a}}ncza} & Proterv-II: An Integrated Production Planning and Scheduling System & \href{../works/KovacsEKV05.pdf}{Yes} & \cite{KovacsEKV05} & 2005 & CP 2005 & 1 & 2 & 3 & \ref{b:KovacsEKV05} & n/a\\
\rowlabel{a:MoffittPP05}MoffittPP05 \href{http://www.aaai.org/Library/AAAI/2005/aaai05-188.php}{MoffittPP05} & \hyperref[auth:a777]{Michael D. Moffitt}, \hyperref[auth:a778]{B. Peintner}, \hyperref[auth:a779]{Martha E. Pollack} & Augmenting Disjunctive Temporal Problems with Finite-Domain Constraints & \href{../works/MoffittPP05.pdf}{Yes} & \cite{MoffittPP05} & 2005 & AAAI 2005 & 6 & 0 & 0 & \ref{b:MoffittPP05} & n/a\\
\rowlabel{a:OddiPCC05}OddiPCC05 \href{http://dx.doi.org/10.1007/0-387-27744-7_7}{OddiPCC05} & \hyperref[auth:a284]{A. Oddi}, \hyperref[auth:a285]{N. Policella}, \hyperref[auth:a286]{A. Cesta}, \hyperref[auth:a287]{G. Cortellessa} & Constraint-Based Random Search for Solving Spacecraft Downlink Scheduling Problems & No & \cite{OddiPCC05} & 2005 & MISTA 2005 & null & 3 & 12 & No & n/a\\
\rowlabel{a:PolicellaWSO05}PolicellaWSO05 \href{http://www.aaai.org/Library/AAAI/2005/aaai05-190.php}{PolicellaWSO05} & \hyperref[auth:a285]{N. Policella}, \hyperref[auth:a1365]{X. Wang}, \hyperref[auth:a300]{Stephen F. Smith}, \hyperref[auth:a284]{A. Oddi} & Exploiting Temporal Flexibility to Obtain High Quality Schedules & \href{../works/PolicellaWSO05.pdf}{Yes} & \cite{PolicellaWSO05} & 2005 & AAAI 2005 & 6 & 0 & 0 & \ref{b:PolicellaWSO05} & n/a\\
\rowlabel{a:QuirogaZH05}QuirogaZH05 \href{https://doi.org/10.1109/ROBOT.2005.1570686}{QuirogaZH05} & \hyperref[auth:a628]{O. Quiroga}, \hyperref[auth:a627]{L. Zeballos}, \hyperref[auth:a594]{Gabriela P. Henning} & A Constraint Programming Approach to Tool Allocation and Resource Scheduling in {FMS} & \href{../works/QuirogaZH05.pdf}{Yes} & \cite{QuirogaZH05} & 2005 & ICRA 2005 & 6 & 2 & 7 & \ref{b:QuirogaZH05} & n/a\\
\rowlabel{a:SchuttWS05}SchuttWS05 \href{https://doi.org/10.1007/11963578_6}{SchuttWS05} & \hyperref[auth:a125]{A. Schutt}, \hyperref[auth:a51]{A. Wolf}, \hyperref[auth:a716]{G. Schrader} & Not-First and Not-Last Detection for Cumulative Scheduling in \emph{O}(\emph{n}\({}^{\mbox{3}}\)log\emph{n}) & \href{../works/SchuttWS05.pdf}{Yes} & \cite{SchuttWS05} & 2005 & INAP 2005 & 15 & 6 & 4 & \ref{b:SchuttWS05} & n/a\\
\rowlabel{a:Vilim05}Vilim05 \href{https://doi.org/10.1007/11493853_29}{Vilim05} & \hyperref[auth:a121]{P. Vil{\'{\i}}m} & Computing Explanations for the Unary Resource Constraint & \href{../works/Vilim05.pdf}{Yes} & \cite{Vilim05} & 2005 & CPAIOR 2005 & 14 & 5 & 8 & \ref{b:Vilim05} & n/a\\
\rowlabel{a:Wolf05}Wolf05 \href{http://dx.doi.org/10.1007/11402763_15}{Wolf05} & \hyperref[auth:a51]{A. Wolf} & Better Propagation for Non-preemptive Single-Resource Constraint Problems & \href{../works/Wolf05.pdf}{Yes} & \cite{Wolf05} & 2005 & CSCLP 2005 & 15 & 4 & 8 & \ref{b:Wolf05} & n/a\\
\rowlabel{a:WolfS05}WolfS05 \href{https://doi.org/10.1007/11963578_8}{WolfS05} & \hyperref[auth:a51]{A. Wolf}, \hyperref[auth:a716]{G. Schrader} & \emph{O}(\emph{n} log\emph{n}) Overload Checking for the Cumulative Constraint and Its Application & \href{../works/WolfS05.pdf}{Yes} & \cite{WolfS05} & 2005 & INAP 2005 & 14 & 6 & 6 & \ref{b:WolfS05} & n/a\\
\rowlabel{a:WuBB05}WuBB05 \href{https://doi.org/10.1007/11564751_110}{WuBB05} & \hyperref[auth:a276]{Christine Wei Wu}, \hyperref[auth:a222]{Kenneth N. Brown}, \hyperref[auth:a89]{J. Christopher Beck} & Scheduling with Uncertain Start Dates & \href{../works/WuBB05.pdf}{Yes} & \cite{WuBB05} & 2005 & CP 2005 & 1 & 0 & 0 & \ref{b:WuBB05} & n/a\\
\rowlabel{a:ArtiguesBF04}ArtiguesBF04 \href{https://doi.org/10.1007/978-3-540-24664-0_3}{ArtiguesBF04} & \hyperref[auth:a6]{C. Artigues}, \hyperref[auth:a386]{S. Belmokhtar}, \hyperref[auth:a359]{D. Feillet} & A New Exact Solution Algorithm for the Job Shop Problem with Sequence-Dependent Setup Times & \href{../works/ArtiguesBF04.pdf}{Yes} & \cite{ArtiguesBF04} & 2004 & CPAIOR 2004 & 13 & 16 & 9 & \ref{b:ArtiguesBF04} & n/a\\
\rowlabel{a:BarbulescuWH04}BarbulescuWH04 \href{http://www.aaai.org/Library/AAAI/2004/aaai04-023.php}{BarbulescuWH04} & \hyperref[auth:a1338]{L. Barbulescu}, \hyperref[auth:a1340]{L. Darrell Whitley}, \hyperref[auth:a1339]{Adele E. Howe} & Leap Before You Look: An Effective Strategy in an Oversubscribed Scheduling Problem & \href{../works/BarbulescuWH04.pdf}{Yes} & \cite{BarbulescuWH04} & 2004 & AAAI 2004 & 6 & 0 & 0 & \ref{b:BarbulescuWH04} & n/a\\
\rowlabel{a:BeckW04}BeckW04 \href{}{BeckW04} & \hyperref[auth:a89]{J. Christopher Beck}, \hyperref[auth:a832]{N. Wilson} & Job Shop Scheduling with Probabilistic Durations & \href{../works/BeckW04.pdf}{Yes} & \cite{BeckW04} & 2004 & ECAI 2004 & 5 & 0 & 0 & \ref{b:BeckW04} & n/a\\
\rowlabel{a:CambazardHDJT04}CambazardHDJT04 \href{https://doi.org/10.1007/978-3-540-30201-8_14}{CambazardHDJT04} & \hyperref[auth:a1011]{H. Cambazard}, \hyperref[auth:a1075]{P. Hladik}, \hyperref[auth:a1076]{A. D{\'{e}}planche}, \hyperref[auth:a249]{N. Jussien}, \hyperref[auth:a1077]{Y. Trinquet} & Decomposition and Learning for a Hard Real Time Task Allocation Problem & \href{../works/CambazardHDJT04.pdf}{Yes} & \cite{CambazardHDJT04} & 2004 & CP 2004 & 15 & 33 & 13 & \ref{b:CambazardHDJT04} & n/a\\
\rowlabel{a:DilkinaH04}DilkinaH04 \href{}{DilkinaH04} & \hyperref[auth:a1360]{Bistra N. Dilkina}, \hyperref[auth:a271]{William S. Havens} & The {U.S.} National Football League Scheduling Problem & \href{../works/DilkinaH04.pdf}{Yes} & \cite{DilkinaH04} & 2004 & AAAI 2004 & 6 & 0 & 0 & \ref{b:DilkinaH04} & n/a\\
\rowlabel{a:GlobusCLP04}GlobusCLP04 \href{}{GlobusCLP04} & \hyperref[auth:a1361]{A. Globus}, \hyperref[auth:a1362]{J. Crawford}, \hyperref[auth:a1363]{Jason D. Lohn}, \hyperref[auth:a1364]{A. Pryor} & A Comparison of Techniques for Scheduling Earth Observing Satellites & \href{../works/GlobusCLP04.pdf}{Yes} & \cite{GlobusCLP04} & 2004 & AAAI 2004 & 8 & 0 & 0 & \ref{b:GlobusCLP04} & n/a\\
\rowlabel{a:HentenryckM04}HentenryckM04 \href{https://doi.org/10.1007/978-3-540-24664-0_22}{HentenryckM04} & \hyperref[auth:a149]{Pascal Van Hentenryck}, \hyperref[auth:a32]{L. Michel} & Scheduling Abstractions for Local Search & \href{../works/HentenryckM04.pdf}{Yes} & \cite{HentenryckM04} & 2004 & CPAIOR 2004 & 16 & 12 & 14 & \ref{b:HentenryckM04} & n/a\\
\rowlabel{a:Hooker04}Hooker04 \href{https://doi.org/10.1007/978-3-540-30201-8_24}{Hooker04} & \hyperref[auth:a161]{John N. Hooker} & A Hybrid Method for Planning and Scheduling & \href{../works/Hooker04.pdf}{Yes} & \cite{Hooker04} & 2004 & CP 2004 & 12 & 39 & 9 & \ref{b:Hooker04} & n/a\\
\rowlabel{a:KovacsV04}KovacsV04 \href{https://doi.org/10.1007/978-3-540-30201-8_26}{KovacsV04} & \hyperref[auth:a147]{A. Kov{\'{a}}cs}, \hyperref[auth:a280]{J. V{\'{a}}ncza} & Completable Partial Solutions in Constraint Programming and Constraint-Based Scheduling & \href{../works/KovacsV04.pdf}{Yes} & \cite{KovacsV04} & 2004 & CP 2004 & 15 & 3 & 12 & \ref{b:KovacsV04} & n/a\\
\rowlabel{a:LimRX04}LimRX04 \href{https://doi.org/10.1007/978-3-540-30201-8_59}{LimRX04} & \hyperref[auth:a281]{A. Lim}, \hyperref[auth:a282]{B. Rodrigues}, \hyperref[auth:a283]{Z. Xu} & Solving the Crane Scheduling Problem Using Intelligent Search Schemes & \href{../works/LimRX04.pdf}{Yes} & \cite{LimRX04} & 2004 & CP 2004 & 5 & 5 & 6 & \ref{b:LimRX04} & n/a\\
\rowlabel{a:MaraveliasG04}MaraveliasG04 \href{https://doi.org/10.1007/978-3-540-24664-0_1}{MaraveliasG04} & \hyperref[auth:a384]{Christos T. Maravelias}, \hyperref[auth:a385]{Ignacio E. Grossmann} & Using {MILP} and {CP} for the Scheduling of Batch Chemical Processes & \href{../works/MaraveliasG04.pdf}{Yes} & \cite{MaraveliasG04} & 2004 & CPAIOR 2004 & 20 & 15 & 15 & \ref{b:MaraveliasG04} & n/a\\
\rowlabel{a:PerronSF04}PerronSF04 \href{https://doi.org/10.1007/978-3-540-30201-8_35}{PerronSF04} & \hyperref[auth:a290]{L. Perron}, \hyperref[auth:a120]{P. Shaw}, \hyperref[auth:a1087]{V. Furnon} & Propagation Guided Large Neighborhood Search & \href{../works/PerronSF04.pdf}{Yes} & \cite{PerronSF04} & 2004 & CP 2004 & 14 & 34 & 8 & \ref{b:PerronSF04} & n/a\\
\rowlabel{a:Sadykov04}Sadykov04 \href{https://doi.org/10.1007/978-3-540-24664-0_31}{Sadykov04} & \hyperref[auth:a387]{R. Sadykov} & A Hybrid Branch-And-Cut Algorithm for the One-Machine Scheduling Problem & \href{../works/Sadykov04.pdf}{Yes} & \cite{Sadykov04} & 2004 & CPAIOR 2004 & 7 & 11 & 7 & \ref{b:Sadykov04} & n/a\\
\rowlabel{a:Vilim04}Vilim04 \href{https://doi.org/10.1007/978-3-540-24664-0_23}{Vilim04} & \hyperref[auth:a121]{P. Vil{\'{\i}}m} & O(n log n) Filtering Algorithms for Unary Resource Constraint & \href{../works/Vilim04.pdf}{Yes} & \cite{Vilim04} & 2004 & CPAIOR 2004 & 13 & 22 & 5 & \ref{b:Vilim04} & n/a\\
\rowlabel{a:VilimBC04}VilimBC04 \href{https://doi.org/10.1007/978-3-540-30201-8_8}{VilimBC04} & \hyperref[auth:a121]{P. Vil{\'{\i}}m}, \hyperref[auth:a153]{R. Bart{\'{a}}k}, \hyperref[auth:a162]{O. Cepek} & Unary Resource Constraint with Optional Activities & \href{../works/VilimBC04.pdf}{Yes} & \cite{VilimBC04} & 2004 & CP 2004 & 15 & 13 & 4 & \ref{b:VilimBC04} & n/a\\
\rowlabel{a:VillaverdeP04}VillaverdeP04 \href{}{VillaverdeP04} & \hyperref[auth:a664]{K. Villaverde}, \hyperref[auth:a33]{E. Pontelli} & An Investigation of Scheduling in Distributed Constraint Logic Programming & No & \cite{VillaverdeP04} & 2004 & ISCA 2004 & 6 & 0 & 0 & No & n/a\\
\rowlabel{a:WolinskiKG04}WolinskiKG04 \href{https://doi.org/10.1109/DSD.2004.1333291}{WolinskiKG04} & \hyperref[auth:a665]{C. Wolinski}, \hyperref[auth:a666]{K. Kuchcinski}, \hyperref[auth:a667]{Maya B. Gokhale} & A Constraints Programming Approach to Communication Scheduling on SoPC Architectures & \href{../works/WolinskiKG04.pdf}{Yes} & \cite{WolinskiKG04} & 2004 & DSD 2004 & 8 & 0 & 9 & \ref{b:WolinskiKG04} & n/a\\
\rowlabel{a:BeckPS03}BeckPS03 \href{http://www.aaai.org/Library/ICAPS/2003/icaps03-027.php}{BeckPS03} & \hyperref[auth:a89]{J. Christopher Beck}, \hyperref[auth:a833]{P. Prosser}, \hyperref[auth:a834]{E. Selensky} & Vehicle Routing and Job Shop Scheduling: What's the Difference? & \href{../works/BeckPS03.pdf}{Yes} & \cite{BeckPS03} & 2003 & ICAPS 2003 & 10 & 0 & 0 & \ref{b:BeckPS03} & n/a\\
\rowlabel{a:BourdaisGP03}BourdaisGP03 \href{https://doi.org/10.1007/978-3-540-45193-8_11}{BourdaisGP03} & \hyperref[auth:a1224]{S. Bourdais}, \hyperref[auth:a1225]{P. Galinier}, \hyperref[auth:a8]{G. Pesant} & {HIBISCUS:} {A} Constraint Programming Application to Staff Scheduling in Health Care & \href{../works/BourdaisGP03.pdf}{Yes} & \cite{BourdaisGP03} & 2003 & CP 2003 & 15 & 29 & 5 & \ref{b:BourdaisGP03} & n/a\\
\rowlabel{a:DannaP03}DannaP03 \href{https://doi.org/10.1007/978-3-540-45193-8_59}{DannaP03} & \hyperref[auth:a289]{E. Danna}, \hyperref[auth:a290]{L. Perron} & Structured vs. Unstructured Large Neighborhood Search: {A} Case Study on Job-Shop Scheduling Problems with Earliness and Tardiness Costs & \href{../works/DannaP03.pdf}{Yes} & \cite{DannaP03} & 2003 & CP 2003 & 5 & 21 & 3 & \ref{b:DannaP03} & n/a\\
\rowlabel{a:FrankK03}FrankK03 \href{http://www.aaai.org/Library/ICAPS/2003/icaps03-023.php}{FrankK03} & \hyperref[auth:a382]{J. Frank}, \hyperref[auth:a383]{E. K{\"{u}}rkl{\"{u}}} & SOFIA's Choice: Scheduling Observations for an Airborne Observatory & \href{../works/FrankK03.pdf}{Yes} & \cite{FrankK03} & 2003 & ICAPS 2003 & 10 & 0 & 0 & \ref{b:FrankK03} & n/a\\
\rowlabel{a:Kumar03}Kumar03 \href{https://doi.org/10.1007/978-3-540-45193-8_45}{Kumar03} & \hyperref[auth:a288]{T. K. Satish Kumar} & Incremental Computation of Resource-Envelopes in Producer-Consumer Models & \href{../works/Kumar03.pdf}{Yes} & \cite{Kumar03} & 2003 & CP 2003 & 15 & 4 & 2 & \ref{b:Kumar03} & n/a\\
\rowlabel{a:OddiPCC03}OddiPCC03 \href{https://doi.org/10.1007/978-3-540-45193-8_39}{OddiPCC03} & \hyperref[auth:a284]{A. Oddi}, \hyperref[auth:a285]{N. Policella}, \hyperref[auth:a286]{A. Cesta}, \hyperref[auth:a287]{G. Cortellessa} & Generating High Quality Schedules for a Spacecraft Memory Downlink Problem & \href{../works/OddiPCC03.pdf}{Yes} & \cite{OddiPCC03} & 2003 & CP 2003 & 15 & 8 & 6 & \ref{b:OddiPCC03} & n/a\\
\rowlabel{a:ValleMGT03}ValleMGT03 \href{https://doi.org/10.1007/978-3-540-45226-3_180}{ValleMGT03} & \hyperref[auth:a672]{Carmelo Del Valle}, \hyperref[auth:a673]{Antonio A. M{\'{a}}rquez}, \hyperref[auth:a674]{Rafael M. Gasca}, \hyperref[auth:a675]{M. Toro} & On Selecting and Scheduling Assembly Plans Using Constraint Programming & \href{../works/ValleMGT03.pdf}{Yes} & \cite{ValleMGT03} & 2003 & KES 2003 & 8 & 7 & 7 & \ref{b:ValleMGT03} & n/a\\
\rowlabel{a:Vilim03}Vilim03 \href{https://doi.org/10.1007/978-3-540-45193-8_124}{Vilim03} & \hyperref[auth:a121]{P. Vil{\'{\i}}m} & Computing Explanations for Global Scheduling Constraints & \href{../works/Vilim03.pdf}{Yes} & \cite{Vilim03} & 2003 & CP 2003 & 1 & 1 & 1 & \ref{b:Vilim03} & n/a\\
\rowlabel{a:Wolf03}Wolf03 \href{https://doi.org/10.1007/978-3-540-45193-8_50}{Wolf03} & \hyperref[auth:a51]{A. Wolf} & Pruning while Sweeping over Task Intervals & \href{../works/Wolf03.pdf}{Yes} & \cite{Wolf03} & 2003 & CP 2003 & 15 & 11 & 7 & \ref{b:Wolf03} & n/a\\
\rowlabel{a:Bartak02}Bartak02 \href{https://doi.org/10.1007/3-540-46135-3_39}{Bartak02} & \hyperref[auth:a153]{R. Bart{\'{a}}k} & Visopt ShopFloor: On the Edge of Planning and Scheduling & \href{../works/Bartak02.pdf}{Yes} & \cite{Bartak02} & 2002 & CP 2002 & 16 & 6 & 4 & \ref{b:Bartak02} & n/a\\
\rowlabel{a:Bartak02a}Bartak02a \href{https://doi.org/10.1007/3-540-36607-5_14}{Bartak02a} & \hyperref[auth:a153]{R. Bart{\'{a}}k} & Visopt ShopFloor: Going Beyond Traditional Scheduling & \href{../works/Bartak02a.pdf}{Yes} & \cite{Bartak02a} & 2002 & ERCIM/CologNet 2002 & 15 & 1 & 9 & \ref{b:Bartak02a} & n/a\\
\rowlabel{a:BeldiceanuC02}BeldiceanuC02 \href{https://doi.org/10.1007/3-540-46135-3_5}{BeldiceanuC02} & \hyperref[auth:a129]{N. Beldiceanu}, \hyperref[auth:a91]{M. Carlsson} & A New Multi-resource cumulatives Constraint with Negative Heights & \href{../works/BeldiceanuC02.pdf}{Yes} & \cite{BeldiceanuC02} & 2002 & CP 2002 & 17 & 33 & 9 & \ref{b:BeldiceanuC02} & n/a\\
\rowlabel{a:BenoistGR02}BenoistGR02 \href{https://doi.org/10.1007/3-540-46135-3_40}{BenoistGR02} & \hyperref[auth:a1182]{T. Benoist}, \hyperref[auth:a1183]{E. Gaudin}, \hyperref[auth:a1184]{B. Rottembourg} & Constraint Programming Contribution to Benders Decomposition: {A} Case Study & \href{../works/BenoistGR02.pdf}{Yes} & \cite{BenoistGR02} & 2002 & CP 2002 & 15 & 13 & 11 & \ref{b:BenoistGR02} & n/a\\
\rowlabel{a:ElkhyariGJ02}ElkhyariGJ02 \href{https://doi.org/10.1007/3-540-46135-3_49}{ElkhyariGJ02} & \hyperref[auth:a294]{A. Elkhyari}, \hyperref[auth:a295]{C. Gu{\'{e}}ret}, \hyperref[auth:a249]{N. Jussien} & Conflict-Based Repair Techniques for Solving Dynamic Scheduling Problems & \href{../works/ElkhyariGJ02.pdf}{Yes} & \cite{ElkhyariGJ02} & 2002 & CP 2002 & 6 & 1 & 6 & \ref{b:ElkhyariGJ02} & n/a\\
\rowlabel{a:ElkhyariGJ02a}ElkhyariGJ02a \href{https://doi.org/10.1007/978-3-540-45157-0_3}{ElkhyariGJ02a} & \hyperref[auth:a294]{A. Elkhyari}, \hyperref[auth:a295]{C. Gu{\'{e}}ret}, \hyperref[auth:a249]{N. Jussien} & Solving Dynamic Resource Constraint Project Scheduling Problems Using New Constraint Programming Tools & \href{../works/ElkhyariGJ02a.pdf}{Yes} & \cite{ElkhyariGJ02a} & 2002 & PATAT 2002 & 24 & 9 & 20 & \ref{b:ElkhyariGJ02a} & n/a\\
\rowlabel{a:FukunagaHFAMN02}FukunagaHFAMN02 \href{http://www.aaai.org/Library/AAAI/2002/aaai02-123.php}{FukunagaHFAMN02} & \hyperref[auth:a1351]{Alex S. Fukunaga}, \hyperref[auth:a1352]{E. Hamilton}, \hyperref[auth:a1353]{J. Fama}, \hyperref[auth:a1354]{D. Andre}, \hyperref[auth:a1355]{O. Matan}, \hyperref[auth:a1356]{Illah R. Nourbakhsh} & Staff Scheduling for Inbound Call Centers and Customer Contact Centers & \href{../works/FukunagaHFAMN02.pdf}{Yes} & \cite{FukunagaHFAMN02} & 2002 & AAAI 2002 & 8 & 0 & 0 & \ref{b:FukunagaHFAMN02} & n/a\\
\rowlabel{a:HookerY02}HookerY02 \href{https://doi.org/10.1007/3-540-46135-3_46}{HookerY02} & \hyperref[auth:a161]{John N. Hooker}, \hyperref[auth:a293]{H. Yan} & A Relaxation of the Cumulative Constraint & \href{../works/HookerY02.pdf}{Yes} & \cite{HookerY02} & 2002 & CP 2002 & 5 & 8 & 7 & \ref{b:HookerY02} & n/a\\
\rowlabel{a:KamarainenS02}KamarainenS02 \href{https://doi.org/10.1007/3-540-46135-3_11}{KamarainenS02} & \hyperref[auth:a292]{O. Kamarainen}, \hyperref[auth:a167]{Hani El Sakkout} & Local Probing Applied to Scheduling & \href{../works/KamarainenS02.pdf}{Yes} & \cite{KamarainenS02} & 2002 & CP 2002 & 17 & 9 & 13 & \ref{b:KamarainenS02} & n/a\\
\rowlabel{a:LimAHO02a}LimAHO02a \href{http://www.aaai.org/Library/AAAI/2002/aaai02-175.php}{LimAHO02a} & \hyperref[auth:a281]{A. Lim}, \hyperref[auth:a1357]{Juay Chin Ang}, \hyperref[auth:a1358]{W. Ho}, \hyperref[auth:a1359]{W. Oon} & UTTSExam: {A} University Examination Timetable Scheduler & \href{../works/LimAHO02a.pdf}{Yes} & \cite{LimAHO02a} & 2002 & AAAI 2002 & 2 & 0 & 0 & \ref{b:LimAHO02a} & n/a\\
\rowlabel{a:Muscettola02}Muscettola02 \href{https://doi.org/10.1007/3-540-46135-3_10}{Muscettola02} & \hyperref[auth:a291]{N. Muscettola} & Computing the Envelope for Stepwise-Constant Resource Allocations & \href{../works/Muscettola02.pdf}{Yes} & \cite{Muscettola02} & 2002 & CP 2002 & 16 & 14 & 4 & \ref{b:Muscettola02} & n/a\\
\rowlabel{a:Vilim02}Vilim02 \href{https://doi.org/10.1007/3-540-46135-3_62}{Vilim02} & \hyperref[auth:a121]{P. Vil{\'{\i}}m} & Batch Processing with Sequence Dependent Setup Times & \href{../works/Vilim02.pdf}{Yes} & \cite{Vilim02} & 2002 & CP 2002 & 1 & 6 & 1 & \ref{b:Vilim02} & n/a\\
\rowlabel{a:ZhuS02}ZhuS02 \href{https://doi.org/10.1007/3-540-47961-9_69}{ZhuS02} & \hyperref[auth:a680]{Kenny Qili Zhu}, \hyperref[auth:a681]{Andrew E. Santosa} & A Meeting Scheduling System Based on Open Constraint Programming & \href{../works/ZhuS02.pdf}{Yes} & \cite{ZhuS02} & 2002 & CAiSE 2002 & 5 & 0 & 5 & \ref{b:ZhuS02} & n/a\\
\rowlabel{a:BeldiceanuC01}BeldiceanuC01 \href{https://doi.org/10.1007/3-540-45578-7_26}{BeldiceanuC01} & \hyperref[auth:a129]{N. Beldiceanu}, \hyperref[auth:a91]{M. Carlsson} & Sweep as a Generic Pruning Technique Applied to the Non-overlapping Rectangles Constraint & \href{../works/BeldiceanuC01.pdf}{Yes} & \cite{BeldiceanuC01} & 2001 & CP 2001 & 15 & 34 & 0 & \ref{b:BeldiceanuC01} & n/a\\
\rowlabel{a:EreminW01}EreminW01 \href{https://doi.org/10.1007/3-540-45578-7_1}{EreminW01} & \hyperref[auth:a1063]{A. Eremin}, \hyperref[auth:a117]{Mark G. Wallace} & Hybrid Benders Decomposition Algorithms in Constraint Logic Programming & \href{../works/EreminW01.pdf}{Yes} & \cite{EreminW01} & 2001 & CP 2001 & 15 & 27 & 7 & \ref{b:EreminW01} & n/a\\
\rowlabel{a:Thorsteinsson01}Thorsteinsson01 \href{https://doi.org/10.1007/3-540-45578-7_2}{Thorsteinsson01} & \hyperref[auth:a881]{Erlendur S. Thorsteinsson} & Branch-and-Check: {A} Hybrid Framework Integrating Mixed Integer Programming and Constraint Logic Programming & \href{../works/Thorsteinsson01.pdf}{Yes} & \cite{Thorsteinsson01} & 2001 & CP 2001 & 15 & 67 & 12 & \ref{b:Thorsteinsson01} & n/a\\
\rowlabel{a:VanczaM01}VanczaM01 \href{https://doi.org/10.1007/3-540-45578-7_60}{VanczaM01} & \hyperref[auth:a280]{J. V{\'{a}}ncza}, \hyperref[auth:a296]{A. M{\'{a}}rkus} & A Constraint Engine for Manufacturing Process Planning & \href{../works/VanczaM01.pdf}{Yes} & \cite{VanczaM01} & 2001 & CP 2001 & 15 & 2 & 19 & \ref{b:VanczaM01} & n/a\\
\rowlabel{a:VerfaillieL01}VerfaillieL01 \href{https://doi.org/10.1007/3-540-45578-7_55}{VerfaillieL01} & \hyperref[auth:a174]{G. Verfaillie}, \hyperref[auth:a173]{M. Lema{\^{\i}}tre} & Selecting and Scheduling Observations for Agile Satellites: Some Lessons from the Constraint Reasoning Community Point of View & \href{../works/VerfaillieL01.pdf}{Yes} & \cite{VerfaillieL01} & 2001 & CP 2001 & 15 & 11 & 6 & \ref{b:VerfaillieL01} & n/a\\
\rowlabel{a:AngelsmarkJ00}AngelsmarkJ00 \href{https://doi.org/10.1007/3-540-45349-0_35}{AngelsmarkJ00} & \hyperref[auth:a297]{O. Angelsmark}, \hyperref[auth:a298]{P. Jonsson} & Some Observations on Durations, Scheduling and Allen's Algebra & \href{../works/AngelsmarkJ00.pdf}{Yes} & \cite{AngelsmarkJ00} & 2000 & CP 2000 & 5 & 1 & 9 & \ref{b:AngelsmarkJ00} & n/a\\
\rowlabel{a:CestaOS00}CestaOS00 \href{http://www.aaai.org/Library/AAAI/2000/aaai00-114.php}{CestaOS00} & \hyperref[auth:a286]{A. Cesta}, \hyperref[auth:a284]{A. Oddi}, \hyperref[auth:a300]{Stephen F. Smith} & Iterative Flattening: {A} Scalable Method for Solving Multi-Capacity Scheduling Problems & \href{../works/CestaOS00.pdf}{Yes} & \cite{CestaOS00} & 2000 & AAAI 2000 & 6 & 0 & 0 & \ref{b:CestaOS00} & n/a\\
\rowlabel{a:FocacciLN00}FocacciLN00 \href{http://www.aaai.org/Library/AIPS/2000/aips00-010.php}{FocacciLN00} & \hyperref[auth:a782]{F. Focacci}, \hyperref[auth:a118]{P. Laborie}, \hyperref[auth:a662]{W. Nuijten} & Solving Scheduling Problems with Setup Times and Alternative Resources & \href{../works/FocacciLN00.pdf}{Yes} & \cite{FocacciLN00} & 2000 & AIPS 2000 & 10 & 0 & 0 & \ref{b:FocacciLN00} & n/a\\
\rowlabel{a:Junker00}Junker00 \href{http://www.aaai.org/Library/AAAI/2000/aaai00-139.php}{Junker00} & \hyperref[auth:a1350]{U. Junker} & Preference-Based Search for Scheduling & \href{../works/Junker00.pdf}{Yes} & \cite{Junker00} & 2000 & AAAI 2000 & 6 & 0 & 0 & \ref{b:Junker00} & n/a\\
\rowlabel{a:Refalo00}Refalo00 \href{https://doi.org/10.1007/3-540-45349-0_27}{Refalo00} & \hyperref[auth:a256]{P. Refalo} & Linear Formulation of Constraint Programming Models and Hybrid Solvers & \href{../works/Refalo00.pdf}{Yes} & \cite{Refalo00} & 2000 & CP 2000 & 15 & 35 & 11 & \ref{b:Refalo00} & n/a\\
\rowlabel{a:TsurutaS00}TsurutaS00 \href{}{TsurutaS00} & \hyperref[auth:a1290]{T. Tsuruta}, \hyperref[auth:a1291]{T. Shintani} & Scheduling Meetings Using Distributed Valued Constraint Satisfaction Algorithm & No & \cite{TsurutaS00} & 2000 & ECAI 2000 & 5 & 0 & 0 & No & n/a\\
\rowlabel{a:WallaceF00}WallaceF00 \href{}{WallaceF00} & \hyperref[auth:a1292]{Richard J. Wallace}, \hyperref[auth:a275]{Eugene C. Freuder} & Dispatchability Conditions for Schedules with Consumable Resources & No & \cite{WallaceF00} & 2000 & ECAI 2000 & 7 & 0 & 0 & No & n/a\\
\rowlabel{a:AbdennadherS99}AbdennadherS99 \href{http://www.aaai.org/Library/IAAI/1999/iaai99-118.php}{AbdennadherS99} & \hyperref[auth:a1341]{S. Abdennadher}, \hyperref[auth:a717]{H. Schlenker} & Nurse Scheduling using Constraint Logic Programming & \href{../works/AbdennadherS99.pdf}{Yes} & \cite{AbdennadherS99} & 1999 & AAAI 1999 & 6 & 0 & 0 & \ref{b:AbdennadherS99} & n/a\\
\rowlabel{a:BeckF99}BeckF99 \href{http://www.aaai.org/Library/AAAI/1999/aaai99-097.php}{BeckF99} & \hyperref[auth:a89]{J. Christopher Beck}, \hyperref[auth:a304]{Mark S. Fox} & Scheduling Alternative Activities & \href{../works/BeckF99.pdf}{Yes} & \cite{BeckF99} & 1999 & AAAI 1999 & 8 & 0 & 0 & \ref{b:BeckF99} & n/a\\
\rowlabel{a:ChunCTY99}ChunCTY99 \href{http://www.aaai.org/Library/IAAI/1999/iaai99-111.php}{ChunCTY99} & \hyperref[auth:a1346]{Andy Hon Wai Chun}, \hyperref[auth:a1347]{Steve Ho Chuen Chan}, \hyperref[auth:a1348]{Francis Ming Fai Tsang}, \hyperref[auth:a1349]{Dennis Wai Ming Yeung} & {HKIA} {SAS:} {A} Constraint-Based Airport Stand Allocation System Developed with Software Components & \href{../works/ChunCTY99.pdf}{Yes} & \cite{ChunCTY99} & 1999 & AAAI 1999 & 8 & 0 & 0 & \ref{b:ChunCTY99} & n/a\\
\rowlabel{a:DorndorfPH99}DorndorfPH99 \href{http://dx.doi.org/10.1007/978-3-642-58409-1_35}{DorndorfPH99} & \hyperref[auth:a911]{U. Dorndorf}, \hyperref[auth:a441]{E. Pesch}, \hyperref[auth:a912]{Toàn Phan Huy} & Recent Developments in Scheduling & No & \cite{DorndorfPH99} & 1999 & Operations Research Proceedings 1999 & null & 0 & 34 & No & n/a\\
\rowlabel{a:JoLLH99}JoLLH99 \href{http://www.aaai.org/Library/IAAI/1999/iaai99-114.php}{JoLLH99} & \hyperref[auth:a1342]{G. Jo}, \hyperref[auth:a1343]{K. Lee}, \hyperref[auth:a1344]{H. Lee}, \hyperref[auth:a1345]{S. Hyun} & Ramp Activity Expert System for Scheduling and Co-ordination at an Airport & \href{../works/JoLLH99.pdf}{Yes} & \cite{JoLLH99} & 1999 & AAAI 1999 & 6 & 0 & 0 & \ref{b:JoLLH99} & n/a\\
\rowlabel{a:KorbaaYG99}KorbaaYG99 \href{https://doi.org/10.23919/ECC.1999.7099947}{KorbaaYG99} & \hyperref[auth:a686]{O. Korbaa}, \hyperref[auth:a687]{P. Yim}, \hyperref[auth:a688]{J. Gentina} & Solving transient scheduling problem for cyclic production using timed Petri nets and constraint programming & \href{../works/KorbaaYG99.pdf}{Yes} & \cite{KorbaaYG99} & 1999 & ECC 1999 & 8 & 1 & 0 & \ref{b:KorbaaYG99} & n/a\\
\rowlabel{a:Simonis99}Simonis99 \href{https://doi.org/10.1007/3-540-45406-3_6}{Simonis99} & \hyperref[auth:a17]{H. Simonis} & Building Industrial Applications with Constraint Programming & \href{../works/Simonis99.pdf}{Yes} & \cite{Simonis99} & 1999 & CCL'99 1999 & 39 & 5 & 18 & \ref{b:Simonis99} & n/a\\
\rowlabel{a:WatsonBHW99}WatsonBHW99 \href{http://www.aaai.org/Library/AAAI/1999/aaai99-098.php}{WatsonBHW99} & \hyperref[auth:a363]{J. Watson}, \hyperref[auth:a1338]{L. Barbulescu}, \hyperref[auth:a1339]{Adele E. Howe}, \hyperref[auth:a1340]{L. Darrell Whitley} & Algorithm Performance and Problem Structure for Flow-shop Scheduling & \href{../works/WatsonBHW99.pdf}{Yes} & \cite{WatsonBHW99} & 1999 & AAAI 1999 & 8 & 0 & 0 & \ref{b:WatsonBHW99} & n/a\\
\rowlabel{a:CestaOS98}CestaOS98 \href{https://doi.org/10.1007/3-540-49481-2_36}{CestaOS98} & \hyperref[auth:a286]{A. Cesta}, \hyperref[auth:a284]{A. Oddi}, \hyperref[auth:a300]{Stephen F. Smith} & Scheduling Multi-capacitated Resources Under Complex Temporal Constraints & \href{../works/CestaOS98.pdf}{Yes} & \cite{CestaOS98} & 1998 & CP 1998 & 1 & 5 & 0 & \ref{b:CestaOS98} & n/a\\
\rowlabel{a:FrostD98}FrostD98 \href{https://doi.org/10.1007/3-540-49481-2_40}{FrostD98} & \hyperref[auth:a301]{D. Frost}, \hyperref[auth:a302]{R. Dechter} & Optimizing with Constraints: {A} Case Study in Scheduling Maintenance of Electric Power Units & \href{../works/FrostD98.pdf}{Yes} & \cite{FrostD98} & 1998 & CP 1998 & 1 & 10 & 2 & \ref{b:FrostD98} & n/a\\
\rowlabel{a:GruianK98}GruianK98 \href{https://doi.org/10.1109/EURMIC.1998.711781}{GruianK98} & \hyperref[auth:a692]{F. Gruian}, \hyperref[auth:a666]{K. Kuchcinski} & Operation Binding and Scheduling for Low Power Using Constraint Logic Programming & \href{../works/GruianK98.pdf}{Yes} & \cite{GruianK98} & 1998 & EUROMICRO 1998 & 8 & 5 & 10 & \ref{b:GruianK98} & n/a\\
\rowlabel{a:PembertonG98}PembertonG98 \href{https://doi.org/10.1090/dimacs/057/06}{PembertonG98} & \hyperref[auth:a690]{Joseph C. Pemberton}, \hyperref[auth:a691]{Flavius Galiber III} & A constraint-based approach to satellite scheduling & \href{../works/PembertonG98.pdf}{Yes} & \cite{PembertonG98} & 1998 & DIMACS 1998 & 14 & 26 & 0 & \ref{b:PembertonG98} & n/a\\
\rowlabel{a:RodosekW98}RodosekW98 \href{https://doi.org/10.1007/3-540-49481-2_28}{RodosekW98} & \hyperref[auth:a299]{R. Rodosek}, \hyperref[auth:a117]{Mark G. Wallace} & A Generic Model and Hybrid Algorithm for Hoist Scheduling Problems & \href{../works/RodosekW98.pdf}{Yes} & \cite{RodosekW98} & 1998 & CP 1998 & 15 & 19 & 10 & \ref{b:RodosekW98} & n/a\\
\rowlabel{a:SakkoutRW98}SakkoutRW98 \href{}{SakkoutRW98} & \hyperref[auth:a167]{Hani El Sakkout}, \hyperref[auth:a1288]{T. Richards}, \hyperref[auth:a1289]{M. Wallace} & Minimal Perturbance in Dynamic Scheduling & No & \cite{SakkoutRW98} & 1998 & ECAI 1998 & 5 & 0 & 0 & No & n/a\\
\rowlabel{a:BaptisteP97}BaptisteP97 \href{https://doi.org/10.1007/BFb0017454}{BaptisteP97} & \hyperref[auth:a163]{P. Baptiste}, \hyperref[auth:a164]{Claude Le Pape} & Constraint Propagation and Decomposition Techniques for Highly Disjunctive and Highly Cumulative Project Scheduling Problems & \href{../works/BaptisteP97.pdf}{Yes} & \cite{BaptisteP97} & 1997 & CP 1997 & 15 & 8 & 10 & \ref{b:BaptisteP97} & n/a\\
\rowlabel{a:BeckDF97}BeckDF97 \href{https://doi.org/10.1007/BFb0017455}{BeckDF97} & \hyperref[auth:a89]{J. Christopher Beck}, \hyperref[auth:a250]{Andrew J. Davenport}, \hyperref[auth:a304]{Mark S. Fox} & Five Pitfalls of Empirical Scheduling Research & \href{../works/BeckDF97.pdf}{Yes} & \cite{BeckDF97} & 1997 & CP 1997 & 15 & 3 & 12 & \ref{b:BeckDF97} & n/a\\
\rowlabel{a:BeckDSF97}BeckDSF97 \href{http://www.aaai.org/Library/AAAI/1997/aaai97-037.php}{BeckDSF97} & \hyperref[auth:a89]{J. Christopher Beck}, \hyperref[auth:a250]{Andrew J. Davenport}, \hyperref[auth:a1311]{Edward M. Sitarski}, \hyperref[auth:a304]{Mark S. Fox} & Beyond Contention: Extending Texture-Based Scheduling Heuristics & \href{../works/BeckDSF97.pdf}{Yes} & \cite{BeckDSF97} & 1997 & AAAI 1997 & 8 & 0 & 0 & \ref{b:BeckDSF97} & n/a\\
\rowlabel{a:BeckDSF97a}BeckDSF97a \href{http://www.aaai.org/Library/AAAI/1997/aaai97-038.php}{BeckDSF97a} & \hyperref[auth:a89]{J. Christopher Beck}, \hyperref[auth:a250]{Andrew J. Davenport}, \hyperref[auth:a1311]{Edward M. Sitarski}, \hyperref[auth:a304]{Mark S. Fox} & Texture-Based Heuristics for Scheduling Revisited & \href{../works/BeckDSF97a.pdf}{Yes} & \cite{BeckDSF97a} & 1997 & AAAI 1997 & 8 & 0 & 0 & \ref{b:BeckDSF97a} & n/a\\
\rowlabel{a:BoucherBVBL97}BoucherBVBL97 \href{}{BoucherBVBL97} & \hyperref[auth:a696]{E. Boucher}, \hyperref[auth:a697]{A. Bachelu}, \hyperref[auth:a698]{C. Varnier}, \hyperref[auth:a699]{P. Baptiste}, \hyperref[auth:a700]{B. Legeard} & Multi-criteria Comparison Between Algorithmic, Constraint Logic and Specific Constraint Programming on a Real Schedulingt Problem & No & \cite{BoucherBVBL97} & 1997 & PACT 1997 & 18 & 0 & 0 & No & n/a\\
\rowlabel{a:Caseau97}Caseau97 \href{https://doi.org/10.1007/BFb0017437}{Caseau97} & \hyperref[auth:a303]{Y. Caseau} & Using Constraint Propagation for Complex Scheduling Problems: Managing Size, Complex Resources and Travel & \href{../works/Caseau97.pdf}{Yes} & \cite{Caseau97} & 1997 & CP 1997 & 4 & 0 & 0 & \ref{b:Caseau97} & n/a\\
\rowlabel{a:GetoorOFC97}GetoorOFC97 \href{http://www.aaai.org/Library/AAAI/1997/aaai97-047.php}{GetoorOFC97} & \hyperref[auth:a1316]{L. Getoor}, \hyperref[auth:a859]{G. Ottosson}, \hyperref[auth:a1317]{Markus P. J. Fromherz}, \hyperref[auth:a1318]{B. Carlson} & Effective Redundant Constraints for Online Scheduling & \href{../works/GetoorOFC97.pdf}{Yes} & \cite{GetoorOFC97} & 1997 & AAAI 1997 & 6 & 0 & 0 & \ref{b:GetoorOFC97} & n/a\\
\rowlabel{a:LeeKLKKYHP97}LeeKLKKYHP97 \href{http://www.aaai.org/Library/IAAI/1997/iaai97-182.php}{LeeKLKKYHP97} & \hyperref[auth:a1326]{Kyoung Jun Lee}, \hyperref[auth:a1327]{Hyun Woo Kim}, \hyperref[auth:a1328]{Jae Kyu Lee}, \hyperref[auth:a1329]{Tae Hwan Kim}, \hyperref[auth:a1330]{Chang Gon Kim}, \hyperref[auth:a1331]{Myoung Kyun Yoon}, \hyperref[auth:a1332]{Eui Jun Hwang}, \hyperref[auth:a1333]{Hyun Jeong Park} & Case and Constraint-Based Apartment Construction Project Planning System: FASTrak-APT & \href{../works/LeeKLKKYHP97.pdf}{Yes} & \cite{LeeKLKKYHP97} & 1997 & AAAI 1997 & 6 & 0 & 0 & \ref{b:LeeKLKKYHP97} & n/a\\
\rowlabel{a:MorgadoM97}MorgadoM97 \href{http://www.aaai.org/Library/IAAI/1997/iaai97-186.php}{MorgadoM97} & \hyperref[auth:a1319]{Ernesto M. Morgado}, \hyperref[auth:a1320]{Jo{\~{a}}o P. Martins} & CREWS{\ }NS: Scheduling Train Crew in The Netherlands & \href{../works/MorgadoM97.pdf}{Yes} & \cite{MorgadoM97} & 1997 & AAAI 1997 & 10 & 0 & 0 & \ref{b:MorgadoM97} & n/a\\
\rowlabel{a:MurphyRFSS97}MurphyRFSS97 \href{http://www.aaai.org/Library/IAAI/1997/iaai97-187.php}{MurphyRFSS97} & \hyperref[auth:a1321]{K. Murphy}, \hyperref[auth:a1322]{E. Ralston}, \hyperref[auth:a1323]{D. Friedlander}, \hyperref[auth:a1324]{R. Swab}, \hyperref[auth:a1325]{P. Steege} & The Scheduling of Rail at Union Pacific Railroad & \href{../works/MurphyRFSS97.pdf}{Yes} & \cite{MurphyRFSS97} & 1997 & AAAI 1997 & 10 & 0 & 0 & \ref{b:MurphyRFSS97} & n/a\\
\rowlabel{a:MurthyRAW97}MurthyRAW97 \href{}{MurthyRAW97} & \hyperref[auth:a1334]{Seshashayee S. Murthy}, \hyperref[auth:a1335]{J. Rachlin}, \hyperref[auth:a1336]{R. Akkiraju}, \hyperref[auth:a1337]{Frederick Y. Wu} & Agent-Based Cooperative Scheduling & No & \cite{MurthyRAW97} & 1997 & AAAI 1997 & 6 & 0 & 0 & No & n/a\\
\rowlabel{a:OddiS97}OddiS97 \href{http://www.aaai.org/Library/AAAI/1997/aaai97-048.php}{OddiS97} & \hyperref[auth:a284]{A. Oddi}, \hyperref[auth:a300]{Stephen F. Smith} & Stochastic Procedures for Generating Feasible Schedules & \href{../works/OddiS97.pdf}{Yes} & \cite{OddiS97} & 1997 & AAAI 1997 & 7 & 0 & 0 & \ref{b:OddiS97} & n/a\\
\rowlabel{a:PapeB97}PapeB97 \href{}{PapeB97} & \hyperref[auth:a164]{Claude Le Pape}, \hyperref[auth:a163]{P. Baptiste} & A Constraint Programming Library for Preemptive and Non-Preemptive Scheduling & No & \cite{PapeB97} & 1997 & PACT 1997 & 20 & 0 & 0 & No & n/a\\
\rowlabel{a:BrusoniCLMMT96}BrusoniCLMMT96 \href{https://doi.org/10.1007/3-540-61286-6_157}{BrusoniCLMMT96} & \hyperref[auth:a728]{V. Brusoni}, \hyperref[auth:a729]{L. Console}, \hyperref[auth:a726]{E. Lamma}, \hyperref[auth:a727]{P. Mello}, \hyperref[auth:a144]{M. Milano}, \hyperref[auth:a730]{P. Terenziani} & Resource-Based vs. Task-Based Approaches for Scheduling Problems & \href{../works/BrusoniCLMMT96.pdf}{Yes} & \cite{BrusoniCLMMT96} & 1996 & ISMIS 1996 & 10 & 1 & 9 & \ref{b:BrusoniCLMMT96} & n/a\\
\rowlabel{a:Colombani96}Colombani96 \href{https://doi.org/10.1007/3-540-61551-2_72}{Colombani96} & \hyperref[auth:a169]{Y. Colombani} & Constraint Programming: an Efficient and Practical Approach to Solving the Job-Shop Problem & \href{../works/Colombani96.pdf}{Yes} & \cite{Colombani96} & 1996 & CP 1996 & 15 & 4 & 5 & \ref{b:Colombani96} & n/a\\
\rowlabel{a:PapeB96}PapeB96 \href{}{PapeB96} & \hyperref[auth:a164]{Claude Le Pape}, \hyperref[auth:a163]{P. Baptiste} & Constraint Propagation Techniques for Disjunctive Scheduling: The Preemptive Case & No & \cite{PapeB96} & 1996 & ECAI 1996 & 5 & 0 & 0 & No & n/a\\
\rowlabel{a:RoweJCA96}RoweJCA96 \href{http://www.aaai.org/Library/IAAI/1996/iaai96-280.php}{RoweJCA96} & \hyperref[auth:a1307]{J. Rowe}, \hyperref[auth:a1308]{K. Jewers}, \hyperref[auth:a1309]{A. Codd}, \hyperref[auth:a1310]{A. Alcock} & Intelligent Retail Logistics Scheduling & \href{../works/RoweJCA96.pdf}{Yes} & \cite{RoweJCA96} & 1996 & AAAI 1996 & 9 & 0 & 0 & \ref{b:RoweJCA96} & n/a\\
\rowlabel{a:Schaerf96}Schaerf96 \href{}{Schaerf96} & \hyperref[auth:a1284]{A. Schaerf} & Scheduling Sport Tournaments using Constraint Logic Programming & No & \cite{Schaerf96} & 1996 & ECAI 1996 & 5 & 0 & 0 & No & n/a\\
\rowlabel{a:StidsenKM96}StidsenKM96 \href{}{StidsenKM96} & \hyperref[auth:a1285]{Thomas R. Stidsen}, \hyperref[auth:a1286]{L. V. Kragelund}, \hyperref[auth:a1287]{O. Mateescu} & Jobshop Scheduling in a Shipyard & No & \cite{StidsenKM96} & 1996 & ECAI 1996 & 8 & 0 & 0 & No & n/a\\
\rowlabel{a:Zhou96}Zhou96 \href{https://doi.org/10.1007/3-540-61551-2_97}{Zhou96} & \hyperref[auth:a177]{J. Zhou} & A Constraint Program for Solving the Job-Shop Problem & \href{../works/Zhou96.pdf}{Yes} & \cite{Zhou96} & 1996 & CP 1996 & 15 & 10 & 7 & \ref{b:Zhou96} & n/a\\
\rowlabel{a:Goltz95}Goltz95 \href{https://doi.org/10.1007/3-540-60299-2_33}{Goltz95} & \hyperref[auth:a306]{H. Goltz} & Reducing Domains for Search in {CLP(FD)} and Its Application to Job-Shop Scheduling & \href{../works/Goltz95.pdf}{Yes} & \cite{Goltz95} & 1995 & CP 1995 & 14 & 7 & 7 & \ref{b:Goltz95} & n/a\\
\rowlabel{a:Puget95}Puget95 \href{https://doi.org/10.1007/3-540-60299-2_43}{Puget95} & \hyperref[auth:a307]{J. Puget} & Applications of Constraint Programming & \href{../works/Puget95.pdf}{Yes} & \cite{Puget95} & 1995 & CP 1995 & 4 & 6 & 2 & \ref{b:Puget95} & n/a\\
\rowlabel{a:Simonis95}Simonis95 \href{https://doi.org/10.1007/3-540-60299-2_42}{Simonis95} & \hyperref[auth:a17]{H. Simonis} & The {CHIP} System and Its Applications & \href{../works/Simonis95.pdf}{Yes} & \cite{Simonis95} & 1995 & CP 1995 & 4 & 7 & 3 & \ref{b:Simonis95} & n/a\\
\rowlabel{a:Simonis95a}Simonis95a \href{https://doi.org/10.1007/3-540-60794-3_11}{Simonis95a} & \hyperref[auth:a17]{H. Simonis} & Application Development with the {CHIP} System & \href{../works/Simonis95a.pdf}{Yes} & \cite{Simonis95a} & 1995 & CONTESSA 1995 & 21 & 1 & 12 & \ref{b:Simonis95a} & n/a\\
\rowlabel{a:SimonisC95}SimonisC95 \href{https://doi.org/10.1007/3-540-60299-2_27}{SimonisC95} & \hyperref[auth:a17]{H. Simonis}, \hyperref[auth:a305]{T. Cornelissens} & Modelling Producer/Consumer Constraints & \href{../works/SimonisC95.pdf}{Yes} & \cite{SimonisC95} & 1995 & CP 1995 & 14 & 17 & 8 & \ref{b:SimonisC95} & n/a\\
\rowlabel{a:Touraivane95}Touraivane95 \href{https://doi.org/10.1007/3-540-60299-2_41}{Touraivane95} & \hyperref[auth:a308]{Toura{\"{\i}}vane} & Constraint Programming and Industrial Applications & \href{../works/Touraivane95.pdf}{Yes} & \cite{Touraivane95} & 1995 & CP 1995 & 3 & 2 & 1 & \ref{b:Touraivane95} & n/a\\
\rowlabel{a:CrawfordB94}CrawfordB94 \href{http://www.aaai.org/Library/AAAI/1994/aaai94-168.php}{CrawfordB94} & \hyperref[auth:a1301]{James M. Crawford}, \hyperref[auth:a1302]{Andrew B. Baker} & Experimental Results on the Application of Satisfiability Algorithms to Scheduling Problems & \href{../works/CrawfordB94.pdf}{Yes} & \cite{CrawfordB94} & 1994 & AAAI 1994 & 6 & 0 & 0 & \ref{b:CrawfordB94} & n/a\\
\rowlabel{a:JourdanFRD94}JourdanFRD94 \href{}{JourdanFRD94} & \hyperref[auth:a703]{J. Jourdan}, \hyperref[auth:a704]{F. Fages}, \hyperref[auth:a705]{D. Rozzonelli}, \hyperref[auth:a706]{A. Demeure} & Data Alignment and Task Scheduling On Parallel Machines Using Concurrent Constraint Model-based Programming & No & \cite{JourdanFRD94} & 1994 & ILPS 1994 & 1 & 0 & 0 & No & n/a\\
\rowlabel{a:Muscettola94}Muscettola94 \href{http://www.aaai.org/Library/AAAI/1994/aaai94-170.php}{Muscettola94} & \hyperref[auth:a291]{N. Muscettola} & On the Utility of Bottleneck Reasoning for Scheduling & \href{../works/Muscettola94.pdf}{Yes} & \cite{Muscettola94} & 1994 & AAAI 1994 & 6 & 0 & 0 & \ref{b:Muscettola94} & n/a\\
\rowlabel{a:NuijtenA94}NuijtenA94 \href{}{NuijtenA94} & \hyperref[auth:a662]{W. Nuijten}, \hyperref[auth:a783]{E. Aarts} & Constraint Satisfaction for Multiple Capacitated Job Shop Scheduling & \href{../works/NuijtenA94.pdf}{Yes} & \cite{NuijtenA94} & 1994 & ECAI 1994 & 5 & 0 & 0 & \ref{b:NuijtenA94} & n/a\\
\rowlabel{a:NuijtenA94a}NuijtenA94a \href{}{NuijtenA94a} & \hyperref[auth:a1278]{W. P. M. Nuijten}, \hyperref[auth:a1279]{Emile H. L. Aarts} & Constraint Satisfaction for Multiple Capacitated Job Shop Scheduling & No & \cite{NuijtenA94a} & 1994 & ECAI 1994 & 5 & 0 & 0 & No & n/a\\
\rowlabel{a:Rodosek94}Rodosek94 \href{}{Rodosek94} & \hyperref[auth:a299]{R. Rodosek} & Combining Constraint Network and Causal Theory to Solve Scheduling Problems from a {CSP} Perspective & No & \cite{Rodosek94} & 1994 & ECAI 1994 & 5 & 0 & 0 & No & n/a\\
\rowlabel{a:Wallace94}Wallace94 \href{}{Wallace94} & \hyperref[auth:a117]{Mark G. Wallace} & Applying Constraints for Scheduling & No & \cite{Wallace94} & 1994 & Constraint Programming 1994 & 19 & 0 & 0 & No & n/a\\
\rowlabel{a:YeGMH94}YeGMH94 \href{}{YeGMH94} & \hyperref[auth:a1280]{P. Ye}, \hyperref[auth:a1281]{D. Glass}, \hyperref[auth:a1282]{Michael F. McTear}, \hyperref[auth:a1283]{John G. Hughes} & Job Cost and Constraint Relaxation for Scheduling Problem Solving in the {CLP} Paradigm & No & \cite{YeGMH94} & 1994 & ECAI 1994 & 5 & 0 & 0 & No & n/a\\
\rowlabel{a:YoshikawaKNW94}YoshikawaKNW94 \href{http://www.aaai.org/Library/AAAI/1994/aaai94-171.php}{YoshikawaKNW94} & \hyperref[auth:a1303]{M. Yoshikawa}, \hyperref[auth:a1304]{K. Kaneko}, \hyperref[auth:a1305]{Y. Nomura}, \hyperref[auth:a1306]{M. Watanabe} & A Constraint-Based Approach to High-School Timetabling Problems: {A} Case Study & \href{../works/YoshikawaKNW94.pdf}{Yes} & \cite{YoshikawaKNW94} & 1994 & AAAI 1994 & 6 & 0 & 0 & \ref{b:YoshikawaKNW94} & n/a\\
\rowlabel{a:SmithC93}SmithC93 \href{http://www.aaai.org/Library/AAAI/1993/aaai93-022.php}{SmithC93} & \hyperref[auth:a300]{Stephen F. Smith}, \hyperref[auth:a1300]{C. Cheng} & Slack-Based Heuristics for Constraint Satisfaction Scheduling & \href{../works/SmithC93.pdf}{Yes} & \cite{SmithC93} & 1993 & AAAI 1993 & 6 & 0 & 0 & \ref{b:SmithC93} & n/a\\
\rowlabel{a:BaptisteLV92}BaptisteLV92 \href{https://doi.org/10.1109/ROBOT.1992.220195}{BaptisteLV92} & \hyperref[auth:a699]{P. Baptiste}, \hyperref[auth:a700]{B. Legeard}, \hyperref[auth:a698]{C. Varnier} & Hoist scheduling problem: an approach based on constraint logic programming & \href{../works/BaptisteLV92.pdf}{Yes} & \cite{BaptisteLV92} & 1992 & ICRA 1992 & 6 & 13 & 6 & \ref{b:BaptisteLV92} & n/a\\
\rowlabel{a:DincbasS91}DincbasS91 \href{}{DincbasS91} & \hyperref[auth:a723]{M. Dincbas}, \hyperref[auth:a17]{H. Simonis} & Apache-a constraint based, automated stand allocation system & \href{../works/DincbasS91.pdf}{Yes} & \cite{DincbasS91} & 1991 & ASTAIR 1991 & 13 & 0 & 0 & \ref{b:DincbasS91} & n/a\\
\rowlabel{a:ErtlK91}ErtlK91 \href{https://doi.org/10.1007/3-540-54444-5_89}{ErtlK91} & \hyperref[auth:a708]{M. Anton Ertl}, \hyperref[auth:a709]{A. Krall} & Optimal Instruction Scheduling using Constraint Logic Programming & \href{../works/ErtlK91.pdf}{Yes} & \cite{ErtlK91} & 1991 & PLILP 1991 & 12 & 14 & 14 & \ref{b:ErtlK91} & n/a\\
\rowlabel{a:Hamscher91}Hamscher91 \href{http://www.aaai.org/Library/AAAI/1991/aaai91-079.php}{Hamscher91} & \hyperref[auth:a1299]{W. Hamscher} & {ACP:} Reason Maintenance and Inference Control for Constraint Propagation Over Intervals & \href{../works/Hamscher91.pdf}{Yes} & \cite{Hamscher91} & 1991 & AAAI 1991 & 6 & 0 & 0 & \ref{b:Hamscher91} & n/a\\
\rowlabel{a:EskeyZ90}EskeyZ90 \href{http://www.aaai.org/Library/AAAI/1990/aaai90-136.php}{EskeyZ90} & \hyperref[auth:a1297]{M. Eskey}, \hyperref[auth:a1298]{M. Zweben} & Learning Search Control for Constraint-Based Scheduling & \href{../works/EskeyZ90.pdf}{Yes} & \cite{EskeyZ90} & 1990 & AAAI 1990 & 8 & 0 & 0 & \ref{b:EskeyZ90} & n/a\\
\rowlabel{a:FoxS90}FoxS90 \href{}{FoxS90} & \hyperref[auth:a304]{Mark S. Fox}, \hyperref[auth:a1058]{Norman M. Sadeh} & Why is Scheduling Difficult? {A} {CSP} Perspective & \href{../works/FoxS90.pdf}{Yes} & \cite{FoxS90} & 1990 & ECAI 1990 & 14 & 0 & 0 & \ref{b:FoxS90} & n/a\\
\rowlabel{a:MintonJPL90}MintonJPL90 \href{http://www.aaai.org/Library/AAAI/1990/aaai90-003.php}{MintonJPL90} & \hyperref[auth:a1230]{S. Minton}, \hyperref[auth:a1231]{Mark D. Johnston}, \hyperref[auth:a1232]{Andrew B. Philips}, \hyperref[auth:a1233]{P. Laird} & Solving Large-Scale Constraint-Satisfaction and Scheduling Problems Using a Heuristic Repair Method & \href{../works/MintonJPL90.pdf}{Yes} & \cite{MintonJPL90} & 1990 & AAAI 1990 & 8 & 0 & 0 & \ref{b:MintonJPL90} & n/a\\
\rowlabel{a:Valdes87}Valdes87 \href{http://www.aaai.org/Library/AAAI/1987/aaai87-046.php}{Valdes87} & \hyperref[auth:a1296]{Ra{\'{u}}l E. Vald{\'{e}}s{-}P{\'{e}}rez} & The Satisfiability of Temporal Constraint Networks & \href{../works/Valdes87.pdf}{Yes} & \cite{Valdes87} & 1987 & AAAI 1987 & 5 & 0 & 0 & \ref{b:Valdes87} & n/a\\
\rowlabel{a:Rit86}Rit86 \href{http://www.aaai.org/Library/AAAI/1986/aaai86-064.php}{Rit86} & \hyperref[auth:a1295]{J. Rit} & Propagating Temporal Constraints for Scheduling & \href{../works/Rit86.pdf}{Yes} & \cite{Rit86} & 1986 & AAAI 1986 & 6 & 0 & 0 & \ref{b:Rit86} & n/a\\
\rowlabel{a:FoxAS82}FoxAS82 \href{http://www.aaai.org/Library/AAAI/1982/aaai82-037.php}{FoxAS82} & \hyperref[auth:a304]{Mark S. Fox}, \hyperref[auth:a1018]{Bradley P. Allen}, \hyperref[auth:a1019]{G. Strohm} & Job-Shop Scheduling: An Investigation in Constraint-Directed Reasoning & \href{../works/FoxAS82.pdf}{Yes} & \cite{FoxAS82} & 1982 & AAAI 1982 & 4 & 0 & 0 & \ref{b:FoxAS82} & n/a\\
\end{longtable}
}



\clearpage
\subsection{Extracted Concepts}
{\scriptsize
\begin{longtable}{p{3cm}rp{4cm}p{1.5cm}p{2cm}p{1.5cm}p{1.5cm}p{1.5cm}p{1.5cm}p{2cm}rp{1.5cm}}
\caption{Automatically Extracted Paper Properties (Requires Local Copy)}\\ \toprule
Work & Pages & Concepts & Classification & Constraints & \shortstack{Prog\\Languages} & \shortstack{CP\\Systems} & Areas & Industries & Benchmarks & Links & Algorithm\\ \midrule\endhead
\bottomrule
\endfoot
\href{papers/AalianPG23.pdf}{AalianPG23}~\cite{AalianPG23} & 16 & scheduling, preempt, activity, flow-shop, order, transportation, machine, make-span, resource &  & cycle, alwaysIn, cumulative, noOverlap, endBeforeStart &  & CPO, Cplex & steel cable & mining industry & real-world & 1 & \\
\href{papers/AbrilSB05.pdf}{AbrilSB05}~\cite{AbrilSB05} & 1 & distributed, scheduling, multi-agent, order &  &  &  &  & railway &  &  & 0 & \\
\href{papers/Acuna-AgostMFG09.pdf}{Acuna-AgostMFG09}~\cite{Acuna-AgostMFG09} & 2 & re-scheduling, order, scheduling, transportation &  &  &  &  & railway &  & Roadef & 1 & \\
\href{papers/AkkerDH07.pdf}{AkkerDH07}~\cite{AkkerDH07} & 15 & resource, due-date, scheduling, make-span, precedence, order, cmax, completion-time, machine, job, lateness, release-date, sequence dependent setup, preempt & RCPSP, single machine, parallel machine & cumulative &  & Cplex &  &  &  & 0 & \\
\href{papers/AlesioNBG14.pdf}{AlesioNBG14}~\cite{AlesioNBG14} & 18 & preempt, job-shop, distributed, scheduling, completion-time, make-span, resource, open-shop, order, job, activity, task &  & alldifferent &  & OPL, Cplex & automotive &  & benchmark & 2 & \\
\href{papers/AngelsmarkJ00.pdf}{AngelsmarkJ00}~\cite{AngelsmarkJ00} & 5 & resource, job, order, scheduling, task, job-shop &  &  &  &  &  &  &  & 0 & \\
\href{papers/AntuoriHHEN21.pdf}{AntuoriHHEN21}~\cite{AntuoriHHEN21} & 16 & release-date, resource, transportation, job, order, due-date, tardiness, scheduling, machine, task, job-shop, precedence &  & cycle & C++, Java & Choco Solver, Gecode & automotive, car manufacturing & automotive industry & gitlab, supplementary material & 1 & \\
\href{papers/ArbaouiY18.pdf}{ArbaouiY18}~\cite{ArbaouiY18} & 10 & setup-time, order, machine, make-span, sequence dependent setup, completion-time, cmax, resource, job, scheduling & single machine, parallel machine & alternative constraint, noOverlap, cumulative & C++ & OZ, Cplex &  &  & benchmark & 0 & \\
\href{papers/ArmstrongGOS21.pdf}{ArmstrongGOS21}~\cite{ArmstrongGOS21} & 18 & machine, transportation, flow-shop, job-shop, scheduling, job, make-span, order, completion-time, sequence dependent setup, preempt, resource, setup-time, precedence, task, cmax & HFF & alternative constraint, cycle, table constraint, circuit, diffn, bin-packing, cumulative & Java, Prolog & OZ, MiniZinc, CPO, Chuffed, Gecode, SICStus, Cplex, CHIP & robot & packaging industry & instance generator, industry partner, zenodo, supplementary material, real-world, industrial partner, benchmark & 1 & energetic reasoning\\
\href{papers/ArmstrongGOS22.pdf}{ArmstrongGOS22}~\cite{ArmstrongGOS22} & 13 & machine, transportation, flow-shop, scheduling, job, re-scheduling, make-span, order, completion-time, resource, task, cmax & HFF, parallel machine & noOverlap, cumulative & Prolog & OZ, OPL, SICStus &  &  & real-world, benchmark & 0 & \\
\href{papers/AronssonBK09.pdf}{AronssonBK09}~\cite{AronssonBK09} & 13 & job-shop, transportation, order, job, task &  & cumulative & Prolog & Cplex, CHIP & railway &  & real-world, real-life & 0 & sweep\\
\href{papers/ArtiguesBF04.pdf}{ArtiguesBF04}~\cite{ArtiguesBF04} & 13 & job, batch process, cmax, make-span, release-date, resource, precedence, completion-time, sequence dependent setup, job-shop, setup-time, preempt, scheduling, order, machine &  & disjunctive & C++ & Ilog Scheduler, Ilog Solver &  &  & benchmark & 0 & edge-finding\\
\href{papers/ArtiouchineB05.pdf}{ArtiouchineB05}~\cite{ArtiouchineB05} & 15 & re-scheduling, release-date, scheduling, order, completion-time, job, resource, make-span, activity, preempt, open-shop, machine, precedence, job-shop & parallel machine, single machine & disjunctive, cumulative &  & Ilog Scheduler & aircraft &  & generated instance, random instance & 0 & not-last, edge-finding, not-first\\
\href{papers/Astrand0F21.pdf}{Astrand0F21}~\cite{Astrand0F21} & 18 & resource, open-shop, task, machine, precedence, job-shop, make-span, order, job, activity, scheduling &  & cycle, disjunctive &  & Gecode & farming, forestry, robot, satellite, agriculture & potash industry, mining industry, mineral industry & benchmark, real-world, real-life, generated instance & 0 & \\
\href{papers/AstrandJZ18.pdf}{AstrandJZ18}~\cite{AstrandJZ18} & 9 & resource, task, machine, make-span, order, activity, scheduling & single machine & disjunctive, cumulative, cycle &  & Gecode & hoist, robot & potash industry &  & 0 & time-tabling\\
\href{papers/BadicaBIL19.pdf}{BadicaBIL19}~\cite{BadicaBIL19} & 11 & completion-time, resource, order, activity, machine, multi-agent, distributed, make-span, scheduling &  & cycle &  & ECLiPSe, Gecode &  &  & github & 0 & \\
\href{papers/Baptiste09.pdf}{Baptiste09}~\cite{Baptiste09} & 1 & scheduling &  &  &  &  &  &  &  & 0 & \\
\href{papers/BaptisteLV92.pdf}{BaptisteLV92}~\cite{BaptisteLV92} & 6 &  &  &  &  &  &  &  &  & 0 & \\
\href{papers/BaptisteP97.pdf}{BaptisteP97}~\cite{BaptisteP97} & 15 & resource, task, preempt, precedence, release-date, flow-shop, job-shop, scheduling, re-scheduling, make-span, order, job, activity, due-date & RCPSP & disjunctive, cumulative & C++ & Claire, CHIP &  &  & benchmark & 0 & edge-finding, edge-finder\\
\href{papers/BarlattCG08.pdf}{BarlattCG08}~\cite{BarlattCG08} & 5 & scheduling, resource, setup-time, job, task, machine, flow-shop, job-shop, transportation &  &  &  &  & automotive, pipeline &  & real-world & 1 & \\
\href{papers/Bartak02.pdf}{Bartak02}~\cite{Bartak02} & 16 & make-span, scheduling, machine, continuous-process, job, resource, activity, lateness, job-shop, task, precedence, earliness, order &  & disjunctive, cumulative & Prolog & SICStus, OZ & dairies &  & real-life & 0 & edge-finding, time-tabling\\
\href{papers/Bartak02a.pdf}{Bartak02a}~\cite{Bartak02a} & 15 & activity, re-scheduling, earliness, job-shop, resource, scheduling, make-span, task, precedence, order, machine, tardiness, job &  & cumulative, disjunctive &  & Ilog Scheduler & dairies &  & benchmark, real-life & 0 & time-tabling, edge-finding\\
\href{papers/BartoliniBBLM14.pdf}{BartoliniBBLM14}~\cite{BartoliniBBLM14} & 16 & resource, tardiness, task, job, activity, make-span, machine, scheduling &  & alternative constraint, cumulative &  &  & super-computer &  &  & 4 & \\
\href{papers/BarzegaranZP20.pdf}{BarzegaranZP20}~\cite{BarzegaranZP20} & 9 & re-scheduling, resource, distributed, machine, task, scheduling, order &  &  & Java & OR-Tools & automotive, robot &  &  & 5 & \\
\href{papers/BeckDF97.pdf}{BeckDF97}~\cite{BeckDF97} & 15 & precedence, release-date, due-date, re-scheduling, make-span, order, scheduling, resource, inventory, machine, job, job-shop, task, activity & single machine & cycle, cumulative &  &  & robot &  & benchmark, real-world & 0 & edge-finding\\
\href{papers/BehrensLM19.pdf}{BehrensLM19}~\cite{BehrensLM19} & 7 & order, setup-time, resource, task, machine, distributed, multi-agent, scheduling, make-span &  &  & Python & OR-Tools, MiniZinc, OZ & robot &  & real-world, github & 0 & \\
\href{papers/BeldiceanuC02.pdf}{BeldiceanuC02}~\cite{BeldiceanuC02} & 17 & order, producer/consumer, scheduling, machine, task, resource, activity & single machine & cumulative & Prolog & SICStus, CHIP, OZ & crew-scheduling &  & real-life, random instance, benchmark & 0 & sweep\\
\href{papers/BeldiceanuCP08.pdf}{BeldiceanuCP08}~\cite{BeldiceanuCP08} & 15 & resource, task, scheduling, order &  & geost, cumulative, disjunctive & Prolog & SICStus, CHIP, OPL & rectangle-packing, perfect-square &  & benchmark & 0 & edge-finding, sweep\\
\href{papers/BeldiceanuP07.pdf}{BeldiceanuP07}~\cite{BeldiceanuP07} & 15 & preempt, scheduling, release-date, task, resource, order, due-date &  & cumulative, disjunctive &  &  &  &  &  & 0 & sweep\\
\href{papers/BenderWS21.pdf}{BenderWS21}~\cite{BenderWS21} & 16 & preempt, activity, task, order, machine, make-span, job, distributed, resource, setup-time, scheduling & RCPSP & noOverlap & Python &  & agriculture &  &  & 9 & \\
\href{papers/BenediktSMVH18.pdf}{BenediktSMVH18}~\cite{BenediktSMVH18} & 10 & job-shop, scheduling, order, job, preempt, resource, machine & single machine, parallel machine & noOverlap &  & OZ, Gurobi & energy-price &  & github, random instance, generated instance & 1 & \\
\href{papers/BeniniBGM06.pdf}{BeniniBGM06}~\cite{BeniniBGM06} & 15 & activity, task, distributed, tardiness, precedence, scheduling, make-span, resource, order, setup-time &  & cycle, cumulative &  & ECLiPSe, Cplex, Ilog Solver, OZ & automotive, pipeline &  & real-life & 0 & \\
\href{papers/BertholdHLMS10.pdf}{BertholdHLMS10}~\cite{BertholdHLMS10} & 5 & precedence, scheduling, order, completion-time, job, resource, preempt & psplib, RCPSP & disjunctive, cumulative &  & Cplex, Z3 &  &  &  & 1 & \\
\href{papers/BessiereHMQW14.pdf}{BessiereHMQW14}~\cite{BessiereHMQW14} & 16 & scheduling, order, job, resource, setup-time, task, machine &  & alldifferent, cycle &  & Choco Solver & satellite & textile industry & benchmark, real-life & 0 & \\
\href{papers/BillautHL12.pdf}{BillautHL12}~\cite{BillautHL12} & 15 & tardiness, precedence, release-date, flow-shop, job-shop, make-span, order, setup-time, job, scheduling, completion-time, due-date, resource, open-shop, machine, cmax & single machine & cycle &  & Mistral, Cplex &  &  & random instance & 0 & \\
\href{papers/Bit-Monnot23.pdf}{Bit-Monnot23}~\cite{Bit-Monnot23} & 8 & precedence, scheduling, machine, distributed, order, job, make-span, open-shop, task, lazy clause generation, job-shop, resource, activity & Open Shop Scheduling Problem, OSP & cycle, cumulative, disjunctive &  & OR-Tools, MiniZinc, CPO, Mistral &  &  & real-world, github, benchmark & 1 & \\
\href{papers/BofillCSV17.pdf}{BofillCSV17}~\cite{BofillCSV17} & 9 & machine, preempt, cmax, lazy clause generation, precedence, scheduling, make-span, resource, order, activity & RCPSP, psplib & cumulative &  & Z3 &  &  & benchmark & 2 & energetic reasoning\\
\href{papers/BofillEGPSV14.pdf}{BofillEGPSV14}~\cite{BofillEGPSV14} & 16 & order, scheduling, lazy clause generation, machine, task &  &  &  & Cplex, Gecode, MiniZinc &  &  & industrial instance & 6 & time-tabling\\
\href{papers/BofillGSV15.pdf}{BofillGSV15}~\cite{BofillGSV15} & 9 & machine, scheduling, order &  &  &  & Cplex &  &  & industrial instance & 3 & time-tabling\\
\href{papers/BogaerdtW19.pdf}{BogaerdtW19}~\cite{BogaerdtW19} & 16 & scheduling, completion-time, order, setup-time, job, machine, job-shop, tardiness, precedence & single machine, parallel machine & noOverlap & C  & OPL, Cplex & railway &  & benchmark & 4 & \\
\href{papers/BonfiettiLBM11.pdf}{BonfiettiLBM11}~\cite{BonfiettiLBM11} & 15 & scheduling, order, job, resource, make-span, activity, machine, precedence, task, job-shop & RCPSP & cumulative, cycle &  & Ilog Solver & hoist, robot &  & generated instance, industrial instance, benchmark & 3 & \\
\href{papers/BonfiettiLBM12.pdf}{BonfiettiLBM12}~\cite{BonfiettiLBM12} & 16 & scheduling, order, job, resource, make-span, activity, distributed, machine, precedence, job-shop & RCPSP & cumulative, cycle &  & Ilog Solver & hoist, robot &  & benchmark & 3 & time-tabling\\
\href{papers/BonfiettiLM13.pdf}{BonfiettiLM13}~\cite{BonfiettiLM13} & 5 & make-span, job-shop, precedence, resource, activity, job, order, scheduling & RCPSP & cumulative, cycle &  & Cplex &  &  &  & 0 & \\
\href{papers/BonfiettiLM14.pdf}{BonfiettiLM14}~\cite{BonfiettiLM14} & 16 & make-span, machine, task, job-shop, precedence, open-shop, resource, activity, job, distributed, order, scheduling & RCPSP, psplib & cumulative &  &  &  &  & real-world, benchmark & 2 & \\
\href{papers/BonfiettiM12.pdf}{BonfiettiM12}~\cite{BonfiettiM12} & 3 & job, task, precedence, job-shop, resource, activity, scheduling, machine & RCPSP & cumulative &  &  & hoist &  & industrial instance & 0 & \\
\href{papers/BonfiettiZLM16.pdf}{BonfiettiZLM16}~\cite{BonfiettiZLM16} & 17 & resource, make-span, activity, precedence, scheduling, order & RCPSP & cumulative, cycle, disjunctive &  & OR-Tools & automotive & automotive industry, control system industry & generated instance, github, industrial instance, benchmark, real-world & 1 & edge-finder, sweep\\
\href{papers/BoothNB16.pdf}{BoothNB16}~\cite{BoothNB16} & 17 & distributed, resource, scheduling, task, machine, precedence, order, activity, re-scheduling &  & disjunctive, cumulative, noOverlap & C++ & Cplex & robot, medical &  & real-world & 0 & \\
\href{papers/BoudreaultSLQ22.pdf}{BoudreaultSLQ22}~\cite{BoudreaultSLQ22} & 16 & lazy clause generation, order, activity, make-span, machine, scheduling, cmax, transportation, distributed, resource, preempt, precedence, task & RCPSP, psplib & disjunctive, cumulative &  & Chuffed, MiniZinc, OR-Tools, OPL & offshore & ship repair industry & benchmark, generated instance, supplementary material, gitlab, real-life, industrial partner, github, real-world & 9 & not-last, energetic reasoning, edge-finding, not-first\\
\href{papers/BridiLBBM16.pdf}{BridiLBBM16}~\cite{BridiLBBM16} & 2 & resource, task, machine, distributed, make-span, order, job, activity, scheduling &  &  &  &  &  &  &  & 0 & \\
\href{papers/BrusoniCLMMT96.pdf}{BrusoniCLMMT96}~\cite{BrusoniCLMMT96} & 10 & resource, activity, precedence, task, distributed, due-date, job-shop, scheduling, order, job &  & disjunctive & Prolog &  & railway &  &  & 0 & \\
\href{papers/BurtLPS15.pdf}{BurtLPS15}~\cite{BurtLPS15} & 17 & task, machine, precedence, order, tardiness, job, job-shop, resource, scheduling, make-span, completion-time & parallel machine, single machine & cumulative, cycle &  & Cplex, Gurobi, Gecode, MiniZinc &  &  & real-world, benchmark, industry partner & 5 & \\
\href{papers/CappartS17.pdf}{CappartS17}~\cite{CappartS17} & 16 & machine, activity, job, precedence, re-scheduling, resource, job-shop, scheduling, task, order, completion-time & TMS & cumulative, noOverlap, alternative constraint, span constraint &  & OPL, OZ & railway &  & bitbucket, random instance, real-life & 1 & \\
\href{papers/CarchraeBF05.pdf}{CarchraeBF05}~\cite{CarchraeBF05} & 1 & scheduling, order, task, make-span &  &  &  &  &  &  &  & 0 & \\
\href{papers/Caseau97.pdf}{Caseau97}~\cite{Caseau97} & 4 & preempt, make-span, order, scheduling, job, resource, job-shop, task &  & cumulative &  &  & robot &  & benchmark & 0 & edge-finding\\
\href{papers/CauwelaertDMS16.pdf}{CauwelaertDMS16}~\cite{CauwelaertDMS16} & 16 & batch process, task, job, job-shop, order, activity, make-span, machine, scheduling, completion-time, setup-time, resource, sequence dependent setup, preempt, precedence &  & cumulative, disjunctive & Java &  & container terminal &  & real-life, bitbucket, benchmark & 2 & not-last, edge-finding, not-first\\
\href{papers/CestaOS98.pdf}{CestaOS98}~\cite{CestaOS98} & 1 & resource, scheduling, job &  &  &  &  & robot &  &  & 0 & \\
\href{papers/ChapadosJR11.pdf}{ChapadosJR11}~\cite{ChapadosJR11} & 6 & activity, scheduling, order, task &  & cycle, cumulative &  & OPL &  & retail industry &  & 0 & time-tabling\\
\href{papers/ChuX05.pdf}{ChuX05}~\cite{ChuX05} & 15 & scheduling, machine, resource, job, release-date, order, due-date, completion-time & single machine & disjunctive, cumulative &  & ECLiPSe &  &  &  & 0 & \\
\href{papers/CireCH13.pdf}{CireCH13}~\cite{CireCH13} & 7 & make-span, tardiness, scheduling, machine, job, resource, precedence, task, order &  & circuit, cumulative &  & OPL, Cplex, OZ &  &  &  & 1 & \\
\href{papers/ClercqPBJ11.pdf}{ClercqPBJ11}~\cite{ClercqPBJ11} & 16 & resource, order, activity, due-date, release-date, distributed, precedence, scheduling, completion-time &  & alldifferent, cumulative & Java & CHIP, Choco Solver &  &  & benchmark & 1 & time-tabling, sweep, energetic reasoning, edge-finding\\
\href{papers/CobanH10.pdf}{CobanH10}~\cite{CobanH10} & 5 & distributed, tardiness, job, preempt, re-scheduling, make-span, order, scheduling &  & circuit, disjunctive &  & OPL, Cplex &  &  &  & 0 & \\
\href{papers/ColT19.pdf}{ColT19}~\cite{ColT19} & 17 & earliness, order, scheduling, precedence, make-span, machine, resource, job, job-shop & JSSP & noOverlap, disjunctive & Java & MiniZinc, CPO, OR-Tools &  &  & github, benchmark, real-world & 2 & \\
\href{papers/Colombani96.pdf}{Colombani96}~\cite{Colombani96} & 15 & job, scheduling, resource, order, task, preempt, activity, due-date, machine, precedence, release-date, job-shop &  & disjunctive &  & CHIP &  &  &  & 0 & \\
\href{papers/DannaP03.pdf}{DannaP03}~\cite{DannaP03} & 5 & machine, job, job-shop, activity, earliness, order, tardiness, scheduling, resource &  & disjunctive &  & Cplex, Ilog Solver, Ilog Scheduler &  &  & benchmark & 0 & \\
\href{papers/Davenport10.pdf}{Davenport10}~\cite{Davenport10} & 5 & resource, release-date, tardiness, scheduling, completion-time, order, earliness, due-date &  &  &  & Cplex & semiconductor &  &  & 0 & \\
\href{papers/DavenportKRSH07.pdf}{DavenportKRSH07}~\cite{DavenportKRSH07} & 13 & make to order, activity, machine, sequence dependent setup, preempt, precedence, resource, inventory, job-shop, order, scheduling, job, setup-time &  & disjunctive, bin-packing & C++ & Cplex, CHIP &  & steel industry &  & 0 & \\
\href{papers/DejemeppeCS15.pdf}{DejemeppeCS15}~\cite{DejemeppeCS15} & 16 & completion-time, tardiness, job-shop, scheduling, sequence dependent setup, make-span, machine, release-date, task, precedence, setup-time, job, resource, order, preempt, activity & single machine & disjunctive, cumulative, cycle &  &  & container terminal &  & real-world, bitbucket, generated instance, benchmark & 4 & not-last, not-first, edge-finding\\
\href{papers/DejemeppeD14.pdf}{DejemeppeD14}~\cite{DejemeppeD14} & 9 & make-span, precedence, job-shop, resource, activity, setup-time, scheduling, order, job &  & cumulative &  &  & medical, patient &  & bitbucket & 0 & \\
\href{papers/DemirovicS18.pdf}{DemirovicS18}~\cite{DemirovicS18} & 18 & scheduling, order, task, resource, activity, precedence &  & cumulative, disjunctive &  & MiniZinc, Gurobi, OZ &  &  & real-world, benchmark & 5 & time-tabling\\
\href{papers/DerrienP14.pdf}{DerrienP14}~\cite{DerrienP14} & 9 & resource, scheduling, activity, order, make-span & psplib, CuSP & cumulative & Java & Choco Solver &  &  & random instance & 0 & sweep, edge-finding, energetic reasoning\\
\href{papers/DerrienPZ14.pdf}{DerrienPZ14}~\cite{DerrienPZ14} & 9 & re-scheduling, make-span, scheduling, resource, order, job, activity, machine, precedence & RCPSP, CuSP & cumulative &  & Choco Solver, CHIP &  &  & benchmark, random instance, real-world & 0 & sweep\\
\href{papers/DilkinaDH05.pdf}{DilkinaDH05}~\cite{DilkinaDH05} & 5 & machine, precedence, job-shop, make-span, job, scheduling, order &  &  &  & OPL &  &  &  & 0 & \\
\href{papers/DoomsH08.pdf}{DoomsH08}~\cite{DoomsH08} & 16 & scheduling, resource, completion-time, machine, job, job-shop, activity, task, order & RCPSP &  &  &  &  & services industry &  & 0 & \\
\href{papers/DoulabiRP14.pdf}{DoulabiRP14}~\cite{DoulabiRP14} & 9 & activity, scheduling, due-date, resource, task, order &  & bin-packing &  & Cplex & nurse, medical, patient &  &  & 0 & \\
\href{papers/EdisO11.pdf}{EdisO11}~\cite{EdisO11} & 7 & task, job, completion-time, activity, lateness, earliness, resource, make-span, scheduling, flow-time, preempt, tardiness, due-date, machine & parallel machine & bin-packing, noOverlap, cumulative &  & OPL, OZ, Cplex &  &  &  & 0 & \\
\href{papers/EfthymiouY23.pdf}{EfthymiouY23}~\cite{EfthymiouY23} & 16 & order, job, make-span, re-scheduling, task, job-shop, scheduling, machine, setup-time & CHSP, JSSP & cumulative, disjunctive, cycle & Python & OPL, OR-Tools & pipeline, hoist, electroplating, satellite &  & benchmark, random instance, generated instance, real-life, industrial instance & 3 & \\
\href{papers/ElkhyariGJ02.pdf}{ElkhyariGJ02}~\cite{ElkhyariGJ02} & 6 & resource, activity, precedence, scheduling, machine, due-date, preempt, make-span, re-scheduling, task & RCPSP & cumulative, disjunctive, table constraint &  &  &  &  &  & 0 & \\
\href{papers/ElkhyariGJ02a.pdf}{ElkhyariGJ02a}~\cite{ElkhyariGJ02a} & 24 & activity, re-scheduling, order, due-date, scheduling, task, precedence, open-shop, resource & RCPSP, psplib & cumulative, disjunctive &  & OZ, OPL &  &  & benchmark, real-life & 0 & time-tabling\\
\href{papers/ErtlK91.pdf}{ErtlK91}~\cite{ErtlK91} & 12 & setup-time, resource, scheduling, order, machine, task &  & cycle & Prolog &  & pipeline &  & real-world, benchmark & 0 & \\
\href{papers/EvenSH15.pdf}{EvenSH15}~\cite{EvenSH15} & 18 & preempt, transportation, order, scheduling, machine, distributed, resource, completion-time, task &  & disjunctive, cumulative &  & OPL, Choco Solver &  &  & real-life, real-world & 0 & sweep\\
\href{papers/FontaineMH16.pdf}{FontaineMH16}~\cite{FontaineMH16} & 11 & order, machine, job, task, completion-time, make-span, job-shop, resource, precedence, scheduling & parallel machine & disjunctive &  & MiniZinc, Gurobi, CHIP &  &  & benchmark & 2 & \\
\href{papers/FortinZDF05.pdf}{FortinZDF05}~\cite{FortinZDF05} & 15 & resource, order, task, activity, temporal constraint reasoning, precedence, make-span, scheduling & psplib &  &  &  &  &  &  & 0 & \\
\href{papers/FrankK05.pdf}{FrankK05}~\cite{FrankK05} & 18 & order, scheduling, job, resource, due-date, task, precedence &  & cycle &  &  & satellite, aircraft &  & benchmark & 0 & \\
\href{papers/FrimodigS19.pdf}{FrimodigS19}~\cite{FrimodigS19} & 17 & resource, order, task, machine, job-shop, job, scheduling &  & regular expression, cumulative, bin-packing & Python & Gecode, Cplex, MiniZinc, OZ & radiation therapy, medical, patient, nurse, physician &  & benchmark, real-world & 1 & \\
\href{papers/FrohnerTR19.pdf}{FrohnerTR19}~\cite{FrohnerTR19} & 9 & scheduling, order, distributed &  &  & Java, Python & MiniZinc, Gecode, Gurobi & nurse &  & benchmark, real-world & 0 & \\
\href{papers/FrostD98.pdf}{FrostD98}~\cite{FrostD98} & 1 & order, scheduling &  &  &  &  &  & power industry &  & 0 & \\
\href{papers/GalleguillosKSB19.pdf}{GalleguillosKSB19}~\cite{GalleguillosKSB19} & 18 & re-scheduling, machine, distributed, resource, order, activity, job, scheduling, make-span & JSSP & cumulative, alternative constraint & Python & OR-Tools, OZ & super-computer, datacenter &  &  & 5 & \\
\href{papers/GarganiR07.pdf}{GarganiR07}~\cite{GarganiR07} & 13 & order, machine, resource, inventory &  & bin-packing & C++ & OPL & steel mill & steel industry & real-life, CSPlib & 0 & \\
\href{papers/GayHLS15.pdf}{GayHLS15}~\cite{GayHLS15} & 9 & precedence, task, order, make-span, resource, scheduling, activity & OSP, psplib, RCPSP & cumulative, disjunctive &  &  &  &  & benchmark, bitbucket & 0 & edge-finding, time-tabling\\
\href{papers/GayHS15.pdf}{GayHS15}~\cite{GayHS15} & 9 & scheduling, precedence, resource, preempt, task, order &  & cumulative, table constraint, disjunctive &  & Choco Solver, OR-Tools, Gecode &  &  & bitbucket & 2 & time-tabling, sweep\\
\href{papers/GaySS14.pdf}{GaySS14}~\cite{GaySS14} & 15 & machine, job, completion-time, activity, order, setup-time, make-span, scheduling, precedence, manpower, continuous-process, resource, job-shop &  & cycle, cumulative, disjunctive &  &  & steel mill &  & real-life, CSPlib & 0 & sweep\\
\href{papers/GeibingerKKMMW21.pdf}{GeibingerKKMMW21}~\cite{GeibingerKKMMW21} & 10 & distributed, scheduling &  &  &  & MiniZinc, OR-Tools, Gurobi, Cplex, Gecode & nurse, physician, COVID, medical, patient & pharmaceutical industry & real-world & 3 & \\
\href{papers/GeibingerMM19.pdf}{GeibingerMM19}~\cite{GeibingerMM19} & 16 & precedence, release-date, resource, activity, re-scheduling, job, order, due-date, completion-time, scheduling, make-span, task & RCPSP & alternative constraint, noOverlap, cumulative, endBeforeStart & Java & CPO, Cplex, Gecode, MiniZinc & automotive &  & real-life, generated instance, industrial partner, real-world, benchmark & 3 & time-tabling\\
\href{papers/GeibingerMM21.pdf}{GeibingerMM21}~\cite{GeibingerMM21} & 9 & lazy clause generation, precedence, release-date, resource, activity, job, order, due-date, completion-time, tardiness, scheduling, machine, task & RCPSP & disjunctive, cumulative &  & CPO, Chuffed, Cplex & nurse &  & real-life, github, generated instance, real-world, benchmark & 0 & time-tabling\\
\href{papers/GeitzGSSW22.pdf}{GeitzGSSW22}~\cite{GeitzGSSW22} & 18 & make-span, order, setup-time, job, scheduling, completion-time, sequence dependent setup, resource, task, machine, preempt, producer/consumer, lateness, lazy clause generation, precedence, job-shop, batch process, transportation & single machine, RCPSP, JSSP & cumulative &  & OZ, OPL & robot &  & real-life, github, real-world & 8 & not-last, sweep\\
\href{papers/GelainPRVW17.pdf}{GelainPRVW17}~\cite{GelainPRVW17} & 16 & resource, scheduling, order &  &  &  &  &  &  & CSPlib, real-life, benchmark & 2 & \\
\href{papers/Geske05.pdf}{Geske05}~\cite{Geske05} & 18 & machine, task, re-scheduling, job, activity, order, distributed, resource, scheduling, lateness, job-shop &  & cumulative & Prolog & CHIP, SICStus & railway &  & real-life & 0 & \\
\href{papers/GilesH16.pdf}{GilesH16}~\cite{GilesH16} & 16 & inventory, setup-time, activity, task, transportation, order, scheduling, resource &  & cumulative, disjunctive &  & Cplex & pipeline & petro-chemical industry, chemical processing industry, chemical industry &  & 0 & \\
\href{papers/GingrasQ16.pdf}{GingrasQ16}~\cite{GingrasQ16} & 7 & resource, scheduling, task, order, make-span, completion-time, precedence & psplib, CuSP, RCPSP & disjunctive, cumulative &  & Choco Solver &  &  & benchmark & 0 & sweep, edge-finder, edge-finding, energetic reasoning\\
\href{papers/GodetLHS20.pdf}{GodetLHS20}~\cite{GodetLHS20} & 8 & lazy clause generation, setup-time, release-date, scheduling, task, order, machine, make-span, cmax, completion-time, resource, job & parallel machine, PMSP, single machine & alldifferent, bin-packing, cumulative, disjunctive &  & OZ, Choco Solver, CHIP, Chuffed & satellite &  & github, real-life, benchmark, generated instance & 0 & not-last, time-tabling\\
\href{papers/GoldwaserS17.pdf}{GoldwaserS17}~\cite{GoldwaserS17} & 16 & scheduling, machine, transportation, due-date, order, lazy clause generation, resource &  & cumulative, disjunctive & Python & Gurobi, Gecode & torpedo & steel industry & instance generator, github, generated instance & 4 & \\
\href{papers/Goltz95.pdf}{Goltz95}~\cite{Goltz95} & 14 & due-date, machine, task, job, completion-time, order, resource, scheduling, precedence, job-shop &  & cumulative, disjunctive & Prolog & CHIP &  &  & benchmark & 0 & edge-finding\\
\href{papers/GomesHS06.pdf}{GomesHS06}~\cite{GomesHS06} & 2 & scheduling, distributed, task, multi-agent, order &  &  &  & Ilog Solver &  &  & real-life & 0 & \\
\href{papers/GrimesH10.pdf}{GrimesH10}~\cite{GrimesH10} & 15 & cmax, machine, job, setup-time, job-shop, flow-shop, sequence dependent setup, open-shop, task, batch process, resource, scheduling, make-span, precedence, order & Open Shop Scheduling Problem & disjunctive, cumulative, cycle &  & OZ &  & steel industry & benchmark & 1 & time-tabling, edge-finding\\
\href{papers/GrimesH11.pdf}{GrimesH11}~\cite{GrimesH11} & 17 & cmax, completion-time, machine, tardiness, job, release-date, earliness, lazy clause generation, job-shop, flow-shop, open-shop, task, due-date, resource, scheduling, make-span, precedence, order & RCPSP & disjunctive, cumulative &  & Cplex, Ilog Scheduler, Ilog Solver, OZ, OPL &  &  & benchmark & 1 & edge-finding\\
\href{papers/GrimesHM09.pdf}{GrimesHM09}~\cite{GrimesHM09} & 9 & make-span, resource, job, precedence, open-shop, scheduling, task, order, job-shop, machine & Open Shop Scheduling Problem, OSP & disjunctive & Java & Choco Solver, Ilog Scheduler, Mistral &  &  & benchmark & 0 & not-last, edge-finding\\
\href{papers/GroleazNS20.pdf}{GroleazNS20}~\cite{GroleazNS20} & 17 & tardiness, precedence, release-date, job-shop, setup-time, job, scheduling, resource, order, machine, inventory, preempt, due-date & GCSP & noOverlap, cycle, cumulative, circuit &  & CPO, OR-Tools &  & food industry & benchmark, industrial instance & 0 & \\
\href{papers/GroleazNS20a.pdf}{GroleazNS20a}~\cite{GroleazNS20a} & 9 & scheduling, machine, inventory, transportation, due-date, distributed, order, tardiness, job, release-date, precedence, resource, setup-time, preempt & parallel machine, RCPSP & cycle, noOverlap, cumulative &  & Cplex, CPO &  & food industry & industrial partner, benchmark & 0 & \\
\href{papers/GruianK98.pdf}{GruianK98}~\cite{GruianK98} & 8 & task, resource, scheduling, order, activity, re-scheduling &  & cumulative, cycle, diffn, circuit &  & OPL, CHIP & pipeline, aircraft &  & benchmark & 0 & \\
\href{papers/GuSS13.pdf}{GuSS13}~\cite{GuSS13} & 7 & lazy clause generation, activity, order, distributed, scheduling, precedence, make-span, machine, resource & single machine & cumulative &  &  &  &  & benchmark & 1 & edge-finding, edge-finder, time-tabling\\
\href{papers/HanenKP21.pdf}{HanenKP21}~\cite{HanenKP21} & 17 & job-shop, resource, scheduling, make-span, completion-time, task, machine, precedence, order, cmax, tardiness, job, lateness, preempt, release-date, due-date & RCPSP, CuSP, parallel machine & cumulative & Python & Claire & pipeline &  & Roadef, generated instance, random instance & 1 & energetic reasoning\\
\href{papers/He0GLW18.pdf}{He0GLW18}~\cite{He0GLW18} & 18 & distributed, machine, precedence, re-scheduling, transportation, multi-agent, order, scheduling &  &  & Python & Gurobi & real-time pricing, energy-price &  & real-world, bitbucket & 8 & \\
\href{papers/HebrardTW05.pdf}{HebrardTW05}~\cite{HebrardTW05} & 1 & order, job, machine, job-shop, scheduling &  &  &  &  &  &  &  & 0 & \\
\href{papers/HechingH16.pdf}{HechingH16}~\cite{HechingH16} & 11 & re-scheduling, job, task, order, scheduling, manpower &  & circuit, noOverlap &  & OPL, Cplex, OZ & patient, medical &  & real-world & 0 & \\
\href{papers/HeinzB12.pdf}{HeinzB12}~\cite{HeinzB12} & 17 & activity, precedence, release-date, due-date, earliness, order, tardiness, scheduling, resource, completion-time, machine, job & single machine & cycle, cumulative, alternative constraint &  & Cplex, Ilog Solver, Ilog Scheduler, OPL &  &  &  & 0 & \\
\href{papers/HeinzKB13.pdf}{HeinzKB13}~\cite{HeinzKB13} & 16 & release-date, job-shop, resource, scheduling, order, machine, tardiness, job & single machine & cumulative &  & OPL, Cplex &  &  &  & 0 & \\
\href{papers/HeinzS11.pdf}{HeinzS11}~\cite{HeinzS11} & 10 & preempt, order, scheduling, resource, completion-time, machine, job & psplib, RCPSP & disjunctive, cumulative &  & Cplex &  &  & benchmark & 1 & energetic reasoning, time-tabling\\
\href{papers/HentenryckM04.pdf}{HentenryckM04}~\cite{HentenryckM04} & 16 & open-shop, resource, order, activity, job, due-date, completion-time, tardiness, scheduling, make-span, machine, task, job-shop, precedence &  & disjunctive, cycle, cumulative &  &  &  &  & benchmark & 0 & \\
\href{papers/HentenryckM08.pdf}{HentenryckM08}~\cite{HentenryckM08} & 5 & order &  & bin-packing &  &  & steel mill &  & CSPlib & 0 & \\
\href{papers/HermenierDL11.pdf}{HermenierDL11}~\cite{HermenierDL11} & 15 & precedence, distributed, resource, order, scheduling, completion-time, producer/consumer, machine, task &  & bin-packing, disjunctive, alldifferent, cumulative, cycle, table constraint &  & OZ, Choco Solver & datacenter &  &  & 1 & \\
\href{papers/HillTV21.pdf}{HillTV21}~\cite{HillTV21} & 19 & scheduling, machine, job, resource, activity, flow-shop, release-date, task, precedence, order, preempt, lazy clause generation, make-span & RCPSP, psplib, single machine & cycle, cumulative, alternative constraint &  &  &  &  & real-world & 0 & \\
\href{papers/HoYCLLCLC18.pdf}{HoYCLLCLC18}~\cite{HoYCLLCLC18} & 6 & resource, task, machine, distributed, re-scheduling, order, job, scheduling &  &  & C  &  & nurse, medical, patient &  & real-world & 0 & \\
\href{papers/HoeveGSL07.pdf}{HoeveGSL07}~\cite{HoeveGSL07} & 6 & re-scheduling, job, precedence, distributed, resource, task, job-shop, multi-agent, scheduling, machine, order &  & disjunctive &  & Ilog Scheduler, Cplex &  &  & benchmark & 0 & edge-finding\\
\href{papers/Hooker04.pdf}{Hooker04}~\cite{Hooker04} & 12 & machine, task, precedence, release-date, make-span, order, tardiness, scheduling, distributed, resource &  & cumulative, circuit, disjunctive &  & Cplex, OPL, Ilog Scheduler &  &  & random instance & 0 & \\
\href{papers/Hooker17.pdf}{Hooker17}~\cite{Hooker17} & 14 & job, due-date, order, tardiness, scheduling, resource &  & circuit &  & OZ &  &  & benchmark, random instance & 0 & \\
\href{papers/HookerY02.pdf}{HookerY02}~\cite{HookerY02} & 5 & resource, scheduling, order, machine, job & RCPSP & disjunctive, cumulative &  &  &  &  &  & 0 & \\
\href{papers/HoundjiSWD14.pdf}{HoundjiSWD14}~\cite{HoundjiSWD14} & 16 & precedence, resource, scheduling, machine, inventory, transportation, due-date, order & single machine & circuit &  &  &  &  & bitbucket, generated instance & 0 & \\
\href{papers/IfrimOS12.pdf}{IfrimOS12}~\cite{IfrimOS12} & 16 & task, order, machine, job, re-scheduling, distributed, due-date, resource, scheduling &  & disjunctive &  &  & datacenter, energy-price &  & real-life & 1 & \\
\href{papers/JuvinHHL23.pdf}{JuvinHHL23}~\cite{JuvinHHL23} & 16 & cmax, resource, job, setup-time, scheduling, task, order, job-shop, due-date, machine, preempt, make-span, flow-shop, completion-time, precedence & JSSP, parallel machine & endBeforeStart, disjunctive, alldifferent, cumulative, noOverlap & C++ & CPO, Mistral &  &  & supplementary material, github, benchmark & 6 & not-last, edge-finding, not-first\\
\href{papers/JuvinHL23.pdf}{JuvinHL23}~\cite{JuvinHL23} & 16 & make-span, completion-time, task, precedence, order, cmax, machine, tardiness, job, setup-time, job-shop, flow-shop, scheduling &  & noOverlap, endBeforeStart &  & Cplex, CPO &  &  & real-world & 0 & \\
\href{papers/KamarainenS02.pdf}{KamarainenS02}~\cite{KamarainenS02} & 17 & machine, job-shop, resource, precedence, transportation, earliness, activity, job, order, preempt, scheduling & KRFP &  &  & ECLiPSe &  &  & real-world, benchmark & 2 & \\
\href{papers/KameugneFGOQ18.pdf}{KameugneFGOQ18}~\cite{KameugneFGOQ18} & 17 & resource, task, cmax, precedence, make-span, scheduling, order, completion-time & RCPSP, CuSP & cumulative, disjunctive & Java & CHIP, Choco Solver &  &  & benchmark, real-world & 0 & time-tabling, not-first, sweep, not-last, energetic reasoning\\
\href{papers/KameugneFND23.pdf}{KameugneFND23}~\cite{KameugneFND23} & 17 & machine, resource, precedence, cmax, order, preempt, scheduling, make-span, completion-time, task, lazy clause generation & psplib, CuSP, RCPSP & disjunctive, cumulative & Java & CHIP, Choco Solver &  &  & benchmark & 5 & sweep, energetic reasoning, edge-finding, not-last, not-first, edge-finder, time-tabling\\
\href{papers/KameugneFSN11.pdf}{KameugneFSN11}~\cite{KameugneFSN11} & 15 & job-shop, release-date, resource, precedence, job, order, preempt, scheduling, make-span, completion-time, task & RCPSP, psplib, CuSP & disjunctive, cumulative &  & Gecode &  &  & benchmark & 1 & edge-finding, not-last, not-first, time-tabling\\
\href{papers/KelarevaTK13.pdf}{KelarevaTK13}~\cite{KelarevaTK13} & 17 & order, tardiness, make-span, re-scheduling, task, resource, lazy clause generation, activity, precedence, scheduling, inventory, transportation, setup-time & Liner Shipping Fleet Repositioning Problem, BPCTOP, LSFRP, Bulk Port Cargo Throughput Optimisation Problem & alldifferent &  & Cplex, MiniZinc, OZ & earth observation, satellite &  & real-world & 5 & \\
\href{papers/KeriK07.pdf}{KeriK07}~\cite{KeriK07} & 14 & due-date, tardiness, temporal constraint reasoning, job, activity, order, earliness, make-span, scheduling, precedence, cmax, resource, job-shop & RCPSP & cycle & C++ &  &  &  &  & 2 & edge-finding\\
\href{papers/KhemmoudjPB06.pdf}{KhemmoudjPB06}~\cite{KhemmoudjPB06} & 13 & resource, stock level, distributed, order, scheduling &  & cycle, cumulative & C++ & CHIP &  &  & real-world & 0 & \\
\href{papers/KimCMLLP23.pdf}{KimCMLLP23}~\cite{KimCMLLP23} & 16 & make-span, job, precedence, open-shop, distributed, tardiness, setup-time, earliness, job-shop, due-date, scheduling, order, transportation, machine & parallel machine, SCC & noOverlap & Python & Gurobi, OR-Tools &  & steel industry & real-world, benchmark, zenodo & 0 & \\
\href{papers/KlankeBYE21.pdf}{KlankeBYE21}~\cite{KlankeBYE21} & 16 & re-scheduling, make-span, order, job, activity, scheduling, completion-time, due-date, resource, task, machine, producer/consumer, job-shop, batch process &  & noOverlap, disjunctive, cumulative, circuit & Python & Gurobi, Cplex, CHIP, OR-Tools &  & food-processing industry & benchmark, random instance, real-life & 0 & \\
\href{papers/KletzanderM17.pdf}{KletzanderM17}~\cite{KletzanderM17} & 15 & scheduling, machine, resource, transportation, order & parallel machine &  &  & OZ & torpedo & steel industry &  & 2 & \\
\href{papers/KorbaaYG99.pdf}{KorbaaYG99}~\cite{KorbaaYG99} & 8 & job, resource, task, job-shop, scheduling, machine, flow-shop, order, transportation, make-span &  & cycle, circuit & Prolog & CHIP, Ilog Solver, OZ & robot, hoist &  &  & 0 & \\
\href{papers/KoschB14.pdf}{KoschB14}~\cite{KoschB14} & 16 & resource, completion-time, batch process, lateness, job-shop, release-date, due-date, multi-agent, order, cmax, make-span, scheduling, machine, distributed, job & single machine, RCPSP & cumulative, bin-packing, disjunctive & Java & Choco Solver, Cplex, OZ & semiconductor &  & benchmark & 0 & \\
\href{papers/KovacsEKV05.pdf}{KovacsEKV05}~\cite{KovacsEKV05} & 1 & scheduling, resource, setup-time, job, job-shop, precedence &  &  &  &  &  &  & real-life & 0 & \\
\href{papers/KovacsTKSG21.pdf}{KovacsTKSG21}~\cite{KovacsTKSG21} & 17 & resource, precedence, job-shop, due-date, preempt, scheduling, order, machine, tardiness, flow-shop, job, inventory, re-scheduling, task, distributed, release-date & RCPSP, single machine & cumulative &  & Gurobi, OR-Tools, Cplex &  &  & github, supplementary material, real-world, benchmark & 2 & \\
\href{papers/KovacsV04.pdf}{KovacsV04}~\cite{KovacsV04} & 15 & job, job-shop, resource, scheduling, make-span, task, machine, precedence, order & single machine & disjunctive, cumulative &  & Ilog Scheduler &  &  & industrial partner, benchmark, real-life & 0 & edge-finding\\
\href{papers/KovacsV06.pdf}{KovacsV06}~\cite{KovacsV06} & 13 & tardiness, job, setup-time, earliness, job-shop, resource, scheduling, make-span, task, machine, precedence, order & RCPSP, single machine & cumulative &  & Ilog Scheduler & automotive &  & industrial partner, benchmark, generated instance & 0 & \\
\href{papers/KreterSS15.pdf}{KreterSS15}~\cite{KreterSS15} & 17 & scheduling, task, order, machine, preempt, activity, make-span, completion-time, resource, lazy clause generation & RCPSP, parallel machine & cumulative, diffn &  & Cplex, MiniZinc, CHIP, Chuffed &  &  & benchmark & 3 & \\
\href{papers/KrogtLPHJ07.pdf}{KrogtLPHJ07}~\cite{KrogtLPHJ07} & 13 & resource, order, job, inventory, activity, due-date, machine, job-shop, precedence, scheduling &  & circuit & Prolog & OPL & semiconductor, aircraft &  & real-world & 0 & \\
\href{papers/Kumar03.pdf}{Kumar03}~\cite{Kumar03} & 15 & activity, order, scheduling, producer/consumer, resource &  & cycle &  &  &  &  &  & 0 & bi-partite matching, max-flow\\
\href{papers/Laborie09.pdf}{Laborie09}~\cite{Laborie09} & 15 & task, precedence, order, machine, tardiness, job, activity, setup-time, release-date, inventory, earliness, sequence dependent setup, due-date, preempt, job-shop, resource, scheduling &  & noOverlap, endBeforeStart, alternative constraint, cumulative, disjunctive & C  & OPL, CPO, OZ & aircraft, satellite &  & real-world, benchmark & 2 & \\
\href{papers/Laborie18a.pdf}{Laborie18a}~\cite{Laborie18a} & 9 & resource, job, release-date, scheduling, task, due-date, machine, precedence &  & cumulative, alternative constraint &  & Ilog Scheduler, CPO, OPL &  &  & real-life, benchmark, real-world & 0 & energetic reasoning\\
\href{papers/LacknerMMWW21.pdf}{LacknerMMWW21}~\cite{LacknerMMWW21} & 18 & release-date, flow-shop, batch process, setup-time, job, order, due-date, tardiness, scheduling, make-span, machine, task, lateness, earliness & parallel machine, OSP, single machine & noOverlap, cumulative, endBeforeStart &  & Chuffed, Cplex, OPL, CPO, OZ, OR-Tools, MiniZinc, Gurobi & semiconductor, oven scheduling & electronics industry, steel industry, manufacturing industry & random instance, industrial partner, benchmark, instance generator, real-life, supplementary material & 3 & \\
\href{papers/LahimerLH11.pdf}{LahimerLH11}~\cite{LahimerLH11} & 14 & resource, task, machine, preempt, cmax, precedence, make-span, order, job, scheduling, completion-time & parallel machine, RCPSP & disjunctive & C++ & Ilog Scheduler &  &  & benchmark & 2 & energetic reasoning\\
\href{papers/LauLN08.pdf}{LauLN08}~\cite{LauLN08} & 5 & order, distributed, inventory, resource, scheduling, flow-shop, transportation, job-shop, machine, job &  &  &  &  &  &  & benchmark, real-world & 0 & \\
\href{papers/LetortBC12.pdf}{LetortBC12}~\cite{LetortBC12} & 16 & order, machine, make-span, precedence, resource, scheduling, task & psplib & cumulative, geost, bin-packing & Java, Prolog & Choco Solver, CHIP, SICStus & datacenter &  & Roadef, benchmark, random instance & 2 & sweep, edge-finding\\
\href{papers/LetortCB13.pdf}{LetortCB13}~\cite{LetortCB13} & 16 & machine, make-span, precedence, resource, scheduling, task, order & psplib, RCPSP & cumulative, disjunctive, bin-packing & Java, Prolog & Choco Solver, SICStus &  &  & Roadef, benchmark, random instance & 2 & energetic reasoning, sweep, edge-finding\\
\href{papers/LiFJZLL22.pdf}{LiFJZLL22}~\cite{LiFJZLL22} & 6 & task, machine, tardiness, job, buffer-capacity, flow-time, setup-time, distributed, job-shop, batch process, transportation, flow-shop, scheduling, make-span, order, completion-time & single machine &  &  & OZ, OPL & robot &  & benchmark & 0 & \\
\href{papers/LimBTBB15.pdf}{LimBTBB15}~\cite{LimBTBB15} & 15 & job-shop, scheduling, multi-agent, order, machine, tardiness, job, re-scheduling, earliness &  &  &  & OPL & HVAC &  & benchmark & 3 & time-tabling\\
\href{papers/LimHTB16.pdf}{LimHTB16}~\cite{LimHTB16} & 18 & machine, activity, re-scheduling, multi-agent, order, scheduling, distributed &  & cumulative &  & OPL & real-time pricing, HVAC, energy-price &  & real-world & 4 & \\
\href{papers/LimRX04.pdf}{LimRX04}~\cite{LimRX04} & 5 & scheduling, preempt, machine, job, completion-time, order, transportation &  &  &  & OZ & container terminal &  & generated instance & 0 & \\
\href{papers/Limtanyakul07.pdf}{Limtanyakul07}~\cite{Limtanyakul07} & 6 & make-span, task, machine, release-date, resource, precedence, job, order, scheduling, due-date &  & cumulative &  & OPL & robot &  & real-life & 0 & energetic reasoning\\
\href{papers/LiuCGM17.pdf}{LiuCGM17}~\cite{LiuCGM17} & 17 & transportation, order, cmax, scheduling, machine, task, activity &  &  & Python & OR-Tools, OPL, MiniZinc &  & tourism industry & github & 11 & \\
\href{papers/LiuJ06.pdf}{LiuJ06}~\cite{LiuJ06} & 5 & make-span, task, order, scheduling, resource &  & cycle, disjunctive &  &  &  &  &  & 0 & \\
\href{papers/LiuLH19.pdf}{LiuLH19}~\cite{LiuLH19} & 9 & order, resource, scheduling &  &  &  & Choco Solver, OZ &  &  & CSPlib, benchmark & 0 & time-tabling\\
\href{papers/LombardiBM15.pdf}{LombardiBM15}~\cite{LombardiBM15} & 16 & completion-time, job-shop, resource, activity, precedence, scheduling, machine, distributed, order, job, make-span, task & JSSP, RCPSP, psplib &  &  &  &  &  & benchmark, real-world & 0 & \\
\href{papers/LombardiBMB11.pdf}{LombardiBMB11}~\cite{LombardiBMB11} & 17 & resource, order, activity, completion-time, scheduling, make-span, machine, task, precedence & RCPSP & cycle, cumulative & C++ &  & hoist &  & benchmark, industrial instance, real-life & 0 & \\
\href{papers/LombardiM09.pdf}{LombardiM09}~\cite{LombardiM09} & 15 & precedence, completion-time, make-span, order, activity, scheduling, resource, task, preempt & RCPSP &  &  & Ilog Solver &  &  & real-world, instance generator & 1 & \\
\href{papers/LombardiM10.pdf}{LombardiM10}~\cite{LombardiM10} & 15 & precedence, completion-time, make-span, order, activity, scheduling, resource, task & RCPSP & disjunctive, cumulative &  & Ilog Solver &  &  & real-world, benchmark & 1 & \\
\href{papers/LombardiM13.pdf}{LombardiM13}~\cite{LombardiM13} & 2 & precedence, make-span, order, activity, scheduling, resource, task & RCPSP, psplib &  &  &  &  &  &  & 0 & \\
\href{papers/Madi-WambaB16.pdf}{Madi-WambaB16}~\cite{Madi-WambaB16} & 16 & precedence, job, order, scheduling, task, resource &  & cumulative & Java & Choco Solver, CHIP &  &  & real-world, benchmark, random instance, generated instance & 3 & \\
\href{papers/Madi-WambaLOBM17.pdf}{Madi-WambaLOBM17}~\cite{Madi-WambaLOBM17} & 8 & machine, task, activity, re-scheduling, job, precedence, distributed, scheduling, order, resource &  & bin-packing, cumulative & Prolog & SICStus & datacenter &  & real-world & 0 & sweep\\
\href{papers/MakMS10.pdf}{MakMS10}~\cite{MakMS10} & 5 & scheduling, due-date, order, machine, inventory, task, job, activity, transportation, precedence, resource &  & cycle &  &  &  &  &  & 0 & \\
\href{papers/MalapertN19.pdf}{MalapertN19}~\cite{MalapertN19} & 17 & make-span, scheduling, completion-time, sequence dependent setup, resource, order, setup-time, job, flow-time, task, machine, cmax & parallel machine, PMSP, PTC, single machine & noOverlap, alwaysIn, cumulative, alternative constraint &  & Cplex, CPO & semiconductor &  & generated instance, benchmark, industrial instance, Roadef & 3 & \\
\href{papers/MaraveliasG04.pdf}{MaraveliasG04}~\cite{MaraveliasG04} & 20 &  &  &  &  & OZ &  &  &  & 0 & \\
\href{papers/Mehdizadeh-Somarin23.pdf}{Mehdizadeh-Somarin23}~\cite{Mehdizadeh-Somarin23} & 14 & multi-agent, job-shop, completion-time, re-scheduling, tardiness, machine, scheduling, cmax, flow-shop, job, task, setup-time, precedence, order, make-span, preempt & parallel machine, JSSP, single machine &  & Python & Cplex, OZ & robot, COVID &  & random instance & 0 & \\
\href{papers/MelgarejoLS15.pdf}{MelgarejoLS15}~\cite{MelgarejoLS15} & 17 & tardiness, scheduling, machine, task, precedence, transportation, setup-time, resource, order, job & single machine & circuit, disjunctive, alldifferent, noOverlap, table constraint &  & OZ, Cplex &  &  & real-world, benchmark & 1 & \\
\href{papers/Mercier-AubinGQ20.pdf}{Mercier-AubinGQ20}~\cite{Mercier-AubinGQ20} & 13 & job, preempt, task, make-span, sequence dependent setup, setup-time, tardiness, precedence, resource, earliness, completion-time, machine, lazy clause generation, activity, job-shop, due-date, scheduling, order & RCPSP & cycle, circuit, cumulative, disjunctive & C++, Python & OPL, MiniZinc &  & textile industry, manufacturing industry & industrial instance, industrial partner & 1 & \\
\href{papers/MonetteDD07.pdf}{MonetteDD07}~\cite{MonetteDD07} & 14 & precedence, job-shop, make-span, job, scheduling, completion-time, resource, open-shop, order, preempt, no preempt, task, machine & Open Shop Scheduling Problem, OSP & disjunctive &  & Gecode &  &  & benchmark & 0 & not-last, not-first, edge-finding\\
\href{papers/MonetteDH09.pdf}{MonetteDH09}~\cite{MonetteDH09} & 8 & precedence, release-date, job-shop, tardiness, make-span, job, scheduling, completion-time, resource, order, preempt, activity, earliness, distributed, due-date, task, machine &  & cycle, disjunctive, cumulative &  &  &  &  & benchmark & 0 & not-last\\
\href{papers/MossigeGSMC17.pdf}{MossigeGSMC17}~\cite{MossigeGSMC17} & 18 & activity, job, distributed, order, completion-time, preempt, scheduling, make-span, machine, task, job-shop, resource, precedence & FJS, single machine, RCPSP & cumulative, cycle, disjunctive & Prolog & SICStus, CHIP & rectangle-packing, robot &  & industrial partner, real-world, benchmark, random instance, CSPlib, generated instance & 4 & \\
\href{papers/MouraSCL08.pdf}{MouraSCL08}~\cite{MouraSCL08} & 16 & scheduling, preempt, activity, order, transportation, inventory, precedence, distributed, resource &  & table constraint, disjunctive, cycle & C++ & Ilog Solver, OZ, Ilog Scheduler & pipeline &  &  & 0 & max-flow\\
\href{papers/MouraSCL08a.pdf}{MouraSCL08a}~\cite{MouraSCL08a} & 8 & transportation, re-scheduling, order, scheduling, due-date, resource, inventory, distributed &  & disjunctive, cumulative & C++ & Ilog Solver, Ilog Scheduler & pipeline &  & real-world, benchmark & 0 & \\
\href{papers/MurinR19.pdf}{MurinR19}~\cite{MurinR19} & 16 & job-shop, task, make-span, transportation, order, resource, scheduling, machine, setup-time, job, activity, completion-time, precedence & JSPT & noOverlap, alternative constraint, endBeforeStart &  & Cplex, OPL & patient, robot &  & real-life, benchmark, github & 3 & \\
\href{papers/MurphyMB15.pdf}{MurphyMB15}~\cite{MurphyMB15} & 17 & scheduling, task, order, machine, activity, re-scheduling, resource &  & cycle, circuit, cumulative, disjunctive & Java & Choco Solver &  &  & real-world & 3 & \\
\href{papers/Muscettola02.pdf}{Muscettola02}~\cite{Muscettola02} & 16 & job-shop, resource, activity, precedence, scheduling, order, job, cmax &  & cycle &  &  &  &  &  & 0 & edge-finding, max-flow\\
\href{papers/NattafM20.pdf}{NattafM20}~\cite{NattafM20} & 16 & setup-time, resource, scheduling, make-span, order, completion-time, machine, job, flow-time & single machine, PMSP, parallel machine, PTC & cumulative, noOverlap &  & CPO, Cplex & semiconductor &  & benchmark, industrial instance & 7 & \\
\href{papers/NishikawaSTT18.pdf}{NishikawaSTT18}~\cite{NishikawaSTT18} & 6 & make-span, order, resource, activity, task, distributed, precedence, scheduling &  & alternative constraint, endBeforeStart &  & Cplex, OZ & pipeline, robot &  & real-world, benchmark & 0 & \\
\href{papers/NishikawaSTT18a.pdf}{NishikawaSTT18a}~\cite{NishikawaSTT18a} & 6 & task, order, activity, make-span, scheduling, distributed, resource, precedence, re-scheduling &  & endBeforeStart, alternative constraint &  & OZ, Cplex & robot, nurse, pipeline &  & real-world, benchmark, real-life & 0 & \\
\href{papers/OddiPCC03.pdf}{OddiPCC03}~\cite{OddiPCC03} & 15 & preempt, distributed, resource, scheduling, precedence, order, completion-time, task, machine, activity & single machine & cycle & Java &  & satellite, earth observation &  & benchmark & 0 & \\
\href{papers/OuelletQ13.pdf}{OuelletQ13}~\cite{OuelletQ13} & 16 & scheduling, task, order, preempt, make-span, completion-time, precedence, resource & CuSP, RCPSP, psplib & cumulative, disjunctive &  & Choco Solver &  &  & benchmark & 1 & edge-finding, not-first, edge-finder, energetic reasoning, not-last, time-tabling, sweep\\
\href{papers/OuelletQ18.pdf}{OuelletQ18}~\cite{OuelletQ18} & 18 & scheduling, task, order, make-span, completion-time, precedence, resource & RCPSP, psplib & cumulative, disjunctive & Java & OZ, Choco Solver &  &  & benchmark, Roadef & 0 & edge-finding, not-first, energetic reasoning, not-last, time-tabling\\
\href{papers/OuelletQ22.pdf}{OuelletQ22}~\cite{OuelletQ22} & 17 & scheduling, task, order, preempt, activity, completion-time, resource, lazy clause generation &  & cumulative, disjunctive & Java & MiniZinc, Choco Solver & nurse &  & github, benchmark, random instance & 1 & edge-finding, not-first, energetic reasoning, not-last, time-tabling, sweep\\
\href{papers/OujanaAYB22.pdf}{OujanaAYB22}~\cite{OujanaAYB22} & 6 & distributed, due-date, tardiness, make to order, precedence, flow-shop, job-shop, batch process, buffer-capacity, make-span, setup-time, job, scheduling, completion-time, sequence dependent setup, resource, open-shop, order, task, machine, preempt & PMSP, parallel machine, FJS, HFF & span constraint, noOverlap, disjunctive &  & CPO, OPL & COVID, robot & food industry, steel industry & benchmark, industrial instance, real-world, real-life & 0 & \\
\href{papers/ParkUJR19.pdf}{ParkUJR19}~\cite{ParkUJR19} & 8 & task, machine, flow-time, order, cmax, tardiness, job, lateness, preempt, no preempt, distributed, due-date, job-shop, flow-shop, resource, scheduling, make-span, open-shop, completion-time & parallel machine, single machine & endBeforeStart, cycle, noOverlap &  &  &  &  & real-world & 0 & \\
\href{papers/PembertonG98.pdf}{PembertonG98}~\cite{PembertonG98} & 14 & job-shop, resource, activity, preempt, scheduling, machine, order, job, task &  & geost, cycle &  & Ilog Solver, OPL & satellite, robot &  &  & 0 & \\
\href{papers/PerezGSL23.pdf}{PerezGSL23}~\cite{PerezGSL23} & 7 & resource, inventory, scheduling, task, order, machine, activity, make-span, completion-time, transportation, re-scheduling &  & table constraint, cumulative &  & OPL & nurse, steel mill, container terminal &  & real-world, generated instance & 0 & \\
\href{papers/PesantRR15.pdf}{PesantRR15}~\cite{PesantRR15} & 16 & activity, transportation, lazy clause generation, scheduling, order &  & cumulative, table constraint &  & Gurobi, Gecode, Ilog Solver &  &  &  & 1 & \\
\href{papers/PoderB08.pdf}{PoderB08}~\cite{PoderB08} & 8 & resource, producer/consumer, release-date, task, activity, preempt, due-date, order, scheduling &  & cumulative &  & CHIP &  &  &  & 0 & sweep\\
\href{papers/PopovicCGNC22.pdf}{PopovicCGNC22}~\cite{PopovicCGNC22} & 15 & order, completion-time, scheduling, make-span, machine, task, resource, transportation, activity & TMS & cumulative, alwaysIn, noOverlap & C++, Prolog & Cplex, SICStus, CHIP, OZ & pipeline & electricity industry &  & 0 & \\
\href{papers/PovedaAA23.pdf}{PovedaAA23}~\cite{PovedaAA23} & 21 & make-span, resource, job, precedence, lazy clause generation, release-date, task, job-shop, scheduling, preempt, activity, order & RCPSP & cumulative, disjunctive & Python & Chuffed, Cplex, MiniZinc, CPO & automotive, aircraft &  & real-world, github, benchmark, industrial instance, real-life & 4 & \\
\href{papers/Pralet17.pdf}{Pralet17}~\cite{Pralet17} & 19 & setup-time, job, activity, precedence, job-shop, due-date, order, sequence dependent setup, make-span, resource, scheduling, machine & RCPSP, psplib, JSSP & cycle, cumulative, disjunctive &  & CPO, Cplex, CHIP & satellite &  & benchmark & 1 & \\
\href{papers/PraletLJ15.pdf}{PraletLJ15}~\cite{PraletLJ15} & 16 & order, job-shop, activity, make-span, precedence, resource, job, due-date, scheduling, tardiness, task & JSSP & alternative constraint, noOverlap, cycle &  & CPO, Cplex & earth observation, satellite &  &  & 0 & \\
\href{papers/Puget95.pdf}{Puget95}~\cite{Puget95} & 4 & resource, job-shop, task, job, activity, order, scheduling, transportation, manpower &  & disjunctive &  & OPL &  &  & benchmark & 0 & \\
\href{papers/QuSN06.pdf}{QuSN06}~\cite{QuSN06} & 4 & task, scheduling, distributed, resource, precedence &  & circuit & Prolog & SICStus &  &  &  & 0 & \\
\href{papers/QuirogaZH05.pdf}{QuirogaZH05}~\cite{QuirogaZH05} & 6 & release-date, tardiness, precedence, flow-shop, scheduling, completion-time, make-span, resource, order, inventory, activity, earliness, due-date, flow-time, task, machine &  &  &  & Ilog Solver, OPL, OZ, Ilog Scheduler, ECLiPSe & robot &  &  & 0 & \\
\href{papers/RendlPHPR12.pdf}{RendlPHPR12}~\cite{RendlPHPR12} & 17 & re-scheduling, job, scheduling, order, machine, transportation &  &  & Java & OZ & medical, patient, nurse &  & real-world, CSPlib, benchmark & 2 & \\
\href{papers/RiahiNS018.pdf}{RiahiNS018}~\cite{RiahiNS018} & 9 & flow-shop, completion-time, job, scheduling, distributed, tardiness, setup-time, order, buffer-capacity, machine, make-span, sequence dependent setup &  &  &  &  &  &  & real-world, real-life, benchmark & 0 & \\
\href{papers/RodosekW98.pdf}{RodosekW98}~\cite{RodosekW98} & 15 & task, order, transportation, machine, activity, make-span, job, resource, scheduling &  & circuit, disjunctive, cycle & Prolog & OPL, CHIP, ECLiPSe, Cplex & hoist, electroplating &  & benchmark & 0 & \\
\href{papers/RossiTHP07.pdf}{RossiTHP07}~\cite{RossiTHP07} & 15 & resource, inventory, scheduling, distributed, stock level, order &  & cumulative, cycle &  & OPL, Choco Solver &  &  &  & 0 & \\
\href{papers/Sadykov04.pdf}{Sadykov04}~\cite{Sadykov04} & 7 & release-date, due-date, preempt, scheduling, completion-time, task, precedence, machine, job, lateness & parallel machine, single machine & disjunctive &  &  &  &  &  & 0 & edge-finding\\
\href{papers/SchuttCSW12.pdf}{SchuttCSW12}~\cite{SchuttCSW12} & 17 & scheduling, resource, order, preempt, activity, lazy clause generation, precedence, make-span &  & cumulative &  & CHIP &  &  & benchmark & 1 & \\
\href{papers/SchuttFS13.pdf}{SchuttFS13}~\cite{SchuttFS13} & 17 & resource, job, lazy clause generation, scheduling, task, order, job-shop, machine, activity, make-span, completion-time, precedence & RCPSP, FJS & disjunctive, span constraint, alternative constraint, cumulative &  & MiniZinc &  &  & benchmark & 1 & time-tabling, energetic reasoning\\
\href{papers/SchuttFSW09.pdf}{SchuttFSW09}~\cite{SchuttFSW09} & 16 & scheduling, resource, open-shop, order, task, machine, preempt, activity, lazy clause generation, precedence, make-span, job & psplib & disjunctive, cumulative &  & ECLiPSe, CHIP, SICStus, OZ &  &  & benchmark, real-world & 1 & edge-finder\\
\href{papers/SchuttS16.pdf}{SchuttS16}~\cite{SchuttS16} & 17 & machine, producer/consumer, precedence, order, inventory, lazy clause generation, activity, preempt, manpower, resource, scheduling, make-span & RCPSP & cumulative &  & Chuffed, MiniZinc, Ilog Scheduler, OPL &  &  & benchmark & 1 & \\
\href{papers/SchuttW10.pdf}{SchuttW10}~\cite{SchuttW10} & 15 & task, order, lazy clause generation, activity, preempt, release-date, due-date, resource, scheduling, make-span & psplib, CuSP, RCPSP & disjunctive, cumulative & Java & CHIP & rectangle-packing &  & benchmark & 1 & edge-finding, not-last, not-first\\
\href{papers/SerraNM12.pdf}{SerraNM12}~\cite{SerraNM12} & 17 & preempt, resource, scheduling, precedence, order, machine, activity, release-date, inventory &  & alwaysIn, cumulative, cycle &  & OPL, Cplex &  &  & benchmark, real-world & 4 & \\
\href{papers/SialaAH15.pdf}{SialaAH15}~\cite{SialaAH15} & 10 & make-span, open-shop, task, machine, precedence, order, cmax, tardiness, job, setup-time, earliness, lazy clause generation, job-shop, resource, scheduling & RCPSP, JSSP & disjunctive, cumulative &  & Mistral &  &  & github, benchmark & 5 & edge-finding\\
\href{papers/SimoninAHL12.pdf}{SimoninAHL12}~\cite{SimoninAHL12} & 15 & resource, activity, precedence, preempt, scheduling, order, task &  & disjunctive, span constraint, cumulative, cycle &  & CHIP & satellite &  &  & 0 & sweep\\
\href{papers/Simonis95.pdf}{Simonis95}~\cite{Simonis95} & 4 & transportation, resource, scheduling, task, machine, producer/consumer, precedence, order &  & cumulative, cycle, diffn, circuit & Prolog & CHIP & aircraft & food industry &  & 0 & \\
\href{papers/SimonisC95.pdf}{SimonisC95}~\cite{SimonisC95} & 14 & manpower, flow-shop, task, order, transportation, machine, inventory, job, batch process, producer/consumer, stock level, resource, continuous-process, job-shop, due-date, scheduling &  & diffn, cumulative & Prolog & OZ, CHIP & aircraft, pipeline & food industry & real-life & 0 & \\
\href{papers/SquillaciPR23.pdf}{SquillaciPR23}~\cite{SquillaciPR23} & 17 & resource, activity, multi-agent, distributed, order, scheduling, task & OSP, Earth Observation Scheduling Problem, EOSP & noOverlap & Python & Cplex & earth orbit, earth observation, satellite &  & github, benchmark & 2 & \\
\href{papers/SunLYL10.pdf}{SunLYL10}~\cite{SunLYL10} & 6 & task, order, scheduling, distributed &  & cycle &  & Cplex, OPL & automotive &  &  & 0 & \\
\href{papers/SzerediS16.pdf}{SzerediS16}~\cite{SzerediS16} & 10 & task, order, machine, preempt, activity, make-span, resource, precedence, lazy clause generation, scheduling & RCPSP, psplib & cumulative &  & Cplex, MiniZinc, Chuffed, Gecode &  &  & benchmark & 2 & \\
\href{papers/TangB20.pdf}{TangB20}~\cite{TangB20} & 16 & batch process, machine, job, flow-shop, precedence, resource, make-span, scheduling, tardiness, due-date, order & 2BPHFSP, single machine & span constraint, bin-packing, alwaysIn, endBeforeStart, cycle & Java & Cplex, CPO & semiconductor & manufacturing industry & real-world & 0 & \\
\href{papers/TardivoDFMP23.pdf}{TardivoDFMP23}~\cite{TardivoDFMP23} & 18 & activity, order, preempt, scheduling, make-span, lazy clause generation, task, resource, precedence & RCPSP, psplib, CuSP & disjunctive, cumulative & C++ & CHIP, Gecode, MiniZinc &  &  & bitbucket, github, benchmark, real-world & 9 & energetic reasoning, not-last, not-first, edge-finding, time-tabling, sweep\\
\href{papers/TasselGS23.pdf}{TasselGS23}~\cite{TasselGS23} & 9 & scheduling, preempt, flow-time, flow-shop, task, order, completion-time, machine, make-span, re-scheduling, job, precedence, tardiness, resource, job-shop & JSSP & cumulative, noOverlap, disjunctive & Java & Choco Solver &  &  & industrial instance, real-world, supplementary material, github, benchmark & 0 & \\
\href{papers/Teppan22.pdf}{Teppan22}~\cite{Teppan22} & 8 & job-shop, task, make-span, order, cmax, preempt, distributed, resource, completion-time, scheduling, machine, setup-time, job, flow-shop & parallel machine, PTC, FJS, JSSP & noOverlap, endBeforeStart & Java & OR-Tools, OPL &  &  & real-life, benchmark & 0 & \\
\href{papers/Tesch16.pdf}{Tesch16}~\cite{Tesch16} & 27 & scheduling, order, job, completion-time, precedence, resource, make-span & CuSP, psplib, RCPSP & cumulative, disjunctive & C++ & OPL &  &  & Roadef & 1 & sweep, edge-finding, energetic reasoning, not-last, time-tabling, not-first\\
\href{papers/Tesch18.pdf}{Tesch18}~\cite{Tesch18} & 17 & scheduling, preempt, due-date, order, machine, task, job, completion-time, precedence, lateness, release-date, resource, make-span & CuSP, psplib, single machine, RCPSP & cumulative &  &  &  &  & Roadef & 0 & sweep, edge-finding, energetic reasoning, not-last, time-tabling\\
\href{papers/ThiruvadyBME09.pdf}{ThiruvadyBME09}~\cite{ThiruvadyBME09} & 15 & tardiness, open-shop, machine, due-date, job, make-span, scheduling, order, resource, setup-time & single machine & cumulative & C++ & Gecode &  &  &  & 0 & \\
\href{papers/Tom19.pdf}{Tom19}~\cite{Tom19} & 6 & job-shop, job, re-scheduling, task, tardiness, activity, resource, make-span, scheduling, machine, transportation & single machine &  & Java & OZ, OPL &  &  & real-world & 0 & \\
\href{papers/TouatBT22.pdf}{TouatBT22}~\cite{TouatBT22} & 8 & tardiness, job, activity, preempt, release-date, no preempt, earliness, distributed, due-date, job-shop, flow-shop, resource, scheduling, make-span, completion-time, task, machine, precedence, order & RCPSP, single machine & noOverlap &  & OZ, OPL, Cplex & robot, container terminal, satellite &  & benchmark, generated instance & 0 & time-tabling\\
\href{papers/Touraivane95.pdf}{Touraivane95}~\cite{Touraivane95} & 3 & scheduling, order, task &  &  & Prolog &  & crew-scheduling &  & real-life & 0 & \\
\href{papers/ValleMGT03.pdf}{ValleMGT03}~\cite{ValleMGT03} & 8 & machine, order, transportation, make-span, resource, job, precedence, task, job-shop, scheduling &  &  &  & Ilog Solver & robot &  & real-life & 0 & edge-finder\\
\href{papers/VanczaM01.pdf}{VanczaM01}~\cite{VanczaM01} & 15 & resource, scheduling, precedence, task, machine, order &  & disjunctive, cycle &  & OZ & robot &  & real-life, real-world & 0 & \\
\href{papers/VerfaillieL01.pdf}{VerfaillieL01}~\cite{VerfaillieL01} & 15 & job, open-shop, order, scheduling, task, job-shop & Open Shop Scheduling Problem & cycle &  & Cplex, OPL & earth observation, satellite &  &  & 0 & \\
\href{papers/Vilim02.pdf}{Vilim02}~\cite{Vilim02} & 1 & scheduling, precedence, sequence dependent setup, batch process, activity, setup-time, resource &  & cumulative, disjunctive &  &  &  &  &  & 0 & edge-finding\\
\href{papers/Vilim03.pdf}{Vilim03}~\cite{Vilim03} & 1 & scheduling, job, open-shop, order, job-shop &  & cumulative, disjunctive &  &  &  &  &  & 0 & not-last, edge-finding\\
\href{papers/Vilim04.pdf}{Vilim04}~\cite{Vilim04} & 13 & scheduling, precedence, sequence dependent setup, batch process, machine, task, job, completion-time, activity, order, setup-time, resource, job-shop &  & cumulative, disjunctive &  &  &  &  & benchmark & 1 & sweep, not-last, edge-finding\\
\href{papers/Vilim05.pdf}{Vilim05}~\cite{Vilim05} & 14 & scheduling, precedence, preempt, machine, task, job, open-shop, completion-time, activity, order, resource, make-span, job-shop &  & cumulative, disjunctive & C++ &  &  &  & benchmark & 4 & not-last\\
\href{papers/Vilim09.pdf}{Vilim09}~\cite{Vilim09} & 15 & scheduling, precedence, preempt, job, completion-time, activity, order, resource, job-shop &  & cumulative, cycle &  & CPO &  &  &  & 0 & energetic reasoning, not-last, edge-finding, not-first\\
\href{papers/Vilim11.pdf}{Vilim11}~\cite{Vilim11} & 16 & scheduling, precedence, preempt, machine, task, completion-time, activity, order, manpower, resource & psplib, RCPSP & cumulative, disjunctive, cycle &  &  &  &  & benchmark & 1 & sweep, energetic reasoning, not-last, time-tabling, edge-finding\\
\href{papers/VilimBC04.pdf}{VilimBC04}~\cite{VilimBC04} & 15 & distributed, job-shop, resource, scheduling, make-span, open-shop, completion-time, machine, precedence, order, job, activity &  & disjunctive, cumulative &  &  &  &  & benchmark, real-life & 0 & not-first, edge-finding, not-last\\
\href{papers/VilimLS15.pdf}{VilimLS15}~\cite{VilimLS15} & 17 & machine, precedence, order, cmax, job, activity, earliness, job-shop, resource, scheduling, make-span, completion-time, task & psplib, RCPSP & noOverlap, disjunctive, cumulative &  & Cplex, CPO, OZ & rectangle-packing &  & benchmark & 8 & time-tabling\\
\href{papers/WangB20.pdf}{WangB20}~\cite{WangB20} & 8 & job, order, machine, task, distributed, resource, scheduling & Fixed Job Scheduling, FJS & alldifferent &  & OZ, Gurobi & aircraft &  & github & 0 & \\
\href{papers/WangB23.pdf}{WangB23}~\cite{WangB23} & 8 & job, lazy clause generation, order, task, transportation, resource, scheduling & Fixed Job Scheduling, FJS & alldifferent &  & Gurobi & crew-scheduling, aircraft &  & real-world, random instance & 0 & \\
\href{papers/WatsonB08.pdf}{WatsonB08}~\cite{WatsonB08} & 15 & job-shop, resource, scheduling, make-span, completion-time, machine, order, cmax, job &  & disjunctive & C++ & Ilog Scheduler &  &  & benchmark, real-world & 1 & \\
\href{papers/WessenCS20.pdf}{WessenCS20}~\cite{WessenCS20} & 10 & make-span, completion-time, precedence, job, scheduling, task, order, job-shop, multi-agent &  & circuit &  & Gecode, OZ & robot &  & real-world & 10 & \\
\href{papers/WinterMMW22.pdf}{WinterMMW22}~\cite{WinterMMW22} & 18 & tardiness, precedence, release-date, setup-time, job, scheduling, completion-time, resource, order, task, machine, distributed, due-date & parallel machine, PMSP & alternative constraint, noOverlap &  & CPO, Gurobi, Cplex & farming & manufacturing industry, agricultural industry & supplementary material, real-life, industry partner, zenodo, industrial partner, benchmark & 0 & \\
\href{papers/Wolf03.pdf}{Wolf03}~\cite{Wolf03} & 15 & completion-time, resource, job, make-span, machine, activity, job-shop, task, order, preempt, scheduling &  & cumulative, disjunctive & Java &  & pipeline &  & benchmark & 0 & not-last, not-first, edge-finding, sweep\\
\href{papers/WolfS05.pdf}{WolfS05}~\cite{WolfS05} & 14 & preempt, activity, order, task, completion-time, scheduling, distributed, resource &  & cumulative &  & CHIP &  &  & real-world & 0 & energetic reasoning, not-last, sweep\\
\href{papers/WolinskiKG04.pdf}{WolinskiKG04}~\cite{WolinskiKG04} & 8 & resource, precedence, scheduling, machine, order, distributed & SCC & cycle & Java &  & pipeline &  &  & 0 & \\
\href{papers/WuBB05.pdf}{WuBB05}~\cite{WuBB05} & 1 & scheduling, resource, job, make-span, release-date &  &  &  & Ilog Scheduler &  &  & benchmark & 0 & \\
\href{papers/YangSS19.pdf}{YangSS19}~\cite{YangSS19} & 10 & resource, completion-time, machine, task, activity, preempt, order, scheduling, lazy clause generation &  & cumulative, disjunctive & Prolog & Choco Solver, Gecode, CHIP, SICStus, OPL, OR-Tools & rectangle-packing &  & generated instance & 1 & not-last, energetic reasoning, edge-finding\\
\href{papers/YoungFS17.pdf}{YoungFS17}~\cite{YoungFS17} & 10 & lazy clause generation, resource, scheduling, make-span, task, machine, precedence, order, activity, preempt & RCPSP, psplib & disjunctive, cumulative &  & Chuffed, MiniZinc &  &  & benchmark, github, instance generator & 6 & time-tabling\\
\href{papers/YuraszeckMC23.pdf}{YuraszeckMC23}~\cite{YuraszeckMC23} & 6 & cmax, job, open-shop, distributed, order, preempt, scheduling, due-date, job-shop, flow-time, make-span, machine, release-date, precedence & OSSP, JSSP & noOverlap &  &  &  &  & github, benchmark & 0 & \\
\href{papers/ZhangJZL22.pdf}{ZhangJZL22}~\cite{ZhangJZL22} & 6 & setup-time, due-date, scheduling, flow-shop, task, order, completion-time, transportation, machine, make-span, job, precedence, tardiness, resource & parallel machine, single machine & alternative constraint, cumulative, noOverlap, endBeforeStart &  & OZ & semiconductor &  & benchmark & 0 & \\
\href{papers/ZhangLS12.pdf}{ZhangLS12}~\cite{ZhangLS12} & 4 & scheduling, order, cmax &  &  &  &  &  &  &  & 0 & time-tabling\\
\href{papers/Zhou96.pdf}{Zhou96}~\cite{Zhou96} & 15 & release-date, job-shop, due-date, task, order, scheduling, precedence, completion-time, job, machine &  & disjunctive & Prolog & Z3 &  &  &  & 0 & edge-finding\\
\href{papers/ZhouGL15.pdf}{ZhouGL15}~\cite{ZhouGL15} & 5 & scheduling, distributed, resource, completion-time, tardiness, machine, setup-time, job, job-shop, flow-shop, task, re-scheduling, make-span, transportation, order, cmax & FJS, HFF, parallel machine & cumulative &  & CHIP, OR-Tools, Gecode, OZ & railway &  & real-world & 0 & \\
\href{papers/ZhuS02.pdf}{ZhuS02}~\cite{ZhuS02} & 5 & activity, scheduling, distributed, resource &  &  &  &  &  &  &  & 0 & \\
\href{papers/ZibranR11.pdf}{ZibranR11}~\cite{ZibranR11} & 4 & scheduling, order, activity &  &  & Java & OPL, Cplex &  &  &  & 0 & \\
\href{papers/ZibranR11a.pdf}{ZibranR11a}~\cite{ZibranR11a} & 10 & scheduling, distributed, order, activity, resource &  &  &  & Cplex, OPL &  &  &  & 0 & time-tabling\\
\href{papers/cp-Hooker05.pdf}{cp-Hooker05}~\cite{cp-Hooker05} & 14 & tardiness, precedence, scheduling, make-span, order, job, due-date, resource, task, machine, release-date &  & circuit, disjunctive, cumulative &  & Cplex, Ilog Scheduler, OPL &  &  &  & 0 & \\
\href{papers/cpaior-GayHS15.pdf}{cpaior-GayHS15}~\cite{cpaior-GayHS15} & 16 & preempt, machine, task, scheduling, order, manpower, resource & psplib, RCPSP & cumulative, disjunctive & Java &  &  &  & bitbucket, benchmark, real-world & 0 & edge-finding, time-tabling, not-first, sweep, not-last, energetic reasoning\\
\href{papers/cpaior-SchuttFS13.pdf}{cpaior-SchuttFS13}~\cite{cpaior-SchuttFS13} & 17 & completion-time, machine, lazy clause generation, activity, scheduling, order, preempt, task, make-span, precedence, resource & RCPSP, psplib & disjunctive, circuit, cumulative &  & CHIP, OZ &  &  & benchmark & 5 & not-last, edge-finding, energetic reasoning\\
\href{papers/cpaior-Vilim09.pdf}{cpaior-Vilim09}~\cite{cpaior-Vilim09} & 15 & scheduling, order, completion-time, preempt, resource, activity, task &  & cycle, cumulative &  & Ilog Scheduler &  &  &  & 1 & energetic reasoning, not-last, edge-finding\\
\end{longtable}
}



\clearpage
\subsection{Manually Defined Fields}
{\scriptsize
\begin{longtable}{>{\raggedright\arraybackslash}p{3cm}>{\raggedright\arraybackslash}p{6cm}p{2cm}rrrrlrr}
\rowcolor{white}\caption{Manually Defined PAPER Properties}\\ \toprule
\rowcolor{white}Key & Title (Local Copy)  & Bench & Links & \shortstack{Data\\Avail} & \shortstack{Sol\\Avail} & \shortstack{Code\\Avail} & \shortstack{Related\\To} & a & b\\ \midrule\endhead
\bottomrule
\endfoot
\rowlabel{c:FalqueALM24}FalqueALM24 \href{https://doi.org/10.1609/aaai.v38i21.30308}{FalqueALM24}~\cite{FalqueALM24} & \href{../works/FalqueALM24.pdf}{Check-In Desk Scheduling Optimisation at {CDG} International Airport} & github, real-world, gitlab, supplementary material & 0 &  &  &  &  & \ref{a:FalqueALM24} & \ref{b:FalqueALM24}\\
\rowlabel{c:LiLZDZW24}LiLZDZW24 \href{https://doi.org/10.1609/aaai.v38i18.29998}{LiLZDZW24}~\cite{LiLZDZW24} & \href{../works/LiLZDZW24.pdf}{Learning to Optimize Permutation Flow Shop Scheduling via Graph-Based Imitation Learning} & benchmark, github, real-life & 0 &  &  &  &  & \ref{a:LiLZDZW24} & \ref{b:LiLZDZW24}\\
\rowlabel{c:AalianPG23}AalianPG23 \href{https://doi.org/10.4230/LIPIcs.CP.2023.6}{AalianPG23}~\cite{AalianPG23} & \href{../works/AalianPG23.pdf}{Optimization of Short-Term Underground Mine Planning Using Constraint Programming} & real-world & 1 & n &  & n &  & \ref{a:AalianPG23} & \ref{b:AalianPG23}\\
\rowlabel{c:Bit-Monnot23}Bit-Monnot23 \href{https://doi.org/10.3233/FAIA230278}{Bit-Monnot23}~\cite{Bit-Monnot23} & \href{../works/Bit-Monnot23.pdf}{Enhancing Hybrid {CP-SAT} Search for Disjunctive Scheduling} & benchmark, real-world, github & 1 & \href{https://github.com/plaans/aries}{y} &  & \href{https://github.com/plaans/aries}{y} & - & \ref{a:Bit-Monnot23} & \ref{b:Bit-Monnot23}\\
\rowlabel{c:EfthymiouY23}EfthymiouY23 \href{https://doi.org/10.1007/978-3-031-33271-5_16}{EfthymiouY23}~\cite{EfthymiouY23} & \href{../works/EfthymiouY23.pdf}{Predicting the Optimal Period for Cyclic Hoist Scheduling Problems} & generated instance, random instance, industrial instance, benchmark, real-life & 3 & n &  & n & - & \ref{a:EfthymiouY23} & \ref{b:EfthymiouY23}\\
\rowlabel{c:JuvinHHL23}JuvinHHL23 \href{https://doi.org/10.4230/LIPIcs.CP.2023.19}{JuvinHHL23}~\cite{JuvinHHL23} & \href{../works/JuvinHHL23.pdf}{An Efficient Constraint Programming Approach to Preemptive Job Shop Scheduling} & github, benchmark, supplementary material & 6 & ref &  & y &  & \ref{a:JuvinHHL23} & \ref{b:JuvinHHL23}\\
\rowlabel{c:JuvinHL23}JuvinHL23 \href{https://doi.org/10.1007/978-3-031-33271-5_23}{JuvinHL23}~\cite{JuvinHL23} & \href{../works/JuvinHL23.pdf}{Constraint Programming for the Robust Two-Machine Flow-Shop Scheduling Problem with Budgeted Uncertainty} & real-world & 0 & ref &  & n & - & \ref{a:JuvinHL23} & \ref{b:JuvinHL23}\\
\rowlabel{c:KameugneFND23}KameugneFND23 \href{https://doi.org/10.4230/LIPIcs.CP.2023.20}{KameugneFND23}~\cite{KameugneFND23} & \href{../works/KameugneFND23.pdf}{Horizontally Elastic Edge Finder Rule for Cumulative Constraint Based on Slack and Density} & benchmark & 5 & \su{BL PSPlib} &  & n & - & \ref{a:KameugneFND23} & \ref{b:KameugneFND23}\\
\rowlabel{c:KimCMLLP23}KimCMLLP23 \href{https://doi.org/10.1007/978-3-031-33271-5_31}{KimCMLLP23}~\cite{KimCMLLP23} & \href{../works/KimCMLLP23.pdf}{Iterated Greedy Constraint Programming for Scheduling Steelmaking Continuous Casting} & real-world, zenodo, benchmark & 0 & \href{https://zenodo.org/records/5126007}{y} &  & n & - & \ref{a:KimCMLLP23} & \ref{b:KimCMLLP23}\\
\rowlabel{c:Mehdizadeh-Somarin23}Mehdizadeh-Somarin23 \href{https://doi.org/10.1007/978-3-031-43670-3_33}{Mehdizadeh-Somarin23}~\cite{Mehdizadeh-Somarin23} & \href{../works/Mehdizadeh-Somarin23.pdf}{A Constraint Programming Model for a Reconfigurable Job Shop Scheduling Problem with Machine Availability} & random instance & 0 & n &  & n & - & \ref{a:Mehdizadeh-Somarin23} & \ref{b:Mehdizadeh-Somarin23}\\
\rowlabel{c:PerezGSL23}PerezGSL23 \href{https://doi.org/10.1109/ICTAI59109.2023.00108}{PerezGSL23}~\cite{PerezGSL23} & \href{../works/PerezGSL23.pdf}{A Constraint Programming Model for Scheduling the Unloading of Trains in Ports} & real-world, generated instance & 0 & n &  & n & - & \ref{a:PerezGSL23} & \ref{b:PerezGSL23}\\
\rowlabel{c:PovedaAA23}PovedaAA23 \href{https://doi.org/10.4230/LIPIcs.CP.2023.31}{PovedaAA23}~\cite{PovedaAA23} & \href{../works/PovedaAA23.pdf}{Partially Preemptive Multi Skill/Mode Resource-Constrained Project Scheduling with Generalized Precedence Relations and Calendars} & industrial instance, real-world, github, benchmark, real-life & 4 & y &  & \href{https://github.com/youngkd/MSPSP-InstLib/blob/master/models/mspsp.mzn}{y} &  & \ref{a:PovedaAA23} & \ref{b:PovedaAA23}\\
\rowlabel{c:SquillaciPR23}SquillaciPR23 \href{https://doi.org/10.1007/978-3-031-33271-5_29}{SquillaciPR23}~\cite{SquillaciPR23} & \href{../works/SquillaciPR23.pdf}{Scheduling Complex Observation Requests for a Constellation of Satellites: Large Neighborhood Search Approaches} & github, benchmark & 2 & \href{https://github.com/ssquilla/Earth_Observing_Satellites_benchmarks}{y} &  & n & - & \ref{a:SquillaciPR23} & \ref{b:SquillaciPR23}\\
\rowlabel{c:TardivoDFMP23}TardivoDFMP23 \href{https://doi.org/10.1007/978-3-031-33271-5_22}{TardivoDFMP23}~\cite{TardivoDFMP23} & \href{../works/TardivoDFMP23.pdf}{Constraint Propagation on {GPU:} {A} Case Study for the Cumulative Constraint} & bitbucket, real-world, benchmark, github & 9 & \href{https://bitbucket.org/constraint-programming/minicpp-benchmarks/src/main/rcpsp/}{\su{PSPLib BL Pack}} &  & y & - & \ref{a:TardivoDFMP23} & \ref{b:TardivoDFMP23}\\
\rowlabel{c:TasselGS23}TasselGS23 \href{https://doi.org/10.1609/icaps.v33i1.27243}{TasselGS23}~\cite{TasselGS23} & \href{../works/TasselGS23.pdf}{An End-to-End Reinforcement Learning Approach for Job-Shop Scheduling Problems Based on Constraint Programming} & industrial instance, real-world, github, benchmark, supplementary material & 0 & ref &  & \href{https://github.com/ingambe/End2End-Job-Shop-Scheduling-CP}{y} & - & \ref{a:TasselGS23} & \ref{b:TasselGS23}\\
\rowlabel{c:WangB23}WangB23 \href{https://doi.org/10.1109/ICTAI59109.2023.00062}{WangB23}~\cite{WangB23} & \href{../works/WangB23.pdf}{Dynamic All-Different and Maximal Cliques Constraints for Fixed Job Scheduling} & random instance, real-world & 0 & (y) &  & n & \cite{WangB20} & \ref{a:WangB23} & \ref{b:WangB23}\\
\rowlabel{c:YuraszeckMC23}YuraszeckMC23 \href{https://doi.org/10.1016/j.procs.2023.03.130}{YuraszeckMC23}~\cite{YuraszeckMC23} & \href{../works/YuraszeckMC23.pdf}{A competitive constraint programming approach for the group shop scheduling problem} & benchmark, github & 0 & ref &  & n & - & \ref{a:YuraszeckMC23} & \ref{b:YuraszeckMC23}\\
\rowlabel{c:ArmstrongGOS22}ArmstrongGOS22 \href{https://doi.org/10.1007/978-3-031-08011-1_1}{ArmstrongGOS22}~\cite{ArmstrongGOS22} & \href{../works/ArmstrongGOS22.pdf}{A Two-Phase Hybrid Approach for the Hybrid Flexible Flowshop with Transportation Times} & real-world, benchmark & 0 & (y) &  & - & \cite{ArmstrongGOS21} & \ref{a:ArmstrongGOS22} & \ref{b:ArmstrongGOS22}\\
\rowlabel{c:BoudreaultSLQ22}BoudreaultSLQ22 \href{https://doi.org/10.4230/LIPIcs.CP.2022.10}{BoudreaultSLQ22}~\cite{BoudreaultSLQ22} & \href{../works/BoudreaultSLQ22.pdf}{A Constraint Programming Approach to Ship Refit Project Scheduling} & gitlab, github, real-world, supplementary material, benchmark, generated instance, real-life, industrial partner & 9 &  &  & \href{https://github.com/raphaelboudreault/chuffed/releases/tag/SBPS}{y} & - & \ref{a:BoudreaultSLQ22} & \ref{b:BoudreaultSLQ22}\\
\rowlabel{c:GeitzGSSW22}GeitzGSSW22 \href{https://doi.org/10.1007/978-3-031-08011-1_10}{GeitzGSSW22}~\cite{GeitzGSSW22} & \href{../works/GeitzGSSW22.pdf}{Solving the Extended Job Shop Scheduling Problem with AGVs - Classical and Quantum Approaches} & real-world, real-life, github & 8 & \href{https://github.com/cgrozea/Data4ExtJSSAGV}{y} &  & n & - & \ref{a:GeitzGSSW22} & \ref{b:GeitzGSSW22}\\
\rowlabel{c:JungblutK22}JungblutK22 \href{https://doi.org/10.1109/IPDPSW55747.2022.00025}{JungblutK22}~\cite{JungblutK22} & \href{../works/JungblutK22.pdf}{Optimal Schedules for High-Level Programming Environments on FPGAs with Constraint Programming} & benchmark, real-world, github & 0 & \href{https://github.com/pascalj/reconf-scheduling}{y} &  & y & - & \ref{a:JungblutK22} & \ref{b:JungblutK22}\\
\rowlabel{c:KotaryFH22}KotaryFH22 \href{https://doi.org/10.1609/aaai.v36i7.20685}{KotaryFH22}~\cite{KotaryFH22} & \href{../works/KotaryFH22.pdf}{Fast Approximations for Job Shop Scheduling: {A} Lagrangian Dual Deep Learning Method} & github, benchmark & 0 &  &  &  &  & \ref{a:KotaryFH22} & \ref{b:KotaryFH22}\\
\rowlabel{c:LiFJZLL22}LiFJZLL22 \href{https://doi.org/10.1109/ICNSC55942.2022.10004158}{LiFJZLL22}~\cite{LiFJZLL22} & \href{../works/LiFJZLL22.pdf}{Constraint Programming for a Novel Integrated Optimization of Blocking Job Shop Scheduling and Variable-Speed Transfer Robot Assignment} & benchmark & 0 & ref &  & n & - & \ref{a:LiFJZLL22} & \ref{b:LiFJZLL22}\\
\rowlabel{c:LuoB22}LuoB22 \href{https://doi.org/10.1007/978-3-031-08011-1_17}{LuoB22}~\cite{LuoB22} & \href{../works/LuoB22.pdf}{Packing by Scheduling: Using Constraint Programming to Solve a Complex 2D Cutting Stock Problem} & real-life, industry partner, real-world, generated instance, github, industrial instance & 2 & n &  & n & - & \ref{a:LuoB22} & \ref{b:LuoB22}\\
\rowlabel{c:OuelletQ22}OuelletQ22 \href{https://doi.org/10.1007/978-3-031-08011-1_21}{OuelletQ22}~\cite{OuelletQ22} & \href{../works/OuelletQ22.pdf}{A MinCumulative Resource Constraint} & github, benchmark, random instance & 1 & \href{https://github.com/yanickouellet/min-cumulative-paper-public}{y} &  & \href{https://github.com/yanickouellet/min-cumulative-paper-public}{y} & - & \ref{a:OuelletQ22} & \ref{b:OuelletQ22}\\
\rowlabel{c:OujanaAYB22}OujanaAYB22 \href{https://doi.org/10.1109/CoDIT55151.2022.9803972}{OujanaAYB22}~\cite{OujanaAYB22} & \href{../works/OujanaAYB22.pdf}{Solving a realistic hybrid and flexible flow shop scheduling problem through constraint programming: industrial case in a packaging company} & real-world, real-life, industrial instance, benchmark & 0 & n &  & n & - & \ref{a:OujanaAYB22} & \ref{b:OujanaAYB22}\\
\rowlabel{c:PopovicCGNC22}PopovicCGNC22 \href{https://doi.org/10.4230/LIPIcs.CP.2022.34}{PopovicCGNC22}~\cite{PopovicCGNC22} & \href{../works/PopovicCGNC22.pdf}{Scheduling the Equipment Maintenance of an Electric Power Transmission Network Using Constraint Programming} &  & 0 & n &  & n & - & \ref{a:PopovicCGNC22} & \ref{b:PopovicCGNC22}\\
\rowlabel{c:Teppan22}Teppan22 \href{https://doi.org/10.5220/0010849900003116}{Teppan22}~\cite{Teppan22} & \href{../works/Teppan22.pdf}{Types of Flexible Job Shop Scheduling: {A} Constraint Programming Experiment} & benchmark, real-life & 0 & ref &  & n & - & \ref{a:Teppan22} & \ref{b:Teppan22}\\
\rowlabel{c:TouatBT22}TouatBT22 \href{}{TouatBT22}~\cite{TouatBT22} & \href{../works/TouatBT22.pdf}{A Constraint Programming Model for the Scheduling Problem with Flexible Maintenance under Human Resource Constraints} & generated instance, benchmark & 0 & n &  & n & - & \ref{a:TouatBT22} & \ref{b:TouatBT22}\\
\rowlabel{c:WinterMMW22}WinterMMW22 \href{https://doi.org/10.4230/LIPIcs.CP.2022.41}{WinterMMW22}~\cite{WinterMMW22} & \href{../works/WinterMMW22.pdf}{Modeling and Solving Parallel Machine Scheduling with Contamination Constraints in the Agricultural Industry} & supplementary material, zenodo, industrial partner, benchmark, real-life, industry partner & 0 & \href{https://zenodo.org/records/6797397}{y} &  & \href{https://zenodo.org/records/6797397}{y} & - & \ref{a:WinterMMW22} & \ref{b:WinterMMW22}\\
\rowlabel{c:ZhangJZL22}ZhangJZL22 \href{https://doi.org/10.1109/ICNSC55942.2022.10004154}{ZhangJZL22}~\cite{ZhangJZL22} & \href{../works/ZhangJZL22.pdf}{Constraint Programming for Modeling and Solving a Hybrid Flow Shop Scheduling Problem} & benchmark & 0 & ref &  & n & - & \ref{a:ZhangJZL22} & \ref{b:ZhangJZL22}\\
\rowlabel{c:AntuoriHHEN21}AntuoriHHEN21 \href{https://doi.org/10.4230/LIPIcs.CP.2021.14}{AntuoriHHEN21}~\cite{AntuoriHHEN21} & \href{../works/AntuoriHHEN21.pdf}{Combining Monte Carlo Tree Search and Depth First Search Methods for a Car Manufacturing Workshop Scheduling Problem} & gitlab, supplementary material & 1 & \href{https://gitlab.laas.fr/vantuori/mcts-cp}{y} &  & \href{https://gitlab.laas.fr/vantuori/mcts-cp}{y} &  & \ref{a:AntuoriHHEN21} & \ref{b:AntuoriHHEN21}\\
\rowlabel{c:ArmstrongGOS21}ArmstrongGOS21 \href{https://doi.org/10.4230/LIPIcs.CP.2021.16}{ArmstrongGOS21}~\cite{ArmstrongGOS21} & \href{../works/ArmstrongGOS21.pdf}{The Hybrid Flexible Flowshop with Transportation Times} & industry partner, instance generator, zenodo, supplementary material, real-world, industrial partner, benchmark & 1 & \href{https://zenodo.org/record/5168966}{y} &  & y & - & \ref{a:ArmstrongGOS21} & \ref{b:ArmstrongGOS21}\\
\rowlabel{c:Astrand0F21}Astrand0F21 \href{https://doi.org/10.1007/978-3-030-78230-6_23}{Astrand0F21}~\cite{Astrand0F21} & \href{../works/Astrand0F21.pdf}{Short-Term Scheduling of Production Fleets in Underground Mines Using CP-Based {LNS}} & benchmark, real-world, generated instance, real-life & 0 & \su{ref generated} &  & n & - & \ref{a:Astrand0F21} & \ref{b:Astrand0F21}\\
\rowlabel{c:BenderWS21}BenderWS21 \href{https://doi.org/10.1007/978-3-030-87672-2_37}{BenderWS21}~\cite{BenderWS21} & \href{../works/BenderWS21.pdf}{Applying Constraint Programming to the Multi-mode Scheduling Problem in Harvest Logistics} &  & 9 & \href{https://tud.link/47mz}{y} &  & n & - & \ref{a:BenderWS21} & \ref{b:BenderWS21}\\
\rowlabel{c:GeibingerKKMMW21}GeibingerKKMMW21 \href{https://doi.org/10.1007/978-3-030-78230-6_29}{GeibingerKKMMW21}~\cite{GeibingerKKMMW21} & \href{../works/GeibingerKKMMW21.pdf}{Physician Scheduling During a Pandemic} & real-world & 3 & \href{https://cdlab-artis.dbai.tuwien.ac.at/papers/pandemic-scheduling/}{y} &  & n & - & \ref{a:GeibingerKKMMW21} & \ref{b:GeibingerKKMMW21}\\
\rowlabel{c:GeibingerMM21}GeibingerMM21 \href{https://doi.org/10.1609/aaai.v35i7.16789}{GeibingerMM21}~\cite{GeibingerMM21} & \href{../works/GeibingerMM21.pdf}{Constraint Logic Programming for Real-World Test Laboratory Scheduling} & real-world, generated instance, github, benchmark, real-life & 0 & \href{dbai.tuwien.ac.at/staff/fmischek/TLSP}{y} &  &  &  & \ref{a:GeibingerMM21} & \ref{b:GeibingerMM21}\\
\rowlabel{c:HanenKP21}HanenKP21 \href{https://doi.org/10.1007/978-3-030-78230-6_14}{HanenKP21}~\cite{HanenKP21} & \href{../works/HanenKP21.pdf}{Two Deadline Reduction Algorithms for Scheduling Dependent Tasks on Parallel Processors} & Roadef, generated instance, random instance & 1 & ref &  & n & - & \ref{a:HanenKP21} & \ref{b:HanenKP21}\\
\rowlabel{c:HillTV21}HillTV21 \href{https://doi.org/10.1007/978-3-030-78230-6_2}{HillTV21}~\cite{HillTV21} & \href{../works/HillTV21.pdf}{A Computational Study of Constraint Programming Approaches for Resource-Constrained Project Scheduling with Autonomous Learning Effects} & real-world & 0 & PSPlib &  & n & - & \ref{a:HillTV21} & \ref{b:HillTV21}\\
\rowlabel{c:KlankeBYE21}KlankeBYE21 \href{https://doi.org/10.1007/978-3-030-78230-6_9}{KlankeBYE21}~\cite{KlankeBYE21} & \href{../works/KlankeBYE21.pdf}{Combining Constraint Programming and Temporal Decomposition Approaches - Scheduling of an Industrial Formulation Plant} & real-life, random instance, benchmark & 0 & n &  & n & - & \ref{a:KlankeBYE21} & \ref{b:KlankeBYE21}\\
\rowlabel{c:KovacsTKSG21}KovacsTKSG21 \href{https://doi.org/10.4230/LIPIcs.CP.2021.36}{KovacsTKSG21}~\cite{KovacsTKSG21} & \href{../works/KovacsTKSG21.pdf}{Utilizing Constraint Optimization for Industrial Machine Workload Balancing} & github, real-world, supplementary material, benchmark & 2 & \href{https://github.com/prosysscience/CPWorkloadBalancing}{y} &  & \href{https://github.com/prosysscience/CPWorkloadBalancing}{y} & - & \ref{a:KovacsTKSG21} & \ref{b:KovacsTKSG21}\\
\rowlabel{c:LacknerMMWW21}LacknerMMWW21 \href{https://doi.org/10.4230/LIPIcs.CP.2021.37}{LacknerMMWW21}~\cite{LacknerMMWW21} & \href{../works/LacknerMMWW21.pdf}{Minimizing Cumulative Batch Processing Time for an Industrial Oven Scheduling Problem} & random instance, supplementary material, benchmark, instance generator, real-life, industrial partner & 3 & \href{https://cdlab-artis.dbai.tuwien.ac.at/papers/ovenscheduling/}{y} &  & \href{https://cdlab-artis.dbai.tuwien.ac.at/papers/ovenscheduling/}{y} &  & \ref{a:LacknerMMWW21} & \ref{b:LacknerMMWW21}\\
\rowlabel{c:AntuoriHHEN20}AntuoriHHEN20 \href{https://doi.org/10.1007/978-3-030-58475-7_38}{AntuoriHHEN20}~\cite{AntuoriHHEN20} & \href{../works/AntuoriHHEN20.pdf}{Leveraging Reinforcement Learning, Constraint Programming and Local Search: {A} Case Study in Car Manufacturing} & random instance, generated instance, gitlab, benchmark, industrial instance & 4 &  &  &  &  & \ref{a:AntuoriHHEN20} & \ref{b:AntuoriHHEN20}\\
\rowlabel{c:BarzegaranZP20}BarzegaranZP20 \href{https://doi.org/10.4230/OASIcs.Fog-IoT.2020.3}{BarzegaranZP20}~\cite{BarzegaranZP20} & \href{../works/BarzegaranZP20.pdf}{Quality-Of-Control-Aware Scheduling of Communication in TSN-Based Fog Computing Platforms Using Constraint Programming} &  & 5 & n &  & n & - & \ref{a:BarzegaranZP20} & \ref{b:BarzegaranZP20}\\
\rowlabel{c:GodetLHS20}GodetLHS20 \href{https://doi.org/10.1609/aaai.v34i02.5510}{GodetLHS20}~\cite{GodetLHS20} & \href{../works/GodetLHS20.pdf}{Using Approximation within Constraint Programming to Solve the Parallel Machine Scheduling Problem with Additional Unit Resources} & real-life, benchmark, generated instance, github & 0 & \href{https://github.com/ArthurGodet/PMSPAUR-public}{JSON} &  & \href{https://github.com/ArthurGodet/PMSPAUR-public}{y} & - & \ref{a:GodetLHS20} & \ref{b:GodetLHS20}\\
\rowlabel{c:GroleazNS20}GroleazNS20 \href{https://doi.org/10.1007/978-3-030-58475-7_36}{GroleazNS20}~\cite{GroleazNS20} & \href{../works/GroleazNS20.pdf}{Solving the Group Cumulative Scheduling Problem with {CPO} and {ACO}} & industrial instance, benchmark & 0 & - &  & - & \cite{GroleazNS20} & \ref{a:GroleazNS20} & \ref{b:GroleazNS20}\\
\rowlabel{c:GroleazNS20a}GroleazNS20a \href{https://doi.org/10.1145/3377930.3389818}{GroleazNS20a}~\cite{GroleazNS20a} & \href{../works/GroleazNS20a.pdf}{{ACO} with automatic parameter selection for a scheduling problem with a group cumulative constraint} & industrial partner, benchmark & 0 & \href{https://perso.citi-lab.fr/csolnon/gc-sched.html}{y} &  & n & - & \ref{a:GroleazNS20a} & \ref{b:GroleazNS20a}\\
\rowlabel{c:Mercier-AubinGQ20}Mercier-AubinGQ20 \href{https://doi.org/10.1007/978-3-030-58942-4_22}{Mercier-AubinGQ20}~\cite{Mercier-AubinGQ20} & \href{../works/Mercier-AubinGQ20.pdf}{Leveraging Constraint Scheduling: {A} Case Study to the Textile Industry} & industrial partner, industrial instance & 1 & a &  & a & - & \ref{a:Mercier-AubinGQ20} & \ref{b:Mercier-AubinGQ20}\\
\rowlabel{c:NattafM20}NattafM20 \href{https://doi.org/10.1007/978-3-030-58475-7_27}{NattafM20}~\cite{NattafM20} & \href{../works/NattafM20.pdf}{Filtering Rules for Flow Time Minimization in a Parallel Machine Scheduling Problem} & industrial instance, benchmark & 7 & - &  & - & \cite{MalapertN19} & \ref{a:NattafM20} & \ref{b:NattafM20}\\
\rowlabel{c:TangB20}TangB20 \href{https://doi.org/10.1007/978-3-030-58942-4_28}{TangB20}~\cite{TangB20} & \href{../works/TangB20.pdf}{{CP} and Hybrid Models for Two-Stage Batching and Scheduling} & real-world & 0 & n &  & n & - & \ref{a:TangB20} & \ref{b:TangB20}\\
\rowlabel{c:ThomasKS20}ThomasKS20 \href{https://doi.org/10.1007/978-3-030-58942-4_30}{ThomasKS20}~\cite{ThomasKS20} & \href{../works/ThomasKS20.pdf}{Insertion Sequence Variables for Hybrid Routing and Scheduling Problems} & benchmark, bitbucket, CSPlib, generated instance & 3 &  &  &  &  & \ref{a:ThomasKS20} & \ref{b:ThomasKS20}\\
\rowlabel{c:WangB20}WangB20 \href{https://doi.org/10.3233/FAIA200114}{WangB20}~\cite{WangB20} & \href{../works/WangB20.pdf}{Global Propagation of Transition Cost for Fixed Job Scheduling} & github & 0 & \href{http://recherche.enac.fr/~wangrx/ecai_gap/}{y} &  & n & - & \ref{a:WangB20} & \ref{b:WangB20}\\
\rowlabel{c:WessenCS20}WessenCS20 \href{https://doi.org/10.1007/978-3-030-58942-4_33}{WessenCS20}~\cite{WessenCS20} & \href{../works/WessenCS20.pdf}{Scheduling of Dual-Arm Multi-tool Assembly Robots and Workspace Layout Optimization} & real-world & 10 & n &  & n & - & \ref{a:WessenCS20} & \ref{b:WessenCS20}\\
\rowlabel{c:BadicaBIL19}BadicaBIL19 \href{https://doi.org/10.1007/978-3-030-32258-8_17}{BadicaBIL19}~\cite{BadicaBIL19} & \href{../works/BadicaBIL19.pdf}{Exploring the Space of Block Structured Scheduling Processes Using Constraint Logic Programming} & github & 0 & dead &  & dead & - & \ref{a:BadicaBIL19} & \ref{b:BadicaBIL19}\\
\rowlabel{c:BehrensLM19}BehrensLM19 \href{https://doi.org/10.1109/ICRA.2019.8794022}{BehrensLM19}~\cite{BehrensLM19} & \href{../works/BehrensLM19.pdf}{A Constraint Programming Approach to Simultaneous Task Allocation and Motion Scheduling for Industrial Dual-Arm Manipulation Tasks} & github, real-world & 0 & \href{https://github.com/boschresearch/STAAMS-SOLVER}{y} &  & \href{https://github.com/boschresearch/STAAMS-SOLVER}{y} & - & \ref{a:BehrensLM19} & \ref{b:BehrensLM19}\\
\rowlabel{c:BogaerdtW19}BogaerdtW19 \href{https://doi.org/10.1007/978-3-030-19212-9_38}{BogaerdtW19}~\cite{BogaerdtW19} & \href{../works/BogaerdtW19.pdf}{Lower Bounds for Uniform Machine Scheduling Using Decision Diagrams} & benchmark & 4 & n &  & n & - & \ref{a:BogaerdtW19} & \ref{b:BogaerdtW19}\\
\rowlabel{c:ColT19}ColT19 \href{https://doi.org/10.1007/978-3-030-30048-7_9}{ColT19}~\cite{ColT19} & \href{../works/ColT19.pdf}{Industrial Size Job Shop Scheduling Tackled by Present Day {CP} Solvers} & benchmark, github, real-world & 2 & \href{https://drive.google.com/drive/folders/1QuKEABR9aiNKPIFe0VMFXP7BNor8KW9b}{y} &  & \href{https://drive.google.com/drive/folders/1QuKEABR9aiNKPIFe0VMFXP7BNor8KW9b}{y} & - & \ref{a:ColT19} & \ref{b:ColT19}\\
\rowlabel{c:FrimodigS19}FrimodigS19 \href{https://doi.org/10.1007/978-3-030-30048-7_25}{FrimodigS19}~\cite{FrimodigS19} & \href{../works/FrimodigS19.pdf}{Models for Radiation Therapy Patient Scheduling} & benchmark, real-world & 1 & n &  & n & - & \ref{a:FrimodigS19} & \ref{b:FrimodigS19}\\
\rowlabel{c:GalleguillosKSB19}GalleguillosKSB19 \href{https://doi.org/10.1007/978-3-030-30048-7_26}{GalleguillosKSB19}~\cite{GalleguillosKSB19} & \href{../works/GalleguillosKSB19.pdf}{Constraint Programming-Based Job Dispatching for Modern {HPC} Applications} &  & 5 &  &  & \href{https://github.com/cgalleguillosm/cp_dispatchers}{y} &  & \ref{a:GalleguillosKSB19} & \ref{b:GalleguillosKSB19}\\
\rowlabel{c:MurinR19}MurinR19 \href{https://doi.org/10.1007/978-3-030-30048-7_27}{MurinR19}~\cite{MurinR19} & \href{../works/MurinR19.pdf}{Scheduling of Mobile Robots Using Constraint Programming} & github, benchmark, real-life & 3 & \href{https://github.com/StanislavMurin/Scheduling-of-Mobile-Robots-using-Constraint-Programming}{y} &  & \href{https://github.com/StanislavMurin/Scheduling-of-Mobile-Robots-using-Constraint-Programming}{y} &  & \ref{a:MurinR19} & \ref{b:MurinR19}\\
\rowlabel{c:SenderovichBB19}SenderovichBB19 \href{https://ojs.aaai.org/index.php/ICAPS/article/view/3504}{SenderovichBB19}~\cite{SenderovichBB19} & \href{../works/SenderovichBB19.pdf}{Learning Scheduling Models from Event Data} & github, benchmark, real-world & 0 &  &  &  &  & \ref{a:SenderovichBB19} & \ref{b:SenderovichBB19}\\
\rowlabel{c:BenediktSMVH18}BenediktSMVH18 \href{https://doi.org/10.1007/978-3-319-93031-2_6}{BenediktSMVH18}~\cite{BenediktSMVH18} & \href{../works/BenediktSMVH18.pdf}{Energy-Aware Production Scheduling with Power-Saving Modes} & generated instance, github, random instance & 1 & \href{https://github.com/CTU-IIG/PSPSM}{y} &  & \href{https://github.com/CTU-IIG/PSPSM}{y} & - & \ref{a:BenediktSMVH18} & \ref{b:BenediktSMVH18}\\
\rowlabel{c:CappartTSR18}CappartTSR18 \href{https://doi.org/10.1007/978-3-319-98334-9_32}{CappartTSR18}~\cite{CappartTSR18} & \href{../works/CappartTSR18.pdf}{A Constraint Programming Approach for Solving Patient Transportation Problems} & bitbucket, real-life, CSPlib & 1 &  &  &  &  & \ref{a:CappartTSR18} & \ref{b:CappartTSR18}\\
\rowlabel{c:He0GLW18}He0GLW18 \href{https://doi.org/10.1007/978-3-319-98334-9_42}{He0GLW18}~\cite{He0GLW18} & \href{../works/He0GLW18.pdf}{A Fast and Scalable Algorithm for Scheduling Large Numbers of Devices Under Real-Time Pricing} & real-world, bitbucket & 8 & \href{https://bitbucket.org/monash-dr/deterministic-rtp-ad/src/master/}{y} &  & \href{https://bitbucket.org/monash-dr/deterministic-rtp-ad/src/master/}{y} & - & \ref{a:He0GLW18} & \ref{b:He0GLW18}\\
\rowlabel{c:CappartS17}CappartS17 \href{https://doi.org/10.1007/978-3-319-59776-8_26}{CappartS17}~\cite{CappartS17} & \href{../works/CappartS17.pdf}{Rescheduling Railway Traffic on Real Time Situations Using Time-Interval Variables} & bitbucket, real-life, random instance & 1 & \href{https://bitbucket.org/qcappart/qcappart_opendata/src/master/}{y} &  & n & - & \ref{a:CappartS17} & \ref{b:CappartS17}\\
\rowlabel{c:GoldwaserS17}GoldwaserS17 \href{https://doi.org/10.1007/978-3-319-66158-2_22}{GoldwaserS17}~\cite{GoldwaserS17} & \href{../works/GoldwaserS17.pdf}{Optimal Torpedo Scheduling} & github, instance generator, generated instance & 4 & \href{https://github.com/AdGold/TorpedoSchedulingInstances}{y} &  & n & - & \ref{a:GoldwaserS17} & \ref{b:GoldwaserS17}\\
\rowlabel{c:LiuCGM17}LiuCGM17 \href{https://doi.org/10.1007/978-3-319-66158-2_24}{LiuCGM17}~\cite{LiuCGM17} & \href{../works/LiuCGM17.pdf}{NightSplitter: {A} Scheduling Tool to Optimize (Sub)group Activities} & github & 11 & n &  & \href{https://cs.unibo.it/t.liu/nightsplitter/mzn.html} & - & \ref{a:LiuCGM17} & \ref{b:LiuCGM17}\\
\rowlabel{c:YoungFS17}YoungFS17 \href{https://doi.org/10.1007/978-3-319-66158-2_20}{YoungFS17}~\cite{YoungFS17} & \href{../works/YoungFS17.pdf}{Constraint Programming Applied to the Multi-Skill Project Scheduling Problem} & benchmark, github, instance generator & 6 &  &  &  &  & \ref{a:YoungFS17} & \ref{b:YoungFS17}\\
\rowlabel{c:AmadiniGM16}AmadiniGM16 \href{http://dx.doi.org/10.1007/978-3-319-50349-3_16}{AmadiniGM16}~\cite{AmadiniGM16} & \href{../works/AmadiniGM16.pdf}{Parallelizing Constraint Solvers for Hard RCPSP Instances} & real-life, benchmark, github & 3 &  &  &  &  & \ref{a:AmadiniGM16} & \ref{b:AmadiniGM16}\\
\rowlabel{c:BonfiettiZLM16}BonfiettiZLM16 \href{https://doi.org/10.1007/978-3-319-44953-1_8}{BonfiettiZLM16}~\cite{BonfiettiZLM16} & \href{../works/BonfiettiZLM16.pdf}{The Multirate Resource Constraint} & generated instance, github, industrial instance, benchmark, real-world & 1 &  &  &  &  & \ref{a:BonfiettiZLM16} & \ref{b:BonfiettiZLM16}\\
\rowlabel{c:CauwelaertDMS16}CauwelaertDMS16 \href{https://doi.org/10.1007/978-3-319-44953-1_33}{CauwelaertDMS16}~\cite{CauwelaertDMS16} & \href{../works/CauwelaertDMS16.pdf}{Efficient Filtering for the Unary Resource with Family-Based Transition Times} & real-life, bitbucket, benchmark & 2 &  &  &  &  & \ref{a:CauwelaertDMS16} & \ref{b:CauwelaertDMS16}\\
\rowlabel{c:CauwelaertLS15}CauwelaertLS15 \href{https://doi.org/10.1007/978-3-319-18008-3_29}{CauwelaertLS15}~\cite{CauwelaertLS15} & \href{../works/CauwelaertLS15.pdf}{Understanding the Potential of Propagators} & benchmark, bitbucket & 0 &  &  &  &  & \ref{a:CauwelaertLS15} & \ref{b:CauwelaertLS15}\\
\rowlabel{c:DejemeppeCS15}DejemeppeCS15 \href{https://doi.org/10.1007/978-3-319-23219-5_7}{DejemeppeCS15}~\cite{DejemeppeCS15} & \href{../works/DejemeppeCS15.pdf}{The Unary Resource with Transition Times} & bitbucket, real-world, generated instance, benchmark & 4 &  &  &  &  & \ref{a:DejemeppeCS15} & \ref{b:DejemeppeCS15}\\
\rowlabel{c:GayHLS15}GayHLS15 \href{https://doi.org/10.1007/978-3-319-23219-5_10}{GayHLS15}~\cite{GayHLS15} & \href{../works/GayHLS15.pdf}{Conflict Ordering Search for Scheduling Problems} & bitbucket, benchmark & 0 &  &  &  &  & \ref{a:GayHLS15} & \ref{b:GayHLS15}\\
\rowlabel{c:GayHS15}GayHS15 \href{https://doi.org/10.1007/978-3-319-23219-5_11}{GayHS15}~\cite{GayHS15} & \href{../works/GayHS15.pdf}{Simple and Scalable Time-Table Filtering for the Cumulative Constraint} & bitbucket & 2 &  &  &  &  & \ref{a:GayHS15} & \ref{b:GayHS15}\\
\rowlabel{c:GayHS15a}GayHS15a \href{https://doi.org/10.1007/978-3-319-18008-3_11}{GayHS15a}~\cite{GayHS15a} & \href{../works/GayHS15a.pdf}{Time-Table Disjunctive Reasoning for the Cumulative Constraint} & real-world, benchmark, bitbucket & 0 &  &  &  &  & \ref{a:GayHS15a} & \ref{b:GayHS15a}\\
\rowlabel{c:SialaAH15}SialaAH15 \href{https://doi.org/10.1007/978-3-319-23219-5_28}{SialaAH15}~\cite{SialaAH15} & \href{../works/SialaAH15.pdf}{Two Clause Learning Approaches for Disjunctive Scheduling} & github, benchmark & 5 &  &  &  &  & \ref{a:SialaAH15} & \ref{b:SialaAH15}\\
\rowlabel{c:DejemeppeD14}DejemeppeD14 \href{https://doi.org/10.1007/978-3-319-07046-9_20}{DejemeppeD14}~\cite{DejemeppeD14} & \href{../works/DejemeppeD14.pdf}{Continuously Degrading Resource and Interval Dependent Activity Durations in Nuclear Medicine Patient Scheduling} & bitbucket & 0 &  &  &  &  & \ref{a:DejemeppeD14} & \ref{b:DejemeppeD14}\\
\rowlabel{c:HoundjiSWD14}HoundjiSWD14 \href{https://doi.org/10.1007/978-3-319-10428-7_29}{HoundjiSWD14}~\cite{HoundjiSWD14} & \href{../works/HoundjiSWD14.pdf}{The StockingCost Constraint} & bitbucket, generated instance & 0 &  &  &  &  & \ref{a:HoundjiSWD14} & \ref{b:HoundjiSWD14}\\
\rowlabel{c:CireCH13}CireCH13 \href{https://doi.org/10.1007/978-3-642-38171-3_22}{CireCH13}~\cite{CireCH13} & \href{../works/CireCH13.pdf}{Mixed Integer Programming vs. Logic-Based Benders Decomposition for Planning and Scheduling} &  & 1 & dead &  & n & - & \ref{a:CireCH13} & \ref{b:CireCH13}\\
\rowlabel{c:GuSS13}GuSS13 \href{https://doi.org/10.1007/978-3-642-38171-3_24}{GuSS13}~\cite{GuSS13} & \href{../works/GuSS13.pdf}{A Lagrangian Relaxation Based Forward-Backward Improvement Heuristic for Maximising the Net Present Value of Resource-Constrained Projects} & benchmark & 1 & dead &  &  & - & \ref{a:GuSS13} & \ref{b:GuSS13}\\
\rowlabel{c:KelarevaTK13}KelarevaTK13 \href{https://doi.org/10.1007/978-3-642-38171-3_8}{KelarevaTK13}~\cite{KelarevaTK13} & \href{../works/KelarevaTK13.pdf}{{CP} Methods for Scheduling and Routing with Time-Dependent Task Costs} & real-world & 5 & ref &  & - & - & \ref{a:KelarevaTK13} & \ref{b:KelarevaTK13}\\
\rowlabel{c:LetortCB13}LetortCB13 \href{https://doi.org/10.1007/978-3-642-38171-3_10}{LetortCB13}~\cite{LetortCB13} & \href{../works/LetortCB13.pdf}{A Synchronized Sweep Algorithm for the \emph{k-dimensional cumulative} Constraint} & benchmark, random instance, Roadef & 2 & PSPlib &  & - & - & \ref{a:LetortCB13} & \ref{b:LetortCB13}\\
\rowlabel{c:SchuttFS13a}SchuttFS13a \href{https://doi.org/10.1007/978-3-642-38171-3_16}{SchuttFS13a}~\cite{SchuttFS13a} & \href{../works/SchuttFS13a.pdf}{Explaining Time-Table-Edge-Finding Propagation for the Cumulative Resource Constraint} & benchmark & 5 & \su{PSPlib AT BL Pack KSD15D PackD} &  & - & - & \ref{a:SchuttFS13a} & \ref{b:SchuttFS13a}\\
\rowlabel{c:SimoninAHL12}SimoninAHL12 \href{https://doi.org/10.1007/978-3-642-33558-7_5}{SimoninAHL12}~\cite{SimoninAHL12} & \href{../works/SimoninAHL12.pdf}{Scheduling Scientific Experiments on the Rosetta/Philae Mission} &  & 0 & n &  & n & - & \ref{a:SimoninAHL12} & \ref{b:SimoninAHL12}\\
\end{longtable}
}



\clearpage
\section{Journal Articles}

\clearpage
\subsection{Articles from bibtex}
{\scriptsize
\begin{longtable}{>{\raggedright\arraybackslash}p{2.5cm}>{\raggedright\arraybackslash}p{4.5cm}>{\raggedright\arraybackslash}p{6.0cm}p{1.0cm}rr>{\raggedright\arraybackslash}p{2.0cm}r>{\raggedright\arraybackslash}p{1cm}p{1cm}p{1cm}p{1cm}}
\rowcolor{white}\caption{ARTICLE (Total 619)}\\ \toprule
\rowcolor{white}\shortstack{Key\\Source} & Authors & Title (Colored by Open Access)& \shortstack{Details\\LC} & Cite & Year & \shortstack{Conference\\/Journal\\/School} & Pages & Relevance &\shortstack{Cites\\OC XR\\SC} & \shortstack{Refs\\OC\\XR} & \shortstack{Links\\Cites\\Refs}\\ \midrule\endhead
\bottomrule
\endfoot
\index{ForbesHJST24}\rowlabel{a:ForbesHJST24}ForbesHJST24 \href{http://dx.doi.org/10.1016/j.ejor.2023.07.032}{ForbesHJST24} & \hyperref[auth:a983]{M. Forbes}, \hyperref[auth:a984]{M. Harris}, \hyperref[auth:a985]{H. Jansen}, \hyperref[auth:a986]{F. van der Schoot}, \hyperref[auth:a987]{T. Taimre} & \cellcolor{gold!20}Combining optimisation and simulation using logic-based Benders decomposition & \hyperref[detail:ForbesHJST24]{Details} \href{../works/ForbesHJST24.pdf}{Yes} & \cite{ForbesHJST24} & 2024 & European Journal of Operational Research & 15 & \noindent{}\textcolor{black!50}{0.00} \textcolor{black!50}{0.00} \textbf{4.07} & 0 0 0 & 26 37 & 9 0 9\\
\index{LuZZYW24}\rowlabel{a:LuZZYW24}LuZZYW24 \href{https://www.mdpi.com/2077-1312/12/1/124}{LuZZYW24} & \hyperref[auth:a1250]{X. Lu}, \hyperref[auth:a1251]{Y. Zhang}, \hyperref[auth:a1252]{L. Zheng}, \hyperref[auth:a1253]{C. Yang}, \hyperref[auth:a1254]{J. Wang} & \cellcolor{gold!20}Integrated Inbound and Outbound Scheduling for Coal Port: Constraint Programming and Adaptive Local Search & \hyperref[detail:LuZZYW24]{Details} \href{../works/LuZZYW24.pdf}{Yes} & \cite{LuZZYW24} & 2024 & Journal of Marine Science and Engineering & 36 & \noindent{}\textbf{1.00} \textbf{1.00} \textbf{71.08} & 0 0 0 & 0 57 & 0 0 0\\
\index{PrataAN23}\rowlabel{a:PrataAN23}PrataAN23 \href{https://www.sciencedirect.com/science/article/pii/S2666720723001522}{PrataAN23} & \hyperref[auth:a385]{B. A. Prata}, \hyperref[auth:a386]{L. R. Abreu}, \hyperref[auth:a387]{M. S. Nagano} & \cellcolor{gold!20}Applications of constraint programming in production scheduling problems: A descriptive bibliometric analysis \hyperref[abs:PrataAN23]{Abstract} & \hyperref[detail:PrataAN23]{Details} \href{../works/PrataAN23.pdf}{Yes} & \cite{PrataAN23} & 2024 & Results in Control and Optimization & 17 & \noindent{}\textbf{1.00} \textbf{1.00} \textbf{54.10} & 0 0 0 & 0 149 & 0 0 0\\
\index{abs-2402-00459}\rowlabel{a:abs-2402-00459}abs-2402-00459 \href{https://doi.org/10.48550/arXiv.2402.00459}{abs-2402-00459} & \hyperref[auth:a395]{S. Nguyen}, \hyperref[auth:a396]{D. R. Thiruvady}, \hyperref[auth:a397]{Y. Sun}, \hyperref[auth:a398]{M. Zhang} & Genetic-based Constraint Programming for Resource Constrained Job Scheduling & \hyperref[detail:abs-2402-00459]{Details} \href{../works/abs-2402-00459.pdf}{Yes} & \cite{abs-2402-00459} & 2024 & CoRR & 21 & \noindent{}\textbf{2.50} \textbf{2.50} \textbf{10.36} & 0 0 0 & 0 0 & 0 0 0\\
\index{AbreuNP23}\rowlabel{a:AbreuNP23}AbreuNP23 \href{https://doi.org/10.1080/00207543.2022.2154404}{AbreuNP23} & \hyperref[auth:a418]{L. R. de Abreu}, \hyperref[auth:a419]{M. S. Nagano}, \hyperref[auth:a385]{B. A. Prata} & A new two-stage constraint programming approach for open shop scheduling problem with machine blocking & \hyperref[detail:AbreuNP23]{Details} \href{../works/AbreuNP23.pdf}{Yes} & \cite{AbreuNP23} & 2023 & \cellcolor{red!20}International Journal of Production Research & 20 & \noindent{}\textbf{1.50} \textbf{1.50} \textbf{44.59} & 1 2 0 & 47 54 & 12 1 11\\
\index{AbreuPNF23}\rowlabel{a:AbreuPNF23}AbreuPNF23 \href{https://www.sciencedirect.com/science/article/pii/S0305054823002502}{AbreuPNF23} & \hyperref[auth:a386]{L. R. Abreu}, \hyperref[auth:a385]{B. A. Prata}, \hyperref[auth:a387]{M. S. Nagano}, \hyperref[auth:a833]{J. M. Framinan} & A constraint programming-based iterated greedy algorithm for the open shop with sequence-dependent processing times and makespan minimization \hyperref[abs:AbreuPNF23]{Abstract} & \hyperref[detail:AbreuPNF23]{Details} \href{../works/AbreuPNF23.pdf}{Yes} & \cite{AbreuPNF23} & 2023 & Computers \  Operations Research & 12 & \noindent{}\textcolor{black!50}{0.00} \textbf{5.00} \textbf{28.79} & 0 3 3 & 46 68 & 15 0 15\\
\index{Adelgren2023}\rowlabel{a:Adelgren2023}Adelgren2023 \href{http://dx.doi.org/10.1016/j.cie.2023.109330}{Adelgren2023} & \hyperref[auth:a967]{N. Adelgren}, \hyperref[auth:a381]{C. T. Maravelias} & On the utility of production scheduling formulations including record keeping variables & \hyperref[detail:Adelgren2023]{Details} \href{../works/Adelgren2023.pdf}{Yes} & \cite{Adelgren2023} & 2023 & Computers \  Industrial Engineering & 12 & \noindent{}\textcolor{black!50}{0.00} \textcolor{black!50}{0.00} \textbf{1.69} & 0 1 1 & 43 52 & 11 0 11\\
\index{AfsarVPG23}\rowlabel{a:AfsarVPG23}AfsarVPG23 \href{http://dx.doi.org/10.1016/j.cie.2023.109454}{AfsarVPG23} & \hyperref[auth:a961]{S. Afsar}, \hyperref[auth:a962]{C. R. Vela}, \hyperref[auth:a963]{J. J. Palacios}, \hyperref[auth:a964]{I. González-Rodríguez} & \cellcolor{gold!20}Mathematical models and benchmarking for the fuzzy job shop scheduling problem & \hyperref[detail:AfsarVPG23]{Details} \href{../works/AfsarVPG23.pdf}{Yes} & \cite{AfsarVPG23} & 2023 & Computers \  Industrial Engineering & 14 & \noindent{}\textcolor{black!50}{0.00} \textcolor{black!50}{0.00} \textbf{22.09} & 0 0 0 & 50 66 & 7 0 7\\
\index{Ahmadi-Javid2023}\rowlabel{a:Ahmadi-Javid2023}Ahmadi-Javid2023 \href{http://dx.doi.org/10.1080/00207543.2023.2230489}{Ahmadi-Javid2023} & \hyperref[auth:a1762]{A. Ahmadi-Javid}, \hyperref[auth:a1763]{M. Haghi}, \hyperref[auth:a1764]{P. Hooshangi-Tabrizi} & Integrated job-shop scheduling in an FMS with heterogeneous transporters: MILP formulation, constraint programming, and branch-and-bound & \hyperref[detail:Ahmadi-Javid2023]{Details} No & \cite{Ahmadi-Javid2023} & 2023 & \cellcolor{red!20}International Journal of Production Research & null & \noindent{}\textbf{2.00} \textbf{2.00} n/a & 0 0 0 & 66 74 & 3 0 3\\
\index{Akan2023}\rowlabel{a:Akan2023}Akan2023 \href{http://dx.doi.org/10.33714/masteb.1324266}{Akan2023} & \hyperref[auth:a1751]{E. Akan}, \hyperref[auth:a1752]{G. Alkan} & Optimizing Shipbuilding Production Project Scheduling Under Resource Constraints Using Genetic Algorithms and Fuzzy Sets \hyperref[abs:Akan2023]{Abstract} & \hyperref[detail:Akan2023]{Details} No & \cite{Akan2023} & 2023 & Marine Science and Technology Bulletin & null & \noindent{}\textcolor{black!50}{0.00} \textcolor{black!50}{0.00} n/a & 0 0 0 & 132 148 & 6 0 6\\
\index{AkramNHRSA23}\rowlabel{a:AkramNHRSA23}AkramNHRSA23 \href{https://doi.org/10.1109/ACCESS.2023.3343409}{AkramNHRSA23} & \hyperref[auth:a399]{B. O. Akram}, \hyperref[auth:a400]{N. K. Noordin}, \hyperref[auth:a401]{F. Hashim}, \hyperref[auth:a402]{M. F. A. Rasid}, \hyperref[auth:a403]{M. I. Salman}, \hyperref[auth:a404]{A. M. Abdulghani} & \cellcolor{gold!20}Joint Scheduling and Routing Optimization for Deterministic Hybrid Traffic in Time-Sensitive Networks Using Constraint Programming & \hyperref[detail:AkramNHRSA23]{Details} \href{../works/AkramNHRSA23.pdf}{Yes} & \cite{AkramNHRSA23} & 2023 & {IEEE} Access & 16 & \noindent{}\textbf{1.00} \textbf{1.00} \textbf{7.96} & 0 0 0 & 0 37 & 0 0 0\\
\index{AlakaP23}\rowlabel{a:AlakaP23}AlakaP23 \href{http://dx.doi.org/10.1007/s00500-023-09105-9}{AlakaP23} & \hyperref[auth:a764]{H. M. Alakaş}, \hyperref[auth:a1384]{M. Pınarbaşı} & \cellcolor{green!10}Balancing of cost-oriented U-type general resource-constrained assembly line: new constraint programming models & \hyperref[detail:AlakaP23]{Details} \href{../works/AlakaP23.pdf}{Yes} & \cite{AlakaP23} & 2023 & Soft Computing & 14 & \noindent{}0.50 0.50 \textbf{23.23} & 0 0 0 & 35 42 & 14 0 14\\
\index{AlfieriGPS23}\rowlabel{a:AlfieriGPS23}AlfieriGPS23 \href{https://www.sciencedirect.com/science/article/pii/S0360835223000074}{AlfieriGPS23} & \hyperref[auth:a729]{A. Alfieri}, \hyperref[auth:a15]{M. Garraffa}, \hyperref[auth:a730]{E. Pastore}, \hyperref[auth:a731]{F. Salassa} & \cellcolor{gold!20}Permutation flowshop problems minimizing core waiting time and core idle time \hyperref[abs:AlfieriGPS23]{Abstract} & \hyperref[detail:AlfieriGPS23]{Details} \href{../works/AlfieriGPS23.pdf}{Yes} & \cite{AlfieriGPS23} & 2023 & Computers \  Industrial Engineering & 13 & \noindent{}\textcolor{black!50}{0.00} \textbf{2.00} \textbf{5.78} & 0 2 3 & 37 45 & 2 0 2\\
\index{Bocewicz2023}\rowlabel{a:Bocewicz2023}Bocewicz2023 \href{http://dx.doi.org/10.3390/app13127165}{Bocewicz2023} & \hyperref[auth:a630]{G. Bocewicz}, \hyperref[auth:a1997]{E. Szwarc}, \hyperref[auth:a2016]{A. Thibbotuwawa}, \hyperref[auth:a1814]{Z. Banaszak} & \cellcolor{gold!20}Project Portfolio Planning Taking into Account the Effect of Loss of Competences of Project Team Members \hyperref[abs:Bocewicz2023]{Abstract} & \hyperref[detail:Bocewicz2023]{Details} No & \cite{Bocewicz2023} & 2023 & Applied Sciences & null & \noindent{}\textcolor{black!50}{0.00} \textbf{1.50} n/a & 0 0 0 & 41 47 & 1 0 1\\
\index{Caballero23}\rowlabel{a:Caballero23}Caballero23 \href{https://doi.org/10.1007/s10601-023-09357-0}{Caballero23} & \hyperref[auth:a102]{J. C. Caballero} & Scheduling through logic-based tools & \hyperref[detail:Caballero23]{Details} \href{../works/Caballero23.pdf}{Yes} & \cite{Caballero23} & 2023 & Constraints An Int. J. & 1 & \noindent{}\textcolor{black!50}{0.00} \textcolor{black!50}{0.00} \textcolor{black!50}{0.00} & 0 0 0 & 0 0 & 0 0 0\\
\index{CzerniachowskaWZ23}\rowlabel{a:CzerniachowskaWZ23}CzerniachowskaWZ23 \href{https://doi.org/10.12913/22998624/166588}{CzerniachowskaWZ23} & \hyperref[auth:a732]{K. Czerniachowska}, \hyperref[auth:a733]{R. Wichniarek}, \hyperref[auth:a734]{K. Żywicki} & \cellcolor{gold!20}Constraint Programming for Flexible Flow Shop Scheduling Problem with Repeated Jobs and Repeated Operations \hyperref[abs:CzerniachowskaWZ23]{Abstract} & \hyperref[detail:CzerniachowskaWZ23]{Details} \href{../works/CzerniachowskaWZ23.pdf}{Yes} & \cite{CzerniachowskaWZ23} & 2023 & Advances in Science and Technology Research Journal & 14 & \noindent{}\textbf{2.00} \textbf{2.00} \textbf{8.41} & 0 0 0 & 0 0 & 0 0 0\\
\index{Danzinger2023}\rowlabel{a:Danzinger2023}Danzinger2023 \href{http://dx.doi.org/10.1145/3546871}{Danzinger2023} & \hyperref[auth:a1484]{P. Danzinger}, \hyperref[auth:a77]{T. Geibinger}, \hyperref[auth:a1485]{D. Janneau}, \hyperref[auth:a80]{F. Mischek}, \hyperref[auth:a45]{N. Musliu}, \hyperref[auth:a1486]{C. Poschalko} & A System for Automated Industrial Test Laboratory Scheduling \hyperref[abs:Danzinger2023]{Abstract} & \hyperref[detail:Danzinger2023]{Details} No & \cite{Danzinger2023} & 2023 & ACM Transactions on Intelligent Systems and Technology & null & \noindent{}\textcolor{black!50}{0.00} \textbf{5.00} n/a & 0 1 1 & 19 26 & 10 0 10\\
\index{Dimny2023}\rowlabel{a:Dimny2023}Dimny2023 \href{http://dx.doi.org/10.1007/s10100-023-00885-x}{Dimny2023} & \hyperref[auth:a1487]{I. Dimény}, \hyperref[auth:a1488]{T. Koltai} & \cellcolor{gold!20}Comparison of MILP and CP models for balancing partially automated assembly lines \hyperref[abs:Dimny2023]{Abstract} & \hyperref[detail:Dimny2023]{Details} No & \cite{Dimny2023} & 2023 & Central European Journal of Operations Research & null & \noindent{}\textcolor{black!50}{0.00} \textbf{5.00} n/a & 0 0 0 & 35 37 & 3 0 3\\
\index{Eiter2023}\rowlabel{a:Eiter2023}Eiter2023 \href{http://dx.doi.org/10.1017/s1471068423000017}{Eiter2023} & \hyperref[auth:a1960]{T. Eiter}, \hyperref[auth:a77]{T. Geibinger}, \hyperref[auth:a45]{N. Musliu}, \hyperref[auth:a1961]{J. Oetsch}, \hyperref[auth:a1962]{P. Skočovský}, \hyperref[auth:a1963]{D. Stepanova} & \cellcolor{gold!20}Answer-Set Programming for Lexicographical Makespan Optimisation in Parallel Machine Scheduling \hyperref[abs:Eiter2023]{Abstract} & \hyperref[detail:Eiter2023]{Details} No & \cite{Eiter2023} & 2023 & Theory and Practice of Logic Programming & null & \noindent{}\textcolor{black!50}{0.00} \textbf{6.01} n/a & 0 1 0 & 27 34 & 3 0 3\\
\index{FahimiQ23}\rowlabel{a:FahimiQ23}FahimiQ23 \href{http://dx.doi.org/10.1287/ijoc.2021.0138}{FahimiQ23} & \hyperref[auth:a122]{H. Fahimi}, \hyperref[auth:a37]{C.-G. Quimper} & Overload-Checking and Edge-Finding for Robust Cumulative Scheduling & \hyperref[detail:FahimiQ23]{Details} No & \cite{FahimiQ23} & 2023 & \cellcolor{red!20}INFORMS Journal on Computing & 20 & \noindent{}\textcolor{black!50}{0.00} \textcolor{black!50}{0.00} n/a & 0 0 0 & 16 21 & 8 0 8\\
\index{Fatemi-AnarakiTFV23}\rowlabel{a:Fatemi-AnarakiTFV23}Fatemi-AnarakiTFV23 \href{http://dx.doi.org/10.1016/j.omega.2022.102770}{Fatemi-AnarakiTFV23} & \hyperref[auth:a735]{S. Fatemi-Anaraki}, \hyperref[auth:a430]{R. Tavakkoli-Moghaddam}, \hyperref[auth:a736]{M. Foumani}, \hyperref[auth:a737]{B. Vahedi-Nouri} & Scheduling of Multi-Robot Job Shop Systems in Dynamic Environments: Mixed-Integer Linear Programming and Constraint Programming Approaches & \hyperref[detail:Fatemi-AnarakiTFV23]{Details} \href{../works/Fatemi-AnarakiTFV23.pdf}{Yes} & \cite{Fatemi-AnarakiTFV23} & 2023 & Omega & 15 & \noindent{}\textbf{2.00} \textbf{2.00} \textbf{26.16} & 7 14 16 & 60 66 & 15 0 15\\
\index{FrimodigECM23}\rowlabel{a:FrimodigECM23}FrimodigECM23 \href{https://doi.org/10.1007/s43069-023-00251-2}{FrimodigECM23} & \hyperref[auth:a95]{S. Frimodig}, \hyperref[auth:a1414]{P. Enqvist}, \hyperref[auth:a91]{M. Carlsson}, \hyperref[auth:a1415]{C. Mercier} & \cellcolor{gold!20}Comparing Optimization Methods for Radiation Therapy Patient Scheduling using Different Objectives & \hyperref[detail:FrimodigECM23]{Details} \href{../works/FrimodigECM23.pdf}{Yes} & \cite{FrimodigECM23} & 2023 & Oper. Res. Forum & 38 & \noindent{}\textcolor{black!50}{0.00} \textcolor{black!50}{0.00} \textbf{11.72} & 0 0 0 & 0 56 & 0 0 0\\
\index{GhasemiMH23}\rowlabel{a:GhasemiMH23}GhasemiMH23 \href{http://dx.doi.org/10.1080/23302674.2023.2224509}{GhasemiMH23} & \hyperref[auth:a981]{S. Ghasemi}, \hyperref[auth:a430]{R. Tavakkoli-Moghaddam}, \hyperref[auth:a982]{M. Hamid} & Operating room scheduling by emphasising human factors and dynamic decision-making styles: a constraint programming method & \hyperref[detail:GhasemiMH23]{Details} No & \cite{GhasemiMH23} & 2023 & \cellcolor{red!20}International Journal of Systems Science: Operations \  Logistics & null & \noindent{}\textbf{1.00} \textbf{1.00} n/a & 0 0 1 & 104 130 & 16 0 16\\
\index{GokPTGO23}\rowlabel{a:GokPTGO23}GokPTGO23 \href{https://ideas.repec.org/a/spr/annopr/v320y2023i2d10.1007_s10479-022-04547-0.html}{GokPTGO23} & \hyperref[auth:a1009]{Y. S. G{\"{o}}k}, \hyperref[auth:a1010]{S. Padr{\'{o}}n}, \hyperref[auth:a1011]{M. Tomasella}, \hyperref[auth:a1012]{D. Guimarans}, \hyperref[auth:a135]{C. {\"{O}}zt{\"{u}}rk} & {Constraint-based robust planning and scheduling of airport apron operations through simheuristics} & \hyperref[detail:GokPTGO23]{Details} \href{../works/GokPTGO23.pdf}{Yes} & \cite{GokPTGO23} & 2023 & Annals of Operations Research & 36 & \noindent{}\textcolor{black!50}{0.00} \textcolor{black!50}{0.00} \textbf{5.61} & 0 0 0 & 0 0 & 0 0 0\\
\index{GunerGSKD23}\rowlabel{a:GunerGSKD23}GunerGSKD23 \href{http://dx.doi.org/10.1080/00207543.2023.2226772}{GunerGSKD23} & \hyperref[auth:a1426]{F. G\"{u}ner}, \hyperref[auth:a1427]{A. K. G\"{o}r\"{u}r}, \hyperref[auth:a1428]{B. Satır}, \hyperref[auth:a1429]{L. Kandiller}, \hyperref[auth:a1430]{J. H. Drake} & A constraint programming approach to a real-world workforce scheduling problem for multi-manned assembly lines with sequence-dependent setup times & \hyperref[detail:GunerGSKD23]{Details} No & \cite{GunerGSKD23} & 2023 & \cellcolor{red!20}International Journal of Production Research & 18 & \noindent{}\textbf{1.00} \textbf{1.00} n/a & 0 3 0 & 43 46 & 7 0 7\\
\index{GuoZ23}\rowlabel{a:GuoZ23}GuoZ23 \href{http://dx.doi.org/10.1016/j.ejor.2023.06.006}{GuoZ23} & \hyperref[auth:a943]{P. Guo}, \hyperref[auth:a944]{J. Zhu} & Capacity reservation for humanitarian relief: A logic-based Benders decomposition method with subgradient cut & \hyperref[detail:GuoZ23]{Details} \href{../works/GuoZ23.pdf}{Yes} & \cite{GuoZ23} & 2023 & European Journal of Operational Research & 29 & \noindent{}\textcolor{black!50}{0.00} \textcolor{black!50}{0.00} 0.20 & 0 1 1 & 112 145 & 18 0 18\\
\index{GurPAE23}\rowlabel{a:GurPAE23}GurPAE23 \href{https://doi.org/10.1007/s10100-022-00835-z}{GurPAE23} & \hyperref[auth:a412]{S. G{\"{u}}r}, \hyperref[auth:a413]{M. Pinarbasi}, \hyperref[auth:a414]{H. M. Alakas}, \hyperref[auth:a415]{T. Eren} & Operating room scheduling with surgical team: a new approach with constraint programming and goal programming & \hyperref[detail:GurPAE23]{Details} \href{../works/GurPAE23.pdf}{Yes} & \cite{GurPAE23} & 2023 & Central Eur. J. Oper. Res. & 25 & \noindent{}\textbf{1.00} \textbf{1.00} \textbf{4.04} & 1 5 3 & 40 46 & 4 1 3\\
\index{Hajji2023}\rowlabel{a:Hajji2023}Hajji2023 \href{http://dx.doi.org/10.3390/computation11070137}{Hajji2023} & \hyperref[auth:a1537]{M. K. Hajji}, \hyperref[auth:a1538]{H. Hadda}, \hyperref[auth:a1539]{N. Dridi} & \cellcolor{gold!20}Makespan Minimization for the Two-Stage Hybrid Flow Shop Problem with Dedicated Machines: A Comprehensive Study of Exact and Heuristic Approaches \hyperref[abs:Hajji2023]{Abstract} & \hyperref[detail:Hajji2023]{Details} No & \cite{Hajji2023} & 2023 & Computation & null & \noindent{}\textcolor{black!50}{0.00} \textbf{2.50} n/a & 0 0 0 & 26 42 & 4 0 4\\
\index{IsikYA23}\rowlabel{a:IsikYA23}IsikYA23 \href{https://doi.org/10.1007/s00500-023-09086-9}{IsikYA23} & \hyperref[auth:a420]{E. E. Isik}, \hyperref[auth:a421]{S. T. Yildiz}, \hyperref[auth:a422]{{\"{O}}zge S. Akpunar} & Constraint programming models for the hybrid flow shop scheduling problem and its extensions & \hyperref[detail:IsikYA23]{Details} \href{../works/IsikYA23.pdf}{Yes} & \cite{IsikYA23} & 2023 & Soft Computing & 28 & \noindent{}\textbf{1.00} \textbf{1.00} \textbf{80.43} & 0 2 2 & 127 141 & 12 0 12\\
\index{JuvinHL23a}\rowlabel{a:JuvinHL23a}JuvinHL23a \href{http://dx.doi.org/10.1016/j.cor.2023.106156}{JuvinHL23a} & \hyperref[auth:a0]{C. Juvin}, \hyperref[auth:a2]{L. Houssin}, \hyperref[auth:a3]{P. Lopez} & \cellcolor{green!10}Logic-based Benders decomposition for the preemptive flexible job-shop scheduling problem & \hyperref[detail:JuvinHL23a]{Details} \href{../works/JuvinHL23a.pdf}{Yes} & \cite{JuvinHL23a} & 2023 & Computers \  Operations Research & 17 & \noindent{}\textcolor{black!50}{0.00} \textcolor{black!50}{0.00} \textbf{9.23} & 0 3 4 & 40 53 & 15 0 15\\
\index{Kasapidis2023}\rowlabel{a:Kasapidis2023}Kasapidis2023 \href{http://dx.doi.org/10.1111/poms.13977}{Kasapidis2023} & \hyperref[auth:a1503]{G. A. Kasapidis}, \hyperref[auth:a1716]{S. Dauzère‐Pérès}, \hyperref[auth:a1504]{D. C. Paraskevopoulos}, \hyperref[auth:a1505]{P. P. Repoussis}, \hyperref[auth:a1506]{C. D. Tarantilis} & \cellcolor{gold!20}On the multiresource flexible job‐shop scheduling problem with arbitrary precedence graphs \hyperref[abs:Kasapidis2023]{Abstract} & \hyperref[detail:Kasapidis2023]{Details} No & \cite{Kasapidis2023} & 2023 & \cellcolor{red!20}Production and Operations Management & null & \noindent{}\textcolor{black!50}{0.00} \textbf{5.00} n/a & 1 4 4 & 13 13 & 2 0 2\\
\index{LacknerMMWW23}\rowlabel{a:LacknerMMWW23}LacknerMMWW23 \href{https://doi.org/10.1007/s10601-023-09347-2}{LacknerMMWW23} & \hyperref[auth:a62]{M.-L. Lackner}, \hyperref[auth:a63]{C. Mrkvicka}, \hyperref[auth:a45]{N. Musliu}, \hyperref[auth:a46]{D. Walkiewicz}, \hyperref[auth:a43]{F. Winter} & \cellcolor{gold!20}Exact methods for the Oven Scheduling Problem & \hyperref[detail:LacknerMMWW23]{Details} \href{../works/LacknerMMWW23.pdf}{Yes} & \cite{LacknerMMWW23} & 2023 & Constraints An Int. J. & 42 & \noindent{}\textcolor{black!50}{0.00} \textcolor{black!50}{0.00} \textbf{46.45} & 0 1 0 & 32 38 & 8 0 8\\
\index{Liu2023}\rowlabel{a:Liu2023}Liu2023 \href{http://dx.doi.org/10.3390/buildings13071867}{Liu2023} & \hyperref[auth:a1244]{S.-S. Liu}, \hyperref[auth:a1718]{P. Utami}, \hyperref[auth:a1719]{A. Budiwirawan}, \hyperref[auth:a1489]{M. F. A. Arifin}, \hyperref[auth:a1720]{F. S. Perdana} & \cellcolor{gold!20}Optimization Model of Maintenance Scheduling Problem for Heritage Buildings with Constraint Programming \hyperref[abs:Liu2023]{Abstract} & \hyperref[detail:Liu2023]{Details} No & \cite{Liu2023} & 2023 & Buildings & null & \noindent{}\textbf{1.00} \textbf{4.01} n/a & 0 1 1 & 48 55 & 2 0 2\\
\index{Lyons2023}\rowlabel{a:Lyons2023}Lyons2023 \href{http://dx.doi.org/10.3390/analytics2030036}{Lyons2023} & \hyperref[auth:a1524]{J. S. F. Lyons}, \hyperref[auth:a836]{M. A. Begen}, \hyperref[auth:a1525]{P. C. Bell} & Surgery Scheduling and Perioperative Care: Smoothing and Visualizing Elective Surgery and Recovery Patient Flow \hyperref[abs:Lyons2023]{Abstract} & \hyperref[detail:Lyons2023]{Details} No & \cite{Lyons2023} & 2023 & Analytics & null & \noindent{}\textcolor{black!50}{0.00} \textbf{3.00} n/a & 0 0 0 & 23 29 & 4 0 4\\
\index{MarliereSPR23}\rowlabel{a:MarliereSPR23}MarliereSPR23 \href{https://www.sciencedirect.com/science/article/pii/S0967066122002611}{MarliereSPR23} & \hyperref[auth:a1018]{G. Marlière}, \hyperref[auth:a1019]{S. {Sobieraj Richard}}, \hyperref[auth:a1020]{P. Pellegrini}, \hyperref[auth:a781]{J. Rodriguez} & \cellcolor{green!10}A conditional time-intervals formulation of the real-time Railway Traffic Management Problem & \hyperref[detail:MarliereSPR23]{Details} \href{../works/MarliereSPR23.pdf}{Yes} & \cite{MarliereSPR23} & 2023 & Control Engineering Practice & 22 & \noindent{}\textcolor{black!50}{0.00} \textcolor{black!50}{0.00} \textbf{12.23} & 1 3 4 & 75 101 & 6 0 6\\
\index{MontemanniD23}\rowlabel{a:MontemanniD23}MontemanniD23 \href{https://doi.org/10.3390/a16010040}{MontemanniD23} & \hyperref[auth:a410]{R. Montemanni}, \hyperref[auth:a411]{M. Dell'Amico} & \cellcolor{gold!20}Solving the Parallel Drone Scheduling Traveling Salesman Problem via Constraint Programming & \hyperref[detail:MontemanniD23]{Details} \href{../works/MontemanniD23.pdf}{Yes} & \cite{MontemanniD23} & 2023 & Algorithms & 13 & \noindent{}\textbf{1.00} \textbf{1.00} \textbf{1.09} & 2 7 8 & 18 21 & 1 1 0\\
\index{MontemanniD23a}\rowlabel{a:MontemanniD23a}MontemanniD23a \href{https://doi.org/10.1016/j.ejco.2023.100078}{MontemanniD23a} & \hyperref[auth:a410]{R. Montemanni}, \hyperref[auth:a411]{M. Dell'Amico} & \cellcolor{gold!20}Constraint programming models for the parallel drone scheduling vehicle routing problem & \hyperref[detail:MontemanniD23a]{Details} \href{../works/MontemanniD23a.pdf}{Yes} & \cite{MontemanniD23a} & 2023 & {EURO} J. Comput. Optim. & 20 & \noindent{}\textbf{1.00} \textbf{1.00} 0.64 & 0 1 1 & 14 19 & 1 0 1\\
\index{NaderiBZ23}\rowlabel{a:NaderiBZ23}NaderiBZ23 \href{http://dx.doi.org/10.2139/ssrn.4494381}{NaderiBZ23} & \hyperref[auth:a726]{B. Naderi}, \hyperref[auth:a836]{M. A. Begen}, \hyperref[auth:a837]{G. Zhang} & Integrated Order Acceptance and Resource Decisions Under Uncertainty: Robust and Stochastic Approaches & \hyperref[detail:NaderiBZ23]{Details} \href{../works/NaderiBZ23.pdf}{Yes} & \cite{NaderiBZ23} & 2023 & SSRN Electronic Journal & 32 & \noindent{}\textcolor{black!50}{0.00} \textcolor{black!50}{0.00} \textbf{10.49} & 0 0 0 & 46 56 & 12 0 12\\
\index{NaderiBZR23}\rowlabel{a:NaderiBZR23}NaderiBZR23 \href{http://dx.doi.org/10.1016/j.omega.2022.102805}{NaderiBZR23} & \hyperref[auth:a726]{B. Naderi}, \hyperref[auth:a836]{M. A. Begen}, \hyperref[auth:a838]{G. S. Zaric}, \hyperref[auth:a728]{V. Roshanaei} & A novel and efficient exact technique for integrated staffing, assignment, routing, and scheduling of home care services under uncertainty & \hyperref[detail:NaderiBZR23]{Details} \href{../works/NaderiBZR23.pdf}{Yes} & \cite{NaderiBZR23} & 2023 & Omega & 15 & \noindent{}\textcolor{black!50}{0.00} \textcolor{black!50}{0.00} \textbf{1.01} & 4 6 6 & 64 80 & 12 0 12\\
\index{NaderiRR23}\rowlabel{a:NaderiRR23}NaderiRR23 \href{https://doi.org/10.1287/ijoc.2023.1287}{NaderiRR23} & \hyperref[auth:a726]{B. Naderi}, \hyperref[auth:a727]{R. Ruiz}, \hyperref[auth:a728]{V. Roshanaei} & Mixed-Integer Programming vs. Constraint Programming for Shop Scheduling Problems: New Results and Outlook & \hyperref[detail:NaderiRR23]{Details} \href{../works/NaderiRR23.pdf}{Yes} & \cite{NaderiRR23} & 2023 & \cellcolor{red!20}INFORMS Journal on Computing & 27 & \noindent{}\textbf{1.00} \textbf{1.00} \textbf{184.97} & 2 7 7 & 50 55 & 22 1 21\\
\index{NouriMHD23}\rowlabel{a:NouriMHD23}NouriMHD23 \href{http://dx.doi.org/10.1080/00207543.2023.2173503}{NouriMHD23} & \hyperref[auth:a737]{B. Vahedi-Nouri}, \hyperref[auth:a430]{R. Tavakkoli-Moghaddam}, \hyperref[auth:a946]{Z. Hanzálek}, \hyperref[auth:a947]{A. Dolgui} & Production scheduling in a reconfigurable manufacturing system benefiting from human-robot collaboration & \hyperref[detail:NouriMHD23]{Details} No & \cite{NouriMHD23} & 2023 & \cellcolor{red!20}International Journal of Production Research & 17 & \noindent{}\textcolor{black!50}{0.00} \textcolor{black!50}{0.00} n/a & 2 6 5 & 44 49 & 11 0 11\\
\index{Oujana2023}\rowlabel{a:Oujana2023}Oujana2023 \href{http://dx.doi.org/10.3390/app13106003}{Oujana2023} & \hyperref[auth:a453]{S. Oujana}, \hyperref[auth:a454]{L. Amodeo}, \hyperref[auth:a455]{F. Yalaoui}, \hyperref[auth:a1477]{D. Brodart} & \cellcolor{gold!20}Mixed-Integer Linear Programming, Constraint Programming and a Novel Dedicated Heuristic for Production Scheduling in a Packaging Plant \hyperref[abs:Oujana2023]{Abstract} & \hyperref[detail:Oujana2023]{Details} No & \cite{Oujana2023} & 2023 & Applied Sciences & null & \noindent{}\textbf{1.00} \textbf{7.01} n/a & 3 3 3 & 55 57 & 4 1 3\\
\index{PenzDN23}\rowlabel{a:PenzDN23}PenzDN23 \href{https://doi.org/10.1016/j.cor.2022.106092}{PenzDN23} & \hyperref[auth:a992]{L. Penz}, \hyperref[auth:a993]{S. Dauz{\`{e}}re-P{\'{e}}r{\`{e}}s}, \hyperref[auth:a81]{M. Nattaf} & \cellcolor{gold!20}Minimizing the sum of completion times on a single machine with health index and flexible maintenance operations & \hyperref[detail:PenzDN23]{Details} \href{../works/PenzDN23.pdf}{Yes} & \cite{PenzDN23} & 2023 & Computers \  Operations Research & 13 & \noindent{}\textcolor{black!50}{0.00} \textcolor{black!50}{0.00} \textcolor{black!50}{0.00} & 0 3 1 & 34 36 & 1 0 1\\
\index{Ramos2023}\rowlabel{a:Ramos2023}Ramos2023 \href{http://dx.doi.org/10.3390/math11020337}{Ramos2023} & \hyperref[auth:a1731]{A. S. Ramos}, \hyperref[auth:a1732]{P. A. Miranda-Gonzalez}, \hyperref[auth:a1733]{S. Nucamendi-Guillén}, \hyperref[auth:a1734]{E. Olivares-Benitez} & \cellcolor{gold!20}A Formulation for the Stochastic Multi-Mode Resource-Constrained Project Scheduling Problem Solved with a Multi-Start Iterated Local Search Metaheuristic \hyperref[abs:Ramos2023]{Abstract} & \hyperref[detail:Ramos2023]{Details} No & \cite{Ramos2023} & 2023 & Mathematics & null & \noindent{}\textcolor{black!50}{0.00} \textcolor{black!50}{0.00} n/a & 0 1 1 & 57 64 & 7 0 7\\
\index{Relich2023}\rowlabel{a:Relich2023}Relich2023 \href{http://dx.doi.org/10.3390/su15097667}{Relich2023} & \hyperref[auth:a1646]{M. Relich} & \cellcolor{gold!20}Predictive and Prescriptive Analytics in Identifying Opportunities for Improving Sustainable Manufacturing \hyperref[abs:Relich2023]{Abstract} & \hyperref[detail:Relich2023]{Details} No & \cite{Relich2023} & 2023 & Sustainability & null & \noindent{}\textcolor{black!50}{0.00} \textbf{3.00} n/a & 0 1 1 & 66 75 & 1 0 1\\
\index{Schweitzer2023}\rowlabel{a:Schweitzer2023}Schweitzer2023 \href{http://dx.doi.org/10.3390/app13020806}{Schweitzer2023} & \hyperref[auth:a1592]{F. Schweitzer}, \hyperref[auth:a1593]{G. Bitsch}, \hyperref[auth:a1594]{L. Louw} & \cellcolor{gold!20}Choosing Solution Strategies for Scheduling Automated Guided Vehicles in Production Using Machine Learning \hyperref[abs:Schweitzer2023]{Abstract} & \hyperref[detail:Schweitzer2023]{Details} No & \cite{Schweitzer2023} & 2023 & Applied Sciences & null & \noindent{}\textcolor{black!50}{0.00} \textbf{1.50} n/a & 2 5 5 & 49 68 & 7 0 7\\
\index{ShaikhK23}\rowlabel{a:ShaikhK23}ShaikhK23 \href{https://doi.org/10.1504/IJESDF.2023.10045616}{ShaikhK23} & \hyperref[auth:a416]{A. A. Shaikh}, \hyperref[auth:a417]{A. A. Khan} & Management of electronic ledger: a constraint programming approach for solving curricula scheduling problems & \hyperref[detail:ShaikhK23]{Details} \href{../works/ShaikhK23.pdf}{Yes} & \cite{ShaikhK23} & 2023 & Int. J. Electron. Secur. Digit. Forensics & 12 & \noindent{}\textbf{1.00} \textbf{1.00} 0.56 & 0 0 0 & 0 0 & 0 0 0\\
\index{Tayyab2023}\rowlabel{a:Tayyab2023}Tayyab2023 \href{http://dx.doi.org/10.3390/app13063616}{Tayyab2023} & \hyperref[auth:a1640]{A. Tayyab}, \hyperref[auth:a1641]{S. Ullah}, \hyperref[auth:a1642]{T. Mahmood}, \hyperref[auth:a1643]{Y. Y. Ghadi}, \hyperref[auth:a1644]{B. Latif}, \hyperref[auth:a1645]{H. Aljuaid} & \cellcolor{gold!20}Modeling of Multi-Level Planning of Shifting Bottleneck Resources Integrated with Downstream Wards in a Hospital \hyperref[abs:Tayyab2023]{Abstract} & \hyperref[detail:Tayyab2023]{Details} No & \cite{Tayyab2023} & 2023 & Applied Sciences & null & \noindent{}\textcolor{black!50}{0.00} \textcolor{black!50}{0.00} n/a & 1 0 1 & 53 57 & 3 0 3\\
\index{WessenCSFPM23}\rowlabel{a:WessenCSFPM23}WessenCSFPM23 \href{https://doi.org/10.1007/s10601-023-09345-4}{WessenCSFPM23} & \hyperref[auth:a90]{J. Wess{\'{e}}n}, \hyperref[auth:a91]{M. Carlsson}, \hyperref[auth:a92]{C. Schulte}, \hyperref[auth:a1416]{P. Flener}, \hyperref[auth:a1417]{F. Pecora}, \hyperref[auth:a1418]{M. Matskin} & \cellcolor{gold!20}A constraint programming model for the scheduling and workspace layout design of a dual-arm multi-tool assembly robot & \hyperref[detail:WessenCSFPM23]{Details} \href{../works/WessenCSFPM23.pdf}{Yes} & \cite{WessenCSFPM23} & 2023 & Constraints An Int. J. & 34 & \noindent{}\textbf{1.00} \textbf{1.00} \textbf{6.97} & 0 0 0 & 38 50 & 6 0 6\\
\index{Xu2023}\rowlabel{a:Xu2023}Xu2023 \href{http://dx.doi.org/10.1108/k-09-2022-1339}{Xu2023} & \hyperref[auth:a1619]{J. Xu}, \hyperref[auth:a1620]{S. Bai} & A reactive scheduling approach for the resource-constrained project scheduling problem with dynamic resource disruption \hyperref[abs:Xu2023]{Abstract} & \hyperref[detail:Xu2023]{Details} No & \cite{Xu2023} & 2023 & Kybernetes & null & \noindent{}\textcolor{black!50}{0.00} \textcolor{black!50}{0.00} n/a & 1 2 2 & 42 51 & 7 1 6\\
\index{YuraszeckMCCR23}\rowlabel{a:YuraszeckMCCR23}YuraszeckMCCR23 \href{https://doi.org/10.1109/ACCESS.2023.3345793}{YuraszeckMCCR23} & \hyperref[auth:a405]{F. Yuraszeck}, \hyperref[auth:a406]{E. Montero}, \hyperref[auth:a407]{D. Canut-de-Bon}, \hyperref[auth:a408]{N. Cuneo}, \hyperref[auth:a409]{M. Rojel} & \cellcolor{gold!20}A Constraint Programming Formulation of the Multi-Mode Resource-Constrained Project Scheduling Problem for the Flexible Job Shop Scheduling Problem & \hyperref[detail:YuraszeckMCCR23]{Details} \href{../works/YuraszeckMCCR23.pdf}{Yes} & \cite{YuraszeckMCCR23} & 2023 & {IEEE} Access & 11 & \noindent{}\textbf{2.50} \textbf{2.50} \textbf{8.55} & 0 0 0 & 0 29 & 0 0 0\\
\index{ZhuSZW23}\rowlabel{a:ZhuSZW23}ZhuSZW23 \href{http://dx.doi.org/10.1016/j.omega.2022.102823}{ZhuSZW23} & \hyperref[auth:a988]{X. Zhu}, \hyperref[auth:a989]{J. Son}, \hyperref[auth:a990]{X. Zhang}, \hyperref[auth:a991]{J. Wu} & Constraint programming and logic-based Benders decomposition for the integrated process planning and scheduling problem & \hyperref[detail:ZhuSZW23]{Details} \href{../works/ZhuSZW23.pdf}{Yes} & \cite{ZhuSZW23} & 2023 & Omega & 22 & \noindent{}\textbf{1.00} \textbf{1.00} \textbf{34.84} & 1 1 1 & 36 50 & 13 0 13\\
\index{abs-2305-19888}\rowlabel{a:abs-2305-19888}abs-2305-19888 \href{https://doi.org/10.48550/arXiv.2305.19888}{abs-2305-19888} & \hyperref[auth:a433]{V. Heinz}, \hyperref[auth:a434]{A. Nov{\'{a}}k}, \hyperref[auth:a311]{M. Vlk}, \hyperref[auth:a116]{Z. Hanz{\'{a}}lek} & Constraint Programming and Constructive Heuristics for Parallel Machine Scheduling with Sequence-Dependent Setups and Common Servers & \hyperref[detail:abs-2305-19888]{Details} \href{../works/abs-2305-19888.pdf}{Yes} & \cite{abs-2305-19888} & 2023 & CoRR & 42 & \noindent{}\textbf{1.50} \textbf{1.50} \textbf{41.88} & 0 0 0 & 0 0 & 0 0 0\\
\index{abs-2306-05747}\rowlabel{a:abs-2306-05747}abs-2306-05747 \href{https://doi.org/10.48550/arXiv.2306.05747}{abs-2306-05747} & \hyperref[auth:a58]{P. Tassel}, \hyperref[auth:a61]{M. Gebser}, \hyperref[auth:a423]{K. Schekotihin} & An End-to-End Reinforcement Learning Approach for Job-Shop Scheduling Problems Based on Constraint Programming & \hyperref[detail:abs-2306-05747]{Details} \href{../works/abs-2306-05747.pdf}{Yes} & \cite{abs-2306-05747} & 2023 & CoRR & 9 & \noindent{}\textbf{2.00} \textbf{2.00} \textbf{12.09} & 0 0 0 & 0 0 & 0 0 0\\
\index{abs-2312-13682}\rowlabel{a:abs-2312-13682}abs-2312-13682 \href{https://doi.org/10.48550/arXiv.2312.13682}{abs-2312-13682} & \hyperref[auth:a425]{G. Perez}, \hyperref[auth:a426]{G. Glorian}, \hyperref[auth:a427]{W. Suijlen}, \hyperref[auth:a428]{A. Lallouet} & A Constraint Programming Model for Scheduling the Unloading of Trains in Ports: Extended & \hyperref[detail:abs-2312-13682]{Details} \href{../works/abs-2312-13682.pdf}{Yes} & \cite{abs-2312-13682} & 2023 & CoRR & 20 & \noindent{}\textbf{1.00} \textbf{1.00} 0.94 & 0 0 0 & 0 0 & 0 0 0\\
\index{AbreuN22}\rowlabel{a:AbreuN22}AbreuN22 \href{https://doi.org/10.1016/j.cie.2022.108128}{AbreuN22} & \hyperref[auth:a418]{L. R. de Abreu}, \hyperref[auth:a419]{M. S. Nagano} & A new hybridization of adaptive large neighborhood search with constraint programming for open shop scheduling with sequence-dependent setup times & \hyperref[detail:AbreuN22]{Details} \href{../works/AbreuN22.pdf}{Yes} & \cite{AbreuN22} & 2022 & Computers \  Industrial Engineering & 20 & \noindent{}\textbf{1.00} \textbf{1.00} \textbf{57.71} & 10 14 13 & 56 74 & 7 2 5\\
\index{AwadMDMT22}\rowlabel{a:AwadMDMT22}AwadMDMT22 \href{http://dx.doi.org/10.1016/j.compchemeng.2021.107565}{AwadMDMT22} & \hyperref[auth:a1171]{M. Awad}, \hyperref[auth:a1172]{K. Mulrennan}, \hyperref[auth:a1173]{J. Donovan}, \hyperref[auth:a1174]{R. Macpherson}, \hyperref[auth:a1175]{D. Tormey} & A constraint programming model for makespan minimisation in batch manufacturing pharmaceutical facilities & \hyperref[detail:AwadMDMT22]{Details} \href{../works/AwadMDMT22.pdf}{Yes} & \cite{AwadMDMT22} & 2022 & Computers \  Chemical Engineering & 22 & \noindent{}\textcolor{black!50}{0.00} \textcolor{black!50}{0.00} \textbf{20.73} & 3 6 6 & 41 53 & 10 2 8\\
\index{BourreauGGLT22}\rowlabel{a:BourreauGGLT22}BourreauGGLT22 \href{https://doi.org/10.1080/00207543.2020.1856436}{BourreauGGLT22} & \hyperref[auth:a441]{E. Bourreau}, \hyperref[auth:a442]{T. Garaix}, \hyperref[auth:a443]{M. Gondran}, \hyperref[auth:a444]{P. Lacomme}, \hyperref[auth:a445]{N. Tchernev} & \cellcolor{green!10}A constraint-programming based decomposition method for the Generalised Workforce Scheduling and Routing Problem {(GWSRP)} & \hyperref[detail:BourreauGGLT22]{Details} \href{../works/BourreauGGLT22.pdf}{Yes} & \cite{BourreauGGLT22} & 2022 & \cellcolor{red!20}International Journal of Production Research & 19 & \noindent{}\textcolor{black!50}{0.00} \textcolor{black!50}{0.00} \textbf{3.25} & 4 6 6 & 44 50 & 2 1 1\\
\index{Braune2022}\rowlabel{a:Braune2022}Braune2022 \href{http://dx.doi.org/10.1007/s10951-022-00750-w}{Braune2022} & \hyperref[auth:a1512]{R. Braune} & \cellcolor{gold!20}Packing-based branch-and-bound for discrete malleable task scheduling \hyperref[abs:Braune2022]{Abstract} & \hyperref[detail:Braune2022]{Details} No & \cite{Braune2022} & 2022 & Journal of Scheduling & null & \noindent{}\textcolor{black!50}{0.00} \textbf{3.00} n/a & 1 2 2 & 34 40 & 6 0 6\\
\index{BulckG22}\rowlabel{a:BulckG22}BulckG22 \href{http://dx.doi.org/10.1007/s10951-021-00717-3}{BulckG22} & \hyperref[auth:a1409]{D. V. Bulck}, \hyperref[auth:a1410]{D. Goossens} & \cellcolor{green!10}Optimizing rest times and differences in games played: an iterative two-phase approach \hyperref[abs:BulckG22]{Abstract} & \hyperref[detail:BulckG22]{Details} \href{../works/BulckG22.pdf}{Yes} & \cite{BulckG22} & 2022 & Journal of Scheduling & 11 & \noindent{}\textcolor{black!50}{0.00} \textcolor{black!50}{0.00} 0.33 & 2 3 3 & 19 22 & 4 0 4\\
\index{CampeauG22}\rowlabel{a:CampeauG22}CampeauG22 \href{https://doi.org/10.1007/s10601-022-09337-w}{CampeauG22} & \hyperref[auth:a103]{L.-P. Campeau}, \hyperref[auth:a9]{M. Gamache} & Short- and medium-term optimization of underground mine planning using constraint programming & \hyperref[detail:CampeauG22]{Details} \href{../works/CampeauG22.pdf}{Yes} & \cite{CampeauG22} & 2022 & Constraints An Int. J. & 18 & \noindent{}\textcolor{black!50}{0.00} \textcolor{black!50}{0.00} \textbf{4.09} & 0 0 1 & 22 26 & 5 0 5\\
\index{CilKLO22}\rowlabel{a:CilKLO22}CilKLO22 \href{http://dx.doi.org/10.1016/j.eswa.2022.117529}{CilKLO22} & \hyperref[auth:a1381]{Z. A. Cil}, \hyperref[auth:a1380]{D. Kizilay}, \hyperref[auth:a1382]{Z. Li}, \hyperref[auth:a1383]{H. \"{O}ztop} & Two-sided disassembly line balancing problem with sequence-dependent setup time: A constraint programming model and artificial bee colony algorithm & \hyperref[detail:CilKLO22]{Details} \href{../works/CilKLO22.pdf}{Yes} & \cite{CilKLO22} & 2022 & Expert Systems with Applications & 19 & \noindent{}\textcolor{black!50}{0.00} \textcolor{black!50}{0.00} \textbf{39.52} & 5 11 10 & 44 52 & 10 1 9\\
\index{ColT22}\rowlabel{a:ColT22}ColT22 \href{http://dx.doi.org/10.1016/j.orp.2022.100249}{ColT22} & \hyperref[auth:a93]{G. D. Col}, \hyperref[auth:a738]{E. C. Teppan} & \cellcolor{gold!20}Industrial-size job shop scheduling with constraint programming & \hyperref[detail:ColT22]{Details} \href{../works/ColT22.pdf}{Yes} & \cite{ColT22} & 2022 & Operations Research Perspectives & 19 & \noindent{}\textbf{2.00} \textbf{2.00} \textbf{33.77} & 3 8 13 & 55 99 & 14 0 14\\
\index{Doolaard2022}\rowlabel{a:Doolaard2022}Doolaard2022 \href{http://dx.doi.org/10.1007/s10472-022-09816-z}{Doolaard2022} & \hyperref[auth:a1900]{F. Doolaard}, \hyperref[auth:a19]{N. Yorke-Smith} & \cellcolor{gold!20}Online learning of variable ordering heuristics for constraint optimisation problems \hyperref[abs:Doolaard2022]{Abstract} & \hyperref[detail:Doolaard2022]{Details} No & \cite{Doolaard2022} & 2022 & Annals of Mathematics and Artificial Intelligence & null & \noindent{}0.50 0.50 n/a & 0 0 1 & 17 29 & 7 0 7\\
\index{El-Kholany2022}\rowlabel{a:El-Kholany2022}El-Kholany2022 \href{http://dx.doi.org/10.1017/s1471068422000217}{El-Kholany2022} & \hyperref[auth:a1496]{M. M. S. El-Kholany}, \hyperref[auth:a61]{M. Gebser}, \hyperref[auth:a423]{K. Schekotihin} & \cellcolor{gold!20}Problem Decomposition and Multi-shot ASP Solving for Job-shop Scheduling \hyperref[abs:El-Kholany2022]{Abstract} & \hyperref[detail:El-Kholany2022]{Details} No & \cite{El-Kholany2022} & 2022 & Theory and Practice of Logic Programming & null & \noindent{}\textcolor{black!50}{0.00} \textbf{4.01} n/a & 6 8 7 & 28 37 & 7 0 7\\
\index{ElciOH22}\rowlabel{a:ElciOH22}ElciOH22 \href{http://dx.doi.org/10.1287/ijoc.2022.1184}{ElciOH22} & \hyperref[auth:a930]{\"{O}zg\"{u}n El\c{c}i}, \hyperref[auth:a160]{J. N. Hooker} & \cellcolor{green!10}Stochastic Planning and Scheduling with Logic-Based Benders Decomposition & \hyperref[detail:ElciOH22]{Details} \href{../works/ElciOH22.pdf}{Yes} & \cite{ElciOH22} & 2022 & \cellcolor{red!20}INFORMS Journal on Computing & 15 & \noindent{}\textcolor{black!50}{0.00} \textcolor{black!50}{0.00} \textbf{3.20} & 2 4 6 & 34 36 & 16 2 14\\
\index{EmdeZD22}\rowlabel{a:EmdeZD22}EmdeZD22 \href{http://dx.doi.org/10.1007/s10479-022-04891-1}{EmdeZD22} & \hyperref[auth:a956]{S. Emde}, \hyperref[auth:a957]{S. Zehtabian}, \hyperref[auth:a958]{Y. Disser} & Point-to-point and milk run delivery scheduling: models, complexity results, and algorithms based on Benders decomposition & \hyperref[detail:EmdeZD22]{Details} \href{../works/EmdeZD22.pdf}{Yes} & \cite{EmdeZD22} & 2022 & Annals of Operations Research & 30 & \noindent{}\textcolor{black!50}{0.00} \textcolor{black!50}{0.00} \textbf{3.40} & 0 0 0 & 52 59 & 11 0 11\\
\index{EtminaniesfahaniGNMS22}\rowlabel{a:EtminaniesfahaniGNMS22}EtminaniesfahaniGNMS22 \href{http://dx.doi.org/10.1007/s42979-022-01487-1}{EtminaniesfahaniGNMS22} & \hyperref[auth:a901]{A. Etminaniesfahani}, \hyperref[auth:a336]{H. Gu}, \hyperref[auth:a902]{L. M. Naeni}, \hyperref[auth:a903]{A. Salehipour} & A Forward–Backward Relax-and-Solve Algorithm for the Resource-Constrained Project Scheduling Problem & \hyperref[detail:EtminaniesfahaniGNMS22]{Details} \href{../works/EtminaniesfahaniGNMS22.pdf}{Yes} & \cite{EtminaniesfahaniGNMS22} & 2022 & SN Computer Science & 10 & \noindent{}\textcolor{black!50}{0.00} \textcolor{black!50}{0.00} \textbf{8.63} & 0 1 2 & 57 66 & 17 0 17\\
\index{FarsiTM22}\rowlabel{a:FarsiTM22}FarsiTM22 \href{https://api.semanticscholar.org/CorpusID:250301745}{FarsiTM22} & \hyperref[auth:a516]{A. Farsi}, \hyperref[auth:a739]{S. A. Torabi}, \hyperref[auth:a515]{M. Mokhtarzadeh} & Integrated surgery scheduling by constraint programming and meta-heuristics & \hyperref[detail:FarsiTM22]{Details} \href{../works/FarsiTM22.pdf}{Yes} & \cite{FarsiTM22} & 2022 & \cellcolor{red!20}International Journal of Management Science and Engineering Management & 14 & \noindent{}\textbf{1.00} \textbf{1.00} \textbf{4.92} & 5 5 8 & 47 50 & 6 1 5\\
\index{Feng2022}\rowlabel{a:Feng2022}Feng2022 \href{http://dx.doi.org/10.3390/app12189062}{Feng2022} & \hyperref[auth:a1738]{C. Feng}, \hyperref[auth:a1739]{S. Hu}, \hyperref[auth:a1740]{Y. Ma}, \hyperref[auth:a1741]{Z. Li} & \cellcolor{gold!20}A Project Scheduling Game Equilibrium Problem Based on Dynamic Resource Supply \hyperref[abs:Feng2022]{Abstract} & \hyperref[detail:Feng2022]{Details} No & \cite{Feng2022} & 2022 & Applied Sciences & null & \noindent{}\textcolor{black!50}{0.00} \textcolor{black!50}{0.00} n/a & 1 1 2 & 27 36 & 3 0 3\\
\index{FetgoD22}\rowlabel{a:FetgoD22}FetgoD22 \href{https://doi.org/10.1007/s43069-022-00172-6}{FetgoD22} & \hyperref[auth:a11]{S. B. Fetgo}, \hyperref[auth:a13]{C. T. Djam{\'{e}}gni} & \cellcolor{green!10}Horizontally Elastic Edge-Finder Algorithm for Cumulative Resource Constraint Revisited & \hyperref[detail:FetgoD22]{Details} \href{../works/FetgoD22.pdf}{Yes} & \cite{FetgoD22} & 2022 & Oper. Res. Forum & 32 & \noindent{}\textcolor{black!50}{0.00} \textcolor{black!50}{0.00} \textbf{12.41} & 0 0 1 & 20 29 & 16 0 16\\
\index{Gao2022}\rowlabel{a:Gao2022}Gao2022 \href{http://dx.doi.org/10.1007/s11227-022-04943-0}{Gao2022} & \hyperref[auth:a1837]{J. Gao}, \hyperref[auth:a1838]{X. Zhu}, \hyperref[auth:a1839]{R. Zhang} & Optimization of parallel test task scheduling with constraint satisfaction & \hyperref[detail:Gao2022]{Details} No & \cite{Gao2022} & 2022 & The Journal of Supercomputing & null & \noindent{}\textbf{2.00} \textbf{2.00} n/a & 2 4 6 & 32 32 & 2 0 2\\
\index{Gembarski2022}\rowlabel{a:Gembarski2022}Gembarski2022 \href{http://dx.doi.org/10.3390/a15090318}{Gembarski2022} & \hyperref[auth:a1991]{P. C. Gembarski} & \cellcolor{gold!20}Joining Constraint Satisfaction Problems and Configurable CAD Product Models: A Step-by-Step Implementation Guide \hyperref[abs:Gembarski2022]{Abstract} & \hyperref[detail:Gembarski2022]{Details} No & \cite{Gembarski2022} & 2022 & Algorithms & null & \noindent{}\textcolor{black!50}{0.00} \textbf{2.00} n/a & 1 2 3 & 19 32 & 2 0 2\\
\index{GhandehariK22}\rowlabel{a:GhandehariK22}GhandehariK22 \href{http://dx.doi.org/10.1016/j.apm.2022.01.001}{GhandehariK22} & \hyperref[auth:a1461]{N. Ghandehari}, \hyperref[auth:a760]{K. Kianfar} & Mixed-integer linear programming, constraint programming and column generation approaches for operating room planning under block strategy \hyperref[abs:GhandehariK22]{Abstract} & \hyperref[detail:GhandehariK22]{Details} \href{../works/GhandehariK22.pdf}{Yes} & \cite{GhandehariK22} & 2022 & APPLIED MATHEMATICAL MODELLING & 16 & \noindent{}\textcolor{black!50}{0.00} \textcolor{black!50}{0.00} \textbf{8.19} & 4 4 4 & 46 55 & 6 1 5\\
\index{Gokgur2022}\rowlabel{a:Gokgur2022}Gokgur2022 \href{http://dx.doi.org/10.35378/gujs.681151}{Gokgur2022} & \hyperref[auth:a1612]{B. Gokgur}, \hyperref[auth:a1613]{S. Özpeyni̇rci̇} & \cellcolor{gold!20}Minimization of Number of Tool Switching Instants in Automated Manufacturing Systems \hyperref[abs:Gokgur2022]{Abstract} & \hyperref[detail:Gokgur2022]{Details} No & \cite{Gokgur2022} & 2022 & Gazi University Journal of Science & null & \noindent{}\textcolor{black!50}{0.00} \textbf{1.50} n/a & 0 0 0 & 29 30 & 6 0 6\\
\index{HeinzNVH22}\rowlabel{a:HeinzNVH22}HeinzNVH22 \href{https://doi.org/10.1016/j.cie.2022.108586}{HeinzNVH22} & \hyperref[auth:a433]{V. Heinz}, \hyperref[auth:a434]{A. Nov{\'{a}}k}, \hyperref[auth:a311]{M. Vlk}, \hyperref[auth:a116]{Z. Hanz{\'{a}}lek} & \cellcolor{green!10}Constraint Programming and constructive heuristics for parallel machine scheduling with sequence-dependent setups and common servers & \hyperref[detail:HeinzNVH22]{Details} \href{../works/HeinzNVH22.pdf}{Yes} & \cite{HeinzNVH22} & 2022 & Computers \  Industrial Engineering & 16 & \noindent{}\textbf{1.50} \textbf{1.50} \textbf{40.68} & 5 7 8 & 25 31 & 7 3 4\\
\index{HillBCGN22}\rowlabel{a:HillBCGN22}HillBCGN22 \href{http://dx.doi.org/10.1287/ijoc.2022.1222}{HillBCGN22} & \hyperref[auth:a64]{A. Hill}, \hyperref[auth:a971]{A. J. Brickey}, \hyperref[auth:a972]{I. Cipriano}, \hyperref[auth:a973]{M. Goycoolea}, \hyperref[auth:a974]{A. Newman} & Optimization Strategies for Resource-Constrained Project Scheduling Problems in Underground Mining & \hyperref[detail:HillBCGN22]{Details} No & \cite{HillBCGN22} & 2022 & \cellcolor{red!20}INFORMS Journal on Computing & 17 & \noindent{}\textcolor{black!50}{0.00} \textcolor{black!50}{0.00} n/a & 0 2 2 & 53 58 & 10 0 10\\
\index{JuvinHL22}\rowlabel{a:JuvinHL22}JuvinHL22 \href{http://dx.doi.org/10.2139/ssrn.4068164}{JuvinHL22} & \hyperref[auth:a0]{C. Juvin}, \hyperref[auth:a2]{L. Houssin}, \hyperref[auth:a3]{P. Lopez} & Logic-Based Benders Decomposition for the Preemptive Flexible Job-Shop Scheduling Problem & \hyperref[detail:JuvinHL22]{Details} \href{../works/JuvinHL22.pdf}{Yes} & \cite{JuvinHL22} & 2022 & SSRN Electronic Journal & 32 & \noindent{}\textcolor{black!50}{0.00} \textcolor{black!50}{0.00} \textbf{12.00} & 0 0 0 & 29 40 & 12 0 12\\
\index{Kuramata2022}\rowlabel{a:Kuramata2022}Kuramata2022 \href{http://dx.doi.org/10.1371/journal.pone.0266846}{Kuramata2022} & \hyperref[auth:a1690]{M. Kuramata}, \hyperref[auth:a1691]{R. Katsuki}, \hyperref[auth:a1692]{K. Nakata} & \cellcolor{gold!20}Solving large break minimization problems in a mirrored double round-robin tournament using quantum annealing \hyperref[abs:Kuramata2022]{Abstract} & \hyperref[detail:Kuramata2022]{Details} No & \cite{Kuramata2022} & 2022 & PLOS ONE & null & \noindent{}\textcolor{black!50}{0.00} \textcolor{black!50}{0.00} n/a & 0 0 0 & 25 36 & 4 0 4\\
\index{MartnezAJ22}\rowlabel{a:MartnezAJ22}MartnezAJ22 \href{http://dx.doi.org/10.1287/ijoc.2021.1079}{MartnezAJ22} & \hyperref[auth:a935]{K. P. Martínez}, \hyperref[auth:a936]{Y. Adulyasak}, \hyperref[auth:a841]{R. Jans} & Logic-Based Benders Decomposition for Integrated Process Configuration and Production Planning Problems & \hyperref[detail:MartnezAJ22]{Details} No & \cite{MartnezAJ22} & 2022 & \cellcolor{red!20}INFORMS Journal on Computing & 15 & \noindent{}\textcolor{black!50}{0.00} \textcolor{black!50}{0.00} n/a & 1 2 2 & 29 29 & 16 1 15\\
\index{MengGRZSC22}\rowlabel{a:MengGRZSC22}MengGRZSC22 \href{http://dx.doi.org/10.1016/j.swevo.2022.101058}{MengGRZSC22} & \hyperref[auth:a500]{L. Meng}, \hyperref[auth:a1176]{K. Gao}, \hyperref[auth:a502]{Y. Ren}, \hyperref[auth:a503]{B. Zhang}, \hyperref[auth:a1158]{H. Sang}, \hyperref[auth:a1177]{Z. Chaoyong} & Novel MILP and CP models for distributed hybrid flowshop scheduling problem with sequence-dependent setup times & \hyperref[detail:MengGRZSC22]{Details} \href{../works/MengGRZSC22.pdf}{Yes} & \cite{MengGRZSC22} & 2022 & Swarm and Evolutionary Computation & 13 & \noindent{}\textbf{1.00} \textbf{1.00} \textbf{24.78} & 38 56 62 & 37 42 & 9 3 6\\
\index{Michels2022}\rowlabel{a:Michels2022}Michels2022 \href{http://dx.doi.org/10.1108/aa-10-2021-0140}{Michels2022} & \hyperref[auth:a1551]{A. S. Michels}, \hyperref[auth:a1552]{A. M. Costa} & Mixed-integer linear programming models for the type-II resource-constrained assembly line balancing problem \hyperref[abs:Michels2022]{Abstract} & \hyperref[detail:Michels2022]{Details} No & \cite{Michels2022} & 2022 & Assembly Automation & null & \noindent{}\textcolor{black!50}{0.00} \textbf{1.50} n/a & 2 2 1 & 26 35 & 4 1 3\\
\index{Misra2022}\rowlabel{a:Misra2022}Misra2022 \href{http://dx.doi.org/10.1016/j.compchemeng.2022.107895}{Misra2022} & \hyperref[auth:a1802]{S. Misra}, \hyperref[auth:a1803]{L. R. Buttazoni}, \hyperref[auth:a1804]{V. Avadiappan}, \hyperref[auth:a1805]{H. J. Lee}, \hyperref[auth:a1806]{M. Yang}, \hyperref[auth:a381]{C. T. Maravelias} & CProS: A web-based application for chemical production scheduling & \hyperref[detail:Misra2022]{Details} No & \cite{Misra2022} & 2022 & Computers \  Chemical Engineering & null & \noindent{}\textcolor{black!50}{0.00} \textcolor{black!50}{0.00} n/a & 2 4 4 & 16 17 & 2 0 2\\
\index{MullerMKP22}\rowlabel{a:MullerMKP22}MullerMKP22 \href{https://doi.org/10.1016/j.ejor.2022.01.034}{MullerMKP22} & \hyperref[auth:a435]{D. M{\"{u}}ller}, \hyperref[auth:a436]{M. G. M{\"{u}}ller}, \hyperref[auth:a437]{D. Kress}, \hyperref[auth:a438]{E. Pesch} & An algorithm selection approach for the flexible job shop scheduling problem: Choosing constraint programming solvers through machine learning & \hyperref[detail:MullerMKP22]{Details} \href{../works/MullerMKP22.pdf}{Yes} & \cite{MullerMKP22} & 2022 & European Journal of Operational Research & 18 & \noindent{}\textbf{2.50} \textbf{2.50} \textbf{12.25} & 17 19 20 & 59 93 & 16 3 13\\
\index{NaderiBZ22}\rowlabel{a:NaderiBZ22}NaderiBZ22 \href{http://dx.doi.org/10.2139/ssrn.4140716}{NaderiBZ22} & \hyperref[auth:a726]{B. Naderi}, \hyperref[auth:a836]{M. A. Begen}, \hyperref[auth:a837]{G. Zhang} & Integrated Order Acceptance and Resource Decisions Under Uncertainty: Robust and Stochastic Approaches & \hyperref[detail:NaderiBZ22]{Details} \href{../works/NaderiBZ22.pdf}{Yes} & \cite{NaderiBZ22} & 2022 & SSRN Electronic Journal & 29 & \noindent{}\textcolor{black!50}{0.00} \textcolor{black!50}{0.00} \textbf{9.27} & 0 0 0 & 44 51 & 11 0 11\\
\index{NaderiBZ22a}\rowlabel{a:NaderiBZ22a}NaderiBZ22a \href{http://dx.doi.org/10.1016/j.cor.2022.105728}{NaderiBZ22a} & \hyperref[auth:a726]{B. Naderi}, \hyperref[auth:a836]{M. A. Begen}, \hyperref[auth:a838]{G. S. Zaric} & Type-2 integrated process-planning and scheduling problem: Reformulation and solution algorithms & \hyperref[detail:NaderiBZ22a]{Details} \href{../works/NaderiBZ22a.pdf}{Yes} & \cite{NaderiBZ22a} & 2022 & Computers \  Operations Research & 19 & \noindent{}\textcolor{black!50}{0.00} \textcolor{black!50}{0.00} \textbf{22.24} & 3 4 4 & 44 54 & 17 2 15\\
\index{NaderiR22}\rowlabel{a:NaderiR22}NaderiR22 \href{http://dx.doi.org/10.1287/ijoo.2021.0056}{NaderiR22} & \hyperref[auth:a726]{B. Naderi}, \hyperref[auth:a728]{V. Roshanaei} & Critical-Path-Search Logic-Based Benders Decomposition Approaches for Flexible Job Shop Scheduling & \hyperref[detail:NaderiR22]{Details} No & \cite{NaderiR22} & 2022 & \cellcolor{red!20}INFORMS Journal on Optimization & 28 & \noindent{}\textcolor{black!50}{0.00} \textcolor{black!50}{0.00} n/a & 5 7 0 & 49 52 & 14 3 11\\
\index{NaqviAIAAA22}\rowlabel{a:NaqviAIAAA22}NaqviAIAAA22 \href{http://dx.doi.org/10.32604/cmc.2022.019653}{NaqviAIAAA22} & \hyperref[auth:a1393]{S. R. Naqvi}, \hyperref[auth:a1394]{A. Ahmad}, \hyperref[auth:a1395]{S. M. R. Islam}, \hyperref[auth:a1396]{T. Akram}, \hyperref[auth:a1397]{M. Abdullah-Al-Wadud}, \hyperref[auth:a1398]{A. Alamri} & \cellcolor{gold!20}Towards Prevention of Sportsmen Burnout: Formal Analysis of Sub-Optimal Tournament Scheduling \hyperref[abs:NaqviAIAAA22]{Abstract} & \hyperref[detail:NaqviAIAAA22]{Details} \href{../works/NaqviAIAAA22.pdf}{Yes} & \cite{NaqviAIAAA22} & 2022 & CMC-COMPUTERS MATERIALS \  CONTINUA & 18 & \noindent{}\textcolor{black!50}{0.00} \textcolor{black!50}{0.00} 0.34 & 0 0 0 & 22 26 & 2 0 2\\
\index{OrnekOS20}\rowlabel{a:OrnekOS20}OrnekOS20 \href{https://ideas.repec.org/a/spr/operea/v22y2022i1d10.1007_s12351-020-00563-9.html}{OrnekOS20} & \hyperref[auth:a138]{A. {\"{O}}rnek}, \hyperref[auth:a135]{C. {\"{O}}zt{\"{u}}rk}, \hyperref[auth:a1013]{I. Sugut} & {Integer and constraint programming model formulations for flight-gate assignment problem} & \hyperref[detail:OrnekOS20]{Details} \href{../works/OrnekOS20.pdf}{Yes} & \cite{OrnekOS20} & 2022 & Operational Research & 29 & \noindent{}\textcolor{black!50}{0.00} \textcolor{black!50}{0.00} \textbf{1.65} & 0 0 0 & 0 0 & 0 0 0\\
\index{Ouellet2022}\rowlabel{a:Ouellet2022}Ouellet2022 \href{http://dx.doi.org/10.1609/aaai.v36i4.20296}{Ouellet2022} & \hyperref[auth:a52]{Y. Ouellet}, \hyperref[auth:a37]{C.-G. Quimper} & The SoftCumulative Constraint with Quadratic Penalty \hyperref[abs:Ouellet2022]{Abstract} & \hyperref[detail:Ouellet2022]{Details} No & \cite{Ouellet2022} & 2022 & Proceedings of the AAAI Conference on Artificial Intelligence & null & \noindent{}\textcolor{black!50}{0.00} \textbf{1.50} n/a & 1 0 0 & 0 0 & 1 1 0\\
\index{PohlAK22}\rowlabel{a:PohlAK22}PohlAK22 \href{https://doi.org/10.1016/j.ejor.2021.08.028}{PohlAK22} & \hyperref[auth:a439]{M. Pohl}, \hyperref[auth:a6]{C. Artigues}, \hyperref[auth:a440]{R. Kolisch} & \cellcolor{green!10}Solving the time-discrete winter runway scheduling problem: {A} column generation and constraint programming approach & \hyperref[detail:PohlAK22]{Details} \href{../works/PohlAK22.pdf}{Yes} & \cite{PohlAK22} & 2022 & European Journal of Operational Research & 16 & \noindent{}\textbf{1.00} \textbf{1.00} \textbf{5.76} & 4 4 4 & 31 44 & 4 1 3\\
\index{Relich2022}\rowlabel{a:Relich2022}Relich2022 \href{http://dx.doi.org/10.3390/app12041921}{Relich2022} & \hyperref[auth:a1646]{M. Relich}, \hyperref[auth:a1705]{I. Nielsen}, \hyperref[auth:a1815]{A. Gola} & \cellcolor{gold!20}Reducing the Total Product Cost at the Product Design Stage \hyperref[abs:Relich2022]{Abstract} & \hyperref[detail:Relich2022]{Details} No & \cite{Relich2022} & 2022 & Applied Sciences & null & \noindent{}\textcolor{black!50}{0.00} \textbf{2.00} n/a & 7 12 13 & 42 55 & 2 0 2\\
\index{ShiYXQ22}\rowlabel{a:ShiYXQ22}ShiYXQ22 \href{https://doi.org/10.1080/00207543.2021.1963496}{ShiYXQ22} & \hyperref[auth:a446]{G. Shi}, \hyperref[auth:a447]{Z. Yang}, \hyperref[auth:a448]{Y. Xu}, \hyperref[auth:a449]{Y. Quan} & Solving the integrated process planning and scheduling problem using an enhanced constraint programming-based approach & \hyperref[detail:ShiYXQ22]{Details} No & \cite{ShiYXQ22} & 2022 & \cellcolor{red!20}International Journal of Production Research & 18 & \noindent{}\textbf{1.00} \textbf{1.00} n/a & 2 3 3 & 45 53 & 5 1 4\\
\index{Song2022}\rowlabel{a:Song2022}Song2022 \href{http://dx.doi.org/10.1016/j.engappai.2021.104603}{Song2022} & \hyperref[auth:a1874]{W. Song}, \hyperref[auth:a1875]{Z. Cao}, \hyperref[auth:a1876]{J. Zhang}, \hyperref[auth:a1877]{C. Xu}, \hyperref[auth:a279]{A. Lim} & \cellcolor{green!10}Learning variable ordering heuristics for solving Constraint Satisfaction Problems & \hyperref[detail:Song2022]{Details} No & \cite{Song2022} & 2022 & Engineering Applications of Artificial Intelligence & null & \noindent{}0.50 0.50 n/a & 12 15 20 & 22 55 & 3 1 2\\
\index{SubulanC22}\rowlabel{a:SubulanC22}SubulanC22 \href{https://doi.org/10.1007/s00500-021-06399-5}{SubulanC22} & \hyperref[auth:a451]{K. Subulan}, \hyperref[auth:a452]{G. {\c{C}}akir} & Constraint programming-based transformation approach for a mixed fuzzy-stochastic resource investment project scheduling problem & \hyperref[detail:SubulanC22]{Details} \href{../works/SubulanC22.pdf}{Yes} & \cite{SubulanC22} & 2022 & Soft Computing & 38 & \noindent{}\textbf{1.50} \textbf{1.50} \textbf{44.71} & 5 7 7 & 86 107 & 5 0 5\\
\index{Tapkan2022}\rowlabel{a:Tapkan2022}Tapkan2022 \href{http://dx.doi.org/10.1080/01605682.2022.2125843}{Tapkan2022} & \hyperref[auth:a1787]{P. Tapkan}, \hyperref[auth:a1788]{S. Kulluk}, \hyperref[auth:a1789]{L. Özbakır}, \hyperref[auth:a1790]{F. Bahar}, \hyperref[auth:a1791]{B. Gülmez} & A constraint programming based column generation approach for crew scheduling: A case study for the Kayseri railway & \hyperref[detail:Tapkan2022]{Details} No & \cite{Tapkan2022} & 2022 & \cellcolor{red!20}Journal of the Operational Research Society & null & \noindent{}\textbf{1.00} \textbf{1.00} n/a & 0 1 1 & 32 38 & 4 0 4\\
\index{Tomczak2022}\rowlabel{a:Tomczak2022}Tomczak2022 \href{http://dx.doi.org/10.3846/jcem.2022.16943}{Tomczak2022} & \hyperref[auth:a1768]{M. Tomczak}, \hyperref[auth:a1769]{P. Jaśkowski} & \cellcolor{gold!20}Scheduling repetitive construction projects: structured literature review \hyperref[abs:Tomczak2022]{Abstract} & \hyperref[detail:Tomczak2022]{Details} No & \cite{Tomczak2022} & 2022 & JOURNAL OF CIVIL ENGINEERING AND MANAGEMENT & null & \noindent{}\textcolor{black!50}{0.00} \textbf{1.00} n/a & 3 5 5 & 191 197 & 6 0 6\\
\index{Valouxis2022}\rowlabel{a:Valouxis2022}Valouxis2022 \href{http://dx.doi.org/10.3390/a15120450}{Valouxis2022} & \hyperref[auth:a1507]{C. Valouxis}, \hyperref[auth:a1508]{C. Gogos}, \hyperref[auth:a1509]{A. Dimitsas}, \hyperref[auth:a1510]{P. Potikas}, \hyperref[auth:a1511]{A. Vittas} & \cellcolor{gold!20}A Hybrid Exact–Local Search Approach for One-Machine Scheduling with Time-Dependent Capacity \hyperref[abs:Valouxis2022]{Abstract} & \hyperref[detail:Valouxis2022]{Details} No & \cite{Valouxis2022} & 2022 & Algorithms & null & \noindent{}\textcolor{black!50}{0.00} \textbf{3.50} n/a & 2 3 3 & 24 28 & 3 0 3\\
\index{YunusogluY22}\rowlabel{a:YunusogluY22}YunusogluY22 \href{https://doi.org/10.1080/00207543.2021.1885068}{YunusogluY22} & \hyperref[auth:a450]{P. Yunusoglu}, \hyperref[auth:a421]{S. T. Yildiz} & Constraint programming approach for multi-resource-constrained unrelated parallel machine scheduling problem with sequence-dependent setup times & \hyperref[detail:YunusogluY22]{Details} \href{../works/YunusogluY22.pdf}{Yes} & \cite{YunusogluY22} & 2022 & \cellcolor{red!20}International Journal of Production Research & 18 & \noindent{}\textbf{2.00} \textbf{2.00} \textbf{69.33} & 20 36 40 & 58 64 & 16 6 10\\
\index{YuraszeckMPV22}\rowlabel{a:YuraszeckMPV22}YuraszeckMPV22 \href{http://dx.doi.org/10.3390/math10030329}{YuraszeckMPV22} & \hyperref[auth:a405]{F. Yuraszeck}, \hyperref[auth:a742]{G. Mejía}, \hyperref[auth:a743]{J. Pereira}, \hyperref[auth:a744]{M. Vilà} & \cellcolor{gold!20}A Novel Constraint Programming Decomposition Approach for the Total Flow Time Fixed Group Shop Scheduling Problem & \hyperref[detail:YuraszeckMPV22]{Details} \href{../works/YuraszeckMPV22.pdf}{Yes} & \cite{YuraszeckMPV22} & 2022 & Mathematics & 26 & \noindent{}\textbf{1.00} \textbf{1.00} \textbf{43.40} & 6 9 9 & 29 37 & 13 1 12\\
\index{Zohali2022}\rowlabel{a:Zohali2022}Zohali2022 \href{http://dx.doi.org/10.1287/ijoc.2020.1015}{Zohali2022} & \hyperref[auth:a1526]{H. Zohali}, \hyperref[auth:a726]{B. Naderi}, \hyperref[auth:a728]{V. Roshanaei} & Solving the Type-2 Assembly Line Balancing with Setups Using Logic-Based Benders Decomposition \hyperref[abs:Zohali2022]{Abstract} & \hyperref[detail:Zohali2022]{Details} No & \cite{Zohali2022} & 2022 & \cellcolor{red!20}INFORMS Journal on Computing & null & \noindent{}\textcolor{black!50}{0.00} \textcolor{black!50}{0.00} n/a & 11 12 12 & 52 53 & 12 2 10\\
\index{abs-2211-14492}\rowlabel{a:abs-2211-14492}abs-2211-14492 \href{https://doi.org/10.48550/arXiv.2211.14492}{abs-2211-14492} & \hyperref[auth:a397]{Y. Sun}, \hyperref[auth:a395]{S. Nguyen}, \hyperref[auth:a396]{D. R. Thiruvady}, \hyperref[auth:a468]{X. Li}, \hyperref[auth:a469]{A. T. Ernst}, \hyperref[auth:a470]{U. Aickelin} & Enhancing Constraint Programming via Supervised Learning for Job Shop Scheduling & \hyperref[detail:abs-2211-14492]{Details} \href{../works/abs-2211-14492.pdf}{Yes} & \cite{abs-2211-14492} & 2022 & CoRR & 17 & \noindent{}\textbf{2.00} \textbf{2.00} \textbf{13.19} & 0 0 0 & 0 0 & 0 0 0\\
\index{AbohashimaEG21}\rowlabel{a:AbohashimaEG21}AbohashimaEG21 \href{https://doi.org/10.1109/ACCESS.2021.3112600}{AbohashimaEG21} & \hyperref[auth:a472]{H. Abohashima}, \hyperref[auth:a473]{A. B. Eltawil}, \hyperref[auth:a474]{M. S. Gheith} & \cellcolor{gold!20}A Mathematical Programming Model and a Firefly-Based Heuristic for Real-Time Traffic Signal Scheduling With Physical Constraints & \hyperref[detail:AbohashimaEG21]{Details} \href{../works/AbohashimaEG21.pdf}{Yes} & \cite{AbohashimaEG21} & 2021 & {IEEE} Access & 14 & \noindent{}\textcolor{black!50}{0.00} \textcolor{black!50}{0.00} \textcolor{black!50}{0.00} & 1 3 3 & 25 27 & 0 0 0\\
\index{AbreuAPNM21}\rowlabel{a:AbreuAPNM21}AbreuAPNM21 \href{http://dx.doi.org/10.1080/0305215x.2021.1957101}{AbreuAPNM21} & \hyperref[auth:a418]{L. R. de Abreu}, \hyperref[auth:a747]{K. A. G. Araújo}, \hyperref[auth:a748]{B. de A. Prata}, \hyperref[auth:a419]{M. S. Nagano}, \hyperref[auth:a749]{J. V. Moccellin} & A new variable neighbourhood search with a constraint programming search strategy for the open shop scheduling problem with operation repetitions & \hyperref[detail:AbreuAPNM21]{Details} \href{../works/AbreuAPNM21.pdf}{Yes} & \cite{AbreuAPNM21} & 2021 & \cellcolor{red!20}Engineering Optimization & 20 & \noindent{}\textbf{1.00} \textbf{1.00} \textbf{32.82} & 5 4 7 & 50 58 & 11 2 9\\
\index{Alaka21}\rowlabel{a:Alaka21}Alaka21 \href{http://dx.doi.org/10.1007/s00500-021-05602-x}{Alaka21} & \hyperref[auth:a764]{H. M. Alakaş} & General resource-constrained assembly line balancing problem: conjunction normal form based constraint programming models & \hyperref[detail:Alaka21]{Details} \href{../works/Alaka21.pdf}{Yes} & \cite{Alaka21} & 2021 & Soft Computing & 11 & \noindent{}0.50 0.50 \textbf{19.49} & 7 9 9 & 20 27 & 11 2 9\\
\index{Bedhief21}\rowlabel{a:Bedhief21}Bedhief21 \href{https://api.semanticscholar.org/CorpusID:240611192}{Bedhief21} & \hyperref[auth:a746]{A. O. Bedhief} & \cellcolor{gold!20}Comparing Mixed-Integer Programming and Constraint Programming Models for the Hybrid Flow Shop Scheduling Problem with Dedicated Machines & \hyperref[detail:Bedhief21]{Details} \href{../works/Bedhief21.pdf}{Yes} & \cite{Bedhief21} & 2021 & Journal Europ{\'e}en des Syst{\`e}mes Automatis{\'e}s & 7 & \noindent{}\textbf{1.50} \textbf{1.50} \textbf{8.02} & 0 0 2 & 0 0 & 0 0 0\\
\index{Bocewicz2021}\rowlabel{a:Bocewicz2021}Bocewicz2021 \href{http://dx.doi.org/10.17531/ein.2021.1.13}{Bocewicz2021} & \hyperref[auth:a630]{G. Bocewicz}, \hyperref[auth:a1997]{E. Szwarc}, \hyperref[auth:a535]{J. Wikarek}, \hyperref[auth:a1527]{P. Nielsen}, \hyperref[auth:a1814]{Z. Banaszak} & \cellcolor{gold!20}A competency-driven staff assignment approach to improving employee scheduling robustness \hyperref[abs:Bocewicz2021]{Abstract} & \hyperref[detail:Bocewicz2021]{Details} No & \cite{Bocewicz2021} & 2021 & Eksploatacja i Niezawodność – Maintenance and Reliability & null & \noindent{}\textcolor{black!50}{0.00} \textbf{2.00} n/a & 5 9 9 & 38 45 & 2 0 2\\
\index{CarlierSJP21}\rowlabel{a:CarlierSJP21}CarlierSJP21 \href{http://dx.doi.org/10.1080/00207543.2021.1923853}{CarlierSJP21} & \hyperref[auth:a845]{J. Carlier}, \hyperref[auth:a928]{A. Sahli}, \hyperref[auth:a929]{A. Jouglet}, \hyperref[auth:a846]{E. Pinson} & A faster checker of the energetic reasoning for the cumulative scheduling problem & \hyperref[detail:CarlierSJP21]{Details} No & \cite{CarlierSJP21} & 2021 & \cellcolor{red!20}International Journal of Production Research & 16 & \noindent{}\textcolor{black!50}{0.00} \textcolor{black!50}{0.00} n/a & 3 6 4 & 26 29 & 12 2 10\\
\index{Chen2021}\rowlabel{a:Chen2021}Chen2021 \href{http://dx.doi.org/10.1177/03611981211036368}{Chen2021} & \hyperref[auth:a1626]{G.-H. Chen}, \hyperref[auth:a1627]{J.-C. Jong}, \hyperref[auth:a1628]{A. F.-W. Han} & \cellcolor{gold!20}Applying Constraint Programming and Integer Programming to Solve the Crew Scheduling Problem for Railroad Systems: Model Formulation and a Case Study \hyperref[abs:Chen2021]{Abstract} & \hyperref[detail:Chen2021]{Details} No & \cite{Chen2021} & 2021 & Transportation Research Record: Journal of the Transportation Research Board & null & \noindent{}\textbf{1.00} \textbf{2.00} n/a & 1 4 4 & 20 25 & 6 0 6\\
\index{Daneshamooz2021}\rowlabel{a:Daneshamooz2021}Daneshamooz2021 \href{http://dx.doi.org/10.1108/k-08-2020-0521}{Daneshamooz2021} & \hyperref[auth:a1728]{F. Daneshamooz}, \hyperref[auth:a1729]{P. Fattahi}, \hyperref[auth:a1730]{S. M. H. Hosseini} & Mathematical modeling and two efficient branch and bound algorithms for job shop scheduling problem followed by an assembly stage \hyperref[abs:Daneshamooz2021]{Abstract} & \hyperref[detail:Daneshamooz2021]{Details} No & \cite{Daneshamooz2021} & 2021 & Kybernetes & null & \noindent{}\textcolor{black!50}{0.00} \textcolor{black!50}{0.00} n/a & 6 7 8 & 21 33 & 2 1 1\\
\index{Edis21}\rowlabel{a:Edis21}Edis21 \href{http://dx.doi.org/10.1016/j.cor.2020.105111}{Edis21} & \hyperref[auth:a346]{E. B. Edis} & Constraint programming approaches to disassembly line balancing problem with sequencing decisions & \hyperref[detail:Edis21]{Details} \href{../works/Edis21.pdf}{Yes} & \cite{Edis21} & 2021 & Computers \  Operations Research & 20 & \noindent{}\textcolor{black!50}{0.00} \textcolor{black!50}{0.00} \textbf{60.40} & 13 19 20 & 48 53 & 10 2 8\\
\index{FanXG21}\rowlabel{a:FanXG21}FanXG21 \href{https://doi.org/10.1016/j.cor.2021.105401}{FanXG21} & \hyperref[auth:a476]{H. Fan}, \hyperref[auth:a477]{H. Xiong}, \hyperref[auth:a478]{M. Goh} & Genetic programming-based hyper-heuristic approach for solving dynamic job shop scheduling problem with extended technical precedence constraints & \hyperref[detail:FanXG21]{Details} \href{../works/FanXG21.pdf}{Yes} & \cite{FanXG21} & 2021 & Computers \  Operations Research & 15 & \noindent{}\textcolor{black!50}{0.00} \textcolor{black!50}{0.00} \textcolor{black!50}{0.00} & 18 27 30 & 57 68 & 1 0 1\\
\index{Grzegorz2021}\rowlabel{a:Grzegorz2021}Grzegorz2021 \href{http://dx.doi.org/10.3390/app11198898}{Grzegorz2021} & \hyperref[auth:a2062]{R. Grzegorz}, \hyperref[auth:a2063]{B. Grzegorz}, \hyperref[auth:a2064]{D. Bogdan}, \hyperref[auth:a2065]{B. Zbigniew} & \cellcolor{gold!20}Reactive Planning-Driven Approach to Online UAVs Mission Rerouting and Rescheduling \hyperref[abs:Grzegorz2021]{Abstract} & \hyperref[detail:Grzegorz2021]{Details} No & \cite{Grzegorz2021} & 2021 & Applied Sciences & null & \noindent{}\textcolor{black!50}{0.00} \textbf{1.50} n/a & 1 3 2 & 30 36 & 1 0 1\\
\index{HamP21}\rowlabel{a:HamP21}HamP21 \href{http://dx.doi.org/10.1109/lra.2021.3056069}{HamP21} & \hyperref[auth:a750]{A. Ham}, \hyperref[auth:a751]{M.-J. Park} & Human–Robot Task Allocation and Scheduling: Boeing 777 Case Study & \hyperref[detail:HamP21]{Details} \href{../works/HamP21.pdf}{Yes} & \cite{HamP21} & 2021 & IEEE Robotics and Automation Letters & 8 & \noindent{}\textcolor{black!50}{0.00} \textcolor{black!50}{0.00} \textbf{6.63} & 13 16 17 & 26 30 & 12 2 10\\
\index{HamPK21}\rowlabel{a:HamPK21}HamPK21 \href{https://api.semanticscholar.org/CorpusID:237898414}{HamPK21} & \hyperref[auth:a750]{A. Ham}, \hyperref[auth:a751]{M.-J. Park}, \hyperref[auth:a752]{K. M. Kim} & \cellcolor{gold!20}Energy-Aware Flexible Job Shop Scheduling Using Mixed Integer Programming and Constraint Programming & \hyperref[detail:HamPK21]{Details} \href{../works/HamPK21.pdf}{Yes} & \cite{HamPK21} & 2021 & Mathematical Problems in Engineering & 12 & \noindent{}\textbf{2.00} \textbf{2.00} \textbf{10.25} & 6 9 11 & 46 51 & 3 1 2\\
\index{Hosseinian2021}\rowlabel{a:Hosseinian2021}Hosseinian2021 \href{http://dx.doi.org/10.1051/ro/2021087}{Hosseinian2021} & \hyperref[auth:a1573]{A. H. Hosseinian}, \hyperref[auth:a1574]{V. Baradaran} & \cellcolor{gold!20}A multi-objective multi-agent optimization algorithm for the multi-skill resource-constrained project scheduling problem with transfer times \hyperref[abs:Hosseinian2021]{Abstract} & \hyperref[detail:Hosseinian2021]{Details} No & \cite{Hosseinian2021} & 2021 & RAIRO - Operations Research & null & \noindent{}\textcolor{black!50}{0.00} \textcolor{black!50}{0.00} n/a & 3 10 10 & 65 77 & 5 0 5\\
\index{HubnerGSV21}\rowlabel{a:HubnerGSV21}HubnerGSV21 \href{https://doi.org/10.1007/s10951-021-00682-x}{HubnerGSV21} & \hyperref[auth:a482]{F. H{\"{u}}bner}, \hyperref[auth:a483]{P. Gerhards}, \hyperref[auth:a484]{C. St{\"{u}}rck}, \hyperref[auth:a485]{R. Volk} & \cellcolor{gold!20}Solving the nuclear dismantling project scheduling problem by combining mixed-integer and constraint programming techniques and metaheuristics & \hyperref[detail:HubnerGSV21]{Details} \href{../works/HubnerGSV21.pdf}{Yes} & \cite{HubnerGSV21} & 2021 & Journal of Scheduling & 22 & \noindent{}\textbf{1.00} \textbf{1.00} \textbf{13.02} & 0 2 1 & 37 46 & 4 0 4\\
\index{Kasapidis2021}\rowlabel{a:Kasapidis2021}Kasapidis2021 \href{http://dx.doi.org/10.1111/poms.13501}{Kasapidis2021} & \hyperref[auth:a1503]{G. A. Kasapidis}, \hyperref[auth:a1504]{D. C. Paraskevopoulos}, \hyperref[auth:a1505]{P. P. Repoussis}, \hyperref[auth:a1506]{C. D. Tarantilis} & \cellcolor{green!10}Flexible Job Shop Scheduling Problems with Arbitrary Precedence Graphs \hyperref[abs:Kasapidis2021]{Abstract} & \hyperref[detail:Kasapidis2021]{Details} No & \cite{Kasapidis2021} & 2021 & \cellcolor{red!20}Production and Operations Management & null & \noindent{}\textcolor{black!50}{0.00} \textbf{3.00} n/a & 7 12 10 & 36 43 & 6 1 5\\
\index{KoehlerBFFHPSSS21}\rowlabel{a:KoehlerBFFHPSSS21}KoehlerBFFHPSSS21 \href{https://doi.org/10.1007/s10601-021-09321-w}{KoehlerBFFHPSSS21} & \hyperref[auth:a104]{J. Koehler}, \hyperref[auth:a105]{J. B{\"{u}}rgler}, \hyperref[auth:a106]{U. Fontana}, \hyperref[auth:a107]{E. Fux}, \hyperref[auth:a108]{F. A. Herzog}, \hyperref[auth:a109]{M. Pouly}, \hyperref[auth:a110]{S. Saller}, \hyperref[auth:a111]{A. Salyaeva}, \hyperref[auth:a112]{P. Scheiblechner}, \hyperref[auth:a113]{K. Waelti} & \cellcolor{gold!20}Cable tree wiring - benchmarking solvers on a real-world scheduling problem with a variety of precedence constraints & \hyperref[detail:KoehlerBFFHPSSS21]{Details} \href{../works/KoehlerBFFHPSSS21.pdf}{Yes} & \cite{KoehlerBFFHPSSS21} & 2021 & Constraints An Int. J. & 51 & \noindent{}\textcolor{black!50}{0.00} \textcolor{black!50}{0.00} \textbf{16.91} & 2 3 2 & 52 66 & 6 0 6\\
\index{Kong2021}\rowlabel{a:Kong2021}Kong2021 \href{http://dx.doi.org/10.1061/(asce)co.1943-7862.0002192}{Kong2021} & \hyperref[auth:a1706]{F. Kong}, \hyperref[auth:a1707]{J. Guo}, \hyperref[auth:a1708]{X. Lv} & Project Resource Input Optimization Problem with Combined Time Constraints Based on Node Network Diagram and Constraint Programming & \hyperref[detail:Kong2021]{Details} No & \cite{Kong2021} & 2021 & Journal of Construction Engineering and Management & null & \noindent{}0.50 0.50 n/a & 1 1 1 & 31 32 & 6 0 6\\
\index{Liu2021}\rowlabel{a:Liu2021}Liu2021 \href{http://dx.doi.org/10.3390/app11041447}{Liu2021} & \hyperref[auth:a1244]{S.-S. Liu}, \hyperref[auth:a1489]{M. F. A. Arifin}, \hyperref[auth:a1490]{W. T. Chen}, \hyperref[auth:a1491]{Y.-H. Huang} & \cellcolor{gold!20}Emergency Repair Scheduling Model for Road Network Integrating Rescheduling Feature \hyperref[abs:Liu2021]{Abstract} & \hyperref[detail:Liu2021]{Details} No & \cite{Liu2021} & 2021 & Applied Sciences & null & \noindent{}\textcolor{black!50}{0.00} \textbf{4.00} n/a & 2 2 2 & 52 53 & 6 0 6\\
\index{Liu2021a}\rowlabel{a:Liu2021a}Liu2021a \href{http://dx.doi.org/10.3390/math9192492}{Liu2021a} & \hyperref[auth:a1244]{S.-S. Liu}, \hyperref[auth:a1719]{A. Budiwirawan}, \hyperref[auth:a1489]{M. F. A. Arifin} & \cellcolor{gold!20}Non-Sequential Linear Construction Project Scheduling Model for Minimizing Idle Equipment Using Constraint Programming (CP) \hyperref[abs:Liu2021a]{Abstract} & \hyperref[detail:Liu2021a]{Details} No & \cite{Liu2021a} & 2021 & Mathematics & null & \noindent{}\textbf{2.00} \textbf{3.00} n/a & 0 1 1 & 49 52 & 5 0 5\\
\index{Liu2021b}\rowlabel{a:Liu2021b}Liu2021b \href{http://dx.doi.org/10.3390/sym13030364}{Liu2021b} & \hyperref[auth:a1244]{S.-S. Liu}, \hyperref[auth:a1719]{A. Budiwirawan}, \hyperref[auth:a1489]{M. F. A. Arifin}, \hyperref[auth:a1490]{W. T. Chen}, \hyperref[auth:a1491]{Y.-H. Huang} & \cellcolor{gold!20}Optimization Model for the Pavement Pothole Repair Problem Considering Consumable Resources \hyperref[abs:Liu2021b]{Abstract} & \hyperref[detail:Liu2021b]{Details} No & \cite{Liu2021b} & 2021 & Symmetry & null & \noindent{}\textcolor{black!50}{0.00} \textbf{6.01} n/a & 2 3 3 & 0 0 & 1 1 0\\
\index{MengLZB21}\rowlabel{a:MengLZB21}MengLZB21 \href{http://dx.doi.org/10.1049/cim2.12005}{MengLZB21} & \hyperref[auth:a500]{L. Meng}, \hyperref[auth:a1157]{C. Lu}, \hyperref[auth:a503]{B. Zhang}, \hyperref[auth:a502]{Y. Ren}, \hyperref[auth:a504]{C. Lv}, \hyperref[auth:a1158]{H. Sang}, \hyperref[auth:a1159]{J. Li}, \hyperref[auth:a501]{C. Zhang} & \cellcolor{gold!20}Constraint programing for solving four complex flexible shop scheduling problems & \hyperref[detail:MengLZB21]{Details} \href{../works/MengLZB21.pdf}{Yes} & \cite{MengLZB21} & 2021 & IET Collaborative Intelligent Manufacturing & 14 & \noindent{}\textcolor{black!50}{0.00} \textcolor{black!50}{0.00} \textbf{28.91} & 5 8 8 & 39 44 & 12 4 8\\
\index{Mischek2021}\rowlabel{a:Mischek2021}Mischek2021 \href{http://dx.doi.org/10.1007/s10951-021-00699-2}{Mischek2021} & \hyperref[auth:a80]{F. Mischek}, \hyperref[auth:a45]{N. Musliu}, \hyperref[auth:a1261]{A. Schaerf} & \cellcolor{gold!20}Local search approaches for the test laboratory scheduling problem with variable task grouping \hyperref[abs:Mischek2021]{Abstract} & \hyperref[detail:Mischek2021]{Details} No & \cite{Mischek2021} & 2021 & Journal of Scheduling & null & \noindent{}\textcolor{black!50}{0.00} \textcolor{black!50}{0.00} n/a & 1 3 2 & 32 38 & 9 1 8\\
\index{Mischek2021a}\rowlabel{a:Mischek2021a}Mischek2021a \href{http://dx.doi.org/10.1007/s10479-021-04007-1}{Mischek2021a} & \hyperref[auth:a80]{F. Mischek}, \hyperref[auth:a45]{N. Musliu} & \cellcolor{gold!20}A local search framework for industrial test laboratory scheduling \hyperref[abs:Mischek2021a]{Abstract} & \hyperref[detail:Mischek2021a]{Details} No & \cite{Mischek2021a} & 2021 & Annals of Operations Research & null & \noindent{}\textcolor{black!50}{0.00} \textbf{2.50} n/a & 3 5 5 & 28 35 & 10 2 8\\
\index{NaderiRBAU21}\rowlabel{a:NaderiRBAU21}NaderiRBAU21 \href{http://dx.doi.org/10.1111/poms.13397}{NaderiRBAU21} & \hyperref[auth:a726]{B. Naderi}, \hyperref[auth:a728]{V. Roshanaei}, \hyperref[auth:a836]{M. A. Begen}, \hyperref[auth:a895]{D. M. Aleman}, \hyperref[auth:a896]{D. R. Urbach} & Increased Surgical Capacity without Additional Resources: Generalized Operating Room Planning and Scheduling & \hyperref[detail:NaderiRBAU21]{Details} No & \cite{NaderiRBAU21} & 2021 & \cellcolor{red!20}Production and Operations Management & 28 & \noindent{}\textcolor{black!50}{0.00} \textcolor{black!50}{0.00} n/a & 22 23 23 & 61 66 & 22 5 17\\
\index{Ortiz-Bayliss2021}\rowlabel{a:Ortiz-Bayliss2021}Ortiz-Bayliss2021 \href{http://dx.doi.org/10.3390/app11062749}{Ortiz-Bayliss2021} & \hyperref[auth:a1603]{J. C. Ortiz-Bayliss}, \hyperref[auth:a1604]{I. Amaya}, \hyperref[auth:a1605]{J. M. Cruz-Duarte}, \hyperref[auth:a1606]{A. E. Gutierrez-Rodriguez}, \hyperref[auth:a1607]{S. E. Conant-Pablos}, \hyperref[auth:a1608]{H. Terashima-Marín} & \cellcolor{gold!20}A General Framework Based on Machine Learning for Algorithm Selection in Constraint Satisfaction Problems \hyperref[abs:Ortiz-Bayliss2021]{Abstract} & \hyperref[detail:Ortiz-Bayliss2021]{Details} No & \cite{Ortiz-Bayliss2021} & 2021 & Applied Sciences & null & \noindent{}0.50 \textbf{1.50} n/a & 2 4 4 & 37 59 & 7 0 7\\
\index{PandeyS21a}\rowlabel{a:PandeyS21a}PandeyS21a \href{https://doi.org/10.1007/s11227-020-03516-3}{PandeyS21a} & \hyperref[auth:a491]{V. Pandey}, \hyperref[auth:a492]{P. Saini} & Constraint programming versus heuristic approach to MapReduce scheduling problem in Hadoop {YARN} for energy minimization & \hyperref[detail:PandeyS21a]{Details} \href{../works/PandeyS21a.pdf}{Yes} & \cite{PandeyS21a} & 2021 & J. Supercomput. & 29 & \noindent{}\textbf{1.00} \textbf{1.00} \textbf{47.79} & 3 3 3 & 32 41 & 8 0 8\\
\index{Pinarbasi21}\rowlabel{a:Pinarbasi21}Pinarbasi21 \href{http://dx.doi.org/10.1080/0305215x.2021.1921171}{Pinarbasi21} & \hyperref[auth:a1384]{M. Pınarbaşı} & New mathematical and constraint programming models for U-type assembly line balancing problems with assignment restrictions & \hyperref[detail:Pinarbasi21]{Details} No & \cite{Pinarbasi21} & 2021 & \cellcolor{red!20}Engineering Optimization & 16 & \noindent{}\textcolor{black!50}{0.00} \textcolor{black!50}{0.00} n/a & 3 6 0 & 46 50 & 8 2 6\\
\index{QinWSLS21}\rowlabel{a:QinWSLS21}QinWSLS21 \href{https://doi.org/10.1109/TASE.2019.2947398}{QinWSLS21} & \hyperref[auth:a486]{M. Qin}, \hyperref[auth:a487]{R. Wang}, \hyperref[auth:a488]{Z. Shi}, \hyperref[auth:a489]{L. Liu}, \hyperref[auth:a490]{L. Shi} & A Genetic Programming-Based Scheduling Approach for Hybrid Flow Shop With a Batch Processor and Waiting Time Constraint & \hyperref[detail:QinWSLS21]{Details} \href{../works/QinWSLS21.pdf}{Yes} & \cite{QinWSLS21} & 2021 & {IEEE} Trans Autom. Sci. Eng. & 12 & \noindent{}\textcolor{black!50}{0.00} \textcolor{black!50}{0.00} \textcolor{black!50}{0.00} & 12 19 0 & 30 30 & 1 0 1\\
\index{RabbaniMM21}\rowlabel{a:RabbaniMM21}RabbaniMM21 \href{http://dx.doi.org/10.1080/17509653.2021.1905096}{RabbaniMM21} & \hyperref[auth:a1246]{M. Rabbani}, \hyperref[auth:a515]{M. Mokhtarzadeh}, \hyperref[auth:a1247]{N. Manavizadeh} & A constraint programming approach and a hybrid of genetic and K-means algorithms to solve the p-hub location-allocation problems & \hyperref[detail:RabbaniMM21]{Details} No & \cite{RabbaniMM21} & 2021 & \cellcolor{red!20}International Journal of Management Science and Engineering Management & 11 & \noindent{}\textcolor{black!50}{0.00} \textcolor{black!50}{0.00} n/a & 4 4 9 & 44 46 & 9 1 8\\
\index{Radzki2021}\rowlabel{a:Radzki2021}Radzki2021 \href{http://dx.doi.org/10.3390/su13095228}{Radzki2021} & \hyperref[auth:a2007]{G. Radzki}, \hyperref[auth:a1705]{I. Nielsen}, \hyperref[auth:a2008]{P. Golińska-Dawson}, \hyperref[auth:a630]{G. Bocewicz}, \hyperref[auth:a1814]{Z. Banaszak} & \cellcolor{gold!20}Reactive UAV Fleet's Mission Planning in Highly Dynamic and Unpredictable Environments \hyperref[abs:Radzki2021]{Abstract} & \hyperref[detail:Radzki2021]{Details} No & \cite{Radzki2021} & 2021 & Sustainability & null & \noindent{}\textcolor{black!50}{0.00} \textbf{1.50} n/a & 13 17 15 & 50 60 & 2 1 1\\
\index{Ramos2021}\rowlabel{a:Ramos2021}Ramos2021 \href{http://dx.doi.org/10.1111/exsy.12830}{Ramos2021} & \hyperref[auth:a1731]{A. S. Ramos}, \hyperref[auth:a1736]{E. Olivares‐Benitez}, \hyperref[auth:a1737]{P. A. Miranda‐Gonzalez} & Multi‐start iterated local search metaheuristic for the multi‐mode resource‐constrained project scheduling problem \hyperref[abs:Ramos2021]{Abstract} & \hyperref[detail:Ramos2021]{Details} No & \cite{Ramos2021} & 2021 & Expert Systems & null & \noindent{}\textcolor{black!50}{0.00} \textcolor{black!50}{0.00} n/a & 3 3 4 & 52 56 & 8 1 7\\
\index{Rieber2021}\rowlabel{a:Rieber2021}Rieber2021 \href{http://dx.doi.org/10.1145/3487922}{Rieber2021} & \hyperref[auth:a1890]{D. Rieber}, \hyperref[auth:a1891]{A. Acosta}, \hyperref[auth:a1892]{H. Fröning} & \cellcolor{gold!20}Joint Program and Layout Transformations to Enable Convolutional Operators on Specialized Hardware Based on Constraint Programming \hyperref[abs:Rieber2021]{Abstract} & \hyperref[detail:Rieber2021]{Details} No & \cite{Rieber2021} & 2021 & ACM Transactions on Architecture and Code Optimization & null & \noindent{}\textcolor{black!50}{0.00} \textbf{2.00} n/a & 0 0 0 & 22 38 & 2 0 2\\
\index{RoshanaeiN21}\rowlabel{a:RoshanaeiN21}RoshanaeiN21 \href{http://dx.doi.org/10.1016/j.ejor.2020.12.004}{RoshanaeiN21} & \hyperref[auth:a728]{V. Roshanaei}, \hyperref[auth:a726]{B. Naderi} & Solving integrated operating room planning and scheduling: Logic-based Benders decomposition versus Branch-Price-and-Cut & \hyperref[detail:RoshanaeiN21]{Details} \href{../works/RoshanaeiN21.pdf}{Yes} & \cite{RoshanaeiN21} & 2021 & European Journal of Operational Research & 14 & \noindent{}\textcolor{black!50}{0.00} \textcolor{black!50}{0.00} \textbf{6.68} & 15 15 15 & 44 51 & 17 7 10\\
\index{Sahli2021}\rowlabel{a:Sahli2021}Sahli2021 \href{http://dx.doi.org/10.1051/ro/2021164}{Sahli2021} & \hyperref[auth:a928]{A. Sahli}, \hyperref[auth:a845]{J. Carlier}, \hyperref[auth:a1170]{A. Moukrim} & \cellcolor{gold!20}Polynomial algorithms for some scheduling problems with one nonrenewable resource \hyperref[abs:Sahli2021]{Abstract} & \hyperref[detail:Sahli2021]{Details} No & \cite{Sahli2021} & 2021 & RAIRO - Operations Research & null & \noindent{}\textcolor{black!50}{0.00} \textcolor{black!50}{0.00} n/a & 0 0 0 & 26 27 & 6 0 6\\
\index{Spieker2021}\rowlabel{a:Spieker2021}Spieker2021 \href{http://dx.doi.org/10.3390/ai2040033}{Spieker2021} & \hyperref[auth:a196]{H. Spieker}, \hyperref[auth:a195]{A. Gotlieb} & \cellcolor{gold!20}Predictive Machine Learning of Objective Boundaries for Solving COPs \hyperref[abs:Spieker2021]{Abstract} & \hyperref[detail:Spieker2021]{Details} No & \cite{Spieker2021} & 2021 & AI & null & \noindent{}\textcolor{black!50}{0.00} \textbf{2.00} n/a & 0 0 0 & 51 75 & 4 0 4\\
\index{Strak2021}\rowlabel{a:Strak2021}Strak2021 \href{http://dx.doi.org/10.5937/tehnika2102239s}{Strak2021} & \hyperref[auth:a2027]{M. Strak}, \hyperref[auth:a2028]{R. Lečić} & Organization of work with clients in the COVID-19 emergency conditions using constraint programming \hyperref[abs:Strak2021]{Abstract} & \hyperref[detail:Strak2021]{Details} No & \cite{Strak2021} & 2021 & Tehnika & null & \noindent{}\textcolor{black!50}{0.00} \textbf{1.00} n/a & 0 0 0 & 6 19 & 1 0 1\\
\index{VlkHT21}\rowlabel{a:VlkHT21}VlkHT21 \href{https://doi.org/10.1016/j.cie.2021.107317}{VlkHT21} & \hyperref[auth:a311]{M. Vlk}, \hyperref[auth:a116]{Z. Hanz{\'{a}}lek}, \hyperref[auth:a475]{S. Tang} & Constraint programming approaches to joint routing and scheduling in time-sensitive networks & \hyperref[detail:VlkHT21]{Details} \href{../works/VlkHT21.pdf}{Yes} & \cite{VlkHT21} & 2021 & Computers \  Industrial Engineering & 14 & \noindent{}\textbf{1.00} \textbf{1.00} \textbf{5.93} & 7 18 20 & 22 36 & 3 0 3\\
\index{Wang2021}\rowlabel{a:Wang2021}Wang2021 \href{http://dx.doi.org/10.1155/2021/5531063}{Wang2021} & \hyperref[auth:a1968]{L. Wang}, \hyperref[auth:a1969]{W. Ma}, \hyperref[auth:a1970]{L. Wang}, \hyperref[auth:a1971]{Y. Ren}, \hyperref[auth:a1972]{C. Yu} & \cellcolor{gold!20}Enabling In-Depot Automated Routing and Recharging Scheduling for Automated Electric Bus Transit Systems \hyperref[abs:Wang2021]{Abstract} & \hyperref[detail:Wang2021]{Details} No & \cite{Wang2021} & 2021 & Journal of Advanced Transportation & null & \noindent{}\textcolor{black!50}{0.00} \textbf{5.00} n/a & 1 4 3 & 34 39 & 3 0 3\\
\index{Xia2021}\rowlabel{a:Xia2021}Xia2021 \href{http://dx.doi.org/10.3233/jifs-189721}{Xia2021} & \hyperref[auth:a1540]{Y. Xia}, \hyperref[auth:a1541]{Z. Xie}, \hyperref[auth:a1542]{Y. Xin}, \hyperref[auth:a1543]{X. Zhang} & A multi-shop integrated scheduling algorithm with fixed output constraint \hyperref[abs:Xia2021]{Abstract} & \hyperref[detail:Xia2021]{Details} No & \cite{Xia2021} & 2021 & Journal of Intelligent \  Fuzzy Systems & null & \noindent{}\textcolor{black!50}{0.00} \textbf{2.50} n/a & 4 6 7 & 17 20 & 2 0 2\\
\index{ZhangYW21}\rowlabel{a:ZhangYW21}ZhangYW21 \href{https://doi.org/10.1016/j.cor.2021.105282}{ZhangYW21} & \hyperref[auth:a479]{L. Zhang}, \hyperref[auth:a480]{C. Yu}, \hyperref[auth:a481]{T. N. Wong} & A graph-based constraint programming approach for the integrated process planning and scheduling problem & \hyperref[detail:ZhangYW21]{Details} \href{../works/ZhangYW21.pdf}{Yes} & \cite{ZhangYW21} & 2021 & Computers \  Operations Research & 10 & \noindent{}\textbf{1.00} \textbf{1.00} \textbf{9.87} & 6 7 8 & 35 41 & 14 3 11\\
\index{Zou2021}\rowlabel{a:Zou2021}Zou2021 \href{http://dx.doi.org/10.1108/ecam-10-2020-0843}{Zou2021} & \hyperref[auth:a756]{X. Zou}, \hyperref[auth:a757]{L. Zhang}, \hyperref[auth:a1483]{Q. Zhang} & Time-cost optimization in repetitive project scheduling with limited resources \hyperref[abs:Zou2021]{Abstract} & \hyperref[detail:Zou2021]{Details} No & \cite{Zou2021} & 2021 & Engineering, Construction and Architectural Management & null & \noindent{}\textcolor{black!50}{0.00} \textbf{4.00} n/a & 3 9 12 & 27 43 & 6 0 6\\
\index{Zuenko2021}\rowlabel{a:Zuenko2021}Zuenko2021 \href{http://dx.doi.org/10.1088/1742-6596/2060/1/012021}{Zuenko2021} & \hyperref[auth:a1994]{A. Zuenko}, \hyperref[auth:a1995]{Y. Oleynik}, \hyperref[auth:a1996]{R. Makedonov} & \cellcolor{gold!20}A method for solving the open-pit mine production scheduling problem using the constraint programming paradigm \hyperref[abs:Zuenko2021]{Abstract} & \hyperref[detail:Zuenko2021]{Details} No & \cite{Zuenko2021} & 2021 & Journal of Physics: Conference Series & null & \noindent{}\textbf{1.00} \textbf{2.00} n/a & 1 1 1 & 4 17 & 1 0 1\\
\index{abs-2102-08778}\rowlabel{a:abs-2102-08778}abs-2102-08778 \href{https://arxiv.org/abs/2102.08778}{abs-2102-08778} & \hyperref[auth:a93]{G. D. Col}, \hyperref[auth:a608]{E. Teppan} & Large-Scale Benchmarks for the Job Shop Scheduling Problem & \hyperref[detail:abs-2102-08778]{Details} \href{../works/abs-2102-08778.pdf}{Yes} & \cite{abs-2102-08778} & 2021 & CoRR & 10 & \noindent{}\textcolor{black!50}{0.00} \textcolor{black!50}{0.00} \textbf{1.00} & 0 0 0 & 0 0 & 0 0 0\\
\index{AbidinK20}\rowlabel{a:AbidinK20}AbidinK20 \href{http://dx.doi.org/10.1016/j.cor.2020.105069}{AbidinK20} & \hyperref[auth:a1381]{Z. A. Cil}, \hyperref[auth:a1380]{D. Kizilay} & Constraint programming model for multi-manned assembly line balancing problem & \hyperref[detail:AbidinK20]{Details} \href{../works/AbidinK20.pdf}{Yes} & \cite{AbidinK20} & 2020 & Computers \  Operations Research & 14 & \noindent{}\textcolor{black!50}{0.00} \textcolor{black!50}{0.00} \textbf{13.74} & 11 14 0 & 27 35 & 6 1 5\\
\index{AlizdehS20}\rowlabel{a:AlizdehS20}AlizdehS20 \href{https://doi.org/10.1504/IJAIP.2020.106687}{AlizdehS20} & \hyperref[auth:a513]{S. Alizdeh}, \hyperref[auth:a514]{S. Saeidi} & Fuzzy project scheduling with critical path including risk and resource constraints using linear programming & \hyperref[detail:AlizdehS20]{Details} No & \cite{AlizdehS20} & 2020 & \cellcolor{red!20}Int. J. Adv. Intell. Paradigms & 14 & \noindent{}\textcolor{black!50}{0.00} \textcolor{black!50}{0.00} n/a & 1 1 3 & 0 0 & 0 0 0\\
\index{AntunesABD20}\rowlabel{a:AntunesABD20}AntunesABD20 \href{https://doi.org/10.1142/S0218213020600076}{AntunesABD20} & \hyperref[auth:a877]{M. Antunes}, \hyperref[auth:a878]{V. Armant}, \hyperref[auth:a217]{K. N. Brown}, \hyperref[auth:a879]{D. A. Desmond}, \hyperref[auth:a880]{G. Escamocher}, \hyperref[auth:a881]{A.-M. George}, \hyperref[auth:a181]{D. Grimes}, \hyperref[auth:a882]{M. O'Keeffe}, \hyperref[auth:a883]{Y. Lin}, \hyperref[auth:a16]{B. O'Sullivan}, \hyperref[auth:a135]{C. {\"{O}}zt{\"{u}}rk}, \hyperref[auth:a884]{L. Quesada}, \hyperref[auth:a129]{M. Siala}, \hyperref[auth:a17]{H. Simonis}, \hyperref[auth:a826]{N. Wilson} & \cellcolor{green!10}Assigning and Scheduling Service Visits in a Mixed Urban/Rural Setting & \hyperref[detail:AntunesABD20]{Details} \href{../works/AntunesABD20.pdf}{Yes} & \cite{AntunesABD20} & 2020 & Int. J. Artif. Intell. Tools & 31 & \noindent{}\textcolor{black!50}{0.00} \textcolor{black!50}{0.00} 0.63 & 0 0 1 & 16 18 & 0 0 0\\
\index{AstrandJZ20}\rowlabel{a:AstrandJZ20}AstrandJZ20 \href{https://doi.org/10.1016/j.cor.2020.105036}{AstrandJZ20} & \hyperref[auth:a74]{M. {\AA}strand}, \hyperref[auth:a75]{M. Johansson}, \hyperref[auth:a199]{A. Zanarini} & Underground mine scheduling of mobile machines using Constraint Programming and Large Neighborhood Search & \hyperref[detail:AstrandJZ20]{Details} \href{../works/AstrandJZ20.pdf}{Yes} & \cite{AstrandJZ20} & 2020 & Computers \  Operations Research & 13 & \noindent{}\textbf{1.50} \textbf{1.50} \textbf{22.04} & 16 19 19 & 24 53 & 9 1 8\\
\index{BadicaBI20}\rowlabel{a:BadicaBI20}BadicaBI20 \href{https://doi.org/10.3233/AIC-200650}{BadicaBI20} & \hyperref[auth:a497]{A. Badica}, \hyperref[auth:a498]{C. Badica}, \hyperref[auth:a499]{M. Ivanovic} & Block structured scheduling using constraint logic programming & \hyperref[detail:BadicaBI20]{Details} \href{../works/BadicaBI20.pdf}{Yes} & \cite{BadicaBI20} & 2020 & {AI} Commun. & 17 & \noindent{}\textbf{1.00} \textbf{1.00} \textbf{9.81} & 2 4 4 & 28 31 & 3 0 3\\
\index{BalochG20}\rowlabel{a:BalochG20}BalochG20 \href{http://dx.doi.org/10.1287/trsc.2019.0928}{BalochG20} & \hyperref[auth:a1237]{G. Baloch}, \hyperref[auth:a1238]{F. Gzara} & Strategic Network Design for Parcel Delivery with Drones Under Competition & \hyperref[detail:BalochG20]{Details} No & \cite{BalochG20} & 2020 & \cellcolor{red!20}Transportation Science & 25 & \noindent{}\textcolor{black!50}{0.00} \textcolor{black!50}{0.00} n/a & 25 32 33 & 46 52 & 9 2 7\\
\index{BenediktMH20}\rowlabel{a:BenediktMH20}BenediktMH20 \href{https://doi.org/10.1007/s10601-020-09317-y}{BenediktMH20} & \hyperref[auth:a114]{O. Benedikt}, \hyperref[auth:a115]{I. M{\'{o}}dos}, \hyperref[auth:a116]{Z. Hanz{\'{a}}lek} & \cellcolor{green!10}Power of pre-processing: production scheduling with variable energy pricing and power-saving states & \hyperref[detail:BenediktMH20]{Details} \href{../works/BenediktMH20.pdf}{Yes} & \cite{BenediktMH20} & 2020 & Constraints An Int. J. & 19 & \noindent{}\textcolor{black!50}{0.00} \textcolor{black!50}{0.00} \textbf{8.72} & 1 2 2 & 18 18 & 3 1 2\\
\index{Caricato2020}\rowlabel{a:Caricato2020}Caricato2020 \href{http://dx.doi.org/10.1007/s00170-020-06176-y}{Caricato2020} & \hyperref[auth:a1499]{P. Caricato}, \hyperref[auth:a1500]{A. Grieco}, \hyperref[auth:a1501]{A. Arigliano}, \hyperref[auth:a1502]{L. Rondone} & \cellcolor{gold!20}Workforce influence on manufacturing machines schedules \hyperref[abs:Caricato2020]{Abstract} & \hyperref[detail:Caricato2020]{Details} No & \cite{Caricato2020} & 2020 & The International Journal of Advanced Manufacturing Technology & null & \noindent{}\textcolor{black!50}{0.00} \textbf{3.00} n/a & 3 4 3 & 20 27 & 3 1 2\\
\index{CarlierPSJ20}\rowlabel{a:CarlierPSJ20}CarlierPSJ20 \href{http://dx.doi.org/10.1016/j.ejor.2020.03.079}{CarlierPSJ20} & \hyperref[auth:a845]{J. Carlier}, \hyperref[auth:a846]{E. Pinson}, \hyperref[auth:a1239]{A. Sahli}, \hyperref[auth:a1240]{A. Jouglet} & \cellcolor{gold!20}An O(n2) algorithm for time-bound adjustments for the cumulative scheduling problem & \hyperref[detail:CarlierPSJ20]{Details} \href{../works/CarlierPSJ20.pdf}{Yes} & \cite{CarlierPSJ20} & 2020 & European Journal of Operational Research & 9 & \noindent{}\textcolor{black!50}{0.00} \textcolor{black!50}{0.00} 0.60 & 6 7 8 & 10 19 & 11 4 7\\
\index{CauwelaertDS20}\rowlabel{a:CauwelaertDS20}CauwelaertDS20 \href{http://dx.doi.org/10.1007/s10951-019-00632-8}{CauwelaertDS20} & \hyperref[auth:a835]{S. V. Cauwelaert}, \hyperref[auth:a202]{C. Dejemeppe}, \hyperref[auth:a147]{P. Schaus} & An Efficient Filtering Algorithm for the Unary Resource Constraint with Transition Times and Optional Activities & \hyperref[detail:CauwelaertDS20]{Details} \href{../works/CauwelaertDS20.pdf}{Yes} & \cite{CauwelaertDS20} & 2020 & Journal of Scheduling & 19 & \noindent{}\textcolor{black!50}{0.00} \textcolor{black!50}{0.00} \textbf{5.18} & 2 2 2 & 21 36 & 16 1 15\\
\index{Danzinger2020}\rowlabel{a:Danzinger2020}Danzinger2020 \href{http://dx.doi.org/10.1609/icaps.v30i1.6681}{Danzinger2020} & \hyperref[auth:a1484]{P. Danzinger}, \hyperref[auth:a77]{T. Geibinger}, \hyperref[auth:a80]{F. Mischek}, \hyperref[auth:a45]{N. Musliu} & Solving the Test Laboratory Scheduling Problem with Variable Task Grouping \hyperref[abs:Danzinger2020]{Abstract} & \hyperref[detail:Danzinger2020]{Details} No & \cite{Danzinger2020} & 2020 & Proceedings of the International Conference on Automated Planning and Scheduling & null & \noindent{}\textcolor{black!50}{0.00} \textbf{3.50} n/a & 1 2 0 & 0 0 & 1 1 0\\
\index{FachiniA20}\rowlabel{a:FachiniA20}FachiniA20 \href{http://dx.doi.org/10.1016/j.cie.2020.106641}{FachiniA20} & \hyperref[auth:a1023]{R. F. Fachini}, \hyperref[auth:a1024]{V. A. Armentano} & Logic-based Benders decomposition for the heterogeneous fixed fleet vehicle routing problem with time windows & \hyperref[detail:FachiniA20]{Details} \href{../works/FachiniA20.pdf}{Yes} & \cite{FachiniA20} & 2020 & Computers \  Industrial Engineering & 18 & \noindent{}\textcolor{black!50}{0.00} \textcolor{black!50}{0.00} \textcolor{black!50}{0.20} & 25 26 35 & 55 68 & 12 2 10\\
\index{FallahiAC20}\rowlabel{a:FallahiAC20}FallahiAC20 \href{https://api.semanticscholar.org/CorpusID:213449737}{FallahiAC20} & \hyperref[auth:a753]{A. E. Fallahi}, \hyperref[auth:a754]{E. Y. Anass}, \hyperref[auth:a755]{M. Cherkaoui} & Tabu search and constraint programming-based approach for a real scheduling and routing problem & \hyperref[detail:FallahiAC20]{Details} \href{../works/FallahiAC20.pdf}{Yes} & \cite{FallahiAC20} & 2020 & International Journal of Applied Management Science & 18 & \noindent{}\textbf{1.00} \textbf{1.00} \textbf{5.04} & 0 0 0 & 0 0 & 0 0 0\\
\index{GuoHLW20}\rowlabel{a:GuoHLW20}GuoHLW20 \href{http://dx.doi.org/10.1080/0305215x.2019.1699919}{GuoHLW20} & \hyperref[auth:a931]{P. Guo}, \hyperref[auth:a932]{X. He}, \hyperref[auth:a933]{Y. Luan}, \hyperref[auth:a934]{Y. Wang} & Logic-based Benders decomposition for gantry crane scheduling with transferring position constraints in a rail–road container terminal & \hyperref[detail:GuoHLW20]{Details} No & \cite{GuoHLW20} & 2020 & \cellcolor{red!20}Engineering Optimization & 21 & \noindent{}\textcolor{black!50}{0.00} \textcolor{black!50}{0.00} n/a & 8 10 8 & 31 34 & 12 0 12\\
\index{Ham20}\rowlabel{a:Ham20}Ham20 \href{http://dx.doi.org/10.1080/00207543.2019.1709671}{Ham20} & \hyperref[auth:a750]{A. Ham} & Transfer-robot task scheduling in job shop & \hyperref[detail:Ham20]{Details} No & \cite{Ham20} & 2020 & \cellcolor{red!20}International Journal of Production Research & 11 & \noindent{}\textcolor{black!50}{0.00} \textcolor{black!50}{0.00} n/a & 27 19 37 & 27 41 & 12 5 7\\
\index{Ham20a}\rowlabel{a:Ham20a}Ham20a \href{http://dx.doi.org/10.1109/tase.2019.2952523}{Ham20a} & \hyperref[auth:a750]{A. Ham} & Drone-Based Material Transfer System in a Robotic Mobile Fulfillment Center & \hyperref[detail:Ham20a]{Details} \href{../works/Ham20a.pdf}{Yes} & \cite{Ham20a} & 2020 & IEEE Transactions on Automation Science and Engineering & 9 & \noindent{}\textcolor{black!50}{0.00} \textcolor{black!50}{0.00} \textbf{9.90} & 15 19 19 & 27 41 & 8 1 7\\
\index{HauderBRPA20}\rowlabel{a:HauderBRPA20}HauderBRPA20 \href{http://dx.doi.org/10.1016/j.cie.2020.106857}{HauderBRPA20} & \hyperref[auth:a550]{V. A. Hauder}, \hyperref[auth:a551]{A. Beham}, \hyperref[auth:a552]{S. Raggl}, \hyperref[auth:a553]{S. N. Parragh}, \hyperref[auth:a554]{M. Affenzeller} & \cellcolor{green!10}Resource-constrained multi-project scheduling with activity and time flexibility & \hyperref[detail:HauderBRPA20]{Details} \href{../works/HauderBRPA20.pdf}{Yes} & \cite{HauderBRPA20} & 2020 & Computers \  Industrial Engineering & 14 & \noindent{}\textcolor{black!50}{0.00} \textcolor{black!50}{0.00} \textbf{25.41} & 14 19 27 & 46 56 & 17 3 14\\
\index{KizilayC20}\rowlabel{a:KizilayC20}KizilayC20 \href{http://dx.doi.org/10.1080/0305215x.2020.1786081}{KizilayC20} & \hyperref[auth:a1380]{D. Kizilay}, \hyperref[auth:a1381]{Z. A. Cil} & Constraint programming approach for multi-objective two-sided assembly line balancing problem with multi-operator stations & \hyperref[detail:KizilayC20]{Details} No & \cite{KizilayC20} & 2020 & \cellcolor{red!20}Engineering Optimization & 16 & \noindent{}\textcolor{black!50}{0.00} \textcolor{black!50}{0.00} n/a & 11 12 14 & 38 39 & 12 6 6\\
\index{Kong2020}\rowlabel{a:Kong2020}Kong2020 \href{http://dx.doi.org/10.1061/(asce)co.1943-7862.0001929}{Kong2020} & \hyperref[auth:a1706]{F. Kong}, \hyperref[auth:a1780]{D. Dou} & RCPSP with Combined Precedence Relations and Resource Calendars & \hyperref[detail:Kong2020]{Details} No & \cite{Kong2020} & 2020 & Journal of Construction Engineering and Management & null & \noindent{}\textcolor{black!50}{0.00} \textcolor{black!50}{0.00} n/a & 5 6 6 & 39 44 & 5 1 4\\
\index{Li2020}\rowlabel{a:Li2020}Li2020 \href{http://dx.doi.org/10.1007/s10732-019-09434-9}{Li2020} & \hyperref[auth:a1796]{H. Li}, \hyperref[auth:a1811]{G. Feng}, \hyperref[auth:a1812]{M. Yin} & On combining variable ordering heuristics for constraint satisfaction problems & \hyperref[detail:Li2020]{Details} No & \cite{Li2020} & 2020 & Journal of Heuristics & null & \noindent{}0.50 0.50 n/a & 2 2 3 & 28 36 & 7 1 6\\
\index{Liu2020}\rowlabel{a:Liu2020}Liu2020 \href{http://dx.doi.org/10.3390/app10248887}{Liu2020} & \hyperref[auth:a1244]{S.-S. Liu}, \hyperref[auth:a1494]{H.-Y. Huang}, \hyperref[auth:a1495]{N. R. D. Kumala} & \cellcolor{gold!20}Two-Stage Optimization Model for Life Cycle Maintenance Scheduling of Bridge Infrastructure \hyperref[abs:Liu2020]{Abstract} & \hyperref[detail:Liu2020]{Details} No & \cite{Liu2020} & 2020 & Applied Sciences & null & \noindent{}\textcolor{black!50}{0.00} \textbf{4.00} n/a & 3 3 3 & 54 60 & 7 2 5\\
\index{LunardiBLRV20}\rowlabel{a:LunardiBLRV20}LunardiBLRV20 \href{https://doi.org/10.1016/j.cor.2020.105020}{LunardiBLRV20} & \hyperref[auth:a505]{W. T. Lunardi}, \hyperref[auth:a506]{E. G. Birgin}, \hyperref[auth:a118]{P. Laborie}, \hyperref[auth:a507]{D. P. Ronconi}, \hyperref[auth:a508]{H. Voos} & \cellcolor{green!10}Mixed Integer linear programming and constraint programming models for the online printing shop scheduling problem & \hyperref[detail:LunardiBLRV20]{Details} \href{../works/LunardiBLRV20.pdf}{Yes} & \cite{LunardiBLRV20} & 2020 & Computers \  Operations Research & 20 & \noindent{}\textbf{1.00} \textbf{1.00} \textbf{46.75} & 30 36 39 & 18 24 & 16 13 3\\
\index{MejiaY20}\rowlabel{a:MejiaY20}MejiaY20 \href{https://doi.org/10.1016/j.ejor.2020.02.010}{MejiaY20} & \hyperref[auth:a424]{G. Mej{\'{\i}}a}, \hyperref[auth:a405]{F. Yuraszeck} & A self-tuning variable neighborhood search algorithm and an effective decoding scheme for open shop scheduling problems with travel/setup times & \hyperref[detail:MejiaY20]{Details} \href{../works/MejiaY20.pdf}{Yes} & \cite{MejiaY20} & 2020 & European Journal of Operational Research & 13 & \noindent{}\textcolor{black!50}{0.00} \textcolor{black!50}{0.00} \textbf{12.71} & 24 29 34 & 45 50 & 6 4 2\\
\index{MengZRZL20}\rowlabel{a:MengZRZL20}MengZRZL20 \href{https://doi.org/10.1016/j.cie.2020.106347}{MengZRZL20} & \hyperref[auth:a500]{L. Meng}, \hyperref[auth:a501]{C. Zhang}, \hyperref[auth:a502]{Y. Ren}, \hyperref[auth:a503]{B. Zhang}, \hyperref[auth:a504]{C. Lv} & Mixed-integer linear programming and constraint programming formulations for solving distributed flexible job shop scheduling problem & \hyperref[detail:MengZRZL20]{Details} \href{../works/MengZRZL20.pdf}{Yes} & \cite{MengZRZL20} & 2020 & Computers \  Industrial Engineering & 13 & \noindent{}\textbf{2.00} \textbf{2.00} \textbf{36.82} & 100 133 152 & 62 69 & 26 16 10\\
\index{Mnif2020}\rowlabel{a:Mnif2020}Mnif2020 \href{http://dx.doi.org/10.4018/ijamc.2020040107}{Mnif2020} & \hyperref[auth:a1964]{M. G. Mnif}, \hyperref[auth:a1965]{S. Bouamama} & Multi-Layer Distributed Constraint Satisfaction for Multi-criteria Optimization Problem \hyperref[abs:Mnif2020]{Abstract} & \hyperref[detail:Mnif2020]{Details} No & \cite{Mnif2020} & 2020 & International Journal of Applied Metaheuristic Computing & null & \noindent{}\textcolor{black!50}{0.00} \textbf{6.01} n/a & 2 2 2 & 11 19 & 1 0 1\\
\index{MokhtarzadehTNF20}\rowlabel{a:MokhtarzadehTNF20}MokhtarzadehTNF20 \href{https://doi.org/10.1080/0951192X.2020.1736713}{MokhtarzadehTNF20} & \hyperref[auth:a515]{M. Mokhtarzadeh}, \hyperref[auth:a430]{R. Tavakkoli-Moghaddam}, \hyperref[auth:a432]{B. V. Nouri}, \hyperref[auth:a516]{A. Farsi} & Scheduling of human-robot collaboration in assembly of printed circuit boards: a constraint programming approach & \hyperref[detail:MokhtarzadehTNF20]{Details} \href{../works/MokhtarzadehTNF20.pdf}{Yes} & \cite{MokhtarzadehTNF20} & 2020 & Int. J. Comput. Integr. Manuf. & 14 & \noindent{}\textbf{1.00} \textbf{1.00} \textbf{15.70} & 25 29 30 & 32 34 & 10 3 7\\
\index{PinarbasiA20}\rowlabel{a:PinarbasiA20}PinarbasiA20 \href{http://dx.doi.org/10.1080/0305215x.2020.1716746}{PinarbasiA20} & \hyperref[auth:a1384]{M. Pınarbaşı}, \hyperref[auth:a764]{H. M. Alakaş} & Balancing stochastic type-II assembly lines: chance-constrained mixed integer and constraint programming models & \hyperref[detail:PinarbasiA20]{Details} No & \cite{PinarbasiA20} & 2020 & \cellcolor{red!20}Engineering Optimization & 18 & \noindent{}\textcolor{black!50}{0.00} \textcolor{black!50}{0.00} n/a & 7 9 11 & 32 36 & 10 3 7\\
\index{Polo-MejiaALB20}\rowlabel{a:Polo-MejiaALB20}Polo-MejiaALB20 \href{https://doi.org/10.1080/00207543.2019.1693654}{Polo-MejiaALB20} & \hyperref[auth:a517]{O. Polo-Mej{\'{\i}}a}, \hyperref[auth:a6]{C. Artigues}, \hyperref[auth:a3]{P. Lopez}, \hyperref[auth:a518]{V. Basini} & \cellcolor{green!10}Mixed-integer/linear and constraint programming approaches for activity scheduling in a nuclear research facility & \hyperref[detail:Polo-MejiaALB20]{Details} \href{../works/Polo-MejiaALB20.pdf}{Yes} & \cite{Polo-MejiaALB20} & 2020 & \cellcolor{red!20}International Journal of Production Research & 18 & \noindent{}\textbf{1.50} \textbf{1.50} \textbf{14.35} & 8 10 11 & 23 36 & 8 2 6\\
\index{QinDCS20}\rowlabel{a:QinDCS20}QinDCS20 \href{https://doi.org/10.1016/j.ejor.2020.02.021}{QinDCS20} & \hyperref[auth:a509]{T. Qin}, \hyperref[auth:a510]{Y. Du}, \hyperref[auth:a511]{J. H. Chen}, \hyperref[auth:a512]{M. Sha} & Combining mixed integer programming and constraint programming to solve the integrated scheduling problem of container handling operations of a single vessel & \hyperref[detail:QinDCS20]{Details} \href{../works/QinDCS20.pdf}{Yes} & \cite{QinDCS20} & 2020 & European Journal of Operational Research & 18 & \noindent{}\textbf{1.00} \textbf{1.00} \textbf{20.41} & 27 30 31 & 30 35 & 9 6 3\\
\index{Relich2020}\rowlabel{a:Relich2020}Relich2020 \href{http://dx.doi.org/10.3390/app10186330}{Relich2020} & \hyperref[auth:a1646]{M. Relich}, \hyperref[auth:a1647]{A. Świć} & \cellcolor{gold!20}Parametric Estimation and Constraint Programming-Based Planning and Simulation of Production Cost of a New Product \hyperref[abs:Relich2020]{Abstract} & \hyperref[detail:Relich2020]{Details} No & \cite{Relich2020} & 2020 & Applied Sciences & null & \noindent{}\textcolor{black!50}{0.00} \textbf{1.00} n/a & 14 15 16 & 24 40 & 5 2 3\\
\index{RoshanaeiBAUB20}\rowlabel{a:RoshanaeiBAUB20}RoshanaeiBAUB20 \href{http://dx.doi.org/10.1016/j.ijpe.2019.07.006}{RoshanaeiBAUB20} & \hyperref[auth:a728]{V. Roshanaei}, \hyperref[auth:a203]{K. E. C. Booth}, \hyperref[auth:a895]{D. M. Aleman}, \hyperref[auth:a896]{D. R. Urbach}, \hyperref[auth:a89]{J. C. Beck} & Branch-and-check methods for multi-level operating room planning and scheduling & \hyperref[detail:RoshanaeiBAUB20]{Details} \href{../works/RoshanaeiBAUB20.pdf}{Yes} & \cite{RoshanaeiBAUB20} & 2020 & International Journal of Production Economics & 19 & \noindent{}\textcolor{black!50}{0.00} \textcolor{black!50}{0.00} \textbf{2.56} & 24 29 29 & 43 56 & 20 9 11\\
\index{SacramentoSP20}\rowlabel{a:SacramentoSP20}SacramentoSP20 \href{https://doi.org/10.1007/s43069-020-00036-x}{SacramentoSP20} & \hyperref[auth:a519]{D. Sacramento}, \hyperref[auth:a85]{C. Solnon}, \hyperref[auth:a520]{D. Pisinger} & \cellcolor{gold!20}Constraint Programming and Local Search Heuristic: a Matheuristic Approach for Routing and Scheduling Feeder Vessels in Multi-terminal Ports & \hyperref[detail:SacramentoSP20]{Details} \href{../works/SacramentoSP20.pdf}{Yes} & \cite{SacramentoSP20} & 2020 & Oper. Res. Forum & 33 & \noindent{}\textbf{1.00} \textbf{1.00} \textbf{14.30} & 2 4 5 & 38 46 & 13 2 11\\
\index{Tesch2020}\rowlabel{a:Tesch2020}Tesch2020 \href{http://dx.doi.org/10.1007/s10951-020-00647-6}{Tesch2020} & \hyperref[auth:a183]{A. Tesch} & \cellcolor{gold!20}A polyhedral study of event-based models for the resource-constrained project scheduling problem \hyperref[abs:Tesch2020]{Abstract} & \hyperref[detail:Tesch2020]{Details} No & \cite{Tesch2020} & 2020 & Journal of Scheduling & null & \noindent{}\textcolor{black!50}{0.00} \textcolor{black!50}{0.00} n/a & 6 7 7 & 29 42 & 12 1 11\\
\index{WallaceY20}\rowlabel{a:WallaceY20}WallaceY20 \href{https://doi.org/10.1007/s10601-020-09316-z}{WallaceY20} & \hyperref[auth:a117]{M. G. Wallace}, \hyperref[auth:a19]{N. Yorke-Smith} & \cellcolor{gold!20}A new constraint programming model and solving for the cyclic hoist scheduling problem & \hyperref[detail:WallaceY20]{Details} \href{../works/WallaceY20.pdf}{Yes} & \cite{WallaceY20} & 2020 & Constraints An Int. J. & 19 & \noindent{}\textbf{1.00} \textbf{1.00} \textbf{4.76} & 5 6 5 & 18 23 & 6 3 3\\
\index{Watermeyer2020}\rowlabel{a:Watermeyer2020}Watermeyer2020 \href{http://dx.doi.org/10.1007/s00291-020-00583-z}{Watermeyer2020} & \hyperref[auth:a1770]{K. Watermeyer}, \hyperref[auth:a1771]{J. Zimmermann} & \cellcolor{gold!20}A branch-and-bound procedure for the resource-constrained project scheduling problem with partially renewable resources and general temporal constraints \hyperref[abs:Watermeyer2020]{Abstract} & \hyperref[detail:Watermeyer2020]{Details} No & \cite{Watermeyer2020} & 2020 & OR Spectrum & null & \noindent{}\textcolor{black!50}{0.00} \textcolor{black!50}{0.00} n/a & 12 17 21 & 26 36 & 7 1 6\\
\index{ZarandiASC20}\rowlabel{a:ZarandiASC20}ZarandiASC20 \href{https://doi.org/10.1007/s10462-018-9667-6}{ZarandiASC20} & \hyperref[auth:a829]{M. H. F. Zarandi}, \hyperref[auth:a830]{A. A. S. Asl}, \hyperref[auth:a831]{S. Sotudian}, \hyperref[auth:a832]{O. Castillo} & A state of the art review of intelligent scheduling & \hyperref[detail:ZarandiASC20]{Details} \href{../works/ZarandiASC20.pdf}{Yes} & \cite{ZarandiASC20} & 2020 & Artif. Intell. Rev. & 93 & \noindent{}\textcolor{black!50}{0.00} \textcolor{black!50}{0.00} \textbf{440.67} & 55 64 66 & 445 538 & 66 3 63\\
\index{ZouZ20}\rowlabel{a:ZouZ20}ZouZ20 \href{https://api.semanticscholar.org/CorpusID:208840808}{ZouZ20} & \hyperref[auth:a756]{X. Zou}, \hyperref[auth:a757]{L. Zhang} & A constraint programming approach for scheduling repetitive projects with atypical activities considering soft logic & \hyperref[detail:ZouZ20]{Details} \href{../works/ZouZ20.pdf}{Yes} & \cite{ZouZ20} & 2020 & Automation in Construction & 10 & \noindent{}\textbf{1.00} \textbf{1.00} \textbf{5.88} & 18 21 19 & 48 52 & 7 3 4\\
\index{Abuwarda2019}\rowlabel{a:Abuwarda2019}Abuwarda2019 \href{http://dx.doi.org/10.1139/cjce-2018-0544}{Abuwarda2019} & \hyperref[auth:a1520]{Z. Abuwarda}, \hyperref[auth:a1521]{T. Hegazy} & \cellcolor{gold!20}Multi-dimensional optimization model for schedule fast-tracking without over-stressing construction workers \hyperref[abs:Abuwarda2019]{Abstract} & \hyperref[detail:Abuwarda2019]{Details} No & \cite{Abuwarda2019} & 2019 & Canadian Journal of Civil Engineering & null & \noindent{}\textcolor{black!50}{0.00} \textbf{2.00} n/a & 4 5 5 & 46 48 & 6 0 6\\
\index{AlakaPY19}\rowlabel{a:AlakaPY19}AlakaPY19 \href{http://dx.doi.org/10.1007/s00500-019-04294-8}{AlakaPY19} & \hyperref[auth:a764]{H. M. Alakaş}, \hyperref[auth:a1384]{M. Pınarbaşı}, \hyperref[auth:a1425]{M. Y\"{u}z\"{u}kırmızı} & Constraint programming model for resource-constrained assembly line balancing problem & \hyperref[detail:AlakaPY19]{Details} \href{../works/AlakaPY19.pdf}{Yes} & \cite{AlakaPY19} & 2019 & Soft Computing & 9 & \noindent{}0.50 0.50 \textbf{14.19} & 15 17 0 & 14 23 & 11 6 5\\
\index{ArkhipovBL19}\rowlabel{a:ArkhipovBL19}ArkhipovBL19 \href{http://dx.doi.org/10.1016/j.ejor.2018.11.005}{ArkhipovBL19} & \hyperref[auth:a924]{D. Arkhipov}, \hyperref[auth:a925]{O. Battaïa}, \hyperref[auth:a926]{A. Lazarev} & \cellcolor{gold!20}An efficient pseudo-polynomial algorithm for finding a lower bound on the makespan for the Resource Constrained Project Scheduling Problem & \hyperref[detail:ArkhipovBL19]{Details} \href{../works/ArkhipovBL19.pdf}{Yes} & \cite{ArkhipovBL19} & 2019 & European Journal of Operational Research & 10 & \noindent{}\textcolor{black!50}{0.00} \textcolor{black!50}{0.00} \textbf{2.65} & 12 13 14 & 24 40 & 12 1 11\\
\index{Benda2019}\rowlabel{a:Benda2019}Benda2019 \href{http://dx.doi.org/10.1007/s00291-019-00567-8}{Benda2019} & \hyperref[auth:a1966]{F. Benda}, \hyperref[auth:a1512]{R. Braune}, \hyperref[auth:a1967]{K. F. Doerner}, \hyperref[auth:a951]{R. F. Hartl} & \cellcolor{gold!20}A machine learning approach for flow shop scheduling problems with alternative resources, sequence-dependent setup times, and blocking \hyperref[abs:Benda2019]{Abstract} & \hyperref[detail:Benda2019]{Details} No & \cite{Benda2019} & 2019 & OR Spectrum & null & \noindent{}\textcolor{black!50}{0.00} \textbf{6.01} n/a & 12 14 18 & 14 22 & 0 0 0\\
\index{Chakrabortty2019}\rowlabel{a:Chakrabortty2019}Chakrabortty2019 \href{http://dx.doi.org/10.1111/itor.12644}{Chakrabortty2019} & \hyperref[auth:a1614]{R. K. Chakrabortty}, \hyperref[auth:a1615]{A. Abbasi}, \hyperref[auth:a1616]{M. J. Ryan} & \cellcolor{gold!20}Multi‐mode resource‐constrained project scheduling using modified variable neighborhood search heuristic \hyperref[abs:Chakrabortty2019]{Abstract} & \hyperref[detail:Chakrabortty2019]{Details} No & \cite{Chakrabortty2019} & 2019 & International Transactions in Operational Research & null & \noindent{}\textcolor{black!50}{0.00} \textcolor{black!50}{0.00} n/a & 29 37 41 & 81 85 & 10 3 7\\
\index{ColT2019a}\rowlabel{a:ColT2019a}ColT2019a \href{http://dx.doi.org/10.4204/eptcs.306.30}{ColT2019a} & \hyperref[auth:a93]{G. D. Col}, \hyperref[auth:a608]{E. Teppan} & \cellcolor{gold!20}Google vs IBM: A Constraint Solving Challenge on the Job-Shop Scheduling Problem & \hyperref[detail:ColT2019a]{Details} \href{../works/ColT2019a.pdf}{Yes} & \cite{ColT2019a} & 2019 & Electronic Proceedings in Theoretical Computer Science & 7 & \noindent{}\textcolor{black!50}{0.00} \textcolor{black!50}{0.00} \textbf{4.42} & 10 13 11 & 10 18 & 10 2 8\\
\index{Cox2019}\rowlabel{a:Cox2019}Cox2019 \href{http://dx.doi.org/10.4114/intartif.vol22iss63pp1-15}{Cox2019} & \hyperref[auth:a1920]{J. L. Cox}, \hyperref[auth:a1921]{S. Lucci}, \hyperref[auth:a1922]{T. Pay} & \cellcolor{gold!20}Effects of Dynamic Variable - Value Ordering  Heuristics on the Search Space of Sudoku Modeled as a Constraint Satisfaction Problem \hyperref[abs:Cox2019]{Abstract} & \hyperref[detail:Cox2019]{Details} No & \cite{Cox2019} & 2019 & Inteligencia Artificial & null & \noindent{}0.50 0.50 n/a & 1 1 2 & 0 0 & 1 1 0\\
\index{EdwardsBSE19}\rowlabel{a:EdwardsBSE19}EdwardsBSE19 \href{http://dx.doi.org/10.1080/01605682.2019.1595192}{EdwardsBSE19} & \hyperref[auth:a892]{S. J. Edwards}, \hyperref[auth:a893]{D. Baatar}, \hyperref[auth:a894]{K. Smith-Miles}, \hyperref[auth:a469]{A. T. Ernst} & Symmetry breaking of identical projects in the high-multiplicity RCPSP/max & \hyperref[detail:EdwardsBSE19]{Details} No & \cite{EdwardsBSE19} & 2019 & \cellcolor{red!20}Journal of the Operational Research Society & 22 & \noindent{}\textcolor{black!50}{0.00} \textcolor{black!50}{0.00} n/a & 3 3 3 & 40 51 & 17 1 16\\
\index{EscobetPQPRA19}\rowlabel{a:EscobetPQPRA19}EscobetPQPRA19 \href{https://doi.org/10.1016/j.compchemeng.2018.08.040}{EscobetPQPRA19} & \hyperref[auth:a525]{T. Escobet}, \hyperref[auth:a526]{V. Puig}, \hyperref[auth:a527]{J. Quevedo}, \hyperref[auth:a528]{P. Pal{\`{a}}-Sch{\"{o}}nw{\"{a}}lder}, \hyperref[auth:a529]{J. Romera}, \hyperref[auth:a530]{W. Adelman} & \cellcolor{green!10}Optimal batch scheduling of a multiproduct dairy process using a combined optimization/constraint programming approach & \hyperref[detail:EscobetPQPRA19]{Details} \href{../works/EscobetPQPRA19.pdf}{Yes} & \cite{EscobetPQPRA19} & 2019 & Computers \  Chemical Engineering & 10 & \noindent{}\textbf{1.00} \textbf{1.00} \textbf{7.68} & 17 17 17 & 18 25 & 6 0 6\\
\index{Geiger2019}\rowlabel{a:Geiger2019}Geiger2019 \href{http://dx.doi.org/10.1613/jair.1.11303}{Geiger2019} & \hyperref[auth:a1829]{M. J. Geiger}, \hyperref[auth:a78]{L. Kletzander}, \hyperref[auth:a45]{N. Musliu} & \cellcolor{gold!20}Solving the Torpedo Scheduling Problem \hyperref[abs:Geiger2019]{Abstract} & \hyperref[detail:Geiger2019]{Details} No & \cite{Geiger2019} & 2019 & Journal of Artificial Intelligence Research & null & \noindent{}\textcolor{black!50}{0.00} \textbf{1.50} n/a & 4 6 6 & 0 0 & 1 1 0\\
\index{GurEA19}\rowlabel{a:GurEA19}GurEA19 \href{https://api.semanticscholar.org/CorpusID:88492001}{GurEA19} & \hyperref[auth:a763]{Şeyda G{\"u}r}, \hyperref[auth:a415]{T. Eren}, \hyperref[auth:a764]{H. M. Alakaş} & \cellcolor{gold!20}Surgical Operation Scheduling with Goal Programming and Constraint Programming: A Case Study & \hyperref[detail:GurEA19]{Details} \href{../works/GurEA19.pdf}{Yes} & \cite{GurEA19} & 2019 & Mathematics & 24 & \noindent{}\textbf{1.00} \textbf{1.00} \textbf{4.44} & 19 21 19 & 30 49 & 4 4 0\\
\index{He2019}\rowlabel{a:He2019}He2019 \href{http://dx.doi.org/10.1007/s10845-019-01518-4}{He2019} & \hyperref[auth:a1547]{L. He}, \hyperref[auth:a308]{M. de Weerdt}, \hyperref[auth:a19]{N. Yorke-Smith} & \cellcolor{gold!20}Time/sequence-dependent scheduling: the design and evaluation of a general purpose tabu-based adaptive large neighbourhood search algorithm \hyperref[abs:He2019]{Abstract} & \hyperref[detail:He2019]{Details} No & \cite{He2019} & 2019 & Journal of Intelligent Manufacturing & null & \noindent{}\textcolor{black!50}{0.00} \textbf{1.50} n/a & 24 33 36 & 34 43 & 3 2 1\\
\index{HechingHK19}\rowlabel{a:HechingHK19}HechingHK19 \href{http://dx.doi.org/10.1287/trsc.2018.0830}{HechingHK19} & \hyperref[auth:a1021]{A. Heching}, \hyperref[auth:a160]{J. N. Hooker}, \hyperref[auth:a1022]{R. Kimura} & \cellcolor{gold!20}A Logic-Based Benders Approach to Home Healthcare Delivery & \hyperref[detail:HechingHK19]{Details} No & \cite{HechingHK19} & 2019 & \cellcolor{red!20}Transportation Science & 13 & \noindent{}\textcolor{black!50}{0.00} \textcolor{black!50}{0.00} n/a & 35 42 37 & 29 32 & 20 8 12\\
\index{Hosseinian2019}\rowlabel{a:Hosseinian2019}Hosseinian2019 \href{http://dx.doi.org/10.1108/jm2-07-2018-0098}{Hosseinian2019} & \hyperref[auth:a1573]{A. H. Hosseinian}, \hyperref[auth:a1574]{V. Baradaran}, \hyperref[auth:a1575]{M. Bashiri} & Modeling of the time-dependent multi-skilled RCPSP considering learning effect \hyperref[abs:Hosseinian2019]{Abstract} & \hyperref[detail:Hosseinian2019]{Details} No & \cite{Hosseinian2019} & 2019 & Journal of Modelling in Management & null & \noindent{}\textcolor{black!50}{0.00} \textcolor{black!50}{0.00} n/a & 19 27 31 & 44 53 & 7 4 3\\
\index{HoundjiSW19}\rowlabel{a:HoundjiSW19}HoundjiSW19 \href{https://doi.org/10.1007/s10601-018-9300-y}{HoundjiSW19} & \hyperref[auth:a223]{V. R. Houndji}, \hyperref[auth:a147]{P. Schaus}, \hyperref[auth:a224]{L. A. Wolsey} & The item dependent stockingcost constraint & \hyperref[detail:HoundjiSW19]{Details} \href{../works/HoundjiSW19.pdf}{Yes} & \cite{HoundjiSW19} & 2019 & Constraints An Int. J. & 27 & \noindent{}\textcolor{black!50}{0.00} \textcolor{black!50}{0.00} \textbf{2.91} & 0 0 0 & 17 28 & 5 0 5\\
\index{Kizilay2019}\rowlabel{a:Kizilay2019}Kizilay2019 \href{http://dx.doi.org/10.3390/a12050100}{Kizilay2019} & \hyperref[auth:a1380]{D. Kizilay}, \hyperref[auth:a1973]{M. F. Tasgetiren}, \hyperref[auth:a1974]{Q.-K. Pan}, \hyperref[auth:a1975]{L. Gao} & \cellcolor{gold!20}A Variable Block Insertion Heuristic for Solving Permutation Flow Shop Scheduling Problem with Makespan Criterion \hyperref[abs:Kizilay2019]{Abstract} & \hyperref[detail:Kizilay2019]{Details} No & \cite{Kizilay2019} & 2019 & Algorithms & null & \noindent{}\textcolor{black!50}{0.00} \textbf{5.00} n/a & 20 21 22 & 36 42 & 2 0 2\\
\index{KonowalenkoMM19}\rowlabel{a:KonowalenkoMM19}KonowalenkoMM19 \href{http://dx.doi.org/10.1109/tla.2019.8932340}{KonowalenkoMM19} & \hyperref[auth:a1466]{E. Konowalenko}, \hyperref[auth:a1467]{W. Meira}, \hyperref[auth:a1468]{L. Magatao} & Constraint Logic Programming Applied to Sequencing Tasks in a Pipeline Network \hyperref[abs:KonowalenkoMM19]{Abstract} & \hyperref[detail:KonowalenkoMM19]{Details} \href{../works/KonowalenkoMM19.pdf}{Yes} & \cite{KonowalenkoMM19} & 2019 & IEEE LATIN AMERICA TRANSACTIONS & 9 & \noindent{}\textbf{1.00} \textbf{4.00} 0.57 & 0 0 0 & 0 0 & 0 0 0\\
\index{Lozano2019}\rowlabel{a:Lozano2019}Lozano2019 \href{http://dx.doi.org/10.1145/3332373}{Lozano2019} & \hyperref[auth:a1522]{R. C. Lozano}, \hyperref[auth:a91]{M. Carlsson}, \hyperref[auth:a1523]{G. H. Blindell}, \hyperref[auth:a92]{C. Schulte} & \cellcolor{green!10}Combinatorial Register Allocation and Instruction Scheduling \hyperref[abs:Lozano2019]{Abstract} & \hyperref[detail:Lozano2019]{Details} No & \cite{Lozano2019} & 2019 & ACM Transactions on Programming Languages and Systems & null & \noindent{}\textcolor{black!50}{0.00} \textbf{1.50} n/a & 10 13 16 & 56 100 & 8 0 8\\
\index{Lozano2019a}\rowlabel{a:Lozano2019a}Lozano2019a \href{http://dx.doi.org/10.1145/3200920}{Lozano2019a} & \hyperref[auth:a1522]{R. C. Lozano}, \hyperref[auth:a92]{C. Schulte} & \cellcolor{green!10}Survey on Combinatorial Register Allocation and Instruction Scheduling \hyperref[abs:Lozano2019a]{Abstract} & \hyperref[detail:Lozano2019a]{Details} No & \cite{Lozano2019a} & 2019 & ACM Computing Surveys & null & \noindent{}\textcolor{black!50}{0.00} \textbf{2.50} n/a & 5 7 8 & 97 154 & 9 0 9\\
\index{NattafDYW19}\rowlabel{a:NattafDYW19}NattafDYW19 \href{https://doi.org/10.1016/j.cor.2019.03.004}{NattafDYW19} & \hyperref[auth:a81]{M. Nattaf}, \hyperref[auth:a993]{S. Dauz{\`{e}}re-P{\'{e}}r{\`{e}}s}, \hyperref[auth:a994]{C. Yugma}, \hyperref[auth:a995]{C.-H. Wu} & \cellcolor{gold!20}Parallel machine scheduling with time constraints on machine qualifications & \hyperref[detail:NattafDYW19]{Details} \href{../works/NattafDYW19.pdf}{Yes} & \cite{NattafDYW19} & 2019 & Computers \  Operations Research & 16 & \noindent{}\textcolor{black!50}{0.00} \textcolor{black!50}{0.00} \textbf{21.82} & 14 22 22 & 21 29 & 6 3 3\\
\index{NattafHKAL19}\rowlabel{a:NattafHKAL19}NattafHKAL19 \href{https://doi.org/10.1016/j.dam.2018.11.008}{NattafHKAL19} & \hyperref[auth:a81]{M. Nattaf}, \hyperref[auth:a996]{M. Horv{\'{a}}th}, \hyperref[auth:a155]{T. Kis}, \hyperref[auth:a6]{C. Artigues}, \hyperref[auth:a3]{P. Lopez} & \cellcolor{gold!20}Polyhedral results and valid inequalities for the continuous energy-constrained scheduling problem & \hyperref[detail:NattafHKAL19]{Details} \href{../works/NattafHKAL19.pdf}{Yes} & \cite{NattafHKAL19} & 2019 & Discrete Applied Mathematics & 16 & \noindent{}\textcolor{black!50}{0.00} \textcolor{black!50}{0.00} 0.49 & 5 6 5 & 12 17 & 5 0 5\\
\index{NishikawaSTT19}\rowlabel{a:NishikawaSTT19}NishikawaSTT19 \href{http://www.ijnc.org/index.php/ijnc/article/view/201}{NishikawaSTT19} & \hyperref[auth:a531]{H. Nishikawa}, \hyperref[auth:a532]{K. Shimada}, \hyperref[auth:a533]{I. Taniguchi}, \hyperref[auth:a534]{H. Tomiyama} & A Constraint Programming Approach to Scheduling of Malleable Tasks & \hyperref[detail:NishikawaSTT19]{Details} \href{../works/NishikawaSTT19.pdf}{Yes} & \cite{NishikawaSTT19} & 2019 & Int. J. Netw. Comput. & 16 & \noindent{}\textbf{2.00} \textbf{2.00} \textbf{51.27} & 3 3 0 & 20 30 & 5 0 5\\
\index{Novas19}\rowlabel{a:Novas19}Novas19 \href{https://doi.org/10.1016/j.cie.2019.07.011}{Novas19} & \hyperref[auth:a524]{J. M. Novas} & Production scheduling and lot streaming at flexible job-shops environments using constraint programming & \hyperref[detail:Novas19]{Details} \href{../works/Novas19.pdf}{Yes} & \cite{Novas19} & 2019 & Computers \  Industrial Engineering & 13 & \noindent{}\textbf{2.00} \textbf{2.00} \textbf{8.25} & 30 35 43 & 29 40 & 10 4 6\\
\index{Ozder2019}\rowlabel{a:Ozder2019}Ozder2019 \href{http://dx.doi.org/10.3390/math7020192}{Ozder2019} & \hyperref[auth:a1753]{E. Özder}, \hyperref[auth:a1754]{E. Özcan}, \hyperref[auth:a415]{T. Eren} & \cellcolor{gold!20}Staff Task-Based Shift Scheduling Solution with an ANP and Goal Programming Method in a Natural Gas Combined Cycle Power Plant \hyperref[abs:Ozder2019]{Abstract} & \hyperref[detail:Ozder2019]{Details} No & \cite{Ozder2019} & 2019 & Mathematics & null & \noindent{}\textcolor{black!50}{0.00} \textcolor{black!50}{0.00} n/a & 24 26 18 & 58 75 & 2 0 2\\
\index{PinarbasiAY19}\rowlabel{a:PinarbasiAY19}PinarbasiAY19 \href{http://dx.doi.org/10.1108/aa-12-2018-0262}{PinarbasiAY19} & \hyperref[auth:a413]{M. Pinarbasi}, \hyperref[auth:a1423]{H. M. Alakas}, \hyperref[auth:a1424]{M. Yuzukirmizi} & A constraint programming approach to type-2 assembly line balancing problem with assignment restrictions & \hyperref[detail:PinarbasiAY19]{Details} \href{../works/PinarbasiAY19.pdf}{Yes} & \cite{PinarbasiAY19} & 2019 & Assembly Automation & 14 & \noindent{}\textcolor{black!50}{0.00} \textcolor{black!50}{0.00} \textbf{26.65} & 16 18 0 & 41 68 & 12 7 5\\
\index{Schwarz2019}\rowlabel{a:Schwarz2019}Schwarz2019 \href{http://dx.doi.org/10.1007/s40685-019-00102-z}{Schwarz2019} & \hyperref[auth:a2013]{K. Schwarz}, \hyperref[auth:a2014]{M. Römer}, \hyperref[auth:a2015]{T. Mellouli} & \cellcolor{gold!20}A data-driven hierarchical MILP approach for scheduling clinical pathways: a real-world case study from a German university hospital \hyperref[abs:Schwarz2019]{Abstract} & \hyperref[detail:Schwarz2019]{Details} No & \cite{Schwarz2019} & 2019 & Business Research & null & \noindent{}\textcolor{black!50}{0.00} \textbf{1.50} n/a & 2 2 3 & 50 58 & 1 0 1\\
\index{SunTB19}\rowlabel{a:SunTB19}SunTB19 \href{http://dx.doi.org/10.1016/j.ejor.2018.08.009}{SunTB19} & \hyperref[auth:a1195]{D. Sun}, \hyperref[auth:a1196]{L. Tang}, \hyperref[auth:a1197]{R. Baldacci} & \cellcolor{green!10}A Benders decomposition-based framework for solving quay crane scheduling problems & \hyperref[detail:SunTB19]{Details} \href{../works/SunTB19.pdf}{Yes} & \cite{SunTB19} & 2019 & European Journal of Operational Research & 12 & \noindent{}\textcolor{black!50}{0.00} \textcolor{black!50}{0.00} \textcolor{black!50}{0.06} & 31 34 37 & 29 31 & 9 3 6\\
\index{TanZWGQ19}\rowlabel{a:TanZWGQ19}TanZWGQ19 \href{http://dx.doi.org/10.1109/tase.2019.2894093}{TanZWGQ19} & \hyperref[auth:a1183]{Y. Tan}, \hyperref[auth:a1184]{M. Zhou}, \hyperref[auth:a1185]{Y. Wang}, \hyperref[auth:a1186]{X. Guo}, \hyperref[auth:a1187]{L. Qi} & A Hybrid MIP–CP Approach to Multistage Scheduling Problem in Continuous Casting and Hot-Rolling Processes & \hyperref[detail:TanZWGQ19]{Details} \href{../works/TanZWGQ19.pdf}{Yes} & \cite{TanZWGQ19} & 2019 & IEEE Transactions on Automation Science and Engineering & 10 & \noindent{}\textbf{1.00} \textbf{1.00} \textbf{7.38} & 40 45 43 & 40 44 & 7 0 7\\
\index{UnsalO19}\rowlabel{a:UnsalO19}UnsalO19 \href{http://dx.doi.org/10.1016/j.tre.2019.03.018}{UnsalO19} & \hyperref[auth:a1217]{O. Unsal}, \hyperref[auth:a347]{C. Oguz} & An exact algorithm for integrated planning of operations in dry bulk terminals & \hyperref[detail:UnsalO19]{Details} \href{../works/UnsalO19.pdf}{Yes} & \cite{UnsalO19} & 2019 & Transportation Research Part E: Logistics and Transportation Review & 19 & \noindent{}\textcolor{black!50}{0.00} \textcolor{black!50}{0.00} \textbf{3.21} & 44 52 54 & 27 34 & 12 4 8\\
\index{WariZ19}\rowlabel{a:WariZ19}WariZ19 \href{http://dx.doi.org/10.1080/00207543.2019.1571250}{WariZ19} & \hyperref[auth:a839]{E. Wari}, \hyperref[auth:a840]{W. Zhu} & A Constraint Programming model for food processing industry: a case for an ice cream processing facility & \hyperref[detail:WariZ19]{Details} No & \cite{WariZ19} & 2019 & \cellcolor{red!20}International Journal of Production Research & 17 & \noindent{}\textcolor{black!50}{0.00} \textcolor{black!50}{0.00} n/a & 11 11 12 & 42 54 & 18 3 15\\
\index{Wikarek2019}\rowlabel{a:Wikarek2019}Wikarek2019 \href{http://dx.doi.org/10.3233/jifs-179364}{Wikarek2019} & \hyperref[auth:a1476]{J. Wikarek}, \hyperref[auth:a1475]{P. Sitek}, \hyperref[auth:a630]{G. Bocewicz} & Resource constrained portfolio scheduling problem (RCPoSP): A hybrid approach & \hyperref[detail:Wikarek2019]{Details} No & \cite{Wikarek2019} & 2019 & Journal of Intelligent \  Fuzzy Systems & null & \noindent{}\textcolor{black!50}{0.00} \textcolor{black!50}{0.00} n/a & 0 0 0 & 14 22 & 4 0 4\\
\index{WikarekS19}\rowlabel{a:WikarekS19}WikarekS19 \href{https://doi.org/10.1142/S2196888819500027}{WikarekS19} & \hyperref[auth:a535]{J. Wikarek}, \hyperref[auth:a536]{P. Sitek} & \cellcolor{gold!20}A Constraint-Based Declarative Programming Framework for Scheduling and Resource Allocation Problems & \hyperref[detail:WikarekS19]{Details} \href{../works/WikarekS19.pdf}{Yes} & \cite{WikarekS19} & 2019 & Vietnam. J. Comput. Sci. & 22 & \noindent{}\textcolor{black!50}{0.00} \textcolor{black!50}{0.00} \textbf{4.30} & 0 0 0 & 11 16 & 6 0 6\\
\index{Xidias2019}\rowlabel{a:Xidias2019}Xidias2019 \href{http://dx.doi.org/10.1017/dsi.2019.292}{Xidias2019} & \hyperref[auth:a1989]{E. Xidias}, \hyperref[auth:a1990]{P. Azariadis} & \cellcolor{gold!20}Energy Efficient Motion Design and Task Scheduling for an Autonomous Vehicle \hyperref[abs:Xidias2019]{Abstract} & \hyperref[detail:Xidias2019]{Details} No & \cite{Xidias2019} & 2019 & Proceedings of the Design Society: International Conference on Engineering Design & null & \noindent{}\textcolor{black!50}{0.00} \textbf{2.00} n/a & 1 2 3 & 16 19 & 1 0 1\\
\index{YounespourAKE19}\rowlabel{a:YounespourAKE19}YounespourAKE19 \href{https://api.semanticscholar.org/CorpusID:208103305}{YounespourAKE19} & \hyperref[auth:a758]{M. Younespour}, \hyperref[auth:a759]{A. Atighehchian}, \hyperref[auth:a760]{K. Kianfar}, \hyperref[auth:a761]{E. T. Esfahani} & Using mixed integer programming and constraint programming for operating rooms scheduling with modified block strategy & \hyperref[detail:YounespourAKE19]{Details} \href{../works/YounespourAKE19.pdf}{Yes} & \cite{YounespourAKE19} & 2019 & Operations research for health care & 11 & \noindent{}\textbf{1.00} \textbf{1.00} \textbf{5.94} & 7 7 11 & 15 26 & 8 4 4\\
\index{Zhang2019}\rowlabel{a:Zhang2019}Zhang2019 \href{http://dx.doi.org/10.1063/1.5053623}{Zhang2019} & \hyperref[auth:a1745]{Z. Zhang}, \hyperref[auth:a1746]{M. Liu}, \hyperref[auth:a1747]{X. Song} & A bi-level fuzzy random model for multi-mode resource-constrained project scheduling problem of photovoltaic power plant \hyperref[abs:Zhang2019]{Abstract} & \hyperref[detail:Zhang2019]{Details} No & \cite{Zhang2019} & 2019 & Journal of Renewable and Sustainable Energy & null & \noindent{}\textcolor{black!50}{0.00} \textcolor{black!50}{0.00} n/a & 4 5 5 & 55 60 & 2 1 1\\
\index{abs-1901-07914}\rowlabel{a:abs-1901-07914}abs-1901-07914 \href{http://arxiv.org/abs/1901.07914}{abs-1901-07914} & \hyperref[auth:a540]{J. K. Behrens}, \hyperref[auth:a541]{R. Lange}, \hyperref[auth:a542]{M. Mansouri} & A Constraint Programming Approach to Simultaneous Task Allocation and Motion Scheduling for Industrial Dual-Arm Manipulation Tasks & \hyperref[detail:abs-1901-07914]{Details} \href{../works/abs-1901-07914.pdf}{Yes} & \cite{abs-1901-07914} & 2019 & CoRR & 8 & \noindent{}\textbf{2.00} \textbf{2.00} \textbf{4.13} & 0 0 0 & 0 0 & 0 0 0\\
\index{abs-1902-01193}\rowlabel{a:abs-1902-01193}abs-1902-01193 \href{http://arxiv.org/abs/1902.01193}{abs-1902-01193} & \hyperref[auth:a548]{O. M. Alade}, \hyperref[auth:a549]{A. O. Amusat} & Solving Nurse Scheduling Problem Using Constraint Programming Technique & \hyperref[detail:abs-1902-01193]{Details} \href{../works/abs-1902-01193.pdf}{Yes} & \cite{abs-1902-01193} & 2019 & CoRR & 9 & \noindent{}\textbf{1.00} \textbf{1.00} \textbf{1.99} & 0 0 0 & 0 0 & 0 0 0\\
\index{abs-1902-09244}\rowlabel{a:abs-1902-09244}abs-1902-09244 \href{http://arxiv.org/abs/1902.09244}{abs-1902-09244} & \hyperref[auth:a550]{V. A. Hauder}, \hyperref[auth:a551]{A. Beham}, \hyperref[auth:a552]{S. Raggl}, \hyperref[auth:a553]{S. N. Parragh}, \hyperref[auth:a554]{M. Affenzeller} & On constraint programming for a new flexible project scheduling problem with resource constraints & \hyperref[detail:abs-1902-09244]{Details} \href{../works/abs-1902-09244.pdf}{Yes} & \cite{abs-1902-09244} & 2019 & CoRR & 62 & \noindent{}\textbf{1.50} \textbf{1.50} \textbf{350.76} & 0 0 0 & 0 0 & 0 0 0\\
\index{abs-1911-04766}\rowlabel{a:abs-1911-04766}abs-1911-04766 \href{http://arxiv.org/abs/1911.04766}{abs-1911-04766} & \hyperref[auth:a77]{T. Geibinger}, \hyperref[auth:a80]{F. Mischek}, \hyperref[auth:a45]{N. Musliu} & Investigating Constraint Programming and Hybrid Methods for Real World Industrial Test Laboratory Scheduling & \hyperref[detail:abs-1911-04766]{Details} \href{../works/abs-1911-04766.pdf}{Yes} & \cite{abs-1911-04766} & 2019 & CoRR & 16 & \noindent{}\textbf{1.00} \textbf{1.00} \textbf{15.95} & 0 0 0 & 0 0 & 0 0 0\\
\index{BaptisteB18}\rowlabel{a:BaptisteB18}BaptisteB18 \href{https://doi.org/10.1016/j.dam.2017.05.001}{BaptisteB18} & \hyperref[auth:a162]{P. Baptiste}, \hyperref[auth:a704]{N. Bonifas} & \cellcolor{gold!20}Redundant cumulative constraints to compute preemptive bounds & \hyperref[detail:BaptisteB18]{Details} \href{../works/BaptisteB18.pdf}{Yes} & \cite{BaptisteB18} & 2018 & Discrete Applied Mathematics & 10 & \noindent{}\textcolor{black!50}{0.00} \textcolor{black!50}{0.00} \textbf{2.63} & 3 4 4 & 13 19 & 8 2 6\\
\index{BorghesiBLMB18}\rowlabel{a:BorghesiBLMB18}BorghesiBLMB18 \href{https://doi.org/10.1016/j.suscom.2018.05.007}{BorghesiBLMB18} & \hyperref[auth:a226]{A. Borghesi}, \hyperref[auth:a225]{A. Bartolini}, \hyperref[auth:a142]{M. Lombardi}, \hyperref[auth:a143]{M. Milano}, \hyperref[auth:a245]{L. Benini} & \cellcolor{green!10}Scheduling-based power capping in high performance computing systems & \hyperref[detail:BorghesiBLMB18]{Details} \href{../works/BorghesiBLMB18.pdf}{Yes} & \cite{BorghesiBLMB18} & 2018 & Sustain. Comput. Informatics Syst. & 13 & \noindent{}\textcolor{black!50}{0.00} \textcolor{black!50}{0.00} \textbf{9.54} & 11 12 19 & 22 66 & 2 0 2\\
\index{BukchinR18}\rowlabel{a:BukchinR18}BukchinR18 \href{http://dx.doi.org/10.1016/j.omega.2017.06.008}{BukchinR18} & \hyperref[auth:a1181]{Y. Bukchin}, \hyperref[auth:a1182]{T. Raviv} & Constraint programming for solving various assembly line balancing problems & \hyperref[detail:BukchinR18]{Details} \href{../works/BukchinR18.pdf}{Yes} & \cite{BukchinR18} & 2018 & Omega & 12 & \noindent{}\textcolor{black!50}{0.00} \textcolor{black!50}{0.00} \textbf{21.89} & 66 68 81 & 29 43 & 23 22 1\\
\index{CauwelaertLS18}\rowlabel{a:CauwelaertLS18}CauwelaertLS18 \href{https://doi.org/10.1007/s10601-017-9277-y}{CauwelaertLS18} & \hyperref[auth:a201]{S. V. Cauwelaert}, \hyperref[auth:a142]{M. Lombardi}, \hyperref[auth:a147]{P. Schaus} & How efficient is a global constraint in practice? - {A} fair experimental framework & \hyperref[detail:CauwelaertLS18]{Details} \href{../works/CauwelaertLS18.pdf}{Yes} & \cite{CauwelaertLS18} & 2018 & Constraints An Int. J. & 36 & \noindent{}\textcolor{black!50}{0.00} \textcolor{black!50}{0.00} \textbf{3.12} & 2 1 1 & 39 61 & 14 1 13\\
\index{Dasygenis2018}\rowlabel{a:Dasygenis2018}Dasygenis2018 \href{http://dx.doi.org/10.1142/s0218213018600023}{Dasygenis2018} & \hyperref[auth:a2000]{M. Dasygenis}, \hyperref[auth:a2001]{K. Stergiou} & Methods for Parallelizing Constraint Propagation through the Use of Strong Local Consistencies \hyperref[abs:Dasygenis2018]{Abstract} & \hyperref[detail:Dasygenis2018]{Details} No & \cite{Dasygenis2018} & 2018 & International Journal on Artificial Intelligence Tools & null & \noindent{}\textcolor{black!50}{0.00} \textbf{1.75} n/a & 1 1 2 & 12 30 & 2 0 2\\
\index{FahimiOQ18}\rowlabel{a:FahimiOQ18}FahimiOQ18 \href{https://doi.org/10.1007/s10601-018-9282-9}{FahimiOQ18} & \hyperref[auth:a122]{H. Fahimi}, \hyperref[auth:a52]{Y. Ouellet}, \hyperref[auth:a37]{C.-G. Quimper} & Linear-time filtering algorithms for the disjunctive constraint and a quadratic filtering algorithm for the cumulative not-first not-last & \hyperref[detail:FahimiOQ18]{Details} \href{../works/FahimiOQ18.pdf}{Yes} & \cite{FahimiOQ18} & 2018 & Constraints An Int. J. & 22 & \noindent{}\textcolor{black!50}{0.00} \textcolor{black!50}{0.00} \textbf{6.07} & 2 2 7 & 20 36 & 19 2 17\\
\index{Gao2018}\rowlabel{a:Gao2018}Gao2018 \href{http://dx.doi.org/10.2355/isijinternational.isijint-2018-305}{Gao2018} & \hyperref[auth:a1712]{C. Gao}, \hyperref[auth:a1713]{D. Qu} & \cellcolor{gold!20}A Modelling and a New Hybrid MILP/CP Decomposition Method for Parallel Continuous Galvanizing Line Scheduling Problem & \hyperref[detail:Gao2018]{Details} No & \cite{Gao2018} & 2018 & ISIJ International & null & \noindent{}\textbf{1.00} \textbf{1.00} n/a & 1 2 2 & 7 16 & 4 0 4\\
\index{GarcaNieves2018}\rowlabel{a:GarcaNieves2018}GarcaNieves2018 \href{http://dx.doi.org/10.1111/mice.12356}{GarcaNieves2018} & \hyperref[auth:a1724]{J. D. García‐Nieves}, \hyperref[auth:a1725]{J. L. Ponz‐Tienda}, \hyperref[auth:a1726]{A. Salcedo‐Bernal}, \hyperref[auth:a1727]{E. Pellicer} & \cellcolor{green!10}The Multimode Resource‐Constrained Project Scheduling Problem for Repetitive Activities in Construction Projects \hyperref[abs:GarcaNieves2018]{Abstract} & \hyperref[detail:GarcaNieves2018]{Details} No & \cite{GarcaNieves2018} & 2018 & Computer-Aided Civil and Infrastructure Engineering & null & \noindent{}\textcolor{black!50}{0.00} \textcolor{black!50}{0.00} n/a & 33 43 46 & 36 44 & 8 4 4\\
\index{GedikKEK18}\rowlabel{a:GedikKEK18}GedikKEK18 \href{https://doi.org/10.1016/j.cie.2018.05.014}{GedikKEK18} & \hyperref[auth:a560]{R. Gedik}, \hyperref[auth:a561]{D. Kalathia}, \hyperref[auth:a562]{G. Egilmez}, \hyperref[auth:a563]{E. Kirac} & A constraint programming approach for solving unrelated parallel machine scheduling problem & \hyperref[detail:GedikKEK18]{Details} \href{../works/GedikKEK18.pdf}{Yes} & \cite{GedikKEK18} & 2018 & Computers \  Industrial Engineering & 11 & \noindent{}\textbf{1.50} \textbf{1.50} \textbf{24.08} & 43 49 47 & 22 31 & 23 20 3\\
\index{GokgurHO18}\rowlabel{a:GokgurHO18}GokgurHO18 \href{https://doi.org/10.1080/00207543.2017.1421781}{GokgurHO18} & \hyperref[auth:a569]{B. G{\"{o}}kg{\"{u}}r}, \hyperref[auth:a137]{B. Hnich}, \hyperref[auth:a570]{S. {\"{O}}zpeynirci} & Parallel machine scheduling with tool loading: a constraint programming approach & \hyperref[detail:GokgurHO18]{Details} \href{../works/GokgurHO18.pdf}{Yes} & \cite{GokgurHO18} & 2018 & \cellcolor{red!20}International Journal of Production Research & 17 & \noindent{}\textbf{1.50} \textbf{1.50} \textbf{73.63} & 31 40 51 & 43 62 & 23 8 15\\
\index{GoldwaserS18}\rowlabel{a:GoldwaserS18}GoldwaserS18 \href{https://doi.org/10.1613/jair.1.11268}{GoldwaserS18} & \hyperref[auth:a189]{A. Goldwaser}, \hyperref[auth:a124]{A. Schutt} & \cellcolor{gold!20}Optimal Torpedo Scheduling & \hyperref[detail:GoldwaserS18]{Details} \href{../works/GoldwaserS18.pdf}{Yes} & \cite{GoldwaserS18} & 2018 & J. Artif. Intell. Res. & 32 & \noindent{}\textcolor{black!50}{0.00} \textcolor{black!50}{0.00} \textbf{3.31} & 8 8 9 & 0 0 & 1 1 0\\
\index{GombolayWS18}\rowlabel{a:GombolayWS18}GombolayWS18 \href{http://dx.doi.org/10.1109/tro.2018.2795034}{GombolayWS18} & \hyperref[auth:a921]{M. C. Gombolay}, \hyperref[auth:a922]{R. J. Wilcox}, \hyperref[auth:a923]{J. A. Shah} & \cellcolor{gold!20}Fast Scheduling of Robot Teams Performing Tasks With Temporospatial Constraints & \hyperref[detail:GombolayWS18]{Details} \href{../works/GombolayWS18.pdf}{Yes} & \cite{GombolayWS18} & 2018 & IEEE Transactions on Robotics & 20 & \noindent{}\textcolor{black!50}{0.00} \textcolor{black!50}{0.00} \textbf{11.87} & 71 80 79 & 75 89 & 12 1 11\\
\index{Ham18}\rowlabel{a:Ham18}Ham18 \href{http://dx.doi.org/10.1016/j.trc.2018.03.025}{Ham18} & \hyperref[auth:a770]{A. M. Ham} & Integrated scheduling of m-truck, m-drone, and m-depot constrained by time-window, drop-pickup, and m-visit using constraint programming & \hyperref[detail:Ham18]{Details} \href{../works/Ham18.pdf}{Yes} & \cite{Ham18} & 2018 & Transportation Research Part C: Emerging Technologies & 14 & \noindent{}\textbf{1.00} \textbf{1.00} \textbf{12.98} & 164 192 197 & 14 30 & 11 7 4\\
\index{Ham18a}\rowlabel{a:Ham18a}Ham18a \href{http://dx.doi.org/10.1109/tsm.2017.2768899}{Ham18a} & \hyperref[auth:a750]{A. Ham} & Scheduling of Dual Resource Constrained Lithography Production: Using CP and MIP/CP & \hyperref[detail:Ham18a]{Details} \href{../works/Ham18a.pdf}{Yes} & \cite{Ham18a} & 2018 & IEEE Transactions on Semiconductor Manufacturing & 10 & \noindent{}\textbf{1.50} \textbf{1.50} \textbf{17.72} & 20 24 28 & 21 28 & 16 5 11\\
\index{KreterSSZ18}\rowlabel{a:KreterSSZ18}KreterSSZ18 \href{https://doi.org/10.1016/j.ejor.2017.10.014}{KreterSSZ18} & \hyperref[auth:a123]{S. Kreter}, \hyperref[auth:a124]{A. Schutt}, \hyperref[auth:a125]{P. J. Stuckey}, \hyperref[auth:a792]{J. Zimmermann} & Mixed-integer linear programming and constraint programming formulations for solving resource availability cost problems & \hyperref[detail:KreterSSZ18]{Details} \href{../works/KreterSSZ18.pdf}{Yes} & \cite{KreterSSZ18} & 2018 & European Journal of Operational Research & 15 & \noindent{}0.50 0.50 \textbf{18.78} & 25 29 26 & 31 54 & 22 14 8\\
\index{LaborieRSV18}\rowlabel{a:LaborieRSV18}LaborieRSV18 \href{https://doi.org/10.1007/s10601-018-9281-x}{LaborieRSV18} & \hyperref[auth:a118]{P. Laborie}, \hyperref[auth:a119]{J. Rogerie}, \hyperref[auth:a120]{P. Shaw}, \hyperref[auth:a121]{P. Vil{\'{\i}}m} & {IBM} {ILOG} {CP} optimizer for scheduling - 20+ years of scheduling with constraints at {IBM/ILOG} & \hyperref[detail:LaborieRSV18]{Details} \href{../works/LaborieRSV18.pdf}{Yes} & \cite{LaborieRSV18} & 2018 & Constraints An Int. J. & 41 & \noindent{}\textbf{1.00} \textbf{1.00} \textbf{54.37} & 148 178 203 & 35 54 & 92 69 23\\
\index{Li2018}\rowlabel{a:Li2018}Li2018 \href{http://dx.doi.org/10.1109/access.2018.2859618}{Li2018} & \hyperref[auth:a1796]{H. Li}, \hyperref[auth:a1801]{Z. Li} & \cellcolor{gold!20}A Novel Strategy of Combining Variable Ordering Heuristics for Constraint Satisfaction Problems & \hyperref[detail:Li2018]{Details} No & \cite{Li2018} & 2018 & IEEE Access & null & \noindent{}0.50 0.50 n/a & 4 4 4 & 17 35 & 5 0 5\\
\index{Ortiz-Bayliss2018}\rowlabel{a:Ortiz-Bayliss2018}Ortiz-Bayliss2018 \href{http://dx.doi.org/10.1155/2018/6103726}{Ortiz-Bayliss2018} & \hyperref[auth:a1781]{J. C. Ortiz-Bayliss}, \hyperref[auth:a1604]{I. Amaya}, \hyperref[auth:a1782]{S. E. Conant-Pablos}, \hyperref[auth:a1608]{H. Terashima-Marín} & \cellcolor{gold!20}Exploring the Impact of Early Decisions in Variable Ordering for Constraint Satisfaction Problems \hyperref[abs:Ortiz-Bayliss2018]{Abstract} & \hyperref[detail:Ortiz-Bayliss2018]{Details} No & \cite{Ortiz-Bayliss2018} & 2018 & Computational Intelligence and Neuroscience & null & \noindent{}0.50 0.50 n/a & 1 1 2 & 26 29 & 5 1 4\\
\index{PourDERB18}\rowlabel{a:PourDERB18}PourDERB18 \href{https://doi.org/10.1016/j.ejor.2017.08.033}{PourDERB18} & \hyperref[auth:a564]{S. M. Pour}, \hyperref[auth:a565]{J. H. Drake}, \hyperref[auth:a566]{L. S. Ejlertsen}, \hyperref[auth:a567]{K. M. Rasmussen}, \hyperref[auth:a568]{E. K. Burke} & \cellcolor{gold!20}A hybrid Constraint Programming/Mixed Integer Programming framework for the preventive signaling maintenance crew scheduling problem & \hyperref[detail:PourDERB18]{Details} \href{../works/PourDERB18.pdf}{Yes} & \cite{PourDERB18} & 2018 & European Journal of Operational Research & 12 & \noindent{}\textbf{1.00} \textbf{1.00} \textbf{30.35} & 41 47 48 & 13 25 & 16 14 2\\
\index{ShinBBHO18}\rowlabel{a:ShinBBHO18}ShinBBHO18 \href{https://doi.org/10.1109/TSMC.2017.2681623}{ShinBBHO18} & \hyperref[auth:a573]{S. Y. Shin}, \hyperref[auth:a574]{Y. Brun}, \hyperref[auth:a575]{H. Balasubramanian}, \hyperref[auth:a576]{P. L. Henneman}, \hyperref[auth:a577]{L. J. Osterweil} & \cellcolor{gold!20}Discrete-Event Simulation and Integer Linear Programming for Constraint-Aware Resource Scheduling & \hyperref[detail:ShinBBHO18]{Details} \href{../works/ShinBBHO18.pdf}{Yes} & \cite{ShinBBHO18} & 2018 & {IEEE} Trans. Syst. Man Cybern. Syst. & 16 & \noindent{}\textcolor{black!50}{0.00} \textcolor{black!50}{0.00} \textcolor{black!50}{0.00} & 9 9 12 & 31 39 & 0 0 0\\
\index{Tang2018}\rowlabel{a:Tang2018}Tang2018 \href{http://dx.doi.org/10.1111/mice.12383}{Tang2018} & \hyperref[auth:a555]{Y. Tang}, \hyperref[auth:a558]{Q. Sun}, \hyperref[auth:a556]{R. Liu}, \hyperref[auth:a557]{F. Wang} & Resource Leveling Based on Line of Balance and Constraint Programming \hyperref[abs:Tang2018]{Abstract} & \hyperref[detail:Tang2018]{Details} No & \cite{Tang2018} & 2018 & Computer-Aided Civil and Infrastructure Engineering & null & \noindent{}0.50 \textbf{1.50} n/a & 18 21 23 & 80 88 & 11 2 9\\
\index{TangLWSK18}\rowlabel{a:TangLWSK18}TangLWSK18 \href{https://doi.org/10.1111/mice.12277}{TangLWSK18} & \hyperref[auth:a555]{Y. Tang}, \hyperref[auth:a556]{R. Liu}, \hyperref[auth:a557]{F. Wang}, \hyperref[auth:a558]{Q. Sun}, \hyperref[auth:a559]{A. A. Kandil} & Scheduling Optimization of Linear Schedule with Constraint Programming & \hyperref[detail:TangLWSK18]{Details} \href{../works/TangLWSK18.pdf}{Yes} & \cite{TangLWSK18} & 2018 & Comput. Aided Civ. Infrastructure Eng. & 28 & \noindent{}\textbf{1.00} \textbf{1.00} \textbf{24.56} & 24 32 30 & 76 88 & 12 5 7\\
\index{TranPZLDB18}\rowlabel{a:TranPZLDB18}TranPZLDB18 \href{https://doi.org/10.1007/s10951-017-0537-x}{TranPZLDB18} & \hyperref[auth:a799]{T. T. Tran}, \hyperref[auth:a800]{M. Padmanabhan}, \hyperref[auth:a801]{P. Y. Zhang}, \hyperref[auth:a802]{H. Li}, \hyperref[auth:a803]{D. G. Down}, \hyperref[auth:a89]{J. C. Beck} & \cellcolor{green!10}Multi-stage resource-aware scheduling for data centers with heterogeneous servers & \hyperref[detail:TranPZLDB18]{Details} \href{../works/TranPZLDB18.pdf}{Yes} & \cite{TranPZLDB18} & 2018 & Journal of Scheduling & 17 & \noindent{}\textcolor{black!50}{0.00} \textcolor{black!50}{0.00} \textcolor{black!50}{0.00} & 8 9 9 & 26 29 & 1 0 1\\
\index{Trker2018}\rowlabel{a:Trker2018}Trker2018 \href{http://dx.doi.org/10.1155/2018/7870849}{Trker2018} & \hyperref[auth:a1714]{T. Türker}, \hyperref[auth:a1715]{A. Demiriz} & \cellcolor{gold!20}An Integrated Approach for Shift Scheduling and Rostering Problems with Break Times for Inbound Call Centers \hyperref[abs:Trker2018]{Abstract} & \hyperref[detail:Trker2018]{Details} No & \cite{Trker2018} & 2018 & Mathematical Problems in Engineering & null & \noindent{}\textcolor{black!50}{0.00} \textbf{5.00} n/a & 7 8 6 & 63 72 & 4 0 4\\
\index{Yvars2018}\rowlabel{a:Yvars2018}Yvars2018 \href{http://dx.doi.org/10.1155/2018/6861429}{Yvars2018} & \hyperref[auth:a1979]{P.-A. Yvars}, \hyperref[auth:a1980]{L. Zimmer} & \cellcolor{gold!20}System Sizing with a Model-Based Approach: Application to the Optimization of a Power Transmission System \hyperref[abs:Yvars2018]{Abstract} & \hyperref[detail:Yvars2018]{Details} No & \cite{Yvars2018} & 2018 & Mathematical Problems in Engineering & null & \noindent{}\textcolor{black!50}{0.00} \textbf{4.50} n/a & 2 3 4 & 16 25 & 2 0 2\\
\index{ZhangW18}\rowlabel{a:ZhangW18}ZhangW18 \href{https://doi.org/10.1109/TEM.2017.2785774}{ZhangW18} & \hyperref[auth:a571]{S. Zhang}, \hyperref[auth:a572]{S. Wang} & Flexible Assembly Job-Shop Scheduling With Sequence-Dependent Setup Times and Part Sharing in a Dynamic Environment: Constraint Programming Model, Mixed-Integer Programming Model, and Dispatching Rules & \hyperref[detail:ZhangW18]{Details} \href{../works/ZhangW18.pdf}{Yes} & \cite{ZhangW18} & 2018 & {IEEE} Trans. Engineering Management & 18 & \noindent{}\textbf{2.00} \textbf{2.00} \textbf{9.42} & 49 56 60 & 28 32 & 4 2 2\\
\index{Balduccini2017}\rowlabel{a:Balduccini2017}Balduccini2017 \href{http://dx.doi.org/10.1017/s1471068417000102}{Balduccini2017} & \hyperref[auth:a1042]{M. Balduccini}, \hyperref[auth:a2051]{Y. Lierler} & \cellcolor{green!10}Constraint answer set solver EZCSP and why integration schemas matter \hyperref[abs:Balduccini2017]{Abstract} & \hyperref[detail:Balduccini2017]{Details} No & \cite{Balduccini2017} & 2017 & Theory and Practice of Logic Programming & null & \noindent{}\textcolor{black!50}{0.00} \textbf{1.00} n/a & 20 21 32 & 34 52 & 1 0 1\\
\index{CarlssonJL17}\rowlabel{a:CarlssonJL17}CarlssonJL17 \href{https://doi.org/10.1016/j.ejor.2016.11.033}{CarlssonJL17} & \hyperref[auth:a91]{M. Carlsson}, \hyperref[auth:a75]{M. Johansson}, \hyperref[auth:a1412]{J. Larson} & \cellcolor{gold!20}Scheduling double round-robin tournaments with divisional play using constraint programming & \hyperref[detail:CarlssonJL17]{Details} \href{../works/CarlssonJL17.pdf}{Yes} & \cite{CarlssonJL17} & 2017 & European Journal of Operational Research & 11 & \noindent{}\textbf{1.00} \textbf{1.00} \textbf{3.77} & 10 11 12 & 14 42 & 6 1 5\\
\index{EmeretlisTAV17}\rowlabel{a:EmeretlisTAV17}EmeretlisTAV17 \href{http://dx.doi.org/10.1145/3133219}{EmeretlisTAV17} & \hyperref[auth:a1227]{A. Emeretlis}, \hyperref[auth:a1228]{G. Theodoridis}, \hyperref[auth:a1229]{P. Alefragis}, \hyperref[auth:a1230]{N. Voros} & Static Mapping of Applications on Heterogeneous Multi-Core Platforms Combining Logic-Based Benders Decomposition with Integer Linear Programming & \hyperref[detail:EmeretlisTAV17]{Details} \href{../works/EmeretlisTAV17.pdf}{Yes} & \cite{EmeretlisTAV17} & 2017 & ACM Transactions on Design Automation of Electronic Systems & 24 & \noindent{}\textcolor{black!50}{0.00} \textcolor{black!50}{0.00} \textbf{5.91} & 4 6 9 & 42 48 & 11 1 10\\
\index{GedikKBR17}\rowlabel{a:GedikKBR17}GedikKBR17 \href{http://dx.doi.org/10.1016/j.cie.2017.03.017}{GedikKBR17} & \hyperref[auth:a560]{R. Gedik}, \hyperref[auth:a563]{E. Kirac}, \hyperref[auth:a1155]{A. B. Milburn}, \hyperref[auth:a1156]{C. Rainwater} & \cellcolor{green!10}A constraint programming approach for the team orienteering problem with time windows & \hyperref[detail:GedikKBR17]{Details} \href{../works/GedikKBR17.pdf}{Yes} & \cite{GedikKBR17} & 2017 & Computers \  Industrial Engineering & 18 & \noindent{}\textcolor{black!50}{0.00} \textcolor{black!50}{0.00} \textbf{3.60} & 20 23 26 & 32 47 & 14 7 7\\
\index{GomesM17}\rowlabel{a:GomesM17}GomesM17 \href{http://dx.doi.org/10.1155/2017/9452762}{GomesM17} & \hyperref[auth:a965]{F. R. A. Gomes}, \hyperref[auth:a966]{G. R. Mateus} & \cellcolor{gold!20}Improved Combinatorial Benders Decomposition for a Scheduling Problem with Unrelated Parallel Machines & \hyperref[detail:GomesM17]{Details} \href{../works/GomesM17.pdf}{Yes} & \cite{GomesM17} & 2017 & Journal of Applied Mathematics & 10 & \noindent{}\textcolor{black!50}{0.00} \textcolor{black!50}{0.00} 0.31 & 1 1 3 & 43 44 & 9 0 9\\
\index{Gonzlez2017}\rowlabel{a:Gonzlez2017}Gonzlez2017 \href{http://dx.doi.org/10.1609/icaps.v27i1.13809}{Gonzlez2017} & \hyperref[auth:a1828]{M. Ángel González}, \hyperref[auth:a282]{A. Oddi}, \hyperref[auth:a1270]{R. Rasconi} & Multi-Objective Optimization in a Job Shop with Energy Costs through Hybrid Evolutionary Techniques \hyperref[abs:Gonzlez2017]{Abstract} & \hyperref[detail:Gonzlez2017]{Details} No & \cite{Gonzlez2017} & 2017 & Proceedings of the International Conference on Automated Planning and Scheduling & null & \noindent{}\textcolor{black!50}{0.00} \textbf{2.00} n/a & 10 12 0 & 0 0 & 1 1 0\\
\index{HamFC17}\rowlabel{a:HamFC17}HamFC17 \href{http://dx.doi.org/10.1109/tsm.2017.2740340}{HamFC17} & \hyperref[auth:a750]{A. Ham}, \hyperref[auth:a1201]{J. W. Fowler}, \hyperref[auth:a875]{E. Cakici} & Constraint Programming Approach for Scheduling Jobs With Release Times, Non-Identical Sizes, and Incompatible Families on Parallel Batching Machines & \hyperref[detail:HamFC17]{Details} \href{../works/HamFC17.pdf}{Yes} & \cite{HamFC17} & 2017 & IEEE Transactions on Semiconductor Manufacturing & 8 & \noindent{}\textbf{2.50} \textbf{2.50} \textbf{15.73} & 21 24 25 & 28 33 & 8 3 5\\
\index{HookerH17}\rowlabel{a:HookerH17}HookerH17 \href{http://dx.doi.org/10.1007/s10601-017-9280-3}{HookerH17} & \hyperref[auth:a160]{J. N. Hooker}, \hyperref[auth:a206]{W.-J. van Hoeve} & Constraint programming and operations research & \hyperref[detail:HookerH17]{Details} \href{../works/HookerH17.pdf}{Yes} & \cite{HookerH17} & 2017 & Constraints An Int. J. & 24 & \noindent{}\textcolor{black!50}{0.00} \textcolor{black!50}{0.00} \textbf{36.80} & 12 13 10 & 189 255 & 61 2 59\\
\index{KreterSS17}\rowlabel{a:KreterSS17}KreterSS17 \href{https://doi.org/10.1007/s10601-016-9266-6}{KreterSS17} & \hyperref[auth:a123]{S. Kreter}, \hyperref[auth:a124]{A. Schutt}, \hyperref[auth:a125]{P. J. Stuckey} & Using constraint programming for solving RCPSP/max-cal & \hyperref[detail:KreterSS17]{Details} \href{../works/KreterSS17.pdf}{Yes} & \cite{KreterSS17} & 2017 & Constraints An Int. J. & 31 & \noindent{}\textcolor{black!50}{0.00} \textcolor{black!50}{0.00} \textbf{8.01} & 15 18 21 & 20 27 & 23 12 11\\
\index{Laborie2017}\rowlabel{a:Laborie2017}Laborie2017 \href{http://dx.doi.org/10.1609/icaps.v27i1.13844}{Laborie2017} & \hyperref[auth:a118]{P. Laborie}, \hyperref[auth:a1550]{B. Messaoudi} & New Results for the GEO-CAPE Observation Scheduling Problem \hyperref[abs:Laborie2017]{Abstract} & \hyperref[detail:Laborie2017]{Details} No & \cite{Laborie2017} & 2017 & Proceedings of the International Conference on Automated Planning and Scheduling & null & \noindent{}\textcolor{black!50}{0.00} \textbf{2.00} n/a & 2 2 0 & 0 0 & 2 2 0\\
\index{Morillo2017}\rowlabel{a:Morillo2017}Morillo2017 \href{http://dx.doi.org/10.1155/2017/4627856}{Morillo2017} & \hyperref[auth:a1735]{D. Morillo}, \hyperref[auth:a271]{F. Barber}, \hyperref[auth:a153]{M. A. Salido} & \cellcolor{gold!20}Mode-Based versus Activity-Based Search for a Nonredundant Resolution of the Multimode Resource-Constrained Project Scheduling Problem \hyperref[abs:Morillo2017]{Abstract} & \hyperref[detail:Morillo2017]{Details} No & \cite{Morillo2017} & 2017 & Mathematical Problems in Engineering & null & \noindent{}\textcolor{black!50}{0.00} \textcolor{black!50}{0.00} n/a & 2 3 3 & 33 37 & 4 0 4\\
\index{Mutha2017}\rowlabel{a:Mutha2017}Mutha2017 \href{http://dx.doi.org/10.1177/1748006x17744380}{Mutha2017} & \hyperref[auth:a1957]{C. Mutha}, \hyperref[auth:a1958]{C. Smidts} & Basis for non-propagation domains, their transformations and their impact on software reliability \hyperref[abs:Mutha2017]{Abstract} & \hyperref[detail:Mutha2017]{Details} No & \cite{Mutha2017} & 2017 & Proceedings of the Institution of Mechanical Engineers, Part O: Journal of Risk and Reliability & null & \noindent{}\textcolor{black!50}{0.00} 0.25 n/a & 0 0 0 & 19 33 & 1 0 1\\
\index{NattafAL17}\rowlabel{a:NattafAL17}NattafAL17 \href{https://doi.org/10.1007/s10601-017-9271-4}{NattafAL17} & \hyperref[auth:a81]{M. Nattaf}, \hyperref[auth:a6]{C. Artigues}, \hyperref[auth:a3]{P. Lopez} & \cellcolor{green!10}Cumulative scheduling with variable task profiles and concave piecewise linear processing rate functions & \hyperref[detail:NattafAL17]{Details} \href{../works/NattafAL17.pdf}{Yes} & \cite{NattafAL17} & 2017 & Constraints An Int. J. & 18 & \noindent{}\textcolor{black!50}{0.00} \textcolor{black!50}{0.00} \textbf{1.71} & 5 5 7 & 10 16 & 8 3 5\\
\index{RoshanaeiLAU17}\rowlabel{a:RoshanaeiLAU17}RoshanaeiLAU17 \href{http://dx.doi.org/10.1016/j.ejor.2016.08.024}{RoshanaeiLAU17} & \hyperref[auth:a728]{V. Roshanaei}, \hyperref[auth:a927]{C. Luong}, \hyperref[auth:a895]{D. M. Aleman}, \hyperref[auth:a896]{D. R. Urbach} & Propagating logic-based Benders' decomposition approaches for distributed operating room scheduling & \hyperref[detail:RoshanaeiLAU17]{Details} \href{../works/RoshanaeiLAU17.pdf}{Yes} & \cite{RoshanaeiLAU17} & 2017 & European Journal of Operational Research & 17 & \noindent{}\textcolor{black!50}{0.00} \textcolor{black!50}{0.00} \textbf{2.97} & 61 66 65 & 46 53 & 24 14 10\\
\index{RoshanaeiLAU17a}\rowlabel{a:RoshanaeiLAU17a}RoshanaeiLAU17a \href{http://dx.doi.org/10.1287/ijoc.2017.0745}{RoshanaeiLAU17a} & \hyperref[auth:a728]{V. Roshanaei}, \hyperref[auth:a927]{C. Luong}, \hyperref[auth:a895]{D. M. Aleman}, \hyperref[auth:a896]{D. R. Urbach} & Collaborative Operating Room Planning and Scheduling & \hyperref[detail:RoshanaeiLAU17a]{Details} No & \cite{RoshanaeiLAU17a} & 2017 & \cellcolor{red!20}INFORMS Journal on Computing & 23 & \noindent{}\textcolor{black!50}{0.00} \textcolor{black!50}{0.00} n/a & 54 55 55 & 42 47 & 18 10 8\\
\index{SchnellH17}\rowlabel{a:SchnellH17}SchnellH17 \href{http://dx.doi.org/10.1016/j.orp.2017.01.002}{SchnellH17} & \hyperref[auth:a950]{A. Schnell}, \hyperref[auth:a951]{R. F. Hartl} & \cellcolor{gold!20}On the generalization of constraint programming and boolean satisfiability solving techniques to schedule a resource-constrained project consisting of multi-mode jobs & \hyperref[detail:SchnellH17]{Details} \href{../works/SchnellH17.pdf}{Yes} & \cite{SchnellH17} & 2017 & Operations Research Perspectives & 11 & \noindent{}\textbf{1.50} \textbf{1.50} \textbf{9.59} & 12 18 21 & 20 37 & 13 4 9\\
\index{Sitek2017}\rowlabel{a:Sitek2017}Sitek2017 \href{http://dx.doi.org/10.1108/imds-10-2016-0465}{Sitek2017} & \hyperref[auth:a536]{P. Sitek}, \hyperref[auth:a535]{J. Wikarek}, \hyperref[auth:a1527]{P. Nielsen} & \cellcolor{gold!20}A constraint-driven approach to food supply chain management \hyperref[abs:Sitek2017]{Abstract} & \hyperref[detail:Sitek2017]{Details} No & \cite{Sitek2017} & 2017 & Industrial Management \  Data Systems & null & \noindent{}\textcolor{black!50}{0.00} \textbf{1.50} n/a & 24 0 26 & 26 0 & 4 1 3\\
\index{TranVNB17}\rowlabel{a:TranVNB17}TranVNB17 \href{https://doi.org/10.1613/jair.5306}{TranVNB17} & \hyperref[auth:a799]{T. T. Tran}, \hyperref[auth:a804]{T. S. Vaquero}, \hyperref[auth:a204]{G. Nejat}, \hyperref[auth:a89]{J. C. Beck} & \cellcolor{gold!20}Robots in Retirement Homes: Applying Off-the-Shelf Planning and Scheduling to a Team of Assistive Robots & \hyperref[detail:TranVNB17]{Details} \href{../works/TranVNB17.pdf}{Yes} & \cite{TranVNB17} & 2017 & J. Artif. Intell. Res. & 68 & \noindent{}\textcolor{black!50}{0.00} \textcolor{black!50}{0.00} \textbf{61.11} & 12 12 21 & 0 0 & 2 2 0\\
\index{Abdul-Niby2016}\rowlabel{a:Abdul-Niby2016}Abdul-Niby2016 \href{http://dx.doi.org/10.48084/etasr.627}{Abdul-Niby2016} & \hyperref[auth:a1855]{M. Abdul-Niby}, \hyperref[auth:a1856]{M. Alameen}, \hyperref[auth:a1857]{A. Salhieh}, \hyperref[auth:a1858]{A. Radhi} & Improved Genetic and Simulating Annealing Algorithms to Solve the Traveling Salesman Problem Using Constraint Programming \hyperref[abs:Abdul-Niby2016]{Abstract} & \hyperref[detail:Abdul-Niby2016]{Details} No & \cite{Abdul-Niby2016} & 2016 & Engineering, Technology \  Applied Science Research & null & \noindent{}\textcolor{black!50}{0.00} 0.50 n/a & 3 4 0 & 7 8 & 1 1 0\\
\index{BlomPS16}\rowlabel{a:BlomPS16}BlomPS16 \href{https://doi.org/10.1287/mnsc.2015.2284}{BlomPS16} & \hyperref[auth:a795]{M. L. Blom}, \hyperref[auth:a324]{A. R. Pearce}, \hyperref[auth:a125]{P. J. Stuckey} & A Decomposition-Based Algorithm for the Scheduling of Open-Pit Networks Over Multiple Time Periods & \hyperref[detail:BlomPS16]{Details} \href{../works/BlomPS16.pdf}{Yes} & \cite{BlomPS16} & 2016 & Manag. Sci. & 26 & \noindent{}\textcolor{black!50}{0.00} \textcolor{black!50}{0.00} \textcolor{black!50}{0.00} & 20 23 25 & 36 46 & 2 0 2\\
\index{Boek2016}\rowlabel{a:Boek2016}Boek2016 \href{http://dx.doi.org/10.1515/mper-2016-0003}{Boek2016} & \hyperref[auth:a1885]{A. Bożek}, \hyperref[auth:a1886]{M. Wysocki} & \cellcolor{gold!20}Off-Line and Dynamic Production Scheduling – A Comparative Case Study \hyperref[abs:Boek2016]{Abstract} & \hyperref[detail:Boek2016]{Details} No & \cite{Boek2016} & 2016 & Management and Production Engineering Review & null & \noindent{}\textcolor{black!50}{0.00} \textbf{5.00} n/a & 5 5 8 & 10 23 & 1 1 0\\
\index{Bonfietti16}\rowlabel{a:Bonfietti16}Bonfietti16 \href{https://doi.org/10.3233/IA-160095}{Bonfietti16} & \hyperref[auth:a198]{A. Bonfietti} & A constraint programming scheduling solver for the MPOpt programming environment & \hyperref[detail:Bonfietti16]{Details} \href{../works/Bonfietti16.pdf}{Yes} & \cite{Bonfietti16} & 2016 & Intelligenza Artificiale & 13 & \noindent{}\textbf{1.00} \textbf{1.00} \textcolor{black!50}{0.13} & 0 0 0 & 19 37 & 1 0 1\\
\index{BoothTNB16}\rowlabel{a:BoothTNB16}BoothTNB16 \href{http://dx.doi.org/10.1109/lra.2016.2522096}{BoothTNB16} & \hyperref[auth:a203]{K. E. C. Booth}, \hyperref[auth:a799]{T. T. Tran}, \hyperref[auth:a204]{G. Nejat}, \hyperref[auth:a89]{J. C. Beck} & \cellcolor{green!10}Mixed-Integer and Constraint Programming Techniques for Mobile Robot Task Planning & \hyperref[detail:BoothTNB16]{Details} \href{../works/BoothTNB16.pdf}{Yes} & \cite{BoothTNB16} & 2016 & IEEE Robotics and Automation Letters & 8 & \noindent{}\textbf{1.00} \textbf{1.00} \textbf{11.60} & 27 28 35 & 21 34 & 14 5 9\\
\index{BridiBLMB16}\rowlabel{a:BridiBLMB16}BridiBLMB16 \href{https://doi.org/10.1109/TPDS.2016.2516997}{BridiBLMB16} & \hyperref[auth:a227]{T. Bridi}, \hyperref[auth:a225]{A. Bartolini}, \hyperref[auth:a142]{M. Lombardi}, \hyperref[auth:a143]{M. Milano}, \hyperref[auth:a245]{L. Benini} & \cellcolor{green!10}A Constraint Programming Scheduler for Heterogeneous High-Performance Computing Machines & \hyperref[detail:BridiBLMB16]{Details} \href{../works/BridiBLMB16.pdf}{Yes} & \cite{BridiBLMB16} & 2016 & {IEEE} Trans. Parallel Distributed Syst. & 14 & \noindent{}0.50 0.50 \textbf{20.89} & 17 18 21 & 22 34 & 4 2 2\\
\index{CireCH16}\rowlabel{a:CireCH16}CireCH16 \href{http://dx.doi.org/10.1017/s0269888916000254}{CireCH16} & \hyperref[auth:a157]{A. A. Cir{\'{e}}}, \hyperref[auth:a335]{E. Coban}, \hyperref[auth:a160]{J. N. Hooker} & \cellcolor{green!10}Logic-based Benders decomposition for planning and scheduling: a computational analysis & \hyperref[detail:CireCH16]{Details} \href{../works/CireCH16.pdf}{Yes} & \cite{CireCH16} & 2016 & The Knowledge Engineering Review & 12 & \noindent{}\textcolor{black!50}{0.00} \textcolor{black!50}{0.00} \textbf{3.51} & 15 17 13 & 21 30 & 25 8 17\\
\index{DoulabiRP16}\rowlabel{a:DoulabiRP16}DoulabiRP16 \href{https://doi.org/10.1287/ijoc.2015.0686}{DoulabiRP16} & \hyperref[auth:a330]{S. H. H. Doulabi}, \hyperref[auth:a326]{L.-M. Rousseau}, \hyperref[auth:a8]{G. Pesant} & A Constraint-Programming-Based Branch-and-Price-and-Cut Approach for Operating Room Planning and Scheduling & \hyperref[detail:DoulabiRP16]{Details} \href{../works/DoulabiRP16.pdf}{Yes} & \cite{DoulabiRP16} & 2016 & \cellcolor{red!20}INFORMS Journal on Computing & 17 & \noindent{}\textcolor{black!50}{0.00} \textcolor{black!50}{0.00} \textbf{3.92} & 56 63 64 & 28 32 & 14 12 2\\
\index{HamC16}\rowlabel{a:HamC16}HamC16 \href{http://dx.doi.org/10.1016/j.cie.2016.11.001}{HamC16} & \hyperref[auth:a770]{A. M. Ham}, \hyperref[auth:a875]{E. Cakici} & Flexible job shop scheduling problem with parallel batch processing machines: MIP and CP approaches & \hyperref[detail:HamC16]{Details} \href{../works/HamC16.pdf}{Yes} & \cite{HamC16} & 2016 & Computers \  Industrial Engineering & 6 & \noindent{}\textbf{2.50} \textbf{2.50} \textbf{8.71} & 50 55 55 & 26 35 & 19 16 3\\
\index{HebrardHJMPV16}\rowlabel{a:HebrardHJMPV16}HebrardHJMPV16 \href{https://doi.org/10.1016/j.dam.2016.07.003}{HebrardHJMPV16} & \hyperref[auth:a1]{E. Hebrard}, \hyperref[auth:a54]{M.-J. Huguet}, \hyperref[auth:a791]{N. Jozefowiez}, \hyperref[auth:a787]{A. Maillard}, \hyperref[auth:a21]{C. Pralet}, \hyperref[auth:a173]{G. Verfaillie} & \cellcolor{gold!20}Approximation of the parallel machine scheduling problem with additional unit resources & \hyperref[detail:HebrardHJMPV16]{Details} \href{../works/HebrardHJMPV16.pdf}{Yes} & \cite{HebrardHJMPV16} & 2016 & Discrete Applied Mathematics & 10 & \noindent{}\textcolor{black!50}{0.00} \textcolor{black!50}{0.00} \textcolor{black!50}{0.00} & 9 10 12 & 8 8 & 1 0 1\\
\index{KuB16}\rowlabel{a:KuB16}KuB16 \href{https://doi.org/10.1016/j.cor.2016.04.006}{KuB16} & \hyperref[auth:a331]{W.-Y. Ku}, \hyperref[auth:a89]{J. C. Beck} & \cellcolor{green!10}Mixed Integer Programming models for job shop scheduling: {A} computational analysis & \hyperref[detail:KuB16]{Details} \href{../works/KuB16.pdf}{Yes} & \cite{KuB16} & 2016 & Computers \  Operations Research & 9 & \noindent{}\textcolor{black!50}{0.00} \textcolor{black!50}{0.00} \textbf{5.53} & 119 132 141 & 17 25 & 25 19 6\\
\index{Li2016}\rowlabel{a:Li2016}Li2016 \href{http://dx.doi.org/10.3141/2549-01}{Li2016} & \hyperref[auth:a2066]{S. Li}, \hyperref[auth:a2067]{R. R. Negenborn}, \hyperref[auth:a2068]{G. Lodewijks} & Approach Integrating Mixed-Integer Programming and Constraint Programming for Planning Rotations of Inland Vessels in a Large Seaport \hyperref[abs:Li2016]{Abstract} & \hyperref[detail:Li2016]{Details} No & \cite{Li2016} & 2016 & Transportation Research Record: Journal of the Transportation Research Board & null & \noindent{}\textcolor{black!50}{0.00} \textbf{1.00} n/a & 3 2 3 & 5 8 & 2 1 1\\
\index{Menouer2016}\rowlabel{a:Menouer2016}Menouer2016 \href{http://dx.doi.org/10.1002/cpe.3840}{Menouer2016} & \hyperref[auth:a1976]{T. Menouer}, \hyperref[auth:a1977]{N. Sukhija}, \hyperref[auth:a1978]{L. C. Bertrand} & A learning Portfolio solver for optimizing the performance of constraint programming problems on multi‐core computing systems \hyperref[abs:Menouer2016]{Abstract} & \hyperref[detail:Menouer2016]{Details} No & \cite{Menouer2016} & 2016 & Concurrency and Computation: Practice and Experience & null & \noindent{}\textcolor{black!50}{0.00} \textbf{4.50} n/a & 2 2 1 & 17 38 & 1 0 1\\
\index{Moreno-Scott2016}\rowlabel{a:Moreno-Scott2016}Moreno-Scott2016 \href{http://dx.doi.org/10.1155/2016/7349070}{Moreno-Scott2016} & \hyperref[auth:a1783]{J. H. Moreno-Scott}, \hyperref[auth:a1781]{J. C. Ortiz-Bayliss}, \hyperref[auth:a1608]{H. Terashima-Marín}, \hyperref[auth:a1782]{S. E. Conant-Pablos} & \cellcolor{gold!20}Experimental Matching of Instances to Heuristics for Constraint Satisfaction Problems \hyperref[abs:Moreno-Scott2016]{Abstract} & \hyperref[detail:Moreno-Scott2016]{Details} No & \cite{Moreno-Scott2016} & 2016 & Computational Intelligence and Neuroscience & null & \noindent{}\textcolor{black!50}{0.00} 0.50 n/a & 4 4 3 & 29 29 & 4 1 3\\
\index{NattafALR16}\rowlabel{a:NattafALR16}NattafALR16 \href{https://doi.org/10.1007/s00291-015-0423-x}{NattafALR16} & \hyperref[auth:a81]{M. Nattaf}, \hyperref[auth:a6]{C. Artigues}, \hyperref[auth:a3]{P. Lopez}, \hyperref[auth:a979]{D. Rivreau} & \cellcolor{green!10}Energetic reasoning and mixed-integer linear programming for scheduling with a continuous resource and linear efficiency functions & \hyperref[detail:NattafALR16]{Details} \href{../works/NattafALR16.pdf}{Yes} & \cite{NattafALR16} & 2016 & {OR} Spectrum & 34 & \noindent{}\textcolor{black!50}{0.00} \textcolor{black!50}{0.00} \textbf{1.68} & 10 10 10 & 15 19 & 6 1 5\\
\index{NovaraNH16}\rowlabel{a:NovaraNH16}NovaraNH16 \href{https://doi.org/10.1016/j.compchemeng.2016.04.030}{NovaraNH16} & \hyperref[auth:a587]{F. M. Novara}, \hyperref[auth:a524]{J. M. Novas}, \hyperref[auth:a588]{G. P. Henning} & A novel constraint programming model for large-scale scheduling problems in multiproduct multistage batch plants: Limited resources and campaign-based operation & \hyperref[detail:NovaraNH16]{Details} \href{../works/NovaraNH16.pdf}{Yes} & \cite{NovaraNH16} & 2016 & Computers \  Chemical Engineering & 17 & \noindent{}\textbf{1.50} \textbf{1.50} \textbf{22.47} & 18 17 19 & 31 40 & 12 8 4\\
\index{OrnekO16}\rowlabel{a:OrnekO16}OrnekO16 \href{https://journals.sfu.ca/ijietap/index.php/ijie/article/view/1930}{OrnekO16} & \hyperref[auth:a138]{A. {\"{O}}rnek}, \hyperref[auth:a135]{C. {\"{O}}zt{\"{u}}rk} & Optimisation and Constraint Based Heuristic Methods for Advanced Planning and Scheduling Systems & \hyperref[detail:OrnekO16]{Details} \href{../works/OrnekO16.pdf}{Yes} & \cite{OrnekO16} & 2016 & International Journal of Industrial Engineering: Theory, Applications and Practice & 25 & \noindent{}\textcolor{black!50}{0.00} \textcolor{black!50}{0.00} \textbf{21.72} & 0 0 0 & 0 0 & 0 0 0\\
\index{QinDS16}\rowlabel{a:QinDS16}QinDS16 \href{http://dx.doi.org/10.1016/j.tre.2016.01.007}{QinDS16} & \hyperref[auth:a509]{T. Qin}, \hyperref[auth:a510]{Y. Du}, \hyperref[auth:a512]{M. Sha} & Evaluating the solution performance of IP and CP for berth allocation with time-varying water depth & \hyperref[detail:QinDS16]{Details} \href{../works/QinDS16.pdf}{Yes} & \cite{QinDS16} & 2016 & Transportation Research Part E: Logistics and Transportation Review & 19 & \noindent{}\textcolor{black!50}{0.00} \textcolor{black!50}{0.00} \textbf{14.66} & 17 18 21 & 40 49 & 14 3 11\\
\index{Ren2016}\rowlabel{a:Ren2016}Ren2016 \href{http://dx.doi.org/10.1155/2016/5201937}{Ren2016} & \hyperref[auth:a1249]{H. Ren}, \hyperref[auth:a1611]{S. Sun} & \cellcolor{gold!20}A Hybrid IP/GA Approach to the Parallel Production Lines Scheduling Problem \hyperref[abs:Ren2016]{Abstract} & \hyperref[detail:Ren2016]{Details} No & \cite{Ren2016} & 2016 & Discrete Dynamics in Nature and Society & null & \noindent{}\textcolor{black!50}{0.00} \textbf{2.00} n/a & 0 0 0 & 32 37 & 5 0 5\\
\index{RiiseML16}\rowlabel{a:RiiseML16}RiiseML16 \href{http://dx.doi.org/10.1016/j.ejor.2016.06.015}{RiiseML16} & \hyperref[auth:a1064]{A. Riise}, \hyperref[auth:a1065]{C. Mannino}, \hyperref[auth:a1066]{L. Lamorgese} & Recursive logic-based Benders' decomposition for multi-mode outpatient scheduling & \hyperref[detail:RiiseML16]{Details} \href{../works/RiiseML16.pdf}{Yes} & \cite{RiiseML16} & 2016 & European Journal of Operational Research & 10 & \noindent{}\textcolor{black!50}{0.00} \textcolor{black!50}{0.00} 0.72 & 27 27 26 & 29 41 & 14 8 6\\
\index{Sitek2016}\rowlabel{a:Sitek2016}Sitek2016 \href{http://dx.doi.org/10.1155/2016/5102616}{Sitek2016} & \hyperref[auth:a1475]{P. Sitek}, \hyperref[auth:a1476]{J. Wikarek} & \cellcolor{gold!20}A Hybrid Programming Framework for Modeling and Solving Constraint Satisfaction and Optimization Problems \hyperref[abs:Sitek2016]{Abstract} & \hyperref[detail:Sitek2016]{Details} No & \cite{Sitek2016} & 2016 & Scientific Programming & null & \noindent{}\textcolor{black!50}{0.00} \textbf{9.01} n/a & 40 39 57 & 11 15 & 9 3 6\\
\index{Teschemacher2016}\rowlabel{a:Teschemacher2016}Teschemacher2016 \href{http://dx.doi.org/10.1016/j.procir.2015.12.071}{Teschemacher2016} & \hyperref[auth:a1905]{U. Teschemacher}, \hyperref[auth:a1906]{G. Reinhart} & \cellcolor{gold!20}Enhancing Constraint Propagation in ACO-based Schedulers for Solving the Job Shop Scheduling Problem & \hyperref[detail:Teschemacher2016]{Details} No & \cite{Teschemacher2016} & 2016 & Procedia CIRP & null & \noindent{}\textbf{3.00} \textbf{3.00} n/a & 5 5 8 & 4 8 & 1 0 1\\
\index{TranAB16}\rowlabel{a:TranAB16}TranAB16 \href{https://doi.org/10.1287/ijoc.2015.0666}{TranAB16} & \hyperref[auth:a799]{T. T. Tran}, \hyperref[auth:a807]{A. Araujo}, \hyperref[auth:a89]{J. C. Beck} & Decomposition Methods for the Parallel Machine Scheduling Problem with Setups & \hyperref[detail:TranAB16]{Details} \href{../works/TranAB16.pdf}{Yes} & \cite{TranAB16} & 2016 & \cellcolor{red!20}INFORMS Journal on Computing & 13 & \noindent{}\textcolor{black!50}{0.00} \textcolor{black!50}{0.00} \textbf{3.31} & 72 75 80 & 28 36 & 31 17 14\\
\index{ZarandiKS16}\rowlabel{a:ZarandiKS16}ZarandiKS16 \href{https://doi.org/10.1007/s10845-013-0860-9}{ZarandiKS16} & \hyperref[auth:a589]{M. H. F. Zarandi}, \hyperref[auth:a590]{H. Khorshidian}, \hyperref[auth:a591]{M. A. Shirazi} & A constraint programming model for the scheduling of {JIT} cross-docking systems with preemption & \hyperref[detail:ZarandiKS16]{Details} \href{../works/ZarandiKS16.pdf}{Yes} & \cite{ZarandiKS16} & 2016 & Journal of Intelligent Manufacturing & 17 & \noindent{}\textbf{1.00} \textbf{1.00} \textbf{5.09} & 28 29 31 & 14 22 & 9 4 5\\
\index{AlesioBNG15}\rowlabel{a:AlesioBNG15}AlesioBNG15 \href{http://dx.doi.org/10.1145/2818640}{AlesioBNG15} & \hyperref[auth:a1223]{S. D. Alesio}, \hyperref[auth:a236]{L. C. Briand}, \hyperref[auth:a235]{S. Nejati}, \hyperref[auth:a195]{A. Gotlieb} & \cellcolor{green!10}Combining Genetic Algorithms and Constraint Programming to Support Stress Testing of Task Deadlines & \hyperref[detail:AlesioBNG15]{Details} \href{../works/AlesioBNG15.pdf}{Yes} & \cite{AlesioBNG15} & 2015 & ACM Transactions on Software Engineering and Methodology & 37 & \noindent{}\textbf{1.00} \textbf{1.00} \textbf{119.50} & 13 14 17 & 51 59 & 10 0 10\\
\index{BajestaniB15}\rowlabel{a:BajestaniB15}BajestaniB15 \href{https://doi.org/10.1007/s10951-015-0416-2}{BajestaniB15} & \hyperref[auth:a817]{M. A. Bajestani}, \hyperref[auth:a89]{J. C. Beck} & A two-stage coupled algorithm for an integrated maintenance planning and flowshop scheduling problem with deteriorating machines & \hyperref[detail:BajestaniB15]{Details} \href{../works/BajestaniB15.pdf}{Yes} & \cite{BajestaniB15} & 2015 & Journal of Scheduling & 16 & \noindent{}\textcolor{black!50}{0.00} \textcolor{black!50}{0.00} \textbf{1.05} & 17 18 20 & 59 69 & 8 1 7\\
\index{Bzdyra2015}\rowlabel{a:Bzdyra2015}Bzdyra2015 \href{http://dx.doi.org/10.4028/www.scientific.net/amm.791.70}{Bzdyra2015} & \hyperref[auth:a1813]{K. Bzdyra}, \hyperref[auth:a630]{G. Bocewicz}, \hyperref[auth:a1814]{Z. Banaszak} & Mass Customized Projects Portfolio Scheduling - Imprecise Operations Time Approach \hyperref[abs:Bzdyra2015]{Abstract} & \hyperref[detail:Bzdyra2015]{Details} No & \cite{Bzdyra2015} & 2015 & Applied Mechanics and Materials & null & \noindent{}\textcolor{black!50}{0.00} \textbf{2.50} n/a & 5 5 0 & 10 14 & 2 1 1\\
\index{EvenSH15a}\rowlabel{a:EvenSH15a}EvenSH15a \href{http://arxiv.org/abs/1505.02487}{EvenSH15a} & \hyperref[auth:a214]{C. Even}, \hyperref[auth:a124]{A. Schutt}, \hyperref[auth:a148]{P. V. Hentenryck} & A Constraint Programming Approach for Non-Preemptive Evacuation Scheduling & \hyperref[detail:EvenSH15a]{Details} \href{../works/EvenSH15a.pdf}{Yes} & \cite{EvenSH15a} & 2015 & CoRR & 16 & \noindent{}\textbf{1.00} \textbf{1.00} 0.42 & 0 0 0 & 0 0 & 0 0 0\\
\index{GoelSHFS15}\rowlabel{a:GoelSHFS15}GoelSHFS15 \href{https://doi.org/10.1016/j.ejor.2014.09.048}{GoelSHFS15} & \hyperref[auth:a592]{V. Goel}, \hyperref[auth:a593]{M. Slusky}, \hyperref[auth:a206]{W.-J. van Hoeve}, \hyperref[auth:a594]{K. C. Furman}, \hyperref[auth:a595]{Y. Shao} & Constraint programming for {LNG} ship scheduling and inventory management & \hyperref[detail:GoelSHFS15]{Details} \href{../works/GoelSHFS15.pdf}{Yes} & \cite{GoelSHFS15} & 2015 & European Journal of Operational Research & 12 & \noindent{}\textbf{1.00} \textbf{1.00} \textbf{8.63} & 48 53 54 & 4 8 & 9 9 0\\
\index{GrimesH15}\rowlabel{a:GrimesH15}GrimesH15 \href{https://doi.org/10.1287/ijoc.2014.0625}{GrimesH15} & \hyperref[auth:a181]{D. Grimes}, \hyperref[auth:a1]{E. Hebrard} & \cellcolor{green!10}Solving Variants of the Job Shop Scheduling Problem Through Conflict-Directed Search & \hyperref[detail:GrimesH15]{Details} \href{../works/GrimesH15.pdf}{Yes} & \cite{GrimesH15} & 2015 & \cellcolor{red!20}INFORMS Journal on Computing & 17 & \noindent{}\textcolor{black!50}{0.00} \textcolor{black!50}{0.00} \textbf{35.48} & 12 13 16 & 41 66 & 28 5 23\\
\index{Kameugne15}\rowlabel{a:Kameugne15}Kameugne15 \href{https://doi.org/10.1007/s10601-015-9227-5}{Kameugne15} & \hyperref[auth:a10]{R. Kameugne} & Propagation techniques of resource constraint for cumulative scheduling & \hyperref[detail:Kameugne15]{Details} \href{../works/Kameugne15.pdf}{Yes} & \cite{Kameugne15} & 2015 & Constraints An Int. J. & 2 & \noindent{}0.75 0.75 \textcolor{black!50}{0.07} & 0 0 0 & 0 0 & 0 0 0\\
\index{LetortCB15}\rowlabel{a:LetortCB15}LetortCB15 \href{https://doi.org/10.1007/s10601-014-9172-8}{LetortCB15} & \hyperref[auth:a127]{A. Letort}, \hyperref[auth:a91]{M. Carlsson}, \hyperref[auth:a128]{N. Beldiceanu} & \cellcolor{green!10}Synchronized sweep algorithms for scalable scheduling constraints & \hyperref[detail:LetortCB15]{Details} \href{../works/LetortCB15.pdf}{Yes} & \cite{LetortCB15} & 2015 & Constraints An Int. J. & 52 & \noindent{}\textcolor{black!50}{0.00} \textcolor{black!50}{0.00} \textbf{18.50} & 2 2 4 & 14 28 & 12 0 12\\
\index{Li2015}\rowlabel{a:Li2015}Li2015 \href{http://dx.doi.org/10.1007/s10732-015-9305-2}{Li2015} & \hyperref[auth:a1796]{H. Li}, \hyperref[auth:a1797]{Y. Liang}, \hyperref[auth:a1798]{N. Zhang}, \hyperref[auth:a1799]{J. Guo}, \hyperref[auth:a1800]{D. Xu}, \hyperref[auth:a1801]{Z. Li} & Improving degree-based variable ordering heuristics for solving constraint satisfaction problems & \hyperref[detail:Li2015]{Details} No & \cite{Li2015} & 2015 & Journal of Heuristics & null & \noindent{}0.50 0.50 n/a & 6 6 12 & 14 25 & 7 3 4\\
\index{Lindauer2015}\rowlabel{a:Lindauer2015}Lindauer2015 \href{http://dx.doi.org/10.1613/jair.4726}{Lindauer2015} & \hyperref[auth:a1942]{M. Lindauer}, \hyperref[auth:a1943]{H. H. Hoos}, \hyperref[auth:a1944]{F. Hutter}, \hyperref[auth:a1945]{T. Schaub} & \cellcolor{gold!20}AutoFolio: An Automatically Configured Algorithm Selector \hyperref[abs:Lindauer2015]{Abstract} & \hyperref[detail:Lindauer2015]{Details} No & \cite{Lindauer2015} & 2015 & Journal of Artificial Intelligence Research & null & \noindent{}\textcolor{black!50}{0.00} 0.50 n/a & 53 58 84 & 0 0 & 1 1 0\\
\index{Mladenovic2015}\rowlabel{a:Mladenovic2015}Mladenovic2015 \href{http://dx.doi.org/10.1002/net.21625}{Mladenovic2015} & \hyperref[auth:a1621]{S. Mladenovic}, \hyperref[auth:a1622]{S. Veskovic}, \hyperref[auth:a1623]{I. Branovic}, \hyperref[auth:a1624]{S. Jankovic}, \hyperref[auth:a1625]{S. Acimovic} & Heuristic Based Real‐Time Train Rescheduling System \hyperref[abs:Mladenovic2015]{Abstract} & \hyperref[detail:Mladenovic2015]{Details} No & \cite{Mladenovic2015} & 2015 & Networks & null & \noindent{}\textcolor{black!50}{0.00} \textbf{2.00} n/a & 2 2 3 & 21 30 & 4 0 4\\
\index{NattafAL15}\rowlabel{a:NattafAL15}NattafAL15 \href{https://doi.org/10.1007/s10601-015-9192-z}{NattafAL15} & \hyperref[auth:a81]{M. Nattaf}, \hyperref[auth:a6]{C. Artigues}, \hyperref[auth:a3]{P. Lopez} & \cellcolor{green!10}A hybrid exact method for a scheduling problem with a continuous resource and energy constraints & \hyperref[detail:NattafAL15]{Details} \href{../works/NattafAL15.pdf}{Yes} & \cite{NattafAL15} & 2015 & Constraints An Int. J. & 21 & \noindent{}\textcolor{black!50}{0.00} \textcolor{black!50}{0.00} 0.96 & 14 15 15 & 13 18 & 7 3 4\\
\index{Oliveira2015}\rowlabel{a:Oliveira2015}Oliveira2015 \href{http://dx.doi.org/10.14807/ijmp.v6i1.262}{Oliveira2015} & \hyperref[auth:a1568]{Renata Melo e Silva de Oliveira}, \hyperref[auth:a1569]{Maria Sofia F. Oliveira de Castro Ribeiro} & Comparing Mixed \  Integer Programming vs. Constraint Programming by solving Job-Shop Scheduling Problems & \hyperref[detail:Oliveira2015]{Details} No & \cite{Oliveira2015} & 2015 & Independent Journal of Management \  Production & null & \noindent{}\textbf{2.00} \textbf{2.00} n/a & 2 1 0 & 0 0 & 2 2 0\\
\index{OzturkTHO15}\rowlabel{a:OzturkTHO15}OzturkTHO15 \href{https://www.sciencedirect.com/science/article/pii/S0278612515000527}{OzturkTHO15} & \hyperref[auth:a135]{C. {\"{O}}zt{\"{u}}rk}, \hyperref[auth:a1016]{S. Tunalı}, \hyperref[auth:a137]{B. Hnich}, \hyperref[auth:a138]{A. {\"{O}}rnek} & Cyclic scheduling of flexible mixed model assembly lines with parallel stations & \hyperref[detail:OzturkTHO15]{Details} \href{../works/OzturkTHO15.pdf}{Yes} & \cite{OzturkTHO15} & 2015 & Journal of Manufacturing Systems & 12 & \noindent{}\textcolor{black!50}{0.00} \textcolor{black!50}{0.00} \textbf{14.33} & 27 28 31 & 17 32 & 9 6 3\\
\index{Sahraeian2015}\rowlabel{a:Sahraeian2015}Sahraeian2015 \href{http://dx.doi.org/10.1016/j.apm.2014.07.032}{Sahraeian2015} & \hyperref[auth:a1863]{R. Sahraeian}, \hyperref[auth:a1864]{M. Namakshenas} & \cellcolor{gold!20}On the optimal modeling and evaluation of job shops with a total weighted tardiness objective: Constraint programming vs. mixed integer programming & \hyperref[detail:Sahraeian2015]{Details} No & \cite{Sahraeian2015} & 2015 & Applied Mathematical Modelling & null & \noindent{}\textbf{1.00} \textbf{1.00} n/a & 2 3 8 & 11 18 & 1 1 0\\
\index{SchnellH15}\rowlabel{a:SchnellH15}SchnellH15 \href{http://dx.doi.org/10.1007/s00291-015-0419-6}{SchnellH15} & \hyperref[auth:a950]{A. Schnell}, \hyperref[auth:a951]{R. F. Hartl} & On the efficient modeling and solution of the multi-mode resource-constrained project scheduling problem with generalized precedence relations & \hyperref[detail:SchnellH15]{Details} \href{../works/SchnellH15.pdf}{Yes} & \cite{SchnellH15} & 2015 & {OR} Spectrum & 21 & \noindent{}\textcolor{black!50}{0.00} \textcolor{black!50}{0.00} \textbf{2.11} & 24 27 31 & 20 30 & 19 8 11\\
\index{Siala15}\rowlabel{a:Siala15}Siala15 \href{https://doi.org/10.1007/s10601-015-9213-y}{Siala15} & \hyperref[auth:a129]{M. Siala} & Search, propagation, and learning in sequencing and scheduling problems & \hyperref[detail:Siala15]{Details} \href{../works/Siala15.pdf}{Yes} & \cite{Siala15} & 2015 & Constraints An Int. J. & 2 & \noindent{}0.50 0.50 \textbf{1.43} & 4 3 0 & 0 0 & 0 0 0\\
\index{SimoninAHL15}\rowlabel{a:SimoninAHL15}SimoninAHL15 \href{https://doi.org/10.1007/s10601-014-9169-3}{SimoninAHL15} & \hyperref[auth:a126]{G. Simonin}, \hyperref[auth:a6]{C. Artigues}, \hyperref[auth:a1]{E. Hebrard}, \hyperref[auth:a3]{P. Lopez} & \cellcolor{green!10}Scheduling scientific experiments for comet exploration & \hyperref[detail:SimoninAHL15]{Details} \href{../works/SimoninAHL15.pdf}{Yes} & \cite{SimoninAHL15} & 2015 & Constraints An Int. J. & 23 & \noindent{}\textcolor{black!50}{0.00} \textcolor{black!50}{0.00} 0.87 & 4 4 6 & 5 8 & 2 0 2\\
\index{Soto2015}\rowlabel{a:Soto2015}Soto2015 \href{http://dx.doi.org/10.1155/2015/580785}{Soto2015} & \hyperref[auth:a1830]{R. Soto}, \hyperref[auth:a1831]{B. Crawford}, \hyperref[auth:a1832]{W. Palma}, \hyperref[auth:a1833]{E. Monfroy}, \hyperref[auth:a1834]{R. Olivares}, \hyperref[auth:a1835]{C. Castro}, \hyperref[auth:a1836]{F. Paredes} & \cellcolor{gold!20}Top-kBased Adaptive Enumeration in Constraint Programming \hyperref[abs:Soto2015]{Abstract} & \hyperref[detail:Soto2015]{Details} No & \cite{Soto2015} & 2015 & Mathematical Problems in Engineering & null & \noindent{}\textcolor{black!50}{0.00} \textbf{1.00} n/a & 8 7 8 & 17 22 & 2 1 1\\
\index{Talbi2015}\rowlabel{a:Talbi2015}Talbi2015 \href{http://dx.doi.org/10.1007/s10479-015-2034-y}{Talbi2015} & \hyperref[auth:a1659]{E.-G. Talbi} & Combining metaheuristics with mathematical programming, constraint programming and machine learning & \hyperref[detail:Talbi2015]{Details} No & \cite{Talbi2015} & 2015 & Annals of Operations Research & null & \noindent{}0.50 0.50 n/a & 84 90 87 & 88 151 & 11 2 9\\
\index{Wang2015}\rowlabel{a:Wang2015}Wang2015 \href{http://dx.doi.org/10.3141/2482-15}{Wang2015} & \hyperref[auth:a1710]{S. Wang}, \hyperref[auth:a1711]{E. Y. Chou} & Cross-Asset Transportation Project Coordination with Integer Programming and Constraint Programming \hyperref[abs:Wang2015]{Abstract} & \hyperref[detail:Wang2015]{Details} No & \cite{Wang2015} & 2015 & Transportation Research Record: Journal of the Transportation Research Board & null & \noindent{}\textcolor{black!50}{0.00} \textbf{1.00} n/a & 1 1 2 & 9 11 & 4 0 4\\
\index{WangMD15}\rowlabel{a:WangMD15}WangMD15 \href{https://doi.org/10.1016/j.ejor.2015.06.008}{WangMD15} & \hyperref[auth:a596]{T. Wang}, \hyperref[auth:a597]{N. Meskens}, \hyperref[auth:a598]{D. Duvivier} & Scheduling operating theatres: Mixed integer programming vs. constraint programming & \hyperref[detail:WangMD15]{Details} \href{../works/WangMD15.pdf}{Yes} & \cite{WangMD15} & 2015 & European Journal of Operational Research & 13 & \noindent{}\textbf{1.00} \textbf{1.00} \textbf{6.87} & 36 38 38 & 33 49 & 21 15 6\\
\index{Amadini2014}\rowlabel{a:Amadini2014}Amadini2014 \href{http://dx.doi.org/10.1017/s1471068414000179}{Amadini2014} & \hyperref[auth:a910]{R. Amadini}, \hyperref[auth:a192]{M. Gabbrielli}, \hyperref[auth:a193]{J. Mauro} & \cellcolor{green!10}SUNNY: a Lazy Portfolio Approach for Constraint Solving \hyperref[abs:Amadini2014]{Abstract} & \hyperref[detail:Amadini2014]{Details} No & \cite{Amadini2014} & 2014 & Theory and Practice of Logic Programming & null & \noindent{}\textcolor{black!50}{0.00} \textbf{1.50} n/a & 19 20 42 & 10 33 & 2 0 2\\
\index{ArtiguesL14}\rowlabel{a:ArtiguesL14}ArtiguesL14 \href{http://dx.doi.org/10.1007/s10951-014-0404-y}{ArtiguesL14} & \hyperref[auth:a6]{C. Artigues}, \hyperref[auth:a3]{P. Lopez} & \cellcolor{green!10}Energetic reasoning for energy-constrained scheduling with a continuous resource & \hyperref[detail:ArtiguesL14]{Details} \href{../works/ArtiguesL14.pdf}{Yes} & \cite{ArtiguesL14} & 2014 & Journal of Scheduling & 17 & \noindent{}\textcolor{black!50}{0.00} \textcolor{black!50}{0.00} \textbf{1.14} & 11 13 14 & 19 26 & 16 8 8\\
\index{Banaszak2014}\rowlabel{a:Banaszak2014}Banaszak2014 \href{http://dx.doi.org/10.1515/fman-2015-0014}{Banaszak2014} & \hyperref[auth:a1814]{Z. Banaszak}, \hyperref[auth:a630]{G. Bocewicz} & Declarative Modeling for Production Order Portfolio Scheduling \hyperref[abs:Banaszak2014]{Abstract} & \hyperref[detail:Banaszak2014]{Details} No & \cite{Banaszak2014} & 2014 & Foundations of Management & null & \noindent{}\textcolor{black!50}{0.00} \textbf{2.50} n/a & 8 8 0 & 0 0 & 1 1 0\\
\index{Bergman2014}\rowlabel{a:Bergman2014}Bergman2014 \href{http://dx.doi.org/10.1613/jair.4199}{Bergman2014} & \hyperref[auth:a1514]{D. Bergman}, \hyperref[auth:a1515]{A. A. Cire}, \hyperref[auth:a1516]{W. V. Hoeve} & \cellcolor{gold!20}MDD Propagation for Sequence Constraints \hyperref[abs:Bergman2014]{Abstract} & \hyperref[detail:Bergman2014]{Details} No & \cite{Bergman2014} & 2014 & Journal of Artificial Intelligence Research & null & \noindent{}\textcolor{black!50}{0.00} \textbf{2.25} n/a & 14 14 19 & 0 0 & 3 3 0\\
\index{BlomBPS14}\rowlabel{a:BlomBPS14}BlomBPS14 \href{https://doi.org/10.1287/ijoc.2013.0590}{BlomBPS14} & \hyperref[auth:a795]{M. L. Blom}, \hyperref[auth:a322]{C. N. Burt}, \hyperref[auth:a324]{A. R. Pearce}, \hyperref[auth:a125]{P. J. Stuckey} & A Decomposition-Based Heuristic for Collaborative Scheduling in a Network of Open-Pit Mines & \hyperref[detail:BlomBPS14]{Details} \href{../works/BlomBPS14.pdf}{Yes} & \cite{BlomBPS14} & 2014 & \cellcolor{red!20}INFORMS Journal on Computing & 19 & \noindent{}\textcolor{black!50}{0.00} \textcolor{black!50}{0.00} \textcolor{black!50}{0.00} & 15 15 16 & 47 55 & 2 1 1\\
\index{BonfiettiLBM14}\rowlabel{a:BonfiettiLBM14}BonfiettiLBM14 \href{https://doi.org/10.1016/j.artint.2013.09.006}{BonfiettiLBM14} & \hyperref[auth:a198]{A. Bonfietti}, \hyperref[auth:a142]{M. Lombardi}, \hyperref[auth:a245]{L. Benini}, \hyperref[auth:a143]{M. Milano} & \cellcolor{gold!20}{CROSS} cyclic resource-constrained scheduling solver & \hyperref[detail:BonfiettiLBM14]{Details} \href{../works/BonfiettiLBM14.pdf}{Yes} & \cite{BonfiettiLBM14} & 2014 & Artificial Intelligence & 28 & \noindent{}\textcolor{black!50}{0.00} \textcolor{black!50}{0.00} \textbf{5.15} & 8 9 8 & 15 35 & 4 3 1\\
\index{Chaleshtarti2014}\rowlabel{a:Chaleshtarti2014}Chaleshtarti2014 \href{http://dx.doi.org/10.1155/2014/634649}{Chaleshtarti2014} & \hyperref[auth:a1755]{A. S. Chaleshtarti}, \hyperref[auth:a1756]{S. Shadrokh}, \hyperref[auth:a1757]{Y. Fathi} & \cellcolor{gold!20}Branch and Bound Algorithms for Resource Constrained Project Scheduling Problem Subject to Nonrenewable Resources with Prescheduled Procurement \hyperref[abs:Chaleshtarti2014]{Abstract} & \hyperref[detail:Chaleshtarti2014]{Details} No & \cite{Chaleshtarti2014} & 2014 & Mathematical Problems in Engineering & null & \noindent{}\textcolor{black!50}{0.00} \textcolor{black!50}{0.00} n/a & 3 3 5 & 26 33 & 7 0 7\\
\index{Dolabi2014}\rowlabel{a:Dolabi2014}Dolabi2014 \href{http://dx.doi.org/10.1016/j.autcon.2014.09.003}{Dolabi2014} & \hyperref[auth:a1748]{H. R. Z. Dolabi}, \hyperref[auth:a1749]{A. Afshar}, \hyperref[auth:a1750]{R. Abbasnia} & CPM/LOB Scheduling Method for Project Deadline Constraint Satisfaction & \hyperref[detail:Dolabi2014]{Details} No & \cite{Dolabi2014} & 2014 & Automation in Construction & null & \noindent{}\textbf{1.00} \textbf{1.00} n/a & 33 38 41 & 30 31 & 5 4 1\\
\index{GrimesIOS14}\rowlabel{a:GrimesIOS14}GrimesIOS14 \href{https://doi.org/10.1016/j.suscom.2014.08.009}{GrimesIOS14} & \hyperref[auth:a181]{D. Grimes}, \hyperref[auth:a182]{G. Ifrim}, \hyperref[auth:a16]{B. O'Sullivan}, \hyperref[auth:a17]{H. Simonis} & \cellcolor{green!10}Analyzing the impact of electricity price forecasting on energy cost-aware scheduling & \hyperref[detail:GrimesIOS14]{Details} \href{../works/GrimesIOS14.pdf}{Yes} & \cite{GrimesIOS14} & 2014 & Sustain. Comput. Informatics Syst. & 16 & \noindent{}\textcolor{black!50}{0.00} \textcolor{black!50}{0.00} 0.85 & 6 6 19 & 7 28 & 1 0 1\\
\index{Han2014}\rowlabel{a:Han2014}Han2014 \href{http://dx.doi.org/10.1007/s10479-014-1619-1}{Han2014} & \hyperref[auth:a1664]{A. F. Han}, \hyperref[auth:a1665]{E. C. Li} & A constraint programming-based approach to the crew scheduling problem of the Taipei mass rapid transit system & \hyperref[detail:Han2014]{Details} No & \cite{Han2014} & 2014 & Annals of Operations Research & null & \noindent{}\textbf{1.00} \textbf{1.00} n/a & 13 15 15 & 21 28 & 9 2 7\\
\index{HarjunkoskiMBC14}\rowlabel{a:HarjunkoskiMBC14}HarjunkoskiMBC14 \href{http://dx.doi.org/10.1016/j.compchemeng.2013.12.001}{HarjunkoskiMBC14} & \hyperref[auth:a871]{I. Harjunkoski}, \hyperref[auth:a381]{C. T. Maravelias}, \hyperref[auth:a937]{P. Bongers}, \hyperref[auth:a891]{P. M. Castro}, \hyperref[auth:a70]{S. Engell}, \hyperref[auth:a382]{I. E. Grossmann}, \hyperref[auth:a160]{J. N. Hooker}, \hyperref[auth:a938]{C. Méndez}, \hyperref[auth:a939]{G. Sand}, \hyperref[auth:a940]{J. Wassick} & \cellcolor{green!10}Scope for industrial applications of production scheduling models and solution methods & \hyperref[detail:HarjunkoskiMBC14]{Details} \href{../works/HarjunkoskiMBC14.pdf}{Yes} & \cite{HarjunkoskiMBC14} & 2014 & Computers \  Chemical Engineering & 33 & \noindent{}\textcolor{black!50}{0.00} \textcolor{black!50}{0.00} \textbf{41.04} & 381 393 418 & 176 229 & 28 10 18\\
\index{Juan2014}\rowlabel{a:Juan2014}Juan2014 \href{http://dx.doi.org/10.1142/s0217595914500419}{Juan2014} & \hyperref[auth:a1981]{Y.-C. Juan}, \hyperref[auth:a1982]{Y.-R. Peng} & A Constraint Satisfaction Coordination Approach for Distributed Supply Chain Production Planning \hyperref[abs:Juan2014]{Abstract} & \hyperref[detail:Juan2014]{Details} No & \cite{Juan2014} & 2014 & Asia-Pacific Journal of Operational Research & null & \noindent{}\textcolor{black!50}{0.00} \textbf{3.00} n/a & 0 0 0 & 14 20 & 1 0 1\\
\index{KameugneFSN14}\rowlabel{a:KameugneFSN14}KameugneFSN14 \href{https://doi.org/10.1007/s10601-013-9157-z}{KameugneFSN14} & \hyperref[auth:a10]{R. Kameugne}, \hyperref[auth:a130]{L. P. Fotso}, \hyperref[auth:a131]{J. D. Scott}, \hyperref[auth:a132]{Y. Ngo-Kateu} & A quadratic edge-finding filtering algorithm for cumulative resource constraints & \hyperref[detail:KameugneFSN14]{Details} \href{../works/KameugneFSN14.pdf}{Yes} & \cite{KameugneFSN14} & 2014 & Constraints An Int. J. & 27 & \noindent{}\textcolor{black!50}{0.00} \textcolor{black!50}{0.00} \textbf{5.63} & 6 6 9 & 10 20 & 16 6 10\\
\index{Kelareva2014}\rowlabel{a:Kelareva2014}Kelareva2014 \href{http://dx.doi.org/10.1007/s13675-014-0022-7}{Kelareva2014} & \hyperref[auth:a332]{E. Kelareva}, \hyperref[auth:a333]{K. Tierney}, \hyperref[auth:a334]{P. Kilby} & \cellcolor{gold!20}CP methods for scheduling and routing with time-dependent task costs & \hyperref[detail:Kelareva2014]{Details} No & \cite{Kelareva2014} & 2014 & EURO Journal on Computational Optimization & null & \noindent{}\textcolor{black!50}{0.00} \textcolor{black!50}{0.00} n/a & 10 12 15 & 49 71 & 8 1 7\\
\index{LaborieR14}\rowlabel{a:LaborieR14}LaborieR14 \href{http://dx.doi.org/10.1007/s10951-014-0408-7}{LaborieR14} & \hyperref[auth:a118]{P. Laborie}, \hyperref[auth:a1069]{J. Rogerie} & Temporal linear relaxation in IBM ILOG CP Optimizer & \hyperref[detail:LaborieR14]{Details} \href{../works/LaborieR14.pdf}{Yes} & \cite{LaborieR14} & 2014 & Journal of Scheduling & 10 & \noindent{}\textcolor{black!50}{0.00} \textcolor{black!50}{0.00} \textbf{4.83} & 17 19 22 & 13 26 & 16 12 4\\
\index{Lambert2014}\rowlabel{a:Lambert2014}Lambert2014 \href{http://dx.doi.org/10.1287/inte.2013.0731}{Lambert2014} & \hyperref[auth:a1558]{W. B. Lambert}, \hyperref[auth:a1559]{A. Brickey}, \hyperref[auth:a1560]{A. M. Newman}, \hyperref[auth:a1561]{K. Eurek} & Open-Pit Block-Sequencing Formulations: A Tutorial \hyperref[abs:Lambert2014]{Abstract} & \hyperref[detail:Lambert2014]{Details} No & \cite{Lambert2014} & 2014 & \cellcolor{red!20}Interfaces & null & \noindent{}\textcolor{black!50}{0.00} \textcolor{black!50}{0.00} n/a & 37 40 48 & 13 22 & 2 2 0\\
\index{Levine2014}\rowlabel{a:Levine2014}Levine2014 \href{http://dx.doi.org/10.1609/icaps.v24i1.13672}{Levine2014} & \hyperref[auth:a1927]{S. Levine}, \hyperref[auth:a1928]{B. Williams} & Concurrent Plan Recognition and Execution for Human-Robot Teams \hyperref[abs:Levine2014]{Abstract} & \hyperref[detail:Levine2014]{Details} No & \cite{Levine2014} & 2014 & Proceedings of the International Conference on Automated Planning and Scheduling & null & \noindent{}\textcolor{black!50}{0.00} 0.50 n/a & 17 20 0 & 0 0 & 1 1 0\\
\index{Li2014}\rowlabel{a:Li2014}Li2014 \href{http://dx.doi.org/10.4028/www.scientific.net/amm.681.265}{Li2014} & \hyperref[auth:a1492]{Y. Li}, \hyperref[auth:a1493]{Z. R. Xiao} & A Constraint-Based Approach for Multi-Skilled Project Scheduling Problem \hyperref[abs:Li2014]{Abstract} & \hyperref[detail:Li2014]{Details} No & \cite{Li2014} & 2014 & Applied Mechanics and Materials & null & \noindent{}\textcolor{black!50}{0.00} \textbf{5.00} n/a & 0 2 2 & 4 5 & 4 0 4\\
\index{Li2014a}\rowlabel{a:Li2014a}Li2014a \href{http://dx.doi.org/10.1177/1063293x14553809}{Li2014a} & \hyperref[auth:a2002]{Y. Li}, \hyperref[auth:a2003]{W. Zhao} & An integrated change propagation scheduling approach for product design \hyperref[abs:Li2014a]{Abstract} & \hyperref[detail:Li2014a]{Details} No & \cite{Li2014a} & 2014 & Concurrent Engineering & null & \noindent{}0.50 \textbf{1.50} n/a & 25 28 36 & 31 37 & 1 0 1\\
\index{Li2014b}\rowlabel{a:Li2014b}Li2014b \href{http://dx.doi.org/10.1155/2014/169097}{Li2014b} & \hyperref[auth:a2002]{Y. Li}, \hyperref[auth:a2003]{W. Zhao}, \hyperref[auth:a2017]{Y. Ma}, \hyperref[auth:a2018]{L. Hu} & \cellcolor{gold!20}Scheduling of Changes in Complex Engineering Design Process via Genetic Algorithm and Elementary Effects Method \hyperref[abs:Li2014b]{Abstract} & \hyperref[detail:Li2014b]{Details} No & \cite{Li2014b} & 2014 & Advances in Mechanical Engineering & null & \noindent{}\textcolor{black!50}{0.00} \textbf{1.25} n/a & 0 0 0 & 22 26 & 1 0 1\\
\index{Lozano2014}\rowlabel{a:Lozano2014}Lozano2014 \href{http://dx.doi.org/10.1145/2666357.2597815}{Lozano2014} & \hyperref[auth:a1522]{R. C. Lozano}, \hyperref[auth:a91]{M. Carlsson}, \hyperref[auth:a1523]{G. H. Blindell}, \hyperref[auth:a92]{C. Schulte} & Combinatorial spill code optimization and ultimate coalescing \hyperref[abs:Lozano2014]{Abstract} & \hyperref[detail:Lozano2014]{Details} No & \cite{Lozano2014} & 2014 & ACM SIGPLAN Notices & null & \noindent{}\textcolor{black!50}{0.00} \textbf{1.00} n/a & 3 2 0 & 12 22 & 1 0 1\\
\index{NovasH14}\rowlabel{a:NovasH14}NovasH14 \href{https://doi.org/10.1016/j.eswa.2013.09.026}{NovasH14} & \hyperref[auth:a524]{J. M. Novas}, \hyperref[auth:a588]{G. P. Henning} & \cellcolor{green!10}Integrated scheduling of resource-constrained flexible manufacturing systems using constraint programming & \hyperref[detail:NovasH14]{Details} \href{../works/NovasH14.pdf}{Yes} & \cite{NovasH14} & 2014 & Expert Systems with Applications & 14 & \noindent{}\textbf{1.50} \textbf{1.50} \textbf{23.96} & 35 37 44 & 26 30 & 18 6 12\\
\index{PengLC14}\rowlabel{a:PengLC14}PengLC14 \href{http://dx.doi.org/10.1155/2014/917685}{PengLC14} & \hyperref[auth:a915]{Y. Peng}, \hyperref[auth:a1385]{D. Lu}, \hyperref[auth:a913]{Y. Chen} & \cellcolor{gold!20}A Constraint Programming Method for Advanced Planning and Scheduling System with Multilevel Structured Products & \hyperref[detail:PengLC14]{Details} \href{../works/PengLC14.pdf}{Yes} & \cite{PengLC14} & 2014 & Discrete Dynamics in Nature and Society & 7 & \noindent{}\textbf{1.00} \textbf{1.00} \textbf{8.94} & 5 4 9 & 13 17 & 7 1 6\\
\index{Silva2014}\rowlabel{a:Silva2014}Silva2014 \href{http://dx.doi.org/10.1590/2238-1031.jtl.v8n4a9}{Silva2014} & \hyperref[auth:a1888]{G. P. Silva}, \hyperref[auth:a1889]{Allexandre Fortes da Silva Reis} & A study of different metaheuristics to solve the urban transit crew scheduling problem \hyperref[abs:Silva2014]{Abstract} & \hyperref[detail:Silva2014]{Details} No & \cite{Silva2014} & 2014 & Journal of Transport Literature & null & \noindent{}\textcolor{black!50}{0.00} \textbf{2.00} n/a & 2 2 0 & 10 22 & 1 1 0\\
\index{Tang2014}\rowlabel{a:Tang2014}Tang2014 \href{http://dx.doi.org/10.1016/j.autcon.2013.09.008}{Tang2014} & \hyperref[auth:a555]{Y. Tang}, \hyperref[auth:a556]{R. Liu}, \hyperref[auth:a558]{Q. Sun} & Schedule control model for linear projects based on linear scheduling method and constraint programming & \hyperref[detail:Tang2014]{Details} No & \cite{Tang2014} & 2014 & Automation in Construction & null & \noindent{}\textbf{1.00} \textbf{1.00} n/a & 32 35 42 & 23 39 & 9 5 4\\
\index{TerekhovTDB14}\rowlabel{a:TerekhovTDB14}TerekhovTDB14 \href{https://doi.org/10.1613/jair.4278}{TerekhovTDB14} & \hyperref[auth:a818]{D. Terekhov}, \hyperref[auth:a799]{T. T. Tran}, \hyperref[auth:a803]{D. G. Down}, \hyperref[auth:a89]{J. C. Beck} & \cellcolor{gold!20}Integrating Queueing Theory and Scheduling for Dynamic Scheduling Problems & \hyperref[detail:TerekhovTDB14]{Details} \href{../works/TerekhovTDB14.pdf}{Yes} & \cite{TerekhovTDB14} & 2014 & J. Artif. Intell. Res. & 38 & \noindent{}\textcolor{black!50}{0.00} \textcolor{black!50}{0.00} \textbf{2.01} & 12 13 17 & 0 0 & 1 1 0\\
\index{ThiruvadyWGS14}\rowlabel{a:ThiruvadyWGS14}ThiruvadyWGS14 \href{https://doi.org/10.1007/s10732-014-9260-3}{ThiruvadyWGS14} & \hyperref[auth:a396]{D. R. Thiruvady}, \hyperref[auth:a117]{M. G. Wallace}, \hyperref[auth:a336]{H. Gu}, \hyperref[auth:a124]{A. Schutt} & \cellcolor{green!10}A Lagrangian relaxation and {ACO} hybrid for resource constrained project scheduling with discounted cash flows & \hyperref[detail:ThiruvadyWGS14]{Details} \href{../works/ThiruvadyWGS14.pdf}{Yes} & \cite{ThiruvadyWGS14} & 2014 & J. Heuristics & 34 & \noindent{}\textcolor{black!50}{0.00} \textcolor{black!50}{0.00} \textbf{1.79} & 19 20 19 & 18 24 & 4 1 3\\
\index{Velez2014}\rowlabel{a:Velez2014}Velez2014 \href{http://dx.doi.org/10.1146/annurev-chembioeng-060713-035859}{Velez2014} & \hyperref[auth:a1480]{S. Velez}, \hyperref[auth:a381]{C. T. Maravelias} & \cellcolor{gold!20}Advances in Mixed-Integer Programming Methods for Chemical Production Scheduling \hyperref[abs:Velez2014]{Abstract} & \hyperref[detail:Velez2014]{Details} No & \cite{Velez2014} & 2014 & Annual Review of Chemical and Biomolecular Engineering & null & \noindent{}\textcolor{black!50}{0.00} \textbf{1.50} n/a & 28 30 25 & 112 121 & 12 1 11\\
\index{Wang2014}\rowlabel{a:Wang2014}Wang2014 \href{http://dx.doi.org/10.1155/2014/271280}{Wang2014} & \hyperref[auth:a2022]{H. Wang}, \hyperref[auth:a2023]{X. Lu}, \hyperref[auth:a2024]{X. Zhang}, \hyperref[auth:a2025]{Q. Wang}, \hyperref[auth:a2026]{Y. Deng} & \cellcolor{gold!20}A Bio-Inspired Method for the Constrained Shortest Path Problem \hyperref[abs:Wang2014]{Abstract} & \hyperref[detail:Wang2014]{Details} No & \cite{Wang2014} & 2014 & The Scientific World Journal & null & \noindent{}\textcolor{black!50}{0.00} \textbf{1.00} n/a & 10 11 19 & 57 62 & 1 0 1\\
\index{ZhaoL14}\rowlabel{a:ZhaoL14}ZhaoL14 \href{http://dx.doi.org/10.1016/j.orhc.2014.05.003}{ZhaoL14} & \hyperref[auth:a1376]{Z. Zhao}, \hyperref[auth:a1377]{X. Li} & Scheduling elective surgeries with sequence-dependent setup times to multiple operating rooms using constraint programming & \hyperref[detail:ZhaoL14]{Details} \href{../works/ZhaoL14.pdf}{Yes} & \cite{ZhaoL14} & 2014 & Operations Research for Health Care & 8 & \noindent{}\textbf{1.00} \textbf{1.00} \textbf{7.14} & 40 40 50 & 23 34 & 6 5 1\\
\index{Ammar2013}\rowlabel{a:Ammar2013}Ammar2013 \href{http://dx.doi.org/10.1061/(asce)co.1943-7862.0000569}{Ammar2013} & \hyperref[auth:a1779]{M. A. Ammar} & LOB and CPM Integrated Method for Scheduling Repetitive Projects & \hyperref[detail:Ammar2013]{Details} No & \cite{Ammar2013} & 2013 & Journal of Construction Engineering and Management & null & \noindent{}\textcolor{black!50}{0.00} \textcolor{black!50}{0.00} n/a & 59 60 65 & 20 22 & 5 5 0\\
\index{ArtiguesLH13}\rowlabel{a:ArtiguesLH13}ArtiguesLH13 \href{http://dx.doi.org/10.1016/j.ijpe.2010.09.030}{ArtiguesLH13} & \hyperref[auth:a6]{C. Artigues}, \hyperref[auth:a3]{P. Lopez}, \hyperref[auth:a1162]{A. Haït} & \cellcolor{green!10}The energy scheduling problem: Industrial case-study and constraint propagation techniques & \hyperref[detail:ArtiguesLH13]{Details} \href{../works/ArtiguesLH13.pdf}{Yes} & \cite{ArtiguesLH13} & 2013 & International Journal of Production Economics & 11 & \noindent{}\textbf{1.50} \textbf{1.50} \textbf{5.79} & 76 82 83 & 16 25 & 11 6 5\\
\index{BajestaniB13}\rowlabel{a:BajestaniB13}BajestaniB13 \href{https://doi.org/10.1613/jair.3902}{BajestaniB13} & \hyperref[auth:a817]{M. A. Bajestani}, \hyperref[auth:a89]{J. C. Beck} & \cellcolor{gold!20}Scheduling a Dynamic Aircraft Repair Shop with Limited Repair Resources & \hyperref[detail:BajestaniB13]{Details} \href{../works/BajestaniB13.pdf}{Yes} & \cite{BajestaniB13} & 2013 & J. Artif. Intell. Res. & 36 & \noindent{}\textcolor{black!50}{0.00} \textcolor{black!50}{0.00} \textbf{33.71} & 14 15 20 & 0 0 & 4 4 0\\
\index{BegB13}\rowlabel{a:BegB13}BegB13 \href{http://doi.acm.org/10.1145/2512470}{BegB13} & \hyperref[auth:a609]{M. O. Beg}, \hyperref[auth:a610]{P. van Beek} & \cellcolor{gold!20}A constraint programming approach for integrated spatial and temporal scheduling for clustered architectures & \hyperref[detail:BegB13]{Details} \href{../works/BegB13.pdf}{Yes} & \cite{BegB13} & 2013 & {ACM} Trans. Embed. Comput. Syst. & 23 & \noindent{}\textbf{1.00} \textbf{1.00} \textbf{4.22} & 1 1 1 & 28 46 & 4 1 3\\
\index{Bocewicz2013}\rowlabel{a:Bocewicz2013}Bocewicz2013 \href{http://dx.doi.org/10.1155/2013/407096}{Bocewicz2013} & \hyperref[auth:a630]{G. Bocewicz}, \hyperref[auth:a1913]{R. Wójcik}, \hyperref[auth:a632]{Z. A. Banaszak}, \hyperref[auth:a1914]{P. Pawlewski} & \cellcolor{gold!20}Multimodal Processes Rescheduling: Cyclic Steady States Space Approach \hyperref[abs:Bocewicz2013]{Abstract} & \hyperref[detail:Bocewicz2013]{Details} No & \cite{Bocewicz2013} & 2013 & Mathematical Problems in Engineering & null & \noindent{}\textcolor{black!50}{0.00} \textbf{1.50} n/a & 13 14 20 & 18 24 & 1 1 0\\
\index{Clautiaux2013}\rowlabel{a:Clautiaux2013}Clautiaux2013 \href{http://dx.doi.org/10.1287/ijoc.1110.0478}{Clautiaux2013} & \hyperref[auth:a1686]{F. Clautiaux}, \hyperref[auth:a929]{A. Jouglet}, \hyperref[auth:a1170]{A. Moukrim} & A New Graph-Theoretical Model for the Guillotine-Cutting Problem \hyperref[abs:Clautiaux2013]{Abstract} & \hyperref[detail:Clautiaux2013]{Details} No & \cite{Clautiaux2013} & 2013 & \cellcolor{red!20}INFORMS Journal on Computing & null & \noindent{}\textcolor{black!50}{0.00} 0.50 n/a & 5 5 7 & 20 22 & 2 0 2\\
\index{HeinzSB13}\rowlabel{a:HeinzSB13}HeinzSB13 \href{https://doi.org/10.1007/s10601-012-9136-9}{HeinzSB13} & \hyperref[auth:a133]{S. Heinz}, \hyperref[auth:a134]{J. Schulz}, \hyperref[auth:a89]{J. C. Beck} & Using dual presolving reductions to reformulate cumulative constraints & \hyperref[detail:HeinzSB13]{Details} \href{../works/HeinzSB13.pdf}{Yes} & \cite{HeinzSB13} & 2013 & Constraints An Int. J. & 36 & \noindent{}\textcolor{black!50}{0.00} \textcolor{black!50}{0.00} \textbf{11.84} & 7 7 9 & 31 41 & 15 2 13\\
\index{Janosikova2013}\rowlabel{a:Janosikova2013}Janosikova2013 \href{http://dx.doi.org/10.26552/com.c.2013.1.39-43}{Janosikova2013} & \hyperref[auth:a2038]{L. Janosikova}, \hyperref[auth:a2039]{T. Hreben} & Mathematical Programming vs. Constraint Programming for Scheduling Problems & \hyperref[detail:Janosikova2013]{Details} No & \cite{Janosikova2013} & 2013 & Communications - Scientific letters of the University of Zilina & null & \noindent{}\textbf{1.00} \textbf{1.00} n/a & 0 0 0 & 4 7 & 3 0 3\\
\index{KameugneF13}\rowlabel{a:KameugneF13}KameugneF13 \href{http://dx.doi.org/10.1007/s13226-013-0005-z}{KameugneF13} & \hyperref[auth:a10]{R. Kameugne}, \hyperref[auth:a130]{L. P. Fotso} & A cumulative not-first/not-last filtering algorithm in O(n 2log(n)) & \hyperref[detail:KameugneF13]{Details} \href{../works/KameugneF13.pdf}{Yes} & \cite{KameugneF13} & 2013 & Indian Journal of Pure and Applied Mathematics & 21 & \noindent{}\textcolor{black!50}{0.00} \textcolor{black!50}{0.00} \textcolor{black!50}{0.00} & 6 8 8 & 4 19 & 10 6 4\\
\index{LombardiMB13}\rowlabel{a:LombardiMB13}LombardiMB13 \href{http://dx.doi.org/10.1109/tc.2011.203}{LombardiMB13} & \hyperref[auth:a142]{M. Lombardi}, \hyperref[auth:a143]{M. Milano}, \hyperref[auth:a245]{L. Benini} & Robust Scheduling of Task Graphs under Execution Time Uncertainty & \hyperref[detail:LombardiMB13]{Details} \href{../works/LombardiMB13.pdf}{Yes} & \cite{LombardiMB13} & 2013 & IEEE Transactions on Computers & 14 & \noindent{}\textcolor{black!50}{0.00} \textcolor{black!50}{0.00} \textbf{10.63} & 28 28 36 & 29 44 & 10 2 8\\
\index{Lorterapong2013}\rowlabel{a:Lorterapong2013}Lorterapong2013 \href{http://dx.doi.org/10.1061/(asce)co.1943-7862.0000582}{Lorterapong2013} & \hyperref[auth:a1792]{P. Lorterapong}, \hyperref[auth:a1793]{M. Ussavadilokrit} & Construction Scheduling Using the Constraint Satisfaction Problem Method & \hyperref[detail:Lorterapong2013]{Details} No & \cite{Lorterapong2013} & 2013 & Journal of Construction Engineering and Management & null & \noindent{}\textbf{1.00} \textbf{1.00} n/a & 14 15 16 & 18 25 & 3 0 3\\
\index{MenciaSV13}\rowlabel{a:MenciaSV13}MenciaSV13 \href{http://dx.doi.org/10.1007/s10845-012-0726-6}{MenciaSV13} & \hyperref[auth:a918]{C. Mencía}, \hyperref[auth:a919]{M. R. Sierra}, \hyperref[auth:a920]{R. Varela} & Intensified iterative deepening A* with application to job shop scheduling & \hyperref[detail:MenciaSV13]{Details} \href{../works/MenciaSV13.pdf}{Yes} & \cite{MenciaSV13} & 2013 & Journal of Intelligent Manufacturing & 11 & \noindent{}\textcolor{black!50}{0.00} \textcolor{black!50}{0.00} \textbf{6.60} & 9 9 12 & 43 55 & 10 0 10\\
\index{MeskensDL13}\rowlabel{a:MeskensDL13}MeskensDL13 \href{http://dx.doi.org/10.1016/j.dss.2012.10.019}{MeskensDL13} & \hyperref[auth:a597]{N. Meskens}, \hyperref[auth:a598]{D. Duvivier}, \hyperref[auth:a1460]{A. Lianset} & Multi-objective operating room scheduling considering desiderata of the surgical team \hyperref[abs:MeskensDL13]{Abstract} & \hyperref[detail:MeskensDL13]{Details} \href{../works/MeskensDL13.pdf}{Yes} & \cite{MeskensDL13} & 2013 & DECISION SUPPORT SYSTEMS & 10 & \noindent{}\textcolor{black!50}{0.00} \textbf{1.00} \textbf{1.50} & 102 102 116 & 31 39 & 5 5 0\\
\index{Nowatzki2013}\rowlabel{a:Nowatzki2013}Nowatzki2013 \href{http://dx.doi.org/10.1145/2499370.2462163}{Nowatzki2013} & \hyperref[auth:a1631]{T. Nowatzki}, \hyperref[auth:a1632]{M. Sartin-Tarm}, \hyperref[auth:a1633]{L. D. Carli}, \hyperref[auth:a1634]{K. Sankaralingam}, \hyperref[auth:a1635]{C. Estan}, \hyperref[auth:a1636]{B. Robatmili} & A general constraint-centric scheduling framework for spatial architectures \hyperref[abs:Nowatzki2013]{Abstract} & \hyperref[detail:Nowatzki2013]{Details} No & \cite{Nowatzki2013} & 2013 & ACM SIGPLAN Notices & null & \noindent{}\textcolor{black!50}{0.00} \textbf{1.50} n/a & 22 21 35 & 36 51 & 3 1 2\\
\index{Ortiz-Bayliss2013}\rowlabel{a:Ortiz-Bayliss2013}Ortiz-Bayliss2013 \href{http://dx.doi.org/10.1016/j.patrec.2012.09.009}{Ortiz-Bayliss2013} & \hyperref[auth:a1781]{J. C. Ortiz-Bayliss}, \hyperref[auth:a1608]{H. Terashima-Marín}, \hyperref[auth:a1782]{S. E. Conant-Pablos} & Learning vector quantization for variable ordering in constraint satisfaction problems & \hyperref[detail:Ortiz-Bayliss2013]{Details} No & \cite{Ortiz-Bayliss2013} & 2013 & Pattern Recognition Letters & null & \noindent{}0.50 0.50 n/a & 22 22 26 & 15 50 & 6 4 2\\
\index{OzturkTHO13}\rowlabel{a:OzturkTHO13}OzturkTHO13 \href{https://doi.org/10.1007/s10601-013-9142-6}{OzturkTHO13} & \hyperref[auth:a135]{C. {\"{O}}zt{\"{u}}rk}, \hyperref[auth:a136]{S. Tunali}, \hyperref[auth:a137]{B. Hnich}, \hyperref[auth:a138]{A. {\"{O}}rnek} & Balancing and scheduling of flexible mixed model assembly lines & \hyperref[detail:OzturkTHO13]{Details} \href{../works/OzturkTHO13.pdf}{Yes} & \cite{OzturkTHO13} & 2013 & Constraints An Int. J. & 36 & \noindent{}\textcolor{black!50}{0.00} \textcolor{black!50}{0.00} \textbf{54.08} & 31 31 34 & 44 62 & 16 6 10\\
\index{Pessoa2013}\rowlabel{a:Pessoa2013}Pessoa2013 \href{http://dx.doi.org/10.3182/20130522-3-br-4036.00069}{Pessoa2013} & \hyperref[auth:a1669]{M. A. O. Pessoa}, \hyperref[auth:a1670]{R. A. E. Montesco}, \hyperref[auth:a1671]{F. Junqueira}, \hyperref[auth:a1672]{Diolino Jose dos Santos Filho}, \hyperref[auth:a1673]{P. E. Miyagi} & \cellcolor{gold!20}Advanced Planning and Scheduling Systems based on Time Windows and Constraint Programming & \hyperref[detail:Pessoa2013]{Details} No & \cite{Pessoa2013} & 2013 & IFAC Proceedings Volumes & null & \noindent{}\textbf{1.00} \textbf{1.00} n/a & 4 4 4 & 13 21 & 2 1 1\\
\index{SchuttFSW13}\rowlabel{a:SchuttFSW13}SchuttFSW13 \href{https://doi.org/10.1007/s10951-012-0285-x}{SchuttFSW13} & \hyperref[auth:a124]{A. Schutt}, \hyperref[auth:a154]{T. Feydy}, \hyperref[auth:a125]{P. J. Stuckey}, \hyperref[auth:a117]{M. G. Wallace} & \cellcolor{green!10}Solving RCPSP/max by lazy clause generation & \hyperref[detail:SchuttFSW13]{Details} \href{../works/SchuttFSW13.pdf}{Yes} & \cite{SchuttFSW13} & 2013 & Journal of Scheduling & 17 & \noindent{}\textcolor{black!50}{0.00} \textcolor{black!50}{0.00} \textbf{5.21} & 43 45 57 & 23 38 & 36 25 11\\
\index{Shobaki2013}\rowlabel{a:Shobaki2013}Shobaki2013 \href{http://dx.doi.org/10.1145/2512432}{Shobaki2013} & \hyperref[auth:a1784]{G. Shobaki}, \hyperref[auth:a1785]{M. Shawabkeh}, \hyperref[auth:a1786]{N. E. A. Rmaileh} & \cellcolor{gold!20}Preallocation instruction scheduling with register pressure minimization using a combinatorial optimization approach \hyperref[abs:Shobaki2013]{Abstract} & \hyperref[detail:Shobaki2013]{Details} No & \cite{Shobaki2013} & 2013 & ACM Transactions on Architecture and Code Optimization & null & \noindent{}\textcolor{black!50}{0.00} \textcolor{black!50}{0.00} n/a & 16 18 21 & 11 19 & 2 2 0\\
\index{SuCC13}\rowlabel{a:SuCC13}SuCC13 \href{http://dx.doi.org/10.1016/j.cie.2013.02.021}{SuCC13} & \hyperref[auth:a1400]{L.-H. Su}, \hyperref[auth:a1401]{Y. Chiu}, \hyperref[auth:a1402]{T. C. E. Cheng} & Sports tournament scheduling to determine the required number of venues subject to the minimum timeslots under given formats \hyperref[abs:SuCC13]{Abstract} & \hyperref[detail:SuCC13]{Details} \href{../works/SuCC13.pdf}{Yes} & \cite{SuCC13} & 2013 & Computers \  Industrial Engineering & 7 & \noindent{}\textcolor{black!50}{0.00} \textbf{1.00} 0.27 & 2 2 4 & 15 16 & 4 0 4\\
\index{Talbi2013}\rowlabel{a:Talbi2013}Talbi2013 \href{http://dx.doi.org/10.1007/s10288-013-0242-3}{Talbi2013} & \hyperref[auth:a1659]{E.-G. Talbi} & Combining metaheuristics with mathematical programming, constraint programming and machine learning & \hyperref[detail:Talbi2013]{Details} No & \cite{Talbi2013} & 2013 & 4OR & null & \noindent{}0.50 0.50 n/a & 15 15 22 & 90 150 & 9 1 8\\
\index{UnsalO13}\rowlabel{a:UnsalO13}UnsalO13 \href{http://dx.doi.org/10.1016/j.tre.2013.08.006}{UnsalO13} & \hyperref[auth:a1217]{O. Unsal}, \hyperref[auth:a347]{C. Oguz} & Constraint programming approach to quay crane scheduling problem & \hyperref[detail:UnsalO13]{Details} \href{../works/UnsalO13.pdf}{Yes} & \cite{UnsalO13} & 2013 & Transportation Research Part E: Logistics and Transportation Review & 15 & \noindent{}\textbf{1.00} \textbf{1.00} \textbf{15.56} & 44 45 54 & 25 34 & 10 6 4\\
\index{Velez2013}\rowlabel{a:Velez2013}Velez2013 \href{http://dx.doi.org/10.1002/aic.14021}{Velez2013} & \hyperref[auth:a1480]{S. Velez}, \hyperref[auth:a1481]{A. Sundaramoorthy}, \hyperref[auth:a381]{C. T. Maravelias} & Valid Inequalities Based on Demand Propagation for Chemical Production Scheduling MIP Models \hyperref[abs:Velez2013]{Abstract} & \hyperref[detail:Velez2013]{Details} No & \cite{Velez2013} & 2013 & AIChE Journal & null & \noindent{}0.50 \textbf{3.75} n/a & 38 39 39 & 43 50 & 7 4 3\\
\index{Wang2013}\rowlabel{a:Wang2013}Wang2013 \href{http://dx.doi.org/10.4028/www.scientific.net/amm.357-360.2720}{Wang2013} & \hyperref[auth:a1903]{H. S. Wang}, \hyperref[auth:a1904]{S. S. Liu} & Road Inspection Scheduling Model Using Constraint Programming \hyperref[abs:Wang2013]{Abstract} & \hyperref[detail:Wang2013]{Details} No & \cite{Wang2013} & 2013 & Applied Mechanics and Materials & null & \noindent{}\textbf{1.00} \textbf{3.00} n/a & 0 0 0 & 6 7 & 2 0 2\\
\index{Zhang2013}\rowlabel{a:Zhang2013}Zhang2013 \href{http://dx.doi.org/10.5772/55956}{Zhang2013} & \hyperref[auth:a1517]{R. Zhang} & \cellcolor{gold!20}A Simulated Annealing-Based Heuristic Algorithm for Job Shop Scheduling to Minimize Lateness \hyperref[abs:Zhang2013]{Abstract} & \hyperref[detail:Zhang2013]{Details} No & \cite{Zhang2013} & 2013 & International Journal of Advanced Robotic Systems & null & \noindent{}\textcolor{black!50}{0.00} \textbf{3.00} n/a & 6 8 13 & 26 27 & 4 1 3\\
\index{Zoulfaghari2013}\rowlabel{a:Zoulfaghari2013}Zoulfaghari2013 \href{http://dx.doi.org/10.4018/jaec.2013040103}{Zoulfaghari2013} & \hyperref[auth:a1758]{H. Zoulfaghari}, \hyperref[auth:a1759]{J. Nematian}, \hyperref[auth:a1760]{N. Mahmoudi}, \hyperref[auth:a1761]{M. Khodabandeh} & A New Genetic Algorithm for the RCPSP in Large Scale \hyperref[abs:Zoulfaghari2013]{Abstract} & \hyperref[detail:Zoulfaghari2013]{Details} No & \cite{Zoulfaghari2013} & 2013 & International Journal of Applied Evolutionary Computation & null & \noindent{}\textcolor{black!50}{0.00} \textcolor{black!50}{0.00} n/a & 5 5 0 & 38 45 & 5 0 5\\
\index{Berbeglia2012}\rowlabel{a:Berbeglia2012}Berbeglia2012 \href{http://dx.doi.org/10.1287/ijoc.1110.0454}{Berbeglia2012} & \hyperref[auth:a1847]{G. Berbeglia}, \hyperref[auth:a1848]{J.-F. Cordeau}, \hyperref[auth:a1074]{G. Laporte} & A Hybrid Tabu Search and Constraint Programming Algorithm for the Dynamic Dial-a-Ride Problem \hyperref[abs:Berbeglia2012]{Abstract} & \hyperref[detail:Berbeglia2012]{Details} No & \cite{Berbeglia2012} & 2012 & \cellcolor{red!20}INFORMS Journal on Computing & null & \noindent{}\textcolor{black!50}{0.00} \textbf{1.00} n/a & 65 68 78 & 24 26 & 1 1 0\\
\index{Eirinakis2012}\rowlabel{a:Eirinakis2012}Eirinakis2012 \href{http://dx.doi.org/10.1287/ijoc.1110.0449}{Eirinakis2012} & \hyperref[auth:a1916]{P. Eirinakis}, \hyperref[auth:a1917]{D. Magos}, \hyperref[auth:a1918]{I. Mourtos}, \hyperref[auth:a1919]{P. Miliotis} & Finding All Stable Pairs and Solutions to the Many-to-Many Stable Matching Problem \hyperref[abs:Eirinakis2012]{Abstract} & \hyperref[detail:Eirinakis2012]{Details} No & \cite{Eirinakis2012} & 2012 & \cellcolor{red!20}INFORMS Journal on Computing & null & \noindent{}\textcolor{black!50}{0.00} \textbf{2.00} n/a & 11 11 13 & 25 30 & 1 0 1\\
\index{Filho2012}\rowlabel{a:Filho2012}Filho2012 \href{http://dx.doi.org/10.1016/j.eswa.2011.07.027}{Filho2012} & \hyperref[auth:a1949]{Cicero Ferreira Fernandes Costa Filho}, \hyperref[auth:a1950]{D. A. R. Rocha}, \hyperref[auth:a1951]{M. G. F. Costa}, \hyperref[auth:a1952]{Wagner Coelho de Albuquerque Pereira} & Using Constraint Satisfaction Problem approach to solve human resource allocation problems in cooperative health services & \hyperref[detail:Filho2012]{Details} No & \cite{Filho2012} & 2012 & Expert Systems with Applications & null & \noindent{}0.50 0.50 n/a & 17 19 24 & 6 10 & 1 0 1\\
\index{GuyonLPR12}\rowlabel{a:GuyonLPR12}GuyonLPR12 \href{http://dx.doi.org/10.1007/s10479-012-1132-3}{GuyonLPR12} & \hyperref[auth:a977]{O. Guyon}, \hyperref[auth:a978]{P. Lemaire}, \hyperref[auth:a846]{E. Pinson}, \hyperref[auth:a979]{D. Rivreau} & \cellcolor{green!10}Solving an integrated job-shop problem with human resource constraints & \hyperref[detail:GuyonLPR12]{Details} \href{../works/GuyonLPR12.pdf}{Yes} & \cite{GuyonLPR12} & 2012 & Annals of Operations Research & 25 & \noindent{}\textcolor{black!50}{0.00} \textcolor{black!50}{0.00} \textbf{1.65} & 32 33 40 & 25 38 & 14 6 8\\
\index{HeinzSSW12}\rowlabel{a:HeinzSSW12}HeinzSSW12 \href{https://doi.org/10.1007/s10601-011-9113-8}{HeinzSSW12} & \hyperref[auth:a133]{S. Heinz}, \hyperref[auth:a139]{T. Schlechte}, \hyperref[auth:a140]{R. Stephan}, \hyperref[auth:a141]{M. Winkler} & Solving steel mill slab design problems & \hyperref[detail:HeinzSSW12]{Details} \href{../works/HeinzSSW12.pdf}{Yes} & \cite{HeinzSSW12} & 2012 & Constraints An Int. J. & 12 & \noindent{}\textcolor{black!50}{0.00} \textcolor{black!50}{0.00} \textcolor{black!50}{0.14} & 10 11 12 & 9 16 & 4 1 3\\
\index{Hoc2012}\rowlabel{a:Hoc2012}Hoc2012 \href{http://dx.doi.org/10.1002/hfm.20359}{Hoc2012} & \hyperref[auth:a2009]{J. Hoc}, \hyperref[auth:a2010]{C. Guerin}, \hyperref[auth:a2011]{N. Mebarki} & The Nature of Expertise in Scheduling: The Case of Timetabling \hyperref[abs:Hoc2012]{Abstract} & \hyperref[detail:Hoc2012]{Details} No & \cite{Hoc2012} & 2012 & Human Factors and Ergonomics in Manufacturing \  Service Industries & null & \noindent{}\textcolor{black!50}{0.00} \textbf{1.50} n/a & 7 7 8 & 29 45 & 1 0 1\\
\index{Junker2012}\rowlabel{a:Junker2012}Junker2012 \href{http://dx.doi.org/10.1017/s0269888912000240}{Junker2012} & \hyperref[auth:a1326]{U. Junker} & Air traffic flow management with heuristic repair \hyperref[abs:Junker2012]{Abstract} & \hyperref[detail:Junker2012]{Details} No & \cite{Junker2012} & 2012 & The Knowledge Engineering Review & null & \noindent{}\textcolor{black!50}{0.00} 0.50 n/a & 3 3 4 & 5 12 & 1 0 1\\
\index{Kelareva2012}\rowlabel{a:Kelareva2012}Kelareva2012 \href{http://dx.doi.org/10.1609/icaps.v22i1.13494}{Kelareva2012} & \hyperref[auth:a332]{E. Kelareva}, \hyperref[auth:a855]{S. Brand}, \hyperref[auth:a334]{P. Kilby}, \hyperref[auth:a1518]{S. Thiebaux}, \hyperref[auth:a1519]{M. Wallace} & CP and MIP Methods for Ship Scheduling with Time-Varying Draft \hyperref[abs:Kelareva2012]{Abstract} & \hyperref[detail:Kelareva2012]{Details} No & \cite{Kelareva2012} & 2012 & Proceedings of the International Conference on Automated Planning and Scheduling & null & \noindent{}\textcolor{black!50}{0.00} \textbf{3.00} n/a & 11 14 0 & 0 0 & 5 5 0\\
\index{LimtanyakulS12}\rowlabel{a:LimtanyakulS12}LimtanyakulS12 \href{https://doi.org/10.1007/s10601-012-9118-y}{LimtanyakulS12} & \hyperref[auth:a144]{K. Limtanyakul}, \hyperref[auth:a145]{U. Schwiegelshohn} & Improvements of constraint programming and hybrid methods for scheduling of tests on vehicle prototypes & \hyperref[detail:LimtanyakulS12]{Details} \href{../works/LimtanyakulS12.pdf}{Yes} & \cite{LimtanyakulS12} & 2012 & Constraints An Int. J. & 32 & \noindent{}\textbf{1.00} \textbf{1.00} \textbf{25.90} & 4 4 5 & 16 27 & 6 1 5\\
\index{LombardiM12}\rowlabel{a:LombardiM12}LombardiM12 \href{https://doi.org/10.1007/s10601-011-9115-6}{LombardiM12} & \hyperref[auth:a142]{M. Lombardi}, \hyperref[auth:a143]{M. Milano} & Optimal methods for resource allocation and scheduling: a cross-disciplinary survey & \hyperref[detail:LombardiM12]{Details} \href{../works/LombardiM12.pdf}{Yes} & \cite{LombardiM12} & 2012 & Constraints An Int. J. & 35 & \noindent{}\textcolor{black!50}{0.00} \textcolor{black!50}{0.00} \textbf{43.11} & 39 39 47 & 68 94 & 41 5 36\\
\index{LombardiM12a}\rowlabel{a:LombardiM12a}LombardiM12a \href{https://doi.org/10.1016/j.artint.2011.12.001}{LombardiM12a} & \hyperref[auth:a142]{M. Lombardi}, \hyperref[auth:a143]{M. Milano} & \cellcolor{gold!20}A min-flow algorithm for Minimal Critical Set detection in Resource Constrained Project Scheduling & \hyperref[detail:LombardiM12a]{Details} \href{../works/LombardiM12a.pdf}{Yes} & \cite{LombardiM12a} & 2012 & Artificial Intelligence & 10 & \noindent{}\textcolor{black!50}{0.00} \textcolor{black!50}{0.00} 0.37 & 3 3 15 & 13 21 & 8 1 7\\
\index{MalapertCGJLR12}\rowlabel{a:MalapertCGJLR12}MalapertCGJLR12 \href{https://doi.org/10.1287/ijoc.1100.0446}{MalapertCGJLR12} & \hyperref[auth:a82]{A. Malapert}, \hyperref[auth:a998]{H. Cambazard}, \hyperref[auth:a293]{C. Gu{\'{e}}ret}, \hyperref[auth:a247]{N. Jussien}, \hyperref[auth:a645]{A. Langevin}, \hyperref[auth:a326]{L.-M. Rousseau} & \cellcolor{green!10}An Optimal Constraint Programming Approach to the Open-Shop Problem & \hyperref[detail:MalapertCGJLR12]{Details} \href{../works/MalapertCGJLR12.pdf}{Yes} & \cite{MalapertCGJLR12} & 2012 & \cellcolor{red!20}INFORMS Journal on Computing & 17 & \noindent{}\textcolor{black!50}{0.00} \textcolor{black!50}{0.00} \textbf{12.50} & 23 24 25 & 21 31 & 20 15 5\\
\index{MalapertGR12}\rowlabel{a:MalapertGR12}MalapertGR12 \href{http://dx.doi.org/10.1016/j.ejor.2012.04.008}{MalapertGR12} & \hyperref[auth:a82]{A. Malapert}, \hyperref[auth:a1375]{C. Guéret}, \hyperref[auth:a326]{L.-M. Rousseau} & \cellcolor{green!10}A constraint programming approach for a batch processing problem with non-identical job sizes & \hyperref[detail:MalapertGR12]{Details} \href{../works/MalapertGR12.pdf}{Yes} & \cite{MalapertGR12} & 2012 & European Journal of Operational Research & 13 & \noindent{}\textbf{1.00} \textbf{1.00} \textbf{10.74} & 43 44 50 & 24 41 & 7 7 0\\
\index{Martin2012}\rowlabel{a:Martin2012}Martin2012 \href{http://dx.doi.org/10.1145/2209285.2209289}{Martin2012} & \hyperref[auth:a1578]{K. Martin}, \hyperref[auth:a659]{C. Wolinski}, \hyperref[auth:a660]{K. Kuchcinski}, \hyperref[auth:a1579]{A. Floch}, \hyperref[auth:a1532]{F. Charot} & Constraint Programming Approach to Reconfigurable Processor Extension Generation and Application Compilation \hyperref[abs:Martin2012]{Abstract} & \hyperref[detail:Martin2012]{Details} No & \cite{Martin2012} & 2012 & ACM Transactions on Reconfigurable Technology and Systems & null & \noindent{}\textcolor{black!50}{0.00} \textbf{2.00} n/a & 15 16 22 & 30 47 & 4 2 2\\
\index{MenciaSV12}\rowlabel{a:MenciaSV12}MenciaSV12 \href{http://dx.doi.org/10.1007/s10479-012-1296-x}{MenciaSV12} & \hyperref[auth:a918]{C. Mencía}, \hyperref[auth:a919]{M. R. Sierra}, \hyperref[auth:a920]{R. Varela} & Depth-first heuristic search for the job shop scheduling problem & \hyperref[detail:MenciaSV12]{Details} \href{../works/MenciaSV12.pdf}{Yes} & \cite{MenciaSV12} & 2012 & Annals of Operations Research & 32 & \noindent{}\textcolor{black!50}{0.00} \textcolor{black!50}{0.00} \textbf{3.53} & 16 17 18 & 40 57 & 13 0 13\\
\index{NovasH12}\rowlabel{a:NovasH12}NovasH12 \href{https://doi.org/10.1016/j.compchemeng.2012.01.005}{NovasH12} & \hyperref[auth:a524]{J. M. Novas}, \hyperref[auth:a588]{G. P. Henning} & A comprehensive constraint programming approach for the rolling horizon-based scheduling of automated wet-etch stations & \hyperref[detail:NovasH12]{Details} \href{../works/NovasH12.pdf}{Yes} & \cite{NovasH12} & 2012 & Computers \  Chemical Engineering & 17 & \noindent{}\textbf{1.00} \textbf{1.00} \textbf{11.52} & 17 17 22 & 15 23 & 5 3 2\\
\index{OzturkTHO12}\rowlabel{a:OzturkTHO12}OzturkTHO12 \href{https://www.sciencedirect.com/science/article/pii/S1474667016331858}{OzturkTHO12} & \hyperref[auth:a1015]{C. {\"{O}}zt{\"{u}}rk}, \hyperref[auth:a1016]{S. Tunalı}, \hyperref[auth:a137]{B. Hnich}, \hyperref[auth:a138]{A. {\"{O}}rnek} & A Constraint Programming Model for Balancing and Scheduling of Flexible Mixed Model Assembly Lines with Parallel Stations & \hyperref[detail:OzturkTHO12]{Details} \href{../works/OzturkTHO12.pdf}{Yes} & \cite{OzturkTHO12} & 2012 & IFAC Proceedings Volumes & 6 & \noindent{}\textbf{1.00} \textbf{1.00} \textbf{7.24} & 5 4 5 & 5 10 & 7 5 2\\
\index{Pesant2012}\rowlabel{a:Pesant2012}Pesant2012 \href{http://dx.doi.org/10.1613/jair.3463}{Pesant2012} & \hyperref[auth:a1586]{G. Pesant}, \hyperref[auth:a1587]{C. Quimper}, \hyperref[auth:a1588]{A. Zanarini} & \cellcolor{gold!20}Counting-Based Search: Branching Heuristics for Constraint Satisfaction Problems \hyperref[abs:Pesant2012]{Abstract} & \hyperref[detail:Pesant2012]{Details} No & \cite{Pesant2012} & 2012 & Journal of Artificial Intelligence Research & null & \noindent{}\textcolor{black!50}{0.00} \textbf{2.00} n/a & 32 32 51 & 0 0 & 6 6 0\\
\index{Pinto2012}\rowlabel{a:Pinto2012}Pinto2012 \href{http://dx.doi.org/10.1007/s10479-012-1256-5}{Pinto2012} & \hyperref[auth:a1598]{G. Pinto}, \hyperref[auth:a1599]{Y. T. Ben-Dov}, \hyperref[auth:a1600]{G. Rabinowitz} & Formulating and solving a multi-mode resource-collaboration and constrained scheduling problem (MRCCSP) & \hyperref[detail:Pinto2012]{Details} No & \cite{Pinto2012} & 2012 & Annals of Operations Research & null & \noindent{}\textcolor{black!50}{0.00} \textcolor{black!50}{0.00} n/a & 5 6 8 & 41 50 & 6 1 5\\
\index{Raffin2012}\rowlabel{a:Raffin2012}Raffin2012 \href{http://dx.doi.org/10.4018/jertcs.2012010101}{Raffin2012} & \hyperref[auth:a1531]{E. Raffin}, \hyperref[auth:a659]{C. Wolinski}, \hyperref[auth:a1532]{F. Charot}, \hyperref[auth:a1533]{E. Casseau}, \hyperref[auth:a1534]{A. Floc’h}, \hyperref[auth:a660]{K. Kuchcinski}, \hyperref[auth:a1535]{S. Chevobbe}, \hyperref[auth:a1536]{S. Guyetant} & \cellcolor{green!10}Scheduling, Binding and Routing System for a Run-Time Reconfigurable Operator Based Multimedia Architecture \hyperref[abs:Raffin2012]{Abstract} & \hyperref[detail:Raffin2012]{Details} No & \cite{Raffin2012} & 2012 & International Journal of Embedded and Real-Time Communication Systems & null & \noindent{}\textcolor{black!50}{0.00} \textbf{2.50} n/a & 0 0 0 & 25 33 & 2 0 2\\
\index{Ribeiro12}\rowlabel{a:Ribeiro12}Ribeiro12 \href{http://dx.doi.org/10.1111/j.1475-3995.2011.00819.x}{Ribeiro12} & \hyperref[auth:a1386]{C. C. Ribeiro} & Sports scheduling: Problems and applications \hyperref[abs:Ribeiro12]{Abstract} & \hyperref[detail:Ribeiro12]{Details} \href{../works/Ribeiro12.pdf}{Yes} & \cite{Ribeiro12} & 2012 & INTERNATIONAL TRANSACTIONS IN OPERATIONAL RESEARCH & 26 & \noindent{}\textcolor{black!50}{0.00} \textbf{1.00} \textbf{1.64} & 47 52 54 & 59 92 & 9 1 8\\
\index{ShangGuan2012}\rowlabel{a:ShangGuan2012}ShangGuan2012 \href{http://dx.doi.org/10.4028/www.scientific.net/amr.443-444.724}{ShangGuan2012} & \hyperref[auth:a1983]{C. X. ShangGuan}, \hyperref[auth:a1984]{J. T. Li}, \hyperref[auth:a1985]{R. F. Shi} & Rescheduling of Parallel Machines under Machine Failures \hyperref[abs:ShangGuan2012]{Abstract} & \hyperref[detail:ShangGuan2012]{Details} No & \cite{ShangGuan2012} & 2012 & Advanced Materials Research & null & \noindent{}\textcolor{black!50}{0.00} \textbf{2.50} n/a & 5 5 6 & 7 15 & 1 0 1\\
\index{TerekhovDOB12}\rowlabel{a:TerekhovDOB12}TerekhovDOB12 \href{https://doi.org/10.1016/j.cie.2012.02.006}{TerekhovDOB12} & \hyperref[auth:a818]{D. Terekhov}, \hyperref[auth:a820]{M. K. Dogru}, \hyperref[auth:a821]{U. {\"{O}}zen}, \hyperref[auth:a89]{J. C. Beck} & Solving two-machine assembly scheduling problems with inventory constraints & \hyperref[detail:TerekhovDOB12]{Details} \href{../works/TerekhovDOB12.pdf}{Yes} & \cite{TerekhovDOB12} & 2012 & Computers \  Industrial Engineering & 15 & \noindent{}\textcolor{black!50}{0.00} \textcolor{black!50}{0.00} \textbf{9.01} & 8 9 16 & 48 59 & 9 2 7\\
\index{TopalogluSS12}\rowlabel{a:TopalogluSS12}TopalogluSS12 \href{http://dx.doi.org/10.1016/j.eswa.2011.09.038}{TopalogluSS12} & \hyperref[auth:a617]{S. Topaloglu}, \hyperref[auth:a1378]{L. Salum}, \hyperref[auth:a1379]{A. A. Supciller} & Rule-based modeling and constraint programming based solution of the assembly line balancing problem & \hyperref[detail:TopalogluSS12]{Details} \href{../works/TopalogluSS12.pdf}{Yes} & \cite{TopalogluSS12} & 2012 & Expert Systems with Applications & 10 & \noindent{}\textcolor{black!50}{0.00} \textcolor{black!50}{0.00} \textbf{6.33} & 31 32 38 & 34 43 & 17 15 2\\
\index{ZarandiB12}\rowlabel{a:ZarandiB12}ZarandiB12 \href{http://dx.doi.org/10.1287/ijoc.1110.0458}{ZarandiB12} & \hyperref[auth:a945]{M. M. Fazel-Zarandi}, \hyperref[auth:a89]{J. C. Beck} & Using Logic-Based Benders Decomposition to Solve the Capacity- and Distance-Constrained Plant Location Problem & \hyperref[detail:ZarandiB12]{Details} No & \cite{ZarandiB12} & 2012 & \cellcolor{red!20}INFORMS Journal on Computing & 12 & \noindent{}\textcolor{black!50}{0.00} \textcolor{black!50}{0.00} n/a & 38 38 42 & 57 61 & 28 18 10\\
\index{ZengM12}\rowlabel{a:ZengM12}ZengM12 \href{http://dx.doi.org/10.1016/j.cor.2011.10.004}{ZengM12} & \hyperref[auth:a1404]{L. Zeng}, \hyperref[auth:a1405]{S. Mizuno} & On the separation in 2-period double round robin tournaments with minimum breaks \hyperref[abs:ZengM12]{Abstract} & \hyperref[detail:ZengM12]{Details} \href{../works/ZengM12.pdf}{Yes} & \cite{ZengM12} & 2012 & Computers \  Operations Research & 9 & \noindent{}\textcolor{black!50}{0.00} \textbf{1.00} \textbf{2.60} & 3 3 4 & 18 25 & 5 0 5\\
\index{Zou2012}\rowlabel{a:Zou2012}Zou2012 \href{http://dx.doi.org/10.14778/2535568.2448945}{Zou2012} & \hyperref[auth:a2054]{T. Zou}, \hyperref[auth:a2055]{R. L. Bras}, \hyperref[auth:a2056]{M. V. Salles}, \hyperref[auth:a2057]{A. Demers}, \hyperref[auth:a2058]{J. Gehrke} & ClouDiA \hyperref[abs:Zou2012]{Abstract} & \hyperref[detail:Zou2012]{Details} No & \cite{Zou2012} & 2012 & Proceedings of the VLDB Endowment & null & \noindent{}\textcolor{black!50}{0.00} \textbf{1.00} n/a & 17 17 14 & 26 69 & 1 0 1\\
\index{Acuna-Agost2011}\rowlabel{a:Acuna-Agost2011}Acuna-Agost2011 \href{http://dx.doi.org/10.1016/j.ejor.2011.05.047}{Acuna-Agost2011} & \hyperref[auth:a354]{R. Acuna-Agost}, \hyperref[auth:a355]{P. Michelon}, \hyperref[auth:a356]{D. Feillet}, \hyperref[auth:a357]{S. Gueye} & SAPI: Statistical Analysis of Propagation of Incidents. A new approach for rescheduling trains after disruptions & \hyperref[detail:Acuna-Agost2011]{Details} No & \cite{Acuna-Agost2011} & 2011 & European Journal of Operational Research & null & \noindent{}0.50 0.50 n/a & 36 37 45 & 26 52 & 3 0 3\\
\index{Artigues2011}\rowlabel{a:Artigues2011}Artigues2011 \href{http://dx.doi.org/10.1016/j.engappai.2010.07.008}{Artigues2011} & \hyperref[auth:a6]{C. Artigues}, \hyperref[auth:a1199]{M.-J. Huguet}, \hyperref[auth:a3]{P. Lopez} & \cellcolor{green!10}Generalized disjunctive constraint propagation for solving the job shop problem with time lags & \hyperref[detail:Artigues2011]{Details} No & \cite{Artigues2011} & 2011 & Engineering Applications of Artificial Intelligence & null & \noindent{}\textbf{1.50} \textbf{1.50} n/a & 22 22 28 & 16 25 & 7 2 5\\
\index{BandaSC11}\rowlabel{a:BandaSC11}BandaSC11 \href{https://doi.org/10.1287/ijoc.1090.0378}{BandaSC11} & \hyperref[auth:a796]{Maria Garcia de la Banda}, \hyperref[auth:a125]{P. J. Stuckey}, \hyperref[auth:a343]{G. Chu} & Solving Talent Scheduling with Dynamic Programming & \hyperref[detail:BandaSC11]{Details} \href{../works/BandaSC11.pdf}{Yes} & \cite{BandaSC11} & 2011 & \cellcolor{red!20}INFORMS Journal on Computing & 18 & \noindent{}\textcolor{black!50}{0.00} \textcolor{black!50}{0.00} 0.79 & 24 25 28 & 17 18 & 1 1 0\\
\index{BartakS11}\rowlabel{a:BartakS11}BartakS11 \href{https://doi.org/10.1007/s10601-011-9109-4}{BartakS11} & \hyperref[auth:a152]{R. Bart{\'{a}}k}, \hyperref[auth:a153]{M. A. Salido} & \cellcolor{green!10}Constraint satisfaction for planning and scheduling problems & \hyperref[detail:BartakS11]{Details} \href{../works/BartakS11.pdf}{Yes} & \cite{BartakS11} & 2011 & Constraints An Int. J. & 5 & \noindent{}\textbf{1.00} \textbf{1.00} \textbf{1.58} & 17 18 21 & 3 7 & 3 2 1\\
\index{BeckFW11}\rowlabel{a:BeckFW11}BeckFW11 \href{https://doi.org/10.1287/ijoc.1100.0388}{BeckFW11} & \hyperref[auth:a89]{J. C. Beck}, \hyperref[auth:a822]{T. K. Feng}, \hyperref[auth:a360]{J.-P. Watson} & Combining Constraint Programming and Local Search for Job-Shop Scheduling & \hyperref[detail:BeckFW11]{Details} \href{../works/BeckFW11.pdf}{Yes} & \cite{BeckFW11} & 2011 & \cellcolor{red!20}INFORMS Journal on Computing & 14 & \noindent{}\textbf{2.00} \textbf{2.00} \textbf{4.48} & 43 46 59 & 23 33 & 19 9 10\\
\index{BeldiceanuCDP11}\rowlabel{a:BeldiceanuCDP11}BeldiceanuCDP11 \href{https://doi.org/10.1007/s10479-010-0731-0}{BeldiceanuCDP11} & \hyperref[auth:a128]{N. Beldiceanu}, \hyperref[auth:a91]{M. Carlsson}, \hyperref[auth:a243]{S. Demassey}, \hyperref[auth:a358]{E. Poder} & New filtering for the \emph{cumulative} constraint in the context of non-overlapping rectangles & \hyperref[detail:BeldiceanuCDP11]{Details} \href{../works/BeldiceanuCDP11.pdf}{Yes} & \cite{BeldiceanuCDP11} & 2011 & Annals of Operations Research & 24 & \noindent{}\textcolor{black!50}{0.00} \textcolor{black!50}{0.00} \textbf{2.13} & 8 8 9 & 8 17 & 6 2 4\\
\index{BeniniLMR11}\rowlabel{a:BeniniLMR11}BeniniLMR11 \href{https://doi.org/10.1007/s10479-010-0718-x}{BeniniLMR11} & \hyperref[auth:a245]{L. Benini}, \hyperref[auth:a142]{M. Lombardi}, \hyperref[auth:a143]{M. Milano}, \hyperref[auth:a718]{M. Ruggiero} & Optimal resource allocation and scheduling for the {CELL} {BE} platform & \hyperref[detail:BeniniLMR11]{Details} \href{../works/BeniniLMR11.pdf}{Yes} & \cite{BeniniLMR11} & 2011 & Annals of Operations Research & 27 & \noindent{}\textcolor{black!50}{0.00} \textcolor{black!50}{0.00} \textbf{15.38} & 18 17 17 & 16 29 & 15 4 11\\
\index{Bourdeaudhuy2011}\rowlabel{a:Bourdeaudhuy2011}Bourdeaudhuy2011 \href{http://dx.doi.org/10.1080/00207543.2010.519113}{Bourdeaudhuy2011} & \hyperref[auth:a1650]{T. Bourdeaud'huy}, \hyperref[auth:a1651]{O. Belkahla}, \hyperref[auth:a681]{P. Yim}, \hyperref[auth:a680]{O. Korbaa}, \hyperref[auth:a1652]{K. Ghedira} & Transient inter-production scheduling based on Petri nets and constraint programming & \hyperref[detail:Bourdeaudhuy2011]{Details} No & \cite{Bourdeaudhuy2011} & 2011 & \cellcolor{red!20}International Journal of Production Research & null & \noindent{}\textbf{1.00} \textbf{1.00} n/a & 5 5 6 & 12 32 & 2 1 1\\
\index{Chun2011}\rowlabel{a:Chun2011}Chun2011 \href{http://dx.doi.org/10.1609/aimag.v32i2.2346}{Chun2011} & \hyperref[auth:a1322]{A. H. W. Chun} & \cellcolor{gold!20}Optimizing Limousine Service with AI \hyperref[abs:Chun2011]{Abstract} & \hyperref[detail:Chun2011]{Details} No & \cite{Chun2011} & 2011 & AI Magazine & null & \noindent{}\textcolor{black!50}{0.00} \textbf{1.50} n/a & 1 1 1 & 15 30 & 2 0 2\\
\index{CobanH11}\rowlabel{a:CobanH11}CobanH11 \href{http://dx.doi.org/10.1007/s10479-011-1031-z}{CobanH11} & \hyperref[auth:a335]{E. Coban}, \hyperref[auth:a160]{J. N. Hooker} & \cellcolor{green!10}Single-facility scheduling by logic-based Benders decomposition & \hyperref[detail:CobanH11]{Details} \href{../works/CobanH11.pdf}{Yes} & \cite{CobanH11} & 2011 & Annals of Operations Research & 28 & \noindent{}\textcolor{black!50}{0.00} \textcolor{black!50}{0.00} \textbf{17.60} & 14 15 17 & 37 44 & 29 8 21\\
\index{Coelho2011}\rowlabel{a:Coelho2011}Coelho2011 \href{http://dx.doi.org/10.1016/j.ejor.2011.03.019}{Coelho2011} & \hyperref[auth:a1555]{J. Coelho}, \hyperref[auth:a1556]{M. Vanhoucke} & \cellcolor{green!10}Multi-mode resource-constrained project scheduling using RCPSP and SAT solvers & \hyperref[detail:Coelho2011]{Details} No & \cite{Coelho2011} & 2011 & European Journal of Operational Research & null & \noindent{}\textcolor{black!50}{0.00} \textcolor{black!50}{0.00} n/a & 82 89 102 & 48 58 & 15 9 6\\
\index{Deblaere2011}\rowlabel{a:Deblaere2011}Deblaere2011 \href{http://dx.doi.org/10.1016/j.cor.2010.01.001}{Deblaere2011} & \hyperref[auth:a1775]{F. Deblaere}, \hyperref[auth:a1090]{E. Demeulemeester}, \hyperref[auth:a1102]{W. Herroelen} & Reactive scheduling in the multi-mode RCPSP & \hyperref[detail:Deblaere2011]{Details} No & \cite{Deblaere2011} & 2011 & Computers \  Operations Research & null & \noindent{}\textcolor{black!50}{0.00} \textcolor{black!50}{0.00} n/a & 77 85 103 & 33 41 & 8 5 3\\
\index{EdisO11a}\rowlabel{a:EdisO11a}EdisO11a \href{http://dx.doi.org/10.1080/03052151003759117}{EdisO11a} & \hyperref[auth:a346]{E. B. Edis}, \hyperref[auth:a348]{I. Ozkarahan} & A combined integer/constraint programming approach to a resource-constrained parallel machine scheduling problem with machine eligibility restrictions & \hyperref[detail:EdisO11a]{Details} No & \cite{EdisO11a} & 2011 & \cellcolor{red!20}Engineering Optimization & 23 & \noindent{}\textbf{2.00} \textbf{2.00} n/a & 43 46 51 & 37 53 & 28 13 15\\
\index{HachemiGR11}\rowlabel{a:HachemiGR11}HachemiGR11 \href{https://doi.org/10.1007/s10479-010-0698-x}{HachemiGR11} & \hyperref[auth:a615]{N. E. Hachemi}, \hyperref[auth:a616]{M. Gendreau}, \hyperref[auth:a326]{L.-M. Rousseau} & A hybrid constraint programming approach to the log-truck scheduling problem & \hyperref[detail:HachemiGR11]{Details} \href{../works/HachemiGR11.pdf}{Yes} & \cite{HachemiGR11} & 2011 & Annals of Operations Research & 16 & \noindent{}\textbf{1.00} \textbf{1.00} \textbf{2.82} & 32 34 38 & 19 30 & 8 3 5\\
\index{Hat2011}\rowlabel{a:Hat2011}Hat2011 \href{http://dx.doi.org/10.1504/ejie.2011.042742}{Hat2011} & \hyperref[auth:a1162]{A. Haït}, \hyperref[auth:a6]{C. Artigues} & \cellcolor{green!10}A hybrid CP/MILP method for scheduling with energy costs & \hyperref[detail:Hat2011]{Details} No & \cite{Hat2011} & 2011 & European J. of Industrial Engineering & null & \noindent{}\textbf{1.00} \textbf{1.00} n/a & 20 20 25 & 0 0 & 1 1 0\\
\index{HeckmanB11}\rowlabel{a:HeckmanB11}HeckmanB11 \href{https://doi.org/10.1007/s10951-009-0113-0}{HeckmanB11} & \hyperref[auth:a823]{I. Heckman}, \hyperref[auth:a89]{J. C. Beck} & Understanding the behavior of Solution-Guided Search for job-shop scheduling & \hyperref[detail:HeckmanB11]{Details} \href{../works/HeckmanB11.pdf}{Yes} & \cite{HeckmanB11} & 2011 & Journal of Scheduling & 20 & \noindent{}\textcolor{black!50}{0.00} \textcolor{black!50}{0.00} \textbf{3.92} & 0 1 3 & 22 39 & 7 0 7\\
\index{KelbelH11}\rowlabel{a:KelbelH11}KelbelH11 \href{https://doi.org/10.1007/s10845-009-0318-2}{KelbelH11} & \hyperref[auth:a618]{J. Kelbel}, \hyperref[auth:a116]{Z. Hanz{\'{a}}lek} & Solving production scheduling with earliness/tardiness penalties by constraint programming & \hyperref[detail:KelbelH11]{Details} \href{../works/KelbelH11.pdf}{Yes} & \cite{KelbelH11} & 2011 & Journal of Intelligent Manufacturing & 10 & \noindent{}\textbf{1.00} \textbf{1.00} \textbf{11.84} & 12 16 18 & 14 25 & 13 5 8\\
\index{KovacsB11}\rowlabel{a:KovacsB11}KovacsB11 \href{https://doi.org/10.1007/s10601-009-9088-x}{KovacsB11} & \hyperref[auth:a146]{A. Kov{\'{a}}cs}, \hyperref[auth:a89]{J. C. Beck} & A global constraint for total weighted completion time for unary resources & \hyperref[detail:KovacsB11]{Details} \href{../works/KovacsB11.pdf}{Yes} & \cite{KovacsB11} & 2011 & Constraints An Int. J. & 24 & \noindent{}\textcolor{black!50}{0.00} \textcolor{black!50}{0.00} \textbf{8.15} & 4 4 9 & 26 36 & 5 1 4\\
\index{KovacsK11}\rowlabel{a:KovacsK11}KovacsK11 \href{https://doi.org/10.1007/s10601-010-9102-3}{KovacsK11} & \hyperref[auth:a146]{A. Kov{\'{a}}cs}, \hyperref[auth:a155]{T. Kis} & Constraint programming approach to a bilevel scheduling problem & \hyperref[detail:KovacsK11]{Details} \href{../works/KovacsK11.pdf}{Yes} & \cite{KovacsK11} & 2011 & Constraints An Int. J. & 24 & \noindent{}\textbf{1.00} \textbf{1.00} \textbf{5.37} & 3 4 5 & 24 37 & 3 0 3\\
\index{LiuW11}\rowlabel{a:LiuW11}LiuW11 \href{http://dx.doi.org/10.1016/j.autcon.2011.04.012}{LiuW11} & \hyperref[auth:a1244]{S.-S. Liu}, \hyperref[auth:a1245]{C.-J. Wang} & Optimizing project selection and scheduling problems with time-dependent resource constraints & \hyperref[detail:LiuW11]{Details} \href{../works/LiuW11.pdf}{Yes} & \cite{LiuW11} & 2011 & Automation in Construction & 10 & \noindent{}\textcolor{black!50}{0.00} \textcolor{black!50}{0.00} \textbf{11.58} & 57 59 71 & 35 48 & 15 5 10\\
\index{Lizarralde2011}\rowlabel{a:Lizarralde2011}Lizarralde2011 \href{http://dx.doi.org/10.3917/proj.007.0089}{Lizarralde2011} & \hyperref[auth:a1478]{I. Lizarralde}, \hyperref[auth:a1248]{P. Esquirol}, \hyperref[auth:a1479]{A. Rivière} & A decision support system to schedule design activities with interdependency and resource constraints \hyperref[abs:Lizarralde2011]{Abstract} & \hyperref[detail:Lizarralde2011]{Details} No & \cite{Lizarralde2011} & 2011 & Projectics / Proyéctica / Projectique & null & \noindent{}\textcolor{black!50}{0.00} \textbf{4.50} n/a & 1 1 0 & 12 34 & 7 1 6\\
\index{ReddyFIBKAJ11}\rowlabel{a:ReddyFIBKAJ11}ReddyFIBKAJ11 \href{https://doi.org/10.1145/1989734.1989745}{ReddyFIBKAJ11} & \hyperref[auth:a1037]{S. Y. Reddy}, \hyperref[auth:a379]{J. Frank}, \hyperref[auth:a1038]{M. Iatauro}, \hyperref[auth:a1039]{M. E. Boyce}, \hyperref[auth:a380]{E. K{\"{u}}rkl{\"{u}}}, \hyperref[auth:a1040]{M. Ai-Chang}, \hyperref[auth:a1041]{A. K. J{\'{o}}nsson} & Planning solar array operations on the international space station & \hyperref[detail:ReddyFIBKAJ11]{Details} \href{../works/ReddyFIBKAJ11.pdf}{Yes} & \cite{ReddyFIBKAJ11} & 2011 & {ACM} Trans. Intell. Syst. Technol. & 24 & \noindent{}\textcolor{black!50}{0.00} \textcolor{black!50}{0.00} 0.59 & 3 3 11 & 8 22 & 1 1 0\\
\index{SchausHMCMD11}\rowlabel{a:SchausHMCMD11}SchausHMCMD11 \href{https://doi.org/10.1007/s10601-010-9100-5}{SchausHMCMD11} & \hyperref[auth:a147]{P. Schaus}, \hyperref[auth:a148]{P. V. Hentenryck}, \hyperref[auth:a149]{J.-N. Monette}, \hyperref[auth:a150]{C. Coffrin}, \hyperref[auth:a32]{L. Michel}, \hyperref[auth:a151]{Y. Deville} & \cellcolor{green!10}Solving Steel Mill Slab Problems with constraint-based techniques: CP, LNS, and {CBLS} & \hyperref[detail:SchausHMCMD11]{Details} \href{../works/SchausHMCMD11.pdf}{Yes} & \cite{SchausHMCMD11} & 2011 & Constraints An Int. J. & 23 & \noindent{}\textcolor{black!50}{0.00} \textcolor{black!50}{0.00} \textbf{3.20} & 14 16 19 & 5 12 & 5 2 3\\
\index{SchuttFSW11}\rowlabel{a:SchuttFSW11}SchuttFSW11 \href{https://doi.org/10.1007/s10601-010-9103-2}{SchuttFSW11} & \hyperref[auth:a124]{A. Schutt}, \hyperref[auth:a154]{T. Feydy}, \hyperref[auth:a125]{P. J. Stuckey}, \hyperref[auth:a117]{M. G. Wallace} & Explaining the cumulative propagator & \hyperref[detail:SchuttFSW11]{Details} \href{../works/SchuttFSW11.pdf}{Yes} & \cite{SchuttFSW11} & 2011 & Constraints An Int. J. & 33 & \noindent{}\textcolor{black!50}{0.00} \textcolor{black!50}{0.00} \textbf{14.77} & 57 61 65 & 23 39 & 48 34 14\\
\index{TopalogluO11}\rowlabel{a:TopalogluO11}TopalogluO11 \href{https://doi.org/10.1016/j.cor.2010.04.018}{TopalogluO11} & \hyperref[auth:a617]{S. Topaloglu}, \hyperref[auth:a348]{I. Ozkarahan} & A constraint programming-based solution approach for medical resident scheduling problems & \hyperref[detail:TopalogluO11]{Details} \href{../works/TopalogluO11.pdf}{Yes} & \cite{TopalogluO11} & 2011 & Computers \  Operations Research & 10 & \noindent{}\textbf{1.00} \textbf{1.00} \textbf{2.44} & 46 47 59 & 24 32 & 14 11 3\\
\index{TrojetHL11}\rowlabel{a:TrojetHL11}TrojetHL11 \href{https://doi.org/10.1016/j.cie.2010.08.014}{TrojetHL11} & \hyperref[auth:a705]{M. Trojet}, \hyperref[auth:a706]{F. H'Mida}, \hyperref[auth:a3]{P. Lopez} & \cellcolor{green!10}Project scheduling under resource constraints: Application of the cumulative global constraint in a decision support framework & \hyperref[detail:TrojetHL11]{Details} \href{../works/TrojetHL11.pdf}{Yes} & \cite{TrojetHL11} & 2011 & Computers \  Industrial Engineering & 7 & \noindent{}\textcolor{black!50}{0.00} \textcolor{black!50}{0.00} \textbf{5.05} & 11 13 12 & 17 32 & 7 3 4\\
\index{ZeballosNH11}\rowlabel{a:ZeballosNH11}ZeballosNH11 \href{http://dx.doi.org/10.1016/j.compchemeng.2011.01.043}{ZeballosNH11} & \hyperref[auth:a621]{L. J. Zeballos}, \hyperref[auth:a524]{J. M. Novas}, \hyperref[auth:a588]{G. P. Henning} & A CP formulation for scheduling multiproduct multistage batch plants & \hyperref[detail:ZeballosNH11]{Details} \href{../works/ZeballosNH11.pdf}{Yes} & \cite{ZeballosNH11} & 2011 & Computers \  Chemical Engineering & 17 & \noindent{}\textbf{1.00} \textbf{1.00} \textbf{28.37} & 26 26 28 & 29 39 & 16 7 9\\
\index{BartakCS10}\rowlabel{a:BartakCS10}BartakCS10 \href{https://doi.org/10.1007/s10479-008-0492-1}{BartakCS10} & \hyperref[auth:a152]{R. Bart{\'{a}}k}, \hyperref[auth:a161]{O. Cepek}, \hyperref[auth:a780]{P. Surynek} & Discovering implied constraints in precedence graphs with alternatives & \hyperref[detail:BartakCS10]{Details} \href{../works/BartakCS10.pdf}{Yes} & \cite{BartakCS10} & 2010 & Annals of Operations Research & 31 & \noindent{}\textcolor{black!50}{0.00} \textcolor{black!50}{0.00} 0.91 & 2 2 2 & 9 21 & 3 1 2\\
\index{BartakSR10}\rowlabel{a:BartakSR10}BartakSR10 \href{https://doi.org/10.1017/S0269888910000202}{BartakSR10} & \hyperref[auth:a152]{R. Bart{\'{a}}k}, \hyperref[auth:a153]{M. A. Salido}, \hyperref[auth:a316]{F. Rossi} & \cellcolor{green!10}New trends in constraint satisfaction, planning, and scheduling: a survey & \hyperref[detail:BartakSR10]{Details} \href{../works/BartakSR10.pdf}{Yes} & \cite{BartakSR10} & 2010 & Knowl. Eng. Rev. & 31 & \noindent{}\textbf{1.00} \textbf{1.00} \textbf{59.34} & 28 29 31 & 47 88 & 18 4 14\\
\index{Bartk2010}\rowlabel{a:Bartk2010}Bartk2010 \href{http://dx.doi.org/10.1177/0142331208100099}{Bartk2010} & \hyperref[auth:a1063]{R. Barták}, \hyperref[auth:a1557]{O. Čepek} & \cellcolor{green!10}Incremental propagation rules for a precedence graph with optional activities and time windows \hyperref[abs:Bartk2010]{Abstract} & \hyperref[detail:Bartk2010]{Details} No & \cite{Bartk2010} & 2010 & Transactions of the Institute of Measurement and Control & null & \noindent{}\textcolor{black!50}{0.00} 0.75 n/a & 5 5 6 & 6 15 & 6 2 4\\
\index{Biswas2010}\rowlabel{a:Biswas2010}Biswas2010 \href{http://dx.doi.org/10.1177/0037549710373601}{Biswas2010} & \hyperref[auth:a2019]{M. Biswas}, \hyperref[auth:a2020]{M. R. Frater}, \hyperref[auth:a2021]{M. Ryan} & Determination of hub port attenuation satisfying radio link path losses for hardware emulation \hyperref[abs:Biswas2010]{Abstract} & \hyperref[detail:Biswas2010]{Details} No & \cite{Biswas2010} & 2010 & SIMULATION & null & \noindent{}\textcolor{black!50}{0.00} \textbf{1.00} n/a & 0 1 1 & 2 9 & 1 0 1\\
\index{ChenGPSH10}\rowlabel{a:ChenGPSH10}ChenGPSH10 \href{http://dx.doi.org/10.1007/s11465-010-0106-x}{ChenGPSH10} & \hyperref[auth:a913]{Y. Chen}, \hyperref[auth:a914]{Z. Guan}, \hyperref[auth:a915]{Y. Peng}, \hyperref[auth:a916]{X. Shao}, \hyperref[auth:a917]{M. Hasseb} & Technology and system of constraint programming for industry production scheduling — Part I: A brief survey and potential directions & \hyperref[detail:ChenGPSH10]{Details} \href{../works/ChenGPSH10.pdf}{Yes} & \cite{ChenGPSH10} & 2010 & Frontiers of Mechanical Engineering in China & 10 & \noindent{}\textbf{1.00} \textbf{1.00} \textbf{20.87} & 2 2 4 & 32 50 & 18 2 16\\
\index{KendallKRU10}\rowlabel{a:KendallKRU10}KendallKRU10 \href{http://dx.doi.org/10.1016/j.cor.2009.05.013}{KendallKRU10} & \hyperref[auth:a1387]{G. Kendall}, \hyperref[auth:a1166]{S. Knust}, \hyperref[auth:a1386]{C. C. Ribeiro}, \hyperref[auth:a1388]{S. Urrutia} & Scheduling in sports: An annotated bibliography \hyperref[abs:KendallKRU10]{Abstract} & \hyperref[detail:KendallKRU10]{Details} \href{../works/KendallKRU10.pdf}{Yes} & \cite{KendallKRU10} & 2010 & Computers \  Operations Research & 19 & \noindent{}\textcolor{black!50}{0.00} \textcolor{black!50}{0.00} \textbf{9.51} & 181 186 220 & 0 0 & 6 6 0\\
\index{LiuGT10}\rowlabel{a:LiuGT10}LiuGT10 \href{http://dx.doi.org/10.3724/sp.j.1004.2010.00603}{LiuGT10} & \hyperref[auth:a1220]{S.-X. Liu}, \hyperref[auth:a1221]{Z. Guo}, \hyperref[auth:a1222]{J.-F. Tang} & Constraint Propagation for Cumulative Scheduling Problems with Precedences: Constraint Propagation for Cumulative Scheduling Problems with Precedences & \hyperref[detail:LiuGT10]{Details} No & \cite{LiuGT10} & 2010 & \cellcolor{red!20}Acta Automatica Sinica & 7 & \noindent{}\textbf{1.50} \textbf{1.50} n/a & 2 2 9 & 15 20 & 11 0 11\\
\index{LombardiM10a}\rowlabel{a:LombardiM10a}LombardiM10a \href{https://doi.org/10.1016/j.artint.2010.02.004}{LombardiM10a} & \hyperref[auth:a142]{M. Lombardi}, \hyperref[auth:a143]{M. Milano} & \cellcolor{gold!20}Allocation and scheduling of Conditional Task Graphs & \hyperref[detail:LombardiM10a]{Details} \href{../works/LombardiM10a.pdf}{Yes} & \cite{LombardiM10a} & 2010 & Artificial Intelligence & 30 & \noindent{}\textcolor{black!50}{0.00} \textcolor{black!50}{0.00} \textbf{10.65} & 8 8 13 & 24 41 & 11 2 9\\
\index{LombardiMRB10}\rowlabel{a:LombardiMRB10}LombardiMRB10 \href{http://dx.doi.org/10.1007/s10951-010-0184-y}{LombardiMRB10} & \hyperref[auth:a142]{M. Lombardi}, \hyperref[auth:a143]{M. Milano}, \hyperref[auth:a718]{M. Ruggiero}, \hyperref[auth:a245]{L. Benini} & Stochastic allocation and scheduling for conditional task graphs in multi-processor systems-on-chip & \hyperref[detail:LombardiMRB10]{Details} \href{../works/LombardiMRB10.pdf}{Yes} & \cite{LombardiMRB10} & 2010 & Journal of Scheduling & 31 & \noindent{}\textcolor{black!50}{0.00} \textcolor{black!50}{0.00} \textbf{17.06} & 24 24 30 & 41 55 & 19 5 14\\
\index{LopesCSM10}\rowlabel{a:LopesCSM10}LopesCSM10 \href{https://doi.org/10.1007/s10601-009-9086-z}{LopesCSM10} & \hyperref[auth:a156]{T. M. T. Lopes}, \hyperref[auth:a157]{A. A. Cir{\'{e}}}, \hyperref[auth:a158]{C. C. de Souza}, \hyperref[auth:a159]{A. V. Moura} & A hybrid model for a multiproduct pipeline planning and scheduling problem & \hyperref[detail:LopesCSM10]{Details} \href{../works/LopesCSM10.pdf}{Yes} & \cite{LopesCSM10} & 2010 & Constraints An Int. J. & 39 & \noindent{}\textcolor{black!50}{0.00} \textcolor{black!50}{0.00} \textbf{11.10} & 31 31 35 & 18 31 & 4 0 4\\
\index{Magato2010}\rowlabel{a:Magato2010}Magato2010 \href{http://dx.doi.org/10.1007/s10951-010-0186-9}{Magato2010} & \hyperref[auth:a1808]{L. Magatão}, \hyperref[auth:a1809]{L. V. R. Arruda}, \hyperref[auth:a1810]{F. Neves-Jr} & A combined CLP-MILP approach for scheduling commodities in a pipeline & \hyperref[detail:Magato2010]{Details} No & \cite{Magato2010} & 2010 & Journal of Scheduling & null & \noindent{}\textbf{1.00} \textbf{1.00} n/a & 27 27 29 & 21 38 & 3 0 3\\
\index{NovasH10}\rowlabel{a:NovasH10}NovasH10 \href{https://doi.org/10.1016/j.compchemeng.2010.07.011}{NovasH10} & \hyperref[auth:a524]{J. M. Novas}, \hyperref[auth:a588]{G. P. Henning} & \cellcolor{green!10}Reactive scheduling framework based on domain knowledge and constraint programming & \hyperref[detail:NovasH10]{Details} \href{../works/NovasH10.pdf}{Yes} & \cite{NovasH10} & 2010 & Computers \  Chemical Engineering & 20 & \noindent{}\textbf{1.00} \textbf{1.00} \textbf{22.75} & 48 49 50 & 19 29 & 4 4 0\\
\index{OzturkTHO10}\rowlabel{a:OzturkTHO10}OzturkTHO10 \href{https://www.sciencedirect.com/science/article/pii/S1571065310000107}{OzturkTHO10} & \hyperref[auth:a135]{C. {\"{O}}zt{\"{u}}rk}, \hyperref[auth:a136]{S. Tunali}, \hyperref[auth:a137]{B. Hnich}, \hyperref[auth:a138]{A. {\"{O}}rnek} & Simultaneous Balancing and Scheduling of Flexible Mixed Model Assembly Lines with Sequence-Dependent Setup Times & \hyperref[detail:OzturkTHO10]{Details} \href{../works/OzturkTHO10.pdf}{Yes} & \cite{OzturkTHO10} & 2010 & Electronic Notes in Discrete Mathematics & 8 & \noindent{}\textcolor{black!50}{0.00} \textcolor{black!50}{0.00} \textbf{1.83} & 15 15 17 & 1 3 & 1 1 0\\
\index{Refanidis2010}\rowlabel{a:Refanidis2010}Refanidis2010 \href{http://dx.doi.org/10.1145/1869397.1869401}{Refanidis2010} & \hyperref[auth:a1546]{I. Refanidis}, \hyperref[auth:a19]{N. Yorke-Smith} & A constraint-based approach to scheduling an individual's activities \hyperref[abs:Refanidis2010]{Abstract} & \hyperref[detail:Refanidis2010]{Details} No & \cite{Refanidis2010} & 2010 & ACM Transactions on Intelligent Systems and Technology & null & \noindent{}\textcolor{black!50}{0.00} \textbf{2.00} n/a & 11 11 20 & 9 26 & 2 0 2\\
\index{Salido10}\rowlabel{a:Salido10}Salido10 \href{https://doi.org/10.1007/s10845-008-0188-z}{Salido10} & \hyperref[auth:a153]{M. A. Salido} & \cellcolor{gold!20}Introduction to planning, scheduling and constraint satisfaction \hyperref[abs:Salido10]{Abstract} & \hyperref[detail:Salido10]{Details} \href{../works/Salido10.pdf}{Yes} & \cite{Salido10} & 2010 & Journal of Intelligent Manufacturing & 4 & \noindent{}\textbf{1.00} \textbf{4.50} \textbf{1.68} & 22 22 17 & 1 3 & 1 0 1\\
\index{Verfaillie2010}\rowlabel{a:Verfaillie2010}Verfaillie2010 \href{http://dx.doi.org/10.1017/s0269888910000172}{Verfaillie2010} & \hyperref[auth:a1722]{G. Verfaillie}, \hyperref[auth:a1897]{C. Pralet}, \hyperref[auth:a2052]{M. Lemaître} & How to model planning and scheduling problems using constraint networks on timelines \hyperref[abs:Verfaillie2010]{Abstract} & \hyperref[detail:Verfaillie2010]{Details} No & \cite{Verfaillie2010} & 2010 & The Knowledge Engineering Review & null & \noindent{}\textcolor{black!50}{0.00} \textbf{1.00} n/a & 18 0 28 & 9 0 & 1 0 1\\
\index{Zeballos10}\rowlabel{a:Zeballos10}Zeballos10 \href{http://dx.doi.org/10.1016/j.rcim.2010.04.005}{Zeballos10} & \hyperref[auth:a621]{L. J. Zeballos} & A constraint programming approach to tool allocation and production scheduling in flexible manufacturing systems & \hyperref[detail:Zeballos10]{Details} \href{../works/Zeballos10.pdf}{Yes} & \cite{Zeballos10} & 2010 & Robotics and Computer-Integrated Manufacturing & 19 & \noindent{}\textbf{1.00} \textbf{1.00} \textbf{16.15} & 41 42 51 & 16 23 & 13 7 6\\
\index{ZeballosCM10}\rowlabel{a:ZeballosCM10}ZeballosCM10 \href{http://dx.doi.org/10.1021/ie1016199}{ZeballosCM10} & \hyperref[auth:a621]{L. J. Zeballos}, \hyperref[auth:a891]{P. M. Castro}, \hyperref[auth:a1190]{C. A. Méndez} & \cellcolor{green!10}Integrated Constraint Programming Scheduling Approach for Automated Wet-Etch Stations in Semiconductor Manufacturing & \hyperref[detail:ZeballosCM10]{Details} \href{../works/ZeballosCM10.pdf}{Yes} & \cite{ZeballosCM10} & 2010 & Industrial \  Engineering Chemistry Research & 11 & \noindent{}\textbf{1.00} \textbf{1.00} \textbf{15.92} & 22 23 29 & 30 39 & 10 3 7\\
\index{ZeballosQH10}\rowlabel{a:ZeballosQH10}ZeballosQH10 \href{https://doi.org/10.1016/j.engappai.2009.07.002}{ZeballosQH10} & \hyperref[auth:a621]{L. J. Zeballos}, \hyperref[auth:a622]{O. Quiroga}, \hyperref[auth:a588]{G. P. Henning} & A constraint programming model for the scheduling of flexible manufacturing systems with machine and tool limitations & \hyperref[detail:ZeballosQH10]{Details} \href{../works/ZeballosQH10.pdf}{Yes} & \cite{ZeballosQH10} & 2010 & Eng. Appl. Artif. Intell. & 20 & \noindent{}\textbf{1.50} \textbf{1.50} \textbf{13.56} & 33 33 43 & 28 41 & 16 10 6\\
\index{abs-1009-0347}\rowlabel{a:abs-1009-0347}abs-1009-0347 \href{http://arxiv.org/abs/1009.0347}{abs-1009-0347} & \hyperref[auth:a124]{A. Schutt}, \hyperref[auth:a154]{T. Feydy}, \hyperref[auth:a125]{P. J. Stuckey}, \hyperref[auth:a117]{M. G. Wallace} & Solving the Resource Constrained Project Scheduling Problem with Generalized Precedences by Lazy Clause Generation & \hyperref[detail:abs-1009-0347]{Details} \href{../works/abs-1009-0347.pdf}{Yes} & \cite{abs-1009-0347} & 2010 & CoRR & 37 & \noindent{}\textcolor{black!50}{0.00} \textcolor{black!50}{0.00} \textbf{4.70} & 0 0 0 & 0 0 & 0 0 0\\
\index{BidotVLB09}\rowlabel{a:BidotVLB09}BidotVLB09 \href{https://doi.org/10.1007/s10951-008-0080-x}{BidotVLB09} & \hyperref[auth:a824]{J. Bidot}, \hyperref[auth:a825]{T. Vidal}, \hyperref[auth:a118]{P. Laborie}, \hyperref[auth:a89]{J. C. Beck} & A theoretic and practical framework for scheduling in a stochastic environment & \hyperref[detail:BidotVLB09]{Details} \href{../works/BidotVLB09.pdf}{Yes} & \cite{BidotVLB09} & 2009 & Journal of Scheduling & 30 & \noindent{}\textcolor{black!50}{0.00} \textcolor{black!50}{0.00} \textbf{9.21} & 58 60 76 & 20 49 & 8 1 7\\
\index{Bocewicz2009}\rowlabel{a:Bocewicz2009}Bocewicz2009 \href{http://dx.doi.org/10.1108/03684920910976989}{Bocewicz2009} & \hyperref[auth:a630]{G. Bocewicz}, \hyperref[auth:a631]{I. Bach}, \hyperref[auth:a1913]{R. Wójcik} & Production flow prototyping subject to imprecise activity specification \hyperref[abs:Bocewicz2009]{Abstract} & \hyperref[detail:Bocewicz2009]{Details} No & \cite{Bocewicz2009} & 2009 & Kybernetes & null & \noindent{}\textcolor{black!50}{0.00} \textbf{13.51} n/a & 2 3 3 & 11 17 & 1 0 1\\
\index{BocewiczBB09}\rowlabel{a:BocewiczBB09}BocewiczBB09 \href{https://doi.org/10.1504/IJIIDS.2009.023038}{BocewiczBB09} & \hyperref[auth:a630]{G. Bocewicz}, \hyperref[auth:a631]{I. Bach}, \hyperref[auth:a632]{Z. A. Banaszak} & Logic-algebraic method based and constraints programming driven approach to AGVs scheduling & \hyperref[detail:BocewiczBB09]{Details} \href{../works/BocewiczBB09.pdf}{Yes} & \cite{BocewiczBB09} & 2009 & Int. J. Intell. Inf. Database Syst. & 19 & \noindent{}\textcolor{black!50}{0.00} \textcolor{black!50}{0.00} 0.91 & 0 0 1 & 0 0 & 0 0 0\\
\index{Capone2009}\rowlabel{a:Capone2009}Capone2009 \href{http://dx.doi.org/10.1002/net.20367}{Capone2009} & \hyperref[auth:a1563]{A. Capone}, \hyperref[auth:a1564]{G. Carello}, \hyperref[auth:a1565]{I. Filippini}, \hyperref[auth:a1566]{S. Gualandi}, \hyperref[auth:a1567]{F. Malucelli} & Solving a resource allocation problem in wireless mesh networks: A comparison between a CP‐based and a classical column generation \hyperref[abs:Capone2009]{Abstract} & \hyperref[detail:Capone2009]{Details} No & \cite{Capone2009} & 2009 & Networks & null & \noindent{}0.50 \textbf{1.00} n/a & 3 14 17 & 16 22 & 3 1 2\\
\index{CarchraeB09}\rowlabel{a:CarchraeB09}CarchraeB09 \href{http://dx.doi.org/10.1007/s10852-008-9100-2}{CarchraeB09} & \hyperref[auth:a272]{T. Carchrae}, \hyperref[auth:a89]{J. C. Beck} & Principles for the Design of Large Neighborhood Search & \hyperref[detail:CarchraeB09]{Details} \href{../works/CarchraeB09.pdf}{Yes} & \cite{CarchraeB09} & 2009 & Journal of Mathematical Modelling and Algorithms & 26 & \noindent{}\textcolor{black!50}{0.00} \textcolor{black!50}{0.00} \textbf{5.14} & 16 17 25 & 19 29 & 17 5 12\\
\index{GarridoAO09}\rowlabel{a:GarridoAO09}GarridoAO09 \href{https://doi.org/10.1007/s10951-008-0083-7}{GarridoAO09} & \hyperref[auth:a633]{A. Garrido}, \hyperref[auth:a634]{M. Arang{\'{u}}}, \hyperref[auth:a635]{E. Onaindia} & A constraint programming formulation for planning: from plan scheduling to plan generation & \hyperref[detail:GarridoAO09]{Details} \href{../works/GarridoAO09.pdf}{Yes} & \cite{GarridoAO09} & 2009 & Journal of Scheduling & 30 & \noindent{}\textbf{1.00} \textbf{1.00} \textbf{20.10} & 5 5 9 & 14 37 & 4 1 3\\
\index{Jans09}\rowlabel{a:Jans09}Jans09 \href{http://dx.doi.org/10.1287/ijoc.1080.0283}{Jans09} & \hyperref[auth:a841]{R. Jans} & \cellcolor{green!10}Solving Lot-Sizing Problems on Parallel Identical Machines Using Symmetry-Breaking Constraints & \hyperref[detail:Jans09]{Details} \href{../works/Jans09.pdf}{Yes} & \cite{Jans09} & 2009 & \cellcolor{red!20}INFORMS Journal on Computing & 14 & \noindent{}\textcolor{black!50}{0.00} \textcolor{black!50}{0.00} \textcolor{black!50}{0.00} & 59 60 61 & 73 77 & 15 0 15\\
\index{Michel2009}\rowlabel{a:Michel2009}Michel2009 \href{http://dx.doi.org/10.1287/ijoc.1080.0313}{Michel2009} & \hyperref[auth:a32]{L. Michel}, \hyperref[auth:a1807]{A. See}, \hyperref[auth:a148]{P. V. Hentenryck} & Transparent Parallelization of Constraint Programming \hyperref[abs:Michel2009]{Abstract} & \hyperref[detail:Michel2009]{Details} No & \cite{Michel2009} & 2009 & \cellcolor{red!20}INFORMS Journal on Computing & null & \noindent{}\textcolor{black!50}{0.00} 0.50 n/a & 28 28 39 & 15 24 & 5 1 4\\
\index{MilanoW09}\rowlabel{a:MilanoW09}MilanoW09 \href{http://dx.doi.org/10.1007/s10479-009-0654-9}{MilanoW09} & \hyperref[auth:a143]{M. Milano}, \hyperref[auth:a117]{M. G. Wallace} & Integrating Operations Research in Constraint Programming & \hyperref[detail:MilanoW09]{Details} \href{../works/MilanoW09.pdf}{Yes} & \cite{MilanoW09} & 2009 & Annals of Operations Research & 40 & \noindent{}\textcolor{black!50}{0.00} \textcolor{black!50}{0.00} \textbf{44.83} & 34 35 41 & 46 77 & 25 6 19\\
\index{OhrimenkoSC09}\rowlabel{a:OhrimenkoSC09}OhrimenkoSC09 \href{http://dx.doi.org/10.1007/s10601-008-9064-x}{OhrimenkoSC09} & \hyperref[auth:a861]{O. Ohrimenko}, \hyperref[auth:a125]{P. J. Stuckey}, \hyperref[auth:a862]{M. Codish} & Propagation via lazy clause generation & \hyperref[detail:OhrimenkoSC09]{Details} \href{../works/OhrimenkoSC09.pdf}{Yes} & \cite{OhrimenkoSC09} & 2009 & Constraints An Int. J. & 35 & \noindent{}\textcolor{black!50}{0.00} \textcolor{black!50}{0.00} \textbf{3.19} & 127 128 198 & 15 35 & 33 31 2\\
\index{RasmussenT09}\rowlabel{a:RasmussenT09}RasmussenT09 \href{http://dx.doi.org/10.1007/s10479-008-0384-4}{RasmussenT09} & \hyperref[auth:a1403]{R. V. Rasmussen}, \hyperref[auth:a1389]{M. A. Trick} & \cellcolor{green!10}The timetable constrained distance minimization problem \hyperref[abs:RasmussenT09]{Abstract} & \hyperref[detail:RasmussenT09]{Details} \href{../works/RasmussenT09.pdf}{Yes} & \cite{RasmussenT09} & 2009 & Annals of Operations Research & 15 & \noindent{}\textcolor{black!50}{0.00} \textbf{1.00} \textbf{1.89} & 8 9 9 & 15 25 & 7 1 6\\
\index{RuggieroBBMA09}\rowlabel{a:RuggieroBBMA09}RuggieroBBMA09 \href{https://doi.org/10.1109/TCAD.2009.2013536}{RuggieroBBMA09} & \hyperref[auth:a718]{M. Ruggiero}, \hyperref[auth:a375]{D. Bertozzi}, \hyperref[auth:a245]{L. Benini}, \hyperref[auth:a143]{M. Milano}, \hyperref[auth:a719]{A. Andrei} & \cellcolor{green!10}Reducing the Abstraction and Optimality Gaps in the Allocation and Scheduling for Variable Voltage/Frequency MPSoC Platforms & \hyperref[detail:RuggieroBBMA09]{Details} \href{../works/RuggieroBBMA09.pdf}{Yes} & \cite{RuggieroBBMA09} & 2009 & {IEEE} Trans. Comput. Aided Des. Integr. Circuits Syst. & 14 & \noindent{}\textcolor{black!50}{0.00} \textcolor{black!50}{0.00} \textbf{5.02} & 9 9 7 & 27 37 & 5 0 5\\
\index{Smith-Miles2009}\rowlabel{a:Smith-Miles2009}Smith-Miles2009 \href{http://dx.doi.org/10.1145/1456650.1456656}{Smith-Miles2009} & \hyperref[auth:a1742]{K. A. Smith-Miles} & Cross-disciplinary perspectives on meta-learning for algorithm selection \hyperref[abs:Smith-Miles2009]{Abstract} & \hyperref[detail:Smith-Miles2009]{Details} No & \cite{Smith-Miles2009} & 2009 & ACM Computing Surveys & null & \noindent{}\textcolor{black!50}{0.00} \textbf{1.50} n/a & 298 307 395 & 46 99 & 3 3 0\\
\index{WuBB09}\rowlabel{a:WuBB09}WuBB09 \href{https://doi.org/10.1016/j.cor.2008.08.008}{WuBB09} & \hyperref[auth:a274]{C. W. Wu}, \hyperref[auth:a217]{K. N. Brown}, \hyperref[auth:a89]{J. C. Beck} & Scheduling with uncertain durations: Modeling beta-robust scheduling with constraints & \hyperref[detail:WuBB09]{Details} \href{../works/WuBB09.pdf}{Yes} & \cite{WuBB09} & 2009 & Computers \  Operations Research & 9 & \noindent{}\textcolor{black!50}{0.00} \textcolor{black!50}{0.00} \textbf{3.11} & 42 42 48 & 5 16 & 3 2 1\\
\index{Yang2009}\rowlabel{a:Yang2009}Yang2009 \href{http://dx.doi.org/10.1007/s10951-009-0106-z}{Yang2009} & \hyperref[auth:a1823]{S. Yang}, \hyperref[auth:a1824]{D. Wang}, \hyperref[auth:a1825]{T. Chai}, \hyperref[auth:a1387]{G. Kendall} & \cellcolor{green!10}An improved constraint satisfaction adaptive neural network for job-shop scheduling & \hyperref[detail:Yang2009]{Details} No & \cite{Yang2009} & 2009 & Journal of Scheduling & null & \noindent{}\textbf{2.00} \textbf{2.00} n/a & 15 15 20 & 25 40 & 4 1 3\\
\index{ZeballosM09}\rowlabel{a:ZeballosM09}ZeballosM09 \href{http://dx.doi.org/10.1021/ie901176n}{ZeballosM09} & \hyperref[auth:a621]{L. J. Zeballos}, \hyperref[auth:a1190]{C. A. Méndez} & \cellcolor{green!10}An Integrated CP-Based Approach for Scheduling of Processing and Transport Units in Pipeless Plants & \hyperref[detail:ZeballosM09]{Details} \href{../works/ZeballosM09.pdf}{Yes} & \cite{ZeballosM09} & 2009 & Industrial \  Engineering Chemistry Research & 13 & \noindent{}\textbf{1.00} \textbf{1.00} \textbf{14.79} & 7 7 7 & 14 23 & 9 1 8\\
\index{abs-0907-0939}\rowlabel{a:abs-0907-0939}abs-0907-0939 \href{http://arxiv.org/abs/0907.0939}{abs-0907-0939} & \hyperref[auth:a221]{T. Petit}, \hyperref[auth:a358]{E. Poder} & The Soft Cumulative Constraint & \hyperref[detail:abs-0907-0939]{Details} \href{../works/abs-0907-0939.pdf}{Yes} & \cite{abs-0907-0939} & 2009 & CoRR & 12 & \noindent{}\textcolor{black!50}{0.00} \textcolor{black!50}{0.00} 0.37 & 0 0 0 & 0 0 & 0 0 0\\
\index{Banaszak2008}\rowlabel{a:Banaszak2008}Banaszak2008 \href{http://dx.doi.org/10.7494/dmms.2008.2.2.5}{Banaszak2008} & \hyperref[auth:a1814]{Z. Banaszak}, \hyperref[auth:a630]{G. Bocewicz}, \hyperref[auth:a631]{I. Bach} & CP-driven Production Process Planning in Multiproject Environment \hyperref[abs:Banaszak2008]{Abstract} & \hyperref[detail:Banaszak2008]{Details} No & \cite{Banaszak2008} & 2008 & Decision Making in Manufacturing and Services & null & \noindent{}\textcolor{black!50}{0.00} \textbf{4.00} n/a & 4 4 0 & 0 0 & 1 1 0\\
\index{BartakSR08}\rowlabel{a:BartakSR08}BartakSR08 \href{http://dx.doi.org/10.1007/s10845-008-0203-4}{BartakSR08} & \hyperref[auth:a1063]{R. Barták}, \hyperref[auth:a153]{M. A. Salido}, \hyperref[auth:a316]{F. Rossi} & \cellcolor{green!10}Constraint satisfaction techniques in planning and scheduling & \hyperref[detail:BartakSR08]{Details} \href{../works/BartakSR08.pdf}{Yes} & \cite{BartakSR08} & 2008 & Journal of Intelligent Manufacturing & 11 & \noindent{}\textbf{1.00} \textbf{1.00} \textbf{12.88} & 54 57 76 & 21 51 & 16 7 9\\
\index{ClautiauxJCM08}\rowlabel{a:ClautiauxJCM08}ClautiauxJCM08 \href{http://dx.doi.org/10.1016/j.cor.2006.05.012}{ClautiauxJCM08} & \hyperref[auth:a1169]{F. Clautiaux}, \hyperref[auth:a929]{A. Jouglet}, \hyperref[auth:a845]{J. Carlier}, \hyperref[auth:a1170]{A. Moukrim} & A new constraint programming approach for the orthogonal packing problem & \hyperref[detail:ClautiauxJCM08]{Details} \href{../works/ClautiauxJCM08.pdf}{Yes} & \cite{ClautiauxJCM08} & 2008 & Computers \  Operations Research & 16 & \noindent{}\textcolor{black!50}{0.00} \textcolor{black!50}{0.00} \textbf{2.75} & 64 65 70 & 14 26 & 9 6 3\\
\index{GarridoOS08}\rowlabel{a:GarridoOS08}GarridoOS08 \href{https://doi.org/10.1016/j.engappai.2008.03.009}{GarridoOS08} & \hyperref[auth:a633]{A. Garrido}, \hyperref[auth:a635]{E. Onaindia}, \hyperref[auth:a640]{{\'{O}}scar Sapena} & Planning and scheduling in an e-learning environment. {A} constraint-programming-based approach & \hyperref[detail:GarridoOS08]{Details} \href{../works/GarridoOS08.pdf}{Yes} & \cite{GarridoOS08} & 2008 & Eng. Appl. Artif. Intell. & 11 & \noindent{}\textcolor{black!50}{0.00} \textcolor{black!50}{0.00} \textbf{15.57} & 22 22 28 & 7 24 & 4 3 1\\
\index{HladikCDJ08}\rowlabel{a:HladikCDJ08}HladikCDJ08 \href{http://dx.doi.org/10.1016/j.jss.2007.02.032}{HladikCDJ08} & \hyperref[auth:a1060]{P.-E. Hladik}, \hyperref[auth:a998]{H. Cambazard}, \hyperref[auth:a1161]{A.-M. Déplanche}, \hyperref[auth:a247]{N. Jussien} & \cellcolor{green!10}Solving a real-time allocation problem with constraint programming & \hyperref[detail:HladikCDJ08]{Details} \href{../works/HladikCDJ08.pdf}{Yes} & \cite{HladikCDJ08} & 2008 & Journal of Systems and Software & 18 & \noindent{}\textcolor{black!50}{0.00} \textcolor{black!50}{0.00} \textbf{9.26} & 36 37 48 & 27 66 & 16 8 8\\
\index{KovacsB08}\rowlabel{a:KovacsB08}KovacsB08 \href{https://doi.org/10.1016/j.engappai.2008.03.004}{KovacsB08} & \hyperref[auth:a146]{A. Kov{\'{a}}cs}, \hyperref[auth:a89]{J. C. Beck} & \cellcolor{green!10}A global constraint for total weighted completion time for cumulative resources & \hyperref[detail:KovacsB08]{Details} \href{../works/KovacsB08.pdf}{Yes} & \cite{KovacsB08} & 2008 & Eng. Appl. Artif. Intell. & 7 & \noindent{}\textcolor{black!50}{0.00} \textcolor{black!50}{0.00} \textbf{2.37} & 5 5 5 & 14 20 & 5 1 4\\
\index{LiW08}\rowlabel{a:LiW08}LiW08 \href{http://dx.doi.org/10.1007/s10951-008-0079-3}{LiW08} & \hyperref[auth:a952]{H. Li}, \hyperref[auth:a953]{K. Womer} & Scheduling projects with multi-skilled personnel by a hybrid MILP/CP benders decomposition algorithm & \hyperref[detail:LiW08]{Details} \href{../works/LiW08.pdf}{Yes} & \cite{LiW08} & 2008 & Journal of Scheduling & 18 & \noindent{}\textbf{1.00} \textbf{1.00} \textbf{23.20} & 113 123 144 & 31 52 & 25 10 15\\
\index{LiessM08}\rowlabel{a:LiessM08}LiessM08 \href{https://doi.org/10.1007/s10479-007-0188-y}{LiessM08} & \hyperref[auth:a639]{O. Liess}, \hyperref[auth:a355]{P. Michelon} & A constraint programming approach for the resource-constrained project scheduling problem & \hyperref[detail:LiessM08]{Details} \href{../works/LiessM08.pdf}{Yes} & \cite{LiessM08} & 2008 & Annals of Operations Research & 12 & \noindent{}\textbf{1.50} \textbf{1.50} \textbf{2.13} & 22 25 28 & 14 17 & 17 9 8\\
\index{Magato2008}\rowlabel{a:Magato2008}Magato2008 \href{http://dx.doi.org/10.1590/s0101-74382008000300007}{Magato2008} & \hyperref[auth:a1637]{L. Magatão}, \hyperref[auth:a1638]{Lúcia Valéria Ramos de Arruda}, \hyperref[auth:a1639]{F. Neves-Jr} & \cellcolor{gold!20}Um modelo híbrido (CLP-MILP) para scheduling de operações em polidutos \hyperref[abs:Magato2008]{Abstract} & \hyperref[detail:Magato2008]{Details} No & \cite{Magato2008} & 2008 & Pesquisa Operacional & null & \noindent{}\textbf{1.00} \textbf{2.00} n/a & 1 1 4 & 24 42 & 4 0 4\\
\index{MalikMB08}\rowlabel{a:MalikMB08}MalikMB08 \href{https://doi.org/10.1142/S0218213008003765}{MalikMB08} & \hyperref[auth:a638]{A. M. Malik}, \hyperref[auth:a641]{J. McInnes}, \hyperref[auth:a610]{P. van Beek} & Optimal Basic Block Instruction Scheduling for Multiple-Issue Processors Using Constraint Programming & \hyperref[detail:MalikMB08]{Details} \href{../works/MalikMB08.pdf}{Yes} & \cite{MalikMB08} & 2008 & Int. J. Artif. Intell. Tools & 18 & \noindent{}\textbf{1.00} \textbf{1.00} \textbf{2.02} & 15 14 14 & 8 12 & 6 5 1\\
\index{MercierH08}\rowlabel{a:MercierH08}MercierH08 \href{http://dx.doi.org/10.1287/ijoc.1070.0226}{MercierH08} & \hyperref[auth:a851]{L. Mercier}, \hyperref[auth:a148]{P. V. Hentenryck} & Edge Finding for Cumulative Scheduling & \hyperref[detail:MercierH08]{Details} \href{../works/MercierH08.pdf}{Yes} & \cite{MercierH08} & 2008 & \cellcolor{red!20}INFORMS Journal on Computing & 11 & \noindent{}\textcolor{black!50}{0.00} \textcolor{black!50}{0.00} 0.71 & 32 33 0 & 5 8 & 31 26 5\\
\index{Petrovic2008}\rowlabel{a:Petrovic2008}Petrovic2008 \href{http://dx.doi.org/10.1142/s0218213008004023}{Petrovic2008} & \hyperref[auth:a1861]{S. Petrovic}, \hyperref[auth:a1862]{S. L. Epstein} & \cellcolor{green!10}Random subsets support learning a mixture of heuristics \hyperref[abs:Petrovic2008]{Abstract} & \hyperref[detail:Petrovic2008]{Details} No & \cite{Petrovic2008} & 2008 & International Journal on Artificial Intelligence Tools & null & \noindent{}\textcolor{black!50}{0.00} 0.50 n/a & 11 11 11 & 5 12 & 1 1 0\\
\index{Salido2008}\rowlabel{a:Salido2008}Salido2008 \href{http://dx.doi.org/10.1016/j.engappai.2008.03.007}{Salido2008} & \hyperref[auth:a153]{M. A. Salido}, \hyperref[auth:a633]{A. Garrido}, \hyperref[auth:a1063]{R. Barták} & Introduction: Special issue on constraint satisfaction techniques for planning and scheduling problems & \hyperref[detail:Salido2008]{Details} No & \cite{Salido2008} & 2008 & Engineering Applications of Artificial Intelligence & null & \noindent{}\textbf{1.00} \textbf{1.00} n/a & 12 13 16 & 6 7 & 6 4 2\\
\index{Salido2008a}\rowlabel{a:Salido2008a}Salido2008a \href{http://dx.doi.org/10.1016/j.engappai.2008.03.006}{Salido2008a} & \hyperref[auth:a153]{M. A. Salido}, \hyperref[auth:a1941]{A. Giret} & \cellcolor{green!10}Feasible distributed CSP models for scheduling problems & \hyperref[detail:Salido2008a]{Details} No & \cite{Salido2008a} & 2008 & Engineering Applications of Artificial Intelligence & null & \noindent{}\textbf{1.00} \textbf{1.00} n/a & 8 8 10 & 14 25 & 1 1 0\\
\index{Tseng2008}\rowlabel{a:Tseng2008}Tseng2008 \href{http://dx.doi.org/10.1145/1367045.1367061}{Tseng2008} & \hyperref[auth:a1682]{I.-L. Tseng}, \hyperref[auth:a1683]{A. Postula} & Partitioning parameterized 45-degree polygons with constraint programming \hyperref[abs:Tseng2008]{Abstract} & \hyperref[detail:Tseng2008]{Details} No & \cite{Tseng2008} & 2008 & ACM Transactions on Design Automation of Electronic Systems & null & \noindent{}\textcolor{black!50}{0.00} 0.50 n/a & 6 6 9 & 18 30 & 2 0 2\\
\index{Wallace2008}\rowlabel{a:Wallace2008}Wallace2008 \href{http://dx.doi.org/10.1142/s0218213008004199}{Wallace2008} & \hyperref[auth:a1268]{R. J. Wallace} & Determining the principles underlying performance variation in csp heuristics \hyperref[abs:Wallace2008]{Abstract} & \hyperref[detail:Wallace2008]{Details} No & \cite{Wallace2008} & 2008 & International Journal on Artificial Intelligence Tools & null & \noindent{}\textcolor{black!50}{0.00} 0.25 n/a & 3 3 4 & 3 8 & 1 1 0\\
\index{Wu2008}\rowlabel{a:Wu2008}Wu2008 \href{http://dx.doi.org/10.1142/s0218213008004187}{Wu2008} & \hyperref[auth:a2060]{H. Wu}, \hyperref[auth:a2061]{P. V. Beek} & Portfolios with deadlines for backtracking search \hyperref[abs:Wu2008]{Abstract} & \hyperref[detail:Wu2008]{Details} No & \cite{Wu2008} & 2008 & International Journal on Artificial Intelligence Tools & null & \noindent{}\textcolor{black!50}{0.00} \textbf{1.50} n/a & 5 5 6 & 13 14 & 2 1 1\\
\index{ArtiguesF07}\rowlabel{a:ArtiguesF07}ArtiguesF07 \href{http://dx.doi.org/10.1007/s10479-007-0283-0}{ArtiguesF07} & \hyperref[auth:a6]{C. Artigues}, \hyperref[auth:a356]{D. Feillet} & \cellcolor{green!10}A branch and bound method for the job-shop problem with sequence-dependent setup times & \hyperref[detail:ArtiguesF07]{Details} \href{../works/ArtiguesF07.pdf}{Yes} & \cite{ArtiguesF07} & 2007 & Annals of Operations Research & 25 & \noindent{}\textcolor{black!50}{0.00} \textcolor{black!50}{0.00} \textbf{6.23} & 49 49 66 & 32 46 & 19 11 8\\
\index{Beck07}\rowlabel{a:Beck07}Beck07 \href{https://doi.org/10.1613/jair.2169}{Beck07} & \hyperref[auth:a89]{J. C. Beck} & \cellcolor{gold!20}Solution-Guided Multi-Point Constructive Search for Job Shop Scheduling & \hyperref[detail:Beck07]{Details} \href{../works/Beck07.pdf}{Yes} & \cite{Beck07} & 2007 & J. Artif. Intell. Res. & 29 & \noindent{}\textcolor{black!50}{0.00} \textcolor{black!50}{0.00} \textbf{2.01} & 34 34 57 & 0 0 & 15 15 0\\
\index{BeckW07}\rowlabel{a:BeckW07}BeckW07 \href{https://doi.org/10.1613/jair.2080}{BeckW07} & \hyperref[auth:a89]{J. C. Beck}, \hyperref[auth:a826]{N. Wilson} & \cellcolor{gold!20}Proactive Algorithms for Job Shop Scheduling with Probabilistic Durations & \hyperref[detail:BeckW07]{Details} \href{../works/BeckW07.pdf}{Yes} & \cite{BeckW07} & 2007 & J. Artif. Intell. Res. & 50 & \noindent{}\textcolor{black!50}{0.00} \textcolor{black!50}{0.00} \textbf{8.34} & 27 31 61 & 0 0 & 4 4 0\\
\index{Choi2007}\rowlabel{a:Choi2007}Choi2007 \href{http://dx.doi.org/10.1145/1276920.1276925}{Choi2007} & \hyperref[auth:a1816]{C. W. Choi}, \hyperref[auth:a1817]{J. H. M. Lee}, \hyperref[auth:a1818]{P. J. Stuckey} & \cellcolor{green!10}Removing propagation redundant constraints in redundant modeling \hyperref[abs:Choi2007]{Abstract} & \hyperref[detail:Choi2007]{Details} No & \cite{Choi2007} & 2007 & ACM Transactions on Computational Logic & null & \noindent{}\textcolor{black!50}{0.00} \textbf{1.25} n/a & 8 8 12 & 10 32 & 4 1 3\\
\index{CorreaLR07}\rowlabel{a:CorreaLR07}CorreaLR07 \href{http://dx.doi.org/10.1016/j.cor.2005.07.004}{CorreaLR07} & \hyperref[auth:a948]{A. I. Corr{\'{e}}a}, \hyperref[auth:a645]{A. Langevin}, \hyperref[auth:a326]{L.-M. Rousseau} & Scheduling and routing of automated guided vehicles: A hybrid approach & \hyperref[detail:CorreaLR07]{Details} \href{../works/CorreaLR07.pdf}{Yes} & \cite{CorreaLR07} & 2007 & Computers \  Operations Research & 20 & \noindent{}\textcolor{black!50}{0.00} \textcolor{black!50}{0.00} \textbf{9.26} & 106 114 137 & 20 28 & 13 4 9\\
\index{Hooker07}\rowlabel{a:Hooker07}Hooker07 \href{http://dx.doi.org/10.1287/opre.1060.0371}{Hooker07} & \hyperref[auth:a160]{J. N. Hooker} & Planning and Scheduling by Logic-Based Benders Decomposition & \hyperref[detail:Hooker07]{Details} \href{../works/Hooker07.pdf}{Yes} & \cite{Hooker07} & 2007 & \cellcolor{red!20}Operations Research & 15 & \noindent{}\textcolor{black!50}{0.00} \textcolor{black!50}{0.00} \textbf{14.07} & 181 197 205 & 19 20 & 66 52 14\\
\index{Lallouet2007}\rowlabel{a:Lallouet2007}Lallouet2007 \href{http://dx.doi.org/10.1142/s0218213007003503}{Lallouet2007} & \hyperref[auth:a428]{A. Lallouet}, \hyperref[auth:a1935]{A. Legtchenko} & Building consistencies for partially defined constraints with decision trees and neural networks \hyperref[abs:Lallouet2007]{Abstract} & \hyperref[detail:Lallouet2007]{Details} No & \cite{Lallouet2007} & 2007 & International Journal on Artificial Intelligence Tools & null & \noindent{}\textcolor{black!50}{0.00} 0.25 n/a & 3 3 6 & 4 12 & 2 1 1\\
\index{MercierH07}\rowlabel{a:MercierH07}MercierH07 \href{http://dx.doi.org/10.1016/j.disopt.2007.01.001}{MercierH07} & \hyperref[auth:a851]{L. Mercier}, \hyperref[auth:a148]{P. V. Hentenryck} & \cellcolor{gold!20}Strong polynomiality of resource constraint propagation & \hyperref[detail:MercierH07]{Details} \href{../works/MercierH07.pdf}{Yes} & \cite{MercierH07} & 2007 & Discrete Optimization & 27 & \noindent{}0.75 0.75 \textbf{11.75} & 5 5 7 & 8 17 & 11 4 7\\
\index{Mladenovic2007}\rowlabel{a:Mladenovic2007}Mladenovic2007 \href{http://dx.doi.org/10.2298/yjor0701009m}{Mladenovic2007} & \hyperref[auth:a1621]{S. Mladenovic}, \hyperref[auth:a1717]{M. Cangalovic} & \cellcolor{gold!20}Heuristic approach to train rescheduling \hyperref[abs:Mladenovic2007]{Abstract} & \hyperref[detail:Mladenovic2007]{Details} No & \cite{Mladenovic2007} & 2007 & Yugoslav Journal of Operations Research & null & \noindent{}\textcolor{black!50}{0.00} \textbf{3.00} n/a & 7 6 9 & 3 3 & 2 2 0\\
\index{RasmussenT07}\rowlabel{a:RasmussenT07}RasmussenT07 \href{http://dx.doi.org/10.1016/j.ejor.2005.10.063}{RasmussenT07} & \hyperref[auth:a1403]{R. V. Rasmussen}, \hyperref[auth:a1389]{M. A. Trick} & A Benders approach for the constrained minimum break problem \hyperref[abs:RasmussenT07]{Abstract} & \hyperref[detail:RasmussenT07]{Details} \href{../works/RasmussenT07.pdf}{Yes} & \cite{RasmussenT07} & 2007 & European Journal of Operational Research & 16 & \noindent{}\textcolor{black!50}{0.00} \textcolor{black!50}{0.00} 0.52 & 60 62 71 & 16 27 & 21 14 7\\
\index{Rodriguez07}\rowlabel{a:Rodriguez07}Rodriguez07 \href{https://www.sciencedirect.com/science/article/pii/S0191261506000233}{Rodriguez07} & \hyperref[auth:a781]{J. Rodriguez} & A constraint programming model for real-time train scheduling at junctions \hyperref[abs:Rodriguez07]{Abstract} & \hyperref[detail:Rodriguez07]{Details} \href{../works/Rodriguez07.pdf}{Yes} & \cite{Rodriguez07} & 2007 & Transportation Research Part B: Methodological & 15 & \noindent{}\textbf{1.00} \textbf{1.50} \textbf{6.88} & 117 121 141 & 6 14 & 11 9 2\\
\index{Simonis07}\rowlabel{a:Simonis07}Simonis07 \href{https://doi.org/10.1007/s10601-006-9011-7}{Simonis07} & \hyperref[auth:a17]{H. Simonis} & Models for Global Constraint Applications & \hyperref[detail:Simonis07]{Details} \href{../works/Simonis07.pdf}{Yes} & \cite{Simonis07} & 2007 & Constraints An Int. J. & 30 & \noindent{}\textcolor{black!50}{0.00} \textcolor{black!50}{0.00} \textbf{10.93} & 10 11 19 & 17 74 & 14 3 11\\
\index{Wang2007}\rowlabel{a:Wang2007}Wang2007 \href{http://dx.doi.org/10.1016/j.omega.2005.06.001}{Wang2007} & \hyperref[auth:a1936]{S. M. Wang}, \hyperref[auth:a1937]{J. C. Chen}, \hyperref[auth:a1938]{K.-J. Wang} & Resource portfolio planning of make-to-stock products using a constraint programming-based genetic algorithm & \hyperref[detail:Wang2007]{Details} No & \cite{Wang2007} & 2007 & Omega & null & \noindent{}0.50 0.50 n/a & 26 28 32 & 13 17 & 1 1 0\\
\index{Bidot2006}\rowlabel{a:Bidot2006}Bidot2006 \href{http://dx.doi.org/10.3182/20060517-3-fr-2903.00313}{Bidot2006} & \hyperref[auth:a824]{J. Bidot}, \hyperref[auth:a118]{P. Laborie}, \hyperref[auth:a89]{J. C. Beck}, \hyperref[auth:a825]{T. Vidal} & \cellcolor{gold!20}Using constraint programming and simulation for execution monitoring and progressive scheduling & \hyperref[detail:Bidot2006]{Details} No & \cite{Bidot2006} & 2006 & IFAC Proceedings Volumes & null & \noindent{}\textbf{1.00} \textbf{1.00} n/a & 2 2 4 & 2 6 & 1 1 0\\
\index{BockmayrP06}\rowlabel{a:BockmayrP06}BockmayrP06 \href{http://dx.doi.org/10.1016/j.cor.2005.01.010}{BockmayrP06} & \hyperref[auth:a908]{A. Bockmayr}, \hyperref[auth:a1178]{N. Pisaruk} & Detecting infeasibility and generating cuts for mixed integer programming using constraint programming & \hyperref[detail:BockmayrP06]{Details} \href{../works/BockmayrP06.pdf}{Yes} & \cite{BockmayrP06} & 2006 & Computers \  Operations Research & 10 & \noindent{}\textcolor{black!50}{0.00} \textcolor{black!50}{0.00} \textbf{2.66} & 12 12 10 & 7 10 & 9 4 5\\
\index{Elkhyari2006}\rowlabel{a:Elkhyari2006}Elkhyari2006 \href{http://dx.doi.org/10.3182/20060517-3-fr-2903.00358}{Elkhyari2006} & \hyperref[auth:a292]{A. Elkhyari}, \hyperref[auth:a2069]{C. Combes} & Using constraint programming for solving dynamic scheduling in the endoscopy unit & \hyperref[detail:Elkhyari2006]{Details} No & \cite{Elkhyari2006} & 2006 & IFAC Proceedings Volumes & null & \noindent{}\textbf{1.00} \textbf{1.00} n/a & 1 1 1 & 1 8 & 1 1 0\\
\index{Frisch2006}\rowlabel{a:Frisch2006}Frisch2006 \href{http://dx.doi.org/10.1016/j.artint.2006.03.002}{Frisch2006} & \hyperref[auth:a1666]{A. M. Frisch}, \hyperref[auth:a137]{B. Hnich}, \hyperref[auth:a97]{Z. Kiziltan}, \hyperref[auth:a1667]{I. Miguel}, \hyperref[auth:a276]{T. Walsh} & \cellcolor{gold!20}Propagation algorithms for lexicographic ordering constraints & \hyperref[detail:Frisch2006]{Details} No & \cite{Frisch2006} & 2006 & Artificial Intelligence & null & \noindent{}0.25 0.25 n/a & 23 23 31 & 9 29 & 5 2 3\\
\index{Gronkvist06}\rowlabel{a:Gronkvist06}Gronkvist06 \href{http://dx.doi.org/10.1016/j.cor.2005.01.017}{Gronkvist06} & \hyperref[auth:a1214]{M. Gr\"{o}nkvist} & Accelerating column generation for aircraft scheduling using constraint propagation & \hyperref[detail:Gronkvist06]{Details} \href{../works/Gronkvist06.pdf}{Yes} & \cite{Gronkvist06} & 2006 & Computers \  Operations Research & 17 & \noindent{}\textbf{1.50} \textbf{1.50} \textbf{2.29} & 28 28 36 & 15 30 & 7 5 2\\
\index{Hooker06}\rowlabel{a:Hooker06}Hooker06 \href{https://doi.org/10.1007/s10601-006-8060-2}{Hooker06} & \hyperref[auth:a160]{J. N. Hooker} & \cellcolor{green!10}An Integrated Method for Planning and Scheduling to Minimize Tardiness & \hyperref[detail:Hooker06]{Details} \href{../works/Hooker06.pdf}{Yes} & \cite{Hooker06} & 2006 & Constraints An Int. J. & 19 & \noindent{}\textcolor{black!50}{0.00} \textcolor{black!50}{0.00} \textbf{8.82} & 19 20 27 & 13 20 & 24 16 8\\
\index{KhayatLR06}\rowlabel{a:KhayatLR06}KhayatLR06 \href{https://doi.org/10.1016/j.ejor.2005.02.077}{KhayatLR06} & \hyperref[auth:a644]{G. E. Khayat}, \hyperref[auth:a645]{A. Langevin}, \hyperref[auth:a646]{D. Riopel} & Integrated production and material handling scheduling using mathematical programming and constraint programming & \hyperref[detail:KhayatLR06]{Details} \href{../works/KhayatLR06.pdf}{Yes} & \cite{KhayatLR06} & 2006 & European Journal of Operational Research & 15 & \noindent{}\textbf{1.00} \textbf{1.00} \textbf{19.97} & 84 89 96 & 14 26 & 14 12 2\\
\index{MilanoW06}\rowlabel{a:MilanoW06}MilanoW06 \href{http://dx.doi.org/10.1007/s10288-006-0019-z}{MilanoW06} & \hyperref[auth:a143]{M. Milano}, \hyperref[auth:a117]{M. G. Wallace} & Integrating operations research in constraint programming & \hyperref[detail:MilanoW06]{Details} \href{../works/MilanoW06.pdf}{Yes} & \cite{MilanoW06} & 2006 & 4OR & 45 & \noindent{}\textcolor{black!50}{0.00} \textcolor{black!50}{0.00} \textbf{47.13} & 18 18 22 & 46 67 & 20 4 16\\
\index{RussellU06}\rowlabel{a:RussellU06}RussellU06 \href{http://dx.doi.org/10.1016/j.cor.2004.09.029}{RussellU06} & \hyperref[auth:a1433]{R. A. Russell}, \hyperref[auth:a1434]{T. L. Urban} & A constraint programming approach to the multiple-venue,  sport-scheduling problem & \hyperref[detail:RussellU06]{Details} \href{../works/RussellU06.pdf}{Yes} & \cite{RussellU06} & 2006 & Computers \  Operations Research & 12 & \noindent{}\textbf{1.00} \textbf{1.00} \textbf{2.33} & 22 22 0 & 16 22 & 11 6 5\\
\index{SadykovW06}\rowlabel{a:SadykovW06}SadykovW06 \href{https://doi.org/10.1287/ijoc.1040.0110}{SadykovW06} & \hyperref[auth:a384]{R. Sadykov}, \hyperref[auth:a224]{L. A. Wolsey} & Integer Programming and Constraint Programming in Solving a Multimachine Assignment Scheduling Problem with Deadlines and Release Dates & \hyperref[detail:SadykovW06]{Details} \href{../works/SadykovW06.pdf}{Yes} & \cite{SadykovW06} & 2006 & \cellcolor{red!20}INFORMS Journal on Computing & 9 & \noindent{}\textbf{1.50} \textbf{1.50} \textbf{8.63} & 45 46 38 & 6 9 & 18 15 3\\
\index{SureshMOK06}\rowlabel{a:SureshMOK06}SureshMOK06 \href{https://doi.org/10.1080/17445760600567842}{SureshMOK06} & \hyperref[auth:a647]{S. Sundaram}, \hyperref[auth:a648]{V. Mani}, \hyperref[auth:a649]{S. N. Omkar}, \hyperref[auth:a650]{H. J. Kim} & Divisible load scheduling in distributed system with buffer constraints: genetic algorithm and linear programming approach & \hyperref[detail:SureshMOK06]{Details} \href{../works/SureshMOK06.pdf}{Yes} & \cite{SureshMOK06} & 2006 & Int. J. Parallel Emergent Distributed Syst. & 19 & \noindent{}\textcolor{black!50}{0.00} \textcolor{black!50}{0.00} \textcolor{black!50}{0.00} & 12 12 13 & 23 39 & 0 0 0\\
\index{Trilling2006}\rowlabel{a:Trilling2006}Trilling2006 \href{http://dx.doi.org/10.3182/20060517-3-fr-2903.00340}{Trilling2006} & \hyperref[auth:a1656]{L. Trilling}, \hyperref[auth:a1657]{A. Guinet}, \hyperref[auth:a1658]{D. L. Magny} & Nurse scheduling using integer linear programming and constraint programming & \hyperref[detail:Trilling2006]{Details} No & \cite{Trilling2006} & 2006 & IFAC Proceedings Volumes & null & \noindent{}\textbf{1.00} \textbf{1.00} n/a & 25 25 0 & 11 13 & 4 3 1\\
\index{Zhu2006}\rowlabel{a:Zhu2006}Zhu2006 \href{http://dx.doi.org/10.1287/ijoc.1040.0121}{Zhu2006} & \hyperref[auth:a1528]{G. Zhu}, \hyperref[auth:a1529]{J. F. Bard}, \hyperref[auth:a1530]{G. Yu} & A Branch-and-Cut Procedure for the Multimode Resource-Constrained Project-Scheduling Problem \hyperref[abs:Zhu2006]{Abstract} & \hyperref[detail:Zhu2006]{Details} No & \cite{Zhu2006} & 2006 & \cellcolor{red!20}INFORMS Journal on Computing & null & \noindent{}\textcolor{black!50}{0.00} \textcolor{black!50}{0.00} n/a & 78 85 118 & 23 23 & 19 15 4\\
\index{DemasseyAM05}\rowlabel{a:DemasseyAM05}DemasseyAM05 \href{http://dx.doi.org/10.1287/ijoc.1030.0043}{DemasseyAM05} & \hyperref[auth:a243]{S. Demassey}, \hyperref[auth:a6]{C. Artigues}, \hyperref[auth:a355]{P. Michelon} & \cellcolor{green!10}Constraint-Propagation-Based Cutting Planes: An Application to the Resource-Constrained Project Scheduling Problem & \hyperref[detail:DemasseyAM05]{Details} \href{../works/DemasseyAM05.pdf}{Yes} & \cite{DemasseyAM05} & 2005 & \cellcolor{red!20}INFORMS Journal on Computing & 14 & \noindent{}\textbf{2.25} \textbf{2.25} \textbf{12.96} & 43 43 51 & 25 30 & 18 6 12\\
\index{Hooker05}\rowlabel{a:Hooker05}Hooker05 \href{https://doi.org/10.1007/s10601-005-2812-2}{Hooker05} & \hyperref[auth:a160]{J. N. Hooker} & \cellcolor{green!10}A Hybrid Method for the Planning and Scheduling & \hyperref[detail:Hooker05]{Details} \href{../works/Hooker05.pdf}{Yes} & \cite{Hooker05} & 2005 & Constraints An Int. J. & 17 & \noindent{}\textcolor{black!50}{0.00} \textcolor{black!50}{0.00} \textbf{7.71} & 68 69 87 & 11 18 & 40 30 10\\
\index{Moccia2005}\rowlabel{a:Moccia2005}Moccia2005 \href{http://dx.doi.org/10.1002/nav.20121}{Moccia2005} & \hyperref[auth:a1589]{L. Moccia}, \hyperref[auth:a1590]{J. Cordeau}, \hyperref[auth:a1591]{M. Gaudioso}, \hyperref[auth:a1074]{G. Laporte} & A branch‐and‐cut algorithm for the quay crane scheduling problem in a container terminal \hyperref[abs:Moccia2005]{Abstract} & \hyperref[detail:Moccia2005]{Details} No & \cite{Moccia2005} & 2005 & Naval Research Logistics (NRL) & null & \noindent{}\textcolor{black!50}{0.00} \textcolor{black!50}{0.00} n/a & 140 147 168 & 13 20 & 4 3 1\\
\index{RoePS05}\rowlabel{a:RoePS05}RoePS05 \href{http://dx.doi.org/10.1016/j.compchemeng.2005.02.024}{RoePS05} & \hyperref[auth:a1241]{B. Roe}, \hyperref[auth:a1242]{L. G. Papageorgiou}, \hyperref[auth:a1243]{N. Shah} & A hybrid MILP/CLP algorithm for multipurpose batch process scheduling & \hyperref[detail:RoePS05]{Details} \href{../works/RoePS05.pdf}{Yes} & \cite{RoePS05} & 2005 & Computers \  Chemical Engineering & 15 & \noindent{}\textbf{1.00} \textbf{1.00} \textbf{16.82} & 48 47 46 & 15 23 & 13 6 7\\
\index{VilimBC05}\rowlabel{a:VilimBC05}VilimBC05 \href{https://doi.org/10.1007/s10601-005-2814-0}{VilimBC05} & \hyperref[auth:a121]{P. Vil{\'{\i}}m}, \hyperref[auth:a152]{R. Bart{\'{a}}k}, \hyperref[auth:a161]{O. Cepek} & Extension of \emph{O}(\emph{n} log \emph{n}) Filtering Algorithms for the Unary Resource Constraint to Optional Activities & \hyperref[detail:VilimBC05]{Details} \href{../works/VilimBC05.pdf}{Yes} & \cite{VilimBC05} & 2005 & Constraints An Int. J. & 23 & \noindent{}\textcolor{black!50}{0.00} \textcolor{black!50}{0.00} \textbf{2.28} & 21 21 32 & 5 16 & 15 12 3\\
\index{Yunes2005}\rowlabel{a:Yunes2005}Yunes2005 \href{http://dx.doi.org/10.1287/trsc.1030.0078}{Yunes2005} & \hyperref[auth:a942]{T. H. Yunes}, \hyperref[auth:a1580]{A. V. Moura}, \hyperref[auth:a170]{C. C. de Souza} & Hybrid Column Generation Approaches for Urban Transit Crew Management Problems \hyperref[abs:Yunes2005]{Abstract} & \hyperref[detail:Yunes2005]{Details} No & \cite{Yunes2005} & 2005 & \cellcolor{red!20}Transportation Science & null & \noindent{}\textcolor{black!50}{0.00} \textbf{2.00} n/a & 42 42 44 & 17 31 & 13 11 2\\
\index{ZeballosH05}\rowlabel{a:ZeballosH05}ZeballosH05 \href{http://journal.iberamia.org/index.php/ia/article/view/452/article\%20\%281\%29.pdf}{ZeballosH05} & \hyperref[auth:a621]{L. J. Zeballos}, \hyperref[auth:a588]{G. P. Henning} & \cellcolor{green!10}A Constraint Programming Approach to {FMS} Scheduling. Consideration of Storage and Transportation Resources & \hyperref[detail:ZeballosH05]{Details} \href{../works/ZeballosH05.pdf}{Yes} & \cite{ZeballosH05} & 2005 & Inteligencia Artif. & 10 & \noindent{}\textbf{1.50} \textbf{1.50} \textbf{8.59} & 0 0 0 & 0 0 & 0 0 0\\
\index{HenzMT04}\rowlabel{a:HenzMT04}HenzMT04 \href{http://dx.doi.org/10.1016/s0377-2217(03)00101-2}{HenzMT04} & \hyperref[auth:a1419]{M. Henz}, \hyperref[auth:a1421]{T. M\"{u}ller}, \hyperref[auth:a1422]{S. Thiel} & Global constraints for round robin tournament scheduling & \hyperref[detail:HenzMT04]{Details} \href{../works/HenzMT04.pdf}{Yes} & \cite{HenzMT04} & 2004 & European Journal of Operational Research & 10 & \noindent{}\textcolor{black!50}{0.00} \textcolor{black!50}{0.00} 0.69 & 44 47 0 & 8 24 & 12 10 2\\
\index{Hindi2004}\rowlabel{a:Hindi2004}Hindi2004 \href{http://dx.doi.org/10.1016/j.cie.2004.03.002}{Hindi2004} & \hyperref[auth:a1826]{K. S. Hindi}, \hyperref[auth:a1827]{K. Fleszar} & A constraint propagation heuristic for the single-hoist, multiple-products scheduling problem & \hyperref[detail:Hindi2004]{Details} No & \cite{Hindi2004} & 2004 & Computers \  Industrial Engineering & null & \noindent{}\textbf{1.50} \textbf{1.50} n/a & 29 30 34 & 8 12 & 1 1 0\\
\index{Kim2004}\rowlabel{a:Kim2004}Kim2004 \href{http://dx.doi.org/10.1016/j.cie.2003.12.017}{Kim2004} & \hyperref[auth:a2029]{K. H. Kim}, \hyperref[auth:a2030]{K. W. Kim}, \hyperref[auth:a2031]{H. Hwang}, \hyperref[auth:a2032]{C. S. Ko} & Operator-scheduling using a constraint satisfaction technique in port container terminals & \hyperref[detail:Kim2004]{Details} No & \cite{Kim2004} & 2004 & Computers \  Industrial Engineering & null & \noindent{}\textbf{1.00} \textbf{1.00} n/a & 22 23 29 & 9 9 & 1 0 1\\
\index{Lim2004}\rowlabel{a:Lim2004}Lim2004 \href{http://dx.doi.org/10.1002/nav.10123}{Lim2004} & \hyperref[auth:a279]{A. Lim}, \hyperref[auth:a280]{B. Rodrigues}, \hyperref[auth:a1743]{F. Xiao}, \hyperref[auth:a1744]{Y. Zhu} & Crane scheduling with spatial constraints \hyperref[abs:Lim2004]{Abstract} & \hyperref[detail:Lim2004]{Details} No & \cite{Lim2004} & 2004 & Naval Research Logistics (NRL) & null & \noindent{}\textcolor{black!50}{0.00} \textcolor{black!50}{0.00} n/a & 103 103 121 & 8 15 & 2 2 0\\
\index{MaraveliasCG04}\rowlabel{a:MaraveliasCG04}MaraveliasCG04 \href{http://dx.doi.org/10.1016/j.compchemeng.2004.03.016}{MaraveliasCG04} & \hyperref[auth:a381]{C. T. Maravelias}, \hyperref[auth:a382]{I. E. Grossmann} & A hybrid MILP/CP decomposition approach for the continuous time scheduling of multipurpose batch plants & \hyperref[detail:MaraveliasCG04]{Details} \href{../works/MaraveliasCG04.pdf}{Yes} & \cite{MaraveliasCG04} & 2004 & Computers \  Chemical Engineering & 29 & \noindent{}\textbf{1.00} \textbf{1.00} \textbf{49.17} & 116 119 130 & 24 29 & 29 23 6\\
\index{Michel2004}\rowlabel{a:Michel2004}Michel2004 \href{http://dx.doi.org/10.1145/976706.976714}{Michel2004} & \hyperref[auth:a32]{L. Michel}, \hyperref[auth:a148]{P. V. Hentenryck} & A decomposition-based implementation of search strategies \hyperref[abs:Michel2004]{Abstract} & \hyperref[detail:Michel2004]{Details} No & \cite{Michel2004} & 2004 & ACM Transactions on Computational Logic & null & \noindent{}\textcolor{black!50}{0.00} \textbf{2.00} n/a & 10 10 11 & 12 34 & 7 1 6\\
\index{OkanoDTRYA04}\rowlabel{a:OkanoDTRYA04}OkanoDTRYA04 \href{https://doi.org/10.1147/rd.485.0811}{OkanoDTRYA04} & \hyperref[auth:a1288]{H. Okano}, \hyperref[auth:a248]{A. J. Davenport}, \hyperref[auth:a1289]{M. Trumbo}, \hyperref[auth:a250]{C. Reddy}, \hyperref[auth:a1290]{K. Yoda}, \hyperref[auth:a1291]{M. Amano} & Finishing Line Scheduling in the steel industry & \hyperref[detail:OkanoDTRYA04]{Details} No & \cite{OkanoDTRYA04} & 2004 & {IBM} J. Res. Dev. & 20 & \noindent{}\textcolor{black!50}{0.00} \textcolor{black!50}{0.00} n/a & 19 20 26 & 0 0 & 0 0 0\\
\index{Ouaja2004}\rowlabel{a:Ouaja2004}Ouaja2004 \href{http://dx.doi.org/10.1002/net.10110}{Ouaja2004} & \hyperref[auth:a1548]{W. Ouaja}, \hyperref[auth:a1549]{B. Richards} & \cellcolor{gold!20}A hybrid multicommodity routing algorithm for traffic engineering \hyperref[abs:Ouaja2004]{Abstract} & \hyperref[detail:Ouaja2004]{Details} No & \cite{Ouaja2004} & 2004 & Networks & null & \noindent{}\textcolor{black!50}{0.00} \textbf{1.50} n/a & 14 14 20 & 10 31 & 3 1 2\\
\index{PoderBS04}\rowlabel{a:PoderBS04}PoderBS04 \href{https://doi.org/10.1016/S0377-2217(02)00756-7}{PoderBS04} & \hyperref[auth:a358]{E. Poder}, \hyperref[auth:a128]{N. Beldiceanu}, \hyperref[auth:a713]{E. Sanlaville} & Computing a lower approximation of the compulsory part of a task with varying duration and varying resource consumption & \hyperref[detail:PoderBS04]{Details} \href{../works/PoderBS04.pdf}{Yes} & \cite{PoderBS04} & 2004 & European Journal of Operational Research & 16 & \noindent{}\textcolor{black!50}{0.00} \textcolor{black!50}{0.00} \textbf{4.99} & 7 7 10 & 8 18 & 7 2 5\\
\index{BeckR03}\rowlabel{a:BeckR03}BeckR03 \href{https://doi.org/10.1023/A:1021849405707}{BeckR03} & \hyperref[auth:a89]{J. C. Beck}, \hyperref[auth:a254]{P. Refalo} & A Hybrid Approach to Scheduling with Earliness and Tardiness Costs & \hyperref[detail:BeckR03]{Details} \href{../works/BeckR03.pdf}{Yes} & \cite{BeckR03} & 2003 & Annals of Operations Research & 23 & \noindent{}\textcolor{black!50}{0.00} \textcolor{black!50}{0.00} \textbf{15.68} & 29 0 45 & 0 0 & 10 10 0\\
\index{ElfJR03}\rowlabel{a:ElfJR03}ElfJR03 \href{http://dx.doi.org/10.1016/s0167-6377(03)00025-7}{ElfJR03} & \hyperref[auth:a1406]{M. Elf}, \hyperref[auth:a1407]{M. Jünger}, \hyperref[auth:a1408]{G. Rinaldi} & \cellcolor{green!10}Minimizing breaks by maximizing cuts \hyperref[abs:ElfJR03]{Abstract} & \hyperref[detail:ElfJR03]{Details} \href{../works/ElfJR03.pdf}{Yes} & \cite{ElfJR03} & 2003 & OPERATIONS RESEARCH LETTERS & 7 & \noindent{}\textcolor{black!50}{0.00} \textbf{1.00} \textcolor{black!50}{0.09} & 41 41 45 & 7 10 & 8 6 2\\
\index{HookerO03}\rowlabel{a:HookerO03}HookerO03 \href{http://dx.doi.org/10.1007/s10107-003-0375-9}{HookerO03} & \hyperref[auth:a160]{J. N. Hooker}, \hyperref[auth:a852]{G. Ottosson} & \cellcolor{green!10}Logic-based Benders decomposition & \hyperref[detail:HookerO03]{Details} \href{../works/HookerO03.pdf}{Yes} & \cite{HookerO03} & 2003 & Mathematical Programming & 28 & \noindent{}\textcolor{black!50}{0.00} \textcolor{black!50}{0.00} 0.61 & 317 333 371 & 0 0 & 78 78 0\\
\index{Kovcs2003}\rowlabel{a:Kovcs2003}Kovcs2003 \href{http://dx.doi.org/10.1016/s1474-6670(17)37762-5}{Kovcs2003} & \hyperref[auth:a1880]{A. Kovács}, \hyperref[auth:a1881]{J. Váncza}, \hyperref[auth:a1882]{B. Kádár}, \hyperref[auth:a1883]{L. Monostori}, \hyperref[auth:a1884]{A. Pfeiffer} & Real-Life Scheduling Using Constraint Programming and Simulation & \hyperref[detail:Kovcs2003]{Details} No & \cite{Kovcs2003} & 2003 & IFAC Proceedings Volumes & null & \noindent{}\textbf{1.00} \textbf{1.00} n/a & 2 2 6 & 8 13 & 2 0 2\\
\index{Kuchcinski03}\rowlabel{a:Kuchcinski03}Kuchcinski03 \href{http://dx.doi.org/10.1145/785411.785416}{Kuchcinski03} & \hyperref[auth:a660]{K. Kuchcinski} & Constraints-driven scheduling and resource assignment & \hyperref[detail:Kuchcinski03]{Details} \href{../works/Kuchcinski03.pdf}{Yes} & \cite{Kuchcinski03} & 2003 & ACM Transactions on Design Automation of Electronic Systems & 29 & \noindent{}\textcolor{black!50}{0.00} \textcolor{black!50}{0.00} \textbf{11.59} & 105 105 116 & 15 42 & 13 11 2\\
\index{KuchcinskiW03}\rowlabel{a:KuchcinskiW03}KuchcinskiW03 \href{https://doi.org/10.1016/S1383-7621(03)00075-4}{KuchcinskiW03} & \hyperref[auth:a660]{K. Kuchcinski}, \hyperref[auth:a659]{C. Wolinski} & Global approach to assignment and scheduling of complex behaviors based on {HCDG} and constraint programming & \hyperref[detail:KuchcinskiW03]{Details} \href{../works/KuchcinskiW03.pdf}{Yes} & \cite{KuchcinskiW03} & 2003 & J. Syst. Archit. & 15 & \noindent{}\textbf{1.00} \textbf{1.00} \textbf{1.08} & 19 19 22 & 18 23 & 7 6 1\\
\index{Laborie03}\rowlabel{a:Laborie03}Laborie03 \href{http://dx.doi.org/10.1016/s0004-3702(02)00362-4}{Laborie03} & \hyperref[auth:a118]{P. Laborie} & \cellcolor{gold!20}Algorithms for propagating resource constraints in AI planning and scheduling: Existing approaches and new results & \hyperref[detail:Laborie03]{Details} \href{../works/Laborie03.pdf}{Yes} & \cite{Laborie03} & 2003 & Artificial Intelligence & 38 & \noindent{}\textcolor{black!50}{0.00} \textcolor{black!50}{0.00} \textbf{8.42} & 128 129 175 & 10 31 & 48 43 5\\
\index{Priore2003}\rowlabel{a:Priore2003}Priore2003 \href{http://dx.doi.org/10.1108/09576060310459456}{Priore2003} & \hyperref[auth:a1819]{P. Priore}, \hyperref[auth:a1820]{D. de la Fuente}, \hyperref[auth:a1821]{R. Pino}, \hyperref[auth:a1822]{J. Puente} & Dynamic scheduling of flexible manufacturing systems using neural networks and inductive learning \hyperref[abs:Priore2003]{Abstract} & \hyperref[detail:Priore2003]{Details} No & \cite{Priore2003} & 2003 & Integrated Manufacturing Systems & null & \noindent{}\textcolor{black!50}{0.00} \textbf{1.25} n/a & 15 15 14 & 23 26 & 1 1 0\\
\index{Sadykov2003}\rowlabel{a:Sadykov2003}Sadykov2003 \href{http://dx.doi.org/10.2139/ssrn.988640}{Sadykov2003} & \hyperref[auth:a384]{R. Sadykov}, \hyperref[auth:a224]{L. A. Wolsey} & Integer Programming and Constraint Programming in Solving a Multi-Machine Assignment Scheduling Problem With Deadlines and Release Dates & \hyperref[detail:Sadykov2003]{Details} No & \cite{Sadykov2003} & 2003 & SSRN Electronic Journal & null & \noindent{}\textbf{1.50} \textbf{1.50} n/a & 3 3 0 & 8 11 & 5 1 4\\
\index{Tsang03}\rowlabel{a:Tsang03}Tsang03 \href{https://doi.org/10.1023/A:1024016929283}{Tsang03} & \hyperref[auth:a665]{E. P. K. Tsang} & Constraint Based Scheduling: Applying Constraint Programming to Scheduling Problems & \hyperref[detail:Tsang03]{Details} \href{../works/Tsang03.pdf}{Yes} & \cite{Tsang03} & 2003 & Journal of Scheduling & 2 & \noindent{}\textbf{1.00} \textbf{1.00} 0.36 & 1 0 0 & 0 0 & 0 0 0\\
\index{Yan2003}\rowlabel{a:Yan2003}Yan2003 \href{http://dx.doi.org/10.1007/bf02948893}{Yan2003} & \hyperref[auth:a2033]{J. Yan}, \hyperref[auth:a2034]{C. Wu} & A constraint satisfaction neural network and heuristic combined approach for concurrent activities scheduling & \hyperref[detail:Yan2003]{Details} No & \cite{Yan2003} & 2003 & Journal of Computer Science and Technology & null & \noindent{}\textbf{1.00} \textbf{1.00} n/a & 3 3 1 & 7 12 & 1 0 1\\
\index{Younes2003}\rowlabel{a:Younes2003}Younes2003 \href{http://dx.doi.org/10.1613/jair.1136}{Younes2003} & \hyperref[auth:a1844]{H. L. S. Younes}, \hyperref[auth:a1845]{R. G. Simmons} & \cellcolor{gold!20}VHPOP: Versatile Heuristic Partial Order Planner \hyperref[abs:Younes2003]{Abstract} & \hyperref[detail:Younes2003]{Details} No & \cite{Younes2003} & 2003 & Journal of Artificial Intelligence Research & null & \noindent{}\textcolor{black!50}{0.00} 0.50 n/a & 54 55 128 & 0 0 & 1 1 0\\
\index{Brucker2002}\rowlabel{a:Brucker2002}Brucker2002 \href{http://dx.doi.org/10.1016/s0166-218x(01)00342-0}{Brucker2002} & \hyperref[auth:a847]{P. Brucker} & \cellcolor{gold!20}Scheduling and constraint propagation & \hyperref[detail:Brucker2002]{Details} No & \cite{Brucker2002} & 2002 & Discrete Applied Mathematics & null & \noindent{}\textbf{1.50} \textbf{1.50} n/a & 40 40 49 & 33 48 & 17 9 8\\
\index{Chan2002}\rowlabel{a:Chan2002}Chan2002 \href{http://dx.doi.org/10.1061/(asce)0733-9364(2002)128:6(513)}{Chan2002} & \hyperref[auth:a1662]{W. T. Chan}, \hyperref[auth:a1663]{H. Hu} & Constraint Programming Approach to Precast Production Scheduling & \hyperref[detail:Chan2002]{Details} No & \cite{Chan2002} & 2002 & Journal of Construction Engineering and Management & null & \noindent{}\textbf{1.00} \textbf{1.00} n/a & 65 69 86 & 9 19 & 10 8 2\\
\index{HarjunkoskiG02}\rowlabel{a:HarjunkoskiG02}HarjunkoskiG02 \href{http://dx.doi.org/10.1016/s0098-1354(02)00100-x}{HarjunkoskiG02} & \hyperref[auth:a871]{I. Harjunkoski}, \hyperref[auth:a382]{I. E. Grossmann} & Decomposition techniques for multistage scheduling problems using mixed-integer and constraint programming methods & \hyperref[detail:HarjunkoskiG02]{Details} \href{../works/HarjunkoskiG02.pdf}{Yes} & \cite{HarjunkoskiG02} & 2002 & Computers \  Chemical Engineering & 20 & \noindent{}\textbf{1.00} \textbf{1.00} \textbf{20.38} & 169 173 192 & 11 25 & 42 39 3\\
\index{Hooker02}\rowlabel{a:Hooker02}Hooker02 \href{http://dx.doi.org/10.1287/ijoc.14.4.295.2828}{Hooker02} & \hyperref[auth:a160]{J. N. Hooker} & Logic, Optimization, and Constraint Programming & \hyperref[detail:Hooker02]{Details} No & \cite{Hooker02} & 2002 & \cellcolor{red!20}INFORMS Journal on Computing & 27 & \noindent{}\textcolor{black!50}{0.00} \textcolor{black!50}{0.00} n/a & 94 93 0 & 84 149 & 40 22 18\\
\index{JussienL02}\rowlabel{a:JussienL02}JussienL02 \href{http://dx.doi.org/10.1016/s0004-3702(02)00221-7}{JussienL02} & \hyperref[auth:a247]{N. Jussien}, \hyperref[auth:a1072]{O. Lhomme} & \cellcolor{gold!20}Local search with constraint propagation and conflict-based heuristics & \hyperref[detail:JussienL02]{Details} \href{../works/JussienL02.pdf}{Yes} & \cite{JussienL02} & 2002 & Artificial Intelligence & 25 & \noindent{}\textcolor{black!50}{0.00} \textcolor{black!50}{0.00} \textbf{4.25} & 88 88 108 & 16 54 & 15 8 7\\
\index{Larrosa2002}\rowlabel{a:Larrosa2002}Larrosa2002 \href{http://dx.doi.org/10.1017/s0960129501003577}{Larrosa2002} & \hyperref[auth:a1794]{J. Larrosa}, \hyperref[auth:a1854]{G. Valiente} & Constraint satisfaction algorithms for graph  pattern matching \hyperref[abs:Larrosa2002]{Abstract} & \hyperref[detail:Larrosa2002]{Details} No & \cite{Larrosa2002} & 2002 & Mathematical Structures in Computer Science & null & \noindent{}\textcolor{black!50}{0.00} 0.50 n/a & 82 81 97 & 0 0 & 3 3 0\\
\index{LorigeonBB02}\rowlabel{a:LorigeonBB02}LorigeonBB02 \href{https://doi.org/10.1057/palgrave.jors.2601421}{LorigeonBB02} & \hyperref[auth:a671]{T. Lorigeon}, \hyperref[auth:a337]{J.-C. Billaut}, \hyperref[auth:a672]{J.-L. Bouquard} & A dynamic programming algorithm for scheduling jobs in a two-machine open shop with an availability constraint & \hyperref[detail:LorigeonBB02]{Details} \href{../works/LorigeonBB02.pdf}{Yes} & \cite{LorigeonBB02} & 2002 & \cellcolor{red!20}Journal of the Operational Research Society & 8 & \noindent{}\textcolor{black!50}{0.00} \textcolor{black!50}{0.00} \textcolor{black!50}{0.00} & 22 23 25 & 0 0 & 0 0 0\\
\index{MilanoORT02}\rowlabel{a:MilanoORT02}MilanoORT02 \href{http://dx.doi.org/10.1287/ijoc.14.4.387.2830}{MilanoORT02} & \hyperref[auth:a143]{M. Milano}, \hyperref[auth:a852]{G. Ottosson}, \hyperref[auth:a254]{P. Refalo}, \hyperref[auth:a874]{E. S. Thorsteinsson} & The Role of Integer Programming Techniques in Constraint Programming's Global Constraints & \hyperref[detail:MilanoORT02]{Details} No & \cite{MilanoORT02} & 2002 & \cellcolor{red!20}INFORMS Journal on Computing & 16 & \noindent{}\textcolor{black!50}{0.00} \textcolor{black!50}{0.00} n/a & 14 14 0 & 31 60 & 18 4 14\\
\index{RodriguezDG02}\rowlabel{a:RodriguezDG02}RodriguezDG02 \href{}{RodriguezDG02} & \hyperref[auth:a781]{J. Rodriguez}, \hyperref[auth:a782]{X. Delorme}, \hyperref[auth:a783]{X. Gandibleux} & Railway infrastructure saturation using constraint programming approach & \hyperref[detail:RodriguezDG02]{Details} \href{../works/RodriguezDG02.pdf}{Yes} & \cite{RodriguezDG02} & 2002 & Computers in Railways VIII & 10 & \noindent{}\textcolor{black!50}{0.00} \textcolor{black!50}{0.00} 0.38 & 0 0 0 & 0 0 & 0 0 0\\
\index{Timpe02}\rowlabel{a:Timpe02}Timpe02 \href{https://doi.org/10.1007/s00291-002-0107-1}{Timpe02} & \hyperref[auth:a673]{C. Timpe} & Solving planning and scheduling problems with combined integer and constraint programming & \hyperref[detail:Timpe02]{Details} \href{../works/Timpe02.pdf}{Yes} & \cite{Timpe02} & 2002 & {OR} Spectrum & 18 & \noindent{}\textbf{1.00} \textbf{1.00} \textbf{7.47} & 42 42 54 & 0 0 & 20 20 0\\
\index{YunG02}\rowlabel{a:YunG02}YunG02 \href{http://dx.doi.org/10.1016/s0360-8352(02)00065-7}{YunG02} & \hyperref[auth:a1472]{Y.-S. Yun}, \hyperref[auth:a1473]{M. Gen} & Advanced scheduling problem using constraint programming techniques in SCM environment & \hyperref[detail:YunG02]{Details} No & \cite{YunG02} & 2002 & Computers \  Industrial Engineering & 17 & \noindent{}\textbf{1.00} \textbf{1.00} n/a & 19 20 27 & 6 19 & 5 5 0\\
\index{Apt2001}\rowlabel{a:Apt2001}Apt2001 \href{http://dx.doi.org/10.1017/s1471068401000072}{Apt2001} & \hyperref[auth:a1887]{K. R. Apt}, \hyperref[auth:a1833]{E. Monfroy} & \cellcolor{green!10}Constraint programming viewed as rule-based programming \hyperref[abs:Apt2001]{Abstract} & \hyperref[detail:Apt2001]{Details} No & \cite{Apt2001} & 2001 & Theory and Practice of Logic Programming & null & \noindent{}\textcolor{black!50}{0.00} \textbf{1.75} n/a & 18 17 30 & 0 0 & 2 2 0\\
\index{BosiM2001}\rowlabel{a:BosiM2001}BosiM2001 \href{http://dx.doi.org/10.1002/1097-024x(200101)31:1<17::aid-spe355>3.0.co;2-l}{BosiM2001} & \hyperref[auth:a1224]{F. Bosi}, \hyperref[auth:a143]{M. Milano} & Enhancing CLP branch and bound techniques for scheduling problems & \hyperref[detail:BosiM2001]{Details} \href{../works/BosiM2001.pdf}{Yes} & \cite{BosiM2001} & 2001 & Software: Practice and Experience & 26 & \noindent{}\textbf{1.00} \textbf{1.00} \textbf{65.58} & 3 3 0 & 12 41 & 9 0 9\\
\index{Caseau2001}\rowlabel{a:Caseau2001}Caseau2001 \href{http://dx.doi.org/10.1017/s0269888901000078}{Caseau2001} & \hyperref[auth:a301]{Y. Caseau}, \hyperref[auth:a1513]{F. Laburthe}, \hyperref[auth:a163]{C. L. Pape}, \hyperref[auth:a1576]{B. Rottembourg} & Combining local and global search in a constraint programming environment \hyperref[abs:Caseau2001]{Abstract} & \hyperref[detail:Caseau2001]{Details} No & \cite{Caseau2001} & 2001 & The Knowledge Engineering Review & null & \noindent{}\textcolor{black!50}{0.00} \textbf{2.00} n/a & 10 11 12 & 0 0 & 4 4 0\\
\index{Farias2001}\rowlabel{a:Farias2001}Farias2001 \href{http://dx.doi.org/10.1017/s0269888901000030}{Farias2001} & \hyperref[auth:a1932]{I. R. D. Farias}, \hyperref[auth:a1933]{E. L. Johnson}, \hyperref[auth:a1934]{G. L. Nemhauser} & Branch-and-cut for combinatorial optimization problems without auxiliary binary variables \hyperref[abs:Farias2001]{Abstract} & \hyperref[detail:Farias2001]{Details} No & \cite{Farias2001} & 2001 & The Knowledge Engineering Review & null & \noindent{}\textcolor{black!50}{0.00} \textbf{1.00} n/a & 33 33 35 & 0 0 & 2 2 0\\
\index{Henz01}\rowlabel{a:Henz01}Henz01 \href{http://dx.doi.org/10.1287/opre.49.1.163.11193}{Henz01} & \hyperref[auth:a1419]{M. Henz} & Scheduling a Major College Basketball Conference—Revisited & \hyperref[detail:Henz01]{Details} No & \cite{Henz01} & 2001 & \cellcolor{red!20}Operations Research & 6 & \noindent{}\textcolor{black!50}{0.00} \textcolor{black!50}{0.00} n/a & 65 68 0 & 9 16 & 18 14 4\\
\index{Hofe2001}\rowlabel{a:Hofe2001}Hofe2001 \href{http://dx.doi.org/10.1142/s0129054101000710}{Hofe2001} & \hyperref[auth:a2012]{H. M. A. Hofe} & Solving rostering tasks by generic methods for constraint optimization \hyperref[abs:Hofe2001]{Abstract} & \hyperref[detail:Hofe2001]{Details} No & \cite{Hofe2001} & 2001 & International Journal of Foundations of Computer Science & null & \noindent{}\textbf{1.00} \textbf{1.50} n/a & 3 3 6 & 3 4 & 1 0 1\\
\index{JainG01}\rowlabel{a:JainG01}JainG01 \href{http://dx.doi.org/10.1287/ijoc.13.4.258.9733}{JainG01} & \hyperref[auth:a844]{V. Jain}, \hyperref[auth:a382]{I. E. Grossmann} & Algorithms for Hybrid MILP/CP Models for a Class of Optimization Problems & \hyperref[detail:JainG01]{Details} \href{../works/JainG01.pdf}{Yes} & \cite{JainG01} & 2001 & \cellcolor{red!20}INFORMS Journal on Computing & 19 & \noindent{}\textcolor{black!50}{0.00} \textcolor{black!50}{0.00} \textbf{29.84} & 279 284 321 & 23 38 & 101 89 12\\
\index{MartinPY01}\rowlabel{a:MartinPY01}MartinPY01 \href{https://doi.org/10.1023/A:1016067230126}{MartinPY01} & \hyperref[auth:a676]{F. Martin}, \hyperref[auth:a677]{A. Pinkney}, \hyperref[auth:a678]{X. Yu} & Cane Railway Scheduling via Constraint Logic Programming: Labelling Order and Constraints in a Real-Life Application & \hyperref[detail:MartinPY01]{Details} \href{../works/MartinPY01.pdf}{Yes} & \cite{MartinPY01} & 2001 & Annals of Operations Research & 17 & \noindent{}\textbf{1.50} \textbf{1.50} \textbf{5.69} & 11 0 14 & 0 0 & 0 0 0\\
\index{Mason01}\rowlabel{a:Mason01}Mason01 \href{https://doi.org/10.1023/A:1016023415105}{Mason01} & \hyperref[auth:a679]{A. J. Mason} & Elastic Constraint Branching, the Wedelin/Carmen Lagrangian Heuristic and Integer Programming for Personnel Scheduling & \hyperref[detail:Mason01]{Details} \href{../works/Mason01.pdf}{Yes} & \cite{Mason01} & 2001 & Annals of Operations Research & 38 & \noindent{}\textcolor{black!50}{0.00} \textcolor{black!50}{0.00} \textcolor{black!50}{0.00} & 5 0 6 & 0 0 & 0 0 0\\
\index{TrentesauxPT01}\rowlabel{a:TrentesauxPT01}TrentesauxPT01 \href{https://www.sciencedirect.com/science/article/pii/S0360835200000784}{TrentesauxPT01} & \hyperref[auth:a1457]{D. Trentesaux}, \hyperref[auth:a1458]{P. Pesin}, \hyperref[auth:a1459]{C. Tahon} & Comparison of constraint logic programming and distributed problem solving: a case study for interactive, efficient and practicable job-shop scheduling \hyperref[abs:TrentesauxPT01]{Abstract} & \hyperref[detail:TrentesauxPT01]{Details} \href{../works/TrentesauxPT01.pdf}{Yes} & \cite{TrentesauxPT01} & 2001 & Computers \  Industrial Engineering & 25 & \noindent{}\textbf{2.00} \textbf{5.00} \textbf{22.74} & 7 7 9 & 9 26 & 3 1 2\\
\index{ArtiguesR00}\rowlabel{a:ArtiguesR00}ArtiguesR00 \href{https://doi.org/10.1016/S0377-2217(99)00496-8}{ArtiguesR00} & \hyperref[auth:a6]{C. Artigues}, \hyperref[auth:a712]{F. Roubellat} & A polynomial activity insertion algorithm in a multi-resource schedule with cumulative constraints and multiple modes & \hyperref[detail:ArtiguesR00]{Details} \href{../works/ArtiguesR00.pdf}{Yes} & \cite{ArtiguesR00} & 2000 & European Journal of Operational Research & 20 & \noindent{}\textcolor{black!50}{0.00} \textcolor{black!50}{0.00} \textcolor{black!50}{0.00} & 84 85 86 & 3 8 & 3 3 0\\
\index{BaptisteP00}\rowlabel{a:BaptisteP00}BaptisteP00 \href{https://doi.org/10.1023/A:1009822502231}{BaptisteP00} & \hyperref[auth:a162]{P. Baptiste}, \hyperref[auth:a163]{C. L. Pape} & Constraint Propagation and Decomposition Techniques for Highly Disjunctive and Highly Cumulative Project Scheduling Problems & \hyperref[detail:BaptisteP00]{Details} \href{../works/BaptisteP00.pdf}{Yes} & \cite{BaptisteP00} & 2000 & Constraints An Int. J. & 21 & \noindent{}\textbf{1.50} \textbf{1.50} \textbf{14.93} & 46 0 62 & 0 0 & 29 29 0\\
\index{BeckF00}\rowlabel{a:BeckF00}BeckF00 \href{https://doi.org/10.1016/S0004-3702(99)00099-5}{BeckF00} & \hyperref[auth:a89]{J. C. Beck}, \hyperref[auth:a302]{M. S. Fox} & \cellcolor{gold!20}Dynamic problem structure analysis as a basis for constraint-directed scheduling heuristics & \hyperref[detail:BeckF00]{Details} \href{../works/BeckF00.pdf}{Yes} & \cite{BeckF00} & 2000 & Artificial Intelligence & 51 & \noindent{}\textcolor{black!50}{0.00} \textcolor{black!50}{0.00} \textbf{13.43} & 24 24 36 & 19 76 & 19 10 9\\
\index{BeckF00a}\rowlabel{a:BeckF00a}BeckF00a \href{http://dx.doi.org/10.1016/s0004-3702(00)00035-7}{BeckF00a} & \hyperref[auth:a89]{J. C. Beck}, \hyperref[auth:a302]{M. S. Fox} & \cellcolor{gold!20}Constraint-directed techniques for scheduling alternative activities & \hyperref[detail:BeckF00a]{Details} \href{../works/BeckF00a.pdf}{Yes} & \cite{BeckF00a} & 2000 & Artificial Intelligence & 40 & \noindent{}\textcolor{black!50}{0.00} \textcolor{black!50}{0.00} \textbf{17.71} & 48 48 60 & 10 44 & 14 9 5\\
\index{BruckerK00}\rowlabel{a:BruckerK00}BruckerK00 \href{http://dx.doi.org/10.1016/s0377-2217(99)00489-0}{BruckerK00} & \hyperref[auth:a847]{P. Brucker}, \hyperref[auth:a1166]{S. Knust} & A linear programming and constraint propagation-based lower bound for the RCPSP & \hyperref[detail:BruckerK00]{Details} \href{../works/BruckerK00.pdf}{Yes} & \cite{BruckerK00} & 2000 & European Journal of Operational Research & 8 & \noindent{}\textcolor{black!50}{0.00} \textcolor{black!50}{0.00} \textbf{1.12} & 66 67 80 & 8 12 & 14 13 1\\
\index{Dorndorf2000}\rowlabel{a:Dorndorf2000}Dorndorf2000 \href{http://dx.doi.org/10.1016/s0004-3702(00)00040-0}{Dorndorf2000} & \hyperref[auth:a904]{U. Dorndorf}, \hyperref[auth:a438]{E. Pesch}, \hyperref[auth:a1046]{T. Phan-Huy} & \cellcolor{gold!20}Constraint propagation techniques for the disjunctive scheduling problem & \hyperref[detail:Dorndorf2000]{Details} \href{../works/Dorndorf2000.pdf}{Yes} & \cite{Dorndorf2000} & 2000 & Artificial Intelligence & 52 & \noindent{}\textbf{1.50} \textbf{1.50} \textbf{18.08} & 47 47 51 & 33 62 & 25 15 10\\
\index{Dorndorf2000a}\rowlabel{a:Dorndorf2000a}Dorndorf2000a \href{http://dx.doi.org/10.1287/mnsc.46.10.1365.12272}{Dorndorf2000a} & \hyperref[auth:a904]{U. Dorndorf}, \hyperref[auth:a438]{E. Pesch}, \hyperref[auth:a1046]{T. Phan-Huy} & A Time-Oriented Branch-and-Bound Algorithm for Resource-Constrained Project Scheduling with Generalised Precedence Constraints \hyperref[abs:Dorndorf2000a]{Abstract} & \hyperref[detail:Dorndorf2000a]{Details} No & \cite{Dorndorf2000a} & 2000 & Management Science & null & \noindent{}\textcolor{black!50}{0.00} \textbf{3.00} n/a & 79 80 87 & 20 28 & 18 9 9\\
\index{HarjunkoskiJG00}\rowlabel{a:HarjunkoskiJG00}HarjunkoskiJG00 \href{http://dx.doi.org/10.1016/s0098-1354(00)00470-1}{HarjunkoskiJG00} & \hyperref[auth:a871]{I. Harjunkoski}, \hyperref[auth:a844]{V. Jain}, \hyperref[auth:a1160]{I. E. Grossman} & Hybrid mixed-integer/constraint logic programming strategies for solving scheduling and combinatorial optimization problems & \hyperref[detail:HarjunkoskiJG00]{Details} \href{../works/HarjunkoskiJG00.pdf}{Yes} & \cite{HarjunkoskiJG00} & 2000 & Computers \  Chemical Engineering & 7 & \noindent{}\textbf{1.00} \textbf{1.00} \textbf{1.64} & 44 44 49 & 3 9 & 15 15 0\\
\index{HeipckeCCS00}\rowlabel{a:HeipckeCCS00}HeipckeCCS00 \href{https://doi.org/10.1023/A:1009860311452}{HeipckeCCS00} & \hyperref[auth:a167]{S. Heipcke}, \hyperref[auth:a168]{Y. Colombani}, \hyperref[auth:a169]{C. C. B. Cavalcante}, \hyperref[auth:a170]{C. C. de Souza} & Scheduling under Labour Resource Constraints & \hyperref[detail:HeipckeCCS00]{Details} \href{../works/HeipckeCCS00.pdf}{Yes} & \cite{HeipckeCCS00} & 2000 & Constraints An Int. J. & 8 & \noindent{}\textcolor{black!50}{0.00} \textcolor{black!50}{0.00} \textbf{2.08} & 5 0 5 & 0 0 & 1 1 0\\
\index{Hentenryck2000}\rowlabel{a:Hentenryck2000}Hentenryck2000 \href{http://dx.doi.org/10.1145/359496.359529}{Hentenryck2000} & \hyperref[auth:a148]{P. V. Hentenryck}, \hyperref[auth:a288]{L. Perron}, \hyperref[auth:a1653]{J.-F. Puget} & Search and strategies in OPL \hyperref[abs:Hentenryck2000]{Abstract} & \hyperref[detail:Hentenryck2000]{Details} No & \cite{Hentenryck2000} & 2000 & ACM Transactions on Computational Logic & null & \noindent{}\textcolor{black!50}{0.00} \textbf{1.00} n/a & 44 43 48 & 14 33 & 9 7 2\\
\index{HookerOTK00}\rowlabel{a:HookerOTK00}HookerOTK00 \href{http://dx.doi.org/10.1017/s0269888900001077}{HookerOTK00} & \hyperref[auth:a160]{J. N. Hooker}, \hyperref[auth:a852]{G. Ottosson}, \hyperref[auth:a1188]{E. S. Thorsteinsson}, \hyperref[auth:a1189]{H.-J. Kim} & \cellcolor{green!10}A scheme for unifying optimization and constraint satisfaction methods & \hyperref[detail:HookerOTK00]{Details} \href{../works/HookerOTK00.pdf}{Yes} & \cite{HookerOTK00} & 2000 & The Knowledge Engineering Review & 20 & \noindent{}\textcolor{black!50}{0.00} \textcolor{black!50}{0.00} \textbf{1.64} & 30 30 44 & 0 0 & 11 11 0\\
\index{Huang2000}\rowlabel{a:Huang2000}Huang2000 \href{http://dx.doi.org/10.1016/s0098-1354(00)00483-x}{Huang2000} & \hyperref[auth:a1648]{W. Huang}, \hyperref[auth:a1649]{P. W. H. Chung} & Scheduling of pipeless batch plants using constraint satisfaction techniques & \hyperref[detail:Huang2000]{Details} No & \cite{Huang2000} & 2000 & Computers \  Chemical Engineering & null & \noindent{}\textbf{1.00} \textbf{1.00} n/a & 15 15 16 & 2 5 & 2 2 0\\
\index{Kambhampati2000}\rowlabel{a:Kambhampati2000}Kambhampati2000 \href{http://dx.doi.org/10.1613/jair.655}{Kambhampati2000} & \hyperref[auth:a1915]{S. Kambhampati} & \cellcolor{gold!20}Planning Graph as a (Dynamic) CSP: Exploiting EBL, DDB and other CSP Search Techniques in Graphplan \hyperref[abs:Kambhampati2000]{Abstract} & \hyperref[detail:Kambhampati2000]{Details} No & \cite{Kambhampati2000} & 2000 & Journal of Artificial Intelligence Research & null & \noindent{}\textcolor{black!50}{0.00} \textbf{1.00} n/a & 31 35 58 & 0 0 & 1 1 0\\
\index{KorbaaYG00}\rowlabel{a:KorbaaYG00}KorbaaYG00 \href{https://doi.org/10.1016/S0947-3580(00)71113-7}{KorbaaYG00} & \hyperref[auth:a680]{O. Korbaa}, \hyperref[auth:a681]{P. Yim}, \hyperref[auth:a682]{J.-C. Gentina} & Solving Transient Scheduling Problems with Constraint Programming & \hyperref[detail:KorbaaYG00]{Details} \href{../works/KorbaaYG00.pdf}{Yes} & \cite{KorbaaYG00} & 2000 & Eur. J. Control & 10 & \noindent{}\textbf{1.00} \textbf{1.00} \textcolor{black!50}{0.00} & 7 7 9 & 4 15 & 2 2 0\\
\index{LopezAKYG00}\rowlabel{a:LopezAKYG00}LopezAKYG00 \href{https://doi.org/10.1016/S0947-3580(00)71114-9}{LopezAKYG00} & \hyperref[auth:a3]{P. Lopez}, \hyperref[auth:a683]{H. Alla}, \hyperref[auth:a680]{O. Korbaa}, \hyperref[auth:a681]{P. Yim}, \hyperref[auth:a682]{J.-C. Gentina} & Discussion on: 'Solving Transient Scheduling Problems with Constraint Programming' by O. Korbaa, P. Yim, and {J.-C.} Gentina & \hyperref[detail:LopezAKYG00]{Details} \href{../works/LopezAKYG00.pdf}{Yes} & \cite{LopezAKYG00} & 2000 & Eur. J. Control & 4 & \noindent{}\textbf{1.00} \textbf{1.00} \textcolor{black!50}{0.00} & 0 0 0 & 0 1 & 0 0 0\\
\index{SakkoutW00}\rowlabel{a:SakkoutW00}SakkoutW00 \href{https://doi.org/10.1023/A:1009856210543}{SakkoutW00} & \hyperref[auth:a166]{H. E. Sakkout}, \hyperref[auth:a117]{M. G. Wallace} & Probe Backtrack Search for Minimal Perturbation in Dynamic Scheduling & \hyperref[detail:SakkoutW00]{Details} \href{../works/SakkoutW00.pdf}{Yes} & \cite{SakkoutW00} & 2000 & Constraints An Int. J. & 30 & \noindent{}\textcolor{black!50}{0.00} \textcolor{black!50}{0.00} \textbf{7.62} & 73 0 105 & 0 0 & 18 18 0\\
\index{SchildW00}\rowlabel{a:SchildW00}SchildW00 \href{https://doi.org/10.1023/A:1009804226473}{SchildW00} & \hyperref[auth:a164]{K. Schild}, \hyperref[auth:a165]{J. W{\"{u}}rtz} & Scheduling of Time-Triggered Real-Time Systems & \hyperref[detail:SchildW00]{Details} \href{../works/SchildW00.pdf}{Yes} & \cite{SchildW00} & 2000 & Constraints An Int. J. & 23 & \noindent{}\textcolor{black!50}{0.00} \textcolor{black!50}{0.00} \textbf{2.74} & 23 0 32 & 0 0 & 1 1 0\\
\index{SimonisCK00}\rowlabel{a:SimonisCK00}SimonisCK00 \href{https://doi.org/10.1109/5254.820326}{SimonisCK00} & \hyperref[auth:a17]{H. Simonis}, \hyperref[auth:a886]{P. Charlier}, \hyperref[auth:a887]{P. Kay} & Constraint Handling in an Integrated Transportation Problem & \hyperref[detail:SimonisCK00]{Details} \href{../works/SimonisCK00.pdf}{Yes} & \cite{SimonisCK00} & 2000 & {IEEE} Intell. Syst. & 7 & \noindent{}\textcolor{black!50}{0.00} \textcolor{black!50}{0.00} 0.48 & 11 11 6 & 5 14 & 10 5 5\\
\index{SourdN00}\rowlabel{a:SourdN00}SourdN00 \href{https://doi.org/10.1287/ijoc.12.4.341.11881}{SourdN00} & \hyperref[auth:a775]{F. Sourd}, \hyperref[auth:a656]{W. Nuijten} & Multiple-Machine Lower Bounds for Shop-Scheduling Problems & \hyperref[detail:SourdN00]{Details} \href{../works/SourdN00.pdf}{Yes} & \cite{SourdN00} & 2000 & \cellcolor{red!20}INFORMS Journal on Computing & 12 & \noindent{}\textcolor{black!50}{0.00} \textcolor{black!50}{0.00} 0.32 & 7 7 8 & 14 23 & 13 2 11\\
\index{TorresL00}\rowlabel{a:TorresL00}TorresL00 \href{http://dx.doi.org/10.1016/s0377-2217(99)00497-x}{TorresL00} & \hyperref[auth:a873]{P. Torres}, \hyperref[auth:a3]{P. Lopez} & \cellcolor{green!10}On Not-First/Not-Last conditions in disjunctive scheduling & \hyperref[detail:TorresL00]{Details} \href{../works/TorresL00.pdf}{Yes} & \cite{TorresL00} & 2000 & European Journal of Operational Research & 12 & \noindent{}\textcolor{black!50}{0.00} \textcolor{black!50}{0.00} \textbf{7.36} & 26 26 26 & 13 27 & 26 20 6\\
\index{Yang2000}\rowlabel{a:Yang2000}Yang2000 \href{http://dx.doi.org/10.1109/72.839016}{Yang2000} & \hyperref[auth:a1912]{S. Yang}, \hyperref[auth:a1824]{D. Wang} & \cellcolor{green!10}Constraint satisfaction adaptive neural network and heuristics combined approaches for generalized job-shop scheduling & \hyperref[detail:Yang2000]{Details} No & \cite{Yang2000} & 2000 & IEEE Transactions on Neural Networks & null & \noindent{}\textbf{2.00} \textbf{2.00} n/a & 37 0 48 & 10 0 & 4 4 0\\
\index{BaptistePN99}\rowlabel{a:BaptistePN99}BaptistePN99 \href{http://dx.doi.org/10.1023/a:1018995000688}{BaptistePN99} & \hyperref[auth:a162]{P. Baptiste}, \hyperref[auth:a163]{C. L. Pape}, \hyperref[auth:a656]{W. Nuijten} & Satisfiability tests and time-bound adjustments for cumulative scheduling problems & \hyperref[detail:BaptistePN99]{Details} \href{../works/BaptistePN99.pdf}{Yes} & \cite{BaptistePN99} & 1999 & Annals of Operations Research & 29 & \noindent{}\textcolor{black!50}{0.00} \textcolor{black!50}{0.00} \textbf{1.74} & 72 0 85 & 0 0 & 19 19 0\\
\index{BensanaLV99}\rowlabel{a:BensanaLV99}BensanaLV99 \href{https://doi.org/10.1023/A:1026488509554}{BensanaLV99} & \hyperref[auth:a171]{E. Bensana}, \hyperref[auth:a172]{M. Lema{\^{\i}}tre}, \hyperref[auth:a173]{G. Verfaillie} & Earth Observation Satellite Management & \hyperref[detail:BensanaLV99]{Details} \href{../works/BensanaLV99.pdf}{Yes} & \cite{BensanaLV99} & 1999 & Constraints An Int. J. & 7 & \noindent{}\textcolor{black!50}{0.00} \textcolor{black!50}{0.00} \textcolor{black!50}{0.07} & 99 0 150 & 0 0 & 4 4 0\\
\index{HookerO99}\rowlabel{a:HookerO99}HookerO99 \href{http://dx.doi.org/10.1016/s0166-218x(99)00100-6}{HookerO99} & \hyperref[auth:a160]{J. N. Hooker}, \hyperref[auth:a1153]{M. Osorio} & \cellcolor{gold!20}Mixed logical-linear programming & \hyperref[detail:HookerO99]{Details} \href{../works/HookerO99.pdf}{Yes} & \cite{HookerO99} & 1999 & Discrete Applied Mathematics & 48 & \noindent{}\textcolor{black!50}{0.00} \textcolor{black!50}{0.00} \textbf{1.61} & 92 95 111 & 48 75 & 19 19 0\\
\index{JainM99}\rowlabel{a:JainM99}JainM99 \href{http://dx.doi.org/10.1016/s0377-2217(98)00113-1}{JainM99} & \hyperref[auth:a954]{A. Jain}, \hyperref[auth:a955]{S. Meeran} & Deterministic job-shop scheduling: Past, present and future & \hyperref[detail:JainM99]{Details} \href{../works/JainM99.pdf}{Yes} & \cite{JainM99} & 1999 & European Journal of Operational Research & 45 & \noindent{}\textcolor{black!50}{0.00} \textcolor{black!50}{0.00} \textbf{7.96} & 490 503 630 & 150 262 & 26 10 16\\
\index{PesantGPR99}\rowlabel{a:PesantGPR99}PesantGPR99 \href{http://dx.doi.org/10.1016/s0377-2217(98)00248-3}{PesantGPR99} & \hyperref[auth:a8]{G. Pesant}, \hyperref[auth:a616]{M. Gendreau}, \hyperref[auth:a1202]{J.-Y. Potvin}, \hyperref[auth:a1203]{J.-M. Rousseau} & On the flexibility of constraint programming models: From single to multiple time windows for the traveling salesman problem & \hyperref[detail:PesantGPR99]{Details} \href{../works/PesantGPR99.pdf}{Yes} & \cite{PesantGPR99} & 1999 & European Journal of Operational Research & 11 & \noindent{}\textcolor{black!50}{0.00} \textcolor{black!50}{0.00} 0.65 & 26 28 32 & 18 24 & 7 4 3\\
\index{RodosekWH99}\rowlabel{a:RodosekWH99}RodosekWH99 \href{http://dx.doi.org/10.1023/a:1018904229454}{RodosekWH99} & \hyperref[auth:a297]{R. Rodosek}, \hyperref[auth:a117]{M. G. Wallace}, \hyperref[auth:a1030]{M. Hajian} & A new approach to integrating mixed integer programming and constraint logic programming & \hyperref[detail:RodosekWH99]{Details} \href{../works/RodosekWH99.pdf}{Yes} & \cite{RodosekWH99} & 1999 & Annals of Operations Research & 25 & \noindent{}\textcolor{black!50}{0.00} \textcolor{black!50}{0.00} \textbf{8.11} & 53 0 67 & 0 0 & 23 23 0\\
\index{BeckDDF98}\rowlabel{a:BeckDDF98}BeckDDF98 \href{http://dx.doi.org/10.1002/(sici)1099-1425(199808)1:2<89::aid-jos9>3.0.co;2-h}{BeckDDF98} & \hyperref[auth:a89]{J. C. Beck}, \hyperref[auth:a248]{A. J. Davenport}, \hyperref[auth:a1218]{E. D. Davis}, \hyperref[auth:a302]{M. S. Fox} & The ODO project: toward a unified basis for constraint-directed scheduling & \hyperref[detail:BeckDDF98]{Details} \href{../works/BeckDDF98.pdf}{Yes} & \cite{BeckDDF98} & 1998 & Journal of Scheduling & 37 & \noindent{}\textcolor{black!50}{0.00} \textcolor{black!50}{0.00} \textbf{22.70} & 9 8 0 & 0 0 & 5 5 0\\
\index{BeckF98}\rowlabel{a:BeckF98}BeckF98 \href{https://doi.org/10.1609/aimag.v19i4.1426}{BeckF98} & \hyperref[auth:a89]{J. C. Beck}, \hyperref[auth:a302]{M. S. Fox} & A Generic Framework for Constraint-Directed Search and Scheduling & \hyperref[detail:BeckF98]{Details} \href{../works/BeckF98.pdf}{Yes} & \cite{BeckF98} & 1998 & {AI} Mag. & 30 & \noindent{}\textcolor{black!50}{0.00} \textcolor{black!50}{0.00} \textbf{9.67} & 0 0 0 & 0 0 & 0 0 0\\
\index{BelhadjiI98}\rowlabel{a:BelhadjiI98}BelhadjiI98 \href{https://doi.org/10.1023/A:1009777711218}{BelhadjiI98} & \hyperref[auth:a174]{S. Belhadji}, \hyperref[auth:a175]{A. Isli} & Temporal Constraint Satisfaction Techniques in Job Shop Scheduling Problem Solving & \hyperref[detail:BelhadjiI98]{Details} \href{../works/BelhadjiI98.pdf}{Yes} & \cite{BelhadjiI98} & 1998 & Constraints An Int. J. & 9 & \noindent{}\textbf{2.00} \textbf{2.00} \textbf{1.70} & 3 0 5 & 0 0 & 1 1 0\\
\index{BockmayrK98}\rowlabel{a:BockmayrK98}BockmayrK98 \href{http://dx.doi.org/10.1287/ijoc.10.3.287}{BockmayrK98} & \hyperref[auth:a908]{A. Bockmayr}, \hyperref[auth:a1045]{T. Kasper} & Branch and Infer: A Unifying Framework for Integer and Finite Domain Constraint Programming & \hyperref[detail:BockmayrK98]{Details} No & \cite{BockmayrK98} & 1998 & \cellcolor{red!20}INFORMS Journal on Computing & 14 & \noindent{}\textcolor{black!50}{0.00} \textcolor{black!50}{0.00} n/a & 79 79 92 & 27 42 & 32 28 4\\
\index{DarbyDowmanL98}\rowlabel{a:DarbyDowmanL98}DarbyDowmanL98 \href{http://dx.doi.org/10.1287/ijoc.10.3.276}{DarbyDowmanL98} & \hyperref[auth:a177]{K. Darby-Dowman}, \hyperref[auth:a178]{J. Little} & Properties of Some Combinatorial Optimization Problems and Their Effect on the Performance of Integer Programming and Constraint Logic Programming & \hyperref[detail:DarbyDowmanL98]{Details} No & \cite{DarbyDowmanL98} & 1998 & \cellcolor{red!20}INFORMS Journal on Computing & 11 & \noindent{}\textcolor{black!50}{0.00} \textcolor{black!50}{0.00} n/a & 28 28 35 & 6 13 & 14 13 1\\
\index{NuijtenP98}\rowlabel{a:NuijtenP98}NuijtenP98 \href{https://doi.org/10.1023/A:1009687210594}{NuijtenP98} & \hyperref[auth:a656]{W. Nuijten}, \hyperref[auth:a163]{C. L. Pape} & Constraint-Based Job Shop Scheduling with {\textbackslash}sc Ilog Scheduler & \hyperref[detail:NuijtenP98]{Details} \href{../works/NuijtenP98.pdf}{Yes} & \cite{NuijtenP98} & 1998 & J. Heuristics & 16 & \noindent{}\textcolor{black!50}{0.00} \textcolor{black!50}{0.00} \textbf{10.57} & 42 0 50 & 0 0 & 24 24 0\\
\index{PapaB98}\rowlabel{a:PapaB98}PapaB98 \href{https://doi.org/10.1023/A:1009723704757}{PapaB98} & \hyperref[auth:a163]{C. L. Pape}, \hyperref[auth:a162]{P. Baptiste} & Resource Constraints for Preemptive Job-shop Scheduling & \hyperref[detail:PapaB98]{Details} \href{../works/PapaB98.pdf}{Yes} & \cite{PapaB98} & 1998 & Constraints An Int. J. & 25 & \noindent{}\textcolor{black!50}{0.00} \textcolor{black!50}{0.00} \textbf{10.97} & 14 0 19 & 0 0 & 4 4 0\\
\index{Richard1998}\rowlabel{a:Richard1998}Richard1998 \href{http://dx.doi.org/10.1051/ro/1998320201251}{Richard1998} & \hyperref[auth:a1684]{P. Richard}, \hyperref[auth:a1685]{C. Proust} & \cellcolor{gold!20}Solving scheduling problems using Petri nets and constraint logic programming & \hyperref[detail:Richard1998]{Details} No & \cite{Richard1998} & 1998 & RAIRO - Operations Research & null & \noindent{}\textbf{1.00} \textbf{1.00} n/a & 4 4 6 & 15 43 & 3 0 3\\
\index{Baykan1997}\rowlabel{a:Baykan1997}Baykan1997 \href{http://dx.doi.org/10.1017/s0890060400003206}{Baykan1997} & \hyperref[auth:a1689]{C. A. Baykan}, \hyperref[auth:a302]{M. S. Fox} & \cellcolor{green!10}Spatial synthesis by disjunctive constraint satisfaction \hyperref[abs:Baykan1997]{Abstract} & \hyperref[detail:Baykan1997]{Details} No & \cite{Baykan1997} & 1997 & Artificial Intelligence for Engineering Design, Analysis and Manufacturing & null & \noindent{}\textcolor{black!50}{0.00} 0.50 n/a & 11 11 19 & 12 35 & 2 0 2\\
\index{Darby-DowmanLMZ97}\rowlabel{a:Darby-DowmanLMZ97}Darby-DowmanLMZ97 \href{https://doi.org/10.1007/BF00137871}{Darby-DowmanLMZ97} & \hyperref[auth:a177]{K. Darby-Dowman}, \hyperref[auth:a178]{J. Little}, \hyperref[auth:a179]{G. Mitra}, \hyperref[auth:a180]{M. Zaffalon} & \cellcolor{green!10}Constraint Logic Programming and Integer Programming Approaches and Their Collaboration in Solving an Assignment Scheduling Problem & \hyperref[detail:Darby-DowmanLMZ97]{Details} \href{../works/Darby-DowmanLMZ97.pdf}{Yes} & \cite{Darby-DowmanLMZ97} & 1997 & Constraints An Int. J. & 20 & \noindent{}\textbf{1.00} \textbf{1.00} \textbf{14.30} & 28 28 32 & 5 22 & 12 12 0\\
\index{Demeulemeester1997}\rowlabel{a:Demeulemeester1997}Demeulemeester1997 \href{http://dx.doi.org/10.1287/mnsc.43.11.1485}{Demeulemeester1997} & \hyperref[auth:a1584]{E. L. Demeulemeester}, \hyperref[auth:a1585]{W. S. Herroelen} & \cellcolor{green!10}New Benchmark Results for the Resource-Constrained Project Scheduling Problem \hyperref[abs:Demeulemeester1997]{Abstract} & \hyperref[detail:Demeulemeester1997]{Details} No & \cite{Demeulemeester1997} & 1997 & Management Science & null & \noindent{}\textcolor{black!50}{0.00} \textcolor{black!50}{0.00} n/a & 158 161 183 & 0 0 & 14 14 0\\
\index{FalaschiGMP97}\rowlabel{a:FalaschiGMP97}FalaschiGMP97 \href{https://doi.org/10.1006/inco.1997.2638}{FalaschiGMP97} & \hyperref[auth:a687]{M. Falaschi}, \hyperref[auth:a192]{M. Gabbrielli}, \hyperref[auth:a688]{K. Marriott}, \hyperref[auth:a689]{C. Palamidessi} & \cellcolor{gold!20}Constraint Logic Programming with Dynamic Scheduling: {A} Semantics Based on Closure Operators & \hyperref[detail:FalaschiGMP97]{Details} \href{../works/FalaschiGMP97.pdf}{Yes} & \cite{FalaschiGMP97} & 1997 & Inf. Comput. & 27 & \noindent{}\textbf{1.00} \textbf{1.00} \textbf{1.33} & 10 10 12 & 9 15 & 0 0 0\\
\index{LammaMM97}\rowlabel{a:LammaMM97}LammaMM97 \href{https://doi.org/10.1016/S0954-1810(96)00002-7}{LammaMM97} & \hyperref[auth:a720]{E. Lamma}, \hyperref[auth:a721]{P. Mello}, \hyperref[auth:a143]{M. Milano} & A distributed constraint-based scheduler & \hyperref[detail:LammaMM97]{Details} \href{../works/LammaMM97.pdf}{Yes} & \cite{LammaMM97} & 1997 & Artif. Intell. Eng. & 15 & \noindent{}\textcolor{black!50}{0.00} \textcolor{black!50}{0.00} \textbf{9.45} & 11 11 13 & 7 28 & 3 2 1\\
\index{PintoG97}\rowlabel{a:PintoG97}PintoG97 \href{https://www.sciencedirect.com/science/article/pii/S0098135496003183}{PintoG97} & \hyperref[auth:a1255]{J. M. Pinto}, \hyperref[auth:a382]{I. E. Grossmann} & A logic-based approach to scheduling problems with resource constraints & \hyperref[detail:PintoG97]{Details} \href{../works/PintoG97.pdf}{Yes} & \cite{PintoG97} & 1997 & Computers \  Chemical Engineering & 18 & \noindent{}\textcolor{black!50}{0.00} \textcolor{black!50}{0.00} \textcolor{black!50}{0.00} & 56 56 60 & 12 24 & 3 3 0\\
\index{Psarras1997}\rowlabel{a:Psarras1997}Psarras1997 \href{http://dx.doi.org/10.1016/s0377-2217(96)00114-2}{Psarras1997} & \hyperref[auth:a2040]{J. Psarras}, \hyperref[auth:a2041]{E. Stefanitsis}, \hyperref[auth:a2042]{N. Christodoulou} & Combination of local search and CLP in the vehicle-fleet scheduling problem & \hyperref[detail:Psarras1997]{Details} No & \cite{Psarras1997} & 1997 & European Journal of Operational Research & null & \noindent{}\textbf{1.00} \textbf{1.00} n/a & 1 1 3 & 11 23 & 1 0 1\\
\index{Zhou97}\rowlabel{a:Zhou97}Zhou97 \href{https://doi.org/10.1023/A:1009757726572}{Zhou97} & \hyperref[auth:a176]{J. Zhou} & A Permutation-Based Approach for Solving the Job-Shop Problem & \hyperref[detail:Zhou97]{Details} \href{../works/Zhou97.pdf}{Yes} & \cite{Zhou97} & 1997 & Constraints An Int. J. & 29 & \noindent{}\textcolor{black!50}{0.00} \textcolor{black!50}{0.00} \textbf{8.24} & 14 0 16 & 0 0 & 5 5 0\\
\index{BlazewiczDP96}\rowlabel{a:BlazewiczDP96}BlazewiczDP96 \href{http://dx.doi.org/10.1016/0377-2217(95)00362-2}{BlazewiczDP96} & \hyperref[auth:a975]{J. Błażewicz}, \hyperref[auth:a976]{W. Domschke}, \hyperref[auth:a438]{E. Pesch} & The job shop scheduling problem: Conventional and new solution techniques & \hyperref[detail:BlazewiczDP96]{Details} \href{../works/BlazewiczDP96.pdf}{Yes} & \cite{BlazewiczDP96} & 1996 & European Journal of Operational Research & 33 & \noindent{}\textcolor{black!50}{0.00} \textcolor{black!50}{0.00} \textbf{14.88} & 344 357 412 & 127 224 & 33 20 13\\
\index{Icmeli1996}\rowlabel{a:Icmeli1996}Icmeli1996 \href{http://dx.doi.org/10.1287/mnsc.42.10.1395}{Icmeli1996} & \hyperref[auth:a1553]{O. Icmeli}, \hyperref[auth:a1554]{S. S. Erenguc} & A Branch and Bound Procedure for the Resource Constrained Project Scheduling Problem with Discounted Cash Flows \hyperref[abs:Icmeli1996]{Abstract} & \hyperref[detail:Icmeli1996]{Details} No & \cite{Icmeli1996} & 1996 & Management Science & null & \noindent{}\textcolor{black!50}{0.00} \textcolor{black!50}{0.00} n/a & 65 67 78 & 0 0 & 8 8 0\\
\index{NuijtenA96}\rowlabel{a:NuijtenA96}NuijtenA96 \href{http://dx.doi.org/10.1016/0377-2217(95)00354-1}{NuijtenA96} & \hyperref[auth:a656]{W. Nuijten}, \hyperref[auth:a777]{E. H. L. Aarts} & A computational study of constraint satisfaction for multiple capacitated job shop scheduling & \hyperref[detail:NuijtenA96]{Details} \href{../works/NuijtenA96.pdf}{Yes} & \cite{NuijtenA96} & 1996 & European Journal of Operational Research & 16 & \noindent{}\textbf{2.00} \textbf{2.00} \textbf{2.88} & 65 65 90 & 6 21 & 27 23 4\\
\index{PeschT96}\rowlabel{a:PeschT96}PeschT96 \href{http://dx.doi.org/10.1287/ijoc.8.2.144}{PeschT96} & \hyperref[auth:a438]{E. Pesch}, \hyperref[auth:a1216]{U. A. W. Tetzlaff} & Constraint Propagation Based Scheduling of Job Shops & \hyperref[detail:PeschT96]{Details} No & \cite{PeschT96} & 1996 & \cellcolor{red!20}INFORMS Journal on Computing & 14 & \noindent{}\textbf{3.00} \textbf{3.00} n/a & 22 23 17 & 0 0 & 7 7 0\\
\index{SadehF96}\rowlabel{a:SadehF96}SadehF96 \href{http://dx.doi.org/10.1016/0004-3702(95)00098-4}{SadehF96} & \hyperref[auth:a1043]{N. M. Sadeh}, \hyperref[auth:a302]{M. S. Fox} & \cellcolor{gold!20}Variable and value ordering heuristics for the job shop scheduling constraint satisfaction problem & \hyperref[detail:SadehF96]{Details} \href{../works/SadehF96.pdf}{Yes} & \cite{SadehF96} & 1996 & Artificial Intelligence & 41 & \noindent{}\textbf{2.50} \textbf{2.50} \textbf{22.27} & 95 97 131 & 17 56 & 19 16 3\\
\index{SmithBHW96}\rowlabel{a:SmithBHW96}SmithBHW96 \href{http://dx.doi.org/10.1007/bf00143880}{SmithBHW96} & \hyperref[auth:a1053]{B. M. Smith}, \hyperref[auth:a1051]{S. C. Brailsford}, \hyperref[auth:a1179]{P. M. Hubbard}, \hyperref[auth:a1180]{H. P. Williams} & The progressive party problem: Integer linear programming and constraint programming compared & \hyperref[detail:SmithBHW96]{Details} \href{../works/SmithBHW96.pdf}{Yes} & \cite{SmithBHW96} & 1996 & Constraints An Int. J. & 20 & \noindent{}\textcolor{black!50}{0.00} \textcolor{black!50}{0.00} \textbf{1.29} & 56 57 61 & 4 9 & 14 14 0\\
\index{Wallace96}\rowlabel{a:Wallace96}Wallace96 \href{https://doi.org/10.1007/BF00143881}{Wallace96} & \hyperref[auth:a117]{M. G. Wallace} & Practical Applications of Constraint Programming & \hyperref[detail:Wallace96]{Details} \href{../works/Wallace96.pdf}{Yes} & \cite{Wallace96} & 1996 & Constraints An Int. J. & 30 & \noindent{}\textcolor{black!50}{0.00} \textcolor{black!50}{0.00} \textbf{16.50} & 87 89 138 & 55 143 & 20 12 8\\
\index{Sadeh1995}\rowlabel{a:Sadeh1995}Sadeh1995 \href{http://dx.doi.org/10.1016/0004-3702(95)00078-s}{Sadeh1995} & \hyperref[auth:a1581]{N. Sadeh}, \hyperref[auth:a1582]{K. Sycara}, \hyperref[auth:a1583]{Y. Xiong} & \cellcolor{gold!20}Backtracking techniques for the job shop scheduling constraint satisfaction problem & \hyperref[detail:Sadeh1995]{Details} No & \cite{Sadeh1995} & 1995 & Artificial Intelligence & null & \noindent{}\textbf{2.00} \textbf{2.00} n/a & 21 20 31 & 12 28 & 5 5 0\\
\index{WeilHFP95}\rowlabel{a:WeilHFP95}WeilHFP95 \href{http://dx.doi.org/10.1109/51.395324}{WeilHFP95} & \hyperref[auth:a1191]{G. Weil}, \hyperref[auth:a1192]{K. Heus}, \hyperref[auth:a1193]{P. Francois}, \hyperref[auth:a1194]{M. Poujade} & Constraint programming for nurse scheduling & \hyperref[detail:WeilHFP95]{Details} \href{../works/WeilHFP95.pdf}{Yes} & \cite{WeilHFP95} & 1995 & IEEE Engineering in Medicine and Biology Magazine & 6 & \noindent{}\textbf{1.00} \textbf{1.00} 0.68 & 56 56 68 & 9 21 & 10 9 1\\
\index{BeldiceanuC94}\rowlabel{a:BeldiceanuC94}BeldiceanuC94 \href{https://www.sciencedirect.com/science/article/pii/0895717794901279}{BeldiceanuC94} & \hyperref[auth:a128]{N. Beldiceanu}, \hyperref[auth:a784]{E. Contejean} & \cellcolor{gold!20}Introducing Global Constraints in {CHIP} \hyperref[abs:BeldiceanuC94]{Abstract} & \hyperref[detail:BeldiceanuC94]{Details} \href{../works/BeldiceanuC94.pdf}{Yes} & \cite{BeldiceanuC94} & 1994 & Mathematical and Computer Modelling & 27 & \noindent{}\textcolor{black!50}{0.00} \textbf{1.00} \textbf{1.72} & 167 169 223 & 8 21 & 37 34 3\\
\index{Pape94}\rowlabel{a:Pape94}Pape94 \href{http://dx.doi.org/10.1049/ise.1994.0009}{Pape94} & \hyperref[auth:a163]{C. L. Pape} & Implementation of resource constraints in ILOG SCHEDULE: a library for the development of constraint-based scheduling systems & \hyperref[detail:Pape94]{Details} \href{../works/Pape94.pdf}{Yes} & \cite{Pape94} & 1994 & Intelligent Systems Engineering & 34 & \noindent{}\textcolor{black!50}{0.00} \textcolor{black!50}{0.00} \textbf{12.56} & 98 98 103 & 0 53 & 38 38 0\\
\index{Schiex1994}\rowlabel{a:Schiex1994}Schiex1994 \href{http://dx.doi.org/10.1142/s0218213094000108}{Schiex1994} & \hyperref[auth:a1721]{T. Schiex}, \hyperref[auth:a1722]{G. Verfaillie} & Nogood recording for static and dynamic constraint satisfaction problems \hyperref[abs:Schiex1994]{Abstract} & \hyperref[detail:Schiex1994]{Details} No & \cite{Schiex1994} & 1994 & International Journal on Artificial Intelligence Tools & null & \noindent{}\textcolor{black!50}{0.00} \textbf{4.00} n/a & 65 66 0 & 0 0 & 2 2 0\\
\index{AggounB93}\rowlabel{a:AggounB93}AggounB93 \href{https://www.sciencedirect.com/science/article/pii/089571779390068A}{AggounB93} & \hyperref[auth:a725]{A. Aggoun}, \hyperref[auth:a128]{N. Beldiceanu} & \cellcolor{gold!20}Extending {CHIP} in order to solve complex scheduling and placement problems \hyperref[abs:AggounB93]{Abstract} & \hyperref[detail:AggounB93]{Details} \href{../works/AggounB93.pdf}{Yes} & \cite{AggounB93} & 1993 & Mathematical and Computer Modelling & 17 & \noindent{}\textcolor{black!50}{0.00} \textbf{3.00} \textbf{3.18} & 187 191 214 & 11 36 & 91 89 2\\
\index{Barber1993}\rowlabel{a:Barber1993}Barber1993 \href{http://dx.doi.org/10.1145/152947.152955}{Barber1993} & \hyperref[auth:a1959]{F. A. Barber} & A metric time-point and duration-based temporal model \hyperref[abs:Barber1993]{Abstract} & \hyperref[detail:Barber1993]{Details} No & \cite{Barber1993} & 1993 & ACM SIGART Bulletin & null & \noindent{}\textcolor{black!50}{0.00} 0.50 n/a & 13 13 0 & 9 26 & 2 0 2\\
\index{Icmeli1993}\rowlabel{a:Icmeli1993}Icmeli1993 \href{http://dx.doi.org/10.1108/01443579310046454}{Icmeli1993} & \hyperref[auth:a1553]{O. Icmeli}, \hyperref[auth:a1554]{S. S. Erenguc}, \hyperref[auth:a1723]{C. J. Zappe} & Project Scheduling Problems: A Survey \hyperref[abs:Icmeli1993]{Abstract} & \hyperref[detail:Icmeli1993]{Details} No & \cite{Icmeli1993} & 1993 & International Journal of Operations \  Production Management & null & \noindent{}\textcolor{black!50}{0.00} \textcolor{black!50}{0.00} n/a & 97 99 0 & 39 56 & 5 3 2\\
\index{Demeulemeester1992}\rowlabel{a:Demeulemeester1992}Demeulemeester1992 \href{http://dx.doi.org/10.1287/mnsc.38.12.1803}{Demeulemeester1992} & \hyperref[auth:a1090]{E. Demeulemeester}, \hyperref[auth:a1102]{W. Herroelen} & A Branch-and-Bound Procedure for the Multiple Resource-Constrained Project Scheduling Problem \hyperref[abs:Demeulemeester1992]{Abstract} & \hyperref[detail:Demeulemeester1992]{Details} No & \cite{Demeulemeester1992} & 1992 & Management Science & null & \noindent{}\textcolor{black!50}{0.00} \textcolor{black!50}{0.00} n/a & 380 387 0 & 0 0 & 22 22 0\\
\index{Elmaghraby1992}\rowlabel{a:Elmaghraby1992}Elmaghraby1992 \href{http://dx.doi.org/10.1287/mnsc.38.9.1245}{Elmaghraby1992} & \hyperref[auth:a1773]{S. E. Elmaghraby}, \hyperref[auth:a1774]{J. Kamburowski} & The Analysis of Activity Networks Under Generalized Precedence Relations (GPRs) \hyperref[abs:Elmaghraby1992]{Abstract} & \hyperref[detail:Elmaghraby1992]{Details} No & \cite{Elmaghraby1992} & 1992 & Management Science & null & \noindent{}\textcolor{black!50}{0.00} \textcolor{black!50}{0.00} n/a & 117 121 0 & 0 0 & 6 6 0\\
\index{MintonJPL92}\rowlabel{a:MintonJPL92}MintonJPL92 \href{http://dx.doi.org/10.1016/0004-3702(92)90007-k}{MintonJPL92} & \hyperref[auth:a1210]{S. Minton}, \hyperref[auth:a1211]{M. D. Johnston}, \hyperref[auth:a1212]{A. B. Philips}, \hyperref[auth:a1213]{P. Laird} & \cellcolor{green!10}Minimizing conflicts: a heuristic repair method for constraint satisfaction and scheduling problems & \hyperref[detail:MintonJPL92]{Details} \href{../works/MintonJPL92.pdf}{Yes} & \cite{MintonJPL92} & 1992 & Artificial Intelligence & 45 & \noindent{}\textbf{1.00} \textbf{1.00} \textbf{1.68} & 437 440 525 & 13 46 & 18 18 0\\
\index{Paredis1992}\rowlabel{a:Paredis1992}Paredis1992 \href{http://dx.doi.org/10.1080/12460125.1992.10511509}{Paredis1992} & \hyperref[auth:a1998]{J. Paredis}, \hyperref[auth:a1999]{T. van Rij} & Simulation and Constraint Programming as Support Methodologies for Job Shop Scheduling & \hyperref[detail:Paredis1992]{Details} No & \cite{Paredis1992} & 1992 & Journal of Decision Systems & null & \noindent{}\textbf{2.00} \textbf{2.00} n/a & 0 2 5 & 3 15 & 1 0 1\\
\index{Tay92}\rowlabel{a:Tay92}Tay92 \href{}{Tay92} & \hyperref[auth:a701]{D. B. H. Tay} & {COPS:} {A} Constraint Programming Approach to Resource-Limited Project Scheduling & \hyperref[detail:Tay92]{Details} No & \cite{Tay92} & 1992 & Comput. J. & null & \noindent{}\textbf{1.50} \textbf{1.50} n/a & 0 0 0 & 0 0 & 0 0 0\\
\index{Clearwater1991}\rowlabel{a:Clearwater1991}Clearwater1991 \href{http://dx.doi.org/10.1126/science.254.5035.1181}{Clearwater1991} & \hyperref[auth:a1776]{S. H. Clearwater}, \hyperref[auth:a1777]{B. A. Huberman}, \hyperref[auth:a1778]{T. Hogg} & Cooperative Solution of Constraint Satisfaction Problems \hyperref[abs:Clearwater1991]{Abstract} & \hyperref[detail:Clearwater1991]{Details} No & \cite{Clearwater1991} & 1991 & Science & null & \noindent{}\textcolor{black!50}{0.00} \textbf{1.00} n/a & 91 91 93 & 4 8 & 3 3 0\\
\index{DincbasSH90}\rowlabel{a:DincbasSH90}DincbasSH90 \href{https://doi.org/10.1016/0743-1066(90)90052-7}{DincbasSH90} & \hyperref[auth:a717]{M. Dincbas}, \hyperref[auth:a17]{H. Simonis}, \hyperref[auth:a148]{P. V. Hentenryck} & \cellcolor{gold!20}Solving Large Combinatorial Problems in Logic Programming & \hyperref[detail:DincbasSH90]{Details} \href{../works/DincbasSH90.pdf}{Yes} & \cite{DincbasSH90} & 1990 & The Journal of Logic Programming & 19 & \noindent{}\textcolor{black!50}{0.00} \textcolor{black!50}{0.00} \textbf{1.34} & 86 85 99 & 9 28 & 17 15 2\\
\index{Davis87}\rowlabel{a:Davis87}Davis87 \href{http://dx.doi.org/10.1016/0004-3702(87)90091-9}{Davis87} & \hyperref[auth:a1215]{E. Davis} & \cellcolor{gold!20}Constraint propagation with interval labels & \hyperref[detail:Davis87]{Details} \href{../works/Davis87.pdf}{Yes} & \cite{Davis87} & 1987 & Artificial Intelligence & 51 & \noindent{}\textcolor{black!50}{0.00} \textcolor{black!50}{0.00} \textbf{2.34} & 308 312 332 & 21 51 & 12 11 1\\
\index{Fisher1985}\rowlabel{a:Fisher1985}Fisher1985 \href{http://dx.doi.org/10.1287/inte.15.2.10}{Fisher1985} & \hyperref[auth:a1772]{M. L. Fisher} & An Applications Oriented Guide to Lagrangian Relaxation \hyperref[abs:Fisher1985]{Abstract} & \hyperref[detail:Fisher1985]{Details} No & \cite{Fisher1985} & 1985 & \cellcolor{red!20}Interfaces & null & \noindent{}\textcolor{black!50}{0.00} \textcolor{black!50}{0.00} n/a & 462 473 517 & 0 0 & 2 2 0\\
\index{Talbot1978}\rowlabel{a:Talbot1978}Talbot1978 \href{http://dx.doi.org/10.1287/mnsc.24.11.1163}{Talbot1978} & \hyperref[auth:a1497]{F. B. Talbot}, \hyperref[auth:a1498]{J. H. Patterson} & An Efficient Integer Programming Algorithm with Network Cuts for Solving Resource-Constrained Scheduling Problems \hyperref[abs:Talbot1978]{Abstract} & \hyperref[detail:Talbot1978]{Details} No & \cite{Talbot1978} & 1978 & Management Science & null & \noindent{}\textcolor{black!50}{0.00} \textcolor{black!50}{0.00} n/a & 149 152 155 & 0 0 & 8 8 0\\
\end{longtable}
}




\clearpage
\subsection{Extracted Concepts}
{\scriptsize
\begin{longtable}{p{3cm}p{4cm}p{2cm}p{2cm}p{2cm}p{2cm}p{2cm}p{2cm}p{2cm}p{2cm}}
\caption{Automatically Extracted Article Properties (Requires Local Copy)}\\ \toprule
Work & Concepts & Classification & Constraints & ProgLanguages & CPSystems & Areas & Industries & Benchmarks & Algorithm\\ \midrule\endhead
\bottomrule
\endfoot
\href{articles/AbohashimaEG21.pdf}{AbohashimaEG21}~\cite{AbohashimaEG21} & scheduling, order, resource, setup-time, cmax, machine, transportation & parallel machine & cycle & Python & Gurobi &  &  & https://, real-world, generated instance, github & \\
\href{articles/AbreuN22.pdf}{AbreuN22}~\cite{AbreuN22} & preempt, make-span, transportation, order, tardiness, inventory, scheduling, flow-time, distributed, resource, completion-time, machine, setup-time, job, job-shop, task, flow-shop, open-shop, batch process, cmax & single machine, Open Shop Scheduling Problem, OSSP & noOverlap, cycle, cumulative & Python & OZ, Cplex, CHIP & medical &  & real-world, benchmark, https://, http:// & \\
\href{articles/AggounB93.pdf}{AggounB93}~\cite{AggounB93} & task, machine, precedence, order, job, activity, due-date, job-shop, flow-shop, resource, scheduling &  & circuit, bin-packing, disjunctive, cumulative & Prolog & OPL, CHIP & perfect-square, rectangle-packing &  & real-world & \\
\href{articles/AkramNHRSA23.pdf}{AkramNHRSA23}~\cite{AkramNHRSA23} & resource, completion-time, preempt, scheduling, order, machine, task, distributed &  & cycle, bin-packing & Python & OR-Tools & medical, agriculture &  & benchmark, https:// & \\
\href{articles/ArtiguesR00.pdf}{ArtiguesR00}~\cite{ArtiguesR00} & no preempt, machine, preempt, release-date, job-shop, transportation, cmax, lateness, precedence, scheduling, completion-time, re-scheduling, make-span, resource, order, setup-time, job, activity, earliness, due-date & RCPSP & cycle, cumulative, disjunctive &  & OZ &  &  &  & \\
\href{articles/AstrandJZ20.pdf}{AstrandJZ20}~\cite{AstrandJZ20} & resource, open-shop, task, machine, precedence, flow-shop, job-shop, re-scheduling, make-span, order, setup-time, job, activity, scheduling, completion-time, due-date & parallel machine, RCPSP & alldifferent, disjunctive, cycle & C++ & OZ, Gecode & robot & potash industry, mining industry, mineral industry & https://, benchmark, real-world, real-life, http:// & \\
\href{articles/BaptisteB18.pdf}{BaptisteB18}~\cite{BaptisteB18} & resource, task, machine, preempt, manpower, lazy clause generation, precedence, scheduling, make-span, order, job & parallel machine, RCPSP, psplib & cumulative, bin-packing &  & CHIP &  &  & http:// & time-tabling, edge-finding, edge-finder\\
\href{articles/BaptisteP00.pdf}{BaptisteP00}~\cite{BaptisteP00} & resource, task, preempt, cmax, precedence, release-date, flow-shop, job-shop, scheduling, re-scheduling, make-span, order, job, activity, due-date & RCPSP & disjunctive, cumulative & C++ & Claire, Ilog Scheduler, CHIP &  &  & http://, benchmark & edge-finding, edge-finder, energetic reasoning\\
\href{articles/BartakS11.pdf}{BartakS11}~\cite{BartakS11} & distributed, resource, scheduling, task, multi-agent, order &  & cumulative &  & OPL &  &  & http://, random instance, real-world, real-life & \\
\href{articles/BeldiceanuCDP11.pdf}{BeldiceanuCDP11}~\cite{BeldiceanuCDP11} & cmax, preempt, resource, task, order, scheduling &  & diffn, geost, disjunctive, cumulative, bin-packing & Prolog & SICStus, CHIP & rectangle-packing, perfect-square &  & http://, benchmark & edge-finding, sweep, energetic reasoning\\
\href{articles/BelhadjiI98.pdf}{BelhadjiI98}~\cite{BelhadjiI98} & precedence, release-date, job-shop, order, job, scheduling, resource, task, machine, preempt, due-date & Temporal Constraint Satisfaction Problem, TCSP, JSSP & disjunctive &  &  &  &  & real-life & \\
\href{articles/BenediktMH20.pdf}{BenediktMH20}~\cite{BenediktMH20} & preempt, order, job, re-scheduling, task, job-shop, scheduling, machine & single machine & noOverlap, endBeforeStart &  & Gurobi & robot &  & https://, github, benchmark, random instance, generated instance & \\
\href{articles/BeniniLMR11.pdf}{BeniniLMR11}~\cite{BeniniLMR11} & resource, order, activity, task, machine, preempt, release-date, tardiness, precedence, scheduling, re-scheduling, make-span & SCC, single machine & table constraint, cumulative, circuit &  & Ilog Scheduler, Cplex, CHIP, OZ & pipeline &  & benchmark, real-world, instance generator & \\
\href{articles/BensanaLV99.pdf}{BensanaLV99}~\cite{BensanaLV99} & order &  & cycle &  & Cplex, Ilog Solver & satellite, earth observation &  & http://, benchmark & \\
\href{articles/BonfiettiLBM14.pdf}{BonfiettiLBM14}~\cite{BonfiettiLBM14} & activity, buffer-capacity, distributed, job, job-shop, machine, make-span, order, precedence, resource, scheduling, task & RCPSP & circuit, cumulative, cycle &  & Ilog Solver & hoist, medical, pipeline, robot &  & benchmark, generated instance, http://, industrial instance, real-world & sweep, time-tabling\\
\href{articles/BorghesiBLMB18.pdf}{BorghesiBLMB18}~\cite{BorghesiBLMB18} & job, re-scheduling, make-span, resource, distributed, activity, task, machine, scheduling, order &  & cumulative, cycle &  &  & super-computer &  & http://, benchmark, real-life, https:// & \\
\href{articles/BourreauGGLT22.pdf}{BourreauGGLT22}~\cite{BourreauGGLT22} & re-scheduling, scheduling, order, manpower, job, resource, precedence, transportation &  & disjunctive, alldifferent, diffn, cycle & C++ & OZ, Choco Solver, Cplex, CHIP & crew-scheduling, nurse &  & real-world, benchmark, http://, https:// & \\
\href{articles/BridiBLMB16.pdf}{BridiBLMB16}~\cite{BridiBLMB16} & re-scheduling, make-span, job, scheduling, resource, order, machine, activity, distributed, tardiness &  & cycle, cumulative, circuit &  & OZ & medical, super-computer &  & real-world, real-life, http:// & \\
\href{articles/Caballero23.pdf}{Caballero23}~\cite{Caballero23} & resource, scheduling & RCPSP &  &  &  &  &  & https://, http:// & \\
\href{articles/CampeauG22.pdf}{CampeauG22}~\cite{CampeauG22} & task, order, activity, make-span, completion-time, precedence, resource, job, scheduling & RCPSP, RCPSPDC & alwaysIn, noOverlap, endBeforeStart, cumulative, cycle & Python & Cplex, OZ &  & mining industry & https://, real-life, real-world & edge-finding\\
\href{articles/Darby-DowmanLMZ97.pdf}{Darby-DowmanLMZ97}~\cite{Darby-DowmanLMZ97} & machine, scheduling, order, task, make-span, resource & MGAP, single machine & span constraint, disjunctive & Prolog & Cplex, ECLiPSe & pipeline, aircraft &  & real-life, real-world, benchmark, http:// & \\
\href{articles/DincbasSH90.pdf}{DincbasSH90}~\cite{DincbasSH90} & task, machine, job-shop, distributed, precedence, scheduling, resource, order, job &  & circuit, disjunctive & Prolog & CHIP, OPL &  &  & real-life & \\
\href{articles/EscobetPQPRA19.pdf}{EscobetPQPRA19}~\cite{EscobetPQPRA19} & task, job-shop, release-date, scheduling, order, batch process, job, resource, activity, distributed, machine, due-date &  & alternative constraint, noOverlap, circuit, cycle &  & OPL, Cplex & energy-price, dairy & food industry, manufacturing industry & https://, http:// & \\
\href{articles/EvenSH15a.pdf}{EvenSH15a}~\cite{EvenSH15a} & preempt, distributed, transportation, resource, scheduling, completion-time, task, machine, order & SCC & disjunctive, cumulative & Java & Choco Solver, OPL &  &  & http://, real-world, real-life & sweep\\
\href{articles/FahimiOQ18.pdf}{FahimiOQ18}~\cite{FahimiOQ18} & completion-time, resource, job, precedence, batch process, lazy clause generation, open-shop, scheduling, distributed, setup-time, task, order, lateness, job-shop, due-date, machine, preempt, make-span, sequence dependent setup & RCPSP, psplib & cumulative, disjunctive, alldifferent &  & Choco Solver &  &  & https://, benchmark, random instance & not-last, time-tabling, sweep, edge-finding, not-first\\
\href{articles/FalaschiGMP97.pdf}{FalaschiGMP97}~\cite{FalaschiGMP97} & order, scheduling &  &  & Prolog &  &  &  &  & \\
\href{articles/FanXG21.pdf}{FanXG21}~\cite{FanXG21} & due-date, no preempt, preempt, tardiness, job, order, batch process, machine, task, earliness, completion-time, flow-shop, distributed, precedence, setup-time, resource, make-span, job-shop, scheduling, flow-time & single machine, parallel machine & cycle & Java, Python & OZ, ECLiPSe, Cplex, Gurobi & semiconductor & manufacturing industry & https://, benchmark & max-flow\\
\href{articles/FetgoD22.pdf}{FetgoD22}~\cite{FetgoD22} & task, precedence, cmax, preempt, lazy clause generation, make-span, order, scheduling, resource, completion-time & CuSP, RCPSP & cumulative & Python, Java & OZ, CHIP, Choco Solver &  &  & benchmark, https://, http://, real-world & not-first, not-last, energetic reasoning, edge-finding, sweep, edge-finder, time-tabling\\
\href{articles/GarridoAO09.pdf}{GarridoAO09}~\cite{GarridoAO09} & re-scheduling, precedence, scheduling, make-span, resource, order, task &  & disjunctive & Java & CPO, OPL, Choco Solver &  &  & http://, benchmark & \\
\href{articles/GarridoOS08.pdf}{GarridoOS08}~\cite{GarridoOS08} & scheduling, make-span, resource, order, activity, task, machine &  &  & Java, C  & Choco Solver, CPO &  &  & http://, real-world & \\
\href{articles/GedikKEK18.pdf}{GedikKEK18}~\cite{GedikKEK18} & cmax, resource, job, setup-time, due-date, scheduling, tardiness, task, order, machine, preempt, make-span, sequence dependent setup, completion-time, transportation & single machine, parallel machine, PMSP & cumulative, noOverlap &  & Cplex, OZ & nurse, medical & manufacturing industry & https://, http://, benchmark & \\
\href{articles/GoelSHFS15.pdf}{GoelSHFS15}~\cite{GoelSHFS15} & precedence, resource, inventory, setup-time, scheduling, activity, task, order, transportation, machine &  & cumulative, noOverlap, disjunctive, alwaysIn &  & OPL, Cplex, CPO & pipeline &  & http:// & \\
\href{articles/GrimesIOS14.pdf}{GrimesIOS14}~\cite{GrimesIOS14} & completion-time, due-date, resource, task, machine, preempt, distributed, re-scheduling, order, activity, scheduling &  & disjunctive &  & Cplex, CHIP & energy-price, real-time pricing, HVAC &  & real-world, real-life, http:// & \\
\href{articles/GurPAE23.pdf}{GurPAE23}~\cite{GurPAE23} & re-scheduling, order, scheduling, distributed, resource, inventory, machine & OSP & cumulative &  & OPL, Cplex, OZ & physician, patient, COVID, nurse &  & https://, real-life & \\
\href{articles/HachemiGR11.pdf}{HachemiGR11}~\cite{HachemiGR11} & task, precedence, job-shop, transportation, make-span, scheduling, resource, order, job, activity &  & cycle, alldifferent &  & OPL, Ilog Scheduler, Cplex & crew-scheduling, forestry & food industry &  & \\
\href{articles/HeinzNVH22.pdf}{HeinzNVH22}~\cite{HeinzNVH22} & activity, make-span, job, precedence, re-scheduling, distributed, resource, setup-time, scheduling, preempt, sequence dependent setup, flow-shop, task, order, completion-time, machine & parallel machine & cumulative, noOverlap, alternative constraint &  & Gurobi & robot, crew-scheduling &  & real-world, generated instance, http://, benchmark, https://, gitlab & \\
\href{articles/HeinzSB13.pdf}{HeinzSB13}~\cite{HeinzSB13} & preempt, due-date, resource, scheduling, precedence, order, completion-time, machine, job, release-date & RCPSP, single machine, psplib & disjunctive, cumulative &  & MiniZinc, CHIP, Cplex & satellite &  & http://, benchmark & time-tabling, edge-finding\\
\href{articles/HeinzSSW12.pdf}{HeinzSSW12}~\cite{HeinzSSW12} & inventory, task, order &  & bin-packing &  & Cplex & steel mill & steel industry, process industry & real-world, http:// & \\
\href{articles/HeipckeCCS00.pdf}{HeipckeCCS00}~\cite{HeipckeCCS00} & make-span, release-date, resource, activity, precedence, completion-time, job-shop, due-date, preempt, scheduling, order, machine, job, task & single machine, RCPSP & disjunctive, cumulative &  &  &  &  & http://, benchmark, instance generator & \\
\href{articles/Hooker05.pdf}{Hooker05}~\cite{Hooker05} & machine, job, task, precedence, release-date, due-date, make-span, order, tardiness, scheduling, distributed, resource &  & cumulative, circuit, disjunctive &  & Cplex, OPL, Ilog Scheduler &  &  & random instance & edge-finding\\
\href{articles/Hooker06.pdf}{Hooker06}~\cite{Hooker06} & machine, job, task, precedence, release-date, due-date, make-span, order, tardiness, scheduling, resource &  & cumulative, circuit, disjunctive &  & Cplex, OPL, Ilog Scheduler &  &  & random instance, http:// & \\
\href{articles/HubnerGSV21.pdf}{HubnerGSV21}~\cite{HubnerGSV21} & completion-time, resource, order, job, inventory, activity, due-date, task, machine, preempt, transportation, cmax, tardiness, make-span, precedence, scheduling & RCPSPDC, RCPSP & cycle, cumulative, endBeforeStart, alternative constraint & C  & Gurobi, Cplex, OPL & automotive &  & http://, https://, benchmark, real-life & \\
\href{articles/IsikYA23.pdf}{IsikYA23}~\cite{IsikYA23} & tardiness, scheduling, machine, distributed, job, resource, completion-time, flow-shop, batch process, setup-time, job-shop, release-date, due-date, task, precedence, transportation, earliness, order, cmax, sequence dependent setup, preempt, make-span & parallel machine, single machine & circuit, noOverlap, cumulative, endBeforeStart &  & OPL, Cplex, OZ & medical, robot & steel industry & real-world, benchmark, https://, generated instance, http://, real-life & energetic reasoning\\
\href{articles/Kameugne15.pdf}{Kameugne15}~\cite{Kameugne15} & resource, scheduling, task, preempt, completion-time &  & cumulative &  &  &  &  & http:// & not-last, edge-finding, not-first\\
\href{articles/KameugneFSN14.pdf}{KameugneFSN14}~\cite{KameugneFSN14} & job-shop, release-date, resource, precedence, job, order, preempt, scheduling, make-span, completion-time, task & RCPSP, psplib, CuSP & disjunctive, cumulative &  & CHIP, Gecode &  &  & random instance, http://, benchmark & energetic reasoning, edge-finding, not-last, not-first, edge-finder, time-tabling\\
\href{articles/KelbelH11.pdf}{KelbelH11}~\cite{KelbelH11} & release-date, inventory, earliness, due-date, preempt, job-shop, resource, scheduling, make-span, distributed, task, precedence, order, completion-time, machine, tardiness, job & JSSP & cumulative, disjunctive &  & Ilog Solver, OPL, Cplex &  &  & http://, benchmark, random instance, generated instance & edge-finder, edge-finding\\
\href{articles/KhayatLR06.pdf}{KhayatLR06}~\cite{KhayatLR06} & job-shop, due-date, scheduling, preempt, task, order, machine, activity, make-span, cmax, job, precedence, resource, setup-time &  &  &  & OPL, Cplex &  &  & http://, real-life, benchmark & \\
\href{articles/KoehlerBFFHPSSS21.pdf}{KoehlerBFFHPSSS21}~\cite{KoehlerBFFHPSSS21} & flow-shop, scheduling, lateness, job, task, make-span, machine, tardiness, precedence, resource, job-shop, flow-time, order & OSP, CTW, single machine & cycle, circuit, cumulative, disjunctive, alldifferent & C , Python & Z3, MiniZinc, OPL, Cplex, Gurobi, OR-Tools, Chuffed & cable tree, automotive, robot &  & real-world, http://, benchmark, github, https:// & \\
\href{articles/KorbaaYG00.pdf}{KorbaaYG00}~\cite{KorbaaYG00} &  &  &  &  &  &  &  &  & \\
\href{articles/KovacsB08.pdf}{KovacsB08}~\cite{KovacsB08} & order, tardiness, job, activity, preempt, release-date, resource, scheduling, completion-time, machine & single machine & bin-packing, disjunctive, cumulative, cycle &  & Ilog Scheduler, Ilog Solver & aircraft &  & benchmark & sweep\\
\href{articles/KovacsB11.pdf}{KovacsB11}~\cite{KovacsB11} & flow-time, precedence, order, tardiness, job, activity, preempt, release-date, earliness, distributed, due-date, job-shop, flow-shop, resource, scheduling, make-span, completion-time, machine & parallel machine, single machine & disjunctive, cumulative, cycle & C++ & Ilog Scheduler, Ilog Solver &  &  & benchmark & edge-finding\\
\href{articles/KovacsK11.pdf}{KovacsK11}~\cite{KovacsK11} & tardiness, job, release-date, earliness, sequence dependent setup, due-date, job-shop, transportation, flow-shop, resource, scheduling, completion-time, task, machine, order & single machine & cycle & C++ & Ilog Solver, Gecode, Cplex &  &  & http:// & \\
\href{articles/KreterSS17.pdf}{KreterSS17}~\cite{KreterSS17} & scheduling, task, order, machine, preempt, activity, make-span, completion-time, precedence, resource, lazy clause generation & RCPSP, OSP, parallel machine & cycle, alwaysIn, cumulative, diffn &  & CPO, Cplex, MiniZinc, CHIP, Chuffed &  &  & http://, benchmark & edge-finding\\
\href{articles/KuchcinskiW03.pdf}{KuchcinskiW03}~\cite{KuchcinskiW03} & scheduling, precedence, resource, distributed, order &  & cycle, circuit & Java & CHIP & pipeline &  & benchmark & \\
\href{articles/LaborieRSV18.pdf}{LaborieRSV18}~\cite{LaborieRSV18} & release-date, job-shop, resource, activity, precedence, sequence dependent setup, earliness, scheduling, machine, inventory, transportation, manpower, due-date, setup-time, batch process, order, tardiness, flow-shop, job, make-span, re-scheduling, task, distributed & psplib, parallel machine, RCPSP, OSP & alternative constraint, cumulative, noOverlap, disjunctive, span constraint, cycle, alwaysIn, endBeforeStart & C , Python, C++, Java & CHIP, Gecode, Ilog Solver, Cplex, Ilog Scheduler, OPL, Choco Solver, CPO & semiconductor, railway, container terminal, satellite, robot, pipeline, aircraft & chemical industry, petro-chemical industry & https://, real-world, http://, benchmark & edge-finding\\
\href{articles/LacknerMMWW23.pdf}{LacknerMMWW23}~\cite{LacknerMMWW23} & release-date, batch process, setup-time, job, order, due-date, tardiness, scheduling, make-span, machine, task, lateness, job-shop, earliness & parallel machine, OSP, single machine & alternative constraint, disjunctive, bin-packing, noOverlap, cumulative, endBeforeStart &  & Chuffed, Cplex, OPL, CPO, OR-Tools, MiniZinc, Gurobi & semiconductor, oven scheduling & electronics industry, steel industry, manufacturing industry & random instance, http://, industrial partner, https://, benchmark, instance generator, zenodo, real-life & time-tabling\\
\href{articles/LammaMM97.pdf}{LammaMM97}~\cite{LammaMM97} & job-shop, resource, scheduling, precedence, order, task, job, distributed & OSP & circuit, disjunctive & C++, Prolog & ECLiPSe, OPL, CHIP & railway &  & real-life & \\
\href{articles/LetortCB15.pdf}{LetortCB15}~\cite{LetortCB15} & machine, make-span, job, precedence, resource, scheduling, task, order & psplib & cumulative, cycle, bin-packing & Java, Prolog & Choco Solver, CHIP, SICStus &  &  & generated instance, benchmark, random instance, http:// & energetic reasoning, sweep, edge-finding\\
\href{articles/LiessM08.pdf}{LiessM08}~\cite{LiessM08} & preempt, resource, scheduling, machine, job, activity, precedence, job-shop, task, make-span, order, cmax & RCPSP, psplib & disjunctive, cumulative & C++ & OZ &  &  & benchmark, http:// & edge-finding\\
\href{articles/LimtanyakulS12.pdf}{LimtanyakulS12}~\cite{LimtanyakulS12} & release-date, scheduling, order, completion-time, job, resource, activity, tardiness, machine, due-date, precedence &  & table constraint, disjunctive, bin-packing, cumulative &  & OZ, Ilog Scheduler, Cplex & robot, automotive & automotive industry & random instance, real-life, generated instance, industrial partner, benchmark, http:// & not-last, energetic reasoning, not-first, edge-finding\\
\href{articles/LombardiM10a.pdf}{LombardiM10a}~\cite{LombardiM10a} & due-date, distributed, order, job, make-span, release-date, re-scheduling, task, completion-time, resource, activity, precedence, preempt, scheduling, machine & TCSP & cycle, span constraint, cumulative, disjunctive, table constraint & C  & Cplex, CHIP &  &  & real-world, http://, benchmark, real-life & sweep\\
\href{articles/LombardiM12.pdf}{LombardiM12}~\cite{LombardiM12} & precedence, flow-shop, job-shop, transportation, completion-time, re-scheduling, make-span, sequence dependent setup, order, setup-time, job, activity, earliness, scheduling, due-date, resource, task, machine, inventory, preempt, distributed, manpower, lazy clause generation, tardiness & parallel machine, RCPSP, psplib & cycle, disjunctive, cumulative, circuit &  & OZ, OR-Tools & aircraft & chemical industry & real-world, benchmark & energetic reasoning, edge-finding\\
\href{articles/LombardiM12a.pdf}{LombardiM12a}~\cite{LombardiM12a} & order, make-span, completion-time, resource, activity, precedence, producer/consumer, scheduling & psplib, RCPSP & disjunctive &  & Ilog Solver &  &  & http://, benchmark & \\
\href{articles/LopesCSM10.pdf}{LopesCSM10}~\cite{LopesCSM10} & distributed, stock level, resource, inventory, job-shop, due-date, scheduling, activity, task, order, transportation, make-span, job, precedence, re-scheduling & OSP & disjunctive, table constraint, cycle, alldifferent & C++ & Ilog Scheduler, Ilog Solver, OZ, OPL & pipeline & oil industry & http://, benchmark, real-world & max-flow\\
\href{articles/LopezAKYG00.pdf}{LopezAKYG00}~\cite{LopezAKYG00} &  &  &  &  &  &  &  &  & \\
\href{articles/LunardiBLRV20.pdf}{LunardiBLRV20}~\cite{LunardiBLRV20} & scheduling, due-date, make-span, machine, completion-time, job-shop, flow-shop, resource, precedence, setup-time, activity, re-scheduling, job, order, tardiness, preempt & FJS & endBeforeStart, noOverlap & Python & Cplex &  &  & benchmark, random instance, generated instance, github, https:// & \\
\href{articles/MartinPY01.pdf}{MartinPY01}~\cite{MartinPY01} & scheduling, task, order, machine, transportation, re-scheduling, resource &  & circuit & Prolog & ECLiPSe, Ilog Solver & railway, aircraft &  & real-life & \\
\href{articles/Mason01.pdf}{Mason01}~\cite{Mason01} & scheduling, order, task, activity, transportation &  &  &  & OPL, OZ, Cplex & railway, crew-scheduling, nurse &  & http:// & \\
\href{articles/MengZRZL20.pdf}{MengZRZL20}~\cite{MengZRZL20} & earliness, job-shop, scheduling, machine, preempt, sequence dependent setup, flow-time, flow-shop, order, completion-time, transportation, make-span, cmax, job, precedence, batch process, open-shop, distributed, tardiness, resource, no preempt, setup-time, task & Open Shop Scheduling Problem, OSP, parallel machine, FJS & alternative constraint, noOverlap, endBeforeStart &  & OPL, Gecode, Gurobi, OR-Tools, Cplex & robot, semiconductor &  & https://, supplementary material, benchmark & \\
\href{articles/MontemanniD23.pdf}{MontemanniD23}~\cite{MontemanniD23} & resource, distributed, order, scheduling, machine, task & OSP & circuit & Python & OPL, CHIP, OR-Tools, Gurobi & robot &  & https://, benchmark, supplementary material & \\
\href{articles/MontemanniD23a.pdf}{MontemanniD23a}~\cite{MontemanniD23a} & order, completion-time, task, transportation, scheduling &  & circuit & Python & OR-Tools &  &  & http://, benchmark, https:// & \\
\href{articles/MullerMKP22.pdf}{MullerMKP22}~\cite{MullerMKP22} & precedence, job-shop, batch process, scheduling, completion-time, make-span, order, setup-time, job, activity, due-date, resource, task, machine, preempt, cmax & FJS, OSP & disjunctive, circuit & Java, Python & Chuffed, MiniZinc, OZ, Gecode, Choco Solver, OPL, Cplex, OR-Tools & robot, semiconductor &  & benchmark, https://, random instance, real-world, github & \\
\href{articles/NattafAL15.pdf}{NattafAL15}~\cite{NattafAL15} & resource, release-date, due-date, scheduling, preempt, task, order, activity, make-span & CECSP, CuSP, RCPSP & cumulative & C++ & Cplex &  &  & generated instance, http:// & sweep, energetic reasoning\\
\href{articles/NattafAL17.pdf}{NattafAL17}~\cite{NattafAL17} & resource, release-date, scheduling, task, order, activity, make-span, job & CECSP & disjunctive, cumulative & C++ & Cplex &  &  & http://, real-world & edge-finding, energetic reasoning\\
\href{articles/NovaraNH16.pdf}{NovaraNH16}~\cite{NovaraNH16} & earliness, machine, make-span, job, precedence, batch process, re-scheduling, tardiness, resource, setup-time, due-date, scheduling, activity, sequence dependent setup, manpower, task, order, completion-time &  & cumulative, noOverlap, endBeforeStart, disjunctive, alternative constraint &  & OPL, Cplex &  & pharmaceutical industry & http://, benchmark & \\
\href{articles/Novas19.pdf}{Novas19}~\cite{Novas19} & inventory, lateness, setup-time, resource, make-span, scheduling, flow-shop, transportation, flow-time, precedence, cmax, release-date, job-shop, sequence dependent setup, due-date, machine, task, tardiness, job, completion-time, activity, order, distributed & parallel machine, FJS & cycle, cumulative, noOverlap, endBeforeStart &  & OPL, OZ, Cplex & medical, semiconductor, robot &  & benchmark, https:// & \\
\href{articles/NovasH10.pdf}{NovasH10}~\cite{NovasH10} & precedence, batch process, due-date, re-scheduling, make-span, earliness, order, tardiness, scheduling, resource, completion-time, machine, setup-time, lateness, job, task, manpower, activity &  &  &  & OZ, OPL, Ilog Scheduler & pipeline &  & http:// & \\
\href{articles/NovasH12.pdf}{NovasH12}~\cite{NovasH12} & precedence, make-span, transportation, order, scheduling, resource, completion-time, machine, job, task, activity &  & cycle &  & Ilog Solver, OZ, OPL, Ilog Scheduler & semiconductor, robot, hoist, electroplating, container terminal &  &  & \\
\href{articles/NovasH14.pdf}{NovasH14}~\cite{NovasH14} & precedence, make-span, transportation, order, scheduling, buffer-capacity, resource, completion-time, machine, job, job-shop, task, activity & parallel machine, single machine &  &  & Ilog Solver, OPL, Ilog Scheduler & robot &  & http://, benchmark & \\
\href{articles/OzturkTHO13.pdf}{OzturkTHO13}~\cite{OzturkTHO13} & order, setup-time, job, activity, scheduling, completion-time, resource, task, machine, preempt, cmax, precedence, flow-shop, make-span & SBSFMMAL & cycle, disjunctive, cumulative &  & OPL, Cplex, CHIP, Ilog Solver, OZ &  &  & http://, real-world, real-life & edge-finding\\
\href{articles/PandeyS21a.pdf}{PandeyS21a}~\cite{PandeyS21a} & make-span, re-scheduling, job, precedence, distributed, resource, task, scheduling, machine, activity, flow-shop, order, completion-time & single machine, parallel machine, PMSP & cumulative, endBeforeStart, alternative constraint &  & OPL, Cplex, OZ & semiconductor &  & benchmark, https:// & \\
\href{articles/PapaB98.pdf}{PapaB98}~\cite{PapaB98} & due-date, preempt, machine, re-scheduling, job, activity, order, task, make-span, completion-time, scheduling, flow-shop, distributed, cmax, setup-time, resource, job-shop & PJSSP, OSP, JSSP & cumulative, table constraint, disjunctive & C++ & Ilog Solver, CHIP, Claire & hoist &  & http://, benchmark & edge-finder, energetic reasoning, edge-finding\\
\href{articles/PoderBS04.pdf}{PoderBS04}~\cite{PoderBS04} & preempt, due-date, resource, scheduling, precedence, order, task, machine, activity, producer/consumer, release-date & RCPSP & cumulative & Prolog & CHIP &  & chemical industry & http:// & \\
\href{articles/PohlAK22.pdf}{PohlAK22}~\cite{PohlAK22} & resource, activity, completion-time, setup-time, lateness, release-date, precedence, transportation, earliness, order, sequence dependent setup, re-scheduling, tardiness, inventory, scheduling, machine, job & SCC, single machine & noOverlap, cumulative & Python & Gurobi, Cplex, OZ & aircraft &  & benchmark, https://, http://, real-world & \\
\href{articles/PourDERB18.pdf}{PourDERB18}~\cite{PourDERB18} & scheduling, task, order, machine, transportation, job &  &  &  & CHIP, Cplex, OR-Tools & crew-scheduling, railway &  & http://, real-life, benchmark, real-world, generated instance & \\
\href{articles/PrataAN23.pdf}{PrataAN23}~\cite{PrataAN23} & machine, tardiness, job, lateness, activity, re-scheduling, flow-time, setup-time, release-date, inventory, earliness, sequence dependent setup, distributed, due-date, preempt, job-shop, batch process, flow-shop, resource, scheduling, make-span, open-shop, completion-time, task, precedence, order & single machine, parallel machine, Open Shop Scheduling Problem & circuit, cumulative &  & OZ, CHIP & robot, aircraft, energy-price, dairy & manufacturing industry & https://, http://, benchmark, real-world, real-life & time-tabling\\
\href{articles/QinDCS20.pdf}{QinDCS20}~\cite{QinDCS20} & transportation, order, cmax, tardiness, scheduling, resource, completion-time, machine, setup-time, job, task, activity, precedence, make-span & parallel machine & endBeforeStart, cycle, noOverlap &  & Cplex, OPL & yard crane, container terminal &  & real-life, https://, benchmark & \\
\href{articles/QinWSLS21.pdf}{QinWSLS21}~\cite{QinWSLS21} & preempt, job-shop, flow-shop, batch process, scheduling, make-span, order, cmax, completion-time, machine, tardiness, job, lateness & single machine, OSP &  & C++ & CHIP, OZ, OPL, Cplex & agriculture, semiconductor &  & https:// & \\
\href{articles/RuggieroBBMA09.pdf}{RuggieroBBMA09}~\cite{RuggieroBBMA09} & scheduling, order, resource, activity, preempt, setup-time, distributed, machine, precedence, task &  & circuit, cumulative, cycle &  & OZ, Ilog Solver, Ilog Scheduler, Cplex, CHIP & pipeline, satellite &  & instance generator, http://, real-life & \\
\href{articles/SacramentoSP20.pdf}{SacramentoSP20}~\cite{SacramentoSP20} & preempt, distributed, machine, precedence, task, flow-shop, job-shop, open-shop, transportation, scheduling, order, completion-time, job, resource, make-span, activity & parallel machine, Open Shop Scheduling Problem & disjunctive, cumulative, alternative constraint, endBeforeStart, noOverlap & Java & Cplex, OZ, CPO & container terminal &  & benchmark, real-life, zenodo, https://, real-world & \\
\href{articles/SakkoutW00.pdf}{SakkoutW00}~\cite{SakkoutW00} & scheduling, distributed, task, order, job-shop, machine, preempt, activity, precedence, transportation, re-scheduling, resource, job & KRFP, single machine & bin-packing, disjunctive &  & CHIP, Cplex & aircraft &  & http://, benchmark, real-world & edge-finding, edge-finder\\
\href{articles/SchausHMCMD11.pdf}{SchausHMCMD11}~\cite{SchausHMCMD11} & order, task & SCC & bin-packing &  &  & steel mill & steel industry & benchmark, generated instance, http:// & \\
\href{articles/SchildW00.pdf}{SchildW00}~\cite{SchildW00} & distributed, job-shop, flow-shop, resource, scheduling, completion-time, task, machine, precedence, order, job & single machine, OSP & disjunctive, cycle, bin-packing &  & OZ, Ilog Solver & automotive & automotive industry, aerospace industry & http:// & time-tabling, edge-finding\\
\href{articles/SchuttFSW11.pdf}{SchuttFSW11}~\cite{SchuttFSW11} & scheduling, completion-time, resource, open-shop, order, task, machine, preempt, activity, lazy clause generation, precedence, make-span & TMS, psplib, RCPSP & disjunctive, cumulative, circuit, span constraint &  & Ilog Scheduler, ECLiPSe, CHIP, SICStus, OZ &  &  & benchmark, real-world, http:// & not-last, not-first, edge-finding, edge-finder\\
\href{articles/ShinBBHO18.pdf}{ShinBBHO18}~\cite{ShinBBHO18} & scheduling, task, order, machine, preempt, activity, transportation, resource, inventory, job & OSP &  &  &  & patient, physician, medical, nurse &  & https://, http://, github, real-world & \\
\href{articles/Siala15.pdf}{Siala15}~\cite{Siala15} & resource, scheduling &  & disjunctive &  &  &  &  & http://, benchmark & \\
\href{articles/SimoninAHL15.pdf}{SimoninAHL15}~\cite{SimoninAHL15} & resource, activity, precedence, preempt, scheduling, order, inventory, transportation, task, make-span &  & disjunctive, span constraint, cumulative, cycle &  & CHIP & earth observation, satellite, pipeline, robot &  &  & sweep\\
\href{articles/Simonis07.pdf}{Simonis07}~\cite{Simonis07} & due-date, job-shop, batch process, transportation, resource, scheduling, make to order, task, machine, producer/consumer, order, bill of material, job, activity, re-scheduling, setup-time, release-date, sequence dependent setup & OSP & disjunctive, cumulative, alldifferent, cycle, diffn, bin-packing & Prolog & OZ, OPL, CHIP, Ilog Scheduler & aircraft, patient, nurse, medical &  &  & time-tabling, sweep, bi-partite matching\\
\href{articles/SubulanC22.pdf}{SubulanC22}~\cite{SubulanC22} & scheduling, tardiness, task, order, due-date, machine, preempt, activity, make-span, BOM, completion-time, precedence, transportation, resource, inventory & RCPSP & endBeforeStart, cumulative &  & Cplex, OZ, OPL & offshore &  & https://, real-life, benchmark, real-world & \\
\href{articles/Timpe02.pdf}{Timpe02}~\cite{Timpe02} & due-date, order, machine, inventory, task, job, activity, stock level, setup-time, resource, make-span, scheduling, producer/consumer &  & cumulative, disjunctive, diffn, cycle & C++ & CHIP, Cplex &  & chemical industry, process industry & http:// & \\
\href{articles/TopalogluO11.pdf}{TopalogluO11}~\cite{TopalogluO11} & order, re-scheduling, task, distributed, transportation, preempt, scheduling & OSP &  &  & Cplex, OPL, OZ, Ilog Solver & nurse, medical, physician, patient &  & http://, real-life & time-tabling\\
\href{articles/TrojetHL11.pdf}{TrojetHL11}~\cite{TrojetHL11} & order, job-shop, machine, activity, make-span, completion-time, job, precedence, distributed, resource, due-date, scheduling, task & RCPSP & cumulative, diffn, disjunctive, cycle, alldifferent & Prolog & OZ, CHIP, SICStus & robot &  & real-world, http:// & \\
\href{articles/Tsang03.pdf}{Tsang03}~\cite{Tsang03} & resource, scheduling &  &  &  &  &  &  & real-life & time-tabling\\
\href{articles/VilimBC05.pdf}{VilimBC05}~\cite{VilimBC05} & setup-time, sequence dependent setup, distributed, job-shop, batch process, resource, scheduling, make-span, open-shop, completion-time, task, machine, precedence, order, job, activity &  & disjunctive, cumulative, cycle &  &  &  &  & http://, benchmark, real-life & not-first, sweep, edge-finding, not-last\\
\href{articles/VlkHT21.pdf}{VlkHT21}~\cite{VlkHT21} & tardiness, due-date, completion-time, order, distributed, precedence, resource, scheduling & PMSP & alternative constraint, noOverlap &  & OPL, Cplex, Gurobi, Z3 & automotive, robot &  & https://, industrial partner, random instance, github, http://, benchmark & \\
\href{articles/Wallace96.pdf}{Wallace96}~\cite{Wallace96} & job-shop, transportation, distributed, task, resource, scheduling, multi-agent, order, machine, job, activity & OSP & circuit, disjunctive, cycle & Prolog, Lisp & CHIP, Ilog Solver, ECLiPSe, OZ, OPL & automotive, aircraft, railway, robot & process industry, automotive industry & http:// & time-tabling\\
\href{articles/WallaceY20.pdf}{WallaceY20}~\cite{WallaceY20} & scheduling, machine, flow-shop, order, transportation, job, lazy clause generation, resource, task, job-shop & CHSP & circuit, cumulative, disjunctive, cycle &  & Chuffed, OPL, Gecode, Gurobi, Cplex, MiniZinc & robot, hoist, electroplating, yard crane, container terminal &  & random instance, https://, real-life, real-world, http://, benchmark & edge-finding, time-tabling\\
\href{articles/WangMD15.pdf}{WangMD15}~\cite{WangMD15} & make-span, scheduling, job, resource, activity, completion-time, job-shop, task, precedence, order, cmax, re-scheduling & OSP & noOverlap, cumulative &  & OPL, Cplex, OZ & nurse, medical, physician, patient &  & real-life, real-world, https://, http:// & time-tabling\\
\href{articles/WikarekS19.pdf}{WikarekS19}~\cite{WikarekS19} & multi-agent, scheduling, machine, preempt, manpower, flow-shop, order, make-span, cmax, resource, inventory, job, precedence, distributed, setup-time, task, job-shop & JSSP, RCPSP & cumulative, disjunctive &  & OZ, Z3, ECLiPSe & robot &  &  & \\
\href{articles/YuraszeckMCCR23.pdf}{YuraszeckMCCR23}~\cite{YuraszeckMCCR23} & setup-time, cmax, activity, make-span, machine, open-shop, precedence, resource, preempt, batch process, task, flow-shop, order, scheduling, job, job-shop, flow-time & RCPSP, Open Shop Scheduling Problem, JSSP, FJS, OSSP & endBeforeStart, cumulative &  & OPL, Cplex &  & pharmaceutical industry & https://, github, real-world, benchmark & \\
\href{articles/ZarandiKS16.pdf}{ZarandiKS16}~\cite{ZarandiKS16} & make-span, job, scheduling, completion-time, resource, order, task, machine, preempt, earliness, distributed, due-date, tardiness, flow-shop, job-shop, transportation & single machine &  &  & Ilog Solver & robot &  & real-world & time-tabling\\
\href{articles/ZeballosH05.pdf}{ZeballosH05}~\cite{ZeballosH05} & transportation, scheduling, buffer-capacity, completion-time, make-span, order, job, activity, due-date, resource, task, machine, tardiness, precedence &  &  &  & Ilog Scheduler, OPL, Ilog Solver & robot &  & http:// & \\
\href{articles/ZeballosQH10.pdf}{ZeballosQH10}~\cite{ZeballosQH10} & cmax, make-span, resource, activity, precedence, completion-time, earliness, job-shop, transportation, due-date, preempt, scheduling, order, machine, tardiness, job, task &  &  &  & ECLiPSe, CHIP, Ilog Solver, OZ, Cplex, Ilog Scheduler, OPL & robot &  & benchmark, real-world, http:// & \\
\href{articles/ZhangW18.pdf}{ZhangW18}~\cite{ZhangW18} & job, completion-time, flow-shop, precedence, lateness, job-shop, re-scheduling, transportation, multi-agent, earliness, order, preempt, flow-time, make-span, distributed, resource, tardiness, scheduling, machine, setup-time & OSP, FJS & noOverlap, cumulative &  & Cplex, Z3, OPL & robot &  & benchmark, http:// & \\
\href{articles/ZhangYW21.pdf}{ZhangYW21}~\cite{ZhangYW21} & cmax, task, machine, job, activity, re-scheduling, release-date, setup-time, preempt, distributed, job-shop, batch process, resource, scheduling, multi-agent, make-span, precedence, order & RCPSP & endBeforeStart, disjunctive &  & Cplex & robot &  & https://, benchmark & \\
\href{articles/Zhou97.pdf}{Zhou97}~\cite{Zhou97} & release-date, job-shop, due-date, task, order, preempt, scheduling, precedence, completion-time, job, machine &  & cumulative, disjunctive & Prolog & CHIP, Ilog Scheduler, Z3 &  &  & benchmark & edge-finding, edge-finder\\
\href{articles/abs-0907-0939.pdf}{abs-0907-0939}~\cite{abs-0907-0939} & resource, order, activity, due-date, preempt, scheduling, make-span, release-date, task &  & cumulative & Java & Choco Solver, CHIP &  &  & http://, real-world & sweep, energetic reasoning, edge-finding\\
\href{articles/abs-1901-07914.pdf}{abs-1901-07914}~\cite{abs-1901-07914} & multi-agent, scheduling, order, resource, make-span, distributed, machine, task &  &  & Python & OZ, MiniZinc, OR-Tools & robot &  & benchmark, real-world, https://, github, http:// & \\
\href{articles/abs-1902-01193.pdf}{abs-1902-01193}~\cite{abs-1902-01193} & order, resource, activity, BOM, task, scheduling & OSP &  & C++, Prolog, Python & Ilog Solver, CHIP, OPL & medical, nurse &  &  & time-tabling\\
\href{articles/abs-1902-09244.pdf}{abs-1902-09244}~\cite{abs-1902-09244} & order, tardiness, completion-time, resource, setup-time, activity, inventory, task, machine, due-date, precedence, transportation, earliness, flow-shop, job-shop, scheduling, job, make-span, release-date & FJS, RCPSP & cumulative, endBeforeStart, cycle &  & Cplex, OZ, OPL & aircraft & steel industry, food-processing industry & benchmark, industry partner, https://, real-world & \\
\href{articles/abs-1911-04766.pdf}{abs-1911-04766}~\cite{abs-1911-04766} & release-date, scheduling, order, completion-time, job, re-scheduling, resource, make-span, activity, due-date, precedence, task & RCPSP & noOverlap, disjunctive, cumulative, alternative constraint, endBeforeStart & Java & OZ, MiniZinc, CPO, Chuffed, Gecode, Cplex & automotive &  & https://, real-world, generated instance, industrial partner, http://, github, benchmark, instance generator, real-life & time-tabling\\
\href{articles/abs-2211-14492.pdf}{abs-2211-14492}~\cite{abs-2211-14492} & resource, setup-time, distributed, activity, due-date, precedence, task, flow-shop, machine, transportation, job-shop, scheduling, order, job, make-span, tardiness, completion-time, cmax & single machine & bin-packing, cumulative, disjunctive & Python & Cplex, OR-Tools, OZ & semiconductor &  & benchmark, https://, random instance, generated instance & \\
\href{articles/abs-2305-19888.pdf}{abs-2305-19888}~\cite{abs-2305-19888} & scheduling, order, job, re-scheduling, make-span, completion-time, cmax, sequence dependent setup, preempt, resource, setup-time, distributed, activity, precedence, task, flow-shop, machine & parallel machine & noOverlap, cumulative, alternative constraint &  & Gurobi & robot &  & https://, real-world, generated instance, http://, gitlab, benchmark & \\
\href{articles/abs-2306-05747.pdf}{abs-2306-05747}~\cite{abs-2306-05747} & job-shop, re-scheduling, flow-time, scheduling, order, completion-time, job, resource, make-span, tardiness, preempt, machine, precedence, task, flow-shop & JSSP & noOverlap, disjunctive, cumulative & Java & Choco Solver &  &  & real-world, supplementary material, github, https://, industrial instance, benchmark & \\
\href{articles/abs-2312-13682.pdf}{abs-2312-13682}~\cite{abs-2312-13682} & re-scheduling, scheduling, order, resource, make-span, activity, machine, transportation, inventory, task & OSP & cumulative, table constraint &  & OPL & steel mill, container terminal, nurse &  & real-world, generated instance & \\
\href{articles/abs-2402-00459.pdf}{abs-2402-00459}~\cite{abs-2402-00459} & machine, due-date, earliness, job-shop, scheduling, order, job, multi-agent, tardiness, completion-time, resource, precedence, task & single machine & disjunctive, bin-packing, cumulative &  & OPL, OR-Tools &  & mining industry & instance generator, https://, real-world, http://, generated instance, github, benchmark & \\
\end{longtable}
}



\clearpage
\subsection{Manually Defined Fields}
{\scriptsize
\begin{longtable}{>{\raggedright\arraybackslash}p{3cm}>{\raggedright\arraybackslash}p{6cm}p{2cm}rrrrlrr}
\rowcolor{white}\caption{Manually Defined ARTICLE Properties}\\ \toprule
\rowcolor{white}Key & Title (Local Copy)  & Bench & Links & \shortstack{Data\\Avail} & \shortstack{Sol\\Avail} & \shortstack{Code\\Avail} & \shortstack{Related\\To} & a & b\\ \midrule\endhead
\bottomrule
\endfoot
\rowlabel{c:ForbesHJST24}ForbesHJST24 \href{http://dx.doi.org/10.1016/j.ejor.2023.07.032}{ForbesHJST24}~\cite{ForbesHJST24} & \href{../works/ForbesHJST24.pdf}{Combining optimisation and simulation using logic-based Benders decomposition} & benchmark, real-life, github & 1 &  &  &  &  & \ref{a:ForbesHJST24} & \ref{b:ForbesHJST24}\\
\rowlabel{c:PrataAN23}PrataAN23 \href{https://www.sciencedirect.com/science/article/pii/S2666720723001522}{PrataAN23}~\cite{PrataAN23} & \href{../works/PrataAN23.pdf}{Applications of constraint programming in production scheduling problems: A descriptive bibliometric analysis} & real-life, benchmark, real-world & 1 & - &  & - & - & \ref{a:PrataAN23} & \ref{b:PrataAN23}\\
\rowlabel{c:abs-2402-00459}abs-2402-00459 \href{https://doi.org/10.48550/arXiv.2402.00459}{abs-2402-00459}~\cite{abs-2402-00459} & \href{../works/abs-2402-00459.pdf}{Genetic-based Constraint Programming for Resource Constrained Job Scheduling} & instance generator, real-world, generated instance, benchmark, github & 2 & \href{https://github.com/andreas-ernst/Mathprog-ORlib/blob/master/data/RCJS_new_instances.zip}{y} &  & n & - & \ref{a:abs-2402-00459} & \ref{b:abs-2402-00459}\\
\rowlabel{c:AbreuNP23}AbreuNP23 \href{https://doi.org/10.1080/00207543.2022.2154404}{AbreuNP23}~\cite{AbreuNP23} & \href{../works/AbreuNP23.pdf}{A new two-stage constraint programming approach for open shop scheduling problem with machine blocking} & real-world, benchmark & 10 & ? &  & ? & ? & \ref{a:AbreuNP23} & \ref{b:AbreuNP23}\\
\rowlabel{c:Adelgren2023}Adelgren2023 \href{http://dx.doi.org/10.1016/j.cie.2023.109330}{Adelgren2023}~\cite{Adelgren2023} & \href{../works/Adelgren2023.pdf}{On the utility of production scheduling formulations including record keeping variables} & benchmark, real-life, github, generated instance, supplementary material & 12 &  &  &  &  & \ref{a:Adelgren2023} & \ref{b:Adelgren2023}\\
\rowlabel{c:AfsarVPG23}AfsarVPG23 \href{http://dx.doi.org/10.1016/j.cie.2023.109454}{AfsarVPG23}~\cite{AfsarVPG23} & \href{../works/AfsarVPG23.pdf}{Mathematical models and benchmarking for the fuzzy job shop scheduling problem} & benchmark, real-life, supplementary material, real-world & 96 &  &  &  &  & \ref{a:AfsarVPG23} & \ref{b:AfsarVPG23}\\
\rowlabel{c:AkramNHRSA23}AkramNHRSA23 \href{https://doi.org/10.1109/ACCESS.2023.3343409}{AkramNHRSA23}~\cite{AkramNHRSA23} & \href{../works/AkramNHRSA23.pdf}{Joint Scheduling and Routing Optimization for Deterministic Hybrid Traffic in Time-Sensitive Networks Using Constraint Programming} & benchmark & 0 & n &  & n & - & \ref{a:AkramNHRSA23} & \ref{b:AkramNHRSA23}\\
\rowlabel{c:Caballero23}Caballero23 \href{https://doi.org/10.1007/s10601-023-09357-0}{Caballero23}~\cite{Caballero23} & \href{../works/Caballero23.pdf}{Scheduling through logic-based tools} &  & 1 & - &  & - & \href{http://hdl.handle.net/10803/667963}{PhD Thesis} & \ref{a:Caballero23} & \ref{b:Caballero23}\\
\rowlabel{c:Fatemi-AnarakiTFV23}Fatemi-AnarakiTFV23 \href{http://dx.doi.org/10.1016/j.omega.2022.102770}{Fatemi-AnarakiTFV23}~\cite{Fatemi-AnarakiTFV23} & \href{../works/Fatemi-AnarakiTFV23.pdf}{Scheduling of Multi-Robot Job Shop Systems in Dynamic Environments: Mixed-Integer Linear Programming and Constraint Programming Approaches} & random instance, github, real-world & 2 &  &  &  &  & \ref{a:Fatemi-AnarakiTFV23} & \ref{b:Fatemi-AnarakiTFV23}\\
\rowlabel{c:GokPTGO23}GokPTGO23 \href{https://ideas.repec.org/a/spr/annopr/v320y2023i2d10.1007_s10479-022-04547-0.html}{GokPTGO23}~\cite{GokPTGO23} & \href{../works/GokPTGO23.pdf}{{Constraint-based robust planning and scheduling of airport apron operations through simheuristics}} & real-world, github & 10 &  &  &  &  & \ref{a:GokPTGO23} & \ref{b:GokPTGO23}\\
\rowlabel{c:GuoZ23}GuoZ23 \href{http://dx.doi.org/10.1016/j.ejor.2023.06.006}{GuoZ23}~\cite{GuoZ23} & \href{../works/GuoZ23.pdf}{Capacity reservation for humanitarian relief: A logic-based Benders decomposition method with subgradient cut} & real-world, supplementary material, benchmark, github & 14 &  &  &  &  & \ref{a:GuoZ23} & \ref{b:GuoZ23}\\
\rowlabel{c:GurPAE23}GurPAE23 \href{https://doi.org/10.1007/s10100-022-00835-z}{GurPAE23}~\cite{GurPAE23} & \href{../works/GurPAE23.pdf}{Operating room scheduling with surgical team: a new approach with constraint programming and goal programming} & real-life & 0 & n &  & n & - & \ref{a:GurPAE23} & \ref{b:GurPAE23}\\
\rowlabel{c:IsikYA23}IsikYA23 \href{https://doi.org/10.1007/s00500-023-09086-9}{IsikYA23}~\cite{IsikYA23} & \href{../works/IsikYA23.pdf}{Constraint programming models for the hybrid flow shop scheduling problem and its extensions} & benchmark, generated instance, real-life, real-world & 4 & \href{https://data.mendeley.com/datasets/n4g8cfjg87/1}{y} &  & \href{https://data.mendeley.com/datasets/n4g8cfjg87/1}{y} & - & \ref{a:IsikYA23} & \ref{b:IsikYA23}\\
\rowlabel{c:LacknerMMWW23}LacknerMMWW23 \href{https://doi.org/10.1007/s10601-023-09347-2}{LacknerMMWW23}~\cite{LacknerMMWW23} & \href{../works/LacknerMMWW23.pdf}{Exact methods for the Oven Scheduling Problem} & zenodo, random instance, benchmark, instance generator, real-life, industrial partner & 0 & \href{https://zenodo.org/records/7456938}{\su{DZN JSON}} &  & \href{https://zenodo.org/records/7456938}{y} & \cite{LacknerMMWW21} & \ref{a:LacknerMMWW23} & \ref{b:LacknerMMWW23}\\
\rowlabel{c:MontemanniD23}MontemanniD23 \href{https://doi.org/10.3390/a16010040}{MontemanniD23}~\cite{MontemanniD23} & \href{../works/MontemanniD23.pdf}{Solving the Parallel Drone Scheduling Traveling Salesman Problem via Constraint Programming} & benchmark, supplementary material & 6 & ref & \href{https://www.mdpi.com/article/10.3390/a16010040/s1}{y} & n & - & \ref{a:MontemanniD23} & \ref{b:MontemanniD23}\\
\rowlabel{c:MontemanniD23a}MontemanniD23a \href{https://doi.org/10.1016/j.ejco.2023.100078}{MontemanniD23a}~\cite{MontemanniD23a} & \href{../works/MontemanniD23a.pdf}{Constraint programming models for the parallel drone scheduling vehicle routing problem} & benchmark & 0 & ref &  & n & - & \ref{a:MontemanniD23a} & \ref{b:MontemanniD23a}\\
\rowlabel{c:NaderiBZR23}NaderiBZR23 \href{http://dx.doi.org/10.1016/j.omega.2022.102805}{NaderiBZR23}~\cite{NaderiBZR23} & \href{../works/NaderiBZR23.pdf}{A novel and efficient exact technique for integrated staffing, assignment, routing, and scheduling of home care services under uncertainty} & supplementary material, benchmark, real-world, real-life, generated instance & 7 &  &  &  &  & \ref{a:NaderiBZR23} & \ref{b:NaderiBZR23}\\
\rowlabel{c:NaderiRR23}NaderiRR23 \href{https://doi.org/10.1287/ijoc.2023.1287}{NaderiRR23}~\cite{NaderiRR23} & \href{../works/NaderiRR23.pdf}{Mixed-Integer Programming vs. Constraint Programming for Shop Scheduling Problems: New Results and Outlook} & github, benchmark & 8 &  &  &  &  & \ref{a:NaderiRR23} & \ref{b:NaderiRR23}\\
\rowlabel{c:ShaikhK23}ShaikhK23 \href{https://doi.org/10.1504/IJESDF.2023.10045616}{ShaikhK23}~\cite{ShaikhK23} & \href{../works/ShaikhK23.pdf}{Management of electronic ledger: a constraint programming approach for solving curricula scheduling problems} & benchmark, real-world & 2 & ? &  & ? & ? & \ref{a:ShaikhK23} & \ref{b:ShaikhK23}\\
\rowlabel{c:YuraszeckMCCR23}YuraszeckMCCR23 \href{https://doi.org/10.1109/ACCESS.2023.3345793}{YuraszeckMCCR23}~\cite{YuraszeckMCCR23} & \href{../works/YuraszeckMCCR23.pdf}{A Constraint Programming Formulation of the Multi-Mode Resource-Constrained Project Scheduling Problem for the Flexible Job Shop Scheduling Problem} & benchmark, github, real-world & 0 & ref &  & n & - & \ref{a:YuraszeckMCCR23} & \ref{b:YuraszeckMCCR23}\\
\rowlabel{c:abs-2305-19888}abs-2305-19888 \href{https://doi.org/10.48550/arXiv.2305.19888}{abs-2305-19888}~\cite{abs-2305-19888} & \href{../works/abs-2305-19888.pdf}{Constraint Programming and Constructive Heuristics for Parallel Machine Scheduling with Sequence-Dependent Setups and Common Servers} & gitlab, generated instance, real-world, benchmark & 1 & \href{https://gitlab.com/vilem_heinz/cp_heur_paper_evalutation}{y} & \href{https://gitlab.com/vilem_heinz/cp_heur_paper_evalutation}{y} & n & - & \ref{a:abs-2305-19888} & \ref{b:abs-2305-19888}\\
\rowlabel{c:abs-2306-05747}abs-2306-05747 \href{https://doi.org/10.48550/arXiv.2306.05747}{abs-2306-05747}~\cite{abs-2306-05747} & \href{../works/abs-2306-05747.pdf}{An End-to-End Reinforcement Learning Approach for Job-Shop Scheduling Problems Based on Constraint Programming} & real-world, github, industrial instance, supplementary material, benchmark & 0 & ref &  & n & - & \ref{a:abs-2306-05747} & \ref{b:abs-2306-05747}\\
\rowlabel{c:abs-2312-13682}abs-2312-13682 \href{https://doi.org/10.48550/arXiv.2312.13682}{abs-2312-13682}~\cite{abs-2312-13682} & \href{../works/abs-2312-13682.pdf}{A Constraint Programming Model for Scheduling the Unloading of Trains in Ports: Extended} & real-world, generated instance & 0 & n &  & n & - & \ref{a:abs-2312-13682} & \ref{b:abs-2312-13682}\\
\rowlabel{c:AbreuN22}AbreuN22 \href{https://doi.org/10.1016/j.cie.2022.108128}{AbreuN22}~\cite{AbreuN22} & \href{../works/AbreuN22.pdf}{A new hybridization of adaptive large neighborhood search with constraint programming for open shop scheduling with sequence-dependent setup times} & real-world, benchmark & 0 & \href{https://bit.ly/392wfZa}{y} &  & n & - & \ref{a:AbreuN22} & \ref{b:AbreuN22}\\
\rowlabel{c:CampeauG22}CampeauG22 \href{https://doi.org/10.1007/s10601-022-09337-w}{CampeauG22}~\cite{CampeauG22} & \href{../works/CampeauG22.pdf}{Short- and medium-term optimization of underground mine planning using constraint programming} & real-life, real-world & 0 & ref &  & n &  & \ref{a:CampeauG22} & \ref{b:CampeauG22}\\
\rowlabel{c:ColT22}ColT22 \href{http://dx.doi.org/10.1016/j.orp.2022.100249}{ColT22}~\cite{ColT22} & \href{../works/ColT22.pdf}{Industrial-size job shop scheduling with constraint programming} & supplementary material, benchmark, generated instance, github, real-life, real-world & 4 &  &  &  &  & \ref{a:ColT22} & \ref{b:ColT22}\\
\rowlabel{c:EmdeZD22}EmdeZD22 \href{http://dx.doi.org/10.1007/s10479-022-04891-1}{EmdeZD22}~\cite{EmdeZD22} & \href{../works/EmdeZD22.pdf}{Point-to-point and milk run delivery scheduling: models, complexity results, and algorithms based on Benders decomposition} & github, random instance & 7 &  &  &  &  & \ref{a:EmdeZD22} & \ref{b:EmdeZD22}\\
\rowlabel{c:FarsiTM22}FarsiTM22 \href{https://api.semanticscholar.org/CorpusID:250301745}{FarsiTM22}~\cite{FarsiTM22} & \href{../works/FarsiTM22.pdf}{Integrated surgery scheduling by constraint programming and meta-heuristics} & supplementary material & 10 &  &  &  &  & \ref{a:FarsiTM22} & \ref{b:FarsiTM22}\\
\rowlabel{c:HeinzNVH22}HeinzNVH22 \href{https://doi.org/10.1016/j.cie.2022.108586}{HeinzNVH22}~\cite{HeinzNVH22} & \href{../works/HeinzNVH22.pdf}{Constraint Programming and constructive heuristics for parallel machine scheduling with sequence-dependent setups and common servers} & real-world, generated instance, benchmark, gitlab & 3 &  &  &  &  & \ref{a:HeinzNVH22} & \ref{b:HeinzNVH22}\\
\rowlabel{c:MullerMKP22}MullerMKP22 \href{https://doi.org/10.1016/j.ejor.2022.01.034}{MullerMKP22}~\cite{MullerMKP22} & \href{../works/MullerMKP22.pdf}{An algorithm selection approach for the flexible job shop scheduling problem: Choosing constraint programming solvers through machine learning} & random instance, benchmark, github, real-world & 3 &  &  &  &  & \ref{a:MullerMKP22} & \ref{b:MullerMKP22}\\
\rowlabel{c:YunusogluY22}YunusogluY22 \href{https://doi.org/10.1080/00207543.2021.1885068}{YunusogluY22}~\cite{YunusogluY22} & \href{../works/YunusogluY22.pdf}{Constraint programming approach for multi-resource-constrained unrelated parallel machine scheduling problem with sequence-dependent setup times} & benchmark, real-life, real-world, generated instance, supplementary material & 10 &  &  &  &  & \ref{a:YunusogluY22} & \ref{b:YunusogluY22}\\
\rowlabel{c:YuraszeckMPV22}YuraszeckMPV22 \href{http://dx.doi.org/10.3390/math10030329}{YuraszeckMPV22}~\cite{YuraszeckMPV22} & \href{../works/YuraszeckMPV22.pdf}{A Novel Constraint Programming Decomposition Approach for the Total Flow Time Fixed Group Shop Scheduling Problem} & real-life, generated instance, benchmark, github & 5 &  &  &  &  & \ref{a:YuraszeckMPV22} & \ref{b:YuraszeckMPV22}\\
\rowlabel{c:AbohashimaEG21}AbohashimaEG21 \href{https://doi.org/10.1109/ACCESS.2021.3112600}{AbohashimaEG21}~\cite{AbohashimaEG21} & \href{../works/AbohashimaEG21.pdf}{A Mathematical Programming Model and a Firefly-Based Heuristic for Real-Time Traffic Signal Scheduling With Physical Constraints} & real-world, generated instance, github & 0 &  &  &  &  & \ref{a:AbohashimaEG21} & \ref{b:AbohashimaEG21}\\
\rowlabel{c:HamP21}HamP21 \href{http://dx.doi.org/10.1109/lra.2021.3056069}{HamP21}~\cite{HamP21} & \href{../works/HamP21.pdf}{Human–Robot Task Allocation and Scheduling: Boeing 777 Case Study} & real-world, benchmark, github & 1 &  &  &  &  & \ref{a:HamP21} & \ref{b:HamP21}\\
\rowlabel{c:HamPK21}HamPK21 \href{https://api.semanticscholar.org/CorpusID:237898414}{HamPK21}~\cite{HamPK21} & \href{../works/HamPK21.pdf}{Energy-Aware Flexible Job Shop Scheduling Using Mixed Integer Programming and Constraint Programming} & benchmark, github & 4 &  &  &  &  & \ref{a:HamPK21} & \ref{b:HamPK21}\\
\rowlabel{c:KoehlerBFFHPSSS21}KoehlerBFFHPSSS21 \href{https://doi.org/10.1007/s10601-021-09321-w}{KoehlerBFFHPSSS21}~\cite{KoehlerBFFHPSSS21} & \href{../works/KoehlerBFFHPSSS21.pdf}{Cable tree wiring - benchmarking solvers on a real-world scheduling problem with a variety of precedence constraints} & real-world, benchmark, github & 9 & \href{https://github.com/kw90/ctw_toolchain}{DZN} &  & y & - & \ref{a:KoehlerBFFHPSSS21} & \ref{b:KoehlerBFFHPSSS21}\\
\rowlabel{c:VlkHT21}VlkHT21 \href{https://doi.org/10.1016/j.cie.2021.107317}{VlkHT21}~\cite{VlkHT21} & \href{../works/VlkHT21.pdf}{Constraint programming approaches to joint routing and scheduling in time-sensitive networks} & benchmark, industrial partner, random instance, github & 0 &  &  &  &  & \ref{a:VlkHT21} & \ref{b:VlkHT21}\\
\rowlabel{c:BenediktMH20}BenediktMH20 \href{https://doi.org/10.1007/s10601-020-09317-y}{BenediktMH20}~\cite{BenediktMH20} & \href{../works/BenediktMH20.pdf}{Power of pre-processing: production scheduling with variable energy pricing and power-saving states} & benchmark, generated instance, random instance, github & 4 & \href{https://github.com/CTU-IIG/EnergyStatesAndCostsSchedulingData}{JSON} &  & \href{https://github.com/CTU-IIG/EnergyStatesAndCostsScheduling}{y} &  & \ref{a:BenediktMH20} & \ref{b:BenediktMH20}\\
\rowlabel{c:CauwelaertDS20}CauwelaertDS20 \href{http://dx.doi.org/10.1007/s10951-019-00632-8}{CauwelaertDS20}~\cite{CauwelaertDS20} & \href{../works/CauwelaertDS20.pdf}{An Efficient Filtering Algorithm for the Unary Resource Constraint with Transition Times and Optional Activities} & real-life, generated instance, benchmark, bitbucket & 2 &  &  &  &  & \ref{a:CauwelaertDS20} & \ref{b:CauwelaertDS20}\\
\rowlabel{c:FachiniA20}FachiniA20 \href{http://dx.doi.org/10.1016/j.cie.2020.106641}{FachiniA20}~\cite{FachiniA20} & \href{../works/FachiniA20.pdf}{Logic-based Benders decomposition for the heterogeneous fixed fleet vehicle routing problem with time windows} & supplementary material, real-world, github, benchmark, real-life & 6 &  &  &  &  & \ref{a:FachiniA20} & \ref{b:FachiniA20}\\
\rowlabel{c:FallahiAC20}FallahiAC20 \href{https://api.semanticscholar.org/CorpusID:213449737}{FallahiAC20}~\cite{FallahiAC20} & \href{../works/FallahiAC20.pdf}{Tabu search and constraint programming-based approach for a real scheduling and routing problem} & real-life, github & 0 &  &  &  &  & \ref{a:FallahiAC20} & \ref{b:FallahiAC20}\\
\rowlabel{c:HauderBRPA20}HauderBRPA20 \href{http://dx.doi.org/10.1016/j.cie.2020.106857}{HauderBRPA20}~\cite{HauderBRPA20} & \href{../works/HauderBRPA20.pdf}{Resource-constrained multi-project scheduling with activity and time flexibility} & industry partner, benchmark, real-world, supplementary material & 0 &  &  &  &  & \ref{a:HauderBRPA20} & \ref{b:HauderBRPA20}\\
\rowlabel{c:LunardiBLRV20}LunardiBLRV20 \href{https://doi.org/10.1016/j.cor.2020.105020}{LunardiBLRV20}~\cite{LunardiBLRV20} & \href{../works/LunardiBLRV20.pdf}{Mixed Integer linear programming and constraint programming models for the online printing shop scheduling problem} & benchmark, github, random instance, generated instance & 1 &  &  &  &  & \ref{a:LunardiBLRV20} & \ref{b:LunardiBLRV20}\\
\rowlabel{c:MejiaY20}MejiaY20 \href{https://doi.org/10.1016/j.ejor.2020.02.010}{MejiaY20}~\cite{MejiaY20} & \href{../works/MejiaY20.pdf}{A self-tuning variable neighborhood search algorithm and an effective decoding scheme for open shop scheduling problems with travel/setup times} & supplementary material, benchmark, generated instance & 2 &  &  &  &  & \ref{a:MejiaY20} & \ref{b:MejiaY20}\\
\rowlabel{c:MengZRZL20}MengZRZL20 \href{https://doi.org/10.1016/j.cie.2020.106347}{MengZRZL20}~\cite{MengZRZL20} & \href{../works/MengZRZL20.pdf}{Mixed-integer linear programming and constraint programming formulations for solving distributed flexible job shop scheduling problem} & benchmark, supplementary material & 0 &  &  &  &  & \ref{a:MengZRZL20} & \ref{b:MengZRZL20}\\
\rowlabel{c:Polo-MejiaALB20}Polo-MejiaALB20 \href{https://doi.org/10.1080/00207543.2019.1693654}{Polo-MejiaALB20}~\cite{Polo-MejiaALB20} & \href{../works/Polo-MejiaALB20.pdf}{Mixed-integer/linear and constraint programming approaches for activity scheduling in a nuclear research facility} & github, Roadef & 2 &  &  &  &  & \ref{a:Polo-MejiaALB20} & \ref{b:Polo-MejiaALB20}\\
\rowlabel{c:SacramentoSP20}SacramentoSP20 \href{https://doi.org/10.1007/s43069-020-00036-x}{SacramentoSP20}~\cite{SacramentoSP20} & \href{../works/SacramentoSP20.pdf}{Constraint Programming and Local Search Heuristic: a Matheuristic Approach for Routing and Scheduling Feeder Vessels in Multi-terminal Ports} & benchmark, real-life, zenodo, real-world & 4 &  &  &  &  & \ref{a:SacramentoSP20} & \ref{b:SacramentoSP20}\\
\rowlabel{c:WallaceY20}WallaceY20 \href{https://doi.org/10.1007/s10601-020-09316-z}{WallaceY20}~\cite{WallaceY20} & \href{../works/WallaceY20.pdf}{A new constraint programming model and solving for the cyclic hoist scheduling problem} & real-world, benchmark, random instance, real-life & 2 & \href{https://data.4tu.nl/articles/_/12912413}{DZN} &  & \href{https://data.4tu.nl/articles/_/12912413}{y} &  & \ref{a:WallaceY20} & \ref{b:WallaceY20}\\
\rowlabel{c:ColT2019a}ColT2019a \href{http://dx.doi.org/10.4204/eptcs.306.30}{ColT2019a}~\cite{ColT2019a} & \href{../works/ColT2019a.pdf}{Google vs IBM: A Constraint Solving Challenge on the Job-Shop Scheduling Problem} & benchmark, github, real-world & 1 &  &  &  &  & \ref{a:ColT2019a} & \ref{b:ColT2019a}\\
\rowlabel{c:HoundjiSW19}HoundjiSW19 \href{https://doi.org/10.1007/s10601-018-9300-y}{HoundjiSW19}~\cite{HoundjiSW19} & \href{../works/HoundjiSW19.pdf}{The item dependent stockingcost constraint} & random instance, benchmark, bitbucket & 2 &  &  &  &  & \ref{a:HoundjiSW19} & \ref{b:HoundjiSW19}\\
\rowlabel{c:SunTB19}SunTB19 \href{http://dx.doi.org/10.1016/j.ejor.2018.08.009}{SunTB19}~\cite{SunTB19} & \href{../works/SunTB19.pdf}{A Benders decomposition-based framework for solving quay crane scheduling problems} & generated instance, instance generator, benchmark, github, real-life & 1 &  &  &  &  & \ref{a:SunTB19} & \ref{b:SunTB19}\\
\rowlabel{c:TanZWGQ19}TanZWGQ19 \href{http://dx.doi.org/10.1109/tase.2019.2894093}{TanZWGQ19}~\cite{TanZWGQ19} & \href{../works/TanZWGQ19.pdf}{A Hybrid MIP–CP Approach to Multistage Scheduling Problem in Continuous Casting and Hot-Rolling Processes} & real-world, generated instance, supplementary material & 0 &  &  &  &  & \ref{a:TanZWGQ19} & \ref{b:TanZWGQ19}\\
\rowlabel{c:abs-1901-07914}abs-1901-07914 \href{http://arxiv.org/abs/1901.07914}{abs-1901-07914}~\cite{abs-1901-07914} & \href{../works/abs-1901-07914.pdf}{A Constraint Programming Approach to Simultaneous Task Allocation and Motion Scheduling for Industrial Dual-Arm Manipulation Tasks} & real-world, github, benchmark & 0 &  &  &  &  & \ref{a:abs-1901-07914} & \ref{b:abs-1901-07914}\\
\rowlabel{c:abs-1911-04766}abs-1911-04766 \href{http://arxiv.org/abs/1911.04766}{abs-1911-04766}~\cite{abs-1911-04766} & \href{../works/abs-1911-04766.pdf}{Investigating Constraint Programming and Hybrid Methods for Real World Industrial Test Laboratory Scheduling} & real-world, benchmark, github, real-life, instance generator, generated instance, industrial partner & 10 &  &  &  &  & \ref{a:abs-1911-04766} & \ref{b:abs-1911-04766}\\
\rowlabel{c:CauwelaertLS18}CauwelaertLS18 \href{https://doi.org/10.1007/s10601-017-9277-y}{CauwelaertLS18}~\cite{CauwelaertLS18} & \href{../works/CauwelaertLS18.pdf}{How efficient is a global constraint in practice? - {A} fair experimental framework} & benchmark, bitbucket & 1 &  &  &  &  & \ref{a:CauwelaertLS18} & \ref{b:CauwelaertLS18}\\
\rowlabel{c:FahimiOQ18}FahimiOQ18 \href{https://doi.org/10.1007/s10601-018-9282-9}{FahimiOQ18}~\cite{FahimiOQ18} & \href{../works/FahimiOQ18.pdf}{Linear-time filtering algorithms for the disjunctive constraint and a quadratic filtering algorithm for the cumulative not-first not-last} & benchmark, random instance & 0 & (y) &  & n &  & \ref{a:FahimiOQ18} & \ref{b:FahimiOQ18}\\
\rowlabel{c:GoldwaserS18}GoldwaserS18 \href{https://doi.org/10.1613/jair.1.11268}{GoldwaserS18}~\cite{GoldwaserS18} & \href{../works/GoldwaserS18.pdf}{Optimal Torpedo Scheduling} & github, instance generator, benchmark, generated instance & 0 &  &  &  &  & \ref{a:GoldwaserS18} & \ref{b:GoldwaserS18}\\
\rowlabel{c:LaborieRSV18}LaborieRSV18 \href{https://doi.org/10.1007/s10601-018-9281-x}{LaborieRSV18}~\cite{LaborieRSV18} & \href{../works/LaborieRSV18.pdf}{{IBM} {ILOG} {CP} optimizer for scheduling - 20+ years of scheduling with constraints at {IBM/ILOG}} & CSPlib, benchmark, real-world & 3 & - &  & - & - & \ref{a:LaborieRSV18} & \ref{b:LaborieRSV18}\\
\rowlabel{c:ShinBBHO18}ShinBBHO18 \href{https://doi.org/10.1109/TSMC.2017.2681623}{ShinBBHO18}~\cite{ShinBBHO18} & \href{../works/ShinBBHO18.pdf}{Discrete-Event Simulation and Integer Linear Programming for Constraint-Aware Resource Scheduling} & github, real-world & 4 &  &  &  &  & \ref{a:ShinBBHO18} & \ref{b:ShinBBHO18}\\
\rowlabel{c:KreterSS17}KreterSS17 \href{https://doi.org/10.1007/s10601-016-9266-6}{KreterSS17}~\cite{KreterSS17} & \href{../works/KreterSS17.pdf}{Using constraint programming for solving RCPSP/max-cal} & benchmark & 5 & dead &  &  & \cite{KreterSS15} & \ref{a:KreterSS17} & \ref{b:KreterSS17}\\
\rowlabel{c:NattafAL17}NattafAL17 \href{https://doi.org/10.1007/s10601-017-9271-4}{NattafAL17}~\cite{NattafAL17} & \href{../works/NattafAL17.pdf}{Cumulative scheduling with variable task profiles and concave piecewise linear processing rate functions} & real-world & 2 & n &  & n & - & \ref{a:NattafAL17} & \ref{b:NattafAL17}\\
\rowlabel{c:SchnellH17}SchnellH17 \href{http://dx.doi.org/10.1016/j.orp.2017.01.002}{SchnellH17}~\cite{SchnellH17} & \href{../works/SchnellH17.pdf}{On the generalization of constraint programming and boolean satisfiability solving techniques to schedule a resource-constrained project consisting of multi-mode jobs} & benchmark, supplementary material & 6 &  &  &  &  & \ref{a:SchnellH17} & \ref{b:SchnellH17}\\
\rowlabel{c:Kameugne15}Kameugne15 \href{https://doi.org/10.1007/s10601-015-9227-5}{Kameugne15}~\cite{Kameugne15} & \href{../works/Kameugne15.pdf}{Propagation techniques of resource constraint for cumulative scheduling} &  & 2 & - &  & - & \href{https://www.a4cp.org/sites/default/files/roger_kameugne_-_propagation_techniques_of_resource_constraint_for_cumulative_scheduling.pdf}{PhDThesis} & \ref{a:Kameugne15} & \ref{b:Kameugne15}\\
\rowlabel{c:LetortCB15}LetortCB15 \href{https://doi.org/10.1007/s10601-014-9172-8}{LetortCB15}~\cite{LetortCB15} & \href{../works/LetortCB15.pdf}{Synchronized sweep algorithms for scalable scheduling constraints} & generated instance, benchmark, random instance, Roadef & 4 & dead &  & - & \cite{LetortCB13} & \ref{a:LetortCB15} & \ref{b:LetortCB15}\\
\rowlabel{c:NattafAL15}NattafAL15 \href{https://doi.org/10.1007/s10601-015-9192-z}{NattafAL15}~\cite{NattafAL15} & \href{../works/NattafAL15.pdf}{A hybrid exact method for a scheduling problem with a continuous resource and energy constraints} & generated instance & 1 & n &  & n &  & \ref{a:NattafAL15} & \ref{b:NattafAL15}\\
\rowlabel{c:SchnellH15}SchnellH15 \href{http://dx.doi.org/10.1007/s00291-015-0419-6}{SchnellH15}~\cite{SchnellH15} & \href{../works/SchnellH15.pdf}{On the efficient modeling and solution of the multi-mode resource-constrained project scheduling problem with generalized precedence relations} & real-life, benchmark, supplementary material & 3 &  &  &  &  & \ref{a:SchnellH15} & \ref{b:SchnellH15}\\
\rowlabel{c:Siala15}Siala15 \href{https://doi.org/10.1007/s10601-015-9213-y}{Siala15}~\cite{Siala15} & \href{../works/Siala15.pdf}{Search, propagation, and learning in sequencing and scheduling problems} & github, Roadef, CSPlib, real-world, benchmark, random instance & 2 & - &  & - & \href{https://www.a4cp.org/sites/default/files/mohamed_siala_-_search_propagation_and_learning_in_sequencing_and_scheduling_problems.pdf}{PhD Thesis} & \ref{a:Siala15} & \ref{b:Siala15}\\
\rowlabel{c:SimoninAHL15}SimoninAHL15 \href{https://doi.org/10.1007/s10601-014-9169-3}{SimoninAHL15}~\cite{SimoninAHL15} & \href{../works/SimoninAHL15.pdf}{Scheduling scientific experiments for comet exploration} &  & 0 & n &  & n & \cite{SimoninAHL12} & \ref{a:SimoninAHL15} & \ref{b:SimoninAHL15}\\
\rowlabel{c:KameugneFSN14}KameugneFSN14 \href{https://doi.org/10.1007/s10601-013-9157-z}{KameugneFSN14}~\cite{KameugneFSN14} & \href{../works/KameugneFSN14.pdf}{A quadratic edge-finding filtering algorithm for cumulative resource constraints} & random instance, benchmark & 2 & \href{https://figshare.com/articles/dataset/Comparison_of_edge_finding_and_extended_edge_finding_filtering_algorithms/736454}{y} &  &  & \cite{KameugneFSN11} & \ref{a:KameugneFSN14} & \ref{b:KameugneFSN14}\\
\rowlabel{c:HeinzSB13}HeinzSB13 \href{https://doi.org/10.1007/s10601-012-9136-9}{HeinzSB13}~\cite{HeinzSB13} & \href{../works/HeinzSB13.pdf}{Using dual presolving reductions to reformulate cumulative constraints} & benchmark & 1 & ref &  & - & - & \ref{a:HeinzSB13} & \ref{b:HeinzSB13}\\
\rowlabel{c:MenciaSV13}MenciaSV13 \href{http://dx.doi.org/10.1007/s10845-012-0726-6}{MenciaSV13}~\cite{MenciaSV13} & \href{../works/MenciaSV13.pdf}{Intensified iterative deepening A* with application to job shop scheduling} & benchmark, real-life, supplementary material & 0 &  &  &  &  & \ref{a:MenciaSV13} & \ref{b:MenciaSV13}\\
\rowlabel{c:OzturkTHO13}OzturkTHO13 \href{https://doi.org/10.1007/s10601-013-9142-6}{OzturkTHO13}~\cite{OzturkTHO13} & \href{../works/OzturkTHO13.pdf}{Balancing and scheduling of flexible mixed model assembly lines} & real-world, real-life & 2 & \href{https://github.com/ozturkcemal/SBSFMMAL}{y} &  & - & - & \ref{a:OzturkTHO13} & \ref{b:OzturkTHO13}\\
\rowlabel{c:SchuttFSW13}SchuttFSW13 \href{https://doi.org/10.1007/s10951-012-0285-x}{SchuttFSW13}~\cite{SchuttFSW13} & \href{../works/SchuttFSW13.pdf}{Solving RCPSP/max by lazy clause generation} & supplementary material, benchmark & 6 &  &  &  &  & \ref{a:SchuttFSW13} & \ref{b:SchuttFSW13}\\
\rowlabel{c:HeinzSSW12}HeinzSSW12 \href{https://doi.org/10.1007/s10601-011-9113-8}{HeinzSSW12}~\cite{HeinzSSW12} & \href{../works/HeinzSSW12.pdf}{Solving steel mill slab design problems} & CSPlib, real-world & 2 & Cplex &  & dead & - & \ref{a:HeinzSSW12} & \ref{b:HeinzSSW12}\\
\rowlabel{c:LimtanyakulS12}LimtanyakulS12 \href{https://doi.org/10.1007/s10601-012-9118-y}{LimtanyakulS12}~\cite{LimtanyakulS12} & \href{../works/LimtanyakulS12.pdf}{Improvements of constraint programming and hybrid methods for scheduling of tests on vehicle prototypes} & real-life, generated instance, industrial partner, benchmark, random instance & 1 & dead &  & - & - & \ref{a:LimtanyakulS12} & \ref{b:LimtanyakulS12}\\
\rowlabel{c:LombardiM12}LombardiM12 \href{https://doi.org/10.1007/s10601-011-9115-6}{LombardiM12}~\cite{LombardiM12} & \href{../works/LombardiM12.pdf}{Optimal methods for resource allocation and scheduling: a cross-disciplinary survey} & real-world, benchmark & 0 & - &  & - & - & \ref{a:LombardiM12} & \ref{b:LombardiM12}\\
\rowlabel{c:BartakS11}BartakS11 \href{https://doi.org/10.1007/s10601-011-9109-4}{BartakS11}~\cite{BartakS11} & \href{../works/BartakS11.pdf}{Constraint satisfaction for planning and scheduling problems} & random instance, real-world, real-life & 2 & - &  & - &  & \ref{a:BartakS11} & \ref{b:BartakS11}\\
\rowlabel{c:KovacsB11}KovacsB11 \href{https://doi.org/10.1007/s10601-009-9088-x}{KovacsB11}~\cite{KovacsB11} & \href{../works/KovacsB11.pdf}{A global constraint for total weighted completion time for unary resources} & benchmark & 2 & n &  & n & - & \ref{a:KovacsB11} & \ref{b:KovacsB11}\\
\rowlabel{c:KovacsK11}KovacsK11 \href{https://doi.org/10.1007/s10601-010-9102-3}{KovacsK11}~\cite{KovacsK11} & \href{../works/KovacsK11.pdf}{Constraint programming approach to a bilevel scheduling problem} &  & 2 & n &  & n & - & \ref{a:KovacsK11} & \ref{b:KovacsK11}\\
\rowlabel{c:SchausHMCMD11}SchausHMCMD11 \href{https://doi.org/10.1007/s10601-010-9100-5}{SchausHMCMD11}~\cite{SchausHMCMD11} & \href{../works/SchausHMCMD11.pdf}{Solving Steel Mill Slab Problems with constraint-based techniques: CP, LNS, and {CBLS}} & CSPlib, generated instance, benchmark & 3 & dead &  &  &  & \ref{a:SchausHMCMD11} & \ref{b:SchausHMCMD11}\\
\rowlabel{c:SchuttFSW11}SchuttFSW11 \href{https://doi.org/10.1007/s10601-010-9103-2}{SchuttFSW11}~\cite{SchuttFSW11} & \href{../works/SchuttFSW11.pdf}{Explaining the cumulative propagator} & benchmark, real-world & 7 & PSPLib &  & - & - & \ref{a:SchuttFSW11} & \ref{b:SchuttFSW11}\\
\rowlabel{c:LopesCSM10}LopesCSM10 \href{https://doi.org/10.1007/s10601-009-9086-z}{LopesCSM10}~\cite{LopesCSM10} & \href{../works/LopesCSM10.pdf}{A hybrid model for a multiproduct pipeline planning and scheduling problem} & real-world, benchmark & 2 & - &  & - & \cite{MouraSCL08,MouraSCL08a} & \ref{a:LopesCSM10} & \ref{b:LopesCSM10}\\
\rowlabel{c:Simonis07}Simonis07 \href{https://doi.org/10.1007/s10601-006-9011-7}{Simonis07}~\cite{Simonis07} & \href{../works/Simonis07.pdf}{Models for Global Constraint Applications} &  & 0 & n &  & n &  & \ref{a:Simonis07} & \ref{b:Simonis07}\\
\rowlabel{c:Hooker06}Hooker06 \href{https://doi.org/10.1007/s10601-006-8060-2}{Hooker06}~\cite{Hooker06} & \href{../works/Hooker06.pdf}{An Integrated Method for Planning and Scheduling to Minimize Tardiness} & random instance & 2 & n &  & n & \cite{Hooker05a} & \ref{a:Hooker06} & \ref{b:Hooker06}\\
\rowlabel{c:Hooker05}Hooker05 \href{https://doi.org/10.1007/s10601-005-2812-2}{Hooker05}~\cite{Hooker05} & \href{../works/Hooker05.pdf}{A Hybrid Method for the Planning and Scheduling} & random instance & 0 & n &  & n & \cite{Hooker04} & \ref{a:Hooker05} & \ref{b:Hooker05}\\
\rowlabel{c:VilimBC05}VilimBC05 \href{https://doi.org/10.1007/s10601-005-2814-0}{VilimBC05}~\cite{VilimBC05} & \href{../works/VilimBC05.pdf}{Extension of \emph{O}(\emph{n} log \emph{n}) Filtering Algorithms for the Unary Resource Constraint to Optional Activities} & benchmark, real-life & 0 & n &  & n & \cite{VilimBC04} & \ref{a:VilimBC05} & \ref{b:VilimBC05}\\
\rowlabel{c:BaptisteP00}BaptisteP00 \href{https://doi.org/10.1023/A:1009822502231}{BaptisteP00}~\cite{BaptisteP00} & \href{../works/BaptisteP00.pdf}{Constraint Propagation and Decomposition Techniques for Highly Disjunctive and Highly Cumulative Project Scheduling Problems} & benchmark & 0 & n &  & n &  & \ref{a:BaptisteP00} & \ref{b:BaptisteP00}\\
\rowlabel{c:HeipckeCCS00}HeipckeCCS00 \href{https://doi.org/10.1023/A:1009860311452}{HeipckeCCS00}~\cite{HeipckeCCS00} & \href{../works/HeipckeCCS00.pdf}{Scheduling under Labour Resource Constraints} & instance generator, benchmark & 0 & dead &  & n & - & \ref{a:HeipckeCCS00} & \ref{b:HeipckeCCS00}\\
\rowlabel{c:SakkoutW00}SakkoutW00 \href{https://doi.org/10.1023/A:1009856210543}{SakkoutW00}~\cite{SakkoutW00} & \href{../works/SakkoutW00.pdf}{Probe Backtrack Search for Minimal Perturbation in Dynamic Scheduling} & benchmark, real-world & 0 & n &  & n & - & \ref{a:SakkoutW00} & \ref{b:SakkoutW00}\\
\rowlabel{c:SchildW00}SchildW00 \href{https://doi.org/10.1023/A:1009804226473}{SchildW00}~\cite{SchildW00} & \href{../works/SchildW00.pdf}{Scheduling of Time-Triggered Real-Time Systems} &  & 0 & n &  & n & - & \ref{a:SchildW00} & \ref{b:SchildW00}\\
\rowlabel{c:BensanaLV99}BensanaLV99 \href{https://doi.org/10.1023/A:1026488509554}{BensanaLV99}~\cite{BensanaLV99} & \href{../works/BensanaLV99.pdf}{Earth Observation Satellite Management} & benchmark & 0 & ? &  & - & - & \ref{a:BensanaLV99} & \ref{b:BensanaLV99}\\
\rowlabel{c:BelhadjiI98}BelhadjiI98 \href{https://doi.org/10.1023/A:1009777711218}{BelhadjiI98}~\cite{BelhadjiI98} & \href{../works/BelhadjiI98.pdf}{Temporal Constraint Satisfaction Techniques in Job Shop Scheduling Problem Solving} & real-life & 0 & n &  & n & - & \ref{a:BelhadjiI98} & \ref{b:BelhadjiI98}\\
\rowlabel{c:PapaB98}PapaB98 \href{https://doi.org/10.1023/A:1009723704757}{PapaB98}~\cite{PapaB98} & \href{../works/PapaB98.pdf}{Resource Constraints for Preemptive Job-shop Scheduling} & benchmark & 0 & dead &  & - & - & \ref{a:PapaB98} & \ref{b:PapaB98}\\
\rowlabel{c:Darby-DowmanLMZ97}Darby-DowmanLMZ97 \href{https://doi.org/10.1007/BF00137871}{Darby-DowmanLMZ97}~\cite{Darby-DowmanLMZ97} & \href{../works/Darby-DowmanLMZ97.pdf}{Constraint Logic Programming and Integer Programming Approaches and Their Collaboration in Solving an Assignment Scheduling Problem} & benchmark, real-life, real-world & 0 & n &  & n & - & \ref{a:Darby-DowmanLMZ97} & \ref{b:Darby-DowmanLMZ97}\\
\rowlabel{c:Zhou97}Zhou97 \href{https://doi.org/10.1023/A:1009757726572}{Zhou97}~\cite{Zhou97} & \href{../works/Zhou97.pdf}{A Permutation-Based Approach for Solving the Job-Shop Problem} & benchmark & 0 & n &  & n & \cite{Zhou96} & \ref{a:Zhou97} & \ref{b:Zhou97}\\
\rowlabel{c:Wallace96}Wallace96 \href{https://doi.org/10.1007/BF00143881}{Wallace96}~\cite{Wallace96} & \href{../works/Wallace96.pdf}{Practical Applications of Constraint Programming} &  & 0 & - &  & - & - & \ref{a:Wallace96} & \ref{b:Wallace96}\\
\end{longtable}
}



\clearpage
\section{Authors}

{\scriptsize
\begin{longtable}{p{4cm}p{20cm}}
\rowcolor{white}\caption{Co-Authors of Articles/Papers}\\ \toprule
\rowcolor{white}Author & Entries \\ \midrule\endhead
\bottomrule
\endfoot
Michela Milano & \href{works/BorghesiBLMB18.pdf}{BorghesiBLMB18}~\cite{BorghesiBLMB18}, \href{works/BonfiettiZLM16.pdf}{BonfiettiZLM16}~\cite{BonfiettiZLM16}, \href{works/BridiBLMB16.pdf}{BridiBLMB16}~\cite{BridiBLMB16}, \href{works/BridiLBBM16.pdf}{BridiLBBM16}~\cite{BridiLBBM16}, \href{works/LombardiBM15.pdf}{LombardiBM15}~\cite{LombardiBM15}, \href{works/BartoliniBBLM14.pdf}{BartoliniBBLM14}~\cite{BartoliniBBLM14}, \href{works/BonfiettiLM14.pdf}{BonfiettiLM14}~\cite{BonfiettiLM14}, \href{works/BonfiettiLBM14.pdf}{BonfiettiLBM14}~\cite{BonfiettiLBM14}, \href{works/BonfiettiLM13.pdf}{BonfiettiLM13}~\cite{BonfiettiLM13}, \href{works/LombardiM13.pdf}{LombardiM13}~\cite{LombardiM13}, \href{works/LombardiM12.pdf}{LombardiM12}~\cite{LombardiM12}, \href{works/BonfiettiLBM12.pdf}{BonfiettiLBM12}~\cite{BonfiettiLBM12}, \href{works/LombardiM12a.pdf}{LombardiM12a}~\cite{LombardiM12a}, \href{works/BonfiettiM12.pdf}{BonfiettiM12}~\cite{BonfiettiM12}, \href{works/BonfiettiLBM11.pdf}{BonfiettiLBM11}~\cite{BonfiettiLBM11}, \href{works/LombardiBMB11.pdf}{LombardiBMB11}~\cite{LombardiBMB11}, \href{works/BeniniLMR11.pdf}{BeniniLMR11}~\cite{BeniniLMR11}, \href{works/LombardiM10.pdf}{LombardiM10}~\cite{LombardiM10}, \href{works/LombardiM10a.pdf}{LombardiM10a}~\cite{LombardiM10a}, \href{works/LombardiM09.pdf}{LombardiM09}~\cite{LombardiM09}, \href{works/RuggieroBBMA09.pdf}{RuggieroBBMA09}~\cite{RuggieroBBMA09}, \href{works/BeniniBGM06.pdf}{BeniniBGM06}~\cite{BeniniBGM06}, \href{works/LammaMM97.pdf}{LammaMM97}~\cite{LammaMM97}, \href{works/BrusoniCLMMT96.pdf}{BrusoniCLMMT96}~\cite{BrusoniCLMMT96}\\
Michele Lombardi & \href{works/BorghesiBLMB18.pdf}{BorghesiBLMB18}~\cite{BorghesiBLMB18}, \href{works/BonfiettiZLM16.pdf}{BonfiettiZLM16}~\cite{BonfiettiZLM16}, \href{works/BridiBLMB16.pdf}{BridiBLMB16}~\cite{BridiBLMB16}, \href{works/BridiLBBM16.pdf}{BridiLBBM16}~\cite{BridiLBBM16}, \href{works/LombardiBM15.pdf}{LombardiBM15}~\cite{LombardiBM15}, \href{works/BartoliniBBLM14.pdf}{BartoliniBBLM14}~\cite{BartoliniBBLM14}, \href{works/BonfiettiLM14.pdf}{BonfiettiLM14}~\cite{BonfiettiLM14}, \href{works/BonfiettiLBM14.pdf}{BonfiettiLBM14}~\cite{BonfiettiLBM14}, \href{works/BonfiettiLM13.pdf}{BonfiettiLM13}~\cite{BonfiettiLM13}, \href{works/LombardiM13.pdf}{LombardiM13}~\cite{LombardiM13}, \href{works/LombardiM12.pdf}{LombardiM12}~\cite{LombardiM12}, \href{works/BonfiettiLBM12.pdf}{BonfiettiLBM12}~\cite{BonfiettiLBM12}, \href{works/LombardiM12a.pdf}{LombardiM12a}~\cite{LombardiM12a}, \href{works/BonfiettiLBM11.pdf}{BonfiettiLBM11}~\cite{BonfiettiLBM11}, \href{works/LombardiBMB11.pdf}{LombardiBMB11}~\cite{LombardiBMB11}, \href{works/BeniniLMR11.pdf}{BeniniLMR11}~\cite{BeniniLMR11}, \href{works/LombardiM10.pdf}{LombardiM10}~\cite{LombardiM10}, \href{works/LombardiM10a.pdf}{LombardiM10a}~\cite{LombardiM10a}, \href{works/LombardiM09.pdf}{LombardiM09}~\cite{LombardiM09}, \href{works/HoeveGSL07.pdf}{HoeveGSL07}~\cite{HoeveGSL07}\\
Andreas Schutt & \href{works/YangSS19.pdf}{YangSS19}~\cite{YangSS19}, \href{works/KreterSS17.pdf}{KreterSS17}~\cite{KreterSS17}, \href{works/YoungFS17.pdf}{YoungFS17}~\cite{YoungFS17}, \href{works/GoldwaserS17.pdf}{GoldwaserS17}~\cite{GoldwaserS17}, \href{works/SchuttS16.pdf}{SchuttS16}~\cite{SchuttS16}, \href{works/SzerediS16.pdf}{SzerediS16}~\cite{SzerediS16}, \href{works/KreterSS15.pdf}{KreterSS15}~\cite{KreterSS15}, \href{works/EvenSH15.pdf}{EvenSH15}~\cite{EvenSH15}, \href{works/EvenSH15a.pdf}{EvenSH15a}~\cite{EvenSH15a}, \href{works/SchuttFS13.pdf}{SchuttFS13}~\cite{SchuttFS13}, \href{works/SchuttFS13a.pdf}{SchuttFS13a}~\cite{SchuttFS13a}, \href{works/GuSS13.pdf}{GuSS13}~\cite{GuSS13}, \href{works/SchuttCSW12.pdf}{SchuttCSW12}~\cite{SchuttCSW12}, \href{works/SchuttFSW11.pdf}{SchuttFSW11}~\cite{SchuttFSW11}, \href{works/SchuttW10.pdf}{SchuttW10}~\cite{SchuttW10}, \href{works/SchuttFSW09.pdf}{SchuttFSW09}~\cite{SchuttFSW09}\\
J. Christopher Beck & \href{works/LuoB22.pdf}{LuoB22}~\cite{LuoB22}, \href{works/TangB20.pdf}{TangB20}~\cite{TangB20}, \href{works/BoothNB16.pdf}{BoothNB16}~\cite{BoothNB16}, \href{works/KoschB14.pdf}{KoschB14}~\cite{KoschB14}, \href{works/HeinzSB13.pdf}{HeinzSB13}~\cite{HeinzSB13}, \href{works/HeinzKB13.pdf}{HeinzKB13}~\cite{HeinzKB13}, \href{works/HeinzB12.pdf}{HeinzB12}~\cite{HeinzB12}, \href{works/KovacsB11.pdf}{KovacsB11}~\cite{KovacsB11}, \href{works/BeckFW11.pdf}{BeckFW11}~\cite{BeckFW11}, \href{works/WatsonB08.pdf}{WatsonB08}~\cite{WatsonB08}, \href{works/KovacsB08.pdf}{KovacsB08}~\cite{KovacsB08}, \href{works/CarchraeBF05.pdf}{CarchraeBF05}~\cite{CarchraeBF05}, \href{works/WuBB05.pdf}{WuBB05}~\cite{WuBB05}, \href{works/BeckDF97.pdf}{BeckDF97}~\cite{BeckDF97}\\
Nicolas Beldiceanu & \href{works/Madi-WambaLOBM17.pdf}{Madi-WambaLOBM17}~\cite{Madi-WambaLOBM17}, \href{works/Madi-WambaB16.pdf}{Madi-WambaB16}~\cite{Madi-WambaB16}, \href{works/LetortCB15.pdf}{LetortCB15}~\cite{LetortCB15}, \href{works/LetortCB13.pdf}{LetortCB13}~\cite{LetortCB13}, \href{works/LetortBC12.pdf}{LetortBC12}~\cite{LetortBC12}, \href{works/ClercqPBJ11.pdf}{ClercqPBJ11}~\cite{ClercqPBJ11}, \href{works/BeldiceanuCDP11.pdf}{BeldiceanuCDP11}~\cite{BeldiceanuCDP11}, \href{works/BeldiceanuCP08.pdf}{BeldiceanuCP08}~\cite{BeldiceanuCP08}, \href{works/PoderB08.pdf}{PoderB08}~\cite{PoderB08}, \href{works/BeldiceanuP07.pdf}{BeldiceanuP07}~\cite{BeldiceanuP07}, \href{works/PoderBS04.pdf}{PoderBS04}~\cite{PoderBS04}, \href{works/BeldiceanuC02.pdf}{BeldiceanuC02}~\cite{BeldiceanuC02}, \href{works/AggounB93.pdf}{AggounB93}~\cite{AggounB93}\\
Emmanuel Hebrard & \href{works/JuvinHHL23.pdf}{JuvinHHL23}~\cite{JuvinHHL23}, \href{works/AntuoriHHEN21.pdf}{AntuoriHHEN21}~\cite{AntuoriHHEN21}, \href{works/GodetLHS20.pdf}{GodetLHS20}~\cite{GodetLHS20}, \href{works/SimoninAHL15.pdf}{SimoninAHL15}~\cite{SimoninAHL15}, \href{works/SialaAH15.pdf}{SialaAH15}~\cite{SialaAH15}, \href{works/BessiereHMQW14.pdf}{BessiereHMQW14}~\cite{BessiereHMQW14}, \href{works/SimoninAHL12.pdf}{SimoninAHL12}~\cite{SimoninAHL12}, \href{works/BillautHL12.pdf}{BillautHL12}~\cite{BillautHL12}, \href{works/GrimesH11.pdf}{GrimesH11}~\cite{GrimesH11}, \href{works/GrimesH10.pdf}{GrimesH10}~\cite{GrimesH10}, \href{works/GrimesHM09.pdf}{GrimesHM09}~\cite{GrimesHM09}, \href{works/HebrardTW05.pdf}{HebrardTW05}~\cite{HebrardTW05}\\
Peter J. Stuckey & \href{works/YangSS19.pdf}{YangSS19}~\cite{YangSS19}, \href{works/DemirovicS18.pdf}{DemirovicS18}~\cite{DemirovicS18}, \href{works/KreterSS17.pdf}{KreterSS17}~\cite{KreterSS17}, \href{works/SchuttS16.pdf}{SchuttS16}~\cite{SchuttS16}, \href{works/KreterSS15.pdf}{KreterSS15}~\cite{KreterSS15}, \href{works/BurtLPS15.pdf}{BurtLPS15}~\cite{BurtLPS15}, \href{works/SchuttFS13.pdf}{SchuttFS13}~\cite{SchuttFS13}, \href{works/SchuttFS13a.pdf}{SchuttFS13a}~\cite{SchuttFS13a}, \href{works/GuSS13.pdf}{GuSS13}~\cite{GuSS13}, \href{works/SchuttCSW12.pdf}{SchuttCSW12}~\cite{SchuttCSW12}, \href{works/SchuttFSW11.pdf}{SchuttFSW11}~\cite{SchuttFSW11}, \href{works/SchuttFSW09.pdf}{SchuttFSW09}~\cite{SchuttFSW09}\\
Roman Bart{\'{a}}k & \href{works/SvancaraB22.pdf}{SvancaraB22}~\cite{SvancaraB22}, \href{works/JelinekB16.pdf}{JelinekB16}~\cite{JelinekB16}, \href{works/BartakV15.pdf}{BartakV15}~\cite{BartakV15}, \href{}{Bartak14}~\cite{Bartak14}, \href{works/BartakS11.pdf}{BartakS11}~\cite{BartakS11}, \href{works/BartakCS10.pdf}{BartakCS10}~\cite{BartakCS10}, \href{works/BartakSR10.pdf}{BartakSR10}~\cite{BartakSR10}, \href{works/VilimBC05.pdf}{VilimBC05}~\cite{VilimBC05}, \href{works/VilimBC04.pdf}{VilimBC04}~\cite{VilimBC04}, \href{works/Bartak02.pdf}{Bartak02}~\cite{Bartak02}, \href{works/Bartak02a.pdf}{Bartak02a}~\cite{Bartak02a}\\
Pierre Lopez & \href{works/JuvinHHL23.pdf}{JuvinHHL23}~\cite{JuvinHHL23}, \href{works/JuvinHL23.pdf}{JuvinHL23}~\cite{JuvinHL23}, \href{works/Polo-MejiaALB20.pdf}{Polo-MejiaALB20}~\cite{Polo-MejiaALB20}, \href{works/NattafAL17.pdf}{NattafAL17}~\cite{NattafAL17}, \href{works/SimoninAHL15.pdf}{SimoninAHL15}~\cite{SimoninAHL15}, \href{works/NattafAL15.pdf}{NattafAL15}~\cite{NattafAL15}, \href{works/SimoninAHL12.pdf}{SimoninAHL12}~\cite{SimoninAHL12}, \href{works/BillautHL12.pdf}{BillautHL12}~\cite{BillautHL12}, \href{works/LahimerLH11.pdf}{LahimerLH11}~\cite{LahimerLH11}, \href{works/TrojetHL11.pdf}{TrojetHL11}~\cite{TrojetHL11}, \href{works/LopezAKYG00.pdf}{LopezAKYG00}~\cite{LopezAKYG00}\\
Petr Vil{\'{\i}}m & \href{works/LaborieRSV18.pdf}{LaborieRSV18}~\cite{LaborieRSV18}, \href{works/VilimLS15.pdf}{VilimLS15}~\cite{VilimLS15}, \href{works/Vilim11.pdf}{Vilim11}~\cite{Vilim11}, \href{works/Vilim09.pdf}{Vilim09}~\cite{Vilim09}, \href{works/Vilim09a.pdf}{Vilim09a}~\cite{Vilim09a}, \href{works/VilimBC05.pdf}{VilimBC05}~\cite{VilimBC05}, \href{works/Vilim05.pdf}{Vilim05}~\cite{Vilim05}, \href{works/VilimBC04.pdf}{VilimBC04}~\cite{VilimBC04}, \href{works/Vilim04.pdf}{Vilim04}~\cite{Vilim04}, \href{works/Vilim03.pdf}{Vilim03}~\cite{Vilim03}, \href{works/Vilim02.pdf}{Vilim02}~\cite{Vilim02}\\
Christian Artigues & \href{works/PovedaAA23.pdf}{PovedaAA23}~\cite{PovedaAA23}, \href{works/PohlAK22.pdf}{PohlAK22}~\cite{PohlAK22}, \href{works/Polo-MejiaALB20.pdf}{Polo-MejiaALB20}~\cite{Polo-MejiaALB20}, \href{works/NattafAL17.pdf}{NattafAL17}~\cite{NattafAL17}, \href{works/SimoninAHL15.pdf}{SimoninAHL15}~\cite{SimoninAHL15}, \href{works/NattafAL15.pdf}{NattafAL15}~\cite{NattafAL15}, \href{works/SialaAH15.pdf}{SialaAH15}~\cite{SialaAH15}, \href{works/SimoninAHL12.pdf}{SimoninAHL12}~\cite{SimoninAHL12}, \href{works/ArtiguesBF04.pdf}{ArtiguesBF04}~\cite{ArtiguesBF04}, \href{works/ArtiguesR00.pdf}{ArtiguesR00}~\cite{ArtiguesR00}\\
Luca Benini & \href{works/BorghesiBLMB18.pdf}{BorghesiBLMB18}~\cite{BorghesiBLMB18}, \href{works/BridiBLMB16.pdf}{BridiBLMB16}~\cite{BridiBLMB16}, \href{works/BridiLBBM16.pdf}{BridiLBBM16}~\cite{BridiLBBM16}, \href{works/BonfiettiLBM14.pdf}{BonfiettiLBM14}~\cite{BonfiettiLBM14}, \href{works/BonfiettiLBM12.pdf}{BonfiettiLBM12}~\cite{BonfiettiLBM12}, \href{works/BonfiettiLBM11.pdf}{BonfiettiLBM11}~\cite{BonfiettiLBM11}, \href{works/LombardiBMB11.pdf}{LombardiBMB11}~\cite{LombardiBMB11}, \href{works/BeniniLMR11.pdf}{BeniniLMR11}~\cite{BeniniLMR11}, \href{works/RuggieroBBMA09.pdf}{RuggieroBBMA09}~\cite{RuggieroBBMA09}, \href{works/BeniniBGM06.pdf}{BeniniBGM06}~\cite{BeniniBGM06}\\
Alessio Bonfietti & \href{works/BonfiettiZLM16.pdf}{BonfiettiZLM16}~\cite{BonfiettiZLM16}, \href{works/Bonfietti16.pdf}{Bonfietti16}~\cite{Bonfietti16}, \href{works/LombardiBM15.pdf}{LombardiBM15}~\cite{LombardiBM15}, \href{works/BonfiettiLM14.pdf}{BonfiettiLM14}~\cite{BonfiettiLM14}, \href{works/BonfiettiLBM14.pdf}{BonfiettiLBM14}~\cite{BonfiettiLBM14}, \href{works/BonfiettiLM13.pdf}{BonfiettiLM13}~\cite{BonfiettiLM13}, \href{works/BonfiettiLBM12.pdf}{BonfiettiLBM12}~\cite{BonfiettiLBM12}, \href{works/BonfiettiM12.pdf}{BonfiettiM12}~\cite{BonfiettiM12}, \href{works/BonfiettiLBM11.pdf}{BonfiettiLBM11}~\cite{BonfiettiLBM11}, \href{works/LombardiBMB11.pdf}{LombardiBMB11}~\cite{LombardiBMB11}\\
Philippe Laborie & \href{works/LunardiBLRV20.pdf}{LunardiBLRV20}~\cite{LunardiBLRV20}, \href{works/LaborieRSV18.pdf}{LaborieRSV18}~\cite{LaborieRSV18}, \href{works/Laborie18a.pdf}{Laborie18a}~\cite{Laborie18a}, \href{works/MelgarejoLS15.pdf}{MelgarejoLS15}~\cite{MelgarejoLS15}, \href{works/VilimLS15.pdf}{VilimLS15}~\cite{VilimLS15}, \href{works/Laborie09.pdf}{Laborie09}~\cite{Laborie09}, \href{}{BaptisteLPN06}~\cite{BaptisteLPN06}, \href{works/GodardLN05.pdf}{GodardLN05}~\cite{GodardLN05}, \href{works/FocacciLN00.pdf}{FocacciLN00}~\cite{FocacciLN00}\\
John N. Hooker & \href{works/Hooker17.pdf}{Hooker17}~\cite{Hooker17}, \href{works/HechingH16.pdf}{HechingH16}~\cite{HechingH16}, \href{works/CireCH13.pdf}{CireCH13}~\cite{CireCH13}, \href{works/CobanH10.pdf}{CobanH10}~\cite{CobanH10}, \href{works/Hooker06.pdf}{Hooker06}~\cite{Hooker06}, \href{works/Hooker05.pdf}{Hooker05}~\cite{Hooker05}, \href{works/Hooker05a.pdf}{Hooker05a}~\cite{Hooker05a}, \href{works/Hooker04.pdf}{Hooker04}~\cite{Hooker04}, \href{works/HookerY02.pdf}{HookerY02}~\cite{HookerY02}\\
Claude{-}Guy Quimper & \href{works/BoudreaultSLQ22.pdf}{BoudreaultSLQ22}~\cite{BoudreaultSLQ22}, \href{works/OuelletQ22.pdf}{OuelletQ22}~\cite{OuelletQ22}, \href{works/Mercier-AubinGQ20.pdf}{Mercier-AubinGQ20}~\cite{Mercier-AubinGQ20}, \href{works/FahimiOQ18.pdf}{FahimiOQ18}~\cite{FahimiOQ18}, \href{works/KameugneFGOQ18.pdf}{KameugneFGOQ18}~\cite{KameugneFGOQ18}, \href{works/OuelletQ18.pdf}{OuelletQ18}~\cite{OuelletQ18}, \href{works/GingrasQ16.pdf}{GingrasQ16}~\cite{GingrasQ16}, \href{works/BessiereHMQW14.pdf}{BessiereHMQW14}~\cite{BessiereHMQW14}, \href{works/OuelletQ13.pdf}{OuelletQ13}~\cite{OuelletQ13}\\
Pierre Schaus & \href{works/CappartS17.pdf}{CappartS17}~\cite{CappartS17}, \href{works/CauwelaertDMS16.pdf}{CauwelaertDMS16}~\cite{CauwelaertDMS16}, \href{works/DejemeppeCS15.pdf}{DejemeppeCS15}~\cite{DejemeppeCS15}, \href{works/GayHLS15.pdf}{GayHLS15}~\cite{GayHLS15}, \href{works/GayHS15.pdf}{GayHS15}~\cite{GayHS15}, \href{works/GayHS15a.pdf}{GayHS15a}~\cite{GayHS15a}, \href{works/HoundjiSWD14.pdf}{HoundjiSWD14}~\cite{HoundjiSWD14}, \href{works/GaySS14.pdf}{GaySS14}~\cite{GaySS14}, \href{works/SchausHMCMD11.pdf}{SchausHMCMD11}~\cite{SchausHMCMD11}\\
Pascal Van Hentenryck & \href{works/FontaineMH16.pdf}{FontaineMH16}~\cite{FontaineMH16}, \href{works/EvenSH15.pdf}{EvenSH15}~\cite{EvenSH15}, \href{works/EvenSH15a.pdf}{EvenSH15a}~\cite{EvenSH15a}, \href{works/SchausHMCMD11.pdf}{SchausHMCMD11}~\cite{SchausHMCMD11}, \href{works/MonetteDH09.pdf}{MonetteDH09}~\cite{MonetteDH09}, \href{works/DoomsH08.pdf}{DoomsH08}~\cite{DoomsH08}, \href{works/HentenryckM08.pdf}{HentenryckM08}~\cite{HentenryckM08}, \href{works/HentenryckM04.pdf}{HentenryckM04}~\cite{HentenryckM04}, \href{works/DincbasSH90.pdf}{DincbasSH90}~\cite{DincbasSH90}\\
Philippe Baptiste & \href{works/BaptisteB18.pdf}{BaptisteB18}~\cite{BaptisteB18}, \href{works/Baptiste09.pdf}{Baptiste09}~\cite{Baptiste09}, \href{}{BaptisteLPN06}~\cite{BaptisteLPN06}, \href{works/ArtiouchineB05.pdf}{ArtiouchineB05}~\cite{ArtiouchineB05}, \href{works/BaptisteP00.pdf}{BaptisteP00}~\cite{BaptisteP00}, \href{works/PapaB98.pdf}{PapaB98}~\cite{PapaB98}, \href{works/BaptisteP97.pdf}{BaptisteP97}~\cite{BaptisteP97}, \href{}{PapeB97}~\cite{PapeB97}\\
Mats Carlsson & \href{works/WessenCS20.pdf}{WessenCS20}~\cite{WessenCS20}, \href{works/MossigeGSMC17.pdf}{MossigeGSMC17}~\cite{MossigeGSMC17}, \href{works/LetortCB15.pdf}{LetortCB15}~\cite{LetortCB15}, \href{works/LetortCB13.pdf}{LetortCB13}~\cite{LetortCB13}, \href{works/LetortBC12.pdf}{LetortBC12}~\cite{LetortBC12}, \href{works/BeldiceanuCDP11.pdf}{BeldiceanuCDP11}~\cite{BeldiceanuCDP11}, \href{works/BeldiceanuCP08.pdf}{BeldiceanuCP08}~\cite{BeldiceanuCP08}, \href{works/BeldiceanuC02.pdf}{BeldiceanuC02}~\cite{BeldiceanuC02}\\
Nysret Musliu & \href{works/LacknerMMWW23.pdf}{LacknerMMWW23}~\cite{LacknerMMWW23}, \href{works/WinterMMW22.pdf}{WinterMMW22}~\cite{WinterMMW22}, \href{works/LacknerMMWW21.pdf}{LacknerMMWW21}~\cite{LacknerMMWW21}, \href{works/GeibingerKKMMW21.pdf}{GeibingerKKMMW21}~\cite{GeibingerKKMMW21}, \href{works/GeibingerMM21.pdf}{GeibingerMM21}~\cite{GeibingerMM21}, \href{works/GeibingerMM19.pdf}{GeibingerMM19}~\cite{GeibingerMM19}, \href{works/abs-1911-04766.pdf}{abs-1911-04766}~\cite{abs-1911-04766}, \href{works/KletzanderM17.pdf}{KletzanderM17}~\cite{KletzanderM17}\\
Helmut Simonis & \href{works/ArmstrongGOS22.pdf}{ArmstrongGOS22}~\cite{ArmstrongGOS22}, \href{works/ArmstrongGOS21.pdf}{ArmstrongGOS21}~\cite{ArmstrongGOS21}, \href{works/GrimesIOS14.pdf}{GrimesIOS14}~\cite{GrimesIOS14}, \href{works/IfrimOS12.pdf}{IfrimOS12}~\cite{IfrimOS12}, \href{works/Simonis07.pdf}{Simonis07}~\cite{Simonis07}, \href{works/SimonisC95.pdf}{SimonisC95}~\cite{SimonisC95}, \href{works/Simonis95.pdf}{Simonis95}~\cite{Simonis95}, \href{works/DincbasSH90.pdf}{DincbasSH90}~\cite{DincbasSH90}\\
Zdenek Hanz{\'{a}}lek & \href{works/Mehdizadeh-Somarin23.pdf}{Mehdizadeh-Somarin23}~\cite{Mehdizadeh-Somarin23}, \href{works/abs-2305-19888.pdf}{abs-2305-19888}~\cite{abs-2305-19888}, \href{works/HeinzNVH22.pdf}{HeinzNVH22}~\cite{HeinzNVH22}, \href{works/VlkHT21.pdf}{VlkHT21}~\cite{VlkHT21}, \href{works/BenediktMH20.pdf}{BenediktMH20}~\cite{BenediktMH20}, \href{works/BenediktSMVH18.pdf}{BenediktSMVH18}~\cite{BenediktSMVH18}, \href{works/KelbelH11.pdf}{KelbelH11}~\cite{KelbelH11}\\
Gabriela P. Henning & \href{works/NovaraNH16.pdf}{NovaraNH16}~\cite{NovaraNH16}, \href{works/NovasH14.pdf}{NovasH14}~\cite{NovasH14}, \href{works/NovasH12.pdf}{NovasH12}~\cite{NovasH12}, \href{works/NovasH10.pdf}{NovasH10}~\cite{NovasH10}, \href{works/ZeballosQH10.pdf}{ZeballosQH10}~\cite{ZeballosQH10}, \href{works/ZeballosH05.pdf}{ZeballosH05}~\cite{ZeballosH05}, \href{works/QuirogaZH05.pdf}{QuirogaZH05}~\cite{QuirogaZH05}\\
Mark Wallace & \href{works/WallaceY20.pdf}{WallaceY20}~\cite{WallaceY20}, \href{works/He0GLW18.pdf}{He0GLW18}~\cite{He0GLW18}, \href{works/SchuttFSW09.pdf}{SchuttFSW09}~\cite{SchuttFSW09}, \href{works/SakkoutW00.pdf}{SakkoutW00}~\cite{SakkoutW00}, \href{works/RodosekW98.pdf}{RodosekW98}~\cite{RodosekW98}, \href{works/Wallace96.pdf}{Wallace96}~\cite{Wallace96}, \href{}{Wallace94}~\cite{Wallace94}\\
Stefan Heinz & \href{works/HeinzSB13.pdf}{HeinzSB13}~\cite{HeinzSB13}, \href{works/HeinzKB13.pdf}{HeinzKB13}~\cite{HeinzKB13}, \href{works/HeinzSSW12.pdf}{HeinzSSW12}~\cite{HeinzSSW12}, \href{works/HeinzB12.pdf}{HeinzB12}~\cite{HeinzB12}, \href{works/HeinzS11.pdf}{HeinzS11}~\cite{HeinzS11}, \href{works/BertholdHLMS10.pdf}{BertholdHLMS10}~\cite{BertholdHLMS10}\\
Andr{\'{a}}s Kov{\'{a}}cs & \href{works/KovacsB11.pdf}{KovacsB11}~\cite{KovacsB11}, \href{works/KovacsK11.pdf}{KovacsK11}~\cite{KovacsK11}, \href{works/KovacsB08.pdf}{KovacsB08}~\cite{KovacsB08}, \href{works/KovacsV06.pdf}{KovacsV06}~\cite{KovacsV06}, \href{works/KovacsEKV05.pdf}{KovacsEKV05}~\cite{KovacsEKV05}, \href{works/KovacsV04.pdf}{KovacsV04}~\cite{KovacsV04}\\
Claude Le Pape & \href{}{BaptisteLPN06}~\cite{BaptisteLPN06}, \href{works/BaptisteP00.pdf}{BaptisteP00}~\cite{BaptisteP00}, \href{works/PapaB98.pdf}{PapaB98}~\cite{PapaB98}, \href{works/NuijtenP98.pdf}{NuijtenP98}~\cite{NuijtenP98}, \href{works/BaptisteP97.pdf}{BaptisteP97}~\cite{BaptisteP97}, \href{}{PapeB97}~\cite{PapeB97}\\
Emmanuel Poder & \href{works/BeldiceanuCDP11.pdf}{BeldiceanuCDP11}~\cite{BeldiceanuCDP11}, \href{works/abs-0907-0939.pdf}{abs-0907-0939}~\cite{abs-0907-0939}, \href{works/BeldiceanuCP08.pdf}{BeldiceanuCP08}~\cite{BeldiceanuCP08}, \href{works/PoderB08.pdf}{PoderB08}~\cite{PoderB08}, \href{works/BeldiceanuP07.pdf}{BeldiceanuP07}~\cite{BeldiceanuP07}, \href{works/PoderBS04.pdf}{PoderBS04}~\cite{PoderBS04}\\
Yves Deville & \href{works/HoundjiSWD14.pdf}{HoundjiSWD14}~\cite{HoundjiSWD14}, \href{works/DejemeppeD14.pdf}{DejemeppeD14}~\cite{DejemeppeD14}, \href{works/SchausHMCMD11.pdf}{SchausHMCMD11}~\cite{SchausHMCMD11}, \href{works/MonetteDH09.pdf}{MonetteDH09}~\cite{MonetteDH09}, \href{works/MonetteDD07.pdf}{MonetteDD07}~\cite{MonetteDD07}\\
Thibaut Feydy & \href{works/YoungFS17.pdf}{YoungFS17}~\cite{YoungFS17}, \href{works/SchuttFS13.pdf}{SchuttFS13}~\cite{SchuttFS13}, \href{works/SchuttFS13a.pdf}{SchuttFS13a}~\cite{SchuttFS13a}, \href{works/SchuttFSW11.pdf}{SchuttFSW11}~\cite{SchuttFSW11}, \href{works/SchuttFSW09.pdf}{SchuttFSW09}~\cite{SchuttFSW09}\\
Roger Kameugne & \href{works/KameugneFND23.pdf}{KameugneFND23}~\cite{KameugneFND23}, \href{works/KameugneFGOQ18.pdf}{KameugneFGOQ18}~\cite{KameugneFGOQ18}, \href{works/Kameugne15.pdf}{Kameugne15}~\cite{Kameugne15}, \href{works/KameugneFSN14.pdf}{KameugneFSN14}~\cite{KameugneFSN14}, \href{works/KameugneFSN11.pdf}{KameugneFSN11}~\cite{KameugneFSN11}\\
Juan M. Novas & \href{works/Novas19.pdf}{Novas19}~\cite{Novas19}, \href{works/NovaraNH16.pdf}{NovaraNH16}~\cite{NovaraNH16}, \href{works/NovasH14.pdf}{NovasH14}~\cite{NovasH14}, \href{works/NovasH12.pdf}{NovasH12}~\cite{NovasH12}, \href{works/NovasH10.pdf}{NovasH10}~\cite{NovasH10}\\
Wim Nuijten & \href{}{BaptisteLPN06}~\cite{BaptisteLPN06}, \href{works/GodardLN05.pdf}{GodardLN05}~\cite{GodardLN05}, \href{works/SourdN00.pdf}{SourdN00}~\cite{SourdN00}, \href{works/FocacciLN00.pdf}{FocacciLN00}~\cite{FocacciLN00}, \href{works/NuijtenP98.pdf}{NuijtenP98}~\cite{NuijtenP98}\\
Louis{-}Martin Rousseau & \href{works/DoulabiRP16.pdf}{DoulabiRP16}~\cite{DoulabiRP16}, \href{works/PesantRR15.pdf}{PesantRR15}~\cite{PesantRR15}, \href{works/DoulabiRP14.pdf}{DoulabiRP14}~\cite{DoulabiRP14}, \href{works/ChapadosJR11.pdf}{ChapadosJR11}~\cite{ChapadosJR11}, \href{works/HachemiGR11.pdf}{HachemiGR11}~\cite{HachemiGR11}\\
Marek Vlk & \href{works/abs-2305-19888.pdf}{abs-2305-19888}~\cite{abs-2305-19888}, \href{works/HeinzNVH22.pdf}{HeinzNVH22}~\cite{HeinzNVH22}, \href{works/VlkHT21.pdf}{VlkHT21}~\cite{VlkHT21}, \href{works/BenediktSMVH18.pdf}{BenediktSMVH18}~\cite{BenediktSMVH18}, \href{works/BartakV15.pdf}{BartakV15}~\cite{BartakV15}\\
Andr{\'{e}} A. Cir{\'{e}} & \href{works/CireCH13.pdf}{CireCH13}~\cite{CireCH13}, \href{works/LopesCSM10.pdf}{LopesCSM10}~\cite{LopesCSM10}, \href{works/MouraSCL08.pdf}{MouraSCL08}~\cite{MouraSCL08}, \href{works/MouraSCL08a.pdf}{MouraSCL08a}~\cite{MouraSCL08a}\\
Andrea Bartolini & \href{works/BorghesiBLMB18.pdf}{BorghesiBLMB18}~\cite{BorghesiBLMB18}, \href{works/BridiBLMB16.pdf}{BridiBLMB16}~\cite{BridiBLMB16}, \href{works/BridiLBBM16.pdf}{BridiLBBM16}~\cite{BridiLBBM16}, \href{works/BartoliniBBLM14.pdf}{BartoliniBBLM14}~\cite{BartoliniBBLM14}\\
Cyrille Dejemeppe & \href{works/CauwelaertDMS16.pdf}{CauwelaertDMS16}~\cite{CauwelaertDMS16}, \href{}{Dejemeppe16}~\cite{Dejemeppe16}, \href{works/DejemeppeCS15.pdf}{DejemeppeCS15}~\cite{DejemeppeCS15}, \href{works/DejemeppeD14.pdf}{DejemeppeD14}~\cite{DejemeppeD14}\\
Steven Gay & \href{works/GayHLS15.pdf}{GayHLS15}~\cite{GayHLS15}, \href{works/GayHS15.pdf}{GayHS15}~\cite{GayHS15}, \href{works/GayHS15a.pdf}{GayHS15a}~\cite{GayHS15a}, \href{works/GaySS14.pdf}{GaySS14}~\cite{GaySS14}\\
Tobias Geibinger & \href{works/GeibingerKKMMW21.pdf}{GeibingerKKMMW21}~\cite{GeibingerKKMMW21}, \href{works/GeibingerMM21.pdf}{GeibingerMM21}~\cite{GeibingerMM21}, \href{works/GeibingerMM19.pdf}{GeibingerMM19}~\cite{GeibingerMM19}, \href{works/abs-1911-04766.pdf}{abs-1911-04766}~\cite{abs-1911-04766}\\
Diarmuid Grimes & \href{works/GrimesIOS14.pdf}{GrimesIOS14}~\cite{GrimesIOS14}, \href{works/GrimesH11.pdf}{GrimesH11}~\cite{GrimesH11}, \href{works/GrimesH10.pdf}{GrimesH10}~\cite{GrimesH10}, \href{works/GrimesHM09.pdf}{GrimesHM09}~\cite{GrimesHM09}\\
Laurent Michel & \href{works/TardivoDFMP23.pdf}{TardivoDFMP23}~\cite{TardivoDFMP23}, \href{works/SchausHMCMD11.pdf}{SchausHMCMD11}~\cite{SchausHMCMD11}, \href{works/HentenryckM08.pdf}{HentenryckM08}~\cite{HentenryckM08}, \href{works/HentenryckM04.pdf}{HentenryckM04}~\cite{HentenryckM04}\\
Florian Mischek & \href{works/GeibingerKKMMW21.pdf}{GeibingerKKMMW21}~\cite{GeibingerKKMMW21}, \href{works/GeibingerMM21.pdf}{GeibingerMM21}~\cite{GeibingerMM21}, \href{works/GeibingerMM19.pdf}{GeibingerMM19}~\cite{GeibingerMM19}, \href{works/abs-1911-04766.pdf}{abs-1911-04766}~\cite{abs-1911-04766}\\
Jean{-}No{\"{e}}l Monette & \href{works/CauwelaertDMS16.pdf}{CauwelaertDMS16}~\cite{CauwelaertDMS16}, \href{works/SchausHMCMD11.pdf}{SchausHMCMD11}~\cite{SchausHMCMD11}, \href{works/MonetteDH09.pdf}{MonetteDH09}~\cite{MonetteDH09}, \href{works/MonetteDD07.pdf}{MonetteDD07}~\cite{MonetteDD07}\\
Margaux Nattaf & \href{works/NattafM20.pdf}{NattafM20}~\cite{NattafM20}, \href{works/MalapertN19.pdf}{MalapertN19}~\cite{MalapertN19}, \href{works/NattafAL17.pdf}{NattafAL17}~\cite{NattafAL17}, \href{works/NattafAL15.pdf}{NattafAL15}~\cite{NattafAL15}\\
Barry O'Sullivan & \href{works/ArmstrongGOS22.pdf}{ArmstrongGOS22}~\cite{ArmstrongGOS22}, \href{works/ArmstrongGOS21.pdf}{ArmstrongGOS21}~\cite{ArmstrongGOS21}, \href{works/GrimesIOS14.pdf}{GrimesIOS14}~\cite{GrimesIOS14}, \href{works/IfrimOS12.pdf}{IfrimOS12}~\cite{IfrimOS12}\\
Yanick Ouellet & \href{works/OuelletQ22.pdf}{OuelletQ22}~\cite{OuelletQ22}, \href{works/FahimiOQ18.pdf}{FahimiOQ18}~\cite{FahimiOQ18}, \href{works/KameugneFGOQ18.pdf}{KameugneFGOQ18}~\cite{KameugneFGOQ18}, \href{works/OuelletQ18.pdf}{OuelletQ18}~\cite{OuelletQ18}\\
Gilles Pesant & \href{works/AalianPG23.pdf}{AalianPG23}~\cite{AalianPG23}, \href{works/DoulabiRP16.pdf}{DoulabiRP16}~\cite{DoulabiRP16}, \href{works/PesantRR15.pdf}{PesantRR15}~\cite{PesantRR15}, \href{works/DoulabiRP14.pdf}{DoulabiRP14}~\cite{DoulabiRP14}\\
Thierry Petit & \href{works/DerrienP14.pdf}{DerrienP14}~\cite{DerrienP14}, \href{works/DerrienPZ14.pdf}{DerrienPZ14}~\cite{DerrienPZ14}, \href{works/ClercqPBJ11.pdf}{ClercqPBJ11}~\cite{ClercqPBJ11}, \href{works/abs-0907-0939.pdf}{abs-0907-0939}~\cite{abs-0907-0939}\\
Christine Solnon & \href{works/GroleazNS20.pdf}{GroleazNS20}~\cite{GroleazNS20}, \href{works/GroleazNS20a.pdf}{GroleazNS20a}~\cite{GroleazNS20a}, \href{works/SacramentoSP20.pdf}{SacramentoSP20}~\cite{SacramentoSP20}, \href{works/MelgarejoLS15.pdf}{MelgarejoLS15}~\cite{MelgarejoLS15}\\
J{\'{o}}zsef V{\'{a}}ncza & \href{works/KovacsV06.pdf}{KovacsV06}~\cite{KovacsV06}, \href{works/KovacsEKV05.pdf}{KovacsEKV05}~\cite{KovacsEKV05}, \href{works/KovacsV04.pdf}{KovacsV04}~\cite{KovacsV04}, \href{works/VanczaM01.pdf}{VanczaM01}~\cite{VanczaM01}\\
Felix Winter & \href{works/LacknerMMWW23.pdf}{LacknerMMWW23}~\cite{LacknerMMWW23}, \href{works/WinterMMW22.pdf}{WinterMMW22}~\cite{WinterMMW22}, \href{works/LacknerMMWW21.pdf}{LacknerMMWW21}~\cite{LacknerMMWW21}, \href{works/GeibingerKKMMW21.pdf}{GeibingerKKMMW21}~\cite{GeibingerKKMMW21}\\
Armin Wolf & \href{works/GeitzGSSW22.pdf}{GeitzGSSW22}~\cite{GeitzGSSW22}, \href{works/SchuttW10.pdf}{SchuttW10}~\cite{SchuttW10}, \href{works/WolfS05.pdf}{WolfS05}~\cite{WolfS05}, \href{works/Wolf03.pdf}{Wolf03}~\cite{Wolf03}\\
Francisco Yuraszeck & \href{works/YuraszeckMCCR23.pdf}{YuraszeckMCCR23}~\cite{YuraszeckMCCR23}, \href{works/YuraszeckMC23.pdf}{YuraszeckMC23}~\cite{YuraszeckMC23}, \href{works/YuraszeckMPV22.pdf}{YuraszeckMPV22}~\cite{YuraszeckMPV22}, \href{works/MejiaY20.pdf}{MejiaY20}~\cite{MejiaY20}\\
Max {\AA}strand & \href{works/Astrand0F21.pdf}{Astrand0F21}~\cite{Astrand0F21}, \href{}{Astrand21}~\cite{Astrand21}, \href{works/AstrandJZ20.pdf}{AstrandJZ20}~\cite{AstrandJZ20}, \href{works/AstrandJZ18.pdf}{AstrandJZ18}~\cite{AstrandJZ18}\\
Miguel A. Salido & \href{works/BartakS11.pdf}{BartakS11}~\cite{BartakS11}, \href{works/BartakSR10.pdf}{BartakSR10}~\cite{BartakSR10}, \href{works/AbrilSB05.pdf}{AbrilSB05}~\cite{AbrilSB05}\\
S{\'{e}}v{\'{e}}rine Betmbe Fetgo & \href{works/KameugneFND23.pdf}{KameugneFND23}~\cite{KameugneFND23}, \href{works/FetgoD22.pdf}{FetgoD22}~\cite{FetgoD22}, \href{works/KameugneFGOQ18.pdf}{KameugneFGOQ18}~\cite{KameugneFGOQ18}\\
Miquel Bofill & \href{works/BofillCSV17.pdf}{BofillCSV17}~\cite{BofillCSV17}, \href{works/BofillGSV15.pdf}{BofillGSV15}~\cite{BofillGSV15}, \href{works/BofillEGPSV14.pdf}{BofillEGPSV14}~\cite{BofillEGPSV14}\\
Thomas Bridi & \href{works/BridiBLMB16.pdf}{BridiBLMB16}~\cite{BridiBLMB16}, \href{works/BridiLBBM16.pdf}{BridiLBBM16}~\cite{BridiLBBM16}, \href{works/BartoliniBBLM14.pdf}{BartoliniBBLM14}~\cite{BartoliniBBLM14}\\
Cid C. de Souza & \href{works/MouraSCL08.pdf}{MouraSCL08}~\cite{MouraSCL08}, \href{works/MouraSCL08a.pdf}{MouraSCL08a}~\cite{MouraSCL08a}, \href{works/HeipckeCCS00.pdf}{HeipckeCCS00}~\cite{HeipckeCCS00}\\
Ondrej Cepek & \href{works/BartakCS10.pdf}{BartakCS10}~\cite{BartakCS10}, \href{works/VilimBC05.pdf}{VilimBC05}~\cite{VilimBC05}, \href{works/VilimBC04.pdf}{VilimBC04}~\cite{VilimBC04}\\
Erich Christian Teppan & \href{works/Teppan22.pdf}{Teppan22}~\cite{Teppan22}, \href{works/ColT22.pdf}{ColT22}~\cite{ColT22}, \href{works/ColT19.pdf}{ColT19}~\cite{ColT19}\\
Giacomo Da Col & \href{works/ColT22.pdf}{ColT22}~\cite{ColT22}, \href{works/abs-2102-08778.pdf}{abs-2102-08778}~\cite{abs-2102-08778}, \href{works/ColT19.pdf}{ColT19}~\cite{ColT19}\\
Sophie Demassey & \href{works/HermenierDL11.pdf}{HermenierDL11}~\cite{HermenierDL11}, \href{works/BeldiceanuCDP11.pdf}{BeldiceanuCDP11}~\cite{BeldiceanuCDP11}, \href{}{Demassey03}~\cite{Demassey03}\\
Alban Derrien & \href{}{Derrien15}~\cite{Derrien15}, \href{works/DerrienP14.pdf}{DerrienP14}~\cite{DerrienP14}, \href{works/DerrienPZ14.pdf}{DerrienPZ14}~\cite{DerrienPZ14}\\
Michele Garraffa & \href{works/AlfieriGPS23.pdf}{AlfieriGPS23}~\cite{AlfieriGPS23}, \href{works/ArmstrongGOS22.pdf}{ArmstrongGOS22}~\cite{ArmstrongGOS22}, \href{works/ArmstrongGOS21.pdf}{ArmstrongGOS21}~\cite{ArmstrongGOS21}\\
Martin Gebser & \href{works/TasselGS23.pdf}{TasselGS23}~\cite{TasselGS23}, \href{works/abs-2306-05747.pdf}{abs-2306-05747}~\cite{abs-2306-05747}, \href{works/KovacsTKSG21.pdf}{KovacsTKSG21}~\cite{KovacsTKSG21}\\
Jean{-}Claude Gentina & \href{works/KorbaaYG00.pdf}{KorbaaYG00}~\cite{KorbaaYG00}, \href{works/LopezAKYG00.pdf}{LopezAKYG00}~\cite{LopezAKYG00}, \href{works/KorbaaYG99.pdf}{KorbaaYG99}~\cite{KorbaaYG99}\\
Renaud Hartert & \href{works/GayHLS15.pdf}{GayHLS15}~\cite{GayHLS15}, \href{works/GayHS15.pdf}{GayHS15}~\cite{GayHS15}, \href{works/GayHS15a.pdf}{GayHS15a}~\cite{GayHS15a}\\
Brahim Hnich & \href{works/GokgurHO18.pdf}{GokgurHO18}~\cite{GokgurHO18}, \href{works/OzturkTHO13.pdf}{OzturkTHO13}~\cite{OzturkTHO13}, \href{works/RossiTHP07.pdf}{RossiTHP07}~\cite{RossiTHP07}\\
Andrew J. Davenport & \href{works/Davenport10.pdf}{Davenport10}~\cite{Davenport10}, \href{works/DavenportKRSH07.pdf}{DavenportKRSH07}~\cite{DavenportKRSH07}, \href{works/BeckDF97.pdf}{BeckDF97}~\cite{BeckDF97}\\
Mikael Johansson & \href{works/Astrand0F21.pdf}{Astrand0F21}~\cite{Astrand0F21}, \href{works/AstrandJZ20.pdf}{AstrandJZ20}~\cite{AstrandJZ20}, \href{works/AstrandJZ18.pdf}{AstrandJZ18}~\cite{AstrandJZ18}\\
Narendra Jussien & \href{works/ClercqPBJ11.pdf}{ClercqPBJ11}~\cite{ClercqPBJ11}, \href{works/ElkhyariGJ02.pdf}{ElkhyariGJ02}~\cite{ElkhyariGJ02}, \href{works/ElkhyariGJ02a.pdf}{ElkhyariGJ02a}~\cite{ElkhyariGJ02a}\\
Tam{\'{a}}s Kis & \href{works/KovacsK11.pdf}{KovacsK11}~\cite{KovacsK11}, \href{works/KeriK07.pdf}{KeriK07}~\cite{KeriK07}, \href{works/KovacsEKV05.pdf}{KovacsEKV05}~\cite{KovacsEKV05}\\
Ouajdi Korbaa & \href{works/KorbaaYG00.pdf}{KorbaaYG00}~\cite{KorbaaYG00}, \href{works/LopezAKYG00.pdf}{LopezAKYG00}~\cite{LopezAKYG00}, \href{works/KorbaaYG99.pdf}{KorbaaYG99}~\cite{KorbaaYG99}\\
Krzysztof Kuchcinski & \href{works/WolinskiKG04.pdf}{WolinskiKG04}~\cite{WolinskiKG04}, \href{works/KuchcinskiW03.pdf}{KuchcinskiW03}~\cite{KuchcinskiW03}, \href{works/GruianK98.pdf}{GruianK98}~\cite{GruianK98}\\
Arnaud Letort & \href{works/LetortCB15.pdf}{LetortCB15}~\cite{LetortCB15}, \href{works/LetortCB13.pdf}{LetortCB13}~\cite{LetortCB13}, \href{works/LetortBC12.pdf}{LetortBC12}~\cite{LetortBC12}\\
Arnaud Malapert & \href{works/NattafM20.pdf}{NattafM20}~\cite{NattafM20}, \href{works/MalapertN19.pdf}{MalapertN19}~\cite{MalapertN19}, \href{works/GrimesHM09.pdf}{GrimesHM09}~\cite{GrimesHM09}\\
Tony Minoru Tamura Lopes & \href{works/LopesCSM10.pdf}{LopesCSM10}~\cite{LopesCSM10}, \href{works/MouraSCL08.pdf}{MouraSCL08}~\cite{MouraSCL08}, \href{works/MouraSCL08a.pdf}{MouraSCL08a}~\cite{MouraSCL08a}\\
Hiroki Nishikawa & \href{works/NishikawaSTT19.pdf}{NishikawaSTT19}~\cite{NishikawaSTT19}, \href{works/NishikawaSTT18.pdf}{NishikawaSTT18}~\cite{NishikawaSTT18}, \href{works/NishikawaSTT18a.pdf}{NishikawaSTT18a}~\cite{NishikawaSTT18a}\\
C{\'{e}}dric Pralet & \href{works/SquillaciPR23.pdf}{SquillaciPR23}~\cite{SquillaciPR23}, \href{works/Pralet17.pdf}{Pralet17}~\cite{Pralet17}, \href{works/PraletLJ15.pdf}{PraletLJ15}~\cite{PraletLJ15}\\
Dhananjay R. Thiruvady & \href{works/abs-2402-00459.pdf}{abs-2402-00459}~\cite{abs-2402-00459}, \href{works/abs-2211-14492.pdf}{abs-2211-14492}~\cite{abs-2211-14492}, \href{works/ThiruvadyBME09.pdf}{ThiruvadyBME09}~\cite{ThiruvadyBME09}\\
Levi Ribeiro de Abreu & \href{works/AbreuNP23.pdf}{AbreuNP23}~\cite{AbreuNP23}, \href{works/AbreuN22.pdf}{AbreuN22}~\cite{AbreuN22}, \href{works/AbreuAPNM21.pdf}{AbreuAPNM21}~\cite{AbreuAPNM21}\\
Jens Schulz & \href{works/HeinzSB13.pdf}{HeinzSB13}~\cite{HeinzSB13}, \href{works/HeinzS11.pdf}{HeinzS11}~\cite{HeinzS11}, \href{works/BertholdHLMS10.pdf}{BertholdHLMS10}~\cite{BertholdHLMS10}\\
Marcelo Seido Nagano & \href{works/AbreuNP23.pdf}{AbreuNP23}~\cite{AbreuNP23}, \href{works/AbreuN22.pdf}{AbreuN22}~\cite{AbreuN22}, \href{works/AbreuAPNM21.pdf}{AbreuAPNM21}~\cite{AbreuAPNM21}\\
Kana Shimada & \href{works/NishikawaSTT19.pdf}{NishikawaSTT19}~\cite{NishikawaSTT19}, \href{works/NishikawaSTT18.pdf}{NishikawaSTT18}~\cite{NishikawaSTT18}, \href{works/NishikawaSTT18a.pdf}{NishikawaSTT18a}~\cite{NishikawaSTT18a}\\
Gilles Simonin & \href{works/GodetLHS20.pdf}{GodetLHS20}~\cite{GodetLHS20}, \href{works/SimoninAHL15.pdf}{SimoninAHL15}~\cite{SimoninAHL15}, \href{works/SimoninAHL12.pdf}{SimoninAHL12}~\cite{SimoninAHL12}\\
Josep Suy & \href{works/BofillCSV17.pdf}{BofillCSV17}~\cite{BofillCSV17}, \href{works/BofillGSV15.pdf}{BofillGSV15}~\cite{BofillGSV15}, \href{works/BofillEGPSV14.pdf}{BofillEGPSV14}~\cite{BofillEGPSV14}\\
Ittetsu Taniguchi & \href{works/NishikawaSTT19.pdf}{NishikawaSTT19}~\cite{NishikawaSTT19}, \href{works/NishikawaSTT18.pdf}{NishikawaSTT18}~\cite{NishikawaSTT18}, \href{works/NishikawaSTT18a.pdf}{NishikawaSTT18a}~\cite{NishikawaSTT18a}\\
Pierre Tassel & \href{works/TasselGS23.pdf}{TasselGS23}~\cite{TasselGS23}, \href{works/abs-2306-05747.pdf}{abs-2306-05747}~\cite{abs-2306-05747}, \href{works/KovacsTKSG21.pdf}{KovacsTKSG21}~\cite{KovacsTKSG21}\\
Hiroyuki Tomiyama & \href{works/NishikawaSTT19.pdf}{NishikawaSTT19}~\cite{NishikawaSTT19}, \href{works/NishikawaSTT18.pdf}{NishikawaSTT18}~\cite{NishikawaSTT18}, \href{works/NishikawaSTT18a.pdf}{NishikawaSTT18a}~\cite{NishikawaSTT18a}\\
Seyda Topaloglu Yildiz & \href{works/IsikYA23.pdf}{IsikYA23}~\cite{IsikYA23}, \href{works/YunusogluY22.pdf}{YunusogluY22}~\cite{YunusogluY22}, \href{works/KucukY19.pdf}{KucukY19}~\cite{KucukY19}\\
Arnaldo Vieira Moura & \href{works/LopesCSM10.pdf}{LopesCSM10}~\cite{LopesCSM10}, \href{works/MouraSCL08.pdf}{MouraSCL08}~\cite{MouraSCL08}, \href{works/MouraSCL08a.pdf}{MouraSCL08a}~\cite{MouraSCL08a}\\
Mateu Villaret & \href{works/BofillCSV17.pdf}{BofillCSV17}~\cite{BofillCSV17}, \href{works/BofillGSV15.pdf}{BofillGSV15}~\cite{BofillGSV15}, \href{works/BofillEGPSV14.pdf}{BofillEGPSV14}~\cite{BofillEGPSV14}\\
Daniel Walkiewicz & \href{works/LacknerMMWW23.pdf}{LacknerMMWW23}~\cite{LacknerMMWW23}, \href{works/WinterMMW22.pdf}{WinterMMW22}~\cite{WinterMMW22}, \href{works/LacknerMMWW21.pdf}{LacknerMMWW21}~\cite{LacknerMMWW21}\\
Toby Walsh & \href{works/GelainPRVW17.pdf}{GelainPRVW17}~\cite{GelainPRVW17}, \href{works/BessiereHMQW14.pdf}{BessiereHMQW14}~\cite{BessiereHMQW14}, \href{works/HebrardTW05.pdf}{HebrardTW05}~\cite{HebrardTW05}\\
Pascal Yim & \href{works/KorbaaYG00.pdf}{KorbaaYG00}~\cite{KorbaaYG00}, \href{works/LopezAKYG00.pdf}{LopezAKYG00}~\cite{LopezAKYG00}, \href{works/KorbaaYG99.pdf}{KorbaaYG99}~\cite{KorbaaYG99}\\
Alessandro Zanarini & \href{works/AstrandJZ20.pdf}{AstrandJZ20}~\cite{AstrandJZ20}, \href{works/AstrandJZ18.pdf}{AstrandJZ18}~\cite{AstrandJZ18}, \href{works/BonfiettiZLM16.pdf}{BonfiettiZLM16}~\cite{BonfiettiZLM16}\\
Luis Zeballos & \href{works/ZeballosQH10.pdf}{ZeballosQH10}~\cite{ZeballosQH10}, \href{works/ZeballosH05.pdf}{ZeballosH05}~\cite{ZeballosH05}, \href{works/QuirogaZH05.pdf}{QuirogaZH05}~\cite{QuirogaZH05}\\
Laurence A. Wolsey & \href{works/HoundjiSWD14.pdf}{HoundjiSWD14}~\cite{HoundjiSWD14}, \href{works/SadykovW06.pdf}{SadykovW06}~\cite{SadykovW06}\\
Bruno A. Prata & \href{works/PrataAN23.pdf}{PrataAN23}~\cite{PrataAN23}, \href{works/AbreuNP23.pdf}{AbreuNP23}~\cite{AbreuNP23}\\
Eddie Armstrong & \href{works/ArmstrongGOS22.pdf}{ArmstrongGOS22}~\cite{ArmstrongGOS22}, \href{works/ArmstrongGOS21.pdf}{ArmstrongGOS21}~\cite{ArmstrongGOS21}\\
Amelia Badica & \href{works/BadicaBI20.pdf}{BadicaBI20}~\cite{BadicaBI20}, \href{works/BadicaBIL19.pdf}{BadicaBIL19}~\cite{BadicaBIL19}\\
Costin Badica & \href{works/BadicaBI20.pdf}{BadicaBI20}~\cite{BadicaBI20}, \href{works/BadicaBIL19.pdf}{BadicaBIL19}~\cite{BadicaBIL19}\\
Pierre Baptiste & \href{}{BoucherBVBL97}~\cite{BoucherBVBL97}, \href{works/BaptisteLV92.pdf}{BaptisteLV92}~\cite{BaptisteLV92}\\
Nicolas Barnier & \href{works/WangB23.pdf}{WangB23}~\cite{WangB23}, \href{works/WangB20.pdf}{WangB20}~\cite{WangB20}\\
Ondrej Benedikt & \href{works/BenediktMH20.pdf}{BenediktMH20}~\cite{BenediktMH20}, \href{works/BenediktSMVH18.pdf}{BenediktSMVH18}~\cite{BenediktSMVH18}\\
Davide Bertozzi & \href{works/RuggieroBBMA09.pdf}{RuggieroBBMA09}~\cite{RuggieroBBMA09}, \href{works/BeniniBGM06.pdf}{BeniniBGM06}~\cite{BeniniBGM06}\\
Jean{-}Charles Billaut & \href{works/BillautHL12.pdf}{BillautHL12}~\cite{BillautHL12}, \href{works/LorigeonBB02.pdf}{LorigeonBB02}~\cite{LorigeonBB02}\\
Andrea Borghesi & \href{works/BorghesiBLMB18.pdf}{BorghesiBLMB18}~\cite{BorghesiBLMB18}, \href{works/BartoliniBBLM14.pdf}{BartoliniBBLM14}~\cite{BartoliniBBLM14}\\
Dario Canut{-}de{-}Bon & \href{works/YuraszeckMCCR23.pdf}{YuraszeckMCCR23}~\cite{YuraszeckMCCR23}, \href{works/YuraszeckMC23.pdf}{YuraszeckMC23}~\cite{YuraszeckMC23}\\
Quentin Cappart & \href{works/PopovicCGNC22.pdf}{PopovicCGNC22}~\cite{PopovicCGNC22}, \href{works/CappartS17.pdf}{CappartS17}~\cite{CappartS17}\\
Amedeo Cesta & \href{works/OddiPCC03.pdf}{OddiPCC03}~\cite{OddiPCC03}, \href{works/CestaOS98.pdf}{CestaOS98}~\cite{CestaOS98}\\
Elvin Coban & \href{works/CireCH13.pdf}{CireCH13}~\cite{CireCH13}, \href{works/CobanH10.pdf}{CobanH10}~\cite{CobanH10}\\
Yves Colombani & \href{works/HeipckeCCS00.pdf}{HeipckeCCS00}~\cite{HeipckeCCS00}, \href{works/Colombani96.pdf}{Colombani96}~\cite{Colombani96}\\
Joseph D. Scott & \href{works/KameugneFSN14.pdf}{KameugneFSN14}~\cite{KameugneFSN14}, \href{works/KameugneFSN11.pdf}{KameugneFSN11}~\cite{KameugneFSN11}\\
Mauro Dell'Amico & \href{works/MontemanniD23.pdf}{MontemanniD23}~\cite{MontemanniD23}, \href{works/MontemanniD23a.pdf}{MontemanniD23a}~\cite{MontemanniD23a}\\
Hani El Sakkout & \href{works/KamarainenS02.pdf}{KamarainenS02}~\cite{KamarainenS02}, \href{works/SakkoutW00.pdf}{SakkoutW00}~\cite{SakkoutW00}\\
Abdallah Elkhyari & \href{works/ElkhyariGJ02.pdf}{ElkhyariGJ02}~\cite{ElkhyariGJ02}, \href{works/ElkhyariGJ02a.pdf}{ElkhyariGJ02a}~\cite{ElkhyariGJ02a}\\
Tamer Eren & \href{works/GurPAE23.pdf}{GurPAE23}~\cite{GurPAE23}, \href{works/GurEA19.pdf}{GurEA19}~\cite{GurEA19}\\
Caroline Even & \href{works/EvenSH15.pdf}{EvenSH15}~\cite{EvenSH15}, \href{works/EvenSH15a.pdf}{EvenSH15a}~\cite{EvenSH15a}\\
Minhaz F. Zibran & \href{works/ZibranR11.pdf}{ZibranR11}~\cite{ZibranR11}, \href{works/ZibranR11a.pdf}{ZibranR11a}~\cite{ZibranR11a}\\
Azadeh Farsi & \href{}{FarsiTM22}~\cite{FarsiTM22}, \href{works/MokhtarzadehTNF20.pdf}{MokhtarzadehTNF20}~\cite{MokhtarzadehTNF20}\\
Dominique Feillet & \href{works/Acuna-AgostMFG09.pdf}{Acuna-AgostMFG09}~\cite{Acuna-AgostMFG09}, \href{works/ArtiguesBF04.pdf}{ArtiguesBF04}~\cite{ArtiguesBF04}\\
Mark G. Wallace & \href{works/SchuttCSW12.pdf}{SchuttCSW12}~\cite{SchuttCSW12}, \href{works/SchuttFSW11.pdf}{SchuttFSW11}~\cite{SchuttFSW11}\\
Maurizio Gabbrielli & \href{works/LiuCGM17.pdf}{LiuCGM17}~\cite{LiuCGM17}, \href{works/FalaschiGMP97.pdf}{FalaschiGMP97}~\cite{FalaschiGMP97}\\
Michel Gamache & \href{works/AalianPG23.pdf}{AalianPG23}~\cite{AalianPG23}, \href{works/CampeauG22.pdf}{CampeauG22}~\cite{CampeauG22}\\
Marc Garcia & \href{works/BofillGSV15.pdf}{BofillGSV15}~\cite{BofillGSV15}, \href{works/BofillEGPSV14.pdf}{BofillEGPSV14}~\cite{BofillEGPSV14}\\
Antonio Garrido & \href{works/GarridoAO09.pdf}{GarridoAO09}~\cite{GarridoAO09}, \href{works/GarridoOS08.pdf}{GarridoOS08}~\cite{GarridoOS08}\\
Vincent Gingras & \href{works/KameugneFGOQ18.pdf}{KameugneFGOQ18}~\cite{KameugneFGOQ18}, \href{works/GingrasQ16.pdf}{GingrasQ16}~\cite{GingrasQ16}\\
Arthur Godet & \href{}{Godet21a}~\cite{Godet21a}, \href{works/GodetLHS20.pdf}{GodetLHS20}~\cite{GodetLHS20}\\
Arnaud Gotlieb & \href{works/MossigeGSMC17.pdf}{MossigeGSMC17}~\cite{MossigeGSMC17}, \href{works/AlesioNBG14.pdf}{AlesioNBG14}~\cite{AlesioNBG14}\\
Lucas Groleaz & \href{works/GroleazNS20.pdf}{GroleazNS20}~\cite{GroleazNS20}, \href{works/GroleazNS20a.pdf}{GroleazNS20a}~\cite{GroleazNS20a}\\
Christelle Gu{\'{e}}ret & \href{works/ElkhyariGJ02.pdf}{ElkhyariGJ02}~\cite{ElkhyariGJ02}, \href{works/ElkhyariGJ02a.pdf}{ElkhyariGJ02a}~\cite{ElkhyariGJ02a}\\
Andy Ham & \href{works/HamPK21.pdf}{HamPK21}~\cite{HamPK21}, \href{works/Ham18.pdf}{Ham18}~\cite{Ham18}\\
Vil{\'{e}}m Heinz & \href{works/abs-2305-19888.pdf}{abs-2305-19888}~\cite{abs-2305-19888}, \href{works/HeinzNVH22.pdf}{HeinzNVH22}~\cite{HeinzNVH22}\\
Seyed Hossein Hashemi Doulabi & \href{works/DoulabiRP16.pdf}{DoulabiRP16}~\cite{DoulabiRP16}, \href{works/DoulabiRP14.pdf}{DoulabiRP14}~\cite{DoulabiRP14}\\
Laurent Houssin & \href{works/JuvinHHL23.pdf}{JuvinHHL23}~\cite{JuvinHHL23}, \href{works/JuvinHL23.pdf}{JuvinHL23}~\cite{JuvinHL23}\\
Georgiana Ifrim & \href{works/GrimesIOS14.pdf}{GrimesIOS14}~\cite{GrimesIOS14}, \href{works/IfrimOS12.pdf}{IfrimOS12}~\cite{IfrimOS12}\\
Mirjana Ivanovic & \href{works/BadicaBI20.pdf}{BadicaBI20}~\cite{BadicaBI20}, \href{works/BadicaBIL19.pdf}{BadicaBIL19}~\cite{BadicaBIL19}\\
Willem Jan van Hoeve & \href{works/HoeveGSL07.pdf}{HoeveGSL07}~\cite{HoeveGSL07}, \href{works/GomesHS06.pdf}{GomesHS06}~\cite{GomesHS06}\\
Carla Juvin & \href{works/JuvinHHL23.pdf}{JuvinHHL23}~\cite{JuvinHHL23}, \href{works/JuvinHL23.pdf}{JuvinHL23}~\cite{JuvinHL23}\\
Chanchal K. Roy & \href{works/ZibranR11.pdf}{ZibranR11}~\cite{ZibranR11}, \href{works/ZibranR11a.pdf}{ZibranR11a}~\cite{ZibranR11a}\\
Lucas Kletzander & \href{works/GeibingerKKMMW21.pdf}{GeibingerKKMMW21}~\cite{GeibingerKKMMW21}, \href{works/KletzanderM17.pdf}{KletzanderM17}~\cite{KletzanderM17}\\
Stefan Kreter & \href{works/KreterSS17.pdf}{KreterSS17}~\cite{KreterSS17}, \href{works/KreterSS15.pdf}{KreterSS15}~\cite{KreterSS15}\\
Jan Kristof Behrens & \href{works/BehrensLM19.pdf}{BehrensLM19}~\cite{BehrensLM19}, \href{works/abs-1901-07914.pdf}{abs-1901-07914}~\cite{abs-1901-07914}\\
Marie{-}Louise Lackner & \href{works/LacknerMMWW23.pdf}{LacknerMMWW23}~\cite{LacknerMMWW23}, \href{works/LacknerMMWW21.pdf}{LacknerMMWW21}~\cite{LacknerMMWW21}\\
Arnaud Lallouet & \href{works/PerezGSL23.pdf}{PerezGSL23}~\cite{PerezGSL23}, \href{works/abs-2312-13682.pdf}{abs-2312-13682}~\cite{abs-2312-13682}\\
Evelina Lamma & \href{works/LammaMM97.pdf}{LammaMM97}~\cite{LammaMM97}, \href{works/BrusoniCLMMT96.pdf}{BrusoniCLMMT96}~\cite{BrusoniCLMMT96}\\
Ralph Lange & \href{works/BehrensLM19.pdf}{BehrensLM19}~\cite{BehrensLM19}, \href{works/abs-1901-07914.pdf}{abs-1901-07914}~\cite{abs-1901-07914}\\
Bruno Legeard & \href{}{BoucherBVBL97}~\cite{BoucherBVBL97}, \href{works/BaptisteLV92.pdf}{BaptisteLV92}~\cite{BaptisteLV92}\\
Michel Lema{\^{\i}}tre & \href{works/VerfaillieL01.pdf}{VerfaillieL01}~\cite{VerfaillieL01}, \href{works/BensanaLV99.pdf}{BensanaLV99}~\cite{BensanaLV99}\\
BoonPing Lim & \href{works/LimHTB16.pdf}{LimHTB16}~\cite{LimHTB16}, \href{works/LimBTBB15.pdf}{LimBTBB15}~\cite{LimBTBB15}\\
Kamol Limtanyakul & \href{works/LimtanyakulS12.pdf}{LimtanyakulS12}~\cite{LimtanyakulS12}, \href{works/Limtanyakul07.pdf}{Limtanyakul07}~\cite{Limtanyakul07}\\
James Little & \href{works/KrogtLPHJ07.pdf}{KrogtLPHJ07}~\cite{KrogtLPHJ07}, \href{works/Darby-DowmanLMZ97.pdf}{Darby-DowmanLMZ97}~\cite{Darby-DowmanLMZ97}\\
Shixin Liu & \href{works/LiFJZLL22.pdf}{LiFJZLL22}~\cite{LiFJZLL22}, \href{works/ZhangJZL22.pdf}{ZhangJZL22}~\cite{ZhangJZL22}\\
Xavier Lorca & \href{works/GodetLHS20.pdf}{GodetLHS20}~\cite{GodetLHS20}, \href{works/HermenierDL11.pdf}{HermenierDL11}~\cite{HermenierDL11}\\
Abid M. Malik & \href{}{Malik08}~\cite{Malik08}, \href{works/MalikMB08.pdf}{MalikMB08}~\cite{MalikMB08}\\
Gilles Madi{-}Wamba & \href{works/Madi-WambaLOBM17.pdf}{Madi-WambaLOBM17}~\cite{Madi-WambaLOBM17}, \href{works/Madi-WambaB16.pdf}{Madi-WambaB16}~\cite{Madi-WambaB16}\\
Masoumeh Mansouri & \href{works/BehrensLM19.pdf}{BehrensLM19}~\cite{BehrensLM19}, \href{works/abs-1901-07914.pdf}{abs-1901-07914}~\cite{abs-1901-07914}\\
Gonzalo Mej{\'{\i}}a & \href{works/YuraszeckMC23.pdf}{YuraszeckMC23}~\cite{YuraszeckMC23}, \href{works/MejiaY20.pdf}{MejiaY20}~\cite{MejiaY20}\\
Paola Mello & \href{works/LammaMM97.pdf}{LammaMM97}~\cite{LammaMM97}, \href{works/BrusoniCLMMT96.pdf}{BrusoniCLMMT96}~\cite{BrusoniCLMMT96}\\
Philippe Michelon & \href{works/Acuna-AgostMFG09.pdf}{Acuna-AgostMFG09}~\cite{Acuna-AgostMFG09}, \href{works/LiessM08.pdf}{LiessM08}~\cite{LiessM08}\\
Mahdi Mokhtarzadeh & \href{}{FarsiTM22}~\cite{FarsiTM22}, \href{works/MokhtarzadehTNF20.pdf}{MokhtarzadehTNF20}~\cite{MokhtarzadehTNF20}\\
Roberto Montemanni & \href{works/MontemanniD23.pdf}{MontemanniD23}~\cite{MontemanniD23}, \href{works/MontemanniD23a.pdf}{MontemanniD23a}~\cite{MontemanniD23a}\\
Christoph Mrkvicka & \href{works/LacknerMMWW23.pdf}{LacknerMMWW23}~\cite{LacknerMMWW23}, \href{works/LacknerMMWW21.pdf}{LacknerMMWW21}~\cite{LacknerMMWW21}\\
Istv{\'{a}}n M{\'{o}}dos & \href{works/BenediktMH20.pdf}{BenediktMH20}~\cite{BenediktMH20}, \href{works/BenediktSMVH18.pdf}{BenediktSMVH18}~\cite{BenediktSMVH18}\\
Kenneth N. Brown & \href{works/MurphyMB15.pdf}{MurphyMB15}~\cite{MurphyMB15}, \href{works/WuBB05.pdf}{WuBB05}~\cite{WuBB05}\\
Samba Ndojh Ndiaye & \href{works/GroleazNS20.pdf}{GroleazNS20}~\cite{GroleazNS20}, \href{works/GroleazNS20a.pdf}{GroleazNS20a}~\cite{GroleazNS20a}\\
Youcheu Ngo{-}Kateu & \href{works/KameugneFSN14.pdf}{KameugneFSN14}~\cite{KameugneFSN14}, \href{works/KameugneFSN11.pdf}{KameugneFSN11}~\cite{KameugneFSN11}\\
Su Nguyen & \href{works/abs-2402-00459.pdf}{abs-2402-00459}~\cite{abs-2402-00459}, \href{works/abs-2211-14492.pdf}{abs-2211-14492}~\cite{abs-2211-14492}\\
Anton{\'{\i}}n Nov{\'{a}}k & \href{works/abs-2305-19888.pdf}{abs-2305-19888}~\cite{abs-2305-19888}, \href{works/HeinzNVH22.pdf}{HeinzNVH22}~\cite{HeinzNVH22}\\
Angelo Oddi & \href{works/OddiPCC03.pdf}{OddiPCC03}~\cite{OddiPCC03}, \href{works/CestaOS98.pdf}{CestaOS98}~\cite{CestaOS98}\\
Eva Onaindia & \href{works/GarridoAO09.pdf}{GarridoAO09}~\cite{GarridoAO09}, \href{works/GarridoOS08.pdf}{GarridoOS08}~\cite{GarridoOS08}\\
Carla P. Gomes & \href{works/HoeveGSL07.pdf}{HoeveGSL07}~\cite{HoeveGSL07}, \href{works/GomesHS06.pdf}{GomesHS06}~\cite{GomesHS06}\\
Laure Pauline Fotso & \href{works/KameugneFSN14.pdf}{KameugneFSN14}~\cite{KameugneFSN14}, \href{works/KameugneFSN11.pdf}{KameugneFSN11}~\cite{KameugneFSN11}\\
Guillaume Perez & \href{works/PerezGSL23.pdf}{PerezGSL23}~\cite{PerezGSL23}, \href{works/abs-2312-13682.pdf}{abs-2312-13682}~\cite{abs-2312-13682}\\
Erwin Pesch & \href{works/MullerMKP22.pdf}{MullerMKP22}~\cite{MullerMKP22}, \href{}{BlazewiczEP19}~\cite{BlazewiczEP19}\\
Enrico Pontelli & \href{works/TardivoDFMP23.pdf}{TardivoDFMP23}~\cite{TardivoDFMP23}, \href{}{VillaverdeP04}~\cite{VillaverdeP04}\\
Oscar Quiroga & \href{works/ZeballosQH10.pdf}{ZeballosQH10}~\cite{ZeballosQH10}, \href{works/QuirogaZH05.pdf}{QuirogaZH05}~\cite{QuirogaZH05}\\
G{\"{u}}nther R. Raidl & \href{works/FrohnerTR19.pdf}{FrohnerTR19}~\cite{FrohnerTR19}, \href{works/RendlPHPR12.pdf}{RendlPHPR12}~\cite{RendlPHPR12}\\
Francesca Rossi & \href{works/GelainPRVW17.pdf}{GelainPRVW17}~\cite{GelainPRVW17}, \href{works/BartakSR10.pdf}{BartakSR10}~\cite{BartakSR10}\\
Martino Ruggiero & \href{works/BeniniLMR11.pdf}{BeniniLMR11}~\cite{BeniniLMR11}, \href{works/RuggieroBBMA09.pdf}{RuggieroBBMA09}~\cite{RuggieroBBMA09}\\
Ruslan Sadykov & \href{works/SadykovW06.pdf}{SadykovW06}~\cite{SadykovW06}, \href{works/Sadykov04.pdf}{Sadykov04}~\cite{Sadykov04}\\
Konstantin Schekotihin & \href{works/TasselGS23.pdf}{TasselGS23}~\cite{TasselGS23}, \href{works/abs-2306-05747.pdf}{abs-2306-05747}~\cite{abs-2306-05747}\\
Christian Schulte & \href{works/WessenCS20.pdf}{WessenCS20}~\cite{WessenCS20}, \href{works/FrimodigS19.pdf}{FrimodigS19}~\cite{FrimodigS19}\\
Bart Selman & \href{works/HoeveGSL07.pdf}{HoeveGSL07}~\cite{HoeveGSL07}, \href{works/GomesHS06.pdf}{GomesHS06}~\cite{GomesHS06}\\
Paul Shaw & \href{works/LaborieRSV18.pdf}{LaborieRSV18}~\cite{LaborieRSV18}, \href{works/VilimLS15.pdf}{VilimLS15}~\cite{VilimLS15}\\
Mohamed Siala & \href{works/Siala15.pdf}{Siala15}~\cite{Siala15}, \href{works/SialaAH15.pdf}{SialaAH15}~\cite{SialaAH15}\\
Wijnand Suijlen & \href{works/PerezGSL23.pdf}{PerezGSL23}~\cite{PerezGSL23}, \href{works/abs-2312-13682.pdf}{abs-2312-13682}~\cite{abs-2312-13682}\\
Yuan Sun & \href{works/abs-2402-00459.pdf}{abs-2402-00459}~\cite{abs-2402-00459}, \href{works/abs-2211-14492.pdf}{abs-2211-14492}~\cite{abs-2211-14492}\\
Andreas T. Ernst & \href{works/abs-2211-14492.pdf}{abs-2211-14492}~\cite{abs-2211-14492}, \href{works/ThiruvadyBME09.pdf}{ThiruvadyBME09}~\cite{ThiruvadyBME09}\\
Reza Tavakkoli{-}Moghaddam & \href{works/Mehdizadeh-Somarin23.pdf}{Mehdizadeh-Somarin23}~\cite{Mehdizadeh-Somarin23}, \href{works/MokhtarzadehTNF20.pdf}{MokhtarzadehTNF20}~\cite{MokhtarzadehTNF20}\\
Cl{\'{e}}mentin Tayou Djam{\'{e}}gni & \href{works/KameugneFND23.pdf}{KameugneFND23}~\cite{KameugneFND23}, \href{works/FetgoD22.pdf}{FetgoD22}~\cite{FetgoD22}\\
Erich Teppan & \href{works/abs-2102-08778.pdf}{abs-2102-08778}~\cite{abs-2102-08778}, \href{}{FriedrichFMRSST14}~\cite{FriedrichFMRSST14}\\
Alexander Tesch & \href{works/Tesch18.pdf}{Tesch18}~\cite{Tesch18}, \href{works/Tesch16.pdf}{Tesch16}~\cite{Tesch16}\\
Sylvie Thi{\'{e}}baux & \href{works/LimHTB16.pdf}{LimHTB16}~\cite{LimHTB16}, \href{works/LimBTBB15.pdf}{LimBTBB15}~\cite{LimBTBB15}\\
Behdin Vahedi Nouri & \href{works/Mehdizadeh-Somarin23.pdf}{Mehdizadeh-Somarin23}~\cite{Mehdizadeh-Somarin23}, \href{works/MokhtarzadehTNF20.pdf}{MokhtarzadehTNF20}~\cite{MokhtarzadehTNF20}\\
Sascha Van Cauwelaert & \href{works/CauwelaertDMS16.pdf}{CauwelaertDMS16}~\cite{CauwelaertDMS16}, \href{works/DejemeppeCS15.pdf}{DejemeppeCS15}~\cite{DejemeppeCS15}\\
Christophe Varnier & \href{}{BoucherBVBL97}~\cite{BoucherBVBL97}, \href{works/BaptisteLV92.pdf}{BaptisteLV92}~\cite{BaptisteLV92}\\
G{\'{e}}rard Verfaillie & \href{works/VerfaillieL01.pdf}{VerfaillieL01}~\cite{VerfaillieL01}, \href{works/BensanaLV99.pdf}{BensanaLV99}~\cite{BensanaLV99}\\
Ruixin Wang & \href{works/WangB23.pdf}{WangB23}~\cite{WangB23}, \href{works/WangB20.pdf}{WangB20}~\cite{WangB20}\\
Jean{-}Paul Watson & \href{works/BeckFW11.pdf}{BeckFW11}~\cite{BeckFW11}, \href{works/WatsonB08.pdf}{WatsonB08}~\cite{WatsonB08}\\
Christophe Wolinski & \href{works/WolinskiKG04.pdf}{WolinskiKG04}~\cite{WolinskiKG04}, \href{works/KuchcinskiW03.pdf}{KuchcinskiW03}~\cite{KuchcinskiW03}\\
Farouk Yalaoui & \href{works/OujanaAYB22.pdf}{OujanaAYB22}~\cite{OujanaAYB22}, \href{works/ArbaouiY18.pdf}{ArbaouiY18}~\cite{ArbaouiY18}\\
Neil Yorke{-}Smith & \href{works/EfthymiouY23.pdf}{EfthymiouY23}~\cite{EfthymiouY23}, \href{works/WallaceY20.pdf}{WallaceY20}~\cite{WallaceY20}\\
Ziyan Zhao & \href{works/LiFJZLL22.pdf}{LiFJZLL22}~\cite{LiFJZLL22}, \href{works/ZhangJZL22.pdf}{ZhangJZL22}~\cite{ZhangJZL22}\\
Jianyang Zhou & \href{works/Zhou97.pdf}{Zhou97}~\cite{Zhou97}, \href{works/Zhou96.pdf}{Zhou96}~\cite{Zhou96}\\
Willem{-}Jan van Hoeve & \href{works/GilesH16.pdf}{GilesH16}~\cite{GilesH16}, \href{works/GoelSHFS15.pdf}{GoelSHFS15}~\cite{GoelSHFS15}\\
Menkes van den Briel & \href{works/LimHTB16.pdf}{LimHTB16}~\cite{LimHTB16}, \href{works/LimBTBB15.pdf}{LimBTBB15}~\cite{LimBTBB15}\\
Peter van Beek & \href{works/BegB13.pdf}{BegB13}~\cite{BegB13}, \href{works/MalikMB08.pdf}{MalikMB08}~\cite{MalikMB08}\\
Florian A. Herzog & \href{works/KoehlerBFFHPSSS21.pdf}{KoehlerBFFHPSSS21}~\cite{KoehlerBFFHPSSS21}\\
J. A. Hoogeveen & \href{works/AkkerDH07.pdf}{AkkerDH07}~\cite{AkkerDH07}\\
M. A. Hakim Newton & \href{works/RiahiNS018.pdf}{RiahiNS018}~\cite{RiahiNS018}\\
Viktoria A. Hauder & \href{works/abs-1902-09244.pdf}{abs-1902-09244}~\cite{abs-1902-09244}\\
Amr A. Kandil & \href{works/TangLWSK18.pdf}{TangLWSK18}~\cite{TangLWSK18}\\
Antonio A. M{\'{a}}rquez & \href{works/ValleMGT03.pdf}{ValleMGT03}~\cite{ValleMGT03}\\
Kennedy A. G. Ara{\'u}jo & \href{works/AbreuAPNM21.pdf}{AbreuAPNM21}~\cite{AbreuAPNM21}\\
Younes Aalian & \href{works/AalianPG23.pdf}{AalianPG23}~\cite{AalianPG23}\\
Hanaa Abohashima & \href{works/AbohashimaEG21.pdf}{AbohashimaEG21}~\cite{AbohashimaEG21}\\
Montserrat Abril & \href{works/AbrilSB05.pdf}{AbrilSB05}~\cite{AbrilSB05}\\
Rodrigo Acuna{-}Agost & \href{works/Acuna-AgostMFG09.pdf}{Acuna-AgostMFG09}~\cite{Acuna-AgostMFG09}\\
W. Adelman & \href{works/EscobetPQPRA19.pdf}{EscobetPQPRA19}~\cite{EscobetPQPRA19}\\
Michael Affenzeller & \href{works/abs-1902-09244.pdf}{abs-1902-09244}~\cite{abs-1902-09244}\\
Abderrahmane Aggoun & \href{works/AggounB93.pdf}{AggounB93}~\cite{AggounB93}\\
Pen{\'{e}}lope Aguiar{-}Melgarejo & \href{works/MelgarejoLS15.pdf}{MelgarejoLS15}~\cite{MelgarejoLS15}\\
Sanjay Ahire & \href{}{KanetAG04}~\cite{KanetAG04}\\
Aftab Ahmed Shaikh & \href{works/ShaikhK23.pdf}{ShaikhK23}~\cite{ShaikhK23}\\
Uwe Aickelin & \href{works/abs-2211-14492.pdf}{abs-2211-14492}~\cite{abs-2211-14492}\\
Mohsen Akbarpour Shirazi & \href{works/ZarandiKS16.pdf}{ZarandiKS16}~\cite{ZarandiKS16}\\
Arianna Alfieri & \href{works/AlfieriGPS23.pdf}{AlfieriGPS23}~\cite{AlfieriGPS23}\\
S. Ali Torabi & \href{}{FarsiTM22}~\cite{FarsiTM22}\\
Samira Alizdeh & \href{}{AlizdehS20}~\cite{AlizdehS20}\\
Hassane Alla & \href{works/LopezAKYG00.pdf}{LopezAKYG00}~\cite{LopezAKYG00}\\
Lionel Amodeo & \href{works/OujanaAYB22.pdf}{OujanaAYB22}~\cite{OujanaAYB22}\\
Alexandru Andrei & \href{works/RuggieroBBMA09.pdf}{RuggieroBBMA09}~\cite{RuggieroBBMA09}\\
Ola Angelsmark & \href{works/AngelsmarkJ00.pdf}{AngelsmarkJ00}~\cite{AngelsmarkJ00}\\
M. Anton Ertl & \href{works/ErtlK91.pdf}{ErtlK91}~\cite{ErtlK91}\\
Zbigniew Antoni Banaszak & \href{works/BocewiczBB09.pdf}{BocewiczBB09}~\cite{BocewiczBB09}\\
Valentin Antuori & \href{works/AntuoriHHEN21.pdf}{AntuoriHHEN21}~\cite{AntuoriHHEN21}\\
Marlene Arang{\'{u}} & \href{works/GarridoAO09.pdf}{GarridoAO09}~\cite{GarridoAO09}\\
Taha Arbaoui & \href{works/ArbaouiY18.pdf}{ArbaouiY18}~\cite{ArbaouiY18}\\
Martin Aronsson & \href{works/AronssonBK09.pdf}{AronssonBK09}~\cite{AronssonBK09}\\
M. Arslan Ornek & \href{works/OzturkTHO13.pdf}{OzturkTHO13}~\cite{OzturkTHO13}\\
Konstantin Artiouchine & \href{works/ArtiouchineB05.pdf}{ArtiouchineB05}~\cite{ArtiouchineB05}\\
Arezoo Atighehchian & \href{works/YounespourAKE19.pdf}{YounespourAKE19}~\cite{YounespourAKE19}\\
Abdullah Ayub Khan & \href{works/ShaikhK23.pdf}{ShaikhK23}~\cite{ShaikhK23}\\
Emrah B. Edis & \href{works/EdisO11.pdf}{EdisO11}~\cite{EdisO11}\\
Amr B. Eltawil & \href{works/AbohashimaEG21.pdf}{AbohashimaEG21}~\cite{AbohashimaEG21}\\
Maya B. Gokhale & \href{works/WolinskiKG04.pdf}{WolinskiKG04}~\cite{WolinskiKG04}\\
David B. H. Tay & \href{}{Tay92}~\cite{Tay92}\\
{\"{O}}zalp Babaoglu & \href{works/GalleguillosKSB19.pdf}{GalleguillosKSB19}~\cite{GalleguillosKSB19}\\
Irena Bach & \href{works/BocewiczBB09.pdf}{BocewiczBB09}~\cite{BocewiczBB09}\\
Astrid Bachelu & \href{}{BoucherBVBL97}~\cite{BoucherBVBL97}\\
Scott Backhaus & \href{works/LimBTBB15.pdf}{LimBTBB15}~\cite{LimBTBB15}\\
Naderi, Bahman & \href{works/NaderiRR23.pdf}{NaderiRR23}~\cite{NaderiRR23}\\
Hari Balasubramanian & \href{works/ShinBBHO18.pdf}{ShinBBHO18}~\cite{ShinBBHO18}\\
Viet Bang Nguyen & \href{works/LauLN08.pdf}{LauLN08}~\cite{LauLN08}\\
Federico Barber & \href{works/AbrilSB05.pdf}{AbrilSB05}~\cite{AbrilSB05}\\
Ada Barlatt & \href{works/BarlattCG08.pdf}{BarlattCG08}~\cite{BarlattCG08}\\
Mohammadreza Barzegaran & \href{works/BarzegaranZP20.pdf}{BarzegaranZP20}~\cite{BarzegaranZP20}\\
Virginie Basini & \href{works/Polo-MejiaALB20.pdf}{Polo-MejiaALB20}~\cite{Polo-MejiaALB20}\\
Andreas Beham & \href{works/abs-1902-09244.pdf}{abs-1902-09244}~\cite{abs-1902-09244}\\
N Beldiceanu & \href{works/BeldiceanuC94.pdf}{BeldiceanuC94}~\cite{BeldiceanuC94}\\
Said Belhadji & \href{works/BelhadjiI98.pdf}{BelhadjiI98}~\cite{BelhadjiI98}\\
Sana Belmokhtar & \href{works/ArtiguesBF04.pdf}{ArtiguesBF04}~\cite{ArtiguesBF04}\\
Fatima Benbouzid{-}Si Tayeb & \href{works/TouatBT22.pdf}{TouatBT22}~\cite{TouatBT22}\\
Till Bender & \href{works/BenderWS21.pdf}{BenderWS21}~\cite{BenderWS21}\\
Belaid Benhamou & \href{works/TouatBT22.pdf}{TouatBT22}~\cite{TouatBT22}\\
Hachemi Bennaceur & \href{works/KhemmoudjPB06.pdf}{KhemmoudjPB06}~\cite{KhemmoudjPB06}\\
E. Bensana & \href{works/BensanaLV99.pdf}{BensanaLV99}~\cite{BensanaLV99}\\
Russell Bent & \href{works/LimBTBB15.pdf}{LimBTBB15}~\cite{LimBTBB15}\\
Timo Berthold & \href{works/BertholdHLMS10.pdf}{BertholdHLMS10}~\cite{BertholdHLMS10}\\
Christian Bessiere & \href{works/BessiereHMQW14.pdf}{BessiereHMQW14}~\cite{BessiereHMQW14}\\
Arthur Bit{-}Monnot & \href{works/Bit-Monnot23.pdf}{Bit-Monnot23}~\cite{Bit-Monnot23}\\
Jacek Blazewicz & \href{}{BlazewiczEP19}~\cite{BlazewiczEP19}\\
Christian Blum & \href{works/ThiruvadyBME09.pdf}{ThiruvadyBME09}~\cite{ThiruvadyBME09}\\
Grzegorz Bocewicz & \href{works/BocewiczBB09.pdf}{BocewiczBB09}~\cite{BocewiczBB09}\\
Markus Bohlin & \href{works/AronssonBK09.pdf}{AronssonBK09}~\cite{AronssonBK09}\\
Nicolas Bonifas & \href{works/BaptisteB18.pdf}{BaptisteB18}~\cite{BaptisteB18}\\
Eric Boucher & \href{}{BoucherBVBL97}~\cite{BoucherBVBL97}\\
Rapha{\"{e}}l Boudreault & \href{works/BoudreaultSLQ22.pdf}{BoudreaultSLQ22}~\cite{BoudreaultSLQ22}\\
Jean{-}Louis Bouquard & \href{works/LorigeonBB02.pdf}{LorigeonBB02}~\cite{LorigeonBB02}\\
Eric Bourreau & \href{works/BourreauGGLT22.pdf}{BourreauGGLT22}~\cite{BourreauGGLT22}\\
Silvia Breitinger & \href{}{BreitingerL95}~\cite{BreitingerL95}\\
Kristen Brent Venable & \href{works/GelainPRVW17.pdf}{GelainPRVW17}~\cite{GelainPRVW17}\\
D. Brodart & \href{works/OujanaAYB22.pdf}{OujanaAYB22}~\cite{OujanaAYB22}\\
Yuriy Brun & \href{works/ShinBBHO18.pdf}{ShinBBHO18}~\cite{ShinBBHO18}\\
Vittorio Brusoni & \href{works/BrusoniCLMMT96.pdf}{BrusoniCLMMT96}~\cite{BrusoniCLMMT96}\\
Josef B{\"{u}}rgler & \href{works/KoehlerBFFHPSSS21.pdf}{KoehlerBFFHPSSS21}~\cite{KoehlerBFFHPSSS21}\\
Cristina C. B. Cavalcante & \href{works/HeipckeCCS00.pdf}{HeipckeCCS00}~\cite{HeipckeCCS00}\\
Lionel C. Briand & \href{works/AlesioNBG14.pdf}{AlesioNBG14}~\cite{AlesioNBG14}\\
Eugene C. Freuder & \href{works/CarchraeBF05.pdf}{CarchraeBF05}~\cite{CarchraeBF05}\\
Kevin C. Furman & \href{works/GoelSHFS15.pdf}{GoelSHFS15}~\cite{GoelSHFS15}\\
Joseph C. Pemberton & \href{works/PembertonG98.pdf}{PembertonG98}~\cite{PembertonG98}\\
Hendrik C. R. Lock & \href{}{BreitingerL95}~\cite{BreitingerL95}\\
Louis{-}Pierre Campeau & \href{works/CampeauG22.pdf}{CampeauG22}~\cite{CampeauG22}\\
Tom Carchrae & \href{works/CarchraeBF05.pdf}{CarchraeBF05}~\cite{CarchraeBF05}\\
Cid Carvalho de Souza & \href{works/LopesCSM10.pdf}{LopesCSM10}~\cite{LopesCSM10}\\
Yves Caseau & \href{works/Caseau97.pdf}{Caseau97}~\cite{Caseau97}\\
Yao{-}Ting Chang & \href{works/HoYCLLCLC18.pdf}{HoYCLLCLC18}~\cite{HoYCLLCLC18}\\
Nicolas Chapados & \href{works/ChapadosJR11.pdf}{ChapadosJR11}~\cite{ChapadosJR11}\\
Mohammad Cherkaoui & \href{works/FallahiAC20.pdf}{FallahiAC20}~\cite{FallahiAC20}\\
Han{-}Mo Chiu & \href{works/HoYCLLCLC18.pdf}{HoYCLLCLC18}~\cite{HoYCLLCLC18}\\
Yeonjun Choi & \href{works/KimCMLLP23.pdf}{KimCMLLP23}~\cite{KimCMLLP23}\\
Geoffrey Chu & \href{works/SchuttCSW12.pdf}{SchuttCSW12}~\cite{SchuttCSW12}\\
Yingyi Chu & \href{works/ChuX05.pdf}{ChuX05}~\cite{ChuX05}\\
Sue{-}Min Chu & \href{works/HoYCLLCLC18.pdf}{HoYCLLCLC18}~\cite{HoYCLLCLC18}\\
Hoong Chuin Lau & \href{works/LauLN08.pdf}{LauLN08}~\cite{LauLN08}\\
Carleton Coffrin & \href{works/SchausHMCMD11.pdf}{SchausHMCMD11}~\cite{SchausHMCMD11}\\
Jordi Coll Caballero & \href{works/Caballero23.pdf}{Caballero23}~\cite{Caballero23}\\
Jordi Coll & \href{works/BofillCSV17.pdf}{BofillCSV17}~\cite{BofillCSV17}\\
Luca Console & \href{works/BrusoniCLMMT96.pdf}{BrusoniCLMMT96}~\cite{BrusoniCLMMT96}\\
E Contejean & \href{works/BeldiceanuC94.pdf}{BeldiceanuC94}~\cite{BeldiceanuC94}\\
Trijntje Cornelissens & \href{works/SimonisC95.pdf}{SimonisC95}~\cite{SimonisC95}\\
Gabriella Cortellessa & \href{works/OddiPCC03.pdf}{OddiPCC03}~\cite{OddiPCC03}\\
Nicol{\'{a}}s Cuneo & \href{works/YuraszeckMCCR23.pdf}{YuraszeckMCCR23}~\cite{YuraszeckMCCR23}\\
Alain C{\^{o}}t{\'{e}} & \href{works/PopovicCGNC22.pdf}{PopovicCGNC22}~\cite{PopovicCGNC22}\\
Kenneth D. Young & \href{works/YoungFS17.pdf}{YoungFS17}~\cite{YoungFS17}\\
Laurent D. Michel & \href{works/FontaineMH16.pdf}{FontaineMH16}~\cite{FontaineMH16}\\
Steven D. Prestwich & \href{works/RossiTHP07.pdf}{RossiTHP07}~\cite{RossiTHP07}\\
Michael D. Moffitt & \href{works/MoffittPP05.pdf}{MoffittPP05}~\cite{MoffittPP05}\\
Emilie Danna & \href{works/DannaP03.pdf}{DannaP03}~\cite{DannaP03}\\
Ken Darby{-}Dowman & \href{works/Darby-DowmanLMZ97.pdf}{Darby-DowmanLMZ97}~\cite{Darby-DowmanLMZ97}\\
Vivian De Smedt & \href{works/GaySS14.pdf}{GaySS14}~\cite{GaySS14}\\
Alexis De Clercq & \href{works/ClercqPBJ11.pdf}{ClercqPBJ11}~\cite{ClercqPBJ11}\\
Rina Dechter & \href{works/FrostD98.pdf}{FrostD98}~\cite{FrostD98}\\
Carmelo Del Valle & \href{works/ValleMGT03.pdf}{ValleMGT03}~\cite{ValleMGT03}\\
Xavier Delorme & \href{works/RodriguezDG02.pdf}{RodriguezDG02}~\cite{RodriguezDG02}\\
Alain Demeure & \href{}{JourdanFRD94}~\cite{JourdanFRD94}\\
Emir Demirovic & \href{works/DemirovicS18.pdf}{DemirovicS18}~\cite{DemirovicS18}\\
Roberto Di Cosmo & \href{works/LiuCGM17.pdf}{LiuCGM17}~\cite{LiuCGM17}\\
Guido Diepen & \href{works/AkkerDH07.pdf}{AkkerDH07}~\cite{AkkerDH07}\\
Bistra Dilkina & \href{works/DilkinaDH05.pdf}{DilkinaDH05}~\cite{DilkinaDH05}\\
Mehmet Dincbas & \href{works/DincbasSH90.pdf}{DincbasSH90}~\cite{DincbasSH90}\\
Gr{\'{e}}goire Dooms & \href{works/DoomsH08.pdf}{DoomsH08}~\cite{DoomsH08}\\
Agostino Dovier & \href{works/TardivoDFMP23.pdf}{TardivoDFMP23}~\cite{TardivoDFMP23}\\
Yuquan Du & \href{works/QinDCS20.pdf}{QinDCS20}~\cite{QinDCS20}\\
Lei Duan & \href{works/DilkinaDH05.pdf}{DilkinaDH05}~\cite{DilkinaDH05}\\
Didier Dubois & \href{works/FortinZDF05.pdf}{FortinZDF05}~\cite{FortinZDF05}\\
Pierre Dupont & \href{works/MonetteDD07.pdf}{MonetteDD07}~\cite{MonetteDD07}\\
David Duvivier & \href{works/WangMD15.pdf}{WangMD15}~\cite{WangMD15}\\
Kyle E. C. Booth & \href{works/BoothNB16.pdf}{BoothNB16}~\cite{BoothNB16}\\
Marco E. L{\"{u}}bbecke & \href{works/BertholdHLMS10.pdf}{BertholdHLMS10}~\cite{BertholdHLMS10}\\
Ignacio E. Grossmann & \href{works/MaraveliasG04.pdf}{MaraveliasG04}~\cite{MaraveliasG04}\\
Andrew E. Santosa & \href{works/ZhuS02.pdf}{ZhuS02}~\cite{ZhuS02}\\
Martha E. Pollack & \href{works/MoffittPP05.pdf}{MoffittPP05}~\cite{MoffittPP05}\\
Nikolaos Efthymiou & \href{works/EfthymiouY23.pdf}{EfthymiouY23}~\cite{EfthymiouY23}\\
Gokhan Egilmez & \href{works/GedikKEK18.pdf}{GedikKEK18}~\cite{GedikKEK18}\\
P{\'{e}}ter Egri & \href{works/KovacsEKV05.pdf}{KovacsEKV05}~\cite{KovacsEKV05}\\
Nizar El Hachemi & \href{works/HachemiGR11.pdf}{HachemiGR11}~\cite{HachemiGR11}\\
Ghada El Khayat & \href{works/KhayatLR06.pdf}{KhayatLR06}~\cite{KhayatLR06}\\
Abdellah El Fallahi & \href{works/FallahiAC20.pdf}{FallahiAC20}~\cite{FallahiAC20}\\
Sebastian Engell & \href{works/KlankeBYE21.pdf}{KlankeBYE21}~\cite{KlankeBYE21}\\
Ey{\"{u}}p Ensar Isik & \href{works/IsikYA23.pdf}{IsikYA23}~\cite{IsikYA23}\\
Teresa Escobet & \href{works/EscobetPQPRA19.pdf}{EscobetPQPRA19}~\cite{EscobetPQPRA19}\\
Joan Espasa & \href{works/BofillEGPSV14.pdf}{BofillEGPSV14}~\cite{BofillEGPSV14}\\
Siham Essodaigui & \href{works/AntuoriHHEN21.pdf}{AntuoriHHEN21}~\cite{AntuoriHHEN21}\\
Stephen F. Smith & \href{works/CestaOS98.pdf}{CestaOS98}~\cite{CestaOS98}\\
Michael F. Gorman & \href{}{KanetAG04}~\cite{KanetAG04}\\
Mohd Fadlee A. Rasid & \href{works/AkramNHRSA23.pdf}{AkramNHRSA23}~\cite{AkramNHRSA23}\\
Fran{\c{c}}ois Fages & \href{}{JourdanFRD94}~\cite{JourdanFRD94}\\
Hamed Fahimi & \href{works/FahimiOQ18.pdf}{FahimiOQ18}~\cite{FahimiOQ18}\\
Moreno Falaschi & \href{works/FalaschiGMP97.pdf}{FalaschiGMP97}~\cite{FalaschiGMP97}\\
Huali Fan & \href{works/FanXG21.pdf}{FanXG21}~\cite{FanXG21}\\
H{\'{e}}l{\`{e}}ne Fargier & \href{works/FortinZDF05.pdf}{FortinZDF05}~\cite{FortinZDF05}\\
Soroush Fatemi-Anaraki & \href{}{Fatemi-AnarakiMFN22}~\cite{Fatemi-AnarakiMFN22}\\
Filippo Focacci & \href{works/FocacciLN00.pdf}{FocacciLN00}~\cite{FocacciLN00}\\
Daniel Fontaine & \href{works/FontaineMH16.pdf}{FontaineMH16}~\cite{FontaineMH16}\\
Urs Fontana & \href{works/KoehlerBFFHPSSS21.pdf}{KoehlerBFFHPSSS21}~\cite{KoehlerBFFHPSSS21}\\
Andrea Formisano & \href{works/TardivoDFMP23.pdf}{TardivoDFMP23}~\cite{TardivoDFMP23}\\
J{\'{e}}r{\^{o}}me Fortin & \href{works/FortinZDF05.pdf}{FortinZDF05}~\cite{FortinZDF05}\\
Mehdi Foumani & \href{}{Fatemi-AnarakiMFN22}~\cite{Fatemi-AnarakiMFN22}\\
Jeremy Frank & \href{works/FrankK05.pdf}{FrankK05}~\cite{FrankK05}\\
Gerhard Friedrich & \href{}{FriedrichFMRSST14}~\cite{FriedrichFMRSST14}\\
Sara Frimodig & \href{works/FrimodigS19.pdf}{FrimodigS19}~\cite{FrimodigS19}\\
Nikolaus Frohner & \href{works/FrohnerTR19.pdf}{FrohnerTR19}~\cite{FrohnerTR19}\\
Daniel Frost & \href{works/FrostD98.pdf}{FrostD98}~\cite{FrostD98}\\
Melanie Fr{\"{u}}hst{\"{u}}ck & \href{}{FriedrichFMRSST14}~\cite{FriedrichFMRSST14}\\
Jun Fu & \href{works/LiFJZLL22.pdf}{LiFJZLL22}~\cite{LiFJZLL22}\\
Etienne Fux & \href{works/KoehlerBFFHPSSS21.pdf}{KoehlerBFFHPSSS21}~\cite{KoehlerBFFHPSSS21}\\
Ernesto G. Birgin & \href{works/LunardiBLRV20.pdf}{LunardiBLRV20}~\cite{LunardiBLRV20}\\
Mohamed Gaha & \href{works/PopovicCGNC22.pdf}{PopovicCGNC22}~\cite{PopovicCGNC22}\\
Flavius Galiber III & \href{works/PembertonG98.pdf}{PembertonG98}~\cite{PembertonG98}\\
Cristian Galleguillos & \href{works/GalleguillosKSB19.pdf}{GalleguillosKSB19}~\cite{GalleguillosKSB19}\\
Xavier Gandibleux & \href{works/RodriguezDG02.pdf}{RodriguezDG02}~\cite{RodriguezDG02}\\
Graeme Gange & \href{works/He0GLW18.pdf}{He0GLW18}~\cite{He0GLW18}\\
Thierry Garaix & \href{works/BourreauGGLT22.pdf}{BourreauGGLT22}~\cite{BourreauGGLT22}\\
Antoine Gargani & \href{works/GarganiR07.pdf}{GarganiR07}~\cite{GarganiR07}\\
Jonathan Gaudreault & \href{works/Mercier-AubinGQ20.pdf}{Mercier-AubinGQ20}~\cite{Mercier-AubinGQ20}\\
Ridvan Gedik & \href{works/GedikKEK18.pdf}{GedikKEK18}~\cite{GedikKEK18}\\
Marc Geitz & \href{works/GeitzGSSW22.pdf}{GeitzGSSW22}~\cite{GeitzGSSW22}\\
Mirco Gelain & \href{works/GelainPRVW17.pdf}{GelainPRVW17}~\cite{GelainPRVW17}\\
Michel Gendreau & \href{works/HachemiGR11.pdf}{HachemiGR11}~\cite{HachemiGR11}\\
Marcus Gerhard M{\"{u}}ller & \href{works/MullerMKP22.pdf}{MullerMKP22}~\cite{MullerMKP22}\\
Patrick Gerhards & \href{works/HubnerGSV21.pdf}{HubnerGSV21}~\cite{HubnerGSV21}\\
Ulrich Geske & \href{works/Geske05.pdf}{Geske05}~\cite{Geske05}\\
Katherine Giles & \href{works/GilesH16.pdf}{GilesH16}~\cite{GilesH16}\\
Ga{\"{e}}l Glorian & \href{works/PerezGSL23.pdf}{PerezGSL23}~\cite{PerezGSL23}\\
Gael Glorian & \href{works/abs-2312-13682.pdf}{abs-2312-13682}~\cite{abs-2312-13682}\\
Daniel Godard & \href{works/GodardLN05.pdf}{GodardLN05}~\cite{GodardLN05}\\
Vikas Goel & \href{works/GoelSHFS15.pdf}{GoelSHFS15}~\cite{GoelSHFS15}\\
Mark Goh & \href{works/FanXG21.pdf}{FanXG21}~\cite{FanXG21}\\
Adrian Goldwaser & \href{works/GoldwaserS17.pdf}{GoldwaserS17}~\cite{GoldwaserS17}\\
Hans{-}Joachim Goltz & \href{works/Goltz95.pdf}{Goltz95}~\cite{Goltz95}\\
Matthieu Gondran & \href{works/BourreauGGLT22.pdf}{BourreauGGLT22}~\cite{BourreauGGLT22}\\
Cristian Grozea & \href{works/GeitzGSSW22.pdf}{GeitzGSSW22}~\cite{GeitzGSSW22}\\
Flavius Gruian & \href{works/GruianK98.pdf}{GruianK98}~\cite{GruianK98}\\
Hanyu Gu & \href{works/GuSS13.pdf}{GuSS13}~\cite{GuSS13}\\
Alessio Guerri & \href{works/BeniniBGM06.pdf}{BeniniBGM06}~\cite{BeniniBGM06}\\
Serigne Gueye & \href{works/Acuna-AgostMFG09.pdf}{Acuna-AgostMFG09}~\cite{Acuna-AgostMFG09}\\
Ying Guo & \href{works/ZhouGL15.pdf}{ZhouGL15}~\cite{ZhouGL15}\\
Şeyda G{\"u}r & \href{works/GurEA19.pdf}{GurEA19}~\cite{GurEA19}\\
Burak G{\"{o}}kg{\"{u}}r & \href{works/GokgurHO18.pdf}{GokgurHO18}~\cite{GokgurHO18}\\
Seyda G{\"{u}}r & \href{works/GurPAE23.pdf}{GurPAE23}~\cite{GurPAE23}\\
Fehmi H'Mida & \href{works/TrojetHL11.pdf}{TrojetHL11}~\cite{TrojetHL11}\\
Rolf H. M{\"{o}}hring & \href{works/BertholdHLMS10.pdf}{BertholdHLMS10}~\cite{BertholdHLMS10}\\
John H. Drake & \href{works/PourDERB18.pdf}{PourDERB18}~\cite{PourDERB18}\\
M. H. Fazel Zarandi & \href{works/ZarandiKS16.pdf}{ZarandiKS16}~\cite{ZarandiKS16}\\
Klaus H. Ecker & \href{}{BlazewiczEP19}~\cite{BlazewiczEP19}\\
Emile H. L. Aarts & \href{works/NuijtenA94.pdf}{NuijtenA94}~\cite{NuijtenA94}\\
Claire Hanen & \href{works/HanenKP21.pdf}{HanenKP21}~\cite{HanenKP21}\\
Jiang Hang Chen & \href{works/QinDCS20.pdf}{QinDCS20}~\cite{QinDCS20}\\
Sue Hanhilammi & \href{works/KrogtLPHJ07.pdf}{KrogtLPHJ07}~\cite{KrogtLPHJ07}\\
Mohamed Haouari & \href{works/LahimerLH11.pdf}{LahimerLH11}~\cite{LahimerLH11}\\
Fazirulhisyam Hashim & \href{works/AkramNHRSA23.pdf}{AkramNHRSA23}~\cite{AkramNHRSA23}\\
Shan He & \href{works/He0GLW18.pdf}{He0GLW18}~\cite{He0GLW18}\\
Susanne Heipcke & \href{works/HeipckeCCS00.pdf}{HeipckeCCS00}~\cite{HeipckeCCS00}\\
Fabien Hermenier & \href{works/HermenierDL11.pdf}{HermenierDL11}~\cite{HermenierDL11}\\
Gerhard Hiermann & \href{works/RendlPHPR12.pdf}{RendlPHPR12}~\cite{RendlPHPR12}\\
Alessandro Hill & \href{works/HillTV21.pdf}{HillTV21}~\cite{HillTV21}\\
Te{-}Wei Ho & \href{works/HoYCLLCLC18.pdf}{HoYCLLCLC18}~\cite{HoYCLLCLC18}\\
Petra Hofstedt & \href{works/LiuLH19.pdf}{LiuLH19}~\cite{LiuLH19}\\
John Hou & \href{works/DavenportKRSH07.pdf}{DavenportKRSH07}~\cite{DavenportKRSH07}\\
Marie{-}Jos{\'{e}} Huguet & \href{works/AntuoriHHEN21.pdf}{AntuoriHHEN21}~\cite{AntuoriHHEN21}\\
Felix H{\"{u}}bner & \href{works/HubnerGSV21.pdf}{HubnerGSV21}~\cite{HubnerGSV21}\\
Amar Isli & \href{works/BelhadjiI98.pdf}{BelhadjiI98}~\cite{BelhadjiI98}\\
Mustafa Ismael Salman & \href{works/AkramNHRSA23.pdf}{AkramNHRSA23}~\cite{AkramNHRSA23}\\
Fernando J. M. Marcellino & \href{works/SerraNM12.pdf}{SerraNM12}~\cite{SerraNM12}\\
Leon J. Osterweil & \href{works/ShinBBHO18.pdf}{ShinBBHO18}~\cite{ShinBBHO18}\\
H. J. Kim & \href{works/SureshMOK06.pdf}{SureshMOK06}~\cite{SureshMOK06}\\
John J. Kanet & \href{}{KanetAG04}~\cite{KanetAG04}\\
Colin J. Layfield & \href{}{Layfield02}~\cite{Layfield02}\\
Andrew J. Mason & \href{works/Mason01.pdf}{Mason01}~\cite{Mason01}\\
Jean Jaubert & \href{works/PraletLJ15.pdf}{PraletLJ15}~\cite{PraletLJ15}\\
Jan Jel{\'{\i}}nek & \href{works/JelinekB16.pdf}{JelinekB16}~\cite{JelinekB16}\\
Yingjun Ji & \href{works/ZhangJZL22.pdf}{ZhangJZL22}~\cite{ZhangJZL22}\\
Zixi Jia & \href{works/LiFJZLL22.pdf}{LiFJZLL22}~\cite{LiFJZLL22}\\
Yunfei Jiang & \href{works/LiuJ06.pdf}{LiuJ06}~\cite{LiuJ06}\\
Yue Jin & \href{works/KrogtLPHJ07.pdf}{KrogtLPHJ07}~\cite{KrogtLPHJ07}\\
Marc Joliveau & \href{works/ChapadosJR11.pdf}{ChapadosJR11}~\cite{ChapadosJR11}\\
Peter Jonsson & \href{works/AngelsmarkJ00.pdf}{AngelsmarkJ00}~\cite{AngelsmarkJ00}\\
Jean Jourdan & \href{}{JourdanFRD94}~\cite{JourdanFRD94}\\
Jae{-}Yoon Jung & \href{works/ParkUJR19.pdf}{ParkUJR19}~\cite{ParkUJR19}\\
Pascal Jungblut & \href{works/JungblutK22.pdf}{JungblutK22}~\cite{JungblutK22}\\
T. K. Satish Kumar & \href{works/Kumar03.pdf}{Kumar03}~\cite{Kumar03}\\
Edmund K. Burke & \href{works/PourDERB18.pdf}{PourDERB18}~\cite{PourDERB18}\\
T. K. Feng & \href{works/BeckFW11.pdf}{BeckFW11}~\cite{BeckFW11}\\
Jayant Kalagnanam & \href{works/DavenportKRSH07.pdf}{DavenportKRSH07}~\cite{DavenportKRSH07}\\
Darshan Kalathia & \href{works/GedikKEK18.pdf}{GedikKEK18}~\cite{GedikKEK18}\\
Olli Kamarainen & \href{works/KamarainenS02.pdf}{KamarainenS02}~\cite{KamarainenS02}\\
Nor Kamariah Noordin & \href{works/AkramNHRSA23.pdf}{AkramNHRSA23}~\cite{AkramNHRSA23}\\
Czerniachowska, Kateryna & \href{works/CzerniachowskaWZ23.pdf}{CzerniachowskaWZ23}~\cite{CzerniachowskaWZ23}\\
Elena Kelareva & \href{works/KelarevaTK13.pdf}{KelarevaTK13}~\cite{KelarevaTK13}\\
Jan Kelbel & \href{works/KelbelH11.pdf}{KelbelH11}~\cite{KelbelH11}\\
H. Khorshidian & \href{works/ZarandiKS16.pdf}{ZarandiKS16}~\cite{ZarandiKS16}\\
Kamran Kianfar & \href{works/YounespourAKE19.pdf}{YounespourAKE19}~\cite{YounespourAKE19}\\
Philip Kilby & \href{works/KelarevaTK13.pdf}{KelarevaTK13}~\cite{KelarevaTK13}\\
Dongyun Kim & \href{works/KimCMLLP23.pdf}{KimCMLLP23}~\cite{KimCMLLP23}\\
Emre Kirac & \href{works/GedikKEK18.pdf}{GedikKEK18}~\cite{GedikKEK18}\\
Zeynep Kiziltan & \href{works/GalleguillosKSB19.pdf}{GalleguillosKSB19}~\cite{GalleguillosKSB19}\\
Christian Klanke & \href{works/KlankeBYE21.pdf}{KlankeBYE21}~\cite{KlankeBYE21}\\
Jana Koehler & \href{works/KoehlerBFFHPSSS21.pdf}{KoehlerBFFHPSSS21}~\cite{KoehlerBFFHPSSS21}\\
Wolfgang Kohlenbrein & \href{works/KovacsTKSG21.pdf}{KovacsTKSG21}~\cite{KovacsTKSG21}\\
Rainer Kolisch & \href{works/PohlAK22.pdf}{PohlAK22}~\cite{PohlAK22}\\
Sebastian Kosch & \href{works/KoschB14.pdf}{KoschB14}~\cite{KoschB14}\\
Benjamin Kov{\'{a}}cs & \href{works/KovacsTKSG21.pdf}{KovacsTKSG21}~\cite{KovacsTKSG21}\\
Matthias Krainz & \href{works/GeibingerKKMMW21.pdf}{GeibingerKKMMW21}~\cite{GeibingerKKMMW21}\\
Andreas Krall & \href{works/ErtlK91.pdf}{ErtlK91}~\cite{ErtlK91}\\
Dieter Kranzlm{\"{u}}ller & \href{works/JungblutK22.pdf}{JungblutK22}~\cite{JungblutK22}\\
Dominik Kress & \href{works/MullerMKP22.pdf}{MullerMKP22}~\cite{MullerMKP22}\\
Per Kreuger & \href{works/AronssonBK09.pdf}{AronssonBK09}~\cite{AronssonBK09}\\
Żywicki, Krzysztof & \href{works/CzerniachowskaWZ23.pdf}{CzerniachowskaWZ23}~\cite{CzerniachowskaWZ23}\\
Wen{-}Yang Ku & \href{works/HeinzKB13.pdf}{HeinzKB13}~\cite{HeinzKB13}\\
Mustafa K{\"u}ç{\"u}k & \href{works/KucukY19.pdf}{KucukY19}~\cite{KucukY19}\\
Elif K{\"{u}}rkl{\"{u}} & \href{works/FrankK05.pdf}{FrankK05}~\cite{FrankK05}\\
Andr{\'{a}}s K{\'{e}}ri & \href{works/KeriK07.pdf}{KeriK07}~\cite{KeriK07}\\
Michael L. Pinedo & \href{works/KimCMLLP23.pdf}{KimCMLLP23}~\cite{KimCMLLP23}\\
Hassan L. Hijazi & \href{works/LimHTB16.pdf}{LimHTB16}~\cite{LimHTB16}\\
Philip L. Henneman & \href{works/ShinBBHO18.pdf}{ShinBBHO18}~\cite{ShinBBHO18}\\
Yiqing L. Luo & \href{works/LuoB22.pdf}{LuoB22}~\cite{LuoB22}\\
Philippe Lacomme & \href{works/BourreauGGLT22.pdf}{BourreauGGLT22}~\cite{BourreauGGLT22}\\
Daniel Lafond & \href{works/BoudreaultSLQ22.pdf}{BoudreaultSLQ22}~\cite{BoudreaultSLQ22}\\
Asma Lahimer & \href{works/LahimerLH11.pdf}{LahimerLH11}~\cite{LahimerLH11}\\
Feipei Lai & \href{works/HoYCLLCLC18.pdf}{HoYCLLCLC18}~\cite{HoYCLLCLC18}\\
Jui{-}Fen Lai & \href{works/HoYCLLCLC18.pdf}{HoYCLLCLC18}~\cite{HoYCLLCLC18}\\
Andr{\'{e}} Langevin & \href{works/KhayatLR06.pdf}{KhayatLR06}~\cite{KhayatLR06}\\
Christophe Lecoutre & \href{works/GayHLS15.pdf}{GayHLS15}~\cite{GayHLS15}\\
Myungho Lee & \href{works/KimCMLLP23.pdf}{KimCMLLP23}~\cite{KimCMLLP23}\\
Kangbok Lee & \href{works/KimCMLLP23.pdf}{KimCMLLP23}~\cite{KimCMLLP23}\\
Solange Lemai{-}Chenevier & \href{works/PraletLJ15.pdf}{PraletLJ15}~\cite{PraletLJ15}\\
Xingyang Li & \href{works/LiFJZLL22.pdf}{LiFJZLL22}~\cite{LiFJZLL22}\\
Siyi Li & \href{works/LiFJZLL22.pdf}{LiFJZLL22}~\cite{LiFJZLL22}\\
Xiaodong Li & \href{works/abs-2211-14492.pdf}{abs-2211-14492}~\cite{abs-2211-14492}\\
Guipeng Li & \href{works/ZhouGL15.pdf}{ZhouGL15}~\cite{ZhouGL15}\\
Hong Li & \href{works/SunLYL10.pdf}{SunLYL10}~\cite{SunLYL10}\\
Nan Li & \href{works/SunLYL10.pdf}{SunLYL10}~\cite{SunLYL10}\\
Yunbo Li & \href{works/Madi-WambaLOBM17.pdf}{Madi-WambaLOBM17}~\cite{Madi-WambaLOBM17}\\
Wan{-}Chung Liao & \href{works/HoYCLLCLC18.pdf}{HoYCLLCLC18}~\cite{HoYCLLCLC18}\\
Ariel Liebman & \href{works/He0GLW18.pdf}{He0GLW18}~\cite{He0GLW18}\\
Olivier Liess & \href{works/LiessM08.pdf}{LiessM08}~\cite{LiessM08}\\
Andrew Lim & \href{works/LimRX04.pdf}{LimRX04}~\cite{LimRX04}\\
Nir Lipovetzky & \href{works/BurtLPS15.pdf}{BurtLPS15}~\cite{BurtLPS15}\\
Tong Liu & \href{works/LiuCGM17.pdf}{LiuCGM17}~\cite{LiuCGM17}\\
Lingxuan Liu & \href{works/QinWSLS21.pdf}{QinWSLS21}~\cite{QinWSLS21}\\
Ke Liu & \href{works/LiuLH19.pdf}{LiuLH19}~\cite{LiuLH19}\\
Rengkui Liu & \href{works/TangLWSK18.pdf}{TangLWSK18}~\cite{TangLWSK18}\\
Yuechang Liu & \href{works/LiuJ06.pdf}{LiuJ06}~\cite{LiuJ06}\\
Doina Logofatu & \href{works/BadicaBIL19.pdf}{BadicaBIL19}~\cite{BadicaBIL19}\\
Thomas Lorigeon & \href{works/LorigeonBB02.pdf}{LorigeonBB02}~\cite{LorigeonBB02}\\
Chang Lv & \href{works/MengZRZL20.pdf}{MengZRZL20}~\cite{MengZRZL20}\\
Zhimin Lv & \href{works/ZhangLS12.pdf}{ZhangLS12}~\cite{ZhangLS12}\\
Sven L{\"{o}}ffler & \href{works/LiuLH19.pdf}{LiuLH19}~\cite{LiuLH19}\\
J. M. van den Akker & \href{works/AkkerDH07.pdf}{AkkerDH07}~\cite{AkkerDH07}\\
Abdulrahman M. Abdulghani & \href{works/AkramNHRSA23.pdf}{AkramNHRSA23}~\cite{AkramNHRSA23}\\
O. M. Alade & \href{works/abs-1902-01193.pdf}{abs-1902-01193}~\cite{abs-1902-01193}\\
Shahrzad M. Pour & \href{works/PourDERB18.pdf}{PourDERB18}~\cite{PourDERB18}\\
Franco M. Novara & \href{works/NovaraNH16.pdf}{NovaraNH16}~\cite{NovaraNH16}\\
Rafael M. Gasca & \href{works/ValleMGT03.pdf}{ValleMGT03}~\cite{ValleMGT03}\\
Jun Ma & \href{works/MakMS10.pdf}{MakMS10}~\cite{MakMS10}\\
Amy Mainville Cohn & \href{works/BarlattCG08.pdf}{BarlattCG08}~\cite{BarlattCG08}\\
Kai{-}Ling Mak & \href{works/MakMS10.pdf}{MakMS10}~\cite{MakMS10}\\
V. Mani & \href{works/SureshMOK06.pdf}{SureshMOK06}~\cite{SureshMOK06}\\
Oscar Manzano & \href{works/MurphyMB15.pdf}{MurphyMB15}~\cite{MurphyMB15}\\
Kourosh Marjani Rasmussen & \href{works/PourDERB18.pdf}{PourDERB18}~\cite{PourDERB18}\\
Kim Marriott & \href{works/FalaschiGMP97.pdf}{FalaschiGMP97}~\cite{FalaschiGMP97}\\
Fae Martin & \href{works/MartinPY01.pdf}{MartinPY01}~\cite{MartinPY01}\\
Jacopo Mauro & \href{works/LiuCGM17.pdf}{LiuCGM17}~\cite{LiuCGM17}\\
Jim McInnes & \href{works/MalikMB08.pdf}{MalikMB08}~\cite{MalikMB08}\\
Zahra Mehdizadeh{-}Somarin & \href{works/Mehdizadeh-Somarin23.pdf}{Mehdizadeh-Somarin23}~\cite{Mehdizadeh-Somarin23}\\
Haci Mehmet Alakas & \href{works/GurPAE23.pdf}{GurPAE23}~\cite{GurPAE23}\\
Hacı Mehmet Alakaş & \href{works/GurEA19.pdf}{GurEA19}~\cite{GurEA19}\\
Sebastian Meiswinkel & \href{works/WinterMMW22.pdf}{WinterMMW22}~\cite{WinterMMW22}\\
Gonzalo Mej{\'i}a & \href{works/YuraszeckMPV22.pdf}{YuraszeckMPV22}~\cite{YuraszeckMPV22}\\
Hein Meling & \href{works/MossigeGSMC17.pdf}{MossigeGSMC17}~\cite{MossigeGSMC17}\\
Julien Menana & \href{}{Menana11}~\cite{Menana11}\\
Jean{-}Marc Menaud & \href{works/Madi-WambaLOBM17.pdf}{Madi-WambaLOBM17}~\cite{Madi-WambaLOBM17}\\
Leilei Meng & \href{works/MengZRZL20.pdf}{MengZRZL20}~\cite{MengZRZL20}\\
Alexandre Mercier{-}Aubin & \href{works/Mercier-AubinGQ20.pdf}{Mercier-AubinGQ20}~\cite{Mercier-AubinGQ20}\\
Vera Mersheeva & \href{}{FriedrichFMRSST14}~\cite{FriedrichFMRSST14}\\
Nadine Meskens & \href{works/WangMD15.pdf}{WangMD15}~\cite{WangMD15}\\
Bernd Meyer & \href{works/ThiruvadyBME09.pdf}{ThiruvadyBME09}~\cite{ThiruvadyBME09}\\
Kyung Min Kim & \href{works/HamPK21.pdf}{HamPK21}~\cite{HamPK21}\\
Gautam Mitra & \href{works/Darby-DowmanLMZ97.pdf}{Darby-DowmanLMZ97}~\cite{Darby-DowmanLMZ97}\\
Elizabeth Montero & \href{works/YuraszeckMCCR23.pdf}{YuraszeckMCCR23}~\cite{YuraszeckMCCR23}\\
Kyungduk Moon & \href{works/KimCMLLP23.pdf}{KimCMLLP23}~\cite{KimCMLLP23}\\
Morten Mossige & \href{works/MossigeGSMC17.pdf}{MossigeGSMC17}~\cite{MossigeGSMC17}\\
Alix Munier Kordon & \href{works/HanenKP21.pdf}{HanenKP21}~\cite{HanenKP21}\\
Stanislav Mur{\'{\i}}n & \href{works/MurinR19.pdf}{MurinR19}~\cite{MurinR19}\\
Nicola Muscettola & \href{works/Muscettola02.pdf}{Muscettola02}~\cite{Muscettola02}\\
David M{\"{u}}ller & \href{works/MullerMKP22.pdf}{MullerMKP22}~\cite{MullerMKP22}\\
Andr{\'{a}}s M{\'{a}}rkus & \href{works/VanczaM01.pdf}{VanczaM01}~\cite{VanczaM01}\\
Marc{-}Andr{\'{e}} M{\'{e}}nard & \href{works/BessiereHMQW14.pdf}{BessiereHMQW14}~\cite{BessiereHMQW14}\\
Christina N. Burt & \href{works/BurtLPS15.pdf}{BurtLPS15}~\cite{BurtLPS15}\\
T. N. Wong & \href{works/ZhangYW21.pdf}{ZhangYW21}~\cite{ZhangYW21}\\
Sophie N. Parragh & \href{works/abs-1902-09244.pdf}{abs-1902-09244}~\cite{abs-1902-09244}\\
S. N. Omkar & \href{works/SureshMOK06.pdf}{SureshMOK06}~\cite{SureshMOK06}\\
Goldie Nejat & \href{works/BoothNB16.pdf}{BoothNB16}~\cite{BoothNB16}\\
Shiva Nejati & \href{works/AlesioNBG14.pdf}{AlesioNBG14}~\cite{AlesioNBG14}\\
Franklin Nguewouo & \href{works/PopovicCGNC22.pdf}{PopovicCGNC22}~\cite{PopovicCGNC22}\\
Alain Nguyen & \href{works/AntuoriHHEN21.pdf}{AntuoriHHEN21}~\cite{AntuoriHHEN21}\\
Gilberto Nishioka & \href{works/SerraNM12.pdf}{SerraNM12}~\cite{SerraNM12}\\
Thierry Noulamo & \href{works/KameugneFND23.pdf}{KameugneFND23}~\cite{KameugneFND23}\\
Jari Nurmi & \href{works/QuSN06.pdf}{QuSN06}~\cite{QuSN06}\\
A. O. Amusat & \href{works/abs-1902-01193.pdf}{abs-1902-01193}~\cite{abs-1902-01193}\\
Ceyda Oguz & \href{works/EdisO11.pdf}{EdisO11}~\cite{EdisO11}\\
Bilal Omar Akram & \href{works/AkramNHRSA23.pdf}{AkramNHRSA23}~\cite{AkramNHRSA23}\\
Mirza Omer Beg & \href{works/BegB13.pdf}{BegB13}~\cite{BegB13}\\
Anne{-}C{\'{e}}cile Orgerie & \href{works/Madi-WambaLOBM17.pdf}{Madi-WambaLOBM17}~\cite{Madi-WambaLOBM17}\\
Mohand Ou Idir Khemmoudj & \href{works/KhemmoudjPB06.pdf}{KhemmoudjPB06}~\cite{KhemmoudjPB06}\\
Pierre Ouellet & \href{works/OuelletQ13.pdf}{OuelletQ13}~\cite{OuelletQ13}\\
Soukaina Oujana & \href{works/OujanaAYB22.pdf}{OujanaAYB22}~\cite{OujanaAYB22}\\
Asma Ouled Bedhief & \href{works/Bedhief21.pdf}{Bedhief21}~\cite{Bedhief21}\\
Irem Ozkarahan & \href{works/TopalogluO11.pdf}{TopalogluO11}~\cite{TopalogluO11}\\
D{\'{e}}bora P. Ronconi & \href{works/LunardiBLRV20.pdf}{LunardiBLRV20}~\cite{LunardiBLRV20}\\
Edward P. K. Tsang & \href{works/Tsang03.pdf}{Tsang03}~\cite{Tsang03}\\
W. P. M. Nuijten & \href{works/NuijtenA94.pdf}{NuijtenA94}~\cite{NuijtenA94}\\
Miquel Palah{\'{\i}} & \href{works/BofillEGPSV14.pdf}{BofillEGPSV14}~\cite{BofillEGPSV14}\\
Catuscia Palamidessi & \href{works/FalaschiGMP97.pdf}{FalaschiGMP97}~\cite{FalaschiGMP97}\\
Pere Pal{\`{a}}{-}Sch{\"{o}}nw{\"{a}}lder & \href{works/EscobetPQPRA19.pdf}{EscobetPQPRA19}~\cite{EscobetPQPRA19}\\
Vaibhav Pandey & \href{works/PandeyS21a.pdf}{PandeyS21a}~\cite{PandeyS21a}\\
Hoonseok Park & \href{works/ParkUJR19.pdf}{ParkUJR19}~\cite{ParkUJR19}\\
Myoung-Ju Park & \href{works/HamPK21.pdf}{HamPK21}~\cite{HamPK21}\\
Erica Pastore & \href{works/AlfieriGPS23.pdf}{AlfieriGPS23}~\cite{AlfieriGPS23}\\
Theo Pedersen & \href{works/HanenKP21.pdf}{HanenKP21}~\cite{HanenKP21}\\
Bart Peintner & \href{works/MoffittPP05.pdf}{MoffittPP05}~\cite{MoffittPP05}\\
Jordi Pereira & \href{works/YuraszeckMPV22.pdf}{YuraszeckMPV22}~\cite{YuraszeckMPV22}\\
Laurent Perron & \href{works/DannaP03.pdf}{DannaP03}~\cite{DannaP03}\\
Mehmet Pinarbasi & \href{works/GurPAE23.pdf}{GurPAE23}~\cite{GurPAE23}\\
Arthur Pinkney & \href{works/MartinPY01.pdf}{MartinPY01}~\cite{MartinPY01}\\
David Pisinger & \href{works/SacramentoSP20.pdf}{SacramentoSP20}~\cite{SacramentoSP20}\\
Maximilian Pohl & \href{works/PohlAK22.pdf}{PohlAK22}~\cite{PohlAK22}\\
Nicola Policella & \href{works/OddiPCC03.pdf}{OddiPCC03}~\cite{OddiPCC03}\\
Oliver Polo{-}Mej{\'{\i}}a & \href{works/Polo-MejiaALB20.pdf}{Polo-MejiaALB20}~\cite{Polo-MejiaALB20}\\
Paul Pop & \href{works/BarzegaranZP20.pdf}{BarzegaranZP20}~\cite{BarzegaranZP20}\\
Louis Popovic & \href{works/PopovicCGNC22.pdf}{PopovicCGNC22}~\cite{PopovicCGNC22}\\
Marc Porcheron & \href{works/KhemmoudjPB06.pdf}{KhemmoudjPB06}~\cite{KhemmoudjPB06}\\
Marc Pouly & \href{works/KoehlerBFFHPSSS21.pdf}{KoehlerBFFHPSSS21}~\cite{KoehlerBFFHPSSS21}\\
Guillaume Pov{\'{e}}da & \href{works/PovedaAA23.pdf}{PovedaAA23}~\cite{PovedaAA23}\\
Matthias Prandtstetter & \href{works/RendlPHPR12.pdf}{RendlPHPR12}~\cite{RendlPHPR12}\\
Jakob Puchinger & \href{works/RendlPHPR12.pdf}{RendlPHPR12}~\cite{RendlPHPR12}\\
Jean{-}Francois Puget & \href{works/Puget95.pdf}{Puget95}~\cite{Puget95}\\
Vicen{\c{c}} Puig & \href{works/EscobetPQPRA19.pdf}{EscobetPQPRA19}~\cite{EscobetPQPRA19}\\
Kenneth Pulliam & \href{works/KrogtLPHJ07.pdf}{KrogtLPHJ07}~\cite{KrogtLPHJ07}\\
Kenny Qili Zhu & \href{works/ZhuS02.pdf}{ZhuS02}~\cite{ZhuS02}\\
Ming Qin & \href{works/QinWSLS21.pdf}{QinWSLS21}~\cite{QinWSLS21}\\
Tianbao Qin & \href{works/QinDCS20.pdf}{QinDCS20}~\cite{QinDCS20}\\
Yang Qu & \href{works/QuSN06.pdf}{QuSN06}~\cite{QuSN06}\\
Yuchen Quan & \href{}{ShiYXQ22}~\cite{ShiYXQ22}\\
Joseba Quevedo & \href{works/EscobetPQPRA19.pdf}{EscobetPQPRA19}~\cite{EscobetPQPRA19}\\
Dominik R. Bleidorn & \href{works/KlankeBYE21.pdf}{KlankeBYE21}~\cite{KlankeBYE21}\\
Aliza R. Heching & \href{works/HechingH16.pdf}{HechingH16}~\cite{HechingH16}\\
Adrian R. Pearce & \href{works/BurtLPS15.pdf}{BurtLPS15}~\cite{BurtLPS15}\\
Levi R. Abreu & \href{works/PrataAN23.pdf}{PrataAN23}~\cite{PrataAN23}\\
Wichniarek, Radosław & \href{works/CzerniachowskaWZ23.pdf}{CzerniachowskaWZ23}~\cite{CzerniachowskaWZ23}\\
Sebastian Raggl & \href{works/abs-1902-09244.pdf}{abs-1902-09244}~\cite{abs-1902-09244}\\
Vinas{\'{e}}tan Ratheil Houndji & \href{works/HoundjiSWD14.pdf}{HoundjiSWD14}~\cite{HoundjiSWD14}\\
Chandra Reddy & \href{works/DavenportKRSH07.pdf}{DavenportKRSH07}~\cite{DavenportKRSH07}\\
Philippe Refalo & \href{works/GarganiR07.pdf}{GarganiR07}~\cite{GarganiR07}\\
Yaping Ren & \href{works/MengZRZL20.pdf}{MengZRZL20}~\cite{MengZRZL20}\\
Andrea Rendl & \href{works/RendlPHPR12.pdf}{RendlPHPR12}~\cite{RendlPHPR12}\\
Hamid Reza Feyzmahdavian & \href{works/Astrand0F21.pdf}{Astrand0F21}~\cite{Astrand0F21}\\
Vahid Riahi & \href{works/RiahiNS018.pdf}{RiahiNS018}~\cite{RiahiNS018}\\
Diane Riopel & \href{works/KhayatLR06.pdf}{KhayatLR06}~\cite{KhayatLR06}\\
Gregory Rix & \href{works/PesantRR15.pdf}{PesantRR15}~\cite{PesantRR15}\\
Robert Rodosek & \href{works/RodosekW98.pdf}{RodosekW98}~\cite{RodosekW98}\\
Brian Rodrigues & \href{works/LimRX04.pdf}{LimRX04}~\cite{LimRX04}\\
Joaquín Rodriguez & \href{works/Rodriguez07.pdf}{Rodriguez07}~\cite{Rodriguez07}\\
Joaquin Rodriguez & \href{works/RodriguezDG02.pdf}{RodriguezDG02}~\cite{RodriguezDG02}\\
Jerome Rogerie & \href{works/LaborieRSV18.pdf}{LaborieRSV18}~\cite{LaborieRSV18}\\
Mohammad Rohaninejad & \href{works/Mehdizadeh-Somarin23.pdf}{Mehdizadeh-Somarin23}~\cite{Mehdizadeh-Somarin23}\\
Maximiliano Rojel & \href{works/YuraszeckMCCR23.pdf}{YuraszeckMCCR23}~\cite{YuraszeckMCCR23}\\
Juli Romera & \href{works/EscobetPQPRA19.pdf}{EscobetPQPRA19}~\cite{EscobetPQPRA19}\\
Roberto Rossi & \href{works/RossiTHP07.pdf}{RossiTHP07}~\cite{RossiTHP07}\\
Fran{\c{c}}ois Roubellat & \href{works/ArtiguesR00.pdf}{ArtiguesR00}~\cite{ArtiguesR00}\\
St{\'{e}}phanie Roussel & \href{works/SquillaciPR23.pdf}{SquillaciPR23}~\cite{SquillaciPR23}\\
Didier Rozzonelli & \href{}{JourdanFRD94}~\cite{JourdanFRD94}\\
Ruiz, Rub\'{e}n & \href{works/NaderiRR23.pdf}{NaderiRR23}~\cite{NaderiRR23}\\
Hana Rudov{\'{a}} & \href{works/MurinR19.pdf}{MurinR19}~\cite{MurinR19}\\
Martin Ruskowski & \href{works/ParkUJR19.pdf}{ParkUJR19}~\cite{ParkUJR19}\\
Anna Ryabokon & \href{}{FriedrichFMRSST14}~\cite{FriedrichFMRSST14}\\
William S. Havens & \href{works/DilkinaDH05.pdf}{DilkinaDH05}~\cite{DilkinaDH05}\\
Mark S. Fox & \href{works/BeckDF97.pdf}{BeckDF97}~\cite{BeckDF97}\\
Marcelo S. Nagano & \href{works/PrataAN23.pdf}{PrataAN23}~\cite{PrataAN23}\\
Mohamed S. Gheith & \href{works/AbohashimaEG21.pdf}{AbohashimaEG21}~\cite{AbohashimaEG21}\\
David Sacramento & \href{works/SacramentoSP20.pdf}{SacramentoSP20}~\cite{SacramentoSP20}\\
Shahram Saeidi & \href{}{AlizdehS20}~\cite{AlizdehS20}\\
Poonam Saini & \href{works/PandeyS21a.pdf}{PandeyS21a}~\cite{PandeyS21a}\\
Fabio Salassa & \href{works/AlfieriGPS23.pdf}{AlfieriGPS23}~\cite{AlfieriGPS23}\\
Sophia Saller & \href{works/KoehlerBFFHPSSS21.pdf}{KoehlerBFFHPSSS21}~\cite{KoehlerBFFHPSSS21}\\
Anastasia Salyaeva & \href{works/KoehlerBFFHPSSS21.pdf}{KoehlerBFFHPSSS21}~\cite{KoehlerBFFHPSSS21}\\
Maria Sander & \href{}{FriedrichFMRSST14}~\cite{FriedrichFMRSST14}\\
Eric Sanlaville & \href{works/PoderBS04.pdf}{PoderBS04}~\cite{PoderBS04}\\
{\'{O}}scar Sapena & \href{works/GarridoOS08.pdf}{GarridoOS08}~\cite{GarridoOS08}\\
{\"{O}}zge Satir Akpunar & \href{works/IsikYA23.pdf}{IsikYA23}~\cite{IsikYA23}\\
Abdul Sattar & \href{works/RiahiNS018.pdf}{RiahiNS018}~\cite{RiahiNS018}\\
Peter Scheiblechner & \href{works/KoehlerBFFHPSSS21.pdf}{KoehlerBFFHPSSS21}~\cite{KoehlerBFFHPSSS21}\\
Klaus Schild & \href{works/SchildW00.pdf}{SchildW00}~\cite{SchildW00}\\
Thomas Schlechte & \href{works/HeinzSSW12.pdf}{HeinzSSW12}~\cite{HeinzSSW12}\\
Thorsten Schmidt & \href{works/BenderWS21.pdf}{BenderWS21}~\cite{BenderWS21}\\
Günter Schmidt & \href{}{BlazewiczEP19}~\cite{BlazewiczEP19}\\
Gunnar Schrader & \href{works/WolfS05.pdf}{WolfS05}~\cite{WolfS05}\\
Philipp Schrott{-}Kostwein & \href{works/KovacsTKSG21.pdf}{KovacsTKSG21}~\cite{KovacsTKSG21}\\
Uwe Schwiegelshohn & \href{works/LimtanyakulS12.pdf}{LimtanyakulS12}~\cite{LimtanyakulS12}\\
Lena Secher Ejlertsen & \href{works/PourDERB18.pdf}{PourDERB18}~\cite{PourDERB18}\\
Thiago Serra & \href{works/SerraNM12.pdf}{SerraNM12}~\cite{SerraNM12}\\
Mei Sha & \href{works/QinDCS20.pdf}{QinDCS20}~\cite{QinDCS20}\\
Yufen Shao & \href{works/GoelSHFS15.pdf}{GoelSHFS15}~\cite{GoelSHFS15}\\
Ganquan Shi & \href{}{ShiYXQ22}~\cite{ShiYXQ22}\\
Zhongshun Shi & \href{works/QinWSLS21.pdf}{QinWSLS21}~\cite{QinWSLS21}\\
Leyuan Shi & \href{works/QinWSLS21.pdf}{QinWSLS21}~\cite{QinWSLS21}\\
Stuart Siegel & \href{works/DavenportKRSH07.pdf}{DavenportKRSH07}~\cite{DavenportKRSH07}\\
Maria Silvia Pini & \href{works/GelainPRVW17.pdf}{GelainPRVW17}~\cite{GelainPRVW17}\\
Vanessa Simard & \href{works/BoudreaultSLQ22.pdf}{BoudreaultSLQ22}~\cite{BoudreaultSLQ22}\\
Pawel Sitek & \href{works/WikarekS19.pdf}{WikarekS19}~\cite{WikarekS19}\\
M. Slusky & \href{works/GoelSHFS15.pdf}{GoelSHFS15}~\cite{GoelSHFS15}\\
Juha{-}Pekka Soininen & \href{works/QuSN06.pdf}{QuSN06}~\cite{QuSN06}\\
Xiaoqing Song & \href{works/ZhangLS12.pdf}{ZhangLS12}~\cite{ZhangLS12}\\
Francis Sourd & \href{works/SourdN00.pdf}{SourdN00}~\cite{SourdN00}\\
Helge Spieker & \href{works/MossigeGSMC17.pdf}{MossigeGSMC17}~\cite{MossigeGSMC17}\\
Samuel Squillaci & \href{works/SquillaciPR23.pdf}{SquillaciPR23}~\cite{SquillaciPR23}\\
Andreas Starzacher & \href{}{FriedrichFMRSST14}~\cite{FriedrichFMRSST14}\\
Wolfgang Steigerwald & \href{works/GeitzGSSW22.pdf}{GeitzGSSW22}~\cite{GeitzGSSW22}\\
R{\"{u}}diger Stephan & \href{works/HeinzSSW12.pdf}{HeinzSSW12}~\cite{HeinzSSW12}\\
Malgorzata Sterna & \href{}{BlazewiczEP19}~\cite{BlazewiczEP19}\\
Robin St{\"{o}}hr & \href{works/GeitzGSSW22.pdf}{GeitzGSSW22}~\cite{GeitzGSSW22}\\
Christian St{\"{u}}rck & \href{works/HubnerGSV21.pdf}{HubnerGSV21}~\cite{HubnerGSV21}\\
Kaile Su & \href{works/RiahiNS018.pdf}{RiahiNS018}~\cite{RiahiNS018}\\
Wei Su & \href{works/MakMS10.pdf}{MakMS10}~\cite{MakMS10}\\
Kemal Subulan & \href{works/SubulanC22.pdf}{SubulanC22}~\cite{SubulanC22}\\
Premysl Sucha & \href{works/BenediktSMVH18.pdf}{BenediktSMVH18}~\cite{BenediktSMVH18}\\
Quanxin Sun & \href{works/TangLWSK18.pdf}{TangLWSK18}~\cite{TangLWSK18}\\
Zheng Sun & \href{works/SunLYL10.pdf}{SunLYL10}~\cite{SunLYL10}\\
Suresh Sundaram & \href{works/SureshMOK06.pdf}{SureshMOK06}~\cite{SureshMOK06}\\
Pavel Surynek & \href{works/BartakCS10.pdf}{BartakCS10}~\cite{BartakCS10}\\
Jir{\'{\i}} Svancara & \href{works/SvancaraB22.pdf}{SvancaraB22}~\cite{SvancaraB22}\\
Ria Szeredi & \href{works/SzerediS16.pdf}{SzerediS16}~\cite{SzerediS16}\\
Alina S{\^{\i}}rbu & \href{works/GalleguillosKSB19.pdf}{GalleguillosKSB19}~\cite{GalleguillosKSB19}\\
Christos T. Maravelias & \href{works/MaraveliasG04.pdf}{MaraveliasG04}~\cite{MaraveliasG04}\\
Willian T. Lunardi & \href{works/LunardiBLRV20.pdf}{LunardiBLRV20}~\cite{LunardiBLRV20}\\
Siyu Tang & \href{works/VlkHT21.pdf}{VlkHT21}~\cite{VlkHT21}\\
Yuanjie Tang & \href{works/TangLWSK18.pdf}{TangLWSK18}~\cite{TangLWSK18}\\
Fabio Tardivo & \href{works/TardivoDFMP23.pdf}{TardivoDFMP23}~\cite{TardivoDFMP23}\\
Armagan Tarim & \href{works/RossiTHP07.pdf}{RossiTHP07}~\cite{RossiTHP07}\\
Ehsan Tarkesh Esfahani & \href{works/YounespourAKE19.pdf}{YounespourAKE19}~\cite{YounespourAKE19}\\
Reza Tavakkoli-Moghaddam & \href{}{Fatemi-AnarakiMFN22}~\cite{Fatemi-AnarakiMFN22}\\
Nikolay Tchernev & \href{works/BourreauGGLT22.pdf}{BourreauGGLT22}~\cite{BourreauGGLT22}\\
Paolo Terenziani & \href{works/BrusoniCLMMT96.pdf}{BrusoniCLMMT96}~\cite{BrusoniCLMMT96}\\
Willian Tessaro Lunardi & \href{}{Lunardi20}~\cite{Lunardi20}\\
Stephan Teuschl & \href{works/FrohnerTR19.pdf}{FrohnerTR19}~\cite{FrohnerTR19}\\
Jordan Ticktin & \href{works/HillTV21.pdf}{HillTV21}~\cite{HillTV21}\\
Kevin Tierney & \href{works/KelarevaTK13.pdf}{KelarevaTK13}~\cite{KelarevaTK13}\\
Christian Timpe & \href{works/Timpe02.pdf}{Timpe02}~\cite{Timpe02}\\
Mary Tom & \href{works/Tom19.pdf}{Tom19}~\cite{Tom19}\\
Seyda Topaloglu & \href{works/TopalogluO11.pdf}{TopalogluO11}~\cite{TopalogluO11}\\
Miguel Toro & \href{works/ValleMGT03.pdf}{ValleMGT03}~\cite{ValleMGT03}\\
Meriem Touat & \href{works/TouatBT22.pdf}{TouatBT22}~\cite{TouatBT22}\\
Toura{\"{\i}}vane & \href{works/Touraivane95.pdf}{Touraivane95}~\cite{Touraivane95}\\
Mariem Trojet & \href{works/TrojetHL11.pdf}{TrojetHL11}~\cite{TrojetHL11}\\
Semra Tunali & \href{works/OzturkTHO13.pdf}{OzturkTHO13}~\cite{OzturkTHO13}\\
Paul Tyler & \href{works/HebrardTW05.pdf}{HebrardTW05}~\cite{HebrardTW05}\\
Jumyung Um & \href{works/ParkUJR19.pdf}{ParkUJR19}~\cite{ParkUJR19}\\
J. V. Moccellin & \href{works/AbreuAPNM21.pdf}{AbreuAPNM21}~\cite{AbreuAPNM21}\\
Behdin Vahedi-Nouri & \href{}{Fatemi-AnarakiMFN22}~\cite{Fatemi-AnarakiMFN22}\\
Roshanaei, Vahid & \href{works/NaderiRR23.pdf}{NaderiRR23}~\cite{NaderiRR23}\\
Karen Villaverde & \href{}{VillaverdeP04}~\cite{VillaverdeP04}\\
Mariona Vil{\`a} & \href{works/YuraszeckMPV22.pdf}{YuraszeckMPV22}~\cite{YuraszeckMPV22}\\
Rebekka Volk & \href{works/HubnerGSV21.pdf}{HubnerGSV21}~\cite{HubnerGSV21}\\
Holger Voos & \href{works/LunardiBLRV20.pdf}{LunardiBLRV20}~\cite{LunardiBLRV20}\\
Thomas W. M. Vossen & \href{works/HillTV21.pdf}{HillTV21}~\cite{HillTV21}\\
Kai Waelti & \href{works/KoehlerBFFHPSSS21.pdf}{KoehlerBFFHPSSS21}~\cite{KoehlerBFFHPSSS21}\\
Runsen Wang & \href{works/QinWSLS21.pdf}{QinWSLS21}~\cite{QinWSLS21}\\
Futian Wang & \href{works/TangLWSK18.pdf}{TangLWSK18}~\cite{TangLWSK18}\\
Shouyang Wang & \href{works/ZhangW18.pdf}{ZhangW18}~\cite{ZhangW18}\\
Tao Wang & \href{works/WangMD15.pdf}{WangMD15}~\cite{WangMD15}\\
Jan Weglarz & \href{}{BlazewiczEP19}~\cite{BlazewiczEP19}\\
Christine Wei Wu & \href{works/WuBB05.pdf}{WuBB05}~\cite{WuBB05}\\
Kong Wei Lye & \href{works/LauLN08.pdf}{LauLN08}~\cite{LauLN08}\\
Johan Wess{\'{e}}n & \href{works/WessenCS20.pdf}{WessenCS20}~\cite{WessenCS20}\\
Jaroslaw Wikarek & \href{works/WikarekS19.pdf}{WikarekS19}~\cite{WikarekS19}\\
Campbell Wilson & \href{works/He0GLW18.pdf}{He0GLW18}~\cite{He0GLW18}\\
Michael Winkler & \href{works/HeinzSSW12.pdf}{HeinzSSW12}~\cite{HeinzSSW12}\\
David Wittwer & \href{works/BenderWS21.pdf}{BenderWS21}~\cite{BenderWS21}\\
J{\"{o}}rg W{\"{u}}rtz & \href{works/SchildW00.pdf}{SchildW00}~\cite{SchildW00}\\
Quanshi Xia & \href{works/ChuX05.pdf}{ChuX05}~\cite{ChuX05}\\
Hegen Xiong & \href{works/FanXG21.pdf}{FanXG21}~\cite{FanXG21}\\
Zhou Xu & \href{works/LimRX04.pdf}{LimRX04}~\cite{LimRX04}\\
Yang Xu & \href{}{ShiYXQ22}~\cite{ShiYXQ22}\\
Tanya Y. Tang & \href{works/TangB20.pdf}{TangB20}~\cite{TangB20}\\
El Yaakoubi Anass & \href{works/FallahiAC20.pdf}{FallahiAC20}~\cite{FallahiAC20}\\
Hong Yan & \href{works/HookerY02.pdf}{HookerY02}~\cite{HookerY02}\\
Moli Yang & \href{works/YangSS19.pdf}{YangSS19}~\cite{YangSS19}\\
Zhouwang Yang & \href{}{ShiYXQ22}~\cite{ShiYXQ22}\\
Jia{-}Sheng Yao & \href{works/HoYCLLCLC18.pdf}{HoYCLLCLC18}~\cite{HoYCLLCLC18}\\
Min Yao & \href{works/SunLYL10.pdf}{SunLYL10}~\cite{SunLYL10}\\
Seung Yeob Shin & \href{works/ShinBBHO18.pdf}{ShinBBHO18}~\cite{ShinBBHO18}\\
Vassilios Yfantis & \href{works/KlankeBYE21.pdf}{KlankeBYE21}~\cite{KlankeBYE21}\\
Maryam Younespour & \href{works/YounespourAKE19.pdf}{YounespourAKE19}~\cite{YounespourAKE19}\\
Chunxia Yu & \href{works/ZhangYW21.pdf}{ZhangYW21}~\cite{ZhangYW21}\\
Xinghuo Yu & \href{works/MartinPY01.pdf}{MartinPY01}~\cite{MartinPY01}\\
Oleg Yu. Gusikhin & \href{works/BarlattCG08.pdf}{BarlattCG08}~\cite{BarlattCG08}\\
Pinar Yunusoglu & \href{works/YunusogluY22.pdf}{YunusogluY22}~\cite{YunusogluY22}\\
Marco Zaffalon & \href{works/Darby-DowmanLMZ97.pdf}{Darby-DowmanLMZ97}~\cite{Darby-DowmanLMZ97}\\
St{\'{e}}phane Zampelli & \href{works/DerrienPZ14.pdf}{DerrienPZ14}~\cite{DerrienPZ14}\\
Bahram Zarrin & \href{works/BarzegaranZP20.pdf}{BarzegaranZP20}~\cite{BarzegaranZP20}\\
Mengjie Zhang & \href{works/abs-2402-00459.pdf}{abs-2402-00459}~\cite{abs-2402-00459}\\
Haotian Zhang & \href{works/ZhangJZL22.pdf}{ZhangJZL22}~\cite{ZhangJZL22}\\
Luping Zhang & \href{works/ZhangYW21.pdf}{ZhangYW21}~\cite{ZhangYW21}\\
Chaoyong Zhang & \href{works/MengZRZL20.pdf}{MengZRZL20}~\cite{MengZRZL20}\\
Biao Zhang & \href{works/MengZRZL20.pdf}{MengZRZL20}~\cite{MengZRZL20}\\
Sicheng Zhang & \href{works/ZhangW18.pdf}{ZhangW18}~\cite{ZhangW18}\\
Xujun Zhang & \href{works/ZhangLS12.pdf}{ZhangLS12}~\cite{ZhangLS12}\\
Lihui Zhang & \href{works/ZouZ20.pdf}{ZouZ20}~\cite{ZouZ20}\\
Jinlian Zhou & \href{works/ZhouGL15.pdf}{ZhouGL15}~\cite{ZhouGL15}\\
Pawel Zielinski & \href{works/FortinZDF05.pdf}{FortinZDF05}~\cite{FortinZDF05}\\
Xin Zou & \href{works/ZouZ20.pdf}{ZouZ20}~\cite{ZouZ20}\\
Mathijs de Weerdt & \href{works/BogaerdtW19.pdf}{BogaerdtW19}~\cite{BogaerdtW19}\\
Bruno de Athayde Prata & \href{works/AbreuAPNM21.pdf}{AbreuAPNM21}~\cite{AbreuAPNM21}\\
Roman van der Krogt & \href{works/KrogtLPHJ07.pdf}{KrogtLPHJ07}~\cite{KrogtLPHJ07}\\
Pim van den Bogaerdt & \href{works/BogaerdtW19.pdf}{BogaerdtW19}~\cite{BogaerdtW19}\\
Stefano {Di Alesio} & \href{works/AlesioNBG14.pdf}{AlesioNBG14}~\cite{AlesioNBG14}\\
Selin {\"{O}}zpeynirci & \href{works/GokgurHO18.pdf}{GokgurHO18}~\cite{GokgurHO18}\\
Cemalettin {\"{O}}zt{\"{u}}rk & \href{works/OzturkTHO13.pdf}{OzturkTHO13}~\cite{OzturkTHO13}\\
Nahum {\'{A}}lvarez & \href{works/PovedaAA23.pdf}{PovedaAA23}~\cite{PovedaAA23}\\
Se{\'{a}}n {\'{O}}g Murphy & \href{works/MurphyMB15.pdf}{MurphyMB15}~\cite{MurphyMB15}\\
Gizem {\c{C}}akir & \href{works/SubulanC22.pdf}{SubulanC22}~\cite{SubulanC22}\\
\end{longtable}
}




\clearpage
\section{Problem Classification}


\begin{table}[htbp]
\caption{\label{tab:classification}Problem Classification Types}
\centering
{\scriptsize
\begin{tabular}{lp{8cm}}\toprule
Code & Name \\ \midrule
JSSP & Job-Shop Scheduling Problem \\
JSPT & Job-Shop Scheduling Problem with Transportation \\
PP-MS-MMRCPSP/max-cal & partially preemptive- multi-skill/mode resource-constrained project scheduling problem with generalized precedence relations and resource calendars\\
RCPSP & Resource Constrained Project Scheduling Problem \\
TMS & Transmission Network Maintenance Planning \\
PMSP & Parallel Machine Scheduling Problem\\
HFF & Hybrid Flexible Flow-shop \\
$HFFm|tt|C_{\max}$ & Hybrid Flexible Flowshop with Transportation Times\\
OSP & Oven Scheduling Problem \\
PTC & Scheduling Problem with Time Constraints\\
GCSP & Group Cumulative Scheduling Problem \\
2BPHFSP & Two-Stage Bin Packing and Hybrid Flow Shop Scheduling Problem\\
CTW & Cable Tree Wiring Problem\\
CHSP & Cyclic Hoist Scheduling Problem \\
CECSP & Continuous Energy-Constrained Scheduling Problem \\
CuSP & Cumulative Scheduling Problem \\
SBSFMMAL & Simultaneous Balancing and Scheduling of Flexible Mixed Model Assembly Lines\\
SMSDP & steel mill slab design problem \\
KRFP & kernel resource feasibility problem\\
TCSP & Temporal Constraint Satisfaction Problem\\
PJSSP & Pre-emptive Job-Shop scheduling Problem\\
MGAP & Modified Generalized Assignment Problem\\
EOSP & Earth Observation Scheduling Problem \\
SCC & Steel-making and continuous casting \\
OSSP & Open Shop Scheduling Problem\\
FJS & Fixed Job Scheduling\\
RCPSPDC & Resource-constrained Project Scheduling Problem with Discounted Cashflow \\
LSFRP & Liner Shipping Fleet Repositioning Problem\\
BPCTOP & Bulk Port Cargo Throughput Optimisation Problem\\

\bottomrule
\end{tabular}
}
\end{table}



\clearpage
\section{Concept Matching}

In order to automatically find out properties of the articles, we try to find certain concepts in the pdf versions of the articles. We manually defined an ontology of important concepts to look for, and defined regular expressions that would recognize these concepts in the text. We use the \emph{pdfgrep} command to search for the number of occurrences of certain regular expressions in the files. This often clearly identifies the constraints used in the model. We group the results by number of occurrences of the concept in the text of the work. Note that this is only approximate, as we do include the full pdf file in the search. A concept might only be mentioned in some of the title of citations used in the paper, we do count them in our results, as we were not able to remove the bibliography from the main body of the work.

Overall, if a work is not mentioned as using the concept, the the text does not contain a match to the corresponding regular expression. A fundamental limitation of this approach is that it only really works for text written in the language the regular expressions are designed for (in our case English), and not those written in another language. We could overcome this limitation by defining all concepts in other languages as well, and then using a language flag to identify the language the text is written in. 

Note that we only show the first 30 matching entries in each concept category, and list the total number of matches if there are more than 30 matches.


\clearpage
\subsection{Concept Type Concepts}
\label{sec:Concepts}
{\scriptsize
\begin{longtable}{lp{3cm}>{\raggedright\arraybackslash}p{6cm}>{\raggedright\arraybackslash}p{6cm}>{\raggedright\arraybackslash}p{8cm}}
\rowcolor{white}\caption{Works for Concepts of Type Concepts}\\ \toprule
\rowcolor{white}Type & Keyword & High & Medium & Low\\ \midrule\endhead
\bottomrule
\endfoot
Concepts & Allen's algebra &  &  & \\
Concepts & BOM & \href{../works/SubulanC22.pdf}{SubulanC22}~\cite{SubulanC22}, \href{../works/OrnekO16.pdf}{OrnekO16}~\cite{OrnekO16} &  & \href{../works/HoundjiSW19.pdf}{HoundjiSW19}~\cite{HoundjiSW19}, \href{../works/abs-1902-01193.pdf}{abs-1902-01193}~\cite{abs-1902-01193}\\
Concepts & Benders Decomposition & \href{../works/ForbesHJST24.pdf}{ForbesHJST24}~\cite{ForbesHJST24}, \href{../works/JuvinHL23a.pdf}{JuvinHL23a}~\cite{JuvinHL23a}, \href{../works/GuoZ23.pdf}{GuoZ23}~\cite{GuoZ23}, \href{../works/ZhuSZW23.pdf}{ZhuSZW23}~\cite{ZhuSZW23}, \href{../works/JuvinHL22.pdf}{JuvinHL22}~\cite{JuvinHL22}, \href{../works/EmdeZD22.pdf}{EmdeZD22}~\cite{EmdeZD22}, \href{../works/ElciOH22.pdf}{ElciOH22}~\cite{ElciOH22}, \href{../works/NaderiBZ22a.pdf}{NaderiBZ22a}~\cite{NaderiBZ22a}, \href{../works/NaderiBZ22.pdf}{NaderiBZ22}~\cite{NaderiBZ22}, \href{../works/VlkHT21.pdf}{VlkHT21}~\cite{VlkHT21}, \href{../works/RoshanaeiBAUB20.pdf}{RoshanaeiBAUB20}~\cite{RoshanaeiBAUB20}, \href{../works/Hooker19.pdf}{Hooker19}~\cite{Hooker19}, \href{../works/TanT18.pdf}{TanT18}~\cite{TanT18}, \href{../works/GombolayWS18.pdf}{GombolayWS18}~\cite{GombolayWS18}, \href{../works/GoldwaserS18.pdf}{GoldwaserS18}~\cite{GoldwaserS18}, \href{../works/GomesM17.pdf}{GomesM17}~\cite{GomesM17}, \href{../works/HookerH17.pdf}{HookerH17}~\cite{HookerH17}, \href{../works/CireCH16.pdf}{CireCH16}~\cite{CireCH16}, \href{../works/Froger16.pdf}{Froger16}~\cite{Froger16}, \href{../works/HechingH16.pdf}{HechingH16}~\cite{HechingH16}, \href{../works/TranAB16.pdf}{TranAB16}~\cite{TranAB16}, \href{../works/BajestaniB15.pdf}{BajestaniB15}~\cite{BajestaniB15}, \href{../works/BajestaniB13.pdf}{BajestaniB13}~\cite{BajestaniB13}, \href{../works/CireCH13.pdf}{CireCH13}~\cite{CireCH13}, \href{../works/HeinzKB13.pdf}{HeinzKB13}~\cite{HeinzKB13}, \href{../works/TranB12.pdf}{TranB12}~\cite{TranB12}, \href{../works/LombardiM12.pdf}{LombardiM12}~\cite{LombardiM12}, \href{../works/LimtanyakulS12.pdf}{LimtanyakulS12}~\cite{LimtanyakulS12}, \href{../works/HeinzB12.pdf}{HeinzB12}~\cite{HeinzB12}... (Total: 47) & \href{../works/NaderiRR23.pdf}{NaderiRR23}~\cite{NaderiRR23}, \href{../works/TangB20.pdf}{TangB20}~\cite{TangB20}, \href{../works/Laborie18a.pdf}{Laborie18a}~\cite{Laborie18a}, \href{../works/TranVNB17.pdf}{TranVNB17}~\cite{TranVNB17}, \href{../works/RoshanaeiLAU17.pdf}{RoshanaeiLAU17}~\cite{RoshanaeiLAU17}, \href{../works/GoldwaserS17.pdf}{GoldwaserS17}~\cite{GoldwaserS17}, \href{../works/HarjunkoskiMBC14.pdf}{HarjunkoskiMBC14}~\cite{HarjunkoskiMBC14}, \href{../works/GuyonLPR12.pdf}{GuyonLPR12}~\cite{GuyonLPR12}, \href{../works/LombardiMRB10.pdf}{LombardiMRB10}~\cite{LombardiMRB10}, \href{../works/BeniniLMR08.pdf}{BeniniLMR08}~\cite{BeniniLMR08}, \href{../works/Hooker05a.pdf}{Hooker05a}~\cite{Hooker05a}, \href{../works/HookerY02.pdf}{HookerY02}~\cite{HookerY02} & \href{../works/PrataAN23.pdf}{PrataAN23}~\cite{PrataAN23}, \href{../works/PovedaAA23.pdf}{PovedaAA23}~\cite{PovedaAA23}, \href{../works/AlfieriGPS23.pdf}{AlfieriGPS23}~\cite{AlfieriGPS23}, \href{../works/JuvinHHL23.pdf}{JuvinHHL23}~\cite{JuvinHHL23}, \href{../works/LuoB22.pdf}{LuoB22}~\cite{LuoB22}, \href{../works/FarsiTM22.pdf}{FarsiTM22}~\cite{FarsiTM22}, \href{../works/Godet21a.pdf}{Godet21a}~\cite{Godet21a}, \href{../works/Mercier-AubinGQ20.pdf}{Mercier-AubinGQ20}~\cite{Mercier-AubinGQ20}, \href{../works/Polo-MejiaALB20.pdf}{Polo-MejiaALB20}~\cite{Polo-MejiaALB20}, \href{../works/QinDCS20.pdf}{QinDCS20}~\cite{QinDCS20}, \href{../works/WallaceY20.pdf}{WallaceY20}~\cite{WallaceY20}, \href{../works/MengZRZL20.pdf}{MengZRZL20}~\cite{MengZRZL20}, \href{../works/AntunesABD20.pdf}{AntunesABD20}~\cite{AntunesABD20}, \href{../works/MurinR19.pdf}{MurinR19}~\cite{MurinR19}, \href{../works/FrimodigS19.pdf}{FrimodigS19}~\cite{FrimodigS19}, \href{../works/LaborieRSV18.pdf}{LaborieRSV18}~\cite{LaborieRSV18}, \href{../works/CappartTSR18.pdf}{CappartTSR18}~\cite{CappartTSR18}, \href{../works/AntunesABD18.pdf}{AntunesABD18}~\cite{AntunesABD18}, \href{../works/BoothNB16.pdf}{BoothNB16}~\cite{BoothNB16}, \href{../works/FontaineMH16.pdf}{FontaineMH16}~\cite{FontaineMH16}, \href{../works/Fahimi16.pdf}{Fahimi16}~\cite{Fahimi16}, \href{../works/EvenSH15a.pdf}{EvenSH15a}~\cite{EvenSH15a}, \href{../works/BurtLPS15.pdf}{BurtLPS15}~\cite{BurtLPS15}, \href{../works/EvenSH15.pdf}{EvenSH15}~\cite{EvenSH15}, \href{../works/LipovetzkyBPS14.pdf}{LipovetzkyBPS14}~\cite{LipovetzkyBPS14}, \href{../works/KoschB14.pdf}{KoschB14}~\cite{KoschB14}, \href{../works/BlomBPS14.pdf}{BlomBPS14}~\cite{BlomBPS14}, \href{../works/KelarevaTK13.pdf}{KelarevaTK13}~\cite{KelarevaTK13}, \href{../works/TerekhovDOB12.pdf}{TerekhovDOB12}~\cite{TerekhovDOB12}... (Total: 39)\\
Concepts & Logic-Based Benders Decomposition & \href{../works/ForbesHJST24.pdf}{ForbesHJST24}~\cite{ForbesHJST24}, \href{../works/GuoZ23.pdf}{GuoZ23}~\cite{GuoZ23}, \href{../works/ZhuSZW23.pdf}{ZhuSZW23}~\cite{ZhuSZW23}, \href{../works/JuvinHL23a.pdf}{JuvinHL23a}~\cite{JuvinHL23a}, \href{../works/ElciOH22.pdf}{ElciOH22}~\cite{ElciOH22}, \href{../works/JuvinHL22.pdf}{JuvinHL22}~\cite{JuvinHL22}, \href{../works/EmdeZD22.pdf}{EmdeZD22}~\cite{EmdeZD22}, \href{../works/NaderiBZ22a.pdf}{NaderiBZ22a}~\cite{NaderiBZ22a}, \href{../works/VlkHT21.pdf}{VlkHT21}~\cite{VlkHT21}, \href{../works/Hooker19.pdf}{Hooker19}~\cite{Hooker19}, \href{../works/GoldwaserS18.pdf}{GoldwaserS18}~\cite{GoldwaserS18}, \href{../works/TanT18.pdf}{TanT18}~\cite{TanT18}, \href{../works/HookerH17.pdf}{HookerH17}~\cite{HookerH17}, \href{../works/HechingH16.pdf}{HechingH16}~\cite{HechingH16}, \href{../works/CireCH16.pdf}{CireCH16}~\cite{CireCH16}, \href{../works/TranAB16.pdf}{TranAB16}~\cite{TranAB16}, \href{../works/BajestaniB15.pdf}{BajestaniB15}~\cite{BajestaniB15}, \href{../works/HeinzKB13.pdf}{HeinzKB13}~\cite{HeinzKB13}, \href{../works/BajestaniB13.pdf}{BajestaniB13}~\cite{BajestaniB13}, \href{../works/CireCH13.pdf}{CireCH13}~\cite{CireCH13}, \href{../works/TranB12.pdf}{TranB12}~\cite{TranB12}, \href{../works/LombardiM12.pdf}{LombardiM12}~\cite{LombardiM12}, \href{../works/BeniniLMR11.pdf}{BeniniLMR11}~\cite{BeniniLMR11}, \href{../works/BajestaniB11.pdf}{BajestaniB11}~\cite{BajestaniB11}, \href{../works/CobanH11.pdf}{CobanH11}~\cite{CobanH11}, \href{../works/Hooker07.pdf}{Hooker07}~\cite{Hooker07}, \href{../works/Hooker05.pdf}{Hooker05}~\cite{Hooker05}, \href{../works/Hooker04.pdf}{Hooker04}~\cite{Hooker04}, \href{../works/HookerO03.pdf}{HookerO03}~\cite{HookerO03} & \href{../works/NaderiRR23.pdf}{NaderiRR23}~\cite{NaderiRR23}, \href{../works/NaderiBZ22.pdf}{NaderiBZ22}~\cite{NaderiBZ22}, \href{../works/RoshanaeiBAUB20.pdf}{RoshanaeiBAUB20}~\cite{RoshanaeiBAUB20}, \href{../works/TangB20.pdf}{TangB20}~\cite{TangB20}, \href{../works/Laborie18a.pdf}{Laborie18a}~\cite{Laborie18a}, \href{../works/GoldwaserS17.pdf}{GoldwaserS17}~\cite{GoldwaserS17}, \href{../works/Froger16.pdf}{Froger16}~\cite{Froger16}, \href{../works/HeinzB12.pdf}{HeinzB12}~\cite{HeinzB12}, \href{../works/GuyonLPR12.pdf}{GuyonLPR12}~\cite{GuyonLPR12}, \href{../works/Lombardi10.pdf}{Lombardi10}~\cite{Lombardi10}, \href{../works/CobanH10.pdf}{CobanH10}~\cite{CobanH10}, \href{../works/MilanoW09.pdf}{MilanoW09}~\cite{MilanoW09}, \href{../works/BeniniLMR08.pdf}{BeniniLMR08}~\cite{BeniniLMR08}, \href{../works/CorreaLR07.pdf}{CorreaLR07}~\cite{CorreaLR07}, \href{../works/Hooker06.pdf}{Hooker06}~\cite{Hooker06}, \href{../works/HookerY02.pdf}{HookerY02}~\cite{HookerY02} & \href{../works/PrataAN23.pdf}{PrataAN23}~\cite{PrataAN23}, \href{../works/JuvinHHL23.pdf}{JuvinHHL23}~\cite{JuvinHHL23}, \href{../works/FarsiTM22.pdf}{FarsiTM22}~\cite{FarsiTM22}, \href{../works/Mercier-AubinGQ20.pdf}{Mercier-AubinGQ20}~\cite{Mercier-AubinGQ20}, \href{../works/QinDCS20.pdf}{QinDCS20}~\cite{QinDCS20}, \href{../works/WallaceY20.pdf}{WallaceY20}~\cite{WallaceY20}, \href{../works/MurinR19.pdf}{MurinR19}~\cite{MurinR19}, \href{../works/CappartTSR18.pdf}{CappartTSR18}~\cite{CappartTSR18}, \href{../works/GombolayWS18.pdf}{GombolayWS18}~\cite{GombolayWS18}, \href{../works/AntunesABD18.pdf}{AntunesABD18}~\cite{AntunesABD18}, \href{../works/LaborieRSV18.pdf}{LaborieRSV18}~\cite{LaborieRSV18}, \href{../works/GomesM17.pdf}{GomesM17}~\cite{GomesM17}, \href{../works/TranVNB17.pdf}{TranVNB17}~\cite{TranVNB17}, \href{../works/RoshanaeiLAU17.pdf}{RoshanaeiLAU17}~\cite{RoshanaeiLAU17}, \href{../works/FontaineMH16.pdf}{FontaineMH16}~\cite{FontaineMH16}, \href{../works/BoothNB16.pdf}{BoothNB16}~\cite{BoothNB16}, \href{../works/Fahimi16.pdf}{Fahimi16}~\cite{Fahimi16}, \href{../works/KoschB14.pdf}{KoschB14}~\cite{KoschB14}, \href{../works/HarjunkoskiMBC14.pdf}{HarjunkoskiMBC14}~\cite{HarjunkoskiMBC14}, \href{../works/TerekhovDOB12.pdf}{TerekhovDOB12}~\cite{TerekhovDOB12}, \href{../works/EdisO11.pdf}{EdisO11}~\cite{EdisO11}, \href{../works/HachemiGR11.pdf}{HachemiGR11}~\cite{HachemiGR11}, \href{../works/KovacsK11.pdf}{KovacsK11}~\cite{KovacsK11}, \href{../works/LombardiMRB10.pdf}{LombardiMRB10}~\cite{LombardiMRB10}, \href{../works/LombardiM10a.pdf}{LombardiM10a}~\cite{LombardiM10a}, \href{../works/RuggieroBBMA09.pdf}{RuggieroBBMA09}~\cite{RuggieroBBMA09}, \href{../works/RodriguezS09.pdf}{RodriguezS09}~\cite{RodriguezS09}, \href{../works/BeniniBGM06.pdf}{BeniniBGM06}~\cite{BeniniBGM06}, \href{../works/MilanoW06.pdf}{MilanoW06}~\cite{MilanoW06}... (Total: 34)\\
Concepts & Pareto & \href{../works/FarsiTM22.pdf}{FarsiTM22}~\cite{FarsiTM22}, \href{../works/Zahout21.pdf}{Zahout21}~\cite{Zahout21}, \href{../works/Lemos21.pdf}{Lemos21}~\cite{Lemos21}, \href{../works/ZarandiASC20.pdf}{ZarandiASC20}~\cite{ZarandiASC20}, \href{../works/Dejemeppe16.pdf}{Dejemeppe16}~\cite{Dejemeppe16}, \href{../works/KovacsK11.pdf}{KovacsK11}~\cite{KovacsK11} & \href{../works/YounespourAKE19.pdf}{YounespourAKE19}~\cite{YounespourAKE19}, \href{../works/DejemeppeD14.pdf}{DejemeppeD14}~\cite{DejemeppeD14}, \href{../works/HeckmanB11.pdf}{HeckmanB11}~\cite{HeckmanB11} & \href{../works/CzerniachowskaWZ23.pdf}{CzerniachowskaWZ23}~\cite{CzerniachowskaWZ23}, \href{../works/LacknerMMWW23.pdf}{LacknerMMWW23}~\cite{LacknerMMWW23}, \href{../works/GokPTGO23.pdf}{GokPTGO23}~\cite{GokPTGO23}, \href{../works/JuvinHL23a.pdf}{JuvinHL23a}~\cite{JuvinHL23a}, \href{../works/JuvinHL22.pdf}{JuvinHL22}~\cite{JuvinHL22}, \href{../works/WinterMMW22.pdf}{WinterMMW22}~\cite{WinterMMW22}, \href{../works/OrnekOS20.pdf}{OrnekOS20}~\cite{OrnekOS20}, \href{../works/Lunardi20.pdf}{Lunardi20}~\cite{Lunardi20}, \href{../works/AntuoriHHEN20.pdf}{AntuoriHHEN20}~\cite{AntuoriHHEN20}, \href{../works/GurEA19.pdf}{GurEA19}~\cite{GurEA19}, \href{../works/EscobetPQPRA19.pdf}{EscobetPQPRA19}~\cite{EscobetPQPRA19}, \href{../works/CappartTSR18.pdf}{CappartTSR18}~\cite{CappartTSR18}, \href{../works/GomesM17.pdf}{GomesM17}~\cite{GomesM17}, \href{../works/Froger16.pdf}{Froger16}~\cite{Froger16}, \href{../works/BridiBLMB16.pdf}{BridiBLMB16}~\cite{BridiBLMB16}, \href{../works/HarjunkoskiMBC14.pdf}{HarjunkoskiMBC14}~\cite{HarjunkoskiMBC14}, \href{../works/RuggieroBBMA09.pdf}{RuggieroBBMA09}~\cite{RuggieroBBMA09}, \href{../works/Baptiste02.pdf}{Baptiste02}~\cite{Baptiste02}, \href{../works/VanczaM01.pdf}{VanczaM01}~\cite{VanczaM01}, \href{../works/FocacciLN00.pdf}{FocacciLN00}~\cite{FocacciLN00}\\
Concepts & activity & \href{../works/TardivoDFMP23.pdf}{TardivoDFMP23}~\cite{TardivoDFMP23}, \href{../works/GokPTGO23.pdf}{GokPTGO23}~\cite{GokPTGO23}, \href{../works/PovedaAA23.pdf}{PovedaAA23}~\cite{PovedaAA23}, \href{../works/AalianPG23.pdf}{AalianPG23}~\cite{AalianPG23}, \href{../works/PenzDN23.pdf}{PenzDN23}~\cite{PenzDN23}, \href{../works/MarliereSPR23.pdf}{MarliereSPR23}~\cite{MarliereSPR23}, \href{../works/CampeauG22.pdf}{CampeauG22}~\cite{CampeauG22}, \href{../works/SvancaraB22.pdf}{SvancaraB22}~\cite{SvancaraB22}, \href{../works/TouatBT22.pdf}{TouatBT22}~\cite{TouatBT22}, \href{../works/SubulanC22.pdf}{SubulanC22}~\cite{SubulanC22}, \href{../works/BenderWS21.pdf}{BenderWS21}~\cite{BenderWS21}, \href{../works/KlankeBYE21.pdf}{KlankeBYE21}~\cite{KlankeBYE21}, \href{../works/Astrand21.pdf}{Astrand21}~\cite{Astrand21}, \href{../works/HubnerGSV21.pdf}{HubnerGSV21}~\cite{HubnerGSV21}, \href{../works/Godet21a.pdf}{Godet21a}~\cite{Godet21a}, \href{../works/ZarandiASC20.pdf}{ZarandiASC20}~\cite{ZarandiASC20}, \href{../works/CauwelaertDS20.pdf}{CauwelaertDS20}~\cite{CauwelaertDS20}, \href{../works/HauderBRPA20.pdf}{HauderBRPA20}~\cite{HauderBRPA20}, \href{../works/Polo-MejiaALB20.pdf}{Polo-MejiaALB20}~\cite{Polo-MejiaALB20}, \href{../works/AstrandJZ20.pdf}{AstrandJZ20}~\cite{AstrandJZ20}, \href{../works/BadicaBI20.pdf}{BadicaBI20}~\cite{BadicaBI20}, \href{../works/ZouZ20.pdf}{ZouZ20}~\cite{ZouZ20}, \href{../works/ThomasKS20.pdf}{ThomasKS20}~\cite{ThomasKS20}, \href{../works/abs-1902-09244.pdf}{abs-1902-09244}~\cite{abs-1902-09244}, \href{../works/GeibingerMM19.pdf}{GeibingerMM19}~\cite{GeibingerMM19}, \href{../works/NattafHKAL19.pdf}{NattafHKAL19}~\cite{NattafHKAL19}, \href{../works/YounespourAKE19.pdf}{YounespourAKE19}~\cite{YounespourAKE19}, \href{../works/Caballero19.pdf}{Caballero19}~\cite{Caballero19}, \href{../works/BadicaBIL19.pdf}{BadicaBIL19}~\cite{BadicaBIL19}... (Total: 173) & \href{../works/BonninMNE24.pdf}{BonninMNE24}~\cite{BonninMNE24}, \href{../works/YuraszeckMCCR23.pdf}{YuraszeckMCCR23}~\cite{YuraszeckMCCR23}, \href{../works/AfsarVPG23.pdf}{AfsarVPG23}~\cite{AfsarVPG23}, \href{../works/Bit-Monnot23.pdf}{Bit-Monnot23}~\cite{Bit-Monnot23}, \href{../works/BoudreaultSLQ22.pdf}{BoudreaultSLQ22}~\cite{BoudreaultSLQ22}, \href{../works/PopovicCGNC22.pdf}{PopovicCGNC22}~\cite{PopovicCGNC22}, \href{../works/Lunardi20.pdf}{Lunardi20}~\cite{Lunardi20}, \href{../works/LunardiBLRV20.pdf}{LunardiBLRV20}~\cite{LunardiBLRV20}, \href{../works/AntunesABD20.pdf}{AntunesABD20}~\cite{AntunesABD20}, \href{../works/GokGSTO20.pdf}{GokGSTO20}~\cite{GokGSTO20}, \href{../works/Hooker19.pdf}{Hooker19}~\cite{Hooker19}, \href{../works/EscobetPQPRA19.pdf}{EscobetPQPRA19}~\cite{EscobetPQPRA19}, \href{../works/Novas19.pdf}{Novas19}~\cite{Novas19}, \href{../works/YangSS19.pdf}{YangSS19}~\cite{YangSS19}, \href{../works/ShinBBHO18.pdf}{ShinBBHO18}~\cite{ShinBBHO18}, \href{../works/SchuttS16.pdf}{SchuttS16}~\cite{SchuttS16}, \href{../works/BoothNB16.pdf}{BoothNB16}~\cite{BoothNB16}, \href{../works/OrnekO16.pdf}{OrnekO16}~\cite{OrnekO16}, \href{../works/TranWDRFOVB16.pdf}{TranWDRFOVB16}~\cite{TranWDRFOVB16}, \href{../works/VilimLS15.pdf}{VilimLS15}~\cite{VilimLS15}, \href{../works/Derrien15.pdf}{Derrien15}~\cite{Derrien15}, \href{../works/GoelSHFS15.pdf}{GoelSHFS15}~\cite{GoelSHFS15}, \href{../works/HarjunkoskiMBC14.pdf}{HarjunkoskiMBC14}~\cite{HarjunkoskiMBC14}, \href{../works/DoulabiRP14.pdf}{DoulabiRP14}~\cite{DoulabiRP14}, \href{../works/LombardiM13.pdf}{LombardiM13}~\cite{LombardiM13}, \href{../works/LombardiMB13.pdf}{LombardiMB13}~\cite{LombardiMB13}, \href{../works/Clercq12.pdf}{Clercq12}~\cite{Clercq12}, \href{../works/BonfiettiM12.pdf}{BonfiettiM12}~\cite{BonfiettiM12}, \href{../works/ChapadosJR11.pdf}{ChapadosJR11}~\cite{ChapadosJR11}... (Total: 52) & \href{../works/PrataAN23.pdf}{PrataAN23}~\cite{PrataAN23}, \href{../works/GuoZ23.pdf}{GuoZ23}~\cite{GuoZ23}, \href{../works/JuvinHL23a.pdf}{JuvinHL23a}~\cite{JuvinHL23a}, \href{../works/abs-2312-13682.pdf}{abs-2312-13682}~\cite{abs-2312-13682}, \href{../works/CzerniachowskaWZ23.pdf}{CzerniachowskaWZ23}~\cite{CzerniachowskaWZ23}, \href{../works/ShaikhK23.pdf}{ShaikhK23}~\cite{ShaikhK23}, \href{../works/SquillaciPR23.pdf}{SquillaciPR23}~\cite{SquillaciPR23}, \href{../works/abs-2305-19888.pdf}{abs-2305-19888}~\cite{abs-2305-19888}, \href{../works/PerezGSL23.pdf}{PerezGSL23}~\cite{PerezGSL23}, \href{../works/PohlAK22.pdf}{PohlAK22}~\cite{PohlAK22}, \href{../works/OuelletQ22.pdf}{OuelletQ22}~\cite{OuelletQ22}, \href{../works/MullerMKP22.pdf}{MullerMKP22}~\cite{MullerMKP22}, \href{../works/JuvinHL22.pdf}{JuvinHL22}~\cite{JuvinHL22}, \href{../works/YunusogluY22.pdf}{YunusogluY22}~\cite{YunusogluY22}, \href{../works/HeinzNVH22.pdf}{HeinzNVH22}~\cite{HeinzNVH22}, \href{../works/abs-2211-14492.pdf}{abs-2211-14492}~\cite{abs-2211-14492}, \href{../works/HebrardALLCMR22.pdf}{HebrardALLCMR22}~\cite{HebrardALLCMR22}, \href{../works/EtminaniesfahaniGNMS22.pdf}{EtminaniesfahaniGNMS22}~\cite{EtminaniesfahaniGNMS22}, \href{../works/Groleaz21.pdf}{Groleaz21}~\cite{Groleaz21}, \href{../works/HillTV21.pdf}{HillTV21}~\cite{HillTV21}, \href{../works/Zahout21.pdf}{Zahout21}~\cite{Zahout21}, \href{../works/GeibingerMM21.pdf}{GeibingerMM21}~\cite{GeibingerMM21}, \href{../works/Astrand0F21.pdf}{Astrand0F21}~\cite{Astrand0F21}, \href{../works/ZhangYW21.pdf}{ZhangYW21}~\cite{ZhangYW21}, \href{../works/PandeyS21a.pdf}{PandeyS21a}~\cite{PandeyS21a}, \href{../works/QinDCS20.pdf}{QinDCS20}~\cite{QinDCS20}, \href{../works/Mercier-AubinGQ20.pdf}{Mercier-AubinGQ20}~\cite{Mercier-AubinGQ20}, \href{../works/SacramentoSP20.pdf}{SacramentoSP20}~\cite{SacramentoSP20}, \href{../works/RoshanaeiBAUB20.pdf}{RoshanaeiBAUB20}~\cite{RoshanaeiBAUB20}... (Total: 92)\\
Concepts & batch process & \href{../works/LacknerMMWW23.pdf}{LacknerMMWW23}~\cite{LacknerMMWW23}, \href{../works/LacknerMMWW21.pdf}{LacknerMMWW21}~\cite{LacknerMMWW21}, \href{../works/QinWSLS21.pdf}{QinWSLS21}~\cite{QinWSLS21}, \href{../works/ZarandiASC20.pdf}{ZarandiASC20}~\cite{ZarandiASC20}, \href{../works/HamC16.pdf}{HamC16}~\cite{HamC16}, \href{../works/NovaraNH16.pdf}{NovaraNH16}~\cite{NovaraNH16}, \href{../works/KoschB14.pdf}{KoschB14}~\cite{KoschB14}, \href{../works/HarjunkoskiMBC14.pdf}{HarjunkoskiMBC14}~\cite{HarjunkoskiMBC14}, \href{../works/Malapert11.pdf}{Malapert11}~\cite{Malapert11} & \href{../works/TangB20.pdf}{TangB20}~\cite{TangB20}, \href{../works/NovasH10.pdf}{NovasH10}~\cite{NovasH10}, \href{../works/Vilim02.pdf}{Vilim02}~\cite{Vilim02}, \href{../works/SimonisC95.pdf}{SimonisC95}~\cite{SimonisC95} & \href{../works/PrataAN23.pdf}{PrataAN23}~\cite{PrataAN23}, \href{../works/IsikYA23.pdf}{IsikYA23}~\cite{IsikYA23}, \href{../works/Adelgren2023.pdf}{Adelgren2023}~\cite{Adelgren2023}, \href{../works/YuraszeckMCCR23.pdf}{YuraszeckMCCR23}~\cite{YuraszeckMCCR23}, \href{../works/MullerMKP22.pdf}{MullerMKP22}~\cite{MullerMKP22}, \href{../works/SvancaraB22.pdf}{SvancaraB22}~\cite{SvancaraB22}, \href{../works/EmdeZD22.pdf}{EmdeZD22}~\cite{EmdeZD22}, \href{../works/LiFJZLL22.pdf}{LiFJZLL22}~\cite{LiFJZLL22}, \href{../works/ColT22.pdf}{ColT22}~\cite{ColT22}, \href{../works/AbreuN22.pdf}{AbreuN22}~\cite{AbreuN22}, \href{../works/GeitzGSSW22.pdf}{GeitzGSSW22}~\cite{GeitzGSSW22}, \href{../works/YunusogluY22.pdf}{YunusogluY22}~\cite{YunusogluY22}, \href{../works/OujanaAYB22.pdf}{OujanaAYB22}~\cite{OujanaAYB22}, \href{../works/LuoB22.pdf}{LuoB22}~\cite{LuoB22}, \href{../works/FanXG21.pdf}{FanXG21}~\cite{FanXG21}, \href{../works/ZhangYW21.pdf}{ZhangYW21}~\cite{ZhangYW21}, \href{../works/KlankeBYE21.pdf}{KlankeBYE21}~\cite{KlankeBYE21}, \href{../works/MengZRZL20.pdf}{MengZRZL20}~\cite{MengZRZL20}, \href{../works/Lunardi20.pdf}{Lunardi20}~\cite{Lunardi20}, \href{../works/CauwelaertDS20.pdf}{CauwelaertDS20}~\cite{CauwelaertDS20}, \href{../works/EscobetPQPRA19.pdf}{EscobetPQPRA19}~\cite{EscobetPQPRA19}, \href{../works/FahimiOQ18.pdf}{FahimiOQ18}~\cite{FahimiOQ18}, \href{../works/Ham18a.pdf}{Ham18a}~\cite{Ham18a}, \href{../works/Ham18.pdf}{Ham18}~\cite{Ham18}, \href{../works/LaborieRSV18.pdf}{LaborieRSV18}~\cite{LaborieRSV18}, \href{../works/Fahimi16.pdf}{Fahimi16}~\cite{Fahimi16}, \href{../works/CauwelaertDMS16.pdf}{CauwelaertDMS16}~\cite{CauwelaertDMS16}, \href{../works/Dejemeppe16.pdf}{Dejemeppe16}~\cite{Dejemeppe16}, \href{../works/Froger16.pdf}{Froger16}~\cite{Froger16}... (Total: 36)\\
Concepts & bi-objective & \href{../works/ZarandiASC20.pdf}{ZarandiASC20}~\cite{ZarandiASC20} & \href{../works/IsikYA23.pdf}{IsikYA23}~\cite{IsikYA23}, \href{../works/AbreuPNF23.pdf}{AbreuPNF23}~\cite{AbreuPNF23}, \href{../works/YunusogluY22.pdf}{YunusogluY22}~\cite{YunusogluY22}, \href{../works/HillTV21.pdf}{HillTV21}~\cite{HillTV21}, \href{../works/Lemos21.pdf}{Lemos21}~\cite{Lemos21}, \href{../works/NattafM20.pdf}{NattafM20}~\cite{NattafM20}, \href{../works/Dejemeppe16.pdf}{Dejemeppe16}~\cite{Dejemeppe16}, \href{../works/DejemeppeD14.pdf}{DejemeppeD14}~\cite{DejemeppeD14} & \href{../works/PrataAN23.pdf}{PrataAN23}~\cite{PrataAN23}, \href{../works/AfsarVPG23.pdf}{AfsarVPG23}~\cite{AfsarVPG23}, \href{../works/GurPAE23.pdf}{GurPAE23}~\cite{GurPAE23}, \href{../works/Mehdizadeh-Somarin23.pdf}{Mehdizadeh-Somarin23}~\cite{Mehdizadeh-Somarin23}, \href{../works/NaderiRR23.pdf}{NaderiRR23}~\cite{NaderiRR23}, \href{../works/abs-2305-19888.pdf}{abs-2305-19888}~\cite{abs-2305-19888}, \href{../works/GokPTGO23.pdf}{GokPTGO23}~\cite{GokPTGO23}, \href{../works/MullerMKP22.pdf}{MullerMKP22}~\cite{MullerMKP22}, \href{../works/PopovicCGNC22.pdf}{PopovicCGNC22}~\cite{PopovicCGNC22}, \href{../works/HeinzNVH22.pdf}{HeinzNVH22}~\cite{HeinzNVH22}, \href{../works/AbreuN22.pdf}{AbreuN22}~\cite{AbreuN22}, \href{../works/FarsiTM22.pdf}{FarsiTM22}~\cite{FarsiTM22}, \href{../works/WinterMMW22.pdf}{WinterMMW22}~\cite{WinterMMW22}, \href{../works/EmdeZD22.pdf}{EmdeZD22}~\cite{EmdeZD22}, \href{../works/Groleaz21.pdf}{Groleaz21}~\cite{Groleaz21}, \href{../works/VlkHT21.pdf}{VlkHT21}~\cite{VlkHT21}, \href{../works/Zahout21.pdf}{Zahout21}~\cite{Zahout21}, \href{../works/HamPK21.pdf}{HamPK21}~\cite{HamPK21}, \href{../works/HauderBRPA20.pdf}{HauderBRPA20}~\cite{HauderBRPA20}, \href{../works/GokGSTO20.pdf}{GokGSTO20}~\cite{GokGSTO20}, \href{../works/MejiaY20.pdf}{MejiaY20}~\cite{MejiaY20}, \href{../works/Lunardi20.pdf}{Lunardi20}~\cite{Lunardi20}, \href{../works/LunardiBLRV20.pdf}{LunardiBLRV20}~\cite{LunardiBLRV20}, \href{../works/RoshanaeiBAUB20.pdf}{RoshanaeiBAUB20}~\cite{RoshanaeiBAUB20}, \href{../works/WallaceY20.pdf}{WallaceY20}~\cite{WallaceY20}, \href{../works/MalapertN19.pdf}{MalapertN19}~\cite{MalapertN19}, \href{../works/abs-1902-09244.pdf}{abs-1902-09244}~\cite{abs-1902-09244}, \href{../works/HamC16.pdf}{HamC16}~\cite{HamC16}, \href{../works/Nattaf16.pdf}{Nattaf16}~\cite{Nattaf16}, \href{../works/BurtLPS15.pdf}{BurtLPS15}~\cite{BurtLPS15}\\
Concepts & bill of material &  & \href{../works/OrnekO16.pdf}{OrnekO16}~\cite{OrnekO16} & \href{../works/Simonis07.pdf}{Simonis07}~\cite{Simonis07}\\
Concepts & blocking constraint & \href{../works/AbreuNP23.pdf}{AbreuNP23}~\cite{AbreuNP23}, \href{../works/RiahiNS018.pdf}{RiahiNS018}~\cite{RiahiNS018} &  & \href{../works/IsikYA23.pdf}{IsikYA23}~\cite{IsikYA23}, \href{../works/LiFJZLL22.pdf}{LiFJZLL22}~\cite{LiFJZLL22}, \href{../works/MengZRZL20.pdf}{MengZRZL20}~\cite{MengZRZL20}, \href{../works/RodriguezS09.pdf}{RodriguezS09}~\cite{RodriguezS09}, \href{../works/Rodriguez07b.pdf}{Rodriguez07b}~\cite{Rodriguez07b}, \href{../works/Rodriguez07.pdf}{Rodriguez07}~\cite{Rodriguez07}\\
Concepts & breakdown & \href{../works/Groleaz21.pdf}{Groleaz21}~\cite{Groleaz21}, \href{../works/FanXG21.pdf}{FanXG21}~\cite{FanXG21}, \href{../works/ZarandiASC20.pdf}{ZarandiASC20}~\cite{ZarandiASC20}, \href{../works/LaborieRSV18.pdf}{LaborieRSV18}~\cite{LaborieRSV18}, \href{../works/ZhangW18.pdf}{ZhangW18}~\cite{ZhangW18}, \href{../works/Froger16.pdf}{Froger16}~\cite{Froger16}, \href{../works/BartakV15.pdf}{BartakV15}~\cite{BartakV15}, \href{../works/NovasH10.pdf}{NovasH10}~\cite{NovasH10}, \href{../works/BidotVLB09.pdf}{BidotVLB09}~\cite{BidotVLB09} & \href{../works/Lunardi20.pdf}{Lunardi20}~\cite{Lunardi20}, \href{../works/GombolayWS18.pdf}{GombolayWS18}~\cite{GombolayWS18}, \href{../works/RoshanaeiLAU17.pdf}{RoshanaeiLAU17}~\cite{RoshanaeiLAU17}, \href{../works/BajestaniB15.pdf}{BajestaniB15}~\cite{BajestaniB15}, \href{../works/ThiruvadyWGS14.pdf}{ThiruvadyWGS14}~\cite{ThiruvadyWGS14}, \href{../works/HarjunkoskiMBC14.pdf}{HarjunkoskiMBC14}~\cite{HarjunkoskiMBC14}, \href{../works/BajestaniB13.pdf}{BajestaniB13}~\cite{BajestaniB13}, \href{../works/BajestaniB11.pdf}{BajestaniB11}~\cite{BajestaniB11}, \href{../works/Elkhyari03.pdf}{Elkhyari03}~\cite{Elkhyari03}, \href{../works/MartinPY01.pdf}{MartinPY01}~\cite{MartinPY01} & \href{../works/Fatemi-AnarakiTFV23.pdf}{Fatemi-AnarakiTFV23}~\cite{Fatemi-AnarakiTFV23}, \href{../works/JuvinHL23.pdf}{JuvinHL23}~\cite{JuvinHL23}, \href{../works/PenzDN23.pdf}{PenzDN23}~\cite{PenzDN23}, \href{../works/IsikYA23.pdf}{IsikYA23}~\cite{IsikYA23}, \href{../works/SubulanC22.pdf}{SubulanC22}~\cite{SubulanC22}, \href{../works/MullerMKP22.pdf}{MullerMKP22}~\cite{MullerMKP22}, \href{../works/ColT22.pdf}{ColT22}~\cite{ColT22}, \href{../works/YunusogluY22.pdf}{YunusogluY22}~\cite{YunusogluY22}, \href{../works/KovacsTKSG21.pdf}{KovacsTKSG21}~\cite{KovacsTKSG21}, \href{../works/AbreuAPNM21.pdf}{AbreuAPNM21}~\cite{AbreuAPNM21}, \href{../works/Astrand21.pdf}{Astrand21}~\cite{Astrand21}, \href{../works/AstrandJZ20.pdf}{AstrandJZ20}~\cite{AstrandJZ20}, \href{../works/HauderBRPA20.pdf}{HauderBRPA20}~\cite{HauderBRPA20}, \href{../works/abs-1902-09244.pdf}{abs-1902-09244}~\cite{abs-1902-09244}, \href{../works/MalapertN19.pdf}{MalapertN19}~\cite{MalapertN19}, \href{../works/GedikKEK18.pdf}{GedikKEK18}~\cite{GedikKEK18}, \href{../works/CappartS17.pdf}{CappartS17}~\cite{CappartS17}, \href{../works/ZarandiKS16.pdf}{ZarandiKS16}~\cite{ZarandiKS16}, \href{../works/CireCH16.pdf}{CireCH16}~\cite{CireCH16}, \href{../works/OzturkTHO15.pdf}{OzturkTHO15}~\cite{OzturkTHO15}, \href{../works/LipovetzkyBPS14.pdf}{LipovetzkyBPS14}~\cite{LipovetzkyBPS14}, \href{../works/BegB13.pdf}{BegB13}~\cite{BegB13}, \href{../works/OzturkTHO13.pdf}{OzturkTHO13}~\cite{OzturkTHO13}, \href{../works/NovasH12.pdf}{NovasH12}~\cite{NovasH12}, \href{../works/BeckFW11.pdf}{BeckFW11}~\cite{BeckFW11}, \href{../works/KovacsK11.pdf}{KovacsK11}~\cite{KovacsK11}, \href{../works/ZeballosQH10.pdf}{ZeballosQH10}~\cite{ZeballosQH10}, \href{../works/Laborie09.pdf}{Laborie09}~\cite{Laborie09}, \href{../works/Malik08.pdf}{Malik08}~\cite{Malik08}... (Total: 34)\\
Concepts & buffer-capacity &  & \href{../works/SureshMOK06.pdf}{SureshMOK06}~\cite{SureshMOK06} & \href{../works/LiFJZLL22.pdf}{LiFJZLL22}~\cite{LiFJZLL22}, \href{../works/OujanaAYB22.pdf}{OujanaAYB22}~\cite{OujanaAYB22}, \href{../works/RiahiNS018.pdf}{RiahiNS018}~\cite{RiahiNS018}, \href{../works/BonfiettiLBM14.pdf}{BonfiettiLBM14}~\cite{BonfiettiLBM14}, \href{../works/NovasH14.pdf}{NovasH14}~\cite{NovasH14}, \href{../works/TerekhovTDB14.pdf}{TerekhovTDB14}~\cite{TerekhovTDB14}, \href{../works/ZeballosH05.pdf}{ZeballosH05}~\cite{ZeballosH05}\\
Concepts & cmax & \href{../works/Fatemi-AnarakiTFV23.pdf}{Fatemi-AnarakiTFV23}~\cite{Fatemi-AnarakiTFV23}, \href{../works/YuraszeckMCCR23.pdf}{YuraszeckMCCR23}~\cite{YuraszeckMCCR23}, \href{../works/KameugneFND23.pdf}{KameugneFND23}~\cite{KameugneFND23}, \href{../works/NaderiRR23.pdf}{NaderiRR23}~\cite{NaderiRR23}, \href{../works/ZhuSZW23.pdf}{ZhuSZW23}~\cite{ZhuSZW23}, \href{../works/JuvinHHL23.pdf}{JuvinHHL23}~\cite{JuvinHHL23}, \href{../works/AbreuNP23.pdf}{AbreuNP23}~\cite{AbreuNP23}, \href{../works/YuraszeckMC23.pdf}{YuraszeckMC23}~\cite{YuraszeckMC23}, \href{../works/abs-2305-19888.pdf}{abs-2305-19888}~\cite{abs-2305-19888}, \href{../works/IsikYA23.pdf}{IsikYA23}~\cite{IsikYA23}, \href{../works/FetgoD22.pdf}{FetgoD22}~\cite{FetgoD22}, \href{../works/EtminaniesfahaniGNMS22.pdf}{EtminaniesfahaniGNMS22}~\cite{EtminaniesfahaniGNMS22}, \href{../works/AbreuN22.pdf}{AbreuN22}~\cite{AbreuN22}, \href{../works/abs-2211-14492.pdf}{abs-2211-14492}~\cite{abs-2211-14492}, \href{../works/YunusogluY22.pdf}{YunusogluY22}~\cite{YunusogluY22}, \href{../works/JuvinHL22.pdf}{JuvinHL22}~\cite{JuvinHL22}, \href{../works/ZhangBB22.pdf}{ZhangBB22}~\cite{ZhangBB22}, \href{../works/ArmstrongGOS21.pdf}{ArmstrongGOS21}~\cite{ArmstrongGOS21}, \href{../works/Godet21a.pdf}{Godet21a}~\cite{Godet21a}, \href{../works/QinWSLS21.pdf}{QinWSLS21}~\cite{QinWSLS21}, \href{../works/Groleaz21.pdf}{Groleaz21}~\cite{Groleaz21}, \href{../works/AbohashimaEG21.pdf}{AbohashimaEG21}~\cite{AbohashimaEG21}, \href{../works/Polo-MejiaALB20.pdf}{Polo-MejiaALB20}~\cite{Polo-MejiaALB20}, \href{../works/MejiaY20.pdf}{MejiaY20}~\cite{MejiaY20}, \href{../works/MengZRZL20.pdf}{MengZRZL20}~\cite{MengZRZL20}, \href{../works/Lunardi20.pdf}{Lunardi20}~\cite{Lunardi20}, \href{../works/QinDCS20.pdf}{QinDCS20}~\cite{QinDCS20}, \href{../works/GodetLHS20.pdf}{GodetLHS20}~\cite{GodetLHS20}, \href{../works/YounespourAKE19.pdf}{YounespourAKE19}~\cite{YounespourAKE19}... (Total: 65) & \href{../works/Mehdizadeh-Somarin23.pdf}{Mehdizadeh-Somarin23}~\cite{Mehdizadeh-Somarin23}, \href{../works/MullerMKP22.pdf}{MullerMKP22}~\cite{MullerMKP22}, \href{../works/ArmstrongGOS22.pdf}{ArmstrongGOS22}~\cite{ArmstrongGOS22}, \href{../works/BoudreaultSLQ22.pdf}{BoudreaultSLQ22}~\cite{BoudreaultSLQ22}, \href{../works/AbreuAPNM21.pdf}{AbreuAPNM21}~\cite{AbreuAPNM21}, \href{../works/HamPK21.pdf}{HamPK21}~\cite{HamPK21}, \href{../works/ArkhipovBL19.pdf}{ArkhipovBL19}~\cite{ArkhipovBL19}, \href{../works/Novas19.pdf}{Novas19}~\cite{Novas19}, \href{../works/ParkUJR19.pdf}{ParkUJR19}~\cite{ParkUJR19}, \href{../works/ArbaouiY18.pdf}{ArbaouiY18}~\cite{ArbaouiY18}, \href{../works/GrimesH15.pdf}{GrimesH15}~\cite{GrimesH15}, \href{../works/WangMD15.pdf}{WangMD15}~\cite{WangMD15}, \href{../works/ZhouGL15.pdf}{ZhouGL15}~\cite{ZhouGL15}, \href{../works/MenciaSV13.pdf}{MenciaSV13}~\cite{MenciaSV13}, \href{../works/MenciaSV12.pdf}{MenciaSV12}~\cite{MenciaSV12}, \href{../works/ZhangLS12.pdf}{ZhangLS12}~\cite{ZhangLS12}, \href{../works/BeckFW11.pdf}{BeckFW11}~\cite{BeckFW11}, \href{../works/OzturkTHO10.pdf}{OzturkTHO10}~\cite{OzturkTHO10}, \href{../works/BartakSR10.pdf}{BartakSR10}~\cite{BartakSR10}, \href{../works/MoffittPP05.pdf}{MoffittPP05}~\cite{MoffittPP05}, \href{../works/Muscettola02.pdf}{Muscettola02}~\cite{Muscettola02}, \href{../works/SourdN00.pdf}{SourdN00}~\cite{SourdN00}, \href{../works/ArtiguesR00.pdf}{ArtiguesR00}~\cite{ArtiguesR00} & \href{../works/JuvinHL23.pdf}{JuvinHL23}~\cite{JuvinHL23}, \href{../works/Teppan22.pdf}{Teppan22}~\cite{Teppan22}, \href{../works/ZhangYW21.pdf}{ZhangYW21}~\cite{ZhangYW21}, \href{../works/HanenKP21.pdf}{HanenKP21}~\cite{HanenKP21}, \href{../works/HubnerGSV21.pdf}{HubnerGSV21}~\cite{HubnerGSV21}, \href{../works/ZarandiASC20.pdf}{ZarandiASC20}~\cite{ZarandiASC20}, \href{../works/GokgurHO18.pdf}{GokgurHO18}~\cite{GokgurHO18}, \href{../works/LiuCGM17.pdf}{LiuCGM17}~\cite{LiuCGM17}, \href{../works/BofillCSV17.pdf}{BofillCSV17}~\cite{BofillCSV17}, \href{../works/OrnekO16.pdf}{OrnekO16}~\cite{OrnekO16}, \href{../works/SialaAH15.pdf}{SialaAH15}~\cite{SialaAH15}, \href{../works/SchnellH15.pdf}{SchnellH15}~\cite{SchnellH15}, \href{../works/KoschB14.pdf}{KoschB14}~\cite{KoschB14}, \href{../works/LombardiMB13.pdf}{LombardiMB13}~\cite{LombardiMB13}, \href{../works/SchuttFSW13.pdf}{SchuttFSW13}~\cite{SchuttFSW13}, \href{../works/Letort13.pdf}{Letort13}~\cite{Letort13}, \href{../works/MalapertCGJLR13.pdf}{MalapertCGJLR13}~\cite{MalapertCGJLR13}, \href{../works/TerekhovDOB12.pdf}{TerekhovDOB12}~\cite{TerekhovDOB12}, \href{../works/GuSW12.pdf}{GuSW12}~\cite{GuSW12}, \href{../works/Schutt11.pdf}{Schutt11}~\cite{Schutt11}, \href{../works/abs-1009-0347.pdf}{abs-1009-0347}~\cite{abs-1009-0347}, \href{../works/LiessM08.pdf}{LiessM08}~\cite{LiessM08}, \href{../works/WatsonB08.pdf}{WatsonB08}~\cite{WatsonB08}, \href{../works/AkkerDH07.pdf}{AkkerDH07}~\cite{AkkerDH07}, \href{../works/KeriK07.pdf}{KeriK07}~\cite{KeriK07}, \href{../works/KhayatLR06.pdf}{KhayatLR06}~\cite{KhayatLR06}, \href{../works/Laborie03.pdf}{Laborie03}~\cite{Laborie03}, \href{../works/BaptisteP00.pdf}{BaptisteP00}~\cite{BaptisteP00}, \href{../works/FocacciLN00.pdf}{FocacciLN00}~\cite{FocacciLN00}\\
Concepts & completion-time & \href{../works/PrataAN23.pdf}{PrataAN23}~\cite{PrataAN23}, \href{../works/BonninMNE24.pdf}{BonninMNE24}~\cite{BonninMNE24}, \href{../works/AbreuNP23.pdf}{AbreuNP23}~\cite{AbreuNP23}, \href{../works/Mehdizadeh-Somarin23.pdf}{Mehdizadeh-Somarin23}~\cite{Mehdizadeh-Somarin23}, \href{../works/ZhuSZW23.pdf}{ZhuSZW23}~\cite{ZhuSZW23}, \href{../works/Fatemi-AnarakiTFV23.pdf}{Fatemi-AnarakiTFV23}~\cite{Fatemi-AnarakiTFV23}, \href{../works/AlfieriGPS23.pdf}{AlfieriGPS23}~\cite{AlfieriGPS23}, \href{../works/AbreuPNF23.pdf}{AbreuPNF23}~\cite{AbreuPNF23}, \href{../works/KameugneFND23.pdf}{KameugneFND23}~\cite{KameugneFND23}, \href{../works/JuvinHL23.pdf}{JuvinHL23}~\cite{JuvinHL23}, \href{../works/PenzDN23.pdf}{PenzDN23}~\cite{PenzDN23}, \href{../works/NaderiRR23.pdf}{NaderiRR23}~\cite{NaderiRR23}, \href{../works/EmdeZD22.pdf}{EmdeZD22}~\cite{EmdeZD22}, \href{../works/OuelletQ22.pdf}{OuelletQ22}~\cite{OuelletQ22}, \href{../works/FetgoD22.pdf}{FetgoD22}~\cite{FetgoD22}, \href{../works/YuraszeckMPV22.pdf}{YuraszeckMPV22}~\cite{YuraszeckMPV22}, \href{../works/JuvinHL22.pdf}{JuvinHL22}~\cite{JuvinHL22}, \href{../works/AbreuN22.pdf}{AbreuN22}~\cite{AbreuN22}, \href{../works/YunusogluY22.pdf}{YunusogluY22}~\cite{YunusogluY22}, \href{../works/SubulanC22.pdf}{SubulanC22}~\cite{SubulanC22}, \href{../works/NaderiBZ22.pdf}{NaderiBZ22}~\cite{NaderiBZ22}, \href{../works/KlankeBYE21.pdf}{KlankeBYE21}~\cite{KlankeBYE21}, \href{../works/Bedhief21.pdf}{Bedhief21}~\cite{Bedhief21}, \href{../works/Groleaz21.pdf}{Groleaz21}~\cite{Groleaz21}, \href{../works/Astrand21.pdf}{Astrand21}~\cite{Astrand21}, \href{../works/ArmstrongGOS21.pdf}{ArmstrongGOS21}~\cite{ArmstrongGOS21}, \href{../works/LunardiBLRV20.pdf}{LunardiBLRV20}~\cite{LunardiBLRV20}, \href{../works/QinDCS20.pdf}{QinDCS20}~\cite{QinDCS20}, \href{../works/CauwelaertDS20.pdf}{CauwelaertDS20}~\cite{CauwelaertDS20}... (Total: 92) & \href{../works/GokPTGO23.pdf}{GokPTGO23}~\cite{GokPTGO23}, \href{../works/AfsarVPG23.pdf}{AfsarVPG23}~\cite{AfsarVPG23}, \href{../works/CzerniachowskaWZ23.pdf}{CzerniachowskaWZ23}~\cite{CzerniachowskaWZ23}, \href{../works/abs-2305-19888.pdf}{abs-2305-19888}~\cite{abs-2305-19888}, \href{../works/LiFJZLL22.pdf}{LiFJZLL22}~\cite{LiFJZLL22}, \href{../works/ZhangBB22.pdf}{ZhangBB22}~\cite{ZhangBB22}, \href{../works/abs-2211-14492.pdf}{abs-2211-14492}~\cite{abs-2211-14492}, \href{../works/MullerMKP22.pdf}{MullerMKP22}~\cite{MullerMKP22}, \href{../works/ColT22.pdf}{ColT22}~\cite{ColT22}, \href{../works/Teppan22.pdf}{Teppan22}~\cite{Teppan22}, \href{../works/NaderiBZ22a.pdf}{NaderiBZ22a}~\cite{NaderiBZ22a}, \href{../works/TouatBT22.pdf}{TouatBT22}~\cite{TouatBT22}, \href{../works/OujanaAYB22.pdf}{OujanaAYB22}~\cite{OujanaAYB22}, \href{../works/HeinzNVH22.pdf}{HeinzNVH22}~\cite{HeinzNVH22}, \href{../works/FanXG21.pdf}{FanXG21}~\cite{FanXG21}, \href{../works/GeibingerMM21.pdf}{GeibingerMM21}~\cite{GeibingerMM21}, \href{../works/QinWSLS21.pdf}{QinWSLS21}~\cite{QinWSLS21}, \href{../works/AbreuAPNM21.pdf}{AbreuAPNM21}~\cite{AbreuAPNM21}, \href{../works/HanenKP21.pdf}{HanenKP21}~\cite{HanenKP21}, \href{../works/NattafM20.pdf}{NattafM20}~\cite{NattafM20}, \href{../works/Mercier-AubinGQ20.pdf}{Mercier-AubinGQ20}~\cite{Mercier-AubinGQ20}, \href{../works/Polo-MejiaALB20.pdf}{Polo-MejiaALB20}~\cite{Polo-MejiaALB20}, \href{../works/abs-1902-09244.pdf}{abs-1902-09244}~\cite{abs-1902-09244}, \href{../works/BogaerdtW19.pdf}{BogaerdtW19}~\cite{BogaerdtW19}, \href{../works/GeibingerMM19.pdf}{GeibingerMM19}~\cite{GeibingerMM19}, \href{../works/ParkUJR19.pdf}{ParkUJR19}~\cite{ParkUJR19}, \href{../works/YangSS19.pdf}{YangSS19}~\cite{YangSS19}, \href{../works/abs-1911-04766.pdf}{abs-1911-04766}~\cite{abs-1911-04766}, \href{../works/MalapertN19.pdf}{MalapertN19}~\cite{MalapertN19}... (Total: 62) & \href{../works/abs-2402-00459.pdf}{abs-2402-00459}~\cite{abs-2402-00459}, \href{../works/TasselGS23.pdf}{TasselGS23}~\cite{TasselGS23}, \href{../works/MontemanniD23a.pdf}{MontemanniD23a}~\cite{MontemanniD23a}, \href{../works/AkramNHRSA23.pdf}{AkramNHRSA23}~\cite{AkramNHRSA23}, \href{../works/IsikYA23.pdf}{IsikYA23}~\cite{IsikYA23}, \href{../works/JuvinHHL23.pdf}{JuvinHHL23}~\cite{JuvinHHL23}, \href{../works/Adelgren2023.pdf}{Adelgren2023}~\cite{Adelgren2023}, \href{../works/abs-2306-05747.pdf}{abs-2306-05747}~\cite{abs-2306-05747}, \href{../works/PerezGSL23.pdf}{PerezGSL23}~\cite{PerezGSL23}, \href{../works/FarsiTM22.pdf}{FarsiTM22}~\cite{FarsiTM22}, \href{../works/PopovicCGNC22.pdf}{PopovicCGNC22}~\cite{PopovicCGNC22}, \href{../works/CampeauG22.pdf}{CampeauG22}~\cite{CampeauG22}, \href{../works/PohlAK22.pdf}{PohlAK22}~\cite{PohlAK22}, \href{../works/GeitzGSSW22.pdf}{GeitzGSSW22}~\cite{GeitzGSSW22}, \href{../works/ZhangJZL22.pdf}{ZhangJZL22}~\cite{ZhangJZL22}, \href{../works/WinterMMW22.pdf}{WinterMMW22}~\cite{WinterMMW22}, \href{../works/ArmstrongGOS22.pdf}{ArmstrongGOS22}~\cite{ArmstrongGOS22}, \href{../works/HubnerGSV21.pdf}{HubnerGSV21}~\cite{HubnerGSV21}, \href{../works/Zahout21.pdf}{Zahout21}~\cite{Zahout21}, \href{../works/VlkHT21.pdf}{VlkHT21}~\cite{VlkHT21}, \href{../works/HamPK21.pdf}{HamPK21}~\cite{HamPK21}, \href{../works/Godet21a.pdf}{Godet21a}~\cite{Godet21a}, \href{../works/PandeyS21a.pdf}{PandeyS21a}~\cite{PandeyS21a}, \href{../works/WessenCS20.pdf}{WessenCS20}~\cite{WessenCS20}, \href{../works/MengZRZL20.pdf}{MengZRZL20}~\cite{MengZRZL20}, \href{../works/GodetLHS20.pdf}{GodetLHS20}~\cite{GodetLHS20}, \href{../works/SacramentoSP20.pdf}{SacramentoSP20}~\cite{SacramentoSP20}, \href{../works/ZouZ20.pdf}{ZouZ20}~\cite{ZouZ20}, \href{../works/AstrandJZ20.pdf}{AstrandJZ20}~\cite{AstrandJZ20}... (Total: 110)\\
Concepts & continuous-process & \href{../works/HarjunkoskiMBC14.pdf}{HarjunkoskiMBC14}~\cite{HarjunkoskiMBC14} &  & \href{../works/FarsiTM22.pdf}{FarsiTM22}~\cite{FarsiTM22}, \href{../works/Dejemeppe16.pdf}{Dejemeppe16}~\cite{Dejemeppe16}, \href{../works/GaySS14.pdf}{GaySS14}~\cite{GaySS14}, \href{../works/Bartak02.pdf}{Bartak02}~\cite{Bartak02}, \href{../works/SimonisC95.pdf}{SimonisC95}~\cite{SimonisC95}\\
Concepts & cyclic scheduling & \href{../works/OzturkTHO15.pdf}{OzturkTHO15}~\cite{OzturkTHO15}, \href{../works/BonfiettiLBM14.pdf}{BonfiettiLBM14}~\cite{BonfiettiLBM14}, \href{../works/HarjunkoskiMBC14.pdf}{HarjunkoskiMBC14}~\cite{HarjunkoskiMBC14}, \href{../works/BonfiettiLM13.pdf}{BonfiettiLM13}~\cite{BonfiettiLM13}, \href{../works/BonfiettiLBM12.pdf}{BonfiettiLBM12}~\cite{BonfiettiLBM12}, \href{../works/LombardiBMB11.pdf}{LombardiBMB11}~\cite{LombardiBMB11}, \href{../works/BonfiettiLBM11.pdf}{BonfiettiLBM11}~\cite{BonfiettiLBM11} & \href{../works/Fatemi-AnarakiTFV23.pdf}{Fatemi-AnarakiTFV23}~\cite{Fatemi-AnarakiTFV23}, \href{../works/BonfiettiZLM16.pdf}{BonfiettiZLM16}~\cite{BonfiettiZLM16}, \href{../works/BonfiettiM12.pdf}{BonfiettiM12}~\cite{BonfiettiM12}, \href{../works/KorbaaYG99.pdf}{KorbaaYG99}~\cite{KorbaaYG99}, \href{../works/RodosekW98.pdf}{RodosekW98}~\cite{RodosekW98} & \href{../works/YuraszeckMPV22.pdf}{YuraszeckMPV22}~\cite{YuraszeckMPV22}, \href{../works/WallaceY20.pdf}{WallaceY20}~\cite{WallaceY20}, \href{../works/MengZRZL20.pdf}{MengZRZL20}~\cite{MengZRZL20}, \href{../works/MusliuSS18.pdf}{MusliuSS18}~\cite{MusliuSS18}, \href{../works/OzturkTHO13.pdf}{OzturkTHO13}~\cite{OzturkTHO13}, \href{../works/OzturkTHO12.pdf}{OzturkTHO12}~\cite{OzturkTHO12}, \href{../works/Menana11.pdf}{Menana11}~\cite{Menana11}, \href{../works/Malik08.pdf}{Malik08}~\cite{Malik08}, \href{../works/Wallace06.pdf}{Wallace06}~\cite{Wallace06}, \href{../works/Mason01.pdf}{Mason01}~\cite{Mason01}\\
Concepts & distributed & \href{../works/PrataAN23.pdf}{PrataAN23}~\cite{PrataAN23}, \href{../works/GuoZ23.pdf}{GuoZ23}~\cite{GuoZ23}, \href{../works/NaderiRR23.pdf}{NaderiRR23}~\cite{NaderiRR23}, \href{../works/Zahout21.pdf}{Zahout21}~\cite{Zahout21}, \href{../works/ZarandiASC20.pdf}{ZarandiASC20}~\cite{ZarandiASC20}, \href{../works/MengZRZL20.pdf}{MengZRZL20}~\cite{MengZRZL20}, \href{../works/He0GLW18.pdf}{He0GLW18}~\cite{He0GLW18}, \href{../works/GombolayWS18.pdf}{GombolayWS18}~\cite{GombolayWS18}, \href{../works/TranPZLDB18.pdf}{TranPZLDB18}~\cite{TranPZLDB18}, \href{../works/RoshanaeiLAU17.pdf}{RoshanaeiLAU17}~\cite{RoshanaeiLAU17}, \href{../works/BridiLBBM16.pdf}{BridiLBBM16}~\cite{BridiLBBM16}, \href{../works/BridiBLMB16.pdf}{BridiBLMB16}~\cite{BridiBLMB16}, \href{../works/ZhouGL15.pdf}{ZhouGL15}~\cite{ZhouGL15}, \href{../works/TerekhovTDB14.pdf}{TerekhovTDB14}~\cite{TerekhovTDB14}, \href{../works/BonfiettiLM14.pdf}{BonfiettiLM14}~\cite{BonfiettiLM14}, \href{../works/BartakS11.pdf}{BartakS11}~\cite{BartakS11}, \href{../works/BartakSR10.pdf}{BartakSR10}~\cite{BartakSR10}, \href{../works/LombardiMRB10.pdf}{LombardiMRB10}~\cite{LombardiMRB10}, \href{../works/WuBB09.pdf}{WuBB09}~\cite{WuBB09}, \href{../works/RuggieroBBMA09.pdf}{RuggieroBBMA09}~\cite{RuggieroBBMA09}, \href{../works/BeckW07.pdf}{BeckW07}~\cite{BeckW07}, \href{../works/HoeveGSL07.pdf}{HoeveGSL07}~\cite{HoeveGSL07}, \href{../works/RossiTHP07.pdf}{RossiTHP07}~\cite{RossiTHP07}, \href{../works/SureshMOK06.pdf}{SureshMOK06}~\cite{SureshMOK06}, \href{../works/GomesHS06.pdf}{GomesHS06}~\cite{GomesHS06}, \href{../works/Geske05.pdf}{Geske05}~\cite{Geske05}, \href{../works/BeckW04.pdf}{BeckW04}~\cite{BeckW04}, \href{../works/Beck99.pdf}{Beck99}~\cite{Beck99}, \href{../works/LammaMM97.pdf}{LammaMM97}~\cite{LammaMM97} & \href{../works/AbreuPNF23.pdf}{AbreuPNF23}~\cite{AbreuPNF23}, \href{../works/ShaikhK23.pdf}{ShaikhK23}~\cite{ShaikhK23}, \href{../works/MarliereSPR23.pdf}{MarliereSPR23}~\cite{MarliereSPR23}, \href{../works/GokPTGO23.pdf}{GokPTGO23}~\cite{GokPTGO23}, \href{../works/AbreuNP23.pdf}{AbreuNP23}~\cite{AbreuNP23}, \href{../works/IsikYA23.pdf}{IsikYA23}~\cite{IsikYA23}, \href{../works/JungblutK22.pdf}{JungblutK22}~\cite{JungblutK22}, \href{../works/NaderiBZ22a.pdf}{NaderiBZ22a}~\cite{NaderiBZ22a}, \href{../works/OrnekOS20.pdf}{OrnekOS20}~\cite{OrnekOS20}, \href{../works/AbreuN22.pdf}{AbreuN22}~\cite{AbreuN22}, \href{../works/OujanaAYB22.pdf}{OujanaAYB22}~\cite{OujanaAYB22}, \href{../works/YuraszeckMPV22.pdf}{YuraszeckMPV22}~\cite{YuraszeckMPV22}, \href{../works/ElciOH22.pdf}{ElciOH22}~\cite{ElciOH22}, \href{../works/Godet21a.pdf}{Godet21a}~\cite{Godet21a}, \href{../works/AbreuAPNM21.pdf}{AbreuAPNM21}~\cite{AbreuAPNM21}, \href{../works/GokGSTO20.pdf}{GokGSTO20}~\cite{GokGSTO20}, \href{../works/MokhtarzadehTNF20.pdf}{MokhtarzadehTNF20}~\cite{MokhtarzadehTNF20}, \href{../works/RoshanaeiBAUB20.pdf}{RoshanaeiBAUB20}~\cite{RoshanaeiBAUB20}, \href{../works/ZouZ20.pdf}{ZouZ20}~\cite{ZouZ20}, \href{../works/Caballero19.pdf}{Caballero19}~\cite{Caballero19}, \href{../works/NishikawaSTT19.pdf}{NishikawaSTT19}~\cite{NishikawaSTT19}, \href{../works/BorghesiBLMB18.pdf}{BorghesiBLMB18}~\cite{BorghesiBLMB18}, \href{../works/ZhangW18.pdf}{ZhangW18}~\cite{ZhangW18}, \href{../works/GomesM17.pdf}{GomesM17}~\cite{GomesM17}, \href{../works/BlomPS16.pdf}{BlomPS16}~\cite{BlomPS16}, \href{../works/ZarandiKS16.pdf}{ZarandiKS16}~\cite{ZarandiKS16}, \href{../works/GrimesH15.pdf}{GrimesH15}~\cite{GrimesH15}, \href{../works/HarjunkoskiMBC14.pdf}{HarjunkoskiMBC14}~\cite{HarjunkoskiMBC14}, \href{../works/BlomBPS14.pdf}{BlomBPS14}~\cite{BlomBPS14}... (Total: 45) & \href{../works/ForbesHJST24.pdf}{ForbesHJST24}~\cite{ForbesHJST24}, \href{../works/Bit-Monnot23.pdf}{Bit-Monnot23}~\cite{Bit-Monnot23}, \href{../works/MontemanniD23.pdf}{MontemanniD23}~\cite{MontemanniD23}, \href{../works/Adelgren2023.pdf}{Adelgren2023}~\cite{Adelgren2023}, \href{../works/abs-2305-19888.pdf}{abs-2305-19888}~\cite{abs-2305-19888}, \href{../works/SquillaciPR23.pdf}{SquillaciPR23}~\cite{SquillaciPR23}, \href{../works/Fatemi-AnarakiTFV23.pdf}{Fatemi-AnarakiTFV23}~\cite{Fatemi-AnarakiTFV23}, \href{../works/YuraszeckMC23.pdf}{YuraszeckMC23}~\cite{YuraszeckMC23}, \href{../works/ZhuSZW23.pdf}{ZhuSZW23}~\cite{ZhuSZW23}, \href{../works/KimCMLLP23.pdf}{KimCMLLP23}~\cite{KimCMLLP23}, \href{../works/AlfieriGPS23.pdf}{AlfieriGPS23}~\cite{AlfieriGPS23}, \href{../works/GurPAE23.pdf}{GurPAE23}~\cite{GurPAE23}, \href{../works/JuvinHL23a.pdf}{JuvinHL23a}~\cite{JuvinHL23a}, \href{../works/AkramNHRSA23.pdf}{AkramNHRSA23}~\cite{AkramNHRSA23}, \href{../works/abs-2211-14492.pdf}{abs-2211-14492}~\cite{abs-2211-14492}, \href{../works/EmdeZD22.pdf}{EmdeZD22}~\cite{EmdeZD22}, \href{../works/NaderiBZ22.pdf}{NaderiBZ22}~\cite{NaderiBZ22}, \href{../works/TouatBT22.pdf}{TouatBT22}~\cite{TouatBT22}, \href{../works/Teppan22.pdf}{Teppan22}~\cite{Teppan22}, \href{../works/BoudreaultSLQ22.pdf}{BoudreaultSLQ22}~\cite{BoudreaultSLQ22}, \href{../works/ColT22.pdf}{ColT22}~\cite{ColT22}, \href{../works/LiFJZLL22.pdf}{LiFJZLL22}~\cite{LiFJZLL22}, \href{../works/FarsiTM22.pdf}{FarsiTM22}~\cite{FarsiTM22}, \href{../works/WinterMMW22.pdf}{WinterMMW22}~\cite{WinterMMW22}, \href{../works/ZhangBB22.pdf}{ZhangBB22}~\cite{ZhangBB22}, \href{../works/HeinzNVH22.pdf}{HeinzNVH22}~\cite{HeinzNVH22}, \href{../works/JuvinHL22.pdf}{JuvinHL22}~\cite{JuvinHL22}, \href{../works/Astrand21.pdf}{Astrand21}~\cite{Astrand21}, \href{../works/FanXG21.pdf}{FanXG21}~\cite{FanXG21}... (Total: 137)\\
Concepts & due-date & \href{../works/AfsarVPG23.pdf}{AfsarVPG23}~\cite{AfsarVPG23}, \href{../works/OujanaAYB22.pdf}{OujanaAYB22}~\cite{OujanaAYB22}, \href{../works/ColT22.pdf}{ColT22}~\cite{ColT22}, \href{../works/NaderiBZ22.pdf}{NaderiBZ22}~\cite{NaderiBZ22}, \href{../works/AntuoriHHEN21.pdf}{AntuoriHHEN21}~\cite{AntuoriHHEN21}, \href{../works/FanXG21.pdf}{FanXG21}~\cite{FanXG21}, \href{../works/Groleaz21.pdf}{Groleaz21}~\cite{Groleaz21}, \href{../works/AntuoriHHEN20.pdf}{AntuoriHHEN20}~\cite{AntuoriHHEN20}, \href{../works/ZarandiASC20.pdf}{ZarandiASC20}~\cite{ZarandiASC20}, \href{../works/TangB20.pdf}{TangB20}~\cite{TangB20}, \href{../works/HauderBRPA20.pdf}{HauderBRPA20}~\cite{HauderBRPA20}, \href{../works/Mercier-AubinGQ20.pdf}{Mercier-AubinGQ20}~\cite{Mercier-AubinGQ20}, \href{../works/Lunardi20.pdf}{Lunardi20}~\cite{Lunardi20}, \href{../works/AntunesABD20.pdf}{AntunesABD20}~\cite{AntunesABD20}, \href{../works/HoundjiSW19.pdf}{HoundjiSW19}~\cite{HoundjiSW19}, \href{../works/Novas19.pdf}{Novas19}~\cite{Novas19}, \href{../works/abs-1911-04766.pdf}{abs-1911-04766}~\cite{abs-1911-04766}, \href{../works/abs-1902-09244.pdf}{abs-1902-09244}~\cite{abs-1902-09244}, \href{../works/GoldwaserS18.pdf}{GoldwaserS18}~\cite{GoldwaserS18}, \href{../works/Tesch18.pdf}{Tesch18}~\cite{Tesch18}, \href{../works/GoldwaserS17.pdf}{GoldwaserS17}~\cite{GoldwaserS17}, \href{../works/Fahimi16.pdf}{Fahimi16}~\cite{Fahimi16}, \href{../works/NovaraNH16.pdf}{NovaraNH16}~\cite{NovaraNH16}, \href{../works/Dejemeppe16.pdf}{Dejemeppe16}~\cite{Dejemeppe16}, \href{../works/BajestaniB15.pdf}{BajestaniB15}~\cite{BajestaniB15}, \href{../works/DoulabiRP14.pdf}{DoulabiRP14}~\cite{DoulabiRP14}, \href{../works/HarjunkoskiMBC14.pdf}{HarjunkoskiMBC14}~\cite{HarjunkoskiMBC14}, \href{../works/KoschB14.pdf}{KoschB14}~\cite{KoschB14}, \href{../works/HoundjiSWD14.pdf}{HoundjiSWD14}~\cite{HoundjiSWD14}... (Total: 58) & \href{../works/PrataAN23.pdf}{PrataAN23}~\cite{PrataAN23}, \href{../works/IsikYA23.pdf}{IsikYA23}~\cite{IsikYA23}, \href{../works/LacknerMMWW23.pdf}{LacknerMMWW23}~\cite{LacknerMMWW23}, \href{../works/NaderiRR23.pdf}{NaderiRR23}~\cite{NaderiRR23}, \href{../works/YunusogluY22.pdf}{YunusogluY22}~\cite{YunusogluY22}, \href{../works/abs-2211-14492.pdf}{abs-2211-14492}~\cite{abs-2211-14492}, \href{../works/WinterMMW22.pdf}{WinterMMW22}~\cite{WinterMMW22}, \href{../works/Godet21a.pdf}{Godet21a}~\cite{Godet21a}, \href{../works/LacknerMMWW21.pdf}{LacknerMMWW21}~\cite{LacknerMMWW21}, \href{../works/GeibingerMM21.pdf}{GeibingerMM21}~\cite{GeibingerMM21}, \href{../works/GroleazNS20a.pdf}{GroleazNS20a}~\cite{GroleazNS20a}, \href{../works/GeibingerMM19.pdf}{GeibingerMM19}~\cite{GeibingerMM19}, \href{../works/AntunesABD18.pdf}{AntunesABD18}~\cite{AntunesABD18}, \href{../works/FahimiOQ18.pdf}{FahimiOQ18}~\cite{FahimiOQ18}, \href{../works/ZarandiKS16.pdf}{ZarandiKS16}~\cite{ZarandiKS16}, \href{../works/CatusseCBL16.pdf}{CatusseCBL16}~\cite{CatusseCBL16}, \href{../works/GrimesH15.pdf}{GrimesH15}~\cite{GrimesH15}, \href{../works/GrimesIOS14.pdf}{GrimesIOS14}~\cite{GrimesIOS14}, \href{../works/HeinzSB13.pdf}{HeinzSB13}~\cite{HeinzSB13}, \href{../works/CobanH11.pdf}{CobanH11}~\cite{CobanH11}, \href{../works/GrimesH11.pdf}{GrimesH11}~\cite{GrimesH11}, \href{../works/Malapert11.pdf}{Malapert11}~\cite{Malapert11}, \href{../works/LombardiM10a.pdf}{LombardiM10a}~\cite{LombardiM10a}, \href{../works/Lombardi10.pdf}{Lombardi10}~\cite{Lombardi10}, \href{../works/MakMS10.pdf}{MakMS10}~\cite{MakMS10}, \href{../works/SchuttW10.pdf}{SchuttW10}~\cite{SchuttW10}, \href{../works/Davenport10.pdf}{Davenport10}~\cite{Davenport10}, \href{../works/ThiruvadyBME09.pdf}{ThiruvadyBME09}~\cite{ThiruvadyBME09}, \href{../works/abs-0907-0939.pdf}{abs-0907-0939}~\cite{abs-0907-0939}... (Total: 45) & \href{../works/abs-2402-00459.pdf}{abs-2402-00459}~\cite{abs-2402-00459}, \href{../works/AbreuPNF23.pdf}{AbreuPNF23}~\cite{AbreuPNF23}, \href{../works/YuraszeckMC23.pdf}{YuraszeckMC23}~\cite{YuraszeckMC23}, \href{../works/JuvinHHL23.pdf}{JuvinHHL23}~\cite{JuvinHHL23}, \href{../works/KimCMLLP23.pdf}{KimCMLLP23}~\cite{KimCMLLP23}, \href{../works/TouatBT22.pdf}{TouatBT22}~\cite{TouatBT22}, \href{../works/YuraszeckMPV22.pdf}{YuraszeckMPV22}~\cite{YuraszeckMPV22}, \href{../works/ElciOH22.pdf}{ElciOH22}~\cite{ElciOH22}, \href{../works/ZhangJZL22.pdf}{ZhangJZL22}~\cite{ZhangJZL22}, \href{../works/SubulanC22.pdf}{SubulanC22}~\cite{SubulanC22}, \href{../works/MullerMKP22.pdf}{MullerMKP22}~\cite{MullerMKP22}, \href{../works/Astrand21.pdf}{Astrand21}~\cite{Astrand21}, \href{../works/HubnerGSV21.pdf}{HubnerGSV21}~\cite{HubnerGSV21}, \href{../works/VlkHT21.pdf}{VlkHT21}~\cite{VlkHT21}, \href{../works/KlankeBYE21.pdf}{KlankeBYE21}~\cite{KlankeBYE21}, \href{../works/Bedhief21.pdf}{Bedhief21}~\cite{Bedhief21}, \href{../works/KovacsTKSG21.pdf}{KovacsTKSG21}~\cite{KovacsTKSG21}, \href{../works/Zahout21.pdf}{Zahout21}~\cite{Zahout21}, \href{../works/HanenKP21.pdf}{HanenKP21}~\cite{HanenKP21}, \href{../works/MejiaY20.pdf}{MejiaY20}~\cite{MejiaY20}, \href{../works/Polo-MejiaALB20.pdf}{Polo-MejiaALB20}~\cite{Polo-MejiaALB20}, \href{../works/GroleazNS20.pdf}{GroleazNS20}~\cite{GroleazNS20}, \href{../works/LunardiBLRV20.pdf}{LunardiBLRV20}~\cite{LunardiBLRV20}, \href{../works/AstrandJZ20.pdf}{AstrandJZ20}~\cite{AstrandJZ20}, \href{../works/Hooker19.pdf}{Hooker19}~\cite{Hooker19}, \href{../works/ParkUJR19.pdf}{ParkUJR19}~\cite{ParkUJR19}, \href{../works/EscobetPQPRA19.pdf}{EscobetPQPRA19}~\cite{EscobetPQPRA19}, \href{../works/GokgurHO18.pdf}{GokgurHO18}~\cite{GokgurHO18}, \href{../works/GedikKEK18.pdf}{GedikKEK18}~\cite{GedikKEK18}... (Total: 86)\\
Concepts & earliness & \href{../works/PrataAN23.pdf}{PrataAN23}~\cite{PrataAN23}, \href{../works/KimCMLLP23.pdf}{KimCMLLP23}~\cite{KimCMLLP23}, \href{../works/PohlAK22.pdf}{PohlAK22}~\cite{PohlAK22}, \href{../works/TouatBT22.pdf}{TouatBT22}~\cite{TouatBT22}, \href{../works/Groleaz21.pdf}{Groleaz21}~\cite{Groleaz21}, \href{../works/ZarandiASC20.pdf}{ZarandiASC20}~\cite{ZarandiASC20}, \href{../works/HauderBRPA20.pdf}{HauderBRPA20}~\cite{HauderBRPA20}, \href{../works/abs-1902-09244.pdf}{abs-1902-09244}~\cite{abs-1902-09244}, \href{../works/LaborieRSV18.pdf}{LaborieRSV18}~\cite{LaborieRSV18}, \href{../works/ZarandiKS16.pdf}{ZarandiKS16}~\cite{ZarandiKS16}, \href{../works/Dejemeppe16.pdf}{Dejemeppe16}~\cite{Dejemeppe16}, \href{../works/GrimesH15.pdf}{GrimesH15}~\cite{GrimesH15}, \href{../works/LombardiM12.pdf}{LombardiM12}~\cite{LombardiM12}, \href{../works/KelbelH11.pdf}{KelbelH11}~\cite{KelbelH11}, \href{../works/GrimesH11.pdf}{GrimesH11}~\cite{GrimesH11}, \href{../works/MonetteDH09.pdf}{MonetteDH09}~\cite{MonetteDH09}, \href{../works/Laborie09.pdf}{Laborie09}~\cite{Laborie09}, \href{../works/KeriK07.pdf}{KeriK07}~\cite{KeriK07}, \href{../works/BeckR03.pdf}{BeckR03}~\cite{BeckR03}, \href{../works/DannaP03.pdf}{DannaP03}~\cite{DannaP03} & \href{../works/FarsiTM22.pdf}{FarsiTM22}~\cite{FarsiTM22}, \href{../works/AntunesABD20.pdf}{AntunesABD20}~\cite{AntunesABD20}, \href{../works/MengZRZL20.pdf}{MengZRZL20}~\cite{MengZRZL20}, \href{../works/TerekhovDOB12.pdf}{TerekhovDOB12}~\cite{TerekhovDOB12}, \href{../works/KovacsB11.pdf}{KovacsB11}~\cite{KovacsB11}, \href{../works/Davenport10.pdf}{Davenport10}~\cite{Davenport10}, \href{../works/Baptiste02.pdf}{Baptiste02}~\cite{Baptiste02} & \href{../works/abs-2402-00459.pdf}{abs-2402-00459}~\cite{abs-2402-00459}, \href{../works/NaderiRR23.pdf}{NaderiRR23}~\cite{NaderiRR23}, \href{../works/AbreuNP23.pdf}{AbreuNP23}~\cite{AbreuNP23}, \href{../works/PenzDN23.pdf}{PenzDN23}~\cite{PenzDN23}, \href{../works/AlfieriGPS23.pdf}{AlfieriGPS23}~\cite{AlfieriGPS23}, \href{../works/LacknerMMWW23.pdf}{LacknerMMWW23}~\cite{LacknerMMWW23}, \href{../works/AbreuPNF23.pdf}{AbreuPNF23}~\cite{AbreuPNF23}, \href{../works/IsikYA23.pdf}{IsikYA23}~\cite{IsikYA23}, \href{../works/EtminaniesfahaniGNMS22.pdf}{EtminaniesfahaniGNMS22}~\cite{EtminaniesfahaniGNMS22}, \href{../works/YunusogluY22.pdf}{YunusogluY22}~\cite{YunusogluY22}, \href{../works/LacknerMMWW21.pdf}{LacknerMMWW21}~\cite{LacknerMMWW21}, \href{../works/FanXG21.pdf}{FanXG21}~\cite{FanXG21}, \href{../works/Polo-MejiaALB20.pdf}{Polo-MejiaALB20}~\cite{Polo-MejiaALB20}, \href{../works/Mercier-AubinGQ20.pdf}{Mercier-AubinGQ20}~\cite{Mercier-AubinGQ20}, \href{../works/ColT19.pdf}{ColT19}~\cite{ColT19}, \href{../works/AntunesABD18.pdf}{AntunesABD18}~\cite{AntunesABD18}, \href{../works/ZhangW18.pdf}{ZhangW18}~\cite{ZhangW18}, \href{../works/German18.pdf}{German18}~\cite{German18}, \href{../works/GokgurHO18.pdf}{GokgurHO18}~\cite{GokgurHO18}, \href{../works/KuB16.pdf}{KuB16}~\cite{KuB16}, \href{../works/NovaraNH16.pdf}{NovaraNH16}~\cite{NovaraNH16}, \href{../works/OrnekO16.pdf}{OrnekO16}~\cite{OrnekO16}, \href{../works/Siala15a.pdf}{Siala15a}~\cite{Siala15a}, \href{../works/VilimLS15.pdf}{VilimLS15}~\cite{VilimLS15}, \href{../works/LimBTBB15.pdf}{LimBTBB15}~\cite{LimBTBB15}, \href{../works/Siala15.pdf}{Siala15}~\cite{Siala15}, \href{../works/SialaAH15.pdf}{SialaAH15}~\cite{SialaAH15}, \href{../works/HarjunkoskiMBC14.pdf}{HarjunkoskiMBC14}~\cite{HarjunkoskiMBC14}, \href{../works/BajestaniB13.pdf}{BajestaniB13}~\cite{BajestaniB13}... (Total: 46)\\
Concepts & energy efficiency & \href{../works/PrataAN23.pdf}{PrataAN23}~\cite{PrataAN23}, \href{../works/PandeyS21a.pdf}{PandeyS21a}~\cite{PandeyS21a}, \href{../works/RuggieroBBMA09.pdf}{RuggieroBBMA09}~\cite{RuggieroBBMA09} & \href{../works/MarliereSPR23.pdf}{MarliereSPR23}~\cite{MarliereSPR23}, \href{../works/Zahout21.pdf}{Zahout21}~\cite{Zahout21}, \href{../works/BenediktMH20.pdf}{BenediktMH20}~\cite{BenediktMH20}, \href{../works/BridiBLMB16.pdf}{BridiBLMB16}~\cite{BridiBLMB16}, \href{../works/Lombardi10.pdf}{Lombardi10}~\cite{Lombardi10} & \href{../works/IsikYA23.pdf}{IsikYA23}~\cite{IsikYA23}, \href{../works/AbreuNP23.pdf}{AbreuNP23}~\cite{AbreuNP23}, \href{../works/abs-2211-14492.pdf}{abs-2211-14492}~\cite{abs-2211-14492}, \href{../works/Lemos21.pdf}{Lemos21}~\cite{Lemos21}, \href{../works/MengZRZL20.pdf}{MengZRZL20}~\cite{MengZRZL20}, \href{../works/ZarandiASC20.pdf}{ZarandiASC20}~\cite{ZarandiASC20}, \href{../works/TranPZLDB18.pdf}{TranPZLDB18}~\cite{TranPZLDB18}, \href{../works/NattafAL17.pdf}{NattafAL17}~\cite{NattafAL17}, \href{../works/Dejemeppe16.pdf}{Dejemeppe16}~\cite{Dejemeppe16}, \href{../works/LombardiMB13.pdf}{LombardiMB13}~\cite{LombardiMB13}, \href{../works/LombardiM12.pdf}{LombardiM12}~\cite{LombardiM12}, \href{../works/BeniniLMR11.pdf}{BeniniLMR11}~\cite{BeniniLMR11}\\
Concepts & flow-shop & \href{../works/BonninMNE24.pdf}{BonninMNE24}~\cite{BonninMNE24}, \href{../works/PrataAN23.pdf}{PrataAN23}~\cite{PrataAN23}, \href{../works/NaderiRR23.pdf}{NaderiRR23}~\cite{NaderiRR23}, \href{../works/AlfieriGPS23.pdf}{AlfieriGPS23}~\cite{AlfieriGPS23}, \href{../works/IsikYA23.pdf}{IsikYA23}~\cite{IsikYA23}, \href{../works/AbreuPNF23.pdf}{AbreuPNF23}~\cite{AbreuPNF23}, \href{../works/AbreuNP23.pdf}{AbreuNP23}~\cite{AbreuNP23}, \href{../works/CzerniachowskaWZ23.pdf}{CzerniachowskaWZ23}~\cite{CzerniachowskaWZ23}, \href{../works/JuvinHL23.pdf}{JuvinHL23}~\cite{JuvinHL23}, \href{../works/ArmstrongGOS22.pdf}{ArmstrongGOS22}~\cite{ArmstrongGOS22}, \href{../works/AbreuN22.pdf}{AbreuN22}~\cite{AbreuN22}, \href{../works/LiFJZLL22.pdf}{LiFJZLL22}~\cite{LiFJZLL22}, \href{../works/OujanaAYB22.pdf}{OujanaAYB22}~\cite{OujanaAYB22}, \href{../works/ColT22.pdf}{ColT22}~\cite{ColT22}, \href{../works/ZhangJZL22.pdf}{ZhangJZL22}~\cite{ZhangJZL22}, \href{../works/Astrand21.pdf}{Astrand21}~\cite{Astrand21}, \href{../works/QinWSLS21.pdf}{QinWSLS21}~\cite{QinWSLS21}, \href{../works/ArmstrongGOS21.pdf}{ArmstrongGOS21}~\cite{ArmstrongGOS21}, \href{../works/Bedhief21.pdf}{Bedhief21}~\cite{Bedhief21}, \href{../works/Groleaz21.pdf}{Groleaz21}~\cite{Groleaz21}, \href{../works/AbreuAPNM21.pdf}{AbreuAPNM21}~\cite{AbreuAPNM21}, \href{../works/MengZRZL20.pdf}{MengZRZL20}~\cite{MengZRZL20}, \href{../works/AstrandJZ20.pdf}{AstrandJZ20}~\cite{AstrandJZ20}, \href{../works/ZarandiASC20.pdf}{ZarandiASC20}~\cite{ZarandiASC20}, \href{../works/Lunardi20.pdf}{Lunardi20}~\cite{Lunardi20}, \href{../works/Novas19.pdf}{Novas19}~\cite{Novas19}, \href{../works/ParkUJR19.pdf}{ParkUJR19}~\cite{ParkUJR19}, \href{../works/ZhangW18.pdf}{ZhangW18}~\cite{ZhangW18}, \href{../works/ZhouGL15.pdf}{ZhouGL15}~\cite{ZhouGL15}... (Total: 38) & \href{../works/JuvinHL23a.pdf}{JuvinHL23a}~\cite{JuvinHL23a}, \href{../works/Mehdizadeh-Somarin23.pdf}{Mehdizadeh-Somarin23}~\cite{Mehdizadeh-Somarin23}, \href{../works/NaderiBZ22.pdf}{NaderiBZ22}~\cite{NaderiBZ22}, \href{../works/YuraszeckMPV22.pdf}{YuraszeckMPV22}~\cite{YuraszeckMPV22}, \href{../works/JuvinHL22.pdf}{JuvinHL22}~\cite{JuvinHL22}, \href{../works/KoehlerBFFHPSSS21.pdf}{KoehlerBFFHPSSS21}~\cite{KoehlerBFFHPSSS21}, \href{../works/Godet21a.pdf}{Godet21a}~\cite{Godet21a}, \href{../works/FanXG21.pdf}{FanXG21}~\cite{FanXG21}, \href{../works/TangB20.pdf}{TangB20}~\cite{TangB20}, \href{../works/HauderBRPA20.pdf}{HauderBRPA20}~\cite{HauderBRPA20}, \href{../works/abs-1902-09244.pdf}{abs-1902-09244}~\cite{abs-1902-09244}, \href{../works/GombolayWS18.pdf}{GombolayWS18}~\cite{GombolayWS18}, \href{../works/LaborieRSV18.pdf}{LaborieRSV18}~\cite{LaborieRSV18}, \href{../works/Fahimi16.pdf}{Fahimi16}~\cite{Fahimi16}, \href{../works/Dejemeppe16.pdf}{Dejemeppe16}~\cite{Dejemeppe16}, \href{../works/GuyonLPR12.pdf}{GuyonLPR12}~\cite{GuyonLPR12}, \href{../works/GrimesH11.pdf}{GrimesH11}~\cite{GrimesH11}, \href{../works/KovacsB11.pdf}{KovacsB11}~\cite{KovacsB11}, \href{../works/BartakSR10.pdf}{BartakSR10}~\cite{BartakSR10}, \href{../works/JainM99.pdf}{JainM99}~\cite{JainM99}, \href{../works/AggounB93.pdf}{AggounB93}~\cite{AggounB93} & \href{../works/TasselGS23.pdf}{TasselGS23}~\cite{TasselGS23}, \href{../works/YuraszeckMCCR23.pdf}{YuraszeckMCCR23}~\cite{YuraszeckMCCR23}, \href{../works/abs-2305-19888.pdf}{abs-2305-19888}~\cite{abs-2305-19888}, \href{../works/JuvinHHL23.pdf}{JuvinHHL23}~\cite{JuvinHHL23}, \href{../works/AfsarVPG23.pdf}{AfsarVPG23}~\cite{AfsarVPG23}, \href{../works/AalianPG23.pdf}{AalianPG23}~\cite{AalianPG23}, \href{../works/abs-2306-05747.pdf}{abs-2306-05747}~\cite{abs-2306-05747}, \href{../works/abs-2211-14492.pdf}{abs-2211-14492}~\cite{abs-2211-14492}, \href{../works/TouatBT22.pdf}{TouatBT22}~\cite{TouatBT22}, \href{../works/Teppan22.pdf}{Teppan22}~\cite{Teppan22}, \href{../works/NaderiBZ22a.pdf}{NaderiBZ22a}~\cite{NaderiBZ22a}, \href{../works/HeinzNVH22.pdf}{HeinzNVH22}~\cite{HeinzNVH22}, \href{../works/HamPK21.pdf}{HamPK21}~\cite{HamPK21}, \href{../works/LacknerMMWW21.pdf}{LacknerMMWW21}~\cite{LacknerMMWW21}, \href{../works/HillTV21.pdf}{HillTV21}~\cite{HillTV21}, \href{../works/Zahout21.pdf}{Zahout21}~\cite{Zahout21}, \href{../works/abs-2102-08778.pdf}{abs-2102-08778}~\cite{abs-2102-08778}, \href{../works/KovacsTKSG21.pdf}{KovacsTKSG21}~\cite{KovacsTKSG21}, \href{../works/PandeyS21a.pdf}{PandeyS21a}~\cite{PandeyS21a}, \href{../works/WallaceY20.pdf}{WallaceY20}~\cite{WallaceY20}, \href{../works/LunardiBLRV20.pdf}{LunardiBLRV20}~\cite{LunardiBLRV20}, \href{../works/SacramentoSP20.pdf}{SacramentoSP20}~\cite{SacramentoSP20}, \href{../works/WikarekS19.pdf}{WikarekS19}~\cite{WikarekS19}, \href{../works/TanT18.pdf}{TanT18}~\cite{TanT18}, \href{../works/RiahiNS018.pdf}{RiahiNS018}~\cite{RiahiNS018}, \href{../works/GokgurHO18.pdf}{GokgurHO18}~\cite{GokgurHO18}, \href{../works/GoldwaserS18.pdf}{GoldwaserS18}~\cite{GoldwaserS18}, \href{../works/HookerH17.pdf}{HookerH17}~\cite{HookerH17}, \href{../works/Nattaf16.pdf}{Nattaf16}~\cite{Nattaf16}... (Total: 63)\\
Concepts & flow-time & \href{../works/BonninMNE24.pdf}{BonninMNE24}~\cite{BonninMNE24}, \href{../works/PenzDN23.pdf}{PenzDN23}~\cite{PenzDN23}, \href{../works/EmdeZD22.pdf}{EmdeZD22}~\cite{EmdeZD22}, \href{../works/YuraszeckMPV22.pdf}{YuraszeckMPV22}~\cite{YuraszeckMPV22}, \href{../works/FanXG21.pdf}{FanXG21}~\cite{FanXG21}, \href{../works/NattafM20.pdf}{NattafM20}~\cite{NattafM20}, \href{../works/ZarandiASC20.pdf}{ZarandiASC20}~\cite{ZarandiASC20}, \href{../works/MalapertN19.pdf}{MalapertN19}~\cite{MalapertN19}, \href{../works/ZhangW18.pdf}{ZhangW18}~\cite{ZhangW18}, \href{../works/TerekhovTDB14.pdf}{TerekhovTDB14}~\cite{TerekhovTDB14}, \href{../works/TranTDB13.pdf}{TranTDB13}~\cite{TranTDB13}, \href{../works/WuBB09.pdf}{WuBB09}~\cite{WuBB09}, \href{../works/Baptiste02.pdf}{Baptiste02}~\cite{Baptiste02} & \href{../works/PrataAN23.pdf}{PrataAN23}~\cite{PrataAN23}, \href{../works/AlfieriGPS23.pdf}{AlfieriGPS23}~\cite{AlfieriGPS23}, \href{../works/YunusogluY22.pdf}{YunusogluY22}~\cite{YunusogluY22}, \href{../works/Malapert11.pdf}{Malapert11}~\cite{Malapert11}, \href{../works/BeckW07.pdf}{BeckW07}~\cite{BeckW07} & \href{../works/YuraszeckMCCR23.pdf}{YuraszeckMCCR23}~\cite{YuraszeckMCCR23}, \href{../works/TasselGS23.pdf}{TasselGS23}~\cite{TasselGS23}, \href{../works/abs-2306-05747.pdf}{abs-2306-05747}~\cite{abs-2306-05747}, \href{../works/YuraszeckMC23.pdf}{YuraszeckMC23}~\cite{YuraszeckMC23}, \href{../works/LiFJZLL22.pdf}{LiFJZLL22}~\cite{LiFJZLL22}, \href{../works/AbreuN22.pdf}{AbreuN22}~\cite{AbreuN22}, \href{../works/KoehlerBFFHPSSS21.pdf}{KoehlerBFFHPSSS21}~\cite{KoehlerBFFHPSSS21}, \href{../works/MengZRZL20.pdf}{MengZRZL20}~\cite{MengZRZL20}, \href{../works/Novas19.pdf}{Novas19}~\cite{Novas19}, \href{../works/ParkUJR19.pdf}{ParkUJR19}~\cite{ParkUJR19}, \href{../works/BajestaniB15.pdf}{BajestaniB15}~\cite{BajestaniB15}, \href{../works/MenciaSV13.pdf}{MenciaSV13}~\cite{MenciaSV13}, \href{../works/MenciaSV12.pdf}{MenciaSV12}~\cite{MenciaSV12}, \href{../works/EdisO11.pdf}{EdisO11}~\cite{EdisO11}, \href{../works/KovacsB11.pdf}{KovacsB11}~\cite{KovacsB11}, \href{../works/QuirogaZH05.pdf}{QuirogaZH05}~\cite{QuirogaZH05}, \href{../works/BeckPS03.pdf}{BeckPS03}~\cite{BeckPS03}, \href{../works/BeckR03.pdf}{BeckR03}~\cite{BeckR03}\\
Concepts & inventory & \href{../works/GuoZ23.pdf}{GuoZ23}~\cite{GuoZ23}, \href{../works/SubulanC22.pdf}{SubulanC22}~\cite{SubulanC22}, \href{../works/Astrand21.pdf}{Astrand21}~\cite{Astrand21}, \href{../works/German18.pdf}{German18}~\cite{German18}, \href{../works/GilesH16.pdf}{GilesH16}~\cite{GilesH16}, \href{../works/GoelSHFS15.pdf}{GoelSHFS15}~\cite{GoelSHFS15}, \href{../works/HarjunkoskiMBC14.pdf}{HarjunkoskiMBC14}~\cite{HarjunkoskiMBC14}, \href{../works/SerraNM12.pdf}{SerraNM12}~\cite{SerraNM12}, \href{../works/TerekhovDOB12.pdf}{TerekhovDOB12}~\cite{TerekhovDOB12}, \href{../works/LopesCSM10.pdf}{LopesCSM10}~\cite{LopesCSM10}, \href{../works/Jans09.pdf}{Jans09}~\cite{Jans09}, \href{../works/RossiTHP07.pdf}{RossiTHP07}~\cite{RossiTHP07}, \href{../works/Timpe02.pdf}{Timpe02}~\cite{Timpe02}, \href{../works/Beck99.pdf}{Beck99}~\cite{Beck99}, \href{../works/BeckDF97.pdf}{BeckDF97}~\cite{BeckDF97} & \href{../works/Adelgren2023.pdf}{Adelgren2023}~\cite{Adelgren2023}, \href{../works/EmdeZD22.pdf}{EmdeZD22}~\cite{EmdeZD22}, \href{../works/ZarandiASC20.pdf}{ZarandiASC20}~\cite{ZarandiASC20}, \href{../works/Novas19.pdf}{Novas19}~\cite{Novas19}, \href{../works/Hooker19.pdf}{Hooker19}~\cite{Hooker19}, \href{../works/Ham18a.pdf}{Ham18a}~\cite{Ham18a}, \href{../works/BajestaniB13.pdf}{BajestaniB13}~\cite{BajestaniB13}, \href{../works/MakMS10.pdf}{MakMS10}~\cite{MakMS10}, \href{../works/LauLN08.pdf}{LauLN08}~\cite{LauLN08}, \href{../works/MouraSCL08a.pdf}{MouraSCL08a}~\cite{MouraSCL08a}, \href{../works/GarganiR07.pdf}{GarganiR07}~\cite{GarganiR07}, \href{../works/DavenportKRSH07.pdf}{DavenportKRSH07}~\cite{DavenportKRSH07}, \href{../works/BeckF00.pdf}{BeckF00}~\cite{BeckF00}, \href{../works/Simonis99.pdf}{Simonis99}~\cite{Simonis99}, \href{../works/BlazewiczDP96.pdf}{BlazewiczDP96}~\cite{BlazewiczDP96}, \href{../works/Simonis95a.pdf}{Simonis95a}~\cite{Simonis95a} & \href{../works/PrataAN23.pdf}{PrataAN23}~\cite{PrataAN23}, \href{../works/PerezGSL23.pdf}{PerezGSL23}~\cite{PerezGSL23}, \href{../works/abs-2312-13682.pdf}{abs-2312-13682}~\cite{abs-2312-13682}, \href{../works/ZhuSZW23.pdf}{ZhuSZW23}~\cite{ZhuSZW23}, \href{../works/GokPTGO23.pdf}{GokPTGO23}~\cite{GokPTGO23}, \href{../works/AlfieriGPS23.pdf}{AlfieriGPS23}~\cite{AlfieriGPS23}, \href{../works/GurPAE23.pdf}{GurPAE23}~\cite{GurPAE23}, \href{../works/PohlAK22.pdf}{PohlAK22}~\cite{PohlAK22}, \href{../works/YunusogluY22.pdf}{YunusogluY22}~\cite{YunusogluY22}, \href{../works/AbreuN22.pdf}{AbreuN22}~\cite{AbreuN22}, \href{../works/Groleaz21.pdf}{Groleaz21}~\cite{Groleaz21}, \href{../works/KovacsTKSG21.pdf}{KovacsTKSG21}~\cite{KovacsTKSG21}, \href{../works/HubnerGSV21.pdf}{HubnerGSV21}~\cite{HubnerGSV21}, \href{../works/HauderBRPA20.pdf}{HauderBRPA20}~\cite{HauderBRPA20}, \href{../works/GroleazNS20a.pdf}{GroleazNS20a}~\cite{GroleazNS20a}, \href{../works/GroleazNS20.pdf}{GroleazNS20}~\cite{GroleazNS20}, \href{../works/YounespourAKE19.pdf}{YounespourAKE19}~\cite{YounespourAKE19}, \href{../works/HoundjiSW19.pdf}{HoundjiSW19}~\cite{HoundjiSW19}, \href{../works/abs-1902-09244.pdf}{abs-1902-09244}~\cite{abs-1902-09244}, \href{../works/WikarekS19.pdf}{WikarekS19}~\cite{WikarekS19}, \href{../works/Ham18.pdf}{Ham18}~\cite{Ham18}, \href{../works/LaborieRSV18.pdf}{LaborieRSV18}~\cite{LaborieRSV18}, \href{../works/ShinBBHO18.pdf}{ShinBBHO18}~\cite{ShinBBHO18}, \href{../works/GomesM17.pdf}{GomesM17}~\cite{GomesM17}, \href{../works/Nattaf16.pdf}{Nattaf16}~\cite{Nattaf16}, \href{../works/SchuttS16.pdf}{SchuttS16}~\cite{SchuttS16}, \href{../works/Froger16.pdf}{Froger16}~\cite{Froger16}, \href{../works/OrnekO16.pdf}{OrnekO16}~\cite{OrnekO16}, \href{../works/OzturkTHO15.pdf}{OzturkTHO15}~\cite{OzturkTHO15}... (Total: 54)\\
Concepts & job & \href{../works/abs-2402-00459.pdf}{abs-2402-00459}~\cite{abs-2402-00459}, \href{../works/PrataAN23.pdf}{PrataAN23}~\cite{PrataAN23}, \href{../works/ForbesHJST24.pdf}{ForbesHJST24}~\cite{ForbesHJST24}, \href{../works/AbreuPNF23.pdf}{AbreuPNF23}~\cite{AbreuPNF23}, \href{../works/JuvinHHL23.pdf}{JuvinHHL23}~\cite{JuvinHHL23}, \href{../works/PenzDN23.pdf}{PenzDN23}~\cite{PenzDN23}, \href{../works/AlfieriGPS23.pdf}{AlfieriGPS23}~\cite{AlfieriGPS23}, \href{../works/YuraszeckMC23.pdf}{YuraszeckMC23}~\cite{YuraszeckMC23}, \href{../works/AfsarVPG23.pdf}{AfsarVPG23}~\cite{AfsarVPG23}, \href{../works/LacknerMMWW23.pdf}{LacknerMMWW23}~\cite{LacknerMMWW23}, \href{../works/Bit-Monnot23.pdf}{Bit-Monnot23}~\cite{Bit-Monnot23}, \href{../works/ZhuSZW23.pdf}{ZhuSZW23}~\cite{ZhuSZW23}, \href{../works/Fatemi-AnarakiTFV23.pdf}{Fatemi-AnarakiTFV23}~\cite{Fatemi-AnarakiTFV23}, \href{../works/Mehdizadeh-Somarin23.pdf}{Mehdizadeh-Somarin23}~\cite{Mehdizadeh-Somarin23}, \href{../works/KimCMLLP23.pdf}{KimCMLLP23}~\cite{KimCMLLP23}, \href{../works/AbreuNP23.pdf}{AbreuNP23}~\cite{AbreuNP23}, \href{../works/IsikYA23.pdf}{IsikYA23}~\cite{IsikYA23}, \href{../works/WangB23.pdf}{WangB23}~\cite{WangB23}, \href{../works/CzerniachowskaWZ23.pdf}{CzerniachowskaWZ23}~\cite{CzerniachowskaWZ23}, \href{../works/abs-2306-05747.pdf}{abs-2306-05747}~\cite{abs-2306-05747}, \href{../works/NaderiRR23.pdf}{NaderiRR23}~\cite{NaderiRR23}, \href{../works/JuvinHL23.pdf}{JuvinHL23}~\cite{JuvinHL23}, \href{../works/TasselGS23.pdf}{TasselGS23}~\cite{TasselGS23}, \href{../works/JuvinHL23a.pdf}{JuvinHL23a}~\cite{JuvinHL23a}, \href{../works/YuraszeckMCCR23.pdf}{YuraszeckMCCR23}~\cite{YuraszeckMCCR23}, \href{../works/EtminaniesfahaniGNMS22.pdf}{EtminaniesfahaniGNMS22}~\cite{EtminaniesfahaniGNMS22}, \href{../works/TouatBT22.pdf}{TouatBT22}~\cite{TouatBT22}, \href{../works/MullerMKP22.pdf}{MullerMKP22}~\cite{MullerMKP22}, \href{../works/ArmstrongGOS22.pdf}{ArmstrongGOS22}~\cite{ArmstrongGOS22}... (Total: 270) & \href{../works/BonninMNE24.pdf}{BonninMNE24}~\cite{BonninMNE24}, \href{../works/ShaikhK23.pdf}{ShaikhK23}~\cite{ShaikhK23}, \href{../works/abs-2305-19888.pdf}{abs-2305-19888}~\cite{abs-2305-19888}, \href{../works/EfthymiouY23.pdf}{EfthymiouY23}~\cite{EfthymiouY23}, \href{../works/Adelgren2023.pdf}{Adelgren2023}~\cite{Adelgren2023}, \href{../works/MarliereSPR23.pdf}{MarliereSPR23}~\cite{MarliereSPR23}, \href{../works/LuoB22.pdf}{LuoB22}~\cite{LuoB22}, \href{../works/HeinzNVH22.pdf}{HeinzNVH22}~\cite{HeinzNVH22}, \href{../works/BourreauGGLT22.pdf}{BourreauGGLT22}~\cite{BourreauGGLT22}, \href{../works/HanenKP21.pdf}{HanenKP21}~\cite{HanenKP21}, \href{../works/Lemos21.pdf}{Lemos21}~\cite{Lemos21}, \href{../works/Mercier-AubinGQ20.pdf}{Mercier-AubinGQ20}~\cite{Mercier-AubinGQ20}, \href{../works/GokGSTO20.pdf}{GokGSTO20}~\cite{GokGSTO20}, \href{../works/MokhtarzadehTNF20.pdf}{MokhtarzadehTNF20}~\cite{MokhtarzadehTNF20}, \href{../works/RoshanaeiBAUB20.pdf}{RoshanaeiBAUB20}~\cite{RoshanaeiBAUB20}, \href{../works/ArkhipovBL19.pdf}{ArkhipovBL19}~\cite{ArkhipovBL19}, \href{../works/EscobetPQPRA19.pdf}{EscobetPQPRA19}~\cite{EscobetPQPRA19}, \href{../works/Tom19.pdf}{Tom19}~\cite{Tom19}, \href{../works/GurEA19.pdf}{GurEA19}~\cite{GurEA19}, \href{../works/German18.pdf}{German18}~\cite{German18}, \href{../works/PourDERB18.pdf}{PourDERB18}~\cite{PourDERB18}, \href{../works/NattafAL17.pdf}{NattafAL17}~\cite{NattafAL17}, \href{../works/CappartS17.pdf}{CappartS17}~\cite{CappartS17}, \href{../works/RoshanaeiLAU17.pdf}{RoshanaeiLAU17}~\cite{RoshanaeiLAU17}, \href{../works/ZarandiKS16.pdf}{ZarandiKS16}~\cite{ZarandiKS16}, \href{../works/TranWDRFOVB16.pdf}{TranWDRFOVB16}~\cite{TranWDRFOVB16}, \href{../works/Madi-WambaB16.pdf}{Madi-WambaB16}~\cite{Madi-WambaB16}, \href{../works/CatusseCBL16.pdf}{CatusseCBL16}~\cite{CatusseCBL16}, \href{../works/LetortCB15.pdf}{LetortCB15}~\cite{LetortCB15}... (Total: 61) & \href{../works/PovedaAA23.pdf}{PovedaAA23}~\cite{PovedaAA23}, \href{../works/GuoZ23.pdf}{GuoZ23}~\cite{GuoZ23}, \href{../works/GokPTGO23.pdf}{GokPTGO23}~\cite{GokPTGO23}, \href{../works/PohlAK22.pdf}{PohlAK22}~\cite{PohlAK22}, \href{../works/CampeauG22.pdf}{CampeauG22}~\cite{CampeauG22}, \href{../works/KlankeBYE21.pdf}{KlankeBYE21}~\cite{KlankeBYE21}, \href{../works/HubnerGSV21.pdf}{HubnerGSV21}~\cite{HubnerGSV21}, \href{../works/AntuoriHHEN21.pdf}{AntuoriHHEN21}~\cite{AntuoriHHEN21}, \href{../works/BenderWS21.pdf}{BenderWS21}~\cite{BenderWS21}, \href{../works/QinDCS20.pdf}{QinDCS20}~\cite{QinDCS20}, \href{../works/Polo-MejiaALB20.pdf}{Polo-MejiaALB20}~\cite{Polo-MejiaALB20}, \href{../works/WessenCS20.pdf}{WessenCS20}~\cite{WessenCS20}, \href{../works/AntuoriHHEN20.pdf}{AntuoriHHEN20}~\cite{AntuoriHHEN20}, \href{../works/FrimodigS19.pdf}{FrimodigS19}~\cite{FrimodigS19}, \href{../works/HoYCLLCLC18.pdf}{HoYCLLCLC18}~\cite{HoYCLLCLC18}, \href{../works/ShinBBHO18.pdf}{ShinBBHO18}~\cite{ShinBBHO18}, \href{../works/CauwelaertLS18.pdf}{CauwelaertLS18}~\cite{CauwelaertLS18}, \href{../works/TangLWSK18.pdf}{TangLWSK18}~\cite{TangLWSK18}, \href{../works/BaptisteB18.pdf}{BaptisteB18}~\cite{BaptisteB18}, \href{../works/TranVNB17.pdf}{TranVNB17}~\cite{TranVNB17}, \href{../works/NovaraNH16.pdf}{NovaraNH16}~\cite{NovaraNH16}, \href{../works/HechingH16.pdf}{HechingH16}~\cite{HechingH16}, \href{../works/WangMD15.pdf}{WangMD15}~\cite{WangMD15}, \href{../works/BurtLPS15.pdf}{BurtLPS15}~\cite{BurtLPS15}, \href{../works/BartakV15.pdf}{BartakV15}~\cite{BartakV15}, \href{../works/LimBTBB15.pdf}{LimBTBB15}~\cite{LimBTBB15}, \href{../works/LombardiBM15.pdf}{LombardiBM15}~\cite{LombardiBM15}, \href{../works/MelgarejoLS15.pdf}{MelgarejoLS15}~\cite{MelgarejoLS15}, \href{../works/DerrienPZ14.pdf}{DerrienPZ14}~\cite{DerrienPZ14}... (Total: 82)\\
Concepts & job-shop & \href{../works/abs-2402-00459.pdf}{abs-2402-00459}~\cite{abs-2402-00459}, \href{../works/PrataAN23.pdf}{PrataAN23}~\cite{PrataAN23}, \href{../works/YuraszeckMCCR23.pdf}{YuraszeckMCCR23}~\cite{YuraszeckMCCR23}, \href{../works/abs-2306-05747.pdf}{abs-2306-05747}~\cite{abs-2306-05747}, \href{../works/JuvinHL23a.pdf}{JuvinHL23a}~\cite{JuvinHL23a}, \href{../works/JuvinHHL23.pdf}{JuvinHHL23}~\cite{JuvinHHL23}, \href{../works/AfsarVPG23.pdf}{AfsarVPG23}~\cite{AfsarVPG23}, \href{../works/AbreuNP23.pdf}{AbreuNP23}~\cite{AbreuNP23}, \href{../works/Mehdizadeh-Somarin23.pdf}{Mehdizadeh-Somarin23}~\cite{Mehdizadeh-Somarin23}, \href{../works/Fatemi-AnarakiTFV23.pdf}{Fatemi-AnarakiTFV23}~\cite{Fatemi-AnarakiTFV23}, \href{../works/ZhuSZW23.pdf}{ZhuSZW23}~\cite{ZhuSZW23}, \href{../works/KimCMLLP23.pdf}{KimCMLLP23}~\cite{KimCMLLP23}, \href{../works/CzerniachowskaWZ23.pdf}{CzerniachowskaWZ23}~\cite{CzerniachowskaWZ23}, \href{../works/Bit-Monnot23.pdf}{Bit-Monnot23}~\cite{Bit-Monnot23}, \href{../works/NaderiRR23.pdf}{NaderiRR23}~\cite{NaderiRR23}, \href{../works/TasselGS23.pdf}{TasselGS23}~\cite{TasselGS23}, \href{../works/Teppan22.pdf}{Teppan22}~\cite{Teppan22}, \href{../works/NaderiBZ22a.pdf}{NaderiBZ22a}~\cite{NaderiBZ22a}, \href{../works/OujanaAYB22.pdf}{OujanaAYB22}~\cite{OujanaAYB22}, \href{../works/LiFJZLL22.pdf}{LiFJZLL22}~\cite{LiFJZLL22}, \href{../works/ColT22.pdf}{ColT22}~\cite{ColT22}, \href{../works/MullerMKP22.pdf}{MullerMKP22}~\cite{MullerMKP22}, \href{../works/ZhangBB22.pdf}{ZhangBB22}~\cite{ZhangBB22}, \href{../works/abs-2211-14492.pdf}{abs-2211-14492}~\cite{abs-2211-14492}, \href{../works/YuraszeckMPV22.pdf}{YuraszeckMPV22}~\cite{YuraszeckMPV22}, \href{../works/GeitzGSSW22.pdf}{GeitzGSSW22}~\cite{GeitzGSSW22}, \href{../works/JuvinHL22.pdf}{JuvinHL22}~\cite{JuvinHL22}, \href{../works/Astrand21.pdf}{Astrand21}~\cite{Astrand21}, \href{../works/KovacsTKSG21.pdf}{KovacsTKSG21}~\cite{KovacsTKSG21}... (Total: 133) & \href{../works/AbreuPNF23.pdf}{AbreuPNF23}~\cite{AbreuPNF23}, \href{../works/PenzDN23.pdf}{PenzDN23}~\cite{PenzDN23}, \href{../works/EfthymiouY23.pdf}{EfthymiouY23}~\cite{EfthymiouY23}, \href{../works/IsikYA23.pdf}{IsikYA23}~\cite{IsikYA23}, \href{../works/AlfieriGPS23.pdf}{AlfieriGPS23}~\cite{AlfieriGPS23}, \href{../works/NaderiBZ22.pdf}{NaderiBZ22}~\cite{NaderiBZ22}, \href{../works/EtminaniesfahaniGNMS22.pdf}{EtminaniesfahaniGNMS22}~\cite{EtminaniesfahaniGNMS22}, \href{../works/TouatBT22.pdf}{TouatBT22}~\cite{TouatBT22}, \href{../works/YunusogluY22.pdf}{YunusogluY22}~\cite{YunusogluY22}, \href{../works/AbreuN22.pdf}{AbreuN22}~\cite{AbreuN22}, \href{../works/LuoB22.pdf}{LuoB22}~\cite{LuoB22}, \href{../works/QinWSLS21.pdf}{QinWSLS21}~\cite{QinWSLS21}, \href{../works/ArmstrongGOS21.pdf}{ArmstrongGOS21}~\cite{ArmstrongGOS21}, \href{../works/KoehlerBFFHPSSS21.pdf}{KoehlerBFFHPSSS21}~\cite{KoehlerBFFHPSSS21}, \href{../works/Godet21a.pdf}{Godet21a}~\cite{Godet21a}, \href{../works/Astrand0F21.pdf}{Astrand0F21}~\cite{Astrand0F21}, \href{../works/MejiaY20.pdf}{MejiaY20}~\cite{MejiaY20}, \href{../works/GroleazNS20.pdf}{GroleazNS20}~\cite{GroleazNS20}, \href{../works/SacramentoSP20.pdf}{SacramentoSP20}~\cite{SacramentoSP20}, \href{../works/ArkhipovBL19.pdf}{ArkhipovBL19}~\cite{ArkhipovBL19}, \href{../works/WikarekS19.pdf}{WikarekS19}~\cite{WikarekS19}, \href{../works/EscobetPQPRA19.pdf}{EscobetPQPRA19}~\cite{EscobetPQPRA19}, \href{../works/GokgurHO18.pdf}{GokgurHO18}~\cite{GokgurHO18}, \href{../works/German18.pdf}{German18}~\cite{German18}, \href{../works/MossigeGSMC17.pdf}{MossigeGSMC17}~\cite{MossigeGSMC17}, \href{../works/CappartS17.pdf}{CappartS17}~\cite{CappartS17}, \href{../works/Derrien15.pdf}{Derrien15}~\cite{Derrien15}, \href{../works/Kameugne14.pdf}{Kameugne14}~\cite{Kameugne14}, \href{../works/BonfiettiLM14.pdf}{BonfiettiLM14}~\cite{BonfiettiLM14}... (Total: 55) & \href{../works/ForbesHJST24.pdf}{ForbesHJST24}~\cite{ForbesHJST24}, \href{../works/BonninMNE24.pdf}{BonninMNE24}~\cite{BonninMNE24}, \href{../works/Adelgren2023.pdf}{Adelgren2023}~\cite{Adelgren2023}, \href{../works/ShaikhK23.pdf}{ShaikhK23}~\cite{ShaikhK23}, \href{../works/PovedaAA23.pdf}{PovedaAA23}~\cite{PovedaAA23}, \href{../works/MarliereSPR23.pdf}{MarliereSPR23}~\cite{MarliereSPR23}, \href{../works/GokPTGO23.pdf}{GokPTGO23}~\cite{GokPTGO23}, \href{../works/YuraszeckMC23.pdf}{YuraszeckMC23}~\cite{YuraszeckMC23}, \href{../works/GuoZ23.pdf}{GuoZ23}~\cite{GuoZ23}, \href{../works/LacknerMMWW23.pdf}{LacknerMMWW23}~\cite{LacknerMMWW23}, \href{../works/JuvinHL23.pdf}{JuvinHL23}~\cite{JuvinHL23}, \href{../works/EmdeZD22.pdf}{EmdeZD22}~\cite{EmdeZD22}, \href{../works/HanenKP21.pdf}{HanenKP21}~\cite{HanenKP21}, \href{../works/Lemos21.pdf}{Lemos21}~\cite{Lemos21}, \href{../works/KlankeBYE21.pdf}{KlankeBYE21}~\cite{KlankeBYE21}, \href{../works/AntuoriHHEN21.pdf}{AntuoriHHEN21}~\cite{AntuoriHHEN21}, \href{../works/Zahout21.pdf}{Zahout21}~\cite{Zahout21}, \href{../works/GokGSTO20.pdf}{GokGSTO20}~\cite{GokGSTO20}, \href{../works/HauderBRPA20.pdf}{HauderBRPA20}~\cite{HauderBRPA20}, \href{../works/AntuoriHHEN20.pdf}{AntuoriHHEN20}~\cite{AntuoriHHEN20}, \href{../works/RoshanaeiBAUB20.pdf}{RoshanaeiBAUB20}~\cite{RoshanaeiBAUB20}, \href{../works/BenediktMH20.pdf}{BenediktMH20}~\cite{BenediktMH20}, \href{../works/WessenCS20.pdf}{WessenCS20}~\cite{WessenCS20}, \href{../works/Mercier-AubinGQ20.pdf}{Mercier-AubinGQ20}~\cite{Mercier-AubinGQ20}, \href{../works/WallaceY20.pdf}{WallaceY20}~\cite{WallaceY20}, \href{../works/NattafDYW19.pdf}{NattafDYW19}~\cite{NattafDYW19}, \href{../works/BogaerdtW19.pdf}{BogaerdtW19}~\cite{BogaerdtW19}, \href{../works/abs-1902-09244.pdf}{abs-1902-09244}~\cite{abs-1902-09244}, \href{../works/Tom19.pdf}{Tom19}~\cite{Tom19}... (Total: 107)\\
Concepts & lateness & \href{../works/Groleaz21.pdf}{Groleaz21}~\cite{Groleaz21}, \href{../works/FahimiOQ18.pdf}{FahimiOQ18}~\cite{FahimiOQ18}, \href{../works/Fahimi16.pdf}{Fahimi16}~\cite{Fahimi16}, \href{../works/Dejemeppe16.pdf}{Dejemeppe16}~\cite{Dejemeppe16}, \href{../works/KoschB14.pdf}{KoschB14}~\cite{KoschB14}, \href{../works/Malapert11.pdf}{Malapert11}~\cite{Malapert11}, \href{../works/BartakSR10.pdf}{BartakSR10}~\cite{BartakSR10}, \href{../works/Geske05.pdf}{Geske05}~\cite{Geske05}, \href{../works/Baptiste02.pdf}{Baptiste02}~\cite{Baptiste02}, \href{../works/ArtiguesR00.pdf}{ArtiguesR00}~\cite{ArtiguesR00}, \href{../works/BlazewiczDP96.pdf}{BlazewiczDP96}~\cite{BlazewiczDP96} & \href{../works/PrataAN23.pdf}{PrataAN23}~\cite{PrataAN23}, \href{../works/PohlAK22.pdf}{PohlAK22}~\cite{PohlAK22}, \href{../works/ZarandiASC20.pdf}{ZarandiASC20}~\cite{ZarandiASC20}, \href{../works/AntunesABD20.pdf}{AntunesABD20}~\cite{AntunesABD20}, \href{../works/ZhangW18.pdf}{ZhangW18}~\cite{ZhangW18}, \href{../works/HarjunkoskiMBC14.pdf}{HarjunkoskiMBC14}~\cite{HarjunkoskiMBC14}, \href{../works/MilanoW09.pdf}{MilanoW09}~\cite{MilanoW09}, \href{../works/AkkerDH07.pdf}{AkkerDH07}~\cite{AkkerDH07}, \href{../works/MilanoW06.pdf}{MilanoW06}~\cite{MilanoW06}, \href{../works/Sadykov04.pdf}{Sadykov04}~\cite{Sadykov04} & \href{../works/LacknerMMWW23.pdf}{LacknerMMWW23}~\cite{LacknerMMWW23}, \href{../works/YunusogluY22.pdf}{YunusogluY22}~\cite{YunusogluY22}, \href{../works/NaderiBZ22.pdf}{NaderiBZ22}~\cite{NaderiBZ22}, \href{../works/GeitzGSSW22.pdf}{GeitzGSSW22}~\cite{GeitzGSSW22}, \href{../works/ColT22.pdf}{ColT22}~\cite{ColT22}, \href{../works/ZhangBB22.pdf}{ZhangBB22}~\cite{ZhangBB22}, \href{../works/LacknerMMWW21.pdf}{LacknerMMWW21}~\cite{LacknerMMWW21}, \href{../works/Godet21a.pdf}{Godet21a}~\cite{Godet21a}, \href{../works/KoehlerBFFHPSSS21.pdf}{KoehlerBFFHPSSS21}~\cite{KoehlerBFFHPSSS21}, \href{../works/HanenKP21.pdf}{HanenKP21}~\cite{HanenKP21}, \href{../works/QinWSLS21.pdf}{QinWSLS21}~\cite{QinWSLS21}, \href{../works/Lunardi20.pdf}{Lunardi20}~\cite{Lunardi20}, \href{../works/Novas19.pdf}{Novas19}~\cite{Novas19}, \href{../works/ArkhipovBL19.pdf}{ArkhipovBL19}~\cite{ArkhipovBL19}, \href{../works/ParkUJR19.pdf}{ParkUJR19}~\cite{ParkUJR19}, \href{../works/AntunesABD18.pdf}{AntunesABD18}~\cite{AntunesABD18}, \href{../works/Tesch18.pdf}{Tesch18}~\cite{Tesch18}, \href{../works/GrimesH15.pdf}{GrimesH15}~\cite{GrimesH15}, \href{../works/BartakV15.pdf}{BartakV15}~\cite{BartakV15}, \href{../works/MenciaSV13.pdf}{MenciaSV13}~\cite{MenciaSV13}, \href{../works/MenciaSV12.pdf}{MenciaSV12}~\cite{MenciaSV12}, \href{../works/TerekhovDOB12.pdf}{TerekhovDOB12}~\cite{TerekhovDOB12}, \href{../works/EdisO11.pdf}{EdisO11}~\cite{EdisO11}, \href{../works/ChenGPSH10.pdf}{ChenGPSH10}~\cite{ChenGPSH10}, \href{../works/NovasH10.pdf}{NovasH10}~\cite{NovasH10}, \href{../works/WuBB09.pdf}{WuBB09}~\cite{WuBB09}, \href{../works/SadykovW06.pdf}{SadykovW06}~\cite{SadykovW06}, \href{../works/Bartak02.pdf}{Bartak02}~\cite{Bartak02}, \href{../works/JainM99.pdf}{JainM99}~\cite{JainM99}\\
Concepts & machine & \href{../works/abs-2402-00459.pdf}{abs-2402-00459}~\cite{abs-2402-00459}, \href{../works/BonninMNE24.pdf}{BonninMNE24}~\cite{BonninMNE24}, \href{../works/PrataAN23.pdf}{PrataAN23}~\cite{PrataAN23}, \href{../works/Fatemi-AnarakiTFV23.pdf}{Fatemi-AnarakiTFV23}~\cite{Fatemi-AnarakiTFV23}, \href{../works/PenzDN23.pdf}{PenzDN23}~\cite{PenzDN23}, \href{../works/YuraszeckMCCR23.pdf}{YuraszeckMCCR23}~\cite{YuraszeckMCCR23}, \href{../works/JuvinHL23a.pdf}{JuvinHL23a}~\cite{JuvinHL23a}, \href{../works/ZhuSZW23.pdf}{ZhuSZW23}~\cite{ZhuSZW23}, \href{../works/AalianPG23.pdf}{AalianPG23}~\cite{AalianPG23}, \href{../works/AbreuPNF23.pdf}{AbreuPNF23}~\cite{AbreuPNF23}, \href{../works/JuvinHHL23.pdf}{JuvinHHL23}~\cite{JuvinHHL23}, \href{../works/abs-2312-13682.pdf}{abs-2312-13682}~\cite{abs-2312-13682}, \href{../works/LacknerMMWW23.pdf}{LacknerMMWW23}~\cite{LacknerMMWW23}, \href{../works/AlfieriGPS23.pdf}{AlfieriGPS23}~\cite{AlfieriGPS23}, \href{../works/AfsarVPG23.pdf}{AfsarVPG23}~\cite{AfsarVPG23}, \href{../works/KimCMLLP23.pdf}{KimCMLLP23}~\cite{KimCMLLP23}, \href{../works/IsikYA23.pdf}{IsikYA23}~\cite{IsikYA23}, \href{../works/CzerniachowskaWZ23.pdf}{CzerniachowskaWZ23}~\cite{CzerniachowskaWZ23}, \href{../works/AbreuNP23.pdf}{AbreuNP23}~\cite{AbreuNP23}, \href{../works/Adelgren2023.pdf}{Adelgren2023}~\cite{Adelgren2023}, \href{../works/NaderiRR23.pdf}{NaderiRR23}~\cite{NaderiRR23}, \href{../works/TasselGS23.pdf}{TasselGS23}~\cite{TasselGS23}, \href{../works/Mehdizadeh-Somarin23.pdf}{Mehdizadeh-Somarin23}~\cite{Mehdizadeh-Somarin23}, \href{../works/JuvinHL23.pdf}{JuvinHL23}~\cite{JuvinHL23}, \href{../works/GuoZ23.pdf}{GuoZ23}~\cite{GuoZ23}, \href{../works/PerezGSL23.pdf}{PerezGSL23}~\cite{PerezGSL23}, \href{../works/EfthymiouY23.pdf}{EfthymiouY23}~\cite{EfthymiouY23}, \href{../works/abs-2306-05747.pdf}{abs-2306-05747}~\cite{abs-2306-05747}, \href{../works/YuraszeckMC23.pdf}{YuraszeckMC23}~\cite{YuraszeckMC23}... (Total: 262) & \href{../works/ForbesHJST24.pdf}{ForbesHJST24}~\cite{ForbesHJST24}, \href{../works/AkramNHRSA23.pdf}{AkramNHRSA23}~\cite{AkramNHRSA23}, \href{../works/GurPAE23.pdf}{GurPAE23}~\cite{GurPAE23}, \href{../works/Bit-Monnot23.pdf}{Bit-Monnot23}~\cite{Bit-Monnot23}, \href{../works/GokPTGO23.pdf}{GokPTGO23}~\cite{GokPTGO23}, \href{../works/OrnekOS20.pdf}{OrnekOS20}~\cite{OrnekOS20}, \href{../works/EtminaniesfahaniGNMS22.pdf}{EtminaniesfahaniGNMS22}~\cite{EtminaniesfahaniGNMS22}, \href{../works/LuoB22.pdf}{LuoB22}~\cite{LuoB22}, \href{../works/ElciOH22.pdf}{ElciOH22}~\cite{ElciOH22}, \href{../works/HillTV21.pdf}{HillTV21}~\cite{HillTV21}, \href{../works/KlankeBYE21.pdf}{KlankeBYE21}~\cite{KlankeBYE21}, \href{../works/Lemos21.pdf}{Lemos21}~\cite{Lemos21}, \href{../works/AbohashimaEG21.pdf}{AbohashimaEG21}~\cite{AbohashimaEG21}, \href{../works/Polo-MejiaALB20.pdf}{Polo-MejiaALB20}~\cite{Polo-MejiaALB20}, \href{../works/RoshanaeiBAUB20.pdf}{RoshanaeiBAUB20}~\cite{RoshanaeiBAUB20}, \href{../works/AntuoriHHEN20.pdf}{AntuoriHHEN20}~\cite{AntuoriHHEN20}, \href{../works/BehrensLM19.pdf}{BehrensLM19}~\cite{BehrensLM19}, \href{../works/GoldwaserS18.pdf}{GoldwaserS18}~\cite{GoldwaserS18}, \href{../works/BaptisteB18.pdf}{BaptisteB18}~\cite{BaptisteB18}, \href{../works/He0GLW18.pdf}{He0GLW18}~\cite{He0GLW18}, \href{../works/Ham18.pdf}{Ham18}~\cite{Ham18}, \href{../works/ShinBBHO18.pdf}{ShinBBHO18}~\cite{ShinBBHO18}, \href{../works/MusliuSS18.pdf}{MusliuSS18}~\cite{MusliuSS18}, \href{../works/FahimiOQ18.pdf}{FahimiOQ18}~\cite{FahimiOQ18}, \href{../works/GoldwaserS17.pdf}{GoldwaserS17}~\cite{GoldwaserS17}, \href{../works/CohenHB17.pdf}{CohenHB17}~\cite{CohenHB17}, \href{../works/KreterSS17.pdf}{KreterSS17}~\cite{KreterSS17}, \href{../works/Pralet17.pdf}{Pralet17}~\cite{Pralet17}, \href{../works/SchuttS16.pdf}{SchuttS16}~\cite{SchuttS16}... (Total: 71) & \href{../works/MarliereSPR23.pdf}{MarliereSPR23}~\cite{MarliereSPR23}, \href{../works/ShaikhK23.pdf}{ShaikhK23}~\cite{ShaikhK23}, \href{../works/KameugneFND23.pdf}{KameugneFND23}~\cite{KameugneFND23}, \href{../works/MontemanniD23.pdf}{MontemanniD23}~\cite{MontemanniD23}, \href{../works/BoudreaultSLQ22.pdf}{BoudreaultSLQ22}~\cite{BoudreaultSLQ22}, \href{../works/PopovicCGNC22.pdf}{PopovicCGNC22}~\cite{PopovicCGNC22}, \href{../works/SubulanC22.pdf}{SubulanC22}~\cite{SubulanC22}, \href{../works/PohlAK22.pdf}{PohlAK22}~\cite{PohlAK22}, \href{../works/GeibingerMM21.pdf}{GeibingerMM21}~\cite{GeibingerMM21}, \href{../works/ArtiguesHQT21.pdf}{ArtiguesHQT21}~\cite{ArtiguesHQT21}, \href{../works/WallaceY20.pdf}{WallaceY20}~\cite{WallaceY20}, \href{../works/BarzegaranZP20.pdf}{BarzegaranZP20}~\cite{BarzegaranZP20}, \href{../works/Mercier-AubinGQ20.pdf}{Mercier-AubinGQ20}~\cite{Mercier-AubinGQ20}, \href{../works/WangB20.pdf}{WangB20}~\cite{WangB20}, \href{../works/ArkhipovBL19.pdf}{ArkhipovBL19}~\cite{ArkhipovBL19}, \href{../works/YounespourAKE19.pdf}{YounespourAKE19}~\cite{YounespourAKE19}, \href{../works/YangSS19.pdf}{YangSS19}~\cite{YangSS19}, \href{../works/NattafHKAL19.pdf}{NattafHKAL19}~\cite{NattafHKAL19}, \href{../works/BadicaBIL19.pdf}{BadicaBIL19}~\cite{BadicaBIL19}, \href{../works/NishikawaSTT19.pdf}{NishikawaSTT19}~\cite{NishikawaSTT19}, \href{../works/Tom19.pdf}{Tom19}~\cite{Tom19}, \href{../works/AntunesABD18.pdf}{AntunesABD18}~\cite{AntunesABD18}, \href{../works/KreterSSZ18.pdf}{KreterSSZ18}~\cite{KreterSSZ18}, \href{../works/HoYCLLCLC18.pdf}{HoYCLLCLC18}~\cite{HoYCLLCLC18}, \href{../works/PourDERB18.pdf}{PourDERB18}~\cite{PourDERB18}, \href{../works/Laborie18a.pdf}{Laborie18a}~\cite{Laborie18a}, \href{../works/CauwelaertLS18.pdf}{CauwelaertLS18}~\cite{CauwelaertLS18}, \href{../works/TranVNB17a.pdf}{TranVNB17a}~\cite{TranVNB17a}, \href{../works/KletzanderM17.pdf}{KletzanderM17}~\cite{KletzanderM17}... (Total: 123)\\
Concepts & make to order &  &  & \href{../works/OujanaAYB22.pdf}{OujanaAYB22}~\cite{OujanaAYB22}, \href{../works/DavenportKRSH07.pdf}{DavenportKRSH07}~\cite{DavenportKRSH07}, \href{../works/Simonis07.pdf}{Simonis07}~\cite{Simonis07}\\
Concepts & make to stock &  &  & \href{../works/HarjunkoskiMBC14.pdf}{HarjunkoskiMBC14}~\cite{HarjunkoskiMBC14}\\
Concepts & make-span & \href{../works/PrataAN23.pdf}{PrataAN23}~\cite{PrataAN23}, \href{../works/Mehdizadeh-Somarin23.pdf}{Mehdizadeh-Somarin23}~\cite{Mehdizadeh-Somarin23}, \href{../works/AbreuNP23.pdf}{AbreuNP23}~\cite{AbreuNP23}, \href{../works/EfthymiouY23.pdf}{EfthymiouY23}~\cite{EfthymiouY23}, \href{../works/PovedaAA23.pdf}{PovedaAA23}~\cite{PovedaAA23}, \href{../works/AfsarVPG23.pdf}{AfsarVPG23}~\cite{AfsarVPG23}, \href{../works/JuvinHL23a.pdf}{JuvinHL23a}~\cite{JuvinHL23a}, \href{../works/abs-2306-05747.pdf}{abs-2306-05747}~\cite{abs-2306-05747}, \href{../works/AalianPG23.pdf}{AalianPG23}~\cite{AalianPG23}, \href{../works/CzerniachowskaWZ23.pdf}{CzerniachowskaWZ23}~\cite{CzerniachowskaWZ23}, \href{../works/AbreuPNF23.pdf}{AbreuPNF23}~\cite{AbreuPNF23}, \href{../works/JuvinHHL23.pdf}{JuvinHHL23}~\cite{JuvinHHL23}, \href{../works/YuraszeckMC23.pdf}{YuraszeckMC23}~\cite{YuraszeckMC23}, \href{../works/ZhuSZW23.pdf}{ZhuSZW23}~\cite{ZhuSZW23}, \href{../works/IsikYA23.pdf}{IsikYA23}~\cite{IsikYA23}, \href{../works/JuvinHL23.pdf}{JuvinHL23}~\cite{JuvinHL23}, \href{../works/AlfieriGPS23.pdf}{AlfieriGPS23}~\cite{AlfieriGPS23}, \href{../works/abs-2305-19888.pdf}{abs-2305-19888}~\cite{abs-2305-19888}, \href{../works/NaderiRR23.pdf}{NaderiRR23}~\cite{NaderiRR23}, \href{../works/TasselGS23.pdf}{TasselGS23}~\cite{TasselGS23}, \href{../works/Bit-Monnot23.pdf}{Bit-Monnot23}~\cite{Bit-Monnot23}, \href{../works/LacknerMMWW23.pdf}{LacknerMMWW23}~\cite{LacknerMMWW23}, \href{../works/AbreuN22.pdf}{AbreuN22}~\cite{AbreuN22}, \href{../works/YunusogluY22.pdf}{YunusogluY22}~\cite{YunusogluY22}, \href{../works/ZhangBB22.pdf}{ZhangBB22}~\cite{ZhangBB22}, \href{../works/HeinzNVH22.pdf}{HeinzNVH22}~\cite{HeinzNVH22}, \href{../works/JuvinHL22.pdf}{JuvinHL22}~\cite{JuvinHL22}, \href{../works/GeitzGSSW22.pdf}{GeitzGSSW22}~\cite{GeitzGSSW22}, \href{../works/BoudreaultSLQ22.pdf}{BoudreaultSLQ22}~\cite{BoudreaultSLQ22}... (Total: 197) & \href{../works/BonninMNE24.pdf}{BonninMNE24}~\cite{BonninMNE24}, \href{../works/KameugneFND23.pdf}{KameugneFND23}~\cite{KameugneFND23}, \href{../works/YuraszeckMCCR23.pdf}{YuraszeckMCCR23}~\cite{YuraszeckMCCR23}, \href{../works/abs-2312-13682.pdf}{abs-2312-13682}~\cite{abs-2312-13682}, \href{../works/Adelgren2023.pdf}{Adelgren2023}~\cite{Adelgren2023}, \href{../works/PerezGSL23.pdf}{PerezGSL23}~\cite{PerezGSL23}, \href{../works/PenzDN23.pdf}{PenzDN23}~\cite{PenzDN23}, \href{../works/MullerMKP22.pdf}{MullerMKP22}~\cite{MullerMKP22}, \href{../works/SvancaraB22.pdf}{SvancaraB22}~\cite{SvancaraB22}, \href{../works/ZhangJZL22.pdf}{ZhangJZL22}~\cite{ZhangJZL22}, \href{../works/abs-2211-14492.pdf}{abs-2211-14492}~\cite{abs-2211-14492}, \href{../works/YuraszeckMPV22.pdf}{YuraszeckMPV22}~\cite{YuraszeckMPV22}, \href{../works/OujanaAYB22.pdf}{OujanaAYB22}~\cite{OujanaAYB22}, \href{../works/LiFJZLL22.pdf}{LiFJZLL22}~\cite{LiFJZLL22}, \href{../works/PandeyS21a.pdf}{PandeyS21a}~\cite{PandeyS21a}, \href{../works/FanXG21.pdf}{FanXG21}~\cite{FanXG21}, \href{../works/QinDCS20.pdf}{QinDCS20}~\cite{QinDCS20}, \href{../works/NattafDYW19.pdf}{NattafDYW19}~\cite{NattafDYW19}, \href{../works/AstrandJZ18.pdf}{AstrandJZ18}~\cite{AstrandJZ18}, \href{../works/Ham18a.pdf}{Ham18a}~\cite{Ham18a}, \href{../works/YoungFS17.pdf}{YoungFS17}~\cite{YoungFS17}, \href{../works/RoshanaeiLAU17.pdf}{RoshanaeiLAU17}~\cite{RoshanaeiLAU17}, \href{../works/KreterSS17.pdf}{KreterSS17}~\cite{KreterSS17}, \href{../works/GingrasQ16.pdf}{GingrasQ16}~\cite{GingrasQ16}, \href{../works/BonfiettiZLM16.pdf}{BonfiettiZLM16}~\cite{BonfiettiZLM16}, \href{../works/HamC16.pdf}{HamC16}~\cite{HamC16}, \href{../works/KuB16.pdf}{KuB16}~\cite{KuB16}, \href{../works/SialaAH15.pdf}{SialaAH15}~\cite{SialaAH15}, \href{../works/DejemeppeCS15.pdf}{DejemeppeCS15}~\cite{DejemeppeCS15}... (Total: 59) & \href{../works/ForbesHJST24.pdf}{ForbesHJST24}~\cite{ForbesHJST24}, \href{../works/GokPTGO23.pdf}{GokPTGO23}~\cite{GokPTGO23}, \href{../works/GuoZ23.pdf}{GuoZ23}~\cite{GuoZ23}, \href{../works/KimCMLLP23.pdf}{KimCMLLP23}~\cite{KimCMLLP23}, \href{../works/TardivoDFMP23.pdf}{TardivoDFMP23}~\cite{TardivoDFMP23}, \href{../works/Fatemi-AnarakiTFV23.pdf}{Fatemi-AnarakiTFV23}~\cite{Fatemi-AnarakiTFV23}, \href{../works/Teppan22.pdf}{Teppan22}~\cite{Teppan22}, \href{../works/CampeauG22.pdf}{CampeauG22}~\cite{CampeauG22}, \href{../works/JungblutK22.pdf}{JungblutK22}~\cite{JungblutK22}, \href{../works/PopovicCGNC22.pdf}{PopovicCGNC22}~\cite{PopovicCGNC22}, \href{../works/FetgoD22.pdf}{FetgoD22}~\cite{FetgoD22}, \href{../works/EmdeZD22.pdf}{EmdeZD22}~\cite{EmdeZD22}, \href{../works/NaderiBZ22.pdf}{NaderiBZ22}~\cite{NaderiBZ22}, \href{../works/KoehlerBFFHPSSS21.pdf}{KoehlerBFFHPSSS21}~\cite{KoehlerBFFHPSSS21}, \href{../works/HanenKP21.pdf}{HanenKP21}~\cite{HanenKP21}, \href{../works/HubnerGSV21.pdf}{HubnerGSV21}~\cite{HubnerGSV21}, \href{../works/Mercier-AubinGQ20.pdf}{Mercier-AubinGQ20}~\cite{Mercier-AubinGQ20}, \href{../works/TangB20.pdf}{TangB20}~\cite{TangB20}, \href{../works/NattafM20.pdf}{NattafM20}~\cite{NattafM20}, \href{../works/CauwelaertDS20.pdf}{CauwelaertDS20}~\cite{CauwelaertDS20}, \href{../works/SacramentoSP20.pdf}{SacramentoSP20}~\cite{SacramentoSP20}, \href{../works/MurinR19.pdf}{MurinR19}~\cite{MurinR19}, \href{../works/abs-1911-04766.pdf}{abs-1911-04766}~\cite{abs-1911-04766}, \href{../works/NishikawaSTT19.pdf}{NishikawaSTT19}~\cite{NishikawaSTT19}, \href{../works/NattafHKAL19.pdf}{NattafHKAL19}~\cite{NattafHKAL19}, \href{../works/BadicaBIL19.pdf}{BadicaBIL19}~\cite{BadicaBIL19}, \href{../works/Tom19.pdf}{Tom19}~\cite{Tom19}, \href{../works/GeibingerMM19.pdf}{GeibingerMM19}~\cite{GeibingerMM19}, \href{../works/Ham18.pdf}{Ham18}~\cite{Ham18}... (Total: 104)\\
Concepts & manpower & \href{../works/NovaraNH16.pdf}{NovaraNH16}~\cite{NovaraNH16} & \href{../works/LaborieRSV18.pdf}{LaborieRSV18}~\cite{LaborieRSV18}, \href{../works/Froger16.pdf}{Froger16}~\cite{Froger16} & \href{../works/BourreauGGLT22.pdf}{BourreauGGLT22}~\cite{BourreauGGLT22}, \href{../works/BadicaBI20.pdf}{BadicaBI20}~\cite{BadicaBI20}, \href{../works/MokhtarzadehTNF20.pdf}{MokhtarzadehTNF20}~\cite{MokhtarzadehTNF20}, \href{../works/HauderBRPA20.pdf}{HauderBRPA20}~\cite{HauderBRPA20}, \href{../works/WikarekS19.pdf}{WikarekS19}~\cite{WikarekS19}, \href{../works/BaptisteB18.pdf}{BaptisteB18}~\cite{BaptisteB18}, \href{../works/MusliuSS18.pdf}{MusliuSS18}~\cite{MusliuSS18}, \href{../works/SchuttS16.pdf}{SchuttS16}~\cite{SchuttS16}, \href{../works/HechingH16.pdf}{HechingH16}~\cite{HechingH16}, \href{../works/GayHS15a.pdf}{GayHS15a}~\cite{GayHS15a}, \href{../works/GaySS14.pdf}{GaySS14}~\cite{GaySS14}, \href{../works/HarjunkoskiMBC14.pdf}{HarjunkoskiMBC14}~\cite{HarjunkoskiMBC14}, \href{../works/Clercq12.pdf}{Clercq12}~\cite{Clercq12}, \href{../works/GuyonLPR12.pdf}{GuyonLPR12}~\cite{GuyonLPR12}, \href{../works/LombardiM12.pdf}{LombardiM12}~\cite{LombardiM12}, \href{../works/SimonisH11.pdf}{SimonisH11}~\cite{SimonisH11}, \href{../works/Menana11.pdf}{Menana11}~\cite{Menana11}, \href{../works/Vilim11.pdf}{Vilim11}~\cite{Vilim11}, \href{../works/NovasH10.pdf}{NovasH10}~\cite{NovasH10}, \href{../works/ChenGPSH10.pdf}{ChenGPSH10}~\cite{ChenGPSH10}, \href{../works/Simonis99.pdf}{Simonis99}~\cite{Simonis99}, \href{../works/NuijtenP98.pdf}{NuijtenP98}~\cite{NuijtenP98}, \href{../works/SimonisC95.pdf}{SimonisC95}~\cite{SimonisC95}, \href{../works/Simonis95a.pdf}{Simonis95a}~\cite{Simonis95a}, \href{../works/Puget95.pdf}{Puget95}~\cite{Puget95}\\
Concepts & multi-agent & \href{../works/SvancaraB22.pdf}{SvancaraB22}~\cite{SvancaraB22}, \href{../works/Zahout21.pdf}{Zahout21}~\cite{Zahout21}, \href{../works/ZarandiASC20.pdf}{ZarandiASC20}~\cite{ZarandiASC20}, \href{../works/BehrensLM19.pdf}{BehrensLM19}~\cite{BehrensLM19}, \href{../works/He0GLW18.pdf}{He0GLW18}~\cite{He0GLW18}, \href{../works/GombolayWS18.pdf}{GombolayWS18}~\cite{GombolayWS18}, \href{../works/HoeveGSL07.pdf}{HoeveGSL07}~\cite{HoeveGSL07} & \href{../works/Lemos21.pdf}{Lemos21}~\cite{Lemos21}, \href{../works/MokhtarzadehTNF20.pdf}{MokhtarzadehTNF20}~\cite{MokhtarzadehTNF20}, \href{../works/abs-1901-07914.pdf}{abs-1901-07914}~\cite{abs-1901-07914}, \href{../works/TranVNB17.pdf}{TranVNB17}~\cite{TranVNB17}, \href{../works/LimHTB16.pdf}{LimHTB16}~\cite{LimHTB16}, \href{../works/BartakSR10.pdf}{BartakSR10}~\cite{BartakSR10}, \href{../works/BocewiczBB09.pdf}{BocewiczBB09}~\cite{BocewiczBB09} & \href{../works/abs-2402-00459.pdf}{abs-2402-00459}~\cite{abs-2402-00459}, \href{../works/Mehdizadeh-Somarin23.pdf}{Mehdizadeh-Somarin23}~\cite{Mehdizadeh-Somarin23}, \href{../works/SquillaciPR23.pdf}{SquillaciPR23}~\cite{SquillaciPR23}, \href{../works/ZhuSZW23.pdf}{ZhuSZW23}~\cite{ZhuSZW23}, \href{../works/GokPTGO23.pdf}{GokPTGO23}~\cite{GokPTGO23}, \href{../works/Fatemi-AnarakiTFV23.pdf}{Fatemi-AnarakiTFV23}~\cite{Fatemi-AnarakiTFV23}, \href{../works/AbreuAPNM21.pdf}{AbreuAPNM21}~\cite{AbreuAPNM21}, \href{../works/ZhangYW21.pdf}{ZhangYW21}~\cite{ZhangYW21}, \href{../works/GokGSTO20.pdf}{GokGSTO20}~\cite{GokGSTO20}, \href{../works/WessenCS20.pdf}{WessenCS20}~\cite{WessenCS20}, \href{../works/MejiaY20.pdf}{MejiaY20}~\cite{MejiaY20}, \href{../works/WikarekS19.pdf}{WikarekS19}~\cite{WikarekS19}, \href{../works/BadicaBIL19.pdf}{BadicaBIL19}~\cite{BadicaBIL19}, \href{../works/ZhangW18.pdf}{ZhangW18}~\cite{ZhangW18}, \href{../works/HookerH17.pdf}{HookerH17}~\cite{HookerH17}, \href{../works/LimBTBB15.pdf}{LimBTBB15}~\cite{LimBTBB15}, \href{../works/KoschB14.pdf}{KoschB14}~\cite{KoschB14}, \href{../works/BartakS11.pdf}{BartakS11}~\cite{BartakS11}, \href{../works/Jans09.pdf}{Jans09}~\cite{Jans09}, \href{../works/GomesHS06.pdf}{GomesHS06}~\cite{GomesHS06}, \href{../works/AbrilSB05.pdf}{AbrilSB05}~\cite{AbrilSB05}, \href{../works/Beck99.pdf}{Beck99}~\cite{Beck99}, \href{../works/BeckF98.pdf}{BeckF98}~\cite{BeckF98}, \href{../works/Wallace96.pdf}{Wallace96}~\cite{Wallace96}, \href{../works/Pape94.pdf}{Pape94}~\cite{Pape94}\\
Concepts & multi-objective & \href{../works/IsikYA23.pdf}{IsikYA23}~\cite{IsikYA23}, \href{../works/AfsarVPG23.pdf}{AfsarVPG23}~\cite{AfsarVPG23}, \href{../works/FarsiTM22.pdf}{FarsiTM22}~\cite{FarsiTM22}, \href{../works/SubulanC22.pdf}{SubulanC22}~\cite{SubulanC22}, \href{../works/YunusogluY22.pdf}{YunusogluY22}~\cite{YunusogluY22}, \href{../works/Lemos21.pdf}{Lemos21}~\cite{Lemos21}, \href{../works/HamPK21.pdf}{HamPK21}~\cite{HamPK21}, \href{../works/ZarandiASC20.pdf}{ZarandiASC20}~\cite{ZarandiASC20}, \href{../works/Tom19.pdf}{Tom19}~\cite{Tom19}, \href{../works/TangLWSK18.pdf}{TangLWSK18}~\cite{TangLWSK18}, \href{../works/Froger16.pdf}{Froger16}~\cite{Froger16}, \href{../works/Dejemeppe16.pdf}{Dejemeppe16}~\cite{Dejemeppe16}, \href{../works/TopalogluO11.pdf}{TopalogluO11}~\cite{TopalogluO11}, \href{../works/ZeballosQH10.pdf}{ZeballosQH10}~\cite{ZeballosQH10} & \href{../works/PrataAN23.pdf}{PrataAN23}~\cite{PrataAN23}, \href{../works/GurPAE23.pdf}{GurPAE23}~\cite{GurPAE23}, \href{../works/CzerniachowskaWZ23.pdf}{CzerniachowskaWZ23}~\cite{CzerniachowskaWZ23}, \href{../works/LacknerMMWW23.pdf}{LacknerMMWW23}~\cite{LacknerMMWW23}, \href{../works/AbreuPNF23.pdf}{AbreuPNF23}~\cite{AbreuPNF23}, \href{../works/LiFJZLL22.pdf}{LiFJZLL22}~\cite{LiFJZLL22}, \href{../works/OrnekOS20.pdf}{OrnekOS20}~\cite{OrnekOS20}, \href{../works/NaderiBZ22a.pdf}{NaderiBZ22a}~\cite{NaderiBZ22a}, \href{../works/AbreuN22.pdf}{AbreuN22}~\cite{AbreuN22}, \href{../works/ZhangJZL22.pdf}{ZhangJZL22}~\cite{ZhangJZL22}, \href{../works/AbreuAPNM21.pdf}{AbreuAPNM21}~\cite{AbreuAPNM21}, \href{../works/FanXG21.pdf}{FanXG21}~\cite{FanXG21}, \href{../works/QinWSLS21.pdf}{QinWSLS21}~\cite{QinWSLS21}, \href{../works/AbohashimaEG21.pdf}{AbohashimaEG21}~\cite{AbohashimaEG21}, \href{../works/ZhangYW21.pdf}{ZhangYW21}~\cite{ZhangYW21}, \href{../works/Zahout21.pdf}{Zahout21}~\cite{Zahout21}, \href{../works/MejiaY20.pdf}{MejiaY20}~\cite{MejiaY20}, \href{../works/ZouZ20.pdf}{ZouZ20}~\cite{ZouZ20}, \href{../works/MengZRZL20.pdf}{MengZRZL20}~\cite{MengZRZL20}, \href{../works/Lunardi20.pdf}{Lunardi20}~\cite{Lunardi20}, \href{../works/YounespourAKE19.pdf}{YounespourAKE19}~\cite{YounespourAKE19}, \href{../works/EscobetPQPRA19.pdf}{EscobetPQPRA19}~\cite{EscobetPQPRA19}, \href{../works/PourDERB18.pdf}{PourDERB18}~\cite{PourDERB18}, \href{../works/CappartTSR18.pdf}{CappartTSR18}~\cite{CappartTSR18}, \href{../works/MenciaSV13.pdf}{MenciaSV13}~\cite{MenciaSV13}, \href{../works/QuirogaZH05.pdf}{QuirogaZH05}~\cite{QuirogaZH05} & \href{../works/abs-2402-00459.pdf}{abs-2402-00459}~\cite{abs-2402-00459}, \href{../works/GokPTGO23.pdf}{GokPTGO23}~\cite{GokPTGO23}, \href{../works/SquillaciPR23.pdf}{SquillaciPR23}~\cite{SquillaciPR23}, \href{../works/MarliereSPR23.pdf}{MarliereSPR23}~\cite{MarliereSPR23}, \href{../works/AlfieriGPS23.pdf}{AlfieriGPS23}~\cite{AlfieriGPS23}, \href{../works/YuraszeckMCCR23.pdf}{YuraszeckMCCR23}~\cite{YuraszeckMCCR23}, \href{../works/GuoZ23.pdf}{GuoZ23}~\cite{GuoZ23}, \href{../works/MullerMKP22.pdf}{MullerMKP22}~\cite{MullerMKP22}, \href{../works/abs-2211-14492.pdf}{abs-2211-14492}~\cite{abs-2211-14492}, \href{../works/ColT22.pdf}{ColT22}~\cite{ColT22}, \href{../works/OujanaAYB22.pdf}{OujanaAYB22}~\cite{OujanaAYB22}, \href{../works/BoudreaultSLQ22.pdf}{BoudreaultSLQ22}~\cite{BoudreaultSLQ22}, \href{../works/TouatBT22.pdf}{TouatBT22}~\cite{TouatBT22}, \href{../works/ArmstrongGOS21.pdf}{ArmstrongGOS21}~\cite{ArmstrongGOS21}, \href{../works/Astrand21.pdf}{Astrand21}~\cite{Astrand21}, \href{../works/KoehlerBFFHPSSS21.pdf}{KoehlerBFFHPSSS21}~\cite{KoehlerBFFHPSSS21}, \href{../works/GroleazNS20a.pdf}{GroleazNS20a}~\cite{GroleazNS20a}, \href{../works/Polo-MejiaALB20.pdf}{Polo-MejiaALB20}~\cite{Polo-MejiaALB20}, \href{../works/SacramentoSP20.pdf}{SacramentoSP20}~\cite{SacramentoSP20}, \href{../works/HauderBRPA20.pdf}{HauderBRPA20}~\cite{HauderBRPA20}, \href{../works/FrohnerTR19.pdf}{FrohnerTR19}~\cite{FrohnerTR19}, \href{../works/KucukY19.pdf}{KucukY19}~\cite{KucukY19}, \href{../works/Novas19.pdf}{Novas19}~\cite{Novas19}, \href{../works/GurEA19.pdf}{GurEA19}~\cite{GurEA19}, \href{../works/abs-1902-09244.pdf}{abs-1902-09244}~\cite{abs-1902-09244}, \href{../works/GeibingerMM19.pdf}{GeibingerMM19}~\cite{GeibingerMM19}, \href{../works/abs-1911-04766.pdf}{abs-1911-04766}~\cite{abs-1911-04766}, \href{../works/Hooker19.pdf}{Hooker19}~\cite{Hooker19}, \href{../works/He0GLW18.pdf}{He0GLW18}~\cite{He0GLW18}... (Total: 58)\\
Concepts & net present value & \href{../works/ThiruvadyWGS14.pdf}{ThiruvadyWGS14}~\cite{ThiruvadyWGS14}, \href{../works/GuSS13.pdf}{GuSS13}~\cite{GuSS13}, \href{../works/SchuttCSW12.pdf}{SchuttCSW12}~\cite{SchuttCSW12}, \href{../works/GuSW12.pdf}{GuSW12}~\cite{GuSW12} & \href{../works/CampeauG22.pdf}{CampeauG22}~\cite{CampeauG22}, \href{../works/HillTV21.pdf}{HillTV21}~\cite{HillTV21}, \href{../works/KelarevaTK13.pdf}{KelarevaTK13}~\cite{KelarevaTK13} & \href{../works/abs-2402-00459.pdf}{abs-2402-00459}~\cite{abs-2402-00459}, \href{../works/EtminaniesfahaniGNMS22.pdf}{EtminaniesfahaniGNMS22}~\cite{EtminaniesfahaniGNMS22}, \href{../works/Astrand21.pdf}{Astrand21}~\cite{Astrand21}, \href{../works/AstrandJZ20.pdf}{AstrandJZ20}~\cite{AstrandJZ20}, \href{../works/ZarandiASC20.pdf}{ZarandiASC20}~\cite{ZarandiASC20}, \href{../works/LaborieRSV18.pdf}{LaborieRSV18}~\cite{LaborieRSV18}, \href{../works/HookerH17.pdf}{HookerH17}~\cite{HookerH17}, \href{../works/MossigeGSMC17.pdf}{MossigeGSMC17}~\cite{MossigeGSMC17}, \href{../works/SzerediS16.pdf}{SzerediS16}~\cite{SzerediS16}, \href{../works/SchuttS16.pdf}{SchuttS16}~\cite{SchuttS16}, \href{../works/SchnellH15.pdf}{SchnellH15}~\cite{SchnellH15}, \href{../works/BlomBPS14.pdf}{BlomBPS14}~\cite{BlomBPS14}, \href{../works/SchuttFS13.pdf}{SchuttFS13}~\cite{SchuttFS13}, \href{../works/Lombardi10.pdf}{Lombardi10}~\cite{Lombardi10}\\
Concepts & no preempt &  &  & \href{../works/ColT22.pdf}{ColT22}~\cite{ColT22}, \href{../works/TouatBT22.pdf}{TouatBT22}~\cite{TouatBT22}, \href{../works/FanXG21.pdf}{FanXG21}~\cite{FanXG21}, \href{../works/Bedhief21.pdf}{Bedhief21}~\cite{Bedhief21}, \href{../works/Lunardi20.pdf}{Lunardi20}~\cite{Lunardi20}, \href{../works/MengZRZL20.pdf}{MengZRZL20}~\cite{MengZRZL20}, \href{../works/ParkUJR19.pdf}{ParkUJR19}~\cite{ParkUJR19}, \href{../works/NattafALR16.pdf}{NattafALR16}~\cite{NattafALR16}, \href{../works/TerekhovTDB14.pdf}{TerekhovTDB14}~\cite{TerekhovTDB14}, \href{../works/LombardiMRB10.pdf}{LombardiMRB10}~\cite{LombardiMRB10}, \href{../works/LiW08.pdf}{LiW08}~\cite{LiW08}, \href{../works/MonetteDD07.pdf}{MonetteDD07}~\cite{MonetteDD07}, \href{../works/BeckW07.pdf}{BeckW07}~\cite{BeckW07}, \href{../works/Baptiste02.pdf}{Baptiste02}~\cite{Baptiste02}, \href{../works/ArtiguesR00.pdf}{ArtiguesR00}~\cite{ArtiguesR00}\\
Concepts & no-wait & \href{../works/PrataAN23.pdf}{PrataAN23}~\cite{PrataAN23}, \href{../works/Fatemi-AnarakiTFV23.pdf}{Fatemi-AnarakiTFV23}~\cite{Fatemi-AnarakiTFV23}, \href{../works/IsikYA23.pdf}{IsikYA23}~\cite{IsikYA23}, \href{../works/AlfieriGPS23.pdf}{AlfieriGPS23}~\cite{AlfieriGPS23}, \href{../works/NaderiRR23.pdf}{NaderiRR23}~\cite{NaderiRR23}, \href{../works/AbreuNP23.pdf}{AbreuNP23}~\cite{AbreuNP23}, \href{../works/HubnerGSV21.pdf}{HubnerGSV21}~\cite{HubnerGSV21}, \href{../works/VlkHT21.pdf}{VlkHT21}~\cite{VlkHT21}, \href{../works/ZarandiASC20.pdf}{ZarandiASC20}~\cite{ZarandiASC20}, \href{../works/Novas19.pdf}{Novas19}~\cite{Novas19}, \href{../works/GrimesH15.pdf}{GrimesH15}~\cite{GrimesH15}, \href{../works/GrimesH11.pdf}{GrimesH11}~\cite{GrimesH11}, \href{../works/GrimesH10.pdf}{GrimesH10}~\cite{GrimesH10}, \href{../works/AkkerDH07.pdf}{AkkerDH07}~\cite{AkkerDH07} & \href{../works/AbreuN22.pdf}{AbreuN22}~\cite{AbreuN22}, \href{../works/AbreuAPNM21.pdf}{AbreuAPNM21}~\cite{AbreuAPNM21}, \href{../works/MengZRZL20.pdf}{MengZRZL20}~\cite{MengZRZL20}, \href{../works/MokhtarzadehTNF20.pdf}{MokhtarzadehTNF20}~\cite{MokhtarzadehTNF20}, \href{../works/MejiaY20.pdf}{MejiaY20}~\cite{MejiaY20}, \href{../works/Dejemeppe16.pdf}{Dejemeppe16}~\cite{Dejemeppe16}, \href{../works/Malapert11.pdf}{Malapert11}~\cite{Malapert11} & \href{../works/AbreuPNF23.pdf}{AbreuPNF23}~\cite{AbreuPNF23}, \href{../works/MarliereSPR23.pdf}{MarliereSPR23}~\cite{MarliereSPR23}, \href{../works/YuraszeckMPV22.pdf}{YuraszeckMPV22}~\cite{YuraszeckMPV22}, \href{../works/BourreauGGLT22.pdf}{BourreauGGLT22}~\cite{BourreauGGLT22}, \href{../works/ArmstrongGOS22.pdf}{ArmstrongGOS22}~\cite{ArmstrongGOS22}, \href{../works/EmdeZD22.pdf}{EmdeZD22}~\cite{EmdeZD22}, \href{../works/LiFJZLL22.pdf}{LiFJZLL22}~\cite{LiFJZLL22}, \href{../works/FarsiTM22.pdf}{FarsiTM22}~\cite{FarsiTM22}, \href{../works/MullerMKP22.pdf}{MullerMKP22}~\cite{MullerMKP22}, \href{../works/NaderiBZ22.pdf}{NaderiBZ22}~\cite{NaderiBZ22}, \href{../works/Bedhief21.pdf}{Bedhief21}~\cite{Bedhief21}, \href{../works/HauderBRPA20.pdf}{HauderBRPA20}~\cite{HauderBRPA20}, \href{../works/abs-1902-09244.pdf}{abs-1902-09244}~\cite{abs-1902-09244}, \href{../works/RiahiNS018.pdf}{RiahiNS018}~\cite{RiahiNS018}, \href{../works/ZhangW18.pdf}{ZhangW18}~\cite{ZhangW18}, \href{../works/ArbaouiY18.pdf}{ArbaouiY18}~\cite{ArbaouiY18}, \href{../works/WangMD15.pdf}{WangMD15}~\cite{WangMD15}, \href{../works/NovasH12.pdf}{NovasH12}~\cite{NovasH12}, \href{../works/HermenierDL11.pdf}{HermenierDL11}~\cite{HermenierDL11}, \href{../works/NovasH10.pdf}{NovasH10}~\cite{NovasH10}, \href{../works/RodriguezS09.pdf}{RodriguezS09}~\cite{RodriguezS09}, \href{../works/Rodriguez07b.pdf}{Rodriguez07b}~\cite{Rodriguez07b}, \href{../works/LammaMM97.pdf}{LammaMM97}~\cite{LammaMM97}, \href{../works/BrusoniCLMMT96.pdf}{BrusoniCLMMT96}~\cite{BrusoniCLMMT96}, \href{../works/BlazewiczDP96.pdf}{BlazewiczDP96}~\cite{BlazewiczDP96}\\
Concepts & one-machine scheduling & \href{../works/MilanoW09.pdf}{MilanoW09}~\cite{MilanoW09}, \href{../works/MilanoW06.pdf}{MilanoW06}~\cite{MilanoW06}, \href{../works/BlazewiczDP96.pdf}{BlazewiczDP96}~\cite{BlazewiczDP96} & \href{../works/ZhangBB22.pdf}{ZhangBB22}~\cite{ZhangBB22}, \href{../works/Schutt11.pdf}{Schutt11}~\cite{Schutt11}, \href{../works/Baptiste02.pdf}{Baptiste02}~\cite{Baptiste02} & \href{../works/PenzDN23.pdf}{PenzDN23}~\cite{PenzDN23}, \href{../works/ColT22.pdf}{ColT22}~\cite{ColT22}, \href{../works/Astrand21.pdf}{Astrand21}~\cite{Astrand21}, \href{../works/FanXG21.pdf}{FanXG21}~\cite{FanXG21}, \href{../works/KoehlerBFFHPSSS21.pdf}{KoehlerBFFHPSSS21}~\cite{KoehlerBFFHPSSS21}, \href{../works/ZarandiASC20.pdf}{ZarandiASC20}~\cite{ZarandiASC20}, \href{../works/Hooker19.pdf}{Hooker19}~\cite{Hooker19}, \href{../works/HookerH17.pdf}{HookerH17}~\cite{HookerH17}, \href{../works/MelgarejoLS15.pdf}{MelgarejoLS15}~\cite{MelgarejoLS15}, \href{../works/BeniniLMR11.pdf}{BeniniLMR11}~\cite{BeniniLMR11}, \href{../works/SadykovW06.pdf}{SadykovW06}~\cite{SadykovW06}, \href{../works/ChuX05.pdf}{ChuX05}~\cite{ChuX05}, \href{../works/BeckW04.pdf}{BeckW04}~\cite{BeckW04}, \href{../works/ArtiguesBF04.pdf}{ArtiguesBF04}~\cite{ArtiguesBF04}, \href{../works/Sadykov04.pdf}{Sadykov04}~\cite{Sadykov04}, \href{../works/HookerO03.pdf}{HookerO03}~\cite{HookerO03}, \href{../works/JainM99.pdf}{JainM99}~\cite{JainM99}\\
Concepts & online scheduling & \href{../works/TerekhovTDB14.pdf}{TerekhovTDB14}~\cite{TerekhovTDB14} & \href{../works/Mehdizadeh-Somarin23.pdf}{Mehdizadeh-Somarin23}~\cite{Mehdizadeh-Somarin23}, \href{../works/Zahout21.pdf}{Zahout21}~\cite{Zahout21}, \href{../works/Groleaz21.pdf}{Groleaz21}~\cite{Groleaz21} & \href{../works/PrataAN23.pdf}{PrataAN23}~\cite{PrataAN23}, \href{../works/MullerMKP22.pdf}{MullerMKP22}~\cite{MullerMKP22}, \href{../works/VlkHT21.pdf}{VlkHT21}~\cite{VlkHT21}, \href{../works/NishikawaSTT19.pdf}{NishikawaSTT19}~\cite{NishikawaSTT19}, \href{../works/TranPZLDB18.pdf}{TranPZLDB18}~\cite{TranPZLDB18}, \href{../works/HebrardHJMPV16.pdf}{HebrardHJMPV16}~\cite{HebrardHJMPV16}, \href{../works/LimHTB16.pdf}{LimHTB16}~\cite{LimHTB16}, \href{../works/ZhouGL15.pdf}{ZhouGL15}~\cite{ZhouGL15}, \href{../works/DoomsH08.pdf}{DoomsH08}~\cite{DoomsH08}, \href{../works/ElkhyariGJ02a.pdf}{ElkhyariGJ02a}~\cite{ElkhyariGJ02a}\\
Concepts & open-shop & \href{../works/PrataAN23.pdf}{PrataAN23}~\cite{PrataAN23}, \href{../works/Bit-Monnot23.pdf}{Bit-Monnot23}~\cite{Bit-Monnot23}, \href{../works/AbreuPNF23.pdf}{AbreuPNF23}~\cite{AbreuPNF23}, \href{../works/AbreuNP23.pdf}{AbreuNP23}~\cite{AbreuNP23}, \href{../works/NaderiRR23.pdf}{NaderiRR23}~\cite{NaderiRR23}, \href{../works/YuraszeckMPV22.pdf}{YuraszeckMPV22}~\cite{YuraszeckMPV22}, \href{../works/AbreuN22.pdf}{AbreuN22}~\cite{AbreuN22}, \href{../works/AbreuAPNM21.pdf}{AbreuAPNM21}~\cite{AbreuAPNM21}, \href{../works/Groleaz21.pdf}{Groleaz21}~\cite{Groleaz21}, \href{../works/ZarandiASC20.pdf}{ZarandiASC20}~\cite{ZarandiASC20}, \href{../works/MejiaY20.pdf}{MejiaY20}~\cite{MejiaY20}, \href{../works/Lunardi20.pdf}{Lunardi20}~\cite{Lunardi20}, \href{../works/FahimiOQ18.pdf}{FahimiOQ18}~\cite{FahimiOQ18}, \href{../works/Fahimi16.pdf}{Fahimi16}~\cite{Fahimi16}, \href{../works/GrimesH15.pdf}{GrimesH15}~\cite{GrimesH15}, \href{../works/Siala15a.pdf}{Siala15a}~\cite{Siala15a}, \href{../works/Siala15.pdf}{Siala15}~\cite{Siala15}, \href{../works/MalapertCGJLR13.pdf}{MalapertCGJLR13}~\cite{MalapertCGJLR13}, \href{../works/MalapertCGJLR12.pdf}{MalapertCGJLR12}~\cite{MalapertCGJLR12}, \href{../works/Malapert11.pdf}{Malapert11}~\cite{Malapert11}, \href{../works/GrimesHM09.pdf}{GrimesHM09}~\cite{GrimesHM09}, \href{../works/OhrimenkoSC09.pdf}{OhrimenkoSC09}~\cite{OhrimenkoSC09}, \href{../works/MonetteDD07.pdf}{MonetteDD07}~\cite{MonetteDD07}, \href{../works/Elkhyari03.pdf}{Elkhyari03}~\cite{Elkhyari03}, \href{../works/LorigeonBB02.pdf}{LorigeonBB02}~\cite{LorigeonBB02}, \href{../works/Baptiste02.pdf}{Baptiste02}~\cite{Baptiste02}, \href{../works/FocacciLN00.pdf}{FocacciLN00}~\cite{FocacciLN00} & \href{../works/ZhuSZW23.pdf}{ZhuSZW23}~\cite{ZhuSZW23}, \href{../works/Godet21a.pdf}{Godet21a}~\cite{Godet21a}, \href{../works/Astrand21.pdf}{Astrand21}~\cite{Astrand21}, \href{../works/SacramentoSP20.pdf}{SacramentoSP20}~\cite{SacramentoSP20}, \href{../works/MengZRZL20.pdf}{MengZRZL20}~\cite{MengZRZL20}, \href{../works/Dejemeppe16.pdf}{Dejemeppe16}~\cite{Dejemeppe16}, \href{../works/TerekhovDOB12.pdf}{TerekhovDOB12}~\cite{TerekhovDOB12}, \href{../works/Schutt11.pdf}{Schutt11}~\cite{Schutt11}, \href{../works/GrimesH10.pdf}{GrimesH10}~\cite{GrimesH10}, \href{../works/Vilim05.pdf}{Vilim05}~\cite{Vilim05}, \href{../works/Demassey03.pdf}{Demassey03}~\cite{Demassey03}, \href{../works/JainM99.pdf}{JainM99}~\cite{JainM99} & \href{../works/BonninMNE24.pdf}{BonninMNE24}~\cite{BonninMNE24}, \href{../works/YuraszeckMCCR23.pdf}{YuraszeckMCCR23}~\cite{YuraszeckMCCR23}, \href{../works/YuraszeckMC23.pdf}{YuraszeckMC23}~\cite{YuraszeckMC23}, \href{../works/KimCMLLP23.pdf}{KimCMLLP23}~\cite{KimCMLLP23}, \href{../works/ShaikhK23.pdf}{ShaikhK23}~\cite{ShaikhK23}, \href{../works/AfsarVPG23.pdf}{AfsarVPG23}~\cite{AfsarVPG23}, \href{../works/NaderiBZ22.pdf}{NaderiBZ22}~\cite{NaderiBZ22}, \href{../works/EmdeZD22.pdf}{EmdeZD22}~\cite{EmdeZD22}, \href{../works/OujanaAYB22.pdf}{OujanaAYB22}~\cite{OujanaAYB22}, \href{../works/ColT22.pdf}{ColT22}~\cite{ColT22}, \href{../works/EtminaniesfahaniGNMS22.pdf}{EtminaniesfahaniGNMS22}~\cite{EtminaniesfahaniGNMS22}, \href{../works/Astrand0F21.pdf}{Astrand0F21}~\cite{Astrand0F21}, \href{../works/abs-2102-08778.pdf}{abs-2102-08778}~\cite{abs-2102-08778}, \href{../works/AstrandJZ20.pdf}{AstrandJZ20}~\cite{AstrandJZ20}, \href{../works/ParkUJR19.pdf}{ParkUJR19}~\cite{ParkUJR19}, \href{../works/GombolayWS18.pdf}{GombolayWS18}~\cite{GombolayWS18}, \href{../works/HookerH17.pdf}{HookerH17}~\cite{HookerH17}, \href{../works/SialaAH15.pdf}{SialaAH15}~\cite{SialaAH15}, \href{../works/Derrien15.pdf}{Derrien15}~\cite{Derrien15}, \href{../works/BonfiettiLM14.pdf}{BonfiettiLM14}~\cite{BonfiettiLM14}, \href{../works/AlesioNBG14.pdf}{AlesioNBG14}~\cite{AlesioNBG14}, \href{../works/BillautHL12.pdf}{BillautHL12}~\cite{BillautHL12}, \href{../works/GrimesH11.pdf}{GrimesH11}~\cite{GrimesH11}, \href{../works/SchuttFSW11.pdf}{SchuttFSW11}~\cite{SchuttFSW11}, \href{../works/ChenGPSH10.pdf}{ChenGPSH10}~\cite{ChenGPSH10}, \href{../works/BartakSR10.pdf}{BartakSR10}~\cite{BartakSR10}, \href{../works/SchuttFSW09.pdf}{SchuttFSW09}~\cite{SchuttFSW09}, \href{../works/ThiruvadyBME09.pdf}{ThiruvadyBME09}~\cite{ThiruvadyBME09}, \href{../works/LiW08.pdf}{LiW08}~\cite{LiW08}... (Total: 37)\\
Concepts & order & \href{../works/PrataAN23.pdf}{PrataAN23}~\cite{PrataAN23}, \href{../works/BonninMNE24.pdf}{BonninMNE24}~\cite{BonninMNE24}, \href{../works/abs-2402-00459.pdf}{abs-2402-00459}~\cite{abs-2402-00459}, \href{../works/GokPTGO23.pdf}{GokPTGO23}~\cite{GokPTGO23}, \href{../works/ZhuSZW23.pdf}{ZhuSZW23}~\cite{ZhuSZW23}, \href{../works/GuoZ23.pdf}{GuoZ23}~\cite{GuoZ23}, \href{../works/EfthymiouY23.pdf}{EfthymiouY23}~\cite{EfthymiouY23}, \href{../works/AbreuNP23.pdf}{AbreuNP23}~\cite{AbreuNP23}, \href{../works/Fatemi-AnarakiTFV23.pdf}{Fatemi-AnarakiTFV23}~\cite{Fatemi-AnarakiTFV23}, \href{../works/Adelgren2023.pdf}{Adelgren2023}~\cite{Adelgren2023}, \href{../works/TasselGS23.pdf}{TasselGS23}~\cite{TasselGS23}, \href{../works/abs-2306-05747.pdf}{abs-2306-05747}~\cite{abs-2306-05747}, \href{../works/JuvinHL23.pdf}{JuvinHL23}~\cite{JuvinHL23}, \href{../works/LacknerMMWW23.pdf}{LacknerMMWW23}~\cite{LacknerMMWW23}, \href{../works/PerezGSL23.pdf}{PerezGSL23}~\cite{PerezGSL23}, \href{../works/IsikYA23.pdf}{IsikYA23}~\cite{IsikYA23}, \href{../works/PenzDN23.pdf}{PenzDN23}~\cite{PenzDN23}, \href{../works/PovedaAA23.pdf}{PovedaAA23}~\cite{PovedaAA23}, \href{../works/JuvinHL23a.pdf}{JuvinHL23a}~\cite{JuvinHL23a}, \href{../works/AlfieriGPS23.pdf}{AlfieriGPS23}~\cite{AlfieriGPS23}, \href{../works/abs-2312-13682.pdf}{abs-2312-13682}~\cite{abs-2312-13682}, \href{../works/CzerniachowskaWZ23.pdf}{CzerniachowskaWZ23}~\cite{CzerniachowskaWZ23}, \href{../works/AalianPG23.pdf}{AalianPG23}~\cite{AalianPG23}, \href{../works/Bit-Monnot23.pdf}{Bit-Monnot23}~\cite{Bit-Monnot23}, \href{../works/AbreuPNF23.pdf}{AbreuPNF23}~\cite{AbreuPNF23}, \href{../works/WangB23.pdf}{WangB23}~\cite{WangB23}, \href{../works/KameugneFND23.pdf}{KameugneFND23}~\cite{KameugneFND23}, \href{../works/JuvinHHL23.pdf}{JuvinHHL23}~\cite{JuvinHHL23}, \href{../works/SquillaciPR23.pdf}{SquillaciPR23}~\cite{SquillaciPR23}... (Total: 406) & \href{../works/ForbesHJST24.pdf}{ForbesHJST24}~\cite{ForbesHJST24}, \href{../works/MontemanniD23a.pdf}{MontemanniD23a}~\cite{MontemanniD23a}, \href{../works/NaderiRR23.pdf}{NaderiRR23}~\cite{NaderiRR23}, \href{../works/TardivoDFMP23.pdf}{TardivoDFMP23}~\cite{TardivoDFMP23}, \href{../works/YuraszeckMC23.pdf}{YuraszeckMC23}~\cite{YuraszeckMC23}, \href{../works/GurPAE23.pdf}{GurPAE23}~\cite{GurPAE23}, \href{../works/ShaikhK23.pdf}{ShaikhK23}~\cite{ShaikhK23}, \href{../works/abs-2305-19888.pdf}{abs-2305-19888}~\cite{abs-2305-19888}, \href{../works/SvancaraB22.pdf}{SvancaraB22}~\cite{SvancaraB22}, \href{../works/ZhangBB22.pdf}{ZhangBB22}~\cite{ZhangBB22}, \href{../works/ArmstrongGOS22.pdf}{ArmstrongGOS22}~\cite{ArmstrongGOS22}, \href{../works/WinterMMW22.pdf}{WinterMMW22}~\cite{WinterMMW22}, \href{../works/ElciOH22.pdf}{ElciOH22}~\cite{ElciOH22}, \href{../works/OrnekOS20.pdf}{OrnekOS20}~\cite{OrnekOS20}, \href{../works/TouatBT22.pdf}{TouatBT22}~\cite{TouatBT22}, \href{../works/OuelletQ22.pdf}{OuelletQ22}~\cite{OuelletQ22}, \href{../works/HeinzNVH22.pdf}{HeinzNVH22}~\cite{HeinzNVH22}, \href{../works/JungblutK22.pdf}{JungblutK22}~\cite{JungblutK22}, \href{../works/BenderWS21.pdf}{BenderWS21}~\cite{BenderWS21}, \href{../works/GeibingerMM21.pdf}{GeibingerMM21}~\cite{GeibingerMM21}, \href{../works/HillTV21.pdf}{HillTV21}~\cite{HillTV21}, \href{../works/abs-2102-08778.pdf}{abs-2102-08778}~\cite{abs-2102-08778}, \href{../works/QinDCS20.pdf}{QinDCS20}~\cite{QinDCS20}, \href{../works/WallaceY20.pdf}{WallaceY20}~\cite{WallaceY20}, \href{../works/AntunesABD20.pdf}{AntunesABD20}~\cite{AntunesABD20}, \href{../works/ZouZ20.pdf}{ZouZ20}~\cite{ZouZ20}, \href{../works/TangB20.pdf}{TangB20}~\cite{TangB20}, \href{../works/GokGSTO20.pdf}{GokGSTO20}~\cite{GokGSTO20}, \href{../works/FrohnerTR19.pdf}{FrohnerTR19}~\cite{FrohnerTR19}... (Total: 112) & \href{../works/Mehdizadeh-Somarin23.pdf}{Mehdizadeh-Somarin23}~\cite{Mehdizadeh-Somarin23}, \href{../works/MontemanniD23.pdf}{MontemanniD23}~\cite{MontemanniD23}, \href{../works/AkramNHRSA23.pdf}{AkramNHRSA23}~\cite{AkramNHRSA23}, \href{../works/JuvinHL22.pdf}{JuvinHL22}~\cite{JuvinHL22}, \href{../works/NaderiBZ22a.pdf}{NaderiBZ22a}~\cite{NaderiBZ22a}, \href{../works/ZhangJZL22.pdf}{ZhangJZL22}~\cite{ZhangJZL22}, \href{../works/ZhangYW21.pdf}{ZhangYW21}~\cite{ZhangYW21}, \href{../works/AbohashimaEG21.pdf}{AbohashimaEG21}~\cite{AbohashimaEG21}, \href{../works/MokhtarzadehTNF20.pdf}{MokhtarzadehTNF20}~\cite{MokhtarzadehTNF20}, \href{../works/RoshanaeiBAUB20.pdf}{RoshanaeiBAUB20}~\cite{RoshanaeiBAUB20}, \href{../works/abs-1902-01193.pdf}{abs-1902-01193}~\cite{abs-1902-01193}, \href{../works/GalleguillosKSB19.pdf}{GalleguillosKSB19}~\cite{GalleguillosKSB19}, \href{../works/KucukY19.pdf}{KucukY19}~\cite{KucukY19}, \href{../works/ArbaouiY18.pdf}{ArbaouiY18}~\cite{ArbaouiY18}, \href{../works/BenediktSMVH18.pdf}{BenediktSMVH18}~\cite{BenediktSMVH18}, \href{../works/He0GLW18.pdf}{He0GLW18}~\cite{He0GLW18}, \href{../works/TranVNB17a.pdf}{TranVNB17a}~\cite{TranVNB17a}, \href{../works/Hooker17.pdf}{Hooker17}~\cite{Hooker17}, \href{../works/HechingH16.pdf}{HechingH16}~\cite{HechingH16}, \href{../works/BridiLBBM16.pdf}{BridiLBBM16}~\cite{BridiLBBM16}, \href{../works/CireCH16.pdf}{CireCH16}~\cite{CireCH16}, \href{../works/Bonfietti16.pdf}{Bonfietti16}~\cite{Bonfietti16}, \href{../works/SzerediS16.pdf}{SzerediS16}~\cite{SzerediS16}, \href{../works/HurleyOS16.pdf}{HurleyOS16}~\cite{HurleyOS16}, \href{../works/Derrien15.pdf}{Derrien15}~\cite{Derrien15}, \href{../works/GayHS15a.pdf}{GayHS15a}~\cite{GayHS15a}, \href{../works/ThiruvadyWGS14.pdf}{ThiruvadyWGS14}~\cite{ThiruvadyWGS14}, \href{../works/DoulabiRP14.pdf}{DoulabiRP14}~\cite{DoulabiRP14}, \href{../works/Kameugne14.pdf}{Kameugne14}~\cite{Kameugne14}... (Total: 65)\\
Concepts & order scheduling & \href{../works/TerekhovDOB12.pdf}{TerekhovDOB12}~\cite{TerekhovDOB12} & \href{../works/PrataAN23.pdf}{PrataAN23}~\cite{PrataAN23}, \href{../works/AbreuPNF23.pdf}{AbreuPNF23}~\cite{AbreuPNF23} & \href{../works/AbreuAPNM21.pdf}{AbreuAPNM21}~\cite{AbreuAPNM21}, \href{../works/QinWSLS21.pdf}{QinWSLS21}~\cite{QinWSLS21}\\
Concepts & periodic & \href{../works/SquillaciPR23.pdf}{SquillaciPR23}~\cite{SquillaciPR23}, \href{../works/Groleaz21.pdf}{Groleaz21}~\cite{Groleaz21}, \href{../works/Lemos21.pdf}{Lemos21}~\cite{Lemos21}, \href{../works/BonfiettiZLM16.pdf}{BonfiettiZLM16}~\cite{BonfiettiZLM16}, \href{../works/Fahimi16.pdf}{Fahimi16}~\cite{Fahimi16}, \href{../works/AlesioNBG14.pdf}{AlesioNBG14}~\cite{AlesioNBG14}, \href{../works/BonfiettiLBM14.pdf}{BonfiettiLBM14}~\cite{BonfiettiLBM14}, \href{../works/TerekhovTDB14.pdf}{TerekhovTDB14}~\cite{TerekhovTDB14}, \href{../works/TranTDB13.pdf}{TranTDB13}~\cite{TranTDB13}, \href{../works/BonfiettiLM13.pdf}{BonfiettiLM13}~\cite{BonfiettiLM13}, \href{../works/SimoninAHL12.pdf}{SimoninAHL12}~\cite{SimoninAHL12}, \href{../works/BonfiettiLBM12.pdf}{BonfiettiLBM12}~\cite{BonfiettiLBM12}, \href{../works/LombardiBMB11.pdf}{LombardiBMB11}~\cite{LombardiBMB11}, \href{../works/Lombardi10.pdf}{Lombardi10}~\cite{Lombardi10}, \href{../works/SchildW00.pdf}{SchildW00}~\cite{SchildW00}, \href{../works/KorbaaYG99.pdf}{KorbaaYG99}~\cite{KorbaaYG99}, \href{../works/PembertonG98.pdf}{PembertonG98}~\cite{PembertonG98} & \href{../works/Mehdizadeh-Somarin23.pdf}{Mehdizadeh-Somarin23}~\cite{Mehdizadeh-Somarin23}, \href{../works/TouatBT22.pdf}{TouatBT22}~\cite{TouatBT22}, \href{../works/Astrand21.pdf}{Astrand21}~\cite{Astrand21}, \href{../works/VlkHT21.pdf}{VlkHT21}~\cite{VlkHT21}, \href{../works/Bonfietti16.pdf}{Bonfietti16}~\cite{Bonfietti16}, \href{../works/BajestaniB15.pdf}{BajestaniB15}~\cite{BajestaniB15}, \href{../works/HarjunkoskiMBC14.pdf}{HarjunkoskiMBC14}~\cite{HarjunkoskiMBC14}, \href{../works/BonfiettiLBM11.pdf}{BonfiettiLBM11}~\cite{BonfiettiLBM11}, \href{../works/Davenport10.pdf}{Davenport10}~\cite{Davenport10}, \href{../works/NovasH10.pdf}{NovasH10}~\cite{NovasH10}, \href{../works/BocewiczBB09.pdf}{BocewiczBB09}~\cite{BocewiczBB09}, \href{../works/BeniniLMR08.pdf}{BeniniLMR08}~\cite{BeniniLMR08} & \href{../works/CzerniachowskaWZ23.pdf}{CzerniachowskaWZ23}~\cite{CzerniachowskaWZ23}, \href{../works/Adelgren2023.pdf}{Adelgren2023}~\cite{Adelgren2023}, \href{../works/PenzDN23.pdf}{PenzDN23}~\cite{PenzDN23}, \href{../works/AbreuPNF23.pdf}{AbreuPNF23}~\cite{AbreuPNF23}, \href{../works/AkramNHRSA23.pdf}{AkramNHRSA23}~\cite{AkramNHRSA23}, \href{../works/abs-2306-05747.pdf}{abs-2306-05747}~\cite{abs-2306-05747}, \href{../works/TasselGS23.pdf}{TasselGS23}~\cite{TasselGS23}, \href{../works/FarsiTM22.pdf}{FarsiTM22}~\cite{FarsiTM22}, \href{../works/OrnekOS20.pdf}{OrnekOS20}~\cite{OrnekOS20}, \href{../works/PopovicCGNC22.pdf}{PopovicCGNC22}~\cite{PopovicCGNC22}, \href{../works/Godet21a.pdf}{Godet21a}~\cite{Godet21a}, \href{../works/AbreuAPNM21.pdf}{AbreuAPNM21}~\cite{AbreuAPNM21}, \href{../works/AntunesABD20.pdf}{AntunesABD20}~\cite{AntunesABD20}, \href{../works/AntuoriHHEN20.pdf}{AntuoriHHEN20}~\cite{AntuoriHHEN20}, \href{../works/AstrandJZ20.pdf}{AstrandJZ20}~\cite{AstrandJZ20}, \href{../works/ZarandiASC20.pdf}{ZarandiASC20}~\cite{ZarandiASC20}, \href{../works/Polo-MejiaALB20.pdf}{Polo-MejiaALB20}~\cite{Polo-MejiaALB20}, \href{../works/Caballero19.pdf}{Caballero19}~\cite{Caballero19}, \href{../works/EscobetPQPRA19.pdf}{EscobetPQPRA19}~\cite{EscobetPQPRA19}, \href{../works/NattafDYW19.pdf}{NattafDYW19}~\cite{NattafDYW19}, \href{../works/CappartTSR18.pdf}{CappartTSR18}~\cite{CappartTSR18}, \href{../works/AntunesABD18.pdf}{AntunesABD18}~\cite{AntunesABD18}, \href{../works/LaborieRSV18.pdf}{LaborieRSV18}~\cite{LaborieRSV18}, \href{../works/TranPZLDB18.pdf}{TranPZLDB18}~\cite{TranPZLDB18}, \href{../works/GombolayWS18.pdf}{GombolayWS18}~\cite{GombolayWS18}, \href{../works/KreterSSZ18.pdf}{KreterSSZ18}~\cite{KreterSSZ18}, \href{../works/AstrandJZ18.pdf}{AstrandJZ18}~\cite{AstrandJZ18}, \href{../works/KreterSS17.pdf}{KreterSS17}~\cite{KreterSS17}, \href{../works/BridiLBBM16.pdf}{BridiLBBM16}~\cite{BridiLBBM16}... (Total: 61)\\
Concepts & planned maintenance &  & \href{../works/Malapert11.pdf}{Malapert11}~\cite{Malapert11}, \href{../works/Davenport10.pdf}{Davenport10}~\cite{Davenport10} & \href{../works/TouatBT22.pdf}{TouatBT22}~\cite{TouatBT22}, \href{../works/KovacsTKSG21.pdf}{KovacsTKSG21}~\cite{KovacsTKSG21}, \href{../works/Astrand21.pdf}{Astrand21}~\cite{Astrand21}, \href{../works/AntunesABD20.pdf}{AntunesABD20}~\cite{AntunesABD20}, \href{../works/BajestaniB15.pdf}{BajestaniB15}~\cite{BajestaniB15}, \href{../works/AkkerDH07.pdf}{AkkerDH07}~\cite{AkkerDH07}\\
Concepts & precedence & \href{../works/BonninMNE24.pdf}{BonninMNE24}~\cite{BonninMNE24}, \href{../works/abs-2402-00459.pdf}{abs-2402-00459}~\cite{abs-2402-00459}, \href{../works/PovedaAA23.pdf}{PovedaAA23}~\cite{PovedaAA23}, \href{../works/YuraszeckMCCR23.pdf}{YuraszeckMCCR23}~\cite{YuraszeckMCCR23}, \href{../works/MarliereSPR23.pdf}{MarliereSPR23}~\cite{MarliereSPR23}, \href{../works/AlfieriGPS23.pdf}{AlfieriGPS23}~\cite{AlfieriGPS23}, \href{../works/JuvinHHL23.pdf}{JuvinHHL23}~\cite{JuvinHHL23}, \href{../works/NaderiRR23.pdf}{NaderiRR23}~\cite{NaderiRR23}, \href{../works/ZhuSZW23.pdf}{ZhuSZW23}~\cite{ZhuSZW23}, \href{../works/IsikYA23.pdf}{IsikYA23}~\cite{IsikYA23}, \href{../works/FetgoD22.pdf}{FetgoD22}~\cite{FetgoD22}, \href{../works/PohlAK22.pdf}{PohlAK22}~\cite{PohlAK22}, \href{../works/CampeauG22.pdf}{CampeauG22}~\cite{CampeauG22}, \href{../works/YunusogluY22.pdf}{YunusogluY22}~\cite{YunusogluY22}, \href{../works/ZhangBB22.pdf}{ZhangBB22}~\cite{ZhangBB22}, \href{../works/EtminaniesfahaniGNMS22.pdf}{EtminaniesfahaniGNMS22}~\cite{EtminaniesfahaniGNMS22}, \href{../works/NaderiBZ22a.pdf}{NaderiBZ22a}~\cite{NaderiBZ22a}, \href{../works/BoudreaultSLQ22.pdf}{BoudreaultSLQ22}~\cite{BoudreaultSLQ22}, \href{../works/GeibingerMM21.pdf}{GeibingerMM21}~\cite{GeibingerMM21}, \href{../works/HanenKP21.pdf}{HanenKP21}~\cite{HanenKP21}, \href{../works/Astrand0F21.pdf}{Astrand0F21}~\cite{Astrand0F21}, \href{../works/Astrand21.pdf}{Astrand21}~\cite{Astrand21}, \href{../works/HillTV21.pdf}{HillTV21}~\cite{HillTV21}, \href{../works/KoehlerBFFHPSSS21.pdf}{KoehlerBFFHPSSS21}~\cite{KoehlerBFFHPSSS21}, \href{../works/FanXG21.pdf}{FanXG21}~\cite{FanXG21}, \href{../works/HubnerGSV21.pdf}{HubnerGSV21}~\cite{HubnerGSV21}, \href{../works/ZhangYW21.pdf}{ZhangYW21}~\cite{ZhangYW21}, \href{../works/Godet21a.pdf}{Godet21a}~\cite{Godet21a}, \href{../works/HamPK21.pdf}{HamPK21}~\cite{HamPK21}... (Total: 177) & \href{../works/GokPTGO23.pdf}{GokPTGO23}~\cite{GokPTGO23}, \href{../works/KameugneFND23.pdf}{KameugneFND23}~\cite{KameugneFND23}, \href{../works/JuvinHL23a.pdf}{JuvinHL23a}~\cite{JuvinHL23a}, \href{../works/TardivoDFMP23.pdf}{TardivoDFMP23}~\cite{TardivoDFMP23}, \href{../works/Bit-Monnot23.pdf}{Bit-Monnot23}~\cite{Bit-Monnot23}, \href{../works/OujanaAYB22.pdf}{OujanaAYB22}~\cite{OujanaAYB22}, \href{../works/SubulanC22.pdf}{SubulanC22}~\cite{SubulanC22}, \href{../works/ColT22.pdf}{ColT22}~\cite{ColT22}, \href{../works/VlkHT21.pdf}{VlkHT21}~\cite{VlkHT21}, \href{../works/AntuoriHHEN21.pdf}{AntuoriHHEN21}~\cite{AntuoriHHEN21}, \href{../works/Zahout21.pdf}{Zahout21}~\cite{Zahout21}, \href{../works/WessenCS20.pdf}{WessenCS20}~\cite{WessenCS20}, \href{../works/MokhtarzadehTNF20.pdf}{MokhtarzadehTNF20}~\cite{MokhtarzadehTNF20}, \href{../works/GokGSTO20.pdf}{GokGSTO20}~\cite{GokGSTO20}, \href{../works/QinDCS20.pdf}{QinDCS20}~\cite{QinDCS20}, \href{../works/GeibingerMM19.pdf}{GeibingerMM19}~\cite{GeibingerMM19}, \href{../works/Novas19.pdf}{Novas19}~\cite{Novas19}, \href{../works/abs-1911-04766.pdf}{abs-1911-04766}~\cite{abs-1911-04766}, \href{../works/BogaerdtW19.pdf}{BogaerdtW19}~\cite{BogaerdtW19}, \href{../works/MurinR19.pdf}{MurinR19}~\cite{MurinR19}, \href{../works/ColT19.pdf}{ColT19}~\cite{ColT19}, \href{../works/Ham18.pdf}{Ham18}~\cite{Ham18}, \href{../works/KameugneFGOQ18.pdf}{KameugneFGOQ18}~\cite{KameugneFGOQ18}, \href{../works/TanT18.pdf}{TanT18}~\cite{TanT18}, \href{../works/MossigeGSMC17.pdf}{MossigeGSMC17}~\cite{MossigeGSMC17}, \href{../works/Madi-WambaLOBM17.pdf}{Madi-WambaLOBM17}~\cite{Madi-WambaLOBM17}, \href{../works/Madi-WambaB16.pdf}{Madi-WambaB16}~\cite{Madi-WambaB16}, \href{../works/KuB16.pdf}{KuB16}~\cite{KuB16}, \href{../works/AmadiniGM16.pdf}{AmadiniGM16}~\cite{AmadiniGM16}... (Total: 80) & \href{../works/PrataAN23.pdf}{PrataAN23}~\cite{PrataAN23}, \href{../works/JuvinHL23.pdf}{JuvinHL23}~\cite{JuvinHL23}, \href{../works/AfsarVPG23.pdf}{AfsarVPG23}~\cite{AfsarVPG23}, \href{../works/Mehdizadeh-Somarin23.pdf}{Mehdizadeh-Somarin23}~\cite{Mehdizadeh-Somarin23}, \href{../works/abs-2306-05747.pdf}{abs-2306-05747}~\cite{abs-2306-05747}, \href{../works/YuraszeckMC23.pdf}{YuraszeckMC23}~\cite{YuraszeckMC23}, \href{../works/KimCMLLP23.pdf}{KimCMLLP23}~\cite{KimCMLLP23}, \href{../works/TasselGS23.pdf}{TasselGS23}~\cite{TasselGS23}, \href{../works/abs-2305-19888.pdf}{abs-2305-19888}~\cite{abs-2305-19888}, \href{../works/MullerMKP22.pdf}{MullerMKP22}~\cite{MullerMKP22}, \href{../works/JuvinHL22.pdf}{JuvinHL22}~\cite{JuvinHL22}, \href{../works/EmdeZD22.pdf}{EmdeZD22}~\cite{EmdeZD22}, \href{../works/BourreauGGLT22.pdf}{BourreauGGLT22}~\cite{BourreauGGLT22}, \href{../works/ZhangJZL22.pdf}{ZhangJZL22}~\cite{ZhangJZL22}, \href{../works/GeitzGSSW22.pdf}{GeitzGSSW22}~\cite{GeitzGSSW22}, \href{../works/TouatBT22.pdf}{TouatBT22}~\cite{TouatBT22}, \href{../works/WinterMMW22.pdf}{WinterMMW22}~\cite{WinterMMW22}, \href{../works/abs-2211-14492.pdf}{abs-2211-14492}~\cite{abs-2211-14492}, \href{../works/HeinzNVH22.pdf}{HeinzNVH22}~\cite{HeinzNVH22}, \href{../works/Lemos21.pdf}{Lemos21}~\cite{Lemos21}, \href{../works/KovacsTKSG21.pdf}{KovacsTKSG21}~\cite{KovacsTKSG21}, \href{../works/PandeyS21a.pdf}{PandeyS21a}~\cite{PandeyS21a}, \href{../works/AbreuAPNM21.pdf}{AbreuAPNM21}~\cite{AbreuAPNM21}, \href{../works/AntunesABD20.pdf}{AntunesABD20}~\cite{AntunesABD20}, \href{../works/GroleazNS20a.pdf}{GroleazNS20a}~\cite{GroleazNS20a}, \href{../works/TangB20.pdf}{TangB20}~\cite{TangB20}, \href{../works/OuelletQ18.pdf}{OuelletQ18}~\cite{OuelletQ18}, \href{../works/DemirovicS18.pdf}{DemirovicS18}~\cite{DemirovicS18}, \href{../works/BaptisteB18.pdf}{BaptisteB18}~\cite{BaptisteB18}... (Total: 106)\\
Concepts & preempt & \href{../works/BonninMNE24.pdf}{BonninMNE24}~\cite{BonninMNE24}, \href{../works/JuvinHL23a.pdf}{JuvinHL23a}~\cite{JuvinHL23a}, \href{../works/JuvinHHL23.pdf}{JuvinHHL23}~\cite{JuvinHHL23}, \href{../works/PovedaAA23.pdf}{PovedaAA23}~\cite{PovedaAA23}, \href{../works/SubulanC22.pdf}{SubulanC22}~\cite{SubulanC22}, \href{../works/JuvinHL22.pdf}{JuvinHL22}~\cite{JuvinHL22}, \href{../works/Groleaz21.pdf}{Groleaz21}~\cite{Groleaz21}, \href{../works/HanenKP21.pdf}{HanenKP21}~\cite{HanenKP21}, \href{../works/ArtiguesHQT21.pdf}{ArtiguesHQT21}~\cite{ArtiguesHQT21}, \href{../works/Godet21a.pdf}{Godet21a}~\cite{Godet21a}, \href{../works/ZarandiASC20.pdf}{ZarandiASC20}~\cite{ZarandiASC20}, \href{../works/Polo-MejiaALB20.pdf}{Polo-MejiaALB20}~\cite{Polo-MejiaALB20}, \href{../works/NattafHKAL19.pdf}{NattafHKAL19}~\cite{NattafHKAL19}, \href{../works/BaptisteB18.pdf}{BaptisteB18}~\cite{BaptisteB18}, \href{../works/FahimiOQ18.pdf}{FahimiOQ18}~\cite{FahimiOQ18}, \href{../works/GokgurHO18.pdf}{GokgurHO18}~\cite{GokgurHO18}, \href{../works/Dejemeppe16.pdf}{Dejemeppe16}~\cite{Dejemeppe16}, \href{../works/ZarandiKS16.pdf}{ZarandiKS16}~\cite{ZarandiKS16}, \href{../works/Fahimi16.pdf}{Fahimi16}~\cite{Fahimi16}, \href{../works/NattafALR16.pdf}{NattafALR16}~\cite{NattafALR16}, \href{../works/EvenSH15.pdf}{EvenSH15}~\cite{EvenSH15}, \href{../works/EvenSH15a.pdf}{EvenSH15a}~\cite{EvenSH15a}, \href{../works/AlesioNBG14.pdf}{AlesioNBG14}~\cite{AlesioNBG14}, \href{../works/LombardiMB13.pdf}{LombardiMB13}~\cite{LombardiMB13}, \href{../works/MenciaSV12.pdf}{MenciaSV12}~\cite{MenciaSV12}, \href{../works/LombardiM12.pdf}{LombardiM12}~\cite{LombardiM12}, \href{../works/BeldiceanuCDP11.pdf}{BeldiceanuCDP11}~\cite{BeldiceanuCDP11}, \href{../works/KovacsB11.pdf}{KovacsB11}~\cite{KovacsB11}, \href{../works/Schutt11.pdf}{Schutt11}~\cite{Schutt11}... (Total: 41) & \href{../works/PrataAN23.pdf}{PrataAN23}~\cite{PrataAN23}, \href{../works/Adelgren2023.pdf}{Adelgren2023}~\cite{Adelgren2023}, \href{../works/abs-2305-19888.pdf}{abs-2305-19888}~\cite{abs-2305-19888}, \href{../works/AbreuPNF23.pdf}{AbreuPNF23}~\cite{AbreuPNF23}, \href{../works/FetgoD22.pdf}{FetgoD22}~\cite{FetgoD22}, \href{../works/HeinzNVH22.pdf}{HeinzNVH22}~\cite{HeinzNVH22}, \href{../works/OuelletQ22.pdf}{OuelletQ22}~\cite{OuelletQ22}, \href{../works/Astrand21.pdf}{Astrand21}~\cite{Astrand21}, \href{../works/Zahout21.pdf}{Zahout21}~\cite{Zahout21}, \href{../works/SacramentoSP20.pdf}{SacramentoSP20}~\cite{SacramentoSP20}, \href{../works/Mercier-AubinGQ20.pdf}{Mercier-AubinGQ20}~\cite{Mercier-AubinGQ20}, \href{../works/Lunardi20.pdf}{Lunardi20}~\cite{Lunardi20}, \href{../works/LunardiBLRV20.pdf}{LunardiBLRV20}~\cite{LunardiBLRV20}, \href{../works/Caballero19.pdf}{Caballero19}~\cite{Caballero19}, \href{../works/ArkhipovBL19.pdf}{ArkhipovBL19}~\cite{ArkhipovBL19}, \href{../works/GombolayWS18.pdf}{GombolayWS18}~\cite{GombolayWS18}, \href{../works/YoungFS17.pdf}{YoungFS17}~\cite{YoungFS17}, \href{../works/OrnekO16.pdf}{OrnekO16}~\cite{OrnekO16}, \href{../works/OzturkTHO15.pdf}{OzturkTHO15}~\cite{OzturkTHO15}, \href{../works/SchnellH15.pdf}{SchnellH15}~\cite{SchnellH15}, \href{../works/NattafAL15.pdf}{NattafAL15}~\cite{NattafAL15}, \href{../works/SimoninAHL15.pdf}{SimoninAHL15}~\cite{SimoninAHL15}, \href{../works/TerekhovTDB14.pdf}{TerekhovTDB14}~\cite{TerekhovTDB14}, \href{../works/OzturkTHO13.pdf}{OzturkTHO13}~\cite{OzturkTHO13}, \href{../works/MenciaSV13.pdf}{MenciaSV13}~\cite{MenciaSV13}, \href{../works/BajestaniB13.pdf}{BajestaniB13}~\cite{BajestaniB13}, \href{../works/OzturkTHO12.pdf}{OzturkTHO12}~\cite{OzturkTHO12}, \href{../works/SimoninAHL12.pdf}{SimoninAHL12}~\cite{SimoninAHL12}, \href{../works/GuyonLPR12.pdf}{GuyonLPR12}~\cite{GuyonLPR12}... (Total: 42) & \href{../works/Mehdizadeh-Somarin23.pdf}{Mehdizadeh-Somarin23}~\cite{Mehdizadeh-Somarin23}, \href{../works/AalianPG23.pdf}{AalianPG23}~\cite{AalianPG23}, \href{../works/KameugneFND23.pdf}{KameugneFND23}~\cite{KameugneFND23}, \href{../works/abs-2306-05747.pdf}{abs-2306-05747}~\cite{abs-2306-05747}, \href{../works/PenzDN23.pdf}{PenzDN23}~\cite{PenzDN23}, \href{../works/NaderiRR23.pdf}{NaderiRR23}~\cite{NaderiRR23}, \href{../works/TasselGS23.pdf}{TasselGS23}~\cite{TasselGS23}, \href{../works/TardivoDFMP23.pdf}{TardivoDFMP23}~\cite{TardivoDFMP23}, \href{../works/YuraszeckMC23.pdf}{YuraszeckMC23}~\cite{YuraszeckMC23}, \href{../works/YuraszeckMCCR23.pdf}{YuraszeckMCCR23}~\cite{YuraszeckMCCR23}, \href{../works/AkramNHRSA23.pdf}{AkramNHRSA23}~\cite{AkramNHRSA23}, \href{../works/AbreuNP23.pdf}{AbreuNP23}~\cite{AbreuNP23}, \href{../works/ZhuSZW23.pdf}{ZhuSZW23}~\cite{ZhuSZW23}, \href{../works/IsikYA23.pdf}{IsikYA23}~\cite{IsikYA23}, \href{../works/AfsarVPG23.pdf}{AfsarVPG23}~\cite{AfsarVPG23}, \href{../works/ZhangBB22.pdf}{ZhangBB22}~\cite{ZhangBB22}, \href{../works/Teppan22.pdf}{Teppan22}~\cite{Teppan22}, \href{../works/EtminaniesfahaniGNMS22.pdf}{EtminaniesfahaniGNMS22}~\cite{EtminaniesfahaniGNMS22}, \href{../works/ColT22.pdf}{ColT22}~\cite{ColT22}, \href{../works/MullerMKP22.pdf}{MullerMKP22}~\cite{MullerMKP22}, \href{../works/YunusogluY22.pdf}{YunusogluY22}~\cite{YunusogluY22}, \href{../works/JungblutK22.pdf}{JungblutK22}~\cite{JungblutK22}, \href{../works/AbreuN22.pdf}{AbreuN22}~\cite{AbreuN22}, \href{../works/NaderiBZ22a.pdf}{NaderiBZ22a}~\cite{NaderiBZ22a}, \href{../works/TouatBT22.pdf}{TouatBT22}~\cite{TouatBT22}, \href{../works/GeitzGSSW22.pdf}{GeitzGSSW22}~\cite{GeitzGSSW22}, \href{../works/BoudreaultSLQ22.pdf}{BoudreaultSLQ22}~\cite{BoudreaultSLQ22}, \href{../works/OujanaAYB22.pdf}{OujanaAYB22}~\cite{OujanaAYB22}, \href{../works/Bedhief21.pdf}{Bedhief21}~\cite{Bedhief21}... (Total: 153)\\
Concepts & preemptive & \href{../works/BonninMNE24.pdf}{BonninMNE24}~\cite{BonninMNE24}, \href{../works/JuvinHL23a.pdf}{JuvinHL23a}~\cite{JuvinHL23a}, \href{../works/JuvinHHL23.pdf}{JuvinHHL23}~\cite{JuvinHHL23}, \href{../works/PovedaAA23.pdf}{PovedaAA23}~\cite{PovedaAA23}, \href{../works/JuvinHL22.pdf}{JuvinHL22}~\cite{JuvinHL22}, \href{../works/ArtiguesHQT21.pdf}{ArtiguesHQT21}~\cite{ArtiguesHQT21}, \href{../works/HanenKP21.pdf}{HanenKP21}~\cite{HanenKP21}, \href{../works/Godet21a.pdf}{Godet21a}~\cite{Godet21a}, \href{../works/ZarandiASC20.pdf}{ZarandiASC20}~\cite{ZarandiASC20}, \href{../works/Polo-MejiaALB20.pdf}{Polo-MejiaALB20}~\cite{Polo-MejiaALB20}, \href{../works/NattafHKAL19.pdf}{NattafHKAL19}~\cite{NattafHKAL19}, \href{../works/GokgurHO18.pdf}{GokgurHO18}~\cite{GokgurHO18}, \href{../works/BaptisteB18.pdf}{BaptisteB18}~\cite{BaptisteB18}, \href{../works/Fahimi16.pdf}{Fahimi16}~\cite{Fahimi16}, \href{../works/Dejemeppe16.pdf}{Dejemeppe16}~\cite{Dejemeppe16}, \href{../works/EvenSH15.pdf}{EvenSH15}~\cite{EvenSH15}, \href{../works/EvenSH15a.pdf}{EvenSH15a}~\cite{EvenSH15a}, \href{../works/AlesioNBG14.pdf}{AlesioNBG14}~\cite{AlesioNBG14}, \href{../works/LombardiMB13.pdf}{LombardiMB13}~\cite{LombardiMB13}, \href{../works/LombardiM12.pdf}{LombardiM12}~\cite{LombardiM12}, \href{../works/MenciaSV12.pdf}{MenciaSV12}~\cite{MenciaSV12}, \href{../works/Schutt11.pdf}{Schutt11}~\cite{Schutt11}, \href{../works/KovacsB11.pdf}{KovacsB11}~\cite{KovacsB11}, \href{../works/BeldiceanuCDP11.pdf}{BeldiceanuCDP11}~\cite{BeldiceanuCDP11}, \href{../works/Lombardi10.pdf}{Lombardi10}~\cite{Lombardi10}, \href{../works/BartakSR10.pdf}{BartakSR10}~\cite{BartakSR10}, \href{../works/MonetteDD07.pdf}{MonetteDD07}~\cite{MonetteDD07}, \href{../works/KovacsB07.pdf}{KovacsB07}~\cite{KovacsB07}, \href{../works/Wolf05.pdf}{Wolf05}~\cite{Wolf05}... (Total: 36) & \href{../works/PrataAN23.pdf}{PrataAN23}~\cite{PrataAN23}, \href{../works/AbreuPNF23.pdf}{AbreuPNF23}~\cite{AbreuPNF23}, \href{../works/Adelgren2023.pdf}{Adelgren2023}~\cite{Adelgren2023}, \href{../works/Groleaz21.pdf}{Groleaz21}~\cite{Groleaz21}, \href{../works/Mercier-AubinGQ20.pdf}{Mercier-AubinGQ20}~\cite{Mercier-AubinGQ20}, \href{../works/SacramentoSP20.pdf}{SacramentoSP20}~\cite{SacramentoSP20}, \href{../works/ArkhipovBL19.pdf}{ArkhipovBL19}~\cite{ArkhipovBL19}, \href{../works/Caballero19.pdf}{Caballero19}~\cite{Caballero19}, \href{../works/FahimiOQ18.pdf}{FahimiOQ18}~\cite{FahimiOQ18}, \href{../works/YoungFS17.pdf}{YoungFS17}~\cite{YoungFS17}, \href{../works/NattafALR16.pdf}{NattafALR16}~\cite{NattafALR16}, \href{../works/ZarandiKS16.pdf}{ZarandiKS16}~\cite{ZarandiKS16}, \href{../works/OrnekO16.pdf}{OrnekO16}~\cite{OrnekO16}, \href{../works/NattafAL15.pdf}{NattafAL15}~\cite{NattafAL15}, \href{../works/OzturkTHO15.pdf}{OzturkTHO15}~\cite{OzturkTHO15}, \href{../works/BajestaniB13.pdf}{BajestaniB13}~\cite{BajestaniB13}, \href{../works/MenciaSV13.pdf}{MenciaSV13}~\cite{MenciaSV13}, \href{../works/OzturkTHO13.pdf}{OzturkTHO13}~\cite{OzturkTHO13}, \href{../works/OzturkTHO12.pdf}{OzturkTHO12}~\cite{OzturkTHO12}, \href{../works/Malapert11.pdf}{Malapert11}~\cite{Malapert11}, \href{../works/SchuttFSW11.pdf}{SchuttFSW11}~\cite{SchuttFSW11}, \href{../works/LombardiMRB10.pdf}{LombardiMRB10}~\cite{LombardiMRB10}, \href{../works/ChenGPSH10.pdf}{ChenGPSH10}~\cite{ChenGPSH10}, \href{../works/Wolf09.pdf}{Wolf09}~\cite{Wolf09}, \href{../works/Laborie09.pdf}{Laborie09}~\cite{Laborie09}, \href{../works/SchuttFSW09.pdf}{SchuttFSW09}~\cite{SchuttFSW09}, \href{../works/KovacsB08.pdf}{KovacsB08}~\cite{KovacsB08}, \href{../works/ArtiouchineB05.pdf}{ArtiouchineB05}~\cite{ArtiouchineB05}, \href{../works/SourdN00.pdf}{SourdN00}~\cite{SourdN00}... (Total: 31) & \href{../works/Mehdizadeh-Somarin23.pdf}{Mehdizadeh-Somarin23}~\cite{Mehdizadeh-Somarin23}, \href{../works/AalianPG23.pdf}{AalianPG23}~\cite{AalianPG23}, \href{../works/abs-2305-19888.pdf}{abs-2305-19888}~\cite{abs-2305-19888}, \href{../works/PenzDN23.pdf}{PenzDN23}~\cite{PenzDN23}, \href{../works/YuraszeckMC23.pdf}{YuraszeckMC23}~\cite{YuraszeckMC23}, \href{../works/NaderiRR23.pdf}{NaderiRR23}~\cite{NaderiRR23}, \href{../works/ColT22.pdf}{ColT22}~\cite{ColT22}, \href{../works/HeinzNVH22.pdf}{HeinzNVH22}~\cite{HeinzNVH22}, \href{../works/MullerMKP22.pdf}{MullerMKP22}~\cite{MullerMKP22}, \href{../works/GeitzGSSW22.pdf}{GeitzGSSW22}~\cite{GeitzGSSW22}, \href{../works/AbreuN22.pdf}{AbreuN22}~\cite{AbreuN22}, \href{../works/SubulanC22.pdf}{SubulanC22}~\cite{SubulanC22}, \href{../works/EtminaniesfahaniGNMS22.pdf}{EtminaniesfahaniGNMS22}~\cite{EtminaniesfahaniGNMS22}, \href{../works/NaderiBZ22a.pdf}{NaderiBZ22a}~\cite{NaderiBZ22a}, \href{../works/AbreuAPNM21.pdf}{AbreuAPNM21}~\cite{AbreuAPNM21}, \href{../works/ArmstrongGOS21.pdf}{ArmstrongGOS21}~\cite{ArmstrongGOS21}, \href{../works/QinWSLS21.pdf}{QinWSLS21}~\cite{QinWSLS21}, \href{../works/ZhangYW21.pdf}{ZhangYW21}~\cite{ZhangYW21}, \href{../works/HillTV21.pdf}{HillTV21}~\cite{HillTV21}, \href{../works/HubnerGSV21.pdf}{HubnerGSV21}~\cite{HubnerGSV21}, \href{../works/Zahout21.pdf}{Zahout21}~\cite{Zahout21}, \href{../works/KovacsTKSG21.pdf}{KovacsTKSG21}~\cite{KovacsTKSG21}, \href{../works/BenderWS21.pdf}{BenderWS21}~\cite{BenderWS21}, \href{../works/GroleazNS20.pdf}{GroleazNS20}~\cite{GroleazNS20}, \href{../works/BenediktMH20.pdf}{BenediktMH20}~\cite{BenediktMH20}, \href{../works/MejiaY20.pdf}{MejiaY20}~\cite{MejiaY20}, \href{../works/CauwelaertDS20.pdf}{CauwelaertDS20}~\cite{CauwelaertDS20}, \href{../works/GroleazNS20a.pdf}{GroleazNS20a}~\cite{GroleazNS20a}, \href{../works/YangSS19.pdf}{YangSS19}~\cite{YangSS19}... (Total: 121)\\
Concepts & producer/consumer & \href{../works/SchuttS16.pdf}{SchuttS16}~\cite{SchuttS16}, \href{../works/PoderBS04.pdf}{PoderBS04}~\cite{PoderBS04}, \href{../works/Kumar03.pdf}{Kumar03}~\cite{Kumar03}, \href{../works/Beck99.pdf}{Beck99}~\cite{Beck99}, \href{../works/SimonisC95.pdf}{SimonisC95}~\cite{SimonisC95} & \href{../works/HermenierDL11.pdf}{HermenierDL11}~\cite{HermenierDL11}, \href{../works/BeldiceanuC02.pdf}{BeldiceanuC02}~\cite{BeldiceanuC02}, \href{../works/Simonis99.pdf}{Simonis99}~\cite{Simonis99}, \href{../works/Simonis95a.pdf}{Simonis95a}~\cite{Simonis95a} & \href{../works/GeitzGSSW22.pdf}{GeitzGSSW22}~\cite{GeitzGSSW22}, \href{../works/KlankeBYE21.pdf}{KlankeBYE21}~\cite{KlankeBYE21}, \href{../works/CappartTSR18.pdf}{CappartTSR18}~\cite{CappartTSR18}, \href{../works/BlomPS16.pdf}{BlomPS16}~\cite{BlomPS16}, \href{../works/LombardiM12a.pdf}{LombardiM12a}~\cite{LombardiM12a}, \href{../works/Wolf11.pdf}{Wolf11}~\cite{Wolf11}, \href{../works/SimonisH11.pdf}{SimonisH11}~\cite{SimonisH11}, \href{../works/LombardiMRB10.pdf}{LombardiMRB10}~\cite{LombardiMRB10}, \href{../works/ChenGPSH10.pdf}{ChenGPSH10}~\cite{ChenGPSH10}, \href{../works/PoderB08.pdf}{PoderB08}~\cite{PoderB08}, \href{../works/Simonis07.pdf}{Simonis07}~\cite{Simonis07}, \href{../works/Timpe02.pdf}{Timpe02}~\cite{Timpe02}, \href{../works/SimonisCK00.pdf}{SimonisCK00}~\cite{SimonisCK00}, \href{../works/Simonis95.pdf}{Simonis95}~\cite{Simonis95}\\
Concepts & re-scheduling & \href{../works/Fatemi-AnarakiTFV23.pdf}{Fatemi-AnarakiTFV23}~\cite{Fatemi-AnarakiTFV23}, \href{../works/MarliereSPR23.pdf}{MarliereSPR23}~\cite{MarliereSPR23}, \href{../works/Astrand21.pdf}{Astrand21}~\cite{Astrand21}, \href{../works/Lemos21.pdf}{Lemos21}~\cite{Lemos21}, \href{../works/HamPK21.pdf}{HamPK21}~\cite{HamPK21}, \href{../works/Groleaz21.pdf}{Groleaz21}~\cite{Groleaz21}, \href{../works/BarzegaranZP20.pdf}{BarzegaranZP20}~\cite{BarzegaranZP20}, \href{../works/ZarandiASC20.pdf}{ZarandiASC20}~\cite{ZarandiASC20}, \href{../works/ZhangW18.pdf}{ZhangW18}~\cite{ZhangW18}, \href{../works/CappartS17.pdf}{CappartS17}~\cite{CappartS17}, \href{../works/Madi-WambaLOBM17.pdf}{Madi-WambaLOBM17}~\cite{Madi-WambaLOBM17}, \href{../works/Froger16.pdf}{Froger16}~\cite{Froger16}, \href{../works/BartakV15.pdf}{BartakV15}~\cite{BartakV15}, \href{../works/HarjunkoskiMBC14.pdf}{HarjunkoskiMBC14}~\cite{HarjunkoskiMBC14}, \href{../works/GrimesIOS14.pdf}{GrimesIOS14}~\cite{GrimesIOS14}, \href{../works/BajestaniB13.pdf}{BajestaniB13}~\cite{BajestaniB13}, \href{../works/TranTDB13.pdf}{TranTDB13}~\cite{TranTDB13}, \href{../works/RendlPHPR12.pdf}{RendlPHPR12}~\cite{RendlPHPR12}, \href{../works/LombardiM12.pdf}{LombardiM12}~\cite{LombardiM12}, \href{../works/IfrimOS12.pdf}{IfrimOS12}~\cite{IfrimOS12}, \href{../works/NovasH10.pdf}{NovasH10}~\cite{NovasH10}, \href{../works/BidotVLB09.pdf}{BidotVLB09}~\cite{BidotVLB09}, \href{../works/Laborie03.pdf}{Laborie03}~\cite{Laborie03}, \href{../works/Baptiste02.pdf}{Baptiste02}~\cite{Baptiste02}, \href{../works/MartinPY01.pdf}{MartinPY01}~\cite{MartinPY01}, \href{../works/ArtiguesR00.pdf}{ArtiguesR00}~\cite{ArtiguesR00} & \href{../works/Mehdizadeh-Somarin23.pdf}{Mehdizadeh-Somarin23}~\cite{Mehdizadeh-Somarin23}, \href{../works/NaderiBZ22a.pdf}{NaderiBZ22a}~\cite{NaderiBZ22a}, \href{../works/Zahout21.pdf}{Zahout21}~\cite{Zahout21}, \href{../works/KovacsTKSG21.pdf}{KovacsTKSG21}~\cite{KovacsTKSG21}, \href{../works/AstrandJZ20.pdf}{AstrandJZ20}~\cite{AstrandJZ20}, \href{../works/AntunesABD20.pdf}{AntunesABD20}~\cite{AntunesABD20}, \href{../works/RoshanaeiBAUB20.pdf}{RoshanaeiBAUB20}~\cite{RoshanaeiBAUB20}, \href{../works/GombolayWS18.pdf}{GombolayWS18}~\cite{GombolayWS18}, \href{../works/TranPZLDB18.pdf}{TranPZLDB18}~\cite{TranPZLDB18}, \href{../works/HoYCLLCLC18.pdf}{HoYCLLCLC18}~\cite{HoYCLLCLC18}, \href{../works/AntunesABD18.pdf}{AntunesABD18}~\cite{AntunesABD18}, \href{../works/HurleyOS16.pdf}{HurleyOS16}~\cite{HurleyOS16}, \href{../works/LimHTB16.pdf}{LimHTB16}~\cite{LimHTB16}, \href{../works/LimBTBB15.pdf}{LimBTBB15}~\cite{LimBTBB15}, \href{../works/CobanH11.pdf}{CobanH11}~\cite{CobanH11}, \href{../works/Lombardi10.pdf}{Lombardi10}~\cite{Lombardi10}, \href{../works/CobanH10.pdf}{CobanH10}~\cite{CobanH10}, \href{../works/Acuna-AgostMFG09.pdf}{Acuna-AgostMFG09}~\cite{Acuna-AgostMFG09}, \href{../works/Elkhyari03.pdf}{Elkhyari03}~\cite{Elkhyari03}, \href{../works/Beck99.pdf}{Beck99}~\cite{Beck99} & \href{../works/PrataAN23.pdf}{PrataAN23}~\cite{PrataAN23}, \href{../works/ForbesHJST24.pdf}{ForbesHJST24}~\cite{ForbesHJST24}, \href{../works/abs-2306-05747.pdf}{abs-2306-05747}~\cite{abs-2306-05747}, \href{../works/abs-2305-19888.pdf}{abs-2305-19888}~\cite{abs-2305-19888}, \href{../works/ShaikhK23.pdf}{ShaikhK23}~\cite{ShaikhK23}, \href{../works/GurPAE23.pdf}{GurPAE23}~\cite{GurPAE23}, \href{../works/NaderiRR23.pdf}{NaderiRR23}~\cite{NaderiRR23}, \href{../works/PerezGSL23.pdf}{PerezGSL23}~\cite{PerezGSL23}, \href{../works/abs-2312-13682.pdf}{abs-2312-13682}~\cite{abs-2312-13682}, \href{../works/GokPTGO23.pdf}{GokPTGO23}~\cite{GokPTGO23}, \href{../works/EfthymiouY23.pdf}{EfthymiouY23}~\cite{EfthymiouY23}, \href{../works/Adelgren2023.pdf}{Adelgren2023}~\cite{Adelgren2023}, \href{../works/TasselGS23.pdf}{TasselGS23}~\cite{TasselGS23}, \href{../works/JuvinHL23a.pdf}{JuvinHL23a}~\cite{JuvinHL23a}, \href{../works/ZhuSZW23.pdf}{ZhuSZW23}~\cite{ZhuSZW23}, \href{../works/BourreauGGLT22.pdf}{BourreauGGLT22}~\cite{BourreauGGLT22}, \href{../works/HeinzNVH22.pdf}{HeinzNVH22}~\cite{HeinzNVH22}, \href{../works/ArmstrongGOS22.pdf}{ArmstrongGOS22}~\cite{ArmstrongGOS22}, \href{../works/LuoB22.pdf}{LuoB22}~\cite{LuoB22}, \href{../works/PohlAK22.pdf}{PohlAK22}~\cite{PohlAK22}, \href{../works/FarsiTM22.pdf}{FarsiTM22}~\cite{FarsiTM22}, \href{../works/YunusogluY22.pdf}{YunusogluY22}~\cite{YunusogluY22}, \href{../works/JuvinHL22.pdf}{JuvinHL22}~\cite{JuvinHL22}, \href{../works/YuraszeckMPV22.pdf}{YuraszeckMPV22}~\cite{YuraszeckMPV22}, \href{../works/ZhangYW21.pdf}{ZhangYW21}~\cite{ZhangYW21}, \href{../works/KlankeBYE21.pdf}{KlankeBYE21}~\cite{KlankeBYE21}, \href{../works/PandeyS21a.pdf}{PandeyS21a}~\cite{PandeyS21a}, \href{../works/BenediktMH20.pdf}{BenediktMH20}~\cite{BenediktMH20}, \href{../works/MejiaY20.pdf}{MejiaY20}~\cite{MejiaY20}... (Total: 91)\\
Concepts & reactive scheduling & \href{../works/NovasH10.pdf}{NovasH10}~\cite{NovasH10} & \href{../works/Groleaz21.pdf}{Groleaz21}~\cite{Groleaz21}, \href{../works/ZarandiASC20.pdf}{ZarandiASC20}~\cite{ZarandiASC20}, \href{../works/BartakV15.pdf}{BartakV15}~\cite{BartakV15}, \href{../works/HarjunkoskiMBC14.pdf}{HarjunkoskiMBC14}~\cite{HarjunkoskiMBC14} & \href{../works/Mehdizadeh-Somarin23.pdf}{Mehdizadeh-Somarin23}~\cite{Mehdizadeh-Somarin23}, \href{../works/FanXG21.pdf}{FanXG21}~\cite{FanXG21}, \href{../works/HubnerGSV21.pdf}{HubnerGSV21}~\cite{HubnerGSV21}, \href{../works/Lunardi20.pdf}{Lunardi20}~\cite{Lunardi20}, \href{../works/EscobetPQPRA19.pdf}{EscobetPQPRA19}~\cite{EscobetPQPRA19}, \href{../works/Fahimi16.pdf}{Fahimi16}~\cite{Fahimi16}, \href{../works/Froger16.pdf}{Froger16}~\cite{Froger16}, \href{../works/NovasH14.pdf}{NovasH14}~\cite{NovasH14}, \href{../works/BonfiettiLM14.pdf}{BonfiettiLM14}~\cite{BonfiettiLM14}, \href{../works/BajestaniB13.pdf}{BajestaniB13}~\cite{BajestaniB13}, \href{../works/NovasH12.pdf}{NovasH12}~\cite{NovasH12}, \href{../works/LombardiM12.pdf}{LombardiM12}~\cite{LombardiM12}, \href{../works/BillautHL12.pdf}{BillautHL12}~\cite{BillautHL12}, \href{../works/LopesCSM10.pdf}{LopesCSM10}~\cite{LopesCSM10}, \href{../works/BidotVLB09.pdf}{BidotVLB09}~\cite{BidotVLB09}, \href{../works/MouraSCL08a.pdf}{MouraSCL08a}~\cite{MouraSCL08a}, \href{../works/BeckW07.pdf}{BeckW07}~\cite{BeckW07}, \href{../works/Elkhyari03.pdf}{Elkhyari03}~\cite{Elkhyari03}, \href{../works/Baptiste02.pdf}{Baptiste02}~\cite{Baptiste02}, \href{../works/BeckF00.pdf}{BeckF00}~\cite{BeckF00}, \href{../works/SakkoutW00.pdf}{SakkoutW00}~\cite{SakkoutW00}, \href{../works/PapaB98.pdf}{PapaB98}~\cite{PapaB98}, \href{../works/NuijtenP98.pdf}{NuijtenP98}~\cite{NuijtenP98}, \href{../works/Wallace96.pdf}{Wallace96}~\cite{Wallace96}\\
Concepts & release-date & \href{../works/BonninMNE24.pdf}{BonninMNE24}~\cite{BonninMNE24}, \href{../works/YunusogluY22.pdf}{YunusogluY22}~\cite{YunusogluY22}, \href{../works/JuvinHL22.pdf}{JuvinHL22}~\cite{JuvinHL22}, \href{../works/YuraszeckMPV22.pdf}{YuraszeckMPV22}~\cite{YuraszeckMPV22}, \href{../works/WinterMMW22.pdf}{WinterMMW22}~\cite{WinterMMW22}, \href{../works/EmdeZD22.pdf}{EmdeZD22}~\cite{EmdeZD22}, \href{../works/Groleaz21.pdf}{Groleaz21}~\cite{Groleaz21}, \href{../works/HanenKP21.pdf}{HanenKP21}~\cite{HanenKP21}, \href{../works/Bedhief21.pdf}{Bedhief21}~\cite{Bedhief21}, \href{../works/Polo-MejiaALB20.pdf}{Polo-MejiaALB20}~\cite{Polo-MejiaALB20}, \href{../works/EscobetPQPRA19.pdf}{EscobetPQPRA19}~\cite{EscobetPQPRA19}, \href{../works/Tesch18.pdf}{Tesch18}~\cite{Tesch18}, \href{../works/KameugneFSN14.pdf}{KameugneFSN14}~\cite{KameugneFSN14}, \href{../works/LimtanyakulS12.pdf}{LimtanyakulS12}~\cite{LimtanyakulS12}, \href{../works/SerraNM12.pdf}{SerraNM12}~\cite{SerraNM12}, \href{../works/TerekhovDOB12.pdf}{TerekhovDOB12}~\cite{TerekhovDOB12}, \href{../works/KameugneFSN11.pdf}{KameugneFSN11}~\cite{KameugneFSN11}, \href{../works/KovacsB11.pdf}{KovacsB11}~\cite{KovacsB11}, \href{../works/Lombardi10.pdf}{Lombardi10}~\cite{Lombardi10}, \href{../works/BartakSR10.pdf}{BartakSR10}~\cite{BartakSR10}, \href{../works/LombardiM10a.pdf}{LombardiM10a}~\cite{LombardiM10a}, \href{../works/abs-0907-0939.pdf}{abs-0907-0939}~\cite{abs-0907-0939}, \href{../works/MercierH08.pdf}{MercierH08}~\cite{MercierH08}, \href{../works/KovacsB07.pdf}{KovacsB07}~\cite{KovacsB07}, \href{../works/Hooker07.pdf}{Hooker07}~\cite{Hooker07}, \href{../works/AkkerDH07.pdf}{AkkerDH07}~\cite{AkkerDH07}, \href{../works/SadykovW06.pdf}{SadykovW06}~\cite{SadykovW06}, \href{../works/ArtiouchineB05.pdf}{ArtiouchineB05}~\cite{ArtiouchineB05}, \href{../works/Hooker05.pdf}{Hooker05}~\cite{Hooker05}... (Total: 36) & \href{../works/PrataAN23.pdf}{PrataAN23}~\cite{PrataAN23}, \href{../works/LacknerMMWW23.pdf}{LacknerMMWW23}~\cite{LacknerMMWW23}, \href{../works/JuvinHL23a.pdf}{JuvinHL23a}~\cite{JuvinHL23a}, \href{../works/LacknerMMWW21.pdf}{LacknerMMWW21}~\cite{LacknerMMWW21}, \href{../works/Godet21a.pdf}{Godet21a}~\cite{Godet21a}, \href{../works/ArtiguesHQT21.pdf}{ArtiguesHQT21}~\cite{ArtiguesHQT21}, \href{../works/GroleazNS20.pdf}{GroleazNS20}~\cite{GroleazNS20}, \href{../works/GroleazNS20a.pdf}{GroleazNS20a}~\cite{GroleazNS20a}, \href{../works/AntuoriHHEN20.pdf}{AntuoriHHEN20}~\cite{AntuoriHHEN20}, \href{../works/ZarandiASC20.pdf}{ZarandiASC20}~\cite{ZarandiASC20}, \href{../works/GeibingerMM19.pdf}{GeibingerMM19}~\cite{GeibingerMM19}, \href{../works/ArkhipovBL19.pdf}{ArkhipovBL19}~\cite{ArkhipovBL19}, \href{../works/abs-1911-04766.pdf}{abs-1911-04766}~\cite{abs-1911-04766}, \href{../works/Dejemeppe16.pdf}{Dejemeppe16}~\cite{Dejemeppe16}, \href{../works/HeinzSB13.pdf}{HeinzSB13}~\cite{HeinzSB13}, \href{../works/KelbelH11.pdf}{KelbelH11}~\cite{KelbelH11}, \href{../works/MilanoW09.pdf}{MilanoW09}~\cite{MilanoW09}, \href{../works/Laborie09.pdf}{Laborie09}~\cite{Laborie09}, \href{../works/Limtanyakul07.pdf}{Limtanyakul07}~\cite{Limtanyakul07}, \href{../works/Simonis07.pdf}{Simonis07}~\cite{Simonis07}, \href{../works/MilanoW06.pdf}{MilanoW06}~\cite{MilanoW06}, \href{../works/Hooker06.pdf}{Hooker06}~\cite{Hooker06}, \href{../works/Hooker05a.pdf}{Hooker05a}~\cite{Hooker05a}, \href{../works/WuBB05.pdf}{WuBB05}~\cite{WuBB05}, \href{../works/Sadykov04.pdf}{Sadykov04}~\cite{Sadykov04}, \href{../works/HarjunkoskiG02.pdf}{HarjunkoskiG02}~\cite{HarjunkoskiG02}, \href{../works/JainG01.pdf}{JainG01}~\cite{JainG01}, \href{../works/TorresL00.pdf}{TorresL00}~\cite{TorresL00}, \href{../works/SourdN00.pdf}{SourdN00}~\cite{SourdN00}... (Total: 31) & \href{../works/ForbesHJST24.pdf}{ForbesHJST24}~\cite{ForbesHJST24}, \href{../works/PovedaAA23.pdf}{PovedaAA23}~\cite{PovedaAA23}, \href{../works/PenzDN23.pdf}{PenzDN23}~\cite{PenzDN23}, \href{../works/IsikYA23.pdf}{IsikYA23}~\cite{IsikYA23}, \href{../works/Adelgren2023.pdf}{Adelgren2023}~\cite{Adelgren2023}, \href{../works/YuraszeckMC23.pdf}{YuraszeckMC23}~\cite{YuraszeckMC23}, \href{../works/PohlAK22.pdf}{PohlAK22}~\cite{PohlAK22}, \href{../works/TouatBT22.pdf}{TouatBT22}~\cite{TouatBT22}, \href{../works/GeibingerMM21.pdf}{GeibingerMM21}~\cite{GeibingerMM21}, \href{../works/HillTV21.pdf}{HillTV21}~\cite{HillTV21}, \href{../works/AbreuAPNM21.pdf}{AbreuAPNM21}~\cite{AbreuAPNM21}, \href{../works/Zahout21.pdf}{Zahout21}~\cite{Zahout21}, \href{../works/Astrand21.pdf}{Astrand21}~\cite{Astrand21}, \href{../works/AntuoriHHEN21.pdf}{AntuoriHHEN21}~\cite{AntuoriHHEN21}, \href{../works/ZhangYW21.pdf}{ZhangYW21}~\cite{ZhangYW21}, \href{../works/KovacsTKSG21.pdf}{KovacsTKSG21}~\cite{KovacsTKSG21}, \href{../works/GodetLHS20.pdf}{GodetLHS20}~\cite{GodetLHS20}, \href{../works/Lunardi20.pdf}{Lunardi20}~\cite{Lunardi20}, \href{../works/MejiaY20.pdf}{MejiaY20}~\cite{MejiaY20}, \href{../works/Hooker19.pdf}{Hooker19}~\cite{Hooker19}, \href{../works/Novas19.pdf}{Novas19}~\cite{Novas19}, \href{../works/Caballero19.pdf}{Caballero19}~\cite{Caballero19}, \href{../works/NattafHKAL19.pdf}{NattafHKAL19}~\cite{NattafHKAL19}, \href{../works/abs-1902-09244.pdf}{abs-1902-09244}~\cite{abs-1902-09244}, \href{../works/LaborieRSV18.pdf}{LaborieRSV18}~\cite{LaborieRSV18}, \href{../works/TanT18.pdf}{TanT18}~\cite{TanT18}, \href{../works/KreterSSZ18.pdf}{KreterSSZ18}~\cite{KreterSSZ18}, \href{../works/Laborie18a.pdf}{Laborie18a}~\cite{Laborie18a}, \href{../works/GokgurHO18.pdf}{GokgurHO18}~\cite{GokgurHO18}... (Total: 87)\\
Concepts & resource & \href{../works/ForbesHJST24.pdf}{ForbesHJST24}~\cite{ForbesHJST24}, \href{../works/BonninMNE24.pdf}{BonninMNE24}~\cite{BonninMNE24}, \href{../works/PrataAN23.pdf}{PrataAN23}~\cite{PrataAN23}, \href{../works/abs-2402-00459.pdf}{abs-2402-00459}~\cite{abs-2402-00459}, \href{../works/Fatemi-AnarakiTFV23.pdf}{Fatemi-AnarakiTFV23}~\cite{Fatemi-AnarakiTFV23}, \href{../works/JuvinHHL23.pdf}{JuvinHHL23}~\cite{JuvinHHL23}, \href{../works/PovedaAA23.pdf}{PovedaAA23}~\cite{PovedaAA23}, \href{../works/ShaikhK23.pdf}{ShaikhK23}~\cite{ShaikhK23}, \href{../works/GuoZ23.pdf}{GuoZ23}~\cite{GuoZ23}, \href{../works/NaderiRR23.pdf}{NaderiRR23}~\cite{NaderiRR23}, \href{../works/GokPTGO23.pdf}{GokPTGO23}~\cite{GokPTGO23}, \href{../works/WangB23.pdf}{WangB23}~\cite{WangB23}, \href{../works/KameugneFND23.pdf}{KameugneFND23}~\cite{KameugneFND23}, \href{../works/MarliereSPR23.pdf}{MarliereSPR23}~\cite{MarliereSPR23}, \href{../works/YuraszeckMCCR23.pdf}{YuraszeckMCCR23}~\cite{YuraszeckMCCR23}, \href{../works/CzerniachowskaWZ23.pdf}{CzerniachowskaWZ23}~\cite{CzerniachowskaWZ23}, \href{../works/abs-2305-19888.pdf}{abs-2305-19888}~\cite{abs-2305-19888}, \href{../works/AlfieriGPS23.pdf}{AlfieriGPS23}~\cite{AlfieriGPS23}, \href{../works/JuvinHL23a.pdf}{JuvinHL23a}~\cite{JuvinHL23a}, \href{../works/AalianPG23.pdf}{AalianPG23}~\cite{AalianPG23}, \href{../works/TardivoDFMP23.pdf}{TardivoDFMP23}~\cite{TardivoDFMP23}, \href{../works/GurPAE23.pdf}{GurPAE23}~\cite{GurPAE23}, \href{../works/AbreuPNF23.pdf}{AbreuPNF23}~\cite{AbreuPNF23}, \href{../works/HeinzNVH22.pdf}{HeinzNVH22}~\cite{HeinzNVH22}, \href{../works/AbreuN22.pdf}{AbreuN22}~\cite{AbreuN22}, \href{../works/OrnekOS20.pdf}{OrnekOS20}~\cite{OrnekOS20}, \href{../works/TouatBT22.pdf}{TouatBT22}~\cite{TouatBT22}, \href{../works/YunusogluY22.pdf}{YunusogluY22}~\cite{YunusogluY22}, \href{../works/SubulanC22.pdf}{SubulanC22}~\cite{SubulanC22}... (Total: 406) & \href{../works/Caballero23.pdf}{Caballero23}~\cite{Caballero23}, \href{../works/abs-2312-13682.pdf}{abs-2312-13682}~\cite{abs-2312-13682}, \href{../works/AfsarVPG23.pdf}{AfsarVPG23}~\cite{AfsarVPG23}, \href{../works/Adelgren2023.pdf}{Adelgren2023}~\cite{Adelgren2023}, \href{../works/TasselGS23.pdf}{TasselGS23}~\cite{TasselGS23}, \href{../works/AbreuNP23.pdf}{AbreuNP23}~\cite{AbreuNP23}, \href{../works/PerezGSL23.pdf}{PerezGSL23}~\cite{PerezGSL23}, \href{../works/IsikYA23.pdf}{IsikYA23}~\cite{IsikYA23}, \href{../works/abs-2306-05747.pdf}{abs-2306-05747}~\cite{abs-2306-05747}, \href{../works/Bit-Monnot23.pdf}{Bit-Monnot23}~\cite{Bit-Monnot23}, \href{../works/ElciOH22.pdf}{ElciOH22}~\cite{ElciOH22}, \href{../works/PohlAK22.pdf}{PohlAK22}~\cite{PohlAK22}, \href{../works/MullerMKP22.pdf}{MullerMKP22}~\cite{MullerMKP22}, \href{../works/SvancaraB22.pdf}{SvancaraB22}~\cite{SvancaraB22}, \href{../works/abs-2211-14492.pdf}{abs-2211-14492}~\cite{abs-2211-14492}, \href{../works/YuraszeckMPV22.pdf}{YuraszeckMPV22}~\cite{YuraszeckMPV22}, \href{../works/WinterMMW22.pdf}{WinterMMW22}~\cite{WinterMMW22}, \href{../works/KlankeBYE21.pdf}{KlankeBYE21}~\cite{KlankeBYE21}, \href{../works/Astrand0F21.pdf}{Astrand0F21}~\cite{Astrand0F21}, \href{../works/TangB20.pdf}{TangB20}~\cite{TangB20}, \href{../works/LunardiBLRV20.pdf}{LunardiBLRV20}~\cite{LunardiBLRV20}, \href{../works/WallaceY20.pdf}{WallaceY20}~\cite{WallaceY20}, \href{../works/MokhtarzadehTNF20.pdf}{MokhtarzadehTNF20}~\cite{MokhtarzadehTNF20}, \href{../works/FrimodigS19.pdf}{FrimodigS19}~\cite{FrimodigS19}, \href{../works/abs-1902-01193.pdf}{abs-1902-01193}~\cite{abs-1902-01193}, \href{../works/ParkUJR19.pdf}{ParkUJR19}~\cite{ParkUJR19}, \href{../works/GedikKEK18.pdf}{GedikKEK18}~\cite{GedikKEK18}, \href{../works/BenediktSMVH18.pdf}{BenediktSMVH18}~\cite{BenediktSMVH18}, \href{../works/HoYCLLCLC18.pdf}{HoYCLLCLC18}~\cite{HoYCLLCLC18}... (Total: 64) & \href{../works/AkramNHRSA23.pdf}{AkramNHRSA23}~\cite{AkramNHRSA23}, \href{../works/PenzDN23.pdf}{PenzDN23}~\cite{PenzDN23}, \href{../works/MontemanniD23.pdf}{MontemanniD23}~\cite{MontemanniD23}, \href{../works/SquillaciPR23.pdf}{SquillaciPR23}~\cite{SquillaciPR23}, \href{../works/ZhuSZW23.pdf}{ZhuSZW23}~\cite{ZhuSZW23}, \href{../works/ZhangJZL22.pdf}{ZhangJZL22}~\cite{ZhangJZL22}, \href{../works/EmdeZD22.pdf}{EmdeZD22}~\cite{EmdeZD22}, \href{../works/Teppan22.pdf}{Teppan22}~\cite{Teppan22}, \href{../works/JungblutK22.pdf}{JungblutK22}~\cite{JungblutK22}, \href{../works/PopovicCGNC22.pdf}{PopovicCGNC22}~\cite{PopovicCGNC22}, \href{../works/ArmstrongGOS22.pdf}{ArmstrongGOS22}~\cite{ArmstrongGOS22}, \href{../works/HamPK21.pdf}{HamPK21}~\cite{HamPK21}, \href{../works/AbreuAPNM21.pdf}{AbreuAPNM21}~\cite{AbreuAPNM21}, \href{../works/AbohashimaEG21.pdf}{AbohashimaEG21}~\cite{AbohashimaEG21}, \href{../works/KoehlerBFFHPSSS21.pdf}{KoehlerBFFHPSSS21}~\cite{KoehlerBFFHPSSS21}, \href{../works/abs-2102-08778.pdf}{abs-2102-08778}~\cite{abs-2102-08778}, \href{../works/AntuoriHHEN21.pdf}{AntuoriHHEN21}~\cite{AntuoriHHEN21}, \href{../works/ArmstrongGOS21.pdf}{ArmstrongGOS21}~\cite{ArmstrongGOS21}, \href{../works/FanXG21.pdf}{FanXG21}~\cite{FanXG21}, \href{../works/MejiaY20.pdf}{MejiaY20}~\cite{MejiaY20}, \href{../works/BarzegaranZP20.pdf}{BarzegaranZP20}~\cite{BarzegaranZP20}, \href{../works/ThomasKS20.pdf}{ThomasKS20}~\cite{ThomasKS20}, \href{../works/NattafM20.pdf}{NattafM20}~\cite{NattafM20}, \href{../works/BadicaBIL19.pdf}{BadicaBIL19}~\cite{BadicaBIL19}, \href{../works/HoundjiSW19.pdf}{HoundjiSW19}~\cite{HoundjiSW19}, \href{../works/KucukY19.pdf}{KucukY19}~\cite{KucukY19}, \href{../works/NattafDYW19.pdf}{NattafDYW19}~\cite{NattafDYW19}, \href{../works/ColT19.pdf}{ColT19}~\cite{ColT19}, \href{../works/ZhangW18.pdf}{ZhangW18}~\cite{ZhangW18}... (Total: 69)\\
Concepts & scheduling & \href{../works/PrataAN23.pdf}{PrataAN23}~\cite{PrataAN23}, \href{../works/ForbesHJST24.pdf}{ForbesHJST24}~\cite{ForbesHJST24}, \href{../works/BonninMNE24.pdf}{BonninMNE24}~\cite{BonninMNE24}, \href{../works/abs-2402-00459.pdf}{abs-2402-00459}~\cite{abs-2402-00459}, \href{../works/AbreuNP23.pdf}{AbreuNP23}~\cite{AbreuNP23}, \href{../works/ZhuSZW23.pdf}{ZhuSZW23}~\cite{ZhuSZW23}, \href{../works/IsikYA23.pdf}{IsikYA23}~\cite{IsikYA23}, \href{../works/AalianPG23.pdf}{AalianPG23}~\cite{AalianPG23}, \href{../works/AbreuPNF23.pdf}{AbreuPNF23}~\cite{AbreuPNF23}, \href{../works/abs-2306-05747.pdf}{abs-2306-05747}~\cite{abs-2306-05747}, \href{../works/JuvinHHL23.pdf}{JuvinHHL23}~\cite{JuvinHHL23}, \href{../works/TardivoDFMP23.pdf}{TardivoDFMP23}~\cite{TardivoDFMP23}, \href{../works/YuraszeckMC23.pdf}{YuraszeckMC23}~\cite{YuraszeckMC23}, \href{../works/Fatemi-AnarakiTFV23.pdf}{Fatemi-AnarakiTFV23}~\cite{Fatemi-AnarakiTFV23}, \href{../works/Mehdizadeh-Somarin23.pdf}{Mehdizadeh-Somarin23}~\cite{Mehdizadeh-Somarin23}, \href{../works/KimCMLLP23.pdf}{KimCMLLP23}~\cite{KimCMLLP23}, \href{../works/AkramNHRSA23.pdf}{AkramNHRSA23}~\cite{AkramNHRSA23}, \href{../works/LacknerMMWW23.pdf}{LacknerMMWW23}~\cite{LacknerMMWW23}, \href{../works/GurPAE23.pdf}{GurPAE23}~\cite{GurPAE23}, \href{../works/AlfieriGPS23.pdf}{AlfieriGPS23}~\cite{AlfieriGPS23}, \href{../works/CzerniachowskaWZ23.pdf}{CzerniachowskaWZ23}~\cite{CzerniachowskaWZ23}, \href{../works/WangB23.pdf}{WangB23}~\cite{WangB23}, \href{../works/JuvinHL23.pdf}{JuvinHL23}~\cite{JuvinHL23}, \href{../works/NaderiRR23.pdf}{NaderiRR23}~\cite{NaderiRR23}, \href{../works/PenzDN23.pdf}{PenzDN23}~\cite{PenzDN23}, \href{../works/TasselGS23.pdf}{TasselGS23}~\cite{TasselGS23}, \href{../works/Bit-Monnot23.pdf}{Bit-Monnot23}~\cite{Bit-Monnot23}, \href{../works/abs-2305-19888.pdf}{abs-2305-19888}~\cite{abs-2305-19888}, \href{../works/abs-2312-13682.pdf}{abs-2312-13682}~\cite{abs-2312-13682}... (Total: 573) & \href{../works/HebrardALLCMR22.pdf}{HebrardALLCMR22}~\cite{HebrardALLCMR22}, \href{../works/Kameugne15.pdf}{Kameugne15}~\cite{Kameugne15}, \href{../works/GayHS15.pdf}{GayHS15}~\cite{GayHS15}, \href{../works/BessiereHMQW14.pdf}{BessiereHMQW14}~\cite{BessiereHMQW14}, \href{../works/HoundjiSWD14.pdf}{HoundjiSWD14}~\cite{HoundjiSWD14}, \href{../works/LetortCB13.pdf}{LetortCB13}~\cite{LetortCB13}, \href{../works/LetortBC12.pdf}{LetortBC12}~\cite{LetortBC12}, \href{../works/ClercqPBJ11.pdf}{ClercqPBJ11}~\cite{ClercqPBJ11}, \href{../works/ChapadosJR11.pdf}{ChapadosJR11}~\cite{ChapadosJR11}, \href{../works/Baptiste09.pdf}{Baptiste09}~\cite{Baptiste09}, \href{../works/abs-0907-0939.pdf}{abs-0907-0939}~\cite{abs-0907-0939}, \href{../works/Acuna-AgostMFG09.pdf}{Acuna-AgostMFG09}~\cite{Acuna-AgostMFG09}, \href{../works/GomesHS06.pdf}{GomesHS06}~\cite{GomesHS06}, \href{../works/DilkinaDH05.pdf}{DilkinaDH05}~\cite{DilkinaDH05}, \href{../works/MoffittPP05.pdf}{MoffittPP05}~\cite{MoffittPP05}, \href{../works/WuBB05.pdf}{WuBB05}~\cite{WuBB05}, \href{../works/HebrardTW05.pdf}{HebrardTW05}~\cite{HebrardTW05}, \href{../works/ValleMGT03.pdf}{ValleMGT03}~\cite{ValleMGT03}, \href{../works/Vilim03.pdf}{Vilim03}~\cite{Vilim03}, \href{../works/HookerY02.pdf}{HookerY02}~\cite{HookerY02}, \href{../works/Vilim02.pdf}{Vilim02}~\cite{Vilim02}, \href{../works/RodriguezDG02.pdf}{RodriguezDG02}~\cite{RodriguezDG02}, \href{../works/FrostD98.pdf}{FrostD98}~\cite{FrostD98}, \href{../works/CestaOS98.pdf}{CestaOS98}~\cite{CestaOS98}, \href{../works/Touraivane95.pdf}{Touraivane95}~\cite{Touraivane95} & \href{../works/Hooker17.pdf}{Hooker17}~\cite{Hooker17}, \href{../works/RossiTHP07.pdf}{RossiTHP07}~\cite{RossiTHP07}, \href{../works/AbrilSB05.pdf}{AbrilSB05}~\cite{AbrilSB05}, \href{../works/VanczaM01.pdf}{VanczaM01}~\cite{VanczaM01}\\
Concepts & sequence dependent setup & \href{../works/Groleaz21.pdf}{Groleaz21}~\cite{Groleaz21}, \href{../works/GedikKEK18.pdf}{GedikKEK18}~\cite{GedikKEK18}, \href{../works/TranAB16.pdf}{TranAB16}~\cite{TranAB16}, \href{../works/HamC16.pdf}{HamC16}~\cite{HamC16}, \href{../works/TranB12.pdf}{TranB12}~\cite{TranB12}, \href{../works/Wolf11.pdf}{Wolf11}~\cite{Wolf11}, \href{../works/FocacciLN00.pdf}{FocacciLN00}~\cite{FocacciLN00} & \href{../works/IsikYA23.pdf}{IsikYA23}~\cite{IsikYA23}, \href{../works/YuraszeckMPV22.pdf}{YuraszeckMPV22}~\cite{YuraszeckMPV22}, \href{../works/GeitzGSSW22.pdf}{GeitzGSSW22}~\cite{GeitzGSSW22}, \href{../works/MengZRZL20.pdf}{MengZRZL20}~\cite{MengZRZL20}, \href{../works/CauwelaertDS20.pdf}{CauwelaertDS20}~\cite{CauwelaertDS20}, \href{../works/ZarandiASC20.pdf}{ZarandiASC20}~\cite{ZarandiASC20}, \href{../works/RiahiNS018.pdf}{RiahiNS018}~\cite{RiahiNS018}, \href{../works/Dejemeppe16.pdf}{Dejemeppe16}~\cite{Dejemeppe16}, \href{../works/GrimesH15.pdf}{GrimesH15}~\cite{GrimesH15}, \href{../works/LombardiM12.pdf}{LombardiM12}~\cite{LombardiM12}, \href{../works/Simonis07.pdf}{Simonis07}~\cite{Simonis07}, \href{../works/ArtiguesBF04.pdf}{ArtiguesBF04}~\cite{ArtiguesBF04} & \href{../works/PrataAN23.pdf}{PrataAN23}~\cite{PrataAN23}, \href{../works/GuoZ23.pdf}{GuoZ23}~\cite{GuoZ23}, \href{../works/abs-2305-19888.pdf}{abs-2305-19888}~\cite{abs-2305-19888}, \href{../works/NaderiRR23.pdf}{NaderiRR23}~\cite{NaderiRR23}, \href{../works/Adelgren2023.pdf}{Adelgren2023}~\cite{Adelgren2023}, \href{../works/YunusogluY22.pdf}{YunusogluY22}~\cite{YunusogluY22}, \href{../works/PohlAK22.pdf}{PohlAK22}~\cite{PohlAK22}, \href{../works/NaderiBZ22a.pdf}{NaderiBZ22a}~\cite{NaderiBZ22a}, \href{../works/HeinzNVH22.pdf}{HeinzNVH22}~\cite{HeinzNVH22}, \href{../works/OujanaAYB22.pdf}{OujanaAYB22}~\cite{OujanaAYB22}, \href{../works/HamPK21.pdf}{HamPK21}~\cite{HamPK21}, \href{../works/ArmstrongGOS21.pdf}{ArmstrongGOS21}~\cite{ArmstrongGOS21}, \href{../works/Bedhief21.pdf}{Bedhief21}~\cite{Bedhief21}, \href{../works/Astrand21.pdf}{Astrand21}~\cite{Astrand21}, \href{../works/Mercier-AubinGQ20.pdf}{Mercier-AubinGQ20}~\cite{Mercier-AubinGQ20}, \href{../works/MejiaY20.pdf}{MejiaY20}~\cite{MejiaY20}, \href{../works/RoshanaeiBAUB20.pdf}{RoshanaeiBAUB20}~\cite{RoshanaeiBAUB20}, \href{../works/MalapertN19.pdf}{MalapertN19}~\cite{MalapertN19}, \href{../works/Novas19.pdf}{Novas19}~\cite{Novas19}, \href{../works/KucukY19.pdf}{KucukY19}~\cite{KucukY19}, \href{../works/Hooker19.pdf}{Hooker19}~\cite{Hooker19}, \href{../works/ArbaouiY18.pdf}{ArbaouiY18}~\cite{ArbaouiY18}, \href{../works/LaborieRSV18.pdf}{LaborieRSV18}~\cite{LaborieRSV18}, \href{../works/FahimiOQ18.pdf}{FahimiOQ18}~\cite{FahimiOQ18}, \href{../works/Ham18.pdf}{Ham18}~\cite{Ham18}, \href{../works/RoshanaeiLAU17.pdf}{RoshanaeiLAU17}~\cite{RoshanaeiLAU17}, \href{../works/Pralet17.pdf}{Pralet17}~\cite{Pralet17}, \href{../works/HookerH17.pdf}{HookerH17}~\cite{HookerH17}, \href{../works/Fahimi16.pdf}{Fahimi16}~\cite{Fahimi16}... (Total: 47)\\
Concepts & setup-time & \href{../works/PrataAN23.pdf}{PrataAN23}~\cite{PrataAN23}, \href{../works/IsikYA23.pdf}{IsikYA23}~\cite{IsikYA23}, \href{../works/AbreuPNF23.pdf}{AbreuPNF23}~\cite{AbreuPNF23}, \href{../works/LacknerMMWW23.pdf}{LacknerMMWW23}~\cite{LacknerMMWW23}, \href{../works/abs-2305-19888.pdf}{abs-2305-19888}~\cite{abs-2305-19888}, \href{../works/AbreuNP23.pdf}{AbreuNP23}~\cite{AbreuNP23}, \href{../works/NaderiRR23.pdf}{NaderiRR23}~\cite{NaderiRR23}, \href{../works/GeitzGSSW22.pdf}{GeitzGSSW22}~\cite{GeitzGSSW22}, \href{../works/NaderiBZ22.pdf}{NaderiBZ22}~\cite{NaderiBZ22}, \href{../works/WinterMMW22.pdf}{WinterMMW22}~\cite{WinterMMW22}, \href{../works/OujanaAYB22.pdf}{OujanaAYB22}~\cite{OujanaAYB22}, \href{../works/YunusogluY22.pdf}{YunusogluY22}~\cite{YunusogluY22}, \href{../works/YuraszeckMPV22.pdf}{YuraszeckMPV22}~\cite{YuraszeckMPV22}, \href{../works/PohlAK22.pdf}{PohlAK22}~\cite{PohlAK22}, \href{../works/HeinzNVH22.pdf}{HeinzNVH22}~\cite{HeinzNVH22}, \href{../works/AbreuN22.pdf}{AbreuN22}~\cite{AbreuN22}, \href{../works/ColT22.pdf}{ColT22}~\cite{ColT22}, \href{../works/Groleaz21.pdf}{Groleaz21}~\cite{Groleaz21}, \href{../works/Astrand21.pdf}{Astrand21}~\cite{Astrand21}, \href{../works/LacknerMMWW21.pdf}{LacknerMMWW21}~\cite{LacknerMMWW21}, \href{../works/Lunardi20.pdf}{Lunardi20}~\cite{Lunardi20}, \href{../works/NattafM20.pdf}{NattafM20}~\cite{NattafM20}, \href{../works/QinDCS20.pdf}{QinDCS20}~\cite{QinDCS20}, \href{../works/GroleazNS20a.pdf}{GroleazNS20a}~\cite{GroleazNS20a}, \href{../works/MejiaY20.pdf}{MejiaY20}~\cite{MejiaY20}, \href{../works/GroleazNS20.pdf}{GroleazNS20}~\cite{GroleazNS20}, \href{../works/Mercier-AubinGQ20.pdf}{Mercier-AubinGQ20}~\cite{Mercier-AubinGQ20}, \href{../works/LunardiBLRV20.pdf}{LunardiBLRV20}~\cite{LunardiBLRV20}, \href{../works/CauwelaertDS20.pdf}{CauwelaertDS20}~\cite{CauwelaertDS20}... (Total: 61) & \href{../works/Adelgren2023.pdf}{Adelgren2023}~\cite{Adelgren2023}, \href{../works/ZhuSZW23.pdf}{ZhuSZW23}~\cite{ZhuSZW23}, \href{../works/AlfieriGPS23.pdf}{AlfieriGPS23}~\cite{AlfieriGPS23}, \href{../works/CzerniachowskaWZ23.pdf}{CzerniachowskaWZ23}~\cite{CzerniachowskaWZ23}, \href{../works/PenzDN23.pdf}{PenzDN23}~\cite{PenzDN23}, \href{../works/KimCMLLP23.pdf}{KimCMLLP23}~\cite{KimCMLLP23}, \href{../works/GokPTGO23.pdf}{GokPTGO23}~\cite{GokPTGO23}, \href{../works/LiFJZLL22.pdf}{LiFJZLL22}~\cite{LiFJZLL22}, \href{../works/Bedhief21.pdf}{Bedhief21}~\cite{Bedhief21}, \href{../works/FanXG21.pdf}{FanXG21}~\cite{FanXG21}, \href{../works/AbreuAPNM21.pdf}{AbreuAPNM21}~\cite{AbreuAPNM21}, \href{../works/ArmstrongGOS21.pdf}{ArmstrongGOS21}~\cite{ArmstrongGOS21}, \href{../works/AstrandJZ20.pdf}{AstrandJZ20}~\cite{AstrandJZ20}, \href{../works/LaborieRSV18.pdf}{LaborieRSV18}~\cite{LaborieRSV18}, \href{../works/HookerH17.pdf}{HookerH17}~\cite{HookerH17}, \href{../works/NovaraNH16.pdf}{NovaraNH16}~\cite{NovaraNH16}, \href{../works/OrnekO16.pdf}{OrnekO16}~\cite{OrnekO16}, \href{../works/HamC16.pdf}{HamC16}~\cite{HamC16}, \href{../works/GaySS14.pdf}{GaySS14}~\cite{GaySS14}, \href{../works/KelarevaTK13.pdf}{KelarevaTK13}~\cite{KelarevaTK13}, \href{../works/OzturkTHO13.pdf}{OzturkTHO13}~\cite{OzturkTHO13}, \href{../works/Wolf11.pdf}{Wolf11}~\cite{Wolf11}, \href{../works/Malapert11.pdf}{Malapert11}~\cite{Malapert11}, \href{../works/ThiruvadyBME09.pdf}{ThiruvadyBME09}~\cite{ThiruvadyBME09}, \href{../works/BeniniBGM06.pdf}{BeniniBGM06}~\cite{BeniniBGM06}, \href{../works/HarjunkoskiG02.pdf}{HarjunkoskiG02}~\cite{HarjunkoskiG02}, \href{../works/Timpe02.pdf}{Timpe02}~\cite{Timpe02}, \href{../works/Vilim02.pdf}{Vilim02}~\cite{Vilim02} & \href{../works/EfthymiouY23.pdf}{EfthymiouY23}~\cite{EfthymiouY23}, \href{../works/YuraszeckMCCR23.pdf}{YuraszeckMCCR23}~\cite{YuraszeckMCCR23}, \href{../works/JuvinHL23.pdf}{JuvinHL23}~\cite{JuvinHL23}, \href{../works/AfsarVPG23.pdf}{AfsarVPG23}~\cite{AfsarVPG23}, \href{../works/JuvinHL23a.pdf}{JuvinHL23a}~\cite{JuvinHL23a}, \href{../works/Mehdizadeh-Somarin23.pdf}{Mehdizadeh-Somarin23}~\cite{Mehdizadeh-Somarin23}, \href{../works/GuoZ23.pdf}{GuoZ23}~\cite{GuoZ23}, \href{../works/Fatemi-AnarakiTFV23.pdf}{Fatemi-AnarakiTFV23}~\cite{Fatemi-AnarakiTFV23}, \href{../works/JuvinHHL23.pdf}{JuvinHHL23}~\cite{JuvinHHL23}, \href{../works/JuvinHL22.pdf}{JuvinHL22}~\cite{JuvinHL22}, \href{../works/abs-2211-14492.pdf}{abs-2211-14492}~\cite{abs-2211-14492}, \href{../works/ZhangJZL22.pdf}{ZhangJZL22}~\cite{ZhangJZL22}, \href{../works/MullerMKP22.pdf}{MullerMKP22}~\cite{MullerMKP22}, \href{../works/Teppan22.pdf}{Teppan22}~\cite{Teppan22}, \href{../works/NaderiBZ22a.pdf}{NaderiBZ22a}~\cite{NaderiBZ22a}, \href{../works/ZhangYW21.pdf}{ZhangYW21}~\cite{ZhangYW21}, \href{../works/AbohashimaEG21.pdf}{AbohashimaEG21}~\cite{AbohashimaEG21}, \href{../works/HamPK21.pdf}{HamPK21}~\cite{HamPK21}, \href{../works/BenderWS21.pdf}{BenderWS21}~\cite{BenderWS21}, \href{../works/Polo-MejiaALB20.pdf}{Polo-MejiaALB20}~\cite{Polo-MejiaALB20}, \href{../works/HauderBRPA20.pdf}{HauderBRPA20}~\cite{HauderBRPA20}, \href{../works/MokhtarzadehTNF20.pdf}{MokhtarzadehTNF20}~\cite{MokhtarzadehTNF20}, \href{../works/GokGSTO20.pdf}{GokGSTO20}~\cite{GokGSTO20}, \href{../works/GodetLHS20.pdf}{GodetLHS20}~\cite{GodetLHS20}, \href{../works/RoshanaeiBAUB20.pdf}{RoshanaeiBAUB20}~\cite{RoshanaeiBAUB20}, \href{../works/Caballero19.pdf}{Caballero19}~\cite{Caballero19}, \href{../works/abs-1902-09244.pdf}{abs-1902-09244}~\cite{abs-1902-09244}, \href{../works/WikarekS19.pdf}{WikarekS19}~\cite{WikarekS19}, \href{../works/BehrensLM19.pdf}{BehrensLM19}~\cite{BehrensLM19}... (Total: 82)\\
Concepts & single-machine scheduling & \href{../works/PenzDN23.pdf}{PenzDN23}~\cite{PenzDN23}, \href{../works/TouatBT22.pdf}{TouatBT22}~\cite{TouatBT22}, \href{../works/ZarandiASC20.pdf}{ZarandiASC20}~\cite{ZarandiASC20}, \href{../works/BajestaniB15.pdf}{BajestaniB15}~\cite{BajestaniB15} & \href{../works/PrataAN23.pdf}{PrataAN23}~\cite{PrataAN23}, \href{../works/AlfieriGPS23.pdf}{AlfieriGPS23}~\cite{AlfieriGPS23}, \href{../works/Groleaz21.pdf}{Groleaz21}~\cite{Groleaz21}, \href{../works/BenediktMH20.pdf}{BenediktMH20}~\cite{BenediktMH20}, \href{../works/BogaerdtW19.pdf}{BogaerdtW19}~\cite{BogaerdtW19}, \href{../works/TerekhovDOB12.pdf}{TerekhovDOB12}~\cite{TerekhovDOB12}, \href{../works/KovacsB11.pdf}{KovacsB11}~\cite{KovacsB11}, \href{../works/WuBB09.pdf}{WuBB09}~\cite{WuBB09}, \href{../works/JainM99.pdf}{JainM99}~\cite{JainM99} & \href{../works/BonninMNE24.pdf}{BonninMNE24}~\cite{BonninMNE24}, \href{../works/Fatemi-AnarakiTFV23.pdf}{Fatemi-AnarakiTFV23}~\cite{Fatemi-AnarakiTFV23}, \href{../works/Mehdizadeh-Somarin23.pdf}{Mehdizadeh-Somarin23}~\cite{Mehdizadeh-Somarin23}, \href{../works/PohlAK22.pdf}{PohlAK22}~\cite{PohlAK22}, \href{../works/ZhangJZL22.pdf}{ZhangJZL22}~\cite{ZhangJZL22}, \href{../works/ElciOH22.pdf}{ElciOH22}~\cite{ElciOH22}, \href{../works/EmdeZD22.pdf}{EmdeZD22}~\cite{EmdeZD22}, \href{../works/KoehlerBFFHPSSS21.pdf}{KoehlerBFFHPSSS21}~\cite{KoehlerBFFHPSSS21}, \href{../works/HamPK21.pdf}{HamPK21}~\cite{HamPK21}, \href{../works/HillTV21.pdf}{HillTV21}~\cite{HillTV21}, \href{../works/QinWSLS21.pdf}{QinWSLS21}~\cite{QinWSLS21}, \href{../works/PandeyS21a.pdf}{PandeyS21a}~\cite{PandeyS21a}, \href{../works/NattafDYW19.pdf}{NattafDYW19}~\cite{NattafDYW19}, \href{../works/NattafHKAL19.pdf}{NattafHKAL19}~\cite{NattafHKAL19}, \href{../works/Tom19.pdf}{Tom19}~\cite{Tom19}, \href{../works/Hooker19.pdf}{Hooker19}~\cite{Hooker19}, \href{../works/MalapertN19.pdf}{MalapertN19}~\cite{MalapertN19}, \href{../works/BenediktSMVH18.pdf}{BenediktSMVH18}~\cite{BenediktSMVH18}, \href{../works/TanT18.pdf}{TanT18}~\cite{TanT18}, \href{../works/Tesch18.pdf}{Tesch18}~\cite{Tesch18}, \href{../works/GomesM17.pdf}{GomesM17}~\cite{GomesM17}, \href{../works/TranWDRFOVB16.pdf}{TranWDRFOVB16}~\cite{TranWDRFOVB16}, \href{../works/ZarandiKS16.pdf}{ZarandiKS16}~\cite{ZarandiKS16}, \href{../works/TranAB16.pdf}{TranAB16}~\cite{TranAB16}, \href{../works/DoulabiRP16.pdf}{DoulabiRP16}~\cite{DoulabiRP16}, \href{../works/BurtLPS15.pdf}{BurtLPS15}~\cite{BurtLPS15}, \href{../works/BajestaniB13.pdf}{BajestaniB13}~\cite{BajestaniB13}, \href{../works/GuSS13.pdf}{GuSS13}~\cite{GuSS13}, \href{../works/TranB12.pdf}{TranB12}~\cite{TranB12}... (Total: 44)\\
Concepts & single-stage scheduling &  & \href{../works/HarjunkoskiG02.pdf}{HarjunkoskiG02}~\cite{HarjunkoskiG02} & \href{../works/TerekhovDOB12.pdf}{TerekhovDOB12}~\cite{TerekhovDOB12}\\
Concepts & stochastic & \href{../works/ForbesHJST24.pdf}{ForbesHJST24}~\cite{ForbesHJST24}, \href{../works/AfsarVPG23.pdf}{AfsarVPG23}~\cite{AfsarVPG23}, \href{../works/GurPAE23.pdf}{GurPAE23}~\cite{GurPAE23}, \href{../works/GuoZ23.pdf}{GuoZ23}~\cite{GuoZ23}, \href{../works/GokPTGO23.pdf}{GokPTGO23}~\cite{GokPTGO23}, \href{../works/PenzDN23.pdf}{PenzDN23}~\cite{PenzDN23}, \href{../works/NaderiBZ22.pdf}{NaderiBZ22}~\cite{NaderiBZ22}, \href{../works/SubulanC22.pdf}{SubulanC22}~\cite{SubulanC22}, \href{../works/ElciOH22.pdf}{ElciOH22}~\cite{ElciOH22}, \href{../works/Astrand21.pdf}{Astrand21}~\cite{Astrand21}, \href{../works/Groleaz21.pdf}{Groleaz21}~\cite{Groleaz21}, \href{../works/AntuoriHHEN20.pdf}{AntuoriHHEN20}~\cite{AntuoriHHEN20}, \href{../works/ZarandiASC20.pdf}{ZarandiASC20}~\cite{ZarandiASC20}, \href{../works/RoshanaeiBAUB20.pdf}{RoshanaeiBAUB20}~\cite{RoshanaeiBAUB20}, \href{../works/Hooker19.pdf}{Hooker19}~\cite{Hooker19}, \href{../works/TranPZLDB18.pdf}{TranPZLDB18}~\cite{TranPZLDB18}, \href{../works/ShinBBHO18.pdf}{ShinBBHO18}~\cite{ShinBBHO18}, \href{../works/Froger16.pdf}{Froger16}~\cite{Froger16}, \href{../works/LombardiBM15.pdf}{LombardiBM15}~\cite{LombardiBM15}, \href{../works/BonfiettiLM14.pdf}{BonfiettiLM14}~\cite{BonfiettiLM14}, \href{../works/TerekhovTDB14.pdf}{TerekhovTDB14}~\cite{TerekhovTDB14}, \href{../works/HarjunkoskiMBC14.pdf}{HarjunkoskiMBC14}~\cite{HarjunkoskiMBC14}, \href{../works/BajestaniB13.pdf}{BajestaniB13}~\cite{BajestaniB13}, \href{../works/LombardiMB13.pdf}{LombardiMB13}~\cite{LombardiMB13}, \href{../works/SchausHMCMD11.pdf}{SchausHMCMD11}~\cite{SchausHMCMD11}, \href{../works/Lombardi10.pdf}{Lombardi10}~\cite{Lombardi10}, \href{../works/LombardiMRB10.pdf}{LombardiMRB10}~\cite{LombardiMRB10}, \href{../works/LombardiM10a.pdf}{LombardiM10a}~\cite{LombardiM10a}, \href{../works/ThiruvadyBME09.pdf}{ThiruvadyBME09}~\cite{ThiruvadyBME09}... (Total: 37) & \href{../works/YuraszeckMC23.pdf}{YuraszeckMC23}~\cite{YuraszeckMC23}, \href{../works/OrnekOS20.pdf}{OrnekOS20}~\cite{OrnekOS20}, \href{../works/YuraszeckMPV22.pdf}{YuraszeckMPV22}~\cite{YuraszeckMPV22}, \href{../works/FarsiTM22.pdf}{FarsiTM22}~\cite{FarsiTM22}, \href{../works/AntuoriHHEN21.pdf}{AntuoriHHEN21}~\cite{AntuoriHHEN21}, \href{../works/HubnerGSV21.pdf}{HubnerGSV21}~\cite{HubnerGSV21}, \href{../works/AstrandJZ20.pdf}{AstrandJZ20}~\cite{AstrandJZ20}, \href{../works/SacramentoSP20.pdf}{SacramentoSP20}~\cite{SacramentoSP20}, \href{../works/Lunardi20.pdf}{Lunardi20}~\cite{Lunardi20}, \href{../works/FrimodigS19.pdf}{FrimodigS19}~\cite{FrimodigS19}, \href{../works/GurEA19.pdf}{GurEA19}~\cite{GurEA19}, \href{../works/ParkUJR19.pdf}{ParkUJR19}~\cite{ParkUJR19}, \href{../works/LaborieRSV18.pdf}{LaborieRSV18}~\cite{LaborieRSV18}, \href{../works/ZhangW18.pdf}{ZhangW18}~\cite{ZhangW18}, \href{../works/RoshanaeiLAU17.pdf}{RoshanaeiLAU17}~\cite{RoshanaeiLAU17}, \href{../works/HookerH17.pdf}{HookerH17}~\cite{HookerH17}, \href{../works/DoulabiRP16.pdf}{DoulabiRP16}~\cite{DoulabiRP16}, \href{../works/LimHTB16.pdf}{LimHTB16}~\cite{LimHTB16}, \href{../works/BajestaniB15.pdf}{BajestaniB15}~\cite{BajestaniB15}, \href{../works/TranTDB13.pdf}{TranTDB13}~\cite{TranTDB13}, \href{../works/LombardiM12a.pdf}{LombardiM12a}~\cite{LombardiM12a}, \href{../works/CobanH11.pdf}{CobanH11}~\cite{CobanH11}, \href{../works/LombardiM10.pdf}{LombardiM10}~\cite{LombardiM10}, \href{../works/WuBB09.pdf}{WuBB09}~\cite{WuBB09}, \href{../works/DilkinaDH05.pdf}{DilkinaDH05}~\cite{DilkinaDH05}, \href{../works/BeckW04.pdf}{BeckW04}~\cite{BeckW04}, \href{../works/JainM99.pdf}{JainM99}~\cite{JainM99} & \href{../works/PrataAN23.pdf}{PrataAN23}~\cite{PrataAN23}, \href{../works/AlfieriGPS23.pdf}{AlfieriGPS23}~\cite{AlfieriGPS23}, \href{../works/JuvinHL23a.pdf}{JuvinHL23a}~\cite{JuvinHL23a}, \href{../works/AbreuPNF23.pdf}{AbreuPNF23}~\cite{AbreuPNF23}, \href{../works/Mehdizadeh-Somarin23.pdf}{Mehdizadeh-Somarin23}~\cite{Mehdizadeh-Somarin23}, \href{../works/AbreuNP23.pdf}{AbreuNP23}~\cite{AbreuNP23}, \href{../works/JuvinHL23.pdf}{JuvinHL23}~\cite{JuvinHL23}, \href{../works/NaderiBZ22a.pdf}{NaderiBZ22a}~\cite{NaderiBZ22a}, \href{../works/AbreuN22.pdf}{AbreuN22}~\cite{AbreuN22}, \href{../works/CampeauG22.pdf}{CampeauG22}~\cite{CampeauG22}, \href{../works/ZhangJZL22.pdf}{ZhangJZL22}~\cite{ZhangJZL22}, \href{../works/PopovicCGNC22.pdf}{PopovicCGNC22}~\cite{PopovicCGNC22}, \href{../works/EmdeZD22.pdf}{EmdeZD22}~\cite{EmdeZD22}, \href{../works/LiFJZLL22.pdf}{LiFJZLL22}~\cite{LiFJZLL22}, \href{../works/EtminaniesfahaniGNMS22.pdf}{EtminaniesfahaniGNMS22}~\cite{EtminaniesfahaniGNMS22}, \href{../works/MullerMKP22.pdf}{MullerMKP22}~\cite{MullerMKP22}, \href{../works/PohlAK22.pdf}{PohlAK22}~\cite{PohlAK22}, \href{../works/AbohashimaEG21.pdf}{AbohashimaEG21}~\cite{AbohashimaEG21}, \href{../works/FanXG21.pdf}{FanXG21}~\cite{FanXG21}, \href{../works/VlkHT21.pdf}{VlkHT21}~\cite{VlkHT21}, \href{../works/AbreuAPNM21.pdf}{AbreuAPNM21}~\cite{AbreuAPNM21}, \href{../works/Lemos21.pdf}{Lemos21}~\cite{Lemos21}, \href{../works/GokGSTO20.pdf}{GokGSTO20}~\cite{GokGSTO20}, \href{../works/HauderBRPA20.pdf}{HauderBRPA20}~\cite{HauderBRPA20}, \href{../works/BadicaBI20.pdf}{BadicaBI20}~\cite{BadicaBI20}, \href{../works/AntunesABD20.pdf}{AntunesABD20}~\cite{AntunesABD20}, \href{../works/ZouZ20.pdf}{ZouZ20}~\cite{ZouZ20}, \href{../works/NattafDYW19.pdf}{NattafDYW19}~\cite{NattafDYW19}, \href{../works/abs-1902-01193.pdf}{abs-1902-01193}~\cite{abs-1902-01193}... (Total: 86)\\
Concepts & stock level & \href{../works/LopesCSM10.pdf}{LopesCSM10}~\cite{LopesCSM10}, \href{../works/SimonisC95.pdf}{SimonisC95}~\cite{SimonisC95} & \href{../works/German18.pdf}{German18}~\cite{German18}, \href{../works/RossiTHP07.pdf}{RossiTHP07}~\cite{RossiTHP07}, \href{../works/Timpe02.pdf}{Timpe02}~\cite{Timpe02}, \href{../works/Simonis99.pdf}{Simonis99}~\cite{Simonis99} & \href{../works/KhemmoudjPB06.pdf}{KhemmoudjPB06}~\cite{KhemmoudjPB06}, \href{../works/SimonisCK00.pdf}{SimonisCK00}~\cite{SimonisCK00}, \href{../works/Beck99.pdf}{Beck99}~\cite{Beck99}, \href{../works/Simonis95a.pdf}{Simonis95a}~\cite{Simonis95a}\\
Concepts & sustainability &  &  & \href{../works/MontemanniD23.pdf}{MontemanniD23}~\cite{MontemanniD23}, \href{../works/MontemanniD23a.pdf}{MontemanniD23a}~\cite{MontemanniD23a}, \href{../works/Mehdizadeh-Somarin23.pdf}{Mehdizadeh-Somarin23}~\cite{Mehdizadeh-Somarin23}, \href{../works/AbreuPNF23.pdf}{AbreuPNF23}~\cite{AbreuPNF23}, \href{../works/PenzDN23.pdf}{PenzDN23}~\cite{PenzDN23}, \href{../works/CzerniachowskaWZ23.pdf}{CzerniachowskaWZ23}~\cite{CzerniachowskaWZ23}, \href{../works/PopovicCGNC22.pdf}{PopovicCGNC22}~\cite{PopovicCGNC22}, \href{../works/MullerMKP22.pdf}{MullerMKP22}~\cite{MullerMKP22}, \href{../works/BenediktMH20.pdf}{BenediktMH20}~\cite{BenediktMH20}, \href{../works/HoYCLLCLC18.pdf}{HoYCLLCLC18}~\cite{HoYCLLCLC18}, \href{../works/Froger16.pdf}{Froger16}~\cite{Froger16}, \href{../works/BridiBLMB16.pdf}{BridiBLMB16}~\cite{BridiBLMB16}, \href{../works/Madi-WambaB16.pdf}{Madi-WambaB16}~\cite{Madi-WambaB16}, \href{../works/GrimesIOS14.pdf}{GrimesIOS14}~\cite{GrimesIOS14}, \href{../works/IfrimOS12.pdf}{IfrimOS12}~\cite{IfrimOS12}\\
Concepts & tardiness & \href{../works/PrataAN23.pdf}{PrataAN23}~\cite{PrataAN23}, \href{../works/NaderiRR23.pdf}{NaderiRR23}~\cite{NaderiRR23}, \href{../works/IsikYA23.pdf}{IsikYA23}~\cite{IsikYA23}, \href{../works/GokPTGO23.pdf}{GokPTGO23}~\cite{GokPTGO23}, \href{../works/KimCMLLP23.pdf}{KimCMLLP23}~\cite{KimCMLLP23}, \href{../works/LacknerMMWW23.pdf}{LacknerMMWW23}~\cite{LacknerMMWW23}, \href{../works/AlfieriGPS23.pdf}{AlfieriGPS23}~\cite{AlfieriGPS23}, \href{../works/AbreuPNF23.pdf}{AbreuPNF23}~\cite{AbreuPNF23}, \href{../works/WinterMMW22.pdf}{WinterMMW22}~\cite{WinterMMW22}, \href{../works/YunusogluY22.pdf}{YunusogluY22}~\cite{YunusogluY22}, \href{../works/OujanaAYB22.pdf}{OujanaAYB22}~\cite{OujanaAYB22}, \href{../works/NaderiBZ22.pdf}{NaderiBZ22}~\cite{NaderiBZ22}, \href{../works/PohlAK22.pdf}{PohlAK22}~\cite{PohlAK22}, \href{../works/TouatBT22.pdf}{TouatBT22}~\cite{TouatBT22}, \href{../works/AbreuN22.pdf}{AbreuN22}~\cite{AbreuN22}, \href{../works/abs-2211-14492.pdf}{abs-2211-14492}~\cite{abs-2211-14492}, \href{../works/Groleaz21.pdf}{Groleaz21}~\cite{Groleaz21}, \href{../works/FanXG21.pdf}{FanXG21}~\cite{FanXG21}, \href{../works/LacknerMMWW21.pdf}{LacknerMMWW21}~\cite{LacknerMMWW21}, \href{../works/AntuoriHHEN21.pdf}{AntuoriHHEN21}~\cite{AntuoriHHEN21}, \href{../works/ZarandiASC20.pdf}{ZarandiASC20}~\cite{ZarandiASC20}, \href{../works/HauderBRPA20.pdf}{HauderBRPA20}~\cite{HauderBRPA20}, \href{../works/GroleazNS20a.pdf}{GroleazNS20a}~\cite{GroleazNS20a}, \href{../works/Mercier-AubinGQ20.pdf}{Mercier-AubinGQ20}~\cite{Mercier-AubinGQ20}, \href{../works/MengZRZL20.pdf}{MengZRZL20}~\cite{MengZRZL20}, \href{../works/TangB20.pdf}{TangB20}~\cite{TangB20}, \href{../works/AntuoriHHEN20.pdf}{AntuoriHHEN20}~\cite{AntuoriHHEN20}, \href{../works/ParkUJR19.pdf}{ParkUJR19}~\cite{ParkUJR19}, \href{../works/abs-1902-09244.pdf}{abs-1902-09244}~\cite{abs-1902-09244}... (Total: 63) & \href{../works/abs-2402-00459.pdf}{abs-2402-00459}~\cite{abs-2402-00459}, \href{../works/AbreuNP23.pdf}{AbreuNP23}~\cite{AbreuNP23}, \href{../works/PenzDN23.pdf}{PenzDN23}~\cite{PenzDN23}, \href{../works/SubulanC22.pdf}{SubulanC22}~\cite{SubulanC22}, \href{../works/FarsiTM22.pdf}{FarsiTM22}~\cite{FarsiTM22}, \href{../works/EmdeZD22.pdf}{EmdeZD22}~\cite{EmdeZD22}, \href{../works/ElciOH22.pdf}{ElciOH22}~\cite{ElciOH22}, \href{../works/ColT22.pdf}{ColT22}~\cite{ColT22}, \href{../works/KovacsTKSG21.pdf}{KovacsTKSG21}~\cite{KovacsTKSG21}, \href{../works/AbreuAPNM21.pdf}{AbreuAPNM21}~\cite{AbreuAPNM21}, \href{../works/GroleazNS20.pdf}{GroleazNS20}~\cite{GroleazNS20}, \href{../works/GokGSTO20.pdf}{GokGSTO20}~\cite{GokGSTO20}, \href{../works/Lunardi20.pdf}{Lunardi20}~\cite{Lunardi20}, \href{../works/GokgurHO18.pdf}{GokgurHO18}~\cite{GokgurHO18}, \href{../works/GedikKEK18.pdf}{GedikKEK18}~\cite{GedikKEK18}, \href{../works/Hooker17.pdf}{Hooker17}~\cite{Hooker17}, \href{../works/CireCH16.pdf}{CireCH16}~\cite{CireCH16}, \href{../works/TranAB16.pdf}{TranAB16}~\cite{TranAB16}, \href{../works/ThiruvadyWGS14.pdf}{ThiruvadyWGS14}~\cite{ThiruvadyWGS14}, \href{../works/TerekhovTDB14.pdf}{TerekhovTDB14}~\cite{TerekhovTDB14}, \href{../works/HarjunkoskiMBC14.pdf}{HarjunkoskiMBC14}~\cite{HarjunkoskiMBC14}, \href{../works/BajestaniB13.pdf}{BajestaniB13}~\cite{BajestaniB13}, \href{../works/Malapert11.pdf}{Malapert11}~\cite{Malapert11}, \href{../works/NovasH10.pdf}{NovasH10}~\cite{NovasH10}, \href{../works/BartakSR10.pdf}{BartakSR10}~\cite{BartakSR10}, \href{../works/Beck06.pdf}{Beck06}~\cite{Beck06}, \href{../works/QuirogaZH05.pdf}{QuirogaZH05}~\cite{QuirogaZH05}, \href{../works/GodardLN05.pdf}{GodardLN05}~\cite{GodardLN05}, \href{../works/Hooker05.pdf}{Hooker05}~\cite{Hooker05}, \href{../works/BeckPS03.pdf}{BeckPS03}~\cite{BeckPS03} & \href{../works/Mehdizadeh-Somarin23.pdf}{Mehdizadeh-Somarin23}~\cite{Mehdizadeh-Somarin23}, \href{../works/JuvinHL23.pdf}{JuvinHL23}~\cite{JuvinHL23}, \href{../works/TasselGS23.pdf}{TasselGS23}~\cite{TasselGS23}, \href{../works/abs-2306-05747.pdf}{abs-2306-05747}~\cite{abs-2306-05747}, \href{../works/LiFJZLL22.pdf}{LiFJZLL22}~\cite{LiFJZLL22}, \href{../works/EtminaniesfahaniGNMS22.pdf}{EtminaniesfahaniGNMS22}~\cite{EtminaniesfahaniGNMS22}, \href{../works/NaderiBZ22a.pdf}{NaderiBZ22a}~\cite{NaderiBZ22a}, \href{../works/ZhangJZL22.pdf}{ZhangJZL22}~\cite{ZhangJZL22}, \href{../works/VlkHT21.pdf}{VlkHT21}~\cite{VlkHT21}, \href{../works/KoehlerBFFHPSSS21.pdf}{KoehlerBFFHPSSS21}~\cite{KoehlerBFFHPSSS21}, \href{../works/HanenKP21.pdf}{HanenKP21}~\cite{HanenKP21}, \href{../works/HamPK21.pdf}{HamPK21}~\cite{HamPK21}, \href{../works/GeibingerMM21.pdf}{GeibingerMM21}~\cite{GeibingerMM21}, \href{../works/Astrand21.pdf}{Astrand21}~\cite{Astrand21}, \href{../works/QinWSLS21.pdf}{QinWSLS21}~\cite{QinWSLS21}, \href{../works/HubnerGSV21.pdf}{HubnerGSV21}~\cite{HubnerGSV21}, \href{../works/Bedhief21.pdf}{Bedhief21}~\cite{Bedhief21}, \href{../works/QinDCS20.pdf}{QinDCS20}~\cite{QinDCS20}, \href{../works/MejiaY20.pdf}{MejiaY20}~\cite{MejiaY20}, \href{../works/LunardiBLRV20.pdf}{LunardiBLRV20}~\cite{LunardiBLRV20}, \href{../works/Polo-MejiaALB20.pdf}{Polo-MejiaALB20}~\cite{Polo-MejiaALB20}, \href{../works/Tom19.pdf}{Tom19}~\cite{Tom19}, \href{../works/Novas19.pdf}{Novas19}~\cite{Novas19}, \href{../works/RiahiNS018.pdf}{RiahiNS018}~\cite{RiahiNS018}, \href{../works/ZhangW18.pdf}{ZhangW18}~\cite{ZhangW18}, \href{../works/KreterSSZ18.pdf}{KreterSSZ18}~\cite{KreterSSZ18}, \href{../works/Ham18a.pdf}{Ham18a}~\cite{Ham18a}, \href{../works/RoshanaeiLAU17.pdf}{RoshanaeiLAU17}~\cite{RoshanaeiLAU17}, \href{../works/HookerH17.pdf}{HookerH17}~\cite{HookerH17}... (Total: 75)\\
Concepts & task & \href{../works/PrataAN23.pdf}{PrataAN23}~\cite{PrataAN23}, \href{../works/ForbesHJST24.pdf}{ForbesHJST24}~\cite{ForbesHJST24}, \href{../works/BonninMNE24.pdf}{BonninMNE24}~\cite{BonninMNE24}, \href{../works/abs-2402-00459.pdf}{abs-2402-00459}~\cite{abs-2402-00459}, \href{../works/JuvinHHL23.pdf}{JuvinHHL23}~\cite{JuvinHHL23}, \href{../works/WangB23.pdf}{WangB23}~\cite{WangB23}, \href{../works/YuraszeckMCCR23.pdf}{YuraszeckMCCR23}~\cite{YuraszeckMCCR23}, \href{../works/PovedaAA23.pdf}{PovedaAA23}~\cite{PovedaAA23}, \href{../works/AfsarVPG23.pdf}{AfsarVPG23}~\cite{AfsarVPG23}, \href{../works/KameugneFND23.pdf}{KameugneFND23}~\cite{KameugneFND23}, \href{../works/GokPTGO23.pdf}{GokPTGO23}~\cite{GokPTGO23}, \href{../works/AkramNHRSA23.pdf}{AkramNHRSA23}~\cite{AkramNHRSA23}, \href{../works/JuvinHL23.pdf}{JuvinHL23}~\cite{JuvinHL23}, \href{../works/CzerniachowskaWZ23.pdf}{CzerniachowskaWZ23}~\cite{CzerniachowskaWZ23}, \href{../works/Fatemi-AnarakiTFV23.pdf}{Fatemi-AnarakiTFV23}~\cite{Fatemi-AnarakiTFV23}, \href{../works/Adelgren2023.pdf}{Adelgren2023}~\cite{Adelgren2023}, \href{../works/abs-2305-19888.pdf}{abs-2305-19888}~\cite{abs-2305-19888}, \href{../works/NaderiBZ22a.pdf}{NaderiBZ22a}~\cite{NaderiBZ22a}, \href{../works/LiFJZLL22.pdf}{LiFJZLL22}~\cite{LiFJZLL22}, \href{../works/CampeauG22.pdf}{CampeauG22}~\cite{CampeauG22}, \href{../works/OuelletQ22.pdf}{OuelletQ22}~\cite{OuelletQ22}, \href{../works/GeitzGSSW22.pdf}{GeitzGSSW22}~\cite{GeitzGSSW22}, \href{../works/HeinzNVH22.pdf}{HeinzNVH22}~\cite{HeinzNVH22}, \href{../works/ColT22.pdf}{ColT22}~\cite{ColT22}, \href{../works/SubulanC22.pdf}{SubulanC22}~\cite{SubulanC22}, \href{../works/FetgoD22.pdf}{FetgoD22}~\cite{FetgoD22}, \href{../works/JuvinHL22.pdf}{JuvinHL22}~\cite{JuvinHL22}, \href{../works/abs-2211-14492.pdf}{abs-2211-14492}~\cite{abs-2211-14492}, \href{../works/ElciOH22.pdf}{ElciOH22}~\cite{ElciOH22}... (Total: 273) & \href{../works/JuvinHL23a.pdf}{JuvinHL23a}~\cite{JuvinHL23a}, \href{../works/MontemanniD23a.pdf}{MontemanniD23a}~\cite{MontemanniD23a}, \href{../works/Bit-Monnot23.pdf}{Bit-Monnot23}~\cite{Bit-Monnot23}, \href{../works/IsikYA23.pdf}{IsikYA23}~\cite{IsikYA23}, \href{../works/MontemanniD23.pdf}{MontemanniD23}~\cite{MontemanniD23}, \href{../works/SquillaciPR23.pdf}{SquillaciPR23}~\cite{SquillaciPR23}, \href{../works/LacknerMMWW23.pdf}{LacknerMMWW23}~\cite{LacknerMMWW23}, \href{../works/ShaikhK23.pdf}{ShaikhK23}~\cite{ShaikhK23}, \href{../works/WinterMMW22.pdf}{WinterMMW22}~\cite{WinterMMW22}, \href{../works/FarsiTM22.pdf}{FarsiTM22}~\cite{FarsiTM22}, \href{../works/OujanaAYB22.pdf}{OujanaAYB22}~\cite{OujanaAYB22}, \href{../works/YuraszeckMPV22.pdf}{YuraszeckMPV22}~\cite{YuraszeckMPV22}, \href{../works/PopovicCGNC22.pdf}{PopovicCGNC22}~\cite{PopovicCGNC22}, \href{../works/MullerMKP22.pdf}{MullerMKP22}~\cite{MullerMKP22}, \href{../works/AbreuN22.pdf}{AbreuN22}~\cite{AbreuN22}, \href{../works/SvancaraB22.pdf}{SvancaraB22}~\cite{SvancaraB22}, \href{../works/HubnerGSV21.pdf}{HubnerGSV21}~\cite{HubnerGSV21}, \href{../works/BenderWS21.pdf}{BenderWS21}~\cite{BenderWS21}, \href{../works/GeibingerMM21.pdf}{GeibingerMM21}~\cite{GeibingerMM21}, \href{../works/ZouZ20.pdf}{ZouZ20}~\cite{ZouZ20}, \href{../works/Polo-MejiaALB20.pdf}{Polo-MejiaALB20}~\cite{Polo-MejiaALB20}, \href{../works/AntuoriHHEN20.pdf}{AntuoriHHEN20}~\cite{AntuoriHHEN20}, \href{../works/BadicaBI20.pdf}{BadicaBI20}~\cite{BadicaBI20}, \href{../works/BarzegaranZP20.pdf}{BarzegaranZP20}~\cite{BarzegaranZP20}, \href{../works/WallaceY20.pdf}{WallaceY20}~\cite{WallaceY20}, \href{../works/WikarekS19.pdf}{WikarekS19}~\cite{WikarekS19}, \href{../works/Caballero19.pdf}{Caballero19}~\cite{Caballero19}, \href{../works/German18.pdf}{German18}~\cite{German18}, \href{../works/DemirovicS18.pdf}{DemirovicS18}~\cite{DemirovicS18}... (Total: 63) & \href{../works/ZhuSZW23.pdf}{ZhuSZW23}~\cite{ZhuSZW23}, \href{../works/TardivoDFMP23.pdf}{TardivoDFMP23}~\cite{TardivoDFMP23}, \href{../works/abs-2306-05747.pdf}{abs-2306-05747}~\cite{abs-2306-05747}, \href{../works/MarliereSPR23.pdf}{MarliereSPR23}~\cite{MarliereSPR23}, \href{../works/NaderiRR23.pdf}{NaderiRR23}~\cite{NaderiRR23}, \href{../works/TasselGS23.pdf}{TasselGS23}~\cite{TasselGS23}, \href{../works/EfthymiouY23.pdf}{EfthymiouY23}~\cite{EfthymiouY23}, \href{../works/PerezGSL23.pdf}{PerezGSL23}~\cite{PerezGSL23}, \href{../works/abs-2312-13682.pdf}{abs-2312-13682}~\cite{abs-2312-13682}, \href{../works/Mehdizadeh-Somarin23.pdf}{Mehdizadeh-Somarin23}~\cite{Mehdizadeh-Somarin23}, \href{../works/GuoZ23.pdf}{GuoZ23}~\cite{GuoZ23}, \href{../works/ZhangJZL22.pdf}{ZhangJZL22}~\cite{ZhangJZL22}, \href{../works/ZhangBB22.pdf}{ZhangBB22}~\cite{ZhangBB22}, \href{../works/EmdeZD22.pdf}{EmdeZD22}~\cite{EmdeZD22}, \href{../works/Teppan22.pdf}{Teppan22}~\cite{Teppan22}, \href{../works/ArmstrongGOS22.pdf}{ArmstrongGOS22}~\cite{ArmstrongGOS22}, \href{../works/abs-2102-08778.pdf}{abs-2102-08778}~\cite{abs-2102-08778}, \href{../works/AntuoriHHEN21.pdf}{AntuoriHHEN21}~\cite{AntuoriHHEN21}, \href{../works/ZhangYW21.pdf}{ZhangYW21}~\cite{ZhangYW21}, \href{../works/FanXG21.pdf}{FanXG21}~\cite{FanXG21}, \href{../works/AbreuAPNM21.pdf}{AbreuAPNM21}~\cite{AbreuAPNM21}, \href{../works/LacknerMMWW21.pdf}{LacknerMMWW21}~\cite{LacknerMMWW21}, \href{../works/HamPK21.pdf}{HamPK21}~\cite{HamPK21}, \href{../works/AstrandJZ20.pdf}{AstrandJZ20}~\cite{AstrandJZ20}, \href{../works/SacramentoSP20.pdf}{SacramentoSP20}~\cite{SacramentoSP20}, \href{../works/BenediktMH20.pdf}{BenediktMH20}~\cite{BenediktMH20}, \href{../works/HauderBRPA20.pdf}{HauderBRPA20}~\cite{HauderBRPA20}, \href{../works/FallahiAC20.pdf}{FallahiAC20}~\cite{FallahiAC20}, \href{../works/MengZRZL20.pdf}{MengZRZL20}~\cite{MengZRZL20}... (Total: 112)\\
Concepts & temporal constraint reasoning &  &  & \href{../works/BartakSR10.pdf}{BartakSR10}~\cite{BartakSR10}, \href{../works/KeriK07.pdf}{KeriK07}~\cite{KeriK07}, \href{../works/FortinZDF05.pdf}{FortinZDF05}~\cite{FortinZDF05}\\
Concepts & transportation & \href{../works/MarliereSPR23.pdf}{MarliereSPR23}~\cite{MarliereSPR23}, \href{../works/GuoZ23.pdf}{GuoZ23}~\cite{GuoZ23}, \href{../works/CzerniachowskaWZ23.pdf}{CzerniachowskaWZ23}~\cite{CzerniachowskaWZ23}, \href{../works/PohlAK22.pdf}{PohlAK22}~\cite{PohlAK22}, \href{../works/BourreauGGLT22.pdf}{BourreauGGLT22}~\cite{BourreauGGLT22}, \href{../works/ArmstrongGOS22.pdf}{ArmstrongGOS22}~\cite{ArmstrongGOS22}, \href{../works/EmdeZD22.pdf}{EmdeZD22}~\cite{EmdeZD22}, \href{../works/GeitzGSSW22.pdf}{GeitzGSSW22}~\cite{GeitzGSSW22}, \href{../works/Lemos21.pdf}{Lemos21}~\cite{Lemos21}, \href{../works/ArmstrongGOS21.pdf}{ArmstrongGOS21}~\cite{ArmstrongGOS21}, \href{../works/ThomasKS20.pdf}{ThomasKS20}~\cite{ThomasKS20}, \href{../works/QinDCS20.pdf}{QinDCS20}~\cite{QinDCS20}, \href{../works/Lunardi20.pdf}{Lunardi20}~\cite{Lunardi20}, \href{../works/SacramentoSP20.pdf}{SacramentoSP20}~\cite{SacramentoSP20}, \href{../works/MurinR19.pdf}{MurinR19}~\cite{MurinR19}, \href{../works/Hooker19.pdf}{Hooker19}~\cite{Hooker19}, \href{../works/Ham18.pdf}{Ham18}~\cite{Ham18}, \href{../works/PourDERB18.pdf}{PourDERB18}~\cite{PourDERB18}, \href{../works/TangLWSK18.pdf}{TangLWSK18}~\cite{TangLWSK18}, \href{../works/CappartTSR18.pdf}{CappartTSR18}~\cite{CappartTSR18}, \href{../works/Froger16.pdf}{Froger16}~\cite{Froger16}, \href{../works/GoelSHFS15.pdf}{GoelSHFS15}~\cite{GoelSHFS15}, \href{../works/NovasH14.pdf}{NovasH14}~\cite{NovasH14}, \href{../works/BlomBPS14.pdf}{BlomBPS14}~\cite{BlomBPS14}, \href{../works/KelarevaTK13.pdf}{KelarevaTK13}~\cite{KelarevaTK13}, \href{../works/NovasH12.pdf}{NovasH12}~\cite{NovasH12}, \href{../works/HachemiGR11.pdf}{HachemiGR11}~\cite{HachemiGR11}, \href{../works/LopesCSM10.pdf}{LopesCSM10}~\cite{LopesCSM10}, \href{../works/BocewiczBB09.pdf}{BocewiczBB09}~\cite{BocewiczBB09}... (Total: 35) & \href{../works/AfsarVPG23.pdf}{AfsarVPG23}~\cite{AfsarVPG23}, \href{../works/KimCMLLP23.pdf}{KimCMLLP23}~\cite{KimCMLLP23}, \href{../works/Fatemi-AnarakiTFV23.pdf}{Fatemi-AnarakiTFV23}~\cite{Fatemi-AnarakiTFV23}, \href{../works/NaderiRR23.pdf}{NaderiRR23}~\cite{NaderiRR23}, \href{../works/GokPTGO23.pdf}{GokPTGO23}~\cite{GokPTGO23}, \href{../works/AbreuPNF23.pdf}{AbreuPNF23}~\cite{AbreuPNF23}, \href{../works/AbreuN22.pdf}{AbreuN22}~\cite{AbreuN22}, \href{../works/SubulanC22.pdf}{SubulanC22}~\cite{SubulanC22}, \href{../works/PopovicCGNC22.pdf}{PopovicCGNC22}~\cite{PopovicCGNC22}, \href{../works/NaderiBZ22.pdf}{NaderiBZ22}~\cite{NaderiBZ22}, \href{../works/ElciOH22.pdf}{ElciOH22}~\cite{ElciOH22}, \href{../works/Astrand21.pdf}{Astrand21}~\cite{Astrand21}, \href{../works/Godet21a.pdf}{Godet21a}~\cite{Godet21a}, \href{../works/AbohashimaEG21.pdf}{AbohashimaEG21}~\cite{AbohashimaEG21}, \href{../works/FallahiAC20.pdf}{FallahiAC20}~\cite{FallahiAC20}, \href{../works/MengZRZL20.pdf}{MengZRZL20}~\cite{MengZRZL20}, \href{../works/MejiaY20.pdf}{MejiaY20}~\cite{MejiaY20}, \href{../works/ZarandiASC20.pdf}{ZarandiASC20}~\cite{ZarandiASC20}, \href{../works/LaborieRSV18.pdf}{LaborieRSV18}~\cite{LaborieRSV18}, \href{../works/EvenSH15.pdf}{EvenSH15}~\cite{EvenSH15}, \href{../works/MelgarejoLS15.pdf}{MelgarejoLS15}~\cite{MelgarejoLS15}, \href{../works/HarjunkoskiMBC14.pdf}{HarjunkoskiMBC14}~\cite{HarjunkoskiMBC14}, \href{../works/RendlPHPR12.pdf}{RendlPHPR12}~\cite{RendlPHPR12}, \href{../works/Malapert11.pdf}{Malapert11}~\cite{Malapert11}, \href{../works/MakMS10.pdf}{MakMS10}~\cite{MakMS10}, \href{../works/MouraSCL08.pdf}{MouraSCL08}~\cite{MouraSCL08}, \href{../works/MouraSCL08a.pdf}{MouraSCL08a}~\cite{MouraSCL08a}, \href{../works/LimRX04.pdf}{LimRX04}~\cite{LimRX04}, \href{../works/Mason01.pdf}{Mason01}~\cite{Mason01}... (Total: 32) & \href{../works/Adelgren2023.pdf}{Adelgren2023}~\cite{Adelgren2023}, \href{../works/AalianPG23.pdf}{AalianPG23}~\cite{AalianPG23}, \href{../works/PerezGSL23.pdf}{PerezGSL23}~\cite{PerezGSL23}, \href{../works/AlfieriGPS23.pdf}{AlfieriGPS23}~\cite{AlfieriGPS23}, \href{../works/ZhuSZW23.pdf}{ZhuSZW23}~\cite{ZhuSZW23}, \href{../works/IsikYA23.pdf}{IsikYA23}~\cite{IsikYA23}, \href{../works/AbreuNP23.pdf}{AbreuNP23}~\cite{AbreuNP23}, \href{../works/abs-2312-13682.pdf}{abs-2312-13682}~\cite{abs-2312-13682}, \href{../works/WangB23.pdf}{WangB23}~\cite{WangB23}, \href{../works/MontemanniD23a.pdf}{MontemanniD23a}~\cite{MontemanniD23a}, \href{../works/NaderiBZ22a.pdf}{NaderiBZ22a}~\cite{NaderiBZ22a}, \href{../works/BoudreaultSLQ22.pdf}{BoudreaultSLQ22}~\cite{BoudreaultSLQ22}, \href{../works/abs-2211-14492.pdf}{abs-2211-14492}~\cite{abs-2211-14492}, \href{../works/ZhangJZL22.pdf}{ZhangJZL22}~\cite{ZhangJZL22}, \href{../works/YuraszeckMPV22.pdf}{YuraszeckMPV22}~\cite{YuraszeckMPV22}, \href{../works/LiFJZLL22.pdf}{LiFJZLL22}~\cite{LiFJZLL22}, \href{../works/ColT22.pdf}{ColT22}~\cite{ColT22}, \href{../works/YunusogluY22.pdf}{YunusogluY22}~\cite{YunusogluY22}, \href{../works/AntuoriHHEN21.pdf}{AntuoriHHEN21}~\cite{AntuoriHHEN21}, \href{../works/HubnerGSV21.pdf}{HubnerGSV21}~\cite{HubnerGSV21}, \href{../works/Bedhief21.pdf}{Bedhief21}~\cite{Bedhief21}, \href{../works/Groleaz21.pdf}{Groleaz21}~\cite{Groleaz21}, \href{../works/GroleazNS20a.pdf}{GroleazNS20a}~\cite{GroleazNS20a}, \href{../works/AntunesABD20.pdf}{AntunesABD20}~\cite{AntunesABD20}, \href{../works/WallaceY20.pdf}{WallaceY20}~\cite{WallaceY20}, \href{../works/HauderBRPA20.pdf}{HauderBRPA20}~\cite{HauderBRPA20}, \href{../works/CauwelaertDS20.pdf}{CauwelaertDS20}~\cite{CauwelaertDS20}, \href{../works/Novas19.pdf}{Novas19}~\cite{Novas19}, \href{../works/HoundjiSW19.pdf}{HoundjiSW19}~\cite{HoundjiSW19}... (Total: 90)\\
Concepts & two-machine scheduling &  &  & \href{../works/AbreuNP23.pdf}{AbreuNP23}~\cite{AbreuNP23}\\
Concepts & two-stage scheduling &  &  & \href{../works/Astrand21.pdf}{Astrand21}~\cite{Astrand21}, \href{../works/QinWSLS21.pdf}{QinWSLS21}~\cite{QinWSLS21}, \href{../works/ZarandiASC20.pdf}{ZarandiASC20}~\cite{ZarandiASC20}, \href{../works/ZouZ20.pdf}{ZouZ20}~\cite{ZouZ20}, \href{../works/TangB20.pdf}{TangB20}~\cite{TangB20}\\
Concepts & unavailability & \href{../works/Lemos21.pdf}{Lemos21}~\cite{Lemos21}, \href{../works/Astrand21.pdf}{Astrand21}~\cite{Astrand21}, \href{../works/LunardiBLRV20.pdf}{LunardiBLRV20}~\cite{LunardiBLRV20}, \href{../works/Lunardi20.pdf}{Lunardi20}~\cite{Lunardi20}, \href{../works/ZhangW18.pdf}{ZhangW18}~\cite{ZhangW18}, \href{../works/Froger16.pdf}{Froger16}~\cite{Froger16}, \href{../works/BajestaniB15.pdf}{BajestaniB15}~\cite{BajestaniB15}, \href{../works/AkkerDH07.pdf}{AkkerDH07}~\cite{AkkerDH07}, \href{../works/KhemmoudjPB06.pdf}{KhemmoudjPB06}~\cite{KhemmoudjPB06} & \href{../works/Mehdizadeh-Somarin23.pdf}{Mehdizadeh-Somarin23}~\cite{Mehdizadeh-Somarin23}, \href{../works/PenzDN23.pdf}{PenzDN23}~\cite{PenzDN23}, \href{../works/TouatBT22.pdf}{TouatBT22}~\cite{TouatBT22}, \href{../works/KovacsTKSG21.pdf}{KovacsTKSG21}~\cite{KovacsTKSG21}, \href{../works/SerraNM12.pdf}{SerraNM12}~\cite{SerraNM12}, \href{../works/LorigeonBB02.pdf}{LorigeonBB02}~\cite{LorigeonBB02} & \href{../works/WangB23.pdf}{WangB23}~\cite{WangB23}, \href{../works/PovedaAA23.pdf}{PovedaAA23}~\cite{PovedaAA23}, \href{../works/GuoZ23.pdf}{GuoZ23}~\cite{GuoZ23}, \href{../works/abs-2305-19888.pdf}{abs-2305-19888}~\cite{abs-2305-19888}, \href{../works/ShaikhK23.pdf}{ShaikhK23}~\cite{ShaikhK23}, \href{../works/YunusogluY22.pdf}{YunusogluY22}~\cite{YunusogluY22}, \href{../works/HeinzNVH22.pdf}{HeinzNVH22}~\cite{HeinzNVH22}, \href{../works/FanXG21.pdf}{FanXG21}~\cite{FanXG21}, \href{../works/PandeyS21a.pdf}{PandeyS21a}~\cite{PandeyS21a}, \href{../works/WangB20.pdf}{WangB20}~\cite{WangB20}, \href{../works/AstrandJZ20.pdf}{AstrandJZ20}~\cite{AstrandJZ20}, \href{../works/KreterSSZ18.pdf}{KreterSSZ18}~\cite{KreterSSZ18}, \href{../works/ArbaouiY18.pdf}{ArbaouiY18}~\cite{ArbaouiY18}, \href{../works/TranVNB17.pdf}{TranVNB17}~\cite{TranVNB17}, \href{../works/KreterSS17.pdf}{KreterSS17}~\cite{KreterSS17}, \href{../works/BurtLPS15.pdf}{BurtLPS15}~\cite{BurtLPS15}, \href{../works/KreterSS15.pdf}{KreterSS15}~\cite{KreterSS15}, \href{../works/GoelSHFS15.pdf}{GoelSHFS15}~\cite{GoelSHFS15}, \href{../works/NovasH14.pdf}{NovasH14}~\cite{NovasH14}, \href{../works/HarjunkoskiMBC14.pdf}{HarjunkoskiMBC14}~\cite{HarjunkoskiMBC14}, \href{../works/GuyonLPR12.pdf}{GuyonLPR12}~\cite{GuyonLPR12}, \href{../works/NovasH10.pdf}{NovasH10}~\cite{NovasH10}\\
\end{longtable}
}


\clearpage
\subsection{Concept Type Classification}
\label{sec:Classification}
{\scriptsize
\begin{longtable}{lp{3cm}>{\raggedright\arraybackslash}p{6cm}>{\raggedright\arraybackslash}p{6cm}>{\raggedright\arraybackslash}p{8cm}}
\rowcolor{white}\caption{Works for Concepts of Type Classification}\\ \toprule
\rowcolor{white}Type & Keyword & High & Medium & Low\\ \midrule\endhead
\bottomrule
\endfoot
Classification & 2BPHFSP & \href{../works/TangB20.pdf}{TangB20}~\cite{TangB20} &  & \\
Classification & BPCTOP & \href{../works/KelarevaTK13.pdf}{KelarevaTK13}~\cite{KelarevaTK13} &  & \\
Classification & Bulk Port Cargo Throughput Optimisation Problem &  &  & \href{../works/KelarevaTK13.pdf}{KelarevaTK13}~\cite{KelarevaTK13}\\
Classification & CECSP & \href{../works/NattafHKAL19.pdf}{NattafHKAL19}~\cite{NattafHKAL19}, \href{../works/NattafAL17.pdf}{NattafAL17}~\cite{NattafAL17}, \href{../works/Nattaf16.pdf}{Nattaf16}~\cite{Nattaf16}, \href{../works/NattafALR16.pdf}{NattafALR16}~\cite{NattafALR16}, \href{../works/NattafAL15.pdf}{NattafAL15}~\cite{NattafAL15} &  & \\
Classification & CHSP & \href{../works/EfthymiouY23.pdf}{EfthymiouY23}~\cite{EfthymiouY23}, \href{../works/WallaceY20.pdf}{WallaceY20}~\cite{WallaceY20} &  & \\
Classification & CTW & \href{../works/KoehlerBFFHPSSS21.pdf}{KoehlerBFFHPSSS21}~\cite{KoehlerBFFHPSSS21} & \href{../works/Lombardi10.pdf}{Lombardi10}~\cite{Lombardi10} & \\
Classification & CuSP & \href{../works/KameugneFND23.pdf}{KameugneFND23}~\cite{KameugneFND23}, \href{../works/FetgoD22.pdf}{FetgoD22}~\cite{FetgoD22}, \href{../works/Tesch18.pdf}{Tesch18}~\cite{Tesch18}, \href{../works/KameugneFGOQ18.pdf}{KameugneFGOQ18}~\cite{KameugneFGOQ18}, \href{../works/Tesch16.pdf}{Tesch16}~\cite{Tesch16}, \href{../works/NattafALR16.pdf}{NattafALR16}~\cite{NattafALR16}, \href{../works/Nattaf16.pdf}{Nattaf16}~\cite{Nattaf16}, \href{../works/Froger16.pdf}{Froger16}~\cite{Froger16}, \href{../works/NattafAL15.pdf}{NattafAL15}~\cite{NattafAL15}, \href{../works/Derrien15.pdf}{Derrien15}~\cite{Derrien15}, \href{../works/Kameugne14.pdf}{Kameugne14}~\cite{Kameugne14}, \href{../works/KameugneFSN14.pdf}{KameugneFSN14}~\cite{KameugneFSN14}, \href{../works/DerrienPZ14.pdf}{DerrienPZ14}~\cite{DerrienPZ14}, \href{../works/KameugneFSN11.pdf}{KameugneFSN11}~\cite{KameugneFSN11}, \href{../works/SchuttW10.pdf}{SchuttW10}~\cite{SchuttW10}, \href{../works/Demassey03.pdf}{Demassey03}~\cite{Demassey03}, \href{../works/BaptistePN99.pdf}{BaptistePN99}~\cite{BaptistePN99} & \href{../works/Fahimi16.pdf}{Fahimi16}~\cite{Fahimi16}, \href{../works/GingrasQ16.pdf}{GingrasQ16}~\cite{GingrasQ16}, \href{../works/OuelletQ13.pdf}{OuelletQ13}~\cite{OuelletQ13}, \href{../works/Elkhyari03.pdf}{Elkhyari03}~\cite{Elkhyari03} & \href{../works/TardivoDFMP23.pdf}{TardivoDFMP23}~\cite{TardivoDFMP23}, \href{../works/HanenKP21.pdf}{HanenKP21}~\cite{HanenKP21}, \href{../works/Zahout21.pdf}{Zahout21}~\cite{Zahout21}, \href{../works/DerrienP14.pdf}{DerrienP14}~\cite{DerrienP14}\\
Classification & EOSP &  & \href{../works/SquillaciPR23.pdf}{SquillaciPR23}~\cite{SquillaciPR23} & \\
Classification & Earth Observation Scheduling Problem &  & \href{../works/SquillaciPR23.pdf}{SquillaciPR23}~\cite{SquillaciPR23} & \\
Classification & FJS & \href{../works/JuvinHL23a.pdf}{JuvinHL23a}~\cite{JuvinHL23a}, \href{../works/WangB23.pdf}{WangB23}~\cite{WangB23}, \href{../works/YuraszeckMCCR23.pdf}{YuraszeckMCCR23}~\cite{YuraszeckMCCR23}, \href{../works/JuvinHL22.pdf}{JuvinHL22}~\cite{JuvinHL22}, \href{../works/MullerMKP22.pdf}{MullerMKP22}~\cite{MullerMKP22}, \href{../works/Teppan22.pdf}{Teppan22}~\cite{Teppan22}, \href{../works/HamPK21.pdf}{HamPK21}~\cite{HamPK21}, \href{../works/WangB20.pdf}{WangB20}~\cite{WangB20}, \href{../works/Lunardi20.pdf}{Lunardi20}~\cite{Lunardi20}, \href{../works/LunardiBLRV20.pdf}{LunardiBLRV20}~\cite{LunardiBLRV20}, \href{../works/ZarandiASC20.pdf}{ZarandiASC20}~\cite{ZarandiASC20}, \href{../works/MengZRZL20.pdf}{MengZRZL20}~\cite{MengZRZL20}, \href{../works/Novas19.pdf}{Novas19}~\cite{Novas19}, \href{../works/MossigeGSMC17.pdf}{MossigeGSMC17}~\cite{MossigeGSMC17}, \href{../works/HamC16.pdf}{HamC16}~\cite{HamC16} & \href{../works/OujanaAYB22.pdf}{OujanaAYB22}~\cite{OujanaAYB22}, \href{../works/HauderBRPA20.pdf}{HauderBRPA20}~\cite{HauderBRPA20}, \href{../works/abs-1902-09244.pdf}{abs-1902-09244}~\cite{abs-1902-09244}, \href{../works/ZhangW18.pdf}{ZhangW18}~\cite{ZhangW18}, \href{../works/SchuttFS13.pdf}{SchuttFS13}~\cite{SchuttFS13} & \href{../works/NaderiRR23.pdf}{NaderiRR23}~\cite{NaderiRR23}, \href{../works/ColT22.pdf}{ColT22}~\cite{ColT22}, \href{../works/ZhouGL15.pdf}{ZhouGL15}~\cite{ZhouGL15}\\
Classification & Fixed Job Scheduling & \href{../works/WangB20.pdf}{WangB20}~\cite{WangB20} & \href{../works/WangB23.pdf}{WangB23}~\cite{WangB23} & \\
Classification & GCSP & \href{../works/Groleaz21.pdf}{Groleaz21}~\cite{Groleaz21}, \href{../works/GroleazNS20.pdf}{GroleazNS20}~\cite{GroleazNS20} &  & \\
Classification & HFF & \href{../works/ArmstrongGOS22.pdf}{ArmstrongGOS22}~\cite{ArmstrongGOS22}, \href{../works/OujanaAYB22.pdf}{OujanaAYB22}~\cite{OujanaAYB22}, \href{../works/ArmstrongGOS21.pdf}{ArmstrongGOS21}~\cite{ArmstrongGOS21}, \href{../works/ZhouGL15.pdf}{ZhouGL15}~\cite{ZhouGL15} &  & \\
Classification & HFFTT & \href{../works/ArmstrongGOS22.pdf}{ArmstrongGOS22}~\cite{ArmstrongGOS22}, \href{../works/ArmstrongGOS21.pdf}{ArmstrongGOS21}~\cite{ArmstrongGOS21} &  & \\
Classification & HFS & \href{../works/IsikYA23.pdf}{IsikYA23}~\cite{IsikYA23}, \href{../works/ZhangJZL22.pdf}{ZhangJZL22}~\cite{ZhangJZL22}, \href{../works/Astrand21.pdf}{Astrand21}~\cite{Astrand21}, \href{../works/ArmstrongGOS21.pdf}{ArmstrongGOS21}~\cite{ArmstrongGOS21}, \href{../works/Bedhief21.pdf}{Bedhief21}~\cite{Bedhief21}, \href{../works/TangB20.pdf}{TangB20}~\cite{TangB20}, \href{../works/MengZRZL20.pdf}{MengZRZL20}~\cite{MengZRZL20}, \href{../works/Baptiste02.pdf}{Baptiste02}~\cite{Baptiste02} &  & \href{../works/ArmstrongGOS22.pdf}{ArmstrongGOS22}~\cite{ArmstrongGOS22}, \href{../works/ZarandiASC20.pdf}{ZarandiASC20}~\cite{ZarandiASC20}, \href{../works/Novas19.pdf}{Novas19}~\cite{Novas19}, \href{../works/ZhouGL15.pdf}{ZhouGL15}~\cite{ZhouGL15}\\
Classification & JSPT &  & \href{../works/MurinR19.pdf}{MurinR19}~\cite{MurinR19} & \\
Classification & JSSP & \href{../works/TasselGS23.pdf}{TasselGS23}~\cite{TasselGS23}, \href{../works/JuvinHL23a.pdf}{JuvinHL23a}~\cite{JuvinHL23a}, \href{../works/JuvinHHL23.pdf}{JuvinHHL23}~\cite{JuvinHHL23}, \href{../works/YuraszeckMC23.pdf}{YuraszeckMC23}~\cite{YuraszeckMC23}, \href{../works/YuraszeckMCCR23.pdf}{YuraszeckMCCR23}~\cite{YuraszeckMCCR23}, \href{../works/abs-2306-05747.pdf}{abs-2306-05747}~\cite{abs-2306-05747}, \href{../works/JuvinHL22.pdf}{JuvinHL22}~\cite{JuvinHL22}, \href{../works/Teppan22.pdf}{Teppan22}~\cite{Teppan22}, \href{../works/ColT22.pdf}{ColT22}~\cite{ColT22}, \href{../works/YuraszeckMPV22.pdf}{YuraszeckMPV22}~\cite{YuraszeckMPV22}, \href{../works/GeitzGSSW22.pdf}{GeitzGSSW22}~\cite{GeitzGSSW22}, \href{../works/Godet21a.pdf}{Godet21a}~\cite{Godet21a}, \href{../works/abs-2102-08778.pdf}{abs-2102-08778}~\cite{abs-2102-08778}, \href{../works/ZarandiASC20.pdf}{ZarandiASC20}~\cite{ZarandiASC20}, \href{../works/ColT19.pdf}{ColT19}~\cite{ColT19}, \href{../works/Pralet17.pdf}{Pralet17}~\cite{Pralet17}, \href{../works/MenciaSV13.pdf}{MenciaSV13}~\cite{MenciaSV13}, \href{../works/MenciaSV12.pdf}{MenciaSV12}~\cite{MenciaSV12}, \href{../works/KelbelH11.pdf}{KelbelH11}~\cite{KelbelH11}, \href{../works/BidotVLB09.pdf}{BidotVLB09}~\cite{BidotVLB09}, \href{../works/GodardLN05.pdf}{GodardLN05}~\cite{GodardLN05}, \href{../works/Baptiste02.pdf}{Baptiste02}~\cite{Baptiste02}, \href{../works/SourdN00.pdf}{SourdN00}~\cite{SourdN00}, \href{../works/TorresL00.pdf}{TorresL00}~\cite{TorresL00}, \href{../works/PapaB98.pdf}{PapaB98}~\cite{PapaB98}, \href{../works/NuijtenP98.pdf}{NuijtenP98}~\cite{NuijtenP98}, \href{../works/NuijtenA96.pdf}{NuijtenA96}~\cite{NuijtenA96}, \href{../works/NuijtenA94.pdf}{NuijtenA94}~\cite{NuijtenA94} & \href{../works/GalleguillosKSB19.pdf}{GalleguillosKSB19}~\cite{GalleguillosKSB19}, \href{../works/LombardiBM15.pdf}{LombardiBM15}~\cite{LombardiBM15}, \href{../works/SialaAH15.pdf}{SialaAH15}~\cite{SialaAH15}, \href{../works/BelhadjiI98.pdf}{BelhadjiI98}~\cite{BelhadjiI98} & \href{../works/Mehdizadeh-Somarin23.pdf}{Mehdizadeh-Somarin23}~\cite{Mehdizadeh-Somarin23}, \href{../works/CzerniachowskaWZ23.pdf}{CzerniachowskaWZ23}~\cite{CzerniachowskaWZ23}, \href{../works/EfthymiouY23.pdf}{EfthymiouY23}~\cite{EfthymiouY23}, \href{../works/WikarekS19.pdf}{WikarekS19}~\cite{WikarekS19}, \href{../works/PraletLJ15.pdf}{PraletLJ15}~\cite{PraletLJ15}, \href{../works/GrimesH15.pdf}{GrimesH15}~\cite{GrimesH15}, \href{../works/BajestaniB11.pdf}{BajestaniB11}~\cite{BajestaniB11}, \href{../works/ChenGPSH10.pdf}{ChenGPSH10}~\cite{ChenGPSH10}\\
Classification & KRFP & \href{../works/KamarainenS02.pdf}{KamarainenS02}~\cite{KamarainenS02}, \href{../works/SakkoutW00.pdf}{SakkoutW00}~\cite{SakkoutW00} &  & \\
Classification & LSFRP & \href{../works/KelarevaTK13.pdf}{KelarevaTK13}~\cite{KelarevaTK13} &  & \\
Classification & Liner Shipping Fleet Repositioning Problem &  & \href{../works/KelarevaTK13.pdf}{KelarevaTK13}~\cite{KelarevaTK13} & \\
Classification & MGAP & \href{../works/Darby-DowmanLMZ97.pdf}{Darby-DowmanLMZ97}~\cite{Darby-DowmanLMZ97} &  & \\
Classification & OSP & \href{../works/NaderiRR23.pdf}{NaderiRR23}~\cite{NaderiRR23}, \href{../works/LacknerMMWW23.pdf}{LacknerMMWW23}~\cite{LacknerMMWW23}, \href{../works/Bit-Monnot23.pdf}{Bit-Monnot23}~\cite{Bit-Monnot23}, \href{../works/LacknerMMWW21.pdf}{LacknerMMWW21}~\cite{LacknerMMWW21}, \href{../works/Groleaz21.pdf}{Groleaz21}~\cite{Groleaz21}, \href{../works/GombolayWS18.pdf}{GombolayWS18}~\cite{GombolayWS18}, \href{../works/GrimesH15.pdf}{GrimesH15}~\cite{GrimesH15}, \href{../works/Siala15.pdf}{Siala15}~\cite{Siala15}, \href{../works/GayHLS15.pdf}{GayHLS15}~\cite{GayHLS15}, \href{../works/Siala15a.pdf}{Siala15a}~\cite{Siala15a}, \href{../works/MalapertCGJLR12.pdf}{MalapertCGJLR12}~\cite{MalapertCGJLR12} & \href{../works/SquillaciPR23.pdf}{SquillaciPR23}~\cite{SquillaciPR23}, \href{../works/GrimesHM09.pdf}{GrimesHM09}~\cite{GrimesHM09}, \href{../works/MonetteDD07.pdf}{MonetteDD07}~\cite{MonetteDD07} & \href{../works/MengZRZL20.pdf}{MengZRZL20}~\cite{MengZRZL20}\\
Classification & OSSP & \href{../works/YuraszeckMC23.pdf}{YuraszeckMC23}~\cite{YuraszeckMC23}, \href{../works/AbreuPNF23.pdf}{AbreuPNF23}~\cite{AbreuPNF23}, \href{../works/AbreuNP23.pdf}{AbreuNP23}~\cite{AbreuNP23}, \href{../works/YuraszeckMPV22.pdf}{YuraszeckMPV22}~\cite{YuraszeckMPV22}, \href{../works/ColT22.pdf}{ColT22}~\cite{ColT22}, \href{../works/AbreuN22.pdf}{AbreuN22}~\cite{AbreuN22}, \href{../works/AbreuAPNM21.pdf}{AbreuAPNM21}~\cite{AbreuAPNM21}, \href{../works/MejiaY20.pdf}{MejiaY20}~\cite{MejiaY20}, \href{../works/Baptiste02.pdf}{Baptiste02}~\cite{Baptiste02} &  & \href{../works/YuraszeckMCCR23.pdf}{YuraszeckMCCR23}~\cite{YuraszeckMCCR23}, \href{../works/ZarandiASC20.pdf}{ZarandiASC20}~\cite{ZarandiASC20}\\
Classification & Open Shop Scheduling Problem & \href{../works/AbreuPNF23.pdf}{AbreuPNF23}~\cite{AbreuPNF23}, \href{../works/AbreuNP23.pdf}{AbreuNP23}~\cite{AbreuNP23}, \href{../works/AbreuN22.pdf}{AbreuN22}~\cite{AbreuN22}, \href{../works/AbreuAPNM21.pdf}{AbreuAPNM21}~\cite{AbreuAPNM21}, \href{../works/MejiaY20.pdf}{MejiaY20}~\cite{MejiaY20}, \href{../works/ZarandiASC20.pdf}{ZarandiASC20}~\cite{ZarandiASC20} & \href{../works/Malapert11.pdf}{Malapert11}~\cite{Malapert11}, \href{../works/LorigeonBB02.pdf}{LorigeonBB02}~\cite{LorigeonBB02} & \href{../works/PrataAN23.pdf}{PrataAN23}~\cite{PrataAN23}, \href{../works/NaderiRR23.pdf}{NaderiRR23}~\cite{NaderiRR23}, \href{../works/Bit-Monnot23.pdf}{Bit-Monnot23}~\cite{Bit-Monnot23}, \href{../works/YuraszeckMCCR23.pdf}{YuraszeckMCCR23}~\cite{YuraszeckMCCR23}, \href{../works/YuraszeckMPV22.pdf}{YuraszeckMPV22}~\cite{YuraszeckMPV22}, \href{../works/ColT22.pdf}{ColT22}~\cite{ColT22}, \href{../works/Groleaz21.pdf}{Groleaz21}~\cite{Groleaz21}, \href{../works/MengZRZL20.pdf}{MengZRZL20}~\cite{MengZRZL20}, \href{../works/SacramentoSP20.pdf}{SacramentoSP20}~\cite{SacramentoSP20}, \href{../works/HookerH17.pdf}{HookerH17}~\cite{HookerH17}, \href{../works/GrimesH15.pdf}{GrimesH15}~\cite{GrimesH15}, \href{../works/MalapertCGJLR13.pdf}{MalapertCGJLR13}~\cite{MalapertCGJLR13}, \href{../works/MalapertCGJLR12.pdf}{MalapertCGJLR12}~\cite{MalapertCGJLR12}, \href{../works/Schutt11.pdf}{Schutt11}~\cite{Schutt11}, \href{../works/GrimesH10.pdf}{GrimesH10}~\cite{GrimesH10}, \href{../works/OhrimenkoSC09.pdf}{OhrimenkoSC09}~\cite{OhrimenkoSC09}, \href{../works/GrimesHM09.pdf}{GrimesHM09}~\cite{GrimesHM09}, \href{../works/MonetteDD07.pdf}{MonetteDD07}~\cite{MonetteDD07}, \href{../works/Baptiste02.pdf}{Baptiste02}~\cite{Baptiste02}, \href{../works/JussienL02.pdf}{JussienL02}~\cite{JussienL02}, \href{../works/VerfaillieL01.pdf}{VerfaillieL01}~\cite{VerfaillieL01}\\
Classification & PJSSP & \href{../works/Baptiste02.pdf}{Baptiste02}~\cite{Baptiste02} & \href{../works/PapaB98.pdf}{PapaB98}~\cite{PapaB98} & \\
Classification & PMSP & \href{../works/NaderiRR23.pdf}{NaderiRR23}~\cite{NaderiRR23}, \href{../works/YunusogluY22.pdf}{YunusogluY22}~\cite{YunusogluY22}, \href{../works/WinterMMW22.pdf}{WinterMMW22}~\cite{WinterMMW22}, \href{../works/PandeyS21a.pdf}{PandeyS21a}~\cite{PandeyS21a}, \href{../works/Godet21a.pdf}{Godet21a}~\cite{Godet21a}, \href{../works/GodetLHS20.pdf}{GodetLHS20}~\cite{GodetLHS20}, \href{../works/MalapertN19.pdf}{MalapertN19}~\cite{MalapertN19}, \href{../works/GedikKEK18.pdf}{GedikKEK18}~\cite{GedikKEK18}, \href{../works/GomesM17.pdf}{GomesM17}~\cite{GomesM17}, \href{../works/TranAB16.pdf}{TranAB16}~\cite{TranAB16}, \href{../works/TranB12.pdf}{TranB12}~\cite{TranB12} & \href{../works/VlkHT21.pdf}{VlkHT21}~\cite{VlkHT21}, \href{../works/NattafM20.pdf}{NattafM20}~\cite{NattafM20} & \href{../works/ColT22.pdf}{ColT22}~\cite{ColT22}, \href{../works/OujanaAYB22.pdf}{OujanaAYB22}~\cite{OujanaAYB22}, \href{../works/ZarandiASC20.pdf}{ZarandiASC20}~\cite{ZarandiASC20}\\
Classification & PTC & \href{../works/NattafM20.pdf}{NattafM20}~\cite{NattafM20}, \href{../works/MalapertN19.pdf}{MalapertN19}~\cite{MalapertN19}, \href{../works/NattafDYW19.pdf}{NattafDYW19}~\cite{NattafDYW19} & \href{../works/NaderiRR23.pdf}{NaderiRR23}~\cite{NaderiRR23} & \href{../works/CzerniachowskaWZ23.pdf}{CzerniachowskaWZ23}~\cite{CzerniachowskaWZ23}, \href{../works/Teppan22.pdf}{Teppan22}~\cite{Teppan22}, \href{../works/Dejemeppe16.pdf}{Dejemeppe16}~\cite{Dejemeppe16}\\
Classification & Partial Order Schedule &  & \href{../works/LombardiBM15.pdf}{LombardiBM15}~\cite{LombardiBM15}, \href{../works/BonfiettiLM14.pdf}{BonfiettiLM14}~\cite{BonfiettiLM14} & \href{../works/Bit-Monnot23.pdf}{Bit-Monnot23}~\cite{Bit-Monnot23}, \href{../works/Astrand0F21.pdf}{Astrand0F21}~\cite{Astrand0F21}, \href{../works/Astrand21.pdf}{Astrand21}~\cite{Astrand21}, \href{../works/CappartTSR18.pdf}{CappartTSR18}~\cite{CappartTSR18}, \href{../works/BonfiettiLBM14.pdf}{BonfiettiLBM14}~\cite{BonfiettiLBM14}, \href{../works/LaborieR14.pdf}{LaborieR14}~\cite{LaborieR14}, \href{../works/GaySS14.pdf}{GaySS14}~\cite{GaySS14}, \href{../works/LombardiM12.pdf}{LombardiM12}~\cite{LombardiM12}, \href{../works/LombardiM12a.pdf}{LombardiM12a}~\cite{LombardiM12a}, \href{../works/LombardiM10.pdf}{LombardiM10}~\cite{LombardiM10}, \href{../works/CarchraeBF05.pdf}{CarchraeBF05}~\cite{CarchraeBF05}\\
Classification & RCMPSP & \href{../works/HauderBRPA20.pdf}{HauderBRPA20}~\cite{HauderBRPA20}, \href{../works/abs-1902-09244.pdf}{abs-1902-09244}~\cite{abs-1902-09244} &  & \href{../works/ArtiguesR00.pdf}{ArtiguesR00}~\cite{ArtiguesR00}\\
Classification & RCPSP & \href{../works/YuraszeckMCCR23.pdf}{YuraszeckMCCR23}~\cite{YuraszeckMCCR23}, \href{../works/GokPTGO23.pdf}{GokPTGO23}~\cite{GokPTGO23}, \href{../works/PovedaAA23.pdf}{PovedaAA23}~\cite{PovedaAA23}, \href{../works/CampeauG22.pdf}{CampeauG22}~\cite{CampeauG22}, \href{../works/BoudreaultSLQ22.pdf}{BoudreaultSLQ22}~\cite{BoudreaultSLQ22}, \href{../works/EtminaniesfahaniGNMS22.pdf}{EtminaniesfahaniGNMS22}~\cite{EtminaniesfahaniGNMS22}, \href{../works/FetgoD22.pdf}{FetgoD22}~\cite{FetgoD22}, \href{../works/SubulanC22.pdf}{SubulanC22}~\cite{SubulanC22}, \href{../works/GeibingerMM21.pdf}{GeibingerMM21}~\cite{GeibingerMM21}, \href{../works/HubnerGSV21.pdf}{HubnerGSV21}~\cite{HubnerGSV21}, \href{../works/Godet21a.pdf}{Godet21a}~\cite{Godet21a}, \href{../works/BenderWS21.pdf}{BenderWS21}~\cite{BenderWS21}, \href{../works/HillTV21.pdf}{HillTV21}~\cite{HillTV21}, \href{../works/Zahout21.pdf}{Zahout21}~\cite{Zahout21}, \href{../works/ArtiguesHQT21.pdf}{ArtiguesHQT21}~\cite{ArtiguesHQT21}, \href{../works/Groleaz21.pdf}{Groleaz21}~\cite{Groleaz21}, \href{../works/ZarandiASC20.pdf}{ZarandiASC20}~\cite{ZarandiASC20}, \href{../works/HauderBRPA20.pdf}{HauderBRPA20}~\cite{HauderBRPA20}, \href{../works/Polo-MejiaALB20.pdf}{Polo-MejiaALB20}~\cite{Polo-MejiaALB20}, \href{../works/GokGSTO20.pdf}{GokGSTO20}~\cite{GokGSTO20}, \href{../works/GeibingerMM19.pdf}{GeibingerMM19}~\cite{GeibingerMM19}, \href{../works/abs-1911-04766.pdf}{abs-1911-04766}~\cite{abs-1911-04766}, \href{../works/Caballero19.pdf}{Caballero19}~\cite{Caballero19}, \href{../works/abs-1902-09244.pdf}{abs-1902-09244}~\cite{abs-1902-09244}, \href{../works/ArkhipovBL19.pdf}{ArkhipovBL19}~\cite{ArkhipovBL19}, \href{../works/KreterSSZ18.pdf}{KreterSSZ18}~\cite{KreterSSZ18}, \href{../works/KameugneFGOQ18.pdf}{KameugneFGOQ18}~\cite{KameugneFGOQ18}, \href{../works/LaborieRSV18.pdf}{LaborieRSV18}~\cite{LaborieRSV18}, \href{../works/TangLWSK18.pdf}{TangLWSK18}~\cite{TangLWSK18}... (Total: 67) & \href{../works/Caballero23.pdf}{Caballero23}~\cite{Caballero23}, \href{../works/KameugneFND23.pdf}{KameugneFND23}~\cite{KameugneFND23}, \href{../works/TardivoDFMP23.pdf}{TardivoDFMP23}~\cite{TardivoDFMP23}, \href{../works/KovacsTKSG21.pdf}{KovacsTKSG21}~\cite{KovacsTKSG21}, \href{../works/GroleazNS20a.pdf}{GroleazNS20a}~\cite{GroleazNS20a}, \href{../works/Tesch18.pdf}{Tesch18}~\cite{Tesch18}, \href{../works/CauwelaertLS18.pdf}{CauwelaertLS18}~\cite{CauwelaertLS18}, \href{../works/BaptisteB18.pdf}{BaptisteB18}~\cite{BaptisteB18}, \href{../works/Dejemeppe16.pdf}{Dejemeppe16}~\cite{Dejemeppe16}, \href{../works/NattafAL15.pdf}{NattafAL15}~\cite{NattafAL15}, \href{../works/GayHLS15.pdf}{GayHLS15}~\cite{GayHLS15}, \href{../works/LombardiBM15.pdf}{LombardiBM15}~\cite{LombardiBM15}, \href{../works/KameugneFSN14.pdf}{KameugneFSN14}~\cite{KameugneFSN14}, \href{../works/LaborieR14.pdf}{LaborieR14}~\cite{LaborieR14}, \href{../works/LombardiM13.pdf}{LombardiM13}~\cite{LombardiM13}, \href{../works/LombardiMB13.pdf}{LombardiMB13}~\cite{LombardiMB13}, \href{../works/KameugneFSN11.pdf}{KameugneFSN11}~\cite{KameugneFSN11}, \href{../works/HeinzS11.pdf}{HeinzS11}~\cite{HeinzS11}, \href{../works/abs-1009-0347.pdf}{abs-1009-0347}~\cite{abs-1009-0347}, \href{../works/KeriK07.pdf}{KeriK07}~\cite{KeriK07}, \href{../works/KovacsV06.pdf}{KovacsV06}~\cite{KovacsV06}, \href{../works/HeipckeCCS00.pdf}{HeipckeCCS00}~\cite{HeipckeCCS00}, \href{../works/ArtiguesR00.pdf}{ArtiguesR00}~\cite{ArtiguesR00} & \href{../works/AbreuPNF23.pdf}{AbreuPNF23}~\cite{AbreuPNF23}, \href{../works/NaderiRR23.pdf}{NaderiRR23}~\cite{NaderiRR23}, \href{../works/GeitzGSSW22.pdf}{GeitzGSSW22}~\cite{GeitzGSSW22}, \href{../works/TouatBT22.pdf}{TouatBT22}~\cite{TouatBT22}, \href{../works/HanenKP21.pdf}{HanenKP21}~\cite{HanenKP21}, \href{../works/Astrand21.pdf}{Astrand21}~\cite{Astrand21}, \href{../works/Lemos21.pdf}{Lemos21}~\cite{Lemos21}, \href{../works/ZhangYW21.pdf}{ZhangYW21}~\cite{ZhangYW21}, \href{../works/Mercier-AubinGQ20.pdf}{Mercier-AubinGQ20}~\cite{Mercier-AubinGQ20}, \href{../works/NattafHKAL19.pdf}{NattafHKAL19}~\cite{NattafHKAL19}, \href{../works/WikarekS19.pdf}{WikarekS19}~\cite{WikarekS19}, \href{../works/OuelletQ18.pdf}{OuelletQ18}~\cite{OuelletQ18}, \href{../works/FahimiOQ18.pdf}{FahimiOQ18}~\cite{FahimiOQ18}, \href{../works/HookerH17.pdf}{HookerH17}~\cite{HookerH17}, \href{../works/GingrasQ16.pdf}{GingrasQ16}~\cite{GingrasQ16}, \href{../works/Tesch16.pdf}{Tesch16}~\cite{Tesch16}, \href{../works/NattafALR16.pdf}{NattafALR16}~\cite{NattafALR16}, \href{../works/BonfiettiZLM16.pdf}{BonfiettiZLM16}~\cite{BonfiettiZLM16}, \href{../works/Fahimi16.pdf}{Fahimi16}~\cite{Fahimi16}, \href{../works/CauwelaertLS15.pdf}{CauwelaertLS15}~\cite{CauwelaertLS15}, \href{../works/Siala15.pdf}{Siala15}~\cite{Siala15}, \href{../works/Siala15a.pdf}{Siala15a}~\cite{Siala15a}, \href{../works/SialaAH15.pdf}{SialaAH15}~\cite{SialaAH15}, \href{../works/GayHS15a.pdf}{GayHS15a}~\cite{GayHS15a}, \href{../works/DerrienPZ14.pdf}{DerrienPZ14}~\cite{DerrienPZ14}, \href{../works/BonfiettiLBM14.pdf}{BonfiettiLBM14}~\cite{BonfiettiLBM14}, \href{../works/KoschB14.pdf}{KoschB14}~\cite{KoschB14}, \href{../works/BonfiettiLM14.pdf}{BonfiettiLM14}~\cite{BonfiettiLM14}, \href{../works/OuelletQ13.pdf}{OuelletQ13}~\cite{OuelletQ13}... (Total: 47)\\
Classification & RCPSPDC &  &  & \href{../works/CampeauG22.pdf}{CampeauG22}~\cite{CampeauG22}, \href{../works/HubnerGSV21.pdf}{HubnerGSV21}~\cite{HubnerGSV21}\\
Classification & RTMP & \href{../works/MarliereSPR23.pdf}{MarliereSPR23}~\cite{MarliereSPR23} &  & \\
Classification & Resource-constrained Project Scheduling Problem & \href{../works/PovedaAA23.pdf}{PovedaAA23}~\cite{PovedaAA23}, \href{../works/SubulanC22.pdf}{SubulanC22}~\cite{SubulanC22}, \href{../works/BoudreaultSLQ22.pdf}{BoudreaultSLQ22}~\cite{BoudreaultSLQ22}, \href{../works/EtminaniesfahaniGNMS22.pdf}{EtminaniesfahaniGNMS22}~\cite{EtminaniesfahaniGNMS22}, \href{../works/HillTV21.pdf}{HillTV21}~\cite{HillTV21}, \href{../works/Godet21a.pdf}{Godet21a}~\cite{Godet21a}, \href{../works/ZarandiASC20.pdf}{ZarandiASC20}~\cite{ZarandiASC20}, \href{../works/Caballero19.pdf}{Caballero19}~\cite{Caballero19}, \href{../works/abs-1902-09244.pdf}{abs-1902-09244}~\cite{abs-1902-09244}, \href{../works/SchnellH15.pdf}{SchnellH15}~\cite{SchnellH15}, \href{../works/HeinzSB13.pdf}{HeinzSB13}~\cite{HeinzSB13}, \href{../works/LombardiM12.pdf}{LombardiM12}~\cite{LombardiM12}, \href{../works/SchuttFSW11.pdf}{SchuttFSW11}~\cite{SchuttFSW11}, \href{../works/Schutt11.pdf}{Schutt11}~\cite{Schutt11}, \href{../works/Lombardi10.pdf}{Lombardi10}~\cite{Lombardi10}, \href{../works/DemasseyAM05.pdf}{DemasseyAM05}~\cite{DemasseyAM05}, \href{../works/Demassey03.pdf}{Demassey03}~\cite{Demassey03} & \href{../works/KameugneFND23.pdf}{KameugneFND23}~\cite{KameugneFND23}, \href{../works/YuraszeckMCCR23.pdf}{YuraszeckMCCR23}~\cite{YuraszeckMCCR23}, \href{../works/Groleaz21.pdf}{Groleaz21}~\cite{Groleaz21}, \href{../works/Astrand21.pdf}{Astrand21}~\cite{Astrand21}, \href{../works/HubnerGSV21.pdf}{HubnerGSV21}~\cite{HubnerGSV21}, \href{../works/GokGSTO20.pdf}{GokGSTO20}~\cite{GokGSTO20}, \href{../works/Polo-MejiaALB20.pdf}{Polo-MejiaALB20}~\cite{Polo-MejiaALB20}, \href{../works/HauderBRPA20.pdf}{HauderBRPA20}~\cite{HauderBRPA20}, \href{../works/ArkhipovBL19.pdf}{ArkhipovBL19}~\cite{ArkhipovBL19}, \href{../works/NattafHKAL19.pdf}{NattafHKAL19}~\cite{NattafHKAL19}, \href{../works/KameugneFGOQ18.pdf}{KameugneFGOQ18}~\cite{KameugneFGOQ18}, \href{../works/BofillCSV17.pdf}{BofillCSV17}~\cite{BofillCSV17}, \href{../works/YoungFS17.pdf}{YoungFS17}~\cite{YoungFS17}, \href{../works/SchuttS16.pdf}{SchuttS16}~\cite{SchuttS16}, \href{../works/Nattaf16.pdf}{Nattaf16}~\cite{Nattaf16}, \href{../works/SzerediS16.pdf}{SzerediS16}~\cite{SzerediS16}, \href{../works/AmadiniGM16.pdf}{AmadiniGM16}~\cite{AmadiniGM16}, \href{../works/VilimLS15.pdf}{VilimLS15}~\cite{VilimLS15}, \href{../works/Kameugne14.pdf}{Kameugne14}~\cite{Kameugne14}, \href{../works/SchuttFSW13.pdf}{SchuttFSW13}~\cite{SchuttFSW13}, \href{../works/SchuttFS13a.pdf}{SchuttFS13a}~\cite{SchuttFS13a}, \href{../works/GuSS13.pdf}{GuSS13}~\cite{GuSS13}, \href{../works/LombardiM12a.pdf}{LombardiM12a}~\cite{LombardiM12a}, \href{../works/SchuttCSW12.pdf}{SchuttCSW12}~\cite{SchuttCSW12}, \href{../works/GuSW12.pdf}{GuSW12}~\cite{GuSW12}, \href{../works/abs-1009-0347.pdf}{abs-1009-0347}~\cite{abs-1009-0347}, \href{../works/LiessM08.pdf}{LiessM08}~\cite{LiessM08}, \href{../works/BeckW07.pdf}{BeckW07}~\cite{BeckW07}, \href{../works/KovacsV04.pdf}{KovacsV04}~\cite{KovacsV04}... (Total: 34) & \href{../works/abs-2402-00459.pdf}{abs-2402-00459}~\cite{abs-2402-00459}, \href{../works/LuZZYW24.pdf}{LuZZYW24}~\cite{LuZZYW24}, \href{../works/Caballero23.pdf}{Caballero23}~\cite{Caballero23}, \href{../works/GokPTGO23.pdf}{GokPTGO23}~\cite{GokPTGO23}, \href{../works/NaderiRR23.pdf}{NaderiRR23}~\cite{NaderiRR23}, \href{../works/CampeauG22.pdf}{CampeauG22}~\cite{CampeauG22}, \href{../works/FetgoD22.pdf}{FetgoD22}~\cite{FetgoD22}, \href{../works/MullerMKP22.pdf}{MullerMKP22}~\cite{MullerMKP22}, \href{../works/ZhangYW21.pdf}{ZhangYW21}~\cite{ZhangYW21}, \href{../works/HanenKP21.pdf}{HanenKP21}~\cite{HanenKP21}, \href{../works/ArtiguesHQT21.pdf}{ArtiguesHQT21}~\cite{ArtiguesHQT21}, \href{../works/GeibingerMM21.pdf}{GeibingerMM21}~\cite{GeibingerMM21}, \href{../works/GroleazNS20a.pdf}{GroleazNS20a}~\cite{GroleazNS20a}, \href{../works/GroleazNS20.pdf}{GroleazNS20}~\cite{GroleazNS20}, \href{../works/AstrandJZ20.pdf}{AstrandJZ20}~\cite{AstrandJZ20}, \href{../works/SacramentoSP20.pdf}{SacramentoSP20}~\cite{SacramentoSP20}, \href{../works/BadicaBI20.pdf}{BadicaBI20}~\cite{BadicaBI20}, \href{../works/abs-1911-04766.pdf}{abs-1911-04766}~\cite{abs-1911-04766}, \href{../works/GalleguillosKSB19.pdf}{GalleguillosKSB19}~\cite{GalleguillosKSB19}, \href{../works/GeibingerMM19.pdf}{GeibingerMM19}~\cite{GeibingerMM19}, \href{../works/CauwelaertLS18.pdf}{CauwelaertLS18}~\cite{CauwelaertLS18}, \href{../works/Tesch18.pdf}{Tesch18}~\cite{Tesch18}, \href{../works/LaborieRSV18.pdf}{LaborieRSV18}~\cite{LaborieRSV18}, \href{../works/KreterSSZ18.pdf}{KreterSSZ18}~\cite{KreterSSZ18}, \href{../works/FahimiOQ18.pdf}{FahimiOQ18}~\cite{FahimiOQ18}, \href{../works/GombolayWS18.pdf}{GombolayWS18}~\cite{GombolayWS18}, \href{../works/BaptisteB18.pdf}{BaptisteB18}~\cite{BaptisteB18}, \href{../works/KreterSS17.pdf}{KreterSS17}~\cite{KreterSS17}, \href{../works/MossigeGSMC17.pdf}{MossigeGSMC17}~\cite{MossigeGSMC17}... (Total: 63)\\
Classification & Resource-constrained Project Scheduling Problem with Discounted Cashflow &  &  & \href{../works/ZarandiASC20.pdf}{ZarandiASC20}~\cite{ZarandiASC20}\\
Classification & SBSFMMAL & \href{../works/OzturkTHO13.pdf}{OzturkTHO13}~\cite{OzturkTHO13}, \href{../works/OzturkTHO10.pdf}{OzturkTHO10}~\cite{OzturkTHO10} & \href{../works/OzturkTHO15.pdf}{OzturkTHO15}~\cite{OzturkTHO15} & \\
Classification & SCC & \href{../works/KimCMLLP23.pdf}{KimCMLLP23}~\cite{KimCMLLP23}, \href{../works/WolinskiKG04.pdf}{WolinskiKG04}~\cite{WolinskiKG04} & \href{../works/SchuttFSW13.pdf}{SchuttFSW13}~\cite{SchuttFSW13}, \href{../works/Lombardi10.pdf}{Lombardi10}~\cite{Lombardi10}, \href{../works/abs-1009-0347.pdf}{abs-1009-0347}~\cite{abs-1009-0347} & \href{../works/PohlAK22.pdf}{PohlAK22}~\cite{PohlAK22}, \href{../works/Zahout21.pdf}{Zahout21}~\cite{Zahout21}, \href{../works/LombardiMB13.pdf}{LombardiMB13}~\cite{LombardiMB13}, \href{../works/BeniniLMR11.pdf}{BeniniLMR11}~\cite{BeniniLMR11}, \href{../works/SchausHMCMD11.pdf}{SchausHMCMD11}~\cite{SchausHMCMD11}, \href{../works/LombardiMRB10.pdf}{LombardiMRB10}~\cite{LombardiMRB10}, \href{../works/BeniniLMR08.pdf}{BeniniLMR08}~\cite{BeniniLMR08}, \href{../works/BeniniLMMR08.pdf}{BeniniLMMR08}~\cite{BeniniLMMR08}\\
Classification & TCSP & \href{../works/BelhadjiI98.pdf}{BelhadjiI98}~\cite{BelhadjiI98} &  & \href{../works/Zahout21.pdf}{Zahout21}~\cite{Zahout21}, \href{../works/BartakSR10.pdf}{BartakSR10}~\cite{BartakSR10}, \href{../works/LombardiM10a.pdf}{LombardiM10a}~\cite{LombardiM10a}, \href{../works/Lombardi10.pdf}{Lombardi10}~\cite{Lombardi10}, \href{../works/Demassey03.pdf}{Demassey03}~\cite{Demassey03}\\
Classification & TMS & \href{../works/PopovicCGNC22.pdf}{PopovicCGNC22}~\cite{PopovicCGNC22}, \href{../works/Froger16.pdf}{Froger16}~\cite{Froger16} & \href{../works/BegB13.pdf}{BegB13}~\cite{BegB13} & \href{../works/CappartS17.pdf}{CappartS17}~\cite{CappartS17}, \href{../works/Siala15a.pdf}{Siala15a}~\cite{Siala15a}, \href{../works/Siala15.pdf}{Siala15}~\cite{Siala15}, \href{../works/JussienL02.pdf}{JussienL02}~\cite{JussienL02}\\
Classification & Temporal Constraint Satisfaction Problem &  & \href{../works/BelhadjiI98.pdf}{BelhadjiI98}~\cite{BelhadjiI98} & \href{../works/BartakSR10.pdf}{BartakSR10}~\cite{BartakSR10}, \href{../works/MoffittPP05.pdf}{MoffittPP05}~\cite{MoffittPP05}, \href{../works/Elkhyari03.pdf}{Elkhyari03}~\cite{Elkhyari03}\\
Classification & parallel machine & \href{../works/PrataAN23.pdf}{PrataAN23}~\cite{PrataAN23}, \href{../works/abs-2305-19888.pdf}{abs-2305-19888}~\cite{abs-2305-19888}, \href{../works/Adelgren2023.pdf}{Adelgren2023}~\cite{Adelgren2023}, \href{../works/IsikYA23.pdf}{IsikYA23}~\cite{IsikYA23}, \href{../works/CzerniachowskaWZ23.pdf}{CzerniachowskaWZ23}~\cite{CzerniachowskaWZ23}, \href{../works/NaderiRR23.pdf}{NaderiRR23}~\cite{NaderiRR23}, \href{../works/YunusogluY22.pdf}{YunusogluY22}~\cite{YunusogluY22}, \href{../works/ZhangJZL22.pdf}{ZhangJZL22}~\cite{ZhangJZL22}, \href{../works/WinterMMW22.pdf}{WinterMMW22}~\cite{WinterMMW22}, \href{../works/HeinzNVH22.pdf}{HeinzNVH22}~\cite{HeinzNVH22}, \href{../works/OujanaAYB22.pdf}{OujanaAYB22}~\cite{OujanaAYB22}, \href{../works/PandeyS21a.pdf}{PandeyS21a}~\cite{PandeyS21a}, \href{../works/Astrand21.pdf}{Astrand21}~\cite{Astrand21}, \href{../works/Godet21a.pdf}{Godet21a}~\cite{Godet21a}, \href{../works/Groleaz21.pdf}{Groleaz21}~\cite{Groleaz21}, \href{../works/ZarandiASC20.pdf}{ZarandiASC20}~\cite{ZarandiASC20}, \href{../works/MengZRZL20.pdf}{MengZRZL20}~\cite{MengZRZL20}, \href{../works/Lunardi20.pdf}{Lunardi20}~\cite{Lunardi20}, \href{../works/GodetLHS20.pdf}{GodetLHS20}~\cite{GodetLHS20}, \href{../works/NattafM20.pdf}{NattafM20}~\cite{NattafM20}, \href{../works/NattafDYW19.pdf}{NattafDYW19}~\cite{NattafDYW19}, \href{../works/MalapertN19.pdf}{MalapertN19}~\cite{MalapertN19}, \href{../works/GokgurHO18.pdf}{GokgurHO18}~\cite{GokgurHO18}, \href{../works/GedikKEK18.pdf}{GedikKEK18}~\cite{GedikKEK18}, \href{../works/ArbaouiY18.pdf}{ArbaouiY18}~\cite{ArbaouiY18}, \href{../works/TanT18.pdf}{TanT18}~\cite{TanT18}, \href{../works/GomesM17.pdf}{GomesM17}~\cite{GomesM17}, \href{../works/HebrardHJMPV16.pdf}{HebrardHJMPV16}~\cite{HebrardHJMPV16}, \href{../works/TranAB16.pdf}{TranAB16}~\cite{TranAB16}... (Total: 35) & \href{../works/NaderiBZ23.pdf}{NaderiBZ23}~\cite{NaderiBZ23}, \href{../works/PenzDN23.pdf}{PenzDN23}~\cite{PenzDN23}, \href{../works/JuvinHL23a.pdf}{JuvinHL23a}~\cite{JuvinHL23a}, \href{../works/Fatemi-AnarakiTFV23.pdf}{Fatemi-AnarakiTFV23}~\cite{Fatemi-AnarakiTFV23}, \href{../works/AbreuPNF23.pdf}{AbreuPNF23}~\cite{AbreuPNF23}, \href{../works/AbreuNP23.pdf}{AbreuNP23}~\cite{AbreuNP23}, \href{../works/Teppan22.pdf}{Teppan22}~\cite{Teppan22}, \href{../works/NaderiBZ22.pdf}{NaderiBZ22}~\cite{NaderiBZ22}, \href{../works/EmdeZD22.pdf}{EmdeZD22}~\cite{EmdeZD22}, \href{../works/ColT22.pdf}{ColT22}~\cite{ColT22}, \href{../works/Zahout21.pdf}{Zahout21}~\cite{Zahout21}, \href{../works/Bedhief21.pdf}{Bedhief21}~\cite{Bedhief21}, \href{../works/MokhtarzadehTNF20.pdf}{MokhtarzadehTNF20}~\cite{MokhtarzadehTNF20}, \href{../works/SacramentoSP20.pdf}{SacramentoSP20}~\cite{SacramentoSP20}, \href{../works/MejiaY20.pdf}{MejiaY20}~\cite{MejiaY20}, \href{../works/ParkUJR19.pdf}{ParkUJR19}~\cite{ParkUJR19}, \href{../works/Novas19.pdf}{Novas19}~\cite{Novas19}, \href{../works/BogaerdtW19.pdf}{BogaerdtW19}~\cite{BogaerdtW19}, \href{../works/Ham18a.pdf}{Ham18a}~\cite{Ham18a}, \href{../works/BenediktSMVH18.pdf}{BenediktSMVH18}~\cite{BenediktSMVH18}, \href{../works/RoshanaeiLAU17.pdf}{RoshanaeiLAU17}~\cite{RoshanaeiLAU17}, \href{../works/CatusseCBL16.pdf}{CatusseCBL16}~\cite{CatusseCBL16}, \href{../works/ZhouGL15.pdf}{ZhouGL15}~\cite{ZhouGL15}, \href{../works/TerekhovTDB14.pdf}{TerekhovTDB14}~\cite{TerekhovTDB14}, \href{../works/TranTDB13.pdf}{TranTDB13}~\cite{TranTDB13}, \href{../works/BajestaniB13.pdf}{BajestaniB13}~\cite{BajestaniB13}, \href{../works/GuyonLPR12.pdf}{GuyonLPR12}~\cite{GuyonLPR12}, \href{../works/KovacsB11.pdf}{KovacsB11}~\cite{KovacsB11}, \href{../works/AkkerDH07.pdf}{AkkerDH07}~\cite{AkkerDH07}... (Total: 31) & \href{../works/KimCMLLP23.pdf}{KimCMLLP23}~\cite{KimCMLLP23}, \href{../works/GuoZ23.pdf}{GuoZ23}~\cite{GuoZ23}, \href{../works/JuvinHHL23.pdf}{JuvinHHL23}~\cite{JuvinHHL23}, \href{../works/LacknerMMWW23.pdf}{LacknerMMWW23}~\cite{LacknerMMWW23}, \href{../works/Mehdizadeh-Somarin23.pdf}{Mehdizadeh-Somarin23}~\cite{Mehdizadeh-Somarin23}, \href{../works/AlfieriGPS23.pdf}{AlfieriGPS23}~\cite{AlfieriGPS23}, \href{../works/JuvinHL22.pdf}{JuvinHL22}~\cite{JuvinHL22}, \href{../works/ArmstrongGOS22.pdf}{ArmstrongGOS22}~\cite{ArmstrongGOS22}, \href{../works/OrnekOS20.pdf}{OrnekOS20}~\cite{OrnekOS20}, \href{../works/EtminaniesfahaniGNMS22.pdf}{EtminaniesfahaniGNMS22}~\cite{EtminaniesfahaniGNMS22}, \href{../works/NaderiBZ22a.pdf}{NaderiBZ22a}~\cite{NaderiBZ22a}, \href{../works/HanenKP21.pdf}{HanenKP21}~\cite{HanenKP21}, \href{../works/FanXG21.pdf}{FanXG21}~\cite{FanXG21}, \href{../works/AbohashimaEG21.pdf}{AbohashimaEG21}~\cite{AbohashimaEG21}, \href{../works/AbreuAPNM21.pdf}{AbreuAPNM21}~\cite{AbreuAPNM21}, \href{../works/HamPK21.pdf}{HamPK21}~\cite{HamPK21}, \href{../works/LacknerMMWW21.pdf}{LacknerMMWW21}~\cite{LacknerMMWW21}, \href{../works/RoshanaeiBAUB20.pdf}{RoshanaeiBAUB20}~\cite{RoshanaeiBAUB20}, \href{../works/GroleazNS20a.pdf}{GroleazNS20a}~\cite{GroleazNS20a}, \href{../works/QinDCS20.pdf}{QinDCS20}~\cite{QinDCS20}, \href{../works/AstrandJZ20.pdf}{AstrandJZ20}~\cite{AstrandJZ20}, \href{../works/NishikawaSTT19.pdf}{NishikawaSTT19}~\cite{NishikawaSTT19}, \href{../works/Hooker19.pdf}{Hooker19}~\cite{Hooker19}, \href{../works/ArkhipovBL19.pdf}{ArkhipovBL19}~\cite{ArkhipovBL19}, \href{../works/Ham18.pdf}{Ham18}~\cite{Ham18}, \href{../works/BaptisteB18.pdf}{BaptisteB18}~\cite{BaptisteB18}, \href{../works/LaborieRSV18.pdf}{LaborieRSV18}~\cite{LaborieRSV18}, \href{../works/HookerH17.pdf}{HookerH17}~\cite{HookerH17}, \href{../works/KletzanderM17.pdf}{KletzanderM17}~\cite{KletzanderM17}... (Total: 51)\\
Classification & psplib & \href{../works/TardivoDFMP23.pdf}{TardivoDFMP23}~\cite{TardivoDFMP23}, \href{../works/Caballero19.pdf}{Caballero19}~\cite{Caballero19}, \href{../works/ArkhipovBL19.pdf}{ArkhipovBL19}~\cite{ArkhipovBL19}, \href{../works/KreterSSZ18.pdf}{KreterSSZ18}~\cite{KreterSSZ18}, \href{../works/OuelletQ18.pdf}{OuelletQ18}~\cite{OuelletQ18}, \href{../works/GayHS15a.pdf}{GayHS15a}~\cite{GayHS15a}, \href{../works/Derrien15.pdf}{Derrien15}~\cite{Derrien15}, \href{../works/LetortCB15.pdf}{LetortCB15}~\cite{LetortCB15}, \href{../works/KameugneFSN14.pdf}{KameugneFSN14}~\cite{KameugneFSN14}, \href{../works/DerrienP14.pdf}{DerrienP14}~\cite{DerrienP14}, \href{../works/Kameugne14.pdf}{Kameugne14}~\cite{Kameugne14}, \href{../works/SchuttFSW13.pdf}{SchuttFSW13}~\cite{SchuttFSW13}, \href{../works/SchuttFS13a.pdf}{SchuttFS13a}~\cite{SchuttFS13a}, \href{../works/HeinzSB13.pdf}{HeinzSB13}~\cite{HeinzSB13}, \href{../works/Letort13.pdf}{Letort13}~\cite{Letort13}, \href{../works/Clercq12.pdf}{Clercq12}~\cite{Clercq12}, \href{../works/SchuttFSW11.pdf}{SchuttFSW11}~\cite{SchuttFSW11}, \href{../works/Schutt11.pdf}{Schutt11}~\cite{Schutt11}, \href{../works/BertholdHLMS10.pdf}{BertholdHLMS10}~\cite{BertholdHLMS10}, \href{../works/SchuttFSW09.pdf}{SchuttFSW09}~\cite{SchuttFSW09}, \href{../works/Demassey03.pdf}{Demassey03}~\cite{Demassey03} & \href{../works/KameugneFND23.pdf}{KameugneFND23}~\cite{KameugneFND23}, \href{../works/BoudreaultSLQ22.pdf}{BoudreaultSLQ22}~\cite{BoudreaultSLQ22}, \href{../works/EtminaniesfahaniGNMS22.pdf}{EtminaniesfahaniGNMS22}~\cite{EtminaniesfahaniGNMS22}, \href{../works/HillTV21.pdf}{HillTV21}~\cite{HillTV21}, \href{../works/BadicaBI20.pdf}{BadicaBI20}~\cite{BadicaBI20}, \href{../works/Tesch18.pdf}{Tesch18}~\cite{Tesch18}, \href{../works/FahimiOQ18.pdf}{FahimiOQ18}~\cite{FahimiOQ18}, \href{../works/BaptisteB18.pdf}{BaptisteB18}~\cite{BaptisteB18}, \href{../works/Tesch16.pdf}{Tesch16}~\cite{Tesch16}, \href{../works/GingrasQ16.pdf}{GingrasQ16}~\cite{GingrasQ16}, \href{../works/Nattaf16.pdf}{Nattaf16}~\cite{Nattaf16}, \href{../works/SzerediS16.pdf}{SzerediS16}~\cite{SzerediS16}, \href{../works/VilimLS15.pdf}{VilimLS15}~\cite{VilimLS15}, \href{../works/GayHLS15.pdf}{GayHLS15}~\cite{GayHLS15}, \href{../works/LombardiBM15.pdf}{LombardiBM15}~\cite{LombardiBM15}, \href{../works/BonfiettiLM14.pdf}{BonfiettiLM14}~\cite{BonfiettiLM14}, \href{../works/LetortCB13.pdf}{LetortCB13}~\cite{LetortCB13}, \href{../works/LombardiM12a.pdf}{LombardiM12a}~\cite{LombardiM12a}, \href{../works/LetortBC12.pdf}{LetortBC12}~\cite{LetortBC12}, \href{../works/HeinzS11.pdf}{HeinzS11}~\cite{HeinzS11}, \href{../works/Vilim11.pdf}{Vilim11}~\cite{Vilim11}, \href{../works/abs-1009-0347.pdf}{abs-1009-0347}~\cite{abs-1009-0347}, \href{../works/SchuttW10.pdf}{SchuttW10}~\cite{SchuttW10} & \href{../works/Godet21a.pdf}{Godet21a}~\cite{Godet21a}, \href{../works/CauwelaertLS18.pdf}{CauwelaertLS18}~\cite{CauwelaertLS18}, \href{../works/LaborieRSV18.pdf}{LaborieRSV18}~\cite{LaborieRSV18}, \href{../works/YoungFS17.pdf}{YoungFS17}~\cite{YoungFS17}, \href{../works/Pralet17.pdf}{Pralet17}~\cite{Pralet17}, \href{../works/BofillCSV17.pdf}{BofillCSV17}~\cite{BofillCSV17}, \href{../works/Dejemeppe16.pdf}{Dejemeppe16}~\cite{Dejemeppe16}, \href{../works/SchnellH15.pdf}{SchnellH15}~\cite{SchnellH15}, \href{../works/CauwelaertLS15.pdf}{CauwelaertLS15}~\cite{CauwelaertLS15}, \href{../works/ThiruvadyWGS14.pdf}{ThiruvadyWGS14}~\cite{ThiruvadyWGS14}, \href{../works/LombardiM13.pdf}{LombardiM13}~\cite{LombardiM13}, \href{../works/OuelletQ13.pdf}{OuelletQ13}~\cite{OuelletQ13}, \href{../works/LombardiM12.pdf}{LombardiM12}~\cite{LombardiM12}, \href{../works/KameugneFSN11.pdf}{KameugneFSN11}~\cite{KameugneFSN11}, \href{../works/LiessM08.pdf}{LiessM08}~\cite{LiessM08}, \href{../works/FortinZDF05.pdf}{FortinZDF05}~\cite{FortinZDF05}, \href{../works/DemasseyAM05.pdf}{DemasseyAM05}~\cite{DemasseyAM05}, \href{../works/ElkhyariGJ02a.pdf}{ElkhyariGJ02a}~\cite{ElkhyariGJ02a}\\
Classification & rtRTMP & \href{../works/MarliereSPR23.pdf}{MarliereSPR23}~\cite{MarliereSPR23} &  & \\
Classification & single machine & \href{../works/BonninMNE24.pdf}{BonninMNE24}~\cite{BonninMNE24}, \href{../works/PrataAN23.pdf}{PrataAN23}~\cite{PrataAN23}, \href{../works/AlfieriGPS23.pdf}{AlfieriGPS23}~\cite{AlfieriGPS23}, \href{../works/LacknerMMWW23.pdf}{LacknerMMWW23}~\cite{LacknerMMWW23}, \href{../works/PenzDN23.pdf}{PenzDN23}~\cite{PenzDN23}, \href{../works/TouatBT22.pdf}{TouatBT22}~\cite{TouatBT22}, \href{../works/HamPK21.pdf}{HamPK21}~\cite{HamPK21}, \href{../works/Groleaz21.pdf}{Groleaz21}~\cite{Groleaz21}, \href{../works/BenediktMH20.pdf}{BenediktMH20}~\cite{BenediktMH20}, \href{../works/ZarandiASC20.pdf}{ZarandiASC20}~\cite{ZarandiASC20}, \href{../works/BogaerdtW19.pdf}{BogaerdtW19}~\cite{BogaerdtW19}, \href{../works/BajestaniB15.pdf}{BajestaniB15}~\cite{BajestaniB15}, \href{../works/BajestaniB13.pdf}{BajestaniB13}~\cite{BajestaniB13}, \href{../works/TerekhovDOB12.pdf}{TerekhovDOB12}~\cite{TerekhovDOB12}, \href{../works/KovacsB11.pdf}{KovacsB11}~\cite{KovacsB11}, \href{../works/ThiruvadyBME09.pdf}{ThiruvadyBME09}~\cite{ThiruvadyBME09}, \href{../works/WuBB09.pdf}{WuBB09}~\cite{WuBB09}, \href{../works/KovacsB07.pdf}{KovacsB07}~\cite{KovacsB07}, \href{../works/SadykovW06.pdf}{SadykovW06}~\cite{SadykovW06}, \href{../works/KanetAG04.pdf}{KanetAG04}~\cite{KanetAG04}, \href{../works/Elkhyari03.pdf}{Elkhyari03}~\cite{Elkhyari03}, \href{../works/Baptiste02.pdf}{Baptiste02}~\cite{Baptiste02}, \href{../works/SourdN00.pdf}{SourdN00}~\cite{SourdN00}, \href{../works/BlazewiczDP96.pdf}{BlazewiczDP96}~\cite{BlazewiczDP96} & \href{../works/NaderiBZ23.pdf}{NaderiBZ23}~\cite{NaderiBZ23}, \href{../works/ZhangBB22.pdf}{ZhangBB22}~\cite{ZhangBB22}, \href{../works/EmdeZD22.pdf}{EmdeZD22}~\cite{EmdeZD22}, \href{../works/NaderiBZ22.pdf}{NaderiBZ22}~\cite{NaderiBZ22}, \href{../works/ElciOH22.pdf}{ElciOH22}~\cite{ElciOH22}, \href{../works/YuraszeckMPV22.pdf}{YuraszeckMPV22}~\cite{YuraszeckMPV22}, \href{../works/Bedhief21.pdf}{Bedhief21}~\cite{Bedhief21}, \href{../works/KoehlerBFFHPSSS21.pdf}{KoehlerBFFHPSSS21}~\cite{KoehlerBFFHPSSS21}, \href{../works/LacknerMMWW21.pdf}{LacknerMMWW21}~\cite{LacknerMMWW21}, \href{../works/PandeyS21a.pdf}{PandeyS21a}~\cite{PandeyS21a}, \href{../works/Astrand21.pdf}{Astrand21}~\cite{Astrand21}, \href{../works/HillTV21.pdf}{HillTV21}~\cite{HillTV21}, \href{../works/Zahout21.pdf}{Zahout21}~\cite{Zahout21}, \href{../works/AbreuAPNM21.pdf}{AbreuAPNM21}~\cite{AbreuAPNM21}, \href{../works/NattafM20.pdf}{NattafM20}~\cite{NattafM20}, \href{../works/Lunardi20.pdf}{Lunardi20}~\cite{Lunardi20}, \href{../works/BenediktSMVH18.pdf}{BenediktSMVH18}~\cite{BenediktSMVH18}, \href{../works/Tesch18.pdf}{Tesch18}~\cite{Tesch18}, \href{../works/TranPZLDB18.pdf}{TranPZLDB18}~\cite{TranPZLDB18}, \href{../works/TanT18.pdf}{TanT18}~\cite{TanT18}, \href{../works/GomesM17.pdf}{GomesM17}~\cite{GomesM17}, \href{../works/TranAB16.pdf}{TranAB16}~\cite{TranAB16}, \href{../works/LaborieR14.pdf}{LaborieR14}~\cite{LaborieR14}, \href{../works/KoschB14.pdf}{KoschB14}~\cite{KoschB14}, \href{../works/BillautHL12.pdf}{BillautHL12}~\cite{BillautHL12}, \href{../works/TranB12.pdf}{TranB12}~\cite{TranB12}, \href{../works/KovacsK11.pdf}{KovacsK11}~\cite{KovacsK11}, \href{../works/Malapert11.pdf}{Malapert11}~\cite{Malapert11}, \href{../works/Beck10.pdf}{Beck10}~\cite{Beck10}... (Total: 38) & \href{../works/abs-2402-00459.pdf}{abs-2402-00459}~\cite{abs-2402-00459}, \href{../works/LuZZYW24.pdf}{LuZZYW24}~\cite{LuZZYW24}, \href{../works/IsikYA23.pdf}{IsikYA23}~\cite{IsikYA23}, \href{../works/NaderiRR23.pdf}{NaderiRR23}~\cite{NaderiRR23}, \href{../works/Fatemi-AnarakiTFV23.pdf}{Fatemi-AnarakiTFV23}~\cite{Fatemi-AnarakiTFV23}, \href{../works/JuvinHL23a.pdf}{JuvinHL23a}~\cite{JuvinHL23a}, \href{../works/Mehdizadeh-Somarin23.pdf}{Mehdizadeh-Somarin23}~\cite{Mehdizadeh-Somarin23}, \href{../works/GeitzGSSW22.pdf}{GeitzGSSW22}~\cite{GeitzGSSW22}, \href{../works/JuvinHL22.pdf}{JuvinHL22}~\cite{JuvinHL22}, \href{../works/ZhangJZL22.pdf}{ZhangJZL22}~\cite{ZhangJZL22}, \href{../works/AbreuN22.pdf}{AbreuN22}~\cite{AbreuN22}, \href{../works/ColT22.pdf}{ColT22}~\cite{ColT22}, \href{../works/abs-2211-14492.pdf}{abs-2211-14492}~\cite{abs-2211-14492}, \href{../works/PohlAK22.pdf}{PohlAK22}~\cite{PohlAK22}, \href{../works/LiFJZLL22.pdf}{LiFJZLL22}~\cite{LiFJZLL22}, \href{../works/Godet21a.pdf}{Godet21a}~\cite{Godet21a}, \href{../works/FanXG21.pdf}{FanXG21}~\cite{FanXG21}, \href{../works/QinWSLS21.pdf}{QinWSLS21}~\cite{QinWSLS21}, \href{../works/KovacsTKSG21.pdf}{KovacsTKSG21}~\cite{KovacsTKSG21}, \href{../works/GodetLHS20.pdf}{GodetLHS20}~\cite{GodetLHS20}, \href{../works/TangB20.pdf}{TangB20}~\cite{TangB20}, \href{../works/ParkUJR19.pdf}{ParkUJR19}~\cite{ParkUJR19}, \href{../works/Tom19.pdf}{Tom19}~\cite{Tom19}, \href{../works/HoundjiSW19.pdf}{HoundjiSW19}~\cite{HoundjiSW19}, \href{../works/NattafDYW19.pdf}{NattafDYW19}~\cite{NattafDYW19}, \href{../works/NattafHKAL19.pdf}{NattafHKAL19}~\cite{NattafHKAL19}, \href{../works/Hooker19.pdf}{Hooker19}~\cite{Hooker19}, \href{../works/MalapertN19.pdf}{MalapertN19}~\cite{MalapertN19}, \href{../works/GedikKEK18.pdf}{GedikKEK18}~\cite{GedikKEK18}... (Total: 88)\\
\end{longtable}
}


\clearpage
\subsection{Concept Type Constraints}
\label{sec:Constraints}
{\scriptsize
\begin{longtable}{lp{3cm}>{\raggedright\arraybackslash}p{6cm}>{\raggedright\arraybackslash}p{6cm}>{\raggedright\arraybackslash}p{8cm}}
\rowcolor{white}\caption{Works for Concepts of Type Constraints}\\ \toprule
\rowcolor{white}Type & Keyword & High & Medium & Low\\ \midrule\endhead
\bottomrule
\endfoot
Constraints & alldifferent & \href{works/JuvinHHL23.pdf}{JuvinHHL23}~\cite{JuvinHHL23}, \href{works/Lemos21.pdf}{Lemos21}~\cite{Lemos21}, \href{works/KoehlerBFFHPSSS21.pdf}{KoehlerBFFHPSSS21}~\cite{KoehlerBFFHPSSS21}, \href{works/Godet21a.pdf}{Godet21a}~\cite{Godet21a}, \href{works/CauwelaertLS18.pdf}{CauwelaertLS18}~\cite{CauwelaertLS18}, \href{works/Dejemeppe16.pdf}{Dejemeppe16}~\cite{Dejemeppe16}, \href{works/Derrien15.pdf}{Derrien15}~\cite{Derrien15}, \href{works/Siala15a.pdf}{Siala15a}~\cite{Siala15a}, \href{works/Malapert11.pdf}{Malapert11}~\cite{Malapert11}, \href{works/Menana11.pdf}{Menana11}~\cite{Menana11}, \href{works/OhrimenkoSC09.pdf}{OhrimenkoSC09}~\cite{OhrimenkoSC09}, \href{works/Simonis07.pdf}{Simonis07}~\cite{Simonis07}, \href{works/KanetAG04.pdf}{KanetAG04}~\cite{KanetAG04} & \href{works/GodetLHS20.pdf}{GodetLHS20}~\cite{GodetLHS20}, \href{works/HookerH18.pdf}{HookerH18}~\cite{HookerH18}, \href{works/BessiereHMQW14.pdf}{BessiereHMQW14}~\cite{BessiereHMQW14}, \href{works/KelarevaTK13.pdf}{KelarevaTK13}~\cite{KelarevaTK13}, \href{works/TerekhovDOB12.pdf}{TerekhovDOB12}~\cite{TerekhovDOB12} & \href{works/WangB23.pdf}{WangB23}~\cite{WangB23}, \href{works/ColT22.pdf}{ColT22}~\cite{ColT22}, \href{works/BourreauGGLT22.pdf}{BourreauGGLT22}~\cite{BourreauGGLT22}, \href{works/FarsiTM22.pdf}{FarsiTM22}~\cite{FarsiTM22}, \href{works/Astrand21.pdf}{Astrand21}~\cite{Astrand21}, \href{works/AstrandJZ20.pdf}{AstrandJZ20}~\cite{AstrandJZ20}, \href{works/WangB20.pdf}{WangB20}~\cite{WangB20}, \href{works/AntuoriHHEN20.pdf}{AntuoriHHEN20}~\cite{AntuoriHHEN20}, \href{works/Lunardi20.pdf}{Lunardi20}~\cite{Lunardi20}, \href{works/MokhtarzadehTNF20.pdf}{MokhtarzadehTNF20}~\cite{MokhtarzadehTNF20}, \href{works/FahimiOQ18.pdf}{FahimiOQ18}~\cite{FahimiOQ18}, \href{works/MelgarejoLS15.pdf}{MelgarejoLS15}~\cite{MelgarejoLS15}, \href{works/AlesioNBG14.pdf}{AlesioNBG14}~\cite{AlesioNBG14}, \href{works/ChuGNSW13.pdf}{ChuGNSW13}~\cite{ChuGNSW13}, \href{works/ClercqPBJ11.pdf}{ClercqPBJ11}~\cite{ClercqPBJ11}, \href{works/HermenierDL11.pdf}{HermenierDL11}~\cite{HermenierDL11}, \href{works/HachemiGR11.pdf}{HachemiGR11}~\cite{HachemiGR11}, \href{works/TrojetHL11.pdf}{TrojetHL11}~\cite{TrojetHL11}, \href{works/LopesCSM10.pdf}{LopesCSM10}~\cite{LopesCSM10}, \href{works/Malik08.pdf}{Malik08}~\cite{Malik08}, \href{works/Thorsteinsson01.pdf}{Thorsteinsson01}~\cite{Thorsteinsson01}, \href{works/Simonis99.pdf}{Simonis99}~\cite{Simonis99}, \href{works/BeldiceanuC94.pdf}{BeldiceanuC94}~\cite{BeldiceanuC94}\\
Constraints & alternative constraint & \href{works/LaborieRSV18.pdf}{LaborieRSV18}~\cite{LaborieRSV18} & \href{works/abs-2305-19888.pdf}{abs-2305-19888}~\cite{abs-2305-19888}, \href{works/MurinR19.pdf}{MurinR19}~\cite{MurinR19}, \href{works/GokgurHO18.pdf}{GokgurHO18}~\cite{GokgurHO18} & \href{works/LacknerMMWW23.pdf}{LacknerMMWW23}~\cite{LacknerMMWW23}, \href{works/NaderiRR23.pdf}{NaderiRR23}~\cite{NaderiRR23}, \href{works/WinterMMW22.pdf}{WinterMMW22}~\cite{WinterMMW22}, \href{works/ZhangJZL22.pdf}{ZhangJZL22}~\cite{ZhangJZL22}, \href{works/SvancaraB22.pdf}{SvancaraB22}~\cite{SvancaraB22}, \href{works/HeinzNVH22.pdf}{HeinzNVH22}~\cite{HeinzNVH22}, \href{works/ArmstrongGOS21.pdf}{ArmstrongGOS21}~\cite{ArmstrongGOS21}, \href{works/HubnerGSV21.pdf}{HubnerGSV21}~\cite{HubnerGSV21}, \href{works/PandeyS21a.pdf}{PandeyS21a}~\cite{PandeyS21a}, \href{works/VlkHT21.pdf}{VlkHT21}~\cite{VlkHT21}, \href{works/HillTV21.pdf}{HillTV21}~\cite{HillTV21}, \href{works/MengZRZL20.pdf}{MengZRZL20}~\cite{MengZRZL20}, \href{works/Polo-MejiaALB20.pdf}{Polo-MejiaALB20}~\cite{Polo-MejiaALB20}, \href{works/SacramentoSP20.pdf}{SacramentoSP20}~\cite{SacramentoSP20}, \href{works/YounespourAKE19.pdf}{YounespourAKE19}~\cite{YounespourAKE19}, \href{works/EscobetPQPRA19.pdf}{EscobetPQPRA19}~\cite{EscobetPQPRA19}, \href{works/GeibingerMM19.pdf}{GeibingerMM19}~\cite{GeibingerMM19}, \href{works/NishikawaSTT19.pdf}{NishikawaSTT19}~\cite{NishikawaSTT19}, \href{works/GalleguillosKSB19.pdf}{GalleguillosKSB19}~\cite{GalleguillosKSB19}, \href{works/MalapertN19.pdf}{MalapertN19}~\cite{MalapertN19}, \href{works/abs-1911-04766.pdf}{abs-1911-04766}~\cite{abs-1911-04766}, \href{works/ArbaouiY18.pdf}{ArbaouiY18}~\cite{ArbaouiY18}, \href{works/Laborie18a.pdf}{Laborie18a}~\cite{Laborie18a}, \href{works/NishikawaSTT18a.pdf}{NishikawaSTT18a}~\cite{NishikawaSTT18a}, \href{works/NishikawaSTT18.pdf}{NishikawaSTT18}~\cite{NishikawaSTT18}, \href{works/CohenHB17.pdf}{CohenHB17}~\cite{CohenHB17}, \href{works/TranVNB17a.pdf}{TranVNB17a}~\cite{TranVNB17a}, \href{works/TranVNB17.pdf}{TranVNB17}~\cite{TranVNB17}, \href{works/CappartS17.pdf}{CappartS17}~\cite{CappartS17}... (Total: 35)\\
Constraints & alwaysIn & \href{works/PopovicCGNC22.pdf}{PopovicCGNC22}~\cite{PopovicCGNC22}, \href{works/SerraNM12.pdf}{SerraNM12}~\cite{SerraNM12} & \href{works/AalianPG23.pdf}{AalianPG23}~\cite{AalianPG23}, \href{works/LuoB22.pdf}{LuoB22}~\cite{LuoB22}, \href{works/TangB20.pdf}{TangB20}~\cite{TangB20}, \href{works/Polo-MejiaALB20.pdf}{Polo-MejiaALB20}~\cite{Polo-MejiaALB20}, \href{works/MalapertN19.pdf}{MalapertN19}~\cite{MalapertN19}, \href{works/LaborieRSV18.pdf}{LaborieRSV18}~\cite{LaborieRSV18}, \href{works/GoelSHFS15.pdf}{GoelSHFS15}~\cite{GoelSHFS15} & \href{works/CampeauG22.pdf}{CampeauG22}~\cite{CampeauG22}, \href{works/KreterSS17.pdf}{KreterSS17}~\cite{KreterSS17}, \href{works/BajestaniB13.pdf}{BajestaniB13}~\cite{BajestaniB13}\\
Constraints & bin-packing & \href{works/Godet21a.pdf}{Godet21a}~\cite{Godet21a}, \href{works/TangB20.pdf}{TangB20}~\cite{TangB20}, \href{works/CauwelaertLS18.pdf}{CauwelaertLS18}~\cite{CauwelaertLS18}, \href{works/LetortCB15.pdf}{LetortCB15}~\cite{LetortCB15}, \href{works/LetortCB13.pdf}{LetortCB13}~\cite{LetortCB13}, \href{works/HeinzSSW12.pdf}{HeinzSSW12}~\cite{HeinzSSW12}, \href{works/LetortBC12.pdf}{LetortBC12}~\cite{LetortBC12}, \href{works/Malapert11.pdf}{Malapert11}~\cite{Malapert11}, \href{works/SchausHMCMD11.pdf}{SchausHMCMD11}~\cite{SchausHMCMD11} & \href{works/LuoB22.pdf}{LuoB22}~\cite{LuoB22}, \href{works/BadicaBI20.pdf}{BadicaBI20}~\cite{BadicaBI20}, \href{works/AntunesABDEGGOL20.pdf}{AntunesABDEGGOL20}~\cite{AntunesABDEGGOL20}, \href{works/FrimodigS19.pdf}{FrimodigS19}~\cite{FrimodigS19}, \href{works/AntunesABDEGGOL18.pdf}{AntunesABDEGGOL18}~\cite{AntunesABDEGGOL18}, \href{works/BaptisteB18.pdf}{BaptisteB18}~\cite{BaptisteB18}, \href{works/GarganiR07.pdf}{GarganiR07}~\cite{GarganiR07}, \href{works/SakkoutW00.pdf}{SakkoutW00}~\cite{SakkoutW00}, \href{works/SchildW00.pdf}{SchildW00}~\cite{SchildW00} & \href{works/abs-2402-00459.pdf}{abs-2402-00459}~\cite{abs-2402-00459}, \href{works/LacknerMMWW23.pdf}{LacknerMMWW23}~\cite{LacknerMMWW23}, \href{works/AkramNHRSA23.pdf}{AkramNHRSA23}~\cite{AkramNHRSA23}, \href{works/abs-2211-14492.pdf}{abs-2211-14492}~\cite{abs-2211-14492}, \href{works/YunusogluY22.pdf}{YunusogluY22}~\cite{YunusogluY22}, \href{works/ArmstrongGOS21.pdf}{ArmstrongGOS21}~\cite{ArmstrongGOS21}, \href{works/GodetLHS20.pdf}{GodetLHS20}~\cite{GodetLHS20}, \href{works/HookerH18.pdf}{HookerH18}~\cite{HookerH18}, \href{works/TranPZLDB18.pdf}{TranPZLDB18}~\cite{TranPZLDB18}, \href{works/Madi-WambaLOBM17.pdf}{Madi-WambaLOBM17}~\cite{Madi-WambaLOBM17}, \href{works/DoulabiRP16.pdf}{DoulabiRP16}~\cite{DoulabiRP16}, \href{works/KoschB14.pdf}{KoschB14}~\cite{KoschB14}, \href{works/DoulabiRP14.pdf}{DoulabiRP14}~\cite{DoulabiRP14}, \href{works/LimtanyakulS12.pdf}{LimtanyakulS12}~\cite{LimtanyakulS12}, \href{works/EdisO11.pdf}{EdisO11}~\cite{EdisO11}, \href{works/HermenierDL11.pdf}{HermenierDL11}~\cite{HermenierDL11}, \href{works/BeldiceanuCDP11.pdf}{BeldiceanuCDP11}~\cite{BeldiceanuCDP11}, \href{works/HartmannB10.pdf}{HartmannB10}~\cite{HartmannB10}, \href{works/Lombardi10.pdf}{Lombardi10}~\cite{Lombardi10}, \href{works/KovacsB08.pdf}{KovacsB08}~\cite{KovacsB08}, \href{works/HentenryckM08.pdf}{HentenryckM08}~\cite{HentenryckM08}, \href{works/Simonis07.pdf}{Simonis07}~\cite{Simonis07}, \href{works/DavenportKRSH07.pdf}{DavenportKRSH07}~\cite{DavenportKRSH07}, \href{works/SimonisCK00.pdf}{SimonisCK00}~\cite{SimonisCK00}, \href{works/BeldiceanuC94.pdf}{BeldiceanuC94}~\cite{BeldiceanuC94}, \href{works/AggounB93.pdf}{AggounB93}~\cite{AggounB93}\\
Constraints & circuit & \href{works/MontemanniD23a.pdf}{MontemanniD23a}~\cite{MontemanniD23a}, \href{works/KlankeBYE21.pdf}{KlankeBYE21}~\cite{KlankeBYE21}, \href{works/Mercier-AubinGQ20.pdf}{Mercier-AubinGQ20}~\cite{Mercier-AubinGQ20}, \href{works/MokhtarzadehTNF20.pdf}{MokhtarzadehTNF20}~\cite{MokhtarzadehTNF20}, \href{works/HookerH18.pdf}{HookerH18}~\cite{HookerH18}, \href{works/Lombardi10.pdf}{Lombardi10}~\cite{Lombardi10}, \href{works/RuggieroBBMA09.pdf}{RuggieroBBMA09}~\cite{RuggieroBBMA09}, \href{works/Rodriguez07.pdf}{Rodriguez07}~\cite{Rodriguez07}, \href{works/RodriguezDG02.pdf}{RodriguezDG02}~\cite{RodriguezDG02}, \href{works/GruianK98.pdf}{GruianK98}~\cite{GruianK98}, \href{works/Wallace96.pdf}{Wallace96}~\cite{Wallace96}, \href{works/BeldiceanuC94.pdf}{BeldiceanuC94}~\cite{BeldiceanuC94} & \href{works/WessenCS20.pdf}{WessenCS20}~\cite{WessenCS20}, \href{works/AntuoriHHEN20.pdf}{AntuoriHHEN20}~\cite{AntuoriHHEN20}, \href{works/Siala15a.pdf}{Siala15a}~\cite{Siala15a}, \href{works/TranB12.pdf}{TranB12}~\cite{TranB12}, \href{works/Malapert11.pdf}{Malapert11}~\cite{Malapert11}, \href{works/KrogtLPHJ07.pdf}{KrogtLPHJ07}~\cite{KrogtLPHJ07}, \href{works/KuchcinskiW03.pdf}{KuchcinskiW03}~\cite{KuchcinskiW03}, \href{works/HookerO03.pdf}{HookerO03}~\cite{HookerO03}, \href{works/Thorsteinsson01.pdf}{Thorsteinsson01}~\cite{Thorsteinsson01}, \href{works/Simonis99.pdf}{Simonis99}~\cite{Simonis99}, \href{works/Simonis95a.pdf}{Simonis95a}~\cite{Simonis95a}, \href{works/DincbasSH90.pdf}{DincbasSH90}~\cite{DincbasSH90} & \href{works/PrataAN23.pdf}{PrataAN23}~\cite{PrataAN23}, \href{works/IsikYA23.pdf}{IsikYA23}~\cite{IsikYA23}, \href{works/MontemanniD23.pdf}{MontemanniD23}~\cite{MontemanniD23}, \href{works/JungblutK22.pdf}{JungblutK22}~\cite{JungblutK22}, \href{works/FarsiTM22.pdf}{FarsiTM22}~\cite{FarsiTM22}, \href{works/ColT22.pdf}{ColT22}~\cite{ColT22}, \href{works/MullerMKP22.pdf}{MullerMKP22}~\cite{MullerMKP22}, \href{works/KoehlerBFFHPSSS21.pdf}{KoehlerBFFHPSSS21}~\cite{KoehlerBFFHPSSS21}, \href{works/ArmstrongGOS21.pdf}{ArmstrongGOS21}~\cite{ArmstrongGOS21}, \href{works/Astrand21.pdf}{Astrand21}~\cite{Astrand21}, \href{works/WallaceY20.pdf}{WallaceY20}~\cite{WallaceY20}, \href{works/GroleazNS20.pdf}{GroleazNS20}~\cite{GroleazNS20}, \href{works/Hooker19.pdf}{Hooker19}~\cite{Hooker19}, \href{works/EscobetPQPRA19.pdf}{EscobetPQPRA19}~\cite{EscobetPQPRA19}, \href{works/CauwelaertLS18.pdf}{CauwelaertLS18}~\cite{CauwelaertLS18}, \href{works/TangLWSK18.pdf}{TangLWSK18}~\cite{TangLWSK18}, \href{works/CappartTSR18.pdf}{CappartTSR18}~\cite{CappartTSR18}, \href{works/Hooker17.pdf}{Hooker17}~\cite{Hooker17}, \href{works/HechingH16.pdf}{HechingH16}~\cite{HechingH16}, \href{works/Dejemeppe16.pdf}{Dejemeppe16}~\cite{Dejemeppe16}, \href{works/Bonfietti16.pdf}{Bonfietti16}~\cite{Bonfietti16}, \href{works/BridiBLMB16.pdf}{BridiBLMB16}~\cite{BridiBLMB16}, \href{works/TranAB16.pdf}{TranAB16}~\cite{TranAB16}, \href{works/MelgarejoLS15.pdf}{MelgarejoLS15}~\cite{MelgarejoLS15}, \href{works/MurphyMB15.pdf}{MurphyMB15}~\cite{MurphyMB15}, \href{works/Derrien15.pdf}{Derrien15}~\cite{Derrien15}, \href{works/BajestaniB15.pdf}{BajestaniB15}~\cite{BajestaniB15}, \href{works/HoundjiSWD14.pdf}{HoundjiSWD14}~\cite{HoundjiSWD14}, \href{works/BonfiettiLBM14.pdf}{BonfiettiLBM14}~\cite{BonfiettiLBM14}... (Total: 54)\\
Constraints & cumulative & \href{works/PovedaAA23.pdf}{PovedaAA23}~\cite{PovedaAA23}, \href{works/TardivoDFMP23.pdf}{TardivoDFMP23}~\cite{TardivoDFMP23}, \href{works/NaderiRR23.pdf}{NaderiRR23}~\cite{NaderiRR23}, \href{works/AalianPG23.pdf}{AalianPG23}~\cite{AalianPG23}, \href{works/KameugneFND23.pdf}{KameugneFND23}~\cite{KameugneFND23}, \href{works/IsikYA23.pdf}{IsikYA23}~\cite{IsikYA23}, \href{works/LacknerMMWW23.pdf}{LacknerMMWW23}~\cite{LacknerMMWW23}, \href{works/FetgoD22.pdf}{FetgoD22}~\cite{FetgoD22}, \href{works/PohlAK22.pdf}{PohlAK22}~\cite{PohlAK22}, \href{works/OuelletQ22.pdf}{OuelletQ22}~\cite{OuelletQ22}, \href{works/ZhangJZL22.pdf}{ZhangJZL22}~\cite{ZhangJZL22}, \href{works/LuoB22.pdf}{LuoB22}~\cite{LuoB22}, \href{works/BoudreaultSLQ22.pdf}{BoudreaultSLQ22}~\cite{BoudreaultSLQ22}, \href{works/Lemos21.pdf}{Lemos21}~\cite{Lemos21}, \href{works/LacknerMMWW21.pdf}{LacknerMMWW21}~\cite{LacknerMMWW21}, \href{works/HanenKP21.pdf}{HanenKP21}~\cite{HanenKP21}, \href{works/KovacsTKSG21.pdf}{KovacsTKSG21}~\cite{KovacsTKSG21}, \href{works/Godet21a.pdf}{Godet21a}~\cite{Godet21a}, \href{works/SacramentoSP20.pdf}{SacramentoSP20}~\cite{SacramentoSP20}, \href{works/Polo-MejiaALB20.pdf}{Polo-MejiaALB20}~\cite{Polo-MejiaALB20}, \href{works/Mercier-AubinGQ20.pdf}{Mercier-AubinGQ20}~\cite{Mercier-AubinGQ20}, \href{works/WallaceY20.pdf}{WallaceY20}~\cite{WallaceY20}, \href{works/GodetLHS20.pdf}{GodetLHS20}~\cite{GodetLHS20}, \href{works/GroleazNS20a.pdf}{GroleazNS20a}~\cite{GroleazNS20a}, \href{works/GroleazNS20.pdf}{GroleazNS20}~\cite{GroleazNS20}, \href{works/Hooker19.pdf}{Hooker19}~\cite{Hooker19}, \href{works/YangSS19.pdf}{YangSS19}~\cite{YangSS19}, \href{works/abs-1911-04766.pdf}{abs-1911-04766}~\cite{abs-1911-04766}, \href{works/Novas19.pdf}{Novas19}~\cite{Novas19}... (Total: 144) & \href{works/PrataAN23.pdf}{PrataAN23}~\cite{PrataAN23}, \href{works/abs-2402-00459.pdf}{abs-2402-00459}~\cite{abs-2402-00459}, \href{works/EfthymiouY23.pdf}{EfthymiouY23}~\cite{EfthymiouY23}, \href{works/abs-2312-13682.pdf}{abs-2312-13682}~\cite{abs-2312-13682}, \href{works/PerezGSL23.pdf}{PerezGSL23}~\cite{PerezGSL23}, \href{works/ColT22.pdf}{ColT22}~\cite{ColT22}, \href{works/YunusogluY22.pdf}{YunusogluY22}~\cite{YunusogluY22}, \href{works/CampeauG22.pdf}{CampeauG22}~\cite{CampeauG22}, \href{works/GeitzGSSW22.pdf}{GeitzGSSW22}~\cite{GeitzGSSW22}, \href{works/AbreuN22.pdf}{AbreuN22}~\cite{AbreuN22}, \href{works/HubnerGSV21.pdf}{HubnerGSV21}~\cite{HubnerGSV21}, \href{works/HillTV21.pdf}{HillTV21}~\cite{HillTV21}, \href{works/KlankeBYE21.pdf}{KlankeBYE21}~\cite{KlankeBYE21}, \href{works/NattafM20.pdf}{NattafM20}~\cite{NattafM20}, \href{works/GalleguillosKSB19.pdf}{GalleguillosKSB19}~\cite{GalleguillosKSB19}, \href{works/NishikawaSTT19.pdf}{NishikawaSTT19}~\cite{NishikawaSTT19}, \href{works/BorghesiBLMB18.pdf}{BorghesiBLMB18}~\cite{BorghesiBLMB18}, \href{works/GedikKEK18.pdf}{GedikKEK18}~\cite{GedikKEK18}, \href{works/TranVNB17a.pdf}{TranVNB17a}~\cite{TranVNB17a}, \href{works/HurleyOS16.pdf}{HurleyOS16}~\cite{HurleyOS16}, \href{works/BoothNB16.pdf}{BoothNB16}~\cite{BoothNB16}, \href{works/BonfiettiZLM16.pdf}{BonfiettiZLM16}~\cite{BonfiettiZLM16}, \href{works/LimHTB16.pdf}{LimHTB16}~\cite{LimHTB16}, \href{works/Bonfietti16.pdf}{Bonfietti16}~\cite{Bonfietti16}, \href{works/GayHLS15.pdf}{GayHLS15}~\cite{GayHLS15}, \href{works/BurtLPS15.pdf}{BurtLPS15}~\cite{BurtLPS15}, \href{works/ThiruvadyWGS14.pdf}{ThiruvadyWGS14}~\cite{ThiruvadyWGS14}, \href{works/GuSS13.pdf}{GuSS13}~\cite{GuSS13}, \href{works/BonfiettiLM13.pdf}{BonfiettiLM13}~\cite{BonfiettiLM13}... (Total: 48) & \href{works/GurPAE23.pdf}{GurPAE23}~\cite{GurPAE23}, \href{works/TasselGS23.pdf}{TasselGS23}~\cite{TasselGS23}, \href{works/abs-2306-05747.pdf}{abs-2306-05747}~\cite{abs-2306-05747}, \href{works/abs-2305-19888.pdf}{abs-2305-19888}~\cite{abs-2305-19888}, \href{works/Bit-Monnot23.pdf}{Bit-Monnot23}~\cite{Bit-Monnot23}, \href{works/YuraszeckMCCR23.pdf}{YuraszeckMCCR23}~\cite{YuraszeckMCCR23}, \href{works/JuvinHHL23.pdf}{JuvinHHL23}~\cite{JuvinHHL23}, \href{works/HeinzNVH22.pdf}{HeinzNVH22}~\cite{HeinzNVH22}, \href{works/PopovicCGNC22.pdf}{PopovicCGNC22}~\cite{PopovicCGNC22}, \href{works/abs-2211-14492.pdf}{abs-2211-14492}~\cite{abs-2211-14492}, \href{works/SubulanC22.pdf}{SubulanC22}~\cite{SubulanC22}, \href{works/HebrardALLCMR22.pdf}{HebrardALLCMR22}~\cite{HebrardALLCMR22}, \href{works/ArmstrongGOS22.pdf}{ArmstrongGOS22}~\cite{ArmstrongGOS22}, \href{works/Astrand21.pdf}{Astrand21}~\cite{Astrand21}, \href{works/PandeyS21a.pdf}{PandeyS21a}~\cite{PandeyS21a}, \href{works/KoehlerBFFHPSSS21.pdf}{KoehlerBFFHPSSS21}~\cite{KoehlerBFFHPSSS21}, \href{works/GeibingerMM21.pdf}{GeibingerMM21}~\cite{GeibingerMM21}, \href{works/ArmstrongGOS21.pdf}{ArmstrongGOS21}~\cite{ArmstrongGOS21}, \href{works/ZouZ20.pdf}{ZouZ20}~\cite{ZouZ20}, \href{works/CauwelaertDS20.pdf}{CauwelaertDS20}~\cite{CauwelaertDS20}, \href{works/abs-1902-09244.pdf}{abs-1902-09244}~\cite{abs-1902-09244}, \href{works/FrimodigS19.pdf}{FrimodigS19}~\cite{FrimodigS19}, \href{works/WikarekS19.pdf}{WikarekS19}~\cite{WikarekS19}, \href{works/YounespourAKE19.pdf}{YounespourAKE19}~\cite{YounespourAKE19}, \href{works/Laborie18a.pdf}{Laborie18a}~\cite{Laborie18a}, \href{works/AstrandJZ18.pdf}{AstrandJZ18}~\cite{AstrandJZ18}, \href{works/ZhangW18.pdf}{ZhangW18}~\cite{ZhangW18}, \href{works/Ham18.pdf}{Ham18}~\cite{Ham18}, \href{works/ArbaouiY18.pdf}{ArbaouiY18}~\cite{ArbaouiY18}... (Total: 98)\\
Constraints & cycle & \href{works/AalianPG23.pdf}{AalianPG23}~\cite{AalianPG23}, \href{works/Astrand0F21.pdf}{Astrand0F21}~\cite{Astrand0F21}, \href{works/Astrand21.pdf}{Astrand21}~\cite{Astrand21}, \href{works/AntuoriHHEN21.pdf}{AntuoriHHEN21}~\cite{AntuoriHHEN21}, \href{works/AbohashimaEG21.pdf}{AbohashimaEG21}~\cite{AbohashimaEG21}, \href{works/GroleazNS20a.pdf}{GroleazNS20a}~\cite{GroleazNS20a}, \href{works/AntuoriHHEN20.pdf}{AntuoriHHEN20}~\cite{AntuoriHHEN20}, \href{works/WallaceY20.pdf}{WallaceY20}~\cite{WallaceY20}, \href{works/AstrandJZ20.pdf}{AstrandJZ20}~\cite{AstrandJZ20}, \href{works/ParkUJR19.pdf}{ParkUJR19}~\cite{ParkUJR19}, \href{works/BorghesiBLMB18.pdf}{BorghesiBLMB18}~\cite{BorghesiBLMB18}, \href{works/AstrandJZ18.pdf}{AstrandJZ18}~\cite{AstrandJZ18}, \href{works/Dejemeppe16.pdf}{Dejemeppe16}~\cite{Dejemeppe16}, \href{works/BridiBLMB16.pdf}{BridiBLMB16}~\cite{BridiBLMB16}, \href{works/BonfiettiLBM14.pdf}{BonfiettiLBM14}~\cite{BonfiettiLBM14}, \href{works/BessiereHMQW14.pdf}{BessiereHMQW14}~\cite{BessiereHMQW14}, \href{works/BegB13.pdf}{BegB13}~\cite{BegB13}, \href{works/Malapert11.pdf}{Malapert11}~\cite{Malapert11}, \href{works/LombardiBMB11.pdf}{LombardiBMB11}~\cite{LombardiBMB11}, \href{works/SunLYL10.pdf}{SunLYL10}~\cite{SunLYL10}, \href{works/BocewiczBB09.pdf}{BocewiczBB09}~\cite{BocewiczBB09}, \href{works/RuggieroBBMA09.pdf}{RuggieroBBMA09}~\cite{RuggieroBBMA09}, \href{works/MalikMB08.pdf}{MalikMB08}~\cite{MalikMB08}, \href{works/Malik08.pdf}{Malik08}~\cite{Malik08}, \href{works/RossiTHP07.pdf}{RossiTHP07}~\cite{RossiTHP07}, \href{works/WolinskiKG04.pdf}{WolinskiKG04}~\cite{WolinskiKG04}, \href{works/KuchcinskiW03.pdf}{KuchcinskiW03}~\cite{KuchcinskiW03}, \href{works/Kumar03.pdf}{Kumar03}~\cite{Kumar03}, \href{works/ArtiguesR00.pdf}{ArtiguesR00}~\cite{ArtiguesR00}... (Total: 38) & \href{works/EfthymiouY23.pdf}{EfthymiouY23}~\cite{EfthymiouY23}, \href{works/CampeauG22.pdf}{CampeauG22}~\cite{CampeauG22}, \href{works/Lemos21.pdf}{Lemos21}~\cite{Lemos21}, \href{works/KoehlerBFFHPSSS21.pdf}{KoehlerBFFHPSSS21}~\cite{KoehlerBFFHPSSS21}, \href{works/HillTV21.pdf}{HillTV21}~\cite{HillTV21}, \href{works/HubnerGSV21.pdf}{HubnerGSV21}~\cite{HubnerGSV21}, \href{works/Godet21a.pdf}{Godet21a}~\cite{Godet21a}, \href{works/CauwelaertDS20.pdf}{CauwelaertDS20}~\cite{CauwelaertDS20}, \href{works/GroleazNS20.pdf}{GroleazNS20}~\cite{GroleazNS20}, \href{works/Lunardi20.pdf}{Lunardi20}~\cite{Lunardi20}, \href{works/ZarandiASC20.pdf}{ZarandiASC20}~\cite{ZarandiASC20}, \href{works/MossigeGSMC17.pdf}{MossigeGSMC17}~\cite{MossigeGSMC17}, \href{works/TranAB16.pdf}{TranAB16}~\cite{TranAB16}, \href{works/SimoninAHL15.pdf}{SimoninAHL15}~\cite{SimoninAHL15}, \href{works/PraletLJ15.pdf}{PraletLJ15}~\cite{PraletLJ15}, \href{works/BurtLPS15.pdf}{BurtLPS15}~\cite{BurtLPS15}, \href{works/Siala15a.pdf}{Siala15a}~\cite{Siala15a}, \href{works/TranTDB13.pdf}{TranTDB13}~\cite{TranTDB13}, \href{works/SchuttFSW13.pdf}{SchuttFSW13}~\cite{SchuttFSW13}, \href{works/SimoninAHL12.pdf}{SimoninAHL12}~\cite{SimoninAHL12}, \href{works/BonfiettiLBM12.pdf}{BonfiettiLBM12}~\cite{BonfiettiLBM12}, \href{works/HachemiGR11.pdf}{HachemiGR11}~\cite{HachemiGR11}, \href{works/KovacsB11.pdf}{KovacsB11}~\cite{KovacsB11}, \href{works/BonfiettiLBM11.pdf}{BonfiettiLBM11}~\cite{BonfiettiLBM11}, \href{works/Vilim11.pdf}{Vilim11}~\cite{Vilim11}, \href{works/Lombardi10.pdf}{Lombardi10}~\cite{Lombardi10}, \href{works/abs-1009-0347.pdf}{abs-1009-0347}~\cite{abs-1009-0347}, \href{works/KovacsB08.pdf}{KovacsB08}~\cite{KovacsB08}, \href{works/Simonis07.pdf}{Simonis07}~\cite{Simonis07}... (Total: 39) & \href{works/Bit-Monnot23.pdf}{Bit-Monnot23}~\cite{Bit-Monnot23}, \href{works/AkramNHRSA23.pdf}{AkramNHRSA23}~\cite{AkramNHRSA23}, \href{works/ZhangBB22.pdf}{ZhangBB22}~\cite{ZhangBB22}, \href{works/BourreauGGLT22.pdf}{BourreauGGLT22}~\cite{BourreauGGLT22}, \href{works/AbreuN22.pdf}{AbreuN22}~\cite{AbreuN22}, \href{works/HamPK21.pdf}{HamPK21}~\cite{HamPK21}, \href{works/ArmstrongGOS21.pdf}{ArmstrongGOS21}~\cite{ArmstrongGOS21}, \href{works/AbreuAPNM21.pdf}{AbreuAPNM21}~\cite{AbreuAPNM21}, \href{works/FanXG21.pdf}{FanXG21}~\cite{FanXG21}, \href{works/FallahiAC20.pdf}{FallahiAC20}~\cite{FallahiAC20}, \href{works/TangB20.pdf}{TangB20}~\cite{TangB20}, \href{works/Mercier-AubinGQ20.pdf}{Mercier-AubinGQ20}~\cite{Mercier-AubinGQ20}, \href{works/QinDCS20.pdf}{QinDCS20}~\cite{QinDCS20}, \href{works/BadicaBI20.pdf}{BadicaBI20}~\cite{BadicaBI20}, \href{works/MokhtarzadehTNF20.pdf}{MokhtarzadehTNF20}~\cite{MokhtarzadehTNF20}, \href{works/Novas19.pdf}{Novas19}~\cite{Novas19}, \href{works/Hooker19.pdf}{Hooker19}~\cite{Hooker19}, \href{works/BadicaBIL19.pdf}{BadicaBIL19}~\cite{BadicaBIL19}, \href{works/abs-1902-09244.pdf}{abs-1902-09244}~\cite{abs-1902-09244}, \href{works/KucukY19.pdf}{KucukY19}~\cite{KucukY19}, \href{works/EscobetPQPRA19.pdf}{EscobetPQPRA19}~\cite{EscobetPQPRA19}, \href{works/TangLWSK18.pdf}{TangLWSK18}~\cite{TangLWSK18}, \href{works/MusliuSS18.pdf}{MusliuSS18}~\cite{MusliuSS18}, \href{works/LaborieRSV18.pdf}{LaborieRSV18}~\cite{LaborieRSV18}, \href{works/Ham18.pdf}{Ham18}~\cite{Ham18}, \href{works/KreterSS17.pdf}{KreterSS17}~\cite{KreterSS17}, \href{works/Pralet17.pdf}{Pralet17}~\cite{Pralet17}, \href{works/DoulabiRP16.pdf}{DoulabiRP16}~\cite{DoulabiRP16}, \href{works/TranDRFWOVB16.pdf}{TranDRFWOVB16}~\cite{TranDRFWOVB16}... (Total: 76)\\
Constraints & diffn & \href{works/ArmstrongGOS21.pdf}{ArmstrongGOS21}~\cite{ArmstrongGOS21}, \href{works/Simonis07.pdf}{Simonis07}~\cite{Simonis07}, \href{works/SimonisCK00.pdf}{SimonisCK00}~\cite{SimonisCK00}, \href{works/BeldiceanuC94.pdf}{BeldiceanuC94}~\cite{BeldiceanuC94} & \href{works/BeldiceanuCDP11.pdf}{BeldiceanuCDP11}~\cite{BeldiceanuCDP11} & \href{works/LuoB22.pdf}{LuoB22}~\cite{LuoB22}, \href{works/BourreauGGLT22.pdf}{BourreauGGLT22}~\cite{BourreauGGLT22}, \href{works/KreterSS17.pdf}{KreterSS17}~\cite{KreterSS17}, \href{works/KreterSS15.pdf}{KreterSS15}~\cite{KreterSS15}, \href{works/TrojetHL11.pdf}{TrojetHL11}~\cite{TrojetHL11}, \href{works/Malapert11.pdf}{Malapert11}~\cite{Malapert11}, \href{works/Timpe02.pdf}{Timpe02}~\cite{Timpe02}, \href{works/Simonis99.pdf}{Simonis99}~\cite{Simonis99}, \href{works/GruianK98.pdf}{GruianK98}~\cite{GruianK98}, \href{works/SimonisC95.pdf}{SimonisC95}~\cite{SimonisC95}, \href{works/Simonis95a.pdf}{Simonis95a}~\cite{Simonis95a}, \href{works/Simonis95.pdf}{Simonis95}~\cite{Simonis95}\\
Constraints & disjunctive & \href{works/JuvinHHL23.pdf}{JuvinHHL23}~\cite{JuvinHHL23}, \href{works/NaderiRR23.pdf}{NaderiRR23}~\cite{NaderiRR23}, \href{works/Bit-Monnot23.pdf}{Bit-Monnot23}~\cite{Bit-Monnot23}, \href{works/YuraszeckMPV22.pdf}{YuraszeckMPV22}~\cite{YuraszeckMPV22}, \href{works/BourreauGGLT22.pdf}{BourreauGGLT22}~\cite{BourreauGGLT22}, \href{works/ZhangBB22.pdf}{ZhangBB22}~\cite{ZhangBB22}, \href{works/Astrand21.pdf}{Astrand21}~\cite{Astrand21}, \href{works/Godet21a.pdf}{Godet21a}~\cite{Godet21a}, \href{works/KoehlerBFFHPSSS21.pdf}{KoehlerBFFHPSSS21}~\cite{KoehlerBFFHPSSS21}, \href{works/GodetLHS20.pdf}{GodetLHS20}~\cite{GodetLHS20}, \href{works/LaborieRSV18.pdf}{LaborieRSV18}~\cite{LaborieRSV18}, \href{works/HookerH18.pdf}{HookerH18}~\cite{HookerH18}, \href{works/FahimiOQ18.pdf}{FahimiOQ18}~\cite{FahimiOQ18}, \href{works/GokgurHO18.pdf}{GokgurHO18}~\cite{GokgurHO18}, \href{works/NattafAL17.pdf}{NattafAL17}~\cite{NattafAL17}, \href{works/Pralet17.pdf}{Pralet17}~\cite{Pralet17}, \href{works/MossigeGSMC17.pdf}{MossigeGSMC17}~\cite{MossigeGSMC17}, \href{works/KuB16.pdf}{KuB16}~\cite{KuB16}, \href{works/FontaineMH16.pdf}{FontaineMH16}~\cite{FontaineMH16}, \href{works/GoelSHFS15.pdf}{GoelSHFS15}~\cite{GoelSHFS15}, \href{works/Siala15a.pdf}{Siala15a}~\cite{Siala15a}, \href{works/GayHS15a.pdf}{GayHS15a}~\cite{GayHS15a}, \href{works/MelgarejoLS15.pdf}{MelgarejoLS15}~\cite{MelgarejoLS15}, \href{works/GrimesH15.pdf}{GrimesH15}~\cite{GrimesH15}, \href{works/SialaAH15.pdf}{SialaAH15}~\cite{SialaAH15}, \href{works/SchuttFS13.pdf}{SchuttFS13}~\cite{SchuttFS13}, \href{works/OzturkTHO13.pdf}{OzturkTHO13}~\cite{OzturkTHO13}, \href{works/SchuttFS13a.pdf}{SchuttFS13a}~\cite{SchuttFS13a}, \href{works/LombardiM12.pdf}{LombardiM12}~\cite{LombardiM12}... (Total: 69) & \href{works/BoudreaultSLQ22.pdf}{BoudreaultSLQ22}~\cite{BoudreaultSLQ22}, \href{works/Astrand0F21.pdf}{Astrand0F21}~\cite{Astrand0F21}, \href{works/GeibingerMM21.pdf}{GeibingerMM21}~\cite{GeibingerMM21}, \href{works/SacramentoSP20.pdf}{SacramentoSP20}~\cite{SacramentoSP20}, \href{works/AstrandJZ20.pdf}{AstrandJZ20}~\cite{AstrandJZ20}, \href{works/MejiaY20.pdf}{MejiaY20}~\cite{MejiaY20}, \href{works/Polo-MejiaALB20.pdf}{Polo-MejiaALB20}~\cite{Polo-MejiaALB20}, \href{works/YangSS19.pdf}{YangSS19}~\cite{YangSS19}, \href{works/CauwelaertLS18.pdf}{CauwelaertLS18}~\cite{CauwelaertLS18}, \href{works/DemirovicS18.pdf}{DemirovicS18}~\cite{DemirovicS18}, \href{works/KameugneFGOQ18.pdf}{KameugneFGOQ18}~\cite{KameugneFGOQ18}, \href{works/Dejemeppe16.pdf}{Dejemeppe16}~\cite{Dejemeppe16}, \href{works/SimoninAHL15.pdf}{SimoninAHL15}~\cite{SimoninAHL15}, \href{works/EvenSH15.pdf}{EvenSH15}~\cite{EvenSH15}, \href{works/EvenSH15a.pdf}{EvenSH15a}~\cite{EvenSH15a}, \href{works/GayHS15.pdf}{GayHS15}~\cite{GayHS15}, \href{works/VilimLS15.pdf}{VilimLS15}~\cite{VilimLS15}, \href{works/LipovetzkyBPS14.pdf}{LipovetzkyBPS14}~\cite{LipovetzkyBPS14}, \href{works/KameugneFSN14.pdf}{KameugneFSN14}~\cite{KameugneFSN14}, \href{works/GaySS14.pdf}{GaySS14}~\cite{GaySS14}, \href{works/KelbelH11.pdf}{KelbelH11}~\cite{KelbelH11}, \href{works/HeinzS11.pdf}{HeinzS11}~\cite{HeinzS11}, \href{works/GrimesH11.pdf}{GrimesH11}~\cite{GrimesH11}, \href{works/HartmannB10.pdf}{HartmannB10}~\cite{HartmannB10}, \href{works/LiessM08.pdf}{LiessM08}~\cite{LiessM08}, \href{works/MouraSCL08a.pdf}{MouraSCL08a}~\cite{MouraSCL08a}, \href{works/MercierH08.pdf}{MercierH08}~\cite{MercierH08}, \href{works/MouraSCL08.pdf}{MouraSCL08}~\cite{MouraSCL08}, \href{works/MonetteDD07.pdf}{MonetteDD07}~\cite{MonetteDD07}... (Total: 37) & \href{works/abs-2402-00459.pdf}{abs-2402-00459}~\cite{abs-2402-00459}, \href{works/LacknerMMWW23.pdf}{LacknerMMWW23}~\cite{LacknerMMWW23}, \href{works/TardivoDFMP23.pdf}{TardivoDFMP23}~\cite{TardivoDFMP23}, \href{works/abs-2306-05747.pdf}{abs-2306-05747}~\cite{abs-2306-05747}, \href{works/KameugneFND23.pdf}{KameugneFND23}~\cite{KameugneFND23}, \href{works/PovedaAA23.pdf}{PovedaAA23}~\cite{PovedaAA23}, \href{works/EfthymiouY23.pdf}{EfthymiouY23}~\cite{EfthymiouY23}, \href{works/TasselGS23.pdf}{TasselGS23}~\cite{TasselGS23}, \href{works/NaderiBZ22.pdf}{NaderiBZ22}~\cite{NaderiBZ22}, \href{works/MullerMKP22.pdf}{MullerMKP22}~\cite{MullerMKP22}, \href{works/OuelletQ22.pdf}{OuelletQ22}~\cite{OuelletQ22}, \href{works/ColT22.pdf}{ColT22}~\cite{ColT22}, \href{works/abs-2211-14492.pdf}{abs-2211-14492}~\cite{abs-2211-14492}, \href{works/OujanaAYB22.pdf}{OujanaAYB22}~\cite{OujanaAYB22}, \href{works/KlankeBYE21.pdf}{KlankeBYE21}~\cite{KlankeBYE21}, \href{works/ZhangYW21.pdf}{ZhangYW21}~\cite{ZhangYW21}, \href{works/Lunardi20.pdf}{Lunardi20}~\cite{Lunardi20}, \href{works/ZarandiASC20.pdf}{ZarandiASC20}~\cite{ZarandiASC20}, \href{works/Mercier-AubinGQ20.pdf}{Mercier-AubinGQ20}~\cite{Mercier-AubinGQ20}, \href{works/CauwelaertDS20.pdf}{CauwelaertDS20}~\cite{CauwelaertDS20}, \href{works/WallaceY20.pdf}{WallaceY20}~\cite{WallaceY20}, \href{works/KucukY19.pdf}{KucukY19}~\cite{KucukY19}, \href{works/abs-1911-04766.pdf}{abs-1911-04766}~\cite{abs-1911-04766}, \href{works/WikarekS19.pdf}{WikarekS19}~\cite{WikarekS19}, \href{works/ColT19.pdf}{ColT19}~\cite{ColT19}, \href{works/Hooker19.pdf}{Hooker19}~\cite{Hooker19}, \href{works/AstrandJZ18.pdf}{AstrandJZ18}~\cite{AstrandJZ18}, \href{works/OuelletQ18.pdf}{OuelletQ18}~\cite{OuelletQ18}, \href{works/CappartTSR18.pdf}{CappartTSR18}~\cite{CappartTSR18}... (Total: 124)\\
Constraints & endBeforeStart & \href{works/SubulanC22.pdf}{SubulanC22}~\cite{SubulanC22}, \href{works/QinDCS20.pdf}{QinDCS20}~\cite{QinDCS20} & \href{works/NaderiRR23.pdf}{NaderiRR23}~\cite{NaderiRR23}, \href{works/IsikYA23.pdf}{IsikYA23}~\cite{IsikYA23}, \href{works/PandeyS21a.pdf}{PandeyS21a}~\cite{PandeyS21a}, \href{works/LunardiBLRV20.pdf}{LunardiBLRV20}~\cite{LunardiBLRV20}, \href{works/Lunardi20.pdf}{Lunardi20}~\cite{Lunardi20}, \href{works/MengZRZL20.pdf}{MengZRZL20}~\cite{MengZRZL20}, \href{works/LaborieRSV18.pdf}{LaborieRSV18}~\cite{LaborieRSV18}, \href{works/NovaraNH16.pdf}{NovaraNH16}~\cite{NovaraNH16}, \href{works/Laborie09.pdf}{Laborie09}~\cite{Laborie09} & \href{works/JuvinHHL23.pdf}{JuvinHHL23}~\cite{JuvinHHL23}, \href{works/YuraszeckMCCR23.pdf}{YuraszeckMCCR23}~\cite{YuraszeckMCCR23}, \href{works/CzerniachowskaWZ23.pdf}{CzerniachowskaWZ23}~\cite{CzerniachowskaWZ23}, \href{works/LacknerMMWW23.pdf}{LacknerMMWW23}~\cite{LacknerMMWW23}, \href{works/JuvinHL23.pdf}{JuvinHL23}~\cite{JuvinHL23}, \href{works/AalianPG23.pdf}{AalianPG23}~\cite{AalianPG23}, \href{works/Teppan22.pdf}{Teppan22}~\cite{Teppan22}, \href{works/YunusogluY22.pdf}{YunusogluY22}~\cite{YunusogluY22}, \href{works/CampeauG22.pdf}{CampeauG22}~\cite{CampeauG22}, \href{works/ZhangJZL22.pdf}{ZhangJZL22}~\cite{ZhangJZL22}, \href{works/HamPK21.pdf}{HamPK21}~\cite{HamPK21}, \href{works/HubnerGSV21.pdf}{HubnerGSV21}~\cite{HubnerGSV21}, \href{works/ZhangYW21.pdf}{ZhangYW21}~\cite{ZhangYW21}, \href{works/LacknerMMWW21.pdf}{LacknerMMWW21}~\cite{LacknerMMWW21}, \href{works/TangB20.pdf}{TangB20}~\cite{TangB20}, \href{works/ZouZ20.pdf}{ZouZ20}~\cite{ZouZ20}, \href{works/SacramentoSP20.pdf}{SacramentoSP20}~\cite{SacramentoSP20}, \href{works/BenediktMH20.pdf}{BenediktMH20}~\cite{BenediktMH20}, \href{works/Polo-MejiaALB20.pdf}{Polo-MejiaALB20}~\cite{Polo-MejiaALB20}, \href{works/MurinR19.pdf}{MurinR19}~\cite{MurinR19}, \href{works/abs-1902-09244.pdf}{abs-1902-09244}~\cite{abs-1902-09244}, \href{works/ParkUJR19.pdf}{ParkUJR19}~\cite{ParkUJR19}, \href{works/GeibingerMM19.pdf}{GeibingerMM19}~\cite{GeibingerMM19}, \href{works/abs-1911-04766.pdf}{abs-1911-04766}~\cite{abs-1911-04766}, \href{works/Novas19.pdf}{Novas19}~\cite{Novas19}, \href{works/NishikawaSTT18a.pdf}{NishikawaSTT18a}~\cite{NishikawaSTT18a}, \href{works/NishikawaSTT18.pdf}{NishikawaSTT18}~\cite{NishikawaSTT18}, \href{works/Ham18.pdf}{Ham18}~\cite{Ham18}, \href{works/HamC16.pdf}{HamC16}~\cite{HamC16}, \href{works/GrimesH15.pdf}{GrimesH15}~\cite{GrimesH15}\\
Constraints & geost & \href{works/BeldiceanuCDP11.pdf}{BeldiceanuCDP11}~\cite{BeldiceanuCDP11} & \href{works/LetortBC12.pdf}{LetortBC12}~\cite{LetortBC12}, \href{works/PembertonG98.pdf}{PembertonG98}~\cite{PembertonG98} & \href{works/Malapert11.pdf}{Malapert11}~\cite{Malapert11}, \href{works/BeldiceanuCP08.pdf}{BeldiceanuCP08}~\cite{BeldiceanuCP08}\\
Constraints & noOverlap & \href{works/abs-2305-19888.pdf}{abs-2305-19888}~\cite{abs-2305-19888}, \href{works/NaderiRR23.pdf}{NaderiRR23}~\cite{NaderiRR23}, \href{works/IsikYA23.pdf}{IsikYA23}~\cite{IsikYA23}, \href{works/JuvinHHL23.pdf}{JuvinHHL23}~\cite{JuvinHHL23}, \href{works/HeinzNVH22.pdf}{HeinzNVH22}~\cite{HeinzNVH22}, \href{works/ColT22.pdf}{ColT22}~\cite{ColT22}, \href{works/PopovicCGNC22.pdf}{PopovicCGNC22}~\cite{PopovicCGNC22}, \href{works/VlkHT21.pdf}{VlkHT21}~\cite{VlkHT21}, \href{works/LunardiBLRV20.pdf}{LunardiBLRV20}~\cite{LunardiBLRV20}, \href{works/Lunardi20.pdf}{Lunardi20}~\cite{Lunardi20}, \href{works/QinDCS20.pdf}{QinDCS20}~\cite{QinDCS20}, \href{works/GedikKEK18.pdf}{GedikKEK18}~\cite{GedikKEK18}, \href{works/MelgarejoLS15.pdf}{MelgarejoLS15}~\cite{MelgarejoLS15} & \href{works/KimCMLLP23.pdf}{KimCMLLP23}~\cite{KimCMLLP23}, \href{works/abs-2306-05747.pdf}{abs-2306-05747}~\cite{abs-2306-05747}, \href{works/LacknerMMWW23.pdf}{LacknerMMWW23}~\cite{LacknerMMWW23}, \href{works/TasselGS23.pdf}{TasselGS23}~\cite{TasselGS23}, \href{works/AbreuN22.pdf}{AbreuN22}~\cite{AbreuN22}, \href{works/YuraszeckMPV22.pdf}{YuraszeckMPV22}~\cite{YuraszeckMPV22}, \href{works/PohlAK22.pdf}{PohlAK22}~\cite{PohlAK22}, \href{works/SvancaraB22.pdf}{SvancaraB22}~\cite{SvancaraB22}, \href{works/KlankeBYE21.pdf}{KlankeBYE21}~\cite{KlankeBYE21}, \href{works/Bedhief21.pdf}{Bedhief21}~\cite{Bedhief21}, \href{works/BenderWS21.pdf}{BenderWS21}~\cite{BenderWS21}, \href{works/BenediktMH20.pdf}{BenediktMH20}~\cite{BenediktMH20}, \href{works/MengZRZL20.pdf}{MengZRZL20}~\cite{MengZRZL20}, \href{works/ZouZ20.pdf}{ZouZ20}~\cite{ZouZ20}, \href{works/SacramentoSP20.pdf}{SacramentoSP20}~\cite{SacramentoSP20}, \href{works/YounespourAKE19.pdf}{YounespourAKE19}~\cite{YounespourAKE19}, \href{works/MalapertN19.pdf}{MalapertN19}~\cite{MalapertN19}, \href{works/MurinR19.pdf}{MurinR19}~\cite{MurinR19}, \href{works/abs-1911-04766.pdf}{abs-1911-04766}~\cite{abs-1911-04766}, \href{works/EscobetPQPRA19.pdf}{EscobetPQPRA19}~\cite{EscobetPQPRA19}, \href{works/Novas19.pdf}{Novas19}~\cite{Novas19}, \href{works/LaborieRSV18.pdf}{LaborieRSV18}~\cite{LaborieRSV18}, \href{works/ZhangW18.pdf}{ZhangW18}~\cite{ZhangW18}, \href{works/ArbaouiY18.pdf}{ArbaouiY18}~\cite{ArbaouiY18}, \href{works/Ham18.pdf}{Ham18}~\cite{Ham18}, \href{works/TranVNB17.pdf}{TranVNB17}~\cite{TranVNB17}, \href{works/CohenHB17.pdf}{CohenHB17}~\cite{CohenHB17}, \href{works/NovaraNH16.pdf}{NovaraNH16}~\cite{NovaraNH16}, \href{works/BoothNB16.pdf}{BoothNB16}~\cite{BoothNB16}... (Total: 32) & \href{works/AbreuNP23.pdf}{AbreuNP23}~\cite{AbreuNP23}, \href{works/JuvinHL23.pdf}{JuvinHL23}~\cite{JuvinHL23}, \href{works/YuraszeckMC23.pdf}{YuraszeckMC23}~\cite{YuraszeckMC23}, \href{works/AalianPG23.pdf}{AalianPG23}~\cite{AalianPG23}, \href{works/CzerniachowskaWZ23.pdf}{CzerniachowskaWZ23}~\cite{CzerniachowskaWZ23}, \href{works/SquillaciPR23.pdf}{SquillaciPR23}~\cite{SquillaciPR23}, \href{works/Teppan22.pdf}{Teppan22}~\cite{Teppan22}, \href{works/YunusogluY22.pdf}{YunusogluY22}~\cite{YunusogluY22}, \href{works/WinterMMW22.pdf}{WinterMMW22}~\cite{WinterMMW22}, \href{works/CampeauG22.pdf}{CampeauG22}~\cite{CampeauG22}, \href{works/OujanaAYB22.pdf}{OujanaAYB22}~\cite{OujanaAYB22}, \href{works/ArmstrongGOS22.pdf}{ArmstrongGOS22}~\cite{ArmstrongGOS22}, \href{works/TouatBT22.pdf}{TouatBT22}~\cite{TouatBT22}, \href{works/ZhangJZL22.pdf}{ZhangJZL22}~\cite{ZhangJZL22}, \href{works/NaderiBZ22.pdf}{NaderiBZ22}~\cite{NaderiBZ22}, \href{works/HamPK21.pdf}{HamPK21}~\cite{HamPK21}, \href{works/AbreuAPNM21.pdf}{AbreuAPNM21}~\cite{AbreuAPNM21}, \href{works/LacknerMMWW21.pdf}{LacknerMMWW21}~\cite{LacknerMMWW21}, \href{works/GroleazNS20.pdf}{GroleazNS20}~\cite{GroleazNS20}, \href{works/GroleazNS20a.pdf}{GroleazNS20a}~\cite{GroleazNS20a}, \href{works/NattafM20.pdf}{NattafM20}~\cite{NattafM20}, \href{works/Polo-MejiaALB20.pdf}{Polo-MejiaALB20}~\cite{Polo-MejiaALB20}, \href{works/BogaerdtW19.pdf}{BogaerdtW19}~\cite{BogaerdtW19}, \href{works/ColT19.pdf}{ColT19}~\cite{ColT19}, \href{works/GeibingerMM19.pdf}{GeibingerMM19}~\cite{GeibingerMM19}, \href{works/KucukY19.pdf}{KucukY19}~\cite{KucukY19}, \href{works/ParkUJR19.pdf}{ParkUJR19}~\cite{ParkUJR19}, \href{works/BenediktSMVH18.pdf}{BenediktSMVH18}~\cite{BenediktSMVH18}, \href{works/CappartTSR18.pdf}{CappartTSR18}~\cite{CappartTSR18}... (Total: 35)\\
Constraints & regular expression &  & \href{works/FrimodigS19.pdf}{FrimodigS19}~\cite{FrimodigS19} & \href{works/HookerH18.pdf}{HookerH18}~\cite{HookerH18}\\
Constraints & span constraint &  & \href{works/CappartS17.pdf}{CappartS17}~\cite{CappartS17}, \href{works/SchuttFS13.pdf}{SchuttFS13}~\cite{SchuttFS13}, \href{works/LombardiM10a.pdf}{LombardiM10a}~\cite{LombardiM10a}, \href{works/Lombardi10.pdf}{Lombardi10}~\cite{Lombardi10}, \href{works/Darby-DowmanLMZ97.pdf}{Darby-DowmanLMZ97}~\cite{Darby-DowmanLMZ97} & \href{works/OujanaAYB22.pdf}{OujanaAYB22}~\cite{OujanaAYB22}, \href{works/ZhangBB22.pdf}{ZhangBB22}~\cite{ZhangBB22}, \href{works/TangB20.pdf}{TangB20}~\cite{TangB20}, \href{works/ZouZ20.pdf}{ZouZ20}~\cite{ZouZ20}, \href{works/YounespourAKE19.pdf}{YounespourAKE19}~\cite{YounespourAKE19}, \href{works/LaborieRSV18.pdf}{LaborieRSV18}~\cite{LaborieRSV18}, \href{works/SimoninAHL15.pdf}{SimoninAHL15}~\cite{SimoninAHL15}, \href{works/SimoninAHL12.pdf}{SimoninAHL12}~\cite{SimoninAHL12}, \href{works/SchuttFSW11.pdf}{SchuttFSW11}~\cite{SchuttFSW11}\\
Constraints & table constraint & \href{works/Lombardi10.pdf}{Lombardi10}~\cite{Lombardi10}, \href{works/LombardiM10a.pdf}{LombardiM10a}~\cite{LombardiM10a}, \href{works/PapaB98.pdf}{PapaB98}~\cite{PapaB98} & \href{works/JelinekB16.pdf}{JelinekB16}~\cite{JelinekB16} & \href{works/PerezGSL23.pdf}{PerezGSL23}~\cite{PerezGSL23}, \href{works/abs-2312-13682.pdf}{abs-2312-13682}~\cite{abs-2312-13682}, \href{works/ArmstrongGOS21.pdf}{ArmstrongGOS21}~\cite{ArmstrongGOS21}, \href{works/CauwelaertLS18.pdf}{CauwelaertLS18}~\cite{CauwelaertLS18}, \href{works/Siala15a.pdf}{Siala15a}~\cite{Siala15a}, \href{works/GayHS15.pdf}{GayHS15}~\cite{GayHS15}, \href{works/PesantRR15.pdf}{PesantRR15}~\cite{PesantRR15}, \href{works/MelgarejoLS15.pdf}{MelgarejoLS15}~\cite{MelgarejoLS15}, \href{works/LimtanyakulS12.pdf}{LimtanyakulS12}~\cite{LimtanyakulS12}, \href{works/BeniniLMR11.pdf}{BeniniLMR11}~\cite{BeniniLMR11}, \href{works/BeckFW11.pdf}{BeckFW11}~\cite{BeckFW11}, \href{works/HermenierDL11.pdf}{HermenierDL11}~\cite{HermenierDL11}, \href{works/LopesCSM10.pdf}{LopesCSM10}~\cite{LopesCSM10}, \href{works/MouraSCL08.pdf}{MouraSCL08}~\cite{MouraSCL08}, \href{works/GodardLN05.pdf}{GodardLN05}~\cite{GodardLN05}, \href{works/Laborie03.pdf}{Laborie03}~\cite{Laborie03}, \href{works/ElkhyariGJ02.pdf}{ElkhyariGJ02}~\cite{ElkhyariGJ02}\\
\end{longtable}
}


\clearpage
\subsection{Concept Type ProgLanguages}
\label{sec:ProgLanguages}
{\scriptsize
\begin{longtable}{lp{3cm}>{\raggedright\arraybackslash}p{6cm}>{\raggedright\arraybackslash}p{6cm}>{\raggedright\arraybackslash}p{8cm}}
\rowcolor{white}\caption{Works for Concepts of Type ProgLanguages}\\ \toprule
\rowcolor{white}Type & Keyword & High & Medium & Low\\ \midrule\endhead
\bottomrule
\endfoot
ProgLanguages & C  & \href{works/KoehlerBFFHPSSS21.pdf}{KoehlerBFFHPSSS21}~\cite{KoehlerBFFHPSSS21} &  & \href{works/EmdeZD22.pdf}{EmdeZD22}~\cite{EmdeZD22}, \href{works/HubnerGSV21.pdf}{HubnerGSV21}~\cite{HubnerGSV21}, \href{works/BogaerdtW19.pdf}{BogaerdtW19}~\cite{BogaerdtW19}, \href{works/TangLWSK18.pdf}{TangLWSK18}~\cite{TangLWSK18}, \href{works/LaborieRSV18.pdf}{LaborieRSV18}~\cite{LaborieRSV18}, \href{works/HoYCLLCLC18.pdf}{HoYCLLCLC18}~\cite{HoYCLLCLC18}, \href{works/LombardiMRB10.pdf}{LombardiMRB10}~\cite{LombardiMRB10}, \href{works/Lombardi10.pdf}{Lombardi10}~\cite{Lombardi10}, \href{works/LombardiM10a.pdf}{LombardiM10a}~\cite{LombardiM10a}, \href{works/Laborie09.pdf}{Laborie09}~\cite{Laborie09}, \href{works/GarridoOS08.pdf}{GarridoOS08}~\cite{GarridoOS08}, \href{works/Layfield02.pdf}{Layfield02}~\cite{Layfield02}\\
ProgLanguages & C++ &  & \href{works/BourreauGGLT22.pdf}{BourreauGGLT22}~\cite{BourreauGGLT22}, \href{works/Demassey03.pdf}{Demassey03}~\cite{Demassey03} & \href{works/TardivoDFMP23.pdf}{TardivoDFMP23}~\cite{TardivoDFMP23}, \href{works/JuvinHHL23.pdf}{JuvinHHL23}~\cite{JuvinHHL23}, \href{works/PopovicCGNC22.pdf}{PopovicCGNC22}~\cite{PopovicCGNC22}, \href{works/ColT22.pdf}{ColT22}~\cite{ColT22}, \href{works/Astrand21.pdf}{Astrand21}~\cite{Astrand21}, \href{works/AntuoriHHEN21.pdf}{AntuoriHHEN21}~\cite{AntuoriHHEN21}, \href{works/QinWSLS21.pdf}{QinWSLS21}~\cite{QinWSLS21}, \href{works/AbreuAPNM21.pdf}{AbreuAPNM21}~\cite{AbreuAPNM21}, \href{works/Lemos21.pdf}{Lemos21}~\cite{Lemos21}, \href{works/Polo-MejiaALB20.pdf}{Polo-MejiaALB20}~\cite{Polo-MejiaALB20}, \href{works/AstrandJZ20.pdf}{AstrandJZ20}~\cite{AstrandJZ20}, \href{works/Mercier-AubinGQ20.pdf}{Mercier-AubinGQ20}~\cite{Mercier-AubinGQ20}, \href{works/abs-1902-01193.pdf}{abs-1902-01193}~\cite{abs-1902-01193}, \href{works/Caballero19.pdf}{Caballero19}~\cite{Caballero19}, \href{works/LaborieRSV18.pdf}{LaborieRSV18}~\cite{LaborieRSV18}, \href{works/ArbaouiY18.pdf}{ArbaouiY18}~\cite{ArbaouiY18}, \href{works/TranPZLDB18.pdf}{TranPZLDB18}~\cite{TranPZLDB18}, \href{works/GomesM17.pdf}{GomesM17}~\cite{GomesM17}, \href{works/NattafAL17.pdf}{NattafAL17}~\cite{NattafAL17}, \href{works/Nattaf16.pdf}{Nattaf16}~\cite{Nattaf16}, \href{works/BoothNB16.pdf}{BoothNB16}~\cite{BoothNB16}, \href{works/Tesch16.pdf}{Tesch16}~\cite{Tesch16}, \href{works/Bonfietti16.pdf}{Bonfietti16}~\cite{Bonfietti16}, \href{works/Fahimi16.pdf}{Fahimi16}~\cite{Fahimi16}, \href{works/NattafAL15.pdf}{NattafAL15}~\cite{NattafAL15}, \href{works/Kameugne14.pdf}{Kameugne14}~\cite{Kameugne14}, \href{works/TranTDB13.pdf}{TranTDB13}~\cite{TranTDB13}, \href{works/SchuttFSW13.pdf}{SchuttFSW13}~\cite{SchuttFSW13}, \href{works/GuSW12.pdf}{GuSW12}~\cite{GuSW12}... (Total: 69)\\
ProgLanguages & Java & \href{works/abs-2102-08778.pdf}{abs-2102-08778}~\cite{abs-2102-08778}, \href{works/Malapert11.pdf}{Malapert11}~\cite{Malapert11} & \href{works/Froger16.pdf}{Froger16}~\cite{Froger16}, \href{works/Wolf11.pdf}{Wolf11}~\cite{Wolf11}, \href{works/KuchcinskiW03.pdf}{KuchcinskiW03}~\cite{KuchcinskiW03} & \href{works/abs-2306-05747.pdf}{abs-2306-05747}~\cite{abs-2306-05747}, \href{works/AlfieriGPS23.pdf}{AlfieriGPS23}~\cite{AlfieriGPS23}, \href{works/TasselGS23.pdf}{TasselGS23}~\cite{TasselGS23}, \href{works/KameugneFND23.pdf}{KameugneFND23}~\cite{KameugneFND23}, \href{works/MullerMKP22.pdf}{MullerMKP22}~\cite{MullerMKP22}, \href{works/FetgoD22.pdf}{FetgoD22}~\cite{FetgoD22}, \href{works/ColT22.pdf}{ColT22}~\cite{ColT22}, \href{works/YuraszeckMPV22.pdf}{YuraszeckMPV22}~\cite{YuraszeckMPV22}, \href{works/OuelletQ22.pdf}{OuelletQ22}~\cite{OuelletQ22}, \href{works/Teppan22.pdf}{Teppan22}~\cite{Teppan22}, \href{works/Groleaz21.pdf}{Groleaz21}~\cite{Groleaz21}, \href{works/FanXG21.pdf}{FanXG21}~\cite{FanXG21}, \href{works/AntuoriHHEN21.pdf}{AntuoriHHEN21}~\cite{AntuoriHHEN21}, \href{works/Lemos21.pdf}{Lemos21}~\cite{Lemos21}, \href{works/ArmstrongGOS21.pdf}{ArmstrongGOS21}~\cite{ArmstrongGOS21}, \href{works/CauwelaertDS20.pdf}{CauwelaertDS20}~\cite{CauwelaertDS20}, \href{works/MejiaY20.pdf}{MejiaY20}~\cite{MejiaY20}, \href{works/SacramentoSP20.pdf}{SacramentoSP20}~\cite{SacramentoSP20}, \href{works/TangB20.pdf}{TangB20}~\cite{TangB20}, \href{works/BarzegaranZP20.pdf}{BarzegaranZP20}~\cite{BarzegaranZP20}, \href{works/abs-1911-04766.pdf}{abs-1911-04766}~\cite{abs-1911-04766}, \href{works/FrohnerTR19.pdf}{FrohnerTR19}~\cite{FrohnerTR19}, \href{works/Tom19.pdf}{Tom19}~\cite{Tom19}, \href{works/ColT19.pdf}{ColT19}~\cite{ColT19}, \href{works/GeibingerMM19.pdf}{GeibingerMM19}~\cite{GeibingerMM19}, \href{works/CauwelaertLS18.pdf}{CauwelaertLS18}~\cite{CauwelaertLS18}, \href{works/OuelletQ18.pdf}{OuelletQ18}~\cite{OuelletQ18}, \href{works/LaborieRSV18.pdf}{LaborieRSV18}~\cite{LaborieRSV18}, \href{works/KameugneFGOQ18.pdf}{KameugneFGOQ18}~\cite{KameugneFGOQ18}... (Total: 55)\\
ProgLanguages & Julia &  &  & \href{works/HebrardALLCMR22.pdf}{HebrardALLCMR22}~\cite{HebrardALLCMR22}, \href{works/Astrand21.pdf}{Astrand21}~\cite{Astrand21}, \href{works/Groleaz21.pdf}{Groleaz21}~\cite{Groleaz21}\\
ProgLanguages & Lisp &  &  & \href{works/Wallace96.pdf}{Wallace96}~\cite{Wallace96}\\
ProgLanguages & Prolog & \href{works/ArmstrongGOS21.pdf}{ArmstrongGOS21}~\cite{ArmstrongGOS21}, \href{works/Simonis99.pdf}{Simonis99}~\cite{Simonis99}, \href{works/FalaschiGMP97.pdf}{FalaschiGMP97}~\cite{FalaschiGMP97}, \href{works/Zhou97.pdf}{Zhou97}~\cite{Zhou97}, \href{works/LammaMM97.pdf}{LammaMM97}~\cite{LammaMM97}, \href{works/Wallace96.pdf}{Wallace96}~\cite{Wallace96}, \href{works/Touraivane95.pdf}{Touraivane95}~\cite{Touraivane95}, \href{works/Simonis95a.pdf}{Simonis95a}~\cite{Simonis95a}, \href{works/Simonis95.pdf}{Simonis95}~\cite{Simonis95}, \href{works/DincbasSH90.pdf}{DincbasSH90}~\cite{DincbasSH90} & \href{works/BadicaBI20.pdf}{BadicaBI20}~\cite{BadicaBI20}, \href{works/MossigeGSMC17.pdf}{MossigeGSMC17}~\cite{MossigeGSMC17}, \href{works/Madi-WambaLOBM17.pdf}{Madi-WambaLOBM17}~\cite{Madi-WambaLOBM17}, \href{works/Malapert11.pdf}{Malapert11}~\cite{Malapert11}, \href{works/MartinPY01.pdf}{MartinPY01}~\cite{MartinPY01}, \href{works/SimonisCK00.pdf}{SimonisCK00}~\cite{SimonisCK00}, \href{works/RodosekW98.pdf}{RodosekW98}~\cite{RodosekW98}, \href{works/Zhou96.pdf}{Zhou96}~\cite{Zhou96}, \href{works/SimonisC95.pdf}{SimonisC95}~\cite{SimonisC95}, \href{works/BeldiceanuC94.pdf}{BeldiceanuC94}~\cite{BeldiceanuC94}, \href{works/AggounB93.pdf}{AggounB93}~\cite{AggounB93} & \href{works/PopovicCGNC22.pdf}{PopovicCGNC22}~\cite{PopovicCGNC22}, \href{works/ArmstrongGOS22.pdf}{ArmstrongGOS22}~\cite{ArmstrongGOS22}, \href{works/ZarandiASC20.pdf}{ZarandiASC20}~\cite{ZarandiASC20}, \href{works/abs-1902-01193.pdf}{abs-1902-01193}~\cite{abs-1902-01193}, \href{works/YangSS19.pdf}{YangSS19}~\cite{YangSS19}, \href{works/CauwelaertLS18.pdf}{CauwelaertLS18}~\cite{CauwelaertLS18}, \href{works/German18.pdf}{German18}~\cite{German18}, \href{works/JelinekB16.pdf}{JelinekB16}~\cite{JelinekB16}, \href{works/LetortCB15.pdf}{LetortCB15}~\cite{LetortCB15}, \href{works/Kameugne14.pdf}{Kameugne14}~\cite{Kameugne14}, \href{works/LetortCB13.pdf}{LetortCB13}~\cite{LetortCB13}, \href{works/Letort13.pdf}{Letort13}~\cite{Letort13}, \href{works/Clercq12.pdf}{Clercq12}~\cite{Clercq12}, \href{works/LetortBC12.pdf}{LetortBC12}~\cite{LetortBC12}, \href{works/Schutt11.pdf}{Schutt11}~\cite{Schutt11}, \href{works/TrojetHL11.pdf}{TrojetHL11}~\cite{TrojetHL11}, \href{works/BeldiceanuCDP11.pdf}{BeldiceanuCDP11}~\cite{BeldiceanuCDP11}, \href{works/Menana11.pdf}{Menana11}~\cite{Menana11}, \href{works/BartakCS10.pdf}{BartakCS10}~\cite{BartakCS10}, \href{works/AronssonBK09.pdf}{AronssonBK09}~\cite{AronssonBK09}, \href{works/BeldiceanuCP08.pdf}{BeldiceanuCP08}~\cite{BeldiceanuCP08}, \href{works/KrogtLPHJ07.pdf}{KrogtLPHJ07}~\cite{KrogtLPHJ07}, \href{works/Simonis07.pdf}{Simonis07}~\cite{Simonis07}, \href{works/QuSN06.pdf}{QuSN06}~\cite{QuSN06}, \href{works/Geske05.pdf}{Geske05}~\cite{Geske05}, \href{works/PoderBS04.pdf}{PoderBS04}~\cite{PoderBS04}, \href{works/Baptiste02.pdf}{Baptiste02}~\cite{Baptiste02}, \href{works/Bartak02.pdf}{Bartak02}~\cite{Bartak02}, \href{works/BeldiceanuC02.pdf}{BeldiceanuC02}~\cite{BeldiceanuC02}... (Total: 37)\\
ProgLanguages & Python & \href{works/KoehlerBFFHPSSS21.pdf}{KoehlerBFFHPSSS21}~\cite{KoehlerBFFHPSSS21} & \href{works/ForbesHJST24.pdf}{ForbesHJST24}~\cite{ForbesHJST24}, \href{works/abs-2211-14492.pdf}{abs-2211-14492}~\cite{abs-2211-14492}, \href{works/AbreuN22.pdf}{AbreuN22}~\cite{AbreuN22}, \href{works/AbreuAPNM21.pdf}{AbreuAPNM21}~\cite{AbreuAPNM21}, \href{works/LaborieRSV18.pdf}{LaborieRSV18}~\cite{LaborieRSV18} & \href{works/EfthymiouY23.pdf}{EfthymiouY23}~\cite{EfthymiouY23}, \href{works/SquillaciPR23.pdf}{SquillaciPR23}~\cite{SquillaciPR23}, \href{works/Mehdizadeh-Somarin23.pdf}{Mehdizadeh-Somarin23}~\cite{Mehdizadeh-Somarin23}, \href{works/AbreuNP23.pdf}{AbreuNP23}~\cite{AbreuNP23}, \href{works/KimCMLLP23.pdf}{KimCMLLP23}~\cite{KimCMLLP23}, \href{works/MontemanniD23.pdf}{MontemanniD23}~\cite{MontemanniD23}, \href{works/PovedaAA23.pdf}{PovedaAA23}~\cite{PovedaAA23}, \href{works/MontemanniD23a.pdf}{MontemanniD23a}~\cite{MontemanniD23a}, \href{works/AkramNHRSA23.pdf}{AkramNHRSA23}~\cite{AkramNHRSA23}, \href{works/NaderiRR23.pdf}{NaderiRR23}~\cite{NaderiRR23}, \href{works/FetgoD22.pdf}{FetgoD22}~\cite{FetgoD22}, \href{works/PohlAK22.pdf}{PohlAK22}~\cite{PohlAK22}, \href{works/MullerMKP22.pdf}{MullerMKP22}~\cite{MullerMKP22}, \href{works/ZhangBB22.pdf}{ZhangBB22}~\cite{ZhangBB22}, \href{works/EtminaniesfahaniGNMS22.pdf}{EtminaniesfahaniGNMS22}~\cite{EtminaniesfahaniGNMS22}, \href{works/LuoB22.pdf}{LuoB22}~\cite{LuoB22}, \href{works/CampeauG22.pdf}{CampeauG22}~\cite{CampeauG22}, \href{works/KlankeBYE21.pdf}{KlankeBYE21}~\cite{KlankeBYE21}, \href{works/FanXG21.pdf}{FanXG21}~\cite{FanXG21}, \href{works/Lemos21.pdf}{Lemos21}~\cite{Lemos21}, \href{works/HanenKP21.pdf}{HanenKP21}~\cite{HanenKP21}, \href{works/BenderWS21.pdf}{BenderWS21}~\cite{BenderWS21}, \href{works/AbohashimaEG21.pdf}{AbohashimaEG21}~\cite{AbohashimaEG21}, \href{works/Lunardi20.pdf}{Lunardi20}~\cite{Lunardi20}, \href{works/LunardiBLRV20.pdf}{LunardiBLRV20}~\cite{LunardiBLRV20}, \href{works/Mercier-AubinGQ20.pdf}{Mercier-AubinGQ20}~\cite{Mercier-AubinGQ20}, \href{works/FrimodigS19.pdf}{FrimodigS19}~\cite{FrimodigS19}, \href{works/BehrensLM19.pdf}{BehrensLM19}~\cite{BehrensLM19}, \href{works/FrohnerTR19.pdf}{FrohnerTR19}~\cite{FrohnerTR19}... (Total: 38)\\
\end{longtable}
}


\clearpage
\subsection{Concept Type CPSystems}
\label{sec:CPSystems}
{\scriptsize
\begin{longtable}{lp{3cm}>{\raggedright\arraybackslash}p{6cm}>{\raggedright\arraybackslash}p{6cm}>{\raggedright\arraybackslash}p{8cm}}
\rowcolor{white}\caption{Works for Concepts of Type CPSystems}\\ \toprule
\rowcolor{white}Type & Keyword & High & Medium & Low\\ \midrule\endhead
\bottomrule
\endfoot
CPSystems & CHIP & \href{works/TrojetHL11.pdf}{TrojetHL11}~\cite{TrojetHL11}, \href{works/Simonis07.pdf}{Simonis07}~\cite{Simonis07}, \href{works/GruianK98.pdf}{GruianK98}~\cite{GruianK98}, \href{works/Wallace96.pdf}{Wallace96}~\cite{Wallace96}, \href{works/Simonis95.pdf}{Simonis95}~\cite{Simonis95}, \href{works/Goltz95.pdf}{Goltz95}~\cite{Goltz95}, \href{works/SimonisC95.pdf}{SimonisC95}~\cite{SimonisC95}, \href{works/BeldiceanuC94.pdf}{BeldiceanuC94}~\cite{BeldiceanuC94}, \href{works/AggounB93.pdf}{AggounB93}~\cite{AggounB93}, \href{works/DincbasSH90.pdf}{DincbasSH90}~\cite{DincbasSH90} & \href{works/ArmstrongGOS21.pdf}{ArmstrongGOS21}~\cite{ArmstrongGOS21}, \href{works/YangSS19.pdf}{YangSS19}~\cite{YangSS19}, \href{works/LaborieRSV18.pdf}{LaborieRSV18}~\cite{LaborieRSV18}, \href{works/Geske05.pdf}{Geske05}~\cite{Geske05}, \href{works/PoderBS04.pdf}{PoderBS04}~\cite{PoderBS04}, \href{works/Timpe02.pdf}{Timpe02}~\cite{Timpe02}, \href{works/RodosekW98.pdf}{RodosekW98}~\cite{RodosekW98}, \href{works/Zhou97.pdf}{Zhou97}~\cite{Zhou97}, \href{works/LammaMM97.pdf}{LammaMM97}~\cite{LammaMM97} & \href{works/PrataAN23.pdf}{PrataAN23}~\cite{PrataAN23}, \href{works/TardivoDFMP23.pdf}{TardivoDFMP23}~\cite{TardivoDFMP23}, \href{works/KameugneFND23.pdf}{KameugneFND23}~\cite{KameugneFND23}, \href{works/LuoB22.pdf}{LuoB22}~\cite{LuoB22}, \href{works/FetgoD22.pdf}{FetgoD22}~\cite{FetgoD22}, \href{works/BourreauGGLT22.pdf}{BourreauGGLT22}~\cite{BourreauGGLT22}, \href{works/PopovicCGNC22.pdf}{PopovicCGNC22}~\cite{PopovicCGNC22}, \href{works/KlankeBYE21.pdf}{KlankeBYE21}~\cite{KlankeBYE21}, \href{works/GodetLHS20.pdf}{GodetLHS20}~\cite{GodetLHS20}, \href{works/abs-1902-01193.pdf}{abs-1902-01193}~\cite{abs-1902-01193}, \href{works/BaptisteB18.pdf}{BaptisteB18}~\cite{BaptisteB18}, \href{works/KameugneFGOQ18.pdf}{KameugneFGOQ18}~\cite{KameugneFGOQ18}, \href{works/GokgurHO18.pdf}{GokgurHO18}~\cite{GokgurHO18}, \href{works/MossigeGSMC17.pdf}{MossigeGSMC17}~\cite{MossigeGSMC17}, \href{works/Pralet17.pdf}{Pralet17}~\cite{Pralet17}, \href{works/KreterSS17.pdf}{KreterSS17}~\cite{KreterSS17}, \href{works/Madi-WambaB16.pdf}{Madi-WambaB16}~\cite{Madi-WambaB16}, \href{works/FontaineMH16.pdf}{FontaineMH16}~\cite{FontaineMH16}, \href{works/ZhouGL15.pdf}{ZhouGL15}~\cite{ZhouGL15}, \href{works/SimoninAHL15.pdf}{SimoninAHL15}~\cite{SimoninAHL15}, \href{works/LetortCB15.pdf}{LetortCB15}~\cite{LetortCB15}, \href{works/KreterSS15.pdf}{KreterSS15}~\cite{KreterSS15}, \href{works/GrimesIOS14.pdf}{GrimesIOS14}~\cite{GrimesIOS14}, \href{works/KameugneFSN14.pdf}{KameugneFSN14}~\cite{KameugneFSN14}, \href{works/DerrienPZ14.pdf}{DerrienPZ14}~\cite{DerrienPZ14}, \href{works/OzturkTHO13.pdf}{OzturkTHO13}~\cite{OzturkTHO13}, \href{works/SchuttFS13a.pdf}{SchuttFS13a}~\cite{SchuttFS13a}, \href{works/SimoninAHL12.pdf}{SimoninAHL12}~\cite{SimoninAHL12}, \href{works/SchuttCSW12.pdf}{SchuttCSW12}~\cite{SchuttCSW12}... (Total: 50)\\
CPSystems & CPO & \href{works/NaderiRR23.pdf}{NaderiRR23}~\cite{NaderiRR23}, \href{works/LacknerMMWW23.pdf}{LacknerMMWW23}~\cite{LacknerMMWW23}, \href{works/JuvinHHL23.pdf}{JuvinHHL23}~\cite{JuvinHHL23}, \href{works/Bit-Monnot23.pdf}{Bit-Monnot23}~\cite{Bit-Monnot23}, \href{works/CzerniachowskaWZ23.pdf}{CzerniachowskaWZ23}~\cite{CzerniachowskaWZ23}, \href{works/WinterMMW22.pdf}{WinterMMW22}~\cite{WinterMMW22}, \href{works/ColT22.pdf}{ColT22}~\cite{ColT22}, \href{works/LacknerMMWW21.pdf}{LacknerMMWW21}~\cite{LacknerMMWW21}, \href{works/ArmstrongGOS21.pdf}{ArmstrongGOS21}~\cite{ArmstrongGOS21}, \href{works/NattafM20.pdf}{NattafM20}~\cite{NattafM20}, \href{works/GroleazNS20.pdf}{GroleazNS20}~\cite{GroleazNS20}, \href{works/Polo-MejiaALB20.pdf}{Polo-MejiaALB20}~\cite{Polo-MejiaALB20}, \href{works/GroleazNS20a.pdf}{GroleazNS20a}~\cite{GroleazNS20a}, \href{works/SacramentoSP20.pdf}{SacramentoSP20}~\cite{SacramentoSP20}, \href{works/GeibingerMM19.pdf}{GeibingerMM19}~\cite{GeibingerMM19}, \href{works/ColT19.pdf}{ColT19}~\cite{ColT19}, \href{works/MalapertN19.pdf}{MalapertN19}~\cite{MalapertN19}, \href{works/LaborieRSV18.pdf}{LaborieRSV18}~\cite{LaborieRSV18}, \href{works/KreterSS17.pdf}{KreterSS17}~\cite{KreterSS17}, \href{works/GoelSHFS15.pdf}{GoelSHFS15}~\cite{GoelSHFS15}, \href{works/PraletLJ15.pdf}{PraletLJ15}~\cite{PraletLJ15}, \href{works/Laborie09.pdf}{Laborie09}~\cite{Laborie09} & \href{works/AalianPG23.pdf}{AalianPG23}~\cite{AalianPG23}, \href{works/abs-1911-04766.pdf}{abs-1911-04766}~\cite{abs-1911-04766}, \href{works/NuijtenA94.pdf}{NuijtenA94}~\cite{NuijtenA94} & \href{works/JuvinHL23.pdf}{JuvinHL23}~\cite{JuvinHL23}, \href{works/PovedaAA23.pdf}{PovedaAA23}~\cite{PovedaAA23}, \href{works/OujanaAYB22.pdf}{OujanaAYB22}~\cite{OujanaAYB22}, \href{works/GeibingerMM21.pdf}{GeibingerMM21}~\cite{GeibingerMM21}, \href{works/abs-2102-08778.pdf}{abs-2102-08778}~\cite{abs-2102-08778}, \href{works/TangB20.pdf}{TangB20}~\cite{TangB20}, \href{works/Laborie18a.pdf}{Laborie18a}~\cite{Laborie18a}, \href{works/Pralet17.pdf}{Pralet17}~\cite{Pralet17}, \href{works/VilimLS15.pdf}{VilimLS15}~\cite{VilimLS15}, \href{works/BartakSR10.pdf}{BartakSR10}~\cite{BartakSR10}, \href{works/GarridoAO09.pdf}{GarridoAO09}~\cite{GarridoAO09}, \href{works/Vilim09.pdf}{Vilim09}~\cite{Vilim09}, \href{works/GarridoOS08.pdf}{GarridoOS08}~\cite{GarridoOS08}, \href{works/BeldiceanuC94.pdf}{BeldiceanuC94}~\cite{BeldiceanuC94}\\
CPSystems & Choco Solver & \href{works/TasselGS23.pdf}{TasselGS23}~\cite{TasselGS23}, \href{works/abs-2306-05747.pdf}{abs-2306-05747}~\cite{abs-2306-05747}, \href{works/LetortCB15.pdf}{LetortCB15}~\cite{LetortCB15}, \href{works/LetortCB13.pdf}{LetortCB13}~\cite{LetortCB13}, \href{works/OuelletQ13.pdf}{OuelletQ13}~\cite{OuelletQ13}, \href{works/LetortBC12.pdf}{LetortBC12}~\cite{LetortBC12}, \href{works/GrimesHM09.pdf}{GrimesHM09}~\cite{GrimesHM09}, \href{works/abs-0907-0939.pdf}{abs-0907-0939}~\cite{abs-0907-0939}, \href{works/GarridoAO09.pdf}{GarridoAO09}~\cite{GarridoAO09}, \href{works/GarridoOS08.pdf}{GarridoOS08}~\cite{GarridoOS08} & \href{works/KameugneFND23.pdf}{KameugneFND23}~\cite{KameugneFND23}, \href{works/MullerMKP22.pdf}{MullerMKP22}~\cite{MullerMKP22}, \href{works/FetgoD22.pdf}{FetgoD22}~\cite{FetgoD22}, \href{works/AntuoriHHEN21.pdf}{AntuoriHHEN21}~\cite{AntuoriHHEN21}, \href{works/LiuLH19.pdf}{LiuLH19}~\cite{LiuLH19}, \href{works/FahimiOQ18.pdf}{FahimiOQ18}~\cite{FahimiOQ18}, \href{works/KameugneFGOQ18.pdf}{KameugneFGOQ18}~\cite{KameugneFGOQ18}, \href{works/LaborieRSV18.pdf}{LaborieRSV18}~\cite{LaborieRSV18}, \href{works/GayHS15.pdf}{GayHS15}~\cite{GayHS15}, \href{works/KoschB14.pdf}{KoschB14}~\cite{KoschB14}, \href{works/DerrienPZ14.pdf}{DerrienPZ14}~\cite{DerrienPZ14}, \href{works/DerrienP14.pdf}{DerrienP14}~\cite{DerrienP14}, \href{works/HermenierDL11.pdf}{HermenierDL11}~\cite{HermenierDL11}, \href{works/ClercqPBJ11.pdf}{ClercqPBJ11}~\cite{ClercqPBJ11} & \href{works/BourreauGGLT22.pdf}{BourreauGGLT22}~\cite{BourreauGGLT22}, \href{works/OuelletQ22.pdf}{OuelletQ22}~\cite{OuelletQ22}, \href{works/GodetLHS20.pdf}{GodetLHS20}~\cite{GodetLHS20}, \href{works/YangSS19.pdf}{YangSS19}~\cite{YangSS19}, \href{works/OuelletQ18.pdf}{OuelletQ18}~\cite{OuelletQ18}, \href{works/GingrasQ16.pdf}{GingrasQ16}~\cite{GingrasQ16}, \href{works/Madi-WambaB16.pdf}{Madi-WambaB16}~\cite{Madi-WambaB16}, \href{works/EvenSH15a.pdf}{EvenSH15a}~\cite{EvenSH15a}, \href{works/MurphyMB15.pdf}{MurphyMB15}~\cite{MurphyMB15}, \href{works/EvenSH15.pdf}{EvenSH15}~\cite{EvenSH15}, \href{works/BessiereHMQW14.pdf}{BessiereHMQW14}~\cite{BessiereHMQW14}, \href{works/BartakSR10.pdf}{BartakSR10}~\cite{BartakSR10}, \href{works/RossiTHP07.pdf}{RossiTHP07}~\cite{RossiTHP07}\\
CPSystems & Chuffed & \href{works/LacknerMMWW23.pdf}{LacknerMMWW23}~\cite{LacknerMMWW23}, \href{works/PovedaAA23.pdf}{PovedaAA23}~\cite{PovedaAA23}, \href{works/BoudreaultSLQ22.pdf}{BoudreaultSLQ22}~\cite{BoudreaultSLQ22}, \href{works/MullerMKP22.pdf}{MullerMKP22}~\cite{MullerMKP22}, \href{works/LacknerMMWW21.pdf}{LacknerMMWW21}~\cite{LacknerMMWW21}, \href{works/GeibingerMM21.pdf}{GeibingerMM21}~\cite{GeibingerMM21}, \href{works/ArmstrongGOS21.pdf}{ArmstrongGOS21}~\cite{ArmstrongGOS21}, \href{works/KoehlerBFFHPSSS21.pdf}{KoehlerBFFHPSSS21}~\cite{KoehlerBFFHPSSS21}, \href{works/WallaceY20.pdf}{WallaceY20}~\cite{WallaceY20}, \href{works/GodetLHS20.pdf}{GodetLHS20}~\cite{GodetLHS20}, \href{works/abs-1911-04766.pdf}{abs-1911-04766}~\cite{abs-1911-04766}, \href{works/YoungFS17.pdf}{YoungFS17}~\cite{YoungFS17}, \href{works/KreterSS17.pdf}{KreterSS17}~\cite{KreterSS17}, \href{works/SzerediS16.pdf}{SzerediS16}~\cite{SzerediS16}, \href{works/KreterSS15.pdf}{KreterSS15}~\cite{KreterSS15} &  & \href{works/SchuttS16.pdf}{SchuttS16}~\cite{SchuttS16}\\
CPSystems & Claire & \href{works/BaptisteP00.pdf}{BaptisteP00}~\cite{BaptisteP00} & \href{works/BaptisteP97.pdf}{BaptisteP97}~\cite{BaptisteP97} & \href{works/HanenKP21.pdf}{HanenKP21}~\cite{HanenKP21}, \href{works/PapaB98.pdf}{PapaB98}~\cite{PapaB98}\\
CPSystems & Cplex & \href{works/CzerniachowskaWZ23.pdf}{CzerniachowskaWZ23}~\cite{CzerniachowskaWZ23}, \href{works/NaderiRR23.pdf}{NaderiRR23}~\cite{NaderiRR23}, \href{works/SubulanC22.pdf}{SubulanC22}~\cite{SubulanC22}, \href{works/BourreauGGLT22.pdf}{BourreauGGLT22}~\cite{BourreauGGLT22}, \href{works/MullerMKP22.pdf}{MullerMKP22}~\cite{MullerMKP22}, \href{works/WinterMMW22.pdf}{WinterMMW22}~\cite{WinterMMW22}, \href{works/HubnerGSV21.pdf}{HubnerGSV21}~\cite{HubnerGSV21}, \href{works/GeibingerKKMMW21.pdf}{GeibingerKKMMW21}~\cite{GeibingerKKMMW21}, \href{works/KoehlerBFFHPSSS21.pdf}{KoehlerBFFHPSSS21}~\cite{KoehlerBFFHPSSS21}, \href{works/PandeyS21a.pdf}{PandeyS21a}~\cite{PandeyS21a}, \href{works/Bedhief21.pdf}{Bedhief21}~\cite{Bedhief21}, \href{works/HamPK21.pdf}{HamPK21}~\cite{HamPK21}, \href{works/QinDCS20.pdf}{QinDCS20}~\cite{QinDCS20}, \href{works/ZouZ20.pdf}{ZouZ20}~\cite{ZouZ20}, \href{works/SacramentoSP20.pdf}{SacramentoSP20}~\cite{SacramentoSP20}, \href{works/MejiaY20.pdf}{MejiaY20}~\cite{MejiaY20}, \href{works/LunardiBLRV20.pdf}{LunardiBLRV20}~\cite{LunardiBLRV20}, \href{works/MengZRZL20.pdf}{MengZRZL20}~\cite{MengZRZL20}, \href{works/MurinR19.pdf}{MurinR19}~\cite{MurinR19}, \href{works/GeibingerMM19.pdf}{GeibingerMM19}~\cite{GeibingerMM19}, \href{works/abs-1911-04766.pdf}{abs-1911-04766}~\cite{abs-1911-04766}, \href{works/NishikawaSTT19.pdf}{NishikawaSTT19}~\cite{NishikawaSTT19}, \href{works/GurEA19.pdf}{GurEA19}~\cite{GurEA19}, \href{works/LaborieRSV18.pdf}{LaborieRSV18}~\cite{LaborieRSV18}, \href{works/NishikawaSTT18.pdf}{NishikawaSTT18}~\cite{NishikawaSTT18}, \href{works/NishikawaSTT18a.pdf}{NishikawaSTT18a}~\cite{NishikawaSTT18a}, \href{works/KreterSS17.pdf}{KreterSS17}~\cite{KreterSS17}, \href{works/NovaraNH16.pdf}{NovaraNH16}~\cite{NovaraNH16}, \href{works/KoschB14.pdf}{KoschB14}~\cite{KoschB14}... (Total: 34) & \href{works/LacknerMMWW23.pdf}{LacknerMMWW23}~\cite{LacknerMMWW23}, \href{works/Mehdizadeh-Somarin23.pdf}{Mehdizadeh-Somarin23}~\cite{Mehdizadeh-Somarin23}, \href{works/AbreuNP23.pdf}{AbreuNP23}~\cite{AbreuNP23}, \href{works/IsikYA23.pdf}{IsikYA23}~\cite{IsikYA23}, \href{works/CampeauG22.pdf}{CampeauG22}~\cite{CampeauG22}, \href{works/YunusogluY22.pdf}{YunusogluY22}~\cite{YunusogluY22}, \href{works/LuoB22.pdf}{LuoB22}~\cite{LuoB22}, \href{works/ColT22.pdf}{ColT22}~\cite{ColT22}, \href{works/TouatBT22.pdf}{TouatBT22}~\cite{TouatBT22}, \href{works/LacknerMMWW21.pdf}{LacknerMMWW21}~\cite{LacknerMMWW21}, \href{works/KovacsTKSG21.pdf}{KovacsTKSG21}~\cite{KovacsTKSG21}, \href{works/QinWSLS21.pdf}{QinWSLS21}~\cite{QinWSLS21}, \href{works/ArmstrongGOS21.pdf}{ArmstrongGOS21}~\cite{ArmstrongGOS21}, \href{works/MokhtarzadehTNF20.pdf}{MokhtarzadehTNF20}~\cite{MokhtarzadehTNF20}, \href{works/NattafM20.pdf}{NattafM20}~\cite{NattafM20}, \href{works/WallaceY20.pdf}{WallaceY20}~\cite{WallaceY20}, \href{works/abs-1902-09244.pdf}{abs-1902-09244}~\cite{abs-1902-09244}, \href{works/MalapertN19.pdf}{MalapertN19}~\cite{MalapertN19}, \href{works/Novas19.pdf}{Novas19}~\cite{Novas19}, \href{works/DoulabiRP16.pdf}{DoulabiRP16}~\cite{DoulabiRP16}, \href{works/HechingH16.pdf}{HechingH16}~\cite{HechingH16}, \href{works/VilimLS15.pdf}{VilimLS15}~\cite{VilimLS15}, \href{works/BofillGSV15.pdf}{BofillGSV15}~\cite{BofillGSV15}, \href{works/NattafAL15.pdf}{NattafAL15}~\cite{NattafAL15}, \href{works/PraletLJ15.pdf}{PraletLJ15}~\cite{PraletLJ15}, \href{works/BofillEGPSV14.pdf}{BofillEGPSV14}~\cite{BofillEGPSV14}, \href{works/GrimesIOS14.pdf}{GrimesIOS14}~\cite{GrimesIOS14}, \href{works/HeinzKB13.pdf}{HeinzKB13}~\cite{HeinzKB13}, \href{works/HeinzB12.pdf}{HeinzB12}~\cite{HeinzB12}... (Total: 41) & \href{works/AlfieriGPS23.pdf}{AlfieriGPS23}~\cite{AlfieriGPS23}, \href{works/JuvinHL23.pdf}{JuvinHL23}~\cite{JuvinHL23}, \href{works/SquillaciPR23.pdf}{SquillaciPR23}~\cite{SquillaciPR23}, \href{works/GurPAE23.pdf}{GurPAE23}~\cite{GurPAE23}, \href{works/PovedaAA23.pdf}{PovedaAA23}~\cite{PovedaAA23}, \href{works/YuraszeckMCCR23.pdf}{YuraszeckMCCR23}~\cite{YuraszeckMCCR23}, \href{works/AalianPG23.pdf}{AalianPG23}~\cite{AalianPG23}, \href{works/FarsiTM22.pdf}{FarsiTM22}~\cite{FarsiTM22}, \href{works/abs-2211-14492.pdf}{abs-2211-14492}~\cite{abs-2211-14492}, \href{works/YuraszeckMPV22.pdf}{YuraszeckMPV22}~\cite{YuraszeckMPV22}, \href{works/PohlAK22.pdf}{PohlAK22}~\cite{PohlAK22}, \href{works/PopovicCGNC22.pdf}{PopovicCGNC22}~\cite{PopovicCGNC22}, \href{works/AbreuN22.pdf}{AbreuN22}~\cite{AbreuN22}, \href{works/ZhangYW21.pdf}{ZhangYW21}~\cite{ZhangYW21}, \href{works/abs-2102-08778.pdf}{abs-2102-08778}~\cite{abs-2102-08778}, \href{works/GeibingerMM21.pdf}{GeibingerMM21}~\cite{GeibingerMM21}, \href{works/FanXG21.pdf}{FanXG21}~\cite{FanXG21}, \href{works/VlkHT21.pdf}{VlkHT21}~\cite{VlkHT21}, \href{works/KlankeBYE21.pdf}{KlankeBYE21}~\cite{KlankeBYE21}, \href{works/AbreuAPNM21.pdf}{AbreuAPNM21}~\cite{AbreuAPNM21}, \href{works/TangB20.pdf}{TangB20}~\cite{TangB20}, \href{works/Polo-MejiaALB20.pdf}{Polo-MejiaALB20}~\cite{Polo-MejiaALB20}, \href{works/GroleazNS20a.pdf}{GroleazNS20a}~\cite{GroleazNS20a}, \href{works/FrimodigS19.pdf}{FrimodigS19}~\cite{FrimodigS19}, \href{works/BogaerdtW19.pdf}{BogaerdtW19}~\cite{BogaerdtW19}, \href{works/EscobetPQPRA19.pdf}{EscobetPQPRA19}~\cite{EscobetPQPRA19}, \href{works/KucukY19.pdf}{KucukY19}~\cite{KucukY19}, \href{works/Ham18.pdf}{Ham18}~\cite{Ham18}, \href{works/PourDERB18.pdf}{PourDERB18}~\cite{PourDERB18}... (Total: 74)\\
CPSystems & ECLiPSe & \href{works/BadicaBI20.pdf}{BadicaBI20}~\cite{BadicaBI20}, \href{works/BadicaBIL19.pdf}{BadicaBIL19}~\cite{BadicaBIL19}, \href{works/RodosekW98.pdf}{RodosekW98}~\cite{RodosekW98} & \href{works/SchuttFSW11.pdf}{SchuttFSW11}~\cite{SchuttFSW11}, \href{works/KamarainenS02.pdf}{KamarainenS02}~\cite{KamarainenS02}, \href{works/Darby-DowmanLMZ97.pdf}{Darby-DowmanLMZ97}~\cite{Darby-DowmanLMZ97}, \href{works/Wallace96.pdf}{Wallace96}~\cite{Wallace96} & \href{works/FanXG21.pdf}{FanXG21}~\cite{FanXG21}, \href{works/MejiaY20.pdf}{MejiaY20}~\cite{MejiaY20}, \href{works/WikarekS19.pdf}{WikarekS19}~\cite{WikarekS19}, \href{works/ZeballosQH10.pdf}{ZeballosQH10}~\cite{ZeballosQH10}, \href{works/SchuttFSW09.pdf}{SchuttFSW09}~\cite{SchuttFSW09}, \href{works/BeniniBGM06.pdf}{BeniniBGM06}~\cite{BeniniBGM06}, \href{works/ChuX05.pdf}{ChuX05}~\cite{ChuX05}, \href{works/QuirogaZH05.pdf}{QuirogaZH05}~\cite{QuirogaZH05}, \href{works/MartinPY01.pdf}{MartinPY01}~\cite{MartinPY01}, \href{works/LammaMM97.pdf}{LammaMM97}~\cite{LammaMM97}\\
CPSystems & Gecode & \href{works/TardivoDFMP23.pdf}{TardivoDFMP23}~\cite{TardivoDFMP23}, \href{works/BadicaBI20.pdf}{BadicaBI20}~\cite{BadicaBI20}, \href{works/AstrandJZ20.pdf}{AstrandJZ20}~\cite{AstrandJZ20}, \href{works/BadicaBIL19.pdf}{BadicaBIL19}~\cite{BadicaBIL19}, \href{works/SzerediS16.pdf}{SzerediS16}~\cite{SzerediS16}, \href{works/ZhouGL15.pdf}{ZhouGL15}~\cite{ZhouGL15}, \href{works/GayHS15.pdf}{GayHS15}~\cite{GayHS15}, \href{works/KameugneFSN14.pdf}{KameugneFSN14}~\cite{KameugneFSN14} & \href{works/MullerMKP22.pdf}{MullerMKP22}~\cite{MullerMKP22}, \href{works/AntuoriHHEN21.pdf}{AntuoriHHEN21}~\cite{AntuoriHHEN21}, \href{works/GeibingerKKMMW21.pdf}{GeibingerKKMMW21}~\cite{GeibingerKKMMW21}, \href{works/Astrand0F21.pdf}{Astrand0F21}~\cite{Astrand0F21}, \href{works/FrohnerTR19.pdf}{FrohnerTR19}~\cite{FrohnerTR19}, \href{works/abs-1911-04766.pdf}{abs-1911-04766}~\cite{abs-1911-04766}, \href{works/GeibingerMM19.pdf}{GeibingerMM19}~\cite{GeibingerMM19}, \href{works/LaborieRSV18.pdf}{LaborieRSV18}~\cite{LaborieRSV18}, \href{works/BurtLPS15.pdf}{BurtLPS15}~\cite{BurtLPS15}, \href{works/BofillEGPSV14.pdf}{BofillEGPSV14}~\cite{BofillEGPSV14}, \href{works/KovacsK11.pdf}{KovacsK11}~\cite{KovacsK11}, \href{works/KameugneFSN11.pdf}{KameugneFSN11}~\cite{KameugneFSN11}, \href{works/ThiruvadyBME09.pdf}{ThiruvadyBME09}~\cite{ThiruvadyBME09} & \href{works/ArmstrongGOS21.pdf}{ArmstrongGOS21}~\cite{ArmstrongGOS21}, \href{works/WessenCS20.pdf}{WessenCS20}~\cite{WessenCS20}, \href{works/WallaceY20.pdf}{WallaceY20}~\cite{WallaceY20}, \href{works/MengZRZL20.pdf}{MengZRZL20}~\cite{MengZRZL20}, \href{works/FrimodigS19.pdf}{FrimodigS19}~\cite{FrimodigS19}, \href{works/YangSS19.pdf}{YangSS19}~\cite{YangSS19}, \href{works/AstrandJZ18.pdf}{AstrandJZ18}~\cite{AstrandJZ18}, \href{works/GoldwaserS17.pdf}{GoldwaserS17}~\cite{GoldwaserS17}, \href{works/PesantRR15.pdf}{PesantRR15}~\cite{PesantRR15}, \href{works/MonetteDD07.pdf}{MonetteDD07}~\cite{MonetteDD07}\\
CPSystems & Gurobi & \href{works/WangB23.pdf}{WangB23}~\cite{WangB23}, \href{works/NaderiRR23.pdf}{NaderiRR23}~\cite{NaderiRR23}, \href{works/LacknerMMWW23.pdf}{LacknerMMWW23}~\cite{LacknerMMWW23}, \href{works/WinterMMW22.pdf}{WinterMMW22}~\cite{WinterMMW22}, \href{works/KovacsTKSG21.pdf}{KovacsTKSG21}~\cite{KovacsTKSG21}, \href{works/GeibingerKKMMW21.pdf}{GeibingerKKMMW21}~\cite{GeibingerKKMMW21}, \href{works/KoehlerBFFHPSSS21.pdf}{KoehlerBFFHPSSS21}~\cite{KoehlerBFFHPSSS21}, \href{works/LacknerMMWW21.pdf}{LacknerMMWW21}~\cite{LacknerMMWW21}, \href{works/WangB20.pdf}{WangB20}~\cite{WangB20}, \href{works/WallaceY20.pdf}{WallaceY20}~\cite{WallaceY20}, \href{works/FrohnerTR19.pdf}{FrohnerTR19}~\cite{FrohnerTR19} & \href{works/VlkHT21.pdf}{VlkHT21}~\cite{VlkHT21}, \href{works/GoldwaserS17.pdf}{GoldwaserS17}~\cite{GoldwaserS17}, \href{works/FontaineMH16.pdf}{FontaineMH16}~\cite{FontaineMH16} & \href{works/KimCMLLP23.pdf}{KimCMLLP23}~\cite{KimCMLLP23}, \href{works/abs-2305-19888.pdf}{abs-2305-19888}~\cite{abs-2305-19888}, \href{works/MontemanniD23.pdf}{MontemanniD23}~\cite{MontemanniD23}, \href{works/HeinzNVH22.pdf}{HeinzNVH22}~\cite{HeinzNVH22}, \href{works/PohlAK22.pdf}{PohlAK22}~\cite{PohlAK22}, \href{works/HubnerGSV21.pdf}{HubnerGSV21}~\cite{HubnerGSV21}, \href{works/FanXG21.pdf}{FanXG21}~\cite{FanXG21}, \href{works/KlankeBYE21.pdf}{KlankeBYE21}~\cite{KlankeBYE21}, \href{works/AbohashimaEG21.pdf}{AbohashimaEG21}~\cite{AbohashimaEG21}, \href{works/BenediktMH20.pdf}{BenediktMH20}~\cite{BenediktMH20}, \href{works/MengZRZL20.pdf}{MengZRZL20}~\cite{MengZRZL20}, \href{works/He0GLW18.pdf}{He0GLW18}~\cite{He0GLW18}, \href{works/DemirovicS18.pdf}{DemirovicS18}~\cite{DemirovicS18}, \href{works/BenediktSMVH18.pdf}{BenediktSMVH18}~\cite{BenediktSMVH18}, \href{works/BurtLPS15.pdf}{BurtLPS15}~\cite{BurtLPS15}, \href{works/PesantRR15.pdf}{PesantRR15}~\cite{PesantRR15}\\
CPSystems & Ilog Scheduler & \href{works/GrimesH11.pdf}{GrimesH11}~\cite{GrimesH11}, \href{works/ZeballosQH10.pdf}{ZeballosQH10}~\cite{ZeballosQH10} & \href{works/LaborieRSV18.pdf}{LaborieRSV18}~\cite{LaborieRSV18}, \href{works/NovasH12.pdf}{NovasH12}~\cite{NovasH12}, \href{works/HeinzB12.pdf}{HeinzB12}~\cite{HeinzB12}, \href{works/LimtanyakulS12.pdf}{LimtanyakulS12}~\cite{LimtanyakulS12}, \href{works/BeckFW11.pdf}{BeckFW11}~\cite{BeckFW11}, \href{works/GrimesHM09.pdf}{GrimesHM09}~\cite{GrimesHM09}, \href{works/WatsonB08.pdf}{WatsonB08}~\cite{WatsonB08}, \href{works/ZeballosH05.pdf}{ZeballosH05}~\cite{ZeballosH05}, \href{works/NuijtenP98.pdf}{NuijtenP98}~\cite{NuijtenP98} & \href{works/Laborie18a.pdf}{Laborie18a}~\cite{Laborie18a}, \href{works/SchuttS16.pdf}{SchuttS16}~\cite{SchuttS16}, \href{works/NovasH14.pdf}{NovasH14}~\cite{NovasH14}, \href{works/BeniniLMR11.pdf}{BeniniLMR11}~\cite{BeniniLMR11}, \href{works/KovacsB11.pdf}{KovacsB11}~\cite{KovacsB11}, \href{works/SchuttFSW11.pdf}{SchuttFSW11}~\cite{SchuttFSW11}, \href{works/LahimerLH11.pdf}{LahimerLH11}~\cite{LahimerLH11}, \href{works/HachemiGR11.pdf}{HachemiGR11}~\cite{HachemiGR11}, \href{works/LopesCSM10.pdf}{LopesCSM10}~\cite{LopesCSM10}, \href{works/NovasH10.pdf}{NovasH10}~\cite{NovasH10}, \href{works/Vilim09a.pdf}{Vilim09a}~\cite{Vilim09a}, \href{works/RuggieroBBMA09.pdf}{RuggieroBBMA09}~\cite{RuggieroBBMA09}, \href{works/KovacsB08.pdf}{KovacsB08}~\cite{KovacsB08}, \href{works/MouraSCL08a.pdf}{MouraSCL08a}~\cite{MouraSCL08a}, \href{works/MouraSCL08.pdf}{MouraSCL08}~\cite{MouraSCL08}, \href{works/HoeveGSL07.pdf}{HoeveGSL07}~\cite{HoeveGSL07}, \href{works/Rodriguez07.pdf}{Rodriguez07}~\cite{Rodriguez07}, \href{works/Simonis07.pdf}{Simonis07}~\cite{Simonis07}, \href{works/KovacsV06.pdf}{KovacsV06}~\cite{KovacsV06}, \href{works/Hooker06.pdf}{Hooker06}~\cite{Hooker06}, \href{works/WuBB05.pdf}{WuBB05}~\cite{WuBB05}, \href{works/ArtiouchineB05.pdf}{ArtiouchineB05}~\cite{ArtiouchineB05}, \href{works/QuirogaZH05.pdf}{QuirogaZH05}~\cite{QuirogaZH05}, \href{works/GodardLN05.pdf}{GodardLN05}~\cite{GodardLN05}, \href{works/Hooker05a.pdf}{Hooker05a}~\cite{Hooker05a}, \href{works/Hooker05.pdf}{Hooker05}~\cite{Hooker05}, \href{works/KovacsV04.pdf}{KovacsV04}~\cite{KovacsV04}, \href{works/ArtiguesBF04.pdf}{ArtiguesBF04}~\cite{ArtiguesBF04}, \href{works/Hooker04.pdf}{Hooker04}~\cite{Hooker04}... (Total: 34)\\
CPSystems & Ilog Solver &  & \href{works/GrimesH11.pdf}{GrimesH11}~\cite{GrimesH11}, \href{works/ZeballosQH10.pdf}{ZeballosQH10}~\cite{ZeballosQH10} & \href{works/abs-1902-01193.pdf}{abs-1902-01193}~\cite{abs-1902-01193}, \href{works/LaborieRSV18.pdf}{LaborieRSV18}~\cite{LaborieRSV18}, \href{works/ZarandiKS16.pdf}{ZarandiKS16}~\cite{ZarandiKS16}, \href{works/PesantRR15.pdf}{PesantRR15}~\cite{PesantRR15}, \href{works/BonfiettiLBM14.pdf}{BonfiettiLBM14}~\cite{BonfiettiLBM14}, \href{works/NovasH14.pdf}{NovasH14}~\cite{NovasH14}, \href{works/OzturkTHO13.pdf}{OzturkTHO13}~\cite{OzturkTHO13}, \href{works/BonfiettiLBM12.pdf}{BonfiettiLBM12}~\cite{BonfiettiLBM12}, \href{works/NovasH12.pdf}{NovasH12}~\cite{NovasH12}, \href{works/HeinzB12.pdf}{HeinzB12}~\cite{HeinzB12}, \href{works/LombardiM12a.pdf}{LombardiM12a}~\cite{LombardiM12a}, \href{works/KelbelH11.pdf}{KelbelH11}~\cite{KelbelH11}, \href{works/BonfiettiLBM11.pdf}{BonfiettiLBM11}~\cite{BonfiettiLBM11}, \href{works/KovacsK11.pdf}{KovacsK11}~\cite{KovacsK11}, \href{works/KovacsB11.pdf}{KovacsB11}~\cite{KovacsB11}, \href{works/TopalogluO11.pdf}{TopalogluO11}~\cite{TopalogluO11}, \href{works/LombardiM10.pdf}{LombardiM10}~\cite{LombardiM10}, \href{works/LopesCSM10.pdf}{LopesCSM10}~\cite{LopesCSM10}, \href{works/LombardiM09.pdf}{LombardiM09}~\cite{LombardiM09}, \href{works/RuggieroBBMA09.pdf}{RuggieroBBMA09}~\cite{RuggieroBBMA09}, \href{works/MouraSCL08a.pdf}{MouraSCL08a}~\cite{MouraSCL08a}, \href{works/MouraSCL08.pdf}{MouraSCL08}~\cite{MouraSCL08}, \href{works/KovacsB08.pdf}{KovacsB08}~\cite{KovacsB08}, \href{works/Rodriguez07.pdf}{Rodriguez07}~\cite{Rodriguez07}, \href{works/GomesHS06.pdf}{GomesHS06}~\cite{GomesHS06}, \href{works/BeniniBGM06.pdf}{BeniniBGM06}~\cite{BeniniBGM06}, \href{works/QuirogaZH05.pdf}{QuirogaZH05}~\cite{QuirogaZH05}, \href{works/ZeballosH05.pdf}{ZeballosH05}~\cite{ZeballosH05}, \href{works/GodardLN05.pdf}{GodardLN05}~\cite{GodardLN05}... (Total: 41)\\
CPSystems & MiniZinc & \href{works/LacknerMMWW23.pdf}{LacknerMMWW23}~\cite{LacknerMMWW23}, \href{works/TardivoDFMP23.pdf}{TardivoDFMP23}~\cite{TardivoDFMP23}, \href{works/ColT22.pdf}{ColT22}~\cite{ColT22}, \href{works/BoudreaultSLQ22.pdf}{BoudreaultSLQ22}~\cite{BoudreaultSLQ22}, \href{works/MullerMKP22.pdf}{MullerMKP22}~\cite{MullerMKP22}, \href{works/JungblutK22.pdf}{JungblutK22}~\cite{JungblutK22}, \href{works/ArmstrongGOS21.pdf}{ArmstrongGOS21}~\cite{ArmstrongGOS21}, \href{works/KoehlerBFFHPSSS21.pdf}{KoehlerBFFHPSSS21}~\cite{KoehlerBFFHPSSS21}, \href{works/LacknerMMWW21.pdf}{LacknerMMWW21}~\cite{LacknerMMWW21}, \href{works/Mercier-AubinGQ20.pdf}{Mercier-AubinGQ20}~\cite{Mercier-AubinGQ20}, \href{works/WallaceY20.pdf}{WallaceY20}~\cite{WallaceY20}, \href{works/abs-1911-04766.pdf}{abs-1911-04766}~\cite{abs-1911-04766}, \href{works/ColT19.pdf}{ColT19}~\cite{ColT19}, \href{works/FrohnerTR19.pdf}{FrohnerTR19}~\cite{FrohnerTR19}, \href{works/GeibingerMM19.pdf}{GeibingerMM19}~\cite{GeibingerMM19}, \href{works/YoungFS17.pdf}{YoungFS17}~\cite{YoungFS17}, \href{works/LiuCGM17.pdf}{LiuCGM17}~\cite{LiuCGM17}, \href{works/SzerediS16.pdf}{SzerediS16}~\cite{SzerediS16}, \href{works/BofillEGPSV14.pdf}{BofillEGPSV14}~\cite{BofillEGPSV14}, \href{works/KelarevaTK13.pdf}{KelarevaTK13}~\cite{KelarevaTK13} & \href{works/PovedaAA23.pdf}{PovedaAA23}~\cite{PovedaAA23}, \href{works/KreterSS17.pdf}{KreterSS17}~\cite{KreterSS17}, \href{works/KreterSS15.pdf}{KreterSS15}~\cite{KreterSS15} & \href{works/Bit-Monnot23.pdf}{Bit-Monnot23}~\cite{Bit-Monnot23}, \href{works/OuelletQ22.pdf}{OuelletQ22}~\cite{OuelletQ22}, \href{works/GeibingerKKMMW21.pdf}{GeibingerKKMMW21}~\cite{GeibingerKKMMW21}, \href{works/abs-2102-08778.pdf}{abs-2102-08778}~\cite{abs-2102-08778}, \href{works/abs-1901-07914.pdf}{abs-1901-07914}~\cite{abs-1901-07914}, \href{works/FrimodigS19.pdf}{FrimodigS19}~\cite{FrimodigS19}, \href{works/BehrensLM19.pdf}{BehrensLM19}~\cite{BehrensLM19}, \href{works/DemirovicS18.pdf}{DemirovicS18}~\cite{DemirovicS18}, \href{works/FontaineMH16.pdf}{FontaineMH16}~\cite{FontaineMH16}, \href{works/SchuttS16.pdf}{SchuttS16}~\cite{SchuttS16}, \href{works/BurtLPS15.pdf}{BurtLPS15}~\cite{BurtLPS15}, \href{works/HeinzSB13.pdf}{HeinzSB13}~\cite{HeinzSB13}, \href{works/SchuttFS13.pdf}{SchuttFS13}~\cite{SchuttFS13}\\
CPSystems & Mistral & \href{works/JuvinHHL23.pdf}{JuvinHHL23}~\cite{JuvinHHL23}, \href{works/GrimesHM09.pdf}{GrimesHM09}~\cite{GrimesHM09} & \href{works/Bit-Monnot23.pdf}{Bit-Monnot23}~\cite{Bit-Monnot23}, \href{works/BillautHL12.pdf}{BillautHL12}~\cite{BillautHL12} & \href{works/SialaAH15.pdf}{SialaAH15}~\cite{SialaAH15}\\
CPSystems & OPL & \href{works/LacknerMMWW23.pdf}{LacknerMMWW23}~\cite{LacknerMMWW23}, \href{works/YunusogluY22.pdf}{YunusogluY22}~\cite{YunusogluY22}, \href{works/MullerMKP22.pdf}{MullerMKP22}~\cite{MullerMKP22}, \href{works/TouatBT22.pdf}{TouatBT22}~\cite{TouatBT22}, \href{works/ColT22.pdf}{ColT22}~\cite{ColT22}, \href{works/LacknerMMWW21.pdf}{LacknerMMWW21}~\cite{LacknerMMWW21}, \href{works/PandeyS21a.pdf}{PandeyS21a}~\cite{PandeyS21a}, \href{works/KoehlerBFFHPSSS21.pdf}{KoehlerBFFHPSSS21}~\cite{KoehlerBFFHPSSS21}, \href{works/QinDCS20.pdf}{QinDCS20}~\cite{QinDCS20}, \href{works/Novas19.pdf}{Novas19}~\cite{Novas19}, \href{works/EscobetPQPRA19.pdf}{EscobetPQPRA19}~\cite{EscobetPQPRA19}, \href{works/TangLWSK18.pdf}{TangLWSK18}~\cite{TangLWSK18}, \href{works/LaborieRSV18.pdf}{LaborieRSV18}~\cite{LaborieRSV18}, \href{works/NovaraNH16.pdf}{NovaraNH16}~\cite{NovaraNH16}, \href{works/AlesioNBG14.pdf}{AlesioNBG14}~\cite{AlesioNBG14}, \href{works/NovasH12.pdf}{NovasH12}~\cite{NovasH12}, \href{works/HachemiGR11.pdf}{HachemiGR11}~\cite{HachemiGR11}, \href{works/ZeballosQH10.pdf}{ZeballosQH10}~\cite{ZeballosQH10}, \href{works/Laborie09.pdf}{Laborie09}~\cite{Laborie09}, \href{works/KhayatLR06.pdf}{KhayatLR06}~\cite{KhayatLR06}, \href{works/AggounB93.pdf}{AggounB93}~\cite{AggounB93} & \href{works/SubulanC22.pdf}{SubulanC22}~\cite{SubulanC22}, \href{works/Teppan22.pdf}{Teppan22}~\cite{Teppan22}, \href{works/Mercier-AubinGQ20.pdf}{Mercier-AubinGQ20}~\cite{Mercier-AubinGQ20}, \href{works/ZouZ20.pdf}{ZouZ20}~\cite{ZouZ20}, \href{works/MurinR19.pdf}{MurinR19}~\cite{MurinR19}, \href{works/Laborie18a.pdf}{Laborie18a}~\cite{Laborie18a}, \href{works/LimBTBB15.pdf}{LimBTBB15}~\cite{LimBTBB15}, \href{works/WangMD15.pdf}{WangMD15}~\cite{WangMD15}, \href{works/EvenSH15a.pdf}{EvenSH15a}~\cite{EvenSH15a}, \href{works/NovasH14.pdf}{NovasH14}~\cite{NovasH14}, \href{works/OzturkTHO13.pdf}{OzturkTHO13}~\cite{OzturkTHO13}, \href{works/SerraNM12.pdf}{SerraNM12}~\cite{SerraNM12}, \href{works/HeinzB12.pdf}{HeinzB12}~\cite{HeinzB12}, \href{works/TopalogluO11.pdf}{TopalogluO11}~\cite{TopalogluO11}, \href{works/EdisO11.pdf}{EdisO11}~\cite{EdisO11}, \href{works/KelbelH11.pdf}{KelbelH11}~\cite{KelbelH11}, \href{works/ZibranR11a.pdf}{ZibranR11a}~\cite{ZibranR11a}, \href{works/NovasH10.pdf}{NovasH10}~\cite{NovasH10}, \href{works/Simonis07.pdf}{Simonis07}~\cite{Simonis07}, \href{works/GarganiR07.pdf}{GarganiR07}~\cite{GarganiR07}, \href{works/KrogtLPHJ07.pdf}{KrogtLPHJ07}~\cite{KrogtLPHJ07}, \href{works/Hooker06.pdf}{Hooker06}~\cite{Hooker06}, \href{works/ZeballosH05.pdf}{ZeballosH05}~\cite{ZeballosH05}, \href{works/QuirogaZH05.pdf}{QuirogaZH05}~\cite{QuirogaZH05}, \href{works/Hooker05a.pdf}{Hooker05a}~\cite{Hooker05a}, \href{works/LorigeonBB02.pdf}{LorigeonBB02}~\cite{LorigeonBB02}, \href{works/VerfaillieL01.pdf}{VerfaillieL01}~\cite{VerfaillieL01}, \href{works/RodosekW98.pdf}{RodosekW98}~\cite{RodosekW98} & \href{works/abs-2402-00459.pdf}{abs-2402-00459}~\cite{abs-2402-00459}, \href{works/GurPAE23.pdf}{GurPAE23}~\cite{GurPAE23}, \href{works/CzerniachowskaWZ23.pdf}{CzerniachowskaWZ23}~\cite{CzerniachowskaWZ23}, \href{works/MontemanniD23.pdf}{MontemanniD23}~\cite{MontemanniD23}, \href{works/IsikYA23.pdf}{IsikYA23}~\cite{IsikYA23}, \href{works/EfthymiouY23.pdf}{EfthymiouY23}~\cite{EfthymiouY23}, \href{works/YuraszeckMCCR23.pdf}{YuraszeckMCCR23}~\cite{YuraszeckMCCR23}, \href{works/PerezGSL23.pdf}{PerezGSL23}~\cite{PerezGSL23}, \href{works/AbreuNP23.pdf}{AbreuNP23}~\cite{AbreuNP23}, \href{works/abs-2312-13682.pdf}{abs-2312-13682}~\cite{abs-2312-13682}, \href{works/GeitzGSSW22.pdf}{GeitzGSSW22}~\cite{GeitzGSSW22}, \href{works/ArmstrongGOS22.pdf}{ArmstrongGOS22}~\cite{ArmstrongGOS22}, \href{works/BoudreaultSLQ22.pdf}{BoudreaultSLQ22}~\cite{BoudreaultSLQ22}, \href{works/OujanaAYB22.pdf}{OujanaAYB22}~\cite{OujanaAYB22}, \href{works/LiFJZLL22.pdf}{LiFJZLL22}~\cite{LiFJZLL22}, \href{works/VlkHT21.pdf}{VlkHT21}~\cite{VlkHT21}, \href{works/Bedhief21.pdf}{Bedhief21}~\cite{Bedhief21}, \href{works/HamPK21.pdf}{HamPK21}~\cite{HamPK21}, \href{works/QinWSLS21.pdf}{QinWSLS21}~\cite{QinWSLS21}, \href{works/abs-2102-08778.pdf}{abs-2102-08778}~\cite{abs-2102-08778}, \href{works/HubnerGSV21.pdf}{HubnerGSV21}~\cite{HubnerGSV21}, \href{works/WallaceY20.pdf}{WallaceY20}~\cite{WallaceY20}, \href{works/MengZRZL20.pdf}{MengZRZL20}~\cite{MengZRZL20}, \href{works/BogaerdtW19.pdf}{BogaerdtW19}~\cite{BogaerdtW19}, \href{works/YounespourAKE19.pdf}{YounespourAKE19}~\cite{YounespourAKE19}, \href{works/abs-1902-09244.pdf}{abs-1902-09244}~\cite{abs-1902-09244}, \href{works/Tom19.pdf}{Tom19}~\cite{Tom19}, \href{works/YangSS19.pdf}{YangSS19}~\cite{YangSS19}, \href{works/abs-1902-01193.pdf}{abs-1902-01193}~\cite{abs-1902-01193}... (Total: 69)\\
CPSystems & OR-Tools & \href{works/abs-2402-00459.pdf}{abs-2402-00459}~\cite{abs-2402-00459}, \href{works/LacknerMMWW23.pdf}{LacknerMMWW23}~\cite{LacknerMMWW23}, \href{works/abs-2211-14492.pdf}{abs-2211-14492}~\cite{abs-2211-14492}, \href{works/ColT22.pdf}{ColT22}~\cite{ColT22}, \href{works/MullerMKP22.pdf}{MullerMKP22}~\cite{MullerMKP22}, \href{works/abs-2102-08778.pdf}{abs-2102-08778}~\cite{abs-2102-08778}, \href{works/KovacsTKSG21.pdf}{KovacsTKSG21}~\cite{KovacsTKSG21}, \href{works/LacknerMMWW21.pdf}{LacknerMMWW21}~\cite{LacknerMMWW21}, \href{works/KoehlerBFFHPSSS21.pdf}{KoehlerBFFHPSSS21}~\cite{KoehlerBFFHPSSS21}, \href{works/FallahiAC20.pdf}{FallahiAC20}~\cite{FallahiAC20}, \href{works/ColT19.pdf}{ColT19}~\cite{ColT19}, \href{works/GayHS15.pdf}{GayHS15}~\cite{GayHS15} & \href{works/EfthymiouY23.pdf}{EfthymiouY23}~\cite{EfthymiouY23}, \href{works/BoudreaultSLQ22.pdf}{BoudreaultSLQ22}~\cite{BoudreaultSLQ22}, \href{works/GeibingerKKMMW21.pdf}{GeibingerKKMMW21}~\cite{GeibingerKKMMW21}, \href{works/BarzegaranZP20.pdf}{BarzegaranZP20}~\cite{BarzegaranZP20}, \href{works/LiuCGM17.pdf}{LiuCGM17}~\cite{LiuCGM17} & \href{works/Bit-Monnot23.pdf}{Bit-Monnot23}~\cite{Bit-Monnot23}, \href{works/KimCMLLP23.pdf}{KimCMLLP23}~\cite{KimCMLLP23}, \href{works/MontemanniD23.pdf}{MontemanniD23}~\cite{MontemanniD23}, \href{works/AkramNHRSA23.pdf}{AkramNHRSA23}~\cite{AkramNHRSA23}, \href{works/MontemanniD23a.pdf}{MontemanniD23a}~\cite{MontemanniD23a}, \href{works/Teppan22.pdf}{Teppan22}~\cite{Teppan22}, \href{works/KlankeBYE21.pdf}{KlankeBYE21}~\cite{KlankeBYE21}, \href{works/MengZRZL20.pdf}{MengZRZL20}~\cite{MengZRZL20}, \href{works/GroleazNS20.pdf}{GroleazNS20}~\cite{GroleazNS20}, \href{works/GalleguillosKSB19.pdf}{GalleguillosKSB19}~\cite{GalleguillosKSB19}, \href{works/BehrensLM19.pdf}{BehrensLM19}~\cite{BehrensLM19}, \href{works/abs-1901-07914.pdf}{abs-1901-07914}~\cite{abs-1901-07914}, \href{works/YangSS19.pdf}{YangSS19}~\cite{YangSS19}, \href{works/PourDERB18.pdf}{PourDERB18}~\cite{PourDERB18}, \href{works/BonfiettiZLM16.pdf}{BonfiettiZLM16}~\cite{BonfiettiZLM16}, \href{works/ZhouGL15.pdf}{ZhouGL15}~\cite{ZhouGL15}, \href{works/LombardiM12.pdf}{LombardiM12}~\cite{LombardiM12}\\
CPSystems & OZ & \href{works/PrataAN23.pdf}{PrataAN23}~\cite{PrataAN23}, \href{works/NaderiRR23.pdf}{NaderiRR23}~\cite{NaderiRR23}, \href{works/CzerniachowskaWZ23.pdf}{CzerniachowskaWZ23}~\cite{CzerniachowskaWZ23}, \href{works/IsikYA23.pdf}{IsikYA23}~\cite{IsikYA23}, \href{works/YunusogluY22.pdf}{YunusogluY22}~\cite{YunusogluY22}, \href{works/WikarekS19.pdf}{WikarekS19}~\cite{WikarekS19}, \href{works/GokgurHO18.pdf}{GokgurHO18}~\cite{GokgurHO18}, \href{works/TopalogluO11.pdf}{TopalogluO11}~\cite{TopalogluO11}, \href{works/NovasH10.pdf}{NovasH10}~\cite{NovasH10}, \href{works/RuggieroBBMA09.pdf}{RuggieroBBMA09}~\cite{RuggieroBBMA09}, \href{works/VanczaM01.pdf}{VanczaM01}~\cite{VanczaM01}, \href{works/SchildW00.pdf}{SchildW00}~\cite{SchildW00}, \href{works/BeldiceanuC94.pdf}{BeldiceanuC94}~\cite{BeldiceanuC94} & \href{works/GeitzGSSW22.pdf}{GeitzGSSW22}~\cite{GeitzGSSW22}, \href{works/BourreauGGLT22.pdf}{BourreauGGLT22}~\cite{BourreauGGLT22}, \href{works/AbreuN22.pdf}{AbreuN22}~\cite{AbreuN22}, \href{works/SubulanC22.pdf}{SubulanC22}~\cite{SubulanC22}, \href{works/PohlAK22.pdf}{PohlAK22}~\cite{PohlAK22}, \href{works/FanXG21.pdf}{FanXG21}~\cite{FanXG21}, \href{works/GodetLHS20.pdf}{GodetLHS20}~\cite{GodetLHS20}, \href{works/AstrandJZ20.pdf}{AstrandJZ20}~\cite{AstrandJZ20}, \href{works/WessenCS20.pdf}{WessenCS20}~\cite{WessenCS20}, \href{works/abs-1901-07914.pdf}{abs-1901-07914}~\cite{abs-1901-07914}, \href{works/LiuLH19.pdf}{LiuLH19}~\cite{LiuLH19}, \href{works/Novas19.pdf}{Novas19}~\cite{Novas19}, \href{works/BehrensLM19.pdf}{BehrensLM19}~\cite{BehrensLM19}, \href{works/Hooker17.pdf}{Hooker17}~\cite{Hooker17}, \href{works/BridiBLMB16.pdf}{BridiBLMB16}~\cite{BridiBLMB16}, \href{works/EdisO11.pdf}{EdisO11}~\cite{EdisO11}, \href{works/GrimesH11.pdf}{GrimesH11}~\cite{GrimesH11}, \href{works/ZeballosQH10.pdf}{ZeballosQH10}~\cite{ZeballosQH10}, \href{works/BocewiczBB09.pdf}{BocewiczBB09}~\cite{BocewiczBB09}, \href{works/LiessM08.pdf}{LiessM08}~\cite{LiessM08}, \href{works/SureshMOK06.pdf}{SureshMOK06}~\cite{SureshMOK06}, \href{works/BeniniBGM06.pdf}{BeniniBGM06}~\cite{BeniniBGM06}, \href{works/GodardLN05.pdf}{GodardLN05}~\cite{GodardLN05}, \href{works/MaraveliasG04.pdf}{MaraveliasG04}~\cite{MaraveliasG04} & \href{works/Mehdizadeh-Somarin23.pdf}{Mehdizadeh-Somarin23}~\cite{Mehdizadeh-Somarin23}, \href{works/GurPAE23.pdf}{GurPAE23}~\cite{GurPAE23}, \href{works/MullerMKP22.pdf}{MullerMKP22}~\cite{MullerMKP22}, \href{works/CampeauG22.pdf}{CampeauG22}~\cite{CampeauG22}, \href{works/ZhangJZL22.pdf}{ZhangJZL22}~\cite{ZhangJZL22}, \href{works/ArmstrongGOS22.pdf}{ArmstrongGOS22}~\cite{ArmstrongGOS22}, \href{works/FetgoD22.pdf}{FetgoD22}~\cite{FetgoD22}, \href{works/TouatBT22.pdf}{TouatBT22}~\cite{TouatBT22}, \href{works/abs-2211-14492.pdf}{abs-2211-14492}~\cite{abs-2211-14492}, \href{works/LiFJZLL22.pdf}{LiFJZLL22}~\cite{LiFJZLL22}, \href{works/PopovicCGNC22.pdf}{PopovicCGNC22}~\cite{PopovicCGNC22}, \href{works/AbreuAPNM21.pdf}{AbreuAPNM21}~\cite{AbreuAPNM21}, \href{works/ArmstrongGOS21.pdf}{ArmstrongGOS21}~\cite{ArmstrongGOS21}, \href{works/Bedhief21.pdf}{Bedhief21}~\cite{Bedhief21}, \href{works/LacknerMMWW21.pdf}{LacknerMMWW21}~\cite{LacknerMMWW21}, \href{works/QinWSLS21.pdf}{QinWSLS21}~\cite{QinWSLS21}, \href{works/PandeyS21a.pdf}{PandeyS21a}~\cite{PandeyS21a}, \href{works/WangB20.pdf}{WangB20}~\cite{WangB20}, \href{works/SacramentoSP20.pdf}{SacramentoSP20}~\cite{SacramentoSP20}, \href{works/FallahiAC20.pdf}{FallahiAC20}~\cite{FallahiAC20}, \href{works/abs-1911-04766.pdf}{abs-1911-04766}~\cite{abs-1911-04766}, \href{works/GurEA19.pdf}{GurEA19}~\cite{GurEA19}, \href{works/Tom19.pdf}{Tom19}~\cite{Tom19}, \href{works/abs-1902-09244.pdf}{abs-1902-09244}~\cite{abs-1902-09244}, \href{works/FrimodigS19.pdf}{FrimodigS19}~\cite{FrimodigS19}, \href{works/NishikawaSTT19.pdf}{NishikawaSTT19}~\cite{NishikawaSTT19}, \href{works/GalleguillosKSB19.pdf}{GalleguillosKSB19}~\cite{GalleguillosKSB19}, \href{works/ArbaouiY18.pdf}{ArbaouiY18}~\cite{ArbaouiY18}, \href{works/BenediktSMVH18.pdf}{BenediktSMVH18}~\cite{BenediktSMVH18}... (Total: 74)\\
CPSystems & SICStus & \href{works/ArmstrongGOS21.pdf}{ArmstrongGOS21}~\cite{ArmstrongGOS21}, \href{works/LetortCB15.pdf}{LetortCB15}~\cite{LetortCB15}, \href{works/LetortCB13.pdf}{LetortCB13}~\cite{LetortCB13}, \href{works/LetortBC12.pdf}{LetortBC12}~\cite{LetortBC12} & \href{works/MossigeGSMC17.pdf}{MossigeGSMC17}~\cite{MossigeGSMC17}, \href{works/SchuttFSW11.pdf}{SchuttFSW11}~\cite{SchuttFSW11}, \href{works/QuSN06.pdf}{QuSN06}~\cite{QuSN06} & \href{works/ArmstrongGOS22.pdf}{ArmstrongGOS22}~\cite{ArmstrongGOS22}, \href{works/PopovicCGNC22.pdf}{PopovicCGNC22}~\cite{PopovicCGNC22}, \href{works/YangSS19.pdf}{YangSS19}~\cite{YangSS19}, \href{works/Madi-WambaLOBM17.pdf}{Madi-WambaLOBM17}~\cite{Madi-WambaLOBM17}, \href{works/JelinekB16.pdf}{JelinekB16}~\cite{JelinekB16}, \href{works/BeldiceanuCDP11.pdf}{BeldiceanuCDP11}~\cite{BeldiceanuCDP11}, \href{works/TrojetHL11.pdf}{TrojetHL11}~\cite{TrojetHL11}, \href{works/BartakCS10.pdf}{BartakCS10}~\cite{BartakCS10}, \href{works/SchuttFSW09.pdf}{SchuttFSW09}~\cite{SchuttFSW09}, \href{works/BeldiceanuCP08.pdf}{BeldiceanuCP08}~\cite{BeldiceanuCP08}, \href{works/Geske05.pdf}{Geske05}~\cite{Geske05}, \href{works/Bartak02.pdf}{Bartak02}~\cite{Bartak02}, \href{works/BeldiceanuC02.pdf}{BeldiceanuC02}~\cite{BeldiceanuC02}\\
CPSystems & Z3 & \href{works/KoehlerBFFHPSSS21.pdf}{KoehlerBFFHPSSS21}~\cite{KoehlerBFFHPSSS21}, \href{works/YounespourAKE19.pdf}{YounespourAKE19}~\cite{YounespourAKE19}, \href{works/SureshMOK06.pdf}{SureshMOK06}~\cite{SureshMOK06} & \href{works/NaderiRR23.pdf}{NaderiRR23}~\cite{NaderiRR23}, \href{works/VlkHT21.pdf}{VlkHT21}~\cite{VlkHT21}, \href{works/WikarekS19.pdf}{WikarekS19}~\cite{WikarekS19}, \href{works/Zhou97.pdf}{Zhou97}~\cite{Zhou97} & \href{works/ZhangW18.pdf}{ZhangW18}~\cite{ZhangW18}, \href{works/BofillCSV17.pdf}{BofillCSV17}~\cite{BofillCSV17}, \href{works/BertholdHLMS10.pdf}{BertholdHLMS10}~\cite{BertholdHLMS10}, \href{works/Rodriguez07.pdf}{Rodriguez07}~\cite{Rodriguez07}, \href{works/Zhou96.pdf}{Zhou96}~\cite{Zhou96}\\
\end{longtable}
}


\clearpage
\subsection{Concept Type ApplicationAreas}
\label{sec:ApplicationAreas}
{\scriptsize
\begin{longtable}{lp{3cm}>{\raggedright\arraybackslash}p{6cm}>{\raggedright\arraybackslash}p{6cm}>{\raggedright\arraybackslash}p{8cm}}
\rowcolor{white}\caption{Works for Concepts of Type ApplicationAreas}\\ \toprule
\rowcolor{white}Type & Keyword & High & Medium & Low\\ \midrule\endhead
\bottomrule
\endfoot
ApplicationAreas & COVID &  & \href{works/GeibingerKKMMW21.pdf}{GeibingerKKMMW21}~\cite{GeibingerKKMMW21} & \href{works/Mehdizadeh-Somarin23.pdf}{Mehdizadeh-Somarin23}~\cite{Mehdizadeh-Somarin23}, \href{works/GurPAE23.pdf}{GurPAE23}~\cite{GurPAE23}, \href{works/OujanaAYB22.pdf}{OujanaAYB22}~\cite{OujanaAYB22}\\
ApplicationAreas & HVAC & \href{works/LimHTB16.pdf}{LimHTB16}~\cite{LimHTB16}, \href{works/LimBTBB15.pdf}{LimBTBB15}~\cite{LimBTBB15}, \href{works/GrimesIOS14.pdf}{GrimesIOS14}~\cite{GrimesIOS14} &  & \\
ApplicationAreas & agriculture &  &  & \href{works/AkramNHRSA23.pdf}{AkramNHRSA23}~\cite{AkramNHRSA23}, \href{works/BenderWS21.pdf}{BenderWS21}~\cite{BenderWS21}, \href{works/HamPK21.pdf}{HamPK21}~\cite{HamPK21}, \href{works/QinWSLS21.pdf}{QinWSLS21}~\cite{QinWSLS21}, \href{works/Astrand0F21.pdf}{Astrand0F21}~\cite{Astrand0F21}, \href{works/MejiaY20.pdf}{MejiaY20}~\cite{MejiaY20}\\
ApplicationAreas & aircraft & \href{works/PohlAK22.pdf}{PohlAK22}~\cite{PohlAK22}, \href{works/WangB20.pdf}{WangB20}~\cite{WangB20}, \href{works/LombardiM12.pdf}{LombardiM12}~\cite{LombardiM12}, \href{works/FrankK05.pdf}{FrankK05}~\cite{FrankK05}, \href{works/ArtiouchineB05.pdf}{ArtiouchineB05}~\cite{ArtiouchineB05} & \href{works/WangB23.pdf}{WangB23}~\cite{WangB23}, \href{works/Ham18.pdf}{Ham18}~\cite{Ham18}, \href{works/Simonis07.pdf}{Simonis07}~\cite{Simonis07}, \href{works/SakkoutW00.pdf}{SakkoutW00}~\cite{SakkoutW00} & \href{works/PrataAN23.pdf}{PrataAN23}~\cite{PrataAN23}, \href{works/PovedaAA23.pdf}{PovedaAA23}~\cite{PovedaAA23}, \href{works/abs-1902-09244.pdf}{abs-1902-09244}~\cite{abs-1902-09244}, \href{works/LaborieRSV18.pdf}{LaborieRSV18}~\cite{LaborieRSV18}, \href{works/Laborie09.pdf}{Laborie09}~\cite{Laborie09}, \href{works/KovacsB08.pdf}{KovacsB08}~\cite{KovacsB08}, \href{works/KrogtLPHJ07.pdf}{KrogtLPHJ07}~\cite{KrogtLPHJ07}, \href{works/MartinPY01.pdf}{MartinPY01}~\cite{MartinPY01}, \href{works/GruianK98.pdf}{GruianK98}~\cite{GruianK98}, \href{works/Darby-DowmanLMZ97.pdf}{Darby-DowmanLMZ97}~\cite{Darby-DowmanLMZ97}, \href{works/Wallace96.pdf}{Wallace96}~\cite{Wallace96}, \href{works/Simonis95.pdf}{Simonis95}~\cite{Simonis95}, \href{works/SimonisC95.pdf}{SimonisC95}~\cite{SimonisC95}\\
ApplicationAreas & automotive &  & \href{works/YuraszeckMPV22.pdf}{YuraszeckMPV22}~\cite{YuraszeckMPV22}, \href{works/LimtanyakulS12.pdf}{LimtanyakulS12}~\cite{LimtanyakulS12}, \href{works/SunLYL10.pdf}{SunLYL10}~\cite{SunLYL10}, \href{works/BarlattCG08.pdf}{BarlattCG08}~\cite{BarlattCG08}, \href{works/SchildW00.pdf}{SchildW00}~\cite{SchildW00} & \href{works/PovedaAA23.pdf}{PovedaAA23}~\cite{PovedaAA23}, \href{works/NaderiRR23.pdf}{NaderiRR23}~\cite{NaderiRR23}, \href{works/CzerniachowskaWZ23.pdf}{CzerniachowskaWZ23}~\cite{CzerniachowskaWZ23}, \href{works/AntuoriHHEN21.pdf}{AntuoriHHEN21}~\cite{AntuoriHHEN21}, \href{works/HubnerGSV21.pdf}{HubnerGSV21}~\cite{HubnerGSV21}, \href{works/AbreuAPNM21.pdf}{AbreuAPNM21}~\cite{AbreuAPNM21}, \href{works/KoehlerBFFHPSSS21.pdf}{KoehlerBFFHPSSS21}~\cite{KoehlerBFFHPSSS21}, \href{works/VlkHT21.pdf}{VlkHT21}~\cite{VlkHT21}, \href{works/BarzegaranZP20.pdf}{BarzegaranZP20}~\cite{BarzegaranZP20}, \href{works/GeibingerMM19.pdf}{GeibingerMM19}~\cite{GeibingerMM19}, \href{works/abs-1911-04766.pdf}{abs-1911-04766}~\cite{abs-1911-04766}, \href{works/BonfiettiZLM16.pdf}{BonfiettiZLM16}~\cite{BonfiettiZLM16}, \href{works/AlesioNBG14.pdf}{AlesioNBG14}~\cite{AlesioNBG14}, \href{works/BeniniBGM06.pdf}{BeniniBGM06}~\cite{BeniniBGM06}, \href{works/KovacsV06.pdf}{KovacsV06}~\cite{KovacsV06}, \href{works/Wallace96.pdf}{Wallace96}~\cite{Wallace96}\\
ApplicationAreas & cable tree & \href{works/KoehlerBFFHPSSS21.pdf}{KoehlerBFFHPSSS21}~\cite{KoehlerBFFHPSSS21} &  & \\
ApplicationAreas & car manufacturing &  & \href{works/AntuoriHHEN21.pdf}{AntuoriHHEN21}~\cite{AntuoriHHEN21} & \href{works/BeldiceanuC94.pdf}{BeldiceanuC94}~\cite{BeldiceanuC94}\\
ApplicationAreas & container terminal & \href{works/QinDCS20.pdf}{QinDCS20}~\cite{QinDCS20}, \href{works/SacramentoSP20.pdf}{SacramentoSP20}~\cite{SacramentoSP20} & \href{works/LaborieRSV18.pdf}{LaborieRSV18}~\cite{LaborieRSV18} & \href{works/abs-2312-13682.pdf}{abs-2312-13682}~\cite{abs-2312-13682}, \href{works/PerezGSL23.pdf}{PerezGSL23}~\cite{PerezGSL23}, \href{works/TouatBT22.pdf}{TouatBT22}~\cite{TouatBT22}, \href{works/WallaceY20.pdf}{WallaceY20}~\cite{WallaceY20}, \href{works/FallahiAC20.pdf}{FallahiAC20}~\cite{FallahiAC20}, \href{works/CauwelaertDMS16.pdf}{CauwelaertDMS16}~\cite{CauwelaertDMS16}, \href{works/DejemeppeCS15.pdf}{DejemeppeCS15}~\cite{DejemeppeCS15}, \href{works/NovasH12.pdf}{NovasH12}~\cite{NovasH12}, \href{works/LimRX04.pdf}{LimRX04}~\cite{LimRX04}\\
ApplicationAreas & crew-scheduling & \href{works/PourDERB18.pdf}{PourDERB18}~\cite{PourDERB18} & \href{works/BourreauGGLT22.pdf}{BourreauGGLT22}~\cite{BourreauGGLT22}, \href{works/Mason01.pdf}{Mason01}~\cite{Mason01}, \href{works/Touraivane95.pdf}{Touraivane95}~\cite{Touraivane95} & \href{works/NaderiRR23.pdf}{NaderiRR23}~\cite{NaderiRR23}, \href{works/WangB23.pdf}{WangB23}~\cite{WangB23}, \href{works/HeinzNVH22.pdf}{HeinzNVH22}~\cite{HeinzNVH22}, \href{works/MokhtarzadehTNF20.pdf}{MokhtarzadehTNF20}~\cite{MokhtarzadehTNF20}, \href{works/TangLWSK18.pdf}{TangLWSK18}~\cite{TangLWSK18}, \href{works/DoulabiRP16.pdf}{DoulabiRP16}~\cite{DoulabiRP16}, \href{works/HachemiGR11.pdf}{HachemiGR11}~\cite{HachemiGR11}, \href{works/BeldiceanuC02.pdf}{BeldiceanuC02}~\cite{BeldiceanuC02}\\
ApplicationAreas & dairies &  &  & \href{works/Bartak02.pdf}{Bartak02}~\cite{Bartak02}, \href{works/Bartak02a.pdf}{Bartak02a}~\cite{Bartak02a}\\
ApplicationAreas & dairy & \href{works/EscobetPQPRA19.pdf}{EscobetPQPRA19}~\cite{EscobetPQPRA19} & \href{works/PrataAN23.pdf}{PrataAN23}~\cite{PrataAN23} & \\
ApplicationAreas & datacenter & \href{works/HermenierDL11.pdf}{HermenierDL11}~\cite{HermenierDL11} &  & \href{works/GalleguillosKSB19.pdf}{GalleguillosKSB19}~\cite{GalleguillosKSB19}, \href{works/Madi-WambaLOBM17.pdf}{Madi-WambaLOBM17}~\cite{Madi-WambaLOBM17}, \href{works/IfrimOS12.pdf}{IfrimOS12}~\cite{IfrimOS12}, \href{works/LetortBC12.pdf}{LetortBC12}~\cite{LetortBC12}\\
ApplicationAreas & datacentre &  &  & \\
ApplicationAreas & day-ahead market &  &  & \\
ApplicationAreas & deep space &  &  & \\
ApplicationAreas & drone & \href{works/MontemanniD23a.pdf}{MontemanniD23a}~\cite{MontemanniD23a}, \href{works/MontemanniD23.pdf}{MontemanniD23}~\cite{MontemanniD23}, \href{works/Ham18.pdf}{Ham18}~\cite{Ham18} &  & \href{works/ShaikhK23.pdf}{ShaikhK23}~\cite{ShaikhK23}, \href{works/Astrand0F21.pdf}{Astrand0F21}~\cite{Astrand0F21}, \href{works/AntuoriHHEN21.pdf}{AntuoriHHEN21}~\cite{AntuoriHHEN21}\\
ApplicationAreas & earth observation & \href{works/SquillaciPR23.pdf}{SquillaciPR23}~\cite{SquillaciPR23}, \href{works/KucukY19.pdf}{KucukY19}~\cite{KucukY19}, \href{works/VerfaillieL01.pdf}{VerfaillieL01}~\cite{VerfaillieL01} & \href{works/BensanaLV99.pdf}{BensanaLV99}~\cite{BensanaLV99} & \href{works/PraletLJ15.pdf}{PraletLJ15}~\cite{PraletLJ15}, \href{works/SimoninAHL15.pdf}{SimoninAHL15}~\cite{SimoninAHL15}, \href{works/KelarevaTK13.pdf}{KelarevaTK13}~\cite{KelarevaTK13}, \href{works/OddiPCC03.pdf}{OddiPCC03}~\cite{OddiPCC03}\\
ApplicationAreas & earth orbit &  &  & \href{works/SquillaciPR23.pdf}{SquillaciPR23}~\cite{SquillaciPR23}\\
ApplicationAreas & electroplating &  & \href{works/RodosekW98.pdf}{RodosekW98}~\cite{RodosekW98} & \href{works/EfthymiouY23.pdf}{EfthymiouY23}~\cite{EfthymiouY23}, \href{works/WallaceY20.pdf}{WallaceY20}~\cite{WallaceY20}, \href{works/NovasH12.pdf}{NovasH12}~\cite{NovasH12}\\
ApplicationAreas & emergency service &  & \href{works/EvenSH15a.pdf}{EvenSH15a}~\cite{EvenSH15a}, \href{works/TopalogluO11.pdf}{TopalogluO11}~\cite{TopalogluO11} & \href{works/EvenSH15.pdf}{EvenSH15}~\cite{EvenSH15}, \href{works/SakkoutW00.pdf}{SakkoutW00}~\cite{SakkoutW00}\\
ApplicationAreas & energy-price & \href{works/GrimesIOS14.pdf}{GrimesIOS14}~\cite{GrimesIOS14}, \href{works/IfrimOS12.pdf}{IfrimOS12}~\cite{IfrimOS12} &  & \href{works/PrataAN23.pdf}{PrataAN23}~\cite{PrataAN23}, \href{works/EscobetPQPRA19.pdf}{EscobetPQPRA19}~\cite{EscobetPQPRA19}, \href{works/BenediktSMVH18.pdf}{BenediktSMVH18}~\cite{BenediktSMVH18}, \href{works/He0GLW18.pdf}{He0GLW18}~\cite{He0GLW18}, \href{works/LimHTB16.pdf}{LimHTB16}~\cite{LimHTB16}\\
ApplicationAreas & farming &  &  & \href{works/WinterMMW22.pdf}{WinterMMW22}~\cite{WinterMMW22}, \href{works/Astrand0F21.pdf}{Astrand0F21}~\cite{Astrand0F21}\\
ApplicationAreas & forestry & \href{works/HachemiGR11.pdf}{HachemiGR11}~\cite{HachemiGR11} &  & \href{works/Astrand0F21.pdf}{Astrand0F21}~\cite{Astrand0F21}\\
ApplicationAreas & hoist & \href{works/EfthymiouY23.pdf}{EfthymiouY23}~\cite{EfthymiouY23}, \href{works/WallaceY20.pdf}{WallaceY20}~\cite{WallaceY20}, \href{works/RodosekW98.pdf}{RodosekW98}~\cite{RodosekW98} & \href{works/NovasH12.pdf}{NovasH12}~\cite{NovasH12}, \href{works/BonfiettiLBM11.pdf}{BonfiettiLBM11}~\cite{BonfiettiLBM11} & \href{works/AstrandJZ18.pdf}{AstrandJZ18}~\cite{AstrandJZ18}, \href{works/BonfiettiLBM14.pdf}{BonfiettiLBM14}~\cite{BonfiettiLBM14}, \href{works/BonfiettiM12.pdf}{BonfiettiM12}~\cite{BonfiettiM12}, \href{works/BonfiettiLBM12.pdf}{BonfiettiLBM12}~\cite{BonfiettiLBM12}, \href{works/LombardiBMB11.pdf}{LombardiBMB11}~\cite{LombardiBMB11}, \href{works/KorbaaYG99.pdf}{KorbaaYG99}~\cite{KorbaaYG99}, \href{works/PapaB98.pdf}{PapaB98}~\cite{PapaB98}\\
ApplicationAreas & medical & \href{works/ShinBBHO18.pdf}{ShinBBHO18}~\cite{ShinBBHO18}, \href{works/WangMD15.pdf}{WangMD15}~\cite{WangMD15}, \href{works/TopalogluO11.pdf}{TopalogluO11}~\cite{TopalogluO11} & \href{works/HechingH16.pdf}{HechingH16}~\cite{HechingH16}, \href{works/DejemeppeD14.pdf}{DejemeppeD14}~\cite{DejemeppeD14}, \href{works/RendlPHPR12.pdf}{RendlPHPR12}~\cite{RendlPHPR12} & \href{works/ShaikhK23.pdf}{ShaikhK23}~\cite{ShaikhK23}, \href{works/AbreuNP23.pdf}{AbreuNP23}~\cite{AbreuNP23}, \href{works/AkramNHRSA23.pdf}{AkramNHRSA23}~\cite{AkramNHRSA23}, \href{works/IsikYA23.pdf}{IsikYA23}~\cite{IsikYA23}, \href{works/YunusogluY22.pdf}{YunusogluY22}~\cite{YunusogluY22}, \href{works/AbreuN22.pdf}{AbreuN22}~\cite{AbreuN22}, \href{works/GeibingerKKMMW21.pdf}{GeibingerKKMMW21}~\cite{GeibingerKKMMW21}, \href{works/AbreuAPNM21.pdf}{AbreuAPNM21}~\cite{AbreuAPNM21}, \href{works/Bedhief21.pdf}{Bedhief21}~\cite{Bedhief21}, \href{works/FallahiAC20.pdf}{FallahiAC20}~\cite{FallahiAC20}, \href{works/abs-1902-01193.pdf}{abs-1902-01193}~\cite{abs-1902-01193}, \href{works/FrimodigS19.pdf}{FrimodigS19}~\cite{FrimodigS19}, \href{works/Novas19.pdf}{Novas19}~\cite{Novas19}, \href{works/GurEA19.pdf}{GurEA19}~\cite{GurEA19}, \href{works/YounespourAKE19.pdf}{YounespourAKE19}~\cite{YounespourAKE19}, \href{works/HoYCLLCLC18.pdf}{HoYCLLCLC18}~\cite{HoYCLLCLC18}, \href{works/GedikKEK18.pdf}{GedikKEK18}~\cite{GedikKEK18}, \href{works/DoulabiRP16.pdf}{DoulabiRP16}~\cite{DoulabiRP16}, \href{works/BridiBLMB16.pdf}{BridiBLMB16}~\cite{BridiBLMB16}, \href{works/BoothNB16.pdf}{BoothNB16}~\cite{BoothNB16}, \href{works/BonfiettiLBM14.pdf}{BonfiettiLBM14}~\cite{BonfiettiLBM14}, \href{works/DoulabiRP14.pdf}{DoulabiRP14}~\cite{DoulabiRP14}, \href{works/Simonis07.pdf}{Simonis07}~\cite{Simonis07}\\
ApplicationAreas & nurse & \href{works/GurPAE23.pdf}{GurPAE23}~\cite{GurPAE23}, \href{works/abs-1902-01193.pdf}{abs-1902-01193}~\cite{abs-1902-01193}, \href{works/HoYCLLCLC18.pdf}{HoYCLLCLC18}~\cite{HoYCLLCLC18}, \href{works/ShinBBHO18.pdf}{ShinBBHO18}~\cite{ShinBBHO18}, \href{works/WangMD15.pdf}{WangMD15}~\cite{WangMD15}, \href{works/RendlPHPR12.pdf}{RendlPHPR12}~\cite{RendlPHPR12}, \href{works/Simonis07.pdf}{Simonis07}~\cite{Simonis07}, \href{works/Mason01.pdf}{Mason01}~\cite{Mason01} & \href{works/OuelletQ22.pdf}{OuelletQ22}~\cite{OuelletQ22}, \href{works/GeibingerKKMMW21.pdf}{GeibingerKKMMW21}~\cite{GeibingerKKMMW21}, \href{works/GeibingerMM21.pdf}{GeibingerMM21}~\cite{GeibingerMM21}, \href{works/YounespourAKE19.pdf}{YounespourAKE19}~\cite{YounespourAKE19}, \href{works/FrohnerTR19.pdf}{FrohnerTR19}~\cite{FrohnerTR19} & \href{works/PerezGSL23.pdf}{PerezGSL23}~\cite{PerezGSL23}, \href{works/abs-2312-13682.pdf}{abs-2312-13682}~\cite{abs-2312-13682}, \href{works/BourreauGGLT22.pdf}{BourreauGGLT22}~\cite{BourreauGGLT22}, \href{works/FallahiAC20.pdf}{FallahiAC20}~\cite{FallahiAC20}, \href{works/FrimodigS19.pdf}{FrimodigS19}~\cite{FrimodigS19}, \href{works/GedikKEK18.pdf}{GedikKEK18}~\cite{GedikKEK18}, \href{works/NishikawaSTT18a.pdf}{NishikawaSTT18a}~\cite{NishikawaSTT18a}, \href{works/DoulabiRP16.pdf}{DoulabiRP16}~\cite{DoulabiRP16}, \href{works/DoulabiRP14.pdf}{DoulabiRP14}~\cite{DoulabiRP14}, \href{works/TopalogluO11.pdf}{TopalogluO11}~\cite{TopalogluO11}\\
ApplicationAreas & offshore &  & \href{works/SubulanC22.pdf}{SubulanC22}~\cite{SubulanC22} & \href{works/BoudreaultSLQ22.pdf}{BoudreaultSLQ22}~\cite{BoudreaultSLQ22}\\
ApplicationAreas & oven scheduling & \href{works/LacknerMMWW23.pdf}{LacknerMMWW23}~\cite{LacknerMMWW23}, \href{works/LacknerMMWW21.pdf}{LacknerMMWW21}~\cite{LacknerMMWW21} &  & \href{works/ColT22.pdf}{ColT22}~\cite{ColT22}\\
ApplicationAreas & patient & \href{works/GurPAE23.pdf}{GurPAE23}~\cite{GurPAE23}, \href{works/GurEA19.pdf}{GurEA19}~\cite{GurEA19}, \href{works/FrimodigS19.pdf}{FrimodigS19}~\cite{FrimodigS19}, \href{works/YounespourAKE19.pdf}{YounespourAKE19}~\cite{YounespourAKE19}, \href{works/ShinBBHO18.pdf}{ShinBBHO18}~\cite{ShinBBHO18}, \href{works/HechingH16.pdf}{HechingH16}~\cite{HechingH16}, \href{works/DoulabiRP16.pdf}{DoulabiRP16}~\cite{DoulabiRP16}, \href{works/WangMD15.pdf}{WangMD15}~\cite{WangMD15}, \href{works/DejemeppeD14.pdf}{DejemeppeD14}~\cite{DejemeppeD14}, \href{works/RendlPHPR12.pdf}{RendlPHPR12}~\cite{RendlPHPR12}, \href{works/TopalogluO11.pdf}{TopalogluO11}~\cite{TopalogluO11} & \href{works/GeibingerKKMMW21.pdf}{GeibingerKKMMW21}~\cite{GeibingerKKMMW21} & \href{works/AlfieriGPS23.pdf}{AlfieriGPS23}~\cite{AlfieriGPS23}, \href{works/AbreuAPNM21.pdf}{AbreuAPNM21}~\cite{AbreuAPNM21}, \href{works/MurinR19.pdf}{MurinR19}~\cite{MurinR19}, \href{works/HoYCLLCLC18.pdf}{HoYCLLCLC18}~\cite{HoYCLLCLC18}, \href{works/DoulabiRP14.pdf}{DoulabiRP14}~\cite{DoulabiRP14}, \href{works/Simonis07.pdf}{Simonis07}~\cite{Simonis07}\\
ApplicationAreas & perfect-square & \href{works/BeldiceanuCDP11.pdf}{BeldiceanuCDP11}~\cite{BeldiceanuCDP11}, \href{works/BeldiceanuCP08.pdf}{BeldiceanuCP08}~\cite{BeldiceanuCP08}, \href{works/AggounB93.pdf}{AggounB93}~\cite{AggounB93} &  & \\
ApplicationAreas & physician & \href{works/GeibingerKKMMW21.pdf}{GeibingerKKMMW21}~\cite{GeibingerKKMMW21}, \href{works/ShinBBHO18.pdf}{ShinBBHO18}~\cite{ShinBBHO18} &  & \href{works/GurPAE23.pdf}{GurPAE23}~\cite{GurPAE23}, \href{works/FrimodigS19.pdf}{FrimodigS19}~\cite{FrimodigS19}, \href{works/WangMD15.pdf}{WangMD15}~\cite{WangMD15}, \href{works/TopalogluO11.pdf}{TopalogluO11}~\cite{TopalogluO11}\\
ApplicationAreas & pipeline & \href{works/BegB13.pdf}{BegB13}~\cite{BegB13}, \href{works/LopesCSM10.pdf}{LopesCSM10}~\cite{LopesCSM10}, \href{works/RuggieroBBMA09.pdf}{RuggieroBBMA09}~\cite{RuggieroBBMA09}, \href{works/MouraSCL08.pdf}{MouraSCL08}~\cite{MouraSCL08}, \href{works/MouraSCL08a.pdf}{MouraSCL08a}~\cite{MouraSCL08a}, \href{works/ErtlK91.pdf}{ErtlK91}~\cite{ErtlK91} & \href{works/ZouZ20.pdf}{ZouZ20}~\cite{ZouZ20}, \href{works/TangLWSK18.pdf}{TangLWSK18}~\cite{TangLWSK18}, \href{works/MalikMB08.pdf}{MalikMB08}~\cite{MalikMB08}, \href{works/BeniniBGM06.pdf}{BeniniBGM06}~\cite{BeniniBGM06}, \href{works/WolinskiKG04.pdf}{WolinskiKG04}~\cite{WolinskiKG04}, \href{works/BeldiceanuC94.pdf}{BeldiceanuC94}~\cite{BeldiceanuC94} & \href{works/EfthymiouY23.pdf}{EfthymiouY23}~\cite{EfthymiouY23}, \href{works/PopovicCGNC22.pdf}{PopovicCGNC22}~\cite{PopovicCGNC22}, \href{works/HanenKP21.pdf}{HanenKP21}~\cite{HanenKP21}, \href{works/NishikawaSTT19.pdf}{NishikawaSTT19}~\cite{NishikawaSTT19}, \href{works/NishikawaSTT18.pdf}{NishikawaSTT18}~\cite{NishikawaSTT18}, \href{works/NishikawaSTT18a.pdf}{NishikawaSTT18a}~\cite{NishikawaSTT18a}, \href{works/LaborieRSV18.pdf}{LaborieRSV18}~\cite{LaborieRSV18}, \href{works/Bonfietti16.pdf}{Bonfietti16}~\cite{Bonfietti16}, \href{works/GilesH16.pdf}{GilesH16}~\cite{GilesH16}, \href{works/GoelSHFS15.pdf}{GoelSHFS15}~\cite{GoelSHFS15}, \href{works/SimoninAHL15.pdf}{SimoninAHL15}~\cite{SimoninAHL15}, \href{works/BonfiettiLBM14.pdf}{BonfiettiLBM14}~\cite{BonfiettiLBM14}, \href{works/BeniniLMR11.pdf}{BeniniLMR11}~\cite{BeniniLMR11}, \href{works/NovasH10.pdf}{NovasH10}~\cite{NovasH10}, \href{works/BarlattCG08.pdf}{BarlattCG08}~\cite{BarlattCG08}, \href{works/KuchcinskiW03.pdf}{KuchcinskiW03}~\cite{KuchcinskiW03}, \href{works/Wolf03.pdf}{Wolf03}~\cite{Wolf03}, \href{works/GruianK98.pdf}{GruianK98}~\cite{GruianK98}, \href{works/Darby-DowmanLMZ97.pdf}{Darby-DowmanLMZ97}~\cite{Darby-DowmanLMZ97}, \href{works/SimonisC95.pdf}{SimonisC95}~\cite{SimonisC95}\\
ApplicationAreas & radiation therapy & \href{works/FrimodigS19.pdf}{FrimodigS19}~\cite{FrimodigS19} &  & \\
ApplicationAreas & railway & \href{works/SvancaraB22.pdf}{SvancaraB22}~\cite{SvancaraB22}, \href{works/PourDERB18.pdf}{PourDERB18}~\cite{PourDERB18}, \href{works/CappartS17.pdf}{CappartS17}~\cite{CappartS17}, \href{works/Acuna-AgostMFG09.pdf}{Acuna-AgostMFG09}~\cite{Acuna-AgostMFG09}, \href{works/AronssonBK09.pdf}{AronssonBK09}~\cite{AronssonBK09}, \href{works/Rodriguez07.pdf}{Rodriguez07}~\cite{Rodriguez07}, \href{works/Geske05.pdf}{Geske05}~\cite{Geske05}, \href{works/RodriguezDG02.pdf}{RodriguezDG02}~\cite{RodriguezDG02}, \href{works/MartinPY01.pdf}{MartinPY01}~\cite{MartinPY01}, \href{works/LammaMM97.pdf}{LammaMM97}~\cite{LammaMM97} & \href{works/LaborieRSV18.pdf}{LaborieRSV18}~\cite{LaborieRSV18}, \href{works/TangLWSK18.pdf}{TangLWSK18}~\cite{TangLWSK18}, \href{works/Mason01.pdf}{Mason01}~\cite{Mason01}, \href{works/BrusoniCLMMT96.pdf}{BrusoniCLMMT96}~\cite{BrusoniCLMMT96} & \href{works/LuoB22.pdf}{LuoB22}~\cite{LuoB22}, \href{works/BogaerdtW19.pdf}{BogaerdtW19}~\cite{BogaerdtW19}, \href{works/ZhouGL15.pdf}{ZhouGL15}~\cite{ZhouGL15}, \href{works/AbrilSB05.pdf}{AbrilSB05}~\cite{AbrilSB05}, \href{works/Wallace96.pdf}{Wallace96}~\cite{Wallace96}\\
ApplicationAreas & real-time pricing &  & \href{works/He0GLW18.pdf}{He0GLW18}~\cite{He0GLW18}, \href{works/GrimesIOS14.pdf}{GrimesIOS14}~\cite{GrimesIOS14} & \href{works/LimHTB16.pdf}{LimHTB16}~\cite{LimHTB16}\\
ApplicationAreas & rectangle-packing & \href{works/YangSS19.pdf}{YangSS19}~\cite{YangSS19}, \href{works/AggounB93.pdf}{AggounB93}~\cite{AggounB93} & \href{works/LuoB22.pdf}{LuoB22}~\cite{LuoB22} & \href{works/MossigeGSMC17.pdf}{MossigeGSMC17}~\cite{MossigeGSMC17}, \href{works/DoulabiRP16.pdf}{DoulabiRP16}~\cite{DoulabiRP16}, \href{works/VilimLS15.pdf}{VilimLS15}~\cite{VilimLS15}, \href{works/BeldiceanuCDP11.pdf}{BeldiceanuCDP11}~\cite{BeldiceanuCDP11}, \href{works/SchuttW10.pdf}{SchuttW10}~\cite{SchuttW10}, \href{works/BeldiceanuCP08.pdf}{BeldiceanuCP08}~\cite{BeldiceanuCP08}\\
ApplicationAreas & robot & \href{works/IsikYA23.pdf}{IsikYA23}~\cite{IsikYA23}, \href{works/LiFJZLL22.pdf}{LiFJZLL22}~\cite{LiFJZLL22}, \href{works/ArmstrongGOS21.pdf}{ArmstrongGOS21}~\cite{ArmstrongGOS21}, \href{works/KoehlerBFFHPSSS21.pdf}{KoehlerBFFHPSSS21}~\cite{KoehlerBFFHPSSS21}, \href{works/WessenCS20.pdf}{WessenCS20}~\cite{WessenCS20}, \href{works/MokhtarzadehTNF20.pdf}{MokhtarzadehTNF20}~\cite{MokhtarzadehTNF20}, \href{works/MurinR19.pdf}{MurinR19}~\cite{MurinR19}, \href{works/abs-1901-07914.pdf}{abs-1901-07914}~\cite{abs-1901-07914}, \href{works/BehrensLM19.pdf}{BehrensLM19}~\cite{BehrensLM19}, \href{works/LaborieRSV18.pdf}{LaborieRSV18}~\cite{LaborieRSV18}, \href{works/MossigeGSMC17.pdf}{MossigeGSMC17}~\cite{MossigeGSMC17}, \href{works/BoothNB16.pdf}{BoothNB16}~\cite{BoothNB16}, \href{works/NovasH14.pdf}{NovasH14}~\cite{NovasH14}, \href{works/NovasH12.pdf}{NovasH12}~\cite{NovasH12}, \href{works/BartakSR10.pdf}{BartakSR10}~\cite{BartakSR10}, \href{works/ValleMGT03.pdf}{ValleMGT03}~\cite{ValleMGT03} & \href{works/PrataAN23.pdf}{PrataAN23}~\cite{PrataAN23}, \href{works/Mehdizadeh-Somarin23.pdf}{Mehdizadeh-Somarin23}~\cite{Mehdizadeh-Somarin23}, \href{works/CzerniachowskaWZ23.pdf}{CzerniachowskaWZ23}~\cite{CzerniachowskaWZ23}, \href{works/TouatBT22.pdf}{TouatBT22}~\cite{TouatBT22}, \href{works/YunusogluY22.pdf}{YunusogluY22}~\cite{YunusogluY22}, \href{works/OujanaAYB22.pdf}{OujanaAYB22}~\cite{OujanaAYB22}, \href{works/Astrand0F21.pdf}{Astrand0F21}~\cite{Astrand0F21}, \href{works/WallaceY20.pdf}{WallaceY20}~\cite{WallaceY20}, \href{works/WikarekS19.pdf}{WikarekS19}~\cite{WikarekS19}, \href{works/NishikawaSTT19.pdf}{NishikawaSTT19}~\cite{NishikawaSTT19}, \href{works/NishikawaSTT18a.pdf}{NishikawaSTT18a}~\cite{NishikawaSTT18a}, \href{works/NishikawaSTT18.pdf}{NishikawaSTT18}~\cite{NishikawaSTT18}, \href{works/VanczaM01.pdf}{VanczaM01}~\cite{VanczaM01} & \href{works/abs-2305-19888.pdf}{abs-2305-19888}~\cite{abs-2305-19888}, \href{works/MontemanniD23.pdf}{MontemanniD23}~\cite{MontemanniD23}, \href{works/HeinzNVH22.pdf}{HeinzNVH22}~\cite{HeinzNVH22}, \href{works/GeitzGSSW22.pdf}{GeitzGSSW22}~\cite{GeitzGSSW22}, \href{works/MullerMKP22.pdf}{MullerMKP22}~\cite{MullerMKP22}, \href{works/ColT22.pdf}{ColT22}~\cite{ColT22}, \href{works/YuraszeckMPV22.pdf}{YuraszeckMPV22}~\cite{YuraszeckMPV22}, \href{works/HamPK21.pdf}{HamPK21}~\cite{HamPK21}, \href{works/ZhangYW21.pdf}{ZhangYW21}~\cite{ZhangYW21}, \href{works/VlkHT21.pdf}{VlkHT21}~\cite{VlkHT21}, \href{works/Bedhief21.pdf}{Bedhief21}~\cite{Bedhief21}, \href{works/FallahiAC20.pdf}{FallahiAC20}~\cite{FallahiAC20}, \href{works/MengZRZL20.pdf}{MengZRZL20}~\cite{MengZRZL20}, \href{works/BenediktMH20.pdf}{BenediktMH20}~\cite{BenediktMH20}, \href{works/MejiaY20.pdf}{MejiaY20}~\cite{MejiaY20}, \href{works/AstrandJZ20.pdf}{AstrandJZ20}~\cite{AstrandJZ20}, \href{works/BarzegaranZP20.pdf}{BarzegaranZP20}~\cite{BarzegaranZP20}, \href{works/Novas19.pdf}{Novas19}~\cite{Novas19}, \href{works/GokgurHO18.pdf}{GokgurHO18}~\cite{GokgurHO18}, \href{works/Ham18.pdf}{Ham18}~\cite{Ham18}, \href{works/ZhangW18.pdf}{ZhangW18}~\cite{ZhangW18}, \href{works/AstrandJZ18.pdf}{AstrandJZ18}~\cite{AstrandJZ18}, \href{works/ZarandiKS16.pdf}{ZarandiKS16}~\cite{ZarandiKS16}, \href{works/DoulabiRP16.pdf}{DoulabiRP16}~\cite{DoulabiRP16}, \href{works/SimoninAHL15.pdf}{SimoninAHL15}~\cite{SimoninAHL15}, \href{works/BonfiettiLBM14.pdf}{BonfiettiLBM14}~\cite{BonfiettiLBM14}, \href{works/LimtanyakulS12.pdf}{LimtanyakulS12}~\cite{LimtanyakulS12}, \href{works/BonfiettiLBM12.pdf}{BonfiettiLBM12}~\cite{BonfiettiLBM12}, \href{works/BonfiettiLBM11.pdf}{BonfiettiLBM11}~\cite{BonfiettiLBM11}... (Total: 43)\\
ApplicationAreas & satellite & \href{works/SquillaciPR23.pdf}{SquillaciPR23}~\cite{SquillaciPR23}, \href{works/GodetLHS20.pdf}{GodetLHS20}~\cite{GodetLHS20}, \href{works/KucukY19.pdf}{KucukY19}~\cite{KucukY19}, \href{works/LaborieRSV18.pdf}{LaborieRSV18}~\cite{LaborieRSV18}, \href{works/PraletLJ15.pdf}{PraletLJ15}~\cite{PraletLJ15}, \href{works/KelarevaTK13.pdf}{KelarevaTK13}~\cite{KelarevaTK13}, \href{works/VerfaillieL01.pdf}{VerfaillieL01}~\cite{VerfaillieL01}, \href{works/BensanaLV99.pdf}{BensanaLV99}~\cite{BensanaLV99}, \href{works/PembertonG98.pdf}{PembertonG98}~\cite{PembertonG98} & \href{works/Laborie09.pdf}{Laborie09}~\cite{Laborie09}, \href{works/FrankK05.pdf}{FrankK05}~\cite{FrankK05} & \href{works/EfthymiouY23.pdf}{EfthymiouY23}~\cite{EfthymiouY23}, \href{works/TouatBT22.pdf}{TouatBT22}~\cite{TouatBT22}, \href{works/Astrand0F21.pdf}{Astrand0F21}~\cite{Astrand0F21}, \href{works/Pralet17.pdf}{Pralet17}~\cite{Pralet17}, \href{works/SimoninAHL15.pdf}{SimoninAHL15}~\cite{SimoninAHL15}, \href{works/BessiereHMQW14.pdf}{BessiereHMQW14}~\cite{BessiereHMQW14}, \href{works/HeinzSB13.pdf}{HeinzSB13}~\cite{HeinzSB13}, \href{works/SimoninAHL12.pdf}{SimoninAHL12}~\cite{SimoninAHL12}, \href{works/RuggieroBBMA09.pdf}{RuggieroBBMA09}~\cite{RuggieroBBMA09}, \href{works/Rodriguez07.pdf}{Rodriguez07}~\cite{Rodriguez07}, \href{works/OddiPCC03.pdf}{OddiPCC03}~\cite{OddiPCC03}, \href{works/NuijtenP98.pdf}{NuijtenP98}~\cite{NuijtenP98}\\
ApplicationAreas & semiconductor & \href{works/MalapertN19.pdf}{MalapertN19}~\cite{MalapertN19}, \href{works/NovasH12.pdf}{NovasH12}~\cite{NovasH12} & \href{works/QinWSLS21.pdf}{QinWSLS21}~\cite{QinWSLS21}, \href{works/GokgurHO18.pdf}{GokgurHO18}~\cite{GokgurHO18}, \href{works/Davenport10.pdf}{Davenport10}~\cite{Davenport10}, \href{works/KrogtLPHJ07.pdf}{KrogtLPHJ07}~\cite{KrogtLPHJ07} & \href{works/LacknerMMWW23.pdf}{LacknerMMWW23}~\cite{LacknerMMWW23}, \href{works/YuraszeckMPV22.pdf}{YuraszeckMPV22}~\cite{YuraszeckMPV22}, \href{works/abs-2211-14492.pdf}{abs-2211-14492}~\cite{abs-2211-14492}, \href{works/MullerMKP22.pdf}{MullerMKP22}~\cite{MullerMKP22}, \href{works/ColT22.pdf}{ColT22}~\cite{ColT22}, \href{works/ZhangJZL22.pdf}{ZhangJZL22}~\cite{ZhangJZL22}, \href{works/FanXG21.pdf}{FanXG21}~\cite{FanXG21}, \href{works/LacknerMMWW21.pdf}{LacknerMMWW21}~\cite{LacknerMMWW21}, \href{works/HamPK21.pdf}{HamPK21}~\cite{HamPK21}, \href{works/PandeyS21a.pdf}{PandeyS21a}~\cite{PandeyS21a}, \href{works/MengZRZL20.pdf}{MengZRZL20}~\cite{MengZRZL20}, \href{works/NattafM20.pdf}{NattafM20}~\cite{NattafM20}, \href{works/TangB20.pdf}{TangB20}~\cite{TangB20}, \href{works/Novas19.pdf}{Novas19}~\cite{Novas19}, \href{works/LaborieRSV18.pdf}{LaborieRSV18}~\cite{LaborieRSV18}, \href{works/Ham18.pdf}{Ham18}~\cite{Ham18}, \href{works/KoschB14.pdf}{KoschB14}~\cite{KoschB14}\\
ApplicationAreas & ship building &  &  & \\
ApplicationAreas & shipping line &  &  & \href{works/QinDCS20.pdf}{QinDCS20}~\cite{QinDCS20}, \href{works/LaborieRSV18.pdf}{LaborieRSV18}~\cite{LaborieRSV18}, \href{works/KelarevaTK13.pdf}{KelarevaTK13}~\cite{KelarevaTK13}\\
ApplicationAreas & steel cable &  &  & \href{works/AalianPG23.pdf}{AalianPG23}~\cite{AalianPG23}\\
ApplicationAreas & steel mill & \href{works/GaySS14.pdf}{GaySS14}~\cite{GaySS14}, \href{works/HeinzSSW12.pdf}{HeinzSSW12}~\cite{HeinzSSW12}, \href{works/SchausHMCMD11.pdf}{SchausHMCMD11}~\cite{SchausHMCMD11}, \href{works/HentenryckM08.pdf}{HentenryckM08}~\cite{HentenryckM08}, \href{works/GarganiR07.pdf}{GarganiR07}~\cite{GarganiR07} &  & \href{works/abs-2312-13682.pdf}{abs-2312-13682}~\cite{abs-2312-13682}, \href{works/PerezGSL23.pdf}{PerezGSL23}~\cite{PerezGSL23}, \href{works/DoulabiRP16.pdf}{DoulabiRP16}~\cite{DoulabiRP16}\\
ApplicationAreas & super-computer & \href{works/BorghesiBLMB18.pdf}{BorghesiBLMB18}~\cite{BorghesiBLMB18}, \href{works/BridiBLMB16.pdf}{BridiBLMB16}~\cite{BridiBLMB16}, \href{works/BartoliniBBLM14.pdf}{BartoliniBBLM14}~\cite{BartoliniBBLM14} &  & \href{works/LuoB22.pdf}{LuoB22}~\cite{LuoB22}, \href{works/GalleguillosKSB19.pdf}{GalleguillosKSB19}~\cite{GalleguillosKSB19}\\
ApplicationAreas & torpedo & \href{works/KletzanderM17.pdf}{KletzanderM17}~\cite{KletzanderM17}, \href{works/GoldwaserS17.pdf}{GoldwaserS17}~\cite{GoldwaserS17} &  & \\
ApplicationAreas & vaccine &  &  & \\
ApplicationAreas & yard crane &  & \href{works/QinDCS20.pdf}{QinDCS20}~\cite{QinDCS20} & \href{works/WallaceY20.pdf}{WallaceY20}~\cite{WallaceY20}\\
\end{longtable}
}


\clearpage
\subsection{Concept Type Industries}
\label{sec:Industries}
{\scriptsize
\begin{longtable}{lp{3cm}>{\raggedright\arraybackslash}p{6cm}>{\raggedright\arraybackslash}p{6cm}>{\raggedright\arraybackslash}p{8cm}}
\rowcolor{white}\caption{Works for Concepts of Type Industries}\\ \toprule
\rowcolor{white}Type & Keyword & High & Medium & Low\\ \midrule\endhead
\bottomrule
\endfoot
Industries & aerospace industry &  &  & \href{works/SchildW00.pdf}{SchildW00}~\cite{SchildW00}\\
Industries & agricultural industry & \href{works/WinterMMW22.pdf}{WinterMMW22}~\cite{WinterMMW22} &  & \\
Industries & automotive industry &  & \href{works/LimtanyakulS12.pdf}{LimtanyakulS12}~\cite{LimtanyakulS12} & \href{works/CzerniachowskaWZ23.pdf}{CzerniachowskaWZ23}~\cite{CzerniachowskaWZ23}, \href{works/EmdeZD22.pdf}{EmdeZD22}~\cite{EmdeZD22}, \href{works/AntuoriHHEN21.pdf}{AntuoriHHEN21}~\cite{AntuoriHHEN21}, \href{works/BonfiettiZLM16.pdf}{BonfiettiZLM16}~\cite{BonfiettiZLM16}, \href{works/SchildW00.pdf}{SchildW00}~\cite{SchildW00}, \href{works/Wallace96.pdf}{Wallace96}~\cite{Wallace96}\\
Industries & chemical industry &  & \href{works/Timpe02.pdf}{Timpe02}~\cite{Timpe02} & \href{works/LaborieRSV18.pdf}{LaborieRSV18}~\cite{LaborieRSV18}, \href{works/GilesH16.pdf}{GilesH16}~\cite{GilesH16}, \href{works/LombardiM12.pdf}{LombardiM12}~\cite{LombardiM12}, \href{works/ChenGPSH10.pdf}{ChenGPSH10}~\cite{ChenGPSH10}, \href{works/PoderBS04.pdf}{PoderBS04}~\cite{PoderBS04}, \href{works/Simonis99.pdf}{Simonis99}~\cite{Simonis99}, \href{works/Simonis95a.pdf}{Simonis95a}~\cite{Simonis95a}\\
Industries & chemical processing industry &  &  & \href{works/GilesH16.pdf}{GilesH16}~\cite{GilesH16}\\
Industries & control system industry &  &  & \href{works/BonfiettiZLM16.pdf}{BonfiettiZLM16}~\cite{BonfiettiZLM16}\\
Industries & electricity industry & \href{works/Froger16.pdf}{Froger16}~\cite{Froger16} &  & \href{works/PopovicCGNC22.pdf}{PopovicCGNC22}~\cite{PopovicCGNC22}, \href{works/Godet21a.pdf}{Godet21a}~\cite{Godet21a}, \href{works/AntunesABDEGGOL20.pdf}{AntunesABDEGGOL20}~\cite{AntunesABDEGGOL20}, \href{works/AntunesABDEGGOL18.pdf}{AntunesABDEGGOL18}~\cite{AntunesABDEGGOL18}\\
Industries & electronics industry &  &  & \href{works/LacknerMMWW23.pdf}{LacknerMMWW23}~\cite{LacknerMMWW23}, \href{works/LacknerMMWW21.pdf}{LacknerMMWW21}~\cite{LacknerMMWW21}\\
Industries & food industry &  & \href{works/Groleaz21.pdf}{Groleaz21}~\cite{Groleaz21} & \href{works/OujanaAYB22.pdf}{OujanaAYB22}~\cite{OujanaAYB22}, \href{works/GroleazNS20a.pdf}{GroleazNS20a}~\cite{GroleazNS20a}, \href{works/GroleazNS20.pdf}{GroleazNS20}~\cite{GroleazNS20}, \href{works/EscobetPQPRA19.pdf}{EscobetPQPRA19}~\cite{EscobetPQPRA19}, \href{works/HachemiGR11.pdf}{HachemiGR11}~\cite{HachemiGR11}, \href{works/SimonisCK00.pdf}{SimonisCK00}~\cite{SimonisCK00}, \href{works/Simonis99.pdf}{Simonis99}~\cite{Simonis99}, \href{works/SimonisC95.pdf}{SimonisC95}~\cite{SimonisC95}, \href{works/Simonis95.pdf}{Simonis95}~\cite{Simonis95}\\
Industries & food-processing industry &  &  & \href{works/KlankeBYE21.pdf}{KlankeBYE21}~\cite{KlankeBYE21}, \href{works/abs-1902-09244.pdf}{abs-1902-09244}~\cite{abs-1902-09244}\\
Industries & manufacturing industry &  &  & \href{works/PrataAN23.pdf}{PrataAN23}~\cite{PrataAN23}, \href{works/CzerniachowskaWZ23.pdf}{CzerniachowskaWZ23}~\cite{CzerniachowskaWZ23}, \href{works/LacknerMMWW23.pdf}{LacknerMMWW23}~\cite{LacknerMMWW23}, \href{works/WinterMMW22.pdf}{WinterMMW22}~\cite{WinterMMW22}, \href{works/YuraszeckMPV22.pdf}{YuraszeckMPV22}~\cite{YuraszeckMPV22}, \href{works/FanXG21.pdf}{FanXG21}~\cite{FanXG21}, \href{works/LacknerMMWW21.pdf}{LacknerMMWW21}~\cite{LacknerMMWW21}, \href{works/Mercier-AubinGQ20.pdf}{Mercier-AubinGQ20}~\cite{Mercier-AubinGQ20}, \href{works/TangB20.pdf}{TangB20}~\cite{TangB20}, \href{works/EscobetPQPRA19.pdf}{EscobetPQPRA19}~\cite{EscobetPQPRA19}, \href{works/GedikKEK18.pdf}{GedikKEK18}~\cite{GedikKEK18}\\
Industries & mineral industry &  &  & \href{works/Astrand21.pdf}{Astrand21}~\cite{Astrand21}, \href{works/Astrand0F21.pdf}{Astrand0F21}~\cite{Astrand0F21}, \href{works/AstrandJZ20.pdf}{AstrandJZ20}~\cite{AstrandJZ20}\\
Industries & mining industry &  & \href{works/AalianPG23.pdf}{AalianPG23}~\cite{AalianPG23} & \href{works/abs-2402-00459.pdf}{abs-2402-00459}~\cite{abs-2402-00459}, \href{works/CampeauG22.pdf}{CampeauG22}~\cite{CampeauG22}, \href{works/Astrand0F21.pdf}{Astrand0F21}~\cite{Astrand0F21}, \href{works/Astrand21.pdf}{Astrand21}~\cite{Astrand21}, \href{works/AstrandJZ20.pdf}{AstrandJZ20}~\cite{AstrandJZ20}, \href{works/ThiruvadyWGS14.pdf}{ThiruvadyWGS14}~\cite{ThiruvadyWGS14}\\
Industries & oil industry &  &  & \href{works/AbreuNP23.pdf}{AbreuNP23}~\cite{AbreuNP23}, \href{works/AbreuAPNM21.pdf}{AbreuAPNM21}~\cite{AbreuAPNM21}, \href{works/LopesCSM10.pdf}{LopesCSM10}~\cite{LopesCSM10}\\
Industries & packaging industry &  &  & \href{works/ArmstrongGOS21.pdf}{ArmstrongGOS21}~\cite{ArmstrongGOS21}\\
Industries & petro-chemical industry &  &  & \href{works/LaborieRSV18.pdf}{LaborieRSV18}~\cite{LaborieRSV18}, \href{works/GilesH16.pdf}{GilesH16}~\cite{GilesH16}\\
Industries & pharmaceutical industry &  &  & \href{works/YuraszeckMCCR23.pdf}{YuraszeckMCCR23}~\cite{YuraszeckMCCR23}, \href{works/CzerniachowskaWZ23.pdf}{CzerniachowskaWZ23}~\cite{CzerniachowskaWZ23}, \href{works/GeibingerKKMMW21.pdf}{GeibingerKKMMW21}~\cite{GeibingerKKMMW21}, \href{works/HamC16.pdf}{HamC16}~\cite{HamC16}, \href{works/NovaraNH16.pdf}{NovaraNH16}~\cite{NovaraNH16}\\
Industries & potash industry &  &  & \href{works/Astrand21.pdf}{Astrand21}~\cite{Astrand21}, \href{works/Astrand0F21.pdf}{Astrand0F21}~\cite{Astrand0F21}, \href{works/AstrandJZ20.pdf}{AstrandJZ20}~\cite{AstrandJZ20}, \href{works/AstrandJZ18.pdf}{AstrandJZ18}~\cite{AstrandJZ18}\\
Industries & power industry & \href{works/Froger16.pdf}{Froger16}~\cite{Froger16} &  & \href{works/FrostD98.pdf}{FrostD98}~\cite{FrostD98}\\
Industries & process industry &  & \href{works/Timpe02.pdf}{Timpe02}~\cite{Timpe02} & \href{works/Nattaf16.pdf}{Nattaf16}~\cite{Nattaf16}, \href{works/BlomPS16.pdf}{BlomPS16}~\cite{BlomPS16}, \href{works/HeinzSSW12.pdf}{HeinzSSW12}~\cite{HeinzSSW12}, \href{works/ChenGPSH10.pdf}{ChenGPSH10}~\cite{ChenGPSH10}, \href{works/Jans09.pdf}{Jans09}~\cite{Jans09}, \href{works/Simonis99.pdf}{Simonis99}~\cite{Simonis99}, \href{works/Wallace96.pdf}{Wallace96}~\cite{Wallace96}\\
Industries & retail industry &  &  & \href{works/ChapadosJR11.pdf}{ChapadosJR11}~\cite{ChapadosJR11}\\
Industries & services industry &  &  & \href{works/DoomsH08.pdf}{DoomsH08}~\cite{DoomsH08}\\
Industries & ship repair industry &  &  & \href{works/BoudreaultSLQ22.pdf}{BoudreaultSLQ22}~\cite{BoudreaultSLQ22}\\
Industries & steel industry &  & \href{works/DavenportKRSH07.pdf}{DavenportKRSH07}~\cite{DavenportKRSH07} & \href{works/LacknerMMWW23.pdf}{LacknerMMWW23}~\cite{LacknerMMWW23}, \href{works/KimCMLLP23.pdf}{KimCMLLP23}~\cite{KimCMLLP23}, \href{works/IsikYA23.pdf}{IsikYA23}~\cite{IsikYA23}, \href{works/OujanaAYB22.pdf}{OujanaAYB22}~\cite{OujanaAYB22}, \href{works/LacknerMMWW21.pdf}{LacknerMMWW21}~\cite{LacknerMMWW21}, \href{works/abs-1902-09244.pdf}{abs-1902-09244}~\cite{abs-1902-09244}, \href{works/GoldwaserS18.pdf}{GoldwaserS18}~\cite{GoldwaserS18}, \href{works/KletzanderM17.pdf}{KletzanderM17}~\cite{KletzanderM17}, \href{works/GoldwaserS17.pdf}{GoldwaserS17}~\cite{GoldwaserS17}, \href{works/HeinzSSW12.pdf}{HeinzSSW12}~\cite{HeinzSSW12}, \href{works/SchausHMCMD11.pdf}{SchausHMCMD11}~\cite{SchausHMCMD11}, \href{works/GrimesH10.pdf}{GrimesH10}~\cite{GrimesH10}, \href{works/GarganiR07.pdf}{GarganiR07}~\cite{GarganiR07}\\
Industries & steel making industry &  &  & \\
Industries & textile industry & \href{works/Mercier-AubinGQ20.pdf}{Mercier-AubinGQ20}~\cite{Mercier-AubinGQ20} &  & \href{works/ZarandiASC20.pdf}{ZarandiASC20}~\cite{ZarandiASC20}, \href{works/BessiereHMQW14.pdf}{BessiereHMQW14}~\cite{BessiereHMQW14}\\
Industries & tourism industry &  &  & \href{works/LiuCGM17.pdf}{LiuCGM17}~\cite{LiuCGM17}\\
\end{longtable}
}


\clearpage
\subsection{Concept Type Benchmarks}
\label{sec:Benchmarks}
\label{Benchmarks}
{\scriptsize
\begin{longtable}{lp{3cm}>{\raggedright\arraybackslash}p{6cm}>{\raggedright\arraybackslash}p{6cm}>{\raggedright\arraybackslash}p{8cm}}
\rowcolor{white}\caption{Works for Concepts of Type Benchmarks (Total 16 Concepts, 16 Used)}\\ \toprule
\rowcolor{white}Type & Keyword & High & Medium & Low\\ \midrule\endhead
\bottomrule
\endfoot
\index{CSPlib}\index{Benchmarks!CSPlib}Benchmarks & CSPlib & \href{../works/LiuLH19a.pdf}{LiuLH19a}~\cite{LiuLH19a}, \href{../works/Siala15.pdf}{Siala15}~\cite{Siala15}, \href{../works/Siala15a.pdf}{Siala15a}~\cite{Siala15a}, \href{../works/SchausHMCMD11.pdf}{SchausHMCMD11}~\cite{SchausHMCMD11}, \href{../works/GarganiR07.pdf}{GarganiR07}~\cite{GarganiR07} & \href{../works/LaborieRSV18.pdf}{LaborieRSV18}~\cite{LaborieRSV18}, \href{../works/German18.pdf}{German18}~\cite{German18}, \href{../works/CappartTSR18.pdf}{CappartTSR18}~\cite{CappartTSR18}, \href{../works/MossigeGSMC17.pdf}{MossigeGSMC17}~\cite{MossigeGSMC17}, \href{../works/NovaraNH16.pdf}{NovaraNH16}~\cite{NovaraNH16}, \href{../works/Letort13.pdf}{Letort13}~\cite{Letort13}, \href{../works/HeinzSSW12.pdf}{HeinzSSW12}~\cite{HeinzSSW12}, \href{../works/BandaSC11.pdf}{BandaSC11}~\cite{BandaSC11} & \href{../works/ThomasKS20.pdf}{ThomasKS20}~\cite{ThomasKS20}, \href{../works/LiuLH19.pdf}{LiuLH19}~\cite{LiuLH19}, \href{../works/LiuLH18.pdf}{LiuLH18}~\cite{LiuLH18}, \href{../works/GelainPRVW17.pdf}{GelainPRVW17}~\cite{GelainPRVW17}, \href{../works/GaySS14.pdf}{GaySS14}~\cite{GaySS14}, \href{../works/RendlPHPR12.pdf}{RendlPHPR12}~\cite{RendlPHPR12}, \href{../works/HentenryckM08.pdf}{HentenryckM08}~\cite{HentenryckM08}\\
\index{Roadef}\index{Benchmarks!Roadef}Benchmarks & Roadef & \href{../works/Froger16.pdf}{Froger16}~\cite{Froger16}, \href{../works/Siala15.pdf}{Siala15}~\cite{Siala15}, \href{../works/Siala15a.pdf}{Siala15a}~\cite{Siala15a} & \href{../works/Nattaf16.pdf}{Nattaf16}~\cite{Nattaf16}, \href{../works/LetortCB15.pdf}{LetortCB15}~\cite{LetortCB15}, \href{../works/Kameugne14.pdf}{Kameugne14}~\cite{Kameugne14}, \href{../works/Letort13.pdf}{Letort13}~\cite{Letort13}, \href{../works/LetortCB13.pdf}{LetortCB13}~\cite{LetortCB13}, \href{../works/LetortBC12.pdf}{LetortBC12}~\cite{LetortBC12} & \href{../works/CzerniachowskaWZ23.pdf}{CzerniachowskaWZ23}~\cite{CzerniachowskaWZ23}, \href{../works/HanenKP21.pdf}{HanenKP21}~\cite{HanenKP21}, \href{../works/Lemos21.pdf}{Lemos21}~\cite{Lemos21}, \href{../works/GokGSTO20.pdf}{GokGSTO20}~\cite{GokGSTO20}, \href{../works/CarlierPSJ20.pdf}{CarlierPSJ20}~\cite{CarlierPSJ20}, \href{../works/Polo-MejiaALB20.pdf}{Polo-MejiaALB20}~\cite{Polo-MejiaALB20}, \href{../works/MalapertN19.pdf}{MalapertN19}~\cite{MalapertN19}, \href{../works/OuelletQ18.pdf}{OuelletQ18}~\cite{OuelletQ18}, \href{../works/Tesch18.pdf}{Tesch18}~\cite{Tesch18}, \href{../works/Fahimi16.pdf}{Fahimi16}~\cite{Fahimi16}, \href{../works/Tesch16.pdf}{Tesch16}~\cite{Tesch16}, \href{../works/Menana11.pdf}{Menana11}~\cite{Menana11}, \href{../works/Acuna-AgostMFG09.pdf}{Acuna-AgostMFG09}~\cite{Acuna-AgostMFG09}, \href{../works/Wallace06.pdf}{Wallace06}~\cite{Wallace06}, \href{../works/Laborie05.pdf}{Laborie05}~\cite{Laborie05}, \href{../works/Elkhyari03.pdf}{Elkhyari03}~\cite{Elkhyari03}\\
\index{benchmark}\index{Benchmarks!benchmark}Benchmarks & benchmark & \href{../works/LiLZDZW24.pdf}{LiLZDZW24}~\cite{LiLZDZW24}, \href{../works/JuvinHL23a.pdf}{JuvinHL23a}~\cite{JuvinHL23a}, \href{../works/IsikYA23.pdf}{IsikYA23}~\cite{IsikYA23}, \href{../works/AlfieriGPS23.pdf}{AlfieriGPS23}~\cite{AlfieriGPS23}, \href{../works/JuvinHHL23.pdf}{JuvinHHL23}~\cite{JuvinHHL23}, \href{../works/Bit-Monnot23.pdf}{Bit-Monnot23}~\cite{Bit-Monnot23}, \href{../works/NaderiBZR23.pdf}{NaderiBZR23}~\cite{NaderiBZR23}, \href{../works/AfsarVPG23.pdf}{AfsarVPG23}~\cite{AfsarVPG23}, \href{../works/YuraszeckMCCR23.pdf}{YuraszeckMCCR23}~\cite{YuraszeckMCCR23}, \href{../works/ShaikhK23.pdf}{ShaikhK23}~\cite{ShaikhK23}, \href{../works/ZhuSZW23.pdf}{ZhuSZW23}~\cite{ZhuSZW23}, \href{../works/NaderiRR23.pdf}{NaderiRR23}~\cite{NaderiRR23}, \href{../works/TasselGS23.pdf}{TasselGS23}~\cite{TasselGS23}, \href{../works/AbreuPNF23.pdf}{AbreuPNF23}~\cite{AbreuPNF23}, \href{../works/TardivoDFMP23.pdf}{TardivoDFMP23}~\cite{TardivoDFMP23}, \href{../works/LacknerMMWW23.pdf}{LacknerMMWW23}~\cite{LacknerMMWW23}, \href{../works/PovedaAA23.pdf}{PovedaAA23}~\cite{PovedaAA23}, \href{../works/abs-2306-05747.pdf}{abs-2306-05747}~\cite{abs-2306-05747}, \href{../works/AbreuNP23.pdf}{AbreuNP23}~\cite{AbreuNP23}, \href{../works/OuelletQ22.pdf}{OuelletQ22}~\cite{OuelletQ22}, \href{../works/ColT22.pdf}{ColT22}~\cite{ColT22}, \href{../works/WinterMMW22.pdf}{WinterMMW22}~\cite{WinterMMW22}, \href{../works/JuvinHL22.pdf}{JuvinHL22}~\cite{JuvinHL22}, \href{../works/NaderiBZ22a.pdf}{NaderiBZ22a}~\cite{NaderiBZ22a}, \href{../works/Teppan22.pdf}{Teppan22}~\cite{Teppan22}, \href{../works/MengGRZSC22.pdf}{MengGRZSC22}~\cite{MengGRZSC22}, \href{../works/ZhangJZL22.pdf}{ZhangJZL22}~\cite{ZhangJZL22}, \href{../works/TouatBT22.pdf}{TouatBT22}~\cite{TouatBT22}, \href{../works/AbreuN22.pdf}{AbreuN22}~\cite{AbreuN22}... (Total: 128) & \href{../works/ForbesHJST24.pdf}{ForbesHJST24}~\cite{ForbesHJST24}, \href{../works/abs-2402-00459.pdf}{abs-2402-00459}~\cite{abs-2402-00459}, \href{../works/NaderiBZ23.pdf}{NaderiBZ23}~\cite{NaderiBZ23}, \href{../works/YuraszeckMC23.pdf}{YuraszeckMC23}~\cite{YuraszeckMC23}, \href{../works/MontemanniD23a.pdf}{MontemanniD23a}~\cite{MontemanniD23a}, \href{../works/MarliereSPR23.pdf}{MarliereSPR23}~\cite{MarliereSPR23}, \href{../works/AkramNHRSA23.pdf}{AkramNHRSA23}~\cite{AkramNHRSA23}, \href{../works/FrimodigECM23.pdf}{FrimodigECM23}~\cite{FrimodigECM23}, \href{../works/IklassovMR023.pdf}{IklassovMR023}~\cite{IklassovMR023}, \href{../works/KameugneFND23.pdf}{KameugneFND23}~\cite{KameugneFND23}, \href{../works/abs-2305-19888.pdf}{abs-2305-19888}~\cite{abs-2305-19888}, \href{../works/NaderiBZ22.pdf}{NaderiBZ22}~\cite{NaderiBZ22}, \href{../works/BourreauGGLT22.pdf}{BourreauGGLT22}~\cite{BourreauGGLT22}, \href{../works/KotaryFH22.pdf}{KotaryFH22}~\cite{KotaryFH22}, \href{../works/ZhangBB22.pdf}{ZhangBB22}~\cite{ZhangBB22}, \href{../works/FetgoD22.pdf}{FetgoD22}~\cite{FetgoD22}, \href{../works/Tassel22.pdf}{Tassel22}~\cite{Tassel22}, \href{../works/OujanaAYB22.pdf}{OujanaAYB22}~\cite{OujanaAYB22}, \href{../works/HeinzNVH22.pdf}{HeinzNVH22}~\cite{HeinzNVH22}, \href{../works/BulckG22.pdf}{BulckG22}~\cite{BulckG22}, \href{../works/MengLZB21.pdf}{MengLZB21}~\cite{MengLZB21}, \href{../works/Astrand21.pdf}{Astrand21}~\cite{Astrand21}, \href{../works/AbreuAPNM21.pdf}{AbreuAPNM21}~\cite{AbreuAPNM21}, \href{../works/KovacsTKSG21.pdf}{KovacsTKSG21}~\cite{KovacsTKSG21}, \href{../works/KletzanderMH21.pdf}{KletzanderMH21}~\cite{KletzanderMH21}, \href{../works/Lunardi20.pdf}{Lunardi20}~\cite{Lunardi20}, \href{../works/SacramentoSP20.pdf}{SacramentoSP20}~\cite{SacramentoSP20}, \href{../works/BenediktMH20.pdf}{BenediktMH20}~\cite{BenediktMH20}, \href{../works/BadicaBI20.pdf}{BadicaBI20}~\cite{BadicaBI20}... (Total: 113) & \href{../works/BonninMNE24.pdf}{BonninMNE24}~\cite{BonninMNE24}, \href{../works/PrataAN23.pdf}{PrataAN23}~\cite{PrataAN23}, \href{../works/MontemanniD23.pdf}{MontemanniD23}~\cite{MontemanniD23}, \href{../works/GuoZ23.pdf}{GuoZ23}~\cite{GuoZ23}, \href{../works/WessenCSFPM23.pdf}{WessenCSFPM23}~\cite{WessenCSFPM23}, \href{../works/Adelgren2023.pdf}{Adelgren2023}~\cite{Adelgren2023}, \href{../works/CzerniachowskaWZ23.pdf}{CzerniachowskaWZ23}~\cite{CzerniachowskaWZ23}, \href{../works/EfthymiouY23.pdf}{EfthymiouY23}~\cite{EfthymiouY23}, \href{../works/KimCMLLP23.pdf}{KimCMLLP23}~\cite{KimCMLLP23}, \href{../works/SquillaciPR23.pdf}{SquillaciPR23}~\cite{SquillaciPR23}, \href{../works/SvancaraB22.pdf}{SvancaraB22}~\cite{SvancaraB22}, \href{../works/JungblutK22.pdf}{JungblutK22}~\cite{JungblutK22}, \href{../works/ElciOH22.pdf}{ElciOH22}~\cite{ElciOH22}, \href{../works/PohlAK22.pdf}{PohlAK22}~\cite{PohlAK22}, \href{../works/YunusogluY22.pdf}{YunusogluY22}~\cite{YunusogluY22}, \href{../works/SubulanC22.pdf}{SubulanC22}~\cite{SubulanC22}, \href{../works/YuraszeckMPV22.pdf}{YuraszeckMPV22}~\cite{YuraszeckMPV22}, \href{../works/AwadMDMT22.pdf}{AwadMDMT22}~\cite{AwadMDMT22}, \href{../works/ArmstrongGOS22.pdf}{ArmstrongGOS22}~\cite{ArmstrongGOS22}, \href{../works/Astrand0F21.pdf}{Astrand0F21}~\cite{Astrand0F21}, \href{../works/VlkHT21.pdf}{VlkHT21}~\cite{VlkHT21}, \href{../works/Zahout21.pdf}{Zahout21}~\cite{Zahout21}, \href{../works/RoshanaeiN21.pdf}{RoshanaeiN21}~\cite{RoshanaeiN21}, \href{../works/HubnerGSV21.pdf}{HubnerGSV21}~\cite{HubnerGSV21}, \href{../works/KlankeBYE21.pdf}{KlankeBYE21}~\cite{KlankeBYE21}, \href{../works/ArmstrongGOS21.pdf}{ArmstrongGOS21}~\cite{ArmstrongGOS21}, \href{../works/AstrandJZ20.pdf}{AstrandJZ20}~\cite{AstrandJZ20}, \href{../works/LunardiBLRV20.pdf}{LunardiBLRV20}~\cite{LunardiBLRV20}, \href{../works/ThomasKS20.pdf}{ThomasKS20}~\cite{ThomasKS20}... (Total: 176)\\
\index{bitbucket}\index{Benchmarks!bitbucket}Benchmarks & bitbucket &  & \href{../works/TardivoDFMP23.pdf}{TardivoDFMP23}~\cite{TardivoDFMP23}, \href{../works/Dejemeppe16.pdf}{Dejemeppe16}~\cite{Dejemeppe16} & \href{../works/ThomasKS20.pdf}{ThomasKS20}~\cite{ThomasKS20}, \href{../works/CauwelaertDS20.pdf}{CauwelaertDS20}~\cite{CauwelaertDS20}, \href{../works/HoundjiSW19.pdf}{HoundjiSW19}~\cite{HoundjiSW19}, \href{../works/CappartTSR18.pdf}{CappartTSR18}~\cite{CappartTSR18}, \href{../works/CauwelaertLS18.pdf}{CauwelaertLS18}~\cite{CauwelaertLS18}, \href{../works/He0GLW18.pdf}{He0GLW18}~\cite{He0GLW18}, \href{../works/CappartS17.pdf}{CappartS17}~\cite{CappartS17}, \href{../works/CauwelaertDMS16.pdf}{CauwelaertDMS16}~\cite{CauwelaertDMS16}, \href{../works/GayHLS15.pdf}{GayHLS15}~\cite{GayHLS15}, \href{../works/CauwelaertLS15.pdf}{CauwelaertLS15}~\cite{CauwelaertLS15}, \href{../works/DejemeppeCS15.pdf}{DejemeppeCS15}~\cite{DejemeppeCS15}, \href{../works/GayHS15a.pdf}{GayHS15a}~\cite{GayHS15a}, \href{../works/GayHS15.pdf}{GayHS15}~\cite{GayHS15}, \href{../works/DejemeppeD14.pdf}{DejemeppeD14}~\cite{DejemeppeD14}, \href{../works/HoundjiSWD14.pdf}{HoundjiSWD14}~\cite{HoundjiSWD14}\\
\index{generated instance}\index{Benchmarks!generated instance}Benchmarks & generated instance & \href{../works/IsikYA23.pdf}{IsikYA23}~\cite{IsikYA23}, \href{../works/LuoB22.pdf}{LuoB22}~\cite{LuoB22}, \href{../works/abs-1911-04766.pdf}{abs-1911-04766}~\cite{abs-1911-04766} & \href{../works/abs-2312-13682.pdf}{abs-2312-13682}~\cite{abs-2312-13682}, \href{../works/PerezGSL23.pdf}{PerezGSL23}~\cite{PerezGSL23}, \href{../works/OrnekOS20.pdf}{OrnekOS20}~\cite{OrnekOS20}, \href{../works/Godet21a.pdf}{Godet21a}~\cite{Godet21a}, \href{../works/GodetLHS20.pdf}{GodetLHS20}~\cite{GodetLHS20}, \href{../works/KletzanderM20.pdf}{KletzanderM20}~\cite{KletzanderM20}, \href{../works/MejiaY20.pdf}{MejiaY20}~\cite{MejiaY20}, \href{../works/SunTB19.pdf}{SunTB19}~\cite{SunTB19}, \href{../works/Madi-WambaB16.pdf}{Madi-WambaB16}~\cite{Madi-WambaB16}, \href{../works/NattafALR16.pdf}{NattafALR16}~\cite{NattafALR16}, \href{../works/Dejemeppe16.pdf}{Dejemeppe16}~\cite{Dejemeppe16}, \href{../works/KelbelH11.pdf}{KelbelH11}~\cite{KelbelH11}, \href{../works/SchausHMCMD11.pdf}{SchausHMCMD11}~\cite{SchausHMCMD11} & \href{../works/abs-2402-00459.pdf}{abs-2402-00459}~\cite{abs-2402-00459}, \href{../works/EfthymiouY23.pdf}{EfthymiouY23}~\cite{EfthymiouY23}, \href{../works/abs-2305-19888.pdf}{abs-2305-19888}~\cite{abs-2305-19888}, \href{../works/Adelgren2023.pdf}{Adelgren2023}~\cite{Adelgren2023}, \href{../works/NaderiBZR23.pdf}{NaderiBZR23}~\cite{NaderiBZR23}, \href{../works/TouatBT22.pdf}{TouatBT22}~\cite{TouatBT22}, \href{../works/ZhangBB22.pdf}{ZhangBB22}~\cite{ZhangBB22}, \href{../works/abs-2211-14492.pdf}{abs-2211-14492}~\cite{abs-2211-14492}, \href{../works/ColT22.pdf}{ColT22}~\cite{ColT22}, \href{../works/YunusogluY22.pdf}{YunusogluY22}~\cite{YunusogluY22}, \href{../works/BoudreaultSLQ22.pdf}{BoudreaultSLQ22}~\cite{BoudreaultSLQ22}, \href{../works/YuraszeckMPV22.pdf}{YuraszeckMPV22}~\cite{YuraszeckMPV22}, \href{../works/HeinzNVH22.pdf}{HeinzNVH22}~\cite{HeinzNVH22}, \href{../works/HanenKP21.pdf}{HanenKP21}~\cite{HanenKP21}, \href{../works/Astrand21.pdf}{Astrand21}~\cite{Astrand21}, \href{../works/AbreuAPNM21.pdf}{AbreuAPNM21}~\cite{AbreuAPNM21}, \href{../works/GeibingerMM21.pdf}{GeibingerMM21}~\cite{GeibingerMM21}, \href{../works/Astrand0F21.pdf}{Astrand0F21}~\cite{Astrand0F21}, \href{../works/AbohashimaEG21.pdf}{AbohashimaEG21}~\cite{AbohashimaEG21}, \href{../works/abs-2102-08778.pdf}{abs-2102-08778}~\cite{abs-2102-08778}, \href{../works/AntuoriHHEN20.pdf}{AntuoriHHEN20}~\cite{AntuoriHHEN20}, \href{../works/CauwelaertDS20.pdf}{CauwelaertDS20}~\cite{CauwelaertDS20}, \href{../works/BenediktMH20.pdf}{BenediktMH20}~\cite{BenediktMH20}, \href{../works/MokhtarzadehTNF20.pdf}{MokhtarzadehTNF20}~\cite{MokhtarzadehTNF20}, \href{../works/RoshanaeiBAUB20.pdf}{RoshanaeiBAUB20}~\cite{RoshanaeiBAUB20}, \href{../works/LunardiBLRV20.pdf}{LunardiBLRV20}~\cite{LunardiBLRV20}, \href{../works/ThomasKS20.pdf}{ThomasKS20}~\cite{ThomasKS20}, \href{../works/Lunardi20.pdf}{Lunardi20}~\cite{Lunardi20}, \href{../works/Ham20a.pdf}{Ham20a}~\cite{Ham20a}... (Total: 68)\\
\index{github}\index{Benchmarks!github}Benchmarks & github & \href{../works/Lemos21.pdf}{Lemos21}~\cite{Lemos21}, \href{../works/Godet21a.pdf}{Godet21a}~\cite{Godet21a}, \href{../works/KoehlerBFFHPSSS21.pdf}{KoehlerBFFHPSSS21}~\cite{KoehlerBFFHPSSS21} & \href{../works/FalqueALM24.pdf}{FalqueALM24}~\cite{FalqueALM24}, \href{../works/PovedaAA23.pdf}{PovedaAA23}~\cite{PovedaAA23}, \href{../works/TardivoDFMP23.pdf}{TardivoDFMP23}~\cite{TardivoDFMP23}, \href{../works/BoudreaultSLQ22.pdf}{BoudreaultSLQ22}~\cite{BoudreaultSLQ22}, \href{../works/JungblutK22.pdf}{JungblutK22}~\cite{JungblutK22}, \href{../works/HamPK21.pdf}{HamPK21}~\cite{HamPK21}, \href{../works/LunardiBLRV20.pdf}{LunardiBLRV20}~\cite{LunardiBLRV20}, \href{../works/GodetLHS20.pdf}{GodetLHS20}~\cite{GodetLHS20}, \href{../works/BenediktMH20.pdf}{BenediktMH20}~\cite{BenediktMH20}, \href{../works/Siala15.pdf}{Siala15}~\cite{Siala15}, \href{../works/Siala15a.pdf}{Siala15a}~\cite{Siala15a} & \href{../works/LiLZDZW24.pdf}{LiLZDZW24}~\cite{LiLZDZW24}, \href{../works/ForbesHJST24.pdf}{ForbesHJST24}~\cite{ForbesHJST24}, \href{../works/abs-2402-00459.pdf}{abs-2402-00459}~\cite{abs-2402-00459}, \href{../works/SquillaciPR23.pdf}{SquillaciPR23}~\cite{SquillaciPR23}, \href{../works/JuvinHHL23.pdf}{JuvinHHL23}~\cite{JuvinHHL23}, \href{../works/YuraszeckMC23.pdf}{YuraszeckMC23}~\cite{YuraszeckMC23}, \href{../works/abs-2306-05747.pdf}{abs-2306-05747}~\cite{abs-2306-05747}, \href{../works/NaderiRR23.pdf}{NaderiRR23}~\cite{NaderiRR23}, \href{../works/Adelgren2023.pdf}{Adelgren2023}~\cite{Adelgren2023}, \href{../works/TasselGS23.pdf}{TasselGS23}~\cite{TasselGS23}, \href{../works/WessenCSFPM23.pdf}{WessenCSFPM23}~\cite{WessenCSFPM23}, \href{../works/YuraszeckMCCR23.pdf}{YuraszeckMCCR23}~\cite{YuraszeckMCCR23}, \href{../works/Fatemi-AnarakiTFV23.pdf}{Fatemi-AnarakiTFV23}~\cite{Fatemi-AnarakiTFV23}, \href{../works/GuoZ23.pdf}{GuoZ23}~\cite{GuoZ23}, \href{../works/GokPTGO23.pdf}{GokPTGO23}~\cite{GokPTGO23}, \href{../works/Bit-Monnot23.pdf}{Bit-Monnot23}~\cite{Bit-Monnot23}, \href{../works/IklassovMR023.pdf}{IklassovMR023}~\cite{IklassovMR023}, \href{../works/OuelletQ22.pdf}{OuelletQ22}~\cite{OuelletQ22}, \href{../works/EmdeZD22.pdf}{EmdeZD22}~\cite{EmdeZD22}, \href{../works/GeitzGSSW22.pdf}{GeitzGSSW22}~\cite{GeitzGSSW22}, \href{../works/KotaryFH22.pdf}{KotaryFH22}~\cite{KotaryFH22}, \href{../works/Tassel22.pdf}{Tassel22}~\cite{Tassel22}, \href{../works/ColT22.pdf}{ColT22}~\cite{ColT22}, \href{../works/MullerMKP22.pdf}{MullerMKP22}~\cite{MullerMKP22}, \href{../works/LuoB22.pdf}{LuoB22}~\cite{LuoB22}, \href{../works/YuraszeckMPV22.pdf}{YuraszeckMPV22}~\cite{YuraszeckMPV22}, \href{../works/KovacsTKSG21.pdf}{KovacsTKSG21}~\cite{KovacsTKSG21}, \href{../works/GeibingerMM21.pdf}{GeibingerMM21}~\cite{GeibingerMM21}, \href{../works/VlkHT21.pdf}{VlkHT21}~\cite{VlkHT21}... (Total: 55)\\
\index{gitlab}\index{Benchmarks!gitlab}Benchmarks & gitlab &  & \href{../works/HeinzNVH22.pdf}{HeinzNVH22}~\cite{HeinzNVH22} & \href{../works/FalqueALM24.pdf}{FalqueALM24}~\cite{FalqueALM24}, \href{../works/abs-2305-19888.pdf}{abs-2305-19888}~\cite{abs-2305-19888}, \href{../works/BoudreaultSLQ22.pdf}{BoudreaultSLQ22}~\cite{BoudreaultSLQ22}, \href{../works/AntuoriHHEN21.pdf}{AntuoriHHEN21}~\cite{AntuoriHHEN21}, \href{../works/AntuoriHHEN20.pdf}{AntuoriHHEN20}~\cite{AntuoriHHEN20}\\
\index{industrial instance}\index{Benchmarks!industrial instance}Benchmarks & industrial instance & \href{../works/LuoB22.pdf}{LuoB22}~\cite{LuoB22}, \href{../works/AntuoriHHEN20.pdf}{AntuoriHHEN20}~\cite{AntuoriHHEN20} & \href{../works/BonfiettiZLM16.pdf}{BonfiettiZLM16}~\cite{BonfiettiZLM16}, \href{../works/BonfiettiLBM14.pdf}{BonfiettiLBM14}~\cite{BonfiettiLBM14}, \href{../works/Schutt11.pdf}{Schutt11}~\cite{Schutt11} & \href{../works/TasselGS23.pdf}{TasselGS23}~\cite{TasselGS23}, \href{../works/PovedaAA23.pdf}{PovedaAA23}~\cite{PovedaAA23}, \href{../works/EfthymiouY23.pdf}{EfthymiouY23}~\cite{EfthymiouY23}, \href{../works/abs-2306-05747.pdf}{abs-2306-05747}~\cite{abs-2306-05747}, \href{../works/Tassel22.pdf}{Tassel22}~\cite{Tassel22}, \href{../works/OujanaAYB22.pdf}{OujanaAYB22}~\cite{OujanaAYB22}, \href{../works/GroleazNS20.pdf}{GroleazNS20}~\cite{GroleazNS20}, \href{../works/NattafM20.pdf}{NattafM20}~\cite{NattafM20}, \href{../works/Mercier-AubinGQ20.pdf}{Mercier-AubinGQ20}~\cite{Mercier-AubinGQ20}, \href{../works/MalapertN19.pdf}{MalapertN19}~\cite{MalapertN19}, \href{../works/Hooker19.pdf}{Hooker19}~\cite{Hooker19}, \href{../works/BofillGSV15.pdf}{BofillGSV15}~\cite{BofillGSV15}, \href{../works/BofillEGPSV14.pdf}{BofillEGPSV14}~\cite{BofillEGPSV14}, \href{../works/ZampelliVSDR13.pdf}{ZampelliVSDR13}~\cite{ZampelliVSDR13}, \href{../works/BonfiettiM12.pdf}{BonfiettiM12}~\cite{BonfiettiM12}, \href{../works/LombardiBMB11.pdf}{LombardiBMB11}~\cite{LombardiBMB11}, \href{../works/BonfiettiLBM11.pdf}{BonfiettiLBM11}~\cite{BonfiettiLBM11}\\
\index{industrial partner}\index{Benchmarks!industrial partner}Benchmarks & industrial partner & \href{../works/BoudreaultSLQ22.pdf}{BoudreaultSLQ22}~\cite{BoudreaultSLQ22}, \href{../works/Lunardi20.pdf}{Lunardi20}~\cite{Lunardi20}, \href{../works/Dejemeppe16.pdf}{Dejemeppe16}~\cite{Dejemeppe16} & \href{../works/LacknerMMWW23.pdf}{LacknerMMWW23}~\cite{LacknerMMWW23}, \href{../works/ArmstrongGOS21.pdf}{ArmstrongGOS21}~\cite{ArmstrongGOS21}, \href{../works/ZampelliVSDR13.pdf}{ZampelliVSDR13}~\cite{ZampelliVSDR13} & \href{../works/WinterMMW22.pdf}{WinterMMW22}~\cite{WinterMMW22}, \href{../works/Tassel22.pdf}{Tassel22}~\cite{Tassel22}, \href{../works/VlkHT21.pdf}{VlkHT21}~\cite{VlkHT21}, \href{../works/LacknerMMWW21.pdf}{LacknerMMWW21}~\cite{LacknerMMWW21}, \href{../works/GroleazNS20a.pdf}{GroleazNS20a}~\cite{GroleazNS20a}, \href{../works/Mercier-AubinGQ20.pdf}{Mercier-AubinGQ20}~\cite{Mercier-AubinGQ20}, \href{../works/AntunesABD20.pdf}{AntunesABD20}~\cite{AntunesABD20}, \href{../works/abs-1911-04766.pdf}{abs-1911-04766}~\cite{abs-1911-04766}, \href{../works/GeibingerMM19.pdf}{GeibingerMM19}~\cite{GeibingerMM19}, \href{../works/AntunesABD18.pdf}{AntunesABD18}~\cite{AntunesABD18}, \href{../works/MossigeGSMC17.pdf}{MossigeGSMC17}~\cite{MossigeGSMC17}, \href{../works/Froger16.pdf}{Froger16}~\cite{Froger16}, \href{../works/HebrardHJMPV16.pdf}{HebrardHJMPV16}~\cite{HebrardHJMPV16}, \href{../works/AlesioBNG15.pdf}{AlesioBNG15}~\cite{AlesioBNG15}, \href{../works/LipovetzkyBPS14.pdf}{LipovetzkyBPS14}~\cite{LipovetzkyBPS14}, \href{../works/LimtanyakulS12.pdf}{LimtanyakulS12}~\cite{LimtanyakulS12}, \href{../works/Malapert11.pdf}{Malapert11}~\cite{Malapert11}, \href{../works/DoRZ08.pdf}{DoRZ08}~\cite{DoRZ08}, \href{../works/KovacsV06.pdf}{KovacsV06}~\cite{KovacsV06}, \href{../works/KovacsV04.pdf}{KovacsV04}~\cite{KovacsV04}, \href{../works/EreminW01.pdf}{EreminW01}~\cite{EreminW01}\\
\index{industry partner}\index{Benchmarks!industry partner}Benchmarks & industry partner & \href{../works/BurtLPS15.pdf}{BurtLPS15}~\cite{BurtLPS15}, \href{../works/LipovetzkyBPS14.pdf}{LipovetzkyBPS14}~\cite{LipovetzkyBPS14} & \href{../works/BlomBPS14.pdf}{BlomBPS14}~\cite{BlomBPS14} & \href{../works/LuoB22.pdf}{LuoB22}~\cite{LuoB22}, \href{../works/WinterMMW22.pdf}{WinterMMW22}~\cite{WinterMMW22}, \href{../works/ArmstrongGOS21.pdf}{ArmstrongGOS21}~\cite{ArmstrongGOS21}, \href{../works/HauderBRPA20.pdf}{HauderBRPA20}~\cite{HauderBRPA20}, \href{../works/abs-1902-09244.pdf}{abs-1902-09244}~\cite{abs-1902-09244}, \href{../works/AntunesABD18.pdf}{AntunesABD18}~\cite{AntunesABD18}, \href{../works/BlomPS16.pdf}{BlomPS16}~\cite{BlomPS16}\\
\index{instance generator}\index{Benchmarks!instance generator}Benchmarks & instance generator & \href{../works/LacknerMMWW23.pdf}{LacknerMMWW23}~\cite{LacknerMMWW23}, \href{../works/LacknerMMWW21.pdf}{LacknerMMWW21}~\cite{LacknerMMWW21} & \href{../works/GoldwaserS18.pdf}{GoldwaserS18}~\cite{GoldwaserS18}, \href{../works/Froger16.pdf}{Froger16}~\cite{Froger16} & \href{../works/abs-2402-00459.pdf}{abs-2402-00459}~\cite{abs-2402-00459}, \href{../works/FrimodigECM23.pdf}{FrimodigECM23}~\cite{FrimodigECM23}, \href{../works/ArmstrongGOS21.pdf}{ArmstrongGOS21}~\cite{ArmstrongGOS21}, \href{../works/Lunardi20.pdf}{Lunardi20}~\cite{Lunardi20}, \href{../works/KletzanderM20.pdf}{KletzanderM20}~\cite{KletzanderM20}, \href{../works/SunTB19.pdf}{SunTB19}~\cite{SunTB19}, \href{../works/abs-1911-04766.pdf}{abs-1911-04766}~\cite{abs-1911-04766}, \href{../works/Caballero19.pdf}{Caballero19}~\cite{Caballero19}, \href{../works/GombolayWS18.pdf}{GombolayWS18}~\cite{GombolayWS18}, \href{../works/GoldwaserS17.pdf}{GoldwaserS17}~\cite{GoldwaserS17}, \href{../works/YoungFS17.pdf}{YoungFS17}~\cite{YoungFS17}, \href{../works/Dejemeppe16.pdf}{Dejemeppe16}~\cite{Dejemeppe16}, \href{../works/UnsalO13.pdf}{UnsalO13}~\cite{UnsalO13}, \href{../works/GuyonLPR12.pdf}{GuyonLPR12}~\cite{GuyonLPR12}, \href{../works/Schutt11.pdf}{Schutt11}~\cite{Schutt11}, \href{../works/BeniniLMR11.pdf}{BeniniLMR11}~\cite{BeniniLMR11}, \href{../works/Lombardi10.pdf}{Lombardi10}~\cite{Lombardi10}, \href{../works/abs-1009-0347.pdf}{abs-1009-0347}~\cite{abs-1009-0347}, \href{../works/RuggieroBBMA09.pdf}{RuggieroBBMA09}~\cite{RuggieroBBMA09}, \href{../works/LombardiM09.pdf}{LombardiM09}~\cite{LombardiM09}, \href{../works/HeipckeCCS00.pdf}{HeipckeCCS00}~\cite{HeipckeCCS00}\\
\index{random instance}\index{Benchmarks!random instance}Benchmarks & random instance & \href{../works/LacknerMMWW21.pdf}{LacknerMMWW21}~\cite{LacknerMMWW21}, \href{../works/WallaceY20.pdf}{WallaceY20}~\cite{WallaceY20}, \href{../works/Dejemeppe16.pdf}{Dejemeppe16}~\cite{Dejemeppe16} & \href{../works/WangB23.pdf}{WangB23}~\cite{WangB23}, \href{../works/LacknerMMWW23.pdf}{LacknerMMWW23}~\cite{LacknerMMWW23}, \href{../works/EfthymiouY23.pdf}{EfthymiouY23}~\cite{EfthymiouY23}, \href{../works/LetortCB15.pdf}{LetortCB15}~\cite{LetortCB15}, \href{../works/KelbelH11.pdf}{KelbelH11}~\cite{KelbelH11} & \href{../works/Mehdizadeh-Somarin23.pdf}{Mehdizadeh-Somarin23}~\cite{Mehdizadeh-Somarin23}, \href{../works/Fatemi-AnarakiTFV23.pdf}{Fatemi-AnarakiTFV23}~\cite{Fatemi-AnarakiTFV23}, \href{../works/OuelletQ22.pdf}{OuelletQ22}~\cite{OuelletQ22}, \href{../works/ElciOH22.pdf}{ElciOH22}~\cite{ElciOH22}, \href{../works/MullerMKP22.pdf}{MullerMKP22}~\cite{MullerMKP22}, \href{../works/EmdeZD22.pdf}{EmdeZD22}~\cite{EmdeZD22}, \href{../works/abs-2211-14492.pdf}{abs-2211-14492}~\cite{abs-2211-14492}, \href{../works/VlkHT21.pdf}{VlkHT21}~\cite{VlkHT21}, \href{../works/Godet21a.pdf}{Godet21a}~\cite{Godet21a}, \href{../works/KlankeBYE21.pdf}{KlankeBYE21}~\cite{KlankeBYE21}, \href{../works/HanenKP21.pdf}{HanenKP21}~\cite{HanenKP21}, \href{../works/AntuoriHHEN20.pdf}{AntuoriHHEN20}~\cite{AntuoriHHEN20}, \href{../works/Lunardi20.pdf}{Lunardi20}~\cite{Lunardi20}, \href{../works/BenediktMH20.pdf}{BenediktMH20}~\cite{BenediktMH20}, \href{../works/LunardiBLRV20.pdf}{LunardiBLRV20}~\cite{LunardiBLRV20}, \href{../works/HoundjiSW19.pdf}{HoundjiSW19}~\cite{HoundjiSW19}, \href{../works/UnsalO19.pdf}{UnsalO19}~\cite{UnsalO19}, \href{../works/FahimiOQ18.pdf}{FahimiOQ18}~\cite{FahimiOQ18}, \href{../works/BenediktSMVH18.pdf}{BenediktSMVH18}~\cite{BenediktSMVH18}, \href{../works/Hooker17.pdf}{Hooker17}~\cite{Hooker17}, \href{../works/MossigeGSMC17.pdf}{MossigeGSMC17}~\cite{MossigeGSMC17}, \href{../works/CappartS17.pdf}{CappartS17}~\cite{CappartS17}, \href{../works/Fahimi16.pdf}{Fahimi16}~\cite{Fahimi16}, \href{../works/Madi-WambaB16.pdf}{Madi-WambaB16}~\cite{Madi-WambaB16}, \href{../works/BoothTNB16.pdf}{BoothTNB16}~\cite{BoothTNB16}, \href{../works/Siala15.pdf}{Siala15}~\cite{Siala15}, \href{../works/Siala15a.pdf}{Siala15a}~\cite{Siala15a}, \href{../works/LarsonJC14.pdf}{LarsonJC14}~\cite{LarsonJC14}, \href{../works/KameugneFSN14.pdf}{KameugneFSN14}~\cite{KameugneFSN14}... (Total: 45)\\
\index{real-life}\index{Benchmarks!real-life}Benchmarks & real-life & \href{../works/GurPAE23.pdf}{GurPAE23}~\cite{GurPAE23}, \href{../works/SubulanC22.pdf}{SubulanC22}~\cite{SubulanC22}, \href{../works/WinterMMW22.pdf}{WinterMMW22}~\cite{WinterMMW22}, \href{../works/HubnerGSV21.pdf}{HubnerGSV21}~\cite{HubnerGSV21}, \href{../works/Astrand21.pdf}{Astrand21}~\cite{Astrand21}, \href{../works/QinDCS20.pdf}{QinDCS20}~\cite{QinDCS20}, \href{../works/KletzanderM20.pdf}{KletzanderM20}~\cite{KletzanderM20}, \href{../works/GurEA19.pdf}{GurEA19}~\cite{GurEA19}, \href{../works/RiiseML16.pdf}{RiiseML16}~\cite{RiiseML16}, \href{../works/WangMD15.pdf}{WangMD15}~\cite{WangMD15}, \href{../works/BartakCS10.pdf}{BartakCS10}~\cite{BartakCS10}, \href{../works/ChenGPSH10.pdf}{ChenGPSH10}~\cite{ChenGPSH10}, \href{../works/BartakSR10.pdf}{BartakSR10}~\cite{BartakSR10}, \href{../works/Baptiste02.pdf}{Baptiste02}~\cite{Baptiste02}, \href{../works/Bartak02a.pdf}{Bartak02a}~\cite{Bartak02a}, \href{../works/MartinPY01.pdf}{MartinPY01}~\cite{MartinPY01} & \href{../works/LuZZYW24.pdf}{LuZZYW24}~\cite{LuZZYW24}, \href{../works/AlakaP23.pdf}{AlakaP23}~\cite{AlakaP23}, \href{../works/AfsarVPG23.pdf}{AfsarVPG23}~\cite{AfsarVPG23}, \href{../works/LacknerMMWW23.pdf}{LacknerMMWW23}~\cite{LacknerMMWW23}, \href{../works/OujanaAYB22.pdf}{OujanaAYB22}~\cite{OujanaAYB22}, \href{../works/BulckG22.pdf}{BulckG22}~\cite{BulckG22}, \href{../works/Lemos21.pdf}{Lemos21}~\cite{Lemos21}, \href{../works/KlankeBYE21.pdf}{KlankeBYE21}~\cite{KlankeBYE21}, \href{../works/Astrand0F21.pdf}{Astrand0F21}~\cite{Astrand0F21}, \href{../works/LacknerMMWW21.pdf}{LacknerMMWW21}~\cite{LacknerMMWW21}, \href{../works/KletzanderMH21.pdf}{KletzanderMH21}~\cite{KletzanderMH21}, \href{../works/FallahiAC20.pdf}{FallahiAC20}~\cite{FallahiAC20}, \href{../works/Lunardi20.pdf}{Lunardi20}~\cite{Lunardi20}, \href{../works/abs-1911-04766.pdf}{abs-1911-04766}~\cite{abs-1911-04766}, \href{../works/PourDERB18.pdf}{PourDERB18}~\cite{PourDERB18}, \href{../works/MusliuSS18.pdf}{MusliuSS18}~\cite{MusliuSS18}, \href{../works/AmadiniGM16.pdf}{AmadiniGM16}~\cite{AmadiniGM16}, \href{../works/Froger16.pdf}{Froger16}~\cite{Froger16}, \href{../works/BartakV15.pdf}{BartakV15}~\cite{BartakV15}, \href{../works/GaySS14.pdf}{GaySS14}~\cite{GaySS14}, \href{../works/LimtanyakulS12.pdf}{LimtanyakulS12}~\cite{LimtanyakulS12}, \href{../works/MenciaSV12.pdf}{MenciaSV12}~\cite{MenciaSV12}, \href{../works/MeskensDHG11.pdf}{MeskensDHG11}~\cite{MeskensDHG11}, \href{../works/LombardiMRB10.pdf}{LombardiMRB10}~\cite{LombardiMRB10}, \href{../works/RuggieroBBMA09.pdf}{RuggieroBBMA09}~\cite{RuggieroBBMA09}, \href{../works/BartakSR08.pdf}{BartakSR08}~\cite{BartakSR08}, \href{../works/Musliu05.pdf}{Musliu05}~\cite{Musliu05}, \href{../works/Tsang03.pdf}{Tsang03}~\cite{Tsang03}, \href{../works/BosiM2001.pdf}{BosiM2001}~\cite{BosiM2001}... (Total: 33) & \href{../works/PrataAN23.pdf}{PrataAN23}~\cite{PrataAN23}, \href{../works/LiLZDZW24.pdf}{LiLZDZW24}~\cite{LiLZDZW24}, \href{../works/BonninMNE24.pdf}{BonninMNE24}~\cite{BonninMNE24}, \href{../works/ForbesHJST24.pdf}{ForbesHJST24}~\cite{ForbesHJST24}, \href{../works/Adelgren2023.pdf}{Adelgren2023}~\cite{Adelgren2023}, \href{../works/AbreuPNF23.pdf}{AbreuPNF23}~\cite{AbreuPNF23}, \href{../works/IsikYA23.pdf}{IsikYA23}~\cite{IsikYA23}, \href{../works/NaderiBZR23.pdf}{NaderiBZR23}~\cite{NaderiBZR23}, \href{../works/EfthymiouY23.pdf}{EfthymiouY23}~\cite{EfthymiouY23}, \href{../works/PovedaAA23.pdf}{PovedaAA23}~\cite{PovedaAA23}, \href{../works/LuoB22.pdf}{LuoB22}~\cite{LuoB22}, \href{../works/GeitzGSSW22.pdf}{GeitzGSSW22}~\cite{GeitzGSSW22}, \href{../works/AwadMDMT22.pdf}{AwadMDMT22}~\cite{AwadMDMT22}, \href{../works/NaderiBZ22.pdf}{NaderiBZ22}~\cite{NaderiBZ22}, \href{../works/YunusogluY22.pdf}{YunusogluY22}~\cite{YunusogluY22}, \href{../works/CampeauG22.pdf}{CampeauG22}~\cite{CampeauG22}, \href{../works/NaqviAIAAA22.pdf}{NaqviAIAAA22}~\cite{NaqviAIAAA22}, \href{../works/YuraszeckMPV22.pdf}{YuraszeckMPV22}~\cite{YuraszeckMPV22}, \href{../works/ColT22.pdf}{ColT22}~\cite{ColT22}, \href{../works/Teppan22.pdf}{Teppan22}~\cite{Teppan22}, \href{../works/BoudreaultSLQ22.pdf}{BoudreaultSLQ22}~\cite{BoudreaultSLQ22}, \href{../works/ElciOH22.pdf}{ElciOH22}~\cite{ElciOH22}, \href{../works/Godet21a.pdf}{Godet21a}~\cite{Godet21a}, \href{../works/Bedhief21.pdf}{Bedhief21}~\cite{Bedhief21}, \href{../works/Alaka21.pdf}{Alaka21}~\cite{Alaka21}, \href{../works/abs-2102-08778.pdf}{abs-2102-08778}~\cite{abs-2102-08778}, \href{../works/GeibingerMM21.pdf}{GeibingerMM21}~\cite{GeibingerMM21}, \href{../works/Groleaz21.pdf}{Groleaz21}~\cite{Groleaz21}, \href{../works/CauwelaertDS20.pdf}{CauwelaertDS20}~\cite{CauwelaertDS20}... (Total: 124)\\
\index{real-world}\index{Benchmarks!real-world}Benchmarks & real-world & \href{../works/LuZZYW24.pdf}{LuZZYW24}~\cite{LuZZYW24}, \href{../works/GokPTGO23.pdf}{GokPTGO23}~\cite{GokPTGO23}, \href{../works/abs-2305-19888.pdf}{abs-2305-19888}~\cite{abs-2305-19888}, \href{../works/HeinzNVH22.pdf}{HeinzNVH22}~\cite{HeinzNVH22}, \href{../works/YunusogluY22.pdf}{YunusogluY22}~\cite{YunusogluY22}, \href{../works/ColT22.pdf}{ColT22}~\cite{ColT22}, \href{../works/GeibingerMM21.pdf}{GeibingerMM21}~\cite{GeibingerMM21}, \href{../works/KoehlerBFFHPSSS21.pdf}{KoehlerBFFHPSSS21}~\cite{KoehlerBFFHPSSS21}, \href{../works/Lemos21.pdf}{Lemos21}~\cite{Lemos21}, \href{../works/Astrand21.pdf}{Astrand21}~\cite{Astrand21}, \href{../works/Lunardi20.pdf}{Lunardi20}~\cite{Lunardi20}, \href{../works/MokhtarzadehTNF20.pdf}{MokhtarzadehTNF20}~\cite{MokhtarzadehTNF20}, \href{../works/HauderBRPA20.pdf}{HauderBRPA20}~\cite{HauderBRPA20}, \href{../works/abs-1911-04766.pdf}{abs-1911-04766}~\cite{abs-1911-04766}, \href{../works/GeibingerMM19.pdf}{GeibingerMM19}~\cite{GeibingerMM19}, \href{../works/abs-1902-09244.pdf}{abs-1902-09244}~\cite{abs-1902-09244}, \href{../works/FrohnerTR19.pdf}{FrohnerTR19}~\cite{FrohnerTR19}, \href{../works/SenderovichBB19.pdf}{SenderovichBB19}~\cite{SenderovichBB19}, \href{../works/GombolayWS18.pdf}{GombolayWS18}~\cite{GombolayWS18}, \href{../works/AgussurjaKL18.pdf}{AgussurjaKL18}~\cite{AgussurjaKL18}, \href{../works/Dejemeppe16.pdf}{Dejemeppe16}~\cite{Dejemeppe16}, \href{../works/MelgarejoLS15.pdf}{MelgarejoLS15}~\cite{MelgarejoLS15}, \href{../works/EvenSH15a.pdf}{EvenSH15a}~\cite{EvenSH15a}, \href{../works/EvenSH15.pdf}{EvenSH15}~\cite{EvenSH15}, \href{../works/UnsalO13.pdf}{UnsalO13}~\cite{UnsalO13}, \href{../works/RendlPHPR12.pdf}{RendlPHPR12}~\cite{RendlPHPR12}, \href{../works/Lombardi10.pdf}{Lombardi10}~\cite{Lombardi10}, \href{../works/MouraSCL08a.pdf}{MouraSCL08a}~\cite{MouraSCL08a}, \href{../works/WatsonBHW99.pdf}{WatsonBHW99}~\cite{WatsonBHW99}... (Total: 32) & \href{../works/PrataAN23.pdf}{PrataAN23}~\cite{PrataAN23}, \href{../works/TasselGS23.pdf}{TasselGS23}~\cite{TasselGS23}, \href{../works/WessenCSFPM23.pdf}{WessenCSFPM23}~\cite{WessenCSFPM23}, \href{../works/abs-2306-05747.pdf}{abs-2306-05747}~\cite{abs-2306-05747}, \href{../works/AbreuNP23.pdf}{AbreuNP23}~\cite{AbreuNP23}, \href{../works/BofillCGGPSV23.pdf}{BofillCGGPSV23}~\cite{BofillCGGPSV23}, \href{../works/IsikYA23.pdf}{IsikYA23}~\cite{IsikYA23}, \href{../works/Fatemi-AnarakiTFV23.pdf}{Fatemi-AnarakiTFV23}~\cite{Fatemi-AnarakiTFV23}, \href{../works/AalianPG23.pdf}{AalianPG23}~\cite{AalianPG23}, \href{../works/AbreuPNF23.pdf}{AbreuPNF23}~\cite{AbreuPNF23}, \href{../works/WangB23.pdf}{WangB23}~\cite{WangB23}, \href{../works/YuraszeckMCCR23.pdf}{YuraszeckMCCR23}~\cite{YuraszeckMCCR23}, \href{../works/OujanaAYB22.pdf}{OujanaAYB22}~\cite{OujanaAYB22}, \href{../works/LuoB22.pdf}{LuoB22}~\cite{LuoB22}, \href{../works/SvancaraB22.pdf}{SvancaraB22}~\cite{SvancaraB22}, \href{../works/MullerMKP22.pdf}{MullerMKP22}~\cite{MullerMKP22}, \href{../works/ArmstrongGOS21.pdf}{ArmstrongGOS21}~\cite{ArmstrongGOS21}, \href{../works/ZarandiASC20.pdf}{ZarandiASC20}~\cite{ZarandiASC20}, \href{../works/WallaceY20.pdf}{WallaceY20}~\cite{WallaceY20}, \href{../works/AntunesABD20.pdf}{AntunesABD20}~\cite{AntunesABD20}, \href{../works/RoshanaeiBAUB20.pdf}{RoshanaeiBAUB20}~\cite{RoshanaeiBAUB20}, \href{../works/AstrandJZ20.pdf}{AstrandJZ20}~\cite{AstrandJZ20}, \href{../works/WessenCS20.pdf}{WessenCS20}~\cite{WessenCS20}, \href{../works/TangB20.pdf}{TangB20}~\cite{TangB20}, \href{../works/ParkUJR19.pdf}{ParkUJR19}~\cite{ParkUJR19}, \href{../works/YounespourAKE19.pdf}{YounespourAKE19}~\cite{YounespourAKE19}, \href{../works/FrimodigS19.pdf}{FrimodigS19}~\cite{FrimodigS19}, \href{../works/PourDERB18.pdf}{PourDERB18}~\cite{PourDERB18}, \href{../works/HoYCLLCLC18.pdf}{HoYCLLCLC18}~\cite{HoYCLLCLC18}... (Total: 56) & \href{../works/FalqueALM24.pdf}{FalqueALM24}~\cite{FalqueALM24}, \href{../works/abs-2402-00459.pdf}{abs-2402-00459}~\cite{abs-2402-00459}, \href{../works/ZhuSZW23.pdf}{ZhuSZW23}~\cite{ZhuSZW23}, \href{../works/GuoZ23.pdf}{GuoZ23}~\cite{GuoZ23}, \href{../works/IklassovMR023.pdf}{IklassovMR023}~\cite{IklassovMR023}, \href{../works/PovedaAA23.pdf}{PovedaAA23}~\cite{PovedaAA23}, \href{../works/Bit-Monnot23.pdf}{Bit-Monnot23}~\cite{Bit-Monnot23}, \href{../works/TardivoDFMP23.pdf}{TardivoDFMP23}~\cite{TardivoDFMP23}, \href{../works/CzerniachowskaWZ23.pdf}{CzerniachowskaWZ23}~\cite{CzerniachowskaWZ23}, \href{../works/FrimodigECM23.pdf}{FrimodigECM23}~\cite{FrimodigECM23}, \href{../works/abs-2312-13682.pdf}{abs-2312-13682}~\cite{abs-2312-13682}, \href{../works/KimCMLLP23.pdf}{KimCMLLP23}~\cite{KimCMLLP23}, \href{../works/NaderiBZR23.pdf}{NaderiBZR23}~\cite{NaderiBZR23}, \href{../works/JuvinHL23.pdf}{JuvinHL23}~\cite{JuvinHL23}, \href{../works/PerezGSL23.pdf}{PerezGSL23}~\cite{PerezGSL23}, \href{../works/NaderiBZ23.pdf}{NaderiBZ23}~\cite{NaderiBZ23}, \href{../works/ShaikhK23.pdf}{ShaikhK23}~\cite{ShaikhK23}, \href{../works/AfsarVPG23.pdf}{AfsarVPG23}~\cite{AfsarVPG23}, \href{../works/MarliereSPR23.pdf}{MarliereSPR23}~\cite{MarliereSPR23}, \href{../works/GeitzGSSW22.pdf}{GeitzGSSW22}~\cite{GeitzGSSW22}, \href{../works/SubulanC22.pdf}{SubulanC22}~\cite{SubulanC22}, \href{../works/BourreauGGLT22.pdf}{BourreauGGLT22}~\cite{BourreauGGLT22}, \href{../works/JungblutK22.pdf}{JungblutK22}~\cite{JungblutK22}, \href{../works/AbreuN22.pdf}{AbreuN22}~\cite{AbreuN22}, \href{../works/FetgoD22.pdf}{FetgoD22}~\cite{FetgoD22}, \href{../works/BoudreaultSLQ22.pdf}{BoudreaultSLQ22}~\cite{BoudreaultSLQ22}, \href{../works/OrnekOS20.pdf}{OrnekOS20}~\cite{OrnekOS20}, \href{../works/Tassel22.pdf}{Tassel22}~\cite{Tassel22}, \href{../works/CampeauG22.pdf}{CampeauG22}~\cite{CampeauG22}... (Total: 177)\\
\index{supplementary material}\index{Benchmarks!supplementary material}Benchmarks & supplementary material & \href{../works/GuoZ23.pdf}{GuoZ23}~\cite{GuoZ23}, \href{../works/FarsiTM22.pdf}{FarsiTM22}~\cite{FarsiTM22}, \href{../works/MejiaY20.pdf}{MejiaY20}~\cite{MejiaY20}, \href{../works/Lunardi20.pdf}{Lunardi20}~\cite{Lunardi20}, \href{../works/TanZWGQ19.pdf}{TanZWGQ19}~\cite{TanZWGQ19} & \href{../works/NaderiBZR23.pdf}{NaderiBZR23}~\cite{NaderiBZR23}, \href{../works/MontemanniD23.pdf}{MontemanniD23}~\cite{MontemanniD23}, \href{../works/AfsarVPG23.pdf}{AfsarVPG23}~\cite{AfsarVPG23}, \href{../works/FachiniA20.pdf}{FachiniA20}~\cite{FachiniA20}, \href{../works/SchuttFSW13.pdf}{SchuttFSW13}~\cite{SchuttFSW13} & \href{../works/FalqueALM24.pdf}{FalqueALM24}~\cite{FalqueALM24}, \href{../works/JuvinHHL23.pdf}{JuvinHHL23}~\cite{JuvinHHL23}, \href{../works/abs-2306-05747.pdf}{abs-2306-05747}~\cite{abs-2306-05747}, \href{../works/TasselGS23.pdf}{TasselGS23}~\cite{TasselGS23}, \href{../works/Adelgren2023.pdf}{Adelgren2023}~\cite{Adelgren2023}, \href{../works/WinterMMW22.pdf}{WinterMMW22}~\cite{WinterMMW22}, \href{../works/ColT22.pdf}{ColT22}~\cite{ColT22}, \href{../works/BoudreaultSLQ22.pdf}{BoudreaultSLQ22}~\cite{BoudreaultSLQ22}, \href{../works/YunusogluY22.pdf}{YunusogluY22}~\cite{YunusogluY22}, \href{../works/AntuoriHHEN21.pdf}{AntuoriHHEN21}~\cite{AntuoriHHEN21}, \href{../works/LacknerMMWW21.pdf}{LacknerMMWW21}~\cite{LacknerMMWW21}, \href{../works/KovacsTKSG21.pdf}{KovacsTKSG21}~\cite{KovacsTKSG21}, \href{../works/ArmstrongGOS21.pdf}{ArmstrongGOS21}~\cite{ArmstrongGOS21}, \href{../works/MengZRZL20.pdf}{MengZRZL20}~\cite{MengZRZL20}, \href{../works/HauderBRPA20.pdf}{HauderBRPA20}~\cite{HauderBRPA20}, \href{../works/SchnellH17.pdf}{SchnellH17}~\cite{SchnellH17}, \href{../works/SchnellH15.pdf}{SchnellH15}~\cite{SchnellH15}, \href{../works/MenciaSV13.pdf}{MenciaSV13}~\cite{MenciaSV13}\\
\index{zenodo}\index{Benchmarks!zenodo}Benchmarks & zenodo & \href{../works/LacknerMMWW23.pdf}{LacknerMMWW23}~\cite{LacknerMMWW23}, \href{../works/SacramentoSP20.pdf}{SacramentoSP20}~\cite{SacramentoSP20} &  & \href{../works/KimCMLLP23.pdf}{KimCMLLP23}~\cite{KimCMLLP23}, \href{../works/WinterMMW22.pdf}{WinterMMW22}~\cite{WinterMMW22}, \href{../works/ArmstrongGOS21.pdf}{ArmstrongGOS21}~\cite{ArmstrongGOS21}\\
\end{longtable}
}


\clearpage
\subsection{Concept Type Algorithms}
\label{sec:Algorithms}
{\scriptsize
\begin{longtable}{lp{3cm}>{\raggedright\arraybackslash}p{6cm}>{\raggedright\arraybackslash}p{6cm}>{\raggedright\arraybackslash}p{8cm}}
\rowcolor{white}\caption{Works for Concepts of Type Algorithms}\\ \toprule
\rowcolor{white}Type & Keyword & High & Medium & Low\\ \midrule\endhead
\bottomrule
\endfoot
Algorithms & GRASP & \href{../works/Lemos21.pdf}{Lemos21}~\cite{Lemos21} & \href{../works/YuraszeckMCCR23.pdf}{YuraszeckMCCR23}~\cite{YuraszeckMCCR23}, \href{../works/PovedaAA23.pdf}{PovedaAA23}~\cite{PovedaAA23}, \href{../works/YunusogluY22.pdf}{YunusogluY22}~\cite{YunusogluY22}, \href{../works/RiahiNS018.pdf}{RiahiNS018}~\cite{RiahiNS018} & \href{../works/LacknerMMWW23.pdf}{LacknerMMWW23}~\cite{LacknerMMWW23}, \href{../works/AkramNHRSA23.pdf}{AkramNHRSA23}~\cite{AkramNHRSA23}, \href{../works/IsikYA23.pdf}{IsikYA23}~\cite{IsikYA23}, \href{../works/SquillaciPR23.pdf}{SquillaciPR23}~\cite{SquillaciPR23}, \href{../works/ArmstrongGOS22.pdf}{ArmstrongGOS22}~\cite{ArmstrongGOS22}, \href{../works/LacknerMMWW21.pdf}{LacknerMMWW21}~\cite{LacknerMMWW21}, \href{../works/Zahout21.pdf}{Zahout21}~\cite{Zahout21}, \href{../works/VlkHT21.pdf}{VlkHT21}~\cite{VlkHT21}, \href{../works/AntuoriHHEN21.pdf}{AntuoriHHEN21}~\cite{AntuoriHHEN21}, \href{../works/GokGSTO20.pdf}{GokGSTO20}~\cite{GokGSTO20}, \href{../works/QinDCS20.pdf}{QinDCS20}~\cite{QinDCS20}, \href{../works/MejiaY20.pdf}{MejiaY20}~\cite{MejiaY20}, \href{../works/GroleazNS20a.pdf}{GroleazNS20a}~\cite{GroleazNS20a}, \href{../works/Caballero19.pdf}{Caballero19}~\cite{Caballero19}, \href{../works/KreterSSZ18.pdf}{KreterSSZ18}~\cite{KreterSSZ18}, \href{../works/ZhouGL15.pdf}{ZhouGL15}~\cite{ZhouGL15}, \href{../works/Siala15.pdf}{Siala15}~\cite{Siala15}, \href{../works/Siala15a.pdf}{Siala15a}~\cite{Siala15a}, \href{../works/SchnellH15.pdf}{SchnellH15}~\cite{SchnellH15}, \href{../works/SerraNM12.pdf}{SerraNM12}~\cite{SerraNM12}, \href{../works/HeinzB12.pdf}{HeinzB12}~\cite{HeinzB12}, \href{../works/Rodriguez07.pdf}{Rodriguez07}~\cite{Rodriguez07}, \href{../works/JainM99.pdf}{JainM99}~\cite{JainM99}\\
Algorithms & IGT & \href{../works/ArmstrongGOS22.pdf}{ArmstrongGOS22}~\cite{ArmstrongGOS22} &  & \\
Algorithms & NEH & \href{../works/AlfieriGPS23.pdf}{AlfieriGPS23}~\cite{AlfieriGPS23}, \href{../works/ArmstrongGOS22.pdf}{ArmstrongGOS22}~\cite{ArmstrongGOS22}, \href{../works/Astrand21.pdf}{Astrand21}~\cite{Astrand21}, \href{../works/RiahiNS018.pdf}{RiahiNS018}~\cite{RiahiNS018} &  & \href{../works/AbreuPNF23.pdf}{AbreuPNF23}~\cite{AbreuPNF23}, \href{../works/IsikYA23.pdf}{IsikYA23}~\cite{IsikYA23}, \href{../works/ZhouGL15.pdf}{ZhouGL15}~\cite{ZhouGL15}\\
Algorithms & bi-partite matching &  &  & \href{../works/Caballero19.pdf}{Caballero19}~\cite{Caballero19}, \href{../works/HookerH17.pdf}{HookerH17}~\cite{HookerH17}, \href{../works/Simonis07.pdf}{Simonis07}~\cite{Simonis07}, \href{../works/Kumar03.pdf}{Kumar03}~\cite{Kumar03}, \href{../works/Simonis99.pdf}{Simonis99}~\cite{Simonis99}\\
Algorithms & edge-finder & \href{../works/KameugneFND23.pdf}{KameugneFND23}~\cite{KameugneFND23}, \href{../works/FetgoD22.pdf}{FetgoD22}~\cite{FetgoD22}, \href{../works/GingrasQ16.pdf}{GingrasQ16}~\cite{GingrasQ16}, \href{../works/KameugneFSN14.pdf}{KameugneFSN14}~\cite{KameugneFSN14}, \href{../works/Lombardi10.pdf}{Lombardi10}~\cite{Lombardi10}, \href{../works/MercierH08.pdf}{MercierH08}~\cite{MercierH08}, \href{../works/BaptisteP00.pdf}{BaptisteP00}~\cite{BaptisteP00} & \href{../works/OuelletQ13.pdf}{OuelletQ13}~\cite{OuelletQ13}, \href{../works/KelbelH11.pdf}{KelbelH11}~\cite{KelbelH11}, \href{../works/PapaB98.pdf}{PapaB98}~\cite{PapaB98} & \href{../works/BaptisteB18.pdf}{BaptisteB18}~\cite{BaptisteB18}, \href{../works/BonfiettiZLM16.pdf}{BonfiettiZLM16}~\cite{BonfiettiZLM16}, \href{../works/Kameugne14.pdf}{Kameugne14}~\cite{Kameugne14}, \href{../works/GuSS13.pdf}{GuSS13}~\cite{GuSS13}, \href{../works/Schutt11.pdf}{Schutt11}~\cite{Schutt11}, \href{../works/SchuttFSW11.pdf}{SchuttFSW11}~\cite{SchuttFSW11}, \href{../works/HeckmanB11.pdf}{HeckmanB11}~\cite{HeckmanB11}, \href{../works/BidotVLB09.pdf}{BidotVLB09}~\cite{BidotVLB09}, \href{../works/MilanoW09.pdf}{MilanoW09}~\cite{MilanoW09}, \href{../works/SchuttFSW09.pdf}{SchuttFSW09}~\cite{SchuttFSW09}, \href{../works/BeckW07.pdf}{BeckW07}~\cite{BeckW07}, \href{../works/MilanoW06.pdf}{MilanoW06}~\cite{MilanoW06}, \href{../works/BeckW05.pdf}{BeckW05}~\cite{BeckW05}, \href{../works/BeckR03.pdf}{BeckR03}~\cite{BeckR03}, \href{../works/ValleMGT03.pdf}{ValleMGT03}~\cite{ValleMGT03}, \href{../works/SakkoutW00.pdf}{SakkoutW00}~\cite{SakkoutW00}, \href{../works/JainM99.pdf}{JainM99}~\cite{JainM99}, \href{../works/Zhou97.pdf}{Zhou97}~\cite{Zhou97}, \href{../works/BaptisteP97.pdf}{BaptisteP97}~\cite{BaptisteP97}\\
Algorithms & edge-finding & \href{../works/KameugneFND23.pdf}{KameugneFND23}~\cite{KameugneFND23}, \href{../works/JuvinHHL23.pdf}{JuvinHHL23}~\cite{JuvinHHL23}, \href{../works/TardivoDFMP23.pdf}{TardivoDFMP23}~\cite{TardivoDFMP23}, \href{../works/OuelletQ22.pdf}{OuelletQ22}~\cite{OuelletQ22}, \href{../works/FetgoD22.pdf}{FetgoD22}~\cite{FetgoD22}, \href{../works/CauwelaertDS20.pdf}{CauwelaertDS20}~\cite{CauwelaertDS20}, \href{../works/YangSS19.pdf}{YangSS19}~\cite{YangSS19}, \href{../works/Caballero19.pdf}{Caballero19}~\cite{Caballero19}, \href{../works/GokgurHO18.pdf}{GokgurHO18}~\cite{GokgurHO18}, \href{../works/FahimiOQ18.pdf}{FahimiOQ18}~\cite{FahimiOQ18}, \href{../works/BaptisteB18.pdf}{BaptisteB18}~\cite{BaptisteB18}, \href{../works/KreterSS17.pdf}{KreterSS17}~\cite{KreterSS17}, \href{../works/HookerH17.pdf}{HookerH17}~\cite{HookerH17}, \href{../works/Fahimi16.pdf}{Fahimi16}~\cite{Fahimi16}, \href{../works/Nattaf16.pdf}{Nattaf16}~\cite{Nattaf16}, \href{../works/Dejemeppe16.pdf}{Dejemeppe16}~\cite{Dejemeppe16}, \href{../works/Derrien15.pdf}{Derrien15}~\cite{Derrien15}, \href{../works/GayHS15a.pdf}{GayHS15a}~\cite{GayHS15a}, \href{../works/Kameugne15.pdf}{Kameugne15}~\cite{Kameugne15}, \href{../works/GrimesH15.pdf}{GrimesH15}~\cite{GrimesH15}, \href{../works/KameugneFSN14.pdf}{KameugneFSN14}~\cite{KameugneFSN14}, \href{../works/Kameugne14.pdf}{Kameugne14}~\cite{Kameugne14}, \href{../works/Letort13.pdf}{Letort13}~\cite{Letort13}, \href{../works/OuelletQ13.pdf}{OuelletQ13}~\cite{OuelletQ13}, \href{../works/SchuttFS13a.pdf}{SchuttFS13a}~\cite{SchuttFS13a}, \href{../works/Clercq12.pdf}{Clercq12}~\cite{Clercq12}, \href{../works/Malapert11.pdf}{Malapert11}~\cite{Malapert11}, \href{../works/KameugneFSN11.pdf}{KameugneFSN11}~\cite{KameugneFSN11}, \href{../works/Vilim11.pdf}{Vilim11}~\cite{Vilim11}... (Total: 50) & \href{../works/BoudreaultSLQ22.pdf}{BoudreaultSLQ22}~\cite{BoudreaultSLQ22}, \href{../works/LaborieRSV18.pdf}{LaborieRSV18}~\cite{LaborieRSV18}, \href{../works/Tesch18.pdf}{Tesch18}~\cite{Tesch18}, \href{../works/GingrasQ16.pdf}{GingrasQ16}~\cite{GingrasQ16}, \href{../works/CauwelaertDMS16.pdf}{CauwelaertDMS16}~\cite{CauwelaertDMS16}, \href{../works/LetortCB15.pdf}{LetortCB15}~\cite{LetortCB15}, \href{../works/DejemeppeCS15.pdf}{DejemeppeCS15}~\cite{DejemeppeCS15}, \href{../works/Siala15a.pdf}{Siala15a}~\cite{Siala15a}, \href{../works/Siala15.pdf}{Siala15}~\cite{Siala15}, \href{../works/MenciaSV13.pdf}{MenciaSV13}~\cite{MenciaSV13}, \href{../works/LetortCB13.pdf}{LetortCB13}~\cite{LetortCB13}, \href{../works/LetortBC12.pdf}{LetortBC12}~\cite{LetortBC12}, \href{../works/LombardiM12.pdf}{LombardiM12}~\cite{LombardiM12}, \href{../works/Lombardi10.pdf}{Lombardi10}~\cite{Lombardi10}, \href{../works/BartakSR10.pdf}{BartakSR10}~\cite{BartakSR10}, \href{../works/LiessM08.pdf}{LiessM08}~\cite{LiessM08}, \href{../works/HoeveGSL07.pdf}{HoeveGSL07}~\cite{HoeveGSL07}, \href{../works/MonetteDD07.pdf}{MonetteDD07}~\cite{MonetteDD07}, \href{../works/Vilim04.pdf}{Vilim04}~\cite{Vilim04}, \href{../works/Bartak02.pdf}{Bartak02}~\cite{Bartak02}, \href{../works/SchildW00.pdf}{SchildW00}~\cite{SchildW00}, \href{../works/Zhou97.pdf}{Zhou97}~\cite{Zhou97} & \href{../works/BonninMNE24.pdf}{BonninMNE24}~\cite{BonninMNE24}, \href{../works/CampeauG22.pdf}{CampeauG22}~\cite{CampeauG22}, \href{../works/Groleaz21.pdf}{Groleaz21}~\cite{Groleaz21}, \href{../works/Astrand21.pdf}{Astrand21}~\cite{Astrand21}, \href{../works/Godet21a.pdf}{Godet21a}~\cite{Godet21a}, \href{../works/WallaceY20.pdf}{WallaceY20}~\cite{WallaceY20}, \href{../works/OuelletQ18.pdf}{OuelletQ18}~\cite{OuelletQ18}, \href{../works/GombolayWS18.pdf}{GombolayWS18}~\cite{GombolayWS18}, \href{../works/CauwelaertLS18.pdf}{CauwelaertLS18}~\cite{CauwelaertLS18}, \href{../works/NattafAL17.pdf}{NattafAL17}~\cite{NattafAL17}, \href{../works/Tesch16.pdf}{Tesch16}~\cite{Tesch16}, \href{../works/OrnekO16.pdf}{OrnekO16}~\cite{OrnekO16}, \href{../works/SialaAH15.pdf}{SialaAH15}~\cite{SialaAH15}, \href{../works/GayHLS15.pdf}{GayHLS15}~\cite{GayHLS15}, \href{../works/DerrienP14.pdf}{DerrienP14}~\cite{DerrienP14}, \href{../works/GuSS13.pdf}{GuSS13}~\cite{GuSS13}, \href{../works/HeinzSB13.pdf}{HeinzSB13}~\cite{HeinzSB13}, \href{../works/OzturkTHO13.pdf}{OzturkTHO13}~\cite{OzturkTHO13}, \href{../works/ChuGNSW13.pdf}{ChuGNSW13}~\cite{ChuGNSW13}, \href{../works/MenciaSV12.pdf}{MenciaSV12}~\cite{MenciaSV12}, \href{../works/LimtanyakulS12.pdf}{LimtanyakulS12}~\cite{LimtanyakulS12}, \href{../works/MalapertCGJLR12.pdf}{MalapertCGJLR12}~\cite{MalapertCGJLR12}, \href{../works/OzturkTHO12.pdf}{OzturkTHO12}~\cite{OzturkTHO12}, \href{../works/HeckmanB11.pdf}{HeckmanB11}~\cite{HeckmanB11}, \href{../works/KovacsB11.pdf}{KovacsB11}~\cite{KovacsB11}, \href{../works/SimonisH11.pdf}{SimonisH11}~\cite{SimonisH11}, \href{../works/BeldiceanuCDP11.pdf}{BeldiceanuCDP11}~\cite{BeldiceanuCDP11}, \href{../works/KelbelH11.pdf}{KelbelH11}~\cite{KelbelH11}, \href{../works/GrimesH11.pdf}{GrimesH11}~\cite{GrimesH11}... (Total: 60)\\
Algorithms & energetic reasoning & \href{../works/TardivoDFMP23.pdf}{TardivoDFMP23}~\cite{TardivoDFMP23}, \href{../works/OuelletQ22.pdf}{OuelletQ22}~\cite{OuelletQ22}, \href{../works/FetgoD22.pdf}{FetgoD22}~\cite{FetgoD22}, \href{../works/HanenKP21.pdf}{HanenKP21}~\cite{HanenKP21}, \href{../works/OuelletQ18.pdf}{OuelletQ18}~\cite{OuelletQ18}, \href{../works/Tesch18.pdf}{Tesch18}~\cite{Tesch18}, \href{../works/CauwelaertLS18.pdf}{CauwelaertLS18}~\cite{CauwelaertLS18}, \href{../works/NattafAL17.pdf}{NattafAL17}~\cite{NattafAL17}, \href{../works/NattafALR16.pdf}{NattafALR16}~\cite{NattafALR16}, \href{../works/Fahimi16.pdf}{Fahimi16}~\cite{Fahimi16}, \href{../works/Tesch16.pdf}{Tesch16}~\cite{Tesch16}, \href{../works/GayHS15a.pdf}{GayHS15a}~\cite{GayHS15a}, \href{../works/NattafAL15.pdf}{NattafAL15}~\cite{NattafAL15}, \href{../works/DerrienP14.pdf}{DerrienP14}~\cite{DerrienP14}, \href{../works/SchuttFS13a.pdf}{SchuttFS13a}~\cite{SchuttFS13a}, \href{../works/LimtanyakulS12.pdf}{LimtanyakulS12}~\cite{LimtanyakulS12}, \href{../works/HeinzS11.pdf}{HeinzS11}~\cite{HeinzS11}, \href{../works/Vilim11.pdf}{Vilim11}~\cite{Vilim11}, \href{../works/Lombardi10.pdf}{Lombardi10}~\cite{Lombardi10}, \href{../works/Laborie03.pdf}{Laborie03}~\cite{Laborie03}, \href{../works/Baptiste02.pdf}{Baptiste02}~\cite{Baptiste02} & \href{../works/KameugneFND23.pdf}{KameugneFND23}~\cite{KameugneFND23}, \href{../works/NattafHKAL19.pdf}{NattafHKAL19}~\cite{NattafHKAL19}, \href{../works/KameugneFGOQ18.pdf}{KameugneFGOQ18}~\cite{KameugneFGOQ18}, \href{../works/Nattaf16.pdf}{Nattaf16}~\cite{Nattaf16}, \href{../works/Kameugne14.pdf}{Kameugne14}~\cite{Kameugne14}, \href{../works/Letort13.pdf}{Letort13}~\cite{Letort13}, \href{../works/SchuttFS13.pdf}{SchuttFS13}~\cite{SchuttFS13}, \href{../works/Schutt11.pdf}{Schutt11}~\cite{Schutt11} & \href{../works/IsikYA23.pdf}{IsikYA23}~\cite{IsikYA23}, \href{../works/BoudreaultSLQ22.pdf}{BoudreaultSLQ22}~\cite{BoudreaultSLQ22}, \href{../works/ArmstrongGOS21.pdf}{ArmstrongGOS21}~\cite{ArmstrongGOS21}, \href{../works/Caballero19.pdf}{Caballero19}~\cite{Caballero19}, \href{../works/YangSS19.pdf}{YangSS19}~\cite{YangSS19}, \href{../works/GokgurHO18.pdf}{GokgurHO18}~\cite{GokgurHO18}, \href{../works/Laborie18a.pdf}{Laborie18a}~\cite{Laborie18a}, \href{../works/BofillCSV17.pdf}{BofillCSV17}~\cite{BofillCSV17}, \href{../works/HookerH17.pdf}{HookerH17}~\cite{HookerH17}, \href{../works/GingrasQ16.pdf}{GingrasQ16}~\cite{GingrasQ16}, \href{../works/LetortCB15.pdf}{LetortCB15}~\cite{LetortCB15}, \href{../works/Derrien15.pdf}{Derrien15}~\cite{Derrien15}, \href{../works/KameugneFSN14.pdf}{KameugneFSN14}~\cite{KameugneFSN14}, \href{../works/LetortCB13.pdf}{LetortCB13}~\cite{LetortCB13}, \href{../works/OuelletQ13.pdf}{OuelletQ13}~\cite{OuelletQ13}, \href{../works/MenciaSV13.pdf}{MenciaSV13}~\cite{MenciaSV13}, \href{../works/Clercq12.pdf}{Clercq12}~\cite{Clercq12}, \href{../works/LombardiM12.pdf}{LombardiM12}~\cite{LombardiM12}, \href{../works/MenciaSV12.pdf}{MenciaSV12}~\cite{MenciaSV12}, \href{../works/GuyonLPR12.pdf}{GuyonLPR12}~\cite{GuyonLPR12}, \href{../works/LahimerLH11.pdf}{LahimerLH11}~\cite{LahimerLH11}, \href{../works/Malapert11.pdf}{Malapert11}~\cite{Malapert11}, \href{../works/ClercqPBJ11.pdf}{ClercqPBJ11}~\cite{ClercqPBJ11}, \href{../works/BeldiceanuCDP11.pdf}{BeldiceanuCDP11}~\cite{BeldiceanuCDP11}, \href{../works/ChenGPSH10.pdf}{ChenGPSH10}~\cite{ChenGPSH10}, \href{../works/abs-0907-0939.pdf}{abs-0907-0939}~\cite{abs-0907-0939}, \href{../works/Vilim09.pdf}{Vilim09}~\cite{Vilim09}, \href{../works/Vilim09a.pdf}{Vilim09a}~\cite{Vilim09a}, \href{../works/Limtanyakul07.pdf}{Limtanyakul07}~\cite{Limtanyakul07}... (Total: 35)\\
Algorithms & max-flow &  & \href{../works/LopesCSM10.pdf}{LopesCSM10}~\cite{LopesCSM10}, \href{../works/MouraSCL08.pdf}{MouraSCL08}~\cite{MouraSCL08}, \href{../works/Muscettola02.pdf}{Muscettola02}~\cite{Muscettola02} & \href{../works/FanXG21.pdf}{FanXG21}~\cite{FanXG21}, \href{../works/ZarandiASC20.pdf}{ZarandiASC20}~\cite{ZarandiASC20}, \href{../works/HoundjiSW19.pdf}{HoundjiSW19}~\cite{HoundjiSW19}, \href{../works/Fahimi16.pdf}{Fahimi16}~\cite{Fahimi16}, \href{../works/Froger16.pdf}{Froger16}~\cite{Froger16}, \href{../works/Kumar03.pdf}{Kumar03}~\cite{Kumar03}\\
Algorithms & not-first & \href{../works/KameugneFND23.pdf}{KameugneFND23}~\cite{KameugneFND23}, \href{../works/FahimiOQ18.pdf}{FahimiOQ18}~\cite{FahimiOQ18}, \href{../works/KameugneFGOQ18.pdf}{KameugneFGOQ18}~\cite{KameugneFGOQ18}, \href{../works/Fahimi16.pdf}{Fahimi16}~\cite{Fahimi16}, \href{../works/Dejemeppe16.pdf}{Dejemeppe16}~\cite{Dejemeppe16}, \href{../works/GayHS15a.pdf}{GayHS15a}~\cite{GayHS15a}, \href{../works/Kameugne14.pdf}{Kameugne14}~\cite{Kameugne14}, \href{../works/Clercq12.pdf}{Clercq12}~\cite{Clercq12}, \href{../works/Schutt11.pdf}{Schutt11}~\cite{Schutt11}, \href{../works/Malapert11.pdf}{Malapert11}~\cite{Malapert11}, \href{../works/SchuttFSW11.pdf}{SchuttFSW11}~\cite{SchuttFSW11}, \href{../works/VilimBC05.pdf}{VilimBC05}~\cite{VilimBC05}, \href{../works/ArtiouchineB05.pdf}{ArtiouchineB05}~\cite{ArtiouchineB05}, \href{../works/Demassey03.pdf}{Demassey03}~\cite{Demassey03}, \href{../works/Baptiste02.pdf}{Baptiste02}~\cite{Baptiste02}, \href{../works/Beck99.pdf}{Beck99}~\cite{Beck99} & \href{../works/TardivoDFMP23.pdf}{TardivoDFMP23}~\cite{TardivoDFMP23}, \href{../works/FetgoD22.pdf}{FetgoD22}~\cite{FetgoD22}, \href{../works/GokgurHO18.pdf}{GokgurHO18}~\cite{GokgurHO18}, \href{../works/OuelletQ18.pdf}{OuelletQ18}~\cite{OuelletQ18}, \href{../works/HookerH17.pdf}{HookerH17}~\cite{HookerH17}, \href{../works/DejemeppeCS15.pdf}{DejemeppeCS15}~\cite{DejemeppeCS15}, \href{../works/Kameugne15.pdf}{Kameugne15}~\cite{Kameugne15}, \href{../works/KameugneFSN14.pdf}{KameugneFSN14}~\cite{KameugneFSN14}, \href{../works/Letort13.pdf}{Letort13}~\cite{Letort13}, \href{../works/OuelletQ13.pdf}{OuelletQ13}~\cite{OuelletQ13}, \href{../works/Lombardi10.pdf}{Lombardi10}~\cite{Lombardi10}, \href{../works/SchuttW10.pdf}{SchuttW10}~\cite{SchuttW10}, \href{../works/BartakSR10.pdf}{BartakSR10}~\cite{BartakSR10}, \href{../works/MonetteDD07.pdf}{MonetteDD07}~\cite{MonetteDD07}, \href{../works/VilimBC04.pdf}{VilimBC04}~\cite{VilimBC04}, \href{../works/Wolf03.pdf}{Wolf03}~\cite{Wolf03}, \href{../works/BeckF00.pdf}{BeckF00}~\cite{BeckF00}, \href{../works/TorresL00.pdf}{TorresL00}~\cite{TorresL00} & \href{../works/JuvinHHL23.pdf}{JuvinHHL23}~\cite{JuvinHHL23}, \href{../works/BoudreaultSLQ22.pdf}{BoudreaultSLQ22}~\cite{BoudreaultSLQ22}, \href{../works/OuelletQ22.pdf}{OuelletQ22}~\cite{OuelletQ22}, \href{../works/Astrand21.pdf}{Astrand21}~\cite{Astrand21}, \href{../works/Groleaz21.pdf}{Groleaz21}~\cite{Groleaz21}, \href{../works/CauwelaertDS20.pdf}{CauwelaertDS20}~\cite{CauwelaertDS20}, \href{../works/CauwelaertLS18.pdf}{CauwelaertLS18}~\cite{CauwelaertLS18}, \href{../works/Tesch16.pdf}{Tesch16}~\cite{Tesch16}, \href{../works/CauwelaertDMS16.pdf}{CauwelaertDMS16}~\cite{CauwelaertDMS16}, \href{../works/GrimesH15.pdf}{GrimesH15}~\cite{GrimesH15}, \href{../works/ChuGNSW13.pdf}{ChuGNSW13}~\cite{ChuGNSW13}, \href{../works/MalapertCGJLR12.pdf}{MalapertCGJLR12}~\cite{MalapertCGJLR12}, \href{../works/LimtanyakulS12.pdf}{LimtanyakulS12}~\cite{LimtanyakulS12}, \href{../works/KameugneFSN11.pdf}{KameugneFSN11}~\cite{KameugneFSN11}, \href{../works/Vilim09.pdf}{Vilim09}~\cite{Vilim09}, \href{../works/Wolf09.pdf}{Wolf09}~\cite{Wolf09}, \href{../works/Wolf05.pdf}{Wolf05}~\cite{Wolf05}, \href{../works/Laborie03.pdf}{Laborie03}~\cite{Laborie03}, \href{../works/SourdN00.pdf}{SourdN00}~\cite{SourdN00}\\
Algorithms & not-last & \href{../works/KameugneFND23.pdf}{KameugneFND23}~\cite{KameugneFND23}, \href{../works/TardivoDFMP23.pdf}{TardivoDFMP23}~\cite{TardivoDFMP23}, \href{../works/KameugneFGOQ18.pdf}{KameugneFGOQ18}~\cite{KameugneFGOQ18}, \href{../works/FahimiOQ18.pdf}{FahimiOQ18}~\cite{FahimiOQ18}, \href{../works/OuelletQ18.pdf}{OuelletQ18}~\cite{OuelletQ18}, \href{../works/Fahimi16.pdf}{Fahimi16}~\cite{Fahimi16}, \href{../works/Dejemeppe16.pdf}{Dejemeppe16}~\cite{Dejemeppe16}, \href{../works/GayHS15a.pdf}{GayHS15a}~\cite{GayHS15a}, \href{../works/Kameugne14.pdf}{Kameugne14}~\cite{Kameugne14}, \href{../works/Clercq12.pdf}{Clercq12}~\cite{Clercq12}, \href{../works/Malapert11.pdf}{Malapert11}~\cite{Malapert11}, \href{../works/Schutt11.pdf}{Schutt11}~\cite{Schutt11}, \href{../works/SchuttW10.pdf}{SchuttW10}~\cite{SchuttW10}, \href{../works/ArtiouchineB05.pdf}{ArtiouchineB05}~\cite{ArtiouchineB05}, \href{../works/SchuttWS05.pdf}{SchuttWS05}~\cite{SchuttWS05}, \href{../works/Vilim05.pdf}{Vilim05}~\cite{Vilim05}, \href{../works/VilimBC05.pdf}{VilimBC05}~\cite{VilimBC05}, \href{../works/Vilim04.pdf}{Vilim04}~\cite{Vilim04}, \href{../works/Wolf03.pdf}{Wolf03}~\cite{Wolf03}, \href{../works/Demassey03.pdf}{Demassey03}~\cite{Demassey03}, \href{../works/Baptiste02.pdf}{Baptiste02}~\cite{Baptiste02}, \href{../works/Beck99.pdf}{Beck99}~\cite{Beck99} & \href{../works/FetgoD22.pdf}{FetgoD22}~\cite{FetgoD22}, \href{../works/CauwelaertDS20.pdf}{CauwelaertDS20}~\cite{CauwelaertDS20}, \href{../works/GokgurHO18.pdf}{GokgurHO18}~\cite{GokgurHO18}, \href{../works/Tesch18.pdf}{Tesch18}~\cite{Tesch18}, \href{../works/Kameugne15.pdf}{Kameugne15}~\cite{Kameugne15}, \href{../works/DejemeppeCS15.pdf}{DejemeppeCS15}~\cite{DejemeppeCS15}, \href{../works/KameugneFSN14.pdf}{KameugneFSN14}~\cite{KameugneFSN14}, \href{../works/SchuttFS13a.pdf}{SchuttFS13a}~\cite{SchuttFS13a}, \href{../works/OuelletQ13.pdf}{OuelletQ13}~\cite{OuelletQ13}, \href{../works/Letort13.pdf}{Letort13}~\cite{Letort13}, \href{../works/SchuttFSW11.pdf}{SchuttFSW11}~\cite{SchuttFSW11}, \href{../works/Vilim11.pdf}{Vilim11}~\cite{Vilim11}, \href{../works/KameugneFSN11.pdf}{KameugneFSN11}~\cite{KameugneFSN11}, \href{../works/Lombardi10.pdf}{Lombardi10}~\cite{Lombardi10}, \href{../works/BartakSR10.pdf}{BartakSR10}~\cite{BartakSR10}, \href{../works/MonetteDD07.pdf}{MonetteDD07}~\cite{MonetteDD07}, \href{../works/Wolf05.pdf}{Wolf05}~\cite{Wolf05}, \href{../works/VilimBC04.pdf}{VilimBC04}~\cite{VilimBC04}, \href{../works/TorresL00.pdf}{TorresL00}~\cite{TorresL00}, \href{../works/BeckF00.pdf}{BeckF00}~\cite{BeckF00} & \href{../works/JuvinHHL23.pdf}{JuvinHHL23}~\cite{JuvinHHL23}, \href{../works/BoudreaultSLQ22.pdf}{BoudreaultSLQ22}~\cite{BoudreaultSLQ22}, \href{../works/GeitzGSSW22.pdf}{GeitzGSSW22}~\cite{GeitzGSSW22}, \href{../works/OuelletQ22.pdf}{OuelletQ22}~\cite{OuelletQ22}, \href{../works/Astrand21.pdf}{Astrand21}~\cite{Astrand21}, \href{../works/Groleaz21.pdf}{Groleaz21}~\cite{Groleaz21}, \href{../works/GodetLHS20.pdf}{GodetLHS20}~\cite{GodetLHS20}, \href{../works/YangSS19.pdf}{YangSS19}~\cite{YangSS19}, \href{../works/CauwelaertLS18.pdf}{CauwelaertLS18}~\cite{CauwelaertLS18}, \href{../works/HookerH17.pdf}{HookerH17}~\cite{HookerH17}, \href{../works/CauwelaertDMS16.pdf}{CauwelaertDMS16}~\cite{CauwelaertDMS16}, \href{../works/Tesch16.pdf}{Tesch16}~\cite{Tesch16}, \href{../works/GrimesH15.pdf}{GrimesH15}~\cite{GrimesH15}, \href{../works/ChuGNSW13.pdf}{ChuGNSW13}~\cite{ChuGNSW13}, \href{../works/LimtanyakulS12.pdf}{LimtanyakulS12}~\cite{LimtanyakulS12}, \href{../works/MalapertCGJLR12.pdf}{MalapertCGJLR12}~\cite{MalapertCGJLR12}, \href{../works/ChenGPSH10.pdf}{ChenGPSH10}~\cite{ChenGPSH10}, \href{../works/Wolf09.pdf}{Wolf09}~\cite{Wolf09}, \href{../works/MonetteDH09.pdf}{MonetteDH09}~\cite{MonetteDH09}, \href{../works/Vilim09a.pdf}{Vilim09a}~\cite{Vilim09a}, \href{../works/GrimesHM09.pdf}{GrimesHM09}~\cite{GrimesHM09}, \href{../works/Vilim09.pdf}{Vilim09}~\cite{Vilim09}, \href{../works/BocewiczBB09.pdf}{BocewiczBB09}~\cite{BocewiczBB09}, \href{../works/WolfS05.pdf}{WolfS05}~\cite{WolfS05}, \href{../works/Laborie03.pdf}{Laborie03}~\cite{Laborie03}, \href{../works/Vilim03.pdf}{Vilim03}~\cite{Vilim03}\\
Algorithms & sweep & \href{../works/Tesch18.pdf}{Tesch18}~\cite{Tesch18}, \href{../works/BonfiettiZLM16.pdf}{BonfiettiZLM16}~\cite{BonfiettiZLM16}, \href{../works/NattafALR16.pdf}{NattafALR16}~\cite{NattafALR16}, \href{../works/Tesch16.pdf}{Tesch16}~\cite{Tesch16}, \href{../works/LetortCB15.pdf}{LetortCB15}~\cite{LetortCB15}, \href{../works/Derrien15.pdf}{Derrien15}~\cite{Derrien15}, \href{../works/SimoninAHL15.pdf}{SimoninAHL15}~\cite{SimoninAHL15}, \href{../works/NattafAL15.pdf}{NattafAL15}~\cite{NattafAL15}, \href{../works/GayHS15.pdf}{GayHS15}~\cite{GayHS15}, \href{../works/DerrienPZ14.pdf}{DerrienPZ14}~\cite{DerrienPZ14}, \href{../works/Letort13.pdf}{Letort13}~\cite{Letort13}, \href{../works/LetortCB13.pdf}{LetortCB13}~\cite{LetortCB13}, \href{../works/Clercq12.pdf}{Clercq12}~\cite{Clercq12}, \href{../works/LetortBC12.pdf}{LetortBC12}~\cite{LetortBC12}, \href{../works/SimoninAHL12.pdf}{SimoninAHL12}~\cite{SimoninAHL12}, \href{../works/ClercqPBJ11.pdf}{ClercqPBJ11}~\cite{ClercqPBJ11}, \href{../works/Malapert11.pdf}{Malapert11}~\cite{Malapert11}, \href{../works/abs-0907-0939.pdf}{abs-0907-0939}~\cite{abs-0907-0939}, \href{../works/BeldiceanuP07.pdf}{BeldiceanuP07}~\cite{BeldiceanuP07}, \href{../works/Wolf05.pdf}{Wolf05}~\cite{Wolf05}, \href{../works/Wolf03.pdf}{Wolf03}~\cite{Wolf03}, \href{../works/BeldiceanuC02.pdf}{BeldiceanuC02}~\cite{BeldiceanuC02} & \href{../works/ArkhipovBL19.pdf}{ArkhipovBL19}~\cite{ArkhipovBL19}, \href{../works/FahimiOQ18.pdf}{FahimiOQ18}~\cite{FahimiOQ18}, \href{../works/GoldwaserS18.pdf}{GoldwaserS18}~\cite{GoldwaserS18}, \href{../works/GayHS15a.pdf}{GayHS15a}~\cite{GayHS15a}, \href{../works/Schutt11.pdf}{Schutt11}~\cite{Schutt11}, \href{../works/AronssonBK09.pdf}{AronssonBK09}~\cite{AronssonBK09}, \href{../works/PoderB08.pdf}{PoderB08}~\cite{PoderB08}, \href{../works/WolfS05.pdf}{WolfS05}~\cite{WolfS05} & \href{../works/BonninMNE24.pdf}{BonninMNE24}~\cite{BonninMNE24}, \href{../works/KameugneFND23.pdf}{KameugneFND23}~\cite{KameugneFND23}, \href{../works/TardivoDFMP23.pdf}{TardivoDFMP23}~\cite{TardivoDFMP23}, \href{../works/HebrardALLCMR22.pdf}{HebrardALLCMR22}~\cite{HebrardALLCMR22}, \href{../works/GeitzGSSW22.pdf}{GeitzGSSW22}~\cite{GeitzGSSW22}, \href{../works/OuelletQ22.pdf}{OuelletQ22}~\cite{OuelletQ22}, \href{../works/FetgoD22.pdf}{FetgoD22}~\cite{FetgoD22}, \href{../works/Godet21a.pdf}{Godet21a}~\cite{Godet21a}, \href{../works/FallahiAC20.pdf}{FallahiAC20}~\cite{FallahiAC20}, \href{../works/HoundjiSW19.pdf}{HoundjiSW19}~\cite{HoundjiSW19}, \href{../works/KameugneFGOQ18.pdf}{KameugneFGOQ18}~\cite{KameugneFGOQ18}, \href{../works/CauwelaertLS18.pdf}{CauwelaertLS18}~\cite{CauwelaertLS18}, \href{../works/Madi-WambaLOBM17.pdf}{Madi-WambaLOBM17}~\cite{Madi-WambaLOBM17}, \href{../works/Fahimi16.pdf}{Fahimi16}~\cite{Fahimi16}, \href{../works/Nattaf16.pdf}{Nattaf16}~\cite{Nattaf16}, \href{../works/GingrasQ16.pdf}{GingrasQ16}~\cite{GingrasQ16}, \href{../works/Dejemeppe16.pdf}{Dejemeppe16}~\cite{Dejemeppe16}, \href{../works/BartakV15.pdf}{BartakV15}~\cite{BartakV15}, \href{../works/EvenSH15.pdf}{EvenSH15}~\cite{EvenSH15}, \href{../works/EvenSH15a.pdf}{EvenSH15a}~\cite{EvenSH15a}, \href{../works/DerrienP14.pdf}{DerrienP14}~\cite{DerrienP14}, \href{../works/BonfiettiLBM14.pdf}{BonfiettiLBM14}~\cite{BonfiettiLBM14}, \href{../works/GaySS14.pdf}{GaySS14}~\cite{GaySS14}, \href{../works/OuelletQ13.pdf}{OuelletQ13}~\cite{OuelletQ13}, \href{../works/SimonisH11.pdf}{SimonisH11}~\cite{SimonisH11}, \href{../works/BeldiceanuCDP11.pdf}{BeldiceanuCDP11}~\cite{BeldiceanuCDP11}, \href{../works/Vilim11.pdf}{Vilim11}~\cite{Vilim11}, \href{../works/Lombardi10.pdf}{Lombardi10}~\cite{Lombardi10}, \href{../works/LombardiM10a.pdf}{LombardiM10a}~\cite{LombardiM10a}... (Total: 37)\\
Algorithms & time-tabling & \href{../works/ShaikhK23.pdf}{ShaikhK23}~\cite{ShaikhK23}, \href{../works/TardivoDFMP23.pdf}{TardivoDFMP23}~\cite{TardivoDFMP23}, \href{../works/OuelletQ22.pdf}{OuelletQ22}~\cite{OuelletQ22}, \href{../works/OrnekOS20.pdf}{OrnekOS20}~\cite{OrnekOS20}, \href{../works/Lemos21.pdf}{Lemos21}~\cite{Lemos21}, \href{../works/DemirovicS18.pdf}{DemirovicS18}~\cite{DemirovicS18}, \href{../works/FahimiOQ18.pdf}{FahimiOQ18}~\cite{FahimiOQ18}, \href{../works/Fahimi16.pdf}{Fahimi16}~\cite{Fahimi16}, \href{../works/GayHS15a.pdf}{GayHS15a}~\cite{GayHS15a}, \href{../works/Kameugne14.pdf}{Kameugne14}~\cite{Kameugne14}, \href{../works/OuelletQ13.pdf}{OuelletQ13}~\cite{OuelletQ13}, \href{../works/Letort13.pdf}{Letort13}~\cite{Letort13}, \href{../works/GuyonLPR12.pdf}{GuyonLPR12}~\cite{GuyonLPR12}, \href{../works/HeinzS11.pdf}{HeinzS11}~\cite{HeinzS11}, \href{../works/Menana11.pdf}{Menana11}~\cite{Menana11}, \href{../works/KanetAG04.pdf}{KanetAG04}~\cite{KanetAG04}, \href{../works/Laborie03.pdf}{Laborie03}~\cite{Laborie03}, \href{../works/ElkhyariGJ02a.pdf}{ElkhyariGJ02a}~\cite{ElkhyariGJ02a}, \href{../works/Wallace96.pdf}{Wallace96}~\cite{Wallace96} & \href{../works/Astrand21.pdf}{Astrand21}~\cite{Astrand21}, \href{../works/Godet21a.pdf}{Godet21a}~\cite{Godet21a}, \href{../works/WallaceY20.pdf}{WallaceY20}~\cite{WallaceY20}, \href{../works/ZarandiASC20.pdf}{ZarandiASC20}~\cite{ZarandiASC20}, \href{../works/abs-1902-01193.pdf}{abs-1902-01193}~\cite{abs-1902-01193}, \href{../works/OuelletQ18.pdf}{OuelletQ18}~\cite{OuelletQ18}, \href{../works/CauwelaertLS18.pdf}{CauwelaertLS18}~\cite{CauwelaertLS18}, \href{../works/Tesch18.pdf}{Tesch18}~\cite{Tesch18}, \href{../works/HookerH17.pdf}{HookerH17}~\cite{HookerH17}, \href{../works/Siala15a.pdf}{Siala15a}~\cite{Siala15a}, \href{../works/Derrien15.pdf}{Derrien15}~\cite{Derrien15}, \href{../works/GayHS15.pdf}{GayHS15}~\cite{GayHS15}, \href{../works/Siala15.pdf}{Siala15}~\cite{Siala15}, \href{../works/BofillGSV15.pdf}{BofillGSV15}~\cite{BofillGSV15}, \href{../works/Vilim11.pdf}{Vilim11}~\cite{Vilim11}, \href{../works/Elkhyari03.pdf}{Elkhyari03}~\cite{Elkhyari03}, \href{../works/Demassey03.pdf}{Demassey03}~\cite{Demassey03}, \href{../works/Bartak02.pdf}{Bartak02}~\cite{Bartak02} & \href{../works/BonninMNE24.pdf}{BonninMNE24}~\cite{BonninMNE24}, \href{../works/PrataAN23.pdf}{PrataAN23}~\cite{PrataAN23}, \href{../works/KameugneFND23.pdf}{KameugneFND23}~\cite{KameugneFND23}, \href{../works/AbreuNP23.pdf}{AbreuNP23}~\cite{AbreuNP23}, \href{../works/Fatemi-AnarakiTFV23.pdf}{Fatemi-AnarakiTFV23}~\cite{Fatemi-AnarakiTFV23}, \href{../works/LacknerMMWW23.pdf}{LacknerMMWW23}~\cite{LacknerMMWW23}, \href{../works/TouatBT22.pdf}{TouatBT22}~\cite{TouatBT22}, \href{../works/FarsiTM22.pdf}{FarsiTM22}~\cite{FarsiTM22}, \href{../works/FetgoD22.pdf}{FetgoD22}~\cite{FetgoD22}, \href{../works/SvancaraB22.pdf}{SvancaraB22}~\cite{SvancaraB22}, \href{../works/GeibingerMM21.pdf}{GeibingerMM21}~\cite{GeibingerMM21}, \href{../works/MokhtarzadehTNF20.pdf}{MokhtarzadehTNF20}~\cite{MokhtarzadehTNF20}, \href{../works/GodetLHS20.pdf}{GodetLHS20}~\cite{GodetLHS20}, \href{../works/LiuLH19.pdf}{LiuLH19}~\cite{LiuLH19}, \href{../works/KucukY19.pdf}{KucukY19}~\cite{KucukY19}, \href{../works/Caballero19.pdf}{Caballero19}~\cite{Caballero19}, \href{../works/Hooker19.pdf}{Hooker19}~\cite{Hooker19}, \href{../works/abs-1911-04766.pdf}{abs-1911-04766}~\cite{abs-1911-04766}, \href{../works/GeibingerMM19.pdf}{GeibingerMM19}~\cite{GeibingerMM19}, \href{../works/ArkhipovBL19.pdf}{ArkhipovBL19}~\cite{ArkhipovBL19}, \href{../works/KameugneFGOQ18.pdf}{KameugneFGOQ18}~\cite{KameugneFGOQ18}, \href{../works/AstrandJZ18.pdf}{AstrandJZ18}~\cite{AstrandJZ18}, \href{../works/BaptisteB18.pdf}{BaptisteB18}~\cite{BaptisteB18}, \href{../works/GoldwaserS18.pdf}{GoldwaserS18}~\cite{GoldwaserS18}, \href{../works/CohenHB17.pdf}{CohenHB17}~\cite{CohenHB17}, \href{../works/YoungFS17.pdf}{YoungFS17}~\cite{YoungFS17}, \href{../works/LuoVLBM16.pdf}{LuoVLBM16}~\cite{LuoVLBM16}, \href{../works/ZarandiKS16.pdf}{ZarandiKS16}~\cite{ZarandiKS16}, \href{../works/Tesch16.pdf}{Tesch16}~\cite{Tesch16}... (Total: 65)\\
\end{longtable}
}





\clearpage
\phantomsection
\addcontentsline{toc}{section}{Bibliography}
\bibliographystyle{plainurl}
\bibliography{bib}



\appendix
\clearpage
\section{Papers and Articles Missing a Local Copy}

This section lists all papers and articles for which we were not able to locate an electronic copy that we could download to our system. This might be because the work is behind a paywall for which we do not have access, or since the paper only exists in hardcopy, for works from the start of the period covered. As in either case we are not able to extract useful information from the work, either automatically, or manually, without the actual text itself, these gaps should be closed where possible.

{\scriptsize
\begin{longtable}{llp{5cm}p{10cm}rp{3cm}l}
\caption{Paper without Local Copy}\\ \toprule
Key & URL & Authors & Title & Year & \shortstack{Conference\\/Journal} & Cite\\ \midrule
\endhead
\bottomrule
\endfoot
FriedrichFMRSST14 & \href{https://doi.org/10.1007/978-3-319-28697-6\_23}{FriedrichFMRSST14} & Gerhard Friedrich and Melanie Fr{\"{u}}hst{\"{u}}ck and Vera Mersheeva and Anna Ryabokon and Maria Sander and Andreas Starzacher and Erich Teppan & Representing Production Scheduling with Constraint Answer Set Programming & 2014 & GOR 2014 & \cite{FriedrichFMRSST14}\\VillaverdeP04 & \href{}{VillaverdeP04} & Karen Villaverde and Enrico Pontelli & An Investigation of Scheduling in Distributed Constraint Logic Programming & 2004 & ISCA 2004 & \cite{VillaverdeP04}\\WolinskiKG04a & \href{https://doi.org/10.1145/968280.968336}{WolinskiKG04a} & Christophe Wolinski and Krzysztof Kuchcinski and Maya B. Gokhale & A constraints programming approach to communication scheduling on SoPC architectures & 2004 & FPGA 2004 & \cite{WolinskiKG04a}\\BoucherBVBL97 & \href{}{BoucherBVBL97} & Eric Boucher and Astrid Bachelu and Christophe Varnier and Pierre Baptiste and Bruno Legeard & Multi-criteria Comparison Between Algorithmic, Constraint Logic and Specific Constraint Programming on a Real Schedulingt Problem & 1997 & PACT 1997 & \cite{BoucherBVBL97}\\PapeB97 & \href{}{PapeB97} & Claude Le Pape and Philippe Baptiste & A Constraint Programming Library for Preemptive and Non-Preemptive Scheduling & 1997 & PACT 1997 & \cite{PapeB97}\\JourdanFRD94 & \href{}{JourdanFRD94} & Jean Jourdan and Fran{\c{c}}ois Fages and Didier Rozzonelli and Alain Demeure & Data Alignment and Task Scheduling On Parallel Machines Using Concurrent Constraint Model-based Programming & 1994 & ILPS 1994 & \cite{JourdanFRD94}\\AggounB92 & \href{}{AggounB92} & Abderrahmane Aggoun and Nicolas Beldiceanu & Extending {CHIP} in order to solve complex scheduling and placement problems & 1992 & JFPL 1992 & \cite{AggounB92}\\\end{longtable}
}



{\scriptsize
\begin{longtable}{p{3cm}p{5cm}p{10cm}p{1cm}rp{2.5cm}l}
\rowcolor{white}\caption{ARTICLE without Local Copy (Total 271)}\\ \toprule
\rowcolor{white}Key/URL & Authors & Title & Relevance &Year & \shortstack{Conference\\/Journal} & Cite\\ \midrule
\endhead
\bottomrule
\endfoot
Ahmadi-Javid2023 \href{http://dx.doi.org/10.1080/00207543.2023.2230489}{Ahmadi-Javid2023} & \hyperref[auth:a1762]{A. Ahmadi-Javid}, \hyperref[auth:a1763]{M. Haghi}, \hyperref[auth:a1764]{P. Hooshangi-Tabrizi} & Integrated job-shop scheduling in an FMS with heterogeneous transporters: MILP formulation, constraint programming, and branch-and-bound & \noindent{}\textbf{2.00} \textbf{2.00} n/a & 2023 & \cellcolor{red!20}International Journal of Production Research & \cite{Ahmadi-Javid2023}\\
Akan2023 \href{http://dx.doi.org/10.33714/masteb.1324266}{Akan2023} & \hyperref[auth:a1751]{E. Akan}, \hyperref[auth:a1752]{G. Alkan} & Optimizing Shipbuilding Production Project Scheduling Under Resource Constraints Using Genetic Algorithms and Fuzzy Sets & \noindent{}\textcolor{black!50}{0.00} \textcolor{black!50}{0.00} n/a & 2023 & Marine Science and Technology Bulletin & \cite{Akan2023}\\
Bocewicz2023 \href{http://dx.doi.org/10.3390/app13127165}{Bocewicz2023} & \hyperref[auth:a630]{G. Bocewicz}, \hyperref[auth:a1997]{E. Szwarc}, \hyperref[auth:a2016]{A. Thibbotuwawa}, \hyperref[auth:a1814]{Z. Banaszak} & \cellcolor{gold!20}Project Portfolio Planning Taking into Account the Effect of Loss of Competences of Project Team Members & \noindent{}\textcolor{black!50}{0.00} \textbf{1.50} n/a & 2023 & Applied Sciences & \cite{Bocewicz2023}\\
Danzinger2023 \href{http://dx.doi.org/10.1145/3546871}{Danzinger2023} & \hyperref[auth:a1484]{P. Danzinger}, \hyperref[auth:a77]{T. Geibinger}, \hyperref[auth:a1485]{D. Janneau}, \hyperref[auth:a80]{F. Mischek}, \hyperref[auth:a45]{N. Musliu}, \hyperref[auth:a1486]{C. Poschalko} & A System for Automated Industrial Test Laboratory Scheduling & \noindent{}\textcolor{black!50}{0.00} \textbf{5.00} n/a & 2023 & ACM Transactions on Intelligent Systems and Technology & \cite{Danzinger2023}\\
Dimny2023 \href{http://dx.doi.org/10.1007/s10100-023-00885-x}{Dimny2023} & \hyperref[auth:a1487]{I. Dimény}, \hyperref[auth:a1488]{T. Koltai} & \cellcolor{gold!20}Comparison of MILP and CP models for balancing partially automated assembly lines & \noindent{}\textcolor{black!50}{0.00} \textbf{5.00} n/a & 2023 & Central European Journal of Operations Research & \cite{Dimny2023}\\
Eiter2023 \href{http://dx.doi.org/10.1017/s1471068423000017}{Eiter2023} & \hyperref[auth:a1960]{T. Eiter}, \hyperref[auth:a77]{T. Geibinger}, \hyperref[auth:a45]{N. Musliu}, \hyperref[auth:a1961]{J. Oetsch}, \hyperref[auth:a1962]{P. Skočovský}, \hyperref[auth:a1963]{D. Stepanova} & \cellcolor{gold!20}Answer-Set Programming for Lexicographical Makespan Optimisation in Parallel Machine Scheduling & \noindent{}\textcolor{black!50}{0.00} \textbf{6.01} n/a & 2023 & Theory and Practice of Logic Programming & \cite{Eiter2023}\\
FahimiQ23 \href{http://dx.doi.org/10.1287/ijoc.2021.0138}{FahimiQ23} & \hyperref[auth:a122]{H. Fahimi}, \hyperref[auth:a37]{C.-G. Quimper} & Overload-Checking and Edge-Finding for Robust Cumulative Scheduling & \noindent{}\textcolor{black!50}{0.00} \textcolor{black!50}{0.00} n/a & 2023 & \cellcolor{red!20}INFORMS Journal on Computing & \cite{FahimiQ23}\\
GhasemiMH23 \href{http://dx.doi.org/10.1080/23302674.2023.2224509}{GhasemiMH23} & \hyperref[auth:a981]{S. Ghasemi}, \hyperref[auth:a430]{R. Tavakkoli-Moghaddam}, \hyperref[auth:a982]{M. Hamid} & Operating room scheduling by emphasising human factors and dynamic decision-making styles: a constraint programming method & \noindent{}\textbf{1.00} \textbf{1.00} n/a & 2023 & \cellcolor{red!20}International Journal of Systems Science: Operations \  Logistics & \cite{GhasemiMH23}\\
GunerGSKD23 \href{http://dx.doi.org/10.1080/00207543.2023.2226772}{GunerGSKD23} & \hyperref[auth:a1426]{F. G\"{u}ner}, \hyperref[auth:a1427]{A. K. G\"{o}r\"{u}r}, \hyperref[auth:a1428]{B. Satır}, \hyperref[auth:a1429]{L. Kandiller}, \hyperref[auth:a1430]{J. H. Drake} & A constraint programming approach to a real-world workforce scheduling problem for multi-manned assembly lines with sequence-dependent setup times & \noindent{}\textbf{1.00} \textbf{1.00} n/a & 2023 & \cellcolor{red!20}International Journal of Production Research & \cite{GunerGSKD23}\\
Hajji2023 \href{http://dx.doi.org/10.3390/computation11070137}{Hajji2023} & \hyperref[auth:a1537]{M. K. Hajji}, \hyperref[auth:a1538]{H. Hadda}, \hyperref[auth:a1539]{N. Dridi} & \cellcolor{gold!20}Makespan Minimization for the Two-Stage Hybrid Flow Shop Problem with Dedicated Machines: A Comprehensive Study of Exact and Heuristic Approaches & \noindent{}\textcolor{black!50}{0.00} \textbf{2.50} n/a & 2023 & Computation & \cite{Hajji2023}\\
Kasapidis2023 \href{http://dx.doi.org/10.1111/poms.13977}{Kasapidis2023} & \hyperref[auth:a1503]{G. A. Kasapidis}, \hyperref[auth:a1716]{S. Dauzère‐Pérès}, \hyperref[auth:a1504]{D. C. Paraskevopoulos}, \hyperref[auth:a1505]{P. P. Repoussis}, \hyperref[auth:a1506]{C. D. Tarantilis} & \cellcolor{gold!20}On the multiresource flexible job‐shop scheduling problem with arbitrary precedence graphs & \noindent{}\textcolor{black!50}{0.00} \textbf{5.00} n/a & 2023 & \cellcolor{red!20}Production and Operations Management & \cite{Kasapidis2023}\\
Liu2023 \href{http://dx.doi.org/10.3390/buildings13071867}{Liu2023} & \hyperref[auth:a1244]{S.-S. Liu}, \hyperref[auth:a1718]{P. Utami}, \hyperref[auth:a1719]{A. Budiwirawan}, \hyperref[auth:a1489]{M. F. A. Arifin}, \hyperref[auth:a1720]{F. S. Perdana} & \cellcolor{gold!20}Optimization Model of Maintenance Scheduling Problem for Heritage Buildings with Constraint Programming & \noindent{}\textbf{1.00} \textbf{4.01} n/a & 2023 & Buildings & \cite{Liu2023}\\
Lyons2023 \href{http://dx.doi.org/10.3390/analytics2030036}{Lyons2023} & \hyperref[auth:a1524]{J. S. F. Lyons}, \hyperref[auth:a836]{M. A. Begen}, \hyperref[auth:a1525]{P. C. Bell} & Surgery Scheduling and Perioperative Care: Smoothing and Visualizing Elective Surgery and Recovery Patient Flow & \noindent{}\textcolor{black!50}{0.00} \textbf{3.00} n/a & 2023 & Analytics & \cite{Lyons2023}\\
NouriMHD23 \href{http://dx.doi.org/10.1080/00207543.2023.2173503}{NouriMHD23} & \hyperref[auth:a737]{B. Vahedi-Nouri}, \hyperref[auth:a430]{R. Tavakkoli-Moghaddam}, \hyperref[auth:a946]{Z. Hanzálek}, \hyperref[auth:a947]{A. Dolgui} & Production scheduling in a reconfigurable manufacturing system benefiting from human-robot collaboration & \noindent{}\textcolor{black!50}{0.00} \textcolor{black!50}{0.00} n/a & 2023 & \cellcolor{red!20}International Journal of Production Research & \cite{NouriMHD23}\\
Oujana2023 \href{http://dx.doi.org/10.3390/app13106003}{Oujana2023} & \hyperref[auth:a453]{S. Oujana}, \hyperref[auth:a454]{L. Amodeo}, \hyperref[auth:a455]{F. Yalaoui}, \hyperref[auth:a1477]{D. Brodart} & \cellcolor{gold!20}Mixed-Integer Linear Programming, Constraint Programming and a Novel Dedicated Heuristic for Production Scheduling in a Packaging Plant & \noindent{}\textbf{1.00} \textbf{7.01} n/a & 2023 & Applied Sciences & \cite{Oujana2023}\\
Ramos2023 \href{http://dx.doi.org/10.3390/math11020337}{Ramos2023} & \hyperref[auth:a1731]{A. S. Ramos}, \hyperref[auth:a1732]{P. A. Miranda-Gonzalez}, \hyperref[auth:a1733]{S. Nucamendi-Guillén}, \hyperref[auth:a1734]{E. Olivares-Benitez} & \cellcolor{gold!20}A Formulation for the Stochastic Multi-Mode Resource-Constrained Project Scheduling Problem Solved with a Multi-Start Iterated Local Search Metaheuristic & \noindent{}\textcolor{black!50}{0.00} \textcolor{black!50}{0.00} n/a & 2023 & Mathematics & \cite{Ramos2023}\\
Relich2023 \href{http://dx.doi.org/10.3390/su15097667}{Relich2023} & \hyperref[auth:a1646]{M. Relich} & \cellcolor{gold!20}Predictive and Prescriptive Analytics in Identifying Opportunities for Improving Sustainable Manufacturing & \noindent{}\textcolor{black!50}{0.00} \textbf{3.00} n/a & 2023 & Sustainability & \cite{Relich2023}\\
Schweitzer2023 \href{http://dx.doi.org/10.3390/app13020806}{Schweitzer2023} & \hyperref[auth:a1592]{F. Schweitzer}, \hyperref[auth:a1593]{G. Bitsch}, \hyperref[auth:a1594]{L. Louw} & \cellcolor{gold!20}Choosing Solution Strategies for Scheduling Automated Guided Vehicles in Production Using Machine Learning & \noindent{}\textcolor{black!50}{0.00} \textbf{1.50} n/a & 2023 & Applied Sciences & \cite{Schweitzer2023}\\
Tayyab2023 \href{http://dx.doi.org/10.3390/app13063616}{Tayyab2023} & \hyperref[auth:a1640]{A. Tayyab}, \hyperref[auth:a1641]{S. Ullah}, \hyperref[auth:a1642]{T. Mahmood}, \hyperref[auth:a1643]{Y. Y. Ghadi}, \hyperref[auth:a1644]{B. Latif}, \hyperref[auth:a1645]{H. Aljuaid} & \cellcolor{gold!20}Modeling of Multi-Level Planning of Shifting Bottleneck Resources Integrated with Downstream Wards in a Hospital & \noindent{}\textcolor{black!50}{0.00} \textcolor{black!50}{0.00} n/a & 2023 & Applied Sciences & \cite{Tayyab2023}\\
Xu2023 \href{http://dx.doi.org/10.1108/k-09-2022-1339}{Xu2023} & \hyperref[auth:a1619]{J. Xu}, \hyperref[auth:a1620]{S. Bai} & A reactive scheduling approach for the resource-constrained project scheduling problem with dynamic resource disruption & \noindent{}\textcolor{black!50}{0.00} \textcolor{black!50}{0.00} n/a & 2023 & Kybernetes & \cite{Xu2023}\\
Braune2022 \href{http://dx.doi.org/10.1007/s10951-022-00750-w}{Braune2022} & \hyperref[auth:a1512]{R. Braune} & \cellcolor{gold!20}Packing-based branch-and-bound for discrete malleable task scheduling & \noindent{}\textcolor{black!50}{0.00} \textbf{3.00} n/a & 2022 & Journal of Scheduling & \cite{Braune2022}\\
Doolaard2022 \href{http://dx.doi.org/10.1007/s10472-022-09816-z}{Doolaard2022} & \hyperref[auth:a1900]{F. Doolaard}, \hyperref[auth:a19]{N. Yorke-Smith} & \cellcolor{gold!20}Online learning of variable ordering heuristics for constraint optimisation problems & \noindent{}0.50 0.50 n/a & 2022 & Annals of Mathematics and Artificial Intelligence & \cite{Doolaard2022}\\
El-Kholany2022 \href{http://dx.doi.org/10.1017/s1471068422000217}{El-Kholany2022} & \hyperref[auth:a1496]{M. M. S. El-Kholany}, \hyperref[auth:a61]{M. Gebser}, \hyperref[auth:a423]{K. Schekotihin} & \cellcolor{gold!20}Problem Decomposition and Multi-shot ASP Solving for Job-shop Scheduling & \noindent{}\textcolor{black!50}{0.00} \textbf{4.01} n/a & 2022 & Theory and Practice of Logic Programming & \cite{El-Kholany2022}\\
Feng2022 \href{http://dx.doi.org/10.3390/app12189062}{Feng2022} & \hyperref[auth:a1738]{C. Feng}, \hyperref[auth:a1739]{S. Hu}, \hyperref[auth:a1740]{Y. Ma}, \hyperref[auth:a1741]{Z. Li} & \cellcolor{gold!20}A Project Scheduling Game Equilibrium Problem Based on Dynamic Resource Supply & \noindent{}\textcolor{black!50}{0.00} \textcolor{black!50}{0.00} n/a & 2022 & Applied Sciences & \cite{Feng2022}\\
Gao2022 \href{http://dx.doi.org/10.1007/s11227-022-04943-0}{Gao2022} & \hyperref[auth:a1837]{J. Gao}, \hyperref[auth:a1838]{X. Zhu}, \hyperref[auth:a1839]{R. Zhang} & Optimization of parallel test task scheduling with constraint satisfaction & \noindent{}\textbf{2.00} \textbf{2.00} n/a & 2022 & The Journal of Supercomputing & \cite{Gao2022}\\
Gembarski2022 \href{http://dx.doi.org/10.3390/a15090318}{Gembarski2022} & \hyperref[auth:a1991]{P. C. Gembarski} & \cellcolor{gold!20}Joining Constraint Satisfaction Problems and Configurable CAD Product Models: A Step-by-Step Implementation Guide & \noindent{}\textcolor{black!50}{0.00} \textbf{2.00} n/a & 2022 & Algorithms & \cite{Gembarski2022}\\
Gokgur2022 \href{http://dx.doi.org/10.35378/gujs.681151}{Gokgur2022} & \hyperref[auth:a1612]{B. Gokgur}, \hyperref[auth:a1613]{S. Özpeyni̇rci̇} & \cellcolor{gold!20}Minimization of Number of Tool Switching Instants in Automated Manufacturing Systems & \noindent{}\textcolor{black!50}{0.00} \textbf{1.50} n/a & 2022 & Gazi University Journal of Science & \cite{Gokgur2022}\\
HillBCGN22 \href{http://dx.doi.org/10.1287/ijoc.2022.1222}{HillBCGN22} & \hyperref[auth:a64]{A. Hill}, \hyperref[auth:a971]{A. J. Brickey}, \hyperref[auth:a972]{I. Cipriano}, \hyperref[auth:a973]{M. Goycoolea}, \hyperref[auth:a974]{A. Newman} & Optimization Strategies for Resource-Constrained Project Scheduling Problems in Underground Mining & \noindent{}\textcolor{black!50}{0.00} \textcolor{black!50}{0.00} n/a & 2022 & \cellcolor{red!20}INFORMS Journal on Computing & \cite{HillBCGN22}\\
Kuramata2022 \href{http://dx.doi.org/10.1371/journal.pone.0266846}{Kuramata2022} & \hyperref[auth:a1690]{M. Kuramata}, \hyperref[auth:a1691]{R. Katsuki}, \hyperref[auth:a1692]{K. Nakata} & \cellcolor{gold!20}Solving large break minimization problems in a mirrored double round-robin tournament using quantum annealing & \noindent{}\textcolor{black!50}{0.00} \textcolor{black!50}{0.00} n/a & 2022 & PLOS ONE & \cite{Kuramata2022}\\
MartnezAJ22 \href{http://dx.doi.org/10.1287/ijoc.2021.1079}{MartnezAJ22} & \hyperref[auth:a935]{K. P. Martínez}, \hyperref[auth:a936]{Y. Adulyasak}, \hyperref[auth:a841]{R. Jans} & Logic-Based Benders Decomposition for Integrated Process Configuration and Production Planning Problems & \noindent{}\textcolor{black!50}{0.00} \textcolor{black!50}{0.00} n/a & 2022 & \cellcolor{red!20}INFORMS Journal on Computing & \cite{MartnezAJ22}\\
Michels2022 \href{http://dx.doi.org/10.1108/aa-10-2021-0140}{Michels2022} & \hyperref[auth:a1551]{A. S. Michels}, \hyperref[auth:a1552]{A. M. Costa} & Mixed-integer linear programming models for the type-II resource-constrained assembly line balancing problem & \noindent{}\textcolor{black!50}{0.00} \textbf{1.50} n/a & 2022 & Assembly Automation & \cite{Michels2022}\\
Misra2022 \href{http://dx.doi.org/10.1016/j.compchemeng.2022.107895}{Misra2022} & \hyperref[auth:a1802]{S. Misra}, \hyperref[auth:a1803]{L. R. Buttazoni}, \hyperref[auth:a1804]{V. Avadiappan}, \hyperref[auth:a1805]{H. J. Lee}, \hyperref[auth:a1806]{M. Yang}, \hyperref[auth:a381]{C. T. Maravelias} & CProS: A web-based application for chemical production scheduling & \noindent{}\textcolor{black!50}{0.00} \textcolor{black!50}{0.00} n/a & 2022 & Computers \  Chemical Engineering & \cite{Misra2022}\\
NaderiR22 \href{http://dx.doi.org/10.1287/ijoo.2021.0056}{NaderiR22} & \hyperref[auth:a726]{B. Naderi}, \hyperref[auth:a728]{V. Roshanaei} & Critical-Path-Search Logic-Based Benders Decomposition Approaches for Flexible Job Shop Scheduling & \noindent{}\textcolor{black!50}{0.00} \textcolor{black!50}{0.00} n/a & 2022 & \cellcolor{red!20}INFORMS Journal on Optimization & \cite{NaderiR22}\\
Ouellet2022 \href{http://dx.doi.org/10.1609/aaai.v36i4.20296}{Ouellet2022} & \hyperref[auth:a52]{Y. Ouellet}, \hyperref[auth:a37]{C.-G. Quimper} & The SoftCumulative Constraint with Quadratic Penalty & \noindent{}\textcolor{black!50}{0.00} \textbf{1.50} n/a & 2022 & Proceedings of the AAAI Conference on Artificial Intelligence & \cite{Ouellet2022}\\
Relich2022 \href{http://dx.doi.org/10.3390/app12041921}{Relich2022} & \hyperref[auth:a1646]{M. Relich}, \hyperref[auth:a1705]{I. Nielsen}, \hyperref[auth:a1815]{A. Gola} & \cellcolor{gold!20}Reducing the Total Product Cost at the Product Design Stage & \noindent{}\textcolor{black!50}{0.00} \textbf{2.00} n/a & 2022 & Applied Sciences & \cite{Relich2022}\\
ShiYXQ22 \href{https://doi.org/10.1080/00207543.2021.1963496}{ShiYXQ22} & \hyperref[auth:a446]{G. Shi}, \hyperref[auth:a447]{Z. Yang}, \hyperref[auth:a448]{Y. Xu}, \hyperref[auth:a449]{Y. Quan} & Solving the integrated process planning and scheduling problem using an enhanced constraint programming-based approach & \noindent{}\textbf{1.00} \textbf{1.00} n/a & 2022 & \cellcolor{red!20}International Journal of Production Research & \cite{ShiYXQ22}\\
Song2022 \href{http://dx.doi.org/10.1016/j.engappai.2021.104603}{Song2022} & \hyperref[auth:a1874]{W. Song}, \hyperref[auth:a1875]{Z. Cao}, \hyperref[auth:a1876]{J. Zhang}, \hyperref[auth:a1877]{C. Xu}, \hyperref[auth:a279]{A. Lim} & \cellcolor{green!10}Learning variable ordering heuristics for solving Constraint Satisfaction Problems & \noindent{}0.50 0.50 n/a & 2022 & Engineering Applications of Artificial Intelligence & \cite{Song2022}\\
Tapkan2022 \href{http://dx.doi.org/10.1080/01605682.2022.2125843}{Tapkan2022} & \hyperref[auth:a1787]{P. Tapkan}, \hyperref[auth:a1788]{S. Kulluk}, \hyperref[auth:a1789]{L. Özbakır}, \hyperref[auth:a1790]{F. Bahar}, \hyperref[auth:a1791]{B. Gülmez} & A constraint programming based column generation approach for crew scheduling: A case study for the Kayseri railway & \noindent{}\textbf{1.00} \textbf{1.00} n/a & 2022 & \cellcolor{red!20}Journal of the Operational Research Society & \cite{Tapkan2022}\\
Tomczak2022 \href{http://dx.doi.org/10.3846/jcem.2022.16943}{Tomczak2022} & \hyperref[auth:a1768]{M. Tomczak}, \hyperref[auth:a1769]{P. Jaśkowski} & \cellcolor{gold!20}Scheduling repetitive construction projects: structured literature review & \noindent{}\textcolor{black!50}{0.00} \textbf{1.00} n/a & 2022 & JOURNAL OF CIVIL ENGINEERING AND MANAGEMENT & \cite{Tomczak2022}\\
Valouxis2022 \href{http://dx.doi.org/10.3390/a15120450}{Valouxis2022} & \hyperref[auth:a1507]{C. Valouxis}, \hyperref[auth:a1508]{C. Gogos}, \hyperref[auth:a1509]{A. Dimitsas}, \hyperref[auth:a1510]{P. Potikas}, \hyperref[auth:a1511]{A. Vittas} & \cellcolor{gold!20}A Hybrid Exact–Local Search Approach for One-Machine Scheduling with Time-Dependent Capacity & \noindent{}\textcolor{black!50}{0.00} \textbf{3.50} n/a & 2022 & Algorithms & \cite{Valouxis2022}\\
Zohali2022 \href{http://dx.doi.org/10.1287/ijoc.2020.1015}{Zohali2022} & \hyperref[auth:a1526]{H. Zohali}, \hyperref[auth:a726]{B. Naderi}, \hyperref[auth:a728]{V. Roshanaei} & Solving the Type-2 Assembly Line Balancing with Setups Using Logic-Based Benders Decomposition & \noindent{}\textcolor{black!50}{0.00} \textcolor{black!50}{0.00} n/a & 2022 & \cellcolor{red!20}INFORMS Journal on Computing & \cite{Zohali2022}\\
Bocewicz2021 \href{http://dx.doi.org/10.17531/ein.2021.1.13}{Bocewicz2021} & \hyperref[auth:a630]{G. Bocewicz}, \hyperref[auth:a1997]{E. Szwarc}, \hyperref[auth:a535]{J. Wikarek}, \hyperref[auth:a1527]{P. Nielsen}, \hyperref[auth:a1814]{Z. Banaszak} & \cellcolor{gold!20}A competency-driven staff assignment approach to improving employee scheduling robustness & \noindent{}\textcolor{black!50}{0.00} \textbf{2.00} n/a & 2021 & Eksploatacja i Niezawodność – Maintenance and Reliability & \cite{Bocewicz2021}\\
CarlierSJP21 \href{http://dx.doi.org/10.1080/00207543.2021.1923853}{CarlierSJP21} & \hyperref[auth:a845]{J. Carlier}, \hyperref[auth:a928]{A. Sahli}, \hyperref[auth:a929]{A. Jouglet}, \hyperref[auth:a846]{E. Pinson} & A faster checker of the energetic reasoning for the cumulative scheduling problem & \noindent{}\textcolor{black!50}{0.00} \textcolor{black!50}{0.00} n/a & 2021 & \cellcolor{red!20}International Journal of Production Research & \cite{CarlierSJP21}\\
Chen2021 \href{http://dx.doi.org/10.1177/03611981211036368}{Chen2021} & \hyperref[auth:a1626]{G.-H. Chen}, \hyperref[auth:a1627]{J.-C. Jong}, \hyperref[auth:a1628]{A. F.-W. Han} & \cellcolor{gold!20}Applying Constraint Programming and Integer Programming to Solve the Crew Scheduling Problem for Railroad Systems: Model Formulation and a Case Study & \noindent{}\textbf{1.00} \textbf{2.00} n/a & 2021 & Transportation Research Record: Journal of the Transportation Research Board & \cite{Chen2021}\\
Daneshamooz2021 \href{http://dx.doi.org/10.1108/k-08-2020-0521}{Daneshamooz2021} & \hyperref[auth:a1728]{F. Daneshamooz}, \hyperref[auth:a1729]{P. Fattahi}, \hyperref[auth:a1730]{S. M. H. Hosseini} & Mathematical modeling and two efficient branch and bound algorithms for job shop scheduling problem followed by an assembly stage & \noindent{}\textcolor{black!50}{0.00} \textcolor{black!50}{0.00} n/a & 2021 & Kybernetes & \cite{Daneshamooz2021}\\
Grzegorz2021 \href{http://dx.doi.org/10.3390/app11198898}{Grzegorz2021} & \hyperref[auth:a2062]{R. Grzegorz}, \hyperref[auth:a2063]{B. Grzegorz}, \hyperref[auth:a2064]{D. Bogdan}, \hyperref[auth:a2065]{B. Zbigniew} & \cellcolor{gold!20}Reactive Planning-Driven Approach to Online UAVs Mission Rerouting and Rescheduling & \noindent{}\textcolor{black!50}{0.00} \textbf{1.50} n/a & 2021 & Applied Sciences & \cite{Grzegorz2021}\\
Hosseinian2021 \href{http://dx.doi.org/10.1051/ro/2021087}{Hosseinian2021} & \hyperref[auth:a1573]{A. H. Hosseinian}, \hyperref[auth:a1574]{V. Baradaran} & \cellcolor{gold!20}A multi-objective multi-agent optimization algorithm for the multi-skill resource-constrained project scheduling problem with transfer times & \noindent{}\textcolor{black!50}{0.00} \textcolor{black!50}{0.00} n/a & 2021 & RAIRO - Operations Research & \cite{Hosseinian2021}\\
Kasapidis2021 \href{http://dx.doi.org/10.1111/poms.13501}{Kasapidis2021} & \hyperref[auth:a1503]{G. A. Kasapidis}, \hyperref[auth:a1504]{D. C. Paraskevopoulos}, \hyperref[auth:a1505]{P. P. Repoussis}, \hyperref[auth:a1506]{C. D. Tarantilis} & \cellcolor{green!10}Flexible Job Shop Scheduling Problems with Arbitrary Precedence Graphs & \noindent{}\textcolor{black!50}{0.00} \textbf{3.00} n/a & 2021 & \cellcolor{red!20}Production and Operations Management & \cite{Kasapidis2021}\\
Kong2021 \href{http://dx.doi.org/10.1061/(asce)co.1943-7862.0002192}{Kong2021} & \hyperref[auth:a1706]{F. Kong}, \hyperref[auth:a1707]{J. Guo}, \hyperref[auth:a1708]{X. Lv} & Project Resource Input Optimization Problem with Combined Time Constraints Based on Node Network Diagram and Constraint Programming & \noindent{}0.50 0.50 n/a & 2021 & Journal of Construction Engineering and Management & \cite{Kong2021}\\
Liu2021 \href{http://dx.doi.org/10.3390/app11041447}{Liu2021} & \hyperref[auth:a1244]{S.-S. Liu}, \hyperref[auth:a1489]{M. F. A. Arifin}, \hyperref[auth:a1490]{W. T. Chen}, \hyperref[auth:a1491]{Y.-H. Huang} & \cellcolor{gold!20}Emergency Repair Scheduling Model for Road Network Integrating Rescheduling Feature & \noindent{}\textcolor{black!50}{0.00} \textbf{4.00} n/a & 2021 & Applied Sciences & \cite{Liu2021}\\
Liu2021a \href{http://dx.doi.org/10.3390/math9192492}{Liu2021a} & \hyperref[auth:a1244]{S.-S. Liu}, \hyperref[auth:a1719]{A. Budiwirawan}, \hyperref[auth:a1489]{M. F. A. Arifin} & \cellcolor{gold!20}Non-Sequential Linear Construction Project Scheduling Model for Minimizing Idle Equipment Using Constraint Programming (CP) & \noindent{}\textbf{2.00} \textbf{3.00} n/a & 2021 & Mathematics & \cite{Liu2021a}\\
Liu2021b \href{http://dx.doi.org/10.3390/sym13030364}{Liu2021b} & \hyperref[auth:a1244]{S.-S. Liu}, \hyperref[auth:a1719]{A. Budiwirawan}, \hyperref[auth:a1489]{M. F. A. Arifin}, \hyperref[auth:a1490]{W. T. Chen}, \hyperref[auth:a1491]{Y.-H. Huang} & \cellcolor{gold!20}Optimization Model for the Pavement Pothole Repair Problem Considering Consumable Resources & \noindent{}\textcolor{black!50}{0.00} \textbf{6.01} n/a & 2021 & Symmetry & \cite{Liu2021b}\\
Mischek2021 \href{http://dx.doi.org/10.1007/s10951-021-00699-2}{Mischek2021} & \hyperref[auth:a80]{F. Mischek}, \hyperref[auth:a45]{N. Musliu}, \hyperref[auth:a1261]{A. Schaerf} & \cellcolor{gold!20}Local search approaches for the test laboratory scheduling problem with variable task grouping & \noindent{}\textcolor{black!50}{0.00} \textcolor{black!50}{0.00} n/a & 2021 & Journal of Scheduling & \cite{Mischek2021}\\
Mischek2021a \href{http://dx.doi.org/10.1007/s10479-021-04007-1}{Mischek2021a} & \hyperref[auth:a80]{F. Mischek}, \hyperref[auth:a45]{N. Musliu} & \cellcolor{gold!20}A local search framework for industrial test laboratory scheduling & \noindent{}\textcolor{black!50}{0.00} \textbf{2.50} n/a & 2021 & Annals of Operations Research & \cite{Mischek2021a}\\
NaderiRBAU21 \href{http://dx.doi.org/10.1111/poms.13397}{NaderiRBAU21} & \hyperref[auth:a726]{B. Naderi}, \hyperref[auth:a728]{V. Roshanaei}, \hyperref[auth:a836]{M. A. Begen}, \hyperref[auth:a895]{D. M. Aleman}, \hyperref[auth:a896]{D. R. Urbach} & Increased Surgical Capacity without Additional Resources: Generalized Operating Room Planning and Scheduling & \noindent{}\textcolor{black!50}{0.00} \textcolor{black!50}{0.00} n/a & 2021 & \cellcolor{red!20}Production and Operations Management & \cite{NaderiRBAU21}\\
Ortiz-Bayliss2021 \href{http://dx.doi.org/10.3390/app11062749}{Ortiz-Bayliss2021} & \hyperref[auth:a1603]{J. C. Ortiz-Bayliss}, \hyperref[auth:a1604]{I. Amaya}, \hyperref[auth:a1605]{J. M. Cruz-Duarte}, \hyperref[auth:a1606]{A. E. Gutierrez-Rodriguez}, \hyperref[auth:a1607]{S. E. Conant-Pablos}, \hyperref[auth:a1608]{H. Terashima-Marín} & \cellcolor{gold!20}A General Framework Based on Machine Learning for Algorithm Selection in Constraint Satisfaction Problems & \noindent{}0.50 \textbf{1.50} n/a & 2021 & Applied Sciences & \cite{Ortiz-Bayliss2021}\\
Pinarbasi21 \href{http://dx.doi.org/10.1080/0305215x.2021.1921171}{Pinarbasi21} & \hyperref[auth:a1384]{M. Pınarbaşı} & New mathematical and constraint programming models for U-type assembly line balancing problems with assignment restrictions & \noindent{}\textcolor{black!50}{0.00} \textcolor{black!50}{0.00} n/a & 2021 & \cellcolor{red!20}Engineering Optimization & \cite{Pinarbasi21}\\
RabbaniMM21 \href{http://dx.doi.org/10.1080/17509653.2021.1905096}{RabbaniMM21} & \hyperref[auth:a1246]{M. Rabbani}, \hyperref[auth:a515]{M. Mokhtarzadeh}, \hyperref[auth:a1247]{N. Manavizadeh} & A constraint programming approach and a hybrid of genetic and K-means algorithms to solve the p-hub location-allocation problems & \noindent{}\textcolor{black!50}{0.00} \textcolor{black!50}{0.00} n/a & 2021 & \cellcolor{red!20}International Journal of Management Science and Engineering Management & \cite{RabbaniMM21}\\
Radzki2021 \href{http://dx.doi.org/10.3390/su13095228}{Radzki2021} & \hyperref[auth:a2007]{G. Radzki}, \hyperref[auth:a1705]{I. Nielsen}, \hyperref[auth:a2008]{P. Golińska-Dawson}, \hyperref[auth:a630]{G. Bocewicz}, \hyperref[auth:a1814]{Z. Banaszak} & \cellcolor{gold!20}Reactive UAV Fleet's Mission Planning in Highly Dynamic and Unpredictable Environments & \noindent{}\textcolor{black!50}{0.00} \textbf{1.50} n/a & 2021 & Sustainability & \cite{Radzki2021}\\
Ramos2021 \href{http://dx.doi.org/10.1111/exsy.12830}{Ramos2021} & \hyperref[auth:a1731]{A. S. Ramos}, \hyperref[auth:a1736]{E. Olivares‐Benitez}, \hyperref[auth:a1737]{P. A. Miranda‐Gonzalez} & Multi‐start iterated local search metaheuristic for the multi‐mode resource‐constrained project scheduling problem & \noindent{}\textcolor{black!50}{0.00} \textcolor{black!50}{0.00} n/a & 2021 & Expert Systems & \cite{Ramos2021}\\
Rieber2021 \href{http://dx.doi.org/10.1145/3487922}{Rieber2021} & \hyperref[auth:a1890]{D. Rieber}, \hyperref[auth:a1891]{A. Acosta}, \hyperref[auth:a1892]{H. Fröning} & \cellcolor{gold!20}Joint Program and Layout Transformations to Enable Convolutional Operators on Specialized Hardware Based on Constraint Programming & \noindent{}\textcolor{black!50}{0.00} \textbf{2.00} n/a & 2021 & ACM Transactions on Architecture and Code Optimization & \cite{Rieber2021}\\
Sahli2021 \href{http://dx.doi.org/10.1051/ro/2021164}{Sahli2021} & \hyperref[auth:a928]{A. Sahli}, \hyperref[auth:a845]{J. Carlier}, \hyperref[auth:a1170]{A. Moukrim} & \cellcolor{gold!20}Polynomial algorithms for some scheduling problems with one nonrenewable resource & \noindent{}\textcolor{black!50}{0.00} \textcolor{black!50}{0.00} n/a & 2021 & RAIRO - Operations Research & \cite{Sahli2021}\\
Spieker2021 \href{http://dx.doi.org/10.3390/ai2040033}{Spieker2021} & \hyperref[auth:a196]{H. Spieker}, \hyperref[auth:a195]{A. Gotlieb} & \cellcolor{gold!20}Predictive Machine Learning of Objective Boundaries for Solving COPs & \noindent{}\textcolor{black!50}{0.00} \textbf{2.00} n/a & 2021 & AI & \cite{Spieker2021}\\
Strak2021 \href{http://dx.doi.org/10.5937/tehnika2102239s}{Strak2021} & \hyperref[auth:a2027]{M. Strak}, \hyperref[auth:a2028]{R. Lečić} & Organization of work with clients in the COVID-19 emergency conditions using constraint programming & \noindent{}\textcolor{black!50}{0.00} \textbf{1.00} n/a & 2021 & Tehnika & \cite{Strak2021}\\
Wang2021 \href{http://dx.doi.org/10.1155/2021/5531063}{Wang2021} & \hyperref[auth:a1968]{L. Wang}, \hyperref[auth:a1969]{W. Ma}, \hyperref[auth:a1970]{L. Wang}, \hyperref[auth:a1971]{Y. Ren}, \hyperref[auth:a1972]{C. Yu} & \cellcolor{gold!20}Enabling In-Depot Automated Routing and Recharging Scheduling for Automated Electric Bus Transit Systems & \noindent{}\textcolor{black!50}{0.00} \textbf{5.00} n/a & 2021 & Journal of Advanced Transportation & \cite{Wang2021}\\
Xia2021 \href{http://dx.doi.org/10.3233/jifs-189721}{Xia2021} & \hyperref[auth:a1540]{Y. Xia}, \hyperref[auth:a1541]{Z. Xie}, \hyperref[auth:a1542]{Y. Xin}, \hyperref[auth:a1543]{X. Zhang} & A multi-shop integrated scheduling algorithm with fixed output constraint & \noindent{}\textcolor{black!50}{0.00} \textbf{2.50} n/a & 2021 & Journal of Intelligent \  Fuzzy Systems & \cite{Xia2021}\\
Zou2021 \href{http://dx.doi.org/10.1108/ecam-10-2020-0843}{Zou2021} & \hyperref[auth:a756]{X. Zou}, \hyperref[auth:a757]{L. Zhang}, \hyperref[auth:a1483]{Q. Zhang} & Time-cost optimization in repetitive project scheduling with limited resources & \noindent{}\textcolor{black!50}{0.00} \textbf{4.00} n/a & 2021 & Engineering, Construction and Architectural Management & \cite{Zou2021}\\
Zuenko2021 \href{http://dx.doi.org/10.1088/1742-6596/2060/1/012021}{Zuenko2021} & \hyperref[auth:a1994]{A. Zuenko}, \hyperref[auth:a1995]{Y. Oleynik}, \hyperref[auth:a1996]{R. Makedonov} & \cellcolor{gold!20}A method for solving the open-pit mine production scheduling problem using the constraint programming paradigm & \noindent{}\textbf{1.00} \textbf{2.00} n/a & 2021 & Journal of Physics: Conference Series & \cite{Zuenko2021}\\
AlizdehS20 \href{https://doi.org/10.1504/IJAIP.2020.106687}{AlizdehS20} & \hyperref[auth:a513]{S. Alizdeh}, \hyperref[auth:a514]{S. Saeidi} & Fuzzy project scheduling with critical path including risk and resource constraints using linear programming & \noindent{}\textcolor{black!50}{0.00} \textcolor{black!50}{0.00} n/a & 2020 & \cellcolor{red!20}Int. J. Adv. Intell. Paradigms & \cite{AlizdehS20}\\
BalochG20 \href{http://dx.doi.org/10.1287/trsc.2019.0928}{BalochG20} & \hyperref[auth:a1237]{G. Baloch}, \hyperref[auth:a1238]{F. Gzara} & Strategic Network Design for Parcel Delivery with Drones Under Competition & \noindent{}\textcolor{black!50}{0.00} \textcolor{black!50}{0.00} n/a & 2020 & \cellcolor{red!20}Transportation Science & \cite{BalochG20}\\
Caricato2020 \href{http://dx.doi.org/10.1007/s00170-020-06176-y}{Caricato2020} & \hyperref[auth:a1499]{P. Caricato}, \hyperref[auth:a1500]{A. Grieco}, \hyperref[auth:a1501]{A. Arigliano}, \hyperref[auth:a1502]{L. Rondone} & \cellcolor{gold!20}Workforce influence on manufacturing machines schedules & \noindent{}\textcolor{black!50}{0.00} \textbf{3.00} n/a & 2020 & The International Journal of Advanced Manufacturing Technology & \cite{Caricato2020}\\
Danzinger2020 \href{http://dx.doi.org/10.1609/icaps.v30i1.6681}{Danzinger2020} & \hyperref[auth:a1484]{P. Danzinger}, \hyperref[auth:a77]{T. Geibinger}, \hyperref[auth:a80]{F. Mischek}, \hyperref[auth:a45]{N. Musliu} & Solving the Test Laboratory Scheduling Problem with Variable Task Grouping & \noindent{}\textcolor{black!50}{0.00} \textbf{3.50} n/a & 2020 & Proceedings of the International Conference on Automated Planning and Scheduling & \cite{Danzinger2020}\\
GuoHLW20 \href{http://dx.doi.org/10.1080/0305215x.2019.1699919}{GuoHLW20} & \hyperref[auth:a931]{P. Guo}, \hyperref[auth:a932]{X. He}, \hyperref[auth:a933]{Y. Luan}, \hyperref[auth:a934]{Y. Wang} & Logic-based Benders decomposition for gantry crane scheduling with transferring position constraints in a rail–road container terminal & \noindent{}\textcolor{black!50}{0.00} \textcolor{black!50}{0.00} n/a & 2020 & \cellcolor{red!20}Engineering Optimization & \cite{GuoHLW20}\\
Ham20 \href{http://dx.doi.org/10.1080/00207543.2019.1709671}{Ham20} & \hyperref[auth:a750]{A. Ham} & Transfer-robot task scheduling in job shop & \noindent{}\textcolor{black!50}{0.00} \textcolor{black!50}{0.00} n/a & 2020 & \cellcolor{red!20}International Journal of Production Research & \cite{Ham20}\\
KizilayC20 \href{http://dx.doi.org/10.1080/0305215x.2020.1786081}{KizilayC20} & \hyperref[auth:a1380]{D. Kizilay}, \hyperref[auth:a1381]{Z. A. Cil} & Constraint programming approach for multi-objective two-sided assembly line balancing problem with multi-operator stations & \noindent{}\textcolor{black!50}{0.00} \textcolor{black!50}{0.00} n/a & 2020 & \cellcolor{red!20}Engineering Optimization & \cite{KizilayC20}\\
Kong2020 \href{http://dx.doi.org/10.1061/(asce)co.1943-7862.0001929}{Kong2020} & \hyperref[auth:a1706]{F. Kong}, \hyperref[auth:a1780]{D. Dou} & RCPSP with Combined Precedence Relations and Resource Calendars & \noindent{}\textcolor{black!50}{0.00} \textcolor{black!50}{0.00} n/a & 2020 & Journal of Construction Engineering and Management & \cite{Kong2020}\\
Li2020 \href{http://dx.doi.org/10.1007/s10732-019-09434-9}{Li2020} & \hyperref[auth:a1796]{H. Li}, \hyperref[auth:a1811]{G. Feng}, \hyperref[auth:a1812]{M. Yin} & On combining variable ordering heuristics for constraint satisfaction problems & \noindent{}0.50 0.50 n/a & 2020 & Journal of Heuristics & \cite{Li2020}\\
Liu2020 \href{http://dx.doi.org/10.3390/app10248887}{Liu2020} & \hyperref[auth:a1244]{S.-S. Liu}, \hyperref[auth:a1494]{H.-Y. Huang}, \hyperref[auth:a1495]{N. R. D. Kumala} & \cellcolor{gold!20}Two-Stage Optimization Model for Life Cycle Maintenance Scheduling of Bridge Infrastructure & \noindent{}\textcolor{black!50}{0.00} \textbf{4.00} n/a & 2020 & Applied Sciences & \cite{Liu2020}\\
Mnif2020 \href{http://dx.doi.org/10.4018/ijamc.2020040107}{Mnif2020} & \hyperref[auth:a1964]{M. G. Mnif}, \hyperref[auth:a1965]{S. Bouamama} & Multi-Layer Distributed Constraint Satisfaction for Multi-criteria Optimization Problem & \noindent{}\textcolor{black!50}{0.00} \textbf{6.01} n/a & 2020 & International Journal of Applied Metaheuristic Computing & \cite{Mnif2020}\\
PinarbasiA20 \href{http://dx.doi.org/10.1080/0305215x.2020.1716746}{PinarbasiA20} & \hyperref[auth:a1384]{M. Pınarbaşı}, \hyperref[auth:a764]{H. M. Alakaş} & Balancing stochastic type-II assembly lines: chance-constrained mixed integer and constraint programming models & \noindent{}\textcolor{black!50}{0.00} \textcolor{black!50}{0.00} n/a & 2020 & \cellcolor{red!20}Engineering Optimization & \cite{PinarbasiA20}\\
Relich2020 \href{http://dx.doi.org/10.3390/app10186330}{Relich2020} & \hyperref[auth:a1646]{M. Relich}, \hyperref[auth:a1647]{A. Świć} & \cellcolor{gold!20}Parametric Estimation and Constraint Programming-Based Planning and Simulation of Production Cost of a New Product & \noindent{}\textcolor{black!50}{0.00} \textbf{1.00} n/a & 2020 & Applied Sciences & \cite{Relich2020}\\
Tesch2020 \href{http://dx.doi.org/10.1007/s10951-020-00647-6}{Tesch2020} & \hyperref[auth:a183]{A. Tesch} & \cellcolor{gold!20}A polyhedral study of event-based models for the resource-constrained project scheduling problem & \noindent{}\textcolor{black!50}{0.00} \textcolor{black!50}{0.00} n/a & 2020 & Journal of Scheduling & \cite{Tesch2020}\\
Watermeyer2020 \href{http://dx.doi.org/10.1007/s00291-020-00583-z}{Watermeyer2020} & \hyperref[auth:a1770]{K. Watermeyer}, \hyperref[auth:a1771]{J. Zimmermann} & \cellcolor{gold!20}A branch-and-bound procedure for the resource-constrained project scheduling problem with partially renewable resources and general temporal constraints & \noindent{}\textcolor{black!50}{0.00} \textcolor{black!50}{0.00} n/a & 2020 & OR Spectrum & \cite{Watermeyer2020}\\
Abuwarda2019 \href{http://dx.doi.org/10.1139/cjce-2018-0544}{Abuwarda2019} & \hyperref[auth:a1520]{Z. Abuwarda}, \hyperref[auth:a1521]{T. Hegazy} & \cellcolor{gold!20}Multi-dimensional optimization model for schedule fast-tracking without over-stressing construction workers & \noindent{}\textcolor{black!50}{0.00} \textbf{2.00} n/a & 2019 & Canadian Journal of Civil Engineering & \cite{Abuwarda2019}\\
Benda2019 \href{http://dx.doi.org/10.1007/s00291-019-00567-8}{Benda2019} & \hyperref[auth:a1966]{F. Benda}, \hyperref[auth:a1512]{R. Braune}, \hyperref[auth:a1967]{K. F. Doerner}, \hyperref[auth:a951]{R. F. Hartl} & \cellcolor{gold!20}A machine learning approach for flow shop scheduling problems with alternative resources, sequence-dependent setup times, and blocking & \noindent{}\textcolor{black!50}{0.00} \textbf{6.01} n/a & 2019 & OR Spectrum & \cite{Benda2019}\\
Chakrabortty2019 \href{http://dx.doi.org/10.1111/itor.12644}{Chakrabortty2019} & \hyperref[auth:a1614]{R. K. Chakrabortty}, \hyperref[auth:a1615]{A. Abbasi}, \hyperref[auth:a1616]{M. J. Ryan} & \cellcolor{gold!20}Multi‐mode resource‐constrained project scheduling using modified variable neighborhood search heuristic & \noindent{}\textcolor{black!50}{0.00} \textcolor{black!50}{0.00} n/a & 2019 & International Transactions in Operational Research & \cite{Chakrabortty2019}\\
Cox2019 \href{http://dx.doi.org/10.4114/intartif.vol22iss63pp1-15}{Cox2019} & \hyperref[auth:a1920]{J. L. Cox}, \hyperref[auth:a1921]{S. Lucci}, \hyperref[auth:a1922]{T. Pay} & \cellcolor{gold!20}Effects of Dynamic Variable - Value Ordering  Heuristics on the Search Space of Sudoku Modeled as a Constraint Satisfaction Problem & \noindent{}0.50 0.50 n/a & 2019 & Inteligencia Artificial & \cite{Cox2019}\\
EdwardsBSE19 \href{http://dx.doi.org/10.1080/01605682.2019.1595192}{EdwardsBSE19} & \hyperref[auth:a892]{S. J. Edwards}, \hyperref[auth:a893]{D. Baatar}, \hyperref[auth:a894]{K. Smith-Miles}, \hyperref[auth:a469]{A. T. Ernst} & Symmetry breaking of identical projects in the high-multiplicity RCPSP/max & \noindent{}\textcolor{black!50}{0.00} \textcolor{black!50}{0.00} n/a & 2019 & \cellcolor{red!20}Journal of the Operational Research Society & \cite{EdwardsBSE19}\\
Geiger2019 \href{http://dx.doi.org/10.1613/jair.1.11303}{Geiger2019} & \hyperref[auth:a1829]{M. J. Geiger}, \hyperref[auth:a78]{L. Kletzander}, \hyperref[auth:a45]{N. Musliu} & \cellcolor{gold!20}Solving the Torpedo Scheduling Problem & \noindent{}\textcolor{black!50}{0.00} \textbf{1.50} n/a & 2019 & Journal of Artificial Intelligence Research & \cite{Geiger2019}\\
He2019 \href{http://dx.doi.org/10.1007/s10845-019-01518-4}{He2019} & \hyperref[auth:a1547]{L. He}, \hyperref[auth:a308]{M. de Weerdt}, \hyperref[auth:a19]{N. Yorke-Smith} & \cellcolor{gold!20}Time/sequence-dependent scheduling: the design and evaluation of a general purpose tabu-based adaptive large neighbourhood search algorithm & \noindent{}\textcolor{black!50}{0.00} \textbf{1.50} n/a & 2019 & Journal of Intelligent Manufacturing & \cite{He2019}\\
HechingHK19 \href{http://dx.doi.org/10.1287/trsc.2018.0830}{HechingHK19} & \hyperref[auth:a1021]{A. Heching}, \hyperref[auth:a160]{J. N. Hooker}, \hyperref[auth:a1022]{R. Kimura} & \cellcolor{gold!20}A Logic-Based Benders Approach to Home Healthcare Delivery & \noindent{}\textcolor{black!50}{0.00} \textcolor{black!50}{0.00} n/a & 2019 & \cellcolor{red!20}Transportation Science & \cite{HechingHK19}\\
Hosseinian2019 \href{http://dx.doi.org/10.1108/jm2-07-2018-0098}{Hosseinian2019} & \hyperref[auth:a1573]{A. H. Hosseinian}, \hyperref[auth:a1574]{V. Baradaran}, \hyperref[auth:a1575]{M. Bashiri} & Modeling of the time-dependent multi-skilled RCPSP considering learning effect & \noindent{}\textcolor{black!50}{0.00} \textcolor{black!50}{0.00} n/a & 2019 & Journal of Modelling in Management & \cite{Hosseinian2019}\\
Kizilay2019 \href{http://dx.doi.org/10.3390/a12050100}{Kizilay2019} & \hyperref[auth:a1380]{D. Kizilay}, \hyperref[auth:a1973]{M. F. Tasgetiren}, \hyperref[auth:a1974]{Q.-K. Pan}, \hyperref[auth:a1975]{L. Gao} & \cellcolor{gold!20}A Variable Block Insertion Heuristic for Solving Permutation Flow Shop Scheduling Problem with Makespan Criterion & \noindent{}\textcolor{black!50}{0.00} \textbf{5.00} n/a & 2019 & Algorithms & \cite{Kizilay2019}\\
Lozano2019 \href{http://dx.doi.org/10.1145/3332373}{Lozano2019} & \hyperref[auth:a1522]{R. C. Lozano}, \hyperref[auth:a91]{M. Carlsson}, \hyperref[auth:a1523]{G. H. Blindell}, \hyperref[auth:a92]{C. Schulte} & \cellcolor{green!10}Combinatorial Register Allocation and Instruction Scheduling & \noindent{}\textcolor{black!50}{0.00} \textbf{1.50} n/a & 2019 & ACM Transactions on Programming Languages and Systems & \cite{Lozano2019}\\
Lozano2019a \href{http://dx.doi.org/10.1145/3200920}{Lozano2019a} & \hyperref[auth:a1522]{R. C. Lozano}, \hyperref[auth:a92]{C. Schulte} & \cellcolor{green!10}Survey on Combinatorial Register Allocation and Instruction Scheduling & \noindent{}\textcolor{black!50}{0.00} \textbf{2.50} n/a & 2019 & ACM Computing Surveys & \cite{Lozano2019a}\\
Ozder2019 \href{http://dx.doi.org/10.3390/math7020192}{Ozder2019} & \hyperref[auth:a1753]{E. Özder}, \hyperref[auth:a1754]{E. Özcan}, \hyperref[auth:a415]{T. Eren} & \cellcolor{gold!20}Staff Task-Based Shift Scheduling Solution with an ANP and Goal Programming Method in a Natural Gas Combined Cycle Power Plant & \noindent{}\textcolor{black!50}{0.00} \textcolor{black!50}{0.00} n/a & 2019 & Mathematics & \cite{Ozder2019}\\
Schwarz2019 \href{http://dx.doi.org/10.1007/s40685-019-00102-z}{Schwarz2019} & \hyperref[auth:a2013]{K. Schwarz}, \hyperref[auth:a2014]{M. Römer}, \hyperref[auth:a2015]{T. Mellouli} & \cellcolor{gold!20}A data-driven hierarchical MILP approach for scheduling clinical pathways: a real-world case study from a German university hospital & \noindent{}\textcolor{black!50}{0.00} \textbf{1.50} n/a & 2019 & Business Research & \cite{Schwarz2019}\\
WariZ19 \href{http://dx.doi.org/10.1080/00207543.2019.1571250}{WariZ19} & \hyperref[auth:a839]{E. Wari}, \hyperref[auth:a840]{W. Zhu} & A Constraint Programming model for food processing industry: a case for an ice cream processing facility & \noindent{}\textcolor{black!50}{0.00} \textcolor{black!50}{0.00} n/a & 2019 & \cellcolor{red!20}International Journal of Production Research & \cite{WariZ19}\\
Wikarek2019 \href{http://dx.doi.org/10.3233/jifs-179364}{Wikarek2019} & \hyperref[auth:a1476]{J. Wikarek}, \hyperref[auth:a1475]{P. Sitek}, \hyperref[auth:a630]{G. Bocewicz} & Resource constrained portfolio scheduling problem (RCPoSP): A hybrid approach & \noindent{}\textcolor{black!50}{0.00} \textcolor{black!50}{0.00} n/a & 2019 & Journal of Intelligent \  Fuzzy Systems & \cite{Wikarek2019}\\
Xidias2019 \href{http://dx.doi.org/10.1017/dsi.2019.292}{Xidias2019} & \hyperref[auth:a1989]{E. Xidias}, \hyperref[auth:a1990]{P. Azariadis} & \cellcolor{gold!20}Energy Efficient Motion Design and Task Scheduling for an Autonomous Vehicle & \noindent{}\textcolor{black!50}{0.00} \textbf{2.00} n/a & 2019 & Proceedings of the Design Society: International Conference on Engineering Design & \cite{Xidias2019}\\
Zhang2019 \href{http://dx.doi.org/10.1063/1.5053623}{Zhang2019} & \hyperref[auth:a1745]{Z. Zhang}, \hyperref[auth:a1746]{M. Liu}, \hyperref[auth:a1747]{X. Song} & A bi-level fuzzy random model for multi-mode resource-constrained project scheduling problem of photovoltaic power plant & \noindent{}\textcolor{black!50}{0.00} \textcolor{black!50}{0.00} n/a & 2019 & Journal of Renewable and Sustainable Energy & \cite{Zhang2019}\\
Dasygenis2018 \href{http://dx.doi.org/10.1142/s0218213018600023}{Dasygenis2018} & \hyperref[auth:a2000]{M. Dasygenis}, \hyperref[auth:a2001]{K. Stergiou} & Methods for Parallelizing Constraint Propagation through the Use of Strong Local Consistencies & \noindent{}\textcolor{black!50}{0.00} \textbf{1.75} n/a & 2018 & International Journal on Artificial Intelligence Tools & \cite{Dasygenis2018}\\
Gao2018 \href{http://dx.doi.org/10.2355/isijinternational.isijint-2018-305}{Gao2018} & \hyperref[auth:a1712]{C. Gao}, \hyperref[auth:a1713]{D. Qu} & \cellcolor{gold!20}A Modelling and a New Hybrid MILP/CP Decomposition Method for Parallel Continuous Galvanizing Line Scheduling Problem & \noindent{}\textbf{1.00} \textbf{1.00} n/a & 2018 & ISIJ International & \cite{Gao2018}\\
GarcaNieves2018 \href{http://dx.doi.org/10.1111/mice.12356}{GarcaNieves2018} & \hyperref[auth:a1724]{J. D. García‐Nieves}, \hyperref[auth:a1725]{J. L. Ponz‐Tienda}, \hyperref[auth:a1726]{A. Salcedo‐Bernal}, \hyperref[auth:a1727]{E. Pellicer} & \cellcolor{green!10}The Multimode Resource‐Constrained Project Scheduling Problem for Repetitive Activities in Construction Projects & \noindent{}\textcolor{black!50}{0.00} \textcolor{black!50}{0.00} n/a & 2018 & Computer-Aided Civil and Infrastructure Engineering & \cite{GarcaNieves2018}\\
Li2018 \href{http://dx.doi.org/10.1109/access.2018.2859618}{Li2018} & \hyperref[auth:a1796]{H. Li}, \hyperref[auth:a1801]{Z. Li} & \cellcolor{gold!20}A Novel Strategy of Combining Variable Ordering Heuristics for Constraint Satisfaction Problems & \noindent{}0.50 0.50 n/a & 2018 & IEEE Access & \cite{Li2018}\\
Ortiz-Bayliss2018 \href{http://dx.doi.org/10.1155/2018/6103726}{Ortiz-Bayliss2018} & \hyperref[auth:a1781]{J. C. Ortiz-Bayliss}, \hyperref[auth:a1604]{I. Amaya}, \hyperref[auth:a1782]{S. E. Conant-Pablos}, \hyperref[auth:a1608]{H. Terashima-Marín} & \cellcolor{gold!20}Exploring the Impact of Early Decisions in Variable Ordering for Constraint Satisfaction Problems & \noindent{}0.50 0.50 n/a & 2018 & Computational Intelligence and Neuroscience & \cite{Ortiz-Bayliss2018}\\
Tang2018 \href{http://dx.doi.org/10.1111/mice.12383}{Tang2018} & \hyperref[auth:a555]{Y. Tang}, \hyperref[auth:a558]{Q. Sun}, \hyperref[auth:a556]{R. Liu}, \hyperref[auth:a557]{F. Wang} & Resource Leveling Based on Line of Balance and Constraint Programming & \noindent{}0.50 \textbf{1.50} n/a & 2018 & Computer-Aided Civil and Infrastructure Engineering & \cite{Tang2018}\\
Trker2018 \href{http://dx.doi.org/10.1155/2018/7870849}{Trker2018} & \hyperref[auth:a1714]{T. Türker}, \hyperref[auth:a1715]{A. Demiriz} & \cellcolor{gold!20}An Integrated Approach for Shift Scheduling and Rostering Problems with Break Times for Inbound Call Centers & \noindent{}\textcolor{black!50}{0.00} \textbf{5.00} n/a & 2018 & Mathematical Problems in Engineering & \cite{Trker2018}\\
Yvars2018 \href{http://dx.doi.org/10.1155/2018/6861429}{Yvars2018} & \hyperref[auth:a1979]{P.-A. Yvars}, \hyperref[auth:a1980]{L. Zimmer} & \cellcolor{gold!20}System Sizing with a Model-Based Approach: Application to the Optimization of a Power Transmission System & \noindent{}\textcolor{black!50}{0.00} \textbf{4.50} n/a & 2018 & Mathematical Problems in Engineering & \cite{Yvars2018}\\
Balduccini2017 \href{http://dx.doi.org/10.1017/s1471068417000102}{Balduccini2017} & \hyperref[auth:a1042]{M. Balduccini}, \hyperref[auth:a2051]{Y. Lierler} & \cellcolor{green!10}Constraint answer set solver EZCSP and why integration schemas matter & \noindent{}\textcolor{black!50}{0.00} \textbf{1.00} n/a & 2017 & Theory and Practice of Logic Programming & \cite{Balduccini2017}\\
Gonzlez2017 \href{http://dx.doi.org/10.1609/icaps.v27i1.13809}{Gonzlez2017} & \hyperref[auth:a1828]{M. Ángel González}, \hyperref[auth:a282]{A. Oddi}, \hyperref[auth:a1270]{R. Rasconi} & Multi-Objective Optimization in a Job Shop with Energy Costs through Hybrid Evolutionary Techniques & \noindent{}\textcolor{black!50}{0.00} \textbf{2.00} n/a & 2017 & Proceedings of the International Conference on Automated Planning and Scheduling & \cite{Gonzlez2017}\\
Laborie2017 \href{http://dx.doi.org/10.1609/icaps.v27i1.13844}{Laborie2017} & \hyperref[auth:a118]{P. Laborie}, \hyperref[auth:a1550]{B. Messaoudi} & New Results for the GEO-CAPE Observation Scheduling Problem & \noindent{}\textcolor{black!50}{0.00} \textbf{2.00} n/a & 2017 & Proceedings of the International Conference on Automated Planning and Scheduling & \cite{Laborie2017}\\
Morillo2017 \href{http://dx.doi.org/10.1155/2017/4627856}{Morillo2017} & \hyperref[auth:a1735]{D. Morillo}, \hyperref[auth:a271]{F. Barber}, \hyperref[auth:a153]{M. A. Salido} & \cellcolor{gold!20}Mode-Based versus Activity-Based Search for a Nonredundant Resolution of the Multimode Resource-Constrained Project Scheduling Problem & \noindent{}\textcolor{black!50}{0.00} \textcolor{black!50}{0.00} n/a & 2017 & Mathematical Problems in Engineering & \cite{Morillo2017}\\
Mutha2017 \href{http://dx.doi.org/10.1177/1748006x17744380}{Mutha2017} & \hyperref[auth:a1957]{C. Mutha}, \hyperref[auth:a1958]{C. Smidts} & Basis for non-propagation domains, their transformations and their impact on software reliability & \noindent{}\textcolor{black!50}{0.00} 0.25 n/a & 2017 & Proceedings of the Institution of Mechanical Engineers, Part O: Journal of Risk and Reliability & \cite{Mutha2017}\\
RoshanaeiLAU17a \href{http://dx.doi.org/10.1287/ijoc.2017.0745}{RoshanaeiLAU17a} & \hyperref[auth:a728]{V. Roshanaei}, \hyperref[auth:a927]{C. Luong}, \hyperref[auth:a895]{D. M. Aleman}, \hyperref[auth:a896]{D. R. Urbach} & Collaborative Operating Room Planning and Scheduling & \noindent{}\textcolor{black!50}{0.00} \textcolor{black!50}{0.00} n/a & 2017 & \cellcolor{red!20}INFORMS Journal on Computing & \cite{RoshanaeiLAU17a}\\
Sitek2017 \href{http://dx.doi.org/10.1108/imds-10-2016-0465}{Sitek2017} & \hyperref[auth:a536]{P. Sitek}, \hyperref[auth:a535]{J. Wikarek}, \hyperref[auth:a1527]{P. Nielsen} & \cellcolor{gold!20}A constraint-driven approach to food supply chain management & \noindent{}\textcolor{black!50}{0.00} \textbf{1.50} n/a & 2017 & Industrial Management \  Data Systems & \cite{Sitek2017}\\
Abdul-Niby2016 \href{http://dx.doi.org/10.48084/etasr.627}{Abdul-Niby2016} & \hyperref[auth:a1855]{M. Abdul-Niby}, \hyperref[auth:a1856]{M. Alameen}, \hyperref[auth:a1857]{A. Salhieh}, \hyperref[auth:a1858]{A. Radhi} & Improved Genetic and Simulating Annealing Algorithms to Solve the Traveling Salesman Problem Using Constraint Programming & \noindent{}\textcolor{black!50}{0.00} 0.50 n/a & 2016 & Engineering, Technology \  Applied Science Research & \cite{Abdul-Niby2016}\\
Boek2016 \href{http://dx.doi.org/10.1515/mper-2016-0003}{Boek2016} & \hyperref[auth:a1885]{A. Bożek}, \hyperref[auth:a1886]{M. Wysocki} & \cellcolor{gold!20}Off-Line and Dynamic Production Scheduling – A Comparative Case Study & \noindent{}\textcolor{black!50}{0.00} \textbf{5.00} n/a & 2016 & Management and Production Engineering Review & \cite{Boek2016}\\
Li2016 \href{http://dx.doi.org/10.3141/2549-01}{Li2016} & \hyperref[auth:a2066]{S. Li}, \hyperref[auth:a2067]{R. R. Negenborn}, \hyperref[auth:a2068]{G. Lodewijks} & Approach Integrating Mixed-Integer Programming and Constraint Programming for Planning Rotations of Inland Vessels in a Large Seaport & \noindent{}\textcolor{black!50}{0.00} \textbf{1.00} n/a & 2016 & Transportation Research Record: Journal of the Transportation Research Board & \cite{Li2016}\\
Menouer2016 \href{http://dx.doi.org/10.1002/cpe.3840}{Menouer2016} & \hyperref[auth:a1976]{T. Menouer}, \hyperref[auth:a1977]{N. Sukhija}, \hyperref[auth:a1978]{L. C. Bertrand} & A learning Portfolio solver for optimizing the performance of constraint programming problems on multi‐core computing systems & \noindent{}\textcolor{black!50}{0.00} \textbf{4.50} n/a & 2016 & Concurrency and Computation: Practice and Experience & \cite{Menouer2016}\\
Moreno-Scott2016 \href{http://dx.doi.org/10.1155/2016/7349070}{Moreno-Scott2016} & \hyperref[auth:a1783]{J. H. Moreno-Scott}, \hyperref[auth:a1781]{J. C. Ortiz-Bayliss}, \hyperref[auth:a1608]{H. Terashima-Marín}, \hyperref[auth:a1782]{S. E. Conant-Pablos} & \cellcolor{gold!20}Experimental Matching of Instances to Heuristics for Constraint Satisfaction Problems & \noindent{}\textcolor{black!50}{0.00} 0.50 n/a & 2016 & Computational Intelligence and Neuroscience & \cite{Moreno-Scott2016}\\
Ren2016 \href{http://dx.doi.org/10.1155/2016/5201937}{Ren2016} & \hyperref[auth:a1249]{H. Ren}, \hyperref[auth:a1611]{S. Sun} & \cellcolor{gold!20}A Hybrid IP/GA Approach to the Parallel Production Lines Scheduling Problem & \noindent{}\textcolor{black!50}{0.00} \textbf{2.00} n/a & 2016 & Discrete Dynamics in Nature and Society & \cite{Ren2016}\\
Sitek2016 \href{http://dx.doi.org/10.1155/2016/5102616}{Sitek2016} & \hyperref[auth:a1475]{P. Sitek}, \hyperref[auth:a1476]{J. Wikarek} & \cellcolor{gold!20}A Hybrid Programming Framework for Modeling and Solving Constraint Satisfaction and Optimization Problems & \noindent{}\textcolor{black!50}{0.00} \textbf{9.01} n/a & 2016 & Scientific Programming & \cite{Sitek2016}\\
Teschemacher2016 \href{http://dx.doi.org/10.1016/j.procir.2015.12.071}{Teschemacher2016} & \hyperref[auth:a1905]{U. Teschemacher}, \hyperref[auth:a1906]{G. Reinhart} & \cellcolor{gold!20}Enhancing Constraint Propagation in ACO-based Schedulers for Solving the Job Shop Scheduling Problem & \noindent{}\textbf{3.00} \textbf{3.00} n/a & 2016 & Procedia CIRP & \cite{Teschemacher2016}\\
Bzdyra2015 \href{http://dx.doi.org/10.4028/www.scientific.net/amm.791.70}{Bzdyra2015} & \hyperref[auth:a1813]{K. Bzdyra}, \hyperref[auth:a630]{G. Bocewicz}, \hyperref[auth:a1814]{Z. Banaszak} & Mass Customized Projects Portfolio Scheduling - Imprecise Operations Time Approach & \noindent{}\textcolor{black!50}{0.00} \textbf{2.50} n/a & 2015 & Applied Mechanics and Materials & \cite{Bzdyra2015}\\
Li2015 \href{http://dx.doi.org/10.1007/s10732-015-9305-2}{Li2015} & \hyperref[auth:a1796]{H. Li}, \hyperref[auth:a1797]{Y. Liang}, \hyperref[auth:a1798]{N. Zhang}, \hyperref[auth:a1799]{J. Guo}, \hyperref[auth:a1800]{D. Xu}, \hyperref[auth:a1801]{Z. Li} & Improving degree-based variable ordering heuristics for solving constraint satisfaction problems & \noindent{}0.50 0.50 n/a & 2015 & Journal of Heuristics & \cite{Li2015}\\
Lindauer2015 \href{http://dx.doi.org/10.1613/jair.4726}{Lindauer2015} & \hyperref[auth:a1942]{M. Lindauer}, \hyperref[auth:a1943]{H. H. Hoos}, \hyperref[auth:a1944]{F. Hutter}, \hyperref[auth:a1945]{T. Schaub} & \cellcolor{gold!20}AutoFolio: An Automatically Configured Algorithm Selector & \noindent{}\textcolor{black!50}{0.00} 0.50 n/a & 2015 & Journal of Artificial Intelligence Research & \cite{Lindauer2015}\\
Mladenovic2015 \href{http://dx.doi.org/10.1002/net.21625}{Mladenovic2015} & \hyperref[auth:a1621]{S. Mladenovic}, \hyperref[auth:a1622]{S. Veskovic}, \hyperref[auth:a1623]{I. Branovic}, \hyperref[auth:a1624]{S. Jankovic}, \hyperref[auth:a1625]{S. Acimovic} & Heuristic Based Real‐Time Train Rescheduling System & \noindent{}\textcolor{black!50}{0.00} \textbf{2.00} n/a & 2015 & Networks & \cite{Mladenovic2015}\\
Oliveira2015 \href{http://dx.doi.org/10.14807/ijmp.v6i1.262}{Oliveira2015} & \hyperref[auth:a1568]{Renata Melo e Silva de Oliveira}, \hyperref[auth:a1569]{Maria Sofia F. Oliveira de Castro Ribeiro} & Comparing Mixed \  Integer Programming vs. Constraint Programming by solving Job-Shop Scheduling Problems & \noindent{}\textbf{2.00} \textbf{2.00} n/a & 2015 & Independent Journal of Management \  Production & \cite{Oliveira2015}\\
Sahraeian2015 \href{http://dx.doi.org/10.1016/j.apm.2014.07.032}{Sahraeian2015} & \hyperref[auth:a1863]{R. Sahraeian}, \hyperref[auth:a1864]{M. Namakshenas} & \cellcolor{gold!20}On the optimal modeling and evaluation of job shops with a total weighted tardiness objective: Constraint programming vs. mixed integer programming & \noindent{}\textbf{1.00} \textbf{1.00} n/a & 2015 & Applied Mathematical Modelling & \cite{Sahraeian2015}\\
Soto2015 \href{http://dx.doi.org/10.1155/2015/580785}{Soto2015} & \hyperref[auth:a1830]{R. Soto}, \hyperref[auth:a1831]{B. Crawford}, \hyperref[auth:a1832]{W. Palma}, \hyperref[auth:a1833]{E. Monfroy}, \hyperref[auth:a1834]{R. Olivares}, \hyperref[auth:a1835]{C. Castro}, \hyperref[auth:a1836]{F. Paredes} & \cellcolor{gold!20}Top-kBased Adaptive Enumeration in Constraint Programming & \noindent{}\textcolor{black!50}{0.00} \textbf{1.00} n/a & 2015 & Mathematical Problems in Engineering & \cite{Soto2015}\\
Talbi2015 \href{http://dx.doi.org/10.1007/s10479-015-2034-y}{Talbi2015} & \hyperref[auth:a1659]{E.-G. Talbi} & Combining metaheuristics with mathematical programming, constraint programming and machine learning & \noindent{}0.50 0.50 n/a & 2015 & Annals of Operations Research & \cite{Talbi2015}\\
Wang2015 \href{http://dx.doi.org/10.3141/2482-15}{Wang2015} & \hyperref[auth:a1710]{S. Wang}, \hyperref[auth:a1711]{E. Y. Chou} & Cross-Asset Transportation Project Coordination with Integer Programming and Constraint Programming & \noindent{}\textcolor{black!50}{0.00} \textbf{1.00} n/a & 2015 & Transportation Research Record: Journal of the Transportation Research Board & \cite{Wang2015}\\
Amadini2014 \href{http://dx.doi.org/10.1017/s1471068414000179}{Amadini2014} & \hyperref[auth:a910]{R. Amadini}, \hyperref[auth:a192]{M. Gabbrielli}, \hyperref[auth:a193]{J. Mauro} & \cellcolor{green!10}SUNNY: a Lazy Portfolio Approach for Constraint Solving & \noindent{}\textcolor{black!50}{0.00} \textbf{1.50} n/a & 2014 & Theory and Practice of Logic Programming & \cite{Amadini2014}\\
Banaszak2014 \href{http://dx.doi.org/10.1515/fman-2015-0014}{Banaszak2014} & \hyperref[auth:a1814]{Z. Banaszak}, \hyperref[auth:a630]{G. Bocewicz} & Declarative Modeling for Production Order Portfolio Scheduling & \noindent{}\textcolor{black!50}{0.00} \textbf{2.50} n/a & 2014 & Foundations of Management & \cite{Banaszak2014}\\
Bergman2014 \href{http://dx.doi.org/10.1613/jair.4199}{Bergman2014} & \hyperref[auth:a1514]{D. Bergman}, \hyperref[auth:a1515]{A. A. Cire}, \hyperref[auth:a1516]{W. V. Hoeve} & \cellcolor{gold!20}MDD Propagation for Sequence Constraints & \noindent{}\textcolor{black!50}{0.00} \textbf{2.25} n/a & 2014 & Journal of Artificial Intelligence Research & \cite{Bergman2014}\\
Chaleshtarti2014 \href{http://dx.doi.org/10.1155/2014/634649}{Chaleshtarti2014} & \hyperref[auth:a1755]{A. S. Chaleshtarti}, \hyperref[auth:a1756]{S. Shadrokh}, \hyperref[auth:a1757]{Y. Fathi} & \cellcolor{gold!20}Branch and Bound Algorithms for Resource Constrained Project Scheduling Problem Subject to Nonrenewable Resources with Prescheduled Procurement & \noindent{}\textcolor{black!50}{0.00} \textcolor{black!50}{0.00} n/a & 2014 & Mathematical Problems in Engineering & \cite{Chaleshtarti2014}\\
Dolabi2014 \href{http://dx.doi.org/10.1016/j.autcon.2014.09.003}{Dolabi2014} & \hyperref[auth:a1748]{H. R. Z. Dolabi}, \hyperref[auth:a1749]{A. Afshar}, \hyperref[auth:a1750]{R. Abbasnia} & CPM/LOB Scheduling Method for Project Deadline Constraint Satisfaction & \noindent{}\textbf{1.00} \textbf{1.00} n/a & 2014 & Automation in Construction & \cite{Dolabi2014}\\
Han2014 \href{http://dx.doi.org/10.1007/s10479-014-1619-1}{Han2014} & \hyperref[auth:a1664]{A. F. Han}, \hyperref[auth:a1665]{E. C. Li} & A constraint programming-based approach to the crew scheduling problem of the Taipei mass rapid transit system & \noindent{}\textbf{1.00} \textbf{1.00} n/a & 2014 & Annals of Operations Research & \cite{Han2014}\\
Juan2014 \href{http://dx.doi.org/10.1142/s0217595914500419}{Juan2014} & \hyperref[auth:a1981]{Y.-C. Juan}, \hyperref[auth:a1982]{Y.-R. Peng} & A Constraint Satisfaction Coordination Approach for Distributed Supply Chain Production Planning & \noindent{}\textcolor{black!50}{0.00} \textbf{3.00} n/a & 2014 & Asia-Pacific Journal of Operational Research & \cite{Juan2014}\\
Kelareva2014 \href{http://dx.doi.org/10.1007/s13675-014-0022-7}{Kelareva2014} & \hyperref[auth:a332]{E. Kelareva}, \hyperref[auth:a333]{K. Tierney}, \hyperref[auth:a334]{P. Kilby} & \cellcolor{gold!20}CP methods for scheduling and routing with time-dependent task costs & \noindent{}\textcolor{black!50}{0.00} \textcolor{black!50}{0.00} n/a & 2014 & EURO Journal on Computational Optimization & \cite{Kelareva2014}\\
Lambert2014 \href{http://dx.doi.org/10.1287/inte.2013.0731}{Lambert2014} & \hyperref[auth:a1558]{W. B. Lambert}, \hyperref[auth:a1559]{A. Brickey}, \hyperref[auth:a1560]{A. M. Newman}, \hyperref[auth:a1561]{K. Eurek} & Open-Pit Block-Sequencing Formulations: A Tutorial & \noindent{}\textcolor{black!50}{0.00} \textcolor{black!50}{0.00} n/a & 2014 & \cellcolor{red!20}Interfaces & \cite{Lambert2014}\\
Levine2014 \href{http://dx.doi.org/10.1609/icaps.v24i1.13672}{Levine2014} & \hyperref[auth:a1927]{S. Levine}, \hyperref[auth:a1928]{B. Williams} & Concurrent Plan Recognition and Execution for Human-Robot Teams & \noindent{}\textcolor{black!50}{0.00} 0.50 n/a & 2014 & Proceedings of the International Conference on Automated Planning and Scheduling & \cite{Levine2014}\\
Li2014 \href{http://dx.doi.org/10.4028/www.scientific.net/amm.681.265}{Li2014} & \hyperref[auth:a1492]{Y. Li}, \hyperref[auth:a1493]{Z. R. Xiao} & A Constraint-Based Approach for Multi-Skilled Project Scheduling Problem & \noindent{}\textcolor{black!50}{0.00} \textbf{5.00} n/a & 2014 & Applied Mechanics and Materials & \cite{Li2014}\\
Li2014a \href{http://dx.doi.org/10.1177/1063293x14553809}{Li2014a} & \hyperref[auth:a2002]{Y. Li}, \hyperref[auth:a2003]{W. Zhao} & An integrated change propagation scheduling approach for product design & \noindent{}0.50 \textbf{1.50} n/a & 2014 & Concurrent Engineering & \cite{Li2014a}\\
Li2014b \href{http://dx.doi.org/10.1155/2014/169097}{Li2014b} & \hyperref[auth:a2002]{Y. Li}, \hyperref[auth:a2003]{W. Zhao}, \hyperref[auth:a2017]{Y. Ma}, \hyperref[auth:a2018]{L. Hu} & \cellcolor{gold!20}Scheduling of Changes in Complex Engineering Design Process via Genetic Algorithm and Elementary Effects Method & \noindent{}\textcolor{black!50}{0.00} \textbf{1.25} n/a & 2014 & Advances in Mechanical Engineering & \cite{Li2014b}\\
Lozano2014 \href{http://dx.doi.org/10.1145/2666357.2597815}{Lozano2014} & \hyperref[auth:a1522]{R. C. Lozano}, \hyperref[auth:a91]{M. Carlsson}, \hyperref[auth:a1523]{G. H. Blindell}, \hyperref[auth:a92]{C. Schulte} & Combinatorial spill code optimization and ultimate coalescing & \noindent{}\textcolor{black!50}{0.00} \textbf{1.00} n/a & 2014 & ACM SIGPLAN Notices & \cite{Lozano2014}\\
Silva2014 \href{http://dx.doi.org/10.1590/2238-1031.jtl.v8n4a9}{Silva2014} & \hyperref[auth:a1888]{G. P. Silva}, \hyperref[auth:a1889]{Allexandre Fortes da Silva Reis} & A study of different metaheuristics to solve the urban transit crew scheduling problem & \noindent{}\textcolor{black!50}{0.00} \textbf{2.00} n/a & 2014 & Journal of Transport Literature & \cite{Silva2014}\\
Tang2014 \href{http://dx.doi.org/10.1016/j.autcon.2013.09.008}{Tang2014} & \hyperref[auth:a555]{Y. Tang}, \hyperref[auth:a556]{R. Liu}, \hyperref[auth:a558]{Q. Sun} & Schedule control model for linear projects based on linear scheduling method and constraint programming & \noindent{}\textbf{1.00} \textbf{1.00} n/a & 2014 & Automation in Construction & \cite{Tang2014}\\
Velez2014 \href{http://dx.doi.org/10.1146/annurev-chembioeng-060713-035859}{Velez2014} & \hyperref[auth:a1480]{S. Velez}, \hyperref[auth:a381]{C. T. Maravelias} & \cellcolor{gold!20}Advances in Mixed-Integer Programming Methods for Chemical Production Scheduling & \noindent{}\textcolor{black!50}{0.00} \textbf{1.50} n/a & 2014 & Annual Review of Chemical and Biomolecular Engineering & \cite{Velez2014}\\
Wang2014 \href{http://dx.doi.org/10.1155/2014/271280}{Wang2014} & \hyperref[auth:a2022]{H. Wang}, \hyperref[auth:a2023]{X. Lu}, \hyperref[auth:a2024]{X. Zhang}, \hyperref[auth:a2025]{Q. Wang}, \hyperref[auth:a2026]{Y. Deng} & \cellcolor{gold!20}A Bio-Inspired Method for the Constrained Shortest Path Problem & \noindent{}\textcolor{black!50}{0.00} \textbf{1.00} n/a & 2014 & The Scientific World Journal & \cite{Wang2014}\\
Ammar2013 \href{http://dx.doi.org/10.1061/(asce)co.1943-7862.0000569}{Ammar2013} & \hyperref[auth:a1779]{M. A. Ammar} & LOB and CPM Integrated Method for Scheduling Repetitive Projects & \noindent{}\textcolor{black!50}{0.00} \textcolor{black!50}{0.00} n/a & 2013 & Journal of Construction Engineering and Management & \cite{Ammar2013}\\
Bocewicz2013 \href{http://dx.doi.org/10.1155/2013/407096}{Bocewicz2013} & \hyperref[auth:a630]{G. Bocewicz}, \hyperref[auth:a1913]{R. Wójcik}, \hyperref[auth:a632]{Z. A. Banaszak}, \hyperref[auth:a1914]{P. Pawlewski} & \cellcolor{gold!20}Multimodal Processes Rescheduling: Cyclic Steady States Space Approach & \noindent{}\textcolor{black!50}{0.00} \textbf{1.50} n/a & 2013 & Mathematical Problems in Engineering & \cite{Bocewicz2013}\\
Clautiaux2013 \href{http://dx.doi.org/10.1287/ijoc.1110.0478}{Clautiaux2013} & \hyperref[auth:a1686]{F. Clautiaux}, \hyperref[auth:a929]{A. Jouglet}, \hyperref[auth:a1170]{A. Moukrim} & A New Graph-Theoretical Model for the Guillotine-Cutting Problem & \noindent{}\textcolor{black!50}{0.00} 0.50 n/a & 2013 & \cellcolor{red!20}INFORMS Journal on Computing & \cite{Clautiaux2013}\\
Janosikova2013 \href{http://dx.doi.org/10.26552/com.c.2013.1.39-43}{Janosikova2013} & \hyperref[auth:a2038]{L. Janosikova}, \hyperref[auth:a2039]{T. Hreben} & Mathematical Programming vs. Constraint Programming for Scheduling Problems & \noindent{}\textbf{1.00} \textbf{1.00} n/a & 2013 & Communications - Scientific letters of the University of Zilina & \cite{Janosikova2013}\\
Lorterapong2013 \href{http://dx.doi.org/10.1061/(asce)co.1943-7862.0000582}{Lorterapong2013} & \hyperref[auth:a1792]{P. Lorterapong}, \hyperref[auth:a1793]{M. Ussavadilokrit} & Construction Scheduling Using the Constraint Satisfaction Problem Method & \noindent{}\textbf{1.00} \textbf{1.00} n/a & 2013 & Journal of Construction Engineering and Management & \cite{Lorterapong2013}\\
Nowatzki2013 \href{http://dx.doi.org/10.1145/2499370.2462163}{Nowatzki2013} & \hyperref[auth:a1631]{T. Nowatzki}, \hyperref[auth:a1632]{M. Sartin-Tarm}, \hyperref[auth:a1633]{L. D. Carli}, \hyperref[auth:a1634]{K. Sankaralingam}, \hyperref[auth:a1635]{C. Estan}, \hyperref[auth:a1636]{B. Robatmili} & A general constraint-centric scheduling framework for spatial architectures & \noindent{}\textcolor{black!50}{0.00} \textbf{1.50} n/a & 2013 & ACM SIGPLAN Notices & \cite{Nowatzki2013}\\
Ortiz-Bayliss2013 \href{http://dx.doi.org/10.1016/j.patrec.2012.09.009}{Ortiz-Bayliss2013} & \hyperref[auth:a1781]{J. C. Ortiz-Bayliss}, \hyperref[auth:a1608]{H. Terashima-Marín}, \hyperref[auth:a1782]{S. E. Conant-Pablos} & Learning vector quantization for variable ordering in constraint satisfaction problems & \noindent{}0.50 0.50 n/a & 2013 & Pattern Recognition Letters & \cite{Ortiz-Bayliss2013}\\
Pessoa2013 \href{http://dx.doi.org/10.3182/20130522-3-br-4036.00069}{Pessoa2013} & \hyperref[auth:a1669]{M. A. O. Pessoa}, \hyperref[auth:a1670]{R. A. E. Montesco}, \hyperref[auth:a1671]{F. Junqueira}, \hyperref[auth:a1672]{Diolino Jose dos Santos Filho}, \hyperref[auth:a1673]{P. E. Miyagi} & \cellcolor{gold!20}Advanced Planning and Scheduling Systems based on Time Windows and Constraint Programming & \noindent{}\textbf{1.00} \textbf{1.00} n/a & 2013 & IFAC Proceedings Volumes & \cite{Pessoa2013}\\
Shobaki2013 \href{http://dx.doi.org/10.1145/2512432}{Shobaki2013} & \hyperref[auth:a1784]{G. Shobaki}, \hyperref[auth:a1785]{M. Shawabkeh}, \hyperref[auth:a1786]{N. E. A. Rmaileh} & \cellcolor{gold!20}Preallocation instruction scheduling with register pressure minimization using a combinatorial optimization approach & \noindent{}\textcolor{black!50}{0.00} \textcolor{black!50}{0.00} n/a & 2013 & ACM Transactions on Architecture and Code Optimization & \cite{Shobaki2013}\\
Talbi2013 \href{http://dx.doi.org/10.1007/s10288-013-0242-3}{Talbi2013} & \hyperref[auth:a1659]{E.-G. Talbi} & Combining metaheuristics with mathematical programming, constraint programming and machine learning & \noindent{}0.50 0.50 n/a & 2013 & 4OR & \cite{Talbi2013}\\
Velez2013 \href{http://dx.doi.org/10.1002/aic.14021}{Velez2013} & \hyperref[auth:a1480]{S. Velez}, \hyperref[auth:a1481]{A. Sundaramoorthy}, \hyperref[auth:a381]{C. T. Maravelias} & Valid Inequalities Based on Demand Propagation for Chemical Production Scheduling MIP Models & \noindent{}0.50 \textbf{3.75} n/a & 2013 & AIChE Journal & \cite{Velez2013}\\
Wang2013 \href{http://dx.doi.org/10.4028/www.scientific.net/amm.357-360.2720}{Wang2013} & \hyperref[auth:a1903]{H. S. Wang}, \hyperref[auth:a1904]{S. S. Liu} & Road Inspection Scheduling Model Using Constraint Programming & \noindent{}\textbf{1.00} \textbf{3.00} n/a & 2013 & Applied Mechanics and Materials & \cite{Wang2013}\\
Zhang2013 \href{http://dx.doi.org/10.5772/55956}{Zhang2013} & \hyperref[auth:a1517]{R. Zhang} & \cellcolor{gold!20}A Simulated Annealing-Based Heuristic Algorithm for Job Shop Scheduling to Minimize Lateness & \noindent{}\textcolor{black!50}{0.00} \textbf{3.00} n/a & 2013 & International Journal of Advanced Robotic Systems & \cite{Zhang2013}\\
Zoulfaghari2013 \href{http://dx.doi.org/10.4018/jaec.2013040103}{Zoulfaghari2013} & \hyperref[auth:a1758]{H. Zoulfaghari}, \hyperref[auth:a1759]{J. Nematian}, \hyperref[auth:a1760]{N. Mahmoudi}, \hyperref[auth:a1761]{M. Khodabandeh} & A New Genetic Algorithm for the RCPSP in Large Scale & \noindent{}\textcolor{black!50}{0.00} \textcolor{black!50}{0.00} n/a & 2013 & International Journal of Applied Evolutionary Computation & \cite{Zoulfaghari2013}\\
Berbeglia2012 \href{http://dx.doi.org/10.1287/ijoc.1110.0454}{Berbeglia2012} & \hyperref[auth:a1847]{G. Berbeglia}, \hyperref[auth:a1848]{J.-F. Cordeau}, \hyperref[auth:a1074]{G. Laporte} & A Hybrid Tabu Search and Constraint Programming Algorithm for the Dynamic Dial-a-Ride Problem & \noindent{}\textcolor{black!50}{0.00} \textbf{1.00} n/a & 2012 & \cellcolor{red!20}INFORMS Journal on Computing & \cite{Berbeglia2012}\\
Eirinakis2012 \href{http://dx.doi.org/10.1287/ijoc.1110.0449}{Eirinakis2012} & \hyperref[auth:a1916]{P. Eirinakis}, \hyperref[auth:a1917]{D. Magos}, \hyperref[auth:a1918]{I. Mourtos}, \hyperref[auth:a1919]{P. Miliotis} & Finding All Stable Pairs and Solutions to the Many-to-Many Stable Matching Problem & \noindent{}\textcolor{black!50}{0.00} \textbf{2.00} n/a & 2012 & \cellcolor{red!20}INFORMS Journal on Computing & \cite{Eirinakis2012}\\
Filho2012 \href{http://dx.doi.org/10.1016/j.eswa.2011.07.027}{Filho2012} & \hyperref[auth:a1949]{Cicero Ferreira Fernandes Costa Filho}, \hyperref[auth:a1950]{D. A. R. Rocha}, \hyperref[auth:a1951]{M. G. F. Costa}, \hyperref[auth:a1952]{Wagner Coelho de Albuquerque Pereira} & Using Constraint Satisfaction Problem approach to solve human resource allocation problems in cooperative health services & \noindent{}0.50 0.50 n/a & 2012 & Expert Systems with Applications & \cite{Filho2012}\\
Hoc2012 \href{http://dx.doi.org/10.1002/hfm.20359}{Hoc2012} & \hyperref[auth:a2009]{J. Hoc}, \hyperref[auth:a2010]{C. Guerin}, \hyperref[auth:a2011]{N. Mebarki} & The Nature of Expertise in Scheduling: The Case of Timetabling & \noindent{}\textcolor{black!50}{0.00} \textbf{1.50} n/a & 2012 & Human Factors and Ergonomics in Manufacturing \  Service Industries & \cite{Hoc2012}\\
Junker2012 \href{http://dx.doi.org/10.1017/s0269888912000240}{Junker2012} & \hyperref[auth:a1326]{U. Junker} & Air traffic flow management with heuristic repair & \noindent{}\textcolor{black!50}{0.00} 0.50 n/a & 2012 & The Knowledge Engineering Review & \cite{Junker2012}\\
Kelareva2012 \href{http://dx.doi.org/10.1609/icaps.v22i1.13494}{Kelareva2012} & \hyperref[auth:a332]{E. Kelareva}, \hyperref[auth:a855]{S. Brand}, \hyperref[auth:a334]{P. Kilby}, \hyperref[auth:a1518]{S. Thiebaux}, \hyperref[auth:a1519]{M. Wallace} & CP and MIP Methods for Ship Scheduling with Time-Varying Draft & \noindent{}\textcolor{black!50}{0.00} \textbf{3.00} n/a & 2012 & Proceedings of the International Conference on Automated Planning and Scheduling & \cite{Kelareva2012}\\
Martin2012 \href{http://dx.doi.org/10.1145/2209285.2209289}{Martin2012} & \hyperref[auth:a1578]{K. Martin}, \hyperref[auth:a659]{C. Wolinski}, \hyperref[auth:a660]{K. Kuchcinski}, \hyperref[auth:a1579]{A. Floch}, \hyperref[auth:a1532]{F. Charot} & Constraint Programming Approach to Reconfigurable Processor Extension Generation and Application Compilation & \noindent{}\textcolor{black!50}{0.00} \textbf{2.00} n/a & 2012 & ACM Transactions on Reconfigurable Technology and Systems & \cite{Martin2012}\\
Pesant2012 \href{http://dx.doi.org/10.1613/jair.3463}{Pesant2012} & \hyperref[auth:a1586]{G. Pesant}, \hyperref[auth:a1587]{C. Quimper}, \hyperref[auth:a1588]{A. Zanarini} & \cellcolor{gold!20}Counting-Based Search: Branching Heuristics for Constraint Satisfaction Problems & \noindent{}\textcolor{black!50}{0.00} \textbf{2.00} n/a & 2012 & Journal of Artificial Intelligence Research & \cite{Pesant2012}\\
Pinto2012 \href{http://dx.doi.org/10.1007/s10479-012-1256-5}{Pinto2012} & \hyperref[auth:a1598]{G. Pinto}, \hyperref[auth:a1599]{Y. T. Ben-Dov}, \hyperref[auth:a1600]{G. Rabinowitz} & Formulating and solving a multi-mode resource-collaboration and constrained scheduling problem (MRCCSP) & \noindent{}\textcolor{black!50}{0.00} \textcolor{black!50}{0.00} n/a & 2012 & Annals of Operations Research & \cite{Pinto2012}\\
Raffin2012 \href{http://dx.doi.org/10.4018/jertcs.2012010101}{Raffin2012} & \hyperref[auth:a1531]{E. Raffin}, \hyperref[auth:a659]{C. Wolinski}, \hyperref[auth:a1532]{F. Charot}, \hyperref[auth:a1533]{E. Casseau}, \hyperref[auth:a1534]{A. Floc’h}, \hyperref[auth:a660]{K. Kuchcinski}, \hyperref[auth:a1535]{S. Chevobbe}, \hyperref[auth:a1536]{S. Guyetant} & \cellcolor{green!10}Scheduling, Binding and Routing System for a Run-Time Reconfigurable Operator Based Multimedia Architecture & \noindent{}\textcolor{black!50}{0.00} \textbf{2.50} n/a & 2012 & International Journal of Embedded and Real-Time Communication Systems & \cite{Raffin2012}\\
ShangGuan2012 \href{http://dx.doi.org/10.4028/www.scientific.net/amr.443-444.724}{ShangGuan2012} & \hyperref[auth:a1983]{C. X. ShangGuan}, \hyperref[auth:a1984]{J. T. Li}, \hyperref[auth:a1985]{R. F. Shi} & Rescheduling of Parallel Machines under Machine Failures & \noindent{}\textcolor{black!50}{0.00} \textbf{2.50} n/a & 2012 & Advanced Materials Research & \cite{ShangGuan2012}\\
ZarandiB12 \href{http://dx.doi.org/10.1287/ijoc.1110.0458}{ZarandiB12} & \hyperref[auth:a945]{M. M. Fazel-Zarandi}, \hyperref[auth:a89]{J. C. Beck} & Using Logic-Based Benders Decomposition to Solve the Capacity- and Distance-Constrained Plant Location Problem & \noindent{}\textcolor{black!50}{0.00} \textcolor{black!50}{0.00} n/a & 2012 & \cellcolor{red!20}INFORMS Journal on Computing & \cite{ZarandiB12}\\
Zou2012 \href{http://dx.doi.org/10.14778/2535568.2448945}{Zou2012} & \hyperref[auth:a2054]{T. Zou}, \hyperref[auth:a2055]{R. L. Bras}, \hyperref[auth:a2056]{M. V. Salles}, \hyperref[auth:a2057]{A. Demers}, \hyperref[auth:a2058]{J. Gehrke} & ClouDiA & \noindent{}\textcolor{black!50}{0.00} \textbf{1.00} n/a & 2012 & Proceedings of the VLDB Endowment & \cite{Zou2012}\\
Acuna-Agost2011 \href{http://dx.doi.org/10.1016/j.ejor.2011.05.047}{Acuna-Agost2011} & \hyperref[auth:a354]{R. Acuna-Agost}, \hyperref[auth:a355]{P. Michelon}, \hyperref[auth:a356]{D. Feillet}, \hyperref[auth:a357]{S. Gueye} & SAPI: Statistical Analysis of Propagation of Incidents. A new approach for rescheduling trains after disruptions & \noindent{}0.50 0.50 n/a & 2011 & European Journal of Operational Research & \cite{Acuna-Agost2011}\\
Artigues2011 \href{http://dx.doi.org/10.1016/j.engappai.2010.07.008}{Artigues2011} & \hyperref[auth:a6]{C. Artigues}, \hyperref[auth:a1199]{M.-J. Huguet}, \hyperref[auth:a3]{P. Lopez} & \cellcolor{green!10}Generalized disjunctive constraint propagation for solving the job shop problem with time lags & \noindent{}\textbf{1.50} \textbf{1.50} n/a & 2011 & Engineering Applications of Artificial Intelligence & \cite{Artigues2011}\\
Bourdeaudhuy2011 \href{http://dx.doi.org/10.1080/00207543.2010.519113}{Bourdeaudhuy2011} & \hyperref[auth:a1650]{T. Bourdeaud'huy}, \hyperref[auth:a1651]{O. Belkahla}, \hyperref[auth:a681]{P. Yim}, \hyperref[auth:a680]{O. Korbaa}, \hyperref[auth:a1652]{K. Ghedira} & Transient inter-production scheduling based on Petri nets and constraint programming & \noindent{}\textbf{1.00} \textbf{1.00} n/a & 2011 & \cellcolor{red!20}International Journal of Production Research & \cite{Bourdeaudhuy2011}\\
Chun2011 \href{http://dx.doi.org/10.1609/aimag.v32i2.2346}{Chun2011} & \hyperref[auth:a1322]{A. H. W. Chun} & \cellcolor{gold!20}Optimizing Limousine Service with AI & \noindent{}\textcolor{black!50}{0.00} \textbf{1.50} n/a & 2011 & AI Magazine & \cite{Chun2011}\\
Coelho2011 \href{http://dx.doi.org/10.1016/j.ejor.2011.03.019}{Coelho2011} & \hyperref[auth:a1555]{J. Coelho}, \hyperref[auth:a1556]{M. Vanhoucke} & \cellcolor{green!10}Multi-mode resource-constrained project scheduling using RCPSP and SAT solvers & \noindent{}\textcolor{black!50}{0.00} \textcolor{black!50}{0.00} n/a & 2011 & European Journal of Operational Research & \cite{Coelho2011}\\
Deblaere2011 \href{http://dx.doi.org/10.1016/j.cor.2010.01.001}{Deblaere2011} & \hyperref[auth:a1775]{F. Deblaere}, \hyperref[auth:a1090]{E. Demeulemeester}, \hyperref[auth:a1102]{W. Herroelen} & Reactive scheduling in the multi-mode RCPSP & \noindent{}\textcolor{black!50}{0.00} \textcolor{black!50}{0.00} n/a & 2011 & Computers \  Operations Research & \cite{Deblaere2011}\\
EdisO11a \href{http://dx.doi.org/10.1080/03052151003759117}{EdisO11a} & \hyperref[auth:a346]{E. B. Edis}, \hyperref[auth:a348]{I. Ozkarahan} & A combined integer/constraint programming approach to a resource-constrained parallel machine scheduling problem with machine eligibility restrictions & \noindent{}\textbf{2.00} \textbf{2.00} n/a & 2011 & \cellcolor{red!20}Engineering Optimization & \cite{EdisO11a}\\
Hat2011 \href{http://dx.doi.org/10.1504/ejie.2011.042742}{Hat2011} & \hyperref[auth:a1162]{A. Haït}, \hyperref[auth:a6]{C. Artigues} & \cellcolor{green!10}A hybrid CP/MILP method for scheduling with energy costs & \noindent{}\textbf{1.00} \textbf{1.00} n/a & 2011 & European J. of Industrial Engineering & \cite{Hat2011}\\
Lizarralde2011 \href{http://dx.doi.org/10.3917/proj.007.0089}{Lizarralde2011} & \hyperref[auth:a1478]{I. Lizarralde}, \hyperref[auth:a1248]{P. Esquirol}, \hyperref[auth:a1479]{A. Rivière} & A decision support system to schedule design activities with interdependency and resource constraints & \noindent{}\textcolor{black!50}{0.00} \textbf{4.50} n/a & 2011 & Projectics / Proyéctica / Projectique & \cite{Lizarralde2011}\\
Bartk2010 \href{http://dx.doi.org/10.1177/0142331208100099}{Bartk2010} & \hyperref[auth:a1063]{R. Barták}, \hyperref[auth:a1557]{O. Čepek} & \cellcolor{green!10}Incremental propagation rules for a precedence graph with optional activities and time windows & \noindent{}\textcolor{black!50}{0.00} 0.75 n/a & 2010 & Transactions of the Institute of Measurement and Control & \cite{Bartk2010}\\
Biswas2010 \href{http://dx.doi.org/10.1177/0037549710373601}{Biswas2010} & \hyperref[auth:a2019]{M. Biswas}, \hyperref[auth:a2020]{M. R. Frater}, \hyperref[auth:a2021]{M. Ryan} & Determination of hub port attenuation satisfying radio link path losses for hardware emulation & \noindent{}\textcolor{black!50}{0.00} \textbf{1.00} n/a & 2010 & SIMULATION & \cite{Biswas2010}\\
LiuGT10 \href{http://dx.doi.org/10.3724/sp.j.1004.2010.00603}{LiuGT10} & \hyperref[auth:a1220]{S.-X. Liu}, \hyperref[auth:a1221]{Z. Guo}, \hyperref[auth:a1222]{J.-F. Tang} & Constraint Propagation for Cumulative Scheduling Problems with Precedences: Constraint Propagation for Cumulative Scheduling Problems with Precedences & \noindent{}\textbf{1.50} \textbf{1.50} n/a & 2010 & \cellcolor{red!20}Acta Automatica Sinica & \cite{LiuGT10}\\
Magato2010 \href{http://dx.doi.org/10.1007/s10951-010-0186-9}{Magato2010} & \hyperref[auth:a1808]{L. Magatão}, \hyperref[auth:a1809]{L. V. R. Arruda}, \hyperref[auth:a1810]{F. Neves-Jr} & A combined CLP-MILP approach for scheduling commodities in a pipeline & \noindent{}\textbf{1.00} \textbf{1.00} n/a & 2010 & Journal of Scheduling & \cite{Magato2010}\\
Refanidis2010 \href{http://dx.doi.org/10.1145/1869397.1869401}{Refanidis2010} & \hyperref[auth:a1546]{I. Refanidis}, \hyperref[auth:a19]{N. Yorke-Smith} & A constraint-based approach to scheduling an individual's activities & \noindent{}\textcolor{black!50}{0.00} \textbf{2.00} n/a & 2010 & ACM Transactions on Intelligent Systems and Technology & \cite{Refanidis2010}\\
Verfaillie2010 \href{http://dx.doi.org/10.1017/s0269888910000172}{Verfaillie2010} & \hyperref[auth:a1722]{G. Verfaillie}, \hyperref[auth:a1897]{C. Pralet}, \hyperref[auth:a2052]{M. Lemaître} & How to model planning and scheduling problems using constraint networks on timelines & \noindent{}\textcolor{black!50}{0.00} \textbf{1.00} n/a & 2010 & The Knowledge Engineering Review & \cite{Verfaillie2010}\\
Bocewicz2009 \href{http://dx.doi.org/10.1108/03684920910976989}{Bocewicz2009} & \hyperref[auth:a630]{G. Bocewicz}, \hyperref[auth:a631]{I. Bach}, \hyperref[auth:a1913]{R. Wójcik} & Production flow prototyping subject to imprecise activity specification & \noindent{}\textcolor{black!50}{0.00} \textbf{13.51} n/a & 2009 & Kybernetes & \cite{Bocewicz2009}\\
Capone2009 \href{http://dx.doi.org/10.1002/net.20367}{Capone2009} & \hyperref[auth:a1563]{A. Capone}, \hyperref[auth:a1564]{G. Carello}, \hyperref[auth:a1565]{I. Filippini}, \hyperref[auth:a1566]{S. Gualandi}, \hyperref[auth:a1567]{F. Malucelli} & Solving a resource allocation problem in wireless mesh networks: A comparison between a CP‐based and a classical column generation & \noindent{}0.50 \textbf{1.00} n/a & 2009 & Networks & \cite{Capone2009}\\
Michel2009 \href{http://dx.doi.org/10.1287/ijoc.1080.0313}{Michel2009} & \hyperref[auth:a32]{L. Michel}, \hyperref[auth:a1807]{A. See}, \hyperref[auth:a148]{P. V. Hentenryck} & Transparent Parallelization of Constraint Programming & \noindent{}\textcolor{black!50}{0.00} 0.50 n/a & 2009 & \cellcolor{red!20}INFORMS Journal on Computing & \cite{Michel2009}\\
Smith-Miles2009 \href{http://dx.doi.org/10.1145/1456650.1456656}{Smith-Miles2009} & \hyperref[auth:a1742]{K. A. Smith-Miles} & Cross-disciplinary perspectives on meta-learning for algorithm selection & \noindent{}\textcolor{black!50}{0.00} \textbf{1.50} n/a & 2009 & ACM Computing Surveys & \cite{Smith-Miles2009}\\
Yang2009 \href{http://dx.doi.org/10.1007/s10951-009-0106-z}{Yang2009} & \hyperref[auth:a1823]{S. Yang}, \hyperref[auth:a1824]{D. Wang}, \hyperref[auth:a1825]{T. Chai}, \hyperref[auth:a1387]{G. Kendall} & \cellcolor{green!10}An improved constraint satisfaction adaptive neural network for job-shop scheduling & \noindent{}\textbf{2.00} \textbf{2.00} n/a & 2009 & Journal of Scheduling & \cite{Yang2009}\\
Banaszak2008 \href{http://dx.doi.org/10.7494/dmms.2008.2.2.5}{Banaszak2008} & \hyperref[auth:a1814]{Z. Banaszak}, \hyperref[auth:a630]{G. Bocewicz}, \hyperref[auth:a631]{I. Bach} & CP-driven Production Process Planning in Multiproject Environment & \noindent{}\textcolor{black!50}{0.00} \textbf{4.00} n/a & 2008 & Decision Making in Manufacturing and Services & \cite{Banaszak2008}\\
Magato2008 \href{http://dx.doi.org/10.1590/s0101-74382008000300007}{Magato2008} & \hyperref[auth:a1637]{L. Magatão}, \hyperref[auth:a1638]{Lúcia Valéria Ramos de Arruda}, \hyperref[auth:a1639]{F. Neves-Jr} & \cellcolor{gold!20}Um modelo híbrido (CLP-MILP) para scheduling de operações em polidutos & \noindent{}\textbf{1.00} \textbf{2.00} n/a & 2008 & Pesquisa Operacional & \cite{Magato2008}\\
Petrovic2008 \href{http://dx.doi.org/10.1142/s0218213008004023}{Petrovic2008} & \hyperref[auth:a1861]{S. Petrovic}, \hyperref[auth:a1862]{S. L. Epstein} & \cellcolor{green!10}Random subsets support learning a mixture of heuristics & \noindent{}\textcolor{black!50}{0.00} 0.50 n/a & 2008 & International Journal on Artificial Intelligence Tools & \cite{Petrovic2008}\\
Salido2008 \href{http://dx.doi.org/10.1016/j.engappai.2008.03.007}{Salido2008} & \hyperref[auth:a153]{M. A. Salido}, \hyperref[auth:a633]{A. Garrido}, \hyperref[auth:a1063]{R. Barták} & Introduction: Special issue on constraint satisfaction techniques for planning and scheduling problems & \noindent{}\textbf{1.00} \textbf{1.00} n/a & 2008 & Engineering Applications of Artificial Intelligence & \cite{Salido2008}\\
Salido2008a \href{http://dx.doi.org/10.1016/j.engappai.2008.03.006}{Salido2008a} & \hyperref[auth:a153]{M. A. Salido}, \hyperref[auth:a1941]{A. Giret} & \cellcolor{green!10}Feasible distributed CSP models for scheduling problems & \noindent{}\textbf{1.00} \textbf{1.00} n/a & 2008 & Engineering Applications of Artificial Intelligence & \cite{Salido2008a}\\
Tseng2008 \href{http://dx.doi.org/10.1145/1367045.1367061}{Tseng2008} & \hyperref[auth:a1682]{I.-L. Tseng}, \hyperref[auth:a1683]{A. Postula} & Partitioning parameterized 45-degree polygons with constraint programming & \noindent{}\textcolor{black!50}{0.00} 0.50 n/a & 2008 & ACM Transactions on Design Automation of Electronic Systems & \cite{Tseng2008}\\
Wallace2008 \href{http://dx.doi.org/10.1142/s0218213008004199}{Wallace2008} & \hyperref[auth:a1268]{R. J. Wallace} & Determining the principles underlying performance variation in csp heuristics & \noindent{}\textcolor{black!50}{0.00} 0.25 n/a & 2008 & International Journal on Artificial Intelligence Tools & \cite{Wallace2008}\\
Wu2008 \href{http://dx.doi.org/10.1142/s0218213008004187}{Wu2008} & \hyperref[auth:a2060]{H. Wu}, \hyperref[auth:a2061]{P. V. Beek} & Portfolios with deadlines for backtracking search & \noindent{}\textcolor{black!50}{0.00} \textbf{1.50} n/a & 2008 & International Journal on Artificial Intelligence Tools & \cite{Wu2008}\\
Choi2007 \href{http://dx.doi.org/10.1145/1276920.1276925}{Choi2007} & \hyperref[auth:a1816]{C. W. Choi}, \hyperref[auth:a1817]{J. H. M. Lee}, \hyperref[auth:a1818]{P. J. Stuckey} & \cellcolor{green!10}Removing propagation redundant constraints in redundant modeling & \noindent{}\textcolor{black!50}{0.00} \textbf{1.25} n/a & 2007 & ACM Transactions on Computational Logic & \cite{Choi2007}\\
Lallouet2007 \href{http://dx.doi.org/10.1142/s0218213007003503}{Lallouet2007} & \hyperref[auth:a428]{A. Lallouet}, \hyperref[auth:a1935]{A. Legtchenko} & Building consistencies for partially defined constraints with decision trees and neural networks & \noindent{}\textcolor{black!50}{0.00} 0.25 n/a & 2007 & International Journal on Artificial Intelligence Tools & \cite{Lallouet2007}\\
Mladenovic2007 \href{http://dx.doi.org/10.2298/yjor0701009m}{Mladenovic2007} & \hyperref[auth:a1621]{S. Mladenovic}, \hyperref[auth:a1717]{M. Cangalovic} & \cellcolor{gold!20}Heuristic approach to train rescheduling & \noindent{}\textcolor{black!50}{0.00} \textbf{3.00} n/a & 2007 & Yugoslav Journal of Operations Research & \cite{Mladenovic2007}\\
Wang2007 \href{http://dx.doi.org/10.1016/j.omega.2005.06.001}{Wang2007} & \hyperref[auth:a1936]{S. M. Wang}, \hyperref[auth:a1937]{J. C. Chen}, \hyperref[auth:a1938]{K.-J. Wang} & Resource portfolio planning of make-to-stock products using a constraint programming-based genetic algorithm & \noindent{}0.50 0.50 n/a & 2007 & Omega & \cite{Wang2007}\\
Bidot2006 \href{http://dx.doi.org/10.3182/20060517-3-fr-2903.00313}{Bidot2006} & \hyperref[auth:a824]{J. Bidot}, \hyperref[auth:a118]{P. Laborie}, \hyperref[auth:a89]{J. C. Beck}, \hyperref[auth:a825]{T. Vidal} & \cellcolor{gold!20}Using constraint programming and simulation for execution monitoring and progressive scheduling & \noindent{}\textbf{1.00} \textbf{1.00} n/a & 2006 & IFAC Proceedings Volumes & \cite{Bidot2006}\\
Elkhyari2006 \href{http://dx.doi.org/10.3182/20060517-3-fr-2903.00358}{Elkhyari2006} & \hyperref[auth:a292]{A. Elkhyari}, \hyperref[auth:a2069]{C. Combes} & Using constraint programming for solving dynamic scheduling in the endoscopy unit & \noindent{}\textbf{1.00} \textbf{1.00} n/a & 2006 & IFAC Proceedings Volumes & \cite{Elkhyari2006}\\
Frisch2006 \href{http://dx.doi.org/10.1016/j.artint.2006.03.002}{Frisch2006} & \hyperref[auth:a1666]{A. M. Frisch}, \hyperref[auth:a137]{B. Hnich}, \hyperref[auth:a97]{Z. Kiziltan}, \hyperref[auth:a1667]{I. Miguel}, \hyperref[auth:a276]{T. Walsh} & \cellcolor{gold!20}Propagation algorithms for lexicographic ordering constraints & \noindent{}0.25 0.25 n/a & 2006 & Artificial Intelligence & \cite{Frisch2006}\\
Trilling2006 \href{http://dx.doi.org/10.3182/20060517-3-fr-2903.00340}{Trilling2006} & \hyperref[auth:a1656]{L. Trilling}, \hyperref[auth:a1657]{A. Guinet}, \hyperref[auth:a1658]{D. L. Magny} & Nurse scheduling using integer linear programming and constraint programming & \noindent{}\textbf{1.00} \textbf{1.00} n/a & 2006 & IFAC Proceedings Volumes & \cite{Trilling2006}\\
Zhu2006 \href{http://dx.doi.org/10.1287/ijoc.1040.0121}{Zhu2006} & \hyperref[auth:a1528]{G. Zhu}, \hyperref[auth:a1529]{J. F. Bard}, \hyperref[auth:a1530]{G. Yu} & A Branch-and-Cut Procedure for the Multimode Resource-Constrained Project-Scheduling Problem & \noindent{}\textcolor{black!50}{0.00} \textcolor{black!50}{0.00} n/a & 2006 & \cellcolor{red!20}INFORMS Journal on Computing & \cite{Zhu2006}\\
Moccia2005 \href{http://dx.doi.org/10.1002/nav.20121}{Moccia2005} & \hyperref[auth:a1589]{L. Moccia}, \hyperref[auth:a1590]{J. Cordeau}, \hyperref[auth:a1591]{M. Gaudioso}, \hyperref[auth:a1074]{G. Laporte} & A branch‐and‐cut algorithm for the quay crane scheduling problem in a container terminal & \noindent{}\textcolor{black!50}{0.00} \textcolor{black!50}{0.00} n/a & 2005 & Naval Research Logistics (NRL) & \cite{Moccia2005}\\
Yunes2005 \href{http://dx.doi.org/10.1287/trsc.1030.0078}{Yunes2005} & \hyperref[auth:a942]{T. H. Yunes}, \hyperref[auth:a1580]{A. V. Moura}, \hyperref[auth:a170]{C. C. de Souza} & Hybrid Column Generation Approaches for Urban Transit Crew Management Problems & \noindent{}\textcolor{black!50}{0.00} \textbf{2.00} n/a & 2005 & \cellcolor{red!20}Transportation Science & \cite{Yunes2005}\\
Hindi2004 \href{http://dx.doi.org/10.1016/j.cie.2004.03.002}{Hindi2004} & \hyperref[auth:a1826]{K. S. Hindi}, \hyperref[auth:a1827]{K. Fleszar} & A constraint propagation heuristic for the single-hoist, multiple-products scheduling problem & \noindent{}\textbf{1.50} \textbf{1.50} n/a & 2004 & Computers \  Industrial Engineering & \cite{Hindi2004}\\
Kim2004 \href{http://dx.doi.org/10.1016/j.cie.2003.12.017}{Kim2004} & \hyperref[auth:a2029]{K. H. Kim}, \hyperref[auth:a2030]{K. W. Kim}, \hyperref[auth:a2031]{H. Hwang}, \hyperref[auth:a2032]{C. S. Ko} & Operator-scheduling using a constraint satisfaction technique in port container terminals & \noindent{}\textbf{1.00} \textbf{1.00} n/a & 2004 & Computers \  Industrial Engineering & \cite{Kim2004}\\
Lim2004 \href{http://dx.doi.org/10.1002/nav.10123}{Lim2004} & \hyperref[auth:a279]{A. Lim}, \hyperref[auth:a280]{B. Rodrigues}, \hyperref[auth:a1743]{F. Xiao}, \hyperref[auth:a1744]{Y. Zhu} & Crane scheduling with spatial constraints & \noindent{}\textcolor{black!50}{0.00} \textcolor{black!50}{0.00} n/a & 2004 & Naval Research Logistics (NRL) & \cite{Lim2004}\\
Michel2004 \href{http://dx.doi.org/10.1145/976706.976714}{Michel2004} & \hyperref[auth:a32]{L. Michel}, \hyperref[auth:a148]{P. V. Hentenryck} & A decomposition-based implementation of search strategies & \noindent{}\textcolor{black!50}{0.00} \textbf{2.00} n/a & 2004 & ACM Transactions on Computational Logic & \cite{Michel2004}\\
OkanoDTRYA04 \href{https://doi.org/10.1147/rd.485.0811}{OkanoDTRYA04} & \hyperref[auth:a1288]{H. Okano}, \hyperref[auth:a248]{A. J. Davenport}, \hyperref[auth:a1289]{M. Trumbo}, \hyperref[auth:a250]{C. Reddy}, \hyperref[auth:a1290]{K. Yoda}, \hyperref[auth:a1291]{M. Amano} & Finishing Line Scheduling in the steel industry & \noindent{}\textcolor{black!50}{0.00} \textcolor{black!50}{0.00} n/a & 2004 & {IBM} J. Res. Dev. & \cite{OkanoDTRYA04}\\
Ouaja2004 \href{http://dx.doi.org/10.1002/net.10110}{Ouaja2004} & \hyperref[auth:a1548]{W. Ouaja}, \hyperref[auth:a1549]{B. Richards} & \cellcolor{gold!20}A hybrid multicommodity routing algorithm for traffic engineering & \noindent{}\textcolor{black!50}{0.00} \textbf{1.50} n/a & 2004 & Networks & \cite{Ouaja2004}\\
Kovcs2003 \href{http://dx.doi.org/10.1016/s1474-6670(17)37762-5}{Kovcs2003} & \hyperref[auth:a1880]{A. Kovács}, \hyperref[auth:a1881]{J. Váncza}, \hyperref[auth:a1882]{B. Kádár}, \hyperref[auth:a1883]{L. Monostori}, \hyperref[auth:a1884]{A. Pfeiffer} & Real-Life Scheduling Using Constraint Programming and Simulation & \noindent{}\textbf{1.00} \textbf{1.00} n/a & 2003 & IFAC Proceedings Volumes & \cite{Kovcs2003}\\
Priore2003 \href{http://dx.doi.org/10.1108/09576060310459456}{Priore2003} & \hyperref[auth:a1819]{P. Priore}, \hyperref[auth:a1820]{D. de la Fuente}, \hyperref[auth:a1821]{R. Pino}, \hyperref[auth:a1822]{J. Puente} & Dynamic scheduling of flexible manufacturing systems using neural networks and inductive learning & \noindent{}\textcolor{black!50}{0.00} \textbf{1.25} n/a & 2003 & Integrated Manufacturing Systems & \cite{Priore2003}\\
Sadykov2003 \href{http://dx.doi.org/10.2139/ssrn.988640}{Sadykov2003} & \hyperref[auth:a384]{R. Sadykov}, \hyperref[auth:a224]{L. A. Wolsey} & Integer Programming and Constraint Programming in Solving a Multi-Machine Assignment Scheduling Problem With Deadlines and Release Dates & \noindent{}\textbf{1.50} \textbf{1.50} n/a & 2003 & SSRN Electronic Journal & \cite{Sadykov2003}\\
Yan2003 \href{http://dx.doi.org/10.1007/bf02948893}{Yan2003} & \hyperref[auth:a2033]{J. Yan}, \hyperref[auth:a2034]{C. Wu} & A constraint satisfaction neural network and heuristic combined approach for concurrent activities scheduling & \noindent{}\textbf{1.00} \textbf{1.00} n/a & 2003 & Journal of Computer Science and Technology & \cite{Yan2003}\\
Younes2003 \href{http://dx.doi.org/10.1613/jair.1136}{Younes2003} & \hyperref[auth:a1844]{H. L. S. Younes}, \hyperref[auth:a1845]{R. G. Simmons} & \cellcolor{gold!20}VHPOP: Versatile Heuristic Partial Order Planner & \noindent{}\textcolor{black!50}{0.00} 0.50 n/a & 2003 & Journal of Artificial Intelligence Research & \cite{Younes2003}\\
Brucker2002 \href{http://dx.doi.org/10.1016/s0166-218x(01)00342-0}{Brucker2002} & \hyperref[auth:a847]{P. Brucker} & \cellcolor{gold!20}Scheduling and constraint propagation & \noindent{}\textbf{1.50} \textbf{1.50} n/a & 2002 & Discrete Applied Mathematics & \cite{Brucker2002}\\
Chan2002 \href{http://dx.doi.org/10.1061/(asce)0733-9364(2002)128:6(513)}{Chan2002} & \hyperref[auth:a1662]{W. T. Chan}, \hyperref[auth:a1663]{H. Hu} & Constraint Programming Approach to Precast Production Scheduling & \noindent{}\textbf{1.00} \textbf{1.00} n/a & 2002 & Journal of Construction Engineering and Management & \cite{Chan2002}\\
Hentenryck02 \href{http://dx.doi.org/10.1287/ijoc.14.4.345.2826}{Hentenryck02} & \hyperref[auth:a148]{P. V. Hentenryck} & Constraint and Integer Programming in OPL & \noindent{}\textcolor{black!50}{0.00} \textcolor{black!50}{0.00} n/a & 2002 & \cellcolor{red!20}INFORMS Journal on Computing & \cite{Hentenryck02}\\
Hooker02 \href{http://dx.doi.org/10.1287/ijoc.14.4.295.2828}{Hooker02} & \hyperref[auth:a160]{J. N. Hooker} & Logic, Optimization, and Constraint Programming & \noindent{}\textcolor{black!50}{0.00} \textcolor{black!50}{0.00} n/a & 2002 & \cellcolor{red!20}INFORMS Journal on Computing & \cite{Hooker02}\\
Larrosa2002 \href{http://dx.doi.org/10.1017/s0960129501003577}{Larrosa2002} & \hyperref[auth:a1794]{J. Larrosa}, \hyperref[auth:a1854]{G. Valiente} & Constraint satisfaction algorithms for graph  pattern matching & \noindent{}\textcolor{black!50}{0.00} 0.50 n/a & 2002 & Mathematical Structures in Computer Science & \cite{Larrosa2002}\\
MilanoORT02 \href{http://dx.doi.org/10.1287/ijoc.14.4.387.2830}{MilanoORT02} & \hyperref[auth:a143]{M. Milano}, \hyperref[auth:a852]{G. Ottosson}, \hyperref[auth:a254]{P. Refalo}, \hyperref[auth:a874]{E. S. Thorsteinsson} & The Role of Integer Programming Techniques in Constraint Programming's Global Constraints & \noindent{}\textcolor{black!50}{0.00} \textcolor{black!50}{0.00} n/a & 2002 & \cellcolor{red!20}INFORMS Journal on Computing & \cite{MilanoORT02}\\
YunG02 \href{http://dx.doi.org/10.1016/s0360-8352(02)00065-7}{YunG02} & \hyperref[auth:a1472]{Y.-S. Yun}, \hyperref[auth:a1473]{M. Gen} & Advanced scheduling problem using constraint programming techniques in SCM environment & \noindent{}\textbf{1.00} \textbf{1.00} n/a & 2002 & Computers \  Industrial Engineering & \cite{YunG02}\\
Apt2001 \href{http://dx.doi.org/10.1017/s1471068401000072}{Apt2001} & \hyperref[auth:a1887]{K. R. Apt}, \hyperref[auth:a1833]{E. Monfroy} & \cellcolor{green!10}Constraint programming viewed as rule-based programming & \noindent{}\textcolor{black!50}{0.00} \textbf{1.75} n/a & 2001 & Theory and Practice of Logic Programming & \cite{Apt2001}\\
Caseau2001 \href{http://dx.doi.org/10.1017/s0269888901000078}{Caseau2001} & \hyperref[auth:a301]{Y. Caseau}, \hyperref[auth:a1513]{F. Laburthe}, \hyperref[auth:a163]{C. L. Pape}, \hyperref[auth:a1576]{B. Rottembourg} & Combining local and global search in a constraint programming environment & \noindent{}\textcolor{black!50}{0.00} \textbf{2.00} n/a & 2001 & The Knowledge Engineering Review & \cite{Caseau2001}\\
Farias2001 \href{http://dx.doi.org/10.1017/s0269888901000030}{Farias2001} & \hyperref[auth:a1932]{I. R. D. Farias}, \hyperref[auth:a1933]{E. L. Johnson}, \hyperref[auth:a1934]{G. L. Nemhauser} & Branch-and-cut for combinatorial optimization problems without auxiliary binary variables & \noindent{}\textcolor{black!50}{0.00} \textbf{1.00} n/a & 2001 & The Knowledge Engineering Review & \cite{Farias2001}\\
Henz01 \href{http://dx.doi.org/10.1287/opre.49.1.163.11193}{Henz01} & \hyperref[auth:a1419]{M. Henz} & Scheduling a Major College Basketball Conference—Revisited & \noindent{}\textcolor{black!50}{0.00} \textcolor{black!50}{0.00} n/a & 2001 & \cellcolor{red!20}Operations Research & \cite{Henz01}\\
Hofe2001 \href{http://dx.doi.org/10.1142/s0129054101000710}{Hofe2001} & \hyperref[auth:a2012]{H. M. A. Hofe} & Solving rostering tasks by generic methods for constraint optimization & \noindent{}\textbf{1.00} \textbf{1.50} n/a & 2001 & International Journal of Foundations of Computer Science & \cite{Hofe2001}\\
LustigP01 \href{http://dx.doi.org/10.1287/inte.31.6.29.9647}{LustigP01} & I. J. Lustig, J.-F. Puget & Program Does Not Equal Program: Constraint Programming and Its Relationship to Mathematical Programming & \noindent{}\textcolor{black!50}{0.00} \textcolor{black!50}{0.00} n/a & 2001 & \cellcolor{red!20}Interfaces & \cite{LustigP01}\\
Dorndorf2000a \href{http://dx.doi.org/10.1287/mnsc.46.10.1365.12272}{Dorndorf2000a} & \hyperref[auth:a904]{U. Dorndorf}, \hyperref[auth:a438]{E. Pesch}, \hyperref[auth:a1046]{T. Phan-Huy} & A Time-Oriented Branch-and-Bound Algorithm for Resource-Constrained Project Scheduling with Generalised Precedence Constraints & \noindent{}\textcolor{black!50}{0.00} \textbf{3.00} n/a & 2000 & Management Science & \cite{Dorndorf2000a}\\
Hentenryck2000 \href{http://dx.doi.org/10.1145/359496.359529}{Hentenryck2000} & \hyperref[auth:a148]{P. V. Hentenryck}, \hyperref[auth:a288]{L. Perron}, \hyperref[auth:a1653]{J.-F. Puget} & Search and strategies in OPL & \noindent{}\textcolor{black!50}{0.00} \textbf{1.00} n/a & 2000 & ACM Transactions on Computational Logic & \cite{Hentenryck2000}\\
Huang2000 \href{http://dx.doi.org/10.1016/s0098-1354(00)00483-x}{Huang2000} & \hyperref[auth:a1648]{W. Huang}, \hyperref[auth:a1649]{P. W. H. Chung} & Scheduling of pipeless batch plants using constraint satisfaction techniques & \noindent{}\textbf{1.00} \textbf{1.00} n/a & 2000 & Computers \  Chemical Engineering & \cite{Huang2000}\\
Kambhampati2000 \href{http://dx.doi.org/10.1613/jair.655}{Kambhampati2000} & \hyperref[auth:a1915]{S. Kambhampati} & \cellcolor{gold!20}Planning Graph as a (Dynamic) CSP: Exploiting EBL, DDB and other CSP Search Techniques in Graphplan & \noindent{}\textcolor{black!50}{0.00} \textbf{1.00} n/a & 2000 & Journal of Artificial Intelligence Research & \cite{Kambhampati2000}\\
Yang2000 \href{http://dx.doi.org/10.1109/72.839016}{Yang2000} & \hyperref[auth:a1912]{S. Yang}, \hyperref[auth:a1824]{D. Wang} & \cellcolor{green!10}Constraint satisfaction adaptive neural network and heuristics combined approaches for generalized job-shop scheduling & \noindent{}\textbf{2.00} \textbf{2.00} n/a & 2000 & IEEE Transactions on Neural Networks & \cite{Yang2000}\\
BockmayrK98 \href{http://dx.doi.org/10.1287/ijoc.10.3.287}{BockmayrK98} & \hyperref[auth:a908]{A. Bockmayr}, \hyperref[auth:a1045]{T. Kasper} & Branch and Infer: A Unifying Framework for Integer and Finite Domain Constraint Programming & \noindent{}\textcolor{black!50}{0.00} \textcolor{black!50}{0.00} n/a & 1998 & \cellcolor{red!20}INFORMS Journal on Computing & \cite{BockmayrK98}\\
DarbyDowmanL98 \href{http://dx.doi.org/10.1287/ijoc.10.3.276}{DarbyDowmanL98} & \hyperref[auth:a177]{K. Darby-Dowman}, \hyperref[auth:a178]{J. Little} & Properties of Some Combinatorial Optimization Problems and Their Effect on the Performance of Integer Programming and Constraint Logic Programming & \noindent{}\textcolor{black!50}{0.00} \textcolor{black!50}{0.00} n/a & 1998 & \cellcolor{red!20}INFORMS Journal on Computing & \cite{DarbyDowmanL98}\\
Richard1998 \href{http://dx.doi.org/10.1051/ro/1998320201251}{Richard1998} & \hyperref[auth:a1684]{P. Richard}, \hyperref[auth:a1685]{C. Proust} & \cellcolor{gold!20}Solving scheduling problems using Petri nets and constraint logic programming & \noindent{}\textbf{1.00} \textbf{1.00} n/a & 1998 & RAIRO - Operations Research & \cite{Richard1998}\\
Baykan1997 \href{http://dx.doi.org/10.1017/s0890060400003206}{Baykan1997} & \hyperref[auth:a1689]{C. A. Baykan}, \hyperref[auth:a302]{M. S. Fox} & \cellcolor{green!10}Spatial synthesis by disjunctive constraint satisfaction & \noindent{}\textcolor{black!50}{0.00} 0.50 n/a & 1997 & Artificial Intelligence for Engineering Design, Analysis and Manufacturing & \cite{Baykan1997}\\
Demeulemeester1997 \href{http://dx.doi.org/10.1287/mnsc.43.11.1485}{Demeulemeester1997} & \hyperref[auth:a1584]{E. L. Demeulemeester}, \hyperref[auth:a1585]{W. S. Herroelen} & \cellcolor{green!10}New Benchmark Results for the Resource-Constrained Project Scheduling Problem & \noindent{}\textcolor{black!50}{0.00} \textcolor{black!50}{0.00} n/a & 1997 & Management Science & \cite{Demeulemeester1997}\\
Psarras1997 \href{http://dx.doi.org/10.1016/s0377-2217(96)00114-2}{Psarras1997} & \hyperref[auth:a2040]{J. Psarras}, \hyperref[auth:a2041]{E. Stefanitsis}, \hyperref[auth:a2042]{N. Christodoulou} & Combination of local search and CLP in the vehicle-fleet scheduling problem & \noindent{}\textbf{1.00} \textbf{1.00} n/a & 1997 & European Journal of Operational Research & \cite{Psarras1997}\\
Icmeli1996 \href{http://dx.doi.org/10.1287/mnsc.42.10.1395}{Icmeli1996} & \hyperref[auth:a1553]{O. Icmeli}, \hyperref[auth:a1554]{S. S. Erenguc} & A Branch and Bound Procedure for the Resource Constrained Project Scheduling Problem with Discounted Cash Flows & \noindent{}\textcolor{black!50}{0.00} \textcolor{black!50}{0.00} n/a & 1996 & Management Science & \cite{Icmeli1996}\\
PeschT96 \href{http://dx.doi.org/10.1287/ijoc.8.2.144}{PeschT96} & \hyperref[auth:a438]{E. Pesch}, \hyperref[auth:a1216]{U. A. W. Tetzlaff} & Constraint Propagation Based Scheduling of Job Shops & \noindent{}\textbf{3.00} \textbf{3.00} n/a & 1996 & \cellcolor{red!20}INFORMS Journal on Computing & \cite{PeschT96}\\
Sadeh1995 \href{http://dx.doi.org/10.1016/0004-3702(95)00078-s}{Sadeh1995} & \hyperref[auth:a1581]{N. Sadeh}, \hyperref[auth:a1582]{K. Sycara}, \hyperref[auth:a1583]{Y. Xiong} & \cellcolor{gold!20}Backtracking techniques for the job shop scheduling constraint satisfaction problem & \noindent{}\textbf{2.00} \textbf{2.00} n/a & 1995 & Artificial Intelligence & \cite{Sadeh1995}\\
Schiex1994 \href{http://dx.doi.org/10.1142/s0218213094000108}{Schiex1994} & \hyperref[auth:a1721]{T. Schiex}, \hyperref[auth:a1722]{G. Verfaillie} & Nogood recording for static and dynamic constraint satisfaction problems & \noindent{}\textcolor{black!50}{0.00} \textbf{4.00} n/a & 1994 & International Journal on Artificial Intelligence Tools & \cite{Schiex1994}\\
Barber1993 \href{http://dx.doi.org/10.1145/152947.152955}{Barber1993} & \hyperref[auth:a1959]{F. A. Barber} & A metric time-point and duration-based temporal model & \noindent{}\textcolor{black!50}{0.00} 0.50 n/a & 1993 & ACM SIGART Bulletin & \cite{Barber1993}\\
Icmeli1993 \href{http://dx.doi.org/10.1108/01443579310046454}{Icmeli1993} & \hyperref[auth:a1553]{O. Icmeli}, \hyperref[auth:a1554]{S. S. Erenguc}, \hyperref[auth:a1723]{C. J. Zappe} & Project Scheduling Problems: A Survey & \noindent{}\textcolor{black!50}{0.00} \textcolor{black!50}{0.00} n/a & 1993 & International Journal of Operations \  Production Management & \cite{Icmeli1993}\\
LubySZ93 \href{http://dx.doi.org/10.1016/0020-0190(93)90029-9}{LubySZ93} & M. Luby, A. Sinclair, D. Zuckerman & Optimal speedup of Las Vegas algorithms & \noindent{}\textcolor{black!50}{0.00} \textcolor{black!50}{0.00} n/a & 1993 & Information Processing Letters & \cite{LubySZ93}\\
Demeulemeester1992 \href{http://dx.doi.org/10.1287/mnsc.38.12.1803}{Demeulemeester1992} & \hyperref[auth:a1090]{E. Demeulemeester}, \hyperref[auth:a1102]{W. Herroelen} & A Branch-and-Bound Procedure for the Multiple Resource-Constrained Project Scheduling Problem & \noindent{}\textcolor{black!50}{0.00} \textcolor{black!50}{0.00} n/a & 1992 & Management Science & \cite{Demeulemeester1992}\\
Elmaghraby1992 \href{http://dx.doi.org/10.1287/mnsc.38.9.1245}{Elmaghraby1992} & \hyperref[auth:a1773]{S. E. Elmaghraby}, \hyperref[auth:a1774]{J. Kamburowski} & The Analysis of Activity Networks Under Generalized Precedence Relations (GPRs) & \noindent{}\textcolor{black!50}{0.00} \textcolor{black!50}{0.00} n/a & 1992 & Management Science & \cite{Elmaghraby1992}\\
Paredis1992 \href{http://dx.doi.org/10.1080/12460125.1992.10511509}{Paredis1992} & \hyperref[auth:a1998]{J. Paredis}, \hyperref[auth:a1999]{T. van Rij} & Simulation and Constraint Programming as Support Methodologies for Job Shop Scheduling & \noindent{}\textbf{2.00} \textbf{2.00} n/a & 1992 & Journal of Decision Systems & \cite{Paredis1992}\\
Tay92 \href{}{Tay92} & \hyperref[auth:a701]{D. B. H. Tay} & {COPS:} {A} Constraint Programming Approach to Resource-Limited Project Scheduling & \noindent{}\textbf{1.50} \textbf{1.50} n/a & 1992 & Comput. J. & \cite{Tay92}\\
Clearwater1991 \href{http://dx.doi.org/10.1126/science.254.5035.1181}{Clearwater1991} & \hyperref[auth:a1776]{S. H. Clearwater}, \hyperref[auth:a1777]{B. A. Huberman}, \hyperref[auth:a1778]{T. Hogg} & Cooperative Solution of Constraint Satisfaction Problems & \noindent{}\textcolor{black!50}{0.00} \textbf{1.00} n/a & 1991 & Science & \cite{Clearwater1991}\\
Fisher1985 \href{http://dx.doi.org/10.1287/inte.15.2.10}{Fisher1985} & \hyperref[auth:a1772]{M. L. Fisher} & An Applications Oriented Guide to Lagrangian Relaxation & \noindent{}\textcolor{black!50}{0.00} \textcolor{black!50}{0.00} n/a & 1985 & \cellcolor{red!20}Interfaces & \cite{Fisher1985}\\
Carlier82 \href{http://dx.doi.org/10.1016/s0377-2217(82)80007-6}{Carlier82} & \hyperref[auth:a845]{J. Carlier} & The one-machine sequencing problem & \noindent{}\textcolor{black!50}{0.00} \textcolor{black!50}{0.00} n/a & 1982 & European Journal of Operational Research & \cite{Carlier82}\\
Lauriere78 \href{http://dx.doi.org/10.1016/0004-3702(78)90029-2}{Lauriere78} & J.-L. Lauriere & A language and a program for stating and solving combinatorial problems & \noindent{}\textcolor{black!50}{0.00} \textcolor{black!50}{0.00} n/a & 1978 & Artificial Intelligence & \cite{Lauriere78}\\
Talbot1978 \href{http://dx.doi.org/10.1287/mnsc.24.11.1163}{Talbot1978} & \hyperref[auth:a1497]{F. B. Talbot}, \hyperref[auth:a1498]{J. H. Patterson} & An Efficient Integer Programming Algorithm with Network Cuts for Solving Resource-Constrained Scheduling Problems & \noindent{}\textcolor{black!50}{0.00} \textcolor{black!50}{0.00} n/a & 1978 & Management Science & \cite{Talbot1978}\\
Mackworth77 \href{http://dx.doi.org/10.1016/0004-3702(77)90007-8}{Mackworth77} & A. K. Mackworth & Consistency in networks of relations & \noindent{}\textcolor{black!50}{0.00} \textcolor{black!50}{0.00} n/a & 1977 & Artificial Intelligence & \cite{Mackworth77}\\
GareyJS76 \href{http://dx.doi.org/10.1287/moor.1.2.117}{GareyJS76} & M. R. Garey, D. S. Johnson, R. Sethi & The Complexity of Flowshop and Jobshop Scheduling & \noindent{}\textcolor{black!50}{0.00} \textcolor{black!50}{0.00} n/a & 1976 & Mathematics of Operations Research & \cite{GareyJS76}\\
PritskerWW69 \href{http://dx.doi.org/10.1287/mnsc.16.1.93}{PritskerWW69} & A. A. B. Pritsker, L. J. Waiters, P. M. Wolfe & Multiproject Scheduling with Limited Resources: A Zero-One Programming Approach & \noindent{}\textcolor{black!50}{0.00} \textcolor{black!50}{0.00} n/a & 1969 & Management Science & \cite{PritskerWW69}\\
\end{longtable}
}



\clearpage
\section{Papers and Articles Without Recognized Concepts}

This section lists papers and articles for which we have a pdf local copy, but where we were not able to extract any of the defined concepts. This can basically have two reasons. We either have included a paper which is not at all related to scheduling, so that none of the defined concepts occur in the paper. A  more likely cause is that the pdf file is a scanned document for which optical character recognition was not run or not successful, so that the pdf consists of a series of bitmap images. In that case, pdfgrep is unable to find any text in the document, and no matches for concepts are found. It may be useful to check the pdf files to see if that is the case.

{\scriptsize
\begin{longtable}{llp{5cm}p{10cm}rp{3cm}l}
\caption{Paper without Concepts}\\ \toprule
Key & \shortstack{Local\\Copy} & Authors & Title & Year & \shortstack{Conference\\/Journal} & Cite\\ \midrule
\endhead
\bottomrule
\endfoot
BaptisteLV92 & \href{papers/BaptisteLV92.pdf}{Yes} & Pierre Baptiste and Bruno Legeard and Christophe Varnier & Hoist scheduling problem: an approach based on constraint logic programming & 1992 & ICRA 1992 & \cite{BaptisteLV92}\\\end{longtable}
}



{\scriptsize
\begin{longtable}{llp{5cm}p{10cm}rp{3cm}lr}
\rowcolor{white}\caption{ARTICLE without Concepts}\\ \toprule
\rowcolor{white}Key & \shortstack{Local\\Copy} & Authors & Title & Year & \shortstack{Conference\\/Journal} & Cite & Pages\\ \midrule
\endhead
\bottomrule
\endfoot
KorbaaYG00 & \href{works/KorbaaYG00.pdf}{Yes} & O. Korbaa, P. Yim, J. Gentina & Solving Transient Scheduling Problems with Constraint Programming & 2000 & Eur. J. Control & \cite{KorbaaYG00} & 10\\LopezAKYG00 & \href{works/LopezAKYG00.pdf}{Yes} & P. Lopez, H. Alla, O. Korbaa, P. Yim, J. Gentina & Discussion on: 'Solving Transient Scheduling Problems with Constraint Programming' by O. Korbaa, P. Yim, and {J.-C.} Gentina & 2000 & Eur. J. Control & \cite{LopezAKYG00} & 4\\\end{longtable}
}



\clearpage
\section{Unmatched Concepts}

This section lists those concepts for which no matches were found. The most likely cause is a mistake in the regular expression used to find the concept, but it is also possible that some concept simply is not mentioned in any of the documents. 

{\scriptsize
\begin{longtable}{lp{10cm}rr}
\rowcolor{white}\caption{Unmatched Concepts}\\ \toprule
\rowcolor{white}Type & Name & CaseSensitive & Revision\\ \midrule
\endhead
\bottomrule
\endfoot
Industries & steel making industry &  & 0\\ApplicationAreas & day-ahead market &  & 0\\ApplicationAreas & ship building &  & 0\\ApplicationAreas & vaccine &  & 0\\Classification & Modified Generalized Assignment Problem &  & 0\\Classification & PP-MS-MMRCPSP & Y & 1\\Classification & Pre-emptive Job-Shop scheduling Problem &  & 0\\Classification & Resource-constrained Project Scheduling Problem with Discounted Cashflow &  & 0\\Classification & SMSDP & Y & 1\\Classification & Steel-making and continuous casting &  & 0\\Concepts & make to stock &  & 1\\\end{longtable}
}



\clearpage
\section{Works by Author}

\subsection{Works by J. Christopher Beck}
\label{sec:a89}
{\scriptsize
\begin{longtable}{>{\raggedright\arraybackslash}p{3cm}>{\raggedright\arraybackslash}p{6cm}>{\raggedright\arraybackslash}p{7cm}rrrp{3cm}rrr}
\rowcolor{white}\caption{Works from bibtex (Total 46)}\\ \toprule
\rowcolor{white}Key & Authors & Title & LC & Cite & Year & \shortstack{Conference\\/Journal} & Pages & b & c \\ \midrule\endhead
\bottomrule
\endfoot
LuoB22 \href{https://doi.org/10.1007/978-3-031-08011-1\_17}{LuoB22} & \hyperref[auth:a754]{Yiqing L. Luo}, \hyperref[auth:a89]{J. Christopher Beck} & Packing by Scheduling: Using Constraint Programming to Solve a Complex 2D Cutting Stock Problem & \href{works/LuoB22.pdf}{Yes} & \cite{LuoB22} & 2022 & CPAIOR 2022 & 17 & \ref{b:LuoB22} & \ref{c:LuoB22}\\
ZhangBB22 \href{https://ojs.aaai.org/index.php/ICAPS/article/view/19826}{ZhangBB22} & \hyperref[auth:a808]{J. Zhang}, \hyperref[auth:a809]{Giovanni Lo Bianco}, \hyperref[auth:a89]{J. Christopher Beck} & Solving Job-Shop Scheduling Problems with QUBO-Based Specialized Hardware & \href{works/ZhangBB22.pdf}{Yes} & \cite{ZhangBB22} & 2022 & ICAPS 2022 & 9 & \ref{b:ZhangBB22} & \ref{c:ZhangBB22}\\
TangB20 \href{https://doi.org/10.1007/978-3-030-58942-4\_28}{TangB20} & \hyperref[auth:a88]{Tanya Y. Tang}, \hyperref[auth:a89]{J. Christopher Beck} & {CP} and Hybrid Models for Two-Stage Batching and Scheduling & \href{works/TangB20.pdf}{Yes} & \cite{TangB20} & 2020 & CPAIOR 2020 & 16 & \ref{b:TangB20} & \ref{c:TangB20}\\
TranPZLDB18 \href{https://doi.org/10.1007/s10951-017-0537-x}{TranPZLDB18} & \hyperref[auth:a810]{Tony T. Tran}, \hyperref[auth:a811]{M. Padmanabhan}, \hyperref[auth:a812]{Peter Yun Zhang}, \hyperref[auth:a813]{H. Li}, \hyperref[auth:a814]{Douglas G. Down}, \hyperref[auth:a89]{J. Christopher Beck} & Multi-stage resource-aware scheduling for data centers with heterogeneous servers & \href{works/TranPZLDB18.pdf}{Yes} & \cite{TranPZLDB18} & 2018 & J. Sched. & 17 & \ref{b:TranPZLDB18} & \ref{c:TranPZLDB18}\\
CohenHB17 \href{https://doi.org/10.1007/978-3-319-66263-3\_10}{CohenHB17} & \hyperref[auth:a816]{E. Cohen}, \hyperref[auth:a817]{G. Huang}, \hyperref[auth:a89]{J. Christopher Beck} & {(I} Can Get) Satisfaction: Preference-Based Scheduling for Concert-Goers at Multi-venue Music Festivals & \href{works/CohenHB17.pdf}{Yes} & \cite{CohenHB17} & 2017 & SAT 2017 & 17 & \ref{b:CohenHB17} & \ref{c:CohenHB17}\\
TranVNB17 \href{https://doi.org/10.1613/jair.5306}{TranVNB17} & \hyperref[auth:a810]{Tony T. Tran}, \hyperref[auth:a815]{Tiago Stegun Vaquero}, \hyperref[auth:a209]{G. Nejat}, \hyperref[auth:a89]{J. Christopher Beck} & Robots in Retirement Homes: Applying Off-the-Shelf Planning and Scheduling to a Team of Assistive Robots & \href{works/TranVNB17.pdf}{Yes} & \cite{TranVNB17} & 2017 & J. Artif. Intell. Res. & 68 & \ref{b:TranVNB17} & \ref{c:TranVNB17}\\
TranVNB17a \href{https://doi.org/10.24963/ijcai.2017/726}{TranVNB17a} & \hyperref[auth:a810]{Tony T. Tran}, \hyperref[auth:a815]{Tiago Stegun Vaquero}, \hyperref[auth:a209]{G. Nejat}, \hyperref[auth:a89]{J. Christopher Beck} & Robots in Retirement Homes: Applying Off-the-Shelf Planning and Scheduling to a Team of Assistive Robots (Extended Abstract) & \href{works/TranVNB17a.pdf}{Yes} & \cite{TranVNB17a} & 2017 & IJCAI 2017 & 5 & \ref{b:TranVNB17a} & \ref{c:TranVNB17a}\\
BoothNB16 \href{https://doi.org/10.1007/978-3-319-44953-1\_34}{BoothNB16} & \hyperref[auth:a208]{Kyle E. C. Booth}, \hyperref[auth:a209]{G. Nejat}, \hyperref[auth:a89]{J. Christopher Beck} & A Constraint Programming Approach to Multi-Robot Task Allocation and Scheduling in Retirement Homes & \href{works/BoothNB16.pdf}{Yes} & \cite{BoothNB16} & 2016 & CP 2016 & 17 & \ref{b:BoothNB16} & \ref{c:BoothNB16}\\
KuB16 \href{https://doi.org/10.1016/j.cor.2016.04.006}{KuB16} & \hyperref[auth:a336]{W. Ku}, \hyperref[auth:a89]{J. Christopher Beck} & Mixed Integer Programming models for job shop scheduling: {A} computational analysis & No & \cite{KuB16} & 2016 & Comput. Oper. Res. & 9 & No & \ref{c:KuB16}\\
LuoVLBM16 \href{http://www.aaai.org/ocs/index.php/KR/KR16/paper/view/12909}{LuoVLBM16} & \hyperref[auth:a824]{R. Luo}, \hyperref[auth:a825]{Richard Anthony Valenzano}, \hyperref[auth:a826]{Y. Li}, \hyperref[auth:a89]{J. Christopher Beck}, \hyperref[auth:a827]{Sheila A. McIlraith} & Using Metric Temporal Logic to Specify Scheduling Problems & \href{works/LuoVLBM16.pdf}{Yes} & \cite{LuoVLBM16} & 2016 & KR 2016 & 4 & \ref{b:LuoVLBM16} & \ref{c:LuoVLBM16}\\
TranAB16 \href{https://doi.org/10.1287/ijoc.2015.0666}{TranAB16} & \hyperref[auth:a810]{Tony T. Tran}, \hyperref[auth:a818]{A. Araujo}, \hyperref[auth:a89]{J. Christopher Beck} & Decomposition Methods for the Parallel Machine Scheduling Problem with Setups & No & \cite{TranAB16} & 2016 & {INFORMS} J. Comput. & 13 & No & \ref{c:TranAB16}\\
TranDRFWOVB16 \href{https://doi.org/10.1609/socs.v7i1.18390}{TranDRFWOVB16} & \hyperref[auth:a810]{Tony T. Tran}, \hyperref[auth:a820]{M. Do}, \hyperref[auth:a821]{Eleanor Gilbert Rieffel}, \hyperref[auth:a383]{J. Frank}, \hyperref[auth:a819]{Z. Wang}, \hyperref[auth:a822]{B. O'Gorman}, \hyperref[auth:a823]{D. Venturelli}, \hyperref[auth:a89]{J. Christopher Beck} & A Hybrid Quantum-Classical Approach to Solving Scheduling Problems & \href{works/TranDRFWOVB16.pdf}{Yes} & \cite{TranDRFWOVB16} & 2016 & SOCS 2016 & 9 & \ref{b:TranDRFWOVB16} & \ref{c:TranDRFWOVB16}\\
TranWDRFOVB16 \href{http://www.aaai.org/ocs/index.php/WS/AAAIW16/paper/view/12664}{TranWDRFOVB16} & \hyperref[auth:a810]{Tony T. Tran}, \hyperref[auth:a819]{Z. Wang}, \hyperref[auth:a820]{M. Do}, \hyperref[auth:a821]{Eleanor Gilbert Rieffel}, \hyperref[auth:a383]{J. Frank}, \hyperref[auth:a822]{B. O'Gorman}, \hyperref[auth:a823]{D. Venturelli}, \hyperref[auth:a89]{J. Christopher Beck} & Explorations of Quantum-Classical Approaches to Scheduling a Mars Lander Activity Problem & \href{works/TranWDRFOVB16.pdf}{Yes} & \cite{TranWDRFOVB16} & 2016 & AAAI 2016 & 9 & \ref{b:TranWDRFOVB16} & \ref{c:TranWDRFOVB16}\\
BajestaniB15 \href{https://doi.org/10.1007/s10951-015-0416-2}{BajestaniB15} & \hyperref[auth:a828]{Maliheh Aramon Bajestani}, \hyperref[auth:a89]{J. Christopher Beck} & A two-stage coupled algorithm for an integrated maintenance planning and flowshop scheduling problem with deteriorating machines & \href{works/BajestaniB15.pdf}{Yes} & \cite{BajestaniB15} & 2015 & J. Sched. & 16 & \ref{b:BajestaniB15} & \ref{c:BajestaniB15}\\
KoschB14 \href{https://doi.org/10.1007/978-3-319-07046-9\_5}{KoschB14} & \hyperref[auth:a332]{S. Kosch}, \hyperref[auth:a89]{J. Christopher Beck} & A New {MIP} Model for Parallel-Batch Scheduling with Non-identical Job Sizes & \href{works/KoschB14.pdf}{Yes} & \cite{KoschB14} & 2014 & CPAIOR 2014 & 16 & \ref{b:KoschB14} & \ref{c:KoschB14}\\
LouieVNB14 \href{https://doi.org/10.1109/ICRA.2014.6907637}{LouieVNB14} & \hyperref[auth:a830]{Wing{-}Yue Geoffrey Louie}, \hyperref[auth:a815]{Tiago Stegun Vaquero}, \hyperref[auth:a209]{G. Nejat}, \hyperref[auth:a89]{J. Christopher Beck} & An autonomous assistive robot for planning, scheduling and facilitating multi-user activities & No & \cite{LouieVNB14} & 2014 & ICRA 2014 & 7 & No & \ref{c:LouieVNB14}\\
TerekhovTDB14 \href{https://doi.org/10.1613/jair.4278}{TerekhovTDB14} & \hyperref[auth:a829]{D. Terekhov}, \hyperref[auth:a810]{Tony T. Tran}, \hyperref[auth:a814]{Douglas G. Down}, \hyperref[auth:a89]{J. Christopher Beck} & Integrating Queueing Theory and Scheduling for Dynamic Scheduling Problems & \href{works/TerekhovTDB14.pdf}{Yes} & \cite{TerekhovTDB14} & 2014 & J. Artif. Intell. Res. & 38 & \ref{b:TerekhovTDB14} & \ref{c:TerekhovTDB14}\\
BajestaniB13 \href{https://doi.org/10.1613/jair.3902}{BajestaniB13} & \hyperref[auth:a828]{Maliheh Aramon Bajestani}, \hyperref[auth:a89]{J. Christopher Beck} & Scheduling a Dynamic Aircraft Repair Shop with Limited Repair Resources & \href{works/BajestaniB13.pdf}{Yes} & \cite{BajestaniB13} & 2013 & J. Artif. Intell. Res. & 36 & \ref{b:BajestaniB13} & \ref{c:BajestaniB13}\\
HeinzKB13 \href{https://doi.org/10.1007/978-3-642-38171-3\_2}{HeinzKB13} & \hyperref[auth:a133]{S. Heinz}, \hyperref[auth:a336]{W. Ku}, \hyperref[auth:a89]{J. Christopher Beck} & Recent Improvements Using Constraint Integer Programming for Resource Allocation and Scheduling & \href{works/HeinzKB13.pdf}{Yes} & \cite{HeinzKB13} & 2013 & CPAIOR 2013 & 16 & \ref{b:HeinzKB13} & \ref{c:HeinzKB13}\\
HeinzSB13 \href{https://doi.org/10.1007/s10601-012-9136-9}{HeinzSB13} & \hyperref[auth:a133]{S. Heinz}, \hyperref[auth:a134]{J. Schulz}, \hyperref[auth:a89]{J. Christopher Beck} & Using dual presolving reductions to reformulate cumulative constraints & \href{works/HeinzSB13.pdf}{Yes} & \cite{HeinzSB13} & 2013 & Constraints An Int. J. & 36 & \ref{b:HeinzSB13} & \ref{c:HeinzSB13}\\
TranTDB13 \href{http://www.aaai.org/ocs/index.php/ICAPS/ICAPS13/paper/view/6005}{TranTDB13} & \hyperref[auth:a810]{Tony T. Tran}, \hyperref[auth:a829]{D. Terekhov}, \hyperref[auth:a814]{Douglas G. Down}, \hyperref[auth:a89]{J. Christopher Beck} & Hybrid Queueing Theory and Scheduling Models for Dynamic Environments with Sequence-Dependent Setup Times & \href{works/TranTDB13.pdf}{Yes} & \cite{TranTDB13} & 2013 & ICAPS 2013 & 9 & \ref{b:TranTDB13} & \ref{c:TranTDB13}\\
HeinzB12 \href{https://doi.org/10.1007/978-3-642-29828-8\_14}{HeinzB12} & \hyperref[auth:a133]{S. Heinz}, \hyperref[auth:a89]{J. Christopher Beck} & Reconsidering Mixed Integer Programming and MIP-Based Hybrids for Scheduling & \href{works/HeinzB12.pdf}{Yes} & \cite{HeinzB12} & 2012 & CPAIOR 2012 & 17 & \ref{b:HeinzB12} & \ref{c:HeinzB12}\\
TerekhovDOB12 \href{https://doi.org/10.1016/j.cie.2012.02.006}{TerekhovDOB12} & \hyperref[auth:a829]{D. Terekhov}, \hyperref[auth:a831]{Mustafa K. Dogru}, \hyperref[auth:a832]{U. {\"{O}}zen}, \hyperref[auth:a89]{J. Christopher Beck} & Solving two-machine assembly scheduling problems with inventory constraints & No & \cite{TerekhovDOB12} & 2012 & Comput. Ind. Eng. & 15 & No & \ref{c:TerekhovDOB12}\\
TranB12 \href{https://doi.org/10.3233/978-1-61499-098-7-774}{TranB12} & \hyperref[auth:a810]{Tony T. Tran}, \hyperref[auth:a89]{J. Christopher Beck} & Logic-based Benders Decomposition for Alternative Resource Scheduling with Sequence Dependent Setups & \href{works/TranB12.pdf}{Yes} & \cite{TranB12} & 2012 & ECAI 2012 & 6 & \ref{b:TranB12} & \ref{c:TranB12}\\
BajestaniB11 \href{http://aaai.org/ocs/index.php/ICAPS/ICAPS11/paper/view/2680}{BajestaniB11} & \hyperref[auth:a828]{Maliheh Aramon Bajestani}, \hyperref[auth:a89]{J. Christopher Beck} & Scheduling an Aircraft Repair Shop & \href{works/BajestaniB11.pdf}{Yes} & \cite{BajestaniB11} & 2011 & ICAPS 2011 & 8 & \ref{b:BajestaniB11} & \ref{c:BajestaniB11}\\
BeckFW11 \href{https://doi.org/10.1287/ijoc.1100.0388}{BeckFW11} & \hyperref[auth:a89]{J. Christopher Beck}, \hyperref[auth:a833]{T. K. Feng}, \hyperref[auth:a364]{J. Watson} & Combining Constraint Programming and Local Search for Job-Shop Scheduling & \href{works/BeckFW11.pdf}{Yes} & \cite{BeckFW11} & 2011 & {INFORMS} J. Comput. & 14 & \ref{b:BeckFW11} & \ref{c:BeckFW11}\\
HeckmanB11 \href{https://doi.org/10.1007/s10951-009-0113-0}{HeckmanB11} & \hyperref[auth:a834]{I. Heckman}, \hyperref[auth:a89]{J. Christopher Beck} & Understanding the behavior of Solution-Guided Search for job-shop scheduling & \href{works/HeckmanB11.pdf}{Yes} & \cite{HeckmanB11} & 2011 & J. Sched. & 20 & \ref{b:HeckmanB11} & \ref{c:HeckmanB11}\\
KovacsB11 \href{https://doi.org/10.1007/s10601-009-9088-x}{KovacsB11} & \hyperref[auth:a146]{A. Kov{\'{a}}cs}, \hyperref[auth:a89]{J. Christopher Beck} & A global constraint for total weighted completion time for unary resources & \href{works/KovacsB11.pdf}{Yes} & \cite{KovacsB11} & 2011 & Constraints An Int. J. & 24 & \ref{b:KovacsB11} & \ref{c:KovacsB11}\\
BidotVLB09 \href{https://doi.org/10.1007/s10951-008-0080-x}{BidotVLB09} & \hyperref[auth:a835]{J. Bidot}, \hyperref[auth:a836]{T. Vidal}, \hyperref[auth:a118]{P. Laborie}, \hyperref[auth:a89]{J. Christopher Beck} & A theoretic and practical framework for scheduling in a stochastic environment & \href{works/BidotVLB09.pdf}{Yes} & \cite{BidotVLB09} & 2009 & J. Sched. & 30 & \ref{b:BidotVLB09} & \ref{c:BidotVLB09}\\
WuBB09 \href{https://doi.org/10.1016/j.cor.2008.08.008}{WuBB09} & \hyperref[auth:a276]{Christine Wei Wu}, \hyperref[auth:a222]{Kenneth N. Brown}, \hyperref[auth:a89]{J. Christopher Beck} & Scheduling with uncertain durations: Modeling beta-robust scheduling with constraints & No & \cite{WuBB09} & 2009 & Comput. Oper. Res. & 9 & No & \ref{c:WuBB09}\\
KovacsB08 \href{https://doi.org/10.1016/j.engappai.2008.03.004}{KovacsB08} & \hyperref[auth:a146]{A. Kov{\'{a}}cs}, \hyperref[auth:a89]{J. Christopher Beck} & A global constraint for total weighted completion time for cumulative resources & \href{works/KovacsB08.pdf}{Yes} & \cite{KovacsB08} & 2008 & Eng. Appl. Artif. Intell. & 7 & \ref{b:KovacsB08} & \ref{c:KovacsB08}\\
WatsonB08 \href{https://doi.org/10.1007/978-3-540-68155-7\_21}{WatsonB08} & \hyperref[auth:a364]{J. Watson}, \hyperref[auth:a89]{J. Christopher Beck} & A Hybrid Constraint Programming / Local Search Approach to the Job-Shop Scheduling Problem & \href{works/WatsonB08.pdf}{Yes} & \cite{WatsonB08} & 2008 & CPAIOR 2008 & 15 & \ref{b:WatsonB08} & \ref{c:WatsonB08}\\
Beck07 \href{https://doi.org/10.1613/jair.2169}{Beck07} & \hyperref[auth:a89]{J. Christopher Beck} & Solution-Guided Multi-Point Constructive Search for Job Shop Scheduling & \href{works/Beck07.pdf}{Yes} & \cite{Beck07} & 2007 & J. Artif. Intell. Res. & 29 & \ref{b:Beck07} & \ref{c:Beck07}\\
BeckW07 \href{https://doi.org/10.1613/jair.2080}{BeckW07} & \hyperref[auth:a89]{J. Christopher Beck}, \hyperref[auth:a837]{N. Wilson} & Proactive Algorithms for Job Shop Scheduling with Probabilistic Durations & \href{works/BeckW07.pdf}{Yes} & \cite{BeckW07} & 2007 & J. Artif. Intell. Res. & 50 & \ref{b:BeckW07} & \ref{c:BeckW07}\\
KovacsB07 \href{https://doi.org/10.1007/978-3-540-72397-4\_9}{KovacsB07} & \hyperref[auth:a146]{A. Kov{\'{a}}cs}, \hyperref[auth:a89]{J. Christopher Beck} & A Global Constraint for Total Weighted Completion Time & \href{works/KovacsB07.pdf}{Yes} & \cite{KovacsB07} & 2007 & CPAIOR 2007 & 15 & \ref{b:KovacsB07} & \ref{c:KovacsB07}\\
Beck06 \href{http://www.aaai.org/Library/ICAPS/2006/icaps06-028.php}{Beck06} & \hyperref[auth:a89]{J. Christopher Beck} & An Empirical Study of Multi-Point Constructive Search for Constraint-Based Scheduling & No & \cite{Beck06} & 2006 & ICAPS 2006 & 10 & No & \ref{c:Beck06}\\
BeckW05 \href{http://ijcai.org/Proceedings/05/Papers/0748.pdf}{BeckW05} & \hyperref[auth:a89]{J. Christopher Beck}, \hyperref[auth:a837]{N. Wilson} & Proactive Algorithms for Scheduling with Probabilistic Durations & \href{works/BeckW05.pdf}{Yes} & \cite{BeckW05} & 2005 & IJCAI 2005 & 6 & \ref{b:BeckW05} & \ref{c:BeckW05}\\
CarchraeBF05 \href{https://doi.org/10.1007/11564751\_80}{CarchraeBF05} & \hyperref[auth:a274]{T. Carchrae}, \hyperref[auth:a89]{J. Christopher Beck}, \hyperref[auth:a275]{Eugene C. Freuder} & Methods to Learn Abstract Scheduling Models & \href{works/CarchraeBF05.pdf}{Yes} & \cite{CarchraeBF05} & 2005 & CP 2005 & 1 & \ref{b:CarchraeBF05} & \ref{c:CarchraeBF05}\\
WuBB05 \href{https://doi.org/10.1007/11564751\_110}{WuBB05} & \hyperref[auth:a276]{Christine Wei Wu}, \hyperref[auth:a222]{Kenneth N. Brown}, \hyperref[auth:a89]{J. Christopher Beck} & Scheduling with Uncertain Start Dates & \href{works/WuBB05.pdf}{Yes} & \cite{WuBB05} & 2005 & CP 2005 & 1 & \ref{b:WuBB05} & \ref{c:WuBB05}\\
BeckW04 \href{}{BeckW04} & \hyperref[auth:a89]{J. Christopher Beck}, \hyperref[auth:a837]{N. Wilson} & Job Shop Scheduling with Probabilistic Durations & No & \cite{BeckW04} & 2004 & ECAI 2004 & 5 & No & \ref{c:BeckW04}\\
BeckPS03 \href{http://www.aaai.org/Library/ICAPS/2003/icaps03-027.php}{BeckPS03} & \hyperref[auth:a89]{J. Christopher Beck}, \hyperref[auth:a838]{P. Prosser}, \hyperref[auth:a839]{E. Selensky} & Vehicle Routing and Job Shop Scheduling: What's the Difference? & No & \cite{BeckPS03} & 2003 & ICAPS 2003 & 10 & No & \ref{c:BeckPS03}\\
BeckR03 \href{https://doi.org/10.1023/A:1021849405707}{BeckR03} & \hyperref[auth:a89]{J. Christopher Beck}, \hyperref[auth:a256]{P. Refalo} & A Hybrid Approach to Scheduling with Earliness and Tardiness Costs & \href{works/BeckR03.pdf}{Yes} & \cite{BeckR03} & 2003 & Ann. Oper. Res. & 23 & \ref{b:BeckR03} & \ref{c:BeckR03}\\
BeckF00 \href{https://doi.org/10.1016/S0004-3702(99)00099-5}{BeckF00} & \hyperref[auth:a89]{J. Christopher Beck}, \hyperref[auth:a304]{Mark S. Fox} & Dynamic problem structure analysis as a basis for constraint-directed scheduling heuristics & \href{works/BeckF00.pdf}{Yes} & \cite{BeckF00} & 2000 & Artif. Intell. & 51 & \ref{b:BeckF00} & \ref{c:BeckF00}\\
Beck99 \href{https://librarysearch.library.utoronto.ca/permalink/01UTORONTO\_INST/14bjeso/alma991106162342106196}{Beck99} & \hyperref[auth:a89]{J. Christopher Beck} & Texture measurements as a basis for heuristic commitment techniques in constraint-directed scheduling & No & \cite{Beck99} & 1999 & n/a & null & No & \ref{c:Beck99}\\
BeckF98 \href{https://doi.org/10.1609/aimag.v19i4.1426}{BeckF98} & \hyperref[auth:a89]{J. Christopher Beck}, \hyperref[auth:a304]{Mark S. Fox} & A Generic Framework for Constraint-Directed Search and Scheduling & \href{works/BeckF98.pdf}{Yes} & \cite{BeckF98} & 1998 & {AI} Mag. & 30 & \ref{b:BeckF98} & \ref{c:BeckF98}\\
BeckDF97 \href{https://doi.org/10.1007/BFb0017455}{BeckDF97} & \hyperref[auth:a89]{J. Christopher Beck}, \hyperref[auth:a250]{Andrew J. Davenport}, \hyperref[auth:a304]{Mark S. Fox} & Five Pitfalls of Empirical Scheduling Research & \href{works/BeckDF97.pdf}{Yes} & \cite{BeckDF97} & 1997 & CP 1997 & 15 & \ref{b:BeckDF97} & \ref{c:BeckDF97}\\
\end{longtable}
}

\subsection{Works by Michela Milano}
\label{sec:a143}
{\scriptsize
\begin{longtable}{>{\raggedright\arraybackslash}p{3cm}>{\raggedright\arraybackslash}p{6cm}>{\raggedright\arraybackslash}p{7cm}rrrp{3cm}rrr}
\rowcolor{white}\caption{Works from bibtex (Total 24)}\\ \toprule
\rowcolor{white}Key & Authors & Title & LC & Cite & Year & \shortstack{Conference\\/Journal} & Pages & b & c \\ \midrule\endhead
\bottomrule
\endfoot
BorghesiBLMB18 \href{https://doi.org/10.1016/j.suscom.2018.05.007}{BorghesiBLMB18} & \hyperref[auth:a231]{A. Borghesi}, \hyperref[auth:a230]{A. Bartolini}, \hyperref[auth:a142]{M. Lombardi}, \hyperref[auth:a143]{M. Milano}, \hyperref[auth:a247]{L. Benini} & Scheduling-based power capping in high performance computing systems & \href{works/BorghesiBLMB18.pdf}{Yes} & \cite{BorghesiBLMB18} & 2018 & Sustain. Comput. Informatics Syst. & 13 & \ref{b:BorghesiBLMB18} & \ref{c:BorghesiBLMB18}\\
BonfiettiZLM16 \href{https://doi.org/10.1007/978-3-319-44953-1\_8}{BonfiettiZLM16} & \hyperref[auth:a203]{A. Bonfietti}, \hyperref[auth:a204]{A. Zanarini}, \hyperref[auth:a142]{M. Lombardi}, \hyperref[auth:a143]{M. Milano} & The Multirate Resource Constraint & \href{works/BonfiettiZLM16.pdf}{Yes} & \cite{BonfiettiZLM16} & 2016 & CP 2016 & 17 & \ref{b:BonfiettiZLM16} & \ref{c:BonfiettiZLM16}\\
BridiBLMB16 \href{https://doi.org/10.1109/TPDS.2016.2516997}{BridiBLMB16} & \hyperref[auth:a232]{T. Bridi}, \hyperref[auth:a230]{A. Bartolini}, \hyperref[auth:a142]{M. Lombardi}, \hyperref[auth:a143]{M. Milano}, \hyperref[auth:a247]{L. Benini} & A Constraint Programming Scheduler for Heterogeneous High-Performance Computing Machines & \href{works/BridiBLMB16.pdf}{Yes} & \cite{BridiBLMB16} & 2016 & {IEEE} Trans. Parallel Distributed Syst. & 14 & \ref{b:BridiBLMB16} & \ref{c:BridiBLMB16}\\
BridiLBBM16 \href{https://doi.org/10.3233/978-1-61499-672-9-1598}{BridiLBBM16} & \hyperref[auth:a232]{T. Bridi}, \hyperref[auth:a142]{M. Lombardi}, \hyperref[auth:a230]{A. Bartolini}, \hyperref[auth:a247]{L. Benini}, \hyperref[auth:a143]{M. Milano} & {DARDIS:} Distributed And Randomized DIspatching and Scheduling & \href{works/BridiLBBM16.pdf}{Yes} & \cite{BridiLBBM16} & 2016 & ECAI 2016 & 2 & \ref{b:BridiLBBM16} & \ref{c:BridiLBBM16}\\
LombardiBM15 \href{https://doi.org/10.1007/978-3-319-23219-5\_20}{LombardiBM15} & \hyperref[auth:a142]{M. Lombardi}, \hyperref[auth:a203]{A. Bonfietti}, \hyperref[auth:a143]{M. Milano} & Deterministic Estimation of the Expected Makespan of a {POS} Under Duration Uncertainty & \href{works/LombardiBM15.pdf}{Yes} & \cite{LombardiBM15} & 2015 & CP 2015 & 16 & \ref{b:LombardiBM15} & \ref{c:LombardiBM15}\\
BartoliniBBLM14 \href{https://doi.org/10.1007/978-3-319-10428-7\_55}{BartoliniBBLM14} & \hyperref[auth:a230]{A. Bartolini}, \hyperref[auth:a231]{A. Borghesi}, \hyperref[auth:a232]{T. Bridi}, \hyperref[auth:a142]{M. Lombardi}, \hyperref[auth:a143]{M. Milano} & Proactive Workload Dispatching on the {EURORA} Supercomputer & \href{works/BartoliniBBLM14.pdf}{Yes} & \cite{BartoliniBBLM14} & 2014 & CP 2014 & 16 & \ref{b:BartoliniBBLM14} & \ref{c:BartoliniBBLM14}\\
BonfiettiLBM14 \href{https://doi.org/10.1016/j.artint.2013.09.006}{BonfiettiLBM14} & \hyperref[auth:a203]{A. Bonfietti}, \hyperref[auth:a142]{M. Lombardi}, \hyperref[auth:a247]{L. Benini}, \hyperref[auth:a143]{M. Milano} & {CROSS} cyclic resource-constrained scheduling solver & \href{works/BonfiettiLBM14.pdf}{Yes} & \cite{BonfiettiLBM14} & 2014 & Artif. Intell. & 28 & \ref{b:BonfiettiLBM14} & \ref{c:BonfiettiLBM14}\\
BonfiettiLM14 \href{https://doi.org/10.1007/978-3-319-07046-9\_15}{BonfiettiLM14} & \hyperref[auth:a203]{A. Bonfietti}, \hyperref[auth:a142]{M. Lombardi}, \hyperref[auth:a143]{M. Milano} & Disregarding Duration Uncertainty in Partial Order Schedules? Yes, We Can! & \href{works/BonfiettiLM14.pdf}{Yes} & \cite{BonfiettiLM14} & 2014 & CPAIOR 2014 & 16 & \ref{b:BonfiettiLM14} & \ref{c:BonfiettiLM14}\\
BonfiettiLM13 \href{http://www.aaai.org/ocs/index.php/ICAPS/ICAPS13/paper/view/6050}{BonfiettiLM13} & \hyperref[auth:a203]{A. Bonfietti}, \hyperref[auth:a142]{M. Lombardi}, \hyperref[auth:a143]{M. Milano} & De-Cycling Cyclic Scheduling Problems & \href{works/BonfiettiLM13.pdf}{Yes} & \cite{BonfiettiLM13} & 2013 & ICAPS 2013 & 5 & \ref{b:BonfiettiLM13} & \ref{c:BonfiettiLM13}\\
LombardiM13 \href{http://www.aaai.org/ocs/index.php/ICAPS/ICAPS13/paper/view/6052}{LombardiM13} & \hyperref[auth:a142]{M. Lombardi}, \hyperref[auth:a143]{M. Milano} & A Min-Flow Algorithm for Minimal Critical Set Detection in Resource Constrained Project Scheduling & \href{works/LombardiM13.pdf}{Yes} & \cite{LombardiM13} & 2013 & ICAPS 2013 & 2 & \ref{b:LombardiM13} & \ref{c:LombardiM13}\\
BonfiettiLBM12 \href{https://doi.org/10.1007/978-3-642-29828-8\_6}{BonfiettiLBM12} & \hyperref[auth:a203]{A. Bonfietti}, \hyperref[auth:a142]{M. Lombardi}, \hyperref[auth:a247]{L. Benini}, \hyperref[auth:a143]{M. Milano} & Global Cyclic Cumulative Constraint & \href{works/BonfiettiLBM12.pdf}{Yes} & \cite{BonfiettiLBM12} & 2012 & CPAIOR 2012 & 16 & \ref{b:BonfiettiLBM12} & \ref{c:BonfiettiLBM12}\\
BonfiettiM12 \href{https://ceur-ws.org/Vol-926/paper2.pdf}{BonfiettiM12} & \hyperref[auth:a203]{A. Bonfietti}, \hyperref[auth:a143]{M. Milano} & A Constraint-based Approach to Cyclic Resource-Constrained Scheduling Problem & \href{works/BonfiettiM12.pdf}{Yes} & \cite{BonfiettiM12} & 2012 & DC SIAAI 2012 & 3 & \ref{b:BonfiettiM12} & \ref{c:BonfiettiM12}\\
LombardiM12 \href{https://doi.org/10.1007/s10601-011-9115-6}{LombardiM12} & \hyperref[auth:a142]{M. Lombardi}, \hyperref[auth:a143]{M. Milano} & Optimal methods for resource allocation and scheduling: a cross-disciplinary survey & \href{works/LombardiM12.pdf}{Yes} & \cite{LombardiM12} & 2012 & Constraints An Int. J. & 35 & \ref{b:LombardiM12} & \ref{c:LombardiM12}\\
LombardiM12a \href{https://doi.org/10.1016/j.artint.2011.12.001}{LombardiM12a} & \hyperref[auth:a142]{M. Lombardi}, \hyperref[auth:a143]{M. Milano} & A min-flow algorithm for Minimal Critical Set detection in Resource Constrained Project Scheduling & \href{works/LombardiM12a.pdf}{Yes} & \cite{LombardiM12a} & 2012 & Artif. Intell. & 10 & \ref{b:LombardiM12a} & \ref{c:LombardiM12a}\\
BeniniLMR11 \href{https://doi.org/10.1007/s10479-010-0718-x}{BeniniLMR11} & \hyperref[auth:a247]{L. Benini}, \hyperref[auth:a142]{M. Lombardi}, \hyperref[auth:a143]{M. Milano}, \hyperref[auth:a727]{M. Ruggiero} & Optimal resource allocation and scheduling for the {CELL} {BE} platform & \href{works/BeniniLMR11.pdf}{Yes} & \cite{BeniniLMR11} & 2011 & Ann. Oper. Res. & 27 & \ref{b:BeniniLMR11} & \ref{c:BeniniLMR11}\\
BonfiettiLBM11 \href{https://doi.org/10.1007/978-3-642-23786-7\_12}{BonfiettiLBM11} & \hyperref[auth:a203]{A. Bonfietti}, \hyperref[auth:a142]{M. Lombardi}, \hyperref[auth:a247]{L. Benini}, \hyperref[auth:a143]{M. Milano} & A Constraint Based Approach to Cyclic {RCPSP} & \href{works/BonfiettiLBM11.pdf}{Yes} & \cite{BonfiettiLBM11} & 2011 & CP 2011 & 15 & \ref{b:BonfiettiLBM11} & \ref{c:BonfiettiLBM11}\\
LombardiBMB11 \href{https://doi.org/10.1007/978-3-642-21311-3\_14}{LombardiBMB11} & \hyperref[auth:a142]{M. Lombardi}, \hyperref[auth:a203]{A. Bonfietti}, \hyperref[auth:a143]{M. Milano}, \hyperref[auth:a247]{L. Benini} & Precedence Constraint Posting for Cyclic Scheduling Problems & \href{works/LombardiBMB11.pdf}{Yes} & \cite{LombardiBMB11} & 2011 & CPAIOR 2011 & 17 & \ref{b:LombardiBMB11} & \ref{c:LombardiBMB11}\\
LombardiM10 \href{https://doi.org/10.1007/978-3-642-15396-9\_32}{LombardiM10} & \hyperref[auth:a142]{M. Lombardi}, \hyperref[auth:a143]{M. Milano} & Constraint Based Scheduling to Deal with Uncertain Durations and Self-Timed Execution & \href{works/LombardiM10.pdf}{Yes} & \cite{LombardiM10} & 2010 & CP 2010 & 15 & \ref{b:LombardiM10} & \ref{c:LombardiM10}\\
LombardiM10a \href{https://doi.org/10.1016/j.artint.2010.02.004}{LombardiM10a} & \hyperref[auth:a142]{M. Lombardi}, \hyperref[auth:a143]{M. Milano} & Allocation and scheduling of Conditional Task Graphs & \href{works/LombardiM10a.pdf}{Yes} & \cite{LombardiM10a} & 2010 & Artif. Intell. & 30 & \ref{b:LombardiM10a} & \ref{c:LombardiM10a}\\
LombardiM09 \href{https://doi.org/10.1007/978-3-642-04244-7\_45}{LombardiM09} & \hyperref[auth:a142]{M. Lombardi}, \hyperref[auth:a143]{M. Milano} & A Precedence Constraint Posting Approach for the {RCPSP} with Time Lags and Variable Durations & \href{works/LombardiM09.pdf}{Yes} & \cite{LombardiM09} & 2009 & CP 2009 & 15 & \ref{b:LombardiM09} & \ref{c:LombardiM09}\\
RuggieroBBMA09 \href{https://doi.org/10.1109/TCAD.2009.2013536}{RuggieroBBMA09} & \hyperref[auth:a727]{M. Ruggiero}, \hyperref[auth:a379]{D. Bertozzi}, \hyperref[auth:a247]{L. Benini}, \hyperref[auth:a143]{M. Milano}, \hyperref[auth:a728]{A. Andrei} & Reducing the Abstraction and Optimality Gaps in the Allocation and Scheduling for Variable Voltage/Frequency MPSoC Platforms & \href{works/RuggieroBBMA09.pdf}{Yes} & \cite{RuggieroBBMA09} & 2009 & {IEEE} Trans. Comput. Aided Des. Integr. Circuits Syst. & 14 & \ref{b:RuggieroBBMA09} & \ref{c:RuggieroBBMA09}\\
BeniniBGM06 \href{https://doi.org/10.1007/11757375\_6}{BeniniBGM06} & \hyperref[auth:a247]{L. Benini}, \hyperref[auth:a379]{D. Bertozzi}, \hyperref[auth:a380]{A. Guerri}, \hyperref[auth:a143]{M. Milano} & Allocation, Scheduling and Voltage Scaling on Energy Aware MPSoCs & \href{works/BeniniBGM06.pdf}{Yes} & \cite{BeniniBGM06} & 2006 & CPAIOR 2006 & 15 & \ref{b:BeniniBGM06} & \ref{c:BeniniBGM06}\\
LammaMM97 \href{https://doi.org/10.1016/S0954-1810(96)00002-7}{LammaMM97} & \hyperref[auth:a729]{E. Lamma}, \hyperref[auth:a730]{P. Mello}, \hyperref[auth:a143]{M. Milano} & A distributed constraint-based scheduler & \href{works/LammaMM97.pdf}{Yes} & \cite{LammaMM97} & 1997 & Artif. Intell. Eng. & 15 & \ref{b:LammaMM97} & \ref{c:LammaMM97}\\
BrusoniCLMMT96 \href{https://doi.org/10.1007/3-540-61286-6\_157}{BrusoniCLMMT96} & \hyperref[auth:a731]{V. Brusoni}, \hyperref[auth:a732]{L. Console}, \hyperref[auth:a729]{E. Lamma}, \hyperref[auth:a730]{P. Mello}, \hyperref[auth:a143]{M. Milano}, \hyperref[auth:a733]{P. Terenziani} & Resource-Based vs. Task-Based Approaches for Scheduling Problems & \href{works/BrusoniCLMMT96.pdf}{Yes} & \cite{BrusoniCLMMT96} & 1996 & ISMIS 1996 & 10 & \ref{b:BrusoniCLMMT96} & \ref{c:BrusoniCLMMT96}\\
\end{longtable}
}

\subsection{Works by Andreas Schutt}
\label{sec:a124}
{\scriptsize
\begin{longtable}{>{\raggedright\arraybackslash}p{3cm}>{\raggedright\arraybackslash}p{6cm}>{\raggedright\arraybackslash}p{7cm}rrrp{3cm}rrr}
\rowcolor{white}\caption{Works from bibtex (Total 24)}\\ \toprule
\rowcolor{white}Key & Authors & Title & LC & Cite & Year & \shortstack{Conference\\/Journal} & Pages & b & c \\ \midrule\endhead
\bottomrule
\endfoot
YangSS19 \href{https://doi.org/10.1007/978-3-030-19212-9\_42}{YangSS19} & \hyperref[auth:a311]{M. Yang}, \hyperref[auth:a124]{A. Schutt}, \hyperref[auth:a125]{Peter J. Stuckey} & Time Table Edge Finding with Energy Variables & \href{works/YangSS19.pdf}{Yes} & \cite{YangSS19} & 2019 & CPAIOR 2019 & 10 & \ref{b:YangSS19} & \ref{c:YangSS19}\\
GoldwaserS18 \href{https://doi.org/10.1613/jair.1.11268}{GoldwaserS18} & \hyperref[auth:a194]{A. Goldwaser}, \hyperref[auth:a124]{A. Schutt} & Optimal Torpedo Scheduling & \href{works/GoldwaserS18.pdf}{Yes} & \cite{GoldwaserS18} & 2018 & J. Artif. Intell. Res. & 32 & \ref{b:GoldwaserS18} & \ref{c:GoldwaserS18}\\
KreterSSZ18 \href{https://doi.org/10.1016/j.ejor.2017.10.014}{KreterSSZ18} & \hyperref[auth:a123]{S. Kreter}, \hyperref[auth:a124]{A. Schutt}, \hyperref[auth:a125]{Peter J. Stuckey}, \hyperref[auth:a803]{J. Zimmermann} & Mixed-integer linear programming and constraint programming formulations for solving resource availability cost problems & No & \cite{KreterSSZ18} & 2018 & Eur. J. Oper. Res. & 15 & No & \ref{c:KreterSSZ18}\\
MusliuSS18 \href{https://doi.org/10.1007/978-3-319-93031-2\_31}{MusliuSS18} & \hyperref[auth:a45]{N. Musliu}, \hyperref[auth:a124]{A. Schutt}, \hyperref[auth:a125]{Peter J. Stuckey} & Solver Independent Rotating Workforce Scheduling & \href{works/MusliuSS18.pdf}{Yes} & \cite{MusliuSS18} & 2018 & CPAIOR 2018 & 17 & \ref{b:MusliuSS18} & \ref{c:MusliuSS18}\\
GoldwaserS17 \href{https://doi.org/10.1007/978-3-319-66158-2\_22}{GoldwaserS17} & \hyperref[auth:a194]{A. Goldwaser}, \hyperref[auth:a124]{A. Schutt} & Optimal Torpedo Scheduling & \href{works/GoldwaserS17.pdf}{Yes} & \cite{GoldwaserS17} & 2017 & CP 2017 & 16 & \ref{b:GoldwaserS17} & \ref{c:GoldwaserS17}\\
KreterSS17 \href{https://doi.org/10.1007/s10601-016-9266-6}{KreterSS17} & \hyperref[auth:a123]{S. Kreter}, \hyperref[auth:a124]{A. Schutt}, \hyperref[auth:a125]{Peter J. Stuckey} & Using constraint programming for solving RCPSP/max-cal & \href{works/KreterSS17.pdf}{Yes} & \cite{KreterSS17} & 2017 & Constraints An Int. J. & 31 & \ref{b:KreterSS17} & \ref{c:KreterSS17}\\
YoungFS17 \href{https://doi.org/10.1007/978-3-319-66158-2\_20}{YoungFS17} & \hyperref[auth:a193]{Kenneth D. Young}, \hyperref[auth:a154]{T. Feydy}, \hyperref[auth:a124]{A. Schutt} & Constraint Programming Applied to the Multi-Skill Project Scheduling Problem & \href{works/YoungFS17.pdf}{Yes} & \cite{YoungFS17} & 2017 & CP 2017 & 10 & \ref{b:YoungFS17} & \ref{c:YoungFS17}\\
SchuttS16 \href{https://doi.org/10.1007/978-3-319-44953-1\_28}{SchuttS16} & \hyperref[auth:a124]{A. Schutt}, \hyperref[auth:a125]{Peter J. Stuckey} & Explaining Producer/Consumer Constraints & \href{works/SchuttS16.pdf}{Yes} & \cite{SchuttS16} & 2016 & CP 2016 & 17 & \ref{b:SchuttS16} & \ref{c:SchuttS16}\\
SzerediS16 \href{https://doi.org/10.1007/978-3-319-44953-1\_31}{SzerediS16} & \hyperref[auth:a205]{R. Szeredi}, \hyperref[auth:a124]{A. Schutt} & Modelling and Solving Multi-mode Resource-Constrained Project Scheduling & \href{works/SzerediS16.pdf}{Yes} & \cite{SzerediS16} & 2016 & CP 2016 & 10 & \ref{b:SzerediS16} & \ref{c:SzerediS16}\\
EvenSH15 \href{https://doi.org/10.1007/978-3-319-23219-5\_40}{EvenSH15} & \hyperref[auth:a219]{C. Even}, \hyperref[auth:a124]{A. Schutt}, \hyperref[auth:a148]{Pascal Van Hentenryck} & A Constraint Programming Approach for Non-preemptive Evacuation Scheduling & \href{works/EvenSH15.pdf}{Yes} & \cite{EvenSH15} & 2015 & CP 2015 & 18 & \ref{b:EvenSH15} & \ref{c:EvenSH15}\\
EvenSH15a \href{http://arxiv.org/abs/1505.02487}{EvenSH15a} & \hyperref[auth:a219]{C. Even}, \hyperref[auth:a124]{A. Schutt}, \hyperref[auth:a148]{Pascal Van Hentenryck} & A Constraint Programming Approach for Non-Preemptive Evacuation Scheduling & \href{works/EvenSH15a.pdf}{Yes} & \cite{EvenSH15a} & 2015 & CoRR & 16 & \ref{b:EvenSH15a} & \ref{c:EvenSH15a}\\
KreterSS15 \href{https://doi.org/10.1007/978-3-319-23219-5\_19}{KreterSS15} & \hyperref[auth:a123]{S. Kreter}, \hyperref[auth:a124]{A. Schutt}, \hyperref[auth:a125]{Peter J. Stuckey} & Modeling and Solving Project Scheduling with Calendars & \href{works/KreterSS15.pdf}{Yes} & \cite{KreterSS15} & 2015 & CP 2015 & 17 & \ref{b:KreterSS15} & \ref{c:KreterSS15}\\
ThiruvadyWGS14 \href{https://doi.org/10.1007/s10732-014-9260-3}{ThiruvadyWGS14} & \hyperref[auth:a400]{Dhananjay R. Thiruvady}, \hyperref[auth:a117]{M. Wallace}, \hyperref[auth:a341]{H. Gu}, \hyperref[auth:a124]{A. Schutt} & A Lagrangian relaxation and {ACO} hybrid for resource constrained project scheduling with discounted cash flows & \href{works/ThiruvadyWGS14.pdf}{Yes} & \cite{ThiruvadyWGS14} & 2014 & J. Heuristics & 34 & \ref{b:ThiruvadyWGS14} & \ref{c:ThiruvadyWGS14}\\
ChuGNSW13 \href{http://www.aaai.org/ocs/index.php/IJCAI/IJCAI13/paper/view/6878}{ChuGNSW13} & \hyperref[auth:a348]{G. Chu}, \hyperref[auth:a804]{S. Gaspers}, \hyperref[auth:a805]{N. Narodytska}, \hyperref[auth:a124]{A. Schutt}, \hyperref[auth:a278]{T. Walsh} & On the Complexity of Global Scheduling Constraints under Structural Restrictions & \href{works/ChuGNSW13.pdf}{Yes} & \cite{ChuGNSW13} & 2013 & IJCAI 2013 & 7 & \ref{b:ChuGNSW13} & \ref{c:ChuGNSW13}\\
GuSS13 \href{https://doi.org/10.1007/978-3-642-38171-3\_24}{GuSS13} & \hyperref[auth:a341]{H. Gu}, \hyperref[auth:a124]{A. Schutt}, \hyperref[auth:a125]{Peter J. Stuckey} & A Lagrangian Relaxation Based Forward-Backward Improvement Heuristic for Maximising the Net Present Value of Resource-Constrained Projects & \href{works/GuSS13.pdf}{Yes} & \cite{GuSS13} & 2013 & CPAIOR 2013 & 7 & \ref{b:GuSS13} & \ref{c:GuSS13}\\
SchuttFS13 \href{https://doi.org/10.1007/978-3-642-40627-0\_47}{SchuttFS13} & \hyperref[auth:a124]{A. Schutt}, \hyperref[auth:a154]{T. Feydy}, \hyperref[auth:a125]{Peter J. Stuckey} & Scheduling Optional Tasks with Explanation & \href{works/SchuttFS13.pdf}{Yes} & \cite{SchuttFS13} & 2013 & CP 2013 & 17 & \ref{b:SchuttFS13} & \ref{c:SchuttFS13}\\
SchuttFS13a \href{https://doi.org/10.1007/978-3-642-38171-3\_16}{SchuttFS13a} & \hyperref[auth:a124]{A. Schutt}, \hyperref[auth:a154]{T. Feydy}, \hyperref[auth:a125]{Peter J. Stuckey} & Explaining Time-Table-Edge-Finding Propagation for the Cumulative Resource Constraint & \href{works/SchuttFS13a.pdf}{Yes} & \cite{SchuttFS13a} & 2013 & CPAIOR 2013 & 17 & \ref{b:SchuttFS13a} & \ref{c:SchuttFS13a}\\
SchuttFSW13 \href{https://doi.org/10.1007/s10951-012-0285-x}{SchuttFSW13} & \hyperref[auth:a124]{A. Schutt}, \hyperref[auth:a154]{T. Feydy}, \hyperref[auth:a125]{Peter J. Stuckey}, \hyperref[auth:a155]{Mark G. Wallace} & Solving RCPSP/max by lazy clause generation & \href{works/SchuttFSW13.pdf}{Yes} & \cite{SchuttFSW13} & 2013 & J. Sched. & 17 & \ref{b:SchuttFSW13} & \ref{c:SchuttFSW13}\\
SchuttCSW12 \href{https://doi.org/10.1007/978-3-642-29828-8\_24}{SchuttCSW12} & \hyperref[auth:a124]{A. Schutt}, \hyperref[auth:a348]{G. Chu}, \hyperref[auth:a125]{Peter J. Stuckey}, \hyperref[auth:a155]{Mark G. Wallace} & Maximising the Net Present Value for Resource-Constrained Project Scheduling & \href{works/SchuttCSW12.pdf}{Yes} & \cite{SchuttCSW12} & 2012 & CPAIOR 2012 & 17 & \ref{b:SchuttCSW12} & \ref{c:SchuttCSW12}\\
SchuttFSW11 \href{https://doi.org/10.1007/s10601-010-9103-2}{SchuttFSW11} & \hyperref[auth:a124]{A. Schutt}, \hyperref[auth:a154]{T. Feydy}, \hyperref[auth:a125]{Peter J. Stuckey}, \hyperref[auth:a155]{Mark G. Wallace} & Explaining the cumulative propagator & \href{works/SchuttFSW11.pdf}{Yes} & \cite{SchuttFSW11} & 2011 & Constraints An Int. J. & 33 & \ref{b:SchuttFSW11} & \ref{c:SchuttFSW11}\\
SchuttW10 \href{https://doi.org/10.1007/978-3-642-15396-9\_36}{SchuttW10} & \hyperref[auth:a124]{A. Schutt}, \hyperref[auth:a51]{A. Wolf} & A New \emph{O}(\emph{n}\({}^{\mbox{2}}\)log\emph{n}) Not-First/Not-Last Pruning Algorithm for Cumulative Resource Constraints & \href{works/SchuttW10.pdf}{Yes} & \cite{SchuttW10} & 2010 & CP 2010 & 15 & \ref{b:SchuttW10} & \ref{c:SchuttW10}\\
abs-1009-0347 \href{http://arxiv.org/abs/1009.0347}{abs-1009-0347} & \hyperref[auth:a124]{A. Schutt}, \hyperref[auth:a154]{T. Feydy}, \hyperref[auth:a125]{Peter J. Stuckey}, \hyperref[auth:a155]{Mark G. Wallace} & Solving the Resource Constrained Project Scheduling Problem with Generalized Precedences by Lazy Clause Generation & \href{works/abs-1009-0347.pdf}{Yes} & \cite{abs-1009-0347} & 2010 & CoRR & 37 & \ref{b:abs-1009-0347} & \ref{c:abs-1009-0347}\\
SchuttFSW09 \href{https://doi.org/10.1007/978-3-642-04244-7\_58}{SchuttFSW09} & \hyperref[auth:a124]{A. Schutt}, \hyperref[auth:a154]{T. Feydy}, \hyperref[auth:a125]{Peter J. Stuckey}, \hyperref[auth:a117]{M. Wallace} & Why Cumulative Decomposition Is Not as Bad as It Sounds & \href{works/SchuttFSW09.pdf}{Yes} & \cite{SchuttFSW09} & 2009 & CP 2009 & 16 & \ref{b:SchuttFSW09} & \ref{c:SchuttFSW09}\\
SchuttWS05 \href{https://doi.org/10.1007/11963578\_6}{SchuttWS05} & \hyperref[auth:a124]{A. Schutt}, \hyperref[auth:a51]{A. Wolf}, \hyperref[auth:a720]{G. Schrader} & Not-First and Not-Last Detection for Cumulative Scheduling in \emph{O}(\emph{n}\({}^{\mbox{3}}\)log\emph{n}) & \href{works/SchuttWS05.pdf}{Yes} & \cite{SchuttWS05} & 2005 & INAP 2005 & 15 & \ref{b:SchuttWS05} & \ref{c:SchuttWS05}\\
\end{longtable}
}

\subsection{Works by Peter J. Stuckey}
\label{sec:a125}
{\scriptsize
\begin{longtable}{>{\raggedright\arraybackslash}p{3cm}>{\raggedright\arraybackslash}p{6cm}>{\raggedright\arraybackslash}p{7cm}rrrp{3cm}rrr}
\rowcolor{white}\caption{Works from bibtex (Total 21)}\\ \toprule
\rowcolor{white}Key & Authors & Title & LC & Cite & Year & \shortstack{Conference\\/Journal} & Pages & b & c \\ \midrule\endhead
\bottomrule
\endfoot
YangSS19 \href{https://doi.org/10.1007/978-3-030-19212-9\_42}{YangSS19} & \hyperref[auth:a311]{M. Yang}, \hyperref[auth:a124]{A. Schutt}, \hyperref[auth:a125]{Peter J. Stuckey} & Time Table Edge Finding with Energy Variables & \href{works/YangSS19.pdf}{Yes} & \cite{YangSS19} & 2019 & CPAIOR 2019 & 10 & \ref{b:YangSS19} & \ref{c:YangSS19}\\
DemirovicS18 \href{https://doi.org/10.1007/978-3-319-93031-2\_10}{DemirovicS18} & \hyperref[auth:a314]{E. Demirovic}, \hyperref[auth:a125]{Peter J. Stuckey} & Constraint Programming for High School Timetabling: {A} Scheduling-Based Model with Hot Starts & \href{works/DemirovicS18.pdf}{Yes} & \cite{DemirovicS18} & 2018 & CPAIOR 2018 & 18 & \ref{b:DemirovicS18} & \ref{c:DemirovicS18}\\
KreterSSZ18 \href{https://doi.org/10.1016/j.ejor.2017.10.014}{KreterSSZ18} & \hyperref[auth:a123]{S. Kreter}, \hyperref[auth:a124]{A. Schutt}, \hyperref[auth:a125]{Peter J. Stuckey}, \hyperref[auth:a803]{J. Zimmermann} & Mixed-integer linear programming and constraint programming formulations for solving resource availability cost problems & No & \cite{KreterSSZ18} & 2018 & Eur. J. Oper. Res. & 15 & No & \ref{c:KreterSSZ18}\\
MusliuSS18 \href{https://doi.org/10.1007/978-3-319-93031-2\_31}{MusliuSS18} & \hyperref[auth:a45]{N. Musliu}, \hyperref[auth:a124]{A. Schutt}, \hyperref[auth:a125]{Peter J. Stuckey} & Solver Independent Rotating Workforce Scheduling & \href{works/MusliuSS18.pdf}{Yes} & \cite{MusliuSS18} & 2018 & CPAIOR 2018 & 17 & \ref{b:MusliuSS18} & \ref{c:MusliuSS18}\\
KreterSS17 \href{https://doi.org/10.1007/s10601-016-9266-6}{KreterSS17} & \hyperref[auth:a123]{S. Kreter}, \hyperref[auth:a124]{A. Schutt}, \hyperref[auth:a125]{Peter J. Stuckey} & Using constraint programming for solving RCPSP/max-cal & \href{works/KreterSS17.pdf}{Yes} & \cite{KreterSS17} & 2017 & Constraints An Int. J. & 31 & \ref{b:KreterSS17} & \ref{c:KreterSS17}\\
BlomPS16 \href{https://doi.org/10.1287/mnsc.2015.2284}{BlomPS16} & \hyperref[auth:a806]{Michelle L. Blom}, \hyperref[auth:a327]{Adrian R. Pearce}, \hyperref[auth:a125]{Peter J. Stuckey} & A Decomposition-Based Algorithm for the Scheduling of Open-Pit Networks Over Multiple Time Periods & No & \cite{BlomPS16} & 2016 & Manag. Sci. & 26 & No & \ref{c:BlomPS16}\\
SchuttS16 \href{https://doi.org/10.1007/978-3-319-44953-1\_28}{SchuttS16} & \hyperref[auth:a124]{A. Schutt}, \hyperref[auth:a125]{Peter J. Stuckey} & Explaining Producer/Consumer Constraints & \href{works/SchuttS16.pdf}{Yes} & \cite{SchuttS16} & 2016 & CP 2016 & 17 & \ref{b:SchuttS16} & \ref{c:SchuttS16}\\
BurtLPS15 \href{https://doi.org/10.1007/978-3-319-18008-3\_7}{BurtLPS15} & \hyperref[auth:a325]{Christina N. Burt}, \hyperref[auth:a326]{N. Lipovetzky}, \hyperref[auth:a327]{Adrian R. Pearce}, \hyperref[auth:a125]{Peter J. Stuckey} & Scheduling with Fixed Maintenance, Shared Resources and Nonlinear Feedrate Constraints: {A} Mine Planning Case Study & \href{works/BurtLPS15.pdf}{Yes} & \cite{BurtLPS15} & 2015 & CPAIOR 2015 & 17 & \ref{b:BurtLPS15} & \ref{c:BurtLPS15}\\
KreterSS15 \href{https://doi.org/10.1007/978-3-319-23219-5\_19}{KreterSS15} & \hyperref[auth:a123]{S. Kreter}, \hyperref[auth:a124]{A. Schutt}, \hyperref[auth:a125]{Peter J. Stuckey} & Modeling and Solving Project Scheduling with Calendars & \href{works/KreterSS15.pdf}{Yes} & \cite{KreterSS15} & 2015 & CP 2015 & 17 & \ref{b:KreterSS15} & \ref{c:KreterSS15}\\
BlomBPS14 \href{https://doi.org/10.1287/ijoc.2013.0590}{BlomBPS14} & \hyperref[auth:a806]{Michelle L. Blom}, \hyperref[auth:a325]{Christina N. Burt}, \hyperref[auth:a327]{Adrian R. Pearce}, \hyperref[auth:a125]{Peter J. Stuckey} & A Decomposition-Based Heuristic for Collaborative Scheduling in a Network of Open-Pit Mines & No & \cite{BlomBPS14} & 2014 & {INFORMS} J. Comput. & 19 & No & \ref{c:BlomBPS14}\\
LipovetzkyBPS14 \href{http://www.aaai.org/ocs/index.php/ICAPS/ICAPS14/paper/view/7942}{LipovetzkyBPS14} & \hyperref[auth:a326]{N. Lipovetzky}, \hyperref[auth:a325]{Christina N. Burt}, \hyperref[auth:a327]{Adrian R. Pearce}, \hyperref[auth:a125]{Peter J. Stuckey} & Planning for Mining Operations with Time and Resource Constraints & \href{works/LipovetzkyBPS14.pdf}{Yes} & \cite{LipovetzkyBPS14} & 2014 & ICAPS 2014 & 9 & \ref{b:LipovetzkyBPS14} & \ref{c:LipovetzkyBPS14}\\
GuSS13 \href{https://doi.org/10.1007/978-3-642-38171-3\_24}{GuSS13} & \hyperref[auth:a341]{H. Gu}, \hyperref[auth:a124]{A. Schutt}, \hyperref[auth:a125]{Peter J. Stuckey} & A Lagrangian Relaxation Based Forward-Backward Improvement Heuristic for Maximising the Net Present Value of Resource-Constrained Projects & \href{works/GuSS13.pdf}{Yes} & \cite{GuSS13} & 2013 & CPAIOR 2013 & 7 & \ref{b:GuSS13} & \ref{c:GuSS13}\\
SchuttFS13 \href{https://doi.org/10.1007/978-3-642-40627-0\_47}{SchuttFS13} & \hyperref[auth:a124]{A. Schutt}, \hyperref[auth:a154]{T. Feydy}, \hyperref[auth:a125]{Peter J. Stuckey} & Scheduling Optional Tasks with Explanation & \href{works/SchuttFS13.pdf}{Yes} & \cite{SchuttFS13} & 2013 & CP 2013 & 17 & \ref{b:SchuttFS13} & \ref{c:SchuttFS13}\\
SchuttFS13a \href{https://doi.org/10.1007/978-3-642-38171-3\_16}{SchuttFS13a} & \hyperref[auth:a124]{A. Schutt}, \hyperref[auth:a154]{T. Feydy}, \hyperref[auth:a125]{Peter J. Stuckey} & Explaining Time-Table-Edge-Finding Propagation for the Cumulative Resource Constraint & \href{works/SchuttFS13a.pdf}{Yes} & \cite{SchuttFS13a} & 2013 & CPAIOR 2013 & 17 & \ref{b:SchuttFS13a} & \ref{c:SchuttFS13a}\\
SchuttFSW13 \href{https://doi.org/10.1007/s10951-012-0285-x}{SchuttFSW13} & \hyperref[auth:a124]{A. Schutt}, \hyperref[auth:a154]{T. Feydy}, \hyperref[auth:a125]{Peter J. Stuckey}, \hyperref[auth:a155]{Mark G. Wallace} & Solving RCPSP/max by lazy clause generation & \href{works/SchuttFSW13.pdf}{Yes} & \cite{SchuttFSW13} & 2013 & J. Sched. & 17 & \ref{b:SchuttFSW13} & \ref{c:SchuttFSW13}\\
GuSW12 \href{https://doi.org/10.1007/978-3-642-33558-7\_55}{GuSW12} & \hyperref[auth:a341]{H. Gu}, \hyperref[auth:a125]{Peter J. Stuckey}, \hyperref[auth:a155]{Mark G. Wallace} & Maximising the Net Present Value of Large Resource-Constrained Projects & \href{works/GuSW12.pdf}{Yes} & \cite{GuSW12} & 2012 & CP 2012 & 15 & \ref{b:GuSW12} & \ref{c:GuSW12}\\
SchuttCSW12 \href{https://doi.org/10.1007/978-3-642-29828-8\_24}{SchuttCSW12} & \hyperref[auth:a124]{A. Schutt}, \hyperref[auth:a348]{G. Chu}, \hyperref[auth:a125]{Peter J. Stuckey}, \hyperref[auth:a155]{Mark G. Wallace} & Maximising the Net Present Value for Resource-Constrained Project Scheduling & \href{works/SchuttCSW12.pdf}{Yes} & \cite{SchuttCSW12} & 2012 & CPAIOR 2012 & 17 & \ref{b:SchuttCSW12} & \ref{c:SchuttCSW12}\\
BandaSC11 \href{https://doi.org/10.1287/ijoc.1090.0378}{BandaSC11} & \hyperref[auth:a807]{Maria Garcia de la Banda}, \hyperref[auth:a125]{Peter J. Stuckey}, \hyperref[auth:a348]{G. Chu} & Solving Talent Scheduling with Dynamic Programming & No & \cite{BandaSC11} & 2011 & {INFORMS} J. Comput. & 18 & No & \ref{c:BandaSC11}\\
SchuttFSW11 \href{https://doi.org/10.1007/s10601-010-9103-2}{SchuttFSW11} & \hyperref[auth:a124]{A. Schutt}, \hyperref[auth:a154]{T. Feydy}, \hyperref[auth:a125]{Peter J. Stuckey}, \hyperref[auth:a155]{Mark G. Wallace} & Explaining the cumulative propagator & \href{works/SchuttFSW11.pdf}{Yes} & \cite{SchuttFSW11} & 2011 & Constraints An Int. J. & 33 & \ref{b:SchuttFSW11} & \ref{c:SchuttFSW11}\\
abs-1009-0347 \href{http://arxiv.org/abs/1009.0347}{abs-1009-0347} & \hyperref[auth:a124]{A. Schutt}, \hyperref[auth:a154]{T. Feydy}, \hyperref[auth:a125]{Peter J. Stuckey}, \hyperref[auth:a155]{Mark G. Wallace} & Solving the Resource Constrained Project Scheduling Problem with Generalized Precedences by Lazy Clause Generation & \href{works/abs-1009-0347.pdf}{Yes} & \cite{abs-1009-0347} & 2010 & CoRR & 37 & \ref{b:abs-1009-0347} & \ref{c:abs-1009-0347}\\
SchuttFSW09 \href{https://doi.org/10.1007/978-3-642-04244-7\_58}{SchuttFSW09} & \hyperref[auth:a124]{A. Schutt}, \hyperref[auth:a154]{T. Feydy}, \hyperref[auth:a125]{Peter J. Stuckey}, \hyperref[auth:a117]{M. Wallace} & Why Cumulative Decomposition Is Not as Bad as It Sounds & \href{works/SchuttFSW09.pdf}{Yes} & \cite{SchuttFSW09} & 2009 & CP 2009 & 16 & \ref{b:SchuttFSW09} & \ref{c:SchuttFSW09}\\
\end{longtable}
}

\subsection{Works by Michele Lombardi}
\label{sec:a142}
{\scriptsize
\begin{longtable}{>{\raggedright\arraybackslash}p{3cm}>{\raggedright\arraybackslash}p{6cm}>{\raggedright\arraybackslash}p{7cm}rrrp{3cm}rrr}
\rowcolor{white}\caption{Works from bibtex (Total 20)}\\ \toprule
\rowcolor{white}Key & Authors & Title & LC & Cite & Year & \shortstack{Conference\\/Journal} & Pages & b & c \\ \midrule\endhead
\bottomrule
\endfoot
BorghesiBLMB18 \href{https://doi.org/10.1016/j.suscom.2018.05.007}{BorghesiBLMB18} & \hyperref[auth:a231]{A. Borghesi}, \hyperref[auth:a230]{A. Bartolini}, \hyperref[auth:a142]{M. Lombardi}, \hyperref[auth:a143]{M. Milano}, \hyperref[auth:a247]{L. Benini} & Scheduling-based power capping in high performance computing systems & \href{works/BorghesiBLMB18.pdf}{Yes} & \cite{BorghesiBLMB18} & 2018 & Sustain. Comput. Informatics Syst. & 13 & \ref{b:BorghesiBLMB18} & \ref{c:BorghesiBLMB18}\\
BonfiettiZLM16 \href{https://doi.org/10.1007/978-3-319-44953-1\_8}{BonfiettiZLM16} & \hyperref[auth:a203]{A. Bonfietti}, \hyperref[auth:a204]{A. Zanarini}, \hyperref[auth:a142]{M. Lombardi}, \hyperref[auth:a143]{M. Milano} & The Multirate Resource Constraint & \href{works/BonfiettiZLM16.pdf}{Yes} & \cite{BonfiettiZLM16} & 2016 & CP 2016 & 17 & \ref{b:BonfiettiZLM16} & \ref{c:BonfiettiZLM16}\\
BridiBLMB16 \href{https://doi.org/10.1109/TPDS.2016.2516997}{BridiBLMB16} & \hyperref[auth:a232]{T. Bridi}, \hyperref[auth:a230]{A. Bartolini}, \hyperref[auth:a142]{M. Lombardi}, \hyperref[auth:a143]{M. Milano}, \hyperref[auth:a247]{L. Benini} & A Constraint Programming Scheduler for Heterogeneous High-Performance Computing Machines & \href{works/BridiBLMB16.pdf}{Yes} & \cite{BridiBLMB16} & 2016 & {IEEE} Trans. Parallel Distributed Syst. & 14 & \ref{b:BridiBLMB16} & \ref{c:BridiBLMB16}\\
BridiLBBM16 \href{https://doi.org/10.3233/978-1-61499-672-9-1598}{BridiLBBM16} & \hyperref[auth:a232]{T. Bridi}, \hyperref[auth:a142]{M. Lombardi}, \hyperref[auth:a230]{A. Bartolini}, \hyperref[auth:a247]{L. Benini}, \hyperref[auth:a143]{M. Milano} & {DARDIS:} Distributed And Randomized DIspatching and Scheduling & \href{works/BridiLBBM16.pdf}{Yes} & \cite{BridiLBBM16} & 2016 & ECAI 2016 & 2 & \ref{b:BridiLBBM16} & \ref{c:BridiLBBM16}\\
LombardiBM15 \href{https://doi.org/10.1007/978-3-319-23219-5\_20}{LombardiBM15} & \hyperref[auth:a142]{M. Lombardi}, \hyperref[auth:a203]{A. Bonfietti}, \hyperref[auth:a143]{M. Milano} & Deterministic Estimation of the Expected Makespan of a {POS} Under Duration Uncertainty & \href{works/LombardiBM15.pdf}{Yes} & \cite{LombardiBM15} & 2015 & CP 2015 & 16 & \ref{b:LombardiBM15} & \ref{c:LombardiBM15}\\
BartoliniBBLM14 \href{https://doi.org/10.1007/978-3-319-10428-7\_55}{BartoliniBBLM14} & \hyperref[auth:a230]{A. Bartolini}, \hyperref[auth:a231]{A. Borghesi}, \hyperref[auth:a232]{T. Bridi}, \hyperref[auth:a142]{M. Lombardi}, \hyperref[auth:a143]{M. Milano} & Proactive Workload Dispatching on the {EURORA} Supercomputer & \href{works/BartoliniBBLM14.pdf}{Yes} & \cite{BartoliniBBLM14} & 2014 & CP 2014 & 16 & \ref{b:BartoliniBBLM14} & \ref{c:BartoliniBBLM14}\\
BonfiettiLBM14 \href{https://doi.org/10.1016/j.artint.2013.09.006}{BonfiettiLBM14} & \hyperref[auth:a203]{A. Bonfietti}, \hyperref[auth:a142]{M. Lombardi}, \hyperref[auth:a247]{L. Benini}, \hyperref[auth:a143]{M. Milano} & {CROSS} cyclic resource-constrained scheduling solver & \href{works/BonfiettiLBM14.pdf}{Yes} & \cite{BonfiettiLBM14} & 2014 & Artif. Intell. & 28 & \ref{b:BonfiettiLBM14} & \ref{c:BonfiettiLBM14}\\
BonfiettiLM14 \href{https://doi.org/10.1007/978-3-319-07046-9\_15}{BonfiettiLM14} & \hyperref[auth:a203]{A. Bonfietti}, \hyperref[auth:a142]{M. Lombardi}, \hyperref[auth:a143]{M. Milano} & Disregarding Duration Uncertainty in Partial Order Schedules? Yes, We Can! & \href{works/BonfiettiLM14.pdf}{Yes} & \cite{BonfiettiLM14} & 2014 & CPAIOR 2014 & 16 & \ref{b:BonfiettiLM14} & \ref{c:BonfiettiLM14}\\
BonfiettiLM13 \href{http://www.aaai.org/ocs/index.php/ICAPS/ICAPS13/paper/view/6050}{BonfiettiLM13} & \hyperref[auth:a203]{A. Bonfietti}, \hyperref[auth:a142]{M. Lombardi}, \hyperref[auth:a143]{M. Milano} & De-Cycling Cyclic Scheduling Problems & \href{works/BonfiettiLM13.pdf}{Yes} & \cite{BonfiettiLM13} & 2013 & ICAPS 2013 & 5 & \ref{b:BonfiettiLM13} & \ref{c:BonfiettiLM13}\\
LombardiM13 \href{http://www.aaai.org/ocs/index.php/ICAPS/ICAPS13/paper/view/6052}{LombardiM13} & \hyperref[auth:a142]{M. Lombardi}, \hyperref[auth:a143]{M. Milano} & A Min-Flow Algorithm for Minimal Critical Set Detection in Resource Constrained Project Scheduling & \href{works/LombardiM13.pdf}{Yes} & \cite{LombardiM13} & 2013 & ICAPS 2013 & 2 & \ref{b:LombardiM13} & \ref{c:LombardiM13}\\
BonfiettiLBM12 \href{https://doi.org/10.1007/978-3-642-29828-8\_6}{BonfiettiLBM12} & \hyperref[auth:a203]{A. Bonfietti}, \hyperref[auth:a142]{M. Lombardi}, \hyperref[auth:a247]{L. Benini}, \hyperref[auth:a143]{M. Milano} & Global Cyclic Cumulative Constraint & \href{works/BonfiettiLBM12.pdf}{Yes} & \cite{BonfiettiLBM12} & 2012 & CPAIOR 2012 & 16 & \ref{b:BonfiettiLBM12} & \ref{c:BonfiettiLBM12}\\
LombardiM12 \href{https://doi.org/10.1007/s10601-011-9115-6}{LombardiM12} & \hyperref[auth:a142]{M. Lombardi}, \hyperref[auth:a143]{M. Milano} & Optimal methods for resource allocation and scheduling: a cross-disciplinary survey & \href{works/LombardiM12.pdf}{Yes} & \cite{LombardiM12} & 2012 & Constraints An Int. J. & 35 & \ref{b:LombardiM12} & \ref{c:LombardiM12}\\
LombardiM12a \href{https://doi.org/10.1016/j.artint.2011.12.001}{LombardiM12a} & \hyperref[auth:a142]{M. Lombardi}, \hyperref[auth:a143]{M. Milano} & A min-flow algorithm for Minimal Critical Set detection in Resource Constrained Project Scheduling & \href{works/LombardiM12a.pdf}{Yes} & \cite{LombardiM12a} & 2012 & Artif. Intell. & 10 & \ref{b:LombardiM12a} & \ref{c:LombardiM12a}\\
BeniniLMR11 \href{https://doi.org/10.1007/s10479-010-0718-x}{BeniniLMR11} & \hyperref[auth:a247]{L. Benini}, \hyperref[auth:a142]{M. Lombardi}, \hyperref[auth:a143]{M. Milano}, \hyperref[auth:a727]{M. Ruggiero} & Optimal resource allocation and scheduling for the {CELL} {BE} platform & \href{works/BeniniLMR11.pdf}{Yes} & \cite{BeniniLMR11} & 2011 & Ann. Oper. Res. & 27 & \ref{b:BeniniLMR11} & \ref{c:BeniniLMR11}\\
BonfiettiLBM11 \href{https://doi.org/10.1007/978-3-642-23786-7\_12}{BonfiettiLBM11} & \hyperref[auth:a203]{A. Bonfietti}, \hyperref[auth:a142]{M. Lombardi}, \hyperref[auth:a247]{L. Benini}, \hyperref[auth:a143]{M. Milano} & A Constraint Based Approach to Cyclic {RCPSP} & \href{works/BonfiettiLBM11.pdf}{Yes} & \cite{BonfiettiLBM11} & 2011 & CP 2011 & 15 & \ref{b:BonfiettiLBM11} & \ref{c:BonfiettiLBM11}\\
LombardiBMB11 \href{https://doi.org/10.1007/978-3-642-21311-3\_14}{LombardiBMB11} & \hyperref[auth:a142]{M. Lombardi}, \hyperref[auth:a203]{A. Bonfietti}, \hyperref[auth:a143]{M. Milano}, \hyperref[auth:a247]{L. Benini} & Precedence Constraint Posting for Cyclic Scheduling Problems & \href{works/LombardiBMB11.pdf}{Yes} & \cite{LombardiBMB11} & 2011 & CPAIOR 2011 & 17 & \ref{b:LombardiBMB11} & \ref{c:LombardiBMB11}\\
LombardiM10 \href{https://doi.org/10.1007/978-3-642-15396-9\_32}{LombardiM10} & \hyperref[auth:a142]{M. Lombardi}, \hyperref[auth:a143]{M. Milano} & Constraint Based Scheduling to Deal with Uncertain Durations and Self-Timed Execution & \href{works/LombardiM10.pdf}{Yes} & \cite{LombardiM10} & 2010 & CP 2010 & 15 & \ref{b:LombardiM10} & \ref{c:LombardiM10}\\
LombardiM10a \href{https://doi.org/10.1016/j.artint.2010.02.004}{LombardiM10a} & \hyperref[auth:a142]{M. Lombardi}, \hyperref[auth:a143]{M. Milano} & Allocation and scheduling of Conditional Task Graphs & \href{works/LombardiM10a.pdf}{Yes} & \cite{LombardiM10a} & 2010 & Artif. Intell. & 30 & \ref{b:LombardiM10a} & \ref{c:LombardiM10a}\\
LombardiM09 \href{https://doi.org/10.1007/978-3-642-04244-7\_45}{LombardiM09} & \hyperref[auth:a142]{M. Lombardi}, \hyperref[auth:a143]{M. Milano} & A Precedence Constraint Posting Approach for the {RCPSP} with Time Lags and Variable Durations & \href{works/LombardiM09.pdf}{Yes} & \cite{LombardiM09} & 2009 & CP 2009 & 15 & \ref{b:LombardiM09} & \ref{c:LombardiM09}\\
HoeveGSL07 \href{http://www.aaai.org/Library/AAAI/2007/aaai07-291.php}{HoeveGSL07} & \hyperref[auth:a651]{Willem Jan van Hoeve}, \hyperref[auth:a652]{Carla P. Gomes}, \hyperref[auth:a653]{B. Selman}, \hyperref[auth:a142]{M. Lombardi} & Optimal Multi-Agent Scheduling with Constraint Programming & \href{works/HoeveGSL07.pdf}{Yes} & \cite{HoeveGSL07} & 2007 & AAAI 2007 & 6 & \ref{b:HoeveGSL07} & \ref{c:HoeveGSL07}\\
\end{longtable}
}

\subsection{Works by Emmanuel Hebrard}
\label{sec:a1}
{\scriptsize
\begin{longtable}{>{\raggedright\arraybackslash}p{3cm}>{\raggedright\arraybackslash}p{6cm}>{\raggedright\arraybackslash}p{7cm}rrrp{3cm}rrr}
\rowcolor{white}\caption{Works from bibtex (Total 17)}\\ \toprule
\rowcolor{white}Key & Authors & Title & LC & Cite & Year & \shortstack{Conference\\/Journal} & Pages & b & c \\ \midrule\endhead
\bottomrule
\endfoot
JuvinHHL23 \href{https://doi.org/10.4230/LIPIcs.CP.2023.19}{JuvinHHL23} & \hyperref[auth:a0]{C. Juvin}, \hyperref[auth:a1]{E. Hebrard}, \hyperref[auth:a2]{L. Houssin}, \hyperref[auth:a3]{P. Lopez} & An Efficient Constraint Programming Approach to Preemptive Job Shop Scheduling & \href{works/JuvinHHL23.pdf}{Yes} & \cite{JuvinHHL23} & 2023 & CP 2023 & 16 & \ref{b:JuvinHHL23} & \ref{c:JuvinHHL23}\\
HebrardALLCMR22 \href{https://doi.org/10.24963/ijcai.2022/643}{HebrardALLCMR22} & \hyperref[auth:a1]{E. Hebrard}, \hyperref[auth:a6]{C. Artigues}, \hyperref[auth:a3]{P. Lopez}, \hyperref[auth:a796]{A. Lusson}, \hyperref[auth:a797]{Steve A. Chien}, \hyperref[auth:a798]{A. Maillard}, \hyperref[auth:a799]{Gregg R. Rabideau} & An Efficient Approach to Data Transfer Scheduling for Long Range Space Exploration & \href{works/HebrardALLCMR22.pdf}{Yes} & \cite{HebrardALLCMR22} & 2022 & IJCAI 2022 & 7 & \ref{b:HebrardALLCMR22} & \ref{c:HebrardALLCMR22}\\
AntuoriHHEN21 \href{https://doi.org/10.4230/LIPIcs.CP.2021.14}{AntuoriHHEN21} & \hyperref[auth:a53]{V. Antuori}, \hyperref[auth:a1]{E. Hebrard}, \hyperref[auth:a54]{M. Huguet}, \hyperref[auth:a55]{S. Essodaigui}, \hyperref[auth:a56]{A. Nguyen} & Combining Monte Carlo Tree Search and Depth First Search Methods for a Car Manufacturing Workshop Scheduling Problem & \href{works/AntuoriHHEN21.pdf}{Yes} & \cite{AntuoriHHEN21} & 2021 & CP 2021 & 16 & \ref{b:AntuoriHHEN21} & \ref{c:AntuoriHHEN21}\\
ArtiguesHQT21 \href{https://doi.org/10.5220/0010190101290136}{ArtiguesHQT21} & \hyperref[auth:a6]{C. Artigues}, \hyperref[auth:a1]{E. Hebrard}, \hyperref[auth:a800]{A. Quilliot}, \hyperref[auth:a801]{H. Toussaint} & Multi-Mode {RCPSP} with Safety Margin Maximization: Models and Algorithms & No & \cite{ArtiguesHQT21} & 2021 & ICORES 2021 & 8 & No & \ref{c:ArtiguesHQT21}\\
AntuoriHHEN20 \href{https://doi.org/10.1007/978-3-030-58475-7\_38}{AntuoriHHEN20} & \hyperref[auth:a53]{V. Antuori}, \hyperref[auth:a1]{E. Hebrard}, \hyperref[auth:a54]{M. Huguet}, \hyperref[auth:a55]{S. Essodaigui}, \hyperref[auth:a56]{A. Nguyen} & Leveraging Reinforcement Learning, Constraint Programming and Local Search: {A} Case Study in Car Manufacturing & \href{works/AntuoriHHEN20.pdf}{Yes} & \cite{AntuoriHHEN20} & 2020 & CP 2020 & 16 & \ref{b:AntuoriHHEN20} & \ref{c:AntuoriHHEN20}\\
GodetLHS20 \href{https://doi.org/10.1609/aaai.v34i02.5510}{GodetLHS20} & \hyperref[auth:a476]{A. Godet}, \hyperref[auth:a246]{X. Lorca}, \hyperref[auth:a1]{E. Hebrard}, \hyperref[auth:a126]{G. Simonin} & Using Approximation within Constraint Programming to Solve the Parallel Machine Scheduling Problem with Additional Unit Resources & \href{works/GodetLHS20.pdf}{Yes} & \cite{GodetLHS20} & 2020 & AAAI 2020 & 8 & \ref{b:GodetLHS20} & \ref{c:GodetLHS20}\\
HebrardHJMPV16 \href{https://doi.org/10.1016/j.dam.2016.07.003}{HebrardHJMPV16} & \hyperref[auth:a1]{E. Hebrard}, \hyperref[auth:a54]{M. Huguet}, \hyperref[auth:a802]{N. Jozefowiez}, \hyperref[auth:a798]{A. Maillard}, \hyperref[auth:a21]{C. Pralet}, \hyperref[auth:a174]{G. Verfaillie} & Approximation of the parallel machine scheduling problem with additional unit resources & \href{works/HebrardHJMPV16.pdf}{Yes} & \cite{HebrardHJMPV16} & 2016 & Discret. Appl. Math. & 10 & \ref{b:HebrardHJMPV16} & \ref{c:HebrardHJMPV16}\\
GrimesH15 \href{https://doi.org/10.1287/ijoc.2014.0625}{GrimesH15} & \hyperref[auth:a182]{D. Grimes}, \hyperref[auth:a1]{E. Hebrard} & Solving Variants of the Job Shop Scheduling Problem Through Conflict-Directed Search & No & \cite{GrimesH15} & 2015 & {INFORMS} J. Comput. & 17 & No & \ref{c:GrimesH15}\\
SialaAH15 \href{https://doi.org/10.1007/978-3-319-23219-5\_28}{SialaAH15} & \hyperref[auth:a129]{M. Siala}, \hyperref[auth:a6]{C. Artigues}, \hyperref[auth:a1]{E. Hebrard} & Two Clause Learning Approaches for Disjunctive Scheduling & \href{works/SialaAH15.pdf}{Yes} & \cite{SialaAH15} & 2015 & CP 2015 & 10 & \ref{b:SialaAH15} & \ref{c:SialaAH15}\\
SimoninAHL15 \href{https://doi.org/10.1007/s10601-014-9169-3}{SimoninAHL15} & \hyperref[auth:a126]{G. Simonin}, \hyperref[auth:a6]{C. Artigues}, \hyperref[auth:a1]{E. Hebrard}, \hyperref[auth:a3]{P. Lopez} & Scheduling scientific experiments for comet exploration & \href{works/SimoninAHL15.pdf}{Yes} & \cite{SimoninAHL15} & 2015 & Constraints An Int. J. & 23 & \ref{b:SimoninAHL15} & \ref{c:SimoninAHL15}\\
BessiereHMQW14 \href{https://doi.org/10.1007/978-3-319-07046-9\_23}{BessiereHMQW14} & \hyperref[auth:a333]{C. Bessiere}, \hyperref[auth:a1]{E. Hebrard}, \hyperref[auth:a334]{M. M{\'{e}}nard}, \hyperref[auth:a37]{C. Quimper}, \hyperref[auth:a278]{T. Walsh} & Buffered Resource Constraint: Algorithms and Complexity & \href{works/BessiereHMQW14.pdf}{Yes} & \cite{BessiereHMQW14} & 2014 & CPAIOR 2014 & 16 & \ref{b:BessiereHMQW14} & \ref{c:BessiereHMQW14}\\
BillautHL12 \href{https://doi.org/10.1007/978-3-642-29828-8\_5}{BillautHL12} & \hyperref[auth:a342]{J. Billaut}, \hyperref[auth:a1]{E. Hebrard}, \hyperref[auth:a3]{P. Lopez} & Complete Characterization of Near-Optimal Sequences for the Two-Machine Flow Shop Scheduling Problem & \href{works/BillautHL12.pdf}{Yes} & \cite{BillautHL12} & 2012 & CPAIOR 2012 & 15 & \ref{b:BillautHL12} & \ref{c:BillautHL12}\\
SimoninAHL12 \href{https://doi.org/10.1007/978-3-642-33558-7\_5}{SimoninAHL12} & \hyperref[auth:a126]{G. Simonin}, \hyperref[auth:a6]{C. Artigues}, \hyperref[auth:a1]{E. Hebrard}, \hyperref[auth:a3]{P. Lopez} & Scheduling Scientific Experiments on the Rosetta/Philae Mission & \href{works/SimoninAHL12.pdf}{Yes} & \cite{SimoninAHL12} & 2012 & CP 2012 & 15 & \ref{b:SimoninAHL12} & \ref{c:SimoninAHL12}\\
GrimesH11 \href{https://doi.org/10.1007/978-3-642-23786-7\_28}{GrimesH11} & \hyperref[auth:a182]{D. Grimes}, \hyperref[auth:a1]{E. Hebrard} & Models and Strategies for Variants of the Job Shop Scheduling Problem & \href{works/GrimesH11.pdf}{Yes} & \cite{GrimesH11} & 2011 & CP 2011 & 17 & \ref{b:GrimesH11} & \ref{c:GrimesH11}\\
GrimesH10 \href{https://doi.org/10.1007/978-3-642-13520-0\_19}{GrimesH10} & \hyperref[auth:a182]{D. Grimes}, \hyperref[auth:a1]{E. Hebrard} & Job Shop Scheduling with Setup Times and Maximal Time-Lags: {A} Simple Constraint Programming Approach & \href{works/GrimesH10.pdf}{Yes} & \cite{GrimesH10} & 2010 & CPAIOR 2010 & 15 & \ref{b:GrimesH10} & \ref{c:GrimesH10}\\
GrimesHM09 \href{https://doi.org/10.1007/978-3-642-04244-7\_33}{GrimesHM09} & \hyperref[auth:a182]{D. Grimes}, \hyperref[auth:a1]{E. Hebrard}, \hyperref[auth:a82]{A. Malapert} & Closing the Open Shop: Contradicting Conventional Wisdom & \href{works/GrimesHM09.pdf}{Yes} & \cite{GrimesHM09} & 2009 & CP 2009 & 9 & \ref{b:GrimesHM09} & \ref{c:GrimesHM09}\\
HebrardTW05 \href{https://doi.org/10.1007/11564751\_117}{HebrardTW05} & \hyperref[auth:a1]{E. Hebrard}, \hyperref[auth:a277]{P. Tyler}, \hyperref[auth:a278]{T. Walsh} & Computing Super-Schedules & \href{works/HebrardTW05.pdf}{Yes} & \cite{HebrardTW05} & 2005 & CP 2005 & 1 & \ref{b:HebrardTW05} & \ref{c:HebrardTW05}\\
\end{longtable}
}

\subsection{Works by Nicolas Beldiceanu}
\label{sec:a128}
{\scriptsize
\begin{longtable}{>{\raggedright\arraybackslash}p{3cm}>{\raggedright\arraybackslash}p{6cm}>{\raggedright\arraybackslash}p{7cm}rrrp{3cm}rrr}
\rowcolor{white}\caption{Works from bibtex (Total 13)}\\ \toprule
\rowcolor{white}Key & Authors & Title & LC & Cite & Year & \shortstack{Conference\\/Journal} & Pages & b & c \\ \midrule\endhead
\bottomrule
\endfoot
Madi-WambaLOBM17 \href{https://doi.org/10.1109/ICPADS.2017.00089}{Madi-WambaLOBM17} & \hyperref[auth:a323]{G. Madi{-}Wamba}, \hyperref[auth:a723]{Y. Li}, \hyperref[auth:a724]{A. Orgerie}, \hyperref[auth:a128]{N. Beldiceanu}, \hyperref[auth:a725]{J. Menaud} & Green Energy Aware Scheduling Problem in Virtualized Datacenters & \href{works/Madi-WambaLOBM17.pdf}{Yes} & \cite{Madi-WambaLOBM17} & 2017 & ICPADS 2017 & 8 & \ref{b:Madi-WambaLOBM17} & \ref{c:Madi-WambaLOBM17}\\
Madi-WambaB16 \href{https://doi.org/10.1007/978-3-319-33954-2\_18}{Madi-WambaB16} & \hyperref[auth:a323]{G. Madi{-}Wamba}, \hyperref[auth:a128]{N. Beldiceanu} & The TaskIntersection Constraint & \href{works/Madi-WambaB16.pdf}{Yes} & \cite{Madi-WambaB16} & 2016 & CPAIOR 2016 & 16 & \ref{b:Madi-WambaB16} & \ref{c:Madi-WambaB16}\\
LetortCB15 \href{https://doi.org/10.1007/s10601-014-9172-8}{LetortCB15} & \hyperref[auth:a127]{A. Letort}, \hyperref[auth:a91]{M. Carlsson}, \hyperref[auth:a128]{N. Beldiceanu} & Synchronized sweep algorithms for scalable scheduling constraints & \href{works/LetortCB15.pdf}{Yes} & \cite{LetortCB15} & 2015 & Constraints An Int. J. & 52 & \ref{b:LetortCB15} & \ref{c:LetortCB15}\\
LetortCB13 \href{https://doi.org/10.1007/978-3-642-38171-3\_10}{LetortCB13} & \hyperref[auth:a127]{A. Letort}, \hyperref[auth:a91]{M. Carlsson}, \hyperref[auth:a128]{N. Beldiceanu} & A Synchronized Sweep Algorithm for the \emph{k-dimensional cumulative} Constraint & \href{works/LetortCB13.pdf}{Yes} & \cite{LetortCB13} & 2013 & CPAIOR 2013 & 16 & \ref{b:LetortCB13} & \ref{c:LetortCB13}\\
LetortBC12 \href{https://doi.org/10.1007/978-3-642-33558-7\_33}{LetortBC12} & \hyperref[auth:a127]{A. Letort}, \hyperref[auth:a128]{N. Beldiceanu}, \hyperref[auth:a91]{M. Carlsson} & A Scalable Sweep Algorithm for the cumulative Constraint & \href{works/LetortBC12.pdf}{Yes} & \cite{LetortBC12} & 2012 & CP 2012 & 16 & \ref{b:LetortBC12} & \ref{c:LetortBC12}\\
BeldiceanuCDP11 \href{https://doi.org/10.1007/s10479-010-0731-0}{BeldiceanuCDP11} & \hyperref[auth:a128]{N. Beldiceanu}, \hyperref[auth:a91]{M. Carlsson}, \hyperref[auth:a245]{S. Demassey}, \hyperref[auth:a362]{E. Poder} & New filtering for the \emph{cumulative} constraint in the context of non-overlapping rectangles & \href{works/BeldiceanuCDP11.pdf}{Yes} & \cite{BeldiceanuCDP11} & 2011 & Ann. Oper. Res. & 24 & \ref{b:BeldiceanuCDP11} & \ref{c:BeldiceanuCDP11}\\
ClercqPBJ11 \href{https://doi.org/10.1007/978-3-642-23786-7\_20}{ClercqPBJ11} & \hyperref[auth:a248]{Alexis De Clercq}, \hyperref[auth:a226]{T. Petit}, \hyperref[auth:a128]{N. Beldiceanu}, \hyperref[auth:a249]{N. Jussien} & Filtering Algorithms for Discrete Cumulative Problems with Overloads of Resource & \href{works/ClercqPBJ11.pdf}{Yes} & \cite{ClercqPBJ11} & 2011 & CP 2011 & 16 & \ref{b:ClercqPBJ11} & \ref{c:ClercqPBJ11}\\
BeldiceanuCP08 \href{https://doi.org/10.1007/978-3-540-68155-7\_5}{BeldiceanuCP08} & \hyperref[auth:a128]{N. Beldiceanu}, \hyperref[auth:a91]{M. Carlsson}, \hyperref[auth:a362]{E. Poder} & New Filtering for the cumulative Constraint in the Context of Non-Overlapping Rectangles & \href{works/BeldiceanuCP08.pdf}{Yes} & \cite{BeldiceanuCP08} & 2008 & CPAIOR 2008 & 15 & \ref{b:BeldiceanuCP08} & \ref{c:BeldiceanuCP08}\\
PoderB08 \href{http://www.aaai.org/Library/ICAPS/2008/icaps08-033.php}{PoderB08} & \hyperref[auth:a362]{E. Poder}, \hyperref[auth:a128]{N. Beldiceanu} & Filtering for a Continuous Multi-Resources cumulative Constraint with Resource Consumption and Production & \href{works/PoderB08.pdf}{Yes} & \cite{PoderB08} & 2008 & ICAPS 2008 & 8 & \ref{b:PoderB08} & \ref{c:PoderB08}\\
BeldiceanuP07 \href{https://doi.org/10.1007/978-3-540-72397-4\_16}{BeldiceanuP07} & \hyperref[auth:a128]{N. Beldiceanu}, \hyperref[auth:a362]{E. Poder} & A Continuous Multi-resources \emph{cumulative} Constraint with Positive-Negative Resource Consumption-Production & \href{works/BeldiceanuP07.pdf}{Yes} & \cite{BeldiceanuP07} & 2007 & CPAIOR 2007 & 15 & \ref{b:BeldiceanuP07} & \ref{c:BeldiceanuP07}\\
PoderBS04 \href{https://doi.org/10.1016/S0377-2217(02)00756-7}{PoderBS04} & \hyperref[auth:a362]{E. Poder}, \hyperref[auth:a128]{N. Beldiceanu}, \hyperref[auth:a722]{E. Sanlaville} & Computing a lower approximation of the compulsory part of a task with varying duration and varying resource consumption & \href{works/PoderBS04.pdf}{Yes} & \cite{PoderBS04} & 2004 & Eur. J. Oper. Res. & 16 & \ref{b:PoderBS04} & \ref{c:PoderBS04}\\
BeldiceanuC02 \href{https://doi.org/10.1007/3-540-46135-3\_5}{BeldiceanuC02} & \hyperref[auth:a128]{N. Beldiceanu}, \hyperref[auth:a91]{M. Carlsson} & A New Multi-resource cumulatives Constraint with Negative Heights & \href{works/BeldiceanuC02.pdf}{Yes} & \cite{BeldiceanuC02} & 2002 & CP 2002 & 17 & \ref{b:BeldiceanuC02} & \ref{c:BeldiceanuC02}\\
AggounB93 \href{https://www.sciencedirect.com/science/article/pii/089571779390068A}{AggounB93} & \hyperref[auth:a734]{A. Aggoun}, \hyperref[auth:a128]{N. Beldiceanu} & Extending {CHIP} in order to solve complex scheduling and placement problems & \href{works/AggounB93.pdf}{Yes} & \cite{AggounB93} & 1993 & Mathematical and Computer Modelling & 17 & \ref{b:AggounB93} & \ref{c:AggounB93}\\
\end{longtable}
}

\subsection{Works by Christian Artigues}
\label{sec:a6}
{\scriptsize
\begin{longtable}{>{\raggedright\arraybackslash}p{3cm}>{\raggedright\arraybackslash}p{6cm}>{\raggedright\arraybackslash}p{7cm}rrrp{3cm}rrr}
\rowcolor{white}\caption{Works from bibtex (Total 12)}\\ \toprule
\rowcolor{white}Key & Authors & Title & LC & Cite & Year & \shortstack{Conference\\/Journal} & Pages & b & c \\ \midrule\endhead
\bottomrule
\endfoot
PovedaAA23 \href{https://doi.org/10.4230/LIPIcs.CP.2023.31}{PovedaAA23} & \hyperref[auth:a4]{G. Pov{\'{e}}da}, \hyperref[auth:a5]{N. {\'{A}}lvarez}, \hyperref[auth:a6]{C. Artigues} & Partially Preemptive Multi Skill/Mode Resource-Constrained Project Scheduling with Generalized Precedence Relations and Calendars & \href{works/PovedaAA23.pdf}{Yes} & \cite{PovedaAA23} & 2023 & CP 2023 & 21 & \ref{b:PovedaAA23} & \ref{c:PovedaAA23}\\
HebrardALLCMR22 \href{https://doi.org/10.24963/ijcai.2022/643}{HebrardALLCMR22} & \hyperref[auth:a1]{E. Hebrard}, \hyperref[auth:a6]{C. Artigues}, \hyperref[auth:a3]{P. Lopez}, \hyperref[auth:a796]{A. Lusson}, \hyperref[auth:a797]{Steve A. Chien}, \hyperref[auth:a798]{A. Maillard}, \hyperref[auth:a799]{Gregg R. Rabideau} & An Efficient Approach to Data Transfer Scheduling for Long Range Space Exploration & \href{works/HebrardALLCMR22.pdf}{Yes} & \cite{HebrardALLCMR22} & 2022 & IJCAI 2022 & 7 & \ref{b:HebrardALLCMR22} & \ref{c:HebrardALLCMR22}\\
PohlAK22 \href{https://doi.org/10.1016/j.ejor.2021.08.028}{PohlAK22} & \hyperref[auth:a444]{M. Pohl}, \hyperref[auth:a6]{C. Artigues}, \hyperref[auth:a445]{R. Kolisch} & Solving the time-discrete winter runway scheduling problem: {A} column generation and constraint programming approach & \href{works/PohlAK22.pdf}{Yes} & \cite{PohlAK22} & 2022 & Eur. J. Oper. Res. & 16 & \ref{b:PohlAK22} & \ref{c:PohlAK22}\\
ArtiguesHQT21 \href{https://doi.org/10.5220/0010190101290136}{ArtiguesHQT21} & \hyperref[auth:a6]{C. Artigues}, \hyperref[auth:a1]{E. Hebrard}, \hyperref[auth:a800]{A. Quilliot}, \hyperref[auth:a801]{H. Toussaint} & Multi-Mode {RCPSP} with Safety Margin Maximization: Models and Algorithms & No & \cite{ArtiguesHQT21} & 2021 & ICORES 2021 & 8 & No & \ref{c:ArtiguesHQT21}\\
Polo-MejiaALB20 \href{https://doi.org/10.1080/00207543.2019.1693654}{Polo-MejiaALB20} & \hyperref[auth:a522]{O. Polo{-}Mej{\'{\i}}a}, \hyperref[auth:a6]{C. Artigues}, \hyperref[auth:a3]{P. Lopez}, \hyperref[auth:a523]{V. Basini} & Mixed-integer/linear and constraint programming approaches for activity scheduling in a nuclear research facility & \href{works/Polo-MejiaALB20.pdf}{Yes} & \cite{Polo-MejiaALB20} & 2020 & Int. J. Prod. Res. & 18 & \ref{b:Polo-MejiaALB20} & \ref{c:Polo-MejiaALB20}\\
NattafAL17 \href{https://doi.org/10.1007/s10601-017-9271-4}{NattafAL17} & \hyperref[auth:a81]{M. Nattaf}, \hyperref[auth:a6]{C. Artigues}, \hyperref[auth:a3]{P. Lopez} & Cumulative scheduling with variable task profiles and concave piecewise linear processing rate functions & \href{works/NattafAL17.pdf}{Yes} & \cite{NattafAL17} & 2017 & Constraints An Int. J. & 18 & \ref{b:NattafAL17} & \ref{c:NattafAL17}\\
NattafAL15 \href{https://doi.org/10.1007/s10601-015-9192-z}{NattafAL15} & \hyperref[auth:a81]{M. Nattaf}, \hyperref[auth:a6]{C. Artigues}, \hyperref[auth:a3]{P. Lopez} & A hybrid exact method for a scheduling problem with a continuous resource and energy constraints & \href{works/NattafAL15.pdf}{Yes} & \cite{NattafAL15} & 2015 & Constraints An Int. J. & 21 & \ref{b:NattafAL15} & \ref{c:NattafAL15}\\
SialaAH15 \href{https://doi.org/10.1007/978-3-319-23219-5\_28}{SialaAH15} & \hyperref[auth:a129]{M. Siala}, \hyperref[auth:a6]{C. Artigues}, \hyperref[auth:a1]{E. Hebrard} & Two Clause Learning Approaches for Disjunctive Scheduling & \href{works/SialaAH15.pdf}{Yes} & \cite{SialaAH15} & 2015 & CP 2015 & 10 & \ref{b:SialaAH15} & \ref{c:SialaAH15}\\
SimoninAHL15 \href{https://doi.org/10.1007/s10601-014-9169-3}{SimoninAHL15} & \hyperref[auth:a126]{G. Simonin}, \hyperref[auth:a6]{C. Artigues}, \hyperref[auth:a1]{E. Hebrard}, \hyperref[auth:a3]{P. Lopez} & Scheduling scientific experiments for comet exploration & \href{works/SimoninAHL15.pdf}{Yes} & \cite{SimoninAHL15} & 2015 & Constraints An Int. J. & 23 & \ref{b:SimoninAHL15} & \ref{c:SimoninAHL15}\\
SimoninAHL12 \href{https://doi.org/10.1007/978-3-642-33558-7\_5}{SimoninAHL12} & \hyperref[auth:a126]{G. Simonin}, \hyperref[auth:a6]{C. Artigues}, \hyperref[auth:a1]{E. Hebrard}, \hyperref[auth:a3]{P. Lopez} & Scheduling Scientific Experiments on the Rosetta/Philae Mission & \href{works/SimoninAHL12.pdf}{Yes} & \cite{SimoninAHL12} & 2012 & CP 2012 & 15 & \ref{b:SimoninAHL12} & \ref{c:SimoninAHL12}\\
ArtiguesBF04 \href{https://doi.org/10.1007/978-3-540-24664-0\_3}{ArtiguesBF04} & \hyperref[auth:a6]{C. Artigues}, \hyperref[auth:a387]{S. Belmokhtar}, \hyperref[auth:a360]{D. Feillet} & A New Exact Solution Algorithm for the Job Shop Problem with Sequence-Dependent Setup Times & \href{works/ArtiguesBF04.pdf}{Yes} & \cite{ArtiguesBF04} & 2004 & CPAIOR 2004 & 13 & \ref{b:ArtiguesBF04} & \ref{c:ArtiguesBF04}\\
ArtiguesR00 \href{https://doi.org/10.1016/S0377-2217(99)00496-8}{ArtiguesR00} & \hyperref[auth:a6]{C. Artigues}, \hyperref[auth:a721]{F. Roubellat} & A polynomial activity insertion algorithm in a multi-resource schedule with cumulative constraints and multiple modes & \href{works/ArtiguesR00.pdf}{Yes} & \cite{ArtiguesR00} & 2000 & Eur. J. Oper. Res. & 20 & \ref{b:ArtiguesR00} & \ref{c:ArtiguesR00}\\
\end{longtable}
}

\subsection{Works by Pierre Lopez}
\label{sec:a3}
{\scriptsize
\begin{longtable}{>{\raggedright\arraybackslash}p{3cm}>{\raggedright\arraybackslash}p{6cm}>{\raggedright\arraybackslash}p{7cm}rrrp{3cm}rrr}
\rowcolor{white}\caption{Works from bibtex (Total 12)}\\ \toprule
\rowcolor{white}Key & Authors & Title & LC & Cite & Year & \shortstack{Conference\\/Journal} & Pages & b & c \\ \midrule\endhead
\bottomrule
\endfoot
JuvinHHL23 \href{https://doi.org/10.4230/LIPIcs.CP.2023.19}{JuvinHHL23} & \hyperref[auth:a0]{C. Juvin}, \hyperref[auth:a1]{E. Hebrard}, \hyperref[auth:a2]{L. Houssin}, \hyperref[auth:a3]{P. Lopez} & An Efficient Constraint Programming Approach to Preemptive Job Shop Scheduling & \href{works/JuvinHHL23.pdf}{Yes} & \cite{JuvinHHL23} & 2023 & CP 2023 & 16 & \ref{b:JuvinHHL23} & \ref{c:JuvinHHL23}\\
JuvinHL23 \href{https://doi.org/10.1007/978-3-031-33271-5\_23}{JuvinHL23} & \hyperref[auth:a0]{C. Juvin}, \hyperref[auth:a2]{L. Houssin}, \hyperref[auth:a3]{P. Lopez} & Constraint Programming for the Robust Two-Machine Flow-Shop Scheduling Problem with Budgeted Uncertainty & \href{works/JuvinHL23.pdf}{Yes} & \cite{JuvinHL23} & 2023 & CPAIOR 2023 & 16 & \ref{b:JuvinHL23} & \ref{c:JuvinHL23}\\
HebrardALLCMR22 \href{https://doi.org/10.24963/ijcai.2022/643}{HebrardALLCMR22} & \hyperref[auth:a1]{E. Hebrard}, \hyperref[auth:a6]{C. Artigues}, \hyperref[auth:a3]{P. Lopez}, \hyperref[auth:a796]{A. Lusson}, \hyperref[auth:a797]{Steve A. Chien}, \hyperref[auth:a798]{A. Maillard}, \hyperref[auth:a799]{Gregg R. Rabideau} & An Efficient Approach to Data Transfer Scheduling for Long Range Space Exploration & \href{works/HebrardALLCMR22.pdf}{Yes} & \cite{HebrardALLCMR22} & 2022 & IJCAI 2022 & 7 & \ref{b:HebrardALLCMR22} & \ref{c:HebrardALLCMR22}\\
Polo-MejiaALB20 \href{https://doi.org/10.1080/00207543.2019.1693654}{Polo-MejiaALB20} & \hyperref[auth:a522]{O. Polo{-}Mej{\'{\i}}a}, \hyperref[auth:a6]{C. Artigues}, \hyperref[auth:a3]{P. Lopez}, \hyperref[auth:a523]{V. Basini} & Mixed-integer/linear and constraint programming approaches for activity scheduling in a nuclear research facility & \href{works/Polo-MejiaALB20.pdf}{Yes} & \cite{Polo-MejiaALB20} & 2020 & Int. J. Prod. Res. & 18 & \ref{b:Polo-MejiaALB20} & \ref{c:Polo-MejiaALB20}\\
NattafAL17 \href{https://doi.org/10.1007/s10601-017-9271-4}{NattafAL17} & \hyperref[auth:a81]{M. Nattaf}, \hyperref[auth:a6]{C. Artigues}, \hyperref[auth:a3]{P. Lopez} & Cumulative scheduling with variable task profiles and concave piecewise linear processing rate functions & \href{works/NattafAL17.pdf}{Yes} & \cite{NattafAL17} & 2017 & Constraints An Int. J. & 18 & \ref{b:NattafAL17} & \ref{c:NattafAL17}\\
NattafAL15 \href{https://doi.org/10.1007/s10601-015-9192-z}{NattafAL15} & \hyperref[auth:a81]{M. Nattaf}, \hyperref[auth:a6]{C. Artigues}, \hyperref[auth:a3]{P. Lopez} & A hybrid exact method for a scheduling problem with a continuous resource and energy constraints & \href{works/NattafAL15.pdf}{Yes} & \cite{NattafAL15} & 2015 & Constraints An Int. J. & 21 & \ref{b:NattafAL15} & \ref{c:NattafAL15}\\
SimoninAHL15 \href{https://doi.org/10.1007/s10601-014-9169-3}{SimoninAHL15} & \hyperref[auth:a126]{G. Simonin}, \hyperref[auth:a6]{C. Artigues}, \hyperref[auth:a1]{E. Hebrard}, \hyperref[auth:a3]{P. Lopez} & Scheduling scientific experiments for comet exploration & \href{works/SimoninAHL15.pdf}{Yes} & \cite{SimoninAHL15} & 2015 & Constraints An Int. J. & 23 & \ref{b:SimoninAHL15} & \ref{c:SimoninAHL15}\\
BillautHL12 \href{https://doi.org/10.1007/978-3-642-29828-8\_5}{BillautHL12} & \hyperref[auth:a342]{J. Billaut}, \hyperref[auth:a1]{E. Hebrard}, \hyperref[auth:a3]{P. Lopez} & Complete Characterization of Near-Optimal Sequences for the Two-Machine Flow Shop Scheduling Problem & \href{works/BillautHL12.pdf}{Yes} & \cite{BillautHL12} & 2012 & CPAIOR 2012 & 15 & \ref{b:BillautHL12} & \ref{c:BillautHL12}\\
SimoninAHL12 \href{https://doi.org/10.1007/978-3-642-33558-7\_5}{SimoninAHL12} & \hyperref[auth:a126]{G. Simonin}, \hyperref[auth:a6]{C. Artigues}, \hyperref[auth:a1]{E. Hebrard}, \hyperref[auth:a3]{P. Lopez} & Scheduling Scientific Experiments on the Rosetta/Philae Mission & \href{works/SimoninAHL12.pdf}{Yes} & \cite{SimoninAHL12} & 2012 & CP 2012 & 15 & \ref{b:SimoninAHL12} & \ref{c:SimoninAHL12}\\
LahimerLH11 \href{https://doi.org/10.1007/978-3-642-21311-3\_12}{LahimerLH11} & \hyperref[auth:a353]{A. Lahimer}, \hyperref[auth:a3]{P. Lopez}, \hyperref[auth:a354]{M. Haouari} & Climbing Depth-Bounded Adjacent Discrepancy Search for Solving Hybrid Flow Shop Scheduling Problems with Multiprocessor Tasks & \href{works/LahimerLH11.pdf}{Yes} & \cite{LahimerLH11} & 2011 & CPAIOR 2011 & 14 & \ref{b:LahimerLH11} & \ref{c:LahimerLH11}\\
TrojetHL11 \href{https://doi.org/10.1016/j.cie.2010.08.014}{TrojetHL11} & \hyperref[auth:a715]{M. Trojet}, \hyperref[auth:a716]{F. H'Mida}, \hyperref[auth:a3]{P. Lopez} & Project scheduling under resource constraints: Application of the cumulative global constraint in a decision support framework & \href{works/TrojetHL11.pdf}{Yes} & \cite{TrojetHL11} & 2011 & Comput. Ind. Eng. & 7 & \ref{b:TrojetHL11} & \ref{c:TrojetHL11}\\
LopezAKYG00 \href{https://doi.org/10.1016/S0947-3580(00)71114-9}{LopezAKYG00} & \hyperref[auth:a3]{P. Lopez}, \hyperref[auth:a693]{H. Alla}, \hyperref[auth:a690]{O. Korbaa}, \hyperref[auth:a691]{P. Yim}, \hyperref[auth:a692]{J. Gentina} & Discussion on: 'Solving Transient Scheduling Problems with Constraint Programming' by O. Korbaa, P. Yim, and {J.-C.} Gentina & \href{works/LopezAKYG00.pdf}{Yes} & \cite{LopezAKYG00} & 2000 & Eur. J. Control & 4 & \ref{b:LopezAKYG00} & \ref{c:LopezAKYG00}\\
\end{longtable}
}

\subsection{Works by Roman Bart{\'{a}}k}
\label{sec:a152}
{\scriptsize
\begin{longtable}{>{\raggedright\arraybackslash}p{3cm}>{\raggedright\arraybackslash}p{6cm}>{\raggedright\arraybackslash}p{7cm}rrrp{3cm}rrr}
\rowcolor{white}\caption{Works from bibtex (Total 11)}\\ \toprule
\rowcolor{white}Key & Authors & Title & LC & Cite & Year & \shortstack{Conference\\/Journal} & Pages & b & c \\ \midrule\endhead
\bottomrule
\endfoot
SvancaraB22 \href{https://doi.org/10.5220/0010869700003116}{SvancaraB22} & \hyperref[auth:a787]{J. Svancara}, \hyperref[auth:a152]{R. Bart{\'{a}}k} & Tackling Train Routing via Multi-agent Pathfinding and Constraint-based Scheduling & \href{works/SvancaraB22.pdf}{Yes} & \cite{SvancaraB22} & 2022 & ICAART 2022 & 8 & \ref{b:SvancaraB22} & \ref{c:SvancaraB22}\\
JelinekB16 \href{https://doi.org/10.1007/978-3-319-28228-2\_1}{JelinekB16} & \hyperref[auth:a788]{J. Jel{\'{\i}}nek}, \hyperref[auth:a152]{R. Bart{\'{a}}k} & Using Constraint Logic Programming to Schedule Solar Array Operations on the International Space Station & \href{works/JelinekB16.pdf}{Yes} & \cite{JelinekB16} & 2016 & PADL 2016 & 10 & \ref{b:JelinekB16} & \ref{c:JelinekB16}\\
BartakV15 \href{}{BartakV15} & \hyperref[auth:a152]{R. Bart{\'{a}}k}, \hyperref[auth:a313]{M. Vlk} & Reactive Recovery from Machine Breakdown in Production Scheduling with Temporal Distance and Resource Constraints & \href{works/BartakV15.pdf}{Yes} & \cite{BartakV15} & 2015 & ICAART 2015 & 12 & \ref{b:BartakV15} & \ref{c:BartakV15}\\
Bartak14 \href{}{Bartak14} & \hyperref[auth:a152]{R. Bart{\'{a}}k} & Planning and Scheduling & No & \cite{Bartak14} & 2014 & n/a & null & No & \ref{c:Bartak14}\\
BartakS11 \href{https://doi.org/10.1007/s10601-011-9109-4}{BartakS11} & \hyperref[auth:a152]{R. Bart{\'{a}}k}, \hyperref[auth:a153]{Miguel A. Salido} & Constraint satisfaction for planning and scheduling problems & \href{works/BartakS11.pdf}{Yes} & \cite{BartakS11} & 2011 & Constraints An Int. J. & 5 & \ref{b:BartakS11} & \ref{c:BartakS11}\\
BartakCS10 \href{https://doi.org/10.1007/s10479-008-0492-1}{BartakCS10} & \hyperref[auth:a152]{R. Bart{\'{a}}k}, \hyperref[auth:a162]{O. Cepek}, \hyperref[auth:a789]{P. Surynek} & Discovering implied constraints in precedence graphs with alternatives & \href{works/BartakCS10.pdf}{Yes} & \cite{BartakCS10} & 2010 & Ann. Oper. Res. & 31 & \ref{b:BartakCS10} & \ref{c:BartakCS10}\\
BartakSR10 \href{https://doi.org/10.1017/S0269888910000202}{BartakSR10} & \hyperref[auth:a152]{R. Bart{\'{a}}k}, \hyperref[auth:a153]{Miguel A. Salido}, \hyperref[auth:a318]{F. Rossi} & New trends in constraint satisfaction, planning, and scheduling: a survey & \href{works/BartakSR10.pdf}{Yes} & \cite{BartakSR10} & 2010 & Knowl. Eng. Rev. & 31 & \ref{b:BartakSR10} & \ref{c:BartakSR10}\\
VilimBC05 \href{https://doi.org/10.1007/s10601-005-2814-0}{VilimBC05} & \hyperref[auth:a121]{P. Vil{\'{\i}}m}, \hyperref[auth:a152]{R. Bart{\'{a}}k}, \hyperref[auth:a162]{O. Cepek} & Extension of \emph{O}(\emph{n} log \emph{n}) Filtering Algorithms for the Unary Resource Constraint to Optional Activities & \href{works/VilimBC05.pdf}{Yes} & \cite{VilimBC05} & 2005 & Constraints An Int. J. & 23 & \ref{b:VilimBC05} & \ref{c:VilimBC05}\\
VilimBC04 \href{https://doi.org/10.1007/978-3-540-30201-8\_8}{VilimBC04} & \hyperref[auth:a121]{P. Vil{\'{\i}}m}, \hyperref[auth:a152]{R. Bart{\'{a}}k}, \hyperref[auth:a162]{O. Cepek} & Unary Resource Constraint with Optional Activities & \href{works/VilimBC04.pdf}{Yes} & \cite{VilimBC04} & 2004 & CP 2004 & 15 & \ref{b:VilimBC04} & \ref{c:VilimBC04}\\
Bartak02 \href{https://doi.org/10.1007/3-540-46135-3\_39}{Bartak02} & \hyperref[auth:a152]{R. Bart{\'{a}}k} & Visopt ShopFloor: On the Edge of Planning and Scheduling & \href{works/Bartak02.pdf}{Yes} & \cite{Bartak02} & 2002 & CP 2002 & 16 & \ref{b:Bartak02} & \ref{c:Bartak02}\\
Bartak02a \href{https://doi.org/10.1007/3-540-36607-5\_14}{Bartak02a} & \hyperref[auth:a152]{R. Bart{\'{a}}k} & Visopt ShopFloor: Going Beyond Traditional Scheduling & \href{works/Bartak02a.pdf}{Yes} & \cite{Bartak02a} & 2002 & ERCIM/CologNet 2002 & 15 & \ref{b:Bartak02a} & \ref{c:Bartak02a}\\
\end{longtable}
}

\subsection{Works by Petr Vil{\'{\i}}m}
\label{sec:a121}
{\scriptsize
\begin{longtable}{>{\raggedright\arraybackslash}p{3cm}>{\raggedright\arraybackslash}p{6cm}>{\raggedright\arraybackslash}p{7cm}rrrp{3cm}rrr}
\rowcolor{white}\caption{Works from bibtex (Total 11)}\\ \toprule
\rowcolor{white}Key & Authors & Title & LC & Cite & Year & \shortstack{Conference\\/Journal} & Pages & b & c \\ \midrule\endhead
\bottomrule
\endfoot
LaborieRSV18 \href{https://doi.org/10.1007/s10601-018-9281-x}{LaborieRSV18} & \hyperref[auth:a118]{P. Laborie}, \hyperref[auth:a119]{J. Rogerie}, \hyperref[auth:a120]{P. Shaw}, \hyperref[auth:a121]{P. Vil{\'{\i}}m} & {IBM} {ILOG} {CP} optimizer for scheduling - 20+ years of scheduling with constraints at {IBM/ILOG} & \href{works/LaborieRSV18.pdf}{Yes} & \cite{LaborieRSV18} & 2018 & Constraints An Int. J. & 41 & \ref{b:LaborieRSV18} & \ref{c:LaborieRSV18}\\
VilimLS15 \href{https://doi.org/10.1007/978-3-319-18008-3\_30}{VilimLS15} & \hyperref[auth:a121]{P. Vil{\'{\i}}m}, \hyperref[auth:a118]{P. Laborie}, \hyperref[auth:a120]{P. Shaw} & Failure-Directed Search for Constraint-Based Scheduling & \href{works/VilimLS15.pdf}{Yes} & \cite{VilimLS15} & 2015 & CPAIOR 2015 & 17 & \ref{b:VilimLS15} & \ref{c:VilimLS15}\\
Vilim11 \href{https://doi.org/10.1007/978-3-642-21311-3\_22}{Vilim11} & \hyperref[auth:a121]{P. Vil{\'{\i}}m} & Timetable Edge Finding Filtering Algorithm for Discrete Cumulative Resources & \href{works/Vilim11.pdf}{Yes} & \cite{Vilim11} & 2011 & CPAIOR 2011 & 16 & \ref{b:Vilim11} & \ref{c:Vilim11}\\
Vilim09 \href{https://doi.org/10.1007/978-3-642-04244-7\_62}{Vilim09} & \hyperref[auth:a121]{P. Vil{\'{\i}}m} & Edge Finding Filtering Algorithm for Discrete Cumulative Resources in \emph{O}(\emph{kn} log \emph{n})\{{\textbackslash}mathcal O\}(kn \{{\textbackslash}rm log\} n) & \href{works/Vilim09.pdf}{Yes} & \cite{Vilim09} & 2009 & CP 2009 & 15 & \ref{b:Vilim09} & \ref{c:Vilim09}\\
Vilim09a \href{https://doi.org/10.1007/978-3-642-01929-6\_22}{Vilim09a} & \hyperref[auth:a121]{P. Vil{\'{\i}}m} & Max Energy Filtering Algorithm for Discrete Cumulative Resources & \href{works/Vilim09a.pdf}{Yes} & \cite{Vilim09a} & 2009 & CPAIOR 2009 & 15 & \ref{b:Vilim09a} & \ref{c:Vilim09a}\\
Vilim05 \href{https://doi.org/10.1007/11493853\_29}{Vilim05} & \hyperref[auth:a121]{P. Vil{\'{\i}}m} & Computing Explanations for the Unary Resource Constraint & \href{works/Vilim05.pdf}{Yes} & \cite{Vilim05} & 2005 & CPAIOR 2005 & 14 & \ref{b:Vilim05} & \ref{c:Vilim05}\\
VilimBC05 \href{https://doi.org/10.1007/s10601-005-2814-0}{VilimBC05} & \hyperref[auth:a121]{P. Vil{\'{\i}}m}, \hyperref[auth:a152]{R. Bart{\'{a}}k}, \hyperref[auth:a162]{O. Cepek} & Extension of \emph{O}(\emph{n} log \emph{n}) Filtering Algorithms for the Unary Resource Constraint to Optional Activities & \href{works/VilimBC05.pdf}{Yes} & \cite{VilimBC05} & 2005 & Constraints An Int. J. & 23 & \ref{b:VilimBC05} & \ref{c:VilimBC05}\\
Vilim04 \href{https://doi.org/10.1007/978-3-540-24664-0\_23}{Vilim04} & \hyperref[auth:a121]{P. Vil{\'{\i}}m} & O(n log n) Filtering Algorithms for Unary Resource Constraint & \href{works/Vilim04.pdf}{Yes} & \cite{Vilim04} & 2004 & CPAIOR 2004 & 13 & \ref{b:Vilim04} & \ref{c:Vilim04}\\
VilimBC04 \href{https://doi.org/10.1007/978-3-540-30201-8\_8}{VilimBC04} & \hyperref[auth:a121]{P. Vil{\'{\i}}m}, \hyperref[auth:a152]{R. Bart{\'{a}}k}, \hyperref[auth:a162]{O. Cepek} & Unary Resource Constraint with Optional Activities & \href{works/VilimBC04.pdf}{Yes} & \cite{VilimBC04} & 2004 & CP 2004 & 15 & \ref{b:VilimBC04} & \ref{c:VilimBC04}\\
Vilim03 \href{https://doi.org/10.1007/978-3-540-45193-8\_124}{Vilim03} & \hyperref[auth:a121]{P. Vil{\'{\i}}m} & Computing Explanations for Global Scheduling Constraints & \href{works/Vilim03.pdf}{Yes} & \cite{Vilim03} & 2003 & CP 2003 & 1 & \ref{b:Vilim03} & \ref{c:Vilim03}\\
Vilim02 \href{https://doi.org/10.1007/3-540-46135-3\_62}{Vilim02} & \hyperref[auth:a121]{P. Vil{\'{\i}}m} & Batch Processing with Sequence Dependent Setup Times & \href{works/Vilim02.pdf}{Yes} & \cite{Vilim02} & 2002 & CP 2002 & 1 & \ref{b:Vilim02} & \ref{c:Vilim02}\\
\end{longtable}
}

\subsection{Works by Luca Benini}
\label{sec:a247}
{\scriptsize
\begin{longtable}{>{\raggedright\arraybackslash}p{3cm}>{\raggedright\arraybackslash}p{6cm}>{\raggedright\arraybackslash}p{7cm}rrrp{3cm}rrr}
\rowcolor{white}\caption{Works from bibtex (Total 10)}\\ \toprule
\rowcolor{white}Key & Authors & Title & LC & Cite & Year & \shortstack{Conference\\/Journal} & Pages & b & c \\ \midrule\endhead
\bottomrule
\endfoot
BorghesiBLMB18 \href{https://doi.org/10.1016/j.suscom.2018.05.007}{BorghesiBLMB18} & \hyperref[auth:a231]{A. Borghesi}, \hyperref[auth:a230]{A. Bartolini}, \hyperref[auth:a142]{M. Lombardi}, \hyperref[auth:a143]{M. Milano}, \hyperref[auth:a247]{L. Benini} & Scheduling-based power capping in high performance computing systems & \href{works/BorghesiBLMB18.pdf}{Yes} & \cite{BorghesiBLMB18} & 2018 & Sustain. Comput. Informatics Syst. & 13 & \ref{b:BorghesiBLMB18} & \ref{c:BorghesiBLMB18}\\
BridiBLMB16 \href{https://doi.org/10.1109/TPDS.2016.2516997}{BridiBLMB16} & \hyperref[auth:a232]{T. Bridi}, \hyperref[auth:a230]{A. Bartolini}, \hyperref[auth:a142]{M. Lombardi}, \hyperref[auth:a143]{M. Milano}, \hyperref[auth:a247]{L. Benini} & A Constraint Programming Scheduler for Heterogeneous High-Performance Computing Machines & \href{works/BridiBLMB16.pdf}{Yes} & \cite{BridiBLMB16} & 2016 & {IEEE} Trans. Parallel Distributed Syst. & 14 & \ref{b:BridiBLMB16} & \ref{c:BridiBLMB16}\\
BridiLBBM16 \href{https://doi.org/10.3233/978-1-61499-672-9-1598}{BridiLBBM16} & \hyperref[auth:a232]{T. Bridi}, \hyperref[auth:a142]{M. Lombardi}, \hyperref[auth:a230]{A. Bartolini}, \hyperref[auth:a247]{L. Benini}, \hyperref[auth:a143]{M. Milano} & {DARDIS:} Distributed And Randomized DIspatching and Scheduling & \href{works/BridiLBBM16.pdf}{Yes} & \cite{BridiLBBM16} & 2016 & ECAI 2016 & 2 & \ref{b:BridiLBBM16} & \ref{c:BridiLBBM16}\\
BonfiettiLBM14 \href{https://doi.org/10.1016/j.artint.2013.09.006}{BonfiettiLBM14} & \hyperref[auth:a203]{A. Bonfietti}, \hyperref[auth:a142]{M. Lombardi}, \hyperref[auth:a247]{L. Benini}, \hyperref[auth:a143]{M. Milano} & {CROSS} cyclic resource-constrained scheduling solver & \href{works/BonfiettiLBM14.pdf}{Yes} & \cite{BonfiettiLBM14} & 2014 & Artif. Intell. & 28 & \ref{b:BonfiettiLBM14} & \ref{c:BonfiettiLBM14}\\
BonfiettiLBM12 \href{https://doi.org/10.1007/978-3-642-29828-8\_6}{BonfiettiLBM12} & \hyperref[auth:a203]{A. Bonfietti}, \hyperref[auth:a142]{M. Lombardi}, \hyperref[auth:a247]{L. Benini}, \hyperref[auth:a143]{M. Milano} & Global Cyclic Cumulative Constraint & \href{works/BonfiettiLBM12.pdf}{Yes} & \cite{BonfiettiLBM12} & 2012 & CPAIOR 2012 & 16 & \ref{b:BonfiettiLBM12} & \ref{c:BonfiettiLBM12}\\
BeniniLMR11 \href{https://doi.org/10.1007/s10479-010-0718-x}{BeniniLMR11} & \hyperref[auth:a247]{L. Benini}, \hyperref[auth:a142]{M. Lombardi}, \hyperref[auth:a143]{M. Milano}, \hyperref[auth:a727]{M. Ruggiero} & Optimal resource allocation and scheduling for the {CELL} {BE} platform & \href{works/BeniniLMR11.pdf}{Yes} & \cite{BeniniLMR11} & 2011 & Ann. Oper. Res. & 27 & \ref{b:BeniniLMR11} & \ref{c:BeniniLMR11}\\
BonfiettiLBM11 \href{https://doi.org/10.1007/978-3-642-23786-7\_12}{BonfiettiLBM11} & \hyperref[auth:a203]{A. Bonfietti}, \hyperref[auth:a142]{M. Lombardi}, \hyperref[auth:a247]{L. Benini}, \hyperref[auth:a143]{M. Milano} & A Constraint Based Approach to Cyclic {RCPSP} & \href{works/BonfiettiLBM11.pdf}{Yes} & \cite{BonfiettiLBM11} & 2011 & CP 2011 & 15 & \ref{b:BonfiettiLBM11} & \ref{c:BonfiettiLBM11}\\
LombardiBMB11 \href{https://doi.org/10.1007/978-3-642-21311-3\_14}{LombardiBMB11} & \hyperref[auth:a142]{M. Lombardi}, \hyperref[auth:a203]{A. Bonfietti}, \hyperref[auth:a143]{M. Milano}, \hyperref[auth:a247]{L. Benini} & Precedence Constraint Posting for Cyclic Scheduling Problems & \href{works/LombardiBMB11.pdf}{Yes} & \cite{LombardiBMB11} & 2011 & CPAIOR 2011 & 17 & \ref{b:LombardiBMB11} & \ref{c:LombardiBMB11}\\
RuggieroBBMA09 \href{https://doi.org/10.1109/TCAD.2009.2013536}{RuggieroBBMA09} & \hyperref[auth:a727]{M. Ruggiero}, \hyperref[auth:a379]{D. Bertozzi}, \hyperref[auth:a247]{L. Benini}, \hyperref[auth:a143]{M. Milano}, \hyperref[auth:a728]{A. Andrei} & Reducing the Abstraction and Optimality Gaps in the Allocation and Scheduling for Variable Voltage/Frequency MPSoC Platforms & \href{works/RuggieroBBMA09.pdf}{Yes} & \cite{RuggieroBBMA09} & 2009 & {IEEE} Trans. Comput. Aided Des. Integr. Circuits Syst. & 14 & \ref{b:RuggieroBBMA09} & \ref{c:RuggieroBBMA09}\\
BeniniBGM06 \href{https://doi.org/10.1007/11757375\_6}{BeniniBGM06} & \hyperref[auth:a247]{L. Benini}, \hyperref[auth:a379]{D. Bertozzi}, \hyperref[auth:a380]{A. Guerri}, \hyperref[auth:a143]{M. Milano} & Allocation, Scheduling and Voltage Scaling on Energy Aware MPSoCs & \href{works/BeniniBGM06.pdf}{Yes} & \cite{BeniniBGM06} & 2006 & CPAIOR 2006 & 15 & \ref{b:BeniniBGM06} & \ref{c:BeniniBGM06}\\
\end{longtable}
}

\subsection{Works by Alessio Bonfietti}
\label{sec:a203}
{\scriptsize
\begin{longtable}{>{\raggedright\arraybackslash}p{3cm}>{\raggedright\arraybackslash}p{6cm}>{\raggedright\arraybackslash}p{7cm}rrrp{3cm}rrr}
\rowcolor{white}\caption{Works from bibtex (Total 10)}\\ \toprule
\rowcolor{white}Key & Authors & Title & LC & Cite & Year & \shortstack{Conference\\/Journal} & Pages & b & c \\ \midrule\endhead
\bottomrule
\endfoot
Bonfietti16 \href{https://doi.org/10.3233/IA-160095}{Bonfietti16} & \hyperref[auth:a203]{A. Bonfietti} & A constraint programming scheduling solver for the MPOpt programming environment & \href{works/Bonfietti16.pdf}{Yes} & \cite{Bonfietti16} & 2016 & Intelligenza Artificiale & 13 & \ref{b:Bonfietti16} & \ref{c:Bonfietti16}\\
BonfiettiZLM16 \href{https://doi.org/10.1007/978-3-319-44953-1\_8}{BonfiettiZLM16} & \hyperref[auth:a203]{A. Bonfietti}, \hyperref[auth:a204]{A. Zanarini}, \hyperref[auth:a142]{M. Lombardi}, \hyperref[auth:a143]{M. Milano} & The Multirate Resource Constraint & \href{works/BonfiettiZLM16.pdf}{Yes} & \cite{BonfiettiZLM16} & 2016 & CP 2016 & 17 & \ref{b:BonfiettiZLM16} & \ref{c:BonfiettiZLM16}\\
LombardiBM15 \href{https://doi.org/10.1007/978-3-319-23219-5\_20}{LombardiBM15} & \hyperref[auth:a142]{M. Lombardi}, \hyperref[auth:a203]{A. Bonfietti}, \hyperref[auth:a143]{M. Milano} & Deterministic Estimation of the Expected Makespan of a {POS} Under Duration Uncertainty & \href{works/LombardiBM15.pdf}{Yes} & \cite{LombardiBM15} & 2015 & CP 2015 & 16 & \ref{b:LombardiBM15} & \ref{c:LombardiBM15}\\
BonfiettiLBM14 \href{https://doi.org/10.1016/j.artint.2013.09.006}{BonfiettiLBM14} & \hyperref[auth:a203]{A. Bonfietti}, \hyperref[auth:a142]{M. Lombardi}, \hyperref[auth:a247]{L. Benini}, \hyperref[auth:a143]{M. Milano} & {CROSS} cyclic resource-constrained scheduling solver & \href{works/BonfiettiLBM14.pdf}{Yes} & \cite{BonfiettiLBM14} & 2014 & Artif. Intell. & 28 & \ref{b:BonfiettiLBM14} & \ref{c:BonfiettiLBM14}\\
BonfiettiLM14 \href{https://doi.org/10.1007/978-3-319-07046-9\_15}{BonfiettiLM14} & \hyperref[auth:a203]{A. Bonfietti}, \hyperref[auth:a142]{M. Lombardi}, \hyperref[auth:a143]{M. Milano} & Disregarding Duration Uncertainty in Partial Order Schedules? Yes, We Can! & \href{works/BonfiettiLM14.pdf}{Yes} & \cite{BonfiettiLM14} & 2014 & CPAIOR 2014 & 16 & \ref{b:BonfiettiLM14} & \ref{c:BonfiettiLM14}\\
BonfiettiLM13 \href{http://www.aaai.org/ocs/index.php/ICAPS/ICAPS13/paper/view/6050}{BonfiettiLM13} & \hyperref[auth:a203]{A. Bonfietti}, \hyperref[auth:a142]{M. Lombardi}, \hyperref[auth:a143]{M. Milano} & De-Cycling Cyclic Scheduling Problems & \href{works/BonfiettiLM13.pdf}{Yes} & \cite{BonfiettiLM13} & 2013 & ICAPS 2013 & 5 & \ref{b:BonfiettiLM13} & \ref{c:BonfiettiLM13}\\
BonfiettiLBM12 \href{https://doi.org/10.1007/978-3-642-29828-8\_6}{BonfiettiLBM12} & \hyperref[auth:a203]{A. Bonfietti}, \hyperref[auth:a142]{M. Lombardi}, \hyperref[auth:a247]{L. Benini}, \hyperref[auth:a143]{M. Milano} & Global Cyclic Cumulative Constraint & \href{works/BonfiettiLBM12.pdf}{Yes} & \cite{BonfiettiLBM12} & 2012 & CPAIOR 2012 & 16 & \ref{b:BonfiettiLBM12} & \ref{c:BonfiettiLBM12}\\
BonfiettiM12 \href{https://ceur-ws.org/Vol-926/paper2.pdf}{BonfiettiM12} & \hyperref[auth:a203]{A. Bonfietti}, \hyperref[auth:a143]{M. Milano} & A Constraint-based Approach to Cyclic Resource-Constrained Scheduling Problem & \href{works/BonfiettiM12.pdf}{Yes} & \cite{BonfiettiM12} & 2012 & DC SIAAI 2012 & 3 & \ref{b:BonfiettiM12} & \ref{c:BonfiettiM12}\\
BonfiettiLBM11 \href{https://doi.org/10.1007/978-3-642-23786-7\_12}{BonfiettiLBM11} & \hyperref[auth:a203]{A. Bonfietti}, \hyperref[auth:a142]{M. Lombardi}, \hyperref[auth:a247]{L. Benini}, \hyperref[auth:a143]{M. Milano} & A Constraint Based Approach to Cyclic {RCPSP} & \href{works/BonfiettiLBM11.pdf}{Yes} & \cite{BonfiettiLBM11} & 2011 & CP 2011 & 15 & \ref{b:BonfiettiLBM11} & \ref{c:BonfiettiLBM11}\\
LombardiBMB11 \href{https://doi.org/10.1007/978-3-642-21311-3\_14}{LombardiBMB11} & \hyperref[auth:a142]{M. Lombardi}, \hyperref[auth:a203]{A. Bonfietti}, \hyperref[auth:a143]{M. Milano}, \hyperref[auth:a247]{L. Benini} & Precedence Constraint Posting for Cyclic Scheduling Problems & \href{works/LombardiBMB11.pdf}{Yes} & \cite{LombardiBMB11} & 2011 & CPAIOR 2011 & 17 & \ref{b:LombardiBMB11} & \ref{c:LombardiBMB11}\\
\end{longtable}
}

\subsection{Works by Philippe Laborie}
\label{sec:a118}
{\scriptsize
\begin{longtable}{>{\raggedright\arraybackslash}p{3cm}>{\raggedright\arraybackslash}p{6cm}>{\raggedright\arraybackslash}p{7cm}rrrp{3cm}rrr}
\rowcolor{white}\caption{Works from bibtex (Total 10)}\\ \toprule
\rowcolor{white}Key & Authors & Title & LC & Cite & Year & \shortstack{Conference\\/Journal} & Pages & b & c \\ \midrule\endhead
\bottomrule
\endfoot
LunardiBLRV20 \href{https://doi.org/10.1016/j.cor.2020.105020}{LunardiBLRV20} & \hyperref[auth:a510]{Willian T. Lunardi}, \hyperref[auth:a511]{Ernesto G. Birgin}, \hyperref[auth:a118]{P. Laborie}, \hyperref[auth:a512]{D{\'{e}}bora P. Ronconi}, \hyperref[auth:a513]{H. Voos} & Mixed Integer linear programming and constraint programming models for the online printing shop scheduling problem & \href{works/LunardiBLRV20.pdf}{Yes} & \cite{LunardiBLRV20} & 2020 & Comput. Oper. Res. & 20 & \ref{b:LunardiBLRV20} & \ref{c:LunardiBLRV20}\\
Laborie18a \href{https://doi.org/10.1007/978-3-319-93031-2\_29}{Laborie18a} & \hyperref[auth:a118]{P. Laborie} & An Update on the Comparison of MIP, {CP} and Hybrid Approaches for Mixed Resource Allocation and Scheduling & \href{works/Laborie18a.pdf}{Yes} & \cite{Laborie18a} & 2018 & CPAIOR 2018 & 9 & \ref{b:Laborie18a} & \ref{c:Laborie18a}\\
LaborieRSV18 \href{https://doi.org/10.1007/s10601-018-9281-x}{LaborieRSV18} & \hyperref[auth:a118]{P. Laborie}, \hyperref[auth:a119]{J. Rogerie}, \hyperref[auth:a120]{P. Shaw}, \hyperref[auth:a121]{P. Vil{\'{\i}}m} & {IBM} {ILOG} {CP} optimizer for scheduling - 20+ years of scheduling with constraints at {IBM/ILOG} & \href{works/LaborieRSV18.pdf}{Yes} & \cite{LaborieRSV18} & 2018 & Constraints An Int. J. & 41 & \ref{b:LaborieRSV18} & \ref{c:LaborieRSV18}\\
MelgarejoLS15 \href{https://doi.org/10.1007/978-3-319-18008-3\_1}{MelgarejoLS15} & \hyperref[auth:a324]{P. Aguiar{-}Melgarejo}, \hyperref[auth:a118]{P. Laborie}, \hyperref[auth:a85]{C. Solnon} & A Time-Dependent No-Overlap Constraint: Application to Urban Delivery Problems & \href{works/MelgarejoLS15.pdf}{Yes} & \cite{MelgarejoLS15} & 2015 & CPAIOR 2015 & 17 & \ref{b:MelgarejoLS15} & \ref{c:MelgarejoLS15}\\
VilimLS15 \href{https://doi.org/10.1007/978-3-319-18008-3\_30}{VilimLS15} & \hyperref[auth:a121]{P. Vil{\'{\i}}m}, \hyperref[auth:a118]{P. Laborie}, \hyperref[auth:a120]{P. Shaw} & Failure-Directed Search for Constraint-Based Scheduling & \href{works/VilimLS15.pdf}{Yes} & \cite{VilimLS15} & 2015 & CPAIOR 2015 & 17 & \ref{b:VilimLS15} & \ref{c:VilimLS15}\\
BidotVLB09 \href{https://doi.org/10.1007/s10951-008-0080-x}{BidotVLB09} & \hyperref[auth:a835]{J. Bidot}, \hyperref[auth:a836]{T. Vidal}, \hyperref[auth:a118]{P. Laborie}, \hyperref[auth:a89]{J. Christopher Beck} & A theoretic and practical framework for scheduling in a stochastic environment & \href{works/BidotVLB09.pdf}{Yes} & \cite{BidotVLB09} & 2009 & J. Sched. & 30 & \ref{b:BidotVLB09} & \ref{c:BidotVLB09}\\
Laborie09 \href{https://doi.org/10.1007/978-3-642-01929-6\_12}{Laborie09} & \hyperref[auth:a118]{P. Laborie} & {IBM} {ILOG} {CP} Optimizer for Detailed Scheduling Illustrated on Three Problems & \href{works/Laborie09.pdf}{Yes} & \cite{Laborie09} & 2009 & CPAIOR 2009 & 15 & \ref{b:Laborie09} & \ref{c:Laborie09}\\
BaptisteLPN06 \href{https://doi.org/10.1016/S1574-6526(06)80026-X}{BaptisteLPN06} & \hyperref[auth:a163]{P. Baptiste}, \hyperref[auth:a118]{P. Laborie}, \hyperref[auth:a164]{Claude Le Pape}, \hyperref[auth:a666]{W. Nuijten} & Constraint-Based Scheduling and Planning & No & \cite{BaptisteLPN06} & 2006 & n/a & 39 & No & \ref{c:BaptisteLPN06}\\
GodardLN05 \href{http://www.aaai.org/Library/ICAPS/2005/icaps05-009.php}{GodardLN05} & \hyperref[auth:a782]{D. Godard}, \hyperref[auth:a118]{P. Laborie}, \hyperref[auth:a666]{W. Nuijten} & Randomized Large Neighborhood Search for Cumulative Scheduling & \href{works/GodardLN05.pdf}{Yes} & \cite{GodardLN05} & 2005 & ICAPS 2005 & 9 & \ref{b:GodardLN05} & \ref{c:GodardLN05}\\
FocacciLN00 \href{http://www.aaai.org/Library/AIPS/2000/aips00-010.php}{FocacciLN00} & \hyperref[auth:a784]{F. Focacci}, \hyperref[auth:a118]{P. Laborie}, \hyperref[auth:a666]{W. Nuijten} & Solving Scheduling Problems with Setup Times and Alternative Resources & \href{works/FocacciLN00.pdf}{Yes} & \cite{FocacciLN00} & 2000 & AIPS 2000 & 10 & \ref{b:FocacciLN00} & \ref{c:FocacciLN00}\\
\end{longtable}
}

\subsection{Works by Nysret Musliu}
\label{sec:a45}
{\scriptsize
\begin{longtable}{>{\raggedright\arraybackslash}p{3cm}>{\raggedright\arraybackslash}p{6cm}>{\raggedright\arraybackslash}p{7cm}rrrp{3cm}rrr}
\rowcolor{white}\caption{Works from bibtex (Total 9)}\\ \toprule
\rowcolor{white}Key & Authors & Title & LC & Cite & Year & \shortstack{Conference\\/Journal} & Pages & b & c \\ \midrule\endhead
\bottomrule
\endfoot
LacknerMMWW23 \href{https://doi.org/10.1007/s10601-023-09347-2}{LacknerMMWW23} & \hyperref[auth:a62]{M. Lackner}, \hyperref[auth:a63]{C. Mrkvicka}, \hyperref[auth:a45]{N. Musliu}, \hyperref[auth:a46]{D. Walkiewicz}, \hyperref[auth:a43]{F. Winter} & Exact methods for the Oven Scheduling Problem & \href{works/LacknerMMWW23.pdf}{Yes} & \cite{LacknerMMWW23} & 2023 & Constraints An Int. J. & 42 & \ref{b:LacknerMMWW23} & \ref{c:LacknerMMWW23}\\
WinterMMW22 \href{https://doi.org/10.4230/LIPIcs.CP.2022.41}{WinterMMW22} & \hyperref[auth:a43]{F. Winter}, \hyperref[auth:a44]{S. Meiswinkel}, \hyperref[auth:a45]{N. Musliu}, \hyperref[auth:a46]{D. Walkiewicz} & Modeling and Solving Parallel Machine Scheduling with Contamination Constraints in the Agricultural Industry & \href{works/WinterMMW22.pdf}{Yes} & \cite{WinterMMW22} & 2022 & CP 2022 & 18 & \ref{b:WinterMMW22} & \ref{c:WinterMMW22}\\
GeibingerKKMMW21 \href{https://doi.org/10.1007/978-3-030-78230-6\_29}{GeibingerKKMMW21} & \hyperref[auth:a77]{T. Geibinger}, \hyperref[auth:a78]{L. Kletzander}, \hyperref[auth:a79]{M. Krainz}, \hyperref[auth:a80]{F. Mischek}, \hyperref[auth:a45]{N. Musliu}, \hyperref[auth:a43]{F. Winter} & Physician Scheduling During a Pandemic & \href{works/GeibingerKKMMW21.pdf}{Yes} & \cite{GeibingerKKMMW21} & 2021 & CPAIOR 2021 & 10 & \ref{b:GeibingerKKMMW21} & \ref{c:GeibingerKKMMW21}\\
GeibingerMM21 \href{https://doi.org/10.1609/aaai.v35i7.16789}{GeibingerMM21} & \hyperref[auth:a77]{T. Geibinger}, \hyperref[auth:a80]{F. Mischek}, \hyperref[auth:a45]{N. Musliu} & Constraint Logic Programming for Real-World Test Laboratory Scheduling & \href{works/GeibingerMM21.pdf}{Yes} & \cite{GeibingerMM21} & 2021 & AAAI 2021 & 9 & \ref{b:GeibingerMM21} & \ref{c:GeibingerMM21}\\
LacknerMMWW21 \href{https://doi.org/10.4230/LIPIcs.CP.2021.37}{LacknerMMWW21} & \hyperref[auth:a62]{M. Lackner}, \hyperref[auth:a63]{C. Mrkvicka}, \hyperref[auth:a45]{N. Musliu}, \hyperref[auth:a46]{D. Walkiewicz}, \hyperref[auth:a43]{F. Winter} & Minimizing Cumulative Batch Processing Time for an Industrial Oven Scheduling Problem & \href{works/LacknerMMWW21.pdf}{Yes} & \cite{LacknerMMWW21} & 2021 & CP 2021 & 18 & \ref{b:LacknerMMWW21} & \ref{c:LacknerMMWW21}\\
GeibingerMM19 \href{https://doi.org/10.1007/978-3-030-19212-9\_20}{GeibingerMM19} & \hyperref[auth:a77]{T. Geibinger}, \hyperref[auth:a80]{F. Mischek}, \hyperref[auth:a45]{N. Musliu} & Investigating Constraint Programming for Real World Industrial Test Laboratory Scheduling & \href{works/GeibingerMM19.pdf}{Yes} & \cite{GeibingerMM19} & 2019 & CPAIOR 2019 & 16 & \ref{b:GeibingerMM19} & \ref{c:GeibingerMM19}\\
abs-1911-04766 \href{http://arxiv.org/abs/1911.04766}{abs-1911-04766} & \hyperref[auth:a77]{T. Geibinger}, \hyperref[auth:a80]{F. Mischek}, \hyperref[auth:a45]{N. Musliu} & Investigating Constraint Programming and Hybrid Methods for Real World Industrial Test Laboratory Scheduling & \href{works/abs-1911-04766.pdf}{Yes} & \cite{abs-1911-04766} & 2019 & CoRR & 16 & \ref{b:abs-1911-04766} & \ref{c:abs-1911-04766}\\
MusliuSS18 \href{https://doi.org/10.1007/978-3-319-93031-2\_31}{MusliuSS18} & \hyperref[auth:a45]{N. Musliu}, \hyperref[auth:a124]{A. Schutt}, \hyperref[auth:a125]{Peter J. Stuckey} & Solver Independent Rotating Workforce Scheduling & \href{works/MusliuSS18.pdf}{Yes} & \cite{MusliuSS18} & 2018 & CPAIOR 2018 & 17 & \ref{b:MusliuSS18} & \ref{c:MusliuSS18}\\
KletzanderM17 \href{https://doi.org/10.1007/978-3-319-59776-8\_28}{KletzanderM17} & \hyperref[auth:a78]{L. Kletzander}, \hyperref[auth:a45]{N. Musliu} & A Multi-stage Simulated Annealing Algorithm for the Torpedo Scheduling Problem & \href{works/KletzanderM17.pdf}{Yes} & \cite{KletzanderM17} & 2017 & CPAIOR 2017 & 15 & \ref{b:KletzanderM17} & \ref{c:KletzanderM17}\\
\end{longtable}
}

\subsection{Works by John N. Hooker}
\label{sec:a161}
{\scriptsize
\begin{longtable}{>{\raggedright\arraybackslash}p{3cm}>{\raggedright\arraybackslash}p{6cm}>{\raggedright\arraybackslash}p{7cm}rrrp{3cm}rrr}
\rowcolor{white}\caption{Works from bibtex (Total 9)}\\ \toprule
\rowcolor{white}Key & Authors & Title & LC & Cite & Year & \shortstack{Conference\\/Journal} & Pages & b & c \\ \midrule\endhead
\bottomrule
\endfoot
Hooker17 \href{https://doi.org/10.1007/978-3-319-66158-2\_36}{Hooker17} & \hyperref[auth:a161]{John N. Hooker} & Job Sequencing Bounds from Decision Diagrams & \href{works/Hooker17.pdf}{Yes} & \cite{Hooker17} & 2017 & CP 2017 & 14 & \ref{b:Hooker17} & \ref{c:Hooker17}\\
HechingH16 \href{https://doi.org/10.1007/978-3-319-33954-2\_14}{HechingH16} & \hyperref[auth:a322]{Aliza R. Heching}, \hyperref[auth:a161]{John N. Hooker} & Scheduling Home Hospice Care with Logic-Based Benders Decomposition & \href{works/HechingH16.pdf}{Yes} & \cite{HechingH16} & 2016 & CPAIOR 2016 & 11 & \ref{b:HechingH16} & \ref{c:HechingH16}\\
CireCH13 \href{https://doi.org/10.1007/978-3-642-38171-3\_22}{CireCH13} & \hyperref[auth:a158]{Andr{\'{e}} A. Cir{\'{e}}}, \hyperref[auth:a340]{E. Coban}, \hyperref[auth:a161]{John N. Hooker} & Mixed Integer Programming vs. Logic-Based Benders Decomposition for Planning and Scheduling & \href{works/CireCH13.pdf}{Yes} & \cite{CireCH13} & 2013 & CPAIOR 2013 & 7 & \ref{b:CireCH13} & \ref{c:CireCH13}\\
CobanH10 \href{https://doi.org/10.1007/978-3-642-13520-0\_11}{CobanH10} & \hyperref[auth:a340]{E. Coban}, \hyperref[auth:a161]{John N. Hooker} & Single-Facility Scheduling over Long Time Horizons by Logic-Based Benders Decomposition & \href{works/CobanH10.pdf}{Yes} & \cite{CobanH10} & 2010 & CPAIOR 2010 & 5 & \ref{b:CobanH10} & \ref{c:CobanH10}\\
Hooker06 \href{https://doi.org/10.1007/s10601-006-8060-2}{Hooker06} & \hyperref[auth:a161]{John N. Hooker} & An Integrated Method for Planning and Scheduling to Minimize Tardiness & \href{works/Hooker06.pdf}{Yes} & \cite{Hooker06} & 2006 & Constraints An Int. J. & 19 & \ref{b:Hooker06} & \ref{c:Hooker06}\\
Hooker05 \href{https://doi.org/10.1007/s10601-005-2812-2}{Hooker05} & \hyperref[auth:a161]{John N. Hooker} & A Hybrid Method for the Planning and Scheduling & \href{works/Hooker05.pdf}{Yes} & \cite{Hooker05} & 2005 & Constraints An Int. J. & 17 & \ref{b:Hooker05} & \ref{c:Hooker05}\\
Hooker05a \href{https://doi.org/10.1007/11564751\_25}{Hooker05a} & \hyperref[auth:a161]{John N. Hooker} & Planning and Scheduling to Minimize Tardiness & \href{works/Hooker05a.pdf}{Yes} & \cite{Hooker05a} & 2005 & CP 2005 & 14 & \ref{b:Hooker05a} & \ref{c:Hooker05a}\\
Hooker04 \href{https://doi.org/10.1007/978-3-540-30201-8\_24}{Hooker04} & \hyperref[auth:a161]{John N. Hooker} & A Hybrid Method for Planning and Scheduling & \href{works/Hooker04.pdf}{Yes} & \cite{Hooker04} & 2004 & CP 2004 & 12 & \ref{b:Hooker04} & \ref{c:Hooker04}\\
HookerY02 \href{https://doi.org/10.1007/3-540-46135-3\_46}{HookerY02} & \hyperref[auth:a161]{John N. Hooker}, \hyperref[auth:a293]{H. Yan} & A Relaxation of the Cumulative Constraint & \href{works/HookerY02.pdf}{Yes} & \cite{HookerY02} & 2002 & CP 2002 & 5 & \ref{b:HookerY02} & \ref{c:HookerY02}\\
\end{longtable}
}

\subsection{Works by Claude{-}Guy Quimper}
\label{sec:a37}
{\scriptsize
\begin{longtable}{>{\raggedright\arraybackslash}p{3cm}>{\raggedright\arraybackslash}p{6cm}>{\raggedright\arraybackslash}p{7cm}rrrp{3cm}rrr}
\rowcolor{white}\caption{Works from bibtex (Total 9)}\\ \toprule
\rowcolor{white}Key & Authors & Title & LC & Cite & Year & \shortstack{Conference\\/Journal} & Pages & b & c \\ \midrule\endhead
\bottomrule
\endfoot
BoudreaultSLQ22 \href{https://doi.org/10.4230/LIPIcs.CP.2022.10}{BoudreaultSLQ22} & \hyperref[auth:a34]{R. Boudreault}, \hyperref[auth:a35]{V. Simard}, \hyperref[auth:a36]{D. Lafond}, \hyperref[auth:a37]{C. Quimper} & A Constraint Programming Approach to Ship Refit Project Scheduling & \href{works/BoudreaultSLQ22.pdf}{Yes} & \cite{BoudreaultSLQ22} & 2022 & CP 2022 & 16 & \ref{b:BoudreaultSLQ22} & \ref{c:BoudreaultSLQ22}\\
OuelletQ22 \href{https://doi.org/10.1007/978-3-031-08011-1\_21}{OuelletQ22} & \hyperref[auth:a52]{Y. Ouellet}, \hyperref[auth:a37]{C. Quimper} & A MinCumulative Resource Constraint & \href{works/OuelletQ22.pdf}{Yes} & \cite{OuelletQ22} & 2022 & CPAIOR 2022 & 17 & \ref{b:OuelletQ22} & \ref{c:OuelletQ22}\\
Mercier-AubinGQ20 \href{https://doi.org/10.1007/978-3-030-58942-4\_22}{Mercier-AubinGQ20} & \hyperref[auth:a86]{A. Mercier{-}Aubin}, \hyperref[auth:a87]{J. Gaudreault}, \hyperref[auth:a37]{C. Quimper} & Leveraging Constraint Scheduling: {A} Case Study to the Textile Industry & \href{works/Mercier-AubinGQ20.pdf}{Yes} & \cite{Mercier-AubinGQ20} & 2020 & CPAIOR 2020 & 13 & \ref{b:Mercier-AubinGQ20} & \ref{c:Mercier-AubinGQ20}\\
FahimiOQ18 \href{https://doi.org/10.1007/s10601-018-9282-9}{FahimiOQ18} & \hyperref[auth:a122]{H. Fahimi}, \hyperref[auth:a52]{Y. Ouellet}, \hyperref[auth:a37]{C. Quimper} & Linear-time filtering algorithms for the disjunctive constraint and a quadratic filtering algorithm for the cumulative not-first not-last & \href{works/FahimiOQ18.pdf}{Yes} & \cite{FahimiOQ18} & 2018 & Constraints An Int. J. & 22 & \ref{b:FahimiOQ18} & \ref{c:FahimiOQ18}\\
KameugneFGOQ18 \href{https://doi.org/10.1007/978-3-319-93031-2\_23}{KameugneFGOQ18} & \hyperref[auth:a10]{R. Kameugne}, \hyperref[auth:a11]{S{\'{e}}v{\'{e}}rine Betmbe Fetgo}, \hyperref[auth:a315]{V. Gingras}, \hyperref[auth:a52]{Y. Ouellet}, \hyperref[auth:a37]{C. Quimper} & Horizontally Elastic Not-First/Not-Last Filtering Algorithm for Cumulative Resource Constraint & \href{works/KameugneFGOQ18.pdf}{Yes} & \cite{KameugneFGOQ18} & 2018 & CPAIOR 2018 & 17 & \ref{b:KameugneFGOQ18} & \ref{c:KameugneFGOQ18}\\
OuelletQ18 \href{https://doi.org/10.1007/978-3-319-93031-2\_34}{OuelletQ18} & \hyperref[auth:a52]{Y. Ouellet}, \hyperref[auth:a37]{C. Quimper} & A O(n {\textbackslash}log {\^{}}2 n) Checker and O(n{\^{}}2 {\textbackslash}log n) Filtering Algorithm for the Energetic Reasoning & \href{works/OuelletQ18.pdf}{Yes} & \cite{OuelletQ18} & 2018 & CPAIOR 2018 & 18 & \ref{b:OuelletQ18} & \ref{c:OuelletQ18}\\
GingrasQ16 \href{http://www.ijcai.org/Abstract/16/440}{GingrasQ16} & \hyperref[auth:a315]{V. Gingras}, \hyperref[auth:a37]{C. Quimper} & Generalizing the Edge-Finder Rule for the Cumulative Constraint & \href{works/GingrasQ16.pdf}{Yes} & \cite{GingrasQ16} & 2016 & IJCAI 2016 & 7 & \ref{b:GingrasQ16} & \ref{c:GingrasQ16}\\
BessiereHMQW14 \href{https://doi.org/10.1007/978-3-319-07046-9\_23}{BessiereHMQW14} & \hyperref[auth:a333]{C. Bessiere}, \hyperref[auth:a1]{E. Hebrard}, \hyperref[auth:a334]{M. M{\'{e}}nard}, \hyperref[auth:a37]{C. Quimper}, \hyperref[auth:a278]{T. Walsh} & Buffered Resource Constraint: Algorithms and Complexity & \href{works/BessiereHMQW14.pdf}{Yes} & \cite{BessiereHMQW14} & 2014 & CPAIOR 2014 & 16 & \ref{b:BessiereHMQW14} & \ref{c:BessiereHMQW14}\\
OuelletQ13 \href{https://doi.org/10.1007/978-3-642-40627-0\_42}{OuelletQ13} & \hyperref[auth:a240]{P. Ouellet}, \hyperref[auth:a37]{C. Quimper} & Time-Table Extended-Edge-Finding for the Cumulative Constraint & \href{works/OuelletQ13.pdf}{Yes} & \cite{OuelletQ13} & 2013 & CP 2013 & 16 & \ref{b:OuelletQ13} & \ref{c:OuelletQ13}\\
\end{longtable}
}

\subsection{Works by Pierre Schaus}
\label{sec:a147}
{\scriptsize
\begin{longtable}{>{\raggedright\arraybackslash}p{3cm}>{\raggedright\arraybackslash}p{6cm}>{\raggedright\arraybackslash}p{7cm}rrrp{3cm}rrr}
\rowcolor{white}\caption{Works from bibtex (Total 9)}\\ \toprule
\rowcolor{white}Key & Authors & Title & LC & Cite & Year & \shortstack{Conference\\/Journal} & Pages & b & c \\ \midrule\endhead
\bottomrule
\endfoot
CappartS17 \href{https://doi.org/10.1007/978-3-319-59776-8\_26}{CappartS17} & \hyperref[auth:a42]{Q. Cappart}, \hyperref[auth:a147]{P. Schaus} & Rescheduling Railway Traffic on Real Time Situations Using Time-Interval Variables & \href{works/CappartS17.pdf}{Yes} & \cite{CappartS17} & 2017 & CPAIOR 2017 & 16 & \ref{b:CappartS17} & \ref{c:CappartS17}\\
CauwelaertDMS16 \href{https://doi.org/10.1007/978-3-319-44953-1\_33}{CauwelaertDMS16} & \hyperref[auth:a206]{Sascha Van Cauwelaert}, \hyperref[auth:a207]{C. Dejemeppe}, \hyperref[auth:a149]{J. Monette}, \hyperref[auth:a147]{P. Schaus} & Efficient Filtering for the Unary Resource with Family-Based Transition Times & \href{works/CauwelaertDMS16.pdf}{Yes} & \cite{CauwelaertDMS16} & 2016 & CP 2016 & 16 & \ref{b:CauwelaertDMS16} & \ref{c:CauwelaertDMS16}\\
DejemeppeCS15 \href{https://doi.org/10.1007/978-3-319-23219-5\_7}{DejemeppeCS15} & \hyperref[auth:a207]{C. Dejemeppe}, \hyperref[auth:a206]{Sascha Van Cauwelaert}, \hyperref[auth:a147]{P. Schaus} & The Unary Resource with Transition Times & \href{works/DejemeppeCS15.pdf}{Yes} & \cite{DejemeppeCS15} & 2015 & CP 2015 & 16 & \ref{b:DejemeppeCS15} & \ref{c:DejemeppeCS15}\\
GayHLS15 \href{https://doi.org/10.1007/978-3-319-23219-5\_10}{GayHLS15} & \hyperref[auth:a216]{S. Gay}, \hyperref[auth:a217]{R. Hartert}, \hyperref[auth:a218]{C. Lecoutre}, \hyperref[auth:a147]{P. Schaus} & Conflict Ordering Search for Scheduling Problems & \href{works/GayHLS15.pdf}{Yes} & \cite{GayHLS15} & 2015 & CP 2015 & 9 & \ref{b:GayHLS15} & \ref{c:GayHLS15}\\
GayHS15 \href{https://doi.org/10.1007/978-3-319-23219-5\_11}{GayHS15} & \hyperref[auth:a216]{S. Gay}, \hyperref[auth:a217]{R. Hartert}, \hyperref[auth:a147]{P. Schaus} & Simple and Scalable Time-Table Filtering for the Cumulative Constraint & \href{works/GayHS15.pdf}{Yes} & \cite{GayHS15} & 2015 & CP 2015 & 9 & \ref{b:GayHS15} & \ref{c:GayHS15}\\
GayHS15a \href{https://doi.org/10.1007/978-3-319-18008-3\_11}{GayHS15a} & \hyperref[auth:a216]{S. Gay}, \hyperref[auth:a217]{R. Hartert}, \hyperref[auth:a147]{P. Schaus} & Time-Table Disjunctive Reasoning for the Cumulative Constraint & \href{works/GayHS15a.pdf}{Yes} & \cite{GayHS15a} & 2015 & CPAIOR 2015 & 16 & \ref{b:GayHS15a} & \ref{c:GayHS15a}\\
GaySS14 \href{https://doi.org/10.1007/978-3-319-10428-7\_59}{GaySS14} & \hyperref[auth:a216]{S. Gay}, \hyperref[auth:a147]{P. Schaus}, \hyperref[auth:a239]{Vivian De Smedt} & Continuous Casting Scheduling with Constraint Programming & \href{works/GaySS14.pdf}{Yes} & \cite{GaySS14} & 2014 & CP 2014 & 15 & \ref{b:GaySS14} & \ref{c:GaySS14}\\
HoundjiSWD14 \href{https://doi.org/10.1007/978-3-319-10428-7\_29}{HoundjiSWD14} & \hyperref[auth:a228]{Vinas{\'{e}}tan Ratheil Houndji}, \hyperref[auth:a147]{P. Schaus}, \hyperref[auth:a229]{Laurence A. Wolsey}, \hyperref[auth:a151]{Y. Deville} & The StockingCost Constraint & \href{works/HoundjiSWD14.pdf}{Yes} & \cite{HoundjiSWD14} & 2014 & CP 2014 & 16 & \ref{b:HoundjiSWD14} & \ref{c:HoundjiSWD14}\\
SchausHMCMD11 \href{https://doi.org/10.1007/s10601-010-9100-5}{SchausHMCMD11} & \hyperref[auth:a147]{P. Schaus}, \hyperref[auth:a148]{Pascal Van Hentenryck}, \hyperref[auth:a149]{J. Monette}, \hyperref[auth:a150]{C. Coffrin}, \hyperref[auth:a32]{L. Michel}, \hyperref[auth:a151]{Y. Deville} & Solving Steel Mill Slab Problems with constraint-based techniques: CP, LNS, and {CBLS} & \href{works/SchausHMCMD11.pdf}{Yes} & \cite{SchausHMCMD11} & 2011 & Constraints An Int. J. & 23 & \ref{b:SchausHMCMD11} & \ref{c:SchausHMCMD11}\\
\end{longtable}
}

\subsection{Works by Tony T. Tran}
\label{sec:a810}
{\scriptsize
\begin{longtable}{>{\raggedright\arraybackslash}p{3cm}>{\raggedright\arraybackslash}p{6cm}>{\raggedright\arraybackslash}p{7cm}rrrp{3cm}rrr}
\rowcolor{white}\caption{Works from bibtex (Total 9)}\\ \toprule
\rowcolor{white}Key & Authors & Title & LC & Cite & Year & \shortstack{Conference\\/Journal} & Pages & b & c \\ \midrule\endhead
\bottomrule
\endfoot
TranPZLDB18 \href{https://doi.org/10.1007/s10951-017-0537-x}{TranPZLDB18} & \hyperref[auth:a810]{Tony T. Tran}, \hyperref[auth:a811]{M. Padmanabhan}, \hyperref[auth:a812]{Peter Yun Zhang}, \hyperref[auth:a813]{H. Li}, \hyperref[auth:a814]{Douglas G. Down}, \hyperref[auth:a89]{J. Christopher Beck} & Multi-stage resource-aware scheduling for data centers with heterogeneous servers & \href{works/TranPZLDB18.pdf}{Yes} & \cite{TranPZLDB18} & 2018 & J. Sched. & 17 & \ref{b:TranPZLDB18} & \ref{c:TranPZLDB18}\\
TranVNB17 \href{https://doi.org/10.1613/jair.5306}{TranVNB17} & \hyperref[auth:a810]{Tony T. Tran}, \hyperref[auth:a815]{Tiago Stegun Vaquero}, \hyperref[auth:a209]{G. Nejat}, \hyperref[auth:a89]{J. Christopher Beck} & Robots in Retirement Homes: Applying Off-the-Shelf Planning and Scheduling to a Team of Assistive Robots & \href{works/TranVNB17.pdf}{Yes} & \cite{TranVNB17} & 2017 & J. Artif. Intell. Res. & 68 & \ref{b:TranVNB17} & \ref{c:TranVNB17}\\
TranVNB17a \href{https://doi.org/10.24963/ijcai.2017/726}{TranVNB17a} & \hyperref[auth:a810]{Tony T. Tran}, \hyperref[auth:a815]{Tiago Stegun Vaquero}, \hyperref[auth:a209]{G. Nejat}, \hyperref[auth:a89]{J. Christopher Beck} & Robots in Retirement Homes: Applying Off-the-Shelf Planning and Scheduling to a Team of Assistive Robots (Extended Abstract) & \href{works/TranVNB17a.pdf}{Yes} & \cite{TranVNB17a} & 2017 & IJCAI 2017 & 5 & \ref{b:TranVNB17a} & \ref{c:TranVNB17a}\\
TranAB16 \href{https://doi.org/10.1287/ijoc.2015.0666}{TranAB16} & \hyperref[auth:a810]{Tony T. Tran}, \hyperref[auth:a818]{A. Araujo}, \hyperref[auth:a89]{J. Christopher Beck} & Decomposition Methods for the Parallel Machine Scheduling Problem with Setups & No & \cite{TranAB16} & 2016 & {INFORMS} J. Comput. & 13 & No & \ref{c:TranAB16}\\
TranDRFWOVB16 \href{https://doi.org/10.1609/socs.v7i1.18390}{TranDRFWOVB16} & \hyperref[auth:a810]{Tony T. Tran}, \hyperref[auth:a820]{M. Do}, \hyperref[auth:a821]{Eleanor Gilbert Rieffel}, \hyperref[auth:a383]{J. Frank}, \hyperref[auth:a819]{Z. Wang}, \hyperref[auth:a822]{B. O'Gorman}, \hyperref[auth:a823]{D. Venturelli}, \hyperref[auth:a89]{J. Christopher Beck} & A Hybrid Quantum-Classical Approach to Solving Scheduling Problems & \href{works/TranDRFWOVB16.pdf}{Yes} & \cite{TranDRFWOVB16} & 2016 & SOCS 2016 & 9 & \ref{b:TranDRFWOVB16} & \ref{c:TranDRFWOVB16}\\
TranWDRFOVB16 \href{http://www.aaai.org/ocs/index.php/WS/AAAIW16/paper/view/12664}{TranWDRFOVB16} & \hyperref[auth:a810]{Tony T. Tran}, \hyperref[auth:a819]{Z. Wang}, \hyperref[auth:a820]{M. Do}, \hyperref[auth:a821]{Eleanor Gilbert Rieffel}, \hyperref[auth:a383]{J. Frank}, \hyperref[auth:a822]{B. O'Gorman}, \hyperref[auth:a823]{D. Venturelli}, \hyperref[auth:a89]{J. Christopher Beck} & Explorations of Quantum-Classical Approaches to Scheduling a Mars Lander Activity Problem & \href{works/TranWDRFOVB16.pdf}{Yes} & \cite{TranWDRFOVB16} & 2016 & AAAI 2016 & 9 & \ref{b:TranWDRFOVB16} & \ref{c:TranWDRFOVB16}\\
TerekhovTDB14 \href{https://doi.org/10.1613/jair.4278}{TerekhovTDB14} & \hyperref[auth:a829]{D. Terekhov}, \hyperref[auth:a810]{Tony T. Tran}, \hyperref[auth:a814]{Douglas G. Down}, \hyperref[auth:a89]{J. Christopher Beck} & Integrating Queueing Theory and Scheduling for Dynamic Scheduling Problems & \href{works/TerekhovTDB14.pdf}{Yes} & \cite{TerekhovTDB14} & 2014 & J. Artif. Intell. Res. & 38 & \ref{b:TerekhovTDB14} & \ref{c:TerekhovTDB14}\\
TranTDB13 \href{http://www.aaai.org/ocs/index.php/ICAPS/ICAPS13/paper/view/6005}{TranTDB13} & \hyperref[auth:a810]{Tony T. Tran}, \hyperref[auth:a829]{D. Terekhov}, \hyperref[auth:a814]{Douglas G. Down}, \hyperref[auth:a89]{J. Christopher Beck} & Hybrid Queueing Theory and Scheduling Models for Dynamic Environments with Sequence-Dependent Setup Times & \href{works/TranTDB13.pdf}{Yes} & \cite{TranTDB13} & 2013 & ICAPS 2013 & 9 & \ref{b:TranTDB13} & \ref{c:TranTDB13}\\
TranB12 \href{https://doi.org/10.3233/978-1-61499-098-7-774}{TranB12} & \hyperref[auth:a810]{Tony T. Tran}, \hyperref[auth:a89]{J. Christopher Beck} & Logic-based Benders Decomposition for Alternative Resource Scheduling with Sequence Dependent Setups & \href{works/TranB12.pdf}{Yes} & \cite{TranB12} & 2012 & ECAI 2012 & 6 & \ref{b:TranB12} & \ref{c:TranB12}\\
\end{longtable}
}

\subsection{Works by Pascal Van Hentenryck}
\label{sec:a148}
{\scriptsize
\begin{longtable}{>{\raggedright\arraybackslash}p{3cm}>{\raggedright\arraybackslash}p{6cm}>{\raggedright\arraybackslash}p{7cm}rrrp{3cm}rrr}
\rowcolor{white}\caption{Works from bibtex (Total 9)}\\ \toprule
\rowcolor{white}Key & Authors & Title & LC & Cite & Year & \shortstack{Conference\\/Journal} & Pages & b & c \\ \midrule\endhead
\bottomrule
\endfoot
FontaineMH16 \href{https://doi.org/10.1007/978-3-319-33954-2\_12}{FontaineMH16} & \hyperref[auth:a320]{D. Fontaine}, \hyperref[auth:a321]{Laurent D. Michel}, \hyperref[auth:a148]{Pascal Van Hentenryck} & Parallel Composition of Scheduling Solvers & \href{works/FontaineMH16.pdf}{Yes} & \cite{FontaineMH16} & 2016 & CPAIOR 2016 & 11 & \ref{b:FontaineMH16} & \ref{c:FontaineMH16}\\
EvenSH15 \href{https://doi.org/10.1007/978-3-319-23219-5\_40}{EvenSH15} & \hyperref[auth:a219]{C. Even}, \hyperref[auth:a124]{A. Schutt}, \hyperref[auth:a148]{Pascal Van Hentenryck} & A Constraint Programming Approach for Non-preemptive Evacuation Scheduling & \href{works/EvenSH15.pdf}{Yes} & \cite{EvenSH15} & 2015 & CP 2015 & 18 & \ref{b:EvenSH15} & \ref{c:EvenSH15}\\
EvenSH15a \href{http://arxiv.org/abs/1505.02487}{EvenSH15a} & \hyperref[auth:a219]{C. Even}, \hyperref[auth:a124]{A. Schutt}, \hyperref[auth:a148]{Pascal Van Hentenryck} & A Constraint Programming Approach for Non-Preemptive Evacuation Scheduling & \href{works/EvenSH15a.pdf}{Yes} & \cite{EvenSH15a} & 2015 & CoRR & 16 & \ref{b:EvenSH15a} & \ref{c:EvenSH15a}\\
SchausHMCMD11 \href{https://doi.org/10.1007/s10601-010-9100-5}{SchausHMCMD11} & \hyperref[auth:a147]{P. Schaus}, \hyperref[auth:a148]{Pascal Van Hentenryck}, \hyperref[auth:a149]{J. Monette}, \hyperref[auth:a150]{C. Coffrin}, \hyperref[auth:a32]{L. Michel}, \hyperref[auth:a151]{Y. Deville} & Solving Steel Mill Slab Problems with constraint-based techniques: CP, LNS, and {CBLS} & \href{works/SchausHMCMD11.pdf}{Yes} & \cite{SchausHMCMD11} & 2011 & Constraints An Int. J. & 23 & \ref{b:SchausHMCMD11} & \ref{c:SchausHMCMD11}\\
MonetteDH09 \href{http://aaai.org/ocs/index.php/ICAPS/ICAPS09/paper/view/712}{MonetteDH09} & \hyperref[auth:a149]{J. Monette}, \hyperref[auth:a151]{Y. Deville}, \hyperref[auth:a148]{Pascal Van Hentenryck} & Just-In-Time Scheduling with Constraint Programming & \href{works/MonetteDH09.pdf}{Yes} & \cite{MonetteDH09} & 2009 & ICAPS 2009 & 8 & \ref{b:MonetteDH09} & \ref{c:MonetteDH09}\\
DoomsH08 \href{https://doi.org/10.1007/978-3-540-68155-7\_8}{DoomsH08} & \hyperref[auth:a363]{G. Dooms}, \hyperref[auth:a148]{Pascal Van Hentenryck} & Gap Reduction Techniques for Online Stochastic Project Scheduling & \href{works/DoomsH08.pdf}{Yes} & \cite{DoomsH08} & 2008 & CPAIOR 2008 & 16 & \ref{b:DoomsH08} & \ref{c:DoomsH08}\\
HentenryckM08 \href{https://doi.org/10.1007/978-3-540-68155-7\_41}{HentenryckM08} & \hyperref[auth:a148]{Pascal Van Hentenryck}, \hyperref[auth:a32]{L. Michel} & The Steel Mill Slab Design Problem Revisited & \href{works/HentenryckM08.pdf}{Yes} & \cite{HentenryckM08} & 2008 & CPAIOR 2008 & 5 & \ref{b:HentenryckM08} & \ref{c:HentenryckM08}\\
HentenryckM04 \href{https://doi.org/10.1007/978-3-540-24664-0\_22}{HentenryckM04} & \hyperref[auth:a148]{Pascal Van Hentenryck}, \hyperref[auth:a32]{L. Michel} & Scheduling Abstractions for Local Search & \href{works/HentenryckM04.pdf}{Yes} & \cite{HentenryckM04} & 2004 & CPAIOR 2004 & 16 & \ref{b:HentenryckM04} & \ref{c:HentenryckM04}\\
DincbasSH90 \href{https://doi.org/10.1016/0743-1066(90)90052-7}{DincbasSH90} & \hyperref[auth:a726]{M. Dincbas}, \hyperref[auth:a17]{H. Simonis}, \hyperref[auth:a148]{Pascal Van Hentenryck} & Solving Large Combinatorial Problems in Logic Programming & \href{works/DincbasSH90.pdf}{Yes} & \cite{DincbasSH90} & 1990 & J. Log. Program. & 19 & \ref{b:DincbasSH90} & \ref{c:DincbasSH90}\\
\end{longtable}
}

\subsection{Works by Philippe Baptiste}
\label{sec:a163}
{\scriptsize
\begin{longtable}{>{\raggedright\arraybackslash}p{3cm}>{\raggedright\arraybackslash}p{6cm}>{\raggedright\arraybackslash}p{7cm}rrrp{3cm}rrr}
\rowcolor{white}\caption{Works from bibtex (Total 8)}\\ \toprule
\rowcolor{white}Key & Authors & Title & LC & Cite & Year & \shortstack{Conference\\/Journal} & Pages & b & c \\ \midrule\endhead
\bottomrule
\endfoot
BaptisteB18 \href{https://doi.org/10.1016/j.dam.2017.05.001}{BaptisteB18} & \hyperref[auth:a163]{P. Baptiste}, \hyperref[auth:a714]{N. Bonifas} & Redundant cumulative constraints to compute preemptive bounds & \href{works/BaptisteB18.pdf}{Yes} & \cite{BaptisteB18} & 2018 & Discret. Appl. Math. & 10 & \ref{b:BaptisteB18} & \ref{c:BaptisteB18}\\
Baptiste09 \href{https://doi.org/10.1007/978-3-642-04244-7\_1}{Baptiste09} & \hyperref[auth:a163]{P. Baptiste} & Constraint-Based Schedulers, Do They Really Work? & \href{works/Baptiste09.pdf}{Yes} & \cite{Baptiste09} & 2009 & CP 2009 & 1 & \ref{b:Baptiste09} & \ref{c:Baptiste09}\\
BaptisteLPN06 \href{https://doi.org/10.1016/S1574-6526(06)80026-X}{BaptisteLPN06} & \hyperref[auth:a163]{P. Baptiste}, \hyperref[auth:a118]{P. Laborie}, \hyperref[auth:a164]{Claude Le Pape}, \hyperref[auth:a666]{W. Nuijten} & Constraint-Based Scheduling and Planning & No & \cite{BaptisteLPN06} & 2006 & n/a & 39 & No & \ref{c:BaptisteLPN06}\\
ArtiouchineB05 \href{https://doi.org/10.1007/11564751\_8}{ArtiouchineB05} & \hyperref[auth:a264]{K. Artiouchine}, \hyperref[auth:a163]{P. Baptiste} & Inter-distance Constraint: An Extension of the All-Different Constraint for Scheduling Equal Length Jobs & \href{works/ArtiouchineB05.pdf}{Yes} & \cite{ArtiouchineB05} & 2005 & CP 2005 & 15 & \ref{b:ArtiouchineB05} & \ref{c:ArtiouchineB05}\\
BaptisteP00 \href{https://doi.org/10.1023/A:1009822502231}{BaptisteP00} & \hyperref[auth:a163]{P. Baptiste}, \hyperref[auth:a164]{Claude Le Pape} & Constraint Propagation and Decomposition Techniques for Highly Disjunctive and Highly Cumulative Project Scheduling Problems & \href{works/BaptisteP00.pdf}{Yes} & \cite{BaptisteP00} & 2000 & Constraints An Int. J. & 21 & \ref{b:BaptisteP00} & \ref{c:BaptisteP00}\\
PapaB98 \href{https://doi.org/10.1023/A:1009723704757}{PapaB98} & \hyperref[auth:a164]{Claude Le Pape}, \hyperref[auth:a163]{P. Baptiste} & Resource Constraints for Preemptive Job-shop Scheduling & \href{works/PapaB98.pdf}{Yes} & \cite{PapaB98} & 1998 & Constraints An Int. J. & 25 & \ref{b:PapaB98} & \ref{c:PapaB98}\\
BaptisteP97 \href{https://doi.org/10.1007/BFb0017454}{BaptisteP97} & \hyperref[auth:a163]{P. Baptiste}, \hyperref[auth:a164]{Claude Le Pape} & Constraint Propagation and Decomposition Techniques for Highly Disjunctive and Highly Cumulative Project Scheduling Problems & \href{works/BaptisteP97.pdf}{Yes} & \cite{BaptisteP97} & 1997 & CP 1997 & 15 & \ref{b:BaptisteP97} & \ref{c:BaptisteP97}\\
PapeB97 \href{}{PapeB97} & \hyperref[auth:a164]{Claude Le Pape}, \hyperref[auth:a163]{P. Baptiste} & A Constraint Programming Library for Preemptive and Non-Preemptive Scheduling & No & \cite{PapeB97} & 1997 & PACT 1997 & 20 & No & \ref{c:PapeB97}\\
\end{longtable}
}

\subsection{Works by Mats Carlsson}
\label{sec:a91}
{\scriptsize
\begin{longtable}{>{\raggedright\arraybackslash}p{3cm}>{\raggedright\arraybackslash}p{6cm}>{\raggedright\arraybackslash}p{7cm}rrrp{3cm}rrr}
\rowcolor{white}\caption{Works from bibtex (Total 8)}\\ \toprule
\rowcolor{white}Key & Authors & Title & LC & Cite & Year & \shortstack{Conference\\/Journal} & Pages & b & c \\ \midrule\endhead
\bottomrule
\endfoot
WessenCS20 \href{https://doi.org/10.1007/978-3-030-58942-4\_33}{WessenCS20} & \hyperref[auth:a90]{J. Wess{\'{e}}n}, \hyperref[auth:a91]{M. Carlsson}, \hyperref[auth:a92]{C. Schulte} & Scheduling of Dual-Arm Multi-tool Assembly Robots and Workspace Layout Optimization & \href{works/WessenCS20.pdf}{Yes} & \cite{WessenCS20} & 2020 & CPAIOR 2020 & 10 & \ref{b:WessenCS20} & \ref{c:WessenCS20}\\
MossigeGSMC17 \href{https://doi.org/10.1007/978-3-319-66158-2\_25}{MossigeGSMC17} & \hyperref[auth:a199]{M. Mossige}, \hyperref[auth:a200]{A. Gotlieb}, \hyperref[auth:a201]{H. Spieker}, \hyperref[auth:a202]{H. Meling}, \hyperref[auth:a91]{M. Carlsson} & Time-Aware Test Case Execution Scheduling for Cyber-Physical Systems & \href{works/MossigeGSMC17.pdf}{Yes} & \cite{MossigeGSMC17} & 2017 & CP 2017 & 18 & \ref{b:MossigeGSMC17} & \ref{c:MossigeGSMC17}\\
LetortCB15 \href{https://doi.org/10.1007/s10601-014-9172-8}{LetortCB15} & \hyperref[auth:a127]{A. Letort}, \hyperref[auth:a91]{M. Carlsson}, \hyperref[auth:a128]{N. Beldiceanu} & Synchronized sweep algorithms for scalable scheduling constraints & \href{works/LetortCB15.pdf}{Yes} & \cite{LetortCB15} & 2015 & Constraints An Int. J. & 52 & \ref{b:LetortCB15} & \ref{c:LetortCB15}\\
LetortCB13 \href{https://doi.org/10.1007/978-3-642-38171-3\_10}{LetortCB13} & \hyperref[auth:a127]{A. Letort}, \hyperref[auth:a91]{M. Carlsson}, \hyperref[auth:a128]{N. Beldiceanu} & A Synchronized Sweep Algorithm for the \emph{k-dimensional cumulative} Constraint & \href{works/LetortCB13.pdf}{Yes} & \cite{LetortCB13} & 2013 & CPAIOR 2013 & 16 & \ref{b:LetortCB13} & \ref{c:LetortCB13}\\
LetortBC12 \href{https://doi.org/10.1007/978-3-642-33558-7\_33}{LetortBC12} & \hyperref[auth:a127]{A. Letort}, \hyperref[auth:a128]{N. Beldiceanu}, \hyperref[auth:a91]{M. Carlsson} & A Scalable Sweep Algorithm for the cumulative Constraint & \href{works/LetortBC12.pdf}{Yes} & \cite{LetortBC12} & 2012 & CP 2012 & 16 & \ref{b:LetortBC12} & \ref{c:LetortBC12}\\
BeldiceanuCDP11 \href{https://doi.org/10.1007/s10479-010-0731-0}{BeldiceanuCDP11} & \hyperref[auth:a128]{N. Beldiceanu}, \hyperref[auth:a91]{M. Carlsson}, \hyperref[auth:a245]{S. Demassey}, \hyperref[auth:a362]{E. Poder} & New filtering for the \emph{cumulative} constraint in the context of non-overlapping rectangles & \href{works/BeldiceanuCDP11.pdf}{Yes} & \cite{BeldiceanuCDP11} & 2011 & Ann. Oper. Res. & 24 & \ref{b:BeldiceanuCDP11} & \ref{c:BeldiceanuCDP11}\\
BeldiceanuCP08 \href{https://doi.org/10.1007/978-3-540-68155-7\_5}{BeldiceanuCP08} & \hyperref[auth:a128]{N. Beldiceanu}, \hyperref[auth:a91]{M. Carlsson}, \hyperref[auth:a362]{E. Poder} & New Filtering for the cumulative Constraint in the Context of Non-Overlapping Rectangles & \href{works/BeldiceanuCP08.pdf}{Yes} & \cite{BeldiceanuCP08} & 2008 & CPAIOR 2008 & 15 & \ref{b:BeldiceanuCP08} & \ref{c:BeldiceanuCP08}\\
BeldiceanuC02 \href{https://doi.org/10.1007/3-540-46135-3\_5}{BeldiceanuC02} & \hyperref[auth:a128]{N. Beldiceanu}, \hyperref[auth:a91]{M. Carlsson} & A New Multi-resource cumulatives Constraint with Negative Heights & \href{works/BeldiceanuC02.pdf}{Yes} & \cite{BeldiceanuC02} & 2002 & CP 2002 & 17 & \ref{b:BeldiceanuC02} & \ref{c:BeldiceanuC02}\\
\end{longtable}
}

\subsection{Works by Helmut Simonis}
\label{sec:a17}
{\scriptsize
\begin{longtable}{>{\raggedright\arraybackslash}p{3cm}>{\raggedright\arraybackslash}p{6cm}>{\raggedright\arraybackslash}p{7cm}rrrp{3cm}rrr}
\rowcolor{white}\caption{Works from bibtex (Total 8)}\\ \toprule
\rowcolor{white}Key & Authors & Title & LC & Cite & Year & \shortstack{Conference\\/Journal} & Pages & b & c \\ \midrule\endhead
\bottomrule
\endfoot
ArmstrongGOS22 \href{https://doi.org/10.1007/978-3-031-08011-1\_1}{ArmstrongGOS22} & \hyperref[auth:a14]{E. Armstrong}, \hyperref[auth:a15]{M. Garraffa}, \hyperref[auth:a16]{B. O'Sullivan}, \hyperref[auth:a17]{H. Simonis} & A Two-Phase Hybrid Approach for the Hybrid Flexible Flowshop with Transportation Times & \href{works/ArmstrongGOS22.pdf}{Yes} & \cite{ArmstrongGOS22} & 2022 & CPAIOR 2022 & 13 & \ref{b:ArmstrongGOS22} & \ref{c:ArmstrongGOS22}\\
ArmstrongGOS21 \href{https://doi.org/10.4230/LIPIcs.CP.2021.16}{ArmstrongGOS21} & \hyperref[auth:a14]{E. Armstrong}, \hyperref[auth:a15]{M. Garraffa}, \hyperref[auth:a16]{B. O'Sullivan}, \hyperref[auth:a17]{H. Simonis} & The Hybrid Flexible Flowshop with Transportation Times & \href{works/ArmstrongGOS21.pdf}{Yes} & \cite{ArmstrongGOS21} & 2021 & CP 2021 & 18 & \ref{b:ArmstrongGOS21} & \ref{c:ArmstrongGOS21}\\
GrimesIOS14 \href{https://doi.org/10.1016/j.suscom.2014.08.009}{GrimesIOS14} & \hyperref[auth:a182]{D. Grimes}, \hyperref[auth:a183]{G. Ifrim}, \hyperref[auth:a16]{B. O'Sullivan}, \hyperref[auth:a17]{H. Simonis} & Analyzing the impact of electricity price forecasting on energy cost-aware scheduling & \href{works/GrimesIOS14.pdf}{Yes} & \cite{GrimesIOS14} & 2014 & Sustain. Comput. Informatics Syst. & 16 & \ref{b:GrimesIOS14} & \ref{c:GrimesIOS14}\\
IfrimOS12 \href{https://doi.org/10.1007/978-3-642-33558-7\_68}{IfrimOS12} & \hyperref[auth:a183]{G. Ifrim}, \hyperref[auth:a16]{B. O'Sullivan}, \hyperref[auth:a17]{H. Simonis} & Properties of Energy-Price Forecasts for Scheduling & \href{works/IfrimOS12.pdf}{Yes} & \cite{IfrimOS12} & 2012 & CP 2012 & 16 & \ref{b:IfrimOS12} & \ref{c:IfrimOS12}\\
Simonis07 \href{https://doi.org/10.1007/s10601-006-9011-7}{Simonis07} & \hyperref[auth:a17]{H. Simonis} & Models for Global Constraint Applications & \href{works/Simonis07.pdf}{Yes} & \cite{Simonis07} & 2007 & Constraints An Int. J. & 30 & \ref{b:Simonis07} & \ref{c:Simonis07}\\
Simonis95 \href{https://doi.org/10.1007/3-540-60299-2\_42}{Simonis95} & \hyperref[auth:a17]{H. Simonis} & The {CHIP} System and Its Applications & \href{works/Simonis95.pdf}{Yes} & \cite{Simonis95} & 1995 & CP 1995 & 4 & \ref{b:Simonis95} & \ref{c:Simonis95}\\
SimonisC95 \href{https://doi.org/10.1007/3-540-60299-2\_27}{SimonisC95} & \hyperref[auth:a17]{H. Simonis}, \hyperref[auth:a305]{T. Cornelissens} & Modelling Producer/Consumer Constraints & \href{works/SimonisC95.pdf}{Yes} & \cite{SimonisC95} & 1995 & CP 1995 & 14 & \ref{b:SimonisC95} & \ref{c:SimonisC95}\\
DincbasSH90 \href{https://doi.org/10.1016/0743-1066(90)90052-7}{DincbasSH90} & \hyperref[auth:a726]{M. Dincbas}, \hyperref[auth:a17]{H. Simonis}, \hyperref[auth:a148]{Pascal Van Hentenryck} & Solving Large Combinatorial Problems in Logic Programming & \href{works/DincbasSH90.pdf}{Yes} & \cite{DincbasSH90} & 1990 & J. Log. Program. & 19 & \ref{b:DincbasSH90} & \ref{c:DincbasSH90}\\
\end{longtable}
}

\subsection{Works by Mark Wallace}
\label{sec:a117}
{\scriptsize
\begin{longtable}{>{\raggedright\arraybackslash}p{3cm}>{\raggedright\arraybackslash}p{6cm}>{\raggedright\arraybackslash}p{7cm}rrrp{3cm}rrr}
\rowcolor{white}\caption{Works from bibtex (Total 8)}\\ \toprule
\rowcolor{white}Key & Authors & Title & LC & Cite & Year & \shortstack{Conference\\/Journal} & Pages & b & c \\ \midrule\endhead
\bottomrule
\endfoot
WallaceY20 \href{https://doi.org/10.1007/s10601-020-09316-z}{WallaceY20} & \hyperref[auth:a117]{M. Wallace}, \hyperref[auth:a19]{N. Yorke{-}Smith} & A new constraint programming model and solving for the cyclic hoist scheduling problem & \href{works/WallaceY20.pdf}{Yes} & \cite{WallaceY20} & 2020 & Constraints An Int. J. & 19 & \ref{b:WallaceY20} & \ref{c:WallaceY20}\\
He0GLW18 \href{https://doi.org/10.1007/978-3-319-98334-9\_42}{He0GLW18} & \hyperref[auth:a185]{S. He}, \hyperref[auth:a117]{M. Wallace}, \hyperref[auth:a186]{G. Gange}, \hyperref[auth:a187]{A. Liebman}, \hyperref[auth:a188]{C. Wilson} & A Fast and Scalable Algorithm for Scheduling Large Numbers of Devices Under Real-Time Pricing & \href{works/He0GLW18.pdf}{Yes} & \cite{He0GLW18} & 2018 & CP 2018 & 18 & \ref{b:He0GLW18} & \ref{c:He0GLW18}\\
ThiruvadyWGS14 \href{https://doi.org/10.1007/s10732-014-9260-3}{ThiruvadyWGS14} & \hyperref[auth:a400]{Dhananjay R. Thiruvady}, \hyperref[auth:a117]{M. Wallace}, \hyperref[auth:a341]{H. Gu}, \hyperref[auth:a124]{A. Schutt} & A Lagrangian relaxation and {ACO} hybrid for resource constrained project scheduling with discounted cash flows & \href{works/ThiruvadyWGS14.pdf}{Yes} & \cite{ThiruvadyWGS14} & 2014 & J. Heuristics & 34 & \ref{b:ThiruvadyWGS14} & \ref{c:ThiruvadyWGS14}\\
SchuttFSW09 \href{https://doi.org/10.1007/978-3-642-04244-7\_58}{SchuttFSW09} & \hyperref[auth:a124]{A. Schutt}, \hyperref[auth:a154]{T. Feydy}, \hyperref[auth:a125]{Peter J. Stuckey}, \hyperref[auth:a117]{M. Wallace} & Why Cumulative Decomposition Is Not as Bad as It Sounds & \href{works/SchuttFSW09.pdf}{Yes} & \cite{SchuttFSW09} & 2009 & CP 2009 & 16 & \ref{b:SchuttFSW09} & \ref{c:SchuttFSW09}\\
SakkoutW00 \href{https://doi.org/10.1023/A:1009856210543}{SakkoutW00} & \hyperref[auth:a167]{Hani El Sakkout}, \hyperref[auth:a117]{M. Wallace} & Probe Backtrack Search for Minimal Perturbation in Dynamic Scheduling & \href{works/SakkoutW00.pdf}{Yes} & \cite{SakkoutW00} & 2000 & Constraints An Int. J. & 30 & \ref{b:SakkoutW00} & \ref{c:SakkoutW00}\\
RodosekW98 \href{https://doi.org/10.1007/3-540-49481-2\_28}{RodosekW98} & \hyperref[auth:a299]{R. Rodosek}, \hyperref[auth:a117]{M. Wallace} & A Generic Model and Hybrid Algorithm for Hoist Scheduling Problems & \href{works/RodosekW98.pdf}{Yes} & \cite{RodosekW98} & 1998 & CP 1998 & 15 & \ref{b:RodosekW98} & \ref{c:RodosekW98}\\
Wallace96 \href{https://doi.org/10.1007/BF00143881}{Wallace96} & \hyperref[auth:a117]{M. Wallace} & Practical Applications of Constraint Programming & \href{works/Wallace96.pdf}{Yes} & \cite{Wallace96} & 1996 & Constraints An Int. J. & 30 & \ref{b:Wallace96} & \ref{c:Wallace96}\\
Wallace94 \href{}{Wallace94} & \hyperref[auth:a117]{M. Wallace} & Applying Constraints for Scheduling & No & \cite{Wallace94} & 1994 & Constraint Programming 1994 & 19 & No & \ref{c:Wallace94}\\
\end{longtable}
}

\subsection{Works by Thibaut Feydy}
\label{sec:a154}
{\scriptsize
\begin{longtable}{>{\raggedright\arraybackslash}p{3cm}>{\raggedright\arraybackslash}p{6cm}>{\raggedright\arraybackslash}p{7cm}rrrp{3cm}rrr}
\rowcolor{white}\caption{Works from bibtex (Total 7)}\\ \toprule
\rowcolor{white}Key & Authors & Title & LC & Cite & Year & \shortstack{Conference\\/Journal} & Pages & b & c \\ \midrule\endhead
\bottomrule
\endfoot
YoungFS17 \href{https://doi.org/10.1007/978-3-319-66158-2\_20}{YoungFS17} & \hyperref[auth:a193]{Kenneth D. Young}, \hyperref[auth:a154]{T. Feydy}, \hyperref[auth:a124]{A. Schutt} & Constraint Programming Applied to the Multi-Skill Project Scheduling Problem & \href{works/YoungFS17.pdf}{Yes} & \cite{YoungFS17} & 2017 & CP 2017 & 10 & \ref{b:YoungFS17} & \ref{c:YoungFS17}\\
SchuttFS13 \href{https://doi.org/10.1007/978-3-642-40627-0\_47}{SchuttFS13} & \hyperref[auth:a124]{A. Schutt}, \hyperref[auth:a154]{T. Feydy}, \hyperref[auth:a125]{Peter J. Stuckey} & Scheduling Optional Tasks with Explanation & \href{works/SchuttFS13.pdf}{Yes} & \cite{SchuttFS13} & 2013 & CP 2013 & 17 & \ref{b:SchuttFS13} & \ref{c:SchuttFS13}\\
SchuttFS13a \href{https://doi.org/10.1007/978-3-642-38171-3\_16}{SchuttFS13a} & \hyperref[auth:a124]{A. Schutt}, \hyperref[auth:a154]{T. Feydy}, \hyperref[auth:a125]{Peter J. Stuckey} & Explaining Time-Table-Edge-Finding Propagation for the Cumulative Resource Constraint & \href{works/SchuttFS13a.pdf}{Yes} & \cite{SchuttFS13a} & 2013 & CPAIOR 2013 & 17 & \ref{b:SchuttFS13a} & \ref{c:SchuttFS13a}\\
SchuttFSW13 \href{https://doi.org/10.1007/s10951-012-0285-x}{SchuttFSW13} & \hyperref[auth:a124]{A. Schutt}, \hyperref[auth:a154]{T. Feydy}, \hyperref[auth:a125]{Peter J. Stuckey}, \hyperref[auth:a155]{Mark G. Wallace} & Solving RCPSP/max by lazy clause generation & \href{works/SchuttFSW13.pdf}{Yes} & \cite{SchuttFSW13} & 2013 & J. Sched. & 17 & \ref{b:SchuttFSW13} & \ref{c:SchuttFSW13}\\
SchuttFSW11 \href{https://doi.org/10.1007/s10601-010-9103-2}{SchuttFSW11} & \hyperref[auth:a124]{A. Schutt}, \hyperref[auth:a154]{T. Feydy}, \hyperref[auth:a125]{Peter J. Stuckey}, \hyperref[auth:a155]{Mark G. Wallace} & Explaining the cumulative propagator & \href{works/SchuttFSW11.pdf}{Yes} & \cite{SchuttFSW11} & 2011 & Constraints An Int. J. & 33 & \ref{b:SchuttFSW11} & \ref{c:SchuttFSW11}\\
abs-1009-0347 \href{http://arxiv.org/abs/1009.0347}{abs-1009-0347} & \hyperref[auth:a124]{A. Schutt}, \hyperref[auth:a154]{T. Feydy}, \hyperref[auth:a125]{Peter J. Stuckey}, \hyperref[auth:a155]{Mark G. Wallace} & Solving the Resource Constrained Project Scheduling Problem with Generalized Precedences by Lazy Clause Generation & \href{works/abs-1009-0347.pdf}{Yes} & \cite{abs-1009-0347} & 2010 & CoRR & 37 & \ref{b:abs-1009-0347} & \ref{c:abs-1009-0347}\\
SchuttFSW09 \href{https://doi.org/10.1007/978-3-642-04244-7\_58}{SchuttFSW09} & \hyperref[auth:a124]{A. Schutt}, \hyperref[auth:a154]{T. Feydy}, \hyperref[auth:a125]{Peter J. Stuckey}, \hyperref[auth:a117]{M. Wallace} & Why Cumulative Decomposition Is Not as Bad as It Sounds & \href{works/SchuttFSW09.pdf}{Yes} & \cite{SchuttFSW09} & 2009 & CP 2009 & 16 & \ref{b:SchuttFSW09} & \ref{c:SchuttFSW09}\\
\end{longtable}
}

\subsection{Works by Zdenek Hanz{\'{a}}lek}
\label{sec:a116}
{\scriptsize
\begin{longtable}{>{\raggedright\arraybackslash}p{3cm}>{\raggedright\arraybackslash}p{6cm}>{\raggedright\arraybackslash}p{7cm}rrrp{3cm}rrr}
\rowcolor{white}\caption{Works from bibtex (Total 7)}\\ \toprule
\rowcolor{white}Key & Authors & Title & LC & Cite & Year & \shortstack{Conference\\/Journal} & Pages & b & c \\ \midrule\endhead
\bottomrule
\endfoot
Mehdizadeh-Somarin23 \href{https://doi.org/10.1007/978-3-031-43670-3\_33}{Mehdizadeh-Somarin23} & \hyperref[auth:a433]{Z. Mehdizadeh{-}Somarin}, \hyperref[auth:a434]{R. Tavakkoli{-}Moghaddam}, \hyperref[auth:a435]{M. Rohaninejad}, \hyperref[auth:a116]{Z. Hanz{\'{a}}lek}, \hyperref[auth:a436]{Behdin Vahedi Nouri} & A Constraint Programming Model for a Reconfigurable Job Shop Scheduling Problem with Machine Availability & \href{works/Mehdizadeh-Somarin23.pdf}{Yes} & \cite{Mehdizadeh-Somarin23} & 2023 & APMS 2023 & 14 & \ref{b:Mehdizadeh-Somarin23} & \ref{c:Mehdizadeh-Somarin23}\\
abs-2305-19888 \href{https://doi.org/10.48550/arXiv.2305.19888}{abs-2305-19888} & \hyperref[auth:a437]{V. Heinz}, \hyperref[auth:a438]{A. Nov{\'{a}}k}, \hyperref[auth:a313]{M. Vlk}, \hyperref[auth:a116]{Z. Hanz{\'{a}}lek} & Constraint Programming and Constructive Heuristics for Parallel Machine Scheduling with Sequence-Dependent Setups and Common Servers & \href{works/abs-2305-19888.pdf}{Yes} & \cite{abs-2305-19888} & 2023 & CoRR & 42 & \ref{b:abs-2305-19888} & \ref{c:abs-2305-19888}\\
HeinzNVH22 \href{https://doi.org/10.1016/j.cie.2022.108586}{HeinzNVH22} & \hyperref[auth:a437]{V. Heinz}, \hyperref[auth:a438]{A. Nov{\'{a}}k}, \hyperref[auth:a313]{M. Vlk}, \hyperref[auth:a116]{Z. Hanz{\'{a}}lek} & Constraint Programming and constructive heuristics for parallel machine scheduling with sequence-dependent setups and common servers & \href{works/HeinzNVH22.pdf}{Yes} & \cite{HeinzNVH22} & 2022 & Comput. Ind. Eng. & 16 & \ref{b:HeinzNVH22} & \ref{c:HeinzNVH22}\\
VlkHT21 \href{https://doi.org/10.1016/j.cie.2021.107317}{VlkHT21} & \hyperref[auth:a313]{M. Vlk}, \hyperref[auth:a116]{Z. Hanz{\'{a}}lek}, \hyperref[auth:a480]{S. Tang} & Constraint programming approaches to joint routing and scheduling in time-sensitive networks & \href{works/VlkHT21.pdf}{Yes} & \cite{VlkHT21} & 2021 & Comput. Ind. Eng. & 14 & \ref{b:VlkHT21} & \ref{c:VlkHT21}\\
BenediktMH20 \href{https://doi.org/10.1007/s10601-020-09317-y}{BenediktMH20} & \hyperref[auth:a114]{O. Benedikt}, \hyperref[auth:a115]{I. M{\'{o}}dos}, \hyperref[auth:a116]{Z. Hanz{\'{a}}lek} & Power of pre-processing: production scheduling with variable energy pricing and power-saving states & \href{works/BenediktMH20.pdf}{Yes} & \cite{BenediktMH20} & 2020 & Constraints An Int. J. & 19 & \ref{b:BenediktMH20} & \ref{c:BenediktMH20}\\
BenediktSMVH18 \href{https://doi.org/10.1007/978-3-319-93031-2\_6}{BenediktSMVH18} & \hyperref[auth:a114]{O. Benedikt}, \hyperref[auth:a312]{P. Sucha}, \hyperref[auth:a115]{I. M{\'{o}}dos}, \hyperref[auth:a313]{M. Vlk}, \hyperref[auth:a116]{Z. Hanz{\'{a}}lek} & Energy-Aware Production Scheduling with Power-Saving Modes & \href{works/BenediktSMVH18.pdf}{Yes} & \cite{BenediktSMVH18} & 2018 & CPAIOR 2018 & 10 & \ref{b:BenediktSMVH18} & \ref{c:BenediktSMVH18}\\
KelbelH11 \href{https://doi.org/10.1007/s10845-009-0318-2}{KelbelH11} & \hyperref[auth:a627]{J. Kelbel}, \hyperref[auth:a116]{Z. Hanz{\'{a}}lek} & Solving production scheduling with earliness/tardiness penalties by constraint programming & \href{works/KelbelH11.pdf}{Yes} & \cite{KelbelH11} & 2011 & J. Intell. Manuf. & 10 & \ref{b:KelbelH11} & \ref{c:KelbelH11}\\
\end{longtable}
}

\subsection{Works by Andr{\'{a}}s Kov{\'{a}}cs}
\label{sec:a146}
{\scriptsize
\begin{longtable}{>{\raggedright\arraybackslash}p{3cm}>{\raggedright\arraybackslash}p{6cm}>{\raggedright\arraybackslash}p{7cm}rrrp{3cm}rrr}
\rowcolor{white}\caption{Works from bibtex (Total 7)}\\ \toprule
\rowcolor{white}Key & Authors & Title & LC & Cite & Year & \shortstack{Conference\\/Journal} & Pages & b & c \\ \midrule\endhead
\bottomrule
\endfoot
KovacsB11 \href{https://doi.org/10.1007/s10601-009-9088-x}{KovacsB11} & \hyperref[auth:a146]{A. Kov{\'{a}}cs}, \hyperref[auth:a89]{J. Christopher Beck} & A global constraint for total weighted completion time for unary resources & \href{works/KovacsB11.pdf}{Yes} & \cite{KovacsB11} & 2011 & Constraints An Int. J. & 24 & \ref{b:KovacsB11} & \ref{c:KovacsB11}\\
KovacsK11 \href{https://doi.org/10.1007/s10601-010-9102-3}{KovacsK11} & \hyperref[auth:a146]{A. Kov{\'{a}}cs}, \hyperref[auth:a156]{T. Kis} & Constraint programming approach to a bilevel scheduling problem & \href{works/KovacsK11.pdf}{Yes} & \cite{KovacsK11} & 2011 & Constraints An Int. J. & 24 & \ref{b:KovacsK11} & \ref{c:KovacsK11}\\
KovacsB08 \href{https://doi.org/10.1016/j.engappai.2008.03.004}{KovacsB08} & \hyperref[auth:a146]{A. Kov{\'{a}}cs}, \hyperref[auth:a89]{J. Christopher Beck} & A global constraint for total weighted completion time for cumulative resources & \href{works/KovacsB08.pdf}{Yes} & \cite{KovacsB08} & 2008 & Eng. Appl. Artif. Intell. & 7 & \ref{b:KovacsB08} & \ref{c:KovacsB08}\\
KovacsB07 \href{https://doi.org/10.1007/978-3-540-72397-4\_9}{KovacsB07} & \hyperref[auth:a146]{A. Kov{\'{a}}cs}, \hyperref[auth:a89]{J. Christopher Beck} & A Global Constraint for Total Weighted Completion Time & \href{works/KovacsB07.pdf}{Yes} & \cite{KovacsB07} & 2007 & CPAIOR 2007 & 15 & \ref{b:KovacsB07} & \ref{c:KovacsB07}\\
KovacsV06 \href{https://doi.org/10.1007/11757375\_13}{KovacsV06} & \hyperref[auth:a146]{A. Kov{\'{a}}cs}, \hyperref[auth:a280]{J. V{\'{a}}ncza} & Progressive Solutions: {A} Simple but Efficient Dominance Rule for Practical {RCPSP} & \href{works/KovacsV06.pdf}{Yes} & \cite{KovacsV06} & 2006 & CPAIOR 2006 & 13 & \ref{b:KovacsV06} & \ref{c:KovacsV06}\\
KovacsEKV05 \href{https://doi.org/10.1007/11564751\_118}{KovacsEKV05} & \hyperref[auth:a146]{A. Kov{\'{a}}cs}, \hyperref[auth:a279]{P. Egri}, \hyperref[auth:a156]{T. Kis}, \hyperref[auth:a280]{J. V{\'{a}}ncza} & Proterv-II: An Integrated Production Planning and Scheduling System & \href{works/KovacsEKV05.pdf}{Yes} & \cite{KovacsEKV05} & 2005 & CP 2005 & 1 & \ref{b:KovacsEKV05} & \ref{c:KovacsEKV05}\\
KovacsV04 \href{https://doi.org/10.1007/978-3-540-30201-8\_26}{KovacsV04} & \hyperref[auth:a146]{A. Kov{\'{a}}cs}, \hyperref[auth:a280]{J. V{\'{a}}ncza} & Completable Partial Solutions in Constraint Programming and Constraint-Based Scheduling & \href{works/KovacsV04.pdf}{Yes} & \cite{KovacsV04} & 2004 & CP 2004 & 15 & \ref{b:KovacsV04} & \ref{c:KovacsV04}\\
\end{longtable}
}

\subsection{Works by Gabriela P. Henning}
\label{sec:a596}
{\scriptsize
\begin{longtable}{>{\raggedright\arraybackslash}p{3cm}>{\raggedright\arraybackslash}p{6cm}>{\raggedright\arraybackslash}p{7cm}rrrp{3cm}rrr}
\rowcolor{white}\caption{Works from bibtex (Total 7)}\\ \toprule
\rowcolor{white}Key & Authors & Title & LC & Cite & Year & \shortstack{Conference\\/Journal} & Pages & b & c \\ \midrule\endhead
\bottomrule
\endfoot
NovaraNH16 \href{https://doi.org/10.1016/j.compchemeng.2016.04.030}{NovaraNH16} & \hyperref[auth:a595]{Franco M. Novara}, \hyperref[auth:a529]{Juan M. Novas}, \hyperref[auth:a596]{Gabriela P. Henning} & A novel constraint programming model for large-scale scheduling problems in multiproduct multistage batch plants: Limited resources and campaign-based operation & \href{works/NovaraNH16.pdf}{Yes} & \cite{NovaraNH16} & 2016 & Comput. Chem. Eng. & 17 & \ref{b:NovaraNH16} & \ref{c:NovaraNH16}\\
NovasH14 \href{https://doi.org/10.1016/j.eswa.2013.09.026}{NovasH14} & \hyperref[auth:a529]{Juan M. Novas}, \hyperref[auth:a596]{Gabriela P. Henning} & Integrated scheduling of resource-constrained flexible manufacturing systems using constraint programming & \href{works/NovasH14.pdf}{Yes} & \cite{NovasH14} & 2014 & Expert Syst. Appl. & 14 & \ref{b:NovasH14} & \ref{c:NovasH14}\\
NovasH12 \href{https://doi.org/10.1016/j.compchemeng.2012.01.005}{NovasH12} & \hyperref[auth:a529]{Juan M. Novas}, \hyperref[auth:a596]{Gabriela P. Henning} & A comprehensive constraint programming approach for the rolling horizon-based scheduling of automated wet-etch stations & \href{works/NovasH12.pdf}{Yes} & \cite{NovasH12} & 2012 & Comput. Chem. Eng. & 17 & \ref{b:NovasH12} & \ref{c:NovasH12}\\
NovasH10 \href{https://doi.org/10.1016/j.compchemeng.2010.07.011}{NovasH10} & \hyperref[auth:a529]{Juan M. Novas}, \hyperref[auth:a596]{Gabriela P. Henning} & Reactive scheduling framework based on domain knowledge and constraint programming & \href{works/NovasH10.pdf}{Yes} & \cite{NovasH10} & 2010 & Comput. Chem. Eng. & 20 & \ref{b:NovasH10} & \ref{c:NovasH10}\\
ZeballosQH10 \href{https://doi.org/10.1016/j.engappai.2009.07.002}{ZeballosQH10} & \hyperref[auth:a630]{L. Zeballos}, \hyperref[auth:a631]{O. Quiroga}, \hyperref[auth:a596]{Gabriela P. Henning} & A constraint programming model for the scheduling of flexible manufacturing systems with machine and tool limitations & \href{works/ZeballosQH10.pdf}{Yes} & \cite{ZeballosQH10} & 2010 & Eng. Appl. Artif. Intell. & 20 & \ref{b:ZeballosQH10} & \ref{c:ZeballosQH10}\\
QuirogaZH05 \href{https://doi.org/10.1109/ROBOT.2005.1570686}{QuirogaZH05} & \hyperref[auth:a631]{O. Quiroga}, \hyperref[auth:a630]{L. Zeballos}, \hyperref[auth:a596]{Gabriela P. Henning} & A Constraint Programming Approach to Tool Allocation and Resource Scheduling in {FMS} & \href{works/QuirogaZH05.pdf}{Yes} & \cite{QuirogaZH05} & 2005 & ICRA 2005 & 6 & \ref{b:QuirogaZH05} & \ref{c:QuirogaZH05}\\
ZeballosH05 \href{http://journal.iberamia.org/index.php/ia/article/view/452/article\%20\%281\%29.pdf}{ZeballosH05} & \hyperref[auth:a630]{L. Zeballos}, \hyperref[auth:a596]{Gabriela P. Henning} & A Constraint Programming Approach to {FMS} Scheduling. Consideration of Storage and Transportation Resources & \href{works/ZeballosH05.pdf}{Yes} & \cite{ZeballosH05} & 2005 & Inteligencia Artif. & 10 & \ref{b:ZeballosH05} & \ref{c:ZeballosH05}\\
\end{longtable}
}

\subsection{Works by Stefan Heinz}
\label{sec:a133}
{\scriptsize
\begin{longtable}{>{\raggedright\arraybackslash}p{3cm}>{\raggedright\arraybackslash}p{6cm}>{\raggedright\arraybackslash}p{7cm}rrrp{3cm}rrr}
\rowcolor{white}\caption{Works from bibtex (Total 6)}\\ \toprule
\rowcolor{white}Key & Authors & Title & LC & Cite & Year & \shortstack{Conference\\/Journal} & Pages & b & c \\ \midrule\endhead
\bottomrule
\endfoot
HeinzKB13 \href{https://doi.org/10.1007/978-3-642-38171-3\_2}{HeinzKB13} & \hyperref[auth:a133]{S. Heinz}, \hyperref[auth:a336]{W. Ku}, \hyperref[auth:a89]{J. Christopher Beck} & Recent Improvements Using Constraint Integer Programming for Resource Allocation and Scheduling & \href{works/HeinzKB13.pdf}{Yes} & \cite{HeinzKB13} & 2013 & CPAIOR 2013 & 16 & \ref{b:HeinzKB13} & \ref{c:HeinzKB13}\\
HeinzSB13 \href{https://doi.org/10.1007/s10601-012-9136-9}{HeinzSB13} & \hyperref[auth:a133]{S. Heinz}, \hyperref[auth:a134]{J. Schulz}, \hyperref[auth:a89]{J. Christopher Beck} & Using dual presolving reductions to reformulate cumulative constraints & \href{works/HeinzSB13.pdf}{Yes} & \cite{HeinzSB13} & 2013 & Constraints An Int. J. & 36 & \ref{b:HeinzSB13} & \ref{c:HeinzSB13}\\
HeinzB12 \href{https://doi.org/10.1007/978-3-642-29828-8\_14}{HeinzB12} & \hyperref[auth:a133]{S. Heinz}, \hyperref[auth:a89]{J. Christopher Beck} & Reconsidering Mixed Integer Programming and MIP-Based Hybrids for Scheduling & \href{works/HeinzB12.pdf}{Yes} & \cite{HeinzB12} & 2012 & CPAIOR 2012 & 17 & \ref{b:HeinzB12} & \ref{c:HeinzB12}\\
HeinzSSW12 \href{https://doi.org/10.1007/s10601-011-9113-8}{HeinzSSW12} & \hyperref[auth:a133]{S. Heinz}, \hyperref[auth:a139]{T. Schlechte}, \hyperref[auth:a140]{R. Stephan}, \hyperref[auth:a141]{M. Winkler} & Solving steel mill slab design problems & \href{works/HeinzSSW12.pdf}{Yes} & \cite{HeinzSSW12} & 2012 & Constraints An Int. J. & 12 & \ref{b:HeinzSSW12} & \ref{c:HeinzSSW12}\\
HeinzS11 \href{https://doi.org/10.1007/978-3-642-20662-7\_34}{HeinzS11} & \hyperref[auth:a133]{S. Heinz}, \hyperref[auth:a134]{J. Schulz} & Explanations for the Cumulative Constraint: An Experimental Study & \href{works/HeinzS11.pdf}{Yes} & \cite{HeinzS11} & 2011 & SEA 2011 & 10 & \ref{b:HeinzS11} & \ref{c:HeinzS11}\\
BertholdHLMS10 \href{https://doi.org/10.1007/978-3-642-13520-0\_34}{BertholdHLMS10} & \hyperref[auth:a355]{T. Berthold}, \hyperref[auth:a133]{S. Heinz}, \hyperref[auth:a356]{Marco E. L{\"{u}}bbecke}, \hyperref[auth:a357]{Rolf H. M{\"{o}}hring}, \hyperref[auth:a134]{J. Schulz} & A Constraint Integer Programming Approach for Resource-Constrained Project Scheduling & \href{works/BertholdHLMS10.pdf}{Yes} & \cite{BertholdHLMS10} & 2010 & CPAIOR 2010 & 5 & \ref{b:BertholdHLMS10} & \ref{c:BertholdHLMS10}\\
\end{longtable}
}

\subsection{Works by Claude Le Pape}
\label{sec:a164}
{\scriptsize
\begin{longtable}{>{\raggedright\arraybackslash}p{3cm}>{\raggedright\arraybackslash}p{6cm}>{\raggedright\arraybackslash}p{7cm}rrrp{3cm}rrr}
\rowcolor{white}\caption{Works from bibtex (Total 6)}\\ \toprule
\rowcolor{white}Key & Authors & Title & LC & Cite & Year & \shortstack{Conference\\/Journal} & Pages & b & c \\ \midrule\endhead
\bottomrule
\endfoot
BaptisteLPN06 \href{https://doi.org/10.1016/S1574-6526(06)80026-X}{BaptisteLPN06} & \hyperref[auth:a163]{P. Baptiste}, \hyperref[auth:a118]{P. Laborie}, \hyperref[auth:a164]{Claude Le Pape}, \hyperref[auth:a666]{W. Nuijten} & Constraint-Based Scheduling and Planning & No & \cite{BaptisteLPN06} & 2006 & n/a & 39 & No & \ref{c:BaptisteLPN06}\\
BaptisteP00 \href{https://doi.org/10.1023/A:1009822502231}{BaptisteP00} & \hyperref[auth:a163]{P. Baptiste}, \hyperref[auth:a164]{Claude Le Pape} & Constraint Propagation and Decomposition Techniques for Highly Disjunctive and Highly Cumulative Project Scheduling Problems & \href{works/BaptisteP00.pdf}{Yes} & \cite{BaptisteP00} & 2000 & Constraints An Int. J. & 21 & \ref{b:BaptisteP00} & \ref{c:BaptisteP00}\\
NuijtenP98 \href{https://doi.org/10.1023/A:1009687210594}{NuijtenP98} & \hyperref[auth:a666]{W. Nuijten}, \hyperref[auth:a164]{Claude Le Pape} & Constraint-Based Job Shop Scheduling with {\textbackslash}sc Ilog Scheduler & \href{works/NuijtenP98.pdf}{Yes} & \cite{NuijtenP98} & 1998 & J. Heuristics & 16 & \ref{b:NuijtenP98} & \ref{c:NuijtenP98}\\
PapaB98 \href{https://doi.org/10.1023/A:1009723704757}{PapaB98} & \hyperref[auth:a164]{Claude Le Pape}, \hyperref[auth:a163]{P. Baptiste} & Resource Constraints for Preemptive Job-shop Scheduling & \href{works/PapaB98.pdf}{Yes} & \cite{PapaB98} & 1998 & Constraints An Int. J. & 25 & \ref{b:PapaB98} & \ref{c:PapaB98}\\
BaptisteP97 \href{https://doi.org/10.1007/BFb0017454}{BaptisteP97} & \hyperref[auth:a163]{P. Baptiste}, \hyperref[auth:a164]{Claude Le Pape} & Constraint Propagation and Decomposition Techniques for Highly Disjunctive and Highly Cumulative Project Scheduling Problems & \href{works/BaptisteP97.pdf}{Yes} & \cite{BaptisteP97} & 1997 & CP 1997 & 15 & \ref{b:BaptisteP97} & \ref{c:BaptisteP97}\\
PapeB97 \href{}{PapeB97} & \hyperref[auth:a164]{Claude Le Pape}, \hyperref[auth:a163]{P. Baptiste} & A Constraint Programming Library for Preemptive and Non-Preemptive Scheduling & No & \cite{PapeB97} & 1997 & PACT 1997 & 20 & No & \ref{c:PapeB97}\\
\end{longtable}
}

\subsection{Works by Emmanuel Poder}
\label{sec:a362}
{\scriptsize
\begin{longtable}{>{\raggedright\arraybackslash}p{3cm}>{\raggedright\arraybackslash}p{6cm}>{\raggedright\arraybackslash}p{7cm}rrrp{3cm}rrr}
\rowcolor{white}\caption{Works from bibtex (Total 6)}\\ \toprule
\rowcolor{white}Key & Authors & Title & LC & Cite & Year & \shortstack{Conference\\/Journal} & Pages & b & c \\ \midrule\endhead
\bottomrule
\endfoot
BeldiceanuCDP11 \href{https://doi.org/10.1007/s10479-010-0731-0}{BeldiceanuCDP11} & \hyperref[auth:a128]{N. Beldiceanu}, \hyperref[auth:a91]{M. Carlsson}, \hyperref[auth:a245]{S. Demassey}, \hyperref[auth:a362]{E. Poder} & New filtering for the \emph{cumulative} constraint in the context of non-overlapping rectangles & \href{works/BeldiceanuCDP11.pdf}{Yes} & \cite{BeldiceanuCDP11} & 2011 & Ann. Oper. Res. & 24 & \ref{b:BeldiceanuCDP11} & \ref{c:BeldiceanuCDP11}\\
abs-0907-0939 \href{http://arxiv.org/abs/0907.0939}{abs-0907-0939} & \hyperref[auth:a226]{T. Petit}, \hyperref[auth:a362]{E. Poder} & The Soft Cumulative Constraint & \href{works/abs-0907-0939.pdf}{Yes} & \cite{abs-0907-0939} & 2009 & CoRR & 12 & \ref{b:abs-0907-0939} & \ref{c:abs-0907-0939}\\
BeldiceanuCP08 \href{https://doi.org/10.1007/978-3-540-68155-7\_5}{BeldiceanuCP08} & \hyperref[auth:a128]{N. Beldiceanu}, \hyperref[auth:a91]{M. Carlsson}, \hyperref[auth:a362]{E. Poder} & New Filtering for the cumulative Constraint in the Context of Non-Overlapping Rectangles & \href{works/BeldiceanuCP08.pdf}{Yes} & \cite{BeldiceanuCP08} & 2008 & CPAIOR 2008 & 15 & \ref{b:BeldiceanuCP08} & \ref{c:BeldiceanuCP08}\\
PoderB08 \href{http://www.aaai.org/Library/ICAPS/2008/icaps08-033.php}{PoderB08} & \hyperref[auth:a362]{E. Poder}, \hyperref[auth:a128]{N. Beldiceanu} & Filtering for a Continuous Multi-Resources cumulative Constraint with Resource Consumption and Production & \href{works/PoderB08.pdf}{Yes} & \cite{PoderB08} & 2008 & ICAPS 2008 & 8 & \ref{b:PoderB08} & \ref{c:PoderB08}\\
BeldiceanuP07 \href{https://doi.org/10.1007/978-3-540-72397-4\_16}{BeldiceanuP07} & \hyperref[auth:a128]{N. Beldiceanu}, \hyperref[auth:a362]{E. Poder} & A Continuous Multi-resources \emph{cumulative} Constraint with Positive-Negative Resource Consumption-Production & \href{works/BeldiceanuP07.pdf}{Yes} & \cite{BeldiceanuP07} & 2007 & CPAIOR 2007 & 15 & \ref{b:BeldiceanuP07} & \ref{c:BeldiceanuP07}\\
PoderBS04 \href{https://doi.org/10.1016/S0377-2217(02)00756-7}{PoderBS04} & \hyperref[auth:a362]{E. Poder}, \hyperref[auth:a128]{N. Beldiceanu}, \hyperref[auth:a722]{E. Sanlaville} & Computing a lower approximation of the compulsory part of a task with varying duration and varying resource consumption & \href{works/PoderBS04.pdf}{Yes} & \cite{PoderBS04} & 2004 & Eur. J. Oper. Res. & 16 & \ref{b:PoderBS04} & \ref{c:PoderBS04}\\
\end{longtable}
}

\subsection{Works by Yves Deville}
\label{sec:a151}
{\scriptsize
\begin{longtable}{>{\raggedright\arraybackslash}p{3cm}>{\raggedright\arraybackslash}p{6cm}>{\raggedright\arraybackslash}p{7cm}rrrp{3cm}rrr}
\rowcolor{white}\caption{Works from bibtex (Total 5)}\\ \toprule
\rowcolor{white}Key & Authors & Title & LC & Cite & Year & \shortstack{Conference\\/Journal} & Pages & b & c \\ \midrule\endhead
\bottomrule
\endfoot
DejemeppeD14 \href{https://doi.org/10.1007/978-3-319-07046-9\_20}{DejemeppeD14} & \hyperref[auth:a207]{C. Dejemeppe}, \hyperref[auth:a151]{Y. Deville} & Continuously Degrading Resource and Interval Dependent Activity Durations in Nuclear Medicine Patient Scheduling & \href{works/DejemeppeD14.pdf}{Yes} & \cite{DejemeppeD14} & 2014 & CPAIOR 2014 & 9 & \ref{b:DejemeppeD14} & \ref{c:DejemeppeD14}\\
HoundjiSWD14 \href{https://doi.org/10.1007/978-3-319-10428-7\_29}{HoundjiSWD14} & \hyperref[auth:a228]{Vinas{\'{e}}tan Ratheil Houndji}, \hyperref[auth:a147]{P. Schaus}, \hyperref[auth:a229]{Laurence A. Wolsey}, \hyperref[auth:a151]{Y. Deville} & The StockingCost Constraint & \href{works/HoundjiSWD14.pdf}{Yes} & \cite{HoundjiSWD14} & 2014 & CP 2014 & 16 & \ref{b:HoundjiSWD14} & \ref{c:HoundjiSWD14}\\
SchausHMCMD11 \href{https://doi.org/10.1007/s10601-010-9100-5}{SchausHMCMD11} & \hyperref[auth:a147]{P. Schaus}, \hyperref[auth:a148]{Pascal Van Hentenryck}, \hyperref[auth:a149]{J. Monette}, \hyperref[auth:a150]{C. Coffrin}, \hyperref[auth:a32]{L. Michel}, \hyperref[auth:a151]{Y. Deville} & Solving Steel Mill Slab Problems with constraint-based techniques: CP, LNS, and {CBLS} & \href{works/SchausHMCMD11.pdf}{Yes} & \cite{SchausHMCMD11} & 2011 & Constraints An Int. J. & 23 & \ref{b:SchausHMCMD11} & \ref{c:SchausHMCMD11}\\
MonetteDH09 \href{http://aaai.org/ocs/index.php/ICAPS/ICAPS09/paper/view/712}{MonetteDH09} & \hyperref[auth:a149]{J. Monette}, \hyperref[auth:a151]{Y. Deville}, \hyperref[auth:a148]{Pascal Van Hentenryck} & Just-In-Time Scheduling with Constraint Programming & \href{works/MonetteDH09.pdf}{Yes} & \cite{MonetteDH09} & 2009 & ICAPS 2009 & 8 & \ref{b:MonetteDH09} & \ref{c:MonetteDH09}\\
MonetteDD07 \href{https://doi.org/10.1007/978-3-540-72397-4\_14}{MonetteDD07} & \hyperref[auth:a149]{J. Monette}, \hyperref[auth:a151]{Y. Deville}, \hyperref[auth:a372]{P. Dupont} & A Position-Based Propagator for the Open-Shop Problem & \href{works/MonetteDD07.pdf}{Yes} & \cite{MonetteDD07} & 2007 & CPAIOR 2007 & 14 & \ref{b:MonetteDD07} & \ref{c:MonetteDD07}\\
\end{longtable}
}

\subsection{Works by Mark G. Wallace}
\label{sec:a155}
{\scriptsize
\begin{longtable}{>{\raggedright\arraybackslash}p{3cm}>{\raggedright\arraybackslash}p{6cm}>{\raggedright\arraybackslash}p{7cm}rrrp{3cm}rrr}
\rowcolor{white}\caption{Works from bibtex (Total 5)}\\ \toprule
\rowcolor{white}Key & Authors & Title & LC & Cite & Year & \shortstack{Conference\\/Journal} & Pages & b & c \\ \midrule\endhead
\bottomrule
\endfoot
SchuttFSW13 \href{https://doi.org/10.1007/s10951-012-0285-x}{SchuttFSW13} & \hyperref[auth:a124]{A. Schutt}, \hyperref[auth:a154]{T. Feydy}, \hyperref[auth:a125]{Peter J. Stuckey}, \hyperref[auth:a155]{Mark G. Wallace} & Solving RCPSP/max by lazy clause generation & \href{works/SchuttFSW13.pdf}{Yes} & \cite{SchuttFSW13} & 2013 & J. Sched. & 17 & \ref{b:SchuttFSW13} & \ref{c:SchuttFSW13}\\
GuSW12 \href{https://doi.org/10.1007/978-3-642-33558-7\_55}{GuSW12} & \hyperref[auth:a341]{H. Gu}, \hyperref[auth:a125]{Peter J. Stuckey}, \hyperref[auth:a155]{Mark G. Wallace} & Maximising the Net Present Value of Large Resource-Constrained Projects & \href{works/GuSW12.pdf}{Yes} & \cite{GuSW12} & 2012 & CP 2012 & 15 & \ref{b:GuSW12} & \ref{c:GuSW12}\\
SchuttCSW12 \href{https://doi.org/10.1007/978-3-642-29828-8\_24}{SchuttCSW12} & \hyperref[auth:a124]{A. Schutt}, \hyperref[auth:a348]{G. Chu}, \hyperref[auth:a125]{Peter J. Stuckey}, \hyperref[auth:a155]{Mark G. Wallace} & Maximising the Net Present Value for Resource-Constrained Project Scheduling & \href{works/SchuttCSW12.pdf}{Yes} & \cite{SchuttCSW12} & 2012 & CPAIOR 2012 & 17 & \ref{b:SchuttCSW12} & \ref{c:SchuttCSW12}\\
SchuttFSW11 \href{https://doi.org/10.1007/s10601-010-9103-2}{SchuttFSW11} & \hyperref[auth:a124]{A. Schutt}, \hyperref[auth:a154]{T. Feydy}, \hyperref[auth:a125]{Peter J. Stuckey}, \hyperref[auth:a155]{Mark G. Wallace} & Explaining the cumulative propagator & \href{works/SchuttFSW11.pdf}{Yes} & \cite{SchuttFSW11} & 2011 & Constraints An Int. J. & 33 & \ref{b:SchuttFSW11} & \ref{c:SchuttFSW11}\\
abs-1009-0347 \href{http://arxiv.org/abs/1009.0347}{abs-1009-0347} & \hyperref[auth:a124]{A. Schutt}, \hyperref[auth:a154]{T. Feydy}, \hyperref[auth:a125]{Peter J. Stuckey}, \hyperref[auth:a155]{Mark G. Wallace} & Solving the Resource Constrained Project Scheduling Problem with Generalized Precedences by Lazy Clause Generation & \href{works/abs-1009-0347.pdf}{Yes} & \cite{abs-1009-0347} & 2010 & CoRR & 37 & \ref{b:abs-1009-0347} & \ref{c:abs-1009-0347}\\
\end{longtable}
}

\subsection{Works by Diarmuid Grimes}
\label{sec:a182}
{\scriptsize
\begin{longtable}{>{\raggedright\arraybackslash}p{3cm}>{\raggedright\arraybackslash}p{6cm}>{\raggedright\arraybackslash}p{7cm}rrrp{3cm}rrr}
\rowcolor{white}\caption{Works from bibtex (Total 5)}\\ \toprule
\rowcolor{white}Key & Authors & Title & LC & Cite & Year & \shortstack{Conference\\/Journal} & Pages & b & c \\ \midrule\endhead
\bottomrule
\endfoot
GrimesH15 \href{https://doi.org/10.1287/ijoc.2014.0625}{GrimesH15} & \hyperref[auth:a182]{D. Grimes}, \hyperref[auth:a1]{E. Hebrard} & Solving Variants of the Job Shop Scheduling Problem Through Conflict-Directed Search & No & \cite{GrimesH15} & 2015 & {INFORMS} J. Comput. & 17 & No & \ref{c:GrimesH15}\\
GrimesIOS14 \href{https://doi.org/10.1016/j.suscom.2014.08.009}{GrimesIOS14} & \hyperref[auth:a182]{D. Grimes}, \hyperref[auth:a183]{G. Ifrim}, \hyperref[auth:a16]{B. O'Sullivan}, \hyperref[auth:a17]{H. Simonis} & Analyzing the impact of electricity price forecasting on energy cost-aware scheduling & \href{works/GrimesIOS14.pdf}{Yes} & \cite{GrimesIOS14} & 2014 & Sustain. Comput. Informatics Syst. & 16 & \ref{b:GrimesIOS14} & \ref{c:GrimesIOS14}\\
GrimesH11 \href{https://doi.org/10.1007/978-3-642-23786-7\_28}{GrimesH11} & \hyperref[auth:a182]{D. Grimes}, \hyperref[auth:a1]{E. Hebrard} & Models and Strategies for Variants of the Job Shop Scheduling Problem & \href{works/GrimesH11.pdf}{Yes} & \cite{GrimesH11} & 2011 & CP 2011 & 17 & \ref{b:GrimesH11} & \ref{c:GrimesH11}\\
GrimesH10 \href{https://doi.org/10.1007/978-3-642-13520-0\_19}{GrimesH10} & \hyperref[auth:a182]{D. Grimes}, \hyperref[auth:a1]{E. Hebrard} & Job Shop Scheduling with Setup Times and Maximal Time-Lags: {A} Simple Constraint Programming Approach & \href{works/GrimesH10.pdf}{Yes} & \cite{GrimesH10} & 2010 & CPAIOR 2010 & 15 & \ref{b:GrimesH10} & \ref{c:GrimesH10}\\
GrimesHM09 \href{https://doi.org/10.1007/978-3-642-04244-7\_33}{GrimesHM09} & \hyperref[auth:a182]{D. Grimes}, \hyperref[auth:a1]{E. Hebrard}, \hyperref[auth:a82]{A. Malapert} & Closing the Open Shop: Contradicting Conventional Wisdom & \href{works/GrimesHM09.pdf}{Yes} & \cite{GrimesHM09} & 2009 & CP 2009 & 9 & \ref{b:GrimesHM09} & \ref{c:GrimesHM09}\\
\end{longtable}
}

\subsection{Works by Roger Kameugne}
\label{sec:a10}
{\scriptsize
\begin{longtable}{>{\raggedright\arraybackslash}p{3cm}>{\raggedright\arraybackslash}p{6cm}>{\raggedright\arraybackslash}p{7cm}rrrp{3cm}rrr}
\rowcolor{white}\caption{Works from bibtex (Total 5)}\\ \toprule
\rowcolor{white}Key & Authors & Title & LC & Cite & Year & \shortstack{Conference\\/Journal} & Pages & b & c \\ \midrule\endhead
\bottomrule
\endfoot
KameugneFND23 \href{https://doi.org/10.4230/LIPIcs.CP.2023.20}{KameugneFND23} & \hyperref[auth:a10]{R. Kameugne}, \hyperref[auth:a11]{S{\'{e}}v{\'{e}}rine Betmbe Fetgo}, \hyperref[auth:a12]{T. Noulamo}, \hyperref[auth:a13]{Cl{\'{e}}mentin Tayou Djam{\'{e}}gni} & Horizontally Elastic Edge Finder Rule for Cumulative Constraint Based on Slack and Density & \href{works/KameugneFND23.pdf}{Yes} & \cite{KameugneFND23} & 2023 & CP 2023 & 17 & \ref{b:KameugneFND23} & \ref{c:KameugneFND23}\\
KameugneFGOQ18 \href{https://doi.org/10.1007/978-3-319-93031-2\_23}{KameugneFGOQ18} & \hyperref[auth:a10]{R. Kameugne}, \hyperref[auth:a11]{S{\'{e}}v{\'{e}}rine Betmbe Fetgo}, \hyperref[auth:a315]{V. Gingras}, \hyperref[auth:a52]{Y. Ouellet}, \hyperref[auth:a37]{C. Quimper} & Horizontally Elastic Not-First/Not-Last Filtering Algorithm for Cumulative Resource Constraint & \href{works/KameugneFGOQ18.pdf}{Yes} & \cite{KameugneFGOQ18} & 2018 & CPAIOR 2018 & 17 & \ref{b:KameugneFGOQ18} & \ref{c:KameugneFGOQ18}\\
Kameugne15 \href{https://doi.org/10.1007/s10601-015-9227-5}{Kameugne15} & \hyperref[auth:a10]{R. Kameugne} & Propagation techniques of resource constraint for cumulative scheduling & \href{works/Kameugne15.pdf}{Yes} & \cite{Kameugne15} & 2015 & Constraints An Int. J. & 2 & \ref{b:Kameugne15} & \ref{c:Kameugne15}\\
KameugneFSN14 \href{https://doi.org/10.1007/s10601-013-9157-z}{KameugneFSN14} & \hyperref[auth:a10]{R. Kameugne}, \hyperref[auth:a130]{Laure Pauline Fotso}, \hyperref[auth:a131]{Joseph D. Scott}, \hyperref[auth:a132]{Y. Ngo{-}Kateu} & A quadratic edge-finding filtering algorithm for cumulative resource constraints & \href{works/KameugneFSN14.pdf}{Yes} & \cite{KameugneFSN14} & 2014 & Constraints An Int. J. & 27 & \ref{b:KameugneFSN14} & \ref{c:KameugneFSN14}\\
KameugneFSN11 \href{https://doi.org/10.1007/978-3-642-23786-7\_37}{KameugneFSN11} & \hyperref[auth:a10]{R. Kameugne}, \hyperref[auth:a130]{Laure Pauline Fotso}, \hyperref[auth:a131]{Joseph D. Scott}, \hyperref[auth:a132]{Y. Ngo{-}Kateu} & A Quadratic Edge-Finding Filtering Algorithm for Cumulative Resource Constraints & \href{works/KameugneFSN11.pdf}{Yes} & \cite{KameugneFSN11} & 2011 & CP 2011 & 15 & \ref{b:KameugneFSN11} & \ref{c:KameugneFSN11}\\
\end{longtable}
}

\subsection{Works by Juan M. Novas}
\label{sec:a529}
{\scriptsize
\begin{longtable}{>{\raggedright\arraybackslash}p{3cm}>{\raggedright\arraybackslash}p{6cm}>{\raggedright\arraybackslash}p{7cm}rrrp{3cm}rrr}
\rowcolor{white}\caption{Works from bibtex (Total 5)}\\ \toprule
\rowcolor{white}Key & Authors & Title & LC & Cite & Year & \shortstack{Conference\\/Journal} & Pages & b & c \\ \midrule\endhead
\bottomrule
\endfoot
Novas19 \href{https://doi.org/10.1016/j.cie.2019.07.011}{Novas19} & \hyperref[auth:a529]{Juan M. Novas} & Production scheduling and lot streaming at flexible job-shops environments using constraint programming & \href{works/Novas19.pdf}{Yes} & \cite{Novas19} & 2019 & Comput. Ind. Eng. & 13 & \ref{b:Novas19} & \ref{c:Novas19}\\
NovaraNH16 \href{https://doi.org/10.1016/j.compchemeng.2016.04.030}{NovaraNH16} & \hyperref[auth:a595]{Franco M. Novara}, \hyperref[auth:a529]{Juan M. Novas}, \hyperref[auth:a596]{Gabriela P. Henning} & A novel constraint programming model for large-scale scheduling problems in multiproduct multistage batch plants: Limited resources and campaign-based operation & \href{works/NovaraNH16.pdf}{Yes} & \cite{NovaraNH16} & 2016 & Comput. Chem. Eng. & 17 & \ref{b:NovaraNH16} & \ref{c:NovaraNH16}\\
NovasH14 \href{https://doi.org/10.1016/j.eswa.2013.09.026}{NovasH14} & \hyperref[auth:a529]{Juan M. Novas}, \hyperref[auth:a596]{Gabriela P. Henning} & Integrated scheduling of resource-constrained flexible manufacturing systems using constraint programming & \href{works/NovasH14.pdf}{Yes} & \cite{NovasH14} & 2014 & Expert Syst. Appl. & 14 & \ref{b:NovasH14} & \ref{c:NovasH14}\\
NovasH12 \href{https://doi.org/10.1016/j.compchemeng.2012.01.005}{NovasH12} & \hyperref[auth:a529]{Juan M. Novas}, \hyperref[auth:a596]{Gabriela P. Henning} & A comprehensive constraint programming approach for the rolling horizon-based scheduling of automated wet-etch stations & \href{works/NovasH12.pdf}{Yes} & \cite{NovasH12} & 2012 & Comput. Chem. Eng. & 17 & \ref{b:NovasH12} & \ref{c:NovasH12}\\
NovasH10 \href{https://doi.org/10.1016/j.compchemeng.2010.07.011}{NovasH10} & \hyperref[auth:a529]{Juan M. Novas}, \hyperref[auth:a596]{Gabriela P. Henning} & Reactive scheduling framework based on domain knowledge and constraint programming & \href{works/NovasH10.pdf}{Yes} & \cite{NovasH10} & 2010 & Comput. Chem. Eng. & 20 & \ref{b:NovasH10} & \ref{c:NovasH10}\\
\end{longtable}
}

\subsection{Works by Wim Nuijten}
\label{sec:a666}
{\scriptsize
\begin{longtable}{>{\raggedright\arraybackslash}p{3cm}>{\raggedright\arraybackslash}p{6cm}>{\raggedright\arraybackslash}p{7cm}rrrp{3cm}rrr}
\rowcolor{white}\caption{Works from bibtex (Total 5)}\\ \toprule
\rowcolor{white}Key & Authors & Title & LC & Cite & Year & \shortstack{Conference\\/Journal} & Pages & b & c \\ \midrule\endhead
\bottomrule
\endfoot
BaptisteLPN06 \href{https://doi.org/10.1016/S1574-6526(06)80026-X}{BaptisteLPN06} & \hyperref[auth:a163]{P. Baptiste}, \hyperref[auth:a118]{P. Laborie}, \hyperref[auth:a164]{Claude Le Pape}, \hyperref[auth:a666]{W. Nuijten} & Constraint-Based Scheduling and Planning & No & \cite{BaptisteLPN06} & 2006 & n/a & 39 & No & \ref{c:BaptisteLPN06}\\
GodardLN05 \href{http://www.aaai.org/Library/ICAPS/2005/icaps05-009.php}{GodardLN05} & \hyperref[auth:a782]{D. Godard}, \hyperref[auth:a118]{P. Laborie}, \hyperref[auth:a666]{W. Nuijten} & Randomized Large Neighborhood Search for Cumulative Scheduling & \href{works/GodardLN05.pdf}{Yes} & \cite{GodardLN05} & 2005 & ICAPS 2005 & 9 & \ref{b:GodardLN05} & \ref{c:GodardLN05}\\
FocacciLN00 \href{http://www.aaai.org/Library/AIPS/2000/aips00-010.php}{FocacciLN00} & \hyperref[auth:a784]{F. Focacci}, \hyperref[auth:a118]{P. Laborie}, \hyperref[auth:a666]{W. Nuijten} & Solving Scheduling Problems with Setup Times and Alternative Resources & \href{works/FocacciLN00.pdf}{Yes} & \cite{FocacciLN00} & 2000 & AIPS 2000 & 10 & \ref{b:FocacciLN00} & \ref{c:FocacciLN00}\\
SourdN00 \href{https://doi.org/10.1287/ijoc.12.4.341.11881}{SourdN00} & \hyperref[auth:a783]{F. Sourd}, \hyperref[auth:a666]{W. Nuijten} & Multiple-Machine Lower Bounds for Shop-Scheduling Problems & \href{works/SourdN00.pdf}{Yes} & \cite{SourdN00} & 2000 & {INFORMS} J. Comput. & 12 & \ref{b:SourdN00} & \ref{c:SourdN00}\\
NuijtenP98 \href{https://doi.org/10.1023/A:1009687210594}{NuijtenP98} & \hyperref[auth:a666]{W. Nuijten}, \hyperref[auth:a164]{Claude Le Pape} & Constraint-Based Job Shop Scheduling with {\textbackslash}sc Ilog Scheduler & \href{works/NuijtenP98.pdf}{Yes} & \cite{NuijtenP98} & 1998 & J. Heuristics & 16 & \ref{b:NuijtenP98} & \ref{c:NuijtenP98}\\
\end{longtable}
}

\subsection{Works by Louis{-}Martin Rousseau}
\label{sec:a331}
{\scriptsize
\begin{longtable}{>{\raggedright\arraybackslash}p{3cm}>{\raggedright\arraybackslash}p{6cm}>{\raggedright\arraybackslash}p{7cm}rrrp{3cm}rrr}
\rowcolor{white}\caption{Works from bibtex (Total 5)}\\ \toprule
\rowcolor{white}Key & Authors & Title & LC & Cite & Year & \shortstack{Conference\\/Journal} & Pages & b & c \\ \midrule\endhead
\bottomrule
\endfoot
DoulabiRP16 \href{https://doi.org/10.1287/ijoc.2015.0686}{DoulabiRP16} & \hyperref[auth:a335]{Seyed Hossein Hashemi Doulabi}, \hyperref[auth:a331]{L. Rousseau}, \hyperref[auth:a8]{G. Pesant} & A Constraint-Programming-Based Branch-and-Price-and-Cut Approach for Operating Room Planning and Scheduling & \href{works/DoulabiRP16.pdf}{Yes} & \cite{DoulabiRP16} & 2016 & {INFORMS} J. Comput. & 17 & \ref{b:DoulabiRP16} & \ref{c:DoulabiRP16}\\
PesantRR15 \href{https://doi.org/10.1007/978-3-319-18008-3\_21}{PesantRR15} & \hyperref[auth:a8]{G. Pesant}, \hyperref[auth:a330]{G. Rix}, \hyperref[auth:a331]{L. Rousseau} & A Comparative Study of {MIP} and {CP} Formulations for the {B2B} Scheduling Optimization Problem & \href{works/PesantRR15.pdf}{Yes} & \cite{PesantRR15} & 2015 & CPAIOR 2015 & 16 & \ref{b:PesantRR15} & \ref{c:PesantRR15}\\
DoulabiRP14 \href{https://doi.org/10.1007/978-3-319-07046-9\_32}{DoulabiRP14} & \hyperref[auth:a335]{Seyed Hossein Hashemi Doulabi}, \hyperref[auth:a331]{L. Rousseau}, \hyperref[auth:a8]{G. Pesant} & A Constraint Programming-Based Column Generation Approach for Operating Room Planning and Scheduling & \href{works/DoulabiRP14.pdf}{Yes} & \cite{DoulabiRP14} & 2014 & CPAIOR 2014 & 9 & \ref{b:DoulabiRP14} & \ref{c:DoulabiRP14}\\
ChapadosJR11 \href{https://doi.org/10.1007/978-3-642-21311-3\_7}{ChapadosJR11} & \hyperref[auth:a349]{N. Chapados}, \hyperref[auth:a350]{M. Joliveau}, \hyperref[auth:a331]{L. Rousseau} & Retail Store Workforce Scheduling by Expected Operating Income Maximization & \href{works/ChapadosJR11.pdf}{Yes} & \cite{ChapadosJR11} & 2011 & CPAIOR 2011 & 6 & \ref{b:ChapadosJR11} & \ref{c:ChapadosJR11}\\
HachemiGR11 \href{https://doi.org/10.1007/s10479-010-0698-x}{HachemiGR11} & \hyperref[auth:a623]{Nizar El Hachemi}, \hyperref[auth:a624]{M. Gendreau}, \hyperref[auth:a331]{L. Rousseau} & A hybrid constraint programming approach to the log-truck scheduling problem & \href{works/HachemiGR11.pdf}{Yes} & \cite{HachemiGR11} & 2011 & Ann. Oper. Res. & 16 & \ref{b:HachemiGR11} & \ref{c:HachemiGR11}\\
\end{longtable}
}

\subsection{Works by Marek Vlk}
\label{sec:a313}
{\scriptsize
\begin{longtable}{>{\raggedright\arraybackslash}p{3cm}>{\raggedright\arraybackslash}p{6cm}>{\raggedright\arraybackslash}p{7cm}rrrp{3cm}rrr}
\rowcolor{white}\caption{Works from bibtex (Total 5)}\\ \toprule
\rowcolor{white}Key & Authors & Title & LC & Cite & Year & \shortstack{Conference\\/Journal} & Pages & b & c \\ \midrule\endhead
\bottomrule
\endfoot
abs-2305-19888 \href{https://doi.org/10.48550/arXiv.2305.19888}{abs-2305-19888} & \hyperref[auth:a437]{V. Heinz}, \hyperref[auth:a438]{A. Nov{\'{a}}k}, \hyperref[auth:a313]{M. Vlk}, \hyperref[auth:a116]{Z. Hanz{\'{a}}lek} & Constraint Programming and Constructive Heuristics for Parallel Machine Scheduling with Sequence-Dependent Setups and Common Servers & \href{works/abs-2305-19888.pdf}{Yes} & \cite{abs-2305-19888} & 2023 & CoRR & 42 & \ref{b:abs-2305-19888} & \ref{c:abs-2305-19888}\\
HeinzNVH22 \href{https://doi.org/10.1016/j.cie.2022.108586}{HeinzNVH22} & \hyperref[auth:a437]{V. Heinz}, \hyperref[auth:a438]{A. Nov{\'{a}}k}, \hyperref[auth:a313]{M. Vlk}, \hyperref[auth:a116]{Z. Hanz{\'{a}}lek} & Constraint Programming and constructive heuristics for parallel machine scheduling with sequence-dependent setups and common servers & \href{works/HeinzNVH22.pdf}{Yes} & \cite{HeinzNVH22} & 2022 & Comput. Ind. Eng. & 16 & \ref{b:HeinzNVH22} & \ref{c:HeinzNVH22}\\
VlkHT21 \href{https://doi.org/10.1016/j.cie.2021.107317}{VlkHT21} & \hyperref[auth:a313]{M. Vlk}, \hyperref[auth:a116]{Z. Hanz{\'{a}}lek}, \hyperref[auth:a480]{S. Tang} & Constraint programming approaches to joint routing and scheduling in time-sensitive networks & \href{works/VlkHT21.pdf}{Yes} & \cite{VlkHT21} & 2021 & Comput. Ind. Eng. & 14 & \ref{b:VlkHT21} & \ref{c:VlkHT21}\\
BenediktSMVH18 \href{https://doi.org/10.1007/978-3-319-93031-2\_6}{BenediktSMVH18} & \hyperref[auth:a114]{O. Benedikt}, \hyperref[auth:a312]{P. Sucha}, \hyperref[auth:a115]{I. M{\'{o}}dos}, \hyperref[auth:a313]{M. Vlk}, \hyperref[auth:a116]{Z. Hanz{\'{a}}lek} & Energy-Aware Production Scheduling with Power-Saving Modes & \href{works/BenediktSMVH18.pdf}{Yes} & \cite{BenediktSMVH18} & 2018 & CPAIOR 2018 & 10 & \ref{b:BenediktSMVH18} & \ref{c:BenediktSMVH18}\\
BartakV15 \href{}{BartakV15} & \hyperref[auth:a152]{R. Bart{\'{a}}k}, \hyperref[auth:a313]{M. Vlk} & Reactive Recovery from Machine Breakdown in Production Scheduling with Temporal Distance and Resource Constraints & \href{works/BartakV15.pdf}{Yes} & \cite{BartakV15} & 2015 & ICAART 2015 & 12 & \ref{b:BartakV15} & \ref{c:BartakV15}\\
\end{longtable}
}

\subsection{Works by Armin Wolf}
\label{sec:a51}
{\scriptsize
\begin{longtable}{>{\raggedright\arraybackslash}p{3cm}>{\raggedright\arraybackslash}p{6cm}>{\raggedright\arraybackslash}p{7cm}rrrp{3cm}rrr}
\rowcolor{white}\caption{Works from bibtex (Total 5)}\\ \toprule
\rowcolor{white}Key & Authors & Title & LC & Cite & Year & \shortstack{Conference\\/Journal} & Pages & b & c \\ \midrule\endhead
\bottomrule
\endfoot
GeitzGSSW22 \href{https://doi.org/10.1007/978-3-031-08011-1\_10}{GeitzGSSW22} & \hyperref[auth:a47]{M. Geitz}, \hyperref[auth:a48]{C. Grozea}, \hyperref[auth:a49]{W. Steigerwald}, \hyperref[auth:a50]{R. St{\"{o}}hr}, \hyperref[auth:a51]{A. Wolf} & Solving the Extended Job Shop Scheduling Problem with AGVs - Classical and Quantum Approaches & \href{works/GeitzGSSW22.pdf}{Yes} & \cite{GeitzGSSW22} & 2022 & CPAIOR 2022 & 18 & \ref{b:GeitzGSSW22} & \ref{c:GeitzGSSW22}\\
SchuttW10 \href{https://doi.org/10.1007/978-3-642-15396-9\_36}{SchuttW10} & \hyperref[auth:a124]{A. Schutt}, \hyperref[auth:a51]{A. Wolf} & A New \emph{O}(\emph{n}\({}^{\mbox{2}}\)log\emph{n}) Not-First/Not-Last Pruning Algorithm for Cumulative Resource Constraints & \href{works/SchuttW10.pdf}{Yes} & \cite{SchuttW10} & 2010 & CP 2010 & 15 & \ref{b:SchuttW10} & \ref{c:SchuttW10}\\
SchuttWS05 \href{https://doi.org/10.1007/11963578\_6}{SchuttWS05} & \hyperref[auth:a124]{A. Schutt}, \hyperref[auth:a51]{A. Wolf}, \hyperref[auth:a720]{G. Schrader} & Not-First and Not-Last Detection for Cumulative Scheduling in \emph{O}(\emph{n}\({}^{\mbox{3}}\)log\emph{n}) & \href{works/SchuttWS05.pdf}{Yes} & \cite{SchuttWS05} & 2005 & INAP 2005 & 15 & \ref{b:SchuttWS05} & \ref{c:SchuttWS05}\\
WolfS05 \href{https://doi.org/10.1007/11963578\_8}{WolfS05} & \hyperref[auth:a51]{A. Wolf}, \hyperref[auth:a720]{G. Schrader} & \emph{O}(\emph{n} log\emph{n}) Overload Checking for the Cumulative Constraint and Its Application & \href{works/WolfS05.pdf}{Yes} & \cite{WolfS05} & 2005 & INAP 2005 & 14 & \ref{b:WolfS05} & \ref{c:WolfS05}\\
Wolf03 \href{https://doi.org/10.1007/978-3-540-45193-8\_50}{Wolf03} & \hyperref[auth:a51]{A. Wolf} & Pruning while Sweeping over Task Intervals & \href{works/Wolf03.pdf}{Yes} & \cite{Wolf03} & 2003 & CP 2003 & 15 & \ref{b:Wolf03} & \ref{c:Wolf03}\\
\end{longtable}
}



\clearpage
\section{Other Works}

\clearpage
\subsection{Books from bibtex}
{\scriptsize
\begin{longtable}{>{\raggedright\arraybackslash}p{3cm}>{\raggedright\arraybackslash}p{6cm}>{\raggedright\arraybackslash}p{6.5cm}rrrp{2.5cm}rrrrr}
\rowcolor{white}\caption{Works from bibtex (Total 3)}\\ \toprule
\rowcolor{white}\shortstack{Key\\Source} & Authors & Title & LC & Cite & Year & \shortstack{Conference\\/Journal\\/School} & Pages & \shortstack{Nr\\Cites} & \shortstack{Nr\\Refs} & b & c \\ \midrule\endhead
\bottomrule
\endfoot
\rowlabel{a:ArtiguesDN08}ArtiguesDN08 \href{http://dx.doi.org/10.1002/9780470611227}{ArtiguesDN08} & \hyperref[auth:a941]{} & Resource Constrained Project Scheduling & No & \cite{ArtiguesDN08} & 2008 & Book & null & 63 & 0 & No & n/a\\
\rowlabel{a:BaptistePN01}BaptistePN01 \href{http://dx.doi.org/10.1007/978-1-4615-1479-4}{BaptistePN01} & \hyperref[auth:a163]{P. Baptiste}, \hyperref[auth:a164]{Claude Le Pape}, \hyperref[auth:a664]{W. Nuijten} & Constraint-Based Scheduling & No & \cite{BaptistePN01} & 2001 & Book & null & 296 & 0 & No & n/a\\
\rowlabel{a:Hooker00}Hooker00 \href{http://dx.doi.org/10.1002/9781118033036}{Hooker00} & \hyperref[auth:a161]{John N. Hooker} & Logic Based Methods for Optimization: Combining Optimization and Constraint Satisfaction & No & \cite{Hooker00} & 2000 & Book & null & 185 & 0 & No & n/a\\
\end{longtable}
}



\clearpage
\subsection{PhDThesis from bibtex}
{\scriptsize
\begin{longtable}{>{\raggedright\arraybackslash}p{2.5cm}>{\raggedright\arraybackslash}p{4.5cm}>{\raggedright\arraybackslash}p{6.0cm}p{1.0cm}rr>{\raggedright\arraybackslash}p{2.0cm}r>{\raggedright\arraybackslash}p{1cm}p{1cm}p{1cm}p{1cm}}
\rowcolor{white}\caption{THESIS (Total 29)}\\ \toprule
\rowcolor{white}\shortstack{Key\\Source} & Authors & Title (Colored by Open Access)& \shortstack{Details\\LC} & Cite & Year & \shortstack{Conference\\/Journal\\/School} & Pages & Relevance &\shortstack{Cites\\OC XR\\SC} & \shortstack{Refs\\OC\\XR} & \shortstack{Links\\Cites\\Refs}\\ \midrule\endhead
\bottomrule
\endfoot
\index{Astrand21}\rowlabel{a:Astrand21}Astrand21 \href{https://nbn-resolving.org/urn:nbn:se:kth:diva-294959}{Astrand21} & \hyperref[auth:a74]{M. {\AA}strand} & Short-term Underground Mine Scheduling: An Industrial Application of Constraint Programming & \hyperref[detail:Astrand21]{Details} \href{../works/Astrand21.pdf}{Yes} & \cite{Astrand21} & 2021 & Royal Institute of Technology, Stockholm, Sweden & 142 & \noindent{}\textbf{1.00} \textbf{1.00} \textbf{310.47} & 0 0 0 & 0 0 & 0 0 0\\
\index{Godet21a}\rowlabel{a:Godet21a}Godet21a \href{https://tel.archives-ouvertes.fr/tel-03681868}{Godet21a} & \hyperref[auth:a470]{A. Godet} & Sur le tri de t{\^{a}}ches pour r{\'{e}}soudre des probl{\`{e}}mes d'ordonnancement avec la programmation par contraintes. (On the use of tasks ordering to solve scheduling problems with constraint programming) & \hyperref[detail:Godet21a]{Details} \href{../works/Godet21a.pdf}{Yes} & \cite{Godet21a} & 2021 & {IMT} Atlantique Bretagne Pays de la Loire, Brest, France & 168 & \noindent{}\textbf{2.50} \textbf{2.50} \textbf{172.67} & 0 0 0 & 0 0 & 0 0 0\\
\index{Groleaz21}\rowlabel{a:Groleaz21}Groleaz21 \href{https://hal.science/tel-03266690}{Groleaz21} & \hyperref[auth:a83]{L. Groleaz} & {The Group Cumulative Scheduling Problem} & \hyperref[detail:Groleaz21]{Details} \href{../works/Groleaz21.pdf}{Yes} & \cite{Groleaz21} & 2021 & {Universit{\'e} de Lyon} & 153 & \noindent{}\textcolor{black!50}{0.00} \textcolor{black!50}{0.00} \textbf{331.76} & 0 0 0 & 0 0 & 0 0 0\\
\index{Lemos21}\rowlabel{a:Lemos21}Lemos21 \href{https://scholar.tecnico.ulisboa.pt/records/u5RPHM-pu_yoOLXJF7BHrgJx47D827b0xHb3}{Lemos21} & \hyperref[auth:a875]{Alexandre Duarte {de Almeida} Lemos} & Solving scheduling problems under disruptions & \hyperref[detail:Lemos21]{Details} \href{../works/Lemos21.pdf}{Yes} & \cite{Lemos21} & 2021 & UNIVERSIDADE DE LISBOA INSTITUTO SUPERIOR TÉCNICO & 188 & \noindent{}\textcolor{black!50}{0.00} \textcolor{black!50}{0.00} \textbf{17.57} & 0 0 0 & 0 0 & 0 0 0\\
\index{Zahout21}\rowlabel{a:Zahout21}Zahout21 \href{https://hal.science/tel-03606639}{Zahout21} & \hyperref[auth:a888]{B. Zahout} & {Algorithmes exacts et approch{\'e}s pour l'ordonnancement des travaux multiressources {\`a} intervalles fixes dans des syst{\`e}mes distribu{\'e}s : approche monocrit{\`e}re et multiagent} & \hyperref[detail:Zahout21]{Details} \href{../works/Zahout21.pdf}{Yes} & \cite{Zahout21} & 2021 & {Universit{\'e} de Tours - LIFAT} & 185 & \noindent{}\textcolor{black!50}{0.00} \textcolor{black!50}{0.00} \textbf{17.91} & 0 0 0 & 0 0 & 0 0 0\\
\index{Lunardi20}\rowlabel{a:Lunardi20}Lunardi20 \href{http://orbilu.uni.lu/handle/10993/43893}{Lunardi20} & \hyperref[auth:a495]{W. T. Lunardi} & A Real-World Flexible Job Shop Scheduling Problem With Sequencing Flexibility: Mathematical Programming, Constraint Programming, and Metaheuristics & \hyperref[detail:Lunardi20]{Details} \href{../works/Lunardi20.pdf}{Yes} & \cite{Lunardi20} & 2020 & University of Luxembourg, Luxembourg City, Luxembourg & 181 & \noindent{}\textbf{2.00} \textbf{2.00} \textbf{239.22} & 0 0 0 & 0 0 & 0 0 0\\
\index{Arkhipov19}\rowlabel{a:Arkhipov19}Arkhipov19 \href{http://www.theses.fr/2019TOU30107}{Arkhipov19} & \hyperref[auth:a1035]{D. Arkhipov} & Planification socio-responsable du travail dans les chaînes de montage d'aéronefs : comment satisfaire à la fois objectifs ergonomiques et économiques & \cellcolor{red!30}\hyperref[detail:Arkhipov19]{Details} No & \cite{Arkhipov19} & 2019 & Toulouse 3 & null & \noindent{}\textcolor{black!50}{0.00} \textcolor{black!50}{0.00} n/a & 0 0 0 & 0 0 & 0 0 0\\
\index{Caballero19}\rowlabel{a:Caballero19}Caballero19 \href{https://www.tesisenred.net/handle/10803/667963#page=1}{Caballero19} & \hyperref[auth:a102]{J. C. Caballero} & Scheduling Through Logic-Based Tools & \hyperref[detail:Caballero19]{Details} \href{../works/Caballero19.pdf}{Yes} & \cite{Caballero19} & 2019 & Universitat de Girona, Spain & 194 & \noindent{}\textcolor{black!50}{0.00} \textcolor{black!50}{0.00} \textbf{43.36} & 0 0 0 & 0 0 & 0 0 0\\
\index{German18}\rowlabel{a:German18}German18 \href{https://theses.hal.science/tel-01896325}{German18} & \hyperref[auth:a889]{G. German} & {Constraint programming for lot-sizing problems} & \hyperref[detail:German18]{Details} \href{../works/German18.pdf}{Yes} & \cite{German18} & 2018 & {Universit{\'e} Grenoble Alpes} & 112 & \noindent{}\textcolor{black!50}{0.00} \textcolor{black!50}{0.00} \textbf{10.90} & 0 0 0 & 0 0 & 0 0 0\\
\index{Dejemeppe16}\rowlabel{a:Dejemeppe16}Dejemeppe16 \href{https://hdl.handle.net/2078.1/178078}{Dejemeppe16} & \hyperref[auth:a202]{C. Dejemeppe} & Constraint programming algorithms and models for scheduling applications & \hyperref[detail:Dejemeppe16]{Details} \href{../works/Dejemeppe16.pdf}{Yes} & \cite{Dejemeppe16} & 2016 & Catholic University of Louvain, Louvain-la-Neuve, Belgium & 274 & \noindent{}\textbf{1.00} \textbf{1.00} \textbf{262.14} & 0 0 0 & 0 0 & 0 0 0\\
\index{Fahimi16}\rowlabel{a:Fahimi16}Fahimi16 \href{http://cp2014.a4cp.org/sites/default/files/hamed_fahimi_-_efficient_algorithms_to_solve_scheduling_problems_with_a_variety_of_optimization_criteria.pdf}{Fahimi16} & \hyperref[auth:a122]{H. Fahimi} & Efficient algorithms to solve scheduling problems with a variety of optimization criteria & \hyperref[detail:Fahimi16]{Details} \href{../works/Fahimi16.pdf}{Yes} & \cite{Fahimi16} & 2016 & Universit{\'{e}} Laval, Quebec, Canada & 120 & \noindent{}\textcolor{black!50}{0.00} \textcolor{black!50}{0.00} \textbf{142.81} & 0 0 0 & 0 0 & 0 0 0\\
\index{Froger16}\rowlabel{a:Froger16}Froger16 \href{https://theses.hal.science/tel-01440836}{Froger16} & \hyperref[auth:a887]{A. Froger} & {Maintenance scheduling in the electricity industry : a particular focus on a problem rising in the onshore wind industry} & \hyperref[detail:Froger16]{Details} \href{../works/Froger16.pdf}{Yes} & \cite{Froger16} & 2016 & {Universit{\'e} d'Angers} & 181 & \noindent{}\textcolor{black!50}{0.00} \textcolor{black!50}{0.00} \textbf{127.37} & 0 0 0 & 0 0 & 0 0 0\\
\index{Nattaf16}\rowlabel{a:Nattaf16}Nattaf16 \href{https://laas.hal.science/tel-01417288}{Nattaf16} & \hyperref[auth:a81]{M. Nattaf} & {Ordonnancement sous contraintes d'{\'e}nergie} & \hyperref[detail:Nattaf16]{Details} \href{../works/Nattaf16.pdf}{Yes} & \cite{Nattaf16} & 2016 & {UPS Toulouse - Universit{\'e} Toulouse 3 Paul Sabatier} & 199 & \noindent{}\textcolor{black!50}{0.00} \textcolor{black!50}{0.00} \textbf{11.76} & 0 0 0 & 0 0 & 0 0 0\\
\index{Derrien15}\rowlabel{a:Derrien15}Derrien15 \href{https://tel.archives-ouvertes.fr/tel-01242789}{Derrien15} & \hyperref[auth:a220]{A. Derrien} & Ordonnancement cumulatif en programmation par contraintes : caract{\'{e}}risation {\'{e}}nerg{\'{e}}tique des raisonnements et solutions robustes. (Cumulative scheduling in constraint programming : energetic characterization of reasoning and robust solutions) & \hyperref[detail:Derrien15]{Details} \href{../works/Derrien15.pdf}{Yes} & \cite{Derrien15} & 2015 & {\'{E}}cole des mines de Nantes, France & 113 & \noindent{}\textbf{1.00} \textbf{1.00} \textbf{3.83} & 0 0 0 & 0 0 & 0 0 0\\
\index{Siala15a}\rowlabel{a:Siala15a}Siala15a \href{https://tel.archives-ouvertes.fr/tel-01164291}{Siala15a} & \hyperref[auth:a129]{M. Siala} & Search, propagation, and learning in sequencing and scheduling problems. (Recherche, propagation et apprentissage dans les probl{\`{e}}mes de s{\'{e}}quencement et d'ordonnancement) & \hyperref[detail:Siala15a]{Details} \href{../works/Siala15a.pdf}{Yes} & \cite{Siala15a} & 2015 & {INSA} Toulouse, France & 199 & \noindent{}0.50 0.50 \textbf{98.99} & 0 0 0 & 0 0 & 0 0 0\\
\index{Kameugne14}\rowlabel{a:Kameugne14}Kameugne14 \href{http://cp2013.a4cp.org/sites/default/files/roger_kameugne_-_propagation_techniques_of_resource_constraint_for_cumulative_scheduling.pdf}{Kameugne14} & \hyperref[auth:a10]{R. Kameugne} & Techniques de Propagation de la Contrainte de Ressource en Ordonnancement Cumulatif & \hyperref[detail:Kameugne14]{Details} \href{../works/Kameugne14.pdf}{Yes} & \cite{Kameugne14} & 2014 & University of Yaounde I, Cameroon & 139 & \noindent{}\textcolor{black!50}{0.00} \textcolor{black!50}{0.00} \textbf{8.15} & 0 0 0 & 0 0 & 0 0 0\\
\index{Letort13}\rowlabel{a:Letort13}Letort13 \href{https://theses.hal.science/tel-00932215}{Letort13} & \hyperref[auth:a127]{A. Letort} & {Passage {\`a} l'{\'e}chelle pour les contraintes d'ordonnancement multi-ressources} & \hyperref[detail:Letort13]{Details} \href{../works/Letort13.pdf}{Yes} & \cite{Letort13} & 2013 & {Ecole des Mines de Nantes} & 132 & \noindent{}\textcolor{black!50}{0.00} \textcolor{black!50}{0.00} \textbf{8.73} & 0 0 0 & 0 0 & 0 0 0\\
\index{Clercq12}\rowlabel{a:Clercq12}Clercq12 \href{https://theses.hal.science/tel-00794323}{Clercq12} & \hyperref[auth:a246]{A. D. Clercq} & {Ordonnancement cumulatif avec d{\'e}passements de capacit{\'e} : Contrainte globale et d{\'e}compositions} & \hyperref[detail:Clercq12]{Details} \href{../works/Clercq12.pdf}{Yes} & \cite{Clercq12} & 2012 & {Ecole des Mines de Nantes} & 196 & \noindent{}\textcolor{black!50}{0.00} \textcolor{black!50}{0.00} \textbf{6.18} & 0 0 0 & 0 0 & 0 0 0\\
\index{Malapert11}\rowlabel{a:Malapert11}Malapert11 \href{https://tel.archives-ouvertes.fr/tel-00630122}{Malapert11} & \hyperref[auth:a82]{A. Malapert} & Techniques d'ordonnancement d'atelier et de fourn{\'{e}}es bas{\'{e}}es sur la programmation par contraintes. (Shop and batch scheduling with constraints) & \hyperref[detail:Malapert11]{Details} \href{../works/Malapert11.pdf}{Yes} & \cite{Malapert11} & 2011 & {\'{E}}cole des mines de Nantes, France & 194 & \noindent{}\textcolor{black!50}{0.00} \textcolor{black!50}{0.00} \textbf{142.49} & 0 0 0 & 0 0 & 0 0 0\\
\index{Menana11}\rowlabel{a:Menana11}Menana11 \href{https://tel.archives-ouvertes.fr/tel-00785838}{Menana11} & \hyperref[auth:a613]{J. Menana} & Automates et programmation par contraintes pour la planification de personnel. (Automata and Constraint Programming for Personnel Scheduling Problems) & \hyperref[detail:Menana11]{Details} \href{../works/Menana11.pdf}{Yes} & \cite{Menana11} & 2011 & University of Nantes, France & 148 & \noindent{}\textbf{1.00} \textbf{1.00} \textbf{2.73} & 0 0 0 & 0 0 & 0 0 0\\
\index{Schutt11}\rowlabel{a:Schutt11}Schutt11 \href{https://www.a4cp.org/sites/default/files/andreas_schutt_-_improving_scheduling_by_learning.pdf}{Schutt11} & \hyperref[auth:a124]{A. Schutt} & Improving Scheduling by Learning & \hyperref[detail:Schutt11]{Details} \href{../works/Schutt11.pdf}{Yes} & \cite{Schutt11} & 2011 & University of Melbourne, Australia & 209 & \noindent{}\textcolor{black!50}{0.00} \textcolor{black!50}{0.00} \textbf{102.66} & 0 0 0 & 0 0 & 0 0 0\\
\index{Lombardi10}\rowlabel{a:Lombardi10}Lombardi10 \href{http://amsdottorato.unibo.it/2961/}{Lombardi10} & \hyperref[auth:a142]{M. Lombardi} & Hybrid Methods for Resource Allocation and Scheduling Problems in Deterministic and Stochastic Environments & \hyperref[detail:Lombardi10]{Details} \href{../works/Lombardi10.pdf}{Yes} & \cite{Lombardi10} & 2010 & University of Bologna, Italy & 175 & \noindent{}\textcolor{black!50}{0.00} \textcolor{black!50}{0.00} \textbf{251.65} & 0 0 0 & 0 0 & 0 0 0\\
\index{Malik08}\rowlabel{a:Malik08}Malik08 \href{https://hdl.handle.net/10012/3612}{Malik08} & \hyperref[auth:a637]{A. M. Malik} & Constraint Programming Techniques for Optimal Instruction Scheduling & \hyperref[detail:Malik08]{Details} \href{../works/Malik08.pdf}{Yes} & \cite{Malik08} & 2008 & University of Waterloo, Ontario, Canada & 151 & \noindent{}\textbf{1.00} \textbf{1.00} \textbf{44.76} & 0 0 0 & 0 0 & 0 0 0\\
\index{Demassey03}\rowlabel{a:Demassey03}Demassey03 \href{https://tel.archives-ouvertes.fr/tel-00293564}{Demassey03} & \hyperref[auth:a243]{S. Demassey} & M{\'{e}}thodes hybrides de programmation par contraintes et programmation lin{\'{e}}aire pour le probl{\`{e}}me d'ordonnancement de projet {\`{a}} contraintes de ressources. (Hybrid Constraint Programming-Integer Linear Programming approaches for the Resource-Constrained Project Scheduling Problem) & \hyperref[detail:Demassey03]{Details} \href{../works/Demassey03.pdf}{Yes} & \cite{Demassey03} & 2003 & University of Avignon, France & 148 & \noindent{}\textbf{1.50} \textbf{1.50} \textbf{15.90} & 0 0 0 & 0 0 & 0 0 0\\
\index{Elkhyari03}\rowlabel{a:Elkhyari03}Elkhyari03 \href{https://theses.hal.science/tel-00008377}{Elkhyari03} & \hyperref[auth:a292]{A. Elkhyari} & {Outils d'aide {\`a} la d{\'e}cision pour des probl{\`e}mes d'ordonnancement dynamiques} & \hyperref[detail:Elkhyari03]{Details} \href{../works/Elkhyari03.pdf}{Yes} & \cite{Elkhyari03} & 2003 & {Universit{\'e} de Nantes} & 333 & \noindent{}\textcolor{black!50}{0.00} \textcolor{black!50}{0.00} \textbf{24.65} & 0 0 0 & 0 0 & 0 0 0\\
\index{Baptiste02}\rowlabel{a:Baptiste02}Baptiste02 \href{https://theses.hal.science/tel-00124998}{Baptiste02} & \hyperref[auth:a162]{P. Baptiste} & {R{\'e}sultats de complexit{\'e} et programmation par contraintes pour l'ordonnancement} & \hyperref[detail:Baptiste02]{Details} \href{../works/Baptiste02.pdf}{Yes} & \cite{Baptiste02} & 2002 & {Universit{\'e} de Technologie de Compi{\`e}gne} & 237 & \noindent{}\textcolor{black!50}{0.00} \textcolor{black!50}{0.00} \textbf{1096.83} & 0 0 0 & 0 0 & 0 0 0\\
\index{Layfield02}\rowlabel{a:Layfield02}Layfield02 \href{http://etheses.whiterose.ac.uk/1301/}{Layfield02} & \hyperref[auth:a669]{C. J. Layfield} & A constraint programming pre-processor for duty scheduling & \hyperref[detail:Layfield02]{Details} \href{../works/Layfield02.pdf}{Yes} & \cite{Layfield02} & 2002 & University of Leeds, {UK} & 230 & \noindent{}\textbf{1.00} \textbf{1.00} \textcolor{black!50}{0.00} & 0 0 0 & 0 0 & 0 0 0\\
\index{Beck99}\rowlabel{a:Beck99}Beck99 \href{https://librarysearch.library.utoronto.ca/permalink/01UTORONTO_INST/14bjeso/alma991106162342106196}{Beck99} & \hyperref[auth:a89]{J. C. Beck} & Texture measurements as a basis for heuristic commitment techniques in constraint-directed scheduling & \hyperref[detail:Beck99]{Details} \href{../works/Beck99.pdf}{Yes} & \cite{Beck99} & 1999 & University of Toronto, Canada & 418 & \noindent{}\textcolor{black!50}{0.00} \textcolor{black!50}{0.00} \textbf{270.43} & 0 0 0 & 0 0 & 0 0 0\\
\index{Nuijten94}\rowlabel{a:Nuijten94}Nuijten94 \href{https://pure.tue.nl/ws/portalfiles/portal/2374269/431902.pdf}{Nuijten94} & \hyperref[auth:a655]{W. Nuijten} & Time and Resource Constrained Scheduling: a Constraint Satisfaction Approach & \hyperref[detail:Nuijten94]{Details} \href{../works/Nuijten94.pdf}{Yes} & \cite{Nuijten94} & 1994 & Eindhoven University of Technology & 172 & \noindent{}\textbf{1.50} \textbf{1.50} \textbf{68.38} & 0 0 0 & 0 0 & 0 0 0\\
\end{longtable}
}



\clearpage
{\scriptsize
\begin{longtable}{>{\raggedright\arraybackslash}p{3cm}r>{\raggedright\arraybackslash}p{4cm}p{1.5cm}p{2cm}p{1.5cm}p{1.5cm}p{1.5cm}p{1.5cm}p{2cm}p{1.5cm}rr}
\rowcolor{white}\caption{Automatically Extracted THESIS Properties (Requires Local Copy)}\\ \toprule
\rowcolor{white}Work & Pages & Concepts & Classification & Constraints & \shortstack{Prog\\Languages} & \shortstack{CP\\Systems} & Areas & Industries & Benchmarks & Algorithm & a & c\\ \midrule\endhead
\bottomrule
\endfoot
\rowlabel{b:Astrand21}\href{../works/Astrand21.pdf}{Astrand21}~\cite{Astrand21} & 142 & distributed, due-date, job-shop, flow-shop, resource, transportation, open-shop, machine, job, re-scheduling, precedence, order, inventory, tardiness, activity, setup-time, preempt, release-date, scheduling, make-span, completion-time, task, sequence dependent setup & RCPSP, parallel machine, HFS, single machine & cumulative, alldifferent, cycle, circuit, disjunctive, Disjunctive constraint, Reified constraint & C++, Julia & Cplex, OPL, Gecode & satellite, drone, agriculture, semiconductor, robot & mineral industry, mining industry, maritime industry, potash industry, shipping industry & real-world, generated instance, real-life, benchmark & time-tabling, not-first, not-last, edge-finding, NEH & \ref{a:Astrand21} & n/a\\
\rowlabel{b:Baptiste02}\href{../works/Baptiste02.pdf}{Baptiste02}~\cite{Baptiste02} & 237 & re-scheduling, resource, release-date, scheduling, preempt, flow-time, task, job-shop, machine, activity, make-span, flow-shop, job, completion-time, precedence, distributed, inventory, no preempt, setup-time, due-date, open-shop, tardiness, order, lateness, earliness, cmax, sequence dependent setup & Open Shop Scheduling Problem, PJSSP, HFS, single machine, RCPSP, OSSP, parallel machine, JSSP & cumulative, circuit, disjunctive, Cardinality constraint, Disjunctive constraint, alternative constraint, table constraint, Arithmetic constraint & Prolog, C++ & Choco Solver, Claire, Ilog Solver, OPL, CHIP, ECLiPSe, Ilog Scheduler, Z3 & hoist &  & real-life, generated instance, benchmark & not-first, energetic reasoning, not-last, edge-finding & \ref{a:Baptiste02} & n/a\\
\rowlabel{b:Beck99}\href{../works/Beck99.pdf}{Beck99}~\cite{Beck99} & 418 & due-date, multi-agent, order, distributed, preempt, scheduling, inventory, machine, release-date, job-shop, task, tardiness, activity, transportation, stock level, precedence, make-span, re-scheduling, resource, job, producer/consumer & single machine & cumulative, Disjunctive constraint, circuit, disjunctive & Prolog, C++ & Ilog Solver, CHIP, Ilog Scheduler, OPL & robot, medical &  & benchmark, real-world & not-last, edge-finding, not-first & \ref{a:Beck99} & n/a\\
\rowlabel{b:Caballero19}\href{../works/Caballero19.pdf}{Caballero19}~\cite{Caballero19} & 194 & resource, machine, setup-time, preempt, lazy clause generation, task, order, activity, distributed, precedence, release-date, cmax, make-span, scheduling, completion-time & psplib, RCPSP & alldifferent, circuit, Cardinality constraint, cycle, Arithmetic constraint, cumulative & C++ & SCIP, CHIP, Z3, CPO, Chuffed, MiniZinc, OPL &  &  & benchmark, real-life, instance generator & energetic reasoning, GRASP, time-tabling, edge-finding, bi-partite matching & \ref{a:Caballero19} & n/a\\
\rowlabel{b:Clercq12}\href{../works/Clercq12.pdf}{Clercq12}~\cite{Clercq12} & 196 & task, order, machine, job, manpower, activity, job-shop, make-span, resource, scheduling, due-date & psplib & Cumulatives constraint, alldifferent, cumulative, disjunctive, SoftCumulativeSum, circuit, SoftCumulative & Prolog & ECLiPSe, SICStus, Choco Solver, CHIP, Gecode & patient &  & benchmark & not-last, energetic reasoning, edge-finding, sweep, time-tabling, not-first & \ref{a:Clercq12} & n/a\\
\rowlabel{b:Dejemeppe16}\href{../works/Dejemeppe16.pdf}{Dejemeppe16}~\cite{Dejemeppe16} & 274 & make-span, sequence dependent setup, open-shop, order, job, activity, continuous-process, machine, preempt, release-date, flow-shop, batch process, tardiness, scheduling, completion-time, re-scheduling, resource, setup-time, earliness, due-date, no-wait, task, job-shop, lateness, precedence & PTC, psplib, single machine, RCPSP & disjunctive, cumulative, Element constraint, Reified constraint, Cumulatives constraint, alldifferent, GCC constraint, cycle, circuit, Disjunctive constraint, Cardinality constraint, Regular constraint &  & Ilog Solver, OPL, Gecode, CHIP, OR-Tools, CPO & medical, patient, super-computer, nurse, physician, robot, container terminal & paper industry & benchmark, instance generator, generated instance, industrial partner, random instance, real-world, bitbucket & not-first, not-last, sweep, edge-finding & \ref{a:Dejemeppe16} & n/a\\
\rowlabel{b:Demassey03}\href{../works/Demassey03.pdf}{Demassey03}~\cite{Demassey03} & 148 & machine, job, precedence, Benders Decomposition, release-date, job-shop, open-shop, activity, flow-shop, order, resource, scheduling, preempt, task & single machine, CuSP, psplib, RCPSP, TCSP & circuit, cumulative, disjunctive, cycle & C++ & Cplex, Claire, Ilog Solver &  &  & benchmark & not-last, edge-finding, time-tabling, not-first & \ref{a:Demassey03} & n/a\\
\rowlabel{b:Derrien15}\href{../works/Derrien15.pdf}{Derrien15}~\cite{Derrien15} & 113 & scheduling, precedence, order, make-span, task, activity, job-shop, resource, machine, job, preempt, open-shop & psplib, CuSP & Disjunctive constraint, cumulative, alldifferent, circuit, disjunctive &  & Claire, Choco Solver & robot &  & benchmark & edge-finding, sweep, time-tabling, energetic reasoning & \ref{a:Derrien15} & n/a\\
\rowlabel{b:Elkhyari03}\href{../works/Elkhyari03.pdf}{Elkhyari03}~\cite{Elkhyari03} & 333 & scheduling, task, job-shop, machine, activity, make-span, flow-shop, cmax, open-shop, tardiness, order, preempt, re-scheduling, resource, job, precedence, release-date & RCPSP, CuSP, parallel machine, Temporal Constraint Satisfaction Problem, single machine & cycle, cumulative, disjunctive &  & CPO, Choco Solver, Claire &  &  & benchmark, Roadef & time-tabling & \ref{a:Elkhyari03} & n/a\\
\rowlabel{b:Fahimi16}\href{../works/Fahimi16.pdf}{Fahimi16}~\cite{Fahimi16} & 120 & completion-time, flow-shop, precedence, batch process, setup-time, due-date, task, open-shop, order, make-span, machine, job, activity, resource, lateness, job-shop, transportation, sequence dependent setup, preempt, tardiness, scheduling, Benders Decomposition & single machine, CuSP, parallel machine, RCPSP & Disjunctive constraint, Cardinality constraint, Cumulatives constraint, alldifferent, cycle, AllDiff constraint, cumulative, alternative constraint, disjunctive & Java, C++ & Choco Solver, CHIP, Ilog Scheduler, Gecode & aircraft &  & benchmark, random instance, real-world, Roadef & time-tabling, not-first, not-last, energetic reasoning, edge-finding, max-flow, sweep & \ref{a:Fahimi16} & n/a\\
\rowlabel{b:Froger16}\href{../works/Froger16.pdf}{Froger16}~\cite{Froger16} & 181 & preempt, distributed, resource, inventory, scheduling, Benders Decomposition, batch process, re-scheduling, task, order, completion-time, machine, job, manpower, release-date, transportation & single machine, CuSP, TMS & disjunctive, cycle, cumulative & Java & Gurobi, OZ, Choco Solver & satellite, energy-price, offshore, train schedule & power industry, electricity industry, energy industry, wind industry & benchmark, real-life, real-world, industrial partner, instance generator, Roadef, generated instance & max-flow & \ref{a:Froger16} & n/a\\
\rowlabel{b:German18}\href{../works/German18.pdf}{German18}~\cite{German18} & 112 & stock level, setup-time, job, task, activity, earliness, machine, resource, job-shop, cmax, order, inventory, scheduling &  & Disjunctive constraint, Cardinality constraint, bin-packing, Balance constraint, cumulative, Among constraint, disjunctive & Prolog & Z3, SICStus, OPL, Choco Solver, Cplex & nurse &  & real-world, benchmark, real-life, CSPlib, generated instance &  & \ref{a:German18} & n/a\\
\rowlabel{b:Godet21a}\href{../works/Godet21a.pdf}{Godet21a}~\cite{Godet21a} & 168 & open-shop, release-date, make-span, transportation, machine, lazy clause generation, distributed, resource, lateness, job-shop, flow-shop, precedence, cmax, preempt, due-date, order, scheduling, Benders Decomposition, completion-time, job, task, activity & single machine, RCPSP, parallel machine, JSSP, PMSP, psplib & AllDiff constraint, bin-packing, GeneralizedAllDiffPrec, disjunctive, BinPacking constraint, cumulative, AllDiffPrec constraint, Disjunctive constraint, Element constraint, alldifferent, Cardinality constraint, cycle &  & OR-Tools, OPL, Claire, Choco Solver, Chuffed, MiniZinc, CHIP & satellite, robot, railway & electricity industry & real-life, github, generated instance, benchmark, random instance & sweep, time-tabling, edge-finding & \ref{a:Godet21a} & n/a\\
\rowlabel{b:Groleaz21}\href{../works/Groleaz21.pdf}{Groleaz21}~\cite{Groleaz21} & 153 & inventory, tardiness, activity, setup-time, preempt, release-date, earliness, scheduling, make-span, completion-time, task, sequence dependent setup, distributed, due-date, job-shop, flow-shop, resource, transportation, cmax, open-shop, machine, job, lateness, re-scheduling, precedence, order & Open Shop Scheduling Problem, single machine, GCSP, RCPSP, OSP, parallel machine & circuit, disjunctive, Disjunctive constraint, span constraint, cumulative, cycle, noOverlap & Julia, Java & Choco Solver, Z3, OPL, OR-Tools, Gurobi, CPO, Gecode, SCIP, Cplex & dairy, robot, automotive & food industry, agrifood industry, dairy industry & benchmark, real-life & edge-finding, not-first, not-last & \ref{a:Groleaz21} & n/a\\
\rowlabel{b:Kameugne14}\href{../works/Kameugne14.pdf}{Kameugne14}~\cite{Kameugne14} & 139 & resource, job, scheduling, task, job-shop, machine, make-span, flow-shop, completion-time, order, preempt & RCPSP, CuSP, parallel machine, psplib & circuit, Disjunctive constraint, Cumulatives constraint, Balance constraint, cumulative, disjunctive & Java, Prolog, C++ & Choco Solver, Claire, Gecode, CHIP, ECLiPSe, SICStus, Cplex, Mistral &  &  & Roadef & not-last, edge-finder, energetic reasoning, time-tabling, edge-finding, not-first & \ref{a:Kameugne14} & n/a\\
\rowlabel{b:Layfield02}\href{../works/Layfield02.pdf}{Layfield02}~\cite{Layfield02} & 230 &  &  &  & C  & OPL, OZ, Z3 &  &  &  &  & \ref{a:Layfield02} & n/a\\
\rowlabel{b:Lemos21}\href{../works/Lemos21.pdf}{Lemos21}~\cite{Lemos21} & 188 & transportation, precedence, job-shop, machine, re-scheduling, distributed, multi-agent, task, job, order, resource, scheduling & RCPSP & cycle, alldifferent, cumulative, Cardinality constraint & Java, C++, Python & OPL, Gurobi, Cplex & surgery, COVID, medical, crew-scheduling, railway, train schedule & railway industry & real-world, github, real-life, benchmark, Roadef & GRASP, time-tabling & \ref{a:Lemos21} & n/a\\
\rowlabel{b:Letort13}\href{../works/Letort13.pdf}{Letort13}~\cite{Letort13} & 132 & machine, resource, job-shop, precedence, cmax, order, scheduling, job, task & psplib & bin-packing, alldifferent, cumulative, geost, Cumulatives constraint, disjunctive & Java, Prolog & SICStus, Claire, Choco Solver, CHIP & steel mill, datacenter &  & Roadef, CSPlib, benchmark & energetic reasoning, edge-finding, sweep, not-first, time-tabling, not-last & \ref{a:Letort13} & n/a\\
\rowlabel{b:Lombardi10}\href{../works/Lombardi10.pdf}{Lombardi10}~\cite{Lombardi10} & 175 & re-scheduling, make-span, job, precedence, Benders Decomposition, lazy clause generation, release-date, distributed, setup-time, job-shop, due-date, activity, completion-time, order, inventory, tardiness, resource, scheduling, preempt, task, machine & single machine, SCC, CTW, RCPSP, TCSP & Disjunctive constraint, cycle, Balance constraint, AllDiff constraint, cumulative, disjunctive, table constraint, span constraint, bin-packing, circuit & C  & OPL, Cplex, Ilog Solver & aircraft, pipeline, semiconductor, medical, automotive & semiconductor industry & generated instance, benchmark, real-world, instance generator, real-life & not-last, sweep, edge-finder, edge-finding, energetic reasoning, time-tabling, not-first & \ref{a:Lombardi10} & n/a\\
\rowlabel{b:Lunardi20}\href{../works/Lunardi20.pdf}{Lunardi20}~\cite{Lunardi20} & 181 & activity, setup-time, release-date, scheduling, make-span, task, cmax, machine, job, lateness, re-scheduling, no preempt, due-date, job-shop, batch process, preempt, flow-shop, resource, transportation, open-shop, precedence, order, completion-time, tardiness & FJS, parallel machine, single machine & cycle, noOverlap, endBeforeStart, alldifferent, disjunctive & Python & CPO, OPL, Cplex & robot & printing industry, glass industry & industrial partner, instance generator, benchmark, random instance, github, supplementary material, real-world, generated instance, real-life &  & \ref{a:Lunardi20} & n/a\\
\rowlabel{b:Malapert11}\href{../works/Malapert11.pdf}{Malapert11}~\cite{Malapert11} & 194 & tardiness, order, lateness, preempt, cmax, batch process, transportation, resource, scheduling, flow-time, task, job-shop, machine, activity, make-span, no-wait, flow-shop, job, completion-time, precedence, inventory, setup-time, due-date, open-shop & Open Shop Scheduling Problem, single machine & cumulative, diffn, circuit, disjunctive, geost, cycle, alldifferent, Element constraint, bin-packing, Disjunctive constraint, Cumulatives constraint & Prolog, C++, Java & Mistral, Choco Solver, Claire, Gecode, ECLiPSe, SICStus, Cplex, OPL, CHIP, Ilog Scheduler & rectangle-packing, robot, semiconductor, patient &  & real-world, industrial partner, generated instance, benchmark & edge-finding, not-first, energetic reasoning, not-last, time-tabling, sweep & \ref{a:Malapert11} & n/a\\
\rowlabel{b:Malik08}\href{../works/Malik08.pdf}{Malik08}~\cite{Malik08} & 151 & order, machine, completion-time, activity, distributed, precedence, task, job, resource, make-span, scheduling &  & alldifferent, Cardinality constraint, cycle &  &  & pipeline &  & real-life, benchmark & edge-finding & \ref{a:Malik08} & n/a\\
\rowlabel{b:Menana11}\href{../works/Menana11.pdf}{Menana11}~\cite{Menana11} & 148 & machine, task, manpower, activity, distributed, resource, precedence, scheduling &  & Regular constraint, alldifferent, Cardinality constraint & Prolog & Z3, CHIP, OPL, Claire, Choco Solver & nurse &  & Roadef, github, benchmark & time-tabling & \ref{a:Menana11} & n/a\\
\rowlabel{b:Nattaf16}\href{../works/Nattaf16.pdf}{Nattaf16}~\cite{Nattaf16} & 199 & order, tardiness, inventory, scheduling, flow-shop, setup-time, job, task, make-span, machine, resource, job-shop, cmax, preempt, due-date & RCPSP, CECSP, psplib, single machine, CuSP, parallel machine & alldifferent, cumulative, disjunctive & C++ & Claire, Cplex & robot & process industry & Roadef & energetic reasoning, edge-finding, sweep & \ref{a:Nattaf16} & n/a\\
\rowlabel{b:Schutt11}\href{../works/Schutt11.pdf}{Schutt11}~\cite{Schutt11} & 209 & lazy clause generation, resource, job-shop, precedence, cmax, preempt, order, tardiness, scheduling, completion-time, machine, setup-time, job, task, activity, open-shop, release-date, make-span & RCPSP, Open Shop Scheduling Problem, psplib & disjunctive, Arithmetic constraint, UTVPI constraint, cumulative, circuit, bin-packing, Reified constraint, Disjunctive constraint, Element constraint, alldifferent, cycle, geost & Prolog, C++ & CHIP, SICStus, Ilog Scheduler, SCIP, ECLiPSe, Ilog Solver & rectangle-packing & carpet industry & benchmark, real-world, industrial instance, instance generator & sweep, edge-finder, time-tabling, not-first, energetic reasoning, edge-finding, not-last & \ref{a:Schutt11} & n/a\\
\rowlabel{b:Siala15a}\href{../works/Siala15a.pdf}{Siala15a}~\cite{Siala15a} & 199 & job-shop, precedence, earliness, cmax, sequence dependent setup, due-date, lazy clause generation, order, tardiness, scheduling, setup-time, task, activity, open-shop, make-span, machine, job, resource & RCPSP, OSP, single machine, TMS & AtMostSeq, table constraint, Balance constraint, cumulative, circuit, Among constraint, AmongSeq constraint, disjunctive, Atmost constraint, Regular constraint, Disjunctive constraint, GCC constraint, Cardinality constraint, CardPath, MultiAtMostSeqCard, AtMostSeqCard, Reified constraint, alldifferent, cycle &  & CHIP, Ilog Solver, Mistral, OPL, Claire & automotive, rectangle-packing &  & github, benchmark, random instance, Roadef, real-world, CSPlib & time-tabling, edge-finding, GRASP & \ref{a:Siala15a} & n/a\\
\rowlabel{b:Zahout21}\href{../works/Zahout21.pdf}{Zahout21}~\cite{Zahout21} & 185 & completion-time, machine, job, activity, release-date, make-span, multi-agent, distributed, resource, job-shop, flow-shop, precedence, preempt, due-date, re-scheduling, task, scheduling & CuSP, parallel machine, RCPSP, SCC, TCSP, single machine & cycle, cumulative, circuit, bin-packing &  & CPO, Cplex, Claire & datacenter, crew-scheduling, satellite &  & benchmark & GRASP & \ref{a:Zahout21} & n/a\\
\end{longtable}
}




\clearpage
\subsection{InBook from bibtex}
{\scriptsize
\begin{longtable}{>{\raggedright\arraybackslash}p{3cm}>{\raggedright\arraybackslash}p{6cm}>{\raggedright\arraybackslash}p{6.5cm}rrrp{2.5cm}rrrrr}
\rowcolor{white}\caption{Works from bibtex (Total 16)}\\ \toprule
\rowcolor{white}\shortstack{Key\\Source} & Authors & Title & LC & Cite & Year & \shortstack{Conference\\/Journal\\/School} & Pages & \shortstack{Nr\\Cites} & \shortstack{Nr\\Refs} & b & c \\ \midrule\endhead
\bottomrule
\endfoot
\rowlabel{a:SchuttFSW15}SchuttFSW15 \href{https://doi.org/10.1007/978-3-319-05443-8_7}{SchuttFSW15} & \hyperref[auth:a125]{A. Schutt}, \hyperref[auth:a155]{T. Feydy}, \hyperref[auth:a126]{Peter J. Stuckey}, \hyperref[auth:a117]{Mark G. Wallace} & A Satisfiability Solving Approach & No & \cite{SchuttFSW15} & 2015 & Handbook on Project Management and Scheduling Vol.1 & 26 & 3 & 28 & No & n/a\\
\rowlabel{a:CestaOPS14}CestaOPS14 \href{http://dx.doi.org/10.1007/978-3-319-05443-8_6}{CestaOPS14} & \hyperref[auth:a286]{A. Cesta}, \hyperref[auth:a284]{A. Oddi}, \hyperref[auth:a285]{N. Policella}, \hyperref[auth:a300]{Stephen F. Smith} & A Precedence Constraint Posting Approach & No & \cite{CestaOPS14} & 2014 & Handbook on Project Management and Scheduling Vol.1 & null & 2 & 17 & No & n/a\\
\rowlabel{a:GuSSWC14}GuSSWC14 \href{http://dx.doi.org/10.1007/978-3-319-05443-8_14}{GuSSWC14} & \hyperref[auth:a339]{H. Gu}, \hyperref[auth:a125]{A. Schutt}, \hyperref[auth:a126]{Peter J. Stuckey}, \hyperref[auth:a117]{Mark G. Wallace}, \hyperref[auth:a346]{G. Chu} & Exact and Heuristic Methods for the Resource-Constrained Net Present Value Problem & No & \cite{GuSSWC14} & 2014 & Handbook on Project Management and Scheduling Vol.1 & null & 5 & 35 & No & n/a\\
\rowlabel{a:Milano11}Milano11 \href{http://dx.doi.org/10.1002/9780470400531.eorms0473}{Milano11} & \hyperref[auth:a144]{M. Milano} & Constraint Programming Links with Math Programming & No & \cite{Milano11} & 2011 & Wiley Encyclopedia of Operations Research and Management Science & null & 0 & 28 & No & n/a\\
\rowlabel{a:CastroGR10}CastroGR10 \href{http://dx.doi.org/10.1007/978-1-4419-1644-0_4}{CastroGR10} & \hyperref[auth:a895]{Pedro M. Castro}, \hyperref[auth:a385]{Ignacio E. Grossmann}, \hyperref[auth:a329]{L. Rousseau} & Decomposition Techniques for Hybrid MILP/CP Models applied to Scheduling and Routing Problems & No & \cite{CastroGR10} & 2010 & Hybrid Optimization & null & 0 & 67 & No & n/a\\
\rowlabel{a:Hooker10}Hooker10 \href{http://dx.doi.org/10.1007/978-1-4419-1644-0_2}{Hooker10} & \hyperref[auth:a161]{John N. Hooker} & Hybrid Modeling & No & \cite{Hooker10} & 2010 & Hybrid Optimization & null & 9 & 39 & No & n/a\\
\rowlabel{a:GongLMW09}GongLMW09 \href{http://dx.doi.org/10.1007/978-0-387-88617-6_11}{GongLMW09} & \hyperref[auth:a1250]{J. Gong}, \hyperref[auth:a1251]{Earl E. Lee}, \hyperref[auth:a1252]{John E. Mitchell}, \hyperref[auth:a1253]{William A. Wallace} & Logic-based MultiObjective Optimization for Restoration Planning & No & \cite{GongLMW09} & 2009 & Optimization and Logistics Challenges in the Enterprise & null & 14 & 13 & No & n/a\\
\rowlabel{a:AggounMV08}AggounMV08 \href{http://dx.doi.org/10.1007/978-0-387-74759-0_396}{AggounMV08} & \hyperref[auth:a728]{A. Aggoun}, \hyperref[auth:a911]{C. Maravelias}, \hyperref[auth:a912]{A. Vazacopoulos} & Mixed Integer Programming/Constraint Programming Hybrid Methods & No & \cite{AggounMV08} & 2008 & Encyclopedia of Optimization & null & 0 & 34 & No & n/a\\
\rowlabel{a:Hooker06a}Hooker06a \href{http://dx.doi.org/10.1016/s1574-6526(06)80019-2}{Hooker06a} & \hyperref[auth:a161]{John N. Hooker} & Operations Research Methods in Constraint Programming & No & \cite{Hooker06a} & 2006 & Foundations of Artificial Intelligence & null & 11 & 44 & No & n/a\\
\rowlabel{a:NeronABCDD06}NeronABCDD06 \href{http://dx.doi.org/10.1007/978-0-387-33768-5_7}{NeronABCDD06} & \hyperref[auth:a903]{E. Néron}, \hyperref[auth:a6]{C. Artigues}, \hyperref[auth:a163]{P. Baptiste}, \hyperref[auth:a849]{J. Carlier}, \hyperref[auth:a904]{J. Damay}, \hyperref[auth:a245]{S. Demassey}, \hyperref[auth:a118]{P. Laborie} & Lower Bounds for Resource Constrained Project Scheduling Problem & No & \cite{NeronABCDD06} & 2006 & Perspectives in Modern Project Scheduling & null & 3 & 34 & No & n/a\\
\rowlabel{a:WolfS05a}WolfS05a \href{http://dx.doi.org/10.1007/11415763_12}{WolfS05a} & \hyperref[auth:a51]{A. Wolf}, \hyperref[auth:a714]{H. Schlenker} & Realising the Alternative Resources Constraint & \href{../works/WolfS05a.pdf}{Yes} & \cite{WolfS05a} & 2005 & Applications of Declarative Programming and Knowledge Management & 15 & 5 & 6 & \ref{b:WolfS05a} & n/a\\
\rowlabel{a:AggounV04}AggounV04 \href{http://dx.doi.org/10.1007/978-3-540-24734-0_15}{AggounV04} & \hyperref[auth:a728]{A. Aggoun}, \hyperref[auth:a912]{A. Vazacopoulos} & Solving Sports Scheduling and Timetabling Problems with Constraint Programming & No & \cite{AggounV04} & 2004 & Economics, Management and Optimization in Sports & null & 7 & 4 & No & n/a\\
\rowlabel{a:AjiliW04}AjiliW04 \href{http://dx.doi.org/10.1007/978-1-4419-8917-8_6}{AjiliW04} & \hyperref[auth:a957]{F. Ajili}, \hyperref[auth:a117]{Mark G. Wallace} & Hybrid Problem Solving in ECLiPSe & No & \cite{AjiliW04} & 2004 & Constraint and Integer Programming & null & 4 & 24 & No & n/a\\
\rowlabel{a:DannaP04}DannaP04 \href{http://dx.doi.org/10.1007/978-1-4419-8917-8_2}{DannaP04} & \hyperref[auth:a289]{E. Danna}, \hyperref[auth:a164]{Claude Le Pape} & Two Generic Schemes for Efficient and Robust Cooperative Algorithms & No & \cite{DannaP04} & 2004 & Constraints and Integer Programming & null & 2 & 34 & No & n/a\\
\rowlabel{a:DomdorfPH03}DomdorfPH03 \href{http://dx.doi.org/10.1007/978-3-642-18965-4_31}{DomdorfPH03} & \hyperref[auth:a967]{U. Domdorf}, \hyperref[auth:a441]{E. Pesch}, \hyperref[auth:a968]{To\"{a}n Phan Huy} & Machine Learning by Schedule Decomposition — Prospects for an Integration of AI and OR Techniques for Job Shop Scheduling & No & \cite{DomdorfPH03} & 2003 & Advances in Evolutionary Computing & null & 0 & 57 & No & n/a\\
\rowlabel{a:DorndorfHP99}DorndorfHP99 \href{http://dx.doi.org/10.1007/978-1-4615-5533-9_10}{DorndorfHP99} & \hyperref[auth:a908]{U. Dorndorf}, \hyperref[auth:a909]{Toàn Phan Huy}, \hyperref[auth:a441]{E. Pesch} & A Survey of Interval Capacity Consistency Tests for Time- and Resource-Constrained Scheduling & No & \cite{DorndorfHP99} & 1999 & Project Scheduling & null & 18 & 20 & No & n/a\\
\end{longtable}
}



\clearpage
\subsection{InCollection from bibtex}
{\scriptsize
\begin{longtable}{>{\raggedright\arraybackslash}p{3cm}>{\raggedright\arraybackslash}p{6cm}>{\raggedright\arraybackslash}p{6.5cm}rrrp{2.5cm}rrrrr}
\rowcolor{white}\caption{Works from bibtex (Total 6)}\\ \toprule
\rowcolor{white}Key & Authors & Title & LC & Cite & Year & \shortstack{Conference\\/Journal} & Pages & \shortstack{Nr\\Cites} & \shortstack{Nr\\Refs} & b & c \\ \midrule\endhead
\bottomrule
\endfoot
\rowlabel{a:BlazewiczEP19}BlazewiczEP19 \href{https://ideas.repec.org/h/spr/ihichp/978-3-319-99849-7_16.html}{BlazewiczEP19} & \hyperref[auth:a774]{J. Blazewicz}, \hyperref[auth:a775]{Klaus H. Ecker}, \hyperref[auth:a443]{E. Pesch}, \hyperref[auth:a776]{G. Schmidt}, \hyperref[auth:a777]{M. Sterna}, \hyperref[auth:a778]{J. Weglarz} & {Constraint Programming and Disjunctive Scheduling} & No & \cite{BlazewiczEP19} & 2019 & {Handbook on Scheduling} & 62 & 38 & 0 & No & \ref{c:BlazewiczEP19}\\
\rowlabel{a:Hooker19}Hooker19 \href{https://ideas.repec.org/h/spr/spochp/978-3-030-22788-3_1.html}{Hooker19} & \hyperref[auth:a161]{John N. Hooker} & {Logic-Based Benders Decomposition for Large-Scale Optimization} & No & \cite{Hooker19} & 2019 & {Large Scale Optimization in Supply Chains and Smart Manufacturing} & 26 & 8 & 0 & No & \ref{c:Hooker19}\\
\rowlabel{a:Bartak14}Bartak14 \href{}{Bartak14} & \hyperref[auth:a152]{R. Bart{\'{a}}k} & Planning and Scheduling & No & \cite{Bartak14} & 2014 & Computing Handbook, Third Edition: Computer Science and Software Engineering & null & 0 & 0 & No & \ref{c:Bartak14}\\
\rowlabel{a:BaptisteLPN06}BaptisteLPN06 \href{https://doi.org/10.1016/S1574-6526(06)80026-X}{BaptisteLPN06} & \hyperref[auth:a163]{P. Baptiste}, \hyperref[auth:a118]{P. Laborie}, \hyperref[auth:a164]{Claude Le Pape}, \hyperref[auth:a666]{W. Nuijten} & Constraint-Based Scheduling and Planning & No & \cite{BaptisteLPN06} & 2006 & Handbook of Constraint Programming & 39 & 30 & 25 & No & \ref{c:BaptisteLPN06}\\
\rowlabel{a:KanetAG04}KanetAG04 \href{http://www.crcnetbase.com/doi/abs/10.1201/9780203489802.ch47}{KanetAG04} & \hyperref[auth:a672]{John J. Kanet}, \hyperref[auth:a673]{S. Ahire}, \hyperref[auth:a674]{Michael F. Gorman} & Constraint Programming for Scheduling & No & \cite{KanetAG04} & 2004 & Handbook of Scheduling - Algorithms, Models, and Performance Analysis & null & 0 & 0 & No & \ref{c:KanetAG04}\\
\rowlabel{a:BreitingerL95}BreitingerL95 \href{}{BreitingerL95} & \hyperref[auth:a705]{S. Breitinger}, \hyperref[auth:a706]{Hendrik C. R. Lock} & Using Constraint Logic Programming for Industrial Scheduling Problems & No & \cite{BreitingerL95} & 1995 & Logic Programming: Formal Methods and Practical Applications, Studies in Computer Science and Artificial Intelligence & 27 & 0 & 0 & No & \ref{c:BreitingerL95}\\
\end{longtable}
}



\clearpage
{\scriptsize
\begin{longtable}{>{\raggedright\arraybackslash}p{3cm}r>{\raggedright\arraybackslash}p{4cm}p{1.5cm}p{2cm}p{1.5cm}p{1.5cm}p{1.5cm}p{1.5cm}p{2cm}p{1.5cm}rr}
\rowcolor{white}\caption{Automatically Extracted INCOLLECTION Properties (Requires Local Copy)}\\ \toprule
\rowcolor{white}Work & Pages & Concepts & Classification & Constraints & \shortstack{Prog\\Languages} & \shortstack{CP\\Systems} & Areas & Industries & Benchmarks & Algorithm & a & c\\ \midrule\endhead
\bottomrule
\endfoot
\rowlabel{b:Hooker19}\href{../works/Hooker19.pdf}{Hooker19}~\cite{Hooker19} & 26 & machine, job, task, activity, release-date, make-span, transportation, distributed, resource, job-shop, sequence dependent setup, due-date, order, tardiness, inventory, scheduling, Benders Decomposition & parallel machine, single machine & cycle, disjunctive, cumulative, circuit &  & OPL, MiniZinc & container terminal, satellite, torpedo, yard crane, operating room, patient, railway, aircraft &  & industrial instance & time-tabling & \ref{a:Hooker19} & n/a\\
\rowlabel{b:HurleyOS16}\href{../works/HurleyOS16.pdf}{HurleyOS16}~\cite{HurleyOS16} & 14 & re-scheduling, resource, scheduling, task, machine, distributed, order &  & cumulative &  &  & energy-price, super-computer, datacentre &  & real-world, benchmark &  & \ref{a:HurleyOS16} & n/a\\
\rowlabel{b:KanetAG04}\href{../works/KanetAG04.pdf}{KanetAG04}~\cite{KanetAG04} & 22 & precedence, order, make-span, completion-time, task, tardiness, activity, earliness, due-date, job-shop, resource, machine, job, inventory, setup-time, transportation, scheduling & single machine, parallel machine & Disjunctive constraint, alldifferent, disjunctive &  & ECLiPSe, Cplex, Ilog Solver, OPL & patient &  &  & time-tabling & \ref{a:KanetAG04} & n/a\\
\end{longtable}
}




% \subsection{CSPLib}

% {\scriptsize
% \begin{longtable}{rlp{8cm}lcccll}
% \caption{\label{tab:csplib}CSPLib scheduling problems}\\ \toprule
% Nr & Name & Description & CP System & Data & Code & Solutions & Classification & Constraints\\ \midrule
% \endfirsthead
% \caption{CSPLib scheduling problems}\\ \toprule
% Nr & Name & Description & CP System & Data & Code & Solutions & Classification & Constraints\\ \midrule
% \endhead
% \bottomrule
% \endfoot
% 59 & \href{https://www.csplib.org/Problems/prob059/}{Energy Cost Aware Scheduling} & & - & 50 TXT & - & - & & \\
% 61 & \href{https://www.csplib.org/Problems/prob061/}{RCPSP} & Resource-Constrained Scheduling Problem & PyCSP3 & PSPLIB & y & PSPLIB & RCPSP & \\
% 73 & \href{https://www.csplib.org/Problems/prob073/}{Test Scheduling Problem} & & \su{ECLiPSe OPL} & 840 Prolog & y & & \\
% 77 & \href{https://www.csplib.org/Problems/prob077/}{Stochastic Assignment and Scheduling Problem} & & MiniZinc & 9 DZN & y & & \\
% \end{longtable}
% }





\end{document}


& \href{papers/.pdf}{} & \cite{} & 2019 & CP & & \su{} & & & & & \su{} \\

& \href{articles/.pdf}{} & \cite{} & & & & & & & & & \\
& \href{articles/.pdf}{} & \cite{} & & & & & & & & & \\
& \href{articles/.pdf}{} & \cite{} & & & & & & & & & \\
& \href{articles/.pdf}{} & \cite{} & & & & & & & & & \\
& \href{articles/.pdf}{} & \cite{} & & & & & & & & & \\
& \href{articles/.pdf}{} & \cite{} & & & & & & & & & \\
& \href{articles/.pdf}{} & \cite{} & & & & & & & & & \\
& \href{articles/.pdf}{} & \cite{} & & & & & & & & & \\

& \href{papers/.pdf}{} & \cite{} & & CP & & & & & & & \\

& \href{papers/.pdf}{} & \cite{} &  & CPAIOR & & & & & & & \\
& \href{papers/.pdf}{} & \cite{} &  & CPAIOR & & & & & & & \\
& \href{papers/.pdf}{} & \cite{} &  & CPAIOR & & & & & & & \\
& \href{papers/.pdf}{} & \cite{} &  & CPAIOR & & & & & & & \\
& \href{papers/.pdf}{} & \cite{} &  & CPAIOR & & & & & & & \\
& \href{papers/.pdf}{} & \cite{} &  & CPAIOR & & & & & & & \\
& \href{papers/.pdf}{} & \cite{} &  & CPAIOR & & & & & & & \\
& \href{papers/.pdf}{} & \cite{} &  & CPAIOR & & & & & & & \\
& \href{papers/.pdf}{} & \cite{} &  & CPAIOR & & & & & & & \\
& \href{papers/.pdf}{} & \cite{} &  & CPAIOR & & & & & & & \\
& \href{papers/.pdf}{} & \cite{} &  & CPAIOR & & & & & & & \\
& \href{papers/.pdf}{} & \cite{} &  & CPAIOR & & & & & & & \\
& \href{papers/.pdf}{} & \cite{} &  & CPAIOR & & & & & & & \\
