{\scriptsize
\begin{longtable}{>{\raggedright\arraybackslash}p{3cm}>{\raggedright\arraybackslash}p{6cm}>{\raggedright\arraybackslash}p{6.5cm}rrrp{2.5cm}rrrrr}
\rowcolor{white}\caption{Works from bibtex (Total 12)}\\ \toprule
\rowcolor{white}Key & Authors & Title & LC & Cite & Year & \shortstack{Conference\\/Journal} & Pages & \shortstack{Nr\\Cites} & \shortstack{Nr\\Refs} & b & c \\ \midrule\endhead
\bottomrule
\endfoot
\rowlabel{a:SchuttFSW15}SchuttFSW15 \href{https://doi.org/10.1007/978-3-319-05443-8_7}{SchuttFSW15} & \hyperref[auth:a125]{A. Schutt}, \hyperref[auth:a155]{T. Feydy}, \hyperref[auth:a126]{Peter J. Stuckey}, \hyperref[auth:a156]{Mark G. Wallace} & A Satisfiability Solving Approach & No & \cite{SchuttFSW15} & 2015 & Handbook on Project Management and Scheduling Vol.1 & 26 & 3 & 28 & No & n/a\\
\rowlabel{a:CestaOPS14}CestaOPS14 \href{http://dx.doi.org/10.1007/978-3-319-05443-8_6}{CestaOPS14} & \hyperref[auth:a287]{A. Cesta}, \hyperref[auth:a285]{A. Oddi}, \hyperref[auth:a286]{N. Policella}, \hyperref[auth:a301]{Stephen F. Smith} & A Precedence Constraint Posting Approach & No & \cite{CestaOPS14} & 2014 & Handbook on Project Management and Scheduling Vol.1 & null & 2 & 17 & No & n/a\\
\rowlabel{a:GuSSWC14}GuSSWC14 \href{http://dx.doi.org/10.1007/978-3-319-05443-8_14}{GuSSWC14} & \hyperref[auth:a342]{H. Gu}, \hyperref[auth:a125]{A. Schutt}, \hyperref[auth:a126]{Peter J. Stuckey}, \hyperref[auth:a156]{Mark G. Wallace}, \hyperref[auth:a349]{G. Chu} & Exact and Heuristic Methods for the Resource-Constrained Net Present Value Problem & No & \cite{GuSSWC14} & 2014 & Handbook on Project Management and Scheduling Vol.1 & null & 5 & 35 & No & n/a\\
\rowlabel{a:Milano11}Milano11 \href{http://dx.doi.org/10.1002/9780470400531.eorms0473}{Milano11} & \hyperref[auth:a144]{M. Milano} & Constraint Programming Links with Math Programming & No & \cite{Milano11} & 2011 & Wiley Encyclopedia of Operations Research and Management Science & null & 0 & 28 & No & n/a\\
\rowlabel{a:CastroGR10}CastroGR10 \href{http://dx.doi.org/10.1007/978-1-4419-1644-0_4}{CastroGR10} & \hyperref[auth:a907]{Pedro M. Castro}, \hyperref[auth:a388]{Ignacio E. Grossmann}, \hyperref[auth:a908]{L. Rousseau} & Decomposition Techniques for Hybrid MILP/CP Models applied to Scheduling and Routing Problems & No & \cite{CastroGR10} & 2010 & Hybrid Optimization & null & 0 & 67 & No & n/a\\
\rowlabel{a:Hooker10}Hooker10 \href{http://dx.doi.org/10.1007/978-1-4419-1644-0_2}{Hooker10} & \hyperref[auth:a162]{John N. Hooker} & Hybrid Modeling & No & \cite{Hooker10} & 2010 & Hybrid Optimization & null & 9 & 39 & No & n/a\\
\rowlabel{a:AggounMV08}AggounMV08 \href{http://dx.doi.org/10.1007/978-0-387-74759-0_396}{AggounMV08} & \hyperref[auth:a734]{A. Aggoun}, \hyperref[auth:a925]{C. Maravelias}, \hyperref[auth:a926]{A. Vazacopoulos} & Mixed Integer Programming/Constraint Programming Hybrid Methods & No & \cite{AggounMV08} & 2008 & Encyclopedia of Optimization & null & 0 & 34 & No & n/a\\
\rowlabel{a:NeronABCDD06}NeronABCDD06 \href{http://dx.doi.org/10.1007/978-0-387-33768-5_7}{NeronABCDD06} & \hyperref[auth:a917]{E. Néron}, \hyperref[auth:a6]{C. Artigues}, \hyperref[auth:a164]{P. Baptiste}, \hyperref[auth:a858]{J. Carlier}, \hyperref[auth:a918]{J. Damay}, \hyperref[auth:a246]{S. Demassey}, \hyperref[auth:a118]{P. Laborie} & Lower Bounds for Resource Constrained Project Scheduling Problem & No & \cite{NeronABCDD06} & 2006 & Perspectives in Modern Project Scheduling & null & 3 & 34 & No & n/a\\
\rowlabel{a:AjiliW04}AjiliW04 \href{http://dx.doi.org/10.1007/978-1-4419-8917-8_6}{AjiliW04} & \hyperref[auth:a972]{F. Ajili}, \hyperref[auth:a156]{Mark G. Wallace} & Hybrid Problem Solving in ECLiPSe & No & \cite{AjiliW04} & 2004 & Constraint and Integer Programming & null & 4 & 24 & No & n/a\\
\rowlabel{a:DannaP04}DannaP04 \href{http://dx.doi.org/10.1007/978-1-4419-8917-8_2}{DannaP04} & \hyperref[auth:a290]{E. Danna}, \hyperref[auth:a165]{Claude Le Pape} & Two Generic Schemes for Efficient and Robust Cooperative Algorithms & No & \cite{DannaP04} & 2004 & Constraints and Integer Programming & null & 2 & 34 & No & n/a\\
\rowlabel{a:DomdorfPH03}DomdorfPH03 \href{http://dx.doi.org/10.1007/978-3-642-18965-4_31}{DomdorfPH03} & \hyperref[auth:a982]{U. Domdorf}, \hyperref[auth:a445]{E. Pesch}, \hyperref[auth:a983]{To\"{a}n Phan Huy} & Machine Learning by Schedule Decomposition — Prospects for an Integration of AI and OR Techniques for Job Shop Scheduling & No & \cite{DomdorfPH03} & 2003 & Advances in Evolutionary Computing & null & 0 & 57 & No & n/a\\
\rowlabel{a:DorndorfHP99}DorndorfHP99 \href{http://dx.doi.org/10.1007/978-1-4615-5533-9_10}{DorndorfHP99} & \hyperref[auth:a922]{U. Dorndorf}, \hyperref[auth:a923]{Toàn Phan Huy}, \hyperref[auth:a445]{E. Pesch} & A Survey of Interval Capacity Consistency Tests for Time- and Resource-Constrained Scheduling & No & \cite{DorndorfHP99} & 1999 & Project Scheduling & null & 18 & 20 & No & n/a\\
\end{longtable}
}

